
\chapter{SQL Style Guide}

Maintaining reproducibility and transparency is a core value of Kickstarter's Data team, and a SQL style guide can help us achieve that goal. Additionally, adhering to the basic rules in this style guide will improve our ability to share, maintain, and extend our research when working with SQL.

If you work in a Team or publish your code it is advisable to follow the Team's styling manual, if there is any for SQL. 

This document is written as a manual for anyone working on the Data team, but also as a guide for anyone at the company who would like to write clean and clear code that is meant to be shared.

The individual tips in this guide are based on a composite of knowledge we've gleaned from experience and our roles at previous jobs.


This style guide is written for use with \href{https://aws.amazon.com/redshift/}{AWS Redshift}/ \href{http://www.postgresql.org/docs/8.0/static/release-8-0-2.html}{[Postgres 8.0.2]}, but much of it can be applied to any SQL database.

\section{Principles}

\begin{itemize}
\item We take a disciplined and practical approach to writing code.
\item We regularly check-in code to Github
\item  We believe consistency in style is very important.
\item  We demonstrate intent explicitly in code, via clear structure and comments where needed.
\end{itemize}

\section{Rules}

\subsection{General stuff}

\begin{itemize}
\item No tabs. 2 spaces per indent.
\item No trailing whitespace.
\item Always capitalize SQL keywords (e.g., `SELECT` or `AS`)
\item Variable names should be underscore separated:
\end{itemize}



\textbf{GOOD:}

\begin{minted}{SQL}
  SELECT COUNT(*) AS backers_count
\end{minted}

\textbf{BAD}
  
\begin{minted}{SQL} 
  SELECT COUNT(*) AS backersCount
\end{minted}  

\section{Comments}

* Comments should go near the top of your query, or at least near the closest `SELECT`

* Try to only comment on things that aren't obvious about the query (e.g., why a particular ID is hardcoded, etc.)

* Don't use single letter variable names be as descriptive as possible given the context:

\textbf{GOOD}

\begin{minted}{SQL}
  `SELECT ksr.backings AS backings_with_creators`
\end{minted}

\textbf{BAD}
\begin{minted}{SQL}
  SELECT ksr.backings AS b
\end{minted}


Use \href{http://www.postgresql.org/docs/8.4/static/queries-with.html}{Common Table Expressions} (CTEs) early and often, and name them well.

\mintinline{HAVING} isn't supported in Redshift, so use |CTE|s instead. If you don't know what this means, ask a friendly Data Team member.

\subsection{SELECT}

Align all columns to the first column on their own line:

\begin{minted}{SQL}
SELECT
  projects.name,
  users.email,
  projects.country,
  COUNT(backings.id) AS backings_count
FROM ...
\end{minted}


\sql{SELECT} always should go on its own line:

\begin{minted}{SQL}
SELECT
  name,
  ...
\end{minted}

Always rename aggregates and function-wrapped columns:

\begin{minted}{SQL}
SELECT
  name,
  SUM(amount) AS sum_amount
FROM ...
\end{minted}



Always rename all columns when selecting with table aliases:

\begin{minted}{SQL}
SELECT
  projects.name AS project_name,
  COUNT(backings.id) AS backings_count
FROM ksr.backings AS backings
INNER JOIN ksr.projects AS projects ON ...
\end{minted}

Always use \sql{AS} to rename columns:

GOOD:

\begin{minted}{SQL}
SELECT
  projects.name AS project_name,
  COUNT(backings.id) AS backings_count
...
\end{minted}

BAD:

\begin{minted}{SQL}
SELECT
  projects.name project_name,
  COUNT(backings.id) backings_count
...
\end{minted}

\endinput
Long Window functions should be split across multiple lines: one for the `PARTITION`, `ORDER` and frame clauses, aligned to the `PARTITION` keyword. Partition keys should be one-per-line, aligned to the first, with aligned commas. Order (`ASC`, `DESC`) should always be explicit. All window functions should be aliased.

```sql
SUM(1) OVER (PARTITION BY category_id,
                          year
             ORDER BY pledged DESC
             ROWS UNBOUNDED PRECEDING) AS category_year
```

## `FROM`

Only one table should be in the `FROM`. Never use `FROM`-joins:

__GOOD__:

```sql
SELECT
  projects.name AS project_name,
  COUNT(backings.id) AS backings_count
FROM ksr.projects AS projects
INNER JOIN ksr.backings AS backings ON backings.project_id = projects.id
...
```

__BAD__:

```sql
SELECT
  projects.name AS project_name,
  COUNT(backings.id) AS backings_count
FROM ksr.projects AS projects, ksr.backings AS backings
WHERE
  backings.project_id = projects.id
...
```

## `JOIN`

Explicitly use `INNER JOIN` not just `JOIN`, making multiple lines of `INNER JOIN`s easier to scan:

__GOOD__:

```sql
SELECT
  projects.name AS project_name,
  COUNT(backings.id) AS backings_count
FROM ksr.projects AS projects
INNER JOIN ksr.backings AS backings ON ...
INNER JOIN ...
LEFT JOIN ksr.backer_rewards AS backer_rewards ON ...
LEFT JOIN ...
```

__BAD__:

```sql
SELECT
  projects.name AS project_name,
  COUNT(backings.id) AS backings_count
FROM ksr.projects AS projects
JOIN ksr.backings AS backings ON ...
LEFT JOIN ksr.backer_rewards AS backer_rewards ON ...
LEFT JOIN ...
```

Additional filters in the `INNER JOIN` go on new indented lines:

```sql
SELECT
  projects.name AS project_name,
  COUNT(backings.id) AS backings_count
FROM ksr.projects AS projects
INNER JOIN ksr.backings AS backings ON projects.id = backings.project_id
  AND backings.project_country != 'US'
...
```

The `ON` keyword and condition goes on the `INNER JOIN` line:

```sql
SELECT
  projects.name AS project_name,
  COUNT(backings.id) AS backings_count
FROM ksr.projects AS projects
INNER JOIN ksr.backings AS backings ON projects.id = backings.project_id
...
```

Begin with `INNER JOIN`s and then list `LEFT JOIN`s, order them semantically, and do not intermingle `LEFT JOIN`s with `INNER JOIN`s unless necessary:

__GOOD__:

```sql
INNER JOIN ksr.backings AS backings ON ...
INNER JOIN ksr.users AS users ON ...
INNER JOIN ksr.locations AS locations ON ...
LEFT JOIN ksr.backer_rewards AS backer_rewards ON ...
LEFT JOIN ...
```

__BAD__:

```sql
LEFT JOIN ksr.backer_rewards AS backer_rewards ON backings
INNER JOIN ksr.users AS users ON ...
LEFT JOIN ...
INNER JOIN ksr.locations AS locations ON ...
```

## `WHERE`

Multiple `WHERE` clauses should go on different lines and begin with the SQL operator:

```sql
SELECT
  name,
  goal
FROM ksr.projects AS projects
WHERE
  country = 'US'
  AND deadline >= '2015-01-01'
...
```

## `CASE`

`CASE` statements aren't always easy to format but try to align `WHEN`, `THEN`, and `ELSE` together inside `CASE` and `END`:

```sql
CASE WHEN category = 'Art'
     THEN backer_id
     ELSE NULL
END
```

## Common Table Expressions (CTEs)

[From AWS](http://docs.aws.amazon.com/redshift/latest/dg/r_WITH_clause.html):

>`WITH` clause subqueries are an efficient way of defining tables that can be used throughout the execution of a single query. In all cases, the same results can be achieved by using subqueries in the main body of the `SELECT` statement, but `WITH` clause subqueries may be simpler to write and read.

The body of a CTE must be one indent further than the `WITH` keyword. Open them at the end of a line and close them on a new line:

```sql
WITH backings_per_category AS (
  SELECT
    category_id,
    deadline,
    ...
)
```

Multiple CTEs should be formatted accordingly:

```sql
WITH backings_per_category AS (
  SELECT
    ...
), backers AS (
  SELECT
    ...
), backers_and_creators AS (
  ...
)
SELECT * FROM backers;
```

If possible, `JOIN` CTEs inside subsequent CTEs, not in the main clause:

__GOOD__:

```sql
WITH backings_per_category AS (
  SELECT
    ...
), backers AS (
  SELECT
    backer_id,
    COUNT(backings_per_category.id) AS projects_backed_per_category
  INNER JOIN ksr.users AS users ON users.id = backings_per_category.backer_id
), backers_and_creators AS (
  ...
)
SELECT * FROM backers_and_creators;
```

__BAD__:

```sql
WITH backings_per_category AS (
  SELECT
    ...
), backers AS (
  SELECT
    backer_id,
    COUNT(backings_per_category.id) AS projects_backed_per_category
), backers_and_creators AS (
  ...
)
SELECT * FROM backers_and_creators
INNER JOIN backers ON backers_and_creators ON backers.backer_id = backers_and_creators.backer_id
```

Always use CTEs over inlined subqueries.

## Tips

* [Sublime](http://www.sublimetext.com/3) is your friend. Configure it to use soft tabs (e.g. 2 spaces), and [trim trailing whitespace](http://nategood.com/sublime-text-strip-whitespace-save)
* Helpful Sublime packages include [Githubinator](https://github.com/ehamiter/GitHubinator), [SendText](https://github.com/wch/SendText), and [Package Control](https://packagecontrol.io/installation).
* Check code into github early and often.
* Always provide a Githubinator permalink in Trello cards where any code is used.