\chapter{Using psql}

Psql is the interactive terminal for working with Postgres. Theres an abundance of flags available for use when working with psql, but lets focus on some of the most important ones, then how to connect:

\begin{verbatim}
-h the host to connect to
-U the user to connect with
-p the port to connect to (default is 5432)
\end{verbatim}


\begin{minted}{bat}
psql -h localhost -U username databasename
\end{minted}

The other option is to use a full string and let psql parse it:

\begin{minted}{bat}
psql "dbname=dbhere host=hosthere user=userhere password=pwhere port=5432 sslmode=require"
\end{minted}


Once you've connected you can begin querying immediately. In addition to basic queries you can also use certain commands. Running |\?| will give you a list of all available commands, though a few key ones are called out below.

List all databases

\begin{minted}{bat}
# \l
          List of databases
Name     |   Owner   | Encoding | Collate | Ctype |  Access privileges
---------+-----------+----------+---------+-------+--------------------
learning |yannis     | UTF8     | C       | UTF-8 |
\end{minted}

Note the use of the backslash |\| prefixing all commands.

I always try and memorize as much as I can, as this improves my productivity. Survival commands are shown below. These a re a minimum set of commands that you need to be productive without consulting the manuals often.

To connect to a database:

\begin{minted}{bat}
# \c  mydb
\end{minted}


To execute a command in shell

\begin{minted}{bat}
\! [COMMAND]  execute command in shell or start interactive shell
\end{minted}







