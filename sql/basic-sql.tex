\chapter{Introduction to Structured Query Language (SQL)}

\section{Creating the Database}

Before you can use a new RDBMS, you must complete two tasks: create the database structure, and create the tables
that will hold the end-user data. To complete the first task, the RDBMS creates the physical files that will hold the
database. When you create a new database, the RDBMS automatically creates the data dictionary tables in which to
store the metadata and creates a default database administrator. Creating the physical files that will hold the database
means interacting with the operating system and the file systems supported by the operating system. Therefore, creating
the database structure is the one feature that tends to differ substantially from one RDBMS to another. However, it
is relatively easy to create a database structure, regardless of which RDBMS you use.

If you use Microsoft Access, creating the database is simple: start Access, click the FILE tab, click New in the left pane,
and then click Blank desktop database in the right pane. Specify the folder in which you want to store the database, and
then name the database. However, if you work in a database environment typically used by larger organizations, you
will probably use an enterprise RDBMS such as Oracle, MS SQL Server, MySQL, or DB2. Given their security requirements
and greater complexity, creating a database with these products is a more elaborate process. 

With the exception of creating the database, most RDBMS vendors use SQL that deviates little from the ANSI standard
SQL. For example, most RDBMSs require each SQL command to end with a semicolon. However, some SQL
implementations do not use a semicolon. 


If you are using an enterprise RDBMS, you must be authenticated by the RDBMS before you can start creating tables.
Authentication is the process the DBMS uses to verify that only registered users access the database. To be authenticated,
you must log on to the RDBMS using a user ID and a password created by the database administrator. In an
enterprise RDBMS, every user ID is associated with a database schema.

My suggestion when you start is to use an open source database such as href{https://www.sqlite.org/index.html}{SQLITE3}. Most of the open source RDMS provide utilities for generating databases via a console. 


\section{Creating Table Structures}

Now you are ready to implement the PRODUCT and VENDOR table structures with the help of SQL, using the

CREATE TABLE syntax shown next.

\begin{minted}{SQL}
CREATE TABLE tablename (
    column1 data type [constraint] [,
    column2 data type [constraint] ] [,
    PRIMARY KEY (column1 [, column2]) ] [,
    FOREIGN KEY (column1 [, column2]) REFERENCES tablename] [,
    CONSTRAINT constraint ] );
\end{minted}    


To make the SQL code more readable, most SQL programmers use one line per column (attribute) definition. In addition,
spaces are used to line up the attribute characteristics and constraints. Finally, both table and attribute names are
fully capitalized. Those conventions are used in the following examples that create VENDOR and PRODUCT tables
and subsequent tables throughout the book.

