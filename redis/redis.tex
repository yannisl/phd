\chapter{Redis}

\section{Introduction}

Redis is an in-memory key-value store known for its flexibility, performance, and wide language support. In this guide, we will demonstrate how to install and configure Redis on an Ubuntu 16.04 server, using the Windows Subsystem for Linux. Amazingly since about a year ago, Microsoft decided to not handicap Windows but to start providing useful tools for Open Tools.


\section{Install Windows Subsystem for Linux (WSL)}


From Start, search for Turn Windows features on or off (type turn)
Select Windows Subsystem for Linux (beta)


Once installed you can run bash on Ubuntu by typing bash from a Windows Command Prompt. To install the latest version of Redis we'll need to use a repository that maintains up-to-date packages for Ubuntu and Debian servers like https://www.dotdeb.org which you can add to Ubuntu's apt-get sources with:


\begin{minted}{bash}
$ echo deb http://packages.dotdeb.org wheezy all >> dotdeb.org.list
$ echo deb-src http://packages.dotdeb.org wheezy all >> dotdeb.org.list
$ sudo mv dotdeb.org.list /etc/apt/sources.list.d
$ wget -q -O - http://www.dotdeb.org/dotdeb.gpg | sudo apt-key add -
\end{minted}

Then after updating our APT cache we can install Redis with:

\begin{minted}{bash}
$ sudo apt-get update
$ sudo apt-get install redis-server
You'll then be able to launch redis with:
\end{minted}


\begin{minted}{bash}
$ redis-server --daemonize yes
\end{minted}

Which will run redis in the background freeing your shell so you can play with it using the redis client:

\begin{minted}{bash}
$ redis-cli
$ 127.0.0.1:6379> SET foo bar
OK
$ 127.0.0.1:6379> GET foo
"bar"
\end{minted}

Which you can connect to from within bash or from your Windows desktop using the redis-cli native Windows binary from MSOpenTech.