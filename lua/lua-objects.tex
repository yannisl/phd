
\parskip10pt

\chapter{Lua Objects}
\label{c:luaobjects}

\section{Class-like tables}
Lua is similar to JavaScript in that the concept of class is not directly supported by the language. In fact, Lua has a very general mechanism for extending the behaviour of tables which makes it straightforward to implement classes. A table’s behaviour is controlled by its metatable. If that metatable has a \cmd{__index} function or table, this will handle looking up anything which is not found in the original table. A class is just a table with an \cmd{__index} key pointing to itself. Creating an object involves making a table and setting its metatable to the class; then when handling obj.fun, Lua first looks up fun in the table obj, and if not found it looks it up in the class. obj:fun(a) is just short for obj.fun(obj,a). So with the metatable mechanism and this bit of syntactic sugar, it is straightforward to implement classic object orientation.

Lua doesn't have any built-in mechanism for classes. They are created through the use of tables
and metatables. A class works as a mold for the creation of objects. Most object-oriented languages
offer the concept of class. In such languages, each object is an instance
of a specific class. Lua does not have the concept of class; each object defines
its own behavior and has a shape of its own. Nevertheless, it is not difficult to
emulate classes in Lua, following the lead from prototype-based languages like
Self and NewtonScript. In these languages, objects have no classes. Instead,
each object may have a prototype, which is a regular object where the first object
looks up any operation that it does not know about. To represent a class in
such languages, we simply create an object to be used exclusively as a prototype
for other objects (its instances). Both classes and prototypes work as a place to
put behavior to be shared by several objects

\emphasis{Animal}
\begin{texexample}{Class-like}{ex:classlike}
\begin{luacode}
local Animal = {}                        -- 1. 

function Animal:new()                    -- 2.
  newObj = {sound = 'woof'}              -- 3.
  self.__index = self                    -- 4.
  return setmetatable(newObj, self)      -- 5.
end

function Animal:makeSound()              -- 6.
  tex.print('I say ' .. self.sound)
end

Dog = Animal:new()                       -- 7.
Dog:makeSound()  -- 'I say woof'         -- 8.

function Dog:makesound()
  s = self.sound .. 'bark '
  tex.print(s)
end

Dog:makesound()
  
\end{luacode}
\end{texexample}

Unsurprisingly Lua uses tables to create class-like objects. In the example we will create a class Animal and then an instance of a Dog. The function \luacmd{tablename:fn()} is the same as \luacmd{tablename.fn(self,...)}. The ‘:’ just adds a first argument to the function called \textit{self}.  The \luacmd{newObj} will be an instance of class \textbf{Animal}. The self = Animal, but inheritance ca change it. 


If needed, a subclass's new() is like the bases's:

\begin{verbatim}
function LoudDog:new()
  newObj = {}
  -- set up newObj
  self.__index = self
  return setmetatable(newObj, self)
end
\end{verbatim}

\begin{texexample}{A shapes librayy}{}
\begin{luacode}
-- Meta class

Shape = {area = 0}
-- Base class method new

function Shape:new (o, side)
  o = o or {}
  setmetatable(o, self)
  self.__index = self
  side = side or 0
  self.area = side*side;
  return o
end

-- Base class method printArea
function Shape:printArea ()
  tex.print("The area is ", self.area)
end

-- We can extend the shape to a square class as shown below.

Square = Shape:new()
-- Derived class method new

function Square:new (o,side)
  o = o or Shape:new(o,side)
  setmetatable(o, self)
  self.__index = self
  return o
end

local square = Square:new(nil, 8)
square:printArea()
\end{luacode}
\end{texexample}

A common error when calling  methods is to forget to call them with (:) syntax. 

\section{Naming conventions and style}

Classes have names that start with a capital letter. Methods should be preferably all lower case. 

\section{Using a library}

Many libraries, such as penlight offer convenient modules to create classes and avoid developing them from scratch. The example that follows is from its documentation.

\begin{texexample}{Using penlight}{}
\begin{luacode}

-- animal.lua

class = require 'pl.class'


class.Animal()

function Animal:_init(name)
    self.name = name
end

function Animal:__tostring()
  return self.name..': '..self:speak()
end

class.Dog(Animal)

function Dog:speak()
  return 'bark'
end

class.Cat(Animal)

function Cat:_init(name,breed)
    self:super(name)  -- must init base!
    self.breed = breed
end

function Cat:speak()
  return 'meow'
end

class.Lion(Cat)

function Lion:speak()
  return 'roar'
end

fido = Dog('Fido')
felix = Cat('Felix','Tabby')
leo = Lion('Leo','African')

tex.print(fido:speak(), leo.breed, fido.name)
\end{luacode}
\end{texexample}









