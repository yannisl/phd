\chapter{Working with TeX's Registers}

%sp = 65536

The tex table contains a large list of virtual internal TEX parameters that are partially writable.
The designation ‘virtual’ means that these items are not properly defined in Lua, but are only
frontends that are handled by a metatable that operates on the actual TEX values. As a result,
most of the Lua table operators (like \luacmd{pairs} and \#) do not work on such items.
At the moment, it is possible to access almost every parameter that has these characteristics:

\begin{enumerate}
\item You can use it after \cs{the}

\item It is a single token.

\item Some special others, see the list below
\end{enumerate}

This excludes parameters that need extra arguments, like |\the\scriptfont|.

The subset comprising simple integer and dimension registers are writable as well as readable
(stuff like |\tracingcommands| and |\parindent|)

\section{Dimension Registers}

The dimension parameters accept Lua numbers (signifying scaled points) or strings (with included dimension). The result is always a number in scaled points.

The LuaTeX manual is very terse and examples are scarce. It assumes that the reader is both an expert in TeX programming as well as Lua. I will carefully go over the tex table which exposes the TeX engine to Lua. In TeX to set a dimension register to a value we use |\parindent=10pt| or |\parindent0pt|. In the following example, we will change the parindent to 20pt via a TeX command and access its value via the tex table.

\begin{texexample}{Accessing Dimension Registers}{}
\begingroup
\parindent20pt
\begin{luacode}
local parindent = tex.parindent
tex.print("parindent = "..parindent.."\\\\")  
tex.print("\\lorem")  
\end{luacode}
\endgroup
\end{texexample}

We notice that the value is returned as \textit{sp} units. To convert we have to divide by 65536 but as LuaTeX has a build-in function \luacmd{tex.sp} we will use this. The \luacmd{tex.sp} can use a string such as |"1pt"| to convert \textit{sp} units to \textit{pt}s.

\begin{texexample}{Accessing Dimension Registers}{}
\parindent20pt
\begin{luacode}
local parindentx = tex.parindent
tex.print("parindent = "..tostring(parindentx/(tex.sp("1pt"))).."pt \\\\")  
tex.print("\\lorem\\par")  

tex.parindent=tex.sp("50pt")
tex.print("\\par\\leavevmode\\alicei".."\\par")
tex.print("parindent = "..tostring(tex.parindent/(tex.sp("1pt"))).."pt \\\\")
 
tex.setglue('parskip', 0, 65536, 0, 0, 0)
tex.print("\\lorem\\par\\lorem\\par")
\end{luacode}
\lorem
\lorem
\end{texexample}

The dimension registers accept Lua numbers (in scaled points) or strings (with an included
absolute dimension; \textit{em} and \textit{ex} and \textit{px} are not allowed). 
The result is always a number in scaled points (\textit{sp}).

\begin{texexample}{Dimensions}{ex:dim3}
\begin{luacode}
local maxdimen = tex.dimen['maxdimen']
maxdimen = string.format("%.2f", maxdimen/tex.sp('1cm'))
print(maxdimen.."cm")
\end{luacode}
\end{texexample}


\section{Using an Inspector}

The Lua manual is terse and sometimes one needs to see the values of a table for a module. The |inspect| library bundled with this manual is based on Enrique García Cota's library \url{https://github.com/kikito/inspect.lua}. I have modified it to change the appearance of tabs and to ensure underscores and other reserved characters are properly escaped. The objective of the library is human understanding.

\luacmd{inspect} has the declaration: |local str = inspect(value, <options>)|, |value| can be any Lua value.

\luacmd{inspect} it transforms simple types (like strings or numbers) into string.

\begin{texexample}{The inspect library}{ex:inspect0}
\begin{luacode}
local inspect = require("inspect.lua")
print(inspect("I am a string."))
\end{luacode}
\end{texexample}

Tables, on the otehr hand are rendered in a way a human can read easily. If a table is\enquote{array-like} it is rendered horizontally.

\begin{texexample}{The inspect library}{ex:inspect1}
\begin{luacode}
local inspect = require("inspect.lua")
local t = {1,2,3,4,5,6}
print(inspect(t))
\end{luacode}
\end{texexample}

If a table is \enquote{dictionary-like} it will be rendered with one element per line:

\begin{texexample}{The inspect library}{ex:inspect2}
\begin{luacode}
local inspect = require("inspect.lua")
local t = {a=1,b=2,c=3,d=4,e=5,f=6}
print(inspect(t))
\end{luacode}
\end{texexample}

Where possible the keys are sorted alphanumerically. Hybrid tables will have the array part on the first line, and the dictionarypart just below them:

\begin{texexample}{The inspect library}{ex:inspect2}
\begin{luacode}
inspect = require("inspect.lua")
local t = {1,2,a=1,b=2,c=3,d=4,e=5,f=6,3,4,10}
print(inspect(t))
\end{luacode}
\end{texexample}

\subsection*{options}

The \luacmd{inspect} has a second parameter, called |options|. This is an optional parameter, but when it is 
provided it must be a table.

\begin{description}
\item[options.depth] sets the maximum depth that will be printed out. When the max depth is reached, inspect will stop parsing tables and just return {...}:
\end{description}

\begin{texexample}{The inspect library}{ex:inspect5}
\begin{luacode}
local inspect = require'inspect'
local t5 = {a = {b = {c = {d = {e = 5}}}}}

print(inspect(t5, {depth=4}))
\end{luacode}
\end{texexample}

\luacmd{options.depth} defaults to infinite (\luacmd{math.huge}).

\begin{texexample}{The inspect library}{ex:inspect6}
\begin{luacode}
local inspect = require('inspect')
local t6 = {a={b=1}}
print(inspect(t6,{newline='\\linebreak',indent='++'})..'\\par')
\end{luacode}
\end{texexample}

\subsubsection*{options.process}

This is a function which allows  altering the passed object before transforming it into a string. A typical way to use it would be to remove certain values so that they don't appear at all.

\luacmd{options.process} has the following signature:

|local processed_item = function(item, path)|

\begin{description}
\item[item] is either a key or a value on the table, or any of its subtables
\item[path] is an array-like table built with all the keys that have been used to reach item, from the root.

For values, it is just a regular list of keys. For example, to reach the 1 in |{a = {b = 1}}|, the path will be |{'a', 'b'}|

For keys, the special value inspect.KEY is inserted. For example, to reach the c in |{a = {b = {c = 1}}}|, the path will be |{'a', 'b', 'c', inspect.KEY }|

For metatables, the special value inspect.METATABLE is inserted. For |{a = {b = 1}}}|, the path |{'a', {b = 1}, inspect.METATABLE}| means "the metatable of the table |{b = 1}|".

\item[processed\_item] is the value returned by options.process. If it is equal to item, then the inspected table will look unchanged. If it is different, then the table will look different; most notably, if it's nil, the item will dissapear on the inspected table.
\end{description}

In the next example we will remove a metatable.
\begin{texexample}{The inspect library}{ex:inspect6}
\begin{luacode}
local inspect = require('inspect')
local t = {1,2,3}
local mt = {b = 2}
setmetatable(t, mt)
print(inspect(t))
print("")

-- remove metatable
print("metatable filtered out".."\\par")
local remove_mt = function(item)
  if item ~= mt then return item end
end

print(inspect(t,{process = remove_mt}))
print("")
\end{luacode}
\end{texexample}

\chapter{The LuaTeX token Library}

The token library provides means to intercept the input and deal with it at the Lua level. The
library provides a basic scanner infrastructure that can be used to write macros that accept
a wide range of arguments. This interface is on purpose kept general and as performance is
quite ok one can build additional parsers without too much overhead. It’s up to macro package
writers to see how they can benefit from this as the main principle behind LuaTEX is to provide a
minimal set of tools and no solutions. The functions provided in the token namespace are given
in the next table:

\begin{texexample}{The inspect library}{ex:inspect9}
\begin{luacode}
local inspect = require('inspect')
local t = token
print(inspect(t))
\end{luacode}
\end{texexample}



\begin{texexample}{The inspect library}{ex:inspect9}
\gdef\bar{bar}
\gdef\foo{foo-\bar}
\begin{luacode}
local inspect = require('inspect')
local t = token
print(t.is_token('bar'))
\end{luacode}
\end{texexample}


\begin{texexample}{The inspect library}{ex:inspect9}
\def\bar{bar}
\def\foo{foo-\bar}

we get:
\directlua{token.scan_string()}{foo} foo full expansion\\
\directlua{token.scan_string()}foo foo letters and others\\
\directlua{z=token.scan_string()}\foo foo-bar meaning\\
\directlua{print(z)}
\end{texexample}

We can see the scanner has absorbed the arguments.


\begin{texexample}{Create a command}{ex:token2}
\edef\relaxing{relaxing}
\def\foo#1{#1}
\begin{luacode}
local t=token.create("relaxing")
local tbl = {cmdname=t.cmdname, command= t.command, id=t.id}
print(inspect(tbl))
print("")
local t=token.create("foo")
local tbl = {cmdname=t.cmdname, command= t.command, id=t.id}
print(inspect(tbl))

print("")
local t=token.create("A")
local s,s1 = "nil","nil"
if t.active then s="active" else s="inactive" end
if t.expandable then s1="expandable" else s1="not expandable" end
local tbl = {cmdname=t.cmdname, command= t.command, id=t.id, active=s, expandable=s1}
print(inspect(tbl))

print("")
local t=token.create("!")
local s,s1 = "nil","nil"
if t.active then s="active" else s="inactive" end
if t.expandable then s1="expandable" else s1="not expandable" end
local tbl = {cmdname=t.cmdname, command= t.command, id=t.id, active=s, expandable=s1}
print(inspect(tbl))
\end{luacode}
\end{texexample}

The numbers that represent a catcode are the same as in TEX itself, so using this information
assumes that you know a bit about TEX’s internals. The other numbers and names are used
consistently but are not frozen. So, when you use them for comparing you can best query a
known primitive or character first to see the values.

\section*{Macros}

The \luacmd{set\_macro} can get up to four arguments. It has the signature:

|setmacro(catcodetable,"csname","content","global")|


\begin{texexample}{Create a command with Lua}{ex:token3}
\begin{luacode}
local MyVal = 123
token.set_macro("MyVal", MyVal, "global")
\end{luacode}
\MyVal
\end{texexample}

I have been using the |luacode| environment for consistency. Like any of LaTeX's environemnts it
creates a group and hence the reason we need to use the \luacmd{global} argument, which will create a
|gdef|. For local definitions \luacmd{directlua} can be used.

\begin{texexample}{Create a command with Lua}{ex:token3}
\luadirect{local test = 123;token.set_macro("foo", test)}
\foo
\end{texexample}

Of course the best application is when we combine it with \tex macros.
\begin{texexample}{Create a command with Lua}{ex:token3}
\def\bar#1{%
\luadirect{local test = #1+8^2;token.set_macro("foo", test)}
\foo
}
\bar{2^16}
\end{texexample}


\subsubsection*{get\_meaning and get\_macro}

The token library has functions starting from |get_| defining getters for macro related
properties.



\begin{texexample}{Get the meaning programmatically}{ex:mean}

\def\alpha#1#2{beta#1 gamma#2 delta}

\begin{luacode}
print(token.get_macro("alpha"))

print(token.get_meaning("alpha").."\\par")


print(token.set_macro("alpha","whatever"))

print(token.get_macro("alpha"))

print(token.get_meaning("alpha").."\\par")
\end{luacode}

\end{texexample}

%https://tex.stackexchange.com/questions/129046/use-lualatex-to-create-macros
%%%%%%%%%%%%%%%%% TOKEN %%%%%%%%%%%%%%%%%%%%%%%%%%%%%%%%%%%%


\begin{luacode}
print("\\parskip1pt")
local inspect = require 'inspect'
local t5 = {a_content = {b_tex = {c_2 = {d = {e = 5}}}}}

print(inspect({1,2,3,4})..'\\par')
print(inspect({a=1,b=2}).."\\par")
print(inspect(t5)..'\\par') 
print(inspect(tex)..'\\par')

local t = tex.primitives()
print(inspect(t)..'\\par')

print(inspect(texconfig)..'\\par')
\end{luacode}

