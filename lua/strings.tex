\chapter{Strings}

Strings in Lua represent text. They can contain a single letter or an entire book. Programs that manipulate 1 M characters are not unusual in Lua. \cite{roberto}

\tex never provided any mechanisms for the manipulation of strings and neither did \latex. This is a serious limitation for libraries that have to deal mostly with the manipulation and typesetting of textual data and one of the reasons to use \LUA. 

The double quote characters (") mark the beginning and end of the string. Marking the beginning and
end is all they do; they are not actually part of the string, which is why print doesn't print them, as in
this example:

\begin{texexample}{Strings}{lua:strings}
\directlua{
    tex.sprint("This is a string")
}
\end{texexample}

Like numbers, strings are values, which means they can be assigned to variables. Here's an example

\begin{texexample}{Strings are values}{lua:svalues}
\directlua{
  local  name, phone = "Jane X. Doe", "248-555-5898"
  tex.sprint(name, phone)
}
\end{texexample}

\subsection{Quoting Strings with Square Brackets}

\begin{texexample}{Multiline strings}{lua:multilinestrings}
\directlua{
 tex.sprint([[
    There are some 
    funky  characters in this
    string. 
]])
}
\end{texexample}

Now what do you do if you want to print to square brackets together?

\begin{texexample}{Comments}{}
\ttfamily \directlua{
 tex.sprint([=[
    [[There]] are some 
    funky  characters in this
    string. 
]=])
}
\end{texexample}

\subsection{Escaping Characters}

\begin{texexample}{Escaping characters}{ex:esc1}
\ttfamily \directlua{
 tex.sprint([=[
    [[There]] are some 
    funky \string\\ characters in this \string\\
    string. 
]=])
}
\end{texexample}


\begin{texexample}{Escaping}{ex:esc}
\ttfamily \directlua{
 tex.sprint("2"+21)
}
\end{texexample}

\subsection{Concatenation operator}

\begin{texexample}{Concatenation operator}{ex:concatop}
\ttfamily \directlua{
 tex.sprint("99" ..  21) -- this is a comment
}
\end{texexample}

\section{Comments}
\begin{texexample}{comments}{ex:comment}
\ttfamily \directlua{
 tex.sprint("99"..21) -- this is a comment
}
\end{texexample}


\section{The kpse library}

This library provides two separate, but nearly identical interfaces to the kpathsea file search
functionality: there is a \enquote{normal} procedural interface that shares its kpathsea instance with
LuaTEX itself, and an object oriented interface that is completely on its own.

\subsection{find\textunderscore file}

The most often used function in the librar is \textsf{find\_file}:

\chapter{Lua Strings Library}

\section{Introduction}


The \LUA string library in \LUA\tex has been extended to provide a number of additional functions: a very useful function is \luacmd{string.explode} which returns a table with the exploded string. 


\begin{docLua}{string.len(string)} 
can be used to find the length of a string.
The |string.len| function is not unicode aware and the slunicode library which be used for cases where utf characters above unicode codepoint 127 is required.
\end{docLua}


\begin{docLua}{string.upper(argument)}
Returns a capitalized representation of the argument.
\end{docLua}
\begin{texexample}{Finding the length of a string}{}
We can get the length of a string using the length operator (denoted by \#):
\begin{luacode}
a = "hello"
tex.print(#a) --> 5
tex.print(#"good bye") --> 8
\end{luacode}
\end{texexample}



\begin{docLua}{string.lower(argument)}
Returns a lower case representation of the argument.
\end{docLua}

\begin{docLua}{string.gsub(\meta{mainString},findString,replaceString)}
Returns a string by replacing occurrences of findString with replaceString.
\end{docLua}

\begin{docLua}{string.reverse(arg)}
Returns a string by reversing the characters of the passed string.
\end{docLua}

\begin{docLua}{string.char(arg)} 
and string.byte(arg)
Returns internal numeric and character representations of input argument.
\end{docLua}

\begin{docLua}{string.rep(string, n))}
Returns a string by repeating the same string n number times.
\end{docLua}



\begin{docLua}{string.format(format,string,...,stringn)}
Returns a formatted string.
\end{docLua}


\begin{texexample}{Formatting strings}{}
\begin{luacode}
string1 = "Lua"
string2 = "Tutorial"
number1 = 10
number2 = 20
-- Basic string formatting
print(string.format("Basic formatting %s %s",string1,string2))

-- Date formatting
date = 2; month = 1; year = 2014
print(string.format("Date formatting %02d/%02d/%03d", date, month, year))

-- Decimal formatting
print(string.format("%.4f",1/3))
\end{luacode}
\end{texexample}




\begin{texexample}{Exploding strings}{}
\begin{luacode}
local z = string.explode("abcd", " ")
local   writeln = function (v) 
   return   tex.print("key = ", v, "\\par")
end

for _,v in ipairs (z) do
  writeln(v)
end

tex.print(unicode.utf8.len("résumé"))
\end{luacode}
\end{texexample}

If you notice in the example above the space at the end of "d" was captured and inserted into the return table. The string.trim trims a string and removes and spaces from the beginning and the end of a string.

\begin{texexample}{Trimming strings}{}
\begin{luacode*}


string.trim = function (s)
   return (string.gsub(s, "^%s*(.-)%s*$", "%1"))
end

-- string.trim = trim

local  writeln = function (v) 
   return   tex.print("key = ", v, "\\par")
end

local str = " a b c d e "

tex.print(str:trim(str).."zzzz\\par")

local z = string.explode(str, " ")
  
for _,v in ipairs (z) do
  writeln(v)
end
\end{luacode*}
\end{texexample}

One of the advantages of Lua, is that any library is just a table and hence can be extended easily. In the example above we extended the string library by adding the \textbf{line string.trim = trim}.  




\begin{texexample}{Finding the length  of a string}{}
\begin{luacode*}
 local z = "ąΒβ"
 tex.print(string.len(z),"\\par")
 tex.print(unicode.utf8.len(z))
 str="äöABCäö"
 i = font.id("arial")
 tex.print(i, font.current(), str)
\end{luacode*}
\end{texexample}

For more details on the rest of the available methods, see the Lua and LuaLaTeX manuals.
