\chapter{Metatables and Metamethods}

\def\textdblunderscore#1{{\bfseries \textunderscore\hskip1pt\textunderscore#1}}

Every language has its own peculiarities and Lua is no exception with its metatables and metamethods. Lua’s metatables allow us to change the behaviour of a value when confronted with an undefined operation. For example when using metatables, we can define how lua computes the expression a+b, where a and b are tables. Whenever Lua tries to add two tables, it checks whether either of them has a \emph{metatable} and whether this metatable has an \textdblunderscore{add} field. If Lua finds this field, it calls the corresponding value---the so-called \emph{metamethod}, which should be a function---to compute the sum.

Lua by default always creates new tables without metatables. We can use \luacmd{setmetatable} to set or change the metatable of any table.

\section{Metatables}
Metatables allow tables to become more powerful than before. They are attached to data and contain values called metamethods. Metamethods are fired when a certain action is used with the datum that it is attached to.

You may think that if you have code like this:
\begin{verbatim}
local list = {1, 2}
print(list[3])
\end{verbatim}

The code will search through the list for the third index in list, realize it’s not there, and return nil. That’s totally wrong. What really happens is the code will search through the list for the third index, realize it’s not there, and then try to see if there’s a metatable attached to the table, returning nil if there isn’t one

\section{Manipulating Metatables}

The two primary functions for giving and finding a table’s metatable, are setmetatable and getmetatable

\begin{texexample}{Metatables}{ex:mt0}
\begin{luacode}
local x = {}
local metaTable = {}       -- metaTables are tables, too!
setmetatable(x, metaTable) -- Give x a metatable called metaTable!
tex.print(getmetatable(x)) --> table: [hexadecimal memory address]
\end{luacode}
\end{texexample}

\section{Metamethods}

Metamethods are the functions that are stored inside a metatable. They can go from
calling a table, to adding a table, to even dividing tables as well. Here’s a list
of metamethods that can be used:



\begin{texexample}{Metatables for Vector library}{ex:mt0}
\begin{luacode}
local inspect = require"inspect"
local vector2 = {__type = "vector2"}
local mt = {__index = vector2}
 
function mt.__div(a, b)
	if (type(a) == "number") then
		-- a is a scalar, b is a vector
		local scalar, vector = a, b
		return vector2.new(scalar / vector.x, scalar / vector.y)
	elseif (type(b) == "number") then
		-- a is a vector, b is a scalar
		local vector, scalar = a, b
		return vector2.new(vector.x / scalar, vector.y / scalar)
	elseif (a.__type and a.__type == "vector2" and b.__type and b.__type == "vector2") then
		-- both a and b are vectors
		return vector2.new(a.x / b.x, a.y / b.y)
	end
end
 
function mt.__tostring(t)
	return t.x .. ", " .. t.y;
end;
 
function vector2.new(x, y)
	local self = setmetatable({}, mt)
	self.x = x or 0
	self.y = y or 0
	return self
end
 
local a = vector2.new(10, 5)
local b = vector2.new(-3, 4)
 
local res = a/b

tex.print(tostring(res).."\\par")   -- -3.3333333333333, 1.25
tex.print(tostring(b / a).."\\par") -- -0.3, 0.8
tex.print(tostring(2 / a).."\\par") -- 0.2, 0.4
tex.print(tostring(a / 2)) -- 5, 2.5
tex.print("")
print(inspect(a))
\end{luacode}
\end{texexample}

It should be noted that when writing functions for either arithmetic or relational metamethods the two function parameters are interchangeable between the table that fired the metamethod and the other value. For example, when doing vector operations with scalars division is not commutative. Therefore if you were writing metamethods for your own vector2 class, you’d want to be careful to account for either scenario.\footnote{See \protect\url{https://developer.roblox.com/articles/Metatables}.}

Another trap to watch is that when you pass the results to TeX for printing, you will need to use type coersion
using |tostring|. 

\section{Arithmetic Metamethods}

In this section, we will introduce a simple example to explain how to use metatables.
Suppose we are using tables to represent sets, with functions to compute
set union, intersection, and the like, as shown in Listing 13.1. To keep our
namespace clean, we store these functions inside a table called Set.

Now, we want to use the addition operator (‘+’) to compute the union of
two sets. For that, we will arrange for all tables representing sets to share a
metatable. This metatable will define how they react to the addition operator.
Our first step is to create a regular table, which we will use as the metatable for
sets:

\begin{verbatim}
 local mt = {} -- metatable for sets
\end{verbatim}

\begin{texexample}{Metamethods}{ex:metamethods}
\begin{luacode}
local mt = {}
Set = {}
function Set.new (l)
   local set = {}
   setmetatable(set, mt)
   for _,v in ipairs(l) do set[v] = true end
   return set
end

function Set.union (a, b)
  local res = Set.new{}
  for k in pairs(a) do res[k] = true end
  for k in pairs(b) do res[k] = true end
  return res
end

function Set.intersection (a, b)
  local res = Set.new{}
  for k in pairs(a) do
  res[k] = b[k]
  end
  return res
end

-- presents a set as a string
function Set.tostring (set)
local l = {} -- list to put all elements from the set
for e in pairs(set) do
l[#l + 1] = e
end
return "{" .. table.concat(l, ", ") .. "}"
end

-- print a set
function Set.print (s)
  print(Set.tostring(s))
end
-- meta methods 
mt.__add = Set.union
s1 = Set.new{10,30,60,90}
s2 = Set.new{30,1}

-- can add the two sets
s3 = s1 + s2

Set.print(s3) --> {1, 90, 10, 30, 60}
\end{luacode}
\end{texexample}

For each arithmetic operator there is a corresponding field name in a metatable.
Besides |__add| and |__mul|, there are |__sub| (for subtraction), |__div| (for division),
|__unm| (for negation), |__mod| (for modulo), and |__pow| (for exponentiation).
We may define also the field |__concat|, to describe a behavior for the concatenation
operator.


\begin{texexample}{Metatable}{}
\begin{luacode}
require"lualibs-dir"
local t, t1 = {}, {}
  print = tex.print
  setmetatable (t, t1)

-- print(getmetatable(t)==t1)

local result = dir.found("./lua/")

if result then tex.print(-2, result) else tex.print('not found')  end

local image_fixed = img.scan({filename = "./images/korea-01.jpg"})

local image     = img.copy(image_fixed)
local halfimage = img.copy(image_fixed)

halfimage.height = halfimage.height *0.3
halfimage.width  = halfimage.width  *0.3

--node.write(img.node(image))


local h=node.hpack(img.node(halfimage), img.node(halfimage))

node.write(h)

tex.print((halfimage.width/65000)..'pt')

\end{luacode}
\end{texexample}
\includegraphics[width=0.3\textwidth,decodearray={0.2 0.2 1 0 1 0.8 1 0}]{./images/korea-01.jpg}








From Lua we can access the metatables only of tables; for any other values we must use C code. 
