\chapter{Pattern Matching}
\label{ch:patternmatching}

\section{Pattern Matching}

In Lua there are two libraries bundled with Lua that can be used for pattern matching. The ones supplied with the |string| library and the |Lpeg| library which we will describe in more detail in its own chapter later on.

The |string| library of Lua, offers a number of pattern matching functions. Unlike many other languages it does not offer a full regular expression library. One can take this at first to be a disantvantage, but the reasons are explained in \textit{Programming in Lua} by Roberto Ierusalimschy, clearly:

\begin{quotation}
Unlike several other scripting languages, Lua uses neither \texttt{POSIX} regex nor
Perl regular expressions for pattern matching. The main reason for this decision
is size: a typical implementation of \texttt{POSIX} regular expressions takes more than
4000 lines of code. This is about the size of all Lua standard libraries together.
In comparison, the implementation of pattern matching in Lua has less than
600 lines. Of course, pattern matching in Lua cannot do all that a full POSIX
implementation does. Nevertheless, pattern matching in Lua is a powerful tool,
and includes some features that are difficult to match with standard POSIX
implementations.
\end{quotation}

The  most powerful functions in the string library are \luacmd{find}, \luacmd{match}, \luacmd{gsub} (Global
Substitution), and \luacmd{gmatch} (Global Match). They all are based on patterns offering there own mini language.\\
\luacmd{string.find}()
searches for a pattern inside a given subject string. The simplest form of a \textit{pattern} is a word, which matches only a copy of itself.

For instance, the pattern ‘Lua’ will search for the substring “Lua” inside the
subject string. When |find| finds its pattern, it returns two values: the \textit{index} where the match begins and the index where the match ends. If it does not find a match, it returns |nil|. Example~\ref{ex:patterns1} demonstrates the |find| function. We use the |luacode| environment to avoid dealing with catcodes and we format the results in maths mode.

\begin{texexample}{Pattern Matching with find}{ex:patterns1}
\begin{luacode*}
s = "hello Lua world"
i, j = string.find(s,"Lua")
tex.print("indices $i="..i..",".."j="..j.."$\\par")
tex.print(string.sub(s,i,j))
\end{luacode*}
\end{texexample}

\noindent\luacmd{string.sub}
When a match succeeds, we can call string.sub with the values returned by
string.find to get the part of the subject string that matched the pattern. For
simple patterns, this is the pattern itself.

The |string.find| function has an optional third parameter: an index that
tells where in the subject string to start the search. This parameter is useful
when we want to process all the indices where a given pattern appears: we
search for a new match repeatedly, each time starting after the position where
we found the previous one. As an example, the following code makes a table
with the positions of all newlines in a string:

We will see later a simpler way to write such loops, using the \luacmd{string.gmatch}()
iterator. In the next example \ref{ex:iteratepatterns}. We search for all the indices
of |\par|. The |\par| token is escaped using |\| provided on the TeX side by |\string|.

\begin{texexample}{Iterating over all matches}{ex:iteratepatterns} 

\edef\tempstring{\string\\par..is this is a is test \string\\par and is this another?}
\begin{luacode}
local s = "\tempstring".."This is is a string and this is another"
local t = {}
local i= 0
while true do
    i = string.find(s, "\\par", i+1)
    if i == nil then break end
    t[#t + 1] = i
    tex.print(i)
end
   
\end{luacode}
\end{texexample}
The result |1, 28| indicates that we found the pattern at the beginning of the string and the next occurence we found it at position |28|.

\section{The string.match function}


The string.match function is similar to string.find, in the sense that it also
searches for a pattern in a string. However, instead of returning the positions
where it found the pattern, it returns the part of the subject string that matched
the pattern:
\begin{verbatim}
print(string.match("hello world", "hello")) --> hello
\end{verbatim}
For fixed patterns like \enquote{hello}, this function is pointless. It shows its power when
used with variable patterns, as in the next example:

\begin{texexample}{string.match}{}
\begin{luacode}
local date = "Today is 18/11/2014 is about time to get serious with LuaTeX."
local d = string.match(date, "%d+/%d+/%d+")
tex.print(d) 
\end{luacode}
\end{texexample}
The string |%d+/%d+/%d+| is named the pattern and is very similar to patterns used in regular expressions. the only difference is we need to prefix the patterns with a percent character (|%|). The |d+| is a character class meaning any digit 1 or more repititions. 

\section{The string.gsub function}

The \luacmd{string.gsub} function---which is an abbreviation for global substitution) takes three mandatory parameters and one optional parameter: the string to be searched, a pattern, and the replacement string. It can be used when we need to replace a string for all occurences of the pattern inside the subject string. The optional parameter is a number that limits the number of substitutions to be made.

\begin{texexample}{The string.gsub function}{ex:stringgsub}
\begin{luacode}
   local s = string.gsub("Y Lazarides & George Lazarides", "&", "\\textbf{and}") 
   tex.print(s)
\end{luacode}
\end{texexample}
In the example we substituted the symbol |&| with |and| and the LaTeX function |\textbf| to print it in bold.


\section{The string.gmatch function}
The string.gmatch function returns a function that iterates over all occurrences
of a pattern in a string. For instance, the following example collects all words in
a given string s:

\begin{texexample}{The string.gmatch function}{ex:gmatch}
\begin{luacode}
   local s = "For all we know this is a success"
   words = {}
   for w in string.gmatch(s, "%a+") do
      words[#words + 1] = w
   end
   tex.print(words[2])   
\end{luacode}
\end{texexample}

\section{More on Patterns}

Patterns can be more useful with \textit{character classes}. 
At the foundation of patterns are Character Classes that represent sets of characters. Each character class matches just one character that conforms to that class. For example, the character class |%d| 
matches any one digit |0-9|, whereas the character class |A| matches the single letter |A|.

More complex structures are \emph{pattern items} that represent combinations of characters. For example, the pattern item |%d+| matches a sequence of one or more digits, and |A+| matches a sequence of |A|'s.

Finally a pattern is a sequence of pattern items that may contain Captures of internal components whose purpose is explained later.

Remember the magic characters
\begin{verbatim}
 % . [ ] ^ $ ( ) * + - ? 
\end{verbatim}
 have special contextual meanings, so usually don't match themselves.
 
A summary of the classes are shown below.  
 
\begin{longtable}{>{\color{blue800}}ll}
Y & represents the character |Y| as long as it is not a \emph{magic} character\\
. & represents any character\\
\%a & letters\\
\%c &control characters\\
\%d &digits\\ 
\%g &printable characters except spaces\\
\%l &lower-case letters\\
\%p &punctuation characters\\
\%s &space characters\\
\%u  &upper-case letters\\
\%w  &alphanumeric characters\\
\%x  &hexadecimal digits\\
\end{longtable}

You can make patterns still more useful with modifiers for repetitions and optional parts. Patterns in Lua offer four modifiers:

\begin{center}
\begin{tabular}{ll}
+	&1 or more repetitions\\
*	&0 or more repetitions\\
-	&also 0 or more repetitions\\
?	&optional (0 or 1 occurrence)\\
\end{tabular}
\end{center}

Anchors can be used to force the pattern to match the begining or end of the string. (\^{}, \$). For example 
|^A%d+Z$| will match a string with A, ending with Z, with only digits in between.

\begin{texexample}{Anchoring}{ex:anchoring}
\begin{luacode}
   local str = "test.lua"
   local pattern = ".+%.lua$" -- anything until a dot and "lua" at the end of the string
   local match = string.match(str, pattern)
   tex.print("String end with "..( (match) and ".lua" or "something different") )
\end{luacode}
\end{texexample}


%https://tug.org/pipermail/luatex/2018-July/006844.html traverse an hbox
\begin{texexample}{Traversing node in an hbox}{}
\setbox0=\hbox{abc}
\begin{luacode}
  local head = tex.box[0].list
 for n in node.traverse_id(node.id("glyph"),head) do
     tex.print(string.char(n.char))
 end
\end{luacode}
\end{texexample}

\section{Performance}

%http://www.wellho.net/mouth/2727_Making-a-Lua-program-run-more-than-10-times-faster.html
For performance try and use specific targets rather than fuzzy expressions. 

Lua's native functions such as |string.find()| are extremely fast, so long as it is being used with exact matches (case sensitive, no globing, no wildcards). Once you start with wildcards it slows down considerably.

While much of the search capabilities looks somewhat like regex in spirit and format, there are significant deficiencies in the string.find() engine that separate it from regex.

\section{alternation}

Unlike regexes Lua does not have an alternation operator. Pipes ($\vert$) cannot be used to separate alternative matches. In regex you can do something like \verb+|cat|dog|horse|+ and the pattern will match a string that contains any of those 3 words. This is not possible in Lua’s |string.find()| engine. Instead, to mimic the capability you have to repeatedly search for the various strings. Something similar to the following:

\begin{texexample}{OR}{ex:or}
\begin{luacode}
   local haystack = "I have a dog and a cat."
   local search = { 'cat', 'dog', 'horse' }
   for _,v in pairs( search ) do
      if haystack:find( v ) then
         tex.print(v)
      end
   end
-- Setup locals
local str = "Hello World!"
local attempts = 1000000
local reuses = 10 -- For the second part of benchmark: Table values are reused 10 times. Change this according to your needs.
local x, c, elapsed, tbl
local print = tex.print

-- "Localize" funcs to minimize lookup overhead
local stringbyte, stringchar, stringsub, stringgsub, stringgmatch = string.byte, string.char, string.sub, string.gsub, string.gmatch

print("-----------------------\\par")
print("Raw speed:")
print("-----------------------\\par")

-- Version 1 - string.sub in loop
x = os.clock()
for j = 1, attempts do
    for i = 1, #str do
        c = stringsub(str, i)
    end
end
elapsed = os.clock() - x
print(string.format("V1: elapsed time: %.3f\\par", elapsed))

-- Version 2 - string.gmatch loop
x = os.clock()
for j = 1, attempts do
    for c in stringgmatch(str, ".") do end
end
elapsed = os.clock() - x
print(string.format("V2: elapsed time: %.3f\\par", elapsed))

-- Version 3 - string.gsub callback
x = os.clock()
for j = 1, attempts do
    stringgsub(str, ".", function(c) end)
end
elapsed = os.clock() - x
print(string.format("V3: elapsed time: %.3f\\par", elapsed))

-- For version 4
local str2table = function(str)
    local ret = {}
    for i = 1, #str do
        ret[i] = stringsub(str, i) -- Note: This is a lot faster than using table.insert
    end
    return ret
end

-- Version 4 - function str2table
x = os.clock()
for j = 1, attempts do
    tbl = str2table(str)
    for i = 1, #tbl do -- Note: This type of loop is a lot faster than "pairs" loop.
        c = tbl[i]
    end
end
elapsed = os.clock() - x
print(string.format("V4: elapsed time: %.3f\\par", elapsed))

-- Version 5 - string.byte
x = os.clock()
for j = 1, attempts do
    tbl = {stringbyte(str, 1, #str)} -- Note: This is about 15% faster than calling string.byte for every character.
    for i = 1, #tbl do
        c = tbl[i] -- Note: produces char codes instead of chars.
    end
end
elapsed = os.clock() - x
print(string.format("V5: elapsed time: %.3f\\par", elapsed))

-- Version 5b - string.byte + conversion back to chars
x = os.clock()
for j = 1, attempts do
    tbl = {stringbyte(str, 1, #str)} -- Note: This is about 15% faster than calling string.byte for every character.
    for i = 1, #tbl do
        c = stringchar(tbl[i])
    end
end
elapsed = os.clock() - x
print(string.format("V5b: elapsed time: %.3f\\par", elapsed))

print("-----------------------\\par")
print("Creating cache table ("..reuses.." reuses):\\par")
print("-----------------------\\par")

-- Version 1 - string.sub in loop
x = os.clock()
for k = 1, attempts do
    tbl = {}
    for i = 1, #str do
        tbl[i] = stringsub(str, i) -- Note: This is a lot faster than using table.insert
    end
    for j = 1, reuses do
        for i = 1, #tbl do
            c = tbl[i]
        end
    end
end
elapsed = os.clock() - x
print(string.format("V1: elapsed time: %.3f\\par", elapsed))

-- Version 2 - string.gmatch loop
x = os.clock()
for k = 1, attempts do
    tbl = {}
    local tblc = 1 -- Note: This is faster than table.insert
    for c in stringgmatch(str, ".") do
        tbl[tblc] = c
        tblc = tblc + 1
    end
    for j = 1, reuses do
        for i = 1, #tbl do
            c = tbl[i]
        end
    end
end
elapsed = os.clock() - x
print(string.format("V2: elapsed time: %.3f\\par", elapsed))

-- Version 3 - string.gsub callback
x = os.clock()
for k = 1, attempts do
    tbl = {}
    local tblc = 1 -- Note: This is faster than table.insert
    stringgsub(str, ".", function(c)
        tbl[tblc] = c
        tblc = tblc + 1
    end)
    for j = 1, reuses do
        for i = 1, #tbl do
            c = tbl[i]
        end
    end
end
elapsed = os.clock() - x
print(string.format("V3: elapsed time: %.3f\\par", elapsed))

-- Version 4 - str2table func before loop
x = os.clock()
for k = 1, attempts do
    tbl = str2table(str)
    for j = 1, reuses do
        for i = 1, #tbl do -- Note: This type of loop is a lot faster than "pairs" loop.
            c = tbl[i]
        end
    end
end
elapsed = os.clock() - x
print(string.format("V4: elapsed time: %.3f\\par", elapsed))

-- Version 5 - string.byte to create table
x = os.clock()
for k = 1, attempts do
    tbl = {stringbyte(str,1,#str)}
    for j = 1, reuses do
        for i = 1, #tbl do
            c = tbl[i]
        end
    end
end
elapsed = os.clock() - x
print(string.format("V5: elapsed time: %.3f\\par", elapsed))

-- Version 5b - string.byte to create table + string.char loop to convert bytes to chars
x = os.clock()
for k = 1, attempts do
    tbl = {stringbyte(str, 1, #str)}
    for i = 1, #tbl do
        tbl[i] = stringchar(tbl[i])
    end
    for j = 1, reuses do
        for i = 1, #tbl do
            c = tbl[i]
        end
    end
end
elapsed = os.clock() - x
print(string.format("V5b: elapsed time: %.3f\\par", elapsed))   
\end{luacode}
\end{texexample}


