
\newfontfamily\hiero{AegyptusR.ttf}
\def\texthiero#1{{\color{black!95}\hiero #1}}
\parindent=1em   

\chapter{Lua Modules}

\epigraph{People think that computer science is the art of geniuses but the actual reality is the opposite, just many people doing things that build on each other, like a wall of mini stones.}{---Donald Knuth}


\section{Modularity}

Modularity is a fundamental requirement for any computer language, as it enables the development of programs using sound software engineering principles, but most importantly it allows different people to work on the codebase simultaneously. Lua is no exception. Lua's modules are easy to develop and when they are all related, they are distributed as a package \textit{package}. They are normally distributed in one folder with submodules in subdirectories. 

For the ecosystem to survive it requires an easy method to load the packages, a place where everything is stored and easily found by loaders and some form of standardization in defining the modules. 

\section{The require Function}

In Lua the \luacmd{require} function treats a module as any code that defines some values, such as table or functions. 

To load  a module, we simply call \textbf{require"modulename"}. The first step of the \textbf{require} function is to check in table \textbf{package.loaded} whether the module is already loaded.

If the module is not loaded Lua will then search for a file name with the module name. If it finds it, it loads it with |loadfile|. 


\begin{texexample}{Findind a module}{ex:modulesearch}
\begin{luacode*}
s = string.gsub(package.cpath,"\\", "\\textbackslash ")
s = string.gsub(s,";", ";\\par ")

tex.print(s)
\end{luacode*}
\end{texexample}


In the example we use the \luacmd{string.gsub} to replace windows( \textbackslash) to a harmless version, so we do not need to worry about catcodes, a quick and dirty solution.  Of course we could have used a catcode table as well.

\section{The Basic Approach for Writing Modules in Lua}

The simplest way to create a module is to create a table and put all functions you want to export inside it, and return this table.

\begin{lualisting}
-- module chicken
-- create a module
local M = {}
-- add a function
function M.chicken ()
  return "chicken"
end
-- add another function
function M.chickens ()
  return "many chickens" 
end
-- and another
function M.ancient_chickens ()
  return "\\bgroup\\panunicode\\char\"13171 \\egroup"
end
return M
\end{lualisting}

In the example that follows, we will load the module |chicken.lua| and then just print a couple of lines.
\begin{luaexample}{Writing a Module}{ex:wmodule}
\begin{luacode*}
local c = require "chicken"
      tex.print(c.chicken(), c.chickens(), c.ancient_chickens())
      tex.print(-2, package.config)
      
      package.preload ["foo"] = function () tex.print "loading foo" end
require "foo"  --> loading foo
\end{luacode*}
\end{luaexample}


\begin{lstlisting}
package main

import (
	"fmt"
	"time"
)

func main() {
	fmt.Println("In main()")
	go longWait()
	go shortWait()
	fmt.Println("About to sleep in main()")
	// sleep works with a Duration in nanoseconds (ns) !
	time.Sleep(6 * 1e9)
	fmt.Println("At the end of main()")
}
func longWait() {
	fmt.Println("Beginning longWait()")
	time.Sleep(5 * 1e9) // sleep for 5 seconds
	fmt.Println("End of longWait()")
}
func shortWait() {
	fmt.Println("Beginning shortWait()")
	time.Sleep(2 * 1e9) // sleep for 2 seconds
	fmt.Println("End of shortWait()")
}

\end{lstlisting}



A second and in my opinion much better way is to return the list of functions you want to export. This way\footnote{This is very similar to Javascript modules.} your code will be much more cleaner and easier to maintain. We rewrite our module |chicken|  to |chickens|. Again the module is saved in a file with the module name and saved in a place where the module loader can find it. 

Once we save the file |chickens.lua|

\begin{texexample}{Writing a Module}{ex:wmodule}
\begin{luacode}
   local c = require"chickens"
   tex.print(c.chicken(), c.chickens(),"\\par","ancient chickens", "{\\Huge" .. c.ancientchickens() .. "}")
\end{luacode}
\end{texexample}



The Lua Manual also describes a way of eliminating the return statement, by assigning it directly into |package.loaded| table:

\begin{verbatim}
local M = {}
package.loaded[...] = M
\end{verbatim}

Remember that \luacmd{require} calls the loader passing the module name as the first argument.

\section{Using Environments}

One drawback of those basic methods for creating modules is that it is all too easy to pollute the global name space, for instance by forgetting a local in a private declaration.

\section{Submodules and Packages}

Lua allows module names to be hierarchical, using a dot to separate name levels. For instance, a module named |mod.sub| is a \textit{submodule} of |mod|. A \textit{package} is a complete tree of modules; it is the unit of distribution in Lua. 

From the point of view of Lua, submodules in the same package have no explicit relationship. Requiring a module a does not automatically load any of its submodules; similarly, requiring |a.b| does not automatically load |a|. Of course, the package author can create these links as she wants. For example, a particular module may start by explicitly requiring one or all of its submodules.

In LuaTeX things are much simpler if everything falls into place and the LuaTeX searchers find the modules. I experimented a bit, before getting the paths right on a MikTeX installation. Normal modules need to be located in a place where \tex can find them.

\begin{filecontents*}{test.config}
    # test.config
    # Read timeout in seconds
    read.timeout=10
    # Write timeout in seconds
    write.timeout=5
   #acceptable ports
   ports = 1002, 1003, 1004
\end{filecontents*}

\begin{texexample}{Lua Modules and Paths}{}

\begin{luacode}
  -- readconfig.lua
  
local config       = require 'pl.config'
local t               = config.read 'test.config'

tex.print("read time out = " .. t.write_timeout, "\\par",
            "ports = " .. "ports" .. "\\par")

 local Map = require"pl.Map"
 Map = require("pl.Map")
   m = Map{one=1, two=2}
   m:update{three=3, four=4,two=20}
   for k,v in pairs (m) do
       tex.print(k .. " = " .. v .. "\\par")
   end 
   
 function search (name)
    altname = string.gsub(name, '.', '\\')
    filename = kpse.find_file(altname, 'lua')	
    if not filename then
      filename = kpse.find_file(name, 'lua')
    end
    if not filename then
      return string.format("[kpse lua searcher] file not found: %s", name)
    else
      return "found" -- loadfile(filename)
    end
end
local z = search ("Map")

tex.print(z)
-- local r = kpse.show_path("lua")
-- tex.print(r)
\end{luacode}
\end{texexample}


LuaTeX uses the |kpse| library to search for files and naturally it is at home, when the files are located somewhere wher kpse can find them. In the example below, I have downloaded the penlight library into the scripts folder. The Map file can be found in:

\begin{texexample}{lookup for files}{}
\begin{luacode}
local f = kpse.lookup("pl/Map", {format = "lua"})

local f1 = kpse.lookup("Map", {format = "lua"})
tex.print("\\par",f)
tex.print("\\par",f1)
\end{luacode}
\end{texexample}

However, most Lua libraries use the dot notation as described above.



\section{The luatexbase package}

Lua's standard function \luacmd{require()} is similar to TEX’s \docAuxCommand{input} primitive but is somehow more
evolved in that it makes a few checks to avoid loading the same module twice. In the \tex
world, this needs to be taken care of by macro packages; in the \latexe world this is done by
\docAuxCommand{usepackage}

But |\usepackage| also takes care of many other things. Most notably, it implements a
complex option system, and does some identification and version checking. The present package
doesn’t try to provide anything for options, but implements a system for identification and
version checking similar to LATEX’s system.

It is important to note that Lua’s function \luacmd{module()} is deprecated in Lua 5.2 and should be
avoided. For examples of good practices for creating modules, see section 1.4. Chapter 15 of
\textit{Programming in Lua}, 3rd ed. discusses various methods for managing package

The package \pkg{luatexbase} which was developed by Heiko Oberdiek Élie Roux, Manuel Pégourié-Gonnard, Philipp Gesang∗
provides the function 

\begin{docLua}{luatexbase.require\_module(name [, required date])}

\end{docLua}

The package function which can be used as a replacement to \luacmd{require()}. The only difference between them being that the |luatexbase.require()|  will check that the module properly identifies itself.


\begin{teXXX}
local err, warn, info, log = luatexbase.provides_module({
     _name       = 'dice',
     _date       = '2014/11/01',
     _version    = '0.0',
     _description = 'simulating a die',
     _author      = 'Y Lazarides',
     _license     =  'LPPL v1.3+' ,
})

local dice =  {}    -- is this better to be local?

function dice.txprint()
    return 'dice'
end    

return dice         -- return the table returns {} 
\end{teXXX} 

%\begin{texexample}{luatexbase module}{}
%
%\begin{luacode}   
%local  f = require("dice")           
%       tex.print('return from dice module = ', f.txprint())
%\end{luacode}
%
%\end{texexample}

Working through a document is a hostile environment, and please do not follow my example. It is best to
load the files and develop them individually. However, it is prone with errors either way and one can easily get confused with local and global variables. Perseverance and patience are required, and the frustrations of \tex will diminish.





\section{Installing the penlight package}

The |penlight| package can provides some very useful routines and we will use it as an example
to understand the module system.

The easiest way to get the package is to use git, as the package can be found at Github. Use |git clone https|  to clone the directory into: 

\begin{quote}
|C:/Program Files/MikTeX 2.9/scripts/| 
\end{quote}

The |scripts| folder is used for normal packages, whereas a similar path exists for |.dlls|, which are placed in a |bin| directory. The path system is a bit of a mess and it requires a few attempts before everything falls into place. 

\begin{texexample}{Using the penlight library}{ex:penlight}
\begin{luacode}
local Map = require("pl.Map")
       m = Map{document="article", 
                     page = 2}
       m:update {font = "Calibri", fontsize = "12pt"}               
       tex.print("\\{\\par")
       for k,v in pairs (m)  do
           tex.print('~~~~',k,' = ', v..',', "\\par")
       end   
       tex.print("\\par\\}\\par") 
\end{luacode}
\end{texexample}


\section{Setting up a Database}

Apache foundation gave to the world many gifts and one of the best is couchdb. The database is especially suited for documents and this is my attempt to integrate it with LuaTeX. Installing couchdb is normally non-eventfull and I also suggest that you get an account at iriscouch.db in order to synchronize your local installation with a remote. Computing is converging to a paradigm where all software can talk to each other remotely. Another possibility is to use squlite for a more structured approach. Before we can use the couchdb we will need  a number of libraries to communicate with the db. First we will load the prerequisites, which are |logging| and a json library. 

\subsection{The logging package}

For windows navigate to |C:\Program Files\MiKTeX 2.9\scripts| and clone the repository for the logging module into logging. 

|git clone https://github.com/Neopallium/lualogging.git  logging|


 
\begin{texexample}{Using the logging module}{}
\begin{luacode}
local logging = require "logging"

local socket = require"logging.socket"

local logger = logging.new(function(self, level, message)
                             tex.print(level, message)
                             return true
                           end)
                           
logger:setLevel (logging.WARN)
logger:log(logging.INFO, "sending email")

logger:info("trying to contact server")
logger:warn("server did not responded yet")
logger:error("server unreachable")

-- dump a table in a log message
local tab = { a = 1, b = 2 }
logger:debug(tab)

-- use string.format() style formatting
logger:info("val1='%s', val2=%d", "string value", 1234)

-- complex log formatting.
local function log_callback(val1, val2)
	-- Do some complex pre-processing of parameters, maybe dump a table to a string.
	return string.format("val1='%s', val2=%d", val1, val2)
end
-- function 'log_callback' will only be called if the current log level is "DEBUG"
logger:debug(log_callback, "string value", 1234)

x = os.getenv('PATH')
s = string.gsub(x,"\\", "\\textbackslash ")
s = string.gsub(x,"_", "\\textunderscore ")
-- tex.print(s)

require("lualibs.lua")

local libloaded = require("lualibs-util-jsn")

local tmp = [[ { "a" : true, "b" : [ 123 , 456E-10, { "a" : true, "b" : [ 123 , 456 ] } ] } ]]

 tmp = utilities.json.tolua(tmp)
 
 
 -- use our own table prettifier
 
 local utils = require("phd-utils")
 
 utils.inspect(tmp)
 
 for k,v in pairs (tmp) do
    tex.print(k, " \\par ")
 end
 
inspect(tmp)
tex.print(inspect(temp))

tmp = utilities.json.tostring(tmp)

tex.print(tmp, " \\par ")


-- tmp = json.tolua(tmp)
-- inspect(tmp)
-- tmp = json.tostring(tmp)
-- inspect(tmp)

-- inspect(json.tostring(true))


local dim =   require("lualibs-util-dim")

local dimstr = string.todimen("10.0mm")

local txprint =function (v)
     				tex.print (v, '\\par')
                  end

dimstr = dimen "10pt" + dimen "20pt" + dimen "200pt" - dimen "100sp" / 10 + "20pt" + "0pt"

txprint(dimstr)

dimstr = dimen "10pt" + dimen "20pt" + dimen "200pt" - dimen "100sp" 

txprint(dimstr)

require("lualibs-gzip") 

str ="Some text that we have put in a file and zipped it."

gzip.save ("test.gz", str)

str  = gzip.load("test.gz")


txprint(str)
\end{luacode}

\end{texexample}


\section{Libraries from ConTeXt}


\subsection*{Dir}

The dir library uses functions of the lfs library that is linked into LuaTEX.
current.

This returns the current directory:

\begin{texexample}{lfs}{}
\begin{luacode}
local write = tex.sprint

write(2, dir.current())

\end{luacode}
\end{texexample}


\begin{docLua}{dir.current()}
Returns the current directory.
\end{docLua}

%\directlua{Stop = os.runtime()
%           diff = Start - Stop 
%           tex.print(Stop, os.clock())
%tex.print(os.which("ps2pdf"))}




\def\function#1{\leavevmode\noindent{\color{teal}
\parindent0pt\leavevmode\par \bfseries #1 }}
 
 
\begin{docLua}{os.which(\meta{filename})}  
This function returns the path of a file and emulates the kpse library function, with similar results. However, the function looks up the local environment path, so if a file is on the path in can find it.
\end{docLua}



\begin{docLua}{os.where()} 
This is an alias for \textbf{os.which()}
\end{docLua}



\begin{teXXX}

-- presents nicely a table 

local M = M or {}

local rep, write = string.rep, tex.print

function M.inspect (tab, offset)
   local openbracket, closebracket, par = "\\{", "\\mbox{..}\\}", "\\par"
   
    offset = offset or ""
    for k, v in pairs (tab) do
        local newoffset = offset .. "\\mbox{~~}"
        if type(v) == "table" then
           write(offset .. k .. " = " .. openbracket .. par)
           M.inspect(v, newoffset)
           write(offset .. closebracket .. par)
        else
         if k~="data" then write(offset..k.." =  ".. tostring(v), "\\par") 
           else
                 write(offset.."k = char data ")
           end
       end
    end
end

return M
\end{teXXX}

\begin{docLua}{os.which(file)}
Locates a file.
\end{docLua}

\begin{texexample}{Test}{ex:which}
\begin{luacode}
local sprint=tex.sprint
sprint(-2,os.which("tex.exe"))
tex.print("\\par")
sprint(-2,os.where("texlua.exe"))
\end{luacode}
\end{texexample}

As file names can cause problems, the |tex.print| statement is loaded with the catcode table -2. 
%\section{phd package utilities}
%
%\function{inspect()} Typesets a Lua table. 
%
%\begin{texexample}{inspecting tables}{}
%\begin{luacode}
%local utils = require"phd-utils"
%local inspect = utils.inspect
%
%inspect(\{a="b", c=\{d="man", e ="woman"\}\})
%
%local cmd = require("cmd")
%
%\end{luacode}
%
%
%\end{texexample}


