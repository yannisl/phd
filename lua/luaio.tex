\chapter{Lua i/o}
\label{ch:luaio}


The I/O library is used for reading and manipulating files in Lua. Like most programming languages there are two kinds of file operations in Lua namely implicit file descriptors and explicit file descriptors.

For the following examples, we will use a sample file test.lua as shown below.

\begin{phdverbatim}
-- sample test.lua
-- sample2 test.lua
\end{phdverbatim}

A simple file open operation uses the following statement.


\begin{docLua}{io.open (filename [, mode])}
A simple file open operation uses the following statement.\\
file = io.open (filename [, mode])
\end{docLua}

The various file modes are listed in the following table.

\begin{longtable}{ll}
Mode	& Description\\
\enquote{r}	& Read-only mode and is the default mode where an existing file is opened.\\
"w"	& Write enabled mode that overwrites the existing file or creates a new file.\\
"a"	& Append mode that opens an existing file or creates a new file for appending.\\
"r+"	& Read and write mode for an existing file.\\
"w+"	& All existing data is removed if file exists or new file is created with read write permissions.\\
"a+"	& Append mode with read mode enabled that opens an existing file or creates a new file.\\
\end{longtable}

\section{Implicit file descriptors}

Implicit file descriptors use the standard input/ output modes or using a single input and single output file. A sample of using implicit file descriptors is shown below.

\begin{texexample}{I/O Operations}{ex:iolua}
\begin{luacode}
fileHandle =io.open("file.txt", "w+")
tex.print(io.type(fileHandle).."\\par")
fileHandle:write("Hello world")
io.close(fileHandle)
tex.print(io.type(fileHandle).."\\par")
\end{luacode}
\end{texexample}

\begin{docLua}{io.flush()}
This function runs |file:flush()| on the default file. When a file is in buffered mode, the data is not
written to the file immediately; it remains in the buffers and is written when the buffer nears getting
full. This function forces the buffers to write to file and clear up.
\end{docLua}

\begin{docLua}{io.input([file])}
This function returns the current default input file when called without any parameters. The parameter
passed can be either a filename or a file handle. When the function is called with a filename, it sets
the handle to this named file as the default input file. If it is called with a file handle, it just sets the
default input file handle to this passed file handle.
\end{docLua}


\begin{docLua}{io.tmpfile()}
This function returns a handle for a temporary file; the file is opened in update mode and removed
automatically when the program ends.
\end{docLua}

The use of a temporary file, is useful in many situations, where we need to write something temporarily
and then read it later on in the program. 

\begin{texexample}{Temporary files}{}
\begin{luacode}
if type(tex)==table then print=print.tex end
fh = io.tmpfile()
fh:write("some sample data on page \\thepage ")
fh:flush()
-- Now let's read the data we just wrote
fh:seek("set", 0)
content = fh: read("*a")
print("We got: ", content)
\end{luacode}
\end{texexample}

\begin{docLua}{io.type(obj)}
This function returns the result if the handle specified by obj is a valid file handle. It returns the string
``file'' if |obj| is an open file handle and the string ``closed file'' if |obj| is a closed file handle. It
returns |nil| if |obj| is not a file handle.
\end{docLua}

\begin{texexample}{Checking the type}{}
\begin{luacode}
if type(tex)==table then print=print.tex end
fh = io.input()
if fh then
    tex.print(io.type(fh)) 
  else
    tex.print("nil")
end
\end{luacode}
\end{texexample} 

 
\begin{filecontents*}{input.txt}
line 1
line 2
line 3
line 4
line 5
\end{filecontents*}


\section*{How to read a file line by line}

Reading a file line by line is easy using an iterator.

\begin{texexample}{Read a file line by line}{}
\begin{luacode}
if type(tex)=='table' then print=tex.print end
for line in io.lines("input.txt") do
  print(line, "\\par")
end
\end{luacode}
\end{texexample}
\endinput
%\newfontfamily\hiero{NotoSansEgyptianHieroglyphs-Regular.ttf}

\section{A more practical example}
In the sections of the documentation, where we describe
the use of unicode for scripts, we provide files with almost
all unicode blocks we describe. For example the file
|hieroglyphics.txt| provides the details for the first
hieroglyphic characters |U+13000-130FF|.

This is parsed using \tex, but could equally well and perhaps easier and more robustly be parsed using Lua.  


\begin{texexample}{Reading a file line by line}{}
\begin{luacode*}
filename = "./languages/hieroglyphics.txt"
fp = io.open( filename, "r" )
for line in fp:lines() do
    tex.print(line..", ")
   end
fp:close()
\end{luacode*}
\end{texexample}

In the second example, we start building our programme
by defining functions to get a glyph using a particular font
command and to typeset it.

The function |getglyph()| takes two arguments, the first is the
name of a command for example for the control sequence \cmd{\hiero} the argument is  ``hiero" and a unicode codepoint in hexadecimal digits (as a string).

\begin{texexample}{Glyph functions}{}
\begin{luacode*}
function getglyph(cmd, codepoint)
  local texstring = "\\Large\\"..cmd.." \\char".."\""..codepoint
  return texstring
end

function printglyph(cmd, codepoint)
  tex.print(getglyph(cmd,codepoint))
end

printglyph("hiero","13000")
printglyph("hiero","13050")
\end{luacode*}
\end{texexample}

You could of course argue, that this is easily be done so far using \tex programming, but soon we will get into areas of the parser that it would be difficult to achieve using \tex only. With Lua also it is easier to add error detection to make our parser more robust.

Back to our parser and hieroglyphics another function we need to develop is a function that can take care of the writing direction. Hieroglyphics, like every written language,  needed conventions to keep writings consistent and readable. For instance English is always read left to right $\rightarrow$. 

Hieroglyphic writing was written in columns or rows. Reading direction is determined by the direction that human and animal figures faced. Reading starts from the direction that figures face and continues in the opposite direction. If the figures look left the direction is left to right and vice versa.

\begin{center}
\bgroup
$\rightarrow$

\scalebox{2}[2]{\hiero\char"13000\char"13011\char"13020}

$\leftarrow$

\scalebox{-2}[2]{\hiero\char"13000\char"13011\char"13020}

\scalebox{2}[2]{\hiero
\char"13000
\char"13011
\char"13020}
\egroup
\end{center}

Vertical texts were written from top to bottom, like the figure shown below. As these were composed from single hieroglyphs the direction the figures are facing is immaterial, other than for symmetry.

\begin{center}
{\hiero
\makebox[3em]{\scalebox{-2}[2]{\hiero
\char"13001}\hss}
\vskip3pt
\makebox[3em]{\scalebox{-2}[2]{\hiero
\char"13006}\hss}
\vskip3pt
\makebox[3em]{\scalebox{-2}[2]{\hiero
\char"13007}\hss}
}
\end{center}

Columns were read down as we would read lines down a page. The Egyptians liked symmetry. If hieroglyphs were inscribed in a column, they would often inscribe the same text in the opposite column, except with the writing reversed. When written vertical they were accompanied by drawings to illustrate the words, pretty much like we do today in publications.

\begin{figure}[hb]
\centering
\includegraphics[width=0.7\textwidth]{./images/senet.jpg}
\caption{Nefertari playing Senet. Painting in tomb of Egyptian Queen Nefertari (1295–1255 BC). \href{http://en.wikipedia.org/wiki/Senet}{Wikimedia}}
\end{figure}



\begin{docCommand}{scalebox} {\oarg{[]} \oarg{[]} \marg{box contents}}
It is much easier to use \tex to achieve the glyph rotation. The package \pkgname{graphicx} provides the command \cmd{\scalebox} which with suitable parameters can achieve this (\seedocs{graphicx}).
\end{docCommand}

In our Lua script we add the functions |getglyphRL()| and |printglyphRL()| to cater for direction. We can also add a helper function for top to bottom scripts:

\begin{verbatim}
function getglyphRL(cmd, codepoint)
  local texstring = "\\scalebox{-5}[5]{\\"..cmd.." \\char".."\""..codepoint.."}"
  return texstring
end

function printglyphRL(cmd, codepoint)
  tex.print(getglyphRL(cmd,codepoint))
end
\end{verbatim}

\section{Parsing transliteration strings}

The input of hieroglyphic texts presents a challenge to computer science. One cannot make a keyboard of hundreds of characters and which in any case would be difficult to use. 
Most Egyptologists tranliterate hieroglyphics, using a scheme that is described in more detail in the Chapter \textit{Egyptian Hieroglyphics}.

A transliteration script consists at is basic form of alphanumeric numbers separated by a "-" to separate the glyphs. It uses the Gardiner classification for numbering the glyphs.

\begin{center}
\texttt{A1*-R2*-R3*-F1*!}
\end{center}

Other symbols such as ":,\^\_*!" are used to denote how the scripts are assembled etc.

\begin{description}
\item [dash]
The dash (-) is used to indicate the end of a glyph. In reality many people in the field use a space which is a good as any symbol. 
\item [Exclamation mark] The exclamation mark indicates a line break. Two exclamation marks(!!) indicate a page break.

\item [Asterisk] an asterisk (*) indicates that the glyph is placed on the same row as the preceding one.

\item[Colon] the colon (:) indicates that the glyph is stacked under the preceding one; glyphs will continue to be stacked under until a stopping command is found (- or ! or !!).

\end{description}

Getting the TeX side of things right is also important, as we may need to scale the glyphs. There are many ways to scale the glyphs; we can put them in boxes and use scalebox. Another good option is to use font sizing commands.

\begin{center}

{\hiero\fontsize{2cm}{0.5em}\selectfont \char"13000}
{\hiero\fontsize{1cm}{0.25em}\selectfont 
\parbox[b]{1cm}{\hiero\char"13000\\ \char"13080}
}
{
\hiero\fontsize{0.66cm}{0.5em}\selectfont 
\fbox{\parbox[b]{1cm}{\hiero\char"13030\\ \char"13080\\ \char"133E5}}
\hiero\fontsize{2cm}{0.5em}\selectfont \char"13000
}



\end{center}

Once the font sizing commands are ready, it is a simple matter to use a stacking mechanism to stack them. If you recall from a parbox can be displayed inline--as far as \tex is concerned it is just another box to be typeset.  

\section{boxes}

\begin{texexample}{Reading values of a box}{}
\bgroup
\setbox0=\hbox{A\hskip 5pt B\hskip 10pt C}
\noindent Using \TeX{} code, box 0 has width \number\wd0\relax \space sp\par
\noindent We can also use Lua and call one of Lua\TeX's functions to get the same
information.\vskip10mm
\noindent From Lua code, box 0 has width 

\directlua{
local boxwidth = tex.box[0].width
tex.print(boxwidth.." sp")
} which, of course, is identical to the value obtained from \TeX{} code.

\egroup
\end{texexample}




