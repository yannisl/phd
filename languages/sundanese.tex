\section{Sundanese}
\epigraph{The married women, when their husband die, must, as point of honour, die with them, and if they should be afraid of death they put into the convents.}{Tomé Pires \textit{Suma Oriental} (1512–1515)}

\label{s:sundanese}

\newfontfamily\sundanese{Noto Sans Sundanese}
The Sundanese (Sundanese: {\sundanese ᮅᮛᮀ ᮞᮥᮔ᮪ᮓ}, Urang Sunda) are an ethnic group native to the western part of the Indonesian island of Java. They number approximately 40 million, and are the second most populous of all the nation's ethnicities. The Sundanese are predominantly Muslim. In their own language, Sundanese, the group is referred to as Urang Sunda, and Orang Sunda or Suku Sunda in the national language, Indonesian.

The Sundanese have traditionally been concentrated in the provinces of West Java, Banten, Jakarta, and the western part of Central Java. Sundanese migrants can also be found in Lampung and South Sumatra. The provinces of Central Java and East Java are home to the Javanese, Indonesia's largest ethnic group.

\begin{figure}[htbp]
\centering
\includegraphics[width=\linewidth-2\parindent]{sundanese}

\caption{Sundanese boys playing Angklung in Garut, c. 1910–1930. \href{https://en.wikipedia.org/wiki/Sundanese_people}{wikipedia}}
\end{figure}

The Sundanese script (Aksara Sunda, {\sundanese ᮃᮊ᮪ᮞᮛ ᮞᮥᮔ᮪ᮓ}) is a writing system which is used by the Sundanese people. It is built based on Old Sundanese script (Aksara Sunda Kuno) which was used by the ancient Sundanese between the 14th and 18th centuries.



\begin{scriptexample}[]{Sundanese}
\unicodetable{sundanese}{"1B80,"1B90,"1BA0,"1BB0}

\sundanese
\obeylines
\bgroup
᮱ {\arial= 1}	᮲ {\arial= 2}	᮳{\arial = 3}
᮴ {\arial= 4}	᮵ {\arial = 5} 	᮶ {\arial= 6}
᮷ {\arial= 7}	᮸ {\arial= 8}	᮹ {\arial= 9}
᮰ {\arial= 0}

\egroup
\end{scriptexample}

\begin{scriptexample}[]{Sundanese}
\bgroup
\sundanese
\centering

◌ᮃᮄᮅᮆᮇᮈᮉᮊᮋᮌᮍᮎᮏᮐᮕᮔᮓᮑᮖᮗᮚᮛᮜᮝᮞᮟᮠᮠ


\egroup
\end{scriptexample}

\bgroup
\def\1{\sundanese ᮱}
\TextOrMath\1\1

$\1$
\egroup

In text In texts, numbers are written surrounded with dual pipe sign \textbar \ldots \textbar. Example: {\textbar \sundanese ᮲᮰᮱᮰ }\textbar = 2010











