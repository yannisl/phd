\newfontfamily\cjk{NotoSerifCJK-Regular.ttc}
\index{Katakana}\index{Hiragana}
\index{Bopomofo}\index{Hangul}\index{Yi}
\index{East Asian Scripts>Katakana}
\index{East Asian Scripts>Hiragana}
\index{East Asian Scripts>Hangul}
\index{East Asian Scripts>Bopomofo}
\index{East Asian Scripts>Yi}
\index{scripts>cjk}
\pagestyle{headings}
\index{Yi fonts>Microsoft Yi Baiti}
\chapter{East Asian Scripts}
\cxset{epigraph width=0.7\linewidth}
\epigraph{

For writing is the foundation of the classics and the arts, the beginning of
royal government. It is the means by which people of the past reach posterity,
by which people of the future know the past. 

{\cjk 蓋文字者,經藝之本,王政之始。前人所以垂後,\\ 後人所以識古。}
}{ Xu Shen  in the ``Postface'' of the \emph{Shuowen}}

\bigskip

\noindent This chapter presents the most common scripts currently in use in East Asia. This includes Chinese, Japanese and Korean. It also discusses several scripts for minority languages spoken in southern China. The scripts discussed are as follows:


\begin{center}
\begin{tabular}{lll}
\nameref{s:han} &Hiragana &Hangul\\
\nameref{s:bopomofo} &Katakana &\nameref{s:yi}\\
\end{tabular}
\end{center}
\bigskip

\parindent1em

\paragraph{Putonghua}The national language of China is Putonghua (Modern Standard Chinese), a standardized version of the Beijing dialect of Mandarin Chinese. As described earlier, there are hundreds of other regional
languages spoken in China, normally referred to as dialects or dialect
groups. Since the founding of the People’s Republic of China in 1949,
knowledge and use of Putonghua has been successfully promoted across
the country by a range of government measures, especially in education.
Most of the population are able to speak the language, and an even
greater percentage, perhaps as many as 90 per cent, can understand it
(Chen 1999: 27–30).

There are more than fifty-five officially recognized minority nationalities,
speaking scores of languages of the Tibeto-Burman, Tai, and Hmong-Mien
families in the south, and the Altaic family in the north (Ramsey 1987:
chs. 10 and 11; Blum 2002). In geographical terms the most widespread are
Uighur/Uyghur, Mongolian, and Tibetan, but these lie outside the geographical
area covered by this book. The greatest degree of linguistic
diversity is in the south and southwest in the provinces of Guangxi,
Guizhou, and Yunnan. Population-wise, the largest non-Sinitic language is
Zhuang (Tai), mainly in Guangxi province, where it has some official
functions. Rather confusingly, there is no one-to-one match between ethnic
nationality names and language names; for example, the nationality
identified as Yi contains speakers of several distinct languages (including,
notably, Lolo). Under the Chinese constitution the national minorities
all have ‘‘the freedom to use and develop their own spoken and written
languages’’, but in practice official support mostly goes to the larger
minorities.


\section{History of the Language}

\epigraph{On the other hand, even well educated people
could not understand the secret meaning of the Taoist characters (fu’s{\cjk 符}), but they
are convinced of the power of those figures.}{---Alex Chengyu Fang and François Thierry, \textit{The Language
and Iconography
of Chinese Charms
Deciphering a Past Belief System} }

The relationship between Chinese and other Sino-Tibetan languages is an area of active research and controversy, as is the attempt to reconstruct Proto-Sino-Tibetan. The main difficulty in both of these efforts is that, while there is very good documentation that allows for the reconstruction of the ancient sounds of Chinese, there is no written documentation of the point where Chinese split from the rest of the Sino-Tibetan languages. This is actually a common problem in historical linguistics, a field which often incorporates the comparative method to deduce these sorts of changes. Unfortunately the use of this technique for Sino-Tibetan languages has not as yet yielded satisfactory results, perhaps because many of the languages that would allow for a more complete reconstruction of Proto-Sino-Tibetan are very poorly documented or understood. Therefore, despite their affinity, the common ancestry of the Chinese and Tibeto-Burman languages remains an unproven hypothesis.[1]

Categorization of the development of Chinese is a subject of scholarly debate. One of the first systems was devised by the Swedish linguist Bernhard Karlgren in the early 1900s. The system was much revised, but always heavily relied on Karlgren's insights and methods.

\begin{figure}[htbp]
\centering

\includegraphics[width=0.45\linewidth]{oracle}

\end{figure}

Oracle bones (Chinese: {\cjk 甲骨}; pinyin: {\cjk jiǎgǔ}) are pieces of ox scapula or turtle plastron, which were used for pyromancy – a form of divination – in ancient China, mainly during the late Shang dynasty. Scapulimancy is the correct term if ox scapulae were used for the divination; plastromancy if turtle plastrons were used.

Diviners would submit questions to deities regarding future weather, crop planting, the fortunes of members of the royal family, military endeavors, and other similar topics.[1] These questions were carved onto the bone or shell in oracle bone script using a sharp tool. Intense heat was then applied with a metal rod until the bone or shell cracked due to thermal expansion. The diviner would then interpret the pattern of cracks and write the prognostication upon the piece as well.[2] By the Zhou dynasty, cinnabar ink and brush had become the preferred writing method, resulting in fewer carved inscriptions and often blank oracle bones being unearthed.

The oracle bones bear the earliest known significant corpus of ancient Chinese writing[a] and contain important historical information such as the complete royal genealogy of the Shang dynasty.[b] When they were discovered and deciphered in the early twentieth century, these records confirmed the existence of the Shang, which some scholars had until then doubted.

Old Chinese, sometimes known as ``Archaic Chinese'', was the language common during the early and middle Zhou Dynasty (1122–256 BC), texts of which include inscriptions on bronze artifacts, the poetry of the Shijing, the history of the Shujing, and portions of the Yijing (I Ching). The phonetic elements found in the majority of Chinese characters also provide hints to their Old Chinese pronunciations. The pronunciation of the borrowed Chinese characters in Japanese, and Vietnamese also provide valuable insights. Old Chinese was not wholly uninflected. It possessed a rich sound system in which aspiration or rough breathing differentiated the consonants, but probably was still without tones. Work on reconstructing Old Chinese started with Qing dynasty philologists.

Middle Chinese was the language used during the Sui, Tang and Song dynasties (6th through 10th centuries AD). It can be divided into an early period, reflected by the Qieyun rime dictionary (AD 601) and its later redaction the Guangyun, and a late period in the 10th century, reflected by rime tables such as the Yunjing. The evidence for the pronunciation of Middle Chinese comes from several sources: modern dialect variations, rime dictionaries, foreign transliterations, rime tables constructed by ancient Chinese philologists to summarize the phonetic system, and Chinese phonetic translations of foreign words. However, all reconstructions are tentative; for example, scholars have shown that trying to reconstruct modern Cantonese from the rimes of modern Cantopop would give a very inaccurate picture of its pronunciation.

The development of the spoken Chinese from early historical times to the present has been complex. Most Chinese people, in Sichuan and in a broad arc from the northeast (Manchuria) to the southwest (Yunnan), use various Mandarin dialects as their home language. The prevalence of Mandarin throughout northern China is largely due to north China's plains. By contrast, the mountains and rivers of southern China promoted linguistic diversity.

Until the mid-20th century, most southern Chinese only spoke their native local variety of Chinese. However, despite the mix of officials and commoners speaking various Chinese dialects, Nanjing Mandarin became dominant at least during the Qing Dynasty. Since the 17th century, the Empire had set up orthoepy academies (simplified Chinese:{\cjk 正音书院}; traditional Chinese: {\cjk 正音書院}; pinyin: Zhèngyīn Shūyuàn) to make pronunciation conform to the Qing capital Beijing's standard, but had little success. During the Qing's last 50 years in the late 19th century, the Beijing Mandarin finally replaced Nanjing Mandarin in the imperial court. For the general population, although variations of Mandarin were already widely spoken in China then, a single standard of Mandarin did not exist. The non-Mandarin speakers in southern China also continued to use their local languages for each and every aspect of life. The new Beijing Mandarin court standard was thus fairly limited.

This situation changed with the creation (in both the PRC and the ROC, but not in Hong Kong and Macau) of an elementary school education system committed to teaching Modern Standard Chinese (Mandarin). As a result, Mandarin is now spoken by virtually all people in mainland China and on Taiwan[citation needed]. At the time of the widespread introduction of Mandarin in mainland China and Taiwan, Hong Kong was a British colony and Mandarin was never used at all. In Hong Kong, Macau, Guangdong and sometimes Guangxi, the language of daily life, education, formal speech and business remains in the local Cantonese. However, Mandarin is becoming increasingly influential, which is seen as a threat by the locals, fearing that their native language might face a decline leading to its death





Settings for |cjk| languages and scripts follow:

\begin{docKey}[phd]{cjk font}{\meta{font name}}{default none, initial code2000.ttf}
This key when set produces all necessary command to set the font for cjk typesetting.
\end{docKey}

If your document is going to be primarily in chinese, you better off to use a dedicated class or package for the whole document. 

The easiest way is (for Simplified Chinese document only):

\begin{dispListing}
% UTF-8 encoding
% Compile with latex+dvipdfmx, pdflatex, xelatex or lualatex
% XeLaTeX is recommanded
\documentclass[UTF8]{ctexart}
\begin{document}
文章内容。
\end{document}
\end{dispListing}

or

\begin{dispListing}
\documentclass{article}
\usepackage[UTF8]{ctex}
...
\end{dispListing}


\begin{dispListing}
% Compile with xelatex
% UTF-8 encoding
\documentclass{article}
\usepackage{xeCJK}
\setCJKmainfont{SimSun}
\begin{document}
文章内容
\end{document}
\end{dispListing}


\parindent1em
\section{Han CJK Unified Ideographs}
\label{s:han}
\index{CJK}
The Chinese, Japanese and Korean (CJK) scripts share a common background. In the process called Han unification the common (shared) characters were identified, and named ``CJK Unified Ideographs''. Unicode defines a total of 74,617 CJK Unified Ideographs.[1]\footnote{\protect\url{http://shahon.org/wp-content/uploads/2010/02/Galambos-2006-Orthography-of-early-Chinese-writing.pdf}}



The terms ideographs or ideograms may be misleading, since the Chinese script is not strictly a picture writing system.
Historically, Vietnam used Chinese ideographs too, so sometimes the abbreviation ``CJKV" is used. This system was replaced by the Latin-based Vietnamese alphabet in the 1920s.

\section{Development of the script}

In the Oracle Bone and early bronze scripts, some but not all of the originally pictographic
characters were already stylized beyond recognition. There was great variation in the writing
of individual characters, and in the strokes used to render them. The subsequent development
of the script is a process of stylization, standardization, and reduction of the process of writing
to the repetition of a small number of stereotyped motions (strokes). Curved lines became
straight or angled, and pictographic iconicity was completely eliminated. 


Following the political unification of China by the first Qin emperor (221 BCE), a standard
script was imposed in place of the regional variants that had sprung up. The regularization of
the script continued into the Han, by which time the more or less modern script had emerged.
Pre-modern forms are still used in some contexts for aesthetic reasons, and various cursive
forms have emerged both as convenient shorthands and as calligraphic art forms, but the Kai
script of the Han dynasty has survived as the model for all subsequent Chinese writing. The
most recent change has been the official PRC simplifications of the 1950s, which reduced the
number of strokes in many characters without fundamentally altering the basic principles
of the script (in many cases by merely giving official blessing to folk shorthand characters).
An example of the historical progression can be seen in Figure~\ref{horsechar}. 

\begin{figure}[htbp]
\includegraphics[width=\textwidth]{chinese-horse-char}
\caption{Evolution of the ma ideogram}
\label{fig:horsechar}
\end{figure}

\unicodetable{cjk}{"4E00,"4E10,"4E20,"4E30,"4E40}




\section{Bopomofo}
\label{s:bopomofo}
Bopomofo is the colloquial name of the \textit{zhuyin fuhao} or \textit{zhuyin} system of phonetic notation for the transcription of spoken Chinese, particularly the Mandarin dialect. Consisting of 37 characters and four tone marks, it transcribes all possible sounds in Mandarin. 

Bopomofo was introduced in China by the Republican Government, in the 1910s and used alongside the Wade-Giles system, which used a modified Latin alphabet. The Wade system was replaced by \textit{Hanyu Pinyin} in 1958 by the Government of the People's Republic of China,[1] at the International Organization for Standardization (ISO) in 1982 (ISO 7098:1982). Bopomofo remains widely used as an educational tool and electronic input method in Taiwan. On Windows the font Microsoft JhengHei can be used. 

Windows fonts that can be used \texttt{Microsoft JhengHei} and \texttt{SimSun}.

U+3100–U+312F
\newfontfamily\bopomofo{Microsoft JhengHei}

\begin{scriptexample}[]{Bopomofo}
{\centering\bopomofo 

伯帛勃脖舶博渤霸壩灞

}

\hfill \texttt{Typeset with \cmd{\bopomofo} and Microsoft JhengHei font }
\end{scriptexample}

\begin{scriptexample}[]{Bopomofo}

{\centering\bopomofo

伯帛勃脖舶博渤霸壩灞

}
\hfill \texttt{Typeset with \cmd{\bopomofo} and JhengHei font }
\end{scriptexample}


The Bopomofo Extended block, running from \unicodenumber{U+31A0-U31BF}, contains less universally recognized Bopomofo characters used to write various non-Mandarin Chinese languages. A few additional tone marks are unified with characters in the Spacing Modifier Letters block. 












\section{Yi}
\label{s:yi}

The Yi script (Yi: {\yi ꆈꌠꁱꂷ} nuosu bburma [nɔ̄sū bū̠mā]; Chinese: {\cjk 彝文}; pinyin: Yí wén) is an umbrella term for two scripts used to write the Yi language; Classical Yi, an ideogram script, the later Yi Syllabary. The script is also historically known in Chinese as Cuan Wen (Chinese: {\cjk 爨文}; pinyin: Cuàn wén) or Wei Shu (simplified Chinese: {\cjk 韪书}; traditional Chinese: {\cjk 違書}; pinyin: Wéi shū) and various other names ({\cjk 夷字、倮語、倮倮文、毕摩文}), among them "tadpole writing" ({\cjk 蝌蚪文}).[1]

This is to be distinguished from romanized Yi ({\yi 彝文罗马拼音} Yiwen Luoma pinyin) which was a system (or systems) invented by missionaries and intermittently used afterwards by some government institutions.[2][3] There was also a Yi abugida or alphasyllabary devised by Sam Pollard, the Pollard script for the Miao language, which he adapted into "Nasu" as well.[4][5] Present day traditional Yi writing can be sub-divided into five main varieties (Huáng Jiànmíng 1993); Nuosu (the prestige form of the Yi language centred on the Liangshan area), Nasu (including the Wusa), Nisu (Southern Yi), Sani ({\yi 撒尼}) and 
Azhe ({\yi 阿哲}).[6][7]


The Yi or Lolo people[3] are an ethnic group in China, Vietnam, and Thailand. Numbering 8 million, they are the seventh largest of the 55 ethnic minority groups officially recognized by the People's Republic of China. They live primarily in rural areas of Sichuan, Yunnan, Guizhou, and Guangxi, usually in mountainous regions. As of 1999, there were 3,300 "Lô Lô" people living in the Hà Giang, Cao Bằng, and Lào Cai provinces in northeastern Vietnam.
The Yi speak various Loloish languages, Sino-Tibetan languages closely related to Burmese. The prestige variety is Nuosu, which is written in the Yi script.

\begin{figure}[htbp]
\includegraphics[width=\linewidth-2\parindent]{yi}

\caption{Yi people in traditional costumes. \href{https://www.dreamstime.com/stock-photos-yi-minority-women-traditional-clothes-image25450383}{dreamsite}}
\end{figure}


The Unicode block for Modern Yi is Yi syllables (U+A000 to U+A48C), and comprises 1,164 syllables (syllables with a diacritic mark are encoded separately, and are not decomposable into syllable plus combining diacritical mark) and one syllable iteration mark (U+A015, incorrectly named YI SYLLABLE WU). In addition, a set of 55 radicals for use in dictionary classification are encoded at U+A490 to U+A4C6 (Yi Radicals).[11] Yi syllables and Yi radicals were added as new blocks to Unicode Standard Version 3.0.[12]

Classical Yi - which is an ideographic script like the Chinese characters - has not yet been encoded in Unicode, but a proposal to encode 88,613 Classical Yi characters was made in 2007.[13]

\bgroup
\yi \char"A000: Yi Syllable It\\

\yi \char"A001: Yi Syllable Ix\\

\yi \char"A002: Yi Syllable I\\
\egroup

\begin{scriptexample}[]{Yi}
\unicodetable{yi}{"A000,"A010,"A020,"A030,"A040,"A050,"A060,"A070,"A080,"A090,"A0A0,"A0B0,"A0C0}
\end{scriptexample}


