\chapter{Luwian}
\newfontfamily\anatolian{Anatolian}

\section{Introduction}

Thanks to the over 33,000 documents from Hattuša, the capital of the Hittite Kingdom, linguists have been able to gain a comprehensive insight into Luwian culture. Some fundamental publications include the book Arzawa, by Susanne Heinhold-Krahmer (1977); The Luwians, edited by H. Craig Melchert (2003)\footcite{Melchert2003}; and Luwian Identities, edited by Alice Mouton and others (2013).\footcite{Mouton2013} Field-oriented excavating archaeologists, on the other hand, never mention Luwians in their explanatory models. The current knowledge regarding the Aegean Bronze Age has been summarized in a number of recently published voluminous works, without attention to any Luwian culture. 

 


\section{Hieroglyphic Script}
Anatolian Hieroglyphs is a Unicode block containing Anatolian hieroglyphs, used to write the extinct Luwian language, because they first appeared on personal seals from Hattusha, the capital of the Hittite Empire. While
Hittites did make use of the characters on seals and on their monumental inscriptions, the characters were
used as text primarily for the related language Luwian; a few glosses in Urartian and some divine names
in Hurrian are known to be written in Anatolian Hieroglyphs. Most of the texts are monumental stone
inscriptions, though some letters and accounting documents have been preserved inscribed on strips of
lead.

The hieroglyphic script is used to record the Luwian language with the help
of pictorial signs. These are written “boustrophedon” or “as the ox ploughs,”
alternating their direction from line to line. As with Egyptian, one reads a line into the faces of the people and animals. Lines are divided from one another by horizontal rules. Within a line, characters are written in vertical columns, though for aesthitic reasons they are sometimes placed out of phonetic or logical order. For modern, lexigographical use it is  normally written as left-to-right.


In structure if not in appearance, the
writing system closely resembles the cuneiform script, distinguishing likewise
three different sign types, logograms, determinatives, and syllabograms. A logogram
represents an entire word with just one picture. In its simplest form, the
glyph depicts the object drawn but it may also depict an object associated with
the intended word, such as the king’s hat as a symbol for the king, or a word of
similar sound. Determinatives are signs used to mark words as belonging to a
specific sphere, maybe comparable to using titles such as Mr. and Mrs. to signify
gender, but extending to a number of categories. Many of these we can understand
while the logic of others eludes us. Syllabic signs are used to represent the
sound of the word written with them; in the hieroglyphic writing system, these
phonetic signs have the structure vowel (V), consonant-vowel (CV) or consonant-
vowel-consonant-vowel (CVCV); a very small number of signs appears not
to adhere to this pattern. Logograms and syllabic signs may be used exclusively
or in combination, thus a word could be written with the logogram—with or
without a phonetic complement, that is, the word end spelled phonetically—with
logogram and full phonetic writing or written purely with phonetic signs. This
type of writing poses one particular problem to modern readers: the practice of
logographic writing may hide the underlying Luwian term, either partially or
completely. Some signs may have either a logographic or phonetic reading, and,
accordingly, have to be interpreted in the context in which they occur.

\begin{figure}[htbp]
\includegraphics[width=0.7\textheight]{katuwa}
\caption{Reliefed orthostat with HLuwian inscription and portrait of the ruler Katuwa from
Carchemish, after Woolley 1921 pi. A13d, \protect\cite{Melchert2003}}
\end{figure}

Visually, the script is called hieroglyphic because it depicts objects, some
of which we can easily identify while others still defy recognition. The majority
of hieroglyphic Luwian inscriptions survives on monuments of stone, and 
there we find two ways of writing: either the signs were incised into a smooth
surface, or the background was chiselled away so that the signs appear in relief.
Among the signs themselves one can differentiate two shapes, a more pictorial,
formal shape, and a more linear, cursive one. Scholars interpret the latter as a
sign of increased handwritten usage, in the same way that the Egnguageyptians used
cursive hieroglyphs on papyrus. That the cursive sign forms that appear on stone
monuments reflect the handwritten variant of the script is born out by the few
surviving handwritten documents. Mainly, these are inscribed strips of lead. But
as very little is preserved outside of the corpus of monumental stone inscriptions,
the development of the script as a handwritten medium is, unfortunately, largely
lost to us.

\paragraph{Punctuation and spacing} In some texts, word division is indicated with the use of {\anatolian\char"145A3}.

\unicodetable{anatolian}{"14400,"14410,"14420,"14430,"14440,"14450,"14460,"14470,"14480,"14490,"144A0,"144B0,"144C0,"144C0,"144D0,"144E0,"144F0,%
 "14500,"14510,"14520,"14530,"14540,"14550,"14560,"14570,"14580,"14590,"145A0,"145B0,"145C0,"145D0,"145E0,"145F0,%
 "14600,"14610,"14620,"14630,"14640}
 
 
\section{Transliteration} 

Current transliteration schemes follow Hawkins, who based his system on Laroche's. Hawkins wrote:

\begin{quotation}
Laroche's system, as has been noted, is about a logical as may be devised for an essentially unsystematic,
script, and it is a system which I and others have used, with the necessesary modifications, over the last twenty-five years. In order to minimize discontinuity, I have opted in the present Corpus against an attempt to devise a new system, and have preferred to adopt and adapt that of Laroche. 
\end{quotation}


\subsection{Sign Lists}

The hieroglyphic script consists of over 500 signs, some with multiple values, which function as 1) logograms, 2) determinatives, and 3) syllabograms, or a combination of these\footcite{Payne2014}. Series of signs either logographic or syllabic, are numbered beginning with the most frequent sign, e.g. NEG, NEG2, NEG3, or sa, s\'a (sa2), s\`a, sa4, sa5. The general principles of tranliteration are indicated under the respective sign types.

All signs are accorded a number basd on Laroche's sign list, commonly denoted as L.No. or *No. An example is shown below:


\def\PES{\bgroup\anatolian\scalebox{1}{\large\char"14463}\egroup\xspace}
\let\pes\PES

\begin{center}
\pes *90 PES \textit{ti}
\end{center} 


The foot, number ninety, has the logographic values \PES, \enquote{foot} and represents the syllable \textit{ti}.

The most up-to-date lists are those published by Marazzi. The earlier publication is restricted to Iron Age signs with extensive bibliographical references. The second volume, the Acts of the Procida Round Table represents the current stage of research and include both Bronze and Iron Age signs.


\def\ego{\bgroup\anatolian\char"14400\egroup}
\def\egoi{\bgroup\anatolian\char"14401\egroup}
\def\egoii{\bgroup\anatolian\char"14402\egroup}
\def\monsii{\bgroup\anatolian\char"14403\egroup}
\def\egoiv{\bgroup\anatolian\char"14404\egroup}

\def\piscis{\bgroup\anatolian\char"144A5\egroup}

\def\ansign#1{
\bgroup
\anatolian\scalebox{2.5}{\char #1}
\egroup
}

{\parindent0pt
\paragraph{\bfseries\large 1 {\anatolian \scalebox{2.5}{\ego}} EGO \foreignquote{french}{JE, MOI.}} 
\medskip
Hands pointing to face. Generally found at the beginning of various texts, denoting \enquote{I am}.

\paragraph{\bfseries\large 2 {\anatolian \scalebox{2.5}{\egoi}} EGO$_1$ \foreignquote{french}{JE, MOI.}} 
\medskip
Hands pointing to face. Generally found at the beginning of various texts, denoting \enquote{I am}.
 
\paragraph{\bfseries\large 3 (a){\anatolian \scalebox{2.5}{\egoii}} EGO$_3$  (b){\anatolian \scalebox{2.5}{\egoiv}} EGO$_3$\foreignquote{french}{JE, MOI.}} 
\medskip
Hands pointing to face. Generally found at the beginning of various texts, denoting \enquote{I am}.
 
%% Sign 4, mons2 
\paragraph{\bfseries\large 4 {\anatolian \scalebox{2.5}{\monsii}} MONS$_2$ \foreignquote{french}{montagne (divine)}} 
\mbox{}
\smallskip

Figure with tiara, open hands in prayer.  
 
%% Sign 5, *5, A005 
 
\paragraph{\bfseries\large 5 \ansign{"14404} } 
\mbox{}
\smallskip

Figure with tiara, open hands in prayer.  
 
%% Sign 6, *6, A005 
 
\paragraph{\bfseries\large *6 \ansign{"14405} {\mdseries\small lat.} ADORARE, \mdseries \small fr. \textit{adorer}, en. \textit{adore}} 
\mbox{}
\smallskip

Figure making a gesture of adoration or prayer.  

%% Sign 7, *7, A005 
 
\paragraph{\bfseries\large *7 \ansign{"14406} {\mdseries\small lat.} EDERE, \mdseries \small fr. \textit{manger}, en. \textit{eat}} 
\mbox{}
\smallskip

Figure making a gesture of adoration or prayer.

%% Sign 8 
 
\paragraph{\bfseries\large *8 \ansign{"14407} {\mdseries\small lat.} BIBERE, \mdseries \small fr. \textit{manger}, en. \textit{drink}} 
\mbox{}
\smallskip

Figure drinking from a bowl.

%% Sign 9 
 
\paragraph{\bfseries\large *9 \ansign{"14408} {\mdseries\small lat.} AMPLECTI, \mdseries \small fr. \textit{croisant le bras}, en. \textit{entwine}} 
\mbox{}
\smallskip

Two figures with entwined arms.

%% Sign 10 
 
\paragraph{\bfseries\large *10 \ansign{"14409} {\mdseries\small lat.} CAPUT, \mdseries \small fr. \textit{homme, personne, tête}, en. \textit{head}} 
\mbox{}
\smallskip

Human head.

%% Sign 11 
 
\paragraph{\bfseries\large *11 \ansign{"1440B} {\mdseries\small lat.} CAPUT, \mdseries \small fr. \textit{homme, personne, tête}, en. \textit{FIXME}} 
\mbox{}
\smallskip

Human head with hat.

%% Sign 12 
 
\paragraph{\bfseries\large *12 \ansign{"1440C} {\mdseries\small lat.} STATUA, \mdseries \small fr. \textit{homme, personne, tête}, en. \textit{FIXME}} 
\mbox{}
\smallskip

Human head with hat.

%% Sign 14 
 
\paragraph{\bfseries\large *14 \ansign{"1440E} {\mdseries\small lat.} PRAE, \mdseries \small fr. \textit{homme, personne, tête}, en. \textit{before}} 
\mbox{}
\smallskip

logosyllabic pari


%% Sign 15 
\paragraph{\bfseries\large *15 \ansign{"1440F} {\mdseries\small lat.} DOMINA, \mdseries \small fr. \textit{homme, personne, tête}, en. \textit{FIXME}} 
\mbox{}
\smallskip

In Laroche (HH), the generic meaning of a woman is attributed to the female head. It is interesting to note that the only attestation that is reported is that of KARKEMISH A5a.3-4, where a mention is made to a TERRA.DEUS.DOMINA, in reference to an underwordly divinity (Earth's Divine Lady).
}

  
 
 
\def\avis{%
\bgroup
$\stackrel{\stackrel{\scalebox{2}{\anatolian\char"1449A}}{\mbox{\small\arial \textsc{AVIS}}}}{\mbox{128}}$
\egroup
} 
 
\def\avisii{%
\bgroup
$\stackrel{\stackrel{\scalebox{2}{\anatolian\char"1449B}}{\mbox{\small\arial \textsc{AVIS2}}}}{\mbox{128}}$
\egroup
} 

\def\avisiii{%
\bgroup
$\stackrel{\stackrel{\scalebox{2}{\anatolian\char"1449C}}{\mbox{\small\arial \textsc{AVIS3}}}}{\mbox{128}}$
\egroup
} 

\def\avisiv{%
\bgroup
$\stackrel{\stackrel{\scalebox{2}{\anatolian\char"1449D}}{\mbox{\small\arial \textsc{AVIS\textsubscript{4}}}}}{\mbox{128}}$
\egroup
} 

\def\avisv{%
\bgroup
$\stackrel{\stackrel{\scalebox{2}{\anatolian\char"1449E}}{\mbox{\small\arial \textsc{AVIS5}}}}{\mbox{128}}$
\egroup
} 

\def\domina{%
\bgroup
$\stackrel{\stackrel{\scalebox{2}{\anatolian\char"1440F}}{\mbox{\small\arial \textsc{DOMINA}}}}{\mbox{\arial 15}}$
\egroup
}

\def\defanatolian#1#2#3{%
  \expandafter\gdef\csname#1\endcsname{%
    $\stackrel{\stackrel{\scalebox{2}{\anatolian\char #2}}{\mbox{\strut\small\ttfamily \uppercase{#1}}}}{\mbox{\ttfamily #3}}$
  }%
}

\defanatolian{avis}{"1449A}{128}
\defanatolian{domina}{"1440F}{15}
\defanatolian{adorare}{"14405}{6}
\defanatolian{bibere}{"14407}{8}



\subsection{Category Lists}

\paragraph{AVIS} signs with Birds Numerous variants of hieroglyphic signs with birds were used \scalebox{1.8}{\anatolian \char"1449A \char"1449B \char"1449C \char"1449D \char"1449E}
 
\begin{center} 
 \avis \avisii \avisiii \avisiv \avisv\\
 \domina \adorare \bibere
\end{center} 

The logogram is represented by theLatin word |AVIS|. Variants are denoted by subscripts. The number is the Laroche sign number.

Alan Gardiner broke the different Egyptian Hieroglyphs in 26 categories, which he labelled with different letters of the alphabet (A, B, C \ldots Y, Z, AA, omitting J). Laroche followed a similar convention and assigned the letters to categories, numbered in Roman numerals:


 
 \subsection{Logograms}
 
 The system devised by Hawkins is that each logogram is represented by a Latin word. This is a similar system as that used by Linear B scholars from the early 1960's. Of course this system is not ideal. Sometimes we may not know the relevant Latin words or these maybe too obscure to be readily intelligible. It is clear that the sign must always be transcribed 
with the same form of the chosen word, which involves ignoring rules of inflection and conguence, producing such grammatical monsters as BOS 9, ``nine oxen'' (singular followed by a number higher thatn 1) and MAGNUS.DOMINA, "(great) queen". The complete list of transliterations is shown in Table .

% https://oi.uchicago.edu/sites/oi.uchicago.edu/files/uploads/shared/docs/Publications/nn234.pdf Nice intro
 
 