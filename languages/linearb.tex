
\section{Linear B}
\label{s:linearb}

The Linear B script is a syllabic writing system that was used on the island of Crete and
parts of the nearby mainland to write the oldest recorded variety of the Greek language.

Linear B clay tablets predate Homeric Greek by some 700 years; the latest tablets date from
the mid- to late thirteenth century bce. Major archaeological sites include Knossos, first
uncovered about 1900 by Sir Arthur Evans, and a major site near Pylos. The majority of
currently known inscriptions are inventories of commodities and accounting records.

The first tablets bearing the scripts were discovered by Sir Arthur Evans (1851-1941) while he was excavating the Minoan palace at Knossos in Crete. 


\medskip

\begin{figure}[ht]
\centering
\begin{minipage}{5cm}
\includegraphics[width=5cm]{./images/iklaina.jpg}
\end{minipage}\hspace{2em}
\begin{minipage}{7cm}
\captionof{figure}{Recently discovered fragment with Linear B, inscription. Found in an olive grove in what's now the village of Iklaina, the tablet was created by a Greek-speaking Mycenaean scribe between 1450 and 1350 B.C. (See \protect\href{http://news.nationalgeographic.com/news/2011/03/110330-oldest-writing-europe-tablet-greece-science-mycenae-greek/}{National Geographic}).}
\end{minipage}

\end{figure}


Early attempts to decipher the script failed until Michael Ventris, an architect and amateur
decipherer, came to the realization that the language might be Greek and not, as previously
thought, a completely unknown language. Ventris worked together with John Chadwick,
and decipherment proceeded quickly. The two published a joint paper in 1953.\citep{ventrisa}


\newfontfamily\linearb{Aegean.ttf}

Linear B was added to the Unicode Standard in April, 2003 with the release of version 4.0.

The Linear B Syllabary block is \unicodenumber{U+10000–U+1007F}. The Linear B Ideograms block is {\smallcps U+10080–U+100FF}. The Unicode block for the related Aegean Numbers is U+10100–U+1013F.

\begin{scriptexample}[]{Linear B}
\unicodetable{linearb}{"10000,"10010,"10020,"10030,"10040,"10050,"10060,"10070}

\captionof{table}{Linear B Typeset with command \protect\string\linearb\ and the \texttt{Aegean} font.}
\end{scriptexample}

\begin{scriptexample}[]{Linear B}
\unicodetable{linearb}{"10080,"10090,"100A0,"100B0,"100C0,"100D0,"100E0,"100F0}
\captionof{table}{Linear B Ideograms. Typeset with command \protect\string\linearb\ and the \texttt{Aegean} font.}
\end{scriptexample}


\begin{scriptexample}[]{Aegean Numbers}
\unicodetable{linearb}{"10100,"10110,"10110,"10120,"10130}

\captionof{table}{Aegean Numbers}
\end{scriptexample}





\section{Phaestos Disc}

\begin{figure}[htp]
\centering

\includegraphics[width=0.67\textwidth]{./phaistosdiscs.jpg}
\caption{Phaistos discs.}
\end{figure}


\PrintUnicodeBlock{./languages/phaistos.txt}{\linearb}


The ideograms are symbols, not pictures of the objects in question, e.g. one tablet records a tripod with missing legs, but the ideogram used is of a tripod with three legs. In modern transcriptions of Linear B tablets, it is typically convenient to represent an ideogram by its Latin or English name or by an abbreviation of the Latin name. Ventris and Chadwick generally used English; Bennett, Latin. Neither the English nor the Latin can be relied upon as an accurate name of the object; in fact, the identification of some of the more obscure objects is a matter of exegesis.

\bgroup

\linearb

Vessels
\let\l\unicodenumber

\begin{tabular}{l>{\smallcps}l>{\smallcps}l>{\smallcps}l>{\smallcps}l}
𐃟	&U+100DF	&200	&\l{sartāgo}	&\l{Boiling Pan}\\
𐃠	&U+100E0	&201	&\l{tripūs}	&\l{Tripod Cauldron}\\
𐃡	&U+100E1	&202	&\l{pōculum}	&\l{Goblet}\\
𐃢	&U+100E2	&203	&\l{urceus}	&\l{Wine Jar?}\\
𐃣	&U+100E3	&204  &\l{Tahirnea}	&\l{Ewer}\\
𐃤	&U+100E4	&205  &\l{Tnhirnula}	&\l{Jug}\\
𐃥	&U+100E5	&206	&\l{hydria}	&Hydria\\
𐃦	&U+100E6	&207	&\l{TRIPOD}  &AMPHORA\\
𐃧	&\l{U+100E7}	&\l{208}	&\l{PAT patera}	&\l{BOWL}\\
𐃨	&U+100E8	&209	&AMPH amphora	&AMPHORA\\
𐃩	&U+100E9	&210	&STIRRIP &JAR\\
𐃪	&U+100EA	&211	&WATER &BOWL?\\
𐃫	&U+100EB	&212	&SIT situla	&WATER JAR?\\
𐃬	&U+100EC	&213	&LANX lanx	&COOKING BOWL\\
\end{tabular}
















\egroup










