\newfontfamily\linearb{Aegean.ttf}
\section{Linear B}
\label{s:linearb}
\index{scripts>Linear B}
The Linear B script is a syllabic writing system that was used on the island of Crete and
parts of the nearby mainland to write the oldest recorded variety of the Greek language.

Linear B clay tablets predate Homeric Greek by some 700 years; the latest tablets date from
the mid- to late thirteenth century \bce. Major archaeological sites include Knossos, first
uncovered about 1900 by Sir Arthur Evans, and a major site near Pylos. The majority of
currently known inscriptions are inventories of commodities and accounting records.

The first tablets bearing the scripts were discovered by Sir Arthur Evans (1851-1941) while he was excavating the Minoan palace at Knossos in Crete. 


\medskip

\begin{figure}[ht]
\centering
\begin{minipage}{5cm}
\includegraphics[width=5cm]{./images/iklaina.jpg}
\end{minipage}\hspace{2em}
\begin{minipage}{7cm}
\captionof{figure}{Recently discovered fragment with Linear B, inscription. Found in an olive grove in what's now the village of Iklaina, the tablet was created by a Greek-speaking Mycenaean scribe between 1450 and 1350 B.C. (See \protect\href{http://news.nationalgeographic.com/news/2011/03/110330-oldest-writing-europe-tablet-greece-science-mycenae-greek/}{National Geographic}).}
\end{minipage}

\end{figure}


Early attempts to decipher the script failed until Michael Ventris, an architect and amateur
decipherer, came to the realization that the language might be Greek and not, as previously
thought, a completely unknown language. Ventris worked together with John Chadwick,
and decipherment proceeded quickly. The two published a joint paper in 1953. See \fullcite{ventrisa}.




Linear B was added to the Unicode Standard in April, 2003 with the release of version 4.0.

The Linear B Syllabary block is \unicodenumber{U+10000–U+1007F}. The Linear B Ideograms block is {\smallcps U+10080–U+100FF}. The Unicode block for the related Aegean Numbers is U+10100–U+1013F.

\begin{scriptexample}[]{Linear B}
\unicodetable{linearb}{"10000,"10010,"10020,"10030,"10040,"10050,"10060,"10070}

\captionof{table}{Linear B Typeset with command \protect\string\linearb\ and the \texttt{Aegean} font.}
\end{scriptexample}

\begin{scriptexample}[]{Linear B}
\unicodetable{linearb}{"10080,"10090,"100A0,"100B0,"100C0,"100D0,"100E0,"100F0}
\captionof{table}{Linear B Ideograms. Typeset with command \protect\string\linearb\ and the \texttt{Aegean} font.}
\end{scriptexample}


\begin{scriptexample}[]{Aegean Numbers}
\unicodetable{linearb}{"10100,"10110,"10110,"10120,"10130}

\captionof{table}{Aegean Numbers}
\end{scriptexample}





\section{Phaestos Disc}


One of the puzzles of Minoan Crete is the Phaestos disc. The Phaistos Disc was discovered in the Minoan palace-site of Phaistos, near Hagia Triada, on the south coast of Crete;[1] specifically the disc was found in the basement of room 8 in building 101 of a group of buildings to the northeast of the main palace. This grouping of four rooms also served as a formal entry into the palace complex. Italian archaeologist Luigi Pernier recovered the intact \enquote{dish}, about 15 cm (5.9 in) in diameter and uniformly slightly more than 1 centimetre (0.39 inches) in thickness, on 3 July 1908 during his excavation of the first Minoan palace.

It was found in the main cell of an underground \enquote{temple depository}. These basement cells, only accessible from above, were neatly covered with a layer of fine plaster. Their content was poor in precious artifacts, but rich in black earth and ashes, mixed with burnt bovine bones. In the northern part of the main cell, in the same black layer, a few inches south-east of the disc and about 20 inches (51 centimetres) above the floor, Linear A tablet PH 1 was also found. The site apparently collapsed as a result of an earthquake, possibly linked with the eruption of the Santorini volcano that affected large parts of the Mediterranean region during the mid second millennium B.C.

\begin{figure}[htp]
\centering

\includegraphics[width=0.67\textwidth]{./phaistosdiscs.jpg}
\caption{Phaistos discs.}
\end{figure}

The Phaistos Disc is generally accepted as authentic by archaeologists.[2] The assumption of authenticity is based on the excavation records by Luigi Pernier. This assumption is supported by the later discovery of the Arkalochori Axe with similar but not identical glyphs.[3]


The possibility that the disc is a 1908 forgery or hoax has been raised by two scholars.[4][5][6] In his 2008 review, Robinson does not endorse the forgery arguments, but argues that \enquote{a thermoluminescence test for the Phaistos Disc is imperative. It will either confirm that new finds are worth hunting for, or it will stop scholars from wasting their effort.}[4]

A gold signet ring from Knossos (the Mavro Spilio ring), found in 1926, contains a Linear A inscription developed in a field defined by a spiral—similar to the Phaistos Disc.\footnote{See University of Cologne website \url{http://arachne.uni-koeln.de/arachne/index.php?view[layout]=objekt_item\&search[constraints][objekt][searchSeriennummer]=159123}} A sealing found in 1955 shows the only known parallel to sign 21 (the \enquote{comb}) of the Phaistos disc.[9] This is considered as evidence that the Phaistos Disc is a genuine Minoan artifact.[10]

\begin{figure}[htbp]
\centering

\includegraphics[width=4.5cm]{crete-spiral-ring}\includegraphics[width=4.5cm]{crete-spiral-ring-01}\includegraphics[width=4.5cm]{crete-spiral-ring-02}

\caption{A gold signet ring from Knossos (the Mavro Spilio ring), found in 1926, contains a Linear A inscription developed in a field defined by a spiral—similar to the Phaistos Disc}
\end{figure}

The disc is made of fine clay.  Both side of the disc carry an inscription arranged in a spiral around the centre. The characters were impressed with a punch or stamp before the clay was fired. There are
241 or 242 characters (one is damaged), which
comprise 45 signs of variable frequency. For
comparison, there are thousands of characters in a few pages of printed English text, comprising the 26 signs we call letters. Lines partition
the disc’s characters into 31 short sections on
side A and 30 on side B, most of which contain
three, four or five characters. It is tempting to
speculate that these sections represent words
in the language of the disc.

That the characters were printed, not carved,
is beyond dispute. But no one knows why the disc’s maker bothered to produce a punch or stamp for each sign, rather than inscribing each character afresh. Egyptian hieroglyphs or Mesopotamian cuneiform of the second
millennium bce are inscribed on stone or clay;
simlarly the Minoan scripts Linear A and B found
at Phaistos, Knossos and other Cretan sites. If
the punch or stamp was to \enquote{print} many copies of documents, one would expect further sam-
ples to have turned up in a century of intensive Mediterranean excavatio

There is patchy and inconclusive evidence for and against the disc’s Cretan origin. The
signs look nothing like those of Linear A, Linear B or any other Minoan script, except coincidentally. This has led some, including Evans and Chadwick, to propose that the disc — and presumably its language, too — was an import.

One sign bears a remarkable resemblance to the architecture of rock tombs found in Anatolia in modern Turkey. One or two others
resemble signs found on a few contemporaneous objects from different sites in Crete. Most
scholars today, including Duhoux, think it a plausible working hypothesis that the disc was made in Crete. Gareth Owens and his Team claim to have read the disc and you can hear how it sounded at a TED Talk\footnote{\url{https://www.youtube.com/watch?v=6Chcplx3tZ8}}.



\subsection{Signs}

There are 242 tokens on the disc, comprising 45 distinct signs. Many of these 45 signs represent easily identifiable every-day things. In addition to these, there is a small diagonal line that occurs underneath the final sign in a group a total of 18 times. The disc shows traces of corrections made by the scribe in several places. The 45 symbols were numbered by Arthur Evans from 01 to 45, and this numbering has become the conventional reference used by most researchers. Some symbols have been compared with Linear A characters by Nahm,[17] Timm,[3] and others. Other scholars (J. Best, S. Davis) have pointed to similar resemblances with the Anatolian hieroglyphs, or with Egyptian hieroglyphs (A. Cuny). In the table below, the character "names" as given by Louis Godart (1995) are given in upper case; where other description or elaboration applies, they are given in lower case.




\PrintUnicodeBlock{./languages/phaistos.txt}{\linearb}




The ideograms are symbols, not pictures of the objects in question, e.g. one tablet records a tripod with missing legs, but the ideogram used is of a tripod with three legs. In modern transcriptions of Linear B tablets, it is typically convenient to represent an ideogram by its Latin or English name or by an abbreviation of the Latin name. Ventris and Chadwick generally used English; Bennett, Latin. Neither the English nor the Latin can be relied upon as an accurate name of the object; in fact, the identification of some of the more obscure objects is a matter of exegesis.

\begingroup

\linearb

Vessels
\let\l\unicodenumber

\begin{tabular}{l>{\smallcps}l>{\smallcps}l>{\smallcps}l>{\smallcps}l}
𐃟	&U+100DF	&200	&\l{sartāgo}	&\l{Boiling Pan}\\
𐃠	&U+100E0	&201	&\l{tripūs}	&\l{Tripod Cauldron}\\
𐃡	&U+100E1	&202	&\l{pōculum}	&\l{Goblet}\\
𐃢	&U+100E2	&203	&\l{urceus}	&\l{Wine Jar?}\\
𐃣	&U+100E3	&204  &\l{Tahirnea}	&\l{Ewer}\\
𐃤	&U+100E4	&205  &\l{Tnhirnula}	&\l{Jug}\\
𐃥	&U+100E5	&206	&\l{hydria}	&Hydria\\
𐃦	&U+100E6	&207	&\l{TRIPOD}  &AMPHORA\\
𐃧	&\l{U+100E7}	&\l{208}	&\l{PAT patera}	&\l{BOWL}\\
𐃨	&U+100E8	&209	&AMPH amphora	&AMPHORA\\
𐃩	&U+100E9	&210	&STIRRIP &JAR\\
𐃪	&U+100EA	&211	&WATER &BOWL?\\
𐃫	&U+100EB	&212	&SIT situla	&WATER JAR?\\
𐃬	&U+100EC	&213	&LANX lanx	&COOKING BOWL\\
\end{tabular}




\subsection{Online Resources}

Corpora and GORILA \url{http://www.people.ku.edu/~jyounger/LinearA/\#3}



\endgroup










