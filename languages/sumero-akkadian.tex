\chapter{Sumero Akkadian Cuneiform}
\label{s:sumero}
\newfontfamily\sumero{NotoSansSumeroAkkadianCuneiform-Regular.ttf}




The cuneiform writing system of the ancient Middle East was deeply influential in
world culture. For over three millennia, until about two thousand years ago, it was the
vehicle of communication from (at its greatest extent) Iran to the Mediterranean,
Anatolia to Egypt. A complex script, written mostly on clay tablets by professional
scribes, it was used to record actions, thoughts, and desires that fundamentally
shaped the modern world, socially, politically, and intellectually. Unlike other ancient
media, such as papyri, writing-boards, or leather rolls, cuneiform tablets survive in their
hundreds of thousands, oW en excavated from the buildings in which they were created,
used, or disposed of. Primary evidence of cuneiform culture thus comes from a wide
variety of physical and social contexts in abundant quantities, which enables the close
study of very particular times and places.


Sumero-Akkadian cuneiform has the advantage that most objects are now in open digitized libraries and easily accessible to specialists as well as the interested general reader. The images are normally accompanied by line diagrams, especially for the more important objects, with details of their provenance and publications. 

\begin{figure}[htbp]
\centering
\includegraphics[height=0.8\textheight]{P222322}
\end{figure}

The line diagram

\begin{figure}[htbp]
\includegraphics[height=0.8\textheight]{P222322_l}
\end{figure}

To compensate for the deficiencies of clay tablets, writing boards (Akkadian lē’u ) with
erasable waxed surfaces were used alongside them from at least the 21st century bc
(Steinkeller 2004 ) , plus papyrus (Akkadian niāru ) from the mid-second millennium
and parchment or leather rolls (Akkadian giṭṭu, magallatu ) from the early fi rst millennium
onwards (see Philippe Clancier in Chapter 35 ) . Practically no such artefacts survive—
apart from a few now surfaceless Neo-Assyrian writing boards—although they
are occasionally mentioned in tablets and sometimes depicted visually (Figure 1.8 ). We
must never forget that cuneiform culture was only one literate culture amongst several
in the ancient Near East, albeit the most longlived and prestigious.



\section{Encoding}

In Unicode, the Sumero-Akkadian Cuneiform script is covered in two blocks:
U+12000–U+1237F Cuneiform
U+12400–U+1247F Cuneiform Numbers and Punctuation
These blocks, in version 6.0, are in the Supplementary Multilingual Plane (SMP).

The sample glyphs in the chart file published by the Unicode Consortium[2] show the characters in their Classical Sumerian form (Early Dynastic period, mid 3rd millennium BCE). The characters as written during the 2nd and 1st millennia BCE, the era during which the vast majority of cuneiform texts were written, are considered font variants of the same characters.


The character set as published in version 5.2 has been criticized, mostly because of its treatment of a number of common characters as ligatures, omitting them from the encoding standard.


\unicodetable{sumero}{"12000,"12010,"12020,"12030,"12040,"12050,"12060,"12070,
"12080,"12090, "120A0, "120B0, "120C0, "120D0, "120E0,"120F0,"12400,"12410,"12420,"12430}


\begin{table}[b]
\begin{scriptexample}{textbox}
\parindent1em
From Plato's dialogue Phaedrus 14, 274c-275b:

Socrates: [274c] I heard, then, that   in Egypt, was one of the ancient gods of that country, the one whose sacred bird is called the ibis, and the name of the god himself was Theuth. He it was who [274d] invented numbers and arithmetic and geometry and astronomy, also draughts and dice, and, most important of all, letters. 

Now the king of all Egypt at that time was the god Thamus, who lived in the great city of the upper region, which the Greeks call the Egyptian Thebes, and they call the god himself Ammon. To him came Theuth to show his inventions, saying that they ought to be imparted to the other Egyptians. But Thamus asked what use there was in each, and as Theuth enumerated their uses, expressed praise or blame, according as he approved [274e] or disapproved.  

"The story goes that Thamus said many things to Theuth in praise or blame of the various arts, which it would take too long to repeat; but when they came to the letters, [274e] “This invention, O king,” said Theuth, “will make the Egyptians wiser and will improve their memories; for it is an elixir of memory and wisdom that I have discovered.” But Thamus replied, “Most ingenious Theuth, one man has the ability to beget arts, but the ability to judge of their usefulness or harmfulness to their users belongs to another; [275a] and now you, who are the father of letters, have been led by your affection to ascribe to them a power the opposite of that which they really possess.  

"For this invention will produce forgetfulness in the minds of those who learn to use it, because they will not practice their memory. Their trust in writing, produced by external characters which are no part of themselves, will discourage the use of their own memory within them. You have invented an elixir not of memory, but of reminding; and you offer your pupils the appearance of wisdom, not true wisdom, for they will read many things without instruction and will therefore seem [275b] to know many things, when they are for the most part ignorant and hard to get along with, since they are not wise, but only appear wise." 
\end{scriptexample}
\end{table}


\printunicodeblock{./languages/cuneiform.txt}{\sumero}



