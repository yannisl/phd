\newfontfamily\korean{NotoSerifCJKkr}

\def\textko#1{\bgroup\korean #1\egroup}

\chapter{Korean}

\section{Origins}

Korea has a fairly homogeneous population so the question as where did the language came from has perplexed linguists. 
This origin question is of ultimate interest to linguists, but it has also captured the imagination of the
Korean lay public, who have tended to conflate the question with broader ones about their own ethnic origin. Linguistic nomenclature has added to the confusion. When specialists speak to the public about \enquote{family trees} and
\enquote{related languages,} the non-specialist naturally thinks that the Korean language
has relatives and a biological family like those people do. And when
a people as homogeneous as Koreans are told that their language belongs to a
family that includes Mongolian and Manchu, they envision their ancestors
arriving in the cul-de-sac of the Korean peninsula as horse-riding warriors.
It becomes a personal kind of romance.\footcite{ki-moon2011}

Nevertheless, the answer to the question of where Korean came from is
still incomplete. In order for a genetic hypothesis to be truly convincing, the
proposed rules of correspondence must lead to additional, often unsuspected
discoveries about the relationship. Concrete facts must emerge about the
history of each language being compared in order to put the hypothesis
beyond challenges to its validity, and that has so far not happened in the case
of Korean. As a result, we cannot yet say with complete certainty what the
origin of Korean was.\footcite{ki-moon2011}



\section{Hangul}
Hangul or Hangeul (English pronunciation: /ˈhɑːnɡuːl/ HAHN-gool;[1] from Korean {\korean 한글} [ha(ː)n.ɡɯl]) is the South Korean term for the Korean alphabet (called Chosŏn'gŭl ({\korean 조선글}) in North Korea), which has been used to write the Korean language since its creation in the 15th century by King Sejong the Great.[2][3]

It is the official writing system of North Korea and South Korea. It is a co-official writing system in the Yanbian Korean Autonomous Prefecture and Changbai Korean Autonomous County in Jilin Province, China. It is also used to write the Cia-Cia language spoken near the town of Bau-Bau, Indonesia.

Korean is known as an alphabetic language in the literature ( Perfetti, 2003;
 Wang et al ., 2003 ; Simpson and Kang, 2004;
 Perfetti and Liu, 2005 ). However, the Korean language has a distinctive feature of a syllabic writing
system. Korean is alphabetic in that each letter maps onto a phoneme and each grapheme ties with
a vowel to form a syllabic unit. Each grapheme in Korean (a consonant or a vowel) has its individual
sound, but a consonant has to glue together with a vowel for the consonant to be vocalized. A consonant
string in the initial, middle, and ending positions is unlikely to occur in Korean. Talylor and Olson (1995), as well as H.K. Pay claims that Korean is an alphabetic syllabary or a syllabic alphabet. \footcite{perfetti2003}


The alphabet consists of 14 consonants and 10 vowels. Its letters are grouped into syllabic blocks, vertically and horizontally. For example, the Korean word for \enquote{honeybee} is written \textko{꿀벌}, not \textko{ㄲㅜㄹㅂㅓㄹ}.[4] As it combines the features of alphabetic and syllabic writing systems, it has been described as an \enquote{alphabetic syllabary} by some linguists.[5][6] As in traditional Chinese writing, Korean texts were traditionally written top to bottom, right to left, and are occasionally still written this way for stylistic purposes. Today, it is typically written from left to right with spaces between words and western-style punctuation.[7]

Some linguists consider it the most logical writing system in the world, partly because the shapes of its consonants mimic the shapes of the speaker's mouth when pronouncing each consonant.[5][7][8]

\section{The transcription of Sino-Korean}

If we take Sejong at his word, the new symbols were devised explicitly to
represent the sounds of Korean. In the preface to the Hunmin cho˘ngu˘m he
wrote:

\begin{quotation}
The sounds of our country’s language are different from those of the Middle Kingdom
and are not smoothly adaptable to those of Chinese characters. Therefore, among the
simple people, there are many who have something they wish to put into words but are
never able to express their feelings. I am distressed by this, and have newly designed
twenty-eight letters. I desire only that everyone practice them at their leisure and make
them convenient for daily use.
\end{quotation}

But from the very beginning, the new letters were used to transcribe
the readings of Chinese characters as well as to write native Korean words,
and both are found together in the texts of the period. As we have said,
these character readings do not represent natural Korean but rather the
prescriptive pronunciations spelled out in detail in the Tongguk chongun
of 1447.

\subsection{Vowels}

Vowel letters are based on three elements:

\begin{enumerate}
\item A horizontal line representing the flat Earth, the essence of yin.

\item A point for the Sun in the heavens, the essence of yang. (This becomes a short stroke when written with a brush.)

\item A vertical line for the upright Human, the neutral mediator between the Heaven and Earth.
\end{enumerate}

Short strokes (dots in the earliest documents) were added to these three basic elements to derive the vowel letter. These are divided into three categories a) simple vowels b) compound vowels and c) iotized vowels. 

\subsubsection{Simple vowels}
\paragraph{Horizontal letters:} these are mid-high back vowels.

bright {\korean ㅗ} [o]

dark {\korean ㅜ} [u]

neutral {\korean ㅡ} eu (ŭ)

\paragraph{Vertical letters:} these were once low vowels.

bright \textko{ㅏ} a

dark {\korean ㅓ} eo (ŏ)

neutral \textko{\char12643 } i

\subsubsection{Compound vowels}

The Korean alphabet never had a w, except in Sino-Korean vocabulary. Since an o or u before an a or eo became a [w] sound, and [w] occurred nowhere else, [w] could always be analyzed as a phonemic o or u, and no letter for [w] was needed. However, vowel harmony is observed: \enquote{dark} \textko{ㅜ} u with \enquote{dark} \textko{ㅓ} eo for \textko{ㅝ} wo; \enquote{bright} \textko{ㅗ} o with \enquote{bright} \textko{ㅏ} a for \textko{ㅘ} wa:

{\korean
ㅘ wa = ㅗ o + ㅏ a

ㅝ wo = ㅜ u + ㅓ eo

ㅙ wae = ㅗ o + ㅐ ae

ㅞ we = ㅜ u + ㅔ e
}


\subsubsection{Iotized vowels}



The table below shows the 21 vowels used in the modern Korean Alphabet in South Korean alphabetic order with Revised Romanization equivalents for each letter. Linguists disagree on the number of phonemes versus diphthongs among vowels in the Korean alphabet.[40]

\bigskip

\begingroup
\setlength{\tabcolsep}{2pt}
\korean
\begin{tabular}{p{2cm}lllllllllllllllllllll}
Letters       &ㅏ &ㅐ	&ㅑ	&ㅒ	&ㅓ	&ㅔ	&ㅕ	&ㅖ	&ㅗ	&ㅘ	&ㅙ	&ㅚ	&ㅛ	&ㅜ	&ㅝ	&ㅞ	&ㅟ	&ㅠ	&ㅡ	&ㅢ	&ㅣ\\
Revised Romanization	&a	&ae	&ya	&yae	&eo	&e	 &yeo	&ye	&o	&wa	&wae	&oe	&yo	&u	&wo	&we	&wi	&yu	&eu	&ui	&i\\
\end{tabular}
\endgroup

\section{The Korean Letters: Jamo}

Every Korean syllable consists of a \textit{lead consonant}, a medial \textit{vowel} and a \textit{tail} consonant. To write syllables with an initial vowel  a special sign for a mute lead consonant must be used. In open syllables (syllables ending in a vowel), the tail consonant is omitted. Isolated vowels can be considered regular syllables with a mute initial and a missing tail.

There are 19 different lead consonants, including the mute consonant. The following table gives the consonants in their canonical order, and their Unicode values. Consonant number 12 is the mute consonant.

\begingroup
\arrayrulecolor{thetablevrulecolor}%
\begin{longtable}{ll >{\korean}l >{\korean}l >{\ttfamily}l}
\toprule
Number	&Lead	&Jamo	& &Character\\ 
        &     &     & &reference\\
\midrule                     
1	&G	&ᄀ	&ㄱ	&U+1100\\
2	&GG	&ᄁ	&ㄲ	&U+1101\\
3	&N	   &ᄂ	&ㄴ	&U+1102\\
4	&D	 &ᄃ	&ㄷ	&U+1103\\
5	&DD	&ᄄ	&ㄸ	&U+1104\\
6	&R	&ᄅ	&ㄹ	&U+1105\\
7	&M	&ᄆ	&ㅁ	&U+1106\\
8	&B	&ᄇ	&ㅂ	&U+1107\\
9	&BB	&ᄈ	&ㅃ	&U+1108\\
10	&S	&ᄉ	&ㅅ	&U+1109\\
11	&SS	&ᄊ	&ㅆ	&U+110A\\
12	&ᄋ	&ㅇ	&ㅇ&U+110B\\
13	&J	  &ᄌ	&ㅈ	&U+110C\\
14	&JJ	&ᄍ	&ㅉ	&U+110D\\
15	&C	   &ᄎ	&ㅊ	&U+110E\\
16	&K	   &ᄏ	&ㅋ	&U+110F\\
17	&T	   &ᄐ	&ㅌ	&U+1110\\
18	&P	   &ᄑ	&ㅍ	&U+1111\\
19	&H	   &ᄒ	&ㅎ	&U+1112\\
\bottomrule
\end{longtable}
\endgroup

\begingroup
\arrayrulecolor{thetablevrulecolor}%
\begin{longtable}{ll >{\korean}l >{\korean}l >{\ttfamily}l}
\toprule
Number	&Lead	&Jamo	& &Character\\ 
        &     &     & &reference\\
\midrule 
1	&G	 &ᆨ	&ㄱ	& U+11A8\\
2	&GG	&ᆩ	&ㄲ	& U+11A9\\
3	&GS	&ᆪ	&ㄳ	& U+11AA\\
4	&N	   &ᆫ	&ㄴ	& U+11AB\\
5	&NJ	&ᆬ	&ㄵ	& U+11AC\\
6	&NH	&ᆭ	&ㄶ	& U+11AD\\
7	&D	&ᆮ	&ㄷ	& U+11AE\\
8	&L	&ᆯ	&ㄹ	& U+11AF\\
9	&LG	&ᆰ	&ㄺ	& U+11B0\\
10	&LM	&ᆱ	&ㄻ	& U+11B1\\
11	&LB	&ᆲ	&ㄼ	& U+11B2\\
12	&LS	&ᆳ	&ㄽ	& U+11B3\\
13	&LT	&ᆴ	&ㄾ	& U+11B4\\
14	&LP	&ᆵ	&ㄿ	& U+11B5\\
15	&LH	&ᆶ	&ㅀ	& U+11B6\\
16	&M	&ᆷ	&ㅁ	& U+11B7\\
17	&B	&ᆸ	&ㅂ	& U+11B8\\
18	&BS	&ᆹ	&ㅄ	& U+11B9\\
19	&S	&ᆺ	&ㅅ	& U+11BA\\
20	&SS	&ᆻ	&ㅆ	& U+11BB\\
21	&NG	&ᆼ	&ㅇ	& U+11BC\\
22	&J	&ᆽ	&ㅈ	& U+11BD\\
23	&C	&ᆾ	&ㅊ	& U+11BE\\
24	&K	&ᆿ	&ㅋ	& U+11BF\\
25	&T	&ᇀ	&ㅌ	& U+11C0\\
26	&P	&ᇁ	&ㅍ	& U+11C1\\
27	&H	&ᇂ	&ㅎ	& U+11C2\\
\bottomrule
\end{longtable}
\endgroup

A useful utility for composing Hangul can be found at gernot-katzers website.
\footnote{\url{http://gernot-katzers-spice-pages.com/var/korean_hangul_unicode.html}} The utility helped me personally to understand how the syllabels are formed from the lead, medial and final letters to syllables and how the syllables are shaped by the text shaping engines. The Tables above and much of the text here are based on these webpages. 

\section{Typography}


\begin{figure}[htbp]
\includegraphics[width=\textwidth]{hangul-direction}
\caption[Korean horizontal and vertical writing]{Horizontal writing and vertical writing (arrow indicates the text direction), from \protect\url{https://www.w3.org/TR/klreq}}
\end{figure}

\paragraph{Line adjustment}

Text can be line adjusted (justified) both in the vertical or horizontal direction. 


\section{LaTeX}

Using one of the newer TeX engines such as LuaLaTeX enables one to typeset Korean almost effortlessly. 
The character \char12643 can give problems sometimes, if you copy paste as it can be confused with the \textbar. There are tow issues for capturing text. If you are on windows you can use a windows virtual keyboard, which tends to work well. The keyboard also has a predictive feature that pops up a menu to choose the correct syllable or word (very similar to pinyn keyboards for Chinese).

\begin{figure}[htbp]
\centering

\includegraphics[width=0.5\textwidth]{hangul-keyboard}

\caption{Virtual Korean keyboard for windows 10. The keyboard can be rendered long or squarish. I personally find the squarish shape more convenient, as it takes less screen space.}

\end{figure}

An alternative way is to use a web based interface such as \href{https://r12a.github.io/pickers/}{r12a}. Google also offers virtual keyboards that can be downloaded.


There are two sets of commands that are usefull utilities for unicode. The first set if you know the name of the 
unicode character it can be used t print the character. These commands are generated by the phd-i18n utility scripts, based
on the latest UCD database (version 11.0) as of this writing.

\begin{texexample}{Printing Korean Characters}{ex:kochar}
\ExplSyntaxOn
\cs_set:cpn {yu}
  {
    HANGUL~JUNGSEONG~YU\par 
    \space
    \large
    \begingroup
      \korean
      \centering
      \char"1172 
    \endgroup
  }
\use:c {yu}
\ExplSyntaxOff
\end{texexample}

The second set of commands is used for transcription of texts, based on the revised romanization scheme. 















