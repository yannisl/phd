\newfontfamily\bassa{JG Bassa Vah Unicode}

The Bassa vah alphabet.

The Bassa script, known as Bassa vah or simply vah ('throwing a sign' in Bassa) is an alphabet for writing the Bassa language of Liberia. It was invented by Dr. Thomas Flo Lewis, who has instigated publishing of limited materials in the language from the mid-1900s through the 1930s, with its height in the 1910s and 1920s. It is alleged that some of the signs are based on native Bassa pictograms revealed by a former slave. It is not clear what connection it may have had with neighboring scripts, but type was cast for it, and an association for its promotion was formed in Liberia in 1959. It is not used contemporarily and has been classified as a failed script.[1] Its creation should be distinguished from other orthographic attempts in the 1830s by European missionaries.
Vah is a true alphabet, with 23 consonant letters, 7 vowel letters, and 5 tone diacritics, which are placed inside the vowels. It also has its own marks for commas and periods.

\bgroup
\obeyline
16AD0 𖫐 BASSA VAH LETTER ENNI
16AD1 𖫑 BASSA VAH LETTER KA
16AD2 𖫒 BASSA VAH LETTER SE
16AD3 𖫓 BASSA VAH LETTER FA
16AD4 𖫔 BASSA VAH LETTER MBE
16AD5 𖫕 BASSA VAH LETTER YIE
16AD6 𖫖 BASSA VAH LETTER GAH
16AD7 𖫗 BASSA VAH LETTER DHII
16AD8 𖫘 BASSA VAH LETTER KPAH
16AD9 𖫙 BASSA VAH LETTER JO
16ADA 𖫚 BASSA VAH LETTER HWAH
16ADB 𖫛 BASSA VAH LETTER WA
16ADC 𖫜 BASSA VAH LETTER ZO
16ADD 𖫝 BASSA VAH LETTER GBU
16ADE 𖫞 BASSA VAH LETTER DO
16ADF 𖫟 BASSA VAH LETTER CE
16AE0 𖫠 BASSA VAH LETTER UWU
16AE1 𖫡 BASSA VAH LETTER TO
16AE2 𖫢 BASSA VAH LETTER BA
16AE3 𖫣 BASSA VAH LETTER VU
16AE4 𖫤 BASSA VAH LETTER YEIN
16AE5 𖫥 BASSA VAH LETTER PA
16AE6 𖫦 BASSA VAH LETTER WADDA

\egroup