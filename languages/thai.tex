\def\thaitext#1{{\thai#1}}

\section{Thai}
\label{s:thai}

``Tai'', ``Thai'', and ``Dai'' are easy to confuse. \textit{Tai} is the standard spelling of the name of the language family. It is also the collective term for the peoples who speak the languages. \textit{Thai} and \textit{Dai}, on the other hand, are the names of two specific Tai groups.\index{Tai}\index{Thai}\index{Dai}

Thai, or Siamese, is the best-known and most important memeber of the Tai family of languages. Its written history dates from 1292, when the inscription of King Rama Khamhaeng of Sukhotai was carved onto a stele in what is now northern Thailand. The inscription is the oldest example of Tai writing. (See Figure~\ref{fig:stele}). The formal name of the stele is the ``King Ram Khamhaeng Inscription''. \footnote{The stele  was added to the Memory of the World Register in 2003 by \textsc{UNESCO}.}

The stone was discovered in 1833 by Mongkut who at the time was a bhikkhu (Buddhist monk), in Wat Mahathat, Sukhothai. The authenticity of the stone – or at least portions of it – has been called into question. Piriya Krairiksh, an academic at the Thai Khadi Research Institute, noted that the stele's treatment of vowels suggests that its creators had been influenced by European alphabet systems. He concluded that the stele was fabricated by someone during the reign of King Mongkut or shortly before. The subject is controversial, since if the stone is a fake, the entire history of the period will have to be re-written.\footfullcite{chamberlain1991}

Scholars are sharply divided on the stele's authenticity. It remains an anomaly amongst contemporary writings, and no other source refers to King Ram Khamhaeng by name. Some scholars claim the inscription was completely a 19th-century fabrication; others claim the first 17 lines are genuine; while a third view is that the inscription was fabricated by King Lithai (a later Sukhothai king). Most Thai scholars hold to the inscription's authenticity.\footfullcite{mukhom2003} The inscription and its image of a Sukhothai utopia remain central to Thai nationalism, and the suggestion it may have been faked caused Michael Wright, an expatriate British scholar, to be threatened with deportation under Thailand's lèse majesté laws. \footfullcite{reynolds2006}

\begin{figure}[htbp]
\parindent0pt
\centering
\includegraphics[width=\textwidth,height=\textheight,keepaspectratio]{stele}
\caption{King Rama Stele, now in the Bangkok Museum.}
\label{fig:stele}
\end{figure}


The Thai language has a complex orthography and system of relational markers. Spoken Thai is mutually intelligible with Laotian, the language of Laos; the two languages are written with slightly different scripts but are linguistically similar.[6]


\begin{scriptexample}[]{Thai}
\bgroup
\centerline{\LARGE\thaitext{◌ะ; ◌ัวะ; เ◌ะ; เ◌อะ; เ◌าะ; เ◌ียะ; เ◌ือะ; แ◌ะ; โ◌ะ}}


\hfill Typeset with \idxfont{IrisUPC} and the command \docAuxCommand{thai}
\egroup
\end{scriptexample}


\section{Tai Languages in China}

In the People's Republic of China the Tai family of languages is known as Zhua\`ang-D\`ong family. Zhu\`ang is the Chinese name for
the largest Tai minority, and D\`ong is the Chinese name for the people who call themselves the Kam. 
Eight minority languages are classified as members of this family: Zhuang, Buyi, Dai, Kam (or Dong), Sui, Mulam, Maonan, and Li.

The family Tai (or Zhuang-Dong) is divided into three distinct branches:

\begin{description}
\item [Zhuang-Dai]
\item [Kam-Sui]
\item[Li] The several dialects of Hainan Island are even more distinctive than Kam or Sui. The status of Li as a member of the Tai family is very much open to question.
\end{description}

Ong Be (native pronunciation: [ʔɑŋ˧ɓe˧]), also known as Bê, or Vo Limgao (Chinese: 臨高; pinyin: Lín'gāo), is a language spoken by 600,000 people, 100,000 of them monolingual, on the north-central coast of Hainan Island, including the suburbs of the provincial capital Haikou. The language is taught in primary schools and broadcast on the radio. Ong Be is thought to be a Tai–Kadai language, but it has no close relatives and its relationship within that family has not been determined.\footnote{Ethnologue classifies Ong Be with the Tai and Kam–Sui languages based on shared vocabulary. However, this is negative evidence, perhaps due to lexical replacement in other branches of the family, and morphological evidence suggests that the Tai and Kam–Sui languages are closer to the Hlai and Kra languages, respectively. The place of Ong Be in this scheme is unknown.}

\paragraph{Zhuang-Dai}
Zhuang is the language of the largest minority group in the People’s Republic of China,
with approximately 18 million speakers.1 The majority of Zhuang speakers live in
Guangxi in Southern China, between 20º 54’ and 26º 20’ north latitude on the
southeastern corner of the Yunnan-Guizhou Plateau. A small number of them are
scattered in adjacent areas of Guangdong, Hunan, Guizhou and Yunnan provinces (see
Map 9.1.2-1). The name Zhuang is not an indigenous cover term, but more of an
administrative term, from which the province of Guangxi, the administrative area of
Zhuang, acquired its name – Guangxi Zhuang Autonomous Region. Although Zhuang
speakers spread all over Guangxi, the majority of them concentrate in four prefectures:
Nanning, Baise, Hechi and Liuzhou along the Xi River system. A passage to Southeast
Asia, the Zhuang area is also inhabited by a number of tribal groups such as the Kam, Sui,
Mulao and Maonan, as well as the Miao, Yao, Lakkja, along with Hakka, Yue and many
varieties of the local Chinese.

\textit{Writing System} One thing that distinguishes Zhuang from other dialects of the Tai family is its unique
writing system. Anthropologists, ethnographers, and historians working in Southern China
would not fail to notice a type of Chinese-based writing system used by the Zhuang people
in Guangxi and the surrounding regions. This kind of writing system is used by shamen or
village headmen to record songs, for account keeping, calendrical reckonings, and other
important things and events. Fan Chengda, a scholar official and a travellor in the Song
dynasty (date), was credited with the discovery of this type of writing (Fan 1175), which
was already nearly 300 years older than the Thai writing system. One can assume that such
a writing system must have existed before Fan’s times. For example, Yupian, a dictionary
compiled shortly after Han times, contains several dozen words like 􅋘, ‘duck’. This is
obviously a Zhuang (Tai) word, which is not found in any Chinese dialect. This character is
now still used in Zhuang area. Mention must also be made of a little song, Yue Ren Ge,
included in Shuo Yuan by Liu Xiang of the Han dynasty. The song was decyphered by a
Zhuang linguist Wei Qingwen, and by Zhengzhang (1991), who believed it to be of Zhuang
(Tai) origin, recorded in Chinese characters. In a way, this could be seen as early attempts at
Zhuang writing.

The Chinese character-based Zhuang writing was modeled on principles of the Chinese
writing system, where each character consists of two components: a meaning part and a
phonetic part. The meaning part often takes the form of radicals while the phonetic part is
represented by certain recurrent and phonetically stable graphs. In some cases, two meaning
components are put together to form a character, with no clue of phonetic readings at all, as
in the case of 􀏡􀀎􄭓, ‘short’, whose meaning resides primarily in the two composite semantic elements, ‘not’ and ‘long’, as neither of the two ideographs was employed to
signal any phonetic content. Such items are only decypherable to trained scholars or shamen
who knew both Zhuang and Chinese well. They would not be understood by outsiders. The
chapter by David Holm in this volume treats this issue in greater detail.

\begin{figure}[htbp]
\includegraphics[width=\textwidth]{zhuang}
\caption{Zhuang women artists.}
\end{figure}

In the late 1950s, after the founding of the People’s Republic of China, a new Zhuang
writing system was introduced using the Roman alphabets. The system was devised by
Chinese linguists with the help of  Russian scholars. The Zhuang dialect of Wuming
was taken as the norm for Standard Zhuang. It was promoted in Zhuang areas through to the
early 1960s before the Cultural Revolution, with some success. The Romanised Zhuang
writing system was reintroduced, with minor revisions, in the early 1980s. A number of
dictionaries have been printed using this writing system.


\paragraph{Kam-Sui Branch} 

The Kam-Sui peoples, and the Kam in particular, are the northernmost Tai-speaking communities. They appear to be aboriginal remnants left unassimilated by the Chinese when they expanded south. 

Kam–Sui includes a dozen languages. The Lakkja and Biao languages are sometimes separated out as a sister branch to Kam–Sui within a "Be–Kam–Tai" branch of Kradai, but this is not well supported. Otherwise the languages are not subclassified.
The better known Kam–Sui languages are Dong (Kam), with over a million speakers, Mulam, Maonan, and Sui. Other Kam–Sui languages include Ai-Cham, Mak, and T’en, and Chadong, which is the most recently discovered Kam–Sui language. Yang (2000) considers Ai-Cham and Mak to be dialects of a single language.[3]

Graham Thurgood (1988) presents the following tentative classification for the Kam–Sui branch.[4] Chadong, a language which has only been recently described by Chinese linguist Jinfang Li, is also included below. It is most closely related to Maonan.



\subparagraph{Kam} The Kam a.k.a. Dong (Chinese: 侗族; pinyin: Dòngzú; endonym: Gaeml [kɐ́m]), a Kam–Sui people of southern China, are one of the 56 ethnic groups officially recognized by the People's Republic of China. They are famed for their native-bred Kam Sweet Rice (Chinese: 香禾糯), carpentry skills, and unique architecture, in particular a form of covered bridge known as the "wind and rain bridge" (Chinese: 风雨桥). The Kam people live mostly in eastern Guizhou, western Hunan, and northern Guangxi in China. Small pockets of Kam speakers are found in Tuyên Quang Province in Vietnam. The Kam people sometimes use Chinese characters to represent the sounds of Kam words. A Latin alphabet was developed in 1958, but it is not much in use due to a lack of printed material and trained teachers. 

The Kam language is noteworthy for its extraordinary large number of tone distinction. Counting six pitch disitinctions in ``checked'' syllables. Most dialects of Kam are said to have fifteen different tones.

\begin{figure}[htbp]
\parindent=0pt
\centering

\includegraphics[width=\textwidth]{zhaoxing}
\caption{Zhaoxing, the largest Dong village in China.}
\end{figure}