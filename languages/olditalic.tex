\section{Old Italic}

\epigraph{A society grows great when old men plant
trees in whose shade they know they will never sit.}{Greek proverb}
\label{s:olditalic}
\index{scripts>Old Italic}
\newfontfamily\olditalic{Noto Sans Old Italic}


Old Italic refers to any of several now extinct alphabet systems used on the Italian Peninsula in ancient times for various Indo-European languages (predominantly Italic) and non-Indo-European (e.g. Etruscan) languages. The alphabets derive from the Euboean Greek Cumaean alphabet, used at Ischia and Cumae in the Bay of Naples in the eighth century BC.

Various Indo-European languages belonging to the Italic branch (Faliscan and members of the Sabellian group, including Oscan, Umbrian, and South Picene, and other Indo-European branches such as Celtic, Venetic and Messapic) originally used the alphabet. Faliscan, Oscan, Umbrian, North Picene, and South Picene all derive from an Etruscan form of the alphabet.

\section{Etruscan}

Many peoples took the system that the Greeks had elaborated and
adapted it to their own language. This was particularly true in Lemnos and
in Etruria, where signs inspired by Greek letters were put to the service of
languages that probably were closely related to Greek—signs that we can
read without fully comprehending them. The Etruscans seem to have used
writing largely for religious purposes. According to Cicero (De divinatione)
they bequeathed their sacred texts to the Romans, who held the Etruscan
religion to be the religion of the Book par excellence.\cite{henri1994}

\begin{figure}[htbp]
\centering
\includegraphics[width=0.7\textwidth]{marsiliana}
\caption{The Marsiliana Tablet}
\end{figure}

The Germanic runic alphabet was derived from one of these alphabets by the 2nd century.


Old Italic is a Unicode block containing a unified repertoire of the three stylistic variants of pre-Roman Italic scripts.

\begin{scriptexample}[]{Testing}
\unicodetable{olditalic}{"10300,"10310,"10320}

{\leavevmode
\hfill\hfill\hfill\footnotesize Typeset with \texttt{Noto Sans Old Italic~}
}
\end{scriptexample}