\section{Avestan script}
\label{s:avestan}
The Avestan alphabet is a writing system developed during Iran's Sassanid era (AD 226–651) to render the Avestan language.
As a side effect of its development, the script was also used for Pazend, a method of writing Middle Persian that was used primarily for the Zend commentaries on the texts of the Avesta. In the texts of Zoroastrian tradition, the alphabet is referred to as \emph{din dabireh} or \emph{din dabiri}, Middle Persian for "the religion's script".

The Avestan alphabet was replaced by the Arabic alphabet after Persia converted to Islam during the 7th century CE. 


Notable Features

The alphabet is written from right to left, in the same way as Syriac, Arabic and Hebrew.
See more at: \url{http://www.iranchamber.com/scripts/avestan_alphabet.php#sthash.ZRu7AkEb.dpuf}


\begin{scriptexample}[]{Avestan}
\ifxetex\TeXXeTstate=1
\beginR\fi
\avestan\raggedleft
𐬄	
𐬅	
𐬆	
𐬇	
𐬈	
𐬉	
𐬊	
𐬋	
𐬌	
𐬍	
𐬎	
𐬏	
𐬐	
	
𐬒	
𐬓	
𐬔	
	
𐬖	
𐬗	
𐬘	
𐬙	
𐬚	
𐬛	
𐬜	
𐬝	
𐬞	
𐬟	
𐬠	
𐬡	
𐬢	
𐬣	
𐬤	
𐬥	
𐬦	
𐬧	
𐬨	
𐬩	
𐬪	
𐬫	
𐬬	
𐬭	
𐬮	
𐬯	
𐬰	
𐬱	
𐬲	
𐬳	
𐬴	
𐬵	
\ifxetex\endR
\TeXXeTstate=0\fi
\end{scriptexample}

The recent Google font \idxfont{NotoSansAvestan-Regular.ttf} can be used to typeset the Avestan script, but I am not sure if it is suitable for any serious study of the language.