
\section{Limbu}
\label{s:limbu}

The Limbu script is used to write the Limbu language. The Limbu script is an abugida derived from the Tibetan script. Limbu is a Tibeto-Burman language spoken mainly in Nepal,[3] significant communities in Bhutan, Sikkim, Darjeeling district, India by the Limbu community. Virtually all Limbus are bilingual in Nepali.

\newfontfamily\limbu{code2000.ttf}

According to traditional histories, the Limbu script was first invented in the late 9th century by King Sirijonga Haang, then fell out of use, to be reintroduced in the 18th century by Te-ongsi Sirijunga Xin Thebe.

To change the inherent vowel, a diacritic is added. Shown here on /k/ ({\limbu ᤁ}):
Appearance	IPA

\begin{table}[htb]
\centering
\begin{tabular}{>{\Large\bfseries\limbu}l>{\arial}l}
ᤁᤡ	&/ki/\\
ᤁᤣ	&/ke/\\
ᤁᤧ	&/kɛ/\\
ᤁᤠ	&/ka/\\
ᤁᤨ	&/kɔ/\\
ᤁᤥ	&/ko/\\
ᤁᤢ	&/ku/\\
ᤁᤤ	&/kai/\\
ᤁᤦ	&/kau/\\
\end{tabular}
\caption{Changing the inherent vowel, using a diacritic.}
\end{table}




\begin{scriptexample}[]{Limbu}
\unicodetable{limbu}{"1900,"1910,"1920,"1930,"1940}
\end{scriptexample}


\printunicodeblock{./languages/limbu.txt}{\limbu}

