\section{Shavian}
\label{s:shavian}
\def\shaviansetup#1{}
\newfontfamily\shavian{code2001.ttf}
^^A\newfontfamily\shavian{NotoSansShavian-Regular.ttf}
\cxset{shavian font/.code=\shaviansetup{#1}}
\cxset{shavian font=shavian}




\begin{scriptexample}[]{shavian}
\shavian

𐑳 𐑡𐑻𐑯𐑰 𐑑 𐑞 𐑕𐑧𐑯𐑑𐑻 𐑝 𐑞 𐑻𐑔
𐑚𐑲 - ·𐑡𐑵𐑤𐑟 ·𐑝𐑻𐑯

𐑗𐑩𐑐𐑑𐑻 1 - 𐑥𐑲 𐑳𐑙𐑒𐑳𐑤 𐑥𐑱𐑒𐑕 𐑳 𐑜𐑮𐑱𐑑 𐑛𐑦𐑕𐑒𐑳𐑝𐑻𐑰

     𐑤𐑫𐑒𐑦𐑙 𐑚𐑩𐑒 𐑑 𐑷𐑤 𐑞𐑩𐑑 𐑣𐑩𐑟 𐑳𐑒𐑻𐑛 𐑑 𐑥𐑰 𐑕𐑦𐑯𐑕 𐑞𐑩𐑑 𐑦𐑝𐑧𐑯𐑑𐑓𐑳𐑤 𐑛𐑱, 𐑲 𐑩𐑥 𐑕𐑒𐑧𐑮𐑕𐑤𐑰 𐑱𐑚𐑳𐑤 𐑑 𐑚𐑦𐑤𐑰𐑝 𐑦𐑯 𐑞 𐑮𐑰𐑩𐑤𐑳𐑑𐑰 𐑝 𐑥𐑲 𐑩𐑛𐑝𐑧𐑯𐑗𐑻𐑟. 𐑞𐑱 𐑢𐑻 𐑑𐑮𐑵𐑤𐑰 𐑕𐑴 𐑢𐑳𐑯𐑛𐑻𐑓𐑳𐑤 𐑞𐑩𐑑 𐑰𐑝𐑦𐑯 𐑯𐑬 𐑲 𐑩𐑥 𐑚𐑦𐑢𐑦𐑤𐑛𐑻𐑛 𐑢𐑧𐑯 𐑲 𐑔𐑦𐑙𐑒 𐑝 𐑞𐑧𐑥.
     𐑥𐑲 𐑳𐑙𐑒𐑳𐑤 𐑢𐑪𐑟 𐑳 𐑡𐑻𐑥𐑳𐑯, 𐑣𐑩𐑝𐑦𐑙 𐑥𐑧𐑮𐑰𐑛 𐑥𐑲 𐑥𐑳𐑞𐑻𐑟 𐑕𐑦𐑕𐑑𐑻, 𐑩𐑯 𐑦𐑙𐑜𐑤𐑦𐑖𐑢𐑫𐑥𐑳𐑯. 𐑚𐑰𐑦𐑙 𐑝𐑧𐑮𐑰 𐑥𐑳𐑗 𐑳𐑑𐑩𐑗𐑑 𐑑 𐑣𐑦𐑟 𐑓𐑪𐑞𐑻𐑤𐑳𐑕 𐑯𐑧𐑓𐑘𐑵, 𐑣𐑰 𐑦𐑯𐑝𐑲𐑑𐑳𐑛 𐑥𐑰 𐑑 𐑕𐑑𐑳𐑛𐑰 𐑳𐑯𐑛𐑻 𐑣𐑦𐑥 𐑦𐑯 𐑣𐑦𐑟 𐑣𐑴𐑥 𐑦𐑯 𐑞 𐑓𐑪𐑞𐑻𐑤𐑩𐑯𐑛. 𐑞𐑦𐑕 𐑣𐑴𐑥 𐑢𐑪𐑟 𐑦𐑯 𐑳 𐑤𐑪𐑮𐑡 𐑑𐑬𐑯, 𐑯 𐑥𐑲 𐑳𐑙𐑒𐑳𐑤 𐑳 𐑐𐑮𐑳𐑓𐑧𐑕𐑻 𐑝 𐑓𐑳𐑤𐑪𐑕𐑳𐑓𐑰, 𐑒𐑧𐑥𐑳𐑕𐑑𐑮𐑰, 𐑡𐑰𐑪𐑤𐑳𐑡𐑰, 𐑥𐑦𐑯𐑻𐑪𐑤𐑳𐑡𐑰, 𐑯 𐑥𐑧𐑯𐑰 𐑳𐑞𐑻 𐑳𐑤𐑴𐑡𐑰𐑕.

\arial

\hfill Excerpt from Jules Vern,  \textit{Journey to the Center of the Earth from \href{http://shavian.weebly.com/}{shavian}}
\end{scriptexample}

The example is typeset using \texttt{code2001.ttf}. There are numerous fonts that provide Shavian glyphs. \texttt{ESL Gothic Unicode} font by Ethan Lamoreaux\footnote{\url{http://www.fontspace.com/ethan-lamoreaux/esl-gothic-unicode}}. The Noto fonts also have a Shavian font. 

You can activate typesetting in Shavian using the key:

\begin{key}{/chapter/shavian font = \meta{font name}} The key will setup the
default font for the Shavian script and define the commands \cmd{\shavian} and \cmd{\textshavian}. 
\end{key}

\PrintUnicodeBlock{./languages/shavian.txt}{\shavian}




