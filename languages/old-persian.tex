\section{Old Persian}
\label{s:oldpersian}


Old Persian, like Hittite an Indo-European language, was written in cuneiforms as of the first millenium BC, mostly between 550 and 350. King Darius’ monumental inscription at
Bisothum – in Old Persian, Elamite and Neo-Babylonian – furnished
the ‘key’ to cuneiform’s decipherment and the reconstruction
of these languages.28 Darius’ Old Persian scribes
effected the most drastic simplification of the borrowed Near
Eastern script (illus. 35). They reduced the cuneiform inventory
to only 41 signs of both syllabic (ka) and phonemic (/k/) values.
Thus, Old Persian cuneiform is ‘half syllabic, half letter writing’.
29 It appears to be on the fence between the Babylonians’
cuneiforms and the Levantines’ consonantal writing, a hybrid
solution using only four logograms and 36 syllabo-phonemic
signs written in wedges. Of particular significance is the fact
that Old Persian also conveys the individual long and short
vowels /a/ (pronounced AH), /i/ (EE) and /u/ (OO) that the
Ugaritic system had conveyed a thousand years earlier.

Old Persian cuneiform is a semi-alphabetic cuneiform script that was the primary script for the Old Persian language. Texts written in this cuneiform were found in Persepolis, Susa, Hamadan, Armenia, and along the Suez Canal.[1] They were mostly inscriptions from the time period of Darius the Great and his son Xerxes. Later kings down to Artaxerxes III used corrupted forms of the language classified as “pre-Middle Persian”.

\begin{scriptexample}[]{Old Persian}
\unicodetable{oldpersian}{"103A0,"103B0,"103C0,"103D0}
\end{scriptexample}

Scholars today mostly agree that the Old Persian script was invented by about 525 BC to provide monument inscriptions for the Achaemenid king Darius I, to be used at Behistun. While a few Old Persian texts seem to be inscribed during the reigns of Cyrus the Great (CMa, CMb, and CMc, all found at Pasargadae), the first Achaemenid emperor, or Arsames and Ariaramnes (AsH and AmH, both found at Hamadan), grandfather and great-grandfather of Darius I, all five, specially the later two, are generally agreed to have been later inscriptions.
Around the time period in which Old Persian was used, nearby languages included Elamite and Akkadian. One of the main differences between the writing systems of these languages is that Old Persian is a semi-alphabet while Elamite and Akkadian were syllabic. In addition, while Old Persian is written in a consistent semi-alphabetic system, Elamite and Akkadian used borrowings from other languages, creating mixed systems.
\medskip

% increa the width by a given dimension to oversize the image and then shift it to the left, irrespective if we are on left
% or right page.
{\leftskip-1.25cm
\includegraphics[width=\textwidth+2.5cm]{./images/naghshe.jpg}
\captionof{figure}{Panoramic view of the Naqsh-e Rustam. This site contains the tombs of four Achaemenid kings, including those of Darius I and Xerxes. (\textit{Wikimedia})}
}