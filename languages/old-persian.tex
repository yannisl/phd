\section{Old Persian}
\label{s:oldpersian}
\newfontfamily\oldpersian{Noto Sans Old Persian}

Old Persian cuneiform is a semi-alphabetic cuneiform script that was the primary script for the Old Persian language. Texts written in this cuneiform were found in Persepolis, Susa, Hamadan, Armenia, and along the Suez Canal.[1] They were mostly inscriptions from the time period of Darius the Great and his son Xerxes. Later kings down to Artaxerxes III used corrupted forms of the language classified as “pre-Middle Persian”.

\begin{scriptexample}[]{Old Persian}
\unicodetable{oldpersian}{"103A0,"103B0,"103C0,"103D0}
\end{scriptexample}

Scholars today mostly agree that the Old Persian script was invented by about 525 BC to provide monument inscriptions for the Achaemenid king Darius I, to be used at Behistun. While a few Old Persian texts seem to be inscribed during the reigns of Cyrus the Great (CMa, CMb, and CMc, all found at Pasargadae), the first Achaemenid emperor, or Arsames and Ariaramnes (AsH and AmH, both found at Hamadan), grandfather and great-grandfather of Darius I, all five, specially the later two, are generally agreed to have been later inscriptions.
Around the time period in which Old Persian was used, nearby languages included Elamite and Akkadian. One of the main differences between the writing systems of these languages is that Old Persian is a semi-alphabet while Elamite and Akkadian were syllabic. In addition, while Old Persian is written in a consistent semi-alphabetic system, Elamite and Akkadian used borrowings from other languages, creating mixed systems.