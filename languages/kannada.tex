\subsection{Kannada alphabet}
\label{s:kannada}
\index{Scripts>Kannada}

\newfontfamily\kannada[Scale=1.0,Script=Kannada]{Lohit-Kannada.ttf}

\def\kannadatext#1{{\kannada #1}}

The Kannada people known as the Kannadigas and Kannadigaru, (sometimes referred to in English as Canarese),[14] are the people who natively speak Kannada.[15] Kannadigas are mainly found in the state of Karnataka in India. Kannada minorities are also found in the neighboring states Maharashtra,[3] Tamil Nadu,[16] Andhra Pradesh, Goa[17][18] and in most Indian states.[3] The English plural is Kannadigas. After a millennium of disintegration from Old Kannada into various languages,[19][12] sister languages[20] and Kannada dialects,[8] modern Kannada stands among 30 most widely spoken languages of the world as of 2001.[7][6] The Kannadiga diaspora can be found all over the world, mainly in the USA, the United Kingdom, Singapore, the UAE and the rest of the Middle East.[21][22][23][24][25][26]\indexindic{Kannada}

\begin{figure}[htbp]
\centering
\includegraphics[width=\linewidth-2\parindent]{kannada}

\caption{Kannada festival.}
\end{figure}



The Kannada alphabet (\kannadatext{ಕನ್ನಡ ಲಿಪಿ}) is an abugida of the Brahmic family,[2] used primarily to write the Kannada language, one of the Dravidian languages of southern India. Several minor languages, such as Tulu, Konkani, Kodava, and Beary, also use alphabets based on the Kannada script.[3] The Kannada and Telugu scripts share high mutual intellegibility with each other, and are often considered to be regional variants of single script. Similarly, Goykanadi, a variant of Old Kannada, has been historically used to write Konkani in the state of Goa.[4]\index{Indic Languages>Konkani}\indexindic{Tulu}\indexindic{Kodava}\indexindic{Beary}



\begin{scriptexample}[]{Kannada}
\centerline{\large\kannadatext{ಙ	ಙ್ಕ	ಙ್ಖ	ಙ್ಗ	ಙ್ಘ	ಙ್ಙ	ಙ್ಚ	ಙ್ಛ	ಙ್ಜ	ಙ್ಝ	ಙ್ಞ	ಙ್ಟ	ಙ್ಠ	ಙ್ಡ	ಙ್ಢ}}
\end{scriptexample}

\medskip

The Kannada script (aksharamale or varnamale) is a phonemic abugida of forty-nine letters, and is written from left to right. The character set is almost identical to that of other Brahmic scripts. Consonantal letters imply an inherent vowel. Letters representing consonants are combined to form digraphs (ottaksharas) when there is no intervening vowel. Otherwise, each letter corresponds to a syllable.

The letters are classified into three categories: swara (vowels), vyanjana (consonants), and yogavaahaka (part vowel, part consonant). \index{swara}\index{vyanjana}\index{yogavaahaka}

The Kannada words for a letter of the script are akshara, akkara, and varna. Each letter has its own form (ākāra) and sound (shabda), providing the visible and audible representations, respectively. Kannada is written from left to right.[7]


% image https://www.quora.com/Why-is-regional-chauvinism-very-high-in-Karnataka