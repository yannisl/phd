\chapter{The Dravidian Languages}

\section{Introduction}



\section{Historical Background}

The major Dravidian scripts used today are those associated with the Kannada,
Telugu, Malayalam and Tamil languages. All these are phonologically based,
and are written from left to right. Historically, they derive from a South Indian
branch of the Brahmi script, used in India from around 250 bce to write Pali (an
Indo-Aryan language) as the medium for Buddhist inscriptions, carved at the orders
of the Emperor Ashoka. The Brahmi script is also the ancestor of scripts
used in North India to the present time, such as Devanagari (used for Sanskrit,
Hindi, Nepali and Marathi); Bengali (used for Bengali and Assamese); Sinhalese
(used for the Sinhala language of Sri Lanka), as well as Tibetan and many
scripts of Southeast Asia, such as Burmese, Siamese (Thai), Laotian and Khmer
(Cambodian).

Many Indologists follow Bühl er (1895, 1896) in believing that both Kharosthi
and Brahmi were derived from Semitic writing - with Brahmi perhaps based on
a Semitic script like the Phoenician, rather than on Aramaic. However, the Indie
systems clearly reflected the phonetic sophistication of the ancient Sanskrit
grammarians. Whereas the Semitic writing systems had no systematic way of indicating
vowels, the inventors of Brahmi introduced a novel method of transcribing
consonants and vowels in a precise way. They did not write consonantal and
vocalic phonemes as independent letters, as is done in Greek; nor did they write
the vowels only with occasional diacritics added to consonant symbols, as is
done in many modem Semitic writing systems. Rather, they adopted the strategy
of writing each consonant-vowel (CV) syllable as a complex unit, called in Sanskrit
an aksara. The general characteristics of the resulting system continue to
be used in the major writing systems of South and Southeast Asia.


Over the centuries, the Brahmi script evolved in a variety of ways in different
parts of the Indian subcontinent, with distinctive developments in the south
(Jensen 1958/1969: 398-403; Dani 1986: 193-214). Among southern scripts, the
Kadamba and Cälukya of the fifth to seventh centuries c e are especially important.
After about the tenth century, these types took on a homogeneous form: the
Old Kanarese script, which was used across the entire Indian peninsula, in the
areas where both Kannada and Telugu are now spoken. By around 1500, this
script had diversified into two closely related varieties, the Kannada and Telugu

Another line of development is reflected in the Chera and Pallava scripts of
South India, dating from the fifth to eighth centuries c e . This eventually took the
form of the Grantha script (from a Sanskrit word meaning ‘book’), which predominated
especially in the Madras area. A western variety of Grantha is the
ancestor of the modem Malayalam system, and an eastern variety of Grantha
was formerly used to write Tamil. However, in the eighth century a competing
script came into use for Tamil - probably reflecting a northern variety of Brahmi,
but with strong influence from Grantha. This system, which has a reduced inventory
of symbols and lacks ‘conjunct’ consonants (see section 2.2), is in fact
better adapted to Classical Tamil phonology than is Grantha, and the newer
script forms the basis for modem Tamil writing. A cursive variety of this Tamil
script, called Vatteluttu, was used between the eighth and fifteenth centuries,
and is said to be still used by the Muslim Mappillas of Malabar for writing
Malayalam.


Tables 2.1a-2.1c show the basic shapes of symbols at major stages in the development
of the scripts mentioned above. The order of listing is the traditional
Indie order (see section 2.2, below). The first column shows probable Semitic
prototypes; the original phonetic values in Semitic are given in parentheses
when these are different from the values in the Brahmi adaptation. The remaining
columns show forms of the Brahmi, Old Kanarese, Kannada, Telugu, Chera,
Grantha, Malayalam and Tamil scripts.

The most commonly spoken Dravidian languages are Tamil (தமிழ்), Telugu (తెలుగు), Kannada (ಕನ್ನಡ), Malayalam (മലയാളം), and Brahui (براہوئی) and Tulu (ತುಳು). There are three subgroups within the Dravidian language family: North Dravidian, Central Dravidian, and South Dravidian, matching for the most part the corresponding regions in the Indian subcontinent.
Dravidian grammatical impact on the structure and syntax of Indo-Aryan languages is considered far greater than the Indo-Aryan grammatical impact on Dravidian. Some linguists explain this anomaly by arguing that Middle Indo-Aryan and New Indo-Aryan were built on a Dravidian substratum.[76] There are also hundreds of Dravidian loanwords in Indo-Aryan languages, and vice versa.



















