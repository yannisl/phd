\chapter{Cypriot Syllabary}
\label{s:cypriot}
\arial
\epigraph{History in this island is almost too profuse.}
{Robert Byron, \textit{The Road to Oxiana} (London, 1937; repr. 1950), 22}
\section{Introduction}

One crucial feature of the Cypro-Archaic period was a continuity of styles, indicating a period of calm without discontinuities caused by hostile invasions.\footcite[page 4, \ldots different foreign powers were perhaps less hostile than is
generally thought. The crucial feature of the Cypro-Archaic period was
not the series of disruptions apparently resulting from invasion, but the
continuity of styles evident in the archaeology.]{Reyes1994} 


Cyprus, an island comprising 9,251 sq. km. or about 3,572 sq. miles,\footcite{Reyes1994} had at least ten separate
kingdoms by the second quarter of the seventh century, according to a prism inscription of the Assyrian king Esarhaddon, dated to 673/2 BC. On that count, the average size of a kingdom would have been 925 sq.km. or 357 sq. miles at most, 'larger than any contemporary Greek polis of the homeland, apart from the two outsize instances of Sparta and
Athens, larger, therefore, than the hundreds of smaller poleis of historical Greece'.20 The extent to which one kingdom interacted with another requires examination, then, since previous studies have simply concentrated on the effects of foreign dominations on the local culture.\footnote{one}\footnote{two} 

\citeauthor{Reyes1994} considered briefly the cultural and social framework of the island, and examined the textual and archaeological sources for Cypriot history from the eighth to the sixth centuries BC, he also investigated the internal and external relations between the different parts of the island.

\section{Cypro-Minoan}

The Cypro-Minoan syllabary (CM) is an undeciphered syllabary used on the island of Cyprus during the late Bronze Age (ca. 1550–1050 BC). The term "Cypro-Minoan" was coined by Arthur Evans in 1909 based on its visual similarity to Linear A on Minoan Crete, from which CM is thought to be derived.[1] Approximately 250 objects—such as clay balls, cylinders, and tablets and votive stands—which bear Cypro-Minoan inscriptions, have been found. Discoveries have been made at various sites around Cyprus, as well as in the ancient city of Ugarit on the Syrian coast.

\newfontfamily\cminoan[Scale=1.5]{Cminoan.ttf}

Before we delve further into the texts and attempts at decipherment, it is prudent to review the conventions and terminology, used in the literature. First we need to understand references to the corpora of epigraphs. These are normally referenced as ENKO Atab 001. The first is an abbreviation for an area, ENKO is for Enkomi, Atab is an abbreviation for the type of object, i.e., ring, stone etc and the number is just an arbitrary number.\footfullcite[][page 27-30]{valerio2016}

\begin{alltt}\cminoan
{\arial ENKO Atab 001} 
󱀀󱀁󱀂󱀃󱀄󱀅󱀆󱀈
 󱀊󱀋󱀌󱀍󱀎󱀏󱀐
󱀑󱀒󱀓󱀔󱀕󱀖
󱀇󱀉
\end{alltt}

\subsection{Place}

\begin{longtable}[l]{>{\ttfamily}ll}
ALAS &Alassa-Palaeotaverna (Limassol)\\
ARPE &Arpera (Larnaca)\\
ATHI &Athienou (Larnaca)\\
CYPR &Cyprus\\
DHEN &Dhenia, or Deneia (Nicosia)\\
ENKO &Enkomi (Famagusta)\\
HALA &Hala Sultan Tekke (Larnaca)\\
IDAL &Idalion (Nicosia)\\
KALA &Kalavasos-Ayios Dhimitrios (Larnaca)\\
KATY &Katydhata (Nicosia)\\
KLAV &Klavdia (Larnaca)\\
KITI &Kition (Larnaca)\\
KOUR &Kourion (Limassol)\\
MAAP &Maa-Palaeokastro (Paphos)\\
MARO &Maroni (Larnaca)\\
MYRT &Myrtou-Pigadhes (Kyrenia)\\
PARA &Ayia Paraskevi (Nicosia)\\
PPAP &Palaeopaphos-Skales (Paphos)\\
PSIL &Psilatos (Famagusta)\\
PYLA &Pyla-Verghi (Famagusta)\\
RASH &Ras Shamra / Ugarit (Syria)\\
SALA &Salamis (Famagusta)\\
SANI &Sanidha (Limassol)\\
SYRI &Syria\\
TIRY &Tiryns (Greece)\\
TOUM &Toumba tou Skourou (Nicosia)\\
\end{longtable}

\subsection{Typological Description}


\begin{longtable}[l]{>{\ttfamily}ll}
Abou & Clay ball (or boule)\\
Adis & Clay disk\\
Aéti & Clay label\\
Aost & Clay ostracon\\
Apes & Clay weight\\
Apla & Clay plaque\\
Arou & Clay cylinder\\
Asta & Clay figurine\\
Atab & Clay tablet\\
Avas & Pottery (complete or fragmentary)\\
Inst & Ivory tool\\
Ipla & Ivory plaque\\
Mbij & Metal jewelry\\
Mexv & Metal ex-voto\\
Mins & Metal tool\\
Mlin & Metal ingot\\
Mvas & Metal base\\
Pblo & Stone block\\
Pfus & Stone spindle whorl\\
Ppla & Stone plaque\\
Psce & Stone seal\\
Vsce & Glass seal\\
\end{longtable}


Since the publication of HoChyMin, it seems to have become the common
practice to cite the inscriptions by absolute sequential number. In this book, however, following I have 
opted to refer to them by label. While this may seem less economical in terms of space,
labels are more informative and probably easier to associate to the actual inscriptions.
Hopefully, this will become clear in Chapter 3, where they are useful to infer the
epigraphical support of particular paleographical variants in the inter-script comparative
tables.

One potential shortcoming of the system, noted by Olivier himself, is that the
typological descriptions are in a certain measure arbitrary and empirical.17 For example,
in the case of |ENKO Apes 001| the name implies that the object inscribed is a weight
("pes"), but this interpretation is debated and it has also been proposed that the support
was actually used as a label.18 Still, such cases are a minority and will be duly signaled.
To facilitate the reading of this dissertation and comparisons with other publications, the
correspondence between sequential numbers and labels is given in the Concordance that
precedes this Introduction.

For reasons of economy, in the transnumeration of sign-sequences I use a system
slightly different from the one found in HoChyMin and now followed on publications
on Cypro-Minoan. Signs whose numbers are below 100 are transnumerated with only
two digits. For example, the sign-group 102-009-082-085 is here given as 102-09-82-85. 
In tentative transliterations, untransliterated signs are kept in transnumeration but
preceded by an asterisk (*): e.g. i?-li?-*71-ni?.

\section{History of Scholarship}

The CM script was studied in 1900 when Arthur Evans analyses three clay \textit{boules}
found during the British excavations in 1896 and the gold ring from Hala Sultan Tekke (HST 3).\footnote{Evans (1909):71, fig.39 (table 3): he matches 15 out of the 15 signs with alleged parallels to Linear A and B; those signs that do not show conformity with the signaries are paralleled with the Cretan Hieroglyphic script.}

In 1935 Evan's reinforced the palaeographical connections between the Aegean scripts and Cypro-Minoan, adding the
evidence provided by a further clay \textit{boule} ENK. 3 the Ayia Paraskevi cylinder seal, and of ceramic fragment Enkomi A 1507 (ENK. 105)


\section{The Signary}



\section{Cypriote syllabary}

The Cypriot or Cypriote syllabary is a syllabic script used in Iron Age Cyprus, from ca. the 11th to the 4th centuries BCE, when it was replaced by the Greek alphabet. A pioneer of that change was king Evagoras of Salamis. It is descended from the Cypro-Minoan syllabary, in turn a variant or derivative of Linear A. Most texts using the script are in the Arcadocypriot dialect of Greek, but some bilingual (Greek and Eteocypriot) inscriptions were found in Amathus.\footfullcite{powell1991}\tcbdocmarginnote{rev. after china}

\begin{figure}[htbp]
\includegraphics[width=\textwidth]{Tablet_cypro-minoan_2_Louvre_AM2336}
\end{figure}

The existence of a local Cypriote script was first demonstrated in
1852 by the collector and antiquarian, the Due de Luynes, on the basis of
some inscribed coins and a few other inscriptions.39 The Assyriologist
George Smith offered the key to decipherment in 1871, though he
remained reluctant, because of the writing's oddity when compared with
Greek alphabetic writing, to conclude that the underlying language was
Greek. By 1875, through the efforts of philologists in several countries, the
decipherment was substantially complete, and the language of most of the
inscriptions was proved to be written in what is now called the Arcado-
Cypriote dialect of Greek. Many later finds allow one to make the
following general description of Cypriote writing.\footcite{powell1991}

From c. 1600 to 1050 B.C. an undeciphered writing similar in form to the
classical Cypriote syllabary was in use on Cyprus and in Ras Shamra in
North Syria. Sir Arthur Evans aptly called this script ``Cypro-Minoan'' by
reason of its formal affinities with Linear A and Β and with the classical
Cypriote writing;40 the term is now standard. Formal similarities make it
probable that Cypro-Minoan is derived from Cretan writing, but their
exact relation cannot be determined. Most will agree that Cypro-Minoan
records pre-Greek languages spoken on Cyprus.

\begin{figure}[htb]
\parindent0pt
\centering

\includegraphics[width=\textwidth]{./images/idalion-tablet.jpg}

\caption[Idalion tablet.]{The bronze Idalion Tablet, from Idalium, (Greek: Ιδάλιον), is from the 5th century BCE Cyprus. The tablet is inscribed on both sides. The script of the tablet is in the Cypro-Minoan syllabary, and the inscription is in Greek. The tablet records a contract between "the king and the city":[1] the topic of the tablet rewards a family of physicians, of the city, for providing free health services to individuals fighting an invading force of Persians.}
\label{fig:idalion}
\end{figure}


The oldest dated inscriptions in the classical Cypriote syllabary are from
the eighth century B.C., very close to the date of the invention of the
alphabet. We are thus left with a troubling hiatus of 300 years between the
latest attestation of Cypro-Minoan writing and the first of classical
Cypriote writing.42 Nonetheless the Cypriote syllabary is doubtless an
adaptation of the Cypro-Minoan. It is notable that the Cypriote syllabary
remained the preferred means of recording Greek on the island of Cyprus,
even after alphabetic writing was also known. The two scripts were used
side-by-side, until, under foreign rule by the Ptolemies, the syllabary was
driven out sometime in the late third century B.C.

About 500 texts written in the Cypriote syllabary are extant. A few
record an unknown, non-greek language usually called Eteocypriote.43
The wide subject matter of the Greek-language texts, inscribed on a
diversity of objects, includes sepulchral, votive, and honorary topics.
There are even four hexameters (below, inrT.). We can identify two
principal varieties of the Cypriote syllabary; one was confined to the
southwest of the island in the area of Old and New Paphos, Rantidi, and
Kition (so-called syllabaire paphien); the other, formally somewhat
different, was used over the rest of the island. The Paphian texts are
written from right to left, the others from left to right.

Cypriote writing is a pure syllabary, without logograms (except for
numerals) and associated indicative signs and devices. Five signs stand for
the pure vowels [a], [e], [i], [o], [u] (just as in Linear B). 

About fifty other signs represent open syllables, consisting of a consonant plus one of the
five vowels (see Table~\ref{tbl:cypriote}). No distinction is made between voiced, aspirated, and unvoiced stops so that, for example, πα, φα, βα are all represented by the same sign, as are τα, θα, δα44 and κα, χα, γα.45 There
seem to be special signs for [xa] and [xe]. Because the syllabograms stand
for open syllables and Greek contains many consonant clusters and final
closed syllables, complicated rules govern the working of Cypriote in the
spelling of Greek (the same is true of Linear B).

The characters are \textit{syllabic}. There is one character for each  vowel, \textit{a, e, i, o, u,} and perhaps one for \textit{o}. There is no distinction between long and short vowels. The other characters represent what are called \textit{open syllables}\footnote{ If a syllable ends with a consonant, it is called a closed syllable. If a syllable ends with a vowel, it is called an open syllable. }, i.e., beginning with a consonant and ending with a vowel. 

No distinction is made between smooth, middle and rough mutes. The same character stands for τά τ\'ασs, δα in Εδαλιον ανδ δα ιν Αθανα  κε, κη, γε, γη, χε, χη. This fact constitutes the greatest difficulty in reading Cypriote.  

Let us now examine a sentence from the celebrated bronze tablet from
Idalion (Fig.~\ref{fig:idalion}), one of the earliest Cypriote inscriptions found, and still
the longest. The tablet, now in the Cabinet des Medailles in the
Bibliotheque Nationale in Paris, was acquired in 1850 by the Due de
Luynes. It had been suspended from an attached ring in the temple of
Athena at Idalion to record an agreement between a certain King
Stasikypros, probably the last king of the city of Idalion, and a physician
by the name of Onasilos, concerning the treatment of the wounded after
a siege of Idalion by the Medes and the people of Kition. The inscription
informs us that the king and the city will reimburse the physicians for their
labors with money and land. 

The document evidently reflects the military
campaigns against Idalion just before Idalion was absorbed into the
kingdom of Kition c. 470; O. Masson dates it to 478-70 B.C. Fig. 9 gives
the Cypriote text with interlinear transliteration into Roman characters.46
The original reads from right to left, but for convenience I have rewritten
the text to read from left to right; numerals in parentheses indicate line
numbers in the original text.

\section{Dedication to Demeter}

\begin{figure}[htbp]
\includegraphics[width=\linewidth]{cypriote-63}
\caption{Dedication to Demeter and kore of Hellooikos, son of Poteisis. Late fourth century bc. \textit{London, British Museum, Reg. no. 96, 2-1, 215. From the Temenos of Demeter, excavations of the British Museum.}}
\end{figure}

Mitford describes the pedestral as:

\begin{quotation}
Pedestral of a fine white marble, undamaged, with molded cap and base. Width 0.28m.; height 0.065 m.; Found in the early months of 1895 in excavations conducted by H.B. Walters "on temple-site (C)" (plan 1).

\indent The inscription, equally undamaged, is composed of two alphabetic lines above one syllabic. Both texts are admirably cut, their characters regular, clear, well formed, without serifs or \textit{apices}. Height of the former from to o.008 m.; of the latter, 0.006 to 0.008 m.
\end{quotation}



\ExplSyntaxOn
\def\startCypriote{\begingroup\par\leavevmode\cypriote
\def\a{%
       $\stackrel{
          \mbox{\strut\char"10800}
          }%
          {\mbox{\strut\arial a-}%    
          }$
   }%  
\def\e{
       $\stackrel{\mbox{\char"10801}}{\mbox{\arial e - }}$
      }
\def\i{
       $\stackrel{\mbox{\strut\char"10802}}{\mbox{\strut\arial i - }}$
      }
\def\o{$\stackrel{\mbox{\char"10803}}{\mbox{\arial o - }}$}
\def\u{$\stackrel{\mbox{\char"10804}}{\mbox{\arial u - }}$}
\def\wa{$\stackrel{\mbox{\char"10832}}{\mbox{\arial wa -}}$}
\def\we{\char"10833}%
\def\wi{\char"10834}%
\def\wo{\char"10835}%
\def\za{\char"1083C}%
\def\zo{\char"1083F}%
\def\ja{}%
\def\jo{}%
\def\ka{\char"1082F}%
\def\ke{\char"1080B}%
\def\ki{\char"1080C}%
\def\ko{\char"1080D}%
\def\ku{\char"1080E}%
\def\la{\char"10816}%
\def\le{\char"10810}%
\def\li{\char"10811}%
\def\lo{\char"10812}%
\def\lu{\char"10813}%
\def\ma{\char"10814}%
\def\me{\char"10815}%
\def\mi{\char"10816}%
\def\mo{\char"10817}%
\def\mu{\char"10818}%
\def\na{\char"10819}
\def\ne{\char"1081A}
\cs_set:Npn\ni{\char"1081B}
\cs_set:Npn\no{\char"1081C}
\cs_set:Npn\nu{\char"1081D}
\def\ksa{\char"10837}
\def\kse{\char"10838}
\def\pa{\char"1081E}
\def\pe{\char"1081F}
\def\pi{\char"10820}
\def\po{\char"10821}
\def\pu{\char"10822}

\def\ra{\char"10823}
\def\re{\char"10824}
\def\ri{\char"10825}
\def\ro{\char"10826}
\def\ru{\char"10827}
% s-
\def\sa{\char"10828}
\def\se{\char"10829}
\def\si{\char"1082A}
\def\so{\char"1082B}
\def\su{\char"1082C}
% t-
\def\ta{\char"1082D}
\def\te{\char"1082E}
\def\ti{\char"1082F}
\def\to{\char"10830}
\def\tu{\char"10831}
}

\def\stopCypriote{\endgroup}
\ExplSyntaxOff

\captionof{table}{The Cypriote Syllabary}
\label{tbl:cypriote}
\startCypriote
\let\ar\arial\cypriote\large
\begin{longtable}{>{\ar}c| c c c c c}
   &\ar a &\ar e &\ar i &\ar o &\ar u\\
   \hline
  &\a &\e &\i &\o &\u\\
 w &\wa &\we         &\wi         &\wo         &           \\ 
 z & \za           &            &            &\zo            &           \\ 
 j &               &            &            &               &           \\
 k-,g-,kh- &\ka       &\ke            &\ki            &\ko               &\ku     \\
 l         &\la       &\le            &\li            &\lo               &\lu     \\
 m         &\ma       &\me            &\mi            &\mo              &\mu        \\
 n         &\na       &\ne            &\ni            &\no               &\nu     \\
 ks-       &\ksa      &\kse           &               &                  &     \\
 p-,b-,ph  &\pa       &\pe            &\pi            &\po               &\pu   \\
 r-        &\ra       &\re            &\ri            &\ro               &\ru   \\
 s-        &\sa       &\se            &\si            &\so               &\su   \\
 t-, d-,th-  &\ta     &\te            &\ti            &\to               &\tu   \\ 
\end{longtable}  

\stopCypriote

From the above the Cypriote syllabary may at first appear ill-suited to the writing of the Greek language. Powell\footcite[][page 44]{powell1991} correctly observes that it is in fact surprisingly well designed as it
imparts phonetic information about the underlying language once one has
mastered the spelling rules. Lacking the apparatus of logograms, sign
indicators, phonetic and semantic complements, and adjective signs of the
ancient logo-syllabic writings, and therefore different in kind from its
Egyptian or Akkadian antecedents, the Cypriote syllabary is a purely
phonetic writing of admirable simplicity and clarity, a high achievement
in the history of writing:

\subsection{Annotations}

The text is annotated in the following lines, where also the rules are outlined.

\begin{enumerate}
\item The script does not distinguish between aspirated and unaspirated vowels. O-TE (line 1) stands for "Οτε and
A stands for ά (line 7).

\item Final consonants are always rendered by the " e " series of syllabic
signs, i.e. the appropriate consonant plus the vowel e (§39.3). Thus the
sign for NE renders final [n] of πτόλιν (line ι), ' Εδάλιον (line 2),
κατέfοργον (line 2), Φιλοκύπρων (line 5), άνωγον (line 9), Όνασίλον
(line 10), τον (line 10), ίνατήραν (line 11), and μισθών (line 17).
SE by the same principle stands for final [s] in κά (lines 3, 12), κετιήρες
(line 4), βασιλεύς (line 6), Στασίκυπρος (line 7), πτόλι (line 8),
Έδαλιήρες (line 9), τό (line 12), κασιγνήτος (line 13), ά(ν)θρώποs
(line 14), and ικμαμένος (line 16).

The appearance of signs in the " e " series in final .position without
word-dividers seems to show that in position before another word
beginning with a vowel final NE or SE are regarded as virtual
consonants; except in the case of diphthongs, or when an internal letter
such as [s] or [p] has dropped out, two or more vowels do not appear
together in the Cypriote syllabary (§35.2—4).
\item  Observe that the prosodic use of word-dividers is not consistent. For
some reason they are particularly apt to be omitted in the first lines of
a text between words in close association, as here between
PO-TO-LI-NE (τττόλι*) and E-TA-LI-O-NE ('Εδάλιον) (line 1);
between PI-LO-KU-PO-RO-NE (Φιλοκύπρων), WE-TE-I
(ρέτει), and TO-O-NA-SA-KO-RA-U (τω Όνασαγόραυ) (lines
4-5); and between TO-NO-NA-SI-KU-PO-RO-NE (τον
Όνασικύπρων) and TO-NI-YA-TE-RA-NE (τον iycnfpav) (lines
10-11). 

Word division is also readily omitted between a subject and
its predicate, as here between \textsc{KA-TE-WO-RO-KO-NE} (κατέρ-opyov) and MA--TO--Ι (Μάδοι) (lines 2—3); and between
A - N O - K O - N E (avcoyov) and O-NA-SI-LO-NE (Όνασίλον)
(line 9).


\item When, in an internal consonant cluster, the consonants belong to
separate syllables (not as in annotation no. 4), then the first consonant
is rendered by the sign that has the vowel belonging to the preceding
syllable (§42.4). Thus:

I-YA-SA-TA-I = Ινάσθαι (line 13)

(But in this case the rule is disguised because the syllable that follows
SA — namely TA - has the same vowel as the syllable that precedes SA
— namely YA).

MI-SI-TO-NE (not *MI-SO-TO-NE) for μισθών (line 17)
MCJ-MA-ME-NO-SE (not *I=KA-MA-ME-NO-SE) for ίκ(?)μαμένοs (lines
15-16).
\end{enumerate}

\section{Writing Medium}
\lorem\lorem\lorem

\begin{figure}[htbp]
\centering

\includegraphics[width=0.6\textwidth]{oldest-book}

\end{figure}

\subsection{Transliteration}
Using a set of macros we can easily type in the syllabary with little effort

{\Large\startCypriote \a{\arial-}\i\o  \te\ta\po\to\to\li\ne\e\ta\li \stopCypriote}


\subsection{Unicode encoding}
The Cypriot syllabary was added to the Unicode Standard in April, 2003 with the release of version 4.0.
The Unicode block for Cypriot is \unicodenumber{U+10800–U+1083F}. The Unicode block for the related Aegean Numbers is \unicodenumber{U+10100–U+1013F}.



\begin{scriptexample}[]{Cypriot Syllabary}
\unicodetable{cypriote}{"10800,"10810,"10820,"10830}

\cypriote \symbol{"10803}
\end{scriptexample}


\section{Refs}

In \cite{Reyes1994} we find the relationships between Cyprus and its neighbours as well as evidence between trading in the City Kingdoms.\footcite[See pages 5-34]{Reyes1994}





\printunicodeblock{./languages/cyprus.txt}{\cypriote}


\bgroup
\newfontfamily\ipafont{Charis SIL}

\ipafont


\begin{IPA}
\textipa{[""Ekspl@"neIS@n]}
\textipa{\;B \;E \;A \;H \;L \;R}\\
\textipa{\!b \!d \!g \!j \!G \!o}\\
\tone{55}ma ‘‘mother’’, \tone{35}ma ‘‘hemp’’


\textipa{iDa}
\end{IPA}

\LARGE
\noindent p t̪ t ʈ t͡ʃ c k\\
pʰ t̪ʰ ʈʰ t͡ʃʰ cʰ kʰ \\
b d̪ ɖ d͡ʒ ɟ ɡ \\
bʱ d̪ʱ ɖʱ d͡ʒʱ ɟʱ ɡʱ \\
f s ʂ ʃ h \\
m n̪ n ɳ ɲ ŋ \\
r ɽ ͏ɻ\\
l ɭ \\
w v j
W+ 
a=sdfɡtⱳsᵭ v

ᵭ ħ ʞ ɭ ɲ ɱ ɭ ᵽ 
\egroup


With Unicode and the right font, there is no problem  in typesetting IPA phonetic symbols. However the problem is the input.

I recommend that you get familiar with a Unicode IPA keyboard overlay. I have used Keyman. When the keyboard is turned on, certian keys (`,@,=) are activated.

As long as your editor allows Unicode input (most do these days) and you're compiling with XeLaTeX or LuaLaTeX, you can just use the IPA keyboard to type directly into the editor just as you can in most other applications. You can also copy and paste your Unicode text from other applications too.

p̛ tt≠pljk ᶊ˥



ɑ ɲ ɲ ɸ β ɹ ɬ ɭ ɮ l ʘ ɓ ǂ ʍ w Waiting ɧ ħ ʌ ɒ 

ɨmnƙ ɮ ʠɰɜɾƭƭyʌiƥ

kʰ 

t̥ataraṭ tʷara zʷara ɣ anɣᵘish ma˨˥˨ a᷅ āltoʃ

This will take time to get upto speed. One can also ofcourse write his own macros for commonly used symbols or words.

d, g,h
˘
d, g, ʂʃs̥s̊å, s, z 

ɬ = 
l̊ at
ɭ  < 
ɮ  >
l̥  ̥
lˈ 
l` l'

I found it frustrating at first to try and remember all the combinations of symbols, especially since I am not a linguist---although I have done a lot of background self-study. In my estimation, handwriting is probably still the fastest way to typeset anything to do with phonetics.

Gâteau Basque, like


ɐ ɑ a ʈ ᵵ ʇ ᴛ ʦ ʧ ʨ ṣ

\textit{da:ta, gaːka, etc.}, 

Hittite vowel phonemes
{\obeylines\large\panunicode
i u ī ū
e a ē ā 
}
Diphthongal combination like that of ˘¯ a and the glides w and y, noted (a-)a-i, (a-)a-u, are
also permitted h₂, h₃
 u, ú, ù, u$_4$


\emph{ši-ú-ni-iš} 

In addition, the order in which the supplementals occur differs in the blue
and red abeecdaria: blue shows the sequence 0, χ, ψ, with the respective values
p\textsuperscript{h}, [kh], [ps] (though the light blue alphabets lack ψ, as discussed above); on
the other hand, the order is χ. Φ. ψ. with the respective values [ks], [ph], [kh],
in the red.
