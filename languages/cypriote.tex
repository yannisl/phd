\section{Cypriot Syllabary}
\label{s:cypriot}
The Cypriot or Cypriote syllabary is a syllabic script used in Iron Age Cyprus, from ca. the 11th to the 4th centuries BCE, when it was replaced by the Greek alphabet. A pioneer of that change was king Evagoras of Salamis. It is descended from the Cypro-Minoan syllabary, in turn a variant or derivative of Linear A. Most texts using the script are in the Arcadocypriot dialect of Greek, but some bilingual (Greek and Eteocypriot) inscriptions were found in Amathus.

\begin{figure}[htb]
\centering
\begin{minipage}{7cm}
\includegraphics[width=7cm]{./images/idalion-tablet.jpg}
\end{minipage}\hspace{1.5em}
\begin{minipage}{6cm}
\captionof{figure}{The bronze Idalion Tablet, from Idalium, (Greek: Ιδάλιον), is from the 5th century BCE Cyprus. The tablet is inscribed on both sides.
The script of the tablet is in the Cypro-Minoan syllabary, and the inscription is in Greek. The tablet records a contract between "the king and the city":[1] the topic of the tablet rewards a family of physicians, of the city, for providing free health services to individuals fighting an invading force of Persians.}
\end{minipage}
\end{figure}

The Cypriot syllabary was added to the Unicode Standard in April, 2003 with the release of version 4.0.
The Unicode block for Cypriot is \unicodenumber{U+10800–U+1083F}. The Unicode block for the related Aegean Numbers is \unicodenumber{U+10100–U+1013F}.

\newfontfamily\cypriote{Aegean.ttf}

\begin{scriptexample}[]{Cypriot Syllabary}
\unicodetable{cypriote}{"10800,"10810,"10820,"10830}

\cypriote \symbol{"10803}
\end{scriptexample}


\printunicodeblock{./languages/cyprus.txt}{\cypriote}
