\section{Brahmi}
\label{s:brahmi}
Brāhmī is the modern name given to one of the oldest writing systems used in the Indian subcontinent and in Central Asia during the final centuries BCE and the early centuries CE. Like its contemporary, Kharoṣṭhī, which was used in what is now Afghanistan and Western Pakistan, Brahmi (native to north and central India) was an \emph{abugida}.

The A´sokan Br¯ahm¯ı of the third century BCE is the mother of all major Indian scripts,
both Indo-Aryan and Dravidian. It was an alpha-syllabic script with diacritics used for
vowels occurring in postconsonantal position. It has separate symbols for the five primary
vowels a i u e o, twenty-five occlusives and eight sonorants and fricatives. The Br¯ahm¯ı
script was used in the rock edicts set up by the Mauryan Emperor A´soka to spread the
Buddhist faith in different parts of the country. The languages represented were Pali
and certain early regional varieties of Middle Indic. The origin of the Br¯ahm¯ı script is
controversial; nearly half of the characters are said to bear similarity to the consonant
symbols employed in the South Semitic script, eventually traceable to Aramaic script of
2000 BCE (Daniels and Bright 1996: §30, 373–83).

The best-known Brahmi inscriptions are the rock-cut edicts of Ashoka in north-central India, dated to 250–232 BCE. The script was deciphered in 1837 by James Prinsep, an archaeologist, philologist, and official of the East India Company.[1] The origin of the script is still much debated, with current Western academic opinion generally agreeing (with some exceptions) that Brahmi was derived from or at least influenced by one or more contemporary Semitic scripts, but a current of opinion in India favors the idea that it is connected to the much older and as-yet undeciphered Indus script

\begin{figure}[htb]
\centering
\includegraphics[width=0.6\textwidth]{./images/ashoka-pillar.jpg}
\caption{Brahmi script on Ashoka Pillar}
\end{figure}



\begin{scriptexample}[]{Brahmi}
\bgroup
\raggedleft
\brahmi

         
   

\arial
\hfill Text: Asokan Edict typeset with \texttt{NotoSansBrahmi-Regular} 
\egroup
\end{scriptexample}

Brahmi is a Unicode block containing characters written in India from the 3rd century BCE through the first millennium CE. It is the predecessor to all modern Indic scripts.

\begin{scriptexample}[]{Brahmi}
\unicodetable{brahmi}{"11000,"11010,"11020,"11030,"11040,"11050,"11060,"11070}
\end{scriptexample}


\printunicodeblock{./languages/brahmi.txt}{\brahmi}







