\chapter{Ugaritic}
\label{s:ugaritic}
\index{Ugaritic fonts>Noto Sans Ugaritic}
\index{Ugaritic}
\index{Akkadian}
\index{Unicode>Ugaritic}
\parindent1em
\newfontfamily\ugaritic{NotoSansUgaritic-Regular.ttf}

\section{Background}
Sometime between 1190-1185 bce, the houses of Ugarit were abandoned by their inhabitants, then pillaged and burned. If they were destroyed by the Sea Peoples we will never know for sure, although this is very likely. This catastrophe ended a history of almost 6000 years. Ugarit was never rebuild and the ruins were buried for centuries before they were discovered in 1929. 

\begin{figure}[htbp]
\centering
\includegraphics[width=\textwidth]{ugarit-excavations}
%http://www.persee.fr/docAsPDF/syria_0039-7946_1936_num_17_2_3887.pdf
\end{figure}

Merchants figure prominently in Ugarit’s archives. The citizens engaged in trade, and many foreign merchants were based in the state, for example from Cyprus exchanging copper ingots in the shape of ox hides. The presence of Minoan and Mycenaean pottery suggests Aegean contacts with the city. It was also the central storage place for grain supplies moving from the wheat plains of northern Syria to the Hittite court.

common defence system (§ 11.5.4.3). The abundance of Cypriot
pottery,173 the Cypro-Minoan texts found in Ugarit ( L i v e r a n i 1979a,
1322-3) as well as letters18 and administrative texts,19 are also witness
to relationhips between the two communities at both the cultural
and the commercial levels. 

Some Cypriots (ally, altyy, DLU, 33)
receive from the Ugaritian administration food and clothing,20 others
belong to the guild of craftsmen.21 On the other hand, from its structure
the administrative text KTU 4.102 = RS 11.857 seems to be
a list of prisoners of war, or of persons detained for some reason,
who come from Cyprus ( V i t a 1995a, 108). An unpublished letter
found in Ras Shamra in 1994, which reports the dispatch of an
emissary of the king of Cyprus to Ugarit to deal with the freeing of
Cypriots detained on Ugaritic soil,22 could support this hypothesis

The \idxlanguage{Ugaritic} language  is written in alphabetic cuneiform. This was an innovative blending of an alphabetic script (like \hyperref[s:hebrew]{Hebrew}) and cuneiform (like Akkadian). The development of alphabetic cuneiform seems to reflect a decline in the use of Akkadian as a \textit{lingua franca} and a transition to alphabetic scripts in the eastern Mediterranean. Ugaritic, as both a cuneiform and alphabetic script, bridges the cuneiform and alphabetic cultures of the ancient Near East.


\begin{figure}[hb]
\centering
\includegraphics[width=\textwidth]{ugaritic-first-tablet}
\caption{A list of offerings with the first tablet number (KTU 1.39 = RS 1.001; Photo: UGARIT - FORSCHUNG Archive)}
\end{figure}

The Ugaritic script is a cuneiform (wedge-shaped) abjad used from around either the fifteenth century BCE[1] or 1300 BCE[2] for Ugaritic, an extinct Northwest Semitic language, and discovered in Ugarit (modern Ras Shamra), Syria, in 1928. It has 30 letters. Other languages (particularly Hurrian) were occasionally written in the Ugaritic script in the area around Ugarit, although not elsewhere.


\section{Material Culture}

Excavations at Ugarit have yielded an abundance of objects of everyday life that we can deduce the every day life of its inhabitants in a higher level of detail than many other civilizations. Objects recovered include mirrors, combs, cooking and drinking utensils, pottery, gems. An interesting item is the clepsydra shown in Figure~\ref{fig:clepsidra} used as a shower head. The religion and cults is also well represented. This is not easy to use as an individual and it was probably used with the help of a servant.

The Ugarites were actively interacting in trading. 

\begin{figure}[htbp]
\includegraphics[width=\textwidth]{clepsidra}
\caption{“Clepsydra” or shower vase RS 81.509
1981, City Center, House E, room 1201. Latakia Museum
H 19.5 cm, Diameter (max.) 18 cm. Fine plain buff pottery with burnished surface. Jug with a large,
ovoid body. The opening is narrow, contracting to a small hole 1 cm in diameter. The bottom is
pierced with 22 small holes to form a strainer. The narrowness of the opening does not permit filling
by any means other than plunging the vase entirely into a large container full of water. It holds about
1 liter. The function of this sort of vase is obvious. The container remained full if the opening was
sealed with one’s thumb to prohibit the entrance of air; the liquid could not flow out through the
bottom. When the thumb was removed (allowing air to enter the jug), the water could flow out
through the bottom, creating a type of shower head.
This object matches the definition of a clepsydra mentioned by ancient authors (Hieron): in its
primary sense, the term clepsydra is not restricted to a measure of time. What we have here is an instrument
used for washing, like a shower in a bathing installation (or shower stall). This was an object
of everyday life, but only in a relatively refined context. This vase was found with other personal
funerary objects fallen from the upper floor of a house of medium status in the city center. Other examples
(e.g., RS 30.325) show that this was not an uncommon item in homes at Ugarit.\\
– Bib.: M. Yon, P. Lombard, and M. Renisio, in RSO III, 1987, p. 106, fig. 87; P. Lombard, ibid., pp. 351–57.}
\label{fig:clepsidra}
\end{figure}

Clay tablets written in Ugaritic provide the earliest evidence of both the North Semitic and South Semitic orders of the alphabet, which gave rise to the alphabetic orders of Arabic (starting with the earliest order of its abjad), the reduced Hebrew, and more distantly the Greek and Latin alphabets on the one hand, and of the Ge'ez alphabet on the other. Arabic and Old South Arabian are the only other Semitic alphabets which have letters for all or almost all of the 29 commonly reconstructed proto-Semitic consonant phonemes. 

According to Dietrich and Loretz in Handbook of Ugaritic Studies (ed. Watson and Wyatt, 1999): "The language they [the 30 signs] represented could be described as an idiom which in terms of content seemed to be comparable to Canaanite texts, but from a phonological perspective, however, was more like Arabic."
The script was written from left to right. Although cuneiform and pressed into clay, its symbols were unrelated to those of the Akkadian cuneiform.

\begin{scriptexample}[]{Ugaritic}
\unicodetable{ugaritic}{"10380,"10390}
\end{scriptexample}

{\let\aegean\arial
\printunicodeblock{./languages/ugaritic.txt}{\ugaritic}
}

\bgroup

\let\a\arial
\Large
\begin{longtable}[l]{%
>{\arial\large}r|
>{\ugaritic}c| 
>{\arial\large}c 
>{\arial\large}c 
>{\arial\large}c >{\arial\large}c
}

&\a Sign	&\a Trans.	&\a IPA	&\a Hebrew	&\a Arabic \\
\hline
\inc &𐎀	&ʾa	& ʔa	&א	&أ \\
\inc &𐎁	&b	& b	    &ב	&ب \\
\inc &𐎂	&g	&ɡ	&ג	&ج\\
\inc &𐎃	&ḫ	&x	&	&خ\\
\inc &𐎄	&d	&d	&ד	&د\\
\inc &𐎅	&h	&h	&ה	&ه\\
\inc &𐎆	&w	&w	&ו	&و\\
\inc &𐎇	&z	&z	&ז	&ز\\
\inc &𐎈	&ḥ	&ħ	&ח	&ح\\
\inc &𐎉	&ṭ	&t̴	&ט	&ط\\
\inc &𐎊	&y	&j	&י	&ي\\
\inc &𐎋	&k	&k	&כ	&ك\\
\inc &𐎌	&š	&ʃ	&ש	&ش\\
\inc &𐎍	&l	&l	&ל	&ل\\
\inc &𐎎	&m	&m	&מ	&م\\
\inc &𐎏	&ḏ	&ð	&	&ذ\\
\inc &𐎐	&n	&n	&נ	&ن\\
\inc &𐎑	&ẓ	&θ̴	&	&ظ\\
\inc &𐎒	&s	&s	&ס	&س\\
\inc &𐎓	&ʿ 	&ʕ	&ע	&ع\\
\inc &𐎔	&p	&p	&פ	&ف\\
\inc &𐎕	&ṣ	&s̴	&צ	&ص\\
\inc &𐎖	&q	&q	&ק	&ق\\
\inc &𐎗	&r	&r	&ר	&ر\\
\inc &𐎘	&ṯ	&θ	&	&ث\\
\inc &𐎙	&ġ	&ɣ	&	&غ\\
\inc &𐎚	&t	&t	&ת	&ت\\
\inc &𐎛	&ʾi	&ʔi	&	&ئ\\
\inc &𐎜	&ʾu	&ʔu	&	&ؤ\\
\end{longtable}
\egroup


\textit{\LARGE$$\stackrel{\mbox{ho}}{.}$$}

% Tranliteration macros 
% 
\bgroup\ugaritic
\def\a{\char"10380}
\def\b{\char"10381}
\def\g{\char"10382}
\LARGE \a \b \g 
\egroup

\section{Online Collections}

http://digital.library.stonybrook.edu/










