\newfontfamily\parthian{NotoSansInscriptionalParthian-Regular.ttf}
\section{Inscriptional Parthian}
\label{s:parthian}
\index{Ancient and Historic Scripts>Inscriptional Parthian}
\index{Inscriptional Parthian fonts>Noto Sans Inscriptional Parthian}
\index{scripts>Inscriptional Parthian}

The Parthian script developed from the Aramaic alphabet around the 2nd century BCE and was used during the Parthian and Sassanid periods of the Persian Empire. The latest known inscription dates from 292 CE. 

We use the font |NotoSansInscriptionalParthian-Regular.ttf| to render the script. 


Inscriptional Parthian is a Unicode block containing characters of the official script of the Sassanid Empire.

\newenvironment{parthiannumbers}{%
\def\1{\parthian\char"10B58}%
\def\2{\parthian\char"10B59}%
\def\3{\text{\parthian\char"10B5A}}%
\def\4{\text{\parthian\char"10B5B}}% 
\TextOrMath\4 \4 %
\TextOrMath\3 \3 %
}{}
\index{Parthian (script)>Parthian numbers}
\begin{scriptexample}[]{}
\unicodetable{parthian}{"10B40,"10B50}
\end{scriptexample}

Inscriptional Parthian has its own numbers, which have right-to-left
directionality. The numbers are built up out of 1, 2, 3, 4, 10, 20, 100, and 1000 which indicates a rather rudimentary numbering system. The inscriptions are not
normalized uniformly. The units are sometimes written with strokes of the same height, or with a final
stroke that is longer, either descending or ascending to show the end of the number; compare 5 in 15 ({\parthian \char"10B59 \char"10B5B}
or 2 + 3) and in 45 (òõ or 1 + 4); compare 6 in 16 (öö or 3 + 3) and in 36 (òôö or 1 + 2 + 3). The
encoding here allows the specialist to choose his or her preferred representation. 

The |phd| package offers only rudimentary support for Parthian numbers in the form of an environment |parthiannumbers|, which can be used as follows:

\begin{texexample}{Inscriptional Parthian numbers}{parth}
\begin{parthiannumbers}
\1 $= 1$
\2 $= 2$
\begin{align*}
\3 &= 3\\
\4 &= 4\\
\3\4 &=7
\end{align*}
\end{parthiannumbers}
\end{texexample}

\subsection{The Inscriptional Parthian Unicode Block in Detail}

\printunicodeblock{./languages/inscriptional-parthian.txt}{\parthian}



\footnote{\url{http://www.unicode.org/L2/L2007/07207-n3286-parthian-pahlavi.pdf}} 