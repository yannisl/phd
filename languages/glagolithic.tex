\newfontfamily\glagolitic{MPH 2B Damase}

\section{Glagolitic}

\epigraph{The average Ph.D. thesis is nothing but a transference of bones from one graveyard to another.}{%
J. Frank Dobie (1888-1964)}


\label{s:glagolitic}
\fboxrule0pt\fboxsep0pt

\noindent
The Glagolitic alphabet /{\glagolitic ˌɡlæɡɵˈlɪtɨk/}, also known as Glagolitsa, is the oldest known Slavic alphabet, from the 9th century.

It was created in the 9th century by Saint Cyril, a Byzantine monk from Thessaloniki. He and his brother, Saint Methodius, were sent by the Byzantine Emperor Michael III in 863 to Great Moravia to spread Christianity among the West Slavs in the area. The brothers decided to translate liturgical books into the Old Slavic language that was understandable to the general population, but as the words of that language could not be easily written by using either the Greek or Latin alphabets, Cyril decided to invent a new script, Glagolitic, which he based on the local dialect of the Slavic tribes from the Byzantine Salonika region.
After the deaths of Cyril and Methodius, the Glagolitic alphabet ceased to be used in Moravia, but their students continued to propagate it in the west and south. 

After a long career, Glagolitic writing stopped being used, except for
religious purposes in certain dioceses of Bosnia and Dalmatia (Croatia).
The Cyrillic alphabet was adopted by all Orthodox Slays and served to note
their literary language. Most of the Slays who rallied to Rome rejected it,
however, which created the paradoxical situation in ex-Yugoslavia, where
two peoples who speak the same language write in different scripts, the
Serbs in Cyrillic and the Croats with Roman characters. Finally, as is
known, the ex-Soviet Union did much to put into writing the languages
spoken by the peoples within its borders, for the most part noting them in
adaptations of the Cyrillic alphabet, while Russian became the language of
culture throughout the Soviet Union.\cite{henri1994}

Slavic printing in Glagolitic characters originated in Venice, where a
\textit{Sluzebnik} (or \textit{Leitourgikon}) was published in 1483, followed by missals and
breviaries, all printed by Andrea Torresani, the future father-in-law and
associate of Aldus Manutius. After 1494 some attempts were made to create
printshops in Croatia itself, first in Senj in 1508, then, after 1530, in
Rijeka (Fiume). The work of these firms was almost totally liturgical (religious,
at any rate), and it had strong competition from manuscript works
that were better adapted to the diversity of local liturgical customs. Religion
also dictated the output of a printshop founded to provide Protestant propaganda
that was set up in Tubingen between 1560 and 1564 by Baron
Hans von Ungnad and that printed the great Lutheran texts in Glagolitic
characters.\footfullcite{henri1994}

Figure~\ref{fig:zograf} illustrates an example of the language.\footnote{\url{https://en.wikipedia.org/wiki/Glagolitic_script\#/media/File:ZographensisColour.jpg}}

\begin{figure}[htbp]
\centering

\includegraphics[width=0.45\linewidth]{glagolitic}
\caption[The first page of the Gospel of Mark from the 10th–11th century Codex Zographensis, found in the Zograf Monastery in 1843.]{The first page of the Gospel of Mark from the 10th–11th century Codex Zographensis, found in the Zograf Monastery in 1843.}
\label{fig:zograf}
\end{figure}

\section{Unicode Support}
The Glagolitic alphabet was added to the Unicode Standard in March 2005 with the release of version 4.1.
The Unicode block for Glagolitic is U+2C00–U+2C5F.



\begin{scriptexample}[]{glacolitic}

\unicodetable{glagolitic}{%
"2C00,"2C10,"2C20,"2C30,"2C40,"2C50}

\texttt{typeset with Damase version 2.0 MPH 2B Damase}
\end{scriptexample}
\bgroup
\glagolitic

The name was not coined until many centuries after its creation, and comes from the Old Church Slavonic glagolъ "utterance" (also the origin of the Slavic name for the letter G). The verb glagoliti means "to speak". It has been conjectured that the name glagolitsa developed in Croatia around the 14th century and was derived from the word glagolity, applied to adherents of the liturgy in Slavonic.[1]

In Old Church Slavonic the name is {\glagolitic ⰍⰫⰓⰊⰎⰎⰑⰂⰋⰜⰀ}, Кѷрїлловица.
The name Glagolitic in Bulgarian, Russian, Macedonian глаголица (glagolica), Belarusian is глаголіца (hłaholica), Croatian glagoljica, Serbian глагољица / glagoljica, Czech hlaholice, Polish głagolica, Slovene glagolica, Slovak hlaholika, and Ukrainian глаголиця (hlaholyća).



\egroup