\section{Ogham}
\label{s:ogham}
\newfontfamily\ogham{code2000.ttf}

Ogham /ˈɒɡəm/[1] (Modern Irish [ˈoːm] or [ˈoːəm]; Old Irish: ogam [ˈɔɣam]) is an Early Medieval alphabet used primarily to write the early Irish language (in the so-called "orthodox" inscriptions, 4th to 6th centuries), and later the Old Irish language (so-called scholastic ogham, 6th to 9th centuries). There are roughly 400 surviving orthodox inscriptions on stone monuments throughout Ireland and western Britain; the bulk of them are in the south of Ireland, in Counties Kerry, Cork and Waterford.\footnote{McManus (1991) is aware of a total of 382 orthodox inscriptions. The later scholastic inscriptions have no definite endpoint and continue into the Middle Irish and even Modern Irish period, and record also names in other languages, such as Old Norse, (Old) Welsh, Latin and possibly Pictish. See Forsyth, K.; "Abstract: The Three Writing Systems of the Picts." in Black et al. Celtic Connections: Proceedings of the Tenth International Congress of Celtic Studies, Vol. 1. East Linton: Tuckwell Press (1999), p. 508; Richard A V Cox, The Language of the Ogam Inscriptions of Scotland, Dept. of Celtic, Aberdeen University ISBN 0-9523911-3-9 [1]; See also The New Companion to the Literature of Wales, by Meic Stephens, page 540.} A rare example of a Christianised Ogham stone can be seen in St. Mary's Collegiate Church Gowran Co. Kilkenny. The largest number outside of Ireland is in Pembrokeshire in Wales.\footnote{O'Kelly, Michael J., '\textit{Early Ireland, an Introduction to Irish Prehistory}', p. 251, Cambridge University Press, 1989} The vast majority of the inscriptions consist of personal names.

Ogham is sometimes called the "Celtic Tree Alphabet", based on a high medieval Bríatharogam tradition ascribing names of trees to the individual letters. The etymology of the word ogam or ogham remains unclear. One possible origin is from the Irish og-úaim 'point-seam', referring to the seam made by the point of a sharp weapon.[4]

Ogham was added to the Unicode Standard in September 1999 with the release of version 3.0.

The spelling of the names given is a standardization dating to 1997, used in Unicode Standard and in Irish Standard 434:1999.
The Unicode block for ogham is \texttt{U+1680–U+169F}.

\begin{scriptexample}[]{Ogham}
\unicodetable{ogham}{"1680,"1690}

With the Titus font

\unicodetable{titus}{"1680,"1690}
\end{scriptexample}


\printunicodeblock{./languages/ogham.txt}{\ogham}
