\newfontfamily\buginese{Noto Sans Buginese}

\section{Buginese}
\label{s:buginese}\index{scripts>Buginese}

\epigraph{The strangest custom I have observed is that some men dress like women, and some women like men, not occassionally, 
but all their lives, devoting themeselves to the occupations and pursuits of their adopted sex. In the case of the males, it seems that the parents of a boy, upon perceiving in him certain effeminancies of habit and appearance, are induced thereby to present him to one of the rajahs, by whom he is received. These youths often acquire much influence over tehir masters\ldots}{James Brooke, in the journal of his vist to Wajo in 1840. }

\begin{figure}[htbp]
\includegraphics[width=\textwidth]{buginese}
\caption{Bugis dressed in traditional clothing (wikimedia).}
\end{figure}

Bugis people is one of Indonesian ethnic which are the inhabitant of South Sulawesi. This tribe is the biggest three after Javanese and Sundanese. Beside the indigenous who live in South Sulawesi, the immigrant of Minangkabau who wander from Sumatra to Sulawesi and the Malay people are being called as Bugis people.

At present time, The Bugis is spread to all over Indonesia like Sulawesi Tenggara, Sulawesi Tengah, Papua, Kalimantan Timur, Kalimantan Selatan even to going abroad. Bugis people are the the most group who insist to spread the Islamic religion.

\begin{figure}[htbp]
\includegraphics[width=\textwidth]{bugi-ship}
\end{figure}

In the past, Bugis people was famous by settled foreigner. The expertise of Bugis-Makassar sailing the ocean was trustworthy, they went up to the overseas territory of Malaysia, the Philippines, Brunei, Thailand, Australia, Madagascar and South Africa. In fact, they also went to Cape Town. Moreover, in South Africa there is a suburb named Maccassar, as a sign of local residents to remember their ancestors homeland.

The reason of Bugis people went to wander because there was conflict among the royal kingdom between Bugis kingdom and Makassar kingdom on 16, 17, 18 and 19 century. The people from the coastal area was mostly wander, instead of they hope for life independence.

Nowadays, Bugis people are mostly doing farming, fishing and trading instead of sailing the ocean. The women are mostly help the men doing farming and some of the elder are still weaving the silk sarong, which is the traditional cloth of Bugis-Makassar.

Many of the marriages are still arranged by parents and ideally take place between cousins. A newlywed couple often lives with the wife’s family for the first few years of their marriage.

In early 1600s after the Islam missionary spread the Islam religion to Makassar, the Bugis held the animist religion, then, they were converted to Islam 1611. A few west coast rulers converted to Christianity in the mid-sixteenth century, but it was failure by the Portuguese at Malacca to provide priests to preach the gospel, it meant that this did not last.

On daily speaking, The Bugis are use their own dialect called “Bahasa Ugi” or Ugi language and they have their own literacy called “Aksara Bugis”. This literacy was founded in 12 century during the entrance of Hindu in Indonesia archipelago.

Bugis people are group who have strong philosophy from their own, there are four principals that they are concern to build their life philosophy, they are:

- The principal of persistent life

- Solidarity and loyalty

- Siri or Pride

- Etiquette and Manners



Those principals are the essential things that build The Bugis to be a social human being. To be a unique group in diversities.

www.indonesia-tourism.com

\paragraph{The Buginese script} is used on the island of Sulawesi, mainly in the southwest. It is of the Brahmic type
and perhaps related to Javanese. It bears some affinity with Tagalog as well, and it apparently does not
traditionally record final consonants (see note on the VIRAMA below). Buginese may be the easternmost
representative of the Brahmi scripts. Sirk (1983) reports that the Buginese language (an Austronesian
language) has a rich traditional literature making it one of the foremost languages of Indonesia. There
may be as many as 2.3 million speakers of Buginese in the southern part of Sulawesi (as of 1971). The
script was reported in some use as of 1983, and a variety of traditional literature has been printed in it.
Buginese literature was studied extensively by B. F. Matthes (a Dutch missionary) in the 19th century.
Matthes published a Buginese-Dutch dictionary in 1874 with a supplement in 1889, as well as a
grammar. The script was previously also used to write the Makassar, Bimanese, and Madurese languages.
For Makassar, Matthes 1858 also gives an older alphabet, which uses different shapes for the letters, and
lacks the HA, but the difference seems to be a change in font style only.


\paragraph{Encoding} Buginese was added to the Unicode Standard in March, 2005 with the release of version 4.1.
The Unicode block for Buginese is U+1A00–U+1A1F:


\unicodetable{buginese}{"1A00,"1A10}


\bgroup
\buginese

ᨀ	ᨀᨗ	ᨀᨘ	ᨀᨙ	ᨀᨛ	ᨀᨚ

\egroup


\paragraph{Punctuation}

There are two punctuation marks the \textit{pallawa} and an end of section marker.

\begin{center}
\begin{tabular}{ll}
\toprule
Pallawa & end section\\
\scalebox{3.5}{\buginese ᨞} &\scalebox{3.5}{\buginese ᨟}\\
\bottomrule
\end{tabular}
\end{center}

Pallawa is used to separate rhythmico-intonational groups, thus functionally corresponds to the period and comma of the Latin script. The pallawa can also be used to denote the doubling of a word or its root.


