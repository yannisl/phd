\section{Gujarati}
\label{s:gujarati}
\idxlanguage{Gujarati}
\index{Unicode>Gujarati}
%FIXME
\index{languages>Gujarati}\index{languages>Gujǎrātī Lipi}
has its own writing system, distinct but related to several other Indian languages' writing systems, such as the one used to write Hindi. Strictly speaking, the Gujarati writing system is what is called an \emph{abugida} (and not an \textit{alphabet}), because the consonant characters all contain an inherent vowel, and other vowels are written as accents added on to the consonant characters. There are also symbols for stand-alone vowels.

The Gujarati script ({\gujarati{ગુજરાતી લિપિ }} Gujǎrātī Lipi), which like all Nāgarī writing systems is strictly speaking an abugida rather than an alphabet, is used to write the Gujarati and Kutchi languages. It is a variant of Devanāgarī script differentiated by the loss of the characteristic horizontal line running above the letters and by a small number of modifications in the remaining characters.
With a few additional characters, added for this purpose, the Gujarati script is also often used to write Sanskrit and Hindi.
Gujarati numerical digits are also different from their Devanagari counterparts.
\medskip

\bgroup
\newfontfamily\gujaratilohit[Script=Gujarati,Scale=1.5]{Lohit-Gujarati.ttf}
\gujarati

\centering

\underline{English/Hindi/Gujarati Alphabets}

\hskip-1.5cm\begin{tabular}{lllllllllllllllllllll}
A &B &bh &C &ch &chh &D &dh &E &F &G &gh &H &I &J &K &kh &L &M &N &O\\

अ &ब &भ &क &च &छ &ड/द &ध/ढ़ &इ &फ &ग &घ &ह &ई &ज &क &ख &ल &म &न/ण &ऑ\\

અ &બ &ભ &ક &ચ &છ &ડ/દ &ધ /ઢ &ઇ &ફ &ગ &ઘ &હ &ઈ &જ &ક &ખ &લ &મ &ન/ણ &ઓ\\

\end{tabular}
\egroup

\medskip

Gujarati has its own set of numeric signs (placed alongside their Hindu-Arabic [or Indo-Arabic] counterparts in the tables below), they are employed in much the same way as English;  that is to say, they are put together in the same manner in order to express larger numbers. It is quite possible to simply substitute the Gujarati numerals for the Hindu-Arabic ones.

The Gujarati words for 1-10 are as follows:
\medskip

\bgroup
\begin{center}
\gujarati
\begin{tabular}{ccl}
Arabic & Gujarati &Name\\
Numeral &Numeral  &\\
0	&૦	&mīṇḍuṃ or shunya\\
1	&૧	&ekaṛo or ek\\
2	&૨	&bagaṛo or bay\\
3	&૩	&tragaṛo or tran\\
4	&૪	&chogaṛo or chaar\\
5	&૫	&pāchaṛo or paanch\\
6	&૬	&chagaṛo or chah\\
7	&૭	&sātaṛo or sāt\\
8	&૮	&āṭhaṛo or āanth\\
9	&૯	&navaṛo or nav\\
10 &૧૦ &દસ das\\

\end{tabular}
\end{center}
\egroup
