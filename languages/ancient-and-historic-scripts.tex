\cxset{steward,
  numbering=arabic,
  custom=stewart,
  offsety=0cm,
  image={fellah-woman.jpg},
  texti={An introduction to the use of font related commands. The chapter also gives a historical background to font selection using \tex and \latex. },
  textii={In this chapter we discuss keys that are available through the \texttt{phd} package and give a background as to how fonts are used
in \latex.
 },
 pagestyle = fancy
}

\chapter{Ancient and Historic Scripts}

Unicode encodes a number of ancient scripts, which have not been in normal use for a millennium or more, as well as historic scripts, whose usage ended in recent centuries. Although these scripts are no longer used to write living languages, documents and inscriptions using these languages exist, both for extinct languages and for precursors of modern languages. The primary user communities for these scripts are scholars, interested in studying the scripts and the languages written in them. A few, such as Coptic, also have contemporary liturgical or other special purposes. Some of the historic scripts are related to each other as well as to modern alphabets. The following are provides as of Unicode version~7.2.
\index{Ancient and Historic Scripts>Ogham}
\index{Ancient and Historic Scripts>Old Italic}
\index{Ancient and Historic Scripts>Runic}
\index{Ancient and Historic Scripts>Gothic}
\index{Ancient and Historic Scripts>Akkadian}
\index{Ancient and Historic Scripts>Old Turkic}
\index{Ancient and Historic Scripts>Hieroglyphs}
\index{Ancient and Historic Scripts>Linear B}
\index{Ancient and Historic Scripts>Linear A}
\index{Ancient and Historic Scripts>Phoenician}
\index{Ancient and Historic Scripts>Old South Arabian}
\index{Ancient and Historic Scripts>Mandaic}
\index{Ancient and Historic Scripts>Avestan}
\index{Ancient Anatolian Alphabets}
\index{Old South Arabian}
\index{Phoenician}
\index{Imperial Aramaic}
\begin{center}
\begin{tabular}{lll}
Ogham (see \S\ref{s:ogham})           
&Ancient Anatolian Alphabets (see \S\ref{s:anatolian})
&Avestan (see \S\ref{s:avestan})\\
Old Italic (see \S\ref{s:olditalic})       
&Old South Arabian (see \S\ref{s:oldsoutharabian})          
&Ugaritic (see \S\ref{s:ugaritic})\\
Runic (see \S\ref{s:runic})            
&Phoenician (see \S\ref{s:phoenician})                  
&Old Persian (see \S\ref{s:oldpersian})\\
Gothic            
&Imperial Aramaic (see \S\ref{s:imperialaramaic})            
&Sumero-Akkadian. (see \S\ref{s:sumero})\\
Old Turkic (see \S\ref{s:oldturkic})     
&Mandaic (see \S\ref{s:mandaic}) 
&Egyptian Hieroglyphs.\\
Linear B (see \S\ref{s:linearb})          
&Inscriptional Parthian (see \S\ref{s:parthian})      
&Meroitic (see \S\ref{s:meroitic})\\
Cypriot (see \S\ref{s:cypriot})
&Inscriptional Pahlavi  (see \S\ref{s:inscriptionalpahlavi})       
&Linear A (see \S\ref{s:linearb})\\
\end{tabular}
\end{center}

The following scripts are also encoded but following the Unicode
convention are described in other sections

\begin{center}
\begin{tabular}{llllll}
Coptic &Glagolithic &Phags-pa. &Kaithi &Kharoshi &Brahmi.\\
\end{tabular}
\end{center}

\section{Old Persian}
\label{s:oldpersian}


Old Persian, like Hittite an Indo-European language, was written in cuneiforms as of the first millenium BC, mostly between 550 and 350. King Darius’ monumental inscription at
Bisothum – in Old Persian, Elamite and Neo-Babylonian – furnished
the ‘key’ to cuneiform’s decipherment and the reconstruction
of these languages.28 Darius’ Old Persian scribes
effected the most drastic simplification of the borrowed Near
Eastern script (illus. 35). They reduced the cuneiform inventory
to only 41 signs of both syllabic (ka) and phonemic (/k/) values.
Thus, Old Persian cuneiform is ‘half syllabic, half letter writing’.
29 It appears to be on the fence between the Babylonians’
cuneiforms and the Levantines’ consonantal writing, a hybrid
solution using only four logograms and 36 syllabo-phonemic
signs written in wedges. Of particular significance is the fact
that Old Persian also conveys the individual long and short
vowels /a/ (pronounced AH), /i/ (EE) and /u/ (OO) that the
Ugaritic system had conveyed a thousand years earlier.

Old Persian cuneiform is a semi-alphabetic cuneiform script that was the primary script for the Old Persian language. Texts written in this cuneiform were found in Persepolis, Susa, Hamadan, Armenia, and along the Suez Canal.[1] They were mostly inscriptions from the time period of Darius the Great and his son Xerxes. Later kings down to Artaxerxes III used corrupted forms of the language classified as “pre-Middle Persian”.

\begin{scriptexample}[]{Old Persian}
\unicodetable{oldpersian}{"103A0,"103B0,"103C0,"103D0}
\end{scriptexample}

Scholars today mostly agree that the Old Persian script was invented by about 525 BC to provide monument inscriptions for the Achaemenid king Darius I, to be used at Behistun. While a few Old Persian texts seem to be inscribed during the reigns of Cyrus the Great (CMa, CMb, and CMc, all found at Pasargadae), the first Achaemenid emperor, or Arsames and Ariaramnes (AsH and AmH, both found at Hamadan), grandfather and great-grandfather of Darius I, all five, specially the later two, are generally agreed to have been later inscriptions.
Around the time period in which Old Persian was used, nearby languages included Elamite and Akkadian. One of the main differences between the writing systems of these languages is that Old Persian is a semi-alphabet while Elamite and Akkadian were syllabic. In addition, while Old Persian is written in a consistent semi-alphabetic system, Elamite and Akkadian used borrowings from other languages, creating mixed systems.
\medskip

{\leftskip-1.25cm
\includegraphics[width=\textwidth+2.5cm]{./images/naghshe.jpg}
\captionof{figure}{Panoramic view of the Naqsh-e Rustam. This site contains the tombs of four Achaemenid kings, including those of Darius I and Xerxes. (\textit{Wikimedia})}
}
\section{Inscriptional Pahlavi}
\label{s:inscriptionalpahlavi}
\newfontfamily\inscriptionalpahlavi{Noto Sans Inscriptional Pahlavi}

Pahlavi or Pahlevi denotes a particular and exclusively written form of various Middle Iranian languages. The essential characteristics of Pahlavi are[1]
the use of a specific Aramaic-derived script, the Pahlavi script;
the high incidence of Aramaic words used as heterograms (called hozwārishn, "archaisms").

Pahlavi compositions have been found for the dialects/ethnolects of Parthia, Parsa, Sogdiana, Scythia, and Khotan.[2] Independent of the variant for which the Pahlavi system was used, the written form of that language only qualifies as Pahlavi when it has the characteristics noted above.


Pahlavi is then an admixture of
written Imperial Aramaic, from which Pahlavi derives its script, logograms, and some of its vocabulary.

spoken Middle Iranian, from which Pahlavi derives its terminations, symbol rules, and most of its vocabulary.
Pahlavi may thus be defined as a system of writing applied to (but not unique for) a specific language group, but with critical features alien to that language group. It has the characteristics of a distinct language, but is not one. It is an exclusively written system, but much Pahlavi literature remains essentially an oral literature committed to writing and so retains many of the characteristics of oral composition.

\begin{scriptexample}[]{Pahlavi}
\unicodetable{inscriptionalpahlavi}{"10B60,"10B70}
\end{scriptexample}




\section{Imperial Aramaic}
\label{s:imperialaramaic}

\subsection{History}

Aramaic is the best-attested and longest-attested
member of the NW Semitic subfamily of languages
(which also includes inter alia \nameref{s:hebrew}, \nameref{s:phoenician},
\nameref{s:ugaritic}, Moabite, Ammonite, and Edomite). The
relatively small proportion of the biblical text
preserved in an Aramaic original (Dan 2:4–7:28; Ezra
4:8–68 and 7:12–26; Jeremiah 10:11; Gen 31:47 [two
words] as well as isolated words and phrases in
Christian Scriptures) belies the importance of this
language for biblical studies and for religious studies
in general, for Aramaic was the primary international
language of literature and communication throughout
the Near East from ca. 600 B.C.E. to ca. 700 C.E. and
was the major spoken language of Palestine, Syria,
and Mesopotamia in the formative periods of
Christianity and rabbinic Judaism. 



Aramaic survived over a period of 3,000 years, during which time its grammar, vocabulary and usage experienced great changes. Aramaic scholars found it useful to divide the several Aramaic dialects into periods, groups and subgroups based both on the chronology as well as the geography.

\begin{enumerate}
\item Old Aramaic
\item Imperial Aramaic
\item  Middle Aramaic
\item Late Aramaic
\item Modern Aramaic
\end{enumerate}


\subsection{Old Aramaic (to ca. 612 BCE)}
This period
witnessed the rise of the Arameans as a major force
in ANE history, the adoption of their language as an
international language of diplomacy in the latter days
of the Neo-Assyrian Empire, and the dispersal of
Aramaic-speaking peoples from Egypt to Lower
Mesopotamia as a result of the Assyrian policies of
deportation. The scattered and generally brief
remains of inscriptions on imperishable materials
preserved from these times are enough to
demonstrate that an international standard dialect had
not yet been developed. The extant texts may be
grouped into several dialects:

\subsection{Middle Aramaic (to ca. 250 C.E.)}
In the Hellenistic and Roman periods, Greek replaced
Aramaic as the administrative language of the Near
East, while in the various Aramaic-speaking regions
the dialects began to develop independently of one
another. Written Aramaic, however, as is the case
with most written languages, by providing a
somewhat artificial, cross-dialectal uniformity,
continued to serve as a vehicle of communication
within and among the various groups. For this
purpose, the literary standard developed in the
previous period, Standard Literary Aramaic, was
used, but lexical and grammatical differences based
on the language(s) and dialect(s) of the local
population are always evident. It is helpful to divide
the texts surviving from this period into two major
categories: epigraphic and canonical.

\subsection{Late Aramaic (to ca. 1200 C.E.)}
The bulk of
our evidence for Aramaic comes from the vast
literature and occasional inscriptions of this period.
During the early centuries of this period Aramaic
dialects were still widely spoken. During the second
half of this period, however, Arabic had already
displaced Aramaic as the spoken language of much
of the population. Consequently, many of our texts
were composed and/or transmitted by persons whose
Aramaic dialect was only a learned language.
Although the dialects of this period were previously
divided into two branches (Eastern and Western), it
now seems best to think rather of three: Palestinian,
Syrian, and Babylonian.

The Aramaic alphabet is adapted from the \nameref{s:phoenician} alphabet and became distinctive from it by the 8th century BCE.  The letters all represent consonants, some of which are \emph{matres lectionis}, which also indicate long vowels.

\subsection{Modern Aramaic (to the present day)}

These dialects can be divided into the same three
geographic groups.

\begin{description}

\item[a. Western]
Here Aramaic is still spoken only in
the town of Ma’lula (ca. 30 miles NNE of Damascus)
and surrounding villages. The vocabulary is heavily
Arabized.

\item[b. Syrian]
Western Syrian (Turoyo) is the language
of Jacobite Christians in the region of Tur-Abdin in
SE Turkey. This dialect is the descendant of
something very like classical Syriac. Eastern Syrian
is spoken in the Kurdistani regions of Iraq, Iran,
Turkey, and Azerbaijan by Christians and, formerly,
by Jews. Substantial communities of the former are
now found in North America. The Jewish speakers
have mostly settled in Israel. These dialects are
widely spoken by their respective communities and
have been studied extensively during the past
century. It has become clear that they are not the
descendants of any known literary Aramaic dialect.

\item[c. Babylonian] 

\nameref{s:mandaic} is still used, at least until
recently, by some Mandaeans in southernmost Iraq
and adjacent areas in Iran.

In addition, in recent years classical \nameref{s:syriac} has
undergone somewhat of a revival as a learned vehicle
of communication for Syriac Christians, both in the
Middle East and among immigrant communities in
Europe and North America.
\end{description}

\begin{figure}[htbp]
\centering
\includegraphics[width=0.6\textwidth]{./images/elephantine-papyrus.jpg}

\caption{The Elephantine papyri are ancient Jewish papyri dating to the 5th century BC, requesting the rebuilding of a Jewish temple. It also name three persons mentioned in Nehemiah: Darius II, Sanballat the Horonite and Johanan the high priest.}

\end{figure}


\subsection{Alphabet and typesetting}

The Aramaic alphabet is historically significant, since virtually all modern Middle Eastern writing systems can be traced back to it, as well as numerous non-Chinese writing systems of Central and East Asia. This is primarily due to the widespread usage of the Aramaic language as both a \emph{lingua franca} and the official language of the Neo-Assyrian Empire, and its successor, the Achaemenid Empire. Among the scripts in modern use, the Hebrew alphabet bears the closest relation to the Imperial Aramaic script of the 5th century BC, with an identical letter inventory and, for the most part, nearly identical letter shapes.

Writing systems that indicate consonants but do not indicate most vowels (like the Aramaic one) or indicate them with added diacritical signs, have been called abjads by Peter T. Daniels to distinguish them from later alphabets, such as Greek, that represent vowels more systematically. This is to avoid the notion that a writing system that represents sounds must be either a syllabary or an alphabet, which implies that a system like Aramaic must be either a syllabary (as argued by Gelb) or an incomplete or deficient alphabet (as most other writers have said); rather, it is a different type.

The Imperial Aramaic alphabet was added to the Unicode Standard in October 2009 with the release of version 5.2.
The Unicode block for Imperial Aramaic is \unicodenumber{U+10840–U+1085F}.

\begin{scriptexample}[]{Aramaic}
\unicodetable{imperialaramaic}{"10840,"10850}
\end{scriptexample}




\PrintUnicodeBlock{./languages/imperial-aramaic.txt}{\imperialaramaic}
\subsection{Ogham}

\newfontfamily\ogham{code2000.ttf}

Ogham was added to the Unicode Standard in September 1999 with the release of version 3.0.
The spelling of the names given is a standardisation dating to 1997, used in Unicode Standard and in Irish Standard 434:1999.
The Unicode block for ogham is \texttt{U+1680–U+169F}.

\begin{scriptexample}[]{Ogham}
\bgroup
\ogham
0	1	2	3	4	5	6	7	8	9	A	B	C	D	E	F\\
U+168x	   	ᚁ	ᚂ	ᚃ	ᚄ	ᚅ	ᚆ	ᚇ	ᚈ	ᚉ	ᚊ	ᚋ	ᚌ	ᚍ	ᚎ	ᚏ\\
U+169x	ᚐ	ᚑ	ᚒ	ᚓ	ᚔ	ᚕ	ᚖ	ᚗ	ᚘ	ᚙ	ᚚ	᚛	᚜	\\

\titus

0	1	2	3	4	5	6	7	8	9	A	B	C	D	E	F\\
U+168x	   	ᚁ	ᚂ	ᚃ	ᚄ	ᚅ	ᚆ	ᚇ	ᚈ	ᚉ	ᚊ	ᚋ	ᚌ	ᚍ	ᚎ	ᚏ\\
U+169x	ᚐ	ᚑ	ᚒ	ᚓ	ᚔ	ᚕ	ᚖ	ᚗ	ᚘ	ᚙ	ᚚ	᚛	᚜
\egroup		
\end{scriptexample}
\section{Ancient Anatolian Alphabets}

The Anatolian scripts described in this section all date from the first millenium BCE, and were used to write various ancient Indo-European languages of western and southwestern Anatolia (now Turkey). All are related to the Greek script and are probably adaptations of it. 

\newfontfamily\lycian{Aegean.ttf}
\let\lydian\lycian
\let\carian\lydian

\begin{description}
\item [Lycian] The Lycian alphabet was used to write the Lycian language. It was an extension of the Greek alphabet, with half a dozen additional letters for sounds not found in Greek. It was largely similar to the Lydian and the Phrygian alphabets.
 
\bgroup
\lydian
\obeylines
0	1	2	3	4	5	6	7	8	9	A	B	C	D	E	F
U+1028x	𐊀	𐊁	𐊂	𐊃	𐊄	𐊅	𐊆	𐊇	𐊈	𐊉	𐊊	𐊋	𐊌	𐊍	𐊎	𐊏
U+1029x	𐊐	𐊑	𐊒	𐊓	𐊔	𐊕	𐊖	𐊗	𐊘	𐊙	𐊚	𐊛	𐊜

Typeset with the \idxfont{Aegean.ttf} and the command \cmd{\lydian}
\egroup

\item[Lydian] Lydian script was used to write the Lydian language. That the language preceded the script is indicated by names in Lydian, which must have existed before they were written. Like other scripts of Anatolia in the Iron Age, the Lydian alphabet is a modification of the East Greek alphabet, but it has unique features. The same Greek letters may not represent the same sounds in both languages or in any other Anatolian language (in some cases it may). Moreover, the Lydian script is alphabetic.
Early Lydian texts are written both from left to right and from right to left. Later texts are exclusively written from right to left. One text is boustrophedon. Spaces separate words except that one text uses dots. Lydian uniquely features a quotation mark in the shape of a right triangle.
The first codification was made by Roberto Gusmani in 1964 in a combined lexicon (vocabulary), grammar, and text collection.


\bgroup
\lycian
\obeylines
	0	1	2	3	4	5	6	7	8	9	A	B	C	D	E	F
U+1092x	𐤠	𐤡	𐤢	𐤣	𐤤	𐤥	𐤦	𐤧	𐤨	𐤩	𐤪	𐤫	𐤬	𐤭	𐤮	𐤯
U+1093x	𐤰	𐤱	𐤲	𐤳	𐤴	𐤵	𐤶	𐤷	𐤸	𐤹						𐤿
Typeset with the \idxfont{Aegean.ttf} and the command \cmd{\lycian}

Examples of words

𐤬𐤭𐤠  - Ora - "Month"

𐤬𐤳𐤦𐤭𐤲𐤬𐤩  - Laqrisa - "Wall"

𐤬𐤭𐤦𐤡  - "House, Home"

\egroup

\item [Carian] The Carian alphabets are a number of regional scripts used to write the Carian language of western Anatolia. They consisted of some 30 alphabetic letters, with several geographic variants in Caria and a homogeneous variant attested from the Nile delta, where Carian mercenaries fought for the Egyptian pharaohs. They were written left-to-right in Caria (apart from the Carian–Lydian city of Tralleis) and right-to-left in Egypt. Carian was deciphered primarily through Egyptian–Carian bilingual tomb inscriptions, starting with John Ray in 1981; previously only a few sound values and the alphabetic nature of the script had been demonstrated. The readings of Ray and subsequent scholars were largely confirmed with a Carian–Greek bilingual inscription discovered in Kaunos in 1996, which for the first time verified personal names, but the identification of many letters remains provisional and debated, and a few are wholly unknown.

\begin{scriptexample}[]{Carian}
\bgroup
\carian
\obeylines
 	0	1	2	3	4	5	6	7	8	9	A	B	C	D	E	F
U+102Ax	𐊠	𐊡	𐊢	𐊣	𐊤	𐊥	𐊦	𐊧	𐊨	𐊩	𐊪	𐊫	𐊬	𐊭	𐊮	𐊯
U+102Bx	𐊰	𐊱	𐊲	𐊳	𐊴	𐊵	𐊶	𐊷	𐊸	𐊹	𐊺	𐊻	𐊼	𐊽	𐊾	𐊿
U+102Cx	𐋀	𐋁	𐋂	𐋃	𐋄	𐋅	𐋆	𐋇	𐋈	𐋉	𐋊	𐋋	𐋌	𐋍	𐋎	𐋏
U+102Dx	𐋐
\egroup
\end{scriptexample}

\newfontfamily\oldpunctuation{code2000.ttf}

Word dividers are infrequent, \emph{scriptio continua}\footnote{a style of writing without word dividers, that is, without spaces or other marks between words or sentences} is common. Words dividers which are attested are U+00B7 (\char"00B7) \textsc{MIDLE DOT} (or U+2E31 word separator middle dot), U+205A TWO DOT PUNCTUATION, and U+205D TRICOLON ({\oldpunctuation\char"205D}). In modern editions U+0020 SPACE may be found.

\end{description}

\section{Phoenician}
\label{s:phoenician}
\arial

The Phoenician alphabet and its successors were widely used over a broad area surrounding the Mediterranean Sea.

\let\phoenician\lycian

\begin{scriptexample}[]{Phoenician}

\unicodetable{phoenician}{"10900,"10910}

\end{scriptexample}

The Phoenician alphabet, called by convention the Proto-Canaanite alphabet for inscriptions older than around 1200 BCE, is the oldest verified consonantal alphabet, or abjad.[1] It was used for the writing of Phoenician, a Northern Semitic language, used by the civilization of Phoenicia. It is classified as an abjad because it records only consonantal sounds (matres lectionis were used for some vowels in certain late varieties).

Phoenician became one of the most widely used writing systems, spread by Phoenician merchants across the Mediterranean world, where it evolved and was assimilated by many other cultures. The Aramaic alphabet, a modified form of Phoenician, was the ancestor of modern Arabic script, while Hebrew script is a stylistic variant of the Aramaic script. The Greek alphabet (and by extension its descendants such as the Latin, the Cyrillic, and the Coptic) was a direct successor of Phoenician, though certain letter values were changed to represent vowels.

\begin{figure}[ht]
\includegraphics[width=\textwidth]{./images/phoenician.jpg}
\captionof{figure}{
Phoenician votive inscription from Idalion (Cyprus), 390 BC. BM 125315 from The Early Alphabet by John F. Healy.}
\end{figure}

As the letters were originally incised with a stylus, most of the shapes are angular and straight, although more cursive versions are increasingly attested in later times, culminating in the Neo-Punic alphabet of Roman-era North Africa. Phoenician was usually written from right to left, although there are some texts written in boustrophedon.


\printunicodeblock{./languages/phoenician.txt}{\phoenician}


\newpage
\section{Palmyrene}
\idxlanguage{Palmyrene}
\arial

Palmyrene is the very widely attested Aramaic dialect and script
of Palmyra in the Syrian desert. The texts date from the midfirst century to the destruction of Palmyra by the Romans in AD 272. Palmyra in the Roman period was a major trading centre.
\medskip

\begin{figure}[ht]
\centering

\includegraphics[width=0.9\textwidth]{./images/palmyrene.jpg}
\captionof{figure}{\protect\arial Limestone bust with Palmyrene inscription. Palmyra late 2nd century AD. BM WA 102612}

\end{figure}

\medskip
The longest of the Palmyrene texts, is the bilingual  taxation tariff written for the year 137 AD in Palmyrene Aramaic and Greek.\footnote{For more details see:MILIK J.T., Dédicaces faites par des dieux (Palmyre, Hatra, 
Tyr) et de thiases sémitiques à l'époque romaine, Paris 1972; ROSENTHAL R., Die 
Sprache der palmyrenischen Inschriften, Leipzig 1936; STARK J.K., Personal Names in 
Palmyrene Inscriptions, Oxford 1971; DRIJVERS H.J.W., The Religion of Palmyra, 
Leiden 1976; TEIXIDOR J., 'Palmyre et son commerce d'Auguste à Caracalla', in 
Semitica 34, (1984) 1-127.  } Trade connections 
took the Palmyrene script to other places, some not far away, such as Dura Europos on the Euphrates, butothers at a great distance. A particular inscription is from South Shields, Roman Arbeia, in the north-east of England, carved on behalf of a Palmyrene mechant for his deceased wife and probably dating to the early third century AD. 

The Palmyrene script existed in two main varieties, a monumental and a cursive one, though the latter is little known and the evidence  mostly from Palmyra itself. The Syriac script of Edessa in southern Turkey, is often regarded as derived or closely related to the Palmyrene---similarities are found in the letters: ', b, g, d, w, h, y, k, l, m, n, `, r and t---though a strong case can also be made for connecting Syriac with a northern Mesopotamian script-family represented principally in texts from Hatra, a city more or less contemporary with Palmyra in Upper Mesopotamia. 


\begin{figure}[ht]
\includegraphics[width=\textwidth]{./images/regina-epigraph.jpg}
\caption{It was customary for Palmyrenes to offer bilingual texts (Greek or Latin) on funerary monuments. The final line of Regina's epitaph is Barates' personal lament in Palmyrene: Regina, freedwoman of Barate, alas. (See \href{http://www2.cnr.edu/home/araia/regina.html}{regina}.)}
\end{figure}

A good article on the classification of Aramaic languages can be found in \textit{The Aramaic language and Its Classification} by Efrem Yildiz.\footnote{\url{http://www.jaas.org/edocs/v14n1/e8.pdf}}








\cxset{quotation font-size=\normalsize,
       quote font-size=\normalsize}


\section{Mandaic}
\label{s:mandaic}
\newfontfamily\mandaic{NotoSansMandaic-Regular.ttf}


The Mandaic script is used to write a dialect of Eastern Aramaic, which, in its classical
form, is currently used as the liturgical language of the Mandaean religion. A living language descended
from Classical Mandaic is spoken by a small number of people living in and around Ahvaz, Khūzestān,
in southwestern Iran; speakers are also found in emigrant communities in Sweden, Australia, and the
United States. There is a considerable amount of Iranian influence on the lexicon of Classical Mandaic,
and Arabic and Persian influence on the grammar and lexicon of the contemporary dialect. The script
itself is likely derived from the Parthian chancery script.

Mandaic is a right-to-left script. It is a true alphabet, using letters regularly for vowels
rather than as the \emph{matres lectionis} from which they derived. The three diacritical marks are used in
teaching materials to differentiate vowel quality. At present, at least, the rule is that they may be omitted
from ordinary text. In this regard they are very like the Arabic fatha, kasra, and damma or the Hebrew
vowel points.

The only so far I could find that can display the script is the Google \idxfont{NotoSansMandaic.ttf}.

\begin{scriptexample}[]{Mandaic}
\bgroup
\unicodetable{mandaic}{"0840,"0850}
\egroup
\end{scriptexample}

In 1943, Lady Ethel Drower published extracts from several magic “recipe books” that served the writers of amulets in Baghdad in the early 20th century, in particular from two manuscripts in her possession, DC 45 and DC 46.

\begin{figure}[hb]
\centering

\includegraphics[height=4cm]{./magic-letters.jpg}
\includegraphics[height=4cm]{./45-453.jpg}
\includegraphics[height=4cm]{./36-448.jpg}

\captionof{figure}{Mandaic Incantation vessels. The left image is from \protect\href{http://thesacredalphabet.blogspot.ae/}{thesacredalphabet}, whereas the last two are from \protect\href{http://www.archaeological-center.com/en/auctions/45-453}{archaeological-center} }
\end{figure}

 While Drower, following her native informants, entitled the work ‘A Mandæan Book of Black Magic’, the manuscripts themselves contain a wide range of formulae for amulets and talismans for various purposes, as Drower herself was well aware. Alongside spells for healing, protection and success, we find others for enflaming love or stirring up enmity.

 The manuscripts themselves appear to have been copied in the late 19th or early 20th centuries; in particular, DC 46, a substantial codex of 264 sides, is written on an extremely modern “clean” paper. DC 45 is written on a rougher paper and appears to be somewhat earlier. It is also more fragmentary, and contains several leaves that were copied by a different hand and inserted into the main part of the manuscript at a later date, though it is clear from their contents that they were intended to replace pages that had been worn or damaged, as they begin and end exactly as required by the preceding and following pages. As it survives today, DC 45 is also considerably shorter than DC 46; however, it also contains several spells that are not found in DC 46.\footnote{\protect\href{http://www.academia.edu/8294938/Arabic_Magic_Texts_in_Mandaic_Script_A_Forgotten_Chapter_in_Near-Eastern_Magic}{Magic Texts}}

Lady Drower inform us that among the Mandaens:

\begin{quote}
Writing in itself is a magic art, and the alphabet is sacred.
Each letter is supposed to invoke a spirit of light and is a thing of power. It is a practice to write the letters separately and to sleep each night with a letter beneath the pillow. If the sleeper sees in a dream something which will enlighten him, the letters upon which he slept that night is taken to a silversmith and a replica in gold or silver is made and worn around the neck as amulet See Mandaic Incantation Texts by Edwin M Yamauchi.
\end{quote}












\newcounter{glyphcount}
^^A\newfontfamily\aegyptus{AegyptusR.ttf}

\chapter{Aegyptian Hieroglyphics}

\index{fonts>Aegyptus}\index{Aegyptus (font)}
\index{fonts>Hieroglyphics}\index{languages>hieroglyphics}

\newfontfamily\hiero{NotoSansEgyptianHieroglyphs-Regular.ttf}

Hieroglyphic writing appeared in Egypt at the end of the fourth millennium bce. The writing
system is pictographic: the glyphs represent tangible objects, most of which modern
scholars have been able to identify. A great many of the pictographs are easily recognizable
even by nonspecialists. Egyptian hieroglyphs represent people and animals, parts of the
bodies of people and animals, clothing, tools, vessels, and so on.

Hieroglyphs were used to write Egyptian for more than 3,000 years, retaining characteristic
features such as use of color and detail in the more elaborated expositions. Throughout the
Old Kingdom, the Middle Kingdom, and the New Kingdom, between 700 and 1,000 hieroglyphs
were in regular use. During the Greco-Roman period, the number of variants, as
distinguished by some modern scholars, grew to somewhere between 6,000 and 8,000.

Hieroglyphs were carved in stone, painted on frescoes, and could also be written with a reed
stylus, though this cursive writing eventually became standardized in what is called \emph{hieratic}
writing. Unicode does not encode the hieratic forms separately, but ust considers them as cursive forms of the hieroglyphs encoded block.

The Demotic script and then later the Coptic script replaced the earlier hieroglyphic and
hieratic forms for much practical writing of Egyptian, but hieroglyphs and hieratic continued
in use until the fourth century ce. An inscription dated August 24, 394 ce has been
found on the Gateway of Hadrian in the temple complex at Philae; this is thought to be
among the latest examples of Ancient Egyptian writing in hieroglyphs

\begin{figure}[htb]
\includegraphics[width=\textwidth]{./images/bookofthedead.jpg}
\end{figure}

In hieroglyphic texts, these drawings are not only simply arranged in sequential order, but also grouped on top of and next to each other. This rather complicates matters trying to register and reproduce hieroglyphic texts using a computer.

\section{Computer Typesetting}

Typesetting hieroglyphics with computers presents a number of problems. First is the method of inputting the characters and second the various methods required to stack hieroglyphics, the direction of writing which can be one of four different directions.

When the first computers were introduced in Egyptology in the late 1970s and the beginning of the 1980s, the graphical capacity of the machines was still in its infancy. Early attempts to register the hieroglyphic pictorial writing on computer therefore chose an encoding system to do this, using alphanumeric codes to represent or replace the graphics. To prevent many people from reinventing the wheel, during the first "Table Ronde Informatique et Egyptologie" in 1984 a committee was charged with the task to develop a uniform system for the encoding of hieroglyphic texts on computer. The resulting Manual for the Encoding of Hieroglyphic Texts for Computer-input (Jan Buurman, Nicolas Grimal, Jochen Hallof, Michael Hainsworth and Dirk van der Plas, Informatique et Egyptologie 2, Paris 1988), simply called Manuel de Codage, presents an easy to use and intuitive way of encoding hieroglyphic writing as well as the abbreviated hieroglyphic transcription (transliteration). The system proposed by the Manuel de Codage has since been adopted by international Egyptology as the official common standard for registering hieroglyphic texts on computer. Mark-Jan Nederhof proposed an enhanced encoding scheme to remove many of the limitations in the Manuel de Codage.

\pkgname{HieroTeX} is a \latexe package developed by to typeset hieroglyphic texts and still works well. The advantages of using \tex is of course its excellent typesetting capabilities and the usage of macros. Although inputting the texts as MdC codes is not that difficult, repeating the same codes over and over can be avoided with easily constructed simple substitution macros. 

\subsection{fonts}

One of the best fonts I came across is \idxfont{Aegyptus} from \url{http://users.teilar.gr/~g1951d/}\footnote{The site also has fonts for Aegean Numbers, Ancient Greek Musical Notation, Ancient Greek Numbers, Ancient Roman Symbols, Arkalochori Axe, Carian, Cypriot Syllabary, Dispilio tablet, Linear A, Linear B Ideograms, Linear B Syllabary, Lycian, Lydian, Old Italic, Old Persian, Phaistos Disc, Phoenician, Phrygian, Sidetic, Troy vessels’ signs and Ugaritic. Cretan Hieroglyphs and Cypro-Minoan script(s) are offered in separate files.}. The font provides all the unicode characters and also offers an additional number of glyphs that are not in the Unicode standard. The font uses the Unicode Private Use Areas to encode the glyphs. 

Another font is the Noto Egyptian Hieroglyphics from Google. This is a lightweight font with the symbols in their proper unicode slots. Mark-Jan Nederhof's \idxfont{NewGardiner} font is another one with support only for the Gardiner set. The codepoint mappings are incorrect, as the font has been  
encoded to EGPZ. The font is similar to the Aegyptus font, however it is just transposed and not recommended unless it is transposed. 

The editor software JSesh\footnote{\protect\url{http://jsesh.qenherkhopeshef.org/}} also provides a free font |JSeshFont.ttf|. This offers a correctly mapped unicode and is another good alternative. The symbols are drawn somewhat simpler and is just a matter of taste what you want to use.

My recommendation is for short demonstration purposes, the Noto font is to be preferred while for more serious work the Aegyptus font will be more useful. Using Lua the font can be transposed automatically to allow the use of commands that refer to unicode numbers. Another advantage of the Aegyptus font is that the glyphs are named with their Gardiner numbers, so it is somewhat easier to programmatically access them by name.\footnote{Unicode does not name the glyphs, but simply calls the Egyptian Hieroglyph $n$. } 

\medskip

\ifxetex
\bgroup
\centering 
\font\myfont = "Aegyptus"
\scalebox{7}{\myfont\XeTeXglyph 201}
\scalebox{7}{\myfont\XeTeXglyph 203}
\scalebox{7}{\myfont\XeTeXglyph 163}
\scalebox{7}{\myfont\XeTeXglyph 164}
\scalebox{7}{\myfont\XeTeXglyph 165}
\scalebox{7}{\myfont\XeTeXglyph 168}
\captionof{table}{Example of Egyptian Hieroglyphics typeset with the \textit{Aegyptus} font.} 
\egroup
\fi

\ifluatex
\bgroup
\centering 
\aegyptus
\scalebox{7}{\char"F300C}
\scalebox{7}{\char"F3001}
\scalebox{7}{\char"F3010}
\scalebox{7}{\char"F308B}
\scalebox{7}{\char"F3097}
\scalebox{7}{\char"F3091}
\captionof{table}{Example of Egyptian Hieroglyphics typeset with the \textit{Aegyptus} font.} 
\egroup

\fi


\subsection{Unicode Block}

Egyptian hieroglyphs is a Unicode block containing the Gardiner's sign list of Egyptian hieroglyphics.
The code points, in the range |0x13000| to |0x1342E|, are available starting from
\href{http://unicode.org/charts/PDF/U13000.pdf}{Unicode 5.2}

\begin{scriptexample}[]{Hieroglyphic}
\bgroup
\unicodetable{hiero}{"13000,"13010,"13020,"13030,"13040,"13050,"13060,"13070,%
"13080,%
"13090,"130A0,"130B0,"130C0,"130D0,"130E0,"130F0,%
"13100,"13110,"13120,"13130,"13140,"13150,"13060,"13070,"13080,"13090}
\egroup
\end{scriptexample}

\subsection{Gardiner's classification}

The standard reference on Egyptian hieroglyphics is Gartiner's Sign List, which lists common Egyptian hieroglyphs. These are grouped in categories from A-Aa. Each category represents a theme for example category A, is "man and his occupations". Based on this list ``Queen with flower" is denoted as \texttt{B7}. 

\subsubsection{Character Names} 

Egyptian hieroglyphic characters have traditionally been designated in
several ways:

\begin{enumerate}
\item  By complex description of the pictographs: \texttt{GOD WITH HEAD OF IBIS}, and so forth.
\item By standardized sign number: C3, E34, G16, G17, G24.
\item For a minority of characters, by transliterated sound.
\end{enumerate}

The characters in the Unicode Standard make use of the standard Egyptological catalog
numbers for the signs. Thus, the name for {\hiero\char"130F9} |U+13049| egyptian hieroglyph e034 refers
uniquely and unambiguously to the Gardiner list sign E34, described as a “{\aegean DESERT HARE}” ({\hiero \char"130FA}) and used for the sound “wn”. The Unicode catalog values are padded to three places with
zeros, so where the Gardiner classification is shown as \texttt{E34}, the unicode value is \texttt{E034}. 

Names for hieroglyphic characters identified explicitly in Gardiner 1953 or other sources as
variants for other hieroglyphic characters are given names by appending “A”, “B”, ... to the sign number. In the sources these are often identified using asterisks. Thus Gardiner’s G7,
G7*, and G7** correspond to U+13146 egyptian sign g007 {\hiero \char"13147}, U+13147 egyptian sign g007a, and U+13148 egyptian sign g007b, respectively.

\def\texthiero#1{{\color{black!95}\hiero #1}}

\begin{longtable}{>{\Large}lll>{\ttfamily}l}
{\hiero \char"13000}&A1-A70 & Man and his occupations &U+13000-1304F\\
{\hiero \char"13050}&B1-B9  &Woman and her occupations &U+13050-13059\\
{\hiero \char"1305A} &C1-C24 &Anthropomorphic Deities &U+1305A-13075\\
{\hiero \char"13076} &D1-D67 &Parts of the Human Body &U+13076-130D1\\
{\hiero \char"130D2} &E1-E38 &Mammals &U+13076-130D1\\
{\hiero \char"130FE}  &F1-F53	&Parts of Mammals &U+130FE-1313E\\
{\hiero\char"1313F}	&G1-G54	&Birds &U+1313F-1317E\\
{\hiero \char"1317F}	&H1-H8	&Parts of Birds &U+1317F-13187\\
\texthiero{\char"13188}	&I1-I15	&Amphibious Animals, Reptiles, etc. &U+13188-1319A\\
\texthiero{\char"1319B}	&K1-K8	&Fishes and Parts of Fishes &U+1319B-131A2\\
\texthiero{\char"131A3}	&L1-L8	&Invertebrata and Lesser Animals &U+131A3-131AC\\
\texthiero{\char"131AD}	&M1-M44	&Trees and Plants &U+13AD-131EE\\
\texthiero{\char"131EF}	&N1-N42	&Sky, Earth, Water &U+131EF-1321F\\
\texthiero{\char"13250}	&O1-O51	&Buildings and Parts of Buildings &U+13250-1329A\\
\texthiero{\char"1329B}	&P1-P11	&Ships and Parts of Ships &U+1329B-132A7\\
\texthiero{\char"132A8}	&Q1-Q7	& Domestic and Funerary Furniture &U+132A8-132AE\\
\texthiero{\char"132AF}	&R1-R29	&Temple Furniture and Sacret Emblems &U+132AF-132D0\\
\texthiero{\char"132D1}	&S1-S46	&Crowns, Dress, Staves, etc. &U+132D1-13306\\
\texthiero{\char"13307}	&T1-T36	&Warfare, Hunting, Butchery &U+13307-13332\\
\texthiero{\char"13333}	&U1-42	&Agriculture, Crafts and Professions &U+13333-13361\\
\texthiero{\char"13362}	&V1-V40a	&Rope, Fibre, Baskets, Bags, etc. &U+13362-133AE\\
\texthiero{\char"133AF}	&W1-W25	&Vessels of Stone and Earthenware &U+133AF-133CE\\
\texthiero{\char"133CF}	&X1-X8a	&Loaves and Cakes &U+133CF-133DA\\
\texthiero{\char"133DB}	&Y1-Y8	&Writing, Games, Music &U+133DB-133E3\\
\texthiero{\char"133E4}	&Z1-Z16H	&Strokes, Geometrical Figures, etc. &U+133E4-1340C\\
\texthiero{\char"1340D}	&Aa1-Aa32	&Unclassified &U+1340D-1342E\\
\end{longtable}

I particularly like the crocodile sign \def\crocodile{\color{teal}{\Huge\texthiero{\char"13188}}} {\crocodile}, as it is applicable to describe people in my field of work. 

\begin{scriptexample}[]{Woman and her occupations}
\unicodetable{hiero}{"13050}
\end{scriptexample}

\section{Positioning}

One of the core assumptions of any hieroglyphic encoding or mark-up scheme following the MdC is that signs and groups of signs maybe positioned next to each other or above each other. The former is indicated by the operator * and the latter by :. One may also use -, which functions as * for horizontal texts and as : for vertical text. 

In some dialects of the MdC relative positioning has been extended by the use of the |&| operator. This is used to form a kind of ligature, such as |D&t| can be defined to represent the \textit{Cobra at rest} sign I10 with sign X1 underneath, as follows:

\begin{center}
{\hiero\HUGE
       \mbox{\rlap{\char"133CF}\char"13193\hfill\hfill}\\
       {\large|insert[bs](I10,X1)|}

\mbox{\rlap{\scalebox{0.5}{\char"133E3}}\char"13193\hfill\hfill}\\
 	
}
\end{center}

This is only a partial solution and to automate it via kerning tables, will require hundreds of entries in the kerning tables. It will also need constant modifications as researchers discover new combinations. A better approach and which is easily applied to \tex based systems would be to adopt Nederhof's method of creating a new command |insert[bs](I10,X1)|. 

In \tex one could simply define a command \cmd{\insert} with one optional argument to handle the positioning. The positioning uses the letters [b,t,s,e] to position the glyph. the letters s and e stand for start and end, whereas b,t for bottom and top respectively. When there are only two symbols involved, this is not such a difficult operation, but when three or more symbols are to be grouped and kerned together, inserting with some form of scaling is necessary.

\subsection{Enclosures}

Enclosures. The two principal names of the king, the \emph{nomen} and \emph{prenomen}, were normally
written inside a \emph{cartouche}: a pictographic representation of a coil of rope.

In the Unicode representation of hieroglyphic text, the beginning and end of the cartouche
are represented by separate paired characters, somewhat like parentheses. The Unicode manual states that `rendering of a full cartouche surrounding a name requires specialized layout software', which is of course an easy task for \tex.

\begin{macro}{\cartouche}
The commands \cmd{\cartouche} and \cmd{\cartouche}, from Peter Wilson's \pkgname{hierglyph} package have been used for many years to demonstrate the use of hieroglyphics with \latexe. 
\end{macro}

There are a several characters for these start and end cartouche characters, reflecting various styles for the enclosures.

\cartouche{{\hiero \char"13147}$sin^{2} x + cos^{2} x = 1$}
\Cartouche{{\hiero \char"13147}$sin^{2} x + cos^{2} x = 1$}

Unicode:{\hiero 𓇓𓏏𓊵𓏙𓊩𓁹𓏃𓋀𓅂𓊹𓉻𓎟𓍋𓈋𓃀𓊖𓏤𓄋𓈐𓎟𓇾𓈅𓏤𓂦𓈉 }

\textpmhg{\HQ} 

\cartouche{\pmglyph{K:l-i-o-p-a-d:r-a}}
%\translitpmhg{\HK\Hl\Hi\Ho\Hp\Ha\Hd\Hr\Ha}

\printunicodeblock{./languages/hieroglyphics.txt}{\hiero}
\printunicodeblock{./languages/hieroglyphics-13100.txt}{\hiero}
\printunicodeblock{./languages/hieroglyphics-13200.txt}{\hiero}
\printunicodeblock{./languages/hieroglyphics-13300.txt}{\hiero}
\printunicodeblock{./languages/hieroglyphics-13400.txt}{\hiero}
\section{Numerals}

Egyptian numbers are encoded following the same principles used for the
encoding of Aegean and Cuneiform numbers. Gardiner does not supply a full set of
numerals with catalog numbers in his Egyptian Grammar, but does describe the system of
numerals in detail, so that it is possible to deduce the required set of numeric characters.

Two conventions of representing Egyptian numerals are supported in the Unicode Standard.
The first relates to the way in which hieratic numerals are represented. Individual
signs for each of the 1s, the 10s, the 100s, the 1000s, and the 10,000s are encoded, because in
hieratic these are written as units, often quite distinct from the hieroglyphic shapes into
which they are transliterated. The other convention is based on the practice of the \emph{Manual
de Codage}, and is comprised of five basic text elements used to build up Egyptian numerals.
There is some overlap between these two systems.

%% Needs some work to get it into LuaLaTeX
%% omitted for the time being
%\ifxetex
%\begin{texexample}{TeXeXglyph}{ex:xetexglyph}
%\raggedright
%\font\myfont = "Aegyptus"
%\setcounter{glyphcount}{136}
%
%\whiledo
%{\value{glyphcount}<\XeTeXcountglyphs\myfont}
%{\arabic{glyphcount}:~
%{\myfont\XeTeXglyph\arabic{glyphcount}}\quad
%\stepcounter{glyphcount}}
%\end{texexample}
%\fi

\section{Input Methods}

If you writing a document with a lot of hieroglyphics inputting of hieroglyphics can be problematic. Most researchers in the field will use special keyboards or editors. They also use MS/Word or OpenOffice. They can both be coerced to produce reasonable documents, but with \tex obviously better results can be achieved. One such editor is \href{http://jsesh.qenherkhopeshef.org/}{jsesh}. 


\begin{luacode*}
    local h = {}
          h = dofile("hiero.lua")
    local options = {style="block",
                     echo=true,
                     direction="RL",
                     size = "\\Huge",
                     color = "green",
                     headings = "captionof{figure}"  -- section/tablecaption/figurecaption
                     }
   -- prints full symbol list
   h.printgardiner(t,options)

   tex.print("\\par")
   local options = {style="block",
                     echo=true,
                     heading="\\par",
                     direction="RL",
                     color = "teal",
                     scale = 8}

   h.printhierochar("hiero","1317D",options)
   h.printhierochar("hiero","13000",{direction="RL",
                                        color = "teal",
                                        scale = 8})
   h.printhierochar("hiero","13003",{direction="LR",
                                        color = "teal",
                                        scale = 1})
   h.parseMdC([[M23-X1-R4-X8-Q2-D4-W17-R14-G4-R8-O29-
               V30-U23-N26-D58-O49-Z1-F13-N31-V30-N16-
               N21-Z1-D45-N25!]])

   tex.print("\\par")
   h.printgardinercat("B")

\end{luacode*}

\newcommand\hierochar[2][direction = "LR",
                         color     = "teal",
                         scale     = 1]{% 
               \luaexec{
                h = h or {}
                h = require("hiero.lua")  
                h.parseMdC(#2,{#1})}}
               
\newcommand\printhierochar[3][direction = "LR",
                              color     = "teal",
                              scale     = 4]{% 
               \luaexec{
                h = h or {}
                h = require("hiero.lua")  
                h.printhierochar(#2,#3,{#1})}}

This file just tests the various commands available for manipulating hieroglyphics. We tried to 
generalize the commands, so they can be re-used for other type of hieroglyphics.

{
\hierochar{"A1-A2-A3!"}

\centering 

\def\options{direction = "LR",
             color     = "teal",
             scale     = 7}

\def\fontname{"hiero"}

\def\hierochar#1{\printhierochar[\options]{\fontname}{#1}}
}


\begin{scriptexample}[]{Some Example}
Sometimes kerning might be required, especially if the
glyphs are scaled.This is easily achieved with a \cmd{\kern}
command and a suitable skip dimension.

\medskip

\bgroup
\fboxsep=0pt\fboxsep.4pt
\def\options{direction = "RL",
             color     = "black!95",
             scale     = 5}
\centering

\color{teal}
\fbox{\hierochar{"13051"}}
\kern-4mm
\hierochar{"13003"}
\def\options{direction = "LR",
             color     = "black!95",
             scale     = 5}
\fbox{\hierochar{"13003"}}\color{red}
\kern-4mm
\hierochar{"13051"}
\color{black!95}
\egroup
\begin{verbatim}
\centering
\hierochar{"13051"}
\kern-4mm
\hierochar{"13003"}
\def\options{direction = "RL",
             color     = "black!95",
             scale     = 5}
\hierochar{"13003"}
\kern-4mm
\hierochar{"13051"}
\end{verbatim}
\end{scriptexample}

A bit of a diversion is appropriate at this point. Our attempt after the historical overview, is to provide some routines for the capturing and display of hieroglyphic texts using LuaTeX. This involves getting low level information from the system regarding fonts. 

\begin{figure}[ht]
\begin{minipage}{0.45\textwidth}
\centering
\includegraphics[width=0.6\textwidth]{./images/fontforge.jpg}
\end{minipage}
\begin{minipage}[t]{0.45\textwidth}
\caption{Viewing font information with fontforge.}
\end{minipage}
\end{figure}

For each glyph, we are interested to get its unicode number, the position in the font table, its name and most importantly the font metrics. The font metrics are a set of parameters that are used to measure the bounding box, any ascenders or descenders and similar information. Using fontforge, these parameters can easily be viewed. However, we are not interested to make any modifications manually; what we are interested is to programmatically obtain this information using Lua. Lua's philosophy and a mantra repeated often by the developers, is that it provides the tools and not the solutions. What this means to the LuaTeX programmer, is that we need to reach very low level  to get this information, which is a road with many bumps. Luckily the tools have been provided by the LuaTeX developers. This comes with a lot of benefits as we can also do our own on the fly mapping, such as creating an index table holding all the Gardiner numbers. 

The |fontloader.open| function loads a font, but it's not usable by itself; the result should be turned into a table with
\textbf{fontloader.to\_table}, as follows:

\begin{verbatim}
  local f = fontloader.open
     ("c:/windows/fonts/NotSansEgyptianHieroglyphics-
       Regulat.ttf")
  fonttable = fontloader.to_table(f)
  fontloader.close(f)
\end{verbatim}

We will use the Google No Tofu Egyptian Hieroglyphic font to experiment with our hieroglyphics. I have used a full path to load the font, which resides on my windows machine in the fonts folder. Once we load all the information in the |fonttable| we use |fontloader.close| to discard the userdata from which the table is extracted. 

What makes OpenType fonts special is that they describe every aspect that you might be able to think of when you think of putting letters together to form words. In addition to the obvious "this is what letters look like" information, OpenType fonts also specify things like the name of each letter that is available in the font, how much of the Unicode standard the font implements, which horizontal and vertical metrics apply to which letters, exactly how the letters are arranged inside the font so that they can quickly be read out, what kind of font classifications apply (is it a fantasy font? is it bold face? is it fixed width? etc), what kind of memory allocation a printer needs to perform in order to be able to even load the font, etc. etc. etc. All these are stored in tables upon tables, similat to a collection of Russian dolls.

To view the values in the fonttable, we will first iterate over the \textbf{fonttable} and extract all the first level keys.

\begin{texexample}{Iterating through a font table}{}
\begin{luacode*}
local z={}
tf=fontloader.to_table(fontloader.open("c:/windows/fonts/NotoSansEgyptianHieroglyphs-Regular.ttf"))

-- we sort the keys to create a table
-- important keys to us are tf.glyphs

for k,v in pairs (tf) do
   --tex.print(k.."\\par")
   table.insert(z, k)
end

table.sort(z)
tex.print("\\begin{multicols}{3}\\raggedright")
for k,v in pairs (z) do
   z[k] = string.gsub(z[k],"%_","\\textunderscore ")
   local s = tf[v]
   tex.print("\\textbullet\\hskip3pt\\hangindent2em " .. z[k].." [\\textit{"..type(s).."}] ","\\par")
end
tex.print("\\end{multicols}")
\end{luacode*}
\end{texexample}

We iterate through the \textbf{fonttable} using the Lua  "pair" iterator and we simply print all the keys and the type of the values in a human readable form as shown in the example. Note the use of |\textunderscore| that replaces all underscores in the fields with its text equivalent to sanitize the output. This is a quick and dirty way to avoid the use of catcodes. Many of the keys, bear intuitive names and are not difficult to discern: \textit{version}, \textit{copyright} and the like. Getting the type of Lua variables is important in order to use them for error trapping. When you attempt for example to print a nil value an error will occur.

Now that we have peeked under the font we will iterate and capture the information of interest, which we will put into another table with two keys \textbf{info}  and \textbf{metrics}. In the metrics file we will get the bounding box related metrics of each and every glyph in the font and save it, into our own table. 

\begin{texexample}{More Metrics}{}
  \begin{luacode*}
   tex.print("units per em = ", tf.units_per_em,"\\par")
   for i,j in ipairs (tf.glyphs[6].boundingbox) do
      tex.print("bounding box["..i.."]".." = ", j,"\\par")
   end 
   local w = (tf.glyphs[6].boundingbox[3]-tf.glyphs[6].boundingbox[1])/tf.units_per_em
   local h = tf.glyphs[6].boundingbox[4]/tf.units_per_em
   tex.print("glyph width = ", w,"em\\par")
   tex.print("glyph height = ", h,"em\\par")

-- presents a nicely typeset table 

local rep, write = string.rep, tex.print
function ExploreTable (tab, offset)
    offset = offset or ""
    for k, v in pairs (tab) do
        local newoffset = offset .. "\\mbox{.}"
        if type(v) == "table" then
           -- if k == "boundingbox" then write("BB") end
           write(offset..k .. " = \\{\\par ")
           ExploreTable(v, newoffset)
           write(offset..newoffset .. "\\}\\par")
         else
           write(offset..k .. " = "..tostring(v),"\\par")
         end
      end
end

write("\\par{\\ttfamily ")
ExploreTable(tf.glyphs[38],"\\mbox{.}")
write("}")
  \end{luacode*}
\end{texexample}

The OpenType fonts standard, provides for so much information that we will ignore most of the items and focus on only a few tables and fields. A small utility after Paul Isambert's article is necessary to enable us to view tables easily within this book,


\begin{texexample}{ExploreTable utility}{}
\begin{luacode*}
-- presents a nicely typeset table 

local rep, write = string.rep, tex.print
function ExploreTable (tab, offset)
    offset = offset or ""
    for k, v in pairs (tab) do
        local newoffset = offset .. "\\mbox{.}"
        if type(v) == "table" then
           -- if k == "boundingbox" then write("BB") end
           write(offset..k .. " = \\{\\par ")
           ExploreTable(v, newoffset)
           write(offset..newoffset .. "\\}\\par")
         else
           write(offset..k .. " = "..tostring(v),"\\par")
         end
      end
end

write("\\par{\\ttfamily ")
ExploreTable(tf.glyphs[38],"\\mbox{.}")
write("}")
  \end{luacode*}
\end{texexample}

A good utility also is |TTX| that will convert an OTF font to XML and back. This requires that you have python installed.\footnote{See some good guidelines as to how to install it at \url{http://www.glyphrstudio.com/ttx/}.} The utility uses python to do the conversion. The archive can be downloaded from \url{http://sourceforge.net/projects/fonttools/files/latest/download}. This is a three prong attack. You need to have python install, the numpy library and then the TTX package. The |TTX| program was written by the font designer Just van Rossum, brother of the creator of the Python language, Guido van Rossum. The tool converts TrueType into human-readable |XML| format. The most attractive feature of this tool is that it also perform the opposite operation that is create a TruType font from an |XML| file. The |XML| format makes the hierarchy of the format clearer. Since SVG fonts are also described in |XML| it becomes an easier task to convert an |SVG| font to a TrueType font. To convert |bar.ttf| into |bar.ttx| you simply write:

\begin{verbatim}
ttx bar.ttf
\end{verbatim}

Similarly for the opposite conversion, from |.ttx| to |.ttf|

\begin{verbatim}
ttx bar.ttx
\end{verbatim}

The generated ttx file is approximately ten times larger than the original |.ttf| file. The files generated are huge affairs and difficult to manage.The command line option |-l| prints a list of the tables in the font. |TTX| is indispensable in the ``humanization'' of TrueType fonts. The details of the tables and what each field represents are eloquently described in that indispensable book by Yannis Haralambous \textit{Fonts \& Encodings.} Although the book is now somewhat dated, it is still the best source of information on many esoteric topics related to fonts. 






\input{./languages/meroitic}
\chapter{Ugaritic}
\label{s:ugaritic}
\index{Ugaritic fonts>Noto Sans Ugaritic}
\index{Ugaritic}
\index{Akkadian}
\index{Unicode>Ugaritic}
\parindent1em
\newfontfamily\ugaritic{NotoSansUgaritic-Regular.ttf}

\section{Background}
Sometime between 1190-1185 bce, the houses of Ugarit were abandoned by their inhabitants, then pillaged and burned. If they were destroyed by the Sea Peoples we will never know for sure, although this is very likely. This catastrophe ended a history of almost 6000 years. Ugarit was never rebuild and the ruins were buried for centuries before they were discovered in 1929. 

\begin{figure}[htbp]
\centering
\includegraphics[width=\textwidth]{ugarit-excavations}
%http://www.persee.fr/docAsPDF/syria_0039-7946_1936_num_17_2_3887.pdf
\end{figure}

Merchants figure prominently in Ugarit’s archives. The citizens engaged in trade, and many foreign merchants were based in the state, for example from Cyprus exchanging copper ingots in the shape of ox hides. The presence of Minoan and Mycenaean pottery suggests Aegean contacts with the city. It was also the central storage place for grain supplies moving from the wheat plains of northern Syria to the Hittite court.

common defence system (§ 11.5.4.3). The abundance of Cypriot
pottery,173 the Cypro-Minoan texts found in Ugarit ( L i v e r a n i 1979a,
1322-3) as well as letters18 and administrative texts,19 are also witness
to relationhips between the two communities at both the cultural
and the commercial levels. 

Some Cypriots (ally, altyy, DLU, 33)
receive from the Ugaritian administration food and clothing,20 others
belong to the guild of craftsmen.21 On the other hand, from its structure
the administrative text KTU 4.102 = RS 11.857 seems to be
a list of prisoners of war, or of persons detained for some reason,
who come from Cyprus ( V i t a 1995a, 108). An unpublished letter
found in Ras Shamra in 1994, which reports the dispatch of an
emissary of the king of Cyprus to Ugarit to deal with the freeing of
Cypriots detained on Ugaritic soil,22 could support this hypothesis

The \idxlanguage{Ugaritic} language  is written in alphabetic cuneiform. This was an innovative blending of an alphabetic script (like \hyperref[s:hebrew]{Hebrew}) and cuneiform (like Akkadian). The development of alphabetic cuneiform seems to reflect a decline in the use of Akkadian as a \textit{lingua franca} and a transition to alphabetic scripts in the eastern Mediterranean. Ugaritic, as both a cuneiform and alphabetic script, bridges the cuneiform and alphabetic cultures of the ancient Near East.


\begin{figure}[hb]
\centering
\includegraphics[width=\textwidth]{ugaritic-first-tablet}
\caption{A list of offerings with the first tablet number (KTU 1.39 = RS 1.001; Photo: UGARIT - FORSCHUNG Archive)}
\end{figure}

The Ugaritic script is a cuneiform (wedge-shaped) abjad used from around either the fifteenth century BCE[1] or 1300 BCE[2] for Ugaritic, an extinct Northwest Semitic language, and discovered in Ugarit (modern Ras Shamra), Syria, in 1928. It has 30 letters. Other languages (particularly Hurrian) were occasionally written in the Ugaritic script in the area around Ugarit, although not elsewhere.


\section{Material Culture}

Excavations at Ugarit have yielded an abundance of objects of everyday life that we can deduce the every day life of its inhabitants in a higher level of detail than many other civilizations. Objects recovered include mirrors, combs, cooking and drinking utensils, pottery, gems. An interesting item is the clepsydra shown in Figure~\ref{fig:clepsidra} used as a shower head. The religion and cults is also well represented. This is not easy to use as an individual and it was probably used with the help of a servant.

The Ugarites were actively interacting in trading. 

\begin{figure}[htbp]
\includegraphics[width=\textwidth]{clepsidra}
\caption{“Clepsydra” or shower vase RS 81.509
1981, City Center, House E, room 1201. Latakia Museum
H 19.5 cm, Diameter (max.) 18 cm. Fine plain buff pottery with burnished surface. Jug with a large,
ovoid body. The opening is narrow, contracting to a small hole 1 cm in diameter. The bottom is
pierced with 22 small holes to form a strainer. The narrowness of the opening does not permit filling
by any means other than plunging the vase entirely into a large container full of water. It holds about
1 liter. The function of this sort of vase is obvious. The container remained full if the opening was
sealed with one’s thumb to prohibit the entrance of air; the liquid could not flow out through the
bottom. When the thumb was removed (allowing air to enter the jug), the water could flow out
through the bottom, creating a type of shower head.
This object matches the definition of a clepsydra mentioned by ancient authors (Hieron): in its
primary sense, the term clepsydra is not restricted to a measure of time. What we have here is an instrument
used for washing, like a shower in a bathing installation (or shower stall). This was an object
of everyday life, but only in a relatively refined context. This vase was found with other personal
funerary objects fallen from the upper floor of a house of medium status in the city center. Other examples
(e.g., RS 30.325) show that this was not an uncommon item in homes at Ugarit.\\
– Bib.: M. Yon, P. Lombard, and M. Renisio, in RSO III, 1987, p. 106, fig. 87; P. Lombard, ibid., pp. 351–57.}
\label{fig:clepsidra}
\end{figure}

Clay tablets written in Ugaritic provide the earliest evidence of both the North Semitic and South Semitic orders of the alphabet, which gave rise to the alphabetic orders of Arabic (starting with the earliest order of its abjad), the reduced Hebrew, and more distantly the Greek and Latin alphabets on the one hand, and of the Ge'ez alphabet on the other. Arabic and Old South Arabian are the only other Semitic alphabets which have letters for all or almost all of the 29 commonly reconstructed proto-Semitic consonant phonemes. 

According to Dietrich and Loretz in Handbook of Ugaritic Studies (ed. Watson and Wyatt, 1999): "The language they [the 30 signs] represented could be described as an idiom which in terms of content seemed to be comparable to Canaanite texts, but from a phonological perspective, however, was more like Arabic."
The script was written from left to right. Although cuneiform and pressed into clay, its symbols were unrelated to those of the Akkadian cuneiform.

\begin{scriptexample}[]{Ugaritic}
\unicodetable{ugaritic}{"10380,"10390}
\end{scriptexample}

{\let\aegean\arial
\printunicodeblock{./languages/ugaritic.txt}{\ugaritic}
}

\bgroup

\let\a\arial
\Large
\begin{longtable}[l]{%
>{\arial\large}r|
>{\ugaritic}c| 
>{\arial\large}c 
>{\arial\large}c 
>{\arial\large}c >{\arial\large}c
}

&\a Sign	&\a Trans.	&\a IPA	&\a Hebrew	&\a Arabic \\
\hline
\inc &𐎀	&ʾa	& ʔa	&א	&أ \\
\inc &𐎁	&b	& b	    &ב	&ب \\
\inc &𐎂	&g	&ɡ	&ג	&ج\\
\inc &𐎃	&ḫ	&x	&	&خ\\
\inc &𐎄	&d	&d	&ד	&د\\
\inc &𐎅	&h	&h	&ה	&ه\\
\inc &𐎆	&w	&w	&ו	&و\\
\inc &𐎇	&z	&z	&ז	&ز\\
\inc &𐎈	&ḥ	&ħ	&ח	&ح\\
\inc &𐎉	&ṭ	&t̴	&ט	&ط\\
\inc &𐎊	&y	&j	&י	&ي\\
\inc &𐎋	&k	&k	&כ	&ك\\
\inc &𐎌	&š	&ʃ	&ש	&ش\\
\inc &𐎍	&l	&l	&ל	&ل\\
\inc &𐎎	&m	&m	&מ	&م\\
\inc &𐎏	&ḏ	&ð	&	&ذ\\
\inc &𐎐	&n	&n	&נ	&ن\\
\inc &𐎑	&ẓ	&θ̴	&	&ظ\\
\inc &𐎒	&s	&s	&ס	&س\\
\inc &𐎓	&ʿ 	&ʕ	&ע	&ع\\
\inc &𐎔	&p	&p	&פ	&ف\\
\inc &𐎕	&ṣ	&s̴	&צ	&ص\\
\inc &𐎖	&q	&q	&ק	&ق\\
\inc &𐎗	&r	&r	&ר	&ر\\
\inc &𐎘	&ṯ	&θ	&	&ث\\
\inc &𐎙	&ġ	&ɣ	&	&غ\\
\inc &𐎚	&t	&t	&ת	&ت\\
\inc &𐎛	&ʾi	&ʔi	&	&ئ\\
\inc &𐎜	&ʾu	&ʔu	&	&ؤ\\
\end{longtable}
\egroup


\textit{\LARGE$$\stackrel{\mbox{ho}}{.}$$}

% Tranliteration macros 
% 
\bgroup\ugaritic
\def\a{\char"10380}
\def\b{\char"10381}
\def\g{\char"10382}
\LARGE \a \b \g 
\egroup

\section{Online Collections}

http://digital.library.stonybrook.edu/











\section{Sumero Akkadian Cuneiform}
\label{s:sumero}
\newfontfamily\sumero{NotoSansSumeroAkkadianCuneiform-Regular.ttf}
In Unicode, the Sumero-Akkadian Cuneiform script is covered in two blocks:
U+12000–U+1237F Cuneiform
U+12400–U+1247F Cuneiform Numbers and Punctuation
These blocks, in version 6.0, are in the Supplementary Multilingual Plane (SMP).

The sample glyphs in the chart file published by the Unicode Consortium[2] show the characters in their Classical Sumerian form (Early Dynastic period, mid 3rd millennium BCE). The characters as written during the 2nd and 1st millennia BCE, the era during which the vast majority of cuneiform texts were written, are considered font variants of the same characters.

The character set as published in version 5.2 has been criticized, mostly because of its treatment of a number of common characters as ligatures, omitting them from the encoding standard.

\begin{scriptexample}[]{Sumero Akkadian}
\unicodetable{sumero}{"12000,"12010,"12020,"12030,"12040,"12050,"12060,"12070,
"12080,"12090,"12400,"12410,"12420,"12430}
\end{scriptexample}

\begin{table}[b]
\begin{scriptexample}[]{textbox}
From Plato's dialogue Phaedrus 14, 274c-275b:

Socrates: [274c] I heard, then, that   in Egypt, was one of the ancient gods of that country, the one whose sacred bird is called the ibis, and the name of the god himself was Theuth. He it was who [274d] invented numbers and arithmetic and geometry and astronomy, also draughts and dice, and, most important of all, letters. 

Now the king of all Egypt at that time was the god Thamus, who lived in the great city of the upper region, which the Greeks call the Egyptian Thebes, and they call the god himself Ammon. To him came Theuth to show his inventions, saying that they ought to be imparted to the other Egyptians. But Thamus asked what use there was in each, and as Theuth enumerated their uses, expressed praise or blame, according as he approved [274e] or disapproved.  

"The story goes that Thamus said many things to Theuth in praise or blame of the various arts, which it would take too long to repeat; but when they came to the letters, [274e] “This invention, O king,” said Theuth, “will make the Egyptians wiser and will improve their memories; for it is an elixir of memory and wisdom that I have discovered.” But Thamus replied, “Most ingenious Theuth, one man has the ability to beget arts, but the ability to judge of their usefulness or harmfulness to their users belongs to another; [275a] and now you, who are the father of letters, have been led by your affection to ascribe to them a power the opposite of that which they really possess.  

"For this invention will produce forgetfulness in the minds of those who learn to use it, because they will not practice their memory. Their trust in writing, produced by external characters which are no part of themselves, will discourage the use of their own memory within them. You have invented an elixir not of memory, but of reminding; and you offer your pupils the appearance of wisdom, not true wisdom, for they will read many things without instruction and will therefore seem [275b] to know many things, when they are for the most part ignorant and hard to get along with, since they are not wise, but only appear wise." 
\end{scriptexample}
\end{table}


\printunicodeblock{./languages/cuneiform.txt}{\sumero}





\section{Inscriptional Parthian}
\label{s:parthian}
\index{Ancient and Historic Scripts>Inscriptional Parthian}
\index{Inscriptional Parthian fonts>Noto Sans Inscriptional Parthian}

The Parthian script developed from the Aramaic alphabet around the 2nd century BCE and was used during the Parthian and Sassanid periods of the Persian Empire. The latest known inscription dates from 292 CE. 

\newfontfamily\parthian{NotoSansInscriptionalParthian-Regular.ttf}
Inscriptional Parthian is a Unicode block containing characters of the official script of the Sassanid Empire.

\newenvironment{parthiannumbers}{^^A
\def\1{\parthian\char"10B58}
\def\2{\parthian\char"10B59}
\def\3{\text{\parthian\char"10B5A}}
\def\4{\text{\parthian\char"10B5B}}^^A 
\TextOrMath\4 \4
\TextOrMath\3 \3
}{}
\index{Parthian numbers}
\begin{scriptexample}[]{}

\unicodetable{parthian}{"10B40,"10B50}



\end{scriptexample}

Inscriptional Parthian has its own numbers, which have right-to-left
directionality. The numbers are built up out of 1, 2, 3, 4, 10, 20, 100, and 1000 which is not such a great scheme. The inscriptions are not
normalized uniformly. The units are sometimes written with strokes of the same height, or with a final
stroke that is longer, either descending or ascending to show the end of the number; compare 5 in 15 ({\parthian \char"10B59 \char"10B5B}
or 2 + 3) and in 45 (òõ or 1 + 4); compare 6 in 16 (öö or 3 + 3) and in 36 (òôö or 1 + 2 + 3). The
encoding here allows the specialist to choose his or her preferred representation. The following is an list
of numbers attested in Inscriptional Parthian. The third column is displayed in visual order.

The |phd| package offers rudimentary support for Parthian numbers in the form of an environment |parthiannumbers|, which can be used as follows:

\begin{texexample}{Inscriptional Parthian numbers}{parth}
\begin{parthiannumbers}
\1 $= 1$
\2 $= 2$
\begin{align*}
\3 &= 3\\
\4 &= 4\\
\3\4 &=7
\end{align*}
\end{parthiannumbers}
\end{texexample}



\printunicodeblock{./languages/inscriptional-parthian.txt}{\parthian}




\footnote{\url{http://www.unicode.org/L2/L2007/07207-n3286-parthian-pahlavi.pdf}} 
\section{Linear A}
\label{s:lineara}
\newfontfamily\lineara{Aegean.ttf}

\section{Aegean and Cypriote Syllabaries}

The Greeks had evidently already occupied the mainland and islands of the
Ægean, including Crete, by the middle of the third millennium
BC. Around 2000 BC, following their consolidation of power on
Crete, new wealth from trade with cosmopolitan Canaan
allowed the creation of a complex palace economy, with major
centres at Knossos, Phaistos and other Cretan sites – Europe’s
first high civilization, the Minoan. Trade with Canaan had evidently
also brought Greeks into contact with Byblos’ pictorial
syllabic writing, whose underlying principle the Minoans borrowed.
Now, Cretans could also write their Minoan Greek language
using a small corpus of syllabo-logographic signs
representing \textit{in-di-vi-du-al} syllables. The signs themselves and
their phonetic values – nearly all V (e) or CV (te) – were wholly
indigenous: what the rebus signs, all originating from the
Cretan world, depicted, one pronounced in Minoan Greek, not
in a Semitic language. (Minoan Greek appears to have been an
archaic sister tongue of the mainland’s Mycenæan Greek.\footnote{A History of Writing. })

Three separate but related forms of syllabo-logographic
writing emerged in the Ægean between c. 2000 and 1200 BC: the
Minoan Greeks’ ‘hieroglyphic’ script and Linear A, and the
later Mycenæan Greeks’ Linear B. Minoan Greeks apparently
also took their writing at an early date to Cyprus, where it experienced
two stages: Cypro-Minoan (evidently derived from
Linear A is one of two currently undeciphered writing systems used in ancient Greece. Cretan hieroglyphic is the other. Linear A was the primary script used in palace and religious writings of the Minoan civilization. It was discovered by archaeologist Arthur Evans. It is the origin of the Linear B script, which was later used by the Mycenaean civilization.

Linear A and its daughter Linear C, the ‘Cypriote Syllabic
Script’. All Ægean and Cypriote scripts are clearly syllabologographic,
as the objective identity of each rebus sign would
have been immediately recognizable to each learner and user. It
seems that determinatives were never employed in any of the
Ægean or Cypriote scripts; however, logograms additionally
depicted most spelt-out items on accounting tablets. All Ægean
and Cypriote scripts, but for these separate logograms, were
completely phonetic.
\medskip

\includegraphics[width=0.8\textwidth]{./images/cretan-hieroglyphs.png}

\medskip
Crete’s `hieroglyphic’ script is the patriarch of this robust
family, its inspiration perhaps derived from Byblos via Cyprus
around 2000 BC. As its name implies, this script used
pictorial signs to reproduce the syllabic inventory of the Minoan
Greek language, here used in rebus fashion as at Byblos. This
writing occurs on seal stones (and their clay impressions), baked
clay, and metal and stone objects, most of these discovered at
Knossos and dating from 2000– 1400 BC (the script was concurrent
with Linear A). There exist about 140 different signs in all –
that is, 70 to 80 syllabic signs and their alloglyphs (different signs
with the same sound value), as well as logograms: human figures,
parts of the body, flora, fauna, boats and geometrical shapes.
Writing direction was open: from left to right, from right to left,
with every other line reversed, even spiral. That this script also
included logograms and numerals suggests that it was initially
used for book-keeping, among other things, until its replacement
in this function with its simplification, Linear A. Thereafter, like
Anatolian hieroglyphs, the Cretan hieroglyphic script appears to
have assumed a ceremonial role in Minoan Greek society,
reserved for sacred inscriptions, dedications and royal proclamations
on round clay disks.

In the 1950s, Linear B was largely udeciphered and found to encode an early form of Greek. Although the two systems share many symbols, this did not lead to a subsequent decipherment of Linear A. Using the values associated with Linear B in Linear A mainly produces unintelligible words. If it uses the same or similar syllabic values as Linear B, then its underlying language appears unrelated to any known language. This has been dubbed the Minoan language.\footnote{\url{http://www.people.ku.edu/~jyounger/LinearA/LinAIdeograms/}}

\begin{scriptexample}[]{Linear A}
\unicodetable{lineara}{  
\number"10600,"10610,"10620,"10630,"10640,"10650,"10660,"10670,
"10680,"10690,"106A0,"106B0,"106C0,"106D0,"106E0,"106F0,"10710,"10720,"10730,"10740,"10750,"10760,"10770}
\end{scriptexample}

Many of the characters form group and specialists name them such as vases in transliterations.

\begin{scriptexample}[]{Vases}
\begin{center}
\scalebox{3}{{\lineara \char"106A6}}
\scalebox{3}{{\lineara \char"106A5}}
\scalebox{3}{{\lineara \char"106A7}}
\scalebox{3}{{\lineara \char"106A9}}
\end{center}
\end{scriptexample}

Linear A contains more than 90 signs (open vowels and consonants+vowels) in regular use and a host of
logograms, many of which are ligatured with syllabograms and/or fractions; about 80\% of these
logograms do not appear in Linear B. While many of Linear A’s signs are also found in Linear B, some
signs are unique to A (e.g., A *301 and following), while some signs found in Linear B are not yet found
in Linear A (e.g., B 12, 14-15, 18-19, 25, 32-33, 36, 42-43, 52, 62-64, 68, 71-72, 75, 83-84, 89-91).

The Unicode Linear A encoding is broadly based on the GORILA ([{\arial ɡɔɹɪˈlɑː}]) catalogue
(Godart and Olivier 1976–1985), which is the basic set of characters used in decipherment efforts.However, “ligatures” which consist of simple horizontal juxtapositions are not uniquely encoded here, as
these may be composed of their constituent parts. On the other hand, “ligatures” which consist of stacked
or touching elements have been encoded. 

\def\codex#1{\emph{Codex #1}\index{codex>#1}}
%\newfontfamily{\gothicfamily}{Noto Sans Gothic}
\newfontfamily{\gothicfamily}{code2001.ttf}
\section{Gothic}

\label{s:gothic}

\subsection{Introduction}

East Germanic Goths rose to prominence during the Great
Migrations of the fourth and fifth centuries AD 31 Their Gothic
languages are primarily known to us today through a few surviving
fragments of Bible translations. It was the Visigothic bishop
Wulfila († AD 383), according to three ecclesiastical historians
writing a century later, who created ‘Gothic letters’ in order to
translate the Bible into the Visigothic language. The fourth century
Greek alphabet was Wulfila’s only apparent source.

Though the bishop’s original Visigothic hand has not survived,
closely related derivative scripts preserved in two later Gothic
manuscripts no older than the sixth century have been preserved
(illus. 116).

‘Wulfila’s script’, as it perhaps should properly be designated,
is an alphabetic script written from left to right without word
separation. Spaces indicate sentences or passages, as does a
colon or a centred dot (as with the Iberian scripts). Nasal suspension
– that is, marking where an /m/ or /n/ should be – is
sometimes indicated by a macron (a topping stroke) above the
preceding letter. Ligatures are even rarer than macrons. There
are frequent contractions: for example, ius is often used to spell
‘Jesus’. Apart from rare profane relics – witness the sixth-century
Latin-Gothic Deed of Naples – Wulfila’s script, measured
by those few inscriptions that have survived, appears to have
conveyed exclusively ecclesiastical texts.

\begin{figure}[htb]
\includegraphics[width=.45\textwidth]{gothic}
\caption{Codex Carolinus}
\end{figure}

The Gothic script that Wulfila devised from the Greek
alphabet did not engender daughter scripts. After the sixth century
AD, it was replaced almost everywhere by related descendants
of Greek and Latin alphabets. Gothic’s last sentinel, the
ninth-century \codex{Vindobonensis} 795, was perhaps by then
only an antiquarian curiosity. The \emph{Codex Carolinus} preserves papal correspondence
with Frankish rulers, including letters exchanged by popes from Gregory III (731-741) to Hadrian I (772-795). the Codex was written in 791 on the orders of Charlemagne in order to rescue papyrus copies threatened with decay. It contains 99 letters, almost exclusively papal, and survives today in Vienna, \"Osterreichische Nationalbibliotek 449, in a copy probably made at Colone during the pontificate of Archbishop Willibert (870-889). The preface of the \codex{Carolinus} appears to refer to a second part that may have contained letters to byzantine rulers, now lost. Parallel copies of the Codex have not turned up. \citep{jasper2001papal}. 

\subsection{Unicode}

The Gothic alphabet was added to the Unicode Standard in March, 2001 with the release of version 3.1.

The Unicode block for Gothic is U+10330–U+1034F in the Supplementary Multilingual Plane. As older software that uses UCS-2 (the predecessor of UTF-16) assumes that all Unicode codepoints can be expressed as 16 bit numbers (U+FFFF or lower, the Basic Multilingual Plane), problems may be encountered using the Gothic alphabet Unicode range and others outside of the Basic Multilingual Plane.

\begin{scriptexample}[]{Gothic}
\unicodetable{gothicfamily}{"10330,"10340}
\end{scriptexample}
{\gothicfamily
𐍀	𐍁	𐍂	𐍃	𐍄	𐍅	𐍆	𐍇	𐍈	𐍉	𐍊}
%http://www.gotica.de/carolinus.html

%\begin{thebibliography}
%\bibitem[Fitzmyer(1995)]{fitzmyer}
%J.~A. Fitzmyer.
%\newblock \emph{The Aramaic inscriptions of Sefīre}, volume~19 of
%  \emph{Biblica et orientalia Sacra Scriptura antiquitatibus orientalibus
%  illustrata}.
%\newblock Pontificial Biblical Institute, Rome, 1995.
%\newblock URL
%  \url{http://web.archive.org/web/20051104215025/http://www.nelc.ucla.edu/Faculty/Schniedewind_files/NWSemitic/Aramaic_ABD.pdf}.
%\end{thebibliography}  











\newfontfamily\linearb{Aegean.ttf}
\section{Linear B}
\label{s:linearb}
\index{scripts>Linear B}
The Linear B script is a syllabic writing system that was used on the island of Crete and
parts of the nearby mainland to write the oldest recorded variety of the Greek language.

Linear B clay tablets predate Homeric Greek by some 700 years; the latest tablets date from
the mid- to late thirteenth century \bce. Major archaeological sites include Knossos, first
uncovered about 1900 by Sir Arthur Evans, and a major site near Pylos. The majority of
currently known inscriptions are inventories of commodities and accounting records.

The first tablets bearing the scripts were discovered by Sir Arthur Evans (1851-1941) while he was excavating the Minoan palace at Knossos in Crete. 


\medskip

\begin{figure}[ht]
\centering
\begin{minipage}{5cm}
\includegraphics[width=5cm]{./images/iklaina.jpg}
\end{minipage}\hspace{2em}
\begin{minipage}{7cm}
\captionof{figure}{Recently discovered fragment with Linear B, inscription. Found in an olive grove in what's now the village of Iklaina, the tablet was created by a Greek-speaking Mycenaean scribe between 1450 and 1350 B.C. (See \protect\href{http://news.nationalgeographic.com/news/2011/03/110330-oldest-writing-europe-tablet-greece-science-mycenae-greek/}{National Geographic}).}
\end{minipage}

\end{figure}


Early attempts to decipher the script failed until Michael Ventris, an architect and amateur
decipherer, came to the realization that the language might be Greek and not, as previously
thought, a completely unknown language. Ventris worked together with John Chadwick,
and decipherment proceeded quickly. The two published a joint paper in 1953. See \fullcite{ventrisa}.




Linear B was added to the Unicode Standard in April, 2003 with the release of version 4.0.

The Linear B Syllabary block is \unicodenumber{U+10000–U+1007F}. The Linear B Ideograms block is {\smallcps U+10080–U+100FF}. The Unicode block for the related Aegean Numbers is U+10100–U+1013F.

\begin{scriptexample}[]{Linear B}
\unicodetable{linearb}{"10000,"10010,"10020,"10030,"10040,"10050,"10060,"10070}

\captionof{table}{Linear B Typeset with command \protect\string\linearb\ and the \texttt{Aegean} font.}
\end{scriptexample}

\begin{scriptexample}[]{Linear B}
\unicodetable{linearb}{"10080,"10090,"100A0,"100B0,"100C0,"100D0,"100E0,"100F0}
\captionof{table}{Linear B Ideograms. Typeset with command \protect\string\linearb\ and the \texttt{Aegean} font.}
\end{scriptexample}


\begin{scriptexample}[]{Aegean Numbers}
\unicodetable{linearb}{"10100,"10110,"10110,"10120,"10130}

\captionof{table}{Aegean Numbers}
\end{scriptexample}





\section{Phaestos Disc}


One of the puzzles of Minoan Crete is the Phaestos disc. The Phaistos Disc was discovered in the Minoan palace-site of Phaistos, near Hagia Triada, on the south coast of Crete;[1] specifically the disc was found in the basement of room 8 in building 101 of a group of buildings to the northeast of the main palace. This grouping of four rooms also served as a formal entry into the palace complex. Italian archaeologist Luigi Pernier recovered the intact \enquote{dish}, about 15 cm (5.9 in) in diameter and uniformly slightly more than 1 centimetre (0.39 inches) in thickness, on 3 July 1908 during his excavation of the first Minoan palace.

It was found in the main cell of an underground \enquote{temple depository}. These basement cells, only accessible from above, were neatly covered with a layer of fine plaster. Their content was poor in precious artifacts, but rich in black earth and ashes, mixed with burnt bovine bones. In the northern part of the main cell, in the same black layer, a few inches south-east of the disc and about 20 inches (51 centimetres) above the floor, Linear A tablet PH 1 was also found. The site apparently collapsed as a result of an earthquake, possibly linked with the eruption of the Santorini volcano that affected large parts of the Mediterranean region during the mid second millennium B.C.

\begin{figure}[htp]
\centering

\includegraphics[width=0.67\textwidth]{./phaistosdiscs.jpg}
\caption{Phaistos discs.}
\end{figure}

The Phaistos Disc is generally accepted as authentic by archaeologists.[2] The assumption of authenticity is based on the excavation records by Luigi Pernier. This assumption is supported by the later discovery of the Arkalochori Axe with similar but not identical glyphs.[3]


The possibility that the disc is a 1908 forgery or hoax has been raised by two scholars.[4][5][6] In his 2008 review, Robinson does not endorse the forgery arguments, but argues that \enquote{a thermoluminescence test for the Phaistos Disc is imperative. It will either confirm that new finds are worth hunting for, or it will stop scholars from wasting their effort.}[4]

A gold signet ring from Knossos (the Mavro Spilio ring), found in 1926, contains a Linear A inscription developed in a field defined by a spiral—similar to the Phaistos Disc.\footnote{See University of Cologne website \url{http://arachne.uni-koeln.de/arachne/index.php?view[layout]=objekt_item\&search[constraints][objekt][searchSeriennummer]=159123}} A sealing found in 1955 shows the only known parallel to sign 21 (the \enquote{comb}) of the Phaistos disc.[9] This is considered as evidence that the Phaistos Disc is a genuine Minoan artifact.[10]

\begin{figure}[htbp]
\centering

\includegraphics[width=4.5cm]{crete-spiral-ring}\includegraphics[width=4.5cm]{crete-spiral-ring-01}\includegraphics[width=4.5cm]{crete-spiral-ring-02}

\caption{A gold signet ring from Knossos (the Mavro Spilio ring), found in 1926, contains a Linear A inscription developed in a field defined by a spiral—similar to the Phaistos Disc}
\end{figure}

The disc is made of fine clay.  Both side of the disc carry an inscription arranged in a spiral around the centre. The characters were impressed with a punch or stamp before the clay was fired. There are
241 or 242 characters (one is damaged), which
comprise 45 signs of variable frequency. For
comparison, there are thousands of characters in a few pages of printed English text, comprising the 26 signs we call letters. Lines partition
the disc’s characters into 31 short sections on
side A and 30 on side B, most of which contain
three, four or five characters. It is tempting to
speculate that these sections represent words
in the language of the disc.

That the characters were printed, not carved,
is beyond dispute. But no one knows why the disc’s maker bothered to produce a punch or stamp for each sign, rather than inscribing each character afresh. Egyptian hieroglyphs or Mesopotamian cuneiform of the second
millennium bce are inscribed on stone or clay;
simlarly the Minoan scripts Linear A and B found
at Phaistos, Knossos and other Cretan sites. If
the punch or stamp was to \enquote{print} many copies of documents, one would expect further sam-
ples to have turned up in a century of intensive Mediterranean excavatio

There is patchy and inconclusive evidence for and against the disc’s Cretan origin. The
signs look nothing like those of Linear A, Linear B or any other Minoan script, except coincidentally. This has led some, including Evans and Chadwick, to propose that the disc — and presumably its language, too — was an import.

One sign bears a remarkable resemblance to the architecture of rock tombs found in Anatolia in modern Turkey. One or two others
resemble signs found on a few contemporaneous objects from different sites in Crete. Most
scholars today, including Duhoux, think it a plausible working hypothesis that the disc was made in Crete. Gareth Owens and his Team claim to have read the disc and you can hear how it sounded at a TED Talk\footnote{\url{https://www.youtube.com/watch?v=6Chcplx3tZ8}}.



\subsection{Signs}

There are 242 tokens on the disc, comprising 45 distinct signs. Many of these 45 signs represent easily identifiable every-day things. In addition to these, there is a small diagonal line that occurs underneath the final sign in a group a total of 18 times. The disc shows traces of corrections made by the scribe in several places. The 45 symbols were numbered by Arthur Evans from 01 to 45, and this numbering has become the conventional reference used by most researchers. Some symbols have been compared with Linear A characters by Nahm,[17] Timm,[3] and others. Other scholars (J. Best, S. Davis) have pointed to similar resemblances with the Anatolian hieroglyphs, or with Egyptian hieroglyphs (A. Cuny). In the table below, the character "names" as given by Louis Godart (1995) are given in upper case; where other description or elaboration applies, they are given in lower case.




\PrintUnicodeBlock{./languages/phaistos.txt}{\linearb}




The ideograms are symbols, not pictures of the objects in question, e.g. one tablet records a tripod with missing legs, but the ideogram used is of a tripod with three legs. In modern transcriptions of Linear B tablets, it is typically convenient to represent an ideogram by its Latin or English name or by an abbreviation of the Latin name. Ventris and Chadwick generally used English; Bennett, Latin. Neither the English nor the Latin can be relied upon as an accurate name of the object; in fact, the identification of some of the more obscure objects is a matter of exegesis.

\begingroup

\linearb

Vessels
\let\l\unicodenumber

\begin{tabular}{l>{\smallcps}l>{\smallcps}l>{\smallcps}l>{\smallcps}l}
𐃟	&U+100DF	&200	&\l{sartāgo}	&\l{Boiling Pan}\\
𐃠	&U+100E0	&201	&\l{tripūs}	&\l{Tripod Cauldron}\\
𐃡	&U+100E1	&202	&\l{pōculum}	&\l{Goblet}\\
𐃢	&U+100E2	&203	&\l{urceus}	&\l{Wine Jar?}\\
𐃣	&U+100E3	&204  &\l{Tahirnea}	&\l{Ewer}\\
𐃤	&U+100E4	&205  &\l{Tnhirnula}	&\l{Jug}\\
𐃥	&U+100E5	&206	&\l{hydria}	&Hydria\\
𐃦	&U+100E6	&207	&\l{TRIPOD}  &AMPHORA\\
𐃧	&\l{U+100E7}	&\l{208}	&\l{PAT patera}	&\l{BOWL}\\
𐃨	&U+100E8	&209	&AMPH amphora	&AMPHORA\\
𐃩	&U+100E9	&210	&STIRRIP &JAR\\
𐃪	&U+100EA	&211	&WATER &BOWL?\\
𐃫	&U+100EB	&212	&SIT situla	&WATER JAR?\\
𐃬	&U+100EC	&213	&LANX lanx	&COOKING BOWL\\
\end{tabular}




\subsection{Online Resources}

Corpora and GORILA \url{http://www.people.ku.edu/~jyounger/LinearA/\#3}



\endgroup











\section{Cypriot Syllabary}
\label{s:cypriot}
The Cypriot or Cypriote syllabary is a syllabic script used in Iron Age Cyprus, from ca. the 11th to the 4th centuries BCE, when it was replaced by the Greek alphabet. A pioneer of that change was king Evagoras of Salamis. It is descended from the Cypro-Minoan syllabary, in turn a variant or derivative of Linear A. Most texts using the script are in the Arcadocypriot dialect of Greek, but some bilingual (Greek and Eteocypriot) inscriptions were found in Amathus.

\begin{figure}[htb]
\centering
\begin{minipage}{7cm}
\includegraphics[width=7cm]{./images/idalion-tablet.jpg}
\end{minipage}\hspace{1.5em}
\begin{minipage}{6cm}
\captionof{figure}{The bronze Idalion Tablet, from Idalium, (Greek: Ιδάλιον), is from the 5th century BCE Cyprus. The tablet is inscribed on both sides.
The script of the tablet is in the Cypro-Minoan syllabary, and the inscription is in Greek. The tablet records a contract between "the king and the city":[1] the topic of the tablet rewards a family of physicians, of the city, for providing free health services to individuals fighting an invading force of Persians.}
\end{minipage}
\end{figure}


The characters are \textit{syllabic}. There is one character for each  vowel, \textit{a, e, i, o, u,} and perhaps one for \textit{o}. There is no distinction between long and short vowels. The other characters represent what are called \textit{open syllables}\footnote{ If a syllable ends with a consonant, it is called a closed syllable. If a syllable ends with a vowel, it is called an open syllable. }, i.e., beginning with a consonant and ending with a vowel. 

No distinction is made between smooth, middle and rough mutes. The same character stands for τά τ\’ασs, δα in Εδαλιον ανδ δα ιν Αθανα  κε, κη, γε, γη, χε, χη. This fact constitutes the greatest difficulty in reading Cypriote.  

The Cypriot syllabary was added to the Unicode Standard in April, 2003 with the release of version 4.0.
The Unicode block for Cypriot is \unicodenumber{U+10800–U+1083F}. The Unicode block for the related Aegean Numbers is \unicodenumber{U+10100–U+1013F}.

\newfontfamily\cypriote{Aegean.ttf}

\begin{scriptexample}[]{Cypriot Syllabary}
\unicodetable{cypriote}{"10800,"10810,"10820,"10830}

\cypriote \symbol{"10803}
\end{scriptexample}


\printunicodeblock{./languages/cyprus.txt}{\cypriote}

\section{Old Italic}

\epigraph{A society grows great when old men plant
trees in whose shade they know they will never sit.}{Greek proverb}
\label{s:olditalic}
\index{scripts>Old Italic}
\newfontfamily\olditalic{Noto Sans Old Italic}


Old Italic refers to any of several now extinct alphabet systems used on the Italian Peninsula in ancient times for various Indo-European languages (predominantly Italic) and non-Indo-European (e.g. Etruscan) languages. The alphabets derive from the Euboean Greek Cumaean alphabet, used at Ischia and Cumae in the Bay of Naples in the eighth century BC.

Various Indo-European languages belonging to the Italic branch (Faliscan and members of the Sabellian group, including Oscan, Umbrian, and South Picene, and other Indo-European branches such as Celtic, Venetic and Messapic) originally used the alphabet. Faliscan, Oscan, Umbrian, North Picene, and South Picene all derive from an Etruscan form of the alphabet.

\section{Etruscan}

Many peoples took the system that the Greeks had elaborated and
adapted it to their own language. This was particularly true in Lemnos and
in Etruria, where signs inspired by Greek letters were put to the service of
languages that probably were closely related to Greek—signs that we can
read without fully comprehending them. The Etruscans seem to have used
writing largely for religious purposes. According to Cicero (De divinatione)
they bequeathed their sacred texts to the Romans, who held the Etruscan
religion to be the religion of the Book par excellence.\cite{henri1994}

\begin{figure}[htbp]
\centering
\includegraphics[width=0.7\textwidth]{marsiliana}
\caption{The Marsiliana Tablet}
\end{figure}

The Germanic runic alphabet was derived from one of these alphabets by the 2nd century.


Old Italic is a Unicode block containing a unified repertoire of the three stylistic variants of pre-Roman Italic scripts.

\begin{scriptexample}[]{Testing}
\unicodetable{olditalic}{"10300,"10310,"10320}

{\leavevmode
\hfill\hfill\hfill\footnotesize Typeset with \texttt{Noto Sans Old Italic~}
}
\end{scriptexample}
\section{Old South Arabian}
\label{s:oldsoutharabian}

\index{Ancient and Historic Scripts>Old South Arabian}
\index{Old South Arabian fonts>Noto Sans Old South Arabian}
\index{alphabets>Yemeni}

\newfontfamily\oldsoutharabian{NotoSansOldSouthArabian-Regular.ttf}

The ancient Yemeni alphabet (Old South Arabian ms3nd; modern Arabic: {\arabicfont المُسنَد‎}  musnad) branched from the Proto-Sinaitic alphabet in about the 9th century BC. It was used for writing the Old South Arabian languages of the Sabaic, Qatabanic, Hadramautic, Minaic (or Madhabic), Himyaritic, and proto-Ge'ez (or proto-Ethiosemitic) in Dʿmt. The earliest inscriptions in the alphabet date to the 9th century BC in Akkele Guzay, Eritrea[3] and in the 10th century BC in Yemen. There are no vowels, instead using the \emph{mater lectionis} to mark them.

Its mature form was reached around 500 BC, and its use continued until the 6th century AD, including Old North Arabian inscriptions in variants of the alphabet, when it was displaced by the Arabic alphabet.[4] In Ethiopia and Eritrea it evolved later into the Ge'ez alphabet,[1][2] which, with added symbols throughout the centuries, has been used to write Amharic, Tigrinya and Tigre, as well as other languages (including various Semitic, Cushitic, and Nilo-Saharan languages).

It is usually written from right to left but can also be written from left to right. When written from left to right the characters are flipped horizontally (see the photo).
The spacing or separation between words is done with a vertical bar mark (\textbar).
Letters in words are not connected together.

Old South Arabian script does not implement any diacritical marks (dots, etc.), differing in this respect from the modern Arabic alphabet.

\begin{scriptexample}[]{South Arabian}
\unicodetable{oldsoutharabian}{"10A60,"10A70}
\end{scriptexample}

Support in \latexe is provided via Peter Wilson's package \pkgname{sarabian}\citep{sarabian}. The package provides all the |metafont| sources as well as transliteration commands and other utilities \seedocs{\SARAB}. The package is based on fonts developed originally by Alan Stanier of Essex University.

The package provides the commands \docAuxCmd{sarabfamily} that selects the South Arabian font family. The family name is \texttt{sarab}. Another command \docAuxCmd{textsarab}\meta{text} typesets \meta{text} in the South Arabian font. The package provides two ways of accessing
glyphs: (a) by \texttt{ASCII} character commands, and (b) via commands. These are illustrated in
Table~\ref{sarabian1} which is a modified version of that provided in the Comprehensive Symbols.



\def\SAtdu{\oldsoutharabian\char"10A77}

A comparison between  the unicode and the rendering (scaled 5) \pkgname{sarabian} is shown below.

\centerline{\scalebox{3}{\SAtdu} \scalebox{3}{\textsarab{\SAtd}}}

There is no real advantage in using unicode fonts, if all you interested is to write some South Arabian text for inscriptions. 

\begin{symtable}[SARAB]{\SARAB\ South Arabian Letters}
\index{South Arabian alphabet}
\index{alphabets>South Arabian}
\label{sarabian1}
\begin{tabular}{*4{ll@{\qquad}}ll}
\K[\textsarab{\SAa}]\SAa   & \K[\textsarab{\SAz}]\SAz   & \K[\textsarab{\SAm}]\SAm   & \K[\textsarab{\SAsd}]\SAsd & \K[\textsarab{\SAdb}]\SAdb \\
\K[\textsarab{\SAb}]\SAb   & \K[\textsarab{\SAhd}]\SAhd & \K[\textsarab{\SAn}]\SAn   & \K[\textsarab{\SAq}]\SAq   & \K[\textsarab{\SAtb}]\SAtb \\
\K[\textsarab{\SAg}]\SAg   & \K[\textsarab{\SAtd}]\SAtd & \K[\textsarab{\SAs}]\SAs   & \K[\textsarab{\SAr}]\SAr   & \K[\textsarab{\SAga}]\SAga \\
\K[\textsarab{\SAd}]\SAd   & \K[\textsarab{\SAy}]\SAy   & \K[\textsarab{\SAf}]\SAf   & \K[\textsarab{\SAsv}]\SAsv & \K[\textsarab{\SAzd}]\SAzd \\
\K[\textsarab{\SAh}]\SAh   & \K[\textsarab{\SAk}]\SAk   & \K[\textsarab{\SAlq}]\SAlq & \K[\textsarab{\SAt}]\SAt   & \K[\textsarab{\SAsa}]\SAsa \\
\K[\textsarab{\SAw}]\SAw   & \K[\textsarab{\SAl}]\SAl   & \K[\textsarab{\SAo}]\SAo   & \K[\textsarab{\SAhu}]\SAhu & \K[\textsarab{\SAdd}]\SAdd \\
\end{tabular}

\bigskip
\begin{tablenote}
  \usefontcmdmessage{\textsarab}{\sarabfamily}.  Single-character
  shortcuts are also supported: Both
  ``\verb+\textsarab{\SAb\SAk\SAn}+'' and ``\verb+\textsarab{bkn}+''
  produce ``\textsarab{bkn}'', for example.  \seedocs{\SARAB}.
\end{tablenote}
\end{symtable}



\section{Avestan script}
\label{s:avestan}
The Avestan alphabet is a writing system developed during Iran's Sassanid era (AD 226–651) to render the Avestan language.
As a side effect of its development, the script was also used for Pazend, a method of writing Middle Persian that was used primarily for the Zend commentaries on the texts of the Avesta. In the texts of Zoroastrian tradition, the alphabet is referred to as \emph{din dabireh} or \emph{din dabiri}, Middle Persian for "the religion's script".

The Avestan alphabet was replaced by the Arabic alphabet after Persia converted to Islam during the 7th century CE. 


Notable Features

The alphabet is written from right to left, in the same way as Syriac, Arabic and Hebrew.
See more at: \url{http://www.iranchamber.com/scripts/avestan_alphabet.php#sthash.ZRu7AkEb.dpuf}

\newfontfamily\avestan{NotoSansAvestan-Regular.ttf}



\begin{scriptexample}[]{Avestan}
\ifxetex\TeXXeTstate=1
\beginR\fi
\avestan\raggedleft
𐬄	
𐬅	
𐬆	
𐬇	
𐬈	
𐬉	
𐬊	
𐬋	
𐬌	
𐬍	
𐬎	
𐬏	
𐬐	
	
𐬒	
𐬓	
𐬔	
	
𐬖	
𐬗	
𐬘	
𐬙	
𐬚	
𐬛	
𐬜	
𐬝	
𐬞	
𐬟	
𐬠	
𐬡	
𐬢	
𐬣	
𐬤	
𐬥	
𐬦	
𐬧	
𐬨	
𐬩	
𐬪	
𐬫	
𐬬	
𐬭	
𐬮	
𐬯	
𐬰	
𐬱	
𐬲	
𐬳	
𐬴	
𐬵	
\ifxetex\endR
\TeXXeTstate=0\fi
\end{scriptexample}

The recent Google font \url{NotoSansAvestan-Regular_0.ttf} can be used to typeset the Avestan script, but really not suitable for any serious study of the language.
\subsection{Old Turkic}

\newfontfamily\oldturkic{Segoe UI Symbol}
\begin{scriptexample}[]{Old Turkish}
\oldturkic
\obeylines
Orkhon	Yenisei
variants	Transliteration / transcription
Old Turkic letter  𐰀	𐰁 𐰂	a, ä
Old Turkic letter  𐰃	𐰄 𐰅	y, i (e)
Old Turkic letter  𐰆		o, u
Old Turkic letter  𐰇	𐰈	ö, ü

	0	1	2	3	4	5	6	7	8	9	A	B	C	D	E	F
U+10C0x	𐰀	𐰁	𐰂	𐰃	𐰄	𐰅	𐰆	𐰇	𐰈	𐰉	𐰊	𐰋	𐰌	𐰍	𐰎	𐰏
U+10C1x	𐰐	𐰑	𐰒	𐰓	𐰔	𐰕	𐰖	𐰗	𐰘	𐰙	𐰚	𐰛	𐰜	𐰝	𐰞	𐰟
U+10C2x	𐰠	𐰡	𐰢	𐰣	𐰤	𐰥	𐰦	𐰧	𐰨	𐰩	𐰪	𐰫	𐰬	𐰭	𐰮	𐰯
U+10C3x	𐰰	𐰱	𐰲	𐰳	𐰴	𐰵	𐰶	𐰷	𐰸	𐰹	𐰺	𐰻	𐰼	𐰽	𐰾	𐰿
U+10C4x	𐱀	𐱁	𐱂	𐱃	𐱄	𐱅	𐱆	𐱇	𐱈	

\hfill  Typeset with \texttt{Segoe UI Symbol} \cmd{\oldturkic} 
\end{scriptexample}

Irk Bitig or Irq Bitig (Old Turkic: {\bfseries\Large\oldturkic 𐰃𐰺𐰴 𐰋𐰃𐱅𐰃𐰏}), known as the Book of Omens or Book of Divination in English, is a 9th-century manuscript book on divination that was discovered in the "Library Cave" of the Mogao Caves in Dunhuang, China, by Aurel Stein in 1907, and is now in the collection of the British Library in London, England. The book is written in Old Turkic using the Old Turkic script (also known as "Orkhon" or "Turkic runes"); it is the only known complete manuscript text written in the Old Turkic script.[1] It is also an important source for early Turkic mythology.

The Old Turkic text does not have any sentence punctuation, but uses two black lines in a red circle as a word separation mark in order to indicate word boundaries as shown in Figure~{\ref{omen}}

\begin{figure}[htb]
\includegraphics[width=0.7\textwidth]{./images/omen.jpg}
\caption{Omen 11 (4-4-3 dice) of the Irk Bitig (folio 13a): "There comes a messenger on a yellow horse (and) an envoy on a dark brown horse, bringing good tidings, it says. Know thus: (The omen) is extremely good."}
\label{omen}
\end{figure}
\section{Runic}
\label{s:runic}
\newfontfamily\runic{NotoSansRunic-Regular.ttf}

Runes (Proto-Norse:{\runic ᚱᚢᚾᛟ }(runo), Old Norse: rún) are the letters in a set of related alphabets known as runic alphabets, which were used to write various Germanic languages before the adoption of the Latin alphabet and for specialised purposes thereafter. The Scandinavian variants are also known as futhark or fuþark (derived from their first six letters of the alphabet: F, U, Þ, A, R, and K); the Anglo-Saxon variant is futhorc or fuþorc (due to sound changes undergone in Old English by the names of those six letters)

\begin{scriptexample}[]{Runic}
 \unicodetable{runic}{"16A0,"16B0,"16C0,"16D0,"16E0,"16F0}
\end{scriptexample}


\printunicodeblock{./languages/runic.txt}{\runic}

