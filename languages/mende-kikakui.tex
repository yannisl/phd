\cxset{offsety = 0pt,
       image   = mende-dance,
       texti   = \lorem,
       textii  = \lorem }   
       
\chapter{Mende Kikakui}

\newfontfamily\kikakui{Kikakui Sans Pro}

The Mende Kikakui script is a syllabary used for writing the Mende language of Sierra Leone.

\begin{figure}[htbp]
\parindent=0pt
\centering

\includegraphics[width=\textwidth]{mende-women}

\caption{Ethnic hairdress styles. From the book: Hair in African Art and Culture
Some of the numerous styles in which the upper Mende women dressed their hair.}
\end{figure}

Mende Kikakui script was added to the Unicode Standard in June 2014 with the release of version 7.0.
The Unicode block for Mende Kikakui is U+1E800–U+1E8DF:


\bgroup
\kikakui


𞠀	𞠁	𞠂	𞠃	𞠄	𞠅	𞠆	𞠇	𞠈	𞠉	𞠊	𞠋	𞠌	𞠍	𞠎	𞠏
U+1E81x	𞠐	𞠑	𞠒	𞠓	𞠔	𞠕	𞠖	𞠗	𞠘	𞠙	𞠚	𞠛	𞠜	𞠝	𞠞	𞠟

\egroup


\begin{scriptexample}[]{Bamum}
\unicodetable{kikakui}{"1E800,"1E810,"1E820,"1E830,"1E840,"1E850,"1E860,"1E870,"1E880,"1E890,"1E8A0,"1E8B0,"1E8C0,"1E8D0}
\end{scriptexample}