\newfontfamily\sikkim{Tibetan Machine Uni}
\newfontfamily\lepcha{Mingzat-R.ttf}
\section{Lepcha}
\label{s:lepcha}
\epigraph{``Had your independence ensured mine, I surely would have greeted you on this moment every year ...''}{
August 15, 2015\\
Chewang Pintso\\
General Secretary, SIBLAC}

\label{s:lepcha}
\index{Scripts>Lepcha}



The Lepcha are also called the Rongkup meaning the children of God and the Rong, Mútuncí Róngkup Rumkup (Lepcha:{\lepcha ᰕᰫ་ᰊᰪᰰ་ᰆᰧᰶ ᰛᰩᰵ་ᰀᰪᰱ ᰛᰪᰮ་ᰀᰪᰱ}; "beloved children of the Róng and of God"), and Rongpa (Sikkimese:{\sikkim རོང་པ་}), are among the indigenous peoples of Sikkim and number between 30,000 and 50,000. Many Lepcha are also found in western and southwestern Bhutan, Tibet, Darjeeling, the Mechi Zone of eastern Nepal, and in the hills of West Bengal. The Lepcha people are composed of four main distinct communities: the Renjóngmú of Sikkim; the Támsángmú of Kalimpong, Kurseong, and Mirik; the ʔilámmú of Ilam District, Nepal; and the Promú of Samtse and Chukha in southwestern Bhutan.[3][2][4]\index{Languages>Lepcha}
\index{Nepal Languages>Lepcha}\index{Bhutan Languages>Lepcha} The Lepcha probably do not exceed 50,000 and hence their language is on the \textsc{UNESCO} endangered list of languages.

\begin{figure}[htbp]
\centering
\includegraphics[width=\linewidth-2\parindent]{lepchas}

\caption{Lepcha manuscript}

\end{figure}

The Lepcha have their own language, also called Lepcha. It belongs to the Bodish–Himalayish group of Tibeto-Burman languages. The Lepcha write their language in their own script, called Róng or Lepcha script, which is derived from the Tibetan script. It was developed between the 17th and 18th centuries, possibly by a Lepcha scholar named Thikúng Mensalóng, during the reign of the third Chogyal (Tibetan king) of Sikkim.[7] The world's largest collection of old Lepcha manuscripts is found with the Himalayan Languages Project in Leiden, Netherlands, with over 180 Lepcha books.

The Lepcha script, or Róng script is an abugida used by the Lepcha people to write the Lepcha language. Unusually for an abugida, syllable-final consonants are written as diacritics.

The United Nations Educational, Scientific and Cultural Organization (UNESCO) lists Lepcha as an endangered language with the following characterization:

The Lepcha language is spoken in Sikkim and Darjeeling district in West Bengal of India. The 1991 Indian census counted 39,342 speakers of Lepcha. Lepcha is considered to be one of the indigenous languages of the area in which it is spoken. Unlike most other languages of the Himalayas, the Lepcha people have their own indigenous script (the world's largest collection of old Lepcha manuscripts is kept in Leiden, with over 180 Lepcha books).

Lepcha is the language of instruction in some schools in Sikkim. In comparison to other Tibeto-Burman languages, it has been given considerable attention in the literature. Nevertheless, many important aspects of the Lepcha language and culture still remain undescribed. 


\begin{figure}[htbp]
\centering
\includegraphics[width=\linewidth-2\parindent]{lepcha}

\caption{A manuscript of the Van Manen Collection at the Kern Institute of Leiden University.}
\end{figure}

{\lepcha
Consonants bear the inherent vowel, but no virama is used to kill this vowel; vowel matras modify it, and
explicit final consonants are used where there is no inherent vowel. Initial vowels are represented with
the vowel matras on the neutral letter £ A. Initial consonants can be followed by the glides ˇ§ -YA and
ˇ• -RA, both of which normally ligate with the consonant they modify; these can also combine to form
ˇˆ -rya, which is simply a glyph ligature of the other two: ÄÙ kya, Äı kra, Ĉ krya. The glide -la is also
found, but is represented not by a ligating combining mark, but by a limited set of letters containing this
glide inherently. With few exceptions, these “combined” letters do not look like a ligature of their base
letters with some mark: Ä ka Å kla, É ga Ñ gla, é pa è pla, ë fa í fla, ì ba î bla, ï ma ñ mla,
ù ha û hla.}

The Mingzat font is still under development by SIL so I am not too sure if the rendering is correct\footnote{\url{http://scripts.sil.org/cms/scripts/page.php?site_id=nrsi&id=Mingzat}}.


\section{Unicode}

Lepcha script was added to the Unicode Standard in April, 2008 with the release of version 5.1.
The Unicode block for Lepcha is U+1C00–U+1C4F:
\begin{scriptexample}[]{Lepcha}
\bgroup
\lepcha
\obeylines
 	    0	1	2	3	4	5	6	7	8	9	A	B	C	D	E	F
U+1C0x	 ᰀ	ᰁ	ᰂ	ᰃ	ᰄ	ᰅ	ᰆ	ᰇ	ᰈ	ᰉ	ᰊ	ᰋ	ᰌ	ᰍ	ᰎ	ᰏ
U+1C1x	 ᰐ	ᰑ	ᰒ	ᰓ	ᰔ	ᰕ	ᰖ	ᰗ	ᰘ	ᰙ	ᰚ	ᰛ	ᰜ	ᰝ	ᰞ	ᰟ
U+1C2x	 ᰠ	ᰡ	ᰢ	ᰣ	ᰤ	ᰥ	ᰦ	ᰧ	ᰨ	ᰩ	ᰪ	ᰫ	ᰬ	ᰭ	ᰮ	ᰯ
U+1C3x	 ᰰ	ᰱ	ᰲ	ᰳ	ᰴ	ᰵ	ᰶ	᰷	x	x	x	᰻	᰼	᰽	᰾	᰿
U+1C4x	 ᱀	᱁	᱂	᱃	᱄	᱅	᱆	᱇	᱈	᱉	x	x	x	ᱍ	ᱎ	ᱏ

\egroup
\end{scriptexample}


