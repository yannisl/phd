\section{Sora Sompeng}
\label{s:sorasompeng}
The Sora Language is part of the Austroasiatic language family. More locally, however, it is a part of the \hyperref[s:munda]{Munda} Languages which include other tribal languages in close proximity to Sora. Sora is unique because although it is surrounded by the Indo-Aryan language \hyperref[s:oriya]{Oriya} and the Dravidian Language Telugu, Sora is more closely related to the languages of Southeast Asia such as Khmer in Cambodia than it is to the predominant languages of India. Moreover, Sora contains very little formal literature but has an abundance of folk tales and traditions. Most of their passed down knowledge is of the oral tradition. Compared to other languages in the Munda family, Sora is decreasing within the Sora tribe at a faster rate. Most speakers are concentrated in Odisha and Andhra Pradesh but smaller communities also exist in Madhya Pradesh, Tamil Nadu, and Bihar.

Sorang Sompeng script is used to write in Sora, a Munda language with 300,000 speakers in India. The script was created by Mangei Gomango in 1936 and is used in religious contexts.[1] He was familiar with Oriya, Telugu and English, so the parent systems of the script are Brahmi and Latin.[2]
The Sora language is also written in the Latin alphabet and the Telugu script.

Sorang Sompeng script was added to the Unicode Standard in January, 2012 with the release of version 6.1. In Windows Nirmala UI.ttf (Windows 10.0) can be used. 

\newfontfamily\NirmalaU{Nirmala UI}


\unicodetable{NirmalaU}{"110D0,"110E0,"110F0}
 	

The Sora Bible employs a Latin-based orthography with a number of Sora-specific
conventions. Sora has also been rendered in the Oriya script in Orissa and in
the Telugu script in Andhra Pradesh, as well as a modified phonetic alphabet in
Ramamurti’s grammatical materials and dictionary. The use of and knowledge of
Sorang Sompeng (N. Zide 1996), the indigenous script, appears to be quite limited.
In many areas Sora remains a vital and thriving language, but one that has no state
or institutional support (sermons and materials are increasingly in Oriya in Gajapati
district which we observed and were told about in Christianized Sora communities).
In other areas, Sora is reportedly being or indeed has already been replaced by Telugu
or Oriya. So, although not an endangered language in sensu stricto, Sora (except in


\section{Numerals}\label{s:munda}

The Sora are unique in their numeral system. Instead of base 10, Sora uses a base 12 system. Only a few other languages in the world share this anomaly. Ekari, for example, uses a base 60 system.[7] For example, 39 in Sora arithmetic would be thought of as (1 * 20)+ 12 + 7. Here are the first 12 numerals in the Sora language :[7]
English: one two three four five six seven eight nine ten eleven twelve

Sora: aboy bago yagi unji monloy tudru gulji thamji tinji gelji gelmuy migel
Similar to how English uses the suffix from the numeral ten after twelve (such as thirteen, fourteen, etc.), Sora also uses a suffix assignment to numerals after 12 and before 20. Thirteen in Sora is expressed as migelboy (12+1), fourteen as migelbagu (12+2), etc.[7] Between numerals 20 and 99, Sora adds the suffix kuri to the first constituent of the numeral. For example, 31 is expressed as bokuri gelmuy and 90 as unjikuri gelji.[7]
