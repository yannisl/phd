\chapter{Middle Eastern Scripts}

The scripts in this section have a common origin in the ancient Phoenician alphabet. They include:

\begin{center}
\begin{tabular}{ll}
Hebrew & Samaritan\\
Arabic & Thaana\\
Syriac &\\
\end{tabular}
\end{center}

The Hebrew script is used in Israel and for languages of the Diaspora. The Arabic script is
used to write many languages throughout the Middle East, North Africa, and certain parts
of Asia. The Syriac script is used to write a number of Middle Eastern languages. These
three also function as major liturgical scripts, used worldwide by various religious groups.

The Samaritan script is used in small communities in Israel and the Palestinian Territories
to write the Samaritan Hebrew and Samaritan Aramaic languages. The Thaana script is
used to write Dhivehi, the language of the Republic of Maldives, an island nation in the
middle of the Indian Ocean. 

Text in these scripts is written from right to left. Arabic and Syriac are cursive scripts even when typeset, unlike Hebrew, Samaritan  and Thaana, where letters are unconnected. Most letters in Arabic and Syriac assume different forms depending on their position in a word. Shaping rules are not required for Hebrew because only five letters have position-dependent forms, and these forms are separately encoded.

Historically, Middle Eastern  scripts did not write short vowels. In modern scripts they are represented  by marks positioned above or below a consonantal letter. Vowels and other
marks of pronunciation (“vocalization”) are encoded as combining characters, so support
for vocalized text necessitates use of composed character sequences. Yiddish, Syriac, and
Thaana are normally written with vocalization; Hebrew, Samaritan, and Arabic are usually written unvocalized. 

\section{Hebrew}
\newfontfamily\hebrew{Miriam}
\fontspec{Arial Unicode MS}
To properly typeset Hebrew texts you first need to choose an appropriate font and also set the directionality of the text. This
is done using the etex commands:

\CMDI{\beginL} and \CMDI{\beginR} 

For \XeTeX\ you also need to add near the top of your document |\TeXXeTstate=1|. The package \pkgname{bidi} can be used to set all parameters. Be warned that it redefines almost all of \latexe's commands, so for short mixed texts, I wouldn't recommend its usage. 



The Hebrew alphabet (Hebrew: אָלֶף־בֵּית עִבְרִי[a], alefbet ʿIvri ), known variously by scholars as the Jewish script, square script, block script, is used in the writing of the Hebrew language, as well as other Jewish languages, most notably Yiddish, Ladino, and Judeo-Arabic. There have been two script forms in use; the original old Hebrew script is known as the paleo-Hebrew script (which has been largely preserved, in an altered form, in the Samaritan script), while the present "square" form of the Hebrew alphabet is a stylized form of the Assyrian script. Various "styles" (in current terms, "fonts") of representation of the letters exist. There is also a cursive Hebrew script, which has also varied over time and place. On Windows you can use the \texttt{Miriam} font or \texttt{Arial Unicode MS} or \texttt{Miriam Fixed}.
\medskip

\topline

\bgroup\TeXXeTstate=1
\raggedleft\hebrew{}\beginR

הכתב הכנעני הקדום הלך והתפשט וסימניו היו מוכרים כל כך, עד כי המשתמשים בו התחילו "להתעצל" בהשלמת הציורים, והניחו כי הקורא יבין גם מתוך שרטוטים סכמתיים באיזו אות מדובר. כך, למשל, הפך הראש למשולש עם צוואר; כף היד מלאת האצבעות הפכה לשרטוט דל, ומהדג נותר רק הזנב. כשהעברים אמצו את הכתב הכנעני הם התקשו לזהות חלק מהציורים המקוריים והניחו למשל כי הסימן המתאר את המילה "זהה" הוא כלי נשק; שזנב הדג המשולש הוא דלת, ושדווקא הנחש הוא דג. כך נולדו שמותיהם העבריים של האותיות זי"ן, דל"ת ונו"ן (נון הוא דג, כמו אמנון, שפמנון וכו'). הציורים שהפכו לסימנים התגלגלו לכתבים נוספים, ואפילו ליוונית וללטינית. גם בכתב העברי המודרני ניתן לזהות המשך התפתחותי ברור מן הכתב הכנעני הקדום, והשתמרות שמות האותיות מקלה מאוד על פענוח המקור.


בתקופת בית שני, אומץ האלפבית הארמי לשימוש השפה העברית במקום האלפבית העברי העתיק, כאשר בזה האחרון נעשה שימוש מועט כגון כתיבת השמות הקדושים והטבעת מטבעות. עם הזמן, נעלם גם שימוש זה של הכתב העתיק. האלפבית העברי של ימינו הוא אפוא פיתוח של האלפבית הארמי ולא של הכתב העברי העתיק.	
{}

 לֹ֥א תִשָּׂ֛א

\endR


\egroup
\bottomline
\medskip

To make all paragraphs  RL use the \cmd{\everypar}\footnote{See discussions at \url{http://tex.stackexchange.com/questions/141867/minimal-bidi-for-typesetting-rl-text} and \url{http://www.tug.org/pipermail/xetex/2004-August/000697.html}}. 

\begin{verbatim}
\newbox\mybox \everypar{\setbox\mybox\lastbox\beginR\box\mybox}
\everypar={% at the start of each paragraph, do....
    \setbox0=\lastbox % save the paragraph indent, if any
    \beginR % set R-L direction
    \box0 % then re-insert the indent
	}
\end{verbatim}

The Hebrew alphabet has 22 letters, of which five have different forms when used at the end of a word. Hebrew is written from right to left. Originally, the alphabet was an abjad consisting only of consonants. Like other \textit{abjads}, such as the Arabic alphabet, means were later devised to indicate vowels by separate vowel points, known in Hebrew as niqqud. In rabbinic Hebrew, the letters א ה ו י are also used as matres lectionis to represent vowels. When used to write Yiddish, the writing system is a true alphabet (except for borrowed Hebrew words). In modern usage of the alphabet, as in the case of Yiddish (except that ע replaces ה) and to some extent modern Israeli Hebrew, vowels may be indicated. Today, the trend is toward full spelling with these letters acting as true vowels.


\subsection{Syriac}

\newfontfamily\syriac{Estrangelo Edessa}

Syriac /ˈsɪriæk/ ({\syriac{ܠܫܢܐ ܣܘܪܝܝܐ}} Leššānā Suryāyā) is a dialect of Middle Aramaic that was once spoken across much of the Fertile Crescent and Eastern Arabia.[1][2][5] Having first appeared as a script in the 1st century AD after being spoken as an unwritten language for five centuries,[6] Classical Syriac became a major literary language throughout the Middle East from the 4th to the 8th centuries,[7] the classical language of Edessa, preserved in a large body of Syriac literature.
It became the vehicle of Syriac Christianity and culture, spreading throughout Asia as far as the Indian Malabar Coast and Eastern China,[8] and was the medium of communication and cultural dissemination for Arabs and, to a lesser extent, Persians. Primarily a Christian medium of expression, Syriac had a fundamental cultural and literary influence on the development of Arabic,[9] which largely replaced it towards the 14th century.[3] Syriac remains the liturgical language of Syriac Christianity.
Syriac is a Middle Aramaic language, and, as such, it is a language of the Northwestern branch of the Semitic family. It is written in the Syriac alphabet, a derivation of the Aramaic alphabet.

\begin{scriptexample}[]{Syriac}
\unicodetable{syriac}{"0700,"0710,"0720,"0730,"0740}
\end{scriptexample}

The Syriac Abbreviation (a type of overline) can be represented with a special control character called the Syriac Abbreviation Mark (U+070F {\syriac \char"070F ܘ}).

\section{Samaritan}
\newfontfamily\samaritan{NotoSansSamaritan-Regular.ttf}

The Samaritan alphabet is used by the Samaritans for religious writings, including the Samaritan Pentateuch, writings in Samaritan Hebrew, and for commentaries and translations in Samaritan Aramaic and occasionally Arabic.

The Samaritans are, consider themselves to be the descendants of the Northern Tribes of Israel that were not sent into Assyrian captivity, and have continuously resided in the land of Israel.

The Torah Scroll of the Samaritans uses an alphabet that is very different from the one used on Jewish Torah Scrolls. According to the Samaritans themselves and Hebrew scholars, this alphabet is the original "Old Hebrew" alphabet.

Even as far back as 1691, this connection between the Samaritan and the "Old" Hebrew alphabets was made by Henry Dodwell; "[the Samaritans] still preserve [the Pentateuch] in the Old Hebrew characters."

Samaritan is a direct descendant of the Paleo-Hebrew alphabet, which was a variety of the Phoenician alphabet in which large parts of the Hebrew Bible were originally penned. All these scripts are believed to be descendants of the Proto-Sinaitic script. That script was used by the ancient Israelites, both Jews and Samaritans. The better-known "square script" Hebrew alphabet traditionally used by Jews is a stylized version of the Aramaic alphabet which they adopted from the Persian Empire (which in turn adopted it from the Arameans). 

After the fall of the Persian Empire, Judaism used both scripts before settling on the Aramaic form. For a limited time thereafter, the use of paleo-Hebrew (proto-Samaritan) among Jews was retained only to write the Tetragrammaton, but soon that custom was also abandoned.



ShofarRegular StamAshkenazCLM.ttf

\begin{scriptexample}[]{Samaritan}
\bgroup
\TeXXeTstate=1
\unicodetable{samaritan}{"0800,"0810,"0820,"0830}
\egroup
\TeXXeTstate=0
\end{scriptexample}

I battled to get an appropriate font for the Samaritan script and had to use the \idxfont{Noto Sans Samaritan} from Google


^^A\printunicodeblock{./languages/samaritan.txt}{\samaritan}


\url{http://www.ancient-hebrew.org/ahh/ahh.htm#_Toc314842274}




\section{Arabic}

\newfontfamily\arabian{Scheherazade-R.ttf}

The Arabic script is a writing system used for writing several languages of Asia and Africa, such as Arabic, Sorani and Luri Dialects of Kurdish language, Persian, Pashto and Urdu.[1] Even until the 16th century, it was used to write some texts in Spanish.[2] After the Latin script, Chinese characters, and Devanagari, it is the fourth-most widely used writing system in the world.[3]
The Arabic script is written from right to left in a cursive style. In most cases the letters transcribe consonants, or consonants and a few vowels, so most Arabic alphabets are abjads.

The script was first used to write texts in Arabic, most notably the Qurʼān, the holy book of Islam. With the spread of Islam, it came to be used to write languages of many language families, leading to the addition of new letters and other symbols, with some versions, such as Kurdish, Uyghur, and old Bosnian being abugidas or true alphabets. It is also the basis for a rich tradition of Arabic calligraphy.

\begin{verbatim}
\begin{Arabic}
ّ هو إذ الغاية؛ شريف الفوائد، جم المذهب، عزيز فنّ التاريخ فنّ أنّ اعلم
والملوك سيرهم، في والأنبياء أخلاقهم، في الأمم من الماضين أحوال على يوقفنا
ّ أحوال في يرومه لمن ذلك في الإقتداء فائدة تتم حتّى وسياستهم؛ دولهم في
والدنيا. الدين
\end{Arabic}
\end{verbatim}




As of Unicode 7.0, the Arabic script is contained in the following blocks:
Arabic (0600—06FF, 255 characters)
Arabic Supplement (0750—077F, 48 characters)
Arabic Extended-A (08A0—08FF, 39 characters)
Arabic Presentation Forms-A (FB50—FDFF, 608 characters)
Arabic Presentation Forms-B (FE70—FEFF, 140 characters)
Rumi Numeral Symbols (10E60—10E7F, 31 characters)
Arabic Mathematical Alphabetic Symbols (1EE00—1EEFF, 143 characters)[1][2]

The basic Arabic range encodes the standard letters and diacritics, but does not encode contextual forms (U+0621–U+0652 being directly based on ISO 8859-6); and also includes the most common diacritics and Arabic-Indic digits. The Arabic Supplement range encodes letter variants mostly used for writing African (non-Arabic) languages. The Arabic Extended-A range encodes additional Qur'anic annotations and letter variants used for various non-Arabic languages. The Arabic Presentation Forms-A range encodes contextual forms and ligatures of letter variants needed for Persian, Urdu, Sindhi and Central Asian languages. The Arabic Presentation Forms-B range encodes spacing forms of Arabic diacritics, and more contextual letter forms. The presentation forms are present only for compatibility with older standards, and are not currently needed for coding text.[3] 

The Arabic Mathematical Alphabetical Symbols block encodes characters used in Arabic mathematical expressions.

\begin{multicols}{3}
\printunicodeblock{./languages/arabic.txt}{\arabian}
\end{multicols}









\section{Thaana}

\newfontfamily\thaana{MV Boli}
Thaana, Taana or Tāna ({\thaana  ތާނަ}‎ in Tāna script) is the modern writing system of the Maldivian language spoken in the Maldives. Thaana has characteristics of both an abugida (diacritic, vowel-killer strokes) and a true alphabet (all vowels are written), with consonants derived from indigenous and Arabic numerals, and vowels derived from the vowel diacritics of the Arabic abjad. Its orthography is largely phonemic.

The Thaana script first appeared in a Maldivian document towards the beginning of the 18th century in a crude initial form known as Gabulhi Thaana which was written scripta continua. This early script slowly developed, its characters slanting 45 degrees, becoming more graceful and spaces were added between words. 

As time went by it gradually replaced the older Dhives Akuru alphabet. The oldest written sample of the Thaana script is found in the island of Kanditheemu in Northern Miladhunmadulu Atoll. It is inscribed on the door posts of the main Hukuru Miskiy (Friday mosque) of the island and dates back to 1008 AH (AD 1599) and 1020 AH (AD 1611) when the roof of the building were built and the renewed during the reigns of Ibrahim Kalaafaan (Sultan Ibrahim III) and Hussain Faamuladeyri Kilege (Sultan Hussain II) respectively.

\begin{scriptexample}[]{Thaana}
\unicodetable{thaana}{"0780,"0790,"07A0,"07B0}

\hfill Typeset with MV Boli and the command \cmd{\thaana}.
\end{scriptexample}


^^A\printunicodeblock{./languages/thaana.txt}{\thaana}



\endinput










