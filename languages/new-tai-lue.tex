\section{New Tai Lue Script}
\label{s:newtailue}
\newfontfamily\tailue{Noto Sans New Tai Lue}


New Tai Lue script, also known as Simplified Tai Lue, is an alphabet used to write the Tai Lü language. Developed in China in the 1950s, New Tai Lue is based on the traditional Tai Le alphabet developed ca. 1200 AD. The government of China promoted the alphabet for use as a replacement for the older script; teaching the script was not mandatory, however, and as a result many are illiterate in New Thai Lue. 

\begin{figure}[htbp]
\centering

\includegraphics[width=\linewidth-2\parindent]{tailue}

\caption{Tai Le costumes. (pininterest)}
\end{figure}

In addition, communities in Burma, Laos, Thailand and Vietnam still use the Tai Le alphabet. There are probably less than one million native speakers of the language who can be found in China, Burma, Laos, Thailand and Vietnam.

\begin{figure}[htbp]
\centering

\includegraphics[width=\linewidth-2\parindent]{tai-lu}

\caption{Tai Le costumes. (pininterest)}
\end{figure}

\begin{scriptexample}[]{Tai Lue}
{\centering\tailue \LARGE

ᦒ	ᦓ	ᦔ	ᦕ	ᦖ	ᦗ	ᦘ	ᦙ	ᦚ	ᦛ	ᦜ	ᦝ	ᦞ	

}
\end{scriptexample}

The New Tai Lue script was added to the Unicode Standard in March, 2005 with the release of version 4.1.

The Unicode block for New Tai Lue is |U+1980|–|U+19DF|:

\begin{scriptexample}[]{New Tai Lue}
\unicodetable{tailue}{"1980,"1990,"19A0,"19B0,"19C0,"19D0}

\texttt{typeset using NotoSansNewTaiLue-Regular.ttf.}
\end{scriptexample}