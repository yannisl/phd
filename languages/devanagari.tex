\section{Devanagari}
\label{s:devanagari}
\parindent1em

Devanagari is part of the Brahmic family of scripts of India, Nepal, Tibet, and South-East Asia.[2] It is a descendant of the Gupta script, along with Siddham and Sharada.[2] Eastern variants of Gupta called nāgarī are first attested from the 7th century CE; from c. 1200 CE these gradually replaced Siddham, which survived as a vehicle for Tantric Buddhism in East Asia, and Sharada, which remained in parallel use in Kashmir. An early version of Devanagari is visible in the Kutila inscription of Bareilly dated to Vikram Samvat 1049 (i.e. 992 CE), which demonstrates the emergence of the horizontal bar to group letters belonging to a word.[3]

Sanskrit nāgarī is the feminine of nāgara \enquote{relating or belonging to a town or city}. It is feminine from its original phrasing with lipi ("script") as nāgarī lipi "script relating to a city", that is, probably from its having originated in some city.[4]

The use of the name devanāgarī is relatively recent, and the older term nāgarī is still common.[2] The rapid spread of the term Devanāgarī may be related to the almost exclusive use of this script to publish Sanskrit texts in print since the 1870s.[2]

In time, Devanagari became India’s principal script. It also
became one of the world’s most important, as it was used to
convey many other languages of the region, such as Hindi, Nepali, Marwari, 
Kumaoni and several non-Indo-Aryan
languages. Devanagari failed to become India’s sole script perhaps
because of the region’s long disunity. Subsequently, it
became the parent of, among other scripts, the Gurmukhi
which the Sikhs elaborated in the 1500s in order to write their
Punjabi language (illus. 78). Today, Devanagari survives in India
alongside some ten other major scripts (including the Latin and
Perso-Arabic alphabets) and about 190 others of lesser significance.\cite{writing}

On Windows use \texttt{Arial Unicode MS} or \texttt{Arial}
\medskip

%\newfontfamily\devanagari[Script=Devanagari,Scale=1.5]{Arial Unicode MS}
\newfontfamily\devanagarilohit[Script=Devanagari,Scale=1.1]{Lohit-Devanagari.ttf}
\let\devanagari\devanagarilohit

\begin{scriptexample}[]{Devanagari}
{\begin{center}\parindent0pt\devanagari

ंःअआइईउऊऋऌऍऎएऐऑऒओऔऔँ \par 

ी	ु	ू	ृ	ॄ	ॅ	ॆ	े	ै	ॉ	ॊ	ो	ौ	्	\par

\bigskip		
\begin{tabular}{lll lll lll l}
०	&१	&२	&३	&४	&५	&६	&७	&८	&९\\
0	&1	&2	&3	&4	&5	&6	&7	&8	&9\\
\end{tabular}
\end{center}	
}
\end{scriptexample}


On Linux \texttt{Lohit} is a font family designed to cover Indic scripts and released by Red Hat. The Lohit fonts currently cover 11 languages: Assamese, Bengali, Gujarati, Hindi, Kannada, Malayalam, Marathi, Oriya, Punjabi, Tamil, Telugu.[1] The fonts were supplied by Modular Infotech and licensed under the GPL. In September 2011, they were retroactively relicensed under the OFL.[2] The Lohit fonts are used as web fonts by some Wikimedia Foundation sites, like Wikipedia, since March 2012.The font currently support 21 Indian languages. 

\let\devanagarilohit\pan

\begin{scriptexample}[]{Devanagari}
\begin{center}\parindent0pt\devanagarilohit

ंःअआइईउऊऋऌऍऎएऐऑऒओऔऔँ \par 

ी	ु	ू	ृ	ॄ	ॅ	ॆ	े	ै	ॉ	ॊ	ो	ौ	्	\par

\bigskip		
\begin{tabular}{lll lll lll l}
०	&१	&२	&३	&४	&५	&६	&७	&८	&९\\
0	&1	&2	&3	&4	&5	&6	&7	&8	&9\\
\end{tabular}
\end{center}
\end{scriptexample}

\subsubsection{Punctuation} 
The end of a sentence or half-verse may be marked with a dot known as a pūrna virām or a vertical line danda: \textbar. The end of a full verse may be marked with two vertical lines: \textbar\textbar. A comma, or alpa virām, is used to denote a natural pause in speech. With expansion of English speakers in India, the full stop is also sometimes used.

\subsection{LaTeX support}

\latex2e support can be found in the \pkgname{sanskrit}. The package contains the font files and pre-processor for printing Sanskrit
text in both devanāgarī and transliterated Roman with diacritics. Another package that can be used with \XeTeX\ is support \pkgname{devnag}.  This was originally developed by Frans Velthuis for the University of Groningen, The Netherlands, and it was the first system to provide
support for the script for \tex. The package was  extended by Anshuman Pandey. The package provides both fonts as well as tranliteration macros.


\printunicodeblock{./languages/devanagari.txt}{\devanagarilohit}





