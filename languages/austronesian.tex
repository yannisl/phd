\cxset{image = philippines}

\chapter{Austronesian}

\epigraph{Five new pieces of debris that could belong to the missing Malaysia Airlines flight \textsc{MH370} have been found in Madagascar.}{bbc}

The Austronesian language family is the largest and most widespread in the
world, with somewhere around 700 (maybe as many as 1,200) languages\footfullcite{adelaar2005}
altogether and 300 million native speakers.\footnote{Ethnologue lists 1,268
Austronesian languages, making it the second largest language family in the world.}
Aside from Southeast Asia, Austronesian languages are found on numerous islands in the eastern and
central Pacific Ocean all the way to Easter Island. There is also a western
outpost language (Malagasy), spoken on the island of Madagascar. The news of the missing Malaysia Airlines debri found in
Madagasgar, gives credence to theories that Austronesian had a foothold by canoes from Indonesia getting lost in a storm and ending up there.\footnote{Cover image shows The Ifugao tribe, one of many mountain tribes in Luzon, has an impressive
farming practice that uses terraces of irrigated fields to grow rice. Members
of the tribe, in ceremonial clothes, stand overlooking the terraced fields
lining a valley in Banaue, Ifugao Province, North Luzon. From Tammy Mildenstein et al. \textit{The Philippines}}

In Southeast Asia the major languages in this family are Indonesian
(Bahasa Indonesia) and Malaysian (Bahasa Melayu), collectively referred
to as Malay (200 million speakers, about 40 million as a first language),
Javanese (75 million speakers), Sundanese (30 million), and Filipino
(Tagalog) (50 million, 17 million as a first language). There are hundreds of
other, closely related languages in Malaysia, Indonesia, and the Philippines.
All the native languages of these countries are Austronesian except for the
Aslian languages of interior Malaysia. There are also some Austronesian
pockets in southern Vietnam and Cambodia, such as the Chamic languages
(Thomas 1998; Grant and Sidwell Forthcoming). Estimates of the number
of Austronesian languages vary a lot, mainly because of difficulties in
drawing the \enquote{language/dialect} distinction.

The Austronesian languages of insular and peninsular Southeast Asia all
belong to the western division of Austronesian. This is usually regarded
as consisting of two further groups, the Central Malayo-Polynesian branch
(in the Moluccas and Lesser Sunda islands, about 100–150 languages) and
the much larger Western Malayo-Polynesian branch (which includes Malay
and the Philippine languages). Another group of about ten Austronesian
languages, in a distantly related subgroup, is located on the island of Formosa
(Taiwan). These languages have about 400,000 speakers.

Table \ref{tbl:mainlanguages} from Robert Blust's book \textit{The Austronesian languages}\footfullcite{robert2009} lists the national languages covered by AN.


\captionof{table}{National/official languages in states with Austronesian-speaking majorities}
\label{tbl:mainlanguages}
\begin{longtable}{p{0.8cm}lrrl}
\toprule
No. &Nation &Area (km$^2$) &Population &Language\\
\midrule
01. &Republic of Indonesia &1,919,440 &248,645,008 &Bahasa Indonesia\\
02. &Republic of the Philippines &300,000 &103,775,002 &Filipino/English\\
03. &Federation of Malaysia &329,750 &29,179,952 &Bahasa Malaysia\\
04. &Malagasy Republic &587,040 &22,005,222 &Malagasy/French\\
05. &Singapore &693 &6,310,129 &Bahasa Melayu\\
06. &Papua New Guinea &462,840 &5,353,494 &Tok Pisin\\
07. &Timor-Leste &15,007 &1,143,667 &Tetun/Portuguese\\
08. &Fiji &18,274 &890,057 &Fijian/English\\
09. &Solomon Islands &28,450 &584,578 &Pijin\\
10. &Brunei Darussalam &5,770 &408,786 &Bahasa Kebangsaan\\
11. &Vanuatu &12,200 &256,155 &Bislama\\
12. &Samoa &2,944 &194,320 &Samoan\\
13. &FSM &702 &106,487 &English\\
14. &Kingdom of Tonga &748 &106,146 &Tongan\\
15. &Kiribati &811 &101,998 &Gilbertese/English\\
16. &Marshall Islands &181 &68,480 &Marshallese/English\\
17. &Republic of Palau &458 &21,032 &Palauan/English\\
18. &Cook Islands &230 &10,777 &Rarotongan/English\\
19. &Tuvalu &26 &10,619 &Tuvaluan\\
20. &Republic of Nauru &21 &9,378 &Nauruan\\
\bottomrule
\end{longtable}

% something went wrong here! 
\parindent1em

Austronesian languages tend to have phoneme inventories which are
medium to small in size. They have abundant morphology,
using affixes of all kinds (prefixes, suffixes, infixes, circumfixes) and
reduplication to build up complex words from simpler ones (see section
3.2.5). All of these languages have an inclusive/exclusive distinction in plural
pronouns, i.e. there are two words for ‘we’—one including the addressee
and one excluding the addressee. This is a feature shared by no other language
family in Southeast Asia or East Asia. 


\begin{figure}[htbp]
\parindent0pt
\centering
\includegraphics[width=\textwidth]{austronesian-map}

\caption[Distribution of Austronesian languages in Southeast Asia]{Distribution of Austronesian languages in Southeast Asia (after \fullcite{comrie2003}).}
\end{figure}


On the western side of insular
Southeast Asia most Austronesian languages have intricate systems of voice
or focus marking (see section 4.4), and many are verb-initial, including Toba
Batak, \hyperref[s:tagalog]{Tagalog}, Chamorro, and the Formosan languages. In the centre and
eastern side of the region, closer to Papua, voice systems are either much
more restricted or lacking altogether, verb-medial constituent order is
common, and subjects and objects are often expressed by pronominal clitics
(Klamer 2000; Grimes 2000).





\section{Balinese}

\epigraph{In Bali the gods are thought of as the \textit{children} of the people, not as august parental figures. Speaking through the lips of those in trance, the gods address the villages as ``papa''  and ```mama'', and the people are said to spoil or indulge their gods\ldots}{Gregory Bateson and Margaret Mead in \textit{Balinese Character: A Photographic Analysis, 1942}}
\label{s:balinese}\index{Balinese}\index{Aksara Bali}\index{Bali}\index{Lombok}

\newfontfamily{\balinese}{AksaraBali.ttf}

Balinese or simply Bali\footnote{Not to be confused with the Nigerian or Papua New Guinea languages also named Bali.} is a Malayo-Polynesian language spoken by 3.3 million people (as of 2000) on the Indonesian island of Bali, as well as northern Nusa Penida, western Lombok and eastern Java.[3] Most Balinese speakers also know Indonesian. Balinese itself is not mutually intelligible with Indonesian, but may be understood by Javanese speakers after some exposure.

In 2011, the Bali Cultural Agency estimates that the number of people still using Balinese language in their daily lives on the Bali Island does not exceed 1 million, as in urban areas their parents only introduce Indonesian language or even English, while daily conversations in the institutions and the mass media have disappeared. The written form of the Balinese language is increasingly unfamiliar and most Balinese people use the Balinese language only as a spoken tool with mixing of Indonesian language in their daily conversation. But in the transmigration areas outside Bali Island, Balinese language is extensively used and believed to play an important role in the survival of the language.[4]

\begin{figure}[htbp]
\centering

\includegraphics[width=\textwidth]{bali-cock.jpg}
\end{figure}

The higher registers of the language borrow extensively from Javanese: an old form of classical Javanese, Kawi, is used in Bali as a religious and ceremonial language.

\paragraph{The Balinese script} is natively known as Aksara Bali and Hanacaraka, is an abugida used in the island of Bali, Indonesia, commonly for writing the Austronesian Balinese language, Old Javanese, and the liturgical language Sanskrit. With some modifications, the script is also used to write the Sasak language, used in the neighboring island of Lombok.[1] 

The script is a descendant of the Brahmi script, and so has many similarities with the modern scripts of South and Southeast Asia. The Balinese script, along with the Javanese script, is considered the most elaborate and ornate among Brahmic scripts of Southeast Asia.[2]

\includegraphics[width=\textwidth]{bali}


Though everyday use of the script has largely been supplanted by the Latin alphabet, the Balinese script has significant prevalence in many of the island's traditional ceremonies and is strongly associated with the Hindu religion. The script is mainly used today for copying lontar or palm leaf manuscripts containing religious texts.[2][3]



{\indicative ◌ }

\newcounter{under}
\setcounter{under}{"1B00}

\def\cb#1 {
\hspace*{2.5pt}
 
 $\text{◌#1}_{\pgfmathparse{Hex(\theunder)}\text{\pgfmathresult}}$
\stepcounter{under}
\vskip5pt\par
}
\begin{scriptexample}[]{Balinese}


\balinese
	 
᭐	᭑	᭒	᭓	᭔	᭕	᭖	᭗	᭘	᭙	᭚	᭛	᭜	᭝	᭞	᭟\\\
 
\def\columnseprulecolor{\color{thegray}}
\columnseprule.4pt
\begin{multicols}{8}

\texttt{U+1B0x}	

\cb{ᬀ }  \cb{ ᬁ } 	\cb{ ᬂ }  	\cb ᬃ	\cb ᬄ 	\cb ᬅ	\cb ᬆ	\cb ᬇ	\cb ᬈ	\cb ᬉ	\cb ᬊ	\cb ᬋ	\cb ᬌ	\cb ᬍ	\cb ᬎ	\cb ᬏ

\columnbreak

\texttt{U+1B1x}	 

\cb ᬐ	 \cb ᬑ 	\cb ᬒ 	\cb ᬓ	\cb ᬔ	\cb ᬕ	\cb ᬖ \cb ᬗ 	\cb ᬘ 	\cb ᬙ 	\cb ᬚ	\cb ᬛ 	\cb ᬜ 	\cb ᬝ 	\cb ᬞ	\cb ᬟ 

\columnbreak

U+1B2x	 

\cb ᬠ◌ 	\cb ᬡ	\cb ᬢ	\cb ᬣ	\cb ᬤ	\cb ᬥ	\cb ᬦ	\cb ᬧ	\cb ᬨ	\cb ᬩ	\cb ᬪ	\cb ᬫ	\cb ᬬ	\cb ᬭ	\cb ᬮ	\cb ᬯ

\columnbreak
U+1B3x 

\cb ᬰ	\cb ᬱ	\cb ᬲ	\cb ᬳ	\cb ᬴	\cb ᬵ	\cb ᬶ	\cb ᬷ	\cb ᬸ	\cb ᬹ	\cb ᬺ	\cb ᬻ	\cb ᬼ	\cb ᬽ	\cb ᬾ	\cb ᬿ


\columnbreak
U+1B4x	 

\cb ᭀ	 \cb ᭁ	\cb ᭂ	\cb ᭃ	\cb ᭄	\cb ᭅ	\cb ᭆ	\cb ᭇ	\cb ᭈ	\cb ᭉ	\cb ᭊ	\cb ᭋ

\columnbreak				
U+1B5x	 

\cb ᭐	\cb ᭑	\cb ᭒	\cb ᭓	\cb ᭔	\cb ᭕	\cb ᭖	\cb ᭗	\cb ᭘	\cb ᭙	\cb ᭚	\cb ᭛	\cb ᭜	\cb ᭝	\cb ᭞	\cb ᭟\\

\columnbreak

U+1B6x 

\cb ᭠	\cb ᭡	\cb ᭢	\cb ᭣	\cb ᭤	\cb ᭥	\cb ᭦	\cb ᭧	\cb ᭨◌ 	\cb ᭩◌ 	\cb ᭪◌ 	\cb ᭫	\cb ᭬	\cb ᭭	\cb ᭮	\cb ᭯

\columnbreak
U+1B7x	 

\cb ᭰	 \cb ᭱  \cb ᭲  \cb ᭳	 \cb ᭴	\cb ᭵	\cb ᭶	\cb ᭷	\cb ᭸	\cb ᭹	\cb ᭺	\cb ᭻	\cb ᭼


\end{multicols}

\end{scriptexample}


One of the most comprehensive fonts is Aksara Bali\footnote{\url{http://www.alanwood.net/downloads/index.html}}. This is obtainable at Alan Wood's website.

\newfontfamily\javanese{Noto Sans Javanese}

%\newfontfamily\javanese{TuladhaJejeg_gr.ttf}

\section{Javanese}
\label{s:javanese}
\index{scripts>Javanese}


The Javanese (Ngoko Javanese: {\javanese ꦮꦺꦴꦁꦗꦮ},[3] Madya Javanese: {\javanese\   ꦠꦶꦪꦁꦗꦮꦶ},[4] Krama Javanese: ꦥꦿꦶꦪꦤ꧀ꦠꦸꦤ꧀ꦗꦮꦶ,[4] Ngoko Gêdrìk: wòng Jåwå, Madya Gêdrìk: tiyang Jawi, Krama Gêdrìk: priyantun Jawi, Indonesian: suku Jawa)[5] are an ethnic group native to the Indonesian island of Java. With approximately 100 million people (as of 2011), they form the largest ethnic group in Indonesia. They are predominantly located in the central to eastern parts of the island. There are also significant numbers of people of Javanese descent in most provinces of Indonesia, Malaysia, Singapore, Suriname, Saudi Arabia and the Netherlands.

The Javanese ethnic group has many sub-groups, such as the Mataram, Cirebonese, Osing, Tenggerese, Samin, Naganese, Banyumasan, etc.[6]

A majority of the Javanese people identify themselves as Muslims, with a minority identifying as Christians and Hindus. However, Javanese civilization has been influenced by more than a millennium of interactions between the native animism Kejawen and the Indian Hindu—Buddhist culture, and this influence is still visible in Javanese history, culture, traditions, and art forms. With a sizeable global population, the Javanese are considered significant as they are the fourth largest ethnic group among Muslims, in the world, after the Arabs,[7] Bengalis[8] and Punjabis.[9]


\paragraph{Javanese} is one of the Austronesian languages, but it is not particularly close to other languages and is difficult to classify. Its closest relatives are the neighbouring languages such as Sundanese, Madurese and Balinese. Most speakers of Javanese also speak Indonesian, the standardized form of Malay spoken in Indonesia, for official and commercial purposes as well as a means to communicate with non-Javanese-speaking Indonesians.

There are speakers of Javanese in Malaysia (concentrated in the states of Selangor and Johor) and Singapore. Some people of Javanese descent in Suriname (the Dutch colony of Suriname until 1975) speak a creole descendant of the language.

\begin{figure}[htbp]
\includegraphics[width=\textwidth]{javanese-people}
\end{figure}

The language is spoken in Yogyakarta, Central and East Java, as well as on the north coast of West Java. It is also spoken elsewhere by the Javanese people in other provinces of Indonesia, which are numerous due to the government-sanctioned transmigration program in the late 20th century, including Lampung, Jambi, and North Sumatra provinces. In Suriname, creolized Javanese is spoken among descendants of plantation migrants brought by the Dutch during the 19th century. In Madura, Bali, Lombok, and the Sunda region of West Java, it is also used as a literary language. It was the court language in Palembang, South Sumatra, until the palace was sacked by the Dutch in the late 18th century.

Javanese is written with the Latin script, Javanese script, and Arabic script.[5] In the present day, the Latin script dominates writings, although the Javanese script is still taught as part of the compulsory Javanese language subject in elementary up to high school levels in Yogyakarta, Central and East Java.

Javanese is the tenth largest language by native speakers and the largest language without official status. It is spoken or understood by approximately 100 million people. At least 45\% of the total population of Indonesia are of Javanese descent or live in an area where Javanese is the dominant language. All seven Indonesian presidents since 1945 have been of Javanese descent.[6] It is therefore not surprising that Javanese has had a deep influence on the development of Indonesian, the national language of Indonesia.

There are three main dialects of the modern language: Central Javanese, Eastern Javanese, and Western Javanese. These three dialects form a dialect continuum from northern Banten in the extreme west of Java to Banyuwangi Regency in the eastern corner of the island. All Javanese dialects are more or less mutually intelligible.


\paragraph{The Javanese script} (Hanacaraka/Carakan) is a script for writing the Javanese language, the native language of one of the peoples of the Island of Java. It is a descendent of the ancient Brahmi script of India, and so has many similarities with modern scripts of South Asia and Southeast Asia. The Javanese script is also used for writing Sanskrit, Old Javanese, and transcriptions of Kawi, as well as the Sundanese language, and the Sasak language.

\begin{figure}[htbp]
\hspace*{-1.5cm}\includegraphics[width=1.2\textwidth]{java-palm-leave-manuscript}
\end{figure}





\begin{scriptexample}[]{Javanese}
\bgroup
\javanese

꧋ꦱꦧꦼꦤ꧀ꦮꦺꦴꦁꦏꦭꦲꦶꦂꦲꦏꦺꦏꦤ꧀ꦛꦶꦩꦂꦢꦶꦏꦭꦤ꧀ꦢꦂꦧꦺꦩꦂꦠꦧꦠ꧀ꦭꦤ꧀ꦲꦏ꧀ꦲꦏ꧀ꦏꦁꦥꦝ꧉

꧋ ꦲꦮꦶꦠ꧀ꦲꦶꦏꦁꦄꦱ꧀ꦩꦄꦭ꧀ꦭꦃ꧈ ꦏꦁꦩꦲꦩꦸꦫꦃꦠꦸꦂ ꦩꦲꦲꦱꦶꦃ꧉ 	 
 ۝꧋ ꦄꦭꦶꦥꦃ꧀ ꦭ ꦩ꧀ ꦫ ꧌ ꦏꦁ — — ꦥꦿꦶꦏ꧀ꦱ ꦏꦉꦪꦥ꧀ꦥꦩꦸꦁꦄꦭ꧀ꦭꦃꦥꦶꦪꦺꦩ꧀ꦧꦏ꧀ ꧌꧉ ꦩꦁꦪꦏꦴꦪꦤꦴ ꦲꦶꦏꦸꦄꦪꦺꦪꦠꦴꦏꦶꦠꦧ꧀ꦑꦸꦂꦄꦤ꧀ꦏꦁꦥꦿꦪꦠꦭ꧉ 	 
᭐	᭑	᭒	᭓	᭔	᭕	᭖	᭗	᭘	᭙	᭚	᭛	᭜	᭝	᭞	᭟
 
\egroup
\end{scriptexample}


The Javanese script was added to Unicode Standard in version 5.2 on the code points \texttt{A980 - A9DF}. There are 91 code points for Javanese script: 53 letters, 19 punctuation marks, 10 numbers, and 9 vowels:
\medskip

\unicodetable{javanese}{"A980,"A990,"A9A0, "A9B0, "A9C0,"A9D0}

\medskip



As of the writing of this document (2017), there are several widely published fonts able to support Javanese, ANSI-based Hanacaraka/Pallawa by Teguh Budi Sayoga,[21] Adjisaka by Sudarto HS/Ki Demang Sokowanten,[22] JG Aksara Jawa by Jason Glavy,[23] Carakan Anyar by Pavkar Dukunov,[24] and Tuladha Jejeg by R.S. Wihananto,[25] which is based on Graphite (SIL) smart font technology. Other fonts with limited publishing includes Surakarta made by Matthew Arciniega in 1992 for Mac's screen font,[26] and Tjarakan developed by AGFA Monotype around 2000.[27] There is also a symbol-based font called Aturra developed by Aditya Bayu in 2012–2013.[28]

Due to the script's complexity, many Javanese fonts have different input method compared to other Indic scripts and may exhibit several flaws. \docFont{JG Aksara Jawa}, in particular, may cause conflicts with other writing system, as the font use code points from other writing systems to complement Javanese's extensive repertoire. This is to be expected, as the font was made before Javanese implementation in Unicode.[29]

Arguably, the most "complete" font, in terms of technicality and glyph count, is \docFont{TuladhaJejeg}. It comes with keyboard facilities, displaying complex syllable structure, and support extensive glyph repertoire including non-standard forms which may not be found in regular Javanese texts, by utilizing Graphite (SIL) smart font technology. |Tuladha Jejeg| uses variable stroke widths on its glyphs with serifs on some glyphs\footnote{\protect\url{https://sites.google.com/site/jawaunicode/main-page}}.

However, as not many writing systems require such complex feature, use is limited to programs with Graphite technology, such as Firefox browser, Thunderbird email client, and several OpenType word processor and of course XeLaTeX. The font was chosen for displaying Javanese script in the Javanese Wikipedia.[16]

\paragraph{jawaTeX} Jawa\TeX{} project is initial effort to make Javanese characters typesetting program using \TeX{}/\LaTeX{}. This project is aimed to make Javanese widely used. The main project is developing transliteration models to transliterate Latin document into Javanese document. Perl and \TeX{}/\LaTeX{} are use in this project, the program are develop to run in text mode (console) both Linux and Windows but not limit on it. Web based program also developed, and automatic embedded Javanese characters in HTML See \href{http://jawatex.org/jawa/jawatex}{jawatex}.



\section{Sundanese}
\epigraph{The married women, when their husband die, must, as point of honour, die with them, and if they should be afraid of death they put into the convents.}{Tomé Pires \textit{Suma Oriental} (1512–1515)}

\label{s:sundanese}

\newfontfamily\sundanese{Noto Sans Sundanese}
The Sundanese (Sundanese: {\sundanese ᮅᮛᮀ ᮞᮥᮔ᮪ᮓ}, Urang Sunda) are an ethnic group native to the western part of the Indonesian island of Java. They number approximately 40 million, and are the second most populous of all the nation's ethnicities. The Sundanese are predominantly Muslim. In their own language, Sundanese, the group is referred to as Urang Sunda, and Orang Sunda or Suku Sunda in the national language, Indonesian.

The Sundanese have traditionally been concentrated in the provinces of West Java, Banten, Jakarta, and the western part of Central Java. Sundanese migrants can also be found in Lampung and South Sumatra. The provinces of Central Java and East Java are home to the Javanese, Indonesia's largest ethnic group.

\begin{figure}[htbp]
\centering
\includegraphics[width=\linewidth-2\parindent]{sundanese}

\caption{Sundanese boys playing Angklung in Garut, c. 1910–1930. \href{https://en.wikipedia.org/wiki/Sundanese_people}{wikipedia}}
\end{figure}

The Sundanese script (Aksara Sunda, {\sundanese ᮃᮊ᮪ᮞᮛ ᮞᮥᮔ᮪ᮓ}) is a writing system which is used by the Sundanese people. It is built based on Old Sundanese script (Aksara Sunda Kuno) which was used by the ancient Sundanese between the 14th and 18th centuries.



\begin{scriptexample}[]{Sundanese}
\unicodetable{sundanese}{"1B80,"1B90,"1BA0,"1BB0}

\sundanese
\obeylines
\bgroup
᮱ {\arial= 1}	᮲ {\arial= 2}	᮳{\arial = 3}
᮴ {\arial= 4}	᮵ {\arial = 5} 	᮶ {\arial= 6}
᮷ {\arial= 7}	᮸ {\arial= 8}	᮹ {\arial= 9}
᮰ {\arial= 0}

\egroup
\end{scriptexample}

\begin{scriptexample}[]{Sundanese}
\bgroup
\sundanese
\centering

◌ᮃᮄᮅᮆᮇᮈᮉᮊᮋᮌᮍᮎᮏᮐᮕᮔᮓᮑᮖᮗᮚᮛᮜᮝᮞᮟᮠᮠ


\egroup
\end{scriptexample}

\bgroup
\def\1{\sundanese ᮱}
\TextOrMath\1\1

$\1$
\egroup

In text In texts, numbers are written surrounded with dual pipe sign \textbar \ldots \textbar. Example: {\textbar \sundanese ᮲᮰᮱᮰ }\textbar = 2010













\newfontfamily\buginese{Noto Sans Buginese}

\section{Buginese}
\label{s:buginese}\index{scripts>Buginese}

\epigraph{The strangest custom I have observed is that some men dress like women, and some women like men, not occassionally, 
but all their lives, devoting themeselves to the occupations and pursuits of their adopted sex. In the case of the males, it seems that the parents of a boy, upon perceiving in him certain effeminancies of habit and appearance, are induced thereby to present him to one of the rajahs, by whom he is received. These youths often acquire much influence over tehir masters\ldots}{James Brooke, in the journal of his vist to Wajo in 1840. }

\begin{figure}[htbp]
\includegraphics[width=\textwidth]{buginese}
\caption{Bugis dressed in traditional clothing (wikimedia).}
\end{figure}

Bugis people is one of Indonesian ethnic which are the inhabitant of South Sulawesi. This tribe is the biggest three after Javanese and Sundanese. Beside the indigenous who live in South Sulawesi, the immigrant of Minangkabau who wander from Sumatra to Sulawesi and the Malay people are being called as Bugis people.

At present time, The Bugis is spread to all over Indonesia like Sulawesi Tenggara, Sulawesi Tengah, Papua, Kalimantan Timur, Kalimantan Selatan even to going abroad. Bugis people are the the most group who insist to spread the Islamic religion.

\begin{figure}[htbp]
\includegraphics[width=\textwidth]{bugi-ship}
\end{figure}

In the past, Bugis people was famous by settled foreigner. The expertise of Bugis-Makassar sailing the ocean was trustworthy, they went up to the overseas territory of Malaysia, the Philippines, Brunei, Thailand, Australia, Madagascar and South Africa. In fact, they also went to Cape Town. Moreover, in South Africa there is a suburb named Maccassar, as a sign of local residents to remember their ancestors homeland.

The reason of Bugis people went to wander because there was conflict among the royal kingdom between Bugis kingdom and Makassar kingdom on 16, 17, 18 and 19 century. The people from the coastal area was mostly wander, instead of they hope for life independence.

Nowadays, Bugis people are mostly doing farming, fishing and trading instead of sailing the ocean. The women are mostly help the men doing farming and some of the elder are still weaving the silk sarong, which is the traditional cloth of Bugis-Makassar.

Many of the marriages are still arranged by parents and ideally take place between cousins. A newlywed couple often lives with the wife’s family for the first few years of their marriage.

In early 1600s after the Islam missionary spread the Islam religion to Makassar, the Bugis held the animist religion, then, they were converted to Islam 1611. A few west coast rulers converted to Christianity in the mid-sixteenth century, but it was failure by the Portuguese at Malacca to provide priests to preach the gospel, it meant that this did not last.

On daily speaking, The Bugis are use their own dialect called “Bahasa Ugi” or Ugi language and they have their own literacy called “Aksara Bugis”. This literacy was founded in 12 century during the entrance of Hindu in Indonesia archipelago.

Bugis people are group who have strong philosophy from their own, there are four principals that they are concern to build their life philosophy, they are:

- The principal of persistent life

- Solidarity and loyalty

- Siri or Pride

- Etiquette and Manners



Those principals are the essential things that build The Bugis to be a social human being. To be a unique group in diversities.

www.indonesia-tourism.com

\paragraph{The Buginese script} is used on the island of Sulawesi, mainly in the southwest. It is of the Brahmic type
and perhaps related to Javanese. It bears some affinity with Tagalog as well, and it apparently does not
traditionally record final consonants (see note on the VIRAMA below). Buginese may be the easternmost
representative of the Brahmi scripts. Sirk (1983) reports that the Buginese language (an Austronesian
language) has a rich traditional literature making it one of the foremost languages of Indonesia. There
may be as many as 2.3 million speakers of Buginese in the southern part of Sulawesi (as of 1971). The
script was reported in some use as of 1983, and a variety of traditional literature has been printed in it.
Buginese literature was studied extensively by B. F. Matthes (a Dutch missionary) in the 19th century.
Matthes published a Buginese-Dutch dictionary in 1874 with a supplement in 1889, as well as a
grammar. The script was previously also used to write the Makassar, Bimanese, and Madurese languages.
For Makassar, Matthes 1858 also gives an older alphabet, which uses different shapes for the letters, and
lacks the HA, but the difference seems to be a change in font style only.


\paragraph{Encoding} Buginese was added to the Unicode Standard in March, 2005 with the release of version 4.1.
The Unicode block for Buginese is U+1A00–U+1A1F:


\unicodetable{buginese}{"1A00,"1A10}


\bgroup
\buginese

ᨀ	ᨀᨗ	ᨀᨘ	ᨀᨙ	ᨀᨛ	ᨀᨚ

\egroup


\paragraph{Punctuation}

There are two punctuation marks the \textit{pallawa} and an end of section marker.

\begin{center}
\begin{tabular}{ll}
\toprule
Pallawa & end section\\
\scalebox{3.5}{\buginese ᨞} &\scalebox{3.5}{\buginese ᨟}\\
\bottomrule
\end{tabular}
\end{center}

Pallawa is used to separate rhythmico-intonational groups, thus functionally corresponds to the period and comma of the Latin script. The pallawa can also be used to denote the doubling of a word or its root.




\section{Philippine Scripts}
\label{s:tagalog}

\newfontfamily\tagalog{Noto Sans Tagalog}

Tagalog (/təˈɡɑːlɒɡ/;[6] Tagalog pronunciation: [tɐˈɡaːloɡ]) is an Austronesian language spoken as a first language by a quarter of the population of the Philippines and as a second language by the majority. Its standardized form, officially named Filipino, is the national language of the Philippines, and is one of two official languages along with English.
It is related to other Philippine languages, such as the Bikol languages, Ilocano, the Visayan languages, Kapampangan and Pangasinan, and more distantly to other Austronesian languages, such as the Formosan languages of Taiwan, Malay (Malaysian \& Indonesian), Hawaiian, Māori and Malagasy.

\begin{figure}[htbp]
\centering
\includegraphics[width=\textwidth]{tagalog-traditional}
\caption{Women In Traditional Dress early 1900s }
\end{figure}

%https://sepiaera.wordpress.com/2013/06/05/women-in-traditional-dress-early-1900s/

\paragraph{Baybayin} Tagalog was written in an abugida—or alphasyllabary—called Baybayin prior to the Spanish colonial period in the Philippines, in the 16th century. This particular writing system was composed of symbols representing three vowels and 14 consonants. Belonging to the Brahmic family of scripts, it shares similarities with the Old Kawi script of Java and is believed to be descended from the script used by the Bugis in Sulawesi.

Although it enjoyed a relatively high level of literacy, Baybayin gradually fell into disuse in favor of the Latin alphabet taught by the Spaniards during their rule.

There has been confusion of how to use Baybayin, which is actually an abugida, or an alphasyllabary, rather than an alphabet. Not every letter in the Latin alphabet is represented with one of those in the Baybayin alphasyllabary. Rather than letters being put together to make sounds as in Western languages, Baybayin uses symbols to represent syllables.

A "kudlit" resembling an apostrophe is used above or below a symbol to change the vowel sound after its consonant. If the kudlit is used above, the vowel is an "E" or "I" sound. If the kudlit is used below, the vowel is an "O" or "U" sound. A special kudlit was later added by Spanish missionaries in which a cross placed below the symbol to get rid of the vowel sound all together, leaving a consonant. Previously, the consonant without a following vowel was simply left out (for example, bundok being rendered as budo), forcing the reader to use context when reading such words.\index{kudlit}

\bigskip

\centerline{
\scalebox{2.5}{b {\tagalog ᜊ᜔}}, 
\scalebox{2.5}{ba/bi {\tagalog ᜊ}},
\scalebox{2.5}{be/bu {\tagalog ᜊᜒ}},
\scalebox{2.5}{bo {\tagalog ᜊᜓ}}
}

{\tagalog \char"170A\char"1714}

\paragraph{Unicode Encoding} Tagalog is a Unicode block containing characters of the pre-Spanish Philippine Baybayin script used for writing the Tagalog language.
\medskip

\unicodetable{tagalog}{"1700,"1710}
\medskip

A sample text from the Bible, The Lord's Prayer (Ama Namin), typeset in Baybayin is shown below.

\bgroup\obeylines
{\tagalog

ᜐᜓᜋᜐᜎᜅᜒᜆ᜔ ᜃ,
ᜐᜋ᜔ᜊᜑᜒᜈ᜔ ᜀᜅ᜔ ᜅᜎᜈ᜔ ᜋᜓ;
ᜋᜉᜐᜀᜋᜒᜈ᜔ ᜀᜅ᜔ ᜃᜑᜍᜒᜀᜈ᜔ ᜋᜓ;
ᜐᜓᜈ᜔ᜇᜒᜈ᜔ ᜀᜅ᜔ ᜎᜓᜂᜊ᜔ ᜋᜓ
ᜇᜒᜆᜓ ᜐ ᜎᜓᜉ, ᜉᜍ ᜈᜅ᜔ ᜐ ᜎᜅᜒᜆ᜔.
ᜊᜒᜄ᜔ᜌᜈ᜔ ᜋᜓ ᜃᜋᜒ ᜅᜌᜓᜈ᜔ ᜅ᜔ ᜀᜋᜒᜅ᜔ ᜃᜃᜈᜒᜈ᜔ ᜐ ᜀᜍᜏ᜔-ᜀᜍᜏ᜔;
ᜀᜆ᜔ ᜉᜆᜏᜍᜒᜈ᜔ ᜋᜓ ᜃᜋᜒ ᜐ ᜀᜋᜒᜅ᜔ ᜋᜅ ᜐᜎ
ᜉᜍ ᜈᜅ᜔ ᜉᜄ᜔ᜉᜉᜆᜏᜇ᜔ ᜈᜋᜒᜈ᜔ ᜐ ᜋᜅ ᜈᜄ᜔ᜃᜃᜐᜎ ᜐ ᜀᜋᜒᜈ᜔;
ᜀᜆ᜔ ᜑᜓᜏᜄ᜔ ᜋᜓ ᜃᜋᜒ ᜁᜉᜑᜒᜈ᜔ᜆᜓᜎᜓᜆ᜔ ᜐ ᜆᜓᜃ᜔ᜐᜓ,
ᜀᜆ᜔ ᜁᜀᜇ᜔ᜌ ᜋᜓ ᜃᜋᜒ ᜐ ᜎᜑᜆ᜔ ᜅ᜔ ᜋᜐᜋ.
ᜐᜉᜄ᜔ᜃᜆ᜔ ᜁᜌᜓ ᜀᜅ᜔ ᜃᜑᜍᜒᜀᜈ᜔, ᜀᜅ᜔ ᜃᜉᜅ᜔ᜌᜍᜒᜑᜈ᜔, 
ᜀᜆ᜔ ᜀᜅ᜔ ᜃᜇᜃᜒᜎᜀᜈ᜔, ᜋᜄ᜔ᜉᜃᜌ᜔ᜎᜈ᜔ᜋᜈ᜔.
ᜀᜋᜒᜈ᜔/ᜐᜒᜌ ᜈᜏ.}

Typeset with \docAuxCommand{tagalog} using Notto Sans Tagalog.

\egroup


% history of philippines https://books.google.com.qa/books?id=4wk8yqCEmJUC&pg=PA22&source=gbs_selected_pages&cad=2#v=onepage&q=tagalog&f=false





\section{Cham}
\label{s:cham}

The Cham alphabet is an abugida used to write Cham, an Austronesian language spoken by some 230,000 Cham people in Vietnam and Cambodia. It is written horizontally left to right, as is English.

\newfontfamily\cham{Noto Sans Cham}

Cham is a Unicode block containing characters for writing the Cham language, primarily used for the Eastern dialect in Cambodia.
Cham script was added to the Unicode Standard in April, 2008 with the release of version 5.1.
The Unicode block for Cham is \textsc{U+AA00–U+AA5F}:

\begin{scriptexample}[]{Cham}
\unicodetable{cham}{"AA10,"AA20,"AA30,"AA40,"AA50}
\end{scriptexample}


\printunicodeblock{./languages/cham.txt}{\cham}


