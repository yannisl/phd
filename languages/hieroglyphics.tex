\subsection{Aegyptian Hieroglyphics}
\index{fonts>Aegyptus}\index{Aegyptus (font)}
\index{fonts>Hieroglyphics}\index{languages>hieroglyphics}
Egyptian hieroglyphics need no introduction. 

One of the best fonts I came across is \idxfont{Aegyptus} from \url{http://users.teilar.gr/~g1951d/} The site also has fonts for Aegean Numbers, Ancient Greek Musical Notation, Ancient Greek Numbers, Ancient Roman Symbols, Arkalochori Axe, Carian, Cypriot Syllabary, Dispilio tablet, Linear A, Linear B Ideograms, Linear B Syllabary, Lycian, Lydian, Old Italic, Old Persian, Phaistos Disc, Phoenician, Phrygian, Sidetic, Troy vessels’ signs and Ugaritic. Cretan Hieroglyphs and Cypro-Minoan script(s) are offered in separate files.
\medskip

\bgroup

\centering 

\font\myfont = "Aegyptus"

\scalebox{7}{\myfont\XeTeXglyph 201}
\scalebox{7}{\myfont\XeTeXglyph 203}
\scalebox{7}{\myfont\XeTeXglyph 163}
\scalebox{7}{\myfont\XeTeXglyph 164}
\scalebox{7}{\myfont\XeTeXglyph 165}
\scalebox{7}{\myfont\XeTeXglyph 168}

\captionof{table}{Example of Egyptian Hieroglyphics typeset with the \textit{Aegyptus} font.} 
\egroup



\begin{texexample}{TeXeXglyph}{ex:xetexglyph}
\raggedright
\font\myfont = "Aegyptus"
\setcounter{glyphcount}{136}

\whiledo
{\value{glyphcount}<\XeTeXcountglyphs\myfont}
{\arabic{glyphcount}:~
{\myfont\XeTeXglyph\arabic{glyphcount}}\quad
\stepcounter{glyphcount}}
\end{texexample}
