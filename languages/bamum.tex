\chapter{Bamum}
\label{s:bamum}
\epigraph{"No known alphabet was ever invented by a European."}{Jeffreys' translation from the Royal script.}

\label{s:bamum}
\index{scripts>Bamum}
\newfontfamily\bamum{NotoSansBamum-Regular.ttf}

The Bamum scripts are an evolutionary series of six scripts created for the Bamum language by King Njoya of Cameroon at the turn of the 20th century. They are notable for evolving from a pictographic system to a partially alphabetic syllabic script in the space of 14 years, from 1896 to 1910. Bamum type was cast in 1918, but the script fell into disuse around 1931.

\begin{figure}[htbp]
\parindent=0pt

\centering

\includegraphics[width=\textwidth]{bamum}

\caption{King Njoya of Bamum receiving an oil painting of Kaiser Wilhelm II. The gift was in return for his support in the German campaign against the Nso'.}
\end{figure}

The Bamum, sometimes called Bamoum, Bamun, Bamoun, or Mum, are a Bantoid ethnic group of Cameroon with around 215,000 members.



\begin{scriptexample}[]{Bamum}
\unicodetable{bamum}{"A6A0,"A6B0,"A6C0,"A6D0,"A6E0,"A6F0}
\end{scriptexample}