\newfontfamily\indus{NFM-Indus Script}
\chapter{The Indus Script}
\label{s:indusscript}
\index{script>Indus Script}\index{Indus Script}

The Indus script (also known as the Harappan script) is a corpus of symbols produced by the Indus Valley Civilization during the Kot Diji and Mature Harappan periods between 3500 and 1900 BCE. Most inscriptions containing these symbols are extremely short, making it extremely difficult to judge whether or not these symbols constitute a script used to record a language, or even symbolise a writing system.[4] In spite of many attempts,[5] 'the script' has not yet been deciphered, but efforts are ongoing. There is no known bilingual inscription to help decipher the script, nor does the script show any significant changes over time. However, some of the syntax (if that is what it may be termed) varies depending upon location.[4]

The first publication of a seal with Harappan symbols dates to 1875, in a drawing by Alexander Cunningham.[6] Since then, over 4,000 inscribed objects have been discovered, some as far afield as Mesopotamia. In the early 1970s, Iravatham Mahadevan published a corpus and concordance of Indus inscriptions listing 3,700 seals and 417 distinct signs in specific patterns. He also found that the average inscription contained five symbols, and the longest inscription contained only 14 symbols in a single line.[7] He also established the direction of writing as right to left.[8]














\bgroup
\def\E#1 {\char"E#1 }
\indus

\char"E000 \char"E088

\E001 \E002 \E003 

\begin{scriptexample}[]{indus}
\unicodetable{indus}{"E000,"E010,"E020,"E030,"E040,"E050, "E060, "E070, "E080,"E090,%
"E0A0,"E0B0,"E0C0,"E0D0,"E0F0,"E100}%
\end{scriptexample}

\egroup


\section{Indus Script Font}

The Indus signs have been under constant analysis and study. These have been subjected to various examinations where these were identified as primary and composite signs.

Asko Parpola has made a continuing contribution to research on the Indus writing system. He collected and critically edited the Indus signs as he attempted at structural analysis. His objectives were to find out the number of graphemes, and the word length. His search for primary signs and identifying composite signs resulted in preparation of the sign list of the Indus script, with principle graphic variants, each with one reference.

The Indus signs have been largely used as drawing images in computational analysis and studies. Present effort is to create the Indus signs in Scalable Vector Graphics (SVG) based font for installing in computers.

National fund for Mohenjodaro has developed this font for installing on computers and embedding on websites by researchers and users around the world. This font is developed by Mr. Shabir Kumbhar, engineering / embedded and mapping by Mr. Amar Fayaz Buriro with the consultation of Dr. Kaleemullah Lashari.

Indus Script font is available to be downloaded for further studies, computational exercises and statistical analysis, free of charge; the only encumbrance is that user acknowledge our website.\href{https://www.mohenjodaroonline.net/index.php/indus-script}{mojenjodaroonline.net}