\section{Balinese}

\epigraph{In Bali the gods are thought of as the \textit{children} of the people, not as august parental figures. Speaking through the lips of those in trance, the gods address the villages as ``papa''  and ```mama'', and the people are said to spoil or indulge their gods\ldots}{Gregory Bateson and Margaret Mead in \textit{Balinese Character: A Photographic Analysis, 1942}}
\label{s:balinese}\index{Balinese}\index{Aksara Bali}\index{Bali}\index{Lombok}

\newfontfamily{\balinese}{AksaraBali.ttf}

Balinese or simply Bali\footnote{Not to be confused with the Nigerian or Papua New Guinea languages also named Bali.} is a Malayo-Polynesian language spoken by 3.3 million people (as of 2000) on the Indonesian island of Bali, as well as northern Nusa Penida, western Lombok and eastern Java.[3] Most Balinese speakers also know Indonesian. Balinese itself is not mutually intelligible with Indonesian, but may be understood by Javanese speakers after some exposure.

In 2011, the Bali Cultural Agency estimates that the number of people still using Balinese language in their daily lives on the Bali Island does not exceed 1 million, as in urban areas their parents only introduce Indonesian language or even English, while daily conversations in the institutions and the mass media have disappeared. The written form of the Balinese language is increasingly unfamiliar and most Balinese people use the Balinese language only as a spoken tool with mixing of Indonesian language in their daily conversation. But in the transmigration areas outside Bali Island, Balinese language is extensively used and believed to play an important role in the survival of the language.[4]

\begin{figure}[htbp]
\centering

\includegraphics[width=\textwidth]{bali-cock.jpg}
\end{figure}

The higher registers of the language borrow extensively from Javanese: an old form of classical Javanese, Kawi, is used in Bali as a religious and ceremonial language.

\paragraph{The Balinese script} is natively known as Aksara Bali and Hanacaraka, is an abugida used in the island of Bali, Indonesia, commonly for writing the Austronesian Balinese language, Old Javanese, and the liturgical language Sanskrit. With some modifications, the script is also used to write the Sasak language, used in the neighboring island of Lombok.[1] 

The script is a descendant of the Brahmi script, and so has many similarities with the modern scripts of South and Southeast Asia. The Balinese script, along with the Javanese script, is considered the most elaborate and ornate among Brahmic scripts of Southeast Asia.[2]

\includegraphics[width=\textwidth]{bali}


Though everyday use of the script has largely been supplanted by the Latin alphabet, the Balinese script has significant prevalence in many of the island's traditional ceremonies and is strongly associated with the Hindu religion. The script is mainly used today for copying lontar or palm leaf manuscripts containing religious texts.[2][3]



{\indicative ◌ }

\newcounter{under}
\setcounter{under}{"1B00}

\def\cb#1 {
\hspace*{2.5pt}
 
 $\text{◌#1}_{\pgfmathparse{Hex(\theunder)}\text{\pgfmathresult}}$
\stepcounter{under}
\vskip5pt\par
}
\begin{scriptexample}[]{Balinese}


\balinese
	 
᭐	᭑	᭒	᭓	᭔	᭕	᭖	᭗	᭘	᭙	᭚	᭛	᭜	᭝	᭞	᭟\\\
 
\def\columnseprulecolor{\color{thegray}}
\columnseprule.4pt
\begin{multicols}{8}

\texttt{U+1B0x}	

\cb{ᬀ }  \cb{ ᬁ } 	\cb{ ᬂ }  	\cb ᬃ	\cb ᬄ 	\cb ᬅ	\cb ᬆ	\cb ᬇ	\cb ᬈ	\cb ᬉ	\cb ᬊ	\cb ᬋ	\cb ᬌ	\cb ᬍ	\cb ᬎ	\cb ᬏ

\columnbreak

\texttt{U+1B1x}	 

\cb ᬐ	 \cb ᬑ 	\cb ᬒ 	\cb ᬓ	\cb ᬔ	\cb ᬕ	\cb ᬖ \cb ᬗ 	\cb ᬘ 	\cb ᬙ 	\cb ᬚ	\cb ᬛ 	\cb ᬜ 	\cb ᬝ 	\cb ᬞ	\cb ᬟ 

\columnbreak

U+1B2x	 

\cb ᬠ◌ 	\cb ᬡ	\cb ᬢ	\cb ᬣ	\cb ᬤ	\cb ᬥ	\cb ᬦ	\cb ᬧ	\cb ᬨ	\cb ᬩ	\cb ᬪ	\cb ᬫ	\cb ᬬ	\cb ᬭ	\cb ᬮ	\cb ᬯ

\columnbreak
U+1B3x 

\cb ᬰ	\cb ᬱ	\cb ᬲ	\cb ᬳ	\cb ᬴	\cb ᬵ	\cb ᬶ	\cb ᬷ	\cb ᬸ	\cb ᬹ	\cb ᬺ	\cb ᬻ	\cb ᬼ	\cb ᬽ	\cb ᬾ	\cb ᬿ


\columnbreak
U+1B4x	 

\cb ᭀ	 \cb ᭁ	\cb ᭂ	\cb ᭃ	\cb ᭄	\cb ᭅ	\cb ᭆ	\cb ᭇ	\cb ᭈ	\cb ᭉ	\cb ᭊ	\cb ᭋ

\columnbreak				
U+1B5x	 

\cb ᭐	\cb ᭑	\cb ᭒	\cb ᭓	\cb ᭔	\cb ᭕	\cb ᭖	\cb ᭗	\cb ᭘	\cb ᭙	\cb ᭚	\cb ᭛	\cb ᭜	\cb ᭝	\cb ᭞	\cb ᭟\\

\columnbreak

U+1B6x 

\cb ᭠	\cb ᭡	\cb ᭢	\cb ᭣	\cb ᭤	\cb ᭥	\cb ᭦	\cb ᭧	\cb ᭨◌ 	\cb ᭩◌ 	\cb ᭪◌ 	\cb ᭫	\cb ᭬	\cb ᭭	\cb ᭮	\cb ᭯

\columnbreak
U+1B7x	 

\cb ᭰	 \cb ᭱  \cb ᭲  \cb ᭳	 \cb ᭴	\cb ᭵	\cb ᭶	\cb ᭷	\cb ᭸	\cb ᭹	\cb ᭺	\cb ᭻	\cb ᭼


\end{multicols}

\end{scriptexample}


One of the most comprehensive fonts is Aksara Bali\footnote{\url{http://www.alanwood.net/downloads/index.html}}. This is obtainable at Alan Wood's website.