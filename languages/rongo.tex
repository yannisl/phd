\cxset{image = rapa-nui}
\chapter{Rongorongo of Easter Island}

Easter Island in the southeastern Pacific is one of the remotest spots
inhabited by humans. From evidence pieced together by archaeologists and
linguists, we now know that this volcanic speck more than 3000 kilometres from
the coast of Chile was probably first colonised by people from eastern Polynesia
in the early centuries AD. But much about the island remains enigmatic,
including its best-known feature, the colossal stone carvings, or moai,
many of which stared inland from the coast. Equally mysterious until now was the
Easter Island script known as rongorongo.

The script takes the form of lines of hieroglyphs, in which every second line
is inscribed upside down. Before missionaries brought Christianity to the
island, Easter Islanders believed that wooden tablets incised with these glyphs
were filled with mana, or spiritual power. The name rongorongo means
“chants” or “recitation”, and the key to understanding it lay with the island’s
royals, chiefs, priests and teachers, who carved the tablets and would recite
their chants at important gatherings. But by the time rongorongo came to the
attention of academic linguists in the late 19th century, none of these experts
remained. The script’s meaning had been lost.

In the 1950s, Thomas Barthel, an ethnologist from the University of
Tübingen in Germany, identified 120 basic elements of
rongorongo—mostly stylised objects or creatures—which combine to
give between 1500 and 2000 glyphs. 

Rongorongo (/ˈrɒŋɡoʊˈrɒŋɡoʊ/; Rapa Nui: [ˈɾoŋoˈɾoŋo]) is a system of glyphs discovered in the 19th century on Easter Island that appears to be writing or proto-writing. Numerous attempts at decipherment have been made, none successfully. Although some calendrical and what might prove to be genealogical information has been identified, none of these glyphs can actually be read. If rongorongo does prove to be writing and proves to be an independent invention, it would be one of very few independent inventions of writing in human history.[1]

\begin{figure}[htbp]
\parindent0pt

\includegraphics[width=\textwidth]{rongo-smithsonian}
\caption{Inscribed tablet (Kohau ronogrongo), 19th century, wood L. 9 1/2 in 24.1cm \textit{Department of Anthropology, Smithsonian Institution, Washington, D.C.}}

\end{figure}

Two dozen wooden objects bearing rongorongo inscriptions, some heavily weathered, burned, or otherwise damaged, were collected in the late 19th century and are now scattered in museums and private collections. None remain on Easter Island. The objects are mostly tablets shaped from irregular pieces of wood, sometimes driftwood, but include a chieftain's staff, a bird-man statuette, and two reimiro ornaments. There are also a few petroglyphs which may include short rongorongo inscriptions. Oral history suggests that only a small elite was ever literate and that the tablets were sacred.

\begin{figure}[htbp]
\includegraphics[width=\textwidth]{rongo-rongo}
\caption{Metropolitan}
\end{figure}



Easter Island is a special territory of Chile that was annexed in 1888. Administratively, it belongs to the Valparaíso Region, and, more specifically, it is the only commune of the Province Isla de Pascua.[8] According to the 2012 Chilean census, the island has about 5,800 residents, of whom some 60 percent are descendants of the aboriginal Rapa Nui.
Easter Island is considered part of Insular Chile.

The name "Easter Island" was given by the island's first recorded European visitor, the Dutch explorer Jacob Roggeveen, who encountered it on Easter Sunday (5 April) in 1722, while searching for Davis or David's island. Roggeveen named it Paasch-Eyland (18th-century Dutch for "Easter Island").[9][10] The island's official Spanish name, Isla de Pascua, also means "Easter Island".

The current Polynesian name of the island, Rapa Nui ("Big Rapa"), was coined after the slave raids of the early 1860s, and refers to the island's topographic resemblance to the island of Rapa in the Bass Islands of the Austral Islands group.[11] However, Norwegian ethnographer Thor Heyerdahl argued that Rapa was the original name of Easter Island and that Rapa Iti was named by refugees from there.\footfullcite{heyerdhal1961}

\begin{figure}[htb]
\parindent0pt

\hspace*{-3.8cm}\includegraphics[width=\pagewidth]{rongo-panorama}

\caption{Panorama of Anakena beach, Easter Island. The moai pictured here was the first to be raised back into place on its ahu in 1955 by Thor Heyerdahl[73] using the labor of islanders and wooden levers.}
\end{figure}



The phrase Te pito o te henua has been said to be the original name of the island since French ethnologist Alphonse Pinart gave it the romantic translation "the Navel of the World" in his Voyage à l'Île de Pâques, published in 1877.[13] William Churchill (1912) inquired about the phrase and was told that there were three te pito o te henua, these being the three capes (land's ends) of the island. The phrase appears to have been used in the same sense as the designation of "Land's End" at the tip of Cornwall. He was unable to elicit a Polynesian name for the island itself, and concluded that there may not have been one.[14]
According to Barthel (1974), oral tradition has it that the island was first named Te pito o te kainga a Hau Maka "The little piece of land of Hau Maka".[15] However, there are two words pronounced pito in Rapa Nui, one meaning 'end' and one 'navel', and the phrase can thus also mean "the Navel of the World". 

Another name, Mata ki te rangi, means "Eyes looking to the sky".[16]
Islanders are referred to in Spanish as pascuense; however it is common to refer to members of the indigenous community as Rapa Nui.
