\chapter{Indo-European Languages}

Indo-European languages were originally, a couple of millennia ago, spoken from Europe
to India. Now, in postcolonial times, they are spoken all over the world. English is thus,
besides being a global lingua franca, spoken in North America, Australia, parts of Africa,
and elsewhere; Spanish in Central and South America (Portuguese in Brazil); French in
parts of Africa, etc. 

According to Glottolog, 584\footnote{As of 2017, \protect\url{https://glottolog.org/resource/languoid/id/anat1257}} IE languages (the number is in reality approximate) are spoken by almost 3 billion people in the world. With regard to the number
of speakers, this makes the IE language family the biggest in the world. Concerning
the number of individual languages, IE is not the largest language family – for instance,
the African Niger-Congo family with more than 1500 languages is much larger. In the
world, there are around 7000 languages (this is not an exact number, since it is extremely
difficult to define linguistically what a language actually is), so in this sense IE languages
are less important, although IE is still one of the biggest language families. Historical
reasons – primarily the concentration of economic power in Europe and later in North
America, together with the culture, science, etc., that goes with it – make IE the bestknown
language family in the world.

In the territories from Europe to India, where IE languages were spoken already a few
thousand years ago, not all languages are IE. Most of the languages spoken in Europe
today are IE, the exceptions (disregarding Balkan Turkish, the languages of the Caucasus,
and recent migrant languages) being Basque, Maltese (an offspring of Arabic), and
the Uralic languages (Hungarian, Estonian, Finnish, and a few smaller ones). Basque is
the only remnant of pre-IE European languages. Other such cases (like Etruscan) have
long disappeared. In the Middle East, we find both IE (like Persian/Farsi, Pashto, Kurdish,
and Tajik) and non-IE languages (like Turkish, Azeri, Arabic, etc.). In the Indian
subcontinent, many languages are IE (like Hindi/Urdu, Bengali, Punjabi, and Nepali),
while many belong to the non-IE Dravidian family (like Tamil and Malayalam) and other
non-IE language families (like smaller languages from the Tibeto-Burman and Austroasiatic families). 

What is a language family? What does it mean when we say that English and Albanian
belong to the IE language family? To put it succinctly, a language family consists of all
the languages that have evolved from a single original proto-language. It is thus a group
of genetically related languages (up to a certain point – it just may well be that all human
languages are related, though this cannot be proven). In the IE case, it means that we can
prove that all the IE languages that we know have evolved through many stages from a
single parent language that we today call Proto-Indo-European. This proto-language was 
spoken some 6000 years ago and, like most proto-languages, is unattested (which means
that we have no written records of PIE). This is not always the case; for instance, Latin (as
the proto-language of all Romance languages) and Old Chinese (as the proto-language of
all modern Chinese “dialects”/languages) are attested.

How can one language split into more than 400 languages? The answer is simple – through language change. All languages change all the time (even today with all our
schools, means of communication, and mass media). Linguistic changes that occur during
the life of an individual speaker are usually slow and gradual (although still perceivable,
whether they have to do with changes in pronunciation, grammatical forms, or vocabulary),
but in time languages change enough to become mutually incomprehensible and
sometimes very different. This is made easy by migration – when two social groups separate,
it is easy to imagine that a previously uniform language can evolve in different directions
(cf. the differences between British, American, and Australian English, although
they are still in contact, mutually comprehensible, and thus considered one language).

That languages like English, Czech, and Welsh have all developed from a single
proto-language was not always apparent. Already in ancient times, people noticed that
there are many similar words in, for instance, Latin and Greek – cf. e.g. Lat. māter
‘mother’, pater ‘father’ and Gr. μήτηρ /mḗtēr/, πατήρ /patḗr/ (or, for that matter, English
mother and father). However, they did not know how to properly explain such similarities.
Thus, there have been many unscientific hypotheses, such as that all languages
are derived from Hebrew (considered a holy language) and so on. In the 15th and 16th
centuries, there were a number of scholars, like Rodolphus Agricola/Roelof Huisman
(1443/4–1485), Sigismund Gelenius/Zikmund Hrubý z Jelení (1497–1554), and Johann
Elichmann (1601/2–1639), who noted correspondences between various IE languages.
However, it is Marcus van Boxhorn (1612–1653), a professor at Leiden University, who 

can be regarded as the first historical linguist and the father of IE linguistics. In 1647
he advanced a theory that Greek, Latin, Persian, Old Saxon, Dutch, German, Gothic,
Russian, Danish, Swedish, Lithuanian, Czech, Croatian, and Welsh (but not Hebrew)
originally stemmed from “Scythian” (= PIE). His friend Claudius Salmasius/Claude de
Saumaise (1588–1653) added Sanskrit to the list. Amazingly, van Boxhorn also noted
the importance of recognizing false cognates, loanwords, systematic correspondences,
plausible semantic agreement, morphological comparison, and synchronically irregular
forms (all basic elements of historical linguistics now called the “comparative method”).2
The European discovery of Sanskrit played an important role in the beginnings of IE
historical and comparative linguistics. Already in 1585, Filippo Sassetti (1540–1588), a
merchant and scholar, noted in a private letter the similarity of some words in Sanskrit
and Italian (e.g., Skr. nava and Ital. nove ‘nine’), and in the 18th century a couple of
scholars wrote on the obvious similarity of Sanskrit to European languages. One of them
was William Jones (1746–1794), who in a lecture in 1786 compared Sanskrit to Greek,
Latin, Gothic, Celtic, and Old Persian, claiming that they “have sprung from some common
source.” Jones is usually wrongly considered the founding father of IE historical
linguistics, although he was neither the first to make such a hypothesis nor the one with
the clearest presentation of the problem. Shortly after, the first modern IE historical
linguists – Jacob Grimm (1785–1863; famed as one of the Grimm brothers), Rasmus Rask
(1787–1832), and Franz Bopp (1791–1867) – started doing serious historical linguistics,
and others followed. The very term Indo-European was coined in 1813 by Thomas Young
(1773–1829). The discovery of Anatolian and Tocharian texts in the beginning of the 20th
century gave new impetus for IE linguistics.

The IE language family comprises ten principal branches (those being the ones sufficiently
attested in ancient times and/or today) and a number of fragmentarily attested
separate languages spoken in ancient times (like Lusitanian or Phrygian) that have since
disappeared. Three main branches (Anatolian, Indo-Iranian, and Greek) are attested
already in the second millennium BCE, two in the first millennium BCE (Italic, Celtic),
and the other five branches later. Three of the principal branches consist of one language
only (Greek, Armenian, and Albanian), although those comprise different dialects
as well. Two major branches (Anatolian and Tocharian) are now extinct, and the same
goes for all fragmentarily attested languages and certain languages from other primary
branches (like Gaulish). The ten principal IE branches are as follows.

\begin{description}
\item [Anatolian]  attested from the 19th/16th to the 1st century BCE
Spoken, until its disappearance, in Asia Minor. Old Anatolian languages (spoken in
the 2nd millennium BCE) are Hittite (the best attested, written in Akkadian cuneiform,
the language of the great Hittite Empire), Cuneiform Luwian (Luvian), Hieroglyphic
Luwian, and Palaic. Hittite is attested from around the 16th century, but a
few names and loanwords are found already in the 19th century BCE in Old Assyrian
texts. New Anatolian languages (spoken in the 1st millennium BCE) are Lydian,
Lycian (Lycian A), Milyan (Lycian B), Carian, Sidetic, and Pisidian.

\item[Indo-Iranian] attested from the 15th–14th century BCE
Spoken mostly in the northern part of the Indian subcontinent and the Middle East
(Eastern Turkey, Caucasus, Iran, Afghanistan). As a result of old migrations, the
Romani language dialects are spoken by the Roma people in Europe. Indo-Iranian
consists of two great groups – Indo-Aryan (Indic) and Iranian. The third, smaller
group is Nuristani (on the border between Afghanistan and Pakistan), and the fourth,
disputably, is Dardic (traditionally considered part of Indo-Aryan). The two major
groups are attested from ancient times and represented by many languages, while
the two small groups are attested only from modern times and consist of a few small
languages spoken in remote parts of northern India, Afghanistan, and Pakistan. IndoAryan
languages are chronologically divided into Old Indic/Indo-Aryan (Vedic – the
language of the Vedas; Sanskrit – the classical language of India), Middle Indic/
Prakrits (Pali, Shauraseni, Maharashtri, Magadhi, etc.), and New Indic (many languages
– Hindi/Urdu, Bengali, Punjabi, Marathi, etc.). Certain words and names of
Indo-Aryan origin were attested in the 15th–14th century BCE in Hurrian texts. The
Rigveda (the oldest part of the Vedas) is usually dated to the second half of the 2nd
millennium BCE but was transmitted orally for a long time. Iranian languages are
chronologically divided into Old Iranian (Avestan – the language of Zarathustra,
from the first half of the 1st millennium BCE, or perhaps even earlier but at first
transmitted orally; Old Persian – the language of the Achaemenid Empire), Middle
Iranian (Pahlavi – the language of the Sassanid Empire; Sogdian; etc.), and New
Iranian (New Persian/Farsi – the official language of Iran; the Kurdish languages;
Pashto – the official language of Afghanistan; etc.).

\item[Greek] – attested from the 15th–14th century BCE\\
Spoken from ancient times until the present in mainland Greece and neighboring
islands. The first attestations were written in the Linear B script in Mycenaean Greek,
the earliest recorded Greek dialect. From the 8th century, Greek has been written in
the Greek alphabet. In the classical period (around the 5th century BCE), the following
Greek dialects existed – North-West dialects, Doric, Aeolic (controversial),
Arcado-Cypriot, and Ionic-Attic (the details of the subgrouping are disputed). Later
stages of Greek are Hellenistic/Koiné, Byzantine/Medieval, and Modern Greek.

\item[Italic] – attested from the 7th century BCE\\
Spoken in ancient Italy. Italic consists of two groups – Latino-Faliscan and Sabellian
(Osco-Umbrian). The Latino-Faliscan group consists of Latin (very well attested)
and Faliscan. The Sabellian group consists of Oscan, Umbrian, and a number of
smaller and poorly attested Italic languages. All Italic languages except Latin disappeared
already in Roman times owing to the expansion of Latin as the language of
the Roman state. Latin later developed into numerous modern Romance languages
(Portuguese, Spanish, Catalan, French, Italian, Romanian, etc.).

\item[Celtic] – attested from the 6th century BCE\\
Now spoken in peripheral areas of the British Isles (Ireland, Wales, Scotland) and
France (Brittany), previously spoken in wide areas of Europe (from the Iberian Peninsula
and France all the way to Asia Minor). The Celtic branch can be divided
(though controversially) into three groups: Celtiberian, Continental Celtic, and Insular
Celtic. Celtiberian is a one-language group, attested in the 2nd to 1st century
BCE in present-day Spain. Continental Celtic languages – Gaulish (in France) and
Lepontic (in northern Italy) – are now extinct. Insular Celtic consists of Brythonic
(Welsh, Breton, and extinct Cornish) and Goidelic (Irish/Gaelic, Scottish Gaelic, and
extinct Manx). Of all currently existing Celtic languages, only Welsh has a more or
less secure future.

\item[Germanic] – attested from the 3rd (?) century BCE\\
Nowadays spoken in North-West Europe. The first Germanic words are attested perhaps
already in the 3rd (?) century BCE on the so-called Negau helmet (found in 
Ženjak in present-day Slovenia). The earliest runic inscriptions date from the first
centuries CE, and the Gothic translation of the Bible from the 4th century. Germanic
languages comprise three groups: East, North, and West. The extinct East Germanic
group consists of Gothic, Vandal, and Burgund. The Northern languages are Old
Norse and, in modern times, the Scandinavian languages. The Western languages
are Old High German, Old Saxonian, Old Low Franconian (Old Dutch), Old Frisian,
and Old English (in modern times, respectively, German, Low German, Dutch, Frisian,
and English).

\item[Armenian] – attested from the 5th century CE\\
Spoken in present-day Armenia and elsewhere in the Caucasus and Middle East as
a minority language, previously also in wide areas of present-day eastern Turkey.
Modern Armenian consist of two dialects, Eastern and Western, both standardized.

\item[Tocharian] – attested from the 4th–5th to the 8th–10th century CE
Spoken in merchant cities on the ancient Silk Road, in the present-day Xinjiang
province of western China (now inhabited by Turkic Uyghurs). There were two
closely related Tocharian languages – Tocharian A and Tocharian B.

\item[Balto-Slavic] – attested from the 9th (Slavic) and 16th (Baltic) century CE
Spoken in the Baltic, the Balkans, and Central and Eastern Europe. The Balto-Slavic
branch consists of West Baltic, East Baltic, and Slavic. The extinct Old Prussian is
West Baltic, and Lithuanian and Latvian are East Baltic (there are a few other poorly
attested and extinct Baltic languages). Slavic consists of East (Russian, Belorussian,
Ukrainian), West (Polish, Czech, Slovak, etc.), and South Slavic (Slovene; Bosnian/
Croatian/Montenegrin/Serbian – traditionally known as Serbo-Croatian; Macedonian;
Bulgarian). The first attested Slavic language, Old Church Slavic, was based
on the Thessaloniki Old Macedonian dialect.

\item[Albanian] – attested from the 15th century CE\\
Spoken in present-day Albania, Kosovo, West Macedonia, South Montenegro,
North-West Greece, and South Italy. There are two dialects – Gheg in the north and
Tosk in the south, the latter being the basis of Standard Albanian.


\end{description}

