\section{Mongolian}
^^A\newfontfamily\mongolian{NotoSansMongolian-Regular.ttf}

The classical Mongolian script (in Mongolian script:{\mongolian ᠮᠣᠩᠭᠣᠯ ᠪᠢᠴᠢᠭ᠌} {\pan Mongγol bičig}; in Mongolian Cyrillic: {\pan Монгол бичиг} Mongol bichig), also known as Uyghurjin Mongol bichig, was the first writing system created specifically for the Mongolian language, and was the most successful until the introduction of Cyrillic in 1946. Derived from Uighur, Mongolian is a true alphabet, with separate letters for consonants and vowels. The Mongolian script has been adapted to write languages such as Oirat and Manchu. Alphabets based on this classical vertical script are used in Inner Mongolia and other parts of China to this day to write Mongolian, Sibe and, experimentally, Evenki.

\begin{scriptexample}[]{Mongolian}
\unicodetable{mongolian}{"1820,"1830,"1840,"1850,"1860,"1870,"1880,"1890,"18A0}
\end{scriptexample}

\PrintUnicodeBlock{./languages/mongolian.txt}{\mongolian}