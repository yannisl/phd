\parindent=1em
\parskip=12pt plus.1pt minus.1pt
\chapter{Lycian}
\label{sec:lycian}

Two Iron Age cultures of western Anatolia that emerged in the aftermath of the
collapse of \ac{LBA} systems merit special discussion because of
their importance both as cultures with indigenous traditions and languages and
as geographical, political, and cultural intermediaries between the Near Eastern
and Aegean worlds: the Lydians in central western Anatolia and the Lycians in
southwestern Anatolia.

Lycian was the indigenous language of Lycia at least during the middle
and late first millennium bce. Recent evidence from the Hieroglyphic Luvian inscription of
Yalburt – specifically, forms of the place names for Tlos, Pinara, and Xanthos – has now
proven that the “Lukka-Lands” of the second-millennium Hittite cuneiform texts do refer
to historical Lycia, that is, roughly the mountainous peninsula on the southwest coast of
Anatolia lying between the Gulf of Telmessos and the Bay of Attaleia (modern Gulf of
Fethiye and Gulf of Antalya; see Poetto 1993). Obviously, without direct textual evidence
from Lycia itself during the second millennium it is quite impossible to characterize with
any precision the language of “Lukka” in that era.\tcbdocmarginnote{23.09.2017}

Lycian shares a number of specific features, including innovations, with Luvian, and it
is widely held that Lycian and Luvian form a subgroup within the Anatolian family; in
other words, that they reflect a prehistoric “Proto-Luvian” language which had developed
out of Proto-Anatolian along different lines from Hittite, Palaic, and Lydian, the other
assured members of the Anatolian group (see, inter alios, Oettinger 1978). One may even
read that Lycian is a later form of Luvian, though not necessarily of that form of Luvian
which is directly attested in the second millennium. The shared features of Lycian and
Luvian are undeniable, but several of these are also common to Lydian, while there are also
crucial divergences between Lycian and Luvian (see Gusmani 1960 and Melchert 1992a).


These divergences make it impossible to reconstruct a coherent Proto-Luvian language
distinct from Proto-Anatolian. One should rather view the common features of Luvian
and Lycian in terms of dialect geography. As the individual languages began to diverge in
their development fromProto-Anatolian, they remained in contact, and innovations which
arose in various places spread in the typical irregular fashion. Luvian, which occupied a
geographically central position, unsurprisingly shares some isoglosses with Lycian (and to
a lesser extent Lydian) to the west, and others with Hittite and Palaic to the east.

\begin{figure}[htb]
\begin{minipage}[b]{0.5\textwidth}
\includegraphics[width=1\linewidth]{./images/xanthian-obelisk.jpg}
\end{minipage}\hspace*{1em}
\begin{minipage}[b]{0.45\textwidth}
\captionof{figure}{Part of the Xanthian obelisk inscription. The Xanthian Obelisk, also known as the Xanthos or Xanthus Stele, the Xanthos or Xanthus Bilingual, the Inscribed Pillar of Xanthos or Xanthus, the Harpagus Stele, and the Columna Xanthiaca, is a stele bearing an inscription currently believed to be trilingual, found on the acropolis of the ancient Lycian city of Xanthos, or Xanthus, near the modern town of Kınık in southern Turkey. The three languages are Ancient Greek, Lycian and Milyan (the last two are Anatolian languages and were previously known as Lycian A and Lycian B respectively).}
\end{minipage}
\end{figure}


For a newer look at the Xanthian Stele see also \cite{Gygax2005}

\section{Sources}

The extant Lycian corpus includes more than 172 inscriptions on stone, over 200 on coins, 
and a handful on other objects. The overwhelming majority of
those on stone are sepulchral texts, with highly stereotyped content.\footcite{Woodard2008} 


Apart from several poorly preserved decrees, the most important exceptions are the inscribed stele of Xanthos,
which describes the military exploits and building activities of a local dynasty, and the
Lycian–Greek–Aramaic trilingual of the Letoon, which records the founding of a cult for
the goddess Leto by the citizens of Xanthos at a temple a few miles south of the city. The
latter text of some forty-one lines has predictably proven to be of immense importance in
advancing understanding of Lycian. Much of the text of the Xanthos stele remains opaque
due to problems of vocabulary which result from the nearly unique subject matter.


Two of the Lycian texts (one of which is the last portion of the Xanthos stele) are written
in a distinct dialect known either as Lycian B (vs. ordinary Lycian A) or as Milyan. The
relationship of the two dialects is indeterminate.Milyan is more archaic than ordinary Lycian
in certain features, and it is noteworthy that bothMilyan texts are in verse (see Eichner 1993
with references). However, it would be dangerous to conclude from these limited facts that
Milyan is merely an older stage of Lycian preserved for special literary purposes. This is only
one of several viable possibilities: see Gusmani (1989–1990) for a useful discussion of the
problem. Unless stated otherwise, the description which follows applies to both forms of
Lycian, but the bulk of the evidence comes from Lycian (A). Extrapolation of the description
to Milyan is often based on very limited evidence and should be viewed as highly provisional.
Special features of Milyan will be explicitly noted where appropriate.

Thanks to the Letoon Trilingual and exploitation of the features shared with Luvian,
understanding of Lycian has improved dramatically in the last two decades (with the notable
exception of the Xanthos stele and Milyan). However, certain features of morphology and
syntax cited below impose some quite serious limitations. One should regard the following
description as intermediate in completeness and reliability between those for Palaic and
Lydian on the one hand, and that for Luvian on the other.


\section{Writing System}
\epigraph{{\panunicode πέμπε δὲ μιν Λυκίηνδε, πόρεν δ’ ὅ γε σήματα λυγρά,
γράψας ἐν πίνακι πτυκτῷ θυμοφθόρα πολλά,
δεῖξαι δ’ ἠνώγειν ὧ πενθερῷ, ὄφρ’ ἀπόλοιτο.}}{(Iliad 6.168-170)}


Lycian iswritten in an alphabet derived fromor closely related to that ofGreek. The details of
the relationship remain unclear: for discussion see Carruba 1978a. The direction of writing
is left to right. Use of word-dividers is frequent, but by no means absolutely consistent. This
fact means that the status of certain morphemes as clitics is, strictly speaking, a matter of
interpretation, which can be supported but not proven by the mode of writing. Problems
involving individual letters will be dealt with below in the phonology.

The Lycian alphabet was used to write the Lycian language. It was an extension of the Greek alphabet, with half a dozen additional letters for sounds not found in Greek. It was largely similar to the Lydian and the Phrygian alphabets.

 
\subsection{Lycian Alphabet} 

{\let\a\arial
\begin{longtable}{>{\lycian\Large}c c c p{5.5cm}}
\toprule
\a Lycian &Transliteration	&Sound 	&Notes\\
\a Letter &                  &IPA    &      \\
\midrule
𐊀	&a	&[a]	&\\
\midrule
𐊂	&b	&[β]	&\\
\midrule
𐊄	&g	&[$\gamma$]	&\\
\midrule
𐊅	&d	&[ð]	&\\
\midrule
𐊆	&i	&[i], [ĩ]	&\\
\midrule
𐊇	&w	&[w]	&\\
\midrule
𐊈	&z	&[t͡s]	&\\
\midrule
𐊛	&h	&[h]	&\\
\midrule
𐊉	&θ	&[θ]	&\\
\midrule
𐊊	&j or y	&[j]	&\\
\midrule
𐊋	&k	&[kʲ]	&[ɡʲ] after nasals\\
\midrule
𐊍	&l	&[l] and [l̩]~[əl]	\\
\midrule
𐊎	&m	&[m]	&\\
\midrule
𐊏	&n	&[n]	&\\
\midrule
𐊒	&u	&[u], [ũ] &\\	
\midrule
𐊓	&p	&[p]	  &[b] after nasals\\
\midrule
𐊔	&κ	&[k]? [kʲ]? [h(e)]	&\\
\midrule
𐊕	&r	&[r] and [r̩]~[ər]	&\\
\midrule
𐊖	&s	&[s]	&\\
\midrule
𐊗	&t	&[t]	&[d] after nasals. ñt is [d] as in {\lycian 𐊑𐊗𐊁𐊎𐊒𐊜𐊍𐊆𐊅𐊀}/ Ñtemuχlida for Greek Δημοκλείδης / Dēmokleídēs.[3]\\
\midrule
𐊁	&e	&[e]	&\\
\midrule
𐊙	&ã	&[ã]	& {\lycian 𐊍𐊒𐊖𐊙𐊗𐊕𐊀}/ Lusãtra for Greek Λύσανδρος / Lúsandros.[4]\\
\midrule
𐊚	&ẽ	&[ẽ]	&\\
\midrule
𐊐	&m̃	&[m̩], [əm], [m.]	&originally perhaps syllabic [m], later coda [m]\\
\midrule
𐊑	&ñ	&[n̩], [ən], [n.]	&originally perhaps syllabic [n], later coda [n]\\
\midrule
𐊘	&τ	&[tʷ]? [t͡ʃ]?	&\\
\midrule
𐊌	&q	&[k]	&[ɡ] after nasals\\
\midrule
𐊃	&β	&[k]? &[kʷ]?	voiced after nasals\\
\midrule
𐊜	&χ	&[q]	&[ɢ] after nasals\\
\bottomrule
\end{longtable}
}

 
\begin{scriptexample}[]{Lydian}
\unicodetable{lydian}{"10280,"10290}

Typeset with the \idxfont{Aegean.ttf} and the command \cmd{\lycian}
\end{scriptexample}






\printunicodeblock{./languages/lycian.txt}{\lycian}