\cxset{chapter format=block}
%\cxset{steward,
%  offsety=0cm,
%  image={europa.jpg},
%  texti={An introduction to the use of font related commands. The chapter also gives a historical background to font selection using \tex and \latex. },
%  textii={In this chapter we discuss keys that are available through the \texttt{phd} package and give a background as to how fonts are used
%in \latex.
% },
% pagestyle = empty,
%}

\chapter{European Alphabetic Scripts}

\section{Introduction}

Modern European alphabetic scripts are derived from or influenced by the Greek script,
which itself was an adaptation of the Phoenician alphabet. A Greek innovation was writing
the letters from left to right, which is the writing direction for all the scripts derived from or
inspired by Greek

The European alphabetic scripts and additional characters described in this chapter follow the Unicode blocks:
\medskip


\begin{center}
\begin{tabular}{lll}
\hyperref[s:latin]{Latin} 
& \hyperref[s:cyrillic]{Cyrillic} 
& \hyperref[s:georgian]{Georgian}\\
  \hyperref[s:greek]{Greek} 
& \hyperref[s:glagolitic]{Glagolitic}
&Modifier letters\\
   \hyperref[s:coptic]{Coptic} 
&  \hyperref[s:armenian]{Armenian} 
&Combining marks\\
\end{tabular}
\end{center}

\section{Latin Script}
\label{s:latin}
Latin script, or Roman script, is an alphabet based on the letters of the classical Latin alphabet. It is used as the standard method of writing in most Western and Central European languages, as well as many languages from other parts of the world. Latin script is the basis for the largest number of alphabets of any writing system[1] and is the most widely adopted writing system in the world (commonly used by about 70\% of world's population). It is also the basis of the International Phonetic Alphabet. The 26 most widespread letters are the letters contained in the ISO basic Latin alphabet.

The script is either called Roman script or Latin script, in reference to its origin in ancient Rome. In the context of transliteration the term "romanization" or "romanisation" is often found.[2][3] Unicode uses the term "Latin"[4] as does the International Organization for Standardization (ISO).[5] The numerals are called Roman numerals.


\subsection{Ligatures}

\newfontfamily\pan{code2000.ttf}

Ligatures for the Latin script are found in the Unicode block Alphabetic Presentation Forms which contains standard ligatures for the Latin, Armenian, and Hebrew scripts.

\begin{scriptexample}[]{Ligatures}
\unicodetable{pan}{"FB00,"FB10,"FB20,"FB30,"FB40}
\end{scriptexample}

\newfontfamily\georgian[Script=Georgian,Scale=1.2]{code2000.ttf}
\newfontfamily\georgianarial[Script=Georgian,Scale=1.2]{Arial Unicode MS}

\section{Georgian}
\label{s:georgian}
The Georgian scripts are the three writing systems used to write the Georgian language: Asomtavruli, Nuskhuri and Mkhedruli. Their letters are equivalent, sharing the same names and alphabetical order and all three are unicameral (make no distinction between upper and lower case). Although each continues to be used, Mkhedruli (see below) is taken as the standard for Georgian and its related Kartvelian languages\footnote{Unicode Standard, V. 6.3. U10A0, p. 3}. 

\bgroup
\topline



\begin{scriptexample}[]{}
\georgian 

\centering
 
ყველა ადამიანი იბადება თავისუფალი და თანასწორი თავისი ღირსებითა და უფლებებით. მათ მინიჭებული აქვთ გონება და სინდისი და ერთმანეთის მიმართ უნდა იქცეოდნენ ძმობის სულისკვეთებით.
\medskip

\georgianarial
ყველა ადამიანი იბადება თავისუფალი და თანასწორი თავისი ღირსებითა და უფლებებით. მათ მინიჭებული აქვთ გონება და სინდისი და ერთმანეთის მიმართ უნდა იქცეოდნენ ძმობის სულისკვეთებით.
\bottomline
\captionof{table}{Article 1 of the Universal Declaration of Human Rights in Georgian, typeset in \texttt{code2000} (top) and \texttt{Arial Unicode MS } (bottom).}

\end{scriptexample}

The scripts originally had 38 letters. Georgian is currently written in a 33-letter alphabet, as five of the letters are obsolete in that language. The Mingrelian alphabet uses 36: the 33 of Georgian, one letter obsolete for that language, and two additional letters specific to Mingrelian and Svan. That same obsolete letter, plus a letter borrowed from Greek, are used in the 35-letter Laz alphabet. The fourth Kartvelian language, Svan, is not commonly written, but when it is it uses the letters of the Mingrelian alphabet, with an additional obsolete Georgian letter and sometimes supplemented by diacritics for its many vowels.

\chapter{Albanian}

Albania's national culture came into being at the crossroads of three great civilizations:
that of Latin Catholicism from the West, that of Byzantine Greek Orthodoxy from the south, and
that of Islam imported by the Ottoman Turks, who had invaded the country in the late 14th
century and who ruled it until the declaration of independence in 1912. Early writing in this tiny
Balkan country, very much a product of these three extremely diverse cultures, was as a result a
hybrid creation. 

\section{Elbasan}
\label{s:elbasan}
\newfontfamily\elbasan{Albanian.otf}
The Elbasan script is a mid 18th-century alphabetic script used for the Albanian language. It was named after the city of Elbasan where it was invented. It was mainly used in the area of Elbasan and Berat. It is widely considered to be the first original alphabet developed for transcribing the Albanian language.

The primary document associated with the alphabet is the Elbasan Gospel Manuscript, known in Albanian as the Anonimi i Elbasanit (The Anonymous of Elbasan).[1] The document was created at St. Jovan Vladimir's Church in central Albania, but is preserved today at the National Archives of Albania in Tirana. Its 59 pages contain Biblical content written in an alphabet of 40 letters,[1] of which 35 frequently recur and 5 are rare. Dots are used on three characters as inherent features of them to indicate varied pronunciation (pre-nasalization and gemination) found in Albanian. The script generally uses Greek letters as numerals with a line on top.

Another original script used for Albanian, was Beitha Kukju's script of the 19th century. This script did not have much influence either.

Elbasan is a simple alphabetic script written from left to right horizontally. The alphabet consists of forty letters.

\subsection{Accents and Other Marks}

The Elbasan manuscript contains breathing accents, similar to
those used in Greek. Those accents do not appear regularly in the orthography and have
not been fully analyzed yet. Raised vertical marks also appear in the manuscript, but are
not specific to the script. Generic combining characters from the Combining Diacritical
Marks block can be used to render these accents and other marks.

\subsection{Names}

The names used for the characters in the Elbasan block are based on those of the
modern Albanian alphabet.

\subsection{Numerals and punctuation}

There are no script-specific numerals or punctuation marks.
A separating dot and spaces appear in the Elbasan manuscript, and may be rendered with
U+00B7 middle dot and U+0020 space, respectively. For numerals, a Greek-like system
of letter and combining overline is in use. Overlines also appear above certain letters in
abbreviations, such as $\overline{\text{\elbasan\char "10507\char"1051D}}$ to indicate Zot (Lord). The overlines in numerals and abbreviations
can be represented with U+0305 combining overline. (See also \href{http://www.unicode.org/charts/}{unicode charts}.)

\subsection{unicode}

Elbasan is a Unicode block containing the historic Elbasan characters for writing the Albanian language. Free fonts for personal use can be found at \href{http://www.fontspace.com/category/unicode\%20font\%20for\%20elbasan}{fontspace}, which I have used here. Commercial fonts can be found at Evertype.

\begin{scriptexample}[]{Elbasan}
\unicodetable{elbasan}{"10500,"10510,"10520}
\end{scriptexample}
\chapter{Armenian}

\label{s:armenian}\index{Armenian}\index{scripts>Armenian}

As we present the scripts in alphabetic order, the first script we will typeset is in Armenian. There are many fonts available for the language. We use two in the example, the first one is \textit{FreeSans} and the second is \textit{Sylphaen} which is found on Windows Operating systems. The language is not supported by the \pkg{Babel} and partially supported by the \pkgname{Polyglossia}. \tcbdocmarginnote{china revision}

\def\ucfirst#1#2;{\MakeUppercase#1#2}


\def\armeniantest#1#2{
  {\parindent0pt
  \topline \vskip3pt
  \noindent\mbox{
     \ucfirst#1;\hfill\hbox{[\texttt{U+0530-U+058F}]}
  }}
 \nobreak

\begin{minipage}{0.45\textwidth}
\bgroup
%\setotherlanguage{#1}
\begin{#1}
#2
[\today]
\end{#1}
\egroup
\end{minipage}\hspace*{1em}
\begin{minipage}{0.45\textwidth}
\bgroup
  \parindent0pt
  \ttfamily\raggedright
  \string\documentclass\{article\}\par
  \string\usepackage[no-math]\{fontspec\}\\
  \string\newfontfamily\textbackslash#1font[Script=\ucfirst #1;,\\   ~~~~~~~Scale=MatchLowercase]
\{FreeSans\}\par
  \string\begin\{document\}\\
  \string\setotherlanguage\{#1\}\\
  \string\begin\{#1\}\\
  \egroup
\begin{#1}
\hskip10pt\vbox{#2}
\end{#1}
\bgroup
  \ttfamily[\detokenize{\today}]\\
  \string\end\{#1\}\\
  \string\end\{document\}
\egroup
\end{minipage}


\textit{FreeSans}: \url{ http://www.gnu.org/software/freefont/}
}

\armeniantest{armenian}{Բոլոր մարդիկ ծնվում են ազատ ու հավասար իրենց
արժանապատվությամբ ու իրավունքներով։       
Նրանք ունեն բանականություն ու խիղճ և միմյանց
պետք է եղբայրաբար վերաբերվեն։}

The Armenian script was invented around 407 AD, by Mesrop Maštoc, a cleric who wanted to 
translate Greek and Syriac scriptures and liturgical texts into Armenian. The system he devised 
is a pure alphabet, closely modelled on the traditional order of Greek phonetic values, with 
additional graphemes to represent Armenian sounds not found in Greek. The orthography is, 
phonetically, a near perfect representation of the Armenian language, and has remained almost 
entirely unchanged since its invention. In recent times, the letterforms in many Armenian 
typefaces have consciously modelled Latin types in their treatment of serifs, stroke weight and 
stress, and other details. This is the approach that Geraldine adopted for the Sylfaen Armenian, 
in order to harmonise the different scripts within the font. 

This kind of harmonisation has to be 
very carefully handled, as there is, of course, a point at which one can corrupt the normative 
letterforms and produce something which will be unacceptable to native readers. Once again, 
we sought expert review of the design, this time from Manvel Shmavonyan, an Armenian type designer, and his Russian colleague Vladimir Yefimov at 
ParaType in Moscow.

\bgroup
\medskip
\fontspec[Script=Armenian,Scale=1.7]{Sylfaen}
\centering

Աա Բբ Գգ Դդ Եե Զզ Էէ Ըը Թթ Ժժ Իի \\
Լլ Խխ Ծծ Կկ Հհ Ձձ Ղղ Ճճ Մմ Յյ Նն \\
Շշ Ոո Չչ Պպ Ջջ Ռռ Սս Վվ Տտ Րր Ցց \\
Ււ Փփ Քք Օօ Ֆֆ / և ﬓ ﬔ ﬕ ﬖ ﬗ\\
\egroup
\captionof{table}{Armenian, showing the basic alphabet (typeset using the \textit{Sylfaen} font.}
\medskip

\bgroup
\def\m#1 #2 #3\\{\makebox[2em]{#1}\makebox[2em]{{\fontspec{code2000.ttf}#2}}\makebox[2em]{\hfill#3 \\ }}
\fontspec[Script=Armenian,Scale=1.1]{Sylfaen}

\begin{multicols}{4}
\m Ա	A	1\\
\m Բ	B	2\\
\m Գ	G	3\\
\m Դ	D	4\\
\m Ե	E	5\\
\m Զ	Z	6\\
\m Է	ē	7\\
\m Ը	ə	8\\
\m Թ	tʿ	9\\
\m Ժ	ž	10\\
\m Ի	I	20\\
\m Լ	L	30\\
\m Խ	X	40\\
\m Ծ	C	50\\
\m Կ	K	60\\
\m Հ	H	70\\
\m Ձ	J	80\\
\m Ղ	ł	90\\
\m Ճ	č	100\\
\m Մ	M	200\\
\m Յ	Y	300\\
\m Ն	N	400\\
\m Շ	š	500\\
\m Ո	O	600\\
\m Չ	čʿ	700\\
\m Պ	P	800\\
\m Ջ	ǰ	900\\
\m Ռ	ṙ	1000\\ 
\m Ս	S	2000\\
\m Վ	V	3000\\
\m Տ	T	4000\\
\m Ր	R	5000\\
\m Ց	cʿ	6000\\
\m Ւ	W	7000\\
\m Փ	pʿ	8000\\
\m Ք	kʿ	9000\\

\end{multicols}
\captionof{table}{Armenian Numerals \textit{(from Wikipedia).}
The first column is the classical Armenian numeral, the second the transliteration and the third the arabic numeral it represents.}

\medskip

Numbers in the Armenian numeral system are obtained by simple addition. Armenian numerals are written left-to-right (as in the Armenian language). Although the order of the numerals is irrelevant since only addition is performed, the convention is to write them in decreasing order of value.

\begin{align*}
\text{ՌՋՀԵ} &= 1975 = 1000 + 900 + 70 + 5\\
\text{ՍՄԻԲ} &= 2222 = 2000 + 200 + 20 + 2\\
\text{ՍԴ}   &= 2004 = 2000 + 4\\
\text{ՃԻ}   &= 120 = 100 + 20\\
\text{Ծ}    &= 50
\end{align*}

To write numbers greater than 9999, it is necessary to have numerals with values greater than 9000. This is done by drawing a line over them, indicating their value is to be multiplied by 10000:

\begin{align*}
\overline{\text{Ա}} &= 10000\\
\overline{\text{Ջ}} &= 9000000\\
\overline{\text{ՌՃԽԳ}}\text{ՌՄԾԵ} &= 11431255
\end{align*}
\egroup

\cxset{image=greek-men}

\parindent1em

\chapter{Greek}
\epigraph{The Pleiads have left the sky, and\\
the moon has vanished. It’s midnight:\\
the time for meeting is over.\\
And me—I am lying, lonely}{Sappho}
\label{s:greek}
\index{languages>Greek}\index{Herodotus}\index{alphabets>Greek}



\enquote{The Phoenicians who came with Kadmos,} wrote Herodotus in the fifth century BC of the legendary Phoenician prince of Tyre and brother of Europa, ``\ldots introduced into Greece, after their settlement in the country, a number of accomplishments of which the most important was writing, an art which probably was unknown to the Greeks until then''. 

A basis for the remarkable history of the Greek language is
the invention of the Greek alphabet. It was modelled after
Semitic scripts, with the important improvement that not only
consonants but also vowels are represented by independent
letters.\footnote{Cover image, from \href{http://www.pappaspost.com/todays-undesirable-muslims-were-yesteryears-greeks-pure-american-no-rats-no-greeks/}{papaspost.com}, showing Greek immigrants arriving at Ellis Island in 1911.}

The poet Sappho had access to an alphabetic script, invented
for the Greek language just a couple of hundred years before her
time. The Greek alphabet is very similar to the Latin one, which
is the one used for English. In fact, the Latin alphabet is derived
from a variant of the Greek one.The similarity is easy to observe.
Here is the original poem, written in the Greek alphabet:\footcite{janson:2002}


\begin{center}
\arial 
ΔΕΔΗΚΕ ΜΕΝ Α ΣΕΛΑΝΝΑ\\
ΚΑΙ ΠΛΕΙΑΔΕΣ. ΜΕΣΑΙ ΔΕ\\
ΝΥΚΤΕΣ. ΠΑΡΑ Δ'ΕΡΧΕΤ'ΩΡΑ.\\
ΕΓΩ ΔΕ ΜΟΝΑ ΚΑΤΕΥΔΩ\\
\end{center} 

Transcribed in the Latin alphabet:

\begin{center}
DEDUKE MEN A SELANNA\\
KAI PLEIADES. MESAI DE\\
NUKTES. PARA D’ ERKHET’ ORA.\\
EGO DE MONA KATEUDO.\\
\end{center}

The Greek alphabet is the script that has been used to write the Greek language since the 8th century BC. It was derived from the earlier Phoenician alphabet, and was in turn the ancestor of numerous other European and Middle Eastern scripts, including Cyrillic and Latin.[3] Apart from its use in writing the Greek language, both in its ancient and its modern forms, the Greek alphabet today also serves as a source of technical symbols and labels in many domains of mathematics, science and other fields.

In its classical and modern forms, the alphabet has 24 letters, ordered from alpha to omega. Like Latin and Cyrillic, Greek originally had only a single form of each letter; it developed the letter case distinction between upper-case and lower-case forms in parallel with Latin during the modern era.

\tex has built-in commands for the usage of the Greek alphabet see section \ref{greek} in the Symbols chapter.

\bgroup
\obeylines
\greek\obeyspaces

Α	ἄλφα	aleph	alpha	[alpʰa]	[ˈalfa]	Listeni/ˈælfə/
Β	βῆτα	beth	beta	[bɛːta]	[ˈvita]	/ˈbiːtə/, US /ˈbeɪtə/
Γ	γάμμα	gimel	gamma	[ɡamma]	[ˈɣama]	/ˈɡæmə/
Δ	δέλτα	daleth	delta	[delta]	[ˈðelta]	/ˈdɛltə/
Η	ἦτα	  heth	   eta	 [hɛːta], [ɛːta]	[ˈita]	/ˈiːtə/, US /ˈeɪtə/
Θ	θῆτα	teth	theta	[tʰɛːta]	[ˈθita]	/ˈθiːtə/, US Listeni/ˈθeɪtə/
Ι	ἰῶτα	yodh	iota	[iɔːta]	[ˈʝota]	Listeni/aɪˈoʊtə/
Κ	κάππα	kaph	kappa	[kappa]	[ˈkapa]	Listeni/ˈkæpə/
Λ	λάμβδα	lamedh	lambda	[lambda]	[ˈlamða]	Listeni/ˈlæmdə/
Μ	μῦ	mem	mu	[myː]	[mi]	Listeni/ˈmjuː/; occasionally US /ˈmuː/
Ν	νῦ	nun	nu	[nyː]	[ni]	/ˈnjuː/ (US /ˈnuː/)
Ρ	ῥῶ	reš	rho	[rɔː]	[ro]	Listeni/ˈroʊ/
Τ	ταῦ	taw	tau	[tau]	[taf]	/ˈtaʊ/ or /ˈtɔː/
\egroup

With a suitable font such as |Arial Unicode MS| you do not need to do anything special to typeset short paragraphs of Greek text. Just use any editor set to encode the text in \utfviii. The example below was just cut and pasted. If you are going to write extensively in Greek it would be preferable to get a virtual keyboard. If you are using windows these are pre-build. 

\topline
\begin{quote}
Ἡροδότου Ἁλικαρνησσέος ἱστορίης ἀπόδεξις ἥδε, ὡς μήτε τὰ γενόμενα ἐξ ἀνθρώπων τῷ χρόνῳ ἐξίτηλα γένηται, μήτε ἔργα μεγάλα τε καὶ θωμαστά, τὰ μὲν Ἕλλησι, τὰ δὲ βαρβάροισι ἀποδεχθέντα, ἀκλεᾶ γένηται, τὰ τε ἄλλα καὶ δι' ἣν αἰτίην ἐπολέμησαν ἀλλήλοισι.[2]

Herodotus of Halicarnassus, his Researches are set down to preserve the memory of the past by putting on record the astonishing achievements of both the Greeks and the Barbarians; and more particularly, to show how they came into conflict.[3]
\end{quote}
\bottomline

\begin{scriptexample}{greek}
\unicodetable{greek}{% 
"0370,"0380,"0390,"03A0,"03B0,"03C0,"03D0,"03E0,"03F0}
\end{scriptexample}

\subsection{Greek diacritics}
\index{Greek>polytonic}

The ancient Greek writing included for many accents and diacritics. The extended unicode standard provides slots for all diacritics. Greek orthography has used a variety of diacritics starting in the Hellenistic period. The complex polytonic orthography notates Ancient Greek phonology. The simple monotonic orthography, introduced in 1982, corresponds to Modern Greek phonology, and requires only two diacritics.

Polytonic orthography (πολύς "much", "many", τόνος "accent") is the standard system for Ancient Greek. The acute accent ( ´ ), the grave accent ( ` ), and the circumflex ( ῀ ) indicate different kinds of pitch accent. The rough breathing ( ῾ ) indicates the presence of an /h/ sound before a letter, while the smooth breathing ( ᾿ ) indicates the absence of /h/.

Since in Modern Greek the pitch accent was replaced by a dynamic accent, and the /h/ was lost, most polytonic diacritics have no phonetic significance, and merely reveal the underlying Ancient Greek etymology.

Monotonic orthography (μόνος "single", τόνος "accent") is the standard system for Modern Greek. It retains a single accent or tonos (΄) to indicate stress and the diaeresis (¨) to indicate a diphthong: compare modern Greek παϊδάκια /pajˈðaca/ "lamb chops", with a diphthong, and παιδάκια /peˈðacia/ "little children" with a simple vowel. Tonos and diaeresis can be combined on a single vowel, as in the verb ταΐζω (/taˈizo/ "to feed").

\medskip
\begin{scriptexample}[]{Greek}
\unicodetable{greek}{%
"1F00,"1F10,"1F20}
\end{scriptexample}

%%%%%%%%%%%%%%%%%%%%%%%%%%%%%%%%%%%%
%    Greek Language
%%%%%%%%%%%%%%%%%%%%%%%%%%%%%%%%%%%%

%\documentclass{book}
%\usepackage{phd}
%\usepackage{philokalia}
%\begin{document}

\subsection{Philokalia}

The \pkgname{philokalia} package by Apostolos Syropoulos provides a Greek font in the style of the Philokalia manuscripts. The package modifies the lettrine package, which we cater for in the \pkgname{phd} and hence we adjusted it slightly for this. Also the package needed some modifications to work with LuaTeX.

The Philokalia (Ancient Greek: φιλοκαλία "love of the beautiful, the good", from φιλία philia "love" and κάλλος kallos "beauty") is "a collection of texts written between the 4th and 15th centuries by spiritual masters"[1] of the Eastern Orthodox hesychast tradition. They were originally written for the guidance and instruction of monks in "the practise of the contemplative life".[2] The collection was compiled in the eighteenth-century by St. Nikodemos of the Holy Mountain and St. Makarios of Corinth.

Although these works were individually known in the monastic culture of Greek Orthodox Christianity before their inclusion in The Philokalia, their presence in this collection resulted in a much wider readership due to its translation into several languages. The earliest translations included a Church Slavonic translation of selected texts by Paisius Velichkovsky (Dobrotolublye) in 1793, a Russian translation by Ignatius Bryanchaninov in 1857, and a five-volume translation into Russian (Dobrotolyubie) by St. Theophan the Recluse in 1877.

There were subsequent Romanian, Italian and French translations.[3][4]
The book is a "principal spiritual text" for all the Eastern Orthodox Churches;[5] the publishers of the current English translation state that "The Philokalia has exercised an influence far greater than that of any book other than the Bible in the recent history of the Orthodox Church."[6]
Philokalia (sometimes Philocalia) is also the name given to an anthology of the writings of Origen compiled by Saint Basil the Great and Saint Gregory Nazianzus. Other works on monastic spirituality have also used the same title over the years.[5][7]

The Philokalia fonts consist of three fonts: one that contains
the normal typeface, one that contains the ligatures and one that contains the special ornament characters that decorate the beginning of each chapter. The glyphs were generated from scanned images of the book pages and Apostolos Syropoulos described the process in detail in \cite{syropoulos}. 


{
%\newfontfamily\plk{Philokalia-Regular}
\plk
%\newfontfamily\PHtitl[Script=Greek,RawFeature=+titl;grek]{Philokalia-Regular}
 %\font\PHtitl="[Philokalia-Regular]/ICU:script=grek,+titl"

 
 \lettrine[lines=3]{\usebox{\philobox}}{ερὶ} ποιητικῆς αὐτῆς τε καὶ τῶν εἰδῶν αὐτῆς, ἥν τινα δύναμιν ἕκαστον ἔχει, 
καὶ πῶς δεῖ συνίστασθαι τοὺς μύθους  εἰ μέλλει καλῶς ἕξειν ἡ ποίησις, ἔτι δὲ ἐκ πόσων καὶ ποίων 
ἐστὶ μορίων, ὁμοίως δὲ καὶ περὶ τῶν ἄλλων ὅσα τῆς αὐτῆς ἐστι μεθόδου, λέγωμεν ἀρξάμενοι κατὰ φύσιν 
πρῶτον ἀπὸ τῶν πρώτων.
 
Ἐποποιία δὴ καὶ ἡ τῆς τραγῳδίας ποίησις ἔτι δὲ κωμῳδία καὶ ἡ διθυραμβοποιητικὴ καὶ τῆς αὐλητικῆς 
ἡ πλείστη καὶ κιθαριστικῆς πᾶσαι τυγχάνουσιν οὖσαι μιμήσεις τὸ σύνολον· διαφέρουσι δὲ ἀλλήλων τρισίν, 
ἢ γὰρ τῷ ἐν ἑτέροις μιμεῖσθαι ἢ τῷ ἕτερα ἢ τῷ ἑτέρως καὶ μὴ τὸν αὐτὸν τρόπον. 

Ὥσπερ γὰρ καὶ χρώμασι καὶ σχήμασι πολλὰ μιμοῦνταί τινες ἀπεικάζοντες (οἱ μὲν [20] διὰ τέχνης οἱ δὲ διὰ συνηθείας),
ἕτεροι δὲ διὰ τῆς φωνῆς, οὕτω κἀν ταῖς εἰρημέναις τέχναις ἅπασαι μὲν ποιοῦνται τὴν μίμησιν ἐν ῥυθμῷ καὶ λόγῳ καὶ
ἁρμονίᾳ, τούτοις δ᾽ ἢ χωρὶς ἢ μεμιγμένοις· οἷον ἁρμονίᾳ μὲν καὶ ῥυθμῷ χρώμεναι μόνον ἥ τε αὐλητικὴ καὶ ἡ κιθαριστικὴ
κἂν εἴ τινες [25] ἕτεραι τυγχάνωσιν οὖσαι τοιαῦται τὴν δύναμιν, οἷον ἡ τῶν συρίγγων, αὐτῷ δὲ τῷ ῥυθμῷ [μιμοῦνται]
χωρὶς ἁρμονίας ἡ τῶν ὀρχηστῶν (καὶ γὰρ οὗτοι διὰ τῶν σχηματιζομένων ῥυθμῶν μιμοῦνται καὶ ἤθη καὶ πάθη καὶ πράξεις)· 
 }

The package also modifies the \pkgname{lettrine} package and hence we have modified the \cmd{\lettrine} command to be called \cmd{\lettrinephilokalia} when used with the |philokalia| package. It is a bit long as a command, but easier to remember. 



\section{Greek-derived scripts}

Because of Greece’s military (Alexander the Great), economic
and cultural influence, the Greek alphabet became the
prototype for the ‘complete’ (that is, fully vowelized) alphabets
that emerged in Europe in the following centuries. These eventually
diffused, almost exclusively through Greek’s granddaughter
alphabets Latin and Cyrillic, throughout the entire
world --- a process still going on over two thousand years later
(illus. \ref{fig:greekderived}).

In first-millenium BC Asia Minor (today’s Turkey), the
Greek alphabet inspired an impressive number of non-Greek
peoples to elaborate their own Anatolian alphabets: \nameref{carian},
\nameref{sec:lydian}, \nameref{sec:lycian}, Pamphylian, Phrygian, Pisidian (of the Roman
period) and Sidetic.  Nonetheless, these scripts failed to
acquire lasting significance because of the region’s declining
economic fortunes followed by several major invasions.

%\documentclass{article}
%\usepackage[margin=1cm]{geometry}
%\usepackage{pdflscape}
%\usepackage{forest}
%\usepackage{hyperref}
%\usetikzlibrary{shadows,arrows}
\newgeometry{left=1cm,right=1cm,bottom=1cm}
\newpage

\tikzset{parent/.style={align=center,text width=2cm, fill=blue!40,rounded corners=2pt,inner sep=2pt},
    child/.style={align=center,text width=2.0cm,fill=orange!60,rounded corners=2pt,inner sep=1pt,outer sep=0pt},
    grandchild/.style={fill=white,text width=1.7cm}
}

%\begin{document}

\begin{landscape}
\begin{forest}
for tree={%
    thick,
    drop shadow,
    l sep=1.0cm,
    s sep=0.6cm,
    node options={draw,font={\rmfamily\small}},
    edge={semithick,-latex},
    where level=0{parent}{},
    where level=1{
        minimum height=0.8cm,
        child,
        parent anchor=south west,
        tier=p,
        l sep=0.25cm,
        for descendants={%
            grandchild,
            minimum height=0.6cm,
            %l sep=0.5cm,
%            s sep=0.5cm,
            anchor=115,
            edge path={
                \noexpand\path[\forestoption{edge}]
                (!to tier=p.parent anchor) |-(.child anchor)\forestoption{edge label};
            },
        }
    }{},
}
[(Phoenician)\\ GREEK
    [Palaeo-Hispanic %heading
        [North-east\\
         Celtiberian
            [South-West\\
                South-east
            ]
        ]
    ],
    [Etruscan
        [LATIN
            [\textit{Rhaetian} 
                [\href{http://en.wikipedia.org/wiki/Gallic}{Gallic}
                    [\href{http://en.wikipedia.org/wiki/Venetic}{Venetic}
                      [Faliscan
                        [Northern Picene
                          Southern Picene\\
                            [Oscan
                              [Umbrian]
                            ]
                          ]
                       ]
                    ]
                ]
            ]
        ]
    ]
    [\href{http://en.wikipedia.org/wiki/Gothic_language}{Gothic}]
    [Glagolithic
       [Croatian]
     ]   
    [Cyrillic
        [Russian
         [Ukrainian
            [Bulgarian
             [Serb]
            ] 
        ]  
     ]  
   ]  
  [Anatolian
    [Carian
      [Lydian
        [Lykian
          [Pamphylian
            [Phrygian
              [Pisidian
                [Sidetic]
            ]
          ]
        ]
      ]
  ]
  ]
  ]
  [Armenian]
  [Georgian]
  [Coptic
    [Nubian]
  ]
]
\end{forest}
\captionof{figure}{Abridged family tree of some Greek-derived scripts.}
\label{fig;greekderived}
\end{landscape}


\restoregeometry
\newpage


%\end{document}

The Armenian monk St Mesrob (c. 345–440) is said to have
elaborated the Armenians’ first script c. AD 405 – Armenian is a
separate branch of the Indo-European superfamily of languages
(to which Greek and Germanic, which includes English, also
belong). Based on the Greek alphabet, the Armenian script
originally consisted of around 36 mainly capital letters. By the
1200s, Armenian notrgir, or cursive writing, had been developed,
then replacing writing in capitals (illus. 100).

St Mesrob is also credited with devising the Georgian alphabet
in the early 400s AD – Georgian is a Caucasian, not an Indo-
European, language – as well as the Albanian alphabet. (Such
multiple attributions suggest that Mesrob’s role was apocryphal.)

The ecclesiastical Georgian script used 38 letters; over
time, several styles of writing Georgian developed, with varying
numbers of letters (illus. 101). The mkhedruli, or ‘lay hand’,
which began as a medium for non-sacral texts, is Georgian’s
most frequently employed script, still in use today.

In Egypt, the Greek alphabet inspired the Coptic
alphabet that replaced one of the world’s oldest writing traditions.
In the Balkans, Greek generated the Glagolitic and
Cyrillic scripts, which eventually generated the Russian script,









\newfontfamily\glagolitic{MPH 2B Damase}

\section{Glagolitic}

\epigraph{The average Ph.D. thesis is nothing but a transference of bones from one graveyard to another.}{%
J. Frank Dobie (1888-1964)}


\label{s:glagolitic}
\fboxrule0pt\fboxsep0pt

\noindent
The Glagolitic alphabet /{\glagolitic ˌɡlæɡɵˈlɪtɨk/}, also known as Glagolitsa, is the oldest known Slavic alphabet, from the 9th century.

It was created in the 9th century by Saint Cyril, a Byzantine monk from Thessaloniki. He and his brother, Saint Methodius, were sent by the Byzantine Emperor Michael III in 863 to Great Moravia to spread Christianity among the West Slavs in the area. The brothers decided to translate liturgical books into the Old Slavic language that was understandable to the general population, but as the words of that language could not be easily written by using either the Greek or Latin alphabets, Cyril decided to invent a new script, Glagolitic, which he based on the local dialect of the Slavic tribes from the Byzantine Salonika region.
After the deaths of Cyril and Methodius, the Glagolitic alphabet ceased to be used in Moravia, but their students continued to propagate it in the west and south. 

After a long career, Glagolitic writing stopped being used, except for
religious purposes in certain dioceses of Bosnia and Dalmatia (Croatia).
The Cyrillic alphabet was adopted by all Orthodox Slays and served to note
their literary language. Most of the Slays who rallied to Rome rejected it,
however, which created the paradoxical situation in ex-Yugoslavia, where
two peoples who speak the same language write in different scripts, the
Serbs in Cyrillic and the Croats with Roman characters. Finally, as is
known, the ex-Soviet Union did much to put into writing the languages
spoken by the peoples within its borders, for the most part noting them in
adaptations of the Cyrillic alphabet, while Russian became the language of
culture throughout the Soviet Union.\cite{henri1994}

Slavic printing in Glagolitic characters originated in Venice, where a
\textit{Sluzebnik} (or \textit{Leitourgikon}) was published in 1483, followed by missals and
breviaries, all printed by Andrea Torresani, the future father-in-law and
associate of Aldus Manutius. After 1494 some attempts were made to create
printshops in Croatia itself, first in Senj in 1508, then, after 1530, in
Rijeka (Fiume). The work of these firms was almost totally liturgical (religious,
at any rate), and it had strong competition from manuscript works
that were better adapted to the diversity of local liturgical customs. Religion
also dictated the output of a printshop founded to provide Protestant propaganda
that was set up in Tubingen between 1560 and 1564 by Baron
Hans von Ungnad and that printed the great Lutheran texts in Glagolitic
characters.\footfullcite{henri1994}

Figure~\ref{fig:zograf} illustrates an example of the language.\footnote{\url{https://en.wikipedia.org/wiki/Glagolitic_script\#/media/File:ZographensisColour.jpg}}

\begin{figure}[htbp]
\centering

\includegraphics[width=0.45\linewidth]{glagolitic}
\caption[The first page of the Gospel of Mark from the 10th–11th century Codex Zographensis, found in the Zograf Monastery in 1843.]{The first page of the Gospel of Mark from the 10th–11th century Codex Zographensis, found in the Zograf Monastery in 1843.}
\label{fig:zograf}
\end{figure}

\section{Unicode Support}
The Glagolitic alphabet was added to the Unicode Standard in March 2005 with the release of version 4.1.
The Unicode block for Glagolitic is U+2C00–U+2C5F.



\begin{scriptexample}[]{glacolitic}

\unicodetable{glagolitic}{%
"2C00,"2C10,"2C20,"2C30,"2C40,"2C50}

\texttt{typeset with Damase version 2.0 MPH 2B Damase}
\end{scriptexample}
\bgroup
\glagolitic

The name was not coined until many centuries after its creation, and comes from the Old Church Slavonic glagolъ "utterance" (also the origin of the Slavic name for the letter G). The verb glagoliti means "to speak". It has been conjectured that the name glagolitsa developed in Croatia around the 14th century and was derived from the word glagolity, applied to adherents of the liturgy in Slavonic.[1]

In Old Church Slavonic the name is {\glagolitic ⰍⰫⰓⰊⰎⰎⰑⰂⰋⰜⰀ}, Кѷрїлловица.
The name Glagolitic in Bulgarian, Russian, Macedonian глаголица (glagolica), Belarusian is глаголіца (hłaholica), Croatian glagoljica, Serbian глагољица / glagoljica, Czech hlaholice, Polish głagolica, Slovene glagolica, Slovak hlaholika, and Ukrainian глаголиця (hlaholyća).



\egroup





