\chapter{Languages of Eastern Asia}


In this chapter we focus on languages of eastern Asia, a region
including China and \hyperref[ch:southeastasia]{Southeast Asia} and defined 
in historical terms as the area
influenced by Classical China, and we will see that contact among languages
again leads to some sharing of grammatical features, creating an impression of a
family relationship and descent from a common ancestral language.1 However,
languages of eastern Asia belong to three major language families: Sino-Tibetan,
Austro-Asiatic and \hyperref[s:tai-kadai]{Tai-Kadai}, which will be the subject of the next three sections.
In the past, Vietnamese and other languages of Southeast Asia were classified as
members of the Sino-Tibetan family; however, their similarities to Chinese are
currently credited to language contact by most linguists outside of China, though
in the Chinese scholarly community, some of the Southeast Asian languages are
still included in the Sino-Tibetan family.

There are also three languages in eastern Asia whose family relationships have
not been determined and remain controversial: Japanese, Korean and Ainu. These
languages will be considered in more detail in Chapter 11.

\section{Sino-Tibetan languages}

The Sino-Tibetan language family comprises 449 languages and is
second only to the Indo-European language family in number of speakers. The
two main branches of the Sino-Tibetan family – as suggested by the name –
are Sinitic (or Chinese) languages and Tibeto-Burman languages. The Sinitic
branch contains only 14 languages, but they are spoken by over a billion people.
Mandarin Chinese alone has some 840 million speakers (92\% of the population
in China speaks Mandarin Chinese), while ``smaller'' Chinese languages
have tens of millions speakers each: for example, Wu Chinese (also known as
Shanghainese) has over 77 million speakers, Yue (also known as Cantonese) has
\begin{figure}[htbp]
\centering

\includegraphics[width=\textwidth]{eastern-asia}

\caption{Languages of eastern Asia.}
\end{figure}
52 million speakers, Jinyu (spoken mainly in Shanxi Province) has 44 million
speakers and Min Nan Chinese (also known as Taiwanese) has 25,700,000 speakers
in China (plus 15,000,000 in Taiwan, 2,660,000 in Malaysia, 1,170,000 in Singapore,
1,080,000 in Thailand and significant communities in Brunei, Indonesia,
the Philippines and elsewhere).2 Naturally, these languages are not uniform and
have many dialects. For instance, Mandarin Chinese has many dialects organized
into four dialect groups: Northern dialects (spoken in Hebei, Henan, Shandong,
Northern Anhui, northeastern provinces and parts of Inner Mongolia), Northwestern
dialects (spoken in Shanxi, Gansu, Qinghai, Ningxia and parts of Inner
Mongolia), Southwestern dialects (spoken in Sichuan, Yunnan, Guizhou, northwest
Guangxi, Hubei and northwest Hunan) and Eastern or Jiang-Huai dialects
(spoken in central Anhui and Jiangsu north of the Yangtze River). Similarly, Wu
Chinese has a number of dialects grouped into Northern and Southern dialects;
the former have been influenced more by the neighboring Mandarin dialects,
especially in the vocabulary. Generally speaking, a higher degree of dialect differentiation
is found in the more densely populated coastal areas in comparison to the interior of China, where there is still much mutual intelligibility among the
different linguistic varieties.

Despite not being mutually intelligible, the various Chinese languages are often
considered to be the same language by the speakers because of the unifying effect
of the writing system. Thus, written language is understood throughout China by
educated, literate speakers across different languages. This is because the Chinese
use a logographic system of writing which does not represent the pronunciation
of words in the same way as alphabetic systems do (such as the Roman alphabet
used for English, French or Croatian). In fact, a Chinese character will typically
have very different sound value from language to language but will have the same
meaning across the various Chinese languages.

\paragraph{Tibeto-Burman} languages are much smaller by number of speakers.
Some of the largest include \hyperref[s:myanmar]{Burmese} (32,000,000 speakers in Myanmar plus
300,000 in Bangladesh), Khams Tibetan (1,490,000 speakers in China), Meitei
(1,370,000 speakers in India plus 15,000 in Bangladesh and 6,000 in Myanmar),
Central Tibetan (1,070,000 speakers in China plus 189,000 in India, 5,280 in
Nepal and 4,800 in Bhutan) and Amdo Tibetan (810,000 speakers in China). But
many Tibeto-Burman languages count less than a 1,000 speakers: for example,
Lunanakha and Seke are spoken by 700 speakers each (in Bhutan and in Nepal,
respectively), while Jad and Brokkat are spoken by 300 speakers each (in India
and in Bhutan, respectively).




There are also some significant linguistic differences between Sinitic and
Tibeto-Burman languages, so that some scholars, such as Christopher Beckwith
(1996) and Roy Andrew Miller, argued that these two families are not related
at all. They point to what they consider an absence of regular sound correspondences,
an absence of reconstructable shared morphology, and evidence that much
shared lexical material has been borrowed from Chinese into Tibeto-Burman. In
opposition to this view, scholars in favor of the Sino-Tibetan hypothesis, such
as W. South Coblin, Graham Thurgood and James Matisoff, have argued that
there are regular correspondences in sounds, as well as in grammar. One of the
main reasons why it is so difficult to apply the comparative method that we are
familiar with from the previous chapters to the Sino-Tibetan languages is the
morphological paucity in many of these languages, including modern Chinese
and Tibetan.

This brings us to consider one of the most characteristic features of Sino-
Tibetan languages – their isolating morphology. In an extreme isolating
language, words are composed of a single morpheme, in contrast to agglutinative
or fusional languages, where words are composed on multiple morphemes. To
illustrate, consider the following sentences from Mandarin Chinese (7.1) and
Burmese (7.2):


Few languages families anywhere are fragmented as that of Tibeto-Burman. How and why did the Tibeto-Burmans become
scattered so widely and into so many small groups? There has been much speculation about this question. Many
believe that the ancestors of teh Tibeto-Burmans lived somewhere in the western reaches of China, and that
cultural or military pressures from the Chinese precipitated a number of early migrations out of China
westward. 



\section{Tai-Kadai Languages }
\label{s:tai-kadai}\index{Tai-Kadai}
The Tai-Kadai language family (also known as Daic, Kadai, Kradai
or Kra-Dai) includes 92 languages found in southern China and Southeast Asia:
in Thailand, Vietnam, Laos and Myanmar. There are three major subfamilies of
the Tai-Kadai family. The largest subfamily is the Kam-Tai branch (also known
as Kam-Sui), which includes 76 languages; the other two branches are the Kadai
subfamily, consisting of 14 languages spoken in Vietnam and China, and the Hlai
subfamily, which includes only 2 languages (both spoken in China).

Probably the best known members of the Tai-Kadai family are \hyperref[s:thai]{Thai} and \hyperref[s:lao]{Lao},
closely related and to some degree mutually intelligible languages.10 Both Thai
and Lao are members of the Southwestern grouping within the Kam-Tai branch
of the Tai-Kadai family. Thai is spoken as a first language by 20 million speakers
and as a second language by 40 million speakers in Thailand, where it serves
as the official language of the country, medium of education and of most mass
communication. It is also spoken in the Midway Islands, Singapore, United Arab
Emirates and the United States. The word thai means ‘free’ in the Thai language;
yet, Thai people have not always been free from outside political influence and
dominance: in fact, they became independent in the mid-thirteenth century, but
prior to that they were dominated by the Mon and then later by the Khmer.

Lao is spoken by 3 million people in Laos, especially in the Mekong River
Valley and Luang Prabang south to Cambodia border. There is a closely related
linguistic variety – Isan – often referred to as Northeastern Thai; 15 million of


Isan speakers (who are ethnically Lao) live on the Khorat Plateau of northeastern
Thailand and comprise roughly a third of Thailand’s population. If Isan speakers
are considered as part of the Lao group, there is more Lao in Thailand than in
Laos. Thus, Thailand and Laos is one of the many areas in the world where
language, ethnicity and nationality do not match up.

It should be noted that while today only a few of the Tai-Kadai languages
are spoken in southern China, Chinese historical records show that Tai-Kadai
speakers used to live further north than they are found today, inhabiting a large
area of China south of the Yangtze River. This theory is further confirmed by
the fact that the diversity of the Tai-Kadai languages reaches its highest degree
in southeastern China, especially in Hunan, suggesting that this is close to their
historic homeland. The migration of Thai-Kadai speakers into the Indochina
peninsula occurred already in historical times, around 2,000 years ago. Due
to Han Chinese expansion, Mongol invasion pressures and a search for lands
more suitable for wet-rice cultivation, the Tai peoples moved south towards
India, down the Mekong River valley and as far south as the Malay Peninsula,
moving into what was formerly Austro-Asiatic territory, inter-marrying with –
and borrowing culture from – local Khmer (Cambodian) and Mon peoples.

Given this history and the relatively high levels of intermarriage between the Thais and
the Chinese, it should not come as a surprise to find strong Chinese influence on
the Thai language. As is the case with Austro-Asiatic languages, Thai developed
certain grammatical characteristics through diffusion from Chinese, among them
isolating morphology and a classifier system. Also, like Chinese and Vietnamese,
Thai is a tonal language, although once again we cannot be sure if this property
of Thai diffused from Chinese/Vietnamese, or arose independently or under areal
pressures.




















