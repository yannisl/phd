\cxset{image = mongolian-people}
\chapter{Mongolian}
The Mongols (Mongolian: Монголчууд, Mongolchuud, [ˈmɔŋ.ɡɔɮ.t͡ʃʊːt]) are an East-Central Asian ethnic group native to Mongolia and China's Inner Mongolia Autonomous Region. They also live as minorities in other regions of China (e.g. Xinjiang), as well as in Russia. Mongolian people belonging to the Buryat and Kalmyk subgroups live predominantly in the Russian federal subjects of Buryatia and Kalmykia.\footnote{Cover image from \href{http://www.bbc.com/news/world-asia-china-25979564}{bbc}}

The Mongols are bound together by a common heritage and ethnic identity. T
heir indigenous dialects are collectively known as the Mongolian language. The ancestors of the modern-day Mongols are referred to as Proto-Mongols.

Mongolian is a member of a language family technically known as “Mongolic”. Apart
from Mongolian, or Mongol proper, the Mongolic language family comprises a dozen
other languages, spoken mainly in regions adjacent to Mongolia. Historically, the
Mongolic language family was formed as a result of the political expansion of the mediaeval,
or “historical”, Mongols under Chinggis Khan (Cingges Xaan) and his descendants
in the 12th–13th centuries. During the initial period of the Mongol empire, the Mongols
controlled, as a politically unified territory, the entire Central Asian belt from the Middle
East to China. The subsequent Mongol dynasty of the Yuan (1279–1368) in the eastern
part of the former Mongol empire, comprised China, Mongolia, Manchuria, Tibet and
Eastern Turkestan.\footcite{book:janhunen2012}

The language of the historical Mongols was based on the local idiom once spoken in
northeastern Mongolia, the native region of Chinggis Khan. With the consolidation of
the political power, this idiom became the koïné of the expanding Mongols, who brought
it to various parts of the empire. The language was widely used in civil and military
administration, and through the Mongol garrisons it gained ground also among local
non-Mongol populations. As a spoken medium, the language of the historical Mongols
is known as Middle Mongol, or Middle Mongolian. Middle Mongol is documented in
a variety of written sources using several different systems of script. With the course of
time, and especially after the collapse of the Mongol empire Middle Mongol was diversified
into several local varieties, from which the modern Mongolic languages have
developed.


Janhunen\footcite{book:janhunen2012} divides the extant Mongolic languages into four geographically
and linguistically distinct branches: Dagur, Common Mongolic,
Shirongolic and Moghol.

\begin{description}

\item[Dagur] branch, located in the northeast (Manchuria) and comprising only the
Dagur language (with several local varieties, including the Amur, Nonni and Hailar
groups of dialects, as well as, since the 18th century, a diaspora group in the Yili
region of Dzungaria); historically, the origins of this branch would seem to be connected
with the earliest breakup of Proto-Mongolic;

\item[Common Mongolic] branch, centered on the traditional homeland of the Mongols
(Mongolia), but extending also to the north (Siberia), east (Manchuria), south
(Ordos) and west (Dzungaria), and comprising a group of closely related forms of
speech, which by the native speakers themselves are often understood as “dialects”
of a single “Mongolian” language;

\item[Shirongolic] branch, located in the Amdo or Kuku Nor (Xeux Noor ‘Blue Lake’)
region of ethnic Tibet (the modern Gansu and Qinghai Provinces of China), and
comprising a number of particularly idiosyncratic and mutually unintelligible
languages spoken by several culturally diversified populations, including Shira
Yughur (Mongolic Yellow Uighur), the Monguor group (Mongghul, Mongghuor,
Mangghuer) and the Bonan group (Bonan, Kangjia, Santa);

\item[Moghol] the Moghol branch, located in Afghanistan and comprising only the Moghol language
(with several local varieties, possibly extinct today).\footcite{book:janhunen2012}
\end{description}

\begin{figure}[htbp]
\includegraphics[width=\textwidth]{mongolian-writing}
\caption{Nova N 176 found in Kyrgyzstan. The manuscript (dating to the 12th century Western Liao) is written in the Mongolic Khitan language using cursive Khitan large script. It has 127 leaves and 15,000 characters.}
\end{figure}

From historical documents it is evident that the lineage represented by the language
of the historical Mongols once had relatives, today technically identified as the Para-
Mongolic languages, spoken until mediaeval times in parts of southwestern Manchuria.

\paragraph{The literary languages}
The earliest known written language for
the historical Mongols was created in the 11th–12th centuries on the basis of a Semitic
alphabet adopted via the Turkic-speaking Ancient Uighurs. The script, in its Mongolian
form, has subsequently become known as the Mongol Script, while the language written
in it is known as Written Mongol or Written Mongolian, or also Literary Mongol
or Literary Mongolian. Written Mongol was reinforced by Chinggis Khan as a general
medium of administration and literature, and in its early form it was essentially identical
with contemporary spoken Middle Mongol, complicated only by certain orthographical
conventions, some of which may actually reflect a stage preceding Middle Mongol and
Proto-Mongolic.

Written Mongol has ever since remained in use as the principal literary language of
the Mongols. Evolving successively through stages termed Pre-Classical (13th to 15th
centuries), Classical (17th to 19th centuries) and Post-Classical (20th century) Written
Mongol, the language, especially as far as its orthographical principles are concerned, still
retains many of its original characteristics. This means that it remains largely unaffected
by the innovations that have taken place in the spoken language and by the diversification
of the latter into the extant modern Mongolic languages. This is particularly true of
the phonological features reflected by the Written Mongol orthography. Written Mongol
has, however, survived only among the speakers of the Common Mongolic idioms, and
even of the latter, the speakers of Buryat and Khamnigan have used it only marginally.
The significance of Written Mongol as a unifying factor for almost all Common
Mongolic speakers can hardly be exaggerated. Even so, its status has been gradually
undermined by the creation of new literary languages, which today cover most of the
Common Mongolic populations living outside of Inner Mongolia. These new literary
languages include:

\begin{enumerate}
\item Written Oirat or the “Clear Script” (Tod Biceg), which was created on the basis of
Written Mongol as early as 1648 for use by the Western Mongols of Dzungaria; this
script is still in use among some of the Oirat groups in Sinkiang;

\item Romanized “Buryat”, which was standardized around 1930 on the basis of what are
actually the Sartul and Tsongol dialects of northern Khalkha, spoken on the Russian
side of the border

\item Cyrillic Buryat, based on the Khori dialect of actual (Eastern) Buryat, which replaced
the earlier Romanized “Buryat” in 1937 and remains in use as the literary language
for the Buryat living in the Russian Federation; the written standard is, however, not
used by the Buryat speakers living in Mongolia and China;

\item Cyrillic Kalmuck, which was standardized in the early 1930s for use by the Volga
Kalmuck, who represent an Oirat diaspora group that has been living under Russian
rule since the 17th century;

\item Cyrillic Khalkha, based on the central dialects of the Khalkha group, which were
developed as the national language of Outer Mongolia after independence, and
which during the 1940s more or less fully replaced Written Mongol as the official
standard language of the country.
\end{enumerate}

\paragraph{Classical Mongolian Script} The classical Mongolian script (in Mongolian script: {\mongolian  ᠮᠣᠩᠭᠣᠯ ᠪᠢᠴᠢᠭ᠌} Mongγol bičig; in Mongolian Cyrillic: Монгол бичиг Mongol bichig), also known as Uyghurjin Mongol bichig, was the first writing system created specifically for the Mongolian language, and was the most successful until the introduction of Cyrillic in 1946. Derived from Uighur, Mongolian is a true alphabet, with separate letters for consonants and vowels. The Mongolian script has been adapted to write languages such as Oirat and Manchu. Alphabets based on this classical vertical script are used in Inner Mongolia and other parts of China to this day to write Mongolian, Sibe and, experimentally, Evenki.
\medskip

\bgroup\par
\noindent
\colorbox{thecodebackground}{\color{black}^^A
\begin{minipage}{\textwidth}^^A
\parindent1pt
\vskip10pt
\leftskip10pt \rightskip\leftskip
\mongolian
\Large
ᠬᠦᠮᠦᠨ ᠪᠦᠷ ᠲᠥᠷᠥᠵᠦ ᠮᠡᠨᠳᠡᠯᠡᠬᠦ ᠡᠷᠬᠡ ᠴᠢᠯᠥᠭᠡ ᠲᠡᠢ᠂ ᠠᠳᠠᠯᠢᠬᠠᠨ ᠨᠡᠷ᠎ᠡ ᠲᠥᠷᠥ ᠲᠡᠢ᠂ ᠢᠵᠢᠯ ᠡᠷᠬᠡ ᠲᠡᠢ ᠪᠠᠢᠠᠭ᠃ ᠣᠶᠤᠨ ᠤᠬᠠᠭᠠᠨ᠂ ᠨᠠᠨᠳᠢᠨ ᠴᠢᠨᠠᠷ ᠵᠠᠶᠠᠭᠠᠰᠠᠨ ᠬᠦᠮᠦᠨ ᠬᠡᠭᠴᠢ ᠥᠭᠡᠷ᠎ᠡ ᠬᠣᠭᠣᠷᠣᠨᠳᠣ᠎ᠨ ᠠᠬᠠᠨ ᠳᠡᠭᠦᠦ ᠢᠨ ᠦᠵᠢᠯ ᠰᠠᠨᠠᠭᠠ ᠥᠠᠷ ᠬᠠᠷᠢᠴᠠᠬᠥ ᠤᠴᠢᠷ ᠲᠠᠢ᠃
\par
\vspace*{10pt}
\end{minipage}
}
\medskip


\paragraph{Unicode Encoding} Mongolian is a Unicode block containing characters for dialects of Mongolian, Manchu, and Sibe languages. It is traditionally written in vertical lines Text direction TDright.svg Top-Down, right across the page, although the Unicode code charts cite the characters rotated to horizontal orientation.
The block has dozens of variation sequences defined for standardized variants.
\bigskip


\unicodetable{mongolian}{"1800,"1810,"1820,"1830,"1840,"1850,"1860,"1870,"1880,"1890,"18A0}
\bigskip

\section{LaTeX}

The \pkg{montex} provides a full system including transliterations.\footcite{montex} There is no as yet support for LuaTeX and I do not see this forthcoming anytime soon. 










