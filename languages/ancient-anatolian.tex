\newfontfamily\brill{brill}
\large\brill
\section{Ancient Anatolian Alphabets}
\label{s:anatolian}
The Anatolian scripts described in this section all date from the first millenium BCE, and were used to write various ancient Indo-European languages of western and southwestern Anatolia (now Turkey). All are related to the Greek script and are probably adaptations of it. 






\section{Lydian}
\label{sec:lydian}
 Lydian script was used to write the Lydian language. That the language preceded the script is indicated by names in Lydian, which must have existed before they were written. Like other scripts of Anatolia in the Iron Age, the Lydian alphabet is a modification of the East Greek alphabet, but it has unique features. The same Greek letters may not represent the same sounds in both languages or in any other Anatolian language (in some cases it may). Moreover, the Lydian script is alphabetic.



Early Lydian texts are written both from left to right and from right to left. Later texts are exclusively written from right to left. One text is boustrophedon. Spaces separate words except that one text uses dots. Lydian uniquely features a quotation mark in the shape of a right triangle.

The first codification was made by Roberto Gusmani in 1964 in a combined lexicon (vocabulary), grammar, and text collection.

\begin{scriptexample}[]{Lydian}
\unicodetable{lydianfont}{"10920,"10930}

\medskip

Typeset with the \idxfont{Aegean.ttf} and the command \cmd{\lydian}
\end{scriptexample}

Examples of words

\bgroup\lydian
𐤬𐤭𐤠  - Ora - "Month"

𐤬𐤳𐤦𐤭𐤲𐤬𐤩  - Laqrisa - "Wall"

𐤬𐤭𐤦𐤡  - "House, Home"

\egroup

Herodotus Hdt. 1.94 
Chapter on the Lydians is well known, but in order to evaluate it properly it will be
helpful to recall exactly what it says54:

\begin{latexquotation}
The Lydians have about the same customs as the Greeks, except that the
Lydians prostitute their female children. The Lydians are the first people
we know to have coined money of silver and gold, and they were the first to
be shopkeepers. The Lydians themselves also claim the invention of the
games that both they and the Greeks now play. They say that the invention
occurred at the same time that they colonized Tyrsenia. What they say
about these things goes like this (the following is in indirect discourse):
In the reign of Atys, son of Manes, there was a terrible famine
throughout Lydia. Although in hard straits, the Lydians persevered for
some time. But finally, when there was no let-up, they sought respite,
some trying one thing and others another. It was then that they invented
dice, and astragals, and ball, and all the other kinds of games, except for
draughts. For the Lydians don't claim to have invented draughts. After
their inventions, this is what they did about the famine. Every second
day they would play, all day, so as not to want food, and on the day
between they would eat, and not play. In this way they persevered for
eighteen years. Since the evil did not abate, but pressed them even
worse, their king divided them up into two parts, by lot: the one group
for staying on, the other group to emigrate from the country. And the
king himself was to be in charge of the group that remained, while in
charge of the departing group was the king's son, whose name was
Tyrsenos. The group whose lot it was to depart from the land went down
to Smyrna and built boats. They put everything they needed into the
boats and sailed away in search of life and land; passing by many
nations, they sailed until they reached the Ombrikians, where they built
cities for themselves and they still live there today. Instead of "Lydians",
they adopted a new name from the king's son, the man who led them.
Taking their eponym from him, they were called Tyrsenoi.

Well, then, the Lydians were enslaved by the Persians\ldots
\end{latexquotation}



\chapter{Carian Language and Scripts}
\def\textcarian#1{\begingroup\bfseries\lydianfont\color{red}#1\endgroup}

\section{Background}

It is not clear when Caria and the Carians enter into ancient History.
This is dependent on equating classical Caria with the land of Karkiya/
Karkisa mentioned in Hittite sources. This supposition, eminently suitable
from a purely linguistic point of view (karkº in Karkisa, Karkiya
is practically identical to the Old Persian word for ‘Carian’, kºka-), is
complicated by the uncertainties regarding the exact location of Karkisa/
Karkiya on the map, a problem intimately bound to the complex question
of Hittite geography, a topic still subject to controversy despite the
great progress made in recent years.\footcite{adiego}

Classical writers did not agree as to the exact extent of Caria. The first mention of Caria is by Homer\footnote{Homer, Iliad, B', 867-869.} (800 BC). According to Homer the Carians lived around Miletus and the Mykale mountain\footnote{Currently in Turkey Samsun Dağı.} and the valley of Meander. 
Strabo\footnote{63 or 64 BC- c.24 AD} considered that the South border of Caria was in the area of Tralleis, south of Meander and that in the valley of Meander there were Carian living, Lydians, Ioanians and Aoelians. 
Xenophon\footnote{Xenophon, Greek, III, 2, 19.} (427-355 BC) places also Tralley in Caria, Diodorus Sicilus (is century BC)\footnote{Diodorus Sicilus was a Greek historian famous for hus monumental universalDiodorus Sicilus, XIV, 36, 3.} wrote that the area was under Ionian rule. 
Accordind to Herodotus (484-425 BC) who was born in Caria, considers that the south west borders of  history Bibliotheca historica, much of which survives, between 60 and 30 BCCaria the Ionian cities of Prieni, Myous and Miletus were part of Caria.

Broadly, one can consider that Caria was within what is now the Turkish province Muğla. The area is rich in ancient ruins, with over 100 excavated sites including the UNESCO Heritage Site of Letoon, near Fethiye.

The history of Caria is entangled with that of the Greek-speaking world; 
the cultural and religious character of the region was shaped by sustained interaction with both east and west.[1] Ionian and Dorian settlements were established along the Anatolian seaboard from the tenth century BCE onwards, and over the subsequent centuries there was sustained contact and assimilation between the \enquote{Karian} communities of the interior and the \enquote{Greek} cities along the coast. Geographically, Karia was recognised in antiquity as the area south of the Maeander River, extending east to the Salbakos Mountains (Map 1); it shared borders with Lydia to the north, Phrygia to the east, and Lykia to the southeast. As an ethnos, the “Karians” are more difficult to define; while the Karian language has now been identified as an Indo-European language of the “Luwic” subgroup, related to other Anatolian languages including Luwian and Lykian,[2] identifying this population in the archaeological record remains problematic. The cultural coherence of the region as a whole is also not assured; the limit of Karia as a geographical unit seems to have been greater than the area traditionally inhabited by the “Karian” people, who are thought to have been concentrated in the southwestern area.[3]\footnote{See Naomi Carless Unwin, \textit{What's in a Name? Linguistic Considerations in the Study of Karian Religion} \protect \url{http://brewminate.com/whats-in-a-name-linguistic-considerations-in-the-study-of-karian-religion/}}

The land of Caria lay during the first millennium BC in the southwest of Anatolia between
Lydia and Lycia. A few dozen texts in the epichoric language, mostly very short or fragmentary,
have been found in Caria itself or on objects likely to have originated there. These
are dated very approximately to the fourth to third centuries BC. There is also a very fragmentary
Carian–Greek bilingual from Athens, dated to the sixth century. By far the largest
number of Carian texts consists of tomb inscriptions and graffiti left by Carian mercenaries
in Egypt, dating from the seventh to fifth centuries BC. A new epoch in Carian studies has
now begun with the dramatic discovery in 1996 of an extensive Carian–Greek bilingual
by Turkish excavators in Kaunos and its remarkably swift publication by Frei and Marek
(1997).\footcite{frei1997}



\label{sec:carian}
The Carian language is an extinct language of the Luwian subgroup of the Anatolian branch of the Indo-European language family. The Carian language was spoken in Caria, a region of western Anatolia between the ancient regions of Lycia and Lydia, by the Carians, a name possibly first mentioned in Hittite sources. Carian is closely related to Lycian and Milyan (Lycian B), and both are closely related to, though not direct descendants of, Luwian. Whether the correspondences between Luwian, Carian, and Lycian are due to direct descent (i.e. a language family as represented by a tree-model), or are due to dialect geography, is disputed.[3]


\section{Decipherment}

Prior to the late 20th century the language remained a total mystery even though many characters of the script appeared to be from the Greek alphabet. Using Greek phonetic values of letters investigators of the 19th and 20th centuries were unable to make headway and classified the language as non-Indo-European. Speculations multiplied, none very substantial. Progress finally came as a result of rejecting the presumption of Greek phonetic values.

The Carian script surely stands in some relationship to the Greek alphabet. The direction
of writing is predominantly right to left in texts from Egypt, and left to right in those from
Caria. \emph{Scriptio continua} is frequent, and use of word-dividers is sporadic.

Decipherment of the Carian script has been a long and arduous task. Pioneering efforts
by A. H. Sayce at the end of the nineteenth century were followed by several false steps
based on the erroneous assumption of a syllabic or semisyllabic system and a long period of
relative neglect. It was the merit of V. Shevoroshkin (1965) to have shown that the Carian
script is an alphabet. However, the specific values he and others assigned to individual letters
led to no breakthrough in our understanding of the language. Particularly striking was the
virtually complete absence of any matches between Carian personal names, as attested in
Greek sources, and putative examples in the native alphabet.

A new era began in 1981 when John Ray first successfully exploited the evidence of
the Carian–Egyptian bilingual tomb inscriptions to establish radically new values for several
Carian letters, as well as to confirm the values of others. Additional investigation,
notably by Ray, Ignacio-Javier Adiego, and Diether Schurr, has led to further revisions ¨
and refinements of the new system. The basic validity of this approach was shown by its
correct prediction of Carian personal names which have subsequently appeared in Greek
sources. Nevertheless, many uncertainties and unsolved problems remained, and several
reputable experts were skeptical of the new interpretation of the Carian alphabet. One can
conveniently gain a sense of the state of Carian studies prior to 1997 from Giannotta et al.
1994.

The new Carian–Greek bilingual from Kaunos has shown conclusively the essential validity
of the Ray–Adiego–Schurr system, while also confirming the suspicion of local variation 
in the use of the Carian alphabet. While some rarer signs remain to be elucidated, the question
of the Carian alphabet may be viewed as decided. The new bilingual has not led to
immediate equally dramatic progress in our grasp of the language. One reason for this is
that the Greek text of the Kaunos Bilingual is a formulaic proxenia decree, while the corresponding
Carian is manifestly quite independent in its phrasing of what must be essentially
the same contents. 

\section{Athens Bilingual}

The clearest of the bilingual inscriptions is the one from Athens, although it is laconic and fragmentary (Adiego 1993, no.18):

\begin{figure}[htbp] 
\centering
\includegraphics[width=.70\textwidth]{greek-carian-athens}
\caption{Greek-Carian bilingual inscription from Athens.}
\end{figure}

\section{The Kaunos Bilingual}

Most of what follows is based on the description of the discovery of the Kaunos bilingual
and the description of the texts and analysis as published by Frei et al.

The Kaunos bilingual, consists of three fragment and in the interest 
of clarity, the three fragments are numbered as:

Fragment I: the upper fragment, the only Carian text (lines 1-12),\\
Fragment II: the lower left fragment, the Carnic (lines ISIS left part) and Greek (lines 1-8 left part) contains text,\\
Fragment III: this was discovered later than the first two fragments (newly found) right lower fragment, the Carian (line 12-17 right-hand part) and Greek (lines 1-7 right part) contains text. It has the dimensions: height 0.24 m, Width 0.14 m, thickness 0.085 m.

The three fragments close together in depth without joints. At the surface, on the other hand, has its edges bumped or chipped off, so that here partly small, partly, in particular between fragment I and Fragments II and III, larger spaces were created,
which led to the almost complete loss of line 12.\footcite{frei1997}

\begin{figure}[htbp]
\includegraphics{kaunos-bilingual}
\caption{The Kaunos bilingual. From:\protect\cite{frei1997}}
\end{figure}

The Kaunos Bilingual has provided welcome confirmation of the view
that Carian is an Indo-European Anatolian language, and indeed, of the western type of
Luvian, Lycian, and Lydian. However, one cannot speak of a complete decipherment until
there are generally accepted interpretations of a substantial body of textsThis 
remark applies even to the new bilingual, as one can easily confirm by
reading the competing linguistic analyses in Blumel, Frei, and Marek 1998.

\begin{quote}
1 Έδοξε Καυν[ί]οις, επί δημιο-\\
2 ργοΰ Ίπποσθένους· Νικοκ-\\
3 λέα Λυοικλέους Άΰηναΐό(ν]\\
4 και Λυσικλέα Λυσικράτ[ους]\\
5 [Α]θηναΐον προξένους ε[ίναι κ-]\\
6 [α] l εύεργέτας Καυνίω[ν αυτό-]\\
7 ύς και έκγόνφυς και [υπάρχει-]\\
8 ν αύτοΐς Ε[-------------]\\
\end{quote}



The Carian people, are renowned for having travelled widely: as mercenaries in the service of foreign armies, Carians are found
throughout the ancient Near East. Their most celebrated achievement away from home was in
Egypt, where a large Carian community flourished on the military needs of an unstable political
climate in the seventh and sixth centuries BC.3 In Babylonia, the presence of Carians (Karsaja or
BannE]aja) is recorded but documentation is thin: a group of Carians, presumably mercenaries,
occupied a fief (Hatru) close to Nippur,4 perhaps already in the time of Cambyses's there were
Carians among the war captives held by Nebuchadnezar II in Babylon (Weidner, Fs. Dussaud);
and finally there were Carians in Borsippa. Caroline Waerzeggers (2006) provides a detailed
article on the \enquote{The Carians of Borsippa} in which she examined tablet evidence of taxation
for the provision of food rations to Carians by inhabitants of Borsippa.\footcite{caroline2006}

It is also possible that they have travelled to what is now Izrael as masons. If they were paid for the work or they were enslaved we probably never know.

Carian is known from a limited number of sources:

\begin{enumerate}
\item Personal names with a suffix of -ασσις (-assis), -ωλλος (-ōllos) or -ωμος (-ōmos) in Greek records.
\item Twenty inscriptions from Caria including four bilinguals.
\item Inscriptions of the Caromemphites, an ethnic enclave at Memphis, Egypt.
\item Graffiti elsewhere in Egypt.
\item Scattered inscriptions elsewhere in the Aegean world.
\item Words stated to be Carian by ancient authors.
\end{enumerate}


\begin{figure}[htbp]
\centering
\includegraphics[width=0.8\textwidth]{carian-inscription}
\caption[Thessaloniki, Sherd with a Carian inscription]{Thessaloniki, Sherd with a Carian inscription, dated ca. 600 BCE-ca. 300 BCE.} 
%\href{credit livius.org}{http://www.livius.org/pictures/greece/thessaloniki/thessaloniki-museum-pieces/thessaloniki-sherd-with-a-carian-inscription/}

\end{figure}


\subsection{Carian Alphabets} 
\parindent1em
\parskip10pt


The Carian alphabets are a number of regional scripts used to write the Carian language of western Anatolia. They consisted of some 30 alphabetic letters, with several geographic variants in Caria and a homogeneous variant attested from the Nile delta, where Carian mercenaries fought for the Egyptian pharaohs. They were written left-to-right in Caria (apart from the Carian–Lydian city of Tralleis) and right-to-left in Egypt. Carian was deciphered primarily through Egyptian–Carian bilingual tomb inscriptions, starting with John Ray in 1981; previously only a few sound values and the alphabetic nature of the script had been demonstrated. The readings of Ray and subsequent scholars were largely confirmed with a Carian–Greek bilingual inscription discovered in Kaunos in 1996, which for the first time verified personal names, but the identification of many letters remains provisional and debated, and a few are wholly unknown.

Adiego \cite{adiego} writes that: \enquote{For years, the temptation has existed to attribute any inscription from
Asia Minor written in an unknown or barely recognizable alphabet to
Carian. In a sort of \textit{obscurum per obscurius}, such materials were classed
as Carian at a time when the Carian alphabet itself remained un-deciphered.

Today, we have a better understanding of the Carian alphabet
(letter values, geographical variants, a complete inventory of signs)
and we can reject the theory that these materials are Carian (canonical
Carian, at least).}


The Carian scripts, which have a common origin, have long puzzled scholars. Most of the letters resemble letters of the Greek alphabet, but their sound values are generally unrelated to the values of the Greek letters. This is unusual among the alphabets of Asia Minor, which generally approximate the Greek alphabet fairly well, both in sound and shape, apart from sounds which had no equivalent in Greek. However, the Carian sound values are not completely disconnected: 𐊠 /a/ (Greek Α), 𐊫 /o/ (Greek Ο), \textcarian{𐊰} /s/ (Greek Ϻ san), and \textcarian{𐊲} /u/ (Greek Υ) are as close to Greek as any Anatolian alphabet, and {\carian 𐊷}, which resembles Greek Β, has the similar sound /p/, which it shares with Greek-derived Lydian \textcarian{𐤡}.

Adiego (2007) therefore suggests that the original Carian script was adopted from cursive Greek, and that it was later restructured, perhaps for monumental inscription, by imitating the form of the most graphically similar Greek print letters without considering their phonetic values. Thus a /t/, which in its cursive form may have had a curved top, was modeled after Greek qoppa (Ϙ) rather than its ancestral tau (Τ) to become \textcarian{𐊭}. Carian /m/, from archaic Greek 𐌌, would have been simplified and was therefore closer in shape to Greek Ν than Μ when it was remodeled as 𐊪. Indeed, many of the regional variants of Carian letters parallel Greek variants: \textcarian{𐊥 𐅝} are common graphic variants of digamma, \textcarian{𐊨 ʘ} of theta, \textcarian{𐊬 Λ} of both gamma and lambda, \textcarian{𐌓 𐊯 𐌃} of rho, \textcarian{𐊵 𐊜} of phi, \textcarian{𐊴 𐊛} of chi, 𐊲 V of upsilon, and \textcarian{𐋏 𐊺} parallel \textcarian{Η 𐌇} eta. This could also explain why one of the rarest letters, \textcarian{𐊱}, has the form of one of the most common Greek letters.[13] 
However, no such proto-Carian cursive script is attested, so these etymologies are speculative.

Further developments occurred within each script; in Kaunos, for example, it would seem that \textcarian{𐊮} /š/ and \textcarian{𐊭} /t/ 
both came to resemble a Greek P, and so were distinguished with an extra line in one: \textcarian{𐌓} /t/, \textcarian{𐊯} /š/


\subsection{Description of Alphabets}

Now it remains to list the letters of the Carian alphabet. Table~\ref{tbl:carian} that follows lists the widely accepted signs and their variants. The last column shows the possible greek origin.\footcite[207ff]{adiego}

\begin{longtable}[L]{ 
>{\carian}l|
>{\carian}l| 
>{\carian}l| 
>{\carian}l| 
>{\carian}l| 
>{\carian}l| 
>{\carian}l| 
>{\carian}l| 
>{\panunicode}l| 
>{\panunicode}l|
p{3.5cm}}
\caption{Carian Alphabets}\label{tbl:carian}\\
\hline
 \rotatebox{-90}{Hyllarima} 
&\rotatebox{-90}{Euromos} 
&\rotatebox{-90}{Mylasa} 
&\rotatebox{-90}{Stratonicea} 
&\rotatebox{-90}{Sinuri-Kildara} 
&\rotatebox{-90}{Kaunos} 
&\rotatebox{-90}{Iasos} 
&\rotatebox{-90}{Mephis} 
&\rotatebox{-90}{transliteration} 
&\rotatebox{-90}{greek origin}\\
\hline
𐊠   &𐊠	 &𐊠	 &𐊠	  &𐊠	 &𐊠	    &𐌀	    &𐊠	       & a	  &Α    \\
𐊡   &--  &-- &« ? &𐋉[4]  &--    &𐋌 𐋍	&𐋌?[5]	   & 𐋌[5] & β    &Not a Greek value; perhaps a ligature of Carian \textcarian{𐊬𐊬}. \textcarian{𐊡} directly from Greek Β.\\
𐊣	&𐊣	 &𐊣	&𐊣	 &𐊣	&𐊣	&𐊣	&𐊣	&l	&Λ\\
𐊤	&𐊤	 &    &𐋐   &𐊤	&𐋈	&𐊤	&𐊤 𐋐?	&𐊤 Ε	&y	&Not a Greek sound value; perhaps a modified \textcarian{Ϝ}.\\
--	&--	 &--  &--  &	&𐊥	&𐊥	&𐊥	&r	&Ρ\\
𐋎	&--  &--  &𐊦   &𐊦	&𐋏	&𐊦	&𐊦	&λ & &Not a Greek value. \textcarian{𐋎} from \textcarian{Λ} plus diacritic, others not Greek\\
\panunicode{ʘ}	& \panunicode{ʘ}	 &\panunicode{ʘ}	&\panunicode{ʘ}	&\panunicode{ʘ} 𐊨?	&𐊨	&𐊨 \panunicode{ʘ}	&𐊨	&q	&Ϙ &\\

Λ	&Λ	&Λ &--	&Λ 𐊬	&Γ	&Λ	 &𐊬 Λ	&b  &  & Archaic form of Β, for example in Crete\\
𐊪	&𐊪	&𐊪	&𐊪	&𐊪 &𝈋	&𝈋	&𝈋	 &𝈋	    &m	&𐌌 \\
𐊫	&𐊫	&𐊫	&𐊫	&𐊫 &𐊫	&𐊫	&𐊫	 &o	    &Ο  &\\
𐊭	&𐊭	&𐊭	&𐊭	&𐊭	&𐌓	&𐊭	&𐊭	 &t	    &Τ  &\\
𐤭	&𐤭	&	𐤭	&𐤭     & 𐤭 𐌓   & 𐊯    & 𐤭 𐤧 𐌃   & 𐊮 Ϸ &{\panunicode š}& &Not a Greek value.\\
𐊰	&𐊰	&𐊰	&𐊰	&𐊰	   &𐊰	&𐊰	&𐊰	  &s	&Ϻ\\
&--&--&--&𐊱	&𐊱	&𐊱		&𐊱	&-- &? &\\
𐊲	&𐊲	&𐊲	&𐊲	&𐊲 V	&𐊲	&𐊲 V	&V 𐊲	&u	&Y &Υ /u/\\
𐊳	&--&--	&𐊳	&𐊳	&𐊳	&-- &--		&\panunicode ñ&\\
	&𐊴	&𐊴	&𐊛	&𐊴	&𐊴	&𐊴 𐊛	&𐊴 𐊛	&k̂	&&Not a Greek value. Maybe a modification of Κ, Χ, or \textcarian{𐊨}.\\
𐊵	&𐊵 &𐊜	&𐊵	&𐊵	&𐊵 𐊜	&𐊵	&𐊵	&𐊜 𐊵	&n	& {\panunicode 𐌍} Archaic form of Ν\\	
𐊷	&	&𐊷	&𐊷	&𐊷	&𐊷	&𐊷	&𐊷	&p	&Β\\
𐊸	&𐊸	&𐊸	&𐊸	&𐊸	&Θ	&𐊸	&𐊸 Θ	&ś	& &Not a Greek value. Perhaps from Ͳ sampi?\\
𝈣	&𐊹-	&⊲-	&𐊮-	&𐤧-	&𐊹	&𐊹	&𐊹	&i	&Ε, ΕΙ, or {\panunicode 𐌇}\\
𐋏	&𐋏	&𐋏	&𐊺	&𐊺	&𐊺	&𐊺	&𐊺	&e	&&Η, {\panunicode 𐌇}\\
\end{longtable}


The Carian alphabetic system has no parallels in other scripts found in Anatolia during the first millenium BC. In all these cases, 
the adaptation of the Greek alphabet was much more straightforward and natural: for sounds existing both in Greek and in
the local language, Greek letters with their sound values were used, and for sounds that were not present in Greek, new
letters were created, or Greek letters still available were recycled (for instance N for {\panunicode ñ} in Lycian). Although we find some
exceptions to this system of adaptation (for instance, the use of  for p (not b) in Lydian, or the use of W for i, not for e,
in Lycian), there are usually clear phonetic or formal reasons for all these exceptions.\footnote{Brun, P., L. Cavalier, K. Konuk et F. Prost, éd. (2013) : Euploia. La Lycie et la Carie antiques. Actes du colloque de Bordeaux 5, 6, 7 novembre 2009, Ausonius Mémoires 34, Bordeaux. 17-28} 


\paragraph{Unicode}
Carian was added to the Unicode Standard in April, 2008 with the release of version 5.1. It is encoded in Plane 1 (Supplementary Multilingual Plane).
The Unicode block for Carian is \unicodenumber{U+102A0–U+102DF}:


\begin{scriptexample}[]{Carian}
\unicodetable{carian}{"102A0,"102B0,"102C0,"102D0}
\end{scriptexample}



\PrintUnicodeBlock{./languages/carian.txt}{\carian}


\section{LaTeX}

The Carian script can be typeset using \latexe, using a number of control sequences provided by the |phd-scripts| package. Although most researchers in the field use different methods and Word, using \latexe can provide many efficiencies in inputting the text. With unicode fonts, the text can also be easily copied and entered into other documents. Like most scripts the following control sequences are provided. Linguistic and archaic texts are time consuming and difficult to set. 

\begin{docEnvironment}{CarianScript}{\oarg{color}}
\end{docEnvironment}

The \docAuxEnv{CarianScript} environment can be used to typeset the Carian script in a more friendly way, rather than looking at the right
unicode codepoint. The letters have menomonics based on their shapes or transliteration values.

\newenvironment{CarianScript}[1][blue]
{
 \def\A{\color{#1}{\carian\char"102A0}\xspace}
 \let\a\A
 \def\B{{\color{#1}\carian\char"102A1}\xspace}
 \let\b\B
 \def\Uuu{{\color{#1}\carian\char"102A4}\xspace} 
 \def\R{{\color{#1}\carian\char"102A5}\xspace}
 \def\Omega{{\color{#1}\carian\char"102B6}\xspace}
 \def\lamda{{\color{#1}\carian\char"102A6}\xspace}
 \def\s{{\color{#1}\carian\char"102B0}\xspace}
 \def\q{{\color{#1}\carian\char"102A8}\xspace}
 \def\m{{\color{#1}\carian\char"102AA}\xspace}
}
{}

\begin{texexample}{Writing in Carian}{ex:carian}
\begin{CarianScript}[red]
The character \B is thought to have been derived from the Greek shaped B, whereas \Omega is almost identical to the Greek form,
where a letter such as \R or \lamda\ldots

The character \q and /m/ is given by \m and /a/ by \a.
\end{CarianScript}
\end{texexample}

\section{The Carian Inscriptions from Egypt}

\subsection{Saqqara}
There are approximately a hundred graffiti in hieroglyphic and demotic, and one solitary example
i Carian. Some of the demotic ones are mason's marks and directions, but the vast majority are
the inscriptions of visitors, mainly in the form 'the worthy servant of Osiris the Ape, X son of Y,
his mother being Z; often these include several members of a single family, as with the Serapeum
stelae. Not all the visitors use the same description; some describe themselves as worthy servants
or souls of Osiris-Apis or occasionally of Thoth. In general the incised graffiti are later than the
ink ones, and the hieroglyphic ones late in the sequence, which perhaps lasts from the fourth
century B.C. into the Roman period; unfortunately, though several are dated to a regnal year, in
no instance is the king's name given. At one point, two adjoining blocks are built into the masonry,
one upside down, bearing a most bizarre Greek inscription.\cite{emery1970}

\begin{figure}[htbp]
\includegraphics[width=0.95\textwidth]{carian-saqqara}
\caption{Stelae with Carian texts; no.5 also bears Egyptian texts. From \protect\cite{emery1970}}
\end{figure}

The Carian Inscriptions from Egypt have been described by Masson and the later Ray.\footcite{ray1982}.
It may as
well be added here that the methods of writing Carian found in Egypt themselves vary,
and the system or systems used at Abu Simbel and Abydos are not the same as the one
used rather later by the Caromemphite community of Saqqara. This may be due to
reasons of chronology as much as to the places of origin of the Carians in question,
and in a settled community such as the Carian one in Memphis there is always the
likelihood of independent development. Ray in his 1982 paper, offered an attempted transliteration of most of the surviving Carian inscriptions from Egypt. 

The original corpus used by Ray was published by Masson. The corpus is numbered by Ray as 
M1\ldots--M$_n$. References are also sometimes extended with a lowercase letter, such as |M10a|. 

\begin{description}
\item[M37] 1. \textit{t}(?)]\textit{-d-a-ld-e-s}\\
2. ]e-a-m-\'s-h-e\\
3. a-d-o-s-h-a-r-k'-o-s \\
\end{description}

`Tdaldes, son of [. ]eam, the adoshark'os.' The first name is tentatively restored from M29; it is a
pity that one cannot be certain about this, as it would confirm that the nominative of such names
ended in -es. The title (?) at the end is discussed by Gusmani, \textit{Kadmos} 17 (1978), 74, since it occurs
in a variant spelling on one of his bronze bowls: cf. also Meriggi, BiOr 37 (I980), 35 who discusses this extensively. 


\begin{description}
\item[M55] 1. \textit{j-\'e-p-s-a-d}\\
           2. \textit{p-u-o-j-\'s a-\'s-r-\'s}\\
           3. \textit{u-r-s-j-a-h-29-h-e} 
\end{description}



Texts published in O. Masson and J. Yoyotte, \textit{Objets pharaoniques a inscription
carienne}, Cairo 1956. Abbreviated MY. He described MY text A-- MY text M. 

The next texts discussed were those collected by V. V. \v{S}evoro\v{s}kin, Issledovanija po desifrovke karijskih nadpisej,
Moscow 1965. Abbreviated \v{S}ev.

In his paper Ray compared the Egyptian to the Carian and provided a detailed study.
of the inscriptions.

\subsection{Graffiti from Abu Simbel}

Graffiti from Abu Simbel were published by O.Masson in \textit{Hommages a la memoire de
S. Sauneron}, Cairo, 1979), 35-49. Abbreviated in the literature as |AS|. 

\begin{description}
\item[AS 1] \textit{p-a-r-\'s-o(?)-d(?)-o-u}\\
   2. ]\textit{-o-j}
\end{description}

Ray writes that the first name may be part of the Para- family, but gives no further commentary.



\^{S}jk`urq. 

\subsection{Labraunda}
Labraunda\footnote{Ancient Greek: Λάβρανδα Labranda or Λάβραυνδα Labraunda} is an  archaeological site five kilometers west of Ortaköy, 
Muğla Province, Turkey, in the mountains near the coast of Caria. In ancient times, it was held sacred by Carians and Mysians alike. 
The site amid its sacred platanus trees\footnote{Herodotus, v.119} was enriched in the Hellenistic style by the Hecatomnid dynasty of Mausolus, satrap (and virtual king) of Persian Caria (c. 377 – 352 BCE), and also later by his successor and brother Idrieus; Labranda was the dynasty's ancestral sacred shrine. The prosperity of a rapidly hellenised Caria occurred in the during the 4th century BCE.[2] Remains of Hellenistic houses and streets can still be traced, and there are numerous inscriptions. The cult icon here was a local Zeus Labrandeus (Ζεὺς Λαβρανδεύς), a standing Zeus with the tall lotus-tipped scepter upright in his left hand and the double-headed axe, the labrys, over his right shoulder. The cult statue was the gift of the founder of the dynasty, Hecatomnus himself, recorded in a surviving inscription.

A grafito from Lambraunda was described in an interesting article in \textit{Kadmos} by Karlsson Lars and Henry Olivier (2009).\footcite{Karlsson2009}

\begin{figure}[htbp]
 \includegraphics[width=\textwidth]{labraunda}
 \end{figure}

\begin{quotation}
 The excavation trenches were laid out in this area, which was used as barracks, i.e.
the rooms in which the soldiers on duty had their living quarters. On
the threshold leading into Room 2, on July 12, 2007, we discovered
the base of an Attic black-gloss bowl. It was decorated on the inside
with palmettes joined by large circle segments. On the underside of the
bowl a Carian graffi to was found, perhaps a name. The base profile
and decoration date it to 375--350 B.C. based on a comparison with
similar examples from the Maussolleion in Halikarnassos,4
 as well
 as already published bases from Labraunda.
 \end{quotation}

The graffito has been named C.La 1, this being the first real Carian text discovered in Labraunda
\begin{description}
  \item[C.La 1]  \textit{bziom}
\end{description}

\section{The Carian Stonemasons}

The Persepolis Treasury Tablets specifically mentioned the Carians as being stonemasons (Hallock 1960:99, Cameron 1965). In the Treasury at Persepolis there are one hundred and eighty-one mason's marks, consisiting of ten basic signs, appear on the earliest stonework in Apadana at Suza. These ten basic signs have parallels amongst the thirty basic signs used at Persepolis and were attributed to Anatolian workmen by Nylander (1974:216-7;1975:322-3). Although the mason's marks differ from sculptors' marks, both have affinities with the Lydian, Aramaic or South Arabian alphabets (Roaf 1983:92-93).

The \enquote{Lydian Wall} at Sardis has mason's marks from Carian, Lydian and Phrygian alphabets (Gusmani 1988:33), while Gosline (1988:59) cites the use of sixty-nine Carian alphabetic masons' marks. Excavations at Carian Labraunda revealed seven mason's marks incised on limestone ashlars\footnote{Ashlar is finely dressed (cut, worked) stone, either an individual stone that has been worked until squared or the structure built of it} (S\={a}flund 1953). Carian alphabetic marks have also been found engraved in stone quarries in Egypt. At Elephantine, four hundred and twenty-eight alphabetic marks have been recognized by Gosline (1992;1998:60).


\section{Postscript}

Writing a book like this one, presents many challenges. One is how to write efficiently, as lingusitic texts are full of citations and also of specialized symbols, diagrams and enumerations.

The choice of programming macros for most of the languages and transliterations, has a big advantage, as one can use the macro names more efficiently than hunting to find the right unicode character. Cut-and-paste from older books is next to impossible in pre-unicode days. Still many journals suffer from this. 