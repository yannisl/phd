\cxset{quotation font-size=\normalsize,
       quote font-size=\normalsize}


\section{Mandaic}
\label{s:mandaic}
\newfontfamily\mandaic{NotoSansMandaic-Regular.ttf}


The Mandaic script is used to write a dialect of Eastern Aramaic, which, in its classical
form, is currently used as the liturgical language of the Mandaean religion. A living language descended
from Classical Mandaic is spoken by a small number of people living in and around Ahvaz, Khūzestān,
in southwestern Iran; speakers are also found in emigrant communities in Sweden, Australia, and the
United States. There is a considerable amount of Iranian influence on the lexicon of Classical Mandaic,
and Arabic and Persian influence on the grammar and lexicon of the contemporary dialect. The script
itself is likely derived from the Parthian chancery script.

Mandaic is a right-to-left script. It is a true alphabet, using letters regularly for vowels
rather than as the \emph{matres lectionis} from which they derived. The three diacritical marks are used in
teaching materials to differentiate vowel quality. At present, at least, the rule is that they may be omitted
from ordinary text. In this regard they are very like the Arabic fatha, kasra, and damma or the Hebrew
vowel points.

The only so far I could find that can display the script is the Google \idxfont{NotoSansMandaic.ttf}.

\begin{scriptexample}[]{Mandaic}
\bgroup
\unicodetable{mandaic}{"0840,"0850}
\egroup
\end{scriptexample}

In 1943, Lady Ethel Drower published extracts from several magic “recipe books” that served the writers of amulets in Baghdad in the early 20th century, in particular from two manuscripts in her possession, DC 45 and DC 46.

\begin{figure}[hb]
\centering

\includegraphics[height=4cm]{./magic-letters.jpg}
\includegraphics[height=4cm]{./45-453.jpg}
\includegraphics[height=4cm]{./36-448.jpg}

\captionof{figure}{Mandaic Incantation vessels. The left image is from \protect\href{http://thesacredalphabet.blogspot.ae/}{thesacredalphabet}, whereas the last two are from \protect\href{http://www.archaeological-center.com/en/auctions/45-453}{archaeological-center} }
\end{figure}

 While Drower, following her native informants, entitled the work ‘A Mandæan Book of Black Magic’, the manuscripts themselves contain a wide range of formulae for amulets and talismans for various purposes, as Drower herself was well aware. Alongside spells for healing, protection and success, we find others for enflaming love or stirring up enmity.

 The manuscripts themselves appear to have been copied in the late 19th or early 20th centuries; in particular, DC 46, a substantial codex of 264 sides, is written on an extremely modern “clean” paper. DC 45 is written on a rougher paper and appears to be somewhat earlier. It is also more fragmentary, and contains several leaves that were copied by a different hand and inserted into the main part of the manuscript at a later date, though it is clear from their contents that they were intended to replace pages that had been worn or damaged, as they begin and end exactly as required by the preceding and following pages. As it survives today, DC 45 is also considerably shorter than DC 46; however, it also contains several spells that are not found in DC 46.\footnote{\protect\href{http://www.academia.edu/8294938/Arabic_Magic_Texts_in_Mandaic_Script_A_Forgotten_Chapter_in_Near-Eastern_Magic}{Magic Texts}}

Lady Drower inform us that among the Mandaens:

\begin{quote}
Writing in itself is a magic art, and the alphabet is sacred.
Each letter is supposed to invoke a spirit of light and is a thing of power. It is a practice to write the letters separately and to sleep each night with a letter beneath the pillow. If the sleeper sees in a dream something which will enlighten him, the letters upon which he slept that night is taken to a silversmith and a replica in gold or silver is made and worn around the neck as amulet See Mandaic Incantation Texts by Edwin M Yamauchi.
\end{quote}











