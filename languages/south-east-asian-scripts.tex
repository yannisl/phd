\chapter{South East Asian Scripts}
\label{ch:southeastasia}
\section{Introduction}

This section documents the facilities offered to typeset Southeast Asian Scripts. These scripts are used in most of Southeast Asia, Indonesia and the Philippines.

\pagestyle{headings}

\begin{table}[htb]
\centering
\begin{tabular}{lll}
  \hyperref[s:thai]{Thai} 
& Tai Tham 
& \hyperref[s:balinese]{Balinese}\\
\hyperref[s:lao]{Lao}  
&Tai Viet  
& \hyperref[s:javanese]{Javanese}\\
Myanmar 
&Kayah Li 
&Rejang\\
 \hyperref[s:khmer]{Khmer} 
&Cham 
&Batak\\
Tai Le 
&Philippine Scripts 
& \hyperref[s:sundanese]{Sundanese}\\
  \hyperref[s:newtailue]{New Tail Lue}
& Buginese\\
\end{tabular}
\end{table}

\section{Balinese}

\epigraph{In Bali the gods are thought of as the \textit{children} of the people, not as august parental figures. Speaking through the lips of those in trance, the gods address the villages as ``papa''  and ```mama'', and the people are said to spoil or indulge their gods\ldots}{Gregory Bateson and Margaret Mead in \textit{Balinese Character: A Photographic Analysis, 1942}}
\label{s:balinese}\index{Balinese}\index{Aksara Bali}\index{Bali}\index{Lombok}

\newfontfamily{\balinese}{AksaraBali.ttf}

Balinese or simply Bali\footnote{Not to be confused with the Nigerian or Papua New Guinea languages also named Bali.} is a Malayo-Polynesian language spoken by 3.3 million people (as of 2000) on the Indonesian island of Bali, as well as northern Nusa Penida, western Lombok and eastern Java.[3] Most Balinese speakers also know Indonesian. Balinese itself is not mutually intelligible with Indonesian, but may be understood by Javanese speakers after some exposure.

In 2011, the Bali Cultural Agency estimates that the number of people still using Balinese language in their daily lives on the Bali Island does not exceed 1 million, as in urban areas their parents only introduce Indonesian language or even English, while daily conversations in the institutions and the mass media have disappeared. The written form of the Balinese language is increasingly unfamiliar and most Balinese people use the Balinese language only as a spoken tool with mixing of Indonesian language in their daily conversation. But in the transmigration areas outside Bali Island, Balinese language is extensively used and believed to play an important role in the survival of the language.[4]

\begin{figure}[htbp]
\centering

\includegraphics[width=\textwidth]{bali-cock.jpg}
\end{figure}

The higher registers of the language borrow extensively from Javanese: an old form of classical Javanese, Kawi, is used in Bali as a religious and ceremonial language.

\paragraph{The Balinese script} is natively known as Aksara Bali and Hanacaraka, is an abugida used in the island of Bali, Indonesia, commonly for writing the Austronesian Balinese language, Old Javanese, and the liturgical language Sanskrit. With some modifications, the script is also used to write the Sasak language, used in the neighboring island of Lombok.[1] 

The script is a descendant of the Brahmi script, and so has many similarities with the modern scripts of South and Southeast Asia. The Balinese script, along with the Javanese script, is considered the most elaborate and ornate among Brahmic scripts of Southeast Asia.[2]

\includegraphics[width=\textwidth]{bali}


Though everyday use of the script has largely been supplanted by the Latin alphabet, the Balinese script has significant prevalence in many of the island's traditional ceremonies and is strongly associated with the Hindu religion. The script is mainly used today for copying lontar or palm leaf manuscripts containing religious texts.[2][3]



{\indicative ◌ }

\newcounter{under}
\setcounter{under}{"1B00}

\def\cb#1 {
\hspace*{2.5pt}
 
 $\text{◌#1}_{\pgfmathparse{Hex(\theunder)}\text{\pgfmathresult}}$
\stepcounter{under}
\vskip5pt\par
}
\begin{scriptexample}[]{Balinese}


\balinese
	 
᭐	᭑	᭒	᭓	᭔	᭕	᭖	᭗	᭘	᭙	᭚	᭛	᭜	᭝	᭞	᭟\\\
 
\def\columnseprulecolor{\color{thegray}}
\columnseprule.4pt
\begin{multicols}{8}

\texttt{U+1B0x}	

\cb{ᬀ }  \cb{ ᬁ } 	\cb{ ᬂ }  	\cb ᬃ	\cb ᬄ 	\cb ᬅ	\cb ᬆ	\cb ᬇ	\cb ᬈ	\cb ᬉ	\cb ᬊ	\cb ᬋ	\cb ᬌ	\cb ᬍ	\cb ᬎ	\cb ᬏ

\columnbreak

\texttt{U+1B1x}	 

\cb ᬐ	 \cb ᬑ 	\cb ᬒ 	\cb ᬓ	\cb ᬔ	\cb ᬕ	\cb ᬖ \cb ᬗ 	\cb ᬘ 	\cb ᬙ 	\cb ᬚ	\cb ᬛ 	\cb ᬜ 	\cb ᬝ 	\cb ᬞ	\cb ᬟ 

\columnbreak

U+1B2x	 

\cb ᬠ◌ 	\cb ᬡ	\cb ᬢ	\cb ᬣ	\cb ᬤ	\cb ᬥ	\cb ᬦ	\cb ᬧ	\cb ᬨ	\cb ᬩ	\cb ᬪ	\cb ᬫ	\cb ᬬ	\cb ᬭ	\cb ᬮ	\cb ᬯ

\columnbreak
U+1B3x 

\cb ᬰ	\cb ᬱ	\cb ᬲ	\cb ᬳ	\cb ᬴	\cb ᬵ	\cb ᬶ	\cb ᬷ	\cb ᬸ	\cb ᬹ	\cb ᬺ	\cb ᬻ	\cb ᬼ	\cb ᬽ	\cb ᬾ	\cb ᬿ


\columnbreak
U+1B4x	 

\cb ᭀ	 \cb ᭁ	\cb ᭂ	\cb ᭃ	\cb ᭄	\cb ᭅ	\cb ᭆ	\cb ᭇ	\cb ᭈ	\cb ᭉ	\cb ᭊ	\cb ᭋ

\columnbreak				
U+1B5x	 

\cb ᭐	\cb ᭑	\cb ᭒	\cb ᭓	\cb ᭔	\cb ᭕	\cb ᭖	\cb ᭗	\cb ᭘	\cb ᭙	\cb ᭚	\cb ᭛	\cb ᭜	\cb ᭝	\cb ᭞	\cb ᭟\\

\columnbreak

U+1B6x 

\cb ᭠	\cb ᭡	\cb ᭢	\cb ᭣	\cb ᭤	\cb ᭥	\cb ᭦	\cb ᭧	\cb ᭨◌ 	\cb ᭩◌ 	\cb ᭪◌ 	\cb ᭫	\cb ᭬	\cb ᭭	\cb ᭮	\cb ᭯

\columnbreak
U+1B7x	 

\cb ᭰	 \cb ᭱  \cb ᭲  \cb ᭳	 \cb ᭴	\cb ᭵	\cb ᭶	\cb ᭷	\cb ᭸	\cb ᭹	\cb ᭺	\cb ᭻	\cb ᭼


\end{multicols}

\end{scriptexample}


One of the most comprehensive fonts is Aksara Bali\footnote{\url{http://www.alanwood.net/downloads/index.html}}. This is obtainable at Alan Wood's website.
\clearpage

%\newfontfamily\javanese{Noto Sans Javanese}

%\newfontfamily\javanese{TuladhaJejeg_gr.ttf}

\section{Javanese}
\label{s:javanese}
\index{scripts>Javanese}


The Javanese (Ngoko Javanese: {\javanese ꦮꦺꦴꦁꦗꦮ},[3] Madya Javanese: {\javanese\   ꦠꦶꦪꦁꦗꦮꦶ},[4] Krama Javanese: ꦥꦿꦶꦪꦤ꧀ꦠꦸꦤ꧀ꦗꦮꦶ,[4] Ngoko Gêdrìk: wòng Jåwå, Madya Gêdrìk: tiyang Jawi, Krama Gêdrìk: priyantun Jawi, Indonesian: suku Jawa)[5] are an ethnic group native to the Indonesian island of Java. With approximately 100 million people (as of 2011), they form the largest ethnic group in Indonesia. They are predominantly located in the central to eastern parts of the island. There are also significant numbers of people of Javanese descent in most provinces of Indonesia, Malaysia, Singapore, Suriname, Saudi Arabia and the Netherlands.

The Javanese ethnic group has many sub-groups, such as the Mataram, Cirebonese, Osing, Tenggerese, Samin, Naganese, Banyumasan, etc.[6]

A majority of the Javanese people identify themselves as Muslims, with a minority identifying as Christians and Hindus. However, Javanese civilization has been influenced by more than a millennium of interactions between the native animism Kejawen and the Indian Hindu—Buddhist culture, and this influence is still visible in Javanese history, culture, traditions, and art forms. With a sizeable global population, the Javanese are considered significant as they are the fourth largest ethnic group among Muslims, in the world, after the Arabs,[7] Bengalis[8] and Punjabis.[9]


\paragraph{Javanese} is one of the Austronesian languages, but it is not particularly close to other languages and is difficult to classify. Its closest relatives are the neighbouring languages such as Sundanese, Madurese and Balinese. Most speakers of Javanese also speak Indonesian, the standardized form of Malay spoken in Indonesia, for official and commercial purposes as well as a means to communicate with non-Javanese-speaking Indonesians.

There are speakers of Javanese in Malaysia (concentrated in the states of Selangor and Johor) and Singapore. Some people of Javanese descent in Suriname (the Dutch colony of Suriname until 1975) speak a creole descendant of the language.

\begin{figure}[htbp]
\includegraphics[width=\textwidth]{javanese-people}
\end{figure}

The language is spoken in Yogyakarta, Central and East Java, as well as on the north coast of West Java. It is also spoken elsewhere by the Javanese people in other provinces of Indonesia, which are numerous due to the government-sanctioned transmigration program in the late 20th century, including Lampung, Jambi, and North Sumatra provinces. In Suriname, creolized Javanese is spoken among descendants of plantation migrants brought by the Dutch during the 19th century. In Madura, Bali, Lombok, and the Sunda region of West Java, it is also used as a literary language. It was the court language in Palembang, South Sumatra, until the palace was sacked by the Dutch in the late 18th century.

Javanese is written with the Latin script, Javanese script, and Arabic script.[5] In the present day, the Latin script dominates writings, although the Javanese script is still taught as part of the compulsory Javanese language subject in elementary up to high school levels in Yogyakarta, Central and East Java.

Javanese is the tenth largest language by native speakers and the largest language without official status. It is spoken or understood by approximately 100 million people. At least 45\% of the total population of Indonesia are of Javanese descent or live in an area where Javanese is the dominant language. All seven Indonesian presidents since 1945 have been of Javanese descent.[6] It is therefore not surprising that Javanese has had a deep influence on the development of Indonesian, the national language of Indonesia.

There are three main dialects of the modern language: Central Javanese, Eastern Javanese, and Western Javanese. These three dialects form a dialect continuum from northern Banten in the extreme west of Java to Banyuwangi Regency in the eastern corner of the island. All Javanese dialects are more or less mutually intelligible.


\paragraph{The Javanese script} (Hanacaraka/Carakan) is a script for writing the Javanese language, the native language of one of the peoples of the Island of Java. It is a descendent of the ancient Brahmi script of India, and so has many similarities with modern scripts of South Asia and Southeast Asia. The Javanese script is also used for writing Sanskrit, Old Javanese, and transcriptions of Kawi, as well as the Sundanese language, and the Sasak language.

\begin{figure}[htbp]
\hspace*{-1.5cm}\includegraphics[width=1.2\textwidth]{java-palm-leave-manuscript}
\end{figure}





\begin{scriptexample}[]{Javanese}
\bgroup
\javanese

꧋ꦱꦧꦼꦤ꧀ꦮꦺꦴꦁꦏꦭꦲꦶꦂꦲꦏꦺꦏꦤ꧀ꦛꦶꦩꦂꦢꦶꦏꦭꦤ꧀ꦢꦂꦧꦺꦩꦂꦠꦧꦠ꧀ꦭꦤ꧀ꦲꦏ꧀ꦲꦏ꧀ꦏꦁꦥꦝ꧉

꧋ ꦲꦮꦶꦠ꧀ꦲꦶꦏꦁꦄꦱ꧀ꦩꦄꦭ꧀ꦭꦃ꧈ ꦏꦁꦩꦲꦩꦸꦫꦃꦠꦸꦂ ꦩꦲꦲꦱꦶꦃ꧉ 	 
 ۝꧋ ꦄꦭꦶꦥꦃ꧀ ꦭ ꦩ꧀ ꦫ ꧌ ꦏꦁ — — ꦥꦿꦶꦏ꧀ꦱ ꦏꦉꦪꦥ꧀ꦥꦩꦸꦁꦄꦭ꧀ꦭꦃꦥꦶꦪꦺꦩ꧀ꦧꦏ꧀ ꧌꧉ ꦩꦁꦪꦏꦴꦪꦤꦴ ꦲꦶꦏꦸꦄꦪꦺꦪꦠꦴꦏꦶꦠꦧ꧀ꦑꦸꦂꦄꦤ꧀ꦏꦁꦥꦿꦪꦠꦭ꧉ 	 
᭐	᭑	᭒	᭓	᭔	᭕	᭖	᭗	᭘	᭙	᭚	᭛	᭜	᭝	᭞	᭟
 
\egroup
\end{scriptexample}


The Javanese script was added to Unicode Standard in version 5.2 on the code points \texttt{A980 - A9DF}. There are 91 code points for Javanese script: 53 letters, 19 punctuation marks, 10 numbers, and 9 vowels:
\medskip

\unicodetable{javanese}{"A980,"A990,"A9A0, "A9B0, "A9C0,"A9D0}

\medskip



As of the writing of this document (2017), there are several widely published fonts able to support Javanese, ANSI-based Hanacaraka/Pallawa by Teguh Budi Sayoga,[21] Adjisaka by Sudarto HS/Ki Demang Sokowanten,[22] JG Aksara Jawa by Jason Glavy,[23] Carakan Anyar by Pavkar Dukunov,[24] and Tuladha Jejeg by R.S. Wihananto,[25] which is based on Graphite (SIL) smart font technology. Other fonts with limited publishing includes Surakarta made by Matthew Arciniega in 1992 for Mac's screen font,[26] and Tjarakan developed by AGFA Monotype around 2000.[27] There is also a symbol-based font called Aturra developed by Aditya Bayu in 2012–2013.[28]

Due to the script's complexity, many Javanese fonts have different input method compared to other Indic scripts and may exhibit several flaws. \docFont{JG Aksara Jawa}, in particular, may cause conflicts with other writing system, as the font use code points from other writing systems to complement Javanese's extensive repertoire. This is to be expected, as the font was made before Javanese implementation in Unicode.[29]

Arguably, the most "complete" font, in terms of technicality and glyph count, is \docFont{TuladhaJejeg}. It comes with keyboard facilities, displaying complex syllable structure, and support extensive glyph repertoire including non-standard forms which may not be found in regular Javanese texts, by utilizing Graphite (SIL) smart font technology. |Tuladha Jejeg| uses variable stroke widths on its glyphs with serifs on some glyphs\footnote{\protect\url{https://sites.google.com/site/jawaunicode/main-page}}.

However, as not many writing systems require such complex feature, use is limited to programs with Graphite technology, such as Firefox browser, Thunderbird email client, and several OpenType word processor and of course XeLaTeX. The font was chosen for displaying Javanese script in the Javanese Wikipedia.[16]

\paragraph{jawaTeX} Jawa\TeX{} project is initial effort to make Javanese characters typesetting program using \TeX{}/\LaTeX{}. This project is aimed to make Javanese widely used. The main project is developing transliteration models to transliterate Latin document into Javanese document. Perl and \TeX{}/\LaTeX{} are use in this project, the program are develop to run in text mode (console) both Linux and Windows but not limit on it. Web based program also developed, and automatic embedded Javanese characters in HTML See \href{http://jawatex.org/jawa/jawatex}{jawatex}.




\section{Khmer}
\label{s:khmer}
\newfontfamily\normaltext{Arial}
\normaltext

\newfontfamily\khmer[Scale=1.05]{NotoSansKhmer-Regular.ttf}
\def\khmertext#1{{\khmer#1}}



The population of Cambodia was estimated at 13.5 million in 2003
(Asian Development Bank, 2003~) Approximately 90 percent of Cambodians
are ethnic Khmer and speak Khmer as their native language. Khmer
belongs to the Mon-Khmer subfamily of the Austro-Asiatic language family;
Vietnamese (in the Vietnamese subfamily) is the only other national language
in this family. The Khmer alphabet began to develop in the seventh
century and influenced the later development of the Thai writing system,
though the two languages are not mutually comprehensible; Khmer itself
was influenced by Pali and Sanskrit, and modern Khmer contains many loan
words from these Indian languages (Diffloth, 1992; Thel, 1985; Weber,
1989). Minority populations in Cambodia include Cham, Chinese, and
Vietnamese, all of which number in the hundreds of thousands (Kosonen,
2004). Somewhat more than 100,000 Cambodians belong to around 30
indigenous ethnic groups living for the most part in the northeast highlands.
Members of the relatively isolated Brao, Jarai, Krung, Tampuan, and other
communities often do not speak Khmer (Thomas, 2002, 2003).

\begin{figure}[htbp]
\includegraphics[width=\textwidth]{khmer}
\caption{Khmer Dancers}
\end{figure}


\begin{docKey}[phd]{khmer font}{=\meta{font name} (Khmer  UI)} {}
Loads the font
command \cmd{\khmer}. When the command is used it typesets text in
khmer unicode. There is no need to load the language, unless it is the main document language. For windows the default font is \texttt{DaunPenh} this font is in general too small to read; a better font to use is Khmer UI.
\end{docKey}




The Khmer script (Khmer: {\Large\khmertext{អក្សរខ្មែរ}}; IPA: [ʔaʔsɑː kʰmaːe]) [2] is an \textit{abugida} (alphasyllabary) script used to write the Khmer language (the official language of Cambodia). It is also used to write Pali among the Buddhist liturgy of Cambodia and Thailand.

It was adapted from the Pallava script, a variant of Grantha alphabet descended from the Brahmi script of India, which was used in southern India and South East Asia during the 5th and 6th Centuries AD.[3] The oldest dated inscription in Khmer was found at Angkor Borei District in Takéo Province south of Phnom Penh and dates from 611.[4] The modern Khmer script differs somewhat from precedent forms seen on the inscriptions of the ruins of Angkor.

Not all Khmer consonants can appear in syllable-final position. The most common syllable-final consonants include {\khmer កងញតនបមល}. The pronunciation of the consonant in final position may differ from it's normal pronunciation.


\begin{tabular}{l l p{9cm}}
\khmertext{ំ}	&nĭkkôhĕt (\khmertext{និគ្គហិត})	&niggahita; nasalizes the inherent vowels and some of the dependent vowels, see anusvara, sometimes used to represent [aɲ] in Sanskrit loanwords\\

\khmertext{ះ}	&reăhmŭkh (\khmertext{រះមុខ})	&"shining face"; adds final aspiration to dependent or inherent vowels, usually omitted, corresponds to the visarga diacritic, it maybe included as dependent vowel symbol\\

\khmertext{ៈ}	&yŭkôleăkpĭntŭ (\khmertext{យុគលពិន្ទុ})	&yugalabindu ("pair of dots"); adds final glottalness to dependent or inherent vowels, usually omitted\\

\khmertext{៉}	 &musĕkâtônd (\khmertext{មូសិកទន្ត})	&mūsikadanta ("mouse teeth"); used to convert some o-series consonants (\khmertext{ង ញ ម យ រ វ}) to the a-series\\

\khmertext{៊}	&treisâpt (\khmertext{ត្រីសព្ទ})	&trīsabda; used to convert some a-series consonants (\khmertext{ស ហ ប អ}) to the o-series\\
\end{tabular}




ុ	kbiĕh kraôm (ក្បៀសក្រោម)	also known as bŏkcheung (បុកជើង); used in place of the diacritics treisâpt and musĕkâtônd when they would be impeded by superscript vowels
់	bântăk (បន្តក់)	used to shorten some vowels; the diacritic is placed on the last consonant of the syllable
៌	rôbat (របាទ)
répheăk (រេផៈ)	rapāda, repha; behave similarly to the tôndâkhéat, corresponds to the Devanagari diacritic repha, however it lost its original function which was to represent a vocalic r
 ៍	tôndâkhéat (ទណ្ឌឃាដ)	daṇḍaghāta; used to render some letters as unpronounced
៎	kakâbat (កាកបាទ)	kākapāda ("crow's foot"); more a punctuation mark than a diacritic; used in writing to indicate the rising intonation of an exclamation or interjection; often placed on particles such as /na/, /nɑː/, /nɛː/, /vəːj/, and the feminine response /cah/
៏	âsda (អស្តា)	denotes stressed intonation in some single-consonant words[5]
័	sanhyoŭk sannha (សំយោគសញ្ញា)	represents a short inherent vowel in Sanskrit and Pali words; usually omitted
៑	vĭréam (វិរាម)	a mostly obsolete diacritic, corresponds to the virāma
្	cheung (ជើង)	a.w. coeng; a sign developed for Unicode to input subscript consonants, appearance of this sign varies among fonts
\section{Sundanese}
\epigraph{The married women, when their husband die, must, as point of honour, die with them, and if they should be afraid of death they put into the convents.}{Tomé Pires \textit{Suma Oriental} (1512–1515)}

\label{s:sundanese}

\newfontfamily\sundanese{Noto Sans Sundanese}
The Sundanese (Sundanese: {\sundanese ᮅᮛᮀ ᮞᮥᮔ᮪ᮓ}, Urang Sunda) are an ethnic group native to the western part of the Indonesian island of Java. They number approximately 40 million, and are the second most populous of all the nation's ethnicities. The Sundanese are predominantly Muslim. In their own language, Sundanese, the group is referred to as Urang Sunda, and Orang Sunda or Suku Sunda in the national language, Indonesian.

The Sundanese have traditionally been concentrated in the provinces of West Java, Banten, Jakarta, and the western part of Central Java. Sundanese migrants can also be found in Lampung and South Sumatra. The provinces of Central Java and East Java are home to the Javanese, Indonesia's largest ethnic group.

\begin{figure}[htbp]
\centering
\includegraphics[width=\linewidth-2\parindent]{sundanese}

\caption{Sundanese boys playing Angklung in Garut, c. 1910–1930. \href{https://en.wikipedia.org/wiki/Sundanese_people}{wikipedia}}
\end{figure}

The Sundanese script (Aksara Sunda, {\sundanese ᮃᮊ᮪ᮞᮛ ᮞᮥᮔ᮪ᮓ}) is a writing system which is used by the Sundanese people. It is built based on Old Sundanese script (Aksara Sunda Kuno) which was used by the ancient Sundanese between the 14th and 18th centuries.



\begin{scriptexample}[]{Sundanese}
\unicodetable{sundanese}{"1B80,"1B90,"1BA0,"1BB0}

\sundanese
\obeylines
\bgroup
᮱ {\arial= 1}	᮲ {\arial= 2}	᮳{\arial = 3}
᮴ {\arial= 4}	᮵ {\arial = 5} 	᮶ {\arial= 6}
᮷ {\arial= 7}	᮸ {\arial= 8}	᮹ {\arial= 9}
᮰ {\arial= 0}

\egroup
\end{scriptexample}

\begin{scriptexample}[]{Sundanese}
\bgroup
\sundanese
\centering

◌ᮃᮄᮅᮆᮇᮈᮉᮊᮋᮌᮍᮎᮏᮐᮕᮔᮓᮑᮖᮗᮚᮛᮜᮝᮞᮟᮠᮠ


\egroup
\end{scriptexample}

\bgroup
\def\1{\sundanese ᮱}
\TextOrMath\1\1

$\1$
\egroup

In text In texts, numbers are written surrounded with dual pipe sign \textbar \ldots \textbar. Example: {\textbar \sundanese ᮲᮰᮱᮰ }\textbar = 2010












%\newfontfamily\hanunoo{NotoSansHanunoo-Regular.ttf}

\section{Hanunó’o}

Hanunó’o is one of the indigenous scripts of the Philippines and is used by the Mangyan peoples of southern Mindoro to write the Hanunó'o language.[1] 

It is an \emphasis{abugida} descended from the Brahmic scripts, closely related to Baybayin, and is famous for being written vertical but written upward, rather than downward as nearly all other scripts (however, it's read horizontally left to right). It is usually written on bamboo by incising characters with a knife.[2][3] Most known Hanunó'o inscriptions are relatively recent because of the perishable nature of bamboo. It is therefore difficult to trace the history of the script



\begin{scriptexample}[width=2cm]{Hanunoo}
\hanunoo

{\Large
\obeylines
ᜠ 
ᜫ
ᜨᜲ
ᜫᜲ
ᜰ
ᜮ
ᜥ
ᜦ᜴}

Typeset with \texttt{NotoSansHanunoo-Regular.ttf} and the command \cmd{\hanunoo}
\end{scriptexample}

Vertically positionning the text is not currently supported by \pkgname{fontspec} and the manual says \textsc{Todo!}. You are your own here, or you can just put the characters in a box and give it a try.

\begin{minipage}[t]{2cm}
\begin{tcolorbox}[width=2cm,colback=graphicbackground,
boxrule=0pt,toprule=0pt,colframe=white]
\Large\hanunoo
ᜩ\\
ᜤ\\
ᜮ\\
ᜥᜳ\\
ᜨ᜴ \\
ᜨ᜴\\
ᜫᜳ\\
ᜥ\\
\end{tcolorbox}
\end{minipage}
\begin{minipage}[t]{2cm}
\begin{tcolorbox}[width=2cm,colback=graphicbackground,
boxrule=0pt,toprule=0pt,colframe=white]
\LARGE\hanunoo
ᜩ\\
ᜤ\\
ᜮ\\
ᜥᜳ\\
ᜨ᜴ \\
ᜨ᜴\\
ᜫᜳ\\
ᜥ\\
\end{tcolorbox}
\end{minipage}
\begin{minipage}[t]{\textwidth-6cm}

The script is written from bottom to top. Typesetting this type of script automatically is not without its problems. One way is to use the build-in features of the font if they are available, but currently this gives problems---at least with the fonts that I have tried. Entering the text is also problematic as you will more than likely see little boxes rather than the actual glyph with most text editors common to \latexe. If you only need a couple of characters or a short sentence, an easy solution is to use |\rotatebox|. Another solution is to use a macro that can add the letters onto a stack, then place them in a box with a limited width. We can use |\@tfor| for this.  
\end{minipage}
\section{New Tai Lue Script}
\label{s:newtailue}
\newfontfamily\tailue{Noto Sans New Tai Lue}


New Tai Lue script, also known as Simplified Tai Lue, is an alphabet used to write the Tai Lü language. Developed in China in the 1950s, New Tai Lue is based on the traditional Tai Le alphabet developed ca. 1200 AD. The government of China promoted the alphabet for use as a replacement for the older script; teaching the script was not mandatory, however, and as a result many are illiterate in New Thai Lue. 

\begin{figure}[htbp]
\centering

\includegraphics[width=\linewidth-2\parindent]{tailue}

\caption{Tai Le costumes. (pininterest)}
\end{figure}

In addition, communities in Burma, Laos, Thailand and Vietnam still use the Tai Le alphabet. There are probably less than one million native speakers of the language who can be found in China, Burma, Laos, Thailand and Vietnam.

\begin{figure}[htbp]
\centering

\includegraphics[width=\linewidth-2\parindent]{tai-lu}

\caption{Tai Le costumes. (pininterest)}
\end{figure}

\begin{scriptexample}[]{Tai Lue}
{\centering\tailue \LARGE

ᦒ	ᦓ	ᦔ	ᦕ	ᦖ	ᦗ	ᦘ	ᦙ	ᦚ	ᦛ	ᦜ	ᦝ	ᦞ	

}
\end{scriptexample}

The New Tai Lue script was added to the Unicode Standard in March, 2005 with the release of version 4.1.

The Unicode block for New Tai Lue is |U+1980|–|U+19DF|:

\begin{scriptexample}[]{New Tai Lue}
\unicodetable{tailue}{"1980,"1990,"19A0,"19B0,"19C0,"19D0}

\texttt{typeset using NotoSansNewTaiLue-Regular.ttf.}
\end{scriptexample}
\section{Myanmar}
\label{s:myanmar}
\index{Myanmar}\index{Burmese}\index{Mon}\index{Unicode>Myanmar}\index{Fonts>Padauk}

%\newfontfamily\myanmar{Padauk}

The Burmese script (Burmese:{\myanmar မြန်မာအက္ခရာ}; MLCTS: mranma akkha.ra; pronounced: [mjəmà ʔɛʔkʰəjà]) is an abugida in the Brahmic family, used for writing Burmese. It is an adaptation of the Old Mon script[2] or the Pyu script. In recent decades, other alphabets using the Mon script, including Shan and Mon itself, have been restructured according to the standard of the now-dominant Burmese alphabet. Besides the Burmese language, the Burmese alphabet is also used for the liturgical languages of Pali and Sanskrit.

The characters are rounded in appearance because the traditional palm leaves used for writing on with a stylus would have been ripped by straight lines.[3] It is written from left to right and requires no spaces between words, although modern writing usually contains spaces after each clause to enhance readability.

The earliest evidence of the Burmese alphabet is dated to 1035, while a casting made in the 18th century of an old stone inscription points to 984.[1] Burmese calligraphy originally followed a square format but the cursive format took hold from the 17th century when popular writing led to the wider use of palm leaves and folded paper known as \emph{parabaiks}.[3] The alphabet has undergone considerable modification to suit the evolving phonology of the Burmese language.

Mon/Burmese script was added to the Unicode Standard in September, 1999 with the release of version 3.0. It was extended in October, 2009 with the release of version 5.2 and again in June, 2014 with the release of version 7.0.

\begin{docKey}[phd]{myanmar font}{=\meta{font name}}{default none initial Padauk}
Loads the font and creates associated environments and commands.
\end{docKey}

\begin{scriptexample}[]{Myanmar}
\unicodetable{myanmar}{"1000,"1010,"1020,"1030,"1040,"1050,"1060,"1070,"1080,"1090}
\end{scriptexample}


{\myanmar
\begin{tabular}{l l l l}
\toprule
Number         &Numeral	 &Written     &\\
\midrule
               &	          &(MLCTS)     &IPA \\
0	            &၀	          &သုည (su.nya.) & IPA: [θòʊɴɲa̰]\\
1	            & ၁	       &တစ် (tac)	 &IPA: [tɪʔ]\\
2              &၂          &နှစ်        &IPA: [n̥ɪʔ]\\
\bottomrule
\end{tabular}


	
1	၁	တစ်
(tac)	IPA: [tɪʔ]
2	၂	နှစ်
(hnac)	IPA: [n̥ɪʔ]
3	၃	သုံး
(sum:)	IPA: [θóʊɴ]
4	၄	လေး
(le:)	IPA: [lé]
5	၅	ငါး
(nga:)	IPA: [ŋá]
6	၆	ခြောက်
(hkrauk)	IPA: [tɕʰaʊʔ]
7	၇	ခုနစ်
(hku. nac)	IPA: [kʰʊ̀ɴ n̥ɪʔ]2
8	၈	ရှစ်
(hrac)	IPA: [ʃɪʔ]
9	၉	ကိုး
(kui:)	IPA: [kó]
10	၁၀	ဆယ်
(ta. hcai)	IPA: [sʰɛ̀]
}




%\section{Oriya}
\label{s:oriya}
\index{Indic scripts>Oriya}
\epigraph{Oṛiyā is encumbered with the drawback of an excessively awkward and cumbrous written character. ... At first glance, an Oṛiyā book seems to be all curves, and it takes a second look to notice that there is something inside each.}{(G. A. Grierson, \textit{Linguistic Survey of India}, 1903)}

\newfontfamily\oriya[Scale=1.1,Script=Oriya]{Noto Sans Oriya}

\def\oriyatext#1{{\oriya#1}}
The Oriya script or Utkala Lipi (Oriya: \oriyatext{ଉତ୍କଳ ଲିପି}) or Utkalakshara (Oriya: \oriyatext{ଉତ୍କଳାକ୍ଷର}) is used to write the Oriya language, and can be used for several other Indian languages, for example, Sanskrit.

\centerline{\Huge\oriyatext{ଉତ୍କଳ ଲିପି}}

\bgroup
\oriya
୦୧୨୩୪୫୬୭୮୯
ଅ ଆ ଇ ଈ ଉ ଊ ଋ ୠ ଌ ୡ ଏ ଐ ଓ ଔ କ ଖ ଗ ଘ ଙ ଚ ଛ ଜ ଝ ଞ ଟ ଠ ଡ ଢ ଣ ତ ଥ ଦ ଧ ନ ପ ଫ ବ ଵ ଭ ମ ଯ ର ଳ ୱ ଶ ଷ ସ ହ ୟ ଲ
\egroup






\begin{figure}[htbp]
\centering

\includegraphics[width=\linewidth-2\parindent]{oriya-people}

\hspace*{-1em}\caption{Children dressed for celebration of Janmashtami, which marks the birth of Lord Krishna. odisha360.com}
\end{figure}

Comparison of Oṛiyā script with its neighbours

At a first look the great number of signs with round shapes suggests a closer relation to the southern neighbour Telugu than to the other neighbours Bengali in the north and Devanāgarī in the west. The reason for the round shapes in Oriya and Telugu (and also in Kannaḍa and Malayāḷam) is the former method of writing using a stylus to scratch the signs into a palm leaf. These tools do not allow for horizontal strokes because that would damage the leaf.

Oriya letters are mostly round shaped whereas in Devanāgarī and Bengali have horizontal lines. So in most cases the reader of Oṛiyā will find the distinctive parts of a letter only below the hoop. Considering this the  closer relation to Devanāgarī and Bengali exists than to any southern script, though both northern and southern scripts have the same origin, Brāhmī.

Oriya (\oriyatext{ଓଡ଼ିଆ} oṛiā), officially spelled Odia,[3][4] is an Indian language belonging to the Indo-Aryan branch of the Indo-European language family. It is the predominant language of the Indian states of Odisha, where native speakers comprise 80\% of the population,[5] and it is spoken in parts of West Bengal, Jharkhand, Chhattisgarh and Andhra Pradesh. Oriya is one of the many official languages in India; it is the official language of Odisha and the second official language of Jharkhand. [6][7][8] Oriya is the sixth Indian language to be designated a Classical Language in India, on the basis of having a long literary history and not having borrowed extensively from other languages.



\printunicodeblock{./languages/oriya.txt}{\oriya}

\section{Numerals}

{\oriya
\obeylines
୦	୧	୨	୩	୪	୫	୬	୭	୮	୯	୵	୶	୷	୲	୳	୴
{\arial 0	1	2	3	4	5	6	7	8	9	¹⁄₁₆	⅛	³⁄₁₆	¼	½	¾}
}





%\cxset{image = mongolian-people}
\chapter{Mongolian}
The Mongols (Mongolian: Монголчууд, Mongolchuud, [ˈmɔŋ.ɡɔɮ.t͡ʃʊːt]) are an East-Central Asian ethnic group native to Mongolia and China's Inner Mongolia Autonomous Region. They also live as minorities in other regions of China (e.g. Xinjiang), as well as in Russia. Mongolian people belonging to the Buryat and Kalmyk subgroups live predominantly in the Russian federal subjects of Buryatia and Kalmykia.\footnote{Cover image from \href{http://www.bbc.com/news/world-asia-china-25979564}{bbc}}

The Mongols are bound together by a common heritage and ethnic identity. T
heir indigenous dialects are collectively known as the Mongolian language. The ancestors of the modern-day Mongols are referred to as Proto-Mongols.

Mongolian is a member of a language family technically known as “Mongolic”. Apart
from Mongolian, or Mongol proper, the Mongolic language family comprises a dozen
other languages, spoken mainly in regions adjacent to Mongolia. Historically, the
Mongolic language family was formed as a result of the political expansion of the mediaeval,
or “historical”, Mongols under Chinggis Khan (Cingges Xaan) and his descendants
in the 12th–13th centuries. During the initial period of the Mongol empire, the Mongols
controlled, as a politically unified territory, the entire Central Asian belt from the Middle
East to China. The subsequent Mongol dynasty of the Yuan (1279–1368) in the eastern
part of the former Mongol empire, comprised China, Mongolia, Manchuria, Tibet and
Eastern Turkestan.\footcite{book:janhunen2012}

The language of the historical Mongols was based on the local idiom once spoken in
northeastern Mongolia, the native region of Chinggis Khan. With the consolidation of
the political power, this idiom became the koïné of the expanding Mongols, who brought
it to various parts of the empire. The language was widely used in civil and military
administration, and through the Mongol garrisons it gained ground also among local
non-Mongol populations. As a spoken medium, the language of the historical Mongols
is known as Middle Mongol, or Middle Mongolian. Middle Mongol is documented in
a variety of written sources using several different systems of script. With the course of
time, and especially after the collapse of the Mongol empire Middle Mongol was diversified
into several local varieties, from which the modern Mongolic languages have
developed.


Janhunen\footcite{book:janhunen2012} divides the extant Mongolic languages into four geographically
and linguistically distinct branches: Dagur, Common Mongolic,
Shirongolic and Moghol.

\begin{description}

\item[Dagur] branch, located in the northeast (Manchuria) and comprising only the
Dagur language (with several local varieties, including the Amur, Nonni and Hailar
groups of dialects, as well as, since the 18th century, a diaspora group in the Yili
region of Dzungaria); historically, the origins of this branch would seem to be connected
with the earliest breakup of Proto-Mongolic;

\item[Common Mongolic] branch, centered on the traditional homeland of the Mongols
(Mongolia), but extending also to the north (Siberia), east (Manchuria), south
(Ordos) and west (Dzungaria), and comprising a group of closely related forms of
speech, which by the native speakers themselves are often understood as “dialects”
of a single “Mongolian” language;

\item[Shirongolic] branch, located in the Amdo or Kuku Nor (Xeux Noor ‘Blue Lake’)
region of ethnic Tibet (the modern Gansu and Qinghai Provinces of China), and
comprising a number of particularly idiosyncratic and mutually unintelligible
languages spoken by several culturally diversified populations, including Shira
Yughur (Mongolic Yellow Uighur), the Monguor group (Mongghul, Mongghuor,
Mangghuer) and the Bonan group (Bonan, Kangjia, Santa);

\item[Moghol] the Moghol branch, located in Afghanistan and comprising only the Moghol language
(with several local varieties, possibly extinct today).\footcite{book:janhunen2012}
\end{description}

\begin{figure}[htbp]
\includegraphics[width=\textwidth]{mongolian-writing}
\caption{Nova N 176 found in Kyrgyzstan. The manuscript (dating to the 12th century Western Liao) is written in the Mongolic Khitan language using cursive Khitan large script. It has 127 leaves and 15,000 characters.}
\end{figure}

From historical documents it is evident that the lineage represented by the language
of the historical Mongols once had relatives, today technically identified as the Para-
Mongolic languages, spoken until mediaeval times in parts of southwestern Manchuria.

\paragraph{The literary languages}
The earliest known written language for
the historical Mongols was created in the 11th–12th centuries on the basis of a Semitic
alphabet adopted via the Turkic-speaking Ancient Uighurs. The script, in its Mongolian
form, has subsequently become known as the Mongol Script, while the language written
in it is known as Written Mongol or Written Mongolian, or also Literary Mongol
or Literary Mongolian. Written Mongol was reinforced by Chinggis Khan as a general
medium of administration and literature, and in its early form it was essentially identical
with contemporary spoken Middle Mongol, complicated only by certain orthographical
conventions, some of which may actually reflect a stage preceding Middle Mongol and
Proto-Mongolic.

Written Mongol has ever since remained in use as the principal literary language of
the Mongols. Evolving successively through stages termed Pre-Classical (13th to 15th
centuries), Classical (17th to 19th centuries) and Post-Classical (20th century) Written
Mongol, the language, especially as far as its orthographical principles are concerned, still
retains many of its original characteristics. This means that it remains largely unaffected
by the innovations that have taken place in the spoken language and by the diversification
of the latter into the extant modern Mongolic languages. This is particularly true of
the phonological features reflected by the Written Mongol orthography. Written Mongol
has, however, survived only among the speakers of the Common Mongolic idioms, and
even of the latter, the speakers of Buryat and Khamnigan have used it only marginally.
The significance of Written Mongol as a unifying factor for almost all Common
Mongolic speakers can hardly be exaggerated. Even so, its status has been gradually
undermined by the creation of new literary languages, which today cover most of the
Common Mongolic populations living outside of Inner Mongolia. These new literary
languages include:

\begin{enumerate}
\item Written Oirat or the “Clear Script” (Tod Biceg), which was created on the basis of
Written Mongol as early as 1648 for use by the Western Mongols of Dzungaria; this
script is still in use among some of the Oirat groups in Sinkiang;

\item Romanized “Buryat”, which was standardized around 1930 on the basis of what are
actually the Sartul and Tsongol dialects of northern Khalkha, spoken on the Russian
side of the border

\item Cyrillic Buryat, based on the Khori dialect of actual (Eastern) Buryat, which replaced
the earlier Romanized “Buryat” in 1937 and remains in use as the literary language
for the Buryat living in the Russian Federation; the written standard is, however, not
used by the Buryat speakers living in Mongolia and China;

\item Cyrillic Kalmuck, which was standardized in the early 1930s for use by the Volga
Kalmuck, who represent an Oirat diaspora group that has been living under Russian
rule since the 17th century;

\item Cyrillic Khalkha, based on the central dialects of the Khalkha group, which were
developed as the national language of Outer Mongolia after independence, and
which during the 1940s more or less fully replaced Written Mongol as the official
standard language of the country.
\end{enumerate}

\paragraph{Classical Mongolian Script} The classical Mongolian script (in Mongolian script: {\mongolian  ᠮᠣᠩᠭᠣᠯ ᠪᠢᠴᠢᠭ᠌} Mongγol bičig; in Mongolian Cyrillic: Монгол бичиг Mongol bichig), also known as Uyghurjin Mongol bichig, was the first writing system created specifically for the Mongolian language, and was the most successful until the introduction of Cyrillic in 1946. Derived from Uighur, Mongolian is a true alphabet, with separate letters for consonants and vowels. The Mongolian script has been adapted to write languages such as Oirat and Manchu. Alphabets based on this classical vertical script are used in Inner Mongolia and other parts of China to this day to write Mongolian, Sibe and, experimentally, Evenki.
\medskip

\bgroup\par
\noindent
\colorbox{thecodebackground}{\color{black}^^A
\begin{minipage}{\textwidth}^^A
\parindent1pt
\vskip10pt
\leftskip10pt \rightskip\leftskip
\mongolian
\Large
ᠬᠦᠮᠦᠨ ᠪᠦᠷ ᠲᠥᠷᠥᠵᠦ ᠮᠡᠨᠳᠡᠯᠡᠬᠦ ᠡᠷᠬᠡ ᠴᠢᠯᠥᠭᠡ ᠲᠡᠢ᠂ ᠠᠳᠠᠯᠢᠬᠠᠨ ᠨᠡᠷ᠎ᠡ ᠲᠥᠷᠥ ᠲᠡᠢ᠂ ᠢᠵᠢᠯ ᠡᠷᠬᠡ ᠲᠡᠢ ᠪᠠᠢᠠᠭ᠃ ᠣᠶᠤᠨ ᠤᠬᠠᠭᠠᠨ᠂ ᠨᠠᠨᠳᠢᠨ ᠴᠢᠨᠠᠷ ᠵᠠᠶᠠᠭᠠᠰᠠᠨ ᠬᠦᠮᠦᠨ ᠬᠡᠭᠴᠢ ᠥᠭᠡᠷ᠎ᠡ ᠬᠣᠭᠣᠷᠣᠨᠳᠣ᠎ᠨ ᠠᠬᠠᠨ ᠳᠡᠭᠦᠦ ᠢᠨ ᠦᠵᠢᠯ ᠰᠠᠨᠠᠭᠠ ᠥᠠᠷ ᠬᠠᠷᠢᠴᠠᠬᠥ ᠤᠴᠢᠷ ᠲᠠᠢ᠃
\par
\vspace*{10pt}
\end{minipage}
}
\medskip


\paragraph{Unicode Encoding} Mongolian is a Unicode block containing characters for dialects of Mongolian, Manchu, and Sibe languages. It is traditionally written in vertical lines Text direction TDright.svg Top-Down, right across the page, although the Unicode code charts cite the characters rotated to horizontal orientation.
The block has dozens of variation sequences defined for standardized variants.
\bigskip


\unicodetable{mongolian}{"1800,"1810,"1820,"1830,"1840,"1850,"1860,"1870,"1880,"1890,"18A0}
\bigskip

\section{LaTeX}

The \pkg{montex} provides a full system including transliterations.\footcite{montex} There is no as yet support for LuaTeX and I do not see this forthcoming anytime soon. 











%\section{Tibetan}
\label{tibetan}
\index{scripts>tibetan}


Another important Northern Indian member perhaps
derived directly from Gupta---and thus a sister script to Nagari,
Sarada and Pali---is Tibetan. However, the Tibetan
language wears this foreign Indo-Aryan script most uncomfortably.\cite{writing}
The script retains the Indic consonantal alphabet with diacritic
attachments to indicate vowels – but with only one vowel
letter, the /a/, which is the same as the system’s own ‘default’ /a/.
This /a/ letter is then used to attach other diacritics in order to
indicate further vowels. Because the Tibetan language has
changed greatly since c. AD 700 (when the script was first elaborated
from Gupta) while the script has remained almost
unchanged, Tibetan is extremely difficult to read today. Its
greatest problem is that it marks none of the tones of its tonal
language. Though Tibetans have long tried to adapt written
Tibetan to spoken Tibetan, high illiteracy has been the price of
failing to achieve this. Tibetan schools in Tibet, by governmental
decree, now teach only the Chinese script and in the Chinese
language.

\newfontfamily\tibetan{TibMachUni.ttf}

\newfontfamily\tibetan{Qomolangma-Chuyig.ttf}

%A should pick it up automatically \tibetan

Fonts described in this section can be obtained from The Tibetan \& Himalayan Library
\footnote{\url{http://www.thlib.org/tools/scripts/wiki/tibetan\%20machine\%20uni.html}  }

I have tried a few \texttt{Tibetan Machine Uni (TMU)} seems to be used by a number of scholars. 

A tip when you are trying to locate fonts is to find a related article in Wikipedia, such as Tibetan alphabet and inspect the element using your browser to see what fonts are being used.


|style="font-family:'Jomolhari','Tibetan Machine Uni','DDC Uchen', 'Kailash';| 


If you cannot see the script and rather than boxes or question marks then you can search and download one of the fonts in |font-family|.



\begin{docKey}[phd]{language}{ = tibetan}{default none, initial english} 
The key |language=tibetan| sets the default language as Tibetan, using the main font given by the key |tibetan font=TibMachUni.ttf|.

It will also create an environment tibetanlanguage.
\end{docKey}

\begin{docKey}[phd]{tibetan font}{= TibMachUni.ttf} {initial = TibMachUni.ttf} 
The key |tibetan font=font-name| sets the default font for the Tibetan language. It will also create the switch \cmd{\tibetan} for typesetting text in Tibetan.
\end{docKey}


\begin{docEnvironment}{tibetan}{}
\end{docEnvironment}

The environment is created automatically
\begin{texexample}{Tibetan language setttings}{ex:tibetan}
\bgroup
\cxset{language=tibetan, tibetan font = TibMachUni.ttf}

\tibetan Tibetan: དབུ་ཅན\par
ཨོཾ་ཨཿཧཱུྂ་བཛྲ་གུ་རུ་པདྨ་སིདྡྷི་ཧཱུྂ༔\par
\egroup

\begin{tibetanlanguage}
The tibetan environment\par
ཨོཾ་ཨཿཧཱུྂ་བཛྲ་གུ་རུ་པདྨ་སིདྡྷི་ཧཱུྂ༔
\end{tibetanlanguage}
\end{texexample}


The Tibetan alphabet is an \emph{abugida} of Indic origin used to write the Tibetan language as well as Dzongkha\footnote{Spoken in Bhutan.}, the Sikkimese language, Ladakhi, and sometimes Balti. 

The printed form of the alphabet is called \textit{uchen} script (Tibetan: དབུ་ཅན་, Wylie: dbu-can; "with a head") while the hand-written cursive form used in everyday writing is called umê script (Tibetan: དབུ་མེད་, Wylie: dbu-med; "headless").

The alphabet is very closely linked to a broad ethnic Tibetan identity. Besides Tibet, it has also been used for Tibetan languages in Bhutan, India, Nepal, and Pakistan.[1] The Tibetan alphabet is ancestral to the Limbu alphabet, the Lepcha alphabet,[2] and the multilingual 'Phags-pa script.[2]


The Tibetan alphabet is romanized in a variety of ways.[3] This article employs the Wylie transliteration system.

The Tibetan alphabet has thirty basic letters, sometimes known as "radicals", for consonants.[2]

{\tibetanfontfamily
ཀ ka /ká/	ཁ kha /kʰá/	ག ga /kà, kʰà/	ང nga /ŋà/\\
ཅ ca /tʃá/	ཆ cha /tʃʰá/	ཇ ja /tʃà/	ཉ nya /ɲà/\\
ཏ ta /tá/	ཐ tha /tʰá/	ད da /tà, tʰà/	ན na /nà/\\
པ pa /pá/	ཕ pha /pʰá/	བ ba /pà, pʰà/	མ ma /mà/\\
ཙ tsa /tsá/	ཚ tsha /tsʰá/	ཛ dza /tsà/	ཝ wa /wà/ (not originally part of the alphabet)[5]\\
ཞ zha /ʃà/[6]	ཟ za /sà/	འ 'a /hà/[7]\\
ཡ ya /jà/	ར ra /rà/	ལ la /là/\\
ཤ sha /ʃá/[6]	ས sa /sá/	ཧ ha /há/[8]\\
ཨ a /á/\\
}


Tibetan is not a difficult script to read or write, but it is a very complex script to deal with in terms of computer processing (as far as complexity goes I would rate it second only to the Mongolian script). The problem is that written Tibetan comprises complex syllable units (known in Tibetan as a tsheg bar {\tibetan ཚེག་བར}) which although written horizontally may include \emph{vertical} clusters of consonants and vowel signs agglutinating around a base consonant (a vertical cluster is known as a "stack"). 

Thus most words have a horizontal and a vertical dimension, with the result that text is not laid out in a straight line as in most scripts. For example, the word bsGrogs བསྒྲོགས་ (pronounced drok ... obviously!) may be analysed as follows :

\definecolor{lavenderblush}{HTML}{FFF0F5}%
\definecolor{beige}{HTML}{F5F5DC}%


{\tibetan 
\HUGE བསྒྲོགས

{\color{beige}%
\symbol{"0F56}\color{blue!40}\color{red}\symbol{"0F66}\symbol{"0F92}\color{blue!80}\symbol{"0FB2}\color{beige}\symbol{"0F7C}\color{blue!25}\symbol{"0F42}\symbol{"0F66}\symbol{"0F0B}}



\begin{tabular}{|l|}
\symbol{"0F56}\symbol{"0F7C}\\
\symbol{"0F42}\symbol{"0F7C}\\
\symbol{"0F66}\symbol{"0F7C}\\
\symbol{"0F40}\symbol{"0F7C}\\
\end{tabular}
}

\subsection{Unicode Block Tibetan}


\bgroup\large\tibetan
\begin{tabular}{llllllllllllllll l}
\toprule
	           &|0|	&|1|	&|2|	&|3|	&|4|	&|5|	&|6|	&|7|	&|8|	&|9|	&|A|	&|B|	&|C|	&|D|	&|E|	&|F|\\
\midrule
\texttt{U+0F0x}	&ༀ	&༁	&༂	&༃	&༄	&༅	&༆	&༇	&༈	&༉	&༊	&་	&༌  &	།	&༎	&༏\\
\midrule
\texttt{U+0F1x} &༐	&༑	&༒	&༓	&༔	&༕	&༖	&༗	&༘&	༙	&༚	&༛	&༜	&༝	&༞	&༟\\
\midrule
\texttt{U+0F2x} &༠	&༡	&༢	&༣	&༤	&༥	&༦	&༧	&༨	&༩	&༪	&༫	&༬	&༭	&༮	&༯\\
\midrule
\texttt{U+0F3x}	&༰ &༱	 &༲ &༳	&༴ &༵	&༶ & ༷	&༸&	༹	&༺&	༻	&༼&	༽	&༾	&༿\\
\midrule
\texttt{U+0F4x} &ཀ	&ཁ	&ག	&གྷ	&ང	&ཅ	&ཆ	&ཇ	&	&ཉ	&ཊ	&ཋ	&ཌ	&ཌྷ	&ཎ	&ཏ\\
\midrule
\texttt{U+0F5x}	 &ཐ	&ད	&དྷ	&ན	&པ	&ཕ	&བ	&བྷ	&མ	&ཙ	&ཚ	&ཛ	&ཛྷ	&ཝ	&ཞ	&ཟ\\
\midrule
\texttt{U+0F6x} &འ	&ཡ	&ར	&ལ	&ཤ	&ཥ	&ས	&ཧ	&ཨ	&ཀྵ	&ཪ	&ཫ	&ཬ	&&&\\
^^A\texttt{U+0F7x}&&	ཱ &	& &ི	ཱི&	ུ&	ཱུ&	ྲྀ&	ཷ&	ླྀ&	ཹ&	ེ&	ཻ&	ོ&	ཽ&	&ཾ	&ཿ\\
\midrule
\texttt{U+0F8x}&    ྀ   & 	ཱྀ&	ྂ&	&ྃ &	྄	&྅&	྆	&྇	ྈ&	ྉ&	ྊ&	ྋ&	ྌ&	ྍ&	ྎ&	ྏ\\
\midrule
\texttt{U+0F9x} &	ྐ&	ྑ   & 	ྒ &	ྒྷ &	ྔ &	ྕ &	ྖ &	ྗ &		ྙ &	ྚ &	ྛ &	ྜ &	ྜྷ &	ྞ &	ྟ\\
\texttt{U+0FAx} &	ྠ &	ྡ &	ྡྷ &	ྣ &	ྤ &	ྥ &		&ྦ	&ྦྷ	ྨ&	ྩ&	ྪ&	ྫ&	ྫྷ&	ྭ&	ྮ&	ྯ\\
\midrule
\texttt{U+0FBx} 
&	  ྰ 
&	
& ྱ  	 
&ྲ	
&ླ	
&ྴ
&	ྵ
&	ྶ
&	ྷ
&ྸ
&
&
&
&	
&྾	
&྿\\
\midrule
\texttt{U+0FCx}	 &࿀&	࿁&	࿂&	࿃&	࿄&	࿅&	&࿇	&࿈	&࿉	&࿊	&࿋	&࿌	&&	࿎	&࿏\\
\midrule
\texttt{U+0FDx}	&࿐	&࿑	&࿒	&࿓	&࿔	&࿕	&࿖	&࿗	&࿘	&࿙	&࿚	&&&&&\\
\midrule
\texttt{U+0FEx} &&&&&&&&&&&&&&&&\\
\midrule
\texttt{U+0FFx}  &&&&&&&&&&&&&&&&\\
\bottomrule
\end{tabular}
\egroup




\subsection{Fonts for Tibetan}

Fonts for Tibetan need to be downloaded one set of fonts are the \texttt{Qomolangma}. They come in different flavours, but they appear
to offer advantages as compared to the Tibetan Machine Uni.
\medskip


\newfontfamily\betsu{Qomolangma-Betsu.ttf}
\newfontfamily\drutsa{Qomolangma-Drutsa.ttf}
\newfontfamily\chuyig{Qomolangma-Chuyig.ttf}
\newfontfamily\tsumachu{Qomolangma-Tsumachu.ttf}
\newfontfamily\uchensutung{Qomolangma-UchenSutung.ttf}
\newfontfamily\uchensuring{Qomolangma-UchenSuring.ttf}
\newfontfamily\uchensarchen{Qomolangma-UchenSarchen.ttf}
\newfontfamily\uchensarchung{Qomolangma-UchenSarchung.ttf}
\newfontfamily\tsuring{Qomolangma-Tsuring.ttf}
\newfontfamily\TMU{TibMachUni.ttf}
\newfontfamily\himalaya{Microsoft Himalaya}


{
\centering

\renewcommand{\arraystretch}{1.5}

\begin{tabular}{lr}
\toprule
|Qomolangma-Betsu.ttf| & {\betsu  དབུ་མེད }\\
\midrule
|Qomolangma-Chuyig.ttf| &{\chuyig  དབུ་མེད}\\
\midrule
|Qomolangma-Drutsa.ttf| &{\drutsa  དབུ་མེད}\\
\midrule
|Qomolangma-Tsumachu.ttf|&{\tsumachu  དབུ་མེད}\\
\midrule
|Qomolangma-Tsuring.ttf| &{\tsuring  དབུ་མེད}\\
\midrule
|Qomolangma-UchenSarchen.ttf| &{\uchensarchen དབུ་མེད}\\
\midrule
|Qomolangma-UchenSarchung.ttf|&{\uchensarchung དབུ་མེད }\\
\midrule
|Qomolangma-UchenSuring.ttf|&{\uchensuring དབུ་མེད}\\
\midrule
|Qomolangma-UchenSutung.ttf|&{\uchensutung དབུ་མེད }\\
\midrule
|TibMachUni.ttf| &{\TMU དབུ་མེད }\\
\midrule
|Microsoft Himalaya| &{\himalaya དབུ་མེད ཽ}\\
\bottomrule
\end{tabular}

}
\bigskip

\bgroup
\LARGE\tsuring
\noindent༆ །ཨ་ཡིག་དཀར་མཛེས་ལས་འཁྲུངས་ཤེས་བློ  འི་\par
གཏེར༑ །ཕས་རྒོལ་ཝ་སྐྱེས་ཟིལ་གནོན་གདོང་ལྔ་བཞིན།།\par
ཆགས་ཐོགས་ཀུན་བྲལ་མཚུངས་མེད་འཇམ་དབྱངསམཐུས།།\par
མཧཱ་མཁས་པའི་གཙོ་བོ་ཉིད་འགྱུར་ཅིག། །མངྒལཾ༎\par
བསྒྲོགས
\egroup

\subsubsection{Tibetan numbers}
\cxset{language=tibetan, tibetan font = TibMachUni.ttf}

{
\obeylines
\small
TIBETAN DIGIT ZERO\tibetan	༠
TIBETAN DIGIT ONE	\tibetan༡	
TIBETAN DIGIT TWO\tibetan	༢	
TIBETAN DIGIT THREE\tibetan	༣	
TIBETAN DIGIT FOUR	\tibetan ༤	
TIBETAN DIGIT FIVE\tibetan	༥	
TIBETAN DIGIT SIX	\tibetan ༦	
TIBETAN DIGIT SEVEN\tibetan	༧	
TIBETAN DIGIT EIGHT\tibetan	༨	
TIBETAN DIGIT NINE\tibetan	༩	
TIBETAN DIGIT HALF ONE	\tibetan༪	
TIBETAN DIGIT HALF TWO	༫	
TIBETAN DIGIT HALF THREE	༬
TIBETAN DIGIT HALF FOUR ༭	
TIBETAN DIGIT HALF FIVE ༯	
TIBETAN DIGIT HALF SIX	 ༯	
TIBETAN DIGIT HALF SEVEN	༰	
TIBETAN DIGIT HALF EIGHT	༱	
TIBETAN DIGIT HALF NINE	༲	
TIBETAN DIGIT HALF ZERO	༳	
}


Tibetan numbers

The usage is not certain. By some interpretations, this has the value of 9.5. Used only in some traditional contexts, these appear as the last digit of a multidigit number, eg. ༤༬ represents 42.5. These are very rarely used, however, and other uses have been postulated.


\PrintUnicodeBlock{./languages/tibetan.txt}{\himalaya}






\chapter{Tamil}

\epigraph{Women live like bats or owls.\\Labour like beasts\\and die like worms\ldots}{Margaret of Newcastle, 1660, England}



\label{s:tamil}
\newfontfamily\tamil[Scale=1.0, Script=Tamil]{code2000.ttf}

\def\tamiltext#1{{\tamil#1}}

\section{Background and History}

Of all the Dravidian languages Tamil has the longest literary tradition, covering
more than two thousand years. The earliest records are cave inscriptions from
the second century \textsc{bce}; the earliest extant literary text is the grammar
Tolkāppiyam (100 \textsc{bce}), which describes the grammar and poetics of Tamil during
that period. The dating of the Tolkāppiyam is still disputed by scholars proposing dates from
5 \textsc{bce} to 600 \textsc{ce}. 

During its two-thousand-year uninterrupted history, Tamil distinguishes
three different stages: Old Tamil (300 \textsc{bce} to 700 \textsc{ce}), Middle Tamil (700
\textsc{ce} to 1600) and Modern Tamil (1600 \textsc{ce} to the present), each with distinct
grammatical characteristics.\index{Dravidian>Tamil}\index{Tamil}


\begin{figure}[htbp]
\bgroup
\parindent=0pt
\centering
\includegraphics[width=0.9\linewidth-2\parindent]{./images/old-tamil-inscription.jpg}

\caption{Mangulam Tamil Brahmi inscription at Dakshin Chithra, Chennai (wikipedia)}

\egroup
\end{figure}

The Tamil script (\tamiltext{தமிழ் அரிச்சுவடி} tamiḻ ariccuvaṭi) is an abugida script that is used by the Tamil people in India, Sri Lanka, Malaysia and elsewhere, to write the Tamil language, as well as to write the liturgical language Sanskrit, using consonants and diacritics not represented in the Tamil alphabet. Certain minority languages such as Saurashtra, Badaga, Irula, and Paniya are also written in the Tamil script. \index{Surashtra}\index{Badaga}\index{Irula}

The Tamil script has 12 vowels (\tamiltext{உயிரெழுத்து} uyireḻuttu ``soul-letters''), 18 consonants (\tamiltext{மெய்யெழுத்து} meyyeḻuttu ``body-letters").
An additional character, the āytam \tamiltext{ஃ (ஆய்தம்)},  is classified in Tamil grammar as being neither a consonant nor a vowel (\tamiltext{அலியெழுத்து} aliyeḻuttu ``the hermaphrodite letter''), though often considered as part of the vowel set (\tamiltext{உயிரெழுத்துக்கள்} uyireḻuttukkaḷ ``vowel class''). The script, however, is syllabic and not alphabetic. The complete script, therefore, consists of the thirty-one letters in their independent form, and an additional 216 combinant letters representing a total 247 combinations (\tamiltext{உயிர்மெய்யெழுத்து} uyirmeyyeḻuttu) of a consonant and a vowel, a mute consonant, or a vowel alone. These combinant letters are formed by adding a vowel marker to the consonant. Some vowels require the basic shape of the consonant to be altered in a way that is specific to that vowel. Others are written by adding a vowel-specific suffix to the consonant, yet others a prefix, and finally some vowels require adding both a prefix and a suffix to the consonant. In every case the vowel marker is different from the standalone character for the vowel.
The Tamil script is written from left to right.\index{hermaphrodite letter}


\section{Unicode}

Tamil is a Unicode block containing characters for the Tamil, Badaga, and Saurashtra languages of Tamil Nadu India, Sri Lanka, Singapore, and Malaysia. In its original incarnation, the code points U+0B02..U+0BCD were a direct copy of the Tamil characters A2-ED from the 1988 ISCII standard. The Devanagari, Bengali, Gurmukhi, Gujarati, Oriya, Telugu, Kannada, and Malayalam blocks were similarly all based on their ISCII encodings.

\begin{scriptexample}[]{Tamil}
\unicodetable{tamil}{"0B80,"0B90,"0BA0,"0BB0,"0BC0,"0BE0,"0BF0}

\hfill  Typeset with \cmd{\tamil} and \texttt{code2000.ttf}
\end{scriptexample}

\subsection{Tamil Numbers and Numerals}

Originally, Tamils did not use zero, nor did they use positional digits (having separate 
symbols for the numbers 10, 100 and 1000). Symbols for the numbers are similar to 
other Tamil letters, with some minor changes. 

For example, the number 3782 is not written as \tamiltext{௩௭௮௨} as in modern usage. Instead it 
is written as \tamiltext{௩ ௲ ௭ ௱ ௮ ௰ ௨}. This would be read as they are written as 
Three Thousands, Seven Hundreds, Eight Tens, Two; or in Tamil as 
\tamiltext{௩௲௭௱௮௰௨ž}.\footnote{https://cloud.github.com/downloads/raaman/Tamil-Numeral/tamilnumbers.html}

\subsection{Dates}

Once the script is loaded the day, month and year can be loaded using the command  \cmd{\tamildate}, which returns the |\today| formatted as per custom Tamil. 

\begin{center}
\bgroup
\tamil
\begin{tabular}{lll}
day	 &month	&year	\\

௳	&௴	      &௵	\\

u	&mee	      &wa	\\
\end{tabular}
\egroup
\end{center}


\section{Grantha}
\label{s:grantha}

Grantha is a Unicode block containing the ancient Grantha script characters of 6th to 19th century Tamil Nadu and Kerala for writing Sanskrit and Manipravalam. Battled to get it working, as I could not find an appropriate unicode font. The font would need remapping. Unfortunately this is a script with no Noto support.

\begin{figure}[htbp]
\bgroup
\parindent=0pt
\centering
\includegraphics[width=\linewidth]{./images/grantha.jpg}

\caption{An image of a palm leaf manuscript with Sanskrit written in Grantha script (wikipedia)}

\egroup
\end{figure}

\newfontfamily\grantha{e-Grantamil 7}%e-Grantamil 7

\begin{scriptexample}[\grantha]{Tamil}
\unicodetable{grantha}{"0D0,"0D1,"0D2,"1133,"1134,"1135,"1136,"1137}

\hfill  Typeset with \cmd{\grantha} and \texttt{e-Granthamil 7.ttf}
\end{scriptexample}

{
\grantha \char"11311

}

%\newfontfamily\freeserif{FreeSerif}
%
%
%\freeserif \lorem
%\begin{tabular}{lll}
%day	 &month	&year	\\
%
%௳	&௴	      &௵	\\
%
%u	&mee	      &wa	\\
%\end{tabular}






\subsection{Kannada alphabet}
\label{s:kannada}
\index{Scripts>Kannada}

\newfontfamily\kannada[Scale=1.0,Script=Kannada]{Lohit-Kannada.ttf}

\def\kannadatext#1{{\kannada #1}}

The Kannada people known as the Kannadigas and Kannadigaru, (sometimes referred to in English as Canarese),[14] are the people who natively speak Kannada.[15] Kannadigas are mainly found in the state of Karnataka in India. Kannada minorities are also found in the neighboring states Maharashtra,[3] Tamil Nadu,[16] Andhra Pradesh, Goa[17][18] and in most Indian states.[3] The English plural is Kannadigas. After a millennium of disintegration from Old Kannada into various languages,[19][12] sister languages[20] and Kannada dialects,[8] modern Kannada stands among 30 most widely spoken languages of the world as of 2001.[7][6] The Kannadiga diaspora can be found all over the world, mainly in the USA, the United Kingdom, Singapore, the UAE and the rest of the Middle East.[21][22][23][24][25][26]\indexindic{Kannada}

\begin{figure}[htbp]
\centering
\includegraphics[width=\linewidth-2\parindent]{kannada}

\caption{Kannada festival.}
\end{figure}



The Kannada alphabet (\kannadatext{ಕನ್ನಡ ಲಿಪಿ}) is an abugida of the Brahmic family,[2] used primarily to write the Kannada language, one of the Dravidian languages of southern India. Several minor languages, such as Tulu, Konkani, Kodava, and Beary, also use alphabets based on the Kannada script.[3] The Kannada and Telugu scripts share high mutual intellegibility with each other, and are often considered to be regional variants of single script. Similarly, Goykanadi, a variant of Old Kannada, has been historically used to write Konkani in the state of Goa.[4]\index{Indic Languages>Konkani}\indexindic{Tulu}\indexindic{Kodava}\indexindic{Beary}



\begin{scriptexample}[]{Kannada}
\centerline{\large\kannadatext{ಙ	ಙ್ಕ	ಙ್ಖ	ಙ್ಗ	ಙ್ಘ	ಙ್ಙ	ಙ್ಚ	ಙ್ಛ	ಙ್ಜ	ಙ್ಝ	ಙ್ಞ	ಙ್ಟ	ಙ್ಠ	ಙ್ಡ	ಙ್ಢ}}
\end{scriptexample}

\medskip

The Kannada script (aksharamale or varnamale) is a phonemic abugida of forty-nine letters, and is written from left to right. The character set is almost identical to that of other Brahmic scripts. Consonantal letters imply an inherent vowel. Letters representing consonants are combined to form digraphs (ottaksharas) when there is no intervening vowel. Otherwise, each letter corresponds to a syllable.

The letters are classified into three categories: swara (vowels), vyanjana (consonants), and yogavaahaka (part vowel, part consonant). \index{swara}\index{vyanjana}\index{yogavaahaka}

The Kannada words for a letter of the script are akshara, akkara, and varna. Each letter has its own form (ākāra) and sound (shabda), providing the visible and audible representations, respectively. Kannada is written from left to right.[7]


% image https://www.quora.com/Why-is-regional-chauvinism-very-high-in-Karnataka


\section{Osmanian Alphabet}

\bgroup
\newfontfamily\osmanian{code2001.ttf}
\osmanian
𐒚𐒁𐒖𐒄 𐒚𐒐 𐒚 𐒎𐒚𐒍𐒚𐒐 𐒑𐒚𐒒𐒠𐒚𐒐 𐒎𐒚𐒑𐒁𐒗 𐒚𐒁𐒖𐒄 𐒚𐒌𐒖𐒄 𐒚𐒁𐒖𐒄𐒖 𐒚
𐒌𐒜
\egroup





\cxset{steward,
  offsety=0cm,
  image={ethiopianbride.jpg},
  texti={An introduction to the use of font related commands. The chapter also gives a historical background to font selection using \tex and \latex. },
  textii={In this chapter we discuss keys that are available through the \texttt{phd} package and give a background as to how fonts are used
in \latex.
 },
 pagestyle = empty,
}




\cxset{steward,
  offsety=0cm,
  image={fellah-woman.jpg},
  texti={An introduction to the use of font related commands. The chapter also gives a historical background to font selection using \tex and \latex. },
  textii={In this chapter we discuss keys that are available through the \texttt{phd} package and give a background as to how fonts are used
in \latex.
 },
 pagestyle = empty
}
