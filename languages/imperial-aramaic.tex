\section{Imperial Aramaic}
\label{s:imperialaramaic}

\newfontfamily\imperialaramaic{NotoSansImperialAramaic-Regular.ttf}
The Aramaic alphabet is adapted from the Phoenician alphabet and became distinctive from it by the 8th century BCE. It was used to write the Aramaic language. The letters all represent consonants, some of which are matres lectionis, which also indicate long vowels.
\medskip

\begin{figure}[htbp]
\centering
\includegraphics[width=0.6\textwidth]{./images/elephantine-papyrus.jpg}

\caption{The Elephantine papyri are ancient Jewish papyri dating to the 5th century BC, requesting the rebuilding of a Jewish temple. It also name three persons mentioned in Nehemiah: Darius II, Sanballat the Horonite and Johanan the high priest.}

\end{figure}


The Aramaic alphabet is historically significant, since virtually all modern Middle Eastern writing systems can be traced back to it, as well as numerous non-Chinese writing systems of Central and East Asia. This is primarily due to the widespread usage of the Aramaic language as both a lingua franca and the official language of the Neo-Assyrian Empire, and its successor, the Achaemenid Empire. Among the scripts in modern use, the Hebrew alphabet bears the closest relation to the Imperial Aramaic script of the 5th century BC, with an identical letter inventory and, for the most part, nearly identical letter shapes.

Writing systems that indicate consonants but do not indicate most vowels (like the Aramaic one) or indicate them with added diacritical signs, have been called abjads by Peter T. Daniels to distinguish them from later alphabets, such as Greek, that represent vowels more systematically. This is to avoid the notion that a writing system that represents sounds must be either a syllabary or an alphabet, which implies that a system like Aramaic must be either a syllabary (as argued by Gelb) or an incomplete or deficient alphabet (as most other writers have said); rather, it is a different type.

The Imperial Aramaic alphabet was added to the Unicode Standard in October 2009 with the release of version 5.2.
The Unicode block for Imperial Aramaic is \unicodenumber{U+10840–U+1085F}.

\begin{scriptexample}[]{Aramaic}
\unicodetable{imperialaramaic}{"10840,"10850}
\end{scriptexample}




\PrintUnicodeBlock{./languages/imperial-aramaic.txt}{\imperialaramaic}