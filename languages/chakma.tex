\cxset{image=chakma.jpg}
\chapter{Chakma}
\label{s:chakma}

\newfontfamily\chakma{RibengUni.ttf}

The Chakma alphabet (Ajhā pāṭh), also called Ojhapath, Ojhopath, Aaojhapath, is an abugida used for the Chakma language and which is being adapted for the Tanchangya language.[1] The forms of the letters are quite similar to those of the Burmese script.

\begin{figure}[htbp]
\includegraphics[width=\linewidth-2\parindent]{chakma}
\end{figure}

The Chakma (Chakma or ), also known as the Changma, are a Tibeto-Burman tribe of the Chittagong Hill Tracts inBangladesh. Today, the geographic distribution of Chakmas is spread across Bangladesh and parts of northeastern India, westernBurma, China and diaspora communities in North America and Europe. Within the CHT, the Chakma are the largest ethnic group and make up half of the region's population. In Burma, they are known as Daingnet people. The Chakma are divided into 46 clans orGozas. They have their own language, customs and culture, and profess Theravada Buddhism. The Chakma Royal Family is one of the major Buddhist royal houses of the South Asia.

Chakmas are Tibeto-Burman, and are thus closely related to tribes in the foothills of the Himalayas. The Chakmas are believed to be originally from Arakan who later on immigrated to Bangladesh in around fifteenth century, settling in the Cox's Bazar District, the Korpos Mohol area, and in the Indian states of Mizoram, Arunachal Pradesh, Tripura.\href{http://thechakmadiary.weebly.com/about.html}{thechakmadiary}

The Arakanese referred to the Chakmas as Saks or Theks. In 1546, when the king of Arakan, Meng Beng, was engaged in a battle with the Burmese, the Sak king appeared from the north and attacked Arakan, and occupied the Ramu of Cox's Bazar, the then territory of the kingdom of Arakan
\bgroup
\obeylines
\chakma
𑄇𑄳𑄇 Kkā = 𑄇 Kā + 𑄳 VIRAMA + 𑄇 Kā
𑄇𑄳𑄑 Ktā = 𑄇 Kā + 𑄳 VIRAMA + 𑄑 Tā
𑄇𑄳𑄖 Ktā = 𑄇 Kā + 𑄳 VIRAMA + 𑄖 Tā
𑄇𑄳𑄟 Kmā = 𑄇 Kā + 𑄳 VIRAMA + 𑄟 Mā
𑄇𑄳𑄌 Kcā = 𑄇 Kā + 𑄳 VIRAMA + 𑄌 Cā
𑄋𑄳𑄇 ńkā = 𑄋 ńā + 𑄳 VIRAMA + 𑄇 Kā
𑄋𑄳𑄉 ńkā = 𑄋 ńā + 𑄳 VIRAMA + 𑄉 Gā
𑄌𑄳𑄌 ccā = 𑄌 cā + 𑄳 VIRAMA + 𑄌 Cā

\egroup

Fonts for the script are not available easily but the
the script can be typeset using \texttt{RibengUni.ttf} which is available at \url{http://uni.hilledu.com/}. 

\begin{scriptexample}[]{Chakma}
\unicodetable{chakma}{"11100,"11110,"11120,"11130,"11140}

\texttt{RibengUni.ttf}
\end{scriptexample}


\printunicodeblock{./languages/chakma.txt}{\chakma}
