\section{Malayalam}
\label{s:malayalam}
\newfontfamily\malayam[Scale=1.1]{Lohit-Malayalam.ttf}

\def\malamtext#1{{\malayam#1}}


Malayalam is a language spoken by the native people of southwestern India (from Thuckalay to Talapady).According to the Indian census of 2011, there were 32,299,239 speakers of Malayalam in Kerala, making up 93.2\% of the total number of Malayalam speakers in India, and 96.74\% of the total population of the state. There were a further 701,673 (2.1\% of the total number) in Karnataka, 957,705 (2.7\%) in Tamil Nadu, and 406,358 (1.2\%) in Maharashtra. The number of Malayalam speakers in Lakshadweep is 51,100, which is only 0.15\% of the total number, but is as much as about 84\% of the population of Lakshadweep. In all, Malayalis made up 3.22\% of the total Indian population in 2011. Of the total 34,713,130 Malayalam speakers in India in 2011, 33,015,420 spoke the standard dialects, 19,643 spoke the Yerava dialect and 31,329 spoke non-standard regional variations like Eranadan.[37] As per the 1991 census data, 28.85\% of all Malayalam speakers in India spoke a second language and 19.64\% of the total knew three or more languages.


\includegraphics[width=\textwidth]{nangeli}
https://feminisminindia.com/2016/09/12/kerala-breast-tax-nangeli/


Large numbers of Malayalis have settled in Delhi, Bangalore, Hyderabad, Mumbai (Bombay), Pune and Chennai (Madras). A large number of Malayalis have also emigrated to the Middle East, the United States, and Europe. There were 179,860 speakers of Malayalam in the United States, according to the 2000 census, with the highest concentrations in Bergen County, New Jersey and Rockland County, New York.[38] There were 7,093 Malayalam speakers in Australia in 2006.[39] The 2001 Canadian census reported 7,070 people who listed Malayalam as their mother tongue, mainly in Toronto. The 2006 New Zealand census reported 2,139 speakers.[40] 134 Malayalam speaking households were reported in 1956 in Fiji. There is also a considerable Malayali population in the Persian Gulf regions, especially in Dubai and Doha. Recently a Keralite is elected as mayor in Loughten town of England.

The Malayalam script (Malayalam: \malamtext{മലയാളലിപി}, Malayāḷalipi, IPA: [mɐləjaːɭɐ lɪβɪ], also known as Kairali script (Malayalam: \malamtext{കൈരളീലിപി}), is a Brahmic script used commonly to write the Malayalam language—which is the principal language of the Indian state of Kerala, spoken by 35 million people in the world.[3] Like many other Indic scripts, it is an alphasyllabary (\textit{abugida}), a writing system that is partially “alphabetic” and partially syllable-based. The modern Malayalam alphabet has 15 vowel letters, 41 consonant letters, and a few other symbols. The Malayalam script is a Vattezhuttu script, which had been extended with Grantha script symbols to represent Indo-Aryan loanwords.[4] The script is also used to write several minority languages such as Paniya, Betta Kurumba, and Ravula.[5] The Malayalam language itself was historically written in several different scripts.

\begin{scriptexample}[]{Malayalam}
\centerline{\Huge\malamtext{കൈരളീലിപി}}
\end{scriptexample}