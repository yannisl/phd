\section{Malayalam}
\label{sec:malayam}
\newfontfamily\malayam[Scale=1.1]{Lohit-Malayalam.ttf}

\def\malamtext#1{{\malayam#1}}

The Malayalam script (Malayalam: \malamtext{മലയാളലിപി}, Malayāḷalipi, IPA: [mɐləjaːɭɐ lɪβɪ], also known as Kairali script (Malayalam: \malamtext{കൈരളീലിപി}), is a Brahmic script used commonly to write the Malayalam language—which is the principal language of the Indian state of Kerala, spoken by 35 million people in the world.[3] Like many other Indic scripts, it is an alphasyllabary (\textit{abugida}), a writing system that is partially “alphabetic” and partially syllable-based. The modern Malayalam alphabet has 15 vowel letters, 41 consonant letters, and a few other symbols. The Malayalam script is a Vattezhuttu script, which had been extended with Grantha script symbols to represent Indo-Aryan loanwords.[4] The script is also used to write several minority languages such as Paniya, Betta Kurumba, and Ravula.[5] The Malayalam language itself was historically written in several different scripts.

\begin{scriptexample}[]{Malayalam}
\centerline{\Huge\malamtext{കൈരളീലിപി}}
\end{scriptexample}