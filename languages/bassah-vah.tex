\cxset{offsety = 0pt,
       image   = bassa,
       texti   = \lorem,
       textii  = \lorem }   
       
\newfontfamily\bassa{JG Bassa Vah Unicode}
\newfontfamily\autonym{autonym}

\chapter{The Bassa Vah Alphabet}

\epigraph{It is said in Africa that when an elder dies, it is as if a library has burned down. }{Melvin Barrolle SYLF Final Report}

The Bassa people of Liberia are a West African ethnic group primarily found in its central coastal regions. They live in Grand Bassa, Rivercess, Margibi and Montserrado counties.[2] In Liberia's capital of Monrovia, they are the largest ethnic group.[3] With an overall population of about 0.57 million, they are the second largest ethnic group in Liberia (13.4\%), after the Kpelle people (20.3\%).[1] Small Bassa communities are also found in Sierra Leone and Ivory Coast.
The Bassa speak the Bassa language, a Kru language that belongs to the Niger-Congo family of languages.[4] They had their own pictographic writing system but it went out of use in the 19th century, was rediscovered among the slaves of Brazil and the West Indies in 1890s, and reconstructed in early 1900 by Thomas Flo Darvin Lewis.[5][6] The revived signs-based script is called Ehni Ka Se Fa.[7]\index{Liberia}

In local languages, the Bassa people are also known as Gboboh, Adbassa or Bambog-Mbog people.[8]

\begin{figure}[htbp]
\parindent=0pt

\includegraphics[width=\linewidth]{bassa}

\caption{Bassah Women, 1922 postcard.}
\end{figure}


From the early nineteenth century, Africa witnessed the appearance
of many indigenous vernacular writing systems, syllabic and alphabetic.
These include scripts for Vai, Mende, Lorna, Kpelle and Bassa, covering
adjacent areas of Liberia and Sierra Leone. Further to the north along the Atlantic
coast other languages including Manding, W olof, Fula, Barnum and Bagam
developed scripts of their own. Many of these scripts were associated at their
inception with revelations and supernatural inspirations of one kind or another.
Their creation and employment was in many cases governed by a spirit of secretiveness,
which may be one of the reasons why relatively little is known about
them. As they have to compete with the widespread Roman and Arabic alphabets,
most of these systems are limited in use with respect to both literary functions
and communities of users, but they are remarkable for their originality and precision
in the transcription of sounds.



The Bassa script, known as Bassa vah or simply vah ('throwing a sign' in Bassa) is an alphabet for writing the Bassa language of Liberia. It was invented by Dr. Thomas Flo Lewis, who has instigated publishing of limited materials in the language from the mid-1900s through the 1930s, with its height in the 1910s and 1920s. It is alleged that some of the signs are based on native Bassa pictograms revealed by a former slave. It is not clear what connection it may have had with neighboring scripts, but type was cast for it, and an association for its promotion was formed in Liberia in 1959. It is not used contemporarily and has been classified as a failed script.[1] Its creation should be distinguished from other orthographic attempts in the 1830s by European missionaries.

\begin{latexquotation}
Some of the most fascinating information I received through the oral testimonies however
concerned the Bassa Vah script itself. It was generally agreed upon by the interviewees that the
Bassa Vah script existed long before Thomas Narven Lewis was born. Oral history suggests it
emerged in the sixteenth or seventeenth century as a tool for resistance against European and
African slave raiders. Members of the Bassa secret societies would chew leaves that left particular
indentations and toss them in areas known for considerable slave raiding activity. This method of
communication was widely understood by the Bassa peoples who heeded the warnings and avoided
these identified areas. This creative resistance strategy adds texture to the contemporary literature on
African resistance to the slave trade, which has so far focused on armed combat and the creation of
maroon spaces.7 This finding also (partly) provides a possible explanation as to why cliometricians
have concluded that the windward coast region was the least affected region in the Trans-Atlantic
Slave Trade.
\end{latexquotation}

Vah is a true alphabet, with 23 consonant letters, 7 vowel letters, and 5 tone diacritics, which are placed inside the vowels. It also has its own marks for commas and periods.

The script is an alphabet with a tone marking system consisting of 23 consonant
letters, seven vowel letters . and five tone diacritics. In table 2 the consonant
letters are arranged in lenis / fortis pairs. The numbering refers to the conventional
order of the alphabet which is also known as ni-ka-se1e using the first four consonant
letter names. Vowels are taken as a separate set not included in the
sequence. In the table they are arranged roughly in phonological order from
unrounded to rounded and front to back vowels. As illustrated for the letter i in
the table, tone marks are placed as diacritics within vowel letters. The sign + is
used as a punctuation mark corresponding to a full stop.


\begin{figure}[htbp]
\parindent0pt
\centering

\includegraphics[width=0.7\textwidth]{bassah-vah}

\caption{Sample of “classic” printed style Bassa Vah text. from unicode.org proposal document.}
\end{figure}

Amazingly are the tone marks.

{
\bassa
𖫪𖫴\kern-4pt𖫫𖫴
}


\bgroup
\obeylines\bassa
16AD0 𖫐 BASSA VAH LETTER ENNI
16AD1 𖫑 BASSA VAH LETTER KA
16AD2 𖫒 BASSA VAH LETTER SE
16AD3 𖫓 BASSA VAH LETTER FA
16AD4 𖫔 BASSA VAH LETTER MBE
16AD5 𖫕 BASSA VAH LETTER YIE
16AD6 𖫖 BASSA VAH LETTER GAH
16AD7 𖫗 BASSA VAH LETTER DHII
16AD8 𖫘 BASSA VAH LETTER KPAH
16AD9 𖫙 BASSA VAH LETTER JO
16ADA 𖫚 BASSA VAH LETTER HWAH
16ADB 𖫛 BASSA VAH LETTER WA
16ADC 𖫜 BASSA VAH LETTER ZO
16ADD 𖫝 BASSA VAH LETTER GBU
16ADE 𖫞 BASSA VAH LETTER DO
16ADF 𖫟 BASSA VAH LETTER CE
16AE0 𖫠 BASSA VAH LETTER UWU
16AE1 𖫡 BASSA VAH LETTER TO
16AE2 𖫢 BASSA VAH LETTER BA
16AE3 𖫣 BASSA VAH LETTER VU
16AE4 𖫤 BASSA VAH LETTER YEIN
16AE5 𖫥 BASSA VAH LETTER PA
16AE6 𖫦 BASSA VAH LETTER WADDA

\egroup
\medskip


\begin{scriptexample}[]{bassah-vah}
\unicodetable{bassa}{"16AD0,"16AE0,"16AF0}

See \url{https://github.com/athinkra/bassa-vah} for font.
\end{scriptexample}

A keyboard to input the script is also available at\url{ http://athinkra.github.io/bassa-vah/tools/\#?load=0x16ad0-bassa_vah.json}. This opens  a browser.

%See also http://www.reocities.com/jglavy/african.html

%http://www.tokyofoundation.org/sylff/pdf/fellows/2011_mb_34.pdf

\bgroup
\obeylines
\pan
Qafár af
Аҧсшәа
Acèh
Адыгэбзэ
Adygabze
زَوُن
Afrikaans
अहिराणी
Akan
Albaamo innaaɬiilka
Gegë
አማርኛ
aragonés
Ænglisc
अङ्गिका
العربية
ܐܪܡܝܐ
mapudungun
Araona
Dziri
Maġribi
مصرى
অসমীয়া
American sign language
asturianu
авар
Kotava
Aymar aru
azərbaycanca
آذربايجانجا
азәрбајҹанҹа
آذربايجانجا
башҡортса
Boarisch
žemaitėška
Batak Toba
بلوچی مکرانی
Bikol Central
беларуская (тарашкевіца)
беларуская
Bahasa Betawi
படகா
български
भोजपुरी
Bislama
Bahasa Banjar
bamanankan
বাংলা
བོད་ཡིག
বিষ্ণুপ্রিয়া মণিপুরী
بختياري
brezhoneg
Bráhuí
बड़ो
bosanski
Iriga Bicolano
ᨅᨔ ᨕᨘᨁᨗ
буряад
català
Chavacano de Zamboanga
Mìng-dĕ̤ng-ngṳ̄
нохчийн
Cebuano
Chamoru
Choctaw
ᏣᎳᎩ
Tsetsêhestâhese
کوردی
corsu
Capiceño
ᓀᐦᐃᔭᐍᐏᐣ
Nēhiyawēwin
qırımtatarca
къырымтатарджа
qırımtatarca
česky
kaszëbsczi
словѣ́ньскъ / ⰔⰎⰑⰂⰡⰐⰠⰔⰍⰟ
Чӑвашла
Cymraeg
dansk
Österreichisches Deutsch
Schweizer Hochdeutsch
Deutsch (Sie-Form)
Deutsch
Zazaki
dolnoserbski
Dusun Bundu-liwan
ދިވެހިބަސް
ཇོང་ཁ
eʋegbe
Emiliàn
Ελληνικά
emiliàn e rumagnòl
Canadian English
British English
English
Esperanto
español de America Latina
español (formal)
español
Yup'ik
eesti
euskara
estremeñu
فارسی
Fulfulde
suomi
Tagalog
meänkieli
Võro
Na Vosa Vakaviti
føroyskt
français
français cadien
arpetan
Nordfriisk
furlan
Frysk
Gaeilge
Gagauz
Alekano
赣语(简体
贛語
Dari
Guadeloupean Creole French
Gàidhlig
galego
گیلکی
Avañe'ẽ
कोंकणी
Konknni
𐌲𐌿𐍄𐌹𐍃𐌺
Ἀρχαία ἑλληνικὴ
Alemannisch
ગુજરાતી
Wayúu
Gurenɛ
Gaelg
هَوُسَ
Hausa
Hak-kâ-fa
Hawai`i
עברית
हिन्दी
Fiji Hindi
फ़ीजी हिन्दी
Fiji Hindi
Ilonggo
छत्तीसगढ़ी
Hiri Motu
hrvatski
hornjoserbsce
湘语
Kreyòl ayisyen
Magyar (magázó)
magyar
Հայերեն
Otsiherero
interlingua
Bahasa Indonesia
Interlingue
Igbo
ꆇꉙ
Iñupiak
ᐃᓄᒃᑎᑐᑦ
inuktitut
Ilokano
ГӀалгӀай
Ido
íslenska
italiano
ᐃᓄᒃᑎᑐᑦ
日本語
Patois
Lojban
jysk
Basa Jawa
ꦧꦱꦗꦮ
ქართული
Qaraqalpaqsha
Taqbaylit
Адыгэбзэ
Qabardjajəbza
Адыгэбзэ
Kabuverdianu
Kongo
Kaingáng
کھوار
Gĩkũyũ
Kırmancki
Kwanyama
қазақша
قازاقشا (تٶتە)
қазақша
qazaqşa
kalaallisut
ភាសាខ្មែរ
ಕನ್ನಡ
한국어 (조선)
한국어
Перем Коми
Kanuri
къарачай-малкъар
Krio
Kinaray-a
Karjala
کٲشُر
कॉशुर
کٲشُر
Bafia
Ripoarisch
Kurdî
كوردي
Kurdî
коми
kernowek
Кыргызча
Latina
Ladino
לאדינו
Lëtzebuergesch
лакку
лезги
Lingua Franca Nova
Luganda
Limburgs
Ligure
Līvõ kēļ
Ladin
lumbaart
lingála
ລາວ
Silozi
lietuvių
latgaļu
Mizo ţawng
latviešu
文言
Lazuri
मैथिली
Basa Banyumasan
мокшень
Morisyen
Malagasy
Ebon
олык марий
Māori
Mi'kmaq
Baso Minangkabau
македонски
മലയാളം
монгол
ᠮᠠᠨᠵᡠ ᡤᡳᠰᡠᠨ
মেইতেই লোন্
ဘာသာ မန်
молдовеняскэ
मराठी
кырык мары
Bahasa Melayu
Malti
Musi
Mvskoke
Mirandés
Behase Mentawei
မြန်မာဘာသာ
эрзянь
مازِرونی
Dorerin Naoero
Nāhuatl
Bân-lâm-gú
Nnapulitano
norsk (bokmål)
Nedersaksisch
Plattdüütsch
नेपाली
नेपाल भाषा
Oshiwambo
ko e vagahau Niuē
Ao
Nederlands (informeel)
Nederlands
norsk (nynorsk)
norsk
Novial
ߒߞߏ
Nouormand
Sesotho sa Leboa
Diné bizaad
Chi-Chewa
occitan
Oromoo
ଓଡ଼ିଆ
Ирон
ਪੰਜਾਬੀ
Pangasinan
Kapampangan
Papiamentu
Picard
Deitsch
Plautdietsch
Pälzisch
पालि
Norfuk / Pitkern
Pijin
Pökoot
polski
Piemontèis
پنجابی
Ποντιακά
Nawat
Prūsiskan
پښتو
português do Brasil
português
Runa Simi
Runa shimi
arero rapa nui
Rumagnôl
Tarifit
ရခိုင်
rumantsch
Romani
Kirundi
română
Armãneashce
tarandíne
Faeag Rotuma
русский
русиньскый
Armãneashce
Влахесте
Megleno-Romanian (Greek script)
Vlăheşte
Kinyarwanda
ʔucināguci
संस्कृतम्
саха тыла
Santali
ꢱꣃꢬꢵꢯ꣄ꢡ꣄ꢬꢵ
sardu
sicilianu
Scots
سنڌي
Sassaresu
sámegiella
Cmique Itom
Sängö
žemaitėška
srpskohrvatski
Tašlḥiyt
ⵜⴰⵛⵍⵃⵉⵜ
Tašlḥiyt
လိၵ်ႈတႆး
සිංහල
Simple English
slovenčina
slovenščina
Schläsch
Salırça
Bahasa Selayar
ܣܘܪܝܝܐ
Gagana Samoa
åarjelsaemien
julevsámegiella
anarâškielâ
sää´mǩiõll
chiShona
Soomaaliga
shqip
3
српски
srpski
Sranantongo
SiSwati
Sesotho
Seeltersk
Basa Sunda
svenska
Kiswahili
Shikomoro
Säggssch
ślůnski
தமிழ்
ತುಳು
తెలుగు
tetun
тоҷикӣ
tojikī
тоҷикӣ
ไทย
ትግርኛ
Türkmençe
ЦӀаьхна миз
Tagalog
толышә зывон
Setswana
lea faka-Tonga
Toki Pona
Tok Pisin
Türkçe
Kokborok (Tripuri)
Ṫuroyo
Xitsonga
Τσακωνικά
татарча
tatarça
Tati
chiTumbuka
Twi
Tweants
Reo Mā`ohi
тыва дыл
ⵜⴰⵎⴰⵣⵉⵖⵜ
удмурт
ئۇيغۇرچە
uyghurche
уйғурчә
українська
اردو
oʻzbekcha
Tshivenda
vèneto
vepsän kel’
Tiếng Việt
West-Vlams
Mainfränkisch
Volapük
Vaďďa
Võro
walon
Winaray
Faka'uvea
Wolof
吴语
хальмг
isiXhosa
მარგალური
Eastern Yiddish
ייִדיש
Yorùbá
Ненэцяʼ вада
ñe'engatú
Maaya T'aan
粵語
Vahcuengh
Zeêuws
中文
文言
中文(中国大陆)
中文(简体)
中文(繁體)
中文(香港)
Bân-lâm-gú
中文(澳門)
中文(马来西亚)
中文(新加坡)
中文(台灣)
粵語
isiZulu
Alemannisch
Thuɔŋjäŋ
ܐܬܘܪܝܐ 
ܣܘܪܝܝܐ
ܣܘܼܪܲܝܬ
ܟܠܕܝܐ


BAMUM

ꚠ	ꚡ	ꚢ	ꚣ	ꚤ	ꚥ	ꚦ	ꚧ	ꚨ	ꚩ	ꚪ	ꚫ	ꚬ	ꚭ	ꚮ	ꚯ

\egroup












