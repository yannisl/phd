\chapter{Tamil}

\epigraph{Women live like bats or owls.\\Labour like beasts\\and die like worms\ldots}{Margaret of Newcastle, 1660, England}



\label{s:tamil}
\newfontfamily\tamil[Scale=1.0, Script=Tamil]{code2000.ttf}

\def\tamiltext#1{{\tamil#1}}

\section{Background and History}

Of all the Dravidian languages Tamil has the longest literary tradition, covering
more than two thousand years. The earliest records are cave inscriptions from
the second century \textsc{bce}; the earliest extant literary text is the grammar
Tolkāppiyam (100 \textsc{bce}), which describes the grammar and poetics of Tamil during
that period. The dating of the Tolkāppiyam is still disputed by scholars proposing dates from
5 \textsc{bce} to 600 \textsc{ce}. 

During its two-thousand-year uninterrupted history, Tamil distinguishes
three different stages: Old Tamil (300 \textsc{bce} to 700 \textsc{ce}), Middle Tamil (700
\textsc{ce} to 1600) and Modern Tamil (1600 \textsc{ce} to the present), each with distinct
grammatical characteristics.\index{Dravidian>Tamil}\index{Tamil}


\begin{figure}[htbp]
\bgroup
\parindent=0pt
\centering
\includegraphics[width=0.9\linewidth-2\parindent]{./images/old-tamil-inscription.jpg}

\caption{Mangulam Tamil Brahmi inscription at Dakshin Chithra, Chennai (wikipedia)}

\egroup
\end{figure}

The Tamil script (\tamiltext{தமிழ் அரிச்சுவடி} tamiḻ ariccuvaṭi) is an abugida script that is used by the Tamil people in India, Sri Lanka, Malaysia and elsewhere, to write the Tamil language, as well as to write the liturgical language Sanskrit, using consonants and diacritics not represented in the Tamil alphabet. Certain minority languages such as Saurashtra, Badaga, Irula, and Paniya are also written in the Tamil script. \index{Surashtra}\index{Badaga}\index{Irula}

The Tamil script has 12 vowels (\tamiltext{உயிரெழுத்து} uyireḻuttu ``soul-letters''), 18 consonants (\tamiltext{மெய்யெழுத்து} meyyeḻuttu ``body-letters").
An additional character, the āytam \tamiltext{ஃ (ஆய்தம்)},  is classified in Tamil grammar as being neither a consonant nor a vowel (\tamiltext{அலியெழுத்து} aliyeḻuttu ``the hermaphrodite letter''), though often considered as part of the vowel set (\tamiltext{உயிரெழுத்துக்கள்} uyireḻuttukkaḷ ``vowel class''). The script, however, is syllabic and not alphabetic. The complete script, therefore, consists of the thirty-one letters in their independent form, and an additional 216 combinant letters representing a total 247 combinations (\tamiltext{உயிர்மெய்யெழுத்து} uyirmeyyeḻuttu) of a consonant and a vowel, a mute consonant, or a vowel alone. These combinant letters are formed by adding a vowel marker to the consonant. Some vowels require the basic shape of the consonant to be altered in a way that is specific to that vowel. Others are written by adding a vowel-specific suffix to the consonant, yet others a prefix, and finally some vowels require adding both a prefix and a suffix to the consonant. In every case the vowel marker is different from the standalone character for the vowel.
The Tamil script is written from left to right.\index{hermaphrodite letter}


\section{Unicode}

Tamil is a Unicode block containing characters for the Tamil, Badaga, and Saurashtra languages of Tamil Nadu India, Sri Lanka, Singapore, and Malaysia. In its original incarnation, the code points U+0B02..U+0BCD were a direct copy of the Tamil characters A2-ED from the 1988 ISCII standard. The Devanagari, Bengali, Gurmukhi, Gujarati, Oriya, Telugu, Kannada, and Malayalam blocks were similarly all based on their ISCII encodings.

\begin{scriptexample}[]{Tamil}
\unicodetable{tamil}{"0B80,"0B90,"0BA0,"0BB0,"0BC0,"0BE0,"0BF0}

\hfill  Typeset with \cmd{\tamil} and \texttt{code2000.ttf}
\end{scriptexample}

\subsection{Tamil Numbers and Numerals}

Originally, Tamils did not use zero, nor did they use positional digits (having separate 
symbols for the numbers 10, 100 and 1000). Symbols for the numbers are similar to 
other Tamil letters, with some minor changes. 

For example, the number 3782 is not written as \tamiltext{௩௭௮௨} as in modern usage. Instead it 
is written as \tamiltext{௩ ௲ ௭ ௱ ௮ ௰ ௨}. This would be read as they are written as 
Three Thousands, Seven Hundreds, Eight Tens, Two; or in Tamil as 
\tamiltext{௩௲௭௱௮௰௨ž}.\footnote{https://cloud.github.com/downloads/raaman/Tamil-Numeral/tamilnumbers.html}

\subsection{Dates}

Once the script is loaded the day, month and year can be loaded using the command  \cmd{\tamildate}, which returns the |\today| formatted as per custom Tamil. 

\begin{center}
\bgroup
\tamil
\begin{tabular}{lll}
day	 &month	&year	\\

௳	&௴	      &௵	\\

u	&mee	      &wa	\\
\end{tabular}
\egroup
\end{center}


\section{Grantha}
\label{s:grantha}

Grantha is a Unicode block containing the ancient Grantha script characters of 6th to 19th century Tamil Nadu and Kerala for writing Sanskrit and Manipravalam. Battled to get it working, as I could not find an appropriate unicode font. The font would need remapping. Unfortunately this is a script with no Noto support.

\begin{figure}[htbp]
\bgroup
\parindent=0pt
\centering
\includegraphics[width=\linewidth]{./images/grantha.jpg}

\caption{An image of a palm leaf manuscript with Sanskrit written in Grantha script (wikipedia)}

\egroup
\end{figure}

\newfontfamily\grantha{e-Grantamil 7}%e-Grantamil 7

\begin{scriptexample}[\grantha]{Tamil}
\unicodetable{grantha}{"0D0,"0D1,"0D2,"1133,"1134,"1135,"1136,"1137}

\hfill  Typeset with \cmd{\grantha} and \texttt{e-Granthamil 7.ttf}
\end{scriptexample}

{
\grantha \char"11311

}

%\newfontfamily\freeserif{FreeSerif}
%
%
%\freeserif \lorem
%\begin{tabular}{lll}
%day	 &month	&year	\\
%
%௳	&௴	      &௵	\\
%
%u	&mee	      &wa	\\
%\end{tabular}


