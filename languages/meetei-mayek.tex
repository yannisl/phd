\section{Meitei Mayek alphabet}
\label{s:meiteimayek}
\index{scripts>Meitei Mayek}
\index{Meetei Mayek}
\newfontfamily\meitei{Noto Sans Meetei Mayek}

\def\textmeitei#1{{\meitei #1}\xspace}

Meithei (Meitei) /ˈmeɪteɪ/,[4] also known as Manipuri /mænɨˈpʊəri/ ({\pan মৈতৈলোন্} \textmeitei{ꯃꯧꯇꯧꯂꯣꯟ} Meitei-lon or {\pan মৈতৈলোল্} \textmeitei{ꯃꯧꯇꯧꯂꯣꯜ} Meitei-lol), is the predominant language and lingua franca in the southeastern Himalayan state of Manipur, in northeastern India. It is the official language in government offices. Meithei is also spoken in the Indian states of Assam and Tripura, and in Bangladesh and Burma (now Myanmar).

The Meitei (also Meetei, Meithei, Manipuri) people are the majority ethnic group of Manipur, a northeastern state of India. Meitei is an endonym or autonym while Manipuri is an exonym. A significant population of the Meitei also are settled in domestic neighboring states such as Assam[1] and Tripura. They have also settled in Bangladesh[2] and Myanmar.[3]
The Meitei people are made up of seven major clans known as Salai Taret.[4] Their written history has been documented to 1445 BC.[5]

Meithei is a Tibeto-Burman language whose exact classification remains unclear, though it shows lexical resemblances to Kuki and Tangkhul Naga.[5] The language is spoken by more than 1.5 million people.

\begin{figure}[htbp]
\centering
\includegraphics[width=\linewidth]{dancing}

\caption{"Khamba-Thoibi" Jagoi 
RKCS paintings on the walls of temple of Ibudhou Thangjing at Moirang, Manipur. 
Picture Courtesy - Recky Maibram.}
\end{figure}

Meithei has proven to be an integrating factor among all ethnic groups in Manipur who use it to communicate among themselves. It has been recognized (as Manipuri), by the Indian Union and has been included in the list of scheduled languages (included in the 8th schedule by the 71st amendment of the constitution in 1992). Meithei is taught as a subject up to the post-graduate level (Ph.D.) in universities of India, apart from being a medium of instruction up to the undergraduate level in Manipur.

\bgroup
\meitei
\begin{tabular}{>{\arial}l
                >{\arial}l
                >{\meitei}l
                >{\arial}l
                >{\arial}l
                >{\meitei}l
               }
1	&ama 	 &ꯑꯃ	       &11	&taramathoi	&\\
2	&ani	   &ꯑꯅꯤ	&12	 &taranithoi	&ky \\
3	&ahum	&ꯑꯍꯨꯝ	   &13	 &tarahumdoi	&ꯇꯔꯥꯍꯨꯝꯗꯣꯢ\\
4	&mari	&ꯃꯔꯤ	   &14  &	taramari	&ꯇꯔꯥꯃꯔꯤ\\
5	&manga	 &ꯃꯉꯥ	   &15	 &taramanga	&ꯇꯔꯥꯃꯉꯥ\\
6	&taruk	 &ꯇꯔꯨꯛ	   &16	 &tarataruk	&ꯇꯔꯥꯇꯔꯨꯛ\\
7	&taret	 &ꯇꯔꯦꯠ	   &17	 &tarataret	&ꯇꯔꯥꯇꯔꯦꯠ\\
8	&nipan &ꯅꯤꯄꯥꯟ	&18	 &taranipan	&ꯇꯔꯥꯅꯤꯄꯥꯟ\\
9	&mapan	 &ꯃꯥꯄꯟ	   &19	 &taramapan	&ꯇꯔꯥꯃꯥꯄꯟ\\
10	&tara	 &ꯇꯔꯥ	   &20	 &kun	&ꯀꯨꯟ\\
\end{tabular}
\egroup





Meitei Mayek script was added to the Unicode Standard in October, 2009 with the release of version 5.2.\index{Meitei Mayek}

The Unicode block for Meitei Mayek, called Meetei Mayek, is \unicodenumber{U+ABC0–U+ABFF}.

Characters for historical orthographies are part of the Meetei Mayek Extensions block at \unicodenumber{U+AAE0–U+AAFF}.

\begin{scriptexample}[]{Meitei}
\unicodetable{meitei}{"ABC0,"ABCD0,"ABE0,"ABF0}
\end{scriptexample}

\begin{scriptexample}[]{Meitei}
\unicodetable{meitei}{"AAE0,"AAF0}
\captionof{table}{Meetei Mayek Extensions}
\end{scriptexample}


\printunicodeblock{./languages/meetei-mayek.txt}{\meitei}


% http://e-pao.net/eyek/tamba/



