\section{Tibetan}
\label{tibetan}
\index{scripts>tibetan}


Another important Northern Indian member perhaps
derived directly from Gupta---and thus a sister script to Nagari,
Sarada and Pali---is Tibetan. However, the Tibetan
language wears this foreign Indo-Aryan script most uncomfortably.\cite{writing}
The script retains the Indic consonantal alphabet with diacritic
attachments to indicate vowels – but with only one vowel
letter, the /a/, which is the same as the system’s own ‘default’ /a/.
This /a/ letter is then used to attach other diacritics in order to
indicate further vowels. Because the Tibetan language has
changed greatly since c. AD 700 (when the script was first elaborated
from Gupta) while the script has remained almost
unchanged, Tibetan is extremely difficult to read today. Its
greatest problem is that it marks none of the tones of its tonal
language. Though Tibetans have long tried to adapt written
Tibetan to spoken Tibetan, high illiteracy has been the price of
failing to achieve this. Tibetan schools in Tibet, by governmental
decree, now teach only the Chinese script and in the Chinese
language.

\newfontfamily\tibetan{TibMachUni.ttf}

\newfontfamily\tibetan{Qomolangma-Chuyig.ttf}

%A should pick it up automatically \tibetan

Fonts described in this section can be obtained from The Tibetan \& Himalayan Library
\footnote{\url{http://www.thlib.org/tools/scripts/wiki/tibetan\%20machine\%20uni.html}  }

I have tried a few \texttt{Tibetan Machine Uni (TMU)} seems to be used by a number of scholars. 

A tip when you are trying to locate fonts is to find a related article in Wikipedia, such as Tibetan alphabet and inspect the element using your browser to see what fonts are being used.


|style="font-family:'Jomolhari','Tibetan Machine Uni','DDC Uchen', 'Kailash';| 


If you cannot see the script and rather than boxes or question marks then you can search and download one of the fonts in |font-family|.



\begin{docKey}[phd]{language}{ = tibetan}{default none, initial english} 
The key |language=tibetan| sets the default language as Tibetan, using the main font given by the key |tibetan font=TibMachUni.ttf|.

It will also create an environment tibetanlanguage.
\end{docKey}

\begin{docKey}[phd]{tibetan font}{= TibMachUni.ttf} {initial = TibMachUni.ttf} 
The key |tibetan font=font-name| sets the default font for the Tibetan language. It will also create the switch \cmd{\tibetan} for typesetting text in Tibetan.
\end{docKey}


\begin{docEnvironment}{tibetan}{}
\end{docEnvironment}

The environment is created automatically
\begin{texexample}{Tibetan language setttings}{ex:tibetan}
\bgroup
\cxset{language=tibetan, tibetan font = TibMachUni.ttf}

\tibetan Tibetan: དབུ་ཅན\par
ཨོཾ་ཨཿཧཱུྂ་བཛྲ་གུ་རུ་པདྨ་སིདྡྷི་ཧཱུྂ༔\par
\egroup

\begin{tibetanlanguage}
The tibetan environment\par
ཨོཾ་ཨཿཧཱུྂ་བཛྲ་གུ་རུ་པདྨ་སིདྡྷི་ཧཱུྂ༔
\end{tibetanlanguage}
\end{texexample}


The Tibetan alphabet is an \emph{abugida} of Indic origin used to write the Tibetan language as well as Dzongkha\footnote{Spoken in Bhutan.}, the Sikkimese language, Ladakhi, and sometimes Balti. 

The printed form of the alphabet is called \textit{uchen} script (Tibetan: དབུ་ཅན་, Wylie: dbu-can; "with a head") while the hand-written cursive form used in everyday writing is called umê script (Tibetan: དབུ་མེད་, Wylie: dbu-med; "headless").

The alphabet is very closely linked to a broad ethnic Tibetan identity. Besides Tibet, it has also been used for Tibetan languages in Bhutan, India, Nepal, and Pakistan.[1] The Tibetan alphabet is ancestral to the Limbu alphabet, the Lepcha alphabet,[2] and the multilingual 'Phags-pa script.[2]


The Tibetan alphabet is romanized in a variety of ways.[3] This article employs the Wylie transliteration system.

The Tibetan alphabet has thirty basic letters, sometimes known as "radicals", for consonants.[2]

{\tibetanfontfamily
ཀ ka /ká/	ཁ kha /kʰá/	ག ga /kà, kʰà/	ང nga /ŋà/\\
ཅ ca /tʃá/	ཆ cha /tʃʰá/	ཇ ja /tʃà/	ཉ nya /ɲà/\\
ཏ ta /tá/	ཐ tha /tʰá/	ད da /tà, tʰà/	ན na /nà/\\
པ pa /pá/	ཕ pha /pʰá/	བ ba /pà, pʰà/	མ ma /mà/\\
ཙ tsa /tsá/	ཚ tsha /tsʰá/	ཛ dza /tsà/	ཝ wa /wà/ (not originally part of the alphabet)[5]\\
ཞ zha /ʃà/[6]	ཟ za /sà/	འ 'a /hà/[7]\\
ཡ ya /jà/	ར ra /rà/	ལ la /là/\\
ཤ sha /ʃá/[6]	ས sa /sá/	ཧ ha /há/[8]\\
ཨ a /á/\\
}


Tibetan is not a difficult script to read or write, but it is a very complex script to deal with in terms of computer processing (as far as complexity goes I would rate it second only to the Mongolian script). The problem is that written Tibetan comprises complex syllable units (known in Tibetan as a tsheg bar {\tibetan ཚེག་བར}) which although written horizontally may include \emph{vertical} clusters of consonants and vowel signs agglutinating around a base consonant (a vertical cluster is known as a "stack"). 

Thus most words have a horizontal and a vertical dimension, with the result that text is not laid out in a straight line as in most scripts. For example, the word bsGrogs བསྒྲོགས་ (pronounced drok ... obviously!) may be analysed as follows :

\definecolor{lavenderblush}{HTML}{FFF0F5}%
\definecolor{beige}{HTML}{F5F5DC}%


{\tibetan 
\HUGE བསྒྲོགས

{\color{beige}%
\symbol{"0F56}\color{blue!40}\color{red}\symbol{"0F66}\symbol{"0F92}\color{blue!80}\symbol{"0FB2}\color{beige}\symbol{"0F7C}\color{blue!25}\symbol{"0F42}\symbol{"0F66}\symbol{"0F0B}}



\begin{tabular}{|l|}
\symbol{"0F56}\symbol{"0F7C}\\
\symbol{"0F42}\symbol{"0F7C}\\
\symbol{"0F66}\symbol{"0F7C}\\
\symbol{"0F40}\symbol{"0F7C}\\
\end{tabular}
}

\subsection{Unicode Block Tibetan}


\bgroup\large\tibetan
\begin{tabular}{llllllllllllllll l}
\toprule
	           &|0|	&|1|	&|2|	&|3|	&|4|	&|5|	&|6|	&|7|	&|8|	&|9|	&|A|	&|B|	&|C|	&|D|	&|E|	&|F|\\
\midrule
\texttt{U+0F0x}	&ༀ	&༁	&༂	&༃	&༄	&༅	&༆	&༇	&༈	&༉	&༊	&་	&༌  &	།	&༎	&༏\\
\midrule
\texttt{U+0F1x} &༐	&༑	&༒	&༓	&༔	&༕	&༖	&༗	&༘&	༙	&༚	&༛	&༜	&༝	&༞	&༟\\
\midrule
\texttt{U+0F2x} &༠	&༡	&༢	&༣	&༤	&༥	&༦	&༧	&༨	&༩	&༪	&༫	&༬	&༭	&༮	&༯\\
\midrule
\texttt{U+0F3x}	&༰ &༱	 &༲ &༳	&༴ &༵	&༶ & ༷	&༸&	༹	&༺&	༻	&༼&	༽	&༾	&༿\\
\midrule
\texttt{U+0F4x} &ཀ	&ཁ	&ག	&གྷ	&ང	&ཅ	&ཆ	&ཇ	&	&ཉ	&ཊ	&ཋ	&ཌ	&ཌྷ	&ཎ	&ཏ\\
\midrule
\texttt{U+0F5x}	 &ཐ	&ད	&དྷ	&ན	&པ	&ཕ	&བ	&བྷ	&མ	&ཙ	&ཚ	&ཛ	&ཛྷ	&ཝ	&ཞ	&ཟ\\
\midrule
\texttt{U+0F6x} &འ	&ཡ	&ར	&ལ	&ཤ	&ཥ	&ས	&ཧ	&ཨ	&ཀྵ	&ཪ	&ཫ	&ཬ	&&&\\
^^A\texttt{U+0F7x}&&	ཱ &	& &ི	ཱི&	ུ&	ཱུ&	ྲྀ&	ཷ&	ླྀ&	ཹ&	ེ&	ཻ&	ོ&	ཽ&	&ཾ	&ཿ\\
\midrule
\texttt{U+0F8x}&    ྀ   & 	ཱྀ&	ྂ&	&ྃ &	྄	&྅&	྆	&྇	ྈ&	ྉ&	ྊ&	ྋ&	ྌ&	ྍ&	ྎ&	ྏ\\
\midrule
\texttt{U+0F9x} &	ྐ&	ྑ   & 	ྒ &	ྒྷ &	ྔ &	ྕ &	ྖ &	ྗ &		ྙ &	ྚ &	ྛ &	ྜ &	ྜྷ &	ྞ &	ྟ\\
\texttt{U+0FAx} &	ྠ &	ྡ &	ྡྷ &	ྣ &	ྤ &	ྥ &		&ྦ	&ྦྷ	ྨ&	ྩ&	ྪ&	ྫ&	ྫྷ&	ྭ&	ྮ&	ྯ\\
\midrule
\texttt{U+0FBx} 
&	  ྰ 
&	
& ྱ  	 
&ྲ	
&ླ	
&ྴ
&	ྵ
&	ྶ
&	ྷ
&ྸ
&
&
&
&	
&྾	
&྿\\
\midrule
\texttt{U+0FCx}	 &࿀&	࿁&	࿂&	࿃&	࿄&	࿅&	&࿇	&࿈	&࿉	&࿊	&࿋	&࿌	&&	࿎	&࿏\\
\midrule
\texttt{U+0FDx}	&࿐	&࿑	&࿒	&࿓	&࿔	&࿕	&࿖	&࿗	&࿘	&࿙	&࿚	&&&&&\\
\midrule
\texttt{U+0FEx} &&&&&&&&&&&&&&&&\\
\midrule
\texttt{U+0FFx}  &&&&&&&&&&&&&&&&\\
\bottomrule
\end{tabular}
\egroup




\subsection{Fonts for Tibetan}

Fonts for Tibetan need to be downloaded one set of fonts are the \texttt{Qomolangma}. They come in different flavours, but they appear
to offer advantages as compared to the Tibetan Machine Uni.
\medskip


\newfontfamily\betsu{Qomolangma-Betsu.ttf}
\newfontfamily\drutsa{Qomolangma-Drutsa.ttf}
\newfontfamily\chuyig{Qomolangma-Chuyig.ttf}
\newfontfamily\tsumachu{Qomolangma-Tsumachu.ttf}
\newfontfamily\uchensutung{Qomolangma-UchenSutung.ttf}
\newfontfamily\uchensuring{Qomolangma-UchenSuring.ttf}
\newfontfamily\uchensarchen{Qomolangma-UchenSarchen.ttf}
\newfontfamily\uchensarchung{Qomolangma-UchenSarchung.ttf}
\newfontfamily\tsuring{Qomolangma-Tsuring.ttf}
\newfontfamily\TMU{TibMachUni.ttf}
\newfontfamily\himalaya{Microsoft Himalaya}


{
\centering

\renewcommand{\arraystretch}{1.5}

\begin{tabular}{lr}
\toprule
|Qomolangma-Betsu.ttf| & {\betsu  དབུ་མེད }\\
\midrule
|Qomolangma-Chuyig.ttf| &{\chuyig  དབུ་མེད}\\
\midrule
|Qomolangma-Drutsa.ttf| &{\drutsa  དབུ་མེད}\\
\midrule
|Qomolangma-Tsumachu.ttf|&{\tsumachu  དབུ་མེད}\\
\midrule
|Qomolangma-Tsuring.ttf| &{\tsuring  དབུ་མེད}\\
\midrule
|Qomolangma-UchenSarchen.ttf| &{\uchensarchen དབུ་མེད}\\
\midrule
|Qomolangma-UchenSarchung.ttf|&{\uchensarchung དབུ་མེད }\\
\midrule
|Qomolangma-UchenSuring.ttf|&{\uchensuring དབུ་མེད}\\
\midrule
|Qomolangma-UchenSutung.ttf|&{\uchensutung དབུ་མེད }\\
\midrule
|TibMachUni.ttf| &{\TMU དབུ་མེད }\\
\midrule
|Microsoft Himalaya| &{\himalaya དབུ་མེད ཽ}\\
\bottomrule
\end{tabular}

}
\bigskip

\bgroup
\LARGE\tsuring
\noindent༆ །ཨ་ཡིག་དཀར་མཛེས་ལས་འཁྲུངས་ཤེས་བློ  འི་\par
གཏེར༑ །ཕས་རྒོལ་ཝ་སྐྱེས་ཟིལ་གནོན་གདོང་ལྔ་བཞིན།།\par
ཆགས་ཐོགས་ཀུན་བྲལ་མཚུངས་མེད་འཇམ་དབྱངསམཐུས།།\par
མཧཱ་མཁས་པའི་གཙོ་བོ་ཉིད་འགྱུར་ཅིག། །མངྒལཾ༎\par
བསྒྲོགས
\egroup

\subsubsection{Tibetan numbers}
\cxset{language=tibetan, tibetan font = TibMachUni.ttf}

{
\obeylines
\small
TIBETAN DIGIT ZERO\tibetan	༠
TIBETAN DIGIT ONE	\tibetan༡	
TIBETAN DIGIT TWO\tibetan	༢	
TIBETAN DIGIT THREE\tibetan	༣	
TIBETAN DIGIT FOUR	\tibetan ༤	
TIBETAN DIGIT FIVE\tibetan	༥	
TIBETAN DIGIT SIX	\tibetan ༦	
TIBETAN DIGIT SEVEN\tibetan	༧	
TIBETAN DIGIT EIGHT\tibetan	༨	
TIBETAN DIGIT NINE\tibetan	༩	
TIBETAN DIGIT HALF ONE	\tibetan༪	
TIBETAN DIGIT HALF TWO	༫	
TIBETAN DIGIT HALF THREE	༬
TIBETAN DIGIT HALF FOUR ༭	
TIBETAN DIGIT HALF FIVE ༯	
TIBETAN DIGIT HALF SIX	 ༯	
TIBETAN DIGIT HALF SEVEN	༰	
TIBETAN DIGIT HALF EIGHT	༱	
TIBETAN DIGIT HALF NINE	༲	
TIBETAN DIGIT HALF ZERO	༳	
}


Tibetan numbers

The usage is not certain. By some interpretations, this has the value of 9.5. Used only in some traditional contexts, these appear as the last digit of a multidigit number, eg. ༤༬ represents 42.5. These are very rarely used, however, and other uses have been postulated.


\PrintUnicodeBlock{./languages/tibetan.txt}{\himalaya}

