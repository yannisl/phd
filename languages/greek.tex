\cxset{image=greek-men}

\parindent1em

\chapter{Greek}
\epigraph{The Pleiads have left the sky, and\\
the moon has vanished. It’s midnight:\\
the time for meeting is over.\\
And me—I am lying, lonely}{Sappho}
\label{s:greek}
\index{languages>Greek}\index{Herodotus}\index{alphabets>Greek}



\enquote{The Phoenicians who came with Kadmos,} wrote Herodotus in the fifth century BC of the legendary Phoenician prince of Tyre and brother of Europa, ``\ldots introduced into Greece, after their settlement in the country, a number of accomplishments of which the most important was writing, an art which probably was unknown to the Greeks until then''. 

A basis for the remarkable history of the Greek language is
the invention of the Greek alphabet. It was modelled after
Semitic scripts, with the important improvement that not only
consonants but also vowels are represented by independent
letters.\footnote{Cover image, from \href{http://www.pappaspost.com/todays-undesirable-muslims-were-yesteryears-greeks-pure-american-no-rats-no-greeks/}{papaspost.com}, showing Greek immigrants arriving at Ellis Island in 1911.}

The poet Sappho had access to an alphabetic script, invented
for the Greek language just a couple of hundred years before her
time. The Greek alphabet is very similar to the Latin one, which
is the one used for English. In fact, the Latin alphabet is derived
from a variant of the Greek one.The similarity is easy to observe.
Here is the original poem, written in the Greek alphabet:\footcite{janson:2002}


\begin{center}
\arial 
ΔΕΔΗΚΕ ΜΕΝ Α ΣΕΛΑΝΝΑ\\
ΚΑΙ ΠΛΕΙΑΔΕΣ. ΜΕΣΑΙ ΔΕ\\
ΝΥΚΤΕΣ. ΠΑΡΑ Δ'ΕΡΧΕΤ'ΩΡΑ.\\
ΕΓΩ ΔΕ ΜΟΝΑ ΚΑΤΕΥΔΩ\\
\end{center} 

Transcribed in the Latin alphabet:

\begin{center}
DEDUKE MEN A SELANNA\\
KAI PLEIADES. MESAI DE\\
NUKTES. PARA D’ ERKHET’ ORA.\\
EGO DE MONA KATEUDO.\\
\end{center}

The Greek alphabet is the script that has been used to write the Greek language since the 8th century BC. It was derived from the earlier Phoenician alphabet, and was in turn the ancestor of numerous other European and Middle Eastern scripts, including Cyrillic and Latin.[3] Apart from its use in writing the Greek language, both in its ancient and its modern forms, the Greek alphabet today also serves as a source of technical symbols and labels in many domains of mathematics, science and other fields.

In its classical and modern forms, the alphabet has 24 letters, ordered from alpha to omega. Like Latin and Cyrillic, Greek originally had only a single form of each letter; it developed the letter case distinction between upper-case and lower-case forms in parallel with Latin during the modern era.

\tex has built-in commands for the usage of the Greek alphabet see section \ref{greek} in the Symbols chapter.

\bgroup
\obeylines
\greek\obeyspaces

Α	ἄλφα	aleph	alpha	[alpʰa]	[ˈalfa]	Listeni/ˈælfə/
Β	βῆτα	beth	beta	[bɛːta]	[ˈvita]	/ˈbiːtə/, US /ˈbeɪtə/
Γ	γάμμα	gimel	gamma	[ɡamma]	[ˈɣama]	/ˈɡæmə/
Δ	δέλτα	daleth	delta	[delta]	[ˈðelta]	/ˈdɛltə/
Η	ἦτα	  heth	   eta	 [hɛːta], [ɛːta]	[ˈita]	/ˈiːtə/, US /ˈeɪtə/
Θ	θῆτα	teth	theta	[tʰɛːta]	[ˈθita]	/ˈθiːtə/, US Listeni/ˈθeɪtə/
Ι	ἰῶτα	yodh	iota	[iɔːta]	[ˈʝota]	Listeni/aɪˈoʊtə/
Κ	κάππα	kaph	kappa	[kappa]	[ˈkapa]	Listeni/ˈkæpə/
Λ	λάμβδα	lamedh	lambda	[lambda]	[ˈlamða]	Listeni/ˈlæmdə/
Μ	μῦ	mem	mu	[myː]	[mi]	Listeni/ˈmjuː/; occasionally US /ˈmuː/
Ν	νῦ	nun	nu	[nyː]	[ni]	/ˈnjuː/ (US /ˈnuː/)
Ρ	ῥῶ	reš	rho	[rɔː]	[ro]	Listeni/ˈroʊ/
Τ	ταῦ	taw	tau	[tau]	[taf]	/ˈtaʊ/ or /ˈtɔː/
\egroup

With a suitable font such as |Arial Unicode MS| you do not need to do anything special to typeset short paragraphs of Greek text. Just use any editor set to encode the text in \utfviii. The example below was just cut and pasted. If you are going to write extensively in Greek it would be preferable to get a virtual keyboard. If you are using windows these are pre-build. 

\topline
\begin{quote}
Ἡροδότου Ἁλικαρνησσέος ἱστορίης ἀπόδεξις ἥδε, ὡς μήτε τὰ γενόμενα ἐξ ἀνθρώπων τῷ χρόνῳ ἐξίτηλα γένηται, μήτε ἔργα μεγάλα τε καὶ θωμαστά, τὰ μὲν Ἕλλησι, τὰ δὲ βαρβάροισι ἀποδεχθέντα, ἀκλεᾶ γένηται, τὰ τε ἄλλα καὶ δι' ἣν αἰτίην ἐπολέμησαν ἀλλήλοισι.[2]

Herodotus of Halicarnassus, his Researches are set down to preserve the memory of the past by putting on record the astonishing achievements of both the Greeks and the Barbarians; and more particularly, to show how they came into conflict.[3]
\end{quote}
\bottomline

\begin{scriptexample}{greek}
\unicodetable{greek}{% 
"0370,"0380,"0390,"03A0,"03B0,"03C0,"03D0,"03E0,"03F0}
\end{scriptexample}

\subsection{Greek diacritics}
\index{Greek>polytonic}

The ancient Greek writing included for many accents and diacritics. The extended unicode standard provides slots for all diacritics. Greek orthography has used a variety of diacritics starting in the Hellenistic period. The complex polytonic orthography notates Ancient Greek phonology. The simple monotonic orthography, introduced in 1982, corresponds to Modern Greek phonology, and requires only two diacritics.

Polytonic orthography (πολύς "much", "many", τόνος "accent") is the standard system for Ancient Greek. The acute accent ( ´ ), the grave accent ( ` ), and the circumflex ( ῀ ) indicate different kinds of pitch accent. The rough breathing ( ῾ ) indicates the presence of an /h/ sound before a letter, while the smooth breathing ( ᾿ ) indicates the absence of /h/.

Since in Modern Greek the pitch accent was replaced by a dynamic accent, and the /h/ was lost, most polytonic diacritics have no phonetic significance, and merely reveal the underlying Ancient Greek etymology.

Monotonic orthography (μόνος "single", τόνος "accent") is the standard system for Modern Greek. It retains a single accent or tonos (΄) to indicate stress and the diaeresis (¨) to indicate a diphthong: compare modern Greek παϊδάκια /pajˈðaca/ "lamb chops", with a diphthong, and παιδάκια /peˈðacia/ "little children" with a simple vowel. Tonos and diaeresis can be combined on a single vowel, as in the verb ταΐζω (/taˈizo/ "to feed").

\medskip
\begin{scriptexample}[]{Greek}
\unicodetable{greek}{%
"1F00,"1F10,"1F20}
\end{scriptexample}

%%%%%%%%%%%%%%%%%%%%%%%%%%%%%%%%%%%%
%    Greek Language
%%%%%%%%%%%%%%%%%%%%%%%%%%%%%%%%%%%%

%\documentclass{book}
%\usepackage{phd}
%\usepackage{philokalia}
%\begin{document}

\subsection{Philokalia}

The \pkgname{philokalia} package by Apostolos Syropoulos provides a Greek font in the style of the Philokalia manuscripts. The package modifies the lettrine package, which we cater for in the \pkgname{phd} and hence we adjusted it slightly for this. Also the package needed some modifications to work with LuaTeX.

The Philokalia (Ancient Greek: φιλοκαλία "love of the beautiful, the good", from φιλία philia "love" and κάλλος kallos "beauty") is "a collection of texts written between the 4th and 15th centuries by spiritual masters"[1] of the Eastern Orthodox hesychast tradition. They were originally written for the guidance and instruction of monks in "the practise of the contemplative life".[2] The collection was compiled in the eighteenth-century by St. Nikodemos of the Holy Mountain and St. Makarios of Corinth.

Although these works were individually known in the monastic culture of Greek Orthodox Christianity before their inclusion in The Philokalia, their presence in this collection resulted in a much wider readership due to its translation into several languages. The earliest translations included a Church Slavonic translation of selected texts by Paisius Velichkovsky (Dobrotolublye) in 1793, a Russian translation by Ignatius Bryanchaninov in 1857, and a five-volume translation into Russian (Dobrotolyubie) by St. Theophan the Recluse in 1877.

There were subsequent Romanian, Italian and French translations.[3][4]
The book is a "principal spiritual text" for all the Eastern Orthodox Churches;[5] the publishers of the current English translation state that "The Philokalia has exercised an influence far greater than that of any book other than the Bible in the recent history of the Orthodox Church."[6]
Philokalia (sometimes Philocalia) is also the name given to an anthology of the writings of Origen compiled by Saint Basil the Great and Saint Gregory Nazianzus. Other works on monastic spirituality have also used the same title over the years.[5][7]

The Philokalia fonts consist of three fonts: one that contains
the normal typeface, one that contains the ligatures and one that contains the special ornament characters that decorate the beginning of each chapter. The glyphs were generated from scanned images of the book pages and Apostolos Syropoulos described the process in detail in \cite{syropoulos}. 


{
%\newfontfamily\plk{Philokalia-Regular}
\plk
%\newfontfamily\PHtitl[Script=Greek,RawFeature=+titl;grek]{Philokalia-Regular}
 %\font\PHtitl="[Philokalia-Regular]/ICU:script=grek,+titl"

 
 \lettrine[lines=3]{\usebox{\philobox}}{ερὶ} ποιητικῆς αὐτῆς τε καὶ τῶν εἰδῶν αὐτῆς, ἥν τινα δύναμιν ἕκαστον ἔχει, 
καὶ πῶς δεῖ συνίστασθαι τοὺς μύθους  εἰ μέλλει καλῶς ἕξειν ἡ ποίησις, ἔτι δὲ ἐκ πόσων καὶ ποίων 
ἐστὶ μορίων, ὁμοίως δὲ καὶ περὶ τῶν ἄλλων ὅσα τῆς αὐτῆς ἐστι μεθόδου, λέγωμεν ἀρξάμενοι κατὰ φύσιν 
πρῶτον ἀπὸ τῶν πρώτων.
 
Ἐποποιία δὴ καὶ ἡ τῆς τραγῳδίας ποίησις ἔτι δὲ κωμῳδία καὶ ἡ διθυραμβοποιητικὴ καὶ τῆς αὐλητικῆς 
ἡ πλείστη καὶ κιθαριστικῆς πᾶσαι τυγχάνουσιν οὖσαι μιμήσεις τὸ σύνολον· διαφέρουσι δὲ ἀλλήλων τρισίν, 
ἢ γὰρ τῷ ἐν ἑτέροις μιμεῖσθαι ἢ τῷ ἕτερα ἢ τῷ ἑτέρως καὶ μὴ τὸν αὐτὸν τρόπον. 

Ὥσπερ γὰρ καὶ χρώμασι καὶ σχήμασι πολλὰ μιμοῦνταί τινες ἀπεικάζοντες (οἱ μὲν [20] διὰ τέχνης οἱ δὲ διὰ συνηθείας),
ἕτεροι δὲ διὰ τῆς φωνῆς, οὕτω κἀν ταῖς εἰρημέναις τέχναις ἅπασαι μὲν ποιοῦνται τὴν μίμησιν ἐν ῥυθμῷ καὶ λόγῳ καὶ
ἁρμονίᾳ, τούτοις δ᾽ ἢ χωρὶς ἢ μεμιγμένοις· οἷον ἁρμονίᾳ μὲν καὶ ῥυθμῷ χρώμεναι μόνον ἥ τε αὐλητικὴ καὶ ἡ κιθαριστικὴ
κἂν εἴ τινες [25] ἕτεραι τυγχάνωσιν οὖσαι τοιαῦται τὴν δύναμιν, οἷον ἡ τῶν συρίγγων, αὐτῷ δὲ τῷ ῥυθμῷ [μιμοῦνται]
χωρὶς ἁρμονίας ἡ τῶν ὀρχηστῶν (καὶ γὰρ οὗτοι διὰ τῶν σχηματιζομένων ῥυθμῶν μιμοῦνται καὶ ἤθη καὶ πάθη καὶ πράξεις)· 
 }

The package also modifies the \pkgname{lettrine} package and hence we have modified the \cmd{\lettrine} command to be called \cmd{\lettrinephilokalia} when used with the |philokalia| package. It is a bit long as a command, but easier to remember. 



\section{Greek-derived scripts}

Because of Greece’s military (Alexander the Great), economic
and cultural influence, the Greek alphabet became the
prototype for the ‘complete’ (that is, fully vowelized) alphabets
that emerged in Europe in the following centuries. These eventually
diffused, almost exclusively through Greek’s granddaughter
alphabets Latin and Cyrillic, throughout the entire
world --- a process still going on over two thousand years later
(illus. \ref{fig:greekderived}).

In first-millenium BC Asia Minor (today’s Turkey), the
Greek alphabet inspired an impressive number of non-Greek
peoples to elaborate their own Anatolian alphabets: \nameref{carian},
\nameref{sec:lydian}, \nameref{sec:lycian}, Pamphylian, Phrygian, Pisidian (of the Roman
period) and Sidetic.  Nonetheless, these scripts failed to
acquire lasting significance because of the region’s declining
economic fortunes followed by several major invasions.

%\documentclass{article}
%\usepackage[margin=1cm]{geometry}
%\usepackage{pdflscape}
%\usepackage{forest}
%\usepackage{hyperref}
%\usetikzlibrary{shadows,arrows}
\newgeometry{left=1cm,right=1cm,bottom=1cm}
\newpage

\tikzset{parent/.style={align=center,text width=2cm, fill=blue!40,rounded corners=2pt,inner sep=2pt},
    child/.style={align=center,text width=2.0cm,fill=orange!60,rounded corners=2pt,inner sep=1pt,outer sep=0pt},
    grandchild/.style={fill=white,text width=1.7cm}
}

%\begin{document}

\begin{landscape}
\begin{forest}
for tree={%
    thick,
    drop shadow,
    l sep=1.0cm,
    s sep=0.6cm,
    node options={draw,font={\rmfamily\small}},
    edge={semithick,-latex},
    where level=0{parent}{},
    where level=1{
        minimum height=0.8cm,
        child,
        parent anchor=south west,
        tier=p,
        l sep=0.25cm,
        for descendants={%
            grandchild,
            minimum height=0.6cm,
            %l sep=0.5cm,
%            s sep=0.5cm,
            anchor=115,
            edge path={
                \noexpand\path[\forestoption{edge}]
                (!to tier=p.parent anchor) |-(.child anchor)\forestoption{edge label};
            },
        }
    }{},
}
[(Phoenician)\\ GREEK
    [Palaeo-Hispanic %heading
        [North-east\\
         Celtiberian
            [South-West\\
                South-east
            ]
        ]
    ],
    [Etruscan
        [LATIN
            [\textit{Rhaetian} 
                [\href{http://en.wikipedia.org/wiki/Gallic}{Gallic}
                    [\href{http://en.wikipedia.org/wiki/Venetic}{Venetic}
                      [Faliscan
                        [Northern Picene
                          Southern Picene\\
                            [Oscan
                              [Umbrian]
                            ]
                          ]
                       ]
                    ]
                ]
            ]
        ]
    ]
    [\href{http://en.wikipedia.org/wiki/Gothic_language}{Gothic}]
    [Glagolithic
       [Croatian]
     ]   
    [Cyrillic
        [Russian
         [Ukrainian
            [Bulgarian
             [Serb]
            ] 
        ]  
     ]  
   ]  
  [Anatolian
    [Carian
      [Lydian
        [Lykian
          [Pamphylian
            [Phrygian
              [Pisidian
                [Sidetic]
            ]
          ]
        ]
      ]
  ]
  ]
  ]
  [Armenian]
  [Georgian]
  [Coptic
    [Nubian]
  ]
]
\end{forest}
\captionof{figure}{Abridged family tree of some Greek-derived scripts.}
\label{fig;greekderived}
\end{landscape}


\restoregeometry
\newpage


%\end{document}

The Armenian monk St Mesrob (c. 345–440) is said to have
elaborated the Armenians’ first script c. AD 405 – Armenian is a
separate branch of the Indo-European superfamily of languages
(to which Greek and Germanic, which includes English, also
belong). Based on the Greek alphabet, the Armenian script
originally consisted of around 36 mainly capital letters. By the
1200s, Armenian notrgir, or cursive writing, had been developed,
then replacing writing in capitals (illus. 100).

St Mesrob is also credited with devising the Georgian alphabet
in the early 400s AD – Georgian is a Caucasian, not an Indo-
European, language – as well as the Albanian alphabet. (Such
multiple attributions suggest that Mesrob’s role was apocryphal.)

The ecclesiastical Georgian script used 38 letters; over
time, several styles of writing Georgian developed, with varying
numbers of letters (illus. 101). The mkhedruli, or ‘lay hand’,
which began as a medium for non-sacral texts, is Georgian’s
most frequently employed script, still in use today.

In Egypt, the Greek alphabet inspired the Coptic
alphabet that replaced one of the world’s oldest writing traditions.
In the Balkans, Greek generated the Glagolitic and
Cyrillic scripts, which eventually generated the Russian script,







