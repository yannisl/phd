\section{Syloti Nagri}
\label{s:sylotinagri}
\newfontfamily\syloti{NotoSansSylotiNagri-Regular.ttf}
\newfontfamily\damase{damase_v.2.ttf}
\index{languages>Sylheti Nagari}

Sylheti or Syloti (i.e. "Silēṭī" Bengali: সিলেটী or "Silôṭī" Bengali: ছিলটী) is one of the Bengali dialects, primarily spoken in the Sylhet Division of northeast Bangladeshi district Moulvibazar,Sylhet,Sunamganj,Hobiganj and the Barak Valley region of southern Assam. (Although sometimes it is considered an independent language for not sharing grammatical mutual intelligibility), it is a similar language to Standard Bengali, with which it shares a high proportion of vocabulary: Spratt and Spratt (1987) report 70\% shared vocabulary, while Chalmers (1996) reports at least 80\% overlap.

Sylheti Nagari or Syloti Nagri (Silôṭi Nagôri) is the original script used for writing the Sylheti language. It is an almost extinct script, this is because the Sylheti Language itself was reduced to only dialect status after Bangladesh gained independence and because it did not make sense for a dialect to have its own script, its use was heavily discouraged. The government of the newly formed Bangladesh did so to promote a greater "Bengali" identity. This led to the informal adoption of the Eastern Nagari script also used for Bengali and Assamese. It is also known as Jalalabadi Nagri, Mosolmani Nagri, Ful Nagri etc.

Sylheti Nagari was added to the Unicode Standard in March, 2005 with the release of version 4.1.
The Unicode block for Sylheti Nagari is U+A800–U+A82F:

\begin{scriptexample}[]{Sylheti}
\unicodetable{damase}{"A800,"A810,"A820}
\end{scriptexample}


\printunicodeblock{./languages/syloti.txt}{\damase}
