\section{Vai}
\label{s:vai}

The Vai syllabary is a syllabic writing system devised for the Vai language by Momolu Duwalu Bukele of Jondu, in what is now Grand Cape Mount County, Liberia.[1] [2] Bukele is regarded within the Vai community, as well as by most scholars, as the syllabary's inventor and chief promoter when it was first documented in the 1830s. It is one of the two most successful indigenous scripts in West Africa.

\newfontfamily\vai{code2000.ttf}
\begin{scriptexample}[]{Vai}
\unicodetable{vai}{"A500,"A510,"A520,"A530,"A540,"A550,"A560,"A570,^^A
"A580,"A590,"A5A0,"A5B0,^^A
"A5C0,"A5D0,"A5E0,"A5F0,"A610,"A620,"A630}
\end{scriptexample}

In the 1920s ten decimal digits were devised for Vai; these were “Vai-style” glyph variants of
European digits (see Figure 11). They were not popular with Vai people  even for historical purposes. All
the modern literature uses European digits.


\begin{scriptexample}[]{Vai}
\bgroup
\vai
\obeylines\Large
0	1	2	3	4	5	6	7	8	9
꘠	꘡	꘢	꘣	꘤	꘥	꘦	꘧	꘨	꘩
\vai
\egroup
\end{scriptexample}



\printunicodeblock{./languages/vai.txt}{\vai}