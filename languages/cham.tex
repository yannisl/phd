\section{Cham}
\label{s:cham}

The Cham alphabet is an abugida used to write Cham, an Austronesian language spoken by some 230,000 Cham people in Vietnam and Cambodia. It is written horizontally left to right, as is English.

\newfontfamily\cham{Noto Sans Cham}

Cham is a Unicode block containing characters for writing the Cham language, primarily used for the Eastern dialect in Cambodia.
Cham script was added to the Unicode Standard in April, 2008 with the release of version 5.1.
The Unicode block for Cham is \textsc{U+AA00–U+AA5F}:

\begin{scriptexample}[]{Cham}
\unicodetable{cham}{"AA10,"AA20,"AA30,"AA40,"AA50}
\end{scriptexample}


\printunicodeblock{./languages/cham.txt}{\cham}