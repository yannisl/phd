\section{Philippine Scripts}
\label{s:tagalog}

\newfontfamily\tagalog{Noto Sans Tagalog}

Tagalog (/təˈɡɑːlɒɡ/;[6] Tagalog pronunciation: [tɐˈɡaːloɡ]) is an Austronesian language spoken as a first language by a quarter of the population of the Philippines and as a second language by the majority. Its standardized form, officially named Filipino, is the national language of the Philippines, and is one of two official languages along with English.
It is related to other Philippine languages, such as the Bikol languages, Ilocano, the Visayan languages, Kapampangan and Pangasinan, and more distantly to other Austronesian languages, such as the Formosan languages of Taiwan, Malay (Malaysian \& Indonesian), Hawaiian, Māori and Malagasy.

\begin{figure}[htbp]
\centering
\includegraphics[width=\textwidth]{tagalog-traditional}
\caption{Women In Traditional Dress early 1900s }
\end{figure}

%https://sepiaera.wordpress.com/2013/06/05/women-in-traditional-dress-early-1900s/

\paragraph{Baybayin} Tagalog was written in an abugida—or alphasyllabary—called Baybayin prior to the Spanish colonial period in the Philippines, in the 16th century. This particular writing system was composed of symbols representing three vowels and 14 consonants. Belonging to the Brahmic family of scripts, it shares similarities with the Old Kawi script of Java and is believed to be descended from the script used by the Bugis in Sulawesi.

Although it enjoyed a relatively high level of literacy, Baybayin gradually fell into disuse in favor of the Latin alphabet taught by the Spaniards during their rule.

There has been confusion of how to use Baybayin, which is actually an abugida, or an alphasyllabary, rather than an alphabet. Not every letter in the Latin alphabet is represented with one of those in the Baybayin alphasyllabary. Rather than letters being put together to make sounds as in Western languages, Baybayin uses symbols to represent syllables.

A "kudlit" resembling an apostrophe is used above or below a symbol to change the vowel sound after its consonant. If the kudlit is used above, the vowel is an "E" or "I" sound. If the kudlit is used below, the vowel is an "O" or "U" sound. A special kudlit was later added by Spanish missionaries in which a cross placed below the symbol to get rid of the vowel sound all together, leaving a consonant. Previously, the consonant without a following vowel was simply left out (for example, bundok being rendered as budo), forcing the reader to use context when reading such words.\index{kudlit}

\bigskip

\centerline{
\scalebox{2.5}{b {\tagalog ᜊ᜔}}, 
\scalebox{2.5}{ba/bi {\tagalog ᜊ}},
\scalebox{2.5}{be/bu {\tagalog ᜊᜒ}},
\scalebox{2.5}{bo {\tagalog ᜊᜓ}}
}

{\tagalog \char"170A\char"1714}

\paragraph{Unicode Encoding} Tagalog is a Unicode block containing characters of the pre-Spanish Philippine Baybayin script used for writing the Tagalog language.
\medskip

\unicodetable{tagalog}{"1700,"1710}
\medskip

A sample text from the Bible, The Lord's Prayer (Ama Namin), typeset in Baybayin is shown below.

\bgroup\obeylines
{\tagalog

ᜐᜓᜋᜐᜎᜅᜒᜆ᜔ ᜃ,
ᜐᜋ᜔ᜊᜑᜒᜈ᜔ ᜀᜅ᜔ ᜅᜎᜈ᜔ ᜋᜓ;
ᜋᜉᜐᜀᜋᜒᜈ᜔ ᜀᜅ᜔ ᜃᜑᜍᜒᜀᜈ᜔ ᜋᜓ;
ᜐᜓᜈ᜔ᜇᜒᜈ᜔ ᜀᜅ᜔ ᜎᜓᜂᜊ᜔ ᜋᜓ
ᜇᜒᜆᜓ ᜐ ᜎᜓᜉ, ᜉᜍ ᜈᜅ᜔ ᜐ ᜎᜅᜒᜆ᜔.
ᜊᜒᜄ᜔ᜌᜈ᜔ ᜋᜓ ᜃᜋᜒ ᜅᜌᜓᜈ᜔ ᜅ᜔ ᜀᜋᜒᜅ᜔ ᜃᜃᜈᜒᜈ᜔ ᜐ ᜀᜍᜏ᜔-ᜀᜍᜏ᜔;
ᜀᜆ᜔ ᜉᜆᜏᜍᜒᜈ᜔ ᜋᜓ ᜃᜋᜒ ᜐ ᜀᜋᜒᜅ᜔ ᜋᜅ ᜐᜎ
ᜉᜍ ᜈᜅ᜔ ᜉᜄ᜔ᜉᜉᜆᜏᜇ᜔ ᜈᜋᜒᜈ᜔ ᜐ ᜋᜅ ᜈᜄ᜔ᜃᜃᜐᜎ ᜐ ᜀᜋᜒᜈ᜔;
ᜀᜆ᜔ ᜑᜓᜏᜄ᜔ ᜋᜓ ᜃᜋᜒ ᜁᜉᜑᜒᜈ᜔ᜆᜓᜎᜓᜆ᜔ ᜐ ᜆᜓᜃ᜔ᜐᜓ,
ᜀᜆ᜔ ᜁᜀᜇ᜔ᜌ ᜋᜓ ᜃᜋᜒ ᜐ ᜎᜑᜆ᜔ ᜅ᜔ ᜋᜐᜋ.
ᜐᜉᜄ᜔ᜃᜆ᜔ ᜁᜌᜓ ᜀᜅ᜔ ᜃᜑᜍᜒᜀᜈ᜔, ᜀᜅ᜔ ᜃᜉᜅ᜔ᜌᜍᜒᜑᜈ᜔, 
ᜀᜆ᜔ ᜀᜅ᜔ ᜃᜇᜃᜒᜎᜀᜈ᜔, ᜋᜄ᜔ᜉᜃᜌ᜔ᜎᜈ᜔ᜋᜈ᜔.
ᜀᜋᜒᜈ᜔/ᜐᜒᜌ ᜈᜏ.}

Typeset with \docAuxCommand{tagalog} using Notto Sans Tagalog.

\egroup


% history of philippines https://books.google.com.qa/books?id=4wk8yqCEmJUC&pg=PA22&source=gbs_selected_pages&cad=2#v=onepage&q=tagalog&f=false



