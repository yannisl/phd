% \iffalse meta-comment
%<*internal>
\iffalse
%</internal>
%<*readme>
----------------------------------------------------------------
phd-pkgmanager --- a package to shorten preambles
E-mail: yannislaz@gmail.com
Released under the LaTeX Project Public License v1.3c or later
See http://www.latex-project.org/lppl.txt
----------------------------------------------------------------
This file provides provides the file phd-colorpalette.sty
%</readme>
%<*readmemd>
###The `phd-colorpalette` LaTeX2e package

The `phd-colorpalette` latex package is part of the `phd` bundle
which provides convenient methods to create new styles for books, reports
and articles. It also loads the most commonly used packages 
and resolves conflicts.

This work consists of the file  `phd-colorpalette.dtx`,
and the derived files   `phd-colorpalette.ins`,  `phd-colorpalette.pdf`, 
and `phd-colorpalette.sty`.

###Installation

run
          phd-lua.bat on windows
          pdflatex phd.dtx
          makeindex -s gind.ist -g phd 

If you have any difficulties with the package come and join us at
http://tex.stackexchange.com and post a new question or
add a comment at http://tex.stackexchange.com/a/45023/963.
or send me a message at  yannislaz at gmail.com

### Documentation

The package was written using the `doc` and `docscript` packages,
so that it is self documented in a literary programming style. 
The .pdf is a fat document, providing over fifty book styles (the
equivalent of classes) plus there is a lot of write-up on the inner
workings of TeX and LaTeX2e. However, you don't need to know much
to use it.

      \usepackage{phd}
      %%%%%%%%%%%%%%%%%%%%%%%%%%%%%%%%%%%%%%%%%%%
%%%%%%  STYLE 13
%%%%%%%%%%%%%%%%%%%%%%%%%%%%%%%%%%%%%%%%%%%

\cxset{style13/.style={
 name={Chapter},
 numbering=arabic,
 number font-size=\HUGE,
 number font-family=\sffamily,
 number font-weight=\bfseries,
 number color=\color{gray!50},
 number before=\par\vspace*{5pt}\hfill\hfill,
 number dot=,
 number after={\hspace*{7pt}\par},
 number position=rightname,
 chapter font-family=\sffamily,
 chapter font-weight=\normalfont,
 chapter font-size=\LARGE,
 chapter before={\thickrule\vspace*{20pt}\par\hfill\hfill},
 chapter after={\vskip0pt\par},
 chapter color={black!50},
 title beforeskip={\vspace*{10pt}},
 title afterskip={\vspace*{50pt}\par},
 title before={\hfill\hfill\raggedleft},
 title after={},
 title font-family=\sffamily,
 title font-color=\color{thered},
 title font-weight=\bfseries,
 title font-size=\huge,
 section indent=-1em,
 section align=\raggedright,
 section numbering=arabic,
 section indent=0pt,
 section beforeskip=0pt,
 section afterskip=\baselineskip,
 subsection align=\raggedright,
 subsection font-family=\sffamily,
 subsection font-weight=\bfseries,
 subsection font-size=\large,
 subsection font-shape=\itshape,
 subparagraph number after=\space,
}
}

\def\setstyle#1{\cxset{style#1}%
 \renewsection\renewsubsection\renewsubsubsection%
 \renewparagraph\renewsubparagraph}

\setstyle{13}


\chapter{Introduction to Chapter\\ Style Thirteen}

\section{A Brief History of Biomedical\\ Fluid Mechanics}
\lorem
\medskip
\begin{figure}[ht]
\centering
\includegraphics[width=0.45\textwidth]{./chapters/chapter14}
\includegraphics[width=0.45\textwidth]{./chapters/chapter14a}
\end{figure}
\lorem


All choices, are made via an extended key-value interface. 
Although not a compliment, it resembles CSS and the keys are a bit verbose but
attributes are easy to change and have a consistent and easy to remember interface.

To set or add a key we only use one command:

      \cxset{chapter name font-size = Huge,
             chapter number font-size = HUGE} 

### Future Development

This is still an experimental version, but I will retain the
interface in future releases. There is a large amount of
work still to be carried out to improve the template styles
provided, to test it more thoroughly and to add a number of
improvements in the special designs. At present I estimate
that I have completed about 70% of the work that needs
to be done.

__The package as it stands is not production stable.__ 


%</readmemd>
%
%<*TODO>
1. On final round add pkg options. This was left as last in order not to solve problems by adding
    options. Too many options are not a good User Interface.
2.  Finish symbol management, both text and math. Math already 60% incorporated.
3.  Better integration of indexing commands.   
4.  Revisit layout manager for Chapters. Broke again in tests.
5.  Docs. Add all references.
6.  Incorporate phd class for more flexibility.
7. Improve package manager.
8. Group script loading for better font management.
9. General font management to relook it again.
10. Add all style sections (about 100 already prepared). Once they
     are all working issue beta version.
%</TODO>
%<*internal>
\fi
\def\nameofplainTeX{plain}
\ifx\fmtname\nameofplainTeX\else
  \expandafter\begingroup
\fi
%</internal>
%<*install>
\input docstrip.tex
\keepsilent
\askforoverwritefalse
\preamble
----------------------------------------------------------------
phd --- A package to beautify documents.
E-mail: yannislaz@gmail.com
Released under the LaTeX Project Public License v1.3c or later
See http://www.latex-project.org/lppl.txt
----------------------------------------------------------------
\endpreamble

%\BaseDirectory{C:/users/admin/my documents/github/phd}
%\usedir{MWE}
\generate{\file{\jobname.sty}{
  \from{\jobname.dtx}{PLT}
  }
  }

%\nopreamble\nopostamble

%</install>

%<install>\endbatchfile
%<*internal>
%\usedir{tex/latex/phd}
\generate{
  \file{\jobname.ins}{\from{\jobname.dtx}{install}}
}
\nopreamble\nopostamble

\generate{
	\file{README.txt}{\from{\jobname.dtx}{readme}}
  }

\generate{
  \file{\jobname.md}{\from{\jobname.dtx}{readmemd}}
}
\generate{
  \file{TODO.tex}{\from{\jobname.dtx}{TODO}}
}

\generate{\file{tikz-page.sty}{\from{\jobname.dtx}{PAGE}}}

%\generate{\file{shellesc.sty}{\from{shellesc.dtx}{package}}}

\ifx\fmtname\nameofplainTeX
  \expandafter\endbatchfile
\else
  \expandafter\endgroup
\fi
%</internal>
%<*driver>

%\listfiles
%gdef\@onlypreamble{} % TO BE REMOVED NEEDED FOR TUTS
\NeedsTeXFormat{LaTeX2e}[2017/04/15]%
\RequirePackage[2017/04/15]{latexrelease}
\documentclass[oneside,10pt,a4paper]{ltxdoc}
\usepackage[bottom=2cm]{geometry}
\savegeometry{std}
% \usepackage[style=mla]{biblatex}
\let\HUGE\huge 
\usepackage{phd}
\usepackage{phd-runningheads}
\usepackage{phd-toc}
\usepackage{phd-lowersections}


\sethyperref
\addbibresource{phd.bib}% Syntax f
\cxset{palette zealous}
\begin{filecontents}{defaults-chapters}
%%    General Defaults for Chapters
\cxset{%    
    chapter title margin-top-width    =  0cm,
    chapter title margin-right-width  =  1cm,
    chapter title margin-bottom-width = 10pt,
    chapter title margin-left-width   = 0pt,
    chapter align                     = left,
    chapter title align               = left, %checked
    chapter name                      = Chapter,
    chapter format                    = stewart,%hdr,
    chapter font-size                 = Huge,
    chapter font-weight               = bold,
    chapter font-family               = sffamily,
    chapter font-shape                = upshape,
    chapter color                     = black,
    chapter number prefix             = ,
    chapter number suffix             = ,
    chapter numbering                 = arabic,
    chapter indent                    = 0pt,
    chapter beforeskip                = -3cm,
    chapter afterskip                 = 30pt,
    chapter afterindent               = off,
    chapter number after              = ,
    chapter arc                       = 0mm,
    chapter background-color          = bgsexy,
    chapter afterindent               = off,
    chapter grow left                 = 0mm,
    chapter grow right                = 0mm, 
    chapter rounded corners           = northeast,
    chapter shadow                    = fuzzy halo,
    chapter border-left-width         = 0pt,
    chapter border-right-width        = 0pt,
    chapter border-top-width          = 0pt,
    chapter border-bottom-width       = 0pt,
    chapter padding-left-width        = 0pt,
    chapter padding-right-width       = 10pt,
    chapter padding-top-width         = 10pt,
    chapter padding-bottom-width      = 10pt,
    chapter number color              = white,
    chapter label color               = white,    
    }
 \cxset{    
    chapter number font-size          = huge,
    chapter number font-weight        = bfseries,
    chapter number font-family        = sffamily,
    chapter number font-shape         = upshape,
    chapter number align              = Centering,
    section number color              = thesectionnumbercolor,
    subsection number color           = thesubsectionnumbercolor,
    subsubsection number color        = thesubsubsectionnumbercolor,
    }
\cxset{%    
     chapter title font-size        = Huge,
     chapter title font-weight      = bold,
     chapter title font-family      = calligra,
     chapter title font-shape       = upshape,
     chapter title color            = black,
     }    
\end{filecontents}
%% LaTeX2e file `defaults-chapters'
%% generated by the `filecontents' environment
%% from source `phd-lowersections' on 2015/07/21.
%%
%%    General Defaults for Chapters
\cxset{%
    chapter title margin-top-width    =  0cm,
    chapter title margin-right-width  =  1cm,
    chapter title margin-bottom-width = 10pt,
    chapter title margin-left-width   = 0pt,
    chapter align                     = RaggedLeft,
    chapter title align               = Centering, %checked
    chapter name                      = Chapter,
    chapter format                    = block,
    chapter font-size                 = HHUGE,
    chapter font-weight               = bold,
    chapter font-family               = sffamily,
    chapter font-shape                = upshape,
    chapter color                     = black,
    chapter number prefix             = ,
    chapter number suffix             = ,
    chapter numbering                 = arabic,
    chapter indent                    = 0pt,
    chapter beforeskip                = -1sp,
    chapter afterskip                 = 30pt,
    chapter afterindent               = off,
    chapter number after              = ,
    chapter arc                       = 0mm,
    chapter background-color       = bgsexy,
    chapter afterindent            = off,
    chapter grow left              = 0mm,
    chapter grow right             = 0mm,
    chapter rounded corners        = northeast,
    chapter shadow                 = fuzzy halo,
    chapter border-left-width      = 0pt,
    chapter border-right-width     = 0pt,
    chapter border-top-width       = 0pt,
    chapter border-bottom-width    = 0pt,
    chapter padding-left-width     = 0pt,
    chapter padding-right-width    = 10pt,
    chapter padding-top-width      = 10pt,
    chapter padding-bottom-width   = 10pt,
    chapter number color           = white,
    chapter label color            = white,
    }
 \cxset{
    chapter number font-size        = huge,
    chapter number font-weight      = bfseries,
    chapter number font-family      = sffamily,
    chapter number font-shape       = upshape,
    chapter number align            = Centering,
    }
\cxset{%
     chapter title font-size        = HHUGE,
     chapter title font-weight      = bold,
     chapter title font-family      = sffamily,
     chapter title font-shape       = upshape,
     chapter title color            = white,
     }
  

\cxset{chapter format=plain,
       chapter title color= creamy,
       chapter label color = creamy,
       chapter number color = creamy,
       chapter number font-size = Huge,
       subsection title color = creamy,
       chapter name = CHAPTER,
       chapter label case = upper,
       chapter number align=left,
       part format = traditional,
       part background-color=spot,
       part beforeskip                = -3cm,
       part afterskip                 = 30pt,
       section color                  = thesectioncolor,
       section font-family      = sffamily,
       subsection font-family   = sffamily,
       }
\usepackage{makeidx}      
\usepackage{phd-documentation}
\usepackage{phd-toc}
\makeindex
\setcounter{tocdepth}{4}
\usepackage{microtype}
\usepackage{tikz-page}
%\fancypagestyle{plain}{
%  \fancyhf{}
%  \chead{\tikzpage}
%}
%\fancypagestyle{titlepage}{
%  \fancyhf{}
%  \chead{\tikzpage}
%}

\begin{document}
\pagestyle{plain}
\parindent1em
\coverpage{asia}{Book Design }{Camel Press}{COLOR PALETTES}{} 
\pagestyle{empty}
%\coverpage{habtoor-city}{Delay Claim}{HLS-DSE/JV}{HABTOOR CITY}{MEP CLAIM} 
\secondpage
\pagestyle{empty}
\clearpage

\tableofcontents

\pagestyle{empty}
\setcounter{secnumdepth}{6}
\parskip0pt plus.1ex minus.1ex
\mainmatter
\pagenumbering{arabic}
\pagestyle{plain} 
\newcommand{\tikzpagelayout}{%
\tpshowframes
\tikzpageputanchors
}
\pagestyle{empty}
\tikzpage
\begin{tikzpicture}[remember picture, overlay] 
  \node[text width=0.5\textwidth] at (page.body center){\includegraphics[width=\textwidth]{balthus-11} This is the caption. \lorem };
   \node[text width=0.3\textwidth, below] at (page.top south){\includegraphics[width=\textwidth]{balthus-11} This is the caption. \lorem };
\end{tikzpicture}  


%\makeatletter
\cxset{defaults/.style ={% 
    chapter title margin-top-width    =  0cm,
    chapter title margin-right-width  =  1cm,
    chapter title margin-bottom-width = 10pt,
    chapter title margin-left-width   = 0pt,
    chapter align                     = left,
    chapter title align               = left, %checked
    chapter name                      = CHAPTER,
    chapter format                    = block,
    chapter font-size                 = Huge,
    chapter font-weight               = bold,
    chapter font-family               = sffamily,
    chapter font-shape                = upshape,
    chapter background-color          = white,
  % chapter label    
    chapter color               = black,
    chapter number prefix             = ,
    chapter number suffix             = ,
    chapter numbering                 = arabic,
    chapter indent                    = 0pt,
    chapter beforeskip                = -3cm,
    chapter afterskip                 = 30pt,
    chapter afterindent               = off,
    chapter number after              = ,
    chapter arc                       = 0mm,
    chapter label background-color    = white,
    chapter label color               = black,
   % chapter afterindent               = on,
    chapter grow left                 = 0mm,
    chapter grow right                = 0mm,
    chapter rounded corners           = northeast,
    chapter shadow                    = fuzzy halo,
    chapter border-left-width         = 0pt,
    chapter border-right-width        = 0pt,
    chapter border-top-width          = 0pt,
    chapter border-bottom-width       = 0pt,
    chapter padding-left-width        = 0pt,
    chapter padding-right-width       = 10pt,
    chapter padding-top-width         = 10pt,
    chapter padding-bottom-width      = 10pt,
    %  
    chapter number color              = black,
    chapter number background-color   = white,
    chapter number font-size        = huge,
    chapter number font-weight      = bfseries,
    chapter number font-family      = sffamily,
    chapter number font-shape       = upshape,
    chapter number align            = Centering,
    %
    chapter title font-size        = Huge,
     chapter title font-weight      = bold,
     chapter title font-family      = sffamily,
     chapter title font-shape       = upshape,
     chapter title color            = black,
     chapter title background-color = white,
     }%
   }  
\makeatother     
%\makeatletter
%\cxset{toc image=\@empty,
%       chapter toc=true,
%       title beforeskip=1pt}
%
%\@specialfalse
%
%
%\renewcommand\stewart[2][]{%
%\fancypagestyle{fancy}{%
%\lhead{}\rhead{}
%\chead{}
%\cfoot{}
%\lfoot{}
%\rfoot{\thepage}
%\def\footrule#1{{\color{blue}%
%  \hrule width\paperwidth}\vskip3pt
%}
%
%\renewcommand{\headrulewidth}{0pt}
%\renewcommand{\footrulewidth}{0.4pt}}
%
%\clearpage
%
%\begin{tikzpicture}[remember picture,overlay]
%% Main shading block
%\node [xshift=5cm,yshift=-\paperheight] at (current page.north west)
%[text width=0.98\textwidth,text height=\paperheight, fill=thecream!30,rounded corners,above right]
%{};
%\node [xshift=6.5cm,yshift=-1.5cm-\soffsety] at (current page.north west)
%[text width=0.9\textwidth,below right]{\sffamily \bfseries \huge #2};
%
%\node [xshift=3cm,yshift=-1.5cm] at (current page.north west)
%[text width=3cm,align=center,minimum height=2.5cm, fill=blue,below right]
%{\[\text{\HHUGE\bfseries\sffamily\color{white}\thechapter}\]
%\par\vspace*{3pt}
%};
%
%\node [xshift=-0.2cm,yshift=-21.5cm] at (current page.north west)
%[text width=3cm,above right]%
%{\includegraphics[width=1.0\paperwidth]{\image@cx}};
%% second box left
%\node [xshift=3cm,yshift=-19.5cm] at (current page.north west)
%[text width=9cm,minimum height=2.5cm,inner sep=0.5em, fill=blue,below right]
%{\color{white}
%  \bfseries\sffamily \texti@cx
%};
%% Last block
%\node [xshift=6.5cm,yshift=-26cm] at (current page.north west)
%[text width=12cm,above right]
%{\textii@cx
%};
%\end{tikzpicture}
%\par
%\clearpage
%}





\cxset{steward,
  chapter numbering=arabic,
  chapter format = stewart,
  offsety=0cm,
  image= {./images/hine02.jpg},
  texti={When Lamport designed the original \LaTeX\ sectioning commands he did not provide a fully comprehensive interface for modifying their design. With current tools available improvements are much easier to program and this chapter provides the details.},
  textii={\precis{In this chapter we discuss a method that allows the production of fancy chapter headings and formatting, based on a set of key values. Central  to this process is the separation of content from presentation.
We also discuss the basic formatting tools that are available and how one can modify them to mould new book designs.}
 }
}


\chapter{Designing Chapter Headings}
\addtocimage{-12pt}{-20pt}{./images/tocblock-man-01.jpg}

\section*{Introduction}

A \textls*{crowded} first page is as unsightly as a crowded title page, wrote De Vinne in \emph{Modern Methods of Book Composition} in 1904.  Not much has changed since. A new chapter must make a good impression and must give an immediate signal that a different topic is going to be discussed. Traditionally chapter openings in LaTeX are an unimpressive and dry event. Our aim is to brighten it up a bit, while keeping true separation of content from presentation, but avoiding the pit traps of over ornamenting the design. A book is to be read and we should provide minimal ornamentation. \index[phdkeys]{chapter> ornamentation}

% \usepackage{array,tabularx}
%\newcolumntype{Y}{>{\raggedleft\arraybackslash}X}% see tabularx
%\tcbset{enhanced,fonttitle=\bfseries\large,fontupper=\normalsize\sffamily,
%colback=yellow!10!white,colframe=red!50!black,colbacktitle=thecodebackground,
%coltitle=black,center title,
%tabularx={X||Y|Y|Y|Y||Y},% this sets ’before upper’ and ’after upper’
%before upper app={Group & One & Two & Three & Four & Sum\\\hline\hline} }
%
%\begin{tcolorbox}[title=My table]
%Red & 1000.00 & 2000.00 & 3000.00 & 4000.00 & 10000.00\\\hline
%Green & 2000.00 & 3000.00 & 4000.00 & 5000.00 & 14000.00\\\hline
%Blue & 3000.00 & 4000.00 & 5000.00 & 6000.00 & 18000.00\\\hline\hline
%Sum & 6000.00 & 9000.00 & 12000.00 & 15000.00 & 42000.00
%\end{tcolorbox}

\begin{figure}[htbp]
\centering
\parindent=0pt
\fbox{\includegraphics[width=\textwidth]{metropolitan-spread}}
\par
\caption{A chapter opening from the Metropolitan Museum of Art publicaion, \textit{Assyrian Reliefs and Ivories} by Vaughn. E. Crawford et. al., 1980. The spread is simple and the chapters are not numbered. This is a common characteristic of many more recently published books.}
\end{figure}


What is to us now a common occurence with instant book-printing was not always so. The cost of illustrated books was a prime factor and as Tschichold wrote:
\begin{quotation}
In the area of book design, in the last few years a revolution has taken place, until recently recognized by only a few. but which now begins to influence a much wider range of action.
It means placing much greater emphasis on the appearance of the book and a wholly contemporary use of typographic and photographic means. Before the invention of printing, literature of that time was spread around by the mouth of the author himself or by professional bards. The books of the Middle Ages - like the "Mannessische Liederhandschrift" - had
\end{quotation}

The type of book you are writing and its contents will determine an appropriate design for chapter headings and the type of design and numbering if any for subsections. Here we are merely providing a mechanism to produce them. These methods can produce a mastepiece or an ugly piece of work. Some simple suggestions follow (from my observations of styles in books I like). In general you need to think what type of book you are developing. For example a novel, should be sectioned very carefully. Many books avoid marking of sections other than chapters totally, perhaps marking them just with a soft ornament such as three centered asterisks.

\section{Numbering of Sections}


In general books do not number sections beyond subsection. You can avoid them all together, if you are not going to reference the sections extensively. 

In works of fiction, authors sometimes number their chapters eccentrically, often as a metafictional statement. For example:
Seiobo There Below by László Krasznahorkai has chapters numbered according to the Fibonacci sequence.

The Curious Incident of the Dog in the Night-Time by Mark Haddon only has chapters which are prime numbers.

At Swim-Two-Birds by Flann O'Brien has the first page titled Chapter 1, but has no further chapter divisions.

God, A Users' Guide by Seán Moncrieff is chaptered backwards (i.e., the first chapter is chapter 20 and the last is chapter 1). The novel The Running Man by Stephen King also uses a similar chapter numbering scheme.
Every novel in the series A Series of Unfortunate Events by Lemony Snicket has thirteen chapters, except the final instalment (The End), which has a fourteenth chapter formatted as its own novel.

Mammoth by John Varley has the chapters ordered chronologically from the point of view of a non-time-traveler, but, as most of the characters travel through time, this leads to the chapters defying the conventional order.


\begin{pgfpicture}
\pgfpathmoveto{\pgfpointorigin}
\pgfpathlineto{\pgfpoint{1cm}{1cm}}
\pgfpathlineto{\pgfpoint{1cm}{0cm}}
\pgfusepath{fill}
\end{pgfpicture}




\begin{figure}[tbp]
\centering
\parindent=0pt
\fbox{\includegraphics[width=\textwidth]{fantasy-architecture}}
\par
\caption{A chapter opening from the Metropolitan Museum of Art publicaion, \textit{Assyrian Reliefs and Ivories} by Vaughn. E. Crawford et. al., 1980. The spread is simple and the chapters are not numbered. This is a common characteristic of many more recent books.}
\end{figure}


\begin{figure}[tbp]
\centering
\parindent=0pt
\fbox{\includegraphics[width=\textwidth]{fantasy-architecture-02}}
\par
\caption{A chapter opening from the Metropolitan Museum of Art publicaion, \textit{Assyrian Reliefs and Ivories} by Vaughn. E. Crawford et. al., 1980. The spread is simple and the chapters are not numbered. This is a common characteristic of many more recent books.}
\end{figure}


\section*{Use of Color}

The modern books that Tschilchod was discussing have long been overwhelmed by the appearance of larger, coffee book type of books. Our brains our now conditioned by branding and graphic design is everywhere. 

Once you have decided that the book is going to be a bit more colorfull, the choice of color will follow. The decision what to color will be an important one, which brings us to color theory. The history of color is perhaps as colorfull as the rest. Attempts to formalize and recognize order date back to Aristotle (384-322 bce) but began in earnest with Leonardo da Vinci (1452-1519) and have progressed ever since. Leonardo noted that certain colors intensify each other, discovering \textit{contrary} and \textit{complementary} colors. The first color wheel was invented by Britain's Sir Isaac Newton (1642-1727), who split white light into red, orange, yellow, green, blue, indigo and violet beams, then joined the two ends of the spectrum to form a circle showing the natural progression of colors. When Newton created the color wheel, he noticed that mixing two colors from opposite positions produced a neutral or \textit{anonymous} color.


\begin{figure}[htbp]
\parindent=0pt
\centering
\fbox{\includegraphics[width=\textwidth]{line-designs} }
\caption{Spread from \textit{Beautiful Geometry}, Eli Maor and Eugen Jost, Princeton Univeristy Press, 2014. A subtle coloring of the chapter heading, de-emphasizing the chapter number and coloring the chapter title. There is no chapter label. A dropcap with the same color starts the first paragraph. This style is easy to achive with the phd system.}
\end{figure}


\begin{figure}[htbp]
\parindent=0pt
\centering
\fbox{\includegraphics[width=\textwidth]{color-book01.jpg} }
\bigskip

\fbox{\includegraphics[width=\textwidth]{color-book02.jpg} }
\end{figure}

One would expect a book written for the sole purpose of describing color theory and its application to the Graphic Arts, is expected to be colorful. Note the de-emphasizing of the label and number. 

\begin{figure}[htbp]
\parindent=0pt
\centering
\fbox{\includegraphics[width=\textwidth]{color-book-03.jpg} }
The chapter heading label and number are almost invisible. The heading text, is typeset in large bold letters, shouting what is coming next. Not your typical scintific book\ldots
\bigskip

\fbox{\includegraphics[width=\textwidth]{color-book-04.jpg} }
\end{figure}

Advertizing people understand that they need to present the message of an advertizement loud and clear so as to catch the busy eye. A heading's message is the title description. Neither the label not the chapter if any are necessary to convey the message. The chapter heading is analogous to the stop at the end of a sentence. The brain gets a signal to absorb what was written before it and get ready for the next. The heading signals the end of a topic. One must not dwell on it.


\section{Contemporary Chapter Headings}

In the book \textit{China} the designer used both a chapter heading on a spread of two images, as well as repeated the chapter number on the text pages \ref{fig:threepage}. The images distill the message of the chapter, although the chapter subtitle is almost unreadable, dominated by the surrounding text. From a technical perspective, the chapter command must paint the two images, set the right type of heading for each page and then without increasing the counter, change the counter to one that displays the chapter number in words and then continue with typesetting the text. A careful choice of images is necessary for such chapters, as well as cropping the images to match the aspect ratio of the book pages. One also needs to be carefull for \latexe not to place any floats in between the page spreads. 

\begin{figure}[htbp]
\parindent=0pt
\centering
\fbox{\includegraphics[width=\textwidth]{beijing.jpg} }\par
\vfill

\fbox{\includegraphics[width=\textwidth]{beijing-01.jpg} }\par
%\fbox{\includegraphics[width=\textwidth]{pearl-river.jpg} }
\caption{A full page chapter spread.}
\label{fig:threepage}
\end{figure}

\begin{figure}[htbp]
\parindent=0pt
\centering
\fbox{\includegraphics[width=\textwidth]{beijing.jpg} }\par
\vfill

\fbox{\includegraphics[width=\textwidth]{beijing-01.jpg} }\par
%\fbox{\includegraphics[width=\textwidth]{pearl-river.jpg} }
\caption{A full page chapter spread.}
\label{fig:threepage}
\end{figure}


\clearpage



In Figure~\ref{fig:photospread} the bands are black, but position low on the page. The size of the pages are 9.69 \texttimes 11.42. The books sections are not numbered. Text i sbroken through inserts of bigger text. Many of the examples here are from
commercial nude photography books, as they tend to break with tradition. In the 1970s and 1980s, fashion photographers began to present a
new, confrontational image of the female body. The pioneer in this
respect was the German Helmut Newton (1920–2004). Newton’s
photographs of nudes were overtly sexual, with an undertone of
menace, and although his models tended to be depicted as part
of the social elite they were often placed, apparently caught out
in reportage style, in sordid environments engaged in fantasy and
fetish. His work made him highly influential in fashion photography,
though some of it was thought too highly sexual for American
magazines and appeared only in those published in Europe.


\begin{figure}[htbp]
\parindent=0pt
\includegraphics[width=\textwidth]{baetens-01.jpg} \par
\vfill\vfill\vfill\vfill
\includegraphics[width=\textwidth]{baetens-02.jpg}\par
\caption{Chapter spread and first pages after the chapter title which is on the right page of the chapter spread. From \textit{New Photography, Art and the Craft}, Pascal Baetens, DK Publications. }
\label{fig:photospread}
\end{figure}

In the 1980s, Newton undressed the dynamic and independent
female in a series called Big Nudes. In this series the women are
indeed naked and very tall, wearing nothing but makeup and high
heels. The Big Nudes were exhibited in the form of life-size prints
that were intended to provoke the viewer by showing self-confident
women who knew what they wanted and were very aware of their
beauty and sexuality



\chapter{Package Usage}

To use the package include it just like any other package:

\begin{teXXX}
\documentclass{book}
\usepackage{phd}
\cxset{style13}
\begin{document}
\chapter{Introduction}
\end{document}
\end{teXXX}

The command \docAuxCommand{cxset} sets the default style for the example to the style defined as \meta{style13}. The package currently offers  100 templates and numerous keys to manipulate them further. Styles are similar to \enquote{themes} used in web programming; they are a collection of keys that resemble in many ways \texttt{css}. Styles can have any names and I am sure as package usage increases and evolve,they will get better names. 

\section{Background}

Before describing in detail how to specify a new layout for headings, we offer an overview of how the task can be accomplished and the design philosophy behind the approach. 

Irrespective of the technique and tools used, the creation of new layouts can always be divided into the following three tasks: constructing a document from “layout bricks”, which we can term as “blocks” or “elements”; establishing the layout semantics of each block; and finally, creating a layout engine supporting any document constructed from such blocks.

\begin{description}
\item [Canned Layouts] At one end of the spectrum, the most accessible approach consists of picking, a canned layout, such as LaTeX itself and perhaps only provide rudimentary macros to manipulate it.
\item [Constraints] Constraints offer a middle ground between canned layouts and handwritten layout engines. Constraints are arguably the most widespread and successful layout programming technique. For, instance, the foundations of \tex are laid upon constraint. CSS, the ubiquitous web template language, also relies on constraints, although in a more restricted and indirect manner.
\end{description}

\subsection{Blocks and Elements}

We define an \emph{element} as a document block, that cannot be subdivided further. For example the chapter title element, is composed of the text of the chapter title. 

A \emph{block} on the other hand is can contain other blocks and or numerous elements. We can consider the chapter headings as \emph{blocks}, composed of three blocks the chapter, number and title. Each block is then composed of elements. Each element has properties and traits. One of these mandary properties is the name. 

Blocks are either \emph{configured} (all constraints are mandatory), or flexible (there are optional/alternative constraints). By bundling optional constraints, flexible blocks make their specification customizable by non-technical users. 

\subsection{Language semantics}

One of the aims of the syntax of the templates was to offer familiar terminology and to remove the use
of \tex macros as far as possible from templates. 
\medskip

{\parindent0pt

 \textit{section}| font-family=serif,|\\
 \textit{section}| font-size=LARGE,|\\
 \textit{section}| font-weight=bold,|\\
}

The restriction I imposed is problematic when dealing with fractions of linewidths and textwidths. So
at present we allow for example |title text-width=0.5\texwidth| or |title text-width=10cm| or any other valid units. Ideas for improvements can only come from user feedback in the future.

Some experimental ideas incorporated are:

\begin{verbatim}
title text-width = 0.5 text-width,
title text-width = 1.2 text-width,
\end{verbatim}

A better parser will need to be programmed for dimensions, which are all currently handled as etex |dimexpr|. 

The syntax must allows both for microtypography as well as macro-typographical features. The former would deal with mostly fonts, spacing and text justification, where the latter deals with layouts, borders shapes and the positioning of elements on the page and also reletively to other elements or blocks.

An advantage of this approach is that it also opens the possibility of parsing the text with a language other than \tex and translating the document to another format, such as |HTML| or |XML| either fully or partially. Next we will describe both the syntax as well as the usage of the settings.

\section{Chapter opening page}

The standard \latexe classes offer only two options to either open a chapter on an odd page or at any page. This package offers five alternatives:

\begin{docKey}[phd]{chapter opening}{=\meta{any, left, right, anywhere, ifafter}}{default none, initial=any}
For documents that are primarily to be read on the web, use |any| for normal books, use \textit{right}. Some templates that we provide use |any| and the examples use |anywhere| to enable us to display the heading at any position on the page.
\end{docKey}

\begin{decription}
\item [any] Opens a chapter at any page, either \textit{verso} or \textit{recto}.
\item [left] Opens a chapter on an even page
\item [right] Opens a chapter on a right page.
\item [anywhere] Opens a chapter at the point where the \cs{chapter} is typed.
\item [none] Alias for \marg{anywhere}.
\item [ifafter] Opens a chapter at the next page if the page has material that does not exceed a certain portion of \cs{textheight}.
\end{description}

\colorlet{theoption}{bgsexy}

To change a setting you just modify the value of the key \oarg{\option{chapter opening}} to one of the values described earlier. 

\begin{dispListing}
\cxset{chapter opening = anywhere}
\end{dispListing}
 
We use this key to print the many examples typesetting chapter heads that follow (see the example~\ref{ex:anywhere}).  


\begin{texexample}{title=Inline Chapter Example}{ex:anywhere}
\cxset{examplestyle/.style = {chapter format = block,
       chapter opening = anywhere,
       chapter name = CHAPTER, 
       %label
       chapter label font-family      = sffamily,
       chapter label color            = primary,
       chapter label background-color = white,
       % number
       chapter number font-family = sffamily,
       chapter number font-size = HUGE,
       chapter number color     = primary,
       chapter label align = centering,
       chapter number background-color = white,
       %title
       chapter title font-family = rmfamily,
       chapter title align = centering,
       chapter title background-color = bgsexy!15,
       chapter title before background-color=white}}
\cxset{examplestyle}       
\lorem
\chapter{Typography Example}
\lorem
\chapter{Another Chapter Heading}
\lorem
\end{texexample}


%\cxset{toc chapter = true}
\addtocounter{chapter}{-1}

Examples for other types of chapter openings follow in the rest of the documentation.

\subsection{Blank pages before chapters}

In the standard LaTeX book class when the \texttt{openany} option is not given or in the report class when the openright is given, chapters start at odd-numbered pages. This can cause a blank page to be printed. Some book designers prefer this page to be completely empty, without any headers or footers. This cannot be done with \lstinline{\thispagestyle} as this command will have to be issued on the \textit{previous} page. However by a suitable redefinition of the
\lstinline{\clearpage} this can be done automatically.
\medskip

\begin{teXXX}
\makeatletter
\def\cleardoublepage{\clearpage\if@twoside\ifodd\c@page\else
  \hbox{}
  \vspace*{\fill}
  \begin{center}
    This page left intentionally blank.
  \end{center}
  \vspace{\fill}
  \thispagestyle{empty}
  \newpage
  \if@twocolumn\hbox{}\newpage\fi\fi\fi}
\makeatother
\end{teXXX}


This is achieved easily by setting the following options:
\bigskip

\begin{tcolorbox}
\lstinline{chapter blank page=empty}\par
\lstinline{chapter blank page text=Some text.}\par
\lstinline{chapter blank page=plain}\par
\end{tcolorbox}
\medskip



The last one refers to a \lstinline!\thispagestyle{plain}!.
\cxset{chapter opening = right, chapter format = block}
\chapter{Test}

\cxset{defaults, chapter opening= anywhere}



\section*{Keys for chapter head formatting}

A chapter heading can be considered of being constructed of several parts, the \textit{chapter number}, the chapter name typically \textit{chapter} and the \textit{title}. Predefined keys handle all the elements of formatting. Additional keys are defined to handle other elements such as inclusion of images or producing complicated examples with graphics constructed with \texttt{TikZ} and other similar packages.


\bigskip\bigskip\bigskip\bigskip
\let\oldrefkey\refKey
\let\refKey\texttt
\makeatletter
\long\def\demobox#1#2{%
\par\bigskip\bigskip\bigskip
\begin{tcolorbox}[enhanced,left=0pt, top=0pt, bottom=0pt,width=\textwidth,
  enlarge top initially by=1cm,enlarge bottom finally by=1cm,left skip=1cm,right skip=1cm,
  colframe=white,colback=white,
  colbacktitle=red!30!white,colupper=black!7!white,
  code={\appto\kvtcb@shadow{%
    \path[fill=white,draw=yellow!50!black,dashed,line width=0.4pt]
      ([xshift=-1cm,yshift=-1cm]frame.south west) rectangle
      ([xshift=1cm,yshift=1cm]frame.north east);
     \path[fill=blue!20!white, 
              opacity=0.3, draw=yellow!50!black,solid,line width=1pt]
      ([xshift=-2cm,yshift=-2cm]frame.south west) rectangle
      ([xshift=2cm,yshift=2cm]frame.north east);  
    }},
  finish={
  \draw[thick,<->] ([yshift=-1.3cm]frame.north west)-- node[below]{\texttt{#1 width}}
    ([yshift=-1.3cm]frame.north east);
  \draw[thick,<->] ([xshift=-15mm]frame.north east)-- node[above]{\refKey{#1 height}}
    ([xshift=-15mm]frame.south east);
  \draw[thick,<->] (frame.north)-- node[right]{\refKey{#1 padding-top}} +(0,1);
  \draw[thick,<->] ([yshift=1cm]frame.north)-- node[right]{\refKey{#1 margin-top}} +(0,1);
  \draw[thick,<->] (frame.south)-- node[right, align=left]{\refKey{#1 padding-bottom}}+(0,-1);
  %left padding
  \draw[thick,<->] (frame.west)-- node[below right,align=center]{\refKey{#1 padding-left }}+(-1,0);
  %left margin
  \draw[thick,<->] ([xshift=-1cm,yshift=-0.9cm]frame.west)-- node[below right,xshift=-1,align=left]{\refKey{#1 margin-left }\\\refKey{#1 grow to left by}}+(-1,0);
  %right padding
  \draw[thick,<->] (frame.east)-- node[below left,align=center]{\refKey{#1 padding-right}}+(1,0);
 %right margin
  \draw[thick,<->] ([xshift=1cm,yshift=-0.9cm]frame.east)-- node[below left,xshift=1, align=right]{\refKey{#1 margin-right}\\\refKey{#1 grow to right by}}+(1,0);
 \draw[thick,<->] ([yshift=-2cm]frame.south)-- node[right, align=left]{\refKey{#1 margin-bottom},\\ \refKey{#1 after skip}}+(0,1);
  }
    ]
#2%
%\hrule width0pt height4.5cm depth0pt\relax% \vspace*{4.5cm}% \lipsum[1]
\end{tcolorbox}\par
\bigskip\bigskip\bigskip}
\makeatother

\demobox{chapter}{\scalebox{1.17}{\HHHUGE Chapter}}

The number box is again drawn in a box similar to a chapter with all properties generalized.

\demobox{number}{\scalebox{1.15}{\HHHUGE Thirteen}}



All parameters shown in the diagram can be set using the command \cs{cxset}. The property names follow conventions similar to those of |css|, rather than typical conventions of \tikzname that are more widely known to the programming community. The prefix to these properties (in the example \textit{chapter}) can be thought of
as similar to a |class| or |id| name in |css|.  

\begin{docCommand}{cxset}{\marg{options}}
  Sets options for every following \refEnv{tcolorbox} inside the current \TeX\ group.
  By default, this does not apply to nested boxes, see \Vref{subsec:everybox}.\par
  For example, the colors of the boxes may be defined for the whole document by this:
\begin{dispListing}
\cxset{chapter numbering = Roman,
       chapter number color = blue}
\end{dispListing}
\end{docCommand}

\begin{docKey}[]{chapter padding-top}{=\meta{dimension}}{no default, initial value 0pt}
All padding keys take one argument, which is a dimension. The length is also stored in a register
\cmd{\chapterpaddingtop}. In this chapter it was set at %\the\chapterpaddingtop.
\begin{dispListing}
\cxset{colback=red!5!white,colframe=red!75!black, chapter padding-top=2pt}
\end{dispListing}
\end{docKey}



\begin{docKey}[]{chapter padding-right}{=\meta{dimension}}{no default, initial value 0pt}
All padding keys take one argument, which is a dimension. The length is also stored in a register
\cmd{\chapterpaddingright}.  In this chapter it was set at %\the\chapterpaddingright.
\end{docKey}

\begin{docKey}[]{chapter padding-bottom}{=\meta{dimension}}{no default, initial value 0pt}
All padding keys take one argument, which is a dimension. The length is also stored in a register
\cmd{\chapterpaddingbottom}.  In this chapter it was set at %\the\chapterpaddingbottom.
\end{docKey}

\begin{docKey}[]{chapter padding-left}{=\meta{dimension}}{no default, initial value 0pt}
All padding keys take one argument, which is a dimension. The length is also stored in a register
\cmd{\chapterpaddingleft}.  In this chapter it was set at %\the\chapterpaddingleft.
\end{docKey}

%% margin

\begin{docKey}[]{chapter margin-top}{=\meta{dimension}}{no default, initial value 0pt}
All padding keys take one argument, which is a dimension. The length is also stored in a register
\cmd{\chaptermargintop}. In this chapter it was set at .
\end{docKey}

\begin{docKey}[]{chapter margin-right}{=\meta{dimension}}{no default, initial value 0pt}
All padding keys take one argument, which is a dimension. The length is also stored in a register
\cmd{\chapterpaddingright}.  In this chapter it was set at %\the\chapterpaddingright.
\end{docKey}

\begin{docKey}[]{chapter margin-bottom}{=\meta{dimension}}{no default, initial value 0pt}
All padding keys take one argument, which is a dimension. The length is also stored in a register
\cmd{\chapterpaddingbottom}.  In this chapter it was set at %\the\chapterpaddingbottom.
\end{docKey}

\begin{docKey}[]{chapter margin-left}{=\meta{dimension}}{no default, initial value 0pt}
All padding keys take one argument, which is a dimension. The length is also stored in a register
\cmd{\chaptermarginleft}.  In this chapter it was set at %\the\chaptermarginleft.
\end{docKey}

\subsection{Borders}

Border have three properties \emph{width, color} and \emph{style}. They can set individually for
each side of the box or using the shorter key .

\begin{docKey}[]{chapter border-top-width}{ = \meta{dimension}}{no default, initial value 0pt}
All border keys take one argument, which is a dimension.
\end{docKey}

\begin{docKey}[]{chapter border-right-width}{=\meta{dimension}}{no default, initial value 0pt}
All border keys take one argument, which is a dimension.
\end{docKey}

\begin{docKey}[]{chapter border-bottom-width}{ = \meta{dimension}}{no default, initial value 0pt}
All border keys take one argument, which is a dimension.
\end{docKey}

\begin{docKey}[]{chapter border-left-width}{ = \meta{dimension}}{no default, initial value 0pt}
All border keys take one argument, which is a dimension.
\end{docKey}

\subsubsection{Border Colors}

The colors follow the same pattern for |border-width| and again they can be set individually or using
a shorter key to set all of them in one color. 

\begin{docKey}[]{chapter border-top-color}{=\meta{color name}}{no default, initial value black}
All border keys take one argument, which is a dimension.
\end{docKey}

\begin{docKey}[]{chapter border-right-color}{=\meta{color name}}{no default, initial value black}
All border keys take one argument, which is a dimension.
\end{docKey}

\begin{docKey}[]{chapter border-bottom-color}{=\meta{color name}}{no default, initial value black}
All border keys take one argument, which is a dimension.
\end{docKey}

\begin{docKey}[]{chapter border-left-color}{=\meta{color name}}{no default, initial value black}
This key is stored in \cmd{\chapterborderrightcolor} and the value in this chapter is 
%\ExplSyntaxOn \l_phd_chapter_border_right_color_tl.
\ExplSyntaxOff
\end{docKey}



\subsubsection{Border Styles}

Standard |css|  offers four styles \emph{dotted, solid, double, dashed}. We offer almost an unlimited set of styles.

\begin{docKey}[phd]{chapter border-top-style}{=\meta{style name}}{no default, initial value \texttt{none}}
The |border-style| properties take a value, which can be |solid, double, dotted, dashed, asterisk|.
\end{docKey}

\begin{docKey}[phd]{chapter border-right-style}{=\meta{style name}}{no default, initial value \texttt{none}}
The |border-style| properties take a value, which can be |solid, double, dotted, dashed, asterisk|.
\end{docKey}

\begin{docKey}[]{chapter border-bottom-style}{=\meta{style name}}{no default, initial value \texttt{none}}
The |border-style| properties take a value, which can be |solid, double, dotted, dashed, asterisk|.
\end{docKey}

\begin{docKey}[]{chapter border-left-style}{=\meta{style name}}{no default, initial value \texttt{none}}
The |border-style| properties take a value, which can be |solid, double, dotted, dashed, asterisk|.
\end{docKey}

\begin{docKey}[phd]{chapter border-style}{=\meta{style name}}{no default, initial value \texttt{none}}
This key sets all chapter-border-\meta{top,right,bottom,left}-style to a single value.
\end{docKey}

\subsubsection{Fonts and colors}

All font parameters can be set using individual keys. The naming scheme in general follows |css| conventions.

\begin{docKey}[phd]{chapter color}{=\meta{color name}}{no default, initial value \texttt{black}}
This key sets the color for the \textit{chapter element}. The color name is stored in \cmd{\chaptercolor@cx}.
The value in this chapter is% \makeatletter\texttt{\chaptercolor@cx}\makeatother.
\end{docKey}

\begin{docKey}[phd]{chapter font-size}{=\meta{Huge, Large}}{no default, initial value \texttt{Huge}}
This sets the size for rendering the \textit{chapter element}. Use one of the following predefined values.
Note that you can either use a command i.e, |chapter font-size=|\cmd{\huge} 
or the command name i.e., |chapter font-size=huge|. The latter is the recommended method.
\end{docKey}

\begin{marglist}
\item [tiny] renders as {\tiny tiny}.
\item[footnotesize] renders as {\footnotesize footnotesize}
\item [small] Opens a chapter on an even page
\item [large] Opens a chapter on a right page.
\item [LARGE] Opens a chapter at the point where the \cs{chapter} is typed.
\item [huge] Alias for \marg{anywhere}.
\item [Huge] Opens a chapter at the next page if the page has material that does not exceed a certain portion of
 \cs{textheight}.
 \item[HUGE] renders as {\HUGE HUGE}.
 \item[HHUGE] renders as {\HHUGE HUGE}.
\end{marglist}

\begin{texexample}{Sizing settings}{}
\cxset{
          chapter format = block,
          chapter label font-size= HUGE,
          chapter name = Chapter,
          chapter format=block,
          chapter number font-size= HUGE,
          chapter title font-size=LARGE,
         % 
         % chapter padding-top=0pt,
         % chapter padding-bottom=0pt,
         % title margin-top=3pt,
         %
          }
\chapter{Setting font-sizes}          
\lorem

\end{texexample}


\begin{docKey}{chapter font-family}{ = \meta{sffamily, rmfamily etc.}}{no default, initial value \texttt{sffamily}}
The |font-family| key accepts \latexe conventional family names or |css| names such as |serif| and |non-serif|. The
value is stored in \docAuxCommand{chapter_font_family}, in this chapter it is set as {\ExplSyntaxOn\meaning\chapter_font_family\ExplSyntaxOff}
\end{docKey}


\begin{marglist}
\item [sffamily] The \emph{chapter element} is rendered in the document default \cmd{\sffamily}.
\item [rmfamily] The \emph{chapter element} is rendered in the document default \cmd{\rmfamily}.
\end{marglist}

%% Font weights
\begin{docKey}[]{chapter font-weight}{=\meta{mdseries,bfseries,etc.}}{no default, initial value \texttt{bfseries}}
The |font-weight| key accepts \latexe conventional family names or |css| names such as |bold| and |bfseries|. The
value is stored in \cmd{\chapterfontweight@cx}, in this chapter it is set as 
{\ExplSyntaxOn\expandafter\string\l_phd_chapter_label_fontweight_tl\ExplSyntaxOff}

\begin{texexample}{Setting chapter element font-weights}{fontweight}
\cxset{chapter label font-weight=normal}
\chapter{Font-weight is normal}
\cxset{chapter label font-weight= bfseries}
\chapter{Font-weight is bfseries}
\lorem
\end{texexample}
\end{docKey}


\begin{marglist}
\item [normal] The \emph{chapter element} is rendered in the document default \cmd{\sffamily}.
\item [bold] The \emph{chapter element} is rendered in the document default \cmd{\rmfamily}.
\item[bfseries] Renders as bold.
\item[mdseries] renders as medium series.
\item[light] This is an alias for normal.
\item[\upshape\ttfamily\string\bfseries] The command version of the setting.
\item[\upshape\ttfamily\string\mdseries] The command version of the setting.
\end{marglist}



\begin{docKey}[]{chapter font-shape}{=\meta{itshape,upshape,etc.}}{no default, initial value \texttt{upshape}}
The |font-weight| key accepts \latexe conventional family names or |css| names such as |bold| and |bfseries|. The
value is stored in |chapter_font_weight|, in this chapter it is set as %\ExplSyntaxOn \texttt{\chapter_font_shape}\ExplSyntaxOff.
\end{docKey}

In |css| the |font-shape| is named as |font-style| so we alias it as well. 

%\begin{marglist}
%\item[normal] normal font-style, defaults to |upshape|.
%\item[upshape] normal font-style, defaults to |upshape|. 
%\item[italic] italic shape, renders as {\itshape italic}. For some fonts it might not be available.
%\item[itshape] italic shape, alias of |italic|.
%\item[oblique] oblique font, in \latexe is equivalent to \cmd{\slshape} and renders as {\slshape slshape}, which might be slightly different than {\itshape italic}.
%\end{marglist}


\begin{texexample}{Setting up Fonts}{chapterfonts}
\cxset{   chapter format = block,
          chapter opening=anywhere,
          chapter label font-weight=normal,
          chapter label font-shape=upshape,
          %chapter border-width=0pt,
          %chapter border-style=none,
          %chapter padding-top=0pt,
          chapter label font-size=large,
          chapter number font-size=large,
          chapter number color=black,
          %title font-size=large,
          }
\chapter[fonts]{Test Font Weights}
\lorem
\cxset{chapter label font-shape=itshape}
\chapter{Test Italic Shape}
\lorem
\cxset{chapter label font-shape=normal}
\chapter{Test normal font-shape}
\lorem
\end{texexample}



The specification of font families is somewhat problematic. In the web the |css| allows |font-family|  to hold several font names as a ``fallback” system. If the browser does not support the first font, it tries the next font.

There are two types of font family names:

\begin{description}
\item[family-name] The name of a font-family, like “times”, “courier”, “arial”, etc.
\item[generic-family] The name of a generic family, like “serif”, “sans-serif”, “cursive”, “fantasy”, “monospace”.
\end{description}

Generally in the \tex community leaving the choice of font  open to what is available on a user’s computer is frowned upon. Knuth’s original aim to render consistently documents, irrespective of a user’s computer installation has served the community well, and it is possible three decades later to produce documents identical in all respects to the original. 

If this is still a valid requirement for documents is debatable. Current document processing requirements are focusing more in the preservation of content and document structure rather than form. Typeset documents in soft copy are now widely preserved in |pdf| or |postcript|  formats. One can archive the |.tex| file as well as the |pdf| file.  Back to the provision of keys, a key defined in a 
similar fashion to those of |css| could help, but there is also the issue of slow compilation. If a font cannot be
found, with the current code, it can slow down compilation tremendously. I am leaving the choice where it belongs to the user and the package writer. It makes no harm if a more flexible definition is provided. The user can then decide to only provide one or many fonts. 

This avoids complicated and almost unintelligible commands such as:

\begin{dispListing}
\setkomafont{subsection}{\usefont{T1}{fvm}{m}{n}}
\setkomafont{section}{\usefont{T1}{fvs}{b}{n}\Large}
\end{dispListing}

Here are some examples. 

\begin{texexample}{Serif and non-serif}{ex:fontfamily}
\cxset{chapter label font-family=serif, 
       chapter opening=anywhere}
\chapter{Serif font}
\lorem
\end{texexample}


\section{Floating and Alignment} 

This particular key bothered me, as the term \emph{float} has a different meaning in \latexe. However, to
be consistent with |css| terminology I have yielded to the temptation and included it.

\begin{docKey}[]{chapter float}{=\meta{left,center,right,none}}{no default, initial value \texttt{none}}
Key that controls the horizontal alignment of the \emph{chapter element}. I order for the
element to float, its |display| property must be set to |inline|.
\end{docKey}

%\begin{texexample}{Floating}{chapter:float}
%\cxset{chapter opening=anywhere, chapter float=center}
%\chapter{Centered Chapter}
%\lorem
%\cxset{chapter float=left}
%\chapter{Left Aligned}
%\lorem
%\cxset{chapter float=right}
%\chapter{Right Aligned}
%\lorem
%\end{texexample}


\subsection{The display property}

Both the |css| box model as well as the \TeX layout engine provide numerous complex algorithms in managing the floating of elements. This is normally controlled using two properties |display| and |float|.


\makeatletter

\begin{docKey}[phd]{chapter position}{ = \meta{absolute, relative}}{no default, initial value black}
This positioning directive instructs the engine to position the element at an exact position.
\end{docKey}



\tcbox[nobeforeafter]{$box_1$}\tcbox[nobeforeafter]{$box_2$}\tcbox[nobeforeafter]{$box_3$}\dotfill\tcbox[nobeforeafter]{$box_n$}
\tcbox[before skip=0.2cm, after skip=0pt, width=1cm, enlarge left by=10cm,width=5cm,enhanced,show bounding box]{title before element}
\tcbox[before skip=0pt, width=1cm, enlarge left by=10cm,width=5cm,enhanced,show bounding box]{
\tcbox{tb}\tcbox{title}\tcbox[nobeforeafter, width=1cm,]{tb}}
\tcbox[before skip=0pt, after skip=12pt, width=1cm, enlarge left by=10cm,width=5cm,enhanced,show bounding box]{\emph{title after} element \fbox{some}}
\makeatother

\begin{docKey}[phd]{chapter float}{=\meta{left,center,right,none}}{no default, initial value \texttt{none}}
Key that controls the horizontal alignment of the \emph{chapter element}. I order for the
element to float, its |display| property must be set to |inline|.
\end{docKey}
In document preparation systems or web page development the layout is user generated, i.e., the user is expected to type the html and the |css| will then specify as to how the page will be rendered by the browser. In our case for documents we can specify how we want the headings to look. The layout manager for each element, creates other associated elements, as shown for the title here. This way most layouts can be accomplished with the declarative visual language of the \pkgname{phd} package. 

\subsubsection{In-line elements}

When an element is specified as |inline| the rendering algorithm places the boxes after each other. This is widely used in |chapter elements| to render the number inline with the chapter name.
\medskip
\bgroup

\noindent
\tcbox[nobeforeafter,width=3cm, height=1cm]{Chapter}\tcbox[nobeforeafter]{twelve}
 
When the property is set as |block| the elements are stacked below each other.
\medskip

\tcbox{chapter  display=block   CHAPTER}
\tcbox{number display=block    TWELVE}

The elements can be considered to be enclosed in a \emph{ghost} element. If the property is set to float we
\begin{figure}[htbp]
\makeatletter
\parindent0pt\fboxsep0pt
\fbox{\vbox to 0pt{\hbox to \dimexpr(\textwidth)\relax{{\hss\tcbox[capture=minipage,width=5cm, height=2cm, top=0pt]{\raggedright number display=block\\ number float=right }}%
}%
}%
}\par
\vspace*{2cm}
\makeatother
\end{figure}
signalling to the layout engine that the element must be placed to the right of the page, as shown in the figure. 


\begin{figure}[htbp]
\makeatletter
\parindent0pt\fboxsep0pt
\fbox{\vbox to 0pt{\hbox to \dimexpr(\textwidth+2cm)\relax{{\hss\tcbox[capture=minipage,width=5cm, height=2cm, top=0pt]{\raggedright number display=block\\ \emph{element} float=right }
\tcbox[capture=minipage,width=5cm, height=2cm, top=0pt]{\raggedright \emph{element} display=block\\ \emph{element} float=right }
}%
}%
}%
}\par
\vspace*{2cm}
\makeatother
\end{figure}

\subsection{Absolute positioning}

Absolute positioning mode, will place an element at an exact position on the page. They are more difficult to
achieve and inflexible. 

\begin{docKey}{position}{=\meta{absolute},\meta{relative}}{no default, initial none}{}

\end{docKey}



This positioning directive instructs the engine to position the element at an exact position.


\begin{docKey}[]{chapter float}{=\meta{left,center,right,none}}{no default, initial value \texttt{none}}
Key that controls the horizontal alignment of the \emph{chapter element}. In order for the
element to float, its |display| property must be set to |inline|.
\end{docKey}
\egroup



\section{Number Element Keys}


\subsection*{Keys for numbering}

Chapter numbering follows that of the standard \LaTeX\ classes and is extended to cover some additional cases such as fully spelled out numbers. This of course is only good for languages that use the arabic numeralsn. For other languages numerals in different formats can be added with simple keys and without the need of \pkgname{polyglossia} or \pkgname{babel}. 

Note that the package uses Heiko Oberdiek's package \pkgname{alphalph} to allow for alphabetic numbering that extends beyond the normal 26 letters of the alphabet. Examples for numbering can be seen in \ref{ex:romannumbering}


\begin{docKey}[phd]{number numbering}{= \oarg{alph,Alph,roman,Roman,none,WORDS,words,none}}{default arabic}
Style of numbering.
\end{docKey}

\begin{marglist}
\item [arabic] Despite that the Arabs call what the West calls Arabic numbers Indian numbers, we provide the value arabic to have normal numbers printed.
\item [alph] Lowercase alphabetic numbering.
\item [Alph] Uppercase alphabetic numbering.
\item [roman] Lowercase roman numbering.
\item [Roman] Uppercase roman numbering.
\item [words] The number is in lowercase words.
\item [WORDS] The number is in uppercase literal numerals.
\item [Words] Prints the number in words and capitalizes the first letter, for example the number 21 will be printed as `Twenty One'\footnote{Currently limited to the first hundred numbers}.
\index{chapter design>numbering>words}
\item [ordinals] Prints the number as ordinal.
\item [Ordinals] Prints the number as Ordinal.
\item [ORDINALS] Prinst the number as ORDINALS.
\item [none] This is equivalent to using the star version of the command. It does not print any number and does not increment the chapter counter.\footnote{I am ambivalent about this, perhaps it will be better to increment it, as it can give a more general approach.}

\end{marglist}
\begin{texexample}{Literal Numbering}{ex:literal}
\cxset{chapter numbering=WORDS} 
\chapter{Literal numbering}
\lorem
\cxset{chapter numbering=words,chapter name=chapter}
\chapter{Literal numbering} 
\lorem
\end{texexample}




\cxset{chapter opening=anywhere, chapter numbering=Roman, chapter number font-shape=upshape}
\index{chapter design>numbering>roman}

\begin{texexample}{Setting up keys for numbering}{ex:romannumberingx}
\bgroup
\cxset{chapter format = traditional, 
       chapter name = CHAPTER, 
       chapter numbering = Roman,
       chapter label color = bgsexy}
\chapter{Roman numbering}
\lorem
\egroup
\end{texexample}





To emulate some old books we also offer an ordinal numbering scheme.

\begin{texexample}{Literal Numbering}{ex:ordinals}
\cxset{chapter numbering=ORDINALS} 
\chapter{Ordinals numbering}
\lorem
\cxset{chapter numbering=words,chapter name=chapter}
\chapter{Literal numbering} 
\lorem
\end{texexample}

\cxset{chapter numbering=arabic}

\subsection{Fonts and colors}
\begin{docKey}[phd]{number color}{=\meta{color name}}{no default, initial value \texttt{black}}
This key sets the color for the \textit{number element}. The color name is stored in %\cmd{\numbercolor@cx}.
The value in this chapter is %\makeatletter\texttt{\numbercolor@cx}\makeatother.
\end{docKey}

\begin{docKey}[phd]{number font-size}{=\meta{Huge, Large}}{no default, initial value \texttt{Huge}}
This sets the size for rendering the \textit{number element}. Use one of the predefined values, as described
in the section for the \emph{chapter} element.
Note that you can either use a command i.e, |number font-size=|\cmd{\huge} 
or the command name i.e., |number font-size=huge|. The latter is the recommended method.
\end{docKey}

Letter spacing can be achieved using the soul package in a combination with the key |spaceout|.
The following examples illustrate the usage.

\index[phdkeys]{{\ttfamily phd/chapter design test}}

%\begin{texexample}{Letter Spacing}{ex:letterspacing}
%\cxset{numbering=Roman,
%        % number letter-spacing=soul,
%        % chapter spaceout=soul,
%         %title spaceout=soul,
%         title font-size=Large,
%         title font-family=rmfamily,
%         title font-shape=scshape}
%\chapter{Letter Spacing}
%
%\lorem
%\end{texexample}

\begin{docKey}[phd]{chapter number letter-spacing}{=\meta{none, true, etc.}}{no default, initial value \texttt{none}}.
\end{docKey}

\begin{marglist}
\item[none] Default value no tracking is used and the letters are spaced as per the basic font information.
\item[inherit] Inherits the letter-spacing settings from the \emph{chapter} element.
\item[true] Letter spacing is employed, using the |soul| package.
\item[false] Alias for |none|.
\item[soul] The \pkgname{soul} package is used for letter-spacing.
\item[microtype] The \pkgname{microtype} package is used for letter-spacing. When the microtype package is used more fine tuning of parameters is available.
\end{marglist}

The example that follows, explains how the features offered by the \pkgname{microtype} package can be used to
set different tracking options.

\begin{texexample}{Microtypography}{micro}
\bgroup

\SetTracking
 [ no ligatures = {f},
 spacing = {600*,-100*, },
 outer spacing = {450,250,150},
 outer kerning = {*,*} ]
 { encoding = * }
 { 100 }

{\huge \textls{Chapter Twenty}}

\SetTracking
 [ no ligatures = {f},
 spacing = {600*,-100*, },
 outer spacing = {450,250,150},
 outer kerning = {*,*} ]
 { encoding = * }
 { 200 }
 
{\huge \textls{Chapter Twenty}}

\egroup
\end{texexample}


\hbox{\drawfontbox{\huge \upshape\textls(Chapter Twenty}}

\hbox{\drawfontbox{\huge \upshape\textls{Chapter Twenty}}}


\section{Styling the chapter title}

Similarly to the number and chapter styling keys exist for styling the chapter title. We summarize the available standard keys below:

\index{chapter design!labels!letter spacing}
\begin{texexample}{Styling the Title}{ex:title} 
\cxset{chapter numbering=arabic, chapter title font-shape=itshape}
\chapter{Chapter title}
\lorem
\end{texexample}


\begin{docKey}[phd]{chapter title font-family}{=\marg{family}}{no default, initial inherit document font}
Selects a predefined font family
\end{docKey}

\begin{texexample}{Title element font styling}{}
\cxset{chapter title font-family=sffamily}
\chapter{Title font family settings}
\lorem
\cxset{chapter title font-shape=itshape}
\chapter{Title font-style settings}
\lorem
\end{texexample}


\begin{docKey}[phd]{chapter title font-weight}{ = \marg{\cs{bfseries},\cs{normalseries}}} {}
Font weight.
\end{docKey}

\begin{docKey}[phd]{chapter title font-size}{= \marg{large, Large, huge, Huge, HUGE, HHuge}}{}
Font sizing commands or their names. Both \docAuxCommand{\HUGE} and HUGE are allowed to be used as values for the key.
\end{docKey}

\begin{docKey}[phd]{chapter title color} { = \marg{color}} {}
The color of the chapter title letters. This takes any predefined color name. 
\end{docKey}


\begin{docKey}[phd]{chapter title spaceout}{ = \marg{soul,none}} {no default, initial = none}
 This key will space out the title. 
\end{docKey}

\begin{texexample}{Title element spacing}{}
\cxset{chapter name=none,
       chapter numbering=none,
       chapter title font-size=Large,
       chapter title color=black,
       chapter title width=0.6\textwidth,
       %title spaceout=soul,
         }
\chapter{The Prehistoric Period in South-East Asia: 2300 BC--AD 400}        
\lorem 
    
\end{texexample}
\cxset{defaults}


\subsection*{Adding content before and after the title element}

Like all the other elements, the title element can be decorated with additional content,
before and after the text. There are two different forms. 

\begin{docKey}[phd]{title before}{=\marg{code}}{default none}
Contents before the title (vertical material)
\end{docKey}

\begin{docKey}[phd]{title after}{=\marg{code}}{default none}
Contents after the title (vertical material)
\end{docKey}

\begin{docKey}[phd]{title content before}{=\marg{code}}{default none}
Contents before the title (horizontal material)
\end{docKey}

\begin{docKey}[phd]{title content after}{=\marg{code}}{default none}
Contents after the title (horizontal material)
\end{docKey}

The difference between the two type of settings, consider the following situation. Assume you have a title that has a rule at the top and bottom and the text is surrounded by two ornaments. The surrounding ornaments will be inserted using the |title before content|, and the rules using the |title before| form. The |title before| is a full fledged element on its own. 

%{
%\hrule
%\centering
%*** Introduction ***
%\par
%\hrule
%}
%
%{
%\MakePercentComment
%\startlineat{200}
%\lstinputlisting{./styles/style13.tex}
%\MakePercentIgnore
%}



 
\begin{docKey}{/phd/ chapter title before skip}{= \marg{soul,none}}{}
Before title string skip.
\end{docKey}

\begin{docKey}{/phd/ chapter title after skip}{ = \marg{soul,none} }{}
After title string skip.
\end{docKey}

\lorem 
%
%\begin{texexample}{letter spacing the chapter title block}{ex:title3}
%
%\cxset{chapter spaceout=none,
%         numbering=arabic}
%         
%\chapter{Chapter Title Styling}
%\end{texexample}
%
%\end{document}



\cxset{chapter opening=right}
\section{Table of Contents}\index{table of contents!key settings}

Traditionally a chapter will be added to the Table of Contents if the \cs{chapter} command is issued. The starred version will not produce a number and will not add a contents line. Since we have adopted an approach where we use a key value interface we can dispense with the starred version of the command, by setting the \option{chapter toc} option to false. For example if we want to define a command for a ``Foreward'' or ``Epiloque'' without wishing them to be added to the table of contents we can use the following setting.\index{Foreward>definitions}\index{Epilogue>definitions}



\begin{texexample}{changing the chapter label name}{}
\cxset{chapter name=Chapteris, chapter numbering=arabic,}
\chapter{Foreward}
\lorem
\end{texexample}

Note that the key \option{numbering=none} still has to be set.


Please note that when \textbf{numbering=none} the chapter number is not available anymore and yo may have to reset it if required again. Although this might be seen as rather cumbersome than simply using \cs{chapter*} the advantage is consistency in the user interface and the use of appropriate semantic definitions for all sectioning commands thus achieving a bit more separation of context from style.


%\cxset{chapter toc=true}

\section{Defining styles}

Named styles can be defined using the standard \textsc{PGF} conventions. To define a style for the forward above we can use:

\begin{texexample}{}{}
\cxset{foreward/.style={chapter numbering=none,
          chapter name=none,
          chapter title font-size= Large,
          chapter title font-family= sffamily,
          chapter numbering=none}}
\cxset{foreward}
\chapter{Foreward.}
\lorem
\end{texexample}



\cxset{chapter numbering=arabic}
\section{Creating semantic names for commands and environments}

To keep our search for semantic commands and true separation of contents it is prudent to define some macros for typesetting the  `foreward' section.

\bgroup
\begin{texexample}{defining a \textit{Foreward} macro.}{}
\begin{lstlisting}
\cxset{foreward/.style={chapter toc=false,
          name=none,
          title font-size = Large,
          title font-family = sffamily,
          numbering=none}}
\newcommand\forewardname{foreward}
\expandafter\newenvironment\expandafter{\forewardname}{%
\cxset{foreward}\chapter{Foreward}}%
{}
\begin{foreward}
\lorem
\end{foreward}
\end{lstlisting}
\end{texexample}
\egroup

Notice the use of a new command \cmd{\forewardname} to allow for internationlization using Babel or other methods. One is tempted to let the English name, but a better approach perhaps is to define both.

\makeatletter




%\makeatletter
\newenvironment{adjustmargins}[2]{%
 \begin{list}{}{%
 \topsep\z@%
 \listparindent\parindent%
 \parsep\parskip%
 \checkoddpage
 \ifoddpage % odd numbered page
 \@ifmtarg{#1}{\setlength{\leftmargin}{\z@}}%
 {\setlength{\leftmargin}{#1}}%
 \@ifmtarg{#2}{\setlength{\rightmargin}{\z@}}%
 {\setlength{\rightmargin}{#2}}%
 \else % even numbered page
 \@ifmtarg{#2}{\setlength{\leftmargin}{\z@}}%
 {\setlength{\leftmargin}{#2}}%
 \@ifmtarg{#1}{\setlength{\rightmargin}{\z@}}%
 {\setlength{\rightmargin}{#1}}%
\fi
}
\item[]}{\end{list}}

\makeatother


\chapter{Pages}

\parindent1em

The page is the main element in a book and its geometry and layout has been studied extensively by typographers. In this chapter we outline the typographical tradition, methods to specify layouts using \latex and associated issues, such as adjusting margins within a page.

Bringhurst notes that ``much typography is based, for the sake of economy on standard industrial sizes, from $35\times45$ inch press sheets to $3 1/2$ x 2 inch conventional business cards. Some formats as the booklets that accompany mobile telephone kits, are condemned to especially rigid restrictions of size.  

There may already be some restrictions on the page size you choose depending on your method of production and distribution. If you aim to output pages on a desktop printer then a standard size like A4 ($297\times210$)mm or US letter ($11\times 8 1/2$ inches) is advisable. If you have the opportunity and necessity of selecting the dimensions of the page you have a great opportunity to enhance the page layout of your book.

\section{Selecting  paper sizes}

Besides the limitations of the method of printing, another consideration is the size of book you writing and the
audience you are addressing it. If you are only producing a 60 page book, paper with smaller dimensions might be more appropriate than a blockbuster novel. 

History, natural science, geometry and mathematics are all relevant to typography.


\begin{figure}[ht]
\centering
\includegraphics[width=0.5\textwidth]{./images/preparing-paper.jpg}
\caption{Getting paper prepared for printing \protect\cite{moxon}.}
\end{figure}

Originally, paper sizes were determined by the moulds the paper was
made in and the use the result was put to. While many hundreds of variations have occurred throughout the centuries, in the main there have seldom been more than six categories of sizes in use since the fourteenth century. These have often come down to us bearing the names of the figures featured in the paper's watermarks, such as \emph{foolscap},
\emph{elephant}, \emph{pot}, and \emph{crown}. To enable the creation of smaller sizes from
existing larger sizes, the sheets have since the Middle Ages been proportioned
with their sides in the ratio of \(1:\sqrt{2}\). For example, quarto (4to,
formerly 4to) and octavo (8vo, formerly 8vo) sizes are obtained by cutting
or folding standard sizes four and eight times respectively.
Former British paper dimensions still used the old sizes before decimalized
versions replaced them; US dimensions still retain most of these
(untrimmed) paper sizes, in inches. Both are still encountered in specialist and bibliographic work, and in reproducing earlier or foreign formats:


\section{Canonical Layouts}

Typographers derive proportions that naturally occur in nature, and pages that embody
them recur in manuscripts and books from Rennaissance Europe, Tang and Song dynasty
China, early Egypt and ancient Rome.  
These numbers are $\pi=$3.14159\ldots , which is the circumference of a circle whose diameter
is one; $\sqrt{2}=$1.41421\ldots , which is the diagonal of a unit square; 
$e=2.71828$  \ldots ,which is the base of the natural logarithms; and $\phi=1.61803$ \ldots ,a number which is discussed later on. Certain of these proportions appear in he structure of the human body; other appear in musical scales. Indeed, one of the simplest of all systems of 
page proportions is based on the familiar intervals of the diatonic scale. Pages that
embody these basic musical proportions have been in common use in Europe for more than a thousand year.

 \begin{figure}
 \makebox[\textwidth]{\makebox[1.1\textwidth][r]{%
 \unitlength=0.0015\textwidth
 \let\ul\unitlength
	\begin{picture}(184,320)(0,-20)
	\put(0,0){\framebox(184,297){}}
	\put(27,70){\makebox(0,0)[bl]{\color{thegray}\rule{113\ul}{184\ul}}}
	\put(92,-20){\makebox(0,0)[t]{Golden number canonical layout}}
	\color{red}
	\put(92,162){\circle{184}}
	\linethickness{.2pt}
	\multiput(0,297)(1.98918918919,-3.21081081082){93}{\line(184,-297){1}}
	\end{picture}%
 \hfill
	\unitlength0.0015\textwidth
	\let\ul\unitlength
		\begin{picture}(210,330)(0,-20)
		\put(0,0){\framebox(210,297){}}
		\put(25,51){\makebox(0,0)[bl]{\color{thegray}\rule{149\ul}{210\ul}}}
		\put(105,-35){\makebox(0,0)[b]{ISO canonical layout}}
		\color{red}
		\put(105,156){\circle{210}}
		\multiput(0,297)(2.27027027027,-3.21081081082){93}{\line(210,-297){1}}
		\end{picture}%
 \hfill
   \unitlength0.001591\textwidth
   \let\ul=\unitlength
   \begin{picture}(220,330)(0,-20)
   		\put(0,0){\framebox(220,280){}}
   		\put(20.742,33.6){\makebox(0,0)[bl]{\color{thegray}\rule{172.86\ul}{220\ul}}}
   		\put(110,-35){\makebox(0,0)[b]{Letter paper canonical layout}}
   		\color{red}
   		\put(110,143.6){\circle{220}}
		\multiput(0,280)(2.37837837838,-3.02702702703){93}{\line(220,-280){1}}
   \end{picture}
 }}%
 \caption{A right page with the relevant diagonal, the text block and the canonical circle.
 In this figure the important information is the page proportions, not the scale; matter of
 fact the letter paper is 17.6 mm shorter than the A4 paper, but the drawings to the same
 height emphasize the relative proportions of the various page parts. \cite{canonicallayout}}
 \label{fig:canoniclayout}
 \end{figure}
 
The package \pkgname{xlayouts} and also Beccari’s \pkgname{canonical} layouts provide both graphical as well as settings for determining page layouts that approach canonical layouts. In reality modern book design has diverged from these principles. 

\begin{figure}[hb]
\cxset{spread xsteps=9,
          spread scale=0.20,
          spread width=0.5\textwidth}
\centering
\drawcanons
\end{figure}


\begin{figure}
  \includegraphics[width=0.5\linewidth]{./graphics/A-sizes.png}
  \caption{When a sheet whose proportions are $1$:$\surd{2}$ is folded in half, the result is a sheet half as large but with \emph{the same proportions}. Standard paper sizes on this principle have been in use in Germany since the early 1920s. The basis of this system is the A0 sheet, which has an are of 1 m$^2$. Yes because it is \textit{reciprocal with nothing but itself}, the ISO page in isolation is the least musical of all the major page shapes. It needs a textblock of another shape or contrast.}
   \label{fig:marginfig1}
\end{figure}

The advantages of basing a paper size upon an aspect ratio of $\surd{2}$ were already noted in 1786 by the German scientist Georg Christoph Lichtenberg, in a letter to Johann Beckmann[2]. The formats that became |A2|, |A3|, |B3|, |B4| and |B5| were developed in France, and published in 1798 during the French Revolution, but were subsequently forgotten. \cite{letimbre2136}

Early in the twentieth century, Dr Walter Porstmann turned Lichtenberg's idea into a proper system of different paper sizes. Porstmann's system was introduced as a DIN standard (DIN 476) in Germany in 1922, replacing a vast variety of other paper formats. Even today the paper sizes are called "DIN Ax" in everyday use in Germany.

The main advantage of this system is its scaling: if a sheet with an aspect ratio of $\surd{2}$ is divided into two equal halves parallel to its shortest sides, then the halves will again have an aspect ratio of $\surd{2}$. Folded brochures of any size can be made by using sheets of the next larger size, e.g. |A4| sheets are folded to make |A5| brochures. The system allows scaling without compromising the aspect ratio from one size to another – as provided by office photocopiers, e.g. enlarging |A4| to |A3| or reducing |A3| to |A4|. Similarly, two sheets of |A4| can be scaled down and fit exactly 1 sheet without any cutoff or margins.

%\cxset{try grid=false}
%\thispagestyle{grid}


The weight of each sheet is also easy to calculate given the basis weight in grams per square metre (g/m² or `'gsm"). Since an |A0| sheet has an area of 1m² , its weight in grams is the same as its basis weight in g/m². A standard |A4| sheet made from 80 g/m² paper weighs 5g, as it is one 16th (four halvings) of an A0 page. Thus the weight, and the associated postage rate, can be easily calculated by counting the number of sheets used.

Unlike the |A4| standard paper, the origin of the dimensions of letter size paper are lost in tradition. The American Forest and Paper Association argues that the dimension originates from the days of manual paper making, and that the 11-inch length of the page is about a quarter of ``the average maximum stretch of an experienced vatman's arms".[1] However, this does not explain the width or aspect ratio. What is known is that Ronald Reagan made this the paper size for U.S. federal forms; previously, the smaller "official" size (8 in × 10½ in or 203.2 mm × 266.7 mm) was used.[1] Letter or US Letter is the most common paper size for office use in the United States and Canada. It is 8$\frac{1}{2}$ by 11 inches (exactly 215.9 mm × 279.4 mm).

\section{The Typearea}

According to \cite{bringhurst2005}, in typography margins must do three things. They must lock the
textblock to the page and lock the facing pages to each other through the force of their proportions. Second, they must frame the textblock in a manner that suits its design. Third, they must protect the textblock, leaving it easy for the reader to see and convenient to handle. 

In most well designed books fifty per cent of the character and integrity of a printed page lies in its letterforms. Much of the other fifty per cent resides in its margins.


\subsection{Readability}

Another aspect that determines the text area, is the readability of the text. Here you need to take into account the readers of your book. For children and older persons a larger type and shorter lines are preferred.

\begin{macro}{\alphabetlength}
The macro |\alphabetlength| prints the length of the alphabet. The length of the alphabet in this text is \alphabetlength. If this is a good choice is debatable, but after all this is just a long document, with many chapters and my aim was to produce a reference and a test document. The macro is defined in the |xlayouts|  package, which is loaded automatically by the |phd| package or class. 
\end{macro}

Traditionally  a line that is approximately 1.4 times the alphabet length is considered good practice. The length of one line of text in this document is \the\textwidth giving a ratio of \alphabetsperline.

\DescribeMacro{\printreadability} prints a small table with some readability figures. If LuaTeX is used, this table is slightly longer and prints some other statistics as well. 

\begin{figure}[htbp]
\drawtriallayout
\bigskip

\printreadability
\captionof{figure}{Page layout diagram and readability statistics (using the \pkgname{xlayouts} package).}
\end{figure}

The macros described above are loaded by the |xlayouts| package, which forms part of the |phd| budle. There are macros for drawing trial layouts 


\section{Examples}
Folowing the nomenclature introduced b Bringhurst in analyzing the examples on the following pages, 
these symbols are used:

%% Align at the = sign 
\begin{table}[htbp]
\begin{tabular}{l l @{ = } p{6cm}}
\textit{Proportions:}      &P  &  page proportion $h/w$\\
~                      &T &  textblock proportion: $d/m$\\
\textit{Page size:}         &w &  width of page (trim-size)\\
~                      &h  & height of page (trim-size)\\
\textit{Textblock:}           &m & measure (width of primary textblock)\\
~                      &d  & depth of primary textblock (excluding running heads, folios etc)\\                      
~                      &$\lambda$ & line height (type size plus added lead)\\
~                      &$n$ & secondary measure (width of secondary column)\\
~                      &$c$  & column width, where there are even multiple columns\\
\textit{Margins}  &$s$  & spine margin (back margin)\\
~                      &$t$   & top margin (header margin)\\
                        &$e$  & fore-edge (front margin)\\
                        &$f$   & foot margin\\
                        &$g$  & internal gutter (on a multiple-column page)\\
\end{tabular}
\caption{Symbols used to demonstrate various ratios in books}
\end{table}
\medskip

\begin{figure}
  \includegraphics[width=\linewidth]{./graphics/page.png}
  \caption{Page nomenclature}
   \label{fig:marginfig1}
\end{figure}

More variables are necessary to specify all the variables handled by a \latex\
page. For the time being the examples refer to dimensions from historical works
in typography and should sufffice.

\subsection{Hypneroto}

\begin{figure}[htbp]
\centering
  \includegraphics[width=\linewidth]{./graphics/hypneroto.jpg}
\caption{The work is lauded for the originality of its
design. Several sequential double page
illustrations add a visual dimension to the
progression of the narrative, and act like an
early form of the strip cartoon. There is an
obsession with movement throughout which is driven
on by the illustrations, resulting in the
impression of bodies moving from one page to the
next. Other typographical innovations include
playing with the traditional layout of the text;
in the opening shown here, for example, the pages
are shaped in the form of goblets. The dimensions
of the text are: $P=1.5[2:3]$; T=1.7 (tall pentagon);
Margins: s=t=w/9; e=2 s. The text is a fantasy
novel, Francesco Colonna's Hypnerotomachia
Poliphili, set in a roman font cut by Francesco
Griffo. (Aldus Manutius, Venice, 1499). Original
size: $20.5\times31$\thinspace cm.}
\label{fig:hypneroto}
\end{figure}


%  \label{fig:layout}



The book was printed by Aldus Manutius in Venice in December 1499. The book is anonymous, but an acrostic formed by the first, elaborately decorated letter in each chapter in the original Italian reads \textsc{\small POLIAM FRATER FRANCISCVS COLVMNA PERAMAVIT}, \enquote{Brother Francesco Colonna has dearly loved Polia.} However, the book has also been attributed to Leon Battista Alberti by several scholars, and earlier, to Lorenzo de Medici. The latest contribution in this respect was the attribution to Aldus Manutius, and arguably, a Francesco Colonna, a wealthy Roman Governor. The author of the illustrations is even less certain, but contemporary opinion gives the work to Benedetto Bordone.

\section{Contemporary book layouts}

All these sound mystical with religious undertones, but we need to remember that early printers made their livelihoods from printing mostly religious books.

From the mystics to the modern, let us study Figure~\ref{fig:nudes}

\begin{figure}[htbp]
\centering

\fbox{\includegraphics[width=\linewidth]{nudes.jpg}}

\caption{In this layout, the placement of various size images on the right pages, makes the margins disappear to the eye. As the whole book, is made of similar pages\ldots }
\end{figure}

Modern designers are more cryptic. One book that I found more useful is Ambrose/Harris \textit{Layout}. The book brings together examples of layout, both contemporary  and historic, from aroudnd the world. It contains examples from leading graphic designers to provide a sample of rich and diverse possibilities for the creative use of layout.

As it will become apparent from what follows, although at first look it appears that all design principles have disappeared into post modern designs, all design is undertaken with reference to a certain set of principles, either by consciously
choosing to follow or by deliberately ignoring or subverting them. The collective body of principles represents different approaches to design and layout construction.

The principles in this section have been used
through the ages to produce designs that are
pleasing to the eye and that organise information
clearly and efficiently, two of the challenges facing
every graphic designer. These principles affect
decisions made at the heart of the design process,
as they provide the basis of how space is divided.

\section{A design must capture the spirit of the times.}

The word \emph{zeitgeist} originates from the German zeit (time) and geist (spirit),
and so literally means spirit of the age. In graphic design, each decade can
be defined by several predominant zeitgeists that somehow seem to capture
their essence. Today, in graphic design, we can see a zeitgeist for the use of
sophisticated computer graphics giving a very close approximation to reality
in addition to another, which is a backlash to this, in the form of rough-and-ready
hand-drawn designs.

\section{Objects on a Page}

How an object is placed on a page has a dramatic
impact on how it is received and interpreted by
the viewer, and the message that it delivers. We
have looked at how grids can be used to guide
element placement on a page, but maintaining a
sense of order is not the only consideration when
laying out a design.

Object placement helps form the narrative of
a design and is constructed from an understanding
of how we read a page. The narrative of a design
can be created and altered by a wide range of
placement and intervention strategies, such as
how white space is used, the balance and relative
weight given to other objects, the juxtaposition or
contrast of objects and so on.

This chapter will outline some of the
fundamental approaches to object placement.

\section{White Space}

White space is not necessarily white, as it refers to any space in the design
without text or graphic elements. Designers naturally insert white space into
their designs to help the composition and make the information the design
contains easier to access, such as leaving margins at the sides of the page that
create space around text blocks. Swiss typographer Jan Tschichold called white
space ‘the lungs of a good design’. Without white space, with every part of the
design area filled, there is a danger that a design would look cramped and
difficult to understand.

White space can instil different perceptions in a viewer depending on how it is
used and the content it is associated with. White space may give the impression
of luxury and extravagance for a full-page photograph. However, it may also give
the impression that there are gaps in a layout that is rather full, or worse, that
there is insufficient content to fill a page. Newspapers try to establish a rational
balance between giving space to page elements to meet the conflicting demands
of the need for typographical sensitivity and readability, while filling a page with
news so that the reader feels they are getting value for money. Habitually readers
expect a newspaper to be ‘full’, which means it is harder to achieve typographic
balance. In contrast, where filling space is of less concern, such as the example
below, white space becomes a more overt part of the design.



\section{Grids}

\subsection{The Baseline Grid}

The baseline grid is the (invisble) graphic foundation upon which a design is constructed and provides a visual guide for positioning and ligning page elements with an accuracy that is difficult to achieve by eye alone. Knuth's TeX focuses almost primarily on getting this one right.

\section{Pace}

It came to me as a big surprise that a books layout must have \textit{pace}. This essentially is the alternation of pages, between say images and text.

\begin{figure}[htbp]
\parindent=0pt
\includegraphics[width=\textwidth]{pace}

\end{figure}

Thumbnails are smaller versions of the spreads of a publication presented on a
page that allow a designer to gauge its pace and balance at the macro level
without focusing on details. Thumbnails allow a designer to look at the
narrative of the publication and tune it as a whole, rather than on a spread-byspread
basis.

Pictured are thumbnails for Miss X, a book for underwear retailer Agent
Provocateur art directed by Mike Figgis and published by Anova, with design by
Gavin Ambrose. The absence of folios and minimal text mean the image flow
takes prominence.

The images can let us set a method for defining such spreads. 


\chapter{Temporarily changing the text width}

\index{pagewidth>change temporarily}


Margins in a page can be changed temporarily by adjusting, the lengths of \cmd{\leftskip} and \cmd{\rightskip}. The |memoir| class provides an environment |adjustwidth| see page 422 (based on This code is based on the \pkgname{chngpage} package.) for doing so and the \class{tufte-book} class provides an environment \textit{fullwidth}. The following code is an adaptation of that found in the \class{memoir} class.


\begin{teXX}
\begin{adjustmargins}{left}{right} 
\end{teXX}


adds the given lengths to the left and
right hand margins. A positive value will shorten the text and a negative value
will widen it. The starred version of the environment will cause the margin changes to switch between odd and even pages. 



\eject
\newgeometry{left=10mm,right=10mm,bottom=1.5cm,top=1cm}

\section*{The \texttt{adjustmargins} environment}
\lorem

\vfill\vfill
\begin{multicols}{2}
\lorem
\end{multicols}

\begin{adjustmargins}{0cm}{0in}
{\leftskip1em\rule{13cm}{.4pt}\par}

\centering



\parbox{\textwidth}{{\leftskip1em\rightskip1em There are no engineers in the hottest parts of hell, because the existence of a 'hottest part' implies a temperature difference, and any marginally competent engineer would immediately use this to run a heat engine and make some other part of hell comfortably cool.  This is obviously impossible.\par}
}
\par
\medskip
\par
\noindent\includegraphics[width=0.9\textwidth]{./graphics/lilian.jpg}\par


\end{adjustmargins}

\clearpage

\restoregeometry


\lipsum[1]


\begin{adjustmargins}{-0.4\textwidth}{0.1\textwidth}
\fboxsep2pt%
\fbox{\includegraphics[width=1.2\textwidth]{./graphics/leoncroll.jpg}}
\end{adjustmargins}

\lipsum[2]

\begin{teX}
\begingroup
\makeatletter
 \catcode`\Q=3
 \long\gdef\@ifmtarg#1{\@xifmtarg#1QQ\@secondoftwo\@firstoftwo\@nil}
 \long\gdef\@xifmtarg#1#2Q#3#4#5\@nil{#4}
 \long\gdef\@ifnotmtarg#1{\@xifmtarg#1QQ\@firstofone\@gobble\@nil}
 \endgroup


\newenvironment{adjustmargins}[2]{%
  \begin{list}{}{%
    \topsep\z@%
    \listparindent\parindent%
    \parsep\parskip%
   \@ifmtarg{#1}{\setlength{\leftmargin}{\z@}}%
   {\setlength{\leftmargin}{#1}}%
   \@ifmtarg{#2}{\setlength{\rightmargin}{\z@}}%
   {\setlength{\rightmargin}{#2}}%
}
\item[]}{\end{list}}
\makeatother
\end{teX}

 
\section{Setting Dimensions in \latex}

To set dimensions for page layout in \latex is not straightforward. You need to adjust several \latex
native dimensions to place a text area where you want. If you want to center the text area in the paper
you use, for example, you have to specify native dimensions as follows:

\begin{verbatim}
\usepackage{calc}
\setlength\textwidth{7in}
\setlength\textheight{10in}
\setlength\oddsidemargin{(\paperwidth-\textwidth)/2 - 1in}
\setlength\topmargin{(\paperheight-\textheight
-\headheight-\headsep-\footskip)/2 - 1in}.
\end{verbatim}

Without package |calc|, the above example would need more tedious settings. To adjust all parameters from scratch one should have a good understanding, of \latexe's definitions of all parameters. The companion package |xlayouts| can be used to display these parameters on an actual printed page. All settings are parameterized and I find the use of colours assists in viewing the rulers better.


\subsection{The Geometry package}

The package \pkg{geometry} \cite{geometry} provides
an easy way to set page layout parameters. In this case, what you have to do is just load the package and set
the page geometry using keys.

\begin{teX}
\usepackage[text={7in,10in},centering]{geometry}.
\end{teX}

Besides centering problem, setting margins from each edge of the paper is also troublesome. But geometry
also make it easy. If you want to set each margin to 1.5in, you can type

\begin{comment}
\label{sec:geometry}

\def\OpenB{{\ttfamily\char`\{}}
 \def\Comma{{\ttfamily\char`,}}
 \def\CloseB{{\ttfamily\char`\}}}
 \def\Gm{\textsf{geometry}}
\newcommand\gpart[1]{\textsf{\textsl{\color[rgb]{.0,.45,.7}#1}}}%

\newcommand\glen[1]{\textsf{#1}}

\bgroup
\makeatletter
 \begin{figure}
  \small
  \unitlength=.65pt
  \begin{picture}(450,250)(0,-10)
  \put(20,0){\framebox(170,230){}}
  \put(20,235){\makebox(170,230)[br]{\gpart{paper}}}
  \begingroup\thicklines
  \put(40,30){\framebox(120,170){}}
  \put(40,30){\makebox(120,165)[tr]{\gpart{total body}~}}
  \put(45,30){\makebox(0,170)[l]{|height|}}
  \put(40,35){\makebox(120,0)[bc]{|width|}}
  \put(50,-20){\makebox(120,0)[bc]{|paperwidth|}}
  \put(10,45){\makebox(0,170)[r]{|paperheight|}}
  \put(90,200){\makebox(0,30)[lc]{|top|}}
  \put(90,0){\makebox(0,30)[lc]{|bottom|}}
  \put(10,70){\makebox(0,0)[r]{|left|}}
  \put(10,55){\makebox(0,0)[r]{(|inner|)}}
  \put(200,70){\makebox(0,0)[l]{|right|}}
  \put(200,55){\makebox(0,0)[l]{(|outer|)}}
  \put(80,230){\vector(0,-1){30}}\put(80,30){\vector(0,-1){30}}
  \put(80,200){\vector(0,1){30}}\put(80,0){\vector(0,1){30}}
  \put(20,70){\vector(1,0){20}}\put(40,70){\vector(-1,0){20}}
  \put(160,70){\vector(1,0){30}}\put(190,70){\vector(-1,0){30}}
  \multiput(160,30)(5,0){24}{\line(1,0){2}}
  \multiput(160,200)(5,0){24}{\line(1,0){2}}
  \begingroup\thicklines
  \put(280,30){\framebox(120,170){}}\endgroup
  \put(283,133){\makebox(0,12)[l]{|textheight|}}
  \put(295,130){\vector(0,-1){100}}\put(295,150){\vector(0,1){50}}
  \multiput(280,220)(5,0){24}{\line(1,0){3}}
  \put(280,208){\makebox(120,20)[bc]{\gpart{head}}}
  \multiput(280,207)(5,0){24}{\line(1,0){3}}
  \put(420,225){\makebox(0,0)[l]{|headheight|}}
  \put(418,225){\line(-2,-1){20}}
  \put(420,213){\makebox(0,0)[l]{|headsep|}}
  \put(418,213){\line(-2,-1){20}}
  \put(420,12){\makebox(0,0)[l]{|footskip|}}
  \put(418,12){\line(-2,1){20}}
  \put(280,40){\makebox(120,140)[c]{\gpart{body}}}
  \put(305,45){\vector(-1,0){25}}\put(375,45){\vector(1,0){25}}
  \put(80,230){\vector(0,-1){30}}\put(80,30){\vector(0,-1){30}}
  \put(280,48){\makebox(120,0)[c]{|textwidth|}}
  \put(280,15){\makebox(120,10)[c]{\gpart{foot}}}
  \multiput(280,14)(5,0){24}{\line(1,0){2}}
  \put(410,30){\dashbox{3}(30,170){}}
  \put(415,30){\makebox(30,170)[l]{\gpart{marginal note}}}
  \put(425,45){\vector(-1,0){15}}\put(425,45){\vector(1,0){15}}
  \put(450,70){\makebox(0,0)[l]{|marginparsep|}}
  \put(448,70){\line(-3,-1){43}}
  \put(450,45){\makebox(0,0)[l]{|marginparwidth|}}
  \end{picture}
\caption{Dimension names used in the geometry package. width $=$ textwidth and height $=$ textheight by default. left, right, top and bottom are margins. If margins on verso pages are swapped by twoside option, margins specified by left and right options are used for the inside and outside margins respectively. inner and outer are aliases of left and right
respectively.}
\label{fig:geometrylayout}
\end{figure}
\makeatother
\egroup
\end{comment}

 The \pkg{geometry} package provides a flexible and easy interface to page dimensions.
 You can change the page layout with intuitive parameters. For instance,
 if you want to set a margin to 2cm from each edge of the paper,
 you can type just |\usepackage[margin=2cm]{geometry}|. 
 The page layout can be changed in the middle of the document
 with \cs{newgeometry} command.  The \ref{fig:geometrylayout}, reproduced from the package documentation, illustrates the variety of parameters that can be set using the package.


\section{Footnotes}
The history of footnotes is as long and complicated as the history of scholarship and commentary. Hebrew scholars more than two thousand years ago used systems of glossing and annotation to work on religious texts. 

Scribes in the Christian tradition in the medieval period made use of annotations in their manuscript copying practices: surrounding the original text with glosses in small letters. After the advent of printing, similar kinds of marginal annotation appeared in printed texts of the late fifteenth century. 

Humanist scholars producing printed editions of classical learning in the sixteenth century also made use of the resources of typography to display both the surviving classical text and their commentary on the same page. References to classical sources - and to modern printed editions - became more systematic, as did the expectation that such references would be consistent with scholarly practices. Scholars increasingly marked their professionalism by using complex citational conventions, which by the seventeenth century were so well established as to be the subject of parody and satire. Scriblerian satire of the early eighteenth century, whose purpose was to mock the pedantry and folly of the works of the learned, frequently included extensive parodies of footnotes and the scholarly contests they encoded. Nonetheless, during the eighteenth century, to appear authoritative and learned an author had to adopt the scholarly machinery of the reference citation.

The footnote was born out of a desire to rationalise the relation between text and citation. 

Robert Connors argues that marginal notations fell out of favour for two practical reasons: they left too much blank paper at the side of the text; and they were difficult for typographers to set. The same notes placed at the bottom of the page were more efficient, both in paper and time[1]. 

Anthony Grafton's The Footnote: A Curious History suggests the modern footnote, inaugurated by Pierre Bayle's Dictionaire Critique et Historique in 1697, signalled an epistemic revolution in historical scholarship, indicating the end of credulous scholasticism and the emergence of analytical historical methodologies. Both scholars note the considerable impact of historians such as David Hume and Edward Gibbon on the stylistic development of the discursive and citational footnote as a location for the display of gentlemanly ease as much as scholarly acumen. In the nineteenth century, German scholars such as Leopold von Ranke and Alexander von Humboldt established a systematic basis for the footnote citation, creating a methodical methodological approach that all competing scholars had to obey. In this way, the idea of the footnote was established, yet no there was no general agreement on the form these footnotes should adopt. A systematic approach to the form of the footnote was needed.

In this section we will discuss how lines and paragraphs are turned into pages and how elements of pages such as footnotes, headers etc are inserted. As with the other chapters we will mix TeX basic commands with the more convenient \LaTeXe\ commnads. We will also look at some of the packages and classes that are availble to assist us with page layouts. 


Besides illustrations that are inserted at the top of a page, plain TEX will also
insert footnotes at the bottom of a page. The ootnote macro is provided
for use within paragraphs;  for example, the footnote in the present sentence was typed
in the following way:


There are two parameters to a footnote[ first comes the reference mark, which will
appear both in the paragraph** and in the footnote itself, and then comes the text of
the footnote.45 The latter text may be several paragraphs long, and it may contain
\footnote{Sidenote: ``Where God meant footnotes to go.'' ---Tufte}

\marginpar{

The history of footnotes is as long and complicated as the history of scholarship and commentary. Hebrew scholars more than two thousand years ago used systems of glossing and annotation to work on religious texts. Scribes in the Christian tradition in the medieval period made use of annotations in their manuscript copying practices: surrounding the original text with glosses in small letters. After the advent of printing, similar kinds of marginal annotation appeared in printed texts of the late fifteenth century. Humanist scholars producing printed editions of classical learning in the sixteenth century also made use of the resources of typography to display both the surviving classical text and their commentary on the same page. References to classical sources - and to modern printed editions - became more systematic, as did the expectation that such references would be consistent with scholarly practices. Scholars increasingly marked their professionalism by using complex citational conventions, which by the seventeenth century were so well established as to be the subject of parody and satire. Scriblerian satire of the early eighteenth century, whose purpose was to mock the pedantry and folly of the works of the learned, frequently included extensive parodies of footnotes and the scholarly contests they encoded. Nonetheless, during the eighteenth century, to appear authoritative and learned an author had to adopt the scholarly machinery of the reference citation.

The footnote was born out of a desire to rationalise the relation between text and citation. Robert Connors argues that marginal notations fell out of favour for two practical reasons: they left too much blank paper at the side of the text; and they were difficult for typographers to set. The same notes placed at the bottom of the page were more efficient, both in paper and time[1]. Anthony Grafton's The Footnote: A Curious History suggests the modern footnote, inaugurated by Pierre Bayle's Dictionaire Critique et Historique in 1697, signalled an epistemic revolution in historical scholarship, indicating the end of credulous scholasticism and the emergence of analytical historical methodologies. Both scholars note the considerable impact of historians such as David Hume and Edward Gibbon on the stylistic development of the discursive and citational footnote as a location for the display of gentlemanly ease as much as scholarly acumen. In the nineteenth century, German scholars such as Leopold von Ranke and Alexander von Humboldt established a systematic basis for the footnote citation, creating a methodical methodological approach that all competing scholars had to obey. In this way, the idea of the footnote was established, yet no there was no general agreement on the form these footnotes should adopt. A systematic approach to the form of the footnote was needed.}

Further reading:

Connors, Robert J., 'The Rhetoric of Citation Systems, Part I: The Development of Annotation Structures from the Renaissance to 1900', Rhetoric Review, 17 (1998), 6-48.

Connors, Robert J., 'The Rhetoric of Citation Systems, Part II: Competing Epistemic Values in Citation', Rhetoric Review, 17 (1999), 219-245.

Grafton, Anthony, The Footnote: A Curious History (London: Faber and Faber, 1997)

Grafton, Anthony, 'The Footnote from De Thou to Ranke', History and Theory, 33 (1994), 53-76

Zerby, Chuck, The Devil's Details: A History of Footnotes (Lancaster: Gazelle, 2002)

[1] Robert J. Connors, The Rhetoric of Citation Systems, Part I: The Development of Annotation Structures from the Renaissance to 1900, Rhetoric Review, 17 (1998), 6-48 (p. 30).
 
%% Note with new geometry paper has to be defined in preamble
% I do not feel very confident of this
% Don't understand it fully how is working

\cxset{geometry oxford/.code={
\newgeometry{a4paper,left=74.8mm,top=27.4mm,headsep=2\baselineskip,%
marginparsep=8.2mm,marginparwidth=49.4mm,textheight=49\baselineskip,headheight=\baselineskip}
\@twosidefalse \@mparswitchfalse % one side option
\reversemarginpar
}}

\cxset{geometry textwidth/.store in=\textwidth@cx,
          geometry textheight/.store in=\textheight@cx,
          geometry tufte/.code={
             \newgeometry{a4paper,left=24.8mm,top=27.4mm,headsep=2\baselineskip,%
             textwidth=107mm,marginparsep=8.2mm,marginparwidth=49.4mm,%
             textheight=\textheight@cx\baselineskip,headheight=\baselineskip}
            \@twosidefalse \@mparswitchfalse % one side option
           %\reversemarginpar
    }
}

\cxset{marginpar push/.store in=\marginparpush@cx,
          marginpar font/.store in=\marginparfont@cx,
          marginpar justification/.is choice,
          marginpar justification/justifying/.code=\gdef\marginparjustification@cx{\justifying},
          marginpar justification/raggedright/.code=\gdef\marginparjustification@cx{\raggedright},
          marginpar justification/RaggedRight/.code=\gdef\marginparjustification@cx{\RaggedRight},
          marginpar justification/RaggedLeft/.code=\gdef\marginparjustification@cx{\RaggedLeft},
 }
\cxset{marginpar push=10pt,
          marginpar font=\normalfont\footnotesize\sffamily,
          marginpar justification=RaggedLeft}

\cxset{style13, geometry textheight=49,
          geometry oxford,
          watermark text=SAMPLE TUFTE VARIANT,
          watermark text color=thered,
          header style=samplepage}

% This is a sidenote without the footnote mark
\newcommand\marginnote[2][0pt]{%
 % \let\cite\@tufte@infootnote@cite%   use the in-sidenote \cite command
  %\gdef\@tufte@citations{}%           clear out any old citations
  \@tufte@margin@par%                 use parindent and parskip settings for marginal text
  \marginpar{\hbox{}\vspace*{#1}\marginparfont@cx\marginparjustification@cx\vspace*{-1\baselineskip}\noindent #2}%
  \@tufte@reset@par%                  use parindent and parskip settings for body text
  %\@tufte@print@citations%            print any citations
  %\let\cite\@tufte@normal@cite%       go back to using normal in-text \cite command
}

\chapter{Geometry and Page Dimensions}

\lipsum[1-4]\marginnote[1pt]{\lorem
    \lorem}

\lipsum[1-2]

%% Stick the caption in the head might as well place the first picture also
\def\asidecaption{\parbox{4.2cm}{{\bfseries Image \thefigure}\par\lorem}%
  % \addtocontents{lof}{This is image 8}
}
\def\ps@caption{%
     \let\@oddfoot\@empty\let\@evenfoot\@empty%
    \def\@evenhead{%
        \begin{picture}(0,0)%
           \put(-150,-80){\asidecaption\par}%
            \stepcounter{figure}
           \put(-150,-370){\asidecaption}%
        \end{picture}%
      }%
    \let\@oddhead\@evenhead%
    \let\@mkboth\@gobbletwo%
    \let\chaptermark\@gobble%
    \let\sectionmark\@gobble%
 }

\def\ps@bigpicture{%
    \setlength\headheight{19cm}%
    \let\@oddfoot\@empty\let\@evenfoot\@empty%
    \def\@evenhead{%
         \begin{picture}(0,0)%
          \put(-149,0){\includegraphics[width=\dimexpr(\textwidth+150pt)]{stuartpearson}}%
         \end{picture}%
      }%
    \let\@oddhead\@evenhead%
    \let\@mkboth\@gobbletwo%
    \let\chaptermark\@gobble%
    \let\sectionmark\@gobble%
 }


\def\doubletakeimage{%
  \renewcommand{\topfraction}{.95}  % ensure seecond image will not float away
  \begin{figure}[t]
    \thispagestyle{caption}
    \includegraphics[width=\textwidth]{matron}%
  \end{figure}

  \begin{figure}[tp]
   \hspace*{-\marginparwidth}\includegraphics[height=0.9\textheight]{stuartpearson}
 \end{figure}
}


\doubletakeimage
\lipsum[1-4]

\restoregeometry
%% RESET EVERYTHING AT END OF CHAPTER
\addtocounter{chapter}{-2}
\@toctrue\@specialtrue

     
%       
%\def\storyi{The best graphics package ever developed is the TikZ package. 
Its parent package is PGF which is short of a miracle that has been programmed
using \tex, a more than thirty years old program. This has taken over almost all other
packages and is very popular with newcomers to \latex. It is frustrating at first, but once 
you over the basic ideas and concepts it opens infinite possibilities for typesetting
great articles and books.}



\cxset{chapter format=stewart,
       texti=\storyi,
       textii=\storyi}

\newcommand\seepgfmanual[1]{%
    \textit{see} the PGFmanual page #1}%
    
%\cxset{chapter format = traditional}    
\chapter{TikZ}

\section{The \protect\texttt{TikZ} package}
\pkg{TikZ}, a high-level interface to \pkg{PGF}, is a language-based tool for specifying graphics.
It uses familiar graphics-related concepts, such as point, line, and circle and
has a concise and natural syntax. It meshes well with pdfLATEX in the sense that
no additional processing steps are needed. Another positive aspect of \pkg{TikZ} is
its ability to blend \tex fonts, symbols, and mathematics within the generated
graphics.


All the TikZ commands can be used inline using \docAuxCommand{tikz} or within the \docAuxCommand{tikzpicture} environment. When we want to use captions and labels, we enclose it in the figure environment or use \docAuxCommand{captionof}, but it can be called anywhere in the text or math of a Tex document:

\begin{teXXX}
\begin{figure}
\centering
%\tikzset{external/force remake}
\begin{tikzpicture}
... TikZ commands ...
\end{tikzpicture}
\caption{A diagram drawn with TikZ.}
\label{Fig:_diagram1}
\end{figure}
\end{teXXX}

We can also use them in math:

\begin{teXXX}
\begin{align*}
\int dx\; f(x) =
\alpha
%\tikzset{external/force remake}
\begin{tikzpicture}
... TikZ commands ...
\end{tikzpicture}
\end{align*}
\end{teXXX}



\section{Draw simple lines}

\begin{texexample}{Draw a Line}{ex:line}
\begin{tikzpicture}
\node[draw] (S1) at (0,0) {Paris};
\node[draw] (S2) at (3,0) {Stratsbourg};
\draw (S1) -- (S2);
\end{tikzpicture}
\end{texexample}


The syntax of the command is:

|\node|\oarg{options} (\meta{name}) at (\meta{position}) |{|\meta{contents}|}|

If we look
 carefully, we see that the two writings give
Slightly different results:
- In the first case, node is an operation executed on a path. We
Can consider each node as a decoration of the point at which it
is associated. The line drawn by the draw command joins two points, the
Nodes are objects added later and centered on points. The option
Draw of the node trace operation the node outline.
- In the second case, \ node is a TikZ command which allows to define
A node, to name it and to draw it. One can then consider the
Nodes as pre-existing objects that will then be linked with the \docAuxCommand{node}.


\begin{texexample}{Draw a Line}{ex:line}
\begin{tikzpicture}
\node[draw] (S1) at (0,0) {Paris};
\node[draw] (S2) at (0,3) {Stratsbourg};
\draw[->] (S1) -- (S2);
\end{tikzpicture}
\end{texexample}

The basic building block of all pictures in \tikzname is the path. A path is a series of straight lines and curves
that are connected (that is not the whole picture, but let us ignore the complications for the moment). You
start a path by specifying the coordinates of the start position as a point in round brackets, as in (0,0).
This is followed by a series of \enquote{path extension operations.}


\begin{texexample}{Draw a Line}{ex:line}
\begin{tikzpicture}
\draw[->] (0,0) -- (1.5,0) -- (0, 1.2);
\end{tikzpicture}
\end{texexample}


\subsection*{Adding Text} 

So far we have seen how to draw lines and arcs. However, an important component is still missing the addition of text. When
\tikzname is constructing a path and it encounters the keyword |node| typically followed by some options  it reads a \textit{node specification}. Options can typically follow and then it terminates by curly brackets. 
 

\begin{texexample}{Draw a Line}{ex:line}
\begin{tikzpicture}
\draw[->] (0,0) -- (1.5,0) node {First Node} -- (0, 1.2) node[shape = circle] {Second Node};
\end{tikzpicture}
\end{texexample}


The \docAuxCommand*{node} can be used to abbreviate the operation. A longer example can demonstrate this better. How can we draw the following figure?

\begin{tikzpicture}
\node[circle,fill=black,inner sep=0.8pt,draw] (a) at (0,0) {};
\node[circle,fill=black,inner sep=0.8pt,draw] (b) at (1.5,0) {};
\node[circle,fill=black,inner sep=1.5pt,draw] (c) at (.75,2) {};
\node[circle,fill=black,inner sep=0.8pt,draw] (d) at (0.75,.75) {};
\node[circle,fill=black,inner sep=0.8pt,draw] (e) at (2,1) {};


\node () at (-0.3,0) {\tiny$1$};
\node () at (0.75,0.45) {\tiny$2$};
\node () at (0.75,2.3) {\tiny$4$};
\node () at (2,1.3) {\tiny$-1$};
\node () at (1.8,0) {\tiny$-1$};

\draw (a)--(b)--(e)--(c) --(a)--(d)--(b)--(c);
\draw (c)--(d);

\node at (3,1) {\Large{$\sim$}};

\begin{scope}[shift={(+4,0)}]
\node[circle,fill=black,inner sep=0.8pt,draw] (a) at (0,0) {};
\node[circle,fill=black,inner sep=0.8pt,draw] (b) at (1.5,0) {};
\node[circle,fill=black,inner sep=0.8pt,draw] (c) at (.75,2) {};
\node[circle,fill=black,inner sep=0.8pt,draw] (d) at (0.75,.75) {};
\node[circle,fill=black,inner sep=0.8pt,draw] (e) at (2,1) {};


\node () at (-0.3,0) {\tiny$2$};
\node () at (0.75,0.45) {\tiny$3$};
\node () at (0.75,2.3) {\tiny$0$};
\node () at (2,1.3) {\tiny$0$};
\node () at (1.8,0) {\tiny$0$};

\draw (a)--(b)--(e)--(c) --(a)--(d)--(b)--(c);
\draw (c)--(d);

\end{scope}
\end{tikzpicture}

\begin{texexample}{A larger example}{ex:larger}
\begin{tikzpicture}
\node[circle,fill=black,inner sep=0.8pt,draw] (a) at (0,0) {};
\node[circle,fill=black,inner sep=0.8pt,draw] (b) at (1.5,0) {};
\node[circle,fill=black,inner sep=1.5pt,draw] (c) at (.75,2) {};
\node[circle,fill=black,inner sep=0.8pt,draw] (d) at (0.75,.75) {};
\node[circle,fill=black,inner sep=0.8pt,draw] (e) at (2,1) {};


\node () at (-0.3,0) {\tiny$1$};
\node () at (0.75,0.45) {\tiny$2$};
\node () at (0.75,2.3) {\tiny$4$};
\node () at (2,1.3) {\tiny$-1$};
\node () at (1.8,0) {\tiny$-1$};

\draw (a)--(b)--(e)--(c) --(a)--(d)--(b)--(c);
\draw (c)--(d);

\node at (3,1) {\Large{$\sim$}};

\begin{scope}[shift={(+4,0)}]
\node[circle,fill=black,inner sep=0.8pt,draw] (a) at (0,0) {};
\node[circle,fill=black,inner sep=0.8pt,draw] (b) at (1.5,0) {};
\node[circle,fill=black,inner sep=0.8pt,draw] (c) at (.75,2) {};
\node[circle,fill=black,inner sep=0.8pt,draw] (d) at (0.75,.75) {};
\node[circle,fill=black,inner sep=0.8pt,draw] (e) at (2,1) {};


\node () at (-0.3,0) {\tiny$2$};
\node () at (0.75,0.45) {\tiny$3$};
\node () at (0.75,2.3) {\tiny$0$};
\node () at (2,1.3) {\tiny$0$};
\node () at (1.8,0) {\tiny$0$};

\draw (a)--(b)--(e)--(c) --(a)--(d)--(b)--(c);
\draw (c)--(d);

\end{scope}
\end{tikzpicture}
\captionof{figure}{The larger vertex fires once to move from the left configuration to the right configuration.}
\end{texexample}

Behind the scenes pgf uses the basic system command \docAuxCommand{pgfnode} to create the nodes. The syntax of the command is given on \seepgfmanual{1026} as:

\begin{docCommand}{pgfnode}{\marg{shape}\marg{anchor}\marg{label text}\marg{name}\marg{path usage command}}
This command creates a new node. The \marg{shape} of the node must have been declared previously using
\lstinline{pgfdeclareshape}.

The shape is shifted such that the \marg{anchor} is at the origin. In order to place the shape somewhere else,
use the coordinate transformation prior to calling this command.
The hnamei is a name for later reference. If no name is given, nothing will be “saved” for the node, it
will just be drawn.

The \marg{path usage command} is executed for the background and the foreground path (if the shape defines
them).
\end{docCommand}


A good workflow, is to first define the nodes, next label them and then draw any connecting lines.

\begin{texexample}{Named nodes}{ex:named} 
\begin{tikzpicture}
\node[circle,fill=black,inner sep=0.8pt,draw] (a) at (0,0) {};
\node[circle,fill=black,inner sep=0.8pt,draw] (b) at (1.5,0) {};
\node[circle,fill=black,inner sep=1.5pt,draw] (c) at (.75,2) {};
\node[circle,fill=black,inner sep=0.8pt,draw] (d) at (0.75,.75) {};
\node[circle,fill=black,inner sep=0.8pt,draw] (e) at (2,1) {};
\end{tikzpicture}
\end{texexample}

\begin{texexample}{Named nodes}{ex:named} 
\begin{tikzpicture}
\node[circle,fill=black,inner sep=0.8pt,draw] (a) at (0,0) {};
\node[circle,fill=black,inner sep=0.8pt,draw] (b) at (1.5,0) {};
\node[circle,fill=black,inner sep=1.5pt,draw] (c) at (.75,2) {};
\node[circle,fill=black,inner sep=0.8pt,draw] (d) at (0.75,.75) {};
\node[circle,fill=black,inner sep=0.8pt,draw] (e) at (2,1) {};
% absolute labelling
\node () at (-0.3,0) {\tiny$1$};
\node () at (0.75,0.45) {\tiny$2$};
\node () at (0.75,2.3) {\tiny$4$};
\node () at (2,1.3) {\tiny$-1$};
\node () at (1.8,0) {\tiny$-1$};
\end{tikzpicture}
\end{texexample}

\begin{texexample}{Named nodes}{ex:named} 
\begin{tikzpicture}
\pgfdeclarelayer{background}
\pgfdeclarelayer{foreground}
\pgfsetlayers{background,main,foreground}
\node[circle,fill=black,inner sep=0.8pt,draw] (a) at (0,0) {};
\node[circle,fill=black,inner sep=0.8pt,draw] (b) at (1.5,0) {};
\node[circle,fill=black,inner sep=1.5pt,draw] (c) at (.75,2) {};
\node[circle,fill=black,inner sep=0.8pt,draw] (d) at (0.75,.75) {};
\node[circle,fill=black,inner sep=0.8pt,draw] (e) at (2,1) {};
% absolute labelling
\node () at (-0.3,0) {\tiny$1$};
\node () at (0.75,0.45) {\tiny$2$};
\node () at (0.75,2.3) {\tiny$4$};
\node () at (2,1.3) {\tiny$-1$};
\node () at (1.8,0) {\tiny$-1$};
% draw connecting lines
\draw (a)--(b)--(e)--(c) --(a)--(d)--(b)--(c);
\draw (c)--(d);
%\begin{pgfonlayer}{background}
\begin{scope}[on background layer={color=blue!10}]
\node [fill=blue!10,fit=(a) (b) (c)
(d) (e)] {};
\end{scope}
%\end{pgfonlayer}
\end{tikzpicture}
\end{texexample}

Just to recap, using \docAuxCommand*{node} and the \textbf{at} we can position accurately any node. We could have used the much longer command |path node|, but in our case above this is unecessary (\seepgfmanual{49}), for more explanations if you are still unsure.

Nodes can be named or unnamed. There are two ways to name them, with the key \docValue{name} or within brackets. The second method is to be preferred. Names for nodes can be pretty arbitrary, but they should not contain commas, periods, parentheses, colons, and some other special characters. However, they can contain underscores and hyphens

\subsection{Layers and Scope}

We can add a backround layer, using the library \textit{backgrounds}, which provides key values for adding backgrounds. \pgfname\ provides a layering mechanism for composing graphics from
multiple layers. (This mechanism is not to be confused with the
conceptual ``software layers'' the \pgfname\ system is composed of.)
Layers are often used in graphic programs. The idea is that you can
draw on the different layers in any order. So you might start drawing
something on the ``background'' layer, then something on the
``foreground'' layer, then something on the ``middle'' layer, and then
something on the background layer once more, and so on. At the end, no
matter in which ordering you drew on the different layers, the layers
are ``stacked on top of each other'' in a fixed ordering to produce
the final picture. Thus, anything drawn on the middle layer would come
on top of everything of the background layer.

Normally, you do not need to use different layers since you will have
little trouble ``ordering'' your graphic commands in such a way that
layers are superfluous. However, in certain situations you only
``know'' what you should draw behind something else after the
``something else'' has been drawn.

For example, suppose you wish to draw a yellow background behind your
picture. The background should be as large as the bounding box of the
picture, plus a little border. If you know the size of the bounding box
of the picture at its beginning, this is easy to accomplish. However,
in general this is not the case and you need to create a
``background'' layer in addition to the standard ``main'' layer. Then,
at the end of the picture, when the bounding box has been established,
you can add a rectangle of the appropriate size to the picture.

\subsection{Declaring Layers}

In \pgfname\ layers are referenced using names. The standard layer,
which is a bit special in certain ways, is called |main|. If nothing
else is specified, all graphic commands are added to the |main|
layer. You can declare a new layer using the following command:

\begin{docCommand}{pgfdeclarelayer}{\marg{name}}
  This command declares a layer named \meta{name} for later
  use. Mainly, this will set up some internal bookkeeping.
\end{docCommand}

The next step toward using a layer is to tell \pgfname\ which layers
will be part of the actual picture and which will be their
ordering. Thus, it is possible to have more layers declared than are
actually used.

\begin{docCommand}{pgfsetlayers}{\marg{layer list}}
  This command tells \pgfname\ which layers will be used in
  pictures. They are stacked on top of each other in the order
  given. The layer |main| should always be part of the list. Here is
  an example:
\begin{codeexample}[code only]
\pgfdeclarelayer{background}
\pgfdeclarelayer{foreground}  
\pgfsetlayers{background,main,foreground}
\end{codeexample}

  This command should be given either outside of any picture or ``directly inside'' of a picture.
  Here, the ``directly inside'' means that there should be no further level of \TeX\ grouping between |\pgfsetlayers| and the matching |\end{pgfpicture}| (no closing braces, no |\end{...}|). It will also work if |\pgfsetlayers| is provided before |\end{tikzpicture}| (with similar restrictions).
\end{docCommand}


\subsection{Using Layers}

Once the layers of your picture have been declared, you can start to
``fill'' them. As said before, all graphics commands are normally
added to the |main| layer. Using the |{pgfonlayer}| environment, you
can tell \pgfname\ that certain commands should, instead, be added to
the given layer.

\begin{docEnvironment}{pgfonlayer}{\marg{layer name}}
\end{docEnvironment}

The whole \meta{environment contents} is added to the layer with the
name \meta{layer name}. This environment can be used anywhere inside
a picture. Thus, even if it is used inside a |{pgfscope}| or a \TeX\
group, the contents will still be added to the ``whole'' picture.
Using this environment multiple times inside the same picture will
cause the \meta{environment contents} to accumulate.

  \emph{Note:} You can \emph{not} add anything to the |main| layer
  using this environment. The only way to add anything to the main
  layer is to give graphic commands outside all |{pgfonlayer}|
  environments. 



\begin{codeexample}[]
\pgfdeclarelayer{background layer}
\pgfdeclarelayer{foreground layer}
\pgfsetlayers{background layer,main,foreground layer}
\begin{tikzpicture}
  % On main layer:
  \fill[blue] (0,0) circle (1cm);
  
  \begin{pgfonlayer}{background layer}
    \fill[yellow] (-1,-1) rectangle (1,1);
  \end{pgfonlayer}
  
  \begin{pgfonlayer}{foreground layer}
    \node[white] {foreground};
  \end{pgfonlayer}
  
  \begin{pgfonlayer}{background layer}
    \fill[black] (-.8,-.8) rectangle (.8,.8);
  \end{pgfonlayer}

  % On main layer again:
  \fill[blue!50] (-.5,-1) rectangle (.5,1);
\end{tikzpicture}
\end{codeexample}



\long\gdef\mytriangle{
\node[circle,fill=black,inner sep=0.8pt,draw] (a) at (0,0) {};
\node[circle,fill=black,inner sep=0.8pt,draw] (b) at (1.5,0) {};
\node[circle,fill=black,inner sep=1.5pt,draw] (c) at (.75,2) {};
\node[circle,fill=black,inner sep=0.8pt,draw] (d) at (0.75,.75) {};
\node[circle,fill=black,inner sep=0.8pt,draw] (e) at (2,1) {};
% absolute labelling
\node () at (-0.3,0) {\tiny$1$};
\node () at (0.75,0.45) {\tiny$2$};
\node () at (0.75,2.3) {\tiny$4$};
\node () at (2,1.3) {\tiny$-1$};
\node () at (1.8,0) {\tiny$-1$};
% draw connecting lines
\draw (a)--(b)--(e)--(c) --(a)--(d)--(b)--(c);
\draw (c)--(d);
}

\begin{texexample}{Adding backgrouns}{ex:backgrounds}
\begin{tikzpicture}
\pgfdeclarelayer{background}
\pgfdeclarelayer{foreground}
\pgfsetlayers{background,main,foreground}
\mytriangle
%\begin{pgfonlayer}{background}
\begin{scope}[on background layer={color=blue!10}]
\mytriangle
\node [fill=blue!10,fit=(a) (b) (c)
(d) (e)] {};
\end{scope}
%\end{pgfonlayer}
\end{tikzpicture}
\end{texexample}


\begin{texexample}{Adding backgrouns}{ex:backgrounds}
\begin{tikzpicture}
\pgfdeclarelayer{background}
\pgfdeclarelayer{foreground}
\pgfsetlayers{background,main,foreground}
\mytriangle
%\begin{pgfonlayer}{background}
\begin{scope}[on background layer={color=blue!10}]
\node [fill=blue!10,fit=(a) (b) (c)
(d) (e)] {};
\end{scope}

\begin{scope}[shift={(+4,0)}]
\mytriangle
\begin{pgfonlayer}{background}
\node [pattern=checkerboard light gray,fit=(a) (b) (c)
(d) (e)] {};
\end{pgfonlayer}
\end{scope}
\end{tikzpicture}
\end{texexample}

This brings us to the end of our discussion. Time for a coffee and a break.                

\section{Adding styles}

In our previous example, we cut and pasted many of the repetitive keys. \pgfname offers a way to set a new key to the values of other keys using the handler |.style|. This is a very powerful way of redefining new keys, but also simplifying the code. Styles in \tikzname can be considered similar to macros in standard LaTeX. When I made a drawing, we can still tweak the styles and look how the drawing changes, until it's perfect. You should never have to tweak each node.

\begin{texexample}{Using styles}{ex:usingstyles}
\tikzset{BN/.style = {circle,fill=black,inner sep=0.8pt,draw},
         tiny/.style = {font=\tiny}, 
}
\begin{tikzpicture}
\node[BN] (a) at (0,0) {};
\node[BN] (b) at (1,0) {};
\node[BN] (c) at (1,1) {};
\node[BN] (d) at (0,1) {};
\node[BN] (e) at (-1,0) {};

\node () at (-1.3,0) [tiny]{$v_1$};
\node () at (-.3,1)  [tiny]{$v_2$};
\node () at (1.3,0)  [tiny]{$w_1$};
\node () at (1.3,1)  [tiny]{$w_2$};

\node[tiny] () at (0.5,-0.2) {$a$};
\node[tiny] () at (0.5,1.2) {$b$};
\node[tiny] () at (0.2,0.5) {$c$};
\node[tiny] () at (-0.5,-.2) {$d$};

\draw (e) -- (a) -- (b) -- (c) -- (d) -- (a);
\draw (e) -- (d);

\end{tikzpicture}
\end{texexample}



\section{Arcs and options for lines}

\begin{texexample}{Draw a Line}{ex:line}
\begin{tikzpicture}
\draw[->] (0,0) -- (1.5,0) node[draw, ellipse] {First Node} -| (0, 1.2) node[draw,ellipse,rotate=45] {Second Node};
\end{tikzpicture}
\end{texexample}

\begin{texexample}{Drawing arcs}{ex:matharcs}
We define 
\begin{gather*}
    \bar{d}_{k,l}:=\hspace{6pt}
    \begin{tikzpicture}[baseline=(current bounding box.center)]
    \draw[->] (3,2) arc (-180:180:5mm);
	  \fill (3.95,2.2) circle [radius=2pt];
    \draw (3.95,1.8) circle [radius=2pt];
    \node at (4.2,1.8) {$l$};
    \node at (4.2,2.2) {$k$};
    \end{tikzpicture}
    \hspace{0.5cm}
    \text{and}
    \hspace{0.5cm}
    d_{k,l}:=\hspace{6pt}
    \begin{tikzpicture}[baseline=(current bounding box.center)]
    \draw[<-] (3,2) arc (-180:180:5mm);
    \fill (3.95,2.2) circle [radius=2pt];
    \draw (3.95,1.8) circle [radius=2pt];
    \node at (4.2,1.8) {$l$};
    \node at (4.2,2.2) {$k$};
    \end{tikzpicture}
    \hspace{0.5cm}
    \text{for}
    \hspace{2mm} k,l\in\mathbb{Z}_{\geq 0}.
\end{gather*}
\end{texexample}


Here is a figure that you should try and reproduce.
\newcommand{\G}{\Gamma}

\begin{tikzpicture}
\draw (-3.5,-1)--(-2.5,0); \draw (-2.5,-1)--(-3.5,0); \draw (-1.5,-1)--(-1.5,0);\draw[fill=black] (-3,-0.5) circle (0.1cm); \draw (-3.5,0)--(-3.5,1); \draw (-2.5,0)--(-1.5,1); \draw (-1.5,0)--(-2.5,1);\draw[fill=black] (-2,0.5) circle (0.1cm); \draw[->] (-3.5,1)--(-2.5,2); \draw[->] (-2.5,1)--(-3.5,2); \draw[->] (-1.5,1)--(-1.5,2); \draw[fill=black] (-3,1.5) circle (0.1cm); \draw (-3.6,0)--(-3.4,0);\draw (-2.6,0)--(-2.4,0);\draw (-1.6,0)--(-1.4,0); \draw (-3.6,1)--(-3.4,1);\draw (-2.6,1)--(-2.4,1);\draw (-1.6,1)--(-1.4,1); \node at (-3.5,-1.2) {$x_1$};\node at (-2.5,-1.2) {$x_2$};\node at (-1.5,-1.2) {$x_3$}; \node at (-3.5,2.2) {$y_1$};\node at (-2.5,2.2) {$y_2$};\node at (-1.5,2.2) {$y_3$}; \node at (-3.8,0) {$t_1$};\node at (-2.2,0) {$t_2$};\node at (-1.2,0) {$t_3$}; \node at (-3.8,1) {$t_4$};\node at (-2.8,1) {$t_5$};\node at (-1.2,1) {$t_6$}; \node at (-2.5,-1.65) {$\Gamma$};
\draw[->] (0,0)--(1,1); \draw[->] (1,0)--(0,1); \draw[fill=black] (0.5,0.5) circle (0.1cm); \draw[->] (2,0)--(3,1); \draw[->] (3,0)--(2,1); \draw[fill=black] (2.5,0.5) circle (0.1cm); \draw[->] (4,0)--(5,1); \draw[->] (5,0)--(4,1); \draw[fill=black] (4.5,0.5) circle (0.1cm); \draw[->] (6,0)--(6,1); \draw[->] (7,0)--(7,1); \draw[->] (8,0)--(8,1);
\node at (0,-.2) {$x_1$};\node at (1,-.2) {$x_2$}; \node at (2,-.2) {$t_2$};\node at (3,-.2) {$t_3$}; \node at (4,-.2) {$t_4$};\node at (5,-.2) {$t_5$}; \node at (6,-.2) {$x_3$}; \node at (7,-.2) {$t_1$}; \node at (8,-.2) {$t_6$};
\node at (0,1.2) {$t_1$};\node at (1,1.2) {$t_2$}; \node at (2,1.2) {$t_5$};\node at (3,1.2) {$t_6$}; \node at (4,1.2) {$y_1$};\node at (5,1.2) {$y_2$}; \node at (6,1.2) {$t_3$}; \node at (7,1.2) {$t_4$}; \node at (8,1.2) {$y_3$};
\node at (0.5,-0.65) {$\G_1$}; \node at (2.5,-0.65) {$\G_2$}; \node at (4.5,-0.65) {$\G_3$}; \node at (6,-0.65) {$\G_4$};\node at (7,-0.65) {$\G_5$};\node at (8,-0.65) {$\G_6$}; 
\end{tikzpicture}

This brings us to the end.




The |node| can take numerous options who are then used to set the typesetting of the text that follows:


\begin{texexample}{Draw a Line}{ex:line}
\begin{tikzpicture}
\draw[->] (0,0) -- (1.5,0) node[draw, ellipse] {First Node} -| (0, 1.2) node[draw,ellipse,rotate=45, text width=3cm, fill=creamy, text justified] {\lorem};
\end{tikzpicture}
\end{texexample}


\begin{texexample}{Draw a Line}{ex:line}
\begin{tikzpicture}[funny ellipse/.style = {draw,ellipse,rotate=45, text width=3cm, fill=creamy, text justified} ]
\draw[->] (0,0) -- (1.5,0) node[draw, ellipse] {First Node} -| (0, 1.2) node[funny ellipse] {\lorem};
\end{tikzpicture}
\end{texexample}

This can also be written by using \docAuxCommand{tikzset} for setting out all the keys. This can written just before the environment or within the scope of the environment. See \href{https://tex.stackexchange.com/questions/52372/should-tikzset-or-tikzstyle-be-used-to-define-tikz-styles}{TX.SX discussion}, for the option to set |\tikzstyle| which should not be used, even if it is quicker to write.


\begin{texexample}{Draw a Line}{ex:line}
\tikzset{funny ellipse/.style = {draw,ellipse,rotate=45, text width=3cm, fill=creamy, text justified} }
\begin{tikzpicture}
\draw[->] (0,0) -- (1.5,0) node[draw, ellipse] {First Node} -| (0, 1.2) node[funny ellipse] {\lorem};
\end{tikzpicture}
\end{texexample}

A |node| can possibly be rendered with a choice from a list of over 720 keys.

ed. 



Using the |TikZ| package you can draw figures and intermingle them with text. To draw a simple diamond as shown in \fref{fig:diamond} we use
the following commands. The package comes with a very comprehensive manual of over 500 pages long. One can state that there is nothing that you cannot draw with PGF/TikZ, if you have the patience and perseverance. TikZ's language has a syntax of its own with very little connection to what we have used so far. You will need to set aside adequate time to study this, especially if your work has a lot of specially drawn figures that you need. The result like anything else in \tex make the effort worthwhile.

\begin{texexample}{Draw a Diamond}{fig:diamond}
\begin{tikzpicture}
 \draw (1,0) -- (0,1) -- (-1,0) -- (0,-1) -- cycle;
\end{tikzpicture}
\end{texexample}


\begin{texexample}{Text long path}{ex:decorations}
\begin{tikzpicture}
\draw [help lines] grid (3,2);
\draw [red, dashed]
[postaction={decoration={text along path, text={a big juicy apple},
text align=fit to path}, decorate}]
(0,0) .. controls (0,2) and (3,2) .. (3,0);
\node (A) at (1.5,0) {!};
\end{tikzpicture}
\end{texexample}


\begin{texexample}{Text long path}{ex:decorations}

Hello \begin{pgfpicture}
\pgfpathrectangle{\pgfpointorigin}{\pgfpoint{2ex}{1ex}}
\pgfusepath{stroke}
\end{pgfpicture} World!

\end{texexample}


\emphasis{-,draw,begin,end,tikzpicture}
\begin{teXXX}
\begin{tikzpicture}
\draw (1,0) -- (0,1) -- (-1,0) -- (0,-1) -- cycle;
\end{tikzpicture}
\end{teXXX}



\makeatletter
The value of $x$ is \pgfsys@markposition{here}important.

Lots of text.
\hbox{\pgfsys@markposition{myorigin}%
\begin{pgfpicture}
% Switch of size protocol
\pgfpathmoveto{\pgfpointorigin}
\pgfusepath{use as bounding box}
\pgfsys@getposition{here}{\hereposition}
\pgfsys@getposition{myorigin}{\thispictureposition}
\pgftransformshift{\pgfpointscale{-1}{\thispictureposition}}
\pgftransformshift{\hereposition}
\pgfpathcircle{\pgfpointorigin}{1cm}
\pgfusepath{draw}
\end{pgfpicture}}

\makeatother


You cannot write directly into a picture environment. The command \docAuxCommand{pgftext} can be used. 

\begin{texexample}{Using text directly}{ex:pgftext}
\tikz{\draw[help lines] (0,0) grid (3,2);
\pgftext[base,x=1cm,y=0.5cm] {lovely}}
\end{texexample}





Sometimes it is quite useful when debugging to add a backround grid. 


\begin{centering}
\begin{tikzpicture}
\draw[step=0.25cm,color=creamy] (-1,-1) grid (1,1);
\draw [color=bgsexy](1,0) -- (0,1) -- (-1,0) -- (0,-1) -- cycle;
\end{tikzpicture}
\captionof{figure}{You can add a background grid using \texttt{step=0.25cm, color=green} as an option}
\end{centering}


\emphasis{step,color,green,grid,begin,end}
\begin{teXXX}
\begin{tikzpicture}
  \draw[step=0.25cm,color=green] (-1,-1) grid (1,1);
  \draw (1,0) -- (0,1) -- (-1,0) -- (0,-1) -- cycle;
\end{tikzpicture}
\end{teXXX}

The grid is specified by providing two diagonally opposing points: (-1,-1)
and (1, 1). The two options supplied give a step size for the grid lines and a
specification for the color of the grid lines, using the \docpkg{xcolor} package

\subsection{Specifying points and paths}

\begin{texexample}{Specifying points and paths}{ex:points}
\centering
\begin{tikzpicture}[scale=1.8]
% Define the points of a regular pentagon
\path (0,0) coordinate (origin);
\path (0:1cm) coordinate (P0);
\path (1*72:1cm) coordinate (P1);
\path (2*72:1cm) coordinate (P2);
\path (3*72:1cm) coordinate (P3);
\path (4*72:1cm) coordinate (P4);
% Draw the edges of the pentagon
\draw[color=bgsexy] (P0) -- (P1) -- (P2) -- (P3) -- (P4) -- cycle;
% Add "spokes"
\draw[color=bgsexy] (origin) -- (P0) (origin) -- (P1) (origin) -- (P2)
(origin) -- (P3) (origin) -- (P4);
\end{tikzpicture}
\captionof{figure}{Drawing a complicated polygon, using paths and the \texttt{draw} command}
\end{texexample}


Two key ideas used in \tikzname\ are points and paths. Both of these ideas were used
in the diamond examples. Much more is possible, however. For example, points
can be specified in any of the following ways:
\begin{enumerate}
\item  Cartesian coordinates
\item  Polar coordinates
\item  Named points
\item  Relative points
\end{enumerate}



\subsection{coordinates}
The cartesian coordinates can be defined and named using the following syntax.

%\emphasis{begin,end,coordinate,at,draw}
%\begin{teXXX}
%\begin{tikzpicture}
%  \coordinate (A) at (0,0);
%  \coordinate (B) at (1.25,0.25);
%  \draw[blue] (A) -- (B);
%\end{tikzpicture}
%\end{teXXX}

\noindent This produces:
\begin{tikzpicture}
\coordinate (A) at (0,0);
\coordinate (B) at (1.25,0.25);
\draw[blue] (A) -- (B);
\end{tikzpicture}


We can add labels to the points by using the |label| option. A label is distinct from the text of a |node|.

\begin{tikzpicture}
\coordinate [label=left:\textcolor{orange}{$A$}] (A) at (0,0);
\coordinate [label=right:\textcolor{orange}{$B$}]  (B) at (1.15,0.25);
\draw[blue] (A) -- (B);
\end{tikzpicture}

\emphasis{label,left,label:,right}
\begin{teXXX}
\begin{tikzpicture}
  \coordinate [label=left:\textcolor{orange}{$A$}] (A) at (0,0);
  \coordinate [label=west:\textcolor{orange}{$B$}] (B) at (1.25,0.25);
  \draw[blue] (A) -- (B);
\end{tikzpicture}
\end{teXXX}




If you tempted to write \texttt{label=top:} it will not work, as the command accepts the following keywords.

\begin{tikzpicture}
  \coordinate [label=left:\textcolor{orange}{east}]  (A) at (0,0);
  \coordinate [label=right:\textcolor{orange}{west}] (B) at (0,0);
  \draw[blue] (A)--(B);
\end{tikzpicture}


\section{Graphic Parameters: Line Width, Line Cap, and Line Join}

The width of lines can be specified using the key:

\begin{docKey}[tikz]{line width}{=\marg{dimension}} {no default, initially 0.4pt}
Specifies the line width \seepgfmanual{166}
\end{docKey}



\bgroup
\def\mkl#1{\tikz \draw[#1] (0,0)--(1.0, 1.5ex);}
\scriptsize\arial
\begin{tabular}{|l|l|l|l|l|l|l|l|}
\hline
\mkl{line width=2pt}& \mkl{ultra thin} &\mkl{very thin} & \mkl{thin} & \mkl{semithick} & \mkl{thick} &\mkl{very thick} &\mkl{ultra thick} \\
\hline
line width=2pt &ultra thin & very thin & thin &semithick & thick & very thick & ultra thick \\
\hline
\end{tabular}
\egroup

\begin{docKey}[tikz]{line cap}{=\marg{dimension}} {no default, initially 0.4pt}
Specifies how lines “end.” Permissible types are round, rect, and butt \seepgfmanual{167}. 
\end{docKey}

\bgroup
\def\mkl#1{\begin{tikzpicture} \draw[line width=10pt, line cap=#1] (0,0)--(1.0, 1.5ex);\draw[white,line width=2pt]
(0,0 )--(1.0,1.5ex);\end{tikzpicture}}
\scriptsize\arial
\begin{tabular}{|l|l|l|}
\hline
\mkl{rect}& \mkl{butt} &\mkl{round}  \\
\hline
rect &butt & round \\
\hline
\end{tabular}
\egroup




\begin{docKey}[tikz]{line join}{=\marg{type}}{no default, initially miter}
Specifies how lines “join.” Permissible type are round, bevel, and miter. They have the following
effects:
\end{docKey}

\begin{texexample}{Joining Lines}{es:joinlines}
\begin{tikzpicture}[line width=10pt]
\draw[line join=round] (0,0) -- ++(.5,1) -- ++(.5,-1);
\draw[line join=bevel] (1.25,0) -- ++(.5,1) -- ++(.5,-1);
\draw[line join=miter] (2.5,0) -- ++(.5,1) -- ++(.5,-1);
\end{tikzpicture}
\end{texexample}


\begin{docKey}[tikz]{dash pattern}{=\marg{dash pattern}}{no default}
Sets the dashing pattern. The syntax is the same as in \metafontlogo. For example following pattern on
2pt off 3pt on 4pt off 4pt means \enquote{draw 2pt, then leave out 3pt, then draw 4pt once more, then
leave out 4pt again, repeat}.
\end{docKey}

\bgroup
\def\ml#1{\tikz \draw[ #1] (0pt,0pt) -- (50pt,0pt);}
\def\alist{solid, dotted, densely dotted, loosely dotted,% 
           dashed,densely dashed, loosely dashed, %
           dash dot, densely dash dot, loosely dash dot, %
           dash dot dot, densely dash dot dot, loosely dash dot dot.}

For patterns there are numerous settings {\arial \alist }


\scriptsize
\begin{tabular}{lll}
\hline
\ml{solid} &  & \\
solid      &  & \\
\hline
\ml{dotted} &\ml{densely dotted} & \ml{loosely dotted}\\
\textit{dotted} & densely dotted  &loosely dotted \\
\hline
\ml{dashed} & \ml{densely dashed} & \ml{loosely dashed}  \\
\textit{dashed}      & densely dashed & loosely dashed            \\
\hline

\ml{dash dot} & \ml{densely dash dot} & \ml{loosely dash dot} \\
\textit{dash dot} & densely dash dot & loosely dash dot \\
\hline

\ml{dash dot dot} & \ml{densely dash dot dot} & \ml{loosely dash dot dot} \\
\textit{dash dot dot} & densely dash dot dot & loosely dash dot dot \\
\hline
\end{tabular}
\egroup


\subsection{Pattern Library}

The library patterns can be used to draw predetermined patterns. This will be a longer than usual section as it explains how to create new patterns. Most of the content is straight from the \pgfname manual. Before we start with the creation f a new pattern let us examine how a pattern is used.

\begin{texexample}{Using Library Patterns}{ex:libpatterns}
\begin{tikzpicture}
\pattern [path fading=west,pattern=checkerboard light gray]
      (0,0) rectangle (5cm,2em);
\end{tikzpicture}
\end{texexample}


\label{section-library-patterns}


The package defines patterns for filling areas. \docAuxCommand*{usetikzlibrary}\marg{patterns}.




\subsection{Form-Only Patterns}

\begin{tabular}{ll}
  \emph{Pattern name} & \emph{Example (pattern in black, blue, and red
    on faded checkerboard)} \\ 
  \patternindex{horizontal lines} 
  \patternindex{vertical lines} 
  \patternindex{north east lines} 
  \patternindex{north west lines} 
  \patternindex{grid} 
  \patternindex{crosshatch} 
  \patternindex{dots} 
  \patternindex{crosshatch dots} 
  \patternindex{fivepointed stars} 
  \patternindex{sixpointed stars} 
  \patternindex{bricks}
  \patternindex{checkerboard}
\end{tabular}
  
\subsection{Inherently Colored Patterns}


\begin{tabular}{ll}
  \emph{Pattern name} & \emph{Example} \\
  \patternindexinherentlycolored{checkerboard light gray} 
  \patternindexinherentlycolored{horizontal lines light gray} 
  \patternindexinherentlycolored{horizontal lines gray} 
  \patternindexinherentlycolored{horizontal lines dark gray} 
  \patternindexinherentlycolored{horizontal lines light blue} 
  \patternindexinherentlycolored{horizontal lines dark blue} 
  \patternindexinherentlycolored{crosshatch dots gray} 
  \patternindexinherentlycolored{crosshatch dots light steel blue} 
\end{tabular}
  


% Copyright 2006 by Till Tantau
%
% This file may be distributed and/or modified
%
% 1. under the LaTeX Project Public License and/or
% 2. under the GNU Free Documentation License.
%
% See the file doc/generic/pgf/licenses/LICENSE for more details.


\section{Creating Patterns}

\label{section-patterns}

\subsection{Overview}

There are many ways of filling a path. First, you can fill it using a
solid color and this is also the fastest method. Second, you can also
fill it using a shading, which means that the color changes smoothly
between two (or more) different colors. Third, you can fill it using a
tiling pattern and it is explained in the following how this is done.

A tiling pattern can be imagined as a rectangular tile (hence the
name) on which a small picture is painted. There is not a single tile,
but (conceptually) an infinite number of tiles, all showing the same
picture, and these tiles are arranged horizontally and vertically to
fill the plane. When you use a tiling pattern to fill a path, what
happens is that the path clips out a ``window'' through which we see
part of this infinite plane.

Patterns come in two versions: \emph{inherently colored patterns} and
\emph{form-only patterns}. (These are often called ``color patterns''
and ``uncolored patterns,'' but these names are misleading since
uncolored patterns do have a color and the color changes. As I said,
the name is misleading\dots) An inherently colored pattern is just a
colored tile like, say, a red star with a black outline. A form-only
pattern can be imagined as a tile that is a kind of rubber stamp. When
this pattern is used, the stamp is used to print copies of the stamp
picture onto the plane, but we can use a different stamp color each
time we use a form-only pattern.

\pgfname\ provides a special support for patterns. You can declare a
pattern and then use it very much like a fill color. \pgfname\
directly maps patterns to the pattern facilities of the underlying
graphic languages (PostScript, \textsc{pdf}, and \textsc{svg}). This
means that filling a path using a pattern will be nearly as fast as if
you used a uniform color.

There are a number of pitfalls and restrictions when using
patterns. First, once a pattern has been declared, you cannot change
it anymore. In particular, it is not possible to enlarge it or change
the line width. Such flexibility would require that the repeating of
the pattern were not done by the graphic language, but on the
\pgfname\ level. This would make patterns orders of magnitude slower
to produce and to render. However, \pgfname{} does provide a
more-or-less successful emulation of ``mutable'' patterns, although
internally, a new (fixed) instance of a pattern is declared when
the parameters of a pattern change.

Second, the phase of patterns is not well-defined, that is, it is not
clear where the origin of the ``first'' tile is. To be more precise,
PostScript and \textsc{pdf} on the one hand and \textsc{svg} on the
other hand define the origin differently. PostScript and \textsc{pdf}
define a fixed origin that is independent of where the path lies. This
has the highly desirable effect that if you use the same pattern to
fill multiple paths, the outcome is the same as if you had filled a 
single path consisting of the union of all these paths. By
comparison, \textsc{svg} uses the upper-left (?) corner of the path to
be filled as the origin. However, the \textsc{svg} specification is a
bit vague on this question.


\subsection{Declaring a Pattern}

Before a pattern can be used, it must be declared. The following
command is used for this:

\begin{docCommand}{pgfdeclarepatternformonly}{%
	\oarg{variables}%
	\marg{name}%
	\marg{bottom left}%
	\marg{top right}%
	\marg{tile size}%
	\marg{code}}

	This command declares a new form-only pattern. The \meta{name} is a
  name for later reference. The two parameters \meta{lower left} and
  \meta{upper right} must describe a bounding box that is large enough
  to encompass the complete tile.
\end{docCommand}

  The size of a tile is given by \meta{tile size}, that is, a tile is
  a rectangle whose lower left   corner is the origin and whose upper
  right corner is given by \meta{tile size}. This might make you
  wonder why the second and third parameters are needed. First, the
  bounding box might be smaller than the tile size if the tile is
  larger than the picture on the tile. Second, the bounding box might
  be bigger, in which case the picture will ``bleed'' over the tile.

  The \meta{code} should be \pgfname\ code than can be protocolled. It
  should not contain any color code.


\begin{codeexample}[]
\pgfdeclarepatternformonly{stars}
{\pgfpointorigin}{\pgfpoint{1cm}{1cm}}
{\pgfpoint{1cm}{1cm}}
{
  \pgftransformshift{\pgfpoint{.5cm}{.5cm}}
  \pgfpathmoveto{\pgfpointpolar{0}{4mm}}
  \pgfpathlineto{\pgfpointpolar{144}{4mm}}
  \pgfpathlineto{\pgfpointpolar{288}{4mm}}
  \pgfpathlineto{\pgfpointpolar{72}{4mm}}
  \pgfpathlineto{\pgfpointpolar{216}{4mm}}
  \pgfpathclose%
  \pgfusepath{fill}
}
\begin{tikzpicture}
  \filldraw[pattern=stars] (0,0)   rectangle (1.5,2);
  \filldraw[pattern=stars,pattern color=red]
                           (1.5,0) rectangle (3,2);
\end{tikzpicture}
\end{codeexample}

	The optional argument \meta{variables} consists of a comma
	separated	list of macros,	registers or keys, representing the
	parameters of the pattern that may vary. If a variable is a key,
	then the full path name must be used (specifically, it must start
	with |/|).
	As an example, a list might look like the following:
	|\mymacro,\mydimen,/pgf/my key|. Note that macros and keys should
	be ``simple''. They should only store values in themselves.
	
	The effect of \meta{variables}, is the following:
  Normally, when this argument is empty, once a pattern has been
  declared, it becomes ``frozen''. This means that it is not possible
  to enlarge the pattern or change the line width later on.
  By specifying \meta{variables}, no pattern is actually created.
  Instead, the arguments are stored away
  (so the macros,	registers or keys do not have to be defined in advance).

  When the fill pattern is set, \pgfname{} checks if the pattern has
  already been created with the \meta{variables} set to their current
  values (\pgfname{} is usually ``smart enough'' to distinguish between
  macros, registers and keys). If so, this already-declared-pattern
  is used as the fill pattern.
  If not, a new instance of the pattern (which will have a
  unique internal name) is declared using the current values of
  \meta{variables}. These values are then saved and the fill pattern
  set accordingly.
	
	The following shows an example of a pattern which varies
	according to the values of the macro |\size|, the key |/tikz/radius|,
	and the \TeX{} dimension |\thickness|.

\begin{texexample}{New Pattern Example}{ex:newpattern}
\pgfdeclarepatternformonly[/tikz/radius,\thickness,\size]{rings}
{\pgfpoint{-0.5*\size}{-0.5*\size}}
{\pgfpoint{0.5*\size}{0.5*\size}}
{\pgfpoint{\size}{\size}}
{
  \pgfsetlinewidth{\thickness}
  \pgfpathcircle\pgfpointorigin{\pgfkeysvalueof{/tikz/radius}}
  \pgfusepath{stroke}
}
\newdimen\thickness
\tikzset{
  radius/.initial=4pt,
  size/.store in=\size, size=20pt,
  thickness/.code={\thickness=#1},
  thickness=0.75pt
}
\begin{tikzpicture}[rings/.style={pattern=rings}]
  \filldraw [rings, radius=2pt, size=6pt]      (0,0)   rectangle +(1.5,2);
  \filldraw [rings, radius=2pt, size=8pt]      (2,0)   rectangle +(1.5,2);
  \filldraw [rings, radius=6pt, thickness=2pt] (0,2.5) rectangle +(1.5,2);
  \filldraw [rings, radius=8pt, thickness=4pt] (2,2.5) rectangle +(1.5,2);
\end{tikzpicture}
\end{texexample}



\begin{docCommand}{pgfdeclarepatterninherentlycolored}{\oarg{variables}
    \marg{name}
    \marg{lower left}
    \marg{upper right}
    \marg{tile size}
    \marg{code}}
  This command works like |\pgfdeclarepatternuncolored|, only the
  pattern will have an inherent color. To set the color, you should
  use \pgfname's color commands, not the |\color| command, since this
  fill is not protocolled.
\end{docCommand}

\begin{texexample}{Inherently Colored}{ex:ingerentlycolored}
\pgfdeclarepatterninherentlycolored{green stars}
{\pgfpointorigin}{\pgfpoint{1cm}{1cm}}
{\pgfpoint{1cm}{1cm}}
{
  \pgfsetfillcolor{green!50!black}
  \pgftransformshift{\pgfpoint{.5cm}{.5cm}}
  \pgfpathmoveto{\pgfpointpolar{0}{4mm}}
  \pgfpathlineto{\pgfpointpolar{144}{4mm}}
  \pgfpathlineto{\pgfpointpolar{288}{4mm}}
  \pgfpathlineto{\pgfpointpolar{72}{4mm}}
  \pgfpathlineto{\pgfpointpolar{216}{4mm}}
  \pgfpathclose%
  \pgfusepath{stroke,fill}
}
\begin{tikzpicture}
  \filldraw[pattern=green stars] (0,0) rectangle (3,2);
\end{tikzpicture}
\end{texexample}



\subsection{Setting a Pattern}

Once a pattern has been declared, it can be used.

\begin{docCommand}{pgfsetfillpattern}{\marg{name}\marg{color}}
  This command specifies that paths that are filled should be filled
  with the ``color'' by the pattern \meta{name}. For an inherently
  colored pattern, the \meta{color} parameter is ignored. For
  form-only patterns, the \meta{color} parameter specifies the color
  to be used for the pattern.
\end{docCommand}
  
\begin{codeexample}[]
\begin{tikzpicture}
  \pgfsetfillpattern{stars}{red}
  \filldraw (0,0) rectangle (1.5,2);

  \pgfsetfillpattern{green stars}{red}
  \filldraw (1.5,0) rectangle (3,2);
\end{tikzpicture}
\end{codeexample}



\endinput
%To summarize, what we have been doing so far is to learn a set of primitive TikZ commands for drawing paths, drawing shapes and labeling them. All TikZ command work by passing options to them. For example to change the above line to an arrow, we just pass the option |->| to the |draw| command.
%

%\begin{tikzpicture}
%  \coordinate [label=left:\textcolor{orange}{$A$}] (A) at (0,0);
%  \coordinate [label=right:\textcolor{orange}{$B$}] (B) at (1.25,0.25);
%  \draw[->,o-stealth] (A)--(B);
%\end{tikzpicture}
%\caption{Effect of the option \protect\texttt{draw[->]}.}

%\emphasis{begin,end,->,draw}
%\begin{teXXX}
%\begin{tikzpicture}
%  ...
%  ...
%  \draw[->,blue] (A)--(B);
%\end{tikzpicture}
%\end{teXXX}
%
%\section*{Relative coordinates}
%\index{TikZ!coordinates, relative}
%A coordinate can be made "relative" by prefixing it with |++|. relative coordinates are useful in many applications.
%\medskip
%
%\noindent The code is simple, except before the coordinate you add the |++| signs. This tells the PGF engine to add the x,y dimensions of the new coordinate to that of its predecessor's. In many instances this is more intuitive and easier to determine.



%\begin{tikzpicture}
%\draw[step=0.5cm,color=gray] (-1,-1) grid (3.5,3);
%\draw[->,red,thick] (0,0) -- ++(1,0) -- ++(0,1) -- ++(-1,0) -- cycle;
%\draw[->,red,thick] (2,0) -- ++(1,0) -- ++(0,1) -- ++(-1,0) -- cycle;
%\draw[arrows=o-stealth,blue] (1.5,1.5) -- ++(1,0) -- ++(0,1) -- ++(-1,0) -- cycle;
%\end{tikzpicture}
%\caption{Example of use of the \protect\texttt{++} to specify relative coordinates.}
%\label{fig:relative}

%\begin{teXXX}
%\begin{tikzpicture}
%  \draw[step=0.5cm,color=gray] (-1,-1) grid (3.5,3);
%  \draw[red,very thick] (0,0) -- ++(1,0) -- ++(0,1) -- ++(-1,0) -- cycle;
%  \draw[red,very thick] (2,0) -- ++(1,0) -- ++(0,1) -- ++(-1,0) -- cycle;
%  \draw[->,red,very thick] (1.5,1.5) -- ++(1,0) -- ++(0,1) -- ++(-1,0) -- cycle;
%\end{tikzpicture}
%\end{teXXX}
%
%Instead of |++| you can also use a single |+|. This also specifies a relative coordinate, but it does not "update"
%the current point for subsequent usages of relative coordinates. Thus, you can use this notation to specify
%numerous points, all relative to the same "initial" point:
%

%\begin{tikzpicture}
%\draw[step=0.5cm,color=gray] (-1,-1) grid (3.5,3);
%\draw[purple, fill=white] (0,0) -- +(1,0) -- +(1,1) -- +(0,1) -- cycle;
%\draw[purple, fill=white] (2,0) -- +(1,0) -- +(1,1) -- +(0,1) -- cycle;
%\draw[purple, fill=white] (1.5,1.5) -- +(1,0) -- +(1,1) -- +(0,1) -- cycle;
%\path (0,0) node [shape=circle,draw]{(0,0)};
%\end{tikzpicture}
%\caption{Example of use of the \protect\texttt{+} to specify relative coordinates.}
%\label{fig:relative1}

%\begin{teXXX}
%  \draw (0,0) -- +(1,0) -- +(1,1) -- +(0,1) -- cycle;
%  \draw (2,0) -- +(1,0) -- +(1,1) -- +(0,1) -- cycle;
%  \draw (1.5,1.5) -- +(1,0) -- +(1,1) -- +(0,1) -- cycle;
%\end{teXXX}
%
%
%Personally, I don't favour this method of specifying co-ordinates, but it can be useful, if you are automating the production of figures through an external script\sidenote{For drawing Bezier curves, the \texttt{+} behaves differently.  You can refer to the PGF Manual for more details.}.
%
%
%\section*{Arrows}
%\index{TikZ>arrows}
%The function |->| creates a tooltip arrow. You can use different arrow tips and there is a long section for them in the PGF manual. You can even define your own.

\bgroup
%\centering
%\begin{tikzpicture}
%  \draw[->] (0,0) -- (2,0);
%  \draw[arrows=o-stealth,blue] (0,-0.3) -- (2,-0.3);
%  \draw[->,o-stealth,orange] (0,-0.6) -- (2,-0.6);
%  \draw[arrows=|-stealth,purple] (0,-0.9) -- (2,-0.9);
%\end{tikzpicture}
%\captionof{figure}{Special arrow endings}
%\label{fig:specials}
\egroup
%
%\emphasis{o,stealth,begin,end,draw}
%\begin{teXXX}
%\begin{tikzpicture}
% \draw[->] (0,0) -- (2,0);
% \draw[arrows=o-stealth,blue] (0,-0.3) -- (2,-0.3);
% \draw[->,o-stealth,orange] (0,-0.6) -- (2,-0.6);
% \draw[arrows=X-stealth,purple] (0,-0.9) -- (2,-0.9);
%\end{tikzpicture}
%\end{teXXX}

%

\begin{verbatim}
\begin{tikzpicture}
% Define the points of a regular pentagon
\path (0,0) coordinate (origin);
\path (0:1cm) coordinate (P0);
\path (1*72:1cm) coordinate (P1);
\path (2*72:1cm) coordinate (P2);
\path (3*72:1cm) coordinate (P3);
\path (4*72:1cm) coordinate (P4);
% Draw the edges of the pentagon
\draw (P0) -- (P1) -- (P2) -- (P3) -- (P4) -- cycle;
% Add "spokes"
\draw (origin) -- (P0) (origin) -- (P1) (origin) -- (P2)
(origin) -- (P3) (origin) -- (P4);
\end{tikzpicture}
\end{verbatim}





\section{Nodes}

A node is a small part of a picture. When a node is created, you provide a position where the node
should be drawn and a shape. A node of shape circle will be drawn as a |circle|, a node of shape |rectangle|
as a rectangle, and so on. A node may also contain same text, which is why they can used nodes to show text.

Finally, a node can get a name for later reference.



\emphasis{node,shape,draw}
\begin{teXXX}
\begin{tikzpicture}
\path ( 0,2) node [shape=circle,draw] {.}
( 0,1) node [shape=circle,draw] {..}
( 0,0) node [shape=circle,draw] {...}
( 1,1) node [shape=rectangle,draw] {....}
(-2,1) node [shape=rectangle,draw] {rectangle (-2,1)};
\end{tikzpicture}
\end{teXXX}
\medskip

\begin{tikzpicture}
\path ( 0,2) node [shape=circle,draw] {1}
( 0,1) node [shape=circle,draw] {\textbf{10}}
( 0,0) node [shape=circle,draw] {\textbf{100}}
( 1,1) node [shape=circle,draw] {\textbf{1000}}
(-2,1) node [shape=circle,draw] {\textbf{10000}};
\end{tikzpicture}

In the above code, this text is empty (because of the
|empty {}|). So, why do we see anything at all at all the nodes? The answer is the draw option for the node operation: It
causes the |shape| around the text" to be drawn. If you have an empty |{}|, PGF still sees the empty space as a character and justs draws around it. The reason is than TikZ automatically adds some space around the text. The amount is set
using the option |inner sep|. So, to increase the size of the nodes. Modifying the example slightly we get.



\begin{tikzpicture}
\path ( 0,2) node [shape=circle,draw] {.}
( 0,1) node [shape=circle,draw] {..}
( 0,0) node [shape=circle,draw] {...}
( 1,1) node [shape=circle,draw] {....}
(-1,1) node [shape=circle,draw] {.....};
\end{tikzpicture}

As you can observe the size of the circle has been adjusted to fit the text that is enclosing it. 
Another way to simply add a node is using the |at| syntax:

\begin{texexample}{The node command}{}
\begin{tikzpicture}
\node at (0,0) [circle, draw] {\textbf{100}};
\node at (1,1) [diamond,draw] {\textbf{100}};
\end{tikzpicture}
\end{texexample}

The \cmd{\node} is an abbreviation of the |\path| node. This is a much shorter syntax than |\path| where one would need to add a lot of redundant move-tos  \seepgfmanual{215}.

If you have many nodes another way of achieving the example outlined above is to use the |\draw| command in comination with node and at.

\begin{texexample}{The node command}{}
\begin{tikzpicture}
\tikz \draw[fill=yellow!80!black]
(0,0) node {first node}
-- (1,1) node[draw, behind path] {second node}
-- (0,2) node[fill=red!20,draw,double,rounded corners] {third node};

\node at (0,0) [circle, draw] {\textbf{100}};
\node at (1,1) [diamond,draw]{\textbf{100}};
\end{tikzpicture}
\end{texexample}

\subsection*{Drawing shapes}

PGF abd \tikzname\ come with a number of predefined shapes:
\begin{itemize}
\item rectangle
\item circle, and
\item coordinate
\end{itemize}


\begin{tikzpicture}
\draw (0,0) circle (1cm);
\draw (0.5,0) circle (0.5cm);
\draw (0,0.5) circle (0.5cm);
\draw (-0.5,0) circle (0.5cm);
\draw (0,-0.5) circle (0.5cm);
\end{tikzpicture}



A circle is specified by providing its center point and the desired radius. The
command:

\medskip

\begin{tikzpicture}
  \draw[step=0.25cm,color=green] (-1,-1) grid (1,1);
  \draw (0,0) circle (1cm);
\end{tikzpicture}
\medskip

\begin{teXXX}
\begin{tikzpicture}
  \draw (x,y) circle (dia);
\end{tikzpicture}
\end{teXXX}



You  can use one |\draw| command to draw multiple circles as shown in \fref{fig:circles}


\begin{tikzpicture} 
 \draw (0,0) 
  circle (1cm)
  circle (0.6cm)
  circle (0.2cm)
 ;
\end{tikzpicture}

\emphasis{circle,begin,end}
\begin{teXXX}
\begin{tikzpicture} 
 \draw (0,0) 
  circle (1cm)
  circle (0.6cm)
  circle (0.2cm)
 ;
\end{tikzpicture}
\end{teXXX}





\begin{center}
\begin{tikzpicture}
\draw (0,0) circle (1cm)
circle (0.6cm)
circle (0.2cm);
\end{tikzpicture}
\captionof{figure}{You can use one draw command to draw multiple circles}
\label{fig:circles}
\end{center}
\captionof{figure}{Drawing multiple circles, using mutiple \texttt{circle} commands}


\subsection{Drawing ellipses}

Ellipses can be drawn in a similar fashion to circles. As an ellipse needs two center points to be specified the command used has the following general form:

\begin{verbatim}
\draw (a,b) ellipse (r1 dim and r2 dim);
\end{verbatim}

We can draw two ellipses as shown in the figure, using the code:
\begin{teX}
\begin{tikzpicture}[scale=0.6]
\draw[color=red] (0,0) ellipse (2cm and 1cm);
\draw[color=red] (0,0) ellipse (1cm and 2cm);
\end{tikzpicture}
\end{teX}

\begin{centering}
\begin{tikzpicture}[scale=0.6]
\draw[color=red] (0,0) ellipse (2cm and 1cm);
\draw[color=red] (0,0) ellipse (1cm and 2cm);
\end{tikzpicture}
\caption[Drawing ellipses]{Use the draw command in combination with ellipse to draw ellipses}
\end{centering}


\begin{teX}
\begin{tikzpicture}
\draw (0,0) ellipse (2cm and 1cm)
ellipse (0.5cm and 1 cm)
ellipse (0.5cm and 0.25cm);
\end{tikzpicture}
\caption{Drawing multiple circles, using mutiple \texttt{draw} commands}
\end{teX}

\section{Drawing more complicated shapes}
we can place a parabola in a rectangle as shown in \fref{fig:parabola}, by using the |rectangle| and the |parabola| options.

\bgroup
\centering

\begin{tikzpicture}
\draw[color=blue] (0,0) rectangle (1,1.5)
(0,0) parabola[color=orange] (1,1.5);
\draw[xshift=1.5cm] (0,0) rectangle (1,1.5)
(0,0) parabola[bend at end] (1,1.5);
\draw[xshift=3cm] (0,0) rectangle (1,1.5)
(0,0) parabola bend (.75,1.75) (1,1.5);
\end{tikzpicture}
\captionof{figure}{Parabolas drawn using the parabola and rectangle options.}
\label{fig:parabola}
\egroup




\emphasis{parabola,rectangle}
\begin{teX}
\begin{tikzpicture}
\draw[color=blue] (0,0) rectangle (1,1.5)
(0,0) parabola[color=orange] (1,1.5);
\draw[xshift=1.5cm] (0,0) rectangle (1,1.5)
(0,0) parabola[bend at end] (1,1.5);
\draw[xshift=3cm] (0,0) rectangle (1,1.5)
(0,0) parabola bend (.75,1.75) (1,1.5);
\end{tikzpicture}
\caption{Parabolas drawn using the parabola command}
\label{fig:parabola}
\end{teX}

\subsection*{The shape library}

\begin{tikzpicture}
\draw [help lines] (0,0) grid (2,2);
\draw [blue, dashed] (1,1) circle(1cm);
\draw [red, dashed] (1,1) circle(.5cm);
\node [star, star point height=.5cm, minimum size=2cm, draw]
at (1,1) {S};
\end{tikzpicture}

\section{Iterations}
One convenient construct provided with TikZ is a |foreach| command sequence

\begin{texexample}{Tikz loops}{tz:ex}
\centering
\begin{tikzpicture}[scale=2, color=bgsexy]
\foreach \i in {1,...,4}
{
  \path (\i,0) coordinate (X\i);
  \fill (X\i) circle (1pt);
}
  \foreach \j in {1,...,3}
{
  \path (\j,1) coordinate (Y\j);
  \fill (Y\j) circle (1pt);
}
\foreach \i in {1,...,4}
{
  \foreach \j in {1,...,3}
  {
     \draw[color=bgsexy] (X\i) -- (Y\j);
  }
}
\end{tikzpicture}
\captionof{figure}{Drawing a bi-partite garph using foreach loops}
\end{texexample}



\section{The pgfplots package}



\subsection{Loading data from files}

Scientific work, especially that associated with research tends to generate
a lot of data. The data would normally come from external applications and stored in files. With |TikZ| one can import the data
by using the word |file|:

\emphasis{addplot,file,x}
\begin{teXXX}
 \addplot file {./raw/wavefunctions/wavefunc\x.dat};
\end{teXXX}

In the example we use a file with a path. The data is saved in
files with the same name but a different ending. We use a |foreach| function to add the ending i.e, the file names are |wavefunc1|, |wavefunc2| and |wavefunc3|. By using external data files and the foreach command it can substantially reduce the amount of text in the macros. This improves debugging and readability.

\begin{texexample}[colback=white]{Loading files}{ex:lfiles}
\centering
\begin{tikzpicture}[scale=0.8]
    \begin{axis}[smooth,
    xlabel=$n$,
    ylabel=$\Theta{j}{n}$]
    \foreach \x in {0,...,2}
    {
        \addplot file {./raw/wavefunctions/wavefunc\x.dat};
    }
    \legend{$j=0$,$j=1$,$j=2$};
    \end{axis}
\end{tikzpicture}
\captionof{figure}{Example plot with data imported from external files, using \texttt{file}}
\end{texexample}


\begin{teXXX}
\begin{tikzpicture}[scale=0.6]
  \begin{axis}[
    xlabel=$n$,
    ylabel=$\Theta{j}{n}$]
    \foreach \x in {0,...,2}
    {
      \addplot file {./raw/wavefunctions/wavefunc\x.dat};
    }
    \legend{$j=0$,$j=1$,$j=2$};
  \end{axis}
\end{tikzpicture}
\end{teXXX}



\section*{Plotting functions}
Functions can be defined for plotting using a variety of methods. They are powerful but generally difficult to remember.



\section{Saving Data to a file}

You can save your data to a file in many ways. One easy way is to use
the \docpkg{filecontents} package. This package extends the LaTeX environment
with the same name, but allows you to overwrite the file {\protect\ctan{filecontents}}.

\begin{teXXX}
\documentclass[justified]{tufte-book}
\usepackage{pgfplots,lipsum,booktabs}
\usepackage{pgfplotstable}
\pgfplotsset{compat=newest}
\usepackage{filecontents*}
\begin{filecontents}{my1.dat}
    Label       value       num
    Integrity     33         4
    Standalone    14         3
    Interface      6         2
    Overall       18         1
\end{filecontents*}
\begin{document}
    your code here ...
\end{document}
\end{teXXX}

It is good practice to keep, such data at the top of your file, although with
the |filecontents| package, they can be inserted anywhere. Sometimes it maybe
easier to have a number of minimal files with the type of charts you using regularly and just update the data on top. In general if the data is entered
by hand rather than generated automatically by software this is a good way
to keep your work tidy.

\newenvironment{Chart}[1][black!70!green]{%
%%  defaults
    \gdef\level##1{Level ##1}
    \def\setchartwidth##1{%
      \def\chartwidth{##1}}%
    \setchartwidth{3.9cm}%
    \def\chartcolor{#1}
    \newcommand\addTitle[2][test]{
    
    
%% For the chart title we set it in a minipage for
%% better control
    \def\charttitle{\minipage{4cm}%
       \footnotesize %
       \centering\textbf{##2}\\##1%
       \endminipage}}%
   \def\xlabel{Completion (\%)}%
%% renders the chart 
    \def\renderChart{%
%%
    \footnotesize%
%%
%%
    \IfFileExists{#1.dat}{Test}{}
   \begin{tikzpicture}
   \begin{axis}[
    xbar, width=\chartwidth,title=\charttitle,
    y=0.5cm, enlarge y limits={true, abs value=0.75},
    xmin=0, xmax=100,enlarge x limits={upper, value=0.25},
    xlabel=\xlabel,
    %ylabel=Label,
    xmajorgrids=true,
    ytick=data,
    yticklabels from table={\dataTable}{Label},
    nodes near coords, nodes near coords align=horizontal
     ]
    \addplot[draw=none, fill=\chartcolor] table [x=value, y=num]
    {\dataTable};
    \end{axis}%
    \end{tikzpicture}}}
{}

\begin{comment}
\begin{figure*}
\centering

\hskip-2cm\begin{Chart}
 \addTitle[Mechanical Systems]{Shangri-la}
 \def\dataTable{SH-mechanical.dat}
 \renderChart
\end{Chart}\hspace{0.3cm}
\begin{Chart}
 \addTitle[FM-200 System]{All areas}
 \def\dataTable{my1.dat}
 \renderChart
\end{Chart}
\begin{Chart}
 \addTitle[Electrical Works]{Merweb}
 \def\dataTable{my6.dat}
 \renderChart
\end{Chart}
\caption{Mechanical Systems Shangrila. Commissioning status}
\end{figure*}


\begin{filecontents*}{my1.dat}
Label     value       num
Integrity         33            4
Standalone      14            3
Interface        6            2
Overall           18            1
\end{filecontents*}

\begin{filecontents*}{SH-mechanical.dat}
Label     value       num
{Fan coil units}       43             8
{Air Handling Units}       13             7 
{CW Pumps}       13             6
{ECU}       11             5
{Pressurization Fans}        15             4
{Smoke Extract Fan}       5             3
{Jet fan}       5             2
{Overall}       12              1
\end{filecontents*}

\begin{filecontents*}{my6.dat}
Label    value         num   other
{Level 7}  50           11   13
L6         90           10   12
L5       80             9    16
L4       90             8    18
L3       70             7    90
L2       80             6    21
L1       70             5    22
\end{filecontents*}

\begin{filecontents*}{carparkventilation.dat}
Label    value         num   other
L5         50           11   13
L4         90           10   12
L3         80           9    16
GR         90           8    18
B1         70           7    90
B2         80           6    21
B3         70           5    22
\end{filecontents*}
%% CO SYSTEM
%% DATA
\begin{filecontents*}{carparkco.dat}
Label    value         num   other
L5         78           7   13
L4         90           6   12
L3         80           5    16
GR         90           4    18
B1         70           3    90
B2         80           2    21
B3         70           1    22
B5         50          {}    {}
\end{filecontents*}

\begin{filecontents*}{carparkco2.dat}
value,   num,   other,
78,       7,   13,
90,       6,   12,
80,       5,    16,
90,       4,    18,
70,       3,    90,
80,       2,    21,
70,       1,    22,
\end{filecontents*}
\end{comment}






















%%\def\chaptername{Chapter}
%\makeatletter
%\cxset{style13}
\cxset{style87a/.style={
 chapter opening=any,
 name=Chapter,
 % positioning and float - inline is 0
 %  float right is 2
 number display=block,
 number float=right,
 number shape=starburst,
 chapter numbering=arabic,
 number spaceout=none,
 number font-size=huge,
 number font-weight=mdseries,
 number font-family=sffamily,
 number font-shape=upshape,
 number before=,
 number display=inline,
 number float=none,
% 
 number border-top-width=0pt,
 number border-right-width=0pt,
 number border-bottom-width=0pt,
 number border-left-width=0pt,
 number border-width=0pt,
%  
 number padding-left=0em,
 number padding-right=0.5em,
 number padding-top=0em,
 number padding-bottom=0pt,
  %number margin-top=, to do
 %number margin-left=0pt,  to create
 %
 number after=,
 number dot=,
 number position=rightname,
 number color=black,
 number background-color=white,
 %chapter name
 chapter display=block,
 chapter float=left,
 chapter shape=ellipse,
 chapter color=white,
 chapter background-color=sweet,
 chapter font-size= Huge,
 chapter font-weight=mdseries,
 chapter font-family=sffamily,
% chapter font-shape=upshape,
 chapter before=,
 chapter spaceout=none,
 chapter after=,
 chapter margin left=0cm,
 chapter margin top=0pt,
 %
 chapter border-width=0pt,
 chapter border-top-width=0pt,
 chapter border-right-width=0pt,
 chapter border-bottom-width=0pt,
 chapter border-left-width=0pt,
% 
 chapter padding-left=0pt,
 chapter padding-right=0pt,
 chapter padding-top=0pt,
 chapter padding-bottom=0pt,
  %chapter title
 title font-family=sffamily,
 title font-color=black!80,
 title font-weight=bfseries,
 title font-size=huge,
 chapter title align=none,
 title margin-left=1cm,
 title margin bottom=1.3cm,
 title margin top=25pt,
 % title borders
 title border-width=0pt,
 title padding=0pt,
 title border-color=black!80,
% title border-top-color=spot!50,
% title border-top-width=20pt,
 title border-left-color=black!80,
 title border-left-width=2pt,
 title border-color=black!80,
 title padding-top=10pt,
 title padding-bottom=10pt,
 title padding-left=10pt,
 title padding-right=0pt,
% title border-right-color=spot!50,
% title border-right-width=20pt,
% title border-bottom-color=spot!50,
% title border-bottom-width=20pt,
 %
 chapter title align=left,
 chapter title text-align=left,
 chapter title width=0.8\textwidth,
 title before=0pt,
 title after=,
 title display=block,
 title beforeskip=,
 title afterskip=,
 author block=false,
 section font-family=rmfamily,
 section font-size=LARGE,
 section font-weight=bfseries,
 section indent=0pt,
 epigraph width=\dimexpr(\textwidth-2cm)\relax,
 epigraph align=center,
 epigraph text align=center,
 section color=spot!50,
 section font-weight=bfseries,
 section align=left,
 section number after=\hskip10pt,
 section font-family=sffamily,
 section numbering prefix=\@arabic\c@chapter.,
 epigraph rule width=0pt,
 header style=plain}}
 \makeatother
 
\cxset{style87a}


\cxset{ 
           %chapter toc=true,
           chapter numbering=arabic,
           chapter number color=black,
           chapter number font-shape=upshape,
           subsubsection numbering=none,
           subsubsection font-family=itshape,
           subsubsection color=black,
           subsection number after=\quad,
          section number after=\quad,
          section color=black,
    }
\def\thesubsubsection{}         
%\pagenumbering{gobble}
\def\JV{HLS DSE-JV\xspace}
\def\letter#1{\texttt{HLSDSEJV/HC/L/YL/#1}\xspace}
\def\KA{K\&A}
\def\DT#1{HLG Transmittal Ref. No.: \texttt{HLG-626-DT-HLS-#1}\xspace}
\def\idxbusbar#1{\index{Busbar Delays>#1}}
\def\idxwestin#1{\index{Westin Delays>#1}}
\def\idxstregis#1{\index{St. Regis Delays>#1}}
\def\idxahu#1{\index{Air Handling Unit Delays>#1}}
\let\idxahus\idxahu
\def\CAR#1{\index{Cost Adjustment Requests>CAR-#1}{\texttt{CAR-#1}}\xspace}
\def\idxbasement#1{\index{Basement delays>#1}}
\let\basement\idxbasement
\def\idxdewa#1{\index{Dewa Approvals>#1}}


\mainmatter
\pagestyle{plain}
\cxset{chapter name=,
          chapter numbering=none}
\chapter{Executive Summary}
\thispagestyle{empty}

This short report provides background information related to  the Habtoor City Project MEP works and the steps taken by the \JV to accelerate the works, under the instructions of the Client, Engineer and Main Contractor.  We mobilized to the Project late August 2013. At the time construction was on-going, with the basements structures mostly completed. On mobilization the only K\&A MEP designs available were those provided with the tender package---which was issued in March~2013. Besides procurement and some engineering activities, the \JV  construction activities were mainly focused on builder's works and remaining underground services until March 2014. 

We started receiving design drawings in March and April 2014. The design was issued piecemeal and in out of sequence fashion for the works to progress as planned and according to the agreed Baseline Program . This enabled us to proceed with works only in the Car Parking Areas of the Basements.  The first partially workable set of design drawings received to enable construction in other areas were the drawings received in September 2014 (Mechanical) and December 2014 (Electrical).


\medskip
		
\paragraph{Delayed Incomplete and Unworkable MEP Designs} The general issue of drawings in September~14, provided general design concepts without concerns for physical plant and ceiling constraints. The Plant rooms at T1 and PD6, as designed were not constructible, as the allocated headroom and space was inadequate. We assisted the Engineer by providing 3D and other drawings to at least fit the equipment in the available space. Fans had to be relocated in ceilings at Podium 1, and ducting was re-routed over the same ceiling void. This delayed finalization of Shop Drawings for essentially all the public areas.


The K\&A \enquote{design} mechanical design for St. Regis was only partially completed in September 2014. This design was deficient in many respects, especially in areas such the Technical floors, and as it stood the design was not constructible. This design was inadequate to close equipment orders for long delivery plant, such as AHU, fans and pumps, as calculations for static pressures could not progress. However, we took the initiative to finalize orders based on estimates and released orders before design finalization. We also assisted the Engineer with solving many of the design issues in order to progress with the works.  In addition the Electrical works suffered because of the designs issued in September 2014, as they have not been co-ordinated with the requirements of the Mechanical plant, Kitchen Contractors etc. \par

The delays  to the completion of the final Project requirements are still on-going with many areas of the Hotels still under design development and without related subcontractors appointed on time.

\begin{table}[ht]
\centering

\begin{tabular}{l l p{3cm}  l l}
\toprule
        &Area         &\raggedright Design required as per baseline program & Design Issued & Delay\\
\midrule        
\inc  &First Floor &16 Apr 14  &5 Jan 15  &  8 months\\
\inc  &Attic Floor & 8 Apr 14  &5 Jan 15  & 8 months \\
\inc  &Podium 6  &1 Apr 14   &6 Sep 14  & 5 months \\
\inc  &Podium 5  &29 Mar 14 &6 Sep 14  & 5 months\\
\inc &Podium 4   &20 Mar 14 &6 Sep 14  & 5.5 months\\
\inc &Podium 3  &12 Mar 14  &6 Sep 14  & 5.5 months\\
\inc &Technical 1 &6 Feb 14   &6Sep 14    &7 months\\
\inc &Mezzanine &26 Dec 13  &6 Sep 14  &9 months\\ 
\bottomrule
\end{tabular}
\caption{Design delays for St. Regis}

\end{table}

The MEP Good for Engineering Designs as received from K\&A enabled part of the Engineering and Procurement activities to start bu the design as it stood was  proceed, they are not sufficient to install MEP services. Drawings from ID Consultants, Lighting Consultants, Kitchen Consultant, ELV Consultants and subcontractor Shop Drawings for the same are necessary. These were mostly unavailable.

\paragraph{Instruction to accelerate the works}
Under this background we received the instruction to  accelerate the works (July 2014). We wrote to to the Main Contractor, requesting that a plan be first agreed as to how program recovery could be achieved and \emph{then} agree to a plan to accelerate the works further, so as to bring the Contract Completion dates forward. The request was to accelerate the St. Regis Hotel first with a Target Completion date of 30 March 2014.

At the time approximately 40\% of the slabs  for St. Regis were incomplete. This included critical plant areas at the two technical floors. Not only the structure had to be completed, but also the technical floor, had to have floating floors casted. The T1 floor was partially handed over to us end October and the PD6 floor in January 2015. As is also evident from the subsequently issued Design MEP Drawings, ID Drawings, Lighting Consultant and ELV Consultant drawings issued, the Professional Team was not ready with their Designs. 

As MEP works are closely interlinked with other trades it is important to note that the Structure Cabling, Kitchen Subcontractors, AV and CCTV Subcontractors were not appointed. 
\medskip

 
   


\label{acceleration}
\index{acceleration>manpower}\index{manpower>acceleration}
\paragraph{JV actions taken to accelerate the works.} Once the information started flowing, we reinforced our Engineering and Site Teams. We also added technicians as areas opened to us for work.
\medskip

\noindent\textit{Workforce}
\medskip

\noindent The \JV upon receipt of the instructions to accelerate, and under the impression that designs and appointments of other subcontractors would be accelerated as well, doubled the workforce in July~2014 and subsequently added technicians and other staff until it is at its current level of approximately 3000 personnel. The deployment of personnel is shown in the table below.

\begin{table}[hbp]
\begin{tabular}{c c c c c c c c c}
\toprule
Item &Sep 13 &Feb 14 &Mar 14 & Jul-14 & Aug-14 &Oct-14 & Jan-15 & Mar-15\\
\midrule
 Site Labour   & 48      &610      & 634     & 1212   &  1300     & 1845   &2 781   & 2 731 \\
\bottomrule
\end{tabular}
\end{table}

Although issues prohibited us from fully handing over areas and ceiling closures, the quantum of the work achieved in this short time can be gauged from the gross claimed amount of close to AED~280,000,000.00. (April~14-April~15). 
\medskip

\noindent\textit{Air-freighting of equipment}
\medskip

\noindent In addition to adding personnel we proceeded to air-freight the following equipment, without which the program recovery would have failed:

\begin{enumerate}
\item Chilled water pumps. The chilled water pumps were necessary to be delivered as early as possible in order to enable piping to be connected and for providing wild air as possible. The first submittal for pumps was made on the 25 February 2014. This was returned on the 26 March 2014. The pumps were again resubmitted in 23 April 2014, after revisions to match changes in equipment. They were returned after 40 days, despite the fact that at the time the Engineer was asking us to accelerate the works. Third and fourth submittals followed and the pumps finally approved on 3 July 2014. Pump heads were reverified to meet new layouts and the order place in August, after opening LCs and finalizing prices with Supplier. Cost AED 50,000.00. 
\index{airfreight>chilled water pumps}
\index{chilled water pumps>air freight costs}

\item First fan coil units deliveries for St Regis. These were subjected to similar delays and 388 fan coil units were air-freighted from Thailand at a cost of 196,539.60~AED. 

\item Air Separator for the St Regis Plantroom was air-freighted at a cost of AED~14,185.00.
\item All fans for St Regis. Many of these fans were to be installed in ceiling voids. These were air-freighted at a cost of AED~221,772.00. This also included air-freighting charges for fire rated motors to be air-freighted from Brazil to the Nuaire Wales factory.

\item The above secured the St Regis Hotel plant room areas.

\item Air-freighting of ECUs and Basement fans was stopped after Client Representative wrote us a letter that they would not consider paying for the above costs. \index{Ecology Units>air freight} \index{air freight costs}

These were sea-freighted, with a consequent further compression in the program of works and delaying completion of the following areas:

\begin{enumerate}
\item Basement areas
\item PD6 St Regis Plantroom
\item Kitchens
\end{enumerate}
\end{enumerate}


These are also expected to delay commissioning of kitchen areas in the basement and the Car Park Ventilation System.

\paragraph{Focus of the claim}

The claim should focus on the following:

\begin{enumerate}
\item Establishing the time extension claim. This should not be too difficult given the delays in design information. Also the casting on all buildings had considerable delays (recorded in the weekly  reports). The biggest delays in casting occured in the "W" hotel. Considerable delays in the issue of provisional sums information was another source of delay. These are listed in the last section of this report. Engineering has all the details as to when the information was released to us. I have recorded the dates we required the information in order not to incur delays.

\item Establishing disruption. Unless this element can be argued successfully, the monetary claim will be insignificant. We should at least try and recover 20\% on the labour component.

\item Establishing acceleration. The Site Team needs to provide the Claim Consultant's an accurate status of the Project and details as to what is still incomplete. Please give attention to teh fact that there are two  non-binding contractual milestones. the completion of basement works and teh completion of the St. Regis Hotel by July 2015. These milestones need to be achieved in order to validate the acceleration part of the claim.
\end{enumerate}

Between the above three components we should be able to Claim in excess of  AED 30 million, which hopefully would recover the higher labour costs incurred.

The Sections that follow are general outlines and an incomplete list of what can be claimed. 
Full information is available with the Engineering Team and the Commercial Team.

\setcounter{chapter}{0}





%\chapter{Summary MEP Progress Report for St Regis Hotel, Habtoor City}
%\pagenumbering{arabic}
%\thispagestyle{plain}
%\section{Current Status}
%
%We have started flushing of the Chilled Water system on the 7 April 2015, as planned and we anticipate to be in a position to progressively provide \emph{wild air} before the end of April, ahead of the scheduled date of the 7 May 2015. In the Basements and in the Guest rooms we have started final fix works, where possible. The Main Plantrooms at Technical Floors 1 and Podium 6, are in the main completed, except final ductwork connections where they impede access. BMS DDC Panels are expected to arrive by the 22 April 2015 and installation expected to be completed within 25-30 days to ensure that by end May we can provide controlled conditions.
%
%Delays have been experienced in the receipt of Electrical panels, such as DBs (delayed due to late deliveries of components by Legrand) and others that were subjected to numerous changes, as described later on.  Other delays were due to late instructions as briefly detailed in Section~\ref{delays}. 
%
%The current outstanding works for the St Regis Hotel are as follows:
%
%\subsection{St Regis Basement}
%
%\begin{description}
%\item[Kitchen Corridors] Some kitchen corridors cable pulling is still under progress. Expected to complete by 30 Apr 2015.
%\item[Main Electrical Room] Delays experienced due to the failure of cable trays during cable pulling and also due to the some of the MDBs being returned to the factory for modifications, as they failed QA/QC Inspections.
%\item[Fan Rooms] Fans scheduled to be delivered 23 Apr 2015.
%\item[BMS] DDC Panels still to be delivered.
%\item[Sump Pumps] Expected to be delivered by 10 May 2015. 
%\item[Others] There are still closure related works, for areas currently inaccessible, such as the new ramp areas, store and office areas. 
%\end{description}
%
%\subsection{Ground Floor}
%\begin{description}
%\item[Ballroom] This area is still under scaffolding being used by the Main Contractor to erect walk-ways in the ceiling. Once the scaffolding is dropped and we are given access to the lower level, we have to install another layer of services, give ceiling grid clearances and upon construction of the ceiling grid we can then install final sprinkler droppers and give clearances for final boarding.
%\item[Banquet Hall] This area has been delayed due to the Iridium Spa delays in Design and appointment of subcontractors. As this area is above the Banquet Hall, coring for drainage pipes delayed the works. This coring is now complete and we expect to ask the Main Contractor to lower the scaffolding and start with the rest of the services.
%\end{description}
%\subsection{Mezzanine}
%\begin{description}
%\item[Festival Dining Restaurant] Currently this area is under nomination, there is no ID Design and final details are still awaited. 
%\item[Security Room] The design for this room has recently changed. The room as shown in the new designs is different from what has been constructed on site and has no space for CCUs. 
%\item[AV Room] Expected to be completed 30 Apr 2015.
%\item[Furniture Store] Expected to be completed 25 Apr 2015.
%\item[Balance Corridors] Expected to be completed 25 Apr 2015.
%\end{description}
%
%\subsection{Podium 1}
%
%\begin{description}
%\item[Banquet A/V Technician] We have no access. This is currently being used as a store.
%\item[Service Corridor] Plan to release for ceiling grid on 23 Apr 2015.
%\item[St Regis Main Kitchen and Corridor] Plan to release on 30 Apr 2015.
%\item[Property Store] Currently no access. If access provided we can release by 30 Apr 2015.
%\item[Steak House Kitchen] Plan to release by 30 Apr 2014.
%\end{description}
%
%\subsection{Podium 2}
%\begin{description}
%\item[Iridium Spa and related areas] We are currently working in the area, which was delayed by late appointment of Finishing Contractor. Still some ID Shop Drawings not available. We expect to catch-up with delays by end May 2015. We plan to complete final fix by 10 June 2015 and Testing and Commissioning by 20 Jul 2015.
%\item[Other Areas] All other areas will be released for closure by 26 Apr 2015.
%\end{description}
%
%\subsection{Podium 3-6}
%
%All guestrooms have been handed over for ceiling closures with the exception of some of the suites, where information and access was provided late. These are the following:
%
%\begin{description}
%\item[Ambassador Suite] Co-ordination ongoing. Expect resolution and final clearances 25 May 2015.
%\item[Bentley Suite] Incomplete information. Completion targets uncertain at this stage.
%\item[Royal Suite] Co-ordination on-going. Expect resolution and final clearances 25 May 2015.
%\end{description}
%
%\subsection{Floor 1}
%
%\begin{description}
%\item[Kitchen 4 and Kitchen 6] Works for walls are progressing, insufficient detail information. Can complete by 15 May 2015, provided all Kitchen Subcontactor’s drawings become available and unimpeded access.
%\end{description}
%
%\section{Delays in Target Dates}
%\label{delays}
%This is a brief summary of recent selected instructions for additional works that have impacted  MEP Progress. 
%In addition to these additional works another critical factor that affected progress was the congestion of services and the numerous RFIs and responses we had to raise in order to resolve them.
%
%\begin{itemize}
%\item Relocation of Kitchen Extract ducting Ground Floor, Mezzanine and Podium BOH areas.
%\item  Additional AV points in all public areas.
%\item  Additional telephone, data and CCTV points in all Public Areas.
%\item  Motorized curtains Meeting Rooms.
%\item Lighting Control System. 
%\item Emergency Lighting System. (see details Chapter~\ref{emergencylights})
%\item Changes to Electrical DBs, SMDBs due to late receipt of DEWA approved drawings. (See Chapter~\ref{electrical})
%\end{itemize}
%
%We have reacted as fast as possible to all instructions and as soon they were received we have added resources to mitigate delays. Where days slipped these are only by a few days and we are confident that by end of this month all physical installations will be completed with the exception of the English Pub, Banquet and Royal Suite. 
%
%\subsection{Back of the House Areas}
%
%All back of the House Areas experienced delays, due to the lack of primary co-ordination at design stage. This caused delays until solutions were found enabling us to install the services. 
%
%The allowable ceiling height in this area was impossible to be achieved and the kitchen extract duct eventually was split in two sections and distributed through two different routes in order to avoid passing it through the corridors which could not accomodate it.
%
%In addition a new roller shutter window was introduced, that made it impossible to install the fresh air ducts feeding the kitchen. After several attempts by |K&A| to find an acceptable solution the roller shutter  window was abandoned as per the instructions of the Client Representative. 
%
%\subsection{Basement Kitchen and Related Areas at B1}
%
%Please note that these areas (with the exception of the corridor) have been cleared for ceiling grid closures in most areas and the balances are as per target to close by the 15 April 2015, including additional works. The additional works were mostly for additional ELV points on walls and for which we have received drawings on the 29 March 2015. We have instituted overtime and added additional crews to complete the works as fast as possible. Most rooms in the area have been affected. 

\begin{comment}
\chapter{Busbar System}

As per the approved Baseline Program we expected to place the busbar order for all three hotels on 27 February 2014. However, HLS DSE-JV were unable to place any orders due to the events that are outlined below, with finality on all busbars only achieved in April 2015. 

\begin{enumerate}
\item On the 23 December 2013 we were requested to change the specification for some busbars via HLG transmittal Ref. No. HLG-626-DT-HLS-0628 dated 23 Decemeber 2013 \textit{Fire Resistance Bus Bar Specification}.

\item On the 25 February 2014 we were issued revised designs via tranmittal Ref. No. HLG-626-DT-HLS-0873 \textit{Revised Electrical Drawings}.

\end{enumerate}


\chapter{Generators}

\section{Generator Ventilation}

\subsection{Background}

The original tender drawings indicated the Generator Ventilation to be by means of Louvres. When such an approach is taken normally the ventilation openings are dictated by the size of the generators.


HLS DSE-JV have submitted as early as 2014 RFIs outlining concerns regarding the adequacy of the ventilation openings and sizing of Generator rooms in the basements.

On the 25 March 2015, we were instructed to proceed with the purchase of additional fans from Systemaire. We issued the order request on the ..... and the order placed on the ......  without formal approval of the amounts in order to speed up the purchase. This affected the commissioning of the generators.

\chapter{Transformer Room Ventilation}

\subsection{Background}

\subsection{Design Errors}
\end{comment}


\cxset{chapter name=Section,
          chapter numbering=arabic}
\chapter{Emergency Lighting System}
\label{emergencylights}
The Emergency Lighting System was finalized on the 22 February 2015. This is impacting on the final fix and commissioning of the Hotel’s Central Battery and Emergency Lighting System. 

\begin{enumerate}
\item As per the approved Baseline Program, we were planning to submit the Material Submission of the Emergency Lighting System by the 25 Feb 2014.
\item On the 25 Nov 2013, we raised RFI \texttt{HLS-DSE/142 JV-RFI-MEP-E028} requesting full details of the Emergency Lights as well as the capacity of the central battery system in order to proceed with Technical Submittals, design of containment system and procurement of equipment.
\item On the 12 Dec 2013 we received an insufficient reply to the above mentioned RFI. We have notified you that the repsonse was insufficient via letter \texttt{HLS DSE/JV/HLG/YL1181} dated 14 Jan 2014, clearly stating that we were unable to proceed further with the submission of the Central Battery System, until the requested information was provided. In our letter we had requested that all details such as diffuser details, base type, IP rating and lamp characteristics are provided. We have also provided details as to Civil Defence requirements.
\item The above concerns were forwarded to the Engineer by the Main Contractor on the 20 Jan 2014. The Engineer instructed us to follow the current design dawings until the completion of the Lighting Consultant’s works.

\item On 10 Feb 2014, we had responded via letter \texttt{HLS-DSE/JVHLG/YL/1227} stating that the information provided by the Engineer, as response to RFI HLS-DSE/142 MEP-E028 was inadequate to produce Shop Drawings and to proceed with material procurement or calculations.
\item On 19 March 2014, once again we responded via letter HLS DSE/JV/626/2.05/YE/nd/2609/14 dated 4 Mar 2014 stating that the inforamtion was inadequate.
\item On 16 April 2014 we sent a clear notification that the lack of information was expected to delay the works via letter \texttt{HLS DSE/JV/HC/L/YL/1322} stating that we were unable to proceed with this portion of the works.

\item On the 20 August 2014 we received via an email instructions to proceed based on a generalized scheme.
\item We raised RFI-MEP-E249 dated 21 Sep 2014, requesting more details on locations and quantities of Emergency Light Fittings. The RFI response was received on 13 Oct 2014 with the response to follow the latest issued Guest Room drawings. 
\item Engineer’s letter \texttt{DU1211/DU/L20054/14} dated 15 Sep 2014, confirmed that due to several ID Design issues the above details were no longer applicable.
\item On 30 Sep 2014 we served notices regarding additional works due to revisions of the Emergency Lighting System for all three hotels.
\item On 15 Nov 2014, we raised concerns due to late finalization of the Central Battery System for W and Westin Hotels. 
\item On the 20 Dec 2014 the we received instructions from the Engineer and Client requesting us to revert back to the original K\&A designs.
\item On the 22 Feb 2015, the Engineer instructed us to procure and install all the Front of House exit lights. We confirmed receipt of the instruction via letter \texttt{YL/1935} dated 24 Mar 2014 once all final details and samples were finalized.
\end{enumerate}


















  
%\chapter{Dewa Approval}
\label{ch:dewa}

As per the approved Baseline Program, we were expected to receive Dewa approved drawings on the 28th November 2013. However, HLS-DSE JV received the LV approved drawings on 15th July 2014, as per HLG transmittal reference No. HLG-626-DT-HLS-1397 dated 15th July 2014. This delayed finalization of orders and progress on site.

 In particular:
 
 \begin{enumerate}
 \item Cables cannot be ordered until such time as approved single line diagrams are available. Once these become available  Shop drawings are prepared and main panels can also be finalized.
 \item MDBs and SMDBs can be finalized and ordered.
 \item Completion of Generator Rooms.
 \item Completion of Transformer and LV Rooms.
  \end{enumerate} 
  
\section{Action by the HLS DSE-JV}
  
Given the enormous task at hand and the instructions received to accelerate the works, we added an Electrical Engineering Manager to assist the Team with the task at hand. We also added additional CAD Operators.

\section{Design Deficiencies}

The Dewa drawings were out of step with the latest revisions of other drawings in terms of architectural, HVAC, Kitchen requirements and other equipment. They also underestimated both the main power required by 2.5MW, as well as the stand-by power required, leading to revisions to the Generator Plant. The Generator Plant is handled under delays of Electrical equipment.

Normally once drawings are submitted and approved by Dewa, the design can be considered complete, however, many areas remained incomplete.

\begin{enumerate}
\item On 20th August 2014, we requested by letter HLSDSEJV/HC/L/YL/1502 to be issued officially a number of revisions we received via email correspondence for the St Regis Hotel. 

\item On 21st August 2014 we confirmed receipt of revised Electrical Drawings from Ground to First Floor via letter ref. no. HLSDSEJV/HC/L/YL/1524. (\CAR{0076})\idxdewa{21 August revisions}

\item On 1 September 2014, we issued delay notice for revised electrical drawings, received by email for St. Regis via letter ref. no.: HLSDSEJV/HC/L/YL/1533 (CAR 83).\CAR{0083}

\item On 11 September, 2014 we confirmed via letter 

\item On 8 September 2014 we received further changes to Electrical Drawings for Westin via HLG Transmittal Ref. No. HLG-626-DT-HLS-1671 dated 8 September 2014.

\item As there was uncertainty over which drawings were to be used, HLG issued us a letter from the Engineer dated 11 September 2004, confirming the following:

      \begin{enumerate}
    	\item  St. Regis Hotel - Electrical Design drawings to be followed as per 1 September 2014 issue drawings (DU/L/18451/14).
    	\item Westin Hotel - Electrical Design drawings to be followed as per the 4 September issued drawings (DU/L/18896/14).
    	\item W Hotel - Electrical Design drawings will be issued after incorporating new Restaurant and ID drawings.
	  \end{enumerate}
\end{enumerate}

\section{30 December 2014 Dewa approved drawings issue}
\label{electrical}

On the 31 December 2014 final Dewa revised drawings were issued. This incorporated further revisons to electrical panels, additional SMDBs, changes to cable sizes, breaker sizes etc. Delay notice was served via letter Ref: YL/1796 date 19/1/2015. Changes affected all areas, including basements, St Regis, Westin and W Hotels. We wrote to the Engineer with suggestions to minimize the impact via letter Ref 25 January 2015 and recording the changes. For the W \& Westin Hotel we did the same via letter ref YL/1843 dated 3 February 2015 (\CAR {0136}).\CAR{0126}

The letters remained unanswered and we issued reminder letter related to these changes via letter ref YL/1907. We also confirmed that the works ere put on hold until such time as we had received confirmation from the Engineer.

Additional works as per letter \texttt{HLG/626/2.05/YE/es/7312/15} dated 6 April 2015. These changes relate to late approval of DEWA drawings. These changes affected all the hotels.

These revisions to the electrical design obstructed us from finalizing and ordering the Electrical Panels including MDBs, MCC, SMDB and electrical cables. The final impact of these changes is described below.

\section{St. Regis}
The following changes were instructed via the above letter and were based on drawing number |EM3300|.
\begin{description}
\item[SMDB-H1-1PLBPR] The works adds outgoing cables feeding |ADD-SS-01| and for |DBP-H1-1PLBPR1|  the cable size was changed from 4c:10mm2 XLPE to 4c:16 mm2 XLPE. The breaker size was changed to 60A MCCB.

\item[SMDB-H1-1TEFCWF] The instruction requests the changing of 15A breaker to 20A for eight CP-H1-1-TEWF/05 T.C.L.-1kW and one CP-H1-1TEWF/09 T.C.L.-1kW.

\item[SMDB-H1-2PL] The instruction requests the following changes:
   \begin{enumerate}
      \item DBP-H1-2PL MCCB 60A change to 80A and cable size 4c:16mm2 XLPE change to 4c:25mm2 XLPE.
      \item BPN-PN-16 and 18 30mA ELCB added.
   \end{enumerate}

\item[SMDB-H1-2PSPA] The instruction requests the following changes:
    \begin{enumerate}
      \item Male and female Jacuzzi bath MCCB 15A change to 20A TCL-3kW.
      \item Female steam room cable size changed (4c:10mm2 XLPE to 4c:16mm2 XLPE).
    \end{enumerate}


\item[SMDB-H1-6PL] The instruction requests the following changes:
   \begin{enumerate}
      \item Additional outgoing feeders for EC-01B, EC-02B, WET-PN-011, WET-PN-017 and WET-PN-020.
      \item 40ATP MCCB removed for FP-H1-1FL
   \end{enumerate}

\end{description}

The following changes were due to drawing No:EM3301

\begin{description}
\item [MDB-H1-B1R1] The instruction requests the following changes:
    \begin{enumerate}
       \item SMDB-H1-GR2 MCCB 200A change to 225A and cable size 4c:95mm2. XLPE change to 4c: 120mm2 XLPE (TCL 116.8kW).
       \item UPS MCCB 60A change to 80A.
    \end{enumerate}
\item[MDB-H1-GR2] The following changes were instructed:
    \begin{enumerate}
       \item Incomer MCCB 200A TP change to 225A TP.
       \item Additional outgoing for WPN-PA-012, WPN-PA-032.
    \end{enumerate}
\end{description}

The following changes were due to drawing No:EM3302

\begin{description}
\item[SMDB-H1-GLSTBR] The following changes were requested:
   \begin{enumerate}
      \item Additional outgping for St Regis, Special Event, St Regis BR.
      \item DBP-H1-GLSTBR MCCB 60A change to 80A.
   \end{enumerate}
\item[SMDB-H1-1PLMK] The following changes were requested:
      \begin{enumerate}
        \item Additional space.
        \item 60A TP MCCB removed.
      \end{enumerate}  
\item[SMDB-H1-2PGSC] The following changes were requested:
     \begin{enumerate}
        \item CAF-SS-01 cable and MCCB size changed from 4c:70mm2 XLPE and 150A TP to 4c:XLPE and 30A TP (TCL-6.5kW).
     \end{enumerate}
\end{description}

The following changes were detailed on drawing No:EM3303

\begin{description}
\item[MDB-H1-B1LR2] SMDB-H1-GL MCCB80A change to 100A.
\item[SMDB-H1-GL] DBP-H1-GLPFA MCCB 60A change 80A and cable size 4c:16mm2 XLPE change 4c:25mm2 XLPE.
\item[SMDB-H1-GLBP1] Additional outgoing for BOQ-KIT-016.
\end{description}

The following changes were detailed on drawing No:EM3304.

\begin{description}
\item[EMDB-H1-B1]  The following changes were requested:
   \begin{enumerate}
      \item ESMDP-H1-GR2 MCCB 80A change to 150A and cable size 4c:35mm2 XLPE change to 4c:70mm2 XLPE.
      \item ESMDB-H1-6PMS1 MCCB 400ATP change to 500ATP.
   \end{enumerate}
\item[EMDB-H1-6PMS1] The following changes were requested:
    \begin{enumerate}
       \item Incomer MCCB 400ATP change to 500ATP.
       \item Additional outgoing for EC-01A,B and future load.
       \item ESMDB-H1-RS cable size changed from 4c:70mm2 XLP (125A TP to 4c:95mm2 XLPE (TCL-55kW).
    \end{enumerate}
\item[ESMDB-H1-GL]
\item[ESMDB-H1-6PMS2]  The incomer to MCCB was changed from 700A TP to 800A TP.
\item[ESMDB-H1-6PL] An additional outgoing cable was requested for EC-02A. For LIFT-H1-SL05 and LIFT-H1-SL06 the cable size was requested to be changed to 4c:35mm2 MGT/XLPE.
\item[ESMDB-H1-2PL] CAF-SK-012, EC-01B MCCB and cable size changed from 30A SP 2c:16mm2 XLPE to 20ASP and 2c:4mm2 PVC (T.C.L.-2.6kW and 0.8kW).
\end{description}

\subsection{St Regis Basement Areas}
The following changes were detailed on drawing No:EM3200.\idxdewa{basements}\idxbasement{SMDB revisions}
\begin{description}
\item[SMDB-BP-1BS1] Additional outgoing circuits were requested for DB-LS-SR2, DB-LS-SR3.
\item[SMDB-BP-1BS3]  An additional outgoing circuit was instructed for DB-LS-SR5.
\item[SMDB-BP-1BS5] An additional outgoing circuit was requested for DB-LS-SR6.
\end{description}

The following changes were detailed on drawing No:EM3201.

\begin{description}
\item[EMDB-BP-1B3] ESMDB-BP-1BS7 MCCB 40A change to 80A and cable size 4c:10mm2 XLPE change 4c:16mm2 XLPE (TCL-17.8kW).\idxbasement{EMDB revisions}\idxbasement{ESMDB revisions}
\item[ESMDB-BP-1BS9] cable size 4c:35mm2 XLPE change to 4c:70mm2 XLPE (TCL-44.4kW).
\item[ESMDB-BP-1B3]  The following changes affected this panel:
     \begin{enumerate}
        \item ESMDB-BP-1BS7 MCCB 40A change to 80A and cable size 4c:10mm2 XLPE change to 4c:16mm2 XLPE    (TCL-17.8kW). 
        \item ESMDB-BP-1BS9 cable size 4c:35mm2 XLPE change to 4c:70mm2 XLPE (TCL-44.4kW). 
        \item ESMDB-BP1BS10 cable size 4c:240mm2 MGT change to 4c:300mm2 MGT(TCL-120kW).
     \end{enumerate}
\item[ESMDB-BP-1BS2]  3 Nos CP-BP-1BE/F1 cable size 4c:16mm2 MGT change to 4c:25mm2 MGT (TCL-17kW).
\item[ESMDB-BP-1BS3]  20ATP, pulse meter, 10mm2 MGT removed for SPCP-BP-1B12.
\item[ESMDB-BP-1BBPA] Incomer MCCB 80A TP change to 100A TP.
\item[ESMDB-BP-1BCOM1] Additional outgoing for COM-IC-001, COM-IC-002, COM-IC-003, COM-IC-006.
\item[USMDB-BP-1BS] UDB-BP-1BS4 and UDB-BP-1BS5 cable size 4c:10mm2 XLPE change to 4c:16mm2 XLPE (TCL-8.8kW and TCL-7.6kw).
\item[ESMDB-BP-1BS1] Incomer MCCB 200A TP change to 250A TP.
\item[ESMDB-BP-1B] ESMDB-BP-1BBPA MCCB 80A change to 100A and cable size 4c:50mm2 XLPE change to 4c:70mm2 XLPE(TCL-44.8kW).
\end{description}

The following changes were due to additional works detailed on drg No: EM3204.

\begin{description}
\item[MDB-BP-2BMEC]
   \begin{enumerate}
     \item Incomer MCCB 80A TP change 100A TP.
     \item FPCP-H1-2B2 cable size 4c:10mm2 XLPE change to 4c:16mm2 XLPE.
     \item FPCP-H1-2B1 cable size 4c:6mm2 XLPE change to 4c:6mm2 XLPE change to 4c:10mm2 XLPE (TCL-5.5kW).
   \end{enumerate}
\item[SMDB-FB-2BMEC]
\end{description}

The following changes were due to drawing No: EM3206.
\begin{description}
\item[MDB-BP-1B2] 
    \begin{enumerate}
       \item MDB-BP-1BCOM Additional outgoings for COM-MP-041.
       \item SMDB-BP-1BS6 MCCB 400A change to 500A and cable size 2x4c:120mm2 XLPE change to 2x4c:150mm2 XLPE (TCL-221kW).
       \item 400A TP+2x4c:120mm2 XLPE removed for FFP-3.
    \end{enumerate}
\item[SMDB-BP-1BS6] Additional outgoing for DB-LS-SR4.
\item[SMDB-BP-1BS10] Additional works were requested as follows:
    \begin{enumerate}
      \item DB-H3-1BSS2 cable size change to 2c:10mm2 XLPE change to 2c:16mm2 XLPE (TCL-1.2kW).
      \item CP-BP-1BTE/F4 cable size change to 4c:16mm2 XLPE change to 4c:25mm2 XLPE MCCB 40A TP Change to 60A TP (TCL-25kW).
      \item CP-BP-1BTF/F2 and CP-BP-1BTE/F2 MCCB 60A TP change to 80A TP (TCL-37kW).
      \item CP-BP-1BTF/F3 and CP-BP-1BTE/F3 MCCB 40A TP change to 60A TP (TCL-22kW and 25kW).
    \end{enumerate}
\end{description}







 
%
\chapter{Delays in Finalizing Requirements for the Busbar System}

As per the approved Baseline Program\footnote{Issued 4 Jan 14 and approved 9 Jan 14, as per HLG letter Ref: HLG/626/2.5/SO/nd/1862/14},  we were planning to order the Busbar on 27 February 2014. The \JV was unable to finalize the Bus Bar Material Submittal due to the numerous revisions issued and the lack of Dewa approved drawings.

\begin{enumerate}
\item On 23 December 2013 we received HLG transmittal ref: no. HLG-626-DT-HLS-0628 dated 23 December 2013 ``Fire Resistance Bus Bar Specification'', instructing us to change some of the busbars to fire rated busbars.
\idxbusbar{change in specification}\idxbusbar{fire rated}
\label{fireratedbusbar}

\item HLG transmittal Ref: No.: HLG-626-DT-HLS-0797 dated 10 February 2014 titled ``Electrical Updated Coordinated Drawings for Basements". (\CAR{0004}).

\item HLG Transmittal Ref. No.: HLG-626-DT-HLS-0873 dated 25 February 2014 ``Revised Electrical Drawings''.

\item On 18 and 20 March 2014 via \DT{0930\&939} we were issued updated drawings for three Hotel (\CAR{0036}).  

\item \DT {10127} dated 10 April 2014 ``Revised Electrical Drawings''. 

\item On 16 June 2014 we were instructed to stop any works on Westin and W Hotel Bus bars due to ``significant comments on Dewa LV approval''. This instruction was received via \KA letter ref. no. DU1211/DU/L/13086. At the time we had completed LV Schematic drawings and had the bus bar isometrics completed. 

\item The approved Dewa LV drawings were received on 15 July 2014 as per \DT {1397} dated 15 July 2014. One month later than the instruction to hold the orders and the works.

\item On 20 August 2014, we requested that we be issued officially the St Regis Hotel revised drawings that we were being send piecemeal by email. \letter{1515}.

\item On 1 September 2014 we had responded to HLG letter ref. no. HLG/626/1.12.AMM/es/4109/14 expressing concerns as to delays in receiving receiving workable design drawings (\letter{1533} (\CAR{0083}).

\item We received revised Electrical Drawings for Westin via \DT {1671} dated 8 September 2014.

\item On 17 September 2014 we received Electrical Drawings for W Hotel via \DT{1728}. 

\item On 27 September we served notice for additional costs and time due to revised electrical drawings related to Mechanical Equipment. The additional loads were received as response to RFI/MEP/E295 for all three Hotels (\CAR{0083}, \CAR{0086}\& \CAR{0087}).

\item On 2 October 2014 we served notices for additional works due to revised Electrical Drawings for Food and Beverage areas. (\CAR{0093}) {\CAR{0093}}

\item We served notice for drawings issued to us for W Hotel from 24 to 27 floor (\CAR{0112}).

\item On 19 February 2015 we issued estimated costs at the Engineer's request for the fire rated bus bars. On 25 March 2015 we were instructed to change all fire rated bus bars to normal bus bars. Accordingly the procurement of this particular bus bars took from 23 December 2013 until 19 February 2015 to be concluded and delayed the works.  (See \ref{fireratedbusbar} referred to when the first instruction was received. )


\end{enumerate}
%

\chapter{Delays in Engineering, Procurement and Construction due to Frequent  Design Changes}

As per the approved Baseline programme the HLS-DSE JV were expecting to receive Good for Construction (GFC) drawings on 14 September 2013 for all areas. This would have ensured that the Contract dates could have been met. GFC drawings are an industry practice by which the Engineer signals the completion of the design and the avoidance of errors omissions and delays by using drawings such as those marked as GFE (Good for Engineering). 

The Consultant has been unable to finalize the design on time and drawings and designs were provided mostly reactively to requests and notices by the Contractor. This has subsequently caused disruption to the \JV Engineering, Submittals, Procurement and Work Progress activities.

\section{Design Changes}

Many design changes were as a response to the \JV RFIs. As of today more than 1300 RFIs have been issued. The events described below are more or less in reverse chronological order from the more recent to the earliest.

Design changes can in general be grouped in the following categories:

\begin{enumerate}
\item As responses to RFIs to resolve, space constraints. The Engineer's design was not coordinated with the basic architectural and structural design. This was most acute in the St Regis Hotel, where large beams and inadequate floor to ceiling height resulted in congested  areas. This was not evident during the tender process and has resulted in additional costs to the Contractor.

\item As responses to RFIs to provide further information due to lack of completeness of the design.

\item Errors and omissions, which either the Engineer corrected or as responses to RFIs.
\end{enumerate}

\begin{enumerate}
\item Variation notification for HVAC works was served under \letter{YL/1922 \& 1931} dated 15 March 2015 and 23 March 2015 (CAR 085). This variations to the works related to:
  \begin{enumerate}
	\item HVAC provisions for Electrical and Telecommunication rooms.
	\item Modifications to duct sizes in St. Regis Mezzanine Floor.
	\item Wow suite toilet exhaust requirements.
	\item Fan air flow changes.
	\item Updates AHUs serving F35 Westin Hotel.
	\item Modifications to duct sizes (W Hotel)
	\item Additional exhaust fan.
  \end{enumerate}
  
 \item Updated electrical drawings (38) were issued on 7 Mar 2015. We served notification as to the time impact in studying the changes (as there were no cloud revisions) and to advice impacts. Costs were reserved under CAR 146. Notifications were served via \letter{1913} and \letter{1920} dated 11 Mar 2015 and 15 Mar 2015.
 
 \item We raised concerns for missing design information for Kitchen equipment revised SLD/power drawings for W Hotel areas in order to proceed with Shop drawings preparation \letter{YL/1856} dated 5 Feb 2015.
 
 \item We raised variation notification due to additional HVAC works \letter{YL/1850} dated 5 Feb 2015.
       \begin{enumerate}
          \item BTU meters for the chilled water system.
          \item Seasonal Taste air distribution details.
          \item Relocation of AHUs at Basement-2
          \item AHU orientation, piping connection and duct re-arrangement.
          \item Additional motorized smoke dampers for standby smoke fans.
       \end{enumerate}

    
\item We requested Specialist subcontractor drawing for Data Centre and Security rooms  to proceed with MEP Shop drawings (affected Mezzanine St. Regis) \letter{YL/1813} dated 20 Jan 2015.\footnote{Security room finalization still pending as of 20 May 2015}. 

\item On 16 December 2014, we advised that MEP delays to the accelerated program were attributed to design delays, civil work delays in the casting of slabs, lack of primary co-ordination by \KA\ delays in appointing subcontractors affecting MEP works (kitchens, ELV works, IT and Audio-visual) and failure of ID designers \letter{YL/1741} dated 16 November 2014.
  
\item We received instructions to provide adaptors and sanitaryware accessories for all Hotels via \letter{Du/26882/14}  dated 16 Nov 2014. 

\item We recorded our concerns due to revisions to the design we received via sketches showing cross-contamination between fresh air intakes and exhaust from kitchens via letter ref: \letter{YL/1760} dated 24 December 2014.

\item Additional fire-fighting works for all hotels were confirmed via letter ref: \letter{YL/1765} dated 28 December 2014.

\item Revised RCP drawings were issued to the \JV which lead to abortive works at St. Regis Ground Floor to Level-2. This was confirmed by \letter{YL/1732} dated 4 December 2014.

\item We received instructions to install earthing system for all structured cabling Telecommunications rooms for all hotels (CAR 130) and \letter{YL/1728} dated 3 Dec 2014.

\item Additional UPS was added to serve the St Regis Hotel Data Center. This was confirmed by letter on 30 October 2014. (CAR 111).

\item Additional floor clean-outs were requested by the Engineer during inspections (CAR-116).

\item FCU type changed from decorative to ducted in W-hotel (CAR-115).

\item Changes to drainage services at the St Regis Hotel due to floor sinks passing over post tension slabs. (CAR 114).

\item Additional Control Panels in St. Regis Hotel Areas. (CAR-113).

\item Revisions to air outlets (CAR-117)

\item Revised Electrical works at St. Regis Hotel (CAR 83) submitted on 4/11/2014.

\item Additional Fans/AHUs that were instructed via RFI responses (CAR 85).

\item  Revised HVAC works  in St Regis Hotel (CAR 116).

\item Additional Electrical Works to St Regis Hotel Data Center (CAR 118)

\item Revised CCTV layouts for W \& Westin Hotel Areas (CAR 087)
  
\item Delays in the finalization of the Central Battery System for W \& Westin Hotels. We raised concerns via letter \letter{YL/1688}  dated 15 October 2014.

\item On 25 September 2014 we confirmed additional works for Drainage Service for one additional toilet at basement. (CAR 92).

\item On 25 September 2014, we confirmed additional works for water supply supply points for pantry at P4 in the St Regis Hotel.

\item On 25 September 2014, we confirmed additional works due to revisions to equipment schedules received via \letter{RF/MEP/M307}  (CAR 85).

\item On 25 September 2014 we confirmed additional works due to revised electrical drawings for Westin Hotel. (CAR 86).

\item On 1 October 2014 we confirmed instructions for additional electrical works due to missing power feeders at the Banquet Hall. received via RFI/MEP306 \& 309 for St Regis Hotel. (CAR 83).

\item On 2 October 2014, we confirmed additional works due to revisions of Mechanical Electrical loads. This was received as a reply to RFI/MEP/E295 for all three Hotels (CAR 83, 86 \&87). 

\item On 2 October 2014 we confirmed additional works due to revised Electrical drawings for Food and Beverage areas (CAR 93).

\item On 2 October 2014, we confirmed additional works due to Revised Mechanical Equipment schedules. This was received via RFI/MEP/M309 for all three Hotels (CAR 85).

\item On 8 October 2014, we were instructed by letter HLG/626/2.05/YE/nd/4327/14 attaching \KA letter Ref. No.: DU1211/DU/L/20844/14 dated 7 September 2004 confirming additional containment for the Access Control System.

\item On 13 October 2014, we confirmed additional works due to the addition of Electrical Heaters to some of the AHUs. This was received via RFI/MEP/M309 for all three hotels (CAR 94).

\item On 6 September 2014, we expressed concerns as to the constructibility and maintainability of the St Regis Technical 1 Plant room and issued proposals to minimize impacts. \letter{1538}. In our letter we requested \KA to resolve the issues latest within 7 days in order to minimize the impact on the accelerated target dates. Further updates were issued via letter \letter{1545} dated 9 September 2014 and \letter{1566} dated 17 September 2014 and \letter{1729} dated 4 December 2014.


\item On 7 September 2014, we requested details for the Access Control System via \letter{1542}.

\item On 9 September 2014, we expressed concerns as to changes to the Architectural design of the St. Regis Fire Pump Room which remained incomplete via \letter{1550} impeding installation of Fire Pumps.

\item On 10 September 2014, we confirmed receipt of revised HVAC drawing via \letter{1549} (CAR 85).

\item On 11 September 2014, we confirmed receipt of revised Electrical drawings for St. Regis Hotel via letter \letter{1555} (CAR 83).

\item On 22 September 2014, we issued notification for additional works related to the Domestic Cold Water system via \letter{1572} (CAR 89).

\item We received revised electrical drawings for Westin via HLG Transmittal Ref. No: HLG-626-DT-HLS-1671 dated 8 September 2014. (CAR 86)

\item We received revised electrical drawings for W Hotel via HLG Transmittal Ref. No.: HLG-626-DT-HLS-1728 dated 17 September 2014.
\end{enumerate}

The above do not record fully the method and lack of detail in issuing design information to the \JV. The general drawing issued to us did not contain adequate information to develop Shop Drawings. Clarifications and proposals were sent to the Engineer for missing information  via RFIs. 




%\makeatletter
\long\def\hlshadi#1{\hl{#1}}
\cxset{enumerate numberingi/.is choice,
  enumerate numberingi/.code={\renewcommand\theenumi {\csname#1\endcsname{enumi}}},
  enumerate numberingii/.code={\renewcommand\theenumii {\csname#1\endcsname{enumii}}},
  enumerate numberingiii/.code={\renewcommand\theenumiii {\csname#1\endcsname{enumiii}}},
  enumerate numberingiv/.code={\renewcommand\theenumiv {\csname#1\endcsname{enumiv}}},
  enumerate labeli punctuation/.store in=\enumeratepunctuationi@cx,
  enumerate labeli/.is choice,
  enumerate labeli/brackets/.code={\renewcommand\labelenumi{(\theenumi\enumeratepunctuationi@cx)}},
  enumerate labeli/square brackets/.code={\renewcommand\labelenumi{[\theenumi\enumeratepunctuationi@cx]}},
  enumerate labeli/right bracket/.code={\renewcommand\labelenumi{\theenumi\enumeratepunctuationi@cx)}},
  enumerate label left/.store in=\enumeratelabelleft@cx,
  enumerate label right/.code=\renewcommand\labelenumi{\enumeratelabelleft@cx\theenumi\enumeratepunctuationi@cx#1},
  enumerate leftmargini/.code={\setlength\leftmargini{#1}},
  enumerate leftmarginii/.code={\setlength\leftmarginii{#1}},
  enumerate leftmarginiii/.code={\setlength\leftmarginiii{#1}},
  enumerate leftmarginiv/.code={\setlength\leftmarginiv{#1}},
  listi topsep/.store in=\listitopsep@cx,
  listi partopsep/.store in=\listipartopsep@cx,
  listi itemsep/.store in=\listiitemsep@cx,
  listi parsep/.store in=\listiparsep@cx,
  listii topsep/.store in=\listiitopsep@cx,
  listii partopsep/.store in=\listiipartopsep@cx,
  listii itemsep/.store in=\listiiitemsep@cx,
  listii parsep/.store in=\listiiparsep@cx,
  listiii topsep/.store in=\listiiitopsep@cx,
  listiii partopsep/.store in=\listiiipartopsep@cx,
  listiii itemsep/.store in=\listiiiitemsep@cx,
  listiii parsep/.store in=\listiiiparsep@cx,
}
\cxset{compact1/.style={%
  enumerate numberingi=arabic,
  enumerate numberingii=alph,
  enumerate numberingiii=alph,
  enumerate numberingiv=roman,
  enumerate labeli punctuation=.,
  enumerate label left=,
  enumerate label right=,
  enumerate leftmargini=2.2em,
  enumerate leftmarginii=2.1em,
  enumerate leftmarginiii=1.5em,
  enumerate leftmarginiv=2em,
  listi topsep=8\p@ \@plus2\p@ \@minus\p@,
  listi itemsep=0\p@ \@plus2\p@ \@minus\p@,
  listi parsep=0\p@ \@plus2\p@ \@minus\p@,
  listii topsep=0\p@ \@plus2\p@ \@minus\p@,
  listii itemsep=0\p@ \@plus2\p@ \@minus\p@,
  listii parsep=0\p@ \@plus2\p@ \@minus\p@,
  listiii topsep=0\p@ \@plus2\p@ \@minus\p@,
  listiii itemsep=0\p@ \@plus2\p@ \@minus\p@,
  listiii parsep=0\p@ \@plus2\p@ \@minus\p@,
}}
\cxset{compact2/.style={%
  enumerate numberingi=alph,
  enumerate numberingii=roman,
  enumerate numberingiii=alph,
  enumerate numberingiv=roman,
  enumerate labeli punctuation=,
  enumerate label left=(,
  enumerate label right=),
  enumerate leftmargini=2.2em,
  enumerate leftmarginii=2.1em,
  enumerate leftmarginiii=1.5em,
  enumerate leftmarginiv=2em,
  listi topsep   = 8\p@ \@plus2\p@ \@minus\p@,
  listi itemsep = 0\p@ \@plus2\p@ \@minus\p@,
  listi parsep   = 0\p@ \@plus2\p@ \@minus\p@,
  listii topsep  = 0\p@ \@plus2\p@ \@minus\p@,
  listii itemsep= 0\p@ \@plus2\p@ \@minus\p@,
  listii parsep  = 0\p@ \@plus2\p@ \@minus\p@,
  listiii topsep = 0\p@ \@plus2\p@ \@minus\p@,
  listiii itemsep= 0\p@ \@plus2\p@ \@minus\p@,
  listiii parsep  = 0\p@ \@plus2\p@ \@minus\p@,
}}

\ExplSyntaxOn
\def\setenumerate#1{
\cxset{#1}
\def\@listi{%
           \leftmargin\leftmargini
            \parsep\listiparsep@cx
            \topsep\listitopsep@cx\relax
            \itemsep\listiitemsep@cx}
            
\def\@listii{\leftmargin\leftmarginii
            \parsep\listiiparsep@cx
            \topsep\listiitopsep@cx\relax
            \itemsep\listiiitemsep@cx}
            
\def\@listiii{\leftmargin\leftmarginiii
            \parsep\listiiiparsep@cx
            \topsep\listiiitopsep@cx\relax
            \itemsep\listiiiitemsep@cx}
}
\ExplSyntaxOff

\setenumerate{compact1}
\makeatother
\def\delay{\textcolor{red}{\Fire}}

\ExplSyntaxOn
% Holds master fields for all ODBs
\clist_new:c {DBSMASTER}

% New DBS
\clist_new:c {DBS}

%% Create the DB. The DB can have any name
%% 
\NewDocumentCommand {\CreateDB} { m }
  {
    \clist_new:c {#1}
    \clist_gput_left:cn {DBS} {#1}
  }   

% add only the number, and this only at the right,
% expecting the user to type it in ascending order and thus make sorting 
% easier
% Note the elements are stores as 1,3,4,68,112 etc.
% #1 DB name
% #2 field index - integer
% 
\NewDocumentCommand \addtoDB { m m  }
  {
    \clist_gput_right:cx { #1 } { #2 }
  }
 
%% Generate some variants
%%

\cs_generate_variant:Nn \clist_sort:Nn {cn}   

%% generalized sort DB
\NewDocumentCommand \SortDB { m }
{
  % remove any duplicates before sorting out
   \clist_remove_duplicates:c { #1 }
   % make variant here
   \clist_sort:cn {#1}
     {
       \int_compare:nNnTF { ##1 } < { ##2 }
        { \sort_reversed: }
        { \sort_ordered: }
    }
}

% #1 DB #2 Suffix
%
 \NewDocumentCommand \printRFI { m m }
  {
  % sort the list in numerical order
    \SortDB { #1 }
     
 % map and print only for category (one to many also possible here)   
     \clist_map_inline:cn { #1 }
      {  
       \cs_if_exist:cT {RFI##1-#2}
          {
            \cs:w RFI##1-#2\cs_end:
            \index {RFI~Mechanical>RFI-M-##1-#2}
          }
       
      }
  }
\ExplSyntaxOff    


 %#1 Number
%#2 Hotel
%#3 Impact
%#4 Description
%#5 Area

\newenvironment {RFI} [3] {%
  \vspace*{12pt}
  \parindent0pt
  \mbox{\bfseries\color{red!80!black}\textsf{RFI-M-#1}}
        \textbf{#2} \textbf{#3} \par
  \begin{enumerate}}
 {\end{enumerate}}

% Optional DB letters W - W hotel
% WE - Westing
% SR  - St Regis

\NewDocumentCommand {\addRFI} { O{SR} +m +m +m +m +g  +g } 
{%
   \expandafter\gdef\csname RFI#2-#1\endcsname
   {%
      \begin{RFI}{#2} {#3} {#4} 
      
        #5
      
      \end{RFI}
       
      \IfNoValueTF{ #6 }{ #6 }{ } 
       
      \IfNoValueTF{ #7 }{ #7 }{ }
    } 
    \addtoDB {MRFI} {#2}
   \par 
}

%% Create DB 
\CreateDB {MRFI}   




\chapter{Mechanical RFIs Related to St Regis}



\addRFI {497} {7 Dec 14 received 19 Jan 2015} {St Regis, Podium 1, Main Kitchen}
{
\item The RFI referred to previous response to RFI, where the Engineer directed the Contractor 
         to their comments on Shop Drawing AHC-HLS-SRH-SDM-AC-P1-0. Contractor re-iterated
         that achievable ceiling in Kitchen area could only be 2300mm and at the kitchen extract hood 2m.
\item This area was almost impossible to complete given the congestion to achieve this ceiling height.         
}


\addRFI {537} {18 Jan 2015 received 22 Jan 2015} {St Regis, Basement 1}%
 {
   \item Engineer issued instruction via drgs DU-1211-DU-00564 on 12 Jan 2014. 
    Engineer issued revised layout for AC-4020.
   \item Suggested re-arrangement blocked access to shaft 35-36/D-F.
   \item No chilled water piping and valving was not shown.
   \item Length of AHU-B2-02 is in excess of 5 meters plus a 500 mm plenum. 
 }



\addRFI{657}
{20-Apr-2015, responded 27 April 2015} {St Regis, Mezzanine, Security Room} 
{
\index{CCU>Security Room}\index{Primary co-ordination>CCTV}
\index{St Regis>Mezzanine>Security Room}

\item The layouts were not matching with latest Architectural drawings.
\item Layout not matching with HVAC Design.
        \begin{enumerate}
        \item The new layouts did not allow space for the CCUs.
        \item The new layouts did not show any access flooring.
        \end{enumerate}
\item Location of CCU 
\item Clear height not achievable.
\item Kitchen extract passing through security room, clashing with Monitor/LED display.
\item Uncertainty if there is access floor or not.
\item Engineer responded to swap store room with security room. The response did not adequately cover all the 
queries. The kitchen headroom or the access flooring was not responded to.
\item The response was followed by workshop meetings. As of May 20 2015, the Main Contractor did not carry the architectural changes required. \delay\delay\delay
}


\addRFI{662} {12-Apr-2014, responded 13-Apr-2014} {St Regis Podium 1 Winter Garden} 
{
\item This RFI dealt with incomplete issues regarding ceilings in Podium 1 Winter Garden.
\item The RFI requested details of return air grilles.
}


\addRFI {0646}{23 April 2015 responded 5 April 2015}{St Regis, Spa Area Podium 2}%
 {
 \index{St Regis>Spa>RFI-0646}
 \item Contractor requested clarifications for discrepancies between BARR \& WRAY requirement drawings and MEP design layout.
 \item Engineer responded to all queries and confirmed requirements.
 \item Area affected Spa. \delay \hl{Shadi to report on nature of delay}
 }{
 The Spa area subcontractor was appointed late. The area at Podium 2, was one of the first areas to be
 completed. The public areas around it, as well as the French Courtyard were designed very late.
 
 The Spa area subcontractor was appointed late. The area at Podium 2, was one of the first areas to be
 completed. The public areas around it, as well as the French Courtyard were designed very late.
}{
 Appointment took place only in January 15.
}

\addRFI [B] {0222}{10 Jun 2014 received 17 Jun 2014 }{Air and Dirt Separators, Boiler Room}
{
\item The Contractor requested clarification on additional air separators in the Boiler Room.
\item Relevant drawings Ref: DU1211-WS2173 Westin Hotel \& DU1211-WS2024 for Basement-1
   \begin{enumerate}
      \item Air and dirt separators are shown in B-1 Boiler plant room for the central water heating. This was not shown on Tender drawings.
      \item Air and Dirt separator is shown in Westin Hotel TE-3 plant room.
      \item Contractor requested verification and noted that the additional equipment shown would lead to additional costs to the Client.
      \item Engineer responded that they should be provided as shown on the revised drawings.
   \end{enumerate}
\item   Later Contractor made arrangements for submittals and for air-freighting later on these equipment. 
}{}{}

\addRFI [B] {0215}{2 Jun 2014 replied 12 Jun 2014} {Basement Duct clashes with Water Feature Plant Room}
{
\item Issues arose due to co-ordination of HV cables tray and ducting.
\item Engineer instructed that the duct be re-sized and re-routed.
}{}{}

\addRFI [WE] {0247} {25 June 2014  returned 5 July 2014} {Westin, Ground Floor, Insufficient ceiling void}
{
   \item Contractor advised that the ID drawings show a gap between the ceiling and the duct of 60mm and that this was going to impede return air.
   \item Please refer to attached modified ceiling levels.
}

\addRFI [B] {0655} {16 Apr 15 received 4 May 15 } {Basement 1, MEP Services in loading Bay}
{
  \item Contractor requested confirmation to Engineer's reply RFI: No. HLG-626-RFI-ME-0618 to proceed with capacity as mentioned in the RFI with one set of fans, one duty and one standby (total 2 fans). 
  \item Engineer responded to refer to RFI reply dated 21 April 2015.
}

\addRFI [W] {0661} {12 Apr 2015 received 22 Apr 2015} {W Mezzanine Floor, Male and Female Accessible Toilets RCP}
  {
    \item Contractor requested co-ordinated RCP layout with diffuser size type and location for male/female and accessible toilet to finalize Shop Drawing.
   \item Engineer responded by providing CAD and PDF files. 
   
   \hlshadi{Shadi to confirm if the drawings received were adequate. (Drawings were received by HLG Feb 15 2015.
     They seem to me they were for a Tender package. We should request Subcontractor Shop Drawings from
     Main Contractor.
    } 
  }
  
\addRFI[B] {0664}  {21 Apr 2015 received 28 Apr 2015} {Basement 1, Loading Bay Area}
 {
   \item Contractor requested clarifications and made proposals for additional fans.
   \item Engineer responded with detailed reply, including tag-number. 
 }
 
\addRFI {0666} {28 April 2015 received 14 May 2015 } {St Regis, Festival Dining Kitchen Podium~1 }
 {
   \item Contractor highlighted that the drainage for this area are not available and would cause problems in finalizing Mezzanine below.
   \item This was responded by CKP Who advised that designs have still not been received from AHG (FNB Division). These are AHG spaces (Specialty Restaurants). \delay\delay\delay
 }


\addRFI [WE] {0676} {7 May 2015 received 18 May 2015} {No access to shaft MR1 at Technical~2 Westin Hotel}
{
  \item Westin Hotel Technical 2 core wall penetration Builder's Works drawings (rev0 and rev1) we proposed 1700 x 950   opening. This was approved by the Engineer. This size has been revised due to the comment, changes were then incorporated in rev 1 layout. At Site the available opening is 2440 x 600. Contractor included an Annexure with proposals.
  \item Engineer responded with proposal to reduce space further between pipes. \hl{Shadi to confirm what eventually was followed.}
}

\printRFI {MRFI} {SR} 

%% Typeset all Chapters
%% 
\chapter {Mechanical RFIs Related to Westin Hotel}
\printRFI {MRFI} {WE}


\chapter{Mechanical RFIs Related to `W'~Hotel}
\printRFI {MRFI} {W}

\chapter{Mechanical RFIs Related to `Basements'  }
\printRFI {MRFI} {B}
  



%
\def\omar#1{\hl{#1}}

\ExplSyntaxOn
 
\edef\aprefix{ERFI}
% #1 DB #2 Suffix
%
 \NewDocumentCommand \printERFI { m m }
  {
  % sort the list in numerical order
    \SortDB { #1 }
     
 % map and print only for category (one to many also possible here)   
     \clist_map_inline:cn { #1 }
      {  
       \cs_if_exist:cT {\aprefix##1-#2}
          {
      %     \PASS RFI-E##1-#2
           \cs:w \aprefix##1-#2\cs_end:
           \index {RFI~Electrical>RFI-E-##1-#2}
           }
         % {\FAIL ##1-#2}\par
      }
  }
  
\ExplSyntaxOff    


 %#1 Number
%#2 Hotel
%#3 Impact
%#4 Description
%#5 Area

\newenvironment {ERFI} [3] {%
  \vspace*{12pt}
  \parindent0pt
  \mbox{\bfseries\color{red!80!black}\textsf{RFI-E-#1}}
        \textbf{#2} \textbf{#3} \par
  \begin{enumerate}}
  {\end{enumerate}}

% Optional DB letters W - W hotel
% WE - Westing
% SR  - St Regis

\NewDocumentCommand {\addERFI} { O{SR} +m +m +m +m +g  +g } 
{%
   \expandafter\gdef\csname \aprefix#2-#1\endcsname
   {%
      \begin{ERFI}{#2} {#3} {#4} 
        #5
     \end{ERFI}
       
      \IfNoValueTF{ #6 }{ #6 }{ } 
       
      \IfNoValueTF{ #7 }{ #7 }{ }
    } 
    \addtoDB {ERFIDB} {#2}
   \par 
}

%% Create DB 
\CreateDB {ERFIDB}   
  
\addERFI[W] {703} {10 May 2015 received 24 May 2015}{Westin and W Guestroom Floors}
{\index{W RFI>RFI-E-703}
   \item Contractor issued query based on HLG letter Ref: HLG/626/2.05/AMM/es/7764/15 dated 4 May 2015
      to request how the lighting in guestroom corridors would be controlled. The HLG letter referred to the Client instructing that lighting control in corridors be cancelled. 
   \item Engineer answered, 'grid switch to be provided in each electrical room for each floor, no of gangs to be as per circuits requirement and to proceed futher as per site conditions.  
   \hl{Omar and PMs to report time implications. Does this involve redrafting of any drawings? If yes please specify}   
}{}{}

\addERFI[WE] {703} {10 May 2015 received 24 May 2015}{Westin and W Guestroom Floors}
{
   \item Contractor issued query based on HLG letter Ref: HLG/626/2.05/AMM/es/7764/15 dated 4 May 2015
      to request how the lighting in guestroom corridors would be controlled. The HLG letter referred to the Client instructing that lighting control in corridors be cancelled. 
   \item Engineer answered, 'grid switch to be provided in each electrical room for each floor, no of gangs to be as per circuits requirement and to proceed futher as per site conditions.  
   \omar{Omar and PMs to report time implications. Does this involve redrafting of any drawings? If yes please specify}   
}{}{}


\addERFI [B] {702} {18 May 2015 received 22 Jan 2015} {St Regis, Basements 1, 2 and 3, Confirmation for lighting circuits for Executive Lift Lobby (B1) and Banquet suite Lift Lobby (B2 and B3) }%
 {
  \item Contractor observed that no lighting design was available for the Executive Lift Lobby (B1) and Banquet suite Lift Lobbies (B2 \& B3). Contractor added circuit references to the light points shown on RCP drawings which were received via \DT{2885}{} dated 5 May 2015 and \DT{2896}{} dated 10 May 2015.
  \item Co-ordinate light location/distribution with ID/Fit Out drawings. Submit RCP for review and approval. Update load schedule based on similar loads.
 }{}{}

\addERFI [W] {701} {12 May 2015 received 23 May 2015} {W Hotel Podium 2, Kitchen, missing circuit references}
 {
   \item Contractor advised of missing DB references and circuits for kitchen areas.
   \item \KA responded to be connected to nearest light DB, advisable DB-FB-2PR2 which has an available spare. \KA also requested for the load schedule to be submitted for approval.
   \omar{Omar I thought we have submitted load schedules, for W?}
 }
 {}{}

\addERFI [B] {696} {7 May 2015 received 19 May 2015} {Basement 1 Janitor Room and Staff Toilets }
{
  \item Contractor noted that the Janitor Room and Staff Toilet lighting and power design is not available.
  \item \KA responded to update the drawing and to be resubmitted for approval!
   \omar{Omar Did we submit?}   
}

\addERFI [SR] {695} {7 May 2015 received 12 May 2015 } {St Regis, PD3, PD4, PD5}
  {
    \item Contractor attached drawing, showing power supply to door requiring not shown in the Electrical design. Contractor requested feeder details for the power socket.
    \item \KA advised to connect to nearby motorized damper.
    \hl{Kyriacos, Omar, how does this affected site. This is clearly a disruptive activity.}
  }

\addERFI [W] {693} {27 Apr 2014 received 30 Apr 2014} {Westin Hotel, discrepancy between Fino RCP drawings and approved shop drawing, Podium 2, Meeting Rooms}
 {
    \item Contractor highlighted conflict between Contractor's approved Shop Drawing and Fino's latest issued RCP drawings. 
    \item \KA respondedn to follow RCP from Fino Int'l for location and number of downlights.
    \omar{Does this mean we need to go back and modify works on site?}
 }

\addERFI {692} {26 Apr 2015 received 2 May 2015 } {All levels having Sofia drawings}
  {
    \item  Contractor referred to nmerous RFI replies   and updated SLD attached.
    \item \KA responded to submit the SLD, for review assesment considering voltage drop, available NOC load etc. \omar{Did we submit? We are not responsible for Sofia loads and hence KA should not be asking us to stay with submitted Dewa loads. If these were exceeded please advise them}
  }

\addERFI [B] {691} {26 April 2015 received 28 April 2015 }{ Basement Loading and unloading area. Missing height of the isolators for scissor lifts.}
 {
   \item Contractor requested information as to the height of required scissor lift isolators.
   \item \KA instructed HLG to co-ordinate with Otis and advise.
   \omar{There are cables in the air in that vicinity, has this been resolved?}  
 }

\addERFI [W] {690} {7 May 2015 received 23 May 2015} {W Hotel, Mezzanine, discrepancy between FA design and architectural drawing.}
 {
 	\item Contractor advised conflicts between architectural drawings and FA design drawings.
 	\item \KA advised to update fire alarm drawings in co-ordination with FA Supplier and UAE code. Shop Drawing to be submitted for approval.
 }
 
\addERFI [SR] {689} {7 May 2015 received 13 May 2015}{St Regis, Podium 1 and TEchnical 1, relocation of amplifier. }
 {
	\item Contractor requested confirmation of location of amplifier rack from Podium-1 to Technical-1.
	\item \KA confirmed and noted routing of containment to be co-ordinated with other services.
	\omar{What amplifier is this? Was containment missed earlier, otherwise this is a disruptive activity.}
 } 

\addERFI [W] {688} {3 May 15 returned 18 May 2015} {W Hotel Ground Floor }
 {
	\item Contractor advised conflicts between current architectural drawings and Approved Fire Alarm Shop drawings.
	\item \KA replied to revise Shop Drawings according to UAE Fire regulations and resubmit.
	\omar {Did we resubmit?}
 }
 
 \addERFI [B] {687} {} {Basement 1 Clashing location of telephone and power sockets with lockers.}
  {
    \item We noticed that the location of power socket and telephone is clashing with locker location and requested clarifications.
    \item \KA provided Mediatech's response to move to nearest available free wall space.
    \omar {Why did we query this? Was it oicked up during an inspection?}
  }

 \addERFI [B] {686} {} {Basement 1 Fire Command and BMS Requirement}
  {
    \item Engineer made comments on \JV Shop drawing. Contractor requested clarifications.
    \item Provision for monitoring of BMS to be provided in the Fire Command Room \JV \& BMS Specialist to confirm that 1 data point and 1 socket is required only.
  }
  
\addERFI [B] {685} {28 Apr 2015 received 6 May 2015} {Basement 1 Fire Command Center - Elevator Panel Details} 
 {
 	\item Contractor requested information for the following:
 	\item \KA replied HLG shall lead the co-ordination with all subcontractors and conclude the requested information as all the submittals are approved. 
 	
 	\delay\delay\delay
 	
 	\omar{Has this now been received?}
 }
 
\addERFI [B] {684} {28 Apr 2015 received 6 May 2015} {Basement 1 Fire Command Center} 
 {
	\item Contractor requested the actual furniture layout for fire conrol room, as commented by Engineer on Shop drawings.
	\item \KA\ instructed HLG to co-ordinate with Furniture Supplier to submit proposal for Operator Approval.
	
	\omar {Zeljko, Rabih/Omar/Rahul we need to finish and get out, request furniture officially from HLG to install computers. } 
	\delay\delay\delay
 }

\addERFI [B] {683}{}{BMS}
 {
 	\item BMS DDC Panel relocation, became necessary due to space constraints.
 	\item \KA responded negatively.
 }

\addERFI [B] {682} { } {Du}
 {
 	\item \lorem
 }


 %%
\chapter{Electrical RFIs St~Regis Hotel}
\printERFI {ERFIDB} {SR}
%
\chapter {Electrical Basements}
\printERFI {ERFIDB} {B}
%
\chapter {Westin Hotel}
\printERFI {ERFIDB} {WE}
%
\chapter {W Hotel}
\printERFI {ERFIDB} {W}











%%internal

\section{Strategy Westin}

\section{Obtain Prices}

Following areas are provisional sums, requrire quotations from subcontractors

\subsection{Westdin}

\begin{enumerate}
\item Westin Spa
\item Round ducting
\item Level 29 and up

\item St Regis Crown and Festival RCP, flooring, ID elevation shop drawings, kitchen shop drawings.
\item Wood deck bar  
\item Royal Suite RCP \& ID elevation shop drawings data \& wifi Shop Drawings AV? Electronics? RMU?
\item Kids party area All MEP design dawings except fire fighting, RCP, Partition, flooring \& ID elevation shop drawings, access control (by Zio).

\item Earthing to be cleared for installation
\item Investigation into costs (masks)
\item Exposed conduits rather than in slabs. (Too costly a difference, also slowed the program).

\item Third fix  St Regis

\item generators

\item Did we achieve recovery? At what cost?

\item RFIs

\item Transformer rooms (difficulties with ventilation requirements)

\item Possible appointment subcontractor.

\item Felxible drops, pull rope to be costed in elev containments. Time consuming and expensive.

\item lift lobbies basements - release of ceilings

\item Containment costs

\item Disruption and Cumulative impact disclaimer at the bottom of all claims. Missing changes to equipment after 1/12/2014 in AHU section.

\item generator ventilation

\item commissioning

\item Stress analysis
\item End June

\item Lifting of equipment
\end{enumerate}


%\def\hot{{\color{red}\scalebox{1.5}{\Fire}} delayed works. }
\def\ghot{{\color{green!80!black}\raggedright\scalebox{1.5}{\Fire}} delayed works, but completed. }
\def\phot{{\color{green!80!black}\raggedright\scalebox{1.5}{\Fire}} delayed works. partially completed. }

\def\check{{$\color{green!80!black}\check$}}
\def\partiald {50\% complete}
\def\unavailable{\hot Design unavailable}
\cxset{subsection color=black}

\chapter{Late Finalization of Provisional Sum Works}

In April 2014, we wrote to the Main Contractor informing them of cut-off dates for providing information related to provisional sums and cut-off dates for their release based on the Baseline Program. The dates varied with the latest one being middle May 2014. Some basic designs were available that enabled some embedded first fix activities to start. However, none of the information was issued on time and according to the baseline program of works. 

The state of the design upon our appointment can be gauged by the value of the provisional works which was in the vicity of 60,000,000.00~AED. This represented approximately 20\% of the base contract value. The sections that follow detail the delays that occured in receiving information.\footnote{A \textcolor{red}{\Fire} indicates the designs arrived late and a \ghot}

\section{Mechanical}

The Total Mechanical Provisionals 19,600,000.00 AED.  Supply, installation, connection, testing and commissioning of all mechanical services including Kitchen hood-make up unit, hood extract ecology unit, AHU, ducting, insulation, air outlets, grills \& diffusers, cold water \& hot water piping, valves, drainage piping, floor drains, gas piping, firefighting piping, sprinklers, fire extinguishers etc (hood fire suppression system by others) to the approval of the Engineer for food services facilities for
\medskip

\bgroup
\raggedright\small
\setcounter{step}{0}
\begin{longtable}{lp{3.4cm}rllp{3.5cm}}
\toprule
\textbf{Item}	& \textbf{Description}	 &\textbf{Amount}&\textbf{Remarks}	&\textbf{1st Fix Start}	&\textbf{K\&A Design}\\
\midrule
1 & Food Service Facilities, Main Receiving Dock and Waste Control \& Centralized Commissary and Stores
& 1,100,000.00 
& B1-SR
&1-Mar-14
& \ghot\\
\midrule
2&Food Service Facilities for St. Regis Hotel	 &1,700,000.00 &&&\\
&&&B2	&30-Mar-14	& \ghot\\
&&&B1	&30-Mar-14	& \ghot\\
&&&GF	&30-Mar-14	& \hot partially still unavailable \\
&&&P1	&10-Apr-14	& \ghot\\
&&&P2	&10-Apr-14	& \ghot\\
\midrule
3 &Food Service Facilities for W. Hotel	 &1,500,000.00 &&&\\
&&&GF	&30-Mar-14	&\hot \\
&&&P1	&30-Mar-14	&\hot \\
&&&P2	&30-Mar-14	&\hot \\
&&&P4	&30-Mar-14	&\hot \\
&&&1st Flr	&30-Mar-14	&\hot \\
&&&24th Flr	&10-May-14	&\hot \\
&&&25th Flr	&10-May-14	&\hot \\
&&&26th Flr	&10-May-14	&\hot \\
\midrule
4&Food Service Facilities for Westin Hotel	 &1,300,000.00 &&&\\
&&&GF	&30-Mar-14	&\hot \partiald  \\
&&&P1	&30-Mar-14	&\hot \\
&&&P2	&30-Mar-14	&\hot \\
&&&1st Flr	&30-Mar-14	&\hot \\
&&&35th Flr	&30-Mar-14	&\hot\\
\midrule
5&Food Service Facilities for Client's Areas	 &1,000,000.00&&&\ghot\\ 
\midrule
6&Laudry Facilities& 1,500,000.00 	&B1-SR		&& \ghot \\
 \bottomrule
\end{longtable}
\egroup

Supply, installation, connection, testing and commissioning of all mechanical services including ducting, insulation, air outlets, grills \& diffusers, chilled water piping, cold water \& hot water piping, valves, drainage piping, floor drains, etc. to the approval of the Engineer for:				
\medskip

\bgroup
\raggedright\small
\setcounter{step}{0}
\begin{longtable}{lp{3.4cm}rlll}
\toprule
\textbf{Item}	& \textbf{Description}	 &\textbf{Amount}&\textbf{Remarks}	&\textbf{1st Fix Start}	&\textbf{K\&A Design}\\
\midrule
7	&Final Fix-Mechanical	 &3,000,000.00 		&&&\hot \\

\multicolumn{2}{c}{\textbf{Total Pro. Sums - Mechanical}}	 &\textbf{19,600,000.00} &&&			\\

\end{longtable}
\egroup

\section{Electrical}

\raggedright\small
\setcounter{step}{0}
\begin{longtable}{cp{3.4cm}rllp{3.5cm}}
\toprule
\textbf{Item}	& \textbf{Description}	 &\textbf{Amount}&\textbf{Remarks}	&\textbf{1st Fix Start}	&\textbf{K\&A Design}\\
\midrule
1	&\hcancel{Supply and fix of ID works for guest elevator cars}	 &\hcancel{2,335,000.00} 	&All	&	& \\
2	&\hcancel[red]{Obstruction lighting}	 &\hcancel[red]{280,000.00} 	&\hcancel{All}	&30-Apr-14	&Cancelled\\
3	&Containment - structured cabling system	 &1,350,000.00 	&All	&15-Apr-14	& \phot \\ 
4	&Containment - CCTV system	 &600,000.00 	&All	&15-Apr-14	& \phot \\
5	&Containment - access control system	 &450,000.00 &All	&15-Apr-14	& \phot \\
6	&Containment - A/V system	 &1,300,000.00 	&All	&15-Apr-14	& \phot \\
7	&Containment - Room Management System	 &1,100,000.00 	&All	&15-Apr-14	& \phot \\
8	&Lighting Control System	 &5,000,000.00 	&All	&30-Apr-14	& \phot\par "BOH partially completed.\\ 
9	&Façade Lighting 	 &1,100,000.00 	&All	&30-Apr-14	& \phot \\
10	&landscape lighting 	& 540,000.00 	&All	&30-Apr-14	& \ghot \\
\midrule
	&\textbf{Total for P.S. Electrical}	 &14,055,000.00 &&&\\

\bottomrule
\end{longtable}


Provisional Sums - Electrical
				
Supply, installation and connection of electrical works including lighting outlets, lighting switches, lighting control, socket outlets, telephone outlet, TV outlet, speakers, internal conduits and wiring, and all necessary accessories				

\newenvironment{pstable}{
\raggedright\small
\setcounter{step}{0}
\begin{longtable}{c>{\raggedright}p{3.4cm}rl p{2cm} p{3.5cm}}
\toprule
\textbf{Item}	& \textbf{Description}	 &\textbf{Amount}&\textbf{Remarks}	&\textbf{Design Cut-off date}
             	&\textbf{K\&A}\\
\midrule}{\bottomrule
\end{longtable}}

\begin{pstable}
a) &Basement/Common Facilities Valets Lifts Lobby 	 &7,865.00 	&B3		& & \ghot\\
b) &Common Facilities: Public Lifts Lobby 2nd Basement Level	 &26,483.00 	&B2	&	& \ghot \\
c) &Common Facilities: Parking Lifts Lobby 2nd Basement Level	 &8,712.00 	&B2		&& \ghot \\
d)	&Common Facilities: Banquet Lifts Lobby 2nd Basement Level	 &16,781.00 	&B2 &		& \ghot \\
e)	&Common Facilities: Staff Cafeteria 2nd Basement Level	 &732,325.00 	&B2	 &	& \ghot\\
   & \textbf{St Regis}                                              &              &     & &\\
\end{pstable}

\begin{pstable}   
f)	&Common Facilities: Valets Lifts Lobby 1st Basement Level	 &7,260.00 	&B1  && \ghot \\
g)	&Common Facilities: Parking Lifts Lobby 1st Basement Level	 &10,868.00 	&B1  && \ghot\\
h)	&Common Facilities: Parking Lifts Lobby 1st Basement Level	 &8,712.00 	&B1  && \ghot \\
i)	&Common Facilities: Royal Suite Lift Lobby 1st Basement Level	 &26,208.00 	&B1 && \ghot \\
j)	&Common Facilities: Exec. Suite Lift Lobby 1st Basement Level	 &8,833.00 	&B1 && \ghot \\
k)	&Common Facilities: Valets Lifts Lobby 1st Basement Level	 &9,504.00 	&B1&& \ghot \\
l)	&Common Facilities: St. Regis Housekeeping 1st Basement Level	 &71,525.00 	&B1 &	1-Apr-14&\ghot \\
m)	&Common Facilities: Maintenance/Engineering Workshop 1st Basement Level	 &267,025.00 	&B1	&1-Apr-14 &\hot \\
n)	&Common Facilities: Laundry 1st Basement Level	 &483,460.00 	&B1	 &1-Apr-14	& \ghot \\
o)	&Common Facilities: Kitchen 1st Basement Level	 &263,495.00 	&B1	 &1-Apr-14	&\ghot \\
p)	&Common Facilities: Chillers/Freezers 1st Basement Level	 &232,220.00 	&B1 &1-Apr-14	& \ghot \\
q)	&Common Facilities: Kitchen Stores 1st Basement Level	 &134,500.00 	&B1 &1-Apr-14	&\ghot \\
r)	&Common Facilities: Office 1st Basement Level	 &13,629.00 	&B1	 &1-Apr-14	&\ghot \\
s)	&Common Facilities: Chillers 1st Basement	 &31,573.00 	&B1	 &1-Apr-14	&\ghot \\
t)	&Common Facilities: Cold Rooms Compressor Room 1st Basement	 &23,216.00 	&B1	 &1-Apr-14	&\ghot \\
u)	&Common Facilities: Toilet 1st Basement Level	 &9,567.00 	&B1	 &1-Apr-14	&\ghot \\
v)	&Common Facilities: Waste Handling Area 1st Basement Level	 &259,105.00 	&B1 &1-Apr-14	&\ghot \\
\midrule
	   &\textbf{Total  P Sums - Electrical}	 &2,652,866.00 &&\\
\end{pstable}

\bigskip

\subsection{Electrical St.Regis Hotel}

Supply, installation and connection of electrical works including lighting outlets, lighting switches, lighting control, socket outlets, telephone outlet, TV outlet, speakers, internal conduits and wiring, and all necessary accessories

\begin{pstable}
1	&St. Regis Hotel Meeting Room 1 Ground Level	 &49,753.00 	&GF	  &27-Apr-14	&\ghot \\
2	&St. Regis Hotel Meeting Room 2 Ground Level	 &26,158.00 	&GF	  &27-Apr-14	&\ghot \\
3	&St. Regis Hotel: Prefunction Area Ground Level	 &326,150.00 	&GF	 &27-Apr-14	&\ghot \\
4	&St. Regis Hotel: Male/Female Toilets Ground Level	 &39,758.00 	&GF	 &27-Apr-14	&\ghot \\
5	&St. Regis Hotel: Boardroom Ground Level	 &43,758.00 	&GF	       &27-Apr-14	&\ghot \\
\end{pstable}

\begin{pstable}
6	&St. Regis Hotel: Lift's Lobby Ground Level	 &42,669.00 	&GF	    &27-Apr-14	&\ghot \\
7	&St. Regis Hotel: Lobby/Sculpture/Reception Ground Level	 &500,429.00 &GF	&27-Apr-14 &\ghot \\
8	&St. Regis Hotel: Pantry Ground Level	 &12,078.00 	&GF	    &27-Apr-14	&\ghot \\
9	&St. Regis Hotel: Lift's Lobby Ground Level	 &47,163.00 	&GF	  &27-Apr-14	&\ghot \\
10	&St. Regis Hotel: Shop Ground Level	 &9,851.00 	&GF	 &27-Apr-14	& \ghot \\
11	&St. Regis Hotel: Valet Parking Cor. Ground Level	 &9,636.00 	&GF	  &27-Apr-14	&\hot \\
12	&St. Regis Hotel: Suite Lift's Lobby Ground Level	 &5,357.00 	&GF	  &27-Apr-14	&\ghot \\
13	&St. Regis Hotel: St. Regis Restaurant Ground Level	 &89,029.00 	&GF	  &27-Apr-14	&\ghot \\
14	&St. Regis Hotel: Cigar Room Ground Level	 &44,798.00 	&GF	       &27-Apr-14	&\ghot \\
15	&St. Regis Hotel: Female Toilet Ground Level	 &16,929.00 	&GF	      &27-Apr-14	&\ghot \\
16	&St. Regis Hotel: Male Toilet Ground Level	 &13,928.00 	&GF	      &27-Apr-14	&\ghot \\
17	&St. Regis Hotel: Corridor Ground Level	 &84,909.00 	&GF	         &27-Apr-14	   &\ghot \\
18	&St. Regis Hotel: Male Toilet Ground Level	 &19,679.00 	&GF      &27-Apr-14   &\ghot \\
19	&St. Regis Hotel: Prefunction Hall Ground Level	 &379,049.00 	&GF	 &27-Apr-14	&\ghot \\
20	&St. Regis Hotel: Pub Kitchen Ground Level	 &40,855.00 	&GF	   &27-Apr-14	&\ghot \\
21	&St. Regis Hotel: St. Regis Ballroom Ground Level	 &455,637.00 	&GF	      &28-Apr-14	&\hot \\
22	&St. Regis Hotel: Banquet Hall Ground Level	 &851,532.00 	&GF	   &24-May-14	&\hot \\
23	&St. Regis Hotel: Banquet Pantry Ground Level	 &160,287.00 &GF	&15-May-14	&\ghot \\

24	&St. Regis Hotel: St. Regis Restaurant Podium 1 Level	 &308,165.00 	&P1	 &15-May-14	&\ghot \\
25	&St. Regis Hotel: Café/Lounge Podium 1 Level	 &417,373.00 	&P1	 &15-May-14	&\ghot \\

26	&St. Regis Hotel: Male/Female Toilets Podium 1 Level	 &43,457.00 	&P1	 &15-May-14	&\ghot \\
27	&St. Regis Hotel: Champagne Bar Podium 1 Level	 &90,418.00 	&P1	 &15-May-14	&\ghot \\

28	&St. Regis Hotel: Pantry Podium 1 Level	 &23,672.00 	&P1	 &15-May-14	&\ghot \\

29	&St. Regis Hotel: Lift's Lobbies Podium 1 Level	 &180,950.00 	&P1	 &15-May-14	&\ghot \\

30	&St. Regis Hotel: St. Regis Main Kitchen Podium 1 Level	 &169,054.00 	&P1	 &15-May-14	&\ghot \\

31	&St. Regis Hotel: Italian Restaurant Podium 1 Level	 &215,254.00 	&P1 &15-May-14	&\ghot \\

32	&St. Regis Hotel: Kitchen Podium 1 Level	 &42,335.00 	&P1	 &15-May-14	&\ghot \\

33	&St. Regis Hotel: Boardroom Podium 1 Level	 &40,798.00 	&P1	 &1-May-14	&\ghot \\

34	&St. Regis Hotel: Break-Out Area Podium 1 Level	 &96,443.00 	&P1  &1-May-14&\ghot \\
35	&St. Regis Hotel: Meeting Room Podium 1 Level	 &25,815.00 	&P1	 & 1-May-14&\ghot \\
36	&St. Regis Hotel: Meeting Room Podium 1 Level	 &52,712.00 	&P1	 &1-May-14&\ghot \\
37	&St. Regis Hotel: Meeting Room Podium 1 Level	 &52,712.00 	&P1	 &1-May-14&\ghot \\
38	&St. Regis Hotel: Meeting Room Podium 1 Level	 &36,452.00 	&P1	 &1-May-14&\ghot \\
39	&St. Regis Hotel: Male Toilet Podium 1 Level	    &14,756.00 	&P1	 &1-May-14&\ghot \\
40	&St. Regis Hotel: Female Toilet Podium 1 Level	 &12,794.00 	&P1	 &1-May-14&\ghot \\

41	&St. Regis Hotel: Gourmet Shop/Cafe Podium 2 Level	 &267,790.00 	&P2	 &1-May-14&\ghot \\
\end{pstable}

\begin{pstable}
42	&St. Regis Hotel: Spa Podium 2 Level	 &541,352.00 	&P2	 &24-Apr-14	&\ghot \\
43	&St. Regis Hotel: Business Centre Podium 2 Level	 &197,690.00 	&P2	 &24-Apr-14	&\ghot \\

44	&St. Regis Hotel: Deluxe Suite Podium 2 Level	 &114,930.00 	&P2 &24-Apr-14	&\ghot \\
45	&St. Regis Hotel: Junior Suites Podium 2 Level	 &100,752.00 	&P2 &24-Apr-14	&\ghot \\

46	&St. Regis Hotel: Junior Suites Podium 2 Level	 &144,084.00 	&P2	 &24-Apr-14	&\hot \\

47	&St. Regis Hotel: Ambassador Suite Podium 2 Level	 &175,242.00 	&P2 &24-Apr-14	&\hot \\
48	&St. Regis Hotel: Executive Suite Podium 2 Level	 &203,292.00 	&P2 &24-Apr-14	&\hot \\

49	&St. Regis Hotel: Water Fountain Podium 2 Level	 &409,475.00 	&P2	 &8-May-14	&\hot \\
50	&St. Regis Hotel: Corridor Podium 2 Level	 &349,470.00 	&P2   &8-May-14	&\hot \\

51	&St. Regis Hotel: Guest Corridor Podium 2 Level	 &447,436.00 	&P2	 &8-May-14	&\hot \\
52	&St. Regis Hotel: Lift's Lobby Podium 2 Level	 &25,342.00 	&P2	 &8-May-14	&\hot \\

53	&St. Regis Hotel: Deluxe Suite Podium 3 Level	 &114,930.00 	&P3	 &4-May-14	&\hot \\
54	&St. Regis Hotel: Ambassador Suite Podium 3 Level	 &175,242.00 	&P3 &4-May-14	&\hot \\

55	&St. Regis Hotel: Junior Suite 1 Podium 3 Level	 &50,340.00 	&P3 &4-May-14	&\hot \\
56	&St. Regis Hotel: Junior Suite 2 Podium 3 Level	 &53,682.00 	&P3	 &4-May-14	&\hot \\

57	&St. Regis Hotel: Junior Suite 3 Podium 3 Level	 &72,042.00 	&P3	 &4-May-14	&\hot \\
58	&St. Regis Hotel: Junior Suite 4 Podium 3 Level	 &50,376.00 	&P3	 &4-May-14	&\hot \\
59	&St. Regis Hotel: Executive Suite Podium 3 Level	 &135,528.00 	&P3	 &4-May-14	&\hot \\

60	&St. Regis Hotel: Executive Suite 1 Podium 3 Level	 &132,060.00 	&P3	 &4-May-14	&\hot  \\

61	&St. Regis Hotel: Corridor Podium 3 Level	 &412,704.00 	&P3	 &18-May-14	&\hot \\
62	&St. Regis Hotel: Deluxe Suite Podium 4 Level	 &114,930.00 	&P4	 &15-May-14	&\hot \\

63	&St. Regis Hotel: Designer Suite Podium 4 Level	 &175,242.00 	&P4	 &15-May-14	&\hot \\
64	&St. Regis Hotel: Junior Suite 1 Podium 4 Level	 &50,340.00 	&P4	 &15-May-14	&\ghot \\

65	&St. Regis Hotel: Junior Suite 2 Podium 4 Level	 &53,682.00 	&P4	 &15-May-14	&\ghot \\

66	&St. Regis Hotel: Junior Suite 3 Podium 4 Level	 &72,042.00 	&P4	 &15-May-14	&\ghot \\

67	&St. Regis Hotel: Junior Suite 4 Podium 4 Level	 &50,376.00 	&P4	 &15-May-14	&\ghot \\

68	&St. Regis Hotel: Executive Suite 2 Podium 4 Level	 &264,120.00 	&P4	 &15-May-14	&\ghot \\
69	&St. Regis Hotel: Corridor Podium 4 Level	 &412,704.00 	&P4	 &30-May-14	&\ghot \\
70	&St. Regis Hotel: Presidential Suite Podium 5 Level	 &145,536.00 	&P5 &22-May-14	&\hot \\
71	&St. Regis Hotel: Designer Suite Podium 5 Level	 &174,918.00 	&P5	 &22-May-14	&\hot \\

72	&St. Regis Hotel: Executive Suite Podium 5 Level	 &271,056.00 	&P5	 &22-May-14	&\hot \\
73	&St. Regis Hotel: Junior Suite 1 Podium 5 Level	 &50,340.00 	&P5	 &22-May-14	&\ghot \\
74	&St. Regis Hotel: Junior Suite 2 Podium 5 Level	 &53,682.00 	&P5	 &22-May-14	&\ghot \\
75	&St. Regis Hotel: Junior Suite 3 Podium 5 Level	 &72,042.00 	&P5	 &22-May-14	&\ghot \\
76	&St. Regis Hotel: Junior Suite 4 Podium 5 Level	 &50,376.00 	&P5	 &22-May-14	&\ghot \\
77	&St. Regis Hotel: Corridor Podium 5 Level	 &461,461.00 	&P5	   &10-May-14	&\ghot \\
78	&St. Regis Hotel: Presidential Suite Podium 6 Level	 &145,536.00 	&P6	 &3-May-14	&\ghot \\
79	&St. Regis Hotel: Royal Suite Podium 6 Level	 &521,004.00 	&P6	 &3-May-14	&\hot \\
80	&St. Regis Hotel: Junior Suite 1 Podium 6 Level	 &72,042.00 	&P6	 &3-May-14	&\ghot \\
81	&St. Regis Hotel: Junior Suite 2 Podium 6 Level	 &50,376.00 	&P6	 &3-May-14	&\ghot \\
82	&St. Regis Hotel: Corridor Podium 6 Level	 &166,150.00 	&P6	 &26-May-14	&\ghot \\
83	&St. Regis Hotel: Special Cabanas F01 Level	 &42,878.00 	&F01	&9-May-14	&\hot \\
84	&St. Regis Hotel: Male \& Female Changing Room F01 Level	 &25,553.00 	&F01	&9-May-14	&\hot \\
85	&St. Regis Hotel: Royal Private Swimming Pool F01 Level	 &105,096.00 	&F01	&9-May-14	&\hot \\
86	&St. Regis Hotel: Pantry F01 Level	 &18,216.00 	                 &F01	&9-May-14	&\hot \\
87	&St. Regis Hotel: Special Cabanas F01 Level	 &42,895.00 	&F01	&9-May-14	&\hot \\
88	&St. Regis Hotel: Kids Play Room F01 Level	 &148,500.00 	&F01	&9-May-14	&\hot \\
89	&St. Regis Hotel: Male \& Female Changing Room F01 	 &52,008.00 	&F01	&9-May-14	&\hot \\
90	&St. Regis Hotel: Pantry F01 Level	           &25,889.00 	&F01	&9-May-14	&\hot \\
91	&St. Regis Hotel: Bar Pagoda F01 Level	 &69,603.00 	&F01	 &9-May-14	&\hot \\
92	&St. Regis Hotel: Lift's Lobbies F01 Level	 &41,250.00 	&F01	&2-May-14	&\hot \\
\midrule
   &\textbf{Total for PS - Electrical}	 &\textbf{13,266,154.00} &&&\\

\end{pstable}

\subsection{Electrical Westin Hotel}

Supply, installation and connection of electrical works including lighting outlets, lighting switches, lighting control, socket outlets, telephone outlet, TV outlet, speakers, internal conduits and wiring, and all necessary accessories

\begin{pstable}
1	&W. Hotel: W. Hotel Entrance Lobby Ground Level	 &191,100.00 	&GF	 &17-Apr-14	&\hot \\
2	&W. Hotel: Theatre Entrance Lobby Mezzanine Level	 &224,437.00 	&MZ	 &17-Apr-14	&\hot \\

3	&W. Hotel: Public Lifts Mezzanine Level	 &35,959.00 	&MZ	 &17-Apr-14	&\hot \\

4	&W. Hotel: Corridor Mezzanine Level	 &10,368.00 	&MZ	 &17-Apr-14	&\hot \\

5	&W. Hotel: Male/Female Toilets Mezzanine Level	 &61,785.00 	&MZ	 &17-Apr-14	&\hot \\


6	&W. Hotel: Kitchen Podium 1 Level	 &62,743.00 	&P1	 &19-Apr-14	& \hot \\

7	&W. Hotel: Kitchen Podium 1 Level	 &44,945.00 	&P1	 &19-Apr-14	& \hot \\

8	&W. Hotel: Kitchen Tech 1 Level	    &29,695.00 	&T1	 &19-Apr-14	&\hot \\

9	&W. Hotel: Festival Dining Restaurant 3 Area 276m2 PL1	 &133,480.00 	&P1	 &30-Apr-14	&\hot \\

10	&W. Hotel: Public Lifts Podium 1 Level	 &146,898.00 	&P1	 &30-Apr-14	& \hot \\

11	&W. Hotel: Male/Female Toilets Tech 1 Level	 &19,929.00 	&T1	 &30-Apr-14	& \hot \\

12	&W. Hotel: Lifts Lobby Tech 1 Level	 &191,363.00 	&T1	 &30-Apr-14	& \hot \\

13	&W. Hotel: KitchenPodium 2 Level	 &38,590.00 	&P2 &22-Apr-14	& \hot \\

14	&W. Hotel: Jazz Room Podium 2 Level	 &306,873.00 	&P2	 &3-Apr-14	& \hot \\

15	&W. Hotel: Public Lift Lobby Podium 2 Level	 &229,092.00 	&P2	 &3-Apr-14	& \hot \\

16	&W. Hotel: Female WC Podium 2 Level	 &8,140.00 	&P2	 &3-Apr-14	& \hot \\

17	&W. Hotel: Male WC Podium 2 Level	 &13,140.00 	&P2	 &3-Apr-14	&\hot \\

18	&W. Hotel: W-Great Room Podium 4 Level	 &434,605.00 	&P4	 &13-Apr-14	&\hot \\

19	&W. Hotel: Great Room Pantry Podium 4 Level	 &83,182.00 	&P4	 &13-Apr-14	&\hot \\

20	&W. Hotel: Business Centre Podium 4 Level	 &50,479.00 	&P4	 &13-Apr-14	&\hot \\

21	&W. Hotel: Male \& Female WC Podium 4 Level	 &37,752.00 	&P4	 &13-Apr-14	&\hot \\

22	&W. Hotel: Lift Lobby Podium 4 Level	 &22,517.00 	&P4	 &13-Apr-14	& \hot \\

23	&W. Hotel: Pre Function Hall Podium 4 Level	 &157,487.00 	&P4	 &13-Apr-14	& \hot \\

24	&W. Hotel: W-Studio 1 Podium 5 Level	 &59,004.00 	&P5	 &23-Apr-14	& \hot \\

25	&W. Hotel: Pre Function Podium 5 Level	 &101,954.00 	&P5	 &23-Apr-14	&\hot \\

26	&W. Hotel: Lift Lobby Podium 5 Level	 &17,226.00 	&P5	 &23-Apr-14	&\hot \\

27	&W. Hotel: W-Studio 4 Podium 5 Level	 &22,154.00 	&P5	 &23-Apr-14	& \hot \\

28	&W. Hotel: W-Studio 5 Podium 5 Level	 &19,399.00 	&P5 &23-Apr-14	& \hot \\

29	&W. Hotel: Male \& Female WC Podium 5 Level	 &40,018.00 	&P5	 &23-Apr-14	& \hot \\

30	&W. Hotel: W-Studio 3 Podium 5 Level	 &28,666.00 	&P5	 &23-Apr-14	& \hot \\

31	&W. Hotel: W-Studio 2 Podium 5	 &28,160.00 	&P5	 &23-Apr-14	& \hot \\

32	&W. Hotel: W Specialty Restaurant F01 Level	 &597,564.00 	&F01	&12-Apr-14	& \hot \\

33	&W. Hotel: WOW Lifts Lobby F01 Level	 &42,884.00 	&F01	&12-Apr-14	& \hot \\

34	&W. Hotel: Fantastic Suite F02 Level	 &51,126.00 	&F02	&8-Apr-14	& \hot \\

35	&W. Hotel: WOW/Parlour Suite F02 Level	 &53,598.00 	&F02	&8-Apr-14	& \hot \\

36	&W. Hotel: Studio Suite F02 Level	 &41,382.00 	&F02	&8-Apr-14	& \hot \\
37	&W. Hotel: Fabulous Room F02 Level	 &29,233.00 	&F02	&8-Apr-14	& \hot \\

38	&W. Hotel: Party Room F02 Level	 &27,005.00 	&F02	&8-Apr-14	& \hot \\

39	&W. Hotel: Corridor F02 Level	 &100,007.00 	&F02	&8-Apr-14	&\hot \\

40	&W. Hotel: Fantastic Suite TP1 Level	 &64,542.00 	&TP1	&15-Apr-14	& \hot \\

41	&W. Hotel: WOW/Parlour Suite TP1 Level	 &67,938.00 	&TP1	&15-Apr-14	& \hot \\

42	&W. Hotel: Studio Suite TP1 Level	 &42,306.00 	&TP1	&15-Apr-14	& \hot \\

43	&W. Hotel: Fabulous Room TP1 Level	 &29,233.00 	&TP1	&15-Apr-14	&\hot \\

44	&W. Hotel: Party Room TP1 Level	 &27,005.00 	&TP1	&15-Apr-14	& \hot \\
45	&W. Hotel Corridor TP1 Level	 &95,348.00 	&TP1	&15-Apr-14	& \hot \\

46	&W. Hotel: Fantastic Suite TP2 Level	 &64,884.00 	&TP2	&12-Apr-14	& \hot \\

47	&W. Hotel: WOW/Parlour Suite TP2 Level	 &69,612.00 	&TP2	&12-Apr-14	& \hot \\

48	&W. Hotel: Studio Suite TP2 Level	 &42,678.00 	&TP2	&12-Apr-14	& \hot \\

49	&W. Hotel: Fabulous Room TP2 Level	 &29,233.00 	&TP2	&12-Apr-14	& \hot \\

50	&W. Hotel: Party Room TP2 Level	 &27,005.00 	&TP2	&12-Apr-14	& \hot \\

51	&W. Hotel: Corridor TP2 Level	 &95,348.00 	&TP2	&12-Apr-14	& \hot \\

52	&W. Hotel: WOW Suite F18 Level	 &124,062.00 	&F18	&12-May-14	&\hot \\

53	&W. Hotel: WOW Suite F18 Level	 &98,436.00 	&F18	&12-May-14	& \hot \\

54	&W. Hotel: Studio Suite F18 Level	 &45,396.00 	&F18	&12-May-14	& \hot \\

55	&W. Hotel: Fabulous Suite F18 Level	 &39,204.00 	&F18	&12-May-14	& \hot \\
56	&W. Hotel: Emergency Lifts Lobby F18 Level	 &19,173.00 	&F18	&12-May-14	& \hot \\
57	&W. Hotel: Corridor F18 Level	 &93,918.00 	&F18	&12-May-14	& \hot \\

58	&W. Hotel: WOW Suite F19 Level	 &124,062.00 	&F19	&25-May-14	& \hot \\

59	&W. Hotel: WOW Suite 2 F19 Level	 &98,436.00 	&F19	&25-May-14	& \hot \\

60	&W. Hotel: Studio Suite F19 Level	 &45,396.00 	&F19	&25-May-14	& \hot \\

61	&W. Hotel: Fabulous Suite F19 Level	 &31,890.00 	&F19	&25-May-14	& \hot \\

62	&W. Hotel: Party Room F19 Level	 &27,005.00 	&F19	&25-May-14	& \hot \\

63	&W. Hotel: Corridor F19 Level	 &93,918.00 	&F19	&25-May-14	& \hot \\

64	&W. Hotel: Extreme WOW Suite F20 Level	 &198,558.00 	&F20	&4-May-14	& \hot \\

65	&W. Hotel: Studio Suite F20 Level	 &45,396.00 	&F20	&4-May-14	 & \hot \\

66	&W. Hotel: Fabulous Suite F20 Level	 &39,204.00 	&F20	&4-May-14	 & \hot \\

67	&W. Hotel: Emergency Lifts Lobby F20 Level	 &19,173.00 	&F20	&4-May-14	& \hot \\

68	&W. Hotel: Corridor F20 Level	 &102,707.00 	&F20	&4-May-14	& \hot \\

69	&W. Hotel: Extreme WOW Suite TP3 Level	 &139,116.00 	&TP3	&15-May-14	& \hot \\
70	&W. Hotel: Studio Suite TP3 Level	 &45,396.00 	&TP3	&15-May-14	& \hot \\

71	&W. Hotel: Fabulous Suite TP3 Level	 &31,890.00 	&TP3	&15-May-14	& \hot \\

72	&W. Hotel: Party Room TP3 Level	 &27,005.00 	&TP3	&15-May-14	& \hot \\

73	&W. Hotel: Corridor TP3 Level	 &91,933.00 	&TP3	&15-May-14	&\hot \\
74	&W. Hotel: Extreme WOW Suite F22 Level	 &198,558.00 	&F22	&25-May-14	&\hot \\
75	&W. Hotel: Studio Suite F22 Level	 &45,396.00 	&F22	&25-May-14	&\hot \\
76	&W. Hotel: Fabulous Suite F22 Level	 &31,890.00 	&F22	&25-May-14	& \hot \\
77	&W. Hotel: Party Room F22 Level	 &27,005.00 	&F22	&25-May-14	& \hot \\
78	&W. Hotel: Corridor F22 Level	 &91,933.00 	&F22	&25-May-14	& \hot \\
79	&W. Hotel: Back of House Pantry F24 (F25) Level	 &32,570.00 	&F25	&12-May-14	&\hot \\
80	&W. Hotel: W-Living Room F24 (F25) Level	 &519,261.00 	&F25	&25-May-14	&\hot \\
81	&W. Hotel: Male \& Female WC F24 (F25) Level	 &52,707.00 	&F25	&25-May-14	&\hot \\
82	&W. Hotel: Lifts Lobby F24 (F25) Level	 &22,077.00 	&F25	&25-May-14	&\hot \\
83	&W. Hotel: Deputy GM F24 (F25) Level	 &25,823.00 	&F25	&25-May-14	&\hot \\
84	&W. Hotel: Lifts Lobby 2 F24 (F25) Level	 &32,791.00 	&F25	&25-May-14	&\hot \\
85	&W. Hotel: Back of House Kitchen F25L (F26) Level	 &24,165.00 	&F26	&16-May-14	&\hot \\
86	&W. Hotel: Restaurant F25L (F26) Level	 &430,694.00 	&F26	&29-May-14	&\hot \\
87	&W. Hotel: Male \& Female WC F25L (F26) Level	 &52,707.00 	&F26	&29-May-14	&\hot \\
88	&W. Hotel: Lifts Lobby 1 F25L (F26) Level	 &21,978.00 	&F26	&29-May-14	&\hot \\
89	&W. Hotel: Lifts Lobby 2 F25L (F26) Level	 &13,211.00 	&F26	&29-May-14	&\hot \\
90	&W. Hotel: Destination Restaurant F26L (F28) Level	 &568,524.00 	&F28	&15-May-14	&\hot \\
91	&W. Hotel: BOH Pantry F26L (F28) Level	 &15,862.00 	&F28	&15-May-14	&\hot \\
92	&W. Hotel: Lift Lobby F26L (F28) Level	 &22,077.00 	&F28	&15-May-14	&\hot \\
93	&W. Hotel: Male \& Female WC F26L (F28) Level	 &36,872.00 	&F28	&15-May-14	&\hot \\
94	&W. Hotel: Entrance/WOW Lift Lobby F26L (F28) Level	 &22,471.00 	&F28	&15-May-14	&\hot \\
95	&W. Hotel: VIP's Deck F27 (F30) Level	 &175,373.00 	&F30	&15-May-14	&\hot \\
\midrule
	&\textbf{Total for Provisional Sums - Electrical}	 &8,494,290.00 &&&\\			

\end{pstable}
\bigskip

\section{Westin Electrical PS}

Supply, installation and connection of electrical works including lighting outlets, lighting switches, lighting control, socket outlets, telephone outlet, TV outlet, speakers, internal conduits and wiring, and all necessary accessories

\begin{pstable}
1	&Westin Hotel: Westin Bar Ground Level	 &139,700.00 &GF	&3-Mar-14	&\hot  \\
2	&Westin Hotel: Westin Bar Kitchen Ground Level	 &26,500.00 	&GF	 &3-Mar-14 &\hot \\
3	&Westin Hotel: Lobby/Reception/Entrance Ground Level	 &676,000.00 	&GF	 &3-Mar-14 &\hot \\
4	&Westin Hotel: Wash Rooms Ground Level	 &41,250.00 	&GF &3-Mar-14	&\hot \\

5	&Westin Hotel: Retail Ground Level	 &13,750.00 	&GF &3-Mar-14 &\hot \\
6	&Westin Hotel: Cocktail Room Ground Level	 &123,200.00 	&Gf	 &3-Mar-14	&\hot \\

7	&Westin Hotel: Concierge Ground Level	 &11,550.00 	&GF	 &3-Mar-14 &\hot \\
8	&Westin Hotel: Pantry Ground Level	 &24,750.00 	&GF	 &3-Mar-14	&\hot \\

9	&Westin Hotel: Westin Bar Upper Level Mezzanine Level	 &112,750.00 	&MZ &3-Mar-14 &\hot \\

10	&Westin Hotel: Food Stations Podium 1 Level	 &853,600.00 	&P1	 &22-Apr-14	& \unavailable \\

11	&Westin Hotel: Coffee Pantry Podium 1 Level	 &57,200.00 	&P1	 &22-Apr-14	& \hot \\
12	&Westin Hotel: Toilets Podium 1 Level	 &42,750.00 	&P1 &22-Apr-14	&\hot \\

13	&Westin Hotel: Kitchen Podium 1 Level	 &148,000.00 	&P1	 &10-Apr-14	&\hot \\
14	&Westin Hotel: Corridor Podium 2 Level	 &71,500.00 	&P2	 &11-Apr-14	& \hot \\

15	&Westin Hotel: Meeting Rooms Podium 2 Level	 &521,400.00 	&P2 &11-Apr-14	&\ghot \\
16	&Westin Hotel: Corridor of Meeting Rooms Podium 2 Level	 &386,100.00 	&P2 &11-Apr-14	&\ghot \\

17	&Westin Hotel: Toilets Podium 2 Level	 &47,250.00 	&P2 &11-Apr-14	&\ghot \\
18	&Westin Hotel: Kitchen Podium 2 Level	 &52,500.00 	&P2	 &29-Apr-14	&\hot \\

19	&Westin Hotel: Administration Offices Podium 4 Level	 &1,037,300.00 	&P4	 &23-Apr-14	&\ghot \\
20	&Westin Hotel: Toilets Podium 4 Level	 &34,200.00 	&P4	 &23-Apr-14	&\ghot \\

21	&Westin Hotel: W/Westin Gym/Spa Lower Level Podium 6 Level &1,125,850.00 	&P6	 &12-Apr-14	&\unavailable \\
22	&Westin Hotel: Corridor \& Lobby F01 Level	 &289,250.00 &F01	&19-Apr-14	&\unavailable \\

23	&Westin Hotel: W/Westin Gym/Spa Upper Level F01 Level	 &676,500.00 &F01	&19-Apr-14 &\unavailable \\
24	&Westin Hotel: W-Sweat Fitness Room F01 Level	 &112,750.00 	&F01	&19-Apr-14	&\unavailable \\

25	&Westin Hotel: Corridor \& Lobby for each floor in F02 to F31 Level	 &5,206,500.00 	&F31	&15-Apr-14& Completed up to 22 Floor as of 20 may 2015.	\\

26	&Westin Hotel: Presidential Suite F32 Level	 &241,800.00 	&F32	&3-May-14	 &\unavailable \\
27	&Westini Hotel: Corridor \& Lobby F32 Level	 &192,400.00 	&F32	&3-May-14	 & \unavailable\\
28	&Westin Hotel: Executive Lounge F33 Level	 &391,600.00 	&F33	&18-May-14	&\unavailable \\

29	&Westin Hotel: Corridor \& Lobby F33 Level	 &189,150.00 	&F33	&18-May-14	&\unavailable \\
30	&Westin Hotel: Royal Suite F34 Level	 &267,600.00 	&F34	&26-May-14	&\unavailable  \\

31	&Westin Hotel: Corridor \& Lobby F34 Level	 &228,150.00 	&F34	&26-May-14	&\unavailable \\
32	&Westin Hotel: Lounge and Restaurant F35 Level	 &482,350.00 	&F35	&25-May-14	&\unavailable \\
33	&Westin Hotel: Outdoor Deck F35 Level	 &102,850.00 	&F35	&25-May-14	&\unavailable \\
34	&Westin Hotel: Specialty Restaurant F35 Level	 &517,000.00 	&F35	&25-May-14	&\unavailable \\
35	&Westin Hotel: Outdoor Deck F35 Level	 &102,850.00 	&F35	&25-May-14	&\unavailable\\
\midrule
	&Total for Provisional Sums - Electrical	 &14,547,850.00 	&&&\\		
\end{pstable}




% \begin{epigraphpage}
 \epigraph{Begin at the beginning,'' the King said, gravely, ``Then
 go till you come to the end; then stop.''}{Lewis Carroll, {\it Alice
 in Wonderland}}

 \epigraph{You can never get a cup of tea large enough or a book long enough to
 suit me''}{C. S. Lewis}
 \end{epigraphpage}

\parindent1em
%\cxset{style13}
%\cxset{title margin bottom=10pt,
%          title beforeskip=1pt}

\chapter{Introduction}
\addtocimage{-12pt}{-20pt}{../images/tocblock-fish}


\epigraph{``Begin at the beginning,'' the king said
"and then go on till you come to the end, then stop."}{
---Lewis Carroll, Alice in Wonderland}

 \parskip3pt plus 5pt 
\noindent This package and its documentation attempts to eliminate some common 
problems encountered when using \LaTeX2e. The first one is the loading of 
recommended packages for a large and perhaps complicated document and 
the second is the re-designing of styles for a document.

 \LaTeX2e, does not provide a standard library, but comes equipped with
 a package mechanism that allows code extensions to be loaded as required.
 This has created a strong vibrant community, hundreds of packages and a 
 headache to both new and seasoned users. What packages are available, when
 to use them and in which order is a common theme for many questions on
 lists and |TX.SE|.

 It is quite common during the writing of a thesis or book
 for the author to keep on adding macros and packages
 at the preamble of the document. In most cases this can
 be satisfactory but in many others it leads to
 incompatibilities and errors. This package aims at
 minimizing one's preamble, by prefetching a number of
 commonly used packages. It also aims at loading them
 in the right order and providing patches for conflicts.
 
 I am hoping that using this package, will lead to less
 frustrations with the intricacies of \LaTeX2e\ packages.

The package code is complicated, but its usage is simple. You first load the package and then
you use one of the available templates:

 \begin{commands}[]{}
 \begin{verbatim}
 \usepackage{phd}
 \usetemplate{style13}
 \end{verbatim}
 \end{commands}

This is what you need to typeset a good looking book or thesis. The rest of this book is a footnote and you can skip them if you want. 

It will be better for the longer projects to just fork the
 package and adapt it to your needs. In this respect, I have
 uploaded the package to |github|.\footnote{\url{https://github.com/yannisl/phd}}

 My goal in selecting the packages and adding a number of 
 commands for the authors was to be able to typeset a 
 document for most common use cases, without the need of
 additional packages. The packages I selected are biased
 towards academic publications, although they can find use
 in almost any fields. The package provides a mechanism via
 PGF keys to provide a settings file. 
 
 Most of the documentation can be found in the implementation part.

Browse any books in a library or bookshop and the striking thing is that their design is very individualistic. They might have similarities but their main features vary. In many respects they resemble people's faces where minor differences have striking effects.

This package arose out of a question at stackexchange. How to redefine chapter heads. Having seen the popularity of the |pgf| package \cite{pkg-pgf} I realized that \latex users prefer this method of styling rather the traditional \latex method.

The user interface can be extended to basically all major packages. The principle is to keep to a minimum changes that can affect the LaTeX core commands. If there are any additions a key setting is provided to be able to revert back to normal LaTeX.

The workflow can be simplified. In addition I want to believe that the interface can provide a useful addition to the open source community and that other people will contribute style libraries, which will be simpler to write. It is also possible
to device an easy and uncomplicated web interface to handle
such a great number of variables.


Most people when they get started with \LaTeX\ will either use one of the standard classes such as the \docFile{book.cls} or one of the generic classes notably koma-script or memoir. Most students will be forced to use on of the many thesis classes available.

\section{The key value concept}

The key-value concept that originated with \LaTeX\ has been extended many times, the last and most serious implementation of it by Tantau in the PGF package. What essentially Tantau developed is a scripting language to script TeX code. The \tikzname and pgfplots packages are two major packaged that use keys effectively. Their popularity is growing and what this package does is to offer a user interface that has been modelled to be similar to that of \texttt{css} (cascade style sheets). 
\smallskip

\begin{scriptexample}{}{}
\textit{chapter number} font-size = Large,\\
\textit{chapter number}     color = theblue
\end{scriptexample}
\smallskip

The main idea behind the package, is that you are configuring a document style by means of \emph{settings} rather than writing macros. In the example above the \emph{number, chapter} can be thought of as class or id names in css style sheets and the |font-size, color| as property settings that apply to the particular element. 


\subsection{Settings}

Settings are activated either by using the command |\cxset|  or by loading a full style sheet. In most cases you will probably import a style sheet and then modify some of the properties using |cxset|.  For example this heading has a dot after the subsection number. This was accomplished by setting,

We can de-activate it for the next and subsequent subsection headings with the setting:

\lorem

\begin{scriptexample}{}{}
\begin{verbatim}
\cxset{subsection number after=\quad}
\end{verbatim}
\end{scriptexample}




\subsection{Cascading}

Most values once set for a higher section will be seen in a cascade by all subsectioning commands in a similar fashion similar to CSS. These include properties such as color, font families and alignment. Best though to specify all of them for maximum flexibility to your users.

\section{On typography}

This package hopefully will assist in improving the typography of books set with \latexe. Any typographical comments on the various styles are just my own ramblingss and not necessarily absolute truths. Like fashion and art typography has opinions rather than absolute truths. In many styles the design is slightly adapted to blend a bit better with this manual. Also I did not select fonts as per the samples but this is left on you the user to decide.



\section{Packages and Fonts}

This manual has been typeset with numerous fonts in order to enable the typsetting of almost all the scripts provided by the Unicode standard. In order to process it from the |.dtx| file, these fonts must be available in your system, otherwise \XeLaTeX\ will have a problem finding the fonts and it will take an awful long time to process. This is especially true for the scripts section, where virtually all the Unicode defined scripts are discussed. You will need a fast computer and a fast hard disk to process the document within a reasonable time. When using \pkg{fontspec} always define your fonts with the \cmd{\newfontfamily} this will speed up processing by an order of magnitude. Compiling from the command prompt will speed up compilation. Average speed 2-3 pages per second.

Many of \tex's parameters are stretched to the limit with a complicated document such as this manual. You will require a full distribution otherwise expect some errors. Important packages is \pkg{morefloats} and \pkg{morewrites}. The package will also expect that you have |e-tex| installed. Ubuntu users are normally one year behind in updates, so you might wish to update manually. It will take upwards of 5 minutes to compile fully on an old laptop and a couple of minutes on a state of the art computer.

The |dtx| should be processed best with its own make file provided for Windows only |phd-lua.bat|. The make file will process the documentation using \lualatex. You can also process the document with \xelatex but is prone to produce errors. Using \latexe the sections on scripts etc will not be printed and a much shorter version of the manual is provided. 

\section{Scripts and Languages}

The package and the documentation offer a full repertoire of font selection keys for different scripts and languages. It hasn't been possible, however hard I tried to compile this section of the documentation with \xelatex, as it kept giving errors of too many files open. This was also not possible even with the \pkg{morewrites} package loaded. With \lualatex the document compiled with no major problems other than the font rendering being of a lower quality to that of XeLaTeX on windows, other than disabling incompatible packages and a number of commands that were redefined. 

Some good news for multi-script typesetting is the |Noto| fonts from Google. These fonts named Noto from "No Tofu" meaning you do not see any little square blocks for undefined glyphs, are fast to load. Disantvantage you need to switch between font commands fairly often.

\section{This book}

When developing the templates, I started using \emph{lorem ipsum} text as samples. Half-way through this
became a jumble mass of uninteresting pages interspersed with code. Headings and the contents of the book
determine both the structure and the selection of fonts, so I went back and wrote narratives  to accompany
the headings. Many of the narratives are semi-autobiographical in nature; others are clustered around books I read and my own interests. Some I stumbled on them accidentally and are mostly there to demonstrate some code.

Besides the templates and the code there is another narrative which is based on notes I kept on \tex and its friends over the years and are offered as a more advanced introduction to coding \latexe and \tex. The whole manual was typeset in a |ltxdoc| class, slightly modified to turn into a book class.

The implementation code is also available and it was mostly for my own benefit. The whole manual with the exception of the |\cxset| introduction, is just a test document. The notes and the “dissection” of the standard \latexe and the standard classes are there to explain the background to the many coding decisions that I took while I was developing the package.

PhD students are notorious for going in all directions and exploring many adjacent fields before they sit down and write their theses. Some become life-time students. To all these new men and women of the Renaissance that slave away to inch knowledge one thesis at a time, I dedicate this book and the name of the package.

\subsection{The TeX hacking sections}

To start programming \tex you need to have a knowldge of \tex basic commands and approach. \latex2015 is a format build on top of \tex to provide a more structured approach. To program \latexe packages you need to understand \latexe concepts, code organization and conventions. To program in \latex3, you need to learn a whole new language and you still need to understand \tex, \latexe and the expl3 language and conventions. To program using LuaTeX, other than the Lua language you need to understand \tex very well.
None of these can be found in one place.  I have gathered a lot of material and put it together. This is not a language you can master easily or quickly, but can teach you a lot about typesetting, computer science and many other interesting topics.


 \section{Version control with Git and Github}
 
 If you are involved with code or a publication that will have frequent changes, you should consider
 some type of version control system. My own recommendation is to use |git| and an online repository such
 as |github|. The latter is currently very fashionable and makes sharing code easier. Note that the |github|
 offers both public as well as private repositories. The general recommendation is that for unpublished work
 such as a thesis or code under development, it is preferable to go for a private repository. 
 
 \lorem\lorem

 \section{Ordering of Packages}
 
One package that normally leads to errors is the 
\pkg{hyperref}. The package which is an outstanding example of software engineering and supported single handledy by Heiko Oberdiek\footcite{hyperref} redefines a a lot of internal commands of the kernel. As a lot of other packages do the same it has to be loaded at the end of the preable with the exception of some packages! 
 
 This manual is typeset according to the conventions of the
 \LaTeX \textsc{docstrip} utility which enables the automatic
 extraction of the \LaTeX{} macro source files~\cite{GOOSSENS94}.

 
 \href{http://tex.stackexchange.com/questions/96350/problem-with-algorithmic-and-hyperref}{problem with algorithmic and hyperref}

 \begin{verbatim}
\usepackage{float}  % load float package first!

\usepackage{hyperref} % let hyperref patch the float package stuff
.
 \usepackage{algorithm} % let algorithm use the patched version of the float package
 \end{verbatim}
 

\section{Known problems}

Perhaps the biggest issue with the package is the speed of
compilation with \XeLaTeX\ or \LuaTeX. This is to be expected, as both engines spend a lot of resources in font management. On demand loading of packages is something I have in the back of my mind. This should be done via document styles i.e., if a book is for the humanities, perhaps only a rudimentary amount of maths packages should be loaded.

\section{Future Directions}

\latexe and \tex usage appears to be increasing. This is mostly by programs that export results with \latexe code rather than authors writing books.  The method adopted here is easier to automate all sorts of reports and automated texts. I would like too develop a web interface for processing such templates and at the same time export into html instead of just producing pdfs. I have already a prototype.   

\section{Tooling}

Some of the scripts on a Windows machine need MSYS\footnote{\url{http://mingw.org/wiki/MSYS}}









%
%\makeatletter\@specialfalse\makeatother
%%%%%%%%%%%%%%%%%%%%%%%%%%%%%%%%%%%%%%%%%%%
%%%%%%  STYLE 01
%%%%%%%%%%%%%%%%%%%%%%%%%%%%%%%%%%%%%%%%%%%


\cxset{
 name={},
 numbering=arabic,
 number font-size=\LARGE,
 number font-family=\rmfamily,
 number font-weight=\bfseries,
 number before=,
 number dot=,
 number after=,
 number position=leftname,
 chapter font-family=\sffamily,
 chapter font-weight=\normalfont,
 chapter font-size=\Large,
 chapter before={\vspace*{20pt}\par},
 chapter after={\hfill\hfill\par},
 chapter color={black!90},
 number color=\color{purple},
 title beforeskip={\vspace*{30pt}},
 title afterskip={\vspace*{40pt}\par},
 title before={},
 title after={},
 title font-family=\sffamily,
 title font-color=\color{purple},
 title font-weight=\bfseries,
 title font-size=\LARGE,
 header style=headings}

\cxset{headings ruled-01}

\chapter{Introduction to Style One}


\begin{summary}
This design is simple and its distinguishing characteristic is a short summary at the beginning of the chapter. This is almost like an abstract typeset in italic font without setting the margins in. We provide a \lstinline{summary} environment for convenience. Note the very simple line in the running head to the left of the page number.
\end{summary}

\medskip
\begin{figure}[ht]
\centering
\includegraphics[width=0.5\textwidth]{./chapters/chapter01}
\end{figure}


%\clearpage
\makeatletter\@debugtrue\makeatother
\cxset{
 chapter toc=true,
 name=CHAPTER,
 chapter numbering=ORDINALS,
 number font-size=Large,
 number font-family=rmfamily,
 number font-weight=bfseries,
 number before=\kern0.5em,
 number dot=,
 number after=\hfill\hfill\par,
 number position=rightname,
 chapter font-family=rmfamily,
 chapter font-weight=bold,
 chapter font-size=Large,
 chapter before={\vspace*{20pt}\par\hfill},
 chapter after=,
 chapter color=black,
 number color=black,
 %
 title margin top=10pt,
 title before=\par\nointerlineskip\hfill,
 title after=\hfill\hfill\par\nointerlineskip,
 title font-family=rmfamily,
 title font-color= black,
 title font-weight=bfseries,
 title font-size=LARGE,
 chapter title width=0.8\textwidth,
 chapter title align=centering,
 title margin-left=0pt,
 author block=false}

\debugtitle
\debugchapter
\chapter[Template 2]{Mondino, the Restorer of Anatomy}

The archive.org is an extraordinary hunting ground  for typographical surprises. On a recent excursion to find some books on Versalius I stubled on a book titled \emph{Andreas Vesalius, the reformer of anatomy} by  Ball, James Moores. It is an old book published in 1910 and has a couple of unusual features. Check the figure below and see if you can identify the challenging feature.

\begin{figure}[ht]
\centering
\includegraphics[width=0.8\textwidth]{versalius}
\caption{J.B. Moore’s \emph{Andreas Versalius, the Reformer of Anatomy} has many unusual features, including chapter numbers using ordinals. }
\end{figure}

\cxset{chapter toc=true,
          chapter opening=anywhere}
          
\chapter{The Template}          
The template is called \emph{Versalius} and is stored under style02. It can be loaded in the normal way using:
\begin{verbatim}
\usepackage[style02]{phd}
\end{verbatim}

I have not reproduced the full extend of the book’s requirements, as some details are quite cumbersome to be automated through \tex. These though can easily be incorporated in a manual way. More about this later.


\section{The Table of Contents}
Another interesting aspect of this book, which is common with many books of its period is the ToC. The ToC shows the full range of the chapter pages, i.e., it is marked as Page 1-16 rather than the common practice nowdays that indicates only the starting page of the chapter. It also has “TABLE OF CONTENTS”  as a heading and not just contents as you would expect from today’s books.

\begin{figure}[ht]
\centering
\includegraphics[width=0.8\textwidth]{versalius-01}
\caption{J.B. Moore’s \emph{Andreas Versalius, the Reformer of Anatomy} has many unusual features, including chapter numbers using ordinals. }
\end{figure}

\section{List of Illustrations}

\begin{figure}[ht]
\centering
\includegraphics[width=0.8\textwidth]{versalius-02}
\caption{J.B. Moore’s \emph{Andreas Versalius, the Reformer of Anatomy} has many unusual features, including chapter numbers using ordinals. }
\end{figure}

\section{The Frontmatter}
As a foreward there is an unumbered chapter called ``Introduction’’. The chapter heading also has a head band.
\begin{figure}[ht]
\centering
\includegraphics[width=0.8\textwidth]{versalius-03}
\caption{J.B. Moore’s \emph{Andreas Versalius, the Reformer of Anatomy} has many unusual features, including chapter numbers using ordinals. }
\label{lettrine}
\end{figure}

\bgroup
\centering
\includegraphics[width=0.7\textwidth]{versalius-headband}

\LARGE\bfseries INTRODUCTION\par
\egroup
\def\dropcapversalius{%
\vbox to 0pt{\vskip6pt\leavevmode\noindent\includegraphics[width=2.39cm]{versalius-dropcap}%
}%
}
\parindent0pt

\hangindent2.6cm \hangafter0
\dropcapversalius \textsc{he dropcap will have to be inserted}, either using the lettrine package or do be achieved via a parshape command and manual entry. You can also write your own macro command using the details we provide under the Paragraphs chapter. On this page I have manually inserted it, as I used an image from the book for the dropcap. If you were to use the template for a full book, it will be then preferable to use

the lettrine package to set the dropcaps. If you observe Figure~\ref{lettrine} carefully, you will notice the first line of theopening paragraph is in small caps. As \tex typesets the full paragraph this is almost an impossible task to achieve through normal \tex commands and in order not to overcomplicate the discussion it can be achieved manually via trial and error. 

\section{Figures}

Most of the figures are wrapped illustrations. A couple are full page figures and bear no caption numbering. One such illustration is shown on page~\pageref{fig:vesalius}. Do note that the List of Illustrations does have the illustrations listed with additional information to that shown in the captions. 

\begin{figure}[p]
\centering
\includegraphics[width=\textwidth]{vesalius}
\centering
ANDREAS VESALIUS\par
(From an old copperplate engraving)\par
\label{fig:vesalius}
\end{figure}







%\newgeometry{top=-10pt,bottom=2cm}

\tcbset{width=\paperwidth,boxrule=0pt,right=3cm,arc=0pt}

\cxset{style03/.style={
 name={},
 numbering=arabic,
 number font-size= HUGE,
 number font-family= rmfamily,
 number font-weight= bfseries,
 number before=\par\vspace*{10pt}\hfill\hfill,
 number dot=.,
 number after=,
 number position=rightname,
 chapter font-family= sffamily,
 chapter font-weight=normalfont,
 chapter font-size= Large,
 chapter before={\hspace*{-2.5cm}\vbox\bgroup\tcolorbox\bgroup\vspace*{20pt}\hfill\hfill},
 chapter after={\par\vspace*{15pt}},
 chapter color=black!90,
 number color= thered,
 title beforeskip={},
 title afterskip={\vspace*{10pt}\par},
 title before={\hfill\hfill},
 title after={\vspace*{60pt}\egroup\endtcolorbox\egroup},
 title font-family=\sffamily,
 title font-color= thered,
 title font-weight=\bfseries,
 title font-size=\Huge}}

\cxset{style03}

\chapter{Introduction Style Three}

This is not an exact reproduction as I am still thinking as to how to use
specials with the package. You can vary it by setting the tcolorbox settings as well as the geometry settings.
\medskip

\begin{figure}[ht]
\centering
\includegraphics[width=0.39\textwidth]{./chapters/chapter03}
\end{figure}

This setting involves changing the geometry of the page as well as adding the chapter name and title in a color box. For this I have used the \lstinline{tcolorbox}. Of course you can use any other shaded environment you feel comfortable with such as mdframed. It is important to set the colorbox parameters.

\begin{lstlisting}
\newgeometry{top=-10pt}
\tcbset{width=\paperwidth,boxrule=0pt,right=3cm,arc=0pt}
\end{lstlisting}

Note that we set the width of the \lstinline{tcolorbox} to \lstinline{\paperwidth} in order for the shading to extend to the full width of the page.

\restoregeometry

%\clearpage
\cxset{style04/.style={
 numbering=Roman,
 number font-size=\Large,
 number font-family=\rmfamily,
 number font-weight=\bfseries,
 number before=,
 number dot=,
 number after=,
 number position=rightname,
 chapter font-family=\rmfamily,
 chapter font-weight=\normalfont,
 chapter font-size=\Large,
 chapter before={\vspace*{20pt}\par\hfill},
 chapter after={\hfill\hfill\par\vspace*{10pt}},
 chapter color={black!90},
 number color=purple,
 title beforeskip={},
 title afterskip={\vspace*{50pt}\par},
 title before={\hfill},
 title after={\hfill\hfill\par},
 title font-family=\rmfamily,
 title font-color= purple,
 title font-weight=\normalfont,
 title font-size=\LARGE,
 section numbering=none,
 section align = center}}

\cxset{style04}

\chapter{INTRODUCTION TO STYLE FOUR}

This is a very simple design applicable perhaps to translations and commentary on older texts.
\medskip
\begin{figure}[ht]
\centering
\includegraphics[width=0.6\textwidth]{./chapters/chapter04.png}
\end{figure}

%
\cxset{style05/.style={
 name={Chapter},
 chapter color = magenta,
 chapter toc = true,
 numbering=arabic,
 number font-size=\Large,
 number font-family=\rmfamily,
 number font-weight=\normalfont\itshape,
 number color= purple,
 number before=\hspace*{-15pt},
 number dot=,
 number after=,
 number position=rightname,
 chapter font-family=sffamily,
 chapter font-weight= \bfseries\itshape,
 chapter font-size=\Large,
 chapter before={\hrule width \columnwidth \kern12.6pt \par\hfill},
 chapter after={\hfill\hfill\par},
 chapter color={magenta},
 chapter spaceout=none,
 title beforeskip={\vspace*{10pt}},
 title afterskip={\vspace*{30pt}\par},
 title before={\hfill},
 title after={\hfill\hfill \vskip12.6pt\hrule width \columnwidth \kern2.6pt },
 title font-family=\rmfamily,
 title font-color=black!90,
 title font-weight=\bfseries,
 title font-size=\huge,
 title font-shape = normal,
 header style= headings}}

\cxset{style05}
\chapter{Introduction to Style Five}\index{ch:style5}

\tcbset{width=\textwidth}
I think this style can be improved with a bit of color. You can experiment with it quite easily. The spacing on top of this style can also be adjusted to suit your typographical taste.
\medskip
\begin{figure}[ht]
\centering
\includegraphics[width=0.6\textwidth]{./chapters/chapter05}
\end{figure}

%\section{General notes on rules}

LaTeX's default rules would normally give problems. Best is to use TeX's primitives to built them.

\index{rules!example color}

\begin{texexample}{}{}
\makeatletter
\hrule width 5cm \kern2.6\p@
AAAAAAAAAAAAAAAAAAAAA
\vskip2.6pt\hrule width 5cm
\medskip

Problem with LaTeX rules.

\rule{5cm}{0.4pt}\par
AAAAAAAAAAAAAAAAAAAAA\par%
\rule[6.5pt]{5cm}{0.4pt}

\def\rule{\@ifnextchar[\@rule{\@rule[\z@]}}
\def\@rule[#1]#2#3{%
 \leavevmode
 \hbox{%
 \setlength\@tempdima{#1}%
 \setlength\@tempdimb{#2}%
 \setlength\@tempdimc{#3}%
 \advance\@tempdimc\@tempdima%
 \vrule\@width\@tempdimb\@height\@tempdimc\@depth-\@tempdima}}

\def\thickrule{\leavevmode \leaders \hrule height 3pt \hfill \kern \z@}

{\color{teal}\hrule width 10.5cm height3pt \kern2.6\p@
    {{\color{black!80}\HUGE CHAPTER TITLE}}\vskip3pt
\hrule width 10.5cm height3pt}
\makeatother
\end{texexample}

%\cxset{chapter format=block}
%\makeatletter
\cxset{defaults/.style ={% 
    chapter title margin-top-width    =  0cm,
    chapter title margin-right-width  =  1cm,
    chapter title margin-bottom-width = 10pt,
    chapter title margin-left-width   = 0pt,
    chapter align                     = left,
    chapter title align               = left, %checked
    chapter name                      = CHAPTER,
    chapter format                    = block,
    chapter font-size                 = Huge,
    chapter font-weight               = bold,
    chapter font-family               = sffamily,
    chapter font-shape                = upshape,
    chapter background-color          = white,
  % chapter label    
    chapter color               = black,
    chapter number prefix             = ,
    chapter number suffix             = ,
    chapter numbering                 = arabic,
    chapter indent                    = 0pt,
    chapter beforeskip                = -3cm,
    chapter afterskip                 = 30pt,
    chapter afterindent               = off,
    chapter number after              = ,
    chapter arc                       = 0mm,
    chapter label background-color    = white,
    chapter label color               = black,
   % chapter afterindent               = on,
    chapter grow left                 = 0mm,
    chapter grow right                = 0mm,
    chapter rounded corners           = northeast,
    chapter shadow                    = fuzzy halo,
    chapter border-left-width         = 0pt,
    chapter border-right-width        = 0pt,
    chapter border-top-width          = 0pt,
    chapter border-bottom-width       = 0pt,
    chapter padding-left-width        = 0pt,
    chapter padding-right-width       = 10pt,
    chapter padding-top-width         = 10pt,
    chapter padding-bottom-width      = 10pt,
    %  
    chapter number color              = black,
    chapter number background-color   = white,
    chapter number font-size        = huge,
    chapter number font-weight      = bfseries,
    chapter number font-family      = sffamily,
    chapter number font-shape       = upshape,
    chapter number align            = Centering,
    %
    chapter title font-size        = Huge,
     chapter title font-weight      = bold,
     chapter title font-family      = sffamily,
     chapter title font-shape       = upshape,
     chapter title color            = black,
     chapter title background-color = white,
     }%
   }  
\makeatother     
%\makeatletter
%\cxset{toc image=\@empty,
%       chapter toc=true,
%       title beforeskip=1pt}
%
%\@specialfalse
%
%
%\renewcommand\stewart[2][]{%
%\fancypagestyle{fancy}{%
%\lhead{}\rhead{}
%\chead{}
%\cfoot{}
%\lfoot{}
%\rfoot{\thepage}
%\def\footrule#1{{\color{blue}%
%  \hrule width\paperwidth}\vskip3pt
%}
%
%\renewcommand{\headrulewidth}{0pt}
%\renewcommand{\footrulewidth}{0.4pt}}
%
%\clearpage
%
%\begin{tikzpicture}[remember picture,overlay]
%% Main shading block
%\node [xshift=5cm,yshift=-\paperheight] at (current page.north west)
%[text width=0.98\textwidth,text height=\paperheight, fill=thecream!30,rounded corners,above right]
%{};
%\node [xshift=6.5cm,yshift=-1.5cm-\soffsety] at (current page.north west)
%[text width=0.9\textwidth,below right]{\sffamily \bfseries \huge #2};
%
%\node [xshift=3cm,yshift=-1.5cm] at (current page.north west)
%[text width=3cm,align=center,minimum height=2.5cm, fill=blue,below right]
%{\[\text{\HHUGE\bfseries\sffamily\color{white}\thechapter}\]
%\par\vspace*{3pt}
%};
%
%\node [xshift=-0.2cm,yshift=-21.5cm] at (current page.north west)
%[text width=3cm,above right]%
%{\includegraphics[width=1.0\paperwidth]{\image@cx}};
%% second box left
%\node [xshift=3cm,yshift=-19.5cm] at (current page.north west)
%[text width=9cm,minimum height=2.5cm,inner sep=0.5em, fill=blue,below right]
%{\color{white}
%  \bfseries\sffamily \texti@cx
%};
%% Last block
%\node [xshift=6.5cm,yshift=-26cm] at (current page.north west)
%[text width=12cm,above right]
%{\textii@cx
%};
%\end{tikzpicture}
%\par
%\clearpage
%}





\cxset{steward,
  chapter numbering=arabic,
  chapter format = stewart,
  offsety=0cm,
  image= {./images/hine02.jpg},
  texti={When Lamport designed the original \LaTeX\ sectioning commands he did not provide a fully comprehensive interface for modifying their design. With current tools available improvements are much easier to program and this chapter provides the details.},
  textii={\precis{In this chapter we discuss a method that allows the production of fancy chapter headings and formatting, based on a set of key values. Central  to this process is the separation of content from presentation.
We also discuss the basic formatting tools that are available and how one can modify them to mould new book designs.}
 }
}


\chapter{Designing Chapter Headings}
\addtocimage{-12pt}{-20pt}{./images/tocblock-man-01.jpg}

\section*{Introduction}

A \textls*{crowded} first page is as unsightly as a crowded title page, wrote De Vinne in \emph{Modern Methods of Book Composition} in 1904.  Not much has changed since. A new chapter must make a good impression and must give an immediate signal that a different topic is going to be discussed. Traditionally chapter openings in LaTeX are an unimpressive and dry event. Our aim is to brighten it up a bit, while keeping true separation of content from presentation, but avoiding the pit traps of over ornamenting the design. A book is to be read and we should provide minimal ornamentation. \index[phdkeys]{chapter> ornamentation}

% \usepackage{array,tabularx}
%\newcolumntype{Y}{>{\raggedleft\arraybackslash}X}% see tabularx
%\tcbset{enhanced,fonttitle=\bfseries\large,fontupper=\normalsize\sffamily,
%colback=yellow!10!white,colframe=red!50!black,colbacktitle=thecodebackground,
%coltitle=black,center title,
%tabularx={X||Y|Y|Y|Y||Y},% this sets ’before upper’ and ’after upper’
%before upper app={Group & One & Two & Three & Four & Sum\\\hline\hline} }
%
%\begin{tcolorbox}[title=My table]
%Red & 1000.00 & 2000.00 & 3000.00 & 4000.00 & 10000.00\\\hline
%Green & 2000.00 & 3000.00 & 4000.00 & 5000.00 & 14000.00\\\hline
%Blue & 3000.00 & 4000.00 & 5000.00 & 6000.00 & 18000.00\\\hline\hline
%Sum & 6000.00 & 9000.00 & 12000.00 & 15000.00 & 42000.00
%\end{tcolorbox}

\begin{figure}[htbp]
\centering
\parindent=0pt
\fbox{\includegraphics[width=\textwidth]{metropolitan-spread}}
\par
\caption{A chapter opening from the Metropolitan Museum of Art publicaion, \textit{Assyrian Reliefs and Ivories} by Vaughn. E. Crawford et. al., 1980. The spread is simple and the chapters are not numbered. This is a common characteristic of many more recently published books.}
\end{figure}


What is to us now a common occurence with instant book-printing was not always so. The cost of illustrated books was a prime factor and as Tschichold wrote:
\begin{quotation}
In the area of book design, in the last few years a revolution has taken place, until recently recognized by only a few. but which now begins to influence a much wider range of action.
It means placing much greater emphasis on the appearance of the book and a wholly contemporary use of typographic and photographic means. Before the invention of printing, literature of that time was spread around by the mouth of the author himself or by professional bards. The books of the Middle Ages - like the "Mannessische Liederhandschrift" - had
\end{quotation}

The type of book you are writing and its contents will determine an appropriate design for chapter headings and the type of design and numbering if any for subsections. Here we are merely providing a mechanism to produce them. These methods can produce a mastepiece or an ugly piece of work. Some simple suggestions follow (from my observations of styles in books I like). In general you need to think what type of book you are developing. For example a novel, should be sectioned very carefully. Many books avoid marking of sections other than chapters totally, perhaps marking them just with a soft ornament such as three centered asterisks.

\section{Numbering of Sections}


In general books do not number sections beyond subsection. You can avoid them all together, if you are not going to reference the sections extensively. 

In works of fiction, authors sometimes number their chapters eccentrically, often as a metafictional statement. For example:
Seiobo There Below by László Krasznahorkai has chapters numbered according to the Fibonacci sequence.

The Curious Incident of the Dog in the Night-Time by Mark Haddon only has chapters which are prime numbers.

At Swim-Two-Birds by Flann O'Brien has the first page titled Chapter 1, but has no further chapter divisions.

God, A Users' Guide by Seán Moncrieff is chaptered backwards (i.e., the first chapter is chapter 20 and the last is chapter 1). The novel The Running Man by Stephen King also uses a similar chapter numbering scheme.
Every novel in the series A Series of Unfortunate Events by Lemony Snicket has thirteen chapters, except the final instalment (The End), which has a fourteenth chapter formatted as its own novel.

Mammoth by John Varley has the chapters ordered chronologically from the point of view of a non-time-traveler, but, as most of the characters travel through time, this leads to the chapters defying the conventional order.


\begin{pgfpicture}
\pgfpathmoveto{\pgfpointorigin}
\pgfpathlineto{\pgfpoint{1cm}{1cm}}
\pgfpathlineto{\pgfpoint{1cm}{0cm}}
\pgfusepath{fill}
\end{pgfpicture}




\begin{figure}[tbp]
\centering
\parindent=0pt
\fbox{\includegraphics[width=\textwidth]{fantasy-architecture}}
\par
\caption{A chapter opening from the Metropolitan Museum of Art publicaion, \textit{Assyrian Reliefs and Ivories} by Vaughn. E. Crawford et. al., 1980. The spread is simple and the chapters are not numbered. This is a common characteristic of many more recent books.}
\end{figure}


\begin{figure}[tbp]
\centering
\parindent=0pt
\fbox{\includegraphics[width=\textwidth]{fantasy-architecture-02}}
\par
\caption{A chapter opening from the Metropolitan Museum of Art publicaion, \textit{Assyrian Reliefs and Ivories} by Vaughn. E. Crawford et. al., 1980. The spread is simple and the chapters are not numbered. This is a common characteristic of many more recent books.}
\end{figure}


\section*{Use of Color}

The modern books that Tschilchod was discussing have long been overwhelmed by the appearance of larger, coffee book type of books. Our brains our now conditioned by branding and graphic design is everywhere. 

Once you have decided that the book is going to be a bit more colorfull, the choice of color will follow. The decision what to color will be an important one, which brings us to color theory. The history of color is perhaps as colorfull as the rest. Attempts to formalize and recognize order date back to Aristotle (384-322 bce) but began in earnest with Leonardo da Vinci (1452-1519) and have progressed ever since. Leonardo noted that certain colors intensify each other, discovering \textit{contrary} and \textit{complementary} colors. The first color wheel was invented by Britain's Sir Isaac Newton (1642-1727), who split white light into red, orange, yellow, green, blue, indigo and violet beams, then joined the two ends of the spectrum to form a circle showing the natural progression of colors. When Newton created the color wheel, he noticed that mixing two colors from opposite positions produced a neutral or \textit{anonymous} color.


\begin{figure}[htbp]
\parindent=0pt
\centering
\fbox{\includegraphics[width=\textwidth]{line-designs} }
\caption{Spread from \textit{Beautiful Geometry}, Eli Maor and Eugen Jost, Princeton Univeristy Press, 2014. A subtle coloring of the chapter heading, de-emphasizing the chapter number and coloring the chapter title. There is no chapter label. A dropcap with the same color starts the first paragraph. This style is easy to achive with the phd system.}
\end{figure}


\begin{figure}[htbp]
\parindent=0pt
\centering
\fbox{\includegraphics[width=\textwidth]{color-book01.jpg} }
\bigskip

\fbox{\includegraphics[width=\textwidth]{color-book02.jpg} }
\end{figure}

One would expect a book written for the sole purpose of describing color theory and its application to the Graphic Arts, is expected to be colorful. Note the de-emphasizing of the label and number. 

\begin{figure}[htbp]
\parindent=0pt
\centering
\fbox{\includegraphics[width=\textwidth]{color-book-03.jpg} }
The chapter heading label and number are almost invisible. The heading text, is typeset in large bold letters, shouting what is coming next. Not your typical scintific book\ldots
\bigskip

\fbox{\includegraphics[width=\textwidth]{color-book-04.jpg} }
\end{figure}

Advertizing people understand that they need to present the message of an advertizement loud and clear so as to catch the busy eye. A heading's message is the title description. Neither the label not the chapter if any are necessary to convey the message. The chapter heading is analogous to the stop at the end of a sentence. The brain gets a signal to absorb what was written before it and get ready for the next. The heading signals the end of a topic. One must not dwell on it.


\section{Contemporary Chapter Headings}

In the book \textit{China} the designer used both a chapter heading on a spread of two images, as well as repeated the chapter number on the text pages \ref{fig:threepage}. The images distill the message of the chapter, although the chapter subtitle is almost unreadable, dominated by the surrounding text. From a technical perspective, the chapter command must paint the two images, set the right type of heading for each page and then without increasing the counter, change the counter to one that displays the chapter number in words and then continue with typesetting the text. A careful choice of images is necessary for such chapters, as well as cropping the images to match the aspect ratio of the book pages. One also needs to be carefull for \latexe not to place any floats in between the page spreads. 

\begin{figure}[htbp]
\parindent=0pt
\centering
\fbox{\includegraphics[width=\textwidth]{beijing.jpg} }\par
\vfill

\fbox{\includegraphics[width=\textwidth]{beijing-01.jpg} }\par
%\fbox{\includegraphics[width=\textwidth]{pearl-river.jpg} }
\caption{A full page chapter spread.}
\label{fig:threepage}
\end{figure}

\begin{figure}[htbp]
\parindent=0pt
\centering
\fbox{\includegraphics[width=\textwidth]{beijing.jpg} }\par
\vfill

\fbox{\includegraphics[width=\textwidth]{beijing-01.jpg} }\par
%\fbox{\includegraphics[width=\textwidth]{pearl-river.jpg} }
\caption{A full page chapter spread.}
\label{fig:threepage}
\end{figure}


\clearpage



In Figure~\ref{fig:photospread} the bands are black, but position low on the page. The size of the pages are 9.69 \texttimes 11.42. The books sections are not numbered. Text i sbroken through inserts of bigger text. Many of the examples here are from
commercial nude photography books, as they tend to break with tradition. In the 1970s and 1980s, fashion photographers began to present a
new, confrontational image of the female body. The pioneer in this
respect was the German Helmut Newton (1920–2004). Newton’s
photographs of nudes were overtly sexual, with an undertone of
menace, and although his models tended to be depicted as part
of the social elite they were often placed, apparently caught out
in reportage style, in sordid environments engaged in fantasy and
fetish. His work made him highly influential in fashion photography,
though some of it was thought too highly sexual for American
magazines and appeared only in those published in Europe.


\begin{figure}[htbp]
\parindent=0pt
\includegraphics[width=\textwidth]{baetens-01.jpg} \par
\vfill\vfill\vfill\vfill
\includegraphics[width=\textwidth]{baetens-02.jpg}\par
\caption{Chapter spread and first pages after the chapter title which is on the right page of the chapter spread. From \textit{New Photography, Art and the Craft}, Pascal Baetens, DK Publications. }
\label{fig:photospread}
\end{figure}

In the 1980s, Newton undressed the dynamic and independent
female in a series called Big Nudes. In this series the women are
indeed naked and very tall, wearing nothing but makeup and high
heels. The Big Nudes were exhibited in the form of life-size prints
that were intended to provoke the viewer by showing self-confident
women who knew what they wanted and were very aware of their
beauty and sexuality



\chapter{Package Usage}

To use the package include it just like any other package:

\begin{teXXX}
\documentclass{book}
\usepackage{phd}
\cxset{style13}
\begin{document}
\chapter{Introduction}
\end{document}
\end{teXXX}

The command \docAuxCommand{cxset} sets the default style for the example to the style defined as \meta{style13}. The package currently offers  100 templates and numerous keys to manipulate them further. Styles are similar to \enquote{themes} used in web programming; they are a collection of keys that resemble in many ways \texttt{css}. Styles can have any names and I am sure as package usage increases and evolve,they will get better names. 

\section{Background}

Before describing in detail how to specify a new layout for headings, we offer an overview of how the task can be accomplished and the design philosophy behind the approach. 

Irrespective of the technique and tools used, the creation of new layouts can always be divided into the following three tasks: constructing a document from “layout bricks”, which we can term as “blocks” or “elements”; establishing the layout semantics of each block; and finally, creating a layout engine supporting any document constructed from such blocks.

\begin{description}
\item [Canned Layouts] At one end of the spectrum, the most accessible approach consists of picking, a canned layout, such as LaTeX itself and perhaps only provide rudimentary macros to manipulate it.
\item [Constraints] Constraints offer a middle ground between canned layouts and handwritten layout engines. Constraints are arguably the most widespread and successful layout programming technique. For, instance, the foundations of \tex are laid upon constraint. CSS, the ubiquitous web template language, also relies on constraints, although in a more restricted and indirect manner.
\end{description}

\subsection{Blocks and Elements}

We define an \emph{element} as a document block, that cannot be subdivided further. For example the chapter title element, is composed of the text of the chapter title. 

A \emph{block} on the other hand is can contain other blocks and or numerous elements. We can consider the chapter headings as \emph{blocks}, composed of three blocks the chapter, number and title. Each block is then composed of elements. Each element has properties and traits. One of these mandary properties is the name. 

Blocks are either \emph{configured} (all constraints are mandatory), or flexible (there are optional/alternative constraints). By bundling optional constraints, flexible blocks make their specification customizable by non-technical users. 

\subsection{Language semantics}

One of the aims of the syntax of the templates was to offer familiar terminology and to remove the use
of \tex macros as far as possible from templates. 
\medskip

{\parindent0pt

 \textit{section}| font-family=serif,|\\
 \textit{section}| font-size=LARGE,|\\
 \textit{section}| font-weight=bold,|\\
}

The restriction I imposed is problematic when dealing with fractions of linewidths and textwidths. So
at present we allow for example |title text-width=0.5\texwidth| or |title text-width=10cm| or any other valid units. Ideas for improvements can only come from user feedback in the future.

Some experimental ideas incorporated are:

\begin{verbatim}
title text-width = 0.5 text-width,
title text-width = 1.2 text-width,
\end{verbatim}

A better parser will need to be programmed for dimensions, which are all currently handled as etex |dimexpr|. 

The syntax must allows both for microtypography as well as macro-typographical features. The former would deal with mostly fonts, spacing and text justification, where the latter deals with layouts, borders shapes and the positioning of elements on the page and also reletively to other elements or blocks.

An advantage of this approach is that it also opens the possibility of parsing the text with a language other than \tex and translating the document to another format, such as |HTML| or |XML| either fully or partially. Next we will describe both the syntax as well as the usage of the settings.

\section{Chapter opening page}

The standard \latexe classes offer only two options to either open a chapter on an odd page or at any page. This package offers five alternatives:

\begin{docKey}[phd]{chapter opening}{=\meta{any, left, right, anywhere, ifafter}}{default none, initial=any}
For documents that are primarily to be read on the web, use |any| for normal books, use \textit{right}. Some templates that we provide use |any| and the examples use |anywhere| to enable us to display the heading at any position on the page.
\end{docKey}

\begin{decription}
\item [any] Opens a chapter at any page, either \textit{verso} or \textit{recto}.
\item [left] Opens a chapter on an even page
\item [right] Opens a chapter on a right page.
\item [anywhere] Opens a chapter at the point where the \cs{chapter} is typed.
\item [none] Alias for \marg{anywhere}.
\item [ifafter] Opens a chapter at the next page if the page has material that does not exceed a certain portion of \cs{textheight}.
\end{description}

\colorlet{theoption}{bgsexy}

To change a setting you just modify the value of the key \oarg{\option{chapter opening}} to one of the values described earlier. 

\begin{dispListing}
\cxset{chapter opening = anywhere}
\end{dispListing}
 
We use this key to print the many examples typesetting chapter heads that follow (see the example~\ref{ex:anywhere}).  


\begin{texexample}{title=Inline Chapter Example}{ex:anywhere}
\cxset{examplestyle/.style = {chapter format = block,
       chapter opening = anywhere,
       chapter name = CHAPTER, 
       %label
       chapter label font-family      = sffamily,
       chapter label color            = primary,
       chapter label background-color = white,
       % number
       chapter number font-family = sffamily,
       chapter number font-size = HUGE,
       chapter number color     = primary,
       chapter label align = centering,
       chapter number background-color = white,
       %title
       chapter title font-family = rmfamily,
       chapter title align = centering,
       chapter title background-color = bgsexy!15,
       chapter title before background-color=white}}
\cxset{examplestyle}       
\lorem
\chapter{Typography Example}
\lorem
\chapter{Another Chapter Heading}
\lorem
\end{texexample}


%\cxset{toc chapter = true}
\addtocounter{chapter}{-1}

Examples for other types of chapter openings follow in the rest of the documentation.

\subsection{Blank pages before chapters}

In the standard LaTeX book class when the \texttt{openany} option is not given or in the report class when the openright is given, chapters start at odd-numbered pages. This can cause a blank page to be printed. Some book designers prefer this page to be completely empty, without any headers or footers. This cannot be done with \lstinline{\thispagestyle} as this command will have to be issued on the \textit{previous} page. However by a suitable redefinition of the
\lstinline{\clearpage} this can be done automatically.
\medskip

\begin{teXXX}
\makeatletter
\def\cleardoublepage{\clearpage\if@twoside\ifodd\c@page\else
  \hbox{}
  \vspace*{\fill}
  \begin{center}
    This page left intentionally blank.
  \end{center}
  \vspace{\fill}
  \thispagestyle{empty}
  \newpage
  \if@twocolumn\hbox{}\newpage\fi\fi\fi}
\makeatother
\end{teXXX}


This is achieved easily by setting the following options:
\bigskip

\begin{tcolorbox}
\lstinline{chapter blank page=empty}\par
\lstinline{chapter blank page text=Some text.}\par
\lstinline{chapter blank page=plain}\par
\end{tcolorbox}
\medskip



The last one refers to a \lstinline!\thispagestyle{plain}!.
\cxset{chapter opening = right, chapter format = block}
\chapter{Test}

\cxset{defaults, chapter opening= anywhere}



\section*{Keys for chapter head formatting}

A chapter heading can be considered of being constructed of several parts, the \textit{chapter number}, the chapter name typically \textit{chapter} and the \textit{title}. Predefined keys handle all the elements of formatting. Additional keys are defined to handle other elements such as inclusion of images or producing complicated examples with graphics constructed with \texttt{TikZ} and other similar packages.


\bigskip\bigskip\bigskip\bigskip
\let\oldrefkey\refKey
\let\refKey\texttt
\makeatletter
\long\def\demobox#1#2{%
\par\bigskip\bigskip\bigskip
\begin{tcolorbox}[enhanced,left=0pt, top=0pt, bottom=0pt,width=\textwidth,
  enlarge top initially by=1cm,enlarge bottom finally by=1cm,left skip=1cm,right skip=1cm,
  colframe=white,colback=white,
  colbacktitle=red!30!white,colupper=black!7!white,
  code={\appto\kvtcb@shadow{%
    \path[fill=white,draw=yellow!50!black,dashed,line width=0.4pt]
      ([xshift=-1cm,yshift=-1cm]frame.south west) rectangle
      ([xshift=1cm,yshift=1cm]frame.north east);
     \path[fill=blue!20!white, 
              opacity=0.3, draw=yellow!50!black,solid,line width=1pt]
      ([xshift=-2cm,yshift=-2cm]frame.south west) rectangle
      ([xshift=2cm,yshift=2cm]frame.north east);  
    }},
  finish={
  \draw[thick,<->] ([yshift=-1.3cm]frame.north west)-- node[below]{\texttt{#1 width}}
    ([yshift=-1.3cm]frame.north east);
  \draw[thick,<->] ([xshift=-15mm]frame.north east)-- node[above]{\refKey{#1 height}}
    ([xshift=-15mm]frame.south east);
  \draw[thick,<->] (frame.north)-- node[right]{\refKey{#1 padding-top}} +(0,1);
  \draw[thick,<->] ([yshift=1cm]frame.north)-- node[right]{\refKey{#1 margin-top}} +(0,1);
  \draw[thick,<->] (frame.south)-- node[right, align=left]{\refKey{#1 padding-bottom}}+(0,-1);
  %left padding
  \draw[thick,<->] (frame.west)-- node[below right,align=center]{\refKey{#1 padding-left }}+(-1,0);
  %left margin
  \draw[thick,<->] ([xshift=-1cm,yshift=-0.9cm]frame.west)-- node[below right,xshift=-1,align=left]{\refKey{#1 margin-left }\\\refKey{#1 grow to left by}}+(-1,0);
  %right padding
  \draw[thick,<->] (frame.east)-- node[below left,align=center]{\refKey{#1 padding-right}}+(1,0);
 %right margin
  \draw[thick,<->] ([xshift=1cm,yshift=-0.9cm]frame.east)-- node[below left,xshift=1, align=right]{\refKey{#1 margin-right}\\\refKey{#1 grow to right by}}+(1,0);
 \draw[thick,<->] ([yshift=-2cm]frame.south)-- node[right, align=left]{\refKey{#1 margin-bottom},\\ \refKey{#1 after skip}}+(0,1);
  }
    ]
#2%
%\hrule width0pt height4.5cm depth0pt\relax% \vspace*{4.5cm}% \lipsum[1]
\end{tcolorbox}\par
\bigskip\bigskip\bigskip}
\makeatother

\demobox{chapter}{\scalebox{1.17}{\HHHUGE Chapter}}

The number box is again drawn in a box similar to a chapter with all properties generalized.

\demobox{number}{\scalebox{1.15}{\HHHUGE Thirteen}}



All parameters shown in the diagram can be set using the command \cs{cxset}. The property names follow conventions similar to those of |css|, rather than typical conventions of \tikzname that are more widely known to the programming community. The prefix to these properties (in the example \textit{chapter}) can be thought of
as similar to a |class| or |id| name in |css|.  

\begin{docCommand}{cxset}{\marg{options}}
  Sets options for every following \refEnv{tcolorbox} inside the current \TeX\ group.
  By default, this does not apply to nested boxes, see \Vref{subsec:everybox}.\par
  For example, the colors of the boxes may be defined for the whole document by this:
\begin{dispListing}
\cxset{chapter numbering = Roman,
       chapter number color = blue}
\end{dispListing}
\end{docCommand}

\begin{docKey}[]{chapter padding-top}{=\meta{dimension}}{no default, initial value 0pt}
All padding keys take one argument, which is a dimension. The length is also stored in a register
\cmd{\chapterpaddingtop}. In this chapter it was set at %\the\chapterpaddingtop.
\begin{dispListing}
\cxset{colback=red!5!white,colframe=red!75!black, chapter padding-top=2pt}
\end{dispListing}
\end{docKey}



\begin{docKey}[]{chapter padding-right}{=\meta{dimension}}{no default, initial value 0pt}
All padding keys take one argument, which is a dimension. The length is also stored in a register
\cmd{\chapterpaddingright}.  In this chapter it was set at %\the\chapterpaddingright.
\end{docKey}

\begin{docKey}[]{chapter padding-bottom}{=\meta{dimension}}{no default, initial value 0pt}
All padding keys take one argument, which is a dimension. The length is also stored in a register
\cmd{\chapterpaddingbottom}.  In this chapter it was set at %\the\chapterpaddingbottom.
\end{docKey}

\begin{docKey}[]{chapter padding-left}{=\meta{dimension}}{no default, initial value 0pt}
All padding keys take one argument, which is a dimension. The length is also stored in a register
\cmd{\chapterpaddingleft}.  In this chapter it was set at %\the\chapterpaddingleft.
\end{docKey}

%% margin

\begin{docKey}[]{chapter margin-top}{=\meta{dimension}}{no default, initial value 0pt}
All padding keys take one argument, which is a dimension. The length is also stored in a register
\cmd{\chaptermargintop}. In this chapter it was set at .
\end{docKey}

\begin{docKey}[]{chapter margin-right}{=\meta{dimension}}{no default, initial value 0pt}
All padding keys take one argument, which is a dimension. The length is also stored in a register
\cmd{\chapterpaddingright}.  In this chapter it was set at %\the\chapterpaddingright.
\end{docKey}

\begin{docKey}[]{chapter margin-bottom}{=\meta{dimension}}{no default, initial value 0pt}
All padding keys take one argument, which is a dimension. The length is also stored in a register
\cmd{\chapterpaddingbottom}.  In this chapter it was set at %\the\chapterpaddingbottom.
\end{docKey}

\begin{docKey}[]{chapter margin-left}{=\meta{dimension}}{no default, initial value 0pt}
All padding keys take one argument, which is a dimension. The length is also stored in a register
\cmd{\chaptermarginleft}.  In this chapter it was set at %\the\chaptermarginleft.
\end{docKey}

\subsection{Borders}

Border have three properties \emph{width, color} and \emph{style}. They can set individually for
each side of the box or using the shorter key .

\begin{docKey}[]{chapter border-top-width}{ = \meta{dimension}}{no default, initial value 0pt}
All border keys take one argument, which is a dimension.
\end{docKey}

\begin{docKey}[]{chapter border-right-width}{=\meta{dimension}}{no default, initial value 0pt}
All border keys take one argument, which is a dimension.
\end{docKey}

\begin{docKey}[]{chapter border-bottom-width}{ = \meta{dimension}}{no default, initial value 0pt}
All border keys take one argument, which is a dimension.
\end{docKey}

\begin{docKey}[]{chapter border-left-width}{ = \meta{dimension}}{no default, initial value 0pt}
All border keys take one argument, which is a dimension.
\end{docKey}

\subsubsection{Border Colors}

The colors follow the same pattern for |border-width| and again they can be set individually or using
a shorter key to set all of them in one color. 

\begin{docKey}[]{chapter border-top-color}{=\meta{color name}}{no default, initial value black}
All border keys take one argument, which is a dimension.
\end{docKey}

\begin{docKey}[]{chapter border-right-color}{=\meta{color name}}{no default, initial value black}
All border keys take one argument, which is a dimension.
\end{docKey}

\begin{docKey}[]{chapter border-bottom-color}{=\meta{color name}}{no default, initial value black}
All border keys take one argument, which is a dimension.
\end{docKey}

\begin{docKey}[]{chapter border-left-color}{=\meta{color name}}{no default, initial value black}
This key is stored in \cmd{\chapterborderrightcolor} and the value in this chapter is 
%\ExplSyntaxOn \l_phd_chapter_border_right_color_tl.
\ExplSyntaxOff
\end{docKey}



\subsubsection{Border Styles}

Standard |css|  offers four styles \emph{dotted, solid, double, dashed}. We offer almost an unlimited set of styles.

\begin{docKey}[phd]{chapter border-top-style}{=\meta{style name}}{no default, initial value \texttt{none}}
The |border-style| properties take a value, which can be |solid, double, dotted, dashed, asterisk|.
\end{docKey}

\begin{docKey}[phd]{chapter border-right-style}{=\meta{style name}}{no default, initial value \texttt{none}}
The |border-style| properties take a value, which can be |solid, double, dotted, dashed, asterisk|.
\end{docKey}

\begin{docKey}[]{chapter border-bottom-style}{=\meta{style name}}{no default, initial value \texttt{none}}
The |border-style| properties take a value, which can be |solid, double, dotted, dashed, asterisk|.
\end{docKey}

\begin{docKey}[]{chapter border-left-style}{=\meta{style name}}{no default, initial value \texttt{none}}
The |border-style| properties take a value, which can be |solid, double, dotted, dashed, asterisk|.
\end{docKey}

\begin{docKey}[phd]{chapter border-style}{=\meta{style name}}{no default, initial value \texttt{none}}
This key sets all chapter-border-\meta{top,right,bottom,left}-style to a single value.
\end{docKey}

\subsubsection{Fonts and colors}

All font parameters can be set using individual keys. The naming scheme in general follows |css| conventions.

\begin{docKey}[phd]{chapter color}{=\meta{color name}}{no default, initial value \texttt{black}}
This key sets the color for the \textit{chapter element}. The color name is stored in \cmd{\chaptercolor@cx}.
The value in this chapter is% \makeatletter\texttt{\chaptercolor@cx}\makeatother.
\end{docKey}

\begin{docKey}[phd]{chapter font-size}{=\meta{Huge, Large}}{no default, initial value \texttt{Huge}}
This sets the size for rendering the \textit{chapter element}. Use one of the following predefined values.
Note that you can either use a command i.e, |chapter font-size=|\cmd{\huge} 
or the command name i.e., |chapter font-size=huge|. The latter is the recommended method.
\end{docKey}

\begin{marglist}
\item [tiny] renders as {\tiny tiny}.
\item[footnotesize] renders as {\footnotesize footnotesize}
\item [small] Opens a chapter on an even page
\item [large] Opens a chapter on a right page.
\item [LARGE] Opens a chapter at the point where the \cs{chapter} is typed.
\item [huge] Alias for \marg{anywhere}.
\item [Huge] Opens a chapter at the next page if the page has material that does not exceed a certain portion of
 \cs{textheight}.
 \item[HUGE] renders as {\HUGE HUGE}.
 \item[HHUGE] renders as {\HHUGE HUGE}.
\end{marglist}

\begin{texexample}{Sizing settings}{}
\cxset{
          chapter format = block,
          chapter label font-size= HUGE,
          chapter name = Chapter,
          chapter format=block,
          chapter number font-size= HUGE,
          chapter title font-size=LARGE,
         % 
         % chapter padding-top=0pt,
         % chapter padding-bottom=0pt,
         % title margin-top=3pt,
         %
          }
\chapter{Setting font-sizes}          
\lorem

\end{texexample}


\begin{docKey}{chapter font-family}{ = \meta{sffamily, rmfamily etc.}}{no default, initial value \texttt{sffamily}}
The |font-family| key accepts \latexe conventional family names or |css| names such as |serif| and |non-serif|. The
value is stored in \docAuxCommand{chapter_font_family}, in this chapter it is set as {\ExplSyntaxOn\meaning\chapter_font_family\ExplSyntaxOff}
\end{docKey}


\begin{marglist}
\item [sffamily] The \emph{chapter element} is rendered in the document default \cmd{\sffamily}.
\item [rmfamily] The \emph{chapter element} is rendered in the document default \cmd{\rmfamily}.
\end{marglist}

%% Font weights
\begin{docKey}[]{chapter font-weight}{=\meta{mdseries,bfseries,etc.}}{no default, initial value \texttt{bfseries}}
The |font-weight| key accepts \latexe conventional family names or |css| names such as |bold| and |bfseries|. The
value is stored in \cmd{\chapterfontweight@cx}, in this chapter it is set as 
{\ExplSyntaxOn\expandafter\string\l_phd_chapter_label_fontweight_tl\ExplSyntaxOff}

\begin{texexample}{Setting chapter element font-weights}{fontweight}
\cxset{chapter label font-weight=normal}
\chapter{Font-weight is normal}
\cxset{chapter label font-weight= bfseries}
\chapter{Font-weight is bfseries}
\lorem
\end{texexample}
\end{docKey}


\begin{marglist}
\item [normal] The \emph{chapter element} is rendered in the document default \cmd{\sffamily}.
\item [bold] The \emph{chapter element} is rendered in the document default \cmd{\rmfamily}.
\item[bfseries] Renders as bold.
\item[mdseries] renders as medium series.
\item[light] This is an alias for normal.
\item[\upshape\ttfamily\string\bfseries] The command version of the setting.
\item[\upshape\ttfamily\string\mdseries] The command version of the setting.
\end{marglist}



\begin{docKey}[]{chapter font-shape}{=\meta{itshape,upshape,etc.}}{no default, initial value \texttt{upshape}}
The |font-weight| key accepts \latexe conventional family names or |css| names such as |bold| and |bfseries|. The
value is stored in |chapter_font_weight|, in this chapter it is set as %\ExplSyntaxOn \texttt{\chapter_font_shape}\ExplSyntaxOff.
\end{docKey}

In |css| the |font-shape| is named as |font-style| so we alias it as well. 

%\begin{marglist}
%\item[normal] normal font-style, defaults to |upshape|.
%\item[upshape] normal font-style, defaults to |upshape|. 
%\item[italic] italic shape, renders as {\itshape italic}. For some fonts it might not be available.
%\item[itshape] italic shape, alias of |italic|.
%\item[oblique] oblique font, in \latexe is equivalent to \cmd{\slshape} and renders as {\slshape slshape}, which might be slightly different than {\itshape italic}.
%\end{marglist}


\begin{texexample}{Setting up Fonts}{chapterfonts}
\cxset{   chapter format = block,
          chapter opening=anywhere,
          chapter label font-weight=normal,
          chapter label font-shape=upshape,
          %chapter border-width=0pt,
          %chapter border-style=none,
          %chapter padding-top=0pt,
          chapter label font-size=large,
          chapter number font-size=large,
          chapter number color=black,
          %title font-size=large,
          }
\chapter[fonts]{Test Font Weights}
\lorem
\cxset{chapter label font-shape=itshape}
\chapter{Test Italic Shape}
\lorem
\cxset{chapter label font-shape=normal}
\chapter{Test normal font-shape}
\lorem
\end{texexample}



The specification of font families is somewhat problematic. In the web the |css| allows |font-family|  to hold several font names as a ``fallback” system. If the browser does not support the first font, it tries the next font.

There are two types of font family names:

\begin{description}
\item[family-name] The name of a font-family, like “times”, “courier”, “arial”, etc.
\item[generic-family] The name of a generic family, like “serif”, “sans-serif”, “cursive”, “fantasy”, “monospace”.
\end{description}

Generally in the \tex community leaving the choice of font  open to what is available on a user’s computer is frowned upon. Knuth’s original aim to render consistently documents, irrespective of a user’s computer installation has served the community well, and it is possible three decades later to produce documents identical in all respects to the original. 

If this is still a valid requirement for documents is debatable. Current document processing requirements are focusing more in the preservation of content and document structure rather than form. Typeset documents in soft copy are now widely preserved in |pdf| or |postcript|  formats. One can archive the |.tex| file as well as the |pdf| file.  Back to the provision of keys, a key defined in a 
similar fashion to those of |css| could help, but there is also the issue of slow compilation. If a font cannot be
found, with the current code, it can slow down compilation tremendously. I am leaving the choice where it belongs to the user and the package writer. It makes no harm if a more flexible definition is provided. The user can then decide to only provide one or many fonts. 

This avoids complicated and almost unintelligible commands such as:

\begin{dispListing}
\setkomafont{subsection}{\usefont{T1}{fvm}{m}{n}}
\setkomafont{section}{\usefont{T1}{fvs}{b}{n}\Large}
\end{dispListing}

Here are some examples. 

\begin{texexample}{Serif and non-serif}{ex:fontfamily}
\cxset{chapter label font-family=serif, 
       chapter opening=anywhere}
\chapter{Serif font}
\lorem
\end{texexample}


\section{Floating and Alignment} 

This particular key bothered me, as the term \emph{float} has a different meaning in \latexe. However, to
be consistent with |css| terminology I have yielded to the temptation and included it.

\begin{docKey}[]{chapter float}{=\meta{left,center,right,none}}{no default, initial value \texttt{none}}
Key that controls the horizontal alignment of the \emph{chapter element}. I order for the
element to float, its |display| property must be set to |inline|.
\end{docKey}

%\begin{texexample}{Floating}{chapter:float}
%\cxset{chapter opening=anywhere, chapter float=center}
%\chapter{Centered Chapter}
%\lorem
%\cxset{chapter float=left}
%\chapter{Left Aligned}
%\lorem
%\cxset{chapter float=right}
%\chapter{Right Aligned}
%\lorem
%\end{texexample}


\subsection{The display property}

Both the |css| box model as well as the \TeX layout engine provide numerous complex algorithms in managing the floating of elements. This is normally controlled using two properties |display| and |float|.


\makeatletter

\begin{docKey}[phd]{chapter position}{ = \meta{absolute, relative}}{no default, initial value black}
This positioning directive instructs the engine to position the element at an exact position.
\end{docKey}



\tcbox[nobeforeafter]{$box_1$}\tcbox[nobeforeafter]{$box_2$}\tcbox[nobeforeafter]{$box_3$}\dotfill\tcbox[nobeforeafter]{$box_n$}
\tcbox[before skip=0.2cm, after skip=0pt, width=1cm, enlarge left by=10cm,width=5cm,enhanced,show bounding box]{title before element}
\tcbox[before skip=0pt, width=1cm, enlarge left by=10cm,width=5cm,enhanced,show bounding box]{
\tcbox{tb}\tcbox{title}\tcbox[nobeforeafter, width=1cm,]{tb}}
\tcbox[before skip=0pt, after skip=12pt, width=1cm, enlarge left by=10cm,width=5cm,enhanced,show bounding box]{\emph{title after} element \fbox{some}}
\makeatother

\begin{docKey}[phd]{chapter float}{=\meta{left,center,right,none}}{no default, initial value \texttt{none}}
Key that controls the horizontal alignment of the \emph{chapter element}. I order for the
element to float, its |display| property must be set to |inline|.
\end{docKey}
In document preparation systems or web page development the layout is user generated, i.e., the user is expected to type the html and the |css| will then specify as to how the page will be rendered by the browser. In our case for documents we can specify how we want the headings to look. The layout manager for each element, creates other associated elements, as shown for the title here. This way most layouts can be accomplished with the declarative visual language of the \pkgname{phd} package. 

\subsubsection{In-line elements}

When an element is specified as |inline| the rendering algorithm places the boxes after each other. This is widely used in |chapter elements| to render the number inline with the chapter name.
\medskip
\bgroup

\noindent
\tcbox[nobeforeafter,width=3cm, height=1cm]{Chapter}\tcbox[nobeforeafter]{twelve}
 
When the property is set as |block| the elements are stacked below each other.
\medskip

\tcbox{chapter  display=block   CHAPTER}
\tcbox{number display=block    TWELVE}

The elements can be considered to be enclosed in a \emph{ghost} element. If the property is set to float we
\begin{figure}[htbp]
\makeatletter
\parindent0pt\fboxsep0pt
\fbox{\vbox to 0pt{\hbox to \dimexpr(\textwidth)\relax{{\hss\tcbox[capture=minipage,width=5cm, height=2cm, top=0pt]{\raggedright number display=block\\ number float=right }}%
}%
}%
}\par
\vspace*{2cm}
\makeatother
\end{figure}
signalling to the layout engine that the element must be placed to the right of the page, as shown in the figure. 


\begin{figure}[htbp]
\makeatletter
\parindent0pt\fboxsep0pt
\fbox{\vbox to 0pt{\hbox to \dimexpr(\textwidth+2cm)\relax{{\hss\tcbox[capture=minipage,width=5cm, height=2cm, top=0pt]{\raggedright number display=block\\ \emph{element} float=right }
\tcbox[capture=minipage,width=5cm, height=2cm, top=0pt]{\raggedright \emph{element} display=block\\ \emph{element} float=right }
}%
}%
}%
}\par
\vspace*{2cm}
\makeatother
\end{figure}

\subsection{Absolute positioning}

Absolute positioning mode, will place an element at an exact position on the page. They are more difficult to
achieve and inflexible. 

\begin{docKey}{position}{=\meta{absolute},\meta{relative}}{no default, initial none}{}

\end{docKey}



This positioning directive instructs the engine to position the element at an exact position.


\begin{docKey}[]{chapter float}{=\meta{left,center,right,none}}{no default, initial value \texttt{none}}
Key that controls the horizontal alignment of the \emph{chapter element}. In order for the
element to float, its |display| property must be set to |inline|.
\end{docKey}
\egroup



\section{Number Element Keys}


\subsection*{Keys for numbering}

Chapter numbering follows that of the standard \LaTeX\ classes and is extended to cover some additional cases such as fully spelled out numbers. This of course is only good for languages that use the arabic numeralsn. For other languages numerals in different formats can be added with simple keys and without the need of \pkgname{polyglossia} or \pkgname{babel}. 

Note that the package uses Heiko Oberdiek's package \pkgname{alphalph} to allow for alphabetic numbering that extends beyond the normal 26 letters of the alphabet. Examples for numbering can be seen in \ref{ex:romannumbering}


\begin{docKey}[phd]{number numbering}{= \oarg{alph,Alph,roman,Roman,none,WORDS,words,none}}{default arabic}
Style of numbering.
\end{docKey}

\begin{marglist}
\item [arabic] Despite that the Arabs call what the West calls Arabic numbers Indian numbers, we provide the value arabic to have normal numbers printed.
\item [alph] Lowercase alphabetic numbering.
\item [Alph] Uppercase alphabetic numbering.
\item [roman] Lowercase roman numbering.
\item [Roman] Uppercase roman numbering.
\item [words] The number is in lowercase words.
\item [WORDS] The number is in uppercase literal numerals.
\item [Words] Prints the number in words and capitalizes the first letter, for example the number 21 will be printed as `Twenty One'\footnote{Currently limited to the first hundred numbers}.
\index{chapter design>numbering>words}
\item [ordinals] Prints the number as ordinal.
\item [Ordinals] Prints the number as Ordinal.
\item [ORDINALS] Prinst the number as ORDINALS.
\item [none] This is equivalent to using the star version of the command. It does not print any number and does not increment the chapter counter.\footnote{I am ambivalent about this, perhaps it will be better to increment it, as it can give a more general approach.}

\end{marglist}
\begin{texexample}{Literal Numbering}{ex:literal}
\cxset{chapter numbering=WORDS} 
\chapter{Literal numbering}
\lorem
\cxset{chapter numbering=words,chapter name=chapter}
\chapter{Literal numbering} 
\lorem
\end{texexample}




\cxset{chapter opening=anywhere, chapter numbering=Roman, chapter number font-shape=upshape}
\index{chapter design>numbering>roman}

\begin{texexample}{Setting up keys for numbering}{ex:romannumberingx}
\bgroup
\cxset{chapter format = traditional, 
       chapter name = CHAPTER, 
       chapter numbering = Roman,
       chapter label color = bgsexy}
\chapter{Roman numbering}
\lorem
\egroup
\end{texexample}





To emulate some old books we also offer an ordinal numbering scheme.

\begin{texexample}{Literal Numbering}{ex:ordinals}
\cxset{chapter numbering=ORDINALS} 
\chapter{Ordinals numbering}
\lorem
\cxset{chapter numbering=words,chapter name=chapter}
\chapter{Literal numbering} 
\lorem
\end{texexample}

\cxset{chapter numbering=arabic}

\subsection{Fonts and colors}
\begin{docKey}[phd]{number color}{=\meta{color name}}{no default, initial value \texttt{black}}
This key sets the color for the \textit{number element}. The color name is stored in %\cmd{\numbercolor@cx}.
The value in this chapter is %\makeatletter\texttt{\numbercolor@cx}\makeatother.
\end{docKey}

\begin{docKey}[phd]{number font-size}{=\meta{Huge, Large}}{no default, initial value \texttt{Huge}}
This sets the size for rendering the \textit{number element}. Use one of the predefined values, as described
in the section for the \emph{chapter} element.
Note that you can either use a command i.e, |number font-size=|\cmd{\huge} 
or the command name i.e., |number font-size=huge|. The latter is the recommended method.
\end{docKey}

Letter spacing can be achieved using the soul package in a combination with the key |spaceout|.
The following examples illustrate the usage.

\index[phdkeys]{{\ttfamily phd/chapter design test}}

%\begin{texexample}{Letter Spacing}{ex:letterspacing}
%\cxset{numbering=Roman,
%        % number letter-spacing=soul,
%        % chapter spaceout=soul,
%         %title spaceout=soul,
%         title font-size=Large,
%         title font-family=rmfamily,
%         title font-shape=scshape}
%\chapter{Letter Spacing}
%
%\lorem
%\end{texexample}

\begin{docKey}[phd]{chapter number letter-spacing}{=\meta{none, true, etc.}}{no default, initial value \texttt{none}}.
\end{docKey}

\begin{marglist}
\item[none] Default value no tracking is used and the letters are spaced as per the basic font information.
\item[inherit] Inherits the letter-spacing settings from the \emph{chapter} element.
\item[true] Letter spacing is employed, using the |soul| package.
\item[false] Alias for |none|.
\item[soul] The \pkgname{soul} package is used for letter-spacing.
\item[microtype] The \pkgname{microtype} package is used for letter-spacing. When the microtype package is used more fine tuning of parameters is available.
\end{marglist}

The example that follows, explains how the features offered by the \pkgname{microtype} package can be used to
set different tracking options.

\begin{texexample}{Microtypography}{micro}
\bgroup

\SetTracking
 [ no ligatures = {f},
 spacing = {600*,-100*, },
 outer spacing = {450,250,150},
 outer kerning = {*,*} ]
 { encoding = * }
 { 100 }

{\huge \textls{Chapter Twenty}}

\SetTracking
 [ no ligatures = {f},
 spacing = {600*,-100*, },
 outer spacing = {450,250,150},
 outer kerning = {*,*} ]
 { encoding = * }
 { 200 }
 
{\huge \textls{Chapter Twenty}}

\egroup
\end{texexample}


\hbox{\drawfontbox{\huge \upshape\textls(Chapter Twenty}}

\hbox{\drawfontbox{\huge \upshape\textls{Chapter Twenty}}}


\section{Styling the chapter title}

Similarly to the number and chapter styling keys exist for styling the chapter title. We summarize the available standard keys below:

\index{chapter design!labels!letter spacing}
\begin{texexample}{Styling the Title}{ex:title} 
\cxset{chapter numbering=arabic, chapter title font-shape=itshape}
\chapter{Chapter title}
\lorem
\end{texexample}


\begin{docKey}[phd]{chapter title font-family}{=\marg{family}}{no default, initial inherit document font}
Selects a predefined font family
\end{docKey}

\begin{texexample}{Title element font styling}{}
\cxset{chapter title font-family=sffamily}
\chapter{Title font family settings}
\lorem
\cxset{chapter title font-shape=itshape}
\chapter{Title font-style settings}
\lorem
\end{texexample}


\begin{docKey}[phd]{chapter title font-weight}{ = \marg{\cs{bfseries},\cs{normalseries}}} {}
Font weight.
\end{docKey}

\begin{docKey}[phd]{chapter title font-size}{= \marg{large, Large, huge, Huge, HUGE, HHuge}}{}
Font sizing commands or their names. Both \docAuxCommand{\HUGE} and HUGE are allowed to be used as values for the key.
\end{docKey}

\begin{docKey}[phd]{chapter title color} { = \marg{color}} {}
The color of the chapter title letters. This takes any predefined color name. 
\end{docKey}


\begin{docKey}[phd]{chapter title spaceout}{ = \marg{soul,none}} {no default, initial = none}
 This key will space out the title. 
\end{docKey}

\begin{texexample}{Title element spacing}{}
\cxset{chapter name=none,
       chapter numbering=none,
       chapter title font-size=Large,
       chapter title color=black,
       chapter title width=0.6\textwidth,
       %title spaceout=soul,
         }
\chapter{The Prehistoric Period in South-East Asia: 2300 BC--AD 400}        
\lorem 
    
\end{texexample}
\cxset{defaults}


\subsection*{Adding content before and after the title element}

Like all the other elements, the title element can be decorated with additional content,
before and after the text. There are two different forms. 

\begin{docKey}[phd]{title before}{=\marg{code}}{default none}
Contents before the title (vertical material)
\end{docKey}

\begin{docKey}[phd]{title after}{=\marg{code}}{default none}
Contents after the title (vertical material)
\end{docKey}

\begin{docKey}[phd]{title content before}{=\marg{code}}{default none}
Contents before the title (horizontal material)
\end{docKey}

\begin{docKey}[phd]{title content after}{=\marg{code}}{default none}
Contents after the title (horizontal material)
\end{docKey}

The difference between the two type of settings, consider the following situation. Assume you have a title that has a rule at the top and bottom and the text is surrounded by two ornaments. The surrounding ornaments will be inserted using the |title before content|, and the rules using the |title before| form. The |title before| is a full fledged element on its own. 

%{
%\hrule
%\centering
%*** Introduction ***
%\par
%\hrule
%}
%
%{
%\MakePercentComment
%\startlineat{200}
%\lstinputlisting{./styles/style13.tex}
%\MakePercentIgnore
%}



 
\begin{docKey}{/phd/ chapter title before skip}{= \marg{soul,none}}{}
Before title string skip.
\end{docKey}

\begin{docKey}{/phd/ chapter title after skip}{ = \marg{soul,none} }{}
After title string skip.
\end{docKey}

\lorem 
%
%\begin{texexample}{letter spacing the chapter title block}{ex:title3}
%
%\cxset{chapter spaceout=none,
%         numbering=arabic}
%         
%\chapter{Chapter Title Styling}
%\end{texexample}
%
%\end{document}



\cxset{chapter opening=right}
\section{Table of Contents}\index{table of contents!key settings}

Traditionally a chapter will be added to the Table of Contents if the \cs{chapter} command is issued. The starred version will not produce a number and will not add a contents line. Since we have adopted an approach where we use a key value interface we can dispense with the starred version of the command, by setting the \option{chapter toc} option to false. For example if we want to define a command for a ``Foreward'' or ``Epiloque'' without wishing them to be added to the table of contents we can use the following setting.\index{Foreward>definitions}\index{Epilogue>definitions}



\begin{texexample}{changing the chapter label name}{}
\cxset{chapter name=Chapteris, chapter numbering=arabic,}
\chapter{Foreward}
\lorem
\end{texexample}

Note that the key \option{numbering=none} still has to be set.


Please note that when \textbf{numbering=none} the chapter number is not available anymore and yo may have to reset it if required again. Although this might be seen as rather cumbersome than simply using \cs{chapter*} the advantage is consistency in the user interface and the use of appropriate semantic definitions for all sectioning commands thus achieving a bit more separation of context from style.


%\cxset{chapter toc=true}

\section{Defining styles}

Named styles can be defined using the standard \textsc{PGF} conventions. To define a style for the forward above we can use:

\begin{texexample}{}{}
\cxset{foreward/.style={chapter numbering=none,
          chapter name=none,
          chapter title font-size= Large,
          chapter title font-family= sffamily,
          chapter numbering=none}}
\cxset{foreward}
\chapter{Foreward.}
\lorem
\end{texexample}



\cxset{chapter numbering=arabic}
\section{Creating semantic names for commands and environments}

To keep our search for semantic commands and true separation of contents it is prudent to define some macros for typesetting the  `foreward' section.

\bgroup
\begin{texexample}{defining a \textit{Foreward} macro.}{}
\begin{lstlisting}
\cxset{foreward/.style={chapter toc=false,
          name=none,
          title font-size = Large,
          title font-family = sffamily,
          numbering=none}}
\newcommand\forewardname{foreward}
\expandafter\newenvironment\expandafter{\forewardname}{%
\cxset{foreward}\chapter{Foreward}}%
{}
\begin{foreward}
\lorem
\end{foreward}
\end{lstlisting}
\end{texexample}
\egroup

Notice the use of a new command \cmd{\forewardname} to allow for internationlization using Babel or other methods. One is tempted to let the English name, but a better approach perhaps is to define both.

\makeatletter



%
\@specialtrue
\cxset{steward,
  numbering=arabic,
  custom=stewart,
  offsety=0cm,
  image=hine03,
  texti={When Lamport designed the original \LaTeX\ sectioning commands, limitations of computer power forced him to restrict the abstraction of complicated chapter layouts. With current tools available improvements are much easier to program.},
%
  textii={In this chapter we discuss a method that allows the production of fancy sectionr headings and formatting, based on a set of key values. Central  to this process is the separation of content from presentation.
We also discuss the basic formatting tools that are available and how one can modify them to mould new book designs.
 }
}



\raggedbottom

\chapter{Lower Level Headings}
\@specialfalse

\section{Introduction}

Good book design dictates that sectioning styles follow that of the general book design and theme. An academic publication for example might have chapters and section numbered in arabic numerals, whereas a high school textbook might have sections marked in colored boxes.

Similarly to the chapter key value interface, the package offers a key value interface to adjust sectioning command parameters.



\cxset{section beforeskip={10pt},
      section indent=0pt}
\cxset{section afterskip={10pt}}
\renewsection

\section{Section styling}

In a similar fashion to the chapter commands the following keys are provided.

\subsection{Fonts and numerals}

Font and numeral keys are shown below.
\medskip

  \keyval{section font-size}{\marg{cmd}}{Font size command such as \cs{large.}}
  \keyval{section font-weight}{\marg{cmd}}{Font weight command such as \cs{bfseries.}}
  \keyval{section font-family}{\marg{cmd}}{Font family command such as \cs{sffamily.}}
  \keyval{section font-shape}{\marg{cmd}}{Font shape command such as \cs{itshape}}
  \keyval{section color}{\marg{color}}{Color of section.}
  \keyval{section numbering}{\marg{arabic|roman|Roman|alph|Alph|words|WORDS}}{Section number style.}
  \begin{marglist}
  \item [arabic] Typesers the section number in arabic numerals.
  \item [roman] Typesets the section number in lowercase roman numerals.
  \item [Roman] Typesets the section number in uppercase roman numerals.
  \item [alph] Typesets the section number in lowercase alphabetic numbering.
  \item [Alph] Typesets the section number in uppercase alphabetic numerals.
  \item [words] Typesets the numbers in words (lowercase).
  \item [WORDS] Typesets the number in words (uppercase).
  \end{marglist}

\subsection{Skip and indentation commands}

The keys for indentaion and above and below skips are shown below.
\medskip

\keyval{section beforeskip}{}{}
\keyval{section afterskip}{}{}
\keyval{section indent}{\marg{dim}}{Indentation from margin as per standard LaTeX class definitions.}
\keyval{section spaceout}{}{}
\begin{marglist}
 \item[soul]
 \item[none]
\end{marglist}

\subsection{align}

\keyval{section align}{\marg{cmd}}{One of the alignment commands centering, ragged right, raggedleft}

\subsection{Hooks}

Hooks for adding material are shown in the following sketch.
\medskip

\fbox{aboveskip}

\fbox{indent} \fbox{number}\fbox{hook}\fbox{title}

\fbox{belowskip}

%\lipsum

\section{Example usage}

\cxset{
 chapter toc=false,
 name=CHAPTER,
 numbering=arabic,
 number font-size=\huge,
 number font-family=\sffamily,
 number font-weight=\bfseries,
 number before=,
 number dot=,
 number after=\hspace{1em},
 number position=rightname,
 chapter opening=anywhere,
 chapter font-family=\sffamily,
 chapter font-weight=\bfseries,
 chapter font-size=\huge,
 chapter before={\vspace*{0.1\textheight}\hfill},
 chapter after={\hfill\hfill\vskip0pt\thinrule\par},
 chapter color={black!90},
 number color=\color{black!90},
 title beforeskip={\vspace*{30pt}},
 title afterskip={\vspace*{30pt}\par},
 title before={\hfill},
 title after={\hfill\hfill},
 title font-family=\sffamily,
 title font-color=\color{black!90},
 title font-weight=\bfseries,
 title font-size=\huge,
%%%%%%%%%% Sections
 section font-size=\LARGE,
 section font-weight=\normalfont,
 section font-family=\sffamily,
 section align=\centering,
 section numbering=arabic,
 section indent=0em,
 section align=\centering,
 section beforeskip=20pt,
 section afterskip=10pt,
 section spaceout=soul,
 section font-shape=\itshape,
}
\cxset{book/.style={
 section numbering=arabic,
 section font-size=\Large,
 section font-weight=\bfseries,
 section font-family=\rmfamily,
 section font-shape=\normalfont,
 section align=\raggedright,
 %section numbering custom=\color{gray}{Section} (\thechapter-\@arabic\c@section),
 subsection font-size=\large
 section indent=0em,
 section beforeskip=-3.5ex \@plus -1ex\@minus -0.2ex,
 section afterskip=2.3ex\@plus.2ex,
 subsection beforeskip=-3.5ex \@plus -1ex\@minus -0.2ex,
 subsection afterskip= 1.5ex \@plus .2ex,
}}


\begin{example}{Adjusting section parameters}{}
\cxset{ section font-size=\LARGE,
 section font-weight=\normalfont,
 section font-family=\sffamily,
 section align=\centering,
 section numbering=(roman),
 section indent=0em,
 section align=\centering,
 section beforeskip=20pt,
 section afterskip=10pt,}
\chapter{A First Look at the Sectioning Keys}
\section{First section}
\lorem
\end{example}

One notable thing to keep in mind is that the numbering of the chapter is independent of that for the section, so if you need to have strange combinations rather define a section numbering custom.\index{section formatting!vertical space}

\cxset{section numbering=arabic}
\subsection{Adjusting vertical spaces}

Perhaps the most important issues we need to consider is the adjusting of vertical spaces; example~\ref{ex:latex}, that follows illustrates settings from the Octavo class and compare them with those of standard the \LaTeXe\ book class. The Octavo class through settings that are based on baselineskip fractions and multiples endeavours to achieve a grid layout. The class also tones down the `loudness' of some of the headings compared to those of the book class.


\cxset{octavo/.style={
 section font-size=\large,
 section font-weight=\normalfont,
 section font-family=\rmfamily,
 section font-shape=\scshape,
 section indent=0em,
 section align=\centering,
 section beforeskip=-1.666\baselineskip\@minus -2\p@,
 section afterskip=0.835\baselineskip \@minus 2\p@,
 subsection numbering=none,
 subsection font-family=\rmfamily,
 subsection font-size=\normalfont,
 subsection font-shape=\scshape,
 subsection font-weight=\normalfont,
 subsection indent=1em,
 subsection align=\raggedright,
 subsection beforeskip=-0.666\baselineskip\@minus -2\p@,
 subsection afterskip=0.333\baselineskip \@minus 2\p@
 }}




\cxset{book/.style={
 section numbering=arabic,
 section font-size=\Large,
 section font-weight=\bfseries,
 section font-family=\rmfamily,
 section font-shape=\normalfont,
 section align=\raggedright,
 %section numbering custom=\color{gray}{Section} (\thechapter-\@arabic\c@section),
 subsection font-size=\large,
 section indent=0em,
 section beforeskip=-3.5ex \@plus -1ex\@minus -0.2ex,
 section afterskip=2.3ex\@plus.2ex,
 subsection font-size=\large,
 subsection font-weight=\bfseries,
 subsection numbering=arabic,
 subsection indent=0pt,
 subsection beforeskip=-3.5ex \@plus -1ex\@minus -0.2ex,
 subsection afterskip= 1.5ex \@plus .2ex,
}}

\cxset{octavo headings/.style={%
 section numbering=none,section font-size=\large,section font-weight=\normalfont,
 section font-family=\rmfamily, section font-shape=\scshape,
 section indent=0em, section align=\centering, section beforeskip=-1.666\baselineskip\@minus -2\p@,
 section afterskip=0.835\baselineskip \@minus 2\p@, subsection numbering=none,
 subsection font-family=\rmfamily, subsection font-size=\normalfont, subsection font-shape=\scshape,
 subsection font-weight=\normalfont, subsection indent=1em, subsection align=\raggedright,
 subsection beforeskip=-0.666\baselineskip\@minus -2\p@,
 subsection afterskip=0.333\baselineskip \@minus 2\p@,
 subsubsection numbering=none,
 subsubsection font-family=\rmfamily,
 subsubsection font-size=\normalfont,
 subsubsection font-shape=\itshape,
 subsubsection font-weight=\normalfont,
 subsubsection indent=1em,
 subsubsection align=\raggedright,
 subsubsection beforeskip=-0.666\baselineskip\@minus -2\p@,
 subsubsection afterskip=0.333\baselineskip \@minus 2\p@,
 paragraph numbering=none,
 paragraph font-family=\rmfamily,
 paragraph font-size=\normalfont,
 paragraph font-shape=\normalfont,
 paragraph font-weight=\normalfont,
 paragraph indent=-1em,
 paragraph align=\raggedright,
 paragraph beforeskip=\z@,
 paragraph afterskip=0\p@,
% subparagraph numbering=none,
% subparagraph font-family=\rmfamily,
% subparagraph font-size=\normalfont,
% subparagraph font-shape=\normalfont,
% subparagraph font-weight=\normalfont,
% subparagraph indent=0em,
% subparagraph align=\raggedright,
% subparagraph beforeskip=\z@,
% subparagraph afterskip=0\p@,
}}
\cxset{octavo headings}
\renewsection\renewsubsection\renewsubsubsection\renewparagraph

\begin{example}{Octavo class headings, settings}{}
\cxset{octavo headings/.style={%
 section numbering=none,section font-size=\large,section font-weight=\normalfont,
 section font-family=\rmfamily, section font-shape=\scshape,
 section indent=0em, section align=\centering, section beforeskip=-1.666\baselineskip\@minus -2\p@,
 section afterskip=0.835\baselineskip \@minus 2\p@, subsection numbering=none,
 subsection font-family=\rmfamily, subsection font-size=\normalfont, subsection font-shape=\scshape,
 subsection font-weight=\normalfont, subsection indent=1em, subsection align=\raggedright,
 subsection beforeskip=-0.666\baselineskip\@minus -2\p@,
 subsection afterskip=0.333\baselineskip \@minus 2\p@,
 subsubsection numbering=none,
 subsubsection font-family=\rmfamily,
 subsubsection font-size=\normalfont,
 subsubsection font-shape=\itshape,
 subsubsection font-weight=\normalfont,
 subsubsection indent=1em,
 subsubsection align=\raggedright,
 subsubsection beforeskip=-0.666\baselineskip\@minus -2\p@,
 subsubsection afterskip=0.333\baselineskip \@minus 2\p@,
 paragraph numbering=none,
 paragraph font-family=\rmfamily,
 paragraph font-size=\normalfont,
 paragraph font-shape=\normalfont,
 paragraph font-weight=\normalfont,
 paragraph indent=-1em,
 paragraph align=\raggedright,
 paragraph beforeskip=\z@,
 paragraph afterskip=0\p@,}}

\cxset{octavo headings}
\renewsection\renewsubsection\renewsubsubsection\renewparagraph
\section{Octavo Class Heading}
\lorem
\subsection{Octavo subsection}
This is some text short text\par
\subsubsection{Octavo sub-subsection}
\lorem
\paragraph{paragraph heading} This is some short text.
\end{example}

\begin{example}{}{}
\cxset{octavo}
\section{Octavo Class Heading}
\lorem
\subsection{Octavo subsection}
\lorem
\subsubsection{Octavo sub-subsection}
\lorem
\paragraph{paragraph heading} This is some short text.
\lorem
\paragraph{paragraph heading} This is some short text.
\lorem
\end{example}



\begin{example}{\LaTeXe\ book class headings settings}{ex:latex}
\cxset{book/.style={
 section numbering=arabic,
 section font-size=\Large,
 section font-weight=\bfseries,
 section font-family=\rmfamily,
 section font-shape=\normalfont,
 section align=\raggedright,
 %section numbering custom=\color{gray}{Section} (\thechapter-\@arabic\c@section),
 subsection font-size=\large,
 section indent=0em,
 section beforeskip=-3.5ex \@plus -1ex\@minus -0.2ex,
 section afterskip=2.3ex\@plus.2ex,
 subsection font-size=\large,
 subsection font-shape=\normalfont,
 subsection font-weight=\bfseries,
 subsection numbering=arabic,
 subsection indent=0pt,
 subsection beforeskip=-3.5ex \@plus -1ex\@minus -0.2ex,
 subsection afterskip= 1.5ex \@plus .2ex,
}}
\cxset{book}
\renewsubsection
\section{LaTeX Book  Class Heading}
\lorem
\subsection{A subsection}
\lorem
\end{example}

\section{Grid example}

One problem sometimes is that the sectioning commands create problems with grid layouts. Example~\ref{ex:grid} shows example settings.

\begin{example}{Section styles from the grid package}{ex:grid}
\cxset{grid/.style={
 section numbering=arabic,
 section font-size=\normalsize,
 section font-weight=\bfseries\mathversion{bold},
 section font-family=\rmfamily,
 section font-shape=\normalfont\bfseries\mathversion{bold},
 section beforeskip=-.999\baselineskip,
 section afterskip=0.001\baselineskip,
 section align=\raggedright,
 %section numbering custom=\color{gray}{Section} (\thechapter-\@arabic\c@section),
 subsection font-size=\normalsize,
 section indent=0em,
% section beforeskip=-3.5ex \@plus -1ex\@minus -0.2ex,
 %section afterskip=2.3ex\@plus.2ex,
 subsection font-shape=,
 subsection font-weight=\bfseries\mathversion{bold},
 subsection numbering=arabic,
 subsection indent=0pt,
 subsection beforeskip=\baselineskip,
 subsection afterskip= -.35\baselineskip,
% subsub section
 subsubsection font-shape=\itshape,
 subsubsection font-weight=\bfseries\mathversion{bold},
 subsubsection numbering=numeric,
 subsubsection indent=0pt,
 subsubsection beforeskip=\baselineskip,
 subsubsection afterskip= -.35\baselineskip,
}}
\cxset{grid}
\renewsubsection
\begin{multicols}{2}
\section{Grid  Class Heading}
\lorem
\subsection{Grid  subsection.}
\lorem
\subsubsection{A subsection grid.}
\lorem
\subsubsection{Another subsection grid.}
\lorem
\end{multicols}
\end{example}



The key \option{\bfseries section numbering custom}=\marg{code} is quite powerfull and can be used to define any type of section number style. Just remember that the numbering so far depends on two counters, the c@chapter and c@section. What the section numbering does, it redefines the macro \cs{thesection} to the new definition provided as argument for the key.

Although the temptation to define a lot of key combinations one would rather define new styles as a more user friendly approach.

\cxset{section numbering=arabic, section align=\raggedright, section font-shape=\upshape, section font-family=\rmfamily}
\section{Handling Other Section Levels}

Other sectioning commands such as \cs{subsubsection}, \cs{paragraph} and \cs{subparagraph} have equivalent keys. Examples can be found in the chapters that follow for specific styles.

\section{Technical discussion}

The standard LaTeX classes, book report and article have sections showing dot leaders, whereas in the article class the sections are shown without the dotted lines, as the l@section macro is redefined for articles.

\index{macros!\textbackslash @seccntformat}

\subsection{Indexing of Lower Section Headings}
\LaTeXe\ offers two pathways in redefining section commands, the first one is @startsection and the second is \cs{@seccntformat} \index{sectioning macros}. It also uses the macro \cs{secdef} to create the starred and unstarred versions of the sectioning commands.

\begin{tcolorbox}{}
\begin{lstlisting}
% \begin{macro}{\l@section}
%    In the article document class the entry in the table of contents
%    for sections looks much like the chapter entries for the report
%    and book document classes.
%
%    First we make sure that if a pagebreak should occur, it occurs
%    \emph{before} this entry. Also a little whitespace is added and a
%    group begun to keep changes local.
% \changes{v1.0h}{1993/12/18}{Replaced -\cs{@secpenalty} by
%    \cs{@secpenalty}.  ASAJ.}
% \changes{v1.2i}{1994/04/28}{Don't print a toc line when the tocdepth
%    counter is less than 1.}
% \changes{v1.4a}{1998/10/12}{we should use \cs{@tocrmarg}; see PR/2881.}
%    \begin{macrocode}
%<*article>
\newcommand*\l@section[2]{%
  \ifnum \c@tocdepth >\z@
    \addpenalty\@secpenalty
    \addvspace{1.0em \@plus\p@}%
%    \end{macrocode}
%
%    The macro |\numberline| requires that the width of the box that
%    holds the part number is stored in \LaTeX's scratch register
%    |\@tempdima|. Therefore we put it there. We begin a group, and
%    change some of the paragraph parameters (see also the remark at
%    \cs{l@part} regarding \cs{rightskip}).
%    \begin{macrocode}
    \setlength\@tempdima{1.5em}%
    \begingroup
      \parindent \z@ \rightskip \@pnumwidth
      \parfillskip -\@pnumwidth
%    \end{macrocode}
%    Then we leave vertical mode and switch to a bold font.
%    \begin{macrocode}
      \leavevmode \bfseries
%    \end{macrocode}
%    Because we do not use |\numberline| here, we have do some fine
%    tuning `by hand', before we can set the entry. We discourage but
%    not disallow a pagebreak immediately after a section entry.
%    \begin{macrocode}
      \advance\leftskip\@tempdima
      \hskip -\leftskip
      #1\nobreak\hfil \nobreak\hb@xt@\@pnumwidth{\hss #2}\par
    \endgroup
  \fi}
%</article>
\end{lstlisting}
\end{tcolorbox}

As you can see the dot leaders are not present in the above definition. Although we can get rid of dot leaders in other section by redefining them, it is not as easy to add them back.

As our aim is to be able to have all the classes used a common denominator we can define a command as follows (using book as a base)

\begin{tcolorbox}{}
\begin{lstlisting}
\def\articlesection{
\newcommand*\l@section[2]{%
  \ifnum \c@tocdepth >\z@
    \addpenalty\@secpenalty
    \addvspace{1.0em \@plus\p@}%
    \setlength\@tempdima{1.5em}%
    \begingroup
      \parindent \z@ \rightskip \@pnumwidth
      \parfillskip -\@pnumwidth
      \leavevmode \bfseries
      \advance\leftskip\@tempdima
      \hskip -\leftskip
      #1\nobreak\hfil \nobreak\hb@xt@\@pnumwidth{\hss #2}\par
    \endgroup
  \fi}
}
\end{lstlisting}
\end{tcolorbox}

%\articlesection

The \cs{@starredsection} macro is one of those locomotive type of commands. It takes 7 required arguments and 2 optional ones and hidden within it are two booleans. The full set looks like this:

\cs{@startsection} \marg{name} \marg{level} \marg{indent} \marg{beforeskip} \marg{afterskip} \marg{style}[*]
  [\marg{altheading}]\marg{heading}.

\begin{marglist}
\item[name] The name of the level command.
\item [level] A number denoting the depth of the section, chapter=1, section=2, etc. A section number will be printed only if \marg{level} is equal or smaller than the value of \textit{secnumdepth}
\item[indent] The indentation of the heading from the left margin.
\item[beforeskip]  The absolute value of this argument is the skip to leave above the heading. If it is negative, then the paragraph indent of the text following the heading is suppressed.
\item [afterskip] If positive, it is the skip to leave below the heading, else it is the skip to the right of a run-in heading.
\item [style] Sets the style of the heading.
\item[\textup{[*]}] When this is missing the heading is numbered and the corresponding counter is incremented.
\item[\textup{[\textit{altheading}]}] Gives an alternative heading to use in the table of contents and in the running heads. This should be present when the * form is used.
\item[heading] The heading of the new section.
\end{marglist}

\begin{example}{Example formatting run-in section}{}
\makeatletter
\bgroup
\renewcommand\section{%
    \@startsection{section}%
    {1}%
    {0em}%
    {-0.8em}%
    {-0.5em}%
    {\large\normalfont\scshape}}
\makeatother
\section[]{test}
\lorem
\egroup
\end{example}

Note we run the example in a group so that we will not influence the formatting of this document.

As mentioned earlier there is an additional way to introduce formatting for sections and this is using the command \cs{@seccntformat}, which is responsible for typesetting the counter part of a section title. The default definition of the command typesets the \cs{the} representation of the section counter.

\begin{example}{}{}
\bgroup
\renewcommand\section{%
    \@startsection{section}%
    {1}%
    {0em}%
    {-0.8em}%
    {-0.5em}%
    {\large\normalfont\scshape}}
\renewcommand\@seccntformat[1]{\fbox
{\csname the#1\endcsname}\hspace{0.5em}}
\makeatother
\section[]{test}\label{sec:ok}
\lorem

See section \ref{sec:ok}.
\egroup
\end{example}

The definition of \cs{@seccntformat} applies to all headings
defined with the \cs{@startsection} command (which is described in the next
section). Therefore, if you wish to use different definitions of \cs{@seccntformat}
for different headings, you must put the appropriate code into every heading
definition.

\begin{tcolorbox}
\begin{lstlisting}
\def\@seccntformat##1{\csname the##1\endcsname{}}
\end{lstlisting}
\end{tcolorbox}

\section{Custom headings}

It is also possible to define section headings without resorting to any of the above. To do this.

\begin{tcolorbox}
\begin{lstlisting}
\newcommand\part{\secdef\cmda\cmdb}
\end{lstlisting}
\end{tcolorbox}

the part and chapter and sometimes appendix are defined this way, but nothing stops us from doing the same for other sections. A generic section command can be defined as follows:

\begin{example}{}{}
\bgroup
\renewcommand\section[2] [?]{% % Complex form:
\refstepcounter{section}% % step counter/ set label
\addcontentsline{toc}{appendix}% % generate toe entry
{\protect\numberline{section-\thesection}#1}%
{\raggedright\large\bfseries section %\appendixname\ % typeset the title
\thesection\par \centering#2\par}% % and number
\sectionmark{#1}% % add to running header
\@afterheading % prepare indentation handling
%\addvspace{\baselineskip}
}
\section{Test}
\lorem
\egroup
\end{example}

Many other strategies can also be implemented that are perhaps easier to grasp.

\begin{example}{}{}
\bgroup
\def\strut{\vrule height12pt depth1pt width0pt}
\renewcommand\section[2] []{% % Complex form:
\refstepcounter{section}% % step counter/ set label
\addcontentsline{toc}{section}% % generate toc entry
{\protect\numberline{\thesection} }%
{\raggedright\large\bfseries\scshape %
\parbox[b]{\dimexpr(\linewidth-0.5\columnsep)}{\colorbox{brown!80}%
{{\vbox{\strut\raise2pt\hbox{#2}}}}}}\vskip0pt% % and number
\sectionmark{#1}% % add to running header
\@afterheading % prepare indentation handling
\vspace{\dimexpr\baselineskip+6pt}%must have a parameter
}
\chapter{Fossil Insects}
\begin{multicols*}{2}\raggedcolumns
\section[Insect Fossilization]{\raggedright \thinspace Insect Fossilization}
\lipsum[1]
\end{multicols*}
\egroup
\end{example}
% To answer http://tex.stackexchange.com/questions/52998/change-title-to-small-caps-but-not-in-toc

Of course some work is needed to center the text properly in the middle of the colour box. For all practical purposes it is lining up as per the sample.

In Chapter we discussed a forward, but this may not apply if there are no chapters or we need to treat these as sections, the example \ref{ex:forwardsection} shows such a method.

\begin{example}{Defining a Foreward Section}{ex:forwardsection}

\newcommand\prematter@sp[1]{% % Complex form:
%\refstepcounter{section}% % step counter/ set label
\addcontentsline{toc}{section}% % generate toe entry
{\protect\numberline{}\textsc{#1}}%
\sectionmark{#1}% % add to running header
{\LARGE\centering\normalfont\sffamily\colorbox{brown!80}{ \textsc{#1}}\par}%
\@afterheading % prepare indentation handling
\addvspace{\baselineskip}
\@afterindentfalse
}

\newenvironment{prematter}[1]{%
   \prematter@sp{#1}}
{}
\begin{multicols}{2}
\label{theok}
\begin{prematter}{Foreward}
\lipsum[1]
\end{prematter}\ref{theok}
\end{multicols}
\end{example}

\section{underlining}

I am aware that some people have no choice but have some sections underlined as dictated by archaic regulations in some establishments for thesis submission. If nobody is forcing you to underline it is best to avoid it. We use Donald Arsenau's ulem package to achieve underlining.

\newfontfamily\aegean{Aegean.ttf}
\makeatletter
%\@debugtrue
\makeatother
%\cxset{steward,
  numbering=arabic,
  custom=stewart,
  offsety=0cm,
  image={fellah-woman.jpg},
  texti={An introduction to the use of font related commands. The chapter also gives a historical background to font selection using \tex and \latex. },
  textii={In this chapter we discuss keys that are available through the \texttt{phd} package and give a background as to how fonts are used
in \latex.
 },
 pagestyle = fancy
}

\pgfpagesuselayout{2 on 1}[a3paper,landscape,border shrink=0mm]

\chapter{Ancient and Historic Scripts}

Writing was perhaps the most important human invention. \tex authors and developers either due to need or fascination developed macros and fonts for many archaic writing systems. Many of these packages are now outdated, as the Unicode standard and the newer engines opened up a fascinating world. My own fascination with writing systems prompted me to add support for such scripts in the \pkgname{phd} package. The development to an extend was frustrating as the overloading of numerous fonts caused compilation to be very slow. Finding the right font was also problematic in many cases, as we opted to identify Open Source fonts. The \tex engine of preference is \luatex. To avoid loading too many fonts, unless they are required, we provide the keys:

\def\loadscripts{}
\cxset{scripts/.store in = \loadscripts}

\begin{key}{/phd/scripts = \meta{all, lineara, linearb, phaestos,\ldots}} The scripts key takes a list of options to enable or disable the loading of fonts and the usage of the key is explained later on. You set it with our only command |\cxset|\meta{key value list}
\end{key}

Supplementary keys, exist for each individual script enabling the setting of specific fonts to a particular script. However, if all the recommended fonts have been installed is quicker to use the |scripts| key.

\def\olmecfontstore{}

\cxset{olmec font/.store in=\olmecfontstore}

\cxset{olmec font=epiolmec}

\begin{key}{/phd/olmec font = \meta{font name}}
\end{key}

The key |script| can be used on its own. It will then load the default fonts built-in, in the |phd| package.

\cxset{script/.store in = \scripttempt}

\begin{key}{/phd/script = \meta{script name}}
\end{key}

\begin{figure}[b]
\centering
\includegraphics[width=0.6\textwidth]{./images/rongo.jpg}
\caption{Rongo rongo writing. Tablet B Aruku kurenga, verso. One of four texts which provided the Jaussen list, the first attempt at decipherment. Made of Pacific rosewood, mid-nineteenth century, Easter Island.
(Collection of the SS.CC., Rome)}
\end{figure}

The first attempt  at communication via writing was through ideographic or mnemonic symbols. Undoubtedly symbolic writing must have existed much earlier than the surviving artifacts, carved in woord or scribled on muddy walls. The earliest surviving symbolic writing are the Jiahu symbols. They were carved on tortoise shells in Jiahu, ca.~6600~BC. Jiahu was a neolithic Peligang culture site found in Henan, China. In Europe the Tărtăria tablets are three tablets, discovered in 1961 by archaeologist Nicolae Vlassa at a Neolithic site in the village of Tărtăria (about 30 km (19 mi) from Alba Iulia), in Romania.[1] The tablets, dated to around 5300 BC,[2] bear incised symbols - the Vinča symbols - and have been the subject of considerable controversy among archaeologists, some of whom claim that the symbols represent the earliest known form of writing in the world. The Indus script appeared ca. 3500 BC and the Nsibidi script of Nigeria, ca. before 500 AD. 

No type of writing system is superior or inferior to another, as the type is often dependent on the language they represent. For example, the syllabary works perfectly fine in Japanese because it can reproduce all Japanese words, but it wouldn't work with English because the English language has a lot of consonant clusters that a syllabary will have trouble to spell out. The pretense that the alphabet is more "efficient" is also flawed. Yes, the number of letters is smaller, but when you read a sentence in English, do you really spell individual letters to form a word? The answer is no. You scan the entire word as if it is a logogram.

And finally, writing system is not a marker of civilization. There are many major urban cultures in the world did not employ writing such as the Andean cultures (Moche, Chimu, Inca, etc), but that didn't prevent them from building impressive states and empires whose complexity rivals those in the Old World

Unicode encodes a number of ancient scripts, which have not been in normal use for a millennium or more, as well as historic scripts, whose usage ended in recent centuries. Although these scripts are no longer used to write living languages, documents and inscriptions using these languages exist, both for extinct languages and for precursors of modern languages. The primary user communities for these scripts are scholars, interested in studying the scripts and the languages written in them. A few, such as Coptic, also have contemporary liturgical or other special purposes. Some of the historic scripts are related to each other as well as to modern alphabets. The following are provides as of Unicode version~7.2.
\index{Ancient and Historic Scripts>Ogham}
\index{Ancient and Historic Scripts>Old Italic}
\index{Ancient and Historic Scripts>Runic}
\index{Ancient and Historic Scripts>Gothic}
\index{Ancient and Historic Scripts>Akkadian}
\index{Ancient and Historic Scripts>Old Turkic}
\index{Ancient and Historic Scripts>Hieroglyphs}
\index{Ancient and Historic Scripts>Linear B}
\index{Ancient and Historic Scripts>Linear A}
\index{Ancient and Historic Scripts>Phoenician}
\index{Ancient and Historic Scripts>Old South Arabian}
\index{Ancient and Historic Scripts>Mandaic}
\index{Ancient and Historic Scripts>Avestan}
\index{Ancient Anatolian Alphabets}
\index{Old South Arabian}
\index{Phoenician}
\index{Imperial Aramaic}
\begin{center}
\begin{tabular}{lll}
Ogham (see \S\ref{s:ogham})           
&Ancient Anatolian Alphabets (see \S\ref{s:anatolian})
&Avestan (see \S\ref{s:avestan})\\
Old Italic (see \S\ref{s:olditalic})       
&Old South Arabian (see \S\ref{s:oldsoutharabian})          
&Ugaritic (see \S\ref{s:ugaritic})\\
Runic (see \S\ref{s:runic})            
&Phoenician (see \S\ref{s:phoenician})                  
&Old Persian (see \S\ref{s:oldpersian})\\
Gothic            
&Imperial Aramaic (see \S\ref{s:imperialaramaic})            
&Sumero-Akkadian. (see \S\ref{s:sumero})\\
Old Turkic (see \S\ref{s:oldturkic})     
&Mandaic (see \S\ref{s:mandaic}) 
&Egyptian Hieroglyphs.\\
Linear B (see \S\ref{s:linearb})          
&Inscriptional Parthian (see \S\ref{s:parthian})      
&Meroitic (see \S\ref{s:meroitic})\\
Cypriot (see \S\ref{s:cypriot})
&Inscriptional Pahlavi  (see \S\ref{s:inscriptionalpahlavi})       
&Linear A (see \S\ref{s:linearb})\\
\end{tabular}
\end{center}

The following scripts are also encoded but following the Unicode
convention are described in other sections

\begin{center}
\begin{tabular}{llllll}
Coptic &Glagolithic &Phags-pa. &Kaithi &Kharoshi &Brahmi.\\
\end{tabular}
\end{center}

Some scripts such as Epi-Olmec are not described in the Unicode standard, but we provide support for them.

\section{Linear A}
\label{s:lineara}
\newfontfamily\lineara{Aegean.ttf}

\section{Aegean and Cypriote Syllabaries}

The Greeks had evidently already occupied the mainland and islands of the
Ægean, including Crete, by the middle of the third millennium
BC. Around 2000 BC, following their consolidation of power on
Crete, new wealth from trade with cosmopolitan Canaan
allowed the creation of a complex palace economy, with major
centres at Knossos, Phaistos and other Cretan sites – Europe’s
first high civilization, the Minoan. Trade with Canaan had evidently
also brought Greeks into contact with Byblos’ pictorial
syllabic writing, whose underlying principle the Minoans borrowed.
Now, Cretans could also write their Minoan Greek language
using a small corpus of syllabo-logographic signs
representing \textit{in-di-vi-du-al} syllables. The signs themselves and
their phonetic values – nearly all V (e) or CV (te) – were wholly
indigenous: what the rebus signs, all originating from the
Cretan world, depicted, one pronounced in Minoan Greek, not
in a Semitic language. (Minoan Greek appears to have been an
archaic sister tongue of the mainland’s Mycenæan Greek.\footnote{A History of Writing. })

Three separate but related forms of syllabo-logographic
writing emerged in the Ægean between c. 2000 and 1200 BC: the
Minoan Greeks’ ‘hieroglyphic’ script and Linear A, and the
later Mycenæan Greeks’ Linear B. Minoan Greeks apparently
also took their writing at an early date to Cyprus, where it experienced
two stages: Cypro-Minoan (evidently derived from
Linear A is one of two currently undeciphered writing systems used in ancient Greece. Cretan hieroglyphic is the other. Linear A was the primary script used in palace and religious writings of the Minoan civilization. It was discovered by archaeologist Arthur Evans. It is the origin of the Linear B script, which was later used by the Mycenaean civilization.

Linear A and its daughter Linear C, the ‘Cypriote Syllabic
Script’. All Ægean and Cypriote scripts are clearly syllabologographic,
as the objective identity of each rebus sign would
have been immediately recognizable to each learner and user. It
seems that determinatives were never employed in any of the
Ægean or Cypriote scripts; however, logograms additionally
depicted most spelt-out items on accounting tablets. All Ægean
and Cypriote scripts, but for these separate logograms, were
completely phonetic.
\medskip

\includegraphics[width=0.8\textwidth]{./images/cretan-hieroglyphs.png}

\medskip
Crete’s `hieroglyphic’ script is the patriarch of this robust
family, its inspiration perhaps derived from Byblos via Cyprus
around 2000 BC. As its name implies, this script used
pictorial signs to reproduce the syllabic inventory of the Minoan
Greek language, here used in rebus fashion as at Byblos. This
writing occurs on seal stones (and their clay impressions), baked
clay, and metal and stone objects, most of these discovered at
Knossos and dating from 2000– 1400 BC (the script was concurrent
with Linear A). There exist about 140 different signs in all –
that is, 70 to 80 syllabic signs and their alloglyphs (different signs
with the same sound value), as well as logograms: human figures,
parts of the body, flora, fauna, boats and geometrical shapes.
Writing direction was open: from left to right, from right to left,
with every other line reversed, even spiral. That this script also
included logograms and numerals suggests that it was initially
used for book-keeping, among other things, until its replacement
in this function with its simplification, Linear A. Thereafter, like
Anatolian hieroglyphs, the Cretan hieroglyphic script appears to
have assumed a ceremonial role in Minoan Greek society,
reserved for sacred inscriptions, dedications and royal proclamations
on round clay disks.

In the 1950s, Linear B was largely udeciphered and found to encode an early form of Greek. Although the two systems share many symbols, this did not lead to a subsequent decipherment of Linear A. Using the values associated with Linear B in Linear A mainly produces unintelligible words. If it uses the same or similar syllabic values as Linear B, then its underlying language appears unrelated to any known language. This has been dubbed the Minoan language.\footnote{\url{http://www.people.ku.edu/~jyounger/LinearA/LinAIdeograms/}}

\begin{scriptexample}[]{Linear A}
\unicodetable{lineara}{  
\number"10600,"10610,"10620,"10630,"10640,"10650,"10660,"10670,
"10680,"10690,"106A0,"106B0,"106C0,"106D0,"106E0,"106F0,"10710,"10720,"10730,"10740,"10750,"10760,"10770}
\end{scriptexample}

Many of the characters form group and specialists name them such as vases in transliterations.

\begin{scriptexample}[]{Vases}
\begin{center}
\scalebox{3}{{\lineara \char"106A6}}
\scalebox{3}{{\lineara \char"106A5}}
\scalebox{3}{{\lineara \char"106A7}}
\scalebox{3}{{\lineara \char"106A9}}
\end{center}
\end{scriptexample}

Linear A contains more than 90 signs (open vowels and consonants+vowels) in regular use and a host of
logograms, many of which are ligatured with syllabograms and/or fractions; about 80\% of these
logograms do not appear in Linear B. While many of Linear A’s signs are also found in Linear B, some
signs are unique to A (e.g., A *301 and following), while some signs found in Linear B are not yet found
in Linear A (e.g., B 12, 14-15, 18-19, 25, 32-33, 36, 42-43, 52, 62-64, 68, 71-72, 75, 83-84, 89-91).

The Unicode Linear A encoding is broadly based on the GORILA ([{\arial ɡɔɹɪˈlɑː}]) catalogue
(Godart and Olivier 1976–1985), which is the basic set of characters used in decipherment efforts.However, “ligatures” which consist of simple horizontal juxtapositions are not uniquely encoded here, as
these may be composed of their constituent parts. On the other hand, “ligatures” which consist of stacked
or touching elements have been encoded. 

\def\codex#1{\emph{Codex #1}\index{codex>#1}}
%\newfontfamily{\gothicfamily}{Noto Sans Gothic}
\newfontfamily{\gothicfamily}{code2001.ttf}
\section{Gothic}

\label{s:gothic}

\subsection{Introduction}

East Germanic Goths rose to prominence during the Great
Migrations of the fourth and fifth centuries AD 31 Their Gothic
languages are primarily known to us today through a few surviving
fragments of Bible translations. It was the Visigothic bishop
Wulfila († AD 383), according to three ecclesiastical historians
writing a century later, who created ‘Gothic letters’ in order to
translate the Bible into the Visigothic language. The fourth century
Greek alphabet was Wulfila’s only apparent source.

Though the bishop’s original Visigothic hand has not survived,
closely related derivative scripts preserved in two later Gothic
manuscripts no older than the sixth century have been preserved
(illus. 116).

‘Wulfila’s script’, as it perhaps should properly be designated,
is an alphabetic script written from left to right without word
separation. Spaces indicate sentences or passages, as does a
colon or a centred dot (as with the Iberian scripts). Nasal suspension
– that is, marking where an /m/ or /n/ should be – is
sometimes indicated by a macron (a topping stroke) above the
preceding letter. Ligatures are even rarer than macrons. There
are frequent contractions: for example, ius is often used to spell
‘Jesus’. Apart from rare profane relics – witness the sixth-century
Latin-Gothic Deed of Naples – Wulfila’s script, measured
by those few inscriptions that have survived, appears to have
conveyed exclusively ecclesiastical texts.

\begin{figure}[htb]
\includegraphics[width=.45\textwidth]{gothic}
\caption{Codex Carolinus}
\end{figure}

The Gothic script that Wulfila devised from the Greek
alphabet did not engender daughter scripts. After the sixth century
AD, it was replaced almost everywhere by related descendants
of Greek and Latin alphabets. Gothic’s last sentinel, the
ninth-century \codex{Vindobonensis} 795, was perhaps by then
only an antiquarian curiosity. The \emph{Codex Carolinus} preserves papal correspondence
with Frankish rulers, including letters exchanged by popes from Gregory III (731-741) to Hadrian I (772-795). the Codex was written in 791 on the orders of Charlemagne in order to rescue papyrus copies threatened with decay. It contains 99 letters, almost exclusively papal, and survives today in Vienna, \"Osterreichische Nationalbibliotek 449, in a copy probably made at Colone during the pontificate of Archbishop Willibert (870-889). The preface of the \codex{Carolinus} appears to refer to a second part that may have contained letters to byzantine rulers, now lost. Parallel copies of the Codex have not turned up. \citep{jasper2001papal}. 

\subsection{Unicode}

The Gothic alphabet was added to the Unicode Standard in March, 2001 with the release of version 3.1.

The Unicode block for Gothic is U+10330–U+1034F in the Supplementary Multilingual Plane. As older software that uses UCS-2 (the predecessor of UTF-16) assumes that all Unicode codepoints can be expressed as 16 bit numbers (U+FFFF or lower, the Basic Multilingual Plane), problems may be encountered using the Gothic alphabet Unicode range and others outside of the Basic Multilingual Plane.

\begin{scriptexample}[]{Gothic}
\unicodetable{gothicfamily}{"10330,"10340}
\end{scriptexample}
{\gothicfamily
𐍀	𐍁	𐍂	𐍃	𐍄	𐍅	𐍆	𐍇	𐍈	𐍉	𐍊}
%http://www.gotica.de/carolinus.html

%\begin{thebibliography}
%\bibitem[Fitzmyer(1995)]{fitzmyer}
%J.~A. Fitzmyer.
%\newblock \emph{The Aramaic inscriptions of Sefīre}, volume~19 of
%  \emph{Biblica et orientalia Sacra Scriptura antiquitatibus orientalibus
%  illustrata}.
%\newblock Pontificial Biblical Institute, Rome, 1995.
%\newblock URL
%  \url{http://web.archive.org/web/20051104215025/http://www.nelc.ucla.edu/Faculty/Schniedewind_files/NWSemitic/Aramaic_ABD.pdf}.
%\end{thebibliography}  











\newfontfamily\linearb{Aegean.ttf}
\section{Linear B}
\label{s:linearb}
\index{scripts>Linear B}
The Linear B script is a syllabic writing system that was used on the island of Crete and
parts of the nearby mainland to write the oldest recorded variety of the Greek language.

Linear B clay tablets predate Homeric Greek by some 700 years; the latest tablets date from
the mid- to late thirteenth century \bce. Major archaeological sites include Knossos, first
uncovered about 1900 by Sir Arthur Evans, and a major site near Pylos. The majority of
currently known inscriptions are inventories of commodities and accounting records.

The first tablets bearing the scripts were discovered by Sir Arthur Evans (1851-1941) while he was excavating the Minoan palace at Knossos in Crete. 


\medskip

\begin{figure}[ht]
\centering
\begin{minipage}{5cm}
\includegraphics[width=5cm]{./images/iklaina.jpg}
\end{minipage}\hspace{2em}
\begin{minipage}{7cm}
\captionof{figure}{Recently discovered fragment with Linear B, inscription. Found in an olive grove in what's now the village of Iklaina, the tablet was created by a Greek-speaking Mycenaean scribe between 1450 and 1350 B.C. (See \protect\href{http://news.nationalgeographic.com/news/2011/03/110330-oldest-writing-europe-tablet-greece-science-mycenae-greek/}{National Geographic}).}
\end{minipage}

\end{figure}


Early attempts to decipher the script failed until Michael Ventris, an architect and amateur
decipherer, came to the realization that the language might be Greek and not, as previously
thought, a completely unknown language. Ventris worked together with John Chadwick,
and decipherment proceeded quickly. The two published a joint paper in 1953. See \fullcite{ventrisa}.




Linear B was added to the Unicode Standard in April, 2003 with the release of version 4.0.

The Linear B Syllabary block is \unicodenumber{U+10000–U+1007F}. The Linear B Ideograms block is {\smallcps U+10080–U+100FF}. The Unicode block for the related Aegean Numbers is U+10100–U+1013F.

\begin{scriptexample}[]{Linear B}
\unicodetable{linearb}{"10000,"10010,"10020,"10030,"10040,"10050,"10060,"10070}

\captionof{table}{Linear B Typeset with command \protect\string\linearb\ and the \texttt{Aegean} font.}
\end{scriptexample}

\begin{scriptexample}[]{Linear B}
\unicodetable{linearb}{"10080,"10090,"100A0,"100B0,"100C0,"100D0,"100E0,"100F0}
\captionof{table}{Linear B Ideograms. Typeset with command \protect\string\linearb\ and the \texttt{Aegean} font.}
\end{scriptexample}


\begin{scriptexample}[]{Aegean Numbers}
\unicodetable{linearb}{"10100,"10110,"10110,"10120,"10130}

\captionof{table}{Aegean Numbers}
\end{scriptexample}





\section{Phaestos Disc}


One of the puzzles of Minoan Crete is the Phaestos disc. The Phaistos Disc was discovered in the Minoan palace-site of Phaistos, near Hagia Triada, on the south coast of Crete;[1] specifically the disc was found in the basement of room 8 in building 101 of a group of buildings to the northeast of the main palace. This grouping of four rooms also served as a formal entry into the palace complex. Italian archaeologist Luigi Pernier recovered the intact \enquote{dish}, about 15 cm (5.9 in) in diameter and uniformly slightly more than 1 centimetre (0.39 inches) in thickness, on 3 July 1908 during his excavation of the first Minoan palace.

It was found in the main cell of an underground \enquote{temple depository}. These basement cells, only accessible from above, were neatly covered with a layer of fine plaster. Their content was poor in precious artifacts, but rich in black earth and ashes, mixed with burnt bovine bones. In the northern part of the main cell, in the same black layer, a few inches south-east of the disc and about 20 inches (51 centimetres) above the floor, Linear A tablet PH 1 was also found. The site apparently collapsed as a result of an earthquake, possibly linked with the eruption of the Santorini volcano that affected large parts of the Mediterranean region during the mid second millennium B.C.

\begin{figure}[htp]
\centering

\includegraphics[width=0.67\textwidth]{./phaistosdiscs.jpg}
\caption{Phaistos discs.}
\end{figure}

The Phaistos Disc is generally accepted as authentic by archaeologists.[2] The assumption of authenticity is based on the excavation records by Luigi Pernier. This assumption is supported by the later discovery of the Arkalochori Axe with similar but not identical glyphs.[3]


The possibility that the disc is a 1908 forgery or hoax has been raised by two scholars.[4][5][6] In his 2008 review, Robinson does not endorse the forgery arguments, but argues that \enquote{a thermoluminescence test for the Phaistos Disc is imperative. It will either confirm that new finds are worth hunting for, or it will stop scholars from wasting their effort.}[4]

A gold signet ring from Knossos (the Mavro Spilio ring), found in 1926, contains a Linear A inscription developed in a field defined by a spiral—similar to the Phaistos Disc.\footnote{See University of Cologne website \url{http://arachne.uni-koeln.de/arachne/index.php?view[layout]=objekt_item\&search[constraints][objekt][searchSeriennummer]=159123}} A sealing found in 1955 shows the only known parallel to sign 21 (the \enquote{comb}) of the Phaistos disc.[9] This is considered as evidence that the Phaistos Disc is a genuine Minoan artifact.[10]

\begin{figure}[htbp]
\centering

\includegraphics[width=4.5cm]{crete-spiral-ring}\includegraphics[width=4.5cm]{crete-spiral-ring-01}\includegraphics[width=4.5cm]{crete-spiral-ring-02}

\caption{A gold signet ring from Knossos (the Mavro Spilio ring), found in 1926, contains a Linear A inscription developed in a field defined by a spiral—similar to the Phaistos Disc}
\end{figure}

The disc is made of fine clay.  Both side of the disc carry an inscription arranged in a spiral around the centre. The characters were impressed with a punch or stamp before the clay was fired. There are
241 or 242 characters (one is damaged), which
comprise 45 signs of variable frequency. For
comparison, there are thousands of characters in a few pages of printed English text, comprising the 26 signs we call letters. Lines partition
the disc’s characters into 31 short sections on
side A and 30 on side B, most of which contain
three, four or five characters. It is tempting to
speculate that these sections represent words
in the language of the disc.

That the characters were printed, not carved,
is beyond dispute. But no one knows why the disc’s maker bothered to produce a punch or stamp for each sign, rather than inscribing each character afresh. Egyptian hieroglyphs or Mesopotamian cuneiform of the second
millennium bce are inscribed on stone or clay;
simlarly the Minoan scripts Linear A and B found
at Phaistos, Knossos and other Cretan sites. If
the punch or stamp was to \enquote{print} many copies of documents, one would expect further sam-
ples to have turned up in a century of intensive Mediterranean excavatio

There is patchy and inconclusive evidence for and against the disc’s Cretan origin. The
signs look nothing like those of Linear A, Linear B or any other Minoan script, except coincidentally. This has led some, including Evans and Chadwick, to propose that the disc — and presumably its language, too — was an import.

One sign bears a remarkable resemblance to the architecture of rock tombs found in Anatolia in modern Turkey. One or two others
resemble signs found on a few contemporaneous objects from different sites in Crete. Most
scholars today, including Duhoux, think it a plausible working hypothesis that the disc was made in Crete. Gareth Owens and his Team claim to have read the disc and you can hear how it sounded at a TED Talk\footnote{\url{https://www.youtube.com/watch?v=6Chcplx3tZ8}}.



\subsection{Signs}

There are 242 tokens on the disc, comprising 45 distinct signs. Many of these 45 signs represent easily identifiable every-day things. In addition to these, there is a small diagonal line that occurs underneath the final sign in a group a total of 18 times. The disc shows traces of corrections made by the scribe in several places. The 45 symbols were numbered by Arthur Evans from 01 to 45, and this numbering has become the conventional reference used by most researchers. Some symbols have been compared with Linear A characters by Nahm,[17] Timm,[3] and others. Other scholars (J. Best, S. Davis) have pointed to similar resemblances with the Anatolian hieroglyphs, or with Egyptian hieroglyphs (A. Cuny). In the table below, the character "names" as given by Louis Godart (1995) are given in upper case; where other description or elaboration applies, they are given in lower case.




\PrintUnicodeBlock{./languages/phaistos.txt}{\linearb}




The ideograms are symbols, not pictures of the objects in question, e.g. one tablet records a tripod with missing legs, but the ideogram used is of a tripod with three legs. In modern transcriptions of Linear B tablets, it is typically convenient to represent an ideogram by its Latin or English name or by an abbreviation of the Latin name. Ventris and Chadwick generally used English; Bennett, Latin. Neither the English nor the Latin can be relied upon as an accurate name of the object; in fact, the identification of some of the more obscure objects is a matter of exegesis.

\begingroup

\linearb

Vessels
\let\l\unicodenumber

\begin{tabular}{l>{\smallcps}l>{\smallcps}l>{\smallcps}l>{\smallcps}l}
𐃟	&U+100DF	&200	&\l{sartāgo}	&\l{Boiling Pan}\\
𐃠	&U+100E0	&201	&\l{tripūs}	&\l{Tripod Cauldron}\\
𐃡	&U+100E1	&202	&\l{pōculum}	&\l{Goblet}\\
𐃢	&U+100E2	&203	&\l{urceus}	&\l{Wine Jar?}\\
𐃣	&U+100E3	&204  &\l{Tahirnea}	&\l{Ewer}\\
𐃤	&U+100E4	&205  &\l{Tnhirnula}	&\l{Jug}\\
𐃥	&U+100E5	&206	&\l{hydria}	&Hydria\\
𐃦	&U+100E6	&207	&\l{TRIPOD}  &AMPHORA\\
𐃧	&\l{U+100E7}	&\l{208}	&\l{PAT patera}	&\l{BOWL}\\
𐃨	&U+100E8	&209	&AMPH amphora	&AMPHORA\\
𐃩	&U+100E9	&210	&STIRRIP &JAR\\
𐃪	&U+100EA	&211	&WATER &BOWL?\\
𐃫	&U+100EB	&212	&SIT situla	&WATER JAR?\\
𐃬	&U+100EC	&213	&LANX lanx	&COOKING BOWL\\
\end{tabular}




\subsection{Online Resources}

Corpora and GORILA \url{http://www.people.ku.edu/~jyounger/LinearA/\#3}



\endgroup











\section{Cypriot Syllabary}
\label{s:cypriot}
The Cypriot or Cypriote syllabary is a syllabic script used in Iron Age Cyprus, from ca. the 11th to the 4th centuries BCE, when it was replaced by the Greek alphabet. A pioneer of that change was king Evagoras of Salamis. It is descended from the Cypro-Minoan syllabary, in turn a variant or derivative of Linear A. Most texts using the script are in the Arcadocypriot dialect of Greek, but some bilingual (Greek and Eteocypriot) inscriptions were found in Amathus.

\begin{figure}[htb]
\centering
\begin{minipage}{7cm}
\includegraphics[width=7cm]{./images/idalion-tablet.jpg}
\end{minipage}\hspace{1.5em}
\begin{minipage}{6cm}
\captionof{figure}{The bronze Idalion Tablet, from Idalium, (Greek: Ιδάλιον), is from the 5th century BCE Cyprus. The tablet is inscribed on both sides.
The script of the tablet is in the Cypro-Minoan syllabary, and the inscription is in Greek. The tablet records a contract between "the king and the city":[1] the topic of the tablet rewards a family of physicians, of the city, for providing free health services to individuals fighting an invading force of Persians.}
\end{minipage}
\end{figure}


The characters are \textit{syllabic}. There is one character for each  vowel, \textit{a, e, i, o, u,} and perhaps one for \textit{o}. There is no distinction between long and short vowels. The other characters represent what are called \textit{open syllables}\footnote{ If a syllable ends with a consonant, it is called a closed syllable. If a syllable ends with a vowel, it is called an open syllable. }, i.e., beginning with a consonant and ending with a vowel. 

No distinction is made between smooth, middle and rough mutes. The same character stands for τά τ\’ασs, δα in Εδαλιον ανδ δα ιν Αθανα  κε, κη, γε, γη, χε, χη. This fact constitutes the greatest difficulty in reading Cypriote.  

The Cypriot syllabary was added to the Unicode Standard in April, 2003 with the release of version 4.0.
The Unicode block for Cypriot is \unicodenumber{U+10800–U+1083F}. The Unicode block for the related Aegean Numbers is \unicodenumber{U+10100–U+1013F}.

\newfontfamily\cypriote{Aegean.ttf}

\begin{scriptexample}[]{Cypriot Syllabary}
\unicodetable{cypriote}{"10800,"10810,"10820,"10830}

\cypriote \symbol{"10803}
\end{scriptexample}


\printunicodeblock{./languages/cyprus.txt}{\cypriote}

\section{Old Persian}
\label{s:oldpersian}


Old Persian, like Hittite an Indo-European language, was written in cuneiforms as of the first millenium BC, mostly between 550 and 350. King Darius’ monumental inscription at
Bisothum – in Old Persian, Elamite and Neo-Babylonian – furnished
the ‘key’ to cuneiform’s decipherment and the reconstruction
of these languages.28 Darius’ Old Persian scribes
effected the most drastic simplification of the borrowed Near
Eastern script (illus. 35). They reduced the cuneiform inventory
to only 41 signs of both syllabic (ka) and phonemic (/k/) values.
Thus, Old Persian cuneiform is ‘half syllabic, half letter writing’.
29 It appears to be on the fence between the Babylonians’
cuneiforms and the Levantines’ consonantal writing, a hybrid
solution using only four logograms and 36 syllabo-phonemic
signs written in wedges. Of particular significance is the fact
that Old Persian also conveys the individual long and short
vowels /a/ (pronounced AH), /i/ (EE) and /u/ (OO) that the
Ugaritic system had conveyed a thousand years earlier.

Old Persian cuneiform is a semi-alphabetic cuneiform script that was the primary script for the Old Persian language. Texts written in this cuneiform were found in Persepolis, Susa, Hamadan, Armenia, and along the Suez Canal.[1] They were mostly inscriptions from the time period of Darius the Great and his son Xerxes. Later kings down to Artaxerxes III used corrupted forms of the language classified as “pre-Middle Persian”.

\begin{scriptexample}[]{Old Persian}
\unicodetable{oldpersian}{"103A0,"103B0,"103C0,"103D0}
\end{scriptexample}

Scholars today mostly agree that the Old Persian script was invented by about 525 BC to provide monument inscriptions for the Achaemenid king Darius I, to be used at Behistun. While a few Old Persian texts seem to be inscribed during the reigns of Cyrus the Great (CMa, CMb, and CMc, all found at Pasargadae), the first Achaemenid emperor, or Arsames and Ariaramnes (AsH and AmH, both found at Hamadan), grandfather and great-grandfather of Darius I, all five, specially the later two, are generally agreed to have been later inscriptions.
Around the time period in which Old Persian was used, nearby languages included Elamite and Akkadian. One of the main differences between the writing systems of these languages is that Old Persian is a semi-alphabet while Elamite and Akkadian were syllabic. In addition, while Old Persian is written in a consistent semi-alphabetic system, Elamite and Akkadian used borrowings from other languages, creating mixed systems.
\medskip

{\leftskip-1.25cm
\includegraphics[width=\textwidth+2.5cm]{./images/naghshe.jpg}
\captionof{figure}{Panoramic view of the Naqsh-e Rustam. This site contains the tombs of four Achaemenid kings, including those of Darius I and Xerxes. (\textit{Wikimedia})}
}
\section{Inscriptional Pahlavi}
\label{s:inscriptionalpahlavi}
\newfontfamily\inscriptionalpahlavi{Noto Sans Inscriptional Pahlavi}

Pahlavi or Pahlevi denotes a particular and exclusively written form of various Middle Iranian languages. The essential characteristics of Pahlavi are[1]
the use of a specific Aramaic-derived script, the Pahlavi script;
the high incidence of Aramaic words used as heterograms (called hozwārishn, "archaisms").

Pahlavi compositions have been found for the dialects/ethnolects of Parthia, Parsa, Sogdiana, Scythia, and Khotan.[2] Independent of the variant for which the Pahlavi system was used, the written form of that language only qualifies as Pahlavi when it has the characteristics noted above.


Pahlavi is then an admixture of
written Imperial Aramaic, from which Pahlavi derives its script, logograms, and some of its vocabulary.

spoken Middle Iranian, from which Pahlavi derives its terminations, symbol rules, and most of its vocabulary.
Pahlavi may thus be defined as a system of writing applied to (but not unique for) a specific language group, but with critical features alien to that language group. It has the characteristics of a distinct language, but is not one. It is an exclusively written system, but much Pahlavi literature remains essentially an oral literature committed to writing and so retains many of the characteristics of oral composition.

\begin{scriptexample}[]{Pahlavi}
\unicodetable{inscriptionalpahlavi}{"10B60,"10B70}
\end{scriptexample}

\section{Imperial Aramaic}
\label{s:imperialaramaic}

\subsection{History}

Aramaic is the best-attested and longest-attested
member of the NW Semitic subfamily of languages
(which also includes inter alia \nameref{s:hebrew}, \nameref{s:phoenician},
\nameref{s:ugaritic}, Moabite, Ammonite, and Edomite). The
relatively small proportion of the biblical text
preserved in an Aramaic original (Dan 2:4–7:28; Ezra
4:8–68 and 7:12–26; Jeremiah 10:11; Gen 31:47 [two
words] as well as isolated words and phrases in
Christian Scriptures) belies the importance of this
language for biblical studies and for religious studies
in general, for Aramaic was the primary international
language of literature and communication throughout
the Near East from ca. 600 B.C.E. to ca. 700 C.E. and
was the major spoken language of Palestine, Syria,
and Mesopotamia in the formative periods of
Christianity and rabbinic Judaism. 



Aramaic survived over a period of 3,000 years, during which time its grammar, vocabulary and usage experienced great changes. Aramaic scholars found it useful to divide the several Aramaic dialects into periods, groups and subgroups based both on the chronology as well as the geography.

\begin{enumerate}
\item Old Aramaic
\item Imperial Aramaic
\item  Middle Aramaic
\item Late Aramaic
\item Modern Aramaic
\end{enumerate}


\subsection{Old Aramaic (to ca. 612 BCE)}
This period
witnessed the rise of the Arameans as a major force
in ANE history, the adoption of their language as an
international language of diplomacy in the latter days
of the Neo-Assyrian Empire, and the dispersal of
Aramaic-speaking peoples from Egypt to Lower
Mesopotamia as a result of the Assyrian policies of
deportation. The scattered and generally brief
remains of inscriptions on imperishable materials
preserved from these times are enough to
demonstrate that an international standard dialect had
not yet been developed. The extant texts may be
grouped into several dialects:

\subsection{Middle Aramaic (to ca. 250 C.E.)}
In the Hellenistic and Roman periods, Greek replaced
Aramaic as the administrative language of the Near
East, while in the various Aramaic-speaking regions
the dialects began to develop independently of one
another. Written Aramaic, however, as is the case
with most written languages, by providing a
somewhat artificial, cross-dialectal uniformity,
continued to serve as a vehicle of communication
within and among the various groups. For this
purpose, the literary standard developed in the
previous period, Standard Literary Aramaic, was
used, but lexical and grammatical differences based
on the language(s) and dialect(s) of the local
population are always evident. It is helpful to divide
the texts surviving from this period into two major
categories: epigraphic and canonical.

\subsection{Late Aramaic (to ca. 1200 C.E.)}
The bulk of
our evidence for Aramaic comes from the vast
literature and occasional inscriptions of this period.
During the early centuries of this period Aramaic
dialects were still widely spoken. During the second
half of this period, however, Arabic had already
displaced Aramaic as the spoken language of much
of the population. Consequently, many of our texts
were composed and/or transmitted by persons whose
Aramaic dialect was only a learned language.
Although the dialects of this period were previously
divided into two branches (Eastern and Western), it
now seems best to think rather of three: Palestinian,
Syrian, and Babylonian.

The Aramaic alphabet is adapted from the \nameref{s:phoenician} alphabet and became distinctive from it by the 8th century BCE.  The letters all represent consonants, some of which are \emph{matres lectionis}, which also indicate long vowels.

\subsection{Modern Aramaic (to the present day)}

These dialects can be divided into the same three
geographic groups.

\begin{description}

\item[a. Western]
Here Aramaic is still spoken only in
the town of Ma’lula (ca. 30 miles NNE of Damascus)
and surrounding villages. The vocabulary is heavily
Arabized.

\item[b. Syrian]
Western Syrian (Turoyo) is the language
of Jacobite Christians in the region of Tur-Abdin in
SE Turkey. This dialect is the descendant of
something very like classical Syriac. Eastern Syrian
is spoken in the Kurdistani regions of Iraq, Iran,
Turkey, and Azerbaijan by Christians and, formerly,
by Jews. Substantial communities of the former are
now found in North America. The Jewish speakers
have mostly settled in Israel. These dialects are
widely spoken by their respective communities and
have been studied extensively during the past
century. It has become clear that they are not the
descendants of any known literary Aramaic dialect.

\item[c. Babylonian] 

\nameref{s:mandaic} is still used, at least until
recently, by some Mandaeans in southernmost Iraq
and adjacent areas in Iran.

In addition, in recent years classical \nameref{s:syriac} has
undergone somewhat of a revival as a learned vehicle
of communication for Syriac Christians, both in the
Middle East and among immigrant communities in
Europe and North America.
\end{description}

\begin{figure}[htbp]
\centering
\includegraphics[width=0.6\textwidth]{./images/elephantine-papyrus.jpg}

\caption{The Elephantine papyri are ancient Jewish papyri dating to the 5th century BC, requesting the rebuilding of a Jewish temple. It also name three persons mentioned in Nehemiah: Darius II, Sanballat the Horonite and Johanan the high priest.}

\end{figure}


\subsection{Alphabet and typesetting}

The Aramaic alphabet is historically significant, since virtually all modern Middle Eastern writing systems can be traced back to it, as well as numerous non-Chinese writing systems of Central and East Asia. This is primarily due to the widespread usage of the Aramaic language as both a \emph{lingua franca} and the official language of the Neo-Assyrian Empire, and its successor, the Achaemenid Empire. Among the scripts in modern use, the Hebrew alphabet bears the closest relation to the Imperial Aramaic script of the 5th century BC, with an identical letter inventory and, for the most part, nearly identical letter shapes.

Writing systems that indicate consonants but do not indicate most vowels (like the Aramaic one) or indicate them with added diacritical signs, have been called abjads by Peter T. Daniels to distinguish them from later alphabets, such as Greek, that represent vowels more systematically. This is to avoid the notion that a writing system that represents sounds must be either a syllabary or an alphabet, which implies that a system like Aramaic must be either a syllabary (as argued by Gelb) or an incomplete or deficient alphabet (as most other writers have said); rather, it is a different type.

The Imperial Aramaic alphabet was added to the Unicode Standard in October 2009 with the release of version 5.2.
The Unicode block for Imperial Aramaic is \unicodenumber{U+10840–U+1085F}.

\begin{scriptexample}[]{Aramaic}
\unicodetable{imperialaramaic}{"10840,"10850}
\end{scriptexample}




\PrintUnicodeBlock{./languages/imperial-aramaic.txt}{\imperialaramaic}
\subsection{Ogham}

\newfontfamily\ogham{code2000.ttf}

Ogham was added to the Unicode Standard in September 1999 with the release of version 3.0.
The spelling of the names given is a standardisation dating to 1997, used in Unicode Standard and in Irish Standard 434:1999.
The Unicode block for ogham is \texttt{U+1680–U+169F}.

\begin{scriptexample}[]{Ogham}
\bgroup
\ogham
0	1	2	3	4	5	6	7	8	9	A	B	C	D	E	F\\
U+168x	   	ᚁ	ᚂ	ᚃ	ᚄ	ᚅ	ᚆ	ᚇ	ᚈ	ᚉ	ᚊ	ᚋ	ᚌ	ᚍ	ᚎ	ᚏ\\
U+169x	ᚐ	ᚑ	ᚒ	ᚓ	ᚔ	ᚕ	ᚖ	ᚗ	ᚘ	ᚙ	ᚚ	᚛	᚜	\\

\titus

0	1	2	3	4	5	6	7	8	9	A	B	C	D	E	F\\
U+168x	   	ᚁ	ᚂ	ᚃ	ᚄ	ᚅ	ᚆ	ᚇ	ᚈ	ᚉ	ᚊ	ᚋ	ᚌ	ᚍ	ᚎ	ᚏ\\
U+169x	ᚐ	ᚑ	ᚒ	ᚓ	ᚔ	ᚕ	ᚖ	ᚗ	ᚘ	ᚙ	ᚚ	᚛	᚜
\egroup		
\end{scriptexample}
\section{Ancient Anatolian Alphabets}

The Anatolian scripts described in this section all date from the first millenium BCE, and were used to write various ancient Indo-European languages of western and southwestern Anatolia (now Turkey). All are related to the Greek script and are probably adaptations of it. 

\newfontfamily\lycian{Aegean.ttf}
\let\lydian\lycian
\let\carian\lydian

\begin{description}
\item [Lycian] The Lycian alphabet was used to write the Lycian language. It was an extension of the Greek alphabet, with half a dozen additional letters for sounds not found in Greek. It was largely similar to the Lydian and the Phrygian alphabets.
 
\bgroup
\lydian
\obeylines
0	1	2	3	4	5	6	7	8	9	A	B	C	D	E	F
U+1028x	𐊀	𐊁	𐊂	𐊃	𐊄	𐊅	𐊆	𐊇	𐊈	𐊉	𐊊	𐊋	𐊌	𐊍	𐊎	𐊏
U+1029x	𐊐	𐊑	𐊒	𐊓	𐊔	𐊕	𐊖	𐊗	𐊘	𐊙	𐊚	𐊛	𐊜

Typeset with the \idxfont{Aegean.ttf} and the command \cmd{\lydian}
\egroup

\item[Lydian] Lydian script was used to write the Lydian language. That the language preceded the script is indicated by names in Lydian, which must have existed before they were written. Like other scripts of Anatolia in the Iron Age, the Lydian alphabet is a modification of the East Greek alphabet, but it has unique features. The same Greek letters may not represent the same sounds in both languages or in any other Anatolian language (in some cases it may). Moreover, the Lydian script is alphabetic.
Early Lydian texts are written both from left to right and from right to left. Later texts are exclusively written from right to left. One text is boustrophedon. Spaces separate words except that one text uses dots. Lydian uniquely features a quotation mark in the shape of a right triangle.
The first codification was made by Roberto Gusmani in 1964 in a combined lexicon (vocabulary), grammar, and text collection.


\bgroup
\lycian
\obeylines
	0	1	2	3	4	5	6	7	8	9	A	B	C	D	E	F
U+1092x	𐤠	𐤡	𐤢	𐤣	𐤤	𐤥	𐤦	𐤧	𐤨	𐤩	𐤪	𐤫	𐤬	𐤭	𐤮	𐤯
U+1093x	𐤰	𐤱	𐤲	𐤳	𐤴	𐤵	𐤶	𐤷	𐤸	𐤹						𐤿
Typeset with the \idxfont{Aegean.ttf} and the command \cmd{\lycian}

Examples of words

𐤬𐤭𐤠  - Ora - "Month"

𐤬𐤳𐤦𐤭𐤲𐤬𐤩  - Laqrisa - "Wall"

𐤬𐤭𐤦𐤡  - "House, Home"

\egroup

\item [Carian] The Carian alphabets are a number of regional scripts used to write the Carian language of western Anatolia. They consisted of some 30 alphabetic letters, with several geographic variants in Caria and a homogeneous variant attested from the Nile delta, where Carian mercenaries fought for the Egyptian pharaohs. They were written left-to-right in Caria (apart from the Carian–Lydian city of Tralleis) and right-to-left in Egypt. Carian was deciphered primarily through Egyptian–Carian bilingual tomb inscriptions, starting with John Ray in 1981; previously only a few sound values and the alphabetic nature of the script had been demonstrated. The readings of Ray and subsequent scholars were largely confirmed with a Carian–Greek bilingual inscription discovered in Kaunos in 1996, which for the first time verified personal names, but the identification of many letters remains provisional and debated, and a few are wholly unknown.

\begin{scriptexample}[]{Carian}
\bgroup
\carian
\obeylines
 	0	1	2	3	4	5	6	7	8	9	A	B	C	D	E	F
U+102Ax	𐊠	𐊡	𐊢	𐊣	𐊤	𐊥	𐊦	𐊧	𐊨	𐊩	𐊪	𐊫	𐊬	𐊭	𐊮	𐊯
U+102Bx	𐊰	𐊱	𐊲	𐊳	𐊴	𐊵	𐊶	𐊷	𐊸	𐊹	𐊺	𐊻	𐊼	𐊽	𐊾	𐊿
U+102Cx	𐋀	𐋁	𐋂	𐋃	𐋄	𐋅	𐋆	𐋇	𐋈	𐋉	𐋊	𐋋	𐋌	𐋍	𐋎	𐋏
U+102Dx	𐋐
\egroup
\end{scriptexample}

\newfontfamily\oldpunctuation{code2000.ttf}

Word dividers are infrequent, \emph{scriptio continua}\footnote{a style of writing without word dividers, that is, without spaces or other marks between words or sentences} is common. Words dividers which are attested are U+00B7 (\char"00B7) \textsc{MIDLE DOT} (or U+2E31 word separator middle dot), U+205A TWO DOT PUNCTUATION, and U+205D TRICOLON ({\oldpunctuation\char"205D}). In modern editions U+0020 SPACE may be found.

\end{description}

\section{Phoenician}
\label{s:phoenician}
\arial

The Phoenician alphabet and its successors were widely used over a broad area surrounding the Mediterranean Sea.

\let\phoenician\lycian

\begin{scriptexample}[]{Phoenician}

\unicodetable{phoenician}{"10900,"10910}

\end{scriptexample}

The Phoenician alphabet, called by convention the Proto-Canaanite alphabet for inscriptions older than around 1200 BCE, is the oldest verified consonantal alphabet, or abjad.[1] It was used for the writing of Phoenician, a Northern Semitic language, used by the civilization of Phoenicia. It is classified as an abjad because it records only consonantal sounds (matres lectionis were used for some vowels in certain late varieties).

Phoenician became one of the most widely used writing systems, spread by Phoenician merchants across the Mediterranean world, where it evolved and was assimilated by many other cultures. The Aramaic alphabet, a modified form of Phoenician, was the ancestor of modern Arabic script, while Hebrew script is a stylistic variant of the Aramaic script. The Greek alphabet (and by extension its descendants such as the Latin, the Cyrillic, and the Coptic) was a direct successor of Phoenician, though certain letter values were changed to represent vowels.

\begin{figure}[ht]
\includegraphics[width=\textwidth]{./images/phoenician.jpg}
\captionof{figure}{
Phoenician votive inscription from Idalion (Cyprus), 390 BC. BM 125315 from The Early Alphabet by John F. Healy.}
\end{figure}

As the letters were originally incised with a stylus, most of the shapes are angular and straight, although more cursive versions are increasingly attested in later times, culminating in the Neo-Punic alphabet of Roman-era North Africa. Phoenician was usually written from right to left, although there are some texts written in boustrophedon.


\printunicodeblock{./languages/phoenician.txt}{\phoenician}


\newpage
\section{Palmyrene}
\idxlanguage{Palmyrene}
\arial

Palmyrene is the very widely attested Aramaic dialect and script
of Palmyra in the Syrian desert. The texts date from the midfirst century to the destruction of Palmyra by the Romans in AD 272. Palmyra in the Roman period was a major trading centre.
\medskip

\begin{figure}[ht]
\centering

\includegraphics[width=0.9\textwidth]{./images/palmyrene.jpg}
\captionof{figure}{\protect\arial Limestone bust with Palmyrene inscription. Palmyra late 2nd century AD. BM WA 102612}

\end{figure}

\medskip
The longest of the Palmyrene texts, is the bilingual  taxation tariff written for the year 137 AD in Palmyrene Aramaic and Greek.\footnote{For more details see:MILIK J.T., Dédicaces faites par des dieux (Palmyre, Hatra, 
Tyr) et de thiases sémitiques à l'époque romaine, Paris 1972; ROSENTHAL R., Die 
Sprache der palmyrenischen Inschriften, Leipzig 1936; STARK J.K., Personal Names in 
Palmyrene Inscriptions, Oxford 1971; DRIJVERS H.J.W., The Religion of Palmyra, 
Leiden 1976; TEIXIDOR J., 'Palmyre et son commerce d'Auguste à Caracalla', in 
Semitica 34, (1984) 1-127.  } Trade connections 
took the Palmyrene script to other places, some not far away, such as Dura Europos on the Euphrates, butothers at a great distance. A particular inscription is from South Shields, Roman Arbeia, in the north-east of England, carved on behalf of a Palmyrene mechant for his deceased wife and probably dating to the early third century AD. 

The Palmyrene script existed in two main varieties, a monumental and a cursive one, though the latter is little known and the evidence  mostly from Palmyra itself. The Syriac script of Edessa in southern Turkey, is often regarded as derived or closely related to the Palmyrene---similarities are found in the letters: ', b, g, d, w, h, y, k, l, m, n, `, r and t---though a strong case can also be made for connecting Syriac with a northern Mesopotamian script-family represented principally in texts from Hatra, a city more or less contemporary with Palmyra in Upper Mesopotamia. 


\begin{figure}[ht]
\includegraphics[width=\textwidth]{./images/regina-epigraph.jpg}
\caption{It was customary for Palmyrenes to offer bilingual texts (Greek or Latin) on funerary monuments. The final line of Regina's epitaph is Barates' personal lament in Palmyrene: Regina, freedwoman of Barate, alas. (See \href{http://www2.cnr.edu/home/araia/regina.html}{regina}.)}
\end{figure}

A good article on the classification of Aramaic languages can be found in \textit{The Aramaic language and Its Classification} by Efrem Yildiz.\footnote{\url{http://www.jaas.org/edocs/v14n1/e8.pdf}}








\cxset{quotation font-size=\normalsize,
       quote font-size=\normalsize}


\section{Mandaic}
\label{s:mandaic}
\newfontfamily\mandaic{NotoSansMandaic-Regular.ttf}


The Mandaic script is used to write a dialect of Eastern Aramaic, which, in its classical
form, is currently used as the liturgical language of the Mandaean religion. A living language descended
from Classical Mandaic is spoken by a small number of people living in and around Ahvaz, Khūzestān,
in southwestern Iran; speakers are also found in emigrant communities in Sweden, Australia, and the
United States. There is a considerable amount of Iranian influence on the lexicon of Classical Mandaic,
and Arabic and Persian influence on the grammar and lexicon of the contemporary dialect. The script
itself is likely derived from the Parthian chancery script.

Mandaic is a right-to-left script. It is a true alphabet, using letters regularly for vowels
rather than as the \emph{matres lectionis} from which they derived. The three diacritical marks are used in
teaching materials to differentiate vowel quality. At present, at least, the rule is that they may be omitted
from ordinary text. In this regard they are very like the Arabic fatha, kasra, and damma or the Hebrew
vowel points.

The only so far I could find that can display the script is the Google \idxfont{NotoSansMandaic.ttf}.

\begin{scriptexample}[]{Mandaic}
\bgroup
\unicodetable{mandaic}{"0840,"0850}
\egroup
\end{scriptexample}

In 1943, Lady Ethel Drower published extracts from several magic “recipe books” that served the writers of amulets in Baghdad in the early 20th century, in particular from two manuscripts in her possession, DC 45 and DC 46.

\begin{figure}[hb]
\centering

\includegraphics[height=4cm]{./magic-letters.jpg}
\includegraphics[height=4cm]{./45-453.jpg}
\includegraphics[height=4cm]{./36-448.jpg}

\captionof{figure}{Mandaic Incantation vessels. The left image is from \protect\href{http://thesacredalphabet.blogspot.ae/}{thesacredalphabet}, whereas the last two are from \protect\href{http://www.archaeological-center.com/en/auctions/45-453}{archaeological-center} }
\end{figure}

 While Drower, following her native informants, entitled the work ‘A Mandæan Book of Black Magic’, the manuscripts themselves contain a wide range of formulae for amulets and talismans for various purposes, as Drower herself was well aware. Alongside spells for healing, protection and success, we find others for enflaming love or stirring up enmity.

 The manuscripts themselves appear to have been copied in the late 19th or early 20th centuries; in particular, DC 46, a substantial codex of 264 sides, is written on an extremely modern “clean” paper. DC 45 is written on a rougher paper and appears to be somewhat earlier. It is also more fragmentary, and contains several leaves that were copied by a different hand and inserted into the main part of the manuscript at a later date, though it is clear from their contents that they were intended to replace pages that had been worn or damaged, as they begin and end exactly as required by the preceding and following pages. As it survives today, DC 45 is also considerably shorter than DC 46; however, it also contains several spells that are not found in DC 46.\footnote{\protect\href{http://www.academia.edu/8294938/Arabic_Magic_Texts_in_Mandaic_Script_A_Forgotten_Chapter_in_Near-Eastern_Magic}{Magic Texts}}

Lady Drower inform us that among the Mandaens:

\begin{quote}
Writing in itself is a magic art, and the alphabet is sacred.
Each letter is supposed to invoke a spirit of light and is a thing of power. It is a practice to write the letters separately and to sleep each night with a letter beneath the pillow. If the sleeper sees in a dream something which will enlighten him, the letters upon which he slept that night is taken to a silversmith and a replica in gold or silver is made and worn around the neck as amulet See Mandaic Incantation Texts by Edwin M Yamauchi.
\end{quote}












\newcounter{glyphcount}
^^A\newfontfamily\aegyptus{AegyptusR.ttf}

\chapter{Aegyptian Hieroglyphics}

\index{fonts>Aegyptus}\index{Aegyptus (font)}
\index{fonts>Hieroglyphics}\index{languages>hieroglyphics}

\newfontfamily\hiero{NotoSansEgyptianHieroglyphs-Regular.ttf}

Hieroglyphic writing appeared in Egypt at the end of the fourth millennium bce. The writing
system is pictographic: the glyphs represent tangible objects, most of which modern
scholars have been able to identify. A great many of the pictographs are easily recognizable
even by nonspecialists. Egyptian hieroglyphs represent people and animals, parts of the
bodies of people and animals, clothing, tools, vessels, and so on.

Hieroglyphs were used to write Egyptian for more than 3,000 years, retaining characteristic
features such as use of color and detail in the more elaborated expositions. Throughout the
Old Kingdom, the Middle Kingdom, and the New Kingdom, between 700 and 1,000 hieroglyphs
were in regular use. During the Greco-Roman period, the number of variants, as
distinguished by some modern scholars, grew to somewhere between 6,000 and 8,000.

Hieroglyphs were carved in stone, painted on frescoes, and could also be written with a reed
stylus, though this cursive writing eventually became standardized in what is called \emph{hieratic}
writing. Unicode does not encode the hieratic forms separately, but ust considers them as cursive forms of the hieroglyphs encoded block.

The Demotic script and then later the Coptic script replaced the earlier hieroglyphic and
hieratic forms for much practical writing of Egyptian, but hieroglyphs and hieratic continued
in use until the fourth century ce. An inscription dated August 24, 394 ce has been
found on the Gateway of Hadrian in the temple complex at Philae; this is thought to be
among the latest examples of Ancient Egyptian writing in hieroglyphs

\begin{figure}[htb]
\includegraphics[width=\textwidth]{./images/bookofthedead.jpg}
\end{figure}

In hieroglyphic texts, these drawings are not only simply arranged in sequential order, but also grouped on top of and next to each other. This rather complicates matters trying to register and reproduce hieroglyphic texts using a computer.

\section{Computer Typesetting}

Typesetting hieroglyphics with computers presents a number of problems. First is the method of inputting the characters and second the various methods required to stack hieroglyphics, the direction of writing which can be one of four different directions.

When the first computers were introduced in Egyptology in the late 1970s and the beginning of the 1980s, the graphical capacity of the machines was still in its infancy. Early attempts to register the hieroglyphic pictorial writing on computer therefore chose an encoding system to do this, using alphanumeric codes to represent or replace the graphics. To prevent many people from reinventing the wheel, during the first "Table Ronde Informatique et Egyptologie" in 1984 a committee was charged with the task to develop a uniform system for the encoding of hieroglyphic texts on computer. The resulting Manual for the Encoding of Hieroglyphic Texts for Computer-input (Jan Buurman, Nicolas Grimal, Jochen Hallof, Michael Hainsworth and Dirk van der Plas, Informatique et Egyptologie 2, Paris 1988), simply called Manuel de Codage, presents an easy to use and intuitive way of encoding hieroglyphic writing as well as the abbreviated hieroglyphic transcription (transliteration). The system proposed by the Manuel de Codage has since been adopted by international Egyptology as the official common standard for registering hieroglyphic texts on computer. Mark-Jan Nederhof proposed an enhanced encoding scheme to remove many of the limitations in the Manuel de Codage.

\pkgname{HieroTeX} is a \latexe package developed by to typeset hieroglyphic texts and still works well. The advantages of using \tex is of course its excellent typesetting capabilities and the usage of macros. Although inputting the texts as MdC codes is not that difficult, repeating the same codes over and over can be avoided with easily constructed simple substitution macros. 

\subsection{fonts}

One of the best fonts I came across is \idxfont{Aegyptus} from \url{http://users.teilar.gr/~g1951d/}\footnote{The site also has fonts for Aegean Numbers, Ancient Greek Musical Notation, Ancient Greek Numbers, Ancient Roman Symbols, Arkalochori Axe, Carian, Cypriot Syllabary, Dispilio tablet, Linear A, Linear B Ideograms, Linear B Syllabary, Lycian, Lydian, Old Italic, Old Persian, Phaistos Disc, Phoenician, Phrygian, Sidetic, Troy vessels’ signs and Ugaritic. Cretan Hieroglyphs and Cypro-Minoan script(s) are offered in separate files.}. The font provides all the unicode characters and also offers an additional number of glyphs that are not in the Unicode standard. The font uses the Unicode Private Use Areas to encode the glyphs. 

Another font is the Noto Egyptian Hieroglyphics from Google. This is a lightweight font with the symbols in their proper unicode slots. Mark-Jan Nederhof's \idxfont{NewGardiner} font is another one with support only for the Gardiner set. The codepoint mappings are incorrect, as the font has been  
encoded to EGPZ. The font is similar to the Aegyptus font, however it is just transposed and not recommended unless it is transposed. 

The editor software JSesh\footnote{\protect\url{http://jsesh.qenherkhopeshef.org/}} also provides a free font |JSeshFont.ttf|. This offers a correctly mapped unicode and is another good alternative. The symbols are drawn somewhat simpler and is just a matter of taste what you want to use.

My recommendation is for short demonstration purposes, the Noto font is to be preferred while for more serious work the Aegyptus font will be more useful. Using Lua the font can be transposed automatically to allow the use of commands that refer to unicode numbers. Another advantage of the Aegyptus font is that the glyphs are named with their Gardiner numbers, so it is somewhat easier to programmatically access them by name.\footnote{Unicode does not name the glyphs, but simply calls the Egyptian Hieroglyph $n$. } 

\medskip

\ifxetex
\bgroup
\centering 
\font\myfont = "Aegyptus"
\scalebox{7}{\myfont\XeTeXglyph 201}
\scalebox{7}{\myfont\XeTeXglyph 203}
\scalebox{7}{\myfont\XeTeXglyph 163}
\scalebox{7}{\myfont\XeTeXglyph 164}
\scalebox{7}{\myfont\XeTeXglyph 165}
\scalebox{7}{\myfont\XeTeXglyph 168}
\captionof{table}{Example of Egyptian Hieroglyphics typeset with the \textit{Aegyptus} font.} 
\egroup
\fi

\ifluatex
\bgroup
\centering 
\aegyptus
\scalebox{7}{\char"F300C}
\scalebox{7}{\char"F3001}
\scalebox{7}{\char"F3010}
\scalebox{7}{\char"F308B}
\scalebox{7}{\char"F3097}
\scalebox{7}{\char"F3091}
\captionof{table}{Example of Egyptian Hieroglyphics typeset with the \textit{Aegyptus} font.} 
\egroup

\fi


\subsection{Unicode Block}

Egyptian hieroglyphs is a Unicode block containing the Gardiner's sign list of Egyptian hieroglyphics.
The code points, in the range |0x13000| to |0x1342E|, are available starting from
\href{http://unicode.org/charts/PDF/U13000.pdf}{Unicode 5.2}

\begin{scriptexample}[]{Hieroglyphic}
\bgroup
\unicodetable{hiero}{"13000,"13010,"13020,"13030,"13040,"13050,"13060,"13070,%
"13080,%
"13090,"130A0,"130B0,"130C0,"130D0,"130E0,"130F0,%
"13100,"13110,"13120,"13130,"13140,"13150,"13060,"13070,"13080,"13090}
\egroup
\end{scriptexample}

\subsection{Gardiner's classification}

The standard reference on Egyptian hieroglyphics is Gartiner's Sign List, which lists common Egyptian hieroglyphs. These are grouped in categories from A-Aa. Each category represents a theme for example category A, is "man and his occupations". Based on this list ``Queen with flower" is denoted as \texttt{B7}. 

\subsubsection{Character Names} 

Egyptian hieroglyphic characters have traditionally been designated in
several ways:

\begin{enumerate}
\item  By complex description of the pictographs: \texttt{GOD WITH HEAD OF IBIS}, and so forth.
\item By standardized sign number: C3, E34, G16, G17, G24.
\item For a minority of characters, by transliterated sound.
\end{enumerate}

The characters in the Unicode Standard make use of the standard Egyptological catalog
numbers for the signs. Thus, the name for {\hiero\char"130F9} |U+13049| egyptian hieroglyph e034 refers
uniquely and unambiguously to the Gardiner list sign E34, described as a “{\aegean DESERT HARE}” ({\hiero \char"130FA}) and used for the sound “wn”. The Unicode catalog values are padded to three places with
zeros, so where the Gardiner classification is shown as \texttt{E34}, the unicode value is \texttt{E034}. 

Names for hieroglyphic characters identified explicitly in Gardiner 1953 or other sources as
variants for other hieroglyphic characters are given names by appending “A”, “B”, ... to the sign number. In the sources these are often identified using asterisks. Thus Gardiner’s G7,
G7*, and G7** correspond to U+13146 egyptian sign g007 {\hiero \char"13147}, U+13147 egyptian sign g007a, and U+13148 egyptian sign g007b, respectively.

\def\texthiero#1{{\color{black!95}\hiero #1}}

\begin{longtable}{>{\Large}lll>{\ttfamily}l}
{\hiero \char"13000}&A1-A70 & Man and his occupations &U+13000-1304F\\
{\hiero \char"13050}&B1-B9  &Woman and her occupations &U+13050-13059\\
{\hiero \char"1305A} &C1-C24 &Anthropomorphic Deities &U+1305A-13075\\
{\hiero \char"13076} &D1-D67 &Parts of the Human Body &U+13076-130D1\\
{\hiero \char"130D2} &E1-E38 &Mammals &U+13076-130D1\\
{\hiero \char"130FE}  &F1-F53	&Parts of Mammals &U+130FE-1313E\\
{\hiero\char"1313F}	&G1-G54	&Birds &U+1313F-1317E\\
{\hiero \char"1317F}	&H1-H8	&Parts of Birds &U+1317F-13187\\
\texthiero{\char"13188}	&I1-I15	&Amphibious Animals, Reptiles, etc. &U+13188-1319A\\
\texthiero{\char"1319B}	&K1-K8	&Fishes and Parts of Fishes &U+1319B-131A2\\
\texthiero{\char"131A3}	&L1-L8	&Invertebrata and Lesser Animals &U+131A3-131AC\\
\texthiero{\char"131AD}	&M1-M44	&Trees and Plants &U+13AD-131EE\\
\texthiero{\char"131EF}	&N1-N42	&Sky, Earth, Water &U+131EF-1321F\\
\texthiero{\char"13250}	&O1-O51	&Buildings and Parts of Buildings &U+13250-1329A\\
\texthiero{\char"1329B}	&P1-P11	&Ships and Parts of Ships &U+1329B-132A7\\
\texthiero{\char"132A8}	&Q1-Q7	& Domestic and Funerary Furniture &U+132A8-132AE\\
\texthiero{\char"132AF}	&R1-R29	&Temple Furniture and Sacret Emblems &U+132AF-132D0\\
\texthiero{\char"132D1}	&S1-S46	&Crowns, Dress, Staves, etc. &U+132D1-13306\\
\texthiero{\char"13307}	&T1-T36	&Warfare, Hunting, Butchery &U+13307-13332\\
\texthiero{\char"13333}	&U1-42	&Agriculture, Crafts and Professions &U+13333-13361\\
\texthiero{\char"13362}	&V1-V40a	&Rope, Fibre, Baskets, Bags, etc. &U+13362-133AE\\
\texthiero{\char"133AF}	&W1-W25	&Vessels of Stone and Earthenware &U+133AF-133CE\\
\texthiero{\char"133CF}	&X1-X8a	&Loaves and Cakes &U+133CF-133DA\\
\texthiero{\char"133DB}	&Y1-Y8	&Writing, Games, Music &U+133DB-133E3\\
\texthiero{\char"133E4}	&Z1-Z16H	&Strokes, Geometrical Figures, etc. &U+133E4-1340C\\
\texthiero{\char"1340D}	&Aa1-Aa32	&Unclassified &U+1340D-1342E\\
\end{longtable}

I particularly like the crocodile sign \def\crocodile{\color{teal}{\Huge\texthiero{\char"13188}}} {\crocodile}, as it is applicable to describe people in my field of work. 

\begin{scriptexample}[]{Woman and her occupations}
\unicodetable{hiero}{"13050}
\end{scriptexample}

\section{Positioning}

One of the core assumptions of any hieroglyphic encoding or mark-up scheme following the MdC is that signs and groups of signs maybe positioned next to each other or above each other. The former is indicated by the operator * and the latter by :. One may also use -, which functions as * for horizontal texts and as : for vertical text. 

In some dialects of the MdC relative positioning has been extended by the use of the |&| operator. This is used to form a kind of ligature, such as |D&t| can be defined to represent the \textit{Cobra at rest} sign I10 with sign X1 underneath, as follows:

\begin{center}
{\hiero\HUGE
       \mbox{\rlap{\char"133CF}\char"13193\hfill\hfill}\\
       {\large|insert[bs](I10,X1)|}

\mbox{\rlap{\scalebox{0.5}{\char"133E3}}\char"13193\hfill\hfill}\\
 	
}
\end{center}

This is only a partial solution and to automate it via kerning tables, will require hundreds of entries in the kerning tables. It will also need constant modifications as researchers discover new combinations. A better approach and which is easily applied to \tex based systems would be to adopt Nederhof's method of creating a new command |insert[bs](I10,X1)|. 

In \tex one could simply define a command \cmd{\insert} with one optional argument to handle the positioning. The positioning uses the letters [b,t,s,e] to position the glyph. the letters s and e stand for start and end, whereas b,t for bottom and top respectively. When there are only two symbols involved, this is not such a difficult operation, but when three or more symbols are to be grouped and kerned together, inserting with some form of scaling is necessary.

\subsection{Enclosures}

Enclosures. The two principal names of the king, the \emph{nomen} and \emph{prenomen}, were normally
written inside a \emph{cartouche}: a pictographic representation of a coil of rope.

In the Unicode representation of hieroglyphic text, the beginning and end of the cartouche
are represented by separate paired characters, somewhat like parentheses. The Unicode manual states that `rendering of a full cartouche surrounding a name requires specialized layout software', which is of course an easy task for \tex.

\begin{macro}{\cartouche}
The commands \cmd{\cartouche} and \cmd{\cartouche}, from Peter Wilson's \pkgname{hierglyph} package have been used for many years to demonstrate the use of hieroglyphics with \latexe. 
\end{macro}

There are a several characters for these start and end cartouche characters, reflecting various styles for the enclosures.

\cartouche{{\hiero \char"13147}$sin^{2} x + cos^{2} x = 1$}
\Cartouche{{\hiero \char"13147}$sin^{2} x + cos^{2} x = 1$}

Unicode:{\hiero 𓇓𓏏𓊵𓏙𓊩𓁹𓏃𓋀𓅂𓊹𓉻𓎟𓍋𓈋𓃀𓊖𓏤𓄋𓈐𓎟𓇾𓈅𓏤𓂦𓈉 }

\textpmhg{\HQ} 

\cartouche{\pmglyph{K:l-i-o-p-a-d:r-a}}
%\translitpmhg{\HK\Hl\Hi\Ho\Hp\Ha\Hd\Hr\Ha}

\printunicodeblock{./languages/hieroglyphics.txt}{\hiero}
\printunicodeblock{./languages/hieroglyphics-13100.txt}{\hiero}
\printunicodeblock{./languages/hieroglyphics-13200.txt}{\hiero}
\printunicodeblock{./languages/hieroglyphics-13300.txt}{\hiero}
\printunicodeblock{./languages/hieroglyphics-13400.txt}{\hiero}
\section{Numerals}

Egyptian numbers are encoded following the same principles used for the
encoding of Aegean and Cuneiform numbers. Gardiner does not supply a full set of
numerals with catalog numbers in his Egyptian Grammar, but does describe the system of
numerals in detail, so that it is possible to deduce the required set of numeric characters.

Two conventions of representing Egyptian numerals are supported in the Unicode Standard.
The first relates to the way in which hieratic numerals are represented. Individual
signs for each of the 1s, the 10s, the 100s, the 1000s, and the 10,000s are encoded, because in
hieratic these are written as units, often quite distinct from the hieroglyphic shapes into
which they are transliterated. The other convention is based on the practice of the \emph{Manual
de Codage}, and is comprised of five basic text elements used to build up Egyptian numerals.
There is some overlap between these two systems.

%% Needs some work to get it into LuaLaTeX
%% omitted for the time being
%\ifxetex
%\begin{texexample}{TeXeXglyph}{ex:xetexglyph}
%\raggedright
%\font\myfont = "Aegyptus"
%\setcounter{glyphcount}{136}
%
%\whiledo
%{\value{glyphcount}<\XeTeXcountglyphs\myfont}
%{\arabic{glyphcount}:~
%{\myfont\XeTeXglyph\arabic{glyphcount}}\quad
%\stepcounter{glyphcount}}
%\end{texexample}
%\fi

\section{Input Methods}

If you writing a document with a lot of hieroglyphics inputting of hieroglyphics can be problematic. Most researchers in the field will use special keyboards or editors. They also use MS/Word or OpenOffice. They can both be coerced to produce reasonable documents, but with \tex obviously better results can be achieved. One such editor is \href{http://jsesh.qenherkhopeshef.org/}{jsesh}. 


\begin{luacode*}
    local h = {}
          h = dofile("hiero.lua")
    local options = {style="block",
                     echo=true,
                     direction="RL",
                     size = "\\Huge",
                     color = "green",
                     headings = "captionof{figure}"  -- section/tablecaption/figurecaption
                     }
   -- prints full symbol list
   h.printgardiner(t,options)

   tex.print("\\par")
   local options = {style="block",
                     echo=true,
                     heading="\\par",
                     direction="RL",
                     color = "teal",
                     scale = 8}

   h.printhierochar("hiero","1317D",options)
   h.printhierochar("hiero","13000",{direction="RL",
                                        color = "teal",
                                        scale = 8})
   h.printhierochar("hiero","13003",{direction="LR",
                                        color = "teal",
                                        scale = 1})
   h.parseMdC([[M23-X1-R4-X8-Q2-D4-W17-R14-G4-R8-O29-
               V30-U23-N26-D58-O49-Z1-F13-N31-V30-N16-
               N21-Z1-D45-N25!]])

   tex.print("\\par")
   h.printgardinercat("B")

\end{luacode*}

\newcommand\hierochar[2][direction = "LR",
                         color     = "teal",
                         scale     = 1]{% 
               \luaexec{
                h = h or {}
                h = require("hiero.lua")  
                h.parseMdC(#2,{#1})}}
               
\newcommand\printhierochar[3][direction = "LR",
                              color     = "teal",
                              scale     = 4]{% 
               \luaexec{
                h = h or {}
                h = require("hiero.lua")  
                h.printhierochar(#2,#3,{#1})}}

This file just tests the various commands available for manipulating hieroglyphics. We tried to 
generalize the commands, so they can be re-used for other type of hieroglyphics.

{
\hierochar{"A1-A2-A3!"}

\centering 

\def\options{direction = "LR",
             color     = "teal",
             scale     = 7}

\def\fontname{"hiero"}

\def\hierochar#1{\printhierochar[\options]{\fontname}{#1}}
}


\begin{scriptexample}[]{Some Example}
Sometimes kerning might be required, especially if the
glyphs are scaled.This is easily achieved with a \cmd{\kern}
command and a suitable skip dimension.

\medskip

\bgroup
\fboxsep=0pt\fboxsep.4pt
\def\options{direction = "RL",
             color     = "black!95",
             scale     = 5}
\centering

\color{teal}
\fbox{\hierochar{"13051"}}
\kern-4mm
\hierochar{"13003"}
\def\options{direction = "LR",
             color     = "black!95",
             scale     = 5}
\fbox{\hierochar{"13003"}}\color{red}
\kern-4mm
\hierochar{"13051"}
\color{black!95}
\egroup
\begin{verbatim}
\centering
\hierochar{"13051"}
\kern-4mm
\hierochar{"13003"}
\def\options{direction = "RL",
             color     = "black!95",
             scale     = 5}
\hierochar{"13003"}
\kern-4mm
\hierochar{"13051"}
\end{verbatim}
\end{scriptexample}

A bit of a diversion is appropriate at this point. Our attempt after the historical overview, is to provide some routines for the capturing and display of hieroglyphic texts using LuaTeX. This involves getting low level information from the system regarding fonts. 

\begin{figure}[ht]
\begin{minipage}{0.45\textwidth}
\centering
\includegraphics[width=0.6\textwidth]{./images/fontforge.jpg}
\end{minipage}
\begin{minipage}[t]{0.45\textwidth}
\caption{Viewing font information with fontforge.}
\end{minipage}
\end{figure}

For each glyph, we are interested to get its unicode number, the position in the font table, its name and most importantly the font metrics. The font metrics are a set of parameters that are used to measure the bounding box, any ascenders or descenders and similar information. Using fontforge, these parameters can easily be viewed. However, we are not interested to make any modifications manually; what we are interested is to programmatically obtain this information using Lua. Lua's philosophy and a mantra repeated often by the developers, is that it provides the tools and not the solutions. What this means to the LuaTeX programmer, is that we need to reach very low level  to get this information, which is a road with many bumps. Luckily the tools have been provided by the LuaTeX developers. This comes with a lot of benefits as we can also do our own on the fly mapping, such as creating an index table holding all the Gardiner numbers. 

The |fontloader.open| function loads a font, but it's not usable by itself; the result should be turned into a table with
\textbf{fontloader.to\_table}, as follows:

\begin{verbatim}
  local f = fontloader.open
     ("c:/windows/fonts/NotSansEgyptianHieroglyphics-
       Regulat.ttf")
  fonttable = fontloader.to_table(f)
  fontloader.close(f)
\end{verbatim}

We will use the Google No Tofu Egyptian Hieroglyphic font to experiment with our hieroglyphics. I have used a full path to load the font, which resides on my windows machine in the fonts folder. Once we load all the information in the |fonttable| we use |fontloader.close| to discard the userdata from which the table is extracted. 

What makes OpenType fonts special is that they describe every aspect that you might be able to think of when you think of putting letters together to form words. In addition to the obvious "this is what letters look like" information, OpenType fonts also specify things like the name of each letter that is available in the font, how much of the Unicode standard the font implements, which horizontal and vertical metrics apply to which letters, exactly how the letters are arranged inside the font so that they can quickly be read out, what kind of font classifications apply (is it a fantasy font? is it bold face? is it fixed width? etc), what kind of memory allocation a printer needs to perform in order to be able to even load the font, etc. etc. etc. All these are stored in tables upon tables, similat to a collection of Russian dolls.

To view the values in the fonttable, we will first iterate over the \textbf{fonttable} and extract all the first level keys.

\begin{texexample}{Iterating through a font table}{}
\begin{luacode*}
local z={}
tf=fontloader.to_table(fontloader.open("c:/windows/fonts/NotoSansEgyptianHieroglyphs-Regular.ttf"))

-- we sort the keys to create a table
-- important keys to us are tf.glyphs

for k,v in pairs (tf) do
   --tex.print(k.."\\par")
   table.insert(z, k)
end

table.sort(z)
tex.print("\\begin{multicols}{3}\\raggedright")
for k,v in pairs (z) do
   z[k] = string.gsub(z[k],"%_","\\textunderscore ")
   local s = tf[v]
   tex.print("\\textbullet\\hskip3pt\\hangindent2em " .. z[k].." [\\textit{"..type(s).."}] ","\\par")
end
tex.print("\\end{multicols}")
\end{luacode*}
\end{texexample}

We iterate through the \textbf{fonttable} using the Lua  "pair" iterator and we simply print all the keys and the type of the values in a human readable form as shown in the example. Note the use of |\textunderscore| that replaces all underscores in the fields with its text equivalent to sanitize the output. This is a quick and dirty way to avoid the use of catcodes. Many of the keys, bear intuitive names and are not difficult to discern: \textit{version}, \textit{copyright} and the like. Getting the type of Lua variables is important in order to use them for error trapping. When you attempt for example to print a nil value an error will occur.

Now that we have peeked under the font we will iterate and capture the information of interest, which we will put into another table with two keys \textbf{info}  and \textbf{metrics}. In the metrics file we will get the bounding box related metrics of each and every glyph in the font and save it, into our own table. 

\begin{texexample}{More Metrics}{}
  \begin{luacode*}
   tex.print("units per em = ", tf.units_per_em,"\\par")
   for i,j in ipairs (tf.glyphs[6].boundingbox) do
      tex.print("bounding box["..i.."]".." = ", j,"\\par")
   end 
   local w = (tf.glyphs[6].boundingbox[3]-tf.glyphs[6].boundingbox[1])/tf.units_per_em
   local h = tf.glyphs[6].boundingbox[4]/tf.units_per_em
   tex.print("glyph width = ", w,"em\\par")
   tex.print("glyph height = ", h,"em\\par")

-- presents a nicely typeset table 

local rep, write = string.rep, tex.print
function ExploreTable (tab, offset)
    offset = offset or ""
    for k, v in pairs (tab) do
        local newoffset = offset .. "\\mbox{.}"
        if type(v) == "table" then
           -- if k == "boundingbox" then write("BB") end
           write(offset..k .. " = \\{\\par ")
           ExploreTable(v, newoffset)
           write(offset..newoffset .. "\\}\\par")
         else
           write(offset..k .. " = "..tostring(v),"\\par")
         end
      end
end

write("\\par{\\ttfamily ")
ExploreTable(tf.glyphs[38],"\\mbox{.}")
write("}")
  \end{luacode*}
\end{texexample}

The OpenType fonts standard, provides for so much information that we will ignore most of the items and focus on only a few tables and fields. A small utility after Paul Isambert's article is necessary to enable us to view tables easily within this book,


\begin{texexample}{ExploreTable utility}{}
\begin{luacode*}
-- presents a nicely typeset table 

local rep, write = string.rep, tex.print
function ExploreTable (tab, offset)
    offset = offset or ""
    for k, v in pairs (tab) do
        local newoffset = offset .. "\\mbox{.}"
        if type(v) == "table" then
           -- if k == "boundingbox" then write("BB") end
           write(offset..k .. " = \\{\\par ")
           ExploreTable(v, newoffset)
           write(offset..newoffset .. "\\}\\par")
         else
           write(offset..k .. " = "..tostring(v),"\\par")
         end
      end
end

write("\\par{\\ttfamily ")
ExploreTable(tf.glyphs[38],"\\mbox{.}")
write("}")
  \end{luacode*}
\end{texexample}

A good utility also is |TTX| that will convert an OTF font to XML and back. This requires that you have python installed.\footnote{See some good guidelines as to how to install it at \url{http://www.glyphrstudio.com/ttx/}.} The utility uses python to do the conversion. The archive can be downloaded from \url{http://sourceforge.net/projects/fonttools/files/latest/download}. This is a three prong attack. You need to have python install, the numpy library and then the TTX package. The |TTX| program was written by the font designer Just van Rossum, brother of the creator of the Python language, Guido van Rossum. The tool converts TrueType into human-readable |XML| format. The most attractive feature of this tool is that it also perform the opposite operation that is create a TruType font from an |XML| file. The |XML| format makes the hierarchy of the format clearer. Since SVG fonts are also described in |XML| it becomes an easier task to convert an |SVG| font to a TrueType font. To convert |bar.ttf| into |bar.ttx| you simply write:

\begin{verbatim}
ttx bar.ttf
\end{verbatim}

Similarly for the opposite conversion, from |.ttx| to |.ttf|

\begin{verbatim}
ttx bar.ttx
\end{verbatim}

The generated ttx file is approximately ten times larger than the original |.ttf| file. The files generated are huge affairs and difficult to manage.The command line option |-l| prints a list of the tables in the font. |TTX| is indispensable in the ``humanization'' of TrueType fonts. The details of the tables and what each field represents are eloquently described in that indispensable book by Yannis Haralambous \textit{Fonts \& Encodings.} Although the book is now somewhat dated, it is still the best source of information on many esoteric topics related to fonts. 






\input{./languages/meroitic}
\chapter{Ugaritic}
\label{s:ugaritic}
\index{Ugaritic fonts>Noto Sans Ugaritic}
\index{Ugaritic}
\index{Akkadian}
\index{Unicode>Ugaritic}
\parindent1em
\newfontfamily\ugaritic{NotoSansUgaritic-Regular.ttf}

\section{Background}
Sometime between 1190-1185 bce, the houses of Ugarit were abandoned by their inhabitants, then pillaged and burned. If they were destroyed by the Sea Peoples we will never know for sure, although this is very likely. This catastrophe ended a history of almost 6000 years. Ugarit was never rebuild and the ruins were buried for centuries before they were discovered in 1929. 

\begin{figure}[htbp]
\centering
\includegraphics[width=\textwidth]{ugarit-excavations}
%http://www.persee.fr/docAsPDF/syria_0039-7946_1936_num_17_2_3887.pdf
\end{figure}

Merchants figure prominently in Ugarit’s archives. The citizens engaged in trade, and many foreign merchants were based in the state, for example from Cyprus exchanging copper ingots in the shape of ox hides. The presence of Minoan and Mycenaean pottery suggests Aegean contacts with the city. It was also the central storage place for grain supplies moving from the wheat plains of northern Syria to the Hittite court.

common defence system (§ 11.5.4.3). The abundance of Cypriot
pottery,173 the Cypro-Minoan texts found in Ugarit ( L i v e r a n i 1979a,
1322-3) as well as letters18 and administrative texts,19 are also witness
to relationhips between the two communities at both the cultural
and the commercial levels. 

Some Cypriots (ally, altyy, DLU, 33)
receive from the Ugaritian administration food and clothing,20 others
belong to the guild of craftsmen.21 On the other hand, from its structure
the administrative text KTU 4.102 = RS 11.857 seems to be
a list of prisoners of war, or of persons detained for some reason,
who come from Cyprus ( V i t a 1995a, 108). An unpublished letter
found in Ras Shamra in 1994, which reports the dispatch of an
emissary of the king of Cyprus to Ugarit to deal with the freeing of
Cypriots detained on Ugaritic soil,22 could support this hypothesis

The \idxlanguage{Ugaritic} language  is written in alphabetic cuneiform. This was an innovative blending of an alphabetic script (like \hyperref[s:hebrew]{Hebrew}) and cuneiform (like Akkadian). The development of alphabetic cuneiform seems to reflect a decline in the use of Akkadian as a \textit{lingua franca} and a transition to alphabetic scripts in the eastern Mediterranean. Ugaritic, as both a cuneiform and alphabetic script, bridges the cuneiform and alphabetic cultures of the ancient Near East.


\begin{figure}[hb]
\centering
\includegraphics[width=\textwidth]{ugaritic-first-tablet}
\caption{A list of offerings with the first tablet number (KTU 1.39 = RS 1.001; Photo: UGARIT - FORSCHUNG Archive)}
\end{figure}

The Ugaritic script is a cuneiform (wedge-shaped) abjad used from around either the fifteenth century BCE[1] or 1300 BCE[2] for Ugaritic, an extinct Northwest Semitic language, and discovered in Ugarit (modern Ras Shamra), Syria, in 1928. It has 30 letters. Other languages (particularly Hurrian) were occasionally written in the Ugaritic script in the area around Ugarit, although not elsewhere.


\section{Material Culture}

Excavations at Ugarit have yielded an abundance of objects of everyday life that we can deduce the every day life of its inhabitants in a higher level of detail than many other civilizations. Objects recovered include mirrors, combs, cooking and drinking utensils, pottery, gems. An interesting item is the clepsydra shown in Figure~\ref{fig:clepsidra} used as a shower head. The religion and cults is also well represented. This is not easy to use as an individual and it was probably used with the help of a servant.

The Ugarites were actively interacting in trading. 

\begin{figure}[htbp]
\includegraphics[width=\textwidth]{clepsidra}
\caption{“Clepsydra” or shower vase RS 81.509
1981, City Center, House E, room 1201. Latakia Museum
H 19.5 cm, Diameter (max.) 18 cm. Fine plain buff pottery with burnished surface. Jug with a large,
ovoid body. The opening is narrow, contracting to a small hole 1 cm in diameter. The bottom is
pierced with 22 small holes to form a strainer. The narrowness of the opening does not permit filling
by any means other than plunging the vase entirely into a large container full of water. It holds about
1 liter. The function of this sort of vase is obvious. The container remained full if the opening was
sealed with one’s thumb to prohibit the entrance of air; the liquid could not flow out through the
bottom. When the thumb was removed (allowing air to enter the jug), the water could flow out
through the bottom, creating a type of shower head.
This object matches the definition of a clepsydra mentioned by ancient authors (Hieron): in its
primary sense, the term clepsydra is not restricted to a measure of time. What we have here is an instrument
used for washing, like a shower in a bathing installation (or shower stall). This was an object
of everyday life, but only in a relatively refined context. This vase was found with other personal
funerary objects fallen from the upper floor of a house of medium status in the city center. Other examples
(e.g., RS 30.325) show that this was not an uncommon item in homes at Ugarit.\\
– Bib.: M. Yon, P. Lombard, and M. Renisio, in RSO III, 1987, p. 106, fig. 87; P. Lombard, ibid., pp. 351–57.}
\label{fig:clepsidra}
\end{figure}

Clay tablets written in Ugaritic provide the earliest evidence of both the North Semitic and South Semitic orders of the alphabet, which gave rise to the alphabetic orders of Arabic (starting with the earliest order of its abjad), the reduced Hebrew, and more distantly the Greek and Latin alphabets on the one hand, and of the Ge'ez alphabet on the other. Arabic and Old South Arabian are the only other Semitic alphabets which have letters for all or almost all of the 29 commonly reconstructed proto-Semitic consonant phonemes. 

According to Dietrich and Loretz in Handbook of Ugaritic Studies (ed. Watson and Wyatt, 1999): "The language they [the 30 signs] represented could be described as an idiom which in terms of content seemed to be comparable to Canaanite texts, but from a phonological perspective, however, was more like Arabic."
The script was written from left to right. Although cuneiform and pressed into clay, its symbols were unrelated to those of the Akkadian cuneiform.

\begin{scriptexample}[]{Ugaritic}
\unicodetable{ugaritic}{"10380,"10390}
\end{scriptexample}

{\let\aegean\arial
\printunicodeblock{./languages/ugaritic.txt}{\ugaritic}
}

\bgroup

\let\a\arial
\Large
\begin{longtable}[l]{%
>{\arial\large}r|
>{\ugaritic}c| 
>{\arial\large}c 
>{\arial\large}c 
>{\arial\large}c >{\arial\large}c
}

&\a Sign	&\a Trans.	&\a IPA	&\a Hebrew	&\a Arabic \\
\hline
\inc &𐎀	&ʾa	& ʔa	&א	&أ \\
\inc &𐎁	&b	& b	    &ב	&ب \\
\inc &𐎂	&g	&ɡ	&ג	&ج\\
\inc &𐎃	&ḫ	&x	&	&خ\\
\inc &𐎄	&d	&d	&ד	&د\\
\inc &𐎅	&h	&h	&ה	&ه\\
\inc &𐎆	&w	&w	&ו	&و\\
\inc &𐎇	&z	&z	&ז	&ز\\
\inc &𐎈	&ḥ	&ħ	&ח	&ح\\
\inc &𐎉	&ṭ	&t̴	&ט	&ط\\
\inc &𐎊	&y	&j	&י	&ي\\
\inc &𐎋	&k	&k	&כ	&ك\\
\inc &𐎌	&š	&ʃ	&ש	&ش\\
\inc &𐎍	&l	&l	&ל	&ل\\
\inc &𐎎	&m	&m	&מ	&م\\
\inc &𐎏	&ḏ	&ð	&	&ذ\\
\inc &𐎐	&n	&n	&נ	&ن\\
\inc &𐎑	&ẓ	&θ̴	&	&ظ\\
\inc &𐎒	&s	&s	&ס	&س\\
\inc &𐎓	&ʿ 	&ʕ	&ע	&ع\\
\inc &𐎔	&p	&p	&פ	&ف\\
\inc &𐎕	&ṣ	&s̴	&צ	&ص\\
\inc &𐎖	&q	&q	&ק	&ق\\
\inc &𐎗	&r	&r	&ר	&ر\\
\inc &𐎘	&ṯ	&θ	&	&ث\\
\inc &𐎙	&ġ	&ɣ	&	&غ\\
\inc &𐎚	&t	&t	&ת	&ت\\
\inc &𐎛	&ʾi	&ʔi	&	&ئ\\
\inc &𐎜	&ʾu	&ʔu	&	&ؤ\\
\end{longtable}
\egroup


\textit{\LARGE$$\stackrel{\mbox{ho}}{.}$$}

% Tranliteration macros 
% 
\bgroup\ugaritic
\def\a{\char"10380}
\def\b{\char"10381}
\def\g{\char"10382}
\LARGE \a \b \g 
\egroup

\section{Online Collections}

http://digital.library.stonybrook.edu/











\section{Sumero Akkadian Cuneiform}
\label{s:sumero}
\newfontfamily\sumero{NotoSansSumeroAkkadianCuneiform-Regular.ttf}
In Unicode, the Sumero-Akkadian Cuneiform script is covered in two blocks:
U+12000–U+1237F Cuneiform
U+12400–U+1247F Cuneiform Numbers and Punctuation
These blocks, in version 6.0, are in the Supplementary Multilingual Plane (SMP).

The sample glyphs in the chart file published by the Unicode Consortium[2] show the characters in their Classical Sumerian form (Early Dynastic period, mid 3rd millennium BCE). The characters as written during the 2nd and 1st millennia BCE, the era during which the vast majority of cuneiform texts were written, are considered font variants of the same characters.

The character set as published in version 5.2 has been criticized, mostly because of its treatment of a number of common characters as ligatures, omitting them from the encoding standard.

\begin{scriptexample}[]{Sumero Akkadian}
\unicodetable{sumero}{"12000,"12010,"12020,"12030,"12040,"12050,"12060,"12070,
"12080,"12090,"12400,"12410,"12420,"12430}
\end{scriptexample}

\begin{table}[b]
\begin{scriptexample}[]{textbox}
From Plato's dialogue Phaedrus 14, 274c-275b:

Socrates: [274c] I heard, then, that   in Egypt, was one of the ancient gods of that country, the one whose sacred bird is called the ibis, and the name of the god himself was Theuth. He it was who [274d] invented numbers and arithmetic and geometry and astronomy, also draughts and dice, and, most important of all, letters. 

Now the king of all Egypt at that time was the god Thamus, who lived in the great city of the upper region, which the Greeks call the Egyptian Thebes, and they call the god himself Ammon. To him came Theuth to show his inventions, saying that they ought to be imparted to the other Egyptians. But Thamus asked what use there was in each, and as Theuth enumerated their uses, expressed praise or blame, according as he approved [274e] or disapproved.  

"The story goes that Thamus said many things to Theuth in praise or blame of the various arts, which it would take too long to repeat; but when they came to the letters, [274e] “This invention, O king,” said Theuth, “will make the Egyptians wiser and will improve their memories; for it is an elixir of memory and wisdom that I have discovered.” But Thamus replied, “Most ingenious Theuth, one man has the ability to beget arts, but the ability to judge of their usefulness or harmfulness to their users belongs to another; [275a] and now you, who are the father of letters, have been led by your affection to ascribe to them a power the opposite of that which they really possess.  

"For this invention will produce forgetfulness in the minds of those who learn to use it, because they will not practice their memory. Their trust in writing, produced by external characters which are no part of themselves, will discourage the use of their own memory within them. You have invented an elixir not of memory, but of reminding; and you offer your pupils the appearance of wisdom, not true wisdom, for they will read many things without instruction and will therefore seem [275b] to know many things, when they are for the most part ignorant and hard to get along with, since they are not wise, but only appear wise." 
\end{scriptexample}
\end{table}


\printunicodeblock{./languages/cuneiform.txt}{\sumero}





\section{Inscriptional Parthian}
\label{s:parthian}
\index{Ancient and Historic Scripts>Inscriptional Parthian}
\index{Inscriptional Parthian fonts>Noto Sans Inscriptional Parthian}

The Parthian script developed from the Aramaic alphabet around the 2nd century BCE and was used during the Parthian and Sassanid periods of the Persian Empire. The latest known inscription dates from 292 CE. 

\newfontfamily\parthian{NotoSansInscriptionalParthian-Regular.ttf}
Inscriptional Parthian is a Unicode block containing characters of the official script of the Sassanid Empire.

\newenvironment{parthiannumbers}{^^A
\def\1{\parthian\char"10B58}
\def\2{\parthian\char"10B59}
\def\3{\text{\parthian\char"10B5A}}
\def\4{\text{\parthian\char"10B5B}}^^A 
\TextOrMath\4 \4
\TextOrMath\3 \3
}{}
\index{Parthian numbers}
\begin{scriptexample}[]{}

\unicodetable{parthian}{"10B40,"10B50}



\end{scriptexample}

Inscriptional Parthian has its own numbers, which have right-to-left
directionality. The numbers are built up out of 1, 2, 3, 4, 10, 20, 100, and 1000 which is not such a great scheme. The inscriptions are not
normalized uniformly. The units are sometimes written with strokes of the same height, or with a final
stroke that is longer, either descending or ascending to show the end of the number; compare 5 in 15 ({\parthian \char"10B59 \char"10B5B}
or 2 + 3) and in 45 (òõ or 1 + 4); compare 6 in 16 (öö or 3 + 3) and in 36 (òôö or 1 + 2 + 3). The
encoding here allows the specialist to choose his or her preferred representation. The following is an list
of numbers attested in Inscriptional Parthian. The third column is displayed in visual order.

The |phd| package offers rudimentary support for Parthian numbers in the form of an environment |parthiannumbers|, which can be used as follows:

\begin{texexample}{Inscriptional Parthian numbers}{parth}
\begin{parthiannumbers}
\1 $= 1$
\2 $= 2$
\begin{align*}
\3 &= 3\\
\4 &= 4\\
\3\4 &=7
\end{align*}
\end{parthiannumbers}
\end{texexample}



\printunicodeblock{./languages/inscriptional-parthian.txt}{\parthian}




\footnote{\url{http://www.unicode.org/L2/L2007/07207-n3286-parthian-pahlavi.pdf}} 


\section{Old Italic}

\epigraph{A society grows great when old men plant
trees in whose shade they know they will never sit.}{Greek proverb}
\label{s:olditalic}
\index{scripts>Old Italic}
\newfontfamily\olditalic{Noto Sans Old Italic}


Old Italic refers to any of several now extinct alphabet systems used on the Italian Peninsula in ancient times for various Indo-European languages (predominantly Italic) and non-Indo-European (e.g. Etruscan) languages. The alphabets derive from the Euboean Greek Cumaean alphabet, used at Ischia and Cumae in the Bay of Naples in the eighth century BC.

Various Indo-European languages belonging to the Italic branch (Faliscan and members of the Sabellian group, including Oscan, Umbrian, and South Picene, and other Indo-European branches such as Celtic, Venetic and Messapic) originally used the alphabet. Faliscan, Oscan, Umbrian, North Picene, and South Picene all derive from an Etruscan form of the alphabet.

\section{Etruscan}

Many peoples took the system that the Greeks had elaborated and
adapted it to their own language. This was particularly true in Lemnos and
in Etruria, where signs inspired by Greek letters were put to the service of
languages that probably were closely related to Greek—signs that we can
read without fully comprehending them. The Etruscans seem to have used
writing largely for religious purposes. According to Cicero (De divinatione)
they bequeathed their sacred texts to the Romans, who held the Etruscan
religion to be the religion of the Book par excellence.\cite{henri1994}

\begin{figure}[htbp]
\centering
\includegraphics[width=0.7\textwidth]{marsiliana}
\caption{The Marsiliana Tablet}
\end{figure}

The Germanic runic alphabet was derived from one of these alphabets by the 2nd century.


Old Italic is a Unicode block containing a unified repertoire of the three stylistic variants of pre-Roman Italic scripts.

\begin{scriptexample}[]{Testing}
\unicodetable{olditalic}{"10300,"10310,"10320}

{\leavevmode
\hfill\hfill\hfill\footnotesize Typeset with \texttt{Noto Sans Old Italic~}
}
\end{scriptexample}
\section{Old South Arabian}
\label{s:oldsoutharabian}

\index{Ancient and Historic Scripts>Old South Arabian}
\index{Old South Arabian fonts>Noto Sans Old South Arabian}
\index{alphabets>Yemeni}

\newfontfamily\oldsoutharabian{NotoSansOldSouthArabian-Regular.ttf}

The ancient Yemeni alphabet (Old South Arabian ms3nd; modern Arabic: {\arabicfont المُسنَد‎}  musnad) branched from the Proto-Sinaitic alphabet in about the 9th century BC. It was used for writing the Old South Arabian languages of the Sabaic, Qatabanic, Hadramautic, Minaic (or Madhabic), Himyaritic, and proto-Ge'ez (or proto-Ethiosemitic) in Dʿmt. The earliest inscriptions in the alphabet date to the 9th century BC in Akkele Guzay, Eritrea[3] and in the 10th century BC in Yemen. There are no vowels, instead using the \emph{mater lectionis} to mark them.

Its mature form was reached around 500 BC, and its use continued until the 6th century AD, including Old North Arabian inscriptions in variants of the alphabet, when it was displaced by the Arabic alphabet.[4] In Ethiopia and Eritrea it evolved later into the Ge'ez alphabet,[1][2] which, with added symbols throughout the centuries, has been used to write Amharic, Tigrinya and Tigre, as well as other languages (including various Semitic, Cushitic, and Nilo-Saharan languages).

It is usually written from right to left but can also be written from left to right. When written from left to right the characters are flipped horizontally (see the photo).
The spacing or separation between words is done with a vertical bar mark (\textbar).
Letters in words are not connected together.

Old South Arabian script does not implement any diacritical marks (dots, etc.), differing in this respect from the modern Arabic alphabet.

\begin{scriptexample}[]{South Arabian}
\unicodetable{oldsoutharabian}{"10A60,"10A70}
\end{scriptexample}

Support in \latexe is provided via Peter Wilson's package \pkgname{sarabian}\citep{sarabian}. The package provides all the |metafont| sources as well as transliteration commands and other utilities \seedocs{\SARAB}. The package is based on fonts developed originally by Alan Stanier of Essex University.

The package provides the commands \docAuxCmd{sarabfamily} that selects the South Arabian font family. The family name is \texttt{sarab}. Another command \docAuxCmd{textsarab}\meta{text} typesets \meta{text} in the South Arabian font. The package provides two ways of accessing
glyphs: (a) by \texttt{ASCII} character commands, and (b) via commands. These are illustrated in
Table~\ref{sarabian1} which is a modified version of that provided in the Comprehensive Symbols.



\def\SAtdu{\oldsoutharabian\char"10A77}

A comparison between  the unicode and the rendering (scaled 5) \pkgname{sarabian} is shown below.

\centerline{\scalebox{3}{\SAtdu} \scalebox{3}{\textsarab{\SAtd}}}

There is no real advantage in using unicode fonts, if all you interested is to write some South Arabian text for inscriptions. 

\begin{symtable}[SARAB]{\SARAB\ South Arabian Letters}
\index{South Arabian alphabet}
\index{alphabets>South Arabian}
\label{sarabian1}
\begin{tabular}{*4{ll@{\qquad}}ll}
\K[\textsarab{\SAa}]\SAa   & \K[\textsarab{\SAz}]\SAz   & \K[\textsarab{\SAm}]\SAm   & \K[\textsarab{\SAsd}]\SAsd & \K[\textsarab{\SAdb}]\SAdb \\
\K[\textsarab{\SAb}]\SAb   & \K[\textsarab{\SAhd}]\SAhd & \K[\textsarab{\SAn}]\SAn   & \K[\textsarab{\SAq}]\SAq   & \K[\textsarab{\SAtb}]\SAtb \\
\K[\textsarab{\SAg}]\SAg   & \K[\textsarab{\SAtd}]\SAtd & \K[\textsarab{\SAs}]\SAs   & \K[\textsarab{\SAr}]\SAr   & \K[\textsarab{\SAga}]\SAga \\
\K[\textsarab{\SAd}]\SAd   & \K[\textsarab{\SAy}]\SAy   & \K[\textsarab{\SAf}]\SAf   & \K[\textsarab{\SAsv}]\SAsv & \K[\textsarab{\SAzd}]\SAzd \\
\K[\textsarab{\SAh}]\SAh   & \K[\textsarab{\SAk}]\SAk   & \K[\textsarab{\SAlq}]\SAlq & \K[\textsarab{\SAt}]\SAt   & \K[\textsarab{\SAsa}]\SAsa \\
\K[\textsarab{\SAw}]\SAw   & \K[\textsarab{\SAl}]\SAl   & \K[\textsarab{\SAo}]\SAo   & \K[\textsarab{\SAhu}]\SAhu & \K[\textsarab{\SAdd}]\SAdd \\
\end{tabular}

\bigskip
\begin{tablenote}
  \usefontcmdmessage{\textsarab}{\sarabfamily}.  Single-character
  shortcuts are also supported: Both
  ``\verb+\textsarab{\SAb\SAk\SAn}+'' and ``\verb+\textsarab{bkn}+''
  produce ``\textsarab{bkn}'', for example.  \seedocs{\SARAB}.
\end{tablenote}
\end{symtable}



\section{Avestan script}
\label{s:avestan}
The Avestan alphabet is a writing system developed during Iran's Sassanid era (AD 226–651) to render the Avestan language.
As a side effect of its development, the script was also used for Pazend, a method of writing Middle Persian that was used primarily for the Zend commentaries on the texts of the Avesta. In the texts of Zoroastrian tradition, the alphabet is referred to as \emph{din dabireh} or \emph{din dabiri}, Middle Persian for "the religion's script".

The Avestan alphabet was replaced by the Arabic alphabet after Persia converted to Islam during the 7th century CE. 


Notable Features

The alphabet is written from right to left, in the same way as Syriac, Arabic and Hebrew.
See more at: \url{http://www.iranchamber.com/scripts/avestan_alphabet.php#sthash.ZRu7AkEb.dpuf}

\newfontfamily\avestan{NotoSansAvestan-Regular.ttf}



\begin{scriptexample}[]{Avestan}
\ifxetex\TeXXeTstate=1
\beginR\fi
\avestan\raggedleft
𐬄	
𐬅	
𐬆	
𐬇	
𐬈	
𐬉	
𐬊	
𐬋	
𐬌	
𐬍	
𐬎	
𐬏	
𐬐	
	
𐬒	
𐬓	
𐬔	
	
𐬖	
𐬗	
𐬘	
𐬙	
𐬚	
𐬛	
𐬜	
𐬝	
𐬞	
𐬟	
𐬠	
𐬡	
𐬢	
𐬣	
𐬤	
𐬥	
𐬦	
𐬧	
𐬨	
𐬩	
𐬪	
𐬫	
𐬬	
𐬭	
𐬮	
𐬯	
𐬰	
𐬱	
𐬲	
𐬳	
𐬴	
𐬵	
\ifxetex\endR
\TeXXeTstate=0\fi
\end{scriptexample}

The recent Google font \url{NotoSansAvestan-Regular_0.ttf} can be used to typeset the Avestan script, but really not suitable for any serious study of the language.
\subsection{Old Turkic}

\newfontfamily\oldturkic{Segoe UI Symbol}
\begin{scriptexample}[]{Old Turkish}
\oldturkic
\obeylines
Orkhon	Yenisei
variants	Transliteration / transcription
Old Turkic letter  𐰀	𐰁 𐰂	a, ä
Old Turkic letter  𐰃	𐰄 𐰅	y, i (e)
Old Turkic letter  𐰆		o, u
Old Turkic letter  𐰇	𐰈	ö, ü

	0	1	2	3	4	5	6	7	8	9	A	B	C	D	E	F
U+10C0x	𐰀	𐰁	𐰂	𐰃	𐰄	𐰅	𐰆	𐰇	𐰈	𐰉	𐰊	𐰋	𐰌	𐰍	𐰎	𐰏
U+10C1x	𐰐	𐰑	𐰒	𐰓	𐰔	𐰕	𐰖	𐰗	𐰘	𐰙	𐰚	𐰛	𐰜	𐰝	𐰞	𐰟
U+10C2x	𐰠	𐰡	𐰢	𐰣	𐰤	𐰥	𐰦	𐰧	𐰨	𐰩	𐰪	𐰫	𐰬	𐰭	𐰮	𐰯
U+10C3x	𐰰	𐰱	𐰲	𐰳	𐰴	𐰵	𐰶	𐰷	𐰸	𐰹	𐰺	𐰻	𐰼	𐰽	𐰾	𐰿
U+10C4x	𐱀	𐱁	𐱂	𐱃	𐱄	𐱅	𐱆	𐱇	𐱈	

\hfill  Typeset with \texttt{Segoe UI Symbol} \cmd{\oldturkic} 
\end{scriptexample}

Irk Bitig or Irq Bitig (Old Turkic: {\bfseries\Large\oldturkic 𐰃𐰺𐰴 𐰋𐰃𐱅𐰃𐰏}), known as the Book of Omens or Book of Divination in English, is a 9th-century manuscript book on divination that was discovered in the "Library Cave" of the Mogao Caves in Dunhuang, China, by Aurel Stein in 1907, and is now in the collection of the British Library in London, England. The book is written in Old Turkic using the Old Turkic script (also known as "Orkhon" or "Turkic runes"); it is the only known complete manuscript text written in the Old Turkic script.[1] It is also an important source for early Turkic mythology.

The Old Turkic text does not have any sentence punctuation, but uses two black lines in a red circle as a word separation mark in order to indicate word boundaries as shown in Figure~{\ref{omen}}

\begin{figure}[htb]
\includegraphics[width=0.7\textwidth]{./images/omen.jpg}
\caption{Omen 11 (4-4-3 dice) of the Irk Bitig (folio 13a): "There comes a messenger on a yellow horse (and) an envoy on a dark brown horse, bringing good tidings, it says. Know thus: (The omen) is extremely good."}
\label{omen}
\end{figure}
\section{Runic}
\label{s:runic}
\newfontfamily\runic{NotoSansRunic-Regular.ttf}

Runes (Proto-Norse:{\runic ᚱᚢᚾᛟ }(runo), Old Norse: rún) are the letters in a set of related alphabets known as runic alphabets, which were used to write various Germanic languages before the adoption of the Latin alphabet and for specialised purposes thereafter. The Scandinavian variants are also known as futhark or fuþark (derived from their first six letters of the alphabet: F, U, Þ, A, R, and K); the Anglo-Saxon variant is futhorc or fuþorc (due to sound changes undergone in Old English by the names of those six letters)

\begin{scriptexample}[]{Runic}
 \unicodetable{runic}{"16A0,"16B0,"16C0,"16D0,"16E0,"16F0}
\end{scriptexample}


\printunicodeblock{./languages/runic.txt}{\runic}


\ifscriptolmec
  \section{Epi-Olmec}
\label{s:olmec}
Epi-Olmec is an ancient Mesoamerican logosyllabic script which has been deciphered by Terrence Kaufman and John Justeson. A complete description of the script has been described by \cite{kaufman}. The most famous inscription is on the Tuxtla Statuette. The Tuxtla Statuette is a small 6.3 inch (16 cm) rounded greenstone figurine, carved to resemble a squat, bullet-shaped human with a duck-like bill and wings. Most researchers believe the statuette represents a shaman wearing a bird mask and bird cloak.[1] It is incised with 75 glyphs of the Epi-Olmec or Isthmian script, one of the few extant examples of this very early Mesoamerican writing system. The Tuxtla Statuette is particularly notable in that its glyphs include the Mesoamerican Long Count calendar date of March 162 CE, which in 1902 was the oldest Long Count date discovered. A product of the final century of the Epi-Olmec culture, the statuette is from the same region and period as La Mojarra Stela 1 and may refer to the same events or persons.[3] Similarities between the Tuxtla Statuette and Cerro de las Mesas Monument 5, a boulder carved to represent a semi-nude figure with a duckbill-like buccal mask, have also been noted.[4]

\begin{figure}[ht]
\centering
\includegraphics[height=0.35\textheight]{./images/tuxtla-statuette.png}\hspace{1em} 
\includegraphics[height=0.35\textheight]{./images/tuxtla-statuette-01.jpg}
\caption{Frontal view of the Tuxtla Statuette. Note the Mesoamerican Long Count calendar date of March 162 CE (8.6.2.4.17) down the front of the statuette. The left figure is from wikipedia and the right from the original \protect\href{http://www.readcube.com/articles/10.1525/aa.1907.9.4.02a00030}{Holmes} paper.}
\end{figure}

\subsection{The epiolmec package}

The script has not been as yet encoded as by the Unicode consortium. Syropoulos \citep{syropoulos} created a font for the script and also wrote an article for TUGboat. Interestingly the paper describes the procedure used to develop the font. The package \pkgname{epiolmec} which is available both in \TeX live and Mik\tex, provides commands to access the glyphs. It is also possibly easier to typeset the script using traditional \latexe techniques, as they provide transcription commands rather than using a unicode font with the glyphs allocated in the private area directly.

\begin{verbatim}
\documentclass{article}
\usepackage{epiolmec,multicol}
\begin{document}
  \begin{center}
      \begin{minipage}{80pt}
      \begin{multicols}{3}
         \EOku\\ \EOji\\  
         \EOtze\\ \EOstep \\
       \end{multicols}    
     \end{minipage}       
  \end{center}
\end{document}
\end{verbatim}

Since the Epi-Olmec script is a logosyllagraphy we
need some practical way to access the symbols of the
script. Originally Syropoulos used the Ω translation
process that mapped words and “syllables” to the
corresponding glyphs of the font. In this way one obtains
a natural way for typing in Epi-Olmec texts. In addition,
in order to avoid the problem mentioned above,
he used a wrapper that typesets the text vertically.
For short texts \cmd{\shortstacks} is adequate, while
for longer texts, he used a |multicols| environment
inside a relatively narrow minipage. 

\begin{scriptexample}{Epi-Olmec}
\bgroup
\HUGE
\centering
\EOpi   \EOofficerI \EOofficerII \EOofficerIII

\captionof{figure}{The output of \string\EOpi, \string\EOofficerI, \string\EOofficerII\ and \string\EOofficerIII\ commands. }
\egroup
\end{scriptexample}

\subsection{Numbering System}\index{Epi-Olmec>vigesimal system}

The Epi-Olmec people used the same numbering system  
 as the Maya. Their numbering system was a vigesimal system and
 the digits were written in a top-down fashion. Thus, we need a macro
 that will typeset numbers in this fashion when it is used with \LaTeX\
 (actually $\epsilon$-\LaTeX). In addition, we need a macro that will
 just output the vigesimal digits. Such a macro could be used with
 $\Lambda$ with the |LTL| text and paragraph directions. To recapitulate,
 we need to define two macros that will basically typeset vigesimal numbers
 in either horizontal or vertical mode.

 For the various calculations that are performed, we need at least three
 counter variables. The fourth is needed for the macro that typesets the
 vigesimal numbers vertically and its usage is explained below. 

\begin{scriptexample}{EpicOtmec}
\def\textb#1{\text{\makebox[6em]{\hss#1~~   \protect\string#1\hfill}}}
\begin{multicols}{3}
\bgroup
\parindent0pt
$\textb{\EOzero}=0$\\
$\textb{\EOi} = 1$\\
$\textb{\EOii} = 2$\\
$\textb{\EOiii} =3$\\
$\textb{\EOiv}  =4$\\
$\textb{\EOv}   =5$\\
$\textb{\EOvi}  =6$\\
$\textb{\EOvii} =7$\\
$\textb{\EOviii} =8$\\
$\textb{\EOix} =9$\\ 
$\textb{\EOx} =10$\\
$\textb{\EOxi} =11$\\
$\textb\EOxii =12$\\
$\textb{\EOxiii} =13$\\
$\textb{\EOxiv} =14$\\
$\textb{\EOxv} =15$\\
$\textb{\EOxvi} =16$\\
$\textb\EOxvii =17$\\
$\textb{\EOxviii} =18$\\
$\textb{\EOxix} =19$\\
$\textb{\EOxx} =20$\\
\egroup
\end{multicols}
\end{scriptexample}


%% TODO add to index all symbols

\begin{multicols}{4}
\bgroup
\def\K#1{\makebox[3em]{{\color{blue}\hss#1\hfill}} \string#1}
\parindent0pt
\K\EOSpan\\ 
\K\EOJI \\
\K\EOvarji\\ 
\K\EOvarki \\
\K\EOpi \\
\K\EOpe \\
\K\EOpuu \\
\K\EOpa \\
\K\EOvarpa\\ 
\K\EOpu \\
\K\EOpo \\
\K\EOti \\
\K\EOte \\
\K\EOtuu \\
\K\EOta \\
\K\EOtu \\
\K\EOto \\
\K\EOtzi \\
\K\EOtze \\
\K\EOtzuu \\
\K\EOtza \\
\K\EOvartza\\ 
\K\EOtzu \\
\K\EOki \\
\K\EOke \\
\K\EOkuu \\
\K\EOvarkuu\\ 
\K\EOku\\ 
\K\EOko \\
\K\EOSi \\
\K\EOvarSi\\ 
\K\EOSuu \\
\K\EOSa \\
\K\EOSu \\
\K\EOSo \\
\K\EOsi \\
\K\EOvarsi\\ 
\K\EOsuu \\
\K\EOsa \\
\K\EOsu \\
\K\EOji \\
\K\EOje \\
\K\EOja \\
\K\EOvarja\\ 
\K\EOju \\
\K\EOjo \\
\K\EOmi \\
\K\EOme \\
\K\EOmuu \\
\K\EOma \\
\K\EOni \\
\K\EOvarni\\
\K\EOne \\
\K\EOnuu \\
\K\EOna \\
\K\EOnu \\
\K\EOwi \\
\K\EOwe \\
\K\EOwuu \\
\K\EOvarwuu\\
\K \EOwa\\
\K\EOwo \\
\K\EOye \\
\K\EOyuu \\
\K\EOya \\
\K\EOkak \\
\K\EOpak \\
\K\EOpuuk\\
\K\EOyaj \\
\K\EOScorpius\\
\K\EODealWith\\
\K\EOYear \\
\K\EOBeardMask \\
\K\EOBlood \\
\K\EOBundle \\
\K\EOChop \\
\K\EOCloth \\
\K\EOSaw \\
\K\EOGuise \\
\K\EOofficerI\\
\K\EOofficerII \\
\K\EOofficerIII \\
\K\EOofficerIV \\
\K\EOKing \\
\K\EOloinCloth \\
\K\EOlongLipII \\
\K\EOLose \\
\K\EOmexNew \\
\K\EOMiddle \\
\K\EOPlant \\
\K\EOPlay \\
\K\EOPrince \\
\K\EOSky \\
\K\EOskyPillar \\
\K\EOSprinkle \\
\K\EOstarWarrior\\
\K\EOTitleII \\
\K\EOtuki \\
\K\EOtzetze\\
\K\EOChronI \\
\K\EOPatron \\
\K\EOandThen\\
\K\EOAppear \\
\K\EODeer \\
\K\EOeat \\
\K\EOPatronII \\
\K\EOPierce \\
\K\EOkij \\
\K\EOstar  \\
\K\EOsnake \\
\K\EOtime \\
\K\EOtukpa  \\
\K\EOflint \\
\K\EOafter \\
\K\EOvarBeardMask \\
\K\EOBedeck \\
\K\EObrace \\
\K\EOflower  \\
\K\EOGod \\
\K\EOMountain \\
\K\EOgovernor \\
\K\EOHallow \\
\K\EOjaguar \\
\K\EOSini \\
\K\EOknottedCloth \\
\K\EOknottedClothStraps \\
\K\EOLord \\
\K\EOmacaw \\
\K\EOmonster \\
\K\EOmacawI \\
\K\EOskyAnimal\\
\K\EOnow \\
\K\EOTitleIV \\
\K\EOpenis \\
\K\EOpriest  \\
\K\EOstep\\
\K\EOsing \\
\K\EOskin \\
\K\EOStarWarrior \\
\K\EOsun \\
\K\EOthrone\\
\K\EOTime \\
\K\EOHallow \\
\K\EOTitle \\
\K\EOturtle \\
\K\EOundef \\
\K\EOGoUp \\
\K\EOLetBlood \\
\K\EORain \\
\K\EOset \\
\K\EOvarYear\\
\K\EOFold \\
\K\EOsacrifice \\
\K\EObuilding \\
\egroup
\end{multicols} 

\subsection{Technical}

The font is defined with the local encoding \texttt{LEO}. 

\begin{verbatim}
\DeclareFontEncoding{LEO}{}{}
\DeclareFontSubstitution{LEO}{cmr}{m}{n}
\DeclareFontFamily{LEO}{cmr}{\hyphenchar\font=-1}
\end{verbatim}

Note the |\hyphenchar\font=-1| that disables hyphenation in the |\DeclareFontFamily|  declaration. You cannot behead the \EOofficerII\ in order to hyphenate the text!



\fi

%\newfontfamily\aegyptus{AegyptusR.ttf}

\chapter{Aegyptian Hieroglyphics}

\index{fonts>Aegyptus}\index{Aegyptus (font)}
\index{fonts>Hieroglyphics}\index{languages>hieroglyphics}

\newfontfamily\hiero{NotoSansEgyptianHieroglyphs-Regular.ttf}

Hieroglyphic writing appeared in Egypt at the end of the fourth millennium bce. The writing
system is pictographic: the glyphs represent tangible objects, most of which modern
scholars have been able to identify. A great many of the pictographs are easily recognizable
even by nonspecialists. Egyptian hieroglyphs represent people and animals, parts of the
bodies of people and animals, clothing, tools, vessels, and so on.

Hieroglyphs were used to write Egyptian for more than 3,000 years, retaining characteristic
features such as use of color and detail in the more elaborated expositions. Throughout the
Old Kingdom, the Middle Kingdom, and the New Kingdom, between 700 and 1,000 hieroglyphs
were in regular use. During the Greco-Roman period, the number of variants, as
distinguished by some modern scholars, grew to somewhere between 6,000 and 8,000.

Hieroglyphs were carved in stone, painted on frescoes, and could also be written with a reed
stylus, though this cursive writing eventually became standardized in what is called \emph{hieratic}
writing. Unicode does not encode the hieratic forms separately, but ust considers them as cursive forms of the hieroglyphs encoded block.

The Demotic script and then later the Coptic script replaced the earlier hieroglyphic and
hieratic forms for much practical writing of Egyptian, but hieroglyphs and hieratic continued
in use until the fourth century ce. An inscription dated August 24, 394 ce has been
found on the Gateway of Hadrian in the temple complex at Philae; this is thought to be
among the latest examples of Ancient Egyptian writing in hieroglyphs

\begin{figure}[htb]
\includegraphics[width=\textwidth]{./images/bookofthedead.jpg}
\end{figure}

In hieroglyphic texts, these drawings are not only simply arranged in sequential order, but also grouped on top of and next to each other. This rather complicates matters trying to register and reproduce hieroglyphic texts using a computer.

\section{Computer Typesetting}

Typesetting hieroglyphics with computers presents a number of problems. First is the method of inputting the characters and second the various methods required to stack hieroglyphics, the direction of writing which can be one of four different directions.

When the first computers were introduced in Egyptology in the late 1970s and the beginning of the 1980s, the graphical capacity of the machines was still in its infancy. Early attempts to register the hieroglyphic pictorial writing on computer therefore chose an encoding system to do this, using alphanumeric codes to represent or replace the graphics. To prevent many people from reinventing the wheel, during the first "Table Ronde Informatique et Egyptologie" in 1984 a committee was charged with the task to develop a uniform system for the encoding of hieroglyphic texts on computer. The resulting Manual for the Encoding of Hieroglyphic Texts for Computer-input (Jan Buurman, Nicolas Grimal, Jochen Hallof, Michael Hainsworth and Dirk van der Plas, Informatique et Egyptologie 2, Paris 1988), simply called Manuel de Codage, presents an easy to use and intuitive way of encoding hieroglyphic writing as well as the abbreviated hieroglyphic transcription (transliteration). The system proposed by the Manuel de Codage has since been adopted by international Egyptology as the official common standard for registering hieroglyphic texts on computer. Mark-Jan Nederhof proposed an enhanced encoding scheme to remove many of the limitations in the Manuel de Codage.

\pkgname{HieroTeX} is a \latexe package developed by to typeset hieroglyphic texts and still works well. The advantages of using \tex is of course its excellent typesetting capabilities and the usage of macros. Although inputting the texts as MdC codes is not that difficult, repeating the same codes over and over can be avoided with easily constructed simple substitution macros. 

\subsection{fonts}

One of the best fonts I came across is \idxfont{Aegyptus} from \url{http://users.teilar.gr/~g1951d/}\footnote{The site also has fonts for Aegean Numbers, Ancient Greek Musical Notation, Ancient Greek Numbers, Ancient Roman Symbols, Arkalochori Axe, Carian, Cypriot Syllabary, Dispilio tablet, Linear A, Linear B Ideograms, Linear B Syllabary, Lycian, Lydian, Old Italic, Old Persian, Phaistos Disc, Phoenician, Phrygian, Sidetic, Troy vessels’ signs and Ugaritic. Cretan Hieroglyphs and Cypro-Minoan script(s) are offered in separate files.}. The font provides all the unicode characters and also offers an additional number of glyphs that are not in the Unicode standard. The font uses the Unicode Private Use Areas to encode the glyphs. 

Another font is the Noto Egyptian Hieroglyphics from Google. This is a lightweight font with the symbols in their proper unicode slots. Mark-Jan Nederhof's \idxfont{NewGardiner} font is another one with support only for the Gardiner set. The codepoint mappings are incorrect, as the font has been  
encoded to EGPZ. The font is similar to the Aegyptus font, however it is just transposed and not recommended unless it is transposed. 

The editor software JSesh\footnote{\protect\url{http://jsesh.qenherkhopeshef.org/}} also provides a free font |JSeshFont.ttf|. This offers a correctly mapped unicode and is another good alternative. The symbols are drawn somewhat simpler and is just a matter of taste what you want to use.

My recommendation is for short demonstration purposes, the Noto font is to be preferred while for more serious work the Aegyptus font will be more useful. Using Lua the font can be transposed automatically to allow the use of commands that refer to unicode numbers. Another advantage of the Aegyptus font is that the glyphs are named with their Gardiner numbers, so it is somewhat easier to programmatically access them by name.\footnote{Unicode does not name the glyphs, but simply calls the Egyptian Hieroglyph $n$. } 

\medskip

\ifxetex
\bgroup
\centering 
\font\myfont = "Aegyptus"
\scalebox{7}{\myfont\XeTeXglyph 201}
\scalebox{7}{\myfont\XeTeXglyph 203}
\scalebox{7}{\myfont\XeTeXglyph 163}
\scalebox{7}{\myfont\XeTeXglyph 164}
\scalebox{7}{\myfont\XeTeXglyph 165}
\scalebox{7}{\myfont\XeTeXglyph 168}
\captionof{table}{Example of Egyptian Hieroglyphics typeset with the \textit{Aegyptus} font.} 
\egroup
\fi

\ifluatex
\bgroup
\centering 
\aegyptus
\scalebox{7}{\char"F300C}
\scalebox{7}{\char"F3001}
\scalebox{7}{\char"F3010}
\scalebox{7}{\char"F308B}
\scalebox{7}{\char"F3097}
\scalebox{7}{\char"F3091}
\captionof{table}{Example of Egyptian Hieroglyphics typeset with the \textit{Aegyptus} font.} 
\egroup

\fi


\subsection{Unicode Block}

Egyptian hieroglyphs is a Unicode block containing the Gardiner's sign list of Egyptian hieroglyphics.
The code points, in the range |0x13000| to |0x1342E|, are available starting from
\href{http://unicode.org/charts/PDF/U13000.pdf}{Unicode 5.2}

\begin{scriptexample}[]{Hieroglyphic}
\bgroup
\unicodetable{hiero}{"13000,"13010,"13020,"13030,"13040,"13050,"13060,"13070,%
"13080,%
"13090,"130A0,"130B0,"130C0,"130D0,"130E0,"130F0,%
"13100,"13110,"13120,"13130,"13140,"13150,"13060,"13070,"13080,"13090}
\egroup
\end{scriptexample}

\subsection{Gardiner's classification}

The standard reference on Egyptian hieroglyphics is Gartiner's Sign List, which lists common Egyptian hieroglyphs. These are grouped in categories from A-Aa. Each category represents a theme for example category A, is "man and his occupations". Based on this list ``Queen with flower" is denoted as \texttt{B7}. 

\subsubsection{Character Names} 

Egyptian hieroglyphic characters have traditionally been designated in
several ways:

\begin{enumerate}
\item  By complex description of the pictographs: \texttt{GOD WITH HEAD OF IBIS}, and so forth.
\item By standardized sign number: C3, E34, G16, G17, G24.
\item For a minority of characters, by transliterated sound.
\end{enumerate}

The characters in the Unicode Standard make use of the standard Egyptological catalog
numbers for the signs. Thus, the name for {\hiero\char"130F9} |U+13049| egyptian hieroglyph e034 refers
uniquely and unambiguously to the Gardiner list sign E34, described as a “{\aegean DESERT HARE}” ({\hiero \char"130FA}) and used for the sound “wn”. The Unicode catalog values are padded to three places with
zeros, so where the Gardiner classification is shown as \texttt{E34}, the unicode value is \texttt{E034}. 

Names for hieroglyphic characters identified explicitly in Gardiner 1953 or other sources as
variants for other hieroglyphic characters are given names by appending “A”, “B”, ... to the sign number. In the sources these are often identified using asterisks. Thus Gardiner’s G7,
G7*, and G7** correspond to U+13146 egyptian sign g007 {\hiero \char"13147}, U+13147 egyptian sign g007a, and U+13148 egyptian sign g007b, respectively.

\def\texthiero#1{{\color{black!95}\hiero #1}}

\begin{longtable}{>{\Large}lll>{\ttfamily}l}
{\hiero \char"13000}&A1-A70 & Man and his occupations &U+13000-1304F\\
{\hiero \char"13050}&B1-B9  &Woman and her occupations &U+13050-13059\\
{\hiero \char"1305A} &C1-C24 &Anthropomorphic Deities &U+1305A-13075\\
{\hiero \char"13076} &D1-D67 &Parts of the Human Body &U+13076-130D1\\
{\hiero \char"130D2} &E1-E38 &Mammals &U+13076-130D1\\
{\hiero \char"130FE}  &F1-F53	&Parts of Mammals &U+130FE-1313E\\
{\hiero\char"1313F}	&G1-G54	&Birds &U+1313F-1317E\\
{\hiero \char"1317F}	&H1-H8	&Parts of Birds &U+1317F-13187\\
\texthiero{\char"13188}	&I1-I15	&Amphibious Animals, Reptiles, etc. &U+13188-1319A\\
\texthiero{\char"1319B}	&K1-K8	&Fishes and Parts of Fishes &U+1319B-131A2\\
\texthiero{\char"131A3}	&L1-L8	&Invertebrata and Lesser Animals &U+131A3-131AC\\
\texthiero{\char"131AD}	&M1-M44	&Trees and Plants &U+13AD-131EE\\
\texthiero{\char"131EF}	&N1-N42	&Sky, Earth, Water &U+131EF-1321F\\
\texthiero{\char"13250}	&O1-O51	&Buildings and Parts of Buildings &U+13250-1329A\\
\texthiero{\char"1329B}	&P1-P11	&Ships and Parts of Ships &U+1329B-132A7\\
\texthiero{\char"132A8}	&Q1-Q7	& Domestic and Funerary Furniture &U+132A8-132AE\\
\texthiero{\char"132AF}	&R1-R29	&Temple Furniture and Sacret Emblems &U+132AF-132D0\\
\texthiero{\char"132D1}	&S1-S46	&Crowns, Dress, Staves, etc. &U+132D1-13306\\
\texthiero{\char"13307}	&T1-T36	&Warfare, Hunting, Butchery &U+13307-13332\\
\texthiero{\char"13333}	&U1-42	&Agriculture, Crafts and Professions &U+13333-13361\\
\texthiero{\char"13362}	&V1-V40a	&Rope, Fibre, Baskets, Bags, etc. &U+13362-133AE\\
\texthiero{\char"133AF}	&W1-W25	&Vessels of Stone and Earthenware &U+133AF-133CE\\
\texthiero{\char"133CF}	&X1-X8a	&Loaves and Cakes &U+133CF-133DA\\
\texthiero{\char"133DB}	&Y1-Y8	&Writing, Games, Music &U+133DB-133E3\\
\texthiero{\char"133E4}	&Z1-Z16H	&Strokes, Geometrical Figures, etc. &U+133E4-1340C\\
\texthiero{\char"1340D}	&Aa1-Aa32	&Unclassified &U+1340D-1342E\\
\end{longtable}

I particularly like the crocodile sign \def\crocodile{\color{teal}{\Huge\texthiero{\char"13188}}} {\crocodile}, as it is applicable to describe people in my field of work. 

\begin{scriptexample}[]{Woman and her occupations}
\unicodetable{hiero}{"13050}
\end{scriptexample}

\section{Positioning}

One of the core assumptions of any hieroglyphic encoding or mark-up scheme following the MdC is that signs and groups of signs maybe positioned next to each other or above each other. The former is indicated by the operator * and the latter by :. One may also use -, which functions as * for horizontal texts and as : for vertical text. 

In some dialects of the MdC relative positioning has been extended by the use of the |&| operator. This is used to form a kind of ligature, such as |D&t| can be defined to represent the \textit{Cobra at rest} sign I10 with sign X1 underneath, as follows:

\begin{center}
{\hiero\HUGE
       \mbox{\rlap{\char"133CF}\char"13193\hfill\hfill}\\
       {\large|insert[bs](I10,X1)|}

\mbox{\rlap{\scalebox{0.5}{\char"133E3}}\char"13193\hfill\hfill}\\
 	
}
\end{center}

This is only a partial solution and to automate it via kerning tables, will require hundreds of entries in the kerning tables. It will also need constant modifications as researchers discover new combinations. A better approach and which is easily applied to \tex based systems would be to adopt Nederhof's method of creating a new command |insert[bs](I10,X1)|. 

In \tex one could simply define a command \cmd{\insert} with one optional argument to handle the positioning. The positioning uses the letters [b,t,s,e] to position the glyph. the letters s and e stand for start and end, whereas b,t for bottom and top respectively. When there are only two symbols involved, this is not such a difficult operation, but when three or more symbols are to be grouped and kerned together, inserting with some form of scaling is necessary.

\subsection{Enclosures}

Enclosures. The two principal names of the king, the \emph{nomen} and \emph{prenomen}, were normally
written inside a \emph{cartouche}: a pictographic representation of a coil of rope.

In the Unicode representation of hieroglyphic text, the beginning and end of the cartouche
are represented by separate paired characters, somewhat like parentheses. The Unicode manual states that `rendering of a full cartouche surrounding a name requires specialized layout software', which is of course an easy task for \tex.

\begin{macro}{\cartouche}
The commands \cmd{\cartouche} and \cmd{\cartouche}, from Peter Wilson's \pkgname{hierglyph} package have been used for many years to demonstrate the use of hieroglyphics with \latexe. 
\end{macro}

There are a several characters for these start and end cartouche characters, reflecting various styles for the enclosures.

\cartouche{{\hiero \char"13147}$sin^{2} x + cos^{2} x = 1$}
\Cartouche{{\hiero \char"13147}$sin^{2} x + cos^{2} x = 1$}

Unicode:{\hiero 𓇓𓏏𓊵𓏙𓊩𓁹𓏃𓋀𓅂𓊹𓉻𓎟𓍋𓈋𓃀𓊖𓏤𓄋𓈐𓎟𓇾𓈅𓏤𓂦𓈉 }

\textpmhg{\HQ} 

\cartouche{\pmglyph{K:l-i-o-p-a-d:r-a}}
%\translitpmhg{\HK\Hl\Hi\Ho\Hp\Ha\Hd\Hr\Ha}

\printunicodeblock{./languages/hieroglyphics.txt}{\hiero}
\printunicodeblock{./languages/hieroglyphics-13100.txt}{\hiero}
\printunicodeblock{./languages/hieroglyphics-13200.txt}{\hiero}
\printunicodeblock{./languages/hieroglyphics-13300.txt}{\hiero}
\printunicodeblock{./languages/hieroglyphics-13400.txt}{\hiero}
\section{Numerals}

Egyptian numbers are encoded following the same principles used for the
encoding of Aegean and Cuneiform numbers. Gardiner does not supply a full set of
numerals with catalog numbers in his Egyptian Grammar, but does describe the system of
numerals in detail, so that it is possible to deduce the required set of numeric characters.

Two conventions of representing Egyptian numerals are supported in the Unicode Standard.
The first relates to the way in which hieratic numerals are represented. Individual
signs for each of the 1s, the 10s, the 100s, the 1000s, and the 10,000s are encoded, because in
hieratic these are written as units, often quite distinct from the hieroglyphic shapes into
which they are transliterated. The other convention is based on the practice of the \emph{Manual
de Codage}, and is comprised of five basic text elements used to build up Egyptian numerals.
There is some overlap between these two systems.

%% Needs some work to get it into LuaLaTeX
%% omitted for the time being
%\ifxetex
%\begin{texexample}{TeXeXglyph}{ex:xetexglyph}
%\raggedright
%\font\myfont = "Aegyptus"
%\setcounter{glyphcount}{136}
%
%\whiledo
%{\value{glyphcount}<\XeTeXcountglyphs\myfont}
%{\arabic{glyphcount}:~
%{\myfont\XeTeXglyph\arabic{glyphcount}}\quad
%\stepcounter{glyphcount}}
%\end{texexample}
%\fi

\section{Input Methods}

If you writing a document with a lot of hieroglyphics inputting of hieroglyphics can be problematic. Most researchers in the field will use special keyboards or editors. They also use MS/Word or OpenOffice. They can both be coerced to produce reasonable documents, but with \tex obviously better results can be achieved. One such editor is \href{http://jsesh.qenherkhopeshef.org/}{jsesh}. 


\begin{luacode*}
    local h = {}
          h = dofile("hiero.lua")
    local options = {style="block",
                     echo=true,
                     direction="RL",
                     size = "\\Huge",
                     color = "green",
                     headings = "captionof{figure}"  -- section/tablecaption/figurecaption
                     }
   -- prints full symbol list
   h.printgardiner(t,options)

   tex.print("\\par")
   local options = {style="block",
                     echo=true,
                     heading="\\par",
                     direction="RL",
                     color = "teal",
                     scale = 8}

   h.printhierochar("hiero","1317D",options)
   h.printhierochar("hiero","13000",{direction="RL",
                                        color = "teal",
                                        scale = 8})
   h.printhierochar("hiero","13003",{direction="LR",
                                        color = "teal",
                                        scale = 1})
   h.parseMdC([[M23-X1-R4-X8-Q2-D4-W17-R14-G4-R8-O29-
               V30-U23-N26-D58-O49-Z1-F13-N31-V30-N16-
               N21-Z1-D45-N25!]])

   tex.print("\\par")
   h.printgardinercat("B")

\end{luacode*}

\newcommand\hierochar[2][direction = "LR",
                         color     = "teal",
                         scale     = 1]{% 
               \luaexec{
                h = h or {}
                h = require("hiero.lua")  
                h.parseMdC(#2,{#1})}}
               
\newcommand\printhierochar[3][direction = "LR",
                              color     = "teal",
                              scale     = 4]{% 
               \luaexec{
                h = h or {}
                h = require("hiero.lua")  
                h.printhierochar(#2,#3,{#1})}}

This file just tests the various commands available for manipulating hieroglyphics. We tried to 
generalize the commands, so they can be re-used for other type of hieroglyphics.

{
\hierochar{"A1-A2-A3!"}

\centering 

\def\options{direction = "LR",
             color     = "teal",
             scale     = 7}

\def\fontname{"hiero"}

\def\hierochar#1{\printhierochar[\options]{\fontname}{#1}}
}


\begin{scriptexample}[]{Some Example}
Sometimes kerning might be required, especially if the
glyphs are scaled.This is easily achieved with a \cmd{\kern}
command and a suitable skip dimension.

\medskip

\bgroup
\fboxsep=0pt\fboxsep.4pt
\def\options{direction = "RL",
             color     = "black!95",
             scale     = 5}
\centering

\color{teal}
\fbox{\hierochar{"13051"}}
\kern-4mm
\hierochar{"13003"}
\def\options{direction = "LR",
             color     = "black!95",
             scale     = 5}
\fbox{\hierochar{"13003"}}\color{red}
\kern-4mm
\hierochar{"13051"}
\color{black!95}
\egroup
\begin{verbatim}
\centering
\hierochar{"13051"}
\kern-4mm
\hierochar{"13003"}
\def\options{direction = "RL",
             color     = "black!95",
             scale     = 5}
\hierochar{"13003"}
\kern-4mm
\hierochar{"13051"}
\end{verbatim}
\end{scriptexample}

A bit of a diversion is appropriate at this point. Our attempt after the historical overview, is to provide some routines for the capturing and display of hieroglyphic texts using LuaTeX. This involves getting low level information from the system regarding fonts. 

\begin{figure}[ht]
\begin{minipage}{0.45\textwidth}
\centering
\includegraphics[width=0.6\textwidth]{./images/fontforge.jpg}
\end{minipage}
\begin{minipage}[t]{0.45\textwidth}
\caption{Viewing font information with fontforge.}
\end{minipage}
\end{figure}

For each glyph, we are interested to get its unicode number, the position in the font table, its name and most importantly the font metrics. The font metrics are a set of parameters that are used to measure the bounding box, any ascenders or descenders and similar information. Using fontforge, these parameters can easily be viewed. However, we are not interested to make any modifications manually; what we are interested is to programmatically obtain this information using Lua. Lua's philosophy and a mantra repeated often by the developers, is that it provides the tools and not the solutions. What this means to the LuaTeX programmer, is that we need to reach very low level  to get this information, which is a road with many bumps. Luckily the tools have been provided by the LuaTeX developers. This comes with a lot of benefits as we can also do our own on the fly mapping, such as creating an index table holding all the Gardiner numbers. 

The |fontloader.open| function loads a font, but it's not usable by itself; the result should be turned into a table with
\textbf{fontloader.to\_table}, as follows:

\begin{verbatim}
  local f = fontloader.open
     ("c:/windows/fonts/NotSansEgyptianHieroglyphics-
       Regulat.ttf")
  fonttable = fontloader.to_table(f)
  fontloader.close(f)
\end{verbatim}

We will use the Google No Tofu Egyptian Hieroglyphic font to experiment with our hieroglyphics. I have used a full path to load the font, which resides on my windows machine in the fonts folder. Once we load all the information in the |fonttable| we use |fontloader.close| to discard the userdata from which the table is extracted. 

What makes OpenType fonts special is that they describe every aspect that you might be able to think of when you think of putting letters together to form words. In addition to the obvious "this is what letters look like" information, OpenType fonts also specify things like the name of each letter that is available in the font, how much of the Unicode standard the font implements, which horizontal and vertical metrics apply to which letters, exactly how the letters are arranged inside the font so that they can quickly be read out, what kind of font classifications apply (is it a fantasy font? is it bold face? is it fixed width? etc), what kind of memory allocation a printer needs to perform in order to be able to even load the font, etc. etc. etc. All these are stored in tables upon tables, similat to a collection of Russian dolls.

To view the values in the fonttable, we will first iterate over the \textbf{fonttable} and extract all the first level keys.

\begin{texexample}{Iterating through a font table}{}
\begin{luacode*}
local z={}
tf=fontloader.to_table(fontloader.open("c:/windows/fonts/NotoSansEgyptianHieroglyphs-Regular.ttf"))

-- we sort the keys to create a table
-- important keys to us are tf.glyphs

for k,v in pairs (tf) do
   --tex.print(k.."\\par")
   table.insert(z, k)
end

table.sort(z)
tex.print("\\begin{multicols}{3}\\raggedright")
for k,v in pairs (z) do
   z[k] = string.gsub(z[k],"%_","\\textunderscore ")
   local s = tf[v]
   tex.print("\\textbullet\\hskip3pt\\hangindent2em " .. z[k].." [\\textit{"..type(s).."}] ","\\par")
end
tex.print("\\end{multicols}")
\end{luacode*}
\end{texexample}

We iterate through the \textbf{fonttable} using the Lua  "pair" iterator and we simply print all the keys and the type of the values in a human readable form as shown in the example. Note the use of |\textunderscore| that replaces all underscores in the fields with its text equivalent to sanitize the output. This is a quick and dirty way to avoid the use of catcodes. Many of the keys, bear intuitive names and are not difficult to discern: \textit{version}, \textit{copyright} and the like. Getting the type of Lua variables is important in order to use them for error trapping. When you attempt for example to print a nil value an error will occur.

Now that we have peeked under the font we will iterate and capture the information of interest, which we will put into another table with two keys \textbf{info}  and \textbf{metrics}. In the metrics file we will get the bounding box related metrics of each and every glyph in the font and save it, into our own table. 

\begin{texexample}{More Metrics}{}
  \begin{luacode*}
   tex.print("units per em = ", tf.units_per_em,"\\par")
   for i,j in ipairs (tf.glyphs[6].boundingbox) do
      tex.print("bounding box["..i.."]".." = ", j,"\\par")
   end 
   local w = (tf.glyphs[6].boundingbox[3]-tf.glyphs[6].boundingbox[1])/tf.units_per_em
   local h = tf.glyphs[6].boundingbox[4]/tf.units_per_em
   tex.print("glyph width = ", w,"em\\par")
   tex.print("glyph height = ", h,"em\\par")

-- presents a nicely typeset table 

local rep, write = string.rep, tex.print
function ExploreTable (tab, offset)
    offset = offset or ""
    for k, v in pairs (tab) do
        local newoffset = offset .. "\\mbox{.}"
        if type(v) == "table" then
           -- if k == "boundingbox" then write("BB") end
           write(offset..k .. " = \\{\\par ")
           ExploreTable(v, newoffset)
           write(offset..newoffset .. "\\}\\par")
         else
           write(offset..k .. " = "..tostring(v),"\\par")
         end
      end
end

write("\\par{\\ttfamily ")
ExploreTable(tf.glyphs[38],"\\mbox{.}")
write("}")
  \end{luacode*}
\end{texexample}

The OpenType fonts standard, provides for so much information that we will ignore most of the items and focus on only a few tables and fields. A small utility after Paul Isambert's article is necessary to enable us to view tables easily within this book,


\begin{texexample}{ExploreTable utility}{}
\begin{luacode*}
-- presents a nicely typeset table 

local rep, write = string.rep, tex.print
function ExploreTable (tab, offset)
    offset = offset or ""
    for k, v in pairs (tab) do
        local newoffset = offset .. "\\mbox{.}"
        if type(v) == "table" then
           -- if k == "boundingbox" then write("BB") end
           write(offset..k .. " = \\{\\par ")
           ExploreTable(v, newoffset)
           write(offset..newoffset .. "\\}\\par")
         else
           write(offset..k .. " = "..tostring(v),"\\par")
         end
      end
end

write("\\par{\\ttfamily ")
ExploreTable(tf.glyphs[38],"\\mbox{.}")
write("}")
  \end{luacode*}
\end{texexample}

A good utility also is |TTX| that will convert an OTF font to XML and back. This requires that you have python installed.\footnote{See some good guidelines as to how to install it at \url{http://www.glyphrstudio.com/ttx/}.} The utility uses python to do the conversion. The archive can be downloaded from \url{http://sourceforge.net/projects/fonttools/files/latest/download}. This is a three prong attack. You need to have python install, the numpy library and then the TTX package. The |TTX| program was written by the font designer Just van Rossum, brother of the creator of the Python language, Guido van Rossum. The tool converts TrueType into human-readable |XML| format. The most attractive feature of this tool is that it also perform the opposite operation that is create a TruType font from an |XML| file. The |XML| format makes the hierarchy of the format clearer. Since SVG fonts are also described in |XML| it becomes an easier task to convert an |SVG| font to a TrueType font. To convert |bar.ttf| into |bar.ttx| you simply write:

\begin{verbatim}
ttx bar.ttf
\end{verbatim}

Similarly for the opposite conversion, from |.ttx| to |.ttf|

\begin{verbatim}
ttx bar.ttx
\end{verbatim}

The generated ttx file is approximately ten times larger than the original |.ttf| file. The files generated are huge affairs and difficult to manage.The command line option |-l| prints a list of the tables in the font. |TTX| is indispensable in the ``humanization'' of TrueType fonts. The details of the tables and what each field represents are eloquently described in that indispensable book by Yannis Haralambous \textit{Fonts \& Encodings.} Although the book is now somewhat dated, it is still the best source of information on many esoteric topics related to fonts. 






\DocInput{\jobname.dtx}
%\chapter{Testing Color Palettes}
\minitoc

\section{First Section}

\lipsum[1]
\subsection{A subsection}
\lipsum[1]
\begin{texexample}{Color Palette} {ex:colors}
% This is a comment
\def\atest{blue}
\atest
% This is another comment
\end{texexample}

%\cxset{steward,
  numbering=arabic,
  custom=stewart,
  offsety=0cm,
  image={europa.jpg},
  texti={An introduction to the use of font related commands. The chapter also gives a historical background to font selection using \tex and \latex. },
  textii={In this chapter we discuss keys that are available through the \texttt{phd} package and give a background as to how fonts are used
in \latex.
 },
 pagestyle = empty,
}

\chapter{European Alphabetic Scripts}

\section{Introduction}

Modern European alphabetic scripts are derived from or influenced by the Greek script,
which itself was an adaptation of the Phoenician alphabet. A Greek innovation was writing
the letters from left to right, which is the writing direction for all the scripts derived from or
inspired by Greek

The European alphabetic scripts and additional characters described in this chapter follow the Unicode blocks:
\medskip


\begin{center}
\begin{tabular}{lll}
Latin &Cyrillic &Georgian.\\
Greek. &Glagolitic. &Modifier letters\\
Coptic &Armenian. &Combining marks\\
\end{tabular}
\end{center}

\section{Latin Script}

Latin script, or Roman script, is an alphabet based on the letters of the classical Latin alphabet. It is used as the standard method of writing in most Western and Central European languages, as well as many languages from other parts of the world. Latin script is the basis for the largest number of alphabets of any writing system[1] and is the most widely adopted writing system in the world (commonly used by about 70\% of world's population). It is also the basis of the International Phonetic Alphabet. The 26 most widespread letters are the letters contained in the ISO basic Latin alphabet.

The script is either called Roman script or Latin script, in reference to its origin in ancient Rome. In the context of transliteration the term "romanization" or "romanisation" is often found.[2][3] Unicode uses the term "Latin"[4] as does the International Organization for Standardization (ISO).[5] The numerals are called Roman numerals.


\subsection{Ligatures}

\newfontfamily\pan{code2000.ttf}

Ligatures for the Latin script are found in the Unicode block Alphabetic Presentation Forms which contains standard ligatures for the Latin, Armenian, and Hebrew scripts.

\begin{scriptexample}[]{Ligatures}
\unicodetable{pan}{"FB00,"FB10,"FB20,"FB30,"FB40}
\end{scriptexample}

\newfontfamily\georgian[Script=Georgian,Scale=1.2]{code2000.ttf}

\newfontfamily\georgianarial[Script=Georgian,Scale=1.2]{Arial Unicode MS}
\section{Georgian}
\label{sec:georgian}
The Georgian scripts are the three writing systems used to write the Georgian language: Asomtavruli, Nuskhuri and Mkhedruli. Their letters are equivalent, sharing the same names and alphabetical order and all three are unicameral (make no distinction between upper and lower case). Although each continues to be used, Mkhedruli (see below) is taken as the standard for Georgian and its related Kartvelian languages\footnote{Unicode Standard, V. 6.3. U10A0, p. 3}. 

\bgroup
\topline



\begin{scriptexample}[]{}
\georgian 

\centering
 
ყველა ადამიანი იბადება თავისუფალი და თანასწორი თავისი ღირსებითა და უფლებებით. მათ მინიჭებული აქვთ გონება და სინდისი და ერთმანეთის მიმართ უნდა იქცეოდნენ ძმობის სულისკვეთებით.
\medskip

\georgianarial
ყველა ადამიანი იბადება თავისუფალი და თანასწორი თავისი ღირსებითა და უფლებებით. მათ მინიჭებული აქვთ გონება და სინდისი და ერთმანეთის მიმართ უნდა იქცეოდნენ ძმობის სულისკვეთებით.
\bottomline
\captionof{table}{Article 1 of the Universal Declaration of Human Rights in Georgian, typeset in \texttt{code2000} (top) and \texttt{Arial Unicode MS } (bottom).}

\end{scriptexample}

The scripts originally had 38 letters. Georgian is currently written in a 33-letter alphabet, as five of the letters are obsolete in that language. The Mingrelian alphabet uses 36: the 33 of Georgian, one letter obsolete for that language, and two additional letters specific to Mingrelian and Svan. That same obsolete letter, plus a letter borrowed from Greek, are used in the 35-letter Laz alphabet. The fourth Kartvelian language, Svan, is not commonly written, but when it is it uses the letters of the Mingrelian alphabet, with an additional obsolete Georgian letter and sometimes supplemented by diacritics for its many vowels.

\chapter{Armenian}

\label{s:armenian}\index{Armenian}\index{scripts>Armenian}

As we present the scripts in alphabetic order, the first script we will typeset is in Armenian. There are many fonts available for the language. We use two in the example, the first one is \textit{FreeSans} and the second is \textit{Sylphaen} which is found on Windows Operating systems. The language is not supported by the \pkg{Babel} and partially supported by the \pkgname{Polyglossia}. \tcbdocmarginnote{china revision}

\def\ucfirst#1#2;{\MakeUppercase#1#2}


\def\armeniantest#1#2{
  {\parindent0pt
  \topline \vskip3pt
  \noindent\mbox{
     \ucfirst#1;\hfill\hbox{[\texttt{U+0530-U+058F}]}
  }}
 \nobreak

\begin{minipage}{0.45\textwidth}
\bgroup
%\setotherlanguage{#1}
\begin{#1}
#2
[\today]
\end{#1}
\egroup
\end{minipage}\hspace*{1em}
\begin{minipage}{0.45\textwidth}
\bgroup
  \parindent0pt
  \ttfamily\raggedright
  \string\documentclass\{article\}\par
  \string\usepackage[no-math]\{fontspec\}\\
  \string\newfontfamily\textbackslash#1font[Script=\ucfirst #1;,\\   ~~~~~~~Scale=MatchLowercase]
\{FreeSans\}\par
  \string\begin\{document\}\\
  \string\setotherlanguage\{#1\}\\
  \string\begin\{#1\}\\
  \egroup
\begin{#1}
\hskip10pt\vbox{#2}
\end{#1}
\bgroup
  \ttfamily[\detokenize{\today}]\\
  \string\end\{#1\}\\
  \string\end\{document\}
\egroup
\end{minipage}


\textit{FreeSans}: \url{ http://www.gnu.org/software/freefont/}
}

\armeniantest{armenian}{Բոլոր մարդիկ ծնվում են ազատ ու հավասար իրենց
արժանապատվությամբ ու իրավունքներով։       
Նրանք ունեն բանականություն ու խիղճ և միմյանց
պետք է եղբայրաբար վերաբերվեն։}

The Armenian script was invented around 407 AD, by Mesrop Maštoc, a cleric who wanted to 
translate Greek and Syriac scriptures and liturgical texts into Armenian. The system he devised 
is a pure alphabet, closely modelled on the traditional order of Greek phonetic values, with 
additional graphemes to represent Armenian sounds not found in Greek. The orthography is, 
phonetically, a near perfect representation of the Armenian language, and has remained almost 
entirely unchanged since its invention. In recent times, the letterforms in many Armenian 
typefaces have consciously modelled Latin types in their treatment of serifs, stroke weight and 
stress, and other details. This is the approach that Geraldine adopted for the Sylfaen Armenian, 
in order to harmonise the different scripts within the font. 

This kind of harmonisation has to be 
very carefully handled, as there is, of course, a point at which one can corrupt the normative 
letterforms and produce something which will be unacceptable to native readers. Once again, 
we sought expert review of the design, this time from Manvel Shmavonyan, an Armenian type designer, and his Russian colleague Vladimir Yefimov at 
ParaType in Moscow.

\bgroup
\medskip
\fontspec[Script=Armenian,Scale=1.7]{Sylfaen}
\centering

Աա Բբ Գգ Դդ Եե Զզ Էէ Ըը Թթ Ժժ Իի \\
Լլ Խխ Ծծ Կկ Հհ Ձձ Ղղ Ճճ Մմ Յյ Նն \\
Շշ Ոո Չչ Պպ Ջջ Ռռ Սս Վվ Տտ Րր Ցց \\
Ււ Փփ Քք Օօ Ֆֆ / և ﬓ ﬔ ﬕ ﬖ ﬗ\\
\egroup
\captionof{table}{Armenian, showing the basic alphabet (typeset using the \textit{Sylfaen} font.}
\medskip

\bgroup
\def\m#1 #2 #3\\{\makebox[2em]{#1}\makebox[2em]{{\fontspec{code2000.ttf}#2}}\makebox[2em]{\hfill#3 \\ }}
\fontspec[Script=Armenian,Scale=1.1]{Sylfaen}

\begin{multicols}{4}
\m Ա	A	1\\
\m Բ	B	2\\
\m Գ	G	3\\
\m Դ	D	4\\
\m Ե	E	5\\
\m Զ	Z	6\\
\m Է	ē	7\\
\m Ը	ə	8\\
\m Թ	tʿ	9\\
\m Ժ	ž	10\\
\m Ի	I	20\\
\m Լ	L	30\\
\m Խ	X	40\\
\m Ծ	C	50\\
\m Կ	K	60\\
\m Հ	H	70\\
\m Ձ	J	80\\
\m Ղ	ł	90\\
\m Ճ	č	100\\
\m Մ	M	200\\
\m Յ	Y	300\\
\m Ն	N	400\\
\m Շ	š	500\\
\m Ո	O	600\\
\m Չ	čʿ	700\\
\m Պ	P	800\\
\m Ջ	ǰ	900\\
\m Ռ	ṙ	1000\\ 
\m Ս	S	2000\\
\m Վ	V	3000\\
\m Տ	T	4000\\
\m Ր	R	5000\\
\m Ց	cʿ	6000\\
\m Ւ	W	7000\\
\m Փ	pʿ	8000\\
\m Ք	kʿ	9000\\

\end{multicols}
\captionof{table}{Armenian Numerals \textit{(from Wikipedia).}
The first column is the classical Armenian numeral, the second the transliteration and the third the arabic numeral it represents.}

\medskip

Numbers in the Armenian numeral system are obtained by simple addition. Armenian numerals are written left-to-right (as in the Armenian language). Although the order of the numerals is irrelevant since only addition is performed, the convention is to write them in decreasing order of value.

\begin{align*}
\text{ՌՋՀԵ} &= 1975 = 1000 + 900 + 70 + 5\\
\text{ՍՄԻԲ} &= 2222 = 2000 + 200 + 20 + 2\\
\text{ՍԴ}   &= 2004 = 2000 + 4\\
\text{ՃԻ}   &= 120 = 100 + 20\\
\text{Ծ}    &= 50
\end{align*}

To write numbers greater than 9999, it is necessary to have numerals with values greater than 9000. This is done by drawing a line over them, indicating their value is to be multiplied by 10000:

\begin{align*}
\overline{\text{Ա}} &= 10000\\
\overline{\text{Ջ}} &= 9000000\\
\overline{\text{ՌՃԽԳ}}\text{ՌՄԾԵ} &= 11431255
\end{align*}
\egroup

\subsection{Greek}
\index{languages>Greek}\index{Herodotus}\index{alphabets>Greek}
\newfontfamily\greek[Script=Greek,Scale=1.02]{NotoSerif-Regular.ttf}
\def\greektext#1{\greek{#1}}

`The Phoenicians who came with Kadmos,' wrote Herodotus in the fifth century BC of the legendary Phoenician prince of Tyre and brother of Europa, `\ldots introduced into Greece, after their settlement in the country, a number of accomplishments of which the most important was writing, an art which probably was unknown to the Greeks until then'. 

The Greek alphabet is the script that has been used to write the Greek language since the 8th century BC.[2] It was derived from the earlier Phoenician alphabet, and was in turn the ancestor of numerous other European and Middle Eastern scripts, including Cyrillic and Latin.[3] Apart from its use in writing the Greek language, both in its ancient and its modern forms, the Greek alphabet today also serves as a source of technical symbols and labels in many domains of mathematics, science and other fields.

In its classical and modern forms, the alphabet has 24 letters, ordered from alpha to omega. Like Latin and Cyrillic, Greek originally had only a single form of each letter; it developed the letter case distinction between upper-case and lower-case forms in parallel with Latin during the modern era.

\bgroup
\greek\obeyspaces

Α	ἄλφα	aleph	alpha	[alpʰa]	[ˈalfa]	Listeni/ˈælfə/
Β	βῆτα	beth	beta	[bɛːta]	[ˈvita]	/ˈbiːtə/, US /ˈbeɪtə/
Γ	γάμμα	gimel	gamma	[ɡamma]	[ˈɣama]	/ˈɡæmə/
Δ	δέλτα	daleth	delta	[delta]	[ˈðelta]	/ˈdɛltə/
Η	ἦτα	  heth	   eta	 [hɛːta], [ɛːta]	[ˈita]	/ˈiːtə/, US /ˈeɪtə/
Θ	θῆτα	teth	theta	[tʰɛːta]	[ˈθita]	/ˈθiːtə/, US Listeni/ˈθeɪtə/
Ι	ἰῶτα	yodh	iota	[iɔːta]	[ˈʝota]	Listeni/aɪˈoʊtə/
Κ	κάππα	kaph	kappa	[kappa]	[ˈkapa]	Listeni/ˈkæpə/
Λ	λάμβδα	lamedh	lambda	[lambda]	[ˈlamða]	Listeni/ˈlæmdə/
Μ	μῦ	mem	mu	[myː]	[mi]	Listeni/ˈmjuː/; occasionally US /ˈmuː/
Ν	νῦ	nun	nu	[nyː]	[ni]	/ˈnjuː/ (US /ˈnuː/)
Ρ	ῥῶ	reš	rho	[rɔː]	[ro]	Listeni/ˈroʊ/
Τ	ταῦ	taw	tau	[tau]	[taf]	/ˈtaʊ/ or /ˈtɔː/

\topline
\begin{quote}
Ἡροδότου Ἁλικαρνησσέος ἱστορίης ἀπόδεξις ἥδε, ὡς μήτε τὰ γενόμενα ἐξ ἀνθρώπων τῷ χρόνῳ ἐξίτηλα γένηται, μήτε ἔργα μεγάλα τε καὶ θωμαστά, τὰ μὲν Ἕλλησι, τὰ δὲ βαρβάροισι ἀποδεχθέντα, ἀκλεᾶ γένηται, τὰ τε ἄλλα καὶ δι' ἣν αἰτίην ἐπολέμησαν ἀλλήλοισι.[2]

Herodotus of Halicarnassus, his Researches are set down to preserve the memory of the past by putting on record the astonishing achievements of both the Greeks and the Barbarians; and more particularly, to show how they came into conflict.[3]
\end{quote}
\bottomline

\symbol{"1F00}
\symbol{"1F01}
\egroup

\newfontfamily\glagolitic{MPH 2B Damase}

\section{Glagolitic}

\epigraph{The average Ph.D. thesis is nothing but a transference of bones from one graveyard to another.}{%
J. Frank Dobie (1888-1964)}


\label{s:glagolitic}
\fboxrule0pt\fboxsep0pt

\noindent
The Glagolitic alphabet /{\glagolitic ˌɡlæɡɵˈlɪtɨk/}, also known as Glagolitsa, is the oldest known Slavic alphabet, from the 9th century.

It was created in the 9th century by Saint Cyril, a Byzantine monk from Thessaloniki. He and his brother, Saint Methodius, were sent by the Byzantine Emperor Michael III in 863 to Great Moravia to spread Christianity among the West Slavs in the area. The brothers decided to translate liturgical books into the Old Slavic language that was understandable to the general population, but as the words of that language could not be easily written by using either the Greek or Latin alphabets, Cyril decided to invent a new script, Glagolitic, which he based on the local dialect of the Slavic tribes from the Byzantine Salonika region.
After the deaths of Cyril and Methodius, the Glagolitic alphabet ceased to be used in Moravia, but their students continued to propagate it in the west and south. 

After a long career, Glagolitic writing stopped being used, except for
religious purposes in certain dioceses of Bosnia and Dalmatia (Croatia).
The Cyrillic alphabet was adopted by all Orthodox Slays and served to note
their literary language. Most of the Slays who rallied to Rome rejected it,
however, which created the paradoxical situation in ex-Yugoslavia, where
two peoples who speak the same language write in different scripts, the
Serbs in Cyrillic and the Croats with Roman characters. Finally, as is
known, the ex-Soviet Union did much to put into writing the languages
spoken by the peoples within its borders, for the most part noting them in
adaptations of the Cyrillic alphabet, while Russian became the language of
culture throughout the Soviet Union.\cite{henri1994}

Slavic printing in Glagolitic characters originated in Venice, where a
\textit{Sluzebnik} (or \textit{Leitourgikon}) was published in 1483, followed by missals and
breviaries, all printed by Andrea Torresani, the future father-in-law and
associate of Aldus Manutius. After 1494 some attempts were made to create
printshops in Croatia itself, first in Senj in 1508, then, after 1530, in
Rijeka (Fiume). The work of these firms was almost totally liturgical (religious,
at any rate), and it had strong competition from manuscript works
that were better adapted to the diversity of local liturgical customs. Religion
also dictated the output of a printshop founded to provide Protestant propaganda
that was set up in Tubingen between 1560 and 1564 by Baron
Hans von Ungnad and that printed the great Lutheran texts in Glagolitic
characters.\footfullcite{henri1994}

Figure~\ref{fig:zograf} illustrates an example of the language.\footnote{\url{https://en.wikipedia.org/wiki/Glagolitic_script\#/media/File:ZographensisColour.jpg}}

\begin{figure}[htbp]
\centering

\includegraphics[width=0.45\linewidth]{glagolitic}
\caption[The first page of the Gospel of Mark from the 10th–11th century Codex Zographensis, found in the Zograf Monastery in 1843.]{The first page of the Gospel of Mark from the 10th–11th century Codex Zographensis, found in the Zograf Monastery in 1843.}
\label{fig:zograf}
\end{figure}

\section{Unicode Support}
The Glagolitic alphabet was added to the Unicode Standard in March 2005 with the release of version 4.1.
The Unicode block for Glagolitic is U+2C00–U+2C5F.



\begin{scriptexample}[]{glacolitic}

\unicodetable{glagolitic}{%
"2C00,"2C10,"2C20,"2C30,"2C40,"2C50}

\texttt{typeset with Damase version 2.0 MPH 2B Damase}
\end{scriptexample}
\bgroup
\glagolitic

The name was not coined until many centuries after its creation, and comes from the Old Church Slavonic glagolъ "utterance" (also the origin of the Slavic name for the letter G). The verb glagoliti means "to speak". It has been conjectured that the name glagolitsa developed in Croatia around the 14th century and was derived from the word glagolity, applied to adherents of the liturgy in Slavonic.[1]

In Old Church Slavonic the name is {\glagolitic ⰍⰫⰓⰊⰎⰎⰑⰂⰋⰜⰀ}, Кѷрїлловица.
The name Glagolitic in Bulgarian, Russian, Macedonian глаголица (glagolica), Belarusian is глаголіца (hłaholica), Croatian glagoljica, Serbian глагољица / glagoljica, Czech hlaholice, Polish głagolica, Slovene glagolica, Slovak hlaholika, and Ukrainian глаголиця (hlaholyća).



\egroup






%\chapter{Additional Modern Scripts}

\begin{center}
\begin{tabular}{lp{5cm}l}
Ethiopic. &Vai. &Deseret.\\
Mongolian. &Bamum. &Shavian.\\
Osmanya.   &Cherokee. &Lisu.\\
Tifinagh.  &Canadian Aboriginal Syllabics. &Miao.\\
N’Ko.&&\\
\end{tabular}
\end{center}

Ethiopic, Mongolian, and Tifinagh are scripts with long histories. Although their roots can
be traced back to the original Semitic and North African writing systems, they would not
be classified as Middle Eastern scripts today

The Cherokee script is a syllabary developed between 1815 and 1821, to write the Cherokee
language, still spoken by small communities in Oklahoma and North Carolina. Canadian
Aboriginal Syllabics were invented in the 1830s for Algonquian languages in Canada. The
system has been extended many times, and is now actively used by other communities, including speakers of Inuktitut and Athapascan languages.

Deseret is a phonemic alphabet devised in the 1850s to write English. It saw limited use for
a few decades by members of The Church of Jesus Christ of Latter-day Saints. Shavian is
another phonemic alphabet, invented in the 1950s to write English. It was used to publish
one book in 1962, but remains of some current interest




\subsection{Ethiopic}
Ge'ez (ግዕዝ Gəʿəz), (also known as Ethiopic) is a script used as an abugida (syllable alphabet) for several languages of Ethiopia and Eritrea. It originated as an abjad (consonant-only alphabet) and was first used to write Ge'ez, now the liturgical language of the Ethiopian Orthodox Tewahedo Church and the Eritrean Orthodox Tewahedo Church. In Amharic and Tigrinya, the script is often called fidäl (ፊደል), meaning "script" or "alphabet".

The Ge'ez script has been adapted to write other, mostly Semitic, languages, particularly Amharic in Ethiopia, and Tigrinya in both Eritrea and Ethiopia. It is also used for Sebatbeit, Me'en, and most other languages of Ethiopia. In Eritrea it is used for Tigre, and it has traditionally been used for Blin, a Cushitic language. Tigre, spoken in western and northern Eritrea, is considered to resemble Ge'ez more than do the other derivative languages.[citation needed] Some other languages in the Horn of Africa, such as Oromo, used to be written using Ge'ez, but have migrated to Latin-based orthographies.
For the representation of sounds, this article uses a system that is common (though not universal) among linguists who work on Ethiopian Semitic languages. This differs somewhat from the conventions of the International Phonetic Alphabet. See the articles on the individual languages for information on the pronunciation.

There are a number of fonts available and we have selected the Google \idxfont{NotoSansEthiopic}
\newfontfamily\ethiopic{NotoSansEthiopic-Bold.ttf}

\begin{scriptexample}[]{Ethiopic}
\unicodetable{ethiopic}{"1200,"1210,"1220,"1230,"1240,"1250,"1260,"1270,"1280,"1290,^^A
"12A0,"12B0,"12C0,"12E0,"12F0,"1300,"1310,"1330,"1340,"1350,"1360,"1370}
\end{scriptexample}
\section{Vai}
\label{s:vai}

The Vai syllabary is a syllabic writing system devised for the Vai language by Momolu Duwalu Bukele of Jondu, in what is now Grand Cape Mount County, Liberia.[1] [2] Bukele is regarded within the Vai community, as well as by most scholars, as the syllabary's inventor and chief promoter when it was first documented in the 1830s. It is one of the two most successful indigenous scripts in West Africa.

\newfontfamily\vai{code2000.ttf}
\begin{scriptexample}[]{Vai}
\unicodetable{vai}{"A500,"A510,"A520,"A530,"A540,"A550,"A560,"A570,^^A
"A580,"A590,"A5A0,"A5B0,^^A
"A5C0,"A5D0,"A5E0,"A5F0,"A610,"A620,"A630}
\end{scriptexample}

In the 1920s ten decimal digits were devised for Vai; these were “Vai-style” glyph variants of
European digits (see Figure 11). They were not popular with Vai people  even for historical purposes. All
the modern literature uses European digits.


\begin{scriptexample}[]{Vai}
\bgroup
\vai
\obeylines\Large
0	1	2	3	4	5	6	7	8	9
꘠	꘡	꘢	꘣	꘤	꘥	꘦	꘧	꘨	꘩
\vai
\egroup
\end{scriptexample}



\printunicodeblock{./languages/vai.txt}{\vai}
\section{Deseret script}
\newfontfamily\deseret{code2001.ttf}

The Deseret alphabet (dɛz.əˈrɛt.) (Deseret: {\deseret 𐐔𐐯𐑅𐐨𐑉𐐯𐐻 or 𐐔𐐯𐑆𐐲𐑉𐐯𐐻}) is a phonemic English spelling reform developed in the mid-19th century by the board of regents of the University of Deseret (later the University of Utah) under the direction of Brigham Young, second president of The Church of Jesus Christ of Latter-day Saints.

In public statements, Young claimed the alphabet was intended to replace the traditional Latin alphabet with an alternative, more phonetically accurate alphabet for the English language. This would offer immigrants an opportunity to learn to read and write English, he said, the orthography of which is often less phonetically consistent than those of many other languages. Similar experiments were not uncommon during the period, the most well-known of which is the Shavian alphabet.

Young also prescribed the learning of Deseret to the school system, stating "It will be the means of introducing uniformity in our orthography, and the years that are now required to learn to read and spell can be devoted to other studies".[2]


Deseret script {\deseret 𐐔𐐯𐑅𐐨𐑉𐐯𐐻}  [U+10400-U+1044F]
\medskip

\bgroup
\par
\noindent
\colorbox{graphicbackground}{\color{black}^^A
\begin{minipage}{\textwidth}^^A
\parindent1pt
\vskip10pt
\leftskip10pt \rightskip\leftskip
\deseret
\large

𐐂 𐑌𐐲𐑉𐑅𐐨𐑉𐐮 𐐮𐑆 𐐪 𐐹𐐨𐑅 𐐱𐑂 𐑊𐐰𐑌𐐼 𐐱𐑌 𐐸𐐶𐐮𐐽 𐑁𐑉𐐭𐐻𐐻𐑉𐐨𐑆 𐐪𐑉 𐑅𐐻𐐪𐑉𐐻𐐯𐐼,


\par
\vspace*{10pt}
\end{minipage}
}

Text: Deseret alphabet http://www.omniglot.com/writing/deseret.htm
\medskip
\egroup

\PrintUnicodeBlock{./languages/deseret.txt}{\deseret}

\chapter{Bamum}
\label{s:bamum}
\epigraph{"No known alphabet was ever invented by a European."}{Jeffreys' translation from the Royal script.}

\label{s:bamum}
\index{scripts>Bamum}
\newfontfamily\bamum{NotoSansBamum-Regular.ttf}

The Bamum scripts are an evolutionary series of six scripts created for the Bamum language by King Njoya of Cameroon at the turn of the 20th century. They are notable for evolving from a pictographic system to a partially alphabetic syllabic script in the space of 14 years, from 1896 to 1910. Bamum type was cast in 1918, but the script fell into disuse around 1931.

\begin{figure}[htbp]
\parindent=0pt

\centering

\includegraphics[width=\textwidth]{bamum}

\caption{King Njoya of Bamum receiving an oil painting of Kaiser Wilhelm II. The gift was in return for his support in the German campaign against the Nso'.}
\end{figure}

The Bamum, sometimes called Bamoum, Bamun, Bamoun, or Mum, are a Bantoid ethnic group of Cameroon with around 215,000 members.



\begin{scriptexample}[]{Bamum}
\unicodetable{bamum}{"A6A0,"A6B0,"A6C0,"A6D0,"A6E0,"A6F0}
\end{scriptexample}
\section{Shavian}
\label{s:shavian}
\def\shaviansetup#1{}
\newfontfamily\shavian{code2001.ttf}
^^A\newfontfamily\shavian{NotoSansShavian-Regular.ttf}
\cxset{shavian font/.code=\shaviansetup{#1}}
\cxset{shavian font=shavian}




\begin{scriptexample}[]{shavian}
\shavian

𐑳 𐑡𐑻𐑯𐑰 𐑑 𐑞 𐑕𐑧𐑯𐑑𐑻 𐑝 𐑞 𐑻𐑔
𐑚𐑲 - ·𐑡𐑵𐑤𐑟 ·𐑝𐑻𐑯

𐑗𐑩𐑐𐑑𐑻 1 - 𐑥𐑲 𐑳𐑙𐑒𐑳𐑤 𐑥𐑱𐑒𐑕 𐑳 𐑜𐑮𐑱𐑑 𐑛𐑦𐑕𐑒𐑳𐑝𐑻𐑰

     𐑤𐑫𐑒𐑦𐑙 𐑚𐑩𐑒 𐑑 𐑷𐑤 𐑞𐑩𐑑 𐑣𐑩𐑟 𐑳𐑒𐑻𐑛 𐑑 𐑥𐑰 𐑕𐑦𐑯𐑕 𐑞𐑩𐑑 𐑦𐑝𐑧𐑯𐑑𐑓𐑳𐑤 𐑛𐑱, 𐑲 𐑩𐑥 𐑕𐑒𐑧𐑮𐑕𐑤𐑰 𐑱𐑚𐑳𐑤 𐑑 𐑚𐑦𐑤𐑰𐑝 𐑦𐑯 𐑞 𐑮𐑰𐑩𐑤𐑳𐑑𐑰 𐑝 𐑥𐑲 𐑩𐑛𐑝𐑧𐑯𐑗𐑻𐑟. 𐑞𐑱 𐑢𐑻 𐑑𐑮𐑵𐑤𐑰 𐑕𐑴 𐑢𐑳𐑯𐑛𐑻𐑓𐑳𐑤 𐑞𐑩𐑑 𐑰𐑝𐑦𐑯 𐑯𐑬 𐑲 𐑩𐑥 𐑚𐑦𐑢𐑦𐑤𐑛𐑻𐑛 𐑢𐑧𐑯 𐑲 𐑔𐑦𐑙𐑒 𐑝 𐑞𐑧𐑥.
     𐑥𐑲 𐑳𐑙𐑒𐑳𐑤 𐑢𐑪𐑟 𐑳 𐑡𐑻𐑥𐑳𐑯, 𐑣𐑩𐑝𐑦𐑙 𐑥𐑧𐑮𐑰𐑛 𐑥𐑲 𐑥𐑳𐑞𐑻𐑟 𐑕𐑦𐑕𐑑𐑻, 𐑩𐑯 𐑦𐑙𐑜𐑤𐑦𐑖𐑢𐑫𐑥𐑳𐑯. 𐑚𐑰𐑦𐑙 𐑝𐑧𐑮𐑰 𐑥𐑳𐑗 𐑳𐑑𐑩𐑗𐑑 𐑑 𐑣𐑦𐑟 𐑓𐑪𐑞𐑻𐑤𐑳𐑕 𐑯𐑧𐑓𐑘𐑵, 𐑣𐑰 𐑦𐑯𐑝𐑲𐑑𐑳𐑛 𐑥𐑰 𐑑 𐑕𐑑𐑳𐑛𐑰 𐑳𐑯𐑛𐑻 𐑣𐑦𐑥 𐑦𐑯 𐑣𐑦𐑟 𐑣𐑴𐑥 𐑦𐑯 𐑞 𐑓𐑪𐑞𐑻𐑤𐑩𐑯𐑛. 𐑞𐑦𐑕 𐑣𐑴𐑥 𐑢𐑪𐑟 𐑦𐑯 𐑳 𐑤𐑪𐑮𐑡 𐑑𐑬𐑯, 𐑯 𐑥𐑲 𐑳𐑙𐑒𐑳𐑤 𐑳 𐑐𐑮𐑳𐑓𐑧𐑕𐑻 𐑝 𐑓𐑳𐑤𐑪𐑕𐑳𐑓𐑰, 𐑒𐑧𐑥𐑳𐑕𐑑𐑮𐑰, 𐑡𐑰𐑪𐑤𐑳𐑡𐑰, 𐑥𐑦𐑯𐑻𐑪𐑤𐑳𐑡𐑰, 𐑯 𐑥𐑧𐑯𐑰 𐑳𐑞𐑻 𐑳𐑤𐑴𐑡𐑰𐑕.

\arial

\hfill Excerpt from Jules Vern,  \textit{Journey to the Center of the Earth from \href{http://shavian.weebly.com/}{shavian}}
\end{scriptexample}

The example is typeset using \texttt{code2001.ttf}. There are numerous fonts that provide Shavian glyphs. \texttt{ESL Gothic Unicode} font by Ethan Lamoreaux\footnote{\url{http://www.fontspace.com/ethan-lamoreaux/esl-gothic-unicode}}. The Noto fonts also have a Shavian font. 

You can activate typesetting in Shavian using the key:

\begin{key}{/chapter/shavian font = \meta{font name}} The key will setup the
default font for the Shavian script and define the commands \cmd{\shavian} and \cmd{\textshavian}. 
\end{key}

\PrintUnicodeBlock{./languages/shavian.txt}{\shavian}





\subsection{Osmanya}

\newfontfamily\osmanya{NotoSansOsmanya-Regular.ttf}

\begin{scriptexample}[]{Osmanya}
\unicodetable{osmanya}{"10480,"10490,"104A0}
\end{scriptexample}

The Osmanya alphabet (Somali: Cismaanya; Osmanya: {\osmanya 𐒋𐒘𐒈𐒑𐒛𐒒𐒕𐒀}), also known as Far Soomaali ("Somali writing"), is a writing script created to transcribe the Somali language. It was invented between 1920 and 1922 by Osman Yusuf Kenadid of the Majeerteen Darod clan, the nephew of Sultan Yusuf Ali Kenadid of the Sultanate of Hobyo.

While Osmanya gained reasonably wide acceptance in Somalia and quickly produced a considerable body of literature, it proved difficult to spread among the population mainly due to stiff competition from the long-established Arabic script as well as the emerging Somali alphabet developed by the Somali linguist, Shire Jama Ahmed, which was based on the Latin script.

As nationalist sentiments grew and since the Somali language had long lost its ancient script,[1] the adoption of a universally recognized writing script for the Somali language became an important point of discussion. After independence, little progress was made on the issue, as opinion was divided over whether the Arabic or Latin scripts should be used instead.

In October 1972, due to its simplicity, the fact that it lent itself well to writing Somali since it could cope with all of the sounds in the language, and the already widespread existence of machines and typewriters designed for its use,[2][3] the government of Somali president Mohamed Siad Barre unilaterally elected to use only the Latin script for writing Somali instead of the Arabic or Osmanya scripts.[4] Barre's administration subsequently launched a massive literacy campaign designed to ensure its sole adoption. This led to a sharp decline in use of Osmanya.
\section{Cherokee}
\index{scripts>Cherokee}
\index{scripts>Cherokee>fonts}
\label{sec:cherokee}
Windows comes with |Plantagenet Cherokee| font. The |code2000| also has good support for the alphabet. The \texttt{SIL font Charis SIL} also has good support and can be downloaded at \href{http://scripts.sil.org/cms/scripts/page.php?item_id=CharisSIL_download}{scripts.sel.org}, the latest version gave me problems when used with Windows. 

  
\def\textcherokee#1{{\cherokee   #1}}


\begin{docKey}[phd]{cherokee font}{ = \meta{font name}} {default none, initial=code2000}
 Loads the font
command \cmd{\cherokee}. When the command is used it typesets text in
cherokee unicode. There is no need to load the language, unless it is the main document language. For windows the default font is  |Plantagenet Cherokee|. Another font is FreeSerif, which we are using here.
\end{docKey}

\begin{scriptexample}[]{Cherokee}
{\cherokee
\begin{tabular}{lp{8.5cm}}
Translation	  &John (ᏣᏂ) 3:16\\
American Bible Society 1860	&ᎾᏍᎩᏰᏃ ᏂᎦᎥᎩ ᎤᏁᎳᏅᎯ ᎤᎨᏳᏒᎩ ᎡᎶᎯ, ᏕᏅᏲᏒᎩ ᎤᏤᎵᎦ ᎤᏪᏥ ᎤᏩᏒᎯᏳ ᎤᏕᏁᎸᎯ, ᎩᎶ ᎾᏍᎩ ᏱᎪᎯᏳᎲᏍᎦ ᎤᏲᎱᎯᏍᏗᏱ ᏂᎨᏒᎾ, ᎬᏂᏛᏉᏍᎩᏂ ᎤᏩᏛᏗ.\\

(Transliteration)	& nasgiyeno nigavgi unelanvhi ugeyusvgi elohi, denvyosvgi utseliga uwetsi uwasvhiyu udenelvhi, gilo nasgi yigohiyuhvsga uyohuhisdiyi nigesvna, gvnidvquosgini uwadvdi.\\
\end{tabular}}
\end{scriptexample}

\begin{texexample}{Using text...}{cherokee}
\bgroup
\cherokee \large\textbf{ᎾᏍᎩᏰᏃ}
\textcherokee{ᎾᏍᎩᏰᏃ}
\egroup
\end{texexample}

If you have trouble getting them to work\footnote{\url{http://tex.stackexchange.com/questions/132087/displaying-cherokee-text}}

\url{http://www.cherokee.org/AboutTheNation/Language/CherokeeFont.aspx}




\section{Tifnagh}

\newfontfamily\tifinagh{code2000.ttf}

Tifinagh (Berber pronunciation: [tifinaɣ]; also written Tifinaɣ in the Berber Latin alphabet, {\tifinagh  ⵜⵉⴼⵉⵏⴰⵖ} in Neo-Tifinagh, and تيفيناغ in the Berber Arabic alphabet) is a series of abjad and alphabetic scripts used by Berber peoples to write Berber languages.[1]
A modern derivate of the traditional script, known as Neo-Tifinagh, was introduced in the 20th century. A slightly modified version of the traditional script, called Tifinagh Ircam, is used in a number of Moroccan elementary schools in teaching the Berber language to children as well as a number of publications.[2][3]

The word tifinagh is thought to be a Berberized feminine plural cognate of Punic, through the Berber feminine prefix ti- and Latin Punicus; thus tifinagh could possibly mean "the Phoenician (letters)"[4][5] or "the Punic letters".

\bgroup

\noindent\tifinagh
\colorbox{thecodebackground}{\color{black}^^A
\begin{minipage}{\textwidth}
\parindent1pt
\vskip10pt
\leftskip10pt \rightskip\leftskip
Tifnagh     ⵜⵉⴼⵉⵏⴰⵖ [U+2D30-U+2D7F]

ⴰⴳⵍⴷⵓⵏ ⴰⵎⵥⵥⴰ

ⵙ ⵡⴰⵡⴰⵍ ⴳⵔⵉ ⵉⴷⵙ, ⵙⵙⵏⵖ ⵢⴰⵜ ⵜⵖⴰⵡⵙⴰ ⵜⵉⵙⵙ ⵙⵏⴰⵜ  ⵉⵅⴰⵜⵔⵏ: ⵉⵜⵔⵉ ⵙⴳ ⴷⴷ ⵉⴷⴷⴰ ⵓⵔ ⵉⵎⵇⵇⵓⵔ, ⵉⵍⵍⴰ ⵖⴰⵙ ⴰⵏⵛⵜ ⵏ ⵢⴰⵜ ⵜⴰⴷⴷⴰⵔⵜ !

ⴰⵢⴰ ⵓⴽⵣⵖ ⵜ. ⵙⵙⵏⵖ ⵉⵙ ⴱⵕⵕⴰ ⵏ ⵉⵜⵔⴰⵏ ⵣⵓⵏⴷ ⴰⴽⴰⵍ, ⵊⵓⴱⵉⵜⵔ, ⵎⴰⵔⵙ, ⴱⵉⵏⵓⵙ – ⵉⵜⵔⴰⵏ ⵎⵉ ⵏⴽⴼⴰ ⵉⵙⵎⴰⵡⵏ – ⵍⵍⴰⵏ ⴷⵉⵖ ⵉⵜⵔⴰⵏ ⵢⴰⴹⵏ ⵎⵥⵥⵉⵢⵏⵉⵏ, ⵡⵉⵏⵏⴰ ⵓⵔ ⵏⵣⵎⵉⵔ ⴰⴷ ⵏⵥⵔ ⵙ ⵓⵜⵉⵍⵉⵙⴽⵓⴱ. ⴰⴷⴷⴰⵢ ⵢⵓⴼⴰ ⵓⴰⵙⵜⵕⵓⵏⵓⵎ ⵢⴰⵏ ⴷⵉⴳⵙⵏ, ⴷⴰ ⵢⴰⵙ ⵉⵜⵜⴳⴰ ⵙ ⵢⵉⵙⵎ ⵢⴰⵏ ⵡⵓⵜⵜⵓⵏ. ⴷⴰ ⵢⴰⵙ ⵉⵇⵇⴰⵔ ⵙ ⵓⵎⴷⵢⴰⵜ : « ⴰⵙⵜⵔⵓⵉⴷ 3251 ».

ⵓⴽⵣⵖ ⵉⵙ ⴷⴷ ⵉⴷⴷⴰ ⵓⴳⵍⴷⵓⵏ ⵎⵥⵥⵉⵢⵏ ⵙⴳ ⵉⵜⵔⵉ ⵎⵉ ⵇⵇⴰⵔⵏ ⴰⵙⵜⵔⵓⵉⴷ ⴱ612. ⴰⵙⵜⵔⵓⵉⴷ ⴰ, ⵓⵔ ⵉⵜⵓⵥⵔⴰ ⴰⵔ 1909 ⵙ ⵓⵜⵉⵍⵉⵙⴽⵓⴱ. ⵉⵥⵔⴰ ⵜ ⵢⴰⵏ ⵓⴰⵙⵜⵕⵓⵏⵓⵎ ⴰⵜⵓⵔⴽⵉⵢ. ⵉⵙⵙⴽⵏ ⵜⵓⴼⴰⵢⵜ ⵏⵏⵙ ⴳ ⵢⴰⵏ ⵓⴳⵔⴰⵡ ⴰⴳⵔⴰⵖⵍⴰⵏ ⵏ ⵍⴰⵙⵜⵕⵓⵏⵓⵎⵢ. ⵎⴰⵛⴰ, ⴰⴽⴷ ⵢⵉⵡⵏ ⵓⵔ ⵜ ⵢⵓⵎⵏ ⴰⵛⴽⵓ ⵉⵍⵍⴰ ⵉⵍⵙⴰ ⵢⴰⵜ ⵎⵍⵙⵉⵡⵜ ⵓⵔ ⵉⴳⵉⵏ ⴰⵎⵎ ⵜⵉⵏ ⵎⴷⴷⵏ. ⵎⴷⴷⵏ ⵉⵎⵇⵔⴰⵏⴻⵏ, ⴰⵎⴽⴰ ⴰⴽⴽ ⴰⵢ ⴳⴰⵏ.

ⵎⴰⵛⴰ ⵙ ⵓⵎⴷⴰⵣ ⵏ ⵜⵓⵙⵙⵏⴰ ⵏ ⴰⵙⵜⵔⵓⵉⴷ ⴱ612, ⵉⴽⴽⵔ ⵢⴰⵏ ⵓⴷⵉⴽⵜⴰⵜⵓⵔ ⴰⵜⵓⵔⴽⵢ, ⵉⴳⴳ ⴰⵙⵏ ⵛⵛⵉⵍ ⵉ ⵎⴷⴷⵏ ⴰⴷ ⵍⵙⵙⴰⵏ ⵎⵍⵙⵉⵡⵜ ⵏ ⵓⵔⵓⴱⵉⵢⵏ, ⵡⴰⵏⵏⴰ ⵢⴰⴳⵉⵏ ⵉⵏⵖ ⵜ. ⴰⵙⵜⵔⵓⵏⵓⵎ ⵏⵏⴰⵖ, ⵢⵓⵍⵙ ⴷⵉⵖ ⵉ ⵜⵎⵙⴽⴰⵏⵜ ⵏⵏⵙ ⴰⵙⴳⴳⴰⵙ ⵏ 1920, ⵜⵉⴽⴽⵍⵜ ⵏⵏⴰⵖ ⵉⵍⵍⴰ ⵉⵍⵙⴰ ⵢⴰⵜ ⵎⵍⵙⵉⵡⵜ ⵢⵖⵓⴷⴰⵏ ⵛⵉⴳⴰⵏ. ⵜⵉⴽⴽⵍⵜ ⵏⵏⴰⵖ, ⵎⴷⴷⵏ ⴰⴽⴽ ⵓⵎⴻⵏ ⴰⵡⴰⵍ ⵏⵏⵙ.
\par
\vspace*{10pt}
\end{minipage}
}

\subsection{Unified Canadian Aboriginal Syllabics}

Unified Canadian Aboriginal Syllabics is a Unicode block containing characters for writing Inuktitut, Carrier, several dialects of Cree, and Canadian Athabascan languages. Additions for some Cree dialects, Ojibwe, and Dene can be found at the Unified Canadian Aboriginal Syllabics Extended block.
\medskip

\newfontfamily\aboriginal{code2000.ttf}
\bgroup
\par
\noindent
\colorbox{graphicbackground}{\color{black}^^A
\begin{minipage}{\textwidth}^^A
\parindent1pt
\vskip10pt
\leftskip10pt \rightskip\leftskip

\aboriginal
ᒥᓯᐌ ᐃᓂᓂᐤ ᑎᐯᓂᒥᑎᓱᐎᓂᐠ ᐁᔑ ᓂᑕᐎᑭᐟ ᓀᐢᑕ ᐯᔭᑾᐣ ᑭᒋ ᐃᔑ
\bfseries ᑲᓇᐗᐸᒥᑯᐎᓯᐟ ᑭᐢᑌᓂᒥᑎᓱᐎᓂᐠ ᓀᐢᑕ ᒥᓂᑯᐎᓯᐎᓇ᙮
Unicode Block: Unified Canadian Aboriginal Syllabics, UCAS Extended
Text: UDHR: Cree, Swampy ᐯᔭᐠ ᐱᐢᑭᑕᓯᓇᐃᑲᐣ ᐁᐢᐱᑕᐢᑲᒥᑲᐠ ᐊᐢᑭᐠ ᑭᒋ ᐃᑗᐎᐣ ᐃᓂᓂᐎ ᒥᓂᑯᐎᓯᐎᓇ ᐅᒋ
\par
\vspace*{10pt}
\end{minipage}
}
\medskip
\egroup
\subsection{Miao}

The Pollard script, also known as Pollard Miao (Chinese: 柏格理苗文 Bó Gélǐ Miao-wen) or Miao, is an abugida loosely based on the Latin alphabet and invented by Methodist missionary Sam Pollard. Pollard invented the script for use with A-Hmao, one of several Miao languages. The script underwent a series of revisions until 1936, when a translation of the New Testament was published using it. The introduction of Christian materials in the script that Pollard invented caused a great impact among the Miao. Part of the reason was that they had a legend about how their ancestors had possessed a script but lost it. According to the legend, the script would be brought back some day. When the script was introduced, many Miao came from far away to see and learn it.[1][2]

Pollard credited the basic idea of the script to the Cree syllabics designed by James Evans in 1838–1841, “While working out the problem, we remembered the case of the syllabics used by a Methodist missionary among the Indians of North America, and resolved to do as he had done” (1919:174). He also gave credit to a Chinese pastor, “Stephen Lee assisted me very ably in this matter, and at last we arrived at a system” (1919:174). In listing the phrases he used to describe devising the script, there is clear indication of intellectual work, not revelation: “we looked about”, “resolved to attempt”, “adapting the system”, “solved our problem” (Pollard 1919:174,175).

Changing politics in China led to the use of several competing scripts, most of which were romanizations. The Pollard script remains popular among Hmong in China, although Hmong outside China tend to use one of the alternative scripts. A revision of the script was completed in 1988, which remains in use.

As with most other abugidas, the Pollard letters represent consonants, whereas vowels are indicated by diacritics. Uniquely, however, the position of this diacritic is varied to represent tone. For example, in Western Hmong, placing the vowel diacritic above the consonant letter indicates that the syllable has a high tone, whereas placing it at the bottom right indicates a low tone.

A still experimental font, that supports Graphite technology is \idxfont{Mia Unicode}\footnote{\url{http://phjamr.github.io/miao.html\#intro}}. The font is licenced under the SIL terms and we are using it in the |phd| package as the default font for the Miao script.

\newfontfamily\miao{MiaoUnicode-Regular.ttf}

\begin{scriptexample}[]{Miao}
\unicodetable{miao}{"16F00,"16F10,"16F20,"16F30,"16F40,"16F70,"16F80,"16F90}
\end{scriptexample}

{\miao 𖼴	𖼵	𖼶	𖼷	𖼸	𖼹	𖼺	}

Features for Miao
There are three features currently available for the Miao script:
\bgroup
\miao
Chuxiong ‘wart’ variant
Stylistic alternates for 𖼳 and 𖼴
Aspiration marker always on right
The ‘wart’ (a translated technical term!) is the small circle in characters like 𖼁, 𖼅, and 𖼾. In the Chuxiong orthography, it is rendered not as a circle but as a dot on the right of the letter, as shown in point 5 here (pdf).

Miao Unicode has a feature called “chux” for handling this. In LibreOffice you can use this style by typing “Miao Unicode:chux=1” into the font field.
\section{N'ko}

\newfontfamily\nko{NotoSansNKo-Regular.ttf}

N'Ko {\nko(ߒߞߏ)} is both a script devised by Solomana Kante in 1949 as a writing system for the Manding languages of West Africa, and the name of the literary language itself written in the script. The term N'Ko means ``I say'' in all Manding languages.

The script has a few similarities to the Arabic script, notably its direction (right-to-left) and the connected letters. It obligatorily marks both tone and vowels.


\begin{scriptexample}[]{N'ko}
\unicodetable{nko}{"07C0,"07D0,"07E0,"07F0}
\end{scriptexample}

The N'Ko alphabet is written from right to left, with letters being connected to one another.

The script is principally used in Guinea and Côte d'Ivoire (respectively by Maninka and Dioula-speakers), with an active user community in Mali (by Bambara-speakers). Publications include a translation of the Qur'an, a variety of textbooks on subjects such as physics and geography, poetic and philosophical works, descriptions of traditional medicine, a dictionary, and several local newspapers. It has been classed as the most successful of the West African scripts.[3] The literary language used is intended as a koine blending elements of the principal Manding languages (which are mutually intelligible), but has a particularly strong Maninka flavour.

The Latin script with several extended characters (phonetic additions) is used for all Manding languages to one degree or another for historical reasons and because of its adoption for "official" transcriptions of the languages by various governments. In some cases, such as with Bambara in Mali, promotion of literacy using this orthography has led to a fair degree of literacy in it. Arabic transcription is commonly used for Mandinka in The Gambia and Senegal.


\subsection{Mongolian}
\newfontfamily\mongolian{NotoSansMongolian-Regular.ttf}

The classical Mongolian script (in Mongolian script:{\mongolian ᠮᠣᠩᠭᠣᠯ ᠪᠢᠴᠢᠭ᠌} Mongγol bičig; in Mongolian Cyrillic: Монгол бичиг Mongol bichig), also known as Uyghurjin Mongol bichig, was the first writing system created specifically for the Mongolian language, and was the most successful until the introduction of Cyrillic in 1946. Derived from Uighur, Mongolian is a true alphabet, with separate letters for consonants and vowels. The Mongolian script has been adapted to write languages such as Oirat and Manchu. Alphabets based on this classical vertical script are used in Inner Mongolia and other parts of China to this day to write Mongolian, Sibe and, experimentally, Evenki.

\begin{scriptexample}[]{Mongolian}
\unicodetable{mongolian}{"1820,"1830,"1840,"1850,"1860,"1870,"1880,"1890,"18A0}
\end{scriptexample}


%\chapter{Middle Eastern Scripts}

The scripts in this section have a common origin in the ancient Phoenician alphabet. They include:

\begin{center}
\begin{tabular}{ll}
Hebrew & Samaritan\\
Arabic & Thaana\\
Syriac &\\
\end{tabular}
\end{center}

The Hebrew script is used in Israel and for languages of the Diaspora. The Arabic script is
used to write many languages throughout the Middle East, North Africa, and certain parts
of Asia. The Syriac script is used to write a number of Middle Eastern languages. These
three also function as major liturgical scripts, used worldwide by various religious groups.

The Samaritan script is used in small communities in Israel and the Palestinian Territories
to write the Samaritan Hebrew and Samaritan Aramaic languages. The Thaana script is
used to write Dhivehi, the language of the Republic of Maldives, an island nation in the
middle of the Indian Ocean. 

Text in these scripts is written from right to left. Arabic and Syriac are cursive scripts even when typeset, unlike Hebrew, Samaritan  and Thaana, where letters are unconnected. Most letters in Arabic and Syriac assume different forms depending on their position in a word. Shaping rules are not required for Hebrew because only five letters have position-dependent forms, and these forms are separately encoded.

Historically, Middle Eastern  scripts did not write short vowels. In modern scripts they are represented  by marks positioned above or below a consonantal letter. Vowels and other
marks of pronunciation (“vocalization”) are encoded as combining characters, so support
for vocalized text necessitates use of composed character sequences. Yiddish, Syriac, and
Thaana are normally written with vocalization; Hebrew, Samaritan, and Arabic are usually written unvocalized. 

\section{Hebrew}
\newfontfamily\hebrew{Miriam}
\fontspec{Arial Unicode MS}
To properly typeset Hebrew texts you first need to choose an appropriate font and also set the directionality of the text. This
is done using the etex commands:

\CMDI{\beginL} and \CMDI{\beginR} 

For \XeTeX\ you also need to add near the top of your document |\TeXXeTstate=1|. The package \pkgname{bidi} can be used to set all parameters. Be warned that it redefines almost all of \latexe's commands, so for short mixed texts, I wouldn't recommend its usage. 



The Hebrew alphabet (Hebrew: אָלֶף־בֵּית עִבְרִי[a], alefbet ʿIvri ), known variously by scholars as the Jewish script, square script, block script, is used in the writing of the Hebrew language, as well as other Jewish languages, most notably Yiddish, Ladino, and Judeo-Arabic. There have been two script forms in use; the original old Hebrew script is known as the paleo-Hebrew script (which has been largely preserved, in an altered form, in the Samaritan script), while the present "square" form of the Hebrew alphabet is a stylized form of the Assyrian script. Various "styles" (in current terms, "fonts") of representation of the letters exist. There is also a cursive Hebrew script, which has also varied over time and place. On Windows you can use the \texttt{Miriam} font or \texttt{Arial Unicode MS} or \texttt{Miriam Fixed}.
\medskip

\topline

\bgroup\TeXXeTstate=1
\raggedleft\hebrew{}\beginR

הכתב הכנעני הקדום הלך והתפשט וסימניו היו מוכרים כל כך, עד כי המשתמשים בו התחילו "להתעצל" בהשלמת הציורים, והניחו כי הקורא יבין גם מתוך שרטוטים סכמתיים באיזו אות מדובר. כך, למשל, הפך הראש למשולש עם צוואר; כף היד מלאת האצבעות הפכה לשרטוט דל, ומהדג נותר רק הזנב. כשהעברים אמצו את הכתב הכנעני הם התקשו לזהות חלק מהציורים המקוריים והניחו למשל כי הסימן המתאר את המילה "זהה" הוא כלי נשק; שזנב הדג המשולש הוא דלת, ושדווקא הנחש הוא דג. כך נולדו שמותיהם העבריים של האותיות זי"ן, דל"ת ונו"ן (נון הוא דג, כמו אמנון, שפמנון וכו'). הציורים שהפכו לסימנים התגלגלו לכתבים נוספים, ואפילו ליוונית וללטינית. גם בכתב העברי המודרני ניתן לזהות המשך התפתחותי ברור מן הכתב הכנעני הקדום, והשתמרות שמות האותיות מקלה מאוד על פענוח המקור.


בתקופת בית שני, אומץ האלפבית הארמי לשימוש השפה העברית במקום האלפבית העברי העתיק, כאשר בזה האחרון נעשה שימוש מועט כגון כתיבת השמות הקדושים והטבעת מטבעות. עם הזמן, נעלם גם שימוש זה של הכתב העתיק. האלפבית העברי של ימינו הוא אפוא פיתוח של האלפבית הארמי ולא של הכתב העברי העתיק.	
{}

 לֹ֥א תִשָּׂ֛א

\endR


\egroup
\bottomline
\medskip

To make all paragraphs  RL use the \cmd{\everypar}\footnote{See discussions at \url{http://tex.stackexchange.com/questions/141867/minimal-bidi-for-typesetting-rl-text} and \url{http://www.tug.org/pipermail/xetex/2004-August/000697.html}}. 

\begin{verbatim}
\newbox\mybox \everypar{\setbox\mybox\lastbox\beginR\box\mybox}
\everypar={% at the start of each paragraph, do....
    \setbox0=\lastbox % save the paragraph indent, if any
    \beginR % set R-L direction
    \box0 % then re-insert the indent
	}
\end{verbatim}

The Hebrew alphabet has 22 letters, of which five have different forms when used at the end of a word. Hebrew is written from right to left. Originally, the alphabet was an abjad consisting only of consonants. Like other \textit{abjads}, such as the Arabic alphabet, means were later devised to indicate vowels by separate vowel points, known in Hebrew as niqqud. In rabbinic Hebrew, the letters א ה ו י are also used as matres lectionis to represent vowels. When used to write Yiddish, the writing system is a true alphabet (except for borrowed Hebrew words). In modern usage of the alphabet, as in the case of Yiddish (except that ע replaces ה) and to some extent modern Israeli Hebrew, vowels may be indicated. Today, the trend is toward full spelling with these letters acting as true vowels.


\subsection{Syriac}

\newfontfamily\syriac{Estrangelo Edessa}

Syriac /ˈsɪriæk/ ({\syriac{ܠܫܢܐ ܣܘܪܝܝܐ}} Leššānā Suryāyā) is a dialect of Middle Aramaic that was once spoken across much of the Fertile Crescent and Eastern Arabia.[1][2][5] Having first appeared as a script in the 1st century AD after being spoken as an unwritten language for five centuries,[6] Classical Syriac became a major literary language throughout the Middle East from the 4th to the 8th centuries,[7] the classical language of Edessa, preserved in a large body of Syriac literature.
It became the vehicle of Syriac Christianity and culture, spreading throughout Asia as far as the Indian Malabar Coast and Eastern China,[8] and was the medium of communication and cultural dissemination for Arabs and, to a lesser extent, Persians. Primarily a Christian medium of expression, Syriac had a fundamental cultural and literary influence on the development of Arabic,[9] which largely replaced it towards the 14th century.[3] Syriac remains the liturgical language of Syriac Christianity.
Syriac is a Middle Aramaic language, and, as such, it is a language of the Northwestern branch of the Semitic family. It is written in the Syriac alphabet, a derivation of the Aramaic alphabet.

\begin{scriptexample}[]{Syriac}
\unicodetable{syriac}{"0700,"0710,"0720,"0730,"0740}
\end{scriptexample}

The Syriac Abbreviation (a type of overline) can be represented with a special control character called the Syriac Abbreviation Mark (U+070F {\syriac \char"070F ܘ}).

\section{Samaritan}
\newfontfamily\samaritan{NotoSansSamaritan-Regular.ttf}

The Samaritan alphabet is used by the Samaritans for religious writings, including the Samaritan Pentateuch, writings in Samaritan Hebrew, and for commentaries and translations in Samaritan Aramaic and occasionally Arabic.

The Samaritans are, consider themselves to be the descendants of the Northern Tribes of Israel that were not sent into Assyrian captivity, and have continuously resided in the land of Israel.

The Torah Scroll of the Samaritans uses an alphabet that is very different from the one used on Jewish Torah Scrolls. According to the Samaritans themselves and Hebrew scholars, this alphabet is the original "Old Hebrew" alphabet.

Even as far back as 1691, this connection between the Samaritan and the "Old" Hebrew alphabets was made by Henry Dodwell; "[the Samaritans] still preserve [the Pentateuch] in the Old Hebrew characters."

Samaritan is a direct descendant of the Paleo-Hebrew alphabet, which was a variety of the Phoenician alphabet in which large parts of the Hebrew Bible were originally penned. All these scripts are believed to be descendants of the Proto-Sinaitic script. That script was used by the ancient Israelites, both Jews and Samaritans. The better-known "square script" Hebrew alphabet traditionally used by Jews is a stylized version of the Aramaic alphabet which they adopted from the Persian Empire (which in turn adopted it from the Arameans). 

After the fall of the Persian Empire, Judaism used both scripts before settling on the Aramaic form. For a limited time thereafter, the use of paleo-Hebrew (proto-Samaritan) among Jews was retained only to write the Tetragrammaton, but soon that custom was also abandoned.



ShofarRegular StamAshkenazCLM.ttf

\begin{scriptexample}[]{Samaritan}
\bgroup
\TeXXeTstate=1
\unicodetable{samaritan}{"0800,"0810,"0820,"0830}
\egroup
\TeXXeTstate=0
\end{scriptexample}

I battled to get an appropriate font for the Samaritan script and had to use the \idxfont{Noto Sans Samaritan} from Google


^^A\printunicodeblock{./languages/samaritan.txt}{\samaritan}


\url{http://www.ancient-hebrew.org/ahh/ahh.htm#_Toc314842274}




\section{Arabic}

\newfontfamily\arabian{Scheherazade-R.ttf}

The Arabic script is a writing system used for writing several languages of Asia and Africa, such as Arabic, Sorani and Luri Dialects of Kurdish language, Persian, Pashto and Urdu.[1] Even until the 16th century, it was used to write some texts in Spanish.[2] After the Latin script, Chinese characters, and Devanagari, it is the fourth-most widely used writing system in the world.[3]
The Arabic script is written from right to left in a cursive style. In most cases the letters transcribe consonants, or consonants and a few vowels, so most Arabic alphabets are abjads.

The script was first used to write texts in Arabic, most notably the Qurʼān, the holy book of Islam. With the spread of Islam, it came to be used to write languages of many language families, leading to the addition of new letters and other symbols, with some versions, such as Kurdish, Uyghur, and old Bosnian being abugidas or true alphabets. It is also the basis for a rich tradition of Arabic calligraphy.

\begin{verbatim}
\begin{Arabic}
ّ هو إذ الغاية؛ شريف الفوائد، جم المذهب، عزيز فنّ التاريخ فنّ أنّ اعلم
والملوك سيرهم، في والأنبياء أخلاقهم، في الأمم من الماضين أحوال على يوقفنا
ّ أحوال في يرومه لمن ذلك في الإقتداء فائدة تتم حتّى وسياستهم؛ دولهم في
والدنيا. الدين
\end{Arabic}
\end{verbatim}




As of Unicode 7.0, the Arabic script is contained in the following blocks:
Arabic (0600—06FF, 255 characters)
Arabic Supplement (0750—077F, 48 characters)
Arabic Extended-A (08A0—08FF, 39 characters)
Arabic Presentation Forms-A (FB50—FDFF, 608 characters)
Arabic Presentation Forms-B (FE70—FEFF, 140 characters)
Rumi Numeral Symbols (10E60—10E7F, 31 characters)
Arabic Mathematical Alphabetic Symbols (1EE00—1EEFF, 143 characters)[1][2]

The basic Arabic range encodes the standard letters and diacritics, but does not encode contextual forms (U+0621–U+0652 being directly based on ISO 8859-6); and also includes the most common diacritics and Arabic-Indic digits. The Arabic Supplement range encodes letter variants mostly used for writing African (non-Arabic) languages. The Arabic Extended-A range encodes additional Qur'anic annotations and letter variants used for various non-Arabic languages. The Arabic Presentation Forms-A range encodes contextual forms and ligatures of letter variants needed for Persian, Urdu, Sindhi and Central Asian languages. The Arabic Presentation Forms-B range encodes spacing forms of Arabic diacritics, and more contextual letter forms. The presentation forms are present only for compatibility with older standards, and are not currently needed for coding text.[3] 

The Arabic Mathematical Alphabetical Symbols block encodes characters used in Arabic mathematical expressions.

\begin{multicols}{3}
\printunicodeblock{./languages/arabic.txt}{\arabian}
\end{multicols}









\section{Thaana}

\newfontfamily\thaana{MV Boli}
Thaana, Taana or Tāna ({\thaana  ތާނަ}‎ in Tāna script) is the modern writing system of the Maldivian language spoken in the Maldives. Thaana has characteristics of both an abugida (diacritic, vowel-killer strokes) and a true alphabet (all vowels are written), with consonants derived from indigenous and Arabic numerals, and vowels derived from the vowel diacritics of the Arabic abjad. Its orthography is largely phonemic.

The Thaana script first appeared in a Maldivian document towards the beginning of the 18th century in a crude initial form known as Gabulhi Thaana which was written scripta continua. This early script slowly developed, its characters slanting 45 degrees, becoming more graceful and spaces were added between words. 

As time went by it gradually replaced the older Dhives Akuru alphabet. The oldest written sample of the Thaana script is found in the island of Kanditheemu in Northern Miladhunmadulu Atoll. It is inscribed on the door posts of the main Hukuru Miskiy (Friday mosque) of the island and dates back to 1008 AH (AD 1599) and 1020 AH (AD 1611) when the roof of the building were built and the renewed during the reigns of Ibrahim Kalaafaan (Sultan Ibrahim III) and Hussain Faamuladeyri Kilege (Sultan Hussain II) respectively.

\begin{scriptexample}[]{Thaana}
\unicodetable{thaana}{"0780,"0790,"07A0,"07B0}

\hfill Typeset with MV Boli and the command \cmd{\thaana}.
\end{scriptexample}


^^A\printunicodeblock{./languages/thaana.txt}{\thaana}



\endinput












\nocite{*}
\printbibliography
\printindex
\end{document}
%</driver>
% \fi
% 
%  \CheckSum{0}
%  \CharacterTable
%  {Upper-case    \A\B\C\D\E\F\G\H\I\J\K\L\M\N\O\P\Q\R\S\T\U\V\W\X\Y\Z
%   Lower-case    \a\b\c\d\e\f\g\h\i\j\k\l\m\n\o\p\q\r\s\t\u\v\w\x\y\z
%   Digits        \0\1\2\3\4\5\6\7\8\9
%   Exclamation   \!     Double quote  \"     Hash (number) \#
%   Dollar        \$     Percent       \%     Ampersand     \&
%   Acute accent  \'     Left paren    \(     Right paren   \)
%   Asterisk      \*     Plus          \+     Comma         \,
%   Minus         \-     Point         \.     Solidus       \/
%   Colon         \:     Semicolon     \;     Less than     \<
%   Equals        \=     Greater than  \>     Question mark \?
%   Commercial at \@     Left bracket  \[     Backslash     \\
%   Right bracket \]     Circumflex    \^     Underscore    \_
%   Grave accent  \`     Left brace    \{     Vertical bar  \|
%   Right brace   \}     Tilde         \~}
%
%
%
% \changes{1.0}{2013/01/26}{Converted to DTX file}
%
% \DoNotIndex{\newcommand,\newenvironment}
% \GetFileInfo{phd.dtx}
% 
%  \def\fileversion{v1.0}          
%  \def\filedate{2012/03/06}
% \title{The \textsf{phd} package.
% \thanks{This
%        file (\texttt{phd.dtx}) has version number \fileversion, last revised
%        \filedate.}
% }
% \author{Dr. Yiannis Lazarides \\ \url{yannislaz@gmail.com}}
% \date{\filedate}
%
%
% 
% ^^A\maketitle
% 
% ^^A\frontmatter
%  ^^A\coverpage{./images/hine02.jpg}{Book Design }{Camel Press}{}{}
%  \newpage
% ^^A\secondpage
% \pagestyle{empty}
%
%
% 
%
%
% \pagestyle{headings}
% \raggedbottom
%  \OnlyDescription
%
% ^^A\StopEventually{\printindex}

% \CodelineNumbered
% \pagestyle{plain}
% 
% 
% ^^A\part{IMPLEMENTATION AND FRIENDS}
% 
%
% \chapter{Code Implementation Objectives and Strategy}
%
% \makeatletter
%  %%
%% This is file `phd-colorpalette.sty',
%% generated with the docstrip utility.
%%
%% The original source files were:
%%
%% phd-colorpalette.dtx  (with options: `PLT')
%% ----------------------------------------------------------------
%% phd --- A package to beautify documents.
%% E-mail: yannislaz@gmail.com
%% Released under the LaTeX Project Public License v1.3c or later
%% See http://www.latex-project.org/lppl.txt
%% ----------------------------------------------------------------



\NeedsTeXFormat{LaTeX2e}[1994/12/01]%
\RequirePackage[2014/05/01]{latexrelease}
\ProvidesFile{phd-colorpalette}[2015/1/13 v1.0 color palettes (YL)]%
\@ifpackageloaded{xcolor}{}%
 {\PassOptionsToPackage{\xcolorkeys@cx}{xcolor}
  \RequirePackage{xcolor}}
\definecolor{glyphbox}{rgb}{0.86,0.86,0.8}
\definecolor{theblue} {rgb}{0.02,0.04,0.48}
\definecolor{thered}  {rgb}{0.65,0.04,0.07}

\colorlet{thefontname}{black}%font examples

\colorlet{thehighlight}{yellow}%soul  highlight
\colorlet{thecancel}{thered}%for cancel commands
\definecolor{thegreen}{rgb}{0.06,0.44,0.08}
\definecolor{thelightgreen}{rgb}{0.06,0.44,0.06}
\definecolor{thegrey} {gray}{0.5}
\definecolor{thegray} {gray}{0.5}
\definecolor{thedarkgray} {gray}{0.95}
\definecolor{lightgray}{gray}{0.6}
\definecolor{shadedcolor}{gray}{0.6}
\definecolor{thelightgray}{gray}{0.6}
\definecolor{theshade}{gray}{0.94}
\definecolor{theframe}{gray}{0.75}
\definecolor{thecream}{rgb}{1,0.95,0.4}
\definecolor{spot}{rgb}{0,0.2,0.6}%some shades of blue
\definecolor{sweet}{rgb}{0,.68,.93}%shades of blue
\definecolor{boxframe}{gray}{0.8}
\definecolor{boxfill}{rgb}{0.95,0.95,0.99}
\definecolor{theoption}{gray}{0.6}

\definecolor{ExampleFrame}{rgb}{0.628,0.705,0.942}
\definecolor{ExampleBack}{rgb}{0.963,0.971,0.994}
\colorlet{preciscolor}{sweet}
\colorlet{toccolor}{sweet}
\definecolor{creamy}{HTML}{FDEBD7}

\ExplSyntaxOn
\cs_gset:Npn \createpalette #1#2#3
  { \tl_gset:cn {auxpalette#1_tl} {}
    \addtotl {#1}{#2}{#3}
    \cxset{palette~#1/.code  = \cs:w auxpalette#1_tl \cs_end: }
}

\cs_set:Npn \addtotl #1 #2 #3
  {
    \tl_put_left:cn {auxpalette#1_tl}
      {
        \definecolor{bgsexy}{HTML}{#2} %{D11C23}
        \colorlet{thechaptercolor}{bgsexy} %font examples
        \colorlet{thesectioncolor}{bgsexy} %font examples
        \colorlet{thesubsectioncolor}{bgsexy} %font examples
        \colorlet{thesubsubsectioncolor}{bgsexy} %font examples
        \colorlet{theparagraphcolor}{bgsexy} %font examples
        \colorlet{thesubparagraphcolor}{bgsexy} %font examples
        \colorlet{thesectionnumbercolor}{bgsexy}
        \colorlet{thesubsectionnumbercolor}{bgsexy}
        \colorlet{thelinkcolor}{bgsexy}
        \colorlet{thecommentstyle}{thegray}
        \colorlet{thecodebackground}{bgsexy!10}
        % used for option
        \colorlet{theoption}{black}
        % headers
        \colorlet{theplainoddheaderbgcolor}{white}
        \colorlet{theplainevenheaderbgcolor}{white}
        \colorlet{theplainoddfooterbgcolor}{white}
        \colorlet{theplainevenfooterbgcolor}{white}
        %
        \colorlet{theheadingsoddheaderbgcolor}{white}
        \colorlet{theheadingsevenheaderbgcolor}{white}
        \colorlet{theheadingsoddfooterbgcolor}{white}
        \colorlet{theheadingsevenfooterbgcolor}{white}
        #3
        % rules
        \colorlet{theplainoddheaderframerule}{bgsexy}
        \colorlet{theplainoddfooterframerule}{bgsexy}
        \colorlet{theplainevenheaderframerule}{bgsexy}
        \colorlet{theplainevenfooterframerule}{bgsexy}
        \colorlet{theheadingsoddheaderframerule}{bgsexy}
        \colorlet{theheadingsoddfooterframerule}{bgsexy}
        \colorlet{theheadingsevenheaderframerule}{bgsexy}
        \colorlet{theheadingsevenfooterframerule}{bgsexy}
        % phd doc specific
       \colorlet{theunicodesymbolcolor}{bgsexy}
    }
}

\createpalette {esquire}       {D11C23} {}  % red shade
\createpalette {fortune}       {EA8A4E} {}  % brick color
\createpalette {oprah}         {F060A8} {}  % nice pinkish modern
\createpalette {vogue}         {F21C93} {}  % nice pinkish modern
\createpalette {architectural} {0168FD} {}  % blue shade
\createpalette {instyle}       {227CE8} {}  % blue shade
\createpalette {smithsonian}   {60A8C0} {}  % milky blue
\createpalette {blueprint}     {486090} {}  % dark milky blue

\createpalette {knoll}         {88A65E} {}  % nice sweet green
\createpalette {living}        {678756} {}  % green shade
\createpalette {spring~onion}  {90D228} {}  % bright green shade
\createpalette {olive}         {EED38D} {}  % washed out not nice
\createpalette {zealous}       {075D6B} {}  %
\createpalette {orange~sakura} {E6781E} {}  % nice modern serious book
\createpalette {orange}        {FF6927} {}  % bright orange
\createpalette {brown}         {AF0C39} {}  % red brown
\createpalette {brown~red}     {8D2420} {}  % brown red, serious luxury
\createpalette {black~tulip}   {420943} {}  % darkish purple
\createpalette {helvetica}     {404547} {}  % darkish purple

\tl_put_right:cn {auxpaletteblueprint_tl}
  {
    \colorlet{theunicodesymbolcolor}{bgsexy}
    \colorlet{thecodebackground}{thelightgray!20}
  }
\ExplSyntaxOff
\cxset{palette smithsonian}

\colorlet{thelinkcolor}{bgsexy}   %linkcolor
\colorlet{theanchorcolor}{bgsexy} %anchorcolor
\colorlet{thecitecolor}{bgsexy}   %citecolor
\colorlet{thefilecolor}{bgsexy}   %filecolor
\colorlet{themenucolor}{bgsexy}   %menucolor
\colorlet{theruncolor}{bgsexy}    %runcolor
\colorlet{theurlcolor}{bgsexy}    %urlcolor
\definecolor{Hyperlink}{rgb}{0.281,0.275,0.485}
\definecolor{lstbgcolor}{rgb}{0.9,0.9,0.9}
\colorlet{examplefill}{yellow!80!black}

\definecolor{thekeywordstyle}{HTML}{000000}
\definecolor{thestringcolor}{HTML}{DF8743}
\definecolor{theemphasiscolor}{HTML}{DF8743}
\definecolor{theframerulecolor}{HTML}{FF6927}
\definecolor{theframefillcolor}{HTML}{000000}
\definecolor{theidentifiercolor}{HTML}{000000}
\definecolor{themacro}{rgb}{0.784,0.06,0.176}
\colorlet{thecmdcolor}{bgsexy}
\def\done{\cellcolor{teal}done}
\def\partialdone{\cellcolor{yellow}done}
\def\hcyan#1{{\color{teal} #1}}
\endinput
%%
%% End of file `phd-colorpalette.sty'.

% \makeatother
%
% \epigraph{
% It's a good thing that when God created the rainbow he didn't consult a decorator or he would still be picking colors.
% }{Sam Levenson}
% 
% 
%
% Having provided for individual control for virtually all document elements, we need
% some form of organizing all the color keys. This package defines different color
% palettes, that make it easier to change all the colors of a document with a single
% key setting (provided of course we have a predefined palette).
%
% We start by outlining what we are trying to achieve with the \pkgname{phd-colorpalette}
% package.
%
% \begin{enumerate}
% \item To provide a declarative interface to enable users to modify colors
%       specifications as groups rather than individual elements..  
% \item To provide a number of predefined color schemes.
% \item To provide a plug-in architecture for extensions.
% \end{enumerate}
% 
% \section{Terminology}
%
%  \begin{description}
%  \item [document] Any written item, as a book, article, or letter, especially 
%                  of a factual or informative nature.
%  \item [heading] A division of a document or document series. For a normal
%        book headings are chapters, sections etc. However we allow for
%        specifying a more complex document divided into books, volumes
%        parts etc. For example the Bible has Books, chapters and verses,
%        where a legal document might require divisions such as clauses.
%        In general these divisions are numbered. These document divisions
%        are stored in the comma list 
%  \item [head] A typeset heading, such as chapter head, or section head.
%        This can include a counter, label and title for example, 
%        \emph{Chapter 1 Introduction}.
%  \item [dom] This is a programming interface that provides a structured
%        representation of the document (a tree) and it defines a way
%        that the structure can be accessed. Although \latexe does not
%        offer a standard way to build such a tree (mainly because
%        \tex does not require the marking of paragraphs, it is 
%        useful to think of the document as a tree structure. We also
%        allow for a semi-automated way to build such a tree (with the 
%        exception that paragraphs are not included).
% \item [element] A part of the document tree that can be styled on
%       its own. For example the chapter label, or the section number.
%
% \end{description}
%
% \section{Users}
%  We classify users according to the \LaTeX3 terminology as a) programmers b) template designers
%  and c) authors.
% \subsection{Author}
%  We assume that the author has an exising template which she is using but might want to do
%  some minor modifications, for example use an italic shape for the font of the mark, but an 
%  upright font for the page numbers. 
%
%
% We follow the idea of representing the basic elements of documents
% as elements, each one having a parent in order to specify
% the element we need to style as accurate as possible. One can think of
% this approach being congruent with objects in other languages.
% As a matter fact nothing stops us from defining a key value
% interface as shown below.
%
%
% This would pehaps make it easier for the template designer, but I have rejected
% the idea as my aim is to make it easy for the author, who can search the template
% and just enter a couple of new proerty values.\url{http:\\test}
%
% \subsection{Template designer}
% \
% The template designer in the example above would have selected the format style
% from a number of predefined formats (templates) or would have created a style
% called \textit{apa} from an existing template and modified it using declarative
% key style.
%
% \subsection{The Programmer}
%
% The programmer in the example above could have created the basic format
% \textit{apa} by using both declarative as well as defining or using existing
% macros. To the programmer we offer an extension mechanism, where the contents
% of a |ps@| command are defined. For example the programmer can define a new
% style using \tikzname, but without having to worry about defining full |ps@|
% and their interface.
%
% \section{Color Palettes}
%
% Although documents would normally have only a small number of colors, the
% number of variables that need a color setting is quite large. In this package
% we introduce the option of a \textit{color palette}, where all color settings can be
% in one place.\index{color palette}
%

% \begin{dispListing}
% \cxset{palette bbc}
% \end{dispListing}
%
%
% \section{Preliminaries}
%
%  Standard file identification. We first announce the package 
%	 and require that it be used with \LaTeX2e. \tcbdocmarginnote{need to upgrade}
% \iffalse
%<*PLT>
% \fi
%    \begin{macrocode}
\NeedsTeXFormat{LaTeX2e}[2017/04/15]%
\RequirePackage[2017/04/15]{latexrelease}
\ProvidesFile{phd-colorpalette}[2015/1/13 v1.0 color palettes (YL)]%
%    \end{macrocode}
%
%    \begin{macrocode}
\@ifpackageloaded{xcolor}{}%
 {\PassOptionsToPackage{\xcolorkeys@cx}{xcolor}
  \RequirePackage{xcolor}}
%    \end{macrocode}
%
%    \begin{macrocode}
\definecolor{glyphbox}{rgb}{0.86,0.86,0.8}
%\definecolor{codebackground}{rgb}{0.8,0.8,1}
\definecolor{theblue} {rgb}{0.02,0.04,0.48}
\definecolor{thered}  {rgb}{0.65,0.04,0.07}


\colorlet{thefontname}{black}%font examples

\colorlet{thehighlight}{yellow}%soul  highlight
\colorlet{thecancel}{thered}%for cancel commands
\definecolor{thegreen}{rgb}{0.06,0.44,0.08}
\definecolor{thelightgreen}{rgb}{0.06,0.44,0.06}
\definecolor{thegrey} {gray}{0.5}
\definecolor{thegray} {gray}{0.5}
\definecolor{thedarkgray} {gray}{0.95}
\definecolor{lightgray}{gray}{0.6}
\definecolor{shadedcolor}{gray}{0.6}
\definecolor{thelightgray}{gray}{0.6}
\definecolor{theshade}{gray}{0.94}
\definecolor{theframe}{gray}{0.75}
\definecolor{thecream}{rgb}{1,0.95,0.4}
\definecolor{spot}{rgb}{0,0.2,0.6}%some shades of blue
\definecolor{sweet}{rgb}{0,.68,.93}%shades of blue
%\colorlet{codebackground}{spot!5!white}
\definecolor{boxframe}{gray}{0.8}
\definecolor{boxfill}{rgb}{0.95,0.95,0.99}
\definecolor{theoption}{gray}{0.6}

\definecolor{ExampleFrame}{rgb}{0.628,0.705,0.942}
\definecolor{ExampleBack}{rgb}{0.963,0.971,0.994}
\colorlet{preciscolor}{sweet}
\colorlet{toccolor}{sweet}
\definecolor{creamy}{HTML}{FDEBD7}
\definecolor{tofu}{HTML}{e7e3d8}
% check this
\newcommand{\TODO}{\textcolor{red}{\bf TODO!}\xspace}
%    \end{macrocode}

% \section{Headings}
%
% The colors \docColor{thechaptercolor}, \docColor{thesectioncolor} and similar color
% definitions are used to color the title of a heading command. 
%
% \section{Creating new palettes}
%
%  \begin{docCommand}{createpalette} { \marg{name of color palette} \marg{hex color code} \marg{optional extra code} } 
%    Creates a palette, given a hex code and a name.
%  \end{docCommand}
%
%  Following the design pattern in other sections we try to generalize the command.
%  Given a color applicable to heading titles, I have set the color commands as best 
%  as I could. For some color combinations this is not satisfactory and modifications
%  are necessary. This can be done by the |\addtocolorpalette| function. 
%    \begin{macrocode} 
\ExplSyntaxOn  
\cs_gset:Npn \createpalette #1#2#3
  { \tl_gset:cn {auxpalette#1_tl} {} 
    \addtotl {#1}{#2}{#3}
    \cxset{palette~#1/.code  = \cs:w auxpalette#1_tl \cs_end: }
    \definecolor{primary}{HTML}{#2}
}

\cs_set:Npn \addtotl #1 #2 #3 
  {
    \tl_put_left:cn {auxpalette#1_tl} 
      {
        \definecolor{bgsexy}{HTML}{#2} %{D11C23}
        \colorlet{thechaptercolor}{bgsexy} %font examples
        \colorlet{thesectioncolor}{bgsexy} %font examples
        \colorlet{thesubsectioncolor}{bgsexy} %font examples
        \colorlet{thesubsubsectioncolor}{bgsexy} %font examples
        \colorlet{theparagraphcolor}{bgsexy} %font examples
        \colorlet{thesubparagraphcolor}{bgsexy} %font examples
        \colorlet{thesectionnumbercolor}{bgsexy}
        \colorlet{thesubsectionnumbercolor}{bgsexy}
         \colorlet{thesubsubsectionnumbercolor}{bgsexy}
        \colorlet{thelinkcolor}{bgsexy}
        \colorlet{thecommentstyle}{thegray}
        \colorlet{thestringstyle}{bgsexy}
        \colorlet{thekeywordstyle}{spot!90}
        % code
        \colorlet{thecodebackground}{bgsexy!10}
        \colorlet{thecodeframe}{bgsexy!20}
        \colorlet{theoption}{black}
        % headers
        \colorlet{theplainoddheaderbgcolor}{white}
        \colorlet{theplainevenheaderbgcolor}{white}
        \colorlet{theplainoddfooterbgcolor}{white}
        \colorlet{theplainevenfooterbgcolor}{white}
        %
        \colorlet{theheadingsoddheaderbgcolor}{white}
        \colorlet{theheadingsevenheaderbgcolor}{white}
        \colorlet{theheadingsoddfooterbgcolor}{white}
        \colorlet{theheadingsevenfooterbgcolor}{white}
        #3
        % rules
        \colorlet{theplainoddheaderframerule}{bgsexy}
        \colorlet{theplainoddfooterframerule}{bgsexy}
        \colorlet{theplainevenheaderframerule}{bgsexy}
        \colorlet{theplainevenfooterframerule}{bgsexy}
        \colorlet{theheadingsoddheaderframerule}{bgsexy}
        \colorlet{theheadingsoddfooterframerule}{bgsexy}
        \colorlet{theheadingsevenheaderframerule}{bgsexy}
        \colorlet{theheadingsevenfooterframerule}{bgsexy}
        % list bullets
        \colorlet{theitemicolor}{bgsexy}
        \colorlet{theitemiicolor}{bgsexy}
        \colorlet{theitemiiicolor}{bgsexy}
        \colorlet{theitemivcolor}{bgsexy}
        \colorlet{theitemvcolor}{bgsexy}
        \colorlet{theitemvicolor}{bgsexy}
        % phd doc specific
        \colorlet{theunicodesymbolcolor}{bgsexy}
        \colorlet{thecmdcolor}{bgsexy}
        % epigraph rule
        \colorlet{theepigraphrulecolor}{bgsexy}
    }
}
% We now create a number of color palettes for convenience.  
% These are given various names, mostly from magazines or the web
% that have inspired the color schemes.
%
\createpalette {esquire}       {D11C23} {}  % red shade
\createpalette {fortune}       {EA8A4E} {}  % brick color
\createpalette {oprah}         {F060A8} {}  % nice pinkish modern
\createpalette {vogue}         {F21C93} {}  % nice pinkish modern
\createpalette {bbc}           {991B1e} {}  % dark red 
\tl_put_right:cn {auxpalettebbc_tl}  
  {
    \colorlet{thecodebackground}{thelightgray!20}
  }
% blue shades
\createpalette {architectural} {0168FD} {}  % blue shade
\createpalette {instyle}       {227CE8} {}  % blue shade    
\createpalette {smithsonian}   {60A8C0} {}  % milky blue 
\createpalette {blueprint}     {486090} {}  % dark milky blue
%
% green shades
\createpalette {knoll}         {88A65E} {}  % nice sweet green
\createpalette {living}        {678756} {}  % green shade
\createpalette {spring~onion}  {90D228} {}  % bright green shade
\createpalette {olive}         {EED38D} {}  % washed out not nice
\createpalette {zealous}       {075D6B} {}  % 
% orangey
\createpalette {orange~sakura} {E6781E} {}  % nice modern serious book
\createpalette {orange}        {FF6927} {}  % bright orange
% brown
\createpalette {brown}         {AF0C39} {}  % red brown  
\createpalette {brown~red}     {8D2420} {}  % brown red, serious luxury
% purples
\createpalette {black~tulip}   {420943} {}  % darkish purple 
\createpalette {helvetica}     {404547} {}  % darkish purple 
%
\createpalette {cerulean}      {9bb7d6} {}  % pantone light milky bluish  
\createpalette {sealife }      {7c7d89} {}  % sea life from https://www.benjaminmoore.com/en-us/color-overview/color-collections/color-trends-2017
\createpalette {rouge}         {D2476F} {}  %rouge like in hot lips
\createpalette{unorange}       {FE6B08} {}  %unbelievable orange 
%  
%  
\tl_put_right:cn {auxpalettecerulean_tl}  
  {
    \colorlet{thecodebackground}{tofu}
  }
  
\tl_put_right:cn {auxpaletteblueprint_tl}  
  {
    \colorlet{theunicodesymbolcolor}{bgsexy}
    \colorlet{thecodebackground}{thelightgray!20}
  }
\ExplSyntaxOff
\cxset{palette orange sakura}    
%    \end{macrocode}

% \section{Hyperlinks}
% 
% These colors are used for hyperlinking. We normally provide a uniform 
% color for all hyperlinks
%
%    \begin{macrocode}
\colorlet{thelinkcolor}{bgsexy}   %linkcolor (also used in toc)
\colorlet{theanchorcolor}{bgsexy} %anchorcolor
\colorlet{thecitecolor}{bgsexy}   %citecolor
\colorlet{thefilecolor}{bgsexy}   %filecolor
\colorlet{themenucolor}{bgsexy}   %menucolor
\colorlet{theruncolor}{bgsexy}    %runcolor
\colorlet{theurlcolor}{bgsexy}    %urlcolor
\definecolor{Hyperlink}{rgb}{0.281,0.275,0.485}                                     
%    \end{macrocode}
%
% \section{Code listings and documentation macros}
%
%  For publications that require listings or self running
%  examples we need to define a number of colors. We also need
%  to cater for control sequences in the text and the like.
%
%    \begin{macrocode}
\definecolor{lstbgcolor}{rgb}{0.9,0.9,0.9}
\colorlet{examplefill}{yellow!80!black}

%\definecolor{thecodebackground}{HTML}{F2F2EA}
%\definecolor{thecodebackground}{HTML}{DED4B9}
%\definecolor{thecodebackground}{HTML}{F3EFE3} % light 
\definecolor{thekeywordstyle}{HTML}{000000}
\definecolor{thecommentstyle}{HTML}{DF8743}
\definecolor{thestringcolor}{HTML}{DF8743}
\definecolor{theemphasiscolor}{HTML}{DF8743}
\definecolor{theframerulecolor}{HTML}{FF6927}
\definecolor{theframefillcolor}{HTML}{000000}
\definecolor{theidentifiercolor}{HTML}{000000}

% tcolorbox commands
\definecolor{themacro}{rgb}{0.784,0.06,0.176}

%    \end{macrocode}
% 
%    \begin{macrocode}
\def\done{\cellcolor{teal}done}  
\def\partialdone{\cellcolor{yellow}done}
\def\hcyan#1{{\color{teal} #1}}
%    \end{macrocode}
%</PLT>


% If you want to simulate the default |fancyhdr| behaviour you can define the
% \docAuxCommand{tikzpagelayout} as following:
% \iffalse
%<*PAGE>
% \fi
\def\pkgfileversion{1.0}
\def\pkgfiledate{2016/08/22}
% \iffalse
%</PAGE>
% \fi
%
% \title{the \textsc{tikz-page} package }
% \author {Sébastien Gross <seb chezwam org>}
% \date{This file describes version \pkgfileversion\  (\pkgfiledate)}
% \maketitle
% \tableofcontents
% \section{Implementation}
% \iffalse
%<*PAGE>
% \fi
%
%    \begin{macrocode}
\NeedsTeXFormat{LaTeX2e}
\ProvidesPackage{tikz-page}[\pkgfiledate\space (v\pkgfileversion)]
%    \end{macrocode}

% The \meta{textpos} option can be used if you want to use |textpos|
% \meta{overlay} option instead of |current page| to position the page
% layout. Beware that |textpos| with \meta{overlay} option maybe incompatible
% with some other packages. On the other hand |tikz| |current page| requires
% at least 2 compilation to work correctly. Thus you might want to use
% \meta{textpos} at conception time and remove this option for your final
% build or if you have incompatibility issues.

%    \begin{macrocode}
\newif\if@tp@use@textpos\@tp@use@textposfalse
\DeclareOption{textpos}{\@tp@use@textpostrue}
\ProcessOptions

\if@tp@use@textpos
\RequirePackage[absolute]{textpos}
\fi
%    \end{macrocode}

%    \begin{macrocode}
\RequirePackage{fancyhdr}
\RequirePackage{tikz}
\usetikzlibrary{plotmarks,calc,shapes,positioning,decorations.text}
\RequirePackage{graphicx}
\RequirePackage{calc}
%    \end{macrocode}


\makeatletter

% All margin sizes are defined in\docLength{@tp@left@margin},
% \docLength{@tp@right@margin}, \docLength{@tp@top@margin},
% \docLength{@tp@bottom@margin} their values are computed by the
% \refCom{tp@compute@margins}.

%    \begin{macrocode}
\newlength{\@tp@left@margin}
\newlength{\@tp@right@margin}
\newlength{\@tp@top@margin}
\newlength{\@tp@bottom@margin}
%    \end{macrocode}

% \begin{docCommand}{@tp@create@length}{\marg{block name}\marg{length name}}

% Generate a \cs{tp@\meta{block name}@\meta{length name}} length. This
% command is intended to be only used to create block length defined below.

%    \begin{macrocode}
\newcommand\@tp@create@length[2]{%
  \expandafter\newskip\csname tp@#1@#2\endcsname%
}%
%    \end{macrocode}
% \end{docCommand}
% 

% For each standard blocks in the page (|page|, |body|, |marginpar|,
% |header|, |footer|) and additionnal blocks (|top|, |right|, |bottom|,
% |left|), 6 lenths are computed in order to define their anchors. Each
% length is defined using the \refCom{@tp@create@length} macro.

%    \begin{macrocode}
\foreach\@@tp@element in {page,body,marginpar,header,footer,top,right,bottom,left}{%
  \foreach\@@tp@len in {xmin,xmax,xmid,ymin,ymax,ymid}{%
    \@tp@create@length{\@@tp@element}{\@@tp@len}%
}}%
%    \end{macrocode}

% \begin{docCommand}{tcflip}{\marg{odd page code}\marg{even page code}}
% Execute \meta{odd page even code} on odd pages and \meta{even page code}
% on even ones.
% \end{docCommand}
%    \begin{macrocode}
\newcommand{\tpflip}[2]{\ifodd\thepage#1\else#2\fi}
%    \end{macrocode}

% \begin{docCommand}{tp@compute@margins}{}
%
% This is where the magic happens. This command sets all \cs{tp@\meta{block
% name}@\meta{length name}} lengths.
%
% \end{docCommand}

%    \begin{macrocode}
\def\tp@compute@margins{%
  \setlength{\tp@page@xmin}{0pt}%
  \setlength{\tp@page@ymin}{0pt}%
  \setlength{\tp@page@xmax}{\paperwidth}%
  \setlength{\tp@page@ymax}{\paperheight}%
  \setlength{\tp@page@xmid}{\dimexpr(\tp@page@xmin+\tp@page@xmax)/2\relax}%
  \setlength{\tp@page@ymid}{\dimexpr(\tp@page@ymin+\tp@page@ymax)/2\relax}%
  %
  \setlength\@tp@left@margin{\dimexpr(1in+\hoffset+\tpflip{\oddsidemargin}{\evensidemargin})\relax}%
  \setlength\@tp@right@margin{\dimexpr(\paperwidth-\@tp@left@margin-\textwidth)\relax}%
  \setlength\@tp@top@margin{\dimexpr(1in+\voffset+\topmargin+\headheight+\headsep)\relax}%
  \setlength\@tp@bottom@margin{\dimexpr(\paperheight-(\textheight+\@tp@top@margin))\relax}%
  %% Body computation
  \setlength\tp@body@xmin{\dimexpr\tp@page@xmin+\@tp@left@margin\relax}%
  \setlength\tp@body@xmax{\dimexpr\tp@page@xmax-\@tp@right@margin\relax}%
  \setlength\tp@body@xmid{\dimexpr((\tp@body@xmax+\tp@body@xmin)/2)\relax}%
  \setlength\tp@body@ymax{\dimexpr(\tp@page@ymax-\@tp@top@margin)\relax}%
  \setlength\tp@body@ymin{\dimexpr\tp@body@ymin+\@tp@bottom@margin\relax}%
  \setlength\tp@body@ymid{\dimexpr(\tp@body@ymin+(\tp@body@ymax-\tp@body@ymin)/2)\relax}%
  %
  %% Margin computation
  %
  \tpflip{%
    \setlength\tp@marginpar@xmin{\dimexpr\tp@body@xmax+\marginparsep\relax}
    \setlength\tp@marginpar@xmax{\dimexpr\tp@marginpar@xmin+\marginparwidth\relax}%
  }{%
    \setlength\tp@marginpar@xmax{\dimexpr\tp@body@xmin-\marginparsep\relax}%
    \setlength\tp@marginpar@xmin{\dimexpr\tp@marginpar@xmax-\marginparwidth\relax}%
  }%
  \setlength\tp@marginpar@xmid{\dimexpr((\tp@marginpar@xmax+\tp@marginpar@xmin)/2)\relax}%
  \setlength\tp@marginpar@ymax{\tp@body@ymax}%
  \setlength\tp@marginpar@ymin{\tp@body@ymin}%
  \setlength\tp@marginpar@ymid{\tp@body@ymid}%
  %
  %% header
  %
  \setlength\tp@header@xmax{\tp@body@xmax}%
  \setlength\tp@header@xmin{\tp@body@xmin}%
  \setlength\tp@header@xmid{\tp@body@xmid}%
  \setlength\tp@header@ymin{\dimexpr\tp@body@ymax+\headsep\relax}%
  \setlength\tp@header@ymax{\dimexpr\tp@header@ymin+\headheight\relax}%
  \setlength\tp@header@ymid{\dimexpr((\tp@header@ymax+\tp@header@ymin)/2)\relax}%
  %
  %% footer
  %
  \setlength\tp@footer@xmax{\tp@body@xmax}%
  \setlength\tp@footer@xmin{\tp@body@xmin}%
  \setlength\tp@footer@xmid{\tp@body@xmid}%
  \setlength\tp@footer@ymin{\dimexpr\tp@body@ymin-\footskip\relax}%
  \setlength\tp@footer@ymax{\tp@footer@ymin}%
  \setlength\tp@footer@ymid{\dimexpr((\tp@footer@ymax+\tp@footer@ymin)/2)\relax}%
  %%
  %% blocks%
  %%
  \setlength\tp@top@xmin{\tp@page@xmin}%
  \setlength\tp@top@xmax{\tp@page@xmax}%
  \setlength\tp@top@xmid{\dimexpr((\tp@top@xmax+\tp@top@xmin)/2)\relax}%
  \setlength\tp@top@ymin{\tp@body@ymax}%
  \setlength\tp@top@ymax{\tp@page@ymax}%
  \setlength\tp@top@ymid{\dimexpr((\tp@top@ymax+\tp@top@ymin)/2)\relax}%
  %%
  \setlength\tp@bottom@xmin{\tp@page@xmin}%
  \setlength\tp@bottom@xmax{\tp@page@xmax}%
  \setlength\tp@bottom@xmid{\dimexpr((\tp@bottom@xmax+\tp@bottom@xmin)/2)\relax}%
  \setlength\tp@bottom@ymin{\tp@page@ymin}%
  \setlength\tp@bottom@ymax{\tp@body@ymin}%
  \setlength\tp@bottom@ymid{\dimexpr((\tp@bottom@ymax+\tp@bottom@ymin)/2)\relax}%
  %%
  \setlength\tp@left@xmin{\tp@page@xmin}%
  \setlength\tp@left@xmax{\tp@body@xmin}%
  \setlength\tp@left@xmid{\dimexpr((\tp@left@xmax+\tp@left@xmin)/2)\relax}%
  \setlength\tp@left@ymin{\tp@body@ymin}%
  \setlength\tp@left@ymax{\tp@body@ymax}%
  \setlength\tp@left@ymid{\dimexpr((\tp@left@ymax+\tp@left@ymin)/2)\relax}%
  %%
  \setlength\tp@right@xmin{\tp@body@xmax}%
  \setlength\tp@right@xmax{\tp@page@xmax}%
  \setlength\tp@right@xmid{\dimexpr((\tp@right@xmax+\tp@right@xmin)/2)\relax}%
  \setlength\tp@right@ymin{\tp@body@ymin}%
  \setlength\tp@right@ymax{\tp@body@ymax}%
  \setlength\tp@right@ymid{\dimexpr((\tp@right@ymax+\tp@right@ymin)/2)\relax}%
}%%    \end{macrocode}
%
% \section*{Generate Anchors}
%
% \begin{docCommand}{@tp@genanchors}{\marg{block name}}
% Generate all 9 anchors (|northwest|, |north|, |northest|, |west|,
% |center|, |east|, |southwest|, |south|, |southest|) for \meta{block name}.
% \end{docCommand}

%    \begin{macrocode}
\def\@tp@genanchors#1{%
  \anchor{#1 north}{\pgf@x=\csname tp@#1@xmid\endcsname  
                    \pgf@y=\csname tp@#1@ymax\endcsname}%
  \anchor{#1 south}{\pgf@x=\csname tp@#1@xmid\endcsname 
                    \pgf@y=\csname tp@#1@ymin\endcsname}%
  \anchor{#1 west}{\pgf@x=\csname tp@#1@xmin\endcsname 
                   \pgf@y=\csname tp@#1@ymid\endcsname}%
  \anchor{#1 northwest}{\pgf@x=\csname tp@#1@xmin\endcsname 
                        \pgf@y=\csname tp@#1@ymax\endcsname}%
  \anchor{#1 southwest}{\pgf@x=\csname tp@#1@xmin\endcsname 
                        \pgf@y=\csname tp@#1@ymin\endcsname}%
  \anchor{#1 east}{\pgf@x=\csname tp@#1@xmax\endcsname 
                   \pgf@y=\csname tp@#1@ymid\endcsname}%
  \anchor{#1 northeast}{\pgf@x=\csname tp@#1@xmax\endcsname 
                        \pgf@y=\csname tp@#1@ymax\endcsname}%
  \anchor{#1 southeast}{\pgf@x=\csname tp@#1@xmax\endcsname 
                        \pgf@y=\csname tp@#1@ymin\endcsname}%
  \anchor{#1 center}{\pgf@x=\csname tp@#1@xmid\endcsname 
                     \pgf@y=\csname tp@#1@ymid\endcsname}%
}% 
%    \end{macrocode}
%
%    \begin{macrocode}
\newcommand\tp@pgfdeclareanchoralias[3]{%
  \expandafter\def\csname pgf@anchor@#1@#3\expandafter\endcsname
    \expandafter{\csname pgf@anchor@#1@#2\endcsname}}
%    \end{macrocode}
% 
% \section*{Declare the page shape}
% \pgfname provides methods to declare new shapes. There are a number of special commands defined
% that can be used for declaring a shape (See PGF manual page 1039). Our shape will be called page.
% We will used this as a key.
% 
% Next we declare the page shape
% \begin{docCommand}{pgfdeclareshape}{}
%
% \end{docCommand}
%    \begin{macrocode}
\pgfdeclareshape{page}{
  \backgroundpath{
    \pgfpathmoveto{\pgfpoint{\tp@page@xmin}{\tp@page@ymin}}
    \pgfpathlineto{\pgfpoint{\tp@page@xmin}{\tp@page@ymax}}
    \pgfpathlineto{\pgfpoint{\tp@page@xmax}{\tp@page@ymax}}
    \pgfpathlineto{\pgfpoint{\tp@page@xmax}{\tp@page@xmin}}
    \pgfpathclose
  }
  %% basic anchors
  \anchor{north}{\pgf@x=\tp@page@xmid \pgf@y=\tp@page@ymax}%
  \anchor{south}{\pgf@x=\tp@page@xmid \pgf@y=\tp@page@ymin}%
  \anchor{west}{\pgf@x=\tp@page@xmin \pgf@y=\tp@page@ymid}%
  \anchor{northwest}{\pgf@x=\tp@page@xmin \pgf@y=\tp@page@ymax}%
  \anchor{southwest}{\pgf@x=\tp@page@xmin \pgf@y=\tp@page@ymin}%
  \anchor{east}{\pgf@x=\tp@page@xmax \pgf@y=\tp@page@ymid}%
  \anchor{northeast}{\pgf@x=\tp@page@xmax \pgf@y=\tp@page@ymax}%
  \anchor{southeast}{\pgf@x=\tp@page@xmax \pgf@y=\tp@page@ymin}%
  %\anchor{center}{\pgfpointorigin}
  \anchor{center}{\pgf@x=\tp@page@xmid \pgf@y=\tp@page@ymid}
  \anchor{origin}{\pgfpointorigin}%\pgf@x=0pt \pgf@y=0pt}
  \@tp@genanchors{page}
  %% Body anchors
  \@tp@genanchors{body}
  \@tp@genanchors{marginpar}
  \@tp@genanchors{header}
  \@tp@genanchors{footer}
  \@tp@genanchors{top}
  \@tp@genanchors{bottom}
  \@tp@genanchors{left}
  \@tp@genanchors{right}
}
%    \end{macrocode}
%
%
%Create a new |tpx| mark to show anchor location when using
%\refCom{tikzpageputanchors} to display anchors on the page.
%
%    \begin{macrocode}
\newdimen\tp@linewidth
\newdimen\tp@marksize
\setlength\tp@marksize{3pt}
\pgfdeclareplotmark{tpx}{
  \setlength{\tp@linewidth}{\pgflinewidth}
  \pgfsetlinewidth{0.1pt}
  \pgfpathmoveto{\pgfpoint{-\tp@marksize}{-\tp@marksize}}
  \pgfpathlineto{\pgfpoint{\tp@marksize}{\tp@marksize}}
  \pgfpathmoveto{\pgfpoint{-\tp@marksize}{\tp@marksize}}
  \pgfpathlineto{\pgfpoint{\tp@marksize}{-\tp@marksize}}
  \pgfusepathqstroke
  \setlength{\pgflinewidth}{\tp@linewidth}
}
%    \end{macrocode}


% Anchors can be displayed block by block (using
% \cs{tikzpageputanchorsdefaults}, \cs{tikzpageputanchors}
% \cs{tikzpageputanchorsmarginpar}, \cs{tikzpageputanchorsheader},
% \cs{tikzpageputanchorsfooter}, \cs{tikzpageputanchorstop},
% \cs{tikzpageputanchorsright}, \cs{tikzpageputanchorsbottom},
% \cs{tikzpageputanchorsleft}) or globally (using \refCom{tikzpageputanchors}).

%    \begin{macrocode}
\def\tikzpageputanchorsdefaults{
  \foreach \anchor/\placement in {%
    northwest/below right%
    ,north/below%
    ,northeast/below left%
    ,west/right%
    ,center/below%
    ,east/left%
    ,southwest/above right%
    ,south/above%
    ,southeast/above left%
  } \draw[red,shift=(page.\anchor)] plot[mark=tpx%% my plot mark
  ] coordinates{(0,0)}
  node[blue,\placement] {\scriptsize\texttt{(page.\anchor)}};
}

\def\tikzpageputanchorsbody{
  \foreach \anchor/\placement in {%
    body northwest/below right%
    ,body north/below%
    ,body northeast/below left%
    ,body west/right%
    ,body center/below%
    ,body east/left%
    ,body southwest/above right%
    ,body south/above%
    ,body southeast/above left%
  } \draw[red,shift=(page.\anchor)] plot[mark=tpx%% my plot mark
  ] coordinates{(0,0)}
  node[blue,\placement] {\scriptsize\texttt{(page.\anchor)}};
}


\def\tikzpageputanchorsmarginpar{
  \foreach \anchor/\placement in {%
    marginpar northwest/below left%
    ,marginpar north/left%
    ,marginpar northeast/above left%
    ,marginpar west/below%
    ,marginpar center/below%
    ,marginpar east/above%
    ,marginpar southwest/below right%
    ,marginpar south/right%
    ,marginpar southeast/above right%
  } \draw[red,shift=(page.\anchor)] plot[mark=tpx%% my plot mark
  ] coordinates{(0,0)}
  node[blue,\placement, rotate=90] {\scriptsize\texttt{(page.\anchor)}};
}


\def\tikzpageputanchorsheader{
  \foreach \anchor/\placement in {%
    header northwest/above right%
    ,header north/above%
    ,header northeast/above left%
    ,header west/right%
    ,header center/right%
    ,header east/left%
    ,header southwest/below right%
    ,header south/below%
    ,header southeast/below left%
  } \draw[red,shift=(page.\anchor)] plot[mark=tpx%% my plot mark
  ] coordinates{(0,0)}
  node[blue,\placement] {\scriptsize\texttt{(page.\anchor)}};
}


\def\tikzpageputanchorsfooter{
  \foreach \anchor/\placement in {%
    footer northwest/above right%
    ,footer north/above%
    ,footer northeast/above left%
    ,footer west/right%
    ,footer center/right%
    ,footer east/left%
    ,footer southwest/below right%
    ,footer south/below%
    ,footer southeast/below left%
  } \draw[red,shift=(page.\anchor)] plot[mark=tpx%% my plot mark
  ] coordinates{(0,0)}
  node[blue,\placement] {\scriptsize\texttt{(page.\anchor)}};
}

\def\tikzpageputanchorstop{
  \foreach \anchor/\placement in {%
    top northwest/below right%
    ,top north/below%
    ,top northeast/below left%
    ,top west/right%
    ,top center/below%
    ,top east/left%
    ,top southwest/above right%
    ,top south/above%
    ,top southeast/above left%
  } \draw[red,shift=(page.\anchor)] plot[mark=tpx%% my plot mark
  ] coordinates{(0,0)}
  node[blue,\placement] {\scriptsize\texttt{(page.\anchor)}};
}


\def\tikzpageputanchorsbottom{
  \foreach \anchor/\placement in {%
    bottom northwest/below right%
    ,bottom north/below%
    ,bottom northeast/below left%
    ,bottom west/right%
    ,bottom center/below%
    ,bottom east/left%
    ,bottom southwest/above right%
    ,bottom south/above%
    ,bottom southeast/above left%
  } \draw[red,shift=(page.\anchor)] plot[mark=ball%% my plot mark
  ] coordinates{(0,0)}
  node[blue,\placement] {\scriptsize\texttt{(page.\anchor)}};
}


\def\tikzpageputanchorsleft{
  \foreach \anchor/\placement in {%
    left northwest/below left%
    ,left north/left%
    ,left northeast/above left%
    ,left west/below%
    ,left center/below%
    ,left east/above%
    ,left southwest/below right%
    ,left south/right%
    ,left southeast/above right%
  } \draw[red,shift=(page.\anchor)] plot[mark=ball%% my plot mark
  ] coordinates{(0,0)}
  node[blue,\placement, rotate=90] {\scriptsize\texttt{(page.\anchor)}};
}

\def\tikzpageputanchorsright{
  \foreach \anchor/\placement in {%
    right northwest/below left%
    ,right north/left%
    ,right northeast/above left%
    ,right west/below%
    ,right center/below%
    ,right east/above%
    ,right southwest/below right%
    ,right south/right%
    ,right southeast/above right%
  } \draw[red,shift=(page.\anchor)] plot[mark=10-pointed star %%tpx%% my plot mark
  ] coordinates{(0,0)}
  node[blue,\placement, rotate=90] {\scriptsize\texttt{(page.\anchor)}};
}
%    \end{macrocode}

% \begin{docCommand}{tikzpageputanchors}{}
% A simple short hand to display all anchors at once.
% \end{docCommand}
%    \begin{macrocode}
\def\tikzpageputanchors{
  \tikzpageputanchorsdefaults
  \tikzpageputanchorsbody
  \tikzpageputanchorsmarginpar
  \tikzpageputanchorsheader
  \tikzpageputanchorsfooter
  \tikzpageputanchorstop
  \tikzpageputanchorsbottom
  \tikzpageputanchorsleft
  \tikzpageputanchorsright
}
%    \end{macrocode}



% \begin{docCommand}{tpshowframes}{}
% 
% Display |top|, |right|, |bottom| and |left| block using a specific
% background. This can be used in conjunction with \refCom{tikzpageputanchors} for
% debuging purposes.
% 
% \end{docCommand}
%    \begin{macrocode}
\tikzset{bottom style/.style={fill=blue!10, opacity=.3, draw}}
\tikzset{top style/.style={fill=blue!10, opacity=.3, draw}}
\def\tpshowframes{
  \draw[top style] (page.bottom northwest) rectangle (page.bottom southeast);
  \draw[fill=yellow!50, opacity=.3, draw] (page.top northwest) rectangle (page.top southeast);
  \draw[fill=red!50, opacity=.3, draw] (page.left northwest) rectangle (page.left southeast);
  \draw[bottom style] (page.right northwest) rectangle (page.right southeast);
}
%    \end{macrocode}


% \begin{docCommand}{tpfancyhdrdefault}{}
% An example to display headers and footer as |fancyhdr| does.
% \end{docCommand}
%    \begin{macrocode}
\def\tpfancyhdrdefault{
  \node [outer sep=0,inner sep=0, anchor=mid] at (page.header center) {};
  \node [outer sep=0,inner sep=0, anchor=mid east] at (page.header east) {\tpflip{\sl\leftmark}{\sl\rightmark}};
  \node [outer sep=0,inner sep=0, anchor=mid west] at (page.header west) {\tpflip{\sl\rightmark}{\sl\leftmark}};
  \node [outer sep=0,inner sep=0, anchor=base east] at (page.footer east) {};
  \node [,outer sep=0,inner sep=0,anchor=base] at (page.footer center) {\thepage};
  \node [outer sep=0,inner sep=0, anchor=base west] at (page.footer west) {};
}
%    \end{macrocode}

% \begin{docCommand}{tikzpage}{}
% Generate a |tikzpicture| for the whole page. if a \cs{tikzpagelayout}
% command exists, it will be executed.
% \end{docCommand}
%    \begin{macrocode}
\newcommand{\tikzpage}{
  \if@tp@use@textpos
  \begin{textblock*}{\textwidth}[0,0](0pt,0pt)%
    \fi
    \tp@compute@margins%
    \if@tp@use@textpos
    \begin{tikzpicture}[]%
      \clip (0,0) rectangle (\paperwidth, \paperheight);
      \else
      \begin{tikzpicture}[remember picture, overlay]%
      \fi
      \if@tp@use@textpos
      \node[anchor=origin,shape=page] (page) {};
      \else
      \node[anchor=origin,shape=page] (page) at (current page.south west) {};
      \fi
      \@ifundefined{tikzpagelayout}{}{\tikzpagelayout}
    \end{tikzpicture}%
  \if@tp@use@textpos
  \end{textblock*}%
  \fi
}
%    \end{macrocode}
%
\makeatother
% \iffalse
%</PAGE>
% \fi
%
% \Finale
%
% \begin{thebibliography}{9}

% \bibitem{Graphical Decoration} Trying to do graphical decorations in
% “ClassicThesis style” \url{http://tex.stackexchange.com/questions/86294}
% \end{thebibliography}
%
% 
% 
\endinput




