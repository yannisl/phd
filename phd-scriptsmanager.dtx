% \iffalse meta-comment
%<*internal>
\iffalse
%</internal>
%<*readme>
----------------------------------------------------------------
phd-scriptsmanager --- a package to shorten preambles
E-mail: yannislaz@gmail.com
Released under the LaTeX Project Public License v1.3c or later
See http://www.latex-project.org/lppl.txt
----------------------------------------------------------------

%</readme>
%<*readmemd>
###The `phd-scriptsmanager` LaTeX2e package version 0.08.0

The `phd-scriptsmanager` latex package and the class 
with the same name provide
convenient methods to create new styles for books, reports
and articles. It also loads the most commonly used packages 
and resolves conflicts.

This work consists of the file  

     `phd-scriptsmanager.dtx`,
     
and the derived files   

     `phd-scriptsmanager.ins`,  
     `phd-scriptsmanager.pdf`, 
     
     and 
     
     `phd-scriptsmanager.sty`.

###Installation

The documentation of this package uses numerous fonts not available in a
normal `TeX` distribution. Before you regenerate it, make sure you install these
fonts first. All fonts are with an open source license. 

The fonts we require for the `phd` system to be fully functional and capable
to typeset almost _any_ script or language that existed or is still live are
the following:

- The [noto](https://www.google.com/get/noto/) fonts from Google. the fonts
  are licenced under the [Apache License Version 2.0](http://www.apache.org/licenses/LICENSE-2.0.html)  or the [SIL Open Font License, Version 1.1]. Download and
  install all the fonts.

- The ancient fonts provided by [George Douros](http://users.teilar.gr/~g1951d/). 

- The Tiresias PCfont font from  [Tiresias](http://www.tiresias.org/fonts/). This font
  is not actually used for scripts, but for experiments in readability. It looks
  very good for headings. The aim of this organization is to make Information
  Communication Technologies accessible to blind and partially sighted people. This
  I thought was a good opportunity to promote their work.
  
- code2000.ttf and code2001.ttf 

- Shonar Bangla font.

- Vrinda

If you have Windows

- Microsoft JhengHei and SimSun

There are many more fonts, I will revisit these docs to provide full documentation.

Once you ready then

run
     
      `phd-lua.bat` on windows
      
          

If you have any difficulties with the package come and join us at
http://tex.stackexchange.com and post a new question or
add a comment at http://tex.stackexchange.com/a/45023/963.
or send me a message at  yannislaz at gmail.com

### Documentation

The package was written using the `doc` and `docscript` packages,
so that it is self documented in a literary programming style. 
The .pdf is a fat document, providing over fifty book styles (the
equivalent of classes) plus there is a lot of write-up on the inner
workings of TeX and LaTeX2e. However, you don't need to know much
to use it.

      \usepackage{phd}
      \input{style13}

All choices, are made via an extended key-value interface. 
Although not a compliment, it resembles CSS and the keys are a bit verbose but
attributes are easy to change and have a consistent and easy to remember interface.

To set or add a key we only use the command `\cxset`:

      \cxset{chapter name font-size = Huge,
             chapter number font-size = HUGE} 

### Future Development

This is still an experimental version, but I will retain the
interface in future releases. There is a large amount of
work still to be carried out to improve the template styles
provided, to test it more thoroughly and to add a number of
improvements in the special designs. At present I estimate
that I have completed about 70% of the work that needs
to be done.

__The package as it stands is not production stable.__ 


%</readmemd>
%
%<*TODO>
## phd-scriptmanager
1.  Document fully all fonts not availabe at CTAN and provide links or folder for download.
2.  More selective de-activation of scripts via keys.
3.  Messaging and errors need to be extended.
4.  Scripts DB and fonts DB for same.
5.  Test Module.
6.  On Demand loading of fonts.
%</TODO>
%<*internal>
\fi
\def\nameofplainTeX{plain}
\ifx\fmtname\nameofplainTeX\else
  \expandafter\begingroup
\fi
%</internal>
%<*install>
\input docstrip.tex
\keepsilent
\askforoverwritefalse
\preamble
----------------------------------------------------------------
phd-scriptsmanager --- A package to beautify documents.
E-mail: yannislaz@gmail.com
Released under the LaTeX Project Public License v1.3c or later
See http://www.latex-project.org/lppl.txt
----------------------------------------------------------------
\endpreamble

%\BaseDirectory{C:/users/admin/my documents/github/phd}
%\usedir{MWE}
\generate{\file{\jobname.sty}{
  \from{\jobname.dtx}{SCRIPTS}}
  }

%\nopreamble\nopostamble

%</install>

%<install>\endbatchfile
%<*internal>
%\usedir{tex/latex/phd}
\generate{
  \file{\jobname.ins}{\from{\jobname.dtx}{install}}
}
\nopreamble\nopostamble

\generate{
	\file{README.txt}{\from{\jobname.dtx}{readme}}
  }

\generate{
  \file{\jobname.md}{\from{\jobname.dtx}{readmemd}}
}
\generate{
  \file{\jobname-todo.tex}{\from{\jobname.dtx}{TODO}}
}

\ifx\fmtname\nameofplainTeX
  \expandafter\endbatchfile
\else
  \expandafter\endgroup
\fi
%</internal>
%<*driver>
%\listfiles
\NeedsTeXFormat{LaTeX2e}[2017/04/15]%
\RequirePackage[2017/04/15]{latexrelease}
\documentclass[twoside,11pt,a4paper]{ltxdoc}
\usepackage[bottom=2cm]{geometry}
\usepackage{microtype}
\savegeometry{std}
\usepackage{phd}

\let\HUGE\Huge %Needs fixing
\usepackage{phd-documentation}
\usepackage{phd-toc}
%\usepackage{phd-runningheads}
\usepackage{phd-lowersections}
\usepackage{makeidx}
\usepackage{phd-lists}

\pagestyle{headings}
\usepackage{bera}
\sethyperref

\addbibresource{phd1.bib}

\cxset{palette architectural}
\makeindex
\cxset{section afterindent=off,
       subsection afterindent=off,
       chapter label background-color=white,
       chapter label font-weight=none,
       chapter number background-color=white,
       chapter background-color=white,
       chapter title before background-color=bgsexy,
       chapter title padding-bottom-width=10pt,
       chapter title margin-bottom-width=10pt,
       chapter title afterskip = 10pt,
       chapter frame-color=white,
       chapter shadow=none,
       chapter title align=RaggedRight,
       section format=hang,
       subsection format=hang,
       subsubsection format=hang,
       chapter opening=any}

     
\begin{document}


\DEBUGOFF
\parindent1em
\coverpage{asia}{Book Design Monographs}{Camel Press}{SCRIPTS}{DESIGN} 
\pagestyle{empty}
%\coverpage{habtoor-city}{Delay Claim}{HLS-DSE/JV}{HABTOOR CITY}{MEP CLAIM} 
\secondpage
\pagestyle{empty}
\clearpage

\tableofcontents

\pagestyle{empty}
\setcounter{secnumdepth}{6}
\parskip0pt plus.1ex minus.1ex
\mainmatter
\pagenumbering{arabic}
\pagestyle{headings} 
\hbadness=10001  

\hfuzz=50pt 
\vfuzz=50pt 

\vbadness=\maxdimen   
\makeatletter
%\@debugtrue

\makeatother
\newfontfamily\aegean{Aegean}
%\newfontfamily\lineara{Aegean.ttf}
%\newfontfamily\cypriote{Aegean.ttf}
\let\lineara\aegean
\let\cypriote\aegean
%\newfontfamily\lycian{Aegean.ttf}
\let\lycian\aegean
\let\lydian\lycian
\let\lydianfont\aegean
\let\carian\lydianfont
\newfontfamily\oldpersian{Noto Sans Old Persian}
\newfontfamily\inscriptionalpahlavi{Noto Sans Inscriptional Pahlavi}
\newfontfamily\imperialaramaic{NotoSansImperialAramaic-Regular.ttf}
\newfontfamily\avestan{NotoSansAvestan-Regular.ttf}
%\newfontfamily\emoji{Symbola}

\chapter{Those Other Languages}
\label{ch:languages}

\epigraph{New York 1. Act making it a misdemeanor to make a speech or talk in public manner, in any language other than English upon any subject relating to the form of a character of the government or the administration or enforcement of the laws of this state or the United States.}{\itshape Introduced in the Assembly by Mr Hamill, Feb.23, and referred to the Codes Committe (A.878.)}

\parindent=1em



\section{The world's scripts and languages}


On May 23, 1918, Iowa Gov. William Harding banned the use of any foreign language in public: in schools, on the streets, in trains, even over the telephone. Frese  \footfullcite{frese} published a detailed history of this event in American history during the Great War years and its effects, which can even today be seen in Iowa. Such events of course during stressful times in a history of a country are not unique to America and similar politics can be observed throughout all human history. Today most Americans' response to the calling of such a law would probably be the Unicode Character \unicodenumber{U+1F4A9}\footnote{\protect\emoji\protect\char"1F4A9 self describing!}. Getting the character to print as a footnote in a document is another story. As many of the world's languages are facing extinction and the inclusion of a section in the |phd| package to deal with different scripts and appropriate fonts has been done in this spirit.

 
Probably there are more users of \latexe whose mother tongue is not English than those who speak the language. \tex out of the box does not offer facilities for using non-latin based scripts easily; this presents numerous problems. The biggest problem---which has been solved to a large extent---was the entering of text without having to mark all the special
characters such as umlauts (\"o) with commands. The second issue and which has been addressed by packages such as Babel, is redefining the strings such as ``Chapter" to another language. In software this is called internationalization and a governing standard is |i18n|. None of the current packages take such an approach and none of them as yet offer a satisfactory solution for |LuaLaTeX|. 


Another issue with writing systems and scripts is finding and using appropriate fonts. Most writing systems that have ever existed are now extinct. Only minute vestiges of one of the most ancient---Egyptian hieroglyphs---live on, unrecognized, in the Latin alphabet in which English, among hundreds of other languages, is conveyed today. The latin \textit{m}, for example, ultimately derives from the Egyptian's n-sign, depicting waves. There may never be a font that includes all the unicode characters (code2000) came close. Good fonts with well over ten thousand characters, keyed to the Unicode system, are now readily available. 

Bringhurst in the Elements of Typographic Style \footcite{Bringhurst2005} critisized the allotment of only 256 characters in the extended |ASCII| specification and other software and considered this practice by software developers as \enquote{typographically sectarian and culturally stunted}. 


Bringhurst comments were unfair to programmers as he was probably unaware of the difficulties. Many  scripts are widely different to the Latin script. Hanunó'o is written vertically from bottom to top, whereas tibetan and sometimes Chinese from top to bottom.  Middle Eastern scripts such as Hebrew and Arabic are written from right to left. Some of the scripts have other peculiarities as they take different forms when they are at the middle of a word or at the end. Ancient scripts such as hieroglyphics could be written from top to botton or from right to left or left to right or boustrostrephon. The glyphs of the latter could also face either left or right and the writing direction can be determined based on the direction the figures are \enquote{seeing}.\index{boustrostrephon}

Note that  we will be using the word ``script" instead of a ``writing system". Many people associate the word ``script" with a small program which is normally used on the command line. Here ``script" means a collection of letters and other characters, meant for writing human languages in a systematic way.  We say that languages such as English, Dutch and Icelandic and Vietnam use the Latin \emph{script}, although they have different repertoires of characters. 


\section{TeX's support for different languages}

\tex's support for languages centered around hyphenation patterns.
Primitives such as \docAuxCommand{language}=\meta{number} can be used to store hyphenation patterns and exceptions for up to 256 different languages. 
This primitive is then used by \tex to apply an appropriate set of hyphenation rules for each paragraph or part of a paragraph in a document\footnote{\url{http://www.tug.org/utilities/plain/cseq.html language-rp}}. 

When \tex begins a ne paragraph it sets the \emph{current language} to \cs{language}. Just before it adds each new character to the paragraph in unrestricted horizontal mode, it compares the current language to \cmd{\language}. If they are different, TeX : 

\begin{enumerate}{}

\item changes the current language to \cmd{\language}; 

\item inserts a whatsit\index{whatsit>language} containing the new language and the values of |\lefthyphenmin| and |\righthyphenmin|; 

\item inserts the character. The |\setlanguage| command should be used to change languages in restricted horizontal mode (i.e., inside an |\hbox|). 
\end{enumerate}
If \meta{number} is less than 0 or greater than 255, 0 is used [455].
  Plain TeX has a \docAuxCommand{newlanguage} command which may be used to allocate numbers for languages [347]. Changes made to \refCom{language} are local to the group containing the change 
  
If you enter, for example, |\newlanguage\Catalan|, then to switch to the hyphenation patterns of the Catalan language, you need to write |\language = \Catalan|. Writing |\Catalan| by itsef is not sufficient. 
More about \tex's support for languages can be found in the \nameref{ch:hypenation}

\section{LaTeX language management}

\latexe follows the same route as \tex and Plain TeX and its only language support is for hyphenation.
In the source2e the File |lthyphen.dtx| describes the approach to loading the default file |hyphen.ltx| . If a file hyphen.cfg is found \latexe will load the appropriate hyphenation patterns. 

Traditionally language management was achieved using Johan 
Braams package \pkg{Babel} which we describe in the next section. Numerous packages to assist in using different languages with \latex can be found at \url{http://www.ctan.org/tex-archive/language/}. 

\section{The Babel Package} 

The package \pkg{Babel} developed by \footfullcite{babel} was the first package to systematically offer foreign language
support for \latex. It has been updated for use with \XeTeX\ and \LuaTeX\ and provides an environment
in which documents can be typeset in a language
other than US English, or in more than one language.
However, no attempt has been done to
take full advantage of the features provided by the
latter, which would require a completely new core
(as for example polyglossia or as part of a future \latex3).

\subsection{Language files}
The package has a number of predefined language files with the extension |ldf|. Each \emph{language definition file} contains commands appropriate for setting strings and hyphenation patterns in the particular language, as well as
many ancillary macros to typeset dates and numbers in the typographical convention of the language. 


\begin{docCommand} {selectlanguage} {\marg{language}} {default none, initial US English}
When a user wants to switch from one language to another he can
do so using the macro |\selectlanguage|. This macro takes the
language, defined previously by a language definition file, as
its argument. It calls several macros that should be defined in
the language definition files to activate the special definitions
for the language chosen. For ``historical reasons'', a macro name is
converted to a language name without the leading |\|; in other words,
the two following declarations are equivalent:
\end{docCommand}
\begin{phdverbatim}
\selectlanguage{german}
\selectlanguage{\german}
\end{phdverbatim}

\begin{docCommand}{foreignlanguage}{\marg{language}\marg{text}}
The command |\foreignlanguage| takes two arguments; the second
argument is a phrase to be typeset according to the rules of the
language named in its first argument. This command (1) only
switches the extra definitions and the hyphenation rules for the
language, \emph{not} the names and dates, (2) does not send
information about the language to auxiliary files (i.e., the
surrounding language is still in force), and (3) it works even if
the language has not been set as package option (but in such a
case it only sets the hyphenation patterns and a warning is shown).
\end{docCommand}

\begin{docCommand}{otherlanguage*} { \marg{language}{otherlanguage*}}
Same as |\foreignlanguage| but as environment. Spaces after the
environment are \textit{not} ignored.
\end{docCommand}


\section{The Polyglossia package}

The \pkg{polyglossia} package has a lot of potential and has solved many issues
but its integration with large parts of the traditional |pdfLaTeX| world
is still under development and will probably take a while before one could
declare it easy to use and bug free \footfullcite{polyglossia}. For example anything with the |bidi| package has issues with loading orders for a number of packages and least of which is with
the Ams packages. So if you are going to mix a number of languages in a \XeTeX\ document
you need to take extra care.

 Polyglossia is a package for facilitating multilingual typesetting with
 \XeLaTeX\ and (at an early stage) \LuaLaTeX.  Basically, it
 can be used as a replacement of \pkg{babel} for performing the following
 tasks automatically:
 
 \begin{enumerate}
 \item Loading the appropriate hyphenation patterns.
 \item Setting the script and language tags of the current font (if possible and
       available), via the package \pkg{fontspec}.
 \item Switching to a font assigned by the user to a particular script or language.
 \item Adjusting some typographical conventions according to the current language
       (such as afterindent, frenchindent, spaces before or after punctuation marks,
       etc.).
 \item Redefining all document strings (like chapter, ``figure'', ``bibliography'').
 \item Adapting the formatting of dates (for non-Gregorian calendars via external
       packages bundled with polyglossia: currently the Hebrew, Islamic and Farsi
       calendars are supported).
 \item For languages that have their own numbering system, modifying the formatting
       of numbers appropriately (this also includes redefining the alphabetic sequence
       for non-Latin alphabets).\footnote{ %
         For the Arabic script this is now done by the bundled package \pkg{arabicnumbers}.}
 \item Ensuring proper directionality if the document contains languages
       that are written from right to left (via the package \pkg{bidi},
       available separately).
 \end{enumerate}
 
 Several features of \pkg{babel} that do not make sense in the \XeTeX\/\luatex world (like font
 encodings, shorthands, etc.) are not supported supported by the package.
 
 Generally speaking, \pkg{polyglossia} aims to remain as compatible as possible
 with the fundamental features of \pkg{babel} while being cleaner, light-weight,
 and modern. The package \pkg{antomega} has been very beneficial in our attempt to
 reach this objective.


\section{Loading language definition files}

The recommended way of \pkg{polyglossia} to load language definition files
is given in the manual as:
 
\begin{docCmd}{setdefaultlanguage}{\oarg{options}\marg{lang}}
 (or equivalently \cmd\setmainlanguage).
\end{docCmd}
 
 Secondary languages can be loaded with

\begin{docCmd} {setotherlanguage}{\oarg{options}\marg{lang}}
\end{docCmd}
 These commands have the advantage of being explicit and of allowing you to set
 language-specific options.\footnote{ %
 More on language-specific options below.}
 It is also possible to load a series of secondary languages at once using

\begin{docCmd}{setotherlanguages} { \marg{lang1,lang2,lang3,\ldots}}
\end{docCmd}

 Language-specific options can be set or changed at any time by means of
\begin{docCmd}{setkeys} { \marg{lang}\marg{opt1=value1,opt2=value2,\ldots}}
\end{docCmd}

\subsection{Bidirectional languages}





\begin{comment}
\begin{Arabic}
ّ هو إذ الغاية؛ شريف الفوائد، جم المذهب، عزيز فنّ التاريخ فنّ أنّ اعلم
والملوك سيرهم، في والأنبياء أخلاقهم، في الأمم من الماضين أحوال على يوقفنا
ّ أحوال في يرومه لمن ذلك في الإقتداء فائدة تتم حتّى وسياستهم؛ دولهم في
والدنيا. الدين
\end{Arabic}
\end{comment}

The Greek language is represented both in modern Greek as well as its ancient variants.

\begin{phdverbatim}
\begin{greek}
\textbf{Η ελληνική γλώσσα} είναι μία από τις ινδοευρωπαϊκές γλώσσες, για την
οποία έχουμε γραπτά κείμενα από τον 15ο αιώνα π.Χ. μέχρι σήμερα. Αποτελεί το
μοναδικό μέλος ενός κλάδου της ινδοευρωπαϊκής οικογένειας γλωσσών. Ανήκει
επίσης στον βαλκανικό γλωσσικό δεσμό.\\	
\end{greek}
\end{phdverbatim}

\topline

\textbf{Η ελληνική γλώσσα} είναι μία από τις ινδοευρωπαϊκές γλώσσες, για την
οποία έχουμε γραπτά κείμενα από τον 15ο αιώνα π.Χ. μέχρι σήμερα. Αποτελεί το
μοναδικό μέλος ενός κλάδου της ινδοευρωπαϊκής οικογένειας γλωσσών. Ανήκει
επίσης στον βαλκανικό γλωσσικό δεσμό.\\	

\bottomline

\begin{verbatim}
\begin{russian}
\textbf{Русский язык} — один из восточнославянских языков, один из 
крупнейших языков мира, в том числе самый распространённый из славянских
языков и самый распространённый язык Европы, как географически, так и по
числу носителей языка как родного (хотя значительная, и географически бо́
льшая, часть русского языкового ареала находится в Азии).	\\

\end{russian}
\end{verbatim}



\textbf{Русский язык} — один из восточнославянских языков, один из крупнейших языков мира, в том числе самый распространённый из славянских языков и самый распространённый язык Европы, как географически, так и по числу носителей языка как родного (хотя значительная, и географически бо́льшая, часть русского языкового ареала находится в Азии).	\\


\section{The Translator package}

The \pkg{translator} package was developed by Till Tantau \cite{translator}. It provides a flexible
mechanism for translating individual words into different languages.
For example, it can be used to translate a word like \enquote{figure} into,
say, the German word \enquote{Abbildung}. Such a translation mechanism is
useful when the author of some package would like to localize the
package such that texts are correctly translated into the language
preferred by the user. The translator package is \emph{not} intended
to be used to automatically translate more than a few words. 

You may wonder whether the translator package is really necessary
since there is the (very nice) |babel| package available for
\LaTeX. This package already provides translations for words like
``figure''. Unfortunately, the architecture of the babel package was
designed in such a way that there is no way of adding translations of
new words to the (very short) list of translations directly build into
babel.

The translator package was specifically designed to allow an easy
extension of the vocabulary. It is both possible to add new words that
should be translated and translations of these words.

\subsection{Using the Translator Package}

  The \pkg{Translator}\footcite{translator} needs to be used with \pkg{Babel} and I am not too sure yet 
  if it is ready  to be used with Polyglossia.

Once the package has loaded a language or a set of languages the optional argument to the
\cmd{\translate} can be used to translate a string. 

\begin{texexample}{Translating strings}{ex:translator}
  \translate[to=german]{rightpagename}
  \translate[to=dutch]{rightpagename}
\end{texexample}

Before you can provide the translations you need to provide your own dictionaries, where you require them. These need to be installed at a place where \tex can find them.

\begin{docCmd} {ProvidesDictionary} { \marg{dictionary file name} \marg{language} }
\end{docCmd}

The dictionary has to be saved in a specific format that relates to the \refCmd{ProvidesDictionary} command. The second argument of the command must be appended to the file name; for the example the file is saved as\footnote{This  example is from the translator package bundle and is under the folder \texttt{base}}:

|translator-basic-dictionary-German|

The concepts take a bit of time to sink in, but once you have everything set up, it is quite easy and straight forward to incorporate it, into your package. 

\begin{teX}
\ProvidesDictionary{translator-basic-dictionary}{German}

\providetranslation{Abstract}{Zusammenfassung}
\providetranslation{Addresses}{Adressen}
\providetranslation{addresses}{Adressen}
\providetranslation{Address}{Adresse}
\providetranslation{address}{Adresse}
\providetranslation{and}{und}
\providetranslation{Appendix}{Anhang}
\providetranslation{Authors}{Autoren}
\providetranslation{authors}{Autoren}
\providetranslation{Author}{Autor}
\providetranslation{author}{Autor}
\end{teX} 

This is in contrast to Babel and Polyglossia that define
commands for each string to be translated such as,

\begin{phdverbatim}
\def\captionsdutch{%
    \def\prefacename{Voorwoord}%
    \def\refname{Referenties}%
    \def\abstractname{Samenvatting}%
    \def\bibname{Bibliografie}%
    \def\chaptername{Hoofdstuk}%
    \def\appendixname{Bijlage}%
    ...
    \def\proofname{Bewijs}%
    \def\glossaryname{Verklarende woordenlijst}%
    \def\today{\number\day~\ifcase\month%
      \or januari\or februari\or maart\or april\or mei\or juni\or
      juli\or augustus\or september\or oktober\or november\or
      december\fi
      \space \number\year}}
\end{phdverbatim}

\begin{docCommand}{usedictionary}{\marg{kind}}
  This command tells the |translator| package, that at the beginning of
  the document it should load \textit{all} dictionaries of kind \meta{kind} for
  the languages used in the document. Note that the dictionaries are
  not loaded immediately, but only at the beginning of the document.

  If no dictionary of the given \emph{kind} exists for one of the
  language, nothing bad happens.

  Invocations of this command accumulate, that is, you can call it
  multiple times for different dictionaries.
\end{docCommand}

\begin{docCommand}{uselanguage}{\marg{list of languages}}
  This command tells the |translator| package that it should load the
  dictionaries for all languages in the \meta{list of languages}. The
  dictionaries are loaded at the beginning of the document.
\end{docCommand}



\chapter{CLDR}

The \pkg{phd} package provides facilities for language handling, but albeitly still at an experimental stage. Sectioning command strings can easily be set in one's language by just typing the key in the appropriate language.

\begin{texexample}{Example of changing language in headings}{ex:lheadings}
\bgroup
\cxset{locale turkish,
       chapter format=block,
      chapter opening=anywhere,
       chapter number color = black,
      }
\chapter{Testing}
        
\egroup
\end{texexample}


The language message text are actually variables (pretty much similar to the message modules of the l3error package. It follows
patterns for defining such messages in other languages and in the Linux kernel. (|get_text|). Actually your error messages
in packages belong here. 

These resources should preferably be put into files that wil be loaded by a library that uses a combination of language and country (also known as the \enquote{locale}) to identfy the right string. Once we have placed these files we can send them to the translation vendor and get back translated files for each locale that your application is going to support.

There are various file formats that make suitable resource files. Popular choices are JSON, XML, gettext or YAML. The translator files that we discussed earlier are very similar in context. 

\begin{phdverbatim}     
     locale/en/names/part~name/.store      = part_name_tl,
     locale/en/names/chapter~name/.store   = chapter_name_tl,
     get_text{en}{chapter_name_tl} -> Chapter
\end{phdverbatim} 

A similar storing technique can be employed for other sections of the CLDR specifications such as delimters:

\begin{phdverbatim}
    locale/en/delimiters/quotation start = “,
    locale/en/delimiters/quotation end =  ”,
    locale/en/delimiters/alternate quotation start = ‘,
    locale/en/delimiters/alternate quotation end = ’,
\end{phdverbatim}

The i18n CLDR discussed in more detailed in the next chapter provide ready made internationally agreed json or xml files
detailing the most common internationalization tasks, such as strings for dates, months, calendars, quotes, common units and
sorting the latter is very important for many tasks, such as bibliographies. Once we have the structure defined
a small Go utility can download all the files and translate them into our resource files. These will be missing the
strings for sectioning, typographical conventions, shorthands and other conventions normally handled by Babel. 

Babel and polyglossia modify the basic LaTeX environments or macros to achieve this. In my opinion it should be the other way out, they should only provide a value to be used by these commands rather than the commands be cloberred at this level.

\section{Numbers}

The formatting of numbers for a locale is specified by the CLDR in files containing the file |numbers|. These can be downloaded as |xml| files or |json| files.
Currently most users of \latexe requiring to format numerals they will use either |numprint| or |SIUnitx|. The latter has many settings as it handles scientific units. For formatting numbers in \latexe the |cldr| specifications and data are somewhat limited. Package options and commands
would normally handle rounding, leading zeroes for decimals, plums minus signs for the combination (+-) and other similar requirements.



\begin{texexample}{Using numprint}{ex:numprint}
% basic command
\numprint{12500678.912345}

% shorter version
\np{12500678.912345}
\end{texexample}



\paragraph{Numbering systems}

Numbering systems are used to show different representations of numeric values. Each numbering system consists of characters that represent numeric digits. In addition, there are also number symbols used with each numbering system that may differ when the numbering system is used in different locales.

The default numbering system for a locale is the numbering system that is normally used to represent numbers in that locale.

\begin{verbatim}
"numbers": {
        "defaultNumberingSystem": "latn",
        "otherNumberingSystems": {
          "native": "latn"
        },
        "minimumGroupingDigits": "1",
        "symbols-numberSystem-latn": {
          "decimal": ".",
          "group": ",",
          "list": ";",
          "percentSign": "%",
          "plusSign": "+",
          "minusSign": "-",
          "exponential": "E",
          "superscriptingExponent": "×",
          "perMille": "‰",
          "infinity": "∞",
          "nan": "NaN",
          "timeSeparator": ":"
        },
\end{verbatim}

The native numbering system for a locale is the numbering system used for native digits, and is normally in the script for the locale's language. Native numbering systems can only use numeric positional decimal digits, like for Latin numbers (0123456789). If the numbering system in your language uses an algorithm to spell out numbers in the language's script, label it as a traditional numbering system instead. The traditional numbering system does not need to be specified if it is the same as the native numbering system.

The default, native and traditional numbering systems for a locale may be different. For example, in Tamil the default numbering system is |latn|, the native numbering system is |tamldec| and the traditional numbering system is |taml|.

\begin{trivlist}\item[]
\begin{tabular}{lll}
\toprule
Code	 & Description	 & Digits\\
\midrule
arab	 & Arabic-Indic digits	&\panunicode ٠١٢٣٤٥٦٧٨٩\\
fullwide &   	Full width digits &\panunicode 	0123456789\\
hant	   & Traditional Chinese numerals — non-decimal	& algorithmic\\
latn	   &Latin digits	 &0123456789\\
\bottomrule
\end{tabular}
\end{trivlist}

\paragraph{Minimum digits for grouping}

In some languages, the grouping separator is suppressed in certain cases. For example, see china-auf-wachstumskurs.gif, where there is a grouping separator in \enquote{12 080} but not in \enquote{4720}. The |minimumGroupingDigits| determines what the default for a locale is. In this case the value should be \enquote{2} to illustrate that the separator only appears once the number of thousands goes into the double-digits (i.e. 10 thousand or above) and not for single-digit thousands (i.e. anything below 10 thousand).


Note that this is just the default, and the grouping separator may be retained in lists, or removed in other circumstances. For example, in English the \enquote{,} is used by default, but not in addresses (\enquote{12345 Baker Street}), in 4-digit years (2014, but 12,000 BC), and certain other cases.

\begin{texexample}{Numprint minimum grouping}{ex:numprint2}
\begingroup
\npfourdigitnosep$\numprint{1234.1234}$, $\numprint{12345.12345}$ 

\npfourdigitsep$\numprint{1234.1234}$, $\numprint{12345.12345}$
\endgroup
\end{texexample}

The much larger package \pkg{siunitx} can also be used to parse and typeset numbers in different formats, using the command \docAuxCommand{num}.


\begin{texexample}{siunitx}{ex:siunitx}
\num{123}\\
\num{1234}\\
\num{12 345}\\
\num{0.123} \\
\num{0,1234}\\
\num{.12345}\\
\num{3.45d-4}\\
\num{-e10}
\end{texexample}

The package also provides commands for formatting angles, ranges and similar. For the latter it also provides a limited set of localization commands by using the \pkg{translator} from the \pkg{Beamer} bundle.

The package defines numerous keys that can be used either at package level or as options to the command num to format and print the numbers. In the next example the group separator is set uing the key |group-separator|. 

\begin{texexample}{siunitx group separator}{ex:siunitx-01}
\num{12345} \\
\num[group-separator = {,}]{12345} \\
\num[group-separator = \text{~}]{12345}
\end{texexample}


\paragraph{Number Symbols} The following symbols are used in formatting numbers. They will be substituted for the placeholders in Number Patterns. 

\begin{longtable}{llp{5cm}}
\toprule
Name	&English Example	&Meaning\\
\midrule
|decimal|	  &2,345.67	 &decimal separator\\
|group|	     &2,345.67	 &grouping separator, typically for thousands\\
|minusSign|  &	+23	*	 &the plus sign used with numbers\\
|plusSign|	  &-23	*	    &the minus sign used with numbers\\ 
|perMille|	  &234‰	*	&the permille sign (out of 1000)\\
|exponential|	      &1.2E3	*	&used in computers for 1.2×10³.\\
|superscriptingExponent|	&1.2×103	* &human-readable format of exponential \\
|infinity|	  &∞	*	&used in +∞ and -∞.\\ 
|nan|	     &NaN	*	&\enquote{not a number}. \\
\bottomrule
\end{longtable}


%The + and - symbols are intended for unary usage, and not for binary usage; thus represents either the positive number or a negative number. For example, in an operation 3 -(-2), the defined symbol would be used for the second minus sign, but not for the subtraction operator. Any directionality markers needed (e.g. <LRM>) to keep with the number should be included.
%percentSign	23.4%%	*	the percent sign (out of 100)

\paragraph{Number Patterns}

Numbers are formatted using patterns, like |#,###.00|. For example, the pattern |#,###.00| when used to format the number 12345.678 could result in "12'345,67". That would happen if the grouping separator for your language is an apostrophe ('), and the decimal separator is a comma (,).  Also see Number Symbols.

Important: The characters . , 0 \# (and others below) are special placeholders; they stand for the decimal separator, and so on, and are NOT real characters. You must NOT "translate" the placeholders; for example, don't change '.' to ',' even though in your language the decimal point is written with a comma.

Here are the special characters used in number patterns.

Whenever any of these symbols are in the English pattern, they must be retained in the pattern for your language. The positions of some of them (\%, ¤) may be changed, or spaces added or removed. The symbols will be replaced by the local equivalents, using the Number Symbols for your language. Verify results by reviewing the dynamic examples in the right-hand pane.


The cldr locale files, provide these patterns. They can then be used to format general purpose numbers, which fall into
five categories.

\begin{longtable}{p{2.5cm}lp{6.5cm}}
\toprule
Type	&English Example	& Meaning\\
\midrule
currency	&¤|#,##0.00|  &Used for currency values. A currency symbol (¤) is will be replaced by the appropriate currency symbol for whatever currency is being formatted. The choice of whether to use the international currency symbols (USD, EUR, JAY, RUB,…) or localized symbols (\$, €, ¥, руб.,…) is up to the application program that uses CLDR. Note: the number of decimals will be set by programs that use CLDR to whatever is appropriate for the currency, so do not change them; keep exactly 2 decimals.\\

currency-accounting	 &¤|#,##0.00|;(¤|#,##0.00|)	&Used for currency formats in accounting contexts.\\
\bottomrule
\end{longtable}

Pattern Characters are shown below.

\begin{longtable}{lp{11cm}}
\caption{Pattern Characters}\\
\toprule
Symbol & Meaning\\
\midrule
.	&Replaced automatically by the character used for the decimal point in your language. Not a real period; must be retained!\\
,	&Replaced by the "grouping" (thousands) separator in your language. Not a real comma; must be retained!\\
0	&Replaced by a digit (or zero if there aren't enough digits).\\
\#	&Replaced by a digit (or nothing if there aren't enough). Often used to show the position of the ",".\\
¤	&This will be replaced by a currency symbol, such as \$ or USD. Note: by default a space is automatically added between letters in a currency symbol and adjacent numbers. So you don't need to add a space between them if your language writes \enquote{\$12} but \enquote{USD 12}.\\
\%	&This marks a percent format. The \% symbol may change position, but must be retained.\\
E	&This marks a scientific format. The E symbol may change position, but must be retained.\\
'	&If any of the above characters are used as literal characters, they must be quoted with ASCII single quotes. For example, in the Short Numbers if a period needs to be used to mark an abbreviation, it would appear as:
0.0 tis'.'
not
0.0 tis.\\
\ldots;\ldots	&If your language uses different formats for negative numbers than just adding "-" at the front, you can put in two patterns, separated by a semicolon. The first will be used for zero and positive values, while the second will be used for negative values.
For example: |#,##|0.00¤;(|#,##|0.00¤) is used to make negative currencies appear like \enquote{(1'234,56£)} instead of \enquote{-1'234,56£}. That is used for formatting currency amounts in English, but not for general-purpose decimal numbers.\\
\bottomrule
\end{longtable}

\section{Characters}
The |<characters>| element provides optional information about characters that are in common use in the locale, and information that can be helpful in picking resources or data appropriate for the locale, such as when choosing among character encodings that are typically used to transmit data in the language of the locale. It typically only occurs in a language locale, not in a language/territory locale.

\begin{quote}
|<exemplarCharacters>[a-zåæø]</exemplarCharacters>|
\end{quote}

The exemplar character set contains the commonly used letters for a given modern form of a language, which can be for testing and for determining the appropriate repertoire of letters for charset conversion or collation. ("Letter" is interpreted broadly, as anything having the property Alphabetic in the [UCD], which also includes syllabaries and ideographs.) It is not a complete set of letters used for a language, nor should it be considered to apply to multiple languages in a particular country. Punctuation and other symbols should not be included.

There are two sets: the main set should contain the minimal set required for users of the language, while the auxiliary exemplar set is designed to encompass additional characters: those non-native or historical characters that would customarily occur in common publications, dictionaries, and so on. So, for example, if Irish newspapers and magazines would commonly have Danish names using å, for example, then it would be appropriate to include å in the auxiliary exemplar characters; just not in the main exemplar set. Major style guidelines are good references for the auxiliary set. Thus for English we have [a-z] in the main set, and [á à ă â å ä ā æ ç é è ĕ ê ë ē í ì ĭ î ï ī ñ ó ò ŏ ô ö ø ō œ ß ú ù ŭ û ü ū ÿ] in the auxiliary set.

In general, the test to see whether or not a letter belongs in the main set is based on whether it is acceptable in that language to always use spellings that avoid that character. For example, the exemplar character set for en (English) is the set [a-z]. This set does not contain the accented letters that are sometimes seen in words like "résumé" or "naïve", because it is acceptable in common practice to spell those words without the accents. The exemplar character set for fr (French), on the other hand, must contain those characters: [a-z é è ù ç à â ê î ô û æ œ ë ï ÿ]. The main set typically includes those letters commonly taught in schools as the "alphabet".

The list of characters is in the Unicode Set format, which allows boolean combinations of sets of letters, including those specified by Unicode properties.

Sequences of characters that act like a single letter in the language — especially in collation — are included within braces, such as [a-z á é í ó ú ö ü ő ű \{cs\} \{dz\} \{dzs\} \{gy\} \ldots]. The characters should be in normalized form (NFC). Where combining marks are used generatively, and apply to a large number of base characters (such as in Indic scripts), the individual combining marks should be included. Where they are used with only a few base characters, the specific combinations should be included. Wherever there is not a precomposed character (e.g. single codepoint) for a given combination, that must be included within braces. For example, to include sequences from the Where is my Character? page on the Unicode site, one would write: [\{ch\} \{tʰ\} \{x̣\} \{ƛ̓\} {ą́} {i̇́} {ト゚}], but for French one would just write [a-z é è ù ...]. When in doubt use braces, since it does no harm to included them around single code points: e.g. [a-z \{é\} \{è\} \{ù\} ...].

If the letter 'z' were only ever used in the combination 'tz', then we might have [a-y {tz}] in the main set. (The language would probably have plain 'z' in the auxiliary set, for use in foreign words.) If combining characters can be used productively in combination with a large number of others (such as say Indic matras), then they are not listed in all the possible combinations, but separately, such as:

{\panunicode [‌ ‍ ॐ ०-९ ऄ-ऋ ॠ ऌ ॡ ऍ-क क़ ख ख़ ग ग़ घ-ज ज़ झ-ड ड़ ढ ढ़ ण-फ फ़ ब-य य़ र-ह ़ ँ-ः ॑-॔ ऽ ् ॽ ा-ॄ ॢ ॣ ॅ-ौ] }

The exemplar character set for Han characters is composed somewhat differently. It is even harder to draw a clear line for Han characters, since usage is more like a frequency curve that slowly trails off to the right in terms of decreasing frequency. So for this case, the exemplar characters simply contain a set of reasonably frequent characters for the language.

The ordering of the characters in the set is irrelevant, but for readability in the XML file the characters should be in sorted order according to the locale's conventions. The set should only contain lower case characters (except for the special case of Turkish and similar languages, where the dotted capital I should be included); the uppercase letters are to be mechanically added when the set is used. For more information, see [Data Formats] and the discussion of Special Casing in the Unicode Character Database.

For example for the locale |se| for Northern Sami, we have:


\begin{longtable}{l p{8cm}}
\toprule
Attribute             & Value \\
\midrule
exemplar characters   & a á b c č d đ e f g h i j k l m n ŋ o p r s š t ŧ u v z ž\\
exemplar characters auxiliary  & à ç é è í ń ñ ó ò q ú w x y ü ø æ å ä ã ö\\
exemplar characters index  &A Á B C Č D Đ E É F G H I J K L M N Ŋ O P Q R S Š T Ŧ U V W X Y Z Ž Ø Æ Å Ä Ö\\
exemplar characters numbers &  , \% ‰ + − 0 1 2 3 4 5 6 7 8 9\\
\bottomrule
\end{longtable}


\begin{longtable}{l p{8cm}}
\toprule
Attribute             & Value \\
\midrule
 exemplar characters &a b c ç d e f g ğ h ı i İ j k l m n o ö p r s ş t u ü v y z\\
exemplar characters  auxiliary & á à ă â å ä ã ā æ é è ĕ ê ë ē í ì ĭ î ï ī ñ ó ò ŏ ô ø ō œ q ß ú ù ŭ û ū w x ÿ\\
 exemplar characters index & A B C Ç D E F G H I İ J K L M N O Ö P Q R S Ş T U Ü V W X Y Z\\
exemplar character numbers & \- , . \% ‰ + 0 1 2 3 4 5 6 7 8 9\\
exemplar characters punctuation &  - ‐ – — , ; : ! ? . … ' ‘ ’ " “ ” ( ) [ ] § @ * / \& \# † ‡ ′ ″\\
\bottomrule
\end{longtable}


\paragraph{ellipsis}The ellipsis element provides patterns for use when truncating strings. There are three versions: initial for removing an initial part of the string (leaving final characters); medial for removing from the center of the string (leaving initial and final characters), and final for removing a final part of the string (leaving initial characters). For example, the following uses the ellipsis character in all three cases (although some languages may have different characters for different positions).

\begin{longtable}{ll}
Ellipsis final & \{0\}… \\           
Ellipsis initial & …\{0\} \\         
Ellipsis medial  & \{0\}…\{1\} \\       
Ellipsis word-final & \{0\} … \\     
Ellipsis word-initial & … \{0\} \\   
Ellipsis word-medial & \{0\} … \{1\}\\
\end{longtable} 

\paragraph{List patterns} List patterns can be used to format variable-length lists of things in a locale-sensitive manner, such as \enquote{Monday, Tuesday, Friday, and Saturday} (in English) versus \enquote{lundi, mardi, vendredi et samedi} (in French). For example, consider the following example:

\begin{phdverbatim}
  <listPatterns>
    <listPattern>
	   <listPatternPart type="start">{0}, {1}</listPatternPart>
		<listPatternPart type="middle">{0}, {1}</listPatternPart>
		<listPatternPart type="end">{0}, and {1}</listPatternPart>
		<listPatternPart type="2">{0} and {1}</listPatternPart>
    </listPattern>
		<listPattern type="or">
			<listPatternPart type="start">{0}, {1}</listPatternPart>
			<listPatternPart type="middle">{0}, {1}</listPatternPart>
			<listPatternPart type="end">{0}, or {1}</listPatternPart>
			<listPatternPart type="2">{0} or {1}</listPatternPart>
	</listPattern>
	<listPattern type="unit">
			<listPatternPart type="start">{0}, {1}</listPatternPart>
			<listPatternPart type="middle">{0}, {1}</listPatternPart>
			<listPatternPart type="end">{0}, {1}</listPatternPart>
			<listPatternPart type="2">{0}, {1}</listPatternPart>
	</listPattern>
	<listPattern type="unit-narrow">
			<listPatternPart type="start">{0} {1}</listPatternPart>
			<listPatternPart type="middle">{0} {1}</listPatternPart>
			<listPatternPart type="end">{0} {1}</listPatternPart>
			<listPatternPart type="2">{0} {1}</listPatternPart>
	</listPattern>
	<listPattern type="unit-short">
			<listPatternPart type="start">{0}, {1}</listPatternPart>
			<listPatternPart type="middle">{0}, {1}</listPatternPart>
			<listPatternPart type="end">{0}, {1}</listPatternPart>
			<listPatternPart type="2">{0}, {1}</listPatternPart>
	</listPattern>
  </listPatterns>
\end{phdverbatim}	

These are not very useful for \tex as most of this type of work can be simply be achieved by just typing the values. the
\pkg{siunitx} offers similar facilities for lists, through the \pkg{translator} and Babel.

\begin{texexample}{clist}{ex:clistuse}
\ExplSyntaxOn
\group_begin:
\def\firsttwowords{~and~}
\def\lasttwowords{ ~and~ }
\def\betweenmorethantwo{ ,~ }
\clist_set:Nn \l_tmpa_clist { a , b , , c , {de} , f }
\clist_use:Nnnn \l_tmpa_clist { \firsttwowords } { ,~ } { ,\lasttwowords }
\group_end:
\ExplSyntaxOff
\end{texexample}


\paragraph{Typographical considerations and Convenience Commands} Some commands,  provided by babel-french are intended to make typesetting according to French typographical conventions easier. Some twenty three conditionals, which more or less affect typographical rules or conventions are mentioned in Babel (see p.43, frencgb.pdf).

    \begin{enumerate}
       \item Hyphenation parameters such as lefthyphenmin and righthyphenmin are defined for many of the languages.
             
             \begin{tabular}{lll}
             \toprule
               Language        & \cs{lefthyphenmin} & \cs{righthyphenmin}\\
             \midrule  
               Finnish         &    2               & 2                   \\
               French          &2                   & 2                   \\
             \bottomrule  
             \end{tabular}
       \item Delimiters (Quotation marks): Delimiters according to CLDR terminology are the characters used for quoting texr. For example in UK English they are the \enquote{curly} right and left forms as in this \enquote{this phrase}. the alternate forms are for embedded quotations such as \enquote{He yelled \enquote{Stop!}, and turned around.} Babel for many of the languages provides macros to enclose text in quotes|\og| and |\fg|.
       \item Typesetting of superscripts such as nth etc. In the French section of Babel this is defined as \docAuxCommand{up}, used
             as M|\up|me \foreignlanguage{french}{M\up{me}}.
       \item French spacing. 
       \item Spacing before punctuation. With Babel and LuaLaTeX a lua script is loaded, that uses callbacks to intercept
             the punctuation and add the appropriate node attributes. The callbacks are fairly comprehensive and cater for
             some edge cases such as 1sp columns etc.
       \item For the other engines it falls back to active characters or to XeTeX character classes. 
       \item Caption separators. In French, captions in figures and tables should never be printed as \enquote{Figure 1:} which is the default in standard \latexe classes; the \enquote{:} is made active too late, no space is added befre it. With \lualatex and \xelatex, this glitch does not occur if you use Babel, you should get \enquote{Figure 1\thinspace:} which is correct in French. 
       \item \textit{Ellipsis Patterns}.  Ellipsis patterns are used in a display when the text is too long to be shown. It will be used in environments where there is very little space. \tex traditionally provided \docAuxCommand{ldots}. With unicode it should be just one character; and where that really can't work, the CLDR specification mentions that it should be as short as possible. 

There are three different possible patterns that need to be translated. Typically the same character is used in all three, but three choices are provided just in case different characters would be appropriate in different contexts, for some languages.

       Babel provides for French macros and switches to allow for the extra spacing required in French typography.

       \item The \pkg{bigfoot} package deeply changes the way footnotes are handled, including providing its own output routine. When |bigfoot| is loaded babel-french drops the customization of footnotes. The layout of footnotes does not depend on the language, as babel's documentation state, it will look wrong if if two footnotes on the same page are looking different because one was called in a French part, the other one in English.  The rest of the code deals in detail as to how to handle the various
       packages and footnotemark.  
     \end{enumerate}
   
   
\paragraph{Babel shorthands}      
My biggest concern with Babel is the ordering of packages due to all the redefinitions and in having to execute most of the code at the |AtBeginDocument| hook.      
    
The way the |PHD| package works is that the user will be provided with a style file, providing all the settings. These can be named. For example |thesis|. Such a style file can be easily be changed to |thesis french|, where for example the field \docAuxKey[phd]{caption separator}{} is set to |caption separator=french colon|.     

\section{Fonts for all the world's scripts and languages}

\epigraph{If you steal from one author it's plagiarrism, if you steal from many, it's research.}{
---Wilson Mizner}

Besides the issues with different languages, hyphenation and caption names, there is also the difficulties with fonts. Unless the current font has the necessary glyphs it will either print junk characters or we get the unicode no glyph symbol.

Many commercial as well as open source fonts exist that can be used to typeset text the world's scripts and languages. The aim of this section of the documentation is to present an overview of the most common scripts represented in the Unicode~7.0 standard. All the examples require the use of the \XeTeX\ or \LUATEX engine. In addition you need to have a copy of the font on your own system. If you do not have them, the font loading mechanism of \XeTeX\ or \LUATEX will take some time to search all the directories and slows compilation tremendously. 

\subsection{Pan-Unicode Fonts}

Thousands of fonts exist on the market, but fewer than a dozen fonts—sometimes described as ``pan-Unicode" fonts—attempt to support the majority of Unicode's character repertoire. Instead, Unicode-based fonts typically focus on supporting only basic |ASCII| and particular scripts or sets of characters or symbols. Several reasons justify this approach: applications and documents rarely need to render characters from more than one or two writing systems; fonts tend to demand resources in computing environments; and operating systems and applications show increasing intelligence in regard to obtaining glyph information from separate font files as needed, i.e. font substitution. Furthermore, designing a consistent set of rendering instructions for tens of thousands of glyphs constitutes a monumental task; such a venture passes the point of diminishing returns for most typefaces.

The \texttt{NotoSerif} fonts from Google\footnote{\protect\url{http://www.google.com/get/noto/}} have good support for 96 language fonts and the list is growing. Since these are widely available most of the scripts that follow use these fonts. Follow the instructions at the website to install them. It is just a matter of dragging them into the fonts folder for most operating systems.

Another freeware pan-Unicode font is \docFont{Titus}
This is an extended version of this font is TITUS Cyberbit Unicode, includes 36,161 characters in v4.0.

On Windows systems |Arial Unicode MS| contains glyphs for all code points within the Unicode Standard version 2.1.  

The code2000 font provides 63546 glyphs and is the nearest font to a universal font to handle Unicode. Unfortunately development stopped in 2008. As a comparison Linux Libertine O, provides 2674 glyphs. \label{code2000}

CJK fonts naturally will have the most glyphs, \idxfont{MingLiU} 34046 glyphs and is a very good font for CJK typesetting. Google in conjunction with Adobe also provides a fee CJK font.

The \href{http://ftp.gnu.org/gnu/freefont/}{FreeFont Project} currently supports most of the useful set of free outline (i.e. OpenType) fonts covering as much as possible of the Unicode character set. The set consists of three typefaces: one monospaced and two proportional (one with uniform and one with modulated stroke). 

The idea of having lots of different writing systems into a single font at all? How good does such a font need to be?
There are two extreme views.  The first one is that glyphs in a font shold comprise a unified design entity. This in practice makes sense only within a single language script. Different script systems, such a Latin, Arabic and Devanagari, have different typesetting traditions and conventions.  A good discussion of the advantages and disadvantages can be found at the gnu website \footnote{\protect\url{https://www.gnu.org/software/freefont/articles/Why_Unicode_fonts.html}}. For TeX it is a better proposition in order to avoid switching of fonts that can distract the writer. At least one requires fonts that are inclusive of one's usage. 

\section{The \texttt{ucharclasses} package}

For multilingual texts font switching can become cumbersome. The use of a pan-Unicode font as the default can help. However, if the languages are distinct enough to use different Unicode blocks, which are not covered by the \pkg{polyglossia} package Mike Kamermans' package \pkg{ucharclasses} can be used. This package only works with \xelatex and does not work with LuaTeX. 

\begin{verbatim}
% and the font switching magic
\usepackage[CJK, Latin, Thai, 
           Sinhala, Malayalam, 
           DominoTiles, 
           MahjongTiles]{ucharclasses}
\usepackage{fontspec}

\ifxetex
% default transition uses the widest coverage font I know of
  \setDefaultTransitions{\fontspec{Code2000.ttf}}{}

% overrides on the default rules for specific informal groups
  \setTransitionsForLatin{\fontspec{Palatino Linotype}}{}
  \setTransitionsForCJK{\fontspec{code2000.ttf}}{}%HAN NOM A
  \setTransitionsForJapanese{\fontspec{code2000.ttf}}{}%Ume Mincho

% overrides on the default rules for specific unicode blocks
  \setTransitionTo{CJKUnifiedIdeographsExtensionB}{\fontspec{SimSun-ExtB}}
  \setTransitionTo{Thai}{\fontspec{IrisUPC}}
  \setTransitionTo{Sinhala}{\fontspec{Iskoola Pota}}
  \setTransitionTo{Malayalam}{\fontspec{Arial Unicode MS}}
\ifxetex
\end{verbatim}

{
\newfontfamily\mahjong{FreeSerif.ttf}
\mahjong
domino tiles, 🁇 🀼 🁐 🁋 🁚 🁝, and mahjong tiles: 🀑 🀑 🀑 🀒 🀒 🀒 🀕 🀕 🀕 🀗 🀗 🀗 🀅 🀅 (using FreeSerif)

}

The interaction between Polyglossia and Fontspec can result in infinite looping and memory leaks. I do not recommend that you use these commands as yet. The use of the charclasses will also slow down compilation possibly by a factor of 10.



\section{PhD Settings}

The \pkg{phd} provides support both for scripts, as well as language settings. A script setting sets the system to use appropriate fonts and if the script is associated with a unique language it will automatically handle language settings. Alternatively for multi-language scripts such as the Latin script, the language key can be used. This will automatically setup the language and an appropriate default font. 

\begin{docKey}[phd]{script} { = \meta{script name}} {default none, initial US English}{}
\end{docKey}

\begin{docKey}{language}{ =\meta{language name}}  {default none, US English}
The key language sets the main language for the document. This language will be used for the sectioning commands and common string translations.

If the language is English Polyglossia or Babel are not loaded automatically. If the language is other than English we load either Babel or Polyglossia depending on the engine used.
\end{docKey}


\begin{docKey}{languages}{ = \meta{language1, language2, language3}}  {}
The key |languages|, determines all the other scripts available for typesetting. For each language default font commands are create automatically. The aim is to be able to run a fully multilingual system with the minimum of upfront settings. These we leave to customize in the style template files.
\end{docKey}

\begin{docKey}{greek font}{ = \meta{options}\meta{font file}}  {}
The package comes with numerous language and appropriate default fonts
for each operating system. 
\end{docKey}

\cxset{chapter opening=any}


\section{IPA Transcriptions}

Language is spoken and writing systems need to cater for the individuality of the sounds for a particular language. Many of the world's languages facing extinction do not have a written representation for their language. The \textit{lingua franca} of linguists is the  International Phonetic Alphabet (IPA). This is an alphabetic system of phonetic notation based primarily on the Latin alphabet. It was devised by the International Phonetic Association in the late 19th century as a standardized representation of the sounds of spoken language. The IPA is used by lexicographers, foreign language students and teachers, linguists, speech-language pathologists, singers, actors, constructed language creators and translators. \footcite{ipa}

In the chapters that follow, I have used it extensively. The IPA Handbook is an essential reference work for all those involved in the analysis of speech. Besides the IPA notation a knowledge of linguistic terms is also necessary. A short guide is provided. In \latex the \pkg{tipa} can be of help, but soon a good keyboard layout will be better.

The IPA Extensions block has been present in Unicode since version 1.0, and was unchanged through the unification with ISO 10646. The block was filled out with extensions for representing disordered speech in version 3.0, and Sinology phonetic symbols in version 4.0.[4]

\bigskip
{\catcode`\"=12
\unicodetable{arial}{"0250,"0260,"0270,"0280,"0290,"02A0}
}
\bigskip

\def\schwa{{\arial \char"0259}}

Besides the symbols, there are numerous diacritics and markers.

With Unicode and the right font, there is no problem  in typesetting IPA phonetic symbols. However the problem is the input.

I recommend that you get familiar with a Unicode IPA keyboard overlay. I have used Keyman. When the keyboard is turned on, certian keys (`,@,=) are activated.

As long as your editor allows Unicode input (most do these days) and you're compiling with XeLaTeX or LuaLaTeX, you can just use the IPA keyboard to type directly into the editor just as you can in most other applications. You can also copy and paste your Unicode text from other applications too. 

For example take the transcription of a Hittite word written as \emph{ši-ú-ni-iš}. Here we can typeset it faster by the Hittite package, and numerous others as |\thittite{si-u-ni-is}|. The software is intelligent enough to add the diacritics. They are also expandable. 

\section{The world's scripts}

Anatolian hieroglyphs were first thought to have been used for the Hittite language, 

Anatolian Hieroglyphs is a Unicode block containing Anatolian hieroglyphs, used to write the extinct Luwian language, because they first appeared on personal seals from Hattusha, the capital of the Hittite Empire. While
Hittites did make use of the characters on seals and on their monumental inscriptions, the characters were
used as text primarily for the related language Luwian; a few glosses in Urartian and some divine names
in Hurrian are known to be written in Anatolian Hieroglyphs. Most of the texts are monumental stone
inscriptions, though some letters and accounting documents have been preserved inscribed on strips of
lead. 

\newfontfamily\anatolian{Anatolian}
{
\catcode`\"=12
\unicodetable{anatolian}{"14400,"14410,"14420,"14430,"14440,"14450,"14460,"14470,"14480,"14490,"144A0,"144B0,"144C0,"144C0,"144D0,"144E0,"144F0,%
 "14500,"14510,"14520,"14530,"14540,"14550,"14560,"14570,"14580,"14590,"145A0,"145B0,"145C0,"145D0,"145E0,"145F0,%
 "14600,"14610,"14620,"14630,"14640}
}
























 OK

%\cxset{image=greek-men}

\parindent1em

\chapter{Greek}
\epigraph{The Pleiads have left the sky, and\\
the moon has vanished. It’s midnight:\\
the time for meeting is over.\\
And me—I am lying, lonely}{Sappho}
\label{s:greek}
\index{languages>Greek}\index{Herodotus}\index{alphabets>Greek}



\enquote{The Phoenicians who came with Kadmos,} wrote Herodotus in the fifth century BC of the legendary Phoenician prince of Tyre and brother of Europa, ``\ldots introduced into Greece, after their settlement in the country, a number of accomplishments of which the most important was writing, an art which probably was unknown to the Greeks until then''. 

A basis for the remarkable history of the Greek language is
the invention of the Greek alphabet. It was modelled after
Semitic scripts, with the important improvement that not only
consonants but also vowels are represented by independent
letters.\footnote{Cover image, from \href{http://www.pappaspost.com/todays-undesirable-muslims-were-yesteryears-greeks-pure-american-no-rats-no-greeks/}{papaspost.com}, showing Greek immigrants arriving at Ellis Island in 1911.}

The poet Sappho had access to an alphabetic script, invented
for the Greek language just a couple of hundred years before her
time. The Greek alphabet is very similar to the Latin one, which
is the one used for English. In fact, the Latin alphabet is derived
from a variant of the Greek one.The similarity is easy to observe.
Here is the original poem, written in the Greek alphabet:\footcite{janson:2002}


\begin{center}
\arial 
ΔΕΔΗΚΕ ΜΕΝ Α ΣΕΛΑΝΝΑ\\
ΚΑΙ ΠΛΕΙΑΔΕΣ. ΜΕΣΑΙ ΔΕ\\
ΝΥΚΤΕΣ. ΠΑΡΑ Δ'ΕΡΧΕΤ'ΩΡΑ.\\
ΕΓΩ ΔΕ ΜΟΝΑ ΚΑΤΕΥΔΩ\\
\end{center} 

Transcribed in the Latin alphabet:

\begin{center}
DEDUKE MEN A SELANNA\\
KAI PLEIADES. MESAI DE\\
NUKTES. PARA D’ ERKHET’ ORA.\\
EGO DE MONA KATEUDO.\\
\end{center}

The Greek alphabet is the script that has been used to write the Greek language since the 8th century BC. It was derived from the earlier Phoenician alphabet, and was in turn the ancestor of numerous other European and Middle Eastern scripts, including Cyrillic and Latin.[3] Apart from its use in writing the Greek language, both in its ancient and its modern forms, the Greek alphabet today also serves as a source of technical symbols and labels in many domains of mathematics, science and other fields.

In its classical and modern forms, the alphabet has 24 letters, ordered from alpha to omega. Like Latin and Cyrillic, Greek originally had only a single form of each letter; it developed the letter case distinction between upper-case and lower-case forms in parallel with Latin during the modern era.

\tex has built-in commands for the usage of the Greek alphabet see section \ref{greek} in the Symbols chapter.

\bgroup
\obeylines
\greek\obeyspaces

Α	ἄλφα	aleph	alpha	[alpʰa]	[ˈalfa]	Listeni/ˈælfə/
Β	βῆτα	beth	beta	[bɛːta]	[ˈvita]	/ˈbiːtə/, US /ˈbeɪtə/
Γ	γάμμα	gimel	gamma	[ɡamma]	[ˈɣama]	/ˈɡæmə/
Δ	δέλτα	daleth	delta	[delta]	[ˈðelta]	/ˈdɛltə/
Η	ἦτα	  heth	   eta	 [hɛːta], [ɛːta]	[ˈita]	/ˈiːtə/, US /ˈeɪtə/
Θ	θῆτα	teth	theta	[tʰɛːta]	[ˈθita]	/ˈθiːtə/, US Listeni/ˈθeɪtə/
Ι	ἰῶτα	yodh	iota	[iɔːta]	[ˈʝota]	Listeni/aɪˈoʊtə/
Κ	κάππα	kaph	kappa	[kappa]	[ˈkapa]	Listeni/ˈkæpə/
Λ	λάμβδα	lamedh	lambda	[lambda]	[ˈlamða]	Listeni/ˈlæmdə/
Μ	μῦ	mem	mu	[myː]	[mi]	Listeni/ˈmjuː/; occasionally US /ˈmuː/
Ν	νῦ	nun	nu	[nyː]	[ni]	/ˈnjuː/ (US /ˈnuː/)
Ρ	ῥῶ	reš	rho	[rɔː]	[ro]	Listeni/ˈroʊ/
Τ	ταῦ	taw	tau	[tau]	[taf]	/ˈtaʊ/ or /ˈtɔː/
\egroup

With a suitable font such as |Arial Unicode MS| you do not need to do anything special to typeset short paragraphs of Greek text. Just use any editor set to encode the text in \utfviii. The example below was just cut and pasted. If you are going to write extensively in Greek it would be preferable to get a virtual keyboard. If you are using windows these are pre-build. 

\topline
\begin{quote}
Ἡροδότου Ἁλικαρνησσέος ἱστορίης ἀπόδεξις ἥδε, ὡς μήτε τὰ γενόμενα ἐξ ἀνθρώπων τῷ χρόνῳ ἐξίτηλα γένηται, μήτε ἔργα μεγάλα τε καὶ θωμαστά, τὰ μὲν Ἕλλησι, τὰ δὲ βαρβάροισι ἀποδεχθέντα, ἀκλεᾶ γένηται, τὰ τε ἄλλα καὶ δι' ἣν αἰτίην ἐπολέμησαν ἀλλήλοισι.[2]

Herodotus of Halicarnassus, his Researches are set down to preserve the memory of the past by putting on record the astonishing achievements of both the Greeks and the Barbarians; and more particularly, to show how they came into conflict.[3]
\end{quote}
\bottomline

\begin{scriptexample}{greek}
\unicodetable{greek}{% 
"0370,"0380,"0390,"03A0,"03B0,"03C0,"03D0,"03E0,"03F0}
\end{scriptexample}

\subsection{Greek diacritics}
\index{Greek>polytonic}

The ancient Greek writing included for many accents and diacritics. The extended unicode standard provides slots for all diacritics. Greek orthography has used a variety of diacritics starting in the Hellenistic period. The complex polytonic orthography notates Ancient Greek phonology. The simple monotonic orthography, introduced in 1982, corresponds to Modern Greek phonology, and requires only two diacritics.

Polytonic orthography (πολύς "much", "many", τόνος "accent") is the standard system for Ancient Greek. The acute accent ( ´ ), the grave accent ( ` ), and the circumflex ( ῀ ) indicate different kinds of pitch accent. The rough breathing ( ῾ ) indicates the presence of an /h/ sound before a letter, while the smooth breathing ( ᾿ ) indicates the absence of /h/.

Since in Modern Greek the pitch accent was replaced by a dynamic accent, and the /h/ was lost, most polytonic diacritics have no phonetic significance, and merely reveal the underlying Ancient Greek etymology.

Monotonic orthography (μόνος "single", τόνος "accent") is the standard system for Modern Greek. It retains a single accent or tonos (΄) to indicate stress and the diaeresis (¨) to indicate a diphthong: compare modern Greek παϊδάκια /pajˈðaca/ "lamb chops", with a diphthong, and παιδάκια /peˈðacia/ "little children" with a simple vowel. Tonos and diaeresis can be combined on a single vowel, as in the verb ταΐζω (/taˈizo/ "to feed").

\medskip
\begin{scriptexample}[]{Greek}
\unicodetable{greek}{%
"1F00,"1F10,"1F20}
\end{scriptexample}

%%%%%%%%%%%%%%%%%%%%%%%%%%%%%%%%%%%%
%    Greek Language
%%%%%%%%%%%%%%%%%%%%%%%%%%%%%%%%%%%%

%\documentclass{book}
%\usepackage{phd}
%\usepackage{philokalia}
%\begin{document}

\subsection{Philokalia}

The \pkgname{philokalia} package by Apostolos Syropoulos provides a Greek font in the style of the Philokalia manuscripts. The package modifies the lettrine package, which we cater for in the \pkgname{phd} and hence we adjusted it slightly for this. Also the package needed some modifications to work with LuaTeX.

The Philokalia (Ancient Greek: φιλοκαλία "love of the beautiful, the good", from φιλία philia "love" and κάλλος kallos "beauty") is "a collection of texts written between the 4th and 15th centuries by spiritual masters"[1] of the Eastern Orthodox hesychast tradition. They were originally written for the guidance and instruction of monks in "the practise of the contemplative life".[2] The collection was compiled in the eighteenth-century by St. Nikodemos of the Holy Mountain and St. Makarios of Corinth.

Although these works were individually known in the monastic culture of Greek Orthodox Christianity before their inclusion in The Philokalia, their presence in this collection resulted in a much wider readership due to its translation into several languages. The earliest translations included a Church Slavonic translation of selected texts by Paisius Velichkovsky (Dobrotolublye) in 1793, a Russian translation by Ignatius Bryanchaninov in 1857, and a five-volume translation into Russian (Dobrotolyubie) by St. Theophan the Recluse in 1877.

There were subsequent Romanian, Italian and French translations.[3][4]
The book is a "principal spiritual text" for all the Eastern Orthodox Churches;[5] the publishers of the current English translation state that "The Philokalia has exercised an influence far greater than that of any book other than the Bible in the recent history of the Orthodox Church."[6]
Philokalia (sometimes Philocalia) is also the name given to an anthology of the writings of Origen compiled by Saint Basil the Great and Saint Gregory Nazianzus. Other works on monastic spirituality have also used the same title over the years.[5][7]

The Philokalia fonts consist of three fonts: one that contains
the normal typeface, one that contains the ligatures and one that contains the special ornament characters that decorate the beginning of each chapter. The glyphs were generated from scanned images of the book pages and Apostolos Syropoulos described the process in detail in \cite{syropoulos}. 


{
%\newfontfamily\plk{Philokalia-Regular}
\plk
%\newfontfamily\PHtitl[Script=Greek,RawFeature=+titl;grek]{Philokalia-Regular}
 %\font\PHtitl="[Philokalia-Regular]/ICU:script=grek,+titl"

 
 \lettrine[lines=3]{\usebox{\philobox}}{ερὶ} ποιητικῆς αὐτῆς τε καὶ τῶν εἰδῶν αὐτῆς, ἥν τινα δύναμιν ἕκαστον ἔχει, 
καὶ πῶς δεῖ συνίστασθαι τοὺς μύθους  εἰ μέλλει καλῶς ἕξειν ἡ ποίησις, ἔτι δὲ ἐκ πόσων καὶ ποίων 
ἐστὶ μορίων, ὁμοίως δὲ καὶ περὶ τῶν ἄλλων ὅσα τῆς αὐτῆς ἐστι μεθόδου, λέγωμεν ἀρξάμενοι κατὰ φύσιν 
πρῶτον ἀπὸ τῶν πρώτων.
 
Ἐποποιία δὴ καὶ ἡ τῆς τραγῳδίας ποίησις ἔτι δὲ κωμῳδία καὶ ἡ διθυραμβοποιητικὴ καὶ τῆς αὐλητικῆς 
ἡ πλείστη καὶ κιθαριστικῆς πᾶσαι τυγχάνουσιν οὖσαι μιμήσεις τὸ σύνολον· διαφέρουσι δὲ ἀλλήλων τρισίν, 
ἢ γὰρ τῷ ἐν ἑτέροις μιμεῖσθαι ἢ τῷ ἕτερα ἢ τῷ ἑτέρως καὶ μὴ τὸν αὐτὸν τρόπον. 

Ὥσπερ γὰρ καὶ χρώμασι καὶ σχήμασι πολλὰ μιμοῦνταί τινες ἀπεικάζοντες (οἱ μὲν [20] διὰ τέχνης οἱ δὲ διὰ συνηθείας),
ἕτεροι δὲ διὰ τῆς φωνῆς, οὕτω κἀν ταῖς εἰρημέναις τέχναις ἅπασαι μὲν ποιοῦνται τὴν μίμησιν ἐν ῥυθμῷ καὶ λόγῳ καὶ
ἁρμονίᾳ, τούτοις δ᾽ ἢ χωρὶς ἢ μεμιγμένοις· οἷον ἁρμονίᾳ μὲν καὶ ῥυθμῷ χρώμεναι μόνον ἥ τε αὐλητικὴ καὶ ἡ κιθαριστικὴ
κἂν εἴ τινες [25] ἕτεραι τυγχάνωσιν οὖσαι τοιαῦται τὴν δύναμιν, οἷον ἡ τῶν συρίγγων, αὐτῷ δὲ τῷ ῥυθμῷ [μιμοῦνται]
χωρὶς ἁρμονίας ἡ τῶν ὀρχηστῶν (καὶ γὰρ οὗτοι διὰ τῶν σχηματιζομένων ῥυθμῶν μιμοῦνται καὶ ἤθη καὶ πάθη καὶ πράξεις)· 
 }

The package also modifies the \pkgname{lettrine} package and hence we have modified the \cmd{\lettrine} command to be called \cmd{\lettrinephilokalia} when used with the |philokalia| package. It is a bit long as a command, but easier to remember. 



\section{Greek-derived scripts}

Because of Greece’s military (Alexander the Great), economic
and cultural influence, the Greek alphabet became the
prototype for the ‘complete’ (that is, fully vowelized) alphabets
that emerged in Europe in the following centuries. These eventually
diffused, almost exclusively through Greek’s granddaughter
alphabets Latin and Cyrillic, throughout the entire
world --- a process still going on over two thousand years later
(illus. \ref{fig:greekderived}).

In first-millenium BC Asia Minor (today’s Turkey), the
Greek alphabet inspired an impressive number of non-Greek
peoples to elaborate their own Anatolian alphabets: \nameref{carian},
\nameref{sec:lydian}, \nameref{sec:lycian}, Pamphylian, Phrygian, Pisidian (of the Roman
period) and Sidetic.  Nonetheless, these scripts failed to
acquire lasting significance because of the region’s declining
economic fortunes followed by several major invasions.

%\documentclass{article}
%\usepackage[margin=1cm]{geometry}
%\usepackage{pdflscape}
%\usepackage{forest}
%\usepackage{hyperref}
%\usetikzlibrary{shadows,arrows}
\newgeometry{left=1cm,right=1cm,bottom=1cm}
\newpage

\tikzset{parent/.style={align=center,text width=2cm, fill=blue!40,rounded corners=2pt,inner sep=2pt},
    child/.style={align=center,text width=2.0cm,fill=orange!60,rounded corners=2pt,inner sep=1pt,outer sep=0pt},
    grandchild/.style={fill=white,text width=1.7cm}
}

%\begin{document}

\begin{landscape}
\begin{forest}
for tree={%
    thick,
    drop shadow,
    l sep=1.0cm,
    s sep=0.6cm,
    node options={draw,font={\rmfamily\small}},
    edge={semithick,-latex},
    where level=0{parent}{},
    where level=1{
        minimum height=0.8cm,
        child,
        parent anchor=south west,
        tier=p,
        l sep=0.25cm,
        for descendants={%
            grandchild,
            minimum height=0.6cm,
            %l sep=0.5cm,
%            s sep=0.5cm,
            anchor=115,
            edge path={
                \noexpand\path[\forestoption{edge}]
                (!to tier=p.parent anchor) |-(.child anchor)\forestoption{edge label};
            },
        }
    }{},
}
[(Phoenician)\\ GREEK
    [Palaeo-Hispanic %heading
        [North-east\\
         Celtiberian
            [South-West\\
                South-east
            ]
        ]
    ],
    [Etruscan
        [LATIN
            [\textit{Rhaetian} 
                [\href{http://en.wikipedia.org/wiki/Gallic}{Gallic}
                    [\href{http://en.wikipedia.org/wiki/Venetic}{Venetic}
                      [Faliscan
                        [Northern Picene
                          Southern Picene\\
                            [Oscan
                              [Umbrian]
                            ]
                          ]
                       ]
                    ]
                ]
            ]
        ]
    ]
    [\href{http://en.wikipedia.org/wiki/Gothic_language}{Gothic}]
    [Glagolithic
       [Croatian]
     ]   
    [Cyrillic
        [Russian
         [Ukrainian
            [Bulgarian
             [Serb]
            ] 
        ]  
     ]  
   ]  
  [Anatolian
    [Carian
      [Lydian
        [Lykian
          [Pamphylian
            [Phrygian
              [Pisidian
                [Sidetic]
            ]
          ]
        ]
      ]
  ]
  ]
  ]
  [Armenian]
  [Georgian]
  [Coptic
    [Nubian]
  ]
]
\end{forest}
\captionof{figure}{Abridged family tree of some Greek-derived scripts.}
\label{fig;greekderived}
\end{landscape}


\restoregeometry
\newpage


%\end{document}

The Armenian monk St Mesrob (c. 345–440) is said to have
elaborated the Armenians’ first script c. AD 405 – Armenian is a
separate branch of the Indo-European superfamily of languages
(to which Greek and Germanic, which includes English, also
belong). Based on the Greek alphabet, the Armenian script
originally consisted of around 36 mainly capital letters. By the
1200s, Armenian notrgir, or cursive writing, had been developed,
then replacing writing in capitals (illus. 100).

St Mesrob is also credited with devising the Georgian alphabet
in the early 400s AD – Georgian is a Caucasian, not an Indo-
European, language – as well as the Albanian alphabet. (Such
multiple attributions suggest that Mesrob’s role was apocryphal.)

The ecclesiastical Georgian script used 38 letters; over
time, several styles of writing Georgian developed, with varying
numbers of letters (illus. 101). The mkhedruli, or ‘lay hand’,
which began as a medium for non-sacral texts, is Georgian’s
most frequently employed script, still in use today.

In Egypt, the Greek alphabet inspired the Coptic
alphabet that replaced one of the world’s oldest writing traditions.
In the Balkans, Greek generated the Glagolitic and
Cyrillic scripts, which eventually generated the Russian script,








%\chapter{Middle Eastern Scripts}
\label{ch:middleeasternscripts}

The scripts in this section have a common origin in the ancient phoenician (\S~\ref{s:phoenician}) alphabet. They include:

\begin{center}
\begin{tabular}{ll}
\nameref{hebrew} & \nameref{s:samaritan}\\
Arabic & Thaana\\
\nameref{s:syriac} &\\
\end{tabular}
\end{center}

The Hebrew script is used in Israel and for languages of the Diaspora. The Arabic script is
used to write many languages throughout the Middle East, North Africa, and certain parts
of Asia. The Syriac script is used to write a number of Middle Eastern languages. These
three also function as major liturgical scripts, used worldwide by various religious groups.

The Samaritan script is used in small communities in Israel and the Palestinian Territories
to write the Samaritan Hebrew and Samaritan Aramaic languages. The Thaana script is
used to write Dhivehi, the language of the Republic of Maldives, an island nation in the
middle of the Indian Ocean. 

Text in these scripts is written from right to left. Arabic and Syriac are cursive scripts even when typeset, unlike Hebrew, Samaritan  and Thaana, where letters are unconnected. Most letters in Arabic and Syriac assume different forms depending on their position in a word. Shaping rules are not required for Hebrew because only five letters have position-dependent forms, and these forms are separately encoded.

Historically, Middle Eastern  scripts did not write short vowels. In modern scripts they are represented  by marks positioned above or below a consonantal letter. Vowels and other
marks of pronunciation (``vocalization’’) are encoded as combining characters, so support
for vocalized text necessitates use of composed character sequences. Yiddish, Syriac, and
Thaana are normally written with vocalization; Hebrew, Samaritan, and Arabic are usually written unvocalized. 


\section{Samaritan}
\label{s:samaritan}
\newfontfamily\samaritan{NotoSansSamaritan-Regular.ttf}

The Samaritan alphabet is used by the Samaritans for religious writings, including the Samaritan Pentateuch, writings in Samaritan Hebrew, and for commentaries and translations in Samaritan Aramaic and occasionally Arabic.

The Samaritans are, consider themselves to be the descendants of the Northern Tribes of Israel that were not sent into Assyrian captivity, and have continuously resided in the land of Israel.

The Torah Scroll of the Samaritans uses an alphabet that is very different from the one used on Jewish Torah Scrolls. According to the Samaritans themselves and Hebrew scholars, this alphabet is the original "Old Hebrew" alphabet.

Even as far back as 1691, this connection between the Samaritan and the "Old" Hebrew alphabets was made by Henry Dodwell; "[the Samaritans] still preserve [the Pentateuch] in the Old Hebrew characters."

Samaritan is a direct descendant of the Paleo-Hebrew alphabet, which was a variety of the Phoenician alphabet in which large parts of the Hebrew Bible were originally penned. All these scripts are believed to be descendants of the Proto-Sinaitic script. That script was used by the ancient Israelites, both Jews and Samaritans. The better-known "square script" Hebrew alphabet traditionally used by Jews is a stylized version of the Aramaic alphabet which they adopted from the Persian Empire (which in turn adopted it from the Arameans). 

After the fall of the Persian Empire, Judaism used both scripts before settling on the Aramaic form. For a limited time thereafter, the use of paleo-Hebrew (proto-Samaritan) among Jews was retained only to write the Tetragrammaton, but soon that custom was also abandoned.



ShofarRegular StamAshkenazCLM.ttf

\begin{scriptexample}[]{Samaritan}
\bgroup
\TeXXeTstate=1
\unicodetable{samaritan}{"0800,"0810,"0820,"0830}
\egroup
\TeXXeTstate=0
\end{scriptexample}

I battled to get an appropriate font for the Samaritan script and had to use the \idxfont{Noto Sans Samaritan} from Google


^^A\printunicodeblock{./languages/samaritan.txt}{\samaritan}


\url{http://www.ancient-hebrew.org/ahh/ahh.htm#_Toc314842274}




\let\luatextextdir\textdir

\section{Hebrew}
\hebrew
\epigraph{Why does the story of creation begin with bet?... In the same manner that the letter bet is closed on all sides and only open in front, similarly you are not permitted to inquire into what is before or what was behind, but only from the actual time of Creation.}{Babylonian Talmud, Tractate Hagigah, 77c}

\label{s:hebrew}

The Hebrew alphabet (Hebrew:{ אָלֶף־בֵּית עִבְרִי }‎[a], Alefbet Ivri), known variously by scholars as the Jewish script, square script and block script, is an abjad script used in the writing of the Hebrew language, also adapted as an alphabet script in the writing of other Jewish languages, most notably in Yiddish (lit. "Jewish" for Judeo-German), Djudío (lit. "Jewish" for Judeo-Spanish), and Judeo-Arabic. Historically, there have been two separate abjad scripts to write Hebrew. The original, old Hebrew script, is known as the paleo-Hebrew alphabet (which has been largely preserved, in an altered form, in the Samaritan alphabet), while the present "Jewish script" or "square script" to write Hebrew is a stylized form of the Aramaic alphabet and was known by Jewish sages as the Ashuri alphabet (lit. "Assyrian"), since its origins were alleged to be from Assyria.[2] Various "styles" (in current terms, "fonts") of representation of the Jewish script letters described in this article also exist, as well as a cursive form which has also varied over time and place, and today is referred to as cursive Hebrew. In the remainder of this article, the term "Hebrew alphabet" refers to the Jewish square script unless otherwise indicated.

To properly typeset Hebrew texts you first need to choose an appropriate font and also set the directionality of the text. This
is done using the etex commands \docAuxCommand{beginL} and \docAuxCommand{beginR} 

For \XeTeX\ you also need to add near the top of your document |\TeXXeTstate=1|. The package \pkgname{bidi} can be used to set all parameters. Be warned that it redefines almost all of \latexe's commands, so for short mixed texts, I wouldn't recommend its usage. 



The Hebrew alphabet (Hebrew: אָלֶף־בֵּית עִבְרִי[a], alefbet ʿIvri ), known variously by scholars as the Jewish script, square script, block script, is used in the writing of the Hebrew language, as well as other Jewish languages, most notably Yiddish, Ladino, and Judeo-Arabic. There have been two script forms in use; the original old Hebrew script is known as the paleo-Hebrew script (which has been largely preserved, in an altered form, in the Samaritan script), while the present "square" form of the Hebrew alphabet is a stylized form of the Assyrian script. Various "styles" (in current terms, "fonts") of representation of the letters exist. There is also a cursive Hebrew script, which has also varied over time and place. On Windows you can use the \texttt{Miriam} font or \texttt{Arial Unicode MS} or \texttt{Miriam Fixed}.
\medskip

\topline
\bgroup
\ifxetex\TeXXeTstate=1\fi
\raggedleft\arial{}\beginR

הכתב הכנעני הקדום הלך והתפשט וסימניו היו מוכרים כל כך, עד כי המשתמשים בו התחילו "להתעצל" בהשלמת הציורים, והניחו כי הקורא יבין גם מתוך שרטוטים סכמתיים באיזו אות מדובר. כך, למשל, הפך הראש למשולש עם צוואר; כף היד מלאת האצבעות הפכה לשרטוט דל, ומהדג נותר רק הזנב. כשהעברים אמצו את הכתב הכנעני הם התקשו לזהות חלק מהציורים המקוריים והניחו למשל כי הסימן המתאר את המילה "זהה" הוא כלי נשק; שזנב הדג המשולש הוא דלת, ושדווקא הנחש הוא דג. כך נולדו שמותיהם העבריים של האותיות זי"ן, דל"ת ונו"ן (נון הוא דג, כמו אמנון, שפמנון וכו'). הציורים שהפכו לסימנים התגלגלו לכתבים נוספים, ואפילו ליוונית וללטינית. גם בכתב העברי המודרני ניתן לזהות המשך התפתחותי ברור מן הכתב הכנעני הקדום, והשתמרות שמות האותיות מקלה מאוד על פענוח המקור.


בתקופת בית שני, אומץ האלפבית הארמי לשימוש השפה העברית במקום האלפבית העברי העתיק, כאשר בזה האחרון נעשה שימוש מועט כגון כתיבת השמות הקדושים והטבעת מטבעות. עם הזמן, נעלם גם שימוש זה של הכתב העתיק. האלפבית העברי של ימינו הוא אפוא פיתוח של האלפבית הארמי ולא של הכתב העברי העתיק.	
{}

 לֹ֥א תִשָּׂ֛א

\endR


\egroup
\bottomline
\medskip

To make all paragraphs  RL use the \cmd{\everypar}\footnote{See discussions at \url{http://tex.stackexchange.com/questions/141867/minimal-bidi-for-typesetting-rl-text} and \url{http://www.tug.org/pipermail/xetex/2004-August/000697.html}}. 

\begin{verbatim}
\newbox\mybox \everypar{\setbox\mybox\lastbox\beginR\box\mybox}
\everypar={% at the start of each paragraph, do....
    \setbox0=\lastbox % save the paragraph indent, if any
    \beginR % set R-L direction
    \box0 % then re-insert the indent
	}
\end{verbatim}

The Hebrew alphabet has 22 letters, of which five have different forms when used at the end of a word. Hebrew is written from right to left. Originally, the alphabet was an abjad consisting only of consonants. Like other \textit{abjads}, such as the Arabic alphabet, means were later devised to indicate vowels by separate vowel points, known in Hebrew as niqqud. In rabbinic Hebrew, the letters א ה ו י are also used as matres lectionis to represent vowels. When used to write Yiddish, the writing system is a true alphabet (except for borrowed Hebrew words). In modern usage of the alphabet, as in the case of Yiddish (except that ע replaces ה) and to some extent modern Israeli Hebrew, vowels may be indicated. Today, the trend is toward full spelling with these letters acting as true vowels.



\section{Syriac}
\label{s:syriac}
\newfontfamily\syriac[Script=Syriac,OpticalSize=11]{Estrangelo Edessa}

Syriac /ˈsɪriæk/ ({\syriac{ܠܫܢܐ ܣܘܪܝܝܐ}} Leššānā Suryāyā) is a dialect of Middle Aramaic that was once spoken across much of the Fertile Crescent and Eastern Arabia.[1][2][5] 

Having first appeared as a script in the 1st century AD after being spoken as an unwritten language for five centuries,[6] Classical Syriac became a major literary language throughout the Middle East from the 4th to the 8th centuries,[7] the classical language of Edessa, preserved in a large body of Syriac literature.


It became the vehicle of Syriac Christianity and culture, spreading throughout Asia as far as the Indian Malabar Coast and Eastern China,[8] and was the medium of communication and cultural dissemination for Arabs and, to a lesser extent, Persians. Primarily a Christian medium of expression, Syriac had a fundamental cultural and literary influence on the development of Arabic,[9] which largely replaced it towards the 14th century.[3] Syriac remains the liturgical language of Syriac Christianity.

\begin{figure}[htb]
\centering
\includegraphics[width=0.7\textwidth]{./images/syriac.jpg}
\caption{11th century book in Syriac Serṭā.}
\end{figure}

Syriac is a Middle Aramaic language, and, as such, it is a language of the Northwestern branch of the Semitic family. It is written in the Syriac alphabet, a derivation of the Aramaic alphabet.

\begin{scriptexample}[]{Syriac}
\unicodetable{syriac}{"0700,"0710,"0720,"0730,"0740}
\end{scriptexample}

The Syriac Abbreviation (a type of overline) can be represented with a special control character called the Syriac Abbreviation Mark (U+070F {\syriac \char"070F ܘ}).


^^A\PrintUnicodeBlock{./languages/syriac.txt}{\syriac}





\section{Arabic}
\label{s:arabic}

The Arabic script is a writing system used for writing several languages of Asia and Africa, such as Arabic, Sorani and Luri Dialects of Kurdish language, Persian, Pashto and Urdu.[1] Even until the 16th century, it was used to write some texts in Spanish.[2] After the Latin script, Chinese characters, and Devanagari, it is the fourth-most widely used writing system in the world.[3]
The Arabic script is written from right to left in a cursive style. In most cases the letters transcribe consonants, or consonants and a few vowels, so most Arabic alphabets are abjads.

The Arabic script has its roots in the Aramaic language and the Nabataen Arabs who wrote in the Aramaic script between the first century \BC{} and third centuries \AD{}. The Nabataens were a gathering of nomadic Arab tribes living
in a region stretching from the Sinai Peninsula to northern
Arabia and eastern Jordan. In the Hellenistic era following
Alexander the Great’s conquests, they formed a kingdom that
lasted from around 150 BC until conquest by the Romans in 105
AD; their capital was the peerless rock city of Petra. Their
Nabatæn form of Aramaic writing became the immediate
parent of Arabic writing.   

The script was first used to write texts in Arabic, most notably the Qurʼān, the holy book of Islam. With the spread of Islam, it came to be used to write languages of many language families, leading to the addition of new letters and other symbols, with some versions, such as Kurdish, Uyghur, and old Bosnian being abugidas or true alphabets. It is also the basis for a rich tradition of Arabic calligraphy. Like Hebrew, Arabic is an important religious script whose
significance, longevity and expansion are owed to its veneration as a vehicle of faith. Once it was chosen to convey the Koran in the seventh
century, its hegemony in the region, and beyond, was assured.
Today, the Arabic consonantal alphabet is read and written on
the Arabian Peninsula, throughout the Near East, in western,
Central and South-East Asia, in parts of Africa and in all areas of
Europe influenced by Islam (illus. 66). 

The Arabic script has
been adapted to more languages belonging to more families
than any other Semitic script, including Berber, Somali, Swahili
(illus. 67), Urdu, Turkish, Uighur, Kazakh, Farsi (Persian),
Kashmiri, Malay, even Spanish and Slavonic in Europe.37 When
borrowed, Arabic letters were never dropped, but new or
derived letters frequently were added to reproduce sounds not
included in the Arabic inventory. Arabic facilitates this process
by distinguishing between some letters only by varying the
number of dots written with each; this function can then easily
be extended by foreign tongues needing new letters compatible
with Arabic’s fundamental appearance.38 Arabic is one of the
world’s great scripts, and will doubtless survive for many more
centuries.


\begin{Arabic}


ّ هو إذ الغاية؛ شريف الفوائد، جم المذهب، عزيز فنّ التاريخ فنّ أنّ اعلم
والملوك سيرهم، في والأنبياء أخلاقهم، في الأمم من الماضين أحوال على يوقفنا
ّ أحوال في يرومه لمن ذلك في الإقتداء فائدة تتم حتّى وسياستهم؛ دولهم في
والدنيا. الدين


\end{Arabic}

Like all Semitic scripts, Arabic uses a consonantal alphabet
commonly indicating word roots, but with a richer inventory of
28 basic letters and additional augmentations, some created by
adding a dot under existing letters (illus. 68). (A ‘29th’ letter is
the ligature of la¯m and ’a¯lif.) Arabic also inherited the long vowel
use of some consonants and the special diacritics to signal
other vowels. However, vowels in Arabic are consistently indicated
only in the Koran and in poetry. All other texts use only
consonantal writing, with diacritics assisting occasionally in
ambiguous readings. The use of ’a¯lif for long /a:/ is an Arabic
innovation. Short /a/, /i/ and /u/ make use of derived forms of
simplified consonants: for /a/, a horizontal bar over the consonant;
for /i/, a similar bar under the consonant; and for /u/, a
small hook over the consonant. If a tiny circle is written above a
consonant, this means no vowel accompanies the consonant. All
but six Arabic letters occur in four different shapes, each determined
by the letter’s position in a word: independent (the neutral
or standard shape), initial, medial or final (illus. 69).39

The oldest Islamic inscription was found in 1999 and described by ‘{}Ali ibn Ibrahim Ghamman in Zuhayr in 
Saudi Arabia and is dated \AD{644-645}\footnote{ 
The inscription of Zuhayr, the oldest Islamicinscription (24 AH/AD 644–645), the rise of theArabic script and the nature of the early Islamic state.} Hoyland\cite{hoyland2010} gives a good review of the development of Arabic as
a written language during the late Roman period in Palestine and Arabia. 


\section{Unicode}

As of Unicode 7.0, the Arabic script is contained in the following blocks:
Arabic (0600—06FF, 255 characters)
Arabic Supplement (0750—077F, 48 characters)
Arabic Extended-A (08A0—08FF, 39 characters)
Arabic Presentation Forms-A (FB50—FDFF, 608 characters)
Arabic Presentation Forms-B (FE70—FEFF, 140 characters)
Rumi Numeral Symbols (10E60—10E7F, 31 characters)
Arabic Mathematical Alphabetic Symbols (1EE00—1EEFF, 143 characters)[1][2]

The basic Arabic range encodes the standard letters and diacritics, but does not encode contextual forms (U+0621–U+0652 being directly based on ISO 8859-6); and also includes the most common diacritics and Arabic-Indic digits. The Arabic Supplement range encodes letter variants mostly used for writing African (non-Arabic) languages. The Arabic Extended-A range encodes additional Qur'anic annotations and letter variants used for various non-Arabic languages. The Arabic Presentation Forms-A range encodes contextual forms and ligatures of letter variants needed for Persian, Urdu, Sindhi and Central Asian languages. The Arabic Presentation Forms-B range encodes spacing forms of Arabic diacritics, and more contextual letter forms. The presentation forms are present only for compatibility with older standards, and are not currently needed for coding text.[3] 

The Arabic Mathematical Alphabetical Symbols block encodes characters used in Arabic mathematical expressions.


Position in word:	Isolated	Final	Medial	Initial
Glyph form:\scalebox{3}[3]{ب}{ـب}‎	ـبـ‎	 \scalebox{3}{بـ}


\printunicodeblock[2]{./languages/arabic.txt}{\arabicfont}





\section{Thaana}

\newfontfamily\thaana{MV Boli}
Thaana, Taana or Tāna ({\thaana  ތާނަ}‎ in Tāna script) is the modern writing system of the Maldivian language spoken in the Maldives. Thaana has characteristics of both an abugida (diacritic, vowel-killer strokes) and a true alphabet (all vowels are written), with consonants derived from indigenous and Arabic numerals, and vowels derived from the vowel diacritics of the Arabic abjad. Its orthography is largely phonemic.

The Thaana script first appeared in a Maldivian document towards the beginning of the 18th century in a crude initial form known as Gabulhi Thaana which was written scripta continua. This early script slowly developed, its characters slanting 45 degrees, becoming more graceful and spaces were added between words. 

As time went by it gradually replaced the older Dhives Akuru alphabet. The oldest written sample of the Thaana script is found in the island of Kanditheemu in Northern Miladhunmadulu Atoll. It is inscribed on the door posts of the main Hukuru Miskiy (Friday mosque) of the island and dates back to 1008 AH (AD 1599) and 1020 AH (AD 1611) when the roof of the building were built and the renewed during the reigns of Ibrahim Kalaafaan (Sultan Ibrahim III) and Hussain Faamuladeyri Kilege (Sultan Hussain II) respectively.

\begin{scriptexample}[]{Thaana}
\unicodetable{thaana}{"0780,"0790,"07A0,"07B0}

\hfill Typeset with MV Boli and the command \cmd{\thaana}.
\end{scriptexample}


^^A\printunicodeblock{./languages/thaana.txt}{\thaana}



\endinput










 OK
%\chapter{Albanian}

Albania's national culture came into being at the crossroads of three great civilizations:
that of Latin Catholicism from the West, that of Byzantine Greek Orthodoxy from the south, and
that of Islam imported by the Ottoman Turks, who had invaded the country in the late 14th
century and who ruled it until the declaration of independence in 1912. Early writing in this tiny
Balkan country, very much a product of these three extremely diverse cultures, was as a result a
hybrid creation. 

\section{Elbasan}
\label{s:elbasan}
\newfontfamily\elbasan{Albanian.otf}
The Elbasan script is a mid 18th-century alphabetic script used for the Albanian language. It was named after the city of Elbasan where it was invented. It was mainly used in the area of Elbasan and Berat. It is widely considered to be the first original alphabet developed for transcribing the Albanian language.

The primary document associated with the alphabet is the Elbasan Gospel Manuscript, known in Albanian as the Anonimi i Elbasanit (The Anonymous of Elbasan).[1] The document was created at St. Jovan Vladimir's Church in central Albania, but is preserved today at the National Archives of Albania in Tirana. Its 59 pages contain Biblical content written in an alphabet of 40 letters,[1] of which 35 frequently recur and 5 are rare. Dots are used on three characters as inherent features of them to indicate varied pronunciation (pre-nasalization and gemination) found in Albanian. The script generally uses Greek letters as numerals with a line on top.

Another original script used for Albanian, was Beitha Kukju's script of the 19th century. This script did not have much influence either.

Elbasan is a simple alphabetic script written from left to right horizontally. The alphabet consists of forty letters.

\subsection{Accents and Other Marks}

The Elbasan manuscript contains breathing accents, similar to
those used in Greek. Those accents do not appear regularly in the orthography and have
not been fully analyzed yet. Raised vertical marks also appear in the manuscript, but are
not specific to the script. Generic combining characters from the Combining Diacritical
Marks block can be used to render these accents and other marks.

\subsection{Names}

The names used for the characters in the Elbasan block are based on those of the
modern Albanian alphabet.

\subsection{Numerals and punctuation}

There are no script-specific numerals or punctuation marks.
A separating dot and spaces appear in the Elbasan manuscript, and may be rendered with
U+00B7 middle dot and U+0020 space, respectively. For numerals, a Greek-like system
of letter and combining overline is in use. Overlines also appear above certain letters in
abbreviations, such as $\overline{\text{\elbasan\char "10507\char"1051D}}$ to indicate Zot (Lord). The overlines in numerals and abbreviations
can be represented with U+0305 combining overline. (See also \href{http://www.unicode.org/charts/}{unicode charts}.)

\subsection{unicode}

Elbasan is a Unicode block containing the historic Elbasan characters for writing the Albanian language. Free fonts for personal use can be found at \href{http://www.fontspace.com/category/unicode\%20font\%20for\%20elbasan}{fontspace}, which I have used here. Commercial fonts can be found at Evertype.

\begin{scriptexample}[]{Elbasan}
\unicodetable{elbasan}{"10500,"10510,"10520}
\end{scriptexample}
%\chapter{South East Asian Scripts}
\label{ch:southeastasia}
\section{Introduction}

This section documents the facilities offered to typeset Southeast Asian Scripts. These scripts are used in most of Southeast Asia, Indonesia and the Philippines.

\pagestyle{headings}

\begin{table}[htb]
\centering
\begin{tabular}{lll}
  \hyperref[s:thai]{Thai} 
& Tai Tham 
& \hyperref[s:balinese]{Balinese}\\
\hyperref[s:lao]{Lao}  
&Tai Viet  
& \hyperref[s:javanese]{Javanese}\\
Myanmar 
&Kayah Li 
&Rejang\\
 \hyperref[s:khmer]{Khmer} 
&Cham 
&Batak\\
Tai Le 
&Philippine Scripts 
& \hyperref[s:sundanese]{Sundanese}\\
  \hyperref[s:newtailue]{New Tail Lue}
& Buginese\\
\end{tabular}
\end{table}

\section{Balinese}

\epigraph{In Bali the gods are thought of as the \textit{children} of the people, not as august parental figures. Speaking through the lips of those in trance, the gods address the villages as ``papa''  and ```mama'', and the people are said to spoil or indulge their gods\ldots}{Gregory Bateson and Margaret Mead in \textit{Balinese Character: A Photographic Analysis, 1942}}
\label{s:balinese}\index{Balinese}\index{Aksara Bali}\index{Bali}\index{Lombok}

\newfontfamily{\balinese}{AksaraBali.ttf}

Balinese or simply Bali\footnote{Not to be confused with the Nigerian or Papua New Guinea languages also named Bali.} is a Malayo-Polynesian language spoken by 3.3 million people (as of 2000) on the Indonesian island of Bali, as well as northern Nusa Penida, western Lombok and eastern Java.[3] Most Balinese speakers also know Indonesian. Balinese itself is not mutually intelligible with Indonesian, but may be understood by Javanese speakers after some exposure.

In 2011, the Bali Cultural Agency estimates that the number of people still using Balinese language in their daily lives on the Bali Island does not exceed 1 million, as in urban areas their parents only introduce Indonesian language or even English, while daily conversations in the institutions and the mass media have disappeared. The written form of the Balinese language is increasingly unfamiliar and most Balinese people use the Balinese language only as a spoken tool with mixing of Indonesian language in their daily conversation. But in the transmigration areas outside Bali Island, Balinese language is extensively used and believed to play an important role in the survival of the language.[4]

\begin{figure}[htbp]
\centering

\includegraphics[width=\textwidth]{bali-cock.jpg}
\end{figure}

The higher registers of the language borrow extensively from Javanese: an old form of classical Javanese, Kawi, is used in Bali as a religious and ceremonial language.

\paragraph{The Balinese script} is natively known as Aksara Bali and Hanacaraka, is an abugida used in the island of Bali, Indonesia, commonly for writing the Austronesian Balinese language, Old Javanese, and the liturgical language Sanskrit. With some modifications, the script is also used to write the Sasak language, used in the neighboring island of Lombok.[1] 

The script is a descendant of the Brahmi script, and so has many similarities with the modern scripts of South and Southeast Asia. The Balinese script, along with the Javanese script, is considered the most elaborate and ornate among Brahmic scripts of Southeast Asia.[2]

\includegraphics[width=\textwidth]{bali}


Though everyday use of the script has largely been supplanted by the Latin alphabet, the Balinese script has significant prevalence in many of the island's traditional ceremonies and is strongly associated with the Hindu religion. The script is mainly used today for copying lontar or palm leaf manuscripts containing religious texts.[2][3]



{\indicative ◌ }

\newcounter{under}
\setcounter{under}{"1B00}

\def\cb#1 {
\hspace*{2.5pt}
 
 $\text{◌#1}_{\pgfmathparse{Hex(\theunder)}\text{\pgfmathresult}}$
\stepcounter{under}
\vskip5pt\par
}
\begin{scriptexample}[]{Balinese}


\balinese
	 
᭐	᭑	᭒	᭓	᭔	᭕	᭖	᭗	᭘	᭙	᭚	᭛	᭜	᭝	᭞	᭟\\\
 
\def\columnseprulecolor{\color{thegray}}
\columnseprule.4pt
\begin{multicols}{8}

\texttt{U+1B0x}	

\cb{ᬀ }  \cb{ ᬁ } 	\cb{ ᬂ }  	\cb ᬃ	\cb ᬄ 	\cb ᬅ	\cb ᬆ	\cb ᬇ	\cb ᬈ	\cb ᬉ	\cb ᬊ	\cb ᬋ	\cb ᬌ	\cb ᬍ	\cb ᬎ	\cb ᬏ

\columnbreak

\texttt{U+1B1x}	 

\cb ᬐ	 \cb ᬑ 	\cb ᬒ 	\cb ᬓ	\cb ᬔ	\cb ᬕ	\cb ᬖ \cb ᬗ 	\cb ᬘ 	\cb ᬙ 	\cb ᬚ	\cb ᬛ 	\cb ᬜ 	\cb ᬝ 	\cb ᬞ	\cb ᬟ 

\columnbreak

U+1B2x	 

\cb ᬠ◌ 	\cb ᬡ	\cb ᬢ	\cb ᬣ	\cb ᬤ	\cb ᬥ	\cb ᬦ	\cb ᬧ	\cb ᬨ	\cb ᬩ	\cb ᬪ	\cb ᬫ	\cb ᬬ	\cb ᬭ	\cb ᬮ	\cb ᬯ

\columnbreak
U+1B3x 

\cb ᬰ	\cb ᬱ	\cb ᬲ	\cb ᬳ	\cb ᬴	\cb ᬵ	\cb ᬶ	\cb ᬷ	\cb ᬸ	\cb ᬹ	\cb ᬺ	\cb ᬻ	\cb ᬼ	\cb ᬽ	\cb ᬾ	\cb ᬿ


\columnbreak
U+1B4x	 

\cb ᭀ	 \cb ᭁ	\cb ᭂ	\cb ᭃ	\cb ᭄	\cb ᭅ	\cb ᭆ	\cb ᭇ	\cb ᭈ	\cb ᭉ	\cb ᭊ	\cb ᭋ

\columnbreak				
U+1B5x	 

\cb ᭐	\cb ᭑	\cb ᭒	\cb ᭓	\cb ᭔	\cb ᭕	\cb ᭖	\cb ᭗	\cb ᭘	\cb ᭙	\cb ᭚	\cb ᭛	\cb ᭜	\cb ᭝	\cb ᭞	\cb ᭟\\

\columnbreak

U+1B6x 

\cb ᭠	\cb ᭡	\cb ᭢	\cb ᭣	\cb ᭤	\cb ᭥	\cb ᭦	\cb ᭧	\cb ᭨◌ 	\cb ᭩◌ 	\cb ᭪◌ 	\cb ᭫	\cb ᭬	\cb ᭭	\cb ᭮	\cb ᭯

\columnbreak
U+1B7x	 

\cb ᭰	 \cb ᭱  \cb ᭲  \cb ᭳	 \cb ᭴	\cb ᭵	\cb ᭶	\cb ᭷	\cb ᭸	\cb ᭹	\cb ᭺	\cb ᭻	\cb ᭼


\end{multicols}

\end{scriptexample}


One of the most comprehensive fonts is Aksara Bali\footnote{\url{http://www.alanwood.net/downloads/index.html}}. This is obtainable at Alan Wood's website.
\clearpage

%\newfontfamily\javanese{Noto Sans Javanese}

%\newfontfamily\javanese{TuladhaJejeg_gr.ttf}

\section{Javanese}
\label{s:javanese}
\index{scripts>Javanese}


The Javanese (Ngoko Javanese: {\javanese ꦮꦺꦴꦁꦗꦮ},[3] Madya Javanese: {\javanese\   ꦠꦶꦪꦁꦗꦮꦶ},[4] Krama Javanese: ꦥꦿꦶꦪꦤ꧀ꦠꦸꦤ꧀ꦗꦮꦶ,[4] Ngoko Gêdrìk: wòng Jåwå, Madya Gêdrìk: tiyang Jawi, Krama Gêdrìk: priyantun Jawi, Indonesian: suku Jawa)[5] are an ethnic group native to the Indonesian island of Java. With approximately 100 million people (as of 2011), they form the largest ethnic group in Indonesia. They are predominantly located in the central to eastern parts of the island. There are also significant numbers of people of Javanese descent in most provinces of Indonesia, Malaysia, Singapore, Suriname, Saudi Arabia and the Netherlands.

The Javanese ethnic group has many sub-groups, such as the Mataram, Cirebonese, Osing, Tenggerese, Samin, Naganese, Banyumasan, etc.[6]

A majority of the Javanese people identify themselves as Muslims, with a minority identifying as Christians and Hindus. However, Javanese civilization has been influenced by more than a millennium of interactions between the native animism Kejawen and the Indian Hindu—Buddhist culture, and this influence is still visible in Javanese history, culture, traditions, and art forms. With a sizeable global population, the Javanese are considered significant as they are the fourth largest ethnic group among Muslims, in the world, after the Arabs,[7] Bengalis[8] and Punjabis.[9]


\paragraph{Javanese} is one of the Austronesian languages, but it is not particularly close to other languages and is difficult to classify. Its closest relatives are the neighbouring languages such as Sundanese, Madurese and Balinese. Most speakers of Javanese also speak Indonesian, the standardized form of Malay spoken in Indonesia, for official and commercial purposes as well as a means to communicate with non-Javanese-speaking Indonesians.

There are speakers of Javanese in Malaysia (concentrated in the states of Selangor and Johor) and Singapore. Some people of Javanese descent in Suriname (the Dutch colony of Suriname until 1975) speak a creole descendant of the language.

\begin{figure}[htbp]
\includegraphics[width=\textwidth]{javanese-people}
\end{figure}

The language is spoken in Yogyakarta, Central and East Java, as well as on the north coast of West Java. It is also spoken elsewhere by the Javanese people in other provinces of Indonesia, which are numerous due to the government-sanctioned transmigration program in the late 20th century, including Lampung, Jambi, and North Sumatra provinces. In Suriname, creolized Javanese is spoken among descendants of plantation migrants brought by the Dutch during the 19th century. In Madura, Bali, Lombok, and the Sunda region of West Java, it is also used as a literary language. It was the court language in Palembang, South Sumatra, until the palace was sacked by the Dutch in the late 18th century.

Javanese is written with the Latin script, Javanese script, and Arabic script.[5] In the present day, the Latin script dominates writings, although the Javanese script is still taught as part of the compulsory Javanese language subject in elementary up to high school levels in Yogyakarta, Central and East Java.

Javanese is the tenth largest language by native speakers and the largest language without official status. It is spoken or understood by approximately 100 million people. At least 45\% of the total population of Indonesia are of Javanese descent or live in an area where Javanese is the dominant language. All seven Indonesian presidents since 1945 have been of Javanese descent.[6] It is therefore not surprising that Javanese has had a deep influence on the development of Indonesian, the national language of Indonesia.

There are three main dialects of the modern language: Central Javanese, Eastern Javanese, and Western Javanese. These three dialects form a dialect continuum from northern Banten in the extreme west of Java to Banyuwangi Regency in the eastern corner of the island. All Javanese dialects are more or less mutually intelligible.


\paragraph{The Javanese script} (Hanacaraka/Carakan) is a script for writing the Javanese language, the native language of one of the peoples of the Island of Java. It is a descendent of the ancient Brahmi script of India, and so has many similarities with modern scripts of South Asia and Southeast Asia. The Javanese script is also used for writing Sanskrit, Old Javanese, and transcriptions of Kawi, as well as the Sundanese language, and the Sasak language.

\begin{figure}[htbp]
\hspace*{-1.5cm}\includegraphics[width=1.2\textwidth]{java-palm-leave-manuscript}
\end{figure}





\begin{scriptexample}[]{Javanese}
\bgroup
\javanese

꧋ꦱꦧꦼꦤ꧀ꦮꦺꦴꦁꦏꦭꦲꦶꦂꦲꦏꦺꦏꦤ꧀ꦛꦶꦩꦂꦢꦶꦏꦭꦤ꧀ꦢꦂꦧꦺꦩꦂꦠꦧꦠ꧀ꦭꦤ꧀ꦲꦏ꧀ꦲꦏ꧀ꦏꦁꦥꦝ꧉

꧋ ꦲꦮꦶꦠ꧀ꦲꦶꦏꦁꦄꦱ꧀ꦩꦄꦭ꧀ꦭꦃ꧈ ꦏꦁꦩꦲꦩꦸꦫꦃꦠꦸꦂ ꦩꦲꦲꦱꦶꦃ꧉ 	 
 ۝꧋ ꦄꦭꦶꦥꦃ꧀ ꦭ ꦩ꧀ ꦫ ꧌ ꦏꦁ — — ꦥꦿꦶꦏ꧀ꦱ ꦏꦉꦪꦥ꧀ꦥꦩꦸꦁꦄꦭ꧀ꦭꦃꦥꦶꦪꦺꦩ꧀ꦧꦏ꧀ ꧌꧉ ꦩꦁꦪꦏꦴꦪꦤꦴ ꦲꦶꦏꦸꦄꦪꦺꦪꦠꦴꦏꦶꦠꦧ꧀ꦑꦸꦂꦄꦤ꧀ꦏꦁꦥꦿꦪꦠꦭ꧉ 	 
᭐	᭑	᭒	᭓	᭔	᭕	᭖	᭗	᭘	᭙	᭚	᭛	᭜	᭝	᭞	᭟
 
\egroup
\end{scriptexample}


The Javanese script was added to Unicode Standard in version 5.2 on the code points \texttt{A980 - A9DF}. There are 91 code points for Javanese script: 53 letters, 19 punctuation marks, 10 numbers, and 9 vowels:
\medskip

\unicodetable{javanese}{"A980,"A990,"A9A0, "A9B0, "A9C0,"A9D0}

\medskip



As of the writing of this document (2017), there are several widely published fonts able to support Javanese, ANSI-based Hanacaraka/Pallawa by Teguh Budi Sayoga,[21] Adjisaka by Sudarto HS/Ki Demang Sokowanten,[22] JG Aksara Jawa by Jason Glavy,[23] Carakan Anyar by Pavkar Dukunov,[24] and Tuladha Jejeg by R.S. Wihananto,[25] which is based on Graphite (SIL) smart font technology. Other fonts with limited publishing includes Surakarta made by Matthew Arciniega in 1992 for Mac's screen font,[26] and Tjarakan developed by AGFA Monotype around 2000.[27] There is also a symbol-based font called Aturra developed by Aditya Bayu in 2012–2013.[28]

Due to the script's complexity, many Javanese fonts have different input method compared to other Indic scripts and may exhibit several flaws. \docFont{JG Aksara Jawa}, in particular, may cause conflicts with other writing system, as the font use code points from other writing systems to complement Javanese's extensive repertoire. This is to be expected, as the font was made before Javanese implementation in Unicode.[29]

Arguably, the most "complete" font, in terms of technicality and glyph count, is \docFont{TuladhaJejeg}. It comes with keyboard facilities, displaying complex syllable structure, and support extensive glyph repertoire including non-standard forms which may not be found in regular Javanese texts, by utilizing Graphite (SIL) smart font technology. |Tuladha Jejeg| uses variable stroke widths on its glyphs with serifs on some glyphs\footnote{\protect\url{https://sites.google.com/site/jawaunicode/main-page}}.

However, as not many writing systems require such complex feature, use is limited to programs with Graphite technology, such as Firefox browser, Thunderbird email client, and several OpenType word processor and of course XeLaTeX. The font was chosen for displaying Javanese script in the Javanese Wikipedia.[16]

\paragraph{jawaTeX} Jawa\TeX{} project is initial effort to make Javanese characters typesetting program using \TeX{}/\LaTeX{}. This project is aimed to make Javanese widely used. The main project is developing transliteration models to transliterate Latin document into Javanese document. Perl and \TeX{}/\LaTeX{} are use in this project, the program are develop to run in text mode (console) both Linux and Windows but not limit on it. Web based program also developed, and automatic embedded Javanese characters in HTML See \href{http://jawatex.org/jawa/jawatex}{jawatex}.




\section{Khmer}
\label{s:khmer}
\newfontfamily\normaltext{Arial}
\normaltext

\newfontfamily\khmer[Scale=1.05]{NotoSansKhmer-Regular.ttf}
\def\khmertext#1{{\khmer#1}}



The population of Cambodia was estimated at 13.5 million in 2003
(Asian Development Bank, 2003~) Approximately 90 percent of Cambodians
are ethnic Khmer and speak Khmer as their native language. Khmer
belongs to the Mon-Khmer subfamily of the Austro-Asiatic language family;
Vietnamese (in the Vietnamese subfamily) is the only other national language
in this family. The Khmer alphabet began to develop in the seventh
century and influenced the later development of the Thai writing system,
though the two languages are not mutually comprehensible; Khmer itself
was influenced by Pali and Sanskrit, and modern Khmer contains many loan
words from these Indian languages (Diffloth, 1992; Thel, 1985; Weber,
1989). Minority populations in Cambodia include Cham, Chinese, and
Vietnamese, all of which number in the hundreds of thousands (Kosonen,
2004). Somewhat more than 100,000 Cambodians belong to around 30
indigenous ethnic groups living for the most part in the northeast highlands.
Members of the relatively isolated Brao, Jarai, Krung, Tampuan, and other
communities often do not speak Khmer (Thomas, 2002, 2003).

\begin{figure}[htbp]
\includegraphics[width=\textwidth]{khmer}
\caption{Khmer Dancers}
\end{figure}


\begin{docKey}[phd]{khmer font}{=\meta{font name} (Khmer  UI)} {}
Loads the font
command \cmd{\khmer}. When the command is used it typesets text in
khmer unicode. There is no need to load the language, unless it is the main document language. For windows the default font is \texttt{DaunPenh} this font is in general too small to read; a better font to use is Khmer UI.
\end{docKey}




The Khmer script (Khmer: {\Large\khmertext{អក្សរខ្មែរ}}; IPA: [ʔaʔsɑː kʰmaːe]) [2] is an \textit{abugida} (alphasyllabary) script used to write the Khmer language (the official language of Cambodia). It is also used to write Pali among the Buddhist liturgy of Cambodia and Thailand.

It was adapted from the Pallava script, a variant of Grantha alphabet descended from the Brahmi script of India, which was used in southern India and South East Asia during the 5th and 6th Centuries AD.[3] The oldest dated inscription in Khmer was found at Angkor Borei District in Takéo Province south of Phnom Penh and dates from 611.[4] The modern Khmer script differs somewhat from precedent forms seen on the inscriptions of the ruins of Angkor.

Not all Khmer consonants can appear in syllable-final position. The most common syllable-final consonants include {\khmer កងញតនបមល}. The pronunciation of the consonant in final position may differ from it's normal pronunciation.


\begin{tabular}{l l p{9cm}}
\khmertext{ំ}	&nĭkkôhĕt (\khmertext{និគ្គហិត})	&niggahita; nasalizes the inherent vowels and some of the dependent vowels, see anusvara, sometimes used to represent [aɲ] in Sanskrit loanwords\\

\khmertext{ះ}	&reăhmŭkh (\khmertext{រះមុខ})	&"shining face"; adds final aspiration to dependent or inherent vowels, usually omitted, corresponds to the visarga diacritic, it maybe included as dependent vowel symbol\\

\khmertext{ៈ}	&yŭkôleăkpĭntŭ (\khmertext{យុគលពិន្ទុ})	&yugalabindu ("pair of dots"); adds final glottalness to dependent or inherent vowels, usually omitted\\

\khmertext{៉}	 &musĕkâtônd (\khmertext{មូសិកទន្ត})	&mūsikadanta ("mouse teeth"); used to convert some o-series consonants (\khmertext{ង ញ ម យ រ វ}) to the a-series\\

\khmertext{៊}	&treisâpt (\khmertext{ត្រីសព្ទ})	&trīsabda; used to convert some a-series consonants (\khmertext{ស ហ ប អ}) to the o-series\\
\end{tabular}




ុ	kbiĕh kraôm (ក្បៀសក្រោម)	also known as bŏkcheung (បុកជើង); used in place of the diacritics treisâpt and musĕkâtônd when they would be impeded by superscript vowels
់	bântăk (បន្តក់)	used to shorten some vowels; the diacritic is placed on the last consonant of the syllable
៌	rôbat (របាទ)
répheăk (រេផៈ)	rapāda, repha; behave similarly to the tôndâkhéat, corresponds to the Devanagari diacritic repha, however it lost its original function which was to represent a vocalic r
 ៍	tôndâkhéat (ទណ្ឌឃាដ)	daṇḍaghāta; used to render some letters as unpronounced
៎	kakâbat (កាកបាទ)	kākapāda ("crow's foot"); more a punctuation mark than a diacritic; used in writing to indicate the rising intonation of an exclamation or interjection; often placed on particles such as /na/, /nɑː/, /nɛː/, /vəːj/, and the feminine response /cah/
៏	âsda (អស្តា)	denotes stressed intonation in some single-consonant words[5]
័	sanhyoŭk sannha (សំយោគសញ្ញា)	represents a short inherent vowel in Sanskrit and Pali words; usually omitted
៑	vĭréam (វិរាម)	a mostly obsolete diacritic, corresponds to the virāma
្	cheung (ជើង)	a.w. coeng; a sign developed for Unicode to input subscript consonants, appearance of this sign varies among fonts
\section{Sundanese}
\epigraph{The married women, when their husband die, must, as point of honour, die with them, and if they should be afraid of death they put into the convents.}{Tomé Pires \textit{Suma Oriental} (1512–1515)}

\label{s:sundanese}

\newfontfamily\sundanese{Noto Sans Sundanese}
The Sundanese (Sundanese: {\sundanese ᮅᮛᮀ ᮞᮥᮔ᮪ᮓ}, Urang Sunda) are an ethnic group native to the western part of the Indonesian island of Java. They number approximately 40 million, and are the second most populous of all the nation's ethnicities. The Sundanese are predominantly Muslim. In their own language, Sundanese, the group is referred to as Urang Sunda, and Orang Sunda or Suku Sunda in the national language, Indonesian.

The Sundanese have traditionally been concentrated in the provinces of West Java, Banten, Jakarta, and the western part of Central Java. Sundanese migrants can also be found in Lampung and South Sumatra. The provinces of Central Java and East Java are home to the Javanese, Indonesia's largest ethnic group.

\begin{figure}[htbp]
\centering
\includegraphics[width=\linewidth-2\parindent]{sundanese}

\caption{Sundanese boys playing Angklung in Garut, c. 1910–1930. \href{https://en.wikipedia.org/wiki/Sundanese_people}{wikipedia}}
\end{figure}

The Sundanese script (Aksara Sunda, {\sundanese ᮃᮊ᮪ᮞᮛ ᮞᮥᮔ᮪ᮓ}) is a writing system which is used by the Sundanese people. It is built based on Old Sundanese script (Aksara Sunda Kuno) which was used by the ancient Sundanese between the 14th and 18th centuries.



\begin{scriptexample}[]{Sundanese}
\unicodetable{sundanese}{"1B80,"1B90,"1BA0,"1BB0}

\sundanese
\obeylines
\bgroup
᮱ {\arial= 1}	᮲ {\arial= 2}	᮳{\arial = 3}
᮴ {\arial= 4}	᮵ {\arial = 5} 	᮶ {\arial= 6}
᮷ {\arial= 7}	᮸ {\arial= 8}	᮹ {\arial= 9}
᮰ {\arial= 0}

\egroup
\end{scriptexample}

\begin{scriptexample}[]{Sundanese}
\bgroup
\sundanese
\centering

◌ᮃᮄᮅᮆᮇᮈᮉᮊᮋᮌᮍᮎᮏᮐᮕᮔᮓᮑᮖᮗᮚᮛᮜᮝᮞᮟᮠᮠ


\egroup
\end{scriptexample}

\bgroup
\def\1{\sundanese ᮱}
\TextOrMath\1\1

$\1$
\egroup

In text In texts, numbers are written surrounded with dual pipe sign \textbar \ldots \textbar. Example: {\textbar \sundanese ᮲᮰᮱᮰ }\textbar = 2010












%\newfontfamily\hanunoo{NotoSansHanunoo-Regular.ttf}

\section{Hanunó’o}

Hanunó’o is one of the indigenous scripts of the Philippines and is used by the Mangyan peoples of southern Mindoro to write the Hanunó'o language.[1] 

It is an \emphasis{abugida} descended from the Brahmic scripts, closely related to Baybayin, and is famous for being written vertical but written upward, rather than downward as nearly all other scripts (however, it's read horizontally left to right). It is usually written on bamboo by incising characters with a knife.[2][3] Most known Hanunó'o inscriptions are relatively recent because of the perishable nature of bamboo. It is therefore difficult to trace the history of the script



\begin{scriptexample}[width=2cm]{Hanunoo}
\hanunoo

{\Large
\obeylines
ᜠ 
ᜫ
ᜨᜲ
ᜫᜲ
ᜰ
ᜮ
ᜥ
ᜦ᜴}

Typeset with \texttt{NotoSansHanunoo-Regular.ttf} and the command \cmd{\hanunoo}
\end{scriptexample}

Vertically positionning the text is not currently supported by \pkgname{fontspec} and the manual says \textsc{Todo!}. You are your own here, or you can just put the characters in a box and give it a try.

\begin{minipage}[t]{2cm}
\begin{tcolorbox}[width=2cm,colback=graphicbackground,
boxrule=0pt,toprule=0pt,colframe=white]
\Large\hanunoo
ᜩ\\
ᜤ\\
ᜮ\\
ᜥᜳ\\
ᜨ᜴ \\
ᜨ᜴\\
ᜫᜳ\\
ᜥ\\
\end{tcolorbox}
\end{minipage}
\begin{minipage}[t]{2cm}
\begin{tcolorbox}[width=2cm,colback=graphicbackground,
boxrule=0pt,toprule=0pt,colframe=white]
\LARGE\hanunoo
ᜩ\\
ᜤ\\
ᜮ\\
ᜥᜳ\\
ᜨ᜴ \\
ᜨ᜴\\
ᜫᜳ\\
ᜥ\\
\end{tcolorbox}
\end{minipage}
\begin{minipage}[t]{\textwidth-6cm}

The script is written from bottom to top. Typesetting this type of script automatically is not without its problems. One way is to use the build-in features of the font if they are available, but currently this gives problems---at least with the fonts that I have tried. Entering the text is also problematic as you will more than likely see little boxes rather than the actual glyph with most text editors common to \latexe. If you only need a couple of characters or a short sentence, an easy solution is to use |\rotatebox|. Another solution is to use a macro that can add the letters onto a stack, then place them in a box with a limited width. We can use |\@tfor| for this.  
\end{minipage}
\section{New Tai Lue Script}
\label{s:newtailue}
\newfontfamily\tailue{Noto Sans New Tai Lue}


New Tai Lue script, also known as Simplified Tai Lue, is an alphabet used to write the Tai Lü language. Developed in China in the 1950s, New Tai Lue is based on the traditional Tai Le alphabet developed ca. 1200 AD. The government of China promoted the alphabet for use as a replacement for the older script; teaching the script was not mandatory, however, and as a result many are illiterate in New Thai Lue. 

\begin{figure}[htbp]
\centering

\includegraphics[width=\linewidth-2\parindent]{tailue}

\caption{Tai Le costumes. (pininterest)}
\end{figure}

In addition, communities in Burma, Laos, Thailand and Vietnam still use the Tai Le alphabet. There are probably less than one million native speakers of the language who can be found in China, Burma, Laos, Thailand and Vietnam.

\begin{figure}[htbp]
\centering

\includegraphics[width=\linewidth-2\parindent]{tai-lu}

\caption{Tai Le costumes. (pininterest)}
\end{figure}

\begin{scriptexample}[]{Tai Lue}
{\centering\tailue \LARGE

ᦒ	ᦓ	ᦔ	ᦕ	ᦖ	ᦗ	ᦘ	ᦙ	ᦚ	ᦛ	ᦜ	ᦝ	ᦞ	

}
\end{scriptexample}

The New Tai Lue script was added to the Unicode Standard in March, 2005 with the release of version 4.1.

The Unicode block for New Tai Lue is |U+1980|–|U+19DF|:

\begin{scriptexample}[]{New Tai Lue}
\unicodetable{tailue}{"1980,"1990,"19A0,"19B0,"19C0,"19D0}

\texttt{typeset using NotoSansNewTaiLue-Regular.ttf.}
\end{scriptexample}
\section{Myanmar}
\label{s:myanmar}
\index{Myanmar}\index{Burmese}\index{Mon}\index{Unicode>Myanmar}\index{Fonts>Padauk}

%\newfontfamily\myanmar{Padauk}

The Burmese script (Burmese:{\myanmar မြန်မာအက္ခရာ}; MLCTS: mranma akkha.ra; pronounced: [mjəmà ʔɛʔkʰəjà]) is an abugida in the Brahmic family, used for writing Burmese. It is an adaptation of the Old Mon script[2] or the Pyu script. In recent decades, other alphabets using the Mon script, including Shan and Mon itself, have been restructured according to the standard of the now-dominant Burmese alphabet. Besides the Burmese language, the Burmese alphabet is also used for the liturgical languages of Pali and Sanskrit.

The characters are rounded in appearance because the traditional palm leaves used for writing on with a stylus would have been ripped by straight lines.[3] It is written from left to right and requires no spaces between words, although modern writing usually contains spaces after each clause to enhance readability.

The earliest evidence of the Burmese alphabet is dated to 1035, while a casting made in the 18th century of an old stone inscription points to 984.[1] Burmese calligraphy originally followed a square format but the cursive format took hold from the 17th century when popular writing led to the wider use of palm leaves and folded paper known as \emph{parabaiks}.[3] The alphabet has undergone considerable modification to suit the evolving phonology of the Burmese language.

Mon/Burmese script was added to the Unicode Standard in September, 1999 with the release of version 3.0. It was extended in October, 2009 with the release of version 5.2 and again in June, 2014 with the release of version 7.0.

\begin{docKey}[phd]{myanmar font}{=\meta{font name}}{default none initial Padauk}
Loads the font and creates associated environments and commands.
\end{docKey}

\begin{scriptexample}[]{Myanmar}
\unicodetable{myanmar}{"1000,"1010,"1020,"1030,"1040,"1050,"1060,"1070,"1080,"1090}
\end{scriptexample}


{\myanmar
\begin{tabular}{l l l l}
\toprule
Number         &Numeral	 &Written     &\\
\midrule
               &	          &(MLCTS)     &IPA \\
0	            &၀	          &သုည (su.nya.) & IPA: [θòʊɴɲa̰]\\
1	            & ၁	       &တစ် (tac)	 &IPA: [tɪʔ]\\
2              &၂          &နှစ်        &IPA: [n̥ɪʔ]\\
\bottomrule
\end{tabular}


	
1	၁	တစ်
(tac)	IPA: [tɪʔ]
2	၂	နှစ်
(hnac)	IPA: [n̥ɪʔ]
3	၃	သုံး
(sum:)	IPA: [θóʊɴ]
4	၄	လေး
(le:)	IPA: [lé]
5	၅	ငါး
(nga:)	IPA: [ŋá]
6	၆	ခြောက်
(hkrauk)	IPA: [tɕʰaʊʔ]
7	၇	ခုနစ်
(hku. nac)	IPA: [kʰʊ̀ɴ n̥ɪʔ]2
8	၈	ရှစ်
(hrac)	IPA: [ʃɪʔ]
9	၉	ကိုး
(kui:)	IPA: [kó]
10	၁၀	ဆယ်
(ta. hcai)	IPA: [sʰɛ̀]
}




%\section{Oriya}
\label{s:oriya}
\index{Indic scripts>Oriya}
\epigraph{Oṛiyā is encumbered with the drawback of an excessively awkward and cumbrous written character. ... At first glance, an Oṛiyā book seems to be all curves, and it takes a second look to notice that there is something inside each.}{(G. A. Grierson, \textit{Linguistic Survey of India}, 1903)}

\newfontfamily\oriya[Scale=1.1,Script=Oriya]{Noto Sans Oriya}

\def\oriyatext#1{{\oriya#1}}
The Oriya script or Utkala Lipi (Oriya: \oriyatext{ଉତ୍କଳ ଲିପି}) or Utkalakshara (Oriya: \oriyatext{ଉତ୍କଳାକ୍ଷର}) is used to write the Oriya language, and can be used for several other Indian languages, for example, Sanskrit.

\centerline{\Huge\oriyatext{ଉତ୍କଳ ଲିପି}}

\bgroup
\oriya
୦୧୨୩୪୫୬୭୮୯
ଅ ଆ ଇ ଈ ଉ ଊ ଋ ୠ ଌ ୡ ଏ ଐ ଓ ଔ କ ଖ ଗ ଘ ଙ ଚ ଛ ଜ ଝ ଞ ଟ ଠ ଡ ଢ ଣ ତ ଥ ଦ ଧ ନ ପ ଫ ବ ଵ ଭ ମ ଯ ର ଳ ୱ ଶ ଷ ସ ହ ୟ ଲ
\egroup






\begin{figure}[htbp]
\centering

\includegraphics[width=\linewidth-2\parindent]{oriya-people}

\hspace*{-1em}\caption{Children dressed for celebration of Janmashtami, which marks the birth of Lord Krishna. odisha360.com}
\end{figure}

Comparison of Oṛiyā script with its neighbours

At a first look the great number of signs with round shapes suggests a closer relation to the southern neighbour Telugu than to the other neighbours Bengali in the north and Devanāgarī in the west. The reason for the round shapes in Oriya and Telugu (and also in Kannaḍa and Malayāḷam) is the former method of writing using a stylus to scratch the signs into a palm leaf. These tools do not allow for horizontal strokes because that would damage the leaf.

Oriya letters are mostly round shaped whereas in Devanāgarī and Bengali have horizontal lines. So in most cases the reader of Oṛiyā will find the distinctive parts of a letter only below the hoop. Considering this the  closer relation to Devanāgarī and Bengali exists than to any southern script, though both northern and southern scripts have the same origin, Brāhmī.

Oriya (\oriyatext{ଓଡ଼ିଆ} oṛiā), officially spelled Odia,[3][4] is an Indian language belonging to the Indo-Aryan branch of the Indo-European language family. It is the predominant language of the Indian states of Odisha, where native speakers comprise 80\% of the population,[5] and it is spoken in parts of West Bengal, Jharkhand, Chhattisgarh and Andhra Pradesh. Oriya is one of the many official languages in India; it is the official language of Odisha and the second official language of Jharkhand. [6][7][8] Oriya is the sixth Indian language to be designated a Classical Language in India, on the basis of having a long literary history and not having borrowed extensively from other languages.



\printunicodeblock{./languages/oriya.txt}{\oriya}

\section{Numerals}

{\oriya
\obeylines
୦	୧	୨	୩	୪	୫	୬	୭	୮	୯	୵	୶	୷	୲	୳	୴
{\arial 0	1	2	3	4	5	6	7	8	9	¹⁄₁₆	⅛	³⁄₁₆	¼	½	¾}
}





%\cxset{image = mongolian-people}
\chapter{Mongolian}
The Mongols (Mongolian: Монголчууд, Mongolchuud, [ˈmɔŋ.ɡɔɮ.t͡ʃʊːt]) are an East-Central Asian ethnic group native to Mongolia and China's Inner Mongolia Autonomous Region. They also live as minorities in other regions of China (e.g. Xinjiang), as well as in Russia. Mongolian people belonging to the Buryat and Kalmyk subgroups live predominantly in the Russian federal subjects of Buryatia and Kalmykia.\footnote{Cover image from \href{http://www.bbc.com/news/world-asia-china-25979564}{bbc}}

The Mongols are bound together by a common heritage and ethnic identity. T
heir indigenous dialects are collectively known as the Mongolian language. The ancestors of the modern-day Mongols are referred to as Proto-Mongols.

Mongolian is a member of a language family technically known as “Mongolic”. Apart
from Mongolian, or Mongol proper, the Mongolic language family comprises a dozen
other languages, spoken mainly in regions adjacent to Mongolia. Historically, the
Mongolic language family was formed as a result of the political expansion of the mediaeval,
or “historical”, Mongols under Chinggis Khan (Cingges Xaan) and his descendants
in the 12th–13th centuries. During the initial period of the Mongol empire, the Mongols
controlled, as a politically unified territory, the entire Central Asian belt from the Middle
East to China. The subsequent Mongol dynasty of the Yuan (1279–1368) in the eastern
part of the former Mongol empire, comprised China, Mongolia, Manchuria, Tibet and
Eastern Turkestan.\footcite{book:janhunen2012}

The language of the historical Mongols was based on the local idiom once spoken in
northeastern Mongolia, the native region of Chinggis Khan. With the consolidation of
the political power, this idiom became the koïné of the expanding Mongols, who brought
it to various parts of the empire. The language was widely used in civil and military
administration, and through the Mongol garrisons it gained ground also among local
non-Mongol populations. As a spoken medium, the language of the historical Mongols
is known as Middle Mongol, or Middle Mongolian. Middle Mongol is documented in
a variety of written sources using several different systems of script. With the course of
time, and especially after the collapse of the Mongol empire Middle Mongol was diversified
into several local varieties, from which the modern Mongolic languages have
developed.


Janhunen\footcite{book:janhunen2012} divides the extant Mongolic languages into four geographically
and linguistically distinct branches: Dagur, Common Mongolic,
Shirongolic and Moghol.

\begin{description}

\item[Dagur] branch, located in the northeast (Manchuria) and comprising only the
Dagur language (with several local varieties, including the Amur, Nonni and Hailar
groups of dialects, as well as, since the 18th century, a diaspora group in the Yili
region of Dzungaria); historically, the origins of this branch would seem to be connected
with the earliest breakup of Proto-Mongolic;

\item[Common Mongolic] branch, centered on the traditional homeland of the Mongols
(Mongolia), but extending also to the north (Siberia), east (Manchuria), south
(Ordos) and west (Dzungaria), and comprising a group of closely related forms of
speech, which by the native speakers themselves are often understood as “dialects”
of a single “Mongolian” language;

\item[Shirongolic] branch, located in the Amdo or Kuku Nor (Xeux Noor ‘Blue Lake’)
region of ethnic Tibet (the modern Gansu and Qinghai Provinces of China), and
comprising a number of particularly idiosyncratic and mutually unintelligible
languages spoken by several culturally diversified populations, including Shira
Yughur (Mongolic Yellow Uighur), the Monguor group (Mongghul, Mongghuor,
Mangghuer) and the Bonan group (Bonan, Kangjia, Santa);

\item[Moghol] the Moghol branch, located in Afghanistan and comprising only the Moghol language
(with several local varieties, possibly extinct today).\footcite{book:janhunen2012}
\end{description}

\begin{figure}[htbp]
\includegraphics[width=\textwidth]{mongolian-writing}
\caption{Nova N 176 found in Kyrgyzstan. The manuscript (dating to the 12th century Western Liao) is written in the Mongolic Khitan language using cursive Khitan large script. It has 127 leaves and 15,000 characters.}
\end{figure}

From historical documents it is evident that the lineage represented by the language
of the historical Mongols once had relatives, today technically identified as the Para-
Mongolic languages, spoken until mediaeval times in parts of southwestern Manchuria.

\paragraph{The literary languages}
The earliest known written language for
the historical Mongols was created in the 11th–12th centuries on the basis of a Semitic
alphabet adopted via the Turkic-speaking Ancient Uighurs. The script, in its Mongolian
form, has subsequently become known as the Mongol Script, while the language written
in it is known as Written Mongol or Written Mongolian, or also Literary Mongol
or Literary Mongolian. Written Mongol was reinforced by Chinggis Khan as a general
medium of administration and literature, and in its early form it was essentially identical
with contemporary spoken Middle Mongol, complicated only by certain orthographical
conventions, some of which may actually reflect a stage preceding Middle Mongol and
Proto-Mongolic.

Written Mongol has ever since remained in use as the principal literary language of
the Mongols. Evolving successively through stages termed Pre-Classical (13th to 15th
centuries), Classical (17th to 19th centuries) and Post-Classical (20th century) Written
Mongol, the language, especially as far as its orthographical principles are concerned, still
retains many of its original characteristics. This means that it remains largely unaffected
by the innovations that have taken place in the spoken language and by the diversification
of the latter into the extant modern Mongolic languages. This is particularly true of
the phonological features reflected by the Written Mongol orthography. Written Mongol
has, however, survived only among the speakers of the Common Mongolic idioms, and
even of the latter, the speakers of Buryat and Khamnigan have used it only marginally.
The significance of Written Mongol as a unifying factor for almost all Common
Mongolic speakers can hardly be exaggerated. Even so, its status has been gradually
undermined by the creation of new literary languages, which today cover most of the
Common Mongolic populations living outside of Inner Mongolia. These new literary
languages include:

\begin{enumerate}
\item Written Oirat or the “Clear Script” (Tod Biceg), which was created on the basis of
Written Mongol as early as 1648 for use by the Western Mongols of Dzungaria; this
script is still in use among some of the Oirat groups in Sinkiang;

\item Romanized “Buryat”, which was standardized around 1930 on the basis of what are
actually the Sartul and Tsongol dialects of northern Khalkha, spoken on the Russian
side of the border

\item Cyrillic Buryat, based on the Khori dialect of actual (Eastern) Buryat, which replaced
the earlier Romanized “Buryat” in 1937 and remains in use as the literary language
for the Buryat living in the Russian Federation; the written standard is, however, not
used by the Buryat speakers living in Mongolia and China;

\item Cyrillic Kalmuck, which was standardized in the early 1930s for use by the Volga
Kalmuck, who represent an Oirat diaspora group that has been living under Russian
rule since the 17th century;

\item Cyrillic Khalkha, based on the central dialects of the Khalkha group, which were
developed as the national language of Outer Mongolia after independence, and
which during the 1940s more or less fully replaced Written Mongol as the official
standard language of the country.
\end{enumerate}

\paragraph{Classical Mongolian Script} The classical Mongolian script (in Mongolian script: {\mongolian  ᠮᠣᠩᠭᠣᠯ ᠪᠢᠴᠢᠭ᠌} Mongγol bičig; in Mongolian Cyrillic: Монгол бичиг Mongol bichig), also known as Uyghurjin Mongol bichig, was the first writing system created specifically for the Mongolian language, and was the most successful until the introduction of Cyrillic in 1946. Derived from Uighur, Mongolian is a true alphabet, with separate letters for consonants and vowels. The Mongolian script has been adapted to write languages such as Oirat and Manchu. Alphabets based on this classical vertical script are used in Inner Mongolia and other parts of China to this day to write Mongolian, Sibe and, experimentally, Evenki.
\medskip

\bgroup\par
\noindent
\colorbox{thecodebackground}{\color{black}^^A
\begin{minipage}{\textwidth}^^A
\parindent1pt
\vskip10pt
\leftskip10pt \rightskip\leftskip
\mongolian
\Large
ᠬᠦᠮᠦᠨ ᠪᠦᠷ ᠲᠥᠷᠥᠵᠦ ᠮᠡᠨᠳᠡᠯᠡᠬᠦ ᠡᠷᠬᠡ ᠴᠢᠯᠥᠭᠡ ᠲᠡᠢ᠂ ᠠᠳᠠᠯᠢᠬᠠᠨ ᠨᠡᠷ᠎ᠡ ᠲᠥᠷᠥ ᠲᠡᠢ᠂ ᠢᠵᠢᠯ ᠡᠷᠬᠡ ᠲᠡᠢ ᠪᠠᠢᠠᠭ᠃ ᠣᠶᠤᠨ ᠤᠬᠠᠭᠠᠨ᠂ ᠨᠠᠨᠳᠢᠨ ᠴᠢᠨᠠᠷ ᠵᠠᠶᠠᠭᠠᠰᠠᠨ ᠬᠦᠮᠦᠨ ᠬᠡᠭᠴᠢ ᠥᠭᠡᠷ᠎ᠡ ᠬᠣᠭᠣᠷᠣᠨᠳᠣ᠎ᠨ ᠠᠬᠠᠨ ᠳᠡᠭᠦᠦ ᠢᠨ ᠦᠵᠢᠯ ᠰᠠᠨᠠᠭᠠ ᠥᠠᠷ ᠬᠠᠷᠢᠴᠠᠬᠥ ᠤᠴᠢᠷ ᠲᠠᠢ᠃
\par
\vspace*{10pt}
\end{minipage}
}
\medskip


\paragraph{Unicode Encoding} Mongolian is a Unicode block containing characters for dialects of Mongolian, Manchu, and Sibe languages. It is traditionally written in vertical lines Text direction TDright.svg Top-Down, right across the page, although the Unicode code charts cite the characters rotated to horizontal orientation.
The block has dozens of variation sequences defined for standardized variants.
\bigskip


\unicodetable{mongolian}{"1800,"1810,"1820,"1830,"1840,"1850,"1860,"1870,"1880,"1890,"18A0}
\bigskip

\section{LaTeX}

The \pkg{montex} provides a full system including transliterations.\footcite{montex} There is no as yet support for LuaTeX and I do not see this forthcoming anytime soon. 











%\section{Tibetan}
\label{tibetan}
\index{scripts>tibetan}


Another important Northern Indian member perhaps
derived directly from Gupta---and thus a sister script to Nagari,
Sarada and Pali---is Tibetan. However, the Tibetan
language wears this foreign Indo-Aryan script most uncomfortably.\cite{writing}
The script retains the Indic consonantal alphabet with diacritic
attachments to indicate vowels – but with only one vowel
letter, the /a/, which is the same as the system’s own ‘default’ /a/.
This /a/ letter is then used to attach other diacritics in order to
indicate further vowels. Because the Tibetan language has
changed greatly since c. AD 700 (when the script was first elaborated
from Gupta) while the script has remained almost
unchanged, Tibetan is extremely difficult to read today. Its
greatest problem is that it marks none of the tones of its tonal
language. Though Tibetans have long tried to adapt written
Tibetan to spoken Tibetan, high illiteracy has been the price of
failing to achieve this. Tibetan schools in Tibet, by governmental
decree, now teach only the Chinese script and in the Chinese
language.

\newfontfamily\tibetan{TibMachUni.ttf}

\newfontfamily\tibetan{Qomolangma-Chuyig.ttf}

%A should pick it up automatically \tibetan

Fonts described in this section can be obtained from The Tibetan \& Himalayan Library
\footnote{\url{http://www.thlib.org/tools/scripts/wiki/tibetan\%20machine\%20uni.html}  }

I have tried a few \texttt{Tibetan Machine Uni (TMU)} seems to be used by a number of scholars. 

A tip when you are trying to locate fonts is to find a related article in Wikipedia, such as Tibetan alphabet and inspect the element using your browser to see what fonts are being used.


|style="font-family:'Jomolhari','Tibetan Machine Uni','DDC Uchen', 'Kailash';| 


If you cannot see the script and rather than boxes or question marks then you can search and download one of the fonts in |font-family|.



\begin{docKey}[phd]{language}{ = tibetan}{default none, initial english} 
The key |language=tibetan| sets the default language as Tibetan, using the main font given by the key |tibetan font=TibMachUni.ttf|.

It will also create an environment tibetanlanguage.
\end{docKey}

\begin{docKey}[phd]{tibetan font}{= TibMachUni.ttf} {initial = TibMachUni.ttf} 
The key |tibetan font=font-name| sets the default font for the Tibetan language. It will also create the switch \cmd{\tibetan} for typesetting text in Tibetan.
\end{docKey}


\begin{docEnvironment}{tibetan}{}
\end{docEnvironment}

The environment is created automatically
\begin{texexample}{Tibetan language setttings}{ex:tibetan}
\bgroup
\cxset{language=tibetan, tibetan font = TibMachUni.ttf}

\tibetan Tibetan: དབུ་ཅན\par
ཨོཾ་ཨཿཧཱུྂ་བཛྲ་གུ་རུ་པདྨ་སིདྡྷི་ཧཱུྂ༔\par
\egroup

\begin{tibetanlanguage}
The tibetan environment\par
ཨོཾ་ཨཿཧཱུྂ་བཛྲ་གུ་རུ་པདྨ་སིདྡྷི་ཧཱུྂ༔
\end{tibetanlanguage}
\end{texexample}


The Tibetan alphabet is an \emph{abugida} of Indic origin used to write the Tibetan language as well as Dzongkha\footnote{Spoken in Bhutan.}, the Sikkimese language, Ladakhi, and sometimes Balti. 

The printed form of the alphabet is called \textit{uchen} script (Tibetan: དབུ་ཅན་, Wylie: dbu-can; "with a head") while the hand-written cursive form used in everyday writing is called umê script (Tibetan: དབུ་མེད་, Wylie: dbu-med; "headless").

The alphabet is very closely linked to a broad ethnic Tibetan identity. Besides Tibet, it has also been used for Tibetan languages in Bhutan, India, Nepal, and Pakistan.[1] The Tibetan alphabet is ancestral to the Limbu alphabet, the Lepcha alphabet,[2] and the multilingual 'Phags-pa script.[2]


The Tibetan alphabet is romanized in a variety of ways.[3] This article employs the Wylie transliteration system.

The Tibetan alphabet has thirty basic letters, sometimes known as "radicals", for consonants.[2]

{\tibetanfontfamily
ཀ ka /ká/	ཁ kha /kʰá/	ག ga /kà, kʰà/	ང nga /ŋà/\\
ཅ ca /tʃá/	ཆ cha /tʃʰá/	ཇ ja /tʃà/	ཉ nya /ɲà/\\
ཏ ta /tá/	ཐ tha /tʰá/	ད da /tà, tʰà/	ན na /nà/\\
པ pa /pá/	ཕ pha /pʰá/	བ ba /pà, pʰà/	མ ma /mà/\\
ཙ tsa /tsá/	ཚ tsha /tsʰá/	ཛ dza /tsà/	ཝ wa /wà/ (not originally part of the alphabet)[5]\\
ཞ zha /ʃà/[6]	ཟ za /sà/	འ 'a /hà/[7]\\
ཡ ya /jà/	ར ra /rà/	ལ la /là/\\
ཤ sha /ʃá/[6]	ས sa /sá/	ཧ ha /há/[8]\\
ཨ a /á/\\
}


Tibetan is not a difficult script to read or write, but it is a very complex script to deal with in terms of computer processing (as far as complexity goes I would rate it second only to the Mongolian script). The problem is that written Tibetan comprises complex syllable units (known in Tibetan as a tsheg bar {\tibetan ཚེག་བར}) which although written horizontally may include \emph{vertical} clusters of consonants and vowel signs agglutinating around a base consonant (a vertical cluster is known as a "stack"). 

Thus most words have a horizontal and a vertical dimension, with the result that text is not laid out in a straight line as in most scripts. For example, the word bsGrogs བསྒྲོགས་ (pronounced drok ... obviously!) may be analysed as follows :

\definecolor{lavenderblush}{HTML}{FFF0F5}%
\definecolor{beige}{HTML}{F5F5DC}%


{\tibetan 
\HUGE བསྒྲོགས

{\color{beige}%
\symbol{"0F56}\color{blue!40}\color{red}\symbol{"0F66}\symbol{"0F92}\color{blue!80}\symbol{"0FB2}\color{beige}\symbol{"0F7C}\color{blue!25}\symbol{"0F42}\symbol{"0F66}\symbol{"0F0B}}



\begin{tabular}{|l|}
\symbol{"0F56}\symbol{"0F7C}\\
\symbol{"0F42}\symbol{"0F7C}\\
\symbol{"0F66}\symbol{"0F7C}\\
\symbol{"0F40}\symbol{"0F7C}\\
\end{tabular}
}

\subsection{Unicode Block Tibetan}


\bgroup\large\tibetan
\begin{tabular}{llllllllllllllll l}
\toprule
	           &|0|	&|1|	&|2|	&|3|	&|4|	&|5|	&|6|	&|7|	&|8|	&|9|	&|A|	&|B|	&|C|	&|D|	&|E|	&|F|\\
\midrule
\texttt{U+0F0x}	&ༀ	&༁	&༂	&༃	&༄	&༅	&༆	&༇	&༈	&༉	&༊	&་	&༌  &	།	&༎	&༏\\
\midrule
\texttt{U+0F1x} &༐	&༑	&༒	&༓	&༔	&༕	&༖	&༗	&༘&	༙	&༚	&༛	&༜	&༝	&༞	&༟\\
\midrule
\texttt{U+0F2x} &༠	&༡	&༢	&༣	&༤	&༥	&༦	&༧	&༨	&༩	&༪	&༫	&༬	&༭	&༮	&༯\\
\midrule
\texttt{U+0F3x}	&༰ &༱	 &༲ &༳	&༴ &༵	&༶ & ༷	&༸&	༹	&༺&	༻	&༼&	༽	&༾	&༿\\
\midrule
\texttt{U+0F4x} &ཀ	&ཁ	&ག	&གྷ	&ང	&ཅ	&ཆ	&ཇ	&	&ཉ	&ཊ	&ཋ	&ཌ	&ཌྷ	&ཎ	&ཏ\\
\midrule
\texttt{U+0F5x}	 &ཐ	&ད	&དྷ	&ན	&པ	&ཕ	&བ	&བྷ	&མ	&ཙ	&ཚ	&ཛ	&ཛྷ	&ཝ	&ཞ	&ཟ\\
\midrule
\texttt{U+0F6x} &འ	&ཡ	&ར	&ལ	&ཤ	&ཥ	&ས	&ཧ	&ཨ	&ཀྵ	&ཪ	&ཫ	&ཬ	&&&\\
^^A\texttt{U+0F7x}&&	ཱ &	& &ི	ཱི&	ུ&	ཱུ&	ྲྀ&	ཷ&	ླྀ&	ཹ&	ེ&	ཻ&	ོ&	ཽ&	&ཾ	&ཿ\\
\midrule
\texttt{U+0F8x}&    ྀ   & 	ཱྀ&	ྂ&	&ྃ &	྄	&྅&	྆	&྇	ྈ&	ྉ&	ྊ&	ྋ&	ྌ&	ྍ&	ྎ&	ྏ\\
\midrule
\texttt{U+0F9x} &	ྐ&	ྑ   & 	ྒ &	ྒྷ &	ྔ &	ྕ &	ྖ &	ྗ &		ྙ &	ྚ &	ྛ &	ྜ &	ྜྷ &	ྞ &	ྟ\\
\texttt{U+0FAx} &	ྠ &	ྡ &	ྡྷ &	ྣ &	ྤ &	ྥ &		&ྦ	&ྦྷ	ྨ&	ྩ&	ྪ&	ྫ&	ྫྷ&	ྭ&	ྮ&	ྯ\\
\midrule
\texttt{U+0FBx} 
&	  ྰ 
&	
& ྱ  	 
&ྲ	
&ླ	
&ྴ
&	ྵ
&	ྶ
&	ྷ
&ྸ
&
&
&
&	
&྾	
&྿\\
\midrule
\texttt{U+0FCx}	 &࿀&	࿁&	࿂&	࿃&	࿄&	࿅&	&࿇	&࿈	&࿉	&࿊	&࿋	&࿌	&&	࿎	&࿏\\
\midrule
\texttt{U+0FDx}	&࿐	&࿑	&࿒	&࿓	&࿔	&࿕	&࿖	&࿗	&࿘	&࿙	&࿚	&&&&&\\
\midrule
\texttt{U+0FEx} &&&&&&&&&&&&&&&&\\
\midrule
\texttt{U+0FFx}  &&&&&&&&&&&&&&&&\\
\bottomrule
\end{tabular}
\egroup




\subsection{Fonts for Tibetan}

Fonts for Tibetan need to be downloaded one set of fonts are the \texttt{Qomolangma}. They come in different flavours, but they appear
to offer advantages as compared to the Tibetan Machine Uni.
\medskip


\newfontfamily\betsu{Qomolangma-Betsu.ttf}
\newfontfamily\drutsa{Qomolangma-Drutsa.ttf}
\newfontfamily\chuyig{Qomolangma-Chuyig.ttf}
\newfontfamily\tsumachu{Qomolangma-Tsumachu.ttf}
\newfontfamily\uchensutung{Qomolangma-UchenSutung.ttf}
\newfontfamily\uchensuring{Qomolangma-UchenSuring.ttf}
\newfontfamily\uchensarchen{Qomolangma-UchenSarchen.ttf}
\newfontfamily\uchensarchung{Qomolangma-UchenSarchung.ttf}
\newfontfamily\tsuring{Qomolangma-Tsuring.ttf}
\newfontfamily\TMU{TibMachUni.ttf}
\newfontfamily\himalaya{Microsoft Himalaya}


{
\centering

\renewcommand{\arraystretch}{1.5}

\begin{tabular}{lr}
\toprule
|Qomolangma-Betsu.ttf| & {\betsu  དབུ་མེད }\\
\midrule
|Qomolangma-Chuyig.ttf| &{\chuyig  དབུ་མེད}\\
\midrule
|Qomolangma-Drutsa.ttf| &{\drutsa  དབུ་མེད}\\
\midrule
|Qomolangma-Tsumachu.ttf|&{\tsumachu  དབུ་མེད}\\
\midrule
|Qomolangma-Tsuring.ttf| &{\tsuring  དབུ་མེད}\\
\midrule
|Qomolangma-UchenSarchen.ttf| &{\uchensarchen དབུ་མེད}\\
\midrule
|Qomolangma-UchenSarchung.ttf|&{\uchensarchung དབུ་མེད }\\
\midrule
|Qomolangma-UchenSuring.ttf|&{\uchensuring དབུ་མེད}\\
\midrule
|Qomolangma-UchenSutung.ttf|&{\uchensutung དབུ་མེད }\\
\midrule
|TibMachUni.ttf| &{\TMU དབུ་མེད }\\
\midrule
|Microsoft Himalaya| &{\himalaya དབུ་མེད ཽ}\\
\bottomrule
\end{tabular}

}
\bigskip

\bgroup
\LARGE\tsuring
\noindent༆ །ཨ་ཡིག་དཀར་མཛེས་ལས་འཁྲུངས་ཤེས་བློ  འི་\par
གཏེར༑ །ཕས་རྒོལ་ཝ་སྐྱེས་ཟིལ་གནོན་གདོང་ལྔ་བཞིན།།\par
ཆགས་ཐོགས་ཀུན་བྲལ་མཚུངས་མེད་འཇམ་དབྱངསམཐུས།།\par
མཧཱ་མཁས་པའི་གཙོ་བོ་ཉིད་འགྱུར་ཅིག། །མངྒལཾ༎\par
བསྒྲོགས
\egroup

\subsubsection{Tibetan numbers}
\cxset{language=tibetan, tibetan font = TibMachUni.ttf}

{
\obeylines
\small
TIBETAN DIGIT ZERO\tibetan	༠
TIBETAN DIGIT ONE	\tibetan༡	
TIBETAN DIGIT TWO\tibetan	༢	
TIBETAN DIGIT THREE\tibetan	༣	
TIBETAN DIGIT FOUR	\tibetan ༤	
TIBETAN DIGIT FIVE\tibetan	༥	
TIBETAN DIGIT SIX	\tibetan ༦	
TIBETAN DIGIT SEVEN\tibetan	༧	
TIBETAN DIGIT EIGHT\tibetan	༨	
TIBETAN DIGIT NINE\tibetan	༩	
TIBETAN DIGIT HALF ONE	\tibetan༪	
TIBETAN DIGIT HALF TWO	༫	
TIBETAN DIGIT HALF THREE	༬
TIBETAN DIGIT HALF FOUR ༭	
TIBETAN DIGIT HALF FIVE ༯	
TIBETAN DIGIT HALF SIX	 ༯	
TIBETAN DIGIT HALF SEVEN	༰	
TIBETAN DIGIT HALF EIGHT	༱	
TIBETAN DIGIT HALF NINE	༲	
TIBETAN DIGIT HALF ZERO	༳	
}


Tibetan numbers

The usage is not certain. By some interpretations, this has the value of 9.5. Used only in some traditional contexts, these appear as the last digit of a multidigit number, eg. ༤༬ represents 42.5. These are very rarely used, however, and other uses have been postulated.


\PrintUnicodeBlock{./languages/tibetan.txt}{\himalaya}






\chapter{Tamil}

\epigraph{Women live like bats or owls.\\Labour like beasts\\and die like worms\ldots}{Margaret of Newcastle, 1660, England}



\label{s:tamil}
\newfontfamily\tamil[Scale=1.0, Script=Tamil]{code2000.ttf}

\def\tamiltext#1{{\tamil#1}}

\section{Background and History}

Of all the Dravidian languages Tamil has the longest literary tradition, covering
more than two thousand years. The earliest records are cave inscriptions from
the second century \textsc{bce}; the earliest extant literary text is the grammar
Tolkāppiyam (100 \textsc{bce}), which describes the grammar and poetics of Tamil during
that period. The dating of the Tolkāppiyam is still disputed by scholars proposing dates from
5 \textsc{bce} to 600 \textsc{ce}. 

During its two-thousand-year uninterrupted history, Tamil distinguishes
three different stages: Old Tamil (300 \textsc{bce} to 700 \textsc{ce}), Middle Tamil (700
\textsc{ce} to 1600) and Modern Tamil (1600 \textsc{ce} to the present), each with distinct
grammatical characteristics.\index{Dravidian>Tamil}\index{Tamil}


\begin{figure}[htbp]
\bgroup
\parindent=0pt
\centering
\includegraphics[width=0.9\linewidth-2\parindent]{./images/old-tamil-inscription.jpg}

\caption{Mangulam Tamil Brahmi inscription at Dakshin Chithra, Chennai (wikipedia)}

\egroup
\end{figure}

The Tamil script (\tamiltext{தமிழ் அரிச்சுவடி} tamiḻ ariccuvaṭi) is an abugida script that is used by the Tamil people in India, Sri Lanka, Malaysia and elsewhere, to write the Tamil language, as well as to write the liturgical language Sanskrit, using consonants and diacritics not represented in the Tamil alphabet. Certain minority languages such as Saurashtra, Badaga, Irula, and Paniya are also written in the Tamil script. \index{Surashtra}\index{Badaga}\index{Irula}

The Tamil script has 12 vowels (\tamiltext{உயிரெழுத்து} uyireḻuttu ``soul-letters''), 18 consonants (\tamiltext{மெய்யெழுத்து} meyyeḻuttu ``body-letters").
An additional character, the āytam \tamiltext{ஃ (ஆய்தம்)},  is classified in Tamil grammar as being neither a consonant nor a vowel (\tamiltext{அலியெழுத்து} aliyeḻuttu ``the hermaphrodite letter''), though often considered as part of the vowel set (\tamiltext{உயிரெழுத்துக்கள்} uyireḻuttukkaḷ ``vowel class''). The script, however, is syllabic and not alphabetic. The complete script, therefore, consists of the thirty-one letters in their independent form, and an additional 216 combinant letters representing a total 247 combinations (\tamiltext{உயிர்மெய்யெழுத்து} uyirmeyyeḻuttu) of a consonant and a vowel, a mute consonant, or a vowel alone. These combinant letters are formed by adding a vowel marker to the consonant. Some vowels require the basic shape of the consonant to be altered in a way that is specific to that vowel. Others are written by adding a vowel-specific suffix to the consonant, yet others a prefix, and finally some vowels require adding both a prefix and a suffix to the consonant. In every case the vowel marker is different from the standalone character for the vowel.
The Tamil script is written from left to right.\index{hermaphrodite letter}


\section{Unicode}

Tamil is a Unicode block containing characters for the Tamil, Badaga, and Saurashtra languages of Tamil Nadu India, Sri Lanka, Singapore, and Malaysia. In its original incarnation, the code points U+0B02..U+0BCD were a direct copy of the Tamil characters A2-ED from the 1988 ISCII standard. The Devanagari, Bengali, Gurmukhi, Gujarati, Oriya, Telugu, Kannada, and Malayalam blocks were similarly all based on their ISCII encodings.

\begin{scriptexample}[]{Tamil}
\unicodetable{tamil}{"0B80,"0B90,"0BA0,"0BB0,"0BC0,"0BE0,"0BF0}

\hfill  Typeset with \cmd{\tamil} and \texttt{code2000.ttf}
\end{scriptexample}

\subsection{Tamil Numbers and Numerals}

Originally, Tamils did not use zero, nor did they use positional digits (having separate 
symbols for the numbers 10, 100 and 1000). Symbols for the numbers are similar to 
other Tamil letters, with some minor changes. 

For example, the number 3782 is not written as \tamiltext{௩௭௮௨} as in modern usage. Instead it 
is written as \tamiltext{௩ ௲ ௭ ௱ ௮ ௰ ௨}. This would be read as they are written as 
Three Thousands, Seven Hundreds, Eight Tens, Two; or in Tamil as 
\tamiltext{௩௲௭௱௮௰௨ž}.\footnote{https://cloud.github.com/downloads/raaman/Tamil-Numeral/tamilnumbers.html}

\subsection{Dates}

Once the script is loaded the day, month and year can be loaded using the command  \cmd{\tamildate}, which returns the |\today| formatted as per custom Tamil. 

\begin{center}
\bgroup
\tamil
\begin{tabular}{lll}
day	 &month	&year	\\

௳	&௴	      &௵	\\

u	&mee	      &wa	\\
\end{tabular}
\egroup
\end{center}


\section{Grantha}
\label{s:grantha}

Grantha is a Unicode block containing the ancient Grantha script characters of 6th to 19th century Tamil Nadu and Kerala for writing Sanskrit and Manipravalam. Battled to get it working, as I could not find an appropriate unicode font. The font would need remapping. Unfortunately this is a script with no Noto support.

\begin{figure}[htbp]
\bgroup
\parindent=0pt
\centering
\includegraphics[width=\linewidth]{./images/grantha.jpg}

\caption{An image of a palm leaf manuscript with Sanskrit written in Grantha script (wikipedia)}

\egroup
\end{figure}

\newfontfamily\grantha{e-Grantamil 7}%e-Grantamil 7

\begin{scriptexample}[\grantha]{Tamil}
\unicodetable{grantha}{"0D0,"0D1,"0D2,"1133,"1134,"1135,"1136,"1137}

\hfill  Typeset with \cmd{\grantha} and \texttt{e-Granthamil 7.ttf}
\end{scriptexample}

{
\grantha \char"11311

}

%\newfontfamily\freeserif{FreeSerif}
%
%
%\freeserif \lorem
%\begin{tabular}{lll}
%day	 &month	&year	\\
%
%௳	&௴	      &௵	\\
%
%u	&mee	      &wa	\\
%\end{tabular}






\subsection{Kannada alphabet}
\label{s:kannada}
\index{Scripts>Kannada}

\newfontfamily\kannada[Scale=1.0,Script=Kannada]{Lohit-Kannada.ttf}

\def\kannadatext#1{{\kannada #1}}

The Kannada people known as the Kannadigas and Kannadigaru, (sometimes referred to in English as Canarese),[14] are the people who natively speak Kannada.[15] Kannadigas are mainly found in the state of Karnataka in India. Kannada minorities are also found in the neighboring states Maharashtra,[3] Tamil Nadu,[16] Andhra Pradesh, Goa[17][18] and in most Indian states.[3] The English plural is Kannadigas. After a millennium of disintegration from Old Kannada into various languages,[19][12] sister languages[20] and Kannada dialects,[8] modern Kannada stands among 30 most widely spoken languages of the world as of 2001.[7][6] The Kannadiga diaspora can be found all over the world, mainly in the USA, the United Kingdom, Singapore, the UAE and the rest of the Middle East.[21][22][23][24][25][26]\indexindic{Kannada}

\begin{figure}[htbp]
\centering
\includegraphics[width=\linewidth-2\parindent]{kannada}

\caption{Kannada festival.}
\end{figure}



The Kannada alphabet (\kannadatext{ಕನ್ನಡ ಲಿಪಿ}) is an abugida of the Brahmic family,[2] used primarily to write the Kannada language, one of the Dravidian languages of southern India. Several minor languages, such as Tulu, Konkani, Kodava, and Beary, also use alphabets based on the Kannada script.[3] The Kannada and Telugu scripts share high mutual intellegibility with each other, and are often considered to be regional variants of single script. Similarly, Goykanadi, a variant of Old Kannada, has been historically used to write Konkani in the state of Goa.[4]\index{Indic Languages>Konkani}\indexindic{Tulu}\indexindic{Kodava}\indexindic{Beary}



\begin{scriptexample}[]{Kannada}
\centerline{\large\kannadatext{ಙ	ಙ್ಕ	ಙ್ಖ	ಙ್ಗ	ಙ್ಘ	ಙ್ಙ	ಙ್ಚ	ಙ್ಛ	ಙ್ಜ	ಙ್ಝ	ಙ್ಞ	ಙ್ಟ	ಙ್ಠ	ಙ್ಡ	ಙ್ಢ}}
\end{scriptexample}

\medskip

The Kannada script (aksharamale or varnamale) is a phonemic abugida of forty-nine letters, and is written from left to right. The character set is almost identical to that of other Brahmic scripts. Consonantal letters imply an inherent vowel. Letters representing consonants are combined to form digraphs (ottaksharas) when there is no intervening vowel. Otherwise, each letter corresponds to a syllable.

The letters are classified into three categories: swara (vowels), vyanjana (consonants), and yogavaahaka (part vowel, part consonant). \index{swara}\index{vyanjana}\index{yogavaahaka}

The Kannada words for a letter of the script are akshara, akkara, and varna. Each letter has its own form (ākāra) and sound (shabda), providing the visible and audible representations, respectively. Kannada is written from left to right.[7]


% image https://www.quora.com/Why-is-regional-chauvinism-very-high-in-Karnataka


\section{Osmanian Alphabet}

\bgroup
\newfontfamily\osmanian{code2001.ttf}
\osmanian
𐒚𐒁𐒖𐒄 𐒚𐒐 𐒚 𐒎𐒚𐒍𐒚𐒐 𐒑𐒚𐒒𐒠𐒚𐒐 𐒎𐒚𐒑𐒁𐒗 𐒚𐒁𐒖𐒄 𐒚𐒌𐒖𐒄 𐒚𐒁𐒖𐒄𐒖 𐒚
𐒌𐒜
\egroup





\cxset{steward,
  offsety=0cm,
  image={ethiopianbride.jpg},
  texti={An introduction to the use of font related commands. The chapter also gives a historical background to font selection using \tex and \latex. },
  textii={In this chapter we discuss keys that are available through the \texttt{phd} package and give a background as to how fonts are used
in \latex.
 },
 pagestyle = empty,
}




\cxset{steward,
  offsety=0cm,
  image={fellah-woman.jpg},
  texti={An introduction to the use of font related commands. The chapter also gives a historical background to font selection using \tex and \latex. },
  textii={In this chapter we discuss keys that are available through the \texttt{phd} package and give a background as to how fonts are used
in \latex.
 },
 pagestyle = empty
}

%
\cxset{chapter afterindent=off,
       subparagraph afterindent=on}
\bgroup
\arial


\chapter{South Asian Scripts}

The scripts of South Asia share so many characteristics that a side by side comparison of a few often reveal structural similarities even in the 
modern letterforms.
\medskip


\begin{center}
\begin{tabular}{lll}
  \hyperref[s:devanagari]{Devanagari} 
& \hyperref[s:gujarati]{Gujarati}
& \hyperref[s:telugu]{Telugu}\\
  \hyperref[s:bengali]{Bengali}
& \hyperref[s:oriya]{Oriya} 
& \hyperref[s:kannada]{Kannada}\\
  \hyperref[s:gurmukhi]{Gurmukhi} 
& \hyperref[s:tamil]{Tamil}
& \hyperref[s:malayalam]{Malayalam}\\
  \hyperref[s:sinhala]{Sinhala} 
& \hyperref[s:kaithi]{Kaithi}  
& \hyperref[s:meeteimayek]{Meetei Mayek}\\
  \hyperref[s:tibetan]{Tibetan} 
& \hyperref[s:saurashtra]{Saurashtra} 
& \hyperref[olchiki]{Ol Chiki}\\
  \hyperref[s:lepcha]{Lepcha}
& \hyperref[s:sharada]{Sharada} 
& \hyperref[s:sorasompeng]{Sora Sompeng}\\
  \hyperref[s:phagspa]{Phags-pa} 
& \hyperref[s:takri]{Takri}  
& \hyperref[s:kharoshthi]{Kharoshthi}\\
  \hyperref[s:limbu]{Limbu} 
& \hyperref[s:chakma]{Chakma}
& \hyperref[s:brahmi]{Brahmi}\\
  \hyperref[s:sylotinagra]{Syloti Nagri} 
& \hyperref[s:mro]{Mro} 
&\\
\end{tabular}
\end{center}

The sections that follow describe the scripts briefly and the |phd| settings
to activate the relevant commands and load appropriate fonts. 

\begin{figure}[htbp]
\includegraphics[width=\textwidth]{./images/indic-language-tree.jpg}
\caption{A family tree of a few of the most important Indic scripts, (adapted from \protect\cite{writing}.}
\end{figure}


\section{Sinhala Alphabet}
\label{s:sinhala}
\index{scripts>Sinhala}

The Sinhala alphabet derives from Pali.

\newfontfamily{\sinhala}{NotoSansSinhala-Regular.ttf}

\begin{docKey}[phd]{sinhala font}{ = \meta{font face}}{default none, initial NotoSansSinhala-Regular.ttf}
\end{docKey}

\begin{docCommand}{textsinhala}{\marg{text}}
Command to typeset sinhalese text.
\end{docCommand}

\begin{docEnvironment}{sinhalascript}{}{}
\end{docEnvironment}

The \refEnv{sinhalascript} environment typesets text inputted in the Sinhala script.

\begin{scriptexample}[]{Sinhala}
\unicodetable{sinhala}{"0D80,"0D90,"0DA0,"0DB0,"0DC0,"0DD0,"0DE0,"0DF0}
\end{scriptexample}

\printunicodeblock{./languages/sinhala.txt}{\sinhala}
\section{Meitei Mayek alphabet}
\label{s:meiteimayek}
\index{scripts>Meitei Mayek}
\index{Meetei Mayek}
\newfontfamily\meitei{Noto Sans Meetei Mayek}

\def\textmeitei#1{{\meitei #1}\xspace}

Meithei (Meitei) /ˈmeɪteɪ/,[4] also known as Manipuri /mænɨˈpʊəri/ ({\pan মৈতৈলোন্} \textmeitei{ꯃꯧꯇꯧꯂꯣꯟ} Meitei-lon or {\pan মৈতৈলোল্} \textmeitei{ꯃꯧꯇꯧꯂꯣꯜ} Meitei-lol), is the predominant language and lingua franca in the southeastern Himalayan state of Manipur, in northeastern India. It is the official language in government offices. Meithei is also spoken in the Indian states of Assam and Tripura, and in Bangladesh and Burma (now Myanmar).

The Meitei (also Meetei, Meithei, Manipuri) people are the majority ethnic group of Manipur, a northeastern state of India. Meitei is an endonym or autonym while Manipuri is an exonym. A significant population of the Meitei also are settled in domestic neighboring states such as Assam[1] and Tripura. They have also settled in Bangladesh[2] and Myanmar.[3]
The Meitei people are made up of seven major clans known as Salai Taret.[4] Their written history has been documented to 1445 BC.[5]

Meithei is a Tibeto-Burman language whose exact classification remains unclear, though it shows lexical resemblances to Kuki and Tangkhul Naga.[5] The language is spoken by more than 1.5 million people.

\begin{figure}[htbp]
\centering
\includegraphics[width=\linewidth]{dancing}

\caption{"Khamba-Thoibi" Jagoi 
RKCS paintings on the walls of temple of Ibudhou Thangjing at Moirang, Manipur. 
Picture Courtesy - Recky Maibram.}
\end{figure}

Meithei has proven to be an integrating factor among all ethnic groups in Manipur who use it to communicate among themselves. It has been recognized (as Manipuri), by the Indian Union and has been included in the list of scheduled languages (included in the 8th schedule by the 71st amendment of the constitution in 1992). Meithei is taught as a subject up to the post-graduate level (Ph.D.) in universities of India, apart from being a medium of instruction up to the undergraduate level in Manipur.

\bgroup
\meitei
\begin{tabular}{>{\arial}l
                >{\arial}l
                >{\meitei}l
                >{\arial}l
                >{\arial}l
                >{\meitei}l
               }
1	&ama 	 &ꯑꯃ	       &11	&taramathoi	&\\
2	&ani	   &ꯑꯅꯤ	&12	 &taranithoi	&ky \\
3	&ahum	&ꯑꯍꯨꯝ	   &13	 &tarahumdoi	&ꯇꯔꯥꯍꯨꯝꯗꯣꯢ\\
4	&mari	&ꯃꯔꯤ	   &14  &	taramari	&ꯇꯔꯥꯃꯔꯤ\\
5	&manga	 &ꯃꯉꯥ	   &15	 &taramanga	&ꯇꯔꯥꯃꯉꯥ\\
6	&taruk	 &ꯇꯔꯨꯛ	   &16	 &tarataruk	&ꯇꯔꯥꯇꯔꯨꯛ\\
7	&taret	 &ꯇꯔꯦꯠ	   &17	 &tarataret	&ꯇꯔꯥꯇꯔꯦꯠ\\
8	&nipan &ꯅꯤꯄꯥꯟ	&18	 &taranipan	&ꯇꯔꯥꯅꯤꯄꯥꯟ\\
9	&mapan	 &ꯃꯥꯄꯟ	   &19	 &taramapan	&ꯇꯔꯥꯃꯥꯄꯟ\\
10	&tara	 &ꯇꯔꯥ	   &20	 &kun	&ꯀꯨꯟ\\
\end{tabular}
\egroup





Meitei Mayek script was added to the Unicode Standard in October, 2009 with the release of version 5.2.\index{Meitei Mayek}

The Unicode block for Meitei Mayek, called Meetei Mayek, is \unicodenumber{U+ABC0–U+ABFF}.

Characters for historical orthographies are part of the Meetei Mayek Extensions block at \unicodenumber{U+AAE0–U+AAFF}.

\begin{scriptexample}[]{Meitei}
\unicodetable{meitei}{"ABC0,"ABCD0,"ABE0,"ABF0}
\end{scriptexample}

\begin{scriptexample}[]{Meitei}
\unicodetable{meitei}{"AAE0,"AAF0}
\captionof{table}{Meetei Mayek Extensions}
\end{scriptexample}


\printunicodeblock{./languages/meetei-mayek.txt}{\meitei}


% http://e-pao.net/eyek/tamba/




%\section{Tibetan}
\label{tibetan}
\index{scripts>tibetan}


Another important Northern Indian member perhaps
derived directly from Gupta---and thus a sister script to Nagari,
Sarada and Pali---is Tibetan. However, the Tibetan
language wears this foreign Indo-Aryan script most uncomfortably.\cite{writing}
The script retains the Indic consonantal alphabet with diacritic
attachments to indicate vowels – but with only one vowel
letter, the /a/, which is the same as the system’s own ‘default’ /a/.
This /a/ letter is then used to attach other diacritics in order to
indicate further vowels. Because the Tibetan language has
changed greatly since c. AD 700 (when the script was first elaborated
from Gupta) while the script has remained almost
unchanged, Tibetan is extremely difficult to read today. Its
greatest problem is that it marks none of the tones of its tonal
language. Though Tibetans have long tried to adapt written
Tibetan to spoken Tibetan, high illiteracy has been the price of
failing to achieve this. Tibetan schools in Tibet, by governmental
decree, now teach only the Chinese script and in the Chinese
language.

\newfontfamily\tibetan{TibMachUni.ttf}

\newfontfamily\tibetan{Qomolangma-Chuyig.ttf}

%A should pick it up automatically \tibetan

Fonts described in this section can be obtained from The Tibetan \& Himalayan Library
\footnote{\url{http://www.thlib.org/tools/scripts/wiki/tibetan\%20machine\%20uni.html}  }

I have tried a few \texttt{Tibetan Machine Uni (TMU)} seems to be used by a number of scholars. 

A tip when you are trying to locate fonts is to find a related article in Wikipedia, such as Tibetan alphabet and inspect the element using your browser to see what fonts are being used.


|style="font-family:'Jomolhari','Tibetan Machine Uni','DDC Uchen', 'Kailash';| 


If you cannot see the script and rather than boxes or question marks then you can search and download one of the fonts in |font-family|.



\begin{docKey}[phd]{language}{ = tibetan}{default none, initial english} 
The key |language=tibetan| sets the default language as Tibetan, using the main font given by the key |tibetan font=TibMachUni.ttf|.

It will also create an environment tibetanlanguage.
\end{docKey}

\begin{docKey}[phd]{tibetan font}{= TibMachUni.ttf} {initial = TibMachUni.ttf} 
The key |tibetan font=font-name| sets the default font for the Tibetan language. It will also create the switch \cmd{\tibetan} for typesetting text in Tibetan.
\end{docKey}


\begin{docEnvironment}{tibetan}{}
\end{docEnvironment}

The environment is created automatically
\begin{texexample}{Tibetan language setttings}{ex:tibetan}
\bgroup
\cxset{language=tibetan, tibetan font = TibMachUni.ttf}

\tibetan Tibetan: དབུ་ཅན\par
ཨོཾ་ཨཿཧཱུྂ་བཛྲ་གུ་རུ་པདྨ་སིདྡྷི་ཧཱུྂ༔\par
\egroup

\begin{tibetanlanguage}
The tibetan environment\par
ཨོཾ་ཨཿཧཱུྂ་བཛྲ་གུ་རུ་པདྨ་སིདྡྷི་ཧཱུྂ༔
\end{tibetanlanguage}
\end{texexample}


The Tibetan alphabet is an \emph{abugida} of Indic origin used to write the Tibetan language as well as Dzongkha\footnote{Spoken in Bhutan.}, the Sikkimese language, Ladakhi, and sometimes Balti. 

The printed form of the alphabet is called \textit{uchen} script (Tibetan: དབུ་ཅན་, Wylie: dbu-can; "with a head") while the hand-written cursive form used in everyday writing is called umê script (Tibetan: དབུ་མེད་, Wylie: dbu-med; "headless").

The alphabet is very closely linked to a broad ethnic Tibetan identity. Besides Tibet, it has also been used for Tibetan languages in Bhutan, India, Nepal, and Pakistan.[1] The Tibetan alphabet is ancestral to the Limbu alphabet, the Lepcha alphabet,[2] and the multilingual 'Phags-pa script.[2]


The Tibetan alphabet is romanized in a variety of ways.[3] This article employs the Wylie transliteration system.

The Tibetan alphabet has thirty basic letters, sometimes known as "radicals", for consonants.[2]

{\tibetanfontfamily
ཀ ka /ká/	ཁ kha /kʰá/	ག ga /kà, kʰà/	ང nga /ŋà/\\
ཅ ca /tʃá/	ཆ cha /tʃʰá/	ཇ ja /tʃà/	ཉ nya /ɲà/\\
ཏ ta /tá/	ཐ tha /tʰá/	ད da /tà, tʰà/	ན na /nà/\\
པ pa /pá/	ཕ pha /pʰá/	བ ba /pà, pʰà/	མ ma /mà/\\
ཙ tsa /tsá/	ཚ tsha /tsʰá/	ཛ dza /tsà/	ཝ wa /wà/ (not originally part of the alphabet)[5]\\
ཞ zha /ʃà/[6]	ཟ za /sà/	འ 'a /hà/[7]\\
ཡ ya /jà/	ར ra /rà/	ལ la /là/\\
ཤ sha /ʃá/[6]	ས sa /sá/	ཧ ha /há/[8]\\
ཨ a /á/\\
}


Tibetan is not a difficult script to read or write, but it is a very complex script to deal with in terms of computer processing (as far as complexity goes I would rate it second only to the Mongolian script). The problem is that written Tibetan comprises complex syllable units (known in Tibetan as a tsheg bar {\tibetan ཚེག་བར}) which although written horizontally may include \emph{vertical} clusters of consonants and vowel signs agglutinating around a base consonant (a vertical cluster is known as a "stack"). 

Thus most words have a horizontal and a vertical dimension, with the result that text is not laid out in a straight line as in most scripts. For example, the word bsGrogs བསྒྲོགས་ (pronounced drok ... obviously!) may be analysed as follows :

\definecolor{lavenderblush}{HTML}{FFF0F5}%
\definecolor{beige}{HTML}{F5F5DC}%


{\tibetan 
\HUGE བསྒྲོགས

{\color{beige}%
\symbol{"0F56}\color{blue!40}\color{red}\symbol{"0F66}\symbol{"0F92}\color{blue!80}\symbol{"0FB2}\color{beige}\symbol{"0F7C}\color{blue!25}\symbol{"0F42}\symbol{"0F66}\symbol{"0F0B}}



\begin{tabular}{|l|}
\symbol{"0F56}\symbol{"0F7C}\\
\symbol{"0F42}\symbol{"0F7C}\\
\symbol{"0F66}\symbol{"0F7C}\\
\symbol{"0F40}\symbol{"0F7C}\\
\end{tabular}
}

\subsection{Unicode Block Tibetan}


\bgroup\large\tibetan
\begin{tabular}{llllllllllllllll l}
\toprule
	           &|0|	&|1|	&|2|	&|3|	&|4|	&|5|	&|6|	&|7|	&|8|	&|9|	&|A|	&|B|	&|C|	&|D|	&|E|	&|F|\\
\midrule
\texttt{U+0F0x}	&ༀ	&༁	&༂	&༃	&༄	&༅	&༆	&༇	&༈	&༉	&༊	&་	&༌  &	།	&༎	&༏\\
\midrule
\texttt{U+0F1x} &༐	&༑	&༒	&༓	&༔	&༕	&༖	&༗	&༘&	༙	&༚	&༛	&༜	&༝	&༞	&༟\\
\midrule
\texttt{U+0F2x} &༠	&༡	&༢	&༣	&༤	&༥	&༦	&༧	&༨	&༩	&༪	&༫	&༬	&༭	&༮	&༯\\
\midrule
\texttt{U+0F3x}	&༰ &༱	 &༲ &༳	&༴ &༵	&༶ & ༷	&༸&	༹	&༺&	༻	&༼&	༽	&༾	&༿\\
\midrule
\texttt{U+0F4x} &ཀ	&ཁ	&ག	&གྷ	&ང	&ཅ	&ཆ	&ཇ	&	&ཉ	&ཊ	&ཋ	&ཌ	&ཌྷ	&ཎ	&ཏ\\
\midrule
\texttt{U+0F5x}	 &ཐ	&ད	&དྷ	&ན	&པ	&ཕ	&བ	&བྷ	&མ	&ཙ	&ཚ	&ཛ	&ཛྷ	&ཝ	&ཞ	&ཟ\\
\midrule
\texttt{U+0F6x} &འ	&ཡ	&ར	&ལ	&ཤ	&ཥ	&ས	&ཧ	&ཨ	&ཀྵ	&ཪ	&ཫ	&ཬ	&&&\\
^^A\texttt{U+0F7x}&&	ཱ &	& &ི	ཱི&	ུ&	ཱུ&	ྲྀ&	ཷ&	ླྀ&	ཹ&	ེ&	ཻ&	ོ&	ཽ&	&ཾ	&ཿ\\
\midrule
\texttt{U+0F8x}&    ྀ   & 	ཱྀ&	ྂ&	&ྃ &	྄	&྅&	྆	&྇	ྈ&	ྉ&	ྊ&	ྋ&	ྌ&	ྍ&	ྎ&	ྏ\\
\midrule
\texttt{U+0F9x} &	ྐ&	ྑ   & 	ྒ &	ྒྷ &	ྔ &	ྕ &	ྖ &	ྗ &		ྙ &	ྚ &	ྛ &	ྜ &	ྜྷ &	ྞ &	ྟ\\
\texttt{U+0FAx} &	ྠ &	ྡ &	ྡྷ &	ྣ &	ྤ &	ྥ &		&ྦ	&ྦྷ	ྨ&	ྩ&	ྪ&	ྫ&	ྫྷ&	ྭ&	ྮ&	ྯ\\
\midrule
\texttt{U+0FBx} 
&	  ྰ 
&	
& ྱ  	 
&ྲ	
&ླ	
&ྴ
&	ྵ
&	ྶ
&	ྷ
&ྸ
&
&
&
&	
&྾	
&྿\\
\midrule
\texttt{U+0FCx}	 &࿀&	࿁&	࿂&	࿃&	࿄&	࿅&	&࿇	&࿈	&࿉	&࿊	&࿋	&࿌	&&	࿎	&࿏\\
\midrule
\texttt{U+0FDx}	&࿐	&࿑	&࿒	&࿓	&࿔	&࿕	&࿖	&࿗	&࿘	&࿙	&࿚	&&&&&\\
\midrule
\texttt{U+0FEx} &&&&&&&&&&&&&&&&\\
\midrule
\texttt{U+0FFx}  &&&&&&&&&&&&&&&&\\
\bottomrule
\end{tabular}
\egroup




\subsection{Fonts for Tibetan}

Fonts for Tibetan need to be downloaded one set of fonts are the \texttt{Qomolangma}. They come in different flavours, but they appear
to offer advantages as compared to the Tibetan Machine Uni.
\medskip


\newfontfamily\betsu{Qomolangma-Betsu.ttf}
\newfontfamily\drutsa{Qomolangma-Drutsa.ttf}
\newfontfamily\chuyig{Qomolangma-Chuyig.ttf}
\newfontfamily\tsumachu{Qomolangma-Tsumachu.ttf}
\newfontfamily\uchensutung{Qomolangma-UchenSutung.ttf}
\newfontfamily\uchensuring{Qomolangma-UchenSuring.ttf}
\newfontfamily\uchensarchen{Qomolangma-UchenSarchen.ttf}
\newfontfamily\uchensarchung{Qomolangma-UchenSarchung.ttf}
\newfontfamily\tsuring{Qomolangma-Tsuring.ttf}
\newfontfamily\TMU{TibMachUni.ttf}
\newfontfamily\himalaya{Microsoft Himalaya}


{
\centering

\renewcommand{\arraystretch}{1.5}

\begin{tabular}{lr}
\toprule
|Qomolangma-Betsu.ttf| & {\betsu  དབུ་མེད }\\
\midrule
|Qomolangma-Chuyig.ttf| &{\chuyig  དབུ་མེད}\\
\midrule
|Qomolangma-Drutsa.ttf| &{\drutsa  དབུ་མེད}\\
\midrule
|Qomolangma-Tsumachu.ttf|&{\tsumachu  དབུ་མེད}\\
\midrule
|Qomolangma-Tsuring.ttf| &{\tsuring  དབུ་མེད}\\
\midrule
|Qomolangma-UchenSarchen.ttf| &{\uchensarchen དབུ་མེད}\\
\midrule
|Qomolangma-UchenSarchung.ttf|&{\uchensarchung དབུ་མེད }\\
\midrule
|Qomolangma-UchenSuring.ttf|&{\uchensuring དབུ་མེད}\\
\midrule
|Qomolangma-UchenSutung.ttf|&{\uchensutung དབུ་མེད }\\
\midrule
|TibMachUni.ttf| &{\TMU དབུ་མེད }\\
\midrule
|Microsoft Himalaya| &{\himalaya དབུ་མེད ཽ}\\
\bottomrule
\end{tabular}

}
\bigskip

\bgroup
\LARGE\tsuring
\noindent༆ །ཨ་ཡིག་དཀར་མཛེས་ལས་འཁྲུངས་ཤེས་བློ  འི་\par
གཏེར༑ །ཕས་རྒོལ་ཝ་སྐྱེས་ཟིལ་གནོན་གདོང་ལྔ་བཞིན།།\par
ཆགས་ཐོགས་ཀུན་བྲལ་མཚུངས་མེད་འཇམ་དབྱངསམཐུས།།\par
མཧཱ་མཁས་པའི་གཙོ་བོ་ཉིད་འགྱུར་ཅིག། །མངྒལཾ༎\par
བསྒྲོགས
\egroup

\subsubsection{Tibetan numbers}
\cxset{language=tibetan, tibetan font = TibMachUni.ttf}

{
\obeylines
\small
TIBETAN DIGIT ZERO\tibetan	༠
TIBETAN DIGIT ONE	\tibetan༡	
TIBETAN DIGIT TWO\tibetan	༢	
TIBETAN DIGIT THREE\tibetan	༣	
TIBETAN DIGIT FOUR	\tibetan ༤	
TIBETAN DIGIT FIVE\tibetan	༥	
TIBETAN DIGIT SIX	\tibetan ༦	
TIBETAN DIGIT SEVEN\tibetan	༧	
TIBETAN DIGIT EIGHT\tibetan	༨	
TIBETAN DIGIT NINE\tibetan	༩	
TIBETAN DIGIT HALF ONE	\tibetan༪	
TIBETAN DIGIT HALF TWO	༫	
TIBETAN DIGIT HALF THREE	༬
TIBETAN DIGIT HALF FOUR ༭	
TIBETAN DIGIT HALF FIVE ༯	
TIBETAN DIGIT HALF SIX	 ༯	
TIBETAN DIGIT HALF SEVEN	༰	
TIBETAN DIGIT HALF EIGHT	༱	
TIBETAN DIGIT HALF NINE	༲	
TIBETAN DIGIT HALF ZERO	༳	
}


Tibetan numbers

The usage is not certain. By some interpretations, this has the value of 9.5. Used only in some traditional contexts, these appear as the last digit of a multidigit number, eg. ༤༬ represents 42.5. These are very rarely used, however, and other uses have been postulated.


\PrintUnicodeBlock{./languages/tibetan.txt}{\himalaya}


\section{Oriya}
\label{s:oriya}
\index{Indic scripts>Oriya}
\epigraph{Oṛiyā is encumbered with the drawback of an excessively awkward and cumbrous written character. ... At first glance, an Oṛiyā book seems to be all curves, and it takes a second look to notice that there is something inside each.}{(G. A. Grierson, \textit{Linguistic Survey of India}, 1903)}

\newfontfamily\oriya[Scale=1.1,Script=Oriya]{Noto Sans Oriya}

\def\oriyatext#1{{\oriya#1}}
The Oriya script or Utkala Lipi (Oriya: \oriyatext{ଉତ୍କଳ ଲିପି}) or Utkalakshara (Oriya: \oriyatext{ଉତ୍କଳାକ୍ଷର}) is used to write the Oriya language, and can be used for several other Indian languages, for example, Sanskrit.

\centerline{\Huge\oriyatext{ଉତ୍କଳ ଲିପି}}

\bgroup
\oriya
୦୧୨୩୪୫୬୭୮୯
ଅ ଆ ଇ ଈ ଉ ଊ ଋ ୠ ଌ ୡ ଏ ଐ ଓ ଔ କ ଖ ଗ ଘ ଙ ଚ ଛ ଜ ଝ ଞ ଟ ଠ ଡ ଢ ଣ ତ ଥ ଦ ଧ ନ ପ ଫ ବ ଵ ଭ ମ ଯ ର ଳ ୱ ଶ ଷ ସ ହ ୟ ଲ
\egroup






\begin{figure}[htbp]
\centering

\includegraphics[width=\linewidth-2\parindent]{oriya-people}

\hspace*{-1em}\caption{Children dressed for celebration of Janmashtami, which marks the birth of Lord Krishna. odisha360.com}
\end{figure}

Comparison of Oṛiyā script with its neighbours

At a first look the great number of signs with round shapes suggests a closer relation to the southern neighbour Telugu than to the other neighbours Bengali in the north and Devanāgarī in the west. The reason for the round shapes in Oriya and Telugu (and also in Kannaḍa and Malayāḷam) is the former method of writing using a stylus to scratch the signs into a palm leaf. These tools do not allow for horizontal strokes because that would damage the leaf.

Oriya letters are mostly round shaped whereas in Devanāgarī and Bengali have horizontal lines. So in most cases the reader of Oṛiyā will find the distinctive parts of a letter only below the hoop. Considering this the  closer relation to Devanāgarī and Bengali exists than to any southern script, though both northern and southern scripts have the same origin, Brāhmī.

Oriya (\oriyatext{ଓଡ଼ିଆ} oṛiā), officially spelled Odia,[3][4] is an Indian language belonging to the Indo-Aryan branch of the Indo-European language family. It is the predominant language of the Indian states of Odisha, where native speakers comprise 80\% of the population,[5] and it is spoken in parts of West Bengal, Jharkhand, Chhattisgarh and Andhra Pradesh. Oriya is one of the many official languages in India; it is the official language of Odisha and the second official language of Jharkhand. [6][7][8] Oriya is the sixth Indian language to be designated a Classical Language in India, on the basis of having a long literary history and not having borrowed extensively from other languages.



\printunicodeblock{./languages/oriya.txt}{\oriya}

\section{Numerals}

{\oriya
\obeylines
୦	୧	୨	୩	୪	୫	୬	୭	୮	୯	୵	୶	୷	୲	୳	୴
{\arial 0	1	2	3	4	5	6	7	8	9	¹⁄₁₆	⅛	³⁄₁₆	¼	½	¾}
}




\section{Mro (Mru language)}
\label{s:mro}

\newfontfamily\mro{MroUnicode-Regular.ttf}
\def\textmro#1{{\mro #1\xspace}}

 Mro (or Mru) is a Tibeto-Burman language spoken primarily in Bangladesh with a few
speakers in India. 

Mru is a Tibeto-Burman language and one of the recognized languages of Bangladesh. It is spoken by a community of Mros (Mru) inhabiting the Chittagong Hill Tracts of Bangladesh and also in Burma with a population of 22,000 in Bangladesh according to the 1991 census. The Mros are the second-largest tribal group in Bandarban District of the Chittagong Hill Tracts. A small group of Mros also live in Rangamati Hill District.

The Mru language is considered "definitely endangered" by UNESCO in June 2010.[4]

The script was invented in the 1980s and is of the class of “messianic”
scripts with no genetic relationship with existing scripts. In the last 10 years there has been an acceptance
among all the Mro to use this script and literacy levels among the 100,000 Mro exceed 80\%.

Some of the characters of the Mro alphabet have a visual similarity to those from other alphabets, but this
relationship is purely coincidental, and the Mro alphabet stands alone as a unity.


The Mro script has no technical complexity: it is a simple left to right alphabet with no
combining characters or characters with special function. There are no tone marks. Some sounds are
represented by more than one letter. The sound [k] is usually represented by \textmro{𖩌} KEAAE kəɘ, as in \textmro{𖩌𖩑𖩗} kow
‘village’, \textmro{𖩄𖩑𖩁𖩌𖩑} boŋko ‘owl’, but in a few words the letter 𖩙 KOO ko is used, as in \textmro{𖩙𖩑} ko ‘gold’. The sound
[m] is usually represented by \textmro{𖩎} MAEM mɘm, as in \textmro{𖩎𖩆𖩁} maŋ ‘go’, \textmro{𖩔𖩎𖩑} śmo ‘fool’, but in a few words the
letter \textmro{𖩃} MIM mim is used, as in \textmro{𖩃𖩊𖩏} min ‘cat’, \textmro{𖩋𖩃𖩊} cmi ‘rice’. The sound [l] is usually represented by \textmro{𖩍} OL
\textmro{ɔl}, as in \textmro{𖩍𖩝𖩁} lɔŋ ‘boat’, \textmro{𖩈𖩍𖩆} khla ‘spoon’, but in a few words the letter \textmro{𖩛} LA la is used, as in \textmro{𖩛𖩆𖩎𖩖} lamɘ
‘moon’, and in a few words \textmro{𖩚} LAN lan is used (we have no example). The vowels \textmro{𖩑𖩖} oɘ are used as a
digraph to describe the vowel [ø].

We are using Philip Reimer's font which is freely available under SIL OFL licence at \href{http://phjamr.github.io/mro.html}{github}. Philip has also produced fonts for two other scripts: Lisu (Fraser) and Miao (Pollard). All three scripts were added to Unicode 7.0 in 2014.



\begin{scriptexample}[]{Mro}
\unicodetable{mro}{"16A40,"16A50,"16A60}
\end{scriptexample}


\printunicodeblock{./languages/mro.txt}{\mro}








\section{Devanagari}
\label{s:devanagari}
\parindent1em

Devanagari is part of the Brahmic family of scripts of India, Nepal, Tibet, and South-East Asia.[2] It is a descendant of the Gupta script, along with Siddham and Sharada.[2] Eastern variants of Gupta called nāgarī are first attested from the 7th century CE; from c. 1200 CE these gradually replaced Siddham, which survived as a vehicle for Tantric Buddhism in East Asia, and Sharada, which remained in parallel use in Kashmir. An early version of Devanagari is visible in the Kutila inscription of Bareilly dated to Vikram Samvat 1049 (i.e. 992 CE), which demonstrates the emergence of the horizontal bar to group letters belonging to a word.[3]

Sanskrit nāgarī is the feminine of nāgara \enquote{relating or belonging to a town or city}. It is feminine from its original phrasing with lipi ("script") as nāgarī lipi "script relating to a city", that is, probably from its having originated in some city.[4]

The use of the name devanāgarī is relatively recent, and the older term nāgarī is still common.[2] The rapid spread of the term Devanāgarī may be related to the almost exclusive use of this script to publish Sanskrit texts in print since the 1870s.[2]

In time, Devanagari became India’s principal script. It also
became one of the world’s most important, as it was used to
convey many other languages of the region, such as Hindi, Nepali, Marwari, 
Kumaoni and several non-Indo-Aryan
languages. Devanagari failed to become India’s sole script perhaps
because of the region’s long disunity. Subsequently, it
became the parent of, among other scripts, the Gurmukhi
which the Sikhs elaborated in the 1500s in order to write their
Punjabi language (illus. 78). Today, Devanagari survives in India
alongside some ten other major scripts (including the Latin and
Perso-Arabic alphabets) and about 190 others of lesser significance.\cite{writing}

On Windows use \texttt{Arial Unicode MS} or \texttt{Arial}
\medskip

%\newfontfamily\devanagari[Script=Devanagari,Scale=1.5]{Arial Unicode MS}
\newfontfamily\devanagarilohit[Script=Devanagari,Scale=1.1]{Lohit-Devanagari.ttf}
\let\devanagari\devanagarilohit

\begin{scriptexample}[]{Devanagari}
{\begin{center}\parindent0pt\devanagari

ंःअआइईउऊऋऌऍऎएऐऑऒओऔऔँ \par 

ी	ु	ू	ृ	ॄ	ॅ	ॆ	े	ै	ॉ	ॊ	ो	ौ	्	\par

\bigskip		
\begin{tabular}{lll lll lll l}
०	&१	&२	&३	&४	&५	&६	&७	&८	&९\\
0	&1	&2	&3	&4	&5	&6	&7	&8	&9\\
\end{tabular}
\end{center}	
}
\end{scriptexample}


On Linux \texttt{Lohit} is a font family designed to cover Indic scripts and released by Red Hat. The Lohit fonts currently cover 11 languages: Assamese, Bengali, Gujarati, Hindi, Kannada, Malayalam, Marathi, Oriya, Punjabi, Tamil, Telugu.[1] The fonts were supplied by Modular Infotech and licensed under the GPL. In September 2011, they were retroactively relicensed under the OFL.[2] The Lohit fonts are used as web fonts by some Wikimedia Foundation sites, like Wikipedia, since March 2012.The font currently support 21 Indian languages. 

\let\devanagarilohit\pan

\begin{scriptexample}[]{Devanagari}
\begin{center}\parindent0pt\devanagarilohit

ंःअआइईउऊऋऌऍऎएऐऑऒओऔऔँ \par 

ी	ु	ू	ृ	ॄ	ॅ	ॆ	े	ै	ॉ	ॊ	ो	ौ	्	\par

\bigskip		
\begin{tabular}{lll lll lll l}
०	&१	&२	&३	&४	&५	&६	&७	&८	&९\\
0	&1	&2	&3	&4	&5	&6	&7	&8	&9\\
\end{tabular}
\end{center}
\end{scriptexample}

\subsubsection{Punctuation} 
The end of a sentence or half-verse may be marked with a dot known as a pūrna virām or a vertical line danda: \textbar. The end of a full verse may be marked with two vertical lines: \textbar\textbar. A comma, or alpa virām, is used to denote a natural pause in speech. With expansion of English speakers in India, the full stop is also sometimes used.

\subsection{LaTeX support}

\latex2e support can be found in the \pkgname{sanskrit}. The package contains the font files and pre-processor for printing Sanskrit
text in both devanāgarī and transliterated Roman with diacritics. Another package that can be used with \XeTeX\ is support \pkgname{devnag}.  This was originally developed by Frans Velthuis for the University of Groningen, The Netherlands, and it was the first system to provide
support for the script for \tex. The package was  extended by Anshuman Pandey. The package provides both fonts as well as tranliteration macros.


\printunicodeblock{./languages/devanagari.txt}{\devanagarilohit}






\section{Bengali}
\label{s:bengali}
\idxlanguage{Bengali}
\index{Bengali fonts>Shonar Bangla}
\index{Bengali fonts>Vrinda}
\index{Bengali fonts>Noto Sans Bengali}
\index{Bengali fonts>Noto Serif Bengali}
\index{Bengali}
%\newfontfamily\bengali[Script=Bengali,Scale=1.3]{Shonar Bangla}
\newfontfamily\bengali[Script=Bengali,Scale=1.0]{Noto Serif Bengali}
There are two Windows fonts that can be used with Windows \textit{Shonar Bangla} and \textit{Vrinda}. For open source fonts one can use, \texttt{Not Serif Bengali}.

\docAuxCommand{bengali} and \docAuxCommand{textbengali} Once the key is set the command \cmd{\bengali} is available for use in typesetting Bengali text.

\bigskip

\bgroup



\bengali
\centering

  অ  আ ই  ঈ  উ  ঊ  ঋ  এ  ঐ\par

%\newfontfamily\bengal[Script=Bengali,Scale=3.2]{Vrinda}

\centering

  অ  আ ই  ঈ  উ  ঊ  ঋ  এ  ঐ\par




\centering

  অ  আ ই  ঈ  উ  ঊ  ঋ  এ  ঐ\par

\captionof{table}{The consonant{\protect\bengali{} ক (kô)} along with the diacritic form of the vowels {\protect\bengali{} অ, আ, ই, ঈ, উ, ঊ, ঋ, এ, ঐ, ও and ঔ} \textit{from Wikipedia}.}
\egroup

\def\indexindic#1{\index{Indic Languages>#1}\index{#1} }

|Bengali| is a Unicode block containing characters for the Bangla, Assamese, Bishnupriya Manipuri, Daphla, Garo, Hallam, Khasi, Mizo, Munda, Naga, Rian, and Santali languages. In its original incarnation, the code points U+0981..U+09CD were a direct copy of the Bengali characters A1-ED from the 1988 ISCII standard, as well as several Assamese ISCII characters in the U+09F0 column. The Devanagari, Gurmukhi, Gujarati, Oriya, Tamil, Telugu, Kannada, and Malayalam blocks were similarly all based on ISCII encodings. \index{Bengali}\index{Indic Languages>Bangla}\index{Indic Languages>Assamese}\index{Indic Languages>Bishnupriya Manipuri}\indexindic{Daphla}\indexindic{Garo}\indexindic{Hallam}\indexindic{Khasi}\indexindic{Mizo}\indexindic{Munda}
\indexindic{Naga}\indexindic{Rian}\indexindic{Santali}

\begin{scriptexample}[]{Bengal}
\unicodetable{bengali}{"0980,"0990,"09A0,"09B0,"09C0,"09D0,"09E0,"09F0}
\end{scriptexample}


\printunicodeblock{./languages/bengali.txt}{\bengali}



\bgroup
\bengali\LARGE
\char"0995 + \color{blue} \char"09BC + \color{red}\char"09AF  = \char"0995\char"09CD \char"09AF
\egroup

Noto has both a serif and a sans font \docFont{Noto Serif Bengali}

See also \url{http://www.nongnu.org/freebangfont/downloads.html} for additional fonts.










\section{Saurashtra}
\label{s:saurashtra}
\idxlanguage{Saurashtra}\idxlanguage{Sourashtra}

\index{Saurashtra fonts>code2000}
\newfontfamily\saurashtra{code2000.ttf}
\def\test{}
\cxset{saurashtra font/.code=\test}
\cxset{saurashtra font=code2000.ttf}

\begin{docKey}[phd]{saurashtra font}{ = \meta{fontname}} {default none, initial = code2000}
  This key sets the saurashtra font.
\end{docKey}

Saurashtra or Sourashtra or {\saurashtra ꢱꣃꢬꢵꢰ꣄ꢜ꣄ꢬꢵ} or Palkar or Patkar (Sanskrit: सौराष्ट्र, Tamil: சௌராட்டிரம்) is an Indo-Aryan language[3] spoken by the Saurashtrian community native to Gujarat, who migrated and settled in Southern India. Madurai in Tamil Nadu has the highest number of people belonging to this community and also remains as their cultural center.

The language is largely only in spoken form even though the language has its own script. The lack of schools teaching Saurashtra script and the language is often cited as a reason for the very few number of people who actually know to read and write in Saurashtra script. Latin, Devanagari or Tamil script is used as alternative for Saurashtra Script by many Saurashtrians.

Census of India places the language under Gujarati. Official figures show the number of speakers as 185,420 (2001 census).[4]


\begin{scriptexample}[]{Saurashtra}
\unicodetable{saurashtra}{"A880,"A890,"A8A0,"A8B0,"A8C0,"A8D0}
\end{scriptexample}


\begin{scriptexample}[]{Saurashtra}
\bgroup
\saurashtra

ꢮꢶꢯ꣄ꢮ ꢱꣃꢬꢵꢰ꣄ꢜ꣄ꢬꢪ꣄ ꢦꢡ꣄ꢬꢶꢒꢾ ꢱꢵꢡ꣄ꢡꢒꢸ ꢂꢮꢬꢾ
ꢮꣁꢭꢱ꣄ꢢꢵꢥꢪꢸꢒ꣄(ꣀꢵꢮꢾꢔꢹ ꢂꢮ꣄ꢬꢶꢫꣁ


\arial

Text: Vishwa Sourashtram \url{http://www.sourashtra.info/ghEr.htm}
\egroup
\end{scriptexample}


\printunicodeblock{./languages/saurashtra.txt}{\saurashtra}

\section{Gujarati}
\label{s:gujarati}
\idxlanguage{Gujarati}
\index{Unicode>Gujarati}
%FIXME
\index{languages>Gujarati}\index{languages>Gujǎrātī Lipi}
has its own writing system, distinct but related to several other Indian languages' writing systems, such as the one used to write Hindi. Strictly speaking, the Gujarati writing system is what is called an \emph{abugida} (and not an \textit{alphabet}), because the consonant characters all contain an inherent vowel, and other vowels are written as accents added on to the consonant characters. There are also symbols for stand-alone vowels.

The Gujarati script ({\gujarati{ગુજરાતી લિપિ }} Gujǎrātī Lipi), which like all Nāgarī writing systems is strictly speaking an abugida rather than an alphabet, is used to write the Gujarati and Kutchi languages. It is a variant of Devanāgarī script differentiated by the loss of the characteristic horizontal line running above the letters and by a small number of modifications in the remaining characters.
With a few additional characters, added for this purpose, the Gujarati script is also often used to write Sanskrit and Hindi.
Gujarati numerical digits are also different from their Devanagari counterparts.
\medskip

\bgroup
\newfontfamily\gujaratilohit[Script=Gujarati,Scale=1.5]{Lohit-Gujarati.ttf}
\gujarati

\centering

\underline{English/Hindi/Gujarati Alphabets}

\hskip-1.5cm\begin{tabular}{lllllllllllllllllllll}
A &B &bh &C &ch &chh &D &dh &E &F &G &gh &H &I &J &K &kh &L &M &N &O\\

अ &ब &भ &क &च &छ &ड/द &ध/ढ़ &इ &फ &ग &घ &ह &ई &ज &क &ख &ल &म &न/ण &ऑ\\

અ &બ &ભ &ક &ચ &છ &ડ/દ &ધ /ઢ &ઇ &ફ &ગ &ઘ &હ &ઈ &જ &ક &ખ &લ &મ &ન/ણ &ઓ\\

\end{tabular}
\egroup

\medskip

Gujarati has its own set of numeric signs (placed alongside their Hindu-Arabic [or Indo-Arabic] counterparts in the tables below), they are employed in much the same way as English;  that is to say, they are put together in the same manner in order to express larger numbers. It is quite possible to simply substitute the Gujarati numerals for the Hindu-Arabic ones.

The Gujarati words for 1-10 are as follows:
\medskip

\bgroup
\begin{center}
\gujarati
\begin{tabular}{ccl}
Arabic & Gujarati &Name\\
Numeral &Numeral  &\\
0	&૦	&mīṇḍuṃ or shunya\\
1	&૧	&ekaṛo or ek\\
2	&૨	&bagaṛo or bay\\
3	&૩	&tragaṛo or tran\\
4	&૪	&chogaṛo or chaar\\
5	&૫	&pāchaṛo or paanch\\
6	&૬	&chagaṛo or chah\\
7	&૭	&sātaṛo or sāt\\
8	&૮	&āṭhaṛo or āanth\\
9	&૯	&navaṛo or nav\\
10 &૧૦ &દસ das\\

\end{tabular}
\end{center}
\egroup

\chapter{Tamil}

\epigraph{Women live like bats or owls.\\Labour like beasts\\and die like worms\ldots}{Margaret of Newcastle, 1660, England}



\label{s:tamil}
\newfontfamily\tamil[Scale=1.0, Script=Tamil]{code2000.ttf}

\def\tamiltext#1{{\tamil#1}}

\section{Background and History}

Of all the Dravidian languages Tamil has the longest literary tradition, covering
more than two thousand years. The earliest records are cave inscriptions from
the second century \textsc{bce}; the earliest extant literary text is the grammar
Tolkāppiyam (100 \textsc{bce}), which describes the grammar and poetics of Tamil during
that period. The dating of the Tolkāppiyam is still disputed by scholars proposing dates from
5 \textsc{bce} to 600 \textsc{ce}. 

During its two-thousand-year uninterrupted history, Tamil distinguishes
three different stages: Old Tamil (300 \textsc{bce} to 700 \textsc{ce}), Middle Tamil (700
\textsc{ce} to 1600) and Modern Tamil (1600 \textsc{ce} to the present), each with distinct
grammatical characteristics.\index{Dravidian>Tamil}\index{Tamil}


\begin{figure}[htbp]
\bgroup
\parindent=0pt
\centering
\includegraphics[width=0.9\linewidth-2\parindent]{./images/old-tamil-inscription.jpg}

\caption{Mangulam Tamil Brahmi inscription at Dakshin Chithra, Chennai (wikipedia)}

\egroup
\end{figure}

The Tamil script (\tamiltext{தமிழ் அரிச்சுவடி} tamiḻ ariccuvaṭi) is an abugida script that is used by the Tamil people in India, Sri Lanka, Malaysia and elsewhere, to write the Tamil language, as well as to write the liturgical language Sanskrit, using consonants and diacritics not represented in the Tamil alphabet. Certain minority languages such as Saurashtra, Badaga, Irula, and Paniya are also written in the Tamil script. \index{Surashtra}\index{Badaga}\index{Irula}

The Tamil script has 12 vowels (\tamiltext{உயிரெழுத்து} uyireḻuttu ``soul-letters''), 18 consonants (\tamiltext{மெய்யெழுத்து} meyyeḻuttu ``body-letters").
An additional character, the āytam \tamiltext{ஃ (ஆய்தம்)},  is classified in Tamil grammar as being neither a consonant nor a vowel (\tamiltext{அலியெழுத்து} aliyeḻuttu ``the hermaphrodite letter''), though often considered as part of the vowel set (\tamiltext{உயிரெழுத்துக்கள்} uyireḻuttukkaḷ ``vowel class''). The script, however, is syllabic and not alphabetic. The complete script, therefore, consists of the thirty-one letters in their independent form, and an additional 216 combinant letters representing a total 247 combinations (\tamiltext{உயிர்மெய்யெழுத்து} uyirmeyyeḻuttu) of a consonant and a vowel, a mute consonant, or a vowel alone. These combinant letters are formed by adding a vowel marker to the consonant. Some vowels require the basic shape of the consonant to be altered in a way that is specific to that vowel. Others are written by adding a vowel-specific suffix to the consonant, yet others a prefix, and finally some vowels require adding both a prefix and a suffix to the consonant. In every case the vowel marker is different from the standalone character for the vowel.
The Tamil script is written from left to right.\index{hermaphrodite letter}


\section{Unicode}

Tamil is a Unicode block containing characters for the Tamil, Badaga, and Saurashtra languages of Tamil Nadu India, Sri Lanka, Singapore, and Malaysia. In its original incarnation, the code points U+0B02..U+0BCD were a direct copy of the Tamil characters A2-ED from the 1988 ISCII standard. The Devanagari, Bengali, Gurmukhi, Gujarati, Oriya, Telugu, Kannada, and Malayalam blocks were similarly all based on their ISCII encodings.

\begin{scriptexample}[]{Tamil}
\unicodetable{tamil}{"0B80,"0B90,"0BA0,"0BB0,"0BC0,"0BE0,"0BF0}

\hfill  Typeset with \cmd{\tamil} and \texttt{code2000.ttf}
\end{scriptexample}

\subsection{Tamil Numbers and Numerals}

Originally, Tamils did not use zero, nor did they use positional digits (having separate 
symbols for the numbers 10, 100 and 1000). Symbols for the numbers are similar to 
other Tamil letters, with some minor changes. 

For example, the number 3782 is not written as \tamiltext{௩௭௮௨} as in modern usage. Instead it 
is written as \tamiltext{௩ ௲ ௭ ௱ ௮ ௰ ௨}. This would be read as they are written as 
Three Thousands, Seven Hundreds, Eight Tens, Two; or in Tamil as 
\tamiltext{௩௲௭௱௮௰௨ž}.\footnote{https://cloud.github.com/downloads/raaman/Tamil-Numeral/tamilnumbers.html}

\subsection{Dates}

Once the script is loaded the day, month and year can be loaded using the command  \cmd{\tamildate}, which returns the |\today| formatted as per custom Tamil. 

\begin{center}
\bgroup
\tamil
\begin{tabular}{lll}
day	 &month	&year	\\

௳	&௴	      &௵	\\

u	&mee	      &wa	\\
\end{tabular}
\egroup
\end{center}


\section{Grantha}
\label{s:grantha}

Grantha is a Unicode block containing the ancient Grantha script characters of 6th to 19th century Tamil Nadu and Kerala for writing Sanskrit and Manipravalam. Battled to get it working, as I could not find an appropriate unicode font. The font would need remapping. Unfortunately this is a script with no Noto support.

\begin{figure}[htbp]
\bgroup
\parindent=0pt
\centering
\includegraphics[width=\linewidth]{./images/grantha.jpg}

\caption{An image of a palm leaf manuscript with Sanskrit written in Grantha script (wikipedia)}

\egroup
\end{figure}

\newfontfamily\grantha{e-Grantamil 7}%e-Grantamil 7

\begin{scriptexample}[\grantha]{Tamil}
\unicodetable{grantha}{"0D0,"0D1,"0D2,"1133,"1134,"1135,"1136,"1137}

\hfill  Typeset with \cmd{\grantha} and \texttt{e-Granthamil 7.ttf}
\end{scriptexample}

{
\grantha \char"11311

}

%\newfontfamily\freeserif{FreeSerif}
%
%
%\freeserif \lorem
%\begin{tabular}{lll}
%day	 &month	&year	\\
%
%௳	&௴	      &௵	\\
%
%u	&mee	      &wa	\\
%\end{tabular}



\section{Malayalam}
\label{s:malayalam}
\newfontfamily\malayam[Scale=1.1]{Lohit-Malayalam.ttf}

\def\malamtext#1{{\malayam#1}}


Malayalam is a language spoken by the native people of southwestern India (from Thuckalay to Talapady).According to the Indian census of 2011, there were 32,299,239 speakers of Malayalam in Kerala, making up 93.2\% of the total number of Malayalam speakers in India, and 96.74\% of the total population of the state. There were a further 701,673 (2.1\% of the total number) in Karnataka, 957,705 (2.7\%) in Tamil Nadu, and 406,358 (1.2\%) in Maharashtra. The number of Malayalam speakers in Lakshadweep is 51,100, which is only 0.15\% of the total number, but is as much as about 84\% of the population of Lakshadweep. In all, Malayalis made up 3.22\% of the total Indian population in 2011. Of the total 34,713,130 Malayalam speakers in India in 2011, 33,015,420 spoke the standard dialects, 19,643 spoke the Yerava dialect and 31,329 spoke non-standard regional variations like Eranadan.[37] As per the 1991 census data, 28.85\% of all Malayalam speakers in India spoke a second language and 19.64\% of the total knew three or more languages.


\includegraphics[width=\textwidth]{nangeli}
https://feminisminindia.com/2016/09/12/kerala-breast-tax-nangeli/


Large numbers of Malayalis have settled in Delhi, Bangalore, Hyderabad, Mumbai (Bombay), Pune and Chennai (Madras). A large number of Malayalis have also emigrated to the Middle East, the United States, and Europe. There were 179,860 speakers of Malayalam in the United States, according to the 2000 census, with the highest concentrations in Bergen County, New Jersey and Rockland County, New York.[38] There were 7,093 Malayalam speakers in Australia in 2006.[39] The 2001 Canadian census reported 7,070 people who listed Malayalam as their mother tongue, mainly in Toronto. The 2006 New Zealand census reported 2,139 speakers.[40] 134 Malayalam speaking households were reported in 1956 in Fiji. There is also a considerable Malayali population in the Persian Gulf regions, especially in Dubai and Doha. Recently a Keralite is elected as mayor in Loughten town of England.

The Malayalam script (Malayalam: \malamtext{മലയാളലിപി}, Malayāḷalipi, IPA: [mɐləjaːɭɐ lɪβɪ], also known as Kairali script (Malayalam: \malamtext{കൈരളീലിപി}), is a Brahmic script used commonly to write the Malayalam language—which is the principal language of the Indian state of Kerala, spoken by 35 million people in the world.[3] Like many other Indic scripts, it is an alphasyllabary (\textit{abugida}), a writing system that is partially “alphabetic” and partially syllable-based. The modern Malayalam alphabet has 15 vowel letters, 41 consonant letters, and a few other symbols. The Malayalam script is a Vattezhuttu script, which had been extended with Grantha script symbols to represent Indo-Aryan loanwords.[4] The script is also used to write several minority languages such as Paniya, Betta Kurumba, and Ravula.[5] The Malayalam language itself was historically written in several different scripts.

\begin{scriptexample}[]{Malayalam}
\centerline{\Huge\malamtext{കൈരളീലിപി}}
\end{scriptexample}
\section{Syloti Nagri}
\label{s:sylotinagri}
\newfontfamily\syloti{NotoSansSylotiNagri-Regular.ttf}
\newfontfamily\damase{damase_v.2.ttf}
\index{languages>Sylheti Nagari}

Sylheti or Syloti (i.e. "Silēṭī" Bengali: সিলেটী or "Silôṭī" Bengali: ছিলটী) is one of the Bengali dialects, primarily spoken in the Sylhet Division of northeast Bangladeshi district Moulvibazar,Sylhet,Sunamganj,Hobiganj and the Barak Valley region of southern Assam. (Although sometimes it is considered an independent language for not sharing grammatical mutual intelligibility), it is a similar language to Standard Bengali, with which it shares a high proportion of vocabulary: Spratt and Spratt (1987) report 70\% shared vocabulary, while Chalmers (1996) reports at least 80\% overlap.

Sylheti Nagari or Syloti Nagri (Silôṭi Nagôri) is the original script used for writing the Sylheti language. It is an almost extinct script, this is because the Sylheti Language itself was reduced to only dialect status after Bangladesh gained independence and because it did not make sense for a dialect to have its own script, its use was heavily discouraged. The government of the newly formed Bangladesh did so to promote a greater "Bengali" identity. This led to the informal adoption of the Eastern Nagari script also used for Bengali and Assamese. It is also known as Jalalabadi Nagri, Mosolmani Nagri, Ful Nagri etc.

Sylheti Nagari was added to the Unicode Standard in March, 2005 with the release of version 4.1.
The Unicode block for Sylheti Nagari is U+A800–U+A82F:

\begin{scriptexample}[]{Sylheti}
\unicodetable{damase}{"A800,"A810,"A820}
\end{scriptexample}


\printunicodeblock{./languages/syloti.txt}{\damase}



\section{Limbu}
\label{s:limbu}

The Limbu script is used to write the Limbu language. The Limbu script is an abugida derived from the Tibetan script. Limbu is a Tibeto-Burman language spoken mainly in Nepal,[3] significant communities in Bhutan, Sikkim, Darjeeling district, India by the Limbu community. Virtually all Limbus are bilingual in Nepali.

\newfontfamily\limbu{code2000.ttf}

According to traditional histories, the Limbu script was first invented in the late 9th century by King Sirijonga Haang, then fell out of use, to be reintroduced in the 18th century by Te-ongsi Sirijunga Xin Thebe.

To change the inherent vowel, a diacritic is added. Shown here on /k/ ({\limbu ᤁ}):
Appearance	IPA

\begin{table}[htb]
\centering
\begin{tabular}{>{\Large\bfseries\limbu}l>{\arial}l}
ᤁᤡ	&/ki/\\
ᤁᤣ	&/ke/\\
ᤁᤧ	&/kɛ/\\
ᤁᤠ	&/ka/\\
ᤁᤨ	&/kɔ/\\
ᤁᤥ	&/ko/\\
ᤁᤢ	&/ku/\\
ᤁᤤ	&/kai/\\
ᤁᤦ	&/kau/\\
\end{tabular}
\caption{Changing the inherent vowel, using a diacritic.}
\end{table}




\begin{scriptexample}[]{Limbu}
\unicodetable{limbu}{"1900,"1910,"1920,"1930,"1940}
\end{scriptexample}


\printunicodeblock{./languages/limbu.txt}{\limbu}



\section{Cham}
\label{s:cham}

The Cham alphabet is an abugida used to write Cham, an Austronesian language spoken by some 230,000 Cham people in Vietnam and Cambodia. It is written horizontally left to right, as is English.

\newfontfamily\cham{Noto Sans Cham}

Cham is a Unicode block containing characters for writing the Cham language, primarily used for the Eastern dialect in Cambodia.
Cham script was added to the Unicode Standard in April, 2008 with the release of version 5.1.
The Unicode block for Cham is \textsc{U+AA00–U+AA5F}:

\begin{scriptexample}[]{Cham}
\unicodetable{cham}{"AA10,"AA20,"AA30,"AA40,"AA50}
\end{scriptexample}


\printunicodeblock{./languages/cham.txt}{\cham}


\section{Sora Sompeng}
\label{s:sorasompeng}
The Sora Language is part of the Austroasiatic language family. More locally, however, it is a part of the \hyperref[s:munda]{Munda} Languages which include other tribal languages in close proximity to Sora. Sora is unique because although it is surrounded by the Indo-Aryan language \hyperref[s:oriya]{Oriya} and the Dravidian Language Telugu, Sora is more closely related to the languages of Southeast Asia such as Khmer in Cambodia than it is to the predominant languages of India. Moreover, Sora contains very little formal literature but has an abundance of folk tales and traditions. Most of their passed down knowledge is of the oral tradition. Compared to other languages in the Munda family, Sora is decreasing within the Sora tribe at a faster rate. Most speakers are concentrated in Odisha and Andhra Pradesh but smaller communities also exist in Madhya Pradesh, Tamil Nadu, and Bihar.

Sorang Sompeng script is used to write in Sora, a Munda language with 300,000 speakers in India. The script was created by Mangei Gomango in 1936 and is used in religious contexts.[1] He was familiar with Oriya, Telugu and English, so the parent systems of the script are Brahmi and Latin.[2]
The Sora language is also written in the Latin alphabet and the Telugu script.

Sorang Sompeng script was added to the Unicode Standard in January, 2012 with the release of version 6.1. In Windows Nirmala UI.ttf (Windows 10.0) can be used. 

\newfontfamily\NirmalaU{Nirmala UI}


\unicodetable{NirmalaU}{"110D0,"110E0,"110F0}
 	

The Sora Bible employs a Latin-based orthography with a number of Sora-specific
conventions. Sora has also been rendered in the Oriya script in Orissa and in
the Telugu script in Andhra Pradesh, as well as a modified phonetic alphabet in
Ramamurti’s grammatical materials and dictionary. The use of and knowledge of
Sorang Sompeng (N. Zide 1996), the indigenous script, appears to be quite limited.
In many areas Sora remains a vital and thriving language, but one that has no state
or institutional support (sermons and materials are increasingly in Oriya in Gajapati
district which we observed and were told about in Christianized Sora communities).
In other areas, Sora is reportedly being or indeed has already been replaced by Telugu
or Oriya. So, although not an endangered language in sensu stricto, Sora (except in


\section{Numerals}\label{s:munda}

The Sora are unique in their numeral system. Instead of base 10, Sora uses a base 12 system. Only a few other languages in the world share this anomaly. Ekari, for example, uses a base 60 system.[7] For example, 39 in Sora arithmetic would be thought of as (1 * 20)+ 12 + 7. Here are the first 12 numerals in the Sora language :[7]
English: one two three four five six seven eight nine ten eleven twelve

Sora: aboy bago yagi unji monloy tudru gulji thamji tinji gelji gelmuy migel
Similar to how English uses the suffix from the numeral ten after twelve (such as thirteen, fourteen, etc.), Sora also uses a suffix assignment to numerals after 12 and before 20. Thirteen in Sora is expressed as migelboy (12+1), fourteen as migelbagu (12+2), etc.[7] Between numerals 20 and 99, Sora adds the suffix kuri to the first constituent of the numeral. For example, 31 is expressed as bokuri gelmuy and 90 as unjikuri gelji.[7]



\section{Ol Chiki script}
\label{s:olchiki}
\arial

The Ol Chiki script, also known as Ol Cemetʼ (Santali: ol 'writing', cemet' 'learning'), Ol Ciki, Ol, and sometimes as the Santali alphabet, was created in 1925 by Raghunath Murmu for the Santali language.

Previously, Santali had been written with the Latin alphabet. But because Santali is not an Indo-Aryan language (like most other languages in the south of India), Indic scripts did not have letters for all of Santali's phonemes, especially its stop consonants and vowels, which made writing the language accurately in an unmodified Indic script difficult. The detailed analysis was given by Dr. Byomkes Chakrabarti in his 'Comparative Study of Santali and Bengali'. Missionaries (first of all Paul Olaf Bodding, a Norwegian) brought the Latin script, which is better at representing Santali stops, phonemes and nasal sounds with the use of diacritical marks and accents. Unlike most Indic scripts, which are derived from Brahmi, Ol Chiki is not an abugida, with vowels given equal representation with consonants. Additionally, it was designed specifically for the language, but one letter could not be assigned to each phoneme because the sixth vowel in Ol Chiki is still problematic.
Ol Chiki has 30 letters, the forms of which are intended to evoke natural shapes. Linguist Norman Zide said "The shapes of the letters are not arbitrary, but reflect the names for the letters, which are words, usually the names of objects or actions representing conventionalized form in the pictorial shape of the characters."[1] It is written from left to right.

\newfontfamily\olchiki{code2000.ttf}

\begin{scriptexample}[]{olchiki}
\bgroup
\olchiki
\obeylines

U+1C5x 	᱐	᱑	᱒	᱓	᱔	᱕	᱖	᱗	᱘	᱙	ᱚ	ᱛ	ᱜ	ᱝ	ᱞ	ᱟ
U+1C6x	   ᱠ	ᱡ	ᱢ	ᱣ	ᱤ	ᱥ	ᱦ	ᱧ	ᱨ	ᱩ	ᱪ	ᱫ	ᱬ	ᱭ	ᱮ	ᱯ
U+1C7x  	ᱰ	ᱱ	ᱲ	ᱳ	ᱴ	ᱵ	ᱶ	ᱷ	ᱸ	ᱹ	ᱺ	ᱻ	ᱼ	ᱽ	᱾	᱿
\egroup

\unicodetable{olchiki}{"1C50,"1C60,"1C70}
\end{scriptexample}

\newfontfamily\sikkim{Tibetan Machine Uni}
\newfontfamily\lepcha{Mingzat-R.ttf}
\section{Lepcha}
\label{s:lepcha}
\epigraph{``Had your independence ensured mine, I surely would have greeted you on this moment every year ...''}{
August 15, 2015\\
Chewang Pintso\\
General Secretary, SIBLAC}

\label{s:lepcha}
\index{Scripts>Lepcha}



The Lepcha are also called the Rongkup meaning the children of God and the Rong, Mútuncí Róngkup Rumkup (Lepcha:{\lepcha ᰕᰫ་ᰊᰪᰰ་ᰆᰧᰶ ᰛᰩᰵ་ᰀᰪᰱ ᰛᰪᰮ་ᰀᰪᰱ}; "beloved children of the Róng and of God"), and Rongpa (Sikkimese:{\sikkim རོང་པ་}), are among the indigenous peoples of Sikkim and number between 30,000 and 50,000. Many Lepcha are also found in western and southwestern Bhutan, Tibet, Darjeeling, the Mechi Zone of eastern Nepal, and in the hills of West Bengal. The Lepcha people are composed of four main distinct communities: the Renjóngmú of Sikkim; the Támsángmú of Kalimpong, Kurseong, and Mirik; the ʔilámmú of Ilam District, Nepal; and the Promú of Samtse and Chukha in southwestern Bhutan.[3][2][4]\index{Languages>Lepcha}
\index{Nepal Languages>Lepcha}\index{Bhutan Languages>Lepcha} The Lepcha probably do not exceed 50,000 and hence their language is on the \textsc{UNESCO} endangered list of languages.

\begin{figure}[htbp]
\centering
\includegraphics[width=\linewidth-2\parindent]{lepchas}

\caption{Lepcha manuscript}

\end{figure}

The Lepcha have their own language, also called Lepcha. It belongs to the Bodish–Himalayish group of Tibeto-Burman languages. The Lepcha write their language in their own script, called Róng or Lepcha script, which is derived from the Tibetan script. It was developed between the 17th and 18th centuries, possibly by a Lepcha scholar named Thikúng Mensalóng, during the reign of the third Chogyal (Tibetan king) of Sikkim.[7] The world's largest collection of old Lepcha manuscripts is found with the Himalayan Languages Project in Leiden, Netherlands, with over 180 Lepcha books.

The Lepcha script, or Róng script is an abugida used by the Lepcha people to write the Lepcha language. Unusually for an abugida, syllable-final consonants are written as diacritics.

The United Nations Educational, Scientific and Cultural Organization (UNESCO) lists Lepcha as an endangered language with the following characterization:

The Lepcha language is spoken in Sikkim and Darjeeling district in West Bengal of India. The 1991 Indian census counted 39,342 speakers of Lepcha. Lepcha is considered to be one of the indigenous languages of the area in which it is spoken. Unlike most other languages of the Himalayas, the Lepcha people have their own indigenous script (the world's largest collection of old Lepcha manuscripts is kept in Leiden, with over 180 Lepcha books).

Lepcha is the language of instruction in some schools in Sikkim. In comparison to other Tibeto-Burman languages, it has been given considerable attention in the literature. Nevertheless, many important aspects of the Lepcha language and culture still remain undescribed. 


\begin{figure}[htbp]
\centering
\includegraphics[width=\linewidth-2\parindent]{lepcha}

\caption{A manuscript of the Van Manen Collection at the Kern Institute of Leiden University.}
\end{figure}

{\lepcha
Consonants bear the inherent vowel, but no virama is used to kill this vowel; vowel matras modify it, and
explicit final consonants are used where there is no inherent vowel. Initial vowels are represented with
the vowel matras on the neutral letter £ A. Initial consonants can be followed by the glides ˇ§ -YA and
ˇ• -RA, both of which normally ligate with the consonant they modify; these can also combine to form
ˇˆ -rya, which is simply a glyph ligature of the other two: ÄÙ kya, Äı kra, Ĉ krya. The glide -la is also
found, but is represented not by a ligating combining mark, but by a limited set of letters containing this
glide inherently. With few exceptions, these “combined” letters do not look like a ligature of their base
letters with some mark: Ä ka Å kla, É ga Ñ gla, é pa è pla, ë fa í fla, ì ba î bla, ï ma ñ mla,
ù ha û hla.}

The Mingzat font is still under development by SIL so I am not too sure if the rendering is correct\footnote{\url{http://scripts.sil.org/cms/scripts/page.php?site_id=nrsi&id=Mingzat}}.


\section{Unicode}

Lepcha script was added to the Unicode Standard in April, 2008 with the release of version 5.1.
The Unicode block for Lepcha is U+1C00–U+1C4F:
\begin{scriptexample}[]{Lepcha}
\bgroup
\lepcha
\obeylines
 	    0	1	2	3	4	5	6	7	8	9	A	B	C	D	E	F
U+1C0x	 ᰀ	ᰁ	ᰂ	ᰃ	ᰄ	ᰅ	ᰆ	ᰇ	ᰈ	ᰉ	ᰊ	ᰋ	ᰌ	ᰍ	ᰎ	ᰏ
U+1C1x	 ᰐ	ᰑ	ᰒ	ᰓ	ᰔ	ᰕ	ᰖ	ᰗ	ᰘ	ᰙ	ᰚ	ᰛ	ᰜ	ᰝ	ᰞ	ᰟ
U+1C2x	 ᰠ	ᰡ	ᰢ	ᰣ	ᰤ	ᰥ	ᰦ	ᰧ	ᰨ	ᰩ	ᰪ	ᰫ	ᰬ	ᰭ	ᰮ	ᰯ
U+1C3x	 ᰰ	ᰱ	ᰲ	ᰳ	ᰴ	ᰵ	ᰶ	᰷	x	x	x	᰻	᰼	᰽	᰾	᰿
U+1C4x	 ᱀	᱁	᱂	᱃	᱄	᱅	᱆	᱇	᱈	᱉	x	x	x	ᱍ	ᱎ	ᱏ

\egroup
\end{scriptexample}




\section{Sharada}
\label{s:sharada}
The Śāradā, or Sharada, script (शारदा) is an abugida writing system of the Brahmic family of scripts, developed around the 8th century. It was used for writing Sanskrit and Kashmiri. The Gurmukhī script was developed from Śāradā. Originally more widespread, its use became later restricted to Kashmir, and it is now rarely used except by the Kashmiri Pandit community for ceremonial purposes. Śāradā is another name for Saraswati, the goddess of learning.
Śāradā script was added to the Unicode Standard in January, 2012 with the release of version 6.1.

The Unicode block for Śāradā script, called Sharada, is U+11180–U+111DF: Unable to locate font in unicode.




%@book{book:1681159,
%   title =     {The Munda Languages},
%   author =    {Norman H. Zide, Gregory D. S. Anderson},
%   publisher = {Routledge},
%   isbn =      {041532890X,9780415328906},
%   year =      {2008},
%   series =    {Routledge Language Family Series},
%   edition =   {},
%   volume =    {},
%   url =       {http://gen.lib.rus.ec/book/index.php?md5=CD5787A1386CD03191256D28E1D5DAD4}}


\chapter{Phags-pa}
\label{s:phagspa}
\newfontfamily\phagspa{code2000.ttf}
\arial 
The 'Phags-pa script, (Mongolian: дөрвөлжин үсэг "Square script") was an alphabet designed by the Tibetan monk and vice-king Drogön Chögyal Phags-pa for the Mongol Yuan emperor Kublai Khan as a unified script for the literary languages of the Yuan. 


It was first promulgated in 1269, although there is an inscription to testify its use before that time. ThePP alphabet is considered to have been designed for all the languages of the Mongol empire, but it appears it was almost used exclusively for Mongolian and Chinese. 

Widespread use was limited to about a hundred years during the Yuan Dynasty, and it fell out of use with the advent of the Ming dynasty. The documentation of its use provides clues about the changes in the varieties of Chinese, the Tibetic languages, Mongolian and other neighboring languages during the Yuan era.

After the fall of the Yuan dynasty in 1368, PP fell out of use, although it may have survived on seals and in some copybooks (although some hold that these descend froma Tibetan seal script rather than from PP).  

PP was based on the Tibetan alphabet and was used for writing both Chinese and Mongolian.

The script as a whole system comes down to us in two different traditions. On the one hand we have the letters and arrangements as they were recorded in the \textit{Shu shih hui yao} and the Fas shu k'ao, two 14th century works on calligraphy. Here the letters are presented in a more Buddist and Tibetan tradition.


\begin{figure}[htbp]
\includegraphics[width=1\linewidth]{./images/phags-pa.jpg}

credit \protect\url{http://turfan.bbaw.de/dta/monght/images/monght009_seite2.jpg}
\end{figure}


\begin{scriptexample}[]{Phags-pa}
\bgroup
\unicodetable{phagspa}{"A840,"A850,"A860,"A870}

\arial
\hfill Typeset with \texttt{code2000.ttf} and \cmd{\phagspa}

\egroup
\end{scriptexample}
\medskip

Phags-pa is a historical script related to Tibetan that was created as the national script of
the Mongol empire. Even though Phags-pa was used mostly in Eastern and Central Asia for
writing text in the Mongolian and Chinese languages, it is discussed in this chapter because
of its close historical connection to the Tibetan script. The script has very limited modern use. It bears similarity to Tibetan and has no case distinctions. It is written vertically in columns running for left to right, like Mongolian. Units are often composed of several syllables and sometimes are separated by whitespace.


\printunicodeblock{./languages/phags-pa.txt}{\phagspa}

\cxset{script/.code={}}
\cxset{script=phags-pa}

\begin{docKey}[phd]{script}{ = \meta{phags-pa}} {}
The key |script| will activate the commands available for typesetting the phags-pa script.
\end{docKey}















\cxset{image=chakma.jpg}
\chapter{Chakma}
\label{s:chakma}

\newfontfamily\chakma{RibengUni.ttf}

The Chakma alphabet (Ajhā pāṭh), also called Ojhapath, Ojhopath, Aaojhapath, is an abugida used for the Chakma language and which is being adapted for the Tanchangya language.[1] The forms of the letters are quite similar to those of the Burmese script.

\begin{figure}[htbp]
\includegraphics[width=\linewidth-2\parindent]{chakma}
\end{figure}

The Chakma (Chakma or ), also known as the Changma, are a Tibeto-Burman tribe of the Chittagong Hill Tracts inBangladesh. Today, the geographic distribution of Chakmas is spread across Bangladesh and parts of northeastern India, westernBurma, China and diaspora communities in North America and Europe. Within the CHT, the Chakma are the largest ethnic group and make up half of the region's population. In Burma, they are known as Daingnet people. The Chakma are divided into 46 clans orGozas. They have their own language, customs and culture, and profess Theravada Buddhism. The Chakma Royal Family is one of the major Buddhist royal houses of the South Asia.

Chakmas are Tibeto-Burman, and are thus closely related to tribes in the foothills of the Himalayas. The Chakmas are believed to be originally from Arakan who later on immigrated to Bangladesh in around fifteenth century, settling in the Cox's Bazar District, the Korpos Mohol area, and in the Indian states of Mizoram, Arunachal Pradesh, Tripura.\href{http://thechakmadiary.weebly.com/about.html}{thechakmadiary}

The Arakanese referred to the Chakmas as Saks or Theks. In 1546, when the king of Arakan, Meng Beng, was engaged in a battle with the Burmese, the Sak king appeared from the north and attacked Arakan, and occupied the Ramu of Cox's Bazar, the then territory of the kingdom of Arakan
\bgroup
\obeylines
\chakma
𑄇𑄳𑄇 Kkā = 𑄇 Kā + 𑄳 VIRAMA + 𑄇 Kā
𑄇𑄳𑄑 Ktā = 𑄇 Kā + 𑄳 VIRAMA + 𑄑 Tā
𑄇𑄳𑄖 Ktā = 𑄇 Kā + 𑄳 VIRAMA + 𑄖 Tā
𑄇𑄳𑄟 Kmā = 𑄇 Kā + 𑄳 VIRAMA + 𑄟 Mā
𑄇𑄳𑄌 Kcā = 𑄇 Kā + 𑄳 VIRAMA + 𑄌 Cā
𑄋𑄳𑄇 ńkā = 𑄋 ńā + 𑄳 VIRAMA + 𑄇 Kā
𑄋𑄳𑄉 ńkā = 𑄋 ńā + 𑄳 VIRAMA + 𑄉 Gā
𑄌𑄳𑄌 ccā = 𑄌 cā + 𑄳 VIRAMA + 𑄌 Cā

\egroup

Fonts for the script are not available easily but the
the script can be typeset using \texttt{RibengUni.ttf} which is available at \url{http://uni.hilledu.com/}. 

\begin{scriptexample}[]{Chakma}
\unicodetable{chakma}{"11100,"11110,"11120,"11130,"11140}

\texttt{RibengUni.ttf}
\end{scriptexample}


\printunicodeblock{./languages/chakma.txt}{\chakma}


\section{Brahmi}
\label{s:brahmi}
Brāhmī is the modern name given to one of the oldest writing systems used in the Indian subcontinent and in Central Asia during the final centuries BCE and the early centuries CE. Like its contemporary, Kharoṣṭhī, which was used in what is now Afghanistan and Western Pakistan, Brahmi (native to north and central India) was an \emph{abugida}.

The A´sokan Br¯ahm¯ı of the third century BCE is the mother of all major Indian scripts,
both Indo-Aryan and Dravidian. It was an alpha-syllabic script with diacritics used for
vowels occurring in postconsonantal position. It has separate symbols for the five primary
vowels a i u e o, twenty-five occlusives and eight sonorants and fricatives. The Br¯ahm¯ı
script was used in the rock edicts set up by the Mauryan Emperor A´soka to spread the
Buddhist faith in different parts of the country. The languages represented were Pali
and certain early regional varieties of Middle Indic. The origin of the Br¯ahm¯ı script is
controversial; nearly half of the characters are said to bear similarity to the consonant
symbols employed in the South Semitic script, eventually traceable to Aramaic script of
2000 BCE (Daniels and Bright 1996: §30, 373–83).

The best-known Brahmi inscriptions are the rock-cut edicts of Ashoka in north-central India, dated to 250–232 BCE. The script was deciphered in 1837 by James Prinsep, an archaeologist, philologist, and official of the East India Company.[1] The origin of the script is still much debated, with current Western academic opinion generally agreeing (with some exceptions) that Brahmi was derived from or at least influenced by one or more contemporary Semitic scripts, but a current of opinion in India favors the idea that it is connected to the much older and as-yet undeciphered Indus script

\begin{figure}[htb]
\centering
\includegraphics[width=0.6\textwidth]{./images/ashoka-pillar.jpg}
\caption{Brahmi script on Ashoka Pillar}
\end{figure}



\begin{scriptexample}[]{Brahmi}
\bgroup
\raggedleft
\brahmi

         
   

\arial
\hfill Text: Asokan Edict typeset with \texttt{NotoSansBrahmi-Regular} 
\egroup
\end{scriptexample}

Brahmi is a Unicode block containing characters written in India from the 3rd century BCE through the first millennium CE. It is the predecessor to all modern Indic scripts.

\begin{scriptexample}[]{Brahmi}
\unicodetable{brahmi}{"11000,"11010,"11020,"11030,"11040,"11050,"11060,"11070}
\end{scriptexample}


\printunicodeblock{./languages/brahmi.txt}{\brahmi}











\egroup
\newfontfamily\cjk{NotoSerifCJK-Regular.ttc}
\index{Katakana}\index{Hiragana}
\index{Bopomofo}\index{Hangul}\index{Yi}
\index{East Asian Scripts>Katakana}
\index{East Asian Scripts>Hiragana}
\index{East Asian Scripts>Hangul}
\index{East Asian Scripts>Bopomofo}
\index{East Asian Scripts>Yi}
\index{scripts>cjk}
\pagestyle{headings}
\index{Yi fonts>Microsoft Yi Baiti}
\chapter{East Asian Scripts}
\cxset{epigraph width=0.7\linewidth}
\epigraph{

For writing is the foundation of the classics and the arts, the beginning of
royal government. It is the means by which people of the past reach posterity,
by which people of the future know the past. 

{\cjk 蓋文字者,經藝之本,王政之始。前人所以垂後,\\ 後人所以識古。}
}{ Xu Shen  in the ``Postface'' of the \emph{Shuowen}}

\bigskip

\noindent This chapter presents the most common scripts currently in use in East Asia. This includes Chinese, Japanese and Korean. It also discusses several scripts for minority languages spoken in southern China. The scripts discussed are as follows:


\begin{center}
\begin{tabular}{lll}
\nameref{s:han} &Hiragana &Hangul\\
\nameref{s:bopomofo} &Katakana &\nameref{s:yi}\\
\end{tabular}
\end{center}
\bigskip

\parindent1em

\paragraph{Putonghua}The national language of China is Putonghua (Modern Standard Chinese), a standardized version of the Beijing dialect of Mandarin Chinese. As described earlier, there are hundreds of other regional
languages spoken in China, normally referred to as dialects or dialect
groups. Since the founding of the People’s Republic of China in 1949,
knowledge and use of Putonghua has been successfully promoted across
the country by a range of government measures, especially in education.
Most of the population are able to speak the language, and an even
greater percentage, perhaps as many as 90 per cent, can understand it
(Chen 1999: 27–30).

There are more than fifty-five officially recognized minority nationalities,
speaking scores of languages of the Tibeto-Burman, Tai, and Hmong-Mien
families in the south, and the Altaic family in the north (Ramsey 1987:
chs. 10 and 11; Blum 2002). In geographical terms the most widespread are
Uighur/Uyghur, Mongolian, and Tibetan, but these lie outside the geographical
area covered by this book. The greatest degree of linguistic
diversity is in the south and southwest in the provinces of Guangxi,
Guizhou, and Yunnan. Population-wise, the largest non-Sinitic language is
Zhuang (Tai), mainly in Guangxi province, where it has some official
functions. Rather confusingly, there is no one-to-one match between ethnic
nationality names and language names; for example, the nationality
identified as Yi contains speakers of several distinct languages (including,
notably, Lolo). Under the Chinese constitution the national minorities
all have ‘‘the freedom to use and develop their own spoken and written
languages’’, but in practice official support mostly goes to the larger
minorities.


\section{History of the Language}

\epigraph{On the other hand, even well educated people
could not understand the secret meaning of the Taoist characters (fu’s{\cjk 符}), but they
are convinced of the power of those figures.}{---Alex Chengyu Fang and François Thierry, \textit{The Language
and Iconography
of Chinese Charms
Deciphering a Past Belief System} }

The relationship between Chinese and other Sino-Tibetan languages is an area of active research and controversy, as is the attempt to reconstruct Proto-Sino-Tibetan. The main difficulty in both of these efforts is that, while there is very good documentation that allows for the reconstruction of the ancient sounds of Chinese, there is no written documentation of the point where Chinese split from the rest of the Sino-Tibetan languages. This is actually a common problem in historical linguistics, a field which often incorporates the comparative method to deduce these sorts of changes. Unfortunately the use of this technique for Sino-Tibetan languages has not as yet yielded satisfactory results, perhaps because many of the languages that would allow for a more complete reconstruction of Proto-Sino-Tibetan are very poorly documented or understood. Therefore, despite their affinity, the common ancestry of the Chinese and Tibeto-Burman languages remains an unproven hypothesis.[1]

Categorization of the development of Chinese is a subject of scholarly debate. One of the first systems was devised by the Swedish linguist Bernhard Karlgren in the early 1900s. The system was much revised, but always heavily relied on Karlgren's insights and methods.

\begin{figure}[htbp]
\centering

\includegraphics[width=0.45\linewidth]{oracle}

\end{figure}

Oracle bones (Chinese: {\cjk 甲骨}; pinyin: {\cjk jiǎgǔ}) are pieces of ox scapula or turtle plastron, which were used for pyromancy – a form of divination – in ancient China, mainly during the late Shang dynasty. Scapulimancy is the correct term if ox scapulae were used for the divination; plastromancy if turtle plastrons were used.

Diviners would submit questions to deities regarding future weather, crop planting, the fortunes of members of the royal family, military endeavors, and other similar topics.[1] These questions were carved onto the bone or shell in oracle bone script using a sharp tool. Intense heat was then applied with a metal rod until the bone or shell cracked due to thermal expansion. The diviner would then interpret the pattern of cracks and write the prognostication upon the piece as well.[2] By the Zhou dynasty, cinnabar ink and brush had become the preferred writing method, resulting in fewer carved inscriptions and often blank oracle bones being unearthed.

The oracle bones bear the earliest known significant corpus of ancient Chinese writing[a] and contain important historical information such as the complete royal genealogy of the Shang dynasty.[b] When they were discovered and deciphered in the early twentieth century, these records confirmed the existence of the Shang, which some scholars had until then doubted.

Old Chinese, sometimes known as ``Archaic Chinese'', was the language common during the early and middle Zhou Dynasty (1122–256 BC), texts of which include inscriptions on bronze artifacts, the poetry of the Shijing, the history of the Shujing, and portions of the Yijing (I Ching). The phonetic elements found in the majority of Chinese characters also provide hints to their Old Chinese pronunciations. The pronunciation of the borrowed Chinese characters in Japanese, and Vietnamese also provide valuable insights. Old Chinese was not wholly uninflected. It possessed a rich sound system in which aspiration or rough breathing differentiated the consonants, but probably was still without tones. Work on reconstructing Old Chinese started with Qing dynasty philologists.

Middle Chinese was the language used during the Sui, Tang and Song dynasties (6th through 10th centuries AD). It can be divided into an early period, reflected by the Qieyun rime dictionary (AD 601) and its later redaction the Guangyun, and a late period in the 10th century, reflected by rime tables such as the Yunjing. The evidence for the pronunciation of Middle Chinese comes from several sources: modern dialect variations, rime dictionaries, foreign transliterations, rime tables constructed by ancient Chinese philologists to summarize the phonetic system, and Chinese phonetic translations of foreign words. However, all reconstructions are tentative; for example, scholars have shown that trying to reconstruct modern Cantonese from the rimes of modern Cantopop would give a very inaccurate picture of its pronunciation.

The development of the spoken Chinese from early historical times to the present has been complex. Most Chinese people, in Sichuan and in a broad arc from the northeast (Manchuria) to the southwest (Yunnan), use various Mandarin dialects as their home language. The prevalence of Mandarin throughout northern China is largely due to north China's plains. By contrast, the mountains and rivers of southern China promoted linguistic diversity.

Until the mid-20th century, most southern Chinese only spoke their native local variety of Chinese. However, despite the mix of officials and commoners speaking various Chinese dialects, Nanjing Mandarin became dominant at least during the Qing Dynasty. Since the 17th century, the Empire had set up orthoepy academies (simplified Chinese:{\cjk 正音书院}; traditional Chinese: {\cjk 正音書院}; pinyin: Zhèngyīn Shūyuàn) to make pronunciation conform to the Qing capital Beijing's standard, but had little success. During the Qing's last 50 years in the late 19th century, the Beijing Mandarin finally replaced Nanjing Mandarin in the imperial court. For the general population, although variations of Mandarin were already widely spoken in China then, a single standard of Mandarin did not exist. The non-Mandarin speakers in southern China also continued to use their local languages for each and every aspect of life. The new Beijing Mandarin court standard was thus fairly limited.

This situation changed with the creation (in both the PRC and the ROC, but not in Hong Kong and Macau) of an elementary school education system committed to teaching Modern Standard Chinese (Mandarin). As a result, Mandarin is now spoken by virtually all people in mainland China and on Taiwan[citation needed]. At the time of the widespread introduction of Mandarin in mainland China and Taiwan, Hong Kong was a British colony and Mandarin was never used at all. In Hong Kong, Macau, Guangdong and sometimes Guangxi, the language of daily life, education, formal speech and business remains in the local Cantonese. However, Mandarin is becoming increasingly influential, which is seen as a threat by the locals, fearing that their native language might face a decline leading to its death





Settings for |cjk| languages and scripts follow:

\begin{docKey}[phd]{cjk font}{\meta{font name}}{default none, initial code2000.ttf}
This key when set produces all necessary command to set the font for cjk typesetting.
\end{docKey}

If your document is going to be primarily in chinese, you better off to use a dedicated class or package for the whole document. 

The easiest way is (for Simplified Chinese document only):

\begin{dispListing}
% UTF-8 encoding
% Compile with latex+dvipdfmx, pdflatex, xelatex or lualatex
% XeLaTeX is recommanded
\documentclass[UTF8]{ctexart}
\begin{document}
文章内容。
\end{document}
\end{dispListing}

or

\begin{dispListing}
\documentclass{article}
\usepackage[UTF8]{ctex}
...
\end{dispListing}


\begin{dispListing}
% Compile with xelatex
% UTF-8 encoding
\documentclass{article}
\usepackage{xeCJK}
\setCJKmainfont{SimSun}
\begin{document}
文章内容
\end{document}
\end{dispListing}


\parindent1em
\section{Han CJK Unified Ideographs}
\label{s:han}
\index{CJK}
The Chinese, Japanese and Korean (CJK) scripts share a common background. In the process called Han unification the common (shared) characters were identified, and named ``CJK Unified Ideographs''. Unicode defines a total of 74,617 CJK Unified Ideographs.[1]\footnote{\protect\url{http://shahon.org/wp-content/uploads/2010/02/Galambos-2006-Orthography-of-early-Chinese-writing.pdf}}



The terms ideographs or ideograms may be misleading, since the Chinese script is not strictly a picture writing system.
Historically, Vietnam used Chinese ideographs too, so sometimes the abbreviation ``CJKV" is used. This system was replaced by the Latin-based Vietnamese alphabet in the 1920s.

\section{Development of the script}

In the Oracle Bone and early bronze scripts, some but not all of the originally pictographic
characters were already stylized beyond recognition. There was great variation in the writing
of individual characters, and in the strokes used to render them. The subsequent development
of the script is a process of stylization, standardization, and reduction of the process of writing
to the repetition of a small number of stereotyped motions (strokes). Curved lines became
straight or angled, and pictographic iconicity was completely eliminated. 


Following the political unification of China by the first Qin emperor (221 BCE), a standard
script was imposed in place of the regional variants that had sprung up. The regularization of
the script continued into the Han, by which time the more or less modern script had emerged.
Pre-modern forms are still used in some contexts for aesthetic reasons, and various cursive
forms have emerged both as convenient shorthands and as calligraphic art forms, but the Kai
script of the Han dynasty has survived as the model for all subsequent Chinese writing. The
most recent change has been the official PRC simplifications of the 1950s, which reduced the
number of strokes in many characters without fundamentally altering the basic principles
of the script (in many cases by merely giving official blessing to folk shorthand characters).
An example of the historical progression can be seen in Figure~\ref{horsechar}. 

\begin{figure}[htbp]
\includegraphics[width=\textwidth]{chinese-horse-char}
\caption{Evolution of the ma ideogram}
\label{fig:horsechar}
\end{figure}

\unicodetable{cjk}{"4E00,"4E10,"4E20,"4E30,"4E40}




\section{Bopomofo}
\label{s:bopomofo}
Bopomofo is the colloquial name of the \textit{zhuyin fuhao} or \textit{zhuyin} system of phonetic notation for the transcription of spoken Chinese, particularly the Mandarin dialect. Consisting of 37 characters and four tone marks, it transcribes all possible sounds in Mandarin. 

Bopomofo was introduced in China by the Republican Government, in the 1910s and used alongside the Wade-Giles system, which used a modified Latin alphabet. The Wade system was replaced by \textit{Hanyu Pinyin} in 1958 by the Government of the People's Republic of China,[1] at the International Organization for Standardization (ISO) in 1982 (ISO 7098:1982). Bopomofo remains widely used as an educational tool and electronic input method in Taiwan. On Windows the font Microsoft JhengHei can be used. 

Windows fonts that can be used \texttt{Microsoft JhengHei} and \texttt{SimSun}.

U+3100–U+312F
\newfontfamily\bopomofo{Microsoft JhengHei}

\begin{scriptexample}[]{Bopomofo}
{\centering\bopomofo 

伯帛勃脖舶博渤霸壩灞

}

\hfill \texttt{Typeset with \cmd{\bopomofo} and Microsoft JhengHei font }
\end{scriptexample}

\begin{scriptexample}[]{Bopomofo}

{\centering\bopomofo

伯帛勃脖舶博渤霸壩灞

}
\hfill \texttt{Typeset with \cmd{\bopomofo} and JhengHei font }
\end{scriptexample}


The Bopomofo Extended block, running from \unicodenumber{U+31A0-U31BF}, contains less universally recognized Bopomofo characters used to write various non-Mandarin Chinese languages. A few additional tone marks are unified with characters in the Spacing Modifier Letters block. 












\section{Yi}
\label{s:yi}

The Yi script (Yi: {\yi ꆈꌠꁱꂷ} nuosu bburma [nɔ̄sū bū̠mā]; Chinese: {\cjk 彝文}; pinyin: Yí wén) is an umbrella term for two scripts used to write the Yi language; Classical Yi, an ideogram script, the later Yi Syllabary. The script is also historically known in Chinese as Cuan Wen (Chinese: {\cjk 爨文}; pinyin: Cuàn wén) or Wei Shu (simplified Chinese: {\cjk 韪书}; traditional Chinese: {\cjk 違書}; pinyin: Wéi shū) and various other names ({\cjk 夷字、倮語、倮倮文、毕摩文}), among them "tadpole writing" ({\cjk 蝌蚪文}).[1]

This is to be distinguished from romanized Yi ({\yi 彝文罗马拼音} Yiwen Luoma pinyin) which was a system (or systems) invented by missionaries and intermittently used afterwards by some government institutions.[2][3] There was also a Yi abugida or alphasyllabary devised by Sam Pollard, the Pollard script for the Miao language, which he adapted into "Nasu" as well.[4][5] Present day traditional Yi writing can be sub-divided into five main varieties (Huáng Jiànmíng 1993); Nuosu (the prestige form of the Yi language centred on the Liangshan area), Nasu (including the Wusa), Nisu (Southern Yi), Sani ({\yi 撒尼}) and 
Azhe ({\yi 阿哲}).[6][7]


The Yi or Lolo people[3] are an ethnic group in China, Vietnam, and Thailand. Numbering 8 million, they are the seventh largest of the 55 ethnic minority groups officially recognized by the People's Republic of China. They live primarily in rural areas of Sichuan, Yunnan, Guizhou, and Guangxi, usually in mountainous regions. As of 1999, there were 3,300 "Lô Lô" people living in the Hà Giang, Cao Bằng, and Lào Cai provinces in northeastern Vietnam.
The Yi speak various Loloish languages, Sino-Tibetan languages closely related to Burmese. The prestige variety is Nuosu, which is written in the Yi script.

\begin{figure}[htbp]
\includegraphics[width=\linewidth-2\parindent]{yi}

\caption{Yi people in traditional costumes. \href{https://www.dreamstime.com/stock-photos-yi-minority-women-traditional-clothes-image25450383}{dreamsite}}
\end{figure}


The Unicode block for Modern Yi is Yi syllables (U+A000 to U+A48C), and comprises 1,164 syllables (syllables with a diacritic mark are encoded separately, and are not decomposable into syllable plus combining diacritical mark) and one syllable iteration mark (U+A015, incorrectly named YI SYLLABLE WU). In addition, a set of 55 radicals for use in dictionary classification are encoded at U+A490 to U+A4C6 (Yi Radicals).[11] Yi syllables and Yi radicals were added as new blocks to Unicode Standard Version 3.0.[12]

Classical Yi - which is an ideographic script like the Chinese characters - has not yet been encoded in Unicode, but a proposal to encode 88,613 Classical Yi characters was made in 2007.[13]

\bgroup
\yi \char"A000: Yi Syllable It\\

\yi \char"A001: Yi Syllable Ix\\

\yi \char"A002: Yi Syllable I\\
\egroup

\begin{scriptexample}[]{Yi}
\unicodetable{yi}{"A000,"A010,"A020,"A030,"A040,"A050,"A060,"A070,"A080,"A090,"A0A0,"A0B0,"A0C0}
\end{scriptexample}



%\cxset{steward,
  chapter format   = stewart,
  chapter numbering=arabic,
  offsety=0cm,
  image={fellah-woman.jpg},
  texti={An introduction to the use of font related commands. The chapter also gives a historical background to font selection using \tex and \latex. },
  textii={In this chapter we discuss keys that are available through the \texttt{phd} package and give a background as to how fonts are used
in \latex.
 },
}

\pagestyle{headings}
%\pgfpagesuselayout{2 on 1}[a3paper,landscape,border shrink=0mm]

\chapter{Ancient and Historic Scripts}

\epigraph{Now that all the old women have died, grandmas, great grandmas and other
assorted persons of that ilk, they have managed to engender within me a heap of
profound perplexities about persons and things old and extiguished forever. As long
as they were alive, I don't know why, I practically never wanted to ask}{---\textit{Beginning of Ioannou 1974.}}

Writing was perhaps the most important human invention. \tex authors and developers either due to need or fascination developed macros and fonts for many archaic writing systems. Many of these packages are now outdated, as the Unicode standard and the newer engines opened up a fascinating world. My own fascination with writing systems prompted me to add support for such scripts in the \pkgname{phd} package. The development to an extend was frustrating as the overloading of numerous fonts caused compilation to be very slow. Finding the right font was also problematic in many cases, as we opted to identify Open Source fonts. The \tex engine of preference is \luatex. To avoid loading too many fonts, unless they are required, we provide the keys:

\def\loadscripts{}
\cxset{scripts/.store in = \loadscripts}

\begin{docKey}[phd]{scripts}{ = \meta{all, lineara, linearb, phaestos,\ldots}} {default none, initial=none}
 The scripts key takes a list of options to enable or disable the loading of fonts and the usage of the key is explained later on. You set it with our only command |\cxset|\meta{key value list}
\end{docKey}

Supplementary keys, exist for each individual script enabling the setting of specific fonts to a particular script. However, if all the recommended fonts have been installed is quicker to use the |scripts| key.

\def\olmecfontstore{}

\cxset{olmec font/.store in=\olmecfontstore}

\cxset{olmec font=epiolmec}

\begin{docKey}[phd]{olmec font}{ = \meta{font name}} {default none, initial=none}
\end{docKey}

The key |script| can be used on its own. It will then load the default fonts built-in, in the |phd| package.

\cxset{script/.store in = \scripttempt}

\begin{docKey}[phd]{script =}{ \meta{script name}}{}
\end{docKey}

\begin{figure}[b]
\centering
\includegraphics[width=0.6\textwidth]{./images/rongo.jpg}
\caption{Rongo rongo writing. Tablet B Aruku kurenga, verso. One of four texts which provided the Jaussen list, the first attempt at decipherment. Made of Pacific rosewood, mid-nineteenth century, Easter Island.
(Collection of the SS.CC., Rome)}

\end{figure}

The first attempt  at communication via writing was through ideographic or mnemonic symbols. Undoubtedly symbolic writing must have existed much earlier than the surviving artifacts, carved in woord or scribled on muddy walls. The earliest surviving symbolic writing are the Jiahu symbols. They were carved on tortoise shells in Jiahu, ca.~6600~BC. Jiahu was a neolithic Peligang culture site found in Henan, China. In Europe the Tărtăria tablets are three tablets, discovered in 1961 by archaeologist Nicolae Vlassa at a Neolithic site in the village of Tărtăria (about 30 km (19 mi) from Alba Iulia), in Romania.[1] The tablets, dated to around 5300 BC,[2] bear incised symbols - the Vinča symbols - and have been the subject of considerable controversy among archaeologists, some of whom claim that the symbols represent the earliest known form of writing in the world. The Indus script appeared ca. 3500 BC and the Nsibidi script of Nigeria, ca. before 500 AD. 

The history of the world was carved in stone, bones, wood. Later they became manuscripts, typographed in books and now locked in os and 1s. To understand the history of the world one needs to understand the scripts. Ideas and society are locked in these scripts. One of the scripts of China is Nüshu, which was exclusively used by women.  During the Cultural Revolution of the 1960s, the Red Guard prosecuted women who used Nushu and burned Nushu books and artifacts. More recently, China has made an effort to preserve Nushu, training five women to be “Nushu transmitters” and teach classes in the Nushu script, but the last woman proficient in Nushu died in September, 2004, at the age of 98.

No type of writing system is superior or inferior to another, as the type is often dependent on the language they represent. For example, the syllabary works perfectly fine in Japanese because it can reproduce all Japanese words, but it wouldn't work with English because the English language has a lot of consonant clusters that a syllabary will have trouble to spell out. The pretense that the alphabet is more \enquote{efficient} is also flawed. Yes, the number of letters is smaller, but when you read a sentence in English, do you really spell individual letters to form a word? The answer is no. You scan the entire word as if it is a logogram.

And finally, writing system is not a marker of civilization. There are many major urban cultures in the world did not employ writing such as the Andean cultures (Moche, Chimu, Inca, etc), but that didn't prevent them from building impressive states and empires whose complexity rivals those in the Old World. For us to preserve the remnants of their civilization we need to write it down in a romanized version.

Unicode encodes a number of ancient scripts, which have not been in normal use for a millennium or more, as well as historic scripts, whose usage ended in recent centuries. Although these scripts are no longer used to write living languages, documents and inscriptions using these languages exist, both for extinct languages and for precursors of modern languages. The primary user communities for these scripts are scholars, interested in studying the scripts and the languages written in them. A few, such as Coptic, also have contemporary liturgical or other special purposes. Some of the historic scripts are related to each other as well as to modern alphabets. The following are provides as of Unicode version~7.2.
\index{Ancient and Historic Scripts>Ogham}
\index{Ancient and Historic Scripts>Old Italic}
\index{Ancient and Historic Scripts>Runic}
\index{Ancient and Historic Scripts>Gothic}
\index{Ancient and Historic Scripts>Akkadian}
\index{Ancient and Historic Scripts>Old Turkic}
\index{Ancient and Historic Scripts>Hieroglyphs}
\index{Ancient and Historic Scripts>Linear B}
\index{Ancient and Historic Scripts>Linear A}
\index{Ancient and Historic Scripts>Phoenician}
\index{Ancient and Historic Scripts>Old South Arabian}
\index{Ancient and Historic Scripts>Mandaic}
\index{Ancient and Historic Scripts>Avestan}
\index{Ancient Anatolian Alphabets}
\index{Old South Arabian}
\index{Phoenician}
\index{Imperial Aramaic}
\begin{center}
\begin{tabular}{lll}
\ref{s:ogham}           
&\ref{s:anatolian}
&\ref{s:avestan}\\

Old Italic \ref{s:olditalic}      
&Old South Arabian \S\ref{s:oldsoutharabian}          
&Ugaritic \S\ref{s:ugaritic} \\
Runic \S\ref{s:runic}
&Phoenician\S\ref{s:phoenician} 
&Old Persian \ref{s:oldpersian} \\
 \S\ref{s:gothic}            
    
&\nameref{s:imperialaramaic}            
&\nameref{s:sumero} \\

    \nameref{s:oldturkic}   
& \nameref{s:mandaic} 
& \nameref{ch:hieroglyphics}\\

 \nameref{s:linearb} 
&\nameref{s:parthian}       
&\nameref{s:meroitic}\\

 Cypriote (\pageref{s:cypriot})
&Inscriptional Pahlavi\S\ \ref{s:inscriptionalpahlavi}       
&Linear A \S\ \ref{s:lineara}\\
\end{tabular}
\end{center}

The following scripts are also encoded but following the Unicode
convention are described in other sections

\begin{center}
\begin{tabular}{llllll}
Coptic &Glagolitic \S\ref{s:glagolitic} &Phags-pa. &Kaithi &Kharoshi &Brahmi \S\ref{s:brahmi}.\\
\end{tabular}
\end{center}

Some scripts such as Olmec \S\ref{s:olmec} are not described in the Unicode standard, but we provide support for them.

%


\chapter{Aegean}

We are fortunate that the frescoes that survived on the walls of the palaces and building at Knossos and other places give us a view of the material culture of the Minoans. These have been studied extensively. Cameron's\footcite{Cameron1976} thesis reviewed the development of the murals beginning with  Neolithic architectural muse of mud plasters, the first painted plasters occuring in EM II settlements and simple decorative schemes in the First Palaces (1900-1700 B.C.). The sudden rise of pictorial naturalism in MM IIIA is explained by native cultural developments of the First Palace Period, not by foreign influences or \enquote{eideticism}  which he rejects convincingly altogether. 
\begin{figure}[htbp]
\centering
\includegraphics[width=0.7\textwidth]{crete-safron-gatherer}
\caption{Painted limestone head from Myccnre. Mainland type with
tattooing}
\end{figure}
A review of the motival repertory leads to considerations of six main \enquote{cycles of ideas} whence the painters derived their themes. The most important, confined to Knossos palace, depicts a major festival of grand processions and athletic activities before the chief Minoan goddess, and it illuminates the palatial architechural design. But five different systems of mural decoration characterise Minoan architecture as a whole, with regional and perhaps autonomous variations at Cycladic sites. Technical considerations confirm \enquote{buon fresco} as the normal painting technique and distinguish Knossian town house and palace murals in construction and purpose. Similar distinctions in compositional design are also described. A review of eleven \enquote{schools} of Knossian painters and of regional artists precedes a detailed reconstruction of the dates of the frescoes on stratigraphical, stylistic and comparative evidence. Cameron suggests that Sir Arthur Evans's fresco dating should generally be lowered by one Minoan phase. Minoan pictorial painting ceases with the palace destruction at Knossos, c.1375 B.C. Major differences appear between pre- and post-LM IB frescoes, tentatively explained on the evidence of Aegean and Egyption pictorial representations by the arrival at Knossos of a Mycenaean military dynasty, c.1450 B.C. Minoan wall painting finally disappeared in the LM IIIB period.

\begin{figure}[htbp]
\centering
\includegraphics[width=0.8\textwidth]{crete-tatoo}
\caption{Painted limestone head from Myccnre. Mainland type with
tattooing}
\end{figure}

Many early societies used tatoos to decorate the body. Glotz\footcite{Glotz2003} thought the some women had tatoos on their body and faces and offered the example of the . Later this form of painting disappeared. The Cretan woman was always depicted elegantly. Tattooing, which is universally practised by primitive
communities, and is perpetuated by the primitive menlbers
of advanced communities, was known among all the Neolithic
populations of the \Ae gean. Crete was no exception. At
Phaistos a figurine of a steatopygous woman is marked with
a little cross on one side. In the metal age the custom
survived in the Cyclades and in Argolis, where faces are often
marked with horizontal lines of red dots, with vertical or
oblique lines, or with circles of dots around a central point.

\begin{figure}[htbp]
\centering
\includegraphics[width=\textwidth]{crete-evans}
\caption{Painted limestone head from Myccnre. Mainland type with
tattooing}
\end{figure}


In the tombs, within reach of the dead, were
placed the tools and vessels required for this ritual operation,
such as needles or awls, vases containing red or blue pigment,
and palettes. But in Crete there is no trace of tattooing
after the Stone Age. Certain small vases, discovered in
Cretan tombs and houses, and mistaken for paint pots, were
really receptacles for offerings.1 The very most that may
be supposed is that the Cretans placed a stigma upon the faces
of their slaves . This would explain the fact that the CupBearer,
a brachycephalic type, has a blue mark carefully
painted on his temple (Fig. 53) , though it is probable that
he is a foreigner bringing tribute. This haste to get rid of
marks disfiguring the face is an early sign of the aesthetic
sense.

\begin{figure}[htbp]
\centering
\includegraphics[width=0.5\textwidth]{crete-parisienne}
\caption{A fragment of the Campstool Fresco showing a woman at a banquet. Knossos, Final-Palatial Period (1400-1350 BCE). In early descriptions, the image is referred to as the "Parisienne", as the Archaeologists thought it bore resemblance to 19th century Parisienne women}
\end{figure}

The scripts during the Aegean Bronze Age form and independent group distinct from the
contemporary Egyptian and Babylonian systems.\footcite{packard1974}\textsuperscript{,}\footcite{booktabs} Four major branches are attested during the
second millenium: the Cretan Hieroglyphic script, Linear A, Linear B, and Cypro-Minoan.  By the end of the Broze Age this family of scripts had fallen out of use, and Greece remained illiterate until the introduction of the Phoeneician script centuries later. A single descendant of the earlier Aegean scripts survived in the first millenium on the island of Cyprus in the form of the Cypriot Syllabary.\footnote{test}\footnote{testing} \label{s:lineara}\index{scripts>Linear A}

\begin{figure}[htbp]
\centering
\includegraphics[width=0.50\textwidth]{iraklion-poppy-goddess}
\caption{Poppy Godess}
\end{figure}

The Greeks had evidently already occupied the mainland and islands of the
Ægean, including Crete, by the middle of the third millennium
BC. Around 2000 BC, following their consolidation of power on
Crete, new wealth from trade with cosmopolitan Canaan
allowed the creation of a complex palace economy, with major
centres at Knossos, Phaistos and other Cretan sites – Europe’s
first high civilization, the Minoan. Trade with Canaan had evidently
also brought Greeks into contact with Byblos’ pictorial
syllabic writing, whose underlying principle the Minoans borrowed.
Now, Cretans could also write their Minoan Greek language
using a small corpus of syllabo-logographic signs
representing \textit{in-di-vi-du-al} syllables. The signs themselves and
their phonetic values – nearly all V (e) or CV (te) – were wholly
indigenous: what the rebus signs, all originating from the
Cretan world, depicted, one pronounced in Minoan Greek, not
in a Semitic language. (Minoan Greek appears to have been an
archaic sister tongue of the mainland’s Mycenæan Greek.\footnote{A History of Writing. })

Three separate but related forms of syllabo-logographic
writing emerged in the Ægean between c. 2000 and 1200 BC: the
Minoan Greeks’ ‘hieroglyphic’ script and Linear A, and the
later Mycenæan Greeks’ Linear B. Minoan Greeks apparently
also took their writing at an early date to Cyprus, where it experienced
two stages: Cypro-Minoan (evidently derived from
Linear A is one of two currently undeciphered writing systems used in ancient Greece. Cretan hieroglyphic is the other. Linear A was the primary script used in palace and religious writings of the Minoan civilization. It was discovered by archaeologist Arthur Evans. It is the origin of the Linear B script, which was later used by the Mycenaean civilization.

Linear A and its daughter Linear C, the ‘Cypriote Syllabic
Script’. All Ægean and Cypriote scripts are clearly syllabologographic,
as the objective identity of each rebus sign would
have been immediately recognizable to each learner and user. It
seems that determinatives were never employed in any of the
Ægean or Cypriote scripts; however, logograms additionally
depicted most spelt-out items on accounting tablets. All Ægean
and Cypriote scripts, but for these separate logograms, were
completely phonetic.
\medskip

\subsection{Cretan Hieroglyphic}

\begin{figure}[htbp]
\centering

\includegraphics[width=0.8\textwidth]{./images/cretan-hieroglyphs.png}

\end{figure}


Crete’s \enquote{hieroglyphic} script is the patriarch of this robust
family, its inspiration perhaps derived from Byblos via Cyprus.\footfullcite{packard1974}
As its name implies, this script used pictorial signs to reproduce the syllabic inventory of the Minoan
Greek language, here used in rebus fashion as at Byblos. This
writing occurs on seal stones (and their clay impressions), baked
clay, and metal and stone objects, most of these discovered at
Knossos and dating from 2000– 1400 BC (the script was concurrent
with Linear A). There exist about 140 different signs in all –
that is, 70 to 80 syllabic signs and their alloglyphs (different signs
with the same sound value), as well as logograms: human figures,
parts of the body, flora, fauna, boats and geometrical shapes.
Writing direction was open: from left to right, from right to left,
with every other line reversed, even spiral. That this script also
included logograms and numerals suggests that it was initially
used for book-keeping, among other things, until its replacement
in this function with its simplification, Linear A. Thereafter, like
Anatolian hieroglyphs, the Cretan hieroglyphic script appears to
have assumed a ceremonial role in Minoan Greek society,
reserved for sacred inscriptions, dedications and royal proclamations
on round clay disks.



In the 1950s, Linear B was largely undeciphered and found to encode an early form of Greek. Although the two systems share many symbols, this did not lead to a subsequent decipherment of Linear A. Using the values associated with Linear B in Linear A mainly produces unintelligible words. If it uses the same or similar syllabic values as Linear B, then its underlying language appears unrelated to any known language. This has been dubbed the Minoan language.\footnote{\url{http://www.people.ku.edu/~jyounger/LinearA/LinAIdeograms/}}

\begin{scriptexample}[]{Linear A}
\unicodetable{lineara}{  
\number"10600,"10610,"10620,"10630,"10640,"10650,"10660,"10670,
"10680,"10690,"106A0,"106B0,"106C0,"106D0,"106E0,"106F0,"10710,"10720,"10730,"10740,"10750,"10760,"10770}
\end{scriptexample}

Many of the characters form group and specialists name them such as vases in transliterations.

\begin{scriptexample}[]{Vases}
\begin{center}
\scalebox{3}{{\lineara \char"106A6}}
\scalebox{3}{{\lineara \char"106A5}}
\scalebox{3}{{\lineara \char"106A7}}
\scalebox{3}{{\lineara \char"106A9}}
\end{center}
\end{scriptexample}

Linear A contains more than 90 signs (open vowels and consonants+vowels) in regular use and a host of
logograms, many of which are ligatured with syllabograms and/or fractions; about 80\% of these
logograms do not appear in Linear B. While many of Linear A’s signs are also found in Linear B, some
signs are unique to A (e.g., A *301 and following), while some signs found in Linear B are not yet found
in Linear A (e.g., B 12, 14-15, 18-19, 25, 32-33, 36, 42-43, 52, 62-64, 68, 71-72, 75, 83-84, 89-91).

The Unicode Linear A encoding is broadly based on the GORILA ([{\arial ɡɔɹɪˈlɑː}]) catalogue
(Godart and Olivier 1976–1985)\parencite{gorila}, which is the basic set of characters used in decipherment efforts.However, “ligatures” which consist of simple horizontal juxtapositions are not uniquely encoded here, as
these may be composed of their constituent parts. On the other hand, “ligatures” which consist of stacked
or touching elements have been encoded.\footnote{An online resource for ancient writing systems in the mediterranean\protect\url{http://lila.sns.it/mnamon/index.php?page=Risorse&id=19&lang=en}. } 








%\newfontfamily\linearb{Aegean.ttf}
\section{Linear B}
\label{s:linearb}
\index{scripts>Linear B}
The Linear B script is a syllabic writing system that was used on the island of Crete and
parts of the nearby mainland to write the oldest recorded variety of the Greek language.

Linear B clay tablets predate Homeric Greek by some 700 years; the latest tablets date from
the mid- to late thirteenth century \bce. Major archaeological sites include Knossos, first
uncovered about 1900 by Sir Arthur Evans, and a major site near Pylos. The majority of
currently known inscriptions are inventories of commodities and accounting records.

The first tablets bearing the scripts were discovered by Sir Arthur Evans (1851-1941) while he was excavating the Minoan palace at Knossos in Crete. 


\medskip

\begin{figure}[ht]
\centering
\begin{minipage}{5cm}
\includegraphics[width=5cm]{./images/iklaina.jpg}
\end{minipage}\hspace{2em}
\begin{minipage}{7cm}
\captionof{figure}{Recently discovered fragment with Linear B, inscription. Found in an olive grove in what's now the village of Iklaina, the tablet was created by a Greek-speaking Mycenaean scribe between 1450 and 1350 B.C. (See \protect\href{http://news.nationalgeographic.com/news/2011/03/110330-oldest-writing-europe-tablet-greece-science-mycenae-greek/}{National Geographic}).}
\end{minipage}

\end{figure}


Early attempts to decipher the script failed until Michael Ventris, an architect and amateur
decipherer, came to the realization that the language might be Greek and not, as previously
thought, a completely unknown language. Ventris worked together with John Chadwick,
and decipherment proceeded quickly. The two published a joint paper in 1953. See \fullcite{ventrisa}.




Linear B was added to the Unicode Standard in April, 2003 with the release of version 4.0.

The Linear B Syllabary block is \unicodenumber{U+10000–U+1007F}. The Linear B Ideograms block is {\smallcps U+10080–U+100FF}. The Unicode block for the related Aegean Numbers is U+10100–U+1013F.

\begin{scriptexample}[]{Linear B}
\unicodetable{linearb}{"10000,"10010,"10020,"10030,"10040,"10050,"10060,"10070}

\captionof{table}{Linear B Typeset with command \protect\string\linearb\ and the \texttt{Aegean} font.}
\end{scriptexample}

\begin{scriptexample}[]{Linear B}
\unicodetable{linearb}{"10080,"10090,"100A0,"100B0,"100C0,"100D0,"100E0,"100F0}
\captionof{table}{Linear B Ideograms. Typeset with command \protect\string\linearb\ and the \texttt{Aegean} font.}
\end{scriptexample}


\begin{scriptexample}[]{Aegean Numbers}
\unicodetable{linearb}{"10100,"10110,"10110,"10120,"10130}

\captionof{table}{Aegean Numbers}
\end{scriptexample}





\section{Phaestos Disc}


One of the puzzles of Minoan Crete is the Phaestos disc. The Phaistos Disc was discovered in the Minoan palace-site of Phaistos, near Hagia Triada, on the south coast of Crete;[1] specifically the disc was found in the basement of room 8 in building 101 of a group of buildings to the northeast of the main palace. This grouping of four rooms also served as a formal entry into the palace complex. Italian archaeologist Luigi Pernier recovered the intact \enquote{dish}, about 15 cm (5.9 in) in diameter and uniformly slightly more than 1 centimetre (0.39 inches) in thickness, on 3 July 1908 during his excavation of the first Minoan palace.

It was found in the main cell of an underground \enquote{temple depository}. These basement cells, only accessible from above, were neatly covered with a layer of fine plaster. Their content was poor in precious artifacts, but rich in black earth and ashes, mixed with burnt bovine bones. In the northern part of the main cell, in the same black layer, a few inches south-east of the disc and about 20 inches (51 centimetres) above the floor, Linear A tablet PH 1 was also found. The site apparently collapsed as a result of an earthquake, possibly linked with the eruption of the Santorini volcano that affected large parts of the Mediterranean region during the mid second millennium B.C.

\begin{figure}[htp]
\centering

\includegraphics[width=0.67\textwidth]{./phaistosdiscs.jpg}
\caption{Phaistos discs.}
\end{figure}

The Phaistos Disc is generally accepted as authentic by archaeologists.[2] The assumption of authenticity is based on the excavation records by Luigi Pernier. This assumption is supported by the later discovery of the Arkalochori Axe with similar but not identical glyphs.[3]


The possibility that the disc is a 1908 forgery or hoax has been raised by two scholars.[4][5][6] In his 2008 review, Robinson does not endorse the forgery arguments, but argues that \enquote{a thermoluminescence test for the Phaistos Disc is imperative. It will either confirm that new finds are worth hunting for, or it will stop scholars from wasting their effort.}[4]

A gold signet ring from Knossos (the Mavro Spilio ring), found in 1926, contains a Linear A inscription developed in a field defined by a spiral—similar to the Phaistos Disc.\footnote{See University of Cologne website \url{http://arachne.uni-koeln.de/arachne/index.php?view[layout]=objekt_item\&search[constraints][objekt][searchSeriennummer]=159123}} A sealing found in 1955 shows the only known parallel to sign 21 (the \enquote{comb}) of the Phaistos disc.[9] This is considered as evidence that the Phaistos Disc is a genuine Minoan artifact.[10]

\begin{figure}[htbp]
\centering

\includegraphics[width=4.5cm]{crete-spiral-ring}\includegraphics[width=4.5cm]{crete-spiral-ring-01}\includegraphics[width=4.5cm]{crete-spiral-ring-02}

\caption{A gold signet ring from Knossos (the Mavro Spilio ring), found in 1926, contains a Linear A inscription developed in a field defined by a spiral—similar to the Phaistos Disc}
\end{figure}

The disc is made of fine clay.  Both side of the disc carry an inscription arranged in a spiral around the centre. The characters were impressed with a punch or stamp before the clay was fired. There are
241 or 242 characters (one is damaged), which
comprise 45 signs of variable frequency. For
comparison, there are thousands of characters in a few pages of printed English text, comprising the 26 signs we call letters. Lines partition
the disc’s characters into 31 short sections on
side A and 30 on side B, most of which contain
three, four or five characters. It is tempting to
speculate that these sections represent words
in the language of the disc.

That the characters were printed, not carved,
is beyond dispute. But no one knows why the disc’s maker bothered to produce a punch or stamp for each sign, rather than inscribing each character afresh. Egyptian hieroglyphs or Mesopotamian cuneiform of the second
millennium bce are inscribed on stone or clay;
simlarly the Minoan scripts Linear A and B found
at Phaistos, Knossos and other Cretan sites. If
the punch or stamp was to \enquote{print} many copies of documents, one would expect further sam-
ples to have turned up in a century of intensive Mediterranean excavatio

There is patchy and inconclusive evidence for and against the disc’s Cretan origin. The
signs look nothing like those of Linear A, Linear B or any other Minoan script, except coincidentally. This has led some, including Evans and Chadwick, to propose that the disc — and presumably its language, too — was an import.

One sign bears a remarkable resemblance to the architecture of rock tombs found in Anatolia in modern Turkey. One or two others
resemble signs found on a few contemporaneous objects from different sites in Crete. Most
scholars today, including Duhoux, think it a plausible working hypothesis that the disc was made in Crete. Gareth Owens and his Team claim to have read the disc and you can hear how it sounded at a TED Talk\footnote{\url{https://www.youtube.com/watch?v=6Chcplx3tZ8}}.



\subsection{Signs}

There are 242 tokens on the disc, comprising 45 distinct signs. Many of these 45 signs represent easily identifiable every-day things. In addition to these, there is a small diagonal line that occurs underneath the final sign in a group a total of 18 times. The disc shows traces of corrections made by the scribe in several places. The 45 symbols were numbered by Arthur Evans from 01 to 45, and this numbering has become the conventional reference used by most researchers. Some symbols have been compared with Linear A characters by Nahm,[17] Timm,[3] and others. Other scholars (J. Best, S. Davis) have pointed to similar resemblances with the Anatolian hieroglyphs, or with Egyptian hieroglyphs (A. Cuny). In the table below, the character "names" as given by Louis Godart (1995) are given in upper case; where other description or elaboration applies, they are given in lower case.




\PrintUnicodeBlock{./languages/phaistos.txt}{\linearb}




The ideograms are symbols, not pictures of the objects in question, e.g. one tablet records a tripod with missing legs, but the ideogram used is of a tripod with three legs. In modern transcriptions of Linear B tablets, it is typically convenient to represent an ideogram by its Latin or English name or by an abbreviation of the Latin name. Ventris and Chadwick generally used English; Bennett, Latin. Neither the English nor the Latin can be relied upon as an accurate name of the object; in fact, the identification of some of the more obscure objects is a matter of exegesis.

\begingroup

\linearb

Vessels
\let\l\unicodenumber

\begin{tabular}{l>{\smallcps}l>{\smallcps}l>{\smallcps}l>{\smallcps}l}
𐃟	&U+100DF	&200	&\l{sartāgo}	&\l{Boiling Pan}\\
𐃠	&U+100E0	&201	&\l{tripūs}	&\l{Tripod Cauldron}\\
𐃡	&U+100E1	&202	&\l{pōculum}	&\l{Goblet}\\
𐃢	&U+100E2	&203	&\l{urceus}	&\l{Wine Jar?}\\
𐃣	&U+100E3	&204  &\l{Tahirnea}	&\l{Ewer}\\
𐃤	&U+100E4	&205  &\l{Tnhirnula}	&\l{Jug}\\
𐃥	&U+100E5	&206	&\l{hydria}	&Hydria\\
𐃦	&U+100E6	&207	&\l{TRIPOD}  &AMPHORA\\
𐃧	&\l{U+100E7}	&\l{208}	&\l{PAT patera}	&\l{BOWL}\\
𐃨	&U+100E8	&209	&AMPH amphora	&AMPHORA\\
𐃩	&U+100E9	&210	&STIRRIP &JAR\\
𐃪	&U+100EA	&211	&WATER &BOWL?\\
𐃫	&U+100EB	&212	&SIT situla	&WATER JAR?\\
𐃬	&U+100EC	&213	&LANX lanx	&COOKING BOWL\\
\end{tabular}




\subsection{Online Resources}

Corpora and GORILA \url{http://www.people.ku.edu/~jyounger/LinearA/\#3}



\endgroup











\newfontfamily\cminoan[Scale=1.5]{Cminoan.ttf}

\newfontfamily\cypriote{Aegean.ttf}

\chapter{Cypriot Syllabary}
\label{s:cypriot}

\epigraph{History in this island is almost too profuse.}
{Robert Byron, \textit{The Road to Oxiana} (London, 1937; repr. 1950), 22}
\section{Introduction}

One crucial feature of the Cypro-Archaic period was a continuity of styles, indicating a period of calm without discontinuities caused by hostile invasions.\footcite[page 4, \ldots different foreign powers were perhaps less hostile than is
generally thought. The crucial feature of the Cypro-Archaic period was
not the series of disruptions apparently resulting from invasion, but the
continuity of styles evident in the archaeology.]{Reyes1994} 


Cyprus, an island comprising 9,251 sq. km. or about 3,572 sq. miles,\footcite{Reyes1994} had at least ten separate
kingdoms by the second quarter of the seventh century, according to a prism inscription of the Assyrian king Esarhaddon, dated to 673/2 BC. On that count, the average size of a kingdom would have been 925 sq.km. or 357 sq. miles at most, 'larger than any contemporary Greek polis of the homeland, apart from the two outsize instances of Sparta and
Athens, larger, therefore, than the hundreds of smaller poleis of historical Greece'.20 The extent to which one kingdom interacted with another requires examination, then, since previous studies have simply concentrated on the effects of foreign dominations on the local culture.\footnote{one}\footnote{two} 

\citeauthor{Reyes1994} considered briefly the cultural and social framework of the island, and examined the textual and archaeological sources for Cypriot history from the eighth to the sixth centuries BC, he also investigated the internal and external relations between the different parts of the island.

\section{Cypro-Minoan}

The Cypro-Minoan syllabary (CM) is an undeciphered syllabary used on the island of Cyprus during the late Bronze Age (ca. 1550–1050 BC). The term "Cypro-Minoan" was coined by Arthur Evans in 1909 based on its visual similarity to Linear A on Minoan Crete, from which CM is thought to be derived.[1] Approximately 250 objects—such as clay balls, cylinders, and tablets and votive stands—which bear Cypro-Minoan inscriptions, have been found. Discoveries have been made at various sites around Cyprus, as well as in the ancient city of Ugarit on the Syrian coast.



Before we delve further into the texts and attempts at decipherment, it is prudent to review the conventions and terminology, used in the literature. First we need to understand references to the corpora of epigraphs. These are normally referenced as ENKO Atab 001. The first is an abbreviation for an area, ENKO is for Enkomi, Atab is an abbreviation for the type of object, i.e., ring, stone etc and the number is just an arbitrary number.\footfullcite[][page 27-30]{valerio2016}

\begin{alltt}\cminoan
{\arial ENKO Atab 001} 
󱀀󱀁󱀂󱀃󱀄󱀅󱀆󱀈
 󱀊󱀋󱀌󱀍󱀎󱀏󱀐
󱀑󱀒󱀓󱀔󱀕󱀖
󱀇󱀉
\end{alltt}

\subsection{Place}

\begin{longtable}[l]{>{\ttfamily}ll}
ALAS &Alassa-Palaeotaverna (Limassol)\\
ARPE &Arpera (Larnaca)\\
ATHI &Athienou (Larnaca)\\
CYPR &Cyprus\\
DHEN &Dhenia, or Deneia (Nicosia)\\
ENKO &Enkomi (Famagusta)\\
HALA &Hala Sultan Tekke (Larnaca)\\
IDAL &Idalion (Nicosia)\\
KALA &Kalavasos-Ayios Dhimitrios (Larnaca)\\
KATY &Katydhata (Nicosia)\\
KLAV &Klavdia (Larnaca)\\
KITI &Kition (Larnaca)\\
KOUR &Kourion (Limassol)\\
MAAP &Maa-Palaeokastro (Paphos)\\
MARO &Maroni (Larnaca)\\
MYRT &Myrtou-Pigadhes (Kyrenia)\\
PARA &Ayia Paraskevi (Nicosia)\\
PPAP &Palaeopaphos-Skales (Paphos)\\
PSIL &Psilatos (Famagusta)\\
PYLA &Pyla-Verghi (Famagusta)\\
RASH &Ras Shamra / Ugarit (Syria)\\
SALA &Salamis (Famagusta)\\
SANI &Sanidha (Limassol)\\
SYRI &Syria\\
TIRY &Tiryns (Greece)\\
TOUM &Toumba tou Skourou (Nicosia)\\
\end{longtable}

\subsection{Typological Description}


\begin{longtable}[l]{>{\ttfamily}ll}
Abou & Clay ball (or boule)\\
Adis & Clay disk\\
Aéti & Clay label\\
Aost & Clay ostracon\\
Apes & Clay weight\\
Apla & Clay plaque\\
Arou & Clay cylinder\\
Asta & Clay figurine\\
Atab & Clay tablet\\
Avas & Pottery (complete or fragmentary)\\
Inst & Ivory tool\\
Ipla & Ivory plaque\\
Mbij & Metal jewelry\\
Mexv & Metal ex-voto\\
Mins & Metal tool\\
Mlin & Metal ingot\\
Mvas & Metal base\\
Pblo & Stone block\\
Pfus & Stone spindle whorl\\
Ppla & Stone plaque\\
Psce & Stone seal\\
Vsce & Glass seal\\
\end{longtable}


Since the publication of HoChyMin, it seems to have become the common
practice to cite the inscriptions by absolute sequential number. In this book, however, following I have 
opted to refer to them by label. While this may seem less economical in terms of space,
labels are more informative and probably easier to associate to the actual inscriptions.
Hopefully, this will become clear in Chapter 3, where they are useful to infer the
epigraphical support of particular paleographical variants in the inter-script comparative
tables.

One potential shortcoming of the system, noted by Olivier himself, is that the
typological descriptions are in a certain measure arbitrary and empirical.17 For example,
in the case of |ENKO Apes 001| the name implies that the object inscribed is a weight
("pes"), but this interpretation is debated and it has also been proposed that the support
was actually used as a label.18 Still, such cases are a minority and will be duly signaled.
To facilitate the reading of this dissertation and comparisons with other publications, the
correspondence between sequential numbers and labels is given in the Concordance that
precedes this Introduction.

For reasons of economy, in the transnumeration of sign-sequences I use a system
slightly different from the one found in HoChyMin and now followed on publications
on Cypro-Minoan. Signs whose numbers are below 100 are transnumerated with only
two digits. For example, the sign-group 102-009-082-085 is here given as 102-09-82-85. 
In tentative transliterations, untransliterated signs are kept in transnumeration but
preceded by an asterisk (*): e.g. i?-li?-*71-ni?.

\section{History of Scholarship}

The CM script was studied in 1900 when Arthur Evans analyses three clay \textit{boules}
found during the British excavations in 1896 and the gold ring from Hala Sultan Tekke (HST 3).\footnote{Evans (1909):71, fig.39 (table 3): he matches 15 out of the 15 signs with alleged parallels to Linear A and B; those signs that do not show conformity with the signaries are paralleled with the Cretan Hieroglyphic script.}

In 1935 Evan's reinforced the palaeographical connections between the Aegean scripts and Cypro-Minoan, adding the
evidence provided by a further clay \textit{boule} ENK. 3 the Ayia Paraskevi cylinder seal, and of ceramic fragment Enkomi A 1507 (ENK. 105)


\section{The Signary}



\section{Cypriote syllabary}

The Cypriot or Cypriote syllabary is a syllabic script used in Iron Age Cyprus, from ca. the 11th to the 4th centuries BCE, when it was replaced by the Greek alphabet. A pioneer of that change was king Evagoras of Salamis. It is descended from the Cypro-Minoan syllabary, in turn a variant or derivative of Linear A. Most texts using the script are in the Arcadocypriot dialect of Greek, but some bilingual (Greek and Eteocypriot) inscriptions were found in Amathus.\footfullcite{powell1991}\tcbdocmarginnote{rev. after china}

\begin{figure}[htbp]
\includegraphics[width=\textwidth]{Tablet_cypro-minoan_2_Louvre_AM2336}
\end{figure}

The existence of a local Cypriote script was first demonstrated in
1852 by the collector and antiquarian, the Due de Luynes, on the basis of
some inscribed coins and a few other inscriptions.39 The Assyriologist
George Smith offered the key to decipherment in 1871, though he
remained reluctant, because of the writing's oddity when compared with
Greek alphabetic writing, to conclude that the underlying language was
Greek. By 1875, through the efforts of philologists in several countries, the
decipherment was substantially complete, and the language of most of the
inscriptions was proved to be written in what is now called the Arcado-
Cypriote dialect of Greek. Many later finds allow one to make the
following general description of Cypriote writing.\footcite{powell1991}

From c. 1600 to 1050 B.C. an undeciphered writing similar in form to the
classical Cypriote syllabary was in use on Cyprus and in Ras Shamra in
North Syria. Sir Arthur Evans aptly called this script ``Cypro-Minoan'' by
reason of its formal affinities with Linear A and Β and with the classical
Cypriote writing;40 the term is now standard. Formal similarities make it
probable that Cypro-Minoan is derived from Cretan writing, but their
exact relation cannot be determined. Most will agree that Cypro-Minoan
records pre-Greek languages spoken on Cyprus.

\begin{figure}[htb]
\parindent0pt
\centering

\includegraphics[width=\textwidth]{./images/idalion-tablet.jpg}

\caption[Idalion tablet.]{The bronze Idalion Tablet, from Idalium, (Greek: Ιδάλιον), is from the 5th century BCE Cyprus. The tablet is inscribed on both sides. The script of the tablet is in the Cypro-Minoan syllabary, and the inscription is in Greek. The tablet records a contract between "the king and the city":[1] the topic of the tablet rewards a family of physicians, of the city, for providing free health services to individuals fighting an invading force of Persians.}
\label{fig:idalion}
\end{figure}


The oldest dated inscriptions in the classical Cypriote syllabary are from
the eighth century B.C., very close to the date of the invention of the
alphabet. We are thus left with a troubling hiatus of 300 years between the
latest attestation of Cypro-Minoan writing and the first of classical
Cypriote writing.42 Nonetheless the Cypriote syllabary is doubtless an
adaptation of the Cypro-Minoan. It is notable that the Cypriote syllabary
remained the preferred means of recording Greek on the island of Cyprus,
even after alphabetic writing was also known. The two scripts were used
side-by-side, until, under foreign rule by the Ptolemies, the syllabary was
driven out sometime in the late third century B.C.

About 500 texts written in the Cypriote syllabary are extant. A few
record an unknown, non-greek language usually called Eteocypriote.43
The wide subject matter of the Greek-language texts, inscribed on a
diversity of objects, includes sepulchral, votive, and honorary topics.
There are even four hexameters (below, inrT.). We can identify two
principal varieties of the Cypriote syllabary; one was confined to the
southwest of the island in the area of Old and New Paphos, Rantidi, and
Kition (so-called syllabaire paphien); the other, formally somewhat
different, was used over the rest of the island. The Paphian texts are
written from right to left, the others from left to right.

Cypriote writing is a pure syllabary, without logograms (except for
numerals) and associated indicative signs and devices. Five signs stand for
the pure vowels [a], [e], [i], [o], [u] (just as in Linear B). 

About fifty other signs represent open syllables, consisting of a consonant plus one of the
five vowels (see Table~\ref{tbl:cypriote}). No distinction is made between voiced, aspirated, and unvoiced stops so that, for example, πα, φα, βα are all represented by the same sign, as are τα, θα, δα44 and κα, χα, γα.45 There
seem to be special signs for [xa] and [xe]. Because the syllabograms stand
for open syllables and Greek contains many consonant clusters and final
closed syllables, complicated rules govern the working of Cypriote in the
spelling of Greek (the same is true of Linear B).

The characters are \textit{syllabic}. There is one character for each  vowel, \textit{a, e, i, o, u,} and perhaps one for \textit{o}. There is no distinction between long and short vowels. The other characters represent what are called \textit{open syllables}\footnote{ If a syllable ends with a consonant, it is called a closed syllable. If a syllable ends with a vowel, it is called an open syllable. }, i.e., beginning with a consonant and ending with a vowel. 

No distinction is made between smooth, middle and rough mutes. The same character stands for τά τ\'ασs, δα in Εδαλιον ανδ δα ιν Αθανα  κε, κη, γε, γη, χε, χη. This fact constitutes the greatest difficulty in reading Cypriote.  

Let us now examine a sentence from the celebrated bronze tablet from
Idalion (Fig.~\ref{fig:idalion}), one of the earliest Cypriote inscriptions found, and still
the longest. The tablet, now in the Cabinet des Medailles in the
Bibliotheque Nationale in Paris, was acquired in 1850 by the Due de
Luynes. It had been suspended from an attached ring in the temple of
Athena at Idalion to record an agreement between a certain King
Stasikypros, probably the last king of the city of Idalion, and a physician
by the name of Onasilos, concerning the treatment of the wounded after
a siege of Idalion by the Medes and the people of Kition. The inscription
informs us that the king and the city will reimburse the physicians for their
labors with money and land. 

The document evidently reflects the military
campaigns against Idalion just before Idalion was absorbed into the
kingdom of Kition c. 470; O. Masson dates it to 478-70 B.C. Fig. 9 gives
the Cypriote text with interlinear transliteration into Roman characters.46
The original reads from right to left, but for convenience I have rewritten
the text to read from left to right; numerals in parentheses indicate line
numbers in the original text.

\section{Dedication to Demeter}

\begin{figure}[htbp]
\includegraphics[width=\linewidth]{cypriote-63}
\caption{Dedication to Demeter and kore of Hellooikos, son of Poteisis. Late fourth century bc. \textit{London, British Museum, Reg. no. 96, 2-1, 215. From the Temenos of Demeter, excavations of the British Museum.}}
\end{figure}

Mitford describes the pedestral as:

\begin{quotation}
Pedestral of a fine white marble, undamaged, with molded cap and base. Width 0.28m.; height 0.065 m.; Found in the early months of 1895 in excavations conducted by H.B. Walters "on temple-site (C)" (plan 1).

\indent The inscription, equally undamaged, is composed of two alphabetic lines above one syllabic. Both texts are admirably cut, their characters regular, clear, well formed, without serifs or \textit{apices}. Height of the former from to o.008 m.; of the latter, 0.006 to 0.008 m.
\end{quotation}



\ExplSyntaxOn
\def\startCypriote{\begingroup\par\leavevmode\cypriote
\def\a{%
       $\stackrel{
          \mbox{\strut\char"10800}
          }%
          {\mbox{\strut\arial a-}%    
          }$
   }%  
\def\e{
       $\stackrel{\mbox{\char"10801}}{\mbox{\arial e - }}$
      }
\def\i{
       $\stackrel{\mbox{\strut\char"10802}}{\mbox{\strut\arial i - }}$
      }
\def\o{$\stackrel{\mbox{\char"10803}}{\mbox{\arial o - }}$}
\def\u{$\stackrel{\mbox{\char"10804}}{\mbox{\arial u - }}$}
\def\wa{$\stackrel{\mbox{\char"10832}}{\mbox{\arial wa -}}$}
\def\we{\char"10833}%
\def\wi{\char"10834}%
\def\wo{\char"10835}%
\def\za{\char"1083C}%
\def\zo{\char"1083F}%
\def\ja{}%
\def\jo{}%
\def\ka{\char"1082F}%
\def\ke{\char"1080B}%
\def\ki{\char"1080C}%
\def\ko{\char"1080D}%
\def\ku{\char"1080E}%
\def\la{\char"10816}%
\def\le{\char"10810}%
\def\li{\char"10811}%
\def\lo{\char"10812}%
\def\lu{\char"10813}%
\def\ma{\char"10814}%
\def\me{\char"10815}%
\def\mi{\char"10816}%
\def\mo{\char"10817}%
\def\mu{\char"10818}%
\def\na{\char"10819}
\def\ne{\char"1081A}
\cs_set:Npn\ni{\char"1081B}
\cs_set:Npn\no{\char"1081C}
\cs_set:Npn\nu{\char"1081D}
\def\ksa{\char"10837}
\def\kse{\char"10838}
\def\pa{\char"1081E}
\def\pe{\char"1081F}
\def\pi{\char"10820}
\def\po{\char"10821}
\def\pu{\char"10822}

\def\ra{\char"10823}
\def\re{\char"10824}
\def\ri{\char"10825}
\def\ro{\char"10826}
\def\ru{\char"10827}
% s-
\def\sa{\char"10828}
\def\se{\char"10829}
\def\si{\char"1082A}
\def\so{\char"1082B}
\def\su{\char"1082C}
% t-
\def\ta{\char"1082D}
\def\te{\char"1082E}
\def\ti{\char"1082F}
\def\to{\char"10830}
\def\tu{\char"10831}
}

\def\stopCypriote{\endgroup}
\ExplSyntaxOff

\captionof{table}{The Cypriote Syllabary}
\label{tbl:cypriote}
\startCypriote
\let\ar\arial\cypriote\large
\begin{longtable}{>{\ar}c| c c c c c}
   &\ar a &\ar e &\ar i &\ar o &\ar u\\
   \hline
  &\a &\e &\i &\o &\u\\
 w &\wa &\we         &\wi         &\wo         &           \\ 
 z & \za           &            &            &\zo            &           \\ 
 j &               &            &            &               &           \\
 k-,g-,kh- &\ka       &\ke            &\ki            &\ko               &\ku     \\
 l         &\la       &\le            &\li            &\lo               &\lu     \\
 m         &\ma       &\me            &\mi            &\mo              &\mu        \\
 n         &\na       &\ne            &\ni            &\no               &\nu     \\
 ks-       &\ksa      &\kse           &               &                  &     \\
 p-,b-,ph  &\pa       &\pe            &\pi            &\po               &\pu   \\
 r-        &\ra       &\re            &\ri            &\ro               &\ru   \\
 s-        &\sa       &\se            &\si            &\so               &\su   \\
 t-, d-,th-  &\ta     &\te            &\ti            &\to               &\tu   \\ 
\end{longtable}  

\stopCypriote

From the above the Cypriote syllabary may at first appear ill-suited to the writing of the Greek language. Powell\footcite[][page 44]{powell1991} correctly observes that it is in fact surprisingly well designed as it
imparts phonetic information about the underlying language once one has
mastered the spelling rules. Lacking the apparatus of logograms, sign
indicators, phonetic and semantic complements, and adjective signs of the
ancient logo-syllabic writings, and therefore different in kind from its
Egyptian or Akkadian antecedents, the Cypriote syllabary is a purely
phonetic writing of admirable simplicity and clarity, a high achievement
in the history of writing:

\subsection{Annotations}

The text is annotated in the following lines, where also the rules are outlined.

\begin{enumerate}
\item The script does not distinguish between aspirated and unaspirated vowels. O-TE (line 1) stands for "Οτε and
A stands for ά (line 7).

\item Final consonants are always rendered by the " e " series of syllabic
signs, i.e. the appropriate consonant plus the vowel e (§39.3). Thus the
sign for NE renders final [n] of πτόλιν (line ι), ' Εδάλιον (line 2),
κατέfοργον (line 2), Φιλοκύπρων (line 5), άνωγον (line 9), Όνασίλον
(line 10), τον (line 10), ίνατήραν (line 11), and μισθών (line 17).
SE by the same principle stands for final [s] in κά (lines 3, 12), κετιήρες
(line 4), βασιλεύς (line 6), Στασίκυπρος (line 7), πτόλι (line 8),
Έδαλιήρες (line 9), τό (line 12), κασιγνήτος (line 13), ά(ν)θρώποs
(line 14), and ικμαμένος (line 16).

The appearance of signs in the " e " series in final .position without
word-dividers seems to show that in position before another word
beginning with a vowel final NE or SE are regarded as virtual
consonants; except in the case of diphthongs, or when an internal letter
such as [s] or [p] has dropped out, two or more vowels do not appear
together in the Cypriote syllabary (§35.2—4).
\item  Observe that the prosodic use of word-dividers is not consistent. For
some reason they are particularly apt to be omitted in the first lines of
a text between words in close association, as here between
PO-TO-LI-NE (τττόλι*) and E-TA-LI-O-NE ('Εδάλιον) (line 1);
between PI-LO-KU-PO-RO-NE (Φιλοκύπρων), WE-TE-I
(ρέτει), and TO-O-NA-SA-KO-RA-U (τω Όνασαγόραυ) (lines
4-5); and between TO-NO-NA-SI-KU-PO-RO-NE (τον
Όνασικύπρων) and TO-NI-YA-TE-RA-NE (τον iycnfpav) (lines
10-11). 

Word division is also readily omitted between a subject and
its predicate, as here between \textsc{KA-TE-WO-RO-KO-NE} (κατέρ-opyov) and MA--TO--Ι (Μάδοι) (lines 2—3); and between
A - N O - K O - N E (avcoyov) and O-NA-SI-LO-NE (Όνασίλον)
(line 9).


\item When, in an internal consonant cluster, the consonants belong to
separate syllables (not as in annotation no. 4), then the first consonant
is rendered by the sign that has the vowel belonging to the preceding
syllable (§42.4). Thus:

I-YA-SA-TA-I = Ινάσθαι (line 13)

(But in this case the rule is disguised because the syllable that follows
SA — namely TA - has the same vowel as the syllable that precedes SA
— namely YA).

MI-SI-TO-NE (not *MI-SO-TO-NE) for μισθών (line 17)
MCJ-MA-ME-NO-SE (not *I=KA-MA-ME-NO-SE) for ίκ(?)μαμένοs (lines
15-16).
\end{enumerate}

\section{Writing Medium}
\lorem\lorem\lorem

\begin{figure}[htbp]
\centering

\includegraphics[width=0.6\textwidth]{oldest-book}

\end{figure}

\subsection{Transliteration}
Using a set of macros we can easily type in the syllabary with little effort

{\Large\startCypriote \a{\arial-}\i\o  \te\ta\po\to\to\li\ne\e\ta\li \stopCypriote}


\subsection{Unicode encoding}
The Cypriot syllabary was added to the Unicode Standard in April, 2003 with the release of version 4.0.
The Unicode block for Cypriot is \unicodenumber{U+10800–U+1083F}. The Unicode block for the related Aegean Numbers is \unicodenumber{U+10100–U+1013F}.



\begin{scriptexample}[]{Cypriot Syllabary}
\unicodetable{cypriote}{"10800,"10810,"10820,"10830}

\cypriote \symbol{"10803}
\end{scriptexample}


\section{Refs}

In \cite{Reyes1994} we find the relationships between Cyprus and its neighbours as well as evidence between trading in the City Kingdoms.\footcite[See pages 5-34]{Reyes1994}





\printunicodeblock{./languages/cyprus.txt}{\cypriote}

\endinput

\bgroup
\newfontfamily\ipafont{Charis SIL}

\ipafont


\begin{IPA}
\textipa{[""Ekspl@"neIS@n]}
\textipa{\;B \;E \;A \;H \;L \;R}\\
\textipa{\!b \!d \!g \!j \!G \!o}\\
\tone{55}ma ‘‘mother’’, \tone{35}ma ‘‘hemp’’


\textipa{iDa}
\end{IPA}

\LARGE
\noindent p t̪ t ʈ t͡ʃ c k\\
pʰ t̪ʰ ʈʰ t͡ʃʰ cʰ kʰ \\
b d̪ ɖ d͡ʒ ɟ ɡ \\
bʱ d̪ʱ ɖʱ d͡ʒʱ ɟʱ ɡʱ \\
f s ʂ ʃ h \\
m n̪ n ɳ ɲ ŋ \\
r ɽ ͏ɻ\\
l ɭ \\
w v j
W+ 
a=sdfɡtⱳsᵭ v

ᵭ ħ ʞ ɭ ɲ ɱ ɭ ᵽ 
\egroup


With Unicode and the right font, there is no problem  in typesetting IPA phonetic symbols. However the problem is the input.

I recommend that you get familiar with a Unicode IPA keyboard overlay. I have used Keyman. When the keyboard is turned on, certian keys (`,@,=) are activated.

As long as your editor allows Unicode input (most do these days) and you're compiling with XeLaTeX or LuaLaTeX, you can just use the IPA keyboard to type directly into the editor just as you can in most other applications. You can also copy and paste your Unicode text from other applications too.

p̛ tt≠pljk ᶊ˥



ɑ ɲ ɲ ɸ β ɹ ɬ ɭ ɮ l ʘ ɓ ǂ ʍ w Waiting ɧ ħ ʌ ɒ 

ɨmnƙ ɮ ʠɰɜɾƭƭyʌiƥ

kʰ 

t̥ataraṭ tʷara zʷara ɣ anɣᵘish ma˨˥˨ a᷅ āltoʃ

This will take time to get upto speed. One can also ofcourse write his own macros for commonly used symbols or words.

d, g,h
˘
d, g, ʂʃs̥s̊å, s, z 

ɬ = 
l̊ at
ɭ  < 
ɮ  >
l̥  ̥
lˈ 
l` l'

I found it frustrating at first to try and remember all the combinations of symbols, especially since I am not a linguist---although I have done a lot of background self-study. In my estimation, handwriting is probably still the fastest way to typeset anything to do with phonetics.

Gâteau Basque, like


ɐ ɑ a ʈ ᵵ ʇ ᴛ ʦ ʧ ʨ ṣ

\textit{da:ta, gaːka, etc.}, 

Hittite vowel phonemes
{\obeylines\large\panunicode
i u ī ū
e a ē ā 
}
Diphthongal combination like that of ˘¯ a and the glides w and y, noted (a-)a-i, (a-)a-u, are
also permitted h₂, h₃
 u, ú, ù, u$_4$


\emph{ši-ú-ni-iš} 

In addition, the order in which the supplementals occur differs in the blue
and red abeecdaria: blue shows the sequence 0, χ, ψ, with the respective values
p\textsuperscript{h}, [kh], [ps] (though the light blue alphabets lack ψ, as discussed above); on
the other hand, the order is χ. Φ. ψ. with the respective values [ks], [ph], [kh],
in the red.

\def\codex#1{\emph{Codex #1}\index{codex>#1}}
%\newfontfamily{\gothicfamily}{Noto Sans Gothic}
\newfontfamily{\gothicfamily}{code2001.ttf}
\section{Gothic}

\label{s:gothic}\index{scripts>Gothic}

\subsection{Introduction}

East Germanic Goths rose to prominence during the Great
Migrations of the fourth and fifth centuries AD 31 Their Gothic
languages are primarily known to us today through a few surviving
fragments of Bible translations. It was the Visigothic bishop
Wulfila († AD 383), according to three ecclesiastical historians
writing a century later, who created ‘Gothic letters’ in order to
translate the Bible into the Visigothic language. The fourth century
Greek alphabet was Wulfila’s only apparent source.

Though the bishop’s original Visigothic hand has not survived,
closely related derivative scripts preserved in two later Gothic
manuscripts no older than the sixth century have been preserved
(illus. 116).

‘Wulfila’s script’, as it perhaps should properly be designated,
is an alphabetic script written from left to right without word
separation. Spaces indicate sentences or passages, as does a
colon or a centred dot (as with the Iberian scripts). Nasal suspension
– that is, marking where an /m/ or /n/ should be – is
sometimes indicated by a macron (a topping stroke) above the
preceding letter. Ligatures are even rarer than macrons. There
are frequent contractions: for example, ius is often used to spell
‘Jesus’. Apart from rare profane relics – witness the sixth-century
Latin-Gothic Deed of Naples – Wulfila’s script, measured
by those few inscriptions that have survived, appears to have
conveyed exclusively ecclesiastical texts.

\begin{figure}[htb]
\includegraphics[width=.45\textwidth]{gothic}
\caption{Codex Carolinus}
\end{figure}

The Gothic script that Wulfila devised from the Greek
alphabet did not engender daughter scripts. After the sixth century
AD, it was replaced almost everywhere by related descendants
of Greek and Latin alphabets. Gothic’s last sentinel, the
ninth-century \codex{Vindobonensis} 795, was perhaps by then
only an antiquarian curiosity. The \emph{Codex Carolinus} preserves papal correspondence
with Frankish rulers, including letters exchanged by popes from Gregory III (731-741) to Hadrian I (772-795). the Codex was written in 791 on the orders of Charlemagne in order to rescue papyrus copies threatened with decay. It contains 99 letters, almost exclusively papal, and survives today in Vienna, \"Osterreichische Nationalbibliotek 449, in a copy probably made at Colone during the pontificate of Archbishop Willibert (870-889). The preface of the \codex{Carolinus} appears to refer to a second part that may have contained letters to byzantine rulers, now lost. Parallel copies of the Codex have not turned up. \citep{jasper2001papal}. 

\subsection{Unicode}

The Gothic alphabet was added to the Unicode Standard in March, 2001 with the release of version 3.1.

The Unicode block for Gothic is U+10330–U+1034F in the Supplementary Multilingual Plane. As older software that uses UCS-2 (the predecessor of UTF-16) assumes that all Unicode codepoints can be expressed as 16 bit numbers (U+FFFF or lower, the Basic Multilingual Plane), problems may be encountered using the Gothic alphabet Unicode range and others outside of the Basic Multilingual Plane.

\begin{scriptexample}[]{Gothic}
\unicodetable{gothicfamily}{"10330,"10340}
\end{scriptexample}
{\gothicfamily
𐍀	𐍁	𐍂	𐍃	𐍄	𐍅	𐍆	𐍇	𐍈	𐍉	𐍊}
%http://www.gotica.de/carolinus.html

%\begin{thebibliography}
%\bibitem[Fitzmyer(1995)]{fitzmyer}
%J.~A. Fitzmyer.
%\newblock \emph{The Aramaic inscriptions of Sefīre}, volume~19 of
%  \emph{Biblica et orientalia Sacra Scriptura antiquitatibus orientalibus
%  illustrata}.
%\newblock Pontificial Biblical Institute, Rome, 1995.
%\newblock URL
%  \url{http://web.archive.org/web/20051104215025/http://www.nelc.ucla.edu/Faculty/Schniedewind_files/NWSemitic/Aramaic_ABD.pdf}.
%\end{thebibliography}  





\chapter{Albanian}

Albania's national culture came into being at the crossroads of three great civilizations:
that of Latin Catholicism from the West, that of Byzantine Greek Orthodoxy from the south, and
that of Islam imported by the Ottoman Turks, who had invaded the country in the late 14th
century and who ruled it until the declaration of independence in 1912. Early writing in this tiny
Balkan country, very much a product of these three extremely diverse cultures, was as a result a
hybrid creation. 

\section{Elbasan}
\label{s:elbasan}
\newfontfamily\elbasan{Albanian.otf}
The Elbasan script is a mid 18th-century alphabetic script used for the Albanian language. It was named after the city of Elbasan where it was invented. It was mainly used in the area of Elbasan and Berat. It is widely considered to be the first original alphabet developed for transcribing the Albanian language.

The primary document associated with the alphabet is the Elbasan Gospel Manuscript, known in Albanian as the Anonimi i Elbasanit (The Anonymous of Elbasan).[1] The document was created at St. Jovan Vladimir's Church in central Albania, but is preserved today at the National Archives of Albania in Tirana. Its 59 pages contain Biblical content written in an alphabet of 40 letters,[1] of which 35 frequently recur and 5 are rare. Dots are used on three characters as inherent features of them to indicate varied pronunciation (pre-nasalization and gemination) found in Albanian. The script generally uses Greek letters as numerals with a line on top.

Another original script used for Albanian, was Beitha Kukju's script of the 19th century. This script did not have much influence either.

Elbasan is a simple alphabetic script written from left to right horizontally. The alphabet consists of forty letters.

\subsection{Accents and Other Marks}

The Elbasan manuscript contains breathing accents, similar to
those used in Greek. Those accents do not appear regularly in the orthography and have
not been fully analyzed yet. Raised vertical marks also appear in the manuscript, but are
not specific to the script. Generic combining characters from the Combining Diacritical
Marks block can be used to render these accents and other marks.

\subsection{Names}

The names used for the characters in the Elbasan block are based on those of the
modern Albanian alphabet.

\subsection{Numerals and punctuation}

There are no script-specific numerals or punctuation marks.
A separating dot and spaces appear in the Elbasan manuscript, and may be rendered with
U+00B7 middle dot and U+0020 space, respectively. For numerals, a Greek-like system
of letter and combining overline is in use. Overlines also appear above certain letters in
abbreviations, such as $\overline{\text{\elbasan\char "10507\char"1051D}}$ to indicate Zot (Lord). The overlines in numerals and abbreviations
can be represented with U+0305 combining overline. (See also \href{http://www.unicode.org/charts/}{unicode charts}.)

\subsection{unicode}

Elbasan is a Unicode block containing the historic Elbasan characters for writing the Albanian language. Free fonts for personal use can be found at \href{http://www.fontspace.com/category/unicode\%20font\%20for\%20elbasan}{fontspace}, which I have used here. Commercial fonts can be found at Evertype.

\begin{scriptexample}[]{Elbasan}
\unicodetable{elbasan}{"10500,"10510,"10520}
\end{scriptexample}

\section{Old Persian}
\label{s:oldpersian}


Old Persian, like Hittite an Indo-European language, was written in cuneiforms as of the first millenium BC, mostly between 550 and 350. King Darius’ monumental inscription at
Bisothum – in Old Persian, Elamite and Neo-Babylonian – furnished
the ‘key’ to cuneiform’s decipherment and the reconstruction
of these languages.28 Darius’ Old Persian scribes
effected the most drastic simplification of the borrowed Near
Eastern script (illus. 35). They reduced the cuneiform inventory
to only 41 signs of both syllabic (ka) and phonemic (/k/) values.
Thus, Old Persian cuneiform is ‘half syllabic, half letter writing’.
29 It appears to be on the fence between the Babylonians’
cuneiforms and the Levantines’ consonantal writing, a hybrid
solution using only four logograms and 36 syllabo-phonemic
signs written in wedges. Of particular significance is the fact
that Old Persian also conveys the individual long and short
vowels /a/ (pronounced AH), /i/ (EE) and /u/ (OO) that the
Ugaritic system had conveyed a thousand years earlier.

Old Persian cuneiform is a semi-alphabetic cuneiform script that was the primary script for the Old Persian language. Texts written in this cuneiform were found in Persepolis, Susa, Hamadan, Armenia, and along the Suez Canal.[1] They were mostly inscriptions from the time period of Darius the Great and his son Xerxes. Later kings down to Artaxerxes III used corrupted forms of the language classified as “pre-Middle Persian”.

\begin{scriptexample}[]{Old Persian}
\unicodetable{oldpersian}{"103A0,"103B0,"103C0,"103D0}
\end{scriptexample}

Scholars today mostly agree that the Old Persian script was invented by about 525 BC to provide monument inscriptions for the Achaemenid king Darius I, to be used at Behistun. While a few Old Persian texts seem to be inscribed during the reigns of Cyrus the Great (CMa, CMb, and CMc, all found at Pasargadae), the first Achaemenid emperor, or Arsames and Ariaramnes (AsH and AmH, both found at Hamadan), grandfather and great-grandfather of Darius I, all five, specially the later two, are generally agreed to have been later inscriptions.
Around the time period in which Old Persian was used, nearby languages included Elamite and Akkadian. One of the main differences between the writing systems of these languages is that Old Persian is a semi-alphabet while Elamite and Akkadian were syllabic. In addition, while Old Persian is written in a consistent semi-alphabetic system, Elamite and Akkadian used borrowings from other languages, creating mixed systems.
\medskip

% increa the width by a given dimension to oversize the image and then shift it to the left, irrespective if we are on left
% or right page.
{\leftskip-1.25cm
\includegraphics[width=\textwidth+2.5cm]{./images/naghshe.jpg}
\captionof{figure}{Panoramic view of the Naqsh-e Rustam. This site contains the tombs of four Achaemenid kings, including those of Darius I and Xerxes. (\textit{Wikimedia})}
}
\section{Inscriptional Pahlavi}
\label{s:inscriptionalpahlavi}

Pahlavi or Pahlevi denotes a particular and exclusively written form of various Middle Iranian languages. The essential characteristics of Pahlavi are[1]
the use of a specific Aramaic-derived script, the Pahlavi script;
the high incidence of Aramaic words used as heterograms (called hozwārishn, "archaisms").

Pahlavi compositions have been found for the dialects/ethnolects of Parthia, Parsa, Sogdiana, Scythia, and Khotan.[2] Independent of the variant for which the Pahlavi system was used, the written form of that language only qualifies as Pahlavi when it has the characteristics noted above.


Pahlavi is then an admixture of
written Imperial Aramaic, from which Pahlavi derives its script, logograms, and some of its vocabulary.

spoken Middle Iranian, from which Pahlavi derives its terminations, symbol rules, and most of its vocabulary.
Pahlavi may thus be defined as a system of writing applied to (but not unique for) a specific language group, but with critical features alien to that language group. It has the characteristics of a distinct language, but is not one. It is an exclusively written system, but much Pahlavi literature remains essentially an oral literature committed to writing and so retains many of the characteristics of oral composition.

\begin{scriptexample}[]{Pahlavi}
\unicodetable{inscriptionalpahlavi}{"10B60,"10B70}
\end{scriptexample}

\section{Imperial Aramaic}
\label{s:imperialaramaic}

\subsection{History}

Aramaic is the best-attested and longest-attested
member of the NW Semitic subfamily of languages
(which also includes inter alia \nameref{s:hebrew}, \nameref{s:phoenician},
\nameref{s:ugaritic}, Moabite, Ammonite, and Edomite). The
relatively small proportion of the biblical text
preserved in an Aramaic original (Dan 2:4–7:28; Ezra
4:8–68 and 7:12–26; Jeremiah 10:11; Gen 31:47 [two
words] as well as isolated words and phrases in
Christian Scriptures) belies the importance of this
language for biblical studies and for religious studies
in general, for Aramaic was the primary international
language of literature and communication throughout
the Near East from ca. 600 B.C.E. to ca. 700 C.E. and
was the major spoken language of Palestine, Syria,
and Mesopotamia in the formative periods of
Christianity and rabbinic Judaism. 



Aramaic survived over a period of 3,000 years, during which time its grammar, vocabulary and usage experienced great changes. Aramaic scholars found it useful to divide the several Aramaic dialects into periods, groups and subgroups based both on the chronology as well as the geography.

\begin{enumerate}
\item Old Aramaic
\item Imperial Aramaic
\item  Middle Aramaic
\item Late Aramaic
\item Modern Aramaic
\end{enumerate}


\subsection{Old Aramaic (to ca. 612 BCE)}
This period
witnessed the rise of the Arameans as a major force
in ANE history, the adoption of their language as an
international language of diplomacy in the latter days
of the Neo-Assyrian Empire, and the dispersal of
Aramaic-speaking peoples from Egypt to Lower
Mesopotamia as a result of the Assyrian policies of
deportation. The scattered and generally brief
remains of inscriptions on imperishable materials
preserved from these times are enough to
demonstrate that an international standard dialect had
not yet been developed. The extant texts may be
grouped into several dialects:

\subsection{Middle Aramaic (to ca. 250 C.E.)}
In the Hellenistic and Roman periods, Greek replaced
Aramaic as the administrative language of the Near
East, while in the various Aramaic-speaking regions
the dialects began to develop independently of one
another. Written Aramaic, however, as is the case
with most written languages, by providing a
somewhat artificial, cross-dialectal uniformity,
continued to serve as a vehicle of communication
within and among the various groups. For this
purpose, the literary standard developed in the
previous period, Standard Literary Aramaic, was
used, but lexical and grammatical differences based
on the language(s) and dialect(s) of the local
population are always evident. It is helpful to divide
the texts surviving from this period into two major
categories: epigraphic and canonical.

\subsection{Late Aramaic (to ca. 1200 C.E.)}
The bulk of
our evidence for Aramaic comes from the vast
literature and occasional inscriptions of this period.
During the early centuries of this period Aramaic
dialects were still widely spoken. During the second
half of this period, however, Arabic had already
displaced Aramaic as the spoken language of much
of the population. Consequently, many of our texts
were composed and/or transmitted by persons whose
Aramaic dialect was only a learned language.
Although the dialects of this period were previously
divided into two branches (Eastern and Western), it
now seems best to think rather of three: Palestinian,
Syrian, and Babylonian.

The Aramaic alphabet is adapted from the \nameref{s:phoenician} alphabet and became distinctive from it by the 8th century BCE.  The letters all represent consonants, some of which are \emph{matres lectionis}, which also indicate long vowels.

\subsection{Modern Aramaic (to the present day)}

These dialects can be divided into the same three
geographic groups.

\begin{description}

\item[a. Western]
Here Aramaic is still spoken only in
the town of Ma’lula (ca. 30 miles NNE of Damascus)
and surrounding villages. The vocabulary is heavily
Arabized.

\item[b. Syrian]
Western Syrian (Turoyo) is the language
of Jacobite Christians in the region of Tur-Abdin in
SE Turkey. This dialect is the descendant of
something very like classical Syriac. Eastern Syrian
is spoken in the Kurdistani regions of Iraq, Iran,
Turkey, and Azerbaijan by Christians and, formerly,
by Jews. Substantial communities of the former are
now found in North America. The Jewish speakers
have mostly settled in Israel. These dialects are
widely spoken by their respective communities and
have been studied extensively during the past
century. It has become clear that they are not the
descendants of any known literary Aramaic dialect.

\item[c. Babylonian] 

Mandaic\ref{s:mandaic} is still used, at least until
recently, by some Mandaeans in southernmost Iraq
and adjacent areas in Iran.

In addition, in recent years classical \nameref{s:syriac} has
undergone somewhat of a revival as a learned vehicle
of communication for Syriac Christians, both in the
Middle East and among immigrant communities in
Europe and North America.
\end{description}

\begin{figure}[htbp]
\centering
\includegraphics[width=0.6\textwidth]{./images/elephantine-papyrus.jpg}

\caption{The Elephantine papyri are ancient Jewish papyri dating to the 5th century BC, requesting the rebuilding of a Jewish temple. It also name three persons mentioned in Nehemiah: Darius II, Sanballat the Horonite and Johanan the high priest.}

\end{figure}


\subsection{Alphabet and typesetting}

The Aramaic alphabet is historically significant, since virtually all modern Middle Eastern writing systems can be traced back to it, as well as numerous non-Chinese writing systems of Central and East Asia. This is primarily due to the widespread usage of the Aramaic language as both a \emph{lingua franca} and the official language of the Neo-Assyrian Empire, and its successor, the Achaemenid Empire. Among the scripts in modern use, the Hebrew alphabet bears the closest relation to the Imperial Aramaic script of the 5th century BC, with an identical letter inventory and, for the most part, nearly identical letter shapes.

Writing systems that indicate consonants but do not indicate most vowels (like the Aramaic one) or indicate them with added diacritical signs, have been called abjads by Peter T. Daniels to distinguish them from later alphabets, such as Greek, that represent vowels more systematically. This is to avoid the notion that a writing system that represents sounds must be either a syllabary or an alphabet, which implies that a system like Aramaic must be either a syllabary (as argued by Gelb) or an incomplete or deficient alphabet (as most other writers have said); rather, it is a different type.

The Imperial Aramaic alphabet was added to the Unicode Standard in October 2009 with the release of version 5.2.
The Unicode block for Imperial Aramaic is \unicodenumber{U+10840–U+1085F}.

\begin{scriptexample}[]{Aramaic}
\unicodetable{imperialaramaic}{"10840,"10850}
\end{scriptexample}




\PrintUnicodeBlock{./languages/imperial-aramaic.txt}{\imperialaramaic}
\section{Ogham}
\label{s:ogham}
\newfontfamily\ogham{code2000.ttf}

Ogham /ˈɒɡəm/[1] (Modern Irish [ˈoːm] or [ˈoːəm]; Old Irish: ogam [ˈɔɣam]) is an Early Medieval alphabet used primarily to write the early Irish language (in the so-called "orthodox" inscriptions, 4th to 6th centuries), and later the Old Irish language (so-called scholastic ogham, 6th to 9th centuries). There are roughly 400 surviving orthodox inscriptions on stone monuments throughout Ireland and western Britain; the bulk of them are in the south of Ireland, in Counties Kerry, Cork and Waterford.\footnote{McManus (1991) is aware of a total of 382 orthodox inscriptions. The later scholastic inscriptions have no definite endpoint and continue into the Middle Irish and even Modern Irish period, and record also names in other languages, such as Old Norse, (Old) Welsh, Latin and possibly Pictish. See Forsyth, K.; "Abstract: The Three Writing Systems of the Picts." in Black et al. Celtic Connections: Proceedings of the Tenth International Congress of Celtic Studies, Vol. 1. East Linton: Tuckwell Press (1999), p. 508; Richard A V Cox, The Language of the Ogam Inscriptions of Scotland, Dept. of Celtic, Aberdeen University ISBN 0-9523911-3-9 [1]; See also The New Companion to the Literature of Wales, by Meic Stephens, page 540.} A rare example of a Christianised Ogham stone can be seen in St. Mary's Collegiate Church Gowran Co. Kilkenny. The largest number outside of Ireland is in Pembrokeshire in Wales.\footnote{O'Kelly, Michael J., '\textit{Early Ireland, an Introduction to Irish Prehistory}', p. 251, Cambridge University Press, 1989} The vast majority of the inscriptions consist of personal names.

Ogham is sometimes called the "Celtic Tree Alphabet", based on a high medieval Bríatharogam tradition ascribing names of trees to the individual letters. The etymology of the word ogam or ogham remains unclear. One possible origin is from the Irish og-úaim 'point-seam', referring to the seam made by the point of a sharp weapon.[4]

Ogham was added to the Unicode Standard in September 1999 with the release of version 3.0.

The spelling of the names given is a standardization dating to 1997, used in Unicode Standard and in Irish Standard 434:1999.
The Unicode block for ogham is \texttt{U+1680–U+169F}.

\begin{scriptexample}[]{Ogham}
\unicodetable{ogham}{"1680,"1690}

With the Titus font

\unicodetable{titus}{"1680,"1690}
\end{scriptexample}


\printunicodeblock{./languages/ogham.txt}{\ogham}

\newfontfamily\brill{brill}
\large\brill
\parindent1em
\parskip10pt

\section{Ancient Anatolian Alphabets}
\label{s:anatolian}
The Anatolian scripts described in this section all date from the first millenium BCE, and were used to write various ancient Indo-European languages of western and southwestern Anatolia (now Turkey). All are related to the Greek script and are probably adaptations of it. 



\section{Lydian}
\label{sec:lydian}
 Lydian script was used to write the Lydian language. That the language preceded the script is indicated by names in Lydian, which must have existed before they were written. Like other scripts of Anatolia in the Iron Age, the Lydian alphabet is a modification of the East Greek alphabet, but it has unique features. The same Greek letters may not represent the same sounds in both languages or in any other Anatolian language (in some cases it may). Moreover, the Lydian script is alphabetic.



Early Lydian texts are written both from left to right and from right to left. Later texts are exclusively written from right to left. One text is boustrophedon. Spaces separate words except that one text uses dots. Lydian uniquely features a quotation mark in the shape of a right triangle.

The first codification was made by Roberto Gusmani in 1964 in a combined lexicon (vocabulary), grammar, and text collection.

\begin{scriptexample}[]{Lydian}
\unicodetable{lydianfont}{"10920,"10930}

\medskip

Typeset with the \idxfont{Aegean.ttf} and the command \cmd{\lydian}
\end{scriptexample}

Examples of words

\bgroup\lydian
𐤬𐤭𐤠  - Ora - "Month"

𐤬𐤳𐤦𐤭𐤲𐤬𐤩  - Laqrisa - "Wall"

𐤬𐤭𐤦𐤡  - "House, Home"

\egroup

Herodotus Hdt. 1.94 
Chapter on the Lydians is well known, but in order to evaluate it properly it will be
helpful to recall exactly what it says54:

\begin{latexquotation}
The Lydians have about the same customs as the Greeks, except that the
Lydians prostitute their female children. The Lydians are the first people
we know to have coined money of silver and gold, and they were the first to
be shopkeepers. The Lydians themselves also claim the invention of the
games that both they and the Greeks now play. They say that the invention
occurred at the same time that they colonized Tyrsenia. What they say
about these things goes like this (the following is in indirect discourse):
In the reign of Atys, son of Manes, there was a terrible famine
throughout Lydia. Although in hard straits, the Lydians persevered for
some time. But finally, when there was no let-up, they sought respite,
some trying one thing and others another. It was then that they invented
dice, and astragals, and ball, and all the other kinds of games, except for
draughts. For the Lydians don't claim to have invented draughts. After
their inventions, this is what they did about the famine. Every second
day they would play, all day, so as not to want food, and on the day
between they would eat, and not play. In this way they persevered for
eighteen years. Since the evil did not abate, but pressed them even
worse, their king divided them up into two parts, by lot: the one group
for staying on, the other group to emigrate from the country. And the
king himself was to be in charge of the group that remained, while in
charge of the departing group was the king's son, whose name was
Tyrsenos. The group whose lot it was to depart from the land went down
to Smyrna and built boats. They put everything they needed into the
boats and sailed away in search of life and land; passing by many
nations, they sailed until they reached the Ombrikians, where they built
cities for themselves and they still live there today. Instead of "Lydians",
they adopted a new name from the king's son, the man who led them.
Taking their eponym from him, they were called Tyrsenoi.

Well, then, the Lydians were enslaved by the Persians\ldots
\end{latexquotation}



\chapter{Carian Language and Scripts}
\def\textcarian#1{\begingroup\bfseries\lydianfont\color{red}#1\endgroup}

\section{Background}

It is not clear when Caria and the Carians enter into ancient History.
This is dependent on equating classical Caria with the land of Karkiya/
Karkisa mentioned in Hittite sources. This supposition, eminently suitable
from a purely linguistic point of view (karkº in Karkisa, Karkiya
is practically identical to the Old Persian word for ‘Carian’, kºka-), is
complicated by the uncertainties regarding the exact location of Karkisa/
Karkiya on the map, a problem intimately bound to the complex question
of Hittite geography, a topic still subject to controversy despite the
great progress made in recent years.\footcite{adiego}

Classical writers did not agree as to the exact extent of Caria. The first mention of Caria is by Homer\footnote{Homer, Iliad, B', 867-869.} (800 BC). According to Homer the Carians lived around Miletus and the Mykale mountain\footnote{Currently in Turkey Samsun Dağı.} and the valley of Meander. 
Strabo\footnote{63 or 64 BC- c.24 AD} considered that the South border of Caria was in the area of Tralleis, south of Meander and that in the valley of Meander there were Carian living, Lydians, Ioanians and Aoelians. 
Xenophon\footnote{Xenophon, Greek, III, 2, 19.} (427-355 BC) places also Tralley in Caria, Diodorus Sicilus (is century BC)\footnote{Diodorus Sicilus was a Greek historian famous for hus monumental universalDiodorus Sicilus, XIV, 36, 3.} wrote that the area was under Ionian rule. 
Accordind to Herodotus (484-425 BC) who was born in Caria, considers that the south west borders of  history Bibliotheca historica, much of which survives, between 60 and 30 BCCaria the Ionian cities of Prieni, Myous and Miletus were part of Caria.

Broadly, one can consider that Caria was within what is now the Turkish province Muğla. The area is rich in ancient ruins, with over 100 excavated sites including the UNESCO Heritage Site of Letoon, near Fethiye.

The history of Caria is entangled with that of the Greek-speaking world; 
the cultural and religious character of the region was shaped by sustained interaction with both east and west.[1] Ionian and Dorian settlements were established along the Anatolian seaboard from the tenth century BCE onwards, and over the subsequent centuries there was sustained contact and assimilation between the \enquote{Karian} communities of the interior and the \enquote{Greek} cities along the coast. Geographically, Karia was recognised in antiquity as the area south of the Maeander River, extending east to the Salbakos Mountains (Map 1); it shared borders with Lydia to the north, Phrygia to the east, and Lykia to the southeast. As an ethnos, the “Karians” are more difficult to define; while the Karian language has now been identified as an Indo-European language of the “Luwic” subgroup, related to other Anatolian languages including Luwian and Lykian,[2] identifying this population in the archaeological record remains problematic. The cultural coherence of the region as a whole is also not assured; the limit of Karia as a geographical unit seems to have been greater than the area traditionally inhabited by the “Karian” people, who are thought to have been concentrated in the southwestern area.[3]\footnote{See Naomi Carless Unwin, \textit{What's in a Name? Linguistic Considerations in the Study of Karian Religion} \protect \url{http://brewminate.com/whats-in-a-name-linguistic-considerations-in-the-study-of-karian-religion/}}

The land of Caria lay during the first millennium BC in the southwest of Anatolia between
Lydia and Lycia. A few dozen texts in the epichoric language, mostly very short or fragmentary,
have been found in Caria itself or on objects likely to have originated there. These
are dated very approximately to the fourth to third centuries BC. There is also a very fragmentary
Carian–Greek bilingual from Athens, dated to the sixth century. By far the largest
number of Carian texts consists of tomb inscriptions and graffiti left by Carian mercenaries
in Egypt, dating from the seventh to fifth centuries BC. A new epoch in Carian studies has
now begun with the dramatic discovery in 1996 of an extensive Carian–Greek bilingual
by Turkish excavators in Kaunos and its remarkably swift publication by Frei and Marek
(1997).\footcite{frei1997}



\label{sec:carian}
The Carian language is an extinct language of the Luwian subgroup of the Anatolian branch of the Indo-European language family. The Carian language was spoken in Caria, a region of western Anatolia between the ancient regions of Lycia and Lydia, by the Carians, a name possibly first mentioned in Hittite sources. Carian is closely related to Lycian and Milyan (Lycian B), and both are closely related to, though not direct descendants of, Luwian. Whether the correspondences between Luwian, Carian, and Lycian are due to direct descent (i.e. a language family as represented by a tree-model), or are due to dialect geography, is disputed.[3]


\section{Decipherment}

Prior to the late 20th century the language remained a total mystery even though many characters of the script appeared to be from the Greek alphabet. Using Greek phonetic values of letters investigators of the 19th and 20th centuries were unable to make headway and classified the language as non-Indo-European. Speculations multiplied, none very substantial. Progress finally came as a result of rejecting the presumption of Greek phonetic values.

The Carian script surely stands in some relationship to the Greek alphabet. The direction
of writing is predominantly right to left in texts from Egypt, and left to right in those from
Caria. \emph{Scriptio continua} is frequent, and use of word-dividers is sporadic.

Decipherment of the Carian script has been a long and arduous task. Pioneering efforts
by A. H. Sayce at the end of the nineteenth century were followed by several false steps
based on the erroneous assumption of a syllabic or semisyllabic system and a long period of
relative neglect. It was the merit of V. Shevoroshkin (1965) to have shown that the Carian
script is an alphabet. However, the specific values he and others assigned to individual letters
led to no breakthrough in our understanding of the language. Particularly striking was the
virtually complete absence of any matches between Carian personal names, as attested in
Greek sources, and putative examples in the native alphabet.

A new era began in 1981 when John Ray first successfully exploited the evidence of
the Carian–Egyptian bilingual tomb inscriptions to establish radically new values for several
Carian letters, as well as to confirm the values of others. Additional investigation,
notably by Ray, Ignacio-Javier Adiego, and Diether Schurr, has led to further revisions ¨
and refinements of the new system. The basic validity of this approach was shown by its
correct prediction of Carian personal names which have subsequently appeared in Greek
sources. Nevertheless, many uncertainties and unsolved problems remained, and several
reputable experts were skeptical of the new interpretation of the Carian alphabet. One can
conveniently gain a sense of the state of Carian studies prior to 1997 from Giannotta et al.
1994.

The new Carian–Greek bilingual from Kaunos has shown conclusively the essential validity
of the Ray–Adiego–Schurr system, while also confirming the suspicion of local variation 
in the use of the Carian alphabet. While some rarer signs remain to be elucidated, the question
of the Carian alphabet may be viewed as decided. The new bilingual has not led to
immediate equally dramatic progress in our grasp of the language. One reason for this is
that the Greek text of the Kaunos Bilingual is a formulaic proxenia decree, while the corresponding
Carian is manifestly quite independent in its phrasing of what must be essentially
the same contents. 

\section{Athens Bilingual}

The clearest of the bilingual inscriptions is the one from Athens, although it is laconic and fragmentary (Adiego 1993, no.18):

\begin{figure}[htbp] 
\centering
\includegraphics[width=.70\textwidth]{greek-carian-athens}
\caption{Greek-Carian bilingual inscription from Athens.}
\end{figure}

\section{The Kaunos Bilingual}

Most of what follows is based on the description of the discovery of the Kaunos bilingual
and the description of the texts and analysis as published by Frei et al.

The Kaunos bilingual, consists of three fragment and in the interest 
of clarity, the three fragments are numbered as:

Fragment I: the upper fragment, the only Carian text (lines 1-12),\\
Fragment II: the lower left fragment, the Carnic (lines ISIS left part) and Greek (lines 1-8 left part) contains text,\\
Fragment III: this was discovered later than the first two fragments (newly found) right lower fragment, the Carian (line 12-17 right-hand part) and Greek (lines 1-7 right part) contains text. It has the dimensions: height 0.24 m, Width 0.14 m, thickness 0.085 m.

The three fragments close together in depth without joints. At the surface, on the other hand, has its edges bumped or chipped off, so that here partly small, partly, in particular between fragment I and Fragments II and III, larger spaces were created,
which led to the almost complete loss of line 12.\footcite{frei1997}

\begin{figure}[htbp]
\includegraphics{kaunos-bilingual}
\caption{The Kaunos bilingual. From:\protect\cite{frei1997}}
\end{figure}

The Kaunos Bilingual has provided welcome confirmation of the view
that Carian is an Indo-European Anatolian language, and indeed, of the western type of
Luvian, Lycian, and Lydian. However, one cannot speak of a complete decipherment until
there are generally accepted interpretations of a substantial body of textsThis 
remark applies even to the new bilingual, as one can easily confirm by
reading the competing linguistic analyses in Blumel, Frei, and Marek 1998.

\begin{quote}
1 Έδοξε Καυν[ί]οις, επί δημιο-\\
2 ργοΰ Ίπποσθένους· Νικοκ-\\
3 λέα Λυοικλέους Άΰηναΐό(ν]\\
4 και Λυσικλέα Λυσικράτ[ους]\\
5 [Α]θηναΐον προξένους ε[ίναι κ-]\\
6 [α] l εύεργέτας Καυνίω[ν αυτό-]\\
7 ύς και έκγόνφυς και [υπάρχει-]\\
8 ν αύτοΐς Ε[-------------]\\
\end{quote}



The Carian people, are renowned for having travelled widely: as mercenaries in the service of foreign armies, Carians are found
throughout the ancient Near East. Their most celebrated achievement away from home was in
Egypt, where a large Carian community flourished on the military needs of an unstable political
climate in the seventh and sixth centuries BC.3 In Babylonia, the presence of Carians (Karsaja or
BannE]aja) is recorded but documentation is thin: a group of Carians, presumably mercenaries,
occupied a fief (Hatru) close to Nippur,4 perhaps already in the time of Cambyses's there were
Carians among the war captives held by Nebuchadnezar II in Babylon (Weidner, Fs. Dussaud);
and finally there were Carians in Borsippa. Caroline Waerzeggers (2006) provides a detailed
article on the \enquote{The Carians of Borsippa} in which she examined tablet evidence of taxation
for the provision of food rations to Carians by inhabitants of Borsippa.\footcite{caroline2006}

It is also possible that they have travelled to what is now Izrael as masons. If they were paid for the work or they were enslaved we probably never know.

Carian is known from a limited number of sources:

\begin{enumerate}
\item Personal names with a suffix of -ασσις (-assis), -ωλλος (-ōllos) or -ωμος (-ōmos) in Greek records.
\item Twenty inscriptions from Caria including four bilinguals.
\item Inscriptions of the Caromemphites, an ethnic enclave at Memphis, Egypt.
\item Graffiti elsewhere in Egypt.
\item Scattered inscriptions elsewhere in the Aegean world.
\item Words stated to be Carian by ancient authors.
\end{enumerate}


\begin{figure}[htbp]
\centering
\includegraphics[width=0.8\textwidth]{carian-inscription}
\caption[Thessaloniki, Sherd with a Carian inscription]{Thessaloniki, Sherd with a Carian inscription, dated ca. 600 BCE-ca. 300 BCE.} 
%\href{credit livius.org}{http://www.livius.org/pictures/greece/thessaloniki/thessaloniki-museum-pieces/thessaloniki-sherd-with-a-carian-inscription/}

\end{figure}


\subsection{Carian Alphabets} 



The Carian alphabets are a number of regional scripts used to write the Carian language of western Anatolia. They consisted of some 30 alphabetic letters, with several geographic variants in Caria and a homogeneous variant attested from the Nile delta, where Carian mercenaries fought for the Egyptian pharaohs. They were written left-to-right in Caria (apart from the Carian–Lydian city of Tralleis) and right-to-left in Egypt. Carian was deciphered primarily through Egyptian–Carian bilingual tomb inscriptions, starting with John Ray in 1981; previously only a few sound values and the alphabetic nature of the script had been demonstrated. The readings of Ray and subsequent scholars were largely confirmed with a Carian–Greek bilingual inscription discovered in Kaunos in 1996, which for the first time verified personal names, but the identification of many letters remains provisional and debated, and a few are wholly unknown.

Adiego \cite{adiego} writes that: \enquote{For years, the temptation has existed to attribute any inscription from
Asia Minor written in an unknown or barely recognizable alphabet to
Carian. In a sort of \textit{obscurum per obscurius}, such materials were classed
as Carian at a time when the Carian alphabet itself remained un-deciphered.

Today, we have a better understanding of the Carian alphabet
(letter values, geographical variants, a complete inventory of signs)
and we can reject the theory that these materials are Carian (canonical
Carian, at least).}


The Carian scripts, which have a common origin, have long puzzled scholars. Most of the letters resemble letters of the Greek alphabet, but their sound values are generally unrelated to the values of the Greek letters. This is unusual among the alphabets of Asia Minor, which generally approximate the Greek alphabet fairly well, both in sound and shape, apart from sounds which had no equivalent in Greek. However, the Carian sound values are not completely disconnected: 𐊠 /a/ (Greek Α), 𐊫 /o/ (Greek Ο), \textcarian{𐊰} /s/ (Greek Ϻ san), and \textcarian{𐊲} /u/ (Greek Υ) are as close to Greek as any Anatolian alphabet, and {\carian 𐊷}, which resembles Greek Β, has the similar sound /p/, which it shares with Greek-derived Lydian \textcarian{𐤡}.

Adiego (2007) therefore suggests that the original Carian script was adopted from cursive Greek, and that it was later restructured, perhaps for monumental inscription, by imitating the form of the most graphically similar Greek print letters without considering their phonetic values. Thus a /t/, which in its cursive form may have had a curved top, was modeled after Greek qoppa (Ϙ) rather than its ancestral tau (Τ) to become \textcarian{𐊭}. Carian /m/, from archaic Greek 𐌌, would have been simplified and was therefore closer in shape to Greek Ν than Μ when it was remodeled as 𐊪. Indeed, many of the regional variants of Carian letters parallel Greek variants: \textcarian{𐊥 𐅝} are common graphic variants of digamma, \textcarian{𐊨 ʘ} of theta, \textcarian{𐊬 Λ} of both gamma and lambda, \textcarian{𐌓 𐊯 𐌃} of rho, \textcarian{𐊵 𐊜} of phi, \textcarian{𐊴 𐊛} of chi, 𐊲 V of upsilon, and \textcarian{𐋏 𐊺} parallel \textcarian{Η 𐌇} eta. This could also explain why one of the rarest letters, \textcarian{𐊱}, has the form of one of the most common Greek letters.[13] 
However, no such proto-Carian cursive script is attested, so these etymologies are speculative.

Further developments occurred within each script; in Kaunos, for example, it would seem that \textcarian{𐊮} /š/ and \textcarian{𐊭} /t/ 
both came to resemble a Greek P, and so were distinguished with an extra line in one: \textcarian{𐌓} /t/, \textcarian{𐊯} /š/


\subsection{Description of Alphabets}

Now it remains to list the letters of the Carian alphabet. Table~\ref{tbl:carian} that follows lists the widely accepted signs and their variants. The last column shows the possible greek origin.\footcite[207ff]{adiego}

\begin{longtable}[L]{ 
>{\carian}l|
>{\carian}l| 
>{\carian}l| 
>{\carian}l| 
>{\carian}l| 
>{\carian}l| 
>{\carian}l| 
>{\carian}l| 
>{\panunicode}l| 
>{\panunicode}l|
p{3.5cm}}
\caption{Carian Alphabets}\label{tbl:carian}\\
\hline
 \rotatebox{90}{Hyllarima} 
&\rotatebox{90}{Euromos} 
&\rotatebox{90}{Mylasa} 
&\rotatebox{90}{Stratonicea} 
&\rotatebox{90}{Sinuri-Kildara} 
&\rotatebox{90}{Kaunos} 
&\rotatebox{90}{Iasos} 
&\rotatebox{90}{Mephis} 
&\rotatebox{90}{transliteration} 
&\rotatebox{90}{greek origin}\\
\hline
𐊠   &𐊠	 &𐊠	 &𐊠	  &𐊠	 &𐊠	    &𐌀	    &𐊠	       & a	  &Α    \\
𐊡   &--  &-- &« ? &𐋉[4]  &--    &𐋌 𐋍	&𐋌?[5]	   & 𐋌[5] & β    &Not a Greek value; perhaps a ligature of Carian \textcarian{𐊬𐊬}. \textcarian{𐊡} directly from Greek Β.\\
𐊣	&𐊣	 &𐊣	&𐊣	 &𐊣	&𐊣	&𐊣	&𐊣	&l	&Λ\\
𐊤	&𐊤	 &    &𐋐   &𐊤	&𐋈	&𐊤	&𐊤 𐋐?	&𐊤 Ε	&y	&Not a Greek sound value; perhaps a modified \textcarian{Ϝ}.\\
--	&--	 &--  &--  &	&𐊥	&𐊥	&𐊥	&r	&Ρ\\
𐋎	&--  &--  &𐊦   &𐊦	&𐋏	&𐊦	&𐊦	&λ & &Not a Greek value. \textcarian{𐋎} from \textcarian{Λ} plus diacritic, others not Greek\\
\panunicode{ʘ}	& \panunicode{ʘ}	 &\panunicode{ʘ}	&\panunicode{ʘ}	&\panunicode{ʘ} 𐊨?	&𐊨	&𐊨 \panunicode{ʘ}	&𐊨	&q	&Ϙ &\\

Λ	&Λ	&Λ &--	&Λ 𐊬	&Γ	&Λ	 &𐊬 Λ	&b  &  & Archaic form of Β, for example in Crete\\
𐊪	&𐊪	&𐊪	&𐊪	&𐊪 &𝈋	&𝈋	&𝈋	 &𝈋	    &m	&𐌌 \\
𐊫	&𐊫	&𐊫	&𐊫	&𐊫 &𐊫	&𐊫	&𐊫	 &o	    &Ο  &\\
𐊭	&𐊭	&𐊭	&𐊭	&𐊭	&𐌓	&𐊭	&𐊭	 &t	    &Τ  &\\
𐤭	&𐤭	&	𐤭	&𐤭     & 𐤭 𐌓   & 𐊯    & 𐤭 𐤧 𐌃   & 𐊮 Ϸ &{\panunicode š}& &Not a Greek value.\\
𐊰	&𐊰	&𐊰	&𐊰	&𐊰	   &𐊰	&𐊰	&𐊰	  &s	&Ϻ\\
&--&--&--&𐊱	&𐊱	&𐊱		&𐊱	&-- &? &\\
𐊲	&𐊲	&𐊲	&𐊲	&𐊲 V	&𐊲	&𐊲 V	&V 𐊲	&u	&Y &Υ /u/\\
𐊳	&--&--	&𐊳	&𐊳	&𐊳	&-- &--		&\panunicode ñ&\\
	&𐊴	&𐊴	&𐊛	&𐊴	&𐊴	&𐊴 𐊛	&𐊴 𐊛	&k̂	&&Not a Greek value. Maybe a modification of Κ, Χ, or \textcarian{𐊨}.\\
𐊵	&𐊵 &𐊜	&𐊵	&𐊵	&𐊵 𐊜	&𐊵	&𐊵	&𐊜 𐊵	&n	& {\panunicode 𐌍} Archaic form of Ν\\	
𐊷	&	&𐊷	&𐊷	&𐊷	&𐊷	&𐊷	&𐊷	&p	&Β\\
𐊸	&𐊸	&𐊸	&𐊸	&𐊸	&Θ	&𐊸	&𐊸 Θ	&ś	& &Not a Greek value. Perhaps from Ͳ sampi?\\
𝈣	&𐊹-	&⊲-	&𐊮-	&𐤧-	&𐊹	&𐊹	&𐊹	&i	&Ε, ΕΙ, or {\panunicode 𐌇}\\
𐋏	&𐋏	&𐋏	&𐊺	&𐊺	&𐊺	&𐊺	&𐊺	&e	&&Η, {\panunicode 𐌇}\\
\end{longtable}


The Carian alphabetic system has no parallels in other scripts found in Anatolia during the first millenium BC. In all these cases, 
the adaptation of the Greek alphabet was much more straightforward and natural: for sounds existing both in Greek and in
the local language, Greek letters with their sound values were used, and for sounds that were not present in Greek, new
letters were created, or Greek letters still available were recycled (for instance N for {\panunicode ñ} in Lycian). Although we find some
exceptions to this system of adaptation (for instance, the use of  for p (not b) in Lydian, or the use of W for i, not for e,
in Lycian), there are usually clear phonetic or formal reasons for all these exceptions.\footnote{Brun, P., L. Cavalier, K. Konuk et F. Prost, éd. (2013) : Euploia. La Lycie et la Carie antiques. Actes du colloque de Bordeaux 5, 6, 7 novembre 2009, Ausonius Mémoires 34, Bordeaux. 17-28} 


\paragraph{Unicode}
Carian was added to the Unicode Standard in April, 2008 with the release of version 5.1. It is encoded in Plane 1 (Supplementary Multilingual Plane).
The Unicode block for Carian is \unicodenumber{U+102A0–U+102DF}:


\begin{scriptexample}[]{Carian}
\unicodetable{carian}{"102A0,"102B0,"102C0,"102D0}
\end{scriptexample}



\PrintUnicodeBlock{./languages/carian.txt}{\carian}


\section{LaTeX}

The Carian script can be typeset using \latexe, using a number of control sequences provided by the |phd-scripts| package. Although most researchers in the field use different methods and Word, using \latexe can provide many efficiencies in inputting the text. With unicode fonts, the text can also be easily copied and entered into other documents. Like most scripts the following control sequences are provided. Linguistic and archaic texts are time consuming and difficult to set. 

\begin{docEnvironment}{CarianScript}{\oarg{color}}
\end{docEnvironment}

The \docAuxEnv{CarianScript} environment can be used to typeset the Carian script in a more friendly way, rather than looking at the right
unicode codepoint. The letters have menomonics based on their shapes or transliteration values.

\newenvironment{CarianScript}[1][blue]
{
 \def\A{\color{#1}{\carian\char"102A0}\xspace}
 \let\a\A
 \def\B{{\color{#1}\carian\char"102A1}\xspace}
 \let\b\B
 \def\Uuu{{\color{#1}\carian\char"102A4}\xspace} 
 \def\R{{\color{#1}\carian\char"102A5}\xspace}
 \def\Omega{{\color{#1}\carian\char"102B6}\xspace}
 \def\lamda{{\color{#1}\carian\char"102A6}\xspace}
 \def\s{{\color{#1}\carian\char"102B0}\xspace}
 \def\q{{\color{#1}\carian\char"102A8}\xspace}
 \def\m{{\color{#1}\carian\char"102AA}\xspace}
}
{}

\begin{texexample}{Writing in Carian}{ex:carian}
\begin{CarianScript}[red]
The character \B is thought to have been derived from the Greek shaped B, whereas \Omega is almost identical to the Greek form,
where a letter such as \R or \lamda\ldots

The character \q and /m/ is given by \m and /a/ by \a.
\end{CarianScript}
\end{texexample}

\section{The Carian Inscriptions from Egypt}

\subsection{Saqqara}
There are approximately a hundred graffiti in hieroglyphic and demotic, and one solitary example
i Carian. Some of the demotic ones are mason's marks and directions, but the vast majority are
the inscriptions of visitors, mainly in the form 'the worthy servant of Osiris the Ape, X son of Y,
his mother being Z; often these include several members of a single family, as with the Serapeum
stelae. Not all the visitors use the same description; some describe themselves as worthy servants
or souls of Osiris-Apis or occasionally of Thoth. In general the incised graffiti are later than the
ink ones, and the hieroglyphic ones late in the sequence, which perhaps lasts from the fourth
century B.C. into the Roman period; unfortunately, though several are dated to a regnal year, in
no instance is the king's name given. At one point, two adjoining blocks are built into the masonry,
one upside down, bearing a most bizarre Greek inscription.\cite{emery1970}

\begin{figure}[htbp]
\includegraphics[width=0.95\textwidth]{carian-saqqara}
\caption{Stelae with Carian texts; no.5 also bears Egyptian texts. From \protect\cite{emery1970}}
\end{figure}

The Carian Inscriptions from Egypt have been described by Masson and the later Ray.\footcite{ray1982}.
It may as
well be added here that the methods of writing Carian found in Egypt themselves vary,
and the system or systems used at Abu Simbel and Abydos are not the same as the one
used rather later by the Caromemphite community of Saqqara. This may be due to
reasons of chronology as much as to the places of origin of the Carians in question,
and in a settled community such as the Carian one in Memphis there is always the
likelihood of independent development. Ray in his 1982 paper, offered an attempted transliteration of most of the surviving Carian inscriptions from Egypt. 

The original corpus used by Ray was published by Masson. The corpus is numbered by Ray as 
M1\ldots--M$_n$. References are also sometimes extended with a lowercase letter, such as |M10a|. 

\begin{description}
\item[M37] 1. \textit{t}(?)]\textit{-d-a-ld-e-s}\\
2. ]e-a-m-\'s-h-e\\
3. a-d-o-s-h-a-r-k'-o-s \\
\end{description}

`Tdaldes, son of [. ]eam, the adoshark'os.' The first name is tentatively restored from M29; it is a
pity that one cannot be certain about this, as it would confirm that the nominative of such names
ended in -es. The title (?) at the end is discussed by Gusmani, \textit{Kadmos} 17 (1978), 74, since it occurs
in a variant spelling on one of his bronze bowls: cf. also Meriggi, BiOr 37 (I980), 35 who discusses this extensively. 


\begin{description}
\item[M55] 1. \textit{j-\'e-p-s-a-d}\\
           2. \textit{p-u-o-j-\'s a-\'s-r-\'s}\\
           3. \textit{u-r-s-j-a-h-29-h-e} 
\end{description}



Texts published in O. Masson and J. Yoyotte, \textit{Objets pharaoniques a inscription
carienne}, Cairo 1956. Abbreviated MY. He described MY text A-- MY text M. 

The next texts discussed were those collected by V. V. \v{S}evoro\v{s}kin, Issledovanija po desifrovke karijskih nadpisej,
Moscow 1965. Abbreviated \v{S}ev.

In his paper Ray compared the Egyptian to the Carian and provided a detailed study.
of the inscriptions.

\subsection{Graffiti from Abu Simbel}

Graffiti from Abu Simbel were published by O.Masson in \textit{Hommages a la memoire de
S. Sauneron}, Cairo, 1979), 35-49. Abbreviated in the literature as |AS|. 

\begin{description}
\item[AS 1] \textit{p-a-r-\'s-o(?)-d(?)-o-u}\\
   2. ]\textit{-o-j}
\end{description}

Ray writes that the first name may be part of the Para- family, but gives no further commentary.



\^{S}jk`urq. 

\subsection{Labraunda}
Labraunda\footnote{Ancient Greek: Λάβρανδα Labranda or Λάβραυνδα Labraunda} is an  archaeological site five kilometers west of Ortaköy, 
Muğla Province, Turkey, in the mountains near the coast of Caria. In ancient times, it was held sacred by Carians and Mysians alike. 
The site amid its sacred platanus trees\footnote{Herodotus, v.119} was enriched in the Hellenistic style by the Hecatomnid dynasty of Mausolus, satrap (and virtual king) of Persian Caria (c. 377 – 352 BCE), and also later by his successor and brother Idrieus; Labranda was the dynasty's ancestral sacred shrine. The prosperity of a rapidly hellenised Caria occurred in the during the 4th century BCE.[2] Remains of Hellenistic houses and streets can still be traced, and there are numerous inscriptions. The cult icon here was a local Zeus Labrandeus (Ζεὺς Λαβρανδεύς), a standing Zeus with the tall lotus-tipped scepter upright in his left hand and the double-headed axe, the labrys, over his right shoulder. The cult statue was the gift of the founder of the dynasty, Hecatomnus himself, recorded in a surviving inscription.

A grafito from Lambraunda was described in an interesting article in \textit{Kadmos} by Karlsson Lars and Henry Olivier (2009).\footcite{Karlsson2009}

\begin{figure}[htbp]
 \includegraphics[width=\textwidth]{labraunda}
 \end{figure}

\begin{quotation}
 The excavation trenches were laid out in this area, which was used as barracks, i.e.
the rooms in which the soldiers on duty had their living quarters. On
the threshold leading into Room 2, on July 12, 2007, we discovered
the base of an Attic black-gloss bowl. It was decorated on the inside
with palmettes joined by large circle segments. On the underside of the
bowl a Carian graffi to was found, perhaps a name. The base profile
and decoration date it to 375--350 B.C. based on a comparison with
similar examples from the Maussolleion in Halikarnassos,4
 as well
 as already published bases from Labraunda.
 \end{quotation}

The graffito has been named C.La 1, this being the first real Carian text discovered in Labraunda
\begin{description}
  \item[C.La 1]  \textit{bziom}
\end{description}

\section{The Carian Stonemasons}

The Persepolis Treasury Tablets specifically mentioned the Carians as being stonemasons (Hallock 1960:99, Cameron 1965). In the Treasury at Persepolis there are one hundred and eighty-one mason's marks, consisiting of ten basic signs, appear on the earliest stonework in Apadana at Suza. These ten basic signs have parallels amongst the thirty basic signs used at Persepolis and were attributed to Anatolian workmen by Nylander (1974:216-7;1975:322-3). Although the mason's marks differ from sculptors' marks, both have affinities with the Lydian, Aramaic or South Arabian alphabets (Roaf 1983:92-93).

The \enquote{Lydian Wall} at Sardis has mason's marks from Carian, Lydian and Phrygian alphabets (Gusmani 1988:33), while Gosline (1988:59) cites the use of sixty-nine Carian alphabetic masons' marks. Excavations at Carian Labraunda revealed seven mason's marks incised on limestone ashlars\footnote{Ashlar is finely dressed (cut, worked) stone, either an individual stone that has been worked until squared or the structure built of it} (S\={a}flund 1953). Carian alphabetic marks have also been found engraved in stone quarries in Egypt. At Elephantine, four hundred and twenty-eight alphabetic marks have been recognized by Gosline (1992;1998:60).


\section{Postscript}

Writing a book like this one, presents many challenges. One is how to write efficiently, as lingusitic texts are full of citations and also of specialized symbols, diagrams and enumerations.

The choice of programming macros for most of the languages and transliterations, has a big advantage, as one can use the macro names more efficiently than hunting to find the right unicode character. Cut-and-paste from older books is next to impossible in pre-unicode days. Still many journals suffer from this. 

\section{Phoenician}
\label{s:phoenician}
\arial

The Phoenician alphabet and its successors were widely used over a broad area surrounding the Mediterranean Sea.

\let\phoenician\lycian

\begin{scriptexample}[]{Phoenician}

\unicodetable{phoenician}{"10900,"10910}

\end{scriptexample}

The Phoenician alphabet, called by convention the Proto-Canaanite alphabet for inscriptions older than around 1200 BCE, is the oldest verified consonantal alphabet, or abjad.[1] It was used for the writing of Phoenician, a Northern Semitic language, used by the civilization of Phoenicia. It is classified as an abjad because it records only consonantal sounds (matres lectionis were used for some vowels in certain late varieties).

Phoenician became one of the most widely used writing systems, spread by Phoenician merchants across the Mediterranean world, where it evolved and was assimilated by many other cultures. The Aramaic alphabet, a modified form of Phoenician, was the ancestor of modern Arabic script, while Hebrew script is a stylistic variant of the Aramaic script. The Greek alphabet (and by extension its descendants such as the Latin, the Cyrillic, and the Coptic) was a direct successor of Phoenician, though certain letter values were changed to represent vowels.

\begin{figure}[ht]
\includegraphics[width=\textwidth]{./images/phoenician.jpg}
\captionof{figure}{
Phoenician votive inscription from Idalion (Cyprus), 390 BC. BM 125315 from The Early Alphabet by John F. Healy.}
\end{figure}

As the letters were originally incised with a stylus, most of the shapes are angular and straight, although more cursive versions are increasingly attested in later times, culminating in the Neo-Punic alphabet of Roman-era North Africa. Phoenician was usually written from right to left, although there are some texts written in boustrophedon.


\PrintUnicodeBlock{./languages/phoenician.txt}{\phoenician}


\newpage
\section{Palmyrene}
\idxlanguage{Palmyrene}
\arial

Palmyrene is the very widely attested Aramaic dialect and script
of Palmyra in the Syrian desert. The texts date from the midfirst century to the destruction of Palmyra by the Romans in AD 272. Palmyra in the Roman period was a major trading centre.
\medskip

\begin{figure}[ht]
\centering

\includegraphics[width=0.9\textwidth]{./images/palmyrene.jpg}
\captionof{figure}{\protect\arial Limestone bust with Palmyrene inscription. Palmyra late 2nd century AD. BM WA 102612}

\end{figure}

\medskip
The longest of the Palmyrene texts, is the bilingual  taxation tariff written for the year 137 AD in Palmyrene Aramaic and Greek.\footnote{For more details see:MILIK J.T., Dédicaces faites par des dieux (Palmyre, Hatra, 
Tyr) et de thiases sémitiques à l'époque romaine, Paris 1972; ROSENTHAL R., Die 
Sprache der palmyrenischen Inschriften, Leipzig 1936; STARK J.K., Personal Names in 
Palmyrene Inscriptions, Oxford 1971; DRIJVERS H.J.W., The Religion of Palmyra, 
Leiden 1976; TEIXIDOR J., 'Palmyre et son commerce d'Auguste à Caracalla', in 
Semitica 34, (1984) 1-127.  } Trade connections 
took the Palmyrene script to other places, some not far away, such as Dura Europos on the Euphrates, butothers at a great distance. A particular inscription is from South Shields, Roman Arbeia, in the north-east of England, carved on behalf of a Palmyrene mechant for his deceased wife and probably dating to the early third century AD. 

The Palmyrene script existed in two main varieties, a monumental and a cursive one, though the latter is little known and the evidence  mostly from Palmyra itself. The Syriac script of Edessa in southern Turkey, is often regarded as derived or closely related to the Palmyrene---similarities are found in the letters: ', b, g, d, w, h, y, k, l, m, n, `, r and t---though a strong case can also be made for connecting Syriac with a northern Mesopotamian script-family represented principally in texts from Hatra, a city more or less contemporary with Palmyra in Upper Mesopotamia. 


\begin{figure}[ht]
\includegraphics[width=\textwidth]{./images/regina-epigraph.jpg}
\caption{It was customary for Palmyrenes to offer bilingual texts (Greek or Latin) on funerary monuments. The final line of Regina's epitaph is Barates' personal lament in Palmyrene: Regina, freedwoman of Barate, alas. (See \href{http://www2.cnr.edu/home/araia/regina.html}{regina}.)}
\end{figure}

A good article on the classification of Aramaic languages can be found in \textit{The Aramaic language and Its Classification} by Efrem Yildiz.\footnote{\url{http://www.jaas.org/edocs/v14n1/e8.pdf}}

I could not find any fonts for Palmyrene paid or unpaid.






\cxset{quotation font-size=\normalsize,
       quote font-size=\normalsize}


\section{Mandaic}
\label{s:mandaic}
\newfontfamily\mandaic{NotoSansMandaic-Regular.ttf}


The Mandaic script is used to write a dialect of Eastern Aramaic, which, in its classical
form, is currently used as the liturgical language of the Mandaean religion. A living language descended
from Classical Mandaic is spoken by a small number of people living in and around Ahvaz, Khūzestān,
in southwestern Iran; speakers are also found in emigrant communities in Sweden, Australia, and the
United States. There is a considerable amount of Iranian influence on the lexicon of Classical Mandaic,
and Arabic and Persian influence on the grammar and lexicon of the contemporary dialect. The script
itself is likely derived from the Parthian chancery script.

Mandaic is a right-to-left script. It is a true alphabet, using letters regularly for vowels
rather than as the \emph{matres lectionis} from which they derived. The three diacritical marks are used in
teaching materials to differentiate vowel quality. At present, at least, the rule is that they may be omitted
from ordinary text. In this regard they are very like the Arabic fatha, kasra, and damma or the Hebrew
vowel points.

The only so far I could find that can display the script is the Google \idxfont{NotoSansMandaic.ttf}.

\begin{scriptexample}[]{Mandaic}
\bgroup
\unicodetable{mandaic}{"0840,"0850}
\egroup
\end{scriptexample}

In 1943, Lady Ethel Drower published extracts from several magic “recipe books” that served the writers of amulets in Baghdad in the early 20th century, in particular from two manuscripts in her possession, DC 45 and DC 46.

\begin{figure}[hb]
\centering

\includegraphics[height=4cm]{./magic-letters.jpg}
\includegraphics[height=4cm]{./45-453.jpg}
\includegraphics[height=4cm]{./36-448.jpg}

\captionof{figure}{Mandaic Incantation vessels. The left image is from \protect\href{http://thesacredalphabet.blogspot.ae/}{thesacredalphabet}, whereas the last two are from \protect\href{http://www.archaeological-center.com/en/auctions/45-453}{archaeological-center} }
\end{figure}

 While Drower, following her native informants, entitled the work ‘A Mandæan Book of Black Magic’, the manuscripts themselves contain a wide range of formulae for amulets and talismans for various purposes, as Drower herself was well aware. Alongside spells for healing, protection and success, we find others for enflaming love or stirring up enmity.

 The manuscripts themselves appear to have been copied in the late 19th or early 20th centuries; in particular, DC 46, a substantial codex of 264 sides, is written on an extremely modern “clean” paper. DC 45 is written on a rougher paper and appears to be somewhat earlier. It is also more fragmentary, and contains several leaves that were copied by a different hand and inserted into the main part of the manuscript at a later date, though it is clear from their contents that they were intended to replace pages that had been worn or damaged, as they begin and end exactly as required by the preceding and following pages. As it survives today, DC 45 is also considerably shorter than DC 46; however, it also contains several spells that are not found in DC 46.\footnote{\protect\href{http://www.academia.edu/8294938/Arabic_Magic_Texts_in_Mandaic_Script_A_Forgotten_Chapter_in_Near-Eastern_Magic}{Magic Texts}}

Lady Drower inform us that among the Mandaens:

\begin{quote}
Writing in itself is a magic art, and the alphabet is sacred.
Each letter is supposed to invoke a spirit of light and is a thing of power. It is a practice to write the letters separately and to sleep each night with a letter beneath the pillow. If the sleeper sees in a dream something which will enlighten him, the letters upon which he slept that night is taken to a silversmith and a replica in gold or silver is made and worn around the neck as amulet See Mandaic Incantation Texts by Edwin M Yamauchi.
\end{quote}













\newfontfamily\aegyptus{AegyptusR.ttf}
\def\texthiero#1{{\color{black!95}\hiero #1}}
\ExplSyntaxOn
\NewDocumentCommand\basket{ o }{
\bgroup
  \aegyptus %use aegyptus font
  \scalebox{7}{\char"F300C} /nb/, `basket';
\egroup
}

\NewDocumentCommand\water{ o }{
\bgroup
  \aegyptus
  \scalebox{7}{\char"F300B} /ne/
\egroup
}


\ExplSyntaxOff

%\bgtitle{Aegyptian\\ Hieroglyphics}{Aegyptian Hieroglyphics}

\chapter{Aegyptian Hieroglyphics}
\label{ch:hieroglyphics}

\index{fonts>Aegyptus}\index{Aegyptus (font)}
\index{fonts>Hieroglyphics}\index{languages>hieroglyphics}

\newfontfamily\hiero{NotoSansEgyptianHieroglyphs-Regular.ttf}

Hieroglyphic writing appeared in Egypt at the end of the fourth millennium bce. The writing
system is pictographic: the glyphs represent tangible objects, most of which modern
scholars have been able to identify. A great many of the pictographs are easily recognizable
even by non-specialists. Egyptian hieroglyphs represent people and animals, parts of the
bodies of people and animals, clothing, tools, vessels, and so on.

\basket

The water is written as,

\water[test]  

Hieroglyphs were used to write Egyptian for more than 3,000 years, retaining characteristic
features such as use of color and detail in the more elaborated expositions. Throughout the
Old Kingdom, the Middle Kingdom, and the New Kingdom, between 700 and 1,000 hieroglyphs
were in regular use. During the Greco-Roman period, the number of variants, as
distinguished by some modern scholars, grew to somewhere between 6,000 and 8,000.

Hieroglyphs were carved in stone, painted on frescoes, and could also be written with a reed
stylus, though this cursive writing eventually became standardized in what is called \emph{hieratic}\index{hieratic}
writing. Unicode does not encode the hieratic forms separately, but considers them as cursive forms of the hieroglyphs encoded block.

The Demotic script and then later the Coptic script replaced the earlier hieroglyphic and
hieratic forms for much practical writing of Egyptian, but hieroglyphs and hieratic continued
in use until the fourth century ce. An inscription dated August 24, 394 ce has been
found on the Gateway of Hadrian in the temple complex at Philae; this is thought to be
among the latest examples of Ancient Egyptian writing in hieroglyphs

\begin{figure}[htb]
\includegraphics[width=\textwidth]{./images/bookofthedead.jpg}
\caption{Extract from \textit{Book of the Dead.}}
\end{figure}


Specialists distinguish six stages in the development of Egyptian:

\begin{enumerate}
\item Old Egyptian of the third millenium BC. These are known from the Pyramid Texts and they represent the most archaic form of Egyptian, and from funerary inscriptions of the fifth and sixth Dynasties.
\item Classical or Middle Egyptian, covering the period 2240-1780 BC (Dynasties 9 to 12).
\item Late Classical: 1780-1350 BC (Dynasties 13 to 18). The \textit{The Book of the Dead} was compiled in this period.
\item Late Egyptian: fourteenth to eighth centuries BC (Dynasties 18 to 24).
\item Demotic: eighth century BC to fifth century AD.
\item Coptic.
\end{enumerate}


The hieroglyphic writing was deciphered by Champollion in the 1820s. Several thousand hieroglyphs are known, many of them being very rare or \textit{hapax legomena}. The hieroglyphic script is sub-divided into:

\begin{description}
\item [Ideograms] These represent objects in purely graphic fashion with no phonetic elements; e.g:
day, sun
house

\item[Phonograms] These are signs indicating pronunciation; e.g. mouth In the course of the centuries such symbols were converted to represent the sign /r/, and a series of such single valued signs ultimately produces an alphabet. At no stage before Coptic are vowels notated. To faciliate pronunciation, modern practice is to vaocalize the Egyptian consonants with /e/. Thus, \textit{pr} is read as /per/;sn `brother', as /sen/, nfr `beautiful' as \texttt{/nefer/}.

\end{description}


In hieroglyphic texts, these drawings are not only simply arranged in sequential order, but also grouped on top of and next to each other. This rather complicates matters trying to register and reproduce hieroglyphic texts using a computer.

\section{Computer Typesetting}

Typesetting hieroglyphics with computers presents a number of problems. First is the method of inputting the characters and second the various methods required to stack hieroglyphics, the direction of writing which can be one of four different directions.

When the first computers were introduced in Egyptology in the late 1970s and the beginning of the 1980s, the graphical capacity of the machines was still in its infancy. Early attempts to register the hieroglyphic pictorial writing on computer therefore chose an encoding system to do this, using alphanumeric codes to represent or replace the graphics. To prevent many people from reinventing the wheel, during the first "Table Ronde Informatique et Egyptologie" in 1984 a committee was charged with the task to develop a uniform system for the encoding of hieroglyphic texts on computer. The resulting Manual for the Encoding of Hieroglyphic Texts for Computer-input (Jan Buurman, Nicolas Grimal, Jochen Hallof, Michael Hainsworth and Dirk van der Plas, Informatique et Egyptologie 2, Paris 1988), simply called Manuel de Codage, presents an easy to use and intuitive way of encoding hieroglyphic writing as well as the abbreviated hieroglyphic transcription (transliteration). The system proposed by the Manuel de Codage has since been adopted by international Egyptology as the official common standard for registering hieroglyphic texts on computer. Mark-Jan Nederhof proposed an enhanced encoding scheme to remove many of the limitations in the Manuel de Codage.

\pkgname{HieroTeX} is a \latexe package developed by to typeset hieroglyphic texts and still works well. The advantages of using \tex is of course its excellent typesetting capabilities and the usage of macros. Although inputting the texts as MdC codes is not that difficult, repeating the same codes over and over can be avoided with easily constructed simple substitution macros. 

\subsection{fonts}

One of the best fonts I came across is \idxfont{Aegyptus} from \url{http://users.teilar.gr/~g1951d/}\footnote{The site also has fonts for Aegean Numbers, Ancient Greek Musical Notation, Ancient Greek Numbers, Ancient Roman Symbols, Arkalochori Axe, Carian, Cypriot Syllabary, Dispilio tablet, Linear A, Linear B Ideograms, Linear B Syllabary, Lycian, Lydian, Old Italic, Old Persian, Phaistos Disc, Phoenician, Phrygian, Sidetic, Troy vessels’ signs and Ugaritic. Cretan Hieroglyphs and Cypro-Minoan script(s) are offered in separate files.}. The font provides all the unicode characters and also offers an additional number of glyphs that are not in the Unicode standard. The font uses the Unicode Private Use Areas to encode the glyphs. 

Another font is the Noto Egyptian Hieroglyphics from Google. This is a lightweight font with the symbols in their proper unicode slots. Mark-Jan Nederhof's \idxfont{NewGardiner} font is another one with support only for the Gardiner set. The codepoint mappings are incorrect, as the font has been  
encoded to EGPZ. The font is similar to the Aegyptus font, however it is just transposed and not recommended unless it is transposed. 

The editor software JSesh\footnote{\protect\url{http://jsesh.qenherkhopeshef.org/}} also provides a free font |JSeshFont.ttf|. This offers a correctly mapped unicode and is another good alternative. The symbols are drawn somewhat simpler and is just a matter of taste what you want to use.

My recommendation is for short demonstration purposes, the Noto font is to be preferred while for more serious work the Aegyptus font will be more useful. Using Lua the font can be transposed automatically to allow the use of commands that refer to unicode numbers. Another advantage of the Aegyptus font is that the glyphs are named with their Gardiner numbers, so it is somewhat easier to programmatically access them by name.\footnote{Unicode does not name the glyphs, but simply calls the Egyptian Hieroglyph $n$. } 

\medskip

\ifxetex
\bgroup
\centering 
\font\myfont = "Aegyptus"
\scalebox{7}{\myfont\XeTeXglyph 201}
\scalebox{7}{\myfont\XeTeXglyph 203}
\scalebox{7}{\myfont\XeTeXglyph 163}
\scalebox{7}{\myfont\XeTeXglyph 164}
\scalebox{7}{\myfont\XeTeXglyph 165}
\scalebox{7}{\myfont\XeTeXglyph 168}
\captionof{table}{Example of Egyptian Hieroglyphics typeset with the \textit{Aegyptus} font.} 
\egroup
\fi

\ifluatex
\bgroup
\centering 
\aegyptus
\scalebox{7}{\char"F300C}
\scalebox{7}{\char"F3001}
\scalebox{7}{\char"F3010}
\scalebox{7}{\char"F308B}
\scalebox{7}{\char"F3097}
\scalebox{7}{\char"F3091}
\captionof{table}{Example of Egyptian Hieroglyphics typeset with the \textit{Aegyptus} font.} 
\egroup

\fi


\subsection{Unicode Block}

Egyptian hieroglyphs is a Unicode block containing the Gardiner's sign list of Egyptian hieroglyphics.
The code points, in the range |0x13000| to |0x1342E|, are available starting from
\href{http://unicode.org/charts/PDF/U13000.pdf}{Unicode 5.2}

\begin{scriptexample}[]{Hieroglyphic}
\bgroup
\unicodetable{hiero}{"13000,"13010,"13020,"13030,"13040,"13050,"13060,"13070,%
"13080,%
"13090,"130A0,"130B0,"130C0,"130D0,"130E0,"130F0,%
"13100,"13110,"13120,"13130,"13140,"13150,"13060,"13070,"13080,"13090}
\egroup
\end{scriptexample}

\subsection{Gardiner's classification}

The standard reference on Egyptian hieroglyphics is Gartiner's Sign List, which lists common Egyptian hieroglyphs. These are grouped in categories from |A-Aa|. Each category represents a theme for example category A, is "man and his occupations". Based on this list ``Queen with flower'' is denoted as \texttt{B7}. 

\subsection{Character Names} 

Egyptian hieroglyphic characters have traditionally been designated in
several ways:

\begin{enumerate}
\item  By complex description of the pictographs: \texttt{GOD WITH HEAD OF IBIS}, and so forth.
\item By standardized sign number: C3, E34, G16, G17, G24.
\item For a minority of characters, by transliterated sound.
\end{enumerate}

The characters in the Unicode Standard make use of the standard Egyptological catalog
numbers for the signs. Thus, the name for {\hiero\char"130F9} |U+13049| egyptian hieroglyph e034 refers
uniquely and unambiguously to the Gardiner list sign E34, described as a “{\aegean DESERT HARE}” ({\hiero \char"130FA}) and used for the sound “wn”. The Unicode catalog values are padded to three places with
zeros, so where the Gardiner classification is shown as \texttt{E34}, the unicode value is \texttt{E034}. 

Names for hieroglyphic characters identified explicitly in Gardiner 1953 or other sources as
variants for other hieroglyphic characters are given names by appending “A”, “B”, ... to the sign number. In the sources these are often identified using asterisks. Thus Gardiner’s G7,
G7*, and G7** correspond to U+13146 egyptian sign g007 {\hiero \char"13147}, U+13147 egyptian sign g007a, and U+13148 egyptian sign g007b, respectively.



\begin{longtable}{>{\Large}lll>{\ttfamily}l}
{\hiero \char"13000}&A1-A70 & Man and his occupations &U+13000-1304F\\
{\hiero \char"13050}&B1-B9  &Woman and her occupations &U+13050-13059\\
{\hiero \char"1305A} &C1-C24 &Anthropomorphic Deities &U+1305A-13075\\
{\hiero \char"13076} &D1-D67 &Parts of the Human Body &U+13076-130D1\\
{\hiero \char"130D2} &E1-E38 &Mammals &U+13076-130D1\\
{\hiero \char"130FE}  &F1-F53	&Parts of Mammals &U+130FE-1313E\\
{\hiero\char"1313F}	&G1-G54	&Birds &U+1313F-1317E\\
{\hiero \char"1317F}	&H1-H8	&Parts of Birds &U+1317F-13187\\
\texthiero{\char"13188}	&I1-I15	&Amphibious Animals, Reptiles, etc. &U+13188-1319A\\
\texthiero{\char"1319B}	&K1-K8	&Fishes and Parts of Fishes &U+1319B-131A2\\
\texthiero{\char"131A3}	&L1-L8	&Invertebrata and Lesser Animals &U+131A3-131AC\\
\texthiero{\char"131AD}	&M1-M44	&Trees and Plants &U+13AD-131EE\\
\texthiero{\char"131EF}	&N1-N42	&Sky, Earth, Water &U+131EF-1321F\\
\texthiero{\char"13250}	&O1-O51	&Buildings and Parts of Buildings &U+13250-1329A\\
\texthiero{\char"1329B}	&P1-P11	&Ships and Parts of Ships &U+1329B-132A7\\
\texthiero{\char"132A8}	&Q1-Q7	& Domestic and Funerary Furniture &U+132A8-132AE\\
\texthiero{\char"132AF}	&R1-R29	&Temple Furniture and Sacret Emblems &U+132AF-132D0\\
\texthiero{\char"132D1}	&S1-S46	&Crowns, Dress, Staves, etc. &U+132D1-13306\\
\texthiero{\char"13307}	&T1-T36	&Warfare, Hunting, Butchery &U+13307-13332\\
\texthiero{\char"13333}	&U1-42	&Agriculture, Crafts and Professions &U+13333-13361\\
\texthiero{\char"13362}	&V1-V40a	&Rope, Fibre, Baskets, Bags, etc. &U+13362-133AE\\
\texthiero{\char"133AF}	&W1-W25	&Vessels of Stone and Earthenware &U+133AF-133CE\\
\texthiero{\char"133CF}	&X1-X8a	&Loaves and Cakes &U+133CF-133DA\\
\texthiero{\char"133DB}	&Y1-Y8	&Writing, Games, Music &U+133DB-133E3\\
\texthiero{\char"133E4}	&Z1-Z16H	&Strokes, Geometrical Figures, etc. &U+133E4-1340C\\
\texthiero{\char"1340D}	&Aa1-Aa32	&Unclassified &U+1340D-1342E\\
\end{longtable}

I particularly like the crocodile sign \def\crocodile{\color{teal}{\Huge\texthiero{\char"13188}}} {\crocodile}, as it is applicable to describe people in my field of work. 

\begin{scriptexample}[]{Woman and her occupations}
\unicodetable{hiero}{"13050}
\end{scriptexample}

\section{Positioning}

One of the core assumptions of any hieroglyphic encoding or mark-up scheme following the MdC is that signs and groups of signs maybe positioned next to each other or above each other. The former is indicated by the operator * and the latter by :. One may also use -, which functions as * for horizontal texts and as : for vertical text. 

In some dialects of the MdC relative positioning has been extended by the use of the |&| operator. This is used to form a kind of ligature, such as |D&t| can be defined to represent the \textit{Cobra at rest} sign I10 with sign X1 underneath, as follows:

\begin{center}
{\hiero\HUGE
       \mbox{\rlap{\char"133CF}\char"13193\hfill\hfill}\\
       {\large|insert[bs](I10,X1)|}

\mbox{\rlap{\scalebox{0.5}{\char"133E3}}\char"13193\hfill\hfill}\\
 	
}
\end{center}

This is only a partial solution and to automate it via kerning tables, will require hundreds of entries in the kerning tables. It will also need constant modifications as researchers discover new combinations. A better approach and which is easily applied to \tex based systems would be to adopt Nederhof's method of creating a new command |insert[bs](I10,X1)|. 

In \tex one could simply define a command \cmd{\insert} with one optional argument to handle the positioning. The positioning uses the letters [b,t,s,e] to position the glyph. the letters s and e stand for start and end, whereas b,t for bottom and top respectively. When there are only two symbols involved, this is not such a difficult operation, but when three or more symbols are to be grouped and kerned together, inserting with some form of scaling is necessary.

\subsection{Enclosures}

Enclosures. The two principal names of the king, the \emph{nomen} and \emph{prenomen}, were normally
written inside a \emph{cartouche}: a pictographic representation of a coil of rope.

In the Unicode representation of hieroglyphic text, the beginning and end of the cartouche
are represented by separate paired characters, somewhat like parentheses. The Unicode manual states that `rendering of a full cartouche surrounding a name requires specialized layout software', which is of course an easy task for \tex.

\begin{macro}{\cartouche}
The commands \cmd{\cartouche} and \cmd{\cartouche}, from Peter Wilson's \pkgname{hierglyph} package have been used for many years to demonstrate the use of hieroglyphics with \latexe. 
\end{macro}

There are a several characters for these start and end cartouche characters, reflecting various styles for the enclosures.

\cartouche{{\hiero \char"13147}$sin^{2} x + cos^{2} x = 1$}
\Cartouche{{\hiero \char"13147}$sin^{2} x + cos^{2} x = 1$}

Unicode:{\hiero 𓇓𓏏𓊵𓏙𓊩𓁹𓏃𓋀𓅂𓊹𓉻𓎟𓍋𓈋𓃀𓊖𓏤𓄋𓈐𓎟𓇾𓈅𓏤𓂦𓈉 }

\textpmhg{\HQ} 

\cartouche{\pmglyph{K:l-i-o-p-a-d:r-a}}
%\translitpmhg{\HK\Hl\Hi\Ho\Hp\Ha\Hd\Hr\Ha}

\printunicodeblock{./languages/hieroglyphics.txt}{\hiero}
\printunicodeblock{./languages/hieroglyphics-13100.txt}{\hiero}
\printunicodeblock{./languages/hieroglyphics-13200.txt}{\hiero}
\printunicodeblock{./languages/hieroglyphics-13300.txt}{\hiero}
\printunicodeblock{./languages/hieroglyphics-13400.txt}{\hiero}
\section{Numerals}

Egyptian numbers are encoded following the same principles used for the
encoding of Aegean and Cuneiform numbers. Gardiner does not supply a full set of
numerals with catalog numbers in his Egyptian Grammar, but does describe the system of
numerals in detail, so that it is possible to deduce the required set of numeric characters.

Two conventions of representing Egyptian numerals are supported in the Unicode Standard.
The first relates to the way in which hieratic numerals are represented. Individual
signs for each of the 1s, the 10s, the 100s, the 1000s, and the 10,000s are encoded, because in
hieratic these are written as units, often quite distinct from the hieroglyphic shapes into
which they are transliterated. The other convention is based on the practice of the \emph{Manual
de Codage}, and is comprised of five basic text elements used to build up Egyptian numerals.
There is some overlap between these two systems.

%% Needs some work to get it into LuaLaTeX
%% omitted for the time being
%\ifxetex
%\begin{texexample}{TeXeXglyph}{ex:xetexglyph}
%\raggedright
%\font\myfont = "Aegyptus"
%\setcounter{glyphcount}{136}
%
%\whiledo
%{\value{glyphcount}<\XeTeXcountglyphs\myfont}
%{\arabic{glyphcount}:~
%{\myfont\XeTeXglyph\arabic{glyphcount}}\quad
%\stepcounter{glyphcount}}
%\end{texexample}
%\fi

\section{Input Methods}

If you writing a document with a lot of hieroglyphics inputting of hieroglyphics can be problematic. Most researchers in the field will use special keyboards or editors. They also use MS/Word or OpenOffice. They can both be coerced to produce reasonable documents, but with \tex obviously better results can be achieved. One such editor is \href{http://jsesh.qenherkhopeshef.org/}{jsesh}. 

Developing a parser either through TeX or Lua or even better with a language such as Go, is not difficult. The advantage of TeX are its boxes, as overlapping of hieroglyphic signs can easily be achieved. The difficulty lies with determining the scaling factors to be used. However, once someone has a table with all the glyph sizes strategies can be developed for overlapping. Such a system is akin to Knuth's math typesetting algorithms. 

\begin{luacode*}
    local h = {}
          h = dofile("hiero.lua")
    local options = {style="block",
                     echo=true,
                     direction="RL",
                     size = "\\Huge",
                     color = "green",
                     headings = "captionof{figure}"  -- section/tablecaption/figurecaption
                     }
   -- prints full symbol list
   h.printgardiner(t,options)

   tex.print("\\par")
   local options = {style="block",
                     echo=true,
                     heading="\\par",
                     direction="RL",
                     color = "teal",
                     scale = 8}

   h.printhierochar("hiero","1317D",options)
   h.printhierochar("hiero","13000",{direction="RL",
                                        color = "teal",
                                        scale = 8})
   h.printhierochar("hiero","13003",{direction="LR",
                                        color = "teal",
                                        scale = 1})
   h.parseMdC([[M23-X1-R4-X8-Q2-D4-W17-R14-G4-R8-O29-
               V30-U23-N26-D58-O49-Z1-F13-N31-V30-N16-
               N21-Z1-D45-N25!]])

   tex.print("\\par")
   h.printgardinercat("B")

\end{luacode*}

\newcommand\hierochar[2][direction = "LR",
                         color     = "teal",
                         scale     = 1]{% 
               \luaexec{
                h = h or {}
                h = require("hiero.lua")  
                h.parseMdC(#2,{#1})}}
               
\newcommand\printhierochar[3][direction = "LR",
                              color     = "teal",
                              scale     = 4]{% 
               \luaexec{
                h = h or {}
                h = require("hiero.lua")  
                h.printhierochar(#2,#3,{#1})}}

This file just tests the various commands available for manipulating hieroglyphics. We tried to 
generalize the commands, so they can be re-used for other type of hieroglyphics.

{
\hierochar{"A1-A2-A3!"}

\centering 

\def\options{direction = "LR",
             color     = "teal",
             scale     = 7}

\def\fontname{"hiero"}

\def\hierochar#1{\printhierochar[\options]{\fontname}{#1}}
}


\begin{scriptexample}[]{Some Example}
Sometimes kerning might be required, especially if the
glyphs are scaled.This is easily achieved with a \cmd{\kern}
command and a suitable skip dimension.

\medskip

\bgroup
\fboxsep=0pt\fboxsep.4pt
\def\options{direction = "RL",
             color     = "black!95",
             scale     = 5}
\centering

\color{teal}
\fbox{\hierochar{"13051"}}
\kern-4mm
\hierochar{"13003"}
\def\options{direction = "LR",
             color     = "black!95",
             scale     = 5}
\fbox{\hierochar{"13003"}}\color{red}
\kern-4mm
\hierochar{"13051"}
\color{black!95}
\egroup
\begin{verbatim}
\centering
\hierochar{"13051"}
\kern-4mm
\hierochar{"13003"}
\def\options{direction = "RL",
             color     = "black!95",
             scale     = 5}
\hierochar{"13003"}
\kern-4mm
\hierochar{"13051"}
\end{verbatim}
\end{scriptexample}

A bit of a diversion is appropriate at this point. Our attempt after the historical overview, is to provide some routines for the capturing and display of hieroglyphic texts using LuaTeX. This involves getting low level information from the system regarding fonts. 

\begin{figure}[ht]
\begin{minipage}{0.45\textwidth}
\centering
\includegraphics[width=0.6\textwidth]{./images/fontforge.jpg}
\end{minipage}
\begin{minipage}[t]{0.45\textwidth}
\caption{Viewing font information with fontforge.}
\end{minipage}
\end{figure}

For each glyph, we are interested to get its unicode number, the position in the font table, its name and most importantly the font metrics. The font metrics are a set of parameters that are used to measure the bounding box, any ascenders or descenders and similar information. Using fontforge, these parameters can easily be viewed. However, we are not interested to make any modifications manually; what we are interested is to programmatically obtain this information using Lua. Lua's philosophy and a mantra repeated often by the developers, is that it provides the tools and not the solutions. What this means to the LuaTeX programmer, is that we need to reach very low level  to get this information, which is a road with many bumps. Luckily the tools have been provided by the LuaTeX developers. This comes with a lot of benefits as we can also do our own on the fly mapping, such as creating an index table holding all the Gardiner numbers. 

The |fontloader.open| function loads a font, but it's not usable by itself; the result should be turned into a table with
\textbf{fontloader.to\_table}, as follows:

\begin{verbatim}
  local f = fontloader.open
     ("c:/windows/fonts/NotSansEgyptianHieroglyphics-
       Regulat.ttf")
  fonttable = fontloader.to_table(f)
  fontloader.close(f)
\end{verbatim}

We will use the Google No Tofu Egyptian Hieroglyphic font to experiment with our hieroglyphics. I have used a full path to load the font, which resides on my windows machine in the fonts folder. Once we load all the information in the |fonttable| we use |fontloader.close| to discard the userdata from which the table is extracted. 

What makes OpenType fonts special is that they describe every aspect that you might be able to think of when you think of putting letters together to form words. In addition to the obvious "this is what letters look like" information, OpenType fonts also specify things like the name of each letter that is available in the font, how much of the Unicode standard the font implements, which horizontal and vertical metrics apply to which letters, exactly how the letters are arranged inside the font so that they can quickly be read out, what kind of font classifications apply (is it a fantasy font? is it bold face? is it fixed width? etc), what kind of memory allocation a printer needs to perform in order to be able to even load the font, etc. etc. etc. All these are stored in tables upon tables, similat to a collection of Russian dolls.

To view the values in the fonttable, we will first iterate over the \textbf{fonttable} and extract all the first level keys.

\begin{texexample}{Iterating through a font table}{}
\begin{luacode*}
local z={}
local src = "c:/windows/fonts/NotoSansEgyptianHieroglyphs-Regular.ttf"
local src = "c:/windows/fonts/Albanian.otf"
tf=fontloader.to_table(fontloader.open(src))

-- we sort the keys to create a table
-- important keys to us are tf.glyphs

for k,v in pairs (tf) do
   --tex.print(k.."\\par")
   table.insert(z, k)
end

table.sort(z)
tex.print("\\begin{multicols}{3}\\raggedright")
for k,v in pairs (z) do
   z[k] = string.gsub(z[k],"%_","\\textunderscore ")
   local s = tf[v]
   tex.print("\\textbullet\\hskip3pt\\hangindent2em " .. z[k].." [\\textit{"..type(s).."}] ","\\par")
end
tex.print("\\end{multicols}")
\end{luacode*}
\end{texexample}

We iterate through the \textbf{fonttable} using the Lua  "pair" iterator and we simply print all the keys and the type of the values in a human readable form as shown in the example. Note the use of |\textunderscore| that replaces all underscores in the fields with its text equivalent to sanitize the output. This is a quick and dirty way to avoid the use of catcodes. Many of the keys, bear intuitive names and are not difficult to discern: \textit{version}, \textit{copyright} and the like. Getting the type of Lua variables is important in order to use them for error trapping. When you attempt for example to print a nil value an error will occur.

Now that we have peeked under the font we will iterate and capture the information of interest, which we will put into another table with two keys \textbf{info}  and \textbf{metrics}. In the metrics file we will get the bounding box related metrics of each and every glyph in the font and save it, into our own table. 

\begin{texexample}{More Metrics}{}
  \begin{luacode*}
   tex.print("units per em = ", tf.units_per_em,"\\par")
   for i,j in ipairs (tf.glyphs[6].boundingbox) do
      tex.print("bounding box["..i.."]".." = ", j,"\\par")
   end 
   local w = (tf.glyphs[6].boundingbox[3]-tf.glyphs[6].boundingbox[1])/tf.units_per_em
   local h = tf.glyphs[6].boundingbox[4]/tf.units_per_em
   tex.print("glyph width = ", w,"em\\par")
   tex.print("glyph height = ", h,"em\\par")

-- presents a nicely typeset table 

local rep, write = string.rep, tex.print
function ExploreTable (tab, offset)
    offset = offset or ""
    for k, v in pairs (tab) do
        local newoffset = offset .. "\\mbox{.}"
        if type(v) == "table" then
           -- if k == "boundingbox" then write("BB") end
           write(offset..k .. " = \\{\\par ")
           ExploreTable(v, newoffset)
           write(offset..newoffset .. "\\}\\par")
         else
           write(offset..k .. " = "..tostring(v),"\\par")
         end
      end
end

write("\\par{\\ttfamily ")
ExploreTable(tf.glyphs[38],"\\mbox{.}")
write("}")
  \end{luacode*}
\end{texexample}

The OpenType fonts standard, provides for so much information that we will ignore most of the items and focus on only a few tables and fields. A small utility after Paul Isambert's article is necessary to enable us to view tables easily within this book,


\begin{texexample}{ExploreTable utility}{}
\begin{luacode*}
-- presents a nicely typeset table 

local rep, write = string.rep, tex.print
function ExploreTable (tab, offset)
    offset = offset or ""
    for k, v in pairs (tab) do
        local newoffset = offset .. "\\mbox{.}"
        if type(v) == "table" then
           -- if k == "boundingbox" then write("BB") end
           write(offset..k .. " = \\{\\par ")
           ExploreTable(v, newoffset)
           write(offset..newoffset .. "\\}\\par")
         else
           write(offset..k .. " = "..tostring(v),"\\par")
         end
      end
end

write("\\par{\\ttfamily ")
ExploreTable(tf.glyphs[38],"\\mbox{.}")
write("}")
  \end{luacode*}
\end{texexample}

A good utility also is |TTX| that will convert an OTF font to XML and back. This requires that you have python installed.\footnote{See some good guidelines as to how to install it at \url{http://www.glyphrstudio.com/ttx/}.} The utility uses python to do the conversion. The archive can be downloaded from \url{http://sourceforge.net/projects/fonttools/files/latest/download}. This is a three prong attack. You need to have python install, the numpy library and then the TTX package. The |TTX| program was written by the font designer Just van Rossum, brother of the creator of the Python language, Guido van Rossum\index{Guido van Rossum}. The tool converts TrueType into human-readable |XML| format. The most attractive feature of this tool is that it also perform the opposite operation that is create a TruType font from an |XML| file. The |XML| format makes the hierarchy of the format clearer. Since SVG fonts are also described in |XML| it becomes an easier task to convert an |SVG| font to a TrueType font. To convert |bar.ttf| into |bar.ttx| you simply write:

\begin{verbatim}
ttx bar.ttf
\end{verbatim}

Similarly for the opposite conversion, from |.ttx| to |.ttf|

\begin{verbatim}
ttx bar.ttx
\end{verbatim}

The generated |ttx| file is approximately ten times larger than the original |.ttf| file. The files generated are huge affairs and difficult to manage.The command line option |-l| prints a list of the tables in the font. |TTX| is indispensable in the ``humanization'' of TrueType fonts. The details of the tables and what each field represents are eloquently described in that indispensable book by \person{Yannis Haralambous} \textit{Fonts \& Encodings.} Although the book is now somewhat dated, it is still the best source of information on many esoteric topics related to fonts. 






\section{Meroitic}
\label{s:meroitic}

The Meroitic script is an alphabetic script, used to write the Meroitic language of the Kingdom of Meroë in Sudan. It was developed in the Napatan Period (about 700–300 BCE), and first appears in the 2nd century BCE. For a time, it was also possibly used to write the Nubian language of the successor Nubian kingdoms. Its use was described by the Greek historian Diodorus Siculus (c. 50 BCE).

Although the Meroitic alphabet did continue in use by the Nubian kingdoms that succeeded the Kingdom of Meroë, it was replaced by the Coptic alphabet with the Christianization of Nubia in the sixth century CE. The Nubian form of the Coptic alphabet retained three Meroitic letters.

The script was deciphered in 1909 by Francis Llewellyn Griffith, a British Egyptologist, based on the Meroitic spellings of Egyptian names. However, the Meroitic language itself has yet to be translated. In late 2008 the first complete royal dedication was found,[1] which may help confirm or refute some of the current hypotheses.

The longest inscription found is in the Museum of Fine Arts, Boston.

\newfontfamily\meroitic{Nilus.ttf}^^A

\unicodetable{meroitic}{"109A0,"109B0,"109C0,"109E0,"109F0}%


The examples here use the \idxfont{Nilus.ttf} font of George Douros\footnote{\url{http://users.teilar.gr/~g1951d/}}.

The name of the queen of Amenhotp III is rendered Teie, i.e. Teye, in the Armana 
tablets. The name of the city dedicated to her in Nubia was therefore pronounced 
Ha-Teye and appears in Meroitic as eyita (1916:119)









\chapter{Ugaritic}
\label{s:ugaritic}
\index{Ugaritic fonts>Noto Sans Ugaritic}
\index{Ugaritic}
\index{Akkadian}
\index{Unicode>Ugaritic}
\parindent1em
\newfontfamily\ugaritic{NotoSansUgaritic-Regular.ttf}

\section{Background}
Sometime between 1190-1185 bce, the houses of Ugarit were abandoned by their inhabitants, then pillaged and burned. If they were destroyed by the Sea Peoples we will never know for sure, although this is very likely. This catastrophe ended a history of almost 6000 years. Ugarit was never rebuild and the ruins were buried for centuries before they were discovered in 1929. 

\begin{figure}[htbp]
\centering
\includegraphics[width=\textwidth]{ugarit-excavations}
%http://www.persee.fr/docAsPDF/syria_0039-7946_1936_num_17_2_3887.pdf
\end{figure}

Merchants figure prominently in Ugarit’s archives. The citizens engaged in trade, and many foreign merchants were based in the state, for example from Cyprus exchanging copper ingots in the shape of ox hides. The presence of Minoan and Mycenaean pottery suggests Aegean contacts with the city. It was also the central storage place for grain supplies moving from the wheat plains of northern Syria to the Hittite court.

common defence system (§ 11.5.4.3). The abundance of Cypriot
pottery,173 the Cypro-Minoan texts found in Ugarit ( L i v e r a n i 1979a,
1322-3) as well as letters18 and administrative texts,19 are also witness
to relationhips between the two communities at both the cultural
and the commercial levels. 

Some Cypriots (ally, altyy, DLU, 33)
receive from the Ugaritian administration food and clothing,20 others
belong to the guild of craftsmen.21 On the other hand, from its structure
the administrative text KTU 4.102 = RS 11.857 seems to be
a list of prisoners of war, or of persons detained for some reason,
who come from Cyprus ( V i t a 1995a, 108). An unpublished letter
found in Ras Shamra in 1994, which reports the dispatch of an
emissary of the king of Cyprus to Ugarit to deal with the freeing of
Cypriots detained on Ugaritic soil,22 could support this hypothesis

The \idxlanguage{Ugaritic} language  is written in alphabetic cuneiform. This was an innovative blending of an alphabetic script (like \hyperref[s:hebrew]{Hebrew}) and cuneiform (like Akkadian). The development of alphabetic cuneiform seems to reflect a decline in the use of Akkadian as a \textit{lingua franca} and a transition to alphabetic scripts in the eastern Mediterranean. Ugaritic, as both a cuneiform and alphabetic script, bridges the cuneiform and alphabetic cultures of the ancient Near East.


\begin{figure}[hb]
\centering
\includegraphics[width=\textwidth]{ugaritic-first-tablet}
\caption{A list of offerings with the first tablet number (KTU 1.39 = RS 1.001; Photo: UGARIT - FORSCHUNG Archive)}
\end{figure}

The Ugaritic script is a cuneiform (wedge-shaped) abjad used from around either the fifteenth century BCE[1] or 1300 BCE[2] for Ugaritic, an extinct Northwest Semitic language, and discovered in Ugarit (modern Ras Shamra), Syria, in 1928. It has 30 letters. Other languages (particularly Hurrian) were occasionally written in the Ugaritic script in the area around Ugarit, although not elsewhere.


\section{Material Culture}

Excavations at Ugarit have yielded an abundance of objects of everyday life that we can deduce the every day life of its inhabitants in a higher level of detail than many other civilizations. Objects recovered include mirrors, combs, cooking and drinking utensils, pottery, gems. An interesting item is the clepsydra shown in Figure~\ref{fig:clepsidra} used as a shower head. The religion and cults is also well represented. This is not easy to use as an individual and it was probably used with the help of a servant.

The Ugarites were actively interacting in trading. 

\begin{figure}[htbp]
\includegraphics[width=\textwidth]{clepsidra}
\caption{“Clepsydra” or shower vase RS 81.509
1981, City Center, House E, room 1201. Latakia Museum
H 19.5 cm, Diameter (max.) 18 cm. Fine plain buff pottery with burnished surface. Jug with a large,
ovoid body. The opening is narrow, contracting to a small hole 1 cm in diameter. The bottom is
pierced with 22 small holes to form a strainer. The narrowness of the opening does not permit filling
by any means other than plunging the vase entirely into a large container full of water. It holds about
1 liter. The function of this sort of vase is obvious. The container remained full if the opening was
sealed with one’s thumb to prohibit the entrance of air; the liquid could not flow out through the
bottom. When the thumb was removed (allowing air to enter the jug), the water could flow out
through the bottom, creating a type of shower head.
This object matches the definition of a clepsydra mentioned by ancient authors (Hieron): in its
primary sense, the term clepsydra is not restricted to a measure of time. What we have here is an instrument
used for washing, like a shower in a bathing installation (or shower stall). This was an object
of everyday life, but only in a relatively refined context. This vase was found with other personal
funerary objects fallen from the upper floor of a house of medium status in the city center. Other examples
(e.g., RS 30.325) show that this was not an uncommon item in homes at Ugarit.\\
– Bib.: M. Yon, P. Lombard, and M. Renisio, in RSO III, 1987, p. 106, fig. 87; P. Lombard, ibid., pp. 351–57.}
\label{fig:clepsidra}
\end{figure}

Clay tablets written in Ugaritic provide the earliest evidence of both the North Semitic and South Semitic orders of the alphabet, which gave rise to the alphabetic orders of Arabic (starting with the earliest order of its abjad), the reduced Hebrew, and more distantly the Greek and Latin alphabets on the one hand, and of the Ge'ez alphabet on the other. Arabic and Old South Arabian are the only other Semitic alphabets which have letters for all or almost all of the 29 commonly reconstructed proto-Semitic consonant phonemes. 

According to Dietrich and Loretz in Handbook of Ugaritic Studies (ed. Watson and Wyatt, 1999): "The language they [the 30 signs] represented could be described as an idiom which in terms of content seemed to be comparable to Canaanite texts, but from a phonological perspective, however, was more like Arabic."
The script was written from left to right. Although cuneiform and pressed into clay, its symbols were unrelated to those of the Akkadian cuneiform.

\begin{scriptexample}[]{Ugaritic}
\unicodetable{ugaritic}{"10380,"10390}
\end{scriptexample}

{\let\aegean\arial
\printunicodeblock{./languages/ugaritic.txt}{\ugaritic}
}

\bgroup

\let\a\arial
\Large
\begin{longtable}[l]{%
>{\arial\large}r|
>{\ugaritic}c| 
>{\arial\large}c 
>{\arial\large}c 
>{\arial\large}c >{\arial\large}c
}

&\a Sign	&\a Trans.	&\a IPA	&\a Hebrew	&\a Arabic \\
\hline
\inc &𐎀	&ʾa	& ʔa	&א	&أ \\
\inc &𐎁	&b	& b	    &ב	&ب \\
\inc &𐎂	&g	&ɡ	&ג	&ج\\
\inc &𐎃	&ḫ	&x	&	&خ\\
\inc &𐎄	&d	&d	&ד	&د\\
\inc &𐎅	&h	&h	&ה	&ه\\
\inc &𐎆	&w	&w	&ו	&و\\
\inc &𐎇	&z	&z	&ז	&ز\\
\inc &𐎈	&ḥ	&ħ	&ח	&ح\\
\inc &𐎉	&ṭ	&t̴	&ט	&ط\\
\inc &𐎊	&y	&j	&י	&ي\\
\inc &𐎋	&k	&k	&כ	&ك\\
\inc &𐎌	&š	&ʃ	&ש	&ش\\
\inc &𐎍	&l	&l	&ל	&ل\\
\inc &𐎎	&m	&m	&מ	&م\\
\inc &𐎏	&ḏ	&ð	&	&ذ\\
\inc &𐎐	&n	&n	&נ	&ن\\
\inc &𐎑	&ẓ	&θ̴	&	&ظ\\
\inc &𐎒	&s	&s	&ס	&س\\
\inc &𐎓	&ʿ 	&ʕ	&ע	&ع\\
\inc &𐎔	&p	&p	&פ	&ف\\
\inc &𐎕	&ṣ	&s̴	&צ	&ص\\
\inc &𐎖	&q	&q	&ק	&ق\\
\inc &𐎗	&r	&r	&ר	&ر\\
\inc &𐎘	&ṯ	&θ	&	&ث\\
\inc &𐎙	&ġ	&ɣ	&	&غ\\
\inc &𐎚	&t	&t	&ת	&ت\\
\inc &𐎛	&ʾi	&ʔi	&	&ئ\\
\inc &𐎜	&ʾu	&ʔu	&	&ؤ\\
\end{longtable}
\egroup


\textit{\LARGE$$\stackrel{\mbox{ho}}{.}$$}

% Tranliteration macros 
% 
\bgroup\ugaritic
\def\a{\char"10380}
\def\b{\char"10381}
\def\g{\char"10382}
\LARGE \a \b \g 
\egroup

\section{Online Collections}

http://digital.library.stonybrook.edu/











%\chapter{Sumero Akkadian Cuneiform}
\label{s:sumero}
\newfontfamily\sumero{NotoSansSumeroAkkadianCuneiform-Regular.ttf}




The cuneiform writing system of the ancient Middle East was deeply influential in
world culture. For over three millennia, until about two thousand years ago, it was the
vehicle of communication from (at its greatest extent) Iran to the Mediterranean,
Anatolia to Egypt. A complex script, written mostly on clay tablets by professional
scribes, it was used to record actions, thoughts, and desires that fundamentally
shaped the modern world, socially, politically, and intellectually. Unlike other ancient
media, such as papyri, writing-boards, or leather rolls, cuneiform tablets survive in their
hundreds of thousands, oW en excavated from the buildings in which they were created,
used, or disposed of. Primary evidence of cuneiform culture thus comes from a wide
variety of physical and social contexts in abundant quantities, which enables the close
study of very particular times and places.


Sumero-Akkadian cuneiform has the advantage that most objects are now in open digitized libraries and easily accessible to specialists as well as the interested general reader. The images are normally accompanied by line diagrams, especially for the more important objects, with details of their provenance and publications. 

\begin{figure}[htbp]
\centering
\includegraphics[height=0.8\textheight]{P222322}
\end{figure}

The line diagram

\begin{figure}[htbp]
\includegraphics[height=0.8\textheight]{P222322_l}
\end{figure}

To compensate for the deficiencies of clay tablets, writing boards (Akkadian lē’u ) with
erasable waxed surfaces were used alongside them from at least the 21st century bc
(Steinkeller 2004 ) , plus papyrus (Akkadian niāru ) from the mid-second millennium
and parchment or leather rolls (Akkadian giṭṭu, magallatu ) from the early fi rst millennium
onwards (see Philippe Clancier in Chapter 35 ) . Practically no such artefacts survive—
apart from a few now surfaceless Neo-Assyrian writing boards—although they
are occasionally mentioned in tablets and sometimes depicted visually (Figure 1.8 ). We
must never forget that cuneiform culture was only one literate culture amongst several
in the ancient Near East, albeit the most longlived and prestigious.



\section{Encoding}

In Unicode, the Sumero-Akkadian Cuneiform script is covered in two blocks:
U+12000–U+1237F Cuneiform
U+12400–U+1247F Cuneiform Numbers and Punctuation
These blocks, in version 6.0, are in the Supplementary Multilingual Plane (SMP).

The sample glyphs in the chart file published by the Unicode Consortium[2] show the characters in their Classical Sumerian form (Early Dynastic period, mid 3rd millennium BCE). The characters as written during the 2nd and 1st millennia BCE, the era during which the vast majority of cuneiform texts were written, are considered font variants of the same characters.


The character set as published in version 5.2 has been criticized, mostly because of its treatment of a number of common characters as ligatures, omitting them from the encoding standard.


\unicodetable{sumero}{"12000,"12010,"12020,"12030,"12040,"12050,"12060,"12070,
"12080,"12090, "120A0, "120B0, "120C0, "120D0, "120E0,"120F0,"12400,"12410,"12420,"12430}


\begin{table}[b]
\begin{scriptexample}{textbox}
\parindent1em
From Plato's dialogue Phaedrus 14, 274c-275b:

Socrates: [274c] I heard, then, that   in Egypt, was one of the ancient gods of that country, the one whose sacred bird is called the ibis, and the name of the god himself was Theuth. He it was who [274d] invented numbers and arithmetic and geometry and astronomy, also draughts and dice, and, most important of all, letters. 

Now the king of all Egypt at that time was the god Thamus, who lived in the great city of the upper region, which the Greeks call the Egyptian Thebes, and they call the god himself Ammon. To him came Theuth to show his inventions, saying that they ought to be imparted to the other Egyptians. But Thamus asked what use there was in each, and as Theuth enumerated their uses, expressed praise or blame, according as he approved [274e] or disapproved.  

"The story goes that Thamus said many things to Theuth in praise or blame of the various arts, which it would take too long to repeat; but when they came to the letters, [274e] “This invention, O king,” said Theuth, “will make the Egyptians wiser and will improve their memories; for it is an elixir of memory and wisdom that I have discovered.” But Thamus replied, “Most ingenious Theuth, one man has the ability to beget arts, but the ability to judge of their usefulness or harmfulness to their users belongs to another; [275a] and now you, who are the father of letters, have been led by your affection to ascribe to them a power the opposite of that which they really possess.  

"For this invention will produce forgetfulness in the minds of those who learn to use it, because they will not practice their memory. Their trust in writing, produced by external characters which are no part of themselves, will discourage the use of their own memory within them. You have invented an elixir not of memory, but of reminding; and you offer your pupils the appearance of wisdom, not true wisdom, for they will read many things without instruction and will therefore seem [275b] to know many things, when they are for the most part ignorant and hard to get along with, since they are not wise, but only appear wise." 
\end{scriptexample}
\end{table}


\printunicodeblock{./languages/cuneiform.txt}{\sumero}





\newfontfamily\parthian{NotoSansInscriptionalParthian-Regular.ttf}
\section{Inscriptional Parthian}
\label{s:parthian}
\index{Ancient and Historic Scripts>Inscriptional Parthian}
\index{Inscriptional Parthian fonts>Noto Sans Inscriptional Parthian}
\index{scripts>Inscriptional Parthian}

The Parthian script developed from the Aramaic alphabet around the 2nd century BCE and was used during the Parthian and Sassanid periods of the Persian Empire. The latest known inscription dates from 292 CE. 

We use the font |NotoSansInscriptionalParthian-Regular.ttf| to render the script. 


Inscriptional Parthian is a Unicode block containing characters of the official script of the Sassanid Empire.

\newenvironment{parthiannumbers}{%
\def\1{\parthian\char"10B58}%
\def\2{\parthian\char"10B59}%
\def\3{\text{\parthian\char"10B5A}}%
\def\4{\text{\parthian\char"10B5B}}% 
\TextOrMath\4 \4 %
\TextOrMath\3 \3 %
}{}
\index{Parthian (script)>Parthian numbers}
\begin{scriptexample}[]{}
\unicodetable{parthian}{"10B40,"10B50}
\end{scriptexample}

Inscriptional Parthian has its own numbers, which have right-to-left
directionality. The numbers are built up out of 1, 2, 3, 4, 10, 20, 100, and 1000 which indicates a rather rudimentary numbering system. The inscriptions are not
normalized uniformly. The units are sometimes written with strokes of the same height, or with a final
stroke that is longer, either descending or ascending to show the end of the number; compare 5 in 15 ({\parthian \char"10B59 \char"10B5B}
or 2 + 3) and in 45 (òõ or 1 + 4); compare 6 in 16 (öö or 3 + 3) and in 36 (òôö or 1 + 2 + 3). The
encoding here allows the specialist to choose his or her preferred representation. 

The |phd| package offers only rudimentary support for Parthian numbers in the form of an environment |parthiannumbers|, which can be used as follows:

\begin{texexample}{Inscriptional Parthian numbers}{parth}
\begin{parthiannumbers}
\1 $= 1$
\2 $= 2$
\begin{align*}
\3 &= 3\\
\4 &= 4\\
\3\4 &=7
\end{align*}
\end{parthiannumbers}
\end{texexample}

\subsection{The Inscriptional Parthian Unicode Block in Detail}

\printunicodeblock{./languages/inscriptional-parthian.txt}{\parthian}



\footnote{\url{http://www.unicode.org/L2/L2007/07207-n3286-parthian-pahlavi.pdf}} 


\section{Old Italic}

\epigraph{A society grows great when old men plant
trees in whose shade they know they will never sit.}{Greek proverb}
\label{s:olditalic}
\index{scripts>Old Italic}
\newfontfamily\olditalic{Noto Sans Old Italic}


Old Italic refers to any of several now extinct alphabet systems used on the Italian Peninsula in ancient times for various Indo-European languages (predominantly Italic) and non-Indo-European (e.g. Etruscan) languages. The alphabets derive from the Euboean Greek Cumaean alphabet, used at Ischia and Cumae in the Bay of Naples in the eighth century BC.

Various Indo-European languages belonging to the Italic branch (Faliscan and members of the Sabellian group, including Oscan, Umbrian, and South Picene, and other Indo-European branches such as Celtic, Venetic and Messapic) originally used the alphabet. Faliscan, Oscan, Umbrian, North Picene, and South Picene all derive from an Etruscan form of the alphabet.

\section{Etruscan}

Many peoples took the system that the Greeks had elaborated and
adapted it to their own language. This was particularly true in Lemnos and
in Etruria, where signs inspired by Greek letters were put to the service of
languages that probably were closely related to Greek—signs that we can
read without fully comprehending them. The Etruscans seem to have used
writing largely for religious purposes. According to Cicero (De divinatione)
they bequeathed their sacred texts to the Romans, who held the Etruscan
religion to be the religion of the Book par excellence.\cite{henri1994}

\begin{figure}[htbp]
\centering
\includegraphics[width=0.7\textwidth]{marsiliana}
\caption{The Marsiliana Tablet}
\end{figure}

The Germanic runic alphabet was derived from one of these alphabets by the 2nd century.


Old Italic is a Unicode block containing a unified repertoire of the three stylistic variants of pre-Roman Italic scripts.

\begin{scriptexample}[]{Testing}
\unicodetable{olditalic}{"10300,"10310,"10320}

{\leavevmode
\hfill\hfill\hfill\footnotesize Typeset with \texttt{Noto Sans Old Italic~}
}
\end{scriptexample}
\section{Old South Arabian}
\label{s:oldsoutharabian}

\index{Ancient and Historic Scripts>Old South Arabian}
\index{Old South Arabian fonts>Noto Sans Old South Arabian}
\index{alphabets>Yemeni}

\newfontfamily\oldsoutharabian{NotoSansOldSouthArabian-Regular.ttf}

The ancient Yemeni alphabet (Old South Arabian ms3nd; modern Arabic: {\arabicfont المُسنَد‎}  musnad) branched from the Proto-Sinaitic alphabet in about the 9th century BC. It was used for writing the Old South Arabian languages of the Sabaic, Qatabanic, Hadramautic, Minaic (or Madhabic), Himyaritic, and proto-Ge'ez (or proto-Ethiosemitic) in Dʿmt. The earliest inscriptions in the alphabet date to the 9th century BC in Akkele Guzay, Eritrea[3] and in the 10th century BC in Yemen. There are no vowels, instead using the \emph{mater lectionis} to mark them.

Its mature form was reached around 500 BC, and its use continued until the 6th century AD, including Old North Arabian inscriptions in variants of the alphabet, when it was displaced by the Arabic alphabet.[4] In Ethiopia and Eritrea it evolved later into the Ge'ez alphabet,[1][2] which, with added symbols throughout the centuries, has been used to write Amharic, Tigrinya and Tigre, as well as other languages (including various Semitic, Cushitic, and Nilo-Saharan languages).

It is usually written from right to left but can also be written from left to right. When written from left to right the characters are flipped horizontally (see the photo).
The spacing or separation between words is done with a vertical bar mark (\textbar).
Letters in words are not connected together.

Old South Arabian script does not implement any diacritical marks (dots, etc.), differing in this respect from the modern Arabic alphabet.

\begin{scriptexample}[]{South Arabian}
\unicodetable{oldsoutharabian}{"10A60,"10A70}
\end{scriptexample}

Support in \latexe is provided via Peter Wilson's package \pkgname{sarabian}\citep{sarabian}. The package provides all the |metafont| sources as well as transliteration commands and other utilities \seedocs{\SARAB}. The package is based on fonts developed originally by Alan Stanier of Essex University.

The package provides the commands \docAuxCmd{sarabfamily} that selects the South Arabian font family. The family name is \texttt{sarab}. Another command \docAuxCmd{textsarab}\meta{text} typesets \meta{text} in the South Arabian font. The package provides two ways of accessing
glyphs: (a) by \texttt{ASCII} character commands, and (b) via commands. These are illustrated in
Table~\ref{sarabian1} which is a modified version of that provided in the Comprehensive Symbols.



\def\SAtdu{\oldsoutharabian\char"10A77}

A comparison between  the unicode and the rendering (scaled 5) \pkgname{sarabian} is shown below.

\centerline{\scalebox{3}{\SAtdu} \scalebox{3}{\textsarab{\SAtd}}}

There is no real advantage in using unicode fonts, if all you interested is to write some South Arabian text for inscriptions. 

\begin{symtable}[SARAB]{\SARAB\ South Arabian Letters}
\index{South Arabian alphabet}
\index{alphabets>South Arabian}
\label{sarabian1}
\begin{tabular}{*4{ll@{\qquad}}ll}
\K[\textsarab{\SAa}]\SAa   & \K[\textsarab{\SAz}]\SAz   & \K[\textsarab{\SAm}]\SAm   & \K[\textsarab{\SAsd}]\SAsd & \K[\textsarab{\SAdb}]\SAdb \\
\K[\textsarab{\SAb}]\SAb   & \K[\textsarab{\SAhd}]\SAhd & \K[\textsarab{\SAn}]\SAn   & \K[\textsarab{\SAq}]\SAq   & \K[\textsarab{\SAtb}]\SAtb \\
\K[\textsarab{\SAg}]\SAg   & \K[\textsarab{\SAtd}]\SAtd & \K[\textsarab{\SAs}]\SAs   & \K[\textsarab{\SAr}]\SAr   & \K[\textsarab{\SAga}]\SAga \\
\K[\textsarab{\SAd}]\SAd   & \K[\textsarab{\SAy}]\SAy   & \K[\textsarab{\SAf}]\SAf   & \K[\textsarab{\SAsv}]\SAsv & \K[\textsarab{\SAzd}]\SAzd \\
\K[\textsarab{\SAh}]\SAh   & \K[\textsarab{\SAk}]\SAk   & \K[\textsarab{\SAlq}]\SAlq & \K[\textsarab{\SAt}]\SAt   & \K[\textsarab{\SAsa}]\SAsa \\
\K[\textsarab{\SAw}]\SAw   & \K[\textsarab{\SAl}]\SAl   & \K[\textsarab{\SAo}]\SAo   & \K[\textsarab{\SAhu}]\SAhu & \K[\textsarab{\SAdd}]\SAdd \\
\end{tabular}

\bigskip
\begin{tablenote}
  \usefontcmdmessage{\textsarab}{\sarabfamily}.  Single-character
  shortcuts are also supported: Both
  ``\verb+\textsarab{\SAb\SAk\SAn}+'' and ``\verb+\textsarab{bkn}+''
  produce ``\textsarab{bkn}'', for example.  \seedocs{\SARAB}.
\end{tablenote}
\end{symtable}

\section{Avestan script}
\label{s:avestan}
The Avestan alphabet is a writing system developed during Iran's Sassanid era (AD 226–651) to render the Avestan language.
As a side effect of its development, the script was also used for Pazend, a method of writing Middle Persian that was used primarily for the Zend commentaries on the texts of the Avesta. In the texts of Zoroastrian tradition, the alphabet is referred to as \emph{din dabireh} or \emph{din dabiri}, Middle Persian for "the religion's script".

The Avestan alphabet was replaced by the Arabic alphabet after Persia converted to Islam during the 7th century CE. 


Notable Features

The alphabet is written from right to left, in the same way as Syriac, Arabic and Hebrew.
See more at: \url{http://www.iranchamber.com/scripts/avestan_alphabet.php#sthash.ZRu7AkEb.dpuf}


\begin{scriptexample}[]{Avestan}
\ifxetex\TeXXeTstate=1
\beginR\fi
\avestan\raggedleft
𐬄	
𐬅	
𐬆	
𐬇	
𐬈	
𐬉	
𐬊	
𐬋	
𐬌	
𐬍	
𐬎	
𐬏	
𐬐	
	
𐬒	
𐬓	
𐬔	
	
𐬖	
𐬗	
𐬘	
𐬙	
𐬚	
𐬛	
𐬜	
𐬝	
𐬞	
𐬟	
𐬠	
𐬡	
𐬢	
𐬣	
𐬤	
𐬥	
𐬦	
𐬧	
𐬨	
𐬩	
𐬪	
𐬫	
𐬬	
𐬭	
𐬮	
𐬯	
𐬰	
𐬱	
𐬲	
𐬳	
𐬴	
𐬵	
\ifxetex\endR
\TeXXeTstate=0\fi
\end{scriptexample}

The recent Google font \idxfont{NotoSansAvestan-Regular.ttf} can be used to typeset the Avestan script, but I am not sure if it is suitable for any serious study of the language.
\section{Old Turkic}
\label{s:oldturkic}

Old Turkic (also East Old Turkic, Orkhon Turkic, Old Uyghur) is the earliest attested form of Turkic, found in Göktürk and Uyghur inscriptions dating from about the 7th century to the 13th century. It is the oldest attested member of the Orkhon branch of Turkic, which is extant in the modern Western Yugur language. Confusingly, it is not the ancestor of the language now called Uighur; the contemporaneous ancestor of Uighur to the west is called Middle Turkic.

Old Turkic is attested in a number of scripts, including the Orkhon-Yenisei runiform script, the Old Uyghur alphabet (a form of the Sogdian alphabet), the Brāhmī script, the Manichean alphabet, and the Perso-Arabic script.

\newfontfamily\oldturkic{Segoe UI Symbol}
\begin{scriptexample}[]{Old Turkish}
\oldturkic
\obeylines
Orkhon	Yenisei
variants	Transliteration / transcription
Old Turkic letter  𐰀	𐰁 𐰂	a, ä
Old Turkic letter  𐰃	𐰄 𐰅	y, i (e)
Old Turkic letter  𐰆		o, u
Old Turkic letter  𐰇	𐰈	ö, ü

	0	1	2	3	4	5	6	7	8	9	A	B	C	D	E	F
U+10C0x	𐰀	𐰁	𐰂	𐰃	𐰄	𐰅	𐰆	𐰇	𐰈	𐰉	𐰊	𐰋	𐰌	𐰍	𐰎	𐰏
U+10C1x	𐰐	𐰑	𐰒	𐰓	𐰔	𐰕	𐰖	𐰗	𐰘	𐰙	𐰚	𐰛	𐰜	𐰝	𐰞	𐰟
U+10C2x	𐰠	𐰡	𐰢	𐰣	𐰤	𐰥	𐰦	𐰧	𐰨	𐰩	𐰪	𐰫	𐰬	𐰭	𐰮	𐰯
U+10C3x	𐰰	𐰱	𐰲	𐰳	𐰴	𐰵	𐰶	𐰷	𐰸	𐰹	𐰺	𐰻	𐰼	𐰽	𐰾	𐰿
U+10C4x	𐱀	𐱁	𐱂	𐱃	𐱄	𐱅	𐱆	𐱇	𐱈	

\hfill  Typeset with \texttt{Segoe UI Symbol} \cmd{\oldturkic} 
\end{scriptexample}

Irk Bitig or Irq Bitig (Old Turkic: {\bfseries\Large\oldturkic 𐰃𐰺𐰴 𐰋𐰃𐱅𐰃𐰏}), known as the Book of Omens or Book of Divination in English, is a 9th-century manuscript book on divination that was discovered in the "Library Cave" of the Mogao Caves in Dunhuang, China, by Aurel Stein in 1907, and is now in the collection of the British Library in London, England. The book is written in Old Turkic using the Old Turkic script (also known as "Orkhon" or "Turkic runes"); it is the only known complete manuscript text written in the Old Turkic script.[1] It is also an important source for early Turkic mythology.

The Old Turkic text does not have any sentence punctuation, but uses two black lines in a red circle as a word separation mark in order to indicate word boundaries as shown in Figure~{\ref{omen}}

\begin{figure}[htb]
\centering

\includegraphics[width=0.5\textwidth]{./images/omen.jpg}
\caption{Omen 11 (4-4-3 dice) of the Irk Bitig (folio 13a): "There comes a messenger on a yellow horse (and) an envoy on a dark brown horse, bringing good tidings, it says. Know thus: (The omen) is extremely good."}
\label{omen}
\end{figure}

\begin{figure}[htb]
\centering

\includegraphics[width=0.8\textwidth]{./images/old-turkic.jpg}
\caption{Runic U 5
Fragment of an Old Turkish Manichaean story in Runic script; well preserved folio of a Manichaean book in codex format. \protect\href{http://turfan.bbaw.de/projekt-en/sprachen-und-schriften}{turfan}.}
\end{figure}


\PrintUnicodeBlock{./languages/old-turkic.txt}{\oldturkic}



Sources of Old Turkic are divided into three corpora:
the 7th to 10th century Orkhon inscriptions in Mongolia and the Yenisey basin (Orkhon Turkic, or Old Turkic proper) and the 650 Elegest inscription about Alp Urungu named a Kyrgyz khan at around Elegest River.

9th to 13th century Uyghur manuscripts from Xinjiang (Old Uyghur), in various scripts including Brahmi, the Manichaean, Syriac and Uyghur alphabets, treating religious (Buddhist, Manichaean and Nestorian), legal, literary, folkloric and astrologic material as well as personal correspondence.



\section{Runic}
\label{s:runic}
\newfontfamily\runic{NotoSansRunic-Regular.ttf}

Runes (Proto-Norse:{\runic ᚱᚢᚾᛟ }(runo), Old Norse: rún) are the letters in a set of related alphabets known as runic alphabets, which were used to write various Germanic languages before the adoption of the Latin alphabet and for specialised purposes thereafter. The Scandinavian variants are also known as futhark or fuþark (derived from their first six letters of the alphabet: F, U, Þ, A, R, and K); the Anglo-Saxon variant is futhorc or fuþorc (due to sound changes undergone in Old English by the names of those six letters)

\begin{scriptexample}[]{Runic}
 \unicodetable{runic}{"16A0,"16B0,"16C0,"16D0,"16E0,"16F0}
\end{scriptexample}


\printunicodeblock{./languages/runic.txt}{\runic}

\newfontfamily\glagolitic{MPH 2B Damase}

\section{Glagolitic}

\epigraph{The average Ph.D. thesis is nothing but a transference of bones from one graveyard to another.}{%
J. Frank Dobie (1888-1964)}


\label{s:glagolitic}
\fboxrule0pt\fboxsep0pt

\noindent
The Glagolitic alphabet /{\glagolitic ˌɡlæɡɵˈlɪtɨk/}, also known as Glagolitsa, is the oldest known Slavic alphabet, from the 9th century.

It was created in the 9th century by Saint Cyril, a Byzantine monk from Thessaloniki. He and his brother, Saint Methodius, were sent by the Byzantine Emperor Michael III in 863 to Great Moravia to spread Christianity among the West Slavs in the area. The brothers decided to translate liturgical books into the Old Slavic language that was understandable to the general population, but as the words of that language could not be easily written by using either the Greek or Latin alphabets, Cyril decided to invent a new script, Glagolitic, which he based on the local dialect of the Slavic tribes from the Byzantine Salonika region.
After the deaths of Cyril and Methodius, the Glagolitic alphabet ceased to be used in Moravia, but their students continued to propagate it in the west and south. 

After a long career, Glagolitic writing stopped being used, except for
religious purposes in certain dioceses of Bosnia and Dalmatia (Croatia).
The Cyrillic alphabet was adopted by all Orthodox Slays and served to note
their literary language. Most of the Slays who rallied to Rome rejected it,
however, which created the paradoxical situation in ex-Yugoslavia, where
two peoples who speak the same language write in different scripts, the
Serbs in Cyrillic and the Croats with Roman characters. Finally, as is
known, the ex-Soviet Union did much to put into writing the languages
spoken by the peoples within its borders, for the most part noting them in
adaptations of the Cyrillic alphabet, while Russian became the language of
culture throughout the Soviet Union.\cite{henri1994}

Slavic printing in Glagolitic characters originated in Venice, where a
\textit{Sluzebnik} (or \textit{Leitourgikon}) was published in 1483, followed by missals and
breviaries, all printed by Andrea Torresani, the future father-in-law and
associate of Aldus Manutius. After 1494 some attempts were made to create
printshops in Croatia itself, first in Senj in 1508, then, after 1530, in
Rijeka (Fiume). The work of these firms was almost totally liturgical (religious,
at any rate), and it had strong competition from manuscript works
that were better adapted to the diversity of local liturgical customs. Religion
also dictated the output of a printshop founded to provide Protestant propaganda
that was set up in Tubingen between 1560 and 1564 by Baron
Hans von Ungnad and that printed the great Lutheran texts in Glagolitic
characters.\footfullcite{henri1994}

Figure~\ref{fig:zograf} illustrates an example of the language.\footnote{\url{https://en.wikipedia.org/wiki/Glagolitic_script\#/media/File:ZographensisColour.jpg}}

\begin{figure}[htbp]
\centering

\includegraphics[width=0.45\linewidth]{glagolitic}
\caption[The first page of the Gospel of Mark from the 10th–11th century Codex Zographensis, found in the Zograf Monastery in 1843.]{The first page of the Gospel of Mark from the 10th–11th century Codex Zographensis, found in the Zograf Monastery in 1843.}
\label{fig:zograf}
\end{figure}

\section{Unicode Support}
The Glagolitic alphabet was added to the Unicode Standard in March 2005 with the release of version 4.1.
The Unicode block for Glagolitic is U+2C00–U+2C5F.



\begin{scriptexample}[]{glacolitic}

\unicodetable{glagolitic}{%
"2C00,"2C10,"2C20,"2C30,"2C40,"2C50}

\texttt{typeset with Damase version 2.0 MPH 2B Damase}
\end{scriptexample}
\bgroup
\glagolitic

The name was not coined until many centuries after its creation, and comes from the Old Church Slavonic glagolъ "utterance" (also the origin of the Slavic name for the letter G). The verb glagoliti means "to speak". It has been conjectured that the name glagolitsa developed in Croatia around the 14th century and was derived from the word glagolity, applied to adherents of the liturgy in Slavonic.[1]

In Old Church Slavonic the name is {\glagolitic ⰍⰫⰓⰊⰎⰎⰑⰂⰋⰜⰀ}, Кѷрїлловица.
The name Glagolitic in Bulgarian, Russian, Macedonian глаголица (glagolica), Belarusian is глаголіца (hłaholica), Croatian glagoljica, Serbian глагољица / glagoljica, Czech hlaholice, Polish głagolica, Slovene glagolica, Slovak hlaholika, and Ukrainian глаголиця (hlaholyća).



\egroup


\ifscriptolmec
  \section{Epi-Olmec}
\label{s:olmec}
Epi-Olmec is an ancient Mesoamerican logosyllabic script which has been deciphered by Terrence Kaufman and John Justeson. A complete description of the script has been described by \cite{kaufman}. The most famous inscription is on the Tuxtla Statuette. The Tuxtla Statuette is a small 6.3 inch (16 cm) rounded greenstone figurine, carved to resemble a squat, bullet-shaped human with a duck-like bill and wings. Most researchers believe the statuette represents a shaman wearing a bird mask and bird cloak.[1] It is incised with 75 glyphs of the Epi-Olmec or Isthmian script, one of the few extant examples of this very early Mesoamerican writing system. The Tuxtla Statuette is particularly notable in that its glyphs include the Mesoamerican Long Count calendar date of March 162 CE, which in 1902 was the oldest Long Count date discovered. A product of the final century of the Epi-Olmec culture, the statuette is from the same region and period as La Mojarra Stela 1 and may refer to the same events or persons.[3] Similarities between the Tuxtla Statuette and Cerro de las Mesas Monument 5, a boulder carved to represent a semi-nude figure with a duckbill-like buccal mask, have also been noted.[4]

\begin{figure}[ht]
\centering
\includegraphics[height=0.35\textheight]{./images/tuxtla-statuette.png}\hspace{1em} 
\includegraphics[height=0.35\textheight]{./images/tuxtla-statuette-01.jpg}
\caption{Frontal view of the Tuxtla Statuette. Note the Mesoamerican Long Count calendar date of March 162 CE (8.6.2.4.17) down the front of the statuette. The left figure is from wikipedia and the right from the original \protect\href{http://www.readcube.com/articles/10.1525/aa.1907.9.4.02a00030}{Holmes} paper.\citep{holmes1907}}
\end{figure}

\subsection{The epiolmec package}

The script has not been as yet encoded as by the Unicode consortium. Syropoulos \citep{syropoulos} created a font for the script and also wrote an article for TUGboat. Interestingly the paper describes the procedure used to develop the font. The package \pkgname{epiolmec} which is available both in \TeX live and Mik\tex, provides commands to access the glyphs. It is also possibly easier to typeset the script using traditional \latexe techniques, as they provide transcription commands rather than using a unicode font with the glyphs allocated in the private area directly.

\begin{verbatim}
\documentclass{article}
\usepackage{epiolmec,multicol}
\begin{document}
  \begin{center}
      \begin{minipage}{80pt}
      \begin{multicols}{3}
         \EOku\\ \EOji\\  
         \EOtze\\ \EOstep \\
       \end{multicols}    
     \end{minipage}       
  \end{center}
\end{document}
\end{verbatim}

Since the Epi-Olmec script is a logosyllagraphy we
need some practical way to access the symbols of the
script. Originally Syropoulos used the Ω translation
process that mapped words and “syllables” to the
corresponding glyphs of the font. In this way one obtains
a natural way for typing in Epi-Olmec texts. In addition,
in order to avoid the problem mentioned above,
he used a wrapper that typesets the text vertically.
For short texts \cmd{\shortstacks} is adequate, while
for longer texts, he used a |multicols| environment
inside a relatively narrow minipage. 

\begin{scriptexample}{Epi-Olmec}
\bgroup
\HUGE
\centering
\EOpi   \EOofficerI \EOofficerII \EOofficerIII

\captionof{figure}{The output of \string\EOpi, \string\EOofficerI, \string\EOofficerII\ and \string\EOofficerIII\ commands. }
\egroup
\end{scriptexample}

\subsection{Numbering System}\index{Epi-Olmec>vigesimal system}

The Epi-Olmec people used the same numbering system  
 as the Maya. Their numbering system was a vigesimal system and
 the digits were written in a top-down fashion. Thus, we need a macro
 that will typeset numbers in this fashion when it is used with \LaTeX\
 (actually $\epsilon$-\LaTeX). In addition, we need a macro that will
 just output the vigesimal digits. Such a macro could be used with
 $\Lambda$ with the |LTL| text and paragraph directions. To recapitulate,
 we need to define two macros that will basically typeset vigesimal numbers
 in either horizontal or vertical mode.

 For the various calculations that are performed, we need at least three
 counter variables. The fourth is needed for the macro that typesets the
 vigesimal numbers vertically and its usage is explained below. 

\begin{scriptexample}{EpicOtmec}
\def\textb#1{\text{\makebox[6em]{\hss#1~~   \protect\string#1\hfill}}}
\begin{multicols}{3}
\bgroup
\parindent0pt
$\textb{\EOzero}=0$\\
$\textb{\EOi} = 1$\\
$\textb{\EOii} = 2$\\
$\textb{\EOiii} =3$\\
$\textb{\EOiv}  =4$\\
$\textb{\EOv}   =5$\\
$\textb{\EOvi}  =6$\\
$\textb{\EOvii} =7$\\
$\textb{\EOviii} =8$\\
$\textb{\EOix} =9$\\ 
$\textb{\EOx} =10$\\
$\textb{\EOxi} =11$\\
$\textb\EOxii =12$\\
$\textb{\EOxiii} =13$\\
$\textb{\EOxiv} =14$\\
$\textb{\EOxv} =15$\\
$\textb{\EOxvi} =16$\\
$\textb\EOxvii =17$\\
$\textb{\EOxviii} =18$\\
$\textb{\EOxix} =19$\\
$\textb{\EOxx} =20$\\
\egroup
\end{multicols}
\end{scriptexample}


%% TODO add to index all symbols

\begin{multicols}{4}
\bgroup
\def\K#1{\makebox[3em]{{\color{theunicodesymbolcolor}\hss#1\hfill}} \string#1}
\parindent0pt
\K\EOSpan\\ 
\K\EOJI \\
\K\EOvarji\\ 
\K\EOvarki \\
\K\EOpi \\
\K\EOpe \\
\K\EOpuu \\
\K\EOpa \\
\K\EOvarpa\\ 
\K\EOpu \\
\K\EOpo \\
\K\EOti \\
\K\EOte \\
\K\EOtuu \\
\K\EOta \\
\K\EOtu \\
\K\EOto \\
\K\EOtzi \\
\K\EOtze \\
\K\EOtzuu \\
\K\EOtza \\
\K\EOvartza\\ 
\K\EOtzu \\
\K\EOki \\
\K\EOke \\
\K\EOkuu \\
\K\EOvarkuu\\ 
\K\EOku\\ 
\K\EOko \\
\K\EOSi \\
\K\EOvarSi\\ 
\K\EOSuu \\
\K\EOSa \\
\K\EOSu \\
\K\EOSo \\
\K\EOsi \\
\K\EOvarsi\\ 
\K\EOsuu \\
\K\EOsa \\
\K\EOsu \\
\K\EOji \\
\K\EOje \\
\K\EOja \\
\K\EOvarja\\ 
\K\EOju \\
\K\EOjo \\
\K\EOmi \\
\K\EOme \\
\K\EOmuu \\
\K\EOma \\
\K\EOni \\
\K\EOvarni\\
\K\EOne \\
\K\EOnuu \\
\K\EOna \\
\K\EOnu \\
\K\EOwi \\
\K\EOwe \\
\K\EOwuu \\
\K\EOvarwuu\\
\K \EOwa\\
\K\EOwo \\
\K\EOye \\
\K\EOyuu \\
\K\EOya \\
\K\EOkak \\
\K\EOpak \\
\K\EOpuuk\\
\K\EOyaj \\
\K\EOScorpius\\
\K\EODealWith\\
\K\EOYear \\
\K\EOBeardMask \\
\K\EOBlood \\
\K\EOBundle \\
\K\EOChop \\
\K\EOCloth \\
\K\EOSaw \\
\K\EOGuise \\
\K\EOofficerI\\
\K\EOofficerII \\
\K\EOofficerIII \\
\K\EOofficerIV \\
\K\EOKing \\
\K\EOloinCloth \\
\K\EOlongLipII \\
\K\EOLose \\
\K\EOmexNew \\
\K\EOMiddle \\
\K\EOPlant \\
\K\EOPlay \\
\K\EOPrince \\
\K\EOSky \\
\K\EOskyPillar \\
\K\EOSprinkle \\
\K\EOstarWarrior\\
\K\EOTitleII \\
\K\EOtuki \\
\K\EOtzetze\\
\K\EOChronI \\
\K\EOPatron \\
\K\EOandThen\\
\K\EOAppear \\
\K\EODeer \\
\K\EOeat \\
\K\EOPatronII \\
\K\EOPierce \\
\K\EOkij \\
\K\EOstar  \\
\K\EOsnake \\
\K\EOtime \\
\K\EOtukpa  \\
\K\EOflint \\
\K\EOafter \\
\K\EOvarBeardMask \\
\K\EOBedeck \\
\K\EObrace \\
\K\EOflower  \\
\K\EOGod \\
\K\EOMountain \\
\K\EOgovernor \\
\K\EOHallow \\
\K\EOjaguar \\
\K\EOSini \\
\K\EOknottedCloth \\
\K\EOknottedClothStraps \\
\K\EOLord \\
\K\EOmacaw \\
\K\EOmonster \\
\K\EOmacawI \\
\K\EOskyAnimal\\
\K\EOnow \\
\K\EOTitleIV \\
\K\EOpenis \\
\K\EOpriest  \\
\K\EOstep\\
\K\EOsing \\
\K\EOskin \\
\K\EOStarWarrior \\
\K\EOsun \\
\K\EOthrone\\
\K\EOTime \\
\K\EOHallow \\
\K\EOTitle \\
\K\EOturtle \\
\K\EOundef \\
\K\EOGoUp \\
\K\EOLetBlood \\
\K\EORain \\
\K\EOset \\
\K\EOvarYear\\
\K\EOFold \\
\K\EOsacrifice \\
\K\EObuilding \\
\egroup
\end{multicols} 

\subsection{Technical}

The font is defined with the local encoding \texttt{LEO}. 

\begin{verbatim}
\DeclareFontEncoding{LEO}{}{}
\DeclareFontSubstitution{LEO}{cmr}{m}{n}
\DeclareFontFamily{LEO}{cmr}{\hyphenchar\font=-1}
\end{verbatim}

Note the |\hyphenchar\font=-1| that disables hyphenation in the |\DeclareFontFamily|  declaration. You cannot behead the \EOofficerII\ in order to hyphenate the text!



\fi
 %OK
%\chapter{Additional Modern Scripts}

\begin{center}
\begin{tabular}{lp{5cm}l}
\hyperref[s:ethiopic]{Ethiopic}
&\hyperref[s:vai]{Vai}
& \hyperref[s:deseret]{Deseret}\\
\hyperref[s:mongolian]{Mongolian} 
&\hyperref[s:bamum]{Bamum} &Shavian.\\
\hyperref[s:osmanya]{Osmanya}
& \hyperref[s:cherokee]{Cherokee} 
& \hyperref[s:lisu]{Lisu}\\
\hyperref[s:tifinagh]{Tifinagh}
&Canadian Aboriginal Syllabics. 
&\hyperref[s:miao]{Miao}\\
\hyperref[s:nko]{N’Ko}&&\\
\end{tabular}
\end{center}

Ethiopic, Mongolian, and Tifinagh are scripts with long histories. Although their roots can
be traced back to the original Semitic and North African writing systems, they would not
be classified as Middle Eastern scripts today

The Cherokee script is a syllabary developed between 1815 and 1821, to write the Cherokee
language, still spoken by small communities in Oklahoma and North Carolina. Canadian
Aboriginal Syllabics were invented in the 1830s for Algonquian languages in Canada. The
system has been extended many times, and is now actively used by other communities, including speakers of Inuktitut and Athapascan languages.

Deseret is a phonemic alphabet devised in the 1850s to write English. It saw limited use for
a few decades by members of The Church of Jesus Christ of Latter-day Saints. Shavian is
another phonemic alphabet, invented in the 1950s to write English. It was used to publish
one book in 1962, but remains of some current interest




\newfontfamily\ethiopic{NotoSansEthiopic-Bold.ttf}

\section{Ethiopic}
\label{s:ethiopic}

Ge'ez ({\ethiopic ግዕዝ} Gəʿəz), (also known as Ethiopic) is a script used as an abugida (syllable alphabet) for several languages of Ethiopia and Eritrea. It originated as an abjad (consonant-only alphabet) and was first used to write Ge'ez, now the liturgical language of the Ethiopian Orthodox Tewahedo Church and the Eritrean Orthodox Tewahedo Church. In Amharic and Tigrinya, the script is often called fidäl ({\ethiopic ፊደል}), meaning "script" or "alphabet".

The Ge'ez script has been adapted to write other, mostly Semitic, languages, particularly Amharic in Ethiopia, and Tigrinya in both Eritrea and Ethiopia. It is also used for Sebatbeit, Me'en, and most other languages of Ethiopia. In Eritrea it is used for Tigre, and it has traditionally been used for Blin, a Cushitic language. Tigre, spoken in western and northern Eritrea, is considered to resemble Ge'ez more than do the other derivative languages.[citation needed] Some other languages in the Horn of Africa, such as Oromo, used to be written using Ge'ez, but have migrated to Latin-based orthographies.

For the representation of sounds, this article uses a system that is common (though not universal) among linguists who work on Ethiopian Semitic languages. This differs somewhat from the conventions of the International Phonetic Alphabet. See the articles on the individual languages for information on the pronunciation.

There are a number of fonts available and we have selected the Google \idxfont{NotoSansEthiopic}


\begin{scriptexample}[]{Ethiopic}
\unicodetable{ethiopic}{"1200,"1210,"1220,"1230,"1240,"1250,"1260,"1270,"1280,"1290,^^A
"12A0,"12B0,"12C0,"12E0,"12F0,"1300,"1310,"1330,"1340,"1350,"1360,"1370}
\end{scriptexample}


\printunicodeblock{./languages/ethiopic.txt}{\ethiopic}




\section{Vai}
\label{s:vai}

The Vai syllabary is a syllabic writing system devised for the Vai language by Momolu Duwalu Bukele of Jondu, in what is now Grand Cape Mount County, Liberia.[1] [2] Bukele is regarded within the Vai community, as well as by most scholars, as the syllabary's inventor and chief promoter when it was first documented in the 1830s. It is one of the two most successful indigenous scripts in West Africa.

\newfontfamily\vai{code2000.ttf}
\begin{scriptexample}[]{Vai}
\unicodetable{vai}{"A500,"A510,"A520,"A530,"A540,"A550,"A560,"A570,^^A
"A580,"A590,"A5A0,"A5B0,^^A
"A5C0,"A5D0,"A5E0,"A5F0,"A610,"A620,"A630}
\end{scriptexample}

In the 1920s ten decimal digits were devised for Vai; these were “Vai-style” glyph variants of
European digits (see Figure 11). They were not popular with Vai people  even for historical purposes. All
the modern literature uses European digits.


\begin{scriptexample}[]{Vai}
\bgroup
\vai
\obeylines\Large
0	1	2	3	4	5	6	7	8	9
꘠	꘡	꘢	꘣	꘤	꘥	꘦	꘧	꘨	꘩
\vai
\egroup
\end{scriptexample}



\printunicodeblock{./languages/vai.txt}{\vai}
\section{Deseret script}
\label{s:deseret}
\newfontfamily\deseret{code2001.ttf}

The Deseret alphabet (dɛz.əˈrɛt.) (Deseret: {\deseret 𐐔𐐯𐑅𐐨𐑉𐐯𐐻 or 𐐔𐐯𐑆𐐲𐑉𐐯𐐻}) is a phonemic English spelling reform developed in the mid-19th century by the board of regents of the University of Deseret (later the University of Utah) under the direction of Brigham Young, second president of The Church of Jesus Christ of Latter-day Saints.

In public statements, Young claimed the alphabet was intended to replace the traditional Latin alphabet with an alternative, more phonetically accurate alphabet for the English language. This would offer immigrants an opportunity to learn to read and write English, he said, the orthography of which is often less phonetically consistent than those of many other languages. Similar experiments were not uncommon during the period, the most well-known of which is the Shavian alphabet.

Young also prescribed the learning of Deseret to the school system, stating "It will be the means of introducing uniformity in our orthography, and the years that are now required to learn to read and spell can be devoted to other studies".[2]


Deseret script {\deseret 𐐔𐐯𐑅𐐨𐑉𐐯𐐻}  [U+10400-U+1044F]
\medskip

\bgroup
\par
\noindent
\colorbox{thecodebackground}{\color{black}^^A
\begin{minipage}{\textwidth}^^A
\parindent1pt
\vskip10pt
\leftskip10pt \rightskip\leftskip
\deseret
\large

𐐂 𐑌𐐲𐑉𐑅𐐨𐑉𐐮 𐐮𐑆 𐐪 𐐹𐐨𐑅 𐐱𐑂 𐑊𐐰𐑌𐐼 𐐱𐑌 𐐸𐐶𐐮𐐽 𐑁𐑉𐐭𐐻𐐻𐑉𐐨𐑆 𐐪𐑉 𐑅𐐻𐐪𐑉𐐻𐐯𐐼,


\par
\vspace*{10pt}
\end{minipage}
}

Text: Deseret alphabet http://www.omniglot.com/writing/deseret.htm
\medskip
\egroup

\PrintUnicodeBlock{./languages/deseret.txt}{\deseret}

\chapter{Bamum}
\label{s:bamum}
\epigraph{"No known alphabet was ever invented by a European."}{Jeffreys' translation from the Royal script.}

\label{s:bamum}
\index{scripts>Bamum}
\newfontfamily\bamum{NotoSansBamum-Regular.ttf}

The Bamum scripts are an evolutionary series of six scripts created for the Bamum language by King Njoya of Cameroon at the turn of the 20th century. They are notable for evolving from a pictographic system to a partially alphabetic syllabic script in the space of 14 years, from 1896 to 1910. Bamum type was cast in 1918, but the script fell into disuse around 1931.

\begin{figure}[htbp]
\parindent=0pt

\centering

\includegraphics[width=\textwidth]{bamum}

\caption{King Njoya of Bamum receiving an oil painting of Kaiser Wilhelm II. The gift was in return for his support in the German campaign against the Nso'.}
\end{figure}

The Bamum, sometimes called Bamoum, Bamun, Bamoun, or Mum, are a Bantoid ethnic group of Cameroon with around 215,000 members.



\begin{scriptexample}[]{Bamum}
\unicodetable{bamum}{"A6A0,"A6B0,"A6C0,"A6D0,"A6E0,"A6F0}
\end{scriptexample}
\section{Shavian}
\label{s:shavian}
\def\shaviansetup#1{}
\newfontfamily\shavian{code2001.ttf}
^^A\newfontfamily\shavian{NotoSansShavian-Regular.ttf}
\cxset{shavian font/.code=\shaviansetup{#1}}
\cxset{shavian font=shavian}




\begin{scriptexample}[]{shavian}
\shavian

𐑳 𐑡𐑻𐑯𐑰 𐑑 𐑞 𐑕𐑧𐑯𐑑𐑻 𐑝 𐑞 𐑻𐑔
𐑚𐑲 - ·𐑡𐑵𐑤𐑟 ·𐑝𐑻𐑯

𐑗𐑩𐑐𐑑𐑻 1 - 𐑥𐑲 𐑳𐑙𐑒𐑳𐑤 𐑥𐑱𐑒𐑕 𐑳 𐑜𐑮𐑱𐑑 𐑛𐑦𐑕𐑒𐑳𐑝𐑻𐑰

     𐑤𐑫𐑒𐑦𐑙 𐑚𐑩𐑒 𐑑 𐑷𐑤 𐑞𐑩𐑑 𐑣𐑩𐑟 𐑳𐑒𐑻𐑛 𐑑 𐑥𐑰 𐑕𐑦𐑯𐑕 𐑞𐑩𐑑 𐑦𐑝𐑧𐑯𐑑𐑓𐑳𐑤 𐑛𐑱, 𐑲 𐑩𐑥 𐑕𐑒𐑧𐑮𐑕𐑤𐑰 𐑱𐑚𐑳𐑤 𐑑 𐑚𐑦𐑤𐑰𐑝 𐑦𐑯 𐑞 𐑮𐑰𐑩𐑤𐑳𐑑𐑰 𐑝 𐑥𐑲 𐑩𐑛𐑝𐑧𐑯𐑗𐑻𐑟. 𐑞𐑱 𐑢𐑻 𐑑𐑮𐑵𐑤𐑰 𐑕𐑴 𐑢𐑳𐑯𐑛𐑻𐑓𐑳𐑤 𐑞𐑩𐑑 𐑰𐑝𐑦𐑯 𐑯𐑬 𐑲 𐑩𐑥 𐑚𐑦𐑢𐑦𐑤𐑛𐑻𐑛 𐑢𐑧𐑯 𐑲 𐑔𐑦𐑙𐑒 𐑝 𐑞𐑧𐑥.
     𐑥𐑲 𐑳𐑙𐑒𐑳𐑤 𐑢𐑪𐑟 𐑳 𐑡𐑻𐑥𐑳𐑯, 𐑣𐑩𐑝𐑦𐑙 𐑥𐑧𐑮𐑰𐑛 𐑥𐑲 𐑥𐑳𐑞𐑻𐑟 𐑕𐑦𐑕𐑑𐑻, 𐑩𐑯 𐑦𐑙𐑜𐑤𐑦𐑖𐑢𐑫𐑥𐑳𐑯. 𐑚𐑰𐑦𐑙 𐑝𐑧𐑮𐑰 𐑥𐑳𐑗 𐑳𐑑𐑩𐑗𐑑 𐑑 𐑣𐑦𐑟 𐑓𐑪𐑞𐑻𐑤𐑳𐑕 𐑯𐑧𐑓𐑘𐑵, 𐑣𐑰 𐑦𐑯𐑝𐑲𐑑𐑳𐑛 𐑥𐑰 𐑑 𐑕𐑑𐑳𐑛𐑰 𐑳𐑯𐑛𐑻 𐑣𐑦𐑥 𐑦𐑯 𐑣𐑦𐑟 𐑣𐑴𐑥 𐑦𐑯 𐑞 𐑓𐑪𐑞𐑻𐑤𐑩𐑯𐑛. 𐑞𐑦𐑕 𐑣𐑴𐑥 𐑢𐑪𐑟 𐑦𐑯 𐑳 𐑤𐑪𐑮𐑡 𐑑𐑬𐑯, 𐑯 𐑥𐑲 𐑳𐑙𐑒𐑳𐑤 𐑳 𐑐𐑮𐑳𐑓𐑧𐑕𐑻 𐑝 𐑓𐑳𐑤𐑪𐑕𐑳𐑓𐑰, 𐑒𐑧𐑥𐑳𐑕𐑑𐑮𐑰, 𐑡𐑰𐑪𐑤𐑳𐑡𐑰, 𐑥𐑦𐑯𐑻𐑪𐑤𐑳𐑡𐑰, 𐑯 𐑥𐑧𐑯𐑰 𐑳𐑞𐑻 𐑳𐑤𐑴𐑡𐑰𐑕.

\arial

\hfill Excerpt from Jules Vern,  \textit{Journey to the Center of the Earth from \href{http://shavian.weebly.com/}{shavian}}
\end{scriptexample}

The example is typeset using \texttt{code2001.ttf}. There are numerous fonts that provide Shavian glyphs. \texttt{ESL Gothic Unicode} font by Ethan Lamoreaux\footnote{\url{http://www.fontspace.com/ethan-lamoreaux/esl-gothic-unicode}}. The Noto fonts also have a Shavian font. 

You can activate typesetting in Shavian using the key:

\begin{docKey}[phd]{shavian font}{= \meta{font name}} {}
The key will setup the
default font for the Shavian script and define the commands \cmd{\shavian} and \cmd{\textshavian}. 
\end{docKey}

\PrintUnicodeBlock{./languages/shavian.txt}{\shavian}





\section{Osmanya}
\label{s:osmanya}
\newfontfamily\osmanya{NotoSansOsmanya-Regular.ttf}

\begin{scriptexample}[]{Osmanya}
\unicodetable{osmanya}{"10480,"10490,"104A0}
\end{scriptexample}

The Osmanya alphabet (Somali: Cismaanya; Osmanya: {\osmanya 𐒋𐒘𐒈𐒑𐒛𐒒𐒕𐒀}), also known as Far Soomaali ("Somali writing"), is a writing script created to transcribe the Somali language. It was invented between 1920 and 1922 by Osman Yusuf Kenadid of the Majeerteen Darod clan, the nephew of Sultan Yusuf Ali Kenadid of the Sultanate of Hobyo.

While Osmanya gained reasonably wide acceptance in Somalia and quickly produced a considerable body of literature, it proved difficult to spread among the population mainly due to stiff competition from the long-established Arabic script as well as the emerging Somali alphabet developed by the Somali linguist, Shire Jama Ahmed, which was based on the Latin script.

As nationalist sentiments grew and since the Somali language had long lost its ancient script,[1] the adoption of a universally recognized writing script for the Somali language became an important point of discussion. After independence, little progress was made on the issue, as opinion was divided over whether the Arabic or Latin scripts should be used instead.

In October 1972, due to its simplicity, the fact that it lent itself well to writing Somali since it could cope with all of the sounds in the language, and the already widespread existence of machines and typewriters designed for its use,[2][3] the government of Somali president Mohamed Siad Barre unilaterally elected to use only the Latin script for writing Somali instead of the Arabic or Osmanya scripts.[4] Barre's administration subsequently launched a massive literacy campaign designed to ensure its sole adoption. This led to a sharp decline in use of Osmanya.

\PrintUnicodeBlock{./languages/osmanya.txt}{\osmanya}

\section{Cherokee}
\index{scripts>Cherokee}
\index{scripts>Cherokee>fonts}
\label{s:cherokee}
Windows comes with |Plantagenet Cherokee| font. The |code2000| also has good support for the alphabet. The \texttt{SIL font Charis SIL} also has good support and can be downloaded at \href{http://scripts.sil.org/cms/scripts/page.php?item_id=CharisSIL_download}{scripts.sel.org}, the latest version gave me problems when used with Windows. 

  
\def\textcherokee#1{{\cherokee   #1}}


\begin{docKey}[phd]{cherokee font}{ = \meta{font name}} {default none, initial=code2000}
 Loads the font
command \cmd{\cherokee}. When the command is used it typesets text in
cherokee unicode. There is no need to load the language, unless it is the main document language. For windows the default font is  |Plantagenet Cherokee|. Another font is FreeSerif, which we are using here.
\end{docKey}

\begin{scriptexample}[]{Cherokee}
{\cherokee
\begin{tabular}{lp{8.5cm}}
Translation	  &John (ᏣᏂ) 3:16\\
American Bible Society 1860	&ᎾᏍᎩᏰᏃ ᏂᎦᎥᎩ ᎤᏁᎳᏅᎯ ᎤᎨᏳᏒᎩ ᎡᎶᎯ, ᏕᏅᏲᏒᎩ ᎤᏤᎵᎦ ᎤᏪᏥ ᎤᏩᏒᎯᏳ ᎤᏕᏁᎸᎯ, ᎩᎶ ᎾᏍᎩ ᏱᎪᎯᏳᎲᏍᎦ ᎤᏲᎱᎯᏍᏗᏱ ᏂᎨᏒᎾ, ᎬᏂᏛᏉᏍᎩᏂ ᎤᏩᏛᏗ.\\

(Transliteration)	& nasgiyeno nigavgi unelanvhi ugeyusvgi elohi, denvyosvgi utseliga uwetsi uwasvhiyu udenelvhi, gilo nasgi yigohiyuhvsga uyohuhisdiyi nigesvna, gvnidvquosgini uwadvdi.\\
\end{tabular}}
\end{scriptexample}

\begin{texexample}{Using text...}{cherokee}
\bgroup
\cherokee \large\textbf{ᎾᏍᎩᏰᏃ}
\textcherokee{ᎾᏍᎩᏰᏃ}
\egroup
\end{texexample}

If you have trouble getting them to work\footnote{\url{http://tex.stackexchange.com/questions/132087/displaying-cherokee-text}}

\url{http://www.cherokee.org/AboutTheNation/Language/CherokeeFont.aspx}



%fonts special
\section{Tifnagh}

\newfontfamily\tifinagh{code2000.ttf}

Tifinagh (Berber pronunciation: [tifinaɣ]; also written Tifinaɣ in the Berber Latin alphabet, {\tifinagh  ⵜⵉⴼⵉⵏⴰⵖ} in Neo-Tifinagh, and تيفيناغ in the Berber Arabic alphabet) is a series of abjad and alphabetic scripts used by Berber peoples to write Berber languages.[1]
A modern derivate of the traditional script, known as Neo-Tifinagh, was introduced in the 20th century. A slightly modified version of the traditional script, called Tifinagh Ircam, is used in a number of Moroccan elementary schools in teaching the Berber language to children as well as a number of publications.[2][3]

The word tifinagh is thought to be a Berberized feminine plural cognate of Punic, through the Berber feminine prefix ti- and Latin Punicus; thus tifinagh could possibly mean "the Phoenician (letters)"[4][5] or "the Punic letters".

\bgroup

\noindent\tifinagh
\colorbox{thecodebackground}{\color{black}^^A
\begin{minipage}{\textwidth}
\parindent1pt
\vskip10pt
\leftskip10pt \rightskip\leftskip
Tifnagh     ⵜⵉⴼⵉⵏⴰⵖ [U+2D30-U+2D7F]

ⴰⴳⵍⴷⵓⵏ ⴰⵎⵥⵥⴰ

ⵙ ⵡⴰⵡⴰⵍ ⴳⵔⵉ ⵉⴷⵙ, ⵙⵙⵏⵖ ⵢⴰⵜ ⵜⵖⴰⵡⵙⴰ ⵜⵉⵙⵙ ⵙⵏⴰⵜ  ⵉⵅⴰⵜⵔⵏ: ⵉⵜⵔⵉ ⵙⴳ ⴷⴷ ⵉⴷⴷⴰ ⵓⵔ ⵉⵎⵇⵇⵓⵔ, ⵉⵍⵍⴰ ⵖⴰⵙ ⴰⵏⵛⵜ ⵏ ⵢⴰⵜ ⵜⴰⴷⴷⴰⵔⵜ !

ⴰⵢⴰ ⵓⴽⵣⵖ ⵜ. ⵙⵙⵏⵖ ⵉⵙ ⴱⵕⵕⴰ ⵏ ⵉⵜⵔⴰⵏ ⵣⵓⵏⴷ ⴰⴽⴰⵍ, ⵊⵓⴱⵉⵜⵔ, ⵎⴰⵔⵙ, ⴱⵉⵏⵓⵙ – ⵉⵜⵔⴰⵏ ⵎⵉ ⵏⴽⴼⴰ ⵉⵙⵎⴰⵡⵏ – ⵍⵍⴰⵏ ⴷⵉⵖ ⵉⵜⵔⴰⵏ ⵢⴰⴹⵏ ⵎⵥⵥⵉⵢⵏⵉⵏ, ⵡⵉⵏⵏⴰ ⵓⵔ ⵏⵣⵎⵉⵔ ⴰⴷ ⵏⵥⵔ ⵙ ⵓⵜⵉⵍⵉⵙⴽⵓⴱ. ⴰⴷⴷⴰⵢ ⵢⵓⴼⴰ ⵓⴰⵙⵜⵕⵓⵏⵓⵎ ⵢⴰⵏ ⴷⵉⴳⵙⵏ, ⴷⴰ ⵢⴰⵙ ⵉⵜⵜⴳⴰ ⵙ ⵢⵉⵙⵎ ⵢⴰⵏ ⵡⵓⵜⵜⵓⵏ. ⴷⴰ ⵢⴰⵙ ⵉⵇⵇⴰⵔ ⵙ ⵓⵎⴷⵢⴰⵜ : « ⴰⵙⵜⵔⵓⵉⴷ 3251 ».

ⵓⴽⵣⵖ ⵉⵙ ⴷⴷ ⵉⴷⴷⴰ ⵓⴳⵍⴷⵓⵏ ⵎⵥⵥⵉⵢⵏ ⵙⴳ ⵉⵜⵔⵉ ⵎⵉ ⵇⵇⴰⵔⵏ ⴰⵙⵜⵔⵓⵉⴷ ⴱ612. ⴰⵙⵜⵔⵓⵉⴷ ⴰ, ⵓⵔ ⵉⵜⵓⵥⵔⴰ ⴰⵔ 1909 ⵙ ⵓⵜⵉⵍⵉⵙⴽⵓⴱ. ⵉⵥⵔⴰ ⵜ ⵢⴰⵏ ⵓⴰⵙⵜⵕⵓⵏⵓⵎ ⴰⵜⵓⵔⴽⵉⵢ. ⵉⵙⵙⴽⵏ ⵜⵓⴼⴰⵢⵜ ⵏⵏⵙ ⴳ ⵢⴰⵏ ⵓⴳⵔⴰⵡ ⴰⴳⵔⴰⵖⵍⴰⵏ ⵏ ⵍⴰⵙⵜⵕⵓⵏⵓⵎⵢ. ⵎⴰⵛⴰ, ⴰⴽⴷ ⵢⵉⵡⵏ ⵓⵔ ⵜ ⵢⵓⵎⵏ ⴰⵛⴽⵓ ⵉⵍⵍⴰ ⵉⵍⵙⴰ ⵢⴰⵜ ⵎⵍⵙⵉⵡⵜ ⵓⵔ ⵉⴳⵉⵏ ⴰⵎⵎ ⵜⵉⵏ ⵎⴷⴷⵏ. ⵎⴷⴷⵏ ⵉⵎⵇⵔⴰⵏⴻⵏ, ⴰⵎⴽⴰ ⴰⴽⴽ ⴰⵢ ⴳⴰⵏ.

ⵎⴰⵛⴰ ⵙ ⵓⵎⴷⴰⵣ ⵏ ⵜⵓⵙⵙⵏⴰ ⵏ ⴰⵙⵜⵔⵓⵉⴷ ⴱ612, ⵉⴽⴽⵔ ⵢⴰⵏ ⵓⴷⵉⴽⵜⴰⵜⵓⵔ ⴰⵜⵓⵔⴽⵢ, ⵉⴳⴳ ⴰⵙⵏ ⵛⵛⵉⵍ ⵉ ⵎⴷⴷⵏ ⴰⴷ ⵍⵙⵙⴰⵏ ⵎⵍⵙⵉⵡⵜ ⵏ ⵓⵔⵓⴱⵉⵢⵏ, ⵡⴰⵏⵏⴰ ⵢⴰⴳⵉⵏ ⵉⵏⵖ ⵜ. ⴰⵙⵜⵔⵓⵏⵓⵎ ⵏⵏⴰⵖ, ⵢⵓⵍⵙ ⴷⵉⵖ ⵉ ⵜⵎⵙⴽⴰⵏⵜ ⵏⵏⵙ ⴰⵙⴳⴳⴰⵙ ⵏ 1920, ⵜⵉⴽⴽⵍⵜ ⵏⵏⴰⵖ ⵉⵍⵍⴰ ⵉⵍⵙⴰ ⵢⴰⵜ ⵎⵍⵙⵉⵡⵜ ⵢⵖⵓⴷⴰⵏ ⵛⵉⴳⴰⵏ. ⵜⵉⴽⴽⵍⵜ ⵏⵏⴰⵖ, ⵎⴷⴷⵏ ⴰⴽⴽ ⵓⵎⴻⵏ ⴰⵡⴰⵍ ⵏⵏⵙ.
\par
\vspace*{10pt}
\end{minipage}
}

\section{Unified Canadian Aboriginal Syllabics}

Unified Canadian Aboriginal Syllabics is a Unicode block containing characters for writing Inuktitut, Carrier, several dialects of Cree, and Canadian Athabascan languages. Additions for some Cree dialects, Ojibwe, and Dene can be found at the Unified Canadian Aboriginal Syllabics Extended block.
\medskip

\newfontfamily\aboriginal{code2000.ttf}

\begin{scriptexample}[]{Aboriginal}
\bgroup
\aboriginal
ᒥᓯᐌ ᐃᓂᓂᐤ ᑎᐯᓂᒥᑎᓱᐎᓂᐠ ᐁᔑ ᓂᑕᐎᑭᐟ ᓀᐢᑕ ᐯᔭᑾᐣ ᑭᒋ ᐃᔑ
\bfseries ᑲᓇᐗᐸᒥᑯᐎᓯᐟ ᑭᐢᑌᓂᒥᑎᓱᐎᓂᐠ ᓀᐢᑕ ᒥᓂᑯᐎᓯᐎᓇ᙮
Unicode Block: Unified Canadian Aboriginal Syllabics, UCAS Extended
Text: UDHR: Cree, Swampy ᐯᔭᐠ ᐱᐢᑭᑕᓯᓇᐃᑲᐣ ᐁᐢᐱᑕᐢᑲᒥᑲᐠ ᐊᐢᑭᐠ ᑭᒋ ᐃᑗᐎᐣ ᐃᓂᓂᐎ ᒥᓂᑯᐎᓯᐎᓇ ᐅᒋ
\egroup
\end{scriptexample}

\begin{scriptexample}[]{Unified Canadian Aboriginal Syllabics }
\unicodetable{aboriginal}{"1400,"1410,"1420,"1430,"1450,"1460,"1470,"1480,"1490,"14A0,^^A
"14B0,"14C0,"14D0,"14E0}
\end{scriptexample}
\cxset{section numbering=arabic}

\newfontfamily\miao{MiaoUnicode-Regular.ttf}


\section{Miao}
\label{s:miao}
\parindent1em

The Pollard script, also known as Pollard Miao (Chinese:{\pan 柏格理苗文 Bó Gélǐ} Miao-wen) or Miao, is an abugida loosely based on the Latin alphabet and invented by Methodist missionary Sam Pollard. 

The history of the Miao writing system is very much the story of the myth about the loss of the old Miao writing
system and how this was later recovered. wrote that in all parts of the areas inhabited by the Miao there are
legends of a lost writing system, Due to the expansion ofthe Han people, the Miao
had to migrate southwards, and in connection with that, the writing was
lost during a river crossing or eaten accidentally.
This kind ofmyth is not unique; a similar legend exists among the Karen
in Burma. In the beginning, when the creator was dispensing books to the
various peoples ofthe earth, the Karen overslept and missed out on the gift
of literacy. In some versions of this myth, they were given a book, but it
was consumed in the fires with which they bum their swidden fields. The
Kachin also have a myth that they devoured their own writing out of
hunger,1 as do the Akha,1 while Graham mentions that the ‘legend of a
lost book’ was also found among the Qiang of west Sichuan.2 This myth
about lost books radically influenced the readiness of the Miao to accept
writing.

Peter Mühlhäusler states that it is almost never the case that writing is
created to meet the needs of an aboriginal society, and that writing systems
introduced from the outside are often met with suspicion as the potentialities
of writing are unknown to the people.3 It may seem quite useless, or at
best as some kind of magic. However, counter-examples do exist, like the
Maori in New Zealand and the Cree in Canada. The Miao have, of course,
had contacts with their neighbours, and, with the Chinese as their main
neighbour, the power of writing must always have been well known to
them. The Chinese have probably attached more importance to writing
than any other people in history and this may have strengthened the need
for explaining the absence of writing in Miao society. 

One version of the myth is that the ancient Miao script survived in their embroidery.

The Miao are
famous for their embroidery and usually attach very strong importance to
the amount and quality of embroidery, especially on wedding dresses and
even on the ordinary dresses still worn in most Miao areas even today. The
myth is also partly an explanation of the intricate patterns found in those
embroideries. One such myth is presented by WangJianguang:

\begin{latexquote}
The Miao people originally had writing, but unfortunately it has not been preserved.
As Chiyou was beaten at the battle of Zhuolu by the Yellow Emperor, the Miao
were driven towards the south. When they had to cross various waters, they did not
have time to build boats, so when they forded the riven they were afraid that their
books should become wet. In order to avoid such a disaster they carried the books
on their heads. In this way the people wandered. When they came to the Yangtze
river they all wanted to cross as quickly as possible, but unfortunately the current was
very strong as they came to the middle, and most ofthem were drowned and only a
few managed to get over. The books were also lost and they could not be retrieved.
As [the migration] continued somebody invented a method of embroidering these
characten onto the clothes as a memorial. Therefore traces of the Miao history are
preserved in their clothes and skirts.6
\end{latexquote}

\begin{figure}[htp]
\includegraphics[width=\textwidth]{miao-01}
\caption{According to lengend the ancient Miao script, survived in the Miao embroidery.
source:\protect\href{http://themiaoculture.tumblr.com/}{themiaoculture}}
\end{figure}
Pollard invented the script for use with A-Hmao, one of several Miao languages. The script underwent a series of revisions until 1936, when a translation of the New Testament was published using it. The introduction of Christian materials in the script that Pollard invented caused a great impact among the Miao. Part of the reason was that they had a legend about how their ancestors had possessed a script but lost it. According to the legend, the script would be brought back some day. When the script was introduced, many Miao came from far away to see and learn it.[1][2]

\subsection{Eating Books and Getting a Good Memory}

Another version of the stories claims that the writing was for some reason
eaten by the Miao, resulting in inner qualities, like a good memory for
traditional songs and stories and general cleverness. The first is a legend from
the ‘short skirt Miao’ in Leishan County, Guizhou Province, recorded by Li
Tinggui during the Spring Festival of 1980:

\begin{latexquotation}
In the past the Miao and the Han were brothers who studied under the same
teacher. Both invented a script. Once they had to cross a river and big brother Miao
carried his younger brother Han on his back and therefore he put his script in his
mouth. As he came to the middle ofthe river he slipped and happened to swallow
the script. Therefore the Miao script is in the stomach and is recorded in the heart,
whereas little brother, who sat on his back, held the script in his hand and preserved
it. Thus the Han have a script which they write with their hands and see with their
eyes.23
\end{latexquotation}

Pollard credited the basic idea of the script to the Cree syllabics designed by James Evans in 1838–1841, “While working out the problem, we remembered the case of the syllabics used by a Methodist missionary among the Indians of North America, and resolved to do as he had done” (1919:174). He also gave credit to a Chinese pastor, “Stephen Lee assisted me very ably in this matter, and at last we arrived at a system” (1919:174). In listing the phrases he used to describe devising the script, there is clear indication of intellectual work, not revelation: “we looked about”, “resolved to attempt”, “adapting the system”, “solved our problem” (Pollard 1919:174,175).

Changing politics in China led to the use of several competing scripts, most of which were romanizations. The Pollard script remains popular among Hmong in China, although Hmong outside China tend to use one of the alternative scripts. A revision of the script was completed in 1988, which remains in use.

As with most other abugidas, the Pollard letters represent consonants, whereas vowels are indicated by diacritics. Uniquely, however, the position of this diacritic is varied to represent tone. For example, in Western Hmong, placing the vowel diacritic above the consonant letter indicates that the syllable has a high tone, whereas placing it at the bottom right indicates a low tone.

A still experimental font, that supports Graphite technology is \idxfont{Mia Unicode}\footnote{\url{http://phjamr.github.io/miao.html\#intro}}. The font is licenced under the SIL terms and we are using it in the |phd| package as the default font for the Miao script.



\begin{scriptexample}[]{Miao}
\unicodetable{miao}{"16F00,"16F10,"16F20,"16F30,"16F40,"16F70,"16F80,"16F90}
\end{scriptexample}

{\LARGE\miao 𖼴	𖼵	𖼶	𖼷	𖼸	𖼹	𖼺	}

Features for Miao
There are three features currently available for the Miao script:


The Chuxiong ‘wart’ variant
Stylistic alternates for {\miao 𖼳} and {\miao 𖼴}
Aspiration marker always on right
The ‘wart’ (a translated technical term!) is the small circle in characters like {\miao 𖼁, 𖼅}, and {\miao 𖼾}. In the Chuxiong orthography, it is rendered not as a circle but as a dot on the right of the letter, as shown in point 5 here (pdf).

Miao Unicode has a feature called “chux” for handling this. In LibreOffice you can use this style by typing “Miao Unicode:chux=1” into the font field.


Samuel Pollard was born in Cornwall in 1864.66 After finishing school he
started to work at a bank in London, but in 1886 he decided to become a
missionary. He arrived in China in the year 1887 in order to work for the
Bible Christian China Mission in north Yunnan. After studying Chinese at
the language school in Anqing he and another young missionary, Frank
Dymond, came to the city of Zhaotong in 1888, where missionary work
had been started just a few months before. Premises had been rented in
Jixian Street near the east gate.

In 1890 Pollard married Emma Hainge, who was a missionary of the
CIM at Kunming. Progress was slow in the missionary work among the
Chinese, and the first two Chinese were baptized in 1893. In 1895-6
Pollard and his wife went to England on their first furlough. On his return
two Chinese students of good family took interest in Christianity and were
baptized. Their names were Li Sitifan ‘Stephen Lee’ and Li Yuehan
‘John Lee’.67 They were to play an important role in the work
among the Miao.










\section{N'ko}

\newfontfamily\nko{NotoSansNKo-Regular.ttf}

N'Ko {\nko(ߒߞߏ)} is both a script devised by Solomana Kante in 1949 as a writing system for the Manding languages of West Africa, and the name of the literary language itself written in the script. The term N'Ko means ``I say'' in all Manding languages.

The script has a few similarities to the Arabic script, notably its direction (right-to-left) and the connected letters. It obligatorily marks both tone and vowels.


\begin{scriptexample}[]{N'ko}
\unicodetable{nko}{"07C0,"07D0,"07E0,"07F0}
\end{scriptexample}

The N'Ko alphabet is written from right to left, with letters being connected to one another.

The script is principally used in Guinea and Côte d'Ivoire (respectively by Maninka and Dioula-speakers), with an active user community in Mali (by Bambara-speakers). Publications include a translation of the Qur'an, a variety of textbooks on subjects such as physics and geography, poetic and philosophical works, descriptions of traditional medicine, a dictionary, and several local newspapers. It has been classed as the most successful of the West African scripts.[3] The literary language used is intended as a koine blending elements of the principal Manding languages (which are mutually intelligible), but has a particularly strong Maninka flavour.

The Latin script with several extended characters (phonetic additions) is used for all Manding languages to one degree or another for historical reasons and because of its adoption for "official" transcriptions of the languages by various governments. In some cases, such as with Bambara in Mali, promotion of literacy using this orthography has led to a fair degree of literacy in it. Arabic transcription is commonly used for Mandinka in The Gambia and Senegal.


\section{Mongolian}
^^A\newfontfamily\mongolian{NotoSansMongolian-Regular.ttf}

The classical Mongolian script (in Mongolian script:{\mongolian ᠮᠣᠩᠭᠣᠯ ᠪᠢᠴᠢᠭ᠌} {\pan Mongγol bičig}; in Mongolian Cyrillic: {\pan Монгол бичиг} Mongol bichig), also known as Uyghurjin Mongol bichig, was the first writing system created specifically for the Mongolian language, and was the most successful until the introduction of Cyrillic in 1946. Derived from Uighur, Mongolian is a true alphabet, with separate letters for consonants and vowels. The Mongolian script has been adapted to write languages such as Oirat and Manchu. Alphabets based on this classical vertical script are used in Inner Mongolia and other parts of China to this day to write Mongolian, Sibe and, experimentally, Evenki.

\begin{scriptexample}[]{Mongolian}
\unicodetable{mongolian}{"1820,"1830,"1840,"1850,"1860,"1870,"1880,"1890,"18A0}
\end{scriptexample}

\PrintUnicodeBlock{./languages/mongolian.txt}{\mongolian}


%\parindent1em


\chapter{Internationalization and Globalization}


\section{Introduction}

In this Chapter we discuss the requirements for localization of software and how this can be applied to \latex. In a way this chapter overlaps the one on languages. However, here we focus mostly on LuaTeX solutions. We also extend the discussion to calendric and solar calculations.

Internationalization is the process of designing a software application so that it can potentially be adapted to various languages and regions without engineering changes. Localization is the process of adapting internationalized software for a specific region or language by adding locale-specific components and translating text. Localization (which is potentially performed multiple times, for different locales) uses the infrastructure or flexibility provided by internationalization (which is ideally performed only once, or as an integral part of ongoing development).\index{internationalization}\index{globalization}

The development of routines for software internationalization and globalization has been an ongoing effort for many years. Currently the accepted method for building such software is the use of i18n. This is an abbreviation of the first letter and last letter of the word internationalization and the 18 is the number of characters in the word.

Internationalization based on i18n is not an easy task for \LaTeX. To an extend some of the issues have been removed with the use of Babel and Polyglossia that provide translation strings for many of the worlds scripts. The de facto resource for internationalization is the Unicode Consortium’s \href{http://cldr.unicode.org/}{CLDR} project.\index{i18n}

\section{Enforcing local styles}

To understand the magnitude of the problem let us look at some of the easier parts of localizing. Consider the Greek days of the week.
\medskip
\begin{trivlist}\item[]\panunicode
\begin{tabular}{llll}
\toprule
Day &Normal Form &Abbreviation &Narrow\\
Monday &Δευτέρα &Δευ. &Δ. \\
\midrule
\end{tabular}
\end{trivlist}

In Greek the abbreviated form, is always capitalized and a stop is provided. The same is true for the month. The narrow form can give problems, unless it is for calendars, where the content is clear. This is because "{\panunicode Π}" are the initials for both ``{\panunicode Πέμπτη}" (Thursday), and ``{\panunicode Παρασκευή}" (Friday). 

In date formats with long month format, that do not include the day, the full month form should be used.
In date formats with long month format, that also include the day, the long date format should be used.

If limited space is available, it is possible to omit the period in the abbreviated form of months, but this should be used only when there is a serious technical restriction

Ultimately, we are aiming at providing the necessary rules to build an automated style that can be used by the system.
                

\section{Locales}
\index{locale}

In computing, a \emph{locale} is a set of parameters that defines the user's language, country and any special variant preferences that the user wants to see in their user interface. Usually a locale identifier consists of at least a \textit{languag}e identifier and a \textit{region} identifier.

On POSIX platforms such as Unix, Linux and others, locale identifiers are defined similar to the BCP 47 definition of language tags, but the locale variant modifier is defined differently, and the character set is included as a part of the identifier. It is defined in this format: |[language[_territory][.codeset][@modifier]]|. (For example, Australian English using the UTF-8 encoding is en\_AU.UTF-8.)

For \latex these ``locales'' can be thought of as the settings of language keys through Babel and Polyglossia. These settings have served the community well for many years, but a litany of duct taping through other packages are a testimony to their limitations. Packages for dates, time and number formatting have been developed to assist. Here is my attempt to put the solution on a better footing and to start providing mechanisms via LuaTeX for a 'plugin'
architecture to find improve solutions. 

\section{Common Locale Data Repository}

The Common Locale Data Repository Project, is a project of the Unicode Consortium to provide locale data in the XML format for use in computer applications. CLDR contains locale specific information that an operating system will typically provide to applications. CLDR is written in LDML (Locale Data Markup Language). The information is currently used in International Components for Unicode, Apple's Mac OS X, OpenOffice.org, and IBM's AIX, among other applications and operating systems

\begin{enumerate}
\item Translations for language names.
\item Translations for territory and country names.
\item Translations for currency names, including singular/plural modifications.
\item Translations for weekday, month, era, period of day, in full and abbreviated forms.
\item Translations for timezones and example cities (or similar) for timezones.
\item Translations for calendar fields. This is useful especially in conjuction with PGF presentational forms.
\item Patterns for formatting/parsing dates or times of day.
\item Examplar sets of characters used for writing the language.
\item Patterns for formatting/parsing numbers.
\item Rules for language adapted collation. \label{collation}
\item Rules for formatting numbers in traditional numeral systems (like Roman numerals, Armenian numerals, ...).
\item Rules for spelling out numbers as words.
\item Rules for transliteration between scripts. A lot of it is based on BGN/PCGN romanization.
\item Rules for \emph{delimiters} such as quotations and question marks.
\end{enumerate}

Currently the consortium’s distribution make the data available in both json and xml formats.  These files hold data for a specific \emph{locale}. Sadly missing are any document sectioning information that would have enabled the incorporation of the above into LaTeX and overcoming some of the Babel and Polyglossia limitations.

We do not need many of the files provided by the CLDR unicode consortium and others we are missing. Take for example the |delimiters| file. 

\bgroup
\scriptsize
\begin{phdverbatim}
  "main" = {
    "ff": {
      "identity": {
        "version": {
          "_cldrVersion": "26",
          "_number": "$Revision: 10739 $"
        },
        "generation": {
          "_date": "$Date: 2014-08-07 12:54:13 -0500 (Thu, 07 Aug 2014) $"
        },
        "language": "ff"
      },
      "delimiters": {
        "quotationStart": "„",
        "quotationEnd": "”",
        "alternateQuotationStart": "‚",
        "alternateQuotationEnd": "’"
      }
    }
  }
}
\end{phdverbatim}
\egroup

Of course the |Json| format as it is, is not readable by Lua a format such as:

\begin{verbatim}
delimiters = {
        quotationStart = "«",
        quotationEnd = "»",
        alternateQuotationStart = "\"",
        alternateQuotationEnd = "\""
      }
\end{verbatim}

\begin{texexample}{i18n}{i18-1}

\panunicode
\begin{luacode*}
-- mock the delimiters from the json
-- file
greekname = 'el'
delimiters = {
        quotationStart = "«",
        quotationEnd = "»",
        alternateQuotationStart = [["]],
        alternateQuotationEnd = [["]]
      }
tex.print(delimiters.quotationStart .. 'test' .. delimiters.quotationEnd)
tex.print ([[\gdef\]] .. greekname .. [[quote#1{\directlua{tex.sprint(delimiters.quotationStart .. '#1' .. delimiters.quotationEnd)}}]])
\end{luacode*}

\def\elquote#1{%
  \directlua {tex.sprint(delimiters.quotationStart .. '#1' .. delimiters.quotationEnd)}
}
\end{texexample}



This is of course a much more simplified way of what one needs to program for a full system. The advantage
of producing the \tex definition also through LuaTeX is that we can keep all the code in one place and econd, we can avoid |\csname| costructs.
\begin{texexample}{elquote}{}
\elquote{This is some longer text in Greek quotes.}
\end{texexample}

I have opted to incorporate these files in the |json| format and provide routines for interfacing via the \pkgname{phd} package.  The reason for opting for a json format, is my other attempts to interface the package with |couchdb|.  My preference for a Nosql type of database, is that  they are better suited in handling data that is commonly  found in documents and also many of the routines will be interchangeable for web applications. I am also hoping that the collation information (see \ref{collation}), will eventually lead to better indices, a subject left untouched in the current distribution.\index{json}

\section{The package phd approach}

The package |phd| packge takes an approach to use only json resource files for the provision of language dependent information, rather than TeX commands alone, as is done by Babel and Polyglossia. 

\section{Language and Region Tags}
\index{tags>regions}\index{tags>language}

Languages are represented by tags such as "en"  for English or "el" for Greek. Other languages have no significant variation and are represented by a language subtag such as "en-US".  The names are mostly intuitive, but in many case bear no relationship to their English names, for example Armenian is coded as \textbf{hy}. There is a useful utility at the SIL website for viewing these codes.\footnote{\protect\url{http://www-01.sil.org/iso639-3/codes.asp?order=reference_name&letter=\%25}.} Note that the CLDR database does not cover all the languages listed in the ISO-639.\footcite{iso639} \index{ISO-639}

The language tags are based on the BGN which is mapped to languages based on ISO-639-1.

ISO 639-2 is the alpha-3 code in Codes for the representation of names of languages-- Part 2. There are 21 languages that have alternative codes for bibliographic or terminology purposes. In those cases, each is listed separately and they are designated as "B" (bibliographic) or "T" (terminology). In all other cases there is only one ISO 639-2 code. Multiple codes assigned to the same language are to be considered synonyms. ISO 639-1 is the alpha-2 code.

We will describe the tables using the English language, which is normally the default and Greek as a second language, as the script is distinctive enough to demonstrate their use. We will also explain Lua routines available via the \pkgname{phd} that are provided as alternatives to Babel and Polyglossia.

{layout.lua}

{layout.orientation.characterOrder} = |left_to_right| or |right_to_left|

layout.orientation.lineOrder = |top_to_bottom|

Example \ref{i18-1} loads the Greek internationalization file |layout| and prints the two fields. Before we send it to
the TeX typesetter we sanitize the string underscores using |gsub|. For illustration purposes we have used |gsub| both as an object method and as a function.

\begin{texexample}{i18n}{i18-1}
\begin{luacode}
local c = require("i18n.el.layout")
local s1 = string.gsub(c.el.layout.orientation.characterOrder, '_', '\\textunderscore ')
local s2 = c.el.layout.orientation.lineOrder:gsub('_', '\\textunderscore ')
tex.print('typeof :', type(c))
tex.print(s1, '\\par', s2)
\end{luacode}
\end{texexample}

Of course for Greek the above information is hardly necessary, but at the level of Lua programming, if we are automating the switching of text direction Greek text might signal a change in direction. Let us have another try using the same code for arabic text. All we have to change is the \textbf{el} to \textbf{ar}.

\begin{texexample}{i18n}{i18-2}
\begin{luacode}
local c = require("i18n.ar.layout")
local s1 = string.gsub(c.ar.layout.orientation.characterOrder, '_', '\\textunderscore ')
local s2 = c.ar.layout.orientation.lineOrder:gsub('_', '\\textunderscore ')
tex.print('typeof :', type(c), '\\par')
tex.print(s1, '\\par', s2)
\end{luacode}
\end{texexample}

In the next example we get the string for the first month of the year in the ``abbreviated'' style. I have changed the json
strings directly to Lua for this file to speed up processing.

\begin{texexample}{i18n}{i18-2}
\begin{luacode}
local c = require("i18n.el.cagregorian")
local months = c.el.dates.calendars.gregorian.months.formats
local days = c.el.dates.calendars.gregorian.days.formats

tex.print("\\begin{tabular}{ll} ")
for i=1,12 do
  tex.sprint(i.." &"..months.wide[i].."\\\\ ")
end
tex.print("\\end{tabular}")
\end{luacode}
\end{texexample}

Printing directly to the document has many benefits but does slow developemnt, both of the code as well as the document. Another distraction is transferring arguments from \tex to Lua and vice versa.

Similarly we can print the months in the Italian language by loading the \textbf{i18n.italian} module and iterating through the month strings. I am still thinking about the interface and the best way forward to provide an easy to use and remember interface. 


Let us now develop a longer example. We will load a number of languages and typeset a table for the different months.
Since we are running the example directly in the document, some patience is required. 

\bigskip 

\begin{texexample}{Month string in various languages}{ex:transl}
\bgroup
\parindent0pt
\newfontfamily\langtable{code2000}
\langtable
\scriptsize
\begin{luacode} 

c = require("i18n.irish")
d = require("i18n.russian")
e = require("i18n.latin")
f = require("i18n.german")
g = require("i18n.kannada")
h = require("i18n.lao")
j = require("i18n.turkish")
k = require("i18n.albanian")

local count=0

local months_irish = c.irish.months
local months_russian = d.russian.months
local months_latin = e.latin.months 
local months_german = f.german.months
local months_kannada = g.kannada.months
local months_lao    = h.lao.months
local months_turkish = j.turkish.months
local months_albanian = k.albanian.months
local centering = function()
                     tex.print("\\centering")
end

local par = function()
               tex.print("\\par")   
end

local tabular = function() 
	tex.print("\\begin{tabular}{clllllll}")
	tex.sprint("\\toprule")
end	


local endtabular = function()
	tex.print("\\bottomrule")
	tex.print("\\end{tabular}")
	tex.print("\\medskip")
end

local eol = function()
  return("\\\\")
end


-- center the table
centering()
tabular()
tex.sprint("Month","&Irish", "&Russian", "&Latin", "&Kannada", "&Lao","&Turkish","&Albanian", eol())
tex.sprint("\\midrule")
for i = 1,12 do
   count = i
   tex.sprint(i.."&", months_irish[i],
                 "&",months_russian[i], 
                 "&",months_latin[i], 
                 "&", months_kannada[i], 
                 "&"..months_lao[i], 
                 "&"..months_turkish[i],
                 "&"..months_albanian[i],
                 eol() )
end  
endtabular()
par()

\end{luacode} 
 
\egroup
\end{texexample}

Now some explanation for the code. We started by loading the necessary libraries for the languages that we wanted to print the month strings and allocated them to local variables.

We then iterated through the twelve months of the gregorian table and typeset them. We could have put the languages in a Lua table and iterated over them. I haven't done it so that the code is clearer. I tried to keep the API functions separate as much as possible. We also defined a font using \docAuxCommand{newfontfamily} of the \pkg{fontspec} package to ensure that we can print the Asian and Cyrillic scripts.

The long javanesque object notations make it difficult to work, but once they are set in functions and locals, development is fast. After the detour to explore the i18n tables and available information, we are now ready to tackle the production of multi-lingual calendars and to complete are library on internationalization. Before we do that a detour to understand
the complexity of calendrical calculations and some historical information is required.
\vfill



 
%\parindent1em
\chapter{The Basic LaTeX3 Syntax and Approach}
 \label{ch:l3intro}
 \cxset{epigraph width=0.7\textwidth}
 \epigraph{A final hint: listen carefully to what language users say they
want, until you have an understanding of what they really want. Then
find some way of achieving the latter at a small fraction of the cost
of the former. This is the test of success in language design, and
of progress in programming methodology. Perhaps these two are the same
subject anyway.}{C.A.R. Hoare, 1973}

		
\epigraph{Frank, in case you needed encouragement, please bear this in mind: I'm very much down at the blunt end of (La)TeX -- almost a total end-user. Following an earlier recommendation in this Q\&A, I visited the expl3 manual and was scared witless... Hope you can understand that---it's not a complaint, just an indication of the intellectual/experience distance from here to there.}{---Brent.Longborough Mar 2 '12 at 9:02 at \href{https://tex.stackexchange.com/questions/45838/what-can-i-do-to-help-the-latex3-project/46427\#46427}{SX.TX}}
 
Niklaus Wirth, the developer of the Pascal language long back in the 70’s wrote a paper titled \emph{On the Design of Programming Languages}. In his paper Wirth advocated that an important aspect of language design is \emph{simplicity}. He later on described the lessons learnt from his own works as:\footnote{\protect\url{http://chrisposkitt.com/tag/wirth/}}:

\begin{enumerate}
\item Writing a program is difficult.
\item Writing a correct program is even more so.
\item Writing a publishable program is exacting.
\item Programs are not written. They grow!
\item Controlling growth needs much discipline.
\item Reducing size and complexity is the triumph.
\item Programs must not be regarded as code for computers, but as literature for humans.
\end{enumerate}

The LaTeX3 syntax can only be described with some awe as `different’, although it retains some remnants of 
\tex’s syntax retaining the backslash, it is so different that many developers and package writers have resisted its adoption irrespective of the fact that it offers some solid code. 

Resistance to the language is understandable and noticed early by Computer Science pioneers. Hoare wrote:


\begin{quotation}
A necessary condition for the achievement of any of these objectives
is the utmost simplicity in the design of the language. Without simplicity,
even the language designer himself cannot evaluate the consequences of his
design decisions. Without simplicity, the compiler writer cannot achieve
even reliability, and certainly cannot construct compact, fast and
- efficient compilers. But the main beneficiary of simplicity is the user
of the language. In all spheres of human intellectual and practical
activity, from carpentry to golf, from sculpture to space travel, the
true craftsman is the one who thoroughly understands his tools. And this
applies to programmers too. A programmer who fully understands his
language can tackle more complex tasks, and complete them quicker and
more satisfactorily than if he did not. In fact, a programmer's need
for an understanding of his language is so great, that it is almost
impossible to persuade him to change to a new one. No matter what the
deficiencies of his current language, he has learned to live with them;
he has learned how to mitigate their effects by discipline and documentation,
and even to take advantage of them in ways which would be impossible
in a new and cleaner language which avoided the deficiency.

It therefore seems especially necessary in the design of a new
programming language, intended to attract programmers away from their
current high level language, to pursue the goal of simplicity to an
extreme, so that a programmer can readily learn and remember all its
features, can select the best facility for each of his purposes, can
fully understand the effects and consequences of each decision, and can
then concentrate the major part of his intellectual effort to understanding
his problem and his programs rather than his tool.
\end{quotation}

I have been programming for many years and have a disdain for languages that---as Hoare
put it--- I cannot remember ``all its features’’.  LaTeX3 has not achieved the level of simplicity required in its core. As a tool it fails the simplicity test and effortful learning is necessary to use it effectively. Currently there are probably less than twenty developers that understand it fully. 

Where, \latex3 excels is its architecture, overall plan and direction and modularizing the code to an extend that the required tools reside in logically set modules or classes in \latex’s terminology. What I can promise you, once you master it, there is no looking back. 

\section{Is it stable?}

One question that often arises is the stability of the current \latex~3 code base. Of course the degree to which software are “stable enough” depends on the requirements. Joseph Wright, answering a question on the SX.TX Q\&A site wrote:

\begin{latexquotation}
If you want 'will never change again', then plain TeX is probably your best bet. Knuth does still fix bugs periodically, but most things are now likely to be regarded as 'features' rather than bugs and so it's extremely likely that a document written in plain today will still work totally unchanged in tens of years (assuming TeX systems continue to be available).

The LaTeX2e kernel is also very unlikely to change further, and so is almost if not quite as stable as TeX itself. The team do fix bugs and do allow a bit more leeway than Knuth does, but even so it's extremely unlikely anything will change with LaTeX2e at the kernel level in a way that would require changes in documents.

There are some LaTeX packages one could reasonably decide to use which are also very stable and unlikely to see changes, either because they are no longer being actively developed or because the authors are careful to only change code related to genuine bugs or new, non-breaking, features. Obvious candidates are keyval, graphicx, etc.: probably there is actually quite a decent list, depending on your requirements.

In the case of the LaTeX3 packages l3kernel and l3packages, 'stable' does not extend as far as 'you will never have to make a change to a document using them', at least at this stage. What it means is that the team will not be making 'arbitrary' changes and will document/announce when this happens. Most of l3kernel is 'done', with the plans primarily focused on addition of new functionality rather than altering existing code. However there are a few places where we know some change may be required, and that will be announced on the LaTeX-L mailing list and documented. Even within these changes, 'breaking' (non-back-compatible) alterations will be small in number, but there is at least one of them we still need to do.

In the case of xparse, \docAuxCommand*{DeclareDocumentCommand} and so on are 'stable' in the sense that they will only be augmented, not removed, but there could be some changes on the more esoteric functions (for example, there are questions centred on the \textbf{g} argument type).

Thus 'stable enough' depends on your use case. If you can live with 'will have to make very occasional changes based on documented and scheduled updates' then expl3 is entirely usable. (I and others use if routinely in packages.) On the other hand, if you want 'this code must work with no changes with all future releases of support code' then we are not quite there yet.
\end{latexquotation}

\section{Getting started}

Other than the obvious of making sure you have the latest distribution from the LaTeX3 repository, 
the first step is to understand the conventions used by the \LaTeX3 developers. Macros are termed \meta{functions} and \meta{variables}. Macro names in general use the underscore and the colon in their names.
This is by design and to be honest is part of what many developers are unhappy about. It does cut down on the readability of the code and the longer names are more difficult to remember. This type of naming convention is similar to Hungarian notation, in which the name of a variable or function indicates its type  or its intended use and it does not have a lot of friends. Unlike \latex2e, expl3 uses the |:| and the underscore |_| extensively to produce |snake_case| like variables and functions. As such if the code is to be used together with \latex2e the code has to be run in a category regime in which spaces are ignored and the |_| and |:| are treated as \enquote{letters}. In this respect the functions |ExplSyntaxOn| and |ExplSyntaxOff| are provided. One of the advantages of |expl3| is that all spaces are ignored, avoiding a lot of issues with missing (\%) and unwanted spaces creeping into typeset material. 


Consider the \tex primitive \docAuxCommand*{meaning}. In \latex3 it has been remapped to \docAuxCommand*{token_to_meaning:N}. Similarly \docAuxCommand*{scan_stop:} has been let to \docAuxCommand*{relax}. We will define 

\begin{texexample}{Getting started}{ex:meaning}
\ExplSyntaxOn
% Define somevar in TeX style
\def\somevar{one}
% Get the meaing in TeX style
\meaning\somevar \\

% The LaTeX3 way
\token_to_meaning:N \somevar \\
\token_to_meaning:N \token_to_meaning:N
\token_to_meaning:N \scan_stop:  \\
\ExplSyntaxOff
\end{texexample}

So what is the advantage of the \latex3 conventions and usually longer names? Firstly as mentioned earlier, by using the convention we namespace our macros and this in itself can avoid errors. In a more canonical form we would write part of the earlier example as:

\begin{texexample}{Getting started}{ex:namespacing}
\ExplSyntaxOn
\cs_set:Npn \l_my_somevar:n #1 {#1}
\l_my_somevar:n {one}
\ExplSyntaxOff
\end{texexample}

As you will observe, we have some unfamiliar syntax but the underlying code is still the same. This time I have introduced |:n| at the tail of the function name.

The part that comes after the colon is termed the \emph{function signature}. For example in |token_to_meaning:N| in Example~\ref{ex:meaning}, the function signature is the \textbf{N}. The individual letter \enquote{N} is termed the argument specifier. Another important part is the prefix of the functions. There are some exceptions but the prefix normally indicates the module where the macro has been defined. So |\token_to_meaning:N|  can be found in the |l3token| package.\footnote{The term module and package are used interchangeably by the \latex3 Team.} The prefixes |l| or |g| are used to indicate local and global variables.
One is not forced to use the conventions for prefixes and |_|, however not using them defeats one of the main purposes of \latex3 which is to force developers to namespace their code.

There are a lot of advantages hiding behind the specifier part. By using the argument specifier, the new kernel
provides families of related functions which avoid the
need for complex |\expandafter| runs. For example,
the \tex primitive |\let| can only be used with a
macro name and a single token; no braces. In latex, the family of |\let|-like macros contains:\tcbdocmarginnote{U 2018-01-05}

\begin{texexample}{let with no expandafters}{ex:l3let}
\ExplSyntaxOn
\cs_set:Npn \Macro_Two {macro two replacement text}
\cs_set_eq:NN \Macro_One \Macro_Two
\cs_set_eq:Nc \Macro_One {Macro_Two}
\cs_set_eq:cN {Macro_One} \Macro_Two
\cs_set_eq:cc {Macro_One} {Macro_Two}

\cs_meaning:N \Macro_One

\ExplSyntaxOff
\end{texexample}

We can also write our own variants easily which we can explore a bit later.


Consider the definition of a simple function  |\phd_print_xy:nn| that accepts two values $x,y$ and prints them. This can be defined by one of the |cs_| type functions.

One way we could have defined the macro using  \tex would be:

\begin{teXXX}
\def\phdprint #1#2{x#1 y#2}
\end{teXXX}

Using \latexe we would have probably used |\newcommand| and if the definition was internal to a package used an |@|. 

\begin{teXXX}
\makeatletter
\newcommand\phd@print [2] {x#1 y#2}
\makeatother
\end{teXXX}

In \latex3 we would use |\cs_set_no_par:Npn|.

\begin{teXXX}
\cs_set_nopar:Npn \phd_print_xy:nn #1#2 { x #1 y #2 }
\end{teXXX}

So what is this mysterious |\cs_set_nopar:Npn|? We can find out by peeking at its meaning. This is shown in Example~\ref{ex:somemeaning}. As you can see behind the new dress is Knuth’s same old workhorse |\def|.

\begin{texexample}{The meaning of a command}{ex:somemeaning}
\ExplSyntaxOn
\token_to_meaning:N  \cs_set_nopar:Npn
\ExplSyntaxOff
\end{texexample}

But first let us examine the |:Npn| part of the |\cs_set_no_par:Npn| more carefully. What this means is the macro has three arguments. The first one is |N-type| which is a \tex token. The second one is |p-type|, which denotes normal \tex parameters such as |#1#2|. Lastly the |n-type| can be either a single token or a bracketted parameter. 

There are many more argument specifiers. Functions can be found with different argument specifiers and these are termed \emph{variants}. Recall that a macro can be defined using |\def|, |\edef| or a |\csname| construct. The argument specifier to the |\cs_setnopar| can be varied to achieve it. 

\begin{texexample}{The meaning of a command}{ex:somemeaning}
\ExplSyntaxOn
\token_to_meaning:N  \cs_set:Npn\\
\token_to_meaning:N  \cs_set_nopar:Npx\\
\token_to_meaning:N  \cs_set_nopar:cpx
\ExplSyntaxOff
\end{texexample}

Consider the use  of a |\csname| construct to define our |\phd_print_xy:nn| macro. The example that follows

\begin{texexample}{ex:csname}{ex:csname}

\ExplSyntaxOn
\expandafter\def\csname phd_print_xy:nn\endcsname #1 #2{x#1 y#2}

\token_to_meaning:N \phd_print_xy:nn\\

\cs_set_nopar:cpx {phd_print_xy:nn} #1 #2 {x#1 y#2}
\token_to_meaning:N \phd_print_xy:nn\\
\ExplSyntaxOff
\end{texexample} 

By using \latex3 functions, we do not need to use the |\expandafter| macro. The macro names are generally longer but the overall code is shorter.

So far we have used the |\token_to_meaning:N|. \latex3 offers similar commands to get the argument specification, the prefix and the replacement specification. When we specify a macro in \latex3 we can capture all its constituent parts and handle them individually if we want.

\begin{texexample}{Dissecting a macro}{ex:dissectmacro}
\ExplSyntaxOn

\cs_set_nopar:Npn \phd_print_xy:nn #1#2! { x #1 y #2 }
Macro meaning: \token_to_meaning:N \phd_print_xy:nn \\
Macro argument specification: \token_get_arg_spec:N \phd_print_xy:nn  \\
Prefix Spec: \token_get_prefix_spec:N \phd_print_xy:nn\\
Replacement Spec: \token_get_replacement_spec:N \phd_print_xy:nn

\ExplSyntaxOff
\end{texexample}


Some argue that the syntax is not syntactic sugar but syntactic cyanide that changes the look and feel both of \latexe and \tex command macros. You should think of |expl3| as a new computer language. It does introduce consistency and offers a full repertoire of tools. The syntactic strangeness of the language does introduce barriers to mastering it, but the advantages far outweigh the difficulties of the language.


The eye tends to miss the argument specifier, it is important to note that the macro
name is \cmd{\test\_something:nn} and not \cmd{\test\_something} and the factory command is |\cs_new:Npn| and not |\cs_new|. If you have been programming using traditional macros this is a common mistake that you will accidentally make and you will get an |error unknown| message.


\section{Where from here}

The chapters of this book follow a logical sequence for learning the language, although most of them can be read as stand alone. 

The steps in learning any computer language require a logical sequence of study:

\begin{enumerate}
\item Understanding the syntax
\item Variables and datatypes
\item Numbers and assignments
\item Control Structures
\item Functions
\item Data structures
\item Ecosystem
\end{enumerate}

In the next chapter we would study the creation of functions in more detail. This is the most important skill to master before you proceed with the rest of the programming constructs, such as iteration, arithmetic operations etc.



\chapter{Defining Functions and Variables}

\section{Defining functions}
There are two main methods to define functions. In the first method you are required to use parameter tex, whereas in the second this can be left out, as it can be inferred from the argument specification of the function being defined. The functions used to create other functions can be found in both forms. For example:

\begin{texexample}{Using parameter text}{}
\ExplSyntaxOn
\cs_set_nopar:Npn \phd_print:n #1 {#1}
\token_to_meaning:N \phd_print:n\\

\cs_set_nopar:Nn  \phd_print:n  {#1}


\cs_meaning:N \phd_print:n\\
\ExplSyntaxOff
\end{texexample}



 Functions can be created with no requirement that they are declared
 first (in contrast to variables, which must always be declared).\footnote{This primarily refers to variables that require a \tex register.}
 Declaring a function before setting up the code means that the name
 chosen will be checked and an error raised if it is already in use.
 The name of a function can be checked at the point of definition using
 the \docAuxCommand*{cs_new}\ldots functions: this is recommended for all
 functions which are defined for the first time.

 There are three primary ways to define new functions, using |new|, |set| or |gset| variations.  The first one is similar to the \latexe |\newcommand|, and produces macros that will generate an error if there is an attempt to redefine them. The other two are variations of the |\def or \edef| and |\gdef or \xdef| \tex commands.
 
 All classes define a function to expand to the substitution text.
 Within the substitution text the actual parameters are substituted
 for the formal parameters (|#1|, |#2|, \ldots).
 
 \begin{description}
   \item[\texttt{new}]
     Create a new function with the \texttt{new} scope,
     such as \docAuxCommand* {cs_new:Npn}.  The definition is global and will result in
     an error if it is already defined.
   \item[\texttt{set}]
     Create a new function with the \texttt{set} scope,
     such as \docAuxCommand* {cs_set:Npn}. The definition is restricted to the current
     \TeX{} group and will not result in an error if the function is already
     defined.
   \item[\texttt{gset}]
     Create a new function with the \texttt{gset} scope,
     such as \docAuxCommand* {cs_gset:Npn}. The definition is global and
     will not result in an error if the function is already defined.
 \end{description}

  Finally, the functions in
 Subsections~\ref{sec:l3basics:defining-new-function-1}~and
 \ref{sec:l3basics:defining-new-function-2} are primarily meant to define
 \emph{base functions} only. Base functions can only have the following
 argument specifiers:
 \begin{description}
   \item[|N| and |n|] No manipulation.
   \item[|T| and |F|] Functionally equivalent to |n| (you are actually
     encouraged to use the family of |\prg_new_conditional:| functions
     described in Section~\ref{sec:l3prg:new-conditional-functions}).
   \item[|p| and |w|] These are special cases.
 \end{description}



 Within each set of scope there are different ways to define a function.
 The differences depend on restrictions on the actual parameters and
 the expandability of the resulting function.
 \begin{description}
   \item[\texttt{nopar}]
      Create a new function with the \texttt{nopar} restriction,
      such as \docAuxCommand*{cs_set_nopar:Npn}. The parameter may not contain
      \docAuxCommand*{par} tokens.
   \item[\texttt{protected}]
      Create a new function with the \texttt{protected} restriction,
      such as \docAuxCommand*{cs_set_protected:Npn}. The parameter may contain
      \docAuxCommand*{par} tokens but the function will not expand within an
      \texttt{x}-type expansion.
 \end{description}
 
 
\subsection{Defining new functions using parameter text}

Theses function are \TeX ish in style, as compared to those functions that use the signature to automatically detect the number of parameters and are more \LaTeX-like. They are mainly used with the |:Npn| signature specification.

\begin{texexample}{Using parameter text}{}
\ExplSyntaxOn
\cs_new:Npn \phd_print:n #1 {#1}

\token_to_meaning:N \cs_new:Npn\\
\token_to_meaning:N \phd_print:n
\ExplSyntaxOff
\end{texexample}

\begin{docCommand}{cs_new:Npn} {\meta{function} \meta{parameters} \marg{code}}
Creates \meta{function} to expand to \meta{code} as replacement text. Within the \meta{code}, the
\meta{parameters} (\#1, \#2, etc.) will be replaced by those absorbed by the function. The
definition is \textbf{global} and an error will result if the \meta{function} is already defined.
Variants with |cpn,Npx,cpx| are predefined by the kernel.
\end{docCommand}

The |:Npn| form can also be used even if there is no parameter text. However this is considered a constant variable and is preferred to be coded as a |tl| such.

\begin{texexample}{Usage of the macro}{ex:csnew}
\ExplSyntaxOn
  \cs_new:Npn \copyrightfootnote: 
    {
      \footnotetext{Copyright~(2014-2015)~of~Yiannis~Lazarides,~distributed~
      under~the~\LaTeX{}~Project~Public~License~(LPPL).}
    }
  \copyrightfootnote:
\ExplSyntaxOff
\end{texexample}

An important point to note is if you use the function signature type you will get an error if the trailing |:| is not used in the macro name. 

\begin{teXXX}
\cs_new:Nn \copyrightafootnote 
  {
    ...
  }
\copyrightafootnote
\end{teXXX}

This will produce an error:\ExplSyntaxOn\copyrightfootnote:\ExplSyntaxOff

\begin{verbatim}
! LaTeX error: "kernel/missing-colon"
! Function '\copyrightafootnote' contains no ':'.
! See the LaTeX3 documentation for further information.
! For immediate help type H <return>.
\end{verbatim}

If the function is redefined, it will produce an error, similar to \latexe |\newcommand|. However, do note that the |set| family of commands can silently overwrite it. 

\begin{texexample}{Usage of the macro \protect\string\cs\_gset:Npn}{ex:csnew}
\ExplSyntaxOn
\cs_gset:Npn \copyrightfootnote: {\footnotetext{Copyright~(2014-2015)~of~
         Yiannis~Lazarides,~distributed~
         under~the~\LaTeX{}~Project~Public~License~(LPPL).}}
\copyrightfootnote:
\ExplSyntaxOff
\end{texexample}

\begin{docCommand}{cs_new_nopar:Npn} {\meta{function} \meta{parameters} \marg{code}}
Creates \meta{function} to expand to \meta{code} as replacement text. Within the \meta{code}, the
\meta{parameters} (\#1, \#2, etc.) will be replaced by those absorbed by the function. When the
\meta{function} is used the hparametersi absorbed cannot contain \par tokens. The definition
is global and an error will result if the \meta{function} is already defined.
\end{docCommand}

\begin{texexample}{Meaning}{}
\ExplSyntaxOn
\token_to_meaning:N \cs_new_nopar:Npn
\ExplSyntaxOff
\end{texexample}

\begin{docCommand}{cs_new_protected:Npn}{\meta{function} \meta{parameters} \marg{code}}
Creates \meta{function} to expand to \meta{code} as replacement text. Within the hcodei, the
hparametersi (\#1, \#2, etc.) will be replaced by those absorbed by the function. The
\meta{function} will not expand within an x-type argument. The definition is global and an
error will result if the hfunctioni is already defined.
\end{docCommand}

\begin{docCommand}{cs_new_protected_nopar:Npn}{\meta{function} \meta{parameters} \marg{code}}
Creates \meta{function} to expand to \meta{code} as replacement text. 
When the \meta{function} is used the \meta{parameters} absorbed cannot contain \docAuxCommand*{par} tokens. The hfunctioni
will not expand within an x-type argument. The definition is global and an error will
result if the \meta{function} is already defined.
\end{docCommand}

This brings us to the end of the |new| type functions that can be used for function definitions. They all have variants of the form |cpn| and |cpx| and the base function for edef also is available. You can consult the manual for more definitions.

\subsubsection{The set type functions}

The rest of the commands are variations using the \cs{cs_}\meta{set} form of function creating macros. These do not issue a  warning if redefined.

 \begin{docCommand}{cs_set:Npn} {\meta{function} \meta{parameters} \marg{code}}
   Sets \meta{function} to expand to \meta{code} as replacement text.
   Within the \meta{code}, the \meta{parameters} (|#1|, |#2|,
   \emph{etc.}) will be replaced by those absorbed by the function.
   The assignment of a meaning to the \meta{function} is restricted to
   the current \TeX{} group level.
\end{docCommand}

\begin{texexample}{Meaning}{}
\ExplSyntaxOn
\token_to_meaning:N \cs_set:Npn
\ExplSyntaxOff
\end{texexample}

As can be seen from the example this is |\protected \long \def|. The |\cs_set_nopar:Npn| in the maeaning in the example is described next and is simply an equivalent function to |\def|.

 \begin{docCommand} {cs_set_nopar:Npn}{\meta{function} \meta{parameters} \marg{code}}
   Sets \meta{function} to expand to \meta{code} as replacement text.
   Within the \meta{code}, the \meta{parameters} (|#1|, |#2|,
   \emph{etc.}) will be replaced by those absorbed by the function.
   When the \meta{function} is used the \meta{parameters} absorbed
   cannot contain \cs{par} tokens. The assignment of a meaning
   to the \meta{function} is restricted to the current \TeX{} group
   level.
 \end{docCommand}
 
 \begin{texexample}{Meaning \textbackslash cs\_set\_nopar:Npn}{}
\ExplSyntaxOn
\token_to_meaning:N \cs_set_nopar:Npn
\ExplSyntaxOff
\end{texexample}
 

\begin{docCommand}{cs_set_protected:Npn} {\meta{function} \meta{parameters} \marg{code}}
   Sets \meta{function} to expand to \meta{code} as replacement text.
   Within the \meta{code}, the \meta{parameters} (|#1|, |#2|,
   \emph{etc.}) will be replaced by those absorbed by the function.
   The assignment of a meaning to the \meta{function} is restricted to
   the current \TeX{} group level. The \meta{function} will
   not expand within an \texttt{x}-type argument.
 \end{docCommand}
 \begin{texexample}{Meaning \textbackslash cs\_set\_protected:Npn}{}
 \ExplSyntaxOn
 \token_to_meaning:N \cs_set_protected:Npn
\ExplSyntaxOff
\end{texexample}
 


\begin{docCommand}{cs_set_protected_nopar:Npn}{\meta{function} \meta{parameters} \marg{code}}
   Sets \meta{function} to expand to \meta{code} as replacement text.
   Within the \meta{code}, the \meta{parameters} (|#1|, |#2|,
   \emph{etc.}) will be replaced by those absorbed by the function.
   When the \meta{function} is used the \meta{parameters} absorbed
   cannot contain \cs{par} tokens. The assignment of a meaning
   to the \meta{function} is restricted to the current \TeX{} group
   level. The \meta{function} will not expand within an
   \texttt{x}-type argument.
\end{docCommand}
\begin{texexample}{Meaning \textbackslash cs\_set\_protected\_nopar:Npn}{}
\ExplSyntaxOn
\token_to_meaning:N \cs_set_protected_nopar:Npn
\ExplSyntaxOff
\end{texexample}
 
Next the above are made available by the \latex3 kernel but all in the |global| form of the command. The syntax is identical except they use |cs_gset|.


\begin{docCommand} {cs_gset:Npn}{\meta{function} \meta{parameters} \marg{code}}
   Globally sets \meta{function} to expand to \meta{code} as replacement
   text. Within the \meta{code}, the \meta{parameters} (|#1|, |#2|,
  \emph{etc.}) will be replaced by those absorbed by the function.
  The assignment of a meaning to the \meta{function} is \emph{not}
   restricted to the current \TeX{} group level: the assignment is
   global.
\end{docCommand}
\begin{texexample}{Meaning \textbackslash cs\_gset:Npn}{}
\ExplSyntaxOn
\token_to_meaning:N \cs_gset:Npn
\ExplSyntaxOff
\end{texexample}

\begin{docCommand}{cs_gset_nopar:Npn} {\meta{function} \meta{parameters} \marg{code}}
   Globally sets \meta{function} to expand to \meta{code} as replacement
   text. Within the \meta{code}, the \meta{parameters} (|#1|, |#2|,
   \emph{etc.}) will be replaced by those absorbed by the function.
   When the \meta{function} is used the \meta{parameters} absorbed
   cannot contain \cs{par} tokens. The assignment of a meaning to the
   \meta{function} is \emph{not} restricted to the current \TeX{}
   group level: the assignment is global.
\end{docCommand}
\begin{texexample}{Meaning \textbackslash cs\_gset\_nopar:Npn}{}
\ExplSyntaxOn
\token_to_meaning:N \cs_gset_nopar:Npn
\ExplSyntaxOff
\end{texexample}


\begin{docCommand} {cs_gset_protected:Npn} {\meta{function} \meta{parameters} \marg{code}}
   Globally sets \meta{function} to expand to \meta{code} as replacement
   text. Within the \meta{code}, the \meta{parameters} (|#1|, |#2|,
   \emph{etc.}) will be replaced by those absorbed by the function.
   The assignment of a meaning to the \meta{function} is \emph{not}
   restricted to the current \TeX{} group level: the assignment is
   global. The \meta{function} will not expand within an
   \texttt{x}-type argument.
\end{docCommand}
\begin{texexample}{Meaning \textbackslash cs\_gset\_protected:Npn}{}
\ExplSyntaxOn
\token_to_meaning:N \cs_gset_protected:Npn
\ExplSyntaxOff
\end{texexample}

\begin{docCommand}{cs_gset_protected_nopar:Npn} {\meta{function} \meta{parameters} \marg{code}}
   Globally sets \meta{function} to expand to \meta{code} as replacement
   text. Within the \meta{code}, the \meta{parameters} (|#1|, |#2|,
   \emph{etc.}) will be replaced by those absorbed by the function.
   When the \meta{function} is used the \meta{parameters} absorbed
   cannot contain \cs{par} tokens. The assignment of a meaning to the
   \meta{function} is \emph{not} restricted to the current \TeX{}
   group level: the assignment is global. The \meta{function} will
   not expand within an \texttt{x}-type argument.
\end{docCommand}
\begin{texexample}{Meaning \textbackslash cs\_gset\_protected\_nopar:Npn}{}
\ExplSyntaxOn
\token_to_meaning:N \cs_gset_protected_nopar:Npn
\ExplSyntaxOff
\end{texexample}

This brings us to the end of the functions available to the developer for defining macros. It’s a lot of them. In the next section some more functions are defined, this time using the signature of the function the function are created automatically without the need to type in the parameter text.


\subsection{Defining new functions using the signature}

The functions outlined below have a simpler form in that they create other commands without the need to specify their arguments. The number of parameters is detected automatically from the function signature. Which method is the best is obvious up to the user preferences.\footnote{See discussion at SX.TX \protect{\url{http://tex.stackexchange.com/questions/240675/differences-in-latex3-function-generation-methods}}} 


\begin{docCommand}{cs_new:Nn}{\meta{function}\marg{code}}
Creates \meta{function} to expand to \meta{code} as replacement text. A nice feature is that within the \meta{code}
the number of parameters is detected automatically from the function signature. These \meta{parameters} (\#1, \#2, etc.) will be replaced by those absorbed by the function. The definition is global and an error will result if the \meta{function} is already defined.\footnote{The definitions of the commands have been taken mostly verbatim from the documentation of the package.}


\begin{texexample}{Signature}{ex:signature}
\ExplSyntaxOn
\cs_new:Nn \exampleone:nn {}
\cs_new:Nn \exampletwo:nn{#1 #2}
\exampleone:nn {one}{two}

\exampletwo:nn{one }{two}

\texttt\textbackslash\cs_to_str:N\exampleone:nn
\ExplSyntaxOff
\end{texexample}
\end{docCommand}

 
 
 
\begin{docCommand}{cs_new_nopar:Nn}{\meta{function} \marg{code}}
   Creates \meta{function} to expand to \meta{code} as replacement text.
   Within the \meta{code}, the number of \meta{parameters} is detected
   automatically from the function signature. These \meta{parameters}
   (|#1|, |#2|, \emph{etc.}) will be replaced by those absorbed by the
   function.  When the \meta{function} is used the \meta{parameters}
   absorbed cannot contain \docAuxCommand*{par} tokens. The definition is global and
   an error will result if the \meta{function} is already defined.
 \end{docCommand}

\begin{function}[pTF]{\cs_if_exist:N}
\begin{docCommand}{cs_new_protected:Nn}{\meta{function} \marg{code}}
   Creates \meta{function} to expand to \meta{code} as replacement text.
   Within the \meta{code}, the number of \meta{parameters} is detected
   automatically from the function signature. These \meta{parameters}
   (|#1|, |#2|, \emph{etc.}) will be replaced by those absorbed by the
   function. The \meta{function} will not expand within an \texttt{x}-type
   argument. The definition is global and
   an error will result if the \meta{function} is already defined.
\end{docCommand}
\end{function}  

%
% \begin{function}
%   {
%     \docAuxCommand*_new_protected_nopar:Nn, \docAuxCommand*_new_protected_nopar:cn,
%     \docAuxCommand*_new_protected_nopar:Nx, \docAuxCommand*_new_protected_nopar:cx
%   }
%   \begin{syntax}
%     \docAuxCommand*{cs_new_protected_nopar:Nn} \meta{function} \Arg{code}
%   \end{syntax}
%   Creates \meta{function} to expand to \meta{code} as replacement text.
%   Within the \meta{code}, the number of \meta{parameters} is detected
%   automatically from the function signature. These \meta{parameters}
%   (|#1|, |#2|, \emph{etc.}) will be replaced by those absorbed by the
%   function.  When the \meta{function} is used the \meta{parameters}
%   absorbed cannot contain \docAuxCommand*{par} tokens. The \meta{function} will not
%   expand within an \texttt{x}-type argument. The definition is global and
%   an error will result if the \meta{function} is already defined.
% \end{function}

Similarly to the |cs_new| commands the |cs_set| functions create other commands, this time
with a local scope. This pattern is followed right through the kernel.

 \begin{docCommand}{cs_set:Nn}{\meta{function}\marg{code}}
   Sets \meta{function} to expand to \meta{code} as replacement text.
   Within the \meta{code}, the number of \meta{parameters} is detected
   automatically from the function signature. These \meta{parameters}
   (|#1|, |#2|, \emph{etc.}) will be replaced by those absorbed by the
   function.
   The assignment of a meaning to the \meta{function} is restricted to
   the current \TeX{} group level.
 \end{docCommand}

\begin{docCommand}{cs_set_nopar:Nn}{\meta{function}\marg{code}}
   Sets \meta{function} to expand to \meta{code} as replacement text.
   Within the \meta{code}, the number of \meta{parameters} is detected
   automatically from the function signature. These \meta{parameters}
   (|#1|, |#2|, \emph{etc.}) will be replaced by those absorbed by the
   function.  When the \meta{function} is used the \meta{parameters}
   absorbed cannot contain \docAuxCommand*{par} tokens.
   The assignment of a meaning to the \meta{function} is restricted to
   the current \TeX{} group level. This is the \tex primitive \docAuxCommand*{def}
\end{docCommand}

\begin{teX}
\tex_let:D \cs_set_nopar:Npn \tex_def:D
748 \tex_let:D \cs_set_nopar:Npx \tex_edef:D
749 \etex_protected:D \cs_set_nopar:Npn \cs_set:Npn
750                     { \tex_long:D \cs_set_nopar:Npn }
751 \etex_protected:D \cs_set_nopar:Npn \cs_set:Npx
752                   { \tex_long:D \cs_set_nopar:Npx }
753 \etex_protected:D \cs_set_nopar:Npn \cs_set_protected_nopar:Npn
754 { \etex_protected:D \cs_set_nopar:Npn }
755 \etex_protected:D \cs_set_nopar:Npn \cs_set_protected_nopar:Npx
756 { \etex_protected:D \cs_set_nopar:Npx }
757 \cs_set_protected_nopar:Npn \cs_set_protected:Npn
758 { \etex_protected:D \tex_long:D \cs_set_nopar:Npn }
759 \cs_set_protected_nopar:Npn \cs_set_protected:Npx
760 { \etex_protected:D \tex_long:D \cs_set_nopar:Npx }
\end{teX}
\ExplSyntaxOn
\meaning\cs_new:Npn
\ExplSyntaxOff


\begin{docCommand}{cs_set_protected:Nn}{\meta{function}\marg{code}}
   Sets \meta{function} to expand to \meta{code} as replacement text.
   Within the \meta{code}, the number of \meta{parameters} is detected
   automatically from the function signature. These \meta{parameters}
   (|#1|, |#2|, \emph{etc.}) will be replaced by those absorbed by the
   function. The \meta{function} will not expand within an \texttt{x}-type
   argument.
   The assignment of a meaning to the \meta{function} is restricted to
   the current \TeX{} group level.
 \end{docCommand}

\begin{docCommand}{cs_set_protected_nopar:Nn}{ \meta{function} \marg{code}}
   Sets \meta{function} to expand to \meta{code} as replacement text.
   Within the \meta{code}, the number of \meta{parameters} is detected
   automatically from the function signature. These \meta{parameters}
   (|#1|, |#2|, \emph{etc.}) will be replaced by those absorbed by the
   function.  When the \meta{function} is used the \meta{parameters}
   absorbed cannot contain \docAuxCommand*{par} tokens. The \meta{function} will not
   expand within an \texttt{x}-type argument.
   The assignment of a meaning to the \meta{function} is restricted to
   the current \TeX{} group level.
 \end{docCommand}

The next commands create functions with global scope.

 \begin{docCommand}{cs_gset:Nn}{ \meta{function} \marg{code}}
   Sets \meta{function} to expand to \meta{code} as replacement text.
   Within the \meta{code}, the number of \meta{parameters} is detected
   automatically from the function signature. These \meta{parameters}
   (|#1|, |#2|, \emph{etc.}) will be replaced by those absorbed by the
   function.
   The assignment of a meaning to the \meta{function} is  global.
 \end{docCommand}

 \begin{docCommand}{cs_gset_nopar:Nn}{ \meta{function} \marg{code}}
   Sets \meta{function} to expand to \meta{code} as replacement text.
   Within the \meta{code}, the number of \meta{parameters} is detected
   automatically from the function signature. These \meta{parameters}
   (|#1|, |#2|, \emph{etc.}) will be replaced by those absorbed by the
   function.  When the \meta{function} is used the \meta{parameters}
   absorbed cannot contain \docAuxCommand*{par} tokens.
   The assignment of a meaning to the \meta{function} is global.
 \end{docCommand}
 

\section{Copying control sequences}

Control sequences (not just functions as defined above) can be set to have the same
meaning using the functions described here. Making two control sequences equivalent
means that the second control sequence is a copy of the first (rather than a pointer to
it). Thus the old and new control sequence are not tied together: changes to one are not
reflected in the other. These are syntactic replacements for the \tex primitive|\let|.

\begin{texexample}{Let}{}
\ExplSyntaxOn
\cs_set_nopar:Nn \testa: {AAA}
\cs_set_eq:NN\testb: \testa:
\token_to_meaning:N \testa:  \\
\cs_set_nopar:Nn \testa: {BBBB}
\testb:  \\
\token_to_meaning:N \testb:  \\
\token_to_meaning:N \testa:  \\
\testa:\\
\testb: \\
\meaning\cs_set_eq:NN

% check if equal to \let
\token_to_meaning:N \let\\
\token_to_meaning:N \cs_set_equal:NN
\ExplSyntaxOff
\end{texexample}

 In the following text \enquote{cs} is used as an abbreviation for
 \enquote{control sequence}.

 \begin{docCommand}{cs_new_eq:NN} {\meta{cs1} \meta{cs2}}
   Globally creates \meta{control sequence 1} and sets it to have the same
   meaning as \meta{control sequence 2} or |<token>|.
   The second control sequence may
   subsequently be altered without affecting the copy.
\end{docCommand}


\begin{docCommand}{cs_set_eq:NN} {\meta{cs1} \meta{cs2}}
   Sets \meta{control sequence1} to have the same meaning as
   \meta{control sequence2} (or |<token>|).
   The second control sequence may subsequently be
   altered without affecting the copy. The assignment of a meaning
   to the \meta{control sequence1} is restricted to the current
   \TeX{} group level.
 \end{docCommand}


\begin{docCommand} {cs_gset_eq:NN} {\meta{cs1} \meta{cs2}}
   Globally sets \meta{control sequence1} to have the same meaning as
   \meta{control sequence2} (or |<token>|).
   The second control sequence may subsequently be
   altered without affecting the copy. The assignment of a meaning to
   the \meta{control sequence1} is \emph{not} restricted to the current
   \TeX{} group level: the assignment is global.
\end{docCommand}

\section{Undefining control sequences}

There are occasions where control sequences need to be deleted. This is handled in a
very simple manner by the use of 
|\cs_undefine:N| \meta{control sequence},
which sets \meta{control sequence} to be globally |undefined|.

\begin{texexample}{Undefining control sequences}{ex:undefine}
\ExplSyntaxOn
\cs_set_nopar:Npn \testa: {AAA}
\cs_set_nopar:cpn {testb} {AAA}

\cs_undefine:N \testa:
\cs_undefine:c {testb}
\token_to_meaning:N \cs_undefine:N\\

\token_to_meaning:N \testa:\\
\token_to_meaning:c {testb}\\
\token_to_meaning:N \token_to_meaning:c
\ExplSyntaxOff
\end{texexample}

The function would simply set the command to the \tex primitive |undefine|, as can be seen from the example.
There is another group of commands associated with constructor functions.

\section{Converting to and from control sequences}

\begin{docCommand}{cs_if_exist_use:N} {\meta{control sequence}}
Tests whether the \meta{control sequence} is currently defined (whether as a function or another
control sequence type), and if it does inserts the \meta{control sequence} into the input stream.
\end{docCommand}

\begin{docCommand}{cs_if_exist_use:NTF} {\meta{control sequence}}
Tests whether the \meta{control sequence} is currently defined (whether as a function or another
control sequence type), and if it does inserts the \meta{control sequence} into the input stream
followed by the \meta{true code}.
\end{docCommand}

\begin{texexample}{Converting to and from control sequences}{ex:ifexists}
\ExplSyntaxOn
\cs_if_exist_use:NTF \test {}{\FALSE}
\ExplSyntaxOff
\end{texexample}

Note that numerous times, I have typed |\cs_if_exists_use:NTF| rather than the more grammatical  |\cs_if_exist_use:NTF| with consequent errors. Grammar is hardwired in the brain and it requires mental effort to write ungrammatical commands. This is an issue that needs to be addressed by the \latex3 developers. 

The famous |\csname| is mapped in this section of the module as well. Unpredictably, it got a shorter name, but a weird suffix |w|! It deserves both as it is the workhorse of \tex. The remapped commands are formally described in the manual as shown below:

\begin{docCommand}{cs:w } {\meta{control sequence name} \textcolor{thecs}{\texttt{cs\_end:}}}
Converts the given \meta{control sequence name} into a single control sequence token. This
process requires one expansion. The content for \meta{control sequence name} may be literal
material or from other expandable functions. The \meta{control sequence name} must, when
fully expanded, consist of character tokens which are not active: typically, they will be
of category code 10 (space), 11 (letter) or 12 (other), or a mixture of these.
\end{docCommand}


\section{User Commands}

All the commands above are at the programming level. For the development of user commands the \pkgname{xparse} package provides some extremely useful commands. These are dealt under \nameref{ch:xparse}
on page \pageref{ch:xparse}.

\begin{teXXX}
\NewDocumentCommand{\kant}{s>{\SplitArgument{1}{-}}O{1-7}}
  {
   \group_begin:
   \IfBooleanTF{#1} (*@\label{starargument}@*)
     { \cs_set_eq:NN \kgl_par: \kgl_star: }
     { \cs_set_eq:NN \kgl_par: \kgl_nostar: }
     \kgl_process:nn #2
    \kgl_print:
   \group_end:
  }
\end{teXXX}

In Line~\ref{starargument} we test for the star version of the command and then we continue examining the optional argument |O{1-7}|, but first and here is the magic, we have passed the argument through a pre-processing macro named |\SplitArgument|, which has captured the splitted argument and placed it, into two braced macros. It then passes it to a second macro |\getwords| that expects two mandatory aruguments and which handles the typesetting of the two words.
    
\begin{texexample}{Split Argument}{}    
\NewDocumentCommand{\separatewords}{>{\SplitArgument{1}{-}}m}{\getwords#1}
\NewDocumentCommand{\getwords}{ m m }{First word:#1  Second~Word:#2}
\separatewords{mail-coach}

\separatewords{mail}

\end{texexample}    

A similar example see TX.SX.\footnote{\protect{\url{http://tex.stackexchange.com/questions/154941/new-command-in-tex-for-fraction/154950\#154950}}}


\chapter{LaTeX3 Control Structures}
 \section{The boolean data type}

 This section describes a boolean data type which is closely
 connected to conditional processing as sometimes you want to
 execute some code depending on the value of a switch
 (\emph{e.g.},~draft/final) and other times you perhaps want to use it as a
 predicate function in an |if_predicate:w| test. The problem of the
 primitive \docAuxCommand*{if_false:} and \docAuxCommand*{if_true:} tokens is that it is not
 always safe to pass them around as they may interfere with scanning
 for termination of primitive conditional processing. In \latex3
 two canonical booleans ar employed: \docAuxCommand*{c_true_bool} or
\docAuxCommand{c_false_bool}. Besides preventing problems as described above. This also let
to the implementation of  a simple boolean parser supporting the
 logical operations And, Or, Not, \emph{etc.}\ which can then be used on
 both the boolean type and predicate functions.

 All conditional |\bool_| functions except assignments are expandable
 and expect the input to also be fully expandable (which will generally
 mean being constructed from predicate functions, possibly nested).
 
Before a boolean can be used it needs to be created with \docAuxCommand{bool_new:N}, but first let us make sure we understand what a boolean is. A Boolean data type is a data type, having two values (usually denoted \emph{true} and \emph{false}), intended to represent the truth values of logic and Boolean algebra. It is named after George Boole, who first defined an algebraic system of logic in the mid 19th century. 

So how does \latex3 construct a boolean? If we examine the code, which we will in a small example, we can see that a boolean variable is just another macro that either stores 0 or 1. If the value is odd then the boolean is \emph{true} else the boolean is \emph{false}. 

\begin{teXXX}
\tex_chardef:D \c_true_bool = 1 ~
\tex_chardef:D \c_false_bool = 0 ~
\end{teXXX}

\begin{teXXX}
 \cs_new_protected:Npn \bool_new:N #1 { \cs_new_eq:NN #1 \c_false_bool }
 \cs_generate_variant:Nn \bool_new:N { c }
\end{teXXX}

When a new boolean is constructed it is always set to false, as is evident from its code. 

Here is the formal syntax of the |\bool_new:N| function.

 \begin{docCommand}{bool_new:N}{\meta{boolean}}
   Creates a new \meta{boolean} or raises an error if the
   name is already taken. The declaration is global. The
   \meta{boolean} will initially be \texttt{false}. Once the boolean is created
   it can be set to logical true or false using \docAuxCommand*{bool_set_false:N} and \docAuxCommand*{bool_set_true:N}.
 \end{docCommand}
 
\begin{texexample}{Booleans}{}
\ExplSyntaxOn
\bool_new:N \mybool
\bool_set_false:N \mybool
\bool_if:NTF\mybool { \PASS } { \FAIL }
\ExplSyntaxOff
\end{texexample}
 

 
The real strength of the \latex~3 macros are the convenience of providing for |Or| and |And|
operations, negation etc.  and for its ability to evaluate fully boolean expressions. 

\begin{docCommand}{bool_if:nTF}{\marg{boolean expression} \marg{true code} \marg{false code}}
   Tests the current truth of \meta{boolean expression}, and
   continues expansion based on this result. The
   \meta{boolean expression} should consist of a series of predicates
   or boolean variables with the logical relationship between these
   defined using |&&| (\enquote{And}), \verb"||" (\enquote{Or}),
   |!| (\enquote{Not}) and parentheses. Minimal evaluation is used
   in the processing, so that once a result is defined there is
   not further expansion of the tests. 
\end{docCommand}   



\begin{texexample}{Booleans}{}
\ExplSyntaxOn
\bool_new:N\chapterfloat
\bool_new:N\numberfloat
\bool_set_false:N\chapterfloat
\bool_set_true:N\numberfloat

\bool_if:nTF {\chapterfloat || \numberfloat}  { \TRUE }{ \FALSE }

\bool_if:nTF {\chapterfloat && \numberfloat}  { \TRUE }{ \FALSE }

\ExplSyntaxOff
\end{texexample}

\subsection{\textbackslash if\_meaning}

The primitive |ifx| conditional has an equivalent in \latex3. This is called more semantically \docAuxCommand*{if_meaning:w}. This compares two tokens based on their meaning.



\begin{texexample}{Test ifx}{}
\ExplSyntaxOn
\group_begin:
  \cs_set_nopar:Npn \a: {BBB}
  \cs_set_nopar:Npn \b: {BBB~}
  \cs_set_nopar:Npn \c: {B~BB}
  \cs_set_nopar:Npn \d: {BBB}   
  
  % Fail
  \if_meaning:w \a:\b: \PASS \else: \FAIL \fi:
  \if_meaning:w \a:\c: \PASS \else: \FAIL \fi:
  % Passes
  \if_meaning:w \a:\d: \PASS \else: \FAIL \fi:
\group_end:  
\ExplSyntaxOff
\end{texexample}

\begin{texexample}{LaTeX2e booleans}{}
\makeatletter
\ExplSyntaxOn
\if@mainmatter
     in~main~text
   \else
    not~in~main~text  
\fi    

 \meaning\@mainmattertrue\\
\bool_new:N \phd_mainmatter_bool 
\meaning\phd_mainmatter_bool
\ExplSyntaxOff
\makeatother  
\end{texexample}

\section{Predicate functions}

Predicate functions are one of the more powerful features of |expl3|. What are predicate functions? They are macros that test a predicate (\meta{true} or \meta{false}) and branch to either a true or false branch or just a single branch depending on the signature of the function. The |expl3| package has numerous such functions for example:

\begin{teXXX}
 \str_if_eq:nnT {}{}{}
\end{teXXX}

accepts two strings and if true does something. The expl3 package, provides a function that can generate such predicate functions fairly easily.

\begin{docCommand}{prg_set_conditional:Npnn }{\meta {function name}: \meta{arg spec} \meta{parameters} \marg{<conditions code>}}

These functions create a family of conditionals using the same \meta{code} to perform the
test created. Those conditionals are expandable if \meta{code} is. The new versions will
check for existing definitions and perform assignments globally (cf. |\cs_new:Npn|) whereas
the set versions do no check and perform assignments locally (cf. |\cs_set:Npn|). The
conditionals created are dependent on the comma-separated list of \meta{conditions}, which
should be one or more of p, T, F and TF.
\end{docCommand}

\begin{teXXX}
\prg_set_conditional:Npnn \cs_if_exist:N #1 { p , T , F , TF }
 {
 \if_meaning:w #1 \scan_stop:
   \prg_return_false:
     \else:
        \if_cs_exist:N #1
           \prg_return_true:
        \else:
          \prg_return_false:
      \fi:
 \fi:
}
2556 \prg_new_conditional:Npnn \token_if_eq_meaning:NN #1#2 { p , T , F , TF }
2557 {
2558 \if_meaning:w #1 #2
2559 \prg_return_true: \else: \prg_return_false: \fi:
2560 }

2201 \prg_new_conditional:Npnn \mode_if_math: { p , T , F , TF }
2202 { \if_mode_math: \prg_return_true: \else: \prg_return_false: \fi: }

\prg_new_conditional:Npnn \int_if_even:n #1 { p , T , F , TF}
3321 {
3322 \if_int_odd:w \__int_eval:w #1 \__int_eval_end:
3323 \prg_return_false:
3324 \else:
3325 \prg_return_true:
3326 \fi:
3327 }
\end{teXXX}

\begin{texexample}{isEven}{}
\ExplSyntaxOn
\prg_new_conditional:Npnn \isEven:n #1 { p, T, F, TF}
{
 \if_int_odd:w \__int_eval:w #1 \__int_eval_end:
    \prg_return_false:
 \else:
    \prg_return_true:
 \fi:
}

\isEven:nTF {2045679}{\PASS}{\FAIL}
\isEven:nTF {1000000}{\PASS}{\FAIL}
\ExplSyntaxOff
\end{texexample}

A common need for programmers is the testing of an integer or real for positiveness  with expl3 we can use predicate functions. In Example~\ref{ex:positive} we define predicate functions \docAuxCommand*{isPositive:nTF} to test an integer expression and feed the results to a true or false branch or according to the function signature. 

\begin{texexample}{isPositive} {ex:positive}
\ExplSyntaxOn
\prg_new_conditional:Npnn \isPositive:n #1 { p, T, F, TF}
{
\if_int_compare:w  \__int_eval:w #1 \__int_eval_end: >\__int_eval:w 0 \__int_eval_end:
     \prg_return_true:
\else:
    \prg_return_false:
\fi:          
}

\prg_new_conditional:Npnn \isNegative:n #1 { p, T, F, TF}
{
\if_int_compare:w  \__int_eval:w #1 \__int_eval_end: >\__int_eval:w 0 \__int_eval_end:
     \prg_return_false:
\else:
    \prg_return_true:
\fi:          
}
\cs_new:Npn \assert_is_positive:n #1 
   {
     \isPositive:nTF {#1} {\PASS #1} {\FAIL #1}
   }  
\cs_new:Npn \assert_is_negative:n #1 
   {
     \isNegative:nTF {#1} {\PASS #1} {\FAIL #1}
   } 
\assert_is_positive:n {2059+23-1245}
\assert_is_positive:n {-2059+23-1245}
\assert_is_negative:n {2059+23-1245}
\assert_is_negative:n {-2059+23-1245}
\ExplSyntaxOff
\end{texexample}

In the next example  we will use a common \tex trick to determine if a number is an integer or not. When \tex tries to convert a number to roman it will not scan past a minus sign .

\begin{texexample}{isInteger} {ex:isinteger}
\ExplSyntaxOn
\prg_set_conditional:Npnn \isInteger:n #1 { p, T, F, TF}
{
   \tl_if_blank:oTF {#1}{\prg_return_false:}
    {
     \tl_if_blank:oTF {  \__int_to_roman:w -\__int_eval:w #1 \__int_eval_end: }
		   {
		     \prg_return_true:
		   }
		   {
		     % not a number, but can be a negative number
		     \prg_return_false:
	         }
   }   
}

\cs_new:Npn \assert_is_integer:n #1 
   {
     \isInteger:nTF {#1} {\PASS\ ~~ #1} {\FAIL\ ~~ #1}\par
   }  
\assert_is_integer:n { }   
\assert_is_integer:n { 12}
\assert_is_integer:n {2059+1}
\assert_is_integer:n {-2059}
\assert_is_integer:n {2059}
%\assert_is_integer:n {ABC-1245}
\ExplSyntaxOff
\end{texexample}

The tests will pass provided even if we pass a  |numexpr|, but the assertion will fails if the number is negative. 
What we should have done was to test first if the head of the string was a (-) and then send it for further processing. 

\begin{texexample}{Testing the head of a string for the minus sign}{ex:string}
\ExplSyntaxOn
\cs_set:Npn \test:#1#2;{
   \str_if_eq:nnTF {-}{#1}{\PASS\par }{\FAIL\par }
   \str_if_eq:nnTF {-}{#1#2}{\PASS\par }{\FAIL\par }
}
\test:-;
\test:-12356;
\test:1234;
\ExplSyntaxOff
\end{texexample}

This passes all the comparison correctly, so we will have to re-write our function to test for the minus sign, before we send it to the main function. The reason I wrote the two tests above, is that a minus sign cannot be considered a number. 




\chapter{LaTeX3 String Manipulation and other Goodies}

 \TeX{} associates each character with a category code: as such, there is no
 concept of a \enquote{string} as commonly understood in many other
 programming languages. However, there are places where we wish to manipulate
 token lists while in some sense \enquote{ignoring} category codes: this is
 done by treating token lists as strings in a \TeX{} sense.

 A \TeX{} string (and thus an \pkg{expl3} string) is a series of characters
 which have category code $12$ (\enquote{other}) with the exception of
 space characters which have category code $10$ (\enquote{space}). Thus
 at a technical level, a \TeX{} string is a token list with the appropriate
 category codes. In this documentation, these will simply be referred to as
 strings: note that they can be stored in token lists as normal.

 The functions documented here take literal token lists,
 convert to strings and then carry out manipulations. Thus they may
 informally be described as \enquote{ignoring} category code. Note that
 the functions \docAuxCommand*{cs_to_str:N}, \docAuxCommand*{tl_to_str:n}, \docAuxCommand*{tl_to_str:N} and
 \docAuxCommand*{token_to_str:N} (and variants) will generate strings from the appropriate
 input: these are documented in \pkg{l3basics}, \pkg{l3tl} and \pkg{l3token},
 respectively.

 \section{The first character from a string}

 \begin{docCommand}{str_head:n}{\docAuxCommand*{str_head:n} \marg{token list}}
   Converts the \meta{token list} into a string, as described for
   \docAuxCommand*{tl_to_str:n}. The \docAuxCommand*{str_head:n} function then leaves
   the first character of this string in the input stream.
   The \docAuxCommand*{str_tail:n} function leaves all characters except
   the first in the input stream. The first character may be
   a space. If the \meta{token list} argument is entirely empty,
   nothing is left in the input stream.
 \end{docCommand}

\begin{texexample}{Strings}{ex:strings}
\ExplSyntaxOn
\group_begin:
\DeclareDocumentCommand\asentence{ m }{
  \str_head:n {#1}\par}
  
\asentence{This is something}  

\str_head:n{\This~is~something}\par
\str_tail:n{\This~is~something}
\group_end:
\ExplSyntaxOff
\end{texexample}

Note that the (\textbackslash) has been captured successfully. Also note that the \emph{tail} is everything after the first token.



 \subsection{Tests on strings}

The package provides some very powerful commands that can be used in string comparisons. Internally the comparisons are carried out using |\pdfstrcmp|. This has some complications in LuaTeX. 

 \begin{docCommand}{str_if_eq_x:nnTF} {\Arg{tl1} \Arg{tl2} \Arg{true code} \Arg{false code}}
%     \docAuxCommand*{str_if_eq:nnTF} \Arg{tl_1} \Arg{tl_2} \Arg{true code} \Arg{false code}
%   \end{syntax}
   Compares the two \meta{token lists} on a character by character
   basis, and is \textit{true} if the two lists contain the same
   characters in the same order. Thus for example
   \begin{verbatim}
     \str_if_eq_p:no { abc } { \tl_to_str:n { abc } }
   \end{verbatim}
   is logically \texttt{true}.
\end{docCommand}


\begin{texexample}{String comparisons}{ex:test}
\ExplSyntaxOn
\cs_set:Npn \abc: {abc}
\str_if_eq_x:nnTF{abc}{abc}{\TRUE}{\FALSE}\\
\str_if_eq_x:nnTF{\abc:}{abc}{\TRUE}{\FALSE}
\ExplSyntaxOff
\end{texexample}

 \section{String manipulation}

 \begin{docCommand}{str_lower_case:n}{\marg{tokens}}
%      \str_lower_case:n, \str_lower_case:f, 
%      \str_upper_case:n, \str_upper_case:f
%   }
%   \begin{syntax}
%     \docAuxCommand*{str_lower_case:n} \Arg{tokens}
%     \docAuxCommand*{str_upper_case:n} \Arg{tokens}
%   \end{syntax}
   Converts the input \meta{tokens} to their string representation, as
   described for \docAuxCommand*{tl_to_str:n}, and then to the lower or upper
   case representation using a one-to-one mapping as described by the
   Unicode Consortium file |UnicodeData.txt|.
   
   These functions are intended for case changing programmatic data in
   places where upper/lower case distinctions are meaningful. One example
   would be automatically generating a function name from user input where
   some case changing is needed. In this situation the input is programmatic,
   not textual, case does have meaning and a language-independent one-to-one
   mapping is appropriate. For example
%   \begin{verbatim}
%     \docAuxCommand*_new_protected:Npn \myfunc:nn #1#2
%       {
%         \docAuxCommand*_set_protected:cpn
%           {
%             user
%             \str_upper_case:f { \tl_head:n {#1} }
%             \str_lower_case:f { \tl_tail:n {#1} }
%           }
%           { #2 }
%       }
%   \end{verbatim}
%   would be used to generate a function with an auto-generated name consisting
%   of the upper case equivalent of the supplied name followed by the lower
%   case equivalent of the rest of the input.
%   
%   These functions should \emph{not} be used for
%   \begin{itemize}
%     \item Caseless comparisons: use \docAuxCommand*{str_fold_case:n} for this
%       situation (case folding is district from lower casing).
%     \item Case changing text for typesetting: see the \docAuxCommand*{tl_lower_case:n(n)},
%       \docAuxCommand*{tl_upper_case:n(n)} and \docAuxCommand*{tl_mixed_case:n(n)} functions which
%       correctly deal with context-dependence and other factors appropriate
%       to text case changing.
%   \end{itemize}
%
%   \begin{texnote}
%     As with all \pkg{expl3} functions, the input supported by
%     \docAuxCommand*{str_fold_case:n} is \emph{engine-native} characters which are or
%     interoperate with \textsc{utf-8}. As such, when used with \pdfTeX{}
%     \emph{only} the Latin alphabet characters A--Z will be case-folded
%     (\emph{i.e.}~the \textsc{ascii} range which coincides with
%     \textsc{utf-8}). Full \textsc{utf-8} support is available with both
%     \XeTeX{} and \LuaTeX{}, subject only to the fact that \XeTeX{} in
%     particular has issues with characters of code above hexadecimal
%     $0\mathrm{xFFF}$ when interacting with \docAuxCommand*{tl_to_str:n}.
%   \end{texnote}
 \end{docCommand}
 
 A common programming task is to convert strings to either uppercase or lowercase equivalents.v
 \begin{texexample}{Converting strings to lower and uppercase}{ex:cases}%TOFIX 
 \ExplSyntaxOn
    \tl_tail:n {TEST} 
   
      \cs_new_protected:Npn \myfunc:nn #1#2
       {
         \cs_set_protected:cpn
           {
             user
             \str_upper_case:f { \tl_head:n {#1} }
             \str_lower_case:f { \tl_tail:n {#1} }
           }
           { #2 }
       }
\docAuxCommand*_new_protected:cpn {yiannis}{Lazarides}
 \ExplSyntaxOff
 \end{texexample}


\endinput
\chapter{LaTeX3 Boxes}

\epigraph{If you go far enough back, your genome connects you with bacteria, butterflies, and barracuda---the great chain of being linked together through DNA.}{---Spencer Wells}

The \pkg{l3box} package, provides numerous commands that deal with boxes. Before you delve in the code you should be familiar with \tex’s concepts of boxes such as \docAuxCommand*{hbox} and \docAuxCommand*{vbox}. The package provides a full repertoire of commands, as well as additional helper functions to reduce the number of commands necessary when storing content in boxes. There is also an additional package for handling the \latexe type |\fbox| and |\makebox| commands, still in experimental stage called \pkgname{xbox}. The latter also is attempting to provide some integration with the \pkgname{xcoffins} package which is an entirely new concept for box manipulation in \latex3. Most of the commands are just syntactic translations of the \latexe macros. 

Do they offer any advantage? I am not too sure if they do at this stage. When it comes to boxes, which is such a fundamental typographic concept users expect much more than these basic commands, however one needs to build up from more basic commands and these have to be re-defined to keep up with the spirit of \latex3.

\section{Storing content in boxes}

\tex’s concept of storing content in boxes is fundamental to any programming effort, where the dimensions of typeset material needs to be determined before further processing.

\subsection{Creating and initializing boxes}
\begin{macro}{\box_new:N, \box_new:c} { \meta{box}}
   Creates a new \meta{box} or raises an error if the name is
   already taken. The declaration is global. The \meta{box} will
   initially be void.
\end{macro}

\begin{texexample}{A new box}{ex:boxnew}
\ExplSyntaxOn
\ttfamily
\token_to_meaning:N \box_new:N\\
\token_to_meaning:N \box
\ExplSyntaxOff
\end{texexample}



Normally three operations are involved. Creating an empty box or using one of the available temporary one, setting the contents in a horizontal or vertical or a combination of both of them, measuring them if necessary 
and then using them.

\begin{texexample}{Store content in a box and use it later}{ex:storebox}
\ExplSyntaxOn
\box_new:N \my_box
\hbox_gset:Nn \my_box {\color{theblue} Some~test.}

\ttfamily
\token_to_meaning:N \hbox_set:Nn
\ExplSyntaxOff
\end{texexample}

Traditionally using \tex the command |\setbox=\hbox{..,}| is used. Note with \latex3 we can use either an |hbox| or |vbox| or variants. Now the above did not typset its contents; for this we need to use the macro |\box_use:N|.

\begin{texexample}{...continued store contents}{}
\ExplSyntaxOn
\box_use:N \my_box
\ExplSyntaxOff
\end{texexample}


\begin{texexample}{Storing content in boxes}{l3box}
\ExplSyntaxOn
\box_new:c {temp_textbox}
\hbox_set_to_wd:cnn { temp_textbox } { 5cm } 
  {
    \tex_hsize:D 5cm
    
    \colorbox{spot!10}{\vbox:n  
       { 
         {\lorem }
       }
       }
  }
\medskip

\leavevmode\vbox{ \box_use:c {temp_textbox  }}\vbox{\box_use:c {temp_textbox  }}\par

\leavevmode\hbox{ \box_use:c {temp_textbox  } \box_use:c {temp_textbox  }}
\medskip

The~box~height~is~\the\box_ht:c {temp_textbox}~and~the~box~width~is~\the\box_wd:c {temp_textbox}.\par
\ExplSyntaxOff  
\end{texexample}


As this is is the \tex primitive |copy| we can use the box as many times as we want.

\begin{texexample}{Measuring the contents}{ex: measure contents}

\ExplSyntaxOn

\box_use:c {temp_textbox}

The~box~height~is~\the\box_ht:c {temp_textbox}~and~the~box~width~is~\the\box_wd:c {temp_textbox}.\par

\box_use:c {temp_textbox}

\ExplSyntaxOff

\tikz\node[draw, fill=spot!20, text width=142.26378pt-10pt, inner sep=5pt, outer sep=0pt, baseline=X.base]{\lorem};
\end{texexample}

The naming schemes are a bit unintuitive but this is inherited from \tex itself. To restrict the |\vbox| you need to set the |\hsize|.  

 \begin{docCommand}{box_move_right:nn}{\docAuxCommand*{box_move_right:nn} \marg{dimexpr} \marg{box function}}
 This function operates in vertical mode, and inserts the
  material specified by the \meta{box function}
  such that its reference point is displaced horizontally by the given
   \meta{dimexpr} from the reference point for typesetting, to the right
   or left as appropriate. The \meta{box function} should be
   a box operation such as |\box_use:N \<box>| or a \enquote{raw}
   box specification such as |\vbox:n { xyz }|.
 \end{docCommand}

 \begin{docCommand}{box_move_up:nn}{\docAuxCommand*{box_move_up:nn} \marg{dimexpr} \marg{box function}}
   This function operates in horizontal mode, and inserts the
   material specified by the \meta{box function}
   such that its reference point is displaced vertical by the given
   \meta{dimexpr} from the reference point for typesetting, up
   or down as appropriate. The \meta{box function} should be
   a box operation such as |\box_use:N \<box>| or a \enquote{raw}
   box specification such as |\vbox:n { xyz }|.
 \end{docCommand}
 
\begin{texexample}{Moving Boxes up or down}{l3boxdown}
\ExplSyntaxOn
\vbox_set:cn{temp_textbox}{abcd}
A \box_move_down:nn{10pt}{\box_use:c { temp_textbox }} 
\ExplSyntaxOff
\end{texexample}

 \section{Measuring and setting box dimensions}

\begin{docCommand}{box_dp:N}{\docAuxCommand*{box_dp:N} \meta{box}}
   Calculates the depth (below the baseline) of the \meta{box}
   in a form suitable for use in a \meta{dimension expression}.
\end{docCommand}

\begin{docCommand}{box_ht:N}{\docAuxCommand*{box_ht:N} \meta{box}}
   Calculates the height (above the baseline) of the \meta{box}
   in a form suitable for use in a \meta{dimension expression}.
  This is the \TeX{} primitive \docAuxCommand*{ht}.
 \end{docCommand}

% \begin{function}{\box_wd:N, \box_wd:c}
%   \begin{syntax}
%     \docAuxCommand*{box_wd:N} \meta{box}
%   \end{syntax}
%   Calculates the width of the \meta{box} in a form
%   suitable for use in a \meta{dimension expression}.
%   \begin{texnote}
%     This is the \TeX{} primitive \tn{wd}.
%   \end{texnote}
% \end{function}

\section{Horizontal Boxes}
\label{l3:hboxes}

So far we have discussed the boxing, unboxing and measuring of box dimensions. In the examples we have used
the \latex3 form of |\hbox| and |\vbox|.  Now time to lose our  beloved |source2e| favoured command \docAuxCommand*{hb@xt@} and friends. 

 \begin{docCommand}{hbox:n}{\docAuxCommand*{hbox:n} \marg{contents}}
   Typesets the \meta{contents} into a horizontal box of line width and then includes this box in the current list for typesetting.
   This is the \TeX{} primitive \docAuxCommand*{hbox}.
 \end{docCommand}

\begin{texexample}{Natural width boxes}{l3:hbox}
\ExplSyntaxOn
\cs_new:Npn \put_image:n #1{
  \par\leavevmode
  \centering
  \hbox:n{\includegraphics[width=#1\textwidth]{latex3}}
  \par
}
\put_image:n {0.6}
\ExplSyntaxOff
\end{texexample}

\begin{texexample}{Natural width boxes}{l3:hbox}
\ExplSyntaxOn
\hbox:n{\includegraphics[width=0.8\textwidth]{latex3}}
\ExplSyntaxOff
\end{texexample}

\definecolor{nice}{HTML}{48CCD8}
{\fboxsep3pt \sffamily \Huge \colorbox{nice}{\color{white}TEXT}}

\begin{docCommand}{hbox_to_wd:nn}{\docAuxCommand*{hbox_to_wd:nn} \marg{dimexpr} \marg{contents}}
   Typesets the \meta{contents} into a horizontal box of width
   \meta{dimexpr} and then includes this box in the current list for
   typesetting.
\end{docCommand}

\begin{texexample}{Natural width boxes}{l3:hbox}
\ExplSyntaxOn
\DeclareDocumentCommand\PutImage{o m}{
  \IfNoValueTF{#1}
      {\putimage{#2}}
      {\putimage{#1}{#2}}
}

\noindent\hbox_to_wd:nn{0.3\textwidth}{\includegraphics[width=0.3\textwidth]{amato}}
\hbox_to_wd:nn{0.3\textwidth}{\includegraphics[width=0.3\textwidth]{amato}}
\ExplSyntaxOff

\noindent\hbox to 0.3\textwidth{\includegraphics[width=0.3\textwidth]{amato}}%
\hbox to 0.3\textwidth{\includegraphics[width=0.3\textwidth]{amato}}
\end{texexample}

Having set our goodbyes to |\hb@xt@| we also don’t feel very sorry for not having to type \% to eliminate wandering spaces. As we delve further into the intricacies of \latex3 we can also start appreciating its advantages.

% \begin{function}{\hbox_to_zero:n}
%   \begin{syntax} 
%     \docAuxCommand*{hbox_to_zero:n} \Arg{contents}
%   \end{syntax}
%   Typesets the \meta{contents} into a horizontal box of zero width
%   and then includes this box in the current list for typesetting.
% \end{function}

\section{Vertical Boxes}
The vertical box equivalents to \tex’s |\vbox|, |\vtop| are provided, as well as helper functions to store contents in a box typeset zero width boxes or lap them left, right or center. The commands are mostly syntactic sugar to the primitive commands. 

\begin{docCommand}{vbox:n}{\marg{contents}}
Typesets the \meta{contents} into a vertical box of natural height and includes this box in the current list for typesetting.
\end{docCommand}

\begin{docCommand}{vbox_to_ht:nn}{\marg{dimexpr}\marg{contents}}
Typesets the \meta{contents} into a vertical box of height \meta{dimexpr} and includes this box in the current list for typesetting.
\end{docCommand}

\begin{docCommand}{vbox_to_zero:n}{\marg{contents}}
Typesets the \meta{contents} into a vertical box of zero height and includes this box in the current list for typesetting.
\end{docCommand}

%\tcbset{listing options={
%              firstnumber=10, stepnumber=1, belowskip=0pt, 
%              escapeinside={(*@}{@*)},
%              backgroundcolor=\color{graphicbackground}
%              }}
\begin{texexample}{vboxes in LaTeX3}{l3:boxes}
\ExplSyntaxOn
    \fbox{\vbox:n{\lorem}}\par
    \fbox{\vbox_to_ht:nn {1.5cm}{\lorem}}\par
    \fbox{\vbox_to_zero:n {\lorem}}
\ExplSyntaxOff
\vspace*{1cm}
\end{texexample}

In Example~\ref{l3:boxes} we use \docAuxCommand*{vbox_to_ht:nn} and \docAuxCommand*{vbox_to_zero:n} to set text in two vertical boxes. The first one is typeset in a vertical box of 2cm height, whereas the second one in a box of zero height. The macro
|\fbox| which we discussed earlier in the \latexe boxes chapter, is also available in \latex3 but as part of the still under trial package \pkgname{xbox}.\footnote{To make matters more complicated, the version used in this document has been redefined further!} 



%\tcbset{listing options={
%              firstnumber=last, stepnumber=1, belowskip=0pt, 
%              escapeinside={(*@}{@*)},
%              backgroundcolor=\color{graphicbackground},
%              upquote=true,
%          }}
              
\begin{texexample}{vboxes in LaTeX3}{l3:boxes}
\ExplSyntaxOn
    \fbox{\vbox:n{\lorem}}\par
    \fbox{\vbox_to_ht:nn {1.5cm}{\lorem}}\par
    \fbox{\vbox_to_zero:n {\lorem}}
\ExplSyntaxOff
\vspace*{1cm}
\end{texexample}

\chapter{LaTeX3 xcoffins, special boxes for special typesetting}

\epigraph{The history of that name (as I remember it at least) goes way back to a stroll in some town in the UK sometime in the last century, probably 1997 (may have been Nottingham, but I don't remember) with David Carlisle and Chris Rowley and perhaps a few others on which we discussed those ideas about boxes with handles and somehow somebody came up with "rather like a coffin" and that is how it got born. And no, I don't remember whether it was David, Chris or myself.

Somehow the name stuck; initially as a working title when we first implemented a prototype, but later I must confess I rather liked it -- a bit morbit for sure, but also catchy :-) ... and it made for few a great lines in my talk in San Francisco, such as: Now in 2010 coffins are back – exhumed, cleaned up – and ready for display
what else can you hope for?}{---Frank Mittelbach}


%\tcbset{listing options={
%              firstnumber=10, stepnumber=1, belowskip=0pt, 
%              escapeinside={(*@}{@*)},
%              backgroundcolor=\color{graphicbackground},
%              upquote=true,
%          }}
          
 In \LaTeX3 terminology, a \enquote{coffin} is a box containing
 typeset material.\footnote{The term `coffin’ was probably coined by Frank Mittelbach (see \protect\url{http://tex.stackexchange.com/questions/147738/origin-of-the-latex3-term-coffin})} Along with the box itself, the coffin structure
 includes information on the size and shape of the box, which makes
 it possible to align two or more coffins easily. This is achieved
 by providing a series of `poles' for each coffin. These
 are horizontal and vertical lines through the coffin at defined
 positions, for example the top or horizontal centre. The points
 where these poles intersect are called \enquote{handles}. Two
 coffins can then be aligned by describing the relationship between
 a handle on one coffin with a handle on the second. In words, an
 example might then read
 \begin{quote}
   Align the top-left handle of coffin A with the bottom-right
   handle of coffin B.
 \end{quote}

 The locations of coffin handles are much easier to understand
 visually. Figure~\ref{fgr:handles} shows the standard handle
 positions for a coffin typeset in horizontal mode (left) and in
 vertical mode (right). Notice that the later case results in a greater
 number of handles being available. As illustrated, each handle
 results from the intersection of two poles. For example, the centre
 of the coffin is marked |(hc,vc)|, \emph{i.e.}~it is the
 point of intersection of the horizontal centre pole with the
 vertical centre pole. New handles are generated automatically when
 poles are added to a coffin: handles are \enquote{dynamic} entities.
 
 \NewCoffin \ExampleCoffin
\begin{figure}[htbp]
   \hfil
    \fboxsep2pc
     \colorbox{black}{\color{white}\begin{minipage}{0.4\textwidth}
     \SetHorizontalCoffin\ExampleCoffin
       {\color{white}\rule{1 in}{1 in}}
  \DisplayCoffinHandles\ExampleCoffin{yellow}
   \end{minipage}}
   \hfil
   \begin{minipage}{0.4\textwidth}
     \SetVerticalCoffin\ExampleCoffin{1 in}
       {\color{black!10!white}\rule{1 in}{1 in}}
     \DisplayCoffinHandles\ExampleCoffin{red}
   \end{minipage}
   \hfil
   \caption{Standard coffin handles: left, horizontal coffin; right,
     vertical coffin}
   \label{fgr:handles}
 \end{figure}


All coffin operations are local to the current \tex group with the exception
of coffin creation. Coffins are also “color safe”: in contrast to the code-level \docAuxCommand*{box_}\ldots.
functions there is no need to add additional grouping to coffins when dealing with color.

The user interface for the command is somewhat complicated. This is an area where the package
can be enhanced in the future and the sole reason is being kept under the \emph{experimental}
branch of \latex3.

\section{Getting Started}

Before a \meta{coffin} can be used, it must be allocated using \docAuxCommand*{NewCoffin}.

\begin{docCommand}{NewCoffin}{\meta{coffin}}
Before a \meta{coffin} can be used, it must be allocated using \docAuxCommand*{NewCoffin}. The name of the
hcoffini should be a control sequence (starting with the escape character, usually \textbackslash ), for
example

\begin{verbatim}
\NewCoffin\MyCoffin
\end{verbatim}

Coffins are allocated globally, and an error will be raised if the name of the \meta{coffin} is
not globally-unique.
\end{docCommand}

\begin{texexample}{Coffins}{ex:coffins}
  \NewCoffin \AnExampleCoffin
  \NewCoffin\Rulei
\end{texexample}

 \begin{docCommand}{SetHorizontalCoffin}{\docAuxCommand*{SetHorizontalCoffin} \meta{coffin} \marg{material}}
   Typesets the \meta{material} in horizontal mode, storing the result
   in the \meta{coffin}. The standard poles for the \meta{coffin} are
   then set up based on the size of the typeset material.
 \end{docCommand}

 \begin{docCommand}{SetVerticalCoffin}{\docAuxCommand*{SetVerticalCoffin} \meta{coffin} \marg{width} \marg{material}}
   Typesets the \meta{material} in vertical mode constrained to the
   given \meta{width} and stores the result in the \meta{coffin}. The
   standard poles for the \meta{coffin} are then set up based on the
   size of the typeset material.
 \end{docCommand}

In Example~\ref{ex:coffins2} we will create a horizontal coffin and then typeset it. 
 
%\tcbset{listing options={
%              firstnumber=last, stepnumber=1, belowskip=0pt, 
%              escapeinside={(*@}{@*)},
%              backgroundcolor=\color{graphicbackground},
%              upquote=true,
%          }}
          
\begin{texexample}{Creating coffins}{ex:coffins2}
\SetHorizontalCoffin\ExampleCoffin
   {\color{red}\rule{4cm}{1pc}}  
\SetHorizontalCoffin\Rulei
   {\color{blue}\rule{6cm}{1pc}}     
   
First coffin\hspace{0.9cm}\DisplayCoffinHandles\ExampleCoffin{black}\hspace{0.9cm}!
  
Second  coffin\hfill \DisplayCoffinHandles\Rulei{blue}

\meaning\Rulei
\end{texexample}
  
\paragraph{How to set the width } The rule was created using \latexe |\rule|  macro and then it was saved in a coffin box named |\ExampleCoffin|. The typesetting was done using |\DisplayCoffinHandles| 

In the next example, we will create a second rule and then demonstrate the joining operation. We will need two more coffins, one to hold the results and the other to hold the material for the second box.

\begin{texexample}{Joining Coffins}{ex:coffins3}
\NewCoffin\ExampleCoffinTwo
\NewCoffin\Result
\SetHorizontalCoffin\ExampleCoffin
   {\color{red}\rule{3cm}{1pc}} 
\SetHorizontalCoffin\ExampleCoffinTwo
   {\color{green}\rule{3cm}{1pc}}    
\JoinCoffins\Result\ExampleCoffin   
\JoinCoffins \Result[\ExampleCoffin-t,\ExampleCoffin-r] \ExampleCoffinTwo [b,l](0pt,2mm)
\TypesetCoffin\Result
\end{texexample}
 
The interesting, but complicated command is |\JoinCoffins|. This takes two arguments, the coffins to be joined, which in turn have optional commands, specifying how the coffins are joined at their poles. 
This is the key operation for coffins,  joining coffins to each other. This
 is always carried out such that the first coffin is the
 \enquote{parent}, and is updated by the alignment. The second
 \enquote{child} coffin is not altered by the alignment process.

 \begin{docCommand}{JoinCoffins}{ \docAuxCommand*{JoinCoffins} *
     ~~\meta{coffin1} [ \meta{coffin1-pole1} , \meta{coffin1-pole2} ]
     ~~\meta{coffin2} [ \meta{coffin2-pole1} , \meta{coffin2-pole2} ]
     ~~( \meta{x-offset} , \meta{y-offset} )}
   Joining of two coffins is carried out by the \docAuxCommand*{JoinCoffins}
   function, which takes two mandatory arguments: the \enquote{parent}
   \meta{coffin1} and the \enquote{child} \meta{coffin2}. All of the
   other arguments shown are optional.
 \end{docCommand}

   The standard \docAuxCommand*{JoinCoffins} functions joins \meta{coffin2} to
   \meta{coffin1} such that the bounding box of \meta{coffin1} after the
   process will expand. The new bounding box will be the smallest
   rectangle covering the bounding boxes of the two input coffins.
   When the starred variant of \docAuxCommand*{JoinCoffins} is used, the bounding
   box of \meta{coffin1} is not altered, \emph{i.e.}~\meta{coffin2} may
   protrude outside of the bounding box of the updated \meta{coffin1}.
   The difference between the two forms of alignment is best illustrated
   using a visual example. In Figure~\ref{fgr:alignment}, the two
   processes are contrasted. In both cases, the small red coffin has been
   aligned with the large grey coffin. In the left-hand illustration,
   the \docAuxCommand*{JoinCoffins} function was used, resulting in an expanded
   bounding box. In contrast, on the right \docAuxCommand*{AttachCoffin} was used,
   meaning that the bounding box does not include the area of the
   smaller coffin.
   
\begin{texexample}{Joining Coffins}{ex:coffins4}
\SetHorizontalCoffin\ExampleCoffin
   {\color{red}\rule{3cm}{1pc}} 
\SetHorizontalCoffin\ExampleCoffinTwo
   {\color{green}\rule{3cm}{1pc}}    
\JoinCoffins\Result\ExampleCoffin   
\JoinCoffins*\Result[\ExampleCoffin-l,\ExampleCoffin-b] \ExampleCoffinTwo [t,l](0pt,2mm)
\TypesetCoffin\Result
\end{texexample}   
   
\section{Controlling coffin poles}

 A number of standard poles are automatically generated when the coffin
 is set or an alignment takes place. The standard poles for all coffins
 are:
 \begin{marglist}
   \item[l] a pole running along the left-hand edge of the bounding
     box of the coffin;
   \item[hc] a pole running vertically through the centre of the coffin
     half-way between the left- and right-hand edges of the bounding
       box (\emph{i.e.}~the \enquote{horizontal centre});
   \item[r] a pole running along the right-hand edge of the bounding
     box of the coffin;
   \item[b] a pole running along the bottom edge of the bounding
     box of the coffin;
   \item[vc] a pole running horizontally through the centre of the
     coffin half-way between the bottom and top edges of the bounding
     box (\emph{i.e.}~the \enquote{vertical centre});
   \item[t] a pole running along the top edge of the bounding
     box of the coffin;
   \item[H] a pole running along the baseline of the typeset material
     contained in the coffin.
 \end{marglist}
 In addition, coffins containing vertical-mode material also
 feature poles which reflect the richer nature of these systems:
 \begin{itemize}
   \item[B] a pole running along the baseline of the material at the
     bottom of the coffin.
   \item[T] a pole running along the baseline of the material at the top
     of the coffin.
 \end{itemize}  
 
\section{A larger example}

Consider the book cover of Judy Estrin’s book, \emph{Closing the Innovation Gap} shown in Example~\ref{ex:covers}. The title elements have been carefully placed by the book designer. This sort
of cover page is within the possibilities of what can be programmed via \latex~3 and the package \pkgname{xcoffins}.

\begin{texexample}{Typesetting Cover Pages}{ex:covers}  
\bgroup
\parindent0pt
% For each element declare a new  coffin
\NewCoffin\ci
\NewCoffin\cii
\NewCoffin\ciii
\NewCoffin\civ

% Always better to give semantic names!
\NewCoffin\slogan
\NewCoffin\ImageCoffin
\NewCoffin\AuthorCoffin

% A convenient commant to set font a
\DeclareDocumentCommand\fonta{}
  {
      \color{white}\LARGE\bfseries\sffamily
  }

% Similar command for font b    
\DeclareDocumentCommand\fontb{}
  {
      \color{white}\large\bfseries\sffamily
  }  
\SetHorizontalCoffin\Result{}
\SetHorizontalCoffin\ci{\fonta\space CLOSING} 
\SetHorizontalCoffin\cii{\fontb THE}
\SetHorizontalCoffin\ciii{\fonta INNOVATION}
\SetHorizontalCoffin\civ{\fonta GAP}

\SetVerticalCoffin\slogan{\CoffinWidth\ciii+30pt}{\vspace*{25pt}\centering
\small\sffamily REIGNITING THE SPARK OF
THE GLOBAL ECONOMY\par}

% set the image coffin
\SetHorizontalCoffin\ImageCoffin{\space\space
  \includegraphics[width=100pt]{./images/innovation-book-cover.jpg}}
  
% set the author  
\SetHorizontalCoffin\AuthorCoffin{\fontb\centering JUDY ESTRIN\par}

% Now join all the coffins check the manual for the handles!    
\JoinCoffins\Result\ci
\JoinCoffins\Result[hc,b]    \cii[hc,t](0pt,-2mm)%the
\JoinCoffins\Result[l,b]       \ciii[l,t](15pt,-2mm)%innovation
\JoinCoffins\Result[\ciii-hc,\ciii-b] \civ[l,t](0pt,-2mm)
\JoinCoffins\Result[l,b]      \slogan[l,t](0pt,-2mm)
\JoinCoffins\Result[hc,b]   \AuthorCoffin[hc,t](0pt,-4mm)
\JoinCoffins\Result[r,b]      \ImageCoffin[l,b](0pt, 0pt)
   \fboxsep1pc
  \colorbox{black}{\color{white}\TypesetCoffin\Result}

% close the group we opened     
\egroup
\end{texexample}

Of course my general advice to anyone programming \latex is to always get professional advice on designing a book cover. Mathematicians, programmers and scientists are not the best of people to design book covers. They can come up with the code, but hardly succeed with the graphics aspects. There are also other methods to design and typeset book covers. An excellent package using \tikzname is \pkgname{bookcover} by Tibor Tómács. 


One tends to forget if the syntax requires to type \textit{t}, \textit{l} or \textit{l}, \textit{t} and this is a common issue with this type of commands. As we said before \latex stresses one’s memory to the limit. It can also be a bit confusing, as to when one needs to use a vertical rather than horizontal coffin.
    
If you stydy the code in Example~\ref{ex:covers} you will notice that the last box, has a width that was set using
\docAuxCommand*{CoffinWidth}. The package provides commands that provide the value of the coffin dimensions. These are described in the next section that together with some other auxiliary helper functions concludes our discussion of the package.

 \section{Measuring coffins}

 There are places in the design process where it is useful to be able to
 measure coffins outside of pole-setting procedures.

 \begin{docCommand}{CoffinDepth}{ \docAuxCommand*{CoffinDepth} \meta{coffin}}
   Calculates the depth (below the baseline) of the \meta{coffin}
   in a form suitable for use in a \meta{dimension expression}, for example
   |\setlength{\mylength}{\CoffinDepth\ExampleCoffin}|.
 \end{docCommand}

 \begin{docCommand}{CoffinHeight}{\docAuxCommand*{CoffinHeight} \meta{coffin}}
   Calculates the height (above the baseline) of the \meta{coffin}
   in a form suitable for use in a \meta{dimension expression}, for example
   |\setlength{\mylength}{\CoffinHeight\ExampleCoffin}|.
 \end{docCommand}

 \begin{docCommand}{CoffinTotalHeight}{\docAuxCommand*{CoffinTotalHeight} \meta{coffin}}
   Calculates the total height of the \meta{coffin}
   in a form suitable for use in a \meta{dimension expression}, for example
   |\setlength{\mylength}{\CoffinTotalHeight\ExampleCoffin}|.
 \end{docCommand}

 \begin{docCommand}{CoffinWidth}{\docAuxCommand*{CoffinWidth} \meta{coffin}}
   Calculates the width of the \meta{coffin} in a form
   suitable for use in a \meta{dimension expression}, for example
   |\setlength{\mylength}{\CoffinWidth\ExampleCoffin}|.
 \end{docCommand} 
    
\section{Debugging}

Debugging code that includes |coffin| functions is made easier when you can view information on the
poles. The pakage provides commands for both printing the information as well as viewing it on the screen.

\begin{docCommand}{DisplayCoffinHandles}{\meta{coffin}meta{color}}
This function first calculates the intersections between all of the hpolesi of the \meta{coffin} to
give a set of \meta{handles}. It then prints the \meta{coffin} at the current location in the source,
with the position of the \meta{handles} marked on the coffin. The \meta{handles} will be labelled
as part of this process: the locations of the \meta{handles} and the labels are both printed in
the \meta{color} specified. This is similar to the |\TypesetCoffin| function, except the former will also print
the handles. 
\end{docCommand}
  
\begin{docCommand}{MarkCoffinHandle}{\meta{coffin}\oarg[\meta{pole1}, \meta{pole2}] \marg{color}}  
This function first calculates the \meta{handle} for the \meta{coffin} as defined by the intersection
of \meta{pole1} and \meta{pole2}. It then marks the position of the \meta{handle} on the \meta{coffin}. The
\meta{handle} will be labelled as part of this process: the location of the \meta{handle} and the
label are both printed in the \meta{color} specified. If no \meta{poles} are give, the default (H,l) is
used.
\end{docCommand}
  
   \begin{figure}
     \hfil
     \SetHorizontalCoffin\ExampleCoffin
       {%
         \color{black!10!white}\rule{0.5 in}{1 in}^^A
         \color{black!20!white}\rule{0.5 in}{1 in}^^A
       }
     \begin{minipage}{0.4\textwidth}
       \DisplayCoffinHandles\ExampleCoffin{blue}
     \end{minipage}
     \hfil
     \begin{minipage}{0.4\textwidth}
       \RotateCoffin\ExampleCoffin{45}
       \DisplayCoffinHandles\ExampleCoffin{red!50!black}
     \end{minipage}
     \hfil
     \caption{Coffin rotation: left, unrotated; right, rotated by
       $45$\textdegree.}
     \label{fgr:rotation}
   \end{figure}
   
%\newpage 
%\newgeometry{margin=5pt,}  
%\null
%
%\newcommand\cbox[2][.8]{{\setlength\fboxsep{0pt}\colorbox[gray]{#1}{#2}}}
%
%
%  \NewCoffin \result
%  \NewCoffin \aaa
%  \NewCoffin \bbb
%  \NewCoffin \ccc
%  \NewCoffin \ddd
%  \NewCoffin \eee
%  \NewCoffin \fff
%  \NewCoffin \rulei
%  \NewCoffin \ruleii
%  \NewCoffin \ruleiii
%
%\SetHorizontalCoffin \result {}
%\SetHorizontalCoffin \aaa {\fontsize{52}{50}\sffamily\bfseries mep progress}
%\SetHorizontalCoffin \bbb {\fontsize{52}{50}\sffamily\bfseries habtoor city}%typographische}habtoor city
%\SetHorizontalCoffin \ccc {\fontsize{12}{10}\sffamily 
%                      \quad zeitschrift des bildungsverbandes der
%                      deutschen buchdrucker leipzig 
%                     \textbullet{} oktoberheft 1925}
%\SetHorizontalCoffin \ddd {\fontsize{28}{20}\sffamily report}%sonderheft}
%\SetVerticalCoffin \eee {180pt}
%                 {\raggedleft\fontsize{31}{36}\sffamily\bfseries 
%                      elementare\\
%                      typographie}
%\SetVerticalCoffin \fff {140pt}
%                 {\raggedright \fontsize{13}{14}\sffamily\bfseries 
%                       yannis lazarides \\
%                       nasser khalf \\
%                       kyriacos savva \\
%                       max burchartz \\
%                       el lissitzky \\
%                       ladislaus moholy-nagy \\
%                       moln\'ar f.~farkas \\
%                       johannes molzahn \\
%                       kurt schwitters \\
%                       mart stam \\
%                       ivan tschichold}
%
%\RotateCoffin \bbb {90}
%\RotateCoffin \ccc {270}
%
%\SetHorizontalCoffin \rulei  {\color{red}\rule{6.5in}{1pc}}
%\SetHorizontalCoffin \ruleii {\color{red}\rule{1pc}{20.5cm}}
%\SetHorizontalCoffin \ruleiii{\color{black}\rule{10pt}{152pt}}
%
%
%\JoinCoffins \result                \aaa 
%\JoinCoffins \result[\aaa-t,\aaa-r] \rulei   [b,r](0pt,2mm)
%\JoinCoffins \result[\aaa-b,\aaa-l] \bbb     [B,r](2pt,0pt)
%\JoinCoffins \result[\bbb-t,\bbb-r] \ruleii  [t,r](-2mm,0pt)
%\JoinCoffins \result[\aaa-B,\aaa-r] \ccc     [B,l](66pt,14pc)
%\JoinCoffins \result[\bbb-l,\ccc-B] \fff     [t,r](-2mm,0pt)
%\JoinCoffins \result[\fff-b,\fff-r] \ruleiii [b,l](2mm,0pt)
%\JoinCoffins \result[\ccc-r,\fff-l] \eee     [B,r]
%\JoinCoffins \result[\eee-T,\eee-r] \ddd     [B,r](0pt,4pc)
%
%
%
%\TypesetCoffin \result
%
%\restoregeometry

\chapter{Expansion and LaTeX3}

Expansion and variants are central to the concept of \latex3. The module |l3expan| provides generic methods for expanding \tex arguments in a systematic manner. The functions in the module all have prefix the |exp|.

The module provides functions to produce \enquote{variants}. This is one of the most fundamental concepts of \latex3 and is good before we proceed further to recap on some of the \latex3 concepts.

\begin{description}
\item [naming conventions] The naming convention for command in \latex3 (expl3)  structures for command names is:

\textbackslash \meta{module}\textunderscore \meta{description}: \meta{arg-specifiers}

\textit{module} identifies the (main) type of data the function manipulates or use (for example, int (integers), prop (property lists), etc., or it might be the name of a package or some specific concept. 

\textit{description} says what is being done, e.g., |put_left|, |get|, |clear|, |count|, etc. If it makes sense the same descriptions are reused, but for special tasks there can, of course, be some that are used only once.

\textit{arg-specifiers} finally describe what arguments the function has and how they should be treated (more on this below).


\item[Base functions] \lorem 

\item[Variant functions]  Any command that uses one or more of these \emph{arg-specifiers} is called a \emph{variant} of the corresponding \emph{base function}. What these functions do is that they modify the argument one way or another and \textbf{only} then pass it to the underlying base function. For example:

\begin{teXXX}
\foo_bar:cVno {cmd} \VAR {text} \CMD
\end{teXXX}

would

\begin{itemize}
\item generate from the string |cmd| the command name \cs{cmd}
\item look up the value of the variable \cs{VAR}
\item leave text alone
\item expand \cs{CMD} once and surround the result with braces
\end{itemize}

It is important to stress that the variants do not produce aliases for the functions, they are also not overloading them. They just expand the base function arguments in a different way. 

\begin{teXXX}
 \foo_bar:Nnnn \cmd {<value-of-\VAR>} {text} {<one-level-expansion-of-\CMD>}
\end{teXXX}

Now ideally we want any possible variant of a base function automatically available for a programmer. Unfortunately, this can only be reliably done if all variants have all been predefined (as TeX doesn't offer you to trap the \enquote{undefined csname} error and do something on the fly).

Given the number of arg-specifier and the possible permutations predefining all variants, of which 90\% would never be needed, is not realistic. As Fank Mittelbach wrote on TEX.SX Q\&A site the \latex3 Team adopted the following strategy:

\begin{enumerate}
\item conceptually all variants are available and everybody can assume this is the case

\item in reality the kernel only defines a small subset that is often needed

\item any variant not defined by the kernel needs to be defined by the programmer using \docAuxCommand*{cs_generate_variant:Nn}

\item \docAuxCommand*{cs_generate_variant:Nn} has been designed in such a way that it doesn't matter if it is called several times: if the variant already exists it will do nothing. So if two programmers define the same variant in their packages it doesn't hurt, the first one executed will define the variant the second one will simply be ignored (with very little overhead).
\end{enumerate}

If some variants are used fairly often they may eventually get defined already in the kernel. Because of the last point it doesn't hurt if some packages still define the variant, i.e., there is no need for programmers to modify their packages in that case.

So in summary: Whenever you need a variant that is not predefined, define it at the beginning of your code. This is even sensible if you need the variant only once, because the code using the variant will be much more readable than any manual preprocessing of the argument and the speed difference is close to zero.

\item[The exp\_args:N.. functions]

Technicically speaking a variant defined via |\cs_generate_variant:Nn| has a very simple definition: |\foo_bar:cVno| above would simply expand to
\end{description}

\section{How to define variants}

The workhorse function used to define variants is:

\begin{docCommand}{cs_generate_variant:Nn} { \meta{parent control sequence} \marg{variant arg-spec}}
This function is used to define arguments-specific variants of the \emph{parent control sequence} for \latex3 
code level macros. The \emph{parent control sequence} is first separated into the \emph{base name} and \emph{original specifier}. 
\end{docCommand}


\begin{texexample}{Generating Variants}{ex:variants}

\ExplSyntaxOn
\group_begin:
  \cs_set:Npn \test: {Some code here}
  \cs_set:Npn \foo:Nn #1 #2 {\csname #1\endcsname \par #2\par}
  
  % creates \foo:cn
  \cs_generate_variant:Nn \foo:Nn {c}
  
  %test 1
  \foo:Nn {\test:}{12}
  
  %test 2
  \foo:cn {test:}{15}
  
\group_end:  
\ExplSyntaxOff
\end{texexample}





 
\input{l3macros}
\parindent1em
\chapter{LaTeX3 counters and registers}

 \section{Introduction}
 
 This \latex3 module is dealing with integer arithmetic as well as utilizing these to create counter data type structures. It can be used to develop counters in a similar fashion to \latexe, although the user level functions are missing.  All control squences in this module are prefixed with |\int|. Some of the examples are prefixed as |phd_counter| and they emulate \latex’s counter commands. In this chapter also we describe a rather long example to typeset a table of numbers in various notations, such as roman, literal, arabic etc.
 
 \section{Integer expressions}

 Calculation and comparison of integer values can be carried out
 using literal numbers, \texttt{int} registers, constants and
 integers stored in token list variables. The standard operators
 \texttt{+}, \texttt{-}, \texttt{/} and \texttt{*} and
 parentheses can be used within such expressions to carry
 arithmetic operations. This module carries out these functions
 on \emph{integer expressions} (\enquote{\texttt{intexpr}}).

 \begin{docCommand}{int_eval:n} {\marg{integer expression}}
    Evaluates the \meta{integer expression}, expanding any
   integer and token list variables within the \meta{expression}
   to their content (without requiring \docAuxCommand*{int_use:N}/\docAuxCommand*{tl_use:N})
   and applying the standard mathematical rules. For example both
 \end{docCommand}
   
   \begin{verbatim}
     \int_eval:n { 5 +  4 * 3 - ( 3 + 4 * 5 ) }
   \end{verbatim}
   and
   \begin{verbatim}
     \tl_new:N  \l_my_tl
     \tl_set:Nn \l_my_tl { 5 }
     \int_new:N  \l_my_int
     \int_set:Nn \l_my_int { 4 }
    \int_eval:n { \l_my_tl +  \l_my_int * 3 - ( 3 + 4 * 5 ) }
   \end{verbatim}
   both evaluate to \( -6 \). The  \marg{integer expression} may
   contain the operators \texttt{+}, \texttt{-}, \texttt{*} and
   \texttt{/}, along with parenthesis \texttt{(} and \texttt{)}.
   Any functions within the expressions should expand to an
   \meta{integer denotation}: a sequence of a sign and digits matching
   the regex |\-?[0-9]+|).
   After expansion \docAuxCommand*{int_eval:n} yields an  \meta{integer denotation}
   which is left in the input stream.
 
  
     Exactly two expansions are needed to evaluate \docAuxCommand*{int eval:n}.
     The result is \emph{not} an \meta{internal integer}, and therefore
     requires suitable termination if used in a \TeX{}-style integer
     assignment.
   
 
  
  If you familiar with e-tex’s |numexpr|, this is equivalent code. 
 
  \begin{texexample}{Integer Evaluation}{ex:numexpr}
  \ExplSyntaxOn
   \int_eval:n { 5 +  4 * 3 - ( 3 + 4 * 5 )+2*2 }\par
   \int_to_arabic:n { ( 2+7 ) / 3 }\par
   \int_to_alph:n { 2+7 } \par
   \int_to_Alph:n { 6 * 2 }\par
   \int_to_roman:n { 9 } \par
   \int_to_Roman:n { 21 } \par
  \ExplSyntaxOff 
  \end{texexample} 
  

 
 \section{Creating and initializing integers} 
  
These are \latex’s equivalents of counters. Numerous commands are provided by the \latex3 kernel and these in my opinion offer a much better interface to lower level commands. A common question is that what one does if it is required to access a \latexe counter. The more-or-less “official” answer was provided by Joseph Wright at the Stack Exchange\footnote{\protect\url{http://tex.stackexchange.com/questions/167094/manipulate-a-latex2e-counter-with-latex3}} Q\&A site with the recommendation that: `mixing up the two interaces is asking for trouble, and while we are working on several areas we’ve not got a “user level” counter approach at yet. (Indeed, the entire question of how variables at the document-level should be handled is open.)’ 
  
  So you have it you keep on using commands such as \docAuxCommand*{setcounter} here is that |\c@..|. is an internal for LaTeX2e, and the entire point of expl3 is to have clear interfaces and internals. There's no reason to abuse the interfaces here (no functionality gain), so stick with them. In my opinion also it is not a good idea to mix |\@| notation with |expl3| notation. When there is a need to use both it is best to clearly separate the two and create an interface if they must somehow share information.

 

  Before we go further with the code is instructive to peek at the \latex3 kernel and understand what is an integer. An integer is simply a \tex \docAuxCommand*{newcount}, as shown from the code below.
  
  \begin{teXXX}
  \cs_new_protected:Npn \int_new:N #1
  {
    \__chk_if_free_cs:N #1(*@\label{allocation}@*)
    \cs:w newcount \cs_end: #1
  }
 \end{teXXX} 
  

   
 \begin{texexample}{allocations}{ex:counter allocations} 
 \ExplSyntaxOn
 \newcount\somecounter
 \meaning\somecounter\par
 
 \int_new:N\someothercounter
 \meaning\someothercounter\par
 
 \tl_map_inline:nn {
    \somecounter
    \someothercounter
  }
  { \cs_undefine:N #1 }
  
  \meaning\somecounter
 \ExplSyntaxOff
 \end{texexample}
 
 As you can observe from the example, using |\int_new:N| to create a counter is identical to the \latexe |\newcount|. It is instructive to keep this in mind later on and in your code, during debugging. Line~\ref{allocation} checks if the |\count| is available and then allocates the counter to the command sequence.
 Once we typeset the example, I have used |\cs_undefined| in a sequence to free the register. This is always good practice.  You must be careful not to get confused here with the terminology, we are dealing with \tex’s primitive |\count| registers.\footnote{To be more specific e-\tex.} Although the original \tex came only with 256 registers the new engines allow up to 65535 count registers. 
  
 \begin{docCommand}{int_new:N}{\meta{integer}}
   Creates a new \meta{integer} or raises an error if the name is
   already taken. The declaration is global. The \meta{integer} will
   initially be equal to $0$.
 \end{docCommand}
 
 In most instances counters involve a three step operation:
 
 \begin{enumerate}
 \item Creating the counter
 \item Adding values
 \item Typesetting the value or using it in another expression
 \end{enumerate}
 

 
 \begin{texexample}{Counters}{ex:l3counters}
 \ExplSyntaxOn
 \int_new:N \exercise
 \int_add:Nn \exercise {12+15}
 \int_to_roman:n \exercise \\
 
 \int_use:N \c_max_register_int
 \ExplSyntaxOff
 \end{texexample}

The \docAuxCommand*{int_use:N} is the \tex primitive \docAuxCommand*{the}. This is one of several \latex3 names of the primitive.


 \begin{docCommand}{int_const:Nn}{\meta{integer} \marg{integer expression}}
   Creates a new constant \meta{integer} or raises an error if the name
   is already taken. The value of the \meta{integer} will be set
   globally to the \meta{integer expression}.
 \end{docCommand}
 
The next commands can be used for round off, absolute functions etc.

 \begin{docCommand}{int_abs:n}{\marg{integer expression}}
   Evaluates the \meta{integer expression} as described for
   \docAuxCommand*{int_eval:n} and leaves the absolute value of the result in
   the input stream as an \meta{integer denotation} after two
   expansions.
 \end{docCommand}
 
 \begin{docCommand}{int_div_round:nn}{\marg{intexpr1} \marg{intexpr2}}
   Evaluates the two \meta{integer expressions} as described earlier,
   then divides the first value by the second, and rounds the result
   to the closest integer.  Ties are rounded away from zero.
   Note that this is identical to using
   |/| directly in an \meta{integer expression}. The result is left in
   the input stream as an \meta{integer denotation} after two expansions.
 \end{docCommand}
 

 \begin{docCommand}{int_div_truncate:nn}{ \marg{intexpr1} \marg{intexpr2}}
   Evaluates the two \meta{integer expressions} as described earlier,
   then divides the first value by the second, and rounds the result
   towards zero.  Note that division using |/|
   rounds the result. The result is left in the input stream as an
   \meta{integer denotation} after two expansions.
 \end{docCommand}
 
 \begin{texexample}{Truncating}{ex:truncate}
 \ExplSyntaxOn
 
 \int_div_round:nn  {10}{3}~~
 \int_div_truncate:nn  {13}{3}
 
 \ExplSyntaxOff
 \end{texexample}




\begin{docCommand}{int_max:nn}{ \marg{intexpr1} \marg{intexpr2}}
   Evaluates the \meta{integer expressions} as described for
   \docAuxCommand*{int_eval:n} and leaves either the larger or smaller value
   in the input stream as an \meta{integer denotation} after two
   expansions. The minimum of two numbers an be fund using |\int_min:nn|
\end{docCommand}

 \begin{texexample}{Finding minima and maxima}{ex:maxima}
 \ExplSyntaxOn
 
 \int_max:nn  {10}{3}~~
 \int_min:nn  {13}{3}
 
 \ExplSyntaxOff
 \end{texexample}


  \begin{docCommand}{int_mod:nn}{ \marg{intexpr1} \marg{intexpr2}}
   Evaluates the two \meta{integer expressions} as described earlier,
   then calculates the integer remainder of dividing the first
   expression by the second.  This is obtained by subtracting
   \docAuxCommand*{int_div_truncate:nn} \marg{intexpr1} \marg{intexpr2} times
   \meta{intexpr2} from \meta{intexpr1}.  Thus, the result has the
   same sign as \meta{intexpr1} and its absolute value is strictly
   less than that of \meta{intexpr2}.  The result is left in the input
   stream as an \meta{integer denotation} after two expansions.
   (See example~\ref{ex:mod}).
   
 \end{docCommand}
  
 \begin{texexample}{Modulus}{ex:mod}
 \ExplSyntaxOn
     \int_mod:nn  {10+13}{3+3}~~
 \ExplSyntaxOff
 \end{texexample}

\subsection{Setting and incrementing integers}

\begin{docCommand}{int_add:Nn}{\meta{integer} \marg{integer expression}}
   Adds the result of the \meta{integer expression} to the current
   content of the \meta{integer}.
 \end{docCommand}

\begin{docCommand}{int_incr:Nn}{\meta{integer} \marg{integer expression}}
 Increases the value stored in \meta{integer} by $1$.
 \end{docCommand}


\begin{docCommand}{int_decr:Nn}{\meta{integer} \marg{integer expression}}
  Decreases the value stored in \meta{integer} by $1$. 
 \end{docCommand}
 
 \begin{docCommand}{int_set:Nn}{ \meta{integer} \marg{integer expression}}
   Sets \meta{integer} to the value of \meta{integer expression},
   which must evaluate to an integer.
 \end{docCommand}
 
 \begin{docCommand}{int_sub:Nn} {\meta{integer} \marg{integer expression}}
   Subtracts the result of the \meta{integer expression} from the
   current content of the \meta{integer}.
 \end{docCommand}  

  \section{Using integers}
  
 Although we have already used \docAuxCommand*{int_use:N} to recover and typeset the value of a counter
 we now give its formal definition and an example of usage. As this is the primitive |\the| some care needs to 
 be  taken with expansion.
 

 \begin{docCommand}{int_use:N}{ \meta{integer}}
   Recovers the content of an \meta{integer} and places it directly
   in the input stream. An error will be raised if the variable does
   not exist or if it is invalid. Can be omitted in places where an
   \meta{integer} is required (such as in the first and third arguments
   of \docAuxCommand*{int_compare:nNnTF}).
 \end{docCommand}
 
 If we are to follow \latexe’s paradigm counters are names and not command sequences at the user level.
 With \latex3 of course we can define them as both. In Example~\ref{ex:intuse}, we use the |:c| variant of the commands to define a counter \meta{somecounter}, add globally an integer expression and then retrieve and typeset its value. 
 
 \begin{texexample}{More on retrieving values}{ex:intuse}
 \ExplSyntaxOn
   \int_new:c   {somecounter}
   \int_gadd:cn {somecounter} {(263+223)/23}
   \int_use:c   {somecounter}
 \ExplSyntaxOff
 \end{texexample}
 
\subsection{Longer example}

We wish to create a table that would list numbers and their literal equivalents.

 \begin{texexample}{Creating a numbered table}{ex:inc}
 \ExplSyntaxOn
 \int_gzero_new:N \phd_step_int
 \cs_new:Nn \g_phd_step_counter:n {
     \int_gincr:N\phd_step_int 
     \int_use:N \phd_step_int
 }
 
 \DeclareDocumentCommand\Inc{}{
    \g_phd_step_counter:n
 }
 \begin{tabular}{ll}
 \Inc  & One \\
 \Inc  & Two\\
 \Inc  & Three\\
 \end{tabular}
 
  \ExplSyntaxOff
 \end{texexample}
 
 With all these syntactic changes to the \latex code and conventions, perhaps we should retrace our steps to Knuth’s original terminology, Lamport’s structural documents concepts and a more simplistic language as expounded by my own philosophy in the phd package.

So what do we need, first we need to think that all these functions and programming are to produce documents, so our higher level macros should be document focused. 

The intention of the design layer is to provide interfaces that allow specifying layout and typography styles in a declarative way. On the implementation side there are a number of prototype implementations (most notably xtemplate and the recent reimplementation of the ldb). Those need to get unified into a common model which requires some more experimentation and probably also some further thoughts.

But the real importance of this layer is not the implementation of its interfaces but the conceptual view of it: provisioning a rich declarative method (or methods) to describe design and layout. I.e., enabling a designer to think not in programs but in visual representations and relationships.

 \begin{texexample}{Counter concepts}{ex:inc2}
 \makeatletter
 \ExplSyntaxOn

  \DeclareDocumentCommand\NewCounter{ m } {
     \int_gzero_new:c {#1}
     % create auxiliary functions
     \cs_set:Nn \g_phd_stepcounter:n {
         \int_gincr:c {#1} 
         \int_use:c {#1}
      }
  }
  
 \DeclareDocumentCommand\StepCounter{ m } { 
    \g_phd_stepcounter:n {#1} 
}  
  
 \DeclareDocumentCommand\SetCounter { m m } {
    \int_gset:cn {#1}{#2}
 }

    
\NewCounter{phd_temp_counter} 

 \DeclareDocumentCommand\Inc{}{
    \StepCounter{phd_temp_counter}
 }
 
 \SetCounter{phd_temp_counter}{12}
 
 \DeclareDocumentCommand\CounterToAlpha{ m }{
     \edef\x{\int_use:c{#1}}
     \int_to_Alph:n {\x}
}   
 \DeclareDocumentCommand\CounterToalpha{ m }{
     \edef\x{\int_use:c{#1}}
     \int_to_alph:n {\x}
}   

\DeclareDocumentCommand\CounterToRoman{ m }{
     \edef\x{\int_use:c{#1}}
     \int_to_Roman:n {\x}
}
\DeclareDocumentCommand\CounterToroman{ m }{
     \edef\x{\int_use:c{#1}}
     \int_to_roman:n {\x}
}
\DeclareDocumentCommand\IncA{}{
    \CounterToAlpha{phd_temp_counter}
}
\DeclareDocumentCommand\Inca{}{
    \CounterToalpha{phd_temp_counter}
}

\DeclareDocumentCommand\IncR{}{
    \CounterToRoman{phd_temp_counter}
}
\DeclareDocumentCommand\Incr{}{
    \CounterToroman{phd_temp_counter}
}
\DeclareDocumentCommand\IncW{}{
    \edef\x{\int_use:c{phd_temp_counter}}
    \expandafter\Words@cx{\x}
}
\DeclareDocumentCommand\Incw{}{
    \edef\x{\int_use:c{phd_temp_counter}}
    \expandafter\words@cx{\x}
}

 \begin{tabular}{c c c c c c c}
 \toprule
 Number & Literal & literal &Alpha &alpha &Roman &roman\\
 \midrule
 \Inc  & \IncW &\Incw &\IncA &\Inca &\IncR &\Incr\\
 \Inc  & \IncW &\Incw &\IncA &\Inca &\IncR &\Incr\\
 \Inc  & \IncW &\Incw &\IncA &\Inca &\IncR &\Incr\\
 \Inc  & \IncW &\Incw &\IncA &\Inca &\IncR &\Incr\\
 \bottomrule
 \end{tabular}
 
 \ExplSyntaxOff
 \makeatother
 \end{texexample}

What have just happened in our example, we have emulated most of \latexe counter macros in a kind of a different way, but there is an important part of the counter macros that is missing. This is the ability to reset counters when a master counter is changed. For example when a chapter counter is incremented the section counters in \latexe will reset and start counting from one again.



\docAuxCommand*{cl@foo} List of counters to be reset when foo stepped. This has a  format
\begin{verbatim}
   \@elt{countera}\@elt{counterb}\@elt{counterc}
\end{verbatim}

\textbf{Adding a prefix to the counters} So what we need to do first decide on some prefixes for counters or another similar convention. I will use a prefix |__counter_| to make the code readable. Although tempting to use |c@|  to make our counters compatible with \latexe counters, as stated earlier this would be against the central philosophy of \latex3 of keeping a onsistent syntax and function aiming conventions. All we have to do is add the prefixes and create the new counter to hold the rest counters and provide helper commands. We can also do some error capturing. 


\begin{phdverbatim}
\DeclareDocumentCommand\NewCounter{ o m } {
% create new counter if it does not exist and also its resets counters
  \int_if_exist:NTF {__counter_#2}
    {
      \msg_error:nn { counters } { counter exists and cannot be created }
    }
    { 
      \int_gzero_new:c {__counter_#2}
      \int_gzero_new:c {__counter_resets_#2}
    }
 
% handle the reset   
  \IfNoValueF{#1}
    {
      \int_if_exist:NTF {__counter_resets_#1} 
                        {add to reset} {false code}
    }    
% create auxiliary functions  to be added later   
       
  }
\end{phdverbatim}
    
This is a rough outline of the code we need to develop. It is considered bad practice to mix too many low level commands with high level commands and we should replace these with auxiliary functions. The auxiliary functions will create automated functions such as |\thechapter| in \latexe.

\begin{teXXX}
\cs_new:Nn \phd_create_new_counter:n {
    \int_if_exist:NTF {__counter_#2}{error}
         { 
             \int_gzero_new:c {__counter_#2}
             \int_gzero_new:c {__counter_resets_#2}
         }
}
\end{teXXX}

We continue our code development, this time we will add a prefix before the counter name. The code is shown
in Example~\ref{ex:createcounters}



\begin{texexample}{Auxiliary constructor function}{ex:createcounters}
\ExplSyntaxOn
\makeatletter
\cs_gset:Nx  \counter_prefix: {c@}
\cs_gset:Nx  \counter_resets_prefix: {__counter_resets_prefix_}

\cs_gset:Nn  \phd_create_new_counter:n {
    \int_if_exist:cTF {\counter_prefix:#1}{ERROR~counter~exists}
        { 
            \int_gzero_new:c {\counter_prefix:#1}
            \seq_new:c {\counter_resets_prefix:#1}
        }
}

\phd_create_new_counter:n {test}

\phd_create_new_counter:n {test2}

\cs_gset:Nn\stepacounter:n {
  \int_gincr:c{\counter_prefix: #1}
}

\cs_gset:Nn\setacounter:cn {
  \int_set:cn {\counter_prefix: #1}{#2}
}

\cs_gset:Nn  \countervalue:n {
    \the\cs:w\counter_prefix: #1\cs_end:\relax
}

\def\makeauxiliaries#1 {\mycommandaux#1\relax}

\def\mycommandaux#1#2\relax{%
       \uppercase{
       \expandafter\gdef\csname #1}
       #2Value
       \endcsname
       {\the\cs:w\counter_prefix: #1#2\cs_end:\relax}
}

\makeauxiliaries {test}

\setacounter:cn {test}{18}

\countervalue:n {test}\par


\makeauxiliaries {section}
first~test~\TestValue\par 
\stepacounter:n {test}
\stepacounter:n {test}

second~test~\TestValue\par 

section~counter~first~test:~\SectionValue\par
section~counter~with~\docAuxCommand*{thesection}:~\thesection\par
\stepacounter:n {section}
section~counter~second~test:~\SectionValue
\ExplSyntaxOff
\makeatother
\end{texexample}



\ExplSyntaxOn
\newcommand{\makeauxiliaryfunctions}[1]{\mycommandaux#1\relax}
\def\mycommandaux#1#2\relax{%
       \uppercase{\expandafter\gdef\cs:w #1}#2Value\cs_end:
       {\tex_the:D\cs:w\counter_prefix: #1#2\cs_end:\relax}%
    }
\ExplSyntaxOff
    
The example above creates a function |\phd_create_new_counter:n| and then tests it. The function uses a conditional to test if the counter exists  and then if it has not been defined earlier it creates the two counters and sets them to zero. If it exists it will typeset an error message. This is of course so that we can view the error in the document, normally this would display an error on the screen. I have not covered the messaging part of the code so far for displaying errors and warnings. These are created with the \pkgname{l3msg} package, which is bundled with |expl3|. We will cover this later on in the book. 

\textbf{Adding the counter to the reset list of another} We now go on to develop a function to add to the reset list of another.

\begin{texexample}{Adding to the reset}{ex:addtoreset}
\ExplSyntaxOn
\cs_gset:Nn \addtoreset:nn {
    \exp_args:Nf\seq_put_left:cn {\counter_resets_prefix:#1}{#2}
    Added~ to~the~ #1 ~ resets~ #2.~The~resets~list~is~now~\seq_use:cn {\counter_resets_prefix:#1}{,}
 }

\ExplSyntaxOff
\end{texexample}

The |\stepcounter:n| will be used to step a counter and to reset all subsidiary counters. 


\begin{texexample}{Adding to the reset}{ex:addtoreset}
\ExplSyntaxOn
\makeatletter
\cs_gset:Npn \resetcounter:n 
  {
    \int_gset:cn {\counter_prefix: #1}{0}
  }
  
\cs_gset:Npn \stepcounter:n {
  \int_gincr:c {\counter_prefix: #1}
  \seq_set_eq:Nc \tempa {__counter_resets_prefix_#1}
  \seq_map_inline:Nn \tempa {\resetcounter:n{##1}}
}      

\stepcounter:n {test}

% Test that the value is captured
\int_use:c {\counter_prefix: test}

\makeatother
\ExplSyntaxOff
\end{texexample}

Now that we have developed the code for |\addtoreset:nn| we are ready to modify and finalize our counter creation macro to the following:

\begin{texexample}{Refactor creation macro}{ex:refactor}
\ExplSyntaxOn
 \DeclareDocumentCommand \NewCounter{ o m } {
    \phd_create_new_counter:n {#2}
    \IfNoValueF{#1}
      {
         \int_if_exist:cT {\counter_resets_prefix:#1} 
             {\addtoreset:nn{#1}{#2}} 
      }    
    \makeauxiliaryfunctions {#1}
 }
%\NewCounter{Chapter}
%\NewCounter[Chapter]{Section}\par
%\NewCounter[Chapter]{Figure}
%\stepcounter:n{Chapter}
%\int_use:c {\counter_prefix: Chapter} 
%\int_use:c {\counter_prefix: Section}
%\countervalue:n{Chapter}
%\ChapterValue
\ExplSyntaxOff
\end{texexample}

%% Rewrite as the examples are in a group and they cannot leak out

\ExplSyntaxOn
\NewDocumentCommand\NewCounter{ o m } {
  \phd_create_new_counter:n {#2}
    \IfNoValueF{#1}
      {
        \int_if_exist:cT {\counter_resets_prefix:#1} 
             {\addtoreset:nn{#1}{#2}} 
      }    
  \makeauxiliaryfunctions {#1}
}
\ExplSyntaxOff

All we have to now do is to write some tests. The \latex3 Team provide testing routines with the tests being run using a Lua script. In our case we can run all our tests within the documentation here. These tests are shown in Example~\ref{ex:countertests}. 


\begin{texexample}{Counter module tests }{ex:countertests}
\ExplSyntaxOn
\NewCounter{Chapter}
\NewCounter[Chapter]{Section}\par
\NewCounter[Chapter]{Figure}\par
\NewCounter[Chapter]{Problems}\par
\stepcounter:n{Chapter}
\stepcounter:n{Chapter}
Value~ of~ Chapte~ counter:~ \int_use:c {\counter_prefix: Chapter}\par 
Value~of~Section~ counter:~\int_use:c {\counter_prefix: Section}\par
Value~of~Chapter~ counter~ using \docAuxCommand*{countervalue:n}:~ \countervalue:n{Chapter}\par
Value~of~Chapter~ counter~ using \docAuxCommand*{ChapterValue}:~ \ChapterValue\par
\ExplSyntaxOff
\end{texexample}


\begin{docCommand}{ChapterCounter}{}
   Typesets the Chapter counter. This is equivalent to |\thechapter|. Decorating the actual counter value, should
   all be based on a key-value system and the author has no access to this function. It is the template designer’s 
   job to define it.
   
  \begin{verbatim}
   chapter numbering = arabic
   chapter font-size = huge
  \end{verbatim} 
  
\end{docCommand}

\begin{docCommand}{ChapterCounterValue}{}
   Typesets the Chapter counter in arabic. This form can also be used in other expressions This is equivalent to \latexe’s \docAuxCommand*{c@chapter}, but not equal, i.e, there is no interface to \latexe counters.
\end{docCommand}
  
  
 \subsection{Integer conditionals}

Comparing the values of two counters can be achieved with the use of conditional expressions. There are numerous commands provided for this purpose and we outline some of the most important ones. Do consult the manual to view others. The first one we will examine is \docAuxCommand*{int_compare:nNnTF}. This function evaluates each of two expressions and branches to the true or false code. 

\begin{docCommand}{int_compare:nNnTF}{\marg{intexpr1} \meta{relation} \marg{intexpr2}
\marg{true code} \marg{false code}}
   This function first evaluates each of the \meta{integer expressions}
   as described for \docAuxCommand*{int_eval:n}. The two results are then
   compared using the \meta{relation}:
   \begin{center}
     \begin{tabular}{ll}
       Equal                 & |=| \\
       Greater than      & |>| \\
       Less than           & |<| \\
     \end{tabular}
   \end{center}
 \end{docCommand}
 
 Consider two counters \docAuxCommand*{counteri} and \docAuxCommand*{counterii} that we need to compare their values and branch to either false or true code.
 
 \begin{texexample}{Integer conditionals}{ex:intconditionals}
 \ExplSyntaxOn
 \int_new:N  \counteri
 \int_new:N  \counterii
 
 \int_compare:nNnTF {\counteri} = {\counterii}
     {true~code\par}{false~ code\par}
     
 \int_gadd:Nn \counterii {15+12}    
 
  \int_compare:nNnTF {\counteri} = {\counterii}
     {true~code~\par}{false~ code\par}
     
\ifnum\counteri<\counterii primitive~ifnum~true\else primitive~false\fi\par
   
 \ExplSyntaxOff
 \end{texexample}
 
 A common error is to include the \meta{relation} code in curly brackets. This leads to errors during parsing. 
 
   
 
 
  \begin{docCommand}{int_case:nnTF} {\marg{test integer expression}
      \{ 
     \marg{intexpr case1} \marg{code case1} 
     \marg{intexpr case2} \marg{code case2} 
     \ldots 
     \marg{intexpr casen} \marg{code casen} 
     \} 
     \marg{true code}
     \marg{false code}}
     
   This function evaluates the \meta{test integer expression} and
   compares this in turn to each of the
   \meta{integer expression cases}. If the two are equal then the
   associated \meta{code} is left in the input stream. If any of the
   cases are matched, the \meta{true code} is also inserted into the
   input stream (after the code for the appropriate case), while if none
   match then the \meta{false code} is inserted. The function
   \docAuxCommand*{int_case:nn}, which does nothing if there is no match, is also
   available. For example
\end{docCommand}   
   \begin{texexample}{Case example}{ex:case}
   \makeatletter
   \ExplSyntaxOn
     \int_case:nnF
       { 2 * 5 }
       {
         { 5 }       { Small }
         { 4 + 6 }   { Medium }
         { -2 * 10 } { Negative }
       }
       { No idea! }
       
   
  \cs_set:Npn \@fnsymbol #1
   {
    \int_case:nnF {#1}
     {
      {0} {}
      {1} { \TextOrMath \textasteriskcentered* }
      {2} { \TextOrMath \textdagger\dagger }
      {3} { \TextOrMath \textdaggerdbl\ddagger }
      {4} { \TextOrMath \textsection\mathsection }
      {5} { \TextOrMath \textparagraph\mathparagraph }
      {6} { \TextOrMath \textbardbl\| }
      {7} { \TextOrMath {\textasteriskcentered\textasteriskcentered}{**} }
      {8} { \TextOrMath {\textdagger\textdagger}{\dagger\dagger} }
      {9} { \TextOrMath {\textdaggerdbl\textdaggerdbl}{\ddagger\ddagger} }
     }
     { \@ctrerr }
   }
   
   
   \@fnsymbol {3}
     \ExplSyntaxOff  
     \makeatother
    \end{texexample}
    
 
 
 The next case example is from \href{https://github.com/wspr/fontspec/blob/master/fontspec.dtx}{fontspec.dtx}. It just wraps the official \latexe definition from \pkgname{fixltx2e} into the |expl| language. If you are going to utilize code from \latexe verbatim, it is always best to use it as is and only using a wrapper.

 \begin{texexample}{Case example}{ex:case2}
   \makeatletter
   \ExplSyntaxOn
   
  \cs_set:Npn \@fnsymbol #1
   {
    \int_case:nnF {#1}
     {
      {0} {}
      {1} { \TextOrMath \textasteriskcentered* }
      {2} { \TextOrMath \textdagger\dagger }
      {3} { \TextOrMath \textdaggerdbl\ddagger }
      {4} { \TextOrMath \textsection\mathsection }
      {5} { \TextOrMath \textparagraph\mathparagraph }
      {6} { \TextOrMath \textbardbl\| }
      {7} { \TextOrMath {\textasteriskcentered\textasteriskcentered}{**} }
      {8} { \TextOrMath {\textdagger\textdagger}{\dagger\dagger} }
      {9} { \TextOrMath {\textdaggerdbl\textdaggerdbl}{\ddagger\ddagger} }
     }
     { \@ctrerr }
 }
 
   
 \@fnsymbol {3}
 \ExplSyntaxOff  
 \makeatother
 \end{texexample}
    

%
% \begin{function}[updated = 2013-01-13, EXP, pTF]{\int_compare:n}
%   \begin{syntax} 
%     \docAuxCommand*{int_compare_p:n} \\
%     ~~\{ \\
%     ~~~~\meta{intexpr_1} \meta{relation_1} \\
%     ~~~~\ldots{} \\
%     ~~~~\meta{intexpr_N} \meta{relation_N} \\
%     ~~~~\meta{intexpr_{N+1}} \\
%     ~~\} \\
%     \docAuxCommand*{int_compare:nTF}
%     ~~\{ \\
%     ~~~~\meta{intexpr_1} \meta{relation_1} \\
%     ~~~~\ldots{} \\
%     ~~~~\meta{intexpr_N} \meta{relation_N} \\
%     ~~~~\meta{intexpr_{N+1}} \\
%     ~~\} \\
%     ~~\Arg{true code} \Arg{false code}
%   \end{syntax}
%   This function evaluates the \meta{integer expressions} as described
%   for \docAuxCommand*{int_eval:n} and compares consecutive result using the
%   corresponding \meta{relation}, namely it compares \meta{intexpr_1}
%   and \meta{intexpr_2} using the \meta{relation_1}, then
%   \meta{intexpr_2} and \meta{intexpr_3} using the \meta{relation_2},
%   until finally comparing \meta{intexpr_N} and \meta{intexpr_{N+1}}
%   using the \meta{relation_N}.  The test yields \texttt{true} if all
%   comparisons are \texttt{true}.  Each \meta{integer expression} is
%   evaluated only once, and the evaluation is lazy, in the sense that
%   if one comparison is \texttt{false}, then no other \meta{integer
%     expression} is evaluated and no other comparison is performed.
%   The \meta{relations} can be any of the following:
%   \begin{center}
%     \begin{tabular}{ll}
%       Equal                    & |=| or |==| \\
%       Greater than or equal to & |>=|        \\
%       Greater than             & |>|         \\
%       Less than or equal to    & |<=|        \\
%       Less than                & |<|         \\
%       Not equal                & |!=|        \\
%     \end{tabular}
%   \end{center}
% \end{function}
%

%
% \begin{function}[EXP,pTF]{\int_if_even:n, \int_if_odd:n}
%   \begin{syntax}
%     \docAuxCommand*{int_if_odd_p:n} \Arg{integer expression}
%     \docAuxCommand*{int_if_odd:nTF} \Arg{integer expression}
%     ~~\Arg{true code} \Arg{false code}
%   \end{syntax}
%   This function first evaluates the \meta{integer expression}
%   as described for \docAuxCommand*{int_eval:n}. It then evaluates if this
%   is odd or even, as appropriate.
% \end{function}
 \subsection{Integer expression loops}
 
 Integer expression loops, bring \latex nearer to the functionality of other computer languages. In Example~\ref{ex:dowhile} a \emph{do}\ldots \emph{while} loop is constructed to print all even numbers from |0..16|. Many variations to looping structures are also provided and these are discussed after the example.
 
 \begin{texexample}{Integer Expression loops}{ex:dowhile}
 \ExplSyntaxOn
 \int_new:N \l_tempa_int
 \int_zero:N \l_tempa_int 
 \int_do_while:nn {\l_tempa_int <= 10 + 6 } {
     \int_use:N \l_tempa_int,~ 
     \int_add:Nn \l_tempa_int {2}
 }
 \ExplSyntaxOff
 \end{texexample}
 
 In the next example we will use \refCom{int_step_function} and call a function at each iteration. 
 
\begin{texexample}{Step function} {ex:stepfunction}
\ExplSyntaxOn
\cs_set:Npn \my_func:n #1 {test~#1}
\int_step_function:nnnN {1} {1} {5} {\my_func:n}
\ExplSyntaxOff
\end{texexample}
 
\section{Summing up} 

This has been a long chapter, and we both deserve some coffee and a break. We have discussed the creation of integer expressions, their use as counters and typesetting commands for counters. We have also examined in depth conditionals associated with integers and also some looping structures that are very robust. I have spent more time than I expected on this module, as it wraps up a lot of the concepts we have been discussing in other chapters and I thought a thorough review and some longer examples would be beneficial. 

By now, if you have been running the examples on your own, you should be more or less start thinking in
\latex3 speak. It takes a while for the syntax and the concepts to sink in. From my own experience, you need to spend at least 2-3 weeks just programming in the |expl| language and you should avoid the temptation to use \latexe macros or \tex primitives. Easier said that done.
 

 
\cxset{section align=left,
       section font-weight=bold}
       
\chapter{The LaTeX3 l3msg Module and how to use it for Error, Warning and other Messages}
\index{messaging>error}\index{messaging>warning}
\section{Introduction}

Messages need to be passed to the user by modules, either when errors occur or to indicate
how the code is proceeding. The l3msg module provides a consistent method for doing
this (as opposed to writing directly to the terminal or log).

The system used by l3msg to create messages divides the process into two distinct
parts. Named messages are created in the first part of the process; at this stage, no
decision is made about the type of output that the message will produce. The second
part of the process is actually producing a message. At this stage a choice of message
class has to be made, for example error, warning or info.

By separating out the creation and use of messages, several benefits are available.
First, the messages can be altered later without needing details of where they are used
in the code. This makes it possible to alter the language used, the detail level and so
on. Secondly, the output which results from a given message can be altered. This can be
done on a message class, module or message name basis. In this way, message behaviour
can be altered and messages can be entirely suppressed.

\section{Creating messages}

Messages \emph{must} be created before they can be used. This has the advantage that they can be used
over and over and also one could start thinking of internationalizing the package.

Messages can be created as either new or set and there is also a TF to check if the message exists:

\begin{docCommand}{msg_new:nnnn}{ \marg{module} \marg{message} \marg{text} \marg{more text} }
Creates a hmessagei for a given hmodulei. The message will be defined to first give htexti
and then \meta{more text} if the user requests it. If no \meta{more text} is available then a standard
text is given instead. Within htexti and more text four parameters (\#1 to \#4) can be
used: these will be supplied at the time the message is used. An error will be raised if
the \meta{message} already exists.
\end{docCommand}

\section{Messaging classes}

Messages are divided into categories termed message classes. 
These are according to \emph{severity}, \emph{fatal}, \emph{critical}, \emph{error}, \emph{warning} and \emph{info}. Each one has its own set of creation functions.

\begin{docCommand*}{msg_fatal:nnnnnn}{\marg{module} \marg{message} \marg{arg one} \marg{arg two} \marg{arg three}
\meta{arg four}}
Issues \meta{module} error \meta{message}, passing \meta{arg one} to \meta{arg four} to the text-creating
functions. After issuing a fatal error the \tex run will halt.
\end{docCommand*}



\begin{texexample}{Typical package error setup}{ex:errors}
% Error message example
%
% simulate LaTeX2e \fmversion
\def\fmversion{2000/11/12}
\makeatletter
\ExplSyntaxOn 

% create error boolean
\bool_new:N \l_mypackage_error_bool
 
% redirect package errors here  %(*@\label{l:warning}@*)
\cs_new_protected:Npn \mypackage_warning:nxx #1 #2 #3 
  {
    \bool_set_true:N \l_mypackage_error_bool
    \msg_info:nnxx { mypackage } { #1 } { #2 } { #3 }
  }
 
% define some error messages
\msg_set:nnnn { mypackage } { old-version }
  { LaTeX~source~files~more~than~5~year~old.~ Is~dated~(year:#1~date:#2-#3) }
  { Please~update~your~distribution~visit~ctan } %(*@\label{test}@*)
   

% check version number   	   	
\cs_new:Npn \mypackage_check_version:n #1 
  {
    \exp_after:wN \l_mypackage_check_version_aux:w #1\q_stop
  }

% check version number auxiliary 
% #1 relation
% #2 true code
% #3 false code  
\cs_new:Npn \l_mypackage_check_version_aux:w #1/#2/#3\q_stop 
  {
    \int_compare:nNnTF { ( \tex_year:D-#1 )*12 + (\tex_month:D-#2) } > { 65 }
      { \FAIL \mypackage_warning:nxx { old-version } { #1 } { #1 / #2 / #3 } }
      { \PASS } 
  }
  
\mypackage_check_version:n \fmversion 
\mypackage_check_version:n \fmtversion 
  
\ExplSyntaxOff
\end{texexample}

In Example~\ref{ex:errors} Line \ref{test} we define our own package command for issuing an error message, rather than typing |\msg_error:nnxx| directly. This is considered good practice and we avoid typing in \enquote{mypackage} all the time. 

\section{How to translate strings}
\index{internationalization>expl3}

I posted similar code to the |TX.SX| Q\&A site to elicit comments from other users, as to recommended best practices.
%http://tex.stackexchange.com/questions/246810/latex3-l3msg-best-practices 
At this moment in time I am not too sure if a LaTeX3 approach to internationalization is appropriate. If one looks at the complexities, it is preferable to use Lua. 






\chapter{Expl3 File Operations}
\label{ch:l3files}


 
\tex provides only some basic primitive control sequences for dealing with files. \tex is also limited to 16 input streams and 16 output streams making it considerably difficult to manipulate too many files. In most of the examples here we have used, streams allocated by the \latexe kernel for temporary operations and hence we can re-use them, but with extreme care. 

Checking for the existence of a file is simple and we can use the |\file_if_exist:nTF| function. 

\section{Creating streams}

Before you can use a file in \tex you need to allocate a stream. With \latex3 you can use \docAuxCommand*{ior_new:N} or \docAuxCommand*{iow_new:N} depending if is a stream for write or read. All I/O operations are global: streams are declared with global names and treated accordingly.   

As one can run out of handles very quickly, the \pkgname{phd} package loads the \pkgname{morewrites} and also patches the \pkgname{filecontents} package to enable us to use it. This was renamed to |phdfilecontents|. In Example~\ref{ex:createstreams} 

\begin{texexample}{File streams}{ex:createstreams}
\ExplSyntaxOn
  \iow_new:N \scratch_filea
  \iow_new:N \scratch_fileb
  \iow_new:N \scratch_filec
  \iow_new:N \scratch_filed
\ExplSyntaxOff
\end{texexample}

Another way used widely by \latexe and package authors is to use \latexe's |\@inputcheck| file handle. This is a file handle used by |filecontents| as well as internally by the \latexe kernel for  and for building the |\IfFileExists| control sequence and hence its name. 


\begin{texexample}{LaTeXe examples}{ex:inputcheck}
\makeatletter
\bgroup
\ttfamily \meaning\@inputcheck\\
\number\@inputcheck %
\egroup
\makeatother
\end{texexample}

Stream management goes back to \tex and Plain\tex which use an allocation meachanism to assign the the names to numbers and then . This mechanism was then moved onto \latex2.  \latex3 uses a slightly different mechanism but the basic logic is still the same. With \latex3 the allocations are kept in a property list. 

\section{File Operations}
\begin{texexample}{File operations}{ex:fileops}
\ExplSyntaxOn
% Check if a file exists 
  \file_if_exist:nTF { filetest.txt } { \PASS } { \FAIL }
% Check for another one
  \file_if_exist:nTF { filetest }     { \PASS } { \FAIL }
  \g_file_current_name_tl
\ExplSyntaxOff
\end{texexample}

\subsection{Handling paths}

The |l3files| module provides a mechanism to add paths to the search path used to search for 
a file. This is pretty much similar to |\graphicspath|.

\begin{docCommand*}{file_path_include:n} {\marg{path}}
Adds \meta{path} to the list of those used to search when reading files. The assignment is local.
The \meta{path} is processed in the same way as a \meta{file name}, i.e., with x-type expansion
except active characters. Spaces are not allowed in the \meta{path}.
\end{docCommand*}

In Example~\ref{ex:corpora1} we add to the search path the directory, where our corpora data is residing. Then we check to see if the file |female.txt| exist. This is a large text file (extension |.txt|), containing common female names
in the US. We will use this file later on for some for examples.

In the |github| distribution that comes with this book, we provide a folder that includes numerous files. In the \ref{ex:corpora1} we demonstrate the use of the |file_path_include:n| function to check if the files exist. We 
weill use these files later on. 

\begin{texexample}{File operations}{ex:corpora1}
\ExplSyntaxOn
% Add path
\file_path_include:n {./corpora/}
\file_if_exist:nTF {female.txt} {\PASS}{\FAIL}
\ExplSyntaxOff
\end{texexample}
\index{path}\index{file operations>path}

\subsection{Loading files}

Loading a full file can of course be achieved with the |\input| command. Under the experimental section of the |\expl3| there is also a function that can load a file on condition that it exists. 

\begin{docCommand} {file_if_exist_input:n} { \marg{file name} }
Searches for \meta{file name} using the current TEX search path and the additional paths
controlled by |\file_path_include:n|). If found, inserts the \meta{true code} then reads in
the file as additional \latex source as described for |\file_input:n|. Note that 
|\file_if_exist_input:n| does not raise an error if the file is not found, in contrast to |\file-input:n|.
\end{docCommand}

\begin{texexample}{Loading a file only if it exists}{}
\ExplSyntaxOn
\file_if_exist_input:n {filetest.txt}
\ExplSyntaxOff
\end{texexample}

Since the file is loaded within the expl3 block, it will cause all spaces to be removed from the output, we can overcome this by following the |expl3| design pattern of declaring a function with |xparse| and calling it outside the block.

\begin{texexample}{Loading a file only if it exists}{}
\ExplSyntaxOn
\DeclareDocumentCommand \CorporaInput{ m }
  {
    \file_if_exist_input:n { #1 }
  }
\ExplSyntaxOff
\CorporaInput {filetest.txt}  
\end{texexample}

As is usual with |expl3| functions the |nTF| signature form of the |\file_if_exist_input:| is also available. 

\begin{texexample}{Loading a file only if it exists}{}
\ExplSyntaxOn
\DeclareDocumentCommand \CorporaInput{ m }
  {
     \file_if_exist_input:nTF {#1}
  }
\ExplSyntaxOff
\CorporaInput {filetest.txt}{ \PASS } { \FAIL }
\CorporaInput { filetest.txt }{ \PASS } { \FAIL }
\end{texexample}

The above code fails if we leave spaces between the \{\verb*+ filetest.txt +\}, we can easily remove them using the  |>{ \TrimSpaces }| argument processor.

\begin{texexample}{Loading a file only if it exists}{ex:trimspaces}
\ExplSyntaxOn
\DeclareDocumentCommand \CorporaInput{  >{  \TrimSpaces } m }
  {
     \file_if_exist_input:nTF {#1}
  }
\ExplSyntaxOff
\CorporaInput {filetest.txt}{ \PASS } { \FAIL }
\CorporaInput { filetest.txt }{ \PASS } { \FAIL }
\makeatletter
%\@ifpackageloaded{test}{\PASS test loaded}{\FAIL}
\makeatother
\end{texexample}

In the next example, we will try and consolidate some of the skills we have been developing so far.
In the \pkgname{phd} package, we are loading over 70 packages through the package manager. We wanted
to automatically keep track of which packages we loaded and which we did not (as they might
not have been in our distribution). \latexe provides a useful macro \docAuxCommand{@ifpackageloaded}
that can check if the package has been loaded or not. 

\begin{texexample}{Loading a Package}{ex:loadpackage}
\ExplSyntaxOn

\clist_new:N \g_packages_loaded_clist
\clist_new:N \g_packages_failed_clists
\clist_new:N \g_packages_loaded_by_others_clist

\makeatletter
%\DeclareDocumentCommand \requirepackage{  >{  \TrimSpaces } m m m }
%  {
%     \file_if_exist:nTF {#1.sty} 
%       { 
%         \@ifpackageloaded{#1} 
%              {
%                   \clist_put_left:Nn \g_packages_loaded_by_others_clist  {#1} 
%               }
%               {
%                   \clist_put_left:Nn \g_packages_loaded_clist  {#1}  
%                 #2
%              }
%        } 
%        { 
%          \clist_put_left:Nn \g_packages_failed_clist  {#1}
%          #3 
%        }
%  }
 

%\@ifpackageloaded{xcolor}{true}{false}
%\@ifpackageloaded{lettrine}{\PASS}{\FAIL}
%\@ifpackagewith{ragged2e}{}{\PASS}{\FAIL}
%\@ifpackagewith{soul}{}{\PASS}{\FAIL}
%\@ifpackagewith{siunitx}{fixed-exponent,scientific-notation}{\PASS}{\FAIL siunitx}
% \makeatother
% 
%\par
%\requirepackage {xcolor}    {\PASS } { \FAIL }
%\requirepackage {soul}      {\PASS } { \FAIL }
%\requirepackage {calligra}  {\PASS } { \FAIL }
%\requirepackage {hyperref}  {\PASS } { \FAIL }
%\requirepackage {layouts}   {\PASS } { \FAIL }
%\par
%
%\clist_map_inline:Nn \g_packages_loaded_by_others_clist 
%  {
%    loaded~by~others~\ldots~ #1,~
%  }
\makeatother
\ExplSyntaxOff
\end{texexample}



\section{Reading and writing to streams}

Typical file operations are reading, writing and appending. Common file management operations are creating, deleting, opening, closing, copying and renaming.

\begin{texexample}{File operations}{ex:fileops}
\edef\someheading{Another test}
\ExplSyntaxOn
\iow_open:Nn \tempstream { filetest.txt }
\iow_now:Nx \tempstream {\someheading}
\iow_close:N \tempstream
\let\getfile\file_input:n
%\file_input:n {filetest.txt}
\ExplSyntaxOff
\getfile {filetest.txt}
%\getfile{./corpora/female.txt}
\end{texexample}   

\section{Input-output streams}

Reading one line at a time from a file, uses the \tex primitive |\read|. One important item to watch is that there are different commands for read and write you need to use the |\io|\textcolor{thered}{\texttt{r}}, rather than |iow_|

Streams  are precious in \tex  as we only have 16 available , so when reading from a file, when we use \LaTeXe, we can use some of \LaTeX build-in streams, so we will be using |\@inputcheck|

\begin{texexample}{File operations}{ex:fileops}
\ExplSyntaxOn

\makeatletter
\global\let\ltx_scratch_stream \@inputcheck
\makeatother
\ior_open:Nn \ltx_scratch_stream {male-a.txt}
\ior_get:NN \ltx_scratch_stream \l_tmpa_tl
\tl_use:N \l_tmpa_tl\par
\ior_get:NN \ltx_scratch_stream \l_tmpa_tl
\tl_use:N \l_tmpa_tl
\ior_close:N \ltx_scratch_stream

\ExplSyntaxOff
\end{texexample}   


Reading line by line is not very useful. What we need is the ability to read all he lines
recursively until the end of the file. 

\begin{texexample}{File operations}{ex:fileops}
\ExplSyntaxOn
% add the file path to the name
\file_path_include:n {./corpora/}

% open the stream
\ior_open:Nn \ltx_scratch_stream {male-a.txt}

% define a macro so we can do recursion
\cs_set:Npn \read_loop {
  \if_eof:w \ltx_scratch_stream
    \ior_close:N \ltx_scratch_stream
    \let\next\relax
 \else:
   \ior_get:NN \ltx_scratch_stream \tmpa
   \tl_use:N \tmpa,~
   \let\next\read_loop
 \fi:      
 \next 
}

% read the file and typeset the words with a comma
\read_loop
\ExplSyntaxOff
\RaggedRight
\end{texexample}   

In the following example we do some more changes to the example. This time instead of typesetting the names,
we will add them to a clist. We will also read a different file that contains an alphabetical list of male names. Then we will check if the name Zacharias is in the list. 

\begin{texexample}{File operations}{ex:fileops}
\ExplSyntaxOn
% \clist
\clist_new:N \males
% open the stream
\file_path_include:n {./corpora/}
\ior_open:Nn \ltx_scratch_stream {male.txt}

% define a macro so we can do recursion
\cs_set:Npn \read_loop {
  \ior_if_eof:NTF \ltx_scratch_stream
    {
      \ior_close:N \ltx_scratch_stream
      \cs_set_eq:NN \next\relax
    }
    {  
      \ior_get:NN \ltx_scratch_stream \tmpa
      \clist_put_right:Nx \males {\tl_use:N \tmpa}
      \cs_set_eq:NN \next\read_loop
   } 
 \next 
}

% read the file and parse the words as a clist \males
\read_loop

% check if Zacharias or Mary are in the list
\clist_if_in:NnTF\males {Zacharias} {\PASS} {\FAIL}
\clist_if_in:NnTF\males {Mary} {\PASS} {\FAIL}
\ExplSyntaxOff
\end{texexample}   

\section{Appending to a file}

To append to a file, first we need to read the contents of the file into a |tl_var| then apend the material and close the file. We then open it again in write mode and write the contents of the |tl_var|. Let us try it out.


    
\section{Writing to the log or aux files}    

There are some constant input-output streams. There is a somewhat different programming philosophy here are these are normally via messages and not directly as shown in the example here.  These are handled in the chapter dealing with messages.

\begin{texexample}{Writing to log and terminal}{ex:log}
\ExplSyntaxOn
\iow_term:x {Something}
\ExplSyntaxOff
\end{texexample}
      
Armed with all these it maybe time to review again our database functions that we have created in the earlier chapter on clists.

                
          
            
              
                
                  
                      




\chapter{LaTeX3 Key value system}
\label{l3:keys}
The key-value system has been discussed earlier but avoided to cover the |l3keys| module of \latex3 until such time as the basics of the expl3 syntax was discussed. 


The l3keys modules provides general purpose keyval processing for |expl3| code. However, it does not interact with LaTeX2e's package or class option system. For that, you need to load some additional code, which is available in the package l3keys2e. This provides the \docAuxCommand*{ProcessKeysOptionscommand} to parse class/package options and process them using keyvals defined by l3keys.

The reason for this separation is that l3keys is intended to form part of a LaTeX3 kernel, while l3keys2e is tied to the LaTeX2e model for processing options. It seems extremely likely that a stand-alone LaTeX3 kernel will use keyval options 'natively' but with a different underlying implementation.

 The high level functions here are intended as a method to create
 key--value controls. Keys are themselves created using a key--value
 interface, minimising the number of functions and arguments
 required. Each key is created by setting one or more \emph{properties}
 of the key:
 \begin{verbatim}
   \keys_define:nn { mymodule }
     {
       key-one .code:n   = code including parameter #1,
       key-two .tl_set:N = \l_mymodule_store_tl
     }
 \end{verbatim}
 
  At a document level, |\keys_set:nn| will be used within a
 document function, for example
 \begin{verbatim}
   \DeclareDocumentCommand \MyModuleSetup { m }
     { \keys_set:nn { mymodule } { #1 }  }
   \DeclareDocumentCommand \MyModuleMacro { o m }
     {
       \group_begin:
         \keys_set:nn { mymodule } { #1 }
        ... Main code for the macro
       \group_end:
     }
 \end{verbatim}
 
 The process of incorporating a key value system into a macro or a package involves three steps. First the keys are defined then processed to set them to some values and lastly incorporated into a function or package.
 
 It is best to illustrate the process with a small example. Example\ref{ex:keyval1} defines two keys that affect the typesetting of paragraphs |parindent| and |parskip|. These are defined using the |.code|, pretty much the same way that |pgfkeys| that we discussed earlier defines keys. 
 
 \begin{texexample}{Key value}{ex:keyval1}
 \ExplSyntaxOn
 \keys_define:nn {scratch}
   {
      parindent .code:n = \parindent#1,
      parskip     .code:n = \parskip#1
   }
   
\DeclareDocumentCommand \MyModuleSetup { m }
     { \keys_set:nn { scratch } { #1 }  }
     
\DeclareDocumentCommand \MyModuleMacro { o }
     {
       \group_begin:
         \keys_set:nn { scratch } { #1 }
         % Main code for \MyModuleMacro
         \lorem\par
         \lorem\par
       \group_end:
     }
 \ExplSyntaxOff   
 \MyModuleSetup{parindent=1em, parskip=1pt}
 \MyModuleMacro [parindent=10pt, parskip=10pt]
 \end{texexample}
 
 
 The definition of the keys was achieved using the command:
 
\begin{docCommand}{keys_define:nn}{\marg{module}\marg{keyval list}}
The command parses the \meta{keyval list} and defines the keys associated there for \meta{module}. 
\end{docCommand}

The \meta{keyval list} should consist of one or more key names along with an associated
key \emph{property}. The properties of a key determine how it acts. The individual properties
are described in the following text; Note that the properties of the key begin from the dot (|.|) after the key name. The various properties available take no arguments or require one or more. All key definitions are local. 
 
 \begin{margoptionslist}
 \item [ .code:n] Stores the \meta{code} for execution when \meta{key} is used. 
 \item [.default:n] \meta{key} |.default:n| = \meta{default} This creates a \meta{default} value for \meta{key} if no value is given. This will be used if only the key name is given, but not if a blank \meta{value} is given. This behaviour is similar to the |pgfkeys| package.
 \item [.initial:n] \meta {key} |.initial:n| = \meta{value} Initialises the \meta{key} with the \meta{value}, equivalent to
|\keys_set:nn| \meta{module} \meta{key} = \meta{value}
 
 \item [.dim_set:N] \meta{key} |.dim_set:N| = \meta{dimension} Defines \meta{key} to set \meta{dimension} to \meta{value} (which must a dimension expression). If the variable does not exist, it will be created globally at the point that the key is set up.
 \end{margoptionslist}
 
%  \begin{texexample}{Key value}{ex:keyval1}
% \ExplSyntaxOn
% \dim_new:N \l_parskip
% \dim_new:N \l_parindent
% \keys_define:nn {scratch}
%   {
%      parindent .dim_set:N = \l_parindent,
%     % parindent .initial:n = 0pt,
%      parskip     .dim:n = \l_parskip,
%      %parskip     .initial:n = 1pt,
%      
%   }
%   
%\DeclareDocumentCommand \MyModuleSetup { m }
%     { \keys_set:nn { scratch } { #1 }  }
%     
%\DeclareDocumentCommand \MyModuleMacro { o }
%     {
%       \group_begin:
%         \dim_set_eq:NN \parindent \l_parindent
%         \dim_set_equal:NN \parskip \l_parskip
%         \keys_set:nn { scratch } { #1 }
%         % Main code for \MyModuleMacro
%         \lorem\par
%         \lorem\par
%       \group_end:
%     }
% \ExplSyntaxOff   
% 
% \MyModuleMacro [parindent=10pt, parskip=10pt]
% \end{texexample}

 
 \subsection{Choice keys}
 
 One of the most powerful features of modern key value packages is the ability to define and set keys for mutally exclusive values. In the |l3keys| module this can be achieved using the choice key.
 
 \begin{margoptionslist}
 \item [.choice] \meta{key} |.choice| This sets \meta{key} to act as a choice key. Each choice is then created, as discussed below:
 \end{margoptionslist}
 
 
 \begin{texexample}{Some choices}{}
 \ExplSyntaxOn
 \keys_define:nn { scratchi }
 {
    mycolor .choice:,
    mycolor/fire .code:n = {\color{red}},
    mycolor/sky .code:n = {\color{blue}},
    mycolor/orange .code:n = {\color{orange}},
    mycolor/lemon .code:n = {\color{yellow}},
    mycolor/grass .code:n = {\color{green}},
    mycolor .initial:n =sky,
    mycolor .default:n=orange,
    unknown .code:n={\color{red} ERROR},
 }

\keys_set:nn { scratchi } { mycolor=fire }  

\DeclareDocumentCommand \MyModuleSetup { m }
     { \keys_set:nn { scratchi } { #1 }  }

\DeclareDocumentCommand \MyModuleMacro { o +m}
     {
       \group_begin:
         \keys_set:nn { scratchi } { #1 }
         #2
         \group_end:
     }
     
 \ExplSyntaxOff
    
 \MyModuleSetup{mycolor=fire}

 \MyModuleMacro [mycolor=grass]{grass,} 
 
 \MyModuleMacro [mycolor]{default}
 
 \MyModuleMacro [apple]{}
 
 \MyModuleMacro [fire]{Fire}
\end{texexample} 
 
The |.choice|  key is a bit different from how it is used in the |xtemplate| package and |pgf| but probably easier to use and define. Of course our example was trivial and the colors should have been achieved with just one code key, capturing the value. It takes some practice to get used to all the types of keys available and to develop error free code easily, but by using a key value system, truly flexible, modern functions can be developed.
 

\subsection{Handling of unknown keys}
 
 Handling of unknown keys is similar to |pgf| where a key defined as |.unknown| is defined. 
 If a key has not previously been defined (is unknown), |\keys_set:nn| will look for a special
unknown key for the same module, and if this is not defined raises an error indicating that
the key name was unknown. This mechanism can be used for example to issue custom
error texts.

\begin{verbatim}
\keys_define:nn { mymodule }
{
unknown .code:n =
You~tried~to~set~key~’\l_keys_key_tl’~to~’#1’.
}
\end{verbatim}
 
 
 As for |pgf| there are many other key types and these are listed in the |l3keys| manual and are not listed here for brevity. 
 
 
\chapter{LaTeX3 properties}


 \LaTeX3 implements a \enquote{property list} data type, which contain
 an \emph{unordered list} of entries each of which consists of a \meta{key} and
 an associated \meta{value}. The \meta{key} and \meta{value} may both be
 any \meta{balanced text}. It is possible to map functions to property lists
 such that the function is applied to every key--value pair within
 the list.

 Each entry in a property list must have a unique \meta{key}: if an entry is
 added to a property list which already contains the \meta{key} then the new
 entry will overwrite the existing one. The \meta{keys} are compared on a
 string basis, using the same method as \docAuxCommand*{str_if_eq:nn}.

 Property lists are intended for storing key-based information for use within
 code.  This is in contrast to key--value lists, which are a form of
 \emph{input} parsed by the \pkg{l3keys} module.

 \section{Creating and initialising property lists}

 \begin{docCommand}{prop_new:N or :c}{\meta{property list}}
   Creates a new \meta{property list} or raises an error if the name is
   already taken. The declaration is global. The \meta{property list} will
   initially contain no entries.
 \end{docCommand}

 \begin{docCommand}{prop_clear:N (:n:c) }{ \meta{property list}}
   Clears all entries from the \meta{property list}.
 \end{docCommand}

 \section{Adding entries to property lists}

 \begin{docCommand}{prop_put:Nnn}{ \meta{property list} \marg{key} \marg{value}}
 
   Adds an entry to the \meta{property list} which may be accessed
   using the \meta{key} and which has \meta{value}. Both the \meta{key}
   and \meta{value} may contain any \meta{balanced text}. The \meta{key}
   is stored after processing with \docAuxCommand*{tl_to_str:n}, meaning that
   category codes are ignored. If the \meta{key} is already present
   in the \meta{property list}, the existing entry is overwritten
   by the new \meta{value}.
 \end{docCommand}
 
 \begin{texexample}{Property lists}{ex:proplits}
 \ExplSyntaxOn
 \prop_new:N \g_tut_temp_prop
 \prop_gput:Nnn \g_tut_temp_prop {symbolic}{true}
 \prop_item:Nn \g_tut_temp_prop {symbolic}
 \ExplSyntaxOff
\end{texexample}

What happened here is we create the property list, using \docAuxCommand{tut_temp_prop}, adding a key and value and the using it through \docAuxCommand {tut_temp_prop}.

 \section{Recovering values from property lists}

   \begin{docCommand}{prop_get:NnN}{ \meta{property list} \marg{key} \meta{tl var}}
   Recovers the \meta{value} stored with \meta{key} from the
   \meta{property list}, and places this in the \meta{token list
   variable}. If the \meta{key} is not found in the
   \meta{property list} then the \meta{token list variable} will
   contain the special marker \docAuxCommand*{q\_no\_value}. The \meta{token list
     variable} is set within the current \TeX{} group. See also
   \docAuxCommand*{prop_get:NnNTF}.
  \end{docCommand}

Consider another example, this time we use a property lists to store information
on scripts and languages. We will use a new property list, name |l_<lang name>_prop| to store the
information.


\begin{texexample}[fontlower=\arial,]{Property Lists}{ex:proplists2}
\ExplSyntaxOn
% create new property list
\prop_new:c    {l_armenian_prop}

% save some properties
\prop_gput:cnn  {l_armenian_prop} {name     } {Armenian}
\prop_gput:cnn  {l_armenian_prop} {wfonts   } {NotoArmenian-Regular.ttf, Arial}
\prop_gput:cnn  {l_armenian_prop} {lfonts   } {NotoArmenian-Regular.ttf, Arial}
\prop_gput:cnn  {l_armenian_prop} {afonts   } {NotoArmenian-Regular.ttf, Arial}
\prop_gput:cnn  {l_armenian_prop} {group    } {Europe}
\prop_gput:cnn  {l_armenian_prop} {direction} {TLT}
\prop_gput:cnn  {l_armenian_prop} {hyphen   } {none}
\prop_gput:cnn  {l_armenian_prop} {sample   } {sample_armenian}

% get the group property and save it in a temporary variable  
\prop_get:cnN   {l_armenian_prop} {group   } \l_tmpa_tl
\tl_use:N \l_tmpa_tl
\ExplSyntaxOff
\end{texexample}

Using a property list is a much neater method than using lower level commands such as
|\csname| to store information. Property lists are used in the two largest \pkg{expl3} packages, \pkg{siunitx} and \pkg{fontspec}. As we will also see in the exampels that 
follow they can be used to automate the generation of code.


\section{Property list conditionals}








\chapter{Sequence lists}

\epigraph{``Where did we get that (equation) from? Nowhere. It is not possible to derive it from anything you know. It came out of the mind of Schrödinger.''}{---Richard Feynman}

One very useful data type, which is incorporated in \latex3 is the ``sequence'' data type. This contains a list of entries which may contain any \meta{balanced text}. One of the most powerful features of lists is that t is possible to map functions to sequences such that the function is applied to every item in the sequence.

Sequences are also used to implement stack functions in \latex3. This is achieved using a number of dedicated stack functions.

\section{Creating sequences}

Like most of the modules new sequences are created using the prefix for the module and the word ``new''.

\begin{docCommand}{seq_new:N}{}
Creates a new \meta{sequence} or raises an error if the name is already taken. The declaration
is global. The \meta{sequence} will initially contain no items.
\end{docCommand}

First let us create and examine the meaning of a simple example of the use of sequences. 

\begin{texexample}{Creating sequences}{ex:seq}
\ExplSyntaxOn
\seq_new:N \g_scratch_seq 
\token_to_meaning:N \g_scratch_seq
\ExplSyntaxOff
\end{texexample}

Examining the meaning of the sequence we created with \refCom{seq_new:N} we observe that there is no magic involved, it is just another macro that holds two others. So let us add some material and see what happens next.

\begin{texexample}{Creating sequences}{ex:seq}
\ExplSyntaxOn
\gdef\tempa {AAA}
\gdef\tempb {BBB}

% Add some material left and right
\seq_gput_left:Nn \g_scratch_seq \tempa
\seq_gput_right:Nn \g_scratch_seq \tempb

% examine the meaning of the \scratch_seq:N
% and the marker at the beginning
\token_to_meaning:N \g_scratch_seq\\
\token_to_meaning:N \s__seq\\
\token_to_meaning:N \s__seq_item:n
\ExplSyntaxOff
\end{texexample}

We have used two more functions that by now you are familiar to put material both at the left and at the right of the sequence, and again examined its meaning. We also examined the meaning of |s__seq| which is the internal command at the start of the list. The concept is very similar to |\@elt| lists

\begin{texexample}{Creating sequences}{ex:seq}
\ExplSyntaxOn
% examine the meaning of the \g_scratch_seq:N
% and the marker at the beginning again
\token_to_meaning:N \g_scratch_seq\\
\token_to_meaning:N \s__seq\\
\token_to_meaning:N \s__seq_item:n\\

% pop the left of the sequence
% store in \l_tmpa
\seq_pop_left:NN \g_scratch_seq \l_tmpa_tl

% typeset contents of left cell
\tl_use:N \l_tmpa_tl

\ExplSyntaxOff
\end{texexample}



\begin{texexample}{Creating sequences}{ex:seq}
\ExplSyntaxOn
\def\urlctan   {\url{\http:ctan.org}}
\def\urlgithub {\url{http:github.org}}

% clearing the sequence
\seq_clear:N \g_scratch_seq 

\seq_gput_left:Nn \g_scratch_seq \urlctan
\seq_gput_left:Nn \g_scratch_seq \urlgithub
% typeset contents of left cell
% pop the left of the sequence
% store in \l_tmpa
\seq_gpop_left:NN \g_scratch_seq \l_tmpa_tl

% typeset contents of left cell
\tl_use:N \l_tmpa_tl

\ExplSyntaxOff
\end{texexample}

\begin{texexample}{An equation database}{ex:seq}
 % #1 name
 % #2 equation
 
\ExplSyntaxOn   
\cs_gset:Npn \addEquation #1#2
  {
    \cs_set:cpn {-#1} {{\bfseries#1}\begin{gather}#2\end{gather}}
    \seq_gput_right:Nn \g_scratch_seq {#1}
  }

\cs_gset:Npn \typesetEquations 
  {
    \seq_map_inline:Nn \g_scratch_seq 
      {
        \cs:w-##1\cs_end:
      }
  }
  
  
% clearing the sequence
\seq_clear:N \g_scratch_seq 

\ExplSyntaxOff

\addEquation {quadratic} 
  {
    ax^2 + bx + c =0
  }
\addEquation {linear}    
  {
    x = \frac{b}{a}
  }
\addEquation {cubic}    
  {
    x^3 + 2x^2 + 10x = 20
  }
    
\typesetEquations

\end{texexample}

By the way Leonardo de Pisa, also known as Fibonacci (1170–1250), was able to find the positive solution to the cubic equation \( x^3 + 2x^2 + 10x = 20\), using the Babylonian numerals. He gave the result as \(1,22,7,42,33,4,40\) (equivalent to \(1 + 22/60 + 7/602 + 42/603 + 33/604 + 4/605 + 40/606)\), which differs from the correct value by only about three trillionths.

But let us fill our little database with the quartic, quintic, sextic and septic functions,
so we can have a few more data in our sequence. Also I suggest you try and run some of the examples on your own to get used to the language, solving syntax errors for typos and the like.

%\tcbset{texexmp/.style={ 
%        colback = white,% background
%        colframe=white, 
%        %bottombox=ignored,   
%        listing options={
%          backgroundcolor=\color{white},
%          keywordstyle=\color{black},
%          breaklines=true,
%          breakatwhitespace=true,
%          commentstyle=\color{thelightgray},
%          emph={addEquation, typesetEquations},
%          emphstyle=\color{thegreen},
%         },
%        }}%

\emphasize{addEquation,typesetEquations}
\begin{texexample}{An equation database}{ex:seq}
% add equations to db
\addEquation {quartic} 
  {
    f(x)=ax^4+bx^3+cx^2+dx+e
  }
  
\addEquation {quintic}    
  {
    g(x)=ax^5+bx^4+cx^3+dx^2+ex+f
  }
  
\addEquation {sextic}    
  {
    ax^6+bx^5+cx^4+dx^3+ex^2+fx+g=0
  }

\addEquation {septic}    
  {
    ax^7+bx^6+cx^5+dx^4+ex^3+fx^2+gx+h=0
  } 
  
\addEquation {BBGKY}
  {\scriptstyle  
   \frac{\partial f_s}{\partial t} + \sum_{i=1}^s \dot{\mathbf{q}}_i \frac{\partial f_s}{\partial \mathbf{q}_i} + \sum_{i=1}^s \left( - \frac{\partial \Phi_i^{ext}}{\partial \mathbf{q}_i} - \sum_{j=1}^s \frac{\partial \Phi_{ij}}{\partial \mathbf{q}_i} \right) \frac{\partial f_s}{\partial \mathbf{p}_i} = (N-s) \sum_{i=1}^s \frac{\partial}{\partial \mathbf{p}_i} \int \frac{\partial \Phi_{is+1}}{\partial \mathbf{q}_i}\cdot f_{s+1} \,d\mathbf{q}_{s+1} d\mathbf{p}_{s+1}.
  }
  
 \addEquation {Borda-Carnot}
  {
    \Delta E\, =\, \frac12\, \rho\, \left( v_3\, -\, v_2 \right)^2\,
           =\, \frac12\, \rho\, \left( \frac{1}{\mu}\, -\, 1 \right)^2\, v_2^2\,
           =\, \frac12\, \rho\, \left( \frac{1}{\mu}\, -\, 1 \right)^2\, \left( \frac{A_1}{A_2} \right)^2\, v_1^2.} %(*@\label{borda}@*)

% typeset all equations in db                      
\typesetEquations 

\end{texexample}

If you observe the last example we have hit a small problem, we had to reduce the size of the display
equation to fit it in. We would have been better to have displayed the equation in a \docAuxEnvironment{multline}
or a \docAuxEnv{brqew environment}, as shown below. 



\begin{multline}
\frac{\partial f_s}{\partial t} + \sum_{i=1}^s \dot{\mathbf{q}}_i \frac{\partial f_s}{\partial \mathbf{q}_i} + \sum_{i=1}^s \left( - \frac{\partial \Phi_i^{ext}}{\partial \mathbf{q}_i} - \sum_{j=1}^s \frac{\partial \Phi_{ij}}{\partial \mathbf{q}_i} \right) \frac{\partial f_s}{\partial \mathbf{p}_i} =\\
 (N-s) \sum_{i=1}^s \frac{\partial}{\partial \mathbf{p}_i} \int \frac{\partial \Phi_{is+1}}{\partial \mathbf{q}_i}\cdot f_{s+1} \,d\mathbf{q}_{s+1} d\mathbf{p}_{s+1}.
\end{multline}

Recall how we defined |\addEquation|,

\begin{teXXX}
\cs_gset:Npn \addEquation #1#2{
  \expandafter\gdef\csname-#1\endcsname 
    {
      {\bfseries#1}\begin{gather}#2\end{gather}
    }
  \seq_gput_right:Nn \g_scratch_seq {#1}
 }
\end{teXXX}

We can change the function to accept an optional argument and a starred or unstarred version to allow the user to add a field to the input that can determine the output. This is fairly easy with \pkgname{xparse}.


\begin{texexample}{An equation database}{ex:seq}
\addEquation {quartic} 
  {
    f(x)=ax^4+bx^3+cx^2+dx+e
  }
\addEquation {quintic}    
  {
    g(x)=ax^5+bx^4+cx^3+dx^2+ex+f
  }
\addEquation {sextic}    
  {
    ax^6+bx^5+cx^4+dx^3+ex^2+fx+g=0
  }

\addEquation {septic}    
  {
    ax^7+bx^6+cx^5+dx^4+ex^3+fx^2+gx+h=0
  } 
\addEquation {BBGKY}
  {\scriptstyle  
   \frac{\partial f_s}{\partial t} + \sum_{i=1}^s \dot{\mathbf{q}}_i \frac{\partial f_s}{\partial \mathbf{q}_i} + \sum_{i=1}^s \left( - \frac{\partial \Phi_i^{ext}}{\partial \mathbf{q}_i} - \sum_{j=1}^s \frac{\partial \Phi_{ij}}{\partial \mathbf{q}_i} \right) \frac{\partial f_s}{\partial \mathbf{p}_i} = (N-s) \sum_{i=1}^s \frac{\partial}{\partial \mathbf{p}_i} \int \frac{\partial \Phi_{is+1}}{\partial \mathbf{q}_i}\cdot f_{s+1} \,d\mathbf{q}_{s+1} d\mathbf{p}_{s+1}.
  }
 \addEquation {Borda-Carnot}
  {
    \Delta E\, =\, \frac12\, \rho\, \left( v_3\, -\, v_2 \right)^2\,
           =\, \frac12\, \rho\, \left( \frac{1}{\mu}\, -\, 1 \right)^2\, v_2^2\,
           =\, \frac12\, \rho\, \left( \frac{1}{\mu}\, -\, 1 \right)^2\, \left( \frac{A_1}{A_2} \right)^2\, v_1^2.}
\typesetEquations 
\end{texexample}





\begin{texexample}{Sequence}{ex:sequence}
\ExplSyntaxOn
\def\exception{}
\NewDocumentCommand\SplitDemo { +m m } 
  {
    \my_seq_split:nn { #1 }{#2}
  }

\tl_new:N \l_first_word_tl

\cs_new_protected:Npn \my_seq_split:nn #1 #2
  { 
    
    \seq_set_split:Nnn \l_tmpa_seq { #2 } { #1 }
    \seq_use:Nn   \l_tmpa_seq {\par}
    \seq_get_left:NN \l_tmpa_seq \l_first_word_tl
    %\textcolor{blue} { \tl_use:N \l_first_word_tl  }
  }
      
\ExplSyntaxOff

\SplitDemo { This is one sentence. 
             This is a second one. 
             This is the third sentence. }{ . }\par
\SplitDemo { The \exception{A.B.C.} corporation. Another sentence. }{ . }
\SplitDemo { The \exception{A.B.C.} corporation. Another sentence. }{~}
\end{texexample}


\paragraph{Capitalization and l3} \latex3 provides capitalization related functions in three modules: str, tl and token. Each module handles different cases. These are still listed under l3candidates and have not as yet been moved into the main modules. We normally have three cases for capitalization, changing a word to all capitals, lowercase or the first letter is capitalized and the rest are lowercased. In |l3| this is termed mixed case (my preference is naming it ucfirst), as mixed case should also include camel case, such as |CamelCase|. In example~\ref{ex:capitalization}, we use the three available functions to see how they are working.

\begin{texexample}{Change Lowercase Letters to Capitals}{ex:capitalization}
\ExplSyntaxOn
\tl_set:Nn \l_tmpa_tl { Hello~WORLD}
Input:~ \tl_use:N \l_tmpa_tl\par

Uppercase:~ \tl_upper_case:n { \l_tmpa_tl }\par

Mixedcase:~ \tl_mixed_case:n {\l_tmpa_tl}\par

Lowercase:~ \tl_lower_case:n {\l_tmpa_tl}\par
\ExplSyntaxOff
\end{texexample}




In the next example we will consider a difficult problem for machines, but an easy problem for humans, the capitalization rules for titles.words that are not normally capitalized in a title.


\begin{texexample}{Sequence}{ex:sequence}
\ExplSyntaxOn
\clist_gset:Nn \title_words_not_capitalized_en 
 {a, an, the, at, by, for, in, of, on, to, up, and, as, but, it, or, nor, do, for, this, be, A, An, The, At, By, For, In, Of, On, To, Up, And, As, But, It, Or, Nor, Do, For, This, Be,abaft, aboard, about, above, absent, across, afore, after, against, along, alongside, amid, amidst, among, amongst, an, anenst, apropos, apud, around, as, aside, astride, at, athwart, atop, barring, before, behind, below, beneath, beside, besides, between, beyond, but, by, circa, concerning, despite, down, during, except, excluding, failing, following, for, forenenst, from, given, in, including, inside, into, lest, like, mid, midst, minus, modulo, near, next, notwithstanding, of, off, on, onto, opposite, out, outside, over, pace, past,  per, plus, pro, qua, regarding, round, sans, save, since, than, through, throughout, till, times, to, toward, towards, under, underneath, unlike, until, unto, up, upon, Versus, versus, via, vice, with, within, without, worth
}

\clist_gset:Nn \abbreviations 
 {
  A.B.C.,iTunes
 }

\clist_gset:Nn \acronyms
  {
    NATO,UN,US
  }  						
    
\cs_new:Npn \ucfirst_aux:w #1#2 \q_stop { \tl_upper_case:n { #1 } #2 }

\cs_new:Npn \ucfirst #1 
	{
		\exp_after:wN \ucfirst_aux:w #1 \q_stop
	}

\cs_new:Npn \lowerfirst #1 
	{
		\tl_lower_case:n {#1}
 	}

\NewDocumentCommand\UppercaseTitle {s +m }
  {
	  \IfBooleanTF { #1 } { {\bfseries {#2} } }
      {     
       	\tex_hyphenpenalty:D = 10000
       	  % split on space
	       \seq_set_split:Nnn \g_tmpa_seq {~} {#2}
	       
	       \seq_use:Nn   \g_tmpa_seq {~}\\
	       
	       % pop the left into a temporary token list
	       % the first letter must always be capitalized 
	       \seq_pop_left:NN \g_tmpa_seq \l_tmpa_tl  
	       
	       % typeset the first letter
	       {\bfseries\ucfirst \l_tmpa_tl \space} 
	       
	       % map it in line
	       \seq_map_inline:Nn \g_tmpa_seq 
	       	{
	         	\clist_if_in:NnTF \title_words_not_capitalized_en { ##1 }
	           { {\bfseries \lowerfirst {##1}~}} { {\bfseries \ucfirst{##1}~ } }    
	            
	         }            
      }    
	} 
	
\ExplSyntaxOff    

 \UppercaseTitle {Top ten things To do in Paris}\\
 \UppercaseTitle {How to use LaTeX sequence lists effectively}\\
 \UppercaseTitle {Senate Votes to Confirm Elena Kagan For U.S. Supreme Court}\\
 \UppercaseTitle {what would be a ``correct'' capitalization for the title of this question?}\\
 \UppercaseTitle* {How about {$e=mc^2$}? }\\
\end{texexample}

The code we just wrote suffers from a lot of deficiencies.


\ExplSyntaxOn

\NewDocumentCommand\UppercaseTitle {s +m }
    {
      \IfBooleanTF { #1 } { {\bfseries {#2} } }
        {     \tex_hyphenpenalty:D = 10000
	        \seq_set_split:Nnn \g_tmpa_seq {~} {#2}
	        \seq_use:Nn   \g_tmpa_seq {~}\\
	        
	        \seq_pop_left:NN \g_tmpa_seq \l_tmpa_tl  
	       
	        {\bfseries\ucfirst \l_tmpa_tl \space} 
	        \seq_map_inline:Nn \g_tmpa_seq 
	           {
	              \clist_if_in:NnTF \title_words_not_capitalized_en { ##1 }
	              { {\bfseries \lowerfirst {##1}~}} { {\bfseries \ucfirst{##1}~ } }    
	            
	           }            
	       
       }    
    } 
\ExplSyntaxOff   
The rules for capitalization of titles varies from publication to publication and from department to department. A look at \href{http://arxiv.org/pdf/1505.04095v1.pdf}{arxiv} yielded a number of papers that do not follow the above rules. This will remain an unsolved problem, but at least we have moved forward. 

\begin{texexample}{Uppercase Title Issues}{}
\UppercaseTitle {Measuring Political Polarization: Twitter shows the two sides of Venezuela}\\
\UppercaseTitle {The Directed Dominating Set Problem: Generalized Leaf Removal and Belief Propagation}\\
\UppercaseTitle {Cities through the Prism of People's Spending Behavior}\\
\UppercaseTitle {On the p-th root of a p-adic number}\\
\UppercaseTitle {Planetary Formation Scenarios Revistied: Core-Accretion Versus Disk Instability}\\
\UppercaseTitle {de Haas-van Alphen effect versus Integer Quantum Hall effect}\\
\UppercaseTitle {A Simple Desultory Philippic (or How I Was Robert McNamara'd into Submission)}
\end{texexample}


The first title, shows a rule in many style manuals that words more than five characters should be capitalized, a rule broken by the third in the list above, although it is a preposition and many style books dictate that all prepositions be lowercase. It would make sense to add prepositions to our list. 

The last example is the Dutch and Afrikaans preposition \emph{de} meaning  \enquote{of} or \enquote{from}. This would make an exception on the first word of the sentence but not the last. The prefix von is not capitalised in German-speaking countries. The Duden dictionary recommends capitalizing the prefix von at the beginning of the sentence, but not in its abbreviated form, in order to avoid confusion with an abbreviated first name: \enquote{Von Humboldt kam später.} and \enquote{v. Humboldt kam später.} (Von Humboldt came later.) The Swiss Neue Zürcher Zeitung, however, recommends omitting the von completely at the beginning of the sentence: \enquote{Humboldt kam später.}


\begin{description}
\item [First and last words] These are always capitalized. There is a general agreement for this one by all guides and editors, however there are exceptions.

\item [Prepositions] Do not capitalize English prepositions in the body of the title, but capitalize them if they are the first word.

\item [Foreign language prepositions] These have their own rules and are discussed later.
\end{description}

Let us now try and improve our code. In Example~\ref{ex:sequence}, we ensured that the first word is always capitalized by popping out the first word from the list and capitalizing it by using:

\begin{teXXX}
\seq_pop_left:NN \g_tmpa_seq \l_tmpa_tl 
\end{teXXX}

The first suggestion that comes to mind is to change is to add a pop left operation and perhaps to add both  to an auxiliary function so that we can later on add exceptions for foreign language names, as desribed in the specification earlier.

\begin{texexample}{Renew \textbackslash UppercaseTitle}{}
\ExplSyntaxOn
%\cs_new:Npn \ucfirst_aux:w #1#2 \q_stop { \tl_upper_case:n { #1 } #2 }
%
%\cs_new:Npn \ucfirst #1 {
%     \exp_after:wN \ucfirst_aux:w #1 \q_stop
%}
%
%\cs_new:Npn \lowerfirst #1 {
%       \tl_lower_case:n {#1}
% }

% Main command
\RenewDocumentCommand\UppercaseTitle {s +m }
    {
      \IfBooleanTF { #1 } { {\bfseries {#2} } }
        {     \tex_hyphenpenalty:D = 10000
	        \seq_set_split:Nnn \g_tmpa_seq {~} {#2}

	        % type splitted sequence	        
	        \seq_use:Nn   \g_tmpa_seq {~}\\
	        
	        % 
	        \pop_first:N  \g_tmpa_seq
	        \seq_pop_right:NN  \g_tmpa_seq \l_tmpb_tl
	        %
	        \seq_map_inline:Nn \g_tmpa_seq 
	           {
	              \clist_if_in:NnTF \title_words_not_capitalized_en { ##1 }
	              { {\bfseries \lowerfirst {##1}~}} { {\bfseries \ucfirst{##1}~ } }    
	            
	           }            
	    {{ \bfseries \ucfirst{\l_tmpb_tl} }}
       }    
    } 

 % Function to pop the first item and decorate it    
\cs_new_nopar:Npn \pop_first:N #1 {
 	        \seq_pop_left:NN \g_tmpa_seq\l_tmpa_tl
	        {\bfseries\ucfirst \l_tmpa_tl \space} 
  }
  
 % Function to pop last word and decorate it 
\cs_new_nopar:Npn \pop_last:N #1 {
	        \seq_pop_right:NN \g_tmpa_seq \l_tmpa_tl
	        {\bfseries\ucfirst  \l_tmpa_tl} 
}        
        
\ExplSyntaxOff    
\UppercaseTitle {What to do with Versus~}\\
\UppercaseTitle {What to do with versus?~}\\
\UppercaseTitle {What to do with versus: Versus or versus~?~}\\
\end{texexample}

What just happened is that we have also created two new auxiliaries one to pop the first word and another to pop the last word. We are now closer to a final solution, but the decoration of the words, needs to be taken care of as well. These are always better to be functions of their own and we can do it quite easily. By decoration we mean adding fonts colors and the like. We do not consider capitalization as decoration. 


\begin{question}
\begin{tasks}(1)
\task Develop a function to detect if a token is composed of all capital letters.
\task Develop a boolean \cs{is_uppecase:nTF} that can return true or false if the text is uppercase.
\task Develop a boolean \cs{is_lowercase:nTF} that can return true or false if the text is all lowercase.
\task Develop a boolean \cs{is_mixedcase:nTF} that can return true or false if the text consists of upper first and then lowercase.
\end{tasks}
\end{question}

\begin{texexample}{Check if uppercase}{ex:seqif}
\ExplSyntaxOn
% create a new sequence
\seq_new:N \alphabet_en_seq

% split the sequence at empty spaces, use gset to pick in the next
% example as well
\seq_gset_split:Nnn \alphabet_en_seq {~} {A~B~C~D~E~F~G~H~I~J~K~L~M~N~O~P~Q~R~S~T~U~V~W~X~Y~Z} 

% create a predicate to check if the letter is uppercase
\prg_new_protected_conditional:Npnn \is_uppercase:n #1  {TF, T, F}{
   \seq_if_in:NnTF \alphabet_en_seq 
   {#1} 
   {\prg_return_true:}
   {\prg_return_false:}
}

\prg_generate_conditional_variant:Nnn \is_uppercase:n {o}{TF, T, F}


Z~is~uppercase~ \is_uppercase:nTF {Z} {\TRUE}{\FALSE}\\

v~is~not~uppercase \is_uppercase:nTF {v} {\TRUE}{\FALSE}
\ExplSyntaxOff 
\end{texexample}

Now that we have the basic definitions, what we have to do is iterate through all the letters of a single
token and carry out the tasks of determining if a string is fully capitalized. Fully capitalized strings,
will be assumed to be acronyms or specific words that the user wishes to have fully capitalized and left in the title as is. This will obviously fail if the title is fully capitalized and we wish to change it to one of the other cases, so first we need to determine that we have a title that is composed of mixed cases.

\begin{texexample}{}{}
\ExplSyntaxOn
% check we have not lost the sequence we defined earlier
%\seq_set_split:Nnn \alphabet_en_seq {~} {A~B~C~D~E~F~G~H~I~J~K~L~M~N~O~P~Q~R~S~T~U~V~W~X~Y~Z} 
\seq_use:Nn \alphabet_en_seq {,}
\ExplSyntaxOff
\end{texexample}

We assume that the title will be split into words and we need to determine if a word is fully
capitalized:

\begin{texexample}{Determine if a word consists of all capital letters}{ex:allcaps}
\ExplSyntaxOn
\tl_gset:Nn \l_my_tl {ALLCAPsa}

\tl_map_inline:Nn \l_my_tl
{
% Do something useful
  \is_uppercase:nTF {#1}{#1~\TRUE}{\FALSE \tl_map_break:}
}
\ExplSyntaxOff
\end{texexample}

We are now on thr right track. We map the word letter by letter and break out at the first occurence of
a lowercase letter. Notice the \cs{l_my_tl} stores |ALLCAPSsa| we broke out using |tl_map_break:| and never typeset the letter. All is left is to define a number of predicates using the macros we have just developed.

\begin{texexample}{Determine the case of a word}{ex:allcaps2}
% ... continued
\ExplSyntaxOn
% create a predicate to check if the letter is uppercase
\bool_new:N \word_allcaps_bool
\bool_new:N \word_mixed_bool
\bool_new:N \word_alllower_bool

\prg_new_protected_conditional:Npnn \if_is_word_uppercase:n #1  {TF, T, F}
  {  
    % Reset all the booleans
     \bool_gset_false:N \word_allcaps_bool
     \bool_gset_false:N \word_alllower_bool
     \bool_gset_false:N \word_mixed_bool
     
      \tl_map_inline:nn {#1}
      {
        \is_uppercase:nTF {##1}
           {\bool_gset_true:N \word_allcaps_bool}
           {
              \bool_gset_true:N \word_alllower_bool
           }
      }
      
    \bool_if:NTF\word_allcaps_bool
       { 
         % Check for mixed case
         \bool_if:NTF \word_alllower_bool
           {
             % If a word has activated all caps 
             % it can still conatin lowercase letters
             % in this case set the mixed case true
             \bool_gset_true:N \word_mixed_bool
             \prg_return_false:
           }
           {\prg_return_true:}
       }
       {
         \prg_return_false:
       }
   }


% Tests true
\if_is_word_uppercase:nTF {ALL}{ALL \TRUE}{\FALSE}\\

% Tests false
\if_is_word_uppercase:nTF {All}{\TRUE}{All \FALSE}\\

% Since the last test is a mixed case check that the boolean is set
\bool_if:NTF\word_mixed_bool {\TRUE}{All \FALSE}\\

% One more test
\if_is_word_uppercase:nTF {N-time}{ALL \TRUE}{\FALSE}\\
\bool_if:NTF\word_mixed_bool {\TRUE}{All \FALSE}

\ExplSyntaxOff
\end{texexample}

There are many more tests that one should include in a comprehensive package, with many more tests for edge cases. For example what to do after punctuation etc. One day this partial code will become a proper
package and integrated into the phd-lowerlevelheadings package.

\subsection{Final approach}

Given the rules above, some words can have three different ways of capitalization depending on their position in the sentence i.e, first, middle or end.

I posted some of the above code on |TEX.SX| and I had some amazing response from two of the developers of |expl3|. Under development there is a version of a |\title_case:n| command, which follows more or less the approach described in the example above.  I  have included the example in the chapter to demonstrate some of the aproaches to programming.

\robustify\url
\robustify\href
\robustify\textbf
\ExplSyntaxOn
\DeclareDocumentCommand \arxiv {g g}
{
  \IfNoValueTF {#1} {\href{http://arxiv.org}{{\color{blue}arxiv}}\xspace}
 {\href{http://arxiv.org/#1}{{\color{blue}#2}}\xspace}
}
\ExplSyntaxOff

\paragraph{The exceptions to the rules}
The article \arxiv{abs/1505.05148}{ALMA maps the Star-Forming Regions in a Dense Gas Disk at z\char`\~3 } has also a problematic title.  Here the first word is an acronym and is left as is,  and the last word is an abbreviation also left as is. Exclusion list, abbreviation lists perhaps need to be build over time similar to hyphenation lists.  Another paper
\arxiv{abs/1505.05156}{Statistics of Measuring Neutron Star Radii: The Bayesian vs. The Frequentist Approach} shows some of the problems with abbreviations, such as \enquote{vs.}, so filtering through a list of exclusions before capitalization is unavoidable. 
\acrodef{QGP}{quark-gluon plasma}

The title  ``\arxiv{abs/1505.04994}{Viscosities of Gluon Dominated QGP Model within Relativistic Non-Abelian Hydrodynamics}'', will only be capitalized properly, except for the  \ac{QGP} acronym, which must be present in an exclusion list.

\paragraph{Problems with NLP}
Some of the problems described in rationalizing an algorithmic approach fall in what is desribed as Natural Language Processing and such problems are not easy to solve. A solution that maybe can be satisfactory for
most cases, would involve at least a couple of passes and it will also assume that the author is somehow correct in typing the text, such a sleaving spaces after a stop. Using a second pass with regular expressions, it maybe possible to start filtering some of the abbreviations and acronyms, as well as do sentence detection.

Thanks to the \latex3 Team some problems taht would seem almost impossible to solve previously, at least now we can contemplate solution.

We will revisit the code in the chapter on Regular Expressions.

\begin{enumerate}
\item Write or ask for a specification. This can clarify requirements and avoid too many iterations of the code development. Have many examples of usage for testing.  Write tests for your code and always test against them. I understood the rules of title casing better from collecting titles from the \arxiv website.

\item Search for similar code and packages before you start developing.

\item Don't be afraid to ask the experts, most of the time they are more than willing to help.

\item Open source development is great. Consider contibuting to it. It is great for you and it is great for the community. The old concept of ``commons'' has more or less disappeared except in programming. Foster it and take care of it.
\end{enumerate}


\section{Summary}

In this chapter we reviewed the basics of the data type \enquote{Sequence lists} and have managed to produce some useful code in our final example. We have also reviewed some of the general concepts behind programming and have even managed to get two of the \latex3 developers to contribute code.

The code with some modifications is included in the \pkgname{phd} to provide title casing for headings and titles. The credit goes to Will Robertson and Joseph Wright. 

This chapter also brings us to the end of the list structures of |expl3|. Lua has its tables which are used to develop any data type and data structure required and similarly |expl3|'s lists can be used to develop and data structure you can imagine. One can think of link lists and tree data structures.

\endinput
A link list can easily be developed as it consists of elements that point to the next element only.  

\begin{verbatim}
\expandafter\def \csname link_item_1_next\endcsname {end list marker}
\end{verbatim}

At creation the link list item will expand to a marker. When the second item is added, the previous item will point to the second element and so on. The advantages of a link list is that if we want to delete an item or insert, we do not have to iterate through the whole list. Of course, if we had to delete it we could just simply mark it as undefined.

All the lists and parsing described in this book depend on one amazing fact, which is what \tex does when scanning the argument specification of a macro (between |\def\acommand | and either the opening bracket |{|. Leverage this fact as much as you can in your parsers. 

During mapping this could be detected and not used. \tex like any language has its own paradigms and we need not   follow other language patterns, but is good to know that we truly have a highly flexible and Turing-complete language available (even if it is a macro language). A macro language is still a language.



\chapter{LaTeX 3 clists module}

\epigraph{``I bet the human brain is a kludge.’’ }{---Marvin Minsky}

\precis{This chapter explores the expl3 comma delimited lists. It provides numerous working examples to demonstrate the use of the numerous available functions, provided by the module.}

\section{Introduction}

One of the most common data structure that computer languages provide are comma delimited lists.
 Comma lists contain ordered\footnote{Ordered does not mean sorted. It means they keep the order they were entered.} data where items can be added to the left
 or right end of the list. The resulting ordered list can then
 be mapped over using \docAuxCommand*{clist_map_function:NN}. Several items can
 be added at once, and spaces are removed from both sides of each item
 on input. Hence,
 \begin{verbatim}
   \clist_new:N \l_my_clist
   \clist_put_left:Nn \l_my_clist { ~ a ~ , ~ {b} ~ }
   \clist_put_right:Nn \l_my_clist { ~ { c ~ } , d }
 \end{verbatim}
 results in \docAuxCommand*{l_my_clist} containing |a,{b},{c~},d|.
 Comma lists cannot contain empty items, thus
 \begin{verbatim}
   \clist_clear_new:N \l_my_clist
   \clist_put_right:Nn \l_my_clist { , ~ , , }
   \clist_if_empty:NTF \l_my_clist { true } { false }
 \end{verbatim}
 will leave \texttt{true} in the input stream. To include an item
 which contains a comma, or starts or ends with a space,
 surround it with braces.  The sequence data type should be preferred
 to comma lists if items are to contain |{|, |}|, or |#| (assuming the
 usual \TeX{} category codes apply).

Implementation of list data structure normally provide the minimum following operations:

\begin{enumerate}
\item a constructor for creating an empty list;
\item an operation for testing whether or not a list is empty;
\item an operation for prepending an entity to a list
\item an operation for appending an entity to a list
\item an operation for determining the first component (or the "head") of a list
\item an operation for referring to the list consisting of all the components of a list except for its first (this is called the "tail" of the list.)
\end{enumerate}

 \section{Creating and initialising comma lists}

 \begin{docCommand}{clist_new:N}{ \meta{comma list}}
   Creates a new \meta{comma list} or raises an error if the name is
   already taken. The declaration is global. The \meta{comma list} will
   initially contain no items.
 \end{docCommand}


\begin{docCommand}{clist_const:Nn}{ \meta{clist~var} \marg{comma list}}
   Creates a new constant \meta{clist~var} or raises an error
   if the name is already taken. The value of the
   \meta{clist~var} will be set globally to the
   \meta{comma list}.
 \end{docCommand}

\begin{docCommand}{clist_clear:N}{ \meta{comma list}}
   Clears all items from the \meta{comma list}.
\end{docCommand}

 \section{Adding data to comma lists}

Adding data to a comma delimited list, is normally done through the use of helper functions and user commands.
If it is to be done once for example at the beginning of a document then it is a once off operation and we can use one of the \docAuxCommand*{clist_set} variants shown below. If the items are to be added programmatically or by the user in more than one place, then one of the functions \docAuxCommand*{clist_put_left} or \docAuxCommand*{clist_out_right} should be used. These prepend or append to the list, so goodbye \docAuxCommand*{@cdr}, \docAuxCommand*{@car} and their friends. 



% \begin{function}[added = 2011-09-06]
%   {
%     \clist_set:Nn,  \clist_set:NV,
%     \clist_set:No,  \clist_set:Nx,
%     \clist_set:cn,  \clist_set:cV,
%     \clist_set:co,  \clist_set:cx,
%     \clist_gset:Nn, \clist_gset:NV,
%     \clist_gset:No, \clist_gset:Nx,
%     \clist_gset:cn, \clist_gset:cV,
%     \clist_gset:co, \clist_gset:cx
%   }
  \begin{docCommand}{clist_set:Nn}{%
      \meta{comma list} \{
      \meta{item$_1$},
      \ldots,
      \meta{item$_n$} 
      \}
      }
   Sets \meta{comma list} to contain the \meta{items},
   removing any previous content from the variable.
   Spaces are removed from both sides of each item. Variants for : Nv,Nx,cV,cx,NV,cV exist.
 \end{docCommand}

\begin{texexample}{Creating a Comma delimited list}{ex:clists}
\ExplSyntaxOn
\clist_gset:Nn \phd_test_clist {one, two, three, four, five}
\phd_test_clist
\clist_gput_right:Nn \phd_test_clist{six, seven, eight}
\cs_new:Nn \nine: {9}
\clist_gput_right:Nn \phd_test_clist\nine:
\clist_gput_right:Nn \phd_test_clist\nine:
\par\phd_test_clist
\par\clist_if_in:NnTF\phd_test_clist {eight} {true} {false}
\ExplSyntaxOff 
\end{texexample}


%   \begin{syntax}
%     \docAuxCommand*{clist_put_left:Nn} \meta{comma list} \meta{item 1},\ldots{},\meta{item n}}
%   \end{syntax}
%   Appends the \meta{items} to the left of the \meta{comma list}.
%   Spaces are removed from both sides of each item.
% \end{function}
%
% \begin{function}[updated = 2011-09-05]
%   {
%     \clist_put_right:Nn,  \clist_put_right:NV,
%     \clist_put_right:No,  \clist_put_right:Nx,
%     \clist_put_right:cn,  \clist_put_right:cV,
%     \clist_put_right:co,  \clist_put_right:cx,
%     \clist_gput_right:Nn, \clist_gput_right:NV,
%     \clist_gput_right:No, \clist_gput_right:Nx,
%     \clist_gput_right:cn, \clist_gput_right:cV,
%     \clist_gput_right:co, \clist_gput_right:cx
%   }
 \begin{docCommand} {clist_put_right:Nn}  {\meta{comma list} \{\meta{item 1},\ldots{},\meta{item n}\}}
   Appends the \meta{items} to the right of the \meta{comma list}.
   Spaces are removed from both sides of each item.
\end{docCommand}

\begin{texexample}{Adding content to the list}{}
\ExplSyntaxOn
\clist_new:N \l_my_clist
\clist_put_right:Nn \l_my_clist{\square, \Diamond, \diamond, d, e, f}
\clist_put_left:Nn \l_my_clist{1,2,3,4,5,6,7,8,9,\hfill, 0}
\clist_use:Nn \l_my_clist{~}
\ExplSyntaxOff
\end{texexample}



\begin{texexample}{Adding content to the list}{}
\ExplSyntaxOn
\clist_put_right:Nn \l_my_clist {\alpha, \beta, \gamma, \delta, \epsilon}
\clist_put_left:Nn \l_my_clist {{\alpha\ldots}}
\[ \clist_use:Nnnn \l_my_clist {,} {,} {,} \]
\ExplSyntaxOff
\end{texexample}

\section{Mapping to comma lists}

\begin{texexample}{Mapping}{ex:longimages}
\ExplSyntaxOn
\clist_set:Nn \imgdb:n {fig145,fig161,fig162,fig163,fig164,fig165,fig166}
\clist_map_inline:Nn \imgdb:n {\includegraphics[width=1.5cm]{./images-01/#1}}
\ExplSyntaxOff
\end{texexample}

If the inline code is long, it might be preferable to use the function version of the map. This callback function should accept one parameter. Note that the mapping command format does not need the \#1 you only provide it with the function name.

\begin{texexample}{Mapping}{ex:longimages}
\ExplSyntaxOn
\cs_set:Npn \put_graphic:n #1 
   {
     \includegraphics[width=1.5cm]{./images-01/#1}
   }
\cs_set:Npn \put_graphic_with_space:n #1 
   {
     \put_graphic:n {#1}
     \hspace{5pt}
   }   
\clist_set:Nn \imgdb:n {fig145,fig161,fig162,fig163,fig164,fig165,fig166}

  \clist_map_function:NN \imgdb:n \put_graphic:n\par  
  \clist_map_function:NN \imgdb:n \put_graphic_with_space:n
\ExplSyntaxOff
\end{texexample}

The inline version is obviously a bit faster, as it does less work, but personally I prefer the callback style as it produces more readable code. Of course we could have used the |clist_if_in:NnTF| conditional. There are numerous conditionals and these are discussed later on.

The mapping function definitions are shown below,

\begin{docCommand}{clist_map_function:NN}{\meta{comma list} \meta{function}}
Applies a callback function to each item stored in the comma list. The function will receive one argument for each iteration. The items are returned from left to right. 
\end{docCommand}

\begin{docCommand}{clist_map_inline:NN}{\meta{comma list} \meta{inline function}}
Applies \meta{inline function} to every \meta{item} stored within the \meta{comma list}. The \meta{inline function} should consist of code which will receive the \meta{item} as \#1. One inline mapping can be nested inside another. The items are returned from left to right.
\end{docCommand}

There are is a third type of mapping function available where each entry in the list is passed to a variable which is then used in a function.

\begin{docCommand}{clist_map_variable:NNn}{\meta{comma list} \meta{tl. var} \marg{inline function}}
Stores each entry in the \meta{comma list} in turn in the \meta{tl var} and applies \meta{function} using \meta{tl var}. the function will usually consist of code making use of the \meta{t var}, but this is not enforced. One variable mapping can be nested inside another. the \meta{items} are returned from left to right.
\end{docCommand}

\subsection{Terminating mapping functions}

All lists in |expl3| can be terminated using a break function. A clist breaks by using the \docAuxCommand*{clist_map_break:n} function. 

\begin{texexample}{Mapping}{ex:longimages}
\ExplSyntaxOn
\cs_set:Npn \put_graphic:n # 1 
   { 
     \includegraphics[width=1.48cm]{./images-01/#1}
   }
\cs_set:Npn \put_graphic_with_space:n #1 
   {
      \parbox[b]{1.52cm}{\put_graphic:n {#1}\par\centering #1}
     \hspace{5pt}
   }   
   
\clist_set:Nn \imgdb:n {fig145,fig161,fig162,fig163,fig164,fig165,fig166}

 \clist_map_function:NN \imgdb:n \put_graphic_with_space:n\par
 \clist_map_inline:Nn \imgdb:n
     {
         \str_if_eq:nnTF {#1} {fig166}
         {\clist_map_break:n { \PASS~ \put_graphic:n {#1} ~#1} }
         {
          \FAIL #1
         }
     }
     
\ExplSyntaxOff
\end{texexample}

What just happened have added \docAuxCommand*{clist_map_inline:Nn} which iterates through all the elements in a list until a search string is found. As you can see as a search function it will be slow as it has to iterate through all the elements of a list. A more efficient way would have been to use \tex’s scanning mechanism of delimited functions to find the item. 

I have named the |clist| in the above examples as |imgdb| as one can easily extend the functions to store other information besides the filename. This can be done in many ways for example using the \meta{property} module of |expl3| or using |\csname|. In Chapter~\ref{ch:longfifgures} \nameref{ch:longfigures} we have used traditional techniques to typeset a lot of figures, in a similar fashion to a long table. Here we provide a similar example using |expliii|.

let us consider the simple case of a record for a person.

\begin{texexample}{Person record}{}
\ExplSyntaxOn
% create a new clist
\clist_new:N \personDB 

% auxiliary function to typeset an image
\cs_gset:Npn \put_graphic:n #1 
   {
     \includegraphics[height=3cm]{#1}
   }

% auxiliary function to enclose the image in a minipage      
\cs_gset:Npn \put_graphic_with_space:n #1 
   {
       \begin{minipage}[b]{3cm}
             \centering
             \put_graphic:n {#1}\par
             \csname#1_name\endcsname\\
             \csname#1_occupation\endcsname\\
      \end{minipage}\hspace{5pt}  
   }   

% helper function to add a person record to the clist (*@\label{lin:personrecord}@*)
\cs_gset:Npn \addtodb:nn #1#2#3
    {
        \cs_gset:cpn { #1_name } { #2 }
        \cs_gset:cpn { #1_occupation } { #3 }
        \clist_gput_left:Nn \personDB { #1 }
    }
    
%        
\addtodb:nn {turner}  {Ted~Turner}  {tycoon}
\addtodb:nn {britney} {Britney Spears} {actress}
\addtodb:nn {che} {Che ~Guevara} {revolutionary}
\clist_map_function:NN \personDB \put_graphic_with_space:n\par
\ExplSyntaxOff
\end{texexample}

The Line~\ref{lin:personrecord} creates to macros, one that will hold the name of the person and another that will hold the occupation. Note that the code would have normally used a |\csname| construction. Here |expl3| takes care of both the |expandafter| as well as the |\csname| construct, simply by using |:cpn| version of |gset|.

What is different with this example, we have added the \docAuxCommand*{addtodb:nn} to add the person names to the list. This still has to be done as an author interface, but as it is just an example, I want to keep the code short. 

\textbf{Automating the addition of fields and records} We have named our database |imageDB| and we have called it a database, but it is so far very unfriendly and all the fields are hard wired in the next Chapter we will create a more appropriate record database.

\textbf{Create a new data base} First we concern ourselves with creating a new database. This is the very first activity we need to define. We store the names of the databases in a master list which we have named |\g_DB_dbs_clist|. We will prefix all our functions and variables with |DB| and we will use this as our module name. 



\textbf{Specifying the database meta data} Our databases will be also records or objects if you want to use an inexactitude name and will also hold information, this is termed \emph{meta data}:

\begin{tabular}{ll}
  name   & \meta{database name} \\
  fields & \meta{list containing the fields as fieldnames}\\
  status & \meta{active or not active}\\
  number of records &\\
  tables & \meta{list}\\
  views  & \\
\end{tabular}

\def\paragraph#1{{\par\leavevmode\bfseries#1}}

\paragraph {Create the master database list} All databases that we will create will be stored
as meta data into another list. This is used only internally at this stage, so we give it an |expl3| sexy name \docAuxCommand{g_db_dbs_clist}.

\begin{texexample}{Creating a database package}{ex:master DB list}
\ExplSyntaxOn
% already defined no need to have it in the example
 \clist_new:N \g_db_dbs_clist
\ExplSyntaxOff
\end{texexample}

\paragraph{Constructor function} Next we create a function that is called when we
need to create a new database.

\begin{texexample}{Continued..}{}
\ExplSyntaxOn
% constructor function
\cs_gset:Npn \g_db_construct_clist:n #1
  { 
% create new DB
  \clist_new:c {#1} 
	% add to master
  \clist_put_left:Nn \g_db_dbs_clist { #1 }
% create meta table
  \g_construct_metatable:n { # 1}
  }
\ExplSyntaxOff
\end{texexample}
		
\paragraph{Creating tables} So far we have created the functions that we need to create a new database. Next we can start writing functions for creating tables for a database. In reality, I called them tables, but this is a misnomer as they hold other stuff as well. 
		
\begin{texexample}{...continued}{ex:db4}		
\ExplSyntaxOn
% persons metatable
% PERSONS-METANAME
% PERSONS-STATUS
% PERSONS-TABLES
\cs_gset:Npn \g_construct_metatable:n #1 
  {
    \cs_gset:cpn   {#1-METANAME  } {   #1    }
    \cs_gset:cpn   {#1-METASTATUS} {-NoValue-}
    \clist_gset:cn {#1-METATABLES} {-NoValue-}
  }		
  
% PERSONS-TABLE-TABLENAME 
% PERSONS-TABLE-TABLENAME-FIELDS (list)		

\cs_gset:Npn \g_construct_table:cc #1 #2 
  {
    \cs_gset:cpn   {#1-TABLE-#2-NAME      } {#2}
    \cs_gset:cpn   {#1-TABLE-#2-STATUS    } {}
    \clist_gset:cn {#1-TABLE-#2-FIELDNAMES} {}
    
    % index key as edef
    \tl_gset:cx  {#1#2-} {name}
    
    % data holding list
    \clist_gset:cn { #1 #2 } { } %(*@\label{lin:personsfamous}@*)
  }
\ExplSyntaxOff  
\end{texexample}

The interesting part is line~\ref{lin:personsfamous} which is a comma delimited list
that will hold all the index keys.

\begin{texexample}{Databases...continued}{ex:fields}
\ExplSyntaxOn
% adds a fieldname to fieldnames
% PERSONS-TABLE-TABLENAME-FIELDNAMES
\cs_gset:Npn \add_fieldname #1 #2 #3
  {
    \clist_gput_left:cx {#1-TABLE-#2-FIELDNAMES} {#3}

  }
%

\cs_gset:Npx \create_index_field #1#2#3#4
  {
    \clist_gput_left:cx {#1#2} {#4}
    
  }  
% create DB table FAMOUS 
\cs_gset:Npx \add_data_index #1#2#3#4
  {
    \clist_gput_left:cx {#1#2} {#4}
    
  } 
  
% add data if is index goes onto clist  
% PERSONS-FAMOUS-ID-SURNAME-VALUE
%   
\cs_gset:Npn \add_field_data #1#2#3#4#5 
  {
   \cs_gset:cpn {#1#2#3#4} 
    { #5    }
  } 
  



% read a field        
\cs_gset:Npn \get_field #1#2#3#4
  { 
    \cs:w #1#2#3#4\cs_end:  
  }
                                
 % create DB PERSONS   
\g_db_construct_clist:n {PERSONS}
\g_construct_table:cc {PERSONS}{FAMOUS}                                
%
\gdef\AddPerson#1#2#3#4{
	\add_data_index {PERSONS} {FAMOUS} {name} {#1}
	\add_field_data {PERSONS} {FAMOUS}{#1} {firstname   } {#1}
	\add_field_data {PERSONS} {FAMOUS}{#1} {surname   } {#2}
	\add_field_data {PERSONS} {FAMOUS}{#1} {occupation} {#3} 
	\add_field_data {PERSONS} {FAMOUS}{#1} {photo} {#4}
}
%
%\get_field {PERSONS} {FAMOUS} {Iggy}  {photo} 
\gdef\PrintImages#1#2{
  \centering 
  \clist_map_inline:cn {#1#2}
    {
      \includegraphics[height=3cm]
      {./martin-schoeller/
        \get_field {#1}{#2}{##1}{photo}
      }\hskip1sp
    }
}
\ExplSyntaxOff
\end{texexample}

What just happened is that we have created two lists one to hold DBs metadata as a simple list and a second |PERSONS|. We have also created the meta-data record. 
 
\begin{texexample}{Database ...continued}{ex:db2}
\AddPerson {Barack} {Obama} {Actor} {barack_obama_2004}
\AddPerson {Iggy} {Pop} {Actor} {iggy_pop_2001}
\AddPerson {Henry} {Kissinger} {Arsehole} {henry_kissinger_2007}
\AddPerson {Frankie} {Velilla} {Student} {frankie_velilla_2001}
\AddPerson {Cindy} {Sheman} {Queen} {cindy_sheman_2000}
\AddPerson {Joe} {Namath} {Tough} {joe_namath_2006}
\AddPerson {Christopher} {Walken} {Tough} {christopher_walken_2000}
\AddPerson {Xiakababoi} {Xiakababoi} {Tough} {xiakababoi_2005}
\AddPerson {Jack} {Nicholson} {Tough} {jack_nicholson_2002}
\AddPerson {Robert} {Deniro} {Actor} {robert_DeNiro_2006}
\PrintImages{PERSONS}{FAMOUS}
\end{texexample}


Now what happens if we decide that we want to add another field in 
our database of famous people, say their biography? we would need to add
another document level command |\AddPersonBio|

\begin{texexample}{adding a bio field}{ex:bio}
\ExplSyntaxOn
\long\gdef\AddPersonBio #1#2 {
   \add_field_data {PERSONS} {FAMOUS} {#1} {bio} {#2}
}
\ExplSyntaxOff
\end{texexample}

Let us add some data for some of the person records we have in our database.

\ExplSyntaxOn
\DeclareDocumentCommand \GetBio {m} {
  \get_field {PERSONS}{FAMOUS}{#1}{bio}
}
\DeclareDocumentCommand \GetPhoto {m} {
  \includegraphics[width=0.8\linewidth] {./martin-schoeller/
    \get_field {PERSONS} {FAMOUS} {#1} {photo}} 
  }
\DeclareDocumentCommand \GetFullName {m} {
    \get_field {PERSONS} {FAMOUS} {#1} {firstname}
    \space 
    \get_field {PERSONS} {FAMOUS} {#1} {surname}
} 
\ExplSyntaxOff


\begin{texexample}{add bio to some records}{ex:bio1}
\ExplSyntaxOn
\DeclareDocumentCommand \GetBio {m} {
  \get_field {PERSONS}{FAMOUS}{#1}{bio}
}
\DeclareDocumentCommand \GetPhoto {m} {
  \includegraphics[width=0.8\linewidth] {./martin-schoeller/
    \get_field {PERSONS} {FAMOUS} {#1} {photo}} 
  }
\DeclareDocumentCommand \GetFullName {m} {
    \get_field {PERSONS} {FAMOUS} {#1} {firstname}
    \space 
    \get_field {PERSONS} {FAMOUS} {#1} {surname}
}    
\ExplSyntaxOff
\end{texexample}
\begin{texexample}{add some more declarations}{ex:2}
\AddPersonBio{Robert}{
  Robert De Niro (/dəˈnɪroʊ/; born August 17, 1943) is an American actor and producer   who has starred in over 90 films. His first major film roles were in the sports drama Bang the Drum Slowly (1973) and Martin Scorsese's crime film Mean Streets (1973). In 1974, after being turned down for the role of Sonny Corleone in the crime film The Godfather (1972), he was cast as the young Vito Corleone in The Godfather Part II (1974), a role for which he won the Academy Award for Best Supporting Actor.

De Niro's longtime collaboration with Scorsese later earned him an Academy Award for Best Actor for his portrayal of Jake LaMotta in the 1980 film Raging Bull. He also earned nominations for the psychological thrillers Taxi Driver (1976) and Cape Fear (1991), both directed by Scorsese. De Niro received additional Academy Award nominations for Michael Cimino's Vietnam war drama The Deer Hunter (1978), Penny Marshall's drama Awakenings (1990), and David O. Russell's romantic comedy-drama Silver Linings Playbook (2012). His portrayal of gangster Jimmy Conway in Scorsese's crime film Goodfellas (1990) earned him a BAFTA nomination in 1990.[1] De Niro has earned four nominations for the Golden Globe Award for Best Actor – Motion Picture Musical or Comedy, for his work in the musical drama New York, New York (1977), opposite Liza Minnelli, the action comedy Midnight Run (1988), the gangster comedy Analyze This (1999), and the comedy Meet the Parents (2000). He has also simultaneously directed and starred in films such as the crime drama A Bronx Tale (1993) and the spy film The Good Shepherd (2006). De Niro has also received the AFI Life Achievement Award in 2003 and the Golden Globe Cecil B. DeMille Award in 2010.}

\AddPersonBio {Jack}{
John Joseph "Jack" Nicholson (born April 22, 1937) is an American actor and filmmaker. Throughout his career, Nicholson has portrayed unique and challenging roles, many of which include dark portrayals of excitable, neurotic and psychopathic characters. Nicholson's 12 Academy Award nominations make him the most nominated male actor in the Academy's history.

Nicholson has won the Academy Award for Best Actor twice, one for the drama One Flew Over the Cuckoo's Nest (1975) and the other for the romantic comedy As Good as It Gets (1997). He also won the Academy Award for Best Supporting Actor for the comedy-drama Terms of Endearment (1983). Nicholson is tied with Walter Brennan and Sir Daniel Day-Lewis as one of three male actors to win three Academy Awards. In 1988 Nicholson won a Grammy Award for Best Album for Children for The Elephant's Child. He is well known for playing Frank Costello in the Martin Scorsese-directed crime drama The Departed (2006), Jack Torrance in the Stanley Kubrick–directed psychological horror film The Shining and the Joker in Batman (1989).

Nicholson is one of only two actors to be nominated for an Academy Award for acting in every decade from the 1960s to the 2000s; the other was Michael Caine. He has won six Golden Globe Awards, and received the Kennedy Center Honor in 2001. In 1994, he became one of the youngest actors to be awarded the American Film Institute's Life Achievement Award. Other notable films in which he has starred include the road movie Easy Rider (1969), the drama Five Easy Pieces (1970), the comedy-drama film The Last Detail (1973), the neo-noir mystery film Chinatown (1974), the drama The Passenger (1975), the epic film Reds (1981), the romantic horror film Wolf (1994), the legal drama A Few Good Men (1992), the Sean Penn-directed mystery film The Pledge (2001), and the comedy-drama About Schmidt (2002).
}
\end{texexample}

Finally continuing our example we will now define a \docAuxCommand{PrintBio} that can be used
to finally extract the data and present typeset it.

  
\begin{texexample}{Printing the Bios}{ex:bio3}
\long\gdef\PrintBio#1{%
\par
 {\pagebreak
 \leavevmode 
 \Huge
 \bfseries
 \centerline{\GetFullName {#1}}}
 \par
 \vspace{20pt}

 {\centering
  \GetPhoto {#1}\par
  \vspace{20pt}}

 \parindent1em
 \GetBio {#1}
 \vfill
}

\PrintBio{Robert}

\PrintBio{Jack}
\end{texexample}

There are a lot of improvements that we can do to the code. Firstly we have not done any error checking. The idea of pre-packaged code is that the finer details can be handled. Error checking should be done for example to verify that an image is available on disk. Also not to hard wired any directories. Sorting is still an issue. Our indexing is also inadequate. What happens if we have Robert DeNiro and Robert Williams? We indexed on the Robert. We would have been better off to add an index key automatically or index by using both name and surname. All these are issues that need to be incorporated. 



\chapter{Queues}
\section{Queue Fundamentals}

A queue is an ordered list in which all insertions are made at one end, called the rear end, while all deletions are made at the other end, called the front end. Given a queue $Q=(a_1,a_2,\dots,a_n)$ with $a_1$ as the front element and $a_n$ as the rear element, we say that $a_{i+1}$ is behind $a_1$ $1 \leq i <n$.

\section{Operations on a Queue}

The operations which are carried on queue are similar to these which are carried on a stack, except their semantics are different. The operations are:

\begin{enumerate}
\item To create a queue
\item To insert an element into the queue
\item To delete an element from the queue
\item To check which element is in the front of  the queue
\item To check whether a queue is empty or not.
\end{enumerate}

\begin{figure}[htbp]
\centering
\includegraphics[width=0.5\textwidth]{queue}
\end{figure}

Since this is a book about typesetting, the next example will create a queue structure that will typeset the operations of a queue and provide diagrams to illustrate the algorithmic steps involved.


\begin{docCommand}{CreateQueue}{\meta{queue name}}
Creates an empty queue .
\end{docCommand}

\def\anitem{{\color{blue}\vrule height1.5cm width0.4cm}\thinspace}
\DeclareDocumentCommand\anitem{O{blue}}{%
{\color{#1}\vrule height1.5cm width0.4cm}\thinspace
}
\NewDocumentCommand\EnqueueString{s}{
  \IfBooleanTF #1
     {Enqueue $\rightarrow$}
     {Enqueue \phantom{$\rightarrow$}}
}

\NewDocumentCommand\DequeueString{s}{
  \IfBooleanTF #1
     {\phantom{$\rightarrow$} Dequeue $\rightarrow$ }
     {\phantom{$\rightarrow$} Dequeue \phantom{$\rightarrow$}}
}
\begin{enumerate}
\item The conventions we will use is that when an item is enqueued it will be typeset in red as shown below, when it enters the front end.

Enqueue $\rightarrow$ \anitem[red]\DequeueString* 

\item When another item is added the above procedure is repeated, but this time the elements not in the front are shown in blue.

\EnqueueString* \anitem[red]\anitem \phantom{$\rightarrow$} Dequeue

\item Enqueue one more item will change the diagram to the following:

\EnqueueString \anitem \anitem \anitem \anitem \anitem \phantom{$\rightarrow$} Dequeue \hfill \anitem[blue!30] 

\item Enqueue one more item will change the diagram to the following:

\EnqueueString \anitem \anitem \anitem \anitem  \DequeueString \hfill \anitem[blue!30] \anitem[blue!30]

\item To summarize the typeset diagram represents the three states of the queue, enqueue, status and dequeue. If a right arrow is shown it either enqueued or dequeued an element. If none is shown it represents the status of the system
\end{enumerate}

I have specifically made the example a bit more complicated, in order to reinforce some of the concepts discussed in other chapters.

The example requires that when a dequeuing command is entered it is indicated with a right arrow (|dequeue \rightarrow|) the arrow is not shown when the enqueuing operation takes place. To keep the length of the diagram spaced properly it requires that a phantom command is used for the enqueing operation.

\subsection{Coding auxiliary macros}

We will need two auxiliary macros to typese the enqueue and dequeue strings with or without arrows. We will use the \pkgname{xparse} package to create the commands. We will use the star version of the command as a toggle to show the \docAuxCommand*{rightarrow} or not. If the command is enetered with a star it will leave the right amount of space to the right of the string, so that all diagrams line nicely to the left. This is achieved using the  \docAuxCommand*{phantom} command that we have encountered earlier.

\emphasis{IfBooleanTF}
\begin{teXXX}
\NewDocumentCommand\EnqueueString{s}{
  \IfBooleanTF #1
     {Enqueue $\rightarrow$ }
     {Enqueue \phantom{$\rightarrow$}}
}
\NewDocumentCommand\DequeueString{s}{
  \IfBooleanTF #1
     {\phantom{$\rightarrow$} Dequeue $\rightarrow$ }
     {\phantom{$\rightarrow$} Dequeue \phantom{$\rightarrow$}}
}
\end{teXXX}

\subsection{Creating the Queue Macros}

In order to typeset the diagrams we will use two queues. One to store the main queue and a second one to store the dequeue items. Before we code the actual functions it will be nice to think of the  commands we want to offer our users. This will also dictate to an extend the code we require.

\begin{verbatim}
\DrawQueStatus
\Enque
\Deque
\DrawEnque
\DrawDeque
\end{verbatim}

\begin{texexample}{Adding content to the sequence}{}
\ExplSyntaxOn
\seq_new:N \g_qlisti
\seq_new:N \g_qlistii
\seq_gpush:Nn \g_qlisti{\anitem[blue]}
\seq_gpush:Nn \g_qlisti{\anitem[blue]}

\seq_push:Nn \g_qlistii{\anitem[blue!30]}

\DeclareDocumentCommand\Enque{O{red}}
   {
      \seq_gpush:Nn \g_qlisti{\anitem[#1]}
   }
   
 \DeclareDocumentCommand\Deque{O{blue!30}}
   {
      \seq_gpop_left:NN \g_qlisti \@tempa
      \seq_gpush:Nn\g_qlisti{\anitem[blue]}
      \seq_gpush:Nn \g_qlistii{\anitem[#1]}
   }  
   
\Enque\Enque\Enque\Enque

\EnqueueString\seq_use:Nn \g_qlisti {} 

\DequeueString*\hfill\seq_use:Nn \g_qlistii{}
\ExplSyntaxOff
%%%%
\end{texexample}
\ExplSyntaxOn
\DeclareDocumentCommand\Enque{O{red}}
   {
      \seq_gpush:Nn \g_qlisti{\anitem[#1]}
   }
   
 \DeclareDocumentCommand\Deque{O{blue!30}}
   {
      \seq_gpop_left:NN \g_qlisti \@tempa
      \seq_gpush:Nn\g_qlisti{\anitem[blue]}
      \seq_gpop_right:NN \g_qlisti \@tempa
      \seq_gpush:Nn \g_qlistii{\anitem[#1]}
   }  
\ExplSyntaxOff
   
One of the characteristics of the programming process is that it is like painting. Some programmers come up with  excellent code on their first attempt, whereas most of us will \emph{refactor} the code over several passes either to improve it, optimize it or catch possible errors.

A subtle issue with the above code is if we enqueue a number of items and then dequeue only the first item will change from red to blue the rest will be still in the que as red. What we will have to do is modify the \docAuxCommand*{Enque} to check if the list is not empty to remove the head item and replace it with a blue box, before effecting the enque operation. This will also give us a chance to use the sequence conditional functions for emptiness. We should also add the conditional in the \docAuxCommand*{Deque} function as well as the author typesetting commands \docAuxCommand*{DrawEnque} and \docAuxCommand*{DrawDeque}.

\begin{texexample}{The drawing functions}{ex:drgfunctions}
\ExplSyntaxOn
\DeclareDocumentCommand\DrawDeque{ O{blue!30} }
  { 
   \EnqueueString  \seq_use:Nn \g_qlisti {} 
   \DequeueString*  \hfill  \seq_use:Nn \g_qlistii{}    
  }
\Deque\Deque\Deque
\DrawDeque
\ExplSyntaxOff  
\end{texexample}

The \docAuxCommand*{DrawQues}, draws the two queues. This is very similar to the other two \docAuxCommand*{Draw}{\meta{deque}} or \meta{enque} functions. It just does not draw the arrows.

\begin{texexample}{The drawing functions}{ex:drgfunctions}
\ExplSyntaxOn
\DeclareDocumentCommand\DrawQues{ O{blue!30} }
  { 
   \EnqueueString  \seq_use:Nn \g_qlisti {} 
   \DequeueString  \hfill  \seq_use:Nn \g_qlistii{}    
  }
\Deque
\DrawQues
\ExplSyntaxOff  
\end{texexample}


 \chapter{Using Comma lists as stacks}
 
 In this chapter, we will look at one common Abstract Data Type (ADT), the stack. A stack is a \emph{collection}, meaning that it is a data structure that contains multiple elements. Other collections we have seen include dictionaries and lists. An ADT is defined by the operations that can be performed on it, which is called an interface. The interface for a stack consists of these operations:

\begin{description}
\item [init] Initialize a new empty stack.
\item [push]
Add a new item to the stack.
\item [pop]
Remove and return an item from the stack. The item that is returned is always the last one that was added.

\item [emptiness] Check whether the stack is empty.
\end{description}

A stack is sometimes called a “Last in, First out” or LIFO data structure, because the last item added is the first to be removed.

 Comma lists can be used as stacks, where data is pushed to and popped
 from the top of the comma list. (The left of a comma list is the top, for
 performance reasons.) The stack functions for comma lists are not
 intended to be mixed with the general ordered data functions detailed
 in the previous section: a comma list should either be used as an
 ordered data type or as a stack, but not in both ways.
 
 \begin{figure}[htbp]
 \hspace*{3cm}%optically center it
 \scalebox{0.7}{\begin{drawstack}
  \startframe
  \cell{First cell}
  \cell{Second cell}
  \finishframe{Some stack frame}
  \cell{Not interesting}
  \startframe
  \cell{Next stack frame}
  \cell{Next stack frame}
  \finishframe{Another stack frame}
\end{drawstack}}
\caption{A stack drawn with the \pkgname{drawstack} package. The package can be used to draw different stacks and their frames.}
\end{figure}

To construct a new empty stack, use the same functions as for a clist or sequence data structure. They are identical and calling them a stack is just syntactic sugar.

 \begin{docCommand}{clist_get:NN}{ \meta{comma list} \meta{token list variable}}
   Stores the left-most item from a \meta{comma list} in the
   \meta{token list variable} without removing it from the
   \meta{comma list}. The \meta{token list variable} is assigned locally.
   If the \meta{comma list} is empty the \meta{token list variable} will
   contain the marker value \docAuxCommand*{q_no_value}.
 \end{docCommand}
\makeatletter 
 \global\let\clistsort\lst@BubbleSort
\makeatother 
\begin{texexample}{Sequence}{ex:sequence}
\makeatletter
\ExplSyntaxOn
\clist_gset:Nn \title_words_not_capitalized_en 
 { 
  a,an,the,at,by,for,in,of,on,to,up,and,as,but,it,or, 
  nor,do,for,this,be,A,An,The,At,By,For,In,Of,On,To,Up, 
  And,As,But,It,Or,Nor,Do,For,This,Be, 
  abaft,aboard,about,above,absent,across,afore,after,against,along,
  alongside,amid,amidst,among,amongst,an,anenst,apropos,apud,around,
  as,aside,astride,at,athwart,atop,barring,before,behind,below,beneath,
  beside,besides,between,beyond,but,by,circa,concerning,despite,down,
  during,except,excluding,failing,following,for,forenenst,from,given,in,
  including,inside,into,lest,like,mid,midst,minus,modulo,near,next,
  notwithstanding,of,off,on,onto,opposite,out,outside,over,pace,past,
  per,plus,pro,qua,regarding,round,sans,save,since,than,through,
  throughout,till,times,to,toward,towards,under,underneath,unlike,
  until,unto,up,upon,Versus,versus,via,vice,with,within,without,worth
}
\clist_gset:Nn \abbreviations 
 {
   A.B.C.,iTunes
 }

\clist_gset:Nn \acronyms
 {
   NATO,UN,US,Scuba,Laser
 }  
\cs_new:Npn \addacronym #1 
 {
   \clist_put_left:Nn \acronyms {#1}
   \lst@BubbleSort\acronyms
   \clist_remove_duplicates:N\acronyms
 }  

\addacronym {EU}
\meaning\acronyms\\
\addacronym {AA}
\acronyms
\ExplSyntaxOff
\makeatother
\end{texexample}


\endinput
\end{document}

\section{Moving items from one list to another}{}

\begin{texexample}{Moving items from one stack to another.}{ex:stacks}
\ExplSyntaxOn
\clist_new:N \phd_stack_a
\clist_new:N \phd_stack_b
\token_to_meaning:N \phd_stack_a
\Expl_SyntaxOff
\end{texexample}

If we examine the meaning of the stacks at this stage, they are just empty macros, not holding any values.

\begin{texexample}{put something into the stacks}{}
\ExplSyntaxOn
\clist_gset:Nn \phd_stack_a {3,4,5,6,}
\clist_gpush:Nn\phd_stack_a {\ldots}

STACK a:~\phd_stack_a\par
\ExplSyntaxOff
\end{texexample}


Let us continue by popping and pushing some more values
\begin{texexample}{Continue}{}
\ExplSyntaxOn
\clist_gpop:NN\phd_stack_a\@tempa
\clist_gpush:Nx\phd_stack_b\@tempa
%% Pop a value
\clist_gpop:NN\phd_stack_a\@tempa
\clist_gpush:Nx\phd_stack_b\@tempa
\clist_gpop:NN\phd_stack_a\@tempa
\clist_gpush:Nx\phd_stack_b\@tempa
\clist_gpop:NN\phd_stack_a\@tempa
\clist_gpush:Nx\phd_stack_b\@tempa
stack a:~\phd_stack_a\par
stack b:~\phd_stack_b
\ExplSyntaxOff
%%%%%%%%%%%%%
\end{texexample}

The much promised freedom from having to deal with \tex expansion has not arrived---although we can save some frustration and typing. When moving items from |stacka| to |stack b| we have used the |:Nx| form of the command
so that the temporary token list variable is expanded. If we do not do that the second stack values will only store the last value of |@tempa|.

In practical applications the second stack is normally used as an array to just store the values. The symbol items before being popped are examined and if for example is a |+| sign the items will be summed up and placed again in the second stack to keep tally of our totals.

Our next step is to refactor the code in our example to recursively empty the first stack.

\begin{texexample}{Moving items from one stack to another}{ex:stacks}
\ExplSyntaxOn
\fboxsep=2pt
\fboxrule=0.4pt
\clist_gset:Nn \phd_stack_a {1,2,3,4,5,6,\ldots}
\clist_gset:Nn \phd_stack_b {}
original~stack a:~
\cs_gset:Nn\recurse:
 {
   \clist_gpop:NNTF\phd_stack_a\@tempa{\clist_gpush:Nx\phd_stack_b\@tempa
      \fbox{\@tempa}~
      \recurse:}{empty~stack\par}
 }  
\recurse: 
stack b:~\phd_stack_b
\ExplSyntaxOff
%%%%%%%%%%%%%
\end{texexample}

\begin{texexample}{Moving items from one stack to another}{ex:stacks}
\ExplSyntaxOn
\fboxsep=2pt
\fboxrule=0.4pt
\cs_gset:Nn\recurseb:
 {
   \clist_gpop:NNTF\phd_stack_b\@tempa{
      \if +\@tempa\relax\else\framebox[1.5em]{\strut\@tempa}\\ \fi
      \recurseb:}{empty~stack\par}
 }  
\recurseb: 
stack b:~\phd_stack_b
\ExplSyntaxOff
%%%%%%%%%%%%%
\end{texexample}

So far so good. We have managed to construct two stacks and to typeset their content in nice boxes. Hopefully, by now if you have been following the examples, you have the rudimentary skills to build our next, more complicate example that would parse a sequence of algebraic expressions and tokenize them. 



\begin{docCommand}{clist_get:NNTF}{ \meta{comma list} \meta{token list variable} \marg{true code} \marg{false code}}
 
   If the \meta{comma list} is empty, leaves the \meta{false code} in the
   input stream.  The value of the \meta{token list variable} is
   not defined in this case and should not be relied upon.  If the
   \meta{comma list} is non-empty, stores the top item from the
   \meta{comma list} in the \meta{token list variable} without removing it
   from the \meta{comma list}. The \meta{token list variable} is assigned
   locally.
 \end{docCommand}


  \begin{docCommand}{clist_pop:NN}{ \meta{comma list} \meta{token list variable}}
   Pops the left-most item from a \meta{comma list} into the
   \meta{token list variable}, \emph{i.e.}~removes the item from the
   comma list and stores it in the \meta{token list variable}.
   Both of the variables are assigned locally.
 \end{docCommand}


  \begin{docCommand}{clist_gpop:NN}{ \meta{comma list} \meta{token list variable}}
   Pops the left-most item from a \meta{comma list} into the
   \meta{token list variable}, \emph{i.e.}~removes the item from the
   comma list and stores it in the \meta{token list variable}.
   The \meta{comma list} is modified globally, while the assignment of
   the \meta{token list variable} is local. Also available as :cN
 \end{docCommand}

 \begin{docCommand}{clist_pop:NNTF}{ \meta{sequence} \meta{token list variable} \marg{true code} \marg{false code}}
   If the \meta{comma list} is empty, leaves the \meta{false code} in the
   input stream.  The value of the \meta{token list variable} is
   not defined in this case and should not be relied upon.  If the
   \meta{comma list} is non-empty, pops the top item from the
   \meta{comma list} in the \meta{token list variable}, \emph{i.e.}~removes
   the item from the \meta{comma list}. Both the \meta{comma list} and the
   \meta{token list variable} are assigned locally.
 \end{docCommand}


   \begin{docCommand}{clist_gpop:NNTF}{\meta{comma list} \meta{token list variable} \marg{true code} \marg{false code}}
     If the \meta{comma list} is empty, leaves the \meta{false code} in the
   input stream.  The value of the \meta{token list variable} is
   not defined in this case and should not be relied upon.  If the
   \meta{comma list} is non-empty, pops the top item from the
   \meta{comma list} in the \meta{token list variable}, \emph{i.e.}~removes
   the item from the \meta{comma list}. The \meta{comma list} is modified
   globally, while the \meta{token list variable} is assigned locally.
 \end{docCommand}

% \begin{function}
%   {
%     \clist_push:Nn,  \clist_push:NV,  \clist_push:No,  \clist_push:Nx,
%     \clist_push:cn,  \clist_push:cV,  \clist_push:co,  \clist_push:cx,
%     \clist_gpush:Nn, \clist_gpush:NV, \clist_gpush:No, \clist_gpush:Nx,
%     \clist_gpush:cn, \clist_gpush:cV, \clist_gpush:co, \clist_gpush:cx
%   }
 \begin{docCommand}{clist_push:Nn}{ \meta{comma list} \marg{items}}
   Adds the \marg{items} to the top of the \meta{comma list}.
   Spaces are removed from both sides of each item.
 \end{docCommand}
%
% \section{Using a single item}
%
% \begin{function}[added = 2014-07-17, EXP]
%   {\clist_item:Nn, \clist_item:cn, \clist_item:nn}
%   \begin{syntax}
%     \docAuxCommand*{clist_item:Nn} \meta{comma list} \Arg{integer expression}
%   \end{syntax}
%   Indexing items in the \meta{comma list} from~$1$ at the top (left), this
%   function will evaluate the \meta{integer expression} and leave the
%   appropriate item from the comma list in the input stream. If the
%   \meta{integer expression} is negative, indexing occurs from the
%   bottom (right) of the comma list. When the \meta{integer expression}
%   is larger than the number of items in the \meta{comma list} (as
%   calculated by \docAuxCommand*{clist_count:N}) then the function will expand to
%   nothing.
%   \begin{texnote}
%     The result is returned within the \tn{unexpanded}
%     primitive (\docAuxCommand*{exp_not:n}), which means that the \meta{item}
%     will not expand further when appearing in an \texttt{x}-type
%     argument expansion.
%   \end{texnote}
% \end{function}


\chapter{LaTeX3 quarks and recursion}
\label{ch:quarks}

\section{What are quarks?}
Quarks and recursion are central to the expl3 language. Quarks are a weird concept and is inherited from \tex’s way of scanning macro arguments.

But before we delve into the details of |expl3|'s quarks let us review \tex's delimited functions with an example. Consider the following example where we delimit the arguments of a macro |\test| with the control sequence |\texquark|. We do not need to define the |\texquark| and as we discussed in the section on macros it can even consist of the macro name itself. \tex will scan the input until the marker is found. It will also absorb the marker and do nothing about it.

\begin{texexample}{TeX quarks!}{}
\def\test#1\texquark{#1}
\test 123456\texquark \\
\def\test#1\test{#1}
\test 123456\test
\end{texexample}

In \latex2e macro delimiters are found all over the place, mostly in the form of \docAuxCommand{@nil}, \docAuxCommand{@nni}  or \docAuxCommand{@@}. See for example, how the \latex2e kernel defines lists.

 In \latex3 these have been termed \enquote{quarks} and \enquote{scan marks}. By convention all constants of type quark start out with |\q_| and scan marks start with |\s_|. Scan marks are reserved for internal use by the kernel and you should avoid using them in your code.\index{scan marks}\index{quarks}

They differ from the simple case above with the \tex example, in that they are used mostly indirectly. The \latex3 quarks, are defined so that they expand to themselves. As such they should never be executed directly in the code. This would cause and endless loop and cause either the program or even your computer to crash. The reason they hold a value, is that they can be tested, using |\ifx| which compares the meaning of two macros without expanding them. The equivalent construction in |expl3| is |\if_meaning:w|. We can use it at the next example. Note I gave used |\def| in th example to make it clearer, but one of course can use |\cs_set:Npn| or an equivalent function.

\begin{texexample}{Checking if is a quark}{ex:quarks}
\ExplSyntaxOn
\def\quark{\quark}
\cs_set_nopar:Npn \b {\quark}
\if_meaning:w  \quark\b
   \PASS
\else:
  \FAIL 
\fi:
\ExplSyntaxOff
\end{texexample}

This ingenious technique employed in Example~\ref{ex:quarks} depends on \tex’s ability to carry out comparisons without expanding the macros being compared. This way semantic definitions can be made for quarks and employed in generic recursive functions. 
Normally, you wouldn’t need to define your own quarks, as the ones made available by |expl3| are adequate for most tasks. If you have to create one, it can be created using:

\begin{docCommand}{quark_new:N}{ \meta{quark}}
Creates a new \meta{quark} which expands only to \meta{quark}. The \meta{quark} is defined globally, and an error message will be raised if the name was already taken.
\end{docCommand}

For example, the kernel defines two flavours of quarks to be used specifically for recursion and which we will use in the next section.

\begin{teXXX}
\quark_new:N \q_recursion_tail
\quark_new:N \q_recursion_stop
\end{teXXX}

Other flavours are lsited in the manual and summarized below:

\begin{docCommand}{q_stop}{ \meta{quark}}
Used as a marker for delimited arguments such as:
\begin{verbatim}
\cs_set:Npn \tmp:w #1#2 \q_stop {#1}
\end{verbatim}
\end{docCommand}





\section{Recursion}
 
One of the problem areas in programming recursion is to have a uniform interface to intercepting and terminating loops when one is doing recursion. \latex3 provides the building blocks.

First let us see an example:

\begin{texexample}{Recursion}{ex:l3recursion}
\ExplSyntaxOn

\cs_new:Npn \__my_decoration_fn:nn #1  {
  \str_if_eq:nnTF{e}{#1}
    {[{\bfseries\color{red}#1}]}
    {[#1]}
}

\cs_new:Npn \mymain #1 
{
      \__my_map:n #1 \q_recursion_tail\q_recursion_stop
}

\cs_new:Npn \__my_map:n #1 
  {
    \quark_if_recursion_tail_stop:n {#1}
    \__my_decoration_fn:nn  {#1} 
    \__my_map:n
  }
\ExplSyntaxOff
 
\mymain {abcdefgh}
\end{texexample}

The main function, will first call a mapping function leaving in the stream the following:
\medskip

\texttt{bcdefgh} {\hl{\textbackslash q\_recursion\_tail} \hl{\textbackslash q\_recursion\_stop}}
\medskip

On the second iteration the stream will be reduced by one token (b) and the remaing value will be:
\medskip

\texttt{cdefgh} {\hl{\textbackslash q\_recursion\_tail} \hl{\textbackslash q\_recursion\_stop}}
\medskip

This is repeated, until the quark is captured which causes the recursion to terminate. The termination is achieved by
the macro |\quark_if_recursion_tail_stop:n|. This will also absorb the |\_recursion_stop| quark. 

While the function is recursing we send the captured letter to a function to decorate and typeset it. This function can be programmed to do whatever you want to achieve. Note in the example it can only accept one argument.

Now what happens, if you wanted to capture two letters at a time or three letters at a time? The program would have to be modified as follows:

\begin{texexample}{Recursion}{ex:l3recursion}
\ExplSyntaxOn
\cs_new:Npn \__my_second_decoration_function:nn #1#2{
   {\color{red}
   [#1#2]}  
}
\cs_set:Npn \mymainother #1
{
 
   \__my_map_other:nn #1  \q_recursion_tail\q_recursion_tail\q_recursion_stop
}

\cs_new:Npn \__my_map_other:nn #1#2
  {
    \quark_if_recursion_tail_stop:n {#1}
    \quark_if_recursion_tail_stop:n {#2}
    \__my_second_decoration_function:nn  {#1}{#2} 
    \__my_map_other:nn
  }

\ExplSyntaxOff 
 
\mymainother {abcdefgh}
\end{texexample}

What just happened, we modified the custom function to accept two arguments, as well as |\_my_map_other:nn|. I also changed  their names to avoid clashes in this document.

In the next example we will iterate through two lists recursively. The first list will provide a string, which we will have to check if it consists of valid character. The valid characters are provided by the second argument of the main macro.


\begin{texexample}{Recursion}{ex:l3recursion}
\ExplSyntaxOn
\cs_new:Npn \ylcompare #1#2
  {
     \__yl_compare_auxi:nN {#2} #1 \q_recursion_tail \q_recursion_stop
  }
  
  
\cs_new:Npn \__yl_compare_auxi:nN #1#2
  {
    \quark_if_recursion_tail_stop:N #2
    \__yl_compare_auxii:nN {#1} #2
    \__yl_compare_auxi:nN {#1}
  }
 
 
\cs_new:Npn \__yl_compare_auxii:nN #1#2
  {
    \__yl_compare_auxiii:NN #2 #1 \q_recursion_tail \q_recursion_stop
  }
\cs_new:Npn \__yl_compare_auxiii:NN #1#2
  {
  % if found not found stop and print
    \quark_if_recursion_tail_stop_do:Nn #2 { \FAIL\  #1 }
  % if not the list end  
    \str_if_eq:nnT {#1} {#2}
      {
        \use_i_delimit_by_q_recursion_stop:nw { \PASS\  #1 }
      }
  % recurse     
    \__yl_compare_auxiii:NN #1
  }
\ExplSyntaxOff

\ylcompare{1234567890AAA}{-1234567890)(}
\ylcompare{text}{abcdefghijklmnopqrst} 
\end{texexample}

How would one modify the above to provide a boolean value if the string is made up only of valid characters? For example for a vowel or alphabet string. This is easy as we can define a boolean, so instead of printing the assertion we would set the boolean at false if it fails. For a number proving string, our method will fail, as we need to test for cases such as |-12345-567|, which is not a valid string also we need to think if we want to allow any spaces. This would probably have to be programmed as a special macro.

\section{Lower level functions}

As we have seen in the section for \tex iteration, one can build almost anything given patience and skills. Many examples can be found in the |expl3| package |fp|. Example~\ref{ex:fp1} is taken from the |fp| package and is a macro to trim leading zeros from a token representing a real number. All the |\@@_| are used in packages to add a prefix when processed through the doc/docstrip system, in this case it will add |fp_|.

\begin{texexample}{Weirds}{ex:fp1}
\makeatletter
\ExplSyntaxOn
 \cs_new:Npn \@@_trim_zeros:w #1 ;
  {
    \@@_trim_zeros_loop:w #1
      ; \@@_trim_zeros_loop:w 0; \@@_trim_zeros_dot:w .; \s__stop
  }
  
\cs_new:Npn \@@_trim_zeros_loop:w #1 0; #2 { #2 #1 ; #2 }

\cs_new:Npn \@@_trim_zeros_dot:w #1 .; { \@@_trim_zeros_end:w #1 ; }

\cs_new:Npn \@@_trim_zeros_end:w #1 ; #2 \s__stop { #1 }
 
 
\@@_trim_zeros:w  121200010.000; 
\ExplSyntaxOff
\end{texexample}


I have removed the |@@_| and replaced them with the |fp_| prefix to make the code more concise and readable.
The main function is delimited with a semi-colon |;| delimited function. Within the macro this is passed onto
|\fp_trim_zeros_loop:w| for further processing. 

\begin{texexample}{Weird}{ex:fp1}
\makeatletter
\ExplSyntaxOn

 \cs_new:Npn \fp_trim_zeros:w #1 ;
  {
    \fp_trim_zeros_loop:w #1;\fp_trim_zeros_loop:w 0; \fp_trim_zeros_dot:w .; \s__stop
  }
  
\cs_new:Npn \fp_trim_zeros_loop:w #1 0; #2 { #2 #1 ; #2 }

\cs_new:Npn \fp_trim_zeros_dot:w #1 .; { \fp_trim_zeros_end:w #1 ; }

\cs_new:Npn \fp_trim_zeros_end:w #1 ; #2 \s__stop { #1 }

 
\fp_trim_zeros:w  131.200010000 ; 
 \ExplSyntaxOff
\end{texexample}

This function is looking for two variables |#1 0; #2| It will scan until its end and then rescan again. The secon time it will absorb |\fp_trim_zeros_dot:w| as its second argument and then continue expanding this function.

\begin{teXXX}
\fp_trim_zeros_loop:w #1;\fp_trim_zeros_loop:w 0; {second macro}
\end{teXXX} 

Weird but wonderful functional programming.

\section{Summary}

This has brought us to almost the end of the |expl3| structures and language. There is much more to cover, but once you become proficient with the syntax and basic usage of its modules, you can pick up the rest through the documentation. 









\chapter{The LaTeX3 l3token package}
\label{ch:l3token}

The \tex concept of tokens is central to its operation. In earlier chapters we discussed extensively the use of category codes and other important aspects of \tex’s tokens. Rememeber a \tex token is either a single character or a control sequence such as a the control sequence |\test|.

A review of all possible tokens is appropriate at this stage, before we examine the module in more detail. We distinguish the meaning of a token which controls the expansion of the token and its effect on \tex’s state,
and its shape, which is used when comparing token lists such as for delimited arguments.
Two tokens of the same shape must have the same meaning, but the converse does not
hold.

A token has one of the following shapes:

\begin{enumerate}
\item A control sequence, characterized by the sequence of characters that constitute its
name: for instance, |\use:n| is a five-letter control sequence.
\end{enumerate}

Now is perhaps a good time to mention some subtleties relating to tokens with
category code 10 (space). Any input character with this category code (normally, space
and tab characters) becomes a normal space, with character code 32 and category code 10.

When a macro takes an undelimited argument, explicit space characters (with character
code 32 and category code 10) are ignored. If the following token is an explicit
character token with category code 1 (begin-group) and an arbitrary character code,
then TEX scans ahead to obtain an equal number of explicit character tokens with category
code 1 (begin-group) and 2 (end-group), and the resulting list of tokens (with outer
braces removed) becomes the argument. Otherwise, a single token is taken as the argument
for the macro: we call such single tokens \enquote{N-type}, as they are suitable to be used
as an argument for a function with the signature :N.



\begin{texexample}{Space tokens}{ex:sptoken}
\ExplSyntaxOn  
 \cs_set:Npn \my_space_token { }
 \token_to_meaning:N \my_space_token\\
 \token_to_meaning:N \c_space_token
 
 % Note that the ~ active character in an ExplSyntaxOn
 % environment has a more complicated definition.
 \token_to_meaning:N ~
\ExplSyntaxOff  
\end{texexample}

The actual definition from the kernel code for the \cs{c_space_token}
\begin{teX}
\use:n { \tex_global:D \tex_let:D \c_space_token = ~ } ~
\end{teX}

\begin{texexample}{makeatletter}{}
\ExplSyntaxOn
\group_begin:
\char_set_catcode_letter:N @
\char_set_catcode_letter:N 1
\def\@store1a{AAAA}
\@store1a\\
\token_to_meaning:N @\\
\token_to_meaning:N 1\\
\char_set_catcode_other:N @
\char_set_catcode_other:N 1
\token_to_meaning:N @\\
\token_to_meaning:N 1\\
\group_end:
\ExplSyntaxOff
\end{texexample}

There are sixteen different commands to set the catcode to any of the predefined groups used by \tex. If you cannot remember the catcode number for a character, try and remember its normal name!

\begin{verbatim}
 \char_set_catcode_escape:N 
 \char_set_catcode_group_begin:N
 \char_set_catcode_group_end:N
 \char_set_catcode_math_toggle:N
 \char_set_catcode_alignment:N
 \char_set_catcode_end_line:N
 \char_set_catcode_parameter:N
 \char_set_catcode_math_superscript:N
 \char_set_catcode_math_subscript:N
 \char_set_catcode_ignore:N
 \char_set_catcode_space:N
 \char_set_catcode_letter:N
 \char_set_catcode_other:N
 \char_set_catcode_active:N
 \char_set_catcode_comment:N
\char_set_catcode_invalid:N
\end{verbatim}

\section{Token predicate functions}

\begin{docCommand}{token_if_macro:NTF} { \meta{token} \marg{true code} \marg{false code}}
tests if the \meta{token} is a \tex macro.
\end{docCommand}

\begin{texexample}{Test if is a macro}{ex:assertt}
\ExplSyntaxOn

% This is a common problem in LaTeXe. 
% \sometest is let tto |\relax| in a csname
 \csname my_sometest\endcsname
 
 % traditional definition using a csname and 
 % \expandafter
 \expandafter\def\csname my_sometesti\endcsname{}
 
 % All tests must pass
 \token_if_macro:NTF \par           { \FAIL } { \PASS } 
 \token_if_macro:NTF \minipage      { \PASS } { \FAIL } 
 \token_if_macro:NTF \my_sometest   { \FAIL } { \PASS }

 \token_if_macro:NTF \my_sometesti  { \PASS } { \FAIL }
 \token_if_macro:NTF Z              { \FAIL } { \PASS } 

 % True was set to relax
 \token_if_eq_meaning:NNTF \my_sometest\relax { \PASS } { \FAIL }
 
 \ExplSyntaxOff
\end{texexample}

Notice the unusual syntax for \cs{ifx} which is named \cs{token_if_eq_meaning:NN}. Also note that Example~\ref{ex:assertt}, uses an assertion style where all tests must return true (\mbox{\PASS}). If you have a lot
of tests in a test file, it is easier to spot what is failing. See below where I redefined the token \enquote{Z} as an active
character and then defined a macro with it. Our test file will then clearly show the test failing. Just a small reminder to turn a character into a macro, you need to set it first to |\active| and then define it. Here is the test file again. 

\begin{texexample}{Test if is a macro}{ex:assertt1}
\ExplSyntaxOn

% \define Z
\group_begin:
\catcode `\Z = \active
\cs_set:Npn  Z {hello~}
% This is a common problem in LaTeXe. 
% \sometest is let tto |\relax| in a csname
 \csname my_sometest\endcsname
 
 % traditional definition using a csname and 
 % \expandafter
 \expandafter\def\csname my_sometesti\endcsname{}
 
 % All tests must pass
 \token_if_macro:NTF \par           { \FAIL } { \PASS } 
 \token_if_macro:NTF \minipage      { \PASS } { \FAIL } 
 \token_if_macro:NTF \my_sometest   { \FAIL } { \PASS }

 \token_if_macro:NTF \my_sometesti  { \PASS } { \FAIL }
 \token_if_macro:NTF Z              { \FAIL } { \PASS } 

 % True was set to relax
 \token_if_eq_meaning:NNTF \my_sometest\relax { \PASS } { \FAIL }
 \group_end:
 \ExplSyntaxOff
\end{texexample}


\bigskip

\begin{question}
It is recommended that you code these exercises as MWEs and try and not refer to the source3 manual,
during your first attempt. Namespace any macros you have to develop as part of the tasks below
with the prefix |yourname|.
\begin{tasks}
\task Define four macros and using \cs{token_if_macro:NTF} typeset a word.
\task Test the meaning of the four macros.
\end{tasks}
\end{question}

\subsection{Test if a control sequence is primitive}

One of the advantages of \latex3 is that it provides new names for all the primitives. This enables
one to check if a primitive has been redefined and to provide suitable tests and replacements.

 If it is a primitive we can find out, using yet another boolean construction \docAuxCommand*{token_if_primitive:NTF}  We can also check its meaning. It is interesting to note that \docAuxCommand*{par} is not a macro. Interestingly we can view what \tex does when we say |\csname somecs\endcsname|. It justs sets it equal to |\relax|. 
 
 Again this is important in parsing and in automating the generation of commands. For example  in the |phd| package, we allow for a key value to be entered either as a control sequence for example, |\Large| or simply as a |large|. A test could be provided before further processing such type of input.

\begin{texexample}{Test if is a macro}{ex:ifprimitive1}
\ExplSyntaxOn
\makeatletter
\token_to_meaning:N \par\\
\token_to_meaning:N \toks
\token_if_primitive:NTF \par       { \PASS } { \FAIL }\\
\token_if_primitive:NTF \@@par     { \PASS } { \FAIL }\\
\token_if_primitive:NTF \tex_par:D { \PASS } { \FAIL }
\makeatother
\ExplSyntaxOff

\end{texexample}

Example~\ref{ex:ifprimitive1} can be used to test if a primitive has been redefined (this can be important for your code and to restore its meaning if necessary or issue an error message.  Another test which is available is to check if a token is a macro. 


\begin{texexample}{Test if a cs is primitive}{ex:primitive}
\ExplSyntaxOn
\group_begin:
\makeatletter
% LaTeX2e normally defines this as @par. 
% Use \par in a group to test.
\def\par{\let\par\@@par\par}
\token_if_primitive:NTF \@par { \PASS } { \FAIL }
\makeatother
\group_end:
\ExplSyntaxOff
\end{texexample}

\subsection{Test for category codes}

The next set of available commands are helper functions equivalent to the output of |\ifcat| 

\begin{docCommand} {token_if_group_begin:NTF} {\meta{token} \marg{true code} \marg{false code}}
Tests if \meta{token} has the category code of a begin group token (\{) when normal TEX
category codes are in force). Note that an explicit begin group token cannot be tested in
this way, as it is not a valid N-type argument. To test it you have to use |\c_group_begin_token|. This is mostly
used in conjuction with |futurelet| type constructions and or parsing.
\end{docCommand}


\begin{texexample} {Test if group begin} {ex:ifgroubbegin}
\ExplSyntaxOn
 \token_if_group_begin:NTF \c_group_begin_token { \PASS } { \FAIL }
 \token_if_group_end:NTF   \c_group_end_token   { \PASS } { \FAIL }\par
 \the\catcode`{
\ExplSyntaxOff
\end{texexample}

Behind the scenes |expl3| uses the |\ifcat| primitive to test the token against the catcode values. Constructions for all categories are available and summarized in the test below rather than described.
\begin{texexample} {Test if group begin} {ex:ifgroubbegin}
\ExplSyntaxOn
 \token_if_group_begin:NTF \c_group_begin_token { \PASS } { \FAIL }
 \token_if_group_end:NTF   \c_group_end_token   { \PASS } { \FAIL }\par
 \token_if_alignment:NTF   \c_alignment_token   { \PASS } { \FAIL }\par
 \token_if_parameter:NTF   \c_parameter_token   { \PASS } { \FAIL }\par
\ExplSyntaxOff
\end{texexample}

Use the constant form of these tokens to avoid errors and to make the code more readable.

The module is feature rich, with too many functions to remember easily. If your code needs to deal
with too many changes of catcodes, lccodes and the like, you will have to study it carefully.



\section{LaTeX3 Futurelet type functions}

In Chapter Futurelet, we spend considerable effort to understand how \tex’s futurelet macro works. There is often a need to look ahead at the next token in the input stream while leaving
it in place. This is handled using the “peek” functions. The generic \docAuxCommand*{peek_after:Nw} is
provided along with a family of predefined tests for common cases. As peeking ahead does
not skip spaces the predefined tests include both a space-respecting and space-skipping
version.

\begin{texexample}{Peek ahead ignoring spaces} {}
\ExplSyntaxOn
\peek_catcode_remove_ignore_spaces:NTF =  
    { 
      \PASS  
      \token_if_letter:NTF
          {l_peek_token ~= ~\token_to_meaning:N \l_peek_token \\  } 
          {   }
    } 
    { \FAIL }  
 = abcde \\
\ExplSyntaxOff
\end{texexample}

Most applications would require to recursively pick up tokens from the input stream and only terminated once a special token is found. This is the most powerful method to parse input strings and create really powerful functions. 

You will understand better if we hide the code in a function.

\begin{texexample}{Peek ahead ignoring spaces} {ex}
\ExplSyntaxOn
\cs_new:Npn \checkletter #1 {
\peek_catcode_remove_ignore_spaces:NTF #1  
    { 
      \PASS  
      \token_if_letter:NTF
          {l_peek_token ~= ~\token_to_meaning:N \l_peek_token \\  } 
          {   }
    } 
    { \FAIL } }

\checkletter {=} =abcde \par
\checkletter {A} Abcde \par
\ExplSyntaxOff
\end{texexample}

\begin{texexample}{Peek ahead ignoring spaces} {}
\ExplSyntaxOn
\cs_set:Npn \check_letter_and_removeall #1 {
\peek_catcode_remove_ignore_spaces:NTF #1  
    { 
      \PASS  
      \removeallaux:w  
    } 
   { \FAIL } 
 }

\cs_set:Npn \removeallaux:w #1; { removed~#1~ }

\check_letter_and_removeall {W}  W 12pt; \par
\ExplSyntaxOff
\end{texexample}

In the next example we will try and remove from the input stream recursively any |;|.
Tests if the next non-space token in the input stream has the same character code as
the test token (as defined by the test \cs{token_if_eq_charcode:NNTF}). Explicit and
implicit space tokens (with character code 32 and category code 10) are ignored and
removed by the test and thehtokeni is removed from the input stream if the test is true.
The function then places either the htrue codei or hfalse codei in the input stream (as
appropriate to the result of the test).
\begin{texexample}{ex:recursivefl}  { }                            
\ExplSyntaxOn

\cs_set:Npn \remove_colon: #1 {
   \peek_charcode_remove_ignore_spaces:NTF#1 
   {
    \TRUE
    \meaning #1 \par
    \peek_charcode_remove_ignore_spaces:NTF#1
   } 
   {
    \meaning#1
    \FALSE
     
   }
}

\remove_colon:;;;;;;;!
\ExplSyntaxOff
\end{texexample}

\subsection{Using higher functions}

\cs{peek_catcode_collect_inline:Nn}\Arg{test token}\Arg{inline code}. Collects and removes tokens from the input stream until finding a token that does not match the \meta{test token}. The colected tokens are passed to the \meta{inline code} as |#1|.   

In the example we collect tokens until we reach the comma (\ExplSyntaxOn\char_value_catcode:n{`\,}\ExplSyntaxOff) character which does not have the same category code as Z ({\ExplSyntaxOn\char_value_catcode:n{`\Z}\ExplSyntaxOff)}. We store the results in the |\grubber|.



\begin{texexample}{Collect tokens}{}
\ExplSyntaxOn

\cs_set:Npn \decorate_and_remove {
    {\space\bfseries \tl_use:N \g_tmpa_tl}
   }

\cs_set:Npn \collect_letters {
  \peek_catcode_collect_inline:Nn Z {\tl_put_right:Nn \g_tmpa_tl {##1}}
}

\cs_set:Npn \collect_others  {
  \peek_catcode_collect_inline:Nn ; {\tl_put_right:Nn \g_tmpb_tl {##1}}
}

\cs_set:Npn \maybe_first_is_surname:w #1   
  {  
    % Clear any contents from the token list 
    \tl_clear:N \g_tmpa_tl
    
    % Collect any letters until catcode is other
    \cs_set:Npn \result {\peek_catcode_collect_inline:Nn Z {\tl_put_right:Nn \g_tmpa_tl {####1}}#1}
    
    \def\removecomma##1##2;;{
      #2
    }
    \removecomma\result;;
    \tl_if_empty:NTF \g_tmpa_tl {\TRUE}
        {}
  }
  
\maybe_first_is_surname:w {Lazarides, Yiannis} 
{\bfseries \tl_use:N \g_tmpa_tl}
    
{\color{red}\tl_use:N \g_tmpb_tl}

%\maybe_first_is_surname:w {Lazarides Yiannis} ;

%\maybe_first_is_surname:n { Lazarides, Yiannis\par }
%
%\maybe_first_is_surname:n { Yiannis;Lazarides   }\par

\ExplSyntaxOff
\end{texexample}






















\chapter{The xtemplate package of LaTeX3 and how to use it effectively}

Back in 1999 Frank Mittelbach together with David Carlisle and Chris Rowley published a paper in TUGboat describing their ideas of  \enquote{New Interfaces for \latex Class Design.} 
 
 \begin{latexquote}
 Traditional \latex class files typically implement one
fixed design via ad hoc, and often low-level, \latex
code. This style of implementation makes it much
harder than is either desirable or necessary to produce
classes that implement a specific visual design.
Moreover, the construction of such classes typically
involves a lot of work that is essentially programming
and thus does not live easily with the declarative
kind of design specification for a document (or
range of documents) that would be produced by a
professional typographic designer.
\end{latexquote}

The \emph{declarative kind} of design specification for a document, mentioned by the authors has been the holy grail of \latex for sometime. With the proliferation of key value packages it came closer to fruition and my own work in the |phd| package had this goal as one of its primary objectives. The \pkg{xtemplate} is at a much lower level than the phd package and I have struggled in my head as to how to integrate the two, so far unsuccessfully. There are very few articles on |xtemplate| but a good introductory one is \emph{Some notes on templates} by  Lars Hellström’s and which was published in TUGboat. The \pkgname{xgalley} still under develpment makes use of templates extensively and is worth to have a good look at the code.

\section{Objects, templates and instances}

\subsection{Object types}

An \emph{object type} sometimes termed \enquote{object} is an abstract idea of a document element that has a fixed number of arguments corresponding to the information from the document author that it is representing.  A sectioning object, for example, might take three inputs: \enquote{title}, \enquote{short title}, and \enquote{label}.

\begin{docCommand} {DeclareObjectType} { \meta{object type} \meta{no of args}}
This function defines an \meta{object type} taking \meta{number of arguments}, where the \meta{object type} is an abstraction as discussed above. For example:
   \begin{verbatim}
     \DeclareObjectType{chapter}{3}
   \end{verbatim}
This would create an object type \enquote{sectioning}, where each use of that object type will need three arguments.   
\end{docCommand}

The object type doesn’t do much when it is declared. It just records the name and the number of arguments in a property store, as can be seen in the code below, extracted from the |xtemplate| package:

\begin{teXXX}
\cs_new_protected:Npn \@@_declare_object_type:nn #1#2
  {
    \int_set:Nn \l_@@_tmp_int {#2}
    \bool_if:nTF
      {
        \int_compare_p:nNn {#2} > \c_nine ||
        \int_compare_p:nNn {#2} < \c_zero
      }
      {
        \msg_error:nnxx { xtemplate } { bad-number-of-arguments }
          {#1} { \exp_not:V \l_@@_tmp_int }
      }
      {
        \msg_info:nnxx { xtemplate } { declare-object-type }
          {#1} {#2}
        \prop_gput:NnV \g_@@_object_type_prop {#1}
          \l_@@_tmp_int
      }
  }
\end{teXXX}

    
\subsection{Templates}

Once an object is created a \emph{template} can be used to generalize a design solution for representing the information of a specific object type. A template has a name and a parent object. There are two important parts to a template:

\begin{enumerate}
\item The parameters it takes to vary the design it is producing.
\item The implementation of the design.
\end{enumerate}

The template definition is split into two parts using \cs{DeclareTemplateInterface} and \cs{DeclareTemplateCode}.
We will first examine \docAuxCommand*{DeclareTemplateInterface}.

\begin{docCommand} {DeclareTemplateInterface} {\meta{object} \marg{key value list}}
The key value list is of the form:
\begin{verbatim}

    key1 : key type1,
    key2 : key type2,
    key3 : key type3  = default3,
    key4 : key type4  = default4,
\end{verbatim}

An important item to note is that spaces in key names are ignored so writing |my key| and |mykey| is one and the same. 

Essentially the |DeclareTemplateInterface| is a command that initializes the list of key values applicable to the object type. The key list must be the same as declared for |object|. I didn’t know the \latex guys were fans of Java. The code pattern here is very similar. You declare an object and then its interface. Once this is done then the code can be developed. The key types available are shown in Table~\ref{tab:key-types}, which has been extracted from the documentation.

\end{docCommand}

\begin{docKey}{boolean} { boolean type for template interface}{\meta{true or false}}
a true or false value
\end{docKey}

   \begin{table}
     \centering
     \begin{tabular}{>{\ttfamily}ll}
       \toprule
       \multicolumn{1}{l}{Key-type} & Description of input \\
       \midrule
       boolean    & \texttt{true} or \texttt{false}            \\
       choice\marg{choices}
         & A list of pre-defined \meta{choices} \\
       code
         & Generalised key type: use |#1| as the input to the key \\
       commalist  & A comma-separated list                        \\
       function\marg{$N$}
         & A function definition with $N$ arguments
          ($N$ from $0$ to $9$) \\
       instance\marg{name}
                      & An instance of type \meta{name} \\
       integer    & An integer or integer expression            \\
       length     & A fixed length                              \\
       muskip    & A math length with shrink and stretch components \\
       real         & A real (floating point) value               \\
       skip         & A length with shrink and stretch components \\
       tokenlist  & A token list: any text or commands          \\
       \bottomrule
     \end{tabular}
     \caption{Key-types for defining template interfaces with
       \cs{DeclareTemplateInterface}.}
     \label{tab:key-types}
   \end{table}
   
\begin{texexample}{xtemplate short example}{}
\DeclareObjectType{obj}{0}
\DeclareTemplateInterface{obj}{tmpt1}{0}
{
  section-name: tokenlist = section,
  section-numbering: tokenlist =Roman,
  section-color: tokenlist = blue,
}
\end{texexample}

The |\DeclareTemplateInterface| part of the code is just a macro, whose fourth argument is written in a funny way.
We could have just written it as:

\begin{teXXX}
\DeclareTemplateInterface{obj}{tmp1}{0}{section-numbering:tokenlist=arabic, ... }
\end{teXXX}

A confusing aspect of the |templates| package is how the code part is defined. Here for each
key declared in the \docAuxCommand{DeclareTemplateInterface} you will need to allocate it an appropriate
macro. This works like in normal \latex keys. 

\begin{texexample}{The template code}{}
\ExplSyntaxOn
\DeclareTemplateCode{obj}{tmpt1}{0}
{
  section-name         = \sectionname,
  section-numbering = \numberingtype, 
  section-color = \colorname,
  }
{
% the implementation part
\AssignTemplateKeys
  \sectionname\ ~ 
  {\cs:w\numberingtype\cs_end: {section}\scan_stop:}\\
}

\DeclareInstance {obj}{inst} {tmpt1}{section-numbering = roman}
\UseInstance{obj}{inst}
\DeclareInstance {obj}{inst2}{tmpt1}{section-name=SECTION,
                                                      section-numbering = arabic}
\UseInstance{obj}{inst2}
\ExplSyntaxOff
\end{texexample}

The implementation part is the part that starts with |\AssignTemplateKeys|. Here we can use the values stored in the key functions to do something useful. Again here, remember, we are using macros and |\AssignTemplateKeys| is a macro with five arguments. This is defined by the package as:

 \begin{verbatim}
   \@@_declare_template_code:nnnnn {#1} {#2} {#3} {#4} {#5}
\end{verbatim}

Looking back at our simple example, the formatting of the section number in |arabic| or |roman| did not make any particular checks for validity. This would have been better programmed as a |choice| key with all the choice words allowed hardcoded in the implementation part. 

\begin{teXXX}
 section-numbering  : choice { arabic, Roman, roman, words, Words, alph, Alph } = arabic
\end{teXXX}                                 

The |choice| key type implements multiple choice input. At the interface level only the list of valid choice is needed:

\begin{teXXX}
\DeclareTemplateInterface{ foo }{ bar }{ 0 }
    { key-name : choice { A, B, C } }
\end{teXXX}

Note that the choices are given in a comma delimited list (which must therefore be wrapped in braces). A default value can also be given:


\begin{teXXX}
\DeclareTemplateInterface{ foo }{ bar }{ 0 }
    { key-name : choice { A, B, C } = A }
\end{teXXX}

\begin{teXXX}
 section-numbering      =
      {
        roman =
          \cs_set_nopar:Npn \numberingtype:
            {
              ... code
            },
        roman  =
          \cs_set_nopar:Npn \numberingtype:
            {
              ... code 
            }
      },
\end{teXXX}

\begin{texexample}{The template code}{ex:unknownkey}
\ExplSyntaxOn
\DeclareTemplateInterface{obj}{section}{0}
{
  section-name: tokenlist = section,
  section-numbering: choice  {arabic, Roman, roman}=roman,
  section-color: tokenlist = blue,
}
\DeclareTemplateCode{obj}{section}{0}
{
  section-name         = \sectionname,
  section-numbering = 
     {
       roman     =   \cs_set_nopar:Npn  \numberingtypei: { \roman{section} \scan_stop: },
       Roman     =  \cs_set_nopar:Npn   \numberingtypei: { \Roman{section} \scan_stop: },
       arabic      =   \cs_set_nopar:Npn  \numberingtypei: { \arabic{section} \scan_stop: },
       unknown =   \cs_set_nopar:Npn  \numberingtypei: { ERROR~unknown~key }
     },  
  section-color = \colorname,
  }
{
% the implementation part

\AssignTemplateKeys
  \sectionname\ ~ 
  \numberingtypei: \par
}

\DeclareInstance {obj}{inst} {section}{section-numbering = roman}
\UseInstance{obj}{inst}
\DeclareInstance {obj}{inst2}{section}
    {
        section-name=SECTION,
        section-numbering = arabic
     }
     
\DeclareInstance {obj}{inst3}{section}
    {
        section-name=SECTION,
        section-numbering = Arabic
     }  
                                                          
\UseInstance{obj}{inst2}
\UseInstance{obj}{inst3}
\ExplSyntaxOff
\meaning\numberingtypei
\end{texexample} 

In Example~\ref{ex:unknownkey} we have introduced the |choice| type key. This also takes an option
\option{unknown}. If a value is given that has not been previously been defined, then it essentially acts as
an |else| branch to the code and executes the definition given, in our example just typesets an 
error message. The code in the example at this stage is very simplistic and it has not been abstracted properly. The example is simply here to demonstrate the various types of keys available. The |length| and the |skip| keys 
accept dimensions or skips and are simply coded. The |function| type of key can be very useful in many situations. 

In the next example we will add some skips before and after the section, as well introduce a boolean to choose betwen a block heading or an inline heading. 


\begin{texexample}{The template code}{ex:unknownkey}
\ExplSyntaxOn
\DeclareTemplateInterface{obj}{headings}{0}
{
  name: tokenlist = section,
  numbering: choice  {arabic, Roman, roman,none} = roman,
  color: tokenlist = blue,
  display: boolean = true,
  aboveskip: skip=10pt,
  belowskip: skip=10pt,
  }
 
\bool_new:N \l_display_bool

\DeclareTemplateCode{obj}{headings}{0}
{
  name         = \sectionname,
  numbering = 
     {
       roman     =  \cs_set_nopar:Npn \numberingtypei: {\roman{section}},
       Roman     = \cs_set_nopar:Npn \numberingtypei: {\Roman{section} },
       arabic      =  \cs_set_nopar:Npn \numberingtypei: {\arabic{section}},
       none       =  \cs_set_nopar:Npn \numberingtypei: {},
       unknown =  \cs_set_nopar:Npn\numberingtypei: {ERROR~unknown~key }
     },  
  color = \colorname,
  display = \l_display_bool,
  aboveskip = \l_tmpa_skip,
  belowskip = \l_tmpb_skip,
 }
 {
% the implementation part
  \AssignTemplateKeys
  \par\skip_vertical:N  \l_tmpa_skip
  \sectionname\ ~ 
  \numberingtypei: \par
}

\DeclareInstance {obj}{part} {headings}
  { name      = PART,
    numbering = Roman
  }
  
\DeclareInstance {obj}{section}{headings}
  { name      = SECTION,
    numbering = arabic
  }
  
\DeclareInstance {obj}{chapter}{headings}
  {
    name      = CHAPTER,
    numbering = Roman, 
    aboveskip = 5pt
  }    
                                                                                                                   
\UseInstance{obj}{part}
\UseInstance{obj}{section}
\UseInstance{obj}{chapter}

\ExplSyntaxOff

\end{texexample}   

Although the |xtemplate| manual recommends that booleans should be preferred over 
|choice| keys, but from a user interface point of view |choice| keys are more powerful. One can define
key variations such as (true, false, on, off, none) and other similar values. 

Another few notes for readers coming from \latexe. The  |\skip_vertical:N| is the
\tex |\vskip|.  There are also some questions arising from the approach, which can affect the
coding. The format and the flexibility of the final settings offered for the user. From a programmer’s
perspective the view is different.  We could view the three basic elements of a heading at a more
elementary level, consisting of a number, a label and a title. Consider the |HTML| element |<span>|,
how can we make an equivalent in \latex3? There are many approaches one could think of, but this time
having covered the basics of how to program templates, we will start from the Designer Level. The Designer
wishes to define commands that are normally used inline and are used for different type of purposes. Such functions can typeset words that are emphasized, others that represent computer code and are typeset verbatim, acronyms and abbreviations. These also can automatically add themselves to an index etc.

Of course we don’t want to offer the user a command called |\span| where he needs to type |\span[emph]|. What we want to offer the user is a series of commands. However at the Design Level, these can be created by means of templates.

\begin{texexample}{A template for spans}{ex:span}
\ExplSyntaxOn
\DeclareObjectType{inlineobj}{1}
\DeclareTemplateInterface{inlineobj}{span}{1}
{
  font-face: tokenlist,
  font-shape: choice {italic, slanted, normal},
  font-weight: choice {bold, normal},
  font-color: tokenlist,
  quote: function 1,
}
\cs_set_nopar:Npn \quote_format:n#1 {\enquote{#1}}
\cs_set_nopar:Npn \quote_format_none:n#1 {#1}

\DeclareTemplateCode{inlineobj}{span}{1}
{
  font-face         =  \l_font_tl,
  font-shape = {
     italic     = \cs_set_nopar:Nn \afontshape: {\itshape},
     slanted = \cs_set_nopar:Nn \afontshape: {\itshape},
     normal = \cs_set_nopar:Nn \afontshape: {\upshape}
  },
  font-weight = {
     bold    = \cs_set_nopar:Nn \afontseries: {\bfseries},
     normal =\cs_set_nopar:Nn \afontseries: {\mdseries}
   },
  font-color = \l_tmpa_tl,  
  quote = \quote_format:n,
}
{
% the implementation part
  \AssignTemplateKeys
  \group_begin:
  \color\l_tmpa_tl
   \cs:w \l_font_tl \cs_end: 
   \afontshape:
   \afontseries: 
       \quote_format:n{\detokenize{#1}} 
   \group_end:
 }
 
\ExplSyntaxOff

\DeclareInstance {inlineobj}{docFunction}{span}
    {
        font-face=arial,
        font-shape=normal, 
        font-weight=bold,
        font-color=green!40!black
     }

\DeclareDocumentCommand\docFunction{ m }{
   \IfInstanceExistTF {inlineobj}{docFunction} 
     {\UseInstance{inlineobj}{docFunction}{#1}}
     {ERROR                                                   }
}

\DeclareInstance {inlineobj}{tn}{span}
    {
        font-face=ttfamily,
        font-shape=normal, 
        font-weight=normal,
        font-color=green!40!black
     }

\DeclareDocumentCommand\tn{ m }{
   \IfInstanceExistTF {inlineobj}{tn} 
     {\UseInstance{inlineobj}{tn}{#1}}
     {ERROR                                                   }
} 
   
The function \docFunction {get_string ( )} is used throughout to get a string in LuaTeX, where macros in text paragraphs are shown as \docFunction\mymacro in green.
\end{texexample}
\ExplSyntaxOn
\DeclareObjectType{inlineobj}{1}
\DeclareTemplateInterface{inlineobj}{span}{1}
{
  font-face: tokenlist,
  font-shape: choice {italic, slanted, normal},
  font-weight: choice {bold, normal},
  font-color: tokenlist,
  quote: function 1,
}
\cs_set_nopar:Npn \quote_format:n#1 {\enquote{#1}}

\DeclareTemplateCode{inlineobj}{span}{1}
{
  font-face         =  \l_font_tl,
  font-shape = {
     italic     = \cs_set_nopar:Nn \afontshape: {\itshape},
     slanted = \cs_set_nopar:Nn \afontshape: {\itshape},
     normal = \cs_set_nopar:Nn \afontshape: {\upshape}
  },
  font-weight = {
     bold    = \cs_set_nopar:Nn \afontseries: {\bfseries},
     normal =\cs_set_nopar:Nn \afontseries: {\mdseries}
   },
  font-color = \l_tmpa_tl,  
  quote = \quote_format:n,
}
{
% the implementation part
  \AssignTemplateKeys
  \group_begin:
  \color\l_tmpa_tl
   \cs:w \l_font_tl \cs_end: 
   \afontshape:
   \afontseries: 
       \quote_format:n{\detokenize{#1}} 
   \group_end:
 }
 
\ExplSyntaxOff

\DeclareInstance {inlineobj}{docFunction}{span}
    {
        font-face=arial,
        font-shape=normal, 
        font-weight=bold,
        font-color=green!40!black
     }

\DeclareDocumentCommand\docFunction{ m }{
   \IfInstanceExistTF {inlineobj}{docFunction} 
     {\UseInstance{inlineobj}{docFunction}{#1}}
     {ERROR}
}

\DeclareInstance {inlineobj}{tn}{span}
    {
        font-face=ttfamily,
        font-shape=normal, 
        font-weight=normal,
        font-color=green!40!black
     }

\DeclareDocumentCommand\tn{ m }{
   \IfInstanceExistTF {inlineobj}{tn} 
     {\UseInstance{inlineobj}{tn}{#1}}
     {ERROR}
} 
   
With the last example I have introduced also the conditional \docAuxCommand*{IfInstanceTF} that provides a test if the template exist. In our case typesets |ERROR| if the instance does not exist.

\begin{docCommand}{IfInstanceExistTF}{\marg{object type} \marg{instance} \marg{true code} \marg{false code}}
Tests if the named \meta{instance} of an \meta{object type} exists, and then inserts the appropriate code into the input stream. 
\end{docCommand}


\section{Summary}

\latex3’s \pkgname{xtemplate} offers a flexible and robust way to enable  declarative
setting of typographical parameters for a document. For the package writer it has one major advantage. It can be used to expose an API through which users communicate with the package's important commands. I would go as far as to say that packages should only expose an API and no settings should occur during loading. This can reduce both errors during package loading with different key values, as well as perhaps stop the race at the |AtBeginDocument|. 

If you want to study a longer non-trivial example you can have a look at the \pkgname{xfrac} package. In this package Will Robertson used |xtemplate| extensively. He also used some of the more esoteric commands of the package and is worth studying the code, before you start using |xtemplate| in your package.

All the functionality made available by the package can easily be provided by |pgfkeys| and the creation of some custom commands. This will remain as a competitor to the package until some of the limitations of |xparse| are addressed. The main limitation currently from my point of view is the addition of custom types  in a similar fashion to pgfkeys \emph{handlers}, although the \tn{code} and the \tn{function} types can be used in this respect.





%\def\Go{\texttt{go}\xspace}

\lstdefinelanguage{Golang}%
  {morekeywords=[1]{package,import,func,type,struct,return,defer,panic,%
     recover,select,var,const,iota,},%
   morekeywords=[2]{string,uint,uint8,uint16,uint32,uint64,int,int8,int16,%
     int32,int64,bool,float32,float64,complex64,complex128,byte,rune,uintptr,%
     error,interface},%
   morekeywords=[3]{map,slice,make,new,nil,len,cap,copy,close,true,false,%
     delete,append,real,imag,complex,chan,},%
   morekeywords=[4]{for,break,continue,range,goto,switch,case,fallthrough,if%
     else,default,},%
   morekeywords=[5]{Println,Printf,Error,},%
   sensitive=true,%
   morecomment=[l]{//},%
   morecomment=[s]{/*}{*/},%
   morestring=[b]',%
   morestring=[b]",%
   morestring=[s]{`}{`},%
   }
   
\lstset{ % add your own preferences
    frame=single,
    basicstyle=\footnotesize\verbatimfont,
    keywordstyle=\color{red},
    numbers=left,
    numbersep=5pt,
    showstringspaces=false, 
    stringstyle=\color{blue},
    tabsize=4,
    language=Golang % this is it !
}
   
\chapter{Getting Started}

\section{Workspaces}

The \Go tool is designed to work with open source code maintained in public repositories,
such as |github.com|. Although you don't need to publish your code, the suggested method
as to how to set your workspace is the same whether you do or not.\footnote{The Go rendering is done with listings and \protect\url{https://github.com/julienc91/listings-golang}}

Go code must be kept into a \emph{workspace}. A workspace is a directory hierarchy with
three directories at its root:

\begin{itemize}
\item |src| contains Go source files organized into packages (one package per directory)
\item |pkg| contains package objects, and
\item |bin| contains executable commands
\end{itemize}

The \Go tool builds source packages and installs the resulting binaries to the |pkg| and
|bin| directories.

\begin{minted}{bash}
bin/
    hello                          # command executable
    outyet                         # command executable (*@\label{outyet}@*)
pkg/
    linux_amd64/
        github.com/golang/example/
            stringutil.a           # package object
src/
    github.com/golang/example/
        .git/                      # Git repository metadata
	hello/
	    hello.go                     # command source
	outyet/
	    main.go                      # command source
	    main_test.go                 # test source
	stringutil/
	    reverse.go                   # package source
	    reverse_test.go              # test source
\end{minted}	    

This workspace contains one repository (\texttt{example}) comprising two commands (\texttt{hello} and \texttt{outyet}) and one library (\texttt{stringutil}).

A typical workspace would contain many source repositories containing many packages and commands. Most Go programmers keep all their Go source code and dependencies in a single workspace.\ref{outyet}

Commands and libraries are built from different kinds of source packages.
Although at first you might think having a set way to organize your workspace is restricting, using a fixed file layout for builds means less configuration. As a fact it also means no configuration. No Makefile, no build.xml. With everyone in the community using the same layout, it makes it easier to
share code and helpd build the Go community.


\section{The \texttt{GOPATH} environment variable}

The \texttt{GOPATH} environment variable specifies the location of your workspace. It is likely the only environment variable you'll need to set when developing Go code.

To get started, create a workspace directory and set |GOPATH| accordingly. Your workspace can be located wherever you like, but we'll use |$HOME/go| in this document. Note that this must not be the same path as your |Go| installation.



For convenience, add the workspace's bin subdirectory to your PATH:

\begin{teX}
$ export PATH=$PATH:$GOPATH/bin
\end{teX}

\begin{minted}{batch}
REM modified version of brainman's batch file at https://groups.google.com/forum/#!topic/golang-nuts/QVPKm7pbhds
 
REM setting goroot to my go installation folder.
set GOROOT=C:\go
 
REM setting gopath to my go playground folder
set GOPATH=C:\Users\****\goplayground
Set GOBIN=%GOPATH%\bin
set PATH=%PATH%;c:\go\bin;%GOBIN%
 
REM finally make my playground source folder as my PWD, write your hello.go here and execute "go install" and check GOBIN folder for exe.
cd %GOPATH%\src
CMD
\end{minted}



\section{Hello World at last}

To compile and run a simple program, first choose a package path (we'll use |github.com/user/hello|) and create a corresponding package directory inside your workspace:

\begin{teX}
$ mkdir $GOPATH/src/github.com/user/hello
mkdir %GOPATH%\src\github.com\yannisl\hello
\end{teX}

Next create a file named |hello.go| inside the directory, containing the
following |Go| code.

\begin{lstlisting}
package main

import "fmt"

func main() {
    fmt.Println("Hello World!")
}
\end{lstlisting}

Now we are ready to install the program with the \Go tool.

\begin{teX}
go install github.com/yannisl/hello
\end{teX}

Note that you can run this command from anywhere on your system. The go tool finds the source code by looking for the |github.com/user/hello| package inside the workspace specified by |GOPATH|.

This command builds the hello command, producing an executable binary. It then installs that binary to the workspace's bin directory as hello (or, under Windows, hello.exe). In our example, that will be |$GOPATH/bin/hello|, which is |$HOME/go/bin/hello|.

The go tool will only print output when an error occurs, so if these commands produce no output they have executed successfully.

You can now run the program by typing its full path at the command line:

Or, as you have added |$GOPATH/bin| to your |PATH|, just type the binary name:

\begin{teX}
$ hello
Hello, world.
\end{teX}

\section{Add the source to git}

Since we will be using a source control system, this is also a good time
to initialize a repository, add the files, and commit your first change.
Again this is not necessary, but it is considered good practice.

\begin{minted}[fontsize=\footnotesize,linenos,escapeinside=VV]{bash}
echo # go >> README.md V\label{readme}V
git init
git add README.md 
git commit -m "first commit"
git remote add origin https://github.com/yannisl/go.git 
git push -u origin master
\end{minted}


We first use the |echo| command to create a |README.md| file in line~\ref{readme}. This is always good practice, not only for open source projects or projects
that have many collaborators, but also for your older self that might not remember
what the younger you did.
 
\section{Your first library}

Let's write a library and use it from the hello program.

Again, the first step is to choose a package path (we'll use github.com/user/stringutil) and create the package directory:

\begin{minted}[fontsize=\footnotesize,linenos,firstnumber=last]{bash}
$ mkdir $GOPATH/src/github.com/user/stringutil
\end{minted}

\section{Package names}

The first statement in a Go source file must be

\begin{minted}[fontsize=\footnotesize,linenos,firstnumber=last,style=friendly,escapeinside=VV]{go}
package name
\end{minted}

where \emph{name} is the package default name for imports. (All files
in a package must use the same \emph{name}.)

Executable commands must always use |package main|.

There is no requirement that package names be unique across all packages linked into a single binary, only that the import paths (their full file names) be unique.

\section{Testing}

|Go| has a lightweight test framework composed of the |go test| command
and the |testing| package. 

Yo can write a test by creating a file with a name ending in |_test.go|
that contains functions named |TestXXX| with a signature |(t *testing.T)|. The test framework runs each such function; if the function calls a failure function such as |t.Error| or |t.Fail|, the test is considered to have failed.



\begin{minted}{go}
package stringutil

import "testing"

func TestReverse(t *testing.T) {V\label{testing}V
	cases := []struct {
		in, want string
	}{
		{"Hello, world", "dlrow ,olleH"},
		{"Hello, 世界", "界世 ,olleH"},
		{"", ""},
	}
	for _, c := range cases {
		got := Reverse(c.in)
		if got != c.want {
			t.Errorf("Reverse(%q) == %q, want %q", c.in, got, c.want)
		}
	}
}
\end{minted}

The test at line~\ref{testing} can be run with |go test|:

\begin{teX}
$ go test github.com/user/stringutil
ok  	github.com/user/stringutil 0.165s
\end{teX}

As always, if you are running the go tool from the package directory, you can omit the package path:

\begin{teX}
$ go test
ok  	github.com/user/stringutil 0.165s
\end{teX}


\section{Remote packages}

An import path can describe how to obtain the package source code using a revision control system such as Git or Mercurial. The go tool uses this property to automatically fetch packages from remote repositories. For instance, the examples described in this document are also kept in a Git repository hosted at GitHub |github.com/golang/example|. If you include the repository |URL| in the package's import path, go get will fetch, build, and install it automatically:

\begin{minted}{bash}
$ go get github.com/golang/example/hello
$ $GOPATH/bin/hello
Hello, Go examples!
\end{minted}

If the specified package is not present in a workspace, go get will place it inside the first workspace specified by |GOPATH|. (If the package does already exist, go get skips the remote fetch and behaves the same as |go install|.)

After issuing the above go get command, the workspace directory tree should now look like this:

\begin{minted}{bash}
bin/
    hello                           # command executable
pkg/
    linux_amd64/
        github.com/golang/example/
            stringutil.a            # package object
        github.com/user/
            stringutil.a            # package object
src/
    github.com/golang/example/
	.git/                             # Git repository metadata
        hello/
            hello.go                # command source
        stringutil/
            reverse.go              # package source
            reverse_test.go         # test source
    github.com/user/
        hello/
            hello.go                # command source
        stringutil/
            reverse.go              # package source
            reverse_test.go         # test source
\end{minted}

This convention is the easiest way to make your Go packages available for others to use. The Go Wiki and godoc.org provide lists of external Go projects.

For more information on using remote repositories with the go tool, see go help \href{https://golang.org/cmd/go/\#hdr-Remote_import_paths}{importpath}.


\section{Installation on Windows WSL}


%https://golang.org/doc/install?download=go1.10.linux-amd64.tar.gz

%https://gist.github.com/nikhita/432436d570b89cab172dcf2894465753

%https://gist.github.com/stefanprodan/29d738c3049a8714297a9bdd8353f31c



This bash script installs Go lang tools in |/usr/local/go|, creates |$HOME/go| directory and sets |GOPATH| environment variable to |$HOME/go/bin|. It is an adapation from a gist by \href{https://gist.github.com/nikhita/432436d570b89cab172dcf2894465753}{nikhita}.

\begin{minted}{bash}
#!/bin/bash
set -e

GVERSION="1.7"
GFILE="go$GVERSION.linux-amd64.tar.gz"

GOPATH="$HOME/go"
GOROOT="/usr/local/go"
if [ -d $GOROOT ]; then
    echo "Installation directory already exists $GOROOT"
    exit 1
fi

mkdir -p "$GOROOT"
chmod 777 "$GOROOT"

wget --no-verbose https://storage.googleapis.com/golang/$GFILE -O $TMPDIR/$GFILE
if [ $? -ne 0 ]; then
    echo "Go download failed! Exiting."
    exit 1
fi

tar -C "/usr/local" -xzf $TMPDIR/$GFILE

touch "$HOME/.bashrc"
{
    echo '# GoLang'
    echo 'export PATH=$PATH:/usr/local/go/bin'
    echo 'export GOPATH=$HOME/go'
    echo 'export PATH=$PATH:$GOPATH/bin'
} >> "$HOME/.bashrc"
source "$HOME/.bashrc"
echo "GOROOT set to $GOROOT"

mkdir -p "$GOPATH" "$GOPATH/src" "$GOPATH/pkg" "$GOPATH/bin" "$GOPATH/out"
chmod 777 "$GOPATH" "$GOPATH/src" "$GOPATH/pkg" "$GOPATH/bin" "$GOPATH/out"
echo "GOPATH set to $GOPATH"

rm -f $TMPDIR/$GFIL
\end{minted}


You can run the one-line installer using this 

\href{https://github.com/udhos/update-golang/blob/master/update-golang.sh}{gist} as source:

\begin{minted}{bash}
cd $HOME
curl -s -L <GIST_RAW_URL> | sudo bash
\end{minted}

The |gist| raw |URL| can be found here.

After running the script, type exit to close the current session. Open a new bash session and run |go env| to verify the installation.





%\chapter{Tools}

\section{gofmt}

Some of Go's concept can be considered controversial. One such tool is |gofmt|. The Go project has 
a canonical program format, implemented by a program that will pick up any program and reformat it.
Every program in the Go source tree has been formatted with |gofmt|. The code review and revision control 
tools check that any new code is gofmt-formatted too. 

The most obvious benefit of using |gofmt| is that when you open an unfamiliar Go program, your brain doesn't get distracted, even subconsciously, about why that brace is in the wrong place; you can focus on the code, not the formatting. But there are many more interesting uses for gofmt. Gofmt can take any file in the Go source tree, parse it into an internal representation, and then write exactly the same bytes back out to the file. There are two reasons this works: the primary reason is that a lot of effort went into gofmt, but an important secondary reason is that gofmt only has to worry about one formatting convention, and we've agreed to accept that as the official one. There is an interesting blog post by Russ Cox at \href{http://research.swtch.com/gofmt}{Gofmt}, 
discussing the tool.

\begin{teX}
gofmt
from
for{
fmt.Println("    I feel pretty." );
       }
\end{teX}
       
to

\begin{teX}
for {
    fmt.Println("I feel pretty.")
}
\end{teX}

One final note. Being able to pick up a program and write it back out, preserving comments, is not an easy task in any language. Robert Griesemer deserves all the credit for the huge effort to make |gofmt| handle real programs so well.

\section{godoc}


\footnote{\protect\url{https://talks.golang.org/2014/hammers}}




%\makeatletter
\cxset{defaults/.style ={% 
    chapter title margin-top-width    =  0cm,
    chapter title margin-right-width  =  1cm,
    chapter title margin-bottom-width = 10pt,
    chapter title margin-left-width   = 0pt,
    chapter align                     = left,
    chapter title align               = left, %checked
    chapter name                      = CHAPTER,
    chapter format                    = block,
    chapter font-size                 = Huge,
    chapter font-weight               = bold,
    chapter font-family               = sffamily,
    chapter font-shape                = upshape,
    chapter background-color          = white,
  % chapter label    
    chapter color               = black,
    chapter number prefix             = ,
    chapter number suffix             = ,
    chapter numbering                 = arabic,
    chapter indent                    = 0pt,
    chapter beforeskip                = -3cm,
    chapter afterskip                 = 30pt,
    chapter afterindent               = off,
    chapter number after              = ,
    chapter arc                       = 0mm,
    chapter label background-color    = white,
    chapter label color               = black,
   % chapter afterindent               = on,
    chapter grow left                 = 0mm,
    chapter grow right                = 0mm,
    chapter rounded corners           = northeast,
    chapter shadow                    = fuzzy halo,
    chapter border-left-width         = 0pt,
    chapter border-right-width        = 0pt,
    chapter border-top-width          = 0pt,
    chapter border-bottom-width       = 0pt,
    chapter padding-left-width        = 0pt,
    chapter padding-right-width       = 10pt,
    chapter padding-top-width         = 10pt,
    chapter padding-bottom-width      = 10pt,
    %  
    chapter number color              = black,
    chapter number background-color   = white,
    chapter number font-size        = huge,
    chapter number font-weight      = bfseries,
    chapter number font-family      = sffamily,
    chapter number font-shape       = upshape,
    chapter number align            = Centering,
    %
    chapter title font-size        = Huge,
     chapter title font-weight      = bold,
     chapter title font-family      = sffamily,
     chapter title font-shape       = upshape,
     chapter title color            = black,
     chapter title background-color = white,
     }%
   }  
\makeatother     
%\makeatletter
%\cxset{toc image=\@empty,
%       chapter toc=true,
%       title beforeskip=1pt}
%
%\@specialfalse
%
%
%\renewcommand\stewart[2][]{%
%\fancypagestyle{fancy}{%
%\lhead{}\rhead{}
%\chead{}
%\cfoot{}
%\lfoot{}
%\rfoot{\thepage}
%\def\footrule#1{{\color{blue}%
%  \hrule width\paperwidth}\vskip3pt
%}
%
%\renewcommand{\headrulewidth}{0pt}
%\renewcommand{\footrulewidth}{0.4pt}}
%
%\clearpage
%
%\begin{tikzpicture}[remember picture,overlay]
%% Main shading block
%\node [xshift=5cm,yshift=-\paperheight] at (current page.north west)
%[text width=0.98\textwidth,text height=\paperheight, fill=thecream!30,rounded corners,above right]
%{};
%\node [xshift=6.5cm,yshift=-1.5cm-\soffsety] at (current page.north west)
%[text width=0.9\textwidth,below right]{\sffamily \bfseries \huge #2};
%
%\node [xshift=3cm,yshift=-1.5cm] at (current page.north west)
%[text width=3cm,align=center,minimum height=2.5cm, fill=blue,below right]
%{\[\text{\HHUGE\bfseries\sffamily\color{white}\thechapter}\]
%\par\vspace*{3pt}
%};
%
%\node [xshift=-0.2cm,yshift=-21.5cm] at (current page.north west)
%[text width=3cm,above right]%
%{\includegraphics[width=1.0\paperwidth]{\image@cx}};
%% second box left
%\node [xshift=3cm,yshift=-19.5cm] at (current page.north west)
%[text width=9cm,minimum height=2.5cm,inner sep=0.5em, fill=blue,below right]
%{\color{white}
%  \bfseries\sffamily \texti@cx
%};
%% Last block
%\node [xshift=6.5cm,yshift=-26cm] at (current page.north west)
%[text width=12cm,above right]
%{\textii@cx
%};
%\end{tikzpicture}
%\par
%\clearpage
%}





\cxset{steward,
  chapter numbering=arabic,
  chapter format = stewart,
  offsety=0cm,
  image= {./images/hine02.jpg},
  texti={When Lamport designed the original \LaTeX\ sectioning commands he did not provide a fully comprehensive interface for modifying their design. With current tools available improvements are much easier to program and this chapter provides the details.},
  textii={\precis{In this chapter we discuss a method that allows the production of fancy chapter headings and formatting, based on a set of key values. Central  to this process is the separation of content from presentation.
We also discuss the basic formatting tools that are available and how one can modify them to mould new book designs.}
 }
}


\chapter{Designing Chapter Headings}
\addtocimage{-12pt}{-20pt}{./images/tocblock-man-01.jpg}

\section*{Introduction}

A \textls*{crowded} first page is as unsightly as a crowded title page, wrote De Vinne in \emph{Modern Methods of Book Composition} in 1904.  Not much has changed since. A new chapter must make a good impression and must give an immediate signal that a different topic is going to be discussed. Traditionally chapter openings in LaTeX are an unimpressive and dry event. Our aim is to brighten it up a bit, while keeping true separation of content from presentation, but avoiding the pit traps of over ornamenting the design. A book is to be read and we should provide minimal ornamentation. \index[phdkeys]{chapter> ornamentation}

% \usepackage{array,tabularx}
%\newcolumntype{Y}{>{\raggedleft\arraybackslash}X}% see tabularx
%\tcbset{enhanced,fonttitle=\bfseries\large,fontupper=\normalsize\sffamily,
%colback=yellow!10!white,colframe=red!50!black,colbacktitle=thecodebackground,
%coltitle=black,center title,
%tabularx={X||Y|Y|Y|Y||Y},% this sets ’before upper’ and ’after upper’
%before upper app={Group & One & Two & Three & Four & Sum\\\hline\hline} }
%
%\begin{tcolorbox}[title=My table]
%Red & 1000.00 & 2000.00 & 3000.00 & 4000.00 & 10000.00\\\hline
%Green & 2000.00 & 3000.00 & 4000.00 & 5000.00 & 14000.00\\\hline
%Blue & 3000.00 & 4000.00 & 5000.00 & 6000.00 & 18000.00\\\hline\hline
%Sum & 6000.00 & 9000.00 & 12000.00 & 15000.00 & 42000.00
%\end{tcolorbox}

\begin{figure}[htbp]
\centering
\parindent=0pt
\fbox{\includegraphics[width=\textwidth]{metropolitan-spread}}
\par
\caption{A chapter opening from the Metropolitan Museum of Art publicaion, \textit{Assyrian Reliefs and Ivories} by Vaughn. E. Crawford et. al., 1980. The spread is simple and the chapters are not numbered. This is a common characteristic of many more recently published books.}
\end{figure}


What is to us now a common occurence with instant book-printing was not always so. The cost of illustrated books was a prime factor and as Tschichold wrote:
\begin{quotation}
In the area of book design, in the last few years a revolution has taken place, until recently recognized by only a few. but which now begins to influence a much wider range of action.
It means placing much greater emphasis on the appearance of the book and a wholly contemporary use of typographic and photographic means. Before the invention of printing, literature of that time was spread around by the mouth of the author himself or by professional bards. The books of the Middle Ages - like the "Mannessische Liederhandschrift" - had
\end{quotation}

The type of book you are writing and its contents will determine an appropriate design for chapter headings and the type of design and numbering if any for subsections. Here we are merely providing a mechanism to produce them. These methods can produce a mastepiece or an ugly piece of work. Some simple suggestions follow (from my observations of styles in books I like). In general you need to think what type of book you are developing. For example a novel, should be sectioned very carefully. Many books avoid marking of sections other than chapters totally, perhaps marking them just with a soft ornament such as three centered asterisks.

\section{Numbering of Sections}


In general books do not number sections beyond subsection. You can avoid them all together, if you are not going to reference the sections extensively. 

In works of fiction, authors sometimes number their chapters eccentrically, often as a metafictional statement. For example:
Seiobo There Below by László Krasznahorkai has chapters numbered according to the Fibonacci sequence.

The Curious Incident of the Dog in the Night-Time by Mark Haddon only has chapters which are prime numbers.

At Swim-Two-Birds by Flann O'Brien has the first page titled Chapter 1, but has no further chapter divisions.

God, A Users' Guide by Seán Moncrieff is chaptered backwards (i.e., the first chapter is chapter 20 and the last is chapter 1). The novel The Running Man by Stephen King also uses a similar chapter numbering scheme.
Every novel in the series A Series of Unfortunate Events by Lemony Snicket has thirteen chapters, except the final instalment (The End), which has a fourteenth chapter formatted as its own novel.

Mammoth by John Varley has the chapters ordered chronologically from the point of view of a non-time-traveler, but, as most of the characters travel through time, this leads to the chapters defying the conventional order.


\begin{pgfpicture}
\pgfpathmoveto{\pgfpointorigin}
\pgfpathlineto{\pgfpoint{1cm}{1cm}}
\pgfpathlineto{\pgfpoint{1cm}{0cm}}
\pgfusepath{fill}
\end{pgfpicture}




\begin{figure}[tbp]
\centering
\parindent=0pt
\fbox{\includegraphics[width=\textwidth]{fantasy-architecture}}
\par
\caption{A chapter opening from the Metropolitan Museum of Art publicaion, \textit{Assyrian Reliefs and Ivories} by Vaughn. E. Crawford et. al., 1980. The spread is simple and the chapters are not numbered. This is a common characteristic of many more recent books.}
\end{figure}


\begin{figure}[tbp]
\centering
\parindent=0pt
\fbox{\includegraphics[width=\textwidth]{fantasy-architecture-02}}
\par
\caption{A chapter opening from the Metropolitan Museum of Art publicaion, \textit{Assyrian Reliefs and Ivories} by Vaughn. E. Crawford et. al., 1980. The spread is simple and the chapters are not numbered. This is a common characteristic of many more recent books.}
\end{figure}


\section*{Use of Color}

The modern books that Tschilchod was discussing have long been overwhelmed by the appearance of larger, coffee book type of books. Our brains our now conditioned by branding and graphic design is everywhere. 

Once you have decided that the book is going to be a bit more colorfull, the choice of color will follow. The decision what to color will be an important one, which brings us to color theory. The history of color is perhaps as colorfull as the rest. Attempts to formalize and recognize order date back to Aristotle (384-322 bce) but began in earnest with Leonardo da Vinci (1452-1519) and have progressed ever since. Leonardo noted that certain colors intensify each other, discovering \textit{contrary} and \textit{complementary} colors. The first color wheel was invented by Britain's Sir Isaac Newton (1642-1727), who split white light into red, orange, yellow, green, blue, indigo and violet beams, then joined the two ends of the spectrum to form a circle showing the natural progression of colors. When Newton created the color wheel, he noticed that mixing two colors from opposite positions produced a neutral or \textit{anonymous} color.


\begin{figure}[htbp]
\parindent=0pt
\centering
\fbox{\includegraphics[width=\textwidth]{line-designs} }
\caption{Spread from \textit{Beautiful Geometry}, Eli Maor and Eugen Jost, Princeton Univeristy Press, 2014. A subtle coloring of the chapter heading, de-emphasizing the chapter number and coloring the chapter title. There is no chapter label. A dropcap with the same color starts the first paragraph. This style is easy to achive with the phd system.}
\end{figure}


\begin{figure}[htbp]
\parindent=0pt
\centering
\fbox{\includegraphics[width=\textwidth]{color-book01.jpg} }
\bigskip

\fbox{\includegraphics[width=\textwidth]{color-book02.jpg} }
\end{figure}

One would expect a book written for the sole purpose of describing color theory and its application to the Graphic Arts, is expected to be colorful. Note the de-emphasizing of the label and number. 

\begin{figure}[htbp]
\parindent=0pt
\centering
\fbox{\includegraphics[width=\textwidth]{color-book-03.jpg} }
The chapter heading label and number are almost invisible. The heading text, is typeset in large bold letters, shouting what is coming next. Not your typical scintific book\ldots
\bigskip

\fbox{\includegraphics[width=\textwidth]{color-book-04.jpg} }
\end{figure}

Advertizing people understand that they need to present the message of an advertizement loud and clear so as to catch the busy eye. A heading's message is the title description. Neither the label not the chapter if any are necessary to convey the message. The chapter heading is analogous to the stop at the end of a sentence. The brain gets a signal to absorb what was written before it and get ready for the next. The heading signals the end of a topic. One must not dwell on it.


\section{Contemporary Chapter Headings}

In the book \textit{China} the designer used both a chapter heading on a spread of two images, as well as repeated the chapter number on the text pages \ref{fig:threepage}. The images distill the message of the chapter, although the chapter subtitle is almost unreadable, dominated by the surrounding text. From a technical perspective, the chapter command must paint the two images, set the right type of heading for each page and then without increasing the counter, change the counter to one that displays the chapter number in words and then continue with typesetting the text. A careful choice of images is necessary for such chapters, as well as cropping the images to match the aspect ratio of the book pages. One also needs to be carefull for \latexe not to place any floats in between the page spreads. 

\begin{figure}[htbp]
\parindent=0pt
\centering
\fbox{\includegraphics[width=\textwidth]{beijing.jpg} }\par
\vfill

\fbox{\includegraphics[width=\textwidth]{beijing-01.jpg} }\par
%\fbox{\includegraphics[width=\textwidth]{pearl-river.jpg} }
\caption{A full page chapter spread.}
\label{fig:threepage}
\end{figure}

\begin{figure}[htbp]
\parindent=0pt
\centering
\fbox{\includegraphics[width=\textwidth]{beijing.jpg} }\par
\vfill

\fbox{\includegraphics[width=\textwidth]{beijing-01.jpg} }\par
%\fbox{\includegraphics[width=\textwidth]{pearl-river.jpg} }
\caption{A full page chapter spread.}
\label{fig:threepage}
\end{figure}


\clearpage



In Figure~\ref{fig:photospread} the bands are black, but position low on the page. The size of the pages are 9.69 \texttimes 11.42. The books sections are not numbered. Text i sbroken through inserts of bigger text. Many of the examples here are from
commercial nude photography books, as they tend to break with tradition. In the 1970s and 1980s, fashion photographers began to present a
new, confrontational image of the female body. The pioneer in this
respect was the German Helmut Newton (1920–2004). Newton’s
photographs of nudes were overtly sexual, with an undertone of
menace, and although his models tended to be depicted as part
of the social elite they were often placed, apparently caught out
in reportage style, in sordid environments engaged in fantasy and
fetish. His work made him highly influential in fashion photography,
though some of it was thought too highly sexual for American
magazines and appeared only in those published in Europe.


\begin{figure}[htbp]
\parindent=0pt
\includegraphics[width=\textwidth]{baetens-01.jpg} \par
\vfill\vfill\vfill\vfill
\includegraphics[width=\textwidth]{baetens-02.jpg}\par
\caption{Chapter spread and first pages after the chapter title which is on the right page of the chapter spread. From \textit{New Photography, Art and the Craft}, Pascal Baetens, DK Publications. }
\label{fig:photospread}
\end{figure}

In the 1980s, Newton undressed the dynamic and independent
female in a series called Big Nudes. In this series the women are
indeed naked and very tall, wearing nothing but makeup and high
heels. The Big Nudes were exhibited in the form of life-size prints
that were intended to provoke the viewer by showing self-confident
women who knew what they wanted and were very aware of their
beauty and sexuality



\chapter{Package Usage}

To use the package include it just like any other package:

\begin{teXXX}
\documentclass{book}
\usepackage{phd}
\cxset{style13}
\begin{document}
\chapter{Introduction}
\end{document}
\end{teXXX}

The command \docAuxCommand{cxset} sets the default style for the example to the style defined as \meta{style13}. The package currently offers  100 templates and numerous keys to manipulate them further. Styles are similar to \enquote{themes} used in web programming; they are a collection of keys that resemble in many ways \texttt{css}. Styles can have any names and I am sure as package usage increases and evolve,they will get better names. 

\section{Background}

Before describing in detail how to specify a new layout for headings, we offer an overview of how the task can be accomplished and the design philosophy behind the approach. 

Irrespective of the technique and tools used, the creation of new layouts can always be divided into the following three tasks: constructing a document from “layout bricks”, which we can term as “blocks” or “elements”; establishing the layout semantics of each block; and finally, creating a layout engine supporting any document constructed from such blocks.

\begin{description}
\item [Canned Layouts] At one end of the spectrum, the most accessible approach consists of picking, a canned layout, such as LaTeX itself and perhaps only provide rudimentary macros to manipulate it.
\item [Constraints] Constraints offer a middle ground between canned layouts and handwritten layout engines. Constraints are arguably the most widespread and successful layout programming technique. For, instance, the foundations of \tex are laid upon constraint. CSS, the ubiquitous web template language, also relies on constraints, although in a more restricted and indirect manner.
\end{description}

\subsection{Blocks and Elements}

We define an \emph{element} as a document block, that cannot be subdivided further. For example the chapter title element, is composed of the text of the chapter title. 

A \emph{block} on the other hand is can contain other blocks and or numerous elements. We can consider the chapter headings as \emph{blocks}, composed of three blocks the chapter, number and title. Each block is then composed of elements. Each element has properties and traits. One of these mandary properties is the name. 

Blocks are either \emph{configured} (all constraints are mandatory), or flexible (there are optional/alternative constraints). By bundling optional constraints, flexible blocks make their specification customizable by non-technical users. 

\subsection{Language semantics}

One of the aims of the syntax of the templates was to offer familiar terminology and to remove the use
of \tex macros as far as possible from templates. 
\medskip

{\parindent0pt

 \textit{section}| font-family=serif,|\\
 \textit{section}| font-size=LARGE,|\\
 \textit{section}| font-weight=bold,|\\
}

The restriction I imposed is problematic when dealing with fractions of linewidths and textwidths. So
at present we allow for example |title text-width=0.5\texwidth| or |title text-width=10cm| or any other valid units. Ideas for improvements can only come from user feedback in the future.

Some experimental ideas incorporated are:

\begin{verbatim}
title text-width = 0.5 text-width,
title text-width = 1.2 text-width,
\end{verbatim}

A better parser will need to be programmed for dimensions, which are all currently handled as etex |dimexpr|. 

The syntax must allows both for microtypography as well as macro-typographical features. The former would deal with mostly fonts, spacing and text justification, where the latter deals with layouts, borders shapes and the positioning of elements on the page and also reletively to other elements or blocks.

An advantage of this approach is that it also opens the possibility of parsing the text with a language other than \tex and translating the document to another format, such as |HTML| or |XML| either fully or partially. Next we will describe both the syntax as well as the usage of the settings.

\section{Chapter opening page}

The standard \latexe classes offer only two options to either open a chapter on an odd page or at any page. This package offers five alternatives:

\begin{docKey}[phd]{chapter opening}{=\meta{any, left, right, anywhere, ifafter}}{default none, initial=any}
For documents that are primarily to be read on the web, use |any| for normal books, use \textit{right}. Some templates that we provide use |any| and the examples use |anywhere| to enable us to display the heading at any position on the page.
\end{docKey}

\begin{decription}
\item [any] Opens a chapter at any page, either \textit{verso} or \textit{recto}.
\item [left] Opens a chapter on an even page
\item [right] Opens a chapter on a right page.
\item [anywhere] Opens a chapter at the point where the \cs{chapter} is typed.
\item [none] Alias for \marg{anywhere}.
\item [ifafter] Opens a chapter at the next page if the page has material that does not exceed a certain portion of \cs{textheight}.
\end{description}

\colorlet{theoption}{bgsexy}

To change a setting you just modify the value of the key \oarg{\option{chapter opening}} to one of the values described earlier. 

\begin{dispListing}
\cxset{chapter opening = anywhere}
\end{dispListing}
 
We use this key to print the many examples typesetting chapter heads that follow (see the example~\ref{ex:anywhere}).  


\begin{texexample}{title=Inline Chapter Example}{ex:anywhere}
\cxset{examplestyle/.style = {chapter format = block,
       chapter opening = anywhere,
       chapter name = CHAPTER, 
       %label
       chapter label font-family      = sffamily,
       chapter label color            = primary,
       chapter label background-color = white,
       % number
       chapter number font-family = sffamily,
       chapter number font-size = HUGE,
       chapter number color     = primary,
       chapter label align = centering,
       chapter number background-color = white,
       %title
       chapter title font-family = rmfamily,
       chapter title align = centering,
       chapter title background-color = bgsexy!15,
       chapter title before background-color=white}}
\cxset{examplestyle}       
\lorem
\chapter{Typography Example}
\lorem
\chapter{Another Chapter Heading}
\lorem
\end{texexample}


%\cxset{toc chapter = true}
\addtocounter{chapter}{-1}

Examples for other types of chapter openings follow in the rest of the documentation.

\subsection{Blank pages before chapters}

In the standard LaTeX book class when the \texttt{openany} option is not given or in the report class when the openright is given, chapters start at odd-numbered pages. This can cause a blank page to be printed. Some book designers prefer this page to be completely empty, without any headers or footers. This cannot be done with \lstinline{\thispagestyle} as this command will have to be issued on the \textit{previous} page. However by a suitable redefinition of the
\lstinline{\clearpage} this can be done automatically.
\medskip

\begin{teXXX}
\makeatletter
\def\cleardoublepage{\clearpage\if@twoside\ifodd\c@page\else
  \hbox{}
  \vspace*{\fill}
  \begin{center}
    This page left intentionally blank.
  \end{center}
  \vspace{\fill}
  \thispagestyle{empty}
  \newpage
  \if@twocolumn\hbox{}\newpage\fi\fi\fi}
\makeatother
\end{teXXX}


This is achieved easily by setting the following options:
\bigskip

\begin{tcolorbox}
\lstinline{chapter blank page=empty}\par
\lstinline{chapter blank page text=Some text.}\par
\lstinline{chapter blank page=plain}\par
\end{tcolorbox}
\medskip



The last one refers to a \lstinline!\thispagestyle{plain}!.
\cxset{chapter opening = right, chapter format = block}
\chapter{Test}

\cxset{defaults, chapter opening= anywhere}



\section*{Keys for chapter head formatting}

A chapter heading can be considered of being constructed of several parts, the \textit{chapter number}, the chapter name typically \textit{chapter} and the \textit{title}. Predefined keys handle all the elements of formatting. Additional keys are defined to handle other elements such as inclusion of images or producing complicated examples with graphics constructed with \texttt{TikZ} and other similar packages.


\bigskip\bigskip\bigskip\bigskip
\let\oldrefkey\refKey
\let\refKey\texttt
\makeatletter
\long\def\demobox#1#2{%
\par\bigskip\bigskip\bigskip
\begin{tcolorbox}[enhanced,left=0pt, top=0pt, bottom=0pt,width=\textwidth,
  enlarge top initially by=1cm,enlarge bottom finally by=1cm,left skip=1cm,right skip=1cm,
  colframe=white,colback=white,
  colbacktitle=red!30!white,colupper=black!7!white,
  code={\appto\kvtcb@shadow{%
    \path[fill=white,draw=yellow!50!black,dashed,line width=0.4pt]
      ([xshift=-1cm,yshift=-1cm]frame.south west) rectangle
      ([xshift=1cm,yshift=1cm]frame.north east);
     \path[fill=blue!20!white, 
              opacity=0.3, draw=yellow!50!black,solid,line width=1pt]
      ([xshift=-2cm,yshift=-2cm]frame.south west) rectangle
      ([xshift=2cm,yshift=2cm]frame.north east);  
    }},
  finish={
  \draw[thick,<->] ([yshift=-1.3cm]frame.north west)-- node[below]{\texttt{#1 width}}
    ([yshift=-1.3cm]frame.north east);
  \draw[thick,<->] ([xshift=-15mm]frame.north east)-- node[above]{\refKey{#1 height}}
    ([xshift=-15mm]frame.south east);
  \draw[thick,<->] (frame.north)-- node[right]{\refKey{#1 padding-top}} +(0,1);
  \draw[thick,<->] ([yshift=1cm]frame.north)-- node[right]{\refKey{#1 margin-top}} +(0,1);
  \draw[thick,<->] (frame.south)-- node[right, align=left]{\refKey{#1 padding-bottom}}+(0,-1);
  %left padding
  \draw[thick,<->] (frame.west)-- node[below right,align=center]{\refKey{#1 padding-left }}+(-1,0);
  %left margin
  \draw[thick,<->] ([xshift=-1cm,yshift=-0.9cm]frame.west)-- node[below right,xshift=-1,align=left]{\refKey{#1 margin-left }\\\refKey{#1 grow to left by}}+(-1,0);
  %right padding
  \draw[thick,<->] (frame.east)-- node[below left,align=center]{\refKey{#1 padding-right}}+(1,0);
 %right margin
  \draw[thick,<->] ([xshift=1cm,yshift=-0.9cm]frame.east)-- node[below left,xshift=1, align=right]{\refKey{#1 margin-right}\\\refKey{#1 grow to right by}}+(1,0);
 \draw[thick,<->] ([yshift=-2cm]frame.south)-- node[right, align=left]{\refKey{#1 margin-bottom},\\ \refKey{#1 after skip}}+(0,1);
  }
    ]
#2%
%\hrule width0pt height4.5cm depth0pt\relax% \vspace*{4.5cm}% \lipsum[1]
\end{tcolorbox}\par
\bigskip\bigskip\bigskip}
\makeatother

\demobox{chapter}{\scalebox{1.17}{\HHHUGE Chapter}}

The number box is again drawn in a box similar to a chapter with all properties generalized.

\demobox{number}{\scalebox{1.15}{\HHHUGE Thirteen}}



All parameters shown in the diagram can be set using the command \cs{cxset}. The property names follow conventions similar to those of |css|, rather than typical conventions of \tikzname that are more widely known to the programming community. The prefix to these properties (in the example \textit{chapter}) can be thought of
as similar to a |class| or |id| name in |css|.  

\begin{docCommand}{cxset}{\marg{options}}
  Sets options for every following \refEnv{tcolorbox} inside the current \TeX\ group.
  By default, this does not apply to nested boxes, see \Vref{subsec:everybox}.\par
  For example, the colors of the boxes may be defined for the whole document by this:
\begin{dispListing}
\cxset{chapter numbering = Roman,
       chapter number color = blue}
\end{dispListing}
\end{docCommand}

\begin{docKey}[]{chapter padding-top}{=\meta{dimension}}{no default, initial value 0pt}
All padding keys take one argument, which is a dimension. The length is also stored in a register
\cmd{\chapterpaddingtop}. In this chapter it was set at %\the\chapterpaddingtop.
\begin{dispListing}
\cxset{colback=red!5!white,colframe=red!75!black, chapter padding-top=2pt}
\end{dispListing}
\end{docKey}



\begin{docKey}[]{chapter padding-right}{=\meta{dimension}}{no default, initial value 0pt}
All padding keys take one argument, which is a dimension. The length is also stored in a register
\cmd{\chapterpaddingright}.  In this chapter it was set at %\the\chapterpaddingright.
\end{docKey}

\begin{docKey}[]{chapter padding-bottom}{=\meta{dimension}}{no default, initial value 0pt}
All padding keys take one argument, which is a dimension. The length is also stored in a register
\cmd{\chapterpaddingbottom}.  In this chapter it was set at %\the\chapterpaddingbottom.
\end{docKey}

\begin{docKey}[]{chapter padding-left}{=\meta{dimension}}{no default, initial value 0pt}
All padding keys take one argument, which is a dimension. The length is also stored in a register
\cmd{\chapterpaddingleft}.  In this chapter it was set at %\the\chapterpaddingleft.
\end{docKey}

%% margin

\begin{docKey}[]{chapter margin-top}{=\meta{dimension}}{no default, initial value 0pt}
All padding keys take one argument, which is a dimension. The length is also stored in a register
\cmd{\chaptermargintop}. In this chapter it was set at .
\end{docKey}

\begin{docKey}[]{chapter margin-right}{=\meta{dimension}}{no default, initial value 0pt}
All padding keys take one argument, which is a dimension. The length is also stored in a register
\cmd{\chapterpaddingright}.  In this chapter it was set at %\the\chapterpaddingright.
\end{docKey}

\begin{docKey}[]{chapter margin-bottom}{=\meta{dimension}}{no default, initial value 0pt}
All padding keys take one argument, which is a dimension. The length is also stored in a register
\cmd{\chapterpaddingbottom}.  In this chapter it was set at %\the\chapterpaddingbottom.
\end{docKey}

\begin{docKey}[]{chapter margin-left}{=\meta{dimension}}{no default, initial value 0pt}
All padding keys take one argument, which is a dimension. The length is also stored in a register
\cmd{\chaptermarginleft}.  In this chapter it was set at %\the\chaptermarginleft.
\end{docKey}

\subsection{Borders}

Border have three properties \emph{width, color} and \emph{style}. They can set individually for
each side of the box or using the shorter key .

\begin{docKey}[]{chapter border-top-width}{ = \meta{dimension}}{no default, initial value 0pt}
All border keys take one argument, which is a dimension.
\end{docKey}

\begin{docKey}[]{chapter border-right-width}{=\meta{dimension}}{no default, initial value 0pt}
All border keys take one argument, which is a dimension.
\end{docKey}

\begin{docKey}[]{chapter border-bottom-width}{ = \meta{dimension}}{no default, initial value 0pt}
All border keys take one argument, which is a dimension.
\end{docKey}

\begin{docKey}[]{chapter border-left-width}{ = \meta{dimension}}{no default, initial value 0pt}
All border keys take one argument, which is a dimension.
\end{docKey}

\subsubsection{Border Colors}

The colors follow the same pattern for |border-width| and again they can be set individually or using
a shorter key to set all of them in one color. 

\begin{docKey}[]{chapter border-top-color}{=\meta{color name}}{no default, initial value black}
All border keys take one argument, which is a dimension.
\end{docKey}

\begin{docKey}[]{chapter border-right-color}{=\meta{color name}}{no default, initial value black}
All border keys take one argument, which is a dimension.
\end{docKey}

\begin{docKey}[]{chapter border-bottom-color}{=\meta{color name}}{no default, initial value black}
All border keys take one argument, which is a dimension.
\end{docKey}

\begin{docKey}[]{chapter border-left-color}{=\meta{color name}}{no default, initial value black}
This key is stored in \cmd{\chapterborderrightcolor} and the value in this chapter is 
%\ExplSyntaxOn \l_phd_chapter_border_right_color_tl.
\ExplSyntaxOff
\end{docKey}



\subsubsection{Border Styles}

Standard |css|  offers four styles \emph{dotted, solid, double, dashed}. We offer almost an unlimited set of styles.

\begin{docKey}[phd]{chapter border-top-style}{=\meta{style name}}{no default, initial value \texttt{none}}
The |border-style| properties take a value, which can be |solid, double, dotted, dashed, asterisk|.
\end{docKey}

\begin{docKey}[phd]{chapter border-right-style}{=\meta{style name}}{no default, initial value \texttt{none}}
The |border-style| properties take a value, which can be |solid, double, dotted, dashed, asterisk|.
\end{docKey}

\begin{docKey}[]{chapter border-bottom-style}{=\meta{style name}}{no default, initial value \texttt{none}}
The |border-style| properties take a value, which can be |solid, double, dotted, dashed, asterisk|.
\end{docKey}

\begin{docKey}[]{chapter border-left-style}{=\meta{style name}}{no default, initial value \texttt{none}}
The |border-style| properties take a value, which can be |solid, double, dotted, dashed, asterisk|.
\end{docKey}

\begin{docKey}[phd]{chapter border-style}{=\meta{style name}}{no default, initial value \texttt{none}}
This key sets all chapter-border-\meta{top,right,bottom,left}-style to a single value.
\end{docKey}

\subsubsection{Fonts and colors}

All font parameters can be set using individual keys. The naming scheme in general follows |css| conventions.

\begin{docKey}[phd]{chapter color}{=\meta{color name}}{no default, initial value \texttt{black}}
This key sets the color for the \textit{chapter element}. The color name is stored in \cmd{\chaptercolor@cx}.
The value in this chapter is% \makeatletter\texttt{\chaptercolor@cx}\makeatother.
\end{docKey}

\begin{docKey}[phd]{chapter font-size}{=\meta{Huge, Large}}{no default, initial value \texttt{Huge}}
This sets the size for rendering the \textit{chapter element}. Use one of the following predefined values.
Note that you can either use a command i.e, |chapter font-size=|\cmd{\huge} 
or the command name i.e., |chapter font-size=huge|. The latter is the recommended method.
\end{docKey}

\begin{marglist}
\item [tiny] renders as {\tiny tiny}.
\item[footnotesize] renders as {\footnotesize footnotesize}
\item [small] Opens a chapter on an even page
\item [large] Opens a chapter on a right page.
\item [LARGE] Opens a chapter at the point where the \cs{chapter} is typed.
\item [huge] Alias for \marg{anywhere}.
\item [Huge] Opens a chapter at the next page if the page has material that does not exceed a certain portion of
 \cs{textheight}.
 \item[HUGE] renders as {\HUGE HUGE}.
 \item[HHUGE] renders as {\HHUGE HUGE}.
\end{marglist}

\begin{texexample}{Sizing settings}{}
\cxset{
          chapter format = block,
          chapter label font-size= HUGE,
          chapter name = Chapter,
          chapter format=block,
          chapter number font-size= HUGE,
          chapter title font-size=LARGE,
         % 
         % chapter padding-top=0pt,
         % chapter padding-bottom=0pt,
         % title margin-top=3pt,
         %
          }
\chapter{Setting font-sizes}          
\lorem

\end{texexample}


\begin{docKey}{chapter font-family}{ = \meta{sffamily, rmfamily etc.}}{no default, initial value \texttt{sffamily}}
The |font-family| key accepts \latexe conventional family names or |css| names such as |serif| and |non-serif|. The
value is stored in \docAuxCommand{chapter_font_family}, in this chapter it is set as {\ExplSyntaxOn\meaning\chapter_font_family\ExplSyntaxOff}
\end{docKey}


\begin{marglist}
\item [sffamily] The \emph{chapter element} is rendered in the document default \cmd{\sffamily}.
\item [rmfamily] The \emph{chapter element} is rendered in the document default \cmd{\rmfamily}.
\end{marglist}

%% Font weights
\begin{docKey}[]{chapter font-weight}{=\meta{mdseries,bfseries,etc.}}{no default, initial value \texttt{bfseries}}
The |font-weight| key accepts \latexe conventional family names or |css| names such as |bold| and |bfseries|. The
value is stored in \cmd{\chapterfontweight@cx}, in this chapter it is set as 
{\ExplSyntaxOn\expandafter\string\l_phd_chapter_label_fontweight_tl\ExplSyntaxOff}

\begin{texexample}{Setting chapter element font-weights}{fontweight}
\cxset{chapter label font-weight=normal}
\chapter{Font-weight is normal}
\cxset{chapter label font-weight= bfseries}
\chapter{Font-weight is bfseries}
\lorem
\end{texexample}
\end{docKey}


\begin{marglist}
\item [normal] The \emph{chapter element} is rendered in the document default \cmd{\sffamily}.
\item [bold] The \emph{chapter element} is rendered in the document default \cmd{\rmfamily}.
\item[bfseries] Renders as bold.
\item[mdseries] renders as medium series.
\item[light] This is an alias for normal.
\item[\upshape\ttfamily\string\bfseries] The command version of the setting.
\item[\upshape\ttfamily\string\mdseries] The command version of the setting.
\end{marglist}



\begin{docKey}[]{chapter font-shape}{=\meta{itshape,upshape,etc.}}{no default, initial value \texttt{upshape}}
The |font-weight| key accepts \latexe conventional family names or |css| names such as |bold| and |bfseries|. The
value is stored in |chapter_font_weight|, in this chapter it is set as %\ExplSyntaxOn \texttt{\chapter_font_shape}\ExplSyntaxOff.
\end{docKey}

In |css| the |font-shape| is named as |font-style| so we alias it as well. 

%\begin{marglist}
%\item[normal] normal font-style, defaults to |upshape|.
%\item[upshape] normal font-style, defaults to |upshape|. 
%\item[italic] italic shape, renders as {\itshape italic}. For some fonts it might not be available.
%\item[itshape] italic shape, alias of |italic|.
%\item[oblique] oblique font, in \latexe is equivalent to \cmd{\slshape} and renders as {\slshape slshape}, which might be slightly different than {\itshape italic}.
%\end{marglist}


\begin{texexample}{Setting up Fonts}{chapterfonts}
\cxset{   chapter format = block,
          chapter opening=anywhere,
          chapter label font-weight=normal,
          chapter label font-shape=upshape,
          %chapter border-width=0pt,
          %chapter border-style=none,
          %chapter padding-top=0pt,
          chapter label font-size=large,
          chapter number font-size=large,
          chapter number color=black,
          %title font-size=large,
          }
\chapter[fonts]{Test Font Weights}
\lorem
\cxset{chapter label font-shape=itshape}
\chapter{Test Italic Shape}
\lorem
\cxset{chapter label font-shape=normal}
\chapter{Test normal font-shape}
\lorem
\end{texexample}



The specification of font families is somewhat problematic. In the web the |css| allows |font-family|  to hold several font names as a ``fallback” system. If the browser does not support the first font, it tries the next font.

There are two types of font family names:

\begin{description}
\item[family-name] The name of a font-family, like “times”, “courier”, “arial”, etc.
\item[generic-family] The name of a generic family, like “serif”, “sans-serif”, “cursive”, “fantasy”, “monospace”.
\end{description}

Generally in the \tex community leaving the choice of font  open to what is available on a user’s computer is frowned upon. Knuth’s original aim to render consistently documents, irrespective of a user’s computer installation has served the community well, and it is possible three decades later to produce documents identical in all respects to the original. 

If this is still a valid requirement for documents is debatable. Current document processing requirements are focusing more in the preservation of content and document structure rather than form. Typeset documents in soft copy are now widely preserved in |pdf| or |postcript|  formats. One can archive the |.tex| file as well as the |pdf| file.  Back to the provision of keys, a key defined in a 
similar fashion to those of |css| could help, but there is also the issue of slow compilation. If a font cannot be
found, with the current code, it can slow down compilation tremendously. I am leaving the choice where it belongs to the user and the package writer. It makes no harm if a more flexible definition is provided. The user can then decide to only provide one or many fonts. 

This avoids complicated and almost unintelligible commands such as:

\begin{dispListing}
\setkomafont{subsection}{\usefont{T1}{fvm}{m}{n}}
\setkomafont{section}{\usefont{T1}{fvs}{b}{n}\Large}
\end{dispListing}

Here are some examples. 

\begin{texexample}{Serif and non-serif}{ex:fontfamily}
\cxset{chapter label font-family=serif, 
       chapter opening=anywhere}
\chapter{Serif font}
\lorem
\end{texexample}


\section{Floating and Alignment} 

This particular key bothered me, as the term \emph{float} has a different meaning in \latexe. However, to
be consistent with |css| terminology I have yielded to the temptation and included it.

\begin{docKey}[]{chapter float}{=\meta{left,center,right,none}}{no default, initial value \texttt{none}}
Key that controls the horizontal alignment of the \emph{chapter element}. I order for the
element to float, its |display| property must be set to |inline|.
\end{docKey}

%\begin{texexample}{Floating}{chapter:float}
%\cxset{chapter opening=anywhere, chapter float=center}
%\chapter{Centered Chapter}
%\lorem
%\cxset{chapter float=left}
%\chapter{Left Aligned}
%\lorem
%\cxset{chapter float=right}
%\chapter{Right Aligned}
%\lorem
%\end{texexample}


\subsection{The display property}

Both the |css| box model as well as the \TeX layout engine provide numerous complex algorithms in managing the floating of elements. This is normally controlled using two properties |display| and |float|.


\makeatletter

\begin{docKey}[phd]{chapter position}{ = \meta{absolute, relative}}{no default, initial value black}
This positioning directive instructs the engine to position the element at an exact position.
\end{docKey}



\tcbox[nobeforeafter]{$box_1$}\tcbox[nobeforeafter]{$box_2$}\tcbox[nobeforeafter]{$box_3$}\dotfill\tcbox[nobeforeafter]{$box_n$}
\tcbox[before skip=0.2cm, after skip=0pt, width=1cm, enlarge left by=10cm,width=5cm,enhanced,show bounding box]{title before element}
\tcbox[before skip=0pt, width=1cm, enlarge left by=10cm,width=5cm,enhanced,show bounding box]{
\tcbox{tb}\tcbox{title}\tcbox[nobeforeafter, width=1cm,]{tb}}
\tcbox[before skip=0pt, after skip=12pt, width=1cm, enlarge left by=10cm,width=5cm,enhanced,show bounding box]{\emph{title after} element \fbox{some}}
\makeatother

\begin{docKey}[phd]{chapter float}{=\meta{left,center,right,none}}{no default, initial value \texttt{none}}
Key that controls the horizontal alignment of the \emph{chapter element}. I order for the
element to float, its |display| property must be set to |inline|.
\end{docKey}
In document preparation systems or web page development the layout is user generated, i.e., the user is expected to type the html and the |css| will then specify as to how the page will be rendered by the browser. In our case for documents we can specify how we want the headings to look. The layout manager for each element, creates other associated elements, as shown for the title here. This way most layouts can be accomplished with the declarative visual language of the \pkgname{phd} package. 

\subsubsection{In-line elements}

When an element is specified as |inline| the rendering algorithm places the boxes after each other. This is widely used in |chapter elements| to render the number inline with the chapter name.
\medskip
\bgroup

\noindent
\tcbox[nobeforeafter,width=3cm, height=1cm]{Chapter}\tcbox[nobeforeafter]{twelve}
 
When the property is set as |block| the elements are stacked below each other.
\medskip

\tcbox{chapter  display=block   CHAPTER}
\tcbox{number display=block    TWELVE}

The elements can be considered to be enclosed in a \emph{ghost} element. If the property is set to float we
\begin{figure}[htbp]
\makeatletter
\parindent0pt\fboxsep0pt
\fbox{\vbox to 0pt{\hbox to \dimexpr(\textwidth)\relax{{\hss\tcbox[capture=minipage,width=5cm, height=2cm, top=0pt]{\raggedright number display=block\\ number float=right }}%
}%
}%
}\par
\vspace*{2cm}
\makeatother
\end{figure}
signalling to the layout engine that the element must be placed to the right of the page, as shown in the figure. 


\begin{figure}[htbp]
\makeatletter
\parindent0pt\fboxsep0pt
\fbox{\vbox to 0pt{\hbox to \dimexpr(\textwidth+2cm)\relax{{\hss\tcbox[capture=minipage,width=5cm, height=2cm, top=0pt]{\raggedright number display=block\\ \emph{element} float=right }
\tcbox[capture=minipage,width=5cm, height=2cm, top=0pt]{\raggedright \emph{element} display=block\\ \emph{element} float=right }
}%
}%
}%
}\par
\vspace*{2cm}
\makeatother
\end{figure}

\subsection{Absolute positioning}

Absolute positioning mode, will place an element at an exact position on the page. They are more difficult to
achieve and inflexible. 

\begin{docKey}{position}{=\meta{absolute},\meta{relative}}{no default, initial none}{}

\end{docKey}



This positioning directive instructs the engine to position the element at an exact position.


\begin{docKey}[]{chapter float}{=\meta{left,center,right,none}}{no default, initial value \texttt{none}}
Key that controls the horizontal alignment of the \emph{chapter element}. In order for the
element to float, its |display| property must be set to |inline|.
\end{docKey}
\egroup



\section{Number Element Keys}


\subsection*{Keys for numbering}

Chapter numbering follows that of the standard \LaTeX\ classes and is extended to cover some additional cases such as fully spelled out numbers. This of course is only good for languages that use the arabic numeralsn. For other languages numerals in different formats can be added with simple keys and without the need of \pkgname{polyglossia} or \pkgname{babel}. 

Note that the package uses Heiko Oberdiek's package \pkgname{alphalph} to allow for alphabetic numbering that extends beyond the normal 26 letters of the alphabet. Examples for numbering can be seen in \ref{ex:romannumbering}


\begin{docKey}[phd]{number numbering}{= \oarg{alph,Alph,roman,Roman,none,WORDS,words,none}}{default arabic}
Style of numbering.
\end{docKey}

\begin{marglist}
\item [arabic] Despite that the Arabs call what the West calls Arabic numbers Indian numbers, we provide the value arabic to have normal numbers printed.
\item [alph] Lowercase alphabetic numbering.
\item [Alph] Uppercase alphabetic numbering.
\item [roman] Lowercase roman numbering.
\item [Roman] Uppercase roman numbering.
\item [words] The number is in lowercase words.
\item [WORDS] The number is in uppercase literal numerals.
\item [Words] Prints the number in words and capitalizes the first letter, for example the number 21 will be printed as `Twenty One'\footnote{Currently limited to the first hundred numbers}.
\index{chapter design>numbering>words}
\item [ordinals] Prints the number as ordinal.
\item [Ordinals] Prints the number as Ordinal.
\item [ORDINALS] Prinst the number as ORDINALS.
\item [none] This is equivalent to using the star version of the command. It does not print any number and does not increment the chapter counter.\footnote{I am ambivalent about this, perhaps it will be better to increment it, as it can give a more general approach.}

\end{marglist}
\begin{texexample}{Literal Numbering}{ex:literal}
\cxset{chapter numbering=WORDS} 
\chapter{Literal numbering}
\lorem
\cxset{chapter numbering=words,chapter name=chapter}
\chapter{Literal numbering} 
\lorem
\end{texexample}




\cxset{chapter opening=anywhere, chapter numbering=Roman, chapter number font-shape=upshape}
\index{chapter design>numbering>roman}

\begin{texexample}{Setting up keys for numbering}{ex:romannumberingx}
\bgroup
\cxset{chapter format = traditional, 
       chapter name = CHAPTER, 
       chapter numbering = Roman,
       chapter label color = bgsexy}
\chapter{Roman numbering}
\lorem
\egroup
\end{texexample}





To emulate some old books we also offer an ordinal numbering scheme.

\begin{texexample}{Literal Numbering}{ex:ordinals}
\cxset{chapter numbering=ORDINALS} 
\chapter{Ordinals numbering}
\lorem
\cxset{chapter numbering=words,chapter name=chapter}
\chapter{Literal numbering} 
\lorem
\end{texexample}

\cxset{chapter numbering=arabic}

\subsection{Fonts and colors}
\begin{docKey}[phd]{number color}{=\meta{color name}}{no default, initial value \texttt{black}}
This key sets the color for the \textit{number element}. The color name is stored in %\cmd{\numbercolor@cx}.
The value in this chapter is %\makeatletter\texttt{\numbercolor@cx}\makeatother.
\end{docKey}

\begin{docKey}[phd]{number font-size}{=\meta{Huge, Large}}{no default, initial value \texttt{Huge}}
This sets the size for rendering the \textit{number element}. Use one of the predefined values, as described
in the section for the \emph{chapter} element.
Note that you can either use a command i.e, |number font-size=|\cmd{\huge} 
or the command name i.e., |number font-size=huge|. The latter is the recommended method.
\end{docKey}

Letter spacing can be achieved using the soul package in a combination with the key |spaceout|.
The following examples illustrate the usage.

\index[phdkeys]{{\ttfamily phd/chapter design test}}

%\begin{texexample}{Letter Spacing}{ex:letterspacing}
%\cxset{numbering=Roman,
%        % number letter-spacing=soul,
%        % chapter spaceout=soul,
%         %title spaceout=soul,
%         title font-size=Large,
%         title font-family=rmfamily,
%         title font-shape=scshape}
%\chapter{Letter Spacing}
%
%\lorem
%\end{texexample}

\begin{docKey}[phd]{chapter number letter-spacing}{=\meta{none, true, etc.}}{no default, initial value \texttt{none}}.
\end{docKey}

\begin{marglist}
\item[none] Default value no tracking is used and the letters are spaced as per the basic font information.
\item[inherit] Inherits the letter-spacing settings from the \emph{chapter} element.
\item[true] Letter spacing is employed, using the |soul| package.
\item[false] Alias for |none|.
\item[soul] The \pkgname{soul} package is used for letter-spacing.
\item[microtype] The \pkgname{microtype} package is used for letter-spacing. When the microtype package is used more fine tuning of parameters is available.
\end{marglist}

The example that follows, explains how the features offered by the \pkgname{microtype} package can be used to
set different tracking options.

\begin{texexample}{Microtypography}{micro}
\bgroup

\SetTracking
 [ no ligatures = {f},
 spacing = {600*,-100*, },
 outer spacing = {450,250,150},
 outer kerning = {*,*} ]
 { encoding = * }
 { 100 }

{\huge \textls{Chapter Twenty}}

\SetTracking
 [ no ligatures = {f},
 spacing = {600*,-100*, },
 outer spacing = {450,250,150},
 outer kerning = {*,*} ]
 { encoding = * }
 { 200 }
 
{\huge \textls{Chapter Twenty}}

\egroup
\end{texexample}


\hbox{\drawfontbox{\huge \upshape\textls(Chapter Twenty}}

\hbox{\drawfontbox{\huge \upshape\textls{Chapter Twenty}}}


\section{Styling the chapter title}

Similarly to the number and chapter styling keys exist for styling the chapter title. We summarize the available standard keys below:

\index{chapter design!labels!letter spacing}
\begin{texexample}{Styling the Title}{ex:title} 
\cxset{chapter numbering=arabic, chapter title font-shape=itshape}
\chapter{Chapter title}
\lorem
\end{texexample}


\begin{docKey}[phd]{chapter title font-family}{=\marg{family}}{no default, initial inherit document font}
Selects a predefined font family
\end{docKey}

\begin{texexample}{Title element font styling}{}
\cxset{chapter title font-family=sffamily}
\chapter{Title font family settings}
\lorem
\cxset{chapter title font-shape=itshape}
\chapter{Title font-style settings}
\lorem
\end{texexample}


\begin{docKey}[phd]{chapter title font-weight}{ = \marg{\cs{bfseries},\cs{normalseries}}} {}
Font weight.
\end{docKey}

\begin{docKey}[phd]{chapter title font-size}{= \marg{large, Large, huge, Huge, HUGE, HHuge}}{}
Font sizing commands or their names. Both \docAuxCommand{\HUGE} and HUGE are allowed to be used as values for the key.
\end{docKey}

\begin{docKey}[phd]{chapter title color} { = \marg{color}} {}
The color of the chapter title letters. This takes any predefined color name. 
\end{docKey}


\begin{docKey}[phd]{chapter title spaceout}{ = \marg{soul,none}} {no default, initial = none}
 This key will space out the title. 
\end{docKey}

\begin{texexample}{Title element spacing}{}
\cxset{chapter name=none,
       chapter numbering=none,
       chapter title font-size=Large,
       chapter title color=black,
       chapter title width=0.6\textwidth,
       %title spaceout=soul,
         }
\chapter{The Prehistoric Period in South-East Asia: 2300 BC--AD 400}        
\lorem 
    
\end{texexample}
\cxset{defaults}


\subsection*{Adding content before and after the title element}

Like all the other elements, the title element can be decorated with additional content,
before and after the text. There are two different forms. 

\begin{docKey}[phd]{title before}{=\marg{code}}{default none}
Contents before the title (vertical material)
\end{docKey}

\begin{docKey}[phd]{title after}{=\marg{code}}{default none}
Contents after the title (vertical material)
\end{docKey}

\begin{docKey}[phd]{title content before}{=\marg{code}}{default none}
Contents before the title (horizontal material)
\end{docKey}

\begin{docKey}[phd]{title content after}{=\marg{code}}{default none}
Contents after the title (horizontal material)
\end{docKey}

The difference between the two type of settings, consider the following situation. Assume you have a title that has a rule at the top and bottom and the text is surrounded by two ornaments. The surrounding ornaments will be inserted using the |title before content|, and the rules using the |title before| form. The |title before| is a full fledged element on its own. 

%{
%\hrule
%\centering
%*** Introduction ***
%\par
%\hrule
%}
%
%{
%\MakePercentComment
%\startlineat{200}
%\lstinputlisting{./styles/style13.tex}
%\MakePercentIgnore
%}



 
\begin{docKey}{/phd/ chapter title before skip}{= \marg{soul,none}}{}
Before title string skip.
\end{docKey}

\begin{docKey}{/phd/ chapter title after skip}{ = \marg{soul,none} }{}
After title string skip.
\end{docKey}

\lorem 
%
%\begin{texexample}{letter spacing the chapter title block}{ex:title3}
%
%\cxset{chapter spaceout=none,
%         numbering=arabic}
%         
%\chapter{Chapter Title Styling}
%\end{texexample}
%
%\end{document}



\cxset{chapter opening=right}
\section{Table of Contents}\index{table of contents!key settings}

Traditionally a chapter will be added to the Table of Contents if the \cs{chapter} command is issued. The starred version will not produce a number and will not add a contents line. Since we have adopted an approach where we use a key value interface we can dispense with the starred version of the command, by setting the \option{chapter toc} option to false. For example if we want to define a command for a ``Foreward'' or ``Epiloque'' without wishing them to be added to the table of contents we can use the following setting.\index{Foreward>definitions}\index{Epilogue>definitions}



\begin{texexample}{changing the chapter label name}{}
\cxset{chapter name=Chapteris, chapter numbering=arabic,}
\chapter{Foreward}
\lorem
\end{texexample}

Note that the key \option{numbering=none} still has to be set.


Please note that when \textbf{numbering=none} the chapter number is not available anymore and yo may have to reset it if required again. Although this might be seen as rather cumbersome than simply using \cs{chapter*} the advantage is consistency in the user interface and the use of appropriate semantic definitions for all sectioning commands thus achieving a bit more separation of context from style.


%\cxset{chapter toc=true}

\section{Defining styles}

Named styles can be defined using the standard \textsc{PGF} conventions. To define a style for the forward above we can use:

\begin{texexample}{}{}
\cxset{foreward/.style={chapter numbering=none,
          chapter name=none,
          chapter title font-size= Large,
          chapter title font-family= sffamily,
          chapter numbering=none}}
\cxset{foreward}
\chapter{Foreward.}
\lorem
\end{texexample}



\cxset{chapter numbering=arabic}
\section{Creating semantic names for commands and environments}

To keep our search for semantic commands and true separation of contents it is prudent to define some macros for typesetting the  `foreward' section.

\bgroup
\begin{texexample}{defining a \textit{Foreward} macro.}{}
\begin{lstlisting}
\cxset{foreward/.style={chapter toc=false,
          name=none,
          title font-size = Large,
          title font-family = sffamily,
          numbering=none}}
\newcommand\forewardname{foreward}
\expandafter\newenvironment\expandafter{\forewardname}{%
\cxset{foreward}\chapter{Foreward}}%
{}
\begin{foreward}
\lorem
\end{foreward}
\end{lstlisting}
\end{texexample}
\egroup

Notice the use of a new command \cmd{\forewardname} to allow for internationlization using Babel or other methods. One is tempted to let the English name, but a better approach perhaps is to define both.

\makeatletter



%\cxset{chapter format=block}
\newfontfamily\emoji{Symbola}

\chapter{Those Other Languages}
\label{ch:languages}

\epigraph{New York 1. Act making it a misdemeanor to make a speech or talk in public manner, in any language other than English upon any subject relating to the form of a character of the government or the administration or enforcement of the laws of this state or the United States.}{\itshape Introduced in the Assembly by Mr Hamill, Feb.23, and referred to the Codes Committe (A.878.)}

\parindent=1em



\section{The world's scripts and languages}


On May 23, 1918, Iowa Gov. William Harding banned the use of any foreign language in public: in schools, on the streets, in trains, even over the telephone. Frese  \footfullcite{frese} published a detailed history of this event in American history during the Great War years and its effects, which can even today be seen in Iowa. Such events of course during stressful times in a history of a country are not unique to America and similar politics can be observed throughout all human history. Today most Americans' response to the calling of such a law would probably be the Unicode Character \unicodenumber{U+1F4A9}\footnote{\protect\emoji\protect\char"1F4A9 self describing!}. Getting the character to print as a footnote in a document is another story. As many of the world's languages are facing extinction and the inclusion of a section in the |phd| package to deal with different scripts and appropriate fonts has been done in this spirit.

 
Probably there are more users of \latexe whose mother tongue is not English than those who speak the language. \tex out of the box does not offer facilities for using non-latin based scripts easily; this presents numerous problems. The biggest problem---which has been solved to a large extent---was the entering of text without having to mark all the special
characters such as umlauts (\"o) with commands. The second issue and which has been addressed by packages such as Babel, is redefining the strings such as ``Chapter" to another language. In software this is called internationalization and a governing standard is |i18n|. None of the current packages take such an approach and none of them as yet offer a satisfactory solution for |LuaLaTeX|. 


Another issue with writing systems and scripts is finding and using appropriate fonts. Most writing systems that have ever existed are now extinct. Only minute vestiges of one of the most ancient---Egyptian hieroglyphs---live on, unrecognized, in the Latin alphabet in which English, among hundreds of other languages, is conveyed today. The latin \textit{m}, for example, ultimately derives from the Egyptian's n-sign, depicting waves. There may never be a font that includes all the unicode characters (code2000) came close. Good fonts with well over ten thousand characters, keyed to the Unicode system, are now readily available. 

Bringhurst in the Elements of Typographic Style \footcite{Bringhurst2005} critisized the allotment of only 256 characters in the extended |ASCII| specification and other software and considered this practice by software developers as \enquote{typographically sectarian and culturally stunted}. 


Bringhurst comments were unfair to programmers as he was probably unaware of the difficulties. Many  scripts are widely different to the Latin script. Hanunó'o is written vertically from bottom to top, whereas tibetan and sometimes Chinese from top to bottom.  Middle Eastern scripts such as Hebrew and Arabic are written from right to left. Some of the scripts have other peculiarities as they take different forms when they are at the middle of a word or at the end. Ancient scripts such as hieroglyphics could be written from top to botton or from right to left or left to right or boustrostrephon. The glyphs of the latter could also face either left or right and the writing direction can be determined based on the direction the figures are \enquote{seeing}.\index{boustrostrephon}

Note that  we will be using the word ``script" instead of a ``writing system". Many people associate the word ``script" with a small program which is normally used on the command line. Here ``script" means a collection of letters and other characters, meant for writing human languages in a systematic way.  We say that languages such as English, Dutch and Icelandic and Vietnam use the Latin \emph{script}, although they have different repertoires of characters. 


\section{TeX's support for different languages}

\tex's support for languages centered around hyphenation patterns.
Primitives such as \docAuxCommand{language}=\meta{number} can be used to store hyphenation patterns and exceptions for up to 256 different languages. 
This primitive is then used by \tex to apply an appropriate set of hyphenation rules for each paragraph or part of a paragraph in a document\footnote{\url{http://www.tug.org/utilities/plain/cseq.html language-rp}}. 

When \tex begins a ne paragraph it sets the \emph{current language} to \cs{language}. Just before it adds each new character to the paragraph in unrestricted horizontal mode, it compares the current language to \cmd{\language}. If they are different, TeX : 

\begin{enumerate}{}

\item changes the current language to \cmd{\language}; 

\item inserts a whatsit\index{whatsit>language} containing the new language and the values of |\lefthyphenmin| and |\righthyphenmin|; 

\item inserts the character. The |\setlanguage| command should be used to change languages in restricted horizontal mode (i.e., inside an |\hbox|). 
\end{enumerate}
If \meta{number} is less than 0 or greater than 255, 0 is used [455].
  Plain TeX has a \docAuxCommand{newlanguage} command which may be used to allocate numbers for languages [347]. Changes made to \refCom{language} are local to the group containing the change 
  
If you enter, for example, |\newlanguage\Catalan|, then to switch to the hyphenation patterns of the Catalan language, you need to write |\language = \Catalan|. Writing |\Catalan| by itsef is not sufficient. 
More about \tex's support for languages can be found in the \nameref{ch:hypenation}

\section{LaTeX language management}

\latexe follows the same route as \tex and Plain TeX and its only language support is for hyphenation.
In the source2e the File |lthyphen.dtx| describes the approach to loading the default file |hyphen.ltx| . If a file hyphen.cfg is found \latexe will load the appropriate hyphenation patterns. 

Traditionally language management was achieved using Johan 
Braams package \pkg{Babel} which we describe in the next section. Numerous packages to assist in using different languages with \latex can be found at \url{http://www.ctan.org/tex-archive/language/}. 

\section{The Babel Package} 

The package \pkg{Babel} developed by \footfullcite{babel} was the first package to systematically offer foreign language
support for \latex. It has been updated for use with \XeTeX\ and \LuaTeX\ and provides an environment
in which documents can be typeset in a language
other than US English, or in more than one language.
However, no attempt has been done to
take full advantage of the features provided by the
latter, which would require a completely new core
(as for example polyglossia or as part of a future \latex3).

\subsection{Language files}
The package has a number of predefined language files with the extension |ldf|. Each \emph{language definition file} contains commands appropriate for setting strings and hyphenation patterns in the particular language, as well as
many ancillary macros to typeset dates and numbers in the typographical convention of the language. 


\begin{docCommand} {selectlanguage} {\marg{language}} {default none, initial US English}
When a user wants to switch from one language to another he can
do so using the macro |\selectlanguage|. This macro takes the
language, defined previously by a language definition file, as
its argument. It calls several macros that should be defined in
the language definition files to activate the special definitions
for the language chosen. For ``historical reasons'', a macro name is
converted to a language name without the leading |\|; in other words,
the two following declarations are equivalent:
\end{docCommand}
\begin{phdverbatim}
\selectlanguage{german}
\selectlanguage{\german}
\end{phdverbatim}

\begin{docCommand}{foreignlanguage}{\marg{language}\marg{text}}
The command |\foreignlanguage| takes two arguments; the second
argument is a phrase to be typeset according to the rules of the
language named in its first argument. This command (1) only
switches the extra definitions and the hyphenation rules for the
language, \emph{not} the names and dates, (2) does not send
information about the language to auxiliary files (i.e., the
surrounding language is still in force), and (3) it works even if
the language has not been set as package option (but in such a
case it only sets the hyphenation patterns and a warning is shown).
\end{docCommand}

\begin{docCommand}{otherlanguage*} { \marg{language}{otherlanguage*}}
Same as |\foreignlanguage| but as environment. Spaces after the
environment are \textit{not} ignored.
\end{docCommand}


\section{The Polyglossia package}

The \pkg{polyglossia} package has a lot of potential and has solved many issues
but its integration with large parts of the traditional |pdfLaTeX| world
is still under development and will probably take a while before one could
declare it easy to use and bug free \footfullcite{polyglossia}. For example anything with the |bidi| package has issues with loading orders for a number of packages and least of which is with
the Ams packages. So if you are going to mix a number of languages in a \XeTeX\ document
you need to take extra care.

 Polyglossia is a package for facilitating multilingual typesetting with
 \XeLaTeX\ and (at an early stage) \LuaLaTeX.  Basically, it
 can be used as a replacement of \pkg{babel} for performing the following
 tasks automatically:
 
 \begin{enumerate}
 \item Loading the appropriate hyphenation patterns.
 \item Setting the script and language tags of the current font (if possible and
       available), via the package \pkg{fontspec}.
 \item Switching to a font assigned by the user to a particular script or language.
 \item Adjusting some typographical conventions according to the current language
       (such as afterindent, frenchindent, spaces before or after punctuation marks,
       etc.).
 \item Redefining all document strings (like chapter, ``figure'', ``bibliography'').
 \item Adapting the formatting of dates (for non-Gregorian calendars via external
       packages bundled with polyglossia: currently the Hebrew, Islamic and Farsi
       calendars are supported).
 \item For languages that have their own numbering system, modifying the formatting
       of numbers appropriately (this also includes redefining the alphabetic sequence
       for non-Latin alphabets).\footnote{ %
         For the Arabic script this is now done by the bundled package \pkg{arabicnumbers}.}
 \item Ensuring proper directionality if the document contains languages
       that are written from right to left (via the package \pkg{bidi},
       available separately).
 \end{enumerate}
 
 Several features of \pkg{babel} that do not make sense in the \XeTeX\/\luatex world (like font
 encodings, shorthands, etc.) are not supported supported by the package.
 
 Generally speaking, \pkg{polyglossia} aims to remain as compatible as possible
 with the fundamental features of \pkg{babel} while being cleaner, light-weight,
 and modern. The package \pkg{antomega} has been very beneficial in our attempt to
 reach this objective.


\section{Loading language definition files}

The recommended way of \pkg{polyglossia} to load language definition files
is given in the manual as:
 
\begin{docCmd}{setdefaultlanguage}{\oarg{options}\marg{lang}}
 (or equivalently \cmd\setmainlanguage).
\end{docCmd}
 
 Secondary languages can be loaded with

\begin{docCmd} {setotherlanguage}{\oarg{options}\marg{lang}}
\end{docCmd}
 These commands have the advantage of being explicit and of allowing you to set
 language-specific options.\footnote{ %
 More on language-specific options below.}
 It is also possible to load a series of secondary languages at once using

\begin{docCmd}{setotherlanguages} { \marg{lang1,lang2,lang3,\ldots}}
\end{docCmd}

 Language-specific options can be set or changed at any time by means of
\begin{docCmd}{setkeys} { \marg{lang}\marg{opt1=value1,opt2=value2,\ldots}}
\end{docCmd}

\subsection{Bidirectional languages}





\begin{comment}
\begin{Arabic}
ّ هو إذ الغاية؛ شريف الفوائد، جم المذهب، عزيز فنّ التاريخ فنّ أنّ اعلم
والملوك سيرهم، في والأنبياء أخلاقهم، في الأمم من الماضين أحوال على يوقفنا
ّ أحوال في يرومه لمن ذلك في الإقتداء فائدة تتم حتّى وسياستهم؛ دولهم في
والدنيا. الدين
\end{Arabic}
\end{comment}

The Greek language is represented both in modern Greek as well as its ancient variants.

\begin{phdverbatim}
\begin{greek}
\textbf{Η ελληνική γλώσσα} είναι μία από τις ινδοευρωπαϊκές γλώσσες, για την
οποία έχουμε γραπτά κείμενα από τον 15ο αιώνα π.Χ. μέχρι σήμερα. Αποτελεί το
μοναδικό μέλος ενός κλάδου της ινδοευρωπαϊκής οικογένειας γλωσσών. Ανήκει
επίσης στον βαλκανικό γλωσσικό δεσμό.\\	
\end{greek}
\end{phdverbatim}

\topline

\textbf{Η ελληνική γλώσσα} είναι μία από τις ινδοευρωπαϊκές γλώσσες, για την
οποία έχουμε γραπτά κείμενα από τον 15ο αιώνα π.Χ. μέχρι σήμερα. Αποτελεί το
μοναδικό μέλος ενός κλάδου της ινδοευρωπαϊκής οικογένειας γλωσσών. Ανήκει
επίσης στον βαλκανικό γλωσσικό δεσμό.\\	

\bottomline

\begin{verbatim}
\begin{russian}
\textbf{Русский язык} — один из восточнославянских языков, один из 
крупнейших языков мира, в том числе самый распространённый из славянских
языков и самый распространённый язык Европы, как географически, так и по
числу носителей языка как родного (хотя значительная, и географически бо́
льшая, часть русского языкового ареала находится в Азии).	\\

\end{russian}
\end{verbatim}



\textbf{Русский язык} — один из восточнославянских языков, один из крупнейших языков мира, в том числе самый распространённый из славянских языков и самый распространённый язык Европы, как географически, так и по числу носителей языка как родного (хотя значительная, и географически бо́льшая, часть русского языкового ареала находится в Азии).	\\


\section{The Translator package}

The \pkg{translator} package was developed by Till Tantau \cite{translator}. It provides a flexible
mechanism for translating individual words into different languages.
For example, it can be used to translate a word like \enquote{figure} into,
say, the German word \enquote{Abbildung}. Such a translation mechanism is
useful when the author of some package would like to localize the
package such that texts are correctly translated into the language
preferred by the user. The translator package is \emph{not} intended
to be used to automatically translate more than a few words. 

You may wonder whether the translator package is really necessary
since there is the (very nice) |babel| package available for
\LaTeX. This package already provides translations for words like
``figure''. Unfortunately, the architecture of the babel package was
designed in such a way that there is no way of adding translations of
new words to the (very short) list of translations directly build into
babel.

The translator package was specifically designed to allow an easy
extension of the vocabulary. It is both possible to add new words that
should be translated and translations of these words.

\subsection{Using the Translator Package}

  The \pkg{Translator}\footcite{translator} needs to be used with \pkg{Babel} and I am not too sure yet 
  if it is ready  to be used with Polyglossia.

Once the package has loaded a language or a set of languages the optional argument to the
\cmd{\translate} can be used to translate a string. 

\begin{texexample}{Translating strings}{ex:translator}
  \translate[to=german]{rightpagename}
  \translate[to=dutch]{rightpagename}
\end{texexample}

Before you can provide the translations you need to provide your own dictionaries, where you require them. These need to be installed at a place where \tex can find them.

\begin{docCmd} {ProvidesDictionary} { \marg{dictionary file name} \marg{language} }
\end{docCmd}

The dictionary has to be saved in a specific format that relates to the \refCmd{ProvidesDictionary} command. The second argument of the command must be appended to the file name; for the example the file is saved as\footnote{This  example is from the translator package bundle and is under the folder \texttt{base}}:

|translator-basic-dictionary-German|

The concepts take a bit of time to sink in, but once you have everything set up, it is quite easy and straight forward to incorporate it, into your package. 

\begin{teX}
\ProvidesDictionary{translator-basic-dictionary}{German}

\providetranslation{Abstract}{Zusammenfassung}
\providetranslation{Addresses}{Adressen}
\providetranslation{addresses}{Adressen}
\providetranslation{Address}{Adresse}
\providetranslation{address}{Adresse}
\providetranslation{and}{und}
\providetranslation{Appendix}{Anhang}
\providetranslation{Authors}{Autoren}
\providetranslation{authors}{Autoren}
\providetranslation{Author}{Autor}
\providetranslation{author}{Autor}
\end{teX} 

This is in contrast to Babel and Polyglossia that define
commands for each string to be translated such as,

\begin{phdverbatim}
\def\captionsdutch{%
    \def\prefacename{Voorwoord}%
    \def\refname{Referenties}%
    \def\abstractname{Samenvatting}%
    \def\bibname{Bibliografie}%
    \def\chaptername{Hoofdstuk}%
    \def\appendixname{Bijlage}%
    ...
    \def\proofname{Bewijs}%
    \def\glossaryname{Verklarende woordenlijst}%
    \def\today{\number\day~\ifcase\month%
      \or januari\or februari\or maart\or april\or mei\or juni\or
      juli\or augustus\or september\or oktober\or november\or
      december\fi
      \space \number\year}}
\end{phdverbatim}

\begin{docCommand}{usedictionary}{\marg{kind}}
  This command tells the |translator| package, that at the beginning of
  the document it should load \textit{all} dictionaries of kind \meta{kind} for
  the languages used in the document. Note that the dictionaries are
  not loaded immediately, but only at the beginning of the document.

  If no dictionary of the given \emph{kind} exists for one of the
  language, nothing bad happens.

  Invocations of this command accumulate, that is, you can call it
  multiple times for different dictionaries.
\end{docCommand}

\begin{docCommand}{uselanguage}{\marg{list of languages}}
  This command tells the |translator| package that it should load the
  dictionaries for all languages in the \meta{list of languages}. The
  dictionaries are loaded at the beginning of the document.
\end{docCommand}



\chapter{CLDR}

The \pkg{phd} package provides facilities for language handling, but albeitly still at an experimental stage. Sectioning command strings can easily be set in one's language by just typing the key in the appropriate language.

\begin{texexample}{Example of changing language in headings}{ex:lheadings}
\bgroup
\cxset{locale turkish,
       chapter format=block,
      chapter opening=anywhere,
       chapter number color = black,
      }
\chapter{Testing}
        
\egroup
\end{texexample}


The language message text are actually variables (pretty much similar to the message modules of the l3error package. It follows
patterns for defining such messages in other languages and in the Linux kernel. (|get_text|). Actually your error messages
in packages belong here. 

These resources should preferably be put into files that wil be loaded by a library that uses a combination of language and country (also known as the \enquote{locale}) to identfy the right string. Once we have placed these files we can send them to the translation vendor and get back translated files for each locale that your application is going to support.

There are various file formats that make suitable resource files. Popular choices are JSON, XML, gettext or YAML. The translator files that we discussed earlier are very similar in context. 

\begin{phdverbatim}     
     locale/en/names/part~name/.store      = part_name_tl,
     locale/en/names/chapter~name/.store   = chapter_name_tl,
     get_text{en}{chapter_name_tl} -> Chapter
\end{phdverbatim} 

A similar storing technique can be employed for other sections of the CLDR specifications such as delimters:

\begin{phdverbatim}
    locale/en/delimiters/quotation start = “,
    locale/en/delimiters/quotation end =  ”,
    locale/en/delimiters/alternate quotation start = ‘,
    locale/en/delimiters/alternate quotation end = ’,
\end{phdverbatim}

The i18n CLDR discussed in more detailed in the next chapter provide ready made internationally agreed json or xml files
detailing the most common internationalization tasks, such as strings for dates, months, calendars, quotes, common units and
sorting the latter is very important for many tasks, such as bibliographies. Once we have the structure defined
a small Go utility can download all the files and translate them into our resource files. These will be missing the
strings for sectioning, typographical conventions, shorthands and other conventions normally handled by Babel. 

Babel and polyglossia modify the basic LaTeX environments or macros to achieve this. In my opinion it should be the other way out, they should only provide a value to be used by these commands rather than the commands be cloberred at this level.

\section{Numbers}

The formatting of numbers for a locale is specified by the CLDR in files containing the file |numbers|. These can be downloaded as |xml| files or |json| files.
Currently most users of \latexe requiring to format numerals they will use either |numprint| or |SIUnitx|. The latter has many settings as it handles scientific units. For formatting numbers in \latexe the |cldr| specifications and data are somewhat limited. Package options and commands
would normally handle rounding, leading zeroes for decimals, plums minus signs for the combination (+-) and other similar requirements.



\begin{texexample}{Using numprint}{ex:numprint}
% basic command
\numprint{12500678.912345}

% shorter version
\np{12500678.912345}
\end{texexample}



\paragraph{Numbering systems}

Numbering systems are used to show different representations of numeric values. Each numbering system consists of characters that represent numeric digits. In addition, there are also number symbols used with each numbering system that may differ when the numbering system is used in different locales.

The default numbering system for a locale is the numbering system that is normally used to represent numbers in that locale.

\begin{verbatim}
"numbers": {
        "defaultNumberingSystem": "latn",
        "otherNumberingSystems": {
          "native": "latn"
        },
        "minimumGroupingDigits": "1",
        "symbols-numberSystem-latn": {
          "decimal": ".",
          "group": ",",
          "list": ";",
          "percentSign": "%",
          "plusSign": "+",
          "minusSign": "-",
          "exponential": "E",
          "superscriptingExponent": "×",
          "perMille": "‰",
          "infinity": "∞",
          "nan": "NaN",
          "timeSeparator": ":"
        },
\end{verbatim}

The native numbering system for a locale is the numbering system used for native digits, and is normally in the script for the locale's language. Native numbering systems can only use numeric positional decimal digits, like for Latin numbers (0123456789). If the numbering system in your language uses an algorithm to spell out numbers in the language's script, label it as a traditional numbering system instead. The traditional numbering system does not need to be specified if it is the same as the native numbering system.

The default, native and traditional numbering systems for a locale may be different. For example, in Tamil the default numbering system is |latn|, the native numbering system is |tamldec| and the traditional numbering system is |taml|.

\begin{trivlist}\item[]
\begin{tabular}{lll}
\toprule
Code	 & Description	 & Digits\\
\midrule
arab	 & Arabic-Indic digits	&\panunicode ٠١٢٣٤٥٦٧٨٩\\
fullwide &   	Full width digits &\panunicode 	0123456789\\
hant	   & Traditional Chinese numerals — non-decimal	& algorithmic\\
latn	   &Latin digits	 &0123456789\\
\bottomrule
\end{tabular}
\end{trivlist}

\paragraph{Minimum digits for grouping}

In some languages, the grouping separator is suppressed in certain cases. For example, see china-auf-wachstumskurs.gif, where there is a grouping separator in \enquote{12 080} but not in \enquote{4720}. The |minimumGroupingDigits| determines what the default for a locale is. In this case the value should be \enquote{2} to illustrate that the separator only appears once the number of thousands goes into the double-digits (i.e. 10 thousand or above) and not for single-digit thousands (i.e. anything below 10 thousand).


Note that this is just the default, and the grouping separator may be retained in lists, or removed in other circumstances. For example, in English the \enquote{,} is used by default, but not in addresses (\enquote{12345 Baker Street}), in 4-digit years (2014, but 12,000 BC), and certain other cases.

\begin{texexample}{Numprint minimum grouping}{ex:numprint2}
\begingroup
\npfourdigitnosep$\numprint{1234.1234}$, $\numprint{12345.12345}$ 

\npfourdigitsep$\numprint{1234.1234}$, $\numprint{12345.12345}$
\endgroup
\end{texexample}

The much larger package \pkg{siunitx} can also be used to parse and typeset numbers in different formats, using the command \docAuxCommand{num}.


\begin{texexample}{siunitx}{ex:siunitx}
\num{123}\\
\num{1234}\\
\num{12 345}\\
\num{0.123} \\
\num{0,1234}\\
\num{.12345}\\
\num{3.45d-4}\\
\num{-e10}
\end{texexample}

The package also provides commands for formatting angles, ranges and similar. For the latter it also provides a limited set of localization commands by using the \pkg{translator} from the \pkg{Beamer} bundle.

The package defines numerous keys that can be used either at package level or as options to the command num to format and print the numbers. In the next example the group separator is set uing the key |group-separator|. 

\begin{texexample}{siunitx group separator}{ex:siunitx-01}
\num{12345} \\
\num[group-separator = {,}]{12345} \\
\num[group-separator = \text{~}]{12345}
\end{texexample}


\paragraph{Number Symbols} The following symbols are used in formatting numbers. They will be substituted for the placeholders in Number Patterns. 

\begin{longtable}{llp{5cm}}
\toprule
Name	&English Example	&Meaning\\
\midrule
|decimal|	  &2,345.67	 &decimal separator\\
|group|	     &2,345.67	 &grouping separator, typically for thousands\\
|minusSign|  &	+23	*	 &the plus sign used with numbers\\
|plusSign|	  &-23	*	    &the minus sign used with numbers\\ 
|perMille|	  &234‰	*	&the permille sign (out of 1000)\\
|exponential|	      &1.2E3	*	&used in computers for 1.2×10³.\\
|superscriptingExponent|	&1.2×103	* &human-readable format of exponential \\
|infinity|	  &∞	*	&used in +∞ and -∞.\\ 
|nan|	     &NaN	*	&\enquote{not a number}. \\
\bottomrule
\end{longtable}


%The + and - symbols are intended for unary usage, and not for binary usage; thus represents either the positive number or a negative number. For example, in an operation 3 -(-2), the defined symbol would be used for the second minus sign, but not for the subtraction operator. Any directionality markers needed (e.g. <LRM>) to keep with the number should be included.
%percentSign	23.4%%	*	the percent sign (out of 100)

\paragraph{Number Patterns}

Numbers are formatted using patterns, like |#,###.00|. For example, the pattern |#,###.00| when used to format the number 12345.678 could result in "12'345,67". That would happen if the grouping separator for your language is an apostrophe ('), and the decimal separator is a comma (,).  Also see Number Symbols.

Important: The characters . , 0 \# (and others below) are special placeholders; they stand for the decimal separator, and so on, and are NOT real characters. You must NOT "translate" the placeholders; for example, don't change '.' to ',' even though in your language the decimal point is written with a comma.

Here are the special characters used in number patterns.

Whenever any of these symbols are in the English pattern, they must be retained in the pattern for your language. The positions of some of them (\%, ¤) may be changed, or spaces added or removed. The symbols will be replaced by the local equivalents, using the Number Symbols for your language. Verify results by reviewing the dynamic examples in the right-hand pane.


The cldr locale files, provide these patterns. They can then be used to format general purpose numbers, which fall into
five categories.

\begin{longtable}{p{2.5cm}lp{6.5cm}}
\toprule
Type	&English Example	& Meaning\\
\midrule
currency	&¤|#,##0.00|  &Used for currency values. A currency symbol (¤) is will be replaced by the appropriate currency symbol for whatever currency is being formatted. The choice of whether to use the international currency symbols (USD, EUR, JAY, RUB,…) or localized symbols (\$, €, ¥, руб.,…) is up to the application program that uses CLDR. Note: the number of decimals will be set by programs that use CLDR to whatever is appropriate for the currency, so do not change them; keep exactly 2 decimals.\\

currency-accounting	 &¤|#,##0.00|;(¤|#,##0.00|)	&Used for currency formats in accounting contexts.\\
\bottomrule
\end{longtable}

Pattern Characters are shown below.

\begin{longtable}{lp{11cm}}
\caption{Pattern Characters}\\
\toprule
Symbol & Meaning\\
\midrule
.	&Replaced automatically by the character used for the decimal point in your language. Not a real period; must be retained!\\
,	&Replaced by the "grouping" (thousands) separator in your language. Not a real comma; must be retained!\\
0	&Replaced by a digit (or zero if there aren't enough digits).\\
\#	&Replaced by a digit (or nothing if there aren't enough). Often used to show the position of the ",".\\
¤	&This will be replaced by a currency symbol, such as \$ or USD. Note: by default a space is automatically added between letters in a currency symbol and adjacent numbers. So you don't need to add a space between them if your language writes \enquote{\$12} but \enquote{USD 12}.\\
\%	&This marks a percent format. The \% symbol may change position, but must be retained.\\
E	&This marks a scientific format. The E symbol may change position, but must be retained.\\
'	&If any of the above characters are used as literal characters, they must be quoted with ASCII single quotes. For example, in the Short Numbers if a period needs to be used to mark an abbreviation, it would appear as:
0.0 tis'.'
not
0.0 tis.\\
\ldots;\ldots	&If your language uses different formats for negative numbers than just adding "-" at the front, you can put in two patterns, separated by a semicolon. The first will be used for zero and positive values, while the second will be used for negative values.
For example: |#,##|0.00¤;(|#,##|0.00¤) is used to make negative currencies appear like \enquote{(1'234,56£)} instead of \enquote{-1'234,56£}. That is used for formatting currency amounts in English, but not for general-purpose decimal numbers.\\
\bottomrule
\end{longtable}

\section{Characters}
The |<characters>| element provides optional information about characters that are in common use in the locale, and information that can be helpful in picking resources or data appropriate for the locale, such as when choosing among character encodings that are typically used to transmit data in the language of the locale. It typically only occurs in a language locale, not in a language/territory locale.

\begin{quote}
|<exemplarCharacters>[a-zåæø]</exemplarCharacters>|
\end{quote}

The exemplar character set contains the commonly used letters for a given modern form of a language, which can be for testing and for determining the appropriate repertoire of letters for charset conversion or collation. ("Letter" is interpreted broadly, as anything having the property Alphabetic in the [UCD], which also includes syllabaries and ideographs.) It is not a complete set of letters used for a language, nor should it be considered to apply to multiple languages in a particular country. Punctuation and other symbols should not be included.

There are two sets: the main set should contain the minimal set required for users of the language, while the auxiliary exemplar set is designed to encompass additional characters: those non-native or historical characters that would customarily occur in common publications, dictionaries, and so on. So, for example, if Irish newspapers and magazines would commonly have Danish names using å, for example, then it would be appropriate to include å in the auxiliary exemplar characters; just not in the main exemplar set. Major style guidelines are good references for the auxiliary set. Thus for English we have [a-z] in the main set, and [á à ă â å ä ā æ ç é è ĕ ê ë ē í ì ĭ î ï ī ñ ó ò ŏ ô ö ø ō œ ß ú ù ŭ û ü ū ÿ] in the auxiliary set.

In general, the test to see whether or not a letter belongs in the main set is based on whether it is acceptable in that language to always use spellings that avoid that character. For example, the exemplar character set for en (English) is the set [a-z]. This set does not contain the accented letters that are sometimes seen in words like "résumé" or "naïve", because it is acceptable in common practice to spell those words without the accents. The exemplar character set for fr (French), on the other hand, must contain those characters: [a-z é è ù ç à â ê î ô û æ œ ë ï ÿ]. The main set typically includes those letters commonly taught in schools as the "alphabet".

The list of characters is in the Unicode Set format, which allows boolean combinations of sets of letters, including those specified by Unicode properties.

Sequences of characters that act like a single letter in the language — especially in collation — are included within braces, such as [a-z á é í ó ú ö ü ő ű \{cs\} \{dz\} \{dzs\} \{gy\} \ldots]. The characters should be in normalized form (NFC). Where combining marks are used generatively, and apply to a large number of base characters (such as in Indic scripts), the individual combining marks should be included. Where they are used with only a few base characters, the specific combinations should be included. Wherever there is not a precomposed character (e.g. single codepoint) for a given combination, that must be included within braces. For example, to include sequences from the Where is my Character? page on the Unicode site, one would write: [\{ch\} \{tʰ\} \{x̣\} \{ƛ̓\} {ą́} {i̇́} {ト゚}], but for French one would just write [a-z é è ù ...]. When in doubt use braces, since it does no harm to included them around single code points: e.g. [a-z \{é\} \{è\} \{ù\} ...].

If the letter 'z' were only ever used in the combination 'tz', then we might have [a-y {tz}] in the main set. (The language would probably have plain 'z' in the auxiliary set, for use in foreign words.) If combining characters can be used productively in combination with a large number of others (such as say Indic matras), then they are not listed in all the possible combinations, but separately, such as:

{\panunicode [‌ ‍ ॐ ०-९ ऄ-ऋ ॠ ऌ ॡ ऍ-क क़ ख ख़ ग ग़ घ-ज ज़ झ-ड ड़ ढ ढ़ ण-फ फ़ ब-य य़ र-ह ़ ँ-ः ॑-॔ ऽ ् ॽ ा-ॄ ॢ ॣ ॅ-ौ] }

The exemplar character set for Han characters is composed somewhat differently. It is even harder to draw a clear line for Han characters, since usage is more like a frequency curve that slowly trails off to the right in terms of decreasing frequency. So for this case, the exemplar characters simply contain a set of reasonably frequent characters for the language.

The ordering of the characters in the set is irrelevant, but for readability in the XML file the characters should be in sorted order according to the locale's conventions. The set should only contain lower case characters (except for the special case of Turkish and similar languages, where the dotted capital I should be included); the uppercase letters are to be mechanically added when the set is used. For more information, see [Data Formats] and the discussion of Special Casing in the Unicode Character Database.

For example for the locale |se| for Northern Sami, we have:


\begin{longtable}{l p{8cm}}
\toprule
Attribute             & Value \\
\midrule
exemplar characters   & a á b c č d đ e f g h i j k l m n ŋ o p r s š t ŧ u v z ž\\
exemplar characters auxiliary  & à ç é è í ń ñ ó ò q ú w x y ü ø æ å ä ã ö\\
exemplar characters index  &A Á B C Č D Đ E É F G H I J K L M N Ŋ O P Q R S Š T Ŧ U V W X Y Z Ž Ø Æ Å Ä Ö\\
exemplar characters numbers &  , \% ‰ + − 0 1 2 3 4 5 6 7 8 9\\
\bottomrule
\end{longtable}


\begin{longtable}{l p{8cm}}
\toprule
Attribute             & Value \\
\midrule
 exemplar characters &a b c ç d e f g ğ h ı i İ j k l m n o ö p r s ş t u ü v y z\\
exemplar characters  auxiliary & á à ă â å ä ã ā æ é è ĕ ê ë ē í ì ĭ î ï ī ñ ó ò ŏ ô ø ō œ q ß ú ù ŭ û ū w x ÿ\\
 exemplar characters index & A B C Ç D E F G H I İ J K L M N O Ö P Q R S Ş T U Ü V W X Y Z\\
exemplar character numbers & \- , . \% ‰ + 0 1 2 3 4 5 6 7 8 9\\
exemplar characters punctuation &  - ‐ – — , ; : ! ? . … ' ‘ ’ " “ ” ( ) [ ] § @ * / \& \# † ‡ ′ ″\\
\bottomrule
\end{longtable}


\paragraph{ellipsis}The ellipsis element provides patterns for use when truncating strings. There are three versions: initial for removing an initial part of the string (leaving final characters); medial for removing from the center of the string (leaving initial and final characters), and final for removing a final part of the string (leaving initial characters). For example, the following uses the ellipsis character in all three cases (although some languages may have different characters for different positions).

\begin{longtable}{ll}
Ellipsis final & \{0\}… \\           
Ellipsis initial & …\{0\} \\         
Ellipsis medial  & \{0\}…\{1\} \\       
Ellipsis word-final & \{0\} … \\     
Ellipsis word-initial & … \{0\} \\   
Ellipsis word-medial & \{0\} … \{1\}\\
\end{longtable} 

\paragraph{List patterns} List patterns can be used to format variable-length lists of things in a locale-sensitive manner, such as \enquote{Monday, Tuesday, Friday, and Saturday} (in English) versus \enquote{lundi, mardi, vendredi et samedi} (in French). For example, consider the following example:

\begin{phdverbatim}
  <listPatterns>
    <listPattern>
	   <listPatternPart type="start">{0}, {1}</listPatternPart>
		<listPatternPart type="middle">{0}, {1}</listPatternPart>
		<listPatternPart type="end">{0}, and {1}</listPatternPart>
		<listPatternPart type="2">{0} and {1}</listPatternPart>
    </listPattern>
		<listPattern type="or">
			<listPatternPart type="start">{0}, {1}</listPatternPart>
			<listPatternPart type="middle">{0}, {1}</listPatternPart>
			<listPatternPart type="end">{0}, or {1}</listPatternPart>
			<listPatternPart type="2">{0} or {1}</listPatternPart>
	</listPattern>
	<listPattern type="unit">
			<listPatternPart type="start">{0}, {1}</listPatternPart>
			<listPatternPart type="middle">{0}, {1}</listPatternPart>
			<listPatternPart type="end">{0}, {1}</listPatternPart>
			<listPatternPart type="2">{0}, {1}</listPatternPart>
	</listPattern>
	<listPattern type="unit-narrow">
			<listPatternPart type="start">{0} {1}</listPatternPart>
			<listPatternPart type="middle">{0} {1}</listPatternPart>
			<listPatternPart type="end">{0} {1}</listPatternPart>
			<listPatternPart type="2">{0} {1}</listPatternPart>
	</listPattern>
	<listPattern type="unit-short">
			<listPatternPart type="start">{0}, {1}</listPatternPart>
			<listPatternPart type="middle">{0}, {1}</listPatternPart>
			<listPatternPart type="end">{0}, {1}</listPatternPart>
			<listPatternPart type="2">{0}, {1}</listPatternPart>
	</listPattern>
  </listPatterns>
\end{phdverbatim}	

These are not very useful for \tex as most of this type of work can be simply be achieved by just typing the values. the
\pkg{siunitx} offers similar facilities for lists, through the \pkg{translator} and Babel.

\begin{texexample}{clist}{ex:clistuse}
\ExplSyntaxOn
\group_begin:
\def\firsttwowords{~and~}
\def\lasttwowords{ ~and~ }
\def\betweenmorethantwo{ ,~ }
\clist_set:Nn \l_tmpa_clist { a , b , , c , {de} , f }
\clist_use:Nnnn \l_tmpa_clist { \firsttwowords } { ,~ } { ,\lasttwowords }
\group_end:
\ExplSyntaxOff
\end{texexample}


\paragraph{Typographical considerations and Convenience Commands} Some commands,  provided by babel-french are intended to make typesetting according to French typographical conventions easier. Some twenty three conditionals, which more or less affect typographical rules or conventions are mentioned in Babel (see p.43, frencgb.pdf).

    \begin{enumerate}
       \item Hyphenation parameters such as lefthyphenmin and righthyphenmin are defined for many of the languages.
             
             \begin{tabular}{lll}
             \toprule
               Language        & \cs{lefthyphenmin} & \cs{righthyphenmin}\\
             \midrule  
               Finnish         &    2               & 2                   \\
               French          &2                   & 2                   \\
             \bottomrule  
             \end{tabular}
       \item Delimiters (Quotation marks): Delimiters according to CLDR terminology are the characters used for quoting texr. For example in UK English they are the \enquote{curly} right and left forms as in this \enquote{this phrase}. the alternate forms are for embedded quotations such as \enquote{He yelled \enquote{Stop!}, and turned around.} Babel for many of the languages provides macros to enclose text in quotes|\og| and |\fg|.
       \item Typesetting of superscripts such as nth etc. In the French section of Babel this is defined as \docAuxCommand{up}, used
             as M|\up|me \foreignlanguage{french}{M\up{me}}.
       \item French spacing. 
       \item Spacing before punctuation. With Babel and LuaLaTeX a lua script is loaded, that uses callbacks to intercept
             the punctuation and add the appropriate node attributes. The callbacks are fairly comprehensive and cater for
             some edge cases such as 1sp columns etc.
       \item For the other engines it falls back to active characters or to XeTeX character classes. 
       \item Caption separators. In French, captions in figures and tables should never be printed as \enquote{Figure 1:} which is the default in standard \latexe classes; the \enquote{:} is made active too late, no space is added befre it. With \lualatex and \xelatex, this glitch does not occur if you use Babel, you should get \enquote{Figure 1\thinspace:} which is correct in French. 
       \item \textit{Ellipsis Patterns}.  Ellipsis patterns are used in a display when the text is too long to be shown. It will be used in environments where there is very little space. \tex traditionally provided \docAuxCommand{ldots}. With unicode it should be just one character; and where that really can't work, the CLDR specification mentions that it should be as short as possible. 

There are three different possible patterns that need to be translated. Typically the same character is used in all three, but three choices are provided just in case different characters would be appropriate in different contexts, for some languages.

       Babel provides for French macros and switches to allow for the extra spacing required in French typography.

       \item The \pkg{bigfoot} package deeply changes the way footnotes are handled, including providing its own output routine. When |bigfoot| is loaded babel-french drops the customization of footnotes. The layout of footnotes does not depend on the language, as babel's documentation state, it will look wrong if if two footnotes on the same page are looking different because one was called in a French part, the other one in English.  The rest of the code deals in detail as to how to handle the various
       packages and footnotemark.  
     \end{enumerate}
   
   
\paragraph{Babel shorthands}      
My biggest concern with Babel is the ordering of packages due to all the redefinitions and in having to execute most of the code at the |AtBeginDocument| hook.      
    
The way the |PHD| package works is that the user will be provided with a style file, providing all the settings. These can be named. For example |thesis|. Such a style file can be easily be changed to |thesis french|, where for example the field \docAuxKey[phd]{caption separator}{} is set to |caption separator=french colon|.     

\section{Fonts for all the world's scripts and languages}

\epigraph{If you steal from one author it's plagiarrism, if you steal from many, it's research.}{
---Wilson Mizner}

Besides the issues with different languages, hyphenation and caption names, there is also the difficulties with fonts. Unless the current font has the necessary glyphs it will either print junk characters or we get the unicode no glyph symbol.

Many commercial as well as open source fonts exist that can be used to typeset text the world's scripts and languages. The aim of this section of the documentation is to present an overview of the most common scripts represented in the Unicode~7.0 standard. All the examples require the use of the \XeTeX\ or \LUATEX engine. In addition you need to have a copy of the font on your own system. If you do not have them, the font loading mechanism of \XeTeX\ or \LUATEX will take some time to search all the directories and slows compilation tremendously. 

\subsection{Pan-Unicode Fonts}

Thousands of fonts exist on the market, but fewer than a dozen fonts—sometimes described as ``pan-Unicode" fonts—attempt to support the majority of Unicode's character repertoire. Instead, Unicode-based fonts typically focus on supporting only basic |ASCII| and particular scripts or sets of characters or symbols. Several reasons justify this approach: applications and documents rarely need to render characters from more than one or two writing systems; fonts tend to demand resources in computing environments; and operating systems and applications show increasing intelligence in regard to obtaining glyph information from separate font files as needed, i.e. font substitution. Furthermore, designing a consistent set of rendering instructions for tens of thousands of glyphs constitutes a monumental task; such a venture passes the point of diminishing returns for most typefaces.

The \texttt{NotoSerif} fonts from Google\footnote{\protect\url{http://www.google.com/get/noto/}} have good support for 96 language fonts and the list is growing. Since these are widely available most of the scripts that follow use these fonts. Follow the instructions at the website to install them. It is just a matter of dragging them into the fonts folder for most operating systems.

Another freeware pan-Unicode font is \docFont{Titus}
This is an extended version of this font is TITUS Cyberbit Unicode, includes 36,161 characters in v4.0.

On Windows systems |Arial Unicode MS| contains glyphs for all code points within the Unicode Standard version 2.1.  

The code2000 font provides 63546 glyphs and is the nearest font to a universal font to handle Unicode. Unfortunately development stopped in 2008. As a comparison Linux Libertine O, provides 2674 glyphs. \label{code2000}

CJK fonts naturally will have the most glyphs, \idxfont{MingLiU} 34046 glyphs and is a very good font for CJK typesetting. Google in conjunction with Adobe also provides a fee CJK font.

The \href{http://ftp.gnu.org/gnu/freefont/}{FreeFont Project} currently supports most of the useful set of free outline (i.e. OpenType) fonts covering as much as possible of the Unicode character set. The set consists of three typefaces: one monospaced and two proportional (one with uniform and one with modulated stroke). 

The idea of having lots of different writing systems into a single font at all? How good does such a font need to be?
There are two extreme views.  The first one is that glyphs in a font shold comprise a unified design entity. This in practice makes sense only within a single language script. Different script systems, such a Latin, Arabic and Devanagari, have different typesetting traditions and conventions.  A good discussion of the advantages and disadvantages can be found at the gnu website \footnote{\protect\url{https://www.gnu.org/software/freefont/articles/Why_Unicode_fonts.html}}. For TeX it is a better proposition in order to avoid switching of fonts that can distract the writer. At least one requires fonts that are inclusive of one's usage. 

\section{The \texttt{ucharclasses} package}

For multilingual texts font switching can become cumbersome. The use of a pan-Unicode font as the default can help. However, if the languages are distinct enough to use different Unicode blocks, which are not covered by the \pkg{polyglossia} package Mike Kamermans' package \pkg{ucharclasses} can be used. This package only works with \xelatex and does not work with LuaTeX. 

\begin{verbatim}
% and the font switching magic
\usepackage[CJK, Latin, Thai, 
           Sinhala, Malayalam, 
           DominoTiles, 
           MahjongTiles]{ucharclasses}
\usepackage{fontspec}

\ifxetex
% default transition uses the widest coverage font I know of
  \setDefaultTransitions{\fontspec{Code2000.ttf}}{}

% overrides on the default rules for specific informal groups
  \setTransitionsForLatin{\fontspec{Palatino Linotype}}{}
  \setTransitionsForCJK{\fontspec{code2000.ttf}}{}%HAN NOM A
  \setTransitionsForJapanese{\fontspec{code2000.ttf}}{}%Ume Mincho

% overrides on the default rules for specific unicode blocks
  \setTransitionTo{CJKUnifiedIdeographsExtensionB}{\fontspec{SimSun-ExtB}}
  \setTransitionTo{Thai}{\fontspec{IrisUPC}}
  \setTransitionTo{Sinhala}{\fontspec{Iskoola Pota}}
  \setTransitionTo{Malayalam}{\fontspec{Arial Unicode MS}}
\ifxetex
\end{verbatim}

{
\newfontfamily\mahjong{FreeSerif.ttf}
\mahjong
domino tiles, 🁇 🀼 🁐 🁋 🁚 🁝, and mahjong tiles: 🀑 🀑 🀑 🀒 🀒 🀒 🀕 🀕 🀕 🀗 🀗 🀗 🀅 🀅 (using FreeSerif)

}

The interaction between Polyglossia and Fontspec can result in infinite looping and memory leaks. I do not recommend that you use these commands as yet. The use of the charclasses will also slow down compilation possibly by a factor of 10.



\section{PhD Settings}

The \pkg{phd} provides support both for scripts, as well as language settings. A script setting sets the system to use appropriate fonts and if the script is associated with a unique language it will automatically handle language settings. Alternatively for multi-language scripts such as the Latin script, the language key can be used. This will automatically setup the language and an appropriate default font. 

\begin{docKey}[phd]{script} { = \meta{script name}} {default none, initial US English}{}
\end{docKey}

\begin{docKey}{language}{ =\meta{language name}}  {default none, US English}
The key language sets the main language for the document. This language will be used for the sectioning commands and common string translations.

If the language is English Polyglossia or Babel are not loaded automatically. If the language is other than English we load either Babel or Polyglossia depending on the engine used.
\end{docKey}


\begin{docKey}{languages}{ = \meta{language1, language2, language3}}  {}
The key |languages|, determines all the other scripts available for typesetting. For each language default font commands are create automatically. The aim is to be able to run a fully multilingual system with the minimum of upfront settings. These we leave to customize in the style template files.
\end{docKey}

\begin{docKey}{greek font}{ = \meta{options}\meta{font file}}  {}
The package comes with numerous language and appropriate default fonts
for each operating system. 
\end{docKey}

\cxset{chapter opening=any}


\section{IPA Transcriptions}

Language is spoken and writing systems need to cater for the individuality of the sounds for a particular language. Many of the world's languages facing extinction do not have a written representation for their language. The \textit{lingua franca} of linguists is the  International Phonetic Alphabet (IPA). This is an alphabetic system of phonetic notation based primarily on the Latin alphabet. It was devised by the International Phonetic Association in the late 19th century as a standardized representation of the sounds of spoken language. The IPA is used by lexicographers, foreign language students and teachers, linguists, speech-language pathologists, singers, actors, constructed language creators and translators. \footcite{ipa}

In the chapters that follow, I have used it extensively. The IPA Handbook is an essential reference work for all those involved in the analysis of speech. Besides the IPA notation a knowledge of linguistic terms is also necessary. A short guide is provided. In \latex the \pkg{tipa} can be of help, but soon a good keyboard layout will be better.

The IPA Extensions block has been present in Unicode since version 1.0, and was unchanged through the unification with ISO 10646. The block was filled out with extensions for representing disordered speech in version 3.0, and Sinology phonetic symbols in version 4.0.[4]

\bigskip
{\catcode`\"=12
\unicodetable{arial}{"0250,"0260,"0270,"0280,"0290,"02A0}
}
\bigskip

\def\schwa{{\arial \char"0259}}

Besides the symbols, there are numerous diacritics and markers.

With Unicode and the right font, there is no problem  in typesetting IPA phonetic symbols. However the problem is the input.

I recommend that you get familiar with a Unicode IPA keyboard overlay. I have used Keyman. When the keyboard is turned on, certian keys (`,@,=) are activated.

As long as your editor allows Unicode input (most do these days) and you're compiling with XeLaTeX or LuaLaTeX, you can just use the IPA keyboard to type directly into the editor just as you can in most other applications. You can also copy and paste your Unicode text from other applications too. 

For example take the transcription of a Hittite word written as \emph{ši-ú-ni-iš}. Here we can typeset it faster by the Hittite package, and numerous others as |\thittite{si-u-ni-is}|. The software is intelligent enough to add the diacritics. They are also expandable. 

\section{The world's scripts}

Anatolian hieroglyphs were first thought to have been used for the Hittite language, 

Anatolian Hieroglyphs is a Unicode block containing Anatolian hieroglyphs, used to write the extinct Luwian language, because they first appeared on personal seals from Hattusha, the capital of the Hittite Empire. While
Hittites did make use of the characters on seals and on their monumental inscriptions, the characters were
used as text primarily for the related language Luwian; a few glosses in Urartian and some divine names
in Hurrian are known to be written in Anatolian Hieroglyphs. Most of the texts are monumental stone
inscriptions, though some letters and accounting documents have been preserved inscribed on strips of
lead. 

\newfontfamily\anatolian{Anatolian}
{
\catcode`\"=12
\unicodetable{anatolian}{"14400,"14410,"14420,"14430,"14440,"14450,"14460,"14470,"14480,"14490,"144A0,"144B0,"144C0,"144C0,"144D0,"144E0,"144F0,%
 "14500,"14510,"14520,"14530,"14540,"14550,"14560,"14570,"14580,"14590,"145A0,"145B0,"145C0,"145D0,"145E0,"145F0,%
 "14600,"14610,"14620,"14630,"14640}
}

























\parindent1em


\chapter{Internationalization and Globalization}


\section{Introduction}

In this Chapter we discuss the requirements for localization of software and how this can be applied to \latex. In a way this chapter overlaps the one on languages. However, here we focus mostly on LuaTeX solutions. We also extend the discussion to calendric and solar calculations.

Internationalization is the process of designing a software application so that it can potentially be adapted to various languages and regions without engineering changes. Localization is the process of adapting internationalized software for a specific region or language by adding locale-specific components and translating text. Localization (which is potentially performed multiple times, for different locales) uses the infrastructure or flexibility provided by internationalization (which is ideally performed only once, or as an integral part of ongoing development).\index{internationalization}\index{globalization}

The development of routines for software internationalization and globalization has been an ongoing effort for many years. Currently the accepted method for building such software is the use of i18n. This is an abbreviation of the first letter and last letter of the word internationalization and the 18 is the number of characters in the word.

Internationalization based on i18n is not an easy task for \LaTeX. To an extend some of the issues have been removed with the use of Babel and Polyglossia that provide translation strings for many of the worlds scripts. The de facto resource for internationalization is the Unicode Consortium’s \href{http://cldr.unicode.org/}{CLDR} project.\index{i18n}

\section{Enforcing local styles}

To understand the magnitude of the problem let us look at some of the easier parts of localizing. Consider the Greek days of the week.
\medskip
\begin{trivlist}\item[]\panunicode
\begin{tabular}{llll}
\toprule
Day &Normal Form &Abbreviation &Narrow\\
Monday &Δευτέρα &Δευ. &Δ. \\
\midrule
\end{tabular}
\end{trivlist}

In Greek the abbreviated form, is always capitalized and a stop is provided. The same is true for the month. The narrow form can give problems, unless it is for calendars, where the content is clear. This is because "{\panunicode Π}" are the initials for both ``{\panunicode Πέμπτη}" (Thursday), and ``{\panunicode Παρασκευή}" (Friday). 

In date formats with long month format, that do not include the day, the full month form should be used.
In date formats with long month format, that also include the day, the long date format should be used.

If limited space is available, it is possible to omit the period in the abbreviated form of months, but this should be used only when there is a serious technical restriction

Ultimately, we are aiming at providing the necessary rules to build an automated style that can be used by the system.
                

\section{Locales}
\index{locale}

In computing, a \emph{locale} is a set of parameters that defines the user's language, country and any special variant preferences that the user wants to see in their user interface. Usually a locale identifier consists of at least a \textit{languag}e identifier and a \textit{region} identifier.

On POSIX platforms such as Unix, Linux and others, locale identifiers are defined similar to the BCP 47 definition of language tags, but the locale variant modifier is defined differently, and the character set is included as a part of the identifier. It is defined in this format: |[language[_territory][.codeset][@modifier]]|. (For example, Australian English using the UTF-8 encoding is en\_AU.UTF-8.)

For \latex these ``locales'' can be thought of as the settings of language keys through Babel and Polyglossia. These settings have served the community well for many years, but a litany of duct taping through other packages are a testimony to their limitations. Packages for dates, time and number formatting have been developed to assist. Here is my attempt to put the solution on a better footing and to start providing mechanisms via LuaTeX for a 'plugin'
architecture to find improve solutions. 

\section{Common Locale Data Repository}

The Common Locale Data Repository Project, is a project of the Unicode Consortium to provide locale data in the XML format for use in computer applications. CLDR contains locale specific information that an operating system will typically provide to applications. CLDR is written in LDML (Locale Data Markup Language). The information is currently used in International Components for Unicode, Apple's Mac OS X, OpenOffice.org, and IBM's AIX, among other applications and operating systems

\begin{enumerate}
\item Translations for language names.
\item Translations for territory and country names.
\item Translations for currency names, including singular/plural modifications.
\item Translations for weekday, month, era, period of day, in full and abbreviated forms.
\item Translations for timezones and example cities (or similar) for timezones.
\item Translations for calendar fields. This is useful especially in conjuction with PGF presentational forms.
\item Patterns for formatting/parsing dates or times of day.
\item Examplar sets of characters used for writing the language.
\item Patterns for formatting/parsing numbers.
\item Rules for language adapted collation. \label{collation}
\item Rules for formatting numbers in traditional numeral systems (like Roman numerals, Armenian numerals, ...).
\item Rules for spelling out numbers as words.
\item Rules for transliteration between scripts. A lot of it is based on BGN/PCGN romanization.
\item Rules for \emph{delimiters} such as quotations and question marks.
\end{enumerate}

Currently the consortium’s distribution make the data available in both json and xml formats.  These files hold data for a specific \emph{locale}. Sadly missing are any document sectioning information that would have enabled the incorporation of the above into LaTeX and overcoming some of the Babel and Polyglossia limitations.

We do not need many of the files provided by the CLDR unicode consortium and others we are missing. Take for example the |delimiters| file. 

\bgroup
\scriptsize
\begin{phdverbatim}
  "main" = {
    "ff": {
      "identity": {
        "version": {
          "_cldrVersion": "26",
          "_number": "$Revision: 10739 $"
        },
        "generation": {
          "_date": "$Date: 2014-08-07 12:54:13 -0500 (Thu, 07 Aug 2014) $"
        },
        "language": "ff"
      },
      "delimiters": {
        "quotationStart": "„",
        "quotationEnd": "”",
        "alternateQuotationStart": "‚",
        "alternateQuotationEnd": "’"
      }
    }
  }
}
\end{phdverbatim}
\egroup

Of course the |Json| format as it is, is not readable by Lua a format such as:

\begin{verbatim}
delimiters = {
        quotationStart = "«",
        quotationEnd = "»",
        alternateQuotationStart = "\"",
        alternateQuotationEnd = "\""
      }
\end{verbatim}

\begin{texexample}{i18n}{i18-1}

\panunicode
\begin{luacode*}
-- mock the delimiters from the json
-- file
greekname = 'el'
delimiters = {
        quotationStart = "«",
        quotationEnd = "»",
        alternateQuotationStart = [["]],
        alternateQuotationEnd = [["]]
      }
tex.print(delimiters.quotationStart .. 'test' .. delimiters.quotationEnd)
tex.print ([[\gdef\]] .. greekname .. [[quote#1{\directlua{tex.sprint(delimiters.quotationStart .. '#1' .. delimiters.quotationEnd)}}]])
\end{luacode*}

\def\elquote#1{%
  \directlua {tex.sprint(delimiters.quotationStart .. '#1' .. delimiters.quotationEnd)}
}
\end{texexample}



This is of course a much more simplified way of what one needs to program for a full system. The advantage
of producing the \tex definition also through LuaTeX is that we can keep all the code in one place and econd, we can avoid |\csname| costructs.
\begin{texexample}{elquote}{}
\elquote{This is some longer text in Greek quotes.}
\end{texexample}

I have opted to incorporate these files in the |json| format and provide routines for interfacing via the \pkgname{phd} package.  The reason for opting for a json format, is my other attempts to interface the package with |couchdb|.  My preference for a Nosql type of database, is that  they are better suited in handling data that is commonly  found in documents and also many of the routines will be interchangeable for web applications. I am also hoping that the collation information (see \ref{collation}), will eventually lead to better indices, a subject left untouched in the current distribution.\index{json}

\section{The package phd approach}

The package |phd| packge takes an approach to use only json resource files for the provision of language dependent information, rather than TeX commands alone, as is done by Babel and Polyglossia. 

\section{Language and Region Tags}
\index{tags>regions}\index{tags>language}

Languages are represented by tags such as "en"  for English or "el" for Greek. Other languages have no significant variation and are represented by a language subtag such as "en-US".  The names are mostly intuitive, but in many case bear no relationship to their English names, for example Armenian is coded as \textbf{hy}. There is a useful utility at the SIL website for viewing these codes.\footnote{\protect\url{http://www-01.sil.org/iso639-3/codes.asp?order=reference_name&letter=\%25}.} Note that the CLDR database does not cover all the languages listed in the ISO-639.\footcite{iso639} \index{ISO-639}

The language tags are based on the BGN which is mapped to languages based on ISO-639-1.

ISO 639-2 is the alpha-3 code in Codes for the representation of names of languages-- Part 2. There are 21 languages that have alternative codes for bibliographic or terminology purposes. In those cases, each is listed separately and they are designated as "B" (bibliographic) or "T" (terminology). In all other cases there is only one ISO 639-2 code. Multiple codes assigned to the same language are to be considered synonyms. ISO 639-1 is the alpha-2 code.

We will describe the tables using the English language, which is normally the default and Greek as a second language, as the script is distinctive enough to demonstrate their use. We will also explain Lua routines available via the \pkgname{phd} that are provided as alternatives to Babel and Polyglossia.

{layout.lua}

{layout.orientation.characterOrder} = |left_to_right| or |right_to_left|

layout.orientation.lineOrder = |top_to_bottom|

Example \ref{i18-1} loads the Greek internationalization file |layout| and prints the two fields. Before we send it to
the TeX typesetter we sanitize the string underscores using |gsub|. For illustration purposes we have used |gsub| both as an object method and as a function.

\begin{texexample}{i18n}{i18-1}
\begin{luacode}
local c = require("i18n.el.layout")
local s1 = string.gsub(c.el.layout.orientation.characterOrder, '_', '\\textunderscore ')
local s2 = c.el.layout.orientation.lineOrder:gsub('_', '\\textunderscore ')
tex.print('typeof :', type(c))
tex.print(s1, '\\par', s2)
\end{luacode}
\end{texexample}

Of course for Greek the above information is hardly necessary, but at the level of Lua programming, if we are automating the switching of text direction Greek text might signal a change in direction. Let us have another try using the same code for arabic text. All we have to change is the \textbf{el} to \textbf{ar}.

\begin{texexample}{i18n}{i18-2}
\begin{luacode}
local c = require("i18n.ar.layout")
local s1 = string.gsub(c.ar.layout.orientation.characterOrder, '_', '\\textunderscore ')
local s2 = c.ar.layout.orientation.lineOrder:gsub('_', '\\textunderscore ')
tex.print('typeof :', type(c), '\\par')
tex.print(s1, '\\par', s2)
\end{luacode}
\end{texexample}

In the next example we get the string for the first month of the year in the ``abbreviated'' style. I have changed the json
strings directly to Lua for this file to speed up processing.

\begin{texexample}{i18n}{i18-2}
\begin{luacode}
local c = require("i18n.el.cagregorian")
local months = c.el.dates.calendars.gregorian.months.formats
local days = c.el.dates.calendars.gregorian.days.formats

tex.print("\\begin{tabular}{ll} ")
for i=1,12 do
  tex.sprint(i.." &"..months.wide[i].."\\\\ ")
end
tex.print("\\end{tabular}")
\end{luacode}
\end{texexample}

Printing directly to the document has many benefits but does slow developemnt, both of the code as well as the document. Another distraction is transferring arguments from \tex to Lua and vice versa.

Similarly we can print the months in the Italian language by loading the \textbf{i18n.italian} module and iterating through the month strings. I am still thinking about the interface and the best way forward to provide an easy to use and remember interface. 


Let us now develop a longer example. We will load a number of languages and typeset a table for the different months.
Since we are running the example directly in the document, some patience is required. 

\bigskip 

\begin{texexample}{Month string in various languages}{ex:transl}
\bgroup
\parindent0pt
\newfontfamily\langtable{code2000}
\langtable
\scriptsize
\begin{luacode} 

c = require("i18n.irish")
d = require("i18n.russian")
e = require("i18n.latin")
f = require("i18n.german")
g = require("i18n.kannada")
h = require("i18n.lao")
j = require("i18n.turkish")
k = require("i18n.albanian")

local count=0

local months_irish = c.irish.months
local months_russian = d.russian.months
local months_latin = e.latin.months 
local months_german = f.german.months
local months_kannada = g.kannada.months
local months_lao    = h.lao.months
local months_turkish = j.turkish.months
local months_albanian = k.albanian.months
local centering = function()
                     tex.print("\\centering")
end

local par = function()
               tex.print("\\par")   
end

local tabular = function() 
	tex.print("\\begin{tabular}{clllllll}")
	tex.sprint("\\toprule")
end	


local endtabular = function()
	tex.print("\\bottomrule")
	tex.print("\\end{tabular}")
	tex.print("\\medskip")
end

local eol = function()
  return("\\\\")
end


-- center the table
centering()
tabular()
tex.sprint("Month","&Irish", "&Russian", "&Latin", "&Kannada", "&Lao","&Turkish","&Albanian", eol())
tex.sprint("\\midrule")
for i = 1,12 do
   count = i
   tex.sprint(i.."&", months_irish[i],
                 "&",months_russian[i], 
                 "&",months_latin[i], 
                 "&", months_kannada[i], 
                 "&"..months_lao[i], 
                 "&"..months_turkish[i],
                 "&"..months_albanian[i],
                 eol() )
end  
endtabular()
par()

\end{luacode} 
 
\egroup
\end{texexample}

Now some explanation for the code. We started by loading the necessary libraries for the languages that we wanted to print the month strings and allocated them to local variables.

We then iterated through the twelve months of the gregorian table and typeset them. We could have put the languages in a Lua table and iterated over them. I haven't done it so that the code is clearer. I tried to keep the API functions separate as much as possible. We also defined a font using \docAuxCommand{newfontfamily} of the \pkg{fontspec} package to ensure that we can print the Asian and Cyrillic scripts.

The long javanesque object notations make it difficult to work, but once they are set in functions and locals, development is fast. After the detour to explore the i18n tables and available information, we are now ready to tackle the production of multi-lingual calendars and to complete are library on internationalization. Before we do that a detour to understand
the complexity of calendrical calculations and some historical information is required.
\vfill




\DocInput{\jobname.dtx}

\nocite{*}
\printbibliography
%\printindex
 %
% 
\end{document}
%</driver>
% \fi
% 
%  \CheckSum{0}
%  \CharacterTable
%  {Upper-case    \A\B\C\D\E\F\G\H\I\J\K\L\M\N\O\P\Q\R\S\T\U\V\W\X\Y\Z
%   Lower-case    \a\b\c\d\e\f\g\h\i\j\k\l\m\n\o\p\q\r\s\t\u\v\w\x\y\z
%   Digits        \0\1\2\3\4\5\6\7\8\9
%   Exclamation   \!     Double quote  \"     Hash (number) \#
%   Dollar        \$     Percent       \%     Ampersand     \&
%   Acute accent  \'     Left paren    \(     Right paren   \)
%   Asterisk      \*     Plus          \+     Comma         \,
%   Minus         \-     Point         \.     Solidus       \/
%   Colon         \:     Semicolon     \;     Less than     \<
%   Equals        \=     Greater than  \>     Question mark \?
%   Commercial at \@     Left bracket  \[     Backslash     \\
%   Right bracket \]     Circumflex    \^     Underscore    \_
%   Grave accent  \`     Left brace    \{     Vertical bar  \|
%   Right brace   \}     Tilde         \~}
%
%
%
% \changes{1.0}{2013/01/26}{Converted to DTX file}
%
% \DoNotIndex{\newcommand,\newenvironment}
% \GetFileInfo{phd.dtx}
% 
%  ^^A\def\fileversion{v1.0}          
%  \def\filedate{2012/03/06}
% \title{The \textsf{phd} package.
% \thanks{This
%        file (\texttt{phd.dtx}) has version number ^^A\fileversion, last revised
%        \filedate.}
% }
% \author{Dr. Yiannis Lazarides \\ \url{yannislaz@gmail.com}}
% \date{\filedate}
%
%
% 
% ^^A\maketitle
% 
% ^^A\frontmatter
%  ^^A\coverpage{./images/hine02.jpg}{Book Design }{Camel Press}{}{}
%  \newpage
% ^^A\secondpage
% \pagestyle{empty}
%
%
% 
%
%
% \pagestyle{headings}
% \raggedbottom
%  \OnlyDescription
%
%^^A\StopEventually{\printindex}

% \CodelineNumbered
% \pagestyle{headings}
% 
% 
% ^^A\part{IMPLEMENTATION AND FRIENDS}
% 
%
% \chapter{Scripts Package Code Implementation Objectives and Strategy}
% 
% \epigraph{
% I was reflecting on the convoluted Java frameworks widely adopted at work. Those hefty frameworks brought coding structures and conventions to large engineering teams; meanwhile, they also sucked the fun of programming like a Pastafarian monster slurping all the tomato sauce on a plate of spaghetti.
%}{\href{http://blog.zmxv.com/2015/07/code-golf-at-google.html}{Zhen Wang}}
%
% \section{Specification}
%   We start by outlining what we are trying to achieve with this package:
%
%   \begin{enumerate}
%   \item To provide a declarative interface to enable modifying headings by
%       setting keys, rather than writing macros.  
%   \item The interface must be exhaustive enabling individual elements of a sectioning
%       block to change colors, widths, fonts etc,
%   \item To provide higher level styles that can style the heading with setting
%       only one key. 
%   \item to devise a scheme where the overall \enquote{shape} of a block be determined
%       using formats.
%   \item To provide a compatibility mode, where documents wishing to test the package
%      can have an easy switch to switch in and out. This is also important for the testing of the package.
%   \item To provide a number of templates that cover most of the typical use case.
%   \item To provide means for a plug-in architecture for extensions.
%   \item Allow the typesetting of any writing system defined in Unicode 10, with the
%      exception of any script that there are no fonts available. 
%   \item Allow the automatic generation of environemnts and text commands for writing systems.
%   \item Interface with the PHD-language modules---which in turn interfaces with Babel and Polyglossia.
%   \item Provide a CLI tool to make it easier to install all the missing fonts.
%   \item Allow for the creation of pan-unicode fonts. 
% \end{enumerate}
% 
% \section{Terminology}
%
%  \begin{description}
%  \item [document] Any written item, as a book, article, or letter, especially 
%                  of a factual or informative nature.
%  \item [heading] A division of a document or document series. For a normal
%        book headings are chapters, sections etc. However we allow for
%        specifying a more complex document divided into books, volumes
%        parts etc. For example the Bible has Books, chapters and verses,
%        where a legal document might require divisions such as clauses.
%        In general these divisions are numbered. These document divisions
%        are stored in the comma list \refCom{phd_book_divisions_clist}.
%  \item [head] A typeset heading, such as chapter head, or section head.
%        This can include a counter, label and title for example, 
%        \emph{Chapter 1 Introduction}.
%  \item [dom] This is a programming interface that provides a structured
%        representation of the document (a tree) and it defines a way
%        that the structure can be accessed. Although \latexe does not
%        offer a standard way to build such a tree (mainly because
%        \tex does not require the marking of paragraphs, it is 
%        useful to think of the document as a tree structure. We also
%        allow for a semi-automated way to build such a tree (with the 
%        exception that paragraphs are not included).
% \item [element] A part of the document tree that can be styled on
%       its own. For example the chapter label, or the section number.
%
% \end{description}
%
% \section{Users}
%  We classify users according to the \LaTeX3 terminology as a) programmers b) template designers
%  and c) authors.
% \subsection{Author}
%  We assume that the author has an exising template which she is using but might want to do
%  some minor modifications, for example use an italic shape for the font of the mark, but an 
%  upright font for the page numbers. 
%
% {\obeylines 
%~~ |\cxset|
%~~~~~|{|
%~~~~~~~~\textit{chapter number color}~~|format          = apa,|
%~~~~~~~~\textit{section title font-size} |font-size   = Large,|
%~~~~~|}|
%}  
%
% We follow the idea of representing the basic elements of documents
% as elements, each one having a parent in order to specify
% the element we need to style as accurate as possible. One can think of
% this approach being congruent with objects in other languages.
% As a matter fact nothing stops us from defining a key value
% interface as shown below.
%
% {\obeylines 
%~~ |\cxset|
%~~~~~|{| 
%~~~~~~~~\textit{header.even.mark.font.size}   = |Large,|
%~~~~~~~~\textit{header.even.mark.font.family} = |serif,|
%~~~~~|}|
%}  
%
% This would pehaps make it easier for the template designer, but I have rejected
% the idea as my aim is to make it easy for the author, who can search the template
% and just enter a couple of new proerty values.
%
% \subsection{Template designer}
% \pagestyle{headings}
% The template designer in the example above would have selected the format style
% from a number of predefined formats (templates) or would have created a style
% called \textit{apa} from an existing template and modified it using declarative
% key style.
%
% \subsection{The programmer}
%
% The programmer in the example above could have created the basic format
% \textit{apa} by using both declarative as well as defining or using existing
% macros. To the programmer we offer an extension mechanism, where the contents
% of a |ps@| command are defined. For example the programmer can define a new
% style using \tikzname, but without having to worry about defining full |ps@|
% and their interface.
%
% \section{Preliminaries}
%
%  Standard file identification. We first announce the package 
%	 and require that it be used with \LaTeX2e. 
% \iffalse
%<*SCRIPTS>
% \fi
%  
%
%    \begin{macrocode}
\NeedsTeXFormat{LaTeX2e}[2017/04/15]%
\RequirePackage[2017/04/15]{latexrelease}
\ProvidesFile{phd-lists}[2015/1/13 v1.0 less preamble (YL)]%
%    \end{macrocode}
%
% \chapter{Scripts and Languages }
% 
%    \begin{macrocode} 
%
\ifengine
  {
    \PassOptionsToPackage{shorthands=off,italian,french,spanish,greek,ngerman,%
                                                 UKenglish}{polyglossia}
    \RequirePackage{polyglossia}
    \setdefaultlanguage{UKenglish}
  }  
  {%
   \PassOptionsToPackage{shorthands=off,italian,french,spanish,greek,ngerman,%
                                                UKenglish}{babel}
   \RequirePackage{babel}
   %babel
   \selectlanguage{UKenglish}
  }
  {\PassOptionsToPackage{shorthands=off,italian,french,spanish,greek,UKenglish }{babel}
    \RequirePackage{babel}
    \setdefaultlanguage{UKenglish}
  }
\begin{otherlanguage}{french}
\global\let\frenchenumerate\enumerate
%\global\let\endfrenchenumerate\endenumerate
\end{otherlanguage}
%
%    \end{macrocode}
%
%    \begin{macrocode}  
\RequirePackage[italian,UKenglish]{xlayouts}
%    \end{macrocode} 
% The clist \refCom{g_phd_scripts_clist} holds a list of all the scripts that have been loaded.
% Managing the user interface is problematic, we will have users that require
% only one script and users that might want all of them.
% There is also the issue between the blurring of alphabets, languages and scripts
% Since we will always specify a pan-unicode font, which we will make available
% with the |phd| package. We map all scripts to this font first.
%
% \begin{docCommand}{g_phd_scripts_clist} {\meta{clist}}
%   Holds a clist of all scripts loaded.
% \end{docCommand}
%
% 
%  Declare two global lists to hold all the scripts available.
% The |\script_prop| holds info for each script loaded
%
%    \begin{macrocode}
\ExplSyntaxOn
\clist_new:N \g_phd_scripts_clist
\clist_new:N \g_phd_noto_clist
\prop_new:N \script_prop
%\input{notolist.txt.tex}
%    \end{macrocode}
%
% \begin{docCommand}{g_phd_noto_clist}{\meta{clist}}
% Holds a list of all noto fonts available.
% \end{docCommand} 
%
% \begin{docCommand}{printnotofontlist}{clist}
% It typesets a list in a two column environment with all the available Noto fonts.
% \end{docCommand}

% 
%    \begin{macrocode}
\cs_set:Npn \printnotofontlist 
  {
    \begin{multicols}{2}
      \clist_map_inline:Nn \g_phd_noto_clist
        {
          ##1\par 
		  }
    \end{multicols}  
  }
%    \end{macrocode}
%	
% 
% 
%    \begin{macrocode}	
\prop_put:Nnn \script_prop {name}{Armenian}
\prop_put:Nnn \script_prop {fonts}{NotoArmenian-Regular.ttf, Others}
\prop_get:NnN \script_prop {fonts}\l_tempa_tl
\prop_put:Nnn \script_prop {group}{Europe}
\prop_get:NnN \script_prop {group} \l_tempa_tl
%    \end{macrocode}
%
% \begin{docCommand}{SetPanUnicodeFont}{\marg{font name}}
%  Sets the pan-unicode font. This font is to be used as a default for all the scripts
%  The user can override it with another font.
% \end{docCommand}
%
%    \begin{macrocode}
\NewDocumentCommand\SetPanUnicodeFont { m }
  {
     \gdef\panunicodefontface{#1}
     \newfontfamily\panunicode[Scale=MatchUppercase]{#1}
  }
%    \end{macrocode}  
%
%  We set the \docAuxCommand{panunicode} to |code2000.ttf|.
%
%    \begin{macrocode}  
\SetPanUnicodeFont{code2000.ttf}    
%    \end{macrocode}

% \begin{docCmd} {makepanfontfamily} { \marg{script name} }
%    
% \end{docCmd}
%    \begin{macrocode}
\cs_gset:Npn \makepanfontfamily#1
  {
%  \newfontfamily\cs:w #1fontfamily\cs_end: { #2 }
  \cs_gset_eq:cN {#1fontfamily} \panunicode
  \cs_gset_eq:cc {#1} {#1fontfamily}
}
%    \end{macrocode}
% 
% \begin{docCmd} {add_a_script:n} { \marg{script name}}
%   Given a script name this function, adds it to the tracking list
%   creates an appropriate envrironment and also a |text<script>| command. 
%   This might overwrite similar commands defined by other
%   packages.
% \end{docCmd}
%    \begin{macrocode}
\cs_gset:Npn \add_a_script:n #1
 {
   \clist_gput_left:Nn \g_phd_scripts_clist {#1 }
   \createscriptenvironment {#1}
   \createtextscript {#1}
 }   
 
 % add a script
\NewDocumentCommand\addascript { m } 
  {
    \add_a_script:n {#1}
  }
  
% Mock an environment 
\gdef\createscriptenvironment #1{
   \exp_after:wN\gdef\csname #1script\endcsname{\group_begin:
      \csname #1fontfamily\endcsname}
   \exp_after:wN\gdef\cs:w end#1script\cs_end:{\group_end: }
}  
\ExplSyntaxOff
%    \end{macrocode}
%  
% \begin{docCommand}{createtextscript}{ \marg{script name}}
%   This creates a command of the form |\text|\meta{script name} i.e., for tibetan
%   it will produce |\texttibetan|
% \end{docCommand}
%    \begin{macrocode}
\ExplSyntaxOn
\cs_gset:Npn \createtextscript #1{
   \long\exp_after:wN\gdef\csname text#1\endcsname ##1
   {
      \group_begin: 
      \cs:w #1fontfamily\cs_end:
        ##1
     \group_end:
   }
}  
%
%
\cs_gset:Npn \makefontfamily#1#2 {
\if_meaning:w\panunicodefontface#2
  \else:
  \exp_after:wN
  \newfontfamily\cs:w #1fontfamily\cs_end: { #2 }
  \cs_gset_eq:cc {#1} {#1fontfamily}
\fi:  
}

\ExplSyntaxOff

\NewDocumentCommand\AddScript { m } {
    \cxset{script/.code=\addascript{##1}}
    \cxset{#1 font/.code=\makefontfamily{#1}{##1}}
    \cxset{script=#1}
    \cxset{#1 font=\panunicodefontface}
}

\cxset{add script/.code = \AddScript{#1}}

\ExplSyntaxOn
\clist_gset:Nn \g_phd_scripts_clist 
  {
      armenian,
      %hebrew,
      % arabic,
      syriac,
      thaana,
      devanagari,
      bamum,
      bengali,
      brahmi,
      buhid,
      bopomofo,
      cham,
      cherokee,
      cjk,
      coptic,
      cypriot,
      %e
      ethiopic,
      georgian,
      glagolitic,
      gurmukhi,
      gujarati,
      kayahli,
      lao,
      lisu,      
      kannada,
      malayalam,
      myanmar,
      ogham,
      oriya,
      oldturkic,
      phoenician, 
      runic,
      tamil,
      thai,
      tibetan,
      tifinagh, 
      telugu, 
      vai,
      rejang,
      saurashtra,
      sinhala,
      sylhetinagari,
      sundanese,%check this
      yi,%check
      meitei,%check
      mongolian,
}

\clist_map_inline:Nn\g_phd_scripts_clist 
  {
    \AddScript{#1}
    \makepanfontfamily {#1}
  }
\ExplSyntaxOff

\newfontfamily\arabicfont[Script=Arabic]{Amiri}
\newfontfamily\arabicfonttt[Script=Arabic,Scale=.75]{DejaVu   Sans Mono}
\newenvironment{Arabic}
   {\bgroup \arabicfont}
   {\egroup}
%    \end{macrocode}
%
% A small utility macro to typeset unicode tables
% examples can be see in the chapters for scripts.
%puts the unicode label (removes last char and adds x)
%
% \begin{docCommand} {putunicode@label} {\marg{unformatted string}} 
%  This macro receives a number in hexadecimal, removes the last
%  0 and replaces it with an x. It then prepends a U+ to fomat it
%  as a Unicode number e.g. U+0100x
% \end{docCommand}
% 
%    \begin{macrocode}
\newcounter{glyph@count}%counts glyphs
%    \end{macrocode}
%		
%		
%    \begin{macrocode}
\ExplSyntaxOn
\def\textU#1{{\unicodenumberfam #1}}
\ExplSyntaxOff
%    \end{macrocode}
%		
%    \begin{macrocode}
\def\putunicode@label#1#2;{%
%    \end{macrocode}
%    
%    \begin{macrocode}
\def\reformat@unicode@string##1{%
   \textU{U+}%
  \let\z\empty%
  \expandafter\@tfor\expandafter\i\expandafter:\expandafter=#2;\do{%
  \if\i;%
    \textU{x}%
  \else%
    \textU{\z}%
  \fi%
  \edef\z{\i}%
 }%
}%
  \makebox[5em]{\reformat@unicode@string{#2}\hfill}%
}
%    \end{macrocode}
% 
% \begin{docCommand} {putchar@cx} {\meta{char}}
% \end{docCommand}
% 
%    \begin{macrocode}


\bgroup \catcode`\"=12 
\def\putchar@cx#1{%

\stepcounter{glyph@count}
%\let\oldactive@prefix\active@prefix
%\let\active@prefix\relax
   \iffontchar\font\n
     \char\the\n$_{\pgfmathparse{Hex(\the\r@cx)}\text{\pgfmathresult}}$%
      %
   \else
    {\arial\graybox}
   \fi
%\let\active@prefix\oldactive@prefix
 }
\global \let\putchar@cx\putchar@cx
\egroup
%    \end{macrocode}
%    
%  typesets one row of a unicode table
%    \begin{macrocode}    
\def\urow@cx#1{%
    \n=#1% 
    \r@cx=0%
    \expandafter\putunicode@label#1;%
    \loop%
        \ifnum\n<\numexpr#1+16\relax%
        \makebox[1.9em]{\expandafter\putchar@cx{#1}}%
        \advance\r@cx by1%  
        \ifnum\r@cx>16\r@cx=1\relax\else\fi
        \advance\n by1%
    \repeat
    \par
}

\def\typeseturows@cx#1{%
\@for\next:=#1\do{%
  \urow@cx\next\vskip3pt}%
}

\newcount\r@cx%
\newcount\n%
\newcommand\unicodetable[2]{%
\bgroup
  % added to ensure csquotes does not interfere
  %\catcode`\"=12 
  \par
  \leavevmode%
   \r@cx=0%
   {\hbox to 5em{\ignorespaces}}%
   \loop%
    \ifnum\r@cx<16\ignorespaces 
    \makebox[1.9em]{\pgfmathparse{Hex(\the\r@cx)}\pgfmathresult}%
    \advance\r@cx by\@ne%  
   \repeat
   \vskip3pt\par
   \@nameuse{#1}%
   \typeseturows@cx{#2}%
\egroup
}
%    \end{macrocode}
% \begin{docCommand} {unicodenumber} {\meta{string}}
% Typesets a string such as |x1020| in a typewriter font.
% \end{docCommand}
%    \begin{macrocode}    
\DeclareRobustCommand\unicodenumber[1]{{\ttfamily #1\xspace}}
%    \end{macrocode}
%    
%    \begin{macrocode}
\def\putdescription#1:{%
  {\parindent0pt 
  \begin{minipage}[t]{4cm}
  \bgroup\panunicode
  \hangindent20pt
  #1\par
  \egroup
  \end{minipage} 
  }
}
%    \end{macrocode}

% \begin{docCmd}{parsefields}{ \marg{unicode number, delimited with:}{rest of line}}
%   parses a line of text of the form \texttt{10900: Phoenician Letter Alf}
% \end{docCmd}
%    \begin{macrocode}
\long\def\parsefields #1:#2\@@{%
    \ifx\par#1
    \else 
        {\small\aegean U+#1}%
         %%\iffontchar\font"#1 %
          \makebox[2.1em]{\color{theunicodesymbolcolor}\symbol{"#1}}% 
          \expandafter\putdescription#2\vskip3pt
        %%\else
          %%{\aegean \makebox[2.1em]{} Unallocated\par}%
        %%\fi
    \fi  
  }%
% Check if it can be saved
\newread\tempstream%s
%    \end{macrocode}
%
% \begin{docCommand}{printunicodeblock}{ \oarg{no columns} \marg{filename} \marg{fontcmd}}
%  The macro prints a unicode table from a file of definitions. This is
%   printed in a two column environment by default. ^^A\Fire fails on Carian
%
%  The lines have the form of:
%
%  \texttt{10900: Phoenician Letter Alf}
% \end{docCommand}
% 
%    \begin{macrocode}
%\ExplSyntaxOn
\DeclareDocumentCommand{\printunicodeblock}{O{2} m m }
  {
    \bgroup
    \leavevmode\parindent0pt\par
    \begin{multicols}{#1}%
     #3
      \openin\@inputcheck=#2
      \loop\unless\ifeof\@inputcheck
      \read\@inputcheck to\fileline %
      %\fileline
      \expandafter\parsefields \fileline:\@@ 
      \repeat
    \end{multicols}%
      \immediate\closein\@inputcheck
      \egroup
  }
\let\PrintUnicodeBlock\printunicodeblock
%\ExplSyntaxOff
%    \end{macrocode}
% 
%    \begin{macrocode}
\let\indicative\pan
\newfontfamily\brahmi{Noto Sans Brahmi}
% ^^A \newfontfamily\bengal[Script=Bengali,Scale=1]{Shonar Bangla}
%    \end{macrocode}
%</SCRIPTS>
\endinput