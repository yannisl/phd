\documentclass[a4paper]{article}
\usepackage{expl3}
\usepackage{etoolbox}

\begin{document}

\ExplSyntaxOn

% Make some commands engine-robust
\robustify\textbf

% Set up to exclude items from case changing
\tl_new:N \l_tl_case_change_exclude_tl

\tl_put_right:Nn \l_tl_case_change_exclude_tl { \NoChangeCase }
\cs_new:Npn \NoChangeCase #1 {#1}

% First split the input string into sentence chunks, where `sentence' means anything after . ? ! : etc.
 
\cs_new:Npn \titlecase #1
  {
    \tl_if_blank:nF {#1}
      { \titlecase_split_period:w #1 . \q_recursion_tail \q_recursion_stop }
  }

%% This is the basic idea of splitting:
%
%\cs_new:Npn \titlecase_split_period:w #1 . #2
%  {
%    \titlecase_split_colon:w #1 : \q_recursion_tail \q_recursion_stop
%    \quark_if_recursion_tail_stop:n {#2}
%    . \c_space_tl
%    \titlecase_split_period:w #2
%  }

% Use an obfuscated macro to avoid lengthy definitions:
\cs_new:Nn \titlecase_new_split_fn:NNNN
  {
    \cs_set:Npn #2 ##1 #1 ##2
      {
        #4 ##1 #3 \q_recursion_tail \q_recursion_stop
        \quark_if_recursion_tail_stop:n {##2}
        #1 \c_space_tl
        #2 ##2
      }

  }

% The reason to use an obfuscated macro is to pass in properly sanitised `characters':
% (Could possibly be done more easily with some clever expansion.)
\exp_after:wN \titlecase_new_split_fn:NNNN
\exp_after:wN .
\exp_after:wN \titlecase_split_period:w \token_to_str:N :
  \titlecase_split_colon:w
\exp_after:wN \titlecase_new_split_fn:NNNN
  \token_to_str:N : \titlecase_split_colon:w ! \titlecase_split_exclam:w
\titlecase_new_split_fn:NNNN ! \titlecase_split_exclam:w  ? \titlecase_split_question:w
\titlecase_new_split_fn:NNNN ? \titlecase_split_question:w  {~} \titlecase_first:w


%% These are the internals that do the processing of each word.

% The first and last words needs special attention;
% they should always be capitalised unless there's a special exception.
\cs_new:Npn \titlecase_firstlast:nn #1 #2
  {
    \titlecase_exact_exceptions:nn {#1} {#2}
    \tl_mixed_case:n {#1} #2
  }

\cs_new:Npn \titlecase_middle:nn #1 #2
  {
    \titlecase_exact_exceptions:nn {#1} {#2} % catch exact matches which will never change
    \titlecase_lc_exceptions:nn {#1} {#2} % catch words that should be lowercase
    \tl_mixed_case:n {#1} #2 % otherwise capitalise normally
  }


%% After all the splitting, we now enter the word-by-word titlecasing.

\cs_new:Npn \titlecase_first:w #1 ~ #2
  {
    \tl_if_blank:nT {#1} { \quark_if_recursion_tail_stop:n {#2} }
    \titlecase_catch_punct:Nn \titlecase_firstlast:nn {#1}
    \use_none_delimit_by_q_nil:w
    \q_nil
    \quark_if_recursion_tail_stop:n {#2}
    \c_space_tl
    \titlecase:w #2
  }

% The regular loop:
\cs_new:Npn \titlecase:w #1 ~ #2
  {
    \quark_if_recursion_tail_stop_do:nn {#2}
      {
        \titlecase_catch_punct:Nn \titlecase_firstlast:nn {#1}
        \use_none_delimit_by_q_nil:w
        \q_nil
      }

    \titlecase_catch_punct:Nn \titlecase_middle:nn {#1}
    \use_none_delimit_by_q_nil:w
    \q_nil
    \c_space_tl
    \titlecase:w #2
  }

\cs_new:Npn \__tl_last:n #1
  {
    \__tl_last:w #1 \q_recursion_tail \q_recursion_tail \q_recursion_stop
  }
\cs_new:Npn \__tl_last:w #1 #2
  {
    \quark_if_recursion_tail_stop_do:nn {#2} {#1}
    \__tl_last:w #2
  }

\cs_new:Npn \titlecase_catch_punct:Nn #1 #2
  {
    \str_if_eq_x:nnT { \__tl_last:n {#2} } {,}
      {
        \exp_args:No #1 { \titlecase_catch_comma:w #2 }{,}
        \use_none_delimit_by_q_nil:w
      }
    \str_if_eq_x:nnT { \__tl_last:n {#2} } {;}
      {
        \exp_args:No #1 { \titlecase_catch_semicolon:w #2 }{;}
        \use_none_delimit_by_q_nil:w
      }
    #1 {#2}
  }

\typeout{ LAST:~\__tl_last:n {foo bar .}}
\cs_new:Npn \titlecase_catch_comma:w #1, {#1}
\cs_new:Npn \titlecase_catch_semicolon:w #1; {#1}


% The exception catching functions contain series of functions like the following to match and if so jump to the end after their specific processing.

\cs_new:Npn \titlecase_exception:nnn #1 #2 #3
  {
    \str_if_eq_x:nnT { \tl_lower_case:n {#1} } {#3}
      { \tl_lower_case:n {#1} #2 \use_none_delimit_by_q_nil:w }
  }

\cs_new:Npn \titlecase_exact_exception:nnn #1 #2 #3
  {
    \str_if_eq:nnT {#1} {#3}
      { #1 #2 \use_none_delimit_by_q_nil:w }
  }


% These are the functions to define the exception catching functions.

\cs_new:Npn \titlecase_set_small_exceptions:n #1
  {
    \seq_set_from_clist:Nn \l_titlecase_seq {#1}
    \cs_set:Npx \titlecase_lc_exceptions:nn ##1 ##2
      {
        \seq_map_function:NN \l_titlecase_seq \titlecase_small_except:n
      }
  }

\cs_new:Npn \titlecase_set_exact_exceptions:n #1
  {
    \seq_set_from_clist:Nn \l_titlecase_seq {#1}
    \cs_set:Npx \titlecase_exact_exceptions:nn ##1 ##2
      {
        \seq_map_function:NN \l_titlecase_seq \titlecase_exact_except:n
      }
  }

\cs_new:Npn \titlecase_small_except:n #1
  {
    \exp_not:N \titlecase_exception:nnn {\exp_not:N ## 1} {\exp_not:N ## 2} {#1}
  }

\cs_new:Npn \titlecase_exact_except:n #1
  {
    \exp_not:N \titlecase_exact_exception:nnn {\exp_not:N ## 1} {\exp_not:N ## 2} {#1}
  }

\titlecase_set_small_exceptions:n {a,an,and,as,at,but,by,en,for,if,in,of,on,or,the,to,v,via,vs}
\titlecase_set_exact_exceptions:n {TV,iTunes,iPhone,AD,A.D.,ad}% an example only

\ExplSyntaxOff

\def\TCTEST#1{{\titlecase{#1}}\par\bigskip}
%\def\TCTEST#1{\typeout{\titlecase{#1}}}

\TCTEST{Modern words like \NoChangeCase{\textbf{iPhone}} and iTunes, \NoChangeCase{eyeTV}, and even just TV are annoying}
\TCTEST{This is a sentence. But this is capitalised.}
\TCTEST{iTunes, here to stay like iPhone}
\TCTEST{Hello friends of the world, whom I am also friends of}
\TCTEST{}
\TCTEST{ }
\TCTEST{and}
\TCTEST{and and}
\TCTEST{and and and}
\TCTEST{This Was All Capitalised. but Some Words Shouldn't Be If They're Small!}
\TCTEST{Colon test: An example of an exception}
\TCTEST{A Question? An answer?}
\TCTEST{A Question! An answer!}
\TCTEST{iTunes and TV}
\TCTEST{David v Goliath and Goliath vs Mittelbach}
\TCTEST{Hard to say whether `quotes are 500 b.c. too hard'}
\TCTEST{Pyramids 4500--2000 b.c. }
\TCTEST{Anno Domini (AD or A.D?)}
\ExplSyntaxOn
\typeout{ [This~ is~ plain~ title_case~ with~ quotes]~ \tl_mixed_case:n {`Hello'} }

[This~ is~ plain~ title_case~ with~ quotes]~ \tl_mixed_case:n {`Hello'}

\par\bigskip [This~ is~ plain~ title_case~ with~ braces]~ \tl_mixed_case:n {{`Hello'}}

\par\bigskip [This~ is~ plain~ title_case~ with~ bold]~ \tl_mixed_case:n {\textbf{`Hello'}}

\par\bigskip [This~ is~ plain~ title_case~ with~ bold~ and~ braces]~ \tl_mixed_case:n {{\textbf{`Hello'}}}

\ExplSyntaxOff

\end{document}