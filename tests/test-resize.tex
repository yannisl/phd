\documentclass{article}

\usepackage{mwe,filecontents}
\usepackage{graphicx}
\usepackage{luacode}
\begin{document}

\resizebox{8mm}{!}{\includegraphics{example-image-16x10}}

\def\preamble{%
\ttfamily
\string\documentclass\{article\}\par
\string\usepackage\{etst\}\par
}
\def\startlua{\preamble\luacode }
\def\endlua{\par\string\end\endluacode}

\newenvironment{luaexample}{\startlua}{\endlua}
\begin{luaexample}
local a, b = 1
local c = a+2
tex.print("a+b")
\end{luaexample}

\end{document}
-- node of the graph. The |name| must be a unique string identifying
-- the node. The newly created vertex will be added to the syntactic
-- digraph. The binding function |everyVertexCreation| will then be
-- called, allowing the binding to store information regarding the newly
-- created vertex.