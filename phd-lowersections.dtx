% \iffalse meta-comment
%<*internal>
\iffalse
%</internal>
%<*readme>
----------------------------------------------------------------
phd-pkgmanager --- a package to shorten preambles
E-mail: yannislaz@gmail.com
Released under the LaTeX Project Public License v1.3c or later
See http://www.latex-project.org/lppl.txt
----------------------------------------------------------------
This file provides a phd for defining a class.
%</readme>
%<*readmemd>
###The `phd` LaTeX2e package

The `phd` latex package and the class with the same name provide
convenient methods to create new styles for books, reports
and articles. It also loads the most commonly used packages 
and resolves conflicts.

This work consists of the file  `phd.dtx`,
and the derived files   `phd.ins`,  `phd.pdf`, and `phd.sty`.

###Installation

run
          phd-lua.bat on windows
           pdflatex phd.dtx
           makeindex -s gind.ist -g phd 

If you have any difficulties with the package come and join us at
http://tex.stackexchange.com and post a new question or
add a comment at http://tex.stackexchange.com/a/45023/963.
or send me a message at  yannislaz at gmail.com

### Documentation

The package was written using the `doc` and `docscript` packages,
so that it is self documented in a literary programming style. 
The .pdf is a fat document, providing over fifty book styles (the
equivalent of classes) plus there is a lot of write-up on the inner
workings of TeX and LaTeX2e. However, you don't need to know much
to use it.

      \usepackage{phd}
      %%%%%%%%%%%%%%%%%%%%%%%%%%%%%%%%%%%%%%%%%%%
%%%%%%  STYLE 13
%%%%%%%%%%%%%%%%%%%%%%%%%%%%%%%%%%%%%%%%%%%

\cxset{style13/.style={
 name={Chapter},
 numbering=arabic,
 number font-size=\HUGE,
 number font-family=\sffamily,
 number font-weight=\bfseries,
 number color=\color{gray!50},
 number before=\par\vspace*{5pt}\hfill\hfill,
 number dot=,
 number after={\hspace*{7pt}\par},
 number position=rightname,
 chapter font-family=\sffamily,
 chapter font-weight=\normalfont,
 chapter font-size=\LARGE,
 chapter before={\thickrule\vspace*{20pt}\par\hfill\hfill},
 chapter after={\vskip0pt\par},
 chapter color={black!50},
 title beforeskip={\vspace*{10pt}},
 title afterskip={\vspace*{50pt}\par},
 title before={\hfill\hfill\raggedleft},
 title after={},
 title font-family=\sffamily,
 title font-color=\color{thered},
 title font-weight=\bfseries,
 title font-size=\huge,
 section indent=-1em,
 section align=\raggedright,
 section numbering=arabic,
 section indent=0pt,
 section beforeskip=0pt,
 section afterskip=\baselineskip,
 subsection align=\raggedright,
 subsection font-family=\sffamily,
 subsection font-weight=\bfseries,
 subsection font-size=\large,
 subsection font-shape=\itshape,
 subparagraph number after=\space,
}
}

\def\setstyle#1{\cxset{style#1}%
 \renewsection\renewsubsection\renewsubsubsection%
 \renewparagraph\renewsubparagraph}

\setstyle{13}


\chapter{Introduction to Chapter\\ Style Thirteen}

\section{A Brief History of Biomedical\\ Fluid Mechanics}
\lorem
\medskip
\begin{figure}[ht]
\centering
\includegraphics[width=0.45\textwidth]{./chapters/chapter14}
\includegraphics[width=0.45\textwidth]{./chapters/chapter14a}
\end{figure}
\lorem


All choices, are made via an extended key-value interface. 
Although not a compliment, it resembles CSS and the keys are a bit verbose but
attributes are easy to change and have a consistent and easy to remember interface.

To set or add a key we only use one command:

      \cxset{chapter name font-size: Huge,
             chapter number font-size: HUGE} 

### Future Development

This is still an experimental version, but I will retain the
interface in future releases. There is a large amount of
work still to be carried out to improve the template styles
provided, to test it more thoroughly and to add a number of
improvements in the special designs. At present I estimate
that I have completed about 70% of the work that needs
to be done.

__The package as it stands is not production stable.__ 


%</readmemd>
%
%<*TODO>
1. On final round add pkg options. This was left as last in order not to solve problems by adding
    options. Too many options are not a good User Interface.
2.  Finish symbol management, both text and math. Math already 60% incorporated.
3.  Better integration of indexing commands.   
4.  Revisit layout manager for Chapters. Broke again in tests.
5.  Docs. Add all references.
6.  Incorporate phd class for more flexibility.
7. Improve package manager.
8. Group script loading for better font management.
9. General font management to relook it again.
10. Add all style sections (about 100 already prepared). Once they
     are all working issue beta version.
%</TODO>
%<*internal>
\fi
\def\nameofplainTeX{plain}
\ifx\fmtname\nameofplainTeX\else
  \expandafter\begingroup
\fi
%</internal>
%<*install>
\input docstrip.tex
\keepsilent
\askforoverwritefalse
\preamble
----------------------------------------------------------------
phd --- A package to beautify documents.
E-mail: yannislaz@gmail.com
Released under the LaTeX Project Public License v1.3c or later
See http://www.latex-project.org/lppl.txt
----------------------------------------------------------------
\endpreamble

%\BaseDirectory{C:/users/admin/my documents/github/phd}
%\usedir{MWE}
\generate{\file{\jobname.sty}{
  \from{\jobname.dtx}{LSECT}}
  }

%\nopreamble\nopostamble

%</install>

%<install>\endbatchfile
%<*internal>
%\usedir{tex/latex/phd}
\generate{
  \file{\jobname.ins}{\from{\jobname.dtx}{install}}
}
\nopreamble\nopostamble

\generate{
	\file{README.txt}{\from{\jobname.dtx}{readme}}
  }

\generate{
  \file{README.md}{\from{\jobname.dtx}{readmemd}}
}
\generate{
  \file{TODO.tex}{\from{\jobname.dtx}{TODO}}
}

\ifx\fmtname\nameofplainTeX
  \expandafter\endbatchfile
\else
  \expandafter\endgroup
\fi
%</internal>
%<*driver>

%\listfiles
%gdef\@onlypreamble{} % TO BE REMOVED NEEDED FOR TUTS
\documentclass[oneside,11pt,a4paper]{ltxdoc}
\usepackage[bottom=2cm]{geometry}
\savegeometry{std}
% \usepackage[style=mla]{biblatex}
\usepackage{phd}
\usepackage{phd-lowersections}
%\usepackage{pkgindoc}             %%% danger
\sethyperref

 
\begin{document}
\coverpage{asia}{Book Design }{Camel Press}{HEADINGS}{DESIGN} 
\secondpage
\frontmatter
\clearpage
\tableofcontents
\setcounter{secnumdepth}{8}
\parskip0pt
\mainmatter
% \begin{epigraphpage}
 \epigraph{Begin at the beginning,'' the King said, gravely, ``Then
 go till you come to the end; then stop.''}{Lewis Carroll, {\it Alice
 in Wonderland}}

 \epigraph{You can never get a cup of tea large enough or a book long enough to
 suit me''}{C. S. Lewis}
 \end{epigraphpage}

\parindent1em
%\cxset{style13}
%\cxset{title margin bottom=10pt,
%          title beforeskip=1pt}

\chapter{Introduction}
\addtocimage{-12pt}{-20pt}{../images/tocblock-fish}


\epigraph{``Begin at the beginning,'' the king said
"and then go on till you come to the end, then stop."}{
---Lewis Carroll, Alice in Wonderland}

 \parskip3pt plus 5pt 
\noindent This package and its documentation attempts to eliminate some common 
problems encountered when using \LaTeX2e. The first one is the loading of 
recommended packages for a large and perhaps complicated document and 
the second is the re-designing of styles for a document.

 \LaTeX2e, does not provide a standard library, but comes equipped with
 a package mechanism that allows code extensions to be loaded as required.
 This has created a strong vibrant community, hundreds of packages and a 
 headache to both new and seasoned users. What packages are available, when
 to use them and in which order is a common theme for many questions on
 lists and |TX.SE|.

 It is quite common during the writing of a thesis or book
 for the author to keep on adding macros and packages
 at the preamble of the document. In most cases this can
 be satisfactory but in many others it leads to
 incompatibilities and errors. This package aims at
 minimizing one's preamble, by prefetching a number of
 commonly used packages. It also aims at loading them
 in the right order and providing patches for conflicts.
 
 I am hoping that using this package, will lead to less
 frustrations with the intricacies of \LaTeX2e\ packages.

The package code is complicated, but its usage is simple. You first load the package and then
you use one of the available templates:

 \begin{commands}[]{}
 \begin{verbatim}
 \usepackage{phd}
 \usetemplate{style13}
 \end{verbatim}
 \end{commands}

This is what you need to typeset a good looking book or thesis. The rest of this book is a footnote and you can skip them if you want. 

It will be better for the longer projects to just fork the
 package and adapt it to your needs. In this respect, I have
 uploaded the package to |github|.\footnote{\url{https://github.com/yannisl/phd}}

 My goal in selecting the packages and adding a number of 
 commands for the authors was to be able to typeset a 
 document for most common use cases, without the need of
 additional packages. The packages I selected are biased
 towards academic publications, although they can find use
 in almost any fields. The package provides a mechanism via
 PGF keys to provide a settings file. 
 
 Most of the documentation can be found in the implementation part.

Browse any books in a library or bookshop and the striking thing is that their design is very individualistic. They might have similarities but their main features vary. In many respects they resemble people's faces where minor differences have striking effects.

This package arose out of a question at stackexchange. How to redefine chapter heads. Having seen the popularity of the |pgf| package \cite{pkg-pgf} I realized that \latex users prefer this method of styling rather the traditional \latex method.

The user interface can be extended to basically all major packages. The principle is to keep to a minimum changes that can affect the LaTeX core commands. If there are any additions a key setting is provided to be able to revert back to normal LaTeX.

The workflow can be simplified. In addition I want to believe that the interface can provide a useful addition to the open source community and that other people will contribute style libraries, which will be simpler to write. It is also possible
to device an easy and uncomplicated web interface to handle
such a great number of variables.


Most people when they get started with \LaTeX\ will either use one of the standard classes such as the \docFile{book.cls} or one of the generic classes notably koma-script or memoir. Most students will be forced to use on of the many thesis classes available.

\section{The key value concept}

The key-value concept that originated with \LaTeX\ has been extended many times, the last and most serious implementation of it by Tantau in the PGF package. What essentially Tantau developed is a scripting language to script TeX code. The \tikzname and pgfplots packages are two major packaged that use keys effectively. Their popularity is growing and what this package does is to offer a user interface that has been modelled to be similar to that of \texttt{css} (cascade style sheets). 
\smallskip

\begin{scriptexample}{}{}
\textit{chapter number} font-size = Large,\\
\textit{chapter number}     color = theblue
\end{scriptexample}
\smallskip

The main idea behind the package, is that you are configuring a document style by means of \emph{settings} rather than writing macros. In the example above the \emph{number, chapter} can be thought of as class or id names in css style sheets and the |font-size, color| as property settings that apply to the particular element. 


\subsection{Settings}

Settings are activated either by using the command |\cxset|  or by loading a full style sheet. In most cases you will probably import a style sheet and then modify some of the properties using |cxset|.  For example this heading has a dot after the subsection number. This was accomplished by setting,

We can de-activate it for the next and subsequent subsection headings with the setting:

\lorem

\begin{scriptexample}{}{}
\begin{verbatim}
\cxset{subsection number after=\quad}
\end{verbatim}
\end{scriptexample}




\subsection{Cascading}

Most values once set for a higher section will be seen in a cascade by all subsectioning commands in a similar fashion similar to CSS. These include properties such as color, font families and alignment. Best though to specify all of them for maximum flexibility to your users.

\section{On typography}

This package hopefully will assist in improving the typography of books set with \latexe. Any typographical comments on the various styles are just my own ramblingss and not necessarily absolute truths. Like fashion and art typography has opinions rather than absolute truths. In many styles the design is slightly adapted to blend a bit better with this manual. Also I did not select fonts as per the samples but this is left on you the user to decide.



\section{Packages and Fonts}

This manual has been typeset with numerous fonts in order to enable the typsetting of almost all the scripts provided by the Unicode standard. In order to process it from the |.dtx| file, these fonts must be available in your system, otherwise \XeLaTeX\ will have a problem finding the fonts and it will take an awful long time to process. This is especially true for the scripts section, where virtually all the Unicode defined scripts are discussed. You will need a fast computer and a fast hard disk to process the document within a reasonable time. When using \pkg{fontspec} always define your fonts with the \cmd{\newfontfamily} this will speed up processing by an order of magnitude. Compiling from the command prompt will speed up compilation. Average speed 2-3 pages per second.

Many of \tex's parameters are stretched to the limit with a complicated document such as this manual. You will require a full distribution otherwise expect some errors. Important packages is \pkg{morefloats} and \pkg{morewrites}. The package will also expect that you have |e-tex| installed. Ubuntu users are normally one year behind in updates, so you might wish to update manually. It will take upwards of 5 minutes to compile fully on an old laptop and a couple of minutes on a state of the art computer.

The |dtx| should be processed best with its own make file provided for Windows only |phd-lua.bat|. The make file will process the documentation using \lualatex. You can also process the document with \xelatex but is prone to produce errors. Using \latexe the sections on scripts etc will not be printed and a much shorter version of the manual is provided. 

\section{Scripts and Languages}

The package and the documentation offer a full repertoire of font selection keys for different scripts and languages. It hasn't been possible, however hard I tried to compile this section of the documentation with \xelatex, as it kept giving errors of too many files open. This was also not possible even with the \pkg{morewrites} package loaded. With \lualatex the document compiled with no major problems other than the font rendering being of a lower quality to that of XeLaTeX on windows, other than disabling incompatible packages and a number of commands that were redefined. 

Some good news for multi-script typesetting is the |Noto| fonts from Google. These fonts named Noto from "No Tofu" meaning you do not see any little square blocks for undefined glyphs, are fast to load. Disantvantage you need to switch between font commands fairly often.

\section{This book}

When developing the templates, I started using \emph{lorem ipsum} text as samples. Half-way through this
became a jumble mass of uninteresting pages interspersed with code. Headings and the contents of the book
determine both the structure and the selection of fonts, so I went back and wrote narratives  to accompany
the headings. Many of the narratives are semi-autobiographical in nature; others are clustered around books I read and my own interests. Some I stumbled on them accidentally and are mostly there to demonstrate some code.

Besides the templates and the code there is another narrative which is based on notes I kept on \tex and its friends over the years and are offered as a more advanced introduction to coding \latexe and \tex. The whole manual was typeset in a |ltxdoc| class, slightly modified to turn into a book class.

The implementation code is also available and it was mostly for my own benefit. The whole manual with the exception of the |\cxset| introduction, is just a test document. The notes and the “dissection” of the standard \latexe and the standard classes are there to explain the background to the many coding decisions that I took while I was developing the package.

PhD students are notorious for going in all directions and exploring many adjacent fields before they sit down and write their theses. Some become life-time students. To all these new men and women of the Renaissance that slave away to inch knowledge one thesis at a time, I dedicate this book and the name of the package.

\subsection{The TeX hacking sections}

To start programming \tex you need to have a knowldge of \tex basic commands and approach. \latex2015 is a format build on top of \tex to provide a more structured approach. To program \latexe packages you need to understand \latexe concepts, code organization and conventions. To program in \latex3, you need to learn a whole new language and you still need to understand \tex, \latexe and the expl3 language and conventions. To program using LuaTeX, other than the Lua language you need to understand \tex very well.
None of these can be found in one place.  I have gathered a lot of material and put it together. This is not a language you can master easily or quickly, but can teach you a lot about typesetting, computer science and many other interesting topics.


 \section{Version control with Git and Github}
 
 If you are involved with code or a publication that will have frequent changes, you should consider
 some type of version control system. My own recommendation is to use |git| and an online repository such
 as |github|. The latter is currently very fashionable and makes sharing code easier. Note that the |github|
 offers both public as well as private repositories. The general recommendation is that for unpublished work
 such as a thesis or code under development, it is preferable to go for a private repository. 
 
 \lorem\lorem

 \section{Ordering of Packages}
 
One package that normally leads to errors is the 
\pkg{hyperref}. The package which is an outstanding example of software engineering and supported single handledy by Heiko Oberdiek\footcite{hyperref} redefines a a lot of internal commands of the kernel. As a lot of other packages do the same it has to be loaded at the end of the preable with the exception of some packages! 
 
 This manual is typeset according to the conventions of the
 \LaTeX \textsc{docstrip} utility which enables the automatic
 extraction of the \LaTeX{} macro source files~\cite{GOOSSENS94}.

 
 \href{http://tex.stackexchange.com/questions/96350/problem-with-algorithmic-and-hyperref}{problem with algorithmic and hyperref}

 \begin{verbatim}
\usepackage{float}  % load float package first!

\usepackage{hyperref} % let hyperref patch the float package stuff
.
 \usepackage{algorithm} % let algorithm use the patched version of the float package
 \end{verbatim}
 

\section{Known problems}

Perhaps the biggest issue with the package is the speed of
compilation with \XeLaTeX\ or \LuaTeX. This is to be expected, as both engines spend a lot of resources in font management. On demand loading of packages is something I have in the back of my mind. This should be done via document styles i.e., if a book is for the humanities, perhaps only a rudimentary amount of maths packages should be loaded.

\section{Future Directions}

\latexe and \tex usage appears to be increasing. This is mostly by programs that export results with \latexe code rather than authors writing books.  The method adopted here is easier to automate all sorts of reports and automated texts. I would like too develop a web interface for processing such templates and at the same time export into html instead of just producing pdfs. I have already a prototype.   

\section{Tooling}

Some of the scripts on a Windows machine need MSYS\footnote{\url{http://mingw.org/wiki/MSYS}}









%\makeatletter\@specialfalse\makeatother
%%%%%%%%%%%%%%%%%%%%%%%%%%%%%%%%%%%%%%%%%%%
%%%%%%  STYLE 01
%%%%%%%%%%%%%%%%%%%%%%%%%%%%%%%%%%%%%%%%%%%


\cxset{
 name={},
 numbering=arabic,
 number font-size=\LARGE,
 number font-family=\rmfamily,
 number font-weight=\bfseries,
 number before=,
 number dot=,
 number after=,
 number position=leftname,
 chapter font-family=\sffamily,
 chapter font-weight=\normalfont,
 chapter font-size=\Large,
 chapter before={\vspace*{20pt}\par},
 chapter after={\hfill\hfill\par},
 chapter color={black!90},
 number color=\color{purple},
 title beforeskip={\vspace*{30pt}},
 title afterskip={\vspace*{40pt}\par},
 title before={},
 title after={},
 title font-family=\sffamily,
 title font-color=\color{purple},
 title font-weight=\bfseries,
 title font-size=\LARGE,
 header style=headings}

\cxset{headings ruled-01}

\chapter{Introduction to Style One}


\begin{summary}
This design is simple and its distinguishing characteristic is a short summary at the beginning of the chapter. This is almost like an abstract typeset in italic font without setting the margins in. We provide a \lstinline{summary} environment for convenience. Note the very simple line in the running head to the left of the page number.
\end{summary}

\medskip
\begin{figure}[ht]
\centering
\includegraphics[width=0.5\textwidth]{./chapters/chapter01}
\end{figure}


%
\@specialtrue
\cxset{steward,
  numbering=arabic,
  custom=stewart,
  offsety=0cm,
  image=hine03,
  texti={When Lamport designed the original \LaTeX\ sectioning commands, limitations of computer power forced him to restrict the abstraction of complicated chapter layouts. With current tools available improvements are much easier to program.},
%
  textii={In this chapter we discuss a method that allows the production of fancy sectionr headings and formatting, based on a set of key values. Central  to this process is the separation of content from presentation.
We also discuss the basic formatting tools that are available and how one can modify them to mould new book designs.
 }
}



\raggedbottom

\chapter{Lower Level Headings}
\@specialfalse

\section{Introduction}

Good book design dictates that sectioning styles follow that of the general book design and theme. An academic publication for example might have chapters and section numbered in arabic numerals, whereas a high school textbook might have sections marked in colored boxes.

Similarly to the chapter key value interface, the package offers a key value interface to adjust sectioning command parameters.



\cxset{section beforeskip={10pt},
      section indent=0pt}
\cxset{section afterskip={10pt}}
\renewsection

\section{Section styling}

In a similar fashion to the chapter commands the following keys are provided.

\subsection{Fonts and numerals}

Font and numeral keys are shown below.
\medskip

  \keyval{section font-size}{\marg{cmd}}{Font size command such as \cs{large.}}
  \keyval{section font-weight}{\marg{cmd}}{Font weight command such as \cs{bfseries.}}
  \keyval{section font-family}{\marg{cmd}}{Font family command such as \cs{sffamily.}}
  \keyval{section font-shape}{\marg{cmd}}{Font shape command such as \cs{itshape}}
  \keyval{section color}{\marg{color}}{Color of section.}
  \keyval{section numbering}{\marg{arabic|roman|Roman|alph|Alph|words|WORDS}}{Section number style.}
  \begin{marglist}
  \item [arabic] Typesers the section number in arabic numerals.
  \item [roman] Typesets the section number in lowercase roman numerals.
  \item [Roman] Typesets the section number in uppercase roman numerals.
  \item [alph] Typesets the section number in lowercase alphabetic numbering.
  \item [Alph] Typesets the section number in uppercase alphabetic numerals.
  \item [words] Typesets the numbers in words (lowercase).
  \item [WORDS] Typesets the number in words (uppercase).
  \end{marglist}

\subsection{Skip and indentation commands}

The keys for indentaion and above and below skips are shown below.
\medskip

\keyval{section beforeskip}{}{}
\keyval{section afterskip}{}{}
\keyval{section indent}{\marg{dim}}{Indentation from margin as per standard LaTeX class definitions.}
\keyval{section spaceout}{}{}
\begin{marglist}
 \item[soul]
 \item[none]
\end{marglist}

\subsection{align}

\keyval{section align}{\marg{cmd}}{One of the alignment commands centering, ragged right, raggedleft}

\subsection{Hooks}

Hooks for adding material are shown in the following sketch.
\medskip

\fbox{aboveskip}

\fbox{indent} \fbox{number}\fbox{hook}\fbox{title}

\fbox{belowskip}

%\lipsum

\section{Example usage}

\cxset{
 chapter toc=false,
 name=CHAPTER,
 numbering=arabic,
 number font-size=\huge,
 number font-family=\sffamily,
 number font-weight=\bfseries,
 number before=,
 number dot=,
 number after=\hspace{1em},
 number position=rightname,
 chapter opening=anywhere,
 chapter font-family=\sffamily,
 chapter font-weight=\bfseries,
 chapter font-size=\huge,
 chapter before={\vspace*{0.1\textheight}\hfill},
 chapter after={\hfill\hfill\vskip0pt\thinrule\par},
 chapter color={black!90},
 number color=\color{black!90},
 title beforeskip={\vspace*{30pt}},
 title afterskip={\vspace*{30pt}\par},
 title before={\hfill},
 title after={\hfill\hfill},
 title font-family=\sffamily,
 title font-color=\color{black!90},
 title font-weight=\bfseries,
 title font-size=\huge,
%%%%%%%%%% Sections
 section font-size=\LARGE,
 section font-weight=\normalfont,
 section font-family=\sffamily,
 section align=\centering,
 section numbering=arabic,
 section indent=0em,
 section align=\centering,
 section beforeskip=20pt,
 section afterskip=10pt,
 section spaceout=soul,
 section font-shape=\itshape,
}
\cxset{book/.style={
 section numbering=arabic,
 section font-size=\Large,
 section font-weight=\bfseries,
 section font-family=\rmfamily,
 section font-shape=\normalfont,
 section align=\raggedright,
 %section numbering custom=\color{gray}{Section} (\thechapter-\@arabic\c@section),
 subsection font-size=\large
 section indent=0em,
 section beforeskip=-3.5ex \@plus -1ex\@minus -0.2ex,
 section afterskip=2.3ex\@plus.2ex,
 subsection beforeskip=-3.5ex \@plus -1ex\@minus -0.2ex,
 subsection afterskip= 1.5ex \@plus .2ex,
}}


\begin{example}{Adjusting section parameters}{}
\cxset{ section font-size=\LARGE,
 section font-weight=\normalfont,
 section font-family=\sffamily,
 section align=\centering,
 section numbering=(roman),
 section indent=0em,
 section align=\centering,
 section beforeskip=20pt,
 section afterskip=10pt,}
\chapter{A First Look at the Sectioning Keys}
\section{First section}
\lorem
\end{example}

One notable thing to keep in mind is that the numbering of the chapter is independent of that for the section, so if you need to have strange combinations rather define a section numbering custom.\index{section formatting!vertical space}

\cxset{section numbering=arabic}
\subsection{Adjusting vertical spaces}

Perhaps the most important issues we need to consider is the adjusting of vertical spaces; example~\ref{ex:latex}, that follows illustrates settings from the Octavo class and compare them with those of standard the \LaTeXe\ book class. The Octavo class through settings that are based on baselineskip fractions and multiples endeavours to achieve a grid layout. The class also tones down the `loudness' of some of the headings compared to those of the book class.


\cxset{octavo/.style={
 section font-size=\large,
 section font-weight=\normalfont,
 section font-family=\rmfamily,
 section font-shape=\scshape,
 section indent=0em,
 section align=\centering,
 section beforeskip=-1.666\baselineskip\@minus -2\p@,
 section afterskip=0.835\baselineskip \@minus 2\p@,
 subsection numbering=none,
 subsection font-family=\rmfamily,
 subsection font-size=\normalfont,
 subsection font-shape=\scshape,
 subsection font-weight=\normalfont,
 subsection indent=1em,
 subsection align=\raggedright,
 subsection beforeskip=-0.666\baselineskip\@minus -2\p@,
 subsection afterskip=0.333\baselineskip \@minus 2\p@
 }}




\cxset{book/.style={
 section numbering=arabic,
 section font-size=\Large,
 section font-weight=\bfseries,
 section font-family=\rmfamily,
 section font-shape=\normalfont,
 section align=\raggedright,
 %section numbering custom=\color{gray}{Section} (\thechapter-\@arabic\c@section),
 subsection font-size=\large,
 section indent=0em,
 section beforeskip=-3.5ex \@plus -1ex\@minus -0.2ex,
 section afterskip=2.3ex\@plus.2ex,
 subsection font-size=\large,
 subsection font-weight=\bfseries,
 subsection numbering=arabic,
 subsection indent=0pt,
 subsection beforeskip=-3.5ex \@plus -1ex\@minus -0.2ex,
 subsection afterskip= 1.5ex \@plus .2ex,
}}

\cxset{octavo headings/.style={%
 section numbering=none,section font-size=\large,section font-weight=\normalfont,
 section font-family=\rmfamily, section font-shape=\scshape,
 section indent=0em, section align=\centering, section beforeskip=-1.666\baselineskip\@minus -2\p@,
 section afterskip=0.835\baselineskip \@minus 2\p@, subsection numbering=none,
 subsection font-family=\rmfamily, subsection font-size=\normalfont, subsection font-shape=\scshape,
 subsection font-weight=\normalfont, subsection indent=1em, subsection align=\raggedright,
 subsection beforeskip=-0.666\baselineskip\@minus -2\p@,
 subsection afterskip=0.333\baselineskip \@minus 2\p@,
 subsubsection numbering=none,
 subsubsection font-family=\rmfamily,
 subsubsection font-size=\normalfont,
 subsubsection font-shape=\itshape,
 subsubsection font-weight=\normalfont,
 subsubsection indent=1em,
 subsubsection align=\raggedright,
 subsubsection beforeskip=-0.666\baselineskip\@minus -2\p@,
 subsubsection afterskip=0.333\baselineskip \@minus 2\p@,
 paragraph numbering=none,
 paragraph font-family=\rmfamily,
 paragraph font-size=\normalfont,
 paragraph font-shape=\normalfont,
 paragraph font-weight=\normalfont,
 paragraph indent=-1em,
 paragraph align=\raggedright,
 paragraph beforeskip=\z@,
 paragraph afterskip=0\p@,
% subparagraph numbering=none,
% subparagraph font-family=\rmfamily,
% subparagraph font-size=\normalfont,
% subparagraph font-shape=\normalfont,
% subparagraph font-weight=\normalfont,
% subparagraph indent=0em,
% subparagraph align=\raggedright,
% subparagraph beforeskip=\z@,
% subparagraph afterskip=0\p@,
}}
\cxset{octavo headings}
\renewsection\renewsubsection\renewsubsubsection\renewparagraph

\begin{example}{Octavo class headings, settings}{}
\cxset{octavo headings/.style={%
 section numbering=none,section font-size=\large,section font-weight=\normalfont,
 section font-family=\rmfamily, section font-shape=\scshape,
 section indent=0em, section align=\centering, section beforeskip=-1.666\baselineskip\@minus -2\p@,
 section afterskip=0.835\baselineskip \@minus 2\p@, subsection numbering=none,
 subsection font-family=\rmfamily, subsection font-size=\normalfont, subsection font-shape=\scshape,
 subsection font-weight=\normalfont, subsection indent=1em, subsection align=\raggedright,
 subsection beforeskip=-0.666\baselineskip\@minus -2\p@,
 subsection afterskip=0.333\baselineskip \@minus 2\p@,
 subsubsection numbering=none,
 subsubsection font-family=\rmfamily,
 subsubsection font-size=\normalfont,
 subsubsection font-shape=\itshape,
 subsubsection font-weight=\normalfont,
 subsubsection indent=1em,
 subsubsection align=\raggedright,
 subsubsection beforeskip=-0.666\baselineskip\@minus -2\p@,
 subsubsection afterskip=0.333\baselineskip \@minus 2\p@,
 paragraph numbering=none,
 paragraph font-family=\rmfamily,
 paragraph font-size=\normalfont,
 paragraph font-shape=\normalfont,
 paragraph font-weight=\normalfont,
 paragraph indent=-1em,
 paragraph align=\raggedright,
 paragraph beforeskip=\z@,
 paragraph afterskip=0\p@,}}

\cxset{octavo headings}
\renewsection\renewsubsection\renewsubsubsection\renewparagraph
\section{Octavo Class Heading}
\lorem
\subsection{Octavo subsection}
This is some text short text\par
\subsubsection{Octavo sub-subsection}
\lorem
\paragraph{paragraph heading} This is some short text.
\end{example}

\begin{example}{}{}
\cxset{octavo}
\section{Octavo Class Heading}
\lorem
\subsection{Octavo subsection}
\lorem
\subsubsection{Octavo sub-subsection}
\lorem
\paragraph{paragraph heading} This is some short text.
\lorem
\paragraph{paragraph heading} This is some short text.
\lorem
\end{example}



\begin{example}{\LaTeXe\ book class headings settings}{ex:latex}
\cxset{book/.style={
 section numbering=arabic,
 section font-size=\Large,
 section font-weight=\bfseries,
 section font-family=\rmfamily,
 section font-shape=\normalfont,
 section align=\raggedright,
 %section numbering custom=\color{gray}{Section} (\thechapter-\@arabic\c@section),
 subsection font-size=\large,
 section indent=0em,
 section beforeskip=-3.5ex \@plus -1ex\@minus -0.2ex,
 section afterskip=2.3ex\@plus.2ex,
 subsection font-size=\large,
 subsection font-shape=\normalfont,
 subsection font-weight=\bfseries,
 subsection numbering=arabic,
 subsection indent=0pt,
 subsection beforeskip=-3.5ex \@plus -1ex\@minus -0.2ex,
 subsection afterskip= 1.5ex \@plus .2ex,
}}
\cxset{book}
\renewsubsection
\section{LaTeX Book  Class Heading}
\lorem
\subsection{A subsection}
\lorem
\end{example}

\section{Grid example}

One problem sometimes is that the sectioning commands create problems with grid layouts. Example~\ref{ex:grid} shows example settings.

\begin{example}{Section styles from the grid package}{ex:grid}
\cxset{grid/.style={
 section numbering=arabic,
 section font-size=\normalsize,
 section font-weight=\bfseries\mathversion{bold},
 section font-family=\rmfamily,
 section font-shape=\normalfont\bfseries\mathversion{bold},
 section beforeskip=-.999\baselineskip,
 section afterskip=0.001\baselineskip,
 section align=\raggedright,
 %section numbering custom=\color{gray}{Section} (\thechapter-\@arabic\c@section),
 subsection font-size=\normalsize,
 section indent=0em,
% section beforeskip=-3.5ex \@plus -1ex\@minus -0.2ex,
 %section afterskip=2.3ex\@plus.2ex,
 subsection font-shape=,
 subsection font-weight=\bfseries\mathversion{bold},
 subsection numbering=arabic,
 subsection indent=0pt,
 subsection beforeskip=\baselineskip,
 subsection afterskip= -.35\baselineskip,
% subsub section
 subsubsection font-shape=\itshape,
 subsubsection font-weight=\bfseries\mathversion{bold},
 subsubsection numbering=numeric,
 subsubsection indent=0pt,
 subsubsection beforeskip=\baselineskip,
 subsubsection afterskip= -.35\baselineskip,
}}
\cxset{grid}
\renewsubsection
\begin{multicols}{2}
\section{Grid  Class Heading}
\lorem
\subsection{Grid  subsection.}
\lorem
\subsubsection{A subsection grid.}
\lorem
\subsubsection{Another subsection grid.}
\lorem
\end{multicols}
\end{example}



The key \option{\bfseries section numbering custom}=\marg{code} is quite powerfull and can be used to define any type of section number style. Just remember that the numbering so far depends on two counters, the c@chapter and c@section. What the section numbering does, it redefines the macro \cs{thesection} to the new definition provided as argument for the key.

Although the temptation to define a lot of key combinations one would rather define new styles as a more user friendly approach.

\cxset{section numbering=arabic, section align=\raggedright, section font-shape=\upshape, section font-family=\rmfamily}
\section{Handling Other Section Levels}

Other sectioning commands such as \cs{subsubsection}, \cs{paragraph} and \cs{subparagraph} have equivalent keys. Examples can be found in the chapters that follow for specific styles.

\section{Technical discussion}

The standard LaTeX classes, book report and article have sections showing dot leaders, whereas in the article class the sections are shown without the dotted lines, as the l@section macro is redefined for articles.

\index{macros!\textbackslash @seccntformat}

\subsection{Indexing of Lower Section Headings}
\LaTeXe\ offers two pathways in redefining section commands, the first one is @startsection and the second is \cs{@seccntformat} \index{sectioning macros}. It also uses the macro \cs{secdef} to create the starred and unstarred versions of the sectioning commands.

\begin{tcolorbox}{}
\begin{lstlisting}
% \begin{macro}{\l@section}
%    In the article document class the entry in the table of contents
%    for sections looks much like the chapter entries for the report
%    and book document classes.
%
%    First we make sure that if a pagebreak should occur, it occurs
%    \emph{before} this entry. Also a little whitespace is added and a
%    group begun to keep changes local.
% \changes{v1.0h}{1993/12/18}{Replaced -\cs{@secpenalty} by
%    \cs{@secpenalty}.  ASAJ.}
% \changes{v1.2i}{1994/04/28}{Don't print a toc line when the tocdepth
%    counter is less than 1.}
% \changes{v1.4a}{1998/10/12}{we should use \cs{@tocrmarg}; see PR/2881.}
%    \begin{macrocode}
%<*article>
\newcommand*\l@section[2]{%
  \ifnum \c@tocdepth >\z@
    \addpenalty\@secpenalty
    \addvspace{1.0em \@plus\p@}%
%    \end{macrocode}
%
%    The macro |\numberline| requires that the width of the box that
%    holds the part number is stored in \LaTeX's scratch register
%    |\@tempdima|. Therefore we put it there. We begin a group, and
%    change some of the paragraph parameters (see also the remark at
%    \cs{l@part} regarding \cs{rightskip}).
%    \begin{macrocode}
    \setlength\@tempdima{1.5em}%
    \begingroup
      \parindent \z@ \rightskip \@pnumwidth
      \parfillskip -\@pnumwidth
%    \end{macrocode}
%    Then we leave vertical mode and switch to a bold font.
%    \begin{macrocode}
      \leavevmode \bfseries
%    \end{macrocode}
%    Because we do not use |\numberline| here, we have do some fine
%    tuning `by hand', before we can set the entry. We discourage but
%    not disallow a pagebreak immediately after a section entry.
%    \begin{macrocode}
      \advance\leftskip\@tempdima
      \hskip -\leftskip
      #1\nobreak\hfil \nobreak\hb@xt@\@pnumwidth{\hss #2}\par
    \endgroup
  \fi}
%</article>
\end{lstlisting}
\end{tcolorbox}

As you can see the dot leaders are not present in the above definition. Although we can get rid of dot leaders in other section by redefining them, it is not as easy to add them back.

As our aim is to be able to have all the classes used a common denominator we can define a command as follows (using book as a base)

\begin{tcolorbox}{}
\begin{lstlisting}
\def\articlesection{
\newcommand*\l@section[2]{%
  \ifnum \c@tocdepth >\z@
    \addpenalty\@secpenalty
    \addvspace{1.0em \@plus\p@}%
    \setlength\@tempdima{1.5em}%
    \begingroup
      \parindent \z@ \rightskip \@pnumwidth
      \parfillskip -\@pnumwidth
      \leavevmode \bfseries
      \advance\leftskip\@tempdima
      \hskip -\leftskip
      #1\nobreak\hfil \nobreak\hb@xt@\@pnumwidth{\hss #2}\par
    \endgroup
  \fi}
}
\end{lstlisting}
\end{tcolorbox}

%\articlesection

The \cs{@starredsection} macro is one of those locomotive type of commands. It takes 7 required arguments and 2 optional ones and hidden within it are two booleans. The full set looks like this:

\cs{@startsection} \marg{name} \marg{level} \marg{indent} \marg{beforeskip} \marg{afterskip} \marg{style}[*]
  [\marg{altheading}]\marg{heading}.

\begin{marglist}
\item[name] The name of the level command.
\item [level] A number denoting the depth of the section, chapter=1, section=2, etc. A section number will be printed only if \marg{level} is equal or smaller than the value of \textit{secnumdepth}
\item[indent] The indentation of the heading from the left margin.
\item[beforeskip]  The absolute value of this argument is the skip to leave above the heading. If it is negative, then the paragraph indent of the text following the heading is suppressed.
\item [afterskip] If positive, it is the skip to leave below the heading, else it is the skip to the right of a run-in heading.
\item [style] Sets the style of the heading.
\item[\textup{[*]}] When this is missing the heading is numbered and the corresponding counter is incremented.
\item[\textup{[\textit{altheading}]}] Gives an alternative heading to use in the table of contents and in the running heads. This should be present when the * form is used.
\item[heading] The heading of the new section.
\end{marglist}

\begin{example}{Example formatting run-in section}{}
\makeatletter
\bgroup
\renewcommand\section{%
    \@startsection{section}%
    {1}%
    {0em}%
    {-0.8em}%
    {-0.5em}%
    {\large\normalfont\scshape}}
\makeatother
\section[]{test}
\lorem
\egroup
\end{example}

Note we run the example in a group so that we will not influence the formatting of this document.

As mentioned earlier there is an additional way to introduce formatting for sections and this is using the command \cs{@seccntformat}, which is responsible for typesetting the counter part of a section title. The default definition of the command typesets the \cs{the} representation of the section counter.

\begin{example}{}{}
\bgroup
\renewcommand\section{%
    \@startsection{section}%
    {1}%
    {0em}%
    {-0.8em}%
    {-0.5em}%
    {\large\normalfont\scshape}}
\renewcommand\@seccntformat[1]{\fbox
{\csname the#1\endcsname}\hspace{0.5em}}
\makeatother
\section[]{test}\label{sec:ok}
\lorem

See section \ref{sec:ok}.
\egroup
\end{example}

The definition of \cs{@seccntformat} applies to all headings
defined with the \cs{@startsection} command (which is described in the next
section). Therefore, if you wish to use different definitions of \cs{@seccntformat}
for different headings, you must put the appropriate code into every heading
definition.

\begin{tcolorbox}
\begin{lstlisting}
\def\@seccntformat##1{\csname the##1\endcsname{}}
\end{lstlisting}
\end{tcolorbox}

\section{Custom headings}

It is also possible to define section headings without resorting to any of the above. To do this.

\begin{tcolorbox}
\begin{lstlisting}
\newcommand\part{\secdef\cmda\cmdb}
\end{lstlisting}
\end{tcolorbox}

the part and chapter and sometimes appendix are defined this way, but nothing stops us from doing the same for other sections. A generic section command can be defined as follows:

\begin{example}{}{}
\bgroup
\renewcommand\section[2] [?]{% % Complex form:
\refstepcounter{section}% % step counter/ set label
\addcontentsline{toc}{appendix}% % generate toe entry
{\protect\numberline{section-\thesection}#1}%
{\raggedright\large\bfseries section %\appendixname\ % typeset the title
\thesection\par \centering#2\par}% % and number
\sectionmark{#1}% % add to running header
\@afterheading % prepare indentation handling
%\addvspace{\baselineskip}
}
\section{Test}
\lorem
\egroup
\end{example}

Many other strategies can also be implemented that are perhaps easier to grasp.

\begin{example}{}{}
\bgroup
\def\strut{\vrule height12pt depth1pt width0pt}
\renewcommand\section[2] []{% % Complex form:
\refstepcounter{section}% % step counter/ set label
\addcontentsline{toc}{section}% % generate toc entry
{\protect\numberline{\thesection} }%
{\raggedright\large\bfseries\scshape %
\parbox[b]{\dimexpr(\linewidth-0.5\columnsep)}{\colorbox{brown!80}%
{{\vbox{\strut\raise2pt\hbox{#2}}}}}}\vskip0pt% % and number
\sectionmark{#1}% % add to running header
\@afterheading % prepare indentation handling
\vspace{\dimexpr\baselineskip+6pt}%must have a parameter
}
\chapter{Fossil Insects}
\begin{multicols*}{2}\raggedcolumns
\section[Insect Fossilization]{\raggedright \thinspace Insect Fossilization}
\lipsum[1]
\end{multicols*}
\egroup
\end{example}
% To answer http://tex.stackexchange.com/questions/52998/change-title-to-small-caps-but-not-in-toc

Of course some work is needed to center the text properly in the middle of the colour box. For all practical purposes it is lining up as per the sample.

In Chapter we discussed a forward, but this may not apply if there are no chapters or we need to treat these as sections, the example \ref{ex:forwardsection} shows such a method.

\begin{example}{Defining a Foreward Section}{ex:forwardsection}

\newcommand\prematter@sp[1]{% % Complex form:
%\refstepcounter{section}% % step counter/ set label
\addcontentsline{toc}{section}% % generate toe entry
{\protect\numberline{}\textsc{#1}}%
\sectionmark{#1}% % add to running header
{\LARGE\centering\normalfont\sffamily\colorbox{brown!80}{ \textsc{#1}}\par}%
\@afterheading % prepare indentation handling
\addvspace{\baselineskip}
\@afterindentfalse
}

\newenvironment{prematter}[1]{%
   \prematter@sp{#1}}
{}
\begin{multicols}{2}
\label{theok}
\begin{prematter}{Foreward}
\lipsum[1]
\end{prematter}\ref{theok}
\end{multicols}
\end{example}

\section{underlining}

I am aware that some people have no choice but have some sections underlined as dictated by archaic regulations in some establishments for thesis submission. If nobody is forcing you to underline it is best to avoid it. We use Donald Arsenau's ulem package to achieve underlining.
e
\DocInput{\jobname.dtx}
% \printindex
 %
% 
\end{document}
%</driver>
% \fi
% 
%  \CheckSum{0}
%  \CharacterTable
%  {Upper-case    \A\B\C\D\E\F\G\H\I\J\K\L\M\N\O\P\Q\R\S\T\U\V\W\X\Y\Z
%   Lower-case    \a\b\c\d\e\f\g\h\i\j\k\l\m\n\o\p\q\r\s\t\u\v\w\x\y\z
%   Digits        \0\1\2\3\4\5\6\7\8\9
%   Exclamation   \!     Double quote  \"     Hash (number) \#
%   Dollar        \$     Percent       \%     Ampersand     \&
%   Acute accent  \'     Left paren    \(     Right paren   \)
%   Asterisk      \*     Plus          \+     Comma         \,
%   Minus         \-     Point         \.     Solidus       \/
%   Colon         \:     Semicolon     \;     Less than     \<
%   Equals        \=     Greater than  \>     Question mark \?
%   Commercial at \@     Left bracket  \[     Backslash     \\
%   Right bracket \]     Circumflex    \^     Underscore    \_
%   Grave accent  \`     Left brace    \{     Vertical bar  \|
%   Right brace   \}     Tilde         \~}
%
%
%
% \changes{1.0}{2013/01/26}{Converted to DTX file}
%
% \DoNotIndex{\newcommand,\newenvironment}
% \GetFileInfo{phd.dtx}
% 
%  \def\fileversion{v1.0}          
%  \def\filedate{2012/03/06}
% \title{The \textsf{phd} package.
% \thanks{This
%        file (\texttt{phd.dtx}) has version number \fileversion, last revised
%        \filedate.}
% }
% \author{Dr. Yiannis Lazarides \\ \url{yannislaz@gmail.com}}
% \date{\filedate}
%
%
% 
% ^^A\maketitle
% 
% ^^A\frontmatter
%  ^^A\coverpage{./images/hine02.jpg}{Book Design }{Camel Press}{}{}
%  \newpage
% ^^A\secondpage
% \pagestyle{empty}
%
%
% 
%
%
% \pagestyle{headings}
% \raggedbottom
%  ^^A\OnlyDescription
%
%  ^^A\StopEventually{\printindex}

% \CodelineNumbered
% \pagestyle{headings}
% 
%<*LSECT>
% \part{IMPLEMENTATION}
% 

% \chapter{Implementation }
%
% 
%
% The original sectioning routines of the \latexe source are somewhat complicated by the
% way optional commands were build. With expl3's xparse module these type of multi-switch commands can be simplified and the code made clearer.
% 
% The aim of this package is to provide:
% \begin{enumerate}
% \item An extended key value interface to lower level sectioning functions.

% \item To provide a compatibility mode, where documents wishing to test the package
% can have an easy switch to switch in and out. This is also important for the testing of the package.

% \item To provide a number of pre-canned templates that cover most of the typical use case.

% \item To provide means for a plug-in architecture for extensions.
% \end{enumerate}
% 
%
% \section{Preliminaries}
%
%  Standard file identification. We first announce the package 
%	 and require that it be used with \LaTeX2e. 
%
%  \lorem
%
%    \begin{macrocode}
\NeedsTeXFormat{LaTeX2e}[1994/12/01]%
\ProvidesFile{phd-lowersections}[2015/1/13 v1.0 less preamble (YL)]%
%    \end{macrocode}
%
% \section{Source2e Interface}
% 
% I am not very fond of mixing expl3 control sequences with source2e commands. Here
% we provide an interface for all these commands we might use. 
% This section can be revisited once expl3 code becomes available.
%
%    \begin{macrocode}
\let\ltxtoday\today
\newif\if@ltxcompat \@ltxcompatfalse
%
\newcommand\tikzi[1][after heading] {%
    \tikz[remember picture,overlay] 
    \draw[<->] (0,0)--(0,.2)--++(-.5,0) node[left,fill=blue!15,text=black]%
       {{\ttfamily\scriptsize #1}};%\space%
  }
%    \end{macrocode}
%
% \section{Key Management}
%
% This part of the code is a bit verbose. We care to provide keys for all
% parameters in order to allow flexibility and easy extensions.
%
% \section{Sections}
%    \begin{macrocode}
\ExplSyntaxOn
%
\dim_gzero_new:N \l_phd_section_title_padding_top_dim
\dim_gzero_new:N \l_phd_section_title_padding_right_dim
\dim_gzero_new:N \l_phd_section_title_padding_bottom_dim
\dim_gzero_new:N \l_phd_section_title_padding_left_dim

\dim_gzero_new:N \l_phd_section_title_border_left_width_dim


\cs_new:Npn \makenewparams #1 
{
 \dim_gzero_new:c {l_phd_#1_padding_top_width_dim}
 \dim_gzero_new:c {l_phd_#1_padding_right_width_dim}
 \dim_gzero_new:c {l_phd_#1_padding_bottom_width_dim}
 \dim_gzero_new:c {l_phd_#1_padding_left_width_dim}
% 
 \dim_gzero_new:c {l_phd_#1_border_left_width_dim}
 \dim_gzero_new:c {l_phd_#1_border_right_width_dim}
 \dim_gzero_new:c {l_phd_#1_border_top_width_dim}
 \dim_gzero_new:c {l_phd_#1_border_bottom_width_dim}
% 
 \dim_gzero_new:c {l_phd_#1title_margin_top_dim}
 \dim_gzero_new:c {l_phd_#1title_margin_right_dim}
 \dim_gzero_new:c {l_phd_#1title_margin_bottom_dim}
 \dim_gzero_new:c {l_phd_#1title_margin_left_dim}
%
 \dim_gzero_new:c {l_phd_#1title_padding_top_width_dim}
 \dim_gzero_new:c {l_phd_#1title_padding_right_width_dim}
 \dim_gzero_new:c {l_phd_#1title_padding_bottom_width_dim}
 \dim_gzero_new:c {l_phd_#1title_padding_left_width_dim}
% 
 \dim_gzero_new:c {l_phd_#1title_border_left_width_dim}
 \dim_gzero_new:c {l_phd_#1title_border_right_width_dim}
 \dim_gzero_new:c {l_phd_#1title_border_top_width_dim}
 \dim_gzero_new:c {l_phd_#1title_border_bottom_width_dim}
}

\clist_new:N \phd_book_divisions_clist
\clist_gset:Nn \phd_book_divisions_clist
  {
    section,subsection,subsubsection,
    paragraph,subparagraph
  }
\clist_map_inline:Nn \phd_book_divisions_clist
  {
    \makenewparams{#1}
  }
%    \endmacrocode}
% 
% 
%    \begin{macrocode}
\ExplSyntaxOn
\def\makekeys#1
 {
  \cxset
    {
% Main elements names etc
     #1~name/.code                 = \cs_gset:cpn {#1name} {##1}, 
     #1~beforeskip/.code           = \cs_gset:cpn {l_phd_#1_before_skip_tl}{##1},
     #1~afterskip/.code            = \cs_gset:cpn {l_phd_#1_after_skip_tl}{##1},
     #1~indent/.code               = \cs_gset:cpn {l_phd_#1_indent_tl}{##1},
% fonts
     #1~font-size/.fontsize        = l_phd_#1_fontsize_tl,
     #1~font-weight/.fontweight    = l_phd_#1_fontweight_tl, 
     #1~font-shape/.fontstyle      = l_phd_#1_fontshape_tl,  
     #1~font-family/.fontfamily    = l_phd_#1_fontfamily_tl,   
% Format
     #1~format/.format~in          = l_phd_#1_format_tl,   
% Colors
     #1~background-color/.colorin =  l_phd_#1_background_color_tl,
     #1~color/.colorin            =  l_phd_#1_color_tl, 
% Main text alignment
     #1~align/.textalign           = l_phd_#1_align_tl,         
% Main element borders
     #1~border-top-width/.code    = \dim_set:cn {l_phd_#1_border_top_width_dim}    {##1},  
     #1~border-right-width/.code  = \dim_set:cn {l_phd_#1_border_right_width_dim}  {##1}, 
     #1~border-bottom-width/.code = \dim_set:cn {l_phd_#1_border_bottom_width_dim} {##1},  
     #1~border-left-width/.code   = \dim_set:cn {l_phd_#1_border_left_width_dim}   {##1}, 
% 
     #1~padding-top-width/.code    = \dim_set:cn {l_phd_#1_padding_top_width_dim}    {##1},  
     #1~padding-right-width/.code  = \dim_set:cn {l_phd_#1_padding_right_width_dim}  {##1}, 
     #1~padding-bottom-width/.code = \dim_set:cn {l_phd_#1_padding_bottom_width_dim} {##1},  
     #1~padding-left-width/.code   = \dim_set:cn {l_phd_#1_padding_left_width_dim}   {##1}, 
%         
     #1~margin-top-width/.code    = \dim_set:cn {l_phd_#1_margin_top_width_dim}    {##1},  
     #1~margin-right-width/.code  = \dim_set:cn {l_phd_#1_margin_right_width_dim}  {##1}, 
     #1~margin-bottom-width/.code = \dim_set:cn {l_phd_#1_margin_bottom_width_dim} {##1},  
     #1~margin-left-width/.code   = \dim_set:cn {l_phd_#1_margin_left_width_dim}   {##1}, 
%     
     #1~title~width/.code          = 
        \expandafter\def\cs:w l_phd_#1_title_width_dim\cs_end:{##1},
% title margins        
      #1~title~margin-top/.code  = \dim_gset:cn {l_phd_sectiontitle_margin_top_dim} {##1},
% Numbering Wow!
    #1~numbering/.is~choice,
    #1~numbering/roman/.code          =
       \cs_gset:cpn {the#1}
         {
           \cs:w l_phd_section_number_prefix_tl \cs_end: 
             \@roman\cs:w c@#1\cs_end:\relax
           \cs:w l_phd_#1_number_suffix_tl \cs_end:   
         },
    #1~numbering/Roman/.code          =
      \cs_gset:cpn {the#1}
        {
          \cs:w l_phd_section_number_prefix_tl \cs_end: 
          \expandafter\@Roman{\cs:w c@#1\cs_end:} \relax
          \cs:w l_phd_#1_number_suffix_tl \cs_end:  
        },
   #1~numbering/(roman)/.code          =
    \cs_gset:cpn {the#1}
       {
       \cs:w l_phd_section_number_prefix_tl \cs_end: 
         (\@roman\cs:w c@#1\cs_end:\relax)
       \cs:w l_phd_#1_number_suffix_tl \cs_end:    
       },
   #1~numbering/(Roman)/.code          =
    \cs_gset:cpn {the#1}
      {
        \cs:w l_phd_section_number_prefix_tl \cs_end: 
        (\@Roman \cs:w c@#1\cs_end:\relax)
        \cs:w l_phd_#1_number_suffix_tl \cs_end:  
      },
  #1~numbering/arabic/.code           =
    \cs_gset:cpn {the#1}
      {
        \cs:w l_phd_section_number_prefix_tl \cs_end: 
        \@arabic\cs:w c@#1\cs_end: \relax
        \cs:w l_phd_#1_number_suffix_tl \cs_end: 
      },
  #1~numbering/numeric/.code          =
    \cs_gset:cpn {the#1}
      {
        \cs:w l_phd_section_number_prefix_tl \cs_end: 
        \@arabic\cs:w c@#1\cs_end:
        \cs:w l_phd_#1_number_suffix_tl \cs_end: 
      },
  #1~numbering/none/.code             = 
    \cs_gset:cpn {the#1} {},
  #1~numbering/alpha/.code            = 
    \cs_gset:cpn {the#1}
      {
        \cs:w l_phd_section_number_prefix_tl \cs_end: 
        \exp_after:wN \alphalph {\cs:w c@#1\cs_end:}
        \cs:w l_phd_#1_number_suffix_tl \cs_end:   
      },
  #1~numbering/Alpha/.code            = 
    \cs_gset:cpn {the#1}
      {
        \cs:w l_phd_section_number_prefix_tl \cs_end: 
          \exp_after:wN \AlphAlph{\cs:w c@#1\cs_end:}
        \cs:w l_phd_#1_number_suffix_tl \cs_end:  
      },
  #1~numbering/words/.code            = 
    \cs_gset:cpn {the#1}
      {
       \cs:w l_phd_section_number_prefix_tl \cs_end: 
       \words@cx{\@arabic\cs:w c@#1\cs_end:}
       \cs:w l_phd_#1_number_suffix_tl \cs_end:  
      },
  #1~numbering/Words/.code            =
    \cs_gset:cpn {the#1}
      {
        \cs:w l_phd_section_number_prefix_tl \cs_end: 
        \Words@cx{\@arabic\cs:w c@#1\cs_end: }
        \cs:w l_phd_#1_number_suffix_tl \cs_end:  
      },
  #1~numbering/WORDS/.code            =
    \cs_gset:cpn {the#1}
      {
       \cs:w l_phd_section_number_prefix_tl \cs_end:   
       \WORDS@cx{\@arabic\cs:w c@#1\cs_end:}
       \cs:w l_phd_#1_number_suffix_tl \cs_end:  
      },
  #1~numbering~custom/.code           = 
    \cs_gset:cpn {the#1} {##1},        
    }   
  }   
%  
%
\clist_map_inline:Nn \phd_book_divisions_clist
  {
    \makekeys{#1}
  }
%  
\ExplSyntaxOff  
%    \end{macrocode}
%    \begin{macrocode} 
\ExplSyntaxOn 
%\cxset
% {  
%   section~title~margin-top/.code         = 
%   \dim_gset:cn {l_phd_sectiontitle_margin_top_dim} {#1},
% }  
%
\cxset{section~title~margin-top = 0pt,} 
\cxset 
  { 
%    
    section~arc/.store~in                  = \l_phd_section_arc_tl,
    section~grow~left/.store~in            = \l_phd_section_grow_left_dim,
    section~grow~right/.store~in           = \l_phd_section_grow_right_dim,
    section~rounded~corners/.store~in      = \l_phd_section_rounded_corners_tl,
    section~number~after/.store~in         = \l_phd_section_number_after_tl,
    section~numbering~prefix/.store~in     = \l_phd_section_number_prefix_tl,
    section~numbering~suffix/.store~in     = \l_phd_section_number_suffix_tl, 
%
 
%
%    section~align/.textalign               = \l_phd_section_align_tl,
%    
    section~afterindent/.onoff             = {afterindent@cx},
%
%    section~beforeskip/.store~in           = \l_phd_section_before_skip_tl,
%    section~afterskip/.store~in            = \l_phd_section_after_skip_tl,
%    section~indent/.store~in               = \l_phd_section_indent_tl,
    section~spaceout/.is~choice,
    section~spaceout/soul/.code            = \@sectionspaceouttrue,
    section~spaceout/none/.code            = \@sectionspaceoutfalse,
 }  
 
\ExplSyntaxOff 
%    \end{macrocode}
% As described earlier boxed headings have numerous elements, each of which can be styled
% on its own. 
% 
% \subsection{Subsection keys}
% From now on almost everything is a repetition of whatever was previously
% defined for higher order sectioning commands.
%
%    \begin{macrocode}
\ExplSyntaxOn
%  \cxset {subsection~format/.format~in     = \l_phd_subsection_format_tl }
%  \cxset {subsection~format = block }
%  \cxset {
%         subsection~background-color/.store~in=\l_phd_subsection_background_color_tl
%         }

\cxset{
%  subsection~font-size/.font-size~in       = \l_phd_subsection_font_size_tl,
%  subsection~font-weight/.font-weight~in   = \l_phd_subsection_font_weight_tl,
%  subsection~font-family/.font-family~in   = \l_phd_subsection_font_family_tl,
%  subsection~font-shape/.font-style~in     = \l_phd_subsection_font_shape_tl,
%  subsection~align/.textalign              = \l_phd_subsection_align_tl,
%  subsection~beforeskip/.store~in          = \l_phd_subsection_before_skip_tl,
%  subsection~afterskip/.store~in           = \l_phd_subsection_after_skip_tl,
%  subsection~indent/.store~in              = \l_phd_subsection_indent_tl,
  subsection~before/.store~in              = \l_phd_subsection_before_tl,
}
\ExplSyntaxOff 
%    \end{macrocode}
%
%    \begin{macrocode}
\ExplSyntaxOn
\def\l_phd_subsection_number_prefix_tl{}
\def\l_phd_subsection_number_suffix_tl{}
\cxset
  {
    subsection~numbering~suffix/.store~in=\l_phd_subsection_number_suffix_tl, 
    subsection~numbering~prefix/.store~in=\l_phd_subsection_number_refix_tl,
%    subsection~numbering/.is~choice,
%    subsection~numbering/roman/.code          =
%       \cs_gset:Npn \thesubsection
%         {
%           \l_phd_subsection_number_prefix_tl
%             \@roman\c@subsection
%           \l_phd_subsection_number_suffix_tl  
%         },
%    subsection~numbering/Roman/.code          =
%      \cs_gset:Npn \thesubsection
%        {
%          \l_phd_subsection_number_prefix_tl
%          \@Roman\c@subsection\relax
%          \l_phd_subsection_number_suffix_tl 
%        },
%  subsection~numbering/(roman)/.code          =
%    \cs_gset:Npn \thesubsection
%       {
%       \l_phd_subsection_number_prefix_tl
%         (\@roman\c@subsection\relax)
%       \l_phd_subsection_number_suffix_tl   
%       },
%  subsection~numbering/(Roman)/.code          =
%    \cs_gset:Npn \thesubsection
%      {
%        \l_phd_subsection_number_prefix_tl
%        (\@Roman\c@subsection)
%        \l_phd_subsection_number_suffix_tl 
%      },
%  subsection~numbering/arabic/.code           =
%    \cs_gset:Npn \thesubsection
%      {
%        \l_phd_subsection_number_prefix_tl
%        \@arabic\c@subsection\relax
%        \l_phd_subsection_number_suffix_tl
%      },
%  subsection~numbering/numeric/.code          =
%    \cs_gset:Npn \thesubsection
%      {
%        \l_phd_subsection_number_prefix_tl
%        \@arabic\c@subsection
%        \l_phd_subsection_number_suffix_tl
%      },
%  subsection~numbering/none/.code             = 
%    \cs_gset:Npn \thesubsection {},
%  subsection~numbering/alpha/.code            = 
%    \cs_gset:Npn \thesubsection 
%      {
%        \l_phd_subsection_number_prefix_tl
%        \alphalph\c@subsection
%        \l_phd_subsection_number_suffix_tl  
%      },
%  subsection~numbering/Alpha/.code            = 
%    \cs_gset:Npn \thesubsection 
%      {
%        \l_phd_subsection_number_prefix_tl
%        \AlphAlph\c@subsection
%        \l_phd_subsection_number_suffix_tl 
%      },
%  subsection~numbering/words/.code            = 
%    \cs_gset:Npn \thesubsection
%      {
%       \l_phd_subsection_number_prefix_tl
%       \words@cx{\@arabic\c@subsection}
%       \l_phd_subsection_number_suffix_tl 
%      },
%  subsection~numbering/Words/.code            =
%    \cs_gset:Npn \thesubsection
%      {
%        \l_phd_subsection_number_prefix_tl
%        \words@cx{\@arabic\c@subsection}
%        \l_phd_subsection_number_suffix_tl 
%      },
%  subsection~numbering/WORDS/.code            =
%    \cs_gset:Npn \thesubsection
%      {
%       \l_phd_subsection_number_prefix_tl  
%       \words@cx{\@arabic\c@subsection}
%       \l_phd_subsection_number_suffix_tl 
%      },
%  subsection numbering custom/.code           = 
%    \cs_gset:Npn \thesubsection {#1},
}
\ExplSyntaxOff
%    \end{macrocode}
%    \begin{macrocode}
\cxset{
     subsection number after/.store in=\subsectionnumberafter@cx,
     }
%    \end{macrocode}
%
% \subsection{Subsubsections}
% 
% 
%    \begin{macrocode}
%
\ExplSyntaxOn 


\cxset{  
%  subsubsection~name/.store~in                   = \subsubsectionname,
%  subsubsection~format/.store~in               = \l_phd_subsubsection_format_tl,  
%  subsubsection~background-color/.store~in      = \l_phd_background_color_tl,
%  subsection~bakground~color/.store~in          = \l_phd_subsection_background_color_tl,
%  subsubsection~font-size/.font-size~in         = \l_phd_subsubsection_fontsize_tl,
%  subsubsection~font-weight/.font-weight~in     = \l_phd_subsubsection_fontweight_tl,
%  subsubsection~font-family/.font-family~in     = \l_phd_subsubsection_fontfamily_tl,
%  subsubsection~font-shape/.font-style~in       = \l_phd_subsubsection_fontshape_tl,
  %subsubsection~color/.store~in                 = \l_phd_subsubsection_color_tl,
}
\ExplSyntaxOff

\def\l_phd_subsubsection_number_prefix_tl{}
\def\l_phd_subsubsection_number_suffix_tl{}
\ExplSyntaxOn
\cxset
  {  
    subsubsection~number~prefix/.store~in        =
     \l_phd_subsubsection_number_prefix_tl,
    subsubsection~number~suffix/.store~in        =
     \l_phd_subsubsection_number_suffix_tl, 
%     
%    subsubsection~numbering/.is~choice,
%    subsubsection~numbering/roman/.code          =
%       \cs_gset:Npn \thesubsubsection
%         {
%           \l_phd_subsubsection_number_prefix_tl
%             \@roman\c@subsubsection
%           \l_phd_subsubsection_number_suffix_tl  
%         },
%    subsubsection~numbering/Roman/.code          =
%      \cs_gset:Npn \thesubsubsection
%        {
%          \l_phd_subsubsection_number_prefix_tl
%          \@Roman\c@subsubsection\relax
%          \l_phd_subsubsection_number_suffix_tl 
%        },
%  subsubsection~numbering/(roman)/.code          =
%    \cs_gset:Npn \thesubsubsection
%       {
%       \l_phd_subsubsection_number_prefix_tl
%         (\@roman\c@subsubsection\relax)
%       \l_phd_subsubsection_number_suffix_tl   
%       },
%  subsubsection~numbering/(Roman)/.code          =
%    \cs_gset:Npn \thesubsubsection
%      {
%        \l_phd_subsubsection_number_prefix_tl
%        (\@Roman\c@subsubsection)
%        \l_phd_subsubsection_number_suffix_tl 
%      },
%  subsubsection~numbering/arabic/.code           =
%    \cs_gset:Npn \thesubsubsection
%      {
%        \l_phd_subsubsection_number_prefix_tl
%        \@arabic\c@subsubsection\relax
%        \l_phd_subsubsection_number_suffix_tl
%      },
%  subsubsection~numbering/numeric/.code          =
%    \cs_gset:Npn \thesubsubsection
%      {
%        \l_phd_subsubsection_number_prefix_tl
%        \@arabic\c@subsubsection
%        \l_phd_subsubsection_number_suffix_tl
%      },
%  subsubsection~numbering/none/.code             = 
%    \cs_gset:Npn \thesubsubsection {},
%  subsubsection~numbering/alpha/.code            = 
%    \cs_gset:Npn \thesubsubsection 
%      {
%        \l_phd_subsubsection_number_prefix_tl
%        \alphalph\c@subsubsection
%        \l_phd_subsubsection_number_suffix_tl  
%      },
%  subsubsection~numbering/Alpha/.code            = 
%    \cs_gset:Npn \thesubsubsection 
%      {
%        \l_phd_subsubsection_number_prefix_tl
%        \AlphAlph\c@subsubsection
%        \l_phd_subsubsection_number_suffix_tl 
%      },
%  subsubsection~numbering/words/.code            = 
%    \cs_gset:Npn \thesubsubsection
%      {
%       \l_phd_subsubsection_number_prefix_tl
%       \words@cx{\@arabic\c@subsubsection}
%       \l_phd_subsubsection_number_suffix_tl 
%      },
%  subsubsection~numbering/Words/.code            =
%    \cs_gset:Npn \thesubsubsection
%      {
%        \l_phd_subsubsection_number_prefix_tl
%        \words@cx{\@arabic\c@subsubsection}
%        \l_phd_subsubsection_number_suffix_tl 
%      },
%  subsubsection~numbering/WORDS/.code            =
%    \cs_gset:Npn \thesubsubsection
%      {
%       \l_phd_subsubsection_number_prefix_tl  
%       \words@cx{\@arabic\c@subsubsection}
%       \l_phd_subsubsection_number_suffix_tl 
%      },
%  section numbering custom/.code                 = 
%    \cs_gset:Npn \thesubsubsection {#1},
%   subsubsection numbering/none/.code            =
%       \gdef\thesubsubsection{},    
%      
%    subsubsection~align/.textalign               = \l_phd_subsubsection_align_tl,
%    subsubsection~afterskip/.store~in            = \l_phd_subsubsection_after_skip_tl,
%    subsubsection~indent/.store~in               = \l_phd_subsubsection_indent_tl,
    subsubsection~number~after/.store~in         = \l_phd_subsubsection_number_after_tl,
}  
\ExplSyntaxOff  
    %
%    \end{macrocode}  
%
% \subsection{Paragraphs}  
% Mostly paragraphs are typeset inline, however, here are some keys, in case a layout requires complicated paragraphs..
%    \begin{macrocode}
%
\ExplSyntaxOn
\def\l_phd_paragraph_number_prefix_tl{}
\def\l_phd_paragraph_number_suffix_tl{}
\cxset
  {
%    paragraph~name/.store~in                     = \paragraphname,
    paragraph~number~prefix/.store~in            = \l_phd_paragraph_number_prefix_tl,
    paragraph~number~suffix/.store~in            = \l_phd_paragraph_number_suffix_tl,  
%    paragraph~format/.store~in                   = \l_phd_paragraph_format_tl,
%    paragraph~name/.store~in                     = \paragraphname,
%    paragraph~font-weight/.font-weight~in        = \l_phd_paragraph_font_weight_tl,
    paragraph~font-family/.font-family~in        = \l_phd_paragraph_font_family_tl,
%    paragraph~font-shape/.font-style~in          = \l_phd_paragraph_font_shape_tl,
    paragraph~color/.store~in                    = \l_ph_paragraph_color_tl,
%    paragraph~numbering/.is~choice,
%    paragraph~numbering/roman/.code          =
%       \cs_gset:Npn \theparagraph
%         {
%           \l_phd_paragraph_number_prefix_tl
%             \@roman\c@paragraph
%           \l_phd_paragraph_number_suffix_tl  
%         },
%    paragraph~numbering/Roman/.code          =
%      \cs_gset:Npn \theparagraph
%        {
%          \l_phd_paragraph_number_prefix_tl
%          \@Roman\c@paragraph\relax
%          \l_phd_paragraph_number_suffix_tl 
%        },
%  paragraph~numbering/(roman)/.code          =
%    \cs_gset:Npn \theparagraph
%       {
%       \l_phd_paragraph_number_prefix_tl
%         (\@roman\c@paragraph\relax)
%       \l_phd_paragraph_number_suffix_tl   
%       },
%  paragraph~numbering/(Roman)/.code          =
%    \cs_gset:Npn \theparagraph
%      {
%        \l_phd_paragraph_number_prefix_tl
%        (\@Roman\c@paragraph)
%        \l_phd_paragraph_number_suffix_tl 
%      },
%  paragraph~numbering/arabic/.code             =
%    \cs_gset:Npn \theparagraph
%      {
%        \l_phd_paragraph_number_prefix_tl
%        \@arabic\c@paragraph\relax
%        \l_phd_paragraph_number_suffix_tl
%      },
%  paragraph~numbering/numeric/.code          =
%    \cs_gset:Npn \theparagraph
%      {
%        \l_phd_paragraph_number_prefix_tl
%        \@arabic\c@paragraph
%        \l_phd_paragraph_number_suffix_tl
%      },
%  paragraph~numbering/none/.code              = 
%    \cs_gset:Npn \theparagraph {},
%  paragraph~numbering/alpha/.code             = 
%    \cs_gset:Npn \theparagraph 
%      {
%        \l_phd_paragraph_number_prefix_tl
%        \alphalph\c@paragraph
%        \l_phd_paragraph_number_suffix_tl  
%      },
%  paragraph~numbering/Alpha/.code            = 
%    \cs_gset:Npn \theparagraph 
%      {
%        \l_phd_paragraph_number_prefix_tl
%        \AlphAlph\c@paragraph
%        \l_phd_paragraph_number_suffix_tl 
%      },
%  paragraph~numbering/words/.code           = 
%    \cs_gset:Npn \theparagraph
%      {
%       \l_phd_paragraph_number_prefix_tl
%       \words@cx{\@arabic\c@paragraph}
%       \l_phd_paragraph_number_suffix_tl 
%      },
%  paragraph~numbering/Words/.code          =
%    \cs_gset:Npn \theparagraph
%      {
%        \l_phd_paragraph_number_prefix_tl
%        \words@cx{\@arabic\c@paragraph}
%        \l_phd_paragraph_number_suffix_tl 
%      },
%  paragraph~numbering/WORDS/.code        =
%    \cs_gset:Npn \theparagraph
%      {
%       \l_phd_paragraph_number_prefix_tl  
%       \words@cx{\@arabic\c@paragraph}
%       \l_phd_paragraph_number_suffix_tl 
%      },
%  paragraph~numbering custom/.code                     = 
%    \cs_gset:Npn \theparagraph {#1},
%    paragraph~align/.textalign                   = \l_phd_paragraph_align_tl,
%    paragraph~beforeskip/.store~in               = \l_phd_paragraph_before_skip_tl,
%    paragraph~afterskip/.store~in                = \l_phd_paragraph_after_skip_tl,
%    paragraph~indent/.store~in                   = \l_phd_paragraph_indent_tl,
    paragraph~number~after/.store~in             = \l_phd_paragraph_number_after_tl,
%    paragraph~background-color/.store~in         = \l_phd_paragraph_background_color_tl,
}  
\ExplSyntaxOff
%    \end{macrocode}
%
% \subsection {Subparagraphs}
%    \begin{macrocode}
%% subparagraphs
%
\ExplSyntaxOn
\def\l_phd_subparagraph_number_prefix_tl{}
\def\l_phd_subparagraph_number_suffix_tl{}
\cxset
  {
%    subparagraph~name/.store~in                 = \subparagraphname,
    subparagraph~number~prefix/.store~in       = \l_phd_subparagraph_number_prefix_tl,
    subparagraph~number~suffix/.store~in       = \l_phd_subparagraph_number_suffix_tl,  
%   subparagraph~format/.format~in             = l_phd_subparagraph_format_tl,
%    subparagraph~font-size/.font-size~in       = \l_phd_subparagraph_fontsize_tl,
%    subparagraph~font-weight/.font-weight~in   = \l_phd_subparagraph_fontweight_tl,
%    subparagraph~font-family/.font-family~in   = \l_phd_subparagraph_fontfamily_tl,
%    subparagraph~font-shape/.font-style~in     = \l_phd_subparagraph_fontshape_tl,
%    subparagraph~color/.store~in               = \l_ph_subparagraph_color_tl,
%    subparagraph~background-color/.store~in    = \l_phd_subparagraph_background_color_tl,
%    subparagraph~numbering/.is~choice,
%    subparagraph~numbering/roman/.code          =
%       \cs_gset:Npn \thesubparagraph
%         {
%           \l_phd_subparagraph_number_prefix_tl
%             \@roman\c@subparagraph
%           \l_phd_subparagraph_number_suffix_tl  
%         },
%    subparagraph~numbering/Roman/.code          =
%      \cs_gset:Npn \thesubparagraph
%        {
%          \l_phd_subparagraph_number_prefix_tl
%          \@Roman\c@subparagraph\relax
%          \l_phd_subparagraph_number_suffix_tl 
%        },
%  subparagraph~numbering/(roman)/.code          =
%    \cs_gset:Npn \thesubparagraph
%       {
%       \l_phd_subparagraph_number_prefix_tl
%         (\@roman\c@subparagraph\relax)
%       \l_phd_subparagraph_number_suffix_tl   
%       },
%  subparagraph~numbering/(Roman)/.code          =
%    \cs_gset:Npn \thesubparagraph
%      {
%        \l_phd_subparagraph_number_prefix_tl
%        (\@Roman\c@subparagraph)
%        \l_phd_subparagraph_number_suffix_tl 
%      },
%  subparagraph~numbering/arabic/.code             =
%    \cs_gset:Npn \thesubparagraph
%      {
%        \l_phd_subparagraph_number_prefix_tl
%        \@arabic\c@subparagraph\relax
%        \l_phd_subparagraph_number_suffix_tl
%      },
%  subparagraph~numbering/numeric/.code          =
%    \cs_gset:Npn \thesubparagraph
%      {
%        \l_phd_subparagraph_number_prefix_tl
%        \@arabic\c@subparagraph
%        \l_phd_subparagraph_number_suffix_tl
%      },
%  subparagraph~numbering/none/.code              = 
%    \cs_gset:Npn \thesubparagraph {},
%  subparagraph~numbering/alpha/.code             = 
%    \cs_gset:Npn \thesubparagraph 
%      {
%        \l_phd_subparagraph_number_prefix_tl
%        \alphalph\c@subparagraph
%        \l_phd_subparagraph_number_suffix_tl  
%      },
%  subparagraph~numbering/Alpha/.code            = 
%    \cs_gset:Npn \thesubparagraph 
%      {
%        \l_phd_subparagraph_number_prefix_tl
%        \AlphAlph\c@subparagraph
%        \l_phd_subparagraph_number_suffix_tl 
%      },
%  subparagraph~numbering/words/.code           = 
%    \cs_gset:Npn \thesubparagraph
%      {
%       \l_phd_subparagraph_number_prefix_tl
%       \words@cx{\@arabic\c@subparagraph}
%       \l_phd_subparagraph_number_suffix_tl 
%      },
%  subparagraph~numbering/Words/.code          =
%    \cs_gset:Npn \thesubparagraph
%      {
%        \l_phd_subparagraph_number_prefix_tl
%        \words@cx{\@arabic\c@subparagraph}
%        \l_phd_subparagraph_number_suffix_tl 
%      },
%  subparagraph~numbering/WORDS/.code        =
%    \cs_gset:Npn \thesubparagraph
%      {
%       \l_phd_subparagraph_number_prefix_tl  
%       \words@cx{\@arabic\c@subparagraph}
%       \l_phd_subparagraph_number_suffix_tl 
%      },
%  subparagraph~numbering custom/.code                     = 
%    \cs_gset:Npn \thesubparagraph {#1},    
%    subparagraph~align/.textalign              = \l_phd_subparagraph_align_tl,
%    subparagraph~beforeskip/.store~in          = \l_phd_subparagraph_before_skip_tl,
%    subparagraph~afterskip/.store~in           = \l_phd_subparagraph_after_skip_tl,
%    subparagraph~indent/.store~in              = \l_phd_subparagraph_indent_tl,
    subparagraph~number~after/.store~in        = \l_phd_subparagraph_number_after_tl,
}
\ExplSyntaxOff
\cxset{subparagraph background-color           = sweet!50}  

%    \end{macrocode}
%

% \section{Renewsection commands}
%
% These have to be called explicitly after key definitions, it is just 
% the way LaTeX works. One could add them in settings or explore a
% bit more deeply. 
%

%
%
% \chapter{Formatters}

% We first define some formatters to tie up with using |format| in sectioning commands.
%
%    \begin{macrocode}
\ExplSyntaxOn
\cs_set:Npn \format_inmargin:nnn #1#2#3
  {
     \tcbdocmarginnote
     { 
       
       \hbox{Section~\@svsec}
       #3
     } 
  }    
\ExplSyntaxOff  
%    \end{macrocode}
%
%    \begin{macrocode}
\ExplSyntaxOn
% 1 skip from left
% 2 
\cs_set:Npn \format_hang:nn #1#2#3 
  {{ \@hangfrom{#2\relax\@svsec%
      \interlinepenalty \@M #3\@@par}%
  }}
\ExplSyntaxOff
%    \end{macrocode}
%
% \subsection{Block format}
%
% Before we start setting boxes within boxes, we need a function to rename
% parameters to enable automatic creation of styles for all the boxed
% elements. The following elements in block formats are contained in
% their own boxes. Each box has its own style.
%  
%    \begin{macrocode}

\ExplSyntaxOn
\cs_set:Npn \phd_set_box_parameters:nn #1 #2
{
  % background color  
    \cs_if_exist:cTF {l_phd_#1_background_color_tl}
      {
        \cs_set:cpn {#1tcbbgcolor} { \cs:w l_phd_#1_background_color_tl\cs_end: }
      }
      {
        \cs_set:cpn {#1tcbbgcolor} {white} 
      }
% 
   \cs_if_exist:cTF {l_phd_#1_arc_tl}
     {
       \cs_set:cpn {tcbarc_#2}
              {\cs:w l_phd_#1_arc_tl \cs_end: }
     }
     {
        \cs_set:cpn {tcbarc_#2} {0pt}
     }
% alignment \FIRE
\cs_if_exist:cTF {l_phd_#1_align_tl}
     {
        \def\tcbalign{\csnamel_#1_align_tl\endcsname}
     }
     {
       \def\tcbalign{}
     }
%  \l_phd_section_grow_left_dim         
\cs_if_exist:cTF {l_phd_#1_grow_left_dim}
    {
      \def\tcbgrowleft{\csname l_phd_#1_grow_left_dim \endcsname} 
    }
    {
      \def\tcbgrowleft{0pt}
    }
\cs_if_exist:cTF {l_phd_#1_grow_right_dim}
    {
      \def\tcbgrowright{\cs:w l_phd_#1_grow_right_dim \cs_end:} 
    }
    {
      \def\tcbgrowright{0pt}
    } 
\cs_if_exist:cTF {l_phd_#1_rounded_corners_tl}
    {
      \def\tcbroundedcorners{\cs:w l_phd_#1_rounded_corners_tl \cs_end:} 
    }
    {
      \def\tcbroundedcorners{all}
    } 
 \cs_if_exist:cTF { l_phd_#1title_margin_top_dim }
    {
      \def\tcbtitlevspace{\cs:w l_phd_#1title_margin_top_dim \cs_end:} 
    }
    {
      \def\tcbtitlevspace{0pt}
    } 
  \dim_if_exist:cTF {l_phd_#1_border_top_width_dim }
    {
      \def\tcbtoprulewidth{\dim_use:c {l_phd_#1_border_top_width_dim} }
    }
    {
      \def\tcbtoprulewidth{0pt}
    }    
 \dim_if_exist:cTF {l_phd_#1_border_left_width_dim }
    {
      \def\tcbleftrulewidth{\dim_use:c {l_phd_#1_border_left_width_dim} }
    }
    {
      \def\tcbleftrulewidth{0pt}
    } 
 
 \dim_if_exist:cTF {l_phd_#1_border_bottom_width_dim }
    {
      \def\tcbbottomrulewidth{\dim_use:c {l_phd_#1_border_bottom_width_dim} }
    }
    {
      \def\tcbbottomrulewidth{0pt}
    }           
 \dim_if_exist:cTF {l_phd_#1_border_right_width_dim }
    {
      \def\tcbrightrulewidth {\dim_use:c {l_phd_#1_border_right_width_dim} }
    }
    {
      \def\tcbrightrulewidth{0pt}
    } 
% Padding       
\dim_if_exist:cTF {l_phd_#1_padding_top_width_dim }
    {
      \cs_set:cpn {#1tcbtopsepwidth} {\dim_use:c {l_phd_#1_padding_top_width_dim} }
    }
    {
      \cs_set:cpn {#1tcbtopsepwidth} {0pt}
    }    
 \dim_if_exist:cTF {l_phd_#1_padding_left_width_dim }
    {
      \cs_set:cpn {#1tcbleftsepwidth} {\dim_use:c {l_phd_#1_padding_left_width_dim} }
    }
    {
      \cs_set:cpn {#1tcbleftsepwidth} {0pt}
    } 
 
 \dim_if_exist:cTF {l_phd_#1_padding_bottom_width_dim }
    {
      \cs_set:cpn {#1tcbbottomsepwidth} {\dim_use:c {l_phd_#1_padding_bottom_width_dim} }
    }
    {
      \cs_set:cpn {#1tcbbottomsepwidth} {0pt}
    }           
 \dim_if_exist:cTF {l_phd_#1_padding_right_width_dim }
    {
      \cs_set:cpn {#1tcbrightsepwidth} {\dim_use:c {l_phd_#1_padding_right_width_dim} }
    }
    {
      \cs_set:cpn {#1tcbrightsepwidth} {0pt}
    }        
    
}         
%    \end{macrocode}
% 
% \begin{docCommand}{make_box_style:n}{ \meta{label name} \meta{style second name} }
%   Every block formatted heading is contained in an outer box. This has
%   a named style such as |outerbox section|. There is a lot of name swapping
%   to generally abstract any element.
% \end{docCommand}
%
%    \begin{macrocode} 
\cs_set:Npn \make_box_style:n #1 #2
  {
    \phd_set_box_parameters:nn {#1}{#2} %section outer
    \tcbset
      {
        #1~#2/.style=
          {
            size               = minimal, %resets
            enhanced,
            colback            = \cs:w #1tcbbgcolor \cs_end:, 
            colframe           = black!80, 
            sharpish~corners,
            sharp~corners      = all,
%            arc                = \tcbarc_outer,
%            auto~outer~arc,                   
%            rounded~corners    = \tcbroundedcorners,
%            fuzzy~shadow       = {2mm}{-1mm}{0mm}{0.1mm}{black!50!white},
% padding           
            left               = \cs:w #1tcbleftsepwidth   \cs_end:,
            right              = \cs:w #1tcbrightsepwidth  \cs_end:,
            bottom             = \cs:w #1tcbbottomsepwidth \cs_end:,
            top                = \cs:w #1tcbtopsepwidth    \cs_end:,
% frame rules           
            toprule            = \tcbtoprulewidth,
            leftrule           = \tcbleftrulewidth,
            rightrule          = \tcbrightrulewidth,
            bottomrule         = \tcbbottomrulewidth,
% border rules           
            grow~to~right~by   = \tcbgrowright,
            grow~to~left~by    = \tcbgrowleft,
            boxsep             = 0pt,
            valign~lower       = center,%not necessary?
%           interior~style     = {left~color=blue!20!white,
%                                 right~color=blue!30!white},
%            borderline~north   = {2pt}{-2pt}{red},  
%            borderline~south   = {2pt}{-2pt}{red},   
%            borderline~east    = {2pt}{0pt}{red},
%            borderline~west    = {2pt}{0pt}{red},                             
          }
    }   
  }   
%    \end{macrocode}
%
% \begin{docCommand}{format_block:nnnn} {\meta{}\meta{}\marg{}\marg{}}
%   This function is the main function for block and fancy headings. It
%   is also used for Chapter and part formatting. Depending on the 
%   settings it contains the following boxes:\\
%   1.0 Label box e.g., section\\
%   2.0 Number box e.g., 1.12.24\\
%   3.0 Title\\
%   It also handles the before and after boxes that handle rules and 
%   ornamentation if any. 
%
% \end{docCommand}
%
%    \begin{macrocode}    
\cs_set:Npn \format_block:nnnn #1#2#3#4
{
 \bgroup
 \make_box_style:n {#1} {outer} 
 \make_box_style:n {#1} {inner} 
 \begin{tcolorbox}[#1~outer]
    \language-1\relax
% number and section name together
    \begin{tcolorbox}[#1~outer,size=minimal]
      \RaggedLeft %this is float
      \begin{tcolorbox}[#1~outer,size=minimal, width=6cm]
        \RaggedLeft    
    %\tcbalign
    #3 
    \cs:w #1name\cs_end: \space 
    ~\@svsec
            \par
    \end{tcolorbox}
    \par
    \end{tcolorbox}
% add vertical space if specified    
    \dim_compare:nNnTF {\tcbtitlevspace} > {0sp}
    {
      \skip_vertical:N \tcbtitlevspace
    }
    {}
% title float box    
    \begin{tcolorbox}[#1~outer,size=minimal]
      \RaggedLeft %this is float
      \begin{tcolorbox}[#1~outer,size=minimal, width=6cm]
        \RaggedLeft
         #4~\fox 
        \par
      \end{tcolorbox}
      \par
    \end{tcolorbox}
 \end{tcolorbox}
 \egroup  
 \par\nobreak\nointerlineskip
}    

\ExplSyntaxOff
%    \end{macrocode}
% 
% \subsection{Display format}
% The display format typesets a heading in a similar fashion to 
% traditional chapters.
%
% \begin{docCommand}{format_display:nn} {\marg{section name}} { \marg{skip after number} \marg {} }
%   Displays a section similar to Chapters
% \end{docCommand}
%  \#1 Section name \\
%  \#2 indent       \\
%  \#3 format para        \\
%  \#4 Title text\\
%  svsec number
%    \begin{macrocode}
\ExplSyntaxOn
\cs_set:Npn \format_display:nnnn #1 #2 #3 #4
{
  \cxset{section~title~margin-top=30pt}
  \format_block:nnnn {#1}{#2}{#3}{#4}
}
 \ExplSyntaxOff
%    \end{macrocode}
%
% \#1 name
% \#2 indent
% \#3 title
%  
%    \begin{macrocode}
\ExplSyntaxOn
\cs_set:Npn \format_inline:nnn #1 #2 #3
  {
   {\bfseries\normalfont
    \theparagraph #3}
   }    
\ExplSyntaxOff  
%    \end{macrocode}
%
% \chapter{Layout Engine Code}
%
% The standard kernel factory commands, they are real locomotives. To hook into them
% we need to dig deep.
%
% We also define \cs{@startsection} as somehow there are problems
% with after indent false. valid |\section*{title}|, |\section[toc-entry]|, 
% |\section [toc-entry] {title}|
%
%  |#1| name i.e, section
%  |#2| level number 2 section
%  |#3| indent
%  |#4| beforeskip
%  |#5| afterskip
%  |#6|  styling command
%
%    \begin{verbatim}
% \def\@startsection#1#2#3#4#5#6{%
%    \if@noskipsec \leavevmode \fi
%    \par
%    \@tempskipa #4\relax 
%    \@afterindenttrue
%    \ifdim \@tempskipa <\z@
%        \@tempskipa -\@tempskipa\@afterindentfalse
%    \fi
%    \if@nobreak
%    \everypar{}%
%    \else
%      \addpenalty\@secpenalty\addvspace\@tempskipa
%    \fi
%   \@ifstar
%   {\@ssect{#3}{#4}{#5}{#6}}%defined in the kernel
%   {\@dblarg{\@sect{#1}{#2}{#3}{#4}{#5}{#6}}}}
%    \end{verbatim}
%   
%    \begin{macrocode}
\ExplSyntaxOn
\DeclareDocumentCommand \start_section:nnnnnnnnn {m m m m m m s o m}      
  {
    \if@noskipsec \leavevmode \fi
    \par
%    check for before skip    
    \l_tmpa_skip #4\relax 
    \@afterindenttrue
    \if_dim:w \l_tmpa_skip <\z@
% make it positive    
      \skip_gset:Nn\l_tmpa_skip {-\l_tmpa_skip}  
      \@afterindentfalse
    \fi:
%    
    \if@nobreak
      \everypar{zz}
    \else
      \addpenalty \@secpenalty
      \addvspace\l_tmpa_skip
    \fi
%
% redirect depending on star or option
%     
    \IfBooleanTF {#7}
      {\@ssect {#3} {#4} {#5} {#6} {#9} }
      {
        \IfValueTF {#8} {\@sect:  {#1} {#2} {#3} {#4} {#5} {#6} [{#8}] {#9} }
                        {\@sect:  {#1} {#2} {#3} {#4} {#5} {#6} [{#8}] {#9} } %sends TF we get it later
      }   
  }

\if@ltxcompat 
  \else 
  \cs_gset_eq:NN \@startsection \start_section:nnnnnnnnn
\fi

\ExplSyntaxOff  
%    \end{macrocode}
%

%    \begin{macrocode}
\ExplSyntaxOn
\cs_set:Npn \@sect: #1 #2 #3 #4 #5 #6 [#7] #8 
  {

%  Decide if we need to add the section in the toc.
    \int_compare:nTF {#2>\c@secnumdepth} 
      {
         \let\@svsec\@empty
      }
      {
        \refstepcounter{#1}
        \protected@edef\@svsec
          {
            \@seccntformat{#1}\relax
          }
        % add short title or long title  
        \IfValueTF{#7}  
          { 
             \cs:w #1mark\cs_end: {#7} 
             \addcontentsline{toc}{#1}{
                \protect\numberline{\csname the#1\endcsname}#7}
          }
          { 
             \cs:w #1mark\cs_end: {#8} 
              \addcontentsline{toc}{#1}{ 
                \protect\numberline{\csname the#1\endcsname}#8}
          }
      } 
     
% 
   \@tempskipa #5\relax
   \gdef\@svsechd{
   %\@seccntformat{#1}#6{\hskip #3\relax #8}
   #8
   }%
%  \ifdim \@tempskipa>\z@
%    \end{macrocode}

%    \begin{macrocode}
    \str_case_x:nnTF {\cs:w l_phd_#1_format_tl \cs_end:}  
      {
          { display } {\format_display:nnnn { #1 } { #3 } {#6} { #8 } 
                        \xsect:n {#5}   } 
          { block   } { \format_block:nnnn  {#1 } { #3 } {#6} { {#8}       } 
                        \xsect:n {#5} 
                      } 
          { plain   } { \format_hang:nn    {#1} { #3 } { #8 }         } 
          { hangs    } { \format_hang:nn    { #1 } { #3 } {{#6#8}} \xsect:n {#5}   }
          { inline   } {  \xsect:n {-3.5ex} }%#5
          { inmargin } {\format_inmargin:nnn {#1} {#3} {#6#8} }
      }
      {
      %{\if@debug~\tiny\csname#1format@cx\endcsname \fi }
      } %true code
      { 
        { 
        %\if@debug~\tiny\csname#1format@cx\endcsname\fi
        }
        \@hangfrom {#6\hskip #3\relax\@svsec}%
        \interlinepenalty \@M #8\@@par  \xsect:n{#5} 
      } %false code 
  %
  }
\ExplSyntaxOff  
%    \end{macrocode}
%
% \begin{docCommand}{@ssect} { {\meta{indent}} {\meta{beforeskip}} {\meta{afterskip}} \meta{styling commands} \meta{arg1} }
% This is the star verson of the command.  What it means is that we want a heading with
% no numbers and not in the toc. Also it does not add it as a mark! This is very limiting
% as originally programmed in the kernel; probably the thinking was to use it to create
% same style headings, that one would use for purposes other than sectioning. In reality
% many books have unnumbered sections and one might want them to go on the headings.
% We modify it to be able to do both based on a settings command.
% So to summarize star section means unumbered. Will use choices as
% to what must be done with it.
%
%  
%  \#1 indent\\
%  \#2 beforeskip\\
%  \#3 afterskip\\
%  \#4 styling command\\
%  \#5 arg1 follows\\
%
% \end{docCommand} 
%    \begin{macrocode}  
\ExplSyntaxOn  
%  
\cs_set:Npn \@ssect #1 #2 #3 #4 #5 {%
  \@tempskipa #3\relax
  \ifdim \@tempskipa>\z@
  \begingroup
    #4{
    \@hangfrom{\hskip #1}%
    \interlinepenalty \@M (#5)\@@par}%
    \endgroup
  \else
  \def\@svsechd{#4{\hskip #1\relax #5}}%
  \fi
% |\xsect:n{afterskip}| then sets the afteskipping as well as the afterindent.   
  \xsect:n{#3} ONLY NEEDED FOR HANG PARA
}
%   
\ExplSyntaxOff   
   
%    \end{macrocode}
%
% \begin{docCommand}{@xsect:n} {\marg{afterskip}}
%  This command sets handles indentation after a sectioning command. It also handles
%  the printing of the title for inline sections (it is saved as |\@svsechd| earlier. It is common
%  for both the star and unstarred versions of |\section|.
% \end{docCommand}
%
%|\@noskipsec| A switch set true by a sectioning command when it is creating an
%in-text heading with |\everypar|.
%    \begin{macrocode}
\ExplSyntaxOn
\cs_set:Npn \xsect:n #1 
  { 
  \l_tmpa_skip #1\relax
  \if_dim:w \l_tmpa_skip>0pt %WATCH better boolean
    \par \nobreak
    \vskip\l_tmpa_skip
    \@afterheading 
   \else:
    \@nobreakfalse
    \global\@noskipsectrue
    \tex_everypar:D 
      {\if@noskipsec
          \global\@noskipsecfalse%resets switch
          {\setbox\z@\lastbox}
          \tex_clubpenalty:D\@M
          \group_begin:
            \parindent0pt\tcbox[size=minimal,
                    nobeforeafter,
                    box~align=base]{\bfseries \@svsechd }%} 
          \group_end:
          \tex_unskip:D
          \l_tmpa_skip #1\relax
          \hskip -\l_tmpa_skip
        \else
          \tex_clubpenalty:D \@clubpenalty
          \tex_everypar:D {\tikzi[every]}
        \fi
      }
  \fi:
  \tex_ignorespaces:D
  }

\cs_set:Npn \after_block:n #1 
 {
   \par \nobreak
    \vskip\l_tmpa_skip
    \@afterheading
 }    
  
  
\ExplSyntaxOff
 \def\@afterheading{%
 \@nobreaktrue
 \everypar{%
   \if@nobreak
     \@nobreakfalse
     \clubpenalty \@M
     \if@afterindent 
     \else
      {\setbox\z@\lastbox}%
     %\tikzi[everypar]
     \fi
 \else
 \clubpenalty \@clubpenalty
 \everypar{
   %\tikzi[cleared]
 }%
 \fi}
 }  
%    \end{macrocode}
%
% 
% When LaTeX is typesetting the section number it calls |\@seccntformat|
% to use it when typsetting a section heading number. This is common for
% all the subsectioning commands. We modify it based on code from \pkgname{sectsty} in order
% to generalize it.
% 
% We first check if \meta{section}|@cntformat| is defined and then we redirect
% to specific section level command.
%
% \begin{docCommand} {@seccntformat} {\marg{section name}}
%  This is a \latexe kernel factory command that produces |thesection| etc.
%  In the kernel it only takes a generic value, where we have 
%  \refCom{section_number_after_tl}.
%  We modify to enable adjustable values for all sectioning commands. 
% \end{docCommand}
%
%    \begin{macrocode}
\ExplSyntaxOn
 \cs_gset:Npn \@seccntformat #1
 {
  \@ifundefined{#1@cntformat}%
  {\csname the#1\endcsname\section_number_after_tl}% default
  {\csname #1_cntformat\endcsname}% individual control
 }
%    \end{macrocode}
%
% \begin{docCommand}{section_number_after_tl} { \meta{void}}
%  This function and its siblings are auxiliary functions.
% \end{docCommand}
%
%    \begin{macrocode}
\tl_set:Nn  \section_number_after_tl{\quad}%default value only space
\tl_set:Nn  \subsection_number_after_tl{\quad}%default value only space
\tl_set:Nn  \subsubsection_number_after_tl{\quad}%default value only space
\tl_set:Nn  \l_phd_paragraph_number_after_tl{\quad}%default value only space
\tl_set:Nn  \subparagraph_number_after_tl{\quad}%default value only space
%
\cs_set:Npn \section_cntformat{\thesection\section_number_after_tl}
\cs_set:Npn \subsection_cntformat{\thesubsection\subsection_number_after_tl}
\cs_set:Npn \subsubsection_cntformat{\thesubsubsection\subsubsection_number_after_tl}
\cs_set:Npn \paragraph_cntformat {\theparagraph\l_phd_paragraph_number_after_tl }
\cs_set:Npn \subparagraph_cntformat {\thesubparagraph\subparagraph_number_after_tl }
\ExplSyntaxOff
%    \end{macrocode}
% 
%    \begin{macrocode}
\ExplSyntaxOn


  \renewcommand\section{%
    \@startsection{section}%
      {1}%level check this conflicts with source2e
      
      {\l_phd_section_indent_tl}%indent#2
      
      {\l_phd_section_before_skip_tl}%before skip#3
      
      {\l_phd_section_after_skip_tl}% after skip#4
      
      {% 
        \setfont@cx
        {\l_phd_section_fontweight_tl}%
        {\l_phd_section_fontfamily_tl}
        {\l_phd_section_fontsize_tl}
        {\l_phd_section_fontshape_tl}%
          \expandafter\setfontparam@cx\l_phd_section_align_tl;%
          \color{\l_phd_section_color_tl}%5
      }
 }%


\ExplSyntaxOff
%    \end{macrocode}
%   
%     \begin{macrocode}
\ExplSyntaxOn
%\if@ltxcompat
%\else
  \renewcommand\subsection
    {
      \@startsection{subsection}
        {1}
        {\l_phd_subsection_indent_tl}%indent
        {\l_phd_subsection_before_skip_tl}%before skip#3
        {\l_phd_subsection_after_skip_tl}% after skip#4
      
        { 
        \setfont@cx
          {\l_phd_subsection_fontweight_tl}
          {\l_phd_subsection_fontfamily_tl}
          {\l_phd_subsection_fontsize_tl}
          {\l_phd_subsection_fontshape_tl}
         % \expandafter\setfontparam@cx\l_phd_subsection_align_tl;%
          \color{\l_phd_subsection_color_tl}%5
        }
   }
%\fi
\ExplSyntaxOff
%\renewcommand\subsection{\@startsection{subsection}{2}{\z@}%
%                                     {-3.25ex\@plus -1ex \@minus -.2ex}%
%                                     {1.5ex \@plus .2ex}%
%                                     {\normalfont\large\bfseries\raggedright}}
%\fi
%
%    \end{macrocode}
%
%  The |subsubsection|  keys need to be activated with a renew command.
%   \begin{macrocode}
\ExplSyntaxOn
\renewcommand{\subsubsection}
{
  \@startsection{subsubsection}%
    {3}%level
    {\l_phd_subsubsection_indent_tl}%indent
    {\l_phd_subsubsection_before_skip_tl}
    {\l_phd_subsubsection_after_skip_tl}
    {\setfont@cx
    {\l_phd_subsubsection_fontweight_tl }
    {\l_phd_subsubsection_fontfamily_tl}
    {\l_phd_subsubsection_fontsize_tl}
    {\l_phd_subsubsection_fontshape_tl}
      \expandafter\setfontparam@cx
      \l_phd_subsubsection_align_tl;
      \color{\l_phd_subsubsection_color_tl}
    }
}       
\ExplSyntaxOff
%    \end{macrocode}
% 

%
% \subsection{Paragraphs}
%
%  We now deal with paragraphs and subparagraphs, normally termed `runin’ heads, as they produce
%  headings that are inlined with the text that follows. We add hooks, so that later the key mechanism
%  can be used to pick-up values. Although they are termed runins, there is no issue to display
%  them as block.
% 
% {macro}{renewparagraph}
%    \begin{macrocode}
\ExplSyntaxOn
%\if@ltxcompat
%\renewcommand\paragraph{\@startsection{paragraph}{4}{\z@}%
%                                    {3.25ex \@plus1ex \@minus.2ex}%
%                                    {-1em}%
%                                    {\normalfont\normalsize\bfseries}}
%\else
%\if@ltxcompat
%\else
  \renewcommand\paragraph{%
     \@startsection{paragraph}%
     {4}%level
     {\l_phd_paragraph_indent_tl}%indent
     {\l_phd_paragraph_before_skip_tl}%
     {\l_phd_paragraph_after_skip_tl}%
     {\setfont@cx
      {\l_phd_paragraph_fontweight_tl}%
     {\l_phd_paragraph_fontfamily_tl}
     {\l_phd_paragraph_fontsize_tl}
     {\l_phd_paragraph_fontshape_tl}%
     \expandafter\setfontparam@cx\l_phd_paragraph_align_tl;%
         \color{\l_ph_paragraph_color_tl}%
     }%
 }
%\fi
\ExplSyntaxOff
%    \end{macrocode}
%

% There is a feature in the standard \latexe classes that a subparagraph is indented
% by the value of \cs{parindent}. This also features in memoir but is absent in the
% KOMA classes. In our defaults we follow the European norm.
%    \begin{macrocode}
\ExplSyntaxOn  
\if@ltxcompat
%\renewcommand\subparagraph{\@startsection{subparagraph}{5}{0pt}%
%                                       {2ex}%
%                                       {-1em}%
%                                      {\normalfont\normalsize\bfseries}}
\renewcommand\subparagraph
      {
         \@startsection{subparagraph}
         {5}%level
         {\l_phd_subparagraph_indent_tl}%indent
         {\l_phd_subparagraph_before_skip_tl}
         {\l_phd_subparagraph_after_skip_tl}
         {
           \setfont@cx
           {\l_phd_subparagraph_fontweight_tl}
           {\l_phd_subparagraph_fontfamily_tl}
           {\l_phd_subparagraph_fontsize_tl}
           {\l_phd_subparagraph_fontshape_tl}%
           \expandafter\setfontparam@cx
             \l_phd_subparagraph_align_tl;
           \color{\l_ph_subparagraph_color_tl}
         }
       } 

\fi
\ExplSyntaxOff
%    \end{macrocode}
%

% 
% \cxset{section numbering=arabic}
% 
%
% \chapter{Default Settings}
%
% Setting default values
%
%
%    \begin{macrocode} 
\cxset
  { section name                   = SECTION,
    section format                 = block,
    section align                  = right,
    section font-size              = LARGE,
    section font-weight            = mdseries,
    section font-family            = sffamily,
    section font-shape             = upshape,
    section color                  = black,
    section numbering prefix       = \thechapter.,
    section numbering suffix       =,
    section numbering              = arabic,
    section indent                 = 0pt,
    section beforeskip             = -3ex,
    section afterskip              = 10pt,
    section afterindent            = off,
    section number after           = \quad,
%    
    section arc                    = 3pt,
    section background-color       = white,
    section afterindent            = off, 
    section grow left              = 0mm,
    section grow right             = 0mm,
    section rounded corners        = northeast,
%    
    section border-left-width      = 0pt,
    section border-right-width     = 0pt,
    section border-top-width       = 2pt,
    section border-bottom-width    = 2pt,
%
    section padding-left-width     = 0pt,
    section padding-right-width    = 10pt,
    section padding-top-width      = 2pt,
    section padding-bottom-width   = 2pt,
  }  
%    \end{macrocode} 
%    \begin{macrocode}
\cxset
  { 
    subsection name                = Subsection,
    subsection format              = block, 
    subsection font-size           = large,  
    subsection font-weight         = bfseries,
    subsection font-family         = rmfamily,
    subsection font-shape          = upshape,
    subsection color               = spot,
    subsection numbering           = arabic,
    subsection align               = raggedleft,
    subsection beforeskip          = -3.25ex\@plus -1ex \@minus -.2ex,
    subsection afterskip           = 1.5ex \@plus .2ex,
    subsection numbering prefix    = \thesection.,
    subsection indent              = 0pt,
    subsection number after        = 0pt,
    subsection background-color    = white,
%    
    subsection border-left-width      = 0pt,
    subsection border-right-width     = 0pt,
    subsection border-top-width       = 5pt,
    subsection border-bottom-width    = 5pt,
%
    subsection padding-left-width     = 0pt,
    subsection padding-right-width    = 0pt,
    subsection padding-top-width      = 20pt,
    subsection padding-bottom-width   = 20pt,    
  }
%    \end{macrocode}
%    \begin{macrocode}
\cxset
  { 
    subsubsection name                    = subsubsection,
    subsubsection format                  = block,  
    subsubsection background-color        = orange!30, 
    subsubsection font-family             = serif, 
    subsubsection font-size               = LARGE,
    subsubsection font-weight             = bfseries,
    subsubsection font-family             = tiresias,
    subsubsection font-shape              = itshape,
    subsubsection color                   = spot,
    subsubsection number prefix           = (,
    subsubsection number suffix           = ),
    subsubsection numbering               = arabic,
    subsubsection indent                  = 0pt,
    subsubsection beforeskip              = -3.25ex\@plus -1ex \@minus -.2ex,
    subsubsection afterskip               = 1.5ex \@plus .2ex,
    subsubsection align                   = flushleft,
    subsubsection number after     =,
%    
    subsubsection border-left-width       = 0pt,
    subsubsection border-right-width      = 0pt,
    subsubsection border-top-width        = 2pt,
    subsubsection border-bottom-width     = 0pt,
%
    subsubsection padding-left-width     = 0pt,
    subsubsection padding-right-width    = 0pt,
    subsubsection padding-top-width      = 20pt,
    subsubsection padding-bottom-width   = 20pt,    
  }
%    \end{macrocode}
%
%    \begin{macrocode}
% paragraph
\cxset
  {
    paragraph name                = paragraph,
    paragraph format              = inline, 
    paragraph name                = paragraph,
    paragraph font-size           = large,
    paragraph font-weight         = bfseries,
    paragraph font-family         = rmfamily,
    paragraph font-shape          = upshape,
    paragraph numbering           = alpha,
    paragraph align               = flushleft,
    paragraph beforeskip          = 3.25ex plus1ex minus.2ex,
    paragraph afterskip           = -1em,
    paragraph indent              = 0pt,
    paragraph number after        = \quad,
    paragraph color               = spot,
    paragraph background-color    = white,
  }
%    \end{macrocode}
%    \begin{macrocode}    
\cxset
  {
    subparagraph name             = subparagraph,
    subparagraph format           = inline,
    subparagraph name             = subparagraph, 
    subparagraph font-size        = large,
    subparagraph font-weight      = bfseries,
    subparagraph font-family      = rmfamily,
    subparagraph font-shape       = upshape,
    subparagraph color            = spot!30,
    subparagraph numbering        = none,
    subparagraph align            = flushleft,
    subparagraph beforeskip       = 3.25ex plus1ex minus .2ex,
    subparagraph afterskip        = -1em,
    subparagraph indent           = 0pt,
    subparagraph number after     = ,
  }
%    \end{macrocode}
%
% \chapter[test]{Tests}
%
%  We prepare a number of tests to verify that all settings work as advertized.
%
% \begin{docCommand}{testsections} {\meta{void}}
%  In honor of Barbara Beeton all testing commands are in lowercase, but we also provide
%  them in mixed case for the rest of the crowd.
% \end{docCommand}
%    \begin{macrocode}
\ExplSyntaxOn
\cs_set:Npn \testsections 
  {
    \section{Sections}
    \lorem\par
    \subsection{Subsections}
    \lorem\par
    \subsubsection{Subsubsections}
    \lorem\par
    \paragraph {Paragraph}
  }
  
\cs_set_eq:NN \TestSections\testsections  
\ExplSyntaxOff
%    \end{macrocode}
%
% \testsections
%
% \lorem |thesection| \par.
%
% \section{Sections}
%
% \lorem
% \subsection{Subsection}
%
% \lorem
%
% \subsubsection{Subsubsection}
%
% \lorem
%
% 
% \paragraph{Block paragraph} 
% 
% \lorem
% 
% \subparagraph{Subparagraph} 
% 
% \lorem
%
% \lorem

% \subparagraph{Subparagraph} 
%\lorem
%
% \cxset{section format=display}
%
% \section{This is a displayed section}
% \lorem

% \paragraph{This is a paragraph}


% \cxset{section format=display}
% \lorem\lorem
%
% \section{This is a displayed section}
%
% \subparagraph{This is a subparagraph}
%
% \cxset{hang/.style = {
%        section format=hang,
%        subsection format=hang,
%^^A     subsubsection format=hang,
%      }}
% \cxset{hang}
% \section{This is a hanged section}
% \lorem\lorem

% \subsection{This is a hanged subsection}
% \lorem\lorem
%
% \subsubsection{This is a hanged subsubsection}
% \lorem\lorem
%
% \paragraph{This is a hanged paragraph.}
% \ExplSyntaxOn
% \lorem
% \clist_new:N \test_numbering_clist 
% \clist_gset:Nn \test_numbering_clist {arabic,roman,Roman,words,Words,WORDS,Alpha,alpha}
% \clist_map_inline:Nn \test_numbering_clist
%  {
%    \cxset{section~numbering=#1,
%           section~format=block,
%           section~numbering~prefix=,
%           section~align=center,
%           section~background-color=white}
%    \section{Section~set~in~#1}
%    \lorem\par
%  }  
% 
% \clist_map_inline:Nn \test_numbering_clist
%  {
%    \cxset{subsection~numbering=#1,
%           subsection~format=block,
%           subsection~numbering~prefix=,
%           subsection~align=center,
%           subsection~border-top-width=0pt,
%           subsection~border-bottom-width=0pt}
%   \subsection{Subsection~set~in~#1}
%    \lorem\par
%  } 
% 
% \chapter{Subsubsection~Tests}
% 
% \clist_map_inline:Nn \test_numbering_clist
%  {
%    \cxset{subsubsection~numbering=#1,
%           subsubsection~format=block,
%           subsubsection~number~prefix=(,
%           subsubsection~align=left}
%    \subsubsection{Subsubsection~set~in~#1}
%    \lorem\par
%  }  
% \chapter{Paragraph~Tests}
% 
% \clist_map_inline:Nn \test_numbering_clist
%  {
%    \cxset{paragraph~format=inline,
%           paragraph~numbering=#1,
%           paragraph~number~prefix=(,
%           paragraph~number~suffix=),
%           }
%    \paragraph{Paragraph~set~in~#1}
%    \lorem\par
%  }  

% \chapter{Subparagraph~Tests}
% 
% \clist_map_inline:Nn \test_numbering_clist
%  {
%    \cxset{subparagraph~numbering=#1,
%           subparagraph~number~prefix=(,
%           subparagraph~number~suffix=),
%           }
%    \subparagraph{Subparagraph~set~in~#1}
%    \lorem\par
%  }  
% \ExplSyntaxOff
%
% \chapter {Test Formatters}

% \cxset{section format = block,
%        section font-size=small,
%        section font-weight=normal,
%        section font-shape=italic,
%        section numbering = arabic}

% \section{Block section} 
%
%\lorem\lorem
%
%
% \section{inline section} \lorem
%
% \cxset{section arc=5mm,
%        section format=block,
%        section beforeskip=-3.25ex,
%        section afterskip=3.25ex,
%        section background-color=spot!15,
%        section title margin-top=0pt,
% }

% \section{block section}\lorem
% \cxset{
%        section arc                = 5mm,
%        section format=block,
%        section beforeskip         = -3.25ex,
%        section afterskip          = 3.25ex,
%        section background-color   = white,
%        section title margin-top   = 10pt,
%        section title width        = 8cm,
%        section border-left-width  = 2pt,
%        section afterindent        = on,
%        section name               = Section,
%        section background-color   = spot!30,
%        section font-size          = huge,
%        section font-weight        = bfseries,  
%        section font-shape         = normal,
%        section font-family        = rmfamily,
%        section border-right-width = 2pt,
% }
%
% \section{display section}\lorem
% \lorem\lorem
%
% \ExplSyntaxOn
% \makeatletter
% \l_phd_section_title_width_dim\\
% \afterindent@cx\\
% \l_phd_section_background_color_tl\\
% \l_phd_section_align_tl\\
% \meaning\l_phd_subsection_align_tl\\
% \makeatother
% \ExplSyntaxOff
%</LSECT>
\endinput
