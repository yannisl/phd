\documentclass{article}
\usepackage{expl3,xparse,phd-lorems}
\usepackage{fontspec}

\begin{document}

\begingroup
\makeatletter
\@ifundefined{arabicfont}{%
  \newfontfamily\arabicfont{Amiri}}{}
\newif\if@RTL
  
\if@RTL\raggedleft\else\raggedright\fi
\makeatother


\arabicfont\linedir TRT
\raggedleft% 
{\linedir TLT\relax \section{Test}}%
\lorem


 للُّغَة العَرَبِيّة هي أكثر اللغات تحدثاً ونطقاً ضمن مجموعة اللغات السامية، وإحدى أكثر اللغات انتشاراً في العالم، يتحدثها أكثر من 467 مليون نسمة،[4](1) ويتوزع متحدثوها في الوطن العربي، بالإضافة إلى العديد من المناطق الأخرى المجاورة كالأحواز وتركيا وتشاد ومالي والسنغال وإرتيريا وإثيوبيا وجنوب السودان وإيران. اللغة العربية ذات أهمية قصوى لدى المسلمين، فهي عندهم لغة مقدسة إذ أنها لغة القرآن، وهي لغة الصلاة وأساسية في القيام بالعديد من العبادات والشعائر الإسلامية.[5][6] العربية هي أيضاً لغة شعائرية رئيسية لدى عدد من الكنائس المسيحية في الوطن العربي، كما كتبت بها كثير من أهم الأعمال الدينية والفكرية اليهودية في العصور الوسطى. ارتفعت مكانة اللغة العربية إثر انتشار الإسلام بين الدول إذ أصبحت لغة السياسة والعلم والأدب لقرون طويلة في الأراضي التي حكمها المسلمون. وللغة العربية تأثيراً مباشراً وغير مباشر على كثير من اللغات الأخرى في العالم الإسلامي، كالتركية والفارسية والأمازيغية والكردية والأردية والماليزية والإندونيسية والألبانية وبعض اللغات الإفريقية الأخرى مثل الهاوسا والسواحيلية والتجرية والأمهرية والصومالية، وبعض اللغات الأوروبية وخاصةً المتوسطية كالإسبانية والبرتغالية والمالطية والصقلية. كما أنها تُدرَّس بشكل رسمي أو غير رسمي في الدول الإسلامية والدول الإفريقية المحاذية للوطن العربي



\endgroup


\linedir TLT\lorem
\edef\primlist{\directlua{tex.print(tex.primitives())}}
\makeatletter
\def\mboxed#1;{%
    \@mboxed#1 \@empty
}
\def\@mboxed#1 #2{%
   \leavevmode\fbox{#1}\space\par  % fbox here to have a visual test
   \ifx #2\@empty\else
    \expandafter\@mboxed
   \fi
   #2%
}

\mboxed\primlist ;
\makeatother
\ExplSyntaxOff
\mathdir TRT 

\begin{equation}
a^2 + z^3 = y
\end{equation}



\end{document}