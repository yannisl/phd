\newacro{SPQR}{Senatus Populusque Romanus}
\newacro{URL}{uniform resource locator}
\newacro{OUP}{Oxford University Press}

\chapter{Acronyms and Abbreviations}


In this section we will discuss the use and typesetting of symbols, abbreviations and acronyms. The |phd| package loads a number of packages and also offers a number of commands in managing symbols, abbreviations and acronyms. The main package we use to manage acronyms is \pkgname{acronym} \cite{acronym}. We also use some build-in commands for abbreviations and to assist in enforcing in-house style guides.  

\section{General Principles}

Abbreviations and symbols represent, through a variety of means, a
shortened form of a word or words. Abbreviations fall into three categories:
only the first of these is technically an abbreviation, though the
term loosely covers them all, and guidelines for their use overlap.

\begin{itemize}
\item \textit{Abbreviations} are formed by omitting the end of a word or words (VCR, lbw, Lieut.).
\item \textit{Contractions} are formed by omitting the middle of a word or words (I've,
mustn't, ne'er-do-well
\item \textit{Acronyms} are formed from the initial letters of words (SALT, Nazi, radar), the results being pronounced as words themselves.
\end{itemize}


\section{Acronyms}

An acronym is distinguished from other abbreviated forms by being a series of letters or 
syllables pronounced as a complete word: \textsc{NATO}
and UEFA are acronyms, but MI6 and BBC are not. Acronyms take no
points, whether all in caps (NAAFI, SALT, WASP), in initial capitals with
upper and lower case (Aga, Fiat, Sogat), or entirely in lower case (derv,
laser, scuba). Since they perform as words they can begin sentences, with
lower-case forms being capitalized normally, such as \textit{Laser treatment}. 

Any all-capital proper-name acronym is, in some house styles, fashioned
with a single initial capital if it exceeds four letters (Basic, Unesco, Unicef).

The \textit{Oxford Guide} suggests that editors should avoid this rule, useful though it is, where the result runs
against the common practice of a discipline (CARPE, SSHRCC, WYSIWYG), or where similar terms would be treated dissimilarly based on length alone.


Acronyms are not new language inventions, they were used well back in antiquity.  For example, the official name for the Roman Empire, and the Republic before it, was abbreviated as \ac{SPQR}. Inscriptions dating from antiquity, both on stone and on coins, use a lot of abbreviations and acronyms to save room and work. For example, Roman first names, of which there was only a small set, were almost always abbreviated. Common terms were abbreviated too, such as writing just "F" for "filius", meaning "son of", a very common part of memorial inscriptions mentioning people. Grammatical markers were abbreviated or left out entirely if they could be inferred from the rest of the text. \ac{SPQR}

So called \textit{Nomina Sacra} were used in many Greek biblical manuscripts. The common words "God" (Θεός), "Jesus" (Ιησούς), "Christ" (Χριστός), and some others, would be abbreviated by their first and last letters, marked with an overline. This was just one of many kinds of conventional scribal abbreviation, used to reduce the time-consuming workload of the scribe and save on valuable writing materials. The same convention is still commonly used in the inscriptions on religious icons and the stamps used to mark the eucharistic bread in eastern churches.

The early Christians in Rome, most of whom were Greek rather than Latin speakers, used the image of a fish as a symbol for Jesus in part because of an acronym—fish in Greek is ΙΧΘΥΣ (ichthys), which was said to stand for Ἰησοῦς Χριστός Θεοῦ Υἱός Σωτήρ (Iesous CHristos THeou hUios Soter: Jesus Christ, God's Son, Savior). Evidence of this interpretation dates from the 2nd and 3rd centuries and is preserved in the catacombs of Rome. And for centuries, the Church has used the inscription INRI over the crucifix, which stands for the Latin \textit{Iesus Nazarenus Rex Iudaeorum} (``Jesus the Nazarene, King of the Jews'').

The Hebrew language has a long history of formation of acronyms pronounced as words, stretching back many centuries. The Hebrew Bible ("Old Testament") is known as "Tanakh", an acronym composed from the Hebrew initial letters of its three major sections: Torah (five books of Moses), Nevi'im (prophets), and K'tuvim (writings). Many rabbinical figures from the Middle Ages onward are referred to in rabbinical literature by their pronounced acronyms, such as Rambam (aka Maimonides, from the initial letters of his full Hebrew name (Rabbi Moshe ben Maimon) and Rashi (Rabbi Shlomo Yitzkhaki).


The main package we load to assist with acronyms and abbreviations is |acronym|, developed by Tobias Oetiker \citeyearpar{acronym}. The package offers a number of useful commands to help with managing acronyms and to produce lists of acronyms and abbreviations. The package works by offering commands that you use to define an acronym as well as an environment serving the same purpose.

\begin{docCommand}{ac}{\meta{short version of the acronym}}
    To enter an acronym inside the text, use the |\ac{NATO}|
\end{docCommand}
    
    \begin{quote}
     |\ac{|\meta{acronym}|}|
    \end{quote}
    command. The first time you use an acronym, the full name of the
    acronym along with the acronym in brackets will be printed. If you
    specify the |footnote| option while loading the package, the full
    name of the acronym is printed as a footnote.
    The next time you access the acronym only the acronym will
    be printed.

\section{Symbols}

Symbols or signs, are a shorthand notation signifying a word or concept, and are frequent features of scientific and technical writing. The distinction between abbreviation and symbol may be blurred when, lie an abbreviation, a symbol is derived directly from a word or words (\textit{Ag} from \textit{argentum}, \textit{Pa} from \textit{pascal}, \textit{U} from \textit{uranium}), and in setting they are often treated similarly. Unlike abbreviations, however, symbols never take points, even if a single letter, or used alone or in conjuction with figures or words: \textit{F}  for \textit{false}, \textit{fluorine}, \textit{phenylalanine}.

Abstract, purely typographical symbols follow similar rules, being either close up (\ding{38}\ding{33}\ding{43})or spaced (\ding{38} \ding{33} \ding{43}). In \latex you can insert a non-breaking space if you want or a |hairsp|.

|\ding{38}~\ding{33}~\ding{43}|

Personally for the example I would prefer not to split them and the non-breaking space is a better option in this instance.

Symbols' uses can differ between disciplines. For example, in philological
works an asterisk (\textasteriskcentered) prefixed to a word signifies a reconstructed
form; in grammatical works it signifies an incorrect or nonstandard
form. A dagger (\textdagger) may signify an obsolete word, or 'deceased' when
placed before a person's name (this convention should be used only in
relation to Christians). In German a double dagger (\textdaggerdbl) follows the name
and signifies `killed in battle', \emph{gefallen} or  \gtrsymKilled.

A full set of these genealogical symbols can be found in the \pkgname{genealogytree} package  developed by  \person{Thomas F. Sturm} \citeyearpar{genealogytree} and are shown below,

\begin{scriptexample}[]{}{}
\textsl{\gtrSymbolsFullLegend[english]}
\end{scriptexample}

The package is loaded automatically by the |phd| package. Besides these symbols numerous other symbols
are loaded and described in the Chapter for Symbols.
\section{Abbreviations}

\subsection{Time Designations}

Most style guides recommend that you spell out the names of the months in the text but abbreviate them in the list  of works cited, except for May, June and July \cite{MLA}. The same manual suggests that words denoting units of time are also spelled out in the text (\textit{second}, \textit{minute}, \textit{week}, \textit{month}, \textit{year}, \textit{century}, some time designations are used only in the abbreviated form (\textit{a.m., p.m., AD, BC, BC, BCE, CE}). The |phd| package provides some assistance by loading the \pkg{datetime} package; more information on using it and of date and time formatting as well as calculations in Handling Dates and Time can be found in Pages~\pageref{ch:dates}--\pageref{datesend}.
\medskip

\begin{longtable}{lp{8cm}}
AD & after the birth of Christ (from the Latin \textit{anno Domini} `in the year of the Lord'; used before numerals ["\AD 14"] and after references for centuries ["twelfth century \AD"]\\
a.m. & before noon (from the Latin \textit{ante meridiem})\\
Apr. &April\\
Aug. &August\\
BC   &before Christ (used after numerals [``18 BC''] and referenced to centuries [``sixth century BC'']\\
BCE &before the common era (used after numerals and references to centuries)\\
CE  &common era (used after numerals and references to centuries)\\
cent. &century\\
Dec. &December\\
Feb  &February\\
Fri. &Friday\\
hr. &hour\\
Jan. &January\\
Mar. &March\\
min. &minute\\
mo. &month\\
Mon. &Monday\\
Nov. &November\\
Oct. &October\\
p.m. &after noon (from the Latin \textit{post meridiem})\\
Sat. &Saturday\\
sec. &second\\
Sept.&September\\
Sun. &Sunday\\
Thurs. &Thursday\\
Tues. &Tuesday\\
Wed. &Wednesday\\
wk. &week\\
yr. &year\\
\end{longtable}

Tables such as the one above, if not provided by the Publisher, can be very helpful, if you develop them on your own and refer back to them for consistency.

\subsection{Geographic Names}

\subsection{Common Scholarly Abbreviations and Reference Words}

\begin{figure}[tp]
\centering
\fbox{\includegraphics[width=1.0\textwidth]{./images/abbreviations.pdf}}
\caption{A typical Abbreviations page. This has been extracted from \protect\cite{bacchae}.}
\end{figure}


\section{The indefinite article with abbreviations}

The choice between \emph{a} and \emph{an} before an abbreviation depends on pronunciation,
not spelling. Use a before abbreviations beginning with a
consonant sound, including an aspirated h and a vowel pronounced with the sound of w or y.

\begin{scriptexample}
a BA degree a KLM flight a BBC announcer
a Herts, address a hilac demonstration a YMCA bed
a SEATO delegate a U-boat captain a UNICEF card
\end{scriptexample}

Use \emph{an} before abbreviations beginning with a vowel sound, including
unaspirated \emph{h}:

an AB degree an MCC ruling an FA cup match
an H-bomb an IOU an MP
an MA an RAC badge an SOS signal

This distinction assumes the reader will pronounce the sounds of the
letters, rather than the words they stand for (a Football Association cup
match, a hydrogen bomb). MS for manuscript is normally pronounced as
the full word, manuscript, and so takes a; MS for multiple sclerosis is
often pronounced em-ess, and so takes an. Likewise 'R.' for rabbi is
pronounced as rabbi ('a R. Shimon wrote'), but 'R' for a restricted classification
is normally pronounced as arr ('an R film').

The difference between sounding and spelling letters is equally important
when choosing the article for abbreviations that are acronyms and
for those that are not: a NASA launch but an NAMB award. The same holds
for names of symbols, which can vary: in America a hash symbol (\#) is a
'number sign' or, more formally, an \textit{octothorp}; in linguistic use an
asterisk may be called a 'star' and in mathematics an exclamation
mark called a 'factorial', 'shriek', or 'bang', so the correct forms are a *
and a ! rather than an * and an !. 

As abbreviated terms enter the
language there can be a period of confusion as to how they are pronounced:
in computing, for example, \ac{URL} is
pronounced by some as an abbreviation (you-are-ell) and others as an
acronym (earl), with the result that some write it as a URL and others as
an URL. Until a single pronunciation becomes generally accepted, the
best practice is simply to ensure consistency within a given work.

\section{Latin abbreviations}
\normalfont

Do not confuse 'e.g.' (\emph{exempli gratia}), meaning \enquote{for example}, with \enquote{i.e.}
{id est), meaning 'that is'. Compare hand tools, e.g. hammer and screwdriver
with hand tools, \ie those able to be held in the user's hands. Print both lower-case roman, with two points and no spaces, and preceded by a
comma. In OUP style 'e.g.' and 'i.e.' are not followed by commas, to avoid
double punctuation; commas are often used in US practice.

Although many people employ 'e.g.' and 'i.e.' quite naturally in speech
as well as writing, prefer 'for example' and 'that is' in running text.
(Since 'e.g.' and 'i.e.' are prone to overuse in text, this convention helps
to limit their profusion.) Conversely, adopt 'e.g.' and 'i.e.' within parentheses
or notes, since abbreviations are preferred there. A sentence in text cannot begin with 'e.g.' or 'i.e.'; however, a note can, in which case
they—exceptionally—remain lower case. The \textit{Oxford Guide} gives an example of exception to the rule

The package offers two commands:

\begin{verbatim}
\newcommand{\ie}{\textit{i.\hairsp{}e.}\xspace}
\newcommand{\eg}{\textit{e.\hairsp{}g.}\xspace}
\end{verbatim}

The commands handle the spacing and if they are to be in italics or not. Renew the commands to set the style you want.

\section{Units}
\label{units}

Most users of \latex will have a need for specifying units in  mathematical or text contexts. We load the \pkgname{siunitx} package. The package was developed by Joseph Wright\cite{siunitx}. The correct application of units of measurement is very important in technical applications. For this reason, carefully-crafted definitions of a coherent units system have been
laid down by the \textit{Conférence Générale des Poids} et Mesures (CGPM): this has resulted in
the \textit{Système International d’Unités} (SI). At the same time, typographic conventions for
correctly displaying both numbers and units exist to ensure that no loss of meaning
occurs in printed matter.

|siunitx| aims to provide a unified method for \latex users to typeset numbers and
units correctly and easily. The design philosophy of |siunitx| is to follow the agreed rules
by default, but to allow variation through option settings. In this way, users can use
|siunitx| to follow the requirements of publishers, co-authors, universities, etc. without needing to alter the input at all.



\begin{ddanger}
Angles can be typeset using the \cs{ang} command.  The
 \meta{angle} can be given either as a decimal number or as a
 semi-colon separated list of degrees, minutes and seconds, which
 is called \enquote{arc format} in this document. The numbers which
 make up an angle are processed using the same system as other numbers.
\end{ddanger}
