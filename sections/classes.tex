\let\sidenote\footnote
\let\citep\footcite
\chapter{How to Develop your Own Class or Package}

\cxset {epigraph width=0.67\textwidth}

\epigraph{First there was one user and I took a lot of time to satisfy myself. Then I had 10 users, and a whole new level of difficulties arose. Then I had a hundred users and another level of things happened. I had a thousand users, I had ten thousand each of those were special phases in the development, important. I couldn't have gone with ten thousand until I'd done
it with a thousand. But each time a new wave of
changes came along, the idea was to have \tex get
better, and not get more diverse as it needed to handle
new things.}{Donald Knuth}

\parindent1em

\section{Introduction}


To \emph{make} a book is an interesting and somewhat involved process\footcite{town}. The text is set in type and printed on pages, the pages are gathered and folded into signatures and these are gathered and folded into signatures and these are then bound and covered. Many of the aspects of this process that has passed down to us by previous generations is discussed extensively in other sections of this book.  Class authors have to distill this knowledge in a set of typographical rules to be described in a class file. The first thing such an author must do is to describe the \emph{rationale} of developing such a class. The \docClass{octavo}\citep{octavo} class was developed to enable printing books in dimensions that follow traditional styles. The \citep{memoir}  class to offer a flexible system on which other classes could be based and so does \citep{koma}. The |tufte-book| and |tufte-handout| classes to provide a style that resembles those found in Tufte books. Many Universities offer \emph{Thesis} classes to standardize the way these are produced. Many of these Universities, translated the styles previously typed and the results are a typographical disaster, only mitigated by the ability to display beautiful mathematics. As these are printed on standard \emph{photocopy paper} one cannot do much with the layout. 
\section{What is a class?}
A class is simply a file with the extension \docExtension{cls} containg a set of macros. 
A class can load another class.
\section{Identifying your class}

The first thing a class must do is to identify any other formats it needs and to announce
its name. This is accomplished using the two commands 
\refCom{NeedsTeXFormat} and \refCom{ProvidesClass}.

The following example, delares the version of \LaTeXe\ that it requires and then
gives the class name. It can be found in the preable of most well written classes. You should also put some remarks to identify you as the author, the version number and other similar details. These are discussed in more detail in the next Chapter, where you will see how to automate documentation for your class.

\begin{teX}
\NeedsTeXFormat{LaTeX2e}[1994/06/01]
\ProvidesClass{myclass-book}[2010/12/11 v3.5.0 myclass-book]
\end{teX}

The above syntax must be followed exactly so that this information can be
used by \texttt{LoadClass} or \texttt{documentclass} (for classes) or \docAuxCommand{RequirePackage}
 or\cmd{usepackage} (for packages) to test that the release is not too old.
The whole of this $<release-info>$ information is displayed by \docAuxCommand{listfiles} and
should therefore not be too long.

\begin{teX}
% Load the common style elements
\input{myclass-common.def}
\end{teX}


Another command that can be used is \docAuxCommand{ProvidesFile}. 
This is similar to the two previous commands except that here the fullname,
including the extension, must be given. It is used for declaring any files other
than main class and package files.

This is useful, if you decide to have your main definitions in a separate file.

\section{Class Options}

Before we see in detail how to add options to a class, we need to review a package called
\pkgname{xkeyval}. Unless you are in the business of re-discovering wheels, this is an absolute must
for developing, readable and maintenable code and your class is to provide many options. 
\begin{teX}
\usepackage[textcolor=red,font=times]{mypack}
\end{teX}

Class options are best set by using booleans\docAuxCommand{newboolean}.

We first set a new boolean that we |name@myclass@afourpaper.| This is used using the package
\texttt{ifthen}\sidenote{The ifthen package was developed by 
David Carlisle, can be downloaded at \url{ http://www.ifi.uio.no/it/latex-links/ifthen.pdf }} 
Then we can |DecalareOptionX| and we set the boolean to default to true. If the user then types

myclass[a4paper]

The a4paper options will be set. This is a much better and concise way of defining options.
\cmd{newboolean}


\begin{teX}
\newboolean{@myclass@afourpaper}
\DeclareOptionX[myclass]<common>{a4paper}
  {
   \setboolean{@myclass@afourpaper}
   {true}
  }
\end{teX}
\medskip

Note that the command provide by \texttt{ifthen} \docAuxCommand{setboolean} takes true or false, as \#2, and sets \#1 accordingly. In the above code we set the option as true. 


It is much easier and most programmers use the \texttt{ifthen} package to check
for option booleans

\begin{teX}
\ifthenelse{\boolean{@myclass@afourpaper}}
  {\geometry{
        a4paper,
        left=24.8mm,
        top=27.4mm,
        headsep=2\baselineskip,
        textwidth=107mm,
        marginparsep=8.2mm,
        marginparwidth=49.4mm,
        textheight=49\baselineskip,
        headheight=\baselineskip
    }
  }
 {}
\end{teX}

\section{Set-up the font sizes}

LaTeX does not provide definitions of all the font-sizes. Unless you are
extending an existing class, this is one of the first tasks you need to 
do in your new class.

Normally class authors will define all the commonly defined size commands,
such as  \cmd{small}, \cmd{normalsize} and other similar commands.

In the example shown below, we first start by defining the \cmd{normalsize} font
size. In this book the \cmd{\normalsize}  is defined as 14pt. We also define the vertical
spaces that we need to have abovedisplay and belowdisplayskip. These are all very difficult to
remember and once you have something you are happy with, just copy from class to class
or even define a samll definition file to keep them all together.


{\fontfamily{phv}\selectfont Helvetica looks like this}
and {\fontencoding{OT1}\fontfamily{ppl} Palatino looks like this}.


 The user has access to a number of commands which change the size of
 the fount, relative to the `main' size used for the bulk of the text.


 These \cmd{size} commands issue a \cmd{@setfontsize}\index{Latex kernel!@setfontsize} 
 command.

\begin{teX}
  \@setfontsize\size\font-size{baselineskip} where:
\end{teX}



  \begin{description}
    \item {font-size} The absolute size of the fount to use from
        now on.
    \item{baselineskip} The normal value of \cmd{baselineskip}
        for the size of the fount selected. (The actual value will be
       % |\baselinestretch| * \meta{baselineskip}.)
    \end{description}

A number of commands, defined in the \LaTeX  kernel, shorten the
following  definitions and are used throughout. These are:

    \begin{center}
    \begin{tabular}{ll@{\qquad}ll@{\qquad}ll}
    \verb=\@vpt= & 5 & \verb=\@vipt= & 6 & \verb=\@viipt= & 7 \\
    \verb=\@viiipt= & 8 & \verb=\@ixpt= & 9 & \verb=\@xpt= & 10 \\
    \verb=\@xipt= & 10.95 & \verb=\@xiipt= & 12 & \verb=\@xivpt= & 14.4\\
    \ldots
    \end{tabular}
    \end{center}


\subsection{Setting up the normalsize}
 The user command to obtain the `main' size is \cmd{normalsize}. \LaTeX\
 uses \cmd{@normalsize} \index{Latex kernel!@normalsize} when referring to the main size and maintains this
 value even if \docAuxCommand{normalsize} is redefined. The \docAuxCommand{normalsize} macro also
  sets values for \cmd{abovedisplayskip}, \cmd{abovedisplayshortskip} and 
\cmd{belowdisplayshortskip}.



\begin{teX}
%%
% Set the font sizes and baselines to match Tufte's books
% normalsize
%%
\renewcommand\normalsize{%
   \@setfontsize\normalsize\@xpt{14}%
   \abovedisplayskip 10\p@ \@plus2\p@ \@minus5\p@
   \abovedisplayshortskip \z@ \@plus3\p@
   \belowdisplayshortskip 6\p@ \@plus3\p@ \@minus3\p@
   \belowdisplayskip \abovedisplayskip
   \let\@listi\@listI}

\normalbaselineskip=14pt
\normalsize
\end{teX}



\begin{teX}
\renewcommand\small{%
   \@setfontsize\small\@ixpt{12}%
   \abovedisplayskip 8.5\p@ \@plus3\p@ \@minus4\p@
   \abovedisplayshortskip \z@ \@plus2\p@
   \belowdisplayshortskip 4\p@ \@plus2\p@ \@minus2\p@
   \def\@listi{\leftmargin\leftmargini
               \topsep 4\p@ \@plus2\p@ \@minus2\p@
               \parsep 2\p@ \@plus\p@ \@minus\p@
               \itemsep \parsep}%
   \belowdisplayskip \abovedisplayskip
}
\renewcommand\footnotesize{%
   \@setfontsize\footnotesize\@viiipt{10}%
   \abovedisplayskip 6\p@ \@plus2\p@ \@minus4\p@
   \abovedisplayshortskip \z@ \@plus\p@
   \belowdisplayshortskip 3\p@ \@plus\p@ \@minus2\p@
   \def\@listi{\leftmargin\leftmargini
               \topsep 3\p@ \@plus\p@ \@minus\p@
               \parsep 2\p@ \@plus\p@ \@minus\p@
               \itemsep \parsep}%
   \belowdisplayskip \abovedisplayskip
}
\renewcommand\scriptsize{\@setfontsize\scriptsize\@viipt\@viiipt}
\renewcommand\tiny{\@setfontsize\tiny\@vpt\@vipt}
\renewcommand\large{\@setfontsize\large\@xipt{15}}
\renewcommand\Large{\@setfontsize\Large\@xiipt{16}}
\renewcommand\LARGE{\@setfontsize\LARGE\@xivpt{18}}
\renewcommand\huge{\@setfontsize\huge\@xxpt{30}}
\renewcommand\Huge{\@setfontsize\Huge{24}{36}}

%% Define a HUGE for fun
\newcommand\HUGE{\@setfontsize\Huge{38}{47}}  
\end{teX}


\section{Adjusting paragraph parameters}

 The parameters which control \TeX 's behaviour when typesetting
 paragraphs receive a bit of a tweak here. Contrary to the usual
 behaviour of modifying the grid with glue when difficulties are
 encountered with vertical space, here we shall try to counteract
 these tendencies and enforce as much as possible uniformity of the 
 grid of lines.

A good value for paragraph indentation is \texttt{parindent 0.5pt}, for vertical spacing between
paragraphs that are indented use 0pt. At this point if you are using any marginals it is a good idea
to allow hyphenation with the \docpkg{ragged2e} package. Since marginals use very narrow paragraphs you may
get a very funny looking marginal text. Using the package, adjustments can be made to hyphenate
the marginal text.

\begin{teXXX}
%%
% \RaggedRight allows hyphenation

\RequirePackage{ragged2e}
\setlength{\RaggedRightRightskip}{\z@ plus 0.08\hsize}
\setlength{\RaggedRightParindent}{1pc}

% Paragraph indentation and separation for normal text
\newcommand{\@tufte@reset@par}{%
  \setlength{\RaggedRightParindent}{1.0pc}%
  \setlength{\parindent}{1pc}%
  \setlength{\parskip}{0pt}%
}
\@tufte@reset@par

% Paragraph indentation and separation for marginal text
\newcommand{\@tufte@margin@par}{%
  \setlength{\RaggedRightParindent}{0.5pc}%
  \setlength{\parindent}{0.5pc}%
  \setlength{\parskip}{0pt}%
}
\end{teXXX}


\section{Formatting Chapters and Sections}

The section on Chapters etc, has more on this, but we will touch on it briefly.
Most recent class developerss use the \pkg{titlesec} and \pkg{titletoc} package to handle the complexity 
of these commands. With the |phd| package this is unecessary. 

\begin{teXXX}
\titleformat{\subsection}%
  [hang]% shape
  {\normalfont\large}% format applied to label+text removed \itshape
  {\thesubsection}% label
  {1em}% horizontal separation between label and title body
  {}% before the title body
  []% after the title body
\end{teXXX}

These are normally followed by the ``titlespacing" commands to define the space around these sections.

\begin{teXXX}
%% We set the titlespacing using the package titlesec and titletoc
%
\titlespacing*{\chapter}{0pt}{20pt}{40pt}
\titlespacing*{\section}{0pt}{3.5ex plus 1ex minus .2ex}{2.3ex plus .2ex}
\titlespacing*{\subsection}{0pt}{3.25ex plus 1ex minus .2ex}{1.5ex plus.2ex}
\end{teXXX}

\section{Adjusting the Index}

For classes representing books, the index is treated like a chapter whereas for others it is normally
treated like a section. Whatever your document ends up like, indices are best done in a multi-column environment.
One possibility is shown below, using the package "multcol".

\begin{teXXX}
\RequirePackage{multicol}
\renewenvironment{theindex}
  {\begin{fullwidth}%
    \small%
    \ifthenelse{\equal{\@tufte@class}{book}}%
      {\chapter{\indexname}}%
      {\section*{\indexname}}%
    \parskip0pt%
    \parindent0pt%
    \let\item\@idxitem%
    \begin{multicols}{3}%
  }
  {\end{multicols}%
    
\renewcommand\@idxitem{\par\hangindent 2em}
\renewcommand\subitem{\par\hangindent 3em\hspace*{1em}}
\renewcommand\subsubitem{
    \par\hangindent 4em\hspace*{2em}
}
\renewcommand\indexspace{
    \par\addvspace{
       1.0\baselineskip plus 0.5ex minus 0.2ex}\relax
    }%
%we now  swallow the letter heading in the index
\newcommand{\lettergroup}[1]{}

\end{teXXX}

The code, renews the "theindex" environment, with minor tweaks and defines it as a three column
layout at "fullwidth".

\section{Provide some hooks}
It is useful at the end of the class to allow for localization of the class
by importing a local file. This is easily achieved by checking if the file exists
and then loading it.  If there is a |myclass-book-local.sty|  file, load it.

\begin{teX}
\IfFileExists{myclass-book-local.tex}
  {input{myclass-book-local}
   \MyClassInfoNL{Loading myclass-book-local.tex}}
  {}
\end{teX}

If you intent to publish your class, you may also want to consider adding a hook for a patch-file.


\section{The final act of kindness to your users}
Many common classes, such as the |memoir| use such a tactic to avoid breaking old code.\index{IfFileExists}

\begin{teX}
 \IfFileExists{mypatch.sty}{%
 \RequirePackage{mypatch}}{}
\end{teX}


\parindent1em
\chapter{How to Package Your Class}

In the previous chapter we have outlined the main sections that you probably need
to define in your class. In the examples we have used we just typed the examples
as |example.cls| or |package.sty|.

In this chapter we will go over the packaging of the class
and automating the generation of user documentation, using the |doc| and \pkg{DocStrip}\footcite{docstrip}
programs in files with an extension |.dtx|. The DocStrip program is an amazing piece of code that was originally
created by Frank Mittelbach to accompany the |doc| package. The idea behind it was to remove comment lines
in order to reduce the execution time of the program. Having created the DocStrip program to remove comment lines from  programs it became feasible to do more than just strip comments.
Wouldn't it be nice to have a way to include parts of the code only when some
condition is set true? Wouldn't it be as nice to have the possibility to split the
source of a \tex program into several smaller files and combine them later into
one `executable'? Both these wishes have been implemented in the DocStrip program.



You should also be
familiar with ``LaTeX2e'' for Class and Package Writers”, which is available
from CTAN (\url{http://www.ctan.org}) and comes with most LaTeX2e" distributions
in a file called clsguide.dvi.\footcite{pakin2004}  Finally, you should know how to
install packages that are shipped as a \texttt{.dtx} file plus a \texttt{.ins} file.

style (.sty) file is primarily a collection of macro and
environment definitions. One or more style files (e.g., a main style file that
\cs{input}  or \cs{RequirePackages} multiple helper files) is called a package.
Packages are loaded into a document with \cs{usepackage}\marg{main .sty fille}.
In the rest of this document, we use the notation \meta{package} to represent
the name of your package.


Motivation The important parts of a package are the code, the documentation
of the code, and the user documentation. Using the \docpkg{Doc}  and
DocStrip programs, it’s possible to combine all three of these into a single,
documented LATEX(.dtx) file. The primary advantage of a .dtx file is that
it enables you to use arbitrary LATEX constructs to comment your code.
Hence, macros, environments, code stanzas, variables, and so forth can be
explained using tables, figures, mathematics, and font changes. Code can
be organized into sections using LATEX’s sectioning commands. Doc even
facilitates generating a unified index that indexes both macro definitions (in
the LATEX code) and macro descriptions (in the user documentation). 

This emphasis on writing verbose, nicely typeset comments for code—essentially
treating a program as a book that describes a set of algorithms—is known
as literate programming \cite{literate} and has been in use since the early days of \tex\ .

Furthermore,
this tutorial shows how to write a single file that serves as both documentation
and driver file, which is a more typical usage of the \texttt{Doc} system than
using separate files.

\subsection{The .ins file}

The first step in preparing a package for distribution is to write an installer
(|.ins|) file. An installer file extracts the code from a |.dtx| file, uses \pkg{docstrip}
to strip off the comments and documentation, and outputs a |.sty| file. The
good news is that a |.ins| file is typically fairly short and doesn’t change
significantly from one package to another.

\paragraph{License} The |ins| files usually start with comments specifying the copyright and license
information:

\begin{minted}{latex}
%%
%% Copyright (C) year by your name %%
%% This file may be distributed and/or modified under the
%% conditions of the LaTeX Project Public License, either
%% version 1.2 of this license or (at your option) any later
%% version. The latest version of this license is in:
%%
%% http://www.latex-project.org/lppl.txt
%%
%% and version 1.2 or later is part of all distributions of
%% LaTeX version 1999/12/01 or later.
%%
\end{minted}

The LATEX Project Public License (LPPL) is the license under which most
packages—and LATEX itself—are distributed. Of course, you can release your
package under any license you want; the LPPL is merely the most common
license for LATEX packages. The LPPL specifies that a user can do whatever
he wants with your package—including sell it and give you nothing in return.
The only restrictions are that he must give you credit for your work, and
he must change the name of the package if he modifies anything to avoid
versioning confusion.
The next step is to load DocStrip:

\begin{teXXX}
%%\input docstrip.tex
%%\keepsilent
\end{teXXX}



By default, DocStrip gives a line-by-line account of its activity. These messages
aren’t terribly useful, so most people turn them off, by using the command \docAuxCommand{keepsilent}:

\begin{teXXX}
\keepsilent
\end{teXXX}


A system administrator can specify the base directory under which all
TEX-related files should be installed, e.g., \texttt{/usr/share/texmf}. (See
\cmd{\BaseDirectory} in the DocStrip manual.) The |ins| file specifies where
its files should be installed relative to that. The following is typical:

\begin{teXXX}
\usedir{tex/latex/packagename}
\preamble
htexti \endpreamble
\end{teXXX}



The next step is to specify a preamble, which is a block of commentary that
will be written to the top of every generated file:

\begin{minted}{latex}
\preamble
----------------------------------------------------------------
phddoc --- A class to typeset LaTeX code.
E-mail: yannislaz@gmail.com
Released under the LaTeX Project Public License v1.3c or later
See http://www.latex-project.org/lppl.txt
----------------------------------------------------------------
\endpreamble
\end{minted}


The preceding preamble would cause |package.sty|  to begin as follows:

\begin{minted}{latex}
%%
%% This is file `phddoc.cls',
%% generated with the docstrip utility.
%%
%% The original source files were:
%%
%% phddoc.dtx  (with options: `class')
%% ----------------------------------------------------------------
%% phddoc --- A class to typeset LaTeX code.
%% E-mail: yannislaz@gmail.com
%% Released under the LaTeX Project Public License v1.3c or later
%% See http://www.latex-project.org/lppl.txt
%% ----------------------------------------------------------------
\end{minted}

We now reach the most important part of a .ins file: the specification of
what files to generate from the |.dtx| file. The following tells DocStrip to
generate hpackagei.sty from hpackagei.dtx by extracting only those parts
marked as `package'  in the .dtx file. (Marking parts of a .dtx file is
described later on.)

\begin{teXXX}
\generate{\file{<package>.sty}{\from{<package>.dtx}{package}}}
\end{teXXX}

\cmd{\generate} can extract any number of files from a given .dtx file. It can
even extract a single file from multiple |.dtx| files. See the DocStrip manual
for details.

Personally I also generate README.md files in |markdown| format as well, so that
when they get uploaded to |github| they can be rendered nicely.

\begin{minted}{latex}
\generate{\file{\jobname.md}{\from{\jobname.dtx}{readmemd}}}
\end{minted}

The text has to be wriiten using `guards' with the tag |readmd|

\begin{minted}{latex}
%<*readmemd>
# The `phddoc` LaTeX2e class

The `phd` latex package and the class with the same name provide
convenient methods to create new styles for books, reports
and articles. It also loads the most commonly used packages 
and resolves conflicts.
%</readmemd>
\end{minted}

\subsection{Generating messages} 

The next part of a |.ins| file consists of commands to output a message to
the user, telling him what files need to be installed and reminding him how
to produce the user documentation. The following set of \cmd{Msg} commands is
typical:

\begin{minted}[
frame=lines,
framesep=2mm,
baselinestretch=1.2,
fontsize=\footnotesize,
linenos
]{latex}
\obeyspaces
\Msg{****************************************************}
\Msg{* *}
\Msg{* To finish the installation you have to move the *}
\Msg{* following file into a directory searched by TeX: *}
\Msg{* *}
\Msg{* packagei.sty *}
\Msg{* *}
\Msg{* To produce the documentation run the file *}
\Msg{* package.dtx through LaTeX. *}
\Msg{* *}
\Msg{* Happy TeXing! *}
\Msg{* *}
\Msg{****************************************************}
Note the use of \obeyspaces to inhibit \tex from collapsing multiple spaces
into one.
\endbatchfile
\end{minted}


Appendix A.1 lists a complete, skeleton .ins file. Appendix A.2 is similar
but contains slight modifications intended to produce a class (|.cls|) file
instead of a style (|.sty|) file

\section{What to put in a  .dtx file}
We started describing the |.ins| install file first. The next file we will describe is
the |.dtx| file. This holds both the code definitions as well as the user documentation.

A |dtx|\ file contains both the commented source code and the user documentation
for the package. Running a |dtx|  file through |latex| typesets the
user documentation, which usually also includes a nicely typeset version of
the commented source code.

Due to some Doc trickery, a |dtx|  file is actually evaluated twice. The first
time, only a small piece of \latex\  driver code is evaluated. The second time,
comments in the |dtx|  file are evaluated, as if there were no `\%'  preceding
them. This can lead to a good deal of confusion when writing |dtx|  files
and occasionally leads to some awkward constructions. Fortunately, once
the basic structure of a |dtx|  file is in place, filling in the code is fairly
straightforward.

\paragraph{Guards} If you open any .dtx file you will notice that the lines either start with a \%
sign or sometimes with a percentage sign and |<|\textit{guard}|>|. The latter is called a guard and they are in a way
like html tags. They have a starting and an ending tag. In the example below there are two different guards
|<*10pt>...</10pt>| and |<*11pt></11pt>|. Unlike html tags guards are boolean expressions! You can use:
\begin{quote}
\textbar  ! \&  
\end{quote}

The \textbar stands for disjunction (OR), the \& stands for conjunction (AND) and the ! (NOT) stands for
negation. The terminal is any sequence of letters and evaluates to true iff it
occurs in the list of options that have to be included.

\begin{minted}{latex}
%<*10pt|11pt|12pt>
... code
%</10pt|11pt|12pt>
\end{minted}

A longer example from KOMA shows the concept better.

\fvset{gobble=0}
\begin{minted}[
frame=lines,
framesep=2mm,
baselinestretch=1.2,
fontsize=\footnotesize,
linenos
]{latex}
%    \begin{macrocode}
\def\normalsize{%
%<*10pt>
  \@setfontsize\normalsize\@xpt\@xiipt
  \abovedisplayskip 10\p@ \@plus2\p@ \@minus5\p@
  \abovedisplayshortskip \z@ \@plus3\p@
  \belowdisplayshortskip 6\p@ \@plus3\p@ \@minus3\p@
%</10pt>
%<*11pt>
  \@setfontsize\normalsize\@xipt{13.6}%
  \abovedisplayskip 11\p@ \@plus3\p@ \@minus6\p@
  \abovedisplayshortskip \z@ \@plus3\p@
  \belowdisplayshortskip 6.5\p@ \@plus3.5\p@ \@minus3\p@
%</11pt>
... 
%    end{macrocode}
\end{minted}

If the guards only contain a one line of text, then a short form is provided as |<10pt>|. It is unecessary to provide a closing tag and the `*' is omitted. The example below from the KOMA classes shows a quite ingenious way of writing the |\ProvidesFile| macro in
the different files; one for each tag. 
Two kinds of optional code are supported: one can either have optional code
that is on one line of tex code.

To distinguish both kinds of optional code the `guard modier' has been introduced. 
The `guard modifier' is one character that immediately follows the < of
the guard. It can be either * for the beginning of a block of code, or / for the end
of a block of code. The beginning and ending guards for a block of code have to
be on a line by themselves.

When a block of code is not included, any guards that occur within that block
are not evaluated.


\begin{minted}{latex}
%    \begin{macrocode}
\ProvidesFile{%
%<10pt>  scrsize10pt.clo%
%<11pt>  scrsize11pt.clo%
%<12pt>  scrsize12pt.clo%
}[\KOMAScriptVersion\space font size class option %
%<10pt>  (10pt)%
%<11pt>  (11pt)%
%<12pt>  (12pt)%
]
%    \end{macrocode}
\end{minted}

In the |.ins| file one could write to generate the various |.clo| files.:

\begin{minted}{latex}
\generate{\usepreamble\defaultpreamble
  \file{scrsize10pt.clo}{%
    \from{scrkernel-version.dtx}{clo,10pt}%
    \from{scrkernel-fonts.dtx}{clo,10pt}%
    \from{scrkernel-paragraphs.dtx}{clo,10pt}%
  }%
  \file{scrsize11pt.clo}{%
    \from{scrkernel-version.dtx}{clo,11pt}%
    \from{scrkernel-fonts.dtx}{clo,11pt}%
    \from{scrkernel-paragraphs.dtx}{clo,11pt}%
  }%
  \file{scrsize12pt.clo}{%
    \from{scrkernel-version.dtx}{clo,12pt}%
    \from{scrkernel-fonts.dtx}{clo,12pt}%
    \from{scrkernel-paragraphs.dtx}{clo,12pt}%
  }%
}%
\end{minted}

Becareful not to introduce spurious empy lines in your generated files by having empty lines in no-man's land, that is between tags.\footnote{In the phd package, I automatically generate the default settings from the |.dtx| files. In this case pgf will complain.}

\begin{minted}{latex}
%</install>

%<install>\endbatchfile
\end{minted}

\paragraph{The character table check } The second mechanism that Doc uses to ensure that a |dtx|  file is uncorrupted
is a character table. If you put the following command verbatim into
your |dtx|  file, then \pkg{Doc} will ensure that no unexpected character translation
took place in transport:

\begin{minted}[
frame=lines,
framesep=2mm,
baselinestretch=1.2,
fontsize=\footnotesize,
linenos,gobble=0,
]{latex}
% \CharacterTable
% {Upper-case \A\B\C\D\E\F\G\H\I\J\K\L\M\N\O\P\Q\R\S\T\U\V\W\X\Y\Z
% Lower-case \a\b\c\d\e\f\g\h\i\j\k\l\m\n\o\p\q\r\s\t\u\v\w\x\y\z
% Digits \0\1\2\3\4\5\6\7\8\9
% Exclamation \! Double quote \" Hash (number) \#
% Dollar \$ Percent \% Ampersand \&
% Acute accent \’ Left paren \( Right paren \)
% Asterisk \* Plus \+ Comma \,
% Minus \- Point \. Solidus \/
% Colon \: Semicolon \; Less than \<
% Equals \= Greater than \> Question mark \?
% Commercial at \@ Left bracket \[ Backslash \\
% Right bracket \] Circumflex \^ Underscore \_
% Grave accent \‘ Left brace \{ Vertical bar \|
% Right brace \} Tilde \~}
A success message looks like this:
***************************
* Character table correct *
***************************

and an error message looks like this:
! Package doc Error: Character table corrupted.
\end{minted}

\paragraph{DoNotIndex} When producing an index, \pkg{doc} normally indexes every control sequence
(i.e., backslashed word or symbol) in the code. The problem with this level
of automation is that many control sequences are uninteresting from the
perspective of understanding the code. For example, a reader probably
doesn’t want to see every location where \cs{if} is used—or \cs{the} or \cs{let} or
\cs{begin} or any of numerous other control sequences.

As its name implies, the \cs{DoNotIndex} command gives |Doc| a list of control
sequences that should not be indexed. \cs{DoNotIndex} can be used any
number of times, and it accepts any number of control sequence names per
invocation:

\begin{minted}[
frame=lines,
framesep=2mm,
baselinestretch=1.2,
bgcolor=white,
fontsize=\footnotesize,
linenos
]{latex}
\DoNotIndex{\#,\$,\%,\&,\@,\\,\{,\},\^,\_,\~,\ }
\DoNotIndex{\@ne}
\DoNotIndex{\advance,\begingroup,\catcode,\closein}
\DoNotIndex{\closeout,\day,\def,\edef,\else,\empty, \endgroup}
\end{minted}


\subsection{User documentation}

We can finally start writing the user documentation. A typical beginning
looks like this:

\begin{minted}[
frame=lines,
framesep=2mm,
baselinestretch=1.2,
bgcolor=white,
fontsize=\footnotesize,
linenos
]{latex}
% \title{The \textsf{package} package\thanks{This document
% corresponds to \textsf{package}~\fileversion,
% dated~\filedate.}}
% \author{your name \\ \texttt{your e-mail address}}
%
% \maketitle
\end{minted}


The title can certainly be more creative, but note that it’s common for
package names to be typeset with \docAuxCommand{textsf} and for \docAuxCommand{thanks} to be used to
specify the package version and date. This yields one of the advantages
of literate programming: Whenever you change the package version (the
optional second argument to \docAuxCommand{ProvidesPackage}), the user documentation
is updated accordingly. Of course, you still have to ensure manually that
the user documentation accurately describes the updated package.

Write the user documentation as you would any \latexe document, except
that you have to precede each line with a |\%|. Note that the |ltxdoc| document
class is derived from article, so the top-level sectioning command is
|\section|, not |\chapter|.



\section{General tips for defining a Class}

Evaluate, if there is a class that is nearer to what you wish to achive. If not do a set of
requirements.

Book structure - start with book or |Octavo| if you need to hack extensively. If not use memoir, |koma| or |tufte-book|.

Paragraph looks

Lists

Figures

Bibliography and citations

Footnotes

Index

Title pages

Book Cover

Language support

Mathematics

Graphs and figures

Typography - fonts, indentations fontsize etc

headers and footers


\section{Declaring Options}

Most classes or packages will have a good deal of options. These are declared using the
\docAuxCommand{DeclareOption} command. In this part no package loading should take place.

\begin{docCommand} {DeclareOption} { \marg{option} \marg{code}}
  The argument option is the name of the option being declared and the \marg{code} is the
  code that will execute if this option is requested.
\end{docCommand}


\begin{docCommand}{DeclareOption*} { \marg{code}}
  The argument \meta{code} in the star version of the command specifies the action to be 
  taken if an unknown option is specified. Within this argument the \docAuxCommand{CurrentOption}
  refers to the name of the option in question. 
  
\end{docCommand}

For example one could pass all such options
  to another package, using:
  \begin{verbatim}
  \DeclareOption*{\PassOptionsToPackage{\CurrentOption}{A}}
  \end{verbatim}


\section{Executing Options}

Normally after the options have been defined, one would need to provide default values and 
the options need to be executed. 

\begin{docCmd} {ExecuteOptions} { \marg{option list}}
  
\end{docCmd}

You can also |\ExecuteOptions| when declaring other options. There is one caveat. This command
can only be executed prior to executing the |\ProcessOptions| command because, as one of
its last actions, the latter command reclaims all of the memory taken up by the code for
the declared options.

\begin{docCmd} {ProcessOptions*} {}
\end{docCmd}

For some packages it is preferable or essential to process options in the order they
appear in the |usepackage| commands rather than using the order given through the
sequence of the \refCmd{DeclareOption} commands. In this case it one has to use
the star version of the command, i.e, |\ProcessOptions*| rather than |\ProcessOptions|.

\section{Special Commands for class files}

It is sometimes preferable to define a new class based on another and hence to extend it.
To support this concept the \latexe kernel provides two commands, \docAuxCommand{LoadClass} and
\docAuxCommand{PassOptionsToClass}. These two commands can then be used to develop a new class, by adding and extending the functionality of the loaded class.

\begin{docCommands}
\refCom{LoadClass}{ \oarg{option list}\marg{class}\oarg{release}}
\end{docCommands}  
  
For example the |ltxdoc| class loads the standard |article| class. The \pkg{tufte-book} class loads
the |book| class. The best way to understand the concepts discussed here is to
study these classes.

\section{A minimal class}

\begin{texexample}{Model Class}{ex:modelclass}

\begin{filecontents}{phdexampleclass.cls}
\NeedsTeXFormat{LaTeX2e}
\ProvidesClass{phdexampleclass}[2015/07/07]
\renewcommand\normalsize{\fontsize{}{10pt}{12pt}\selectfont}
\setlength\textwidth{6.5in}
\setlength\textheight{5in}
\pagenumbering{arabic}
\end{filecontents}

\end{texexample}

\vfill
\endinput




















