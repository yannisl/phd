\large
\parindent1em
\parskip1ex
\chapter{Set Theory}
\epigraph{Every mathematician agrees that every mathematician must know some 
set theory; the disagreement begins in trying to decide how much is some. }{---P. N. Halmos in \textit{Naive Set Theory}}

\section{Introduction}

One of the advantages in studying set theory is that by its nature it does not depend on any previous mathematical knowledge. Even if you have not studied set theory before you will at least I hope to give you an introductory knowledge into what it is. A secondary  aim of this chapter is to introduce set theory as a background to what we need to typeset and provide the necessary material to typeset it.

Set theory is a branch of mathematical logic that studies sets, which informally are collections of objects. Although any type of object can be collected into a set, set theory is applied most often to objects that are relevant to mathematics. The language of set theory can be used to define nearly all mathematical objects.

The modern study of set theory was initiated by Georg Cantor and Richard Dedekind in the 1870s. After the discovery of paradoxes in naive set theory, such as Russell's paradox, numerous axiom systems were proposed in the early twentieth century, of which the Zermelo–Fraenkel axioms, with or without the axiom of choice, are the best-known.

Set theory is commonly employed as a foundational system for mathematics, particularly in the form of Zermelo–Fraenkel set theory with the axiom of choice. Beyond its foundational role, set theory is a branch of mathematics in its own right, with an active research community. Contemporary research into set theory includes a diverse collection of topics, ranging from the structure of the real number line to the study of the consistency of large cardinals.

Halmos\footcite{halmos1960} is a good book to start with. Halmos explains the Naive Set Theory well and is a good introductory book.
The reasons behind the development of set theory is the interesting part. I have found Truss J.K. book\footcite{truss1997} as one of the better books on the subject.

\section{Set Definitions and Notation}

A \textbf{set} is a collection of objects known as \textbf{elements}. An element can be
anything we care to imagine, such as numbers, functions, geometric shapes, or lines. A set is a single
object that can contain many elements. This concept, in its  complete generality, is of great importance in mathematics since all of mathematics can be developed by starting from it. 


We express the fact that an object $x$ is an element of the set $E$ by writing 
\[x \in E. \] 
If $x$ is not an element of $E$ then we write $x \not\in E$. We say two sets $E,F$ are equal and write $E=F$, if they contain precisely the same elements. Most mathematical authors will capitalize the set and use lowercase letters for elements for clarity.

In depicting a set it is often convenient to simply list its elements. There is a standard notation for this: we enclose the list between \enquote{curly brackets} or \enquote{braces} as in
\[ E = \{x,y,z\dots\} \]
This is known as the \emph{roster method} of writing a set, and the list is known as a \textbf{roster}. 

The order in which elements of a set are listed is irrelevant. Hence the set $\{2,3,4\}$ is the same as $\{4,3,2\}$. Each element of a set is listed once and only once, so we would never write $\{1,2,2,3\}$.

The empty set is denoted by $\{\}$ or $\varnothing$.  We never write $\{\varnothing\}$ for the empty set, as the set has one element in it.

Another way to denote a set is the \textbf{Set-Builder Notation}, rather than
listing the items in the set, the members are described by some proposition, $P(x)$.
This is a useful way of doing things as some sets defy listing. An example of this
notation is
\[ C = \{x : x \text{ is a letter in the English Alphabet}\}\]
which is of the form $\{x:P(x)\}$. The colon is usually translated as \enquote{such that} or
\enquote{so that,} and then followed by a description. This is read as, ``$C$ is the set of $x$ such
that $x$ is a letter in the English Alphabet.'' Some mathematicians use a $\vert$ instead of a colon.
\[ \{x \vert P(x)\}\]

\makeatletter
\newif\if@suchthatsymbol
\@suchthatsymbolfalse
\def\suchthat{%
  \if@suchthatsymbol
    \expandafter\vert
  \else
    \expandafter\text{ such that }  
  \fi  
}
\makeatother

\begin{eg} set-builder notation for x in the set of all natural numbers.
\[ D = \{x \suchthat x \in \mathbb N\}\]
\end{eg}

\begin{eg} The set of positive divisors of 12 can be written
\[ \{1,2,3,4,6,12\}\]
We have $6 \in E \text{ but } 7\notin E$
\end{eg}

In functional analysis for programming the set-builder notation is called \textbf{set comprehension}.

In set theory, perhaps more than in any other branch of mathematics,
it is vital to set up a collection of symbolic abbreviations for various
logical concepts. Because the basic assumptions of set theory are absolutely
minimal, all but the most trivial assertions about sets tend to be
logically complex, and a good system of abbreviations helps to make otherwise
complex statements readable. For instance, the symbol $\in$ has already
been introduced to abbreviate the phrase \enquote{is an element of} I also make
considerable use of the following (standard) logical symbols:


\captionof{table}{Notation for Logic and Sets}
\begin{longtable}{llll}
$\in$         & \cs{in}         & abbreviates & is a member of\\
$\notin$      & \cs{notin}      & abbreviates & is not a member of\\
$\ni$         & \cs{ni}         & abbreviates & owns (has member)\\ 
$\subset$     & \cs{subset}     & abbreviates & is a proper subset of\\
$\subseteq$   & \cs{subseteq}   & abbreviates & is proper subset of \\
$\rightarrow$ & \cs{rightarrow} & abbreviates & implies\\
${\iff}$      &\cs{iff}         & abbreviates & if and only if\\ 
$\neg$        &\cs{neg}         & abbreviates & not            \\  
$\land, \wedge$ &\cs{land}      & abbreviates & and           \\
$\lor$        & \cs{lor}        & abbreviates & or (logical or)\\
$\exists$     &\cs{exists}      & abbreviates & there exists\\
$\exists!$    &\cs{exists!}     & abbreviates & there is no\\
$\forall$     &\cs{forall}      & abbreviates & for all\\
$\varnothing$ & \cs{varnothing} & abbreviates & the empty set\\
$\{\}$        & \cs{\{}\cs{\}}  & abbreviates & the set \\
$\mathbb{N}$  & |\textbb{N}|    & abbreviates & set of natural numbers\\
\end{longtable}


Memorizing the notation is necessary before any progress can be achieved. Many students find the notation cumbersome. If you do
write the first in English using full sentences as follows:

\def\fsolid{\tikz\draw[line width=0.8pt](0,0) -- (2.5,0);}
\def\fdashed{\tikz\draw[line width=0.8pt,dashed](0,0)--(2.5,0);}

\medskip
\begin{tabular}{lll}
\fsolid $\:\rightarrow\:$ \fdashed &for&\textit{if} \fsolid \textit{then} \fdashed.\\
\end{tabular}
\medskip

I found this hint in \textit{Set Theory: an introduction} by Robert L.~Vaught. The book is an elementary level text, but quite thorough ad written well.\footcite[See the beginning of Chapter 1 in: ][]{vaught1994}

\parindent1em
\parskip1ex
Vaught went further advocating the use of pure English where possible when writing mathematics and not symbolic logic, as in ordinary language one has a rich tradition of conventions for how to write grammatically and even how to argue. On the other hand \enquote{popular symbolic logic} there is no convention for dealing (as in a proof) with more than one sentence at a time. 


A set $x$ is formed by choosing the sets which are to be members of $x$. Are there any restrictions on the sets which are to be members of $x$? As we will see below there are. These are the paradoxes the most famous of which is the Russel paradox. Let $r$ be the set whose members are all sets $x$ such that $x$ is not a member of $x$. Then for every set $x$,
\[x \in r \iff x \notin x.\]
Substituting $r$ for $x$, we get a contradiction.

\section{Axioms}

\begin{axiom}[Axiom of Extensionality] 
If for any $x, x \in A$ if and only if $x\in B,$ then $A=B$. 
\end{axiom}

It is important that sets should be 
extensional; that is, a set is determined by which objects are or are not in it, 
irrespective of the manner in which it is presented to us. So, for example,

\[\{2,3,5\}\] 
\[\{p:p \text{ is prime and }p<6\}\]
are all the same set.

Clearly, in the expression $[x:\dots]$, the dots are always to be replaced by an \textit{asserting expression}. An extension of this usage is \texttt{- - -}$: \dots$, the \ldots is still to be asserting, but the dashed is clearly to be a naming expression. These two set-builder notions do not have to be taken as primitive or governed by the English meaning, but can be introduced in a simple way by definition, as follows:

\begin{Definition}
\item ${x: \mathcal{P}x}$ = the unique A such that for any $x, x, \in A$
\end{Definition}

\begin{axiom}[Separation or Aussonderungs (Zermelo)] 
$\{x:x \in B \text{ and } \mathcal{P}x \}$ always exists.
\end{axiom}



\section{Sets of Sets}

\section{Relations}

Boolean Operations on Sets were first studied by George Boole in 1847 and by A.~de~Morgan in the same period. We define:

\[
\begin{cases}
\begin{aligned}
A \cup B &= \{x: x \in A \text{ or } x \in B\}, \text{called the \textit{union} of $A$ and $B$}\\ 
A \cap B &= \{x: x \in A \text{ and } x \in B\}, \text{the \textit{intersection} of $A$ and $B$}\\
A - B    &= \{x: x\in A \text{ and } x \in B\},   A \text{ \textit{minus} } B.\\
\varnothing &= \{x: x \not= x \}, \mbox{\text{\parbox[t]{0.6\linewidth}{the \textit{empty} set (clearly by Extensionality the unique set with no members)}}}. 
\end{aligned}
\end{cases}
\]

\makeatletter
\def\oversortoftilde#1{\mathop{\vbox{\m@th\ialign{##\crcr\noalign{\kern1\p@}%
      \sortoftildefill\crcr\noalign{\kern2\p@\nointerlineskip}%
      $\hfil\displaystyle{#1}\hfil$\crcr}}}\limits}

\def\sortoftildefill{$\m@th \setbox\z@\hbox{$\braceld$}%
  \braceld\leaders\vrule \@height\ht\z@ \@depth\z@\hfill\braceru$}

\makeatother


%\[ \widetilde{abcdefghijklmnopqrstuvwxyz}\]
%\[ \oversortoftilde{abcdefghijklmnopqrstuvwxyz}\]
%\[\widetilde{A \cup B}\]


De Morgan's law $\oversortoftilde{A \cup B}= \tilde{A} \cap \tilde{B}$, 

A better notation is to use overlines rather than tildes

\[
\begin{align}
  \overline{A \cup B} &= \overline{A} \cap \overline{B}, \\
  \overline{A \cap B} &= \overline{A} \cup \overline{B},
\end{align}
\]


\section{Venn Diagrams}

Almost all of us have seen and used Venn Diagrams. Though they are no longer quite as ubiquitous as they were
during the heyday of “The New Math” (has it already been 35 years?), they are still a staple of Discrete Mathematics courses.
Nobody calls them anything except “Venn Diagrams,” and several modern sources (e. g. [R]) refer to John Venn’s 1880
article [V], so it is natural to assume that Venn discovered them. That would be the end of it, unless we read Venn’s article
 and see that Venn doesn’t call them “Venn Diagrams.” He calls them
“Eulerian Circles”! We quote his opening sentences:
\marginpar{
\includegraphics[width=0.8\linewidth]{john-venn}
}

\begin{quote}
Schemes of diagrammatic representation have been so familiarly introduced into logical
treatises during the last century or so, that many readers, even those who have made no
professional study of logic, may be supposed to be acquainted with the general nation and
object of such devices. Of these schemes one only, viz. that commonly called “Eulerian
circles,” has met with any general acceptance.\ldots
\end{quote}


\newcommand{\vennAll}[2]{
\begin{center}
\begin{tikzpicture}[scale=0.5]
%\draw (-4,-2.5) -- (4,-2.5) -- (4,2.5) -- (-4,2.5) -- cycle;
\draw (1,0) circle [radius=2] (1,1.5) node {\scriptsize #1};
\draw (1.5,-.5) circle [radius=1.0] (1.5,0) node {\scriptsize #2};
\end{tikzpicture}
\end{center}}
\vennAll{mortal}{men}

Sun-Joo Shin\footnote{Sun-Joo Shin is Professor of Philosophy at Yale University. Originally from Korea she is widely
known for her groundbreaking work on diagrammatic reasoning. In her first
book, The Logical Status of Diagrams (1994), she gave the first fully
worked-out example of a logical system based on diagrams (in this case,
Venn diagrams). Her second book, The Iconic Logic of Peirce’s Graphs
(2002), drew attention to a neglected diagrammatic logical system designed
by Charles Peirce, his “existential graphs.” She designed new ways of reading these diagrams and showed how and why the specific diagrammatic
strengths of Peirce’s graphs have been systematically neglected and misunderstood by later logicians. } in \emph{The Iconic Logic of Peirces's Graphs} says that:
Our ordinary reasoning typically involves information
obtained through more than one medium—sentences, diagrams,
smells, sounds, and so on. Recognizing the actual practice of this multimodal
reasoning, researchers have started focusing on multi-modal,
or heterogeneous, representation systems, which employ both symbolic
and diagrammatic elements. This is a clear departure from the major
direction taken by logicians and mathematicians since the development
of modern logic: For more than a century, symbolic representation systems
have been the exclusive subject for formal logic.\footcite{shin2002}
\marginpar{\includegraphics[width=0.9\linewidth]{peirce01}}

Peirce had a different notion for logic graphs. 

%\begin{eg}
%Let $A=\{ 1,2,4,8\}$, $B=\{ 8, 2, 1, 4\}$, $C=\{1,2,1,4,8\}$, and $D=\{1,2,3,4,8\}$.  Of these sets, $A=B=C$ because they each contain the same elements.  It doesn't matter that we listed the elements in a different order when we defined $A$ than when we defined $B$, or that we listed $1$ as an element of $C$ more than once.  The set $D$ is different from the others since it contains the element $3$ and the other sets do not.
%\end{eg}


%\begin{enumerate}
%\item[a.] $A\setminus(B\cup C)=(A\setminus B)\cap(A\setminus C)$, and
%\item[b.] $A\setminus(B\cap C)=(A\setminus B)\cup(A\setminus C)$.
%\end{enumerate}
