\chapter{Category Theory}

Category theory is the mathematician's attempt to lay bare some of the 
underlying principles common to diverse fields in the mathematical sciences. It 
has become, as well, an area of pure mathematics in its own right. Briefly, a 
category is a domain of mathematical discourse characterized in a very general 
way, and \emph{category theory} is thus an array of tools for stating results which can 
be used across a wide mathematical spectrum.\footcite{arbib1975} 

Arbib's first chapter is titled \emph{Learning to Think with Arrows}, which amptly describes the main tools of category theory.

\fullcite{arbib1975}

The reader is familiar with sets as collections of elements, and with functions 
as assignments of an element in one set to each element of another set. In other 
words, the usual approach to set theory starts with elements and builds all its 
notions in terms of these. In this chapter, we introduce a different approach to 
set theory, which builds all its notions in terms of arrows, the symbols $f: A \to B$ 
which represent a function as a unitary whole rather than in element-by-element 
terms. Then, in subsequent chapters, we shall see that this 'arrow-language' of 
category theory allows us to specify, once and for all, concepts which play an 
important role in many different areas of mathemetics even though their element- 
by-element definitions are drastically different in different domains of discourse.\footcite{arbib1975} 

\[
\begin{tikzcd}[scale=2.0]
A \arrow[dr, swap, "g \circ f"] \arrow[r, "g"] & B  \arrow[d]\\
& C\\
\end{tikzcd}
\]

Such diagrams — in which, given a starting point and a destination (such as 
$A$ and $C$ in the above diagram), different paths yield the same overall function are called 
\textbf{commutative diagrams} — \textit{commutare} is the Latin for exchange, and we say that a diagram 
commutes if we can exchange paths, between two given points, with impunity. 
Commutative diagrams is the categorist's method of writing equations. 


The historical development of the subject has been, very roughly, as follows:\footcite{awodey2010}

1945 Eilenberg and Mac Lane’s “General theory of natural equivalences” was
the original paper, in which the theory was first formulated.
Late 1940s The main applications were originally in the fields of algebraic
topology, particularly homology theory, and abstract algebra.\footcite{samuel1945}

1950s A. Grothendieck et al. began using category theory with great success in
algebraic geometry.

1960s F.W. Lawvere and others began applying categories to logic, revealing
some deep and surprising connections.

1970s Applications were already appearing in computer science, linguistics,
cognitive science, philosophy, and many other areas.

\section*{Resources}

If you are not a mathematician perhaps starting with Lawvere's book is the best option. I read the book, which is at a more elementary level and helped me understand the concepts and learn the terminology easy.

When I was through all the textbooks, I reread Eilenberg and Mac Lane's paper and I was still puzzled and confused about some of the concepts. You cannot win them all.

I kept these notes as a summary for both the mathematics as well as the typesetting, as I have done for most sections of this book.


\section{Typography}

The notation is not difficult to understand and typeset. However the commutative diagrams are somehow difficult to produce right from the first time. At the moment most authors of longer texts tend to gravitate in using tikz and the tikz-cd package. Authors of papers tend to use the older package |xy-pic|. They are both large packages with tonnes of options. 

Egbert Rijke published a short guide how to use LaTeX to write documents in category theory\footcite{rijke2015} and using \tikzname for drawing diagrams. 

%% https://github.com/EgbertRijke/CategoryTheory_Course

%%%% Macros for mathematical notation
\newcommand{\cat}{%
  \mathbf%
}
\newcommand{\domain}[1]{%
  \mathrm{dom}(#1)%
}
\newcommand{\codomain}[1]{%
  \mathrm{cod}(#1)%
}
\newcommand{\idarrow}[1][]{%
  \mathbf{1}_{#1}%
}

%%%% Macros for use in text
\newcommand{\incode}{%
  \texttt%
}
\newcommand{\name}{%
  \textsc%
}

\section{Category theory in \LaTeX}

\subsection{The xy-pic package}  

The package xypic\footcite{xypic} is perhaps the simplest to use.\footnote{If you are using LuaLaTeX ensure that pdfliterals are defined. } We can achieve most of our commutative diagrams with knowing the two commands
\[ xypic = \{ \text{\cs{xymatrix}},\text{\cs{ar}}, \&, |\\|, \} \]  
and have a basic understanding of tabular environments. \cs{xymatrix} encloses a tabular
environment, so we have to use |&| and |\\| to indicate the end of a row. \cs{ar} draws an arrow. The arrow can be labelled by using either a superscript or subscript, depending if we want to typeset on top or below the arrow.

The |\xymatrix| is a marix environment. A simple example is shown below.
\begin{texexample}{xymatrix basics}{ex:basic}
\[
\xymatrix{1&2\\3&4}
\]
\end{texexample}
It is unecessary for the last line to end with a carriage return. Also missing cells can be omitted.
\begin{texexample}{xymatrix basics}{ex:basic}
\[
\xymatrix{1&2\\3}
\]
\end{texexample} 
To connect two nodes with an arrow we use |ar| with a suitable method denoting the direction of the arrow. In the example |\ar[dr]| the letters \textbf{dr} mean down right
\begin{texexample}{xymatrix basics}{ex:basic}
\[
\xymatrix{
a \ar[dr] & b \\
c         & d
}
\]
\end{texexample}
Arrows can be labelled either on top or the bottom using |^_|.
\begin{texexample}{xymatrix basics}{ex:basic}
\[
A = \xymatrix{
a \ar[dr]^f & b \\
c         & d
} 
\quad B =
\xymatrix{
a \ar[dr]_f & b \\
c           & d
}
\quad C = 
\xymatrix{
a \ar[dr]_f^g \ar[r] & b \ar[d] \\
c             & d  }
\]
\end{texexample}



\begin{figure}[htp]
\begin{center}
\begin{tabular}{cccc}
\verb+[F]+ & {\xymatrix{*+[F]{\txt{Simples}}}} & \verb+[F=]+ &
      {\xymatrix{*+[F=]{\txt{Duplo}}}} \\
 & & & \\
\verb+[F.]+ & {\xymatrix{*+[F.]{\txt{Pontilhado}}}} & \verb+[F--]+ &
      {\xymatrix{*+[F--]{\txt{Tracejado}}}} \\
 & & & \\
\verb+[F-,]+ & {\xymatrix{*+[F-,]{\txt{Sombra}}}} & \verb+[F-:<3pt>]+ &
      {\xymatrix{*+[F-:<3pt>]{\txt{Arredondado}}}} \\
 & & & \\
\verb+[o][F-]+ & {\xymatrix{*+[o][F-]{\txt{Redondo}}}}
\end{tabular}
\end{center}
\caption{Frames em diagramas em matriz}\label{tab:framematriz}
\end{figure}

A problem that can arise is that sometimes the vertical bar is used as a short verbatim, you can delete it and restore it using |DeleteShortVerb| or |MakeShortVerb|.

There are many possible arrow styles. An arrow is defined within the method |@{}|. Each arrow style has
a tail a shaft and a tip.
\begin{texexample}{Arrows}{ex;arrows}
\[
\DeleteShortVerb{\|}
\xymatrix @R=0pt {A\ar @{|->}[r] & B\\
          A\ar @{~>}[l] & B\\
          A\ar @{.>}[r] & B\\
          A\ar @{-->}[r] & B\\
          A\ar @{=>}[r] & B\\
          A\ar @{^{(}->}[r] & B\\
          A\ar @{_{(}->}[r] & B\\
}          
\]
\end{texexample}
Note the way the uphook is defined.


\begin{displaymath}
  \xymatrix{
    {A} \ar@{>>->>}[]+<0ex,-2ex>;[d]
    & {C} \ar@{>>->>}[d]\\
    {B}
    & {D}
  }
\end{displaymath}


\newcommand{\AB}[2]{
\xymatrix{
#1 \ar[r] & #2
}
}


\[
\AB{X_1}{X_2}
\AB{X_3}{X_4}
\]



\begin{texexample}{Thinking in arrows}{ex:arrows}
\[ \xymatrix{A\ar[r]^f &B\\ } \]

\[ \xymatrix{A\ar[r]_f & B\\ } \]

\[  \xymatrix{A \ar@{>->}[r]_f & B\\ }\]

\[  \xymatrix{A \ar@{~>}[r]_f & B\\ }\]

\[  \xymatrix{A \ar@/^/[r]^f & B\\ }\]

% fix tilde to vertical
\[
\xygraph{
  []L :@/_/ [r] {M_1} :@/^/ [r]{M^2}
      :@{~>}[r]R      :@/_1em/"L"
}
\]

\[
\xymatrix@!0{
& \lambda\omega \ar@{-}[rr]\ar@{-}'[d][dd]
& & \lambda C \ar@{-}[dd]
\\
\lambda2 \ar@{-}[ur]\ar@{-}[rr]\ar@{-}[dd]
& & \lambda P2 \ar@{-}[ur]\ar@{-}[dd]
\\
& \lambda\underline\omega \ar@{-}'[r][rr]
& & \lambda P\underline\omega
\\
\lambda{\to} \ar@{-}[rr]\ar@{-}[ur]
& & \lambda P \ar@{-}[ur]
}
\]
\end{texexample}

Arrows are decorated by using a |\ar@{}| specification. This is very similar to \tikzname.  

\begin{texexample}{Using the xy package}{ex:xy}
\[
\xymatrix{
    A\ar[r]^f\ar[d]_h & B\ar[d]^g\\
    C\ar[r]_i         & D}
\]
\end{texexample}

\subsection*{The Barr package for xy-pic}

An extension to the xy-pic package was developed by Barr and provides somehow better semantically defined macros.

$$\bfig
\square[A`B`C`D;e`f`g`m] 
\efig
$$

The package is a book of its own and many evenings of experimentation.

The xy-pic package was used to typeset commutative diagrams in both spivak's as well as Tom 
Leinster's book which are both freely available the latter at arxiv.org\footnote{\protect\url{https://arxiv.org/abs/1612.09375v1}}

\begin{Definition}

Let $X$ and $Y$ be sets. The {\em product of $X$ and $Y$}, denoted $X\times Y$,\index{a symbol!$\times$} is defined as the set of ordered pairs $(x,y)$ where $x\in X$ and $y\in Y$. Symbolically, $$X\times Y=\{(x,y)\|x\in X,\;\; y\in Y\}.$$ There are two natural {\em projection functions} $\pi_1\taking X\times Y\to X$ and $\pi_2\taking X\times Y\to Y$.\index{projection functions}\index{product!projection functions}
$$\xymatrix@=15pt{&X\times Y\ar[ddr]^{\pi_2}\ar[ddl]_{\pi_1}\\\\X&&Y}$$

\end{Definition}

\subsection{Theorems, definitions and exercises}
Theorem environments, which are provided by the \pkg{amsthm} package, can be
used to declare environments for theorems, lemmas, definitions, exercises, and
the like. To declare the environments for definitions and exercises, we have
included the lines
\begin{quote}
\begin{minted}{tex}
\theoremstyle{definition}
\newtheorem{defn}{Definition}[section]
\newtheorem{ex}{Exercise}
\end{minted}
\end{quote}
in the preamble.

Now we can start writing definitions. For instance, the definition of a category
is written in the \docAuxEnvironment{document}-environment as:

\begin{quote}
\begin{minted}{tex}
\begin{Definition}
A \emph{category} \(\cat{C}\) consists of
\begin{itemize}
\item a collection of objects: \(A\), \(B\), \(C\),
  \ldots
\item a collection of arrows: \(f\), \(g\), \(h\),
  \ldots
\item for each arrow \(f\) objects \(\domain{f}\) and
  \(\codomain{f}\) called the \emph{domain} and 
\emph{codomain} of \(f\). If \(\domain{f}=A\) and 
  \(\codomain{f}=B\), we also write \(f:A\to B\),
\item given \(f:A\to B\) and \(g:B\to C\), so that 
  \(\domain{g}=\codomain{f}\), there is an arrow 
  \(g\circ f:A\to C\),
\item an arrow \(\idarrow[A]:A\to A\) for every 
  object \(A\) of \(\cat{C}\),
\end{itemize}
such that
\begin{description}
\item[(Associative law)] for every \(f:A\to B\), 
  \(g:B\to C\) and \(h:C\to C\) we have
\begin{equation*}
h\circ(g\circ f)=(h\circ g)\circ f,
\end{equation*}
\item[(Unit laws)] for every \(f:A\to B\) we have
\begin{equation*}
f\circ\idarrow[A]=f=\idarrow[B]\circ f.
\end{equation*}
\end{description}
\end{Definition}
\end{minted}
\end{quote}

This results in:
\begin{Definition}
A \emph{category} \(\cat{C}\) consists of
\begin{itemize}
\item a collection of objects: \(A\), \(B\), \(C\), \ldots
\item a collection of arrows: \(f\), \(g\), \(h\), \ldots
\item for each arrow \(f\) objects \(\domain{f}\) and \(\codomain{f}\) called
the \emph{domain} and \emph{codomain} of \(f\). If \(\domain{f}=A\) and \(\codomain{f}=B\),
we also write \(f:A\to B\),
\item given \(f:A\to B\) and \(g:B\to C\), so that \(\domain{g}=\codomain{f}\), there
is an arrow \(g\circ f:A\to C\),
\item an arrow \(\idarrow[A]:A\to A\) for every object \(A\) of \(\cat{C}\),
\end{itemize}
such that
\begin{description}
\item[(Associative law)] for every \(f:A\to B\), \(g:B\to C\) and \(g:C\to D\) we have
\begin{equation*}
h\circ(g\circ f)=(h\circ g)\circ f,
\end{equation*}
\item[(Unit laws)] for every \(f:A\to B\) we have
\begin{equation*}
f\circ\idarrow[A]=f=\idarrow[B]\circ f.
\end{equation*}
\end{description}
\end{Definition}


\subsection{Drawing diagrams}

For drawing diagrams, we recommend the \pkg{tikz-cd} package. The documentation
for the \pkg{tikz-cd} package is available on \url{http://www.ctan.org/pkg/tikz-cd}.
Be sure you use the latest version of \pkg{tikz-cd}, because its features and
syntax has been changed recently.

\begin{texexample}{Using the tikz-cd package}{ex:tikzcd}
\[
\begin{tikzcd}
A \arrow[dr, swap,"g \circ f"] \arrow[r, "g"] & B  \arrow[d]\\
& C\\
\end{tikzcd}
\]
\end{texexample}

As an example of a diagram drawn with \pkg{tikz-cd}, the following code displays
the diagram that has been used in class to demonstrate the associative law:
\begin{quote}
\begin{minted}{tex}
\begin{equation*}
\begin{tikzcd}
A \arrow[r,"f"]
  \arrow[dr,swap,"g\circ f"]
  &
B \arrow[dr,"g\circ h"]
  \arrow[d,swap,"g"]
  \\
  {}&
C \arrow[r,swap,"h"]
  &
D
\end{tikzcd}
\end{equation*}
\end{minted}
\end{quote}
The output of the above code is the diagram
\begin{equation*}
\begin{tikzcd}
A \arrow[r,"f"]
  \arrow[dr,swap,"g\circ f"]
  &
B \arrow[dr,"g\circ h"]
  \arrow[d,swap,"g"]
  \\
  {}&
C \arrow[r,swap,"h"]
  &
D
\end{tikzcd}
\end{equation*}
It is also possible to draw parallel arrows, for instance, to display coequalizer
diagrams, and dotted arrows to display universal properties. The code
\begin{quote}
\begin{minted}{tex}
\begin{equation*}
\begin{tikzcd}
A \arrow[yshift=.7ex,r,"f"]
  \arrow[yshift=-.7ex,r,swap,"g"]
  &
B \arrow[r,"e"]
  \arrow[dr,swap,"h"]
  &
C \arrow[densely dotted,d,"\exists!"]
  \\
& &
D
\end{tikzcd}
\end{equation*}
\end{minted}
\end{quote}
results in the diagram
\begin{equation*}
\begin{tikzcd}
A \arrow[yshift=.7ex,r,"f"]
  \arrow[yshift=-.7ex,r,swap,"g"]
  &
B \arrow[r,"e"]
  \arrow[dr,swap,"h"]
  &
C \arrow[densely dotted,d,"\exists!"]
  \\
& &
D
\end{tikzcd}
\end{equation*}
When the diagram seems too packed with information, it sometimes helps to
separate the rows and columns more by starting the |tikzcd|-environment
with the option |column sep=huge| or |row sep=large|, for example.
More precisely, the environment would look like
\begin{quote}
\begin{minted}{tex}
\begin{tikzcd}[column sep=huge]
...
\end{tikzcd}
\end{minted}
\end{quote}
The package documentation contains more examples which will prove useful in
displaying diagrams.

A very useful tool is an online website \href{https://tikzcd.yichuanshen.de/}{tikzcd} where you can use to draw the diagrams fairly fast. 


\begin{Rule}
Use tikz-cd for notes and arxiv.org articles, bt you may have to use xy-pic for papers as the journal might not allow the use of tikz. You will benefit if you have a basic knowledge of both.
\end{Rule}


\section{Sets, maps composition}

Before we give a precise definition of \enquote{category}, we should become familiar with one example, the \textbf{category of finite sets and maps}.

An object in the category is a finite set or collection. Here are some examples.

(the set of all students in the class) is one objct,\\
(the set of all desks in the classroom) is another,\\
(the set of all letters of the alphabet) is another.

Consider the finite set of three students in a class:
 \[\[ John, Mary, Sam\]\]
We can picture them as:
\medskip

\begin{center}
\includegraphics[width=0.7\textwidth]{johnandmary}
\end{center}
where a dot represents each element, and we are then free to leave off the labels when for one reason or another they are temporarily irrelevant to the discussion, and picture this set as:

\begin{center}
\includegraphics[scale=0.5]{seta}
\end{center}
Such a picture, labeled or not is called an \emph{internal diagram} of the set.

A \textbf{map} $f$ in this category consists of three things:
\begin{enumerate}
\item a set $A$, called the \texit{domain of the map.
\item a set $B$, called the codomain of the map,
\item a rule assigning to each element $a$ in the domain, an element $b$ in the
codomain. This $b$ is denoted by $f\circ a$ (or sometimes $f(a)$ read $f$ of $a$).
\end{enumerate}

This type of simple diagram representing a category with an ellipse and dots representing elements
is so common that we must spent some time to understand how to draw it and then develop some techniques
to speed it up. As these are then connected we might need to define a macro of the form:

|\catellipse{a1->x1,x1->z1,...}|

Our procedure would best take one argument describing the relationship between the elements of the individual sets.
\[d = \{ a, b, c\}\]

\begin{tikzpicture}[ele/.style={fill=black,circle,minimum width=.8pt,inner sep=1pt},every fit/.style={ellipse,draw,inner sep=-2pt}]
  \node[ele,label=left:$a$] (a1) at (0,4) {};    
  \node[ele,label=left:$b$] (a2) at (0,3) {};    
  \node[ele,label=left:$c$] (a3) at (0,2) {};
  \node[ele,label=left:$d$] (a4) at (0,1) {};

  \node[ele,,label=right:$1$] (b1) at (4,4) {};
  \node[ele,,label=right:$2$] (b2) at (4,3) {};
  \node[ele,,label=right:$3$] (b3) at (4,2) {};
  \node[ele,,label=right:$4$] (b4) at (4,1) {};

  \node[draw,fit= (a1) (a2) (a3) (a4),minimum width=2cm] {} ;
  \node[draw,fit= (b1) (b2) (b3) (b4),minimum width=2cm] {} ;  
  \draw[->,thick,shorten <=2pt,shorten >=2pt] (a1) -- (b4);
  \draw[->,thick,shorten <=2pt,shorten >=2] (a2) -- (b2);
  \draw[->,thick,shorten <=2pt,shorten >=2] (a3) -- (b1);
  \draw[->,thick,shorten <=2pt,shorten >=2] (a4) -- (b3);
 \end{tikzpicture}

\begin{center}
\begin{tikzpicture}[dot/.style={draw,fill,circle,inner sep=1pt},]
%    \node (b)  [label=below:$ b $]{};
%    \node (a1) [label=left:$ a_{1} $]{};
%    \node (a2) [label=left:$ a_{2} $, above = .5cm of a1]{};
%    \node (c1) [label=right:$ c_{1} $, above right = 1cm and 2cm of b]{};
%    \node (c2) [label=right:$ c_{2} $, above = .5cm of c1]{};
%    %\node (d)  [label=above:$ d $, above left = 1cm and 2cm of c2]{};
%
%   \draw (a2) -- (c2);
%   \draw (a1) -- (c1);
   \draw (-3,0) circle [x radius=2, y radius=1.5];
   \draw (3,0) circle [x radius=2, y radius=1.5];
   \node[dot,label=left:$a_1$] (a1) at (-3,1) {};
   \node[dot,label=left:$a_2$] (a2) at (-3.7,.1) {};
   \node[dot,label=left:$a_3$] (a3) at (-2.5,-1) {};
 \end{tikzpicture}
\end{center}



\begin{tikzpicture}[
  every node/.style={on grid},
  setA/.style={circle,inner sep=2pt},
  setB/.style={inner sep=2pt},
  setC/.style={fill=red,rectangle,inner sep=2pt},
  every fit/.style={draw,ellipse,text width=35pt},
  >=latex
]

% set A
\node[setA,label=left:$a$] (a) {{Mary}};
\node [setA,below = of a,label=left:$b$] (b) {John};
\node [setA,below = of b,label=left:$c$] (c) {Wei};
\node[above=of a,anchor=south] {$A$};

% set B
\node[setB, inner sep=0pt,right=3.5cm of a, label=right:$x$] (x) {$x$};
\node[setB, inner sep=0pt, below = of x] (y) {$\bullet_1$};
\node[setB, inner sep=0pt,below = of y] (z) {$\circ$};
\node[above=of x,anchor=south] {$B$};

% set C
\node[setC,label=right:$m$,right = 3.5cm of x] (m) {};
\node[setC,label=right:$n$,below = of m] (n) {};
\node[setC,label=right:$p$,below = of n] (p) {};
\node[above=of m,anchor=south] {$C$};

% the arrows
\draw[->,shorten >= 3pt] (a) -- node[label=above:$f$] {} (x);
\draw[->,shorten >= 3pt] (b) -- node[label=above:$f$] {} (x);
\draw[->] (c) -- node[label=above:$g$] {} (y);
\draw[->,shorten <= 3pt] (x) -- node[label=above:$h$] {} (m);
\draw[->] (n) -- node[label=above:$u$] {} (y);

% the boxes around the sets
%\begin{pgfonlayer}{background}
\node[fit= (a)  (c) ] {};
\node[fit= (x) (z) ] {};
\node[fit= (m) (p)] {};
%\end{pgfonlayer}
\end{tikzpicture}




$\xy\Loop(0,0)A(ur,ul)\endxy$

$\xy*\cir<4pt>{l^r}\endxy$


Testing
\[ 
\xy 
{\ar@{=>} (0,15)*{}; (10,15)*{}}; 
%{\ar@3{->} (0,10)*{}; (10,10)*{}}; 
%{\ar@2{:>} (0,5)*{}; (10,5)*{}}; 
%{\ar@{|->} (0,0)*{}; (10,0)*{}}; 
%{\ar@2{~>} (15,0)*{}; (25,0)*{}}; 
%{\ar@{->>} (15,5)*{}; (25,5)*{}}; 
%{\ar@{<->} (15,10)*{};(25,10)*{}}; 
%{\ar@/^1pc/(15,15)*{};(25,15)*{}}; 
\endxy 
\]
Testing
\[
\xy
(0,-5)*\ellipse(3,1){.};
(0,-5)*\ellipse(3,1)__,=:a(-180){-};
\endxy
\]


$$\xymatrix{
A \ar@(l,dl) &
    B \ar@(l,d) &
    C \ar@(l,dr) \\
D \ar@(dl,d) &
    E \ar@(dl,dr) &
    F \ar@(dl,r)
}$$


Sometime we need to adjust the spacing between columns, we can do this by using
|C=0pt| and rows can be adjusted using |@R=|\meta{dimen}. Sometimes a certain degree
of experimentation is necessary.

\[
B= \xymatrix @C=-15pt @R=1pc{
 {\bullet}\ar[d]    & *!L{male}\\
 {\bullet}\ar@(l,d) & *!L{female}\\ 
 }
\]

\noindent \begin{tabular}{lr}
\begin{minipage}{7cm}
\begin{verbatim}
\xy
(0,0)*+{A};(10,10)*+{B}**\dir{>}
\endxy
\end{verbatim}
\end{minipage} &
\begin{minipage}{6cm}
$$ 
\xy
(0,0)*+{A};(10,10)*+{B} **\dir{>}
\endxy
$$
\end{minipage}
\end{tabular}


\[
\xymatrix{
A \ar[r]^f & B \ar[d]^f
& A \ar[r]_{g_1} & B \ar[d]_{g_1}
& A \ar[r]|h & B \ar[d]|h \\
D \ar[u]^f & C \ar[l]^f
& D \ar[u]_{g_1} & C \ar[l]_{g_1}
& D \ar[u]|h & C \ar[l]|h
}
\]


\[
\xymatrix{
X {\scriptscriptstyle}\ar@(ur,r)[F=]^m}
\]

\[ 
 X^{\circlearrowleft m}  X^{\mkern-1mu\rotatebox[origin=c]{-270}{\circlearrowleft} m} \rightarrow B 
\]

$$
\xy
\xymatrix{
\ar[ul] &\ar[ul]{~>} &\ar[ur];
}
\endxy
$$

$$ 
\xy
(0,0)*\cir<30pt>{};
{\ar@{->} (-4,0)*{}; (-14,0)*+{1}};
{\ar@{->} (4,0)*{}; (14,0)*+{r}};
%{\ar@{->} (0,4)*{}; (0,14)*+{u}};
%{\ar@{->} (0,-4)*{}; (0,-14)*+{d}};
%{\ar@{->} (3,3)*{}; (12,12)*+{ur=ru}};
%{\ar@{->} (-3,3)*{}; (-12,12)*+{ul=lu}};
%{\ar@{->} (-3,-3)*{}; (-12,-12)*+{dl=ld}};
%{\ar@{->} (3,-3)*{}; (12,-12)*+{dr=rd}};
\endxy
$$

\[
\xymatrix{
*!R{A \times B}& *!L{C} \\
*!R{D} & *!L{E \times F}
}
\]

\[
\xymatrix @R=4pt @C=-10pt{
*!R{A \times B}& *!L{C} \\
*!R{D} & *!L{E \times F}
}
\]

\[
\xymatrix{
*!R=0{A \times B}& *!L=0{C} \\
*!R=0{D} & *!L=0{E \times F}
}
\]

\[
\xymatrix{
A \times B & C\phantom{\times E} \\
\relax\phantom{E\times}D & E \times F
}
\]

It follows that the unique
homomorphism $\phi': M(\ell)\longrightarrow M$ with $\phi'(m_+)=v$,  renders the following diagram commutative. 
\[
\xymatrix{ 
&M(\ell)\ar[d]^\phi\ar@{-->}[ld]_{\phi'}\\
M\ar@{->>}[r]_\psi &N.
}
\]
This proves that $M(\ell)$ is projective in $\CO_{int}$. 



\newcommand{\CF}{{\mathcal F}}
To prove the sliding relations, let us denote 
\[
\Psi_+:=\widehat{\CF}\left(
\begin{picture}(80, 30)(-12,15)
\qbezier(0, 45)(0, 45)(22, 0)
\qbezier(0, 0)(10, 13)(12, 15)
\qbezier(17, 21)(35, 50)(60, 0)
\put(-8, 35){$a$}
\put(32, 18){$v$}
%\put(75, 15){$=$}
\end{picture}
\right),
%
\qquad \quad
%
%
\Psi_+':=\widehat{\CF}\left(
\begin{picture}(80, 30)(-12,15)
\qbezier(0, 0)(25, 50)(45, 20)
\qbezier(49, 15)(50, 15)(60, 0)
\qbezier(60, 45)(60, 45)(38, 0)
\put(62, 35){$a$}
\put(22, 18){$v$}
\end{picture}
\right), 
\]
\[
\Psi_-:=\widehat{\CF}\left(
\begin{picture}(80, 30)(-12,15)
\qbezier(0, 0)(30, 60)(60, 0)
\qbezier(0, 45)(0, 45)(10, 22)
\qbezier(15, 15)(15, 15)(22, 0)
%\put(65, 0){, }
\put(-8, 35){$a$}
\put(32, 18){$v$}
%\put(65, 0){, }
\end{picture}\right), 
%
\qquad \quad
%
\Psi'_-:=\widehat{\CF}\left(
\begin{picture}(80, 30)(-8,15)
\qbezier(0, 0)(30, 60)(60, 0)
\qbezier(60, 45)(60, 45)(50, 22)
\qbezier(45, 15)(45, 15)(38, 0)
\put(62, 35){$a$}
\put(22, 18){$v$}
\end{picture}\right).
\]















%% https://github.com/appliedcategorytheory/TikZWD/blob/master/Collected_WDs_TikZ_Pictures_wiring_diagrams.tex