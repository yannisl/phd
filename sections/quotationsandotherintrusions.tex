\chapter{Quotations and Other Intrusions}
\normalfont

\epigraph{“What is a quote? A quote (cognate with quota) is a cut, a section, a slice of someone else’s orange. You suck the slice, toss the rind, skate away. Part of what you enjoy in a documentary technique is the sense of banditry. To loot someone else’s life or sentences and make off with a point of view, which is called “objective” because you can make anything into an object by treating it this way, is exciting and dangerous.”}{--- Anne Carson, \textit{Decreation}}

\epigraph{The great thing about printing is it should be invisible}{Beatrice Warde}
\section{Introduction}

For centuries quotations were not used in books. In the earliest printed books, a quotation was marked merely by naming the speaker.

\begin{scriptexample}[]{Poliphilus}
And after shee sayde, Poliphilus lette vs goe and ascende vp this mount nexte the Garden, and Thelemia remayning at the stayre foote, wee ascended vp to the playne toppe. Where shee shewed vnto mee, with a heauenly eloquence, a Garden of a large compasse, made in the forme of an intricate Laborynth allyes and wayes, not to bee troden, but sayled about, for insteade of allyes to treade vppon, there were ryuers of water.
\end{scriptexample}

Three forms of quotation mark are still in common use. Inverted and raised commas --- ``quote'' and
`quote' --- are generally favoured in Britain and North America. But baseline and inverted commas --
are still widely used in Germany. Many typographers prefer them to take the shape of
sloped primes (\enquote{---}) instead of tailed commas. 

To enforce a good style in English is easy. To try and internationalize it, is short of a nightmare.

\begin{texexample}{German quotes with babel and csquote}{ex:csquote}
\selectlanguage{ngerman}
The quotes for the German l"anguage \foreignquote{ngerman}{with some french \foreignquote*{french}{inner text}}
\selectlanguage{english}
\end{texexample}


\begin{scriptexample}[]{German quotation marks}

Andreas fragte mich: „Hast du den Artikel‚ EU-Erweiterung ‘gelesen?“ (Andreas asked me: \enquote{Have you read the \enquote{EU-Expansion} article?})
\end{scriptexample}

\begin{docCommand}{guillemotleft}{}
\end{docCommand}

\begin{docCommand}{guillemotright}{}
\end{docCommand}

\guillemotleft quote\guillemotright\ and <<quote>>. \footnote{Use the \texttt{\textbackslash{guillemotleft}} and \texttt{\textbackslash{guillemotright}} commands. You can also use the \protect\index{csquote} csquote package}Guillemets or otherwise known as duck foot quotation marks, chevrons, or angle quotes - <<quote>> and <quote> - are the normal form in France and Italy and are widely used in the rest of Europe. German typographers set their guillemets the 
>>opposite way<<. In either case, thin spaces are customary between the guillemets and the text they enclose. Other languages and scripts can use a plethora of different characters to denote quotation marks. For example in Chinese you can find {\panunicode 「」︰單引號 (Mandarin: dān yǐn hào, Jyutping: daan1 jan5 hou6, lit: "Single quotation mark")
『』︰雙引號 (Mandarin: shuāng yǐn hào, Jyutping: soeng1 jan5 hou6, lit: "Double quotation mark"}). 

\bgroup
\LARGE
\noindent\panunicode
﹁\\
︰\\
﹂\\
\egroup

When quotation marks (including guillemets) are used, the question remains, how many should be there?
The usual British practice is to use single quotes first, and doubles with singles.




\paragraph{Greek} uses angled quotation marks for the primary quotes.


\begin{scriptexample}[\panunicode]{}
Greek uses angled quotation marks (εισαγωγικά – isagogiká):

«Μιλάει σοβαρά;» ρώτησε την Μαρία.\\
«Ναι, σίγουρα», αποκρίθηκε.\\
and the quotation dash (παύλα – pávla):\\

― Μιλάει σοβαρά; ρώτησε την Μαρία.\\
― Ναι, σίγουρα, αποκρίθηκε.\\
which translate to:

“Is he serious?” he asked Maria.\\
“Yes, certainly,” she replied.
\end{scriptexample}

A closing quotation mark (») is added to the beginning of each new quoted paragraph.

\begin{scriptexample}[\panunicode]{}
« Η Βικιπαίδεια ή Wikipedia είναι ένα συλλογικό εγκυκλοπαιδικό\\
» εγχείρημα που έχει συσταθεί στο Διαδίκτυο, παγκόσμιο, πολύγλωσσο,\\
» που λειτουργεί με την αρχή του wiki. »
\end{scriptexample}

When quotations are nested, double and then single quotation marks are used:

«\ldots“\ldots‘\ldots’\ldots”\ldots».

\paragraph{Spanish} uses angled quotation marks (comillas latinas or angulares) as well, but always without the spaces.

\begin{scriptexample}[\panunicode]{}
\selectlanguage{spanish}
\enquote{Esto es un ejemplo de cómo se suele hacer una cita literal en español}. 
\foreignquote{english}{This is an example of how a literal quotation is usually written in Spanish.}
\selectlanguage{english}
\end{scriptexample}

And, when quotations are nested in more levels than inner and outer quotation, the system is:[65]

\begin{scriptexample}[\panunicode]{}
«Antonio me dijo: “Vaya ‘cacharro’ que se ha comprado Julián”».
“Antonio told me, ‘What a piece of “junk” Julián has purchased for himself’.”
\end{scriptexample}

As in French, the use of English quotation marks is increasing in Spanish, and the El País style guide, which is widely followed in Spain, recommends them. Hispanic Americans often use them, owing to influence from the United States.

\section{The apostrophe}

The most common error in text is to use |'s| for plurals of numbers, or for multiple letters. This is unecessary, use the \emph{2010s} or the ABCs.

It is normal to avoid the period after metric units and other self-evident abbreviations. Set 11.3 m and 520 cm but 36 in. or 36", and in bibliographical references, p 36f, or pp 306-314. You can also use the \pkg{siunitx} to give you a consistent set of units in scientific texts.

\section{Parentheses}

I used to introduce a lot of parentheses in my writings until I was shocked by the AP Manual of Style:
\emph{The perceived need for parentheses is an indication that your sentence is becoming concorted}. If you do use a parentheses, follow these guidelines:\index{style!parentheses}\index{style!brackets}

\begin{itemize}
\item If the material is inside a sentence, place the period outside the parentheses.
\item If the whole sentence is within brackets, put the full stop inside. (Please remember this.)
\end{itemize}

According to Robert Bringhurst's \index{Robert Bringhurst} \textit{Elements of Typographic Style}, the details of typesetting ellipsis depend on the character and size of the font being set and the typographer's preference. Bringhurst writes that a full space between each dot is "another Victorian eccentricity." In most contexts, the Chicago ellipsis is much too wide"—he recommends using flush dots, or thin-spaced dots (up to one-fifth of an em), or the prefabricated ellipsis character (Unicode \unicodenumber{U+2026} ({\pan \char"2026}), Latin entity \&hellip;) \citep{Bringhurst2005}.\index{ellipis>unicode}\index{ellipsis>\protect\string \ldots}

Bringhurst suggests that normally an ellipsis should be spaced fore-and-aft to separate it from the text, but when it combines with other punctuation, the leading space disappears and the other punctuation follows. He provides the following examples:
i\ldots j	k\ldots.	l\ldots l	l, \ldots l	m\ldots?	n\ldots!

[\dots]\lorem

$[\ldots]$\lorem

...\lorem



This all makes for nice-looking output, but it unfortunately adds a bit
of a burden to your job as a typist, because \tex's rule for determining the end of
a sentence doesn't always work. The problem is that a period sometimes comes
in the middle of a sentence \dots like when it is used (as here) to make an ellipsis" of three dots.

Moreover, if you try to specify \ldots by typing three periods in a row,
you get `...' the dots are too close together. One way to handle this is to go
into mathematics mode, using the |\ldots| control sequence defined in plain TEX
format. For example, if you type

Hmmm |$\ldots$| I wonder why?

the result is `Hmmm $\ldots$ I wonder why?'. This works because math formulas are
exempt from the normal text spacing rules.


\begin{teXXX}
\mathchardef\ldotp="613A % ldot as a punctuation mark
\def\ldots{\mathinner{\ldotp\ldotp\ldotp}}
\end{teXXX}




\section{Foreign Words and Romanization}

Foreign words and phrases used in an English text should be italicised (no
inverted commas) and should have the appropriate accents, e.g. \textit{inter alia,
raison d'\^{e}tre}.

Exceptions: words and phrases now in common use and/or considered part of
the English language, e.g. role, ad hoc, per capita, per se, etc.

\begin{enumerate}
\item Personal names should retain their original accents, e.g. Grybauskait\.{e},
Potočnik, Wallstr\"{o}m. Not to forget as Smith tells us to use the diaresis `where the dividing of two vowels makes two different vowels together may be taken for a dipthong, and make the verse fall short of its measure; as might have happened to the lines underneath, had no di\ae resis been used to prevent it; viz.

{\hskip3cm \narrower\narrower\it

 The Swans that in C\"ayster's water burn.\\
 In flames C\"aicus, Peneus, Alpheus, roll'd.\\
 The Tan\"ais smokes amid the boiling wave.\\

}

\item Quotations. Place verbatim quotations in foreign languages in quotation marks
without italicising the text.

\item Latin. Avoid obscure Latin phrases if writing for a broad readership. When
faced with such phrases as a translator, check whether they have the same
currency and meaning when used in English.

\item The expression `per diem' (`daily allowance') and many others have English
equivalents, which should be preferred e.g. `a year' or `per year' rather than `per annum'.

In general Greek, Cyrillic, Chinese or Arabic scripts should be transliterated, except in specialist texts where the author is sure that his audience has knowledge of the language.

\end{enumerate}

The European Commission Directorate-General for Translation has an English Style Guide that deals in detail with foreign words and phrases in english text and romanization systems.



















