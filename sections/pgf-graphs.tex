\chapter{Graphs}

Graphs are so named because they can be represented graphically, and it is this
graphical representation which helps us understand many of their properties. Each
vertex is indicated by a point, and each edge by a line joining the points representing
its ends. \tikzname is especially suited for this type of graphics. The first example draws a \textit{tree}, using the
\docValue{graphdrawing library}.


\begin{texexample}{Drawing Graphs}{ex:graph}
\bgroup
\begin{tikzpicture} [tree layout, sibling distance=8mm]
\graph [nodes={circle, draw, inner sep=1.5pt, outer sep=0pt}]{
1 -- {2 -- 3 -- { 4 -- 5, 6 -- { 7, 8, 9 }}, 10 -- 11 -- { 12, 13 }, 14 -- 15 -- {16, 17} }
};
\end{tikzpicture}
\egroup
\end{texexample}

We can use this to change from science, to management and have a simple organization chart.
\begin{texexample}{Drawing Graphs}{ex:graph}

\begin{tikzpicture} [tree layout, sibling distance=9mm]
\graph [nodes={circle, draw, inner sep=1pt, outer sep=1pt, font={\footnotesize\arial}, minimum size=2em},level distance=1.5cm]{%
PD -- {CM1 -- { SM1 -- FM1,  SM2 -- { FM2, FM3, FM4 }}, CM2 -- {SE1, SE2 }, CM3  -- {SE3, SE4}, EM--{LE1,LM1}, SM, CM}
};
\end{tikzpicture}

\end{texexample}

There are literally 10's of keys that can be used to change the appearance and the layout of the tree. The algorithms used are also described in the manual and is a must read for anyone wanting to delve deeper into how such layouts can be determined algorithmically. Get prepared to spend some time to understand all the parameters that can be chang