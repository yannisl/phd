\chapter{Numbers}
\label{ch:numbers}
\newfontfamily\bonum{texgyreheros-regular.otf}
\section{General principles}

When describing an arithmetic value the terms \textit{number, numeral}, and \textit{figure} are largely interchangeable, though \textit{figure} signifies a numerical symbol, especially any of the ten arabic numbers, rather than a representation in words. \textit{Number} is the general term for both arabic as well as roman figures; standards such as |BS 2961| recommends the term \textit{numeral}, while many people reserve instead for roman figures. Do not use \textit{figure} when confusion between numbers and illustrations may result.

\section{Ranging (lining) and (non-lining) figures }

In typography, two different varieties or 'cuts' of type are used to set
figures. The \emph{old style}, also called \emph{non-ranging} or \emph{non-lining}, has descenders
and a few ascenders: \bgroup\bonum 
abcde 0123456789\egroup. The new style, also called \emph{ranging},
\emph{lining}, or \emph{modern}, has uniform ascenders and no descenders: 0123456789; these are used especially in scientific and technical work.

It is not recommended to mix old- and new-style figures in the same book without special
directions. There are, however, contexts in which mixing is a benefit.

For example, a different style should be used for superior figures indicating
editions or manuscript sigla, to avoid confusion with cues for
note references.

The |phd| package loads appropriate default fonts for the type of document being used. Normally it defaults to old style numerals for the text and for lining figures, in tables and sectioning commands. (See also Chapter~\ref{ch:fonts}  Page~\pageref{ch:fonts} for a more technical discussion on fonts.) 

When the |phd| package is used with |XeLaTeX| it loads |fontspec|, which provides settings for not only lining and old style figures, but also if the font has the appropriate glyph has an option |SlashedZero|. This is mostly used for publications describing computer related subjects and where there must be a distinction between the letter `O', the normal zero `0' and the slashed zero. 

\begin{comment}
\begin{texexample}{Lining and Old style Numerals}{oldstyle}
\bgroup
\def\temp{
\fontspec[Numbers=Lining]{TeX Gyre Bonum}
0123456789

\fontspec[Numbers=SlashedZero]{TeX Gyre Bonum}
0123456789

\fontspec[Numbers=OldStyle]{TeX Gyre Bonum}
abcde 0123456789
}
\ifxetex 
\temp
\else
  abcde \oldstylenums{0123456789}
\fi
\egroup
\end{texexample}
\end{comment}

\section{Figures and Words}

One common question that comes in publications dealing with guidelines for writing, is when to use words for figures. In non-technical contexts, OUP style recommends the words for numbers below
100. When a sentence contains one or more figures of 100 or above,
however, use arabic figures throughout for consistency within that
sentence: print for example 90 to 100 (not \textit{ninety} to 100), 30, 76, and 105
(not \textit{thirty}, \textit{seventy-six}, and 108). This convention holds only for the sentence
where this combination of numbers occurs: it does not influence
usage elsewhere in the text unless a similar situation exists.
However, clarity for the reader is always more important than blind adherence
to rule, and in some contexts a different approach is necessary. 

For example, it is sometimes clearer when two sets of figures are mixed to
use words for one and figures for the other, as in thirty 10-page pamphlets,
nine 6-room flats. This is especially useful when the two sets run throughout
a sustained expanse of text (as in comparing quantities):
\begin{quote}
The manuscript comprises thirty-five folios with 22 lines of writing, twenty
with 21 lines, and twenty-two with 20 lines.
\end{quote}

Anything more complicated, or involving more than two sets of quantities,
is probably presented more clearly in a table.

In technical contexts, it is preferable  spelling out numbers below
ten. Similar rules govern this convention: in a sentence containing
numbers above and below ten, style the numbers as figures rather
than words.

Use figures with all abbreviated forms of units, including units of time,
and with symbols:


|6'2" 250 BC 11a.m. 13 mm|

For units also see page~\pageref{units}, where we describe the use of the |siunitx| package. Spaces between numbers and units are important and also the package introduces, a non-breaking space, preventing printing the number in a different line during hyphenation.

\section{Numbers and Punctuation}

In nontechnical contexts, use commas in numbers of more than four
figures. Although optional, it is OUP style to use the comma in figures up
to 9,999, such as 1,863 or 6,523. Do not use a comma to separate groups of
three digits in technical and foreign-language work:
Continental languages,
and International Standards Organization (ISO) publications in
English, use a comma to denote a decimal sign, so that 2.3 becomes 2,3.

Use instead a thin space where necessary to separate numbers with five
or more digits either side of the decimal point:
14 785 652 1000000 3.141592 65 0.000 025.

Four-digit figures are not split with thin spaces—3.1416—except in
tabular matter, where four-digit figures are aligned with larger numbers (see the Chapter on Tables at page~\ref{ch:tables}).

Use the |siunitx| for consistency. The separator used between groups of digits is stored by the group-separator option.
This takes literal input and may be used in math mode: for a text-mode full space use the (\texttt{~}).

\begin{texexample}{Printing numbers}{}
\text{~}.
\num{12345} \\
\num[group-separator = {,}]{12345} \\
\num[group-separator = \text{~}]{12345}
\end{texexample}

For very large numbers use the \cmd{\numprint} from the \pkg{numprint} package \citep{numprint}. This package is also loaded automatically by |phd|. The options provided are not as extensive as those of the |siunitx| package, but it handles huge numbers robustly: \numprint{34567890768966645345}. 

\section{Ranges}

For a span of numbers generally, use an en rule, eliding to the fewest
number of figures possible: 30-1, 42-3, 132-6, 1841-5. But in each hundred
do not elide digits in the group 10 to 19, as these represent single
rather than compound numbers: 10-12,15-19,114-18, 214-15, 310-11.

For larger ranges give only the last two digits of the second number unless more are necessary.

\begin{longtable}{ll}
98-103    &923-1,003\\
103-05    &1,005-12\\
567-892   &1,669-1,722\\
\end{longtable}

The MLA manual \citep{MLA} recommends that for years beginning in AD 1000 or later, omit the first two digits of the second year if they are the same as the first two digits of the first year. Otherwise, write both years in full.

2005-09

1867-1901

For years that begin before AD 1 do not abbreviate any ranges.

741-560 BC

\BC{142}-\AD{158}


When describing a range in figures, repeat the quantity as necessary to
avoid ambiguity: 1~thousand to 2 thousand litres, 1 billion to 2 billion light years
away. The elision 1 to 2 thousand litres means the amount starts at only
1 litre, and 1 to 2 billion light years away means the distance begins only
1 light year away. Add non breaking spaces to avoid splitting |1~thousand| and similar words.


Use a comma to separate successive references to individual page
numbers: 6, 7, 8; use an en rule to connect the numbers if the subject
is continuous from one page to another: 6-8. OUP prefers references to
provide exact page extents; where this is impossible, print 51 f. if the
reference extends only to page 52, but 51 ff. if the reference is to more than one following page. For scientific work the \textit{folio} is not preferred and rather use |pages| or abbreviated forms \textit{pg.}. 


 For lists and ranges when a single unit is given, \pkg{siunitx} will
 automatically \enquote{compress} exponents when a fixed exponent is in use.

\begin{texexample}{Printing ranges with siunitx}{ex:ranges}
  \sisetup{
    fixed-exponent      = 3        ,
    list-units          = brackets ,
    range-units         = brackets ,
    scientific-notation = fixed
  }%
  \SIrange{1e3}{7e3}{\metre} \\
  \SIlist{1e3;2e3;3e3}{\kg}
\end{texexample}

If your document uses a lot of numbers and units, it will pay you to study the \pkg{siunitx}.

\section{Fractions}

This is an area where \tex excels and where possible always use math mode.

In statistical matter use one-piece (cased) fractions where available (\textonehalf). If these are not available, use split fractions (e.g. $2/3$).

In non-technical running text, set complex fractions in font-size numerals
with a solidus (\thinspace\textfractionsolidus\thinspace) between (19/100). Known also as 'shilling 
7 I Numbers 1 71
tions', they represent a quantity without needing extra interlinear space
to be displayed on more than one line. Decimal fractions are similarly
useful: 12.66 rather than $12\frac{2}{3}$, 99.9 rather than 99 and 9/10.

In mathematical texts the tradition is to use a \textit{virgule} and not a solidus.

\[ a/b + c/d + e/f = 1\]

Spell out simple fractions in textual matter, for example one-half, two-thirds,
one and three-quarters. Hyphenate compounded numerals in compound
fractions such as nine thirty-seconds, forty-seven sixty-fourths; the
numerator and denominator are hyphenated unless either already contains
a hyphen. Do not use a hyphen between a whole number and a
fraction: twenty-six and nine-tenths. Combinations such as half a mile, half a
dozen contain no hyphens, but write half-mile, half-dozen, etc. 

If at all possible, do not break spelt-out fractions at line endings.

\subsection{The xfrac package}

The |phd| package loads the package \pkg{xfrac} by default. The package was developed using \latex3 and offers an interface of declaring fonts via the \cmd{\DeclareInstance}. What the package attempts to do is to provide user commands for more granular choices for text fractions. For maths the recommendation is to leave everything to the \tex engine.

\DeclareInstance{xfrac}{cmr}{text}
{slash-symbol-font = ptm}

The package provides the command \cmd{\sfrac}, which produces text fractions such as \sfrac{7}{9} which you can compare with their maths siblings $\sfrac{7}{9}$. The package was developed by the \latex team, but was primarily the effort of Wills Robertson \citep{xfrac}. 

\section{Currency}

\subsection{The Euro}
\index{currency!euro}
The \textit{euro} can be typeset using the command \cmd{\texteuro} which produces the euro symbol (\texteuro). It can also be entered directly if you are using the XeLaTeX or LuaLaTeX engines. 

\subsection{British pounds}
\index{currency!pound}\index{currency!pennies}
Amounts in whole pounds should be printed with the £ symbol, numerals
and unit abbreviation close up: £2,542, £3m., £7.47m. Print 00
after the decimal point only if a sum appears in context with other
fractional amounts: They bought at £8.00 and sold at £9.50.

Amounts in pence are set with the numeral close up to the abbreviation,
which has no full point: 56p rather than £0.56. Mixed amounts
always extend to two places after the decimal point, and do not include
the pence abbreviation: £15.30, £15.79.

Amounts expressed in pre-decimal currency---£.s.d. (italic)---will
continue to be found in copy and must be retained. They will naturally occur in resetting books published before 1971, and in new books in
which the author refers to events and conditions, or quotations from
work dating, before 1971. For example:

\begin{quote}
In 1969 income tax stood at 8s. 3d. in the £.\\
The tenth edition cost 10s. 6d. in 1956.
\end{quote}

In new books, and in annotated editions of reset works, a decision must
be made as to whether to introduce decimal equivalents, for elucidation
or for ease of comparison (e.g. in statistics). Note the distinction in the
pound symbol between the earlier style of, for example, `£44. 3s. 10d.'
and the earlier style of `44l. 3s. 10d.'. In both these styles a normal space
of the line separates the elements. Note the spelling fourpence, ninepenny, etc. for amounts in pre-decimal pennies.  

\section{US currency}

\index{currency!US}
Sums of money in dollars and cents are treated like those in pounds and
pence: \$4,542, \$3m., \$7.47 m., 56c. In older books one can find ``cents'' abbreviated as ``\textcent'', but this is no more common. The \cmd{\textcent} can be used to typeset the \textcent symbol. There is no need to load any packages, as the |phd| package will load the appropriate package based on the \TeX engine used.

\section{Old currency Symbols}
\index{currency!denarius}
\index{currency!florin}
\index{\string \denarius}

The Denarius (\Denarius) and Florin (\Florin) glyphs can be found both in |UTF8| fonts as well as using packages with |pdfLaTeX|. The package \pkg{marvosym} \citep{marvosym} provides many such commands and they belong to specialized books, although now and then one can find such symbols being used for mathematics. You can view many of these symbols in the implementation part of this manual at Page~\pageref{currencysymbols}.

\section{Calendar}

Variations in time-reckoning systems between cultures and eras can
lead both author and editor to error. The following section offers some
guidance for those working with unfamiliar calendars; fuller explanation
may be found in Blackburn and Holford-Stevens, \textit{The Oxford Companion
to the Year}

\subsection{Old and New Style}

These terms are often applied to two different sets of facts. In 1582 Pope
Gregory XIII decreed that, in order to correct the Julian calendar, the
days 5-14 October ofthat year should be omitted and no future centennial
year (e.g. 1700, 1800, 1900) should be a leap year unless it was
divisible by four (e.g. 1600, 2000). This reformed 'Gregorian' calendar
was quickly adopted in Roman Catholic countries, more slowly elsewhere:
in Britain not till 1752 (when the days 3-13 September were
omitted), in Russia not till 1918 (when the days 16-28 February were
omitted). The discrepancy between the Julian and Gregorian calendars
was ten days until 28 February/10 March 1700, eleven days from 29
February/11 March 1700 to 28 February/11 March 1800, twelve days
from 29 February/12 March 1800 to 28 February/12 March 1900, and
thirteen days from 29 February/13 March 1900; it will become fourteen
days on 29 February/14 March 2100.

Until the middle of the eighteenth century, not all states reckoned the
new year from the same day: whereas France (which had previously
counted from Easter) adopted 1 January from 1563, and Scotland from
1600, England counted from 25 March in official usage as late as 1751; so
until 1749 did Florence and Pisa, but whereas Florence, like England,
counted from 25 March AD 1, Pisa counted from 25 March 1 BC, SO that the
Pisan year was always one higher than the Florentine. Thus the execution
of Charles~I was officially dated 30 January 1648 in England, but 30
January 1649 in Scotland. In Florence---which used the Gregorian calendar---
the same day was 9 February 1648, in France and Pisa 9 February
1649. Furthermore, although both Shakespeare and Cervantes died on 23
April 1616 according to their respective calendars, 23 April in Spain (and
other Roman Catholic countries) was only 13 April in England, 23 April in
England was 3 May in Spain, and in Pisa the year was 1617.

Confusion is caused in English-language writing by the adoption, in
England, Ireland, and the American colonies, of two reforms in quick
succession. The year 1751 began on 1 January in Scotland and on 25
March in England, but ended throughout Great Britain and its colonies
on 31 December, so that 1752 began on 1 January. So, whereas 1 January
1752 corresponded to 12 January in most Continental countries, from 14
September onwards there was no discrepancy.
As a result, many writers treat the two reforms as one, using Old Style
and New Style indiscriminately for the start of the new year and the
form of calendar, even with reference to countries in which the two
reforms were adopted at different times. This is unfortunate: Old Style
should be reserved for the Julian calendar and New Style for the Gregorian;
the 1 January reckoning should be called 'modern style' (or 'Circumcision
style'), that from 25 March 'Annunciation' or 'Lady Day' style (or
'Florentine' and 'Pisan' style with reference to those cities), and others
as appropriate, such as 'Easter style', 'Nativity style', 'Venetian style' =
1 March, 'Byzantine style' = 1 September

\subsection{Typesetting old and new style dates}

It is customary to give dates in Old or New Style according to the system
in force at the time in the country chiefly discussed; if the system may be
unfamiliar to the reader, a brief explanation should be added.

The OUP recommends that any dates
in the other style should be given in parentheses with an equals sign
preceding the date and the abbreviation of the style following it: 23
August 1637 NS(=13 August OS) in a history of England, or 13 August 1637
OS (= 23 August NS) in one of France. In either case, 13/23 August 1637 may
be used for short, but when citing documents take care to use this form
only when the original itself employs it. On the other hand, it is normal
to treat the year as beginning on 1 January: modern histories of England
date the execution of Charles I to 30 January 1649. When it is necessary to
keep both styles in mind, it is normal to write 30 January 1648/9, subject to
the same qualification when citing documents; otherwise the date
should be given as 30 January 1648 (= modern 1649). (Contemporary accounts
could manage this as well: George Washington's date of birth was
recorded in the family Bible as ye 11th day of February 1731/2.) 

Original
documents may exhibit the split fraction, for example 172\sfrac{1}{2}, this should
be used only when exact transcription is required. For dates in Pisan style
between 25 March and 31 December inclusive, write 15 August 1737/6. 









