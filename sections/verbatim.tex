\chapter{Verbatim}
\label{ch:verbatim}
\epigraph{Please do not attempt to simplify this code.\\
Keep the space shuttle flying.}{Note form the Kubernettes code at github.com}

\section{Introduction}

The verbatim environment uses the monospaced |\ttfamily| font, turns blanks into spaces, starts a new line  for each carriage return and interprets \emph{every}  character literally. What this means is that all special characters are catcoded to other.

The command \cs{verb} produces in-line verbatim text, where the argument is delimited by any pair of characters.

The star variants of the commands are the same, except that spaces print as in the \tex{}book's space character instead of as blank spaces such as \verb* +test this+. This is font dependent. Some of the fonts may not contain the character.

 \LaTeX's \texttt{verbatim} and \texttt{verbatim*} environments
 have a few features that may give rise to problems. These are:
 \begin{itemize}
   \item
     Due to the method used to detect the closing |\end{verbatim}|
     (i.e.\ macro parameter delimiting) you cannot leave spaces
     between the |\end| token and |{verbatim}|.
   \item
     Since \TeX{} has to read all the text between the
     |\begin{verbatim}| and the |\end{verbatim}| before it can output
     anything, long verbatim listings may overflow \TeX's memory.
 \end{itemize}
 Whereas the first     of these points can be considered
 only a minor nuisance the other one is a real limitation, for the older engines. For the newer engines this is not a real issue now.
 
The package \pkg{verbatim} eliminates the above limitatiosn by searching for the |\end{verbatim}| string rather than absorbing the full argument of the pseudo environment. The \latexe verbatim code just absorbs the whole argument.

The package \pkg{Fancyvrb} provides a more advanced interface, but also has its limitations.

Full parsers such as listings are discussed in other sections of this Monograph.




\section{LaTeX's verbatim environment}



The environment and the command are provided in source2e under the \docFile{ltmiscen.dtx}.

A number of other packages were also developed over the years. These will be reviewed here as well as add examples as to how to handle the programming of such matters.

\section{How verbatim code works?}

\begin{texexample}{vobeyspaces}{ex:obeyspaces}
\makeatletter
\begingroup
\obeylines
\@xobeysp

\ttfamily
         This is some text
           Try any shape of line       test

           
\endgroup


\makeatother
\end{texexample}
\subsection{Verbatim}


  The verbatim environment uses the fixed-width |\ttfamily| font, turns
  blanks into spaces, starts a new line for each carriage return (or
  sequence of consecutive carriage returns), and interprets
  \emph{every} character literally.
  I.e., all special characters |\, {, $|, etc.
   are |\catcode|'d to 'other'.

 The command |\verb| produces in-line verbatim text, where the argument
 is delimited by any pair of characters.  E.g., |\verb #...#| takes
  `|...|' as its argument, and sets it verbatim in |\ttfamily| font.

The *-variants of these commands are the same, except that spaces
print as the \TeX{}book's space character instead of as blank spaces.

\begin{macro}{\@vobeyspaces}
\begin{teX}
{\catcode`\ =\active%
\gdef\@vobeyspaces{\catcode`\ \active\let \@xobeysp}}
\end{teX}
\end{macro}
%
\begin{macro}{\@xobeysp}
Moved to ltspace.dtx
% \changes{v1.0z}{1995/07/13}{Use \cs{nobreak}}
% \changes{v1.1f}{1996/09/28}{Moved to ltspace.dtx}
\end{macro}
%
The next two macros, which follow conventions for star and unstarred versions
use a common trickery to define environments that capture their contents
in a pseudo-environment.

Just to refresh a delimited macro can be delimited essentially with almost anything except 
braces, \# and some other esoteric characters that can confuse \tex's parser. In the 
example below we will create a delimited macro \cs{test}, which is delimited by
\cs{endtest}. 

\begin{texexample}{Delimited Refresher}{ex:delim1}
\def\test#1\endtest{#1}
\test
  100
\endtest
\end{texexample}

If the braces are changed to the catcode for other so that they can be used
to delimit the macro and by adding a matching end{} we can trick LaTeX that everything
is in order.

Our exampel will be a bit more complicated than our normal examples. We will develop a rudimentary environment named \docAuxEnvironment{zverbatim} that will capture the user input, typeset it verbatim and colorize comment strings.

We will mostly use |expl3| syntax to continue our study of the language and also to make the code a bit more readable. The description of the functions are listed below:



\begin{texexample}{}{}
\ExplSyntaxOn
\tl_new:N \g_store_tl

\cs_new:Npn \make_at_letter:
 {
  \char_set_catcode_letter:N @
 }
  
\cs_new:Npn \make_at_other:
 {
  \char_set_catcode_letter:N @
 }  
 
\cs_new:Npn \make_other:n #1 
  {
 		\char_set_catcode_other:N #1 \relax
  } 
  
\make_at_letter:
\begingroup
\char_set_catcode_escape:N | 
\char_set_catcode_group_begin:N [
\char_set_catcode_group_end:N ] 
\char_set_catcode_other:N\{
\char_set_catcode_other:N\}
\char_set_catcode_other:N\\


|cs_gset:Npn|phd_xverbatim:n#1\end{zverbatim}
            [|process_argument[#1]|end[zverbatim]]
|endgroup


\cs_set:Npn \zverbatim_aux: 
   {
     \bfseries
     \let\do\make_other:n\dospecials
     \obeylines \verbatimfont \@noligs
     \hyphenchar\font\m@ne
   }

% just set some preliminaries with
% |\@zverbatim|
\cs_gset:Npn \zverbatim
   {
     \zverbatim_aux: 
     \phd_xverbatim:n
   }

% Cannot do anything more at this time
\cs_gset:Npn \endzverbatim {}

% Process the argument
\gdef\process_argument#1 {
   
   \tl_set:Nn\g_store_tl{#1}
  
   \regex_replace_all:nnN 
   {\%[a-zA-Z\ \#\d]+}
   { \cB{\c{color}\cB{green800\cE}\0\cE} }
   \g_store_tl
   \tl_use:N \g_store_tl
}
\make_at_other:

\ExplSyntaxOff


% Output to test that everything is # working
\begin{zverbatim}
% Test to see a comment #1  
    \testing \our 
      \macrotest{\someother}
\end{zverbatim}


\end{texexample}
 

Now that we have more or less understood how verbatim pseudo-environments are
created, let us examine the \latexe code a bit more carefully and then we will
revisit our example in a bit more detail to add a few more features.


Catcodes are set in a group and the macros |\@xverbatim| and |\@sverbatim|
are defined exactly in a similar fashion to that of our example.

\begin{macro}{\@xverbatim,\@sxverbatim}
\begin{teX}
\begingroup \catcode `|=0 \catcode `[= 1
\catcode`]=2 \catcode `\{=12 \catcode `\}=12
\catcode`\\=12 |gdef|@xverbatim#1\end{verbatim}[#1|end[verbatim]]
|gdef|@sxverbatim#1\end{verbatim*}[#1|end[verbatim*]]
|endgroup
\end{teX}
\end{macro}


The next macro is the starting macro that is call by |\verbatim| at the beginning
of the environment. It is responsible for setting decorative parameters. This is
done by the use of a |trivlst|. 

\begin{macro}{\@verbatim}
\begin{teX}
\def\@verbatim{\trivlist \item\relax
  \if@minipage\else\vskip\parskip\fi
  \leftskip\@totalleftmargin\rightskip\z@skip
  \parindent\z@\parfillskip\@flushglue\parskip\z@skip
\end{teX}
% \changes{LaTeX2.09}{1991/08/26}{\cs{@@par} added}
    Added |\@@par| to clear possible |\parshape| definition
    from a surrounding list (the verbatim guru says).
% \changes{v0.9p}{1994/01/18}
%         {Only add \cs{penalty} if in hmode}
\begin{teX}
  \@@par
  \@tempswafalse
  \def\par{%
    \if@tempswa
\end{teX}
    A |\leavevmode| added: needed if, for example, a blank verbatim
    line is the first thing in a list item (wow!).
% \changes{v1.0f}{1994/04/29}{\cs{leavevmode} added}
\begin{teX}
      \leavevmode \null \@@par\penalty\interlinepenalty
    \else
      \@tempswatrue
      \ifhmode\@@par\penalty\interlinepenalty\fi
    \fi}%
\end{teX}
    To allow customization we hide the font used in a separate macro.
  \changes{v0.9a}{1993/11/21}{use \cs{verbatim@font} instead of \cs{tt}}
  \changes{v0.9h}{1993/12/13}{Readded \cs{@noligs}}
  \changes{v1.1d}{1996/06/03}{Exchanged the following two code lines
           so that \cs{dospecials} cannot reset the category code
           of characters handled by \cs{@noligs}.}
  \changes{v1.1h}{2000/01/07}{Disable hyphenation even if the font allows it.}
\begin{teX}
  \let\do\@makeother \dospecials
  \obeylines \verbatim@font \@noligs
  \hyphenchar\font\m@ne
\end{teX}
To avoid a breakpoint after the labels box, we remove the penalty
put there by the list macros: another use of |\unpenalty|!
% \changes{v1.0f}{1994/04/29}{Change to \cs{everypar} added}
\begin{teX}
  \everypar \expandafter{\the\everypar \unpenalty}%
}
\end{teX}
\end{macro}
%

\begin{docCommand}{verbatim}{}
\begin{docCommand}{endverbatim}{}
%    (RmS 93/09/19) Protected against `missing item' error message
%               triggered by empty verbatim environment.
\begin{teX}
\def\verbatim{\@verbatim \frenchspacing\@vobeyspaces \@xverbatim}
\def\endverbatim{\if@newlist \leavevmode\fi\endtrivlist}
\end{teX}
\end{docCommand}
\end{docCommand}
%
\begin{macro}{\verbatim@font}
    Macro to select the font  used for verbatim typesetting.
    It also does other work if necessary for the font used.
\begin{teX}
\def\verbatim@font{\normalfont\ttfamily}
\end{teX}
\end{macro}


\begin{environment}{verbatim*}
\begin{teX}
\@namedef{verbatim*}{\@verbatim\@sxverbatim}
\expandafter\let\csname endverbatim*\endcsname =\endverbatim
\end{teX}
\end{environment}

The following code needs no explanation, in our example we collected it at the beginning
of our code with the possible intention to make a small package with utilities.
\begin{macro}{\@makeother}
\begin{teX}
\def\@makeother#1{\catcode`#112\relax}
\end{teX}
\end{macro}

\begin{question}
\begin{tasks}(1)
 \task What is the main disantvantages of the method used here to define
       the |verbatim| environment?
 \task Using our example, try and insert verbatim code into a list. Is this satisfactory.
\end{tasks}
\end{question}


\begin{texexample}{}{}
\begin{itemize}
\item First verbatim
      \begin{verbatim}
      \lorem
      \end{verbatim}
\item Second verbatim
      \begin{verbatim}
      Second verbatim
       \end{verbatim}
\end{itemize}

\end{texexample}

\subsection{Definition of inline verbatim \textbackslash verb}

\begin{macro}{\verb@balance@group}
% \changes{LaTeX2.09}{1993/09/07}
%     {(RmS) Changed definition of \cs{verb} so that it detects a
%              missing second delimiter.}
\begin{teX}
\let\verb@balance@group\@empty
\end{teX}
\end{macro}
%
\begin{macro}{\verb@egroup}
\begin{teX}
\def\verb@egroup{\global\let\verb@balance@group\@empty\egroup}
\end{teX}
\end{macro}
%
\begin{macro}{\verb@eol@error}
\begin{teX}
\begingroup
  \obeylines%
  \gdef\verb@eol@error{\obeylines%
    \def^^M{\verb@egroup\@latex@error{%
            \noexpand\verb ended by end of line}\@ehc}}%
\endgroup
\end{teX}
\end{macro}
%
\begin{macro}{\verb}
% \changes{LaTeX2.09}{1992/08/24}
%         {Changed \cs{verb} and \cs{@sverb} to work correctly
%            in math mode}
% \changes{v0.9a}{1993/11/21}{Use \cs{verbatim@font} instead of
%                             \cs{tt}.}
% \changes{v1.1a}{1995/09/19}{Put \cs{@noligs} after
%                    \cs{verbatim@font} where it belongs.}
%    Typesetting a small piece verbatim.
%  \changes{v1.1d}{1996/06/03}{Put setting of verbatim font after
%           \cs{dospecials}
%           so that \cs{dospecials} cannot reset the category code
%           of characters handled by \cs{@noligs}.}
\begin{teX}
\def\verb{\relax\ifmmode\hbox\else\leavevmode\null\fi
  \bgroup
    \verb@eol@error \let\do\@makeother \dospecials
    \verbatim@font\@noligs
    % common latex method for defining star and unstarred commands
    \@ifstar\@sverb\@verb}
\end{teX}
\end{macro}


\begin{macro}{\@sverb}
\begin{teX}
\def\@sverb#1{%
  \catcode`#1\active
  \lccode`\~`#1%
  \gdef\verb@balance@group{\verb@egroup
     \@latex@error{\noexpand\verb illegal in command argument}\@ehc}%
  \aftergroup\verb@balance@group
  \lowercase{\let~\verb@egroup}}%
\end{teX}
\end{macro}

%
\begin{macro}{\@verb}
\begin{teX}
\def\@verb{\@vobeyspaces \frenchspacing \@sverb}
\end{teX}
\end{macro}
%
\begin{macro}{\verbatim@nolig@list}

\begin{teX}
\def\verbatim@nolig@list{\do\`\do\<\do\>\do\,\do\'\do\-}
\end{teX}
\end{macro}
%
\begin{macro}{\do@noligs}
\begin{teX}
\def\do@noligs#1{%
  \catcode`#1\active
  \begingroup
     \lccode`\~`#1\relax
     \lowercase{\endgroup\def~{\leavevmode\kern\z@\char`#1}}}
\end{teX}
\end{macro}
%
\begin{macro}{\@noligs}
To stay compatible with packages that use |\@noligs| we keep it.
\begin{teX}
\def\@noligs{\let\do\do@noligs \verbatim@nolig@list}
\end{teX}
\end{macro}



\section{The fancyvrb package}


The code works by scanning a line at a time from an environment or a file. This allows it to pre-process each line, rejecting it, removing spaces, numbering it, etc., before going on to execute the body of the line with the appropriate catcodes set.

According to Poore\footcite{poore} the first public release of the package was in January 1998, but from the copyright note its first version must have been in 1994; irrespective the package has remained almost  unchanged except for a few bug fixes. \pkg{fancyvrb} has become one of the primary \latex packages for working with verbatim text.

Poore extended the package with \pkg{fvextra}. All commands defined by |fancyverb| can still be used, but additional
features and keys are made available. The package is used by minted and pythontex which are the only alternatives to
\pkg{listings}.


\begin{texexample}{Using fancyvrb}{ex:fancyvrb}
\begin{Verbatim}
  First verbatim line.
  Second verbatim line.
\end{Verbatim}
\end{texexample}

 \subsection{Fonts}

All four axis of the NFSS scheme can be set using a key value interface.

\begin{texexample}{Using fancyvrb}{ex:fancyvrblarge}
\begin{Verbatim}[commentchar=!,gobble=0,fontfamily=tt]
 A comment
 Verbatim line.
! A comment that you will not see
\end{Verbatim}

\begin{Verbatim}[commentchar=!,gobble=0,fontfamily=tt]
 A comment
 Verbatim line.
! A comment that you will not see
\end{Verbatim}
\end{texexample}



\begin{texexample}{Using fancyvrb}{ex:fancyvrb1}
\begin{Verbatim}[commentchar=!,gobble=0]
 A comment
 Verbatim line.
! A comment that you will not see
\end{Verbatim}
\end{texexample}


\begin{texexample}{Using fancyvrb}{ex:fancyvrb2}
 \fvset{gobble=0}
 \begin{Verbatim}[frame=single,
 label=My text]
 First verbatim line.
 Second verbatim line.
 \end{Verbatim}
\end{texexample}

\begin{texexample}{Using fancyvrb}{ex:fancyvrb3}
  \begin{Verbatim*}
   Verbatim line.
  \end{Verbatim*}
\end{texexample}

\begin{docKey}{defineactive}{ = \meta{code}}{default=[]}
 
\end{docKey}

\begin{texexample}{Adding an active character and a command}{ex:defineactive}
\begin{Verbatim}[frame=single,numbers=left,numbersep=3pt]
! A small "hello" program

program hello
  print *, "Hello World"
\end{Verbatim}

\fvset{frame=single,numbers=left,numbersep=3pt}
\begin{Verbatim}
! A small "hello" program

program hello
  print *, "Hello World"
\end{Verbatim}


\def\ExclamationPoint{\char33}
 \catcode`!=\active
 
\begin{Verbatim}[defineactive=\def!{\color{cyan}\itshape\ExclamationPoint}]
! A small "hello" program

program hello
  print *, "Hello World"
\end{Verbatim}
\end{texexample}

\subsection{Writing and reading verbatim to files.}

To write data verbatim to a file the environment \docAuxEnvironment{VerbatimOut} is available.
It takes one mandatory argument: the file name into which to write the data. If you try and use the
environment directly inside your own environment, the moment we start the |VerbatimOut| environment everythingis absorbed without processing and so the end of your own environment is not recognized.
As a solution the package offers the command |\VerbatimEnvironment|, which is executed within the |\begin|
code of your environment, ensures that the end tag of your environment will be recognized in verbatim mode and the corresponding code executed. To read the file back the command |\VerbatimInput| is used.

\begin{texexample}{Write and read verbatim to file}{ex:vrbout}
\newenvironment{myexample} 
{\VerbatimEnvironment\begin{VerbatimOut}{test.vrb}} 
{\end{VerbatimOut}
\VerbatimInput[numbers=left,firstnumber=2,firstline=2,fontshape=it,fontseries=b]{test.vrb}}% 

\begin{myexample}
first line
second line
third line
fourth line
\end{myexample}
\end{texexample}

Since we can input and type our verbatim text, it opens the oportunity to have self-running examples, similar to the example boxes I have been using for this book.

\begin{texexample}{Self-running examples}{ex:srexamples}
\newcommand\Example{%
\VerbatimEnvironment
\begin{VerbatimOut}{\jobname.vrb}}

\def\endExample{%
\end{VerbatimOut}
{\centering \input{\jobname.vrb}}
\VerbatimInput{\jobname.vrb}
\input{\jobname.vrb}
}

\begin{Example}
    \def\test{Some test}
    \test
\end{Example}

\end{texexample}



\subsection{Saving and re-using verbatim text}

The package can be used to save and re-use verbatim texts in a document. This enables us to place verbatim in normally inaccessible areas such as the argument of sectioning commands (not recommended) or |\marginpar|.

\DescribeMacro{\SaveVerb}{\oarg{key/val list}=data=}{}
The syntax of the command is unusual in that we need to write it the same way as we write
a |\verb| command. It takes one mandatory argument, a \textit{label} denoting the storage
bin in which to save the parsed data. 

\DescribeMacro{\UseVerb*}{\oarg{key/val-list}\Arg{label}}{}

Not recommended for general use.

\begin{texexample}{Saving Verbatim Text}{ex:saveverb}
\begin{minipage}{0.5\linewidth}
 \SaveVerb{danger}=\test \something=
A verbatim \footnote{\UseVerb*{danger}} test in a minpage.
 \end{minipage}
\end{texexample}

In the LaTeX Companion there is an example using the commands to define a macro |\vitem| that can be used
to replace |\item| 


\section{The alltt package}

This package defines the eponymous \docAuxEnvironment{alltt} environment, which is like the \docAuxEnvironment{verbatim} environment except that |\|, |{|, and |}| have their usual meanings.
Thus, other commands and environments can appear within an |alltt| environment.



\begin{texexample}{Using the alltt package}{ex:alltt} 
\begin{alltt} 
\rmfamily
One can have font changes, 
\emph{emphasized text}. 
Some special characters: # & $ ^ _
\lorem 
\end{alltt} 
\end{texexample}


Should you need to typeset mathematical material you will have to use the \latexe constructs |\[|\ldots|\]| as the math toggle |$| is disabled in an |alltt| environment. For subscripts or superscripts you  will need to use the little known \latexe commands \docAuxCommand{sb} or \docAuxCommand{sp} that refer to subscript or superscript respectively.\tcbdocmarginnote{Added Dec 2018}

\begin{texexample}{Using the alltt package}{ex:alltt1} 
\begin{alltt} 
\rmfamily
One can have font changes, 
\emph{emphasized text}. 
Some special characters: # & $ ^ _
\lorem 

\[ a\sb{2} + b\sb{5}  \]

\end{alltt} 
\end{texexample}



