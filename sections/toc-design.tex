\chapter{Designing Lists of Contents}

Figure \ref{fig:tocsteward} shows another |TOC| this time from a mathematics textbook. This is a much more complicated layout and includes images.

If we take a similarly flexible approach of redefining l@chapter we can try and format the toc shown in Figure \ref{fig:tocsteward}.

\begin{figure}[tp]
\centering

\includegraphics[width=0.8\textwidth]{contents02.png}
\caption{Complex table of contents layout.}
\label{fig:tocsteward}
\end{figure}

\begin{figure}[tp]
\centering
\fbox{\includegraphics[width=0.5\textwidth]{contents01.png}}
\caption{Complex table of contents layout.}
\label{fig:toc}
\end{figure}
\begin{figure}[htbp]
\bgroup
\parindent=0pt

\fbox{\includegraphics[width=0.45\textwidth]{./images/content-page-01.pdf}}\hfill\fbox{\includegraphics[width=0.45\textwidth]{./images/content-page-01.pdf}}
\caption{Contents page of Tschitchold's.}

\egroup
\end{figure}
The last example is from an Oxford University Press publication \textit{Portraiture}. It would never pass my mind to design such a Contents page style, but it does look and is a good test for our code  (Figure~\ref{tocsample}). Our spacing looks odd, as the geometry of the page is different as well as the selection of font. This design is found in the full Oxford History of Art series.

\begin{figure}[htbp]
\centering
\includegraphics[width=0.8\textwidth]{oxford-toc}
\caption{Spread with ToC starting page from \textit{Portraiture}.}
\label{tocsample}
\end{figure}

While we at it, let us go over one more example. This time the template is shown in Figure~\ref{fig:reinvent}. This is a ruled example. The top rule will be at the beginning of the ToC and the bottom will be at the beginning of the Part. We will come back for the latter later.

\section{Rules for Good Typesetting}

Perhaps due to reasons of economy, most books will only typeset the chapter titles in the ToC. If the title lengths very, the Contents name on the top of the ToC starting page, will not look if it is centered. Best to have it flush right to balance the page (see Figure~\ref{fig:unlikely}). This is from an Oxford University book \textit{An Unlikely Audience}. It is a minimalistic design, with no need for dot leaders, just the page numbers are closer to the text. 

\begin{figure}[htbp]
\centering
\includegraphics[width=0.5\textwidth]{reinvent}
\caption{Spread with ToC starting page from \textit{Reinvent Yourself}.}
\label{fig:reinvent}
\end{figure}

%\begin{texexample}{Contents heading and hooks}{}
%\bgroup
%\def\doublerule{\hrule width\textwidth height1pt depth0pt\vskip1pt
%     \hrule width\textwidth height3.5pt depth0pt\vskip24pt\relax}
%\cxset{toc name=Contents,
%          toc name case=none,
%          toc name color=black,
%          toc name align=center,
%          toc name indent=, 
%          toc name before= doublerule,
%          toc name font-size=Huge,
%          toc name font-family=rmfamily}
% \sampletoctitle
%\egroup 
%\end{texexample}

What we have just done we defined a double rule macro and just inserted it as the argument before the name hook. The ``contents'' was typeset as is and centered align. 


\begin{figure}[htbp]
\centering
\fbox{\includegraphics[width=0.7\textwidth]{unlikely-audience}}
\caption{Spread with ToC starting page from \textit{An Unlikely Audience}.}
\label{fig:unlikely}
\end{figure}