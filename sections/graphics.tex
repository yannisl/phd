\chapter{Drawing pictures and graphs}
\epigraph{Dear God\break If I have but one hour remaining to live, please allow me to spend this time
in a mathematics class so that it will seem to last forever.}{\textit{---A bored student's prayer}}


\begin{figure}%
  \centering
  \includegraphics[width=0.3\linewidth]{./graphics/pic37.png}
  \caption{During the early days of typography fonts were designed to emulate the looks of calligraphic texts.}
  \label{fig:marginfig1}
\end{figure}

\parindent=0em
\parskip=0.25\baselineskip plus .25pt minus .25pt\relax

\section{Inserting figures}

In order to insert figures, the \pkgname{graphicx} package has to included in the preamble (before the |\begin{document}|-command) of your LaTeX-document:

\begin{verbatim}
\usepackage{graphicx}
\end{verbatim}

Originally only EPS-figures could be inserted with the \pkgname{graphic}package. This has now been developed into the  \pkgname{graphicx}, which allows almost any common format to be inserted. 

The simplest way of including a graphic looks like this:


\CMDI{\includegraphics}\marg{filename}


If the image is not located in the same folder as the tex-file, you will have to specify the path relative to the tex-file.

\begin{verbatim}
\includegraphics{./images/filename}
\end{verbatim}


\subsection{Scaling and resizing images}

If you want the image to appear in a different size, you can specifiy the size as a parameter of the |\includegraphics|-command::

\begin{commands}[]{ex:graphics}
\cmd{\includegraphics}\oarg{width=3.9cm}\marg{filename}
\end{commands}

This will scale the image to the width of 3.9 centimeters. 

Use |\textwidth| command if you don't want to specify an absolute size but rather want the actual size to depend on the text width of the page. You can use any of the normal \tex units such as \texttt{em, pt, cm, in}:

\begin{commands}[]{ex:graphics}
\cmd{\includegraphics}\oarg{width=0.5\string\textwidth}\marg{filename}
\end{commands}

\noindent will scale the image to half of the text width. The images in the
figure below were produced by three |\includegraphics| commands. You can have as many as you like and the \tex engine will treat them the same way as text. If you a leave a space between the commands, they will be positioned vertically as they are treated as paragraphs.

\medskip

\begin{commands}[]{}
\begingroup

\centering
\includegraphics[width=0.3\textwidth]{./graphics/amato.jpg}
\includegraphics[width=0.3\textwidth]{./graphics/amato.jpg}
\includegraphics[width=0.3\textwidth]{./graphics/amato.jpg}

\endgroup

\begin{verbatim}
\begingroup

\centering
\includegraphics[width=0.3\textwidth]{./graphics/amato.jpg}
\includegraphics[width=0.3\textwidth]{./graphics/amato.jpg}
\includegraphics[width=0.3\textwidth]{./graphics/amato.jpg}

\endgroup
\end{verbatim}
\captionof{figure}{Images aligned horizontally.}
\end{commands}

The three photos were centered using the |\centering| command, within a group. The |\begingroup..\endgroup| is necessary to limit the effect of centering to
the group only, otherwise \tex would center everything from this point onwards.

\subsection{Controlling the aspect ratio}

You can control the picture aspect ratio by using the command:

\begin{commands}[]{}
\cmd{\includegraphics}\marg{keepaspectratio,width=3cm, height=3cm}\oarg{filename}
\end{commands}

If the key |keepaspectratio| is set to true then specifying 
both |width| and |height| (or |totalheight|) does not distort the figure but 
scales such that neither of the specified dimensions is exceeded.

\medskip
\begin{commands}[]{}
\begingroup

\centering
\includegraphics[width=0.3\textwidth, height=5cm]{./images/amato.jpg}
\includegraphics[keepaspectratio=true,width=4cm, height=5cm]{./images/amato.jpg}
\includegraphics[width=3cm]{./images/amato.jpg}

\endgroup

\begin{verbatim}
\begingroup

\centering
\includegraphics[width=0.3\textwidth, height=5cm]{./images/amato.jpg}
\includegraphics[keepaspectratio=true,width=4cm, height=5cm]{./images/amato.jpg}
\includegraphics[width=3cm]{./images/amato.jpg}

\endgroup

\end{verbatim}
\captionof{figure}{Controlling the aspect ratio.}
\end{commands}

This can be very useful if you have images shown side by side with different
aspect ratios. 


\subsection{Paths and file types}

For larger projects you will probably find it more convenient to have 
images in different folders. You can specify default paths using:


\CMDI{\graphicspath}\marg{dir-list}

This optional declaration may be used to specify a list of directories in which to
search for graphics files. The format is the same as for the \latexe primitive
|\input@path|. A list of directories, each in a \{\} group (even if there is only one
in the list). For example:


\graybox{\texttt{\textbackslash graphicspath\{\{eps/\}\{tiff/\}\}}}


The default image formats can be declared using:

\CMDI{\DeclareGraphicsExtensions}\marg{png, jpg}

This specifies the behaviour of the system when no file extension is specified in 
the argument to |\includegraphics|. \texttt{\{ext-list\}} should be a comma separated 
list of file extensions. (White space is ignored between the entries.) A file name
is produced by appending one extension from the list. If a file is found, the
system acts as if that extension had been specified. If not, the next extension
in \texttt{ext-list} is tried.



\subsection{The figure environment}

You use the figure-environment to let your image appear in a \emph{floating} environment, that is \latex will place it at the right position of a page and even on the next page:

\begin{teX}
\begin{figure}
  \includegraphics{filename.jpg}
  \caption{title of your figure}
  \label{labelname}
\end{figure}
\end{teX}

Here |\caption{...}| defines the title of the figure which will appear beneath the figure. |\label{..}| defines the label which can be used inside the document in order to insert references to the figure:

The figure

|\ref{labelname} on page \pageref{labelname} ..|

The|\label-command| inside the |\figure|-envirnonment hast to appear just after the|\caption|-command.
placing figures

If figures reside inside a |\figure|-environment, this will cause LaTeX to choose the actual location of the figure inside the document. There are different parameters for the placement strategy:

\begin{description}
\item[h (here)] Try to place the figure just where the command is located.

\item [t (top)] Try to place the figure at the top of the page.

\item[b (bottom)] Try to place the figure at the bottom of the page.

\item [p (float page)] Try to place the figure on a page which contains only floating elements.
\end{description}

The order of these parameters doesn't matter since placement is always tried in the order \textbf{h, t, b, p,} if these parameters are present:

If no parameter is present, the default order is  \texttt{[tbp]}.


The command for a figure-environment might for example look like this:

\begin{teX}
\begin{figure}[htbp]
...
\end{figure}
\end{teX}



\subsection{Table of figures}
\index{figures!Table of figures}
A table of figures is inserted (where you place the command) using the command


\begin{teX}
   \listoffigures
\end{teX}

The caption given in the \cmd{caption} command is also used in the list of figures. 
If you want to use different captions, you may add a parameter to the |\caption| command:
|\caption[caption for listoffigures]{caption inside the document}|


\subsection{Figures with a border}

Although drawing frames around tables should be discouraged, if you find the need
to draw them there are  two possible ways to achieve it: either only the figure itself is bordered or there is a border around the figure and its caption. You place a border around the figure using the \cmd{\fbox} command or the \cmd{\framebox}.

\emphasis{fbox,minipage}
\begin{teX}
\begin{figure}[htbp]
  \centering
  \fbox{
    \includegraphics{filename}
  }
  \caption{caption}
  \label{Labelname}
\end{figure}
\end{teX}

Placing a border around the figure and its title is a little more tricky: You need to place the figure and the title in a |\minipage| environment which is bordered again with the |\fbox| command:

\begin{figure}[htbp]
\begin{commands}[]{}
\centering
  \fbox{
    \begin{minipage}{.95\linewidth}
      \mbox{}
      \centering
      
      \includegraphics[width=.9\linewidth]{./images/asia.jpg}
      \caption{How to place a border around an image. }
      \label{labelname}
    \end{minipage}}
 
\begin{verbatim}
\begin{figure}[htbp]
  \centering
    \begin{minipage}{width=.8\linewidth}
     \centering
     
      \includegraphics[.9\linewidth]{filename}
      \caption{caption}
      \label{labelname}
    \end{minipage}
 \end{figure}
\end{verbatim}
\end{commands}
\end{figure}



Unfortunately the width of the border cannot be determined automatically. It has to be specified as a parameter of the |\minipage| environment. However, you may be bale to develop a macro to do this,  based on the ImageSize routines we developed in section.


\section{Complex Layouts}
\label{looting}
In reality most professionally typeset books will have their own style for image pages. In Figure~\ref{complex}
three images are set in a non-symmetrical layout. This type of setting is difficult to automate and manual intervention is possible.

This layout will require four minipages. Two for the top figure (one for the image and one for the caption) and two for the two bottom figures. The rightmost bottom figure will have to be put in a zero height box to let it overflow to the top. The figure has been reproduced from an Oriental Institute publication \emph{Catastrophe! The Looting and Destruction of Iraq’s Past} \cite{looting}. The book is interesting both for its contents as well as its simple but effective typography and appropriate for the topic. The volume has been pblished in conjuction with the exhibition titled as the name of the book, that described the loss of Iraq’s archaeological past to looters and to the war. 

\begin{figure}[p]
\centering
\includegraphics[height=0.8\textheight]{oriental}
\caption{More complex layouts. \emph{Copyright the Oriental Institute of the University of Chicago.}}
\label{complex}
\end{figure}

The style is reproduced in a \pkgname{phd} template (style 56) and both code and details can be found in the relevant pages.





\section{Side by side figures}

You might want to place to figures side by side but to use only one caption. This is achieved by placing both figures in its own |\minipage| which reside in the same |\figure|.

if only one |\caption| command is used, both figures will have a common title:

\medskip
\begin{verbatim}
\begin{figure}[htbp]
  \centering
  \begin{minipage}[b]{5 cm}
    \includegraphics{filename 1}  
  \end{minipage}
  \begin{minipage}[b]{5 cm}
    \includegraphics{filename 2}  
  \end{minipage}
  \caption{common caption}
  \label{Labelname}
\end{figure}
\end{verbatim}
\medskip

The first parameter of the |\minipage| environment determines how both graphics are aligned to each other. b (bottom) aligns the bottom borders of the figures, \textbf{t} (top) aligns the top borders and \textbf{c} aligns the centers.

If you want distinct titles for the two figures you will only have to supply a |\caption| command for both |\minipage|environments:

\begin{teX}
\begin{figure}[htbp]
  \centering
  \begin{minipage}[b]{5 cm}
    \includegraphics{filename 1} 
    \caption{caption 1}
    \label{labelname 1}
  \end{minipage}
  \begin{minipage}[b]{5 cm}
    \includegraphics{filename 2}  
    \caption{caption 2}
    \label{labelname 2}
  \end{minipage}
\end{figure}
\end{teX}


If you want to have subfigures with distinct caption, you use the |\subfig| package:


You can put as many figures as you like on a page, but a word of warning, you may need to make some manual adjustments before you get it right. The package provides support for the manipulation and reference of small or ‘sub’ floats within a single floating (e.g., figure or table) environment1 It is convenient to use this
package when your sub-floats are to be separately captioned, referenced, or when such
sub-captions are to be included on a List-of-Floats page.

The package is a replacement for the subfigure package, from which it was derived.
However, the new subfig package is not completely backward compatible.
Therefore, a new name was called for. The newer package is smaller and easier to use
than the older package, however, it now uses the following additional packages, 
caption (required), 
everysel (optional), 
keyval (required), 
ragged2e (optional).

It will work without the \pkgname{ragged2e} and \pkgname{everysel} packages if you do not use the following
justification options: ‘Center’, ‘RaggedRight’ and ‘RaggedLeft’. The other justification
options ‘center’, ‘raggedright’ and ‘raggedleft’ will work without the above two packages. If the ragged2e package is present, than the caption package will load it and it
will, in turn, load the everysel package. This happens whether or not you will be using
the justification options that require it. If it cannot find the ragged2e package, than the
caption package will print a message that ‘RaggedRight’, etc. will not be available.


\begin{figure}[htb]
\includegraphics[height=5cm]{dotty}
\includegraphics[height=5cm]{bette}
\includegraphics[height=5cm]{dotty}
\end{figure}

 A low bottle-shaped vase, of yellowish ware, with flaring rim and somewhat flattened body. Height, 5 inches; width 5 inches. \ref{fig:one}

A well-made bottle shaped vase, with low neck and globular body, somewhat conical above. Color dark brownish. 7½ inches in height. Shown in \ref{fig:two}


\begin{figure}
  \centering
  \includegraphics[width=0.7\linewidth]{./graphics/fig175.jpg}
   \centerline{From the tomb of a Pull\= arius.}
  \label{fig:marginfig1}
\end{figure}

The above figure is an effigy vase of the dark ware. The body is globular. A kneeling human figure forms the neck. The mouth of the vessel occurs at the back of the head—a rule in this class of vessels. Is is finely made and symmetrical. 9¾ inches high and 7 inches in diameter. being larger than the above two it is preferable to scale it to give the reader an indication. Based on the figure width, you may also need to adjust the distance between the figures to ensure that the whitespace is just about right. For screen reading this can be increased and for printed works you may wish to make it less.



\section{The wrapfig package}


\captionsetup[wrapfigure]{margin=10pt,font=small,labelfont=bf, name=Fig.} % [wrapfigure]{name=Fig.}


Donald Arseneau has created the \pkg{wrapfig} package to allow people to place figures or
tables at the side of a page and wrap text around them. The package provides the
environments wrapfigure and wraptable. Both environments have two required and
two optional arguments. You can see an example taht uses the package to wrap a picture into such a paragraph of text.

\begin{figure}[htbp]
   \includegraphics[width=\linewidth]{./graphics/cyprus.jpg} 
   \caption{\small Cyprian limestone group of Phoenician dancers, about 6½ in. high. There is a somewhat similar group, also from Cyprus, in the British Museum. The dress, a hooded cowl, appears to be of great antiquity.}
\end{figure}

\begin{wrapfigure}[20]{l}{3.8cm}
\centering\small
\includegraphics[width=\linewidth]{./graphics/egyptdance.jpg}  
\caption{\small The hieroglyphics describe the dance.}
\end{wrapfigure}
Amongst the earliest representations that are comprehensible, we have certain Egyptian paintings, and some of these exhibit postures that evidently had even then a settled meaning, and were a phrase in the sentences of the art. Not only were they settled at such an early period (B.C. 3000, fig. 1) but they appear to have been accepted and handed down to succeeding generations (fig. 2), and what is remarkable in some countries, even to our own times. The accompanying illustrations from Egypt and Greece exhibit what was evidently a traditional attitude. The hand-in-hand dance is another of these.

The earliest accompaniments to dancing appear to have been the clapping of hands, the pipes,[1] the guitar, the tambourine, the castanets, the cymbals, the tambour, and sometimes in the street, the drum.

The following account of Egyptian dancing is from Sir Gardiner Wilkinson's "Ancient Egypt"[2]:—
\begin{figure}
   \includegraphics[width=0.3\linewidth]{./graphics/lotus.jpg} 
   \caption{\small Cyprian limestone group of Phoenician dancers, about 6½ in. high. There is a somewhat similar group, also from Cyprus, in the British Museum. The dress, a hooded cowl, appears to be of great antiquity.}
\end{figure}
"The dance consisted mostly of a succession of figures, in which the performers endeavoured to exhibit a great variety of gesture. Men and women danced at the same time, or in separate parties, but the latter were generally preferred for their superior grace and elegance. Some danced to slow airs, adapted to the style of their movement; the attitudes they assumed frequently partook of a grace not unworthy of the Greeks; and some credit is due to the skill of the artist who represented the subject, which excites additional interest from its being in one of the oldest tombs of Thebes (B.C. 1450, Amenophis II.). Others preferred a lively step, regulated by an appropriate tune; and men sometimes danced with great spirit, bounding from the ground, more in the manner of Europeans than of Eastern people. On these occasions the music was not always composed of many instruments, and here we find only the cylindrical maces and a woman snapping her fingers in the time, in lieu of cymbals or castanets.

\begin{figure}
   \includegraphics[width=0.3\linewidth]{./graphics/patera.jpg} 
   \caption{\small Cyprian limestone group of Phoenician dancers, about 6½ in. high. There is a somewhat similar group, also from Cyprus, in the British Museum. The dress, a hooded cowl, appears to be of great antiquity.}
\end{figure}

"Graceful attitudes and gesticulations were the general style of their dance, but, as in all other countries, the taste of the performance varied according to the rank of the person by whom they were employed, or their own skill, and the dance at the house of a priest differed from that among the uncouth peasantry, etc.

"It was not customary for the upper orders of Egyptians to indulge in this amusement, either in public or private assemblies, and none appear to have practised it but the lower ranks of society, and those who gained their livelihood by attending festive meetings.

"Many of these postures resembled those of the modern ballet, and the pirouette delighted an Egyptian party 3,500 years ago.
\medskip

The wrapped figure is positioned using the \texttt{wrapfigure} environment, as shown below:

\begin{teX}
\begin{wrapfigure}[nlines]{placement}[overhang ]{width }
   \includegraphics[width=3.8cm]{./graphics/egyptdance} 
   \caption{\small The hieroglyphics describe the dance.}
\end{wrapfigure}
\end{teX}

The parameter |nlines|  is the number of narrow lines, and placement is one of r, l, i, o, R, L, I, or
O for right, le, inside, and outside, respectively. The uppercase placement specifiers
differ from their lowercase counterparts in that they force \latex to put the float \emph{here},
whereas the lowercase placement specifiers just give a hint to \latex to place them
\texttt{here}. The \meta{width} argument is the width of the figure or table that appears in the body
of the environment. Finally, \texttt{overhang} tells \latex how much the figure should hang out
into the margin of the page. Here is how one may create dangerous paragraphs bends!

The |wrapfig| package is compatible with the |caption| package. You can set the caption parameters using:---

\begin{teX}
\captionsetup[wrapfigure]{<options>}
\end{teX}

If you are probably wondering how |wrapfig| achieves this, you should read the package code. It basically uses \refCom{everypar}, and hence the limitations with |\par|. Here is an extract from the class.

\begin{teX}

% Subvert \everypar to float fig and do wrapping.  
% Also for non-float.
\def\WF@startfloating{%
 \WF@everypar\expandafter{\the\everypar}\let\everypar\WF@everypar
 \WF@@everypar{\ifvoid\WF@box\else\WF@floathand\fi \the\everypar
 \WF@wraphand
}}
\end{teX}

Moving now to a more scientific example that the previous ones, we will place two figures
one on top of each other and give them individual, sub-captions as shown in \ref{fig:honey}.
 
\captionsetup[figure]{margin=10pt,font=small,labelfont=bf,format=hang}%

\begin{figure}[htbp]
\centering
  \begin{subfigure}[b]{0.5\textwidth}
  \includegraphics[width=\linewidth]{./graphics/honey.png}
  \caption{Taylor instability in the surface of the honey in an inverted honey jar.}\label{fig:honey}
    \hspace{1cm}
  \end{subfigure}

  \begin{subfigure}[b]{0.9\textwidth}
     \centering
     \includegraphics[width=9cm]{./graphics/honeydrops.png}
     \caption{Taylor instability in the interface of the water condensing on the underside of a small water pipe.}
  \end{subfigure}  
  \caption{Two examples of Taylor instabilities that are commonly found.}%
    \label{fig:Athird}%
\end{figure}

The figures are from \textit{A Heat Transfer Textbook}, by J.H.Lienhard, which incidentally was typeset using
\tex . It is a McGrawHill publication. 

\begin{teX}
\begin{figure}[htbp]
    \captionsetup[figure]{margin=10pt}%
    \subfloat[Taylor instability...]
     {{\includegraphics[width=8cm]{./graphics/honey}}}
    \hspace{1cm}
     \subfloat[Taylor instability in the...]%
      {\includegraphics[width=9cm]{./graphics/honeydrops}}  
     \\[-10pt]
   \caption{Taylor instability in...}%
    \label{fig:Afirst}%
    \caption{Two examples of... }%
    \label{fig:honey}%
\end{figure}
\end{teX}


The text can have more than one paragraph. It is also possible to include figures
generated by |TikZ/pgf|, as shown in the next example, drawn from real code
in the book.


\begin{wrapfigure}[14]{l}{3.0cm}
\pgfplotsset{width=5.0cm,compat=1.3}
\begin{tikzpicture}
\begin{axis}[minor y tick num=4, 
minor x tick num=4, 
xmin=0,xmax=300,
ymin=0,ymax=60,
xlabel=\textsf{liquidus ($l/s$)},
ylabel=\textsf{capitis ($m$)}, 
ytick={0,15,30,45,60,75},
xtick={0,100,200,300}
]
\addplot[color=blue,mark=x, smooth] coordinates {
(0,44)
(50,43)
(100,42)
(150,40)
(200,33)
(220,29)
};

\end{axis}
\end{tikzpicture}
\caption{Pump headum and flowm}
\end{wrapfigure}


\providecommand\addcredit[1]{%
 \vspace*{-6.5pt}
 \scriptsize%
 \flushright%
 \textit{Credit: #1}%
}

\begin{figure}[htp]
\centering

\captionsetup{name=Photo., labelsep=period}%
   \begin{minipage}[t]{0.48\textwidth}
      \includegraphics[width=\textwidth]{./graphics/movingup.jpg}%
      \addcredit{U.S. DoD.}%
     \caption{The effects of the credit going past the edge of the figure. This can be corrected by adding a minipage to hold both commands. }
\end{minipage}\hfill\hfill
\begin{minipage}[t]{0.48\textwidth}
      \includegraphics[width=\textwidth]{./graphics/survivors.jpg}%
      \addcredit{U.S. DoD.}%
    {\footnotesize Marines awaiting resting before moving on to Japan. }
\end{minipage}

% \begin{minipage}[t]{0.48\textwidth}
%      \includegraphics[width=\textwidth]{./graphics/img009.jpg}%
%      \addcredit{U.S. DoD.}%
%     \caption{Engineer Construction Troops in Liberia, July 1942.}
%\end{minipage}\hfill\hfill
%\begin{minipage}[t]{0.48\textwidth}
%      \includegraphics[width=\textwidth]{./graphics/survivors.jpg}%
%      \addcredit{U.S. DoD.}%
%     \caption{The effects of the credit going past the edge of the figure. This can be corrected by adding a minipage to hold both commands. }
%\end{minipage}
% \begin{minipage}[t]{0.48\textwidth}
%      \includegraphics[width=\textwidth]{./graphics/img126.jpg}%
%      \addcredit{U.S. DoD.}%
%     \caption{Marine Reinforcements.
%A light machine gun squad of 3d Battalion, 1st Marines, arrives during the battle for ``Boulder City.'' }
%\end{minipage}\hfill\hfill
%\begin{minipage}[t]{0.48\textwidth}
%      \includegraphics[width=\textwidth]{./graphics/img124.jpg}%
%      \addcredit{U.S. DoD.}%
%     \caption{Brothers Under the Skin, inductees at Fort Sam Houston, Texas, 1953. }
%\end{minipage}
\end{figure}
\newpage


Armed with all these packages you can help the Gutenburg organization to transcribe
some of the old books that they have online. 

\clearpage







