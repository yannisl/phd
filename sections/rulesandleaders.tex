%\cxset{custom = fashion,
%          fashion image=./images/venus.jpg}

\chapter{Rules and Leaders}
\pagestyle{headings}

\epigraph{He had forty-two boxes, all carefully packed,
With his name painted clearly on each:
But, since he omitted to mention the fact,
They were all left behind on the beach.}{---Lewis Carroll, The Hunting of the Snark}

\section{Rules}

Rules, both horizontal and vertical, are traditionally used in typesetting. In
\tex, a rule does not necessarily have to be long and thin; it has three dimensions,
like a box, and can have any rectangular shape. There are two types of rules, |\hrule| and |\vrule|.

\begin{docCommand}{hrule}{ height\meta{dimen} width \meta{dimen} depth\meta{dimen} }
Draws a rule in vertical mode.
\end{docCommand}

\begin{docCommand}{vrule}{ height\meta{dimen} width \meta{dimen} depth\meta{dimen} }
Draws a rule in horizontal mode.
\end{docCommand}

The shape of the rule does not depend on whether it is \textsc{h} or \textsc{v}, and the difference
between the two types is in the context in which they can be used, not in their
shapes. An |\hrule| is considered vertical material and can be part of a vertical list.

A |\vrule| is the opposite and can only appear in horizontal lists. The reason for
this convention is that a horizontal rule is a good separator between items stacked
vertically, whereas a vertical rule is a natural separator for items laid horizontally,
from left to right.

As a result, a |\vrule| should be used inside a paragraph, such as this \vrule, or in
an |\hbox|. An |\hrule| should be used between paragraphs or in a |\vbox|.

Any unspecified dimensions of a rule are determined [221] by these defaults:

\begin{enumerate}
\item The height of an |\hrule| is 0.4pt, and the depth is 0pt.
\item The width of a |\vrule| is 0.4pt.
\item Other dimensions are determined by extending the rule to the size of the smallest
box containing it. An example of this rule is the |\vrule| above. Its depth is set
equal to the depth of the line it happens to be on.
\end{enumerate}



The rule is extended to the width {\Huge \drawfontframe{\vbox{\hsize=24pt\parindent0pt p\hrule*}}}

\paragraph{Struts} The word \emph{strut} has already been mentioned. It refers to a \refCom{vrule} with width zero. It refers to a |\vrule| with
width zero. A standard strut is part of the plain format and is defined, on [353], as
|\vrule height8.5pt depth3. 5pt width0pt| (the actual definition is slightly more
complicated and takes into account the current mode). Such a rule does not show
up in print and is used to open up boxes. Inexperienced users find it hard to believe
that such a rule can be useful, but a glance at [478] shows that it is one of the most
frequently mentioned terms in the \texbook.

A horizontal strut can also be defined. It is an |\hrule| with height and depth
of zero. Surprisingly, such a thing is rarely used (but see discussion of |\hphantom|
in section 3.24)

\begin{texexample}{Drawing a Ruler}{ex:ruler}
\bgroup

\def\1{\vrule height 0pt depth 2pt}

\def\2{\vrule height 0pt depth 4pt}

\def\3{\vrule height 0pt depth 6pt}

\def\4{\vrule height 0pt depth 8pt}

\def\ruler#1#2#3{%
    \leftline{$\vcenter{%
    \hrule\hbox{\4#1}}\,\,\rm#2\,{#3}$}}%
  
\def\\#1{\hbox to .125in{\hfil#1}}
  
\def\8{\\\1\\\2\\\1\\\3\\\1\\\2\\\1\\\4}%
  
\ruler{\8\8\8\8}4{in}
\egroup
\end{texexample}

Lamport in \latex developed a macro |\rule| to enable users to draw lines without remembering all the rules for horizontal or vertical modes and the like.\footnote{In the latest releases this has been changed to a robust macro, using \textbackslash DeclareRobustCommand.}

\begin{docCommand}{rule}{\oarg{raised}\marg{width}\marg{height} }
Typesets a rule with a  \meta{width} and\meta{height}, raised by \meta{raised}.
\end{docCommand}

\begin{teX}
\def\rule{\@ifnextchar[\@rule{\@rule[\z@]}}%
\def\@rule[#1]#2#3{%
\leavevmode
\hbox{%
  \setlength\@tempdima{#1}%
  \setlength\@tempdimb{#2}%
  \setlength\@tempdimc{#3}%
  \advance\@tempdimc\@tempdima
  \vrule\@width\@tempdimb\@height\@tempdimc\@depth-\@tempdima}
}
\end{teX}

The important macro is |@rule| which sets the lengths and widths to the parameters required by the user. The raising of the rule is achieved by adjusting the depth to the given amount of length to raise the rule.

This is a Lamport rule |\rule[6.5pt]{4pt}{7pt}| typeset as:\rule[6.5pt]{4pt}{7pt} Many \latexe packages 
provide rules for common cases, such as \pkg{booktabs} providing rules that can be used in tables. 

Another useful \latexe macro is |\underline| that can be used to underline text. The \latex version is a modification of the \textsc{plain} version to enable it to be used in math mode. The \textsc{plain} version can still be used in \latexe by using |\@@underline|. 

\section{Applications}

One example of \refCom{vrule} is to provide the color background of a box. This method is used for
example by the \pkg{xcolor} to provide generic drivers. First a |vrule| with the require box dimensions
is typeset in a zero width box using \refCom{rlap} and then the text is overwritten to provide the typeset box, with a background color. One can extend such macros to draw numerous lines at different colors to also 
achieve  a gradient effect.


\begin{texexample}{}{}
\makeatletter
\bgroup
\renewcommand*\color@block[3]%
{{%
\color{blue}%
    \rlap{%
      \ifcolors@
        \vrule\@width#1\@height#2\@depth#3
      \fi
    }%
}} 
\hbox{\color@block{80pt}{30pt}{3.5pt}%
      \sffamily\bfseries\Huge\color{white}FFji}
\egroup 
\makeatother 
\end{texexample}

Of course the example is trivial. In a more detailed macro, it would be preferable to measure the dimensions
of the text and size the background accordingly. 

\section{Leaders}

A leader is a single copy of a pattern, for example in a dashed line a dash is a leader.
Dot leaders are a row of dots that visually connect the chapter titles and section headings to their corresponding page numbers. 

Leaders don't have to be composed of dots, with \tex leaders can be used fill a space with copies of a pattern,
\eg, to put repeated dots between a title and a page number in a table
of contents. 

The Plain Format provides six standard leader definitions. All these definitions are equivalent to an |\hfill| type of horizontal glue.

\medskip


\begin{tabular}{lp{3cm}}
\docAuxCommand{hrulefill}     & \hrulefill\\[0.3em]
\docAuxCommand{dotfill}        & x\dotfill x \\[0.3em]
\docAuxCommand{leftarrowfill} & \leftarrowfill\\[0.3em]
\docAuxCommand{rightarrowfill} & \rightarrowfill\\[0.3em]
\docAuxCommand{downbracefill} & \downbracefill\\[0.3em]
\docAuxCommand{upbracefill} & \upbracefill\\
\end{tabular}

\bigskip


A leader is a single copy of the pattern. The specification of
leaders contains three pieces of information:

\begin{enumerate}
\item  what a single leader is
\item  how much space needs to be filled
\item  how the copies of the pattern should be arranged within the space
\end{enumerate}

In \tex leaders are actually \emph{visual glue}. Wherever glue can go a row of leaders can go.

\begin{texexample}{Leaders}{ex:leaders}
\meaning\dotfill  \par
\meaning\hrulefill\par
\meaning\downbracefill\par
\end{texexample}

\begin{docCommand}{leaders}{}
\tex applies an imaginary window and only those leader boxes are printed which fully fit into the window. This ensures that the leader dots of different lines line up vertically.
\end{docCommand}


\begin{docCommand}{cleaders}{}
\end{docCommand}

\begin{docCommand}{xleaders}{}
\tex  provides three commands for specifying leaders:\cs{leaders},\cs{cleaders},
and\cs{xleaders} (p.~174). The argument of each command specifies the
leader. The command must be followed by glue; the size of the glue specifies
how much space is to be filled. The choice of command determines how
the leaders are arranged within the space.
\end{docCommand}

Rule leaders \textit{fill} the specified amount of space with a rule extending in the direction of the skip
specified. \index{rules and leaders>rule leaders}

The most common application for leaders is to fill the space with either a rule or with dots, such as shown in Example~\ref{leaders} below.

\emphasis{leaders,hbox,hfill}
\begin{texexample}{Leader example}{leaders}
\hbox{Exa\leaders\hrule\hskip20pt e}
\hbox to \linewidth{Section 1.2 \leaders\hbox{..}\hfill\space 15}
Section 1.3 \leaders\hbox{..}\hfill\space 15

\parfillskip=0pt plus1fil

\lipsum*[1]\leaders\hbox{..}\hfill\space 15
\end{texexample}

Leaders must be in a box, such as an \cs{hbox}. If they are not in a box an error is issued by \tex.

\begin{texexample}{}{hboxleaders}
\hbox to \textwidth{g\leaders\hbox{+}\hfill 112}
\end{texexample}

because a horizontal rule has a default height of |.4pt|. On the other hand,\index{Rules and Leaders>default value}

\verb+\hbox{g\leaders\vrule\hskip10pt f}+

gives

\hbox{g\leaders\vrule\hskip10pt f}

because the height and depth of a vertical rule by default fill the surrounding box.
Spurious rule dimensions are ignored: in horizontal mode

\verb+\leaders\hrule width 10pt \hskip 20pt+

is equivalent to

\verb+\leaders\hrule \hskip 20pt+

If the width or height-plus-depth of either the skip or the box is negative, TEX uses ordinary glue
instead of leaders.

\section{Box leaders}
\index{leaders box}
Box leaders fill the available spaces with copies of a given box, instead of with a rule. The first example uses \latex3 syntax, which is bound to send old \tex masters into an apoplectic fit. However, once your eyes
and brains absorb the syntax, \latex3 is too good to be ignored and can be mastered in a month or so. The
underscores still bother me, as well as the Hungarian notation, but I have mellowed as I grew older and
have now accepted it as an essential toolbox for latexing.

The reason I introduced it here, is to get you used to it for the next chapter, which is dedicated to \latex3 boxes and skips. This will bring us to a full round. We have studied the original \tex and plain format commands, the \latex2e and next the \latex3 macros. 

\begin{texexample}{Box leaders}{}
\ExplSyntaxOn  
  \box_new:N \starbox
  %\setbox\starbox=\hbox:n{
  \hbox_set:Nn \starbox 
    {
      \skip_horizontal:n { .2em  }
      \box_move_down:nn { 2.5pt }
                        {\hbox:n{*}}
      \skip_horizontal:n {.2em}
    }

  
  \hbox_to_wd:nn {\textwidth} 
    {
       \null \tex_leaders:D\box_use:N \starbox \hfill \null
    }.
\ExplSyntaxOff
\end{texexample}

If you notice you have to use the \cs{copy} command rather than \cs{usebox}, as we cannot use the |\leavevmode| with leaders

\begin{verbatim}
\usebox unchanged
81 \def\usebox#1{\leavevmode\copy #1\relax}
\end{verbatim}

That is, copies of the box register fill up the available space.

Dot leaders, as in the above example, are often used for tables of contents. In such applications it
is desirable that dots on subsequent lines are vertically aligned. The\cs{leaders} command does this
automatically:


The mechanism behind this is the following: TEX acts as if an infinite row of boxes starts (invisibly)
at the left edge of the surrounding box, and the row of copies actually placed is merely the part of
this row that is not obscured by the other contents of the box.

Stated differently, box leaders are a window on an infinite row of boxes, and the row starts at the
left edge of the surrounding box. Consider the following example:

\begin{texexample}{}{}
\hbox to 8cm {\leaders\copy\centerdot\hfil}
\hbox to 8cm {word\leaders\copy\centerdot\hfil}
\end{texexample}

which gives,

\hbox to 8cm {\leaders\copy\centerdot\hfil}
\hbox to 8cm {word\leaders\copy\centerdot\hfil}

The row of leaders boxes becomes visible as soon as it does not coincide with other material.
The above discussion only talked about leaders in horizontal mode. Leaders can equally well be
placed in vertical mode; for box leaders the \textit{infinite row} then starts at the top of the surrounding
box.


\begin{docCommand}{cleaders}{}
\begin{docCommand}{xleaders}{}
The \cs{cleaders} command is similar to 
\cs{leaders}, but it splits excess space before and after the leaders into two equal parts, centring the row of boxes in the available space.
The \cs{xleaders} command is also similar, but spreads the space between and after the leaders evenly between all the boxes.
\end{docCommand}
\end{docCommand}

The differences are best explained with an example.

\emphasis{leaders,cleaders,xleaders}
\begin{texexample}{}{}
\def\leaderpattern{\hbox{\kern0.5em-\kern0.5em-\kern0.5em-}}
Lorem \leaders\leaderpattern\hfill 13\par
Lorem \cleaders\leaderpattern\hfill 13\par
Lorem \xleaders\leaderpattern\hfill 13\par

\meaning\xleaders
\end{texexample}




\section{Vertical leaders}

If vertical glue commands such as \cs{vfill} is used it is possible to have
vertical leaders. In Example~\ref{vleaders} we use a centered dot \cs{cdot} to fill the space between two paragraphs with leaders. We define a command
\cs{vdotfill} to do this that contains the instructions.

\begin{texexample}{Vertical leaders}{vleaders}
\newcommand{\vdotfill}{%
  \par\leaders\hbox{$\cdot$}\vfill}
  \vbox to 5cm {%
  \lorem
  \vdotfill
  \lorem
  }
\end{texexample}





\section{Leaders and shifted margins}

If margins have been shifted, leaders may look different depending on how the shift has been realized.
For an illustration of how\cs    {hangindent} and\cs{leftskip} influence the look of leaders, consider
the following examples, where

\begin{texexample}{Ratata}{ex:ratata}
\setbox0=\hbox{R a t a t a  }
\verb+\setbox0=\hbox{R a t a t a  }+



\hbox{\kern1em\hbox{\leaders\copy0\hskip5cm}}

\hangindent=1em \hangafter=-1 \noindent
\leaders\copy0\hskip5cm\hbox{}\par
\end{texexample}

gives (note the shift with respect to the previous example)
\medskip

{\hbox{\kern1em\hbox{\leaders\copy0\hskip5cm}}
\hangindent=1em \hangafter=-1 \noindent
\leaders\copy0\hskip5cm\hbox{}\par}

In the first paragraph the\cs{leftskip} glue only obscures the first leader box; in the second paragraph
the hanging indentation actually shifts the orientation point for the row of leaders. Hanging
indentation is performed in TEX by a\cs{moveright} of the boxes containing the lines of the
paragraph.

   

Leaders are a powerful tool, they take a little bit of time to understand, but once you familiar with them you can achieve all sorts of layouts with them.


\section{Applications}

Most of the useage of leaders is in table of contents and old tables fashioned the old way. The package \pkg{arydshln} by Hiroshi Nakashima uses \cs{xleaders} to give \latex’s \pkg{array} and \pkg{tabular} environments the capability to draw horizontal/vertical dash-lines. You can refer to it for more examples.

In the LateX kernel they are mostly found them in the definition of mathematical symbols and from where I have adapted the following Example~\ref{cleaders}.

\begin{texexample}{cleaders example}{cleaders}
 \makeatletter
 \def\rightarrowfill{$\m@th\smash-\mkern-7mu%
  \cleaders\hbox{$\mkern-2mu\smash-\mkern-2mu$}\hfill
  \mkern-7mu\mathord\rightarrow$}
 \makeatother
From here to \rightarrowfill the end.
\end{texexample}

Note in the example the use of mathematical kerns (|\mkern|) and the use of 
|\smash|. Another interesting area was the definition of various commands in the
picture environment using solely leaders.


Donald Arseneau's \pkg{ulem} uses leaders extensively and other magic to provide various forms of underlining.

\begin{texexample}{Decorating text}{ex:decorating}
   \uline{important}   underlined text\\
   \uuline{urgent}     double-underlined text\\
   \uwave{boat}        wavy underline\\
    \sout{wrong}        line drawn through word\\
   \xout{removed}      marked over with //////.\\
   \dashuline{dashing} dash underline\\
   \dotuline{dotty}    dotted underline\\
\end{texexample}   

The package has another useful feature. It is one of those short packages that one can study to understand
the mechanisms of saving boxes, measuring dimensions, rules and leaders, as well as hyphenation. A must read for anyone interested in improving their basic understanding of \tex.

\vfill













