\pgfplotsset{compat/show suggested version=false}
\cxset{chapter format=traditional}
\chapter{Noise Review}

This is a preliminary response to the observations of \textsc{AURECON} on the noise measures provided for the MEP installations. Conspel during the design and procurement phase employed an International Acoustic Specialist firm, who recommended the use of low noise pipes, as a precautionary measure, where pipes run over residential sensitive areas.

\section{Drainage Piping}

AURECON raised the concern of the possibility of objectionable waterborne sound being generated at the point where the junction of the floor drain in the bathrooms above intersect with the horizontal pipe.

\paragraph{Installation Materials}

The material used in the installation is low noise pipes by Rehau. The material is a multi-layer pipe construction which increases the pipe rigidity and enhances the sound insulation properties. The inner layer has a low roughness which optimizes the flow, while the impact-resistant outer layer provides robustness for handling on construction sites. Rehau has been manufacturing the material since the 1990s. Special supports are part of the range, isolating the pipe from walls and ceilings to reduce structural borne noise.

\paragraph{Noise Generated by waste water flow} There is no general method for calculating the noise generated by waste water flow in the literature. The European norm is to rely on the provision of standards, that enable the testing of materials by accredited laboratory. Results from these laboratory tests are then used to demonstrate compliance with requirements and or legislation.  The current European standard for such measurements is DIN EN 14366 \textit{Laboratory measurement of noise from waste water installations}, which was introduced in 2002.  

These laboratory tests are carried out, in specified arrangements and flow ranges of 0.5-4.0\thinspace$l/s$. For most applications the flow that needs to be checked is 2.0 $l/s$ which is the estimated flow that results from a toilet flush. In the Doha Oasis Project, the flow under study are much lower, as the system is a 2-pipe system separating the waste water from the soil water. In the case highlighted by AURECON the maximum possible is if there is a simultaneous use of awash-hand basin and showers. This is a highly unlikely event. However, in order to err on the side of caution we have combined the flows, as shown in \eqref{eq:flow} below.


\begin{align}
Q_t &= Q_{whb} + Q_{shower} \\
     &= 0.0.037 + 0.26\\
     &= 0.297\thinspace l/s \label{eq:flow}
\end{align}

In order to estimate the noise level we refer to experiments carried out by T. Sheers \textit{et al.}, and from where Figure~\ref{fig:graph} has been extracted.\footnote{T Scheers and M Vercammen, \textit{Noise from Watse Water pipes above a Suspended Ceiling}. EuroNoise Proceedings, 2015.}

\begin{figure}[htbp]
\centering

\includegraphics[width=0.7\linewidth]{drain-graph}
\caption{Measured sound pressure level at the tested flow rates for three piping materials. }
\label{fig:graph}
\end{figure}

As it can be observed from the figure the flow of 0.297\thinspace$l/s$ results in a sound pressure level of approximately 40 db$_A$. Considering the attenuation provided by the ceiling, as well as the effect of the ceiling void acting as an absorbing plenum (due to the insulated ducts in the vicinity), we can estimate the resulting room noise at the point of the observer at $<25$\thinspace NC. 


For laminar flows it is generally accepted, that the noise generated will be imperceptible. The only point where turbulence is expected is at the gulley trap discharge and inlets. However, the design of the gulley-trap is of heavier construction than that of the pipe in order to dampen any additional noise generated by the water flowing in and out of the trap. As a check we could use established hydraulic methods to check if the flow is laminar or turbulent. Figure~\ref{Manning} shows the flow through a gravity pipe. This in Fluid Mechanics is considered as an open channel flow, and calculations can be made using the Manning equation, which is given by eqn \eqref{maneq}, which follows:

\begin{equation}
Q=\frac{K}{n}R^{\frac{2}{3}} S^{\frac{1}{2}}_{f}A \label{maneq}
\end{equation}

where,

\medskip
\begin{flalign*}
Q &= \text{discharge}&\\
K_n &= \text{unit conversion factor (1.0 for SI units)}&\\
n   &=   \text{Manning roughness coefficient (0.011)}&\\
R   &=   \text{hydraulic radius, $R = A/P_w$}&\\
A   &=   \text{cross sectional area}&\\
P_w &= \text{wetted perimeter}&\\
S_f &=  \text{friction (i.e, energy grade line)} &
\end{flalign*}
\medskip

Assuming the pipe is half-full during flow, and a slope of 1:50, we calculate the flow to be 2.3 l/s, velocity 2.35\thinspace{m/s}  and a Reynolds value of 208, which is well within the laminar regime of flow.

\begin{figure}[htbp]
\centering

\includegraphics[width=0.7\linewidth]{flow}

\caption{Gravity flow through pipe.}
\label{Manning}
\end{figure}

\paragraph{Independent Certification and Test}

The Rehau literature provides acoustic results from accredited laboratories and Consultancies including AECOM\footnote{\protect\scriptsize\url{https://www.rehau.com/download/1782060/raupiano-anz-technical-information.pdf}}. This reference provides results from tests carried out in Australian laboratories and tested against Australian standards. The AECOM letter referring to test with a ceiling construction similar to the Doha Oasis states: ``\ldots we are of the opinion this will generally not be noticeable to the human ear.'' 
The second reference provides similar tests with similar results from German State laboratories
\footnote{\protect\scriptsize\url{https://www.byko.is/media/frarennslislagnir/Rehau-Raupiano---taekniupplysingar.pdf}}

\paragraph{Recommended Action}  We have verified that the installation complies with International standards and the MEP specification. This was done both by examining the literature and providing calculations, as well as with reference to tests in accredited laboratories. As the installation complies with the requirements we do recommend that it be accepted as is. 




\chapter{HVAC Related}

The AURECON report also provided three additional concerns:
 
\begin{enumerate}
\item To verify that all fcus are within specification.
\item To take measures to avoid cross-talk.
\item The Residential Plant rooms are checked.
\end{enumerate}

Separate reports will be issued for these within 7 days.

\endinput
\luadirect{
local pi = math.pi
K=1
d = 0.050
n = 0.011
Pw = pi*d/2
A = (pi*d*d/4)/2
R = pi*(d)/2
S= 1/50 
Q=(K/n)*math.pow(R,2/3)*math.pow(S,0.5)*A
velocity = Q/A
nee = 8.90e-4
reynolds = (velocity * Pw)/nee
tex.print("flow", Q,velocity, reynolds)

}




