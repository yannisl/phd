\chapter[Data Structures]{Data Structures}

\parindent1em 
In computer science, a data structure is a particular way of organizing data in a computer so that it can be
used efficiently. \tex has only one type of data structure: the \emph{token list}, although one can consider a macro storing only contents as another primitive data structure. The original \tex had only 256 token list\footnote{See also etex for a more modern version.} registers that are
available to the user.\tex also offers some special token lists: the |\every|... variables, |\errhelp|,
and |\output|. Lamport in \latex developed additional data structures, using \tex’s primitive commands and  these are discussed in the chapters discussing the \latex kernel. In addition the \latex3 team is busy developing more familiar data structures and associated manipulation routines that one can truly say that the \tex programmer has now a full kit to program any complicated piece of code.


A token register stores a token list. A macro also stores a token list in its \meta{replacement text}, so you can perfectly get away without using token registers in most of your \tex programming. So where is the difference?

\begin{enumerate}
\item Token registers are faster.
\item Token registers will \emph{never} be expanded.
\end{enumerate} 


\begin{description}
\item [toks] Prefix for a token list register.
\item [toksdef]  Define a control sequence to be a synonym for a |\toks| register.
\item [newtoks] Macro that allocates a token list register in |plain.tex|.
\end{description}


Token lists are probably among the least obvious components of \tex: most \tex users will never
find occasion for their use, but format designers and other macro writers can find interesting
applications. Following are some examples of the sorts of things that can be done with token lists.

The number of primitive operations available for token lists is rather limited: assignment\index{token lists!assignment} and
unpacking\index{token lists>unpacking}. However, these are sufficient to implement other operations such as appending\index{token list>appending}.

\section{How to allocate to a token register}

\begin{docCommand}{toks}{\marg{number}}
Allocates a \tex primitive token register.
\end{docCommand}

\begin{texexample}{Basic usage}{ex:toksusage}
\bgroup
\def\a{a test.}
\toks0={\a}
\edef\b{\the\toks0 }
\def\c{\a}
\edef\e{\c}
\def\f{\the\toks0 }
\meaning\b

\meaning\e

\meaning\f
\egroup
\end{texexample}

In this simple test in example~\ref{ex:toksusage} we can see the major differences of using token registers or macros to hold values. An |\edef| holding the |\the\toks0 |, did not expand the values, where |\c| expanded the macro. This is an important difference and can enable one to build lists. Once you have lists, you have a full Turing complete language.

\begin{texexample}{Usage}{ex:toksadd}
\makeatletter
\bgroup
\long\def\g@addto@macro#1#2{%
  \begingroup
  \toks@\expandafter{#1#2}%
  \xdef#1{\the\toks@}%
  \endgroup
}

\def\a{My macro.\kern5pt}

\g@addto@macro\a{My other text }

\meaning\a

\a
\egroup
\makeatother
\end{texexample}

As lightly different macro can do a similar job:

\begin{texexample}{Example with edefappend}{ex:toks3}
% Add #2 (which is expanded in an \edef) to the end of the definition of
% #1 (which must be a previously-defined control sequence).  This is a
% way to construct simple lists. (xeplain)
% 
\makeatletter
\bgroup
\def\edefappend#1#2{%
  \toks@ = \expandafter{#1}%
  \edef#1{\the\toks@ #2}%
}%
%
\def\a{My macro.\kern5pt}
\edefappend\a{My other text. }

\meaning\a
\egroup
\makeatother
\end{texexample}

So far we have used the register |toks0| and |toks@| which are the same register. This is bad practice, as it may already
be in use. We can use \docAuxCommand{newtoks} to define a new register using a name.

\begin{docCommand}{newtoks}{\marg{name}}
Allocates a new token register.
\end{docCommand}


\begin{texexample}{Token List Allocation}{ex:toks}
\newtoks\alist 
\alist={token1,token2,token3}
\meaning\alist
\end{texexample}

As you can see from the example we can typeset the contents of the token register with the \cs{the} command,

\begin{texexample}{Token List with macros}{}
\begingroup
\def\c{one}
\def\d{two}
\newtoks\blist 
\blist={{\c} {\d}}
\the\blist. 
\endgroup
\end{texexample}


\begin{texexample}{Token List with macros}{}
\begingroup
\def\c{one}
\def\d{two}
\newtoks\blist 
\blist={\c \d}
\the\blist. 
\endgroup
\end{texexample}

New token lists can be created, by either using the \cmd{\newtoks} or the \cmd{\toks}\meta{number}. Where |\toks0| defines the zero register etc. It is always better to use \cmd{\newtoks} in order not to affect existing token registers used by the system or other packages.

Token register can store anything for example they can store \cmd{\vfil} or \cmd{\hfill}. \latexe uses them extensively, using one or two registers but mostly using |\@temptoken|. Do not be tempted to use this variable as it can affect the marks in your document. Good practice in your package is to define a scratch register |\mypackagetemptokena| etc.

\section{Operations}

\subsection{copying}

You can copy the contents of one token register to another by assignment. The copying is carried out \emph{without expansion}.

\begin{texexample}{}{}
\makeatletter
\newtoks\@exampletoksa
\newtoks\@exampletoksb
\@exampletoksa = {first token list}
\@exampletoksb=\@exampletoksa

% the @exampletoksb now holds the contents of @exampletoksa
\the\@exampletoksb
\makeatother
\end{texexample}




\section{Collecting Information with Token Registers}

One important use of token registers is to store information. Normally the package will provide some commands to add tokens, remove comments etc.

\begin{teXXX}
\def\addinfo #1{%
    \expandafter\expandafter\expandafter\collecttokens\expandafter{%
          \the\collecttokens #1}
}
\end{teXXX}


\begin{texexample}{Adding to a token list}{ex:adding}

\bgroup
\makeatletter
\def\addinfo#1{
  \expandafter\expandafter\expandafter
    \@exampletoksa\expandafter{%
     \the\@exampletoksa #1 }%
}
\addinfo{\hfill}
\addinfo{CHAPTER}
\addinfo{\kern0.5em}
\addinfo{50}
\the\@exampletoksa

\makeatother
\egroup

\end{texexample}


\section{Joining two token registers}

We have see so far how to allocate a token register to another, we can also join two token registers by expanding both in a macro or a token register:

\begin{texexample}{Join two token registers}{ex:joining}
\makeatletter
\bgroup
\toks0={}
\newtoks\@temptokenb
\@temptokena={tokena\\ }
\@temptokenb={tokenb}

\def\jointoks#1#2#3{%
  #1=\expandafter\expandafter\expandafter
    {\expandafter\the\expandafter#2\the#3}}

\jointoks{\toks0}{\@temptokena}{\@temptokenb}

\the\toks0 
\egroup
\makeatother
\end{texexample}




\begin{teXXX}
% 1. Vereinigung zweier token register
%
% Ergebnis " #1={ <Inhalt von #2> <Inhalt von #3>} "
%
\def\JoinToks#1=(#2+#3){#1=\expandafter\expandafter\expandafter
{\expandafter\the\expandafter#2\the#3}}
%===============================================================
%
% 2. Ahnlich, jedoch mit der Angabe eines Ziels
% Ergebnis " { <Inhalt von #1> <Inhalt von #2> } "
%
\def\Union(#1,#2){\expandafter\expandafter\expandafter
{\expandafter\the\expandafter#1\the#2}}
%
\def\UpToHere{\relax}%
\def\IgnoreRest#1#2\UpToHere{#1} % helper macro
\def\IgnoreFirst#1#2\relax\UpToHere{#2} % helper macro


%===============================================================
% 3. liefert das erste Element eines token register #1
%
\def\First#1{\expandafter\IgnoreRest\the#1{}\UpToHere}
%===============================================================
% 4. liefert das erste Element eines token register #1 mit
% umgebenden Klammern " { ... } "
%
\def\FirstOf#1{\expandafter\expandafter\expandafter
{\expandafter\IgnoreRest\the#1{}\UpToHere}}
%===============================================================
% 5. weist das erste Element eines token register #1
% auf das zweite token register #2 zu
%
\def\MoveFirst(#1to#2){#2=\FirstOf{#1}}
%===============================================================
% 6. gibt alle Elemente aus dem token register #1,
% außer dem ersten aus.
%
\def\Rest#1{\expandafter\IgnoreFirst\the#1\relax\UpToHere}
%===============================================================
% 7. wie in (6), jedoch mit umgebenden Klammern " { ... }"
%
\def\RestOf#1{\expandafter\expandafter\expandafter
{\expandafter\IgnoreFirst\the#1\relax\UpToHere}}
%===============================================================
% 8. weist alle Elemente des token register #1 außer dem
% ersten auf #2 zu
%
\end{teXXX}

\def\MoveRest(#1to#2){#2=\RestOf{#1}}

We can write some macros to manipulate the toks registers as shown in listing no 3. By capturing the firsttoken and the rest of tokens, you can actually write a macro to transverse the token list.

\begin{minipage}[t]{4cm}
\begin{teX}
\toks1={one}                         
\toks2={two}                         
\toks3={{one}{two}}            
\ToksOne={\number1}          
\ToksTwo=\toks2                   
\ToksThree=\toks3                
\end{teX}

\end{minipage}
\hspace{1.5cm}
\begin{minipage}[t]{4cm}
\begin{teX}
\toks4=\FirstOf{\toks1}
\toks5=\RestOf{\toks2}
\toks7=\FirstOf{\toks3}
\toks8=\RestOf\ToksOne
\toks9=\Union(\toks1,\toks2)
\MoveRest(\toks9 to\toks0)
\end{teX}
\end{minipage}
\bigskip  

The result is 

\bigskip

{\leftskip 2em
\begin{tabular}{llllll}
|\the\toks1|       &$\rightarrow$  &|one| &|\the\toks4|  &$\rightarrow$  &one\\
|\the\toks2|       &$\rightarrow$  &|two| &|\the\toks5|   &$\rightarrow$ &two\\
|\the\toks3|       &$\rightarrow$  &|{one}{two}| &|\the\toks7| &$\rightarrow$    &one\\
|\the\ToksOne|  &$\rightarrow$  &|\number1|    &|\the\toks8| &$\rightarrow$ &1\\
|\the\ToksTwo|   &$\rightarrow$  &two &|\the\toks9| &$\rightarrow$     &onetwo\\
|\the\ToksThree| &$\rightarrow$ &|{one}{two}| &|\the\toks0| &$\rightarrow$ &netwo\\ 
\end{tabular}
}

\section{Joining two registers}

Now, that we have described the basic commands of creating a token register (assignment) and unpacking it with the \cmd{the}, we can develop some macros to manipulate such lists. The first macro we will develop is a macro to \textit{join}
two lists and store the result in a third token list \cmd{\result}.

\begin{texexample}{Joining two registers}{}
\def\JoinToks#1#2+#3;{#1=\expandafter\expandafter\expandafter
    {\expandafter\the\expandafter#2\the#3}}

\toks1={{Alice }{John }{Mary }}
\toks2={{Marilou }{Maria }{Marianne }}
\toks3={{John}{Yannis}{Yiannis}}

\newtoks\result
\JoinToks\result\toks1+\toks2;
\the\result
\JoinToks\result\result+\toks3;
\end{texexample}

LaTeX2e has a temp scratch register \cmd{\@temptokena}.


Note that the equal sign in |\JoinToks\result=(\toks1+\toks2)| and the brackets are by design, ie, by the definition of \cmd{JoinToks}. The same with the plus sign (+). You could omit all of them and just use commas or just spaces. Also remember to separate the different elements of the list by using brackets curly brackets. If you omit them, \tex will only read the first letter!



\section{Union}
Similarly we can define a command \cmd{\Union} which is very similar to \cmd{\JoinToks} and is perhaps more intuitive.


\begin{texexample}{Union of Two Token Lists}{}
\toks1{test1,test2,test3}
\toks2{test1,test4,test5}
\def\Union(#1,#2){\expandafter\expandafter\expandafter
{\expandafter\the\expandafter#1\the#2}}
\toks9=\Union(\toks1,\toks2) 
\the\toks9
\end{texexample}





\subsection{Helper Macros}
The next macros are helper macros to assist in the rest of the definitions. The \cmd{\UpToHere} macro is a simple \cmd{\relax}, where the \cmd{\IgnoreRest} and \cmd{\IgnoreFirst} are defined as per their names.

\begin{teX}
% Helper macros 
%
\def\UpToHere{\relax}%
\def\IgnoreRest#1#2\UpToHere{#1} % helper macro
\def\IgnoreFirst#1#2\relax\UpToHere{#2} % helper macro
\end{teX}


\begin{teX}

% 3. Returns the first element of a token register #1
%
\def\First#1{\expandafter\IgnoreRest\the#1{}\UpToHere}

% 4. returns the first element of a token register # 1 with
% Surrounding brackets "{...}"
%
\def\FirstOf#1{\expandafter\expandafter\expandafter
{\expandafter\IgnoreRest\the#1{}\UpToHere}}

% 5. Move  the first element of a token register # 1
% To the second token register # 2 to
%
\def\MoveFirst(#1to#2){#2=\FirstOf{#1}}

% 6. Move all elements of the token register # 1,
% Except for the first out.
%
\def\Rest#1{\expandafter\IgnoreFirst\the#1\relax\UpToHere}

% 7. as in (6), but with surrounding brackets "{...}"
%
\def\RestOf#1{\expandafter\expandafter\expandafter
{\expandafter\IgnoreFirst\the#1\relax\UpToHere}}

% 8. weist alle Elemente des token register #1 auer dem
% ersten auf #2 zu

\end{teX}

If you have read up to here, you would be wondering as to how to access the \textit{length} of the token register, pop and push, slice etc. Common terminologies of lists and arrays\footnote{Remember that a list is a one dimensional array. It is quite possible to build all these commands using \TeX\ and as a matter of fact \LaTeX has all these constructs built-in.} We would demonstrate this by example.


\subsection{How to find the length of a token list}
\index{token lists!length}

These examples are from \TeX by Topic. They have been modified to print their
output, rather than |\message{}|, in order to print the result here.


We first define a toks register and a count register, named \cmd{auxlist} and \cmd{auxcount}.

\begin{teX}
\newtoks\auxlist 
\newcount\auxcount
\end{teX}



First of all there must be an operation to add auxiliary files:

\begin{teX}
\def\NewAuxFile#1{\AddToAuxList{#1}%
% plus other actions
}
\end{teX}

\def\NewAuxFile#1{\AddToAuxList{#1}%
% plus other actions
}

\noindent Next we define a macro that adds a token to the token list, but also delimits it using \docAuxCommand{elt}. The token |\@elt| is commonly used by \latex in list constructions. It has been borrowed from Lisp which and is a short for element. Knuth used throughout |plain| two backslashes (\textbackslash\textbackslash). As a matter of fact you can use any marker you want. It is preferable though to adhere to these two conventions as they are commonly used throughout packages and in the literature.

Map takes a function \textbf{f} and a list $xs$ and applies f
to every element of xs. For example,

$$
\mbox{Map}\,[1,2,3] = [f1,f2,f3]
$$



\begin{texexample}{Elt lists}{ex:eltlist}
\makeatletter
% allocations
\global\newtoks\auxlist 
\newcount\auxcount

% Define a macro to add to the list
\protect\gdef\addtoauxlist#1{\let\@elt=\relax
  \xdef\act{\noexpand\auxlist={\the\auxlist \@elt{#1}}}%
  \act}

% Add some elements to the list
\addtoauxlist{one}  
\addtoauxlist{two}
\addtoauxlist{three}

\the\auxlist

\the\auxlist
\makeatother
\end{texexample}



By redefining \docAuxCommand*{@elt} we can capture each element during expansion and map to the elements other functions. Consider that we just want to print the elements.


\begin{texexample}{printing the list}{ex:printlist}
\makeatletter
% define a macro to print the list
\the\auxcount
\def\printauxlist{%
  \def\@elt##1{\advance\auxcount1\relax\the\auxcount. \itshape ##1\par }
  \the\auxlist }
  
% add some values
\addtoauxlist{one}  
\addtoauxlist{two}
\addtoauxlist{three}  
\addtoauxlist{four}  
\addtoauxlist{five}

% print the results
\printauxlist
\makeatother  
\end{texexample}

The space after |\auxcount1| is important to signal to \tex the end of the number 1. It is better to actually write |\relax|. Try it without and it will just keep on printing only the dot only.



\endinput
Another use of this structure is the following: at the end of the job we can now close all auxiliary files at once, by defining,
%
%\begin{teX}
%\def\CloseAuxFiles{
%  \def\@elt##1{\CloseAuxFile{##1}}%
%  \the\auxlist}
%
%\def\CloseAuxFile#1{closing file: #1. %
%% plus other actions
%}
%\end{teX}
%\def\CloseAuxFiles{
%  \def\@elt##1{\CloseAuxFile{##1}}%
%  \the\auxlist}
%
%\def\CloseAuxFile#1{closing file: #1. %
%% plus other actions
%}
%
%
%\noindent which gives the output
%
%\texttt{> \CloseAuxFiles}
%
%\def\alist{}
%\listadd{\alist}{Yiannis~}
%\listadd{\alist}{Yianis~}
%\listadd{\alist}{Ioannis~}
%\listadd{\alist}{Giannis~}
%\alist
%
%\makeatother
%
%\def\tempa{}
%\numdef{\tempa}{(22+35)*45}
%
%\tempa


%
%\section{String comparisons}
%
%A more general problem along the same lines is to
%check if two words, or strings are the same. We can
%use |\ifx| for this as well. When |\ifx| compares two
%tokens that are macro names, the result is true if
%the macros have been defined in the same way, and
%if their first level replacement texts are the same.
%So, we define two macros whose replacement texts
%are the strings, and compare these.
%
%\numberLineAt{50}
%\begin{teX}
%\newif\ifsame
%\newcommand{\strcomp}[2]{%
% \samefalse
% \begingroup
%   \def\1{#1}\def\2{#2}%
%   \ifx\1\2\endgroup True \sametrue
%   \else False 
% \endgroup
%\fi}
%\strcomp{Yiannis}{Yiannis}\\
%\strcomp{Yiannis}{YIANNIS}\\
%\end{teX}
%
%\newif\ifsame
%\newcommand{\strcomp}[2]{%
%\samefalse
%\begingroup
%\def\1{#1}\def\2{#2}%
%\ifx\1\2\endgroup True \sametrue
%\else False \endgroup
%\fi}
%
%
%\printf{\strcomp{Yiannis}{Yiannis}}
%\printf{\strcomp{Yiannis}{YIANNIS}}
%
%
%
%\section{Lists}
%
%A list is simply a one dimensional array, normally delimited by commas:
%
%\begin{teX}
%  \def\somelist(1,2,3,4,5,6,7,8)
%\end{teX}
%
%In TeX you will probably better off defining it as:
%
%\begin{teX}
%  \def\somelist{1,2,3,4,5,6,7,8}
%\end{teX}
%
%The package \cmd{coolist} provides basic control sequences for manipulating such lists
%
% Lists are defined as a sequence of tokens separated by a comma.  The \texttt{coollist} package allows the user
% to access certain elements of the list while neglecting others---essentially turning lists into a sort of
% array. 
%
% List elements are accessed by specifying the position of the object within the list (the index of the item) and
% all lists start indexing at |1|.
%
% 
% \begin{tabular}{ll}
% |\listval{1,2,3,4}{2}|                & \listval{1,2,3,4}{2} (the null string)        \\
% |$\listval{\alpha,\beta,\gamma}{2}$|  & $\listval{\alpha,\beta,\gamma}{2}$            \\
% |\listval{a,b,c}{4}|                  & \listval{a,b,c}{4} (the null string)
% \end{tabular}
%
%The \pmac{coolist}{liststore} stores the length of the comma delimited list into the counter. It does so by creating variables with the same name as the argument.
%
%\begin{teX}
%\liststore{1,2,3,4}{temp}
%\tempi;\tempii;\tempiii;\tempiv 
%\end{teX}
%
%produces 
%
%\liststore{1,2,3,4}{temp}
%\tempi;\tempii;\tempiii;\tempiv 
%
%This can be used quite effectively to create a number of on the fly variables.
%
%The list can be numerical or alpha
%
%\begin{teX}
%\liststore{alpha,beta}{temp}
%\end{teX}
%
%
%will produce
%
%\liststore{alpha,beta}{temp}
%
%\texttt{\medskip\tempi; \tempii \medskip }
%
%\section*{length of list}
%
%The length of the list can be obtained by using the \pmac{listcool}{listlen} or \pmac{listcool}{listlenstore}
%
%\begin{teX}
%\listlen{1,2,3,4,5} 
%\listlen{} 
%\listlen{1,2} 
%\listlen{1} 
%\end{teX}
%
%\medskip
%
%|\listlen{1,2,3,4,5}| \texttt{\listlen{1,2,3,4,5}}\\
%
%
%\listlen{} 
%\listlen{1,2} 
%\listlen{1} 
%\medskip
%
%You can copy one list into another using \cmd{listcopy}
%
%\begin{teX}
%\liststore{1,2,3}{temp}
%\listcopy{temp}{copiedlist}
%\copiedlisti;\copiedlistii;\copiedlistiii 
%\end{teX}
%
%{\obeylines\tt
%\liststore{1,2,3}{temp}
%\listcopy{temp}{copiedlist}
%\copiedlisti;\copiedlistii;\copiedlistiii 
%}
%
%
%You can get the sum of a list by using \cmd{listsum}
%
%
%\listsum{1,2,3,4,5}{\thelistsum}
%\thelistsum 
%\listsum{1,2,3,a,b,a,a}{\thelistsum}
%\thelistsum 
%\liststore{1,2,3,5,j,k,j}{temp}
%
%
%\listsum[liststored=true]{temp}{\thelistsum}
%\thelistsum 11+2j+k
%\listsum{a,b,c,d}{\thelistsum}
%\thelistsum
%
%
%An ingenious way of providing a consistent user interface with the |coolllist| commands is in the package \docpkg{cool}, by the same author. The code below is from the package and is used to define the display of a Fibonacci number:
%
%\begin{teX}
%$$\Fibonacci{n,x}  or~ \Fibonacci{n}$$
%$$\Fibonacci{n,x+1}  or~ \Fibonacci{n}$$
%$$\Multinomial{n_1, n_2, \ldots, n_m}$$
%\end{teX}
%
%$$\Fibonacci{n,x}  or~ \Fibonacci{n}$$
%$$\Fibonacci{n,x+1}  or~ \Fibonacci{n}$$
%$$\Multinomial{n_1, n_2, \ldots, n_m}$$
%
%
%
%Note the $\ldots$ are not lost!
%
%\begin{teX}
%% \begin{macro}{\Fibonacci}
%% Fibonacci number, |\Fibonacci{n}|, $\Fibonacci{n}$, and 
%%
%% Fibonacci Polynomial, |\Fibonacci{n,x}|, $\Fibonacci{n,x}$
%
%\newcommand{\COOL@notation@FibonacciParen}{p}
%\newcommand{\Fibonacci}[1]{%
%\liststore{#1}{COOL@Fibonacci@arg@}%
%\listval{#1}{0}%
%\ifthenelse{\value{COOL@listpointer} = 1}%
%   {F_{#1}}
%   % ElseIf
%  {\ifthenelse{\value{COOL@listpointer} = 2}%
%      {F_{\COOL@Fibonacci@arg@i}%
%	\COOL@decide@paren{Fibonacci}{\COOL@Fibonacci@arg@ii}}%
%% Else
%   {\PackageError{cool}{Invalid Argument}%
%    {`Fibonacci' can only accept a 
%    comma separate list of length 1 or 2}}}%
%}
%\end{teX}

