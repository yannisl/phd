\chapter{Margin Material}

\section{Introduction}

Since the beginning of writing margins were always used to add notes and other marginal material such as citations, notes and even figures. The \pkgname{marginnote} developed by \citeyearpar{marginnote} does not float the marginal
notes and for a good reason, as most of the time they end up in the wrong place in any case. This chapter describes
the usage of marginpar and not the technicalities of the definitions. The definitions are described in the |float.dtx| in 
\nameref{ltx:marginalnotes}

\section{How to Switch Margin paragraphs}

Marginal notes use the same mechanism as floats to communicate with the \cmd{\output} routine.
Marginal notes are distinguished from floats by having a negative placement specification. [372-373] The \CMDI{\marginpar}\oarg{left text}\marg{right text} generates a marginal note in parbox, using either the left 
text or the right text, depending on the placement. It default to righttext=lefttext. 

\begin{figure}
\includegraphics[width=\textwidth]{marginpar-01}
\end{figure}

Most designers appear to prefer to use only the one side of the page, when margin materials form part of the design of the book, such as the tufte-class. 

Marginal notes are normally put on the outside of the page if the switch \CMDI{\@mparswitch} is true, and on the right if |@mparswitch=false|. The command \CMDI{\reversemarginpar} reverses the side where they are put. \CMDI{\normalmarginpar} undoes |\reversemarginpar|. These commands have no effect in two-colum output.


\begin{figure}
\includegraphics[width=\textwidth]{marginpar-02}
\end{figure}


\lipsum