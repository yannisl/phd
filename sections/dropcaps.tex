% !TEX TS-program = pdflatex
% !TEX encoding = UTF-8 Unicode
% arara: pdflatex: { synctex: true }
%%% cfr-initials.tex
%%% Copyright 2015 Clea F. Rees
%%
%% This work may be distributed and/or modified under the
%% conditions of the LaTeX Project Public License, either version 1.3
%% of this license or (at your option) any later version.
%% The latest version of this license is in
%%   http://www.latex-project.org/lppl.txt
%% and version 1.3 or later is part of all distributions of LaTeX
%% version 2005/12/01 or later.
%%
%% This work has the LPPL maintenance status `maintained'.
%%
%% The Current Maintainer of this work is Clea F. Rees.
%%
%% This work consists of all files listed in manifest.txt.
%\listfiles
%\documentclass[11pt,british,a4paper]{article}
%\usepackage{babel}
%\usepackage[utf8]{inputenc}
%\usepackage[T1]{fontenc}
%\usepackage{textcomp,cfr-lm}
%\usepackage{fancyhdr,pageslts}
%\usepackage{longtable,verbatim}
%\usepackage{lettrine}
%\usepackage{booktabs,url}
%\usepackage{microtype}
%\usepackage[headheight=14pt,vscale=.71]{geometry}	% use 14pt for 11pt text, 15pt for 12pt text
%\usepackage{parskip}
%\usepackage{Acorn, AnnSton, ArtNouv, ArtNouvc, Carrickc, Eichenla, Eileen, EileenBl, Elzevier, GotIn, GoudyIn, Kinigcap, Konanur, Kramer, MorrisIn, Nouveaud, Romantik, Rothdn, Royal, Sanremo, Starburst, Typocaps, Zallman}
%
%\title{cfr-initials}
%\author{Clea F.\ Rees\footnote{reesc21 <at> cardiff <dot> ac <dot> uk}}
%\def\dyddiad{2015--04--06}
%\def\fyversion{Version 1.01}
%\date{\fyversion\ --- \dyddiad}
%\pagestyle{fancy}
%\fancyhf[lh]{\itshape \fyversion}
%\fancyhf[rh]{\itshape \dyddiad}
%\fancyhf[ch]{\itshape cfr-initials}
%\fancyhf[lf]{}
%\fancyhf[rf]{}
%\fancyhf[cf]{\itshape --- \thepage~of~\lastpageref*{LastPage} ---}

\chapter {Decorating Text}

\section{Initials}


In a written or published work, an initial is a letter at the beginning of a word, a chapter, or a paragraph that is larger than the rest of the text. The word is derived from the Latin initialis, which means standing at the beginning. An initial often is several lines in height and in older books or manuscripts, sometimes ornately decorated.

In illuminated manuscripts, initials with images inside them, such as those illustrated here, are known as historiated initials. They were an invention of the Insular art of the British Isles in the eighth century. Initials containing, typically, plant-form spirals with small figures of animals or humans that do not represent a specific person or scene are known as "inhabited" initials. Certain important initials, such as the B of Beatus vir... at the opening of Psalm 1 at the start of a vulgate Latin psalter, could occupy a whole page of a manuscript.

These specific initials, in an illuminated manuscript, also were called Initiums.

\section{Brief history of the initial}

\dropcap{A}{set} of sixteenth-century initial capitals, which is missing a few letters
The classical tradition was late to use capital letters for initials at all; in surviving Roman texts it often is difficult even to separate the words as spacing was not used either. In the Late Antique period both came into common use in Italy, the initials usually were set in the left margin (as in the third example below), as though to cut them off from the rest of the text, and about twice as tall as the other letters. The radical innovation of insular manuscripts was to make initials much larger, not indented, and for the letters immediately following the initial also to be larger, but diminishing in size (called the "diminuendo" effect, after the musical notation). Subsequently they became larger still, coloured, and penetrated farther and farther into the rest of the text, until the whole page might be taken over. The decoration of insular initials, especially large ones, was generally abstract and geometrical, or featured animals in patterns. Historiated initials were an Insular invention, but did not come into their own until the later developments of Ottonian art, Anglo-Saxon art, and the Romanesque style in particular. After this period, in Gothic art large paintings of scenes tended to go in rectangular framed spaces, and the initial, although often still historiated, tended to become smaller again.

In the very early history of printing the typesetters would leave blank the necessary space, so that the initials could be added later by a scribe or miniature painter. Later initials were printed using separate blocks in woodcut or metalcut techniques.


\ExplSyntaxOn
\clist_new:N\dropcapslist
\clist_gset:Nn \dropcapslist
    {\Royal,\Romantik,\EileenBlfamily,\Zallmanfamily,\Konanurfamily,\Starburstfamily,\Typocapsfamily, \ArtNouvcfamily,\Kramerfamily,\GotInfamily,\Sanremofamily,\ArtNouvfamily,}
\ExplSyntaxOff

\setcounter{DefaultLines}{3}%

\makeatletter

\long\def\lettrinetest#1{%
\renewcommand\LettrineFontHook{\color{bgsexy}#1}
\leavevmode
\dropcap{G}{e} any dedicated reader can clearly see, the Ideal of practical reason is a representation of, as far as I know, the things in themselves; as I have shown elsewhere, the phenomena should only be
used as a canon for our understanding. This is of course some nonsense text to see what is goinf wrong, with some calculations.
\texttt{\string#1} \the\@tempcnta, \the\@tempcntb

\drawfontbox{{\Huge\color{thegreen}#1Q}}
\par
}
	

 
\ExplSyntaxOn
 \clist_map_inline:Nn\dropcapslist {
     \lettrinetest{#1}\relax
        }
\ExplSyntaxOff
	
\makeatother


\begin{latexquote}
  \hspace*{-\parindent}\pkgname{cfr-initials} by Clea F.\ Rees is a set of 23 tiny packages designed to make it easier to use the decorative and ornamental initials provided by \pkgname{initials} in \LaTeX.
  \pkgname{initials} provides 23 such fonts in type 1 format, together with the support files required to use them.
  They cannot be used straightforwardly in \LaTeX, however, because the lack of package files providing ready-to-use commands is complicated by the non-standard naming of the font definition files.
  \pkgname{cfr-initials} is designed to make good that deficit.|http://www.1001fonts.com/users/steffmann/|
\end{latexquote}

\section{Using the fonts}\label{sec:usage}

To access the fonts, you simply load the relevant package in your preamble.
For example:
\begin{verbatim}
  \usepackage{Zallman}
\end{verbatim}

Each package provides two new commands.
The first is a font \emph{switch}.
It switches to the relevant set of initials until the end of the group, or until another command switches to a different family.
You will rarely wish to use these commands directly but they are useful in the definitions of macros.
For example, these are the commands you will need if using the fonts with the \pkgname{lettrine} package.
(See section \ref{sec:lettrine} for examples.)

The second takes a single, mandatory argument and typesets that argument in the appropriate font.

\begin{longtable}{llllll}
  \toprule
  \bfseries Family	& \bfseries Package	&	\bfseries Switch	&	\bfseries Command	& \verb|ABC| & \verb|abc|	\\\midrule\endhead
  \bottomrule\endfoot
  Acorn & \pkgname{Acorn} & \verb|\Acornfamily| & \verb|\acorn{}| & \acorn{ABC} & \acorn{abc} \\
  AnnSton & \pkgname{AnnSton} & \verb|\AnnStonfamily| & \verb|\astone{}| & \astone{ABC} & \astone{abc} \\
  ArtNouv & \pkgname{ArtNouv} & \verb|\ArtNouvfamily| & \verb|\artnouv{}| & \artnouv{ABC} & --- \\
  ArtNouvc & \pkgname{ArtNouvc} & \verb|\ArtNouvcfamily| & \verb|\artnouvc{}| & \artnouvc{ABC} & \artnouvc{abc} \\
  Carrickc & \pkgname{Carrickc} & \verb|\Carrickcfamily| & \verb|\carr{}| & \carr{ABC} & \carr{abc} \\
  Eichenla & \pkgname{Eichenla} & \verb|\Eichenlafamily| & \verb|\eichen{}| & \eichen{ABC} & \eichen{abc} \\
  Eileen & \pkgname{Eileen} & \verb|\Eileenfamily| & \verb|\eileen{}| & \eileen{ABC} & \eileen{abc} \\
  EileenBl & \pkgname{EileenBl} & \verb|\EileenBlfamily| & \verb|\eileenbl{}| & \eileenbl{ABC} & \eileenbl{abc} \\
  Elzevier & \pkgname{Elzevier} & \verb|\Elzevier| & \verb|\elz{}| & \elz{ABC} & \elz{abc} \\
  GotIn & \pkgname{GotIn} & \verb|\GotInfamily| & \verb|\gotin{}| & \gotin{ABC} & --- \\
  GoudyIn & \pkgname{GoudyIn} & \verb|\GoudyInfamily| & \verb|\goudyin{}| & \goudyin{ABC} & --- \\
  Kinigcap & \pkgname{Kinigcap} & \verb|\Kinigcapfamily| & \verb|\kinig{}| & \kinig{ABC} & \kinig{abc} \\
  Konanur & \pkgname{Konanur} & \verb|\Konanurfamily| & \verb|\konanur{}| & \konanur{ABC} & \konanur{abc} \\
  Kramer & \pkgname{Kramer} & \verb|\Kramerfamily| & \verb|\kramer{}| & \kramer{ABC} & \kramer{abc} \\
  MorrisIn & \pkgname{MorrisIn} & \verb|\MorrisInfamily| & \verb|\morrisin{}| & \morrisin{ABC} & \morrisin{abc} \\
  Nouveaud & \pkgname{Nouveaud} & \verb|\Nouveaudfamily| & \verb|\nouvd{}| & \nouvd{ABC} & \nouvd{abc} \\
  Romantik & \pkgname{Romantik} & \verb|\Romantik| & \verb|\romantik{}| & \romantik{ABC} & \romantik{abc} \\
  Rothdn & \pkgname{Rothdn} & \verb|\Rothdnfamily| & \verb|\roth{}| & \roth{ABC} & \roth{abc} \\
  Royal & \pkgname{Royal} & \verb|\Royal| & \verb|\royal{}| & \royal{ABC} & \royal{QFR}\\
  Sanremo & \pkgname{Sanremo} & \verb|\Sanremofamily| & \verb|\sanremo{}| & \sanremo{ABC} & \sanremo{abc} \\
  Starburst & \pkgname{Starburst} & \verb|\Starburstfamily| & \verb|\starburst{}| & \starburst{ABC} & \starburst{abc} \\
  Typocaps & \pkgname{Typocaps} & \verb|\Typocapsfamily| & \verb|\typocap{}| & \typocap{ABC} & \typocap{abc} \\
  Zallman & \pkgname{Zallman} & \verb|\Zallmanfamily| & \verb|\zall{}| & \zall{ABC} & \zall{abc} \\
\end{longtable}
\clearpage

\section{Sample lettrines}\label{sec:lettrine}

\lettrinetest{\Royal,\Romantik,\ArtNouvfamily,\EileenBlfamily,\Zallmanfamily,\Konanurfamily,\Starburstfamily,\Typocapsfamily,\ArtNouvcfamily}%,\Kinigcapfamily,\Kramerfamily,\GotInfamily}


\endinput

