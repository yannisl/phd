% Note with new geometry paper has to be defined in preamble
% I do not feel very confident of this
% Don't understand it fully how is working

\cxset{geometry oxford/.code={
\newgeometry{a4paper,left=74.8mm,top=27.4mm,headsep=2\baselineskip,%
marginparsep=8.2mm,marginparwidth=49.4mm,textheight=49\baselineskip,headheight=\baselineskip}
\@twosidefalse \@mparswitchfalse % one side option
\reversemarginpar
}}

\cxset{geometry textwidth/.store in=\textwidth@cx,
          geometry textheight/.store in=\textheight@cx,
          geometry tufte/.code={
             \newgeometry{a4paper,left=24.8mm,top=27.4mm,headsep=2\baselineskip,%
             textwidth=107mm,marginparsep=8.2mm,marginparwidth=49.4mm,%
             textheight=\textheight@cx\baselineskip,headheight=\baselineskip}
            \@twosidefalse \@mparswitchfalse % one side option
           %\reversemarginpar
    }
}

\cxset{marginpar push/.store in=\marginparpush@cx,
          marginpar font/.store in=\marginparfont@cx,
          marginpar justification/.is choice,
          marginpar justification/justifying/.code=\gdef\marginparjustification@cx{\justifying},
          marginpar justification/raggedright/.code=\gdef\marginparjustification@cx{\raggedright},
          marginpar justification/RaggedRight/.code=\gdef\marginparjustification@cx{\RaggedRight},
          marginpar justification/RaggedLeft/.code=\gdef\marginparjustification@cx{\RaggedLeft},
 }
\cxset{marginpar push=10pt,
          marginpar font=\normalfont\footnotesize\sffamily,
          marginpar justification=RaggedLeft}

\cxset{style13, geometry textheight=49,
          geometry oxford,
          watermark text=SAMPLE TUFTE VARIANT,
          watermark text color=thered,
          header style=samplepage}

% This is a sidenote without the footnote mark
\newcommand\marginnote[2][0pt]{%
 % \let\cite\@tufte@infootnote@cite%   use the in-sidenote \cite command
  %\gdef\@tufte@citations{}%           clear out any old citations
  \@tufte@margin@par%                 use parindent and parskip settings for marginal text
  \marginpar{\hbox{}\vspace*{#1}\marginparfont@cx\marginparjustification@cx\vspace*{-1\baselineskip}\noindent #2}%
  \@tufte@reset@par%                  use parindent and parskip settings for body text
  %\@tufte@print@citations%            print any citations
  %\let\cite\@tufte@normal@cite%       go back to using normal in-text \cite command
}

\chapter{Geometry and Page Dimensions}

\lipsum[1-4]\marginnote[1pt]{\lorem
    \lorem}

\lipsum[1-2]

%% Stick the caption in the head might as well place the first picture also
\def\asidecaption{\parbox{4.2cm}{{\bfseries Image \thefigure}\par\lorem}%
  % \addtocontents{lof}{This is image 8}
}
\def\ps@caption{%
     \let\@oddfoot\@empty\let\@evenfoot\@empty%
    \def\@evenhead{%
        \begin{picture}(0,0)%
           \put(-150,-80){\asidecaption\par}%
            \stepcounter{figure}
           \put(-150,-370){\asidecaption}%
        \end{picture}%
      }%
    \let\@oddhead\@evenhead%
    \let\@mkboth\@gobbletwo%
    \let\chaptermark\@gobble%
    \let\sectionmark\@gobble%
 }

\def\ps@bigpicture{%
    \setlength\headheight{19cm}%
    \let\@oddfoot\@empty\let\@evenfoot\@empty%
    \def\@evenhead{%
         \begin{picture}(0,0)%
          \put(-149,0){\includegraphics[width=\dimexpr(\textwidth+150pt)]{stuartpearson}}%
         \end{picture}%
      }%
    \let\@oddhead\@evenhead%
    \let\@mkboth\@gobbletwo%
    \let\chaptermark\@gobble%
    \let\sectionmark\@gobble%
 }


\def\doubletakeimage{%
  \renewcommand{\topfraction}{.95}  % ensure seecond image will not float away
  \begin{figure}[t]
    \thispagestyle{caption}
    \includegraphics[width=\textwidth]{matron}%
  \end{figure}

  \begin{figure}[tp]
   \hspace*{-\marginparwidth}\includegraphics[height=0.9\textheight]{stuartpearson}
 \end{figure}
}


\doubletakeimage
\lipsum[1-4]

\restoregeometry
%% RESET EVERYTHING AT END OF CHAPTER
\addtocounter{chapter}{-2}
\@toctrue\@specialtrue

