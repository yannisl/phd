
\cxset{lineskip/.code=\setlength\lineskip{#1},
       lineskip/.default=1pt,
          normallineskip/.code=\setlength\normallineskip{#1},
          parindent/.code=\setlength\parindent{#1},
          parskip/.code=\setlength\parskip{#1},
          text-indent/.code=\setlength\parindent{#1},
          baselinestretch/.code=\renewcommand\baselinestretch{#1},
          single spacing/.code=\singlespacing,
          single spacing/.default=\singlespacing,
          double spacing/.code=\doublespacing}

\cxset{lineskip=1pt,
          normallineskip=1pt,
          parindent=1em,
          parskip=1pt,
          text-indent=1em,
          baselinestretch={},
          single spacing}

\makeatletter\@specialtrue\makeatother
\cxset{steward,
  numbering=arabic,
  custom=stewart,
  offsety=0cm,
  image={./images/hine05.jpg},
  texti={When Lamport designed the original \LaTeX\ sectioning commands, limitations of computer power forced him to restrict the abstraction of complicated chapter layouts. With current tools available improvements are much easier to program.},
  textii={In this chapter we discuss a method that allows the production of fancy chapter headings and formatting, based on a set of key values. Central  to this process is the separation of content from presentation.
We also discuss the basic formatting tools that are available and how one can modify them to mould new book designs.
 }
}
\cxset{chapter opening=left}

\chapter{General Settings}

\section{Introduction}

Here we define and set general paragraph settings. The parameters which control \TeX's behaviour when typesetting paragraphs can receive a bit of a tweak here. We also describe a set of options to handle parameters that can influence grid typesetting. This is especially important for two or more column typesetting. The commands act only on the text within a grouped environment. They do not affect captions or footnotes. Use anything over \emph{single spacing} with care, as books are meant to be single spaced.  



\section{Controlling inter-line spacing}
\index{line spacing}
Interline spacing traditionally has been controlled using the \pkgname{setspace} or by setting appropriate primitive \tex commands \cite{setspace}. The \pkgname{phd} loads the |setspace| package and then provides parameterized commands for setting styles. 

\begin{key}{/chapter/single spacing} 
	The Lineskip parameter emulates \TeX's \cmd{\parindent} command.
\end{key}
\begin{key}{/chapter/one half spacing} 
	The Lineskip parameter emulates \TeX's \cmd{\parindent} command.
\end{key}
\begin{key}{/chapter/double spacing} 
	Sets the document line-spacing to double.
\end{key}

If you want to use larger inter-line spacing in a document, you can change its value by putting the

\CMDI{\linespread}\meta{factor} Use |\linespread{1.3}| for "one and a half" line spacing, and |\linespread{1.6}| for "double" line spacing. Normally the lines are not spread, so the default line spread factor is~1.

The setspace package allows more fine-grained control over line spacing. To set "one and a half" line spacing document-wide, but not where it is usually unnecessary (e.g. footnotes, captions):

\begin{teXXX}
\usepackage{setspace}
%\singlespacing
\onehalfspacing
%\doublespacing
%\setstretch{1.1}
\end{teXXX}

The |phd| package provides the settings

\begin{key}{/chapter/single spacing}
We use the \pkgname{setspace} to effect the desired line spread effect.
\end{key}


These command offer little value over the normal \TeX\ macros other than keeping the interface, uniform. One can also extend the interface to cover CSS style commands:

\begin{verbatim}
\cxset{text-indent=50pt}

\cxset{double spacing}
\lipsum*[1]

\cxset{single spacing}
\lipsum*[1]
\end{verbatim}



\subsection{Parameters controlling paragraphs}\index{Paragraphs!controlling parameters}
The parameters \cs{lineskip} and \cs{normallineskip} influence \TeX\ when two lines come two close.
\medskip



\begin{key}{/chapter/lineskip=1pt} 
	The Lineskip parameter emulates \TeX's \cmd{\lineskip} command.
\end{key}

\begin{key}{/chapter/normallineskip=\marg{dim}} 
	The normallineskip parameter emulates \TeX's \cmd{\normallineskip} command.
\end{key}

\begin{key}{/chapter/lineskiplimit=\marg{dim}} 
	The Lineskip parameter emulates \TeX's \cmd{\lineskiplimit} command.
\end{key}

\begin{key}{/chapter/parindent=\marg{dim}} 
	The Lineskip parameter emulates \TeX's \cmd{\parindent} command.
\end{key}

\keyval{parindent}{\marg{dim}}{Paragraph indentation.}
\keyval{text-indent}{\marg{dim}}{Alias for \cs{parindent}.}
\keyval{parskip}{\marg{dim}}{Spacing between paragraphs.}


Another advantage, the package offers a few pre-configured styles, just setting a style to latex will revert everything back to latex.

\section{Technical discussion}

Most classes, including the standard \LaTeXe\ classes as well as packages attempting to achieve a grid typesetting try define a text height that is a multiple of \cs{baselineskip}. This way they give little opportunity to TeX to adjust the vertical glue to achieve a flush bottom.

\section{Dropcaps and Lettrines}\index{Lettrine!basic typesetting}

Dropcaps or lettrines are those letters that start paragraphs with a fancy larger letter. The class uses a parameterized version of the lettrine package of Daniel Flipo. Lettrine letters are easily typed and produced, but they are notoriously difficult to get right and no-one seems to agree on settings. These settings depend on the font the sizing of the text and the personal taste of the book interior designer. As I don't profess to be one, I have done what I think Knuth have done (just studied existing sources) allowed programming hooks and provided defaults as close as possible to the originals.


