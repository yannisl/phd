\chapter{Colors}
\section{Introduction}

One area where \latex shows its age is the manipulation and use of colors. When \tex was developed color printing was not possible and the introduction of colors is now only predominantly via the use of the |xcolor| package. Despite such limitations colors can be used fairly easily.

The figure below, shows the wavelengths in nm of the visible light. It has been drawn using the \docpkg{xcolor} package and the native \latex environment \cs{picture}. The colors can be typest using the wavelength of light.

\smallskip

\begin{texexample}{}{}
  \hbox{\color{thered} A TesT}
\end{texexample}




\newcount\WL \unitlength.75pt

\begin{figure}
\hskip-3pt\scalebox{0.9}{
\noindent

\begin{picture}(460,60)(355,-10)
\sffamily \tiny \linethickness{1.25\unitlength} \WL=360
\multiput(360,0)(1,0){456}%
{{\color[wave]{\the\WL}\line(0,1){50}}\global\advance\WL1}
\linethickness{0.25\unitlength}\WL=360
\multiput(360,0)(20,0){23}%
{\picture(0,0)
\line(0,-1){5} \multiput(5,0)(5,0){3}{\line(0,-1){2.5}}
\put(0,-10){\makebox(0,0){\the\WL}}\global\advance\WL20
\endpicture}
\end{picture}}
\caption{The visible spectrum nm}
\end{figure}



\section{Specifying colors by name}

The easier way to specify colors is to use the pre-build names available
with the package drivers.


\section{Color boxes}

Now and then users want to place text in colorboxes. The macro \cs{colorbox} can be used to provide background text to a box.

\begin{docCommand}{colorbox}{}
The colorbox macro takes one command. As it is modelled on the same concept as an
\cs{fbox} it will add a bit of space around the containing text. If you want
to remove it you will have to set the \cs{fboxsep=0pt}.
\end{docCommand}


\begin{texexample}{Color boxes}{}
\fboxsep0pt
\colorbox{green}{\begin{minipage}{3cm}
  Some text
\end{minipage}}

\fboxrule.2pt\fboxsep-.2pt
\fbox{\begin{minipage}{3cm}
  Some text
\end{minipage}}
\end{texexample}

\begin{docCommand}{textcolor}{\marg{color name}\marg{text}}{}
The \cs{textcolor} takes two arguments and typesets the contents of the second argument, using the color specified in the first argument.
\end{docCommand}

\begin{texexample}{Color text}{ex:textcolor}
\textcolor{blue}{This is typeset in blue color.}

\end{texexample}

\section{Mixing colors}

Colors can be mixed by using a convention using (!).

\begin{texexample}{Mixing colors}{ex:colormix}
% set the color
\color{blue!40!yellow}
\lorem 
\end{texexample}

One can also mix colors from different color schemes and |xcolor| will take care of the calculations.

\section{Defining colors}

There are two commands that one can use to define colors.

\begin{docCommand}{definecolor}{\oarg{type}\marg{name}\marg{model-list}\marg{spec-list} }
The |\definecolor| command is one of the commands that may be used to assign a \textit{name} to a specific color. Afterwards, this color is known to the system (in the current group) and may be used in color expressions. Note that an existing color name will be overwritten.
\end{docCommand} 

\begin{texexample}{Defining colors}{}
\definecolor{red}{rgb}{1,0,0}
\definecolor{mygray}{named}{black}
\colorlet{myblack}{black}

\color{myblack}\lorem
\end{texexample}

\section{Palettes}

The \pkgname{phd} offers the concept of a palette. A palette is a set of colours that can be used to colour sectioning and other document elements in a consistent and easy way.

It also offers the misnamed command \cmd{\inherit} that allows the inheritance of a color from a higher level section to a lower level.

If you study the styling commands each and every sectioning command has a colour attribute. This colour attribute if not set it defaults to the standard color used for sectioning in palettes.

\cxset{palette/.store in=\palette}

document text color
document headings color
document 











