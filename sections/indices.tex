\let\bs\textbackslash
\parskip1.5pt 


\chapter{Indices}
\pagebreak

\starttemplate{kroll}
\thispagestyle{empty}
    \begin{leftcolumn}
       \begin{center} 
          \huge \noindent PREPARING\\
                   INDEXES
       \end{center}
     
      \medskip

       {\justifying \small\noindent And in such indexes, although small pricks\\
To their subsequent volumes, there is seen\\
The baby figure of the giant mass\\
Of things to come at large. \par
\hfill \textit{--Shakespeare}\par
\hfill\hfill{ \RaggedRight from \textit{Troilus and Cressida}}}
\medskip
       \putimage[width=1.0\linewidth]{./images/animalium01.jpg}\par
       \aheader{Handwritten cards compiled by  Sherborn for his publication \textit{Index Animalium}}
   \end{leftcolumn}
   \begin{rightcolumn}
       \putimage[width=\linewidth]{./images/animalium.jpg}
       \onelinecaption{{\resizebox{\linewidth}{5.5pt}{\bfseries Sherborn’s `Index  Animalium'. Credit Natural History Museum, London\footnote{This monumental publication became the basis for all zoological nomenclature work having gathered together all the relevant data in one place, just as an online database does today.}}}\par}
%  \centerline{\onelineheader{TYPESETTING AN INDEX}}
      \begin{multicols}{2}
        
\parindent1em      \lettrine{T}{he} first English Language index, appeared in Christopher Marlowe's \textit{Hero and Leander} in 1593. At that period, as often as not, by an ``index to a book'' was meant what we should now call a table of contents. Among the first indexes---in the modern sense---to a book in the English language was one in Plutarch's Parallel Lives, in Sir Thomas North's 1595 translation\footnote{Borko, Harold \& Bernier, Charles L. (1978). \textit{Indexing Concepts and Methods}, ISBN 0-12-118660-1.}.  

A section entitled ``An Alphabetical Table of the most material contents of the whole book'' may be found in Henry Scobell's Acts and Ordinances of Parliament of 1658. This section comes after ``An index of the general titles comprised in the ensuing Table''. Both of these indexes predate the index to Alexander Cruden's Concordance (1737) see Farrows\footcite{farrow96}, which is erroneously held to be the earliest index found in an English book.

      \end{multicols}
   \end{rightcolumn}
\stoptemplate

\pagestyle{headings}


\index{Quantum Mechanics>History|(}
\setlength{\columnsep}{2em}

\begin{multicols}{2}

\section{Preparing an index}


In order to produce an index, we need to load the
package \pkg{makeidx}, mark the indices we need using \docAuxCommand{index}  and compile the doument using  \docAuxCommand{makeindex}.\footnote{You will also need to at least run the file once using |MakeIndex| on |MikTex|. Check your distribution if you getting problems.}
At the place where we want the index to be printed,we use \docAuxCommand{printindex}.
\parindent1em

In \latex, we include a word
in the index by using the command \cmd{\index}\meta{arg}, so if the word Kroll should be included in
the index, we should use the command |\index{Kroll}|.

If the word is to be printed in bold, we use

% : = |
% = = @
% \catcode `|
\bgroup
 \catcode`|=11
\gdef\idxmain#1{%
   \def\idxmainentryi##1##2##3;{%
      \index{Kroll=\textbf{#1}|textbf}
    }
   \idxmainentryi#1;    
}  
\egroup

\idxmain{Kroll}

\index{Kroll=\textbf{Kroll}>Leon Kroll}


\emphasis{makeidx,makeindex,printindex,index}
\begin{teX}
\documentclass{article}
\usepackage{makeidx}
\makeindex
\begin{document}
   To prepare an index, 
   just include the
   package\index{package}
\printindex
\end{document}
\end{teX}

When the input file is proceesed with one of the LaTeX engines the \docAuxCommand{index} command
writes an appropriate index with the current page number, to a special file. The file has the same
name as the \latex input file, but with the extension (|.idx|). This file can them be processed with the |makeindex|
program by typing in the command line:

\begin{minted}{bash}
makeindex filename
\end{minted}

Note that you do not have to type the \textit{filename} extension. The program will look for a filename.idx and use that.
The |makeindex| program generates a sorted index with the same base fille name, but this time with the extension |.ind|. if now
the LaTeX input file is processed again, this sorted index gets included into the document at the point where \latex
finds \docAuxCommand{printindex}.

There is a  special character |@| which is used to denote that what appears on its left side must be
typeset as it appears on its right side.  The first occurrence of perl will also be used
by the sorting algorithm. is is very useful since what is used for sorting and what
will be printed may be different! For example,we may want to have the name 
\texttt{Donald Knuth}  under the letter K., we should write

\begin{sverbatim}
\index{Knuth@{Donald Knuth}}
\end{sverbatim}

Another thing we may want to change is the way that the page number is typeset.
If we want, for example, to have the page number in bold, we would write
\verb+\index{perl|textbf}+.  Notice that we wrote "textbf"  without the backslash. Of course,
the above can be combined with the  command:

\begin{sverbatim}
\index{perl@\textbf{perl}|textit}
\end{sverbatim}

will print the word perl in the index (the entry will be typeset in boldface type) sorted
as "perl",’ and its page number will be italic. A common application of this is through
the command . If we want to send the reader to another index entry, say, to send
the reader from the \verb+$\omega$+ to the \verb+$\Omega$+ command, we can write

\verb+\index{omega@$\omega$|see{$\Omega$}}+


Here, we ask for the entry to be sorted according to the word omega and, in its place,
the program must use 

\begin{teX}
$\omega$|see{$\Omega$}
\end{teX}

If a word is used repeatedly in a range of pages and we want to have this range
in the index, we do not write the relative \texttt{index} command all of the time. Instead,
we write \verb+\index\{convex\|(\}+  at the place where we have the first occurrence and
\verb+\index\{convex|)\}+  at the place where we have the last occurrence. This will produce a
page range in the index for the word "convex".

\subsection{Subindices}
Subindices can be produced using an exclamation mark. If we want the word `Zeus'
to appear in the category of Greek which is in the category of Gods, we will write:

\begin{sverbatim}
\index{Gods!Greek!Zeus}
\end{sverbatim}

The actual symbol used will depend on the |.ist| file used when the file is compiled. For example in the |ltxdoc| and \docClass{phddoc} classes this is redefined as |>| and the above should be written as:

\begin{sverbatim}
\index{Gods>Greek>Zeus}
\end{sverbatim}

The sympbol to be used is called |index level| and is defined in the |.ist| file that is used in conjuction with the |MakeIndex| program to produce the index.

\section{Multi-page Indexing}

To perform multi-page indexing, add a |( and |) to the end of the \cmd{\index} command, as in 
\index{Indexing>multi-page}.

{\small
\verb+\index{Quantum Mechanics!History|(}+

\narrower\narrower
In 1901, Max Planck released his theory of radiation dependant 
on quantized energy. While this explained the ultraviolet catastrophe
 in the spectrum of blackbody radiation, this had far larger consequences 
as the beginnings of quantum mechanics.\ldots

\verb+\index{Quantum Mechanics>History|)}+
}

\end{multicols}

\index{Quantum Mechanics>History|)}


\section{Summary of commands}

\begin{tabular}{lll}
\toprule
Example	&Index Entry	&Comment\\
\midrule
\textbackslash index\{hello\}	          &hello, 1	&Plain entry\\
\textbackslash index\{hello!Peter\}	      &Peter, 3	&Subentry under 'hello'\\
\textbackslash index\{Sam@\textbackslash textsl\{Sam\}\}	&Sam, 2	&Formatted entry\\
\textbackslash index\{Lin@\textbackslash textbf\{Lin\}\}	&\textbf{Lin}, 7	&Same as above\\
\textbackslash index\{Jennytextbf\}	     &Jenny, 3	&Formatted page number\\
\textbackslash index\{Joe textit\}	&Joe, 5	          &Same as above\\
\textbackslash index\{ecole@\'ecole\}	&école, 4	&Handling of accents\\
\textbackslash index\{Peter see\{hello\}\}	&Peter, see hello	&Cross-references\\
\textbackslash index\{Jen see also\{Jenny\}\}	&Jen, see also Jenny	 &Same as above\\
\bottomrule
\end{tabular}

\section{Indexing Class Documentation}

\index{Indexing=\textbf{Indexing}}
\index{Indexing>general}
\index{Indexing>doc}

Indexing \latex2e classes or package documentation produced with the \docClass{ltxdoc} is normally achieved using the \pkg{doc}\footcite{doc} and \pkg{docstrip}\footcite{docstrip} program, which are sometime difficult to use, if you need to deviate from their standard methods. The important thing here to remember is that you need to use different characters |=| |>| |*|.
\medskip

\begin{tabular}{ll}
\toprule
normal    & doc \\
\midrule
\string @ & \texttt{=} \\
\string ! & \texttt{>}\\
\bottomrule
\end{tabular}

\begin{verbatim}
\index{Indexing=\textbf}
\index{Indexing>general}
\index{Indexing>doc}
\end{verbatim}

This manual, was build using a large |ltxdoc| class and these problems appeared while I was developing it. As normal with such problems, they were very time consuming to debug. There are still issues in some parts and one day, I am hoping to come back and correct them. One needs at this point to query the need to use the |doc| and |docstrip| method of documenting macros and if it shouldn't have a pre-processor written in a higher language to ease development. 

\begin{texexample}{Checking usage and main}{ex:docmain}
\meaning\main \\
\meaning\usage
\end{texexample}





\chapter{The makeindex program}

The makeindex program was developed by Pehong Chen and Michael Harrison\footcite{chen01} in the eighties and is still used by \latexe. It is a remarkable, flexible program, found on most distributions. It is also used in GNU EMACS. 

The input format consists of a list of \meta{specifier, attribute} tuples. These are the essential tokens and delimiters needed in scanning the input index file. There are numerous such specifiers (you can think of them as the names of functions or variables) and the attributes as their value. 

\captionof{table}{Index input style parameters}
\begin{longtable}{l l l l r}
\toprule
S/No & Specifier &Attribute &Default & Meaning\\
\midrule
\inc & |keyword|     & string &|"\\indexentry"| & index command\\
\inc & |arg_open|    & char   &|'{'|            & argument opening delimiter\\
\inc & |arg_close|   & char   &|'}'|            & argument closing delimiter\\
\inc & |range_open|  & char   &|'('|            & page range opening delimiter\\
\inc & |range_close| & char   &|')'|           & page range closing delimiter\\
\inc & |level|       & char   &|'!'|             & index level delimiter\\     
\inc & |actual|      & char   &|'@'|             & actual key designator\\
\inc & |encap|       & char   & \textbar             & page number encapsulator\\
\inc & |quote|       & char & |'"'|            & quote symbol\\
\inc & |escape|      & char & |'\\'|           & symbol which escapes quote\\
\inc & |page_compositor| & string & ''-''       & composite page delimiter\\
\bottomrule
\end{longtable}


\begin{description}

\item[composite page delimiter] A page number can be a composite of one or more fields separated by a certain delimiter bound to a \textit{page\_compositor}  e.g. II-13 for page 13 of Chapter II. This attribute allows the lexical analyzer of the makeindex program to seprate these fields, making the sorting of page numbers easier.
\end{description}


\section{Output format}

The output format is governed by an index style \docExtension{.ist}. Writing new ones is a black art.

The file \docFile{gind.ist} is shown below
\begin{teXXX}
actual '='
encap '|'
level '>'
quote '!'
preamble
"\n \\begin{theindex} \n \\makeatletter\\scan@allowedfalse\n"
postamble
"\n\n \\end{theindex}\n"
item_x1   "\\efill \n \\subitem "
item_x2   "\\efill \n \\subsubitem "
delim_0   "\\pfill "
delim_1   "\\pfill "
delim_2   "\\pfill "
% The next lines will produce some warnings when
% running Makeindex as they try to cover two different
% versions of the program:
%lethead_prefix   "{\\bfseries\\hfil "
%lethead_suffix   "\\hfil}\\nopagebreak\n"
%lethead_flag       1
heading_prefix   "{\\bfseries\\hfil "
heading_suffix   "\\hfil}\\nopagebreak\n"
headings_flag       1
%%
%%
%% End of file `gind.ist'.
\end{teXXX}


\section{Customization}\index{Indexing>customizing}

When creating an index with \pkg{makeindex} one can create a \docFile{sample.ist} file that can be used together with the |makeidx| program to customize the way the index will look.

\begin{verbatim}
heading_prefix "{\\bfseries\\hfil "
heading_suffix "\\hfil}\\nopagebreak\n"
headings_flag 1
delim_0 "\\dotfill"
delim_1 "\\dotfill"
delim_2 "\\dotfill"
\end{verbatim}

This will write the first alphabet symbol in bold font, and uses dots as delimiters. This file is generally  used jointly with \texttt{makeindex} using

\begin{teX}
makeindex -s sample.ist filename.idx
\end{teX}

where |filename.idx| has been craeted by executing |latex| or one of the other engine commands such as |pdflatex| on |filename.tex|.


According to \fullcite{gabora}, you may use

\begin{teX}
sort_rule "\." "\b\."
sort_rule "\:" "\b\:"
sort_rule "\," "\b\,"
\end{teX}


\section{Writing custom indexing commands}

For complex documents it is easier to write a number of macros to assist with indexing and to also provide consistency. For example if you want to index the Devanagari alphabet we might need to get quite creative as to how to both index it as well as get the symbols in the index.
%\DeclareRobustCommand\ta{{\tibetan ༃ }}

%\begin{texexample}{Writing Indexing Commands}{ex:zs}
%\gdef\ZZs#1{\incsyms%
%   \indexcommand[\string#1]{#1}%
%   #1}
%\DeclareRobustCommand\ta{{\tibetan ༃}\xparse}
%\ta
%\end{texexample}
%
%The command |\ZZs{\ta}| typesets the command in the document, as (\ZZs{\ta}) and also adds it to the index and typesets the symbol.
%
%
%\begin{phdverbatim}
%\def\ZZs#1{\incsyms%
%   \indexcommand[\string#1]{#1}
%   \string#1}
%\end{phdverbatim}

\section{Multiple indices}

The \pkg{multind}\footcite{multind} extends the |makeidx| macros to enable multiple indices. The commands are the same as those for |mkidx|, but you have to pass an extra parameter |name| as the first argument to the commands.

\emph{printindex}
\begin{phdverbatim}
\usepackage{multind}
\makeindex{books}
\makeindex{authors}
...
\index{books}{A book to index}
\index{authors}{Put this author in the index}
...
\printindex{books}{The Books index}
\printindex{authors}{The Authors index}
\end{phdverbatim}

This package has multiple issues. It is not found in normal TeX distributions and also has compatibility issues with \AmS classes and packages. It is also fairly old, as it was written for \latex 2.09. A series of newer packages are now available.

The package \pkg{imakeidx}\footcite{imakeidx} by Enrico Gregorio 

\section{How to I add index to the table of contents?}
By default \latexe will not show in Table of Contents. It can be added manually. To add index as a chapter, use the control sequence \docAuxCommand{addcontentsline}, as shown below:

\emphasis[thered]{addcontentsline}
\begin{teX}
\clearpage
\addcontentsline{toc}{chapter}{Index}
\printindex
\end{teX}

\emphasis[thered]{printindex}
\begin{teX}
\clearpage
\addcontentsline{toc}{chapter}{Index}
\printindex
\end{teX}

\parindent=1em
\section{Limitations of the MakeIndex index processor}

The |MakeIndex| program has served the \latex community well for many years. With the newer engines it struggles
with indexing |utf| documents and this is its major limitation. Its limitations with multiple indexing can be by-passed
using the \pkg{imakeidx} package.

\section{The xindy index processor}

A more recent index processor is \href{http://www.xindy.org}{xindy}\index{xindy}. 

xindy means fle\textbf{x}ible \textbf{ind}exing s\textbf{y}stem. It is an indexing system that can be used to generate book-like indexes for arbitrary document preparation systems. This term includes systems such as TeX and LaTeX , the Nroff-family or SGML-based systems (e.g. HTML) that process some kind of text and generate indexing information. It is not fixed to any specific system, but can be configured for a wide variety of purposes.

Authors sometimes wish to include an index into their document, but very often their document preparation systems aren't able to produce indexes on their own, or the capabilities of the built-in indexers are insufficient and produce ugly looking results. More often they use separate specialized tools for this purpose, sometimes called index processors.

In comparison to other index processors \index{main>xindy} has several powerful features that make it an ideal framework for describing and generating complex indexes. Its most interesting features are

\subsection{Internationalization}

xindy can be configured to process indexes for many languages with different letter sets and different sorting rules. For example, many roman languages such as Italian, French, Portuguese or Spanish contain accentuated letters such as À, Á, ñ. Other languages from northern Europe have letters like Ä, Ø, æ or ß which often can't even be processed by many index processors let alone sorting them correctly into an index. The xindy system can be configured to process these alphabets by defining sort and merge rules that allow expressing language specific rules. One example of such a rule would be

(sort-rule "ä" "ae")

defining that a word containing the umlaut-a will be sorted as if it contained the letters ae instead. This is one form of how the umlaut-a ("ä") is sorted into german indexes. With an appropriate set of rules on can express the complete rules of a specific language.
\endinput




















