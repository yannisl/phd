\newfontfamily\symbola{Symbola_hint.ttf}
\let\codetwothousandone\pan
\let\codetwothousand\pan
\newcommand{\mf}{{\fontencoding{U}\fontfamily{zmf}\selectfont METAFONT}}

\newcommand{\pcstrut}{\vrule height11pt width0pt}

\newcommand{\sample}{Typographia Ars Artium Omnium}% Conservatrix}

\newcommand{\thefont}[4][OT1]{%
	\textcolor{thefontname}{#2}&%
	\pcstrut\fontencoding{#1}\fontfamily{#3}\selectfont#4\\}

\newcommand{\fonttitle}[1]{%
	\multicolumn2{p{\columnwidth}}{\vrule height1.5pc width0pt
	\fontseries{b}\selectfont\textcolor{black}{#1}}\\[3pt]}


\newfontfamily\oldlondon{OldLondon.ttf}

\newfontfamily\nivea{Eagle-Bold.otf}

\newfontfamily\marlboro{Marlboro.ttf}

\def\marlborologo{%
{\fboxsep=2pt\marlboro \colorbox{red800}{\textcolor{white}{MARLBORO}}}
}

\cxset{
      chapter format = stewart,
       offsety=0cm,
  image={hine03.jpg},
  texti={An introduction to the use of font related commands. The chapter also gives a historical background to font selection using \tex and \latex. },
  textii={In this chapter we discuss keys that are available through the \texttt{phd} package and give a background as to how fonts are used
in \latex.
 }
 }


\chapter{Setting up Fonts}
\label{ch:fonts}
\section{Introduction}

\epigraph{I am happy to say that it's
possible now to buy Kentucky Fried Chicken in Stockholm. Pita bread, too, he added as an
afterthought., He wasn't sure how it had happened, but lately pita had grown to seem as American
as hot dogs.}{Ann Tyler---The Accidental Tourist.}

\pagestyle{headings}
\index{fonts>serif}\index{fonts>non-serif}
Selecting the right fonts for a book is a job best left to the book designer. Despite this good advice most \latex authors get their hands dirty trying to play the role of the book designer. A word of advice is that most of them make a royal mess of it. Irrespective of the \tex engine employed, being \tex, \latexe, \lualatex or \xelatex there are two issues in using fonts. How to select them and specify them and what fonts to use. We will dwell on the technical aspects of font selection in this Chapter.

The aesthetic nature of writing means that writers may choose certain fonts or visual styles because
of their beauty or the qualities they express, and
readers must deal with these variations. For example, {\oldlondon Old London Court} may appear on a street sign
not because this font is easy for modern readers to
read, but to convey adherence to tradition. The aesthetic benefits come with a cost: Readers must learn
to deal with different forms of symbols, placing
 in the same category as ‹A›. For a child, it may
be by no means obvious that these forms belong
together.\footnote{The Old London font can be downloaded from \protect\url{https://www.dafont.com/theme.php?cat=401}}

But fonts dont just mess with children's minds but also with the adult brain. When physicist Fabiola Gianotti announced the possible discovery of the Higgs boson in July 2012, her presentation slides were dominated by the rounded typeface of Comic Sans. As the \textit{New Scientist} put it, Comic Sans is the most divisive font in the world.\footnote{\protect\url{New Scientist, 29 December 2012, p. 68-69.}}  Reactions ranged from outrage to calls for it to be renamed Comic Cerns. 

\section*{Fonts and Human Perception}

There is another more serious aspect in selecting fonts based on \enquote{physiological} considerations. 
All text must be written in some particular style of lettering, known as its typeface or font. If it is once
accepted that different typefaces generate their own connotations then every written word originates
two meanings. Doyle and Bottomley (2009) carried out three experiments that demonstrated that people who are asked to evaluate products are affected by transfer of meaning from the typeface in which they are presented. The meaning of the word is pulled towards the meaning of the typeface (assimilation), rather than pushed away from it.\footcite{doyle2009}

Previous research has revealed that different typefaces are often perceived as having visible personality traits (which we Jordan et al, call print personalities) with the ability to convey semantic information beyond the meaning provided linguistically by the words themselves. Over time, this capacity for typefaces to elicit feelings in readers has been referred to variously as atmosphere value (Poffenberger and Franken, 1923), congeniality (Zachrisson, 1965), semantic quality (Bartram, 1982), topographical allusion (Lewis and Walker, 1989), personality (Striver, 2001), and rhetorical effects (Mackiewicz and Moeller, 2004). Indeed, several reports have argued that the visual attributes of written words have a subtle influence on perception extending beyond matters of legibility (e.g., Kostelnick, 1990; Brumberger, 2003; Mackiewicz, 2004). Sushan and Wright (1989), for example, claim that each typeface has a discrete personality and can be characterized in many ways, including confident, elegant, casual, bold, romantic, friendly, nostalgic, modern, delicate, and sassy, with as many potential personalities as there are actual typefaces. In addition, Parker (1997) proposed that typefaces with rounded serifs are perceived as friendly and open whilst typefaces with square serifs are perceived as formal and proper. Whether or not a typeface has a serif also affects the number of affective characteristics ascribed to it, with serif typefaces eliciting more emotion-laden adjectives than sans serif typefaces (Tantillo et al., 1995). Serif typefaces, for example Times New Roman are typically described as reliable and bookish. 


\subsection*{Legibility}

A considerable body of literature deals with various aspects of typography. A large number of studies have been concerned with legibility, others with comprehension, aesthetic merits, \enquote{liking}, \enquote{pleasingness} and eye fatigue. Another aspect is the \enquote{appropriateness} of typefaces for different kings of subject matter.

Boris Veytsman in an article in TUGboat \citep{boris2012} reviewed the literature comparing fonts for readbility as well as the ``trustability'' of the text based on different fonts. Experiments carried out by Morris \footcite{morris2012a} concluded that fonts affect the reader's attitude towards the text. Baskerville scored the highest and Comic Sans the lowest.
Interestingly Computer Modern, the default typeface of \tex, scored high in the test.  Other tests carried out by \cite{boris2012a} also concluded that there are no noticable differences between serif and non-serif fonts in reading comprehension for cyrillic adult readers and that comprehension and reading speed might be affected by factors other than the font serifs alone. 

How the human brain perceives text is an area still under active research. The Bouma theory assumes that
text is perceived word by word rather than letter by letter and perhaps can partially explain Veytsman's results.  The supporters of the theory claim that readers perceive words as clusters of letters, similar to
logotypes, which communicate through their overall shape and outline rather than through their precise spelling. The Bouma model evolved from evidence that
was gathered in the fi eld of cognitive psychology since
the 1970s. Contemporary psychologists use the word
‘Bouma’ synonymously with \enquote{word shape} in tribute to
Herman Bouma, who discussed the concept in his paper
\enquote{Visual Interference in the Parafoveal Recognition of Initial
and Final Letters of Words} in 1973. The fovea is the
location at the back of the eye that allows us to depict
objects in detail. Surrounding the fovea is the parafoveal
area that is responsible for our peripheral vision. Herman
Bouma claimed that readers look at the centre of a word
while they recognise the surrounding letters using their
peripheral vision.

James Cattel is considered by many to be the first
psychologist to propose (in 1886) that reading results
from the recognition of complete words rather than
individual letters. The Bouma theory is supported by the
fact that spelling mistakes are missed significantly more
often where the shape of the incorrect word is consistent
with that of the correct word. On the other hand, the idea
of reading words on the basis of either their outlines or
their rhythm of ascending and descending letters has
raised doubts because the outline of words, as well as
the rhythm of ascending and descending letters, are
evidently not suffi cient for the recognition of words. But
the Bouma theory also relies on other criteria such as
word boundaries, which are determined by the blank
space between words, the frequency of words within
a text, as well as consistency in the linguistic structure
of phrases (for example, subject-verb-object). So the
hypothesis retains much of its credibility.

\subsection*{How type sells}

Of course the biggest effect on readers is when fonts are used for ``branding'' a product into consumers minds. 
Marketers have been brainwashing
consumers for years through the use of fonts. In \textit{Branding
With Type}  by Rogener, Pool, and Packhauser
(1995), a fervent argument is made for unique
but consistent typefaces as a crucial element of
corporate branding. Rogener et al. describe the
fonts used by IBM, Mercedes, Nivea, and
\marlborologo\ as instantly recognisable
internationally, and imply that the significant
investment by such companies in design and
copyright of trademarked fonts is worthwhile. 

For example, \citeauthor{rogener1995} discuss the Nivea
Bold typeface developed in 1992 by Gunther
Heinrich at advertising agency \textsc{TBWA} in
Hamburg, Germany, for skincare brand Nivea,
and claim that the Nivea Bold typeface has
effectively embodied the {\fboxsep=2pt \nivea \colorbox{blue700}{\textcolor{white}{NIVEA}}} brand’s \enquote{pure and simple} product philosophy {\fboxsep=2pt\nivea \colorbox{blue700}{\textcolor{white}{NONAME}}}. They link the font directly to profitability and Nivea’s worldwide 
product category market share of 35\% \footcite[p. 91]{rogener1995} 

\subsection*{Recommendations}

Having put all the research literature behind us, it is time to summarize the findings in a set of recommendations that we can apply in our own work. I call these recommendations as you may want to bend them a little, they are not really rules, like in other sections of this book. 

\begin{Rule}
Select a font family that is appropriate for your readers.
\end{Rule}

The reader persona and the expectations of the reader is the most important criterion in selecting fonts.
If you writing lecture notes for your 3rd grade students select a different font that the one you would use
for your graduate students. Study books in your area of specialization and research what is appropriate.

People do not like risk. Read Anne Tyler’s book \textit{The Accidental Tourist} (Vintage, 1995) for an example of
a risk-free life. The central character writes travel guides for Americans who
would rather not travel. The purpose of the guides is to insulate people against
risk. For example, he advises travellers always to have a good book with them
so that they don’t have to talk to the person next to them on the plane.
Similarly, the restaurants he recommends in Paris are all American.

\begin{Rule}
Select a font family that is appropriate for the medium. A printed article requires a different set of
fonts that one published on the web.
\end{Rule}

\begin{Rule}
Do not use more than three font faces on a page (headings, running titles, body of text). For backmatter and front matter a similar number of fonts can be used.
\end{Rule}

\begin{Rule}
Vary the size, decoration to distingusish the different parts of the document rather than the font face.
\end{Rule}

\begin{Rule} Consistency
\end{Rule}
A promise is implicit in a brand. From the combination of corporate messages and experience we believe that we will receive a certain experience from a brand.

Do not vary the \enquote{style} of your reports, lecture notes or books; be consistent in order to establish 
your own brand. What do you want to project? Most probably professionalism, knowledge, expertise? Project it through the contents and quality of your publications\footnote{Yes lecture notes, are publications}. If you change it, consider why? Companies rebrand but there is always a good reason behind it. They express the idea that \ldots hey look at us we changed.

\chapter{Fonts TeX and LaTeX}

\section{The choice of engine}

If you use only |pdfLaTeX| the range of fonts is rather limiting and I would highly recommend for any serious typesetting work to move onto |XeLaTeX| and the use of the package \pkg{fontspec} \citep{fontspec}. Another alternative is to use \lualatex. The latter is becoming more stable and is production ready to a large extend. It is expected to be the successor to pdfTeX.

One of the things I wanted to achieve with the \pkgname{phd} package was  to take care of different \tex engines, and to ensure that the package can be used irrespective of the \TeX\ engine used. 

Before we start outlining the scheme let us start, by demonstrating how to load one of the standard fonts provided by \latexe. We will load the Computer Modern font.\index{Computer Modern (font)} 

\begin{texexample}{How to load a font}{ex:fonts}
\newcommand{\fontdemo}[4][OT1]{
    \leavevmode
    \textcolor{thefontname}{#2}
    \fontencoding{#1}\fontfamily{#3}\selectfont#4 }
\bgroup
\fontdemo{CM}{cmtt}{ \alphabet\par}

\fox
\egroup

\hrule
\vskip 1in
\centerline{\bf A SHORT STORY}
\vskip 6pt
\centerline{\sl    by A. U. Thor} % !`?`?! (modified)
\vskip .5cm
Once upon a time, in a distant
  galaxy called \"O\"o\c c,
there lived a computer
named R.~J. Drofnats.

Mr.~Drofnats---or ``R. J.,'' as
he preferred to be called---% error has been fixed!
was happiest when he was at work
typesetting beautiful documents.
\vskip 1in
\hrule
%\vfill\eject

\meaning\ttdefault 
\end{texexample}

In the example we have used a number of convenience commands that are provided by the |phd| package.

\begin{docCmd} {alphabet} {}
  Typesets the letters of the English alphabet
\end{docCmd}  


\begin{docCmd}{fox}{}
  Typesets the fox passage
\end{docCmd}

The example  creates a convenience command to call the |computer modern typewriter| font and to print the alphabet.\footnote{The command \cs{alphabet} is provided by the \texttt{phd} package.} In this case we are asking \latex to load a font from the |cmtt| family. 

To load a font two things are required the encoding scheme [|OT1|] in the example and the somewhat cryptic font family name [|cmtt|].

\section{What is a character? And a glyph?}
\index{glyph}\index{character}
A character is an abstract
concept: the letter “A” is a character, while any
particular drawing of that character is a glyph. In many
cases there is one glyph for each character and one character
for each glyph, but not always.

The glyph used for the Latin letter “A” may also be
used for the Greek letter “Alpha”, while in Arabic writing
most Arabic letters have at least four different glyphs
(often vastly more) depending on what other letters are
around them.

\section{What's a font?}

As the \pkgname{fontinst} manual says: ``Once upon a time, this question was easily answered: a font is a set of type
in one size, style, etc. There used to be no ambiguity, because a font was a
collection of chunks of type metal kept in a drawer, one drawer for each font'' \citet{fontinst}.


With digital typesetting, things are more complicated. What a font
\textit{is} isn't easy to pin down. A typical use of a PostScript font with \latex might
use these elements:

\begin{enumerate}
\item Type 1 printer font file
\item Bitmap screen font file
\item Adobe font metric file (afm file)
\item \tex font metric file (tfm file)
\item Virtual font file (vf file)
\item font definition file (fd file)
\end{enumerate}

Looked at from a particular point of view, each of these files \textit{is} the font. So
what’s going on? Every text font in \latex has five attributes:

\index{encoding schemes>OML}\index{encoding schemes>OMS}\index{encoding schemes>OMX}\index{encoding schemes>U}\index{encoding schemes>OML}
\index{encoding schemes}\index{encoding schemes>OT1}



\section{The low-level interface}

The low level commands are mainly  used to define new commands in packages and classes.
It is helpful to understand the internal organization of fonts in \latex's font selection
scheme New Font Selection Scheme (NFSS). Although termed new, one needs to understand
that this is now almost twenty years old.

One of the goals of \latex's font selection scheme was to allow a rational selection with
algorithms guided by the principles of generic markup \citep{companion}. 

Internally the \latex kernel keeps track of five independent attributes. The ``current encoding",
the ``current family'', the ``current series'', the ``current shape'', and the ``current size''. This was introduced in NFSS release 2 after it became clear that real support in multiple languages would be possible only by maintaining the character-encoding scheme independly of the
other font attributes.


\subsection{Encoding Schemes}

The \textit{encoding} scheme (in the example |OT1|) provides information as to which glyph goes into what slot in a font table. These font tables can be printed using |fonttest.tex|. We show the test for |cmtt10| in Figure~\ref{fig:fonttest}. The
most common values for the font encoding are:
\medskip

\begin{longtable}{ll}
OT1   & TEX text\\
T1     & TEX extended text\\
OML  & TEX math italic\\
OMS  & TEX math symbols\\
OMX  & TEX math large symbols\\
U       & Unknown\\ 
L\meta{xx}  &A local encoding\\
\end{longtable}
\medskip
\begin{description}
\item[family]\index{fonts>family}\index{fonts>cmr}\index{fonts>cmss}
\index{fonts>cmtt}
The name for a collection of fonts, usually grouped under a common
name by the font foundry. For example, `Adobe Times', `ITC Garamond',
and Knuth's `Computer Modern Roman' are all font families.

There are far too many font families to list them all, but some common ones
are:

\begin{longtable}{rl}
cmr  &Computer Modern Roman\\
cmss &Computer Modern Sans\\
cmtt &Computer Modern Typewriter\\
cmm  &Computer Modern Math Italic\\
cmsy &Computer Modern Math Symbols\\
cmex &Computer Modern Math Extensions\\
ptm  &Adobe Times\\
phv  &Adobe Helvetica\\
pcr  &Adobe Courier\\
\end{longtable}

\item[series] How heavy or expanded a font is. For example, `medium weight', `narrow'
and `bold extended' are all series.

\item[shape] The form of the letters within a font family. For example, `italic',
`oblique' and `upright' (sometimes called `roman') are all font shapes. The most common values for the font shape are:

\begin{longtable}{ll}
n  &Normal (that is `upright' or `roman')\\
it &Italic\\
sl &Slanted (or `oblique')\\
sc &Caps and small caps\\
ui & upright italic shape\\
ol &  ouline shape\\
\end{longtable}

To change the shape attribute the \docAuxCommand{fontshape} is used to change the shape
attribute. If you try this and it does not work, it means that the font you have selected
does not have the font shape you have requested. You may need to specify an appropriate
font family as well:

\begin{texexample}{usefont}{ex:usefont}
{\usefont{T1}{cmr}{m}{sc}  \raggedright \lorem}
\end{texexample}

\item[size] The design size of the font, for example `10pt'. If no dimension is specified, `pt' is assumed.
\end{description}

These five parameters specify every \latex
font, for example:

\begin{longtable}{lll}
|LaTeX| specification &Font  &TEX font name\\
|OT1 cmr m n 10|      &Computer Modern Roman 10 point &cmr10\\
|OT1 cmss m sl 1pc|   &Computer Modern Sans Oblique 1 pica &cmssi12\\
|OML cmm m it 10pt|   &Computer Modern Math Italic 10 point &cmmi10\\
|T1 ptm b it 1in|  &Adobe Times Bold Italic 1 inch &ptmb8t at 1in\\
\end{longtable}

\begin{texexample}{Th usefont command}{ex:usefont}
\bgroup
\fontsize{12}{14pt} \lorem
\lorem
\egroup
\end{texexample}

When you get a font error or an underfull or overfull box \tex always will print an error with the font specification in full as shown below:

\begin{teX}
LaTeX Font Warning: Font shape `EU1/cmr/m/sc' undefined
(Font)              using `EU1/cmr/m/n' instead on input line 160.
\end{teX}

\section{Setting several font attributes}

Often it is required that several attributes of a particular font need to be set. For this
task \latex provides the command \docAuxCommand{usefont}. This command takes four arguments: the encoding, family, series, and shape. The command updates those and then calls \docAuxCommand{selectfont}. If you also need to change the size and baseline skip, place
a \docAuxCommand{fontsize} command in front of it. 

\begin{texexample}{Th usefont command}{ex:usefont}
\bgroup
\fontsize{12}{14pt}\usefont{OT1}{cmdh}{bc}{it}
\lorem\par

\egroup
\end{texexample}


%\begin{tabbing}
%\ttverb\textvisiblespace\quad\=bbbbbbbbbbbbbbbbbbbbbbbbbbbbbbb\=b'b'\=cccccccccccccc\kill
%\ttverb\`{}               \>OT1, T1, EU1, EU2\>   \a`{}\> (grave)      \\
%\ttverb\'{}               \>OT1, T1, EU1, EU2\>   \a'{}\> (acute)      \\
%\ttverb\^{}               \>OT1, T1, EU1, EU2\>   \^{}\>  (circumflex) \\
%\ttverb\~{}               \>OT1, T1, EU1, EU2\>   \~{}\>  (tilde)      \\
%\ttverb\"{}               \>OT1, T1, EU1, EU2\>   \"{}\>  (umlaut)     \\
%\ttverb\H{}               \>OT1, T1, EU1, EU2\>   \H{}\>  (Hungarian umlaut) \\
%\ttverb\r{}               \>OT1, T1, EU1, EU2\>   \r{}\>  (ring)       \\
%\ttverb\v{}               \>OT1, T1, EU1, EU2\>   \v{}\>  (ha\v{c}ek)  \\
%\ttverb\u{}               \>OT1, T1, EU1, EU2\>   \u{}\>  (breve)      \\
%\ttverb\t{}               \>OT1, T1, EU1, EU2\>   \t{}\>  (tie)        \\
%\ttverb\={}               \>OT1, T1, EU1, EU2\>   \a={}\> (macron)     \\
%\ttverb\.{}               \>OT1, T1, EU1, EU2\>   \.{}\>  (dot)        \\
%\ttverb\b{}               \>OT1, T1, EU1, EU2\>   \b{}\>  (underbar)   \\
%\ttverb\c{}               \>OT1, T1, EU1, EU2\>   \c{}\>  (cedilla)    \\
%\ttverb\d{}               \>OT1, T1, EU1, EU2\>   \d{}\>  (dot under)  \\
%\ttverb\k{}               \>T1    \>   \k{}\>  (ogonek)     \\
%\ttverb\AE                \>OT1, T1, EU1, EU2\>   \AE \>               \\
%\ttverb\DH                \>T1    \>   \DH \>               \\
%\ttverb\DJ                \>T1    \>   \DJ \>               \\
%\ttverb\L                 \>OT1, T1, EU1, EU2\>   \L  \>               \\
%\ttverb\NG                \>T1    \>   \NG \>               \\
%\ttverb\OE                \>OT1, T1, EU1, EU2\>   \OE \>               \\
%\ttverb\O                 \>OT1, T1, EU1, EU2\>   \O  \>               \\
%\ttverb\SS                \>OT1, T1, EU1, EU2\>   \SS \>               \\
%\ttverb\TH                \>T1    \>   \TH \>               \\
%\ttverb\ae                \>OT1, T1, EU1, EU2\>   \ae \>               \\
%\ttverb\dh                \>T1    \>   \dh \>               \\
%\ttverb\dj                \>T1    \>   \dj \>               \\
%\ttverb\guillemotleft     \>T1    \>   \guillemotleft  \> (guillemet) \\
%\ttverb\guillemotright    \>T1    \>   \guillemotright \> (guillemet) \\
%\ttverb\guilsinglleft     \>T1    \>   \guilsinglleft  \> (guillemet) \\
%\ttverb\guilsinglright    \>T1    \>   \guilsinglright \> (guillemet) \\
%\ttverb\i                 \>OT1, T1, EU1, EU2\>   \i  \>               \\
%\ttverb\j                 \>OT1, T1, EU1, EU2\>   \j  \>               \\
%\ttverb\l                 \>OT1, T1, EU1, EU2\>   \l  \>               \\
%\ttverb\ng                \>T1    \>   \ng \>               \\
%\ttverb\oe                \>OT1, T1, EU1, EU2\>   \oe \>               \\
%\ttverb\o                 \>OT1, T1, EU1, EU2\>   \o  \>               \\
%\ttverb\quotedblbase      \>T1    \>   \quotedblbase   \>   \\
%\ttverb\quotesinglbase    \>T1    \>   \quotesinglbase \>   \\
%\ttverb\ss                \>OT1, T1, EU1, EU2\>   \ss \>               \\
%\ttverb\textasciicircum   \>OT1, T1, EU1, EU2\>   \textasciicircum \>  \\
%\ttverb\textasciitilde    \>OT1, T1, EU1, EU2\>   \textasciitilde  \>  \\
%\ttverb\textbackslash     \>OT1, T1, EU1, EU2\>   \textbackslash   \>  \\
%\ttverb\textbar           \>OT1, T1, EU1, EU2\>   \textbar         \>  \\
%\ttverb\textbraceleft     \>OT1, T1, EU1, EU2\>   \textbraceleft   \>  \\
%\ttverb\textbraceright    \>OT1, T1, EU1, EU2\>   \textbraceright  \>  \\
%\ttverb\textcompwordmark  \>OT1, T1, EU1, EU2\>   \textcompwordmark\> (invisible) \\
%\ttverb\textdollar        \>OT1, T1, EU1, EU2\>   \textdollar      \>  \\
%\ttverb\textemdash        \>OT1, T1, EU1, EU2\>   \textemdash      \>  \\
%\ttverb\textendash        \>OT1, T1, EU1, EU2\>   \textendash      \>  \\
%\ttverb\textexclamdown    \>OT1, T1, EU1, EU2\>   \textexclamdown  \>  \\
%\ttverb\textgreater       \>OT1, T1, EU1, EU2\>   \textgreater     \>  \\
%\ttverb\textless          \>OT1, T1, EU1, EU2\>   \textless        \>  \\
%\ttverb\textquestiondown  \>OT1, T1, EU1, EU2\>   \textquestiondown\>  \\
%\ttverb\textquotedbl      \>T1    \>   \textquotedbl    \>  \\
%\ttverb\textquotedblleft  \>OT1, T1, EU1, EU2\>   \textquotedblleft\>  \\
%\ttverb\textquotedblright \>OT1, T1, EU1, EU2\>   \textquotedblright\> \\
%\ttverb\textquoteleft     \>OT1, T1, EU1, EU2\>   \textquoteleft   \>  \\
%\ttverb\textquoteright    \>OT1, T1, EU1, EU2\>   \textquoteright  \>  \\
%\ttverb\textregistered    \>OT1, T1, EU1, EU2\>   \textregistered  \>  \\
%\ttverb\textsection       \>OT1, T1, EU1, EU2\>   \textsection     \>  \\
%\ttverb\textsterling      \>OT1, T1, EU1, EU2\>   \textsterling    \>  \\
%\ttverb\texttrademark     \>OT1, T1, EU1, EU2\>   \texttrademark   \>  \\
%\ttverb\textunderscore    \>OT1, T1, EU1, EU2\>   \textunderscore  \>  \\
%\ttverb\textvisiblespace  \>OT1, T1, EU1, EU2\>   \textvisiblespace\>  \\
%\ttverb\th                \>T1    \>   \th              \>
%\end{tabbing}                        

Do note that when you use the \pkgname{hyperref}, you will get a surprise, all the commands have been converted to "PU" encoding. This is mostly harmless and is  done in order for |hyperref| to mark bookmarks\footnote{http://tex.stackexchange.com/questions/198810/why-does-the-hyperref-package-changes-encoding-of-font-commands} in a safe way.

\begin{texexample}{font encoding}{ex:encoding}
\meaning\textasciitilde\\
\meaning\"\\
\meaning\NG\\
\meaning\k\\
\meaning\alpha
\meaning\printfontparams

\printfontparams
\end{texexample}

A peek at the \docFile{puenc.def} reveals the inner workings
of the encoding mechanism.

\begin{phdverbatim}
\ProvidesFile{puenc.def}
  [2003/01/20 v6.73l
  Hyperref: PDF Unicode definition (HO)]
\DeclareFontEncoding{PU}{}{}
\DeclareTextCommand{\textLF}{PU}{\80\012} % line feed
\DeclareTextCommand{\textCR}{PU}{\80\015} % carriage return
\DeclareTextCommand{\textHT}{PU}{\80\011} % horizontal tab
\DeclareTextCommand{\textBS}{PU}{\80\010} % backspace
\DeclareTextCommand{\textFF}{PU}{\80\014} % formfeed
\DeclareTextAccent{\`}{PU}{\textgrave}
\DeclareTextAccent{\'}{PU}{\textacute}
\DeclareTextAccent{\^}{PU}{\textcircumflex}
\end{phdverbatim}

\printfontparams 


\latex uses a number of other files to get to the particular file that contains the font metrics file |cmtt10| and to find the appropriate file. For the original Knuth fonts the filenames have been kept the same, essentially as a request from Knuth that one should not change them.

Most of the difficulty in selecting and using fonts is figuring the encoding scheme and the Karl Berry naming scheme. In the Example~\ref{ex:fonts} we select the \cs{fontfamily} |cmtt| which is computer modern type writer and then we invoke the macros for the shape \cs{itshape} and print the |alphabet|. The macro \cmd{\alphabet} is build-in the |phd| package as we use it in a few places.

\begin{figure}[htbp]
\centering

\hspace*{-2cm}\includegraphics[width=\linewidth]{./images/testfont-output.pdf}

\caption{Output from testfont.tex for cmtt10 font}
\label{fig:fonttest}
\end{figure}


\subsection{The Postscript fonts}

With Adobe reader a number of fonts come pre-packaged and these have been incorporated into \latex2e. These fonts can be found in all \tex distributions. The \textit{Times New Roman} is named |ptm|. 

\begin{texexample}{The Postscript fonts}{ex:postscriptfonts}
\raggedright
\begin{tabular}{@{}>{\sffamily\bfseries}rl}
\fonttitle{The Adobe `LaserWriter 35', 10 typefaces in a total of 35
different styles, standard on all PostScript printers}

%\thefont{Avant Garde Book}{pag}{\fontsize{9}{9}\selectfont\sample}
%\thefont{Bookman Light}{pbk}{\sample}
\thefont{Courier}{pcr}{\sample}
\thefont{Helvetica}{phv}{\sample}
\thefont{New Century Schoolbook}{pnc}{\sample}
\thefont{Palatino}{ppl}{\sample}
\thefont[U]{Symbol}{psy}{\sample}
\thefont{Zapf Chancery Medium Italic}{pzc}{\fontsize{12}{12}\selectfont\itshape\sample}
\thefont[U]{Zapf Dingbats}{pzd}{\sample}
\end{tabular}
\end{texexample}

Using the |phd| package we can come closer to the |fontspec| or LuaTeX way of doing things and use longer font names as those found in the operating system.


Activating the key will set the font to |pzc| and unless is within a group
will typeset the rest of the document with this typeface.




\makeatletter
\def\fontname@cx{}
\cxset{font name/.is choice,
       font name/Zapf Chancery Medium Italic/.code={\fontfamily{pzc}\selectfont},
 font name/courier/.code={\fontfamily{pcr}\selectfont},
font name/Helvetica/.code={\fontfamily{phv}\selectfont},
font name/helvetica/.code={\fontfamily{phv}\selectfont},
font name/Bookman Light/.code={\fontfamily{pbk}\selectfont},
font name/bookman/.code={\fontfamily{pbk}\selectfont},
font name/Utopia/.code={\fontfamily{put}\selectfont},
font name/Palatino/.code={\fontfamily{put}\selectfont},
font name/Old Standard/.store in=\fontname@cx,
font name/Junicode/.code={
\fontspec{Junicode}\addfontfeature{StylisticSet=2}}
}
\makeatother

\begin{docKey}[key]{font name} { = \marg{Zapf Chancery Medium Italic}} {}
\cxset{font name=Zapf Chancery Medium Italic}
\bgroup \itshape This is how it is typeset\egroup
\end{docKey}


\begin{docKey}[phd] {font name}{ = \marg{Old Standard}} {}
Setting the key to \texttt{Old Standard} will typeset the next sample in \texttt{OldStandard-Regular}, |Stylistic Set=2|. 

\bgroup
\parindent1em\itshape
\cxset{font name=Old Standard}

\aliceii

abcdefg
\egroup
\end{docKey}



\begin{docKey}[phd]{font name}{ = \marg{Junicode}} {}
Setting the key to \texttt{Junicode} will typeset the next sample in \texttt{Junicode}, \texttt{Stylistic Set=2}. 

\bgroup
\parindent1em\itshape
\cxset{font name=Junicode}

\aliceii

abcdefg
\egroup
\end{docKey}




\begin{docKey}[phd] {font name }{=\marg{Bookman Light or bookman}} {}
Bookman Light or |bookman|
\end{docKey}

\bgroup
\cxset{font name=bookman}
\aliceiii
\egroup


\begin{docKey}[phd] {font name }{ =\marg{Utopia or utopia}} {}

\end{docKey}

\bgroup
\cxset{font name=Utopia}

\renewcommand{\LettrineFontHook}{\fontfamily{put}\fontseries{bx}}%
\par\leavevmode

\lettrine[lines=5, lhang=0.1,lraise=0.28,findent=1pt]{g}{oats} are animals found in all sort of places. The paragraph has been set using the font family |utopia|. The comment about the goats was just to get the letter g.
comfortable in mountain areas. I don't recall Alice  They are more
comfortable in mountain areas. I don't recall Alice  They are more
comfortable in mountain areas. I don't recall Alice  They are more
comfortable in mountain areas. I don't recall Alice  They are more
comfortable in mountain areas. I don't recall Alice  They are more
comfortable in mountain areas. I don't recall Alice  They are more
comfortable in mountain areas. I don't recall Alice 
\egroup

\renewcommand{\LettrineFontHook}{\fontfamily{phv}\fontseries{bx}}%



\par\leavevmode

\lettrine[lines=5, lhang=0.1,lraise=0.28,findent=1pt]{g}{oats} are animals found in all sort of places. The paragraph has been set using the font family |utopia|. The comment about the goats was just to get the letter g.
comfortable in mountain areas. I don't recall Alice  They are more
comfortable in mountain areas. I don't recall Alice   They are more
comfortable in mountain areas. I don't recall Alice   They are more
comfortable in mountain areas. I don't recall Alice  They are more
comfortable in mountain areas. I don't recall Alice  They are more
comfortable in mountain areas. I don't recall Alice  They are more
comfortable in mountain areas. I don't recall Alice 

\medskip



\lettrine{G}{o}ats are among the earliest animals domesticated by humans. The most recent genetic analysis confirms the archaeological evidence that the wild Bezoar ibex of the Zagros Mountains are the likely origin of almost all domestic goats today. Neolithic farmers began to herd wild goats for easy access to milk and meat, primarily, as well as for their dung, which was used as fuel, and their bones, hair, and sinew for clothing, building, and tools. The earliest remnants of domesticated goats dating 10,000 years before present are found in Ganj Dareh in Iran. Goat remains have been found at archaeological sites in Jericho, Choga Mami Djeitun and Çay\"on\"u, dating the domestication of goats in Western Asia at between 8000 and 9000 years ago.\footnote{Text is from wikipedia's article for the domesticated goat.}

As you have observed we did not change the normal size of paragraphs, but the examples demonstrate that differences in font families also affect the visual size of the typeset text. |Helvetica| is normally scaled down to 0.95 and |Chancery| is scaled a little bit up or we use a larger font size.




\subsection{Additional free fonts for use with \LaTeX}

A number of archaic and other fonts are available in the \latexe historical collection. These are very impressive. They also provide in most instances transliterations.

\begin{tabular}{@{}>{\sffamily\bfseries}rl}
\fonttitle{\textit{The Historical Collection}}
\thefont{Cypriot}{cypr}{\fontsize{7}{7}\selectfont\sample}
\thefont{Linear `B'}{linb}{\fontsize{8}{8}\selectfont\sample}
\thefont{Phoenician}{phnc}{\sample}
\thefont{Runic}{fut}{TYPOGRAPHIA ARS ARTIUM OMNIUM CONSERVATRIX}
%\thefont{Rustic}{rust}{\sample}
\thefont[U]{Bard}{zba}{\sample}
\thefont{Uncial}{uncl}{\sample}[-3pt]
\end{tabular}

\subsection{Uncial fonts}

\newcommand{\ABC}{ABCDEFGHIJKLMNOPQRSTUVWXYZ}
%\newcommand{\abc}{abcdefghijkl mnopqrstuvwxyz}
\newcommand{\punct}{.,;:!?`' \&{} () []}
\newcommand{\figs}{0123456789}
\newcommand{\dashes}{- -- ---}
\newcommand{\sentence}{%
this is an example of the uncial font. now is the time for all good
men, and women, to come to the aid of the party while the quick brown fox
jumps over the lazy dog:}


\newcommand{\Sentence}{%
This is an example of the Uncial font. Now is the time for all good
men, and women, to come to the aid of the party while the quick brown fox
jumps over the lazy dog:}

Peter Wilson's \pkgname{uncial} package provides a useful uncial font and is easily used by just including the file. 

\begin{texexample}{Unical fonts example}{}
\begin{center}
The Uncial Huge normal font. \\ \par
{\unclfamily\Huge \ABC\\ \alphabet\\ \punct{}\dashes\\ \figs\\ \par }
\end{center}
\end{texexample}




%The following fonts are all selections from Yiannis Haralambous collection and we categorize them as other scripts collection.
%
%\begin{tabular}{@{}>{\sffamily\bfseries}rl}
%\fonttitle{\textit{The Other Scripts Collection}}
%\thefont{Calligraphic}{zca}{\fontsize{15}{15}\selectfont\sample}
%\thefont[U]{Fraktur}{yfrak}{%
%	Alle\char'215\ Verg\"angliche ist nur ein Gleichni
%	Da\char'215\ Unzul\"angliche hier wird'\char'215\
%	Ereigni\char'215;}
%
%\thefont[U]{Schwabacher}{yswab}{%
%	Da\char'215\ Unbeschreibliche hier wird'\char'215\ getan / 
%	Da\char'215\ Ewig-Weibliche zieht un\char'215\ hinan!}
%\thefont[U]{`Gothic'}{ygoth}{If it plese ony man spirituel or temporel
%to bye any pye\char'140\ of two and thre comemoraci\~o\char'140}[6pt]
%\thefont[U]{Decorative Initials}{yinit}{\fontsize{8}{8}\selectfont
%\raisebox{-12pt}{YIANNIS}}
%\end{tabular}


\section{Dingbat and Symbol Fonts}
\index{fonts>Zapf Dingbats}

Fonts containing collections of special symbols, which are normally not found in a text font, are called  \textit{dingbats}. One such font, the PostScript font Zapf Dingbats, is available if you use the |pifont| package, originally written by Sebastian Rahtz, and now part of |PSNFSS|. This is loaded automatically by the |phd| package. (See also implementation code at Page \pageref{dingbats}).

The parameter for the \cs{ding} command is an integer that specifies the character to be typeset according to Table~\ref{tbl:dingbats}. For example |\ding{38}| gives \ding{38}.

For Open Type fonts the |Wingdings| family can be found on Windows systems. The advent of Unicode and the universal character set allowed commonly used dingbats to be given their own character codes. Although fonts claiming Unicode coverage will contain glyphs for dingbats \textit{in addition} to alphabetic characters continue to be popular, primarily for ease of input. Such fonts are sometimes known as \textit{pi fonts}.\index{fonts>pi fonts}

\subsection{Unicode Dingbats block}

The Dingbats block |U+2700-U+27BF| was added to the Unicode Standard in June, 1993, with the release of version 1.1. This code block  contains decorative character variants, and other marks of emphasis and non-textual symbolism. Most of its characters were taken from Zapf Dingbats. 

The Ornamental Dingbats block (|U+1F650–U+1F67F|) was added to the Unicode Standard in June 2014 with the release of version 7.0. This code block contains ornamental leaves, punctuation, and ampersands, quilt squares, and checkerboard patterns. It is a subset of dingbat fonts Webdings, Wingdings, and Wingdings 2. \footnote{See \url{http://std.dkuug.dk/jtc1/sc2/wg2/docs/n4115.pdf}}

A font that we will be using for many of the \XeLaTeX examples is |code2000|
and |code2001|. The fonts were designed by James Kas
\footnote{They can be downloaded at \url{http://www.alanwood.net/downloads/index.html}}. They are True type fonts. The fonts contain a respectable collection of more or less exotic Unicode characters both within the Basic Multilingual Plane (BMP). They were designed by James Kass and were freeware. Sadly the website is no longer available, but the files can be downloaded in the links I have provided. I have also included them in the distribution for the |phd| package, as they are such a useful tool.

\index{Unicode}\index{Basic Multilingual Plane}

\begin{docCmd} {codetwothousand} {}
   Loads the TrueType font \texttt{code2000.ttf}\index{fonts>code2000}
\end{docCmd} \index{code2001}

\begin{docCmd} {codetwothousandone} {}
   Loads the TrueType font \texttt{code2001}
\end{docCmd}

\begin{docCmd} {symbola} {}
  Loads the TrueType font \texttt{symbola}\index{fonts>Symbola}
\end{docCmd}

\index{fonts>Symbola}
\index{fonts>code2000}
\index{fonts>code2001}
\begin{phdverbatim}
\newfontfamily{\codetwothousand}{code2000.ttf}
  \newfontfamily{\codetwothousandone}{code2001.ttf}
  \newfontfamily{\symbola}{symbola.ttf}
\end{phdverbatim}




\index{fonts>wingdings}
\begin{texexample}{Wingdings}{ex:wingdings}
\ifxetex
   {\codetwothousand \symbol{9742} \symbol{9743}
    Katakana (片仮名, カタカナ)
   \codetwothousandone \symbol{57508}
   \symbola \symbol{9816}
  }
\else
  \ifluatex
  {\codetwothousand \symbol{9742} \symbol{9743}
    Katakana (片仮名, カタカナ)
   \codetwothousandone \symbol{57508}
   \symbola \symbol{9816}
  }
  \else
   Compile the document with XeTeX to see the example
  \fi 
\fi
\end{texexample}

Another useful font for experimenting if you are using a Windows computer is |Arial Unicode MS|.
\index{Arial Unicode MS (font)}\index{fonts>Arial Unicode MS}

%\begin{smallverbatim}
%\documentclass{article}
%\usepackage{fontspec}
%\setmainfont{Arial Unicode MS}
%\usepackage{multicol}
%\setlength{\columnseprule}{0.4pt}
%\usepackage{multido}
%\setlength{\parindent}{0pt}
%\begin{document}
%
%\begin{multicols}{8}
%\multido{\i=0+1}{"10000}{^^A from U+0000 to U+FFFF
%  \iffontchar\font\i %
%    \makebox[3em][l]{\i}%
%    \symbol{\i}\endgraf
%  \fi
%}
%\end{multicols}
%\centering
%\symbol{57352}
%\end{document}
%\end{smallverbatim} 


%
%\begin{multicols}{8}
%\ExplSyntaxOn
%\bgroup
%\symbola
%\multido{\next=0+1}{"10000}{
%  \iffontchar\font\next %
%     \makebox[3em][l]{\next}%
%    \symbol{\next}\endgraf
%  \fi
%\egroup  
%\ExplSyntaxOff
%}
%\end{multicols}

The |Symbola Font| has many other symbols, including chess and even Mahjong symbols\index{Mahjong}.\footnote{\url{http://users.teilar.gr/~g1951d/Symbola.pdf}}. \person{George Douros} has packaged many of the fonts for archaic languages, but sadly the substitution mechanisms of \latexe do not always map the fonts properly.

With |LuaLaTeX| and |XeLaTeX|, \tex has moved into the twenty-first century and its usefulness can now be extended to many other languages and fields. 


\section{Naming digital fonts}

Commercial and Open Source fonts come as a set of several files. The \enquote{.pfb} file and less frequently, a |.pfa| file or other files depending on the type of font and the operating system and provider. The metric information file resides in an associated |.afm| file. Other files, with extensions |.inf| (information) and |.pfm| are irrelevant to \latex and \tex.

Fonts already have names given them by their designers. The problem lies in associating this name with the font files. Restriction of operating systems originally from PC-DOS dates, restricted to the initial part of file names to eight characters.

\subsection{Karl Berry naming scheme}\index{Karl Berry Scheme}\index{fonts>Karl Berry scheme}

The original inspiration for Fontname was Frank Mittelbach and Rainer Schoepf's article in TUGboat 11(2) (June 1990). This led to an article by Karl Berry in TUGboat 11(4) (November pages 512-519).

Karl Berry then suggested a system---with many limitations, but perhaps the best that could have been done in its time, for mapping a lengthy font name into a file name that was eight or fewer characters long. If the font files are renamed accordingly then we can deduce the nature of the font by examining its file name. The scheme did not apply to the original Computer Modern fonts that retained their original names \citep{fontname}.

This scheme assumes that only eight characters or fewer can be available for naming the font. These eight characters look like,

\begin{verbatim}
FNNW[S][V]7V
\end{verbatim}

Some additional comments on this shorthand notation is in order. 
The most common foundry abbreviations are |p| for Adobe (from PostScript), \textbf{b} for BitStream, and \textbf{m} Monotype. A font flouting this scheme will begin with a z.

The next two letters are reserved for the typeface name. The hundreds and hundred of available faces guarantees  that many of these will be cryptic, even for the most common typefaces---Adobe Garamond is |ad|. 





\section{The phd package interface.}

By design feature options for XeTeX/XeLaTeX have currently been restricted. The reason behind this decision is that I was concerned that I would have added a complicated interface with very little reason as to its use. I opted for a more semantic approach and expect the user to define custom macros to handle anything else.
\medskip

\subsection{Sizes}

The standard LaTeX classes, as well as classes such as Koma or Memoir allow for the user to select
the default document font size. Based on this size LaTeX will load a |.clo| file with commands
to set the scale of the document, such as |\small|, |\large| etc. It also defines relevant spacing
for the |\small| and |\footnotesize| including list margins.

\begin{docKey}[phd]{main font-size}{= \meta{dim}}{11pt}
The \meta{dim} can take any value withing the range of valid font sizes with the largest being over
2,000pts.\footnote{Currently there is an experimental value that can be used in lieu of \meta{dim}, which is \meta{auto}. This will automatically size the font size and lineheight based on the width of the textblock.}
\end{docKey}

The key is set automatically by the |phddoc| class. However, it can be overwritten at any stage.

\subsection{Font face}
 A particular font-face for a document division or other document element can be set, using \meta{element name}{}|font-size| = |Georgia|

\begin{docKey}{main font-face}{= font name}{default = Georgia} 
\end{docKey}

\keyval{mainfont}{\marg{font1,font2,font3}}{A comma separated list of one or more font-names. The main font will be set to the first font found.}
\keyval{chapterfont}{\marg{font1,font2,font3}}{A comma separated list of one or more font-names. The main font will be set to the first font found.}
\keyval{sectionfont}{\marg{font1,font2,font3}}{A comma separated list of one or more font-names. The main font will be set to the first font found.}
\keyval{contentsfont}{\marg{font1,font2,font3}}{A comma separated list of one or more font-names. The main font will be set to the first font found.}

\begin{docKey}[phd] {bibliography font}{ = \marg{font1,font2,font3}} {}
A comma separated list of one or more font-names. The main font will be set to the first font found.
\end{docKey}

Note that the package will first check if is running under XeTeX. If it does it will execute the commands and load the macros, otherwise it will fall back on pdfLaTeX commands.

\section{Viewing and selecting fonts}

\subsection{Typefaces that come with the standard \LaTeX\ distribution}
{
\raggedright
\begin{tabular}{@{}>{\sffamily\bfseries}rl}
\fonttitle{Computer Modern (CM), \LaTeX's default typeface}
\thefont{CM Roman}{cmr}{\sample}
\thefont{CM Italic}{cmr}{\itshape\sample}
\thefont{CM Slanted (Oblique)}{cmr}{\slshape\fox}
\thefont{CM Bold}{cmr}{\fontseries{b}\selectfont\sample}
\thefont{CM Bold Extended}{cmr}{\bfseries\sample}
\thefont{CM Bold Italic}{cmr}{\itshape\bfseries\sample}
\thefont{CM Bold Slanted}{cmr}{\slshape\bfseries\sample}
\thefont{CM Caps \& Small Caps}{cmr}{\scshape\sample}
\thefont{CM Sans-Serif}{cmss}{\sample}
\thefont{CM Sans-Serif Oblique}{cmss}{\itshape\sample}
\thefont{CM Sans-Serif Bold}{cmss}{\bfseries\sample}
\thefont{CM Typewriter}{cmtt}{\sample}
\thefont{CM Typewriter Italic}{cmtt}{\itshape\sample}
\thefont{CM Typewriter Bold}{cmtt}{\bfseries\sample}
\thefont{CM Typewriter C\&SC}{cmtt}{\scshape\sample}
\thefont[OMS]{CM Mathematics}{cmsy}{$E=mc^2$\qquad}
\thefont{CM `Dunhill'}{cmdh}{\sample}
\thefont{CM `Fibonacci'}{cmfib}{\sample}
\end{tabular}
}\index{fonts>Fibonacci}\index{fonts>Dunhill}
\section{Discussion}


Unfortunately, even with the best will loading fonts will always be a difficult task in \TeX\. Hopefully the interface provided will result in better separation of presentation from content and offers consistency in the styling of documents. Nothing prevents you from adding normal macros to styles. Each style can be treated as a package in many respects.



\section{XeLaTeX and LuaLaTeX}


\bgroup
\fontspec{Verdana}
\begin{minipage}[t]{.2\linewidth}
\hbox to \linewidth{\hfill\hfill Verdana\hspace{2em}}
\end{minipage}
\begin{minipage}[t]{.75\linewidth}
\addfontfeature{ItalicFeatures={Alternate = 1}}
\noindent\fox\\
\alphabet\\
\punctuation\\
\frogking
\end{minipage}
\egroup

\bgroup
\fontspec{Calibri}
\begin{minipage}[t]{.2\linewidth}
\hbox to \linewidth{\hfill\hfill Calibri\hspace{2em}}
\end{minipage}
\begin{minipage}[t]{.65\linewidth}
\addfontfeature{ItalicFeatures={Alternate = 1}}
\noindent\fox\\
\alphabet\\
\textsc{\alphabet}\\
\punctuation\\
\frogking
Θαμκυαμ πλαθονεμ ραθιονιβυς ναμ ει, δυις περπετυα σιθ αδ, νες ιδ δισυντ σοντεντιωνες. Κυι σινθ μυνδι εα, φιμ αν γραεσω ιυδισαβιτ, εραθ δολορ φιρθυθε υθ δυο. Συ νοσθερ οπθιων ευμ, μει ερος προβο φιερενθ ευ. Ιυς μανδαμυς τωρκυαθος εξπεθενδις ιδ. Σεδ θε νιβχ νονυμυ δελισαθισιμι, φιμ νο νιβχ λαβωραμυς, σεα εα δισο ποσιμ αντιωπαμ.
\end{minipage}
\egroup

\normalfont
\section{Utilities for testing fonts}

The package \pkg{fonttable} is an extension and re-implementation of Donald Knuth’s \docFile{testfont.tex}, which
is available from CTAN. The package was developed by \person {Peter}{Wilson} and currently maintained by \person{Will}{Robertson}\citep{fonttable}. It provides a number of utility commands for typesetting font tables.


The {fonttable}\marg{font} takes the font file as an  argument and typesets it in a nice table. 

% \ifengine
%\ifxetex
% \else
%  \fonttable{pzdr}
%\fi

A great tool to inspect a True Type font on the command line is \texttt{luaotfload-tool}:

\begin{teXXX}
  luaotfload-tool --find="Iwona" --inspect
\end{teXXX}

\section{ \texttt{.tfm } files}


When you tell \tex that you will be using a particular font, it has to find out information about that font. This information is stored in a file with the extension \docfileextension{.tfm}. For example when you say:

|\font a=cmr10|

\noindent \tex looks for  a file named |cmr10.tfm|. If this is not found then an error is issued |Lookup failed on file CMR10.TFM|

Generall speaking, a font's |.tfm| file contains information about the height, width and depth of all the characters in the font plus kerning and ligature information. So, |cmr10.tfm| might say that the lower-case "d" in CMR10 is 5.5 points wide, 6.94 points high, etc. This is the information that \tex uses to make its lowest-level boxes---those around characters. See the \tex manual for information about what \tex does with these boxes. Note the |.tfm| files do not contain any information that is device dependent. Only device-drivers read \tex's |dvi| output files can use that sort of information.


\section{Fonts for Far East Languages}

In internationalization, CJK is a collective term for the Chinese, Japanese, and Korean languages, all of which use Chinese characters and derivatives (collectively, CJK characters) in their writing systems. Occasionally, Vietnamese is included, making the abbreviation CJKV, since Vietnamese historically used Chinese characters as well.
The characters are known as hànzì in Chinese, kanji in Japanese, hanja in Korean, and Chữ Nôm in Vietnamese.\index{kanji}\index{hanja}\index{hànzì}\index{CJK}\index{CJKV}


\subsection{Selecting a font}

The easiest way is with Will Robertson's \pkgname{fontspec} package. In this sample, we have used the \texttt{SimSun} font, which can be found on windows machines:

\begin{verbatim}
\usepackage{fontspec}
\setromanfont{SimSun}
\end{verbatim}
in the preamble, to use a Far Eastern font as the initial default typeface.

\section{Entering CJK text}

You can just enter Unicode text directly in the document: 你好. Don't use legacy \LaTeX\ packages such as \verb|inputenc| or \verb|CJK|, as \XeTeX\ reads the text as Unicode characters, not the separate byte codes of UTF-8 sequences, and passes them directly to the Unicode font. (Actually, it would probably be possible to use \verb|\XeTeXinputencoding "bytes"| and work with legacy \LaTeX\ input encoding support. But then you're pretty much committed to all the old encoding and font machinery, and there's not much point in using the \XeTeX\ engine at all.)


\begin{comment}
\setromanfont{Times New Roman}

\subsection{A CJK environment}

\newenvironment{CJK}{\fontspec[Scale=0.9]{SimSun}}{}

\newcommand{\cjk}[1]{{\fontspec[Scale=0.9]{SimSun}#1}}

Rather than selecting a CJK font as the main document typeface, you might want to define a CJK environment for text fragments used in the midst of a document using a normal Roman font. This allows me to say \verb|\begin{CJK}東光\end{CJK}| to generate \begin{CJK}東光\end{CJK}, without putting the whole paragraph into the Far Eastern font. Or I could define a command that takes the CJK text as an argument, so that \verb|\cjk{北京}| produces \cjk{北京}. It's that easy! Such an environment can easily be set using the \cmd{newfamily} or \cmd{\fontspec}.

\begin{verbatim}
\newenvironment{CJK}{\fontspec[Scale=0.9]{SimSun}}{}
\newcommand{\cjk}[1]{{\fontspec[Scale=0.9]{SimSun}#1}}
\end{verbatim}
\end{comment}


\normalfont

\section{Changing the font size in LaTeX}

\index{fonts>sizing commands}

Changing the font size in LaTeX can be done at two levels, 
affecting the whole document or elements with in it. 
Using a different font size on a
global level will affect all normal-sized text as well
as the sizes of headings, footnotes, etc. By changing
the font size locally, however, a single word, a few
lines of text, a large table, or a heading throughout
the document may be modified. Fortunately, there is
no need for the writer to juggle with numbers when
doing so. \latex provides a set of macros for changing
the font size locally, taking into consideration the
document’s global font size. \citep{thurnherr2012}

\subsection{Changing the font size on the document-wide level}

The standard classes article, report and book support
three different font sizes: 10pt, 11pt, 12pt. By
default, the font size is set to 10pt and can be modified
by passing any of the previously-mentioned
value as a class option. As an example, suppose you
want to change the font size for normal text to 12pt
throughout the document. For the class report, this
is how you would do that:

\begin{quote}
|\documentclass[12pt]{report}|
\end{quote}

In most cases, the available font sizes for the
standard classes are sufficient and you do not have to
bother about loading special packages that provide
more options.

\section{Changing the font size locally}

A common scenario is that the author of a document
needs to change the font size for a word or paragraph,
decrease the font size of a large table to make it fit on
a page or increase the size of a heading throughout
the document. LaTeX implements a set of macros
which allow changing font size from Huge to tiny,
literally. That way, the author does not have to
worry about numbers. The macros, including the
exact font size in points, are summarized in table 1.
A good rule of thumb is not to use too many
different sizes and not to make things too small or
too big.

\latexe provides two different ways to use these
font size modifier macros: inline or as an environment
using |\begin...\end|:

\begin{texexample}{Changing the size locally}{ex:size}
{\Large This is some large text.\par}
\begin{footnotesize}
This is some footnote-sized text.
\end{footnotesize}
\end{texexample}

The |\par| command at the end of the inline
example adjusts baselineskip, the minimum space
between the bottom of two successive lines



A number of packages exist \ctan{moresize} \citep{moresize},
 \ctan{anyfont} by \citeauthor{anyfont}. The standard classes, \docClass{memoir}
 \docClass{KOMA} classes and most journals also provide their own
 defined sizing commands.
 
%\fontsizetest\textfontten{10/12pt}

\begin{trivlist}\item[]
\begin{tabular}{llll}
\toprule
Class option &10pt &11pt &12pt\\
\midrule
|\Huge|      &25pt &25pt &25pt\\
|\huge|      &20pt &20pt &25pt\\
|\LARGE|     &17pt &17pt &20pt\\
|\Large|     &14pt &14pt &17pt\\
|\large|     &12pt &12pt &14pt\\
|\normalsize| (default) &10pt &11pt &12pt\\
|\small|     &9pt &10pt &11pt\\
|\footnotesize| &8pt &9pt &10pt\\
|\scriptsize| &7pt &8pt &8pt\\
|\tiny|       &5pt &6pt &6pt\\
\bottomrule
\end{tabular}
\end{trivlist}




\let \handlegroupnormalbefore \relax
\let \handlegroupnormalafter  \relax

\protected \def \handlegroupnormal #1#2{%
  \bgroup
  \def \handlegroupbefore {#1}%
  \def \handlegroupafter  {#2}%
  \afterassignment \handlegroupnormalbefore
  \let \next =%
}

\def \handlegroupnormalbefore {%
  \bgroup 
  \handlegroupbefore
  \bgroup 
  \aftergroup \handlegroupnormalafter%
}

\def \handlegroupnormalafter {%
  \handlegroupafter
  \egroup 
  \egroup 
}

\let \groupedcommand \handlegroupnormal 

\def \definehighlight [#1][#2]{%
  \ifcsname #1\endcsname\else
    \expandafter\def\csname #1\endcsname{%
      \leavevmode
      \groupedcommand {#2}\empty%
    }
  \fi%
}



\def\restoreunderscore{\catcode`\_=12\relax}

\definehighlight     [fileent][\ttfamily\restoreunderscore]         %% files, dirs
\definehighlight     [texmacro][\sffamily\itshape\textbackslash]     %% cs
\definehighlight     [luafunction][\sffamily\itshape\restoreunderscore] %% lua identifiers
\definehighlight     [identifier][\ttfamily]                           %% names
\definehighlight     [abbrev][\rmfamily\scshape]                   %% acronyms
\definehighlight     [emphasis][\rmfamily\slshape]                   %% level 1 emph

\definehighlight     [Largefont][\Large]                              %% font size
\definehighlight     [smallcaps][\scfamily]                                 %% font feature
\definehighlight     [nonproportional][\ttfamily]                                 %% font switch
%\let \inlinecode \lstinline
\protected \def \inlinecode {\lstinline}

\chapter{LuaTeX and Fonts}
\label{c:luatexfonts}


\section{Introduction}

Font management and installation has always been painful with \tex.  A
lot of files are needed for one font ({tfm},{pfb},
{map}, {fd}, {vf}), and due to the 8-Bit encoding
each font is limited to 256 characters.

But the font world has evolved since the original \tex, and new
typographic systems have appeared, most notably the so called
\emph{smart font} technologies like \OpenType fonts ({otf}).

These fonts can contain many more characters than \tex fonts, as well
as additional functionality like ligatures, old-style numbers, small
capitals, etc., and support more complex writing systems like Arabic
and Indic. Unfortunately, \pkg{luaotfload} doesn‘t support many Indic
  scripts right now.
.

With \LuaTeX/\LuaLaTeX a completely different mechanism to that of \XeLaTeX or \pdfLaTeX is used to load fonts.

In \identifier{luaotfload}, the canonical syntax for font requests
requires a \emphasis{prefix}:
%
\begin{quote}
  \nonproportional{\string\font\string\fontname\space= }%
  \meta{prefix}%
  \nonproportional{:}%
  \meta{fontname}%
  \dots
\end{quote}
%
where \meta{prefix} is either \inlinecode{file:} or \inlinecode {name:}.
  The development version also knows two further prefixes,
  \inlinecode {kpse:} and \inlinecode {my:}.
  %
  A \inlinecode {kpse} lookup is restricted to files that can be found by
  \identifier{kpathsea} and
  will not attempt to locate system fonts.
  %
  This behavior can be of value when an extra degree of encapsulation is
  needed, for instance when supplying a customized tex distribution.

  The \inlinecode {my} lookup takes this a step further: it lets you define
  a custom resolver function and hook it into the 

   \verb+\luafunction{resolve_font}+

  callback.
  %
  This ensures full control over how a file is located \footnote{For a working example see the
  \protect\url {https://bitbucket.org/phg/lua-la-tex-tests/src/5f6a535d/pln-lookup-callback-1.tex}.}
%
It determines whether the font loader should interpret the request as
a \emphasis{file name} or
 \emphasis{font name}, respectively,
which again influences how it will attempt to locate the font.
%
Examples for font names are
\begin{verbatim}
            “Latin Modern Italic”,
            “GFS Bodoni Rg”, and
            “PT Serif Caption”
\end{verbatim}
         
-- they are the human readable identifiers
usually listed in drop-down menus and the like.\footnote{%
  Font names may appear like a great choice at first because they
  offer seemingly more intuitive identifiers in comparison to arguably
  cryptic file names:
  %
  “PT Sans Bold” is a lot more descriptive than \fileent{PTS75F.ttf}.
  On the other hand, font names are quite arbitrary and there is no
  universal method to determine their meaning.
  %
  While \identifier{luaotfload} provides fairly sophisticated heuristic
  to figure out a matching font style, weight, and optical size, it
  cannot be relied upon to work satisfactorily for all font files.
  %
  For an in-depth analysis of the situation and how broken font names
  are, please refer to
  \hyperlink [this post]{http://www.ntg.nl/pipermail/ntg-context/2013/073889.html}
  by Hans Hagen, the author of the font loader.
  %
  If in doubt, use filenames.
  %
  \fileent{luaotfload-tool} can perform the matching for you with the
  option \inlinecode {--find=<name>}, and you can use the file name it returns
  in your font definition.}

%

\subsection {Compatibility Layer}

In addition to the regular prefixed requests, \identifier{luaotfload}
accepts loading fonts the \XETEX way.
%
There are again two modes: bracketed and unbracketed.
A bracketed request looks as follows.

\begin{quote}
  \nonproportional{\string\font\string\fontname\space = [}%
  \meta{/path/to/file}%
  \nonproportional{]}
\end{quote}

\noindent
Inside the square brackets, every character except for a closing
bracket is permitted, allowing for specifying paths to a font file.
%
Naturally, path-less file names are equally valid and processed the
same way as an ordinary \inlinecode {file:} lookup.

\begin{quote}
  \nonproportional{\string\font\string\fontname\space= }%
  \meta{font name}
  \ldots
\end{quote}

Unbracketed (or, for lack of a better word: \emphasis{anonymous})
font requests resemble the conventional \TEX syntax.
%
However, they have a broader spectrum of possible interpretations:
before anything else, \identifier{luaotfload} attempts to load a
traditional \TEX Font Metric (\abbrev{tfm} or \abbrev{ofm}).
%
If this fails, it performs a \inlinecode {name:} lookup, which itself will
fall back to a \inlinecode {file:} lookup if no database entry matches
\meta{font name}.

Furthermore, \identifier{luaotfload} supports the slashed (shorthand)
font style notation from \XETEX.

\begin{quote}
  \nonproportional{\string\font\string\fontname\space= }%
  \meta{font name}%
  \nonproportional{/}%
  \meta{modifier}
  \dots
\end{quote}





\texttt{OpenType} fonts are widely deployed and available for all modern
operating systems.

As of 2014 they have become the de facto standard for advanced text
layout.

However, until recently the only way to use them directly in the \tex
world was with the \XeTeX engine.

Unlike \XeTeX, \LUATEX has no built-in support for \OpenType or
technologies other than the original \tex fonts.

Instead, it provides hooks for executing LUA code during the TEX run
that allow implementing extensions for loading fonts and manipulating
how input text is processed without modifying the underlying engine.

This is where \pkg{luaotfload} comes into play:
Based on code from \CONTEXT, it extends \LUATEX with functionality necessary
for handling \OpenType fonts.

Additionally, it provides means for accessing fonts known to the operating system conveniently by indexing the metadata. The |luaotfload| package is an adaptation of the \CONTEXT font loading system, allowing for loading \OpenType fonts with an extended syntax and adds support for a variety of font features. With current developments of moving \xetex to \luatex it is the expected way of \latex to evolve.

\section{Loading Fonts}

\docAuxCommand{luaotfload} supports an extended syntax for font loading, similar to that used by \xetex.



\begin{docCommand}{font}{=\meta{prefix}:\meta{font name}:\meta{font features}\meta{\tex features}}
The curly brackets are optional and escape the spaces in the enclosed
font name. Alternatively, double quotes serve the same purpose.\footnote{A surprising feature is with LuaTeX capitalization is not significant, the font name can be typed in lower or upper case and will still find the file. See \protect\url{http://tex.stackexchange.com/questions/223236/why-is-setmainfont-case-sensitive-with-xelatex-but-not-with-lualatex}.}
\end{docCommand}

\section{Below the Hood}

Most \tex users like to have full control and are curious to understand how things work. Using |luaotfload| we can have a look at the |otf| tables, and although a bit error prone can 
see how the fonts store information. The first example we will examine is a short programme to print 50 of the glyph names. In Example \ref{ex:symbola}, we use the built-in  \luacmd{fontloader} function. Many font symbolic names have underscores or other problematic characters. We change the underscore using \luacmd{string.gsub} from the |strings| module, which is loaded automatically with LuaTeX.

%\newfontfamily{\symbola}{Symbola.ttf}

\begin{texexample}{Getting the name of Glyphs}{ex:symbola}
\bgroup
\begin{luacode}
tex.print("\\footnotesize")
f=fontloader.open("c:/windows/fonts/Symbola.ttf")
tex.print("\\begin{multicols}{4}")
local i = 175
while (i < 200) do  
local g = f.glyphs[i]
if g then
  local s = string.gsub(g.name, '%_', '\\textunderscore  ')
  tex.print(i,"\\ttfamily "..s .. "  (\\symbola\\char" .. g.unicode ..")\\par" )
end
  i = i + 1
end
fontloader.close(f)
tex.print("\\end{multicols}")
\end{luacode}
\egroup
\end{texexample}

If you are compiling this document in Windows, the above example would probably find your fonts. However, this is not a very good way to search for fonts. The TeX world has settled over many years on two principles when distributing files. The TeX Directory Structure (TDS) which is a directory hierarchy for macros, fonts, and the other implementation-independent TeX system files and the kpathsea utility for locating these files and for running the many scripts that are necessary while typesetting.

\begin{texexample}{Getting the name of Glyphs}{}
\begin{luacode}
local filename = "Symbola.ttf"
local fullname = kpse.find_file(filename, 'truetype fonts') 
if fullname then tex.print(fullname) else tex.print("font ".. filename .. " not found") end
\end{luacode}
\end{texexample}

The various fontloader programs go at great lengths to ensure that all possible directory paths are searched and the main reason, why when using a font for the first time it takes so long to process. With Lua once the
 file is located, we can read it and manipulate the data in any way we want.
 
 
















