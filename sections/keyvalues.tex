\chapter{Key Value Interfaces}

The key value system greatly simplifies the \tex interface for authors. As \cite{joseph2009} wrote this ease of use was not transferred into settting up key-value systems for authors of pre-packaged \tex code. This Chapter and the one that follows that focus specifically on the \pkg{pgfkeys} package provides an overview and describes some of the more difficult areas. The TUGboat article referenced earlier and written by Joseph Wright \textit{et.al} has an excellent introduction to the available packages and some longer examples for comparison. Chapter~\ref{ch:l3keys}
\nameref{ch:l3keys} discusses the |expl3| key-value functions.

\section{keyval}

The \pkgname{keyval} written by David Carlisle is still widely used by package authors to provide the means for users to easily specify numerous optional arguments for macros \cite{keyval}. The main advantages of using keyval are that  (1) the number of optional arguments is no longer limited to 9 and that (2) the arguments are named, and hence there is less chance of confusion about the syntax of a macro.

\section{xkeyval}

A more recent package, \pkgname{xkeyval} provides improvements for programming keys and  also
provides a more advanced interface for the namespacing of keys and families.
Before you start experimenting with the xkeyval package, I suggest that you load the package \pkg{xkview}. This is part of the \ctan{xkeyval}  bundle and can help you to view key value parameters in various ways. The \pkg{xkeyval} package was developed by Hendri Adriaens and Uwe Kern \citep{xkeyval}. This package is an extension of the well-known |keyval| package. The package provides more flexible commands and syntax enhancements as well as newer option processing mechanism.

The main change of the |xkeyval| package is that it provides a means to namespace the keys, which all have the form |\KV@family@keyname|, where the KV is a literal prefix to avoid collisions. They take one argument to handle user input.

The main commands of the package are the same as those of keyval. 

\begin{texexample}{xkeyval }{}
\makeatletter
\define@key{phd}{pi}{\setlength{\parindent}{#1}}
\setkeys{phd}{pi=50pt}
\makeatother
\lorem\par
\setkeys{phd}{pi=0pt}
\lorem\par
\end{texexample}

Defining a default key, i.e., a key that can be used as |indent| or |indent=30pt| will stretch your memory, as it has an optional parameter as its third argument. 

\begin{verbatim}
\define@key{family}{key}[none]{The input is: #1}
\end{verbatim}

\begin{texexample}{xkeyval }{}
\makeatletter

\define@key{phd}{pi}[30pt]{\setlength{\parindent}{#1}}

\setkeys{phd}{pi}

\lorem

or \setkeys{phd}{pi=0pt}

\lorem
\makeatother
\end{texexample}

\section{Ordinary Keys}
\makeatletter
\define@key{phd}{pi}[1em]{\setlength{\parindent}{#1}}
\makeatother


Ordinary keys are keys that have values such as \texttt{animal=elephant} and your macro can be called like \texttt{animals[animal=elephant]\{14\}}.

   

\section{Keys and values in package options}

First of all, the package supplies macros to declare class or package options, execute them and process
them. The macros are available under the usual
\latex names, but all with the suffix \textbf{X}, namely

\begin{docCommand}{DeclareOptionX}{}
\begin{docCommand}{DeclareOptionX*}{}
\begin{docCommand}{ExecuteOptionsX}{}
\begin{docCommand}{ProcessOptionsX}{}
These commands allow the user to assign a value to
an option just like when using |\setkeys|. The first
macro is based on |\define@key| and the final two
are based on |\setkeys|. Supposing that a package
|mypack| is set up with these commands, a user could
for instance do
\end{docCommand}
\end{docCommand}
\end{docCommand}
\end{docCommand}

\begin{verbatim}
\usepackage[textcolor=red,font=times]{mypack}
\end{verbatim}

These |xkeyval|macros are fully compatible with the \latex option conventions. They will allow packages to copy global options specified in the |\documentclass| command, to pass options to other classes or packages and to update the list of unused global options that will be displayed by \latex in the log file. 


\section{kvoptions}

Another package \pkgname{kvoptions} by Heiko Oberdiek is used extensively in the large suite of packages
developed by Heiko \cite{kvoptions}. The package originally formed part of the \pkgname{hyperref} and later branched into
an independent package. The package provides a number of additional commands to those found in the \latexe kernel and a comparison of the commands is shown in the table below. It is a good alternative to single purpose
package writers. An important feature of the package is its ability to process options both globally as well as locally avoiding conflicts when options are specified both globally as well as locally. Heiko provides an example from his bookmark package \cite{bookmark}, which provides the option \option{open}
that specifies whether the bookmarks are opened or closed initially. It’s values are
true or false. Since KOMA-Script version 3.00 the KOMA classes also introduces
option open with values right and any and a complete different meaning.
Such conflicts can be resolved by marking all or part of options as local by
|\DeclareLocalOption| or |\DeclareLocalOptions|. Then the packages ignores
global occurences of these options



\input{./sections/pgfmanual-en-pgfkeys}











