\chapter{ltfloat.dtx}

 \section{Float types}

  The different types of floats are identified by a \meta{type} name,
  which is the name of the counter for that kind of float.  For
  example, figures are of type `figure' and tables are of type `table'.
  Each \meta{type} has associated a positive \meta{type number}, which
  is a power of two e.g.,\\
  figures might be have type number~1, tables type number~2, programs
  type number~4, etc. See \urlhttp://tex.stackexchange.com/questions/39017/how-to-influence-the-position-of-float-environments-like-figure-and-table-in-lat/39020#39020{}

  The locations where a float can go are specified by a
  \meta{placement specifier}, which is a list of the possible
  locations, each denoted by a letter as follows:

    \begin{center}
    \begin{tabular}{l@{ : }l@{ --- }l}
     h & here   & at the current location in the text.\\
     t & top    & at the top of a text page.\\
     b & bottom & at the bottom of a text page.\\
     p & page   & on a separate float page
    \end{tabular}
    \end{center}

  In addition, in conjunction with these, you can use `!' which means
  that the current values of the float positioning parameters are
  ignored for this float. (Has no effect on `p', float page
  positioning.)
  For example, `pht' specifies that the float can appear in any of
  three locations: page, here or top. The order of specifying the placement
  specifiers is irrelevant to the float algorithm.


\subsection{Floating Environments}
    \begin{teX}
\message{floats,}
    \end{teX}



 Where floats may appear on a page, and how many may appear there
 are specified by the following float placement parameters.  The
 numbers are named like counters so the user can set them with
 the ordinary counter-setting commands.

\begin{tabular}{lp{6cm}}

  \cs{c@topnumber}      & Number of floats allowed at the top of a column.\\
  \cs{topfraction}      & Fraction of column that can be devoted to floats.
  \cs{c@dbltopnumber}, \cs{dbltopfraction} \\
                    & Same as above, but for double-column floats.\\
  \cs{c@bottomnumber}, \cs{bottomfraction}\\ 
                    & Same as above for bottom of page.\\
  \cs{c@totalnumber}    & Number of floats allowed in a single column,
                          including in-text floats.\\
  \cs{textfraction}     &Minimum fraction of column that must contain text.\\
  \cs{floatpagefraction}& Minimum fraction of page that must be taken
                          up by float page.\\
  \cs{dblfloatpagefraction} 
                    & Same as above, for double-column floats.\\
\end{tabular}


 The document style must define the following.

\begin{longtable}{lp{6cm}}
    \cs{fps@TYPE}   & The default placement specifier for floats of type
                  TYPE. \\
    \cs{ftype@TYPE} & The type number for floats of type TYPE.\\
    \cs{ext@TYPE}   & The file extension indicating the file on which the
                  contents list for float type TYPE is stored.
                    For example,  \cs{ext@figure = 'lof'}.\\
    \cs{fnum@TYPE}  & A macro to generate the figure number for a caption.
                  For example, \cs{fnum@TYPE} == Figure \cs{thefigure}.\\
    \cs{@makecaption}{NUM}{TEXT} & 
              A macro to make a caption, with NUM the value
              produced by \cs{fnum@}... and TEXT the text of the caption.
              It can assume it's in a \cs{parbox} of the appropriate width.\\
\end{longtable}

\begin{teX}
 \@float{type}[placement] : This macro begins a float environment for a (*@ float @*)
     single-column float of type TYPE with PLACEMENT as the placement
     specifier.  The default value of PLACEMENT is defined by
     \fps@TYPE.  The environment is ended by \end@float.
     E.g., \figure == \@float{figure}, \endfigure == \end@float.

  \@float{TYPE}[PLACEMENT] ==
   BEGIN
     if hmode then \@bsphack
                          \@floatpenalty := -10002
              else      \@floatpenalty := -10003
     fi
     \@captype ==L TYPE
     \@dblflset
     \@fps     ==L PLACEMENT
     \@onelevel@sanitize \@fps 
     add default PLACEMENT if at most ! in PLACEMENT == \@fpsadddefault
     if inner
       then LaTeX Error: 'Not in outer paragraph mode.'
            \@floatpenalty := 0
       else if \@freelist nonempty
              then \@currbox  :=L head of \@freelist
                   \@freelist :=G tail of \@freelist
                   \count\@currbox :=G 32*\ftype@TYPE + 
                                          bits determined by PLACEMENT
              else \@floatpenalty := 0
                   LaTeX Error: 'Too many unprocessed floats'
            fi
     fi
     \@currbox :=G   \color@vbox
                       \normalcolor
                         \vbox{
                          %% 15 Dec 87 --
                          %% removed \boxmaxdepth :=L 0pt
                          %% that made box 0 depth because it screwed
                          %% things up. Instead, added \vskip0pt at end
                               \hsize = \columnwidth
                               \@parboxrestore
                               \@floatboxreset
   END

  \caption ==
    BEGIN
     \refstepcounter{\@captype}
     \@dblarg{\@caption{\@captype}}
    END

 In following definition, \par moved from after \addcontentsline to
 before \addcontentsline because the \write could cause
 an extra blank line to be added to the paragraph above the
 caption.  (Change made 12 Jun 87)

  \@caption{TYPE}[STEXT]{TEXT} ==
   BEGIN
     \par
     \addcontentsline{\ext@TYPE}{TYPE}{\numberline{\theTYPE}{STEXT}}
     \begingroup
       \@parboxrestore
       \@normalsize
       \@makecaption{\fnum@TYPE}{TEXT}
       \par
     \endgroup
   END


  \@dblfloat{TYPE}[PLACEMENT] : Macro to begin a float environment for
     a double-column float of type TYPE with PLACEMENT as the placement
     specifier.  The default value of PLACEMENT is 'tp'
     The environment is ended by \end@dblfloat.
     E.g., \figure* == \@dblfloat{figure}, 
           \endfigure* == \end@dblfloat.

  \@dblfloat{TYPE}[PLACEMENT] ==
     Identical to \@float{TYPE}[PLACEMENT] except \hsize and \linewidth
     are set to \textwidth.
\end{teX}

 \begin{macro}{\@floatpenalty}  
The float penalty is saved in a counter to enable ease of use.
 \end{macro}
    \begin{teX}
\newcount\@floatpenalty
    \end{teX}
 


\begin{macro}{\caption}

    This is set to be an error message outside a float since no
    captype is defined there; this may need to be changed by some 
    classes. Note if the caption is outside a float it triggers
	an error.

    \begin{teX}
\def\caption{%
   \ifx\@captype\@undefined
     \@latex@error{\noexpand\caption outside float}\@ehd
     \expandafter\@gobble
   \else
     \refstepcounter\@captype
     \expandafter\@firstofone
   \fi
   {\@dblarg{\@caption\@captype}}%
} 
    \end{teX}
 \end{macro}

 \begin{macro}{\@caption}
    \begin{teX}
\long\def\@caption#1[#2]#3{%
  \par
  \addcontentsline{\csname ext@#1\endcsname}{#1}%
    {\protect\numberline{\csname the#1\endcsname}{\ignorespaces #2}}%
  \begingroup
    \end{teX}

 The paragraph setting parameters are normalised at this point, however
 |\@parboxrestore| resets |\everypar| which is not correct in this
 context so |\@setminipage| is called if needed.

 The float mechanism, like minipage, sets the flag |@minipage| true
 before executing the user-supplied text. Many \LaTeX\ constructs
 test for this flag and do not add vertical space when it is true.
 The intention is that this emulates \TeX's `top of page' behaviour.
 The flag must be set false at the start of the first paragraph. This
 is achieved by a redefinition of |\everypar|, but the call to
 |\@parboxrestore| removes that redefinition, so it is re-inserted 
 if needed. If the flag is already false then the |\caption| was not
 the first entry in the float, and so some other paragraph has already
 activated the special |\everypar|. In this case no further action is
 needed.
    \begin{teX}
    \@parboxrestore
    \if@minipage
      \@setminipage
    \fi
    \end{teX}

    \begin{teX}
    \normalsize
    \@makecaption{\csname fnum@#1\endcsname}{\ignorespaces #3}\par
  \endgroup}
    \end{teX}
 \end{macro}

 \begin{macro}{\@float} Just a reminder that this is used to define new floating
 environments, such as \textit{figure} and \textit{table}.  The |fps@#1| has been
 defined in the standard classes and defines the default float specifier. In |book.cls|
is defined as |\def\fps@figure{tbp}|.

 \begin{macro}{\@dblflset}
    \begin{teX}
\def\@float#1{%
  \@ifnextchar[%
    {\@xfloat{#1}}%
    {\edef\reserved@a{\noexpand\@xfloat{#1}[\csname fps@#1\endcsname]}%
     \reserved@a}}
    \end{teX}
    
 \end{macro}
 \end{macro}

  \begin{macro}{\@dblfloat}

    \begin{teX}
\def\@dblfloat{%
  \if@twocolumn\let\reserved@a\@dbflt\else\let\reserved@a\@float\fi
  \reserved@a}
    \end{teX}
  \end{macro}

    
  \begin{macro}{\fps@dbl}
  Note that all double floats have default fps `tp'.
  \end{macro}
  
  \begin{macro}{\@setfps}
    This sets the fps, dealing with error conditions by adding
    the default.
  \end{macro}

  \begin{macro}{\@xfloat}

   The first part of this sets the count register that stores all
   the information about the type and fps of the float.

    We assume here that the default specifiers already contain no
    active characters.

    It may be better to store the defaults as numbers, rather than
    symbol strings.

   The |\@nodocument| is defined in the ltxerror and is the error produced if paragraphs are typeset in the preamble. So here LaTeX is checking if a float was specified in a preamble. Why here, is because
we cannot rely on user defined environments to carry out these tests.

What does the ! do? This goes to default, so it is interesting to find out why the bang was specified
in the first place. It just starts overrides all the constraints.

    \begin{teX}
\def\@xfloat #1[#2]{%
  \@nodocument 
  \def \@captype {#1}%
   \def \@fps {#2}%
   \@onelevel@sanitize \@fps 
   \def \reserved@b {!}%
   \ifx \reserved@b \@fps
     \@fpsadddefault
   \else
     \ifx \@fps \@empty
       \@fpsadddefault
     \fi
   \fi
   \ifhmode
     \@bsphack
     \@floatpenalty -\@Mii
   \else
     \@floatpenalty-\@Miii
   \fi
  \ifinner
     \@parmoderr\@floatpenalty\z@
  \else
    \@next\@currbox\@freelist
      {%
       \@tempcnta \sixt@@n
       \expandafter \@tfor \expandafter \reserved@a
         \expandafter :\expandafter =\@fps 
         \do
          {%
           \if \reserved@a h%
             \ifodd \@tempcnta
             \else
               \advance \@tempcnta \@ne
             \fi
           \fi
           \if \reserved@a t%
             \@setfpsbit \tw@
           \fi
           \if \reserved@a b%
             \@setfpsbit 4%
           \fi
           \if \reserved@a p%
             \@setfpsbit 8%
           \fi
           \if \reserved@a !%
             \ifnum \@tempcnta>15
               \advance\@tempcnta -\sixt@@n\relax
             \fi
           \fi
           }%
       \@tempcntb \csname ftype@\@captype \endcsname
       \multiply \@tempcntb \@xxxii
       \advance \@tempcnta \@tempcntb
       \global \count\@currbox \@tempcnta
       }%
    \@fltovf  %This is for too many floats error for marginpars
  \fi
    \end{teX}
    The remainder sets up the box in which the float is typeset, and
    the typesetting environment to be used.  It is essential to have
    the extra box to avoid the unwanted space that would otherwise
    often be put at the top of the float.

    It ends with a hook; not sure how useful this is but it is needed
    at present to deal with double-column floats.
    \begin{teX}
  \global \setbox\@currbox
    \color@vbox
      \normalcolor
      \vbox \bgroup
        \hsize\columnwidth
        \@parboxrestore
        \@floatboxreset
}
    \end{teX}
  \end{macro}
  
  \begin{macro}{\@floatboxreset}
    
 The rational for allowing these normally global flags to be set
 locally here, via |\@parboxrestore|, was stated originally by
 Donald Arseneau and extended by Chris Rowley.
 It is because these flags are only set globally to
 true by section commands, and these should never appear within
 marginals or floats or, indeed, in any group; and they are only ever
 set globally to false when they are definitely true.

 If anyone is unhappy with this argument then both flags should be
 treated as in |\set@nobreak|; otherwise this command will be
 redundant. 
     {Added local settings of flags: dangerous!!}
    \begin{teX}
\def \@floatboxreset {%
        \reset@font
        \normalsize
        \@setminipage
}
    \end{teX}
  \end{macro}
  
  \begin{macro}{\@setnobreak}
    \begin{teX}
\def \@setnobreak{%
  \if@nobreak
    \let\outer@nobreak\@nobreaktrue
    \@nobreakfalse
  \fi
}
    \end{teX}
  \end{macro}

  \begin{macro}{\@setminipage}
    \begin{teX}
\def \@setminipage{%
  \@minipagetrue
  \everypar{\@minipagefalse\everypar{}}%
}
    \end{teX}
  \end{macro}

 \begin{macro}{\end@float}
    \begin{teX}
\def\end@float{%
  \@endfloatbox
  \ifnum\@floatpenalty <\z@
    \end{teX}
 We make sure that we never exceed |\textheight|, otherwise float
 will never get typeset (91/03/15 FMi).
    \begin{teX}
    \@largefloatcheck
    \@cons\@currlist\@currbox
    \ifnum\@floatpenalty <-\@Mii
      \penalty -\@Miv
    \end{teX}
 Saving and restoring |\prevdepth| added 26 May 87 to prevent extra
 vertical space when used in vertical mode.
    \begin{teX}
      \@tempdima\prevdepth
      \vbox{}%
      \prevdepth\@tempdima
    \end{teX}

    \begin{teX}
      \penalty\@floatpenalty
    \end{teX}
 
    \begin{teX}
    \else
      \vadjust{\penalty -\@Miv \vbox{}\penalty\@floatpenalty}\@Esphack
    \fi
  \fi
}
    \end{teX}
 \end{macro}

 \begin{macro}{\end@dblfloat}
    \begin{teX}
\def\end@dblfloat{%
\if@twocolumn
  \@endfloatbox
  \ifnum\@floatpenalty <\z@
    \end{teX}
 We make sure that we never exceed |\textheight|, otherwise float
 will never get typeset (91/03/15 FMi).
    \begin{teX}
    \@largefloatcheck
    \@cons\@dbldeferlist\@currbox
  \fi
    \end{teX}
 RmS 92/03/18 changed |\@esphack| to |\@Esphack|.
    \begin{teX}
    \ifnum \@floatpenalty =-\@Mii \@Esphack\fi
\else
  \end@float
\fi
}
    \end{teX}
 \end{macro}
 
 \begin{macro}{\@endfloatbox}
    This macro is not intended to be a hook; it is designed to help
    maintain the integrity of this code, which is used twice and, as
    can be seen, is subject to frequent changes.
    \begin{teX}
\def \@endfloatbox{%
      \par\vskip\z@skip      %% \par\vskip\z@ added 15 Dec 87
    \end{teX}
   
    \begin{teX}
      \@minipagefalse   
      \outer@nobreak
    \egroup                  %% end of vbox
  \color@endbox
}
% 
 \begin{macro}{\outer@nobreak}
    \begin{teX}
\let\outer@nobreak\@empty
    \end{teX}
  \end{macro}
 

  \begin{macro}{\@largefloatcheck}
 
    This calculates by how much a float is oversize for the page and
    prints this in a warning message.
    
    \begin{teX}  
\def \@largefloatcheck{%
  \ifdim \ht\@currbox>\textheight
    \@tempdima -\textheight
    \advance \@tempdima \ht\@currbox
    \end{teX}

    \begin{teX}
    \@latex@warning {Float too large for page by \the\@tempdima}%
    \ht\@currbox \textheight
  \fi
}
    \end{teX}
  \end{macro}

  \begin{macro}{\@dbflt}
  \begin{macro}{\@xdblfloat}
    \begin{teX}
\def\@dbflt#1{\@ifnextchar[{\@xdblfloat{#1}}{\@xdblfloat{#1}[tp]}}
\def\@xdblfloat#1[#2]{%
  \@xfloat{#1}[#2]\hsize\textwidth\linewidth\textwidth}
    \end{teX}
  \end{macro}
  \end{macro}

    \begin{teX}
\newcount\c@topnumber
\newcount\c@dbltopnumber
\newcount\c@bottomnumber
\newcount\c@totalnumber
    \end{teX}

 An analysis of |\@floatplacement|:

 This should be called whenever |\@colht| has been set.
    \begin{teX}
\def\@floatplacement{\global\@topnum\c@topnumber
    % Textpage bit, global:
   \global\@toproom \topfraction\@colht
   \global\@botnum  \c@bottomnumber
   \global\@botroom \bottomfraction\@colht
   \global\@colnum  \c@totalnumber
    % Floatpage bit, local:
   \@fpmin   \floatpagefraction\@colht}
    \end{teX}

  \begin{macro}{\@dblfloatplacement}
 
     This should be called only within a group.  Now changed to
     provide extra checks in |\@addtodblcol|, needed when processing a
     BANG float.
    
    \begin{teX}  
\def \@dblfloatplacement {%
    \end{teX}
    Textpage bit: global, but need not be.
    \begin{teX}  
  \global \@dbltopnum \c@dbltopnumber
  \global \@dbltoproom \dbltopfraction\@colht
    \end{teX}
   This new bit uses |\@textmin| to locally store the amount of extra
   room in the column.   
    \begin{teX}
  \@textmin \@colht
  \advance \@textmin -\@dbltoproom
    \end{teX}
    Floatpage bit: must be local.
    \begin{teX}
  \@fpmin \dblfloatpagefraction\textheight
  \@fptop \@dblfptop
  \@fpsep \@dblfpsep
  \@fpbot \@dblfpbot
}
    \end{teX}
  \end{macro}

\section{Marginal Notes}

   Marginal notes use the same mechanism as floats to communicate
   with the \cs{output} routine.  Marginal notes are distinguished from
   floats by having a negative placement specification.  The command
   \cs{marginpar}[LTEXT]{RTEXT} generates a marginal note in a parbox,
   using LTEXT if it's on the left and RTEXT if it's on the right.
   (Default is RTEXT = LTEXT.)  It uses the following parameters.

% \begin{oldcomments}
   \marginparwidth : Width of marginal notes.
   \marginparsep   : Distance between marginal note and text.
        the page layout to determine how to move the marginal
        note into the margin.   E.g., \@leftmarginskip ==
        \hskip -\marginparwidth \hskip -\marginparsep .
   \marginparpush  :  Minimum vertical separation between \marginpar's

  Marginal notes are normally put on the outside of the page
  if @mparswitch = true, and on the right if @mparswitch = false.
  The command \reversemarginpar reverses the side where they
  are put.  \normalmarginpar undoes \reversemarginpar.
  These commands have no effect for two-column output.

  SURPRISE: if two marginal notes appear on the same line of
  text, then the second one could appear on the next page, in
  a funny position. (I was unable to reproduce the error).


  \marginpar [LTEXT]{RTEXT} ==
   BEGIN
     if hmode then \@bsphack
                   \@floatpenalty := -10002
              else \@floatpenalty := -10003
     fi
     if inner
       then LaTeX Error: 'Not in outer paragraph mode.'
            \@floatpenalty := 0
       else if \@freelist has two elements:
              then get \@marbox, \@currbox  from \@freelist
                   \count\@marbox :=G -1
              else \@floatpenalty := 0
                   LaTeX Error: 'Too many unprocessed floats'
                   \@currbox, \@marbox := \@tempboxa    %%use \def
            fi
     fi
     if optional argument
       then %% \@xmpar ==
            \@savemarbox\@marbox{LTEXT}
            \@savemarbox\@currbox{RTEXT}
       else %% \@ympar ==
            \@savemarbox\@marbox{RTEXT}
            \box\@currbox :=G \box\@marbox
    fi
    \@xympar 
   END

 \reversemarginpar == BEGIN \@mparbottom   :=G 0
                            @reversemargin :=G true
                      END

 \normalmarginpar  == BEGIN \@mparbottom   :=G 0
                            @reversemargin :=G false
                      END

  \end{oldcomments}
%
 \begin{macro}{\marginpar}
    \begin{teX}
\def\marginpar{%
  \ifhmode
    \@bsphack
    \@floatpenalty -\@Mii
  \else
    \@floatpenalty-\@Miii
  \fi
  \ifinner
    \@parmoderr
    \@floatpenalty\z@
  \else
    \@next\@currbox\@freelist{}{}%
    \@next\@marbox\@freelist{\global\count\@marbox\m@ne}%
       {\@floatpenalty\z@
        \@fltovf\def\@currbox{\@tempboxa}\def\@marbox{\@tempboxa}}%
  \fi
  \@ifnextchar [\@xmpar\@ympar}
    \end{teX}
 \end{macro}
%
% \begin{macro}{\@xmpar}
%    \begin{teX}
\long\def\@xmpar[#1]#2{%
  \@savemarbox\@marbox{#1}%
  \@savemarbox\@currbox{#2}%
  \@xympar}
%    \end{teX}
% \end{macro}

This is the main macro for a marginpar command that does not have left or
right text. Note it call \cs{@xympar}
 \begin{macro}{\@ympar}
    \begin{teX}
\long\def\@ympar#1{%
  \@savemarbox\@marbox{#1}%
  \global\setbox\@currbox\copy\@marbox
  \@xympar}
    \end{teX}
 \end{macro}
 
 \begin{macro}{\@savemarbox}
 sets up the vboxes including a color@vbox to correctly handle colour. 
    \begin{teX}
\long\def \@savemarbox #1#2{%
  \global\setbox #1%
    \color@vbox
      \vtop{%
        \hsize\marginparwidth
        \@parboxrestore 
        \@marginparreset
        #2%
        \@minipagefalse   
        \outer@nobreak
        }%
    \color@endbox
}
    \end{teX}
  \end{macro}
 
%  \begin{macro}{\@marginparreset}
% \changes{v1.1f}{1994/11/21}{Macro added}
%
% The rational for allowing these normally global flags to be set
% locally here, via |\@parboxrestore| was stated originally by
% Donald Arsenau and extended by Chris Rowley.
% It is because these flags are only set globally to
% true by section commands, and these should never appear within
% marginals or floats or, indeed, in any group; and they are only ever
% set globally to false when they are definitely true.
%
% If anyone is unhappy with this argument then both flags should be
% treated as in |\set@nobreak|; otherwise this command will be
% redundant. 
% \changes{v1.1p}{1996/10/24}
%     {Added local settings of flags: dangerous!!}
%    \begin{teX}
\def \@marginparreset {%
        \reset@font
        \normalsize
%        \let\if@nobreak\iffalse
%        \let\if@noskipsec\iffalse
%        \@setnobreak
        \@setminipage
}
%    \end{teX}
%  \end{macro}
%
% \begin{macro}{\@xympar}
%
%    \begin{teX}
\def \@xympar{%
  \ifnum\@floatpenalty <\z@\@cons\@currlist\@marbox\fi
  \setbox\@tempboxa
    \color@vbox
      \vbox \bgroup
  \end@float
  \@ignorefalse
  \@esphack
}
%    \end{teX}
% \end{macro}
%
 \begin{macro}{\reversemarginpar}
 \begin{macro}{\normalmarginpar}
    \begin{teX}
\def\reversemarginpar{\global\@mparbottom\z@ \@reversemargintrue}
\def\normalmarginpar{\global\@mparbottom\z@ \@reversemarginfalse}
    \end{teX}
 \end{macro}
 \end{macro}

    \begin{teX}
\message{footnotes,}
    \end{teX}

 \section{Footnotes}

We start with a summary of all user commands.

\begin{tabular}{lp{6cm}}
   \cs{footnote}\marg{text} &User command to insert a footnote.\\
   \cs{footnote}\oarg[NUM]\marg{NOTE} &User command to insert a footnote numbered, NUM, where NUM is a number -- 1, 2,
                       etc.  For example, if footnotes are numbered
                       *, **, etc. within pages, then \cs{footnote}|[2]{...}|
                       produces footnote '**'.  This command does not
                       step the footnote counter.
\end{tabular}

\begin{oldcomments}
%   \footnotemark[NUM] : Command to produce just the footnote mark in
%                        the text, but no footnote.  With no argument,
%                        it steps the footnote counter before generating
%                        the mark.
%
%   \footnotetext[NUM]{TEXT} : Command to produce the footnote but
%                              no mark.  \footnote is equivalent to
%                              \footnotemark \footnotetext .
%

%   As in PLAIN, footnotes use \insert\footins, and the following
%   parameters: 
%
%   \footnotesize   : Size-changing command for footnotes.
%
%   \footnotesep    : The height of a strut placed at the beginning of
%                     every footnote.
%   \skip\footins   : Space between main text and footnotes.  The rule
%                     separating footnotes from text occurs in this
%                     space. This space lies above the strut of height
%                     \footnotesep which is at the beginning of the
%                     first footnote.
%   \footnoterule   : Macro to draw the rule separating footnotes from
%                     text. It is executed right after a \vspace of
%                     \skip\footins. It should take zero vertical
%                     space--i.e., it should to a negative skip to
%                     compensate for any positive space it occupies.
%                     (See PLAIN.TEX.)
%
%   \interfootnotelinepenalty : Interline penalty for footnotes.
%
%   \thefootnote : In usual LaTeX style, produces the footnote number.
%                  If footnotes are to be numbered within pages, then
%                  the document style file must include an \@addtoreset
%                  command to cause the footnote counter to be reset
%                  when the page counter is stepped.  This is not a good
%                  idea, though, because the counter will not always be
%                  reset in time to ensure that the first footnote on a
%                  page is footnote number one.
%
%   \@thefnmark : Holds the current footnote's mark--e.g., \dag or '1'
%                 or 'a'. 
%
%   \@mpfnnumber  : A macro that generates the numbers for \footnote
%                  and \footnotemark commands. It == \thefootnote
%                  outside a minipage environment, but can be
%                  changed inside to generate numbers for
%                  \footnote's.
%
%   \@makefnmark : A macro to generate the footnote marker from
%                 \@thefnmark The default definition was
%                 \hbox{$^\@thefnmark$}.
%
%                 This is now replaced by
%                 \textsuperscript{\@thefnmark}
%
%   \@makefntext{NOTE} :
%        Must produce the actual footnote, using \@thefnmark as the mark
%        of the footnote and NOTE as the text.  It is called when
%        effectively  inside a \parbox, with \hsize = \columnwidth.
%          For example, it might be as simple as
%               $^{\@thefnmark}$  NOTE
%
% In a minipage environment, \footnote and \footnotetext are redefined
% so that
%    (a) they use the counter mpfootnote
%    (b) the footnotes they produce go at the bottom of the minipage.
% The switch is accomplished by letting \@mpfn == footnote or mpfootnote
% and \thempfn == \thefootnote or \thempfootnote, and by redefining
% \@footnotetext to be \@mpfootnotetext in the minipage.
%
% \footnote{NOTE}  ==
%  BEGIN
%    \stepcounter{\@mpfn}
%    begingroup
%       \protect == \noexpand
%       \@thefnmark :=G eval (\thempfn)
%    endgroup
%    \@footnotemark
%    \@footnotetext{NOTE}
%  END
%
% \footnote[NUM]{NOTE} ==
%  BEGIN
%    begingroup
%       \protect == \noexpand
%       counter \@mpfn :=L NUM
%       \@thefnmark :=G eval (\thempfn)
%    endgroup
%    \@footnotemark
%    \@footnotetext{NOTE}
%  END
%
% \footnotemark      ==
%  BEGIN \stepcounter{footnote}
%        begingroup
%           \protect == \noexpand
%           \@thefnmark :=G eval(\thefootnote)
%        endgroup
%        \@footnotemark
%  END
%
% \footnotemark[NUM] ==
%   BEGIN
%       begingroup
%         footnote counter :=L NUM
%         \protect == \noexpand
%        \@thefnmark :=G eval(\thefootnote)
%       endgroup
%       \@footnotemark
%   END
%
% \@footnotemark ==
%   BEGIN
%    \leavevmode
%    IF hmode THEN \@x@sf := \the\spacefactor FI
%    \@makefnmark          % put number in main text
%    IF hmode THEN \spacefactor := \@x@sf FI
%   END
%
% \footnotetext      ==
%    BEGIN begingroup \protect == \noexpand
%                     \@thefnmark :=G eval (\thempfn)
%          endgroup
%          \@footnotetext
%    END
%
% \footnotetext[NUM] ==
%    BEGIN begingroup  counter \@mpfn :=L NUM
%                      \protect == \noexpand
%                      \@thefnmark :=G eval (\thempfn)
%          endgroup
%          \@footnotetext
%    END
%
% \end{oldcomments}

\begin{algorithm}
\cs{footnotetext}[NUM] ==\\
\Begin{
 begingroup\\
  counter \cs{@mpfn} := L NUM\\
  \cs{protect} == \cs{noexpand}\\
  \cs{@thefnmark} :=G eval(\cs{thempfn})\\
 endgroup\\
 \cs{@footnotetext}\\
}
\end{algorithm}


 \begin{macro}{\footins}
 \LaTeX\ does use the same insert for footnotes as PLAIN.
    \begin{teX}
\newinsert\footins
    \end{teX}

 \LaTeX\ leaves these initializations for the |\footins| insert.

    \begin{teX}
\skip\footins=\bigskipamount % space added when footnote is present
\count\footins=1000 % footnote magnification factor (1 to 1)
\dimen\footins=8in % maximum footnotes per page
    \end{teX}
 \end{macro}


 \begin{macro}{\footnoterule}
 \LaTeX\ keeps PLAIN \TeX's |\footnoterule| as the default.

    \begin{teX}
\def\footnoterule{\kern-3\p@
  \hrule \@width 2in \kern 2.6\p@} % the \hrule is .4pt high
    \end{teX}
 \end{macro}

 \begin{macro}{\thefootnote}
    \begin{teX}
\@definecounter{footnote}
\def\thefootnote{\@arabic\c@footnote}
    \end{teX}
 \end{macro}

 \begin{macro}{\thempfootnote}
    The default display for the footnote counter in minipages is to
    use italic letters. We use |\itshape| not |\textit| as the latter
    would add an italic correction.
    \begin{teX}
\@definecounter{mpfootnote}
\def\thempfootnote{{\itshape\@alph\c@mpfootnote}}
    \end{teX}
 \end{macro}

 \begin{macro}{\@makefnmark}
    \begin{teX}
\def\@makefnmark{\hbox{$^{\@thefnmark}\m@th$}}
\def\@makefnmark{\hbox{\@textsuperscript{\normalfont\@thefnmark}}}
    \end{teX}
 \end{macro}

  \begin{macro}{\textsuperscript}
    This command provides superscript characters in the current text
    font. It's implementation might change!!!
    \begin{teX}
\DeclareRobustCommand*\textsuperscript[1]{%
  \@textsuperscript{\selectfont#1}}
    \end{teX}
  \end{macro}

  \begin{macro}{\@textsuperscript}
    This command should not be used directly, but may be used to define
   other commands |\textsuperscript|, |\@makefnmark|. |#1| should
   always start with a font selection command, to activate the font
   size switch.
    \begin{teX}
\def\@textsuperscript#1{%
  {\m@th\ensuremath{^{\mbox{\fontsize\sf@size\z@#1}}}}}
    \end{teX}
  \end{macro}

 \begin{macro}{\footnotesep}
    \begin{teX}
\newdimen\footnotesep
    \end{teX}
 \end{macro}

 \begin{macro}{\footnote}

    \begin{teX}
\def\footnote{\@ifnextchar[\@xfootnote{\stepcounter\@mpfn
     \protected@xdef\@thefnmark{\thempfn}%
     \@footnotemark\@footnotetext}}
    \end{teX}
 \end{macro}

 \begin{macro}{\@xfootnote}
    \begin{teX}
\def\@xfootnote[#1]{%
   \begingroup 
     \csname c@\@mpfn\endcsname #1\relax
     \unrestored@protected@xdef\@thefnmark{\thempfn}%
   \endgroup
   \@footnotemark\@footnotetext}
    \end{teX}
 \end{macro}

 \begin{macro}{\@footnotetext}
    \begin{teX}
\long\def\@footnotetext#1{\insert\footins{%
    \reset@font\footnotesize
    \interlinepenalty\interfootnotelinepenalty
    \splittopskip\footnotesep
    \splitmaxdepth \dp\strutbox \floatingpenalty \@MM
    \hsize\columnwidth \@parboxrestore
    \protected@edef\@currentlabel{%
       \csname p@footnote\endcsname\@thefnmark
    }% 
    \color@begingroup
      \@makefntext{%
        \rule\z@\footnotesep\ignorespaces#1\@finalstrut\strutbox}%
    \color@endgroup}}%
    \end{teX}
 \end{macro}

 \begin{macro}{\footnotemark}

    \cs{footnotemark}.} 
    \begin{teX}
\def\footnotemark{%
   \@ifnextchar[\@xfootnotemark
     {\stepcounter{footnote}%
      \protected@xdef\@thefnmark{\thefootnote}%
      \@footnotemark}}
    \end{teX}
 \end{macro}

 \begin{macro}{\@xfootnotemark}
    \begin{teX}
\def\@xfootnotemark[#1]{%
   \begingroup 
      \c@footnote #1\relax
      \unrestored@protected@xdef\@thefnmark{\thefootnote}%
   \endgroup
   \@footnotemark}
    \end{teX}
 \end{macro}

 \begin{macro}{\@footnotemark}

         {Add \cs{nobreak} to allow hyphenation. latex/1605}
    \begin{teX}
\def\@footnotemark{%
  \leavevmode
  \ifhmode\edef\@x@sf{\the\spacefactor}\nobreak\fi
  \@makefnmark
  \ifhmode\spacefactor\@x@sf\fi
  \relax}
    \end{teX}
 \end{macro}

 \begin{macro}{\footnotetext}
    \begin{teX}
\def\footnotetext{%
     \@ifnextchar [\@xfootnotenext
       {\protected@xdef\@thefnmark{\thempfn}%
    \@footnotetext}}
    \end{teX}
 \end{macro}

 \begin{macro}{\@xfootnotenext}
    \begin{teX}
\def\@xfootnotenext[#1]{%
  \begingroup 
     \csname c@\@mpfn\endcsname #1\relax
     \unrestored@protected@xdef\@thefnmark{\thempfn}%
  \endgroup
  \@footnotetext}
    \end{teX}
 \end{macro}

 \begin{macro}{\thempfn}
 \begin{macro}{\@mpfn}
    \begin{teX}
\def\@mpfn{footnote}
\def\thempfn{\thefootnote}

    \end{teX}
 \end{macro}
 \end{macro}

