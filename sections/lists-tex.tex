\chapter{Lists and PlainTeX}

\section{Introduction}

Lists or seriation as sometimes referred in Style Manuals helps the reader understand the organization of key points within sections, paragraphs,
and sentences. In any series, all items should be syntactically and conceptually
parallel.

Separate paragraphs in a series, such as itemized conclusions or steps in a procedure,
are identified by an Arabic numeral followed by a period but not enclosed m or
followed by parentheses. Separate sentences in a series are also identified by an ArabiC
numeral followed by a period; the first word is capitalized, and the sentence ends with
a period or correct punctuation.

Using the learned helplessness theory, we predicted that the depressed and
non de pressed participants would make the following judgments of control:

\begin{enumerate}
  \item Individuals who ... [paragraph continues].
  \item Nondepressed persons exposed to ... [paragraph continues].
  \item Depressed persons exposed to ... [paragraph continues].
  \item Depressed and nondepressed participants in the no-noise groups ... [paragraph continues].
\end{enumerate}

\paragraph{Lists (Seriation, Enumeration).} The Chicago Manual of Style, uses the term \enquote{lists} to refer to enumeration, or seriation.

Seriation is the process of (1) listing a series of topics, (2) marked by numbers or letters in parentheses,
(3) to delineate subjects that merit individual attention in the text. This paragraph gives an example of a
run-in list in Chicago terminology. 

\paragraph{Paragraph seriation.} If each element in the series requires a separate paragraph, or is a complete
sentence, these are set flush with the left margin, with longer sentences aligned as text blocks to the right
of the number. An introductory clause or sentence ending with a colon typically introduces the series:

      
\begin{enumerate}
\item This form of seriation or verical list is useful in detailing and summarizing an argument, or perhaps the
results of a research study. \enquote{If items run over a line, the second and subsequent lines are usually
indented [hanging indent]. . . . aligned with the first word following the numeral}  (CMS 2003, 272).

\item \enquote{All items in a list should be syntactically alike--that is, all should be noun forms, phrases, full
sentences, or whatever the context requires} (CMS 2003, 271).

\item Lower level run-in heading? The paragraph list format merges into that for the lowest level
heading. The two can readily be combined when a fourth level heading is needed.
\end{enumerate}

\paragraph{Bullet lists.} These ue heavy dots •  or squares to  make visual signposts in unnumbered lists but ``can lose their force
if used too frequently” (CMS 2003, 272)
Knuth did not spend much time on lists, probably because he never intended to
provide a fully fledged system.

\begin{itemize}
\item \lorem
\item \lorem
      \lorem
\end{itemize}

\paragraph{Typography}

The simplicity of the APA and Harvard requirements serves most cases. In many disciplines of the humanities more complex
list constructions can be found. For example \ref{fig:woodard} there are at list four different types of lists used in the text. Note the traditional typographic way denoting alpha lists after the letter Z. It starts with AA, BB, CC, rather than AA, AB, AC etc. 

\begin{texexample}{Enumeration}{ex:enum}
\ExplSyntaxOn
\int_to_Alph:n {28}
\ExplSyntaxOff
\end{texexample}

English lacks a standard set of numerals, other than the arabic numerals that are used daily. Other languages such as Greek and of course Latin, which used Roman numerals  had their own numeral system.  

\begin{figure}
\centering
\fboxrule=2pt
 \fbox{\includegraphics[height=0.8\textheight,keepaspectratio]{woodard.pdf}}
 \caption{Enumerated lists, using uppercase alphabetic characters for the second level. At the first level the numbering is continuous through sections and in brackets, without an ending stop, whereas at the second level the numbers are followed by stops. The right margin is set at the same width than that of the text.}
 \label{fig:woodard}
\end{figure}

\begin{figure}
\centering
\fboxrule=2pt
 \fbox{\includegraphics[height=0.8\textheight,keepaspectratio]{grammar-01.pdf}}
 \caption{Enumerated lists, using uppercase alphabetic characters for the second level. At the first level the numbering is continuous through sections and in brackets, without an ending stop, whereas at the second level the numbers are followed by stops. The right margin is set at the same width than that of the text.}
 \label{fig:woodard}
\end{figure}

\paragraph{Knuth's lists in \PlainTeX}

Compared to the code that follows this chapter things are much simpler using \PlainTeX. Knuth
wrote:

\begin{quote}
Paragraph shapes of a limited but important kind are provided by |\item|,
\cs{itemitem}, and \cs{narrower}. There are also two macros that haven't been mentioned
before: (1) |\hang| causes hanging indentation by the normal amount of |\parindent|,
after th first line; thus, the entire paragraph will be indented by the same amount
(unless it began with |\noindent|). (2) |\textindent|{stuff} is like |\indent|, but it puts
the `stuff' into the indentation, push right except for an en space; it also removes spaces
that might follow the right brace in `{stuff}'. For example, the present paragraph
was typeset by the commands `|\textindent{$\bullet$}| Paragraph shapes\ldots'; the
opening \enquote{P} occurs at the normal position for a paragraph's first letter. TeXbook{352}
\end{quote}

The definitions as described by Knuth above, are simpler than those provided by \latexe. They only deal
with the shaping of the paragraph. They are defined as follows:

\begin{teXXX}
\def\hang{\hangindent\parindent}
\def\item{\par\hang\textindent}
\def\itemitem{\par\indent \hangindent2\parindent \textindent}
\def\textindent#1{\indent\llap{#1\enspace}\ignorespaces}
\def\narrower{\advance\leftskip by\parindent
  \advance\rightskip by\parindent}
\end{teXXX}

The usage of the commands is shown in Example~\ref{ex:tex-lists.} that follows. 

\begin{texexample}{TeX list example}{ex:tex-lists}
\bgroup
\parindent2em
% Define the macros
\def\hang{\hangindent\parindent}
\def\item{\par\hang\textindent}
\def\itemitem{\par\indent \hangindent2\parindent \textindent}
\def\textindent#1{\indent\llap{#1\enspace}\ignorespaces}
\def\narrower{\advance\leftskip by\parindent
\advance\rightskip by\parindent}

% Example usage
\item{1} \lorem
\itemitem{2} \lorem
\egroup
\end{texexample}
\vskip0pt plus10pt minus10pt

Knuth was of the opinion that every book had to have its own commands defined specifically for the book. The format
\PlainTeX\ was never intended to be a generic framework that could be adapted. Lamport's \latexe that followed had the opposite idea. He developed \latex as a framework that could be moulded by the classes into different styles. His idea of enclosing
text in \emph{environments} enabled users to star separating the styling of the pages from the structure of the document.
We will review these ideas and the code in the next chapter.


\vfill





