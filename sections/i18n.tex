\parindent1em
\cxset{section font-family=tiresias,
      section font-family=sffamily,
      chapter format=block}

\chapter{Internationalization and Globalization}


\section{Introduction}

In this Chapter we discuss the requirements for localization of software and how this can be applied to \latex. In a way this chapter overlaps the one on languages. However, here we focus mostly on LuaTeX solutions. We also extend the discussion to calendric and solar calculations.

Internationalization is the process of designing a software application so that it can potentially be adapted to various languages and regions without engineering changes. Localization is the process of adapting internationalized software for a specific region or language by adding locale-specific components and translating text. Localization (which is potentially performed multiple times, for different locales) uses the infrastructure or flexibility provided by internationalization (which is ideally performed only once, or as an integral part of ongoing development).\index{internationalization}\index{globalization}

The development of routines for software internationalization and globalization has been an ongoing effort for many years. Currently the accepted method for building such software is the use of i18n. This is an abbreviation of the first letter and last letter of the word internationalization and the 18 is the number of characters in the word.

Internationalization based on i18n is not an easy task for \LaTeX. To an extend some of the issues have been removed with the use of Babel and Polyglossia that provide translation strings for many of the worlds scripts. The de facto resource for internationalization is the Unicode Consortium’s \href{http://cldr.unicode.org/}{CLDR} project.\index{i18n}

\section{Enforcing local styles}

To understand the magnitude of the problem let us look at some of the easier parts of localizing. Consider the Greek days of the week.
\medskip
\begin{trivlist}\item[]\panunicode
\begin{tabular}{llll}
\toprule
Day &Normal Form &Abbreviation &Narrow\\
Monday &Δευτέρα &Δευ. &Δ. \\
\midrule
\end{tabular}
\end{trivlist}

In Greek the abbreviated form, is always capitalized and a stop is provided. The same is true for the month. The narrow form can give problems, unless it is for calendars, where the content is clear. This is because "{\panunicode Π}" are the initials for both ``{\panunicode Πέμπτη}" (Thursday), and ``{\panunicode Παρασκευή}" (Friday). 

In date formats with long month format, that do not include the day, the full month form should be used.
In date formats with long month format, that also include the day, the long date format should be used.

If limited space is available, it is possible to omit the period in the abbreviated form of months, but this should be used only when there is a serious technical restriction

Ultimately, we are aiming at providing the necessary rules to build an automated style that can be used by the system.
                

\section{Locales}
\index{locale}

In computing, a \emph{locale} is a set of parameters that defines the user's language, country and any special variant preferences that the user wants to see in their user interface. Usually a locale identifier consists of at least a \textit{languag}e identifier and a \textit{region} identifier.

On POSIX platforms such as Unix, Linux and others, locale identifiers are defined similar to the BCP 47 definition of language tags, but the locale variant modifier is defined differently, and the character set is included as a part of the identifier. It is defined in this format: |[language[_territory][.codeset][@modifier]]|. (For example, Australian English using the UTF-8 encoding is en\_AU.UTF-8.)

For \latex these ``locales'' can be thought of as the settings of language keys through Babel and Polyglossia. These settings have served the community well for many years, but a litany of duct taping through other packages are a testimony to their limitations. Packages for dates, time and number formatting have been developed to assist. Here is my attempt to put the solution on a better footing and to start providing mechanisms via LuaTeX for a 'plugin'
architecture to find improve solutions. 

\section{Common Locale Data Repository}

The Common Locale Data Repository Project, is a project of the Unicode Consortium to provide locale data in the XML format for use in computer applications. CLDR contains locale specific information that an operating system will typically provide to applications. CLDR is written in LDML (Locale Data Markup Language). The information is currently used in International Components for Unicode, Apple's Mac OS X, OpenOffice.org, and IBM's AIX, among other applications and operating systems

\begin{enumerate}
\item Translations for language names.
\item Translations for territory and country names.
\item Translations for currency names, including singular/plural modifications.
\item Translations for weekday, month, era, period of day, in full and abbreviated forms.
\item Translations for timezones and example cities (or similar) for timezones.
\item Translations for calendar fields. This is useful especially in conjuction with PGF presentational forms.
\item Patterns for formatting/parsing dates or times of day.
\item Examplar sets of characters used for writing the language.
\item Patterns for formatting/parsing numbers.
\item Rules for language adapted collation. \label{collation}
\item Rules for formatting numbers in traditional numeral systems (like Roman numerals, Armenian numerals, ...).
\item Rules for spelling out numbers as words.
\item Rules for transliteration between scripts. A lot of it is based on BGN/PCGN romanization.
\item Rules for \emph{delimiters} such as quotations and question marks.
\end{enumerate}

Currently the consortium’s distribution make the data available in both json and xml formats.  These files hold data for a specific \emph{locale}. Sadly missing are any document sectioning information that would have enabled the incorporation of the above into LaTeX and overcoming some of the Babel and Polyglossia limitations.

We do not need many of the files provided by the CLDR unicode consortium and others we are missing. Take for example the |delimiters| file. 

\bgroup
\scriptsize
\begin{phdverbatim}
  "main" = {
    "ff": {
      "identity": {
        "version": {
          "_cldrVersion": "26",
          "_number": "$Revision: 10739 $"
        },
        "generation": {
          "_date": "$Date: 2014-08-07 12:54:13 -0500 (Thu, 07 Aug 2014) $"
        },
        "language": "ff"
      },
      "delimiters": {
        "quotationStart": "„",
        "quotationEnd": "”",
        "alternateQuotationStart": "‚",
        "alternateQuotationEnd": "’"
      }
    }
  }
}
\end{phdverbatim}
\egroup

Of course the |Json| format as it is, is not readable by Lua a format such as:

\begin{verbatim}
delimiters = {
        quotationStart = "«",
        quotationEnd = "»",
        alternateQuotationStart = "\"",
        alternateQuotationEnd = "\""
      }
\end{verbatim}

\begin{texexample}{i18n}{i18-1}

\panunicode
\begin{luacode*}
-- mock the delimiters from the json
-- file
greekname = 'el'
delimiters = {
        quotationStart = "«",
        quotationEnd = "»",
        alternateQuotationStart = [["]],
        alternateQuotationEnd = [["]]
      }
tex.print(delimiters.quotationStart .. 'test' .. delimiters.quotationEnd)
tex.print ([[\gdef\]] .. greekname .. [[quote#1{\directlua{tex.sprint(delimiters.quotationStart .. '#1' .. delimiters.quotationEnd)}}]])
\end{luacode*}

\def\elquote#1{%
  \directlua {tex.sprint(delimiters.quotationStart .. '#1' .. delimiters.quotationEnd)}
}
\end{texexample}



This is of course a much more simplified way of what one needs to program for a full system. The advantage
of producing the \tex definition also through LuaTeX is that we can keep all the code in one place and econd, we can avoid |\csname| costructs.
\begin{texexample}{elquote}{}
\elquote{This is some longer text in Greek quotes.}
\end{texexample}

I have opted to incorporate these files in the |json| format and provide routines for interfacing via the \pkgname{phd} package.  The reason for opting for a json format, is my other attempts to interface the package with |couchdb|.  My preference for a Nosql type of database, is that  they are better suited in handling data that is commonly  found in documents and also many of the routines will be interchangeable for web applications. I am also hoping that the collation information (see \ref{collation}), will eventually lead to better indices, a subject left untouched in the current distribution.\index{json}

\section{The package phd approach}

The package |phd| packge takes an approach to use only json resource files for the provision of language dependent information, rather than TeX commands alone, as is done by Babel and Polyglossia. 

\section{Language and Region Tags}
\index{tags>regions}\index{tags>language}

Languages are represented by tags such as "en"  for English or "el" for Greek. Other languages have no significant variation and are represented by a language subtag such as "en-US".  The names are mostly intuitive, but in many case bear no relationship to their English names, for example Armenian is coded as \textbf{hy}. There is a useful utility at the SIL website for viewing these codes.\footnote{\protect\url{http://www-01.sil.org/iso639-3/codes.asp?order=reference_name&letter=\%25}.} Note that the CLDR database does not cover all the languages listed in the ISO-639.\footcite{iso639} \index{ISO-639}

The language tags are based on the BGN which is mapped to languages based on ISO-639-1.

ISO 639-2 is the alpha-3 code in Codes for the representation of names of languages-- Part 2. There are 21 languages that have alternative codes for bibliographic or terminology purposes. In those cases, each is listed separately and they are designated as "B" (bibliographic) or "T" (terminology). In all other cases there is only one ISO 639-2 code. Multiple codes assigned to the same language are to be considered synonyms. ISO 639-1 is the alpha-2 code.

We will describe the tables using the English language, which is normally the default and Greek as a second language, as the script is distinctive enough to demonstrate their use. We will also explain Lua routines available via the \pkgname{phd} that are provided as alternatives to Babel and Polyglossia.

{layout.lua}

{layout.orientation.characterOrder} = |left_to_right| or |right_to_left|

layout.orientation.lineOrder = |top_to_bottom|

Example \ref{i18-1} loads the Greek internationalization file |layout| and prints the two fields. Before we send it to
the TeX typesetter we sanitize the string underscores using |gsub|. For illustration purposes we have used |gsub| both as an object method and as a function.

\begin{texexample}{i18n}{i18-1}
\begin{luacode}
local c = require("i18n.el.layout")
local s1 = string.gsub(c.el.layout.orientation.characterOrder, '_', '\\textunderscore ')
local s2 = c.el.layout.orientation.lineOrder:gsub('_', '\\textunderscore ')
tex.print('typeof :', type(c))
tex.print(s1, '\\par', s2)
\end{luacode}
\end{texexample}

Of course for Greek the above information is hardly necessary, but at the level of Lua programming, if we are automating the switching of text direction Greek text might signal a change in direction. Let us have another try using the same code for arabic text. All we have to change is the \textbf{el} to \textbf{ar}.

\begin{texexample}{i18n}{i18-2}
\begin{luacode}
local c = require("i18n.ar.layout")
local s1 = string.gsub(c.ar.layout.orientation.characterOrder, '_', '\\textunderscore ')
local s2 = c.ar.layout.orientation.lineOrder:gsub('_', '\\textunderscore ')
tex.print('typeof :', type(c), '\\par')
tex.print(s1, '\\par', s2)
\end{luacode}
\end{texexample}

In the next example we get the string for the first month of the year in the ``abbreviated'' style. I have changed the json
strings directly to Lua for this file to speed up processing.

\begin{texexample}{i18n}{i18-2}
\begin{luacode}
local c = require("i18n.el.cagregorian")
local months = c.el.dates.calendars.gregorian.months.formats
local days = c.el.dates.calendars.gregorian.days.formats

tex.print("\\begin{tabular}{ll} ")
for i=1,12 do
  tex.sprint(i.." &"..months.wide[i].."\\\\ ")
end
tex.print("\\end{tabular}")
\end{luacode}
\end{texexample}

Printing directly to the document has many benefits but does slow developemnt, both of the code as well as the document. Another distraction is transferring arguments from \tex to Lua and vice versa.

Similarly we can print the months in the Italian language by loading the \textbf{i18n.italian} module and iterating through the month strings. I am still thinking about the interface and the best way forward to provide an easy to use and remember interface. 


Let us now develop a longer example. We will load a number of languages and typeset a table for the different months.
Since we are running the example directly in the document, some patience is required. 

\bigskip 

\begin{texexample}{Month string in various languages}{ex:transl}
\bgroup
\parindent0pt
\newfontfamily\langtable{code2000}
\langtable
\scriptsize
\begin{luacode} 

c = require("i18n.irish")
d = require("i18n.russian")
e = require("i18n.latin")
f = require("i18n.german")
g = require("i18n.kannada")
h = require("i18n.lao")
j = require("i18n.turkish")
k = require("i18n.albanian")

local count=0

local months_irish = c.irish.months
local months_russian = d.russian.months
local months_latin = e.latin.months 
local months_german = f.german.months
local months_kannada = g.kannada.months
local months_lao    = h.lao.months
local months_turkish = j.turkish.months
local months_albanian = k.albanian.months
local centering = function()
                     tex.print("\\centering")
end

local par = function()
               tex.print("\\par")   
end

local tabular = function() 
	tex.print("\\begin{tabular}{clllllll}")
	tex.sprint("\\toprule")
end	


local endtabular = function()
	tex.print("\\bottomrule")
	tex.print("\\end{tabular}")
	tex.print("\\medskip")
end

local eol = function()
  return("\\\\")
end


-- center the table
centering()
tabular()
tex.sprint("Month","&Irish", "&Russian", "&Latin", "&Kannada", "&Lao","&Turkish","&Albanian", eol())
tex.sprint("\\midrule")
for i = 1,12 do
   count = i
   tex.sprint(i.."&", months_irish[i],
                 "&",months_russian[i], 
                 "&",months_latin[i], 
                 "&", months_kannada[i], 
                 "&"..months_lao[i], 
                 "&"..months_turkish[i],
                 "&"..months_albanian[i],
                 eol() )
end  
endtabular()
par()

\end{luacode} 
 
\egroup
\end{texexample}

Now some explanation for the code. We started by loading the necessary libraries for the languages that we wanted to print the month strings and allocated them to local variables.

We then iterated through the twelve months of the gregorian table and typeset them. We could have put the languages in a Lua table and iterated over them. I haven't done it so that the code is clearer. I tried to keep the API functions separate as much as possible. We also defined a font using \docAuxCommand{newfontfamily} of the \pkg{fontspec} package to ensure that we can print the Asian and Cyrillic scripts.

The long javanesque object notations make it difficult to work, but once they are set in functions and locals, development is fast. After the detour to explore the i18n tables and available information, we are now ready to tackle the production of multi-lingual calendars and to complete are library on internationalization. Before we do that a detour to understand
the complexity of calendrical calculations and some historical information is required.
\vfill



