
\cxset{epigraph width=0.7\textwidth}

\chapter{Epigraphs}\index{epigraphs}
\label{c:epigraphs}

\epigraph{Please give examples of good use of epigraphs in fiction.

I mean them quoted dealies they sometimes put at the start of chapters.

What counts as ``good use" is whatever you think counts. Part of my goal is to understand what people like about these things.
.}{\href{http://ask.metafilter.com/207423/Good-use-of-epigraphs-in-fiction}{Stebulus}}



\section{Introduction}

Epigraphs or quotations before or after chapters are quite common in books. Peter Wilson's epigraph package \citep{epigraph}, 
does a good job and we have adapted it where necessary to allow for a key value interface. The command:


\begin{docCommand}{epigraph}{ \marg{text}\marg{source}}{ }
\end{docCommand}

By default the epigraph is placed at the right
hand side of the textblock, and the \marg{source} is typeset at the bottom right of the \marg{text}. 
Numerous settings allow for manipulating the width of the epigraph, the location and other 
variables. If the package is available we use it otherwise we use other internal commands.

All key values for epigraphs, start with the keyword \emph{epigraph}. You can think of the epigraph of a block of text that can go anywhere on a page and has some formatting rules that are set 

\section{Key-value interface}
The key value interface provided by the package is shown below. It mostly follows the 
naming conventions of the epigraph package to make the transition easier for experienced users. Use any dimension or a dimension expression.
\medskip

\begin{docKey}[phd]{epigraph width}{ = \marg{dim} }{no default, initial=0.6\cs{textwidth}}
  Sets the width of the epigraph block. 
\end{docKey}



\begin{docKey}[phd]{epigraph align}{ = \marg{left\textbar center\textbar right}}{no default, initial=right}
 A font-size command such as \cs{footnotesize}, 
\cs{small} and other similar commands. This will align the full block containing the epigraph, left right or center according to the setting of the key. Most epigraphs are aligned right.
\end{docKey}

\begin{texexample}{Setting epigraph widths}{ep:align}
% set properties
 \cxset{epigraph align=left, 
           epigraph width=300pt}
% write the epigraph           
 \epigraph{Example is the school of mankind, and they
   will learn at no other.}{unknown}
\end{texexample}

\begin{docKey}[phd]{epigraph rule width}{ = \marg{dim}} {no default, initial=0.4pt }
 The width of the rule separating the epigraph from the source. Set to 0pt,if you do not want a rule.
\end{docKey}



\begin{docKey}[phd]{ epigraph font-size}{ = \marg{font sizing cmd}} {no default,}
Use a font sizing command such as \cmd{\footnotesize}
\end{docKey}

\begin{docKey}[phd]{ epigraph beforeskip}{ = \marg{dim}}{no default, }
Space before the epigraph.
\end{docKey}

\begin{docKey}[phd]{ epigraph afterskip}{ = \marg{dim}}{no default, }
Space after the epigraph.
\end{docKey}

\subsection{Styling the source part}

\begin{docKey}[phd]{ epigraph source align}{ = \marg{left\textbar center\textbar right}}{no default, initial = right}
Align the source text to the right, left or center.
\end{docKey}

\begin{docKey}[phd]{ epigraph source font-size}{ = \meta{tiny, scriptsize, footnotesize, small etc.} } {default small }
Align the source text to the right, left or center.
\end{docKey}

\begin{docKey}[phd]{ epigraph source font-shape}{ = \marg{dim}}{no default, }
Align the source text to the right, left or center.
\end{docKey}

\begin{docKey}[phd]{ epigraph source font-family}{ = \marg{dim}}{no default, }
Align the source text to the right, left or center.
\end{docKey}


\begin{docKey}[phd]{epigraph source font-weight}{ = \marg{bold,normal}}{no default, }
Align the source text to the right, left or center.
\end{docKey}


Usage examples can be found in relevant style examples (See Chapter~\ref{ch:41}) for a rather 
nice example with non-traditional alignment.

\section{Epigraphs on empty pages}

When a chapter open on an odd page sometimes the  previous page is left empty. Some book designers 
add the words ``this page left intentionally blank'' and other might add a quote. To add such a quote use:

%\begin{verbatim}
%\cxset{blank page text=\epigraph{The great tragedy of science is 
%                                   the slaying of a beautiful theory
%                                  by an ugly fact.}{Thomas Huxley}}
%\end{verbatim}

\endinput

