\cxset{chapter name = Chapter}

\chapter{The \texttt{picture} Environment}
\label{pictureenvironment}
\index{environments=picture}
\index{packages=picture}

When TeX was developed, the notion of graphic output was very limited, although Knuth presented a method
using boxes to draw primitive commands at any point on the page. This of course is achieved using zero width or height |\hbox| or |\vbox| commands. LaTeX uses a similar approach with the picture environment. 
The |picture| environment comes straight out of the box and can be used to draw simple figures. For more sophisticated graphics |TikZ| is a better choice. It can be used in package documentation and simple tasks. The learning curve for using it is minimal.

Using the picture environment is much easier to code for drawing shapes or rulers around sectioning commands.
This type of heading is very popular in many modern books. Figure~\ref{fig:picture-sections}

\begin{figure}[htbp]
\includegraphics[width=\textwidth]{./images/picture-sections.jpg}
\caption{A section with some fancy lines around the text. From \textit{Probabilities and Statistics for Engineers and Scientists}, by Walpole \textit{et.al}, 2011. }
\label{fig:picture-sections}
\end{figure}

Of course this is also achievable without the picture environment, simpy using TeX commands or with tikZ. From graphics point of view, the environment is also useful for short mathematical diagrams.

\section{The Basic Commands}

\begin{docEnvironment}{picture}{}{}
\end{docEnvironment}
The |picture| environment is created using one of two commands.

\emphasis{picture}
\begin{teXXX}
 \begin{picture}(x, y). . . \end{picture}
\end{teXXX}

\noindent or

\begin{teXX}
  \begin{picture}(x, y)(x0,y0). . . \end{picture}
\end{teXX}

\begin{docCommand}{unitlength}{\marg{dim}}
Most people prefer the first type which they combine, with a |setlength| command that sets the \cs{unitlength}.
\end{docCommand}

The optional argument gives the coordinates of the point at the lower-left corner of the picture (thereby determining the origin). For example, if \cs{unitlength} has been set to 1mm, the command

\begin{texexample}{}{}
  \setlength\unitlength{1mm}
  \begin{picture}(40,40)(0,0)
    \put(10,30){\vector(0,-1){30}}
    \put(10,30){\vector(1,0){30}}
    \put(25,30.5){$a$} 
  \end{picture}
\end{texexample}

produces a picture of width 100 millimeters and height 200 millimeters, whose lower-left corner is the point (10,20) and whose upper-right corner is therefore the point (110,220). When you first draw a picture, you will omit the optional argument, leaving the origin at the lower-left corner. If you then want to modify your picture by shifting everything, you just add the appropriate optional argument.

\section{Text and Formulae}

%\begin{macro}{\linethickness}
%\begin{macro}{\thicklines}
%\begin{macro}{\thinlines}
Text and formulas can be written into a picture
environment with the \cs{put} command in the usual way. The line thickness can be
set by using \cs{linethickness}\marg{dim}. The command \cs{thinlines} is half the thickness of the \cs{linethickness} dimension and \cs{thicklines} is the current line width. The \cs{linethickness} does not change width of slanted lines
or circles as it is drawn using a font and would render badly.
%\end{macro}
%\end{macro}
%\end{macro}

\emphasis{thicklines}
\begin{texexample}{Text and Formulae}{}
\setlength{\unitlength}{0.8cm}
\begin{picture}(6,5)
 \thicklines
 \put(1,0.5){\line(2,1){3}}
 \put(4,2){\line(-2,1){2}}
 \put(2,3){\line(-2,-5){1}}
 \put(0.7,0.3){$A$}
 \put(4.05,1.9){$B$}
 \put(1.7,2.95){$C$}
 \put(3.1,2.5){$a$}
 \put(1.3,1.7){$b$}
 \put(2.5,1.05){$c$}
 \put(0.3,4){$F=
 \sqrt{s(s-a)(s-b)(s-c)}$}
 \put(3.5,0.4){$\displaystyle
 s:=\frac{a+b+c}{2}$}
\end{picture}
\end{texexample}



\setlength{\unitlength}{5cm}
\begin{picture}(1,1)
\put(0,0){\line(0,1){1}}
\put(0,0){\line(1,0){1}}
\put(0,0){\color{blue}\line(1,1){1}}
\put(0,0){\color{orange}\line(1,2){0.5}}
\end{picture}


\section{multiput and linethickness}
The \cmd{\multiput} is used to place multiple objects onto the picture. It has the general format shown below:

\setlength{\unitlength}{2mm}
\begin{picture}(30,20)
  \color{green}
   \linethickness{0.075mm}
   \multiput(0,0)(1,0){25}%
   {\line(0,1){20}}
   \multiput(0,0)(0,1){21}%
   {\line(1,0){25}}
   \linethickness{0.15mm}
   \multiput(0,0)(5,0){6}%
   {\line(0,1){20}}
   \multiput(0,0)(0,5){5}%
   {\line(1,0){25}}
   \linethickness{0.3mm}
   \multiput(5,0)(10,0){2}%
    {\line(0,1){20}}
   \multiput(0,5)(0,10){2}%
   {\line(1,0){25}}
\end{picture}



\begin{docCommand}{multiput} {(x,y) (Dx, Dy) \marg{n} \marg{object} }
The command |\multiput| allows to repeat
a \cmd{\put} a number of times.
\end{docCommand}

\begin{figure}
\setlength{\unitlength}{0.8cm}
\begin{picture}(6,5)
 \thicklines
 \put(1,0.5){\line(2,1){3}}
 \put(4,2){\line(-2,1){2}}
 \put(2,3){\line(-2,-5){1}}
 \put(0.7,0.3){$A$}
 \put(4.05,1.9){$B$}
 \put(1.7,2.95){$C$}
 \put(3.1,2.5){$a$}
 \put(1.3,1.7){$b$}
 \put(2.5,1.05){$c$}
 \put(0.3,4){$F=
 \sqrt{s(s-a)(s-b)(s-c)}$}
 \put(3.5,0.4){$\displaystyle
 s:=\frac{a+b+c}{2}$}
\end{picture}
\caption{Figures can have captions, if you enclose in a figure environment}
\end{figure}

\begin{figure}
\scalebox{0.7}{
\setlength{\unitlength}{0.5mm}
\begin{picture}(120,168)
\newsavebox{\foldera}
\savebox{\foldera}
(40,32)[bl]{% definition
\multiput(0,0)(0,28){2}
{\line(1,0){40}}
\multiput(0,0)(40,0){2}
{\line(0,1){28}}
\put(1,28){\oval(2,2)[tl]}
\put(1,29){\line(1,0){5}}
\put(9,29){\oval(6,6)[tl]}
\put(9,32){\line(1,0){8}}
\put(17,29){\oval(6,6)[tr]}
\put(20,29){\line(1,0){19}}
\put(39,28){\oval(2,2)[tr]}
}
\newsavebox{\folderb}
\savebox{\folderb}
(40,32)[l]{% definition
\put(0,14){\line(1,0){8}}
\put(8,0){\usebox{\foldera}}
\put(0.2,1.4)
{$\beta=v/c=\tanh\chi$}
}
\put(34,26){\line(0,1){102}}
\put(14,128){\usebox{\foldera}}
\multiput(34,86)(0,-37){3}
{\usebox{\folderb}}
\end{picture}}
\caption{Pictures can be scaled using \protect\textbackslash scalebox.}
\end{figure}

\section{Some examples}
Any vertex-symmetric graph is regular, but edge-symmetric graphs
need not be regular. For example,
\begin{verbatim}
$$\unitlength=10pt
\def\putdisk(#1,#2){\put(#1,#2){\disk{.4}}}
$\vcenter{
\hbox{\beginpicture(2,1.5)(0,0)
\putdisk(0,0)
\putdisk(2,0)
\putdisk(1,.5)
\putdisk(1,1.5)
\put(0,0){\line(2,1){1}}
\put(2,0){\line(-2,1){1}}
\put(1,.5){\line(0,1){1}}
\endpicture}}
\quad&\hbox{is edge-symmetric, not vertex-symmetric;}\cr
\noalign{\smallskip}
\vcenter{
\hbox{\beginpicture(2,2)(0,0)
\putdisk(1,0)
\putdisk(1,2)
\putdisk(0,.5)
\putdisk(0,1.5)
\putdisk(2,.5)
\putdisk(2,1.5)
\put(0,.5){\line(2,1){2}}
\put(2,.5){\line(-2,1){2}}
\put(0,.5){\line(2,3){1}}
\put(2,.5){\line(-2,3){1}}
\put(0,.5){\line(1,0){2}}
\put(0,1.5){\line(1,0){2}}
\put(1,0){\line(-2,3){1}}
\put(1,0){\line(2,3){1}}
\put(1,0){\line(0,1){2}}
\endpicture}}
\quad&\hbox{is vertex-symmetric, not edge-symmetric.}\qquad
 (\vcenter{\hbox{\beginpicture(1,2)(0,0)
\putdisk(.5,0)\putdisk(.5,2)\put(.5,0){\line(0,1){2}}\endpicture}}
\hbox{ is a maximal clique})\cr}$$
\end{verbatim}



\section{picture package}

The \pkg{picture} package by Heiko Oberdiek redefines the default \pkg{picture} macros and adds code that detects
whether such an argument is given as number or as length. In the latter case, the
length is used directly without multiplying with \cs{unitlength}. Th following
example i from the documentation of the package.

 \setlength{\unitlength}{1pt}
 \begin{picture}(\widthof{Hello World}, 10mm)
   \put(0, 0){\makebox(0,0)[lb]{Hello World}}%
   \put(0, \heightof{Hello World} + \fboxsep){%
   \line(1, 0){\widthof{Hello World}}%
 }%
 \put(\widthof{Hello World}, 10mm){%
   \line(0, -1){10mm}%
 }%
 \put(0,0){\line(966,259){8}}
 \end{picture}

The package |calc| is used for calculations or etex. The picture package requires that the package |calc| is loaded before
the |picture| package and is loaded correctly by |phd|.

The package also supports the packages \pkg{pspicture} and \pkg{pict2e}, but they must be loaded before package picture.

\section{pict2e}

The package pict2e by Hubert G\"a\ss lein, Rolf Niepraschk and Joseph Tkadlec extends the existing LATEX picture environment, using the familiar
technique (cf. the graphics and color packages) of driver files. In the user-level part of
this documentation there is a fair number of examples of use, showing where things are
improved by comparison with the Standard LaTeX picture environment.

The package is loaded automatically by |phd|.






