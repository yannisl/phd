\chapter{PREFACE}

Over the course of last year, I have been experimenting in using |\LaTeX\| for the production of reports in our
organization. The results have been more than satisfactory and as a byproduct these reports now are more
readable and are produced quickly and efficiently. The process involved some difficulties. Firstly these reports
are produced weekly under strict deadlines. Another difficulty is that a number of people contribute
to these reports that were not familiar with \LaTeX. Some of them don't even realize that something other
than Word can be used to produce a report.

My first attempt was to use one of the standard classes. This worked well, but by the time I got the layout to somewhere near to what I wanted, the preamble had grown to something like two pages and twenty to thirty packages. The next attempt started from the thesis class. It looked a bit better, but still was difficult to adapt it into something that would fit in with a  corporate environment. Nobody complained about the looks of the printed pages, but the companied centered around the Title pages. Corporate environments are concerned about maintaining the brand image\footnote{even if this looks terrible!}, corporate colors and the like. I then tried the US Corps of Engineers class, which looked closer to what I has a mind. Another class that we have tried and still use for some type of reports is the \texttt{Tufte-book} class. This went well with almost everyone, especially for reports that had a lot of margin figures and really a lot of sidenotes. For progress reports, the sidenotes were used to almost `talk' to the reader.

In the meantime I produced every conceivable document with \LaTeX\, except letters and minutes. This includes a lot of the run-of-the-mill documents such as method statements, reports, progress reports, planning reports, cost reports and the like. Many co-existed and still co-exist with similar documents that were previously produced with Word.

One of the main points I want to emphasize, is that if you are to succeed, you need to give thought to the user interface. Class authors, like the \texttt{memoir} and \texttt{koma} have produced countless of commands, possibly running into their hundreds, for adapting the class to almost any layout conceivable. Anyone not familiar with them with a deadline to get a report out by next Thursday will quickly abandoned the idea. However, no one argued with an approach such as the minimal below:

\begin{verbatim}
\documentclass[themea, themeb]{weekly-report}
\begin{document}
    .....
\end{document}
\end{verbatim}

We tried to keep the interface as simple as possible and move the complexity into the class and sub-classes. Neither Lamport nor Knuth in my opinion gave any indication that one should develop a mother of all classes and use it and modify as you go along. The actual class \textit{HLS} has a hook at the end, where localizations are possible.



\section*{Where to start?}

When starting to develop a new class or embark in a serious attempt to modify an existing one, always start from another base. A number of classes are available. Which ever class you decide to use as a base, firstly produce a real case document with it, but without any modifications or additions of too many packages. Add packages as you need them and make notes in the text regarding your choice. You might not like the looks and figures and floats might not be where you want them to be, but just press on with finishing at least a couple of chapters from your book and to include all the necessary sectioning such as bibliography index etc.

At this point if you have started from the standard LaTeX classes, you should be able to experiment and use all the classes on your short list, without any problems. At this point and is worth printing all the samples out decide on the base class to use.


\section*{Conventions}

\section*{Consistency}


\section*{The Spiral Technique}

The Mathematician P. Halmos, described his technique for writing his books. He called it the spiral technique:

\begin{quotation}
\ldots When you finish with Chapter 2, you go and edit Chapter 1 and when you finish Chapter 3 you edit Chapter 1 and Chapter 2.

\end{quotation}

This is a good technique for writing a book, but it is also a good technique for writing a LaTeX class; the only problem is what are the Chapters of the Class? So the first thing one needs to do is to decide as to what makes a `report` a `book` or a `letter`. Here are the chapters for reports.

\subsection{Document sectioning}

\begin{verbatim}
title
secondpage
acknowledgements
summary/abstract
parts
chapters
--sections
--subsections
references
bibliography
Appendices
Colophon
Page Elements
   paragraphs
   lists
   headers
   footers
   margin pars
   footers
   images
   tables
\end{verbatim}

The presentational elements should be considered on their own, although they are in reality intertwined with the sectioning commands.

\begin{verbatim}
Geometry
Fonts
   sizing
   family
   serified or not
Spacing (indentation)
Layout
Color
Decorative Elements
Page numbering
\end{verbatim}

\section{Geometry}

Although page geometry should be the easiest thing to decide, the type of paper normally used being A4, after having through a number of experiments
we decided to use the \textit{imperial}  size, with proportions as per the Octavo class. This is slightly smaller by about 20 mm both ways, than an A4 page, but one can proportion the text better. We print crop marks and for important publications, we trim the paper after printing at a print shop.

\begin{Verbatim}
{\setlength\paperheight{279mm}%
    \setlength\paperwidth{191mm}}
 210 × 297 mm is A4
\end{Verbatim}
