\section{The shape of a book}

    Books come in many shapes and sizes, but over the centuries certain
shapes have been found to be more pleasurable and convenient than others.
Thus books, except for a very very few, are rectangular in shape. The 
exceptions on the whole are books for young children, although I do
have a book edited by Fritz Spiegl and published by Pan Books entitled
\textit{A Small Book of Grave Humour}, which is in the shape of a tombstone
--- this is an anthology of epitaphs. Normally the height of a book, when 
closed, is greater than the width. Apart from any aesthetic reasons, 
a book of this shape is physically more comfortable to hold than one which 
is wider than it is high.

    It might appear that the designer has great freedom in choosing the
size of the work, but for economic reasons this is not normally the case.
Much typographical design is based upon the availabilty of certain 
standard industrial sizes of sheets of paper\index{paper!size}. 
A page size of \abybm{12}{8}{inches} will be much more expensive than one 
which fits on a standard
US letter sheet\index{paper!size!letterpaper} 
of \abybm{11}{8\:\Mfrac{1}{2}}{inches}. 
Similarly, one of the standard sizes
for a business envelope is \abybm{4\:\Mfrac{1}{8}}{9\:\Mfrac{1}{2}}{inches}. 
Brochures for mailing
should be designed so that they can be inserted into the envelope with 
minimal folding. Thus a brochure size of \abybm{5}{10}{inches} will be 
highly inconvenient, no matter how good it looks visually.

    Over the years books have been produced in an almost infinite variety
of proportions,
where by \emph{proportion}\index{proportion}
I mean the ratio of the height to the width of a
rectangle. However, certain proportions occur time after time throughout
the centuries and across many different countries and 
civilizations. This is because some proportions are inherently
more pleasing to the eye than others are. These pleasing proportions are
also commonly found in nature --- in  physical, biological, and chemical
systems and constructs. 

\index{proportion!page|(}

    Some examples of pleasing proportions can be
seen in Japanese wood block prints, such as the \textit{Hoso-ye} size
(\ratio{2}{1}) which is a double square, the \textit{Oban} (\ratio{3}{2}), %($3 : 2$),
the \textit{Chuban} (\ratio{11}{8}) and the \textit{Koban} size
(\ratio{{\sqrt{2}}}{1}). Sometimes these prints were made up into books, but
were often published as stand-alone art work. Similarly Indian paintings,
at least in the 16th to the 18th century,
often come in the range \ratio{1.701}{1} to \ratio{13}{9}, thus being around
\ratio{3}{2} in proportion.

    In medieval Europe page proportions were generally in the range
\ratio{1.25}{1} to \ratio{1.5}{1}. Sheets of paper\index{paper} were typically 
produced in the
proportion \ratio{4}{3} (\ratio{1.33}{1}) or \ratio{3}{2} 
(\ratio{1.5}{1}). 
All sheet proportions
have the property that they are reproduced with each alternate
folding of the sheet.
For example, if a sheet starts at a size of \abyb{60}{40} 
(i.e., \ratio{3}{2}),
then the first fold will make a double sheet of size \abyb{30}{40}
(i.e., \ratio{3}{4}). The next fold will produce a quadrupled sheet of size
\abyb{30}{20}, which is again \ratio{3}{2}, and so on. 
 The Renaissance typographers tended to like taller books, and their 
proportions would go up to \ratio{1.87}{1}
or so. The style nowadays has tended to go back towards the medieval
proportions.

    The standard ISO page proportions are 
\ratio{{\sqrt{2}}}{1} (\ratio{1.414}{1}). These
have a similar folding property to the other proportions, except in this case
each fold reproduces the original page proportion.
Thus halving an A0 sheet 
(size \abybm{1189}{841}{mm}) produces an A1 size sheet (\abyb{594}{841}),
which in turn being halved produces the A2 sheet (\abyb{420}{594}), down
through the A3, A4 (\abybm{210}{297}{mm}), A5, \ldots sheets.

For many years it was thought that it was impossible to fold a sheet of 
paper\index{paper}, no matter how large and thin, more than eight times 
altogether. This is not so as in 2002 a high school student, Britney Gallivan,
managed to fold a sheet of paper in half twelve times (see, for example,
\url{http://mathworld.wolfram.com/Folding.html}).


   There is no one perfect proportion for a page, 
although some are clearly better
than others. For ordinary books both publishers and readers tend to prefer
books whose proportions range from the light 
\ratio{9}{5} (\ratio{1.8}{1}) to the heavy
\ratio{5}{4} (\ratio{1.25}{1}). Some examples are shown in \fref{flpage:prop}.
 Wider pages, those with proportions less than
\ratio{{\sqrt{2}}}{1} (\ratio{1.414}{1}),
are principally useful for documents that need
extra width for tables\index{table}, marginal notes\index{marginalia}, 
or where multi-column\index{column!multiple} printing is preferred. 

\begin{figure}
\centering
\setlength{\unitlength}{1pc}
\begin{picture}(24,38)
\put(0,4){\begin{picture}(24,34)
  \put(0,0){\framebox(24,34){}}
  \thicklines \put(19.78,0){\line(0,1){34}}
  \thinlines
  \put(16,0){\line(0,1){34}}
  \put(17.78,0){\line(0,1){34}}
  \put(18.48,0){\line(0,1){34}}
  \put(19.2,0){\line(0,1){34}}
  \put(20.81,0){\line(0,1){34}}
  \put(21.33,0){\line(0,1){34}}
  \put(22.63,0){\line(0,1){34}}
  \put(0,-0.5){\begin{picture}(24,2)
    \put(16,0){\makebox(0,0){\textsc{a}}}
    \put(17.78,0){\makebox(0,0){\textsc{b}}}
    \put(18.48,0){\makebox(0,0){\textsc{c}}}
    \put(19.2,0){\makebox(0,0){\textsc{d}}}
    \put(19.78,0){\makebox(0,0){{$\varphi$}}}
    \put(20.81,0){\makebox(0,0){\textsc{e}}}
    \put(21.33,0){\makebox(0,0){\textsc{f}}}
    \put(22.63,0){\makebox(0,0){\textsc{g}}}
    \put(24,0){\makebox(0,0){\textsc{h}}}
    \end{picture}}
  \end{picture}}
  \put(0,0){\begin{picture}(24,4)
    \put(0,0){\begin{picture}(8,4)
      \put(0,2){\textsc{a} $2 : 1$}
      \put(0,1){\textsc{b} $9 : 5$}
      \put(0,0){\textsc{c} $1.732 : 1$ ($\sqrt{3}{} : 1$)}
      \end{picture}}
    \put(8,0){\begin{picture}(8,4)
      \put(0,2){\textsc{d} $5 : 3$}
      \put(0,1){{$\varphi$} $1.618 : 1$ ($\varphi{} : 1$)}
      \put(0,0){\textsc{e} $1.538 : 1$}
      \end{picture}}
    \put(16,0){\begin{picture}(8,4)
      \put(0,2){\textsc{f} $3 : 2$}
      \put(0,1){\textsc{g} $1.414 : 1$ ($\sqrt{2}{} : 1$)}
      \put(0,0){\textsc{h} $4 : 3$}
      \end{picture}}
    \end{picture}}
\end{picture}
\setlength{\unitlength}{1pt}
\caption{Some page proportions} \label{flpage:prop}
\end{figure}

    In books where the illustrations\index{illustration} are the primary 
concern, the shape of the illustrations\index{illustration} is generally 
the major influence on the page proportion.
The page size should be somewhat higher than that of the average 
illustration\index{illustration}. The extra height is required for the 
insertion of captions\index{caption} describing the
illustration\index{illustration}. 
A proportion of \ratio{{\pi}}{e} (\ratio{1.156}{1}), 
which is slightly higher
than a perfect square, is good for square illustrations.\footnote{Both $e$
and $\pi$ are well known mathematical numbers. $e$ ($= 2.718 \ldots$)
is the base of natural logarithms and $\pi$ ($= 3.141 \ldots$) is the
ratio of the circumference of a circle to its diameter.}
The \ratio{e}{{$\pi$}}
(\ratio{0.864}{1}) proportion is useful for landscape 
photographs  taken with a \abyb{4}{5}
format camera, while those from a 35mm camera (which produces a negative
with a \ratio{2}{3} proportion) are better accomodated on 
an \ratio{0.83}{1} page.
\index{proportion!page|)}