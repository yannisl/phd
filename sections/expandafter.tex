\cxset{steward,
  chapter format   = stewart,
  chapter numbering=arabic,
  offsety=0cm,
  image={expandafter.jpg},
  texti={An introduction to the use of font related commands. The chapter also gives a historical background to font selection using \tex and \latex. },
  textii={In this chapter we discuss keys that are available through the \texttt{phd} package and give a background as to how fonts are used
in \latex.
 },
}
%https://bookaroundthecorner.files.wordpress.com/2016/11/hogue_erosion1.jpg
\chapter{Expandafter}

One of the most often misunderstood \TeX\ commands is \docAuxCommand*{expandafter}. This
is an instruction that reverses the order of expansion. It is not a typesetting instruction, but an instruction that influences the expansion of macros. But what is \textit{expansion}? The term expansion means the replacement of the macro and its arguments, if there are any, by the \textit{replacement} text of the macro. If we have defined a macro

\begin{teX}
\def\test{ABC};
\end{teX}


\noindent then the replacement text of |\test| is |ABC| and the \textit{expansion} of |\test| is |ABC|.

As a control sequence |expandafter| can be followed by any number of tokens.

\begin{commands}[]{}
\cmd{\expandafter}\string\token$_e$\string\token$_1$\string\token$_2$\string\token$_n$ etc
\end{commands}

\noindent then the following rules describe the execution of |expandafter|:

\begin{enumerate}
\item  \meta{$<token_e$}, the token immediately following |\expandafter|, is saved without expansion.
\item \meta{$<token_1>$}, which is the token after the saved $token_e$, is analyzed. The following cases can be distinguished:
\begin{enumerate}
\item If is a macro: The macro will be expanded. In other words, the macro and its arguments, if any, will be replaced by the replacement text. After this \tex will \textbf{not} look at the first token of this new replacement text to expand it again or to execute it.
\end{enumerate}



\begin{teX}
\def\xx [#1]{[#1]}
\def\yy{[ABC]}

\expandafter\xx\yy
\end{teX}

This results in 
\def\xx [#1]{[#1]}
\def\yy{[ABC]}

\texttt{> \expandafter\xx\yy}


\item token1 is primitive: Normally a primitive token can not be expanded so the |\expandafter| has no effect; but there are exceptions, which we will discuss after the example.

\begin{texexample}{Expansion}{ex:expandafter}
\expandafter AB
\end{texexample}

Character A is saved. Then \tex\ tries to expand it, but \textit{not} print B, because B cannot be expanded. Finally A is put back in front of the B ; in other words, the two characters are printed in the given order, and we may well have omitted the |\expandafter|. So what's the point here? |\expandafter| reverses the order of expansion, not of execution.

\noindent But there are exceptions to the above:
\begin{enumerate}
\item \textbf{temporarily suspend an opening curly brace} token 1 is is an opening curly brace which leads to the opening curly brace temporarily suspended. This is listed as a separate case because it has some interesting, applications;

\begin{teX}
\newtoks\ta
\newtoks\tb
\ta = {\a\b\c}
\tb=\expandafter{\the\ta}
\tb={\the\ta}
\tb
\end{teX}

\begin{texexample}{Expansion}{}
\begingroup

\def\a{A}
\def\b{B}
\def\c{C}
\newtoks\ta
\newtoks\tb
\ta = {\a\b\c}
\tb=\expandafter{\the\ta}
\tb={\the\ta}

\texttt{> \the\tb}

\texttt{> \the\ta}

\endgroup
\end{texexample}

\item \meta{$token_1$} is another expandafter. The best way to understand this is to write a \tex mnmal example and watch it in action

\begin{teX}
\tracingmacros=2  \tracingcommands=2
\def\a{A}
\def\b{B}
\def\c{C}

\expandafter\expandafter\expandafter\a\expandafter\b\c

\bye
\end{teX}

Checking the log file with |\tracingmacros=2 \tracingcommands=2| we get

\begin{verbatim}
{vertical mode: \def}
{blank space  }
{\def}
{blank space  }
{\def}
{blank space  }
{\par}
{\expandafter}
{\expandafter}
{\expandafter}

\c ->C
{\expandafter}

\b ->B

\a ->A
{the letter A}
{horizontal mode: the letter A}
{\par}

\meaning\futurenonspacelet
\end{verbatim}


\end{enumerate}
\end{enumerate}

\section{Common usage patterns}

One common usage of |\expandafter| is when |\csname| is used to define a macro. The |\expandafter| command is used to suspend the |\def| expansion temporarily, so that the name can be be first defined using the |\csname|
construct.

\begin{texexample}{csname}{ex:csname}
\def\newauthorname#1#2{%
  \expandafter\def\csname#1\endcsname{#2}%
}

\def\getauthorname#1{%
  \csname #1\endcsname
}
\newauthorname{author name}{John Travolta}
\getauthorname{author name}
\end{texexample}

Another example is letting one control sequence equal to another. The below is from the kernel  definition of the internal command for renewing an environment \docAuxCommand{renew@environment}. 

\begin{teX}
 \expandafter\let\csname#1\endcsname\relax
  \expandafter\let\csname end#1\endcsname\relax
\end{teX}

\section{Suppressing expansion }

Expansion of a \meta{token} can be supressed by using the \tex primitive \docAuxCommand{noexpand}. This primitive is typically employed to:

\begin{enumerate}
\item To prevent expansion of tokens in |\edef|\rq{}s.
\item To prevent the expansion of tokens in |\write| operations.
\end{enumerate}

There is no counter part  ``|\expand|'', which could be used in |\def|-based macro defintion to force the expansion of tokens when a macro is defined.

\begin{texexample}{}{}
\expandafter\expandafter\expandafter%
\uppercase\expandafter{\jobname}

\newcount\n
\n=10
\uppercase\expandafter{\romannumeral\n}

\uppercase{\romannumeral\n}

\MakeUppercase{\jobname}


\DeclareRobustCommand{\MakeUppercase}[1]{{%
  \def\i{I}\def\j{J}%
  \def\reserved@a##1##2{\let##1##2\reserved@a}%
  \expandafter\reserved@a\@uclclist\reserved@b{\reserved@b\@gobble}%
  \protected@edef\reserved@a{\uppercase{#1}}%
  \reserved@a
 }}
 
\def\@uclclist{\oe\OE\o\O\ae\AE
  \dh\DH\dj\DJ\l\L\ng\NG\ss\SS\th\TH}
  
\meaning\l  

\meaning\L

\meaning\ng
\end{texexample}







