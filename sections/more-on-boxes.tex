\cxset{style87/.style={
 chapter opening=any,
 name=Chapter,
 % positioning and float - inline is 0
 %  float right is 2
 number display=block,
 number float=right,
 number shape=starburst,
 numbering=Words,
 number spaceout=none,
 number font-size=huge,
 number font-weight=bold,
 number font-family=rmfamily,
 number font-shape=normal,
 number before=,
 number display=inline,
 number float=none,
% 
 number border-top-width=0pt,
 number border-right-width=0pt,
 number border-bottom-width=0pt,
 number border-left-width=0pt,
 number border-width=0pt,
%  
 number padding-left=0em,
 number padding-right=0.5em,
 number padding-top=0em,
 number padding-bottom=0pt,
  %number margin-top=, to do
 %number margin-left=0pt,  to create
 %
 number after=\par,
 number dot=,
 number position=rightname,
 number color=sweet,
 number background-color=white,
 %chapter name
 chapter display=block,
 chapter float=left,
 chapter shape=ellipse,
 chapter color=black,
 chapter background-color=sweet,
 chapter font-size= Huge,
 chapter font-weight=bfseries,
 chapter font-family=itshape,
 chapter before=,
 chapter spaceout=none,
 chapter after=,
 chapter margin-left=0cm,
 chapter margin-top=0pt,
 %
 chapter border-width=2pt,
 chapter border-top-width=1pt,
 chapter border-right-width=1pt,
 chapter border-bottom-width=1pt,
 chapter border-left-width=4pt,
% 
 chapter padding-left=20pt,
 chapter padding-right=20pt,
 chapter padding-top=20pt,
 chapter padding-bottom=10pt,
  %chapter title
 title font-family=rmfamily,
 title font-color=black!80,
 title font-weight=bfseries,
 title font-size=huge,
 chapter title align=none,
 title margin-left=1cm,
 title margin bottom=1.3cm,
 title margin top=30pt,
 % title borders
 title border-width=0pt,
 title padding=0pt,
 title border-color=black!80,
% title border-top-color=spot!50,
% title border-top-width=20pt,
 title border-left-color=black!80,
 title border-left-width=2pt,
 title border-color=black!80,
 title padding-top=10pt,
 title padding-bottom=10pt,
 title padding-left=10pt,
 title padding-right=0pt,
% title border-right-color=spot!50,
% title border-right-width=20pt,
% title border-bottom-color=spot!50,
% title border-bottom-width=20pt,
 %
 chapter title align=left,
 chapter title text-align=left,
 chapter title width=0.8\textwidth,
 title before=,
 title after=,
 title display=block,
 title beforeskip=12pt,
 title afterskip=12pt,
 author block=false,
 section font-family=rmfamily,
 section font-size=LARGE,
 section font-weight=bfseries,
 section indent=0pt,
  section font-weight=mdseries,
 section align=left,
 subsubsection font-family=tiresias,
 subsubsection font-shape=upshape,
 subsubsection font-weight=mdseries,
 subsubsection align=flushleft,
 epigraph width=\dimexpr(\textwidth-2cm)\relax,
 epigraph align=center,
 epigraph text align=center,
 epigraph rule width=0pt,
 header style=plain}}
 
\cxset{style87}
\renewsection\renewsubsection\renewsubsubsection

\makeatletter
\cxset{enumerate numberingi/.is choice,
  enumerate numberingi/.code={\renewcommand\theenumi {\csname#1\endcsname{enumi}}},
  enumerate numberingii/.code={\renewcommand\theenumii {\csname#1\endcsname{enumii}}},
  enumerate numberingiii/.code={\renewcommand\theenumiii {\csname#1\endcsname{enumiii}}},
  enumerate numberingiv/.code={\renewcommand\theenumiv {\csname#1\endcsname{enumiv}}},
  enumerate labeli punctuation/.store in=\enumeratepunctuationi@cx,
  enumerate labeli/.is choice,
  enumerate labeli/brackets/.code={\renewcommand\labelenumi{(\theenumi\enumeratepunctuationi@cx)}},
  enumerate labeli/square brackets/.code={\renewcommand\labelenumi{[\theenumi\enumeratepunctuationi@cx]}},
  enumerate labeli/right bracket/.code={\renewcommand\labelenumi{\theenumi\enumeratepunctuationi@cx)}},
  enumerate label left/.store in=\enumeratelabelleft@cx,
  enumerate label right/.code=\renewcommand\labelenumi{\enumeratelabelleft@cx\theenumi\enumeratepunctuationi@cx#1},
  enumerate leftmargini/.code={\setlength\leftmargini{#1}},
  enumerate leftmarginii/.code={\setlength\leftmarginii{#1}},
  enumerate leftmarginiii/.code={\setlength\leftmarginiii{#1}},
  enumerate leftmarginiv/.code={\setlength\leftmarginiv{#1}},
  listi topsep/.store in=\listitopsep@cx,
  listi partopsep/.store in=\listipartopsep@cx,
  listi itemsep/.store in=\listiitemsep@cx,
  listi parsep/.store in=\listiparsep@cx,
  listii topsep/.store in=\listiitopsep@cx,
  listii partopsep/.store in=\listiipartopsep@cx,
  listii itemsep/.store in=\listiiitemsep@cx,
  listii parsep/.store in=\listiiparsep@cx,
  listiii topsep/.store in=\listiiitopsep@cx,
  listiii partopsep/.store in=\listiiipartopsep@cx,
  listiii itemsep/.store in=\listiiiitemsep@cx,
  listiii parsep/.store in=\listiiiparsep@cx,
}
\cxset{compact1/.style={%
  enumerate numberingi=arabic,
  enumerate numberingii=alph,
  enumerate numberingiii=alph,
  enumerate numberingiv=roman,
  enumerate labeli punctuation=.,
  enumerate label left=,
  enumerate label right=,
  enumerate leftmargini=2.2em,
  enumerate leftmarginii=2.1em,
  enumerate leftmarginiii=1.5em,
  enumerate leftmarginiv=2em,
  listi topsep=8\p@ \@plus2\p@ \@minus\p@,
  listi itemsep=0\p@ \@plus2\p@ \@minus\p@,
  listi parsep=0\p@ \@plus2\p@ \@minus\p@,
  listii topsep=0\p@ \@plus2\p@ \@minus\p@,
  listii itemsep=0\p@ \@plus2\p@ \@minus\p@,
  listii parsep=0\p@ \@plus2\p@ \@minus\p@,
  listiii topsep=0\p@ \@plus2\p@ \@minus\p@,
  listiii itemsep=0\p@ \@plus2\p@ \@minus\p@,
  listiii parsep=0\p@ \@plus2\p@ \@minus\p@,
}}
\cxset{compact2/.style={%
  enumerate numberingi=alph,
  enumerate numberingii=roman,
  enumerate numberingiii=alph,
  enumerate numberingiv=roman,
  enumerate labeli punctuation=,
  enumerate label left=(,
  enumerate label right=),
  enumerate leftmargini=2.2em,
  enumerate leftmarginii=2.1em,
  enumerate leftmarginiii=1.5em,
  enumerate leftmarginiv=2em,
  listi topsep   = 8\p@ \@plus2\p@ \@minus\p@,
  listi itemsep = 0\p@ \@plus2\p@ \@minus\p@,
  listi parsep   = 0\p@ \@plus2\p@ \@minus\p@,
  listii topsep  = 0\p@ \@plus2\p@ \@minus\p@,
  listii itemsep= 0\p@ \@plus2\p@ \@minus\p@,
  listii parsep  = 0\p@ \@plus2\p@ \@minus\p@,
  listiii topsep = 0\p@ \@plus2\p@ \@minus\p@,
  listiii itemsep= 0\p@ \@plus2\p@ \@minus\p@,
  listiii parsep  = 0\p@ \@plus2\p@ \@minus\p@,
}}

\ExplSyntaxOn
\def\setenumerate#1{
\cxset{#1}
\def\@listi{%
           \leftmargin\leftmargini
            \parsep\listiparsep@cx
            \topsep\listitopsep@cx\relax
            \itemsep\listiitemsep@cx}
            
\def\@listii{\leftmargin\leftmarginii
            \parsep\listiiparsep@cx
            \topsep\listiitopsep@cx\relax
            \itemsep\listiiitemsep@cx}
            
\def\@listiii{\leftmargin\leftmarginiii
            \parsep\listiiiparsep@cx
            \topsep\listiiitopsep@cx\relax
            \itemsep\listiiiitemsep@cx}
}


\setenumerate{compact1}
\ExplSyntaxOff
\makeatother
\MakePercentComment


\chapter[Template 87]{More on Boxes: Using packages to automate boxing calculations and the drawing of borders.}
\thispagestyle{plain}
\pagestyle{headings}
\large


The boxing of contents is such an important concept in \tex and also in typography that it is worth examining some of the available packages that can be used.

The most elaborate package is Martin Scharrer’s package \pkgname{adjustbox}. The package uses numerous keys
to draw borders, adjust spacing and margings but also for the clipping of images. At the background the package uses and extends the \pkgname{graphicx} key value system.

This package allows to adjust general \latexe material in several w+ays using a key=value interface.
It got inspired by the interface of \cmd{\includegraphics} from the \pkg{graphicx} package.
 This package also loads the \pkg{trimclip} package which code was once included in this package.


\adjustbox{frame}{This is some text}

 \subsection{Trimming and Clipping}
 
 Trimming and clipping is achieved by loading the \pkgname{trimclip}. This package forms part of the suit of packages developed by Martin Scharrer and or related to adjusting boxes and their sizes. The package allows for
 verbatim material as well. 
 
 \let\Macro\cmd
 
 The following keys allow content to be trimmed (i.e.\ the official size is made smaller, so the remaining material
 laps over the official boundaries) or clipped (overlapping material is not displayed).
 These keys come in different variants, where the lower-case keys represent the behavior of
 the corresponding \Macro\includegraphics keys. The corresponding macros (\Macro\trimbox, \Macro\clipbox, etc.)
 and environments (\env{trimbox}, \env{clipbox}, etc.) are included in the
 accompanying \pkg{trimclip} package and are explained in its manual.

 
 This key represents the original \option{trim} key of \Macro\includegraphics but accepts its value in different forms.
 Unlike most other keys it always acts on the original content independent in which order it is used with other keys.
 The key trims the given amounts from the lower left (ll) and the upper right (ur) corner of the box. This means that
 the amount \meta{llx} is trimmed from the left side, \meta{lly} from the bottom and \meta{urx} and \meta{ury} from the
 right and top of the box, respectively.
 If only one value is given it will be used for all four sites.
 If only two values are given the first one will be used for the left and right side (llx, urx) and the second for the
 bottom and top side (lly, ury).
 
\begin{texexample}{Example}{clipping}
\adjustbox{Clip=1, min width=8cm, center,}{This is some test}
\medskip

The untrimmed version is shown below
\medskip

\adjustbox{Clip=.1, min width=8cm, center,}{This is some test}%
\end{texexample}
 
 
 