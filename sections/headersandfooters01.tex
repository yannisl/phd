%% DESIGNING HEADERS AND FOOTERS  **************************


\chapter{Running Titles and Paging}

\section{Introduction}
\thispagestyle{plain}

Early printed books had no running title or paging figures. The first attempt to satisfy this need of the reader was to repeat the number of the chapter at the head of each page.\footnote{De Vinne, pg 142.}  As books and styles evolved, if the words of the running title or chapter began appearing together with the page number. Practical considerations regarding the wearing of plates, school-books and all works that were printed frequently had running titles in capitals of light-faced antique. 

\begin{figure}[htp]
\includegraphics[width=\textwidth]{./images/beauty-and-art-spread.jpg}
\end{figure}

Headlines are those headings which, appearing as a rule at the top of the page, do not introduce a new part of the book
but repeat a eading or the title of the book itself. In a reference book they indicate the contents of each opening, by repeating the book and part or chapter titles, or the part
and chapter titles, or the chapter and section titles, or any combination
the author or editor or typographer may prefer.\footcite{williamson1956}

He saw the main reason for headlines that in a book designed for
sustained reading, headlines help the reader to find his place on taking up the book after an interruption. Headlines may be the same in content and
style on facing pages, or may differ from recto to verso.

The \emph{running headline} is usually understood to mean a headline that
runs unchanged throughout the book, and which normally consists of the
book's title. It is worth using only for books of permanent value. These
may have to be re-bound over and over again by libraries, and may be kept
for scores of years by private owners. Running headlines minimize the
risk of mislaying pages or sections when numbers of books are broken up
together for rebinding, or disintegrate on the shelf; they do not assist the
act of reading.\footcite{williamson1956}


The \emph{section headline} consists of the titles of the book's subdivisions,
whether chapters, parts, books, or other divisions. A headline which
consists of the titles of subdivisions of chapters might be described as a
subsection headline; in using this kind of headline, some care may be
necessary to prevent its appearing directly over the subsection heading.

The \emph{page headline} refers only to the text which appears below it, and
so has to be written by the author in page proof. Since it may require an
extra stage of proof, delays the correction of proofs in page, and adds to
the cost of composition, it is useful only when essential. In works of
reference generally it may be valuable, but usually section headlines, perhaps
reinforced by dates in historical books, are adequate.

\section{Headline Styles}

The style in which headlines are set is a matter of choice. They should
be distinct not only from the text itself but from any parts of the text;
italic upper and lower-case of text size, for instance, should not be used
for a text which contains much italic, whether in extracts or in single
phrases. Since headlines are subordinate to the text, they should not be
more emphatic ; capitals of the text fount are quite big enough for, say,
books larger than demy 8vo (since the size of the headline should be

related to the area as well as to the fount of the text), and for demy 8vo and
smaller books, capitals smaller than those of the text fount are to be
preferred.

When the headlines on facing pages differ in importance, as when there
is a running and a chapter headline, they are sometimes set in different
styles, such as small capitals for one, and italic upper and lower-case for
the other. This is a logical arrangement, though it unbalances the facing
pages and may complicate the typography of the opening.

In position the headline should not be too widely separated from the
text, in case it looks isolated ; 6 points of space is usually quite enough.
Conventionally the headline is placed over the text, but in an unconventionally
designed book it could fulfil its purpose just as well if tucked away
at the foot of the page, and would be less conspicuous than at the head.
The most popular lateral positions for the headline are centred in the text
measure or flush with the inner margin ; the position and style of setting
of the headline, however, are among the points at which unconventional
typography tends to depart from custom.

Headlines may have a certain decorative value, relieving and setting off
a solid page of text, if for example well-designed small capitals are carefully
letter-spaced, or if a particularly attractive upper and lower-case
italic is used. No additional ornament is necessary, but in the more fanciful
kind of book the typographer may wish to decorate the headline with a
fleuron or rule or strip of border ; before he does so he should consider
whether this ornament will retain its charm after appearing on scores or
hundreds of pages.

Almost every type of design has been adopted by typographers and book designers; sometimes the text is centered and in other cases it is set flush up to the inner or outer margin of the facing pages. The book chapter and the section of the book is sometimes specified in the running title, the chapter name on the left and the section on the right. When the running title consists of the name of the book, it was sometimes divided so that one half only of this name would appear on one page and the other half on the facing page. De Vrinde was highly critical of such practices and remarked `Nor is this a commendable fashion, for a line of many words can seldom be evenly divided; if it is not so divided, one heading will be longer than the other.’  Some modern books that follow in a similar fashion would place the chapter label and number at the left and the chapter title on the right. 

I am unsure if repeating the name of the book in its running title has any benefits to the reader, especially if the name of the book is well known to the reader. This title is most useful when it explains or to some extend defines the matter on the page, and this explanation should refer not to the first but to the last paragraph on that page.  Many authors prefer to not have sections in chapters and in such cases running the book name in the header rather than having left and right headers that just repeat the chapter name is preferable. An example of this is Tufte’s \textit{Beautiful Evidence}.  Tufte’s books do not have any footer material.  Many specialist scientific books have multi-authors, sometimes the running head includes the authors name (See figure from ). This particular illustration also shows the use of rules. Traditionally the rules were applied to protect the top of the block from mechanical wear during printing. 

\begin{figure}[hb]
\includegraphics[width=\textwidth]{./images/headers/header-humidification-odd.jpg}
\includegraphics[width=\textwidth]{./images/headers/header-humidification-even.jpg}
\end{figure}


As a rule,  paging with arabic figures begins with the text of the book. The matter before the text (as the title, preface, introduction, etc., which are printed last of all) is paged with roman lower-case numerals. Appendices, indices and all additions to the text take arabic figures for paging, but publisher’s advertisements at the end of the book should receive their special paging in a figure of a different face. Maps, portraits, and illustrations made on separate pages never receive printed paging, although they may be reckoned as pages in the table of contents or the index. 
\begin{figure}[htb]
\includegraphics[width=\textwidth]{./images/headers/architect.jpg}
\caption{The headers here, have a background shading.}
\end{figure}

\begin{figure}[htb]
\includegraphics[width=\textwidth]{logic.jpg}
\caption{The headers shown here include small dotted rules, running to the outer page end. This type of header can be build by adding properties and inheriting the properties of other headers.}
\end{figure}

\begin{figure}[htb]
\hskip-.1\textwidth\includegraphics[width=1.2\textwidth]{./images/headers/tulip-01.jpg}

\vspace*{1cm}

\hskip-.1\textwidth\hbox to 0pt{\includegraphics[width=1.2\textwidth]{./images/headers/tulip-02.jpg}}
\caption{The headers here, have a background shading.}
\end{figure}

\begin{figure}[htp]
\includegraphics[width=1\textwidth]{./images/headers/small-flash-01.jpg}

\vspace*{1cm}
\includegraphics[width=1\textwidth]{./images/headers/small-flash-02.jpg}

\caption{The headers here, have a background shading.}
\end{figure}


\begin{figure}[htp]
\includegraphics[width=1\textwidth]{./images/headers/power-and-politics-01}

\vspace*{1cm}
\includegraphics[width=1\textwidth]{./images/headers/power-and-politics-02}

\caption{The headers here, have a background shading.}
\end{figure}

\begin{figure}[htp]
\includegraphics[width=1\textwidth]{./images/headers/economic-warfare-01}

\vspace*{1cm}
\includegraphics[width=1\textwidth]{./images/headers/economic-warfare-02}

\caption{The headers here, have a background shading.}
\end{figure}


\section{The Requirements}

The brief discussion above and the examples from various publications can help in definind the final requirements of what we are about to program. The header or the footer for that matter as they are very similar needs to communicate with the page that is currently being processed to obtain the page number and any other marks that need to go into the running head.

\begin{tabular}{>{\raggedright}p{5cm}l}
Access to the page number & \\
Build up string from sections, chapters, titles or subtitles &\\
Distingusih between left and right numbers &\\
Add user data  &\\
Provide an intuitive user interface&\\
\end{tabular}

A more modern approach would be to offer a small templating language to deal with the headers and footers. This is for example, now common in web applications where variables are sent by the server to the web page being build and transformed in templates.

Another approach is to use a graphical language, such as metapost.

Since we have to deal with odd and even pages and a header and or a footer, the minimum variables needed to hold this information is four. 

A graphicablock can also happily contain the necessary information.

The algorithm is described below:

\begin{enumerate}
\item Set the variable headerleft and headerright to indicate one page or two page printing.
\item Define text block templates as macros to set the typesetting to a named style. Each header style
         will have its own name. Standardize parameters to enable easy redefinition of commands. As a 
         final fully flexible approach the key header = custom will provide full capabilities for any user
         defined design.
\item Distinguish how headers and footers will be typeset on title pages, chapter openings, bibliographies, 
         automatically generated pages, such as float pages etc.
\item Hook into LaTeX’s output routine to obtain information about the top and bottom inserts and other marks.         
\item Inherit properties, such as language and directionality.
\item Provide less intrusive ways to define different styles by the user.
\item define block commands to mark start of different headings for example |\mainmatter|. This will define
         the start of the main text of the publication and issue a command to process the pages that follow.
\end{enumerate}

A special type of header is something that will be repeated on every page, say a watermark of some sorts. These are dealt as backgrounds.
 
\section{Traditional LaTeX page style commands}
  
One of the first tasks of any \LaTeXe\ class is to redefine the headers and footers. The format of the running headers or footers in \LaTeX\ terminology is called the \textit{page style}. Each different format is given names like \textit{empty} or \textit{plain} to make it easier to select and remember. 

\begin{figure}[hbt]
\includegraphics[width=\textwidth]{./images/headers/Running-heads-lace.png}
\caption{This last example shows what kind of atmosphere you can create with running heads. Here a bit of lace texture has been softened and graduated, creating a kind of gentle, suggestive frame around these pages. I’ve also used line drawings, logos, and other graphic elements to dress up running heads like these. From the \protect\href{bookk  }{bookdesigner.com}}
\end{figure}


The LaTeX kernel\footnote{In File J file{ltpage.dtx}, page 311.} defines two commands for selecting the running heads:

\begin{lstlisting}
\pagestyle{<style>} : sets the page style of the current and succeeding pages to style
\thispagestyle{<style>} : sets the page style of the current page only to style.
\end{lstlisting}

\section{Traditional LaTeX page style definition}

To define a page style \textit{style}, you must define the \lstinline{\ps@style} to set the page parameters.

\subsection{How a page style makes running heads and feet}
The \lstinline{\ps@}. . . command defines the macros \lstinline{\@oddhead}, \lstinline{\@oddfoot}, \lstinline{\@evenhead},
and \lstinline{\@evenfoot} to define the running heads and feet. (See output routine.) As some headings contain information such as the chapter name or section number these
headings are based on the sectioning commands, which define them. The page style defines the commands




\verb!\chaptermark,\sectionmark!, etc., where

\verb+\chaptermark{<text>}+ is called by \verb+\chapter+ to set a mark. The  ...mark commands and the ...head
macros are defined with the help of the following macros.
%(All the \ ...mark commands should be initialized to no-ops.)



\subsection{marking conventions}

LaTeX produces two kinds of marks a `left' and a `right' mark using the following commands.

markboth

markright



\section{The low level page style interface}

The basic mechanics of defining page styles is provided in the \LaTeXe\ kernel and it  involves defining or redefining four macros:

\begin{marglist}
\item [\docAuxCommand{oddhead}] For two-sided printing, it generates the header for the odd-numbered
pages; otherwise, it generates the header for all pages.

\item [\cs{oddfoot}] For two-sided printing, it generates the footer for the odd-numbered pages; otherwise, it generates the footer for all pages.

\item [\cs{evenhead}] For two-sided printing, it generates the header of the even-numbered
pages; it is ignored in one-sided printing.

\item [\cs{evenfoot}] For two-sided printing, it generates the footer of the even-numbered
pages; it is ignored in one-sided printing.

\end{marglist}
A named page style, involves the redefinition of these commands stored in a macro \cs{ps@<style>}.
The \cs{pagestyle}\marg{plain} is defined as:



%\begin{tcolorbox}
%\begin{lstlisting}
%\newcommand\ps@plain{%
%  \renewcommand\@oddhead{}%
%  \let\@evenhead\@oddhead
%  \renewcommand\@evenfoot{%
%  {\hfil\normalfont\textrm{\thepage}\hfil}}%
%  \let\@oddfoot\@evenfoot
%}
%\end{lstlisting}
%\end{tcolorbox}

Since the \textit{plain} style treats both the odd and even pages the same way, the \cs{@evenfoot} and \cs{@evenhead} are let to the \cs{@oddhead} and \cs{@oddfoot} commands. The style only prints a page number at the center of the footer.


\subsection{A longer example}

\index{watermark}\index{water mark!sample page style}
\thispagestyle{samplepage}
Consider the case, where we need to print on a page the words \textsc{sample page}, as you might have noticed in some places of this document and at the margin of this page. Sometimes this type of mark is called a \textit{watermark.}

We will call this type of page style \textit{samplepage} and we will activate it on a particular page by typing \cs{thispagestyle}\marg{samplepage}.




%\begin{tcolorbox}
%\begin{lstlisting}
%%% Some special styles
%\IfFileExists{rotating.sty}{\RequirePackage{rotating}}{}
%
%\def\even@samplepage{%
% \begin{picture}(0,0)
%   \put(\Xeven,\Yeven){\turnbox{90}{\Huge \textcolor{\watermark@textcolor}{\watermark@text}}}
%\end{picture}
%}
%
%\def\odd@samplepage{%
% \begin{picture}(0,0)
%   \put(\Xodd,\Yodd){\turnbox{90}{\Huge \textcolor{\watermark@textcolor}{\watermark@text}}}
% \end{picture}
%}
%
%\def\watermarktext#1{\gdef\watermark@text{\fontfamily{phv}\selectfont#1}}
%\def\watermarktextcolor#1{\gdef\watermark@textcolor{#1}}
%\watermarktext{SAMPLE PAGE}
%\watermarktextcolor{purple}
%
%\def\ps@samplepage{\let\@mkboth\@gobbletwo
% \let\@oddhead\odd@samplepage\def\@oddfoot{\reset@font\hfil\thepage}
% \let\@evenhead\even@samplepage\def\@evenfoot{\reset@font\thepage\hfil}}
%
%\def\Xodd{500}
%\def\Xeven{-70}\def\Yeven{-810}
%\def\Yeven{-\expandafter\strip@pt\textheight}
%\let\Yodd\Yeven
%\end{lstlisting}
%\end{tcolorbox}

If you study the code in the example, you will notice that we are using \LaTeXe's \env{picture} environment to
place the text exactly where we need it. This is one way of absolutely positioning text on a page, another way is to use |pgf|’s absolute positioning methods.




\subsection{The key value interface}

The key value interface provides a number of mechanisms to tap into the page styles, enabling consistency in the user interface.

\medskip

\keyval{header style}{\marg{text}}{Triggers a page style for one page only.} The following values can be used.

\begin{marglist}
\item [empty] Standard class empty headers.
\item [plain] Standard class plain headers.
\item [headings] Standard class headings.
\item [fancy] If you use the fancyhdr package any fancy header style.
\item [sample page] Prints sample at the edge of the paper.
\item [preprint] Prints preprint at the edge of the paper.
\item [watermark] Prints a watermark at predefined places.
\end{marglist}

\keyval{watermark}{\marg{true|false}}{Prints a watermark on all pages, defaults to false.}
\keyval{watermark text}{\marg{text}}{The watermark text.}
\keyval{watermark text left}{\marg{text}}{The watermark text on left pages.}
\keyval{watermark text right}{\marg{text}}{The watermark text on right pages.}
\keyval{watermark angle}{\marg{number}}{The rotation angle of the water mark}




%\cxset{ watermark text/.store in=\watermark@text,
%           watermark text color/.store in=\watermark@textcolor,
%           watermark font-size/.store in=\watermarkfontsize@cx,
%           watermark odd x/.store in=\watermarkoddx@cx,
%           watermark even x/.store in=\watermarkevenx@cx,
%           watermark even y/.store in=\watermarkeveny@cx}
%
%\cxset{watermark text= PRE-PRINT,
%          watermark text color=theblue,
%          watermark font-size=\huge,
%          watermark odd x=470,
%          watermark even y=700,
%          watermark even x=60}
%
%\def\Xodd{\watermarkoddx@cx}
%\def\Xeven{-\watermarkevenx@cx}
%\def\Yeven{-\watermarkeveny@cx}
%%\def\Yeven{-\expandafter\strip@pt\textheight}
%\let\Yodd\Yeven
%
%\def\even@samplepage{%
% \begin{picture}(0,0)
%   \put(\Xeven,\Yeven){\turnbox{60}{\watermarkfontsize@cx \textcolor{\watermark@textcolor}{\watermark@text}}}
%\end{picture}
%}
%
%\def\odd@samplepage{%
% \begin{picture}(0,0)
%   \put(\Xodd,\Yodd){\turnbox{90}{\watermarkfontsize@cx\textcolor{\watermark@textcolor}{\watermark@text}}}
% \end{picture}
%}


. 



\subsection{Using the headings as hooks}

Since the headings are added to the page during processing of the output routine, they are sometimes used
to insert material on the page at places other than the head, through the use of a zero width box. For example we
can use this approach to add a watermark on a page. Other approaches to position material at absolute positions
on a page, is to hook at \emph{shipout}. Some packages such as TikZ can also be used through the |remember picture, overlay |  key settings. 

The |phd| package has a predefined style, named samplepage that can be used to typeset some text at the outer margin of a page. The text is configurable and you can set it for example to typeset “PRE-PRINT” rather than the “SAMPLE PAGE” string. 

\begin{tcolorbox}
\begin{lstlisting}
\cxset{
     watermark text= PRE-PRINT,
     watermark text color=theblue,
     watermark font-size=\huge
}
\end{lstlisting}
\end{tcolorbox}

\makeatletter
\cxset{watermark text/.code =\watermarktext }
\makeatother

\watermarktext{PRE-PRINT}
   
\thispagestyle{samplepage}


\section{Adding marks}

Most books will have headers that include marks such as the chapter name and number and or other combinations together with section numbers.

The standard book class include two styles one called \textit{headings} and another called \textit{myheadings} that style such headers.


\section{What are Marks?}

Let me now define marks in TeX. The purpose of marks in \tex is to generate 
the text in running heads, where the tex in these running head is based on the 
titles of chapters, headings of sections, and so forth. I will first explain the mark 
mechanism of \tex, before I show how the marking mechanism of \tex is applied 
to typeset running heads in this series. 

Marks are in a \mark{similar} to \docAuxCommand{specials} in the sense that the 
associated information is not typeset, but is accessible in a different way. In the 
case of marks this information is not written to the dvi file but accessible using 
\docAuxCommand{botmark}, \docAuxCommand{topmark} and \docAuxCommand{firstmark}



\subsection{Generation of a Mark Using \textbackslash mark}
 
The primitive |\mark| is used to inform TeX about the text of a mark. For instance, 
|\mark|\marg{This is fun} records the given text as a mark. What such a |\mark| is for 
will be discussed shortly. Notice that the text of a |\mark| is expanded right when 
the |\mark| is called, not when the text is extracted from |\topmark|, |\firstmark| 
or |\botmark| (see below). 

\begin{texexample}{Exploring marks}{ex:marks}
\meaning\mark

\meaning\topmark

\meaning\firstmark

\meaning\botmark
\end{texexample}



