\chapter{Environments}
\label{ch:environments}

\section{Defining Environments}
Environments are used to format blocks of text in \latex2e documents. 
Knuth's design of macro definition commands makes use of delimited arguments. Using this one could device a way to create blocks of text enclosed in an environment.\tcbdocmarginnote{1 Jun 20018. \\New example\\ added.}

\begin{texexample}{Knuthian Environments}{ex:knuthianenvironments}
\long\def\startenvironment#1\stopenvironment{
\bgroup
 \hsize=\linewidth 
 \vskip10pt
  \centering 
  \leavevmode #1\par
 \egroup 
}
% use the environment
\startenvironment
\lorem
\stopenvironment
\end{texexample}

Defining environments this way, is not recommended, as \latexe provides a consistent and familiar interface to users. The environment definition commands also provided by \latexe are more sophisticated, check that the command name has not been defined earlier, as well as carry out a number of other usefull house-keeping tasks.

To define the same environment as use in \ref{ex:knuthianenvironments} we will use the following:


\begin{texexample}{LaTeX Environments}{ex:ltxenvironments}
\newenvironment{loremenvironment}[1]{
   \vskip10pt
 
  \centering 
  \leavevmode #1\par
 }% 
{}
% use the environment
\begin{loremenvironment}
 \lorem
\end{loremenvironment}
\end{texexample}








\subsection{The \protect\latex way}
LaTeX implements macros |\begin| and |\end|. These are a generic pair whose argument determines the environment that's being begun or ended.

LaTeX makes it much easier to code environments. Here's a generic environment definition:

\begin{docCommand}{newenvironment}{\marg{environment name}\marg{stuff to do before text} \marg{stuff to do after text}}
Creates a new environment
\end{docCommand}


The \docAuxCommand{newenvironment} macro takes three arguments:

\begin{enumerate}
\item The name of the environment being created
\item The stuff to do before the text being assigned that environment
\item The stuff to do after the text being assigned that environment
\end{enumerate}

The resemblance to \tex paired macros is obvious, but \latex  environments make it generic across all environments, and place the beginning and ending code in one place. Not only that, but because the environment has one name instead of two different names, it's very easy for a front end like LyX to assign environments to highlighted stretches of text.

The |\newenvironment|  macro works only when the environment name is undefined. If there's already an environment with that name, use \docAuxCommand{renewenvironment} instead. 

The \pkg{phd} package defines a number of environments for general use:

\begin{docEnvironment}{absolutequote}{}
\end{docEnvironment}

The Example~\ref{ex:absolutequote} illustrates the redefinition of this environment to set its margins a bit more in.

\begin{texexample}{Absolute quote example}{ex:absolutequote}
\makeatletter
\renewenvironment{absolutequote}
               {\list{}{\listparindent 1.5em%
                  \itemindent \listparindent
                  \leftmargin2cm\rightmargin\leftmargin
                  \parsep 0pt plus 0pt%
                  }%
                  \item\relax\footnotesize
                }%
               {\endlist}               
\makeatother  
\lipsum[1]
\begin{quotation}
\lipsum[3]
\end{quotation}
\lipsum[1]

\end{texexample}

\subsection{Using xparse}

With the \pkg{xparse} from the \latex3 suite, we can use the |NewDocumentEnvironment| to define environments. This is very useful, if we have a mixture of optional and mandatory commands, as well as when we need to access the end code.

\begin{docCommand}{NewDocumentEnvironment}{\marg{environment}\marg{arg spec}\marg{start code}\marg{end code}}
Creates new environments.
\end{docCommand}

An advantage of the command is that both the \meta{start code}
and \meta{end code} can access the arguments as defined by \meta{arg spec}. You can learn more about these commands in the LaTeX3 Monograph. 


\begin{texexample}{Absolute quote example}{ex:absolutequote2}
\makeatletter
\NewDocumentEnvironment{absolutequotation}{ o }
               {\list{}{\listparindent 1.5em%
                  \itemindent \listparindent
                  \leftmargin2cm\rightmargin\leftmargin
                  \parsep 0pt plus 0pt%
                  }%
                  \item\relax\footnotesize
                }%
               {\endlist}               
\makeatother  
\lipsum[1]
\begin{absolutequotation}
\lipsum[3]
\end{absolutequotation}
\lipsum[1]

\end{texexample}

Lamport, cleverly defined macros that automatically, create the necessary \tex \cs{begingroup} and \cs{endgroup} commands. You can find the code in the |source2e| file and which is shown below:\footnote{You can find the full code in \texttt{File y, for ltmiscen.dtx}}

\begin{teXXX}
\def\begin#1{%
  \@ifundefined{#1}%
  {\def\reserved@a{\@latex@error{Environment #1 undefined}\@eha}}%
  {\def\reserved@a{\def\@currenvir{#1}%
  \edef\@currenvline{\on@line}%
  \csname #1\endcsname}}%
  \@ignorefalse
  \begingroup\@endpefalse\reserved@a}


\def\end#1{%
  \csname end#1\endcsname\@checkend{#1}%
  \expandafter\endgroup\if@endpe\@doendpe\fi
  \if@ignore\@ignorefalse\ignorespaces\fi}
\end{teXXX}

In \latex environments are defined as 
|\begin{foo}| and |\end{foo}| which are are used to delimit environment |foo|.
|\begin{foo}| starts a group and calls |\foo| if it is defined, otherwise it does
nothing.

|\end{foo}| checks to see that it matches the corresponding |\begin| and if so,
it calls |\endfoo| and does an |\endgroup|. Otherwise, |\end{foo}| does nothing.
If |\end{foo}| needs to ignore blanks after it, then |\endfoo| should globally set
the |@ignore| switch true with |\@ignoretrue| (this will automatically be global).

NOTE: |\@@end| is defined to be the |\end| command of TEX82.
|\enddocument| is the user's command for ending the manuscript file.
|\stop| is a panic button to end \tex in the middle.


\section{Checking the environment}

An interesting aspect of \latexe is that we can use \docAuxCommand{@currenvir} to check if a command is within a particular environment. The following code will be used to typeout the environment.

\begin{texexample}{Checking for an Environment}{ex:currenenvir}
\makeatletter
The current environment is \@currenvir
\makeatother
\end{texexample}


The \cs{@checkend} \index{Latex kernel=@checkend} uses the \cs{@currenvir}\index{Latex kernel=@currenvir} to see if there is a matching
begin environment and if it cannot find it produces an error.

\begin{teX}
\def\@checkend#1{%
   \def\reserved@a{#1}
   \ifx\reserved@a\@currenvir 
   \else
     \@badend{#1}
   \fi
}
\end{teX}

It is a pity that there is no real guide for explaining the \latex macros, other than just reading through them. Lamport and later his associates managed to produce code that offers the user a friendly API. Besides the scenes of this API, it also offers the package writers hundreds of useful commands.

\endinput
\newlength{\egwidth}\setlength{\egwidth}{0.48\textwidth}

\newenvironment{ega}%
{\begin{list}{}{\setlength{\leftmargin}{0.02\textwidth}%
\setlength{\rightmargin}{\leftmargin}}\item[]\footnotesize}%
{\end{list}}

\newenvironment{egbox}%
{\begin{minipage}[t]{\egwidth}}%
{\end{minipage}}

\newcommand{\egstart}{\begin{ega}\begin{egbox}}
\newcommand{\egmid}{\end{egbox}\hfill\begin{egbox}}
\newcommand{\egend}{\end{egbox}\end{ega}}

% one or two other commands

\newcommand{\fn}[1]{\hbox{\tt #1}}
\newcommand{\llo}[1]{(see line #1)}
\newcommand{\lls}[1]{(see lines #1)}


\egstart
Here is some advice to remember:
\begin{quotation}
Environments for making
...other things as well.
Many problems
...environments.
\end{quotation}
\egmid%
Environments for making quotations
can be used for other things as well.
Many problems can be solved by
novel applications of existing
environments.
\egend


The \cs{tabbing} environment overcomes this problem. Within it you set
tabstops and tab to them much like you do on a typewriter.  Tabstops are
set with the |\=| command, and the |\>| command moves to the
next stop.  The
|\\| command is used to separate each line.  A line that ends |\kill|
produces no output, and can be used to set tabstops:


\begin{texexample}{Tabbing Example}{ex:tabbing}
\begin{tabbing}
 Income \=Expenditure \= \kill
 Income \>Expenditure \>Result\\
 20s 0d  \>19s 11d \>Happiness\\
 20s 0d  \>20s 1d  \>Misery \\
\end{tabbing}
\end{texexample}



Unlike a typewriter's tab key, the |\>| command always moves to the next
tabstop in sequence, even if this means moving to the left.  This can cause
text to be overwritten if the gap between two tabstops is too small.
