%&program=xetex
%&encoding=UTF-8 Unicode

% Arabic text from http://www.unicode.org/standard/translations/arabic.html
%
% Uses Geeza Pro font (optional install with Mac OS X), or Scheherazade if available
%

%&encoding=UTF-8 Unicode

% Macros to try and find available fonts for XeTeX sample docs
%
% Usage:
%
% \testFontIsAvailable{font-name}
%   sets \ifFontIsAvailable according to whether or not it could be found
%
% \FindAnInstalledFont{font-name/alternative/another/yet-another}{\cs}
%   searches for an available font from among the names given,
%   and \def's the control sequence \cs to the first one found
%   or to <No suitable font found> if none (which will subsequently
%   cause an error when used in a \font command). 

\newif\ifFontIsAvailable
\def\testFontAvailability#1{
  \count255=\interactionmode
  \batchmode
  \let\preload=\nullfont
  \font\preload="#1" at 10pt
  \ifx\preload\nullfont \FontIsAvailablefalse
  \else \FontIsAvailabletrue \fi
  \interactionmode=\count255
}

\def\FindAnInstalledFont#1#2{
  \expandafter\getFirstFontName#1/\end
  \let\next\gobbleTwo
  \ifx\trialFontName\empty
    \def#2{<No suitable font found>}%
  \else
    \testFontAvailability{\trialFontName}
    \ifFontIsAvailable
      \edef#2{\trialFontName}%
    \else
      \let\next\FindAnInstalledFont
    \fi
  \fi
  \expandafter\next\expandafter{\remainingNames}{#2}
}
\def\getFirstFontName#1/#2\end{\def\trialFontName{#1}\def\remainingNames{#2}}
\def\gobbleTwo#1#2{}

\TeXXeTstate=1
\nopagenumbers \frenchspacing
%\input testfontavailability
\FindAnInstalledFont{Geeza Pro Bold/Geeza Pro/Scheherazade/Arabic Typesetting/}{\titlefont}
\font\title="\titlefont" at 18pt
\font\heading="\titlefont" at 18pt
\FindAnInstalledFont{Scheherazadex/Geeza Pro}{\bodyfont}
\font\body="\bodyfont" at 15pt \body
\font\romfont="Times New Roman" at 12pt \def\rom#1{{\beginL\romfont #1\endL}}
\parindent=0.5in \baselineskip=22pt \lineskiplimit=-1000pt

\def\s#1{\bigskip \rightline{\beginR\heading #1\endR}\nobreak\medskip}

\centerline{\beginR\title ما هي الشفرة الموحدة يونِكود؟\endR}
\everypar={\setbox0=\lastbox \beginR \box0 }
\bigskip
الأخرى بعد أن تُعطي رقما معينا لكل أساسًا، تتعامل الحواسيب فقط مع الأرقام، وتقوم بتخزين الأحرف والمحارف كان هناك مئات الأنظمة للتشفير وتخصيص هذه الأرقام للمحارف، ولم يوجدواحد منها. وقبل اختراع يونِكود، فإن الاتحاد الأوروبي لوحده، احتوى نظام تشفير واحد يحتوي على جميع المحارف الضرورية. وعلى سبيل المثال، جميع اللغات المستخدمة في الاتحاد. وحتى لو اعتبرنا لغة واحدة، كاللغةالعديد من الشفرات المختلفة ليغطي الترقيم والرموز الفنية والعلمية الإنجليزية، فإن جدول شفرة واحد لم يكف لاستيعاب جميع الأحرف وعلامات الشائعة الاستعمال.

وبعبارة أخرى، يمكن أن يستخدِم جدولي وتجدر الملاحظة أن أنظمة التشفير المختلفة تتعارض مع بعضها البعض. حاسوب، مختلفين، أو رقمين مختلفين لتمثيل نفس المحرف. ولو أخذنا أي جهاز شفرة نفس الرقم لتمثيل محرفين  وبخاصة جهاز النادل (\rom{server}) على التعامل مع عدد كبير من الشفرات المختلفة، ويتم، فيجب أن تكون لديه القدرة خطورة لضياع أو تحريفالأساس. ومع ذلك، فعندما تمر البيانات عبر أنظمة مختلفة، توجد هناك تصميمه على هذا بعض هذه البيانات.

\s{يونِكود تغير هذا كليـا !}

العالمية، وذلك بغض النظر عن نوع تخصص الشفرة الموحدة يونِكود رقما وحيدا لكل محرف في جميع اللغات فـي تـم تبني مواصفة يونِكود مــن قبـل قادة الصانعين لأنظمة الحواسيب الحاسوب أو البرامج المستخدمة. وقد. العالم، مثل شركات آي.بي.إم (\rom{IBM})  أبـل، (\rom{APPLE})، هِيـْولِـت بـاكـرد (\rom{Hewlett-Packard})،  مايكروسوفت (\rom{Microsoft})،  أوراكِـل (\rom{Oracle})، صن (\rom{Sun})  والمقاييس الحديثة (مثل لغة البرمجة جافا وغيرها. كما أن المواصفات \rom{JAVA}  ولغة إكس إم إل \rom{XML}، فإنالتي تستخدم لبرمجة الانترنيت) تتطلب استخدام يونِكود. علاوة على ذلك إيزو يونِكود هي الطـريـقـة الرسـمية لتطبيق المقيـاس الـعـالـمي  ١٠٦٤٦   (\rom{ISO 10646}). 

تستخدمه وتدعمه، يعتبر من أهم الاختراعات الحديثة في عولمة إن بزوغ مواصفة يونِكود وتوفُّر الأنظمة التي سيؤدي إلى توفير كبير مقارنة مع لجميع اللغات في العالم. وإن استخدام يونِكود في عالم الانترنيت البرمجيات المشفرة. كما أن استخدام يونِكود سيُمكِّن المبرمج من كتابة البرنامجاستخدام المجموعات التقليدية للمحارف دولة في العالم أينما كانت، دون الحاجة مرة واحدة، واستخدامه على أي نوع من الأجهزة أو الأنظمة، ولأي لغة أو الأنظمة تعديل. وأخيرا، فإن استخدام يونِكود سيمكن البيانات من الانتقال عبر لإعادة البرمجة أو إجراء أي الصانعة للأنظمة واللغات، والدول التي تمر من والأجهزة المختلفة دون أي خطورة لتحريفها، مهما تعددت الشركات خلالها هذه البيانات.


This is some test
\bye
