\chapter{Captions}

\parindent1em

\section{Setting the caption options}

Captions are very visual and both the text as well as its typography need careful consideration. Most readers will read the captions of figures, before reading the text. We will now in the sections that follow use the caption package to change all the parameters of the caption. This is achieved mainly through one macro, with key value styles.


%\DeclareRobustCommand\acaption{\protect\RaggedRight Lorem ipsum caption \protect\ldots.}
\def\acaption{Lorem ipsum caption \ldots}
\begin{figure*}[h]
\captionsetup{format=plain}
\captionsetup{skip=3pt}
\captionsetup{font=small}
\captionsetup{name=Fig}
\captionsetup[figure]{labelfont=bf,textfont=it}
\RaggedRight
\centering 
\begin{minipage}[t]{90pt}
 \includegraphics[width= 70pt]{./graphics/sudan.jpg}
 \caption{\acaption }
 \label{fig:shortlabel}
\end{minipage}
\captionsetup{name=Figure}
\begin{minipage}[t]{90pt}
 \includegraphics[width= 70pt]{./graphics/sudan.jpg}
 \caption{\acaption }
\end{minipage}
\captionsetup{name=Fig,labelsep=space}
\begin{minipage}[t]{90pt}
 \includegraphics[width= 70pt]{./graphics/sudan.jpg}
 \caption{\acaption }
\end{minipage}
\end{figure*}


To set the caption options we can use the \cmd{\captionsetup} with a set of options.
\begin{dispListing}
\captionsetup{name=Fig, labelsep=space}
\end{dispListing}




It is highly recommended to use the \texttt{caption} package to setup the captions of figures. This package developed by Axel Sommerfeldt offers customization of captions in floating environments such
figure and table and cooperates with many other packages. Most classes provide build-in options and commands for customizing captions. 

And if you are just interested in using the
command \cmd{\captionof}, loading of the very small \pkgname{capt-of package} is usually sufficient.

For wrapped figures the label name is preferable to be shorter, otherwise it leads to text that is either underfull or overfull. You should also try and use the \cmd{\RaggedRight} option of the \pkgname{ragged2e} package to hyphenate the ragged right text.

Figure~\ref{fig:shortlabel}, has its label shortened by using ``Fig'' rather than "Figure". I have done this as the space available is narrow. The setup is achieved using the \texttt{caption} package's \verb+\captiosetup+ command. We will use this command to specify, the fonts, numbering, labels, separators and other parameters of the captions.

\subsection{Adjusting the label}%%

The \emph{label} is the name of the figure. Sometimes it is abbreviated, sometimes it is not. Adjusting the label, is achieved by setting the key parameter |name| in \cmd{\captionsetup}. 

\begin{commands}[]{}
\cmd{\captionsetup}\marg{name=Figure}
\end{commands}



The figures were typeset by using a different setup style. The first one displays the  label fully, the second uses an abbreviation and the third has a new line, before the caption text is displayed.

\subsection{Fonts}

There are three font options which affects different parts of the caption: One affecting the
whole caption (font), one which only affects the caption label and separator (labelfont) and at least one which only affects the caption text (textfont). You set them up using the options shown in the table below:

\begin{table}[htp]
\centering
\smaller
\caption{Key values for fonts, using the caption package}
\begin{tabular}{ll}
\toprule
normalfont &Normal shape\\
up &Upright shape\\
it &Italic shape \\
sl &Slanted shape\\
sc & \textsc{Small Caps Shape}\\
md &Medium series\\
bf &Bold series\\
rm &Roman family\\
sf &Sans Serif family\\
tt &Typewriter family\\
\bottomrule
\end{tabular}
\end{table}

\emphasis{captionsetup,captionof}
\begin{teXXX}
\captionsetup{name=Figure.}
\captionof{figure}{\acaption}
\end{teXXX}


\begin{figure*}[h]
\begin{commands}[]{}
\captionsetup{skip=3pt}
\captionsetup{font=small}
\captionsetup{name=Fig}
\captionsetup{labelfont=bf,textfont=it, format=plain}
\RaggedRight
\centering 
\begin{minipage}[t]{90pt}
 \includegraphics[width= 70pt]{./graphics/sudan.jpg}
 \caption{\acaption }
\end{minipage}
\captionsetup{name=Figure}
\begin{minipage}[t]{90pt}
 \includegraphics[width= 70pt]{./graphics/sudan.jpg}
 \caption{\acaption }
\end{minipage}
\captionsetup{name=Fig,labelsep=space}
\begin{minipage}[t]{90pt}
 \includegraphics[width= 70pt]{./graphics/sudan.jpg}
 \caption{\acaption }
\end{minipage}
 \caption{Three boys example (changing the figure name).}
 \end{commands}
\end{figure*}


\section{Adjusting the Separator}


The separator can be adjusted in a similar manner. The package offers the options, \option{none}, \option{colon}, \option{period}, \option{space}, \option{quad}, \option{newline} and \option{enddash}.  The various options are illustrated
in \hbox{Figures~18-23}.


\section{Adjusting spacing before and after the figure}

Skips are the amount of vertical space between the caption and the figure. The caption package offers the option
\option{skip=amount}.\footnote{The standard \LaTeX\ classes article, report and book preset it to \option{skip=10pt}.} We will now make some recommendations as to how to adjust this spacing.

\medskip

\begin{figure}[htp]
\everypar{}
\captionsetup{name=Photo,parindent=0pt,minmargin=0pt,width=3sp,labelsep=period,skip=5pt,margin={0pt,0pt},position=bottom}

\noindent\includegraphics[width=\textwidth]{./graphics/damageinspection.jpg}

\noindent\caption{Damage Inspection.A squadron operations officer of the 332d Fighter Group points out a cannon hole to ground crew, Italy, 1945.}\par
\end{figure}

\medskip

The space between the image and the caption should be approximately half the point size of the text. The photo above had the following settings:


\begin{teX}

\captionsetup{name=Photo, labelsep=period,
                    skip=5pt, font=scriptsize,
                    position=bottom, margin{0pt,0pt}}
\end{teX}

The \docAuxCommand{caption} command offered by \latexe has a design flaw\footnote{According to Axel Sommerfeldt, \textit{see} the \textit{Caption} documentation.}: The command does not
know if it stands on the beginning of the figure or table, or at the end. Therefore it does
not know where to put the space separating the caption from the content of the figure
or table. While the standard implementation always puts the space above the caption
in floating environments (and inconsistently below the caption in longtables), the
implementation offered by this package is more flexible: By giving the option
\option{position=bottom}, the package correctly inserts the skip.  You can also try the \option{position=auto}.
\medskip

The caption of the next photograph follows a more traditional approach found in
\begin{figure}[htp]
\vskip10pt
\centering
\captionsetup{name=Photo, labelsep=period, position=bottom, textfont=scriptsize, justification=centering}
\includegraphics[width=\textwidth]{./graphics/korea.jpg}

\caption*{\textsc{25th Division Troops Unload Trucks and Equipment}\par
\textit{at Sasebo Railway Station, Japan, for transport to Korea, 1950.}}
\vskip10pt
\end{figure}
many books where, there is no label or number and the text is split into two lines. The first line is a photograph heading and the second line is printed in italics with some explanatory stuff about the photo.

To achieve this result we need to firstly use the \emph{starred} form of the caption command and override the formatting commands of the caption.

\begin{teX}
\begin{figure}[htp]
\vskip10pt
\centering
\captionsetup{...}
\includegraphics[width=\textwidth]{filename}
\caption*{\textsc{25th Division Troops Unload Trucks and Equipment}\par
\textit{at Sasebo Railway Station, Japan, for transport to Korea, 1950.}}
\vskip10pt
\end{figure}
\end{teX}

You will notice that the photograph is between the lines of the paragraph, so I have added some small skips to arrange proper spacing around it.


To my knowledge, you cannot customize the caption package to get the heading for the caption text. You can define your own command to do so:
\begin{phdverbatim}
\newcommand\captionx[2]{\par%
     \leavevmode 
     \caption*{\textsc{#1}\par%
     \textit{#1}}%
}
\end{phdverbatim}

\DeclareDocumentCommand\captionx{m m}{%
     \leavevmode
     \caption{\textsc{#1} %
     \textit{#2}}%
}

With photographs you need sometimes to add a "credit" to credit the photographer or even a copyright notice. This is necessary, especially if you have licensed images from an agency. For this I would prefer a simple solution where we
just define an \verb+addcredit+ macro. More customization might be possible, as well as a few setup macros. As an exercise have a look at some publications and see how they handle this type of photographs.

\begin{teX}
\newcommand\addcredit[1]{%
   \vspace*{-10.5pt}%
   \scriptsize
   \hfill\hfill
   \textit{Credit: #1}%
}
\end{teX}

\providecommand\addcredit[1]{%
 \scriptsize%
 \vspace*{-10.5pt}%
 \hfill\hfill\textit{Credit: #1}%
 \vspace{10pt}
}

The results of the code so farm can be seen in the photograph that follows. The credit has been added and
the text has been centered and styled as required.

The full code is now shown below:

\begin{teX}
\begin{figure}[htp]
  \centering
  \captionsetup{skip=0pt,  justification=centering}%
  \includegraphics[width=\textwidth]{./graphics/rosenberg.jpg}%
  \addcredit{U.S. DoD.}%
  \captionx{Assistant Secretary Rosenberg}{talks ...}
\end{figure}
\end{teX}

\begin{figure}[htp]
  \centering
  \captionsetup{name=Photo, labelsep=period, skip=0pt, position=top, textfont=scriptsize,    justification=centering}%
\includegraphics[width=\textwidth]{./graphics/rosenberg.jpg}%
\addcredit{U.S. DoD.}%
\captionx{Assistant Secretary Rosenberg}{talks with men of the 140th Medium Tank Battalion during a Far East tour.}
\vspace{10pt}
\end{figure}

It all looks perfect, but there is a snag. If the photo is narrower, there will be nothing to stop it floating past the edge of the photo. This can be corrected by enclosing the commands within a minipage.


\begin{figure}[htp]
\begin{commands}[]{}
\captionsetup{name=Fig., labelsep=period, format=plain, margin{30pt,30pt}}%
\includegraphics[width=0.97\textwidth]{./graphics/movingup.jpg}%
\addcredit{U.S. DoD.}%
\caption{The effects of the credit going past the edge of the figure. This can be corrected by adding a minipage to hold both the include graaphics, as well as the addcredit command. }

\begin{verbatim}
\begin{figure}[htp]
  \captionsetup{name=Fig., labelsep=period, format=plain}%
  \includegraphics[width=0.97\textwidth]{./graphics/movingup.jpg}%
  \addcredit{U.S. DoD.}%
  \caption{The effects of the credit going past the edge of the figure. This can be corrected by adding a minipage to hold both the include graaphics, as well as the addcredit command. }
\end{figure}
\end{verbatim}
\end{commands}
\end{figure}

\section{Presentation}

Presentation of a lot of figures can influence the appearance of a book tremendously. Like sectioning commands and text styling, magazines and books can be recognized from the styling of their pictures. In figures we imitated the appearance of photographs in Life Magazine. Life in the forties was in the front with the troops and had some great photographers.  It had a style still very hard to improve on.

\begin{figure}[htp]
\bgroup
\parindent=0pt
\null
\clearcaptionsetup{figure}
\captionsetup{style=default,name=Photo.,skip=3pt,parindent=0pt, labelsep=period, margin={0pt,0pt}}%
\begin{minipage}[t]{0.48\textwidth}%
      \includegraphics[width=\textwidth]{./graphics/movingup.jpg}%
      \addcredit{U.S. DoD.}\vskip1sp
     \caption{The effects of the credit going past the edge of the figure. This can be corrected by adding a minipage to hold both commands.}%
\end{minipage}\hfill\hfill
\begin{minipage}[t]{0.48\textwidth}
\includegraphics[width=\linewidth]{./graphics/survivors.jpg}%
%      \addcredit{U.S. DoD.}%
\caption{The effects of the credit going past the edge of the figure. This can be corrected by adding a minipage to hold both commands. }
    
\end{minipage}

 \begin{minipage}[t]{0.48\linewidth}
      \includegraphics[width=\linewidth]{./graphics/img009.jpg}%
      \addcredit{U.S. DoD.}%
     \caption{Engineer Construction Troops in Liberia, July 1942.}
\end{minipage}\hfill\hfill
\begin{minipage}[t]{0.48\textwidth}
      \includegraphics[width=\textwidth]{./graphics/survivors.jpg}%
      \addcredit{U.S. DoD.}%
     \caption{The effects of the credit going past the edge of the figure. This can be corrected by adding a minipage to hold both commands. }
\end{minipage}
\begin{minipage}[t]{0.48\textwidth}
      \includegraphics[width=\textwidth]{./graphics/img126.jpg}%
      \addcredit{U.S. DoD.}%
     \caption{Marine Reinforcements.
A light machine gun squad of 3d Battalion, 1st Marines, arrives during the battle for ``Boulder City.'' }
\end{minipage}\hfill\hfill
\begin{minipage}[t]{0.48\textwidth}
      \includegraphics[width=\textwidth]{./graphics/img124.jpg}%
      \addcredit{U.S. DoD.}%
     \caption{Brothers Under the Skin, inductees at Fort Sam Houston, Texas, 1953. }
\end{minipage}
\egroup
\end{figure}
\newpage


\endinput