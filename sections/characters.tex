\chapter{Characters and TeX}
\label{ch:characters}

\normalsize

\tex\ works internally by translating characters into character codes. The way characters are encoded in a computer
may differ from system to system.\index{characters>encoding}


There are 256 characters that \tex\  might encounter at
each step, in a file or in a line of text typed directly on your terminal. These
256 characters are classified into 16 categories (catcodes) numbered 0 to 15:\index{characters>catcodes}\index{catcodes}



\begin{table}[htbp]
\centering
\begin{tabular}{rll}
\toprule
Code & Description & Representation\\
\midrule
0 & Escape character & (\textbackslash in this book)\\
1 & Beginning of group & (|{| in this book)\\
2 & End of group & (|\}| in this book )\\
3 & Math shift & (|\$| in this book)\\
4 & Alignment tab & (|\&| in this book)\\
5 & End of line &(return in this book)\\
6 & Parameter &(|\#| in this book\\
7 & Supescript &(|\^| in this book)\\
8 & Subscript &(|\_| in this bookl)\\
9 & Ignored character &(null in this manual)\\
10 & Space &(\textvisiblespace in this book)\\
11 &Letter &(A,\ldots,Z and a,\ldots z)\\
12 &Other character &(none of the above or below)\\
13 &Active character &(|\~| in this manual)\\
14 &Comment character &(|\%| in this book)\\
15 &Invalid character &(delete in this book)\\
\bottomrule
\end{tabular}
\captionof{table}{Character Codes}
\end{table}
\medskip

When \tex reads a line of text from a file, or a line of text that you entered
directly on your keyboard, it converts that text into a list of \cmd{\tokens}. A
token is either (a) a single character with an attached category code, or (b) a control
sequence. For example, if the normal conventions of plain \tex  are in force, the text
\verb*+ `{\hskip 36 pt}'+  is converted into a list of \textit{eight} tokens:
\medskip

$ \{_{1}$ hskip $3_{12}~~6_{12}~~\_{10}~~p_{11}~~t_{11}~~\}_2 $

\medskip
The subscripts here are the category codes, as listed earlier:
\begin{itemize}
\item[1] for beginning of group,
\item[12] for |other| character," and so on. The |hskip| doesn't get a subscript, because it
represents a control sequence token instead of a character token. Notice that the space
after \cmd{hskip} does not get into the token list, because it follows a control word.
\end{itemize}

Knuth in the \texbook notes that:

\begin{latexquotation}

It is important to understand the idea of token lists, if you want to gain a
thorough understanding of \tex, and it is convenient to learn the concept by
thinking of \tex as if it were a living organism. The individual lines of input in your
files are seen only by \tex's \textit{eyes} and \textit{mouth}; but after that text has been gobbled
up, it is sent to \tex's \textit{stomach} in the form of a token list, and the digestive processes
that do the actual typesetting are based entirely on tokens. As far as the stomach is
concerned, the input 
flows in as a stream of tokens, somewhat as if your \tex manuscript
had been typed all on one extremely long line.
\end{latexquotation}

\section{Control sequences for characters}

\begin{docCommand}{char}{\meta{number}}
There are a number of ways in which a control sequence can denote a character. The \docAuxCommand{char} command specifies a character to be typeset; the \cmd{\let} command introduces a synonym for a \emph{character
token}, that is, the combination of character code and category code.
\index{\string\char}
Characters can be denoted numerically by, for example, \verb+ \char98 +. This command tells \tex to add
character number 98 of the current font to the horizontal list currently under construction.

Instead of decimal notation, it is often more convenient to use octal or hexadecimal notation. For
octal the single quote is used: \verb+ \char`142+; hexadecimal uses the double quote: \verb+ \char"62+. Of course there are very few uses for octal numbers nowdays. However the hexadecimal numbers can be very useful for unicode manipulation. We will use these extensively in routines for language scripts in other sections.

\end{docCommand}

\begin{texexample}{Characters}{ex:chars}
\char65, 
\char`b 
\char`\b,
\char"70
\end{texexample}

The reverse operation is achieved via prefixing with \docAuxCommand{number}. 

\begin{texexample}{Getting the character code of character}{}
\number`a \\
\meaning a \\
\def\charnum#1{char `#1'  character code = \number`#1}
\charnum A\\
\charnum 6
\end{texexample}

\verb+ \char`'62+  is incorrect; the process that replaces two quotes by a double quote works at a later
stage of processing (the visual processor) than number scanning (the execution processor).

Because of the explicit conversion to character codes by the back quote character it is also possible
to get a ‘b’ – provided that you are using a font organized a bit like the ASCII table – with \verb+ \char‘b+
or \verb+ \char‘\b+.

The \cmd{\char} command looks superficially a bit like the \verb+  ^^+ substitution mechanism.

Both mechanisms access characters without directly denoting them. However, the \verb+ ^^+ mechanism
operates in a very early stage of processing (in the input processor of \tex, but before category
code assignment); the \cmd{\char} command, on the other hand, comes in the final stages of processing.
In effect it says ‘typeset character number so-and-so’.

\CMDI{\Uchar} The LuaTeX expandable command \cmd{\Uchar} reads a number between 0 and 1,114,111 and expands to the associated Unicode character. 

\begin{docCommand*}{chardef} {\meta{control sequence}=\meta{number}}

There is a construction to let a control sequence stand for some character code: the \cmd{\chardef}
where the number can be an explicit representation or a counter value, but it can also be a character
code obtained using the left quote command (see above; the full definition of number is
given in Chapter 7). In the plain format the latter possibility is used in definitions such as
\end{docCommand*}

\verb+ \chardef\%=‘\%+

or 

\verb+ \chardef\%=37   +

command to typeset character 37 (usually the per cent character).\index{characters>percent character}

A control sequence that has been defined with a \cmd{\chardef} command can also be used as a \meta{number}.
This fact is used in allocation commands such as \verb+ \char{newbox}+ (see Chapters 7 and 31). Tokens defined
with \verb+ \char{mathchardef}+ can also be used this way.


But \tex\ actually provides another kind of number that makes it unnecessary
for you to know \texttt{ASCII} at all! The token `12 (left quote), when followed by
any character token or by any control sequence token whose name is a single character,
stands for \tex's internal code for the character in question. For example, \verb+\char`b+ and
\verb+ \char`\b+ are also equivalent to \verb+ \char98+. If you look in Appendix B to see how \verb+ \%+ is
defined, you'll notice that the definition is

\verb+\def\%{\char`\%}+

instead of \verb+ \char37+  as claimed above.

\section{Special notation for invisible characters}

\tex has a standard way to refer to the invisible characters of |ASCII|: 

Code 0 can be typed as the sequence of three characters \verb+ ^^@+, code 1 can be typed
\verb+ ^^A+, and so on up to code 31, which is \verb+ ^^_  +(see Appendix C). If the character following
\verb+ ^^+ has an internal code between 64 and 127, TEX subtracts 64 from the code; if the
code is between 0 and 63, \tex adds 64. 

Hence code 127 can be typed \verb+^^?+, and
the dangerous bend sign can be obtained by saying \verb+{\manual^^?}+. However, you must
change the category code of character 127 before using it, since this character ordinarily
has category 15 (invalid); say, e.g., 

\verb+ \catcode`\^^?=12 +

The \verb+ ^^+ notation is different from
\cmd{\char}, because \verb+ ^^+ combinations are like single characters; for example, it would not
be permissible to say \verb+ \catcode`\char127+, but \verb+^^+ symbols can even be used as letters within control words.

\begin{texexample}{Special notation}{ex:texbook1}
\def\^^zz{test}
\^^zz
\end{texexample}


Most of the \verb+ ^^+ codes are unimportant except in unusual applications. But
\verb+ ^^M+ is particularly noteworthy because it is code 13, the |ASCII| return that
\tex normally places at the right end of every line of your input file. By changing the
category of \verb+ ^^M+  you can obtain useful special effects, as we shall see later.

\section{Upper and Lowercase characters}



The twin operations \docAuxCommand{uppercase}\marg{token list} and \docAuxCommand{lowercase}\marg{token list}
go through a given token list and convert all of the character tokens to their
``uppercase"  or ``lowercase'' equivalents.

\verb*+\lccode +the character code for the lowercase form of a letter (p. 103)

\begin{texexample}{Uppercase and Lowercase}{ex:lowercase} 
\uppercase{abcdefgh} 
\lowercase{ABCDEFGH}
\end{texexample}

Here's how: Each of the 256 possible characters has two associated values called the \cmd{\uccode} and the \cmd{\lccode}; these values are
changeable just as a \cmd{\catcode} is. Conversion to uppercase means that a character
is replaced by its \cmd{\uccode} value, unless the \cmd{\uccode} value is zero (when no change
is made). Conversion to lowercase is similar, using the
\verb+  \lccode+. The category codes
aren't changed. 

When |INITEX| begins, all \verb+ \uccode+ and \verb+ \lccode+ values are zero except
that the letters a to z and A to Z have \verb+\uccode+ values A to Z and \verb+\lccode+ values a to z.

These two control sequences are used to build a hash table, mapping all the capital and lowercase letters to their respective character codes.
(see pg 41 \texbook.)

\section{Category codes}

One of the most infamous, but powerful \tex command is |catcode|. The command is used to change the category code of a character.

As we do not want anything to break we run the code between a group.

\subsection{Active characters}

The term \emph{active character} is another name for a user defined command, which consists of only one character. You can think of active characters as control sequences without a slash consisting of only one character. 
Control sequences can be classifed as \emph{primitive}, a \emph{character} or \emph{user defined}. The latter category is further subdivided into a macro or an active character. To define such a commands, we first have to assign it a category 13. 

In the example below we will define the tilde as category 13 and then define it to expand so that it typesets a subscript.

\begin{texexample}{Active character test}{}
\bgroup
    \catcode `~=13
    \def~{$_1$}

 a~a~a~a~a~
 \egroup
\end{texexample}

When a command is used very often it is preferable to be as short as possible so we would normally define it as an active character. 

In \latex2e and Plain the category code number is stored as |\active| by using |\chardef|

|\chardef\active=13|

so we could rewrite the above as:

\begin{texexample}{Active character test}{}
\bgroup
    \catcode `~=\active
    \def~{$_2$}

 a~a~a~a~a~
 \egroup
\end{texexample}

Although spaces deserve special treatment they can also be made active.

\begin{texexample}{Active character test}{active space}
\bgroup
\makeatletter
\def\showvisiblespaces#1{#1}%
\catcode` =\active%
\def {\textunderscore}%
\showvisiblespaces{This is some test command}%
\makeatother%
\egroup%
\end{texexample}

Let us create another example. This time we will make the mathshift character to be the bar {\textbar} character and the dolar sign \$ to be letter.

\begin{texexample}{Category code manipulation}{}
\bgroup
\catcode`$ = 11
\catcode`| = 3

It used to cost $0.5 for a pie, now it is $5.0.
Let's do the math: |5/.5=10|. 

This equation uses display math.
||50+50=100||
\egroup
\end{texexample}

\latexe has a number of kernel command to assist with |\catcode| changes.  First the |\makeatletter| and |makeatother|. Another one of them is |\@sanitize|

\begin{texexample}{Category code manipulation}{}
\bgroup
\def\@makeother#1{\catcode`#1=12\relax}
\makeatletter
% define a test
\def\test{\@relax }

% definition from source2e
\def \@onelevel@sanitize #1{
   \edef #1{\expandafter\strip@prefix
      \meaning #1}
 }

% sanitize text
 \@onelevel@sanitize\test
 \test
 
 \makeatother
\egroup 
\end{texexample}

In \latex the |@| character is special and reserved for internal commands only. It is defined in the source2e kernel as:

\begin{teXXX}
\def\makeatletter{\catcode`\@11\relax}
\def\makeatother{\catcode`\@12\relax}
\end{teXXX}


The \CMDI{\makeatother} sets it back to be a category 12 code.   Another command \CMDI{\@makeother} turns any character to character code 12. \latex3 makes both the underscore as well as the colon letters, so that they can be used in control sequences.


\begin{texexample}{Category code manipulation}{}
\bgroup
\catcode `\_=12%
\catcode `\: =12
\def\test_long_command:Nnn {latex 3 style control sequence}
\test_long_command:Nnn
\egroup 
\end{texexample}

%This is some comments that you need to write and others










