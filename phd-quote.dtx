% \iffalse meta-comment
%<*internal>
\iffalse
%</internal>
%<*readme>
----------------------------------------------------------------
phd-pkgmanager --- a package to shorten preambles
E-mail: yannislaz@gmail.com
Released under the LaTeX Project Public License v1.3c or later
See http://www.latex-project.org/lppl.txt
----------------------------------------------------------------

%</readme>
%<*readmemd>
###The `phd-quote` LaTeX2e package version 0.08.0

The `phd` latex package and the class with the same name provide
convenient methods to create new styles for books, reports
and articles. It also loads the most commonly used packages 
and resolves conflicts.

This work consists of the file  

   `phd-quote.dtx`,
   
and the derived files   

   `phd-quote.ins`,  
   `phd.pdf`, 
   `phd.sty`.

###Installation

run the provide script

     `phd-lua phd-quote.dtx` on windows
     
     or manually on other OSs.
          

If you have any difficulties with the package come and join us at
http://tex.stackexchange.com and post a new question or
add a comment at http://tex.stackexchange.com/a/45023/963.
or send me a message at  yannislaz at gmail.com

### Documentation

The package was written using the `doc` and `docscript` packages,
so that it is self documented in a literary programming style. 
The .pdf is a fat document, providing over fifty book styles (the
equivalent of classes) plus there is a lot of write-up on the inner
workings of TeX and LaTeX2e. However, you don't need to know much
to use it.

      \usepackage{phd}
      \input{style13}

All choices, are made via an extended key-value interface. 
Although not a compliment, it resembles CSS and the keys are a bit verbose but
attributes are easy to change and have a consistent and easy to remember interface.

To set or add a key we only use one command:

      \cxset{chapter name font-size = Huge,
             chapter number font-size = HUGE} 

### Future Development

This is still an experimental version, but I will retain the
interface in future releases. There is a large amount of
work still to be carried out to improve the template styles
provided, to test it more thoroughly and to add a number of
improvements in the special designs. At present I estimate
that I have completed about 70% of the work that needs
to be done.

__The package as it stands is not production stable.__ 


%</readmemd>
%
%<*TODO>
1. On final round add pkg options. This was left as last in order not to solve problems by adding
    options. Too many options are not a good User Interface.
2.  Finish symbol management, both text and math. Math already 80% incorporated.
3.  Better integration of indexing commands.   
4.  Revisit layout manager for Chapters. Broke again in tests.
5.  Docs. Add all references.
6.  Incorporate phd class for more flexibility.
7.  Improve package manager.
8.  Group script loading for better font management.
9.  General font management to relook it again.
10. Add all style sections (about 100 already prepared). Once they
     are all working issue beta version.
%</TODO>
%<*internal>
\fi
\def\nameofplainTeX{plain}
\ifx\fmtname\nameofplainTeX\else
  \expandafter\begingroup
\fi
%</internal>
%<*install>
\input docstrip.tex
\keepsilent
\askforoverwritefalse
\preamble
----------------------------------------------------------------
phd --- A package to beautify documents.
E-mail: yannislaz@gmail.com
Released under the LaTeX Project Public License v1.3c or later
See http://www.latex-project.org/lppl.txt
----------------------------------------------------------------
\endpreamble

%\BaseDirectory{C:/users/admin/my documents/github/phd}
%\usedir{MWE}
\generate{\file{\jobname.sty}{
  \from{\jobname.dtx}{QUOTE}}
  }

%\nopreamble\nopostamble

%</install>

%<install>\endbatchfile
%<*internal>
%\usedir{tex/latex/phd}
\generate{
  \file{\jobname.ins}{\from{\jobname.dtx}{install}}
}
\nopreamble\nopostamble

\generate{
	\file{README.txt}{\from{\jobname.dtx}{readme}}
  }

\generate{
  \file{\jobname.md}{\from{\jobname.dtx}{readmemd}}
}
\generate{
  \file{\jobname-todo.md}{\from{\jobname.dtx}{TODO}}
}

\ifx\fmtname\nameofplainTeX
  \expandafter\endbatchfile
\else
  \expandafter\endgroup
\fi
%</internal>
%<*driver>

%\listfiles
%gdef\@onlypreamble{} % TO BE REMOVED NEEDED FOR TUTS
\NeedsTeXFormat{LaTeX2e}[2017/04/15]%
\RequirePackage[2017/04/15]{latexrelease}
\documentclass[twoside,11pt,a4paper]{ltxdoc}
\usepackage[bottom=2cm]{geometry}
\savegeometry{std}
% \usepackage[style=mla]{biblatex}
\usepackage{phd}

\usepackage{phd-handlers}
\usepackage{phd-documentation}

\usepackage{phd-toc}
\usepackage{phd-runningheads}
\usepackage{phd-lowersections}
\usepackage{makeidx}
%

\pagestyle{headings}
\sethyperref
\usepackage{phd-quote}
\usepackage{phd-lists}
\cxset{palette bbc}
\makeindex
\begin{filecontents}{defaults-chapters}
%%    General Defaults for Chapters
\cxset{%    
    chapter title margin-top-width    =  0cm,
    chapter title margin-right-width  =  1cm,
    chapter title margin-bottom-width = 10pt,
    chapter title margin-left-width   = 0pt,
    chapter align                     = left,
    chapter title align               = left, %checked
    chapter name                      = hang,
    chapter format                    = fashion,
    chapter font-size                 = Huge,
    chapter font-weight               = bold,
    chapter font-family               = sffamily,
    chapter font-shape                = upshape,
    chapter color                     = black,
    chapter number prefix             = ,
    chapter number suffix             = ,
    chapter numbering                 = arabic,
    chapter indent                    = 0pt,
    chapter beforeskip                = -3cm,
    chapter afterskip                 = 30pt,
    chapter afterindent               = off,
    chapter number after              = ,
    chapter arc                       = 0mm,
    chapter background-color          = bgsexy,
    chapter afterindent               = off,
    chapter grow left                 = 0mm,
    chapter grow right                = 0mm, 
    chapter rounded corners           = northeast,
    chapter shadow                    = fuzzy halo,
    chapter border-left-width         = 0pt,
    chapter border-right-width        = 0pt,
    chapter border-top-width          = 0pt,
    chapter border-bottom-width       = 0pt,
    chapter padding-left-width        = 0pt,
    chapter padding-right-width       = 10pt,
    chapter padding-top-width         = 10pt,
    chapter padding-bottom-width      = 10pt,
    chapter number color              = white,
    chapter label color               = white,    
    }
 \cxset{    
    chapter number font-size        = huge,
    chapter number font-weight      = bfseries,
    chapter number font-family      = sffamily,
    chapter number font-shape       = upshape,
    chapter number align            = Centering,
    }
\cxset{%    
     chapter title font-size        = Huge,
     chapter title font-weight      = bold,
     chapter title font-family      = calligra,
     chapter title font-shape       = upshape,
     chapter title color            = black,
      subsection afterindent=off,
       section afterindent=off,
     }    
\end{filecontents}
%% LaTeX2e file `defaults-chapters'
%% generated by the `filecontents' environment
%% from source `phd-documentation' on 2018/10/28.
%%
%%    General Defaults for Chapters
\cxset{%
    chapter title margin-top-width    =  0cm,
    chapter title margin-right-width  =  1cm,
    chapter title margin-bottom-width = 10pt,
    chapter title margin-left-width   = 0pt,
    chapter align                     = left,
    chapter title align               = left, %checked
    chapter name                      = hang,
    chapter format                    = hdr,
    chapter font-size                 = Huge,
    chapter font-weight               = bvar,
    chapter font-family               = sffamily,
    chapter font-shape                = upshape,
    chapter color                     = black,
    chapter number prefix             = ,
    chapter number suffix             = ,
    chapter numbering                 = arabic,
    chapter indent                    = 0pt,
    chapter beforeskip                = -3cm,
    chapter afterskip                 = 30pt,
    chapter afterindent               = off,
    chapter number after              = ,
    chapter arc                       = 0mm,
    chapter background-color          = bgsexy,
    chapter afterindent               = off,
    chapter grow left                 = 0mm,
    chapter grow right                = 0mm,
    chapter rounded corners           = northeast,
    chapter shadow                    = fuzzy halo,
    chapter border-left-width         = 0pt,
    chapter border-right-width     = 0pt,
    chapter border-top-width       = 0pt,
    chapter border-bottom-width    = 0pt,
    chapter padding-left-width     = 0pt,
    chapter padding-right-width    = 10pt,
    chapter padding-top-width      = 10pt,
    chapter padding-bottom-width   = 10pt,
    chapter number color           = white,
    chapter label color            = white,
    }
 \cxset{
    chapter number font-size        = huge,
    chapter number font-weight      = bfseries,
    chapter number font-family      = sffamily,
    chapter number font-shape       = upshape,
    chapter number align            = Centering,
    }
\cxset{%
     chapter title font-size        = Huge,
     chapter title font-weight      = bvar,
     chapter title font-family      = calligra,
     chapter title font-shape       = upshape,
     chapter title color            = black,
     }
  
%\definecolor{bgsexy}{HTML}{FF6927}
%
%\definecolor{creamy}{HTML}{FDEBD7}
\cxset{chapter title color= creamy,
       chapter label color = creamy,
       chapter number color = creamy,
       chapter number font-size = Huge,
       subsection title color = creamy,
       chapter name = CHAPTER,
       chapter label case = upper,
       chapter number align=left,
       part format = traditional,
       part background-color=spot,
       part beforeskip                = -3cm,
       part afterskip                 = 30pt,
       section format=hang,
             }
\usepackage{microtype}  

\begin{document}
\DEBUGOFF
\parindent1em
\coverpage{korea-01}{Book Design Monographs}{Camel Press}{QUOTE}{DESIGN} 
\pagestyle{empty}
\secondpage
\pagestyle{empty}
\clearpage
\tableofcontents
\pagestyle{empty}
\setcounter{secnumdepth}{4}
\parskip0pt plus.1ex minus.1ex
\mainmatter
\pagenumbering{arabic}
\pagestyle{headings}        
\DocInput{\jobname.dtx}
\printindex
 %
% 
\end{document}
%</driver>
% \fi
% 
%  \CheckSum{0}
%  \CharacterTable
%  {Upper-case    \A\B\C\D\E\F\G\H\I\J\K\L\M\N\O\P\Q\R\S\T\U\V\W\X\Y\Z
%   Lower-case    \a\b\c\d\e\f\g\h\i\j\k\l\m\n\o\p\q\r\s\t\u\v\w\x\y\z
%   Digits        \0\1\2\3\4\5\6\7\8\9
%   Exclamation   \!     Double quote  \"     Hash (number) \#
%   Dollar        \$     Percent       \%     Ampersand     \&
%   Acute accent  \'     Left paren    \(     Right paren   \)
%   Asterisk      \*     Plus          \+     Comma         \,
%   Minus         \-     Point         \.     Solidus       \/
%   Colon         \:     Semicolon     \;     Less than     \<
%   Equals        \=     Greater than  \>     Question mark \?
%   Commercial at \@     Left bracket  \[     Backslash     \\
%   Right bracket \]     Circumflex    \^     Underscore    \_
%   Grave accent  \`     Left brace    \{     Vertical bar  \|
%   Right brace   \}     Tilde         \~}
%
%
%
% \changes{1.0}{2013/01/26}{Converted to DTX file}
% \changes{1.0}{2017/08/08}{Update to latest LaTeX distribution}
%
% \DoNotIndex{\newcommand,\newenvironment}
% \GetFileInfo{phd.dtx}
% 
%  \def\fileversion{v1.0}          
%  \def\filedate{2012/03/06}
%  \title{The \textsf{phd} package.
%  \thanks{This
%        file (\texttt{phd.dtx}) has version number \fileversion, last revised
%        \filedate.}
%  }
%  \author{Dr. Yiannis Lazarides \\ \url{yannislaz@gmail.com}}
%  \date{\filedate}
%
%
% 
% ^^A\maketitle
% 
% ^^A\frontmatter
%  ^^A\coverpage{./images/hine02.jpg}{Book Design }{Camel Press}{}{}
%  \newpage
% ^^A\secondpage
% \pagestyle{empty}
%
%
% 
%
%
% \pagestyle{headings}
% \raggedbottom
%
% \chapter{User Manual}
%
% \section{Introduction}
%
% The usage of the package is fairly simple. Include it in your preamble as you would normally
%  include a package.
%
%  Select the style you need by:
%
%     |\cxset{quotation tugboat}|
%
%
%  \section{Define your own style}
%
% To defne your own style use the \docAuxCommand{cxset} command again as follows:
%
% \begin{teXXX}
%\cxset{quotation myjournal/.style={
%  quotation above = 5pt plus 2pt minus 1pt,
%  quotation left margin=2pc,
%  quotation right margin=2pc,
%  quotation parindent=10pt,
%}}
% \end{teXXX}
%
% Preferably do this in the preamble of your document.
%
% \begin{texexample}{Quotation Environment Example}{ex:quot}
% \begingroup
% \cxset{quotation tugboat}
% \cxset{quotation font-size=small}
% \lorem\lorem
% \begin{quotation}
% \lorem
% \end{quotation}
% \endgroup
% \begin{latexquotation}
% \lorem
% \end{latexquotation}
% \end{texexample}
%  \OnlyDescription
%
%  ^^A\StopEventually{\printindex}

% \CodelineNumbered
% \pagestyle{headings}
% 
% 
% ^^A\part{IMPLEMENTATION AND FRIENDS}
% \chapter{PHD-Quote Package Objectives}
% 
% \epigraph{
% I was reflecting on the convoluted Java frameworks widely adopted at work. Those hefty frameworks brought coding structures and conventions to large engineering teams; meanwhile, they also sucked the fun of programming like a Pastafarian monster slurping all the tomato sauce on a plate of spaghetti.
%}{\href{http://blog.zmxv.com/2015/07/code-golf-at-google.html}{Zhen Wang}}
%
% We start by outlining what we are trying to achieve with this package:
% \cxset{enumerate numberingi =arabic,
%        enumerate labeli punctuation=,}
% \begin{enumerate}
% \item To provide a declarative interface to enable users to modify the styling of 
%         quotations by using a key value concept. These can be added in an overall style
%         file to enable a unified interface to the styling of a document.
% \item The interface must be able to manupulate all the properties of a quotation or quote environment.
% \item To provide a compatibility mode, where documents wishing to test the package
% can have an easy switch to switch in and out. This is also important for the testing of the package.
% \item To provide a number of pre-defined styles.
% \item To provide means for a plug-in architecture for extensions.
% \end{enumerate}
% 
% \section{Terminology}
%
%  \begin{description}
%  \item [document] Any written item, as a book, article, or letter, especially 
%                  of a factual or informative nature.
%  \item [heading] A division of a document or document series. For a normal
%        book headings are chapters, sections etc. However we allow for
%        specifying a more complex document divided into books, volumes
%        parts etc. For example the Bible has Books, chapters and verses,
%        where a legal document might require divisions such as clauses.
%        In general these divisions are numbered. These document divisions
%        are stored in the comma list \refCom{phd_book_divisions_clist}.
%  \item [head] A typeset heading, such as chapter head, or section head.
%        This can include a counter, label and title for example, 
%        \emph{Chapter 1 Introduction}.
%  \item [dom] This is a programming interface that provides a structured
%        representation of the document (a tree) and it defines a way
%        that the structure can be accessed. Although \latexe does not
%        offer a standard way to build such a tree (mainly because
%        \tex does not require the marking of paragraphs, it is 
%        useful to think of the document as a tree structure. We also
%        allow for a semi-automated way to build such a tree (with the 
%        exception that paragraphs are not included).
% \item [element] A part of the document tree that can be styled on
%       its own. For example the chapter label, or the section number.
%
% \end{description}
%
% \section{Users}
%  We classify users according to the \LaTeX3 terminology as a) programmers b) template designers
%  and c) authors.
% \subsection{Author}
%  We assume that the author has an exising template which she is using but might want to do
%  some minor modifications, for example use an italic shape for the font of the mark, but an 
%  upright font for the page numbers. 
%
% {\obeylines 
%~~ |\cxset|
%~~~~~|{|
%~~~~~~~~\textit{chapter number color}~~|format          = apa,|
%~~~~~~~~\textit{section title font-size} |font-size   = Large,|
%~~~~~|}|
%}  
%
% We follow the idea of representing the basic elements of documents
% as elements, each one having a parent in order to specify
% the element we need to style as accurate as possible. One can think of
% this approach being congruent with objects in other languages.
% As a matter fact nothing stops us from defining a key value
% interface as shown below.
%
% {\obeylines 
%~~ |\cxset|
%~~~~~|{| 
%~~~~~~~~\textit{header.even.mark.font.size}   = |Large,|
%~~~~~~~~\textit{header.even.mark.font.family} = |serif,|
%~~~~~|}|
%}  
%
% This would pehaps make it easier for the template designer, but I have rejected
% the idea as my aim is to make it easy for the author, who can search the template
% and just enter a couple of new proerty values.
%
% \subsection{Template designer}
% \pagestyle{headings}
% The template designer in the example above would have selected the format style
% from a number of predefined formats (templates) or would have created a style
% called \textit{apa} from an existing template and modified it using declarative
% key style.
%
% \subsection{The programmer}
%
% The programmer in the example above could have created the basic format
% \textit{apa} by using both declarative as well as defining or using existing
% macros. To the programmer we offer an extension mechanism, where the contents
% of a |ps@| command are defined. For example the programmer can define a new
% style using \tikzname, but without having to worry about defining full |ps@|
% and their interface.
%
% \section{Preliminaries}
%
%  Standard file identification. We first announce the package 
%	 and require that it be used with \LaTeX2e. 
% \iffalse
%<*QUOTE>
% \fi
%  
%
%    \begin{macrocode}
\NeedsTeXFormat{LaTeX2e}[2017/04/15]%
\ProvidesFile{phd-quote}[2017/04/15 v1.0 quotation management (YL)]%
%    \end{macrocode}
%
% 
% \section{Source2e Interface}
% 
% I am not very fond of mixing expl3 control sequences with source2e commands. Here
% we provide an interface for all these commands we might use. 
% This section can be revisited once expl3 code becomes available.
%
%    \begin{macrocode}
\let\ltxtoday\today
\newif\if@ltxcompat \@ltxcompatfalse
\renewenvironment{quotation}
               {\par\addvspace{6pt}
                \list{}{\listparindent12pt %
                        \leftmargin=\quotationleftmargin@cx %
                        \itemindent    \quotationparindent@cx
                        \rightmargin   \quotationrightmargin@cx
                        \parsep \quotationparsep@cx 
                        \expandafter\csname thequotationfontsize\endcsname   %
                        \quotationfontname@cx\relax } 
                \item\relax\hspace{0pt}\relax }
               {\endlist}
 \let\latexquotation\quotation
\let\endlatexquotation\endquotation
\let\latexquote\quote
\let\endlatexquote\endquote
%    \end{macrocode}
%
% \section{Key Management}
%
% For all keys we use the \pgfname package.
%
%    \begin{macrocode}
\ExplSyntaxOn
\cxset{
  quotation~above/.store~in=\quotationabove@cx,
  quotation~left~margin/.store~in=\quotationleftmargin@cx,
  quotation~right~margin/.store~in=\quotationrightmargin@cx,
  quotation~parsep/.store~in=\quotationparsep@cx,
  quotation~font-size/.fontsize= thequotationfontsize,
  quotation~parindent/.store~in=\quotationparindent@cx,
  quotation~font-name/.store~in=\quotationfontname@cx,
 }
\ExplSyntaxOff 
 \cxset{quotation default/.style={%
  quotation above=0pt, 
  quotation left margin=25pt,
  quotation right margin=25pt,
  quotation parsep=2pt,
  quotation font-size=normal,
  quotation parindent=4em,
  quotation font-name=\arial,}} 
 \cxset{quotation default}
%    \end{macrocode}
% 
%
%    \begin{macrocode}

\cxset{quotation acmtrans2e/.style={
  quotation above = 5pt plus 2pt minus 1pt,
  quotation left margin=2pc,
  quotation right margin=2pc,
  quotation parindent=10pt,
}}

\cxset{quotation ametsoc/.style={
  quotation above = 5pt plus 2pt minus 1pt,
  quotation left margin=2pc,
  quotation right margin=2pc,
  quotation parindent=1.5em,
  %quotation item indent = 1.5em, % 
}}


\cxset{quotation phil/.style={
  quotation above = 5pt plus 2pt minus 1pt,
  quotation left margin=13.5pt,
  %quotation right margin=13.5pt,
  quotation right margin=13.5pt,
  quotation parindent=1.5em,
  %quotation item indent = 1.5em, % 
}}

\cxset{quotation tugboat/.style={
  quotation default,
  quotation above = 5pt plus 2pt minus 1pt,
  quotation left margin=4em,
  quotation right margin=1em,
  quotation parindent=1.5em,
  %quotation item indent = 1.5em, % 
}}

\cxset{quotation wide/.style={
  quotation above = 5pt plus 2pt minus 1pt,
  quotation left margin=3em,
  quotation right margin=3em,
  quotation parindent=1.5em,
  %quotation item indent = 1.5em, % 
}}
\def\longitem{\list{---}{\labelwidth\z@
 \leftmargin\z@ \itemindent\parindent \advance\itemindent\labelsep}}
\let\endlongitem\endlist
%    \end{macrocode}
% \hrule
% \lorem 
% \hrule
% \cxset{quotation acmtrans2e}
% 
% \begin{quotation}
% 
% \lorem 
% \lipsum[1]
% \lorem
% \hrule
% \end{quotation}
% \hrule
%
% \begin{longitem}
% \item \lorem
% \item \lorem
% \end{longitem}
%
% \section{Philosophers Imprint Style}
% \lorem
% \cxset{quotation phil}
%
% \begin{quotation}
% \lorem
%
% \lorem
% \end{quotation}
%
% \subsection{TUGboat}
%
% This style is interesting as it has uneven left and right margins.
%
% \lorem
% 
% \cxset{quotation tugboat}
% \begin{quotation}
% \lorem
%
% \lorem
% \end{quotation}
% 
% 
% \subsection{International Journal of Digital Curation (dcpaper)}
%\cxset{quotation wide}
%\begin{quotation}
%‘Cras porttitor dictum lacus. Class aptent taciti sociosqu ad litora torquent per conubia nostra, per inceptos hymenaeos. In consectetuer, diam at volutpat elementum, libero lectus pulvinar sem.’ (Borgman, 2007)
%\end{quotation}
%
% 
% \begin{docCommand} {setquote} { \meta{keys} }

% \end{docCommand}
%    \begin{macrocode}
\cxset{
  quote above/.store in=\quoteabove@cx,
  quote left margin/.store in=\quoteleftmargin@cx,
  quote right margin/.store in=\quoterightmargin@cx,
  quote parsep/.store in=\quoteparsep@cx,
  quote font-size/.store in=\quotefontsize@cx,
  quote parindent/.store in=\quoteparindent@cx,
  quote font-name/.store in=\quotefontname@cx,
 }

\def\setquote#1{%
  \cxset{#1}
  \renewenvironment{quote}
               {\par\addvspace{\quoteabove@cx}
                \list{}{\listparindent\quoteparindent@cx%
                        \leftmargin=\quoteleftmargin@cx%
                        \itemindent  \listparindent
                        \rightmargin\leftmargin
                        \parsep=\quoteparsep@cx%
                        \quotefontsize@cx\quotefontname@cx}%
                \item\relax\hskip-\listparindent}
               {\endlist}
  }

%\setquote{%
%  quote above=0pt,
%  quote left margin=50pt,
%  quote parsep=0pt,
%  quote font-size=\small,
%  quote parindent=12pt,
%  quote font-name=,
% }
%    \end{macrocode}
% \begin{quote}
%   \lorem
%
%   \lorem
% \end{quote}
%</QUOTE>
\endinput

% 
%