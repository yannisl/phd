% \iffalse meta-comment
%<*internal>
\iffalse
%</internal>
%<*readme>
----------------------------------------------------------------
phd-pkgmanager --- a package to shorten preambles
E-mail: yannislaz@gmail.com
Released under the LaTeX Project Public License v1.3c or later
See http://www.latex-project.org/lppl.txt
----------------------------------------------------------------

%</readme>
%<*readmemd>
###The `phd-quote` LaTeX2e package version 0.08.0

The `phd` latex package and the class with the same name provide
convenient methods to create new styles for books, reports
and articles. It also loads the most commonly used packages 
and resolves conflicts.

This work consists of the file  

   `phd-quote.dtx`,
   
and the derived files   

   `phd-quote.ins`,  
   `phd.pdf`, 
   `phd.sty`.

###Installation

run the provide script

     `phd-lua phd-quote.dtx` on windows
     
     or manually on other OSs.
          

If you have any difficulties with the package come and join us at
http://tex.stackexchange.com and post a new question or
add a comment at http://tex.stackexchange.com/a/45023/963.
or send me a message at  yannislaz at gmail.com

### Documentation

The package was written using the `doc` and `docscript` packages,
so that it is self documented in a literary programming style. 
The .pdf is a fat document, providing over fifty book styles (the
equivalent of classes) plus there is a lot of write-up on the inner
workings of TeX and LaTeX2e. However, you don't need to know much
to use it.

      \usepackage{phd}
      \input{style13}

All choices, are made via an extended key-value interface. 
Although not a compliment, it resembles CSS and the keys are a bit verbose but
attributes are easy to change and have a consistent and easy to remember interface.

To set or add a key we only use one command:

      \cxset{chapter name font-size = Huge,
             chapter number font-size = HUGE} 

### Future Development

This is still an experimental version, but I will retain the
interface in future releases. There is a large amount of
work still to be carried out to improve the template styles
provided, to test it more thoroughly and to add a number of
improvements in the special designs. At present I estimate
that I have completed about 70% of the work that needs
to be done.

__The package as it stands is not production stable.__ 


%</readmemd>
%
%<*TODO>
1. On final round add pkg options. This was left as last in order not to solve problems by adding
    options. Too many options are not a good User Interface.
2.  Finish symbol management, both text and math. Math already 80% incorporated.
3.  Better integration of indexing commands.   
4.  Revisit layout manager for Chapters. Broke again in tests.
5.  Docs. Add all references.
6.  Incorporate phd class for more flexibility.
7.  Improve package manager.
8.  Group script loading for better font management.
9.  General font management to relook it again.
10. Add all style sections (about 100 already prepared). Once they
     are all working issue beta version.
%</TODO>
%<*internal>
\fi
\def\nameofplainTeX{plain}
\ifx\fmtname\nameofplainTeX\else
  \expandafter\begingroup
\fi
%</internal>
%<*install>
\input docstrip.tex
\keepsilent
\askforoverwritefalse
\preamble
----------------------------------------------------------------
phd --- A package to beautify documents.
E-mail: yannislaz@gmail.com
Released under the LaTeX Project Public License v1.3c or later
See http://www.latex-project.org/lppl.txt
----------------------------------------------------------------
\endpreamble

%\BaseDirectory{C:/users/admin/my documents/github/phd}
%\usedir{MWE}
\generate{\file{\jobname.sty}{
  \from{\jobname.dtx}{QUOTE}}
  }

%\nopreamble\nopostamble

%</install>

%<install>\endbatchfile
%<*internal>
%\usedir{tex/latex/phd}
\generate{
  \file{\jobname.ins}{\from{\jobname.dtx}{install}}
}
\nopreamble\nopostamble

\generate{
	\file{README.txt}{\from{\jobname.dtx}{readme}}
  }

\generate{
  \file{\jobname.md}{\from{\jobname.dtx}{readmemd}}
}
\generate{
  \file{\jobname-todo.tex}{\from{\jobname.dtx}{TODO}}
}

\ifx\fmtname\nameofplainTeX
  \expandafter\endbatchfile
\else
  \expandafter\endgroup
\fi
%</internal>
%<*driver>

%\listfiles
%gdef\@onlypreamble{} % TO BE REMOVED NEEDED FOR TUTS
\NeedsTeXFormat{LaTeX2e}[2017/04/15]%
\RequirePackage[2017/04/15]{latexrelease}
\documentclass[twoside,11pt,a4paper]{ltxdoc}
\usepackage[bottom=2cm]{geometry}
\savegeometry{std}
% \usepackage[style=mla]{biblatex}
\usepackage{phd}

\usepackage{phd-handlers}
\usepackage{phd-documentation}

\usepackage{phd-toc}
\usepackage{phd-runningheads}
\usepackage{phd-lowersections}
\usepackage{makeidx}
%

\pagestyle{headings}
\sethyperref
\usepackage{phd-quote}
\usepackage{phd-lists}
\cxset{palette bbc}
\makeindex
\begin{filecontents}{defaults-chapters}
%%    General Defaults for Chapters
\cxset{%    
    chapter title margin-top-width    =  0cm,
    chapter title margin-right-width  =  1cm,
    chapter title margin-bottom-width = 10pt,
    chapter title margin-left-width   = 0pt,
    chapter align                     = left,
    chapter title align               = left, %checked
    chapter name                      = hang,
    chapter format                    = fashion,
    chapter font-size                 = Huge,
    chapter font-weight               = bold,
    chapter font-family               = sffamily,
    chapter font-shape                = upshape,
    chapter color                     = black,
    chapter number prefix             = ,
    chapter number suffix             = ,
    chapter numbering                 = arabic,
    chapter indent                    = 0pt,
    chapter beforeskip                = -3cm,
    chapter afterskip                 = 30pt,
    chapter afterindent               = off,
    chapter number after              = ,
    chapter arc                       = 0mm,
    chapter background-color          = bgsexy,
    chapter afterindent               = off,
    chapter grow left                 = 0mm,
    chapter grow right                = 0mm, 
    chapter rounded corners           = northeast,
    chapter shadow                    = fuzzy halo,
    chapter border-left-width         = 0pt,
    chapter border-right-width        = 0pt,
    chapter border-top-width          = 0pt,
    chapter border-bottom-width       = 0pt,
    chapter padding-left-width        = 0pt,
    chapter padding-right-width       = 10pt,
    chapter padding-top-width         = 10pt,
    chapter padding-bottom-width      = 10pt,
    chapter number color              = white,
    chapter label color               = white,    
    }
 \cxset{    
    chapter number font-size        = huge,
    chapter number font-weight      = bfseries,
    chapter number font-family      = sffamily,
    chapter number font-shape       = upshape,
    chapter number align            = Centering,
    }
\cxset{%    
     chapter title font-size        = Huge,
     chapter title font-weight      = bold,
     chapter title font-family      = calligra,
     chapter title font-shape       = upshape,
     chapter title color            = black,
      subsection afterindent=off,
       section afterindent=off,
     }    
\end{filecontents}
%% LaTeX2e file `defaults-chapters'
%% generated by the `filecontents' environment
%% from source `phd-documentation' on 2018/10/28.
%%
%%    General Defaults for Chapters
\cxset{%
    chapter title margin-top-width    =  0cm,
    chapter title margin-right-width  =  1cm,
    chapter title margin-bottom-width = 10pt,
    chapter title margin-left-width   = 0pt,
    chapter align                     = left,
    chapter title align               = left, %checked
    chapter name                      = hang,
    chapter format                    = hdr,
    chapter font-size                 = Huge,
    chapter font-weight               = bvar,
    chapter font-family               = sffamily,
    chapter font-shape                = upshape,
    chapter color                     = black,
    chapter number prefix             = ,
    chapter number suffix             = ,
    chapter numbering                 = arabic,
    chapter indent                    = 0pt,
    chapter beforeskip                = -3cm,
    chapter afterskip                 = 30pt,
    chapter afterindent               = off,
    chapter number after              = ,
    chapter arc                       = 0mm,
    chapter background-color          = bgsexy,
    chapter afterindent               = off,
    chapter grow left                 = 0mm,
    chapter grow right                = 0mm,
    chapter rounded corners           = northeast,
    chapter shadow                    = fuzzy halo,
    chapter border-left-width         = 0pt,
    chapter border-right-width     = 0pt,
    chapter border-top-width       = 0pt,
    chapter border-bottom-width    = 0pt,
    chapter padding-left-width     = 0pt,
    chapter padding-right-width    = 10pt,
    chapter padding-top-width      = 10pt,
    chapter padding-bottom-width   = 10pt,
    chapter number color           = white,
    chapter label color            = white,
    }
 \cxset{
    chapter number font-size        = huge,
    chapter number font-weight      = bfseries,
    chapter number font-family      = sffamily,
    chapter number font-shape       = upshape,
    chapter number align            = Centering,
    }
\cxset{%
     chapter title font-size        = Huge,
     chapter title font-weight      = bvar,
     chapter title font-family      = calligra,
     chapter title font-shape       = upshape,
     chapter title color            = black,
     }
  
%\definecolor{bgsexy}{HTML}{FF6927}
%
%\definecolor{creamy}{HTML}{FDEBD7}
\cxset{chapter title color= creamy,
       chapter label color = creamy,
       chapter number color = creamy,
       chapter number font-size = Huge,
       subsection title color = creamy,
       chapter name = CHAPTER,
       chapter label case = upper,
       chapter number align=left,
       part format = traditional,
       part background-color=spot,
       part beforeskip                = -3cm,
       part afterskip                 = 30pt,
       section format=hang,
             }
\usepackage{microtype}  

\begin{document}
\DEBUGOFF
\parindent1em
\coverpage{korea-01}{Book Design Monographs}{Camel Press}{QUOTE}{DESIGN} 
\pagestyle{empty}
%\coverpage{habtoor-city}{Delay Claim}{HLS-DSE/JV}{HABTOOR CITY}{MEP CLAIM} 
\secondpage
\pagestyle{empty}
\clearpage

\tableofcontents

\pagestyle{empty}
\setcounter{secnumdepth}{6}
\parskip0pt plus.1ex minus.1ex
\mainmatter
\pagenumbering{arabic}
\pagestyle{headings}        

%%FIX LIST DIAGRAM

\cxset{steward,
  chapter name=chapter,
  chapter format= stewart,
   image={sweepers.jpg},
  texti={Lists are essential elements of any document style and perhaps the most troublesome to get right.
         In this chapter we discuss the construction of lists and offer a key value interface.},
  textii={The Chapter discusses in detail the construction of lists. It reviews the mechanisms offered
          by LaTeX and outlines a key value approach to building lists. We define a standard interface that does not
          interfere with the original commands. The three standard list styles \textit{enumerate, itemize} and \textit{description} are redesigned to accept a key value interface. The photograph is Lewis Hine's which noted: ``Ivey Mill Company, Hickory, N.C. Some doffers and sweepers. Plenty of them.'' Location: Hickory, Catawba County Date: November 1908. Photographs like this were used by Hine to campaign against child labour.
         }
}

\def\storyi{Lists are essential elements of any document style and perhaps the most troublesome to get right.
         In this chapter we discuss the standard lists offered in the LaTeX classes and describe how new lists can be constructed. We review and use:
         
         \begin{enumerate}
           \item enumerate
           \item itemize
           \item description
           \item trivlist
         \end{enumerate}
   
Some commonly used packages are also reviewed.        
         }

\cxset{palette spring onion}
\pagestyle{headings-spring-onion}
\makeatletter
\def\imagewidth@cx{6cm}
\makeatother
\cxset{chapter format=fashion,
       fashion image=fashion-pngtree.png}

\chapter{Standard \LaTeX\ Lists}

\pagebreak

\section{Introduction}

There are four environments for producing formatted lists:\footnote{There are also other environments, such as \emph{quote}, %
\emph{quotation}, \emph{verbatim}, which behind the scenes are also lists.}

\begin{trivlist}
\item |\begin|\marg{trivlist} list text |\end{trivlist}|
\item |\begin|\marg{itemize} list text |\end{itemize}|
\item |\begin|\marg{enumerate} list text |\end{enumerate}|
\item |\begin|\marg{description} list text |\end{description}|
\end{trivlist}

Lists shape their contents so that 
 the \emph{list text} is indented from the left margin
and a label, or marker, is included. What type of label is used depends
on the selected list environment. The command to produce the label is |\item|. Any following paragraphs, i.e., paragraphs types without being prefixed by |\item| are at the same distance from the margin. 

Lists can be nested either mixed or of one type to a depth of four levels. The type of label used depends on the level of nesting. The indentation is always relative to the left margin or right margin of the enclosing list.


\cxset{label itemi = \textbullet,
       label itemii = *}

\begin{itemize}
\item This is the first level of the list.
  \begin{itemize}
     \item This is the second level of the list.
  \end{itemize}
\item And back to the first level.
\end{itemize}

The optional argument of the |\item| can be used to change the label in the itemize and enumerate environments. The optional
argument takes precedence over the standard label. For the enumerate environment, this means that the corresponding counter is not automatically incremented. You will need to do the numbering manually.



\begin{texexample}{Example with manual settings}{ex:settings}
\section{Example of the itemize environment}
\begin{itemize}
\item[---] This is the first level of the list.
  \begin{itemize}
     \item[\textbullet] This is the second level of the list.
  \end{itemize}
\item[---] And back to the first level.
\end{itemize}
\end{texexample}

The optional label appears right justified within the area reserved for
the label. The width of this area is the amount of indentation at that level
less the separation between label and text; this means that the left edge
of the label area is flush with the left margin of the enclosing level.

It is also possible to change the standard labels for all or part of the
document. The labels are generated with the internal commands

\bgroup
\trivlist\item
\cs{labelitemi}, \cs{labelitemii}, \cs{labelitemiii}, \cs{labelitemiv}, 
\cs{labelenumi}, \cs{labelenumii}, \cs{labelenumiii}, \cs{labelenumiv}
\endtrivlist
\egroup

The endings i, ii, iii, and iv refer to the four possible levels.
These commands may be altered with \cs{renewcommand}. For example,
to change the label of the third level of the itemize environment to a checkmark (\ding{51}), we can write:



\begin{quote}\small
|\renewcommand{\labelitemiii}{\ding{51}}|
\end{quote}

The symbol |\ding{31}| is available by loading the \pkg{pifont}\footcite{pifont} or another package that can be used for such symbols such as \pkg{bbding} or create your own using a suitable font such as \pkg{symbola}
or \pkg{fontawesome}. 

\begin{texexample}{Changing the label symbols}{ex:symbols}
\renewcommand{\labelitemiii}{\ding{51}}
\begin{itemize}
\item This is the first level of the list.
  \begin{itemize}
     \item This is the second level of the list.
     \begin{itemize}
     \item This is the third level of the list.
     \end{itemize}
  \end{itemize}
\item And back to the first level.
\end{itemize}

\end{texexample}

We can use symbols, if we need them as in the following example:

\begin{texexample}{A yes and no list}{ex:yesno}
\newcommand{\Yess}{\ding{51}}
\newcommand{\Noo}{\ding{55}}

\begin{enumerate}
 \item[\Yess] Learn about lists.
 \item[\Noo] Learn about the \tex's output routine.
\end{enumerate}
\end{texexample}

As we can see from the example, we can use the argument of an |item| and it can change an enumerated list into an itemized list, provided we type the argument manually. The |enumi| will not be incremented. 



\section{Generalized lists}

Lists such as those in the three environments itemize, enumerate, and
description can be formed in a quite general way. The type of label and
its width, the depth of indentation, spacings for paragraphs and labels,
and so on, may be wholly or partially set by the user by means of the list
environment:

\bgroup
\trivlist\item 
     \cs{begin}\{list\}\marg{std\_label}\marg{list\_declarations} item list \cs{end}\{list\}
\endtrivlist
\egroup


Here item list consists of the text for the listed entries, each of which
begins with an |\item| command that generates the corresponding label.
The \(stnd\_label\) contains the definition of the label to be produced by the
\cs{item} command when the optional argument is missing.  

The first argument in the list environment defines the |standard label|, that
is, the label that is produced by the \cs{item} command when it appears
without an argument. In the case of an unchanging label, such as for the
itemize environment, this is simply the desired symbol. If this is to be a
mathematical symbol, it must be given as \$symbol name\$, enclosed in \$
signs. For example, to select a right arrow symbol ($\Rightarrow$) as the label, 
the \emph{std\_label} must be defined to be |\$\Rightarrow\$|.



\newcounter{steps}
\setcounter{steps}{0}

\begin{list}{\bfseries\upshape Step \arabic{steps}:}
{%
\usecounter{steps}
\setlength{\labelwidth}{2cm}\setlength{\leftmargin}{2.6cm}
\setlength{\labelsep}{0.5cm}\setlength{\rightmargin}{1cm}
\setlength{\parsep}{0.5ex plus0.2ex minus0.1ex}
\setlength{\itemsep}{0ex plus0.2ex minus0pt}\relax \slshape %
}
\item Melt the butter and dark chocolate
\item Prepare the egg and sugar mix
\item Cool the butter and dark chocolate
\item Set out the milk and white chocolate
\item Prepare the brownie tin

\item[]\ldots
    
\item Fold in the chocolate to the eggy mousse\ldots
\item Add the flour and cocoa\ldots
\item Get the tin in the oven.
\end{list}



\begin{texexample}[listing only]{Generalized Lists}{ex:genlists}
% create a new counter for the list
%\newcounter{steps}
%\setcounter{steps}{0}
Continuing with our recipe\ldots

\makeatletter
\def\usecounter#1{\@nmbrlisttrue\def\@listctr{#1}}


\begin{list}{\bfseries\upshape Step \arabic{steps}:}%
{ 
% this has to be on the first line 
\usecounter{steps}
\setlength{\itemsep}{0ex} 
\setlength{\labelwidth}{2cm}
\setlength{\leftmargin}{2.6cm}
\setlength{\labelsep}{0.5cm}
\setlength{\rightmargin}{1cm}
\setlength{\parsep}{0.5ex plus.2pt}
 }
\item Melt the butter and dark chocolate
\item Prepare the egg and sugar mix
\item Cool the butter and dark chocolate
\item Set out the milk and white chocolate
\item Prepare the brownie tin
      \ldots
\item Fold in the chocolate to the eggy mousse\ldots
\item Add the flour and cocoa\ldots
\item Get the tin in the oven.

% end the list
\end{list}
\makeatother


This brings us to the end of our cooking lessons.
\end{texexample}

This brings us to the next step. 

\makeatletter
\gdef\resume{\def\usecounter##1{\@nmbrlisttrue\def\@listctr{##1}}\relax}
\gdef\reset{\def\usecounter##1{\@nmbrlisttrue\def\@listctr{##1}\setcounter{##1}{0}\relax}}
\makeatother

\resume
\begin{list}{\bfseries\upshape A \arabic{steps}:}
{%
\usecounter{steps}
\setlength{\labelwidth}{2cm}\setlength{\leftmargin}{2.6cm}
\setlength{\labelsep}{0.5cm}\setlength{\rightmargin}{1cm}
\setlength{\parsep}{0.5ex plus0.2ex minus0.1ex}
\setlength{\itemsep}{0ex plus0.2ex minus0pt}\relax \slshape %
}
\item Melt the butter and dark chocolate
\item Prepare the egg and sugar mix
\item Cool the butter and dark chocolate
\item Set out the milk and white chocolate
\item Prepare the brownie tin
\end{list}

\reset
\begin{list}{\bfseries\upshape A \arabic{steps}:}
{%
\usecounter{steps}
\setlength{\labelwidth}{2cm}\setlength{\leftmargin}{2.6cm}
\setlength{\labelsep}{0.5cm}\setlength{\rightmargin}{1cm}
\setlength{\parsep}{0.5ex plus0.2ex minus0.1ex}
\setlength{\itemsep}{0ex plus0.2ex minus0pt}\relax \slshape %
}
\item Melt the butter and dark chocolate
\item Prepare the egg and sugar mix
\item Cool the butter and dark chocolate
\item Set out the milk and white chocolate
\item \lorem
\end{list}


Using either |newenvironment| we can

\emphasize{recipe}
\begin{texcode}{Creating a new named List}{ex:newlist}
\newenvironment{recipe}{\list{\bfseries\upshape Step \arabic{steps}:}
{%
\usecounter{steps}
\setlength{\labelwidth}{2cm}\setlength{\leftmargin}{2.6cm}
\setlength{\labelsep}{0.5cm}\setlength{\rightmargin}{1cm}
\setlength{\parsep}{0.5ex plus0.2ex minus0.1ex}
\setlength{\itemsep}{0ex plus0.2ex minus0pt}\relax \slshape %
}}
{\endlist}

\begin{recipe}
\item Have a nice afternoon\ldots
\end{recipe}
\end{texcode}

\newenvironment{recipe}{\list{\bfseries\upshape Step \arabic{steps}:}
{%
\usecounter{steps}
\setlength{\labelwidth}{2cm}\setlength{\leftmargin}{2.6cm}
\setlength{\labelsep}{0.5cm}\setlength{\rightmargin}{1cm}
\setlength{\parsep}{0.5ex plus0.2ex minus0.1ex}
\setlength{\itemsep}{0ex plus0.2ex minus0pt}\relax \slshape %
}}
{\endlist}

\begin{recipe}
\item Have a nice afternoon\ldots
\end{recipe}

% something gone terribly wrong, need to restore the settings
\makeatletter
%\@restorepar\@nobreakfalse\@nmbrlistfalse
\let\par\@@par
\makeatother


\subsection{trivlist}

The simplest of all lists is |\trivlist|. In simple terms the |trivlist| environment turns each item into a paragraph and thus it is easier to apply formatting information to a list of paragraphs or items if you want to think of them this way. As it carries common information to all lists, it is used to build up more complicated structures. A good use of trivlists is to simplify the writing of table heads and produce semantic table environments. They are also used to define teh spacing around the verbatim environments. A simple author use is shown in Example~\ref{ex:trivlists}.

\begin{texexample}{Trivlists}{ex:trivlists}
\newenvironment{name}
  {\trivlist\item
   \tabular{@{}ll@{}}}
  {\endtabular\endtrivlist}
  
% test it  
\begin{name}
   First    & Mary  \\
   Second   & Jones \\
   Nickname & --- \\
\end{name}
 

But whatever you call the comma, is it right or wrong? There’re fair arguments on both sides.  One might be concerned about limiting ambiguity. Alas, including the Oxford comma can lead to ambiguity, but omitting it can lead to ambiguity as well.  Consider (3) and (4):
\begin{trivlist}
\item[(3a)] I own pictures of my friends, Hugh Grant, and Dolly Parton.
\item[(3b)] I own pictures of my friends, Hugh Grant and Dolly Parton.
\item[] 
\item[(4a)] I am writing to my Congresswoman, Alia Shawkat, and Michael Cera.
\item[(4b)] I am writing to my Congresswoman, Alia Shawkat and Michael Cera.
\end{trivlist}
\end{texexample}
           


The general parameters affecting a general list is shown in the  diagram  below\footnote{Produced using the \texttt{layouts} package.}. LaTeX offers three general list structures, enumerate, itemize and description.
%\begin{figure}[hp]
%\listdiagram
%\caption{Layout of an \texttt{enumerate} list} \label{fig:lstenum}
%\end{figure}

\section{The list geometry}

You can draw a list diagram as shown below, using the function \docAuxCmd{drawlistdiagram} from
the \pkgname{xlayouts} package, which is bundled with the \pkgname{phd} package.


\begin{figure}[htbp]
\bgroup
\centering
\cxset{geometry units=mm}
\drawlistdiagram
\caption[List geoemetry]{\latexe list diagram.}
\egroup
\end{figure}


List in \latexe are shaped using |\parshape|. Sometimes as you change parameters things react unintuitevely. An easy wway to remember is that the parameter |\leftmargin| determines the first line of the indentation and |\linewidth| is the
|hsize-leftmargin-rightmargin|



What is important to notice here is that all the standard list parameters are left essentially unchanged. The only item that is affected is \refCom{makelabel}, which is redefined in \lstinline{description} label.

\section{Packages}


 Most journals develped their own lists and hard-wired them. Current packages are:
 \pkg{enumitem}, \pkg{enumerate}, \pkg{paralist}.



\paragraph{Enumerate} This package gives the \refEnv{enumerate} environment an optional argument which determines
 the style in which the counter is printed. An occurence of one of the tokens 
 \textbf{A a I i } or \textbf{1} produces the value of the counter printed with
 (respectively) \cmd{\Alph} \cmd{\alph} \cmd{\Roman} \cmd{\roman} or \cmd{\arabic}. 
 These letters may be surrounded by any string involving any other \tex expressions, 
 however the tokens \textbf{A a I i 1} must be inside a \{\} group if they are
 not to be taken as special.

\begin{texcode}{Example using the package enumerate}{ex:enum}
 \begin{enumerate}[EX i.]
   \item one one one one one one one
         one one one one\label{LA}
   \item two
      \begin{enumerate}[{example} a)]
        \item one of two one of two
          one of two\label{LB}
        \item two of two
       \end{enumerate}
   \item two of two
 \end{enumerate}
 

 \begin{enumerate}[{A}-1]
 \item one\label{LC}
 \item two
 \end{enumerate}
\end{texcode}

 This package minimally changes the original \latexe definitions. It is very convenient when you
 want now and then to change labels in a document. The central idea behind the |phd| group of
 packages and eponymous classes, is that there is a distinction between the author, template designer and 
 programmer. The packages at the level of the author always use a key value interface, that
 limits the involvement of the author and to a large extend the template designer to that of 
 setting a number of keys. 

\paragraph{Babel} The babel package\footcite{babel} will redefine enumerate for a number of languages such as french and greek. It is best in these cases, if you still need to use a different list to create a new list with one of the other packages such as enumitem or using the phd-lists package.
\bgroup
\selectlanguage{french}
\extrasfrench

\lorem
\begin{frenchenumerate}
\item \lorem
      \lorem
        \begin{frenchenumerate}
         \item Something
         \item \lorem
               
               \lorem
        \end{frenchenumerate}
\item \lorem
\end{frenchenumerate}

\lorem


\egroup
\paragraph{phd-lists} The package, which is described in the next chapter uses a key value approach in setting new lists
and adjusting their parameters.

\begin{texexample}{Example using phd-lists}{ex:phd-lists}
\cxset{enumerate numberingi   = Alpha,
       enumerate numberingii  = alpha,
       enumerate numberingiii = Roman,}

 
 \begin{enumerate}
   \item This is the top level. \lorem
      \begin{enumerate}
        \item This is the second level. \lorem
          \begin{enumerate}
            \item This is the third level. \lorem
          \end{enumerate}. 
      \end{enumerate}
 \end{enumerate}
\end{texexample}

\vfill
\endinput

\section{Creating new description like environments}

The macro \docAuxCmd*{NewDescriptionEnvironment} can be used to redefine new description like environments, using a key value interface.

%\begin{texexample}{Define a new description list environment}{ex:newdesclist}
%% define the orange description environment
%\NewDescriptionEnvironment[description centered]{orangedescription}
%
%% Sample
%The \texttt{orangedescription} environment in action.
%
%\begin{orangedescription}
%
% \item[One] \lorem
% \item[Two] \lorem
% \item[Three] \lorem
%
%\end{orangedescription}
%
%\lorem
%
%\makeatother
%\end{texexample}



\section{Example: redefining a description list}
We will now develop a description environment, that can be useful for the documentation of packages to describe options. We will use a description list as the basis of the environment. We define the following key values.
|\itemindent-\leftmargin|

\section{Enumerated lists}


\begin{enumerate}
\item one
\item two
\item three
\end{enumerate}

Enumerated (numbered) list environments are characterized by numbering. They use a variety of fields and counters as shown in table.

\subsection{Vertical skips}

By default LaTeX adds vertical skips, as shown in figure 1. The definition of these skips is influenced by the font size and are defined in the \texttt{bk10.clo} files, hence hard to find and change. Each level of the list has its own definition as \lstinline{\@listi}.

\bigskip
\tcbset{width=\linewidth,arc=1mm,before=\bigskip,left=8mm}

\begin{teXXX}
\def\@listi{\leftmargin\leftmargini
            \parsep 40pt plus20pt minus0pt
            \topsep 80pt plus20pt minus40pt
            \itemsep40pt plus20pt minus0pt}
\let\@listI\@listi
\@listi

\def\@listii {\leftmargin\leftmarginii
              \labelwidth\leftmarginii
              \advance\labelwidth-\labelsep
              \topsep    40pt plus20pt minus0pt
              \parsep    20pt plus0pt  minus0pt
              \itemsep   \parsep}
\def\@listiii{\leftmargin\leftmarginiii
              \labelwidth\leftmarginiii
              \advance\labelwidth-\labelsep
              \topsep    20pt plus0ptminus0pt
              \parsep    1pt
              \partopsep 0pt plus\z@ minus0pt
              \itemsep   \topsep}
\def\@listiv {\leftmargin\leftmarginiv
              \labelwidth\leftmarginiv
              \advance\labelwidth-\labelsep}
\def\@listv  {\leftmargin\leftmarginv
              \labelwidth\leftmarginv
              \advance\labelwidth-\labelsep}
\def\@listvi {\leftmargin\leftmarginvi
              \labelwidth\leftmarginvi
              \advance\labelwidth-\labelsep}
\end{teXXX}


\begin{texexample}{Compact Styles}{ex:compact2}
\makeatletter
\cxset{enumerate compact/.style={%
  enumerate numberingi=alpha,
  enumerate numberingii=roman,
  enumerate numberingiii=alpha,
  enumerate numberingiv=roman,
  enumerate labeli punctuation=,
  enumerate label left=(,
  enumerate label right=),
  enumerate leftmargini=2.2em,
  enumerate leftmarginii=2.1em,
  enumerate leftmarginiii=1.5em,
  enumerate leftmarginiv=2em,
  listi topsep=8pt plus2pt minus0pt,
  listi itemsep=0pt plus2pt minus0pt,
  listi parsep=0pt plus2pt minus0pt,
  listi parindent=0pt,
  listii parindent=1em,
  listiii parindent=1em,
  listii topsep=0pt plus2pt minus0pt,
  listii itemsep=0pt plus2pt minus0pt,
  listii parsep=0pt plus2pt minus0pt,
  listiii topsep=0pt plus2pt minus0pt,
  listiii itemsep=0pt plus2pt minus0pt,
  listiv parindent=0pt,
  listiv parsep=0pt plus2pt minus0pt,
  listiv parsep=0pt plus2pt minus0pt,
  listiv topsep=0pt plus2pt minus0pt,
  listiv itemsep=0pt plus2pt minus0pt,
  listiv parsep=0pt plus2pt minus0pt,
}}
\makeatother


\begin{enumerate}
\item Does this project actually merit the use of the Minor Works Form or Intermediate Form instead of their `grown up' relatives?
\item Do the number of PC or prime cost items mean that it would be more desirable to use a re-measurable form?
\item Is this a contract which merits the production of full scale bills
of quantities or is something more standardised going to suffice?
\end{enumerate}
\end{texexample}



As you will observe the numbering in the above example has been enclosed in round brackets, using:


\begin{texcode}{Bracketting a numeral}{ex:brackets}
  enumerate label left=(,
  enumerate label right=),
\end{texcode}


The next example is from the \textit{LaTeX Companion}. In example~\ref{ex:companion}, the first-level list elements are decorated with the section sign (\S) as a prefix and a period as a suffix (omitted in references). We will
define this as a style named \textit{paragraphsymbol} for the lack of any better name. This style can sometimes be found in legal texts.

\begin{texexample}{Paragraph symbols in enumerate}{ex:companion}
\cxset{paragraphsymbol/.style={%
  enumerate numberingi=arabic,
  enumerate labeli punctuation=.,
  enumerate label left=\S,
  enumerate label right=,
}}

\setenumerate{paragraphsymbol}
\begin{enumerate}
\item \lorem
\item \lorem
\item \lorem
\end{enumerate}
\end{texexample}


%\section{Creating enumerated environments}
%
%New enumerated environments cab be created by using the macro \lstinline{\newenumeratedenvironment}. Keys are set as either styles or individually.
%
%%% verbatim from latex
%
%
%\begin{texexample}{An enumerated list factory}{ex:listfactory}
%
%
%\makeatletter
%\newenvironment#1#2{
%#2\enumerate}{\endenumerate}
%\makeatother
%
%\newenumeratedenvironment{paragraphsymbol}{
%  enumerate numberingi=alpha,
%  enumerate numberingii=roman,
%  enumerate numberingiii=alpha,
%  enumerate numberingiv=roman,
%  enumerate labeli punctuation=,
%  enumerate label left=(,
%  enumerate label right=),
%  enumerate leftmargini=2.2em,
%  enumerate leftmarginii=2.1em,
%  enumerate leftmarginiii=1.5em,
%  enumerate leftmarginiv=2em,
%  listi topsep=8pt plus2pt minus0pt,
%  listi itemsep=0pt plus2pt minus0pt,
%  listi parsep=0pt plus2pt minus0pt,
%  listi parindent=0pt,
%  listii parindent=1em,
%  listiii parindent=1em,
%  listii topsep=0pt plus2pt minus0pt,
%  listii itemsep=0pt plus2pt minus0pt,
%  listii parsep=0pt plus2pt minus0pt,
%  listiii topsep=0pt plus2pt minus0pt,
%  listiii itemsep=0pt plus2pt minus0pt,
%  listiv parindent=0pt,
%  listiv parsep=0pt plus2pt minus0pt,
%  listiv parsep=0pt plus2pt minus0pt,
%  listiv topsep=0pt plus2pt minus0pt,
%  listiv itemsep=0pt plus2pt minus0pt,
%  listiv parsep=0pt plus2pt minus0pt,
%  enumerate numberingi=alpha,
%  enumerate labeli punctuation=.,
%  enumerate label left={\P},
%  enumerate label right=,
%}
%
%
%\begin{paragraphsymbol}
%\item \lorem
%\item \lorem
%\item \lorem
%\end{paragraphsymbol}
%
%\end{texexample}

\section{The description list environment}

You can use the description list environment as you would normally use it with \latexe.

\begin{docEnv} {description} {}
\end{docEnv}

However, a number of settings are available to modify the styling of the environment. These 
include settings for fonts and color, as well as spacing and margins.

\begin{docKey} {description label font-face} { = \meta{font shape} } {initial, default=inherit}
\end{docKey}

\begin{docKey} {description label font-family} { = \meta{font shape} } {initial, default=inherit}
\end{docKey}

\begin{docKey} {description label font-size} { = \meta{font size} } {initial, default=inherit}
\end{docKey}

\begin{docKey} {description label font-weight} { = \meta{font weight} } {initial, default=inherit}
\end{docKey}

\begin{docKey} {description label font-shape} { = \meta{font shape} } {initial, default=inherit}
\end{docKey}

\begin{docKey} {description label color} { = \meta{color name} } {initial, default=thedescriptionlabelcolor}
  Setts the description label color
\end{docKey}

\begin{docKey} {description label sep} { = \meta{dim} } {initial, default = 1em}
\end{docKey}

\begin{docKey} {description label width} { = \meta{dim} } {initial, default = 1em}
\end{docKey}

\begin{docKey} {description margin left} { = \meta{dim} } {initial, default = 0em}
\end{docKey}

\begin{docKey} {description margin right} { = \meta{dim} } {initial, default = 0em}
\end{docKey}

\begin{docKey} {description item indent} { = \meta{dim} } {initial, default = 0em}
\end{docKey}

Unlike the enumerate and itemize environment, the description list environment is defined in the book class.
The environment is defined as:

\refCom{descriptionlabel} sets the typesetting of the description label.
\section{Itemized lists}

The itemized \LaTeX\ lists are similar to those for the enumerated lists. However they are somehow simpler as there is no need for counters.

\bigskip
\begin{tcolorbox}[width=\linewidth,arc=2mm,title=Default \LaTeX\ parameters for itemized lists]
\begin{lstlisting}
\newcommand\labelitemi{\textbullet}
\newcommand\labelitemii{\normalfont\bfseries \textendash}
\newcommand\labelitemiii{\textasteriskcentered}
\newcommand\labelitemiv{\textperiodcentered}
\end{lstlisting}
\end{tcolorbox}




\cxset{red/.style={
 labelitemi={{\color{green}\ding{'64}}},
 labelitemii=\color{red}\textendash,
 labelitemiii=\textasteriskcentered,
 labelitemiv=\textperiodcentered,
}}

Now that we have managed to abstract the itemized environment we can generate a new environment factory.

\makeatletter
\def\newitemizedenvironment#1#2{
\@itemdepth=0
\expandafter\def\csname#1\endcsname{%
 \cxset{#2}%
 \ifnum \@itemdepth >\thr@@\@toodeep\else
 \advance\@itemdepth\@ne
 \edef\@itemitem{labelitem\romannumeral\the\@itemdepth}%
 \expandafter
 \list
 \csname\@itemitem\endcsname
 {\def\makelabel####1{\hss\llap{####1}}}%
 \fi}
 \expandafter\let\csname end#1\endcsname=\endlist
}
\makeatother

%\newitemizedenvironment{reditemize}{black}
%
%
%\begin{reditemize}
%\item Test.
%   \begin{reditemize}
%    \item test.
%   \end{reditemize}
%\end{reditemize}
%
%\begin{itemize}
%\item Level i
%      \begin{itemize}
%       \item Level ii
%          \begin{itemize}
%            \item Level iii
%              \begin{itemize}
%                \item Level iv. \lipsum*[1]
%              \end{itemize}
%          \end{itemize}
%      \end{itemize}
%\end{itemize}


\section{Itemized lists with ding symbols}

So far we have used both standard symbols as well as those provided by the pifont that offers numerous,
dingbang symbols. The pifont package also offers environments to do that more easily.


\begin{texcode}{dinglist}{ex:dinglists}
\begin{dinglist}{"E4}
\item The first item. \item The second
item in the list.
\end{dinglist}

\end{texcode}

This will give us:

\begin{dinglist}{"E4}
\item The first item. \item The second
item in the list.
\end{dinglist}



%\begin{dingautolist}{'30}
%\item The first item in the list.\label{lst:a}
%\item The second item in the list.\label{lst:b}
%\item The third item in the list.\label{lst:c}
%\item The fourth item in the list.\label{lst:d}
%\end{dingautolist}
%
%\newenvironment{steps}{\dingautolist{'30}}{\enddingautolist}
%
%\begin{steps}
%\item The first item in the list.\label{lst:a}
%\item The second item in the list.\label{lst:b}
%\item The third item in the list.\label{lst:c}
%\item The fourth item in the list.\label{lst:d}
%\end{steps}


\endinput

\def\start@SFBbox{\@next\@currbox\@freelist{}{}%
 \global\setbox\@currbox
 \vbox\bgroup
  \hsize \textwidth
  \@parboxrestore
}
\def\finish@SFBbox{\par\vskip -\dbltextfloatsep
  \egroup
  \global\count\@currbox\tw@
  \global\@dbltopnum\@ne
  \global\@dbltoproom\maxdimen\@addtodblcol
  \global\vsize\@colht
  \global\@colroom\@colht
}

\newif\ifSFB@keywords
\def\keywords{\if@twocolumn
  \start@SFBbox\@keywords
 \else
  \@keywords
 \fi
}
\def\@keywords{\list{}{%
    \labelwidth\z@ \labelsep\z@
    \leftmargin 11pc\rightmargin\z@  % was 11pc\right....
    \parsep 0pt plus 1pt}\item[]\reset@font\large{\bf Key words: }%
}
\def\endkeywords{\if@twocolumn
  \endlist\addvspace{37pt plus 0.5\baselineskip}\finish@SFBbox
 \else
  \endlist
\fi










 

\DocInput{\jobname.dtx}
\printindex
 %
% 
\end{document}
%</driver>
% \fi
% 
%  \CheckSum{0}
%  \CharacterTable
%  {Upper-case    \A\B\C\D\E\F\G\H\I\J\K\L\M\N\O\P\Q\R\S\T\U\V\W\X\Y\Z
%   Lower-case    \a\b\c\d\e\f\g\h\i\j\k\l\m\n\o\p\q\r\s\t\u\v\w\x\y\z
%   Digits        \0\1\2\3\4\5\6\7\8\9
%   Exclamation   \!     Double quote  \"     Hash (number) \#
%   Dollar        \$     Percent       \%     Ampersand     \&
%   Acute accent  \'     Left paren    \(     Right paren   \)
%   Asterisk      \*     Plus          \+     Comma         \,
%   Minus         \-     Point         \.     Solidus       \/
%   Colon         \:     Semicolon     \;     Less than     \<
%   Equals        \=     Greater than  \>     Question mark \?
%   Commercial at \@     Left bracket  \[     Backslash     \\
%   Right bracket \]     Circumflex    \^     Underscore    \_
%   Grave accent  \`     Left brace    \{     Vertical bar  \|
%   Right brace   \}     Tilde         \~}
%
%
%
% \changes{1.0}{2013/01/26}{Converted to DTX file}
% \changes{1.0}{2017/08/08}{Update to latest LaTeX distribution}
%
% \DoNotIndex{\newcommand,\newenvironment}
% \GetFileInfo{phd.dtx}
% 
%  \def\fileversion{v1.0}          
%  \def\filedate{2012/03/06}
%  \title{The \textsf{phd} package.
%  \thanks{This
%        file (\texttt{phd.dtx}) has version number \fileversion, last revised
%        \filedate.}
%  }
%  \author{Dr. Yiannis Lazarides \\ \url{yannislaz@gmail.com}}
%  \date{\filedate}
%
%
% 
% ^^A\maketitle
% 
% ^^A\frontmatter
%  ^^A\coverpage{./images/hine02.jpg}{Book Design }{Camel Press}{}{}
%  \newpage
% ^^A\secondpage
% \pagestyle{empty}
%
%
% 
%
%
% \pagestyle{headings}
% \raggedbottom
%  \OnlyDescription
%
%  ^^A\StopEventually{\printindex}

% \CodelineNumbered
% \pagestyle{headings}
% 
% 
% ^^A\part{IMPLEMENTATION AND FRIENDS}
% 
%
% \chapter{PHD-Quote Package Code Implementation Objectives and Strategy}
% 
% \epigraph{
% I was reflecting on the convoluted Java frameworks widely adopted at work. Those hefty frameworks brought coding structures and conventions to large engineering teams; meanwhile, they also sucked the fun of programming like a Pastafarian monster slurping all the tomato sauce on a plate of spaghetti.
%}{\href{http://blog.zmxv.com/2015/07/code-golf-at-google.html}{Zhen Wang}}
%
% We start by outlining what we are trying to achieve with this package:
% \cxset{enumerate numberingi =arabic,
%        enumerate labeli punctuation=,}
% \begin{enumerate}
% \item To provide a declarative interface to enable users to modify headings by
%       setting keys, rather than writing macros.  
% \item The interface must be able to manupulate properties of headings down to
%       the last detail.
% \item To provide a compatibility mode, where documents wishing to test the package
% can have an easy switch to switch in and out. This is also important for the testing of the package.
% \item To provide a number of templates that cover most of the typical use case.
% \item To provide means for a plug-in architecture for extensions.

% \end{enumerate}
% 
% \section{Terminology}
%
%  \begin{description}
%  \item [document] Any written item, as a book, article, or letter, especially 
%                  of a factual or informative nature.
%  \item [heading] A division of a document or document series. For a normal
%        book headings are chapters, sections etc. However we allow for
%        specifying a more complex document divided into books, volumes
%        parts etc. For example the Bible has Books, chapters and verses,
%        where a legal document might require divisions such as clauses.
%        In general these divisions are numbered. These document divisions
%        are stored in the comma list \refCom{phd_book_divisions_clist}.
%  \item [head] A typeset heading, such as chapter head, or section head.
%        This can include a counter, label and title for example, 
%        \emph{Chapter 1 Introduction}.
%  \item [dom] This is a programming interface that provides a structured
%        representation of the document (a tree) and it defines a way
%        that the structure can be accessed. Although \latexe does not
%        offer a standard way to build such a tree (mainly because
%        \tex does not require the marking of paragraphs, it is 
%        useful to think of the document as a tree structure. We also
%        allow for a semi-automated way to build such a tree (with the 
%        exception that paragraphs are not included).
% \item [element] A part of the document tree that can be styled on
%       its own. For example the chapter label, or the section number.
%
% \end{description}
%
% \section{Users}
%  We classify users according to the \LaTeX3 terminology as a) programmers b) template designers
%  and c) authors.
% \subsection{Author}
%  We assume that the author has an exising template which she is using but might want to do
%  some minor modifications, for example use an italic shape for the font of the mark, but an 
%  upright font for the page numbers. 
%
% {\obeylines 
%~~ |\cxset|
%~~~~~|{|
%~~~~~~~~\textit{chapter number color}~~|format          = apa,|
%~~~~~~~~\textit{section title font-size} |font-size   = Large,|
%~~~~~|}|
%}  
%
% We follow the idea of representing the basic elements of documents
% as elements, each one having a parent in order to specify
% the element we need to style as accurate as possible. One can think of
% this approach being congruent with objects in other languages.
% As a matter fact nothing stops us from defining a key value
% interface as shown below.
%
% {\obeylines 
%~~ |\cxset|
%~~~~~|{| 
%~~~~~~~~\textit{header.even.mark.font.size}   = |Large,|
%~~~~~~~~\textit{header.even.mark.font.family} = |serif,|
%~~~~~|}|
%}  
%
% This would pehaps make it easier for the template designer, but I have rejected
% the idea as my aim is to make it easy for the author, who can search the template
% and just enter a couple of new proerty values.
%
% \subsection{Template designer}
% \pagestyle{headings}
% The template designer in the example above would have selected the format style
% from a number of predefined formats (templates) or would have created a style
% called \textit{apa} from an existing template and modified it using declarative
% key style.
%
% \subsection{The programmer}
%
% The programmer in the example above could have created the basic format
% \textit{apa} by using both declarative as well as defining or using existing
% macros. To the programmer we offer an extension mechanism, where the contents
% of a |ps@| command are defined. For example the programmer can define a new
% style using \tikzname, but without having to worry about defining full |ps@|
% and their interface.
%
% \section{Preliminaries}
%
%  Standard file identification. We first announce the package 
%	 and require that it be used with \LaTeX2e. 
% \iffalse
%<*QUOTE>
% \fi
%  
%
%    \begin{macrocode}
\NeedsTeXFormat{LaTeX2e}[2017/04/15]%
\ProvidesFile{phd-quote}[2017/04/15 v1.0 quotation management (YL)]%
%    \end{macrocode}
%
% 
% \section{Source2e Interface}
% 
% I am not very fond of mixing expl3 control sequences with source2e commands. Here
% we provide an interface for all these commands we might use. 
% This section can be revisited once expl3 code becomes available.
%
%    \begin{macrocode}
\let\ltxtoday\today
\newif\if@ltxcompat \@ltxcompatfalse
\renewenvironment{quotation}
               {\par\addvspace{6pt}
                \list{}{\listparindent12pt %
                        \leftmargin=\quotationleftmargin@cx %
                        \itemindent    \quotationparindent@cx
                        \rightmargin   \quotationrightmargin@cx
                        \parsep \quotationparsep@cx 
                        \expandafter\csname thequotationfontsize\endcsname   %
                        \quotationfontname@cx\relax } 
                \item\relax\hspace{0pt}\relax }
               {\endlist}
 \let\latexquotation\quotation
\let\endlatexquotation\endquotation
\let\latexquote\quote
\let\endlatexquote\endquote
%    \end{macrocode}
%
% \section{Key Management}
%
% For all keys we use the \pgfname package.
%
%    \begin{macrocode}
\ExplSyntaxOn
\cxset{
  quotation~above/.store~in=\quotationabove@cx,
  quotation~left~margin/.store~in=\quotationleftmargin@cx,
  quotation~right~margin/.store~in=\quotationrightmargin@cx,
  quotation~parsep/.store~in=\quotationparsep@cx,
  quotation~font-size/.fontsize= thequotationfontsize,
  quotation~parindent/.store~in=\quotationparindent@cx,
  quotation~font-name/.store~in=\quotationfontname@cx,
 }
\ExplSyntaxOff 
 \cxset{quotation default/.style={%
  quotation above=0pt, 
  quotation left margin=25pt,
  quotation right margin=25pt,
  quotation parsep=2pt,
  quotation font-size=normal,
  quotation parindent=4em,
  quotation font-name=\arial,}} 
 \cxset{quotation default}
%    \end{macrocode}
% 
%
%    \begin{macrocode}

\cxset{quotation acmtrans2e/.style={
  quotation above = 5pt plus 2pt minus 1pt,
  quotation left margin=2pc,
  quotation right margin=2pc,
  quotation parindent=10pt,
}}

\cxset{quotation ametsoc/.style={
  quotation above = 5pt plus 2pt minus 1pt,
  quotation left margin=2pc,
  quotation right margin=2pc,
  quotation parindent=1.5em,
  %quotation item indent = 1.5em, % 
}}


\cxset{quotation phil/.style={
  quotation above = 5pt plus 2pt minus 1pt,
  quotation left margin=13.5pt,
  %quotation right margin=13.5pt,
  quotation right margin=13.5pt,
  quotation parindent=1.5em,
  %quotation item indent = 1.5em, % 
}}

\cxset{quotation tugboat/.style={
  quotation above = 5pt plus 2pt minus 1pt,
  quotation left margin=4em,
  quotation right margin=1em,
  quotation parindent=1.5em,
  %quotation item indent = 1.5em, % 
}}

\cxset{quotation wide/.style={
  quotation above = 5pt plus 2pt minus 1pt,
  quotation left margin=3em,
  quotation right margin=3em,
  quotation parindent=1.5em,
  %quotation item indent = 1.5em, % 
}}
\def\longitem{\list{---}{\labelwidth\z@
 \leftmargin\z@ \itemindent\parindent \advance\itemindent\labelsep}}
\let\endlongitem\endlist
%    \end{macrocode}
% \hrule
% \lorem 
% \hrule
% \cxset{quotation acmtrans2e}
% 
% \begin{quotation}
% 
% \lorem 
% \lipsum[1]
% \lorem
% \hrule
% \end{quotation}
% \hrule
%
% \begin{longitem}
% \item \lorem
% \item \lorem
% \end{longitem}
%
% \section{Philosophers Imprint Style}
% \lorem
% \cxset{quotation phil}
%
% \begin{quotation}
% \lorem
%
% \lorem
% \end{quotation}
%
% \subsection{TUGboat}
%
% This style is interesting as it has uneven left and right margins.
%
% \lorem
% 
% \cxset{quotation tugboat}
% \begin{quotation}
% \lorem
%
% \lorem
% \end{quotation}
% 
% 
% \subsection{International Journal of Digital Curation (dcpaper)}
%\cxset{quotation wide}
%\begin{quotation}
%‘Cras porttitor dictum lacus. Class aptent taciti sociosqu ad litora torquent per conubia nostra, per inceptos hymenaeos. In consectetuer, diam at volutpat elementum, libero lectus pulvinar sem.’ (Borgman, 2007)
%\end{quotation}
%
% 
% \begin{docCommand} {setquote} { \meta{keys} }

% \end{docCommand}
%    \begin{macrocode}
\cxset{
  quote above/.store in=\quoteabove@cx,
  quote left margin/.store in=\quoteleftmargin@cx,
  quote right margin/.store in=\quoterightmargin@cx,
  quote parsep/.store in=\quoteparsep@cx,
  quote font-size/.store in=\quotefontsize@cx,
  quote parindent/.store in=\quoteparindent@cx,
  quote font-name/.store in=\quotefontname@cx,
 }

\def\setquote#1{%
  \cxset{#1}
  \renewenvironment{quote}
               {\par\addvspace{\quoteabove@cx}
                \list{}{\listparindent\quoteparindent@cx%
                        \leftmargin=\quoteleftmargin@cx%
                        \itemindent  \listparindent
                        \rightmargin\leftmargin
                        \parsep=\quoteparsep@cx%
                        \quotefontsize@cx\quotefontname@cx}%
                \item\relax\hskip-\listparindent}
               {\endlist}
  }

%\setquote{%
%  quote above=0pt,
%  quote left margin=50pt,
%  quote parsep=0pt,
%  quote font-size=\small,
%  quote parindent=12pt,
%  quote font-name=,
% }
%    \end{macrocode}
% \begin{quote}
%   \lorem
%
%   \lorem
% \end{quote}
%</QUOTE>
\endinput

% 
%