% \iffalse meta-comment
%
% Copyright 1993-2014
%
% The LaTeX3 Project and any individual authors listed elsewhere
% in this file.
%
% This file is part of the Standard LaTeX `Tools Bundle'.
% -------------------------------------------------------
%
% It may be distributed and/or modified under the
% conditions of the LaTeX Project Public License, either version 1.3c
% of this license or (at your option) any later version.
% The latest version of this license is in
%    https://www.latex-project.org/lppl.txt
% and version 1.3c or later is part of all distributions of LaTeX
% version 2005/12/01 or later.
%
% The list of all files belonging to the LaTeX `Tools Bundle' is
% given in the file `manifest.txt'.
%
% \fi
%
%\iffalse   % this is a METACOMMENT !
%
%
%% Package `verbatim' to use with LaTeX2e
%% Copyright (C) 1989--2003 by Rainer Sch\"opf. All rights reserved.
%
% Copying of this file is authorized only if either
% (1) you make absolutely no changes to your copy, including name, or
% (2) if you do make changes, you name it something other than
%     verbatim.dtx.
% This restriction helps ensure that all styles developed here
% remain identical.
%
%
%
% \section{Producing the documentation}
%
% We provide a short driver file that can be extracted by the
% \textsf{DocStrip} program using the conditional `\textsf{driver}'.
%
%    \begin{macrocode}
%<*driver>

\documentclass[book,oneside,10pt,a4paper,
               microtype=on]{phddoc}
               
               
%%%
%% This is file `phd-documentation-defaults.def',
%% generated with the docstrip utility.
%%
%% The original source files were:
%%
%% phd-fontmanager.dtx  (with options: `DFLT')
%% phd-colorpalette.dtx  (with options: `DFLT')
%% phd-lowersections.dtx  (with options: `DFLT')
%% phd-toc.dtx  (with options: `DFLT')
%% phd-documentation.dtx  (with options: `DFLT')
%% ----------------------------------------------------------------
%% phd --- A package to beautify documents.
%% E-mail: yannislaz@gmail.com
%% Released under the LaTeX Project Public License v1.3c or later
%% See http://www.latex-project.org/lppl.txt
%% ----------------------------------------------------------------
\cxset{
   % settings for document fonts.
    main font-size                 = 10pt,
    main font-face                 = Georgia,
    main sans font-face            = Georgia, %Arial,
    main mono font-face            = B-612,%Source Code Pro, %Consolas, %Apl385, %Consolas,%Source Code Pro,
    chapter label font-face        = Georgia,
    chapter number font-face       = Arial,
    chapter title font-face        = Times New Roman,
    section label font-face        = Arial,
    section number font-face       = Arial,
    section title font-face        = Arial,
    subsection label font-face     = Arial,
    subsection number font-face    = Arial,
    subsection title font-face     = Arial,
    subsubsection label font-face  = Arial,
    subsubsection number font-face = Arial,
    subsubsection title font-face  = Arial,
    paragraph label font-face      = Arial,
    paragraph number font-face     = Arial,
    paragraph title font-face      = Arial,
    subparagraph label font-face   = Arial,
    subparagraph number font-face  = Arial,
    subparagraph title font-face   = Arial,
    % default palette
    palette orange sakura,
    part format                       = traditional,
    chapter title margin-top-width    =  0cm,
    chapter title margin-right-width  =  1cm,
    chapter title margin-bottom-width = 10pt,
    chapter title margin-left-width   = 0pt,
    chapter align                     = left,
    chapter title align               = left, %checked
    chapter name                      = chapter,
    chapter format                    = fashion,
    chapter font-size                 = Huge,
    chapter font-weight               = bold,
    chapter font-family               = sffamily,
    chapter font-shape                = upshape,
    chapter color                     = black,
    chapter number prefix             = ,
    chapter number suffix             = ,
    chapter numbering                 = arabic,
    chapter indent                    = 0pt,
    chapter beforeskip                = -3cm,
    chapter afterskip                 = 30pt,
    chapter afterindent               = off,
    chapter number after              = ,
    chapter arc                       = 0mm,
    chapter background-color          = white,
    chapter afterindent               = off,
    chapter grow left                 = 0mm,
    chapter grow right                = 0mm,
    chapter rounded corners           = northeast,
    chapter shadow                    = fuzzy halo,
    chapter border-left-width         = 0pt,
    chapter border-right-width        = 0pt,
    chapter border-top-width          = 0pt,
    chapter border-bottom-width       = 0pt,
    chapter padding-left-width        = 0pt,
    chapter padding-right-width       = 10pt,
    chapter padding-top-width         = 10pt,
    chapter padding-bottom-width      = 10pt,
    chapter number color              = white,
    chapter label color               = black,
    chapter number font-size        = huge,
    chapter number font-weight      = bfseries,
    chapter number font-family      = sffamily,
    chapter number font-shape       = upshape,
    chapter number align            = Centering,
    chapter title font-size        = Huge,
    chapter title font-weight      = bold,
    chapter title font-family      = sffamily,
    chapter title font-shape       = upshape,
    chapter title color            = black,
    section name                   = Section,
    section format                 = traditional,
    section align                  = Centering,
    section title align            = Centering, %checked
    section font-size              = Large,
    section font-weight            = bfseries,
    section font-family            = serif,
    section font-shape             = upshape,
    section number font-size       = Large,
    section number font-weight     = bfseries,
    section number font-family     = serif,
    section number font-shape      = upshape,
    section title font-size        = Large,
    section title font-weight      = bfseries,
    section title font-family      = serif,
    section title font-shape       = upshape,
    section color                  = black,
    section number prefix          = \thechapter.,
    section number suffix          =,
    section numbering              = arabic,
    section indent                 = 0pt,
    section beforeskip             = 3ex,
    section afterskip              = 1.5ex plus .1ex,
    section afterindent            = on,
    section number after           = \quad,
    section arc                            = 3pt,
    section background-color       = white,
    section afterindent                = on,
    section grow left                   = 0mm,
    section grow right                 = 0mm,
    section rounded corners        = northeast,
    section border-left-width      = 0pt,
    section border-right-width     = 0pt,
    section border-top-width       = 2pt,
    section border-bottom-width    = 2pt,
    section padding-left-width     = 0pt,
    section padding-right-width    = 10pt,
    section padding-top-width      = 2pt,
    section padding-bottom-width   = 2pt,
    section title margin-top-width = 2pt,
    section title color            = thesectiontitlecolor,
    section shadow                 = no shadow,
%% sybsection
    subsection name                   = Subsection,
    subsection format                 = hang,
    subsection font-size              = large,
    subsection font-weight            = bfseries,
    subsection font-family            = rmfamily,
    subsection font-shape             = upshape,
    subsection number font-size       = large,
    subsection number font-weight     = bfseries,
    subsection number font-family     = rmfamily,
    subsection number font-shape      = upshape,
    subsection title font-size        = Large,
    subsection title font-weight      = bfseries,
    subsection title font-family      = sffamily,
    subsection title font-shape       = upshape,
    subsection title color            = bgsexy,
    subsection color                  = bgsexy,
    subsection numbering              = arabic,
    subsection align                  = Centering, %checked
    subsection title align            = Centering, %checked
    subsection beforeskip             = -3.25explus -1ex minus -.2ex,
    subsection afterskip              = 1.5ex plus .2ex,
    subsection number prefix          = \thesection.,
    subsection indent                 = 0pt,
    subsection number after           = 0pt,
    subsection background-color       = white,
    subsection border-left-width      = 0pt,
    subsection border-right-width     = 0pt,
    subsection border-top-width       = 5pt,
    subsection border-bottom-width    = 5pt,
    subsection padding-left-width     = 0pt,
    subsection padding-right-width    = 0pt,
    subsection padding-top-width      = 20pt,
    subsection padding-bottom-width   = 20pt,
    subsection shadow                 = drop shadow,
    subsubsection name                    = Subsubsection,
    subsubsection format                  = hang,
    subsubsection background-color        = white, %checked
    subsubsection afterindent             = off,
    subsubsection font-family             = rmfamily,
    subsubsection font-size               = large,
    subsubsection font-weight             = bfseries,
    subsubsection font-family             = tiresias,
    subsubsection font-shape              = upshape,
    subsubsection font-family             = sffamily,
    subsubsection font-size               = large,
    subsubsection font-weight             = bfseries,
    subsubsection font-family             = tiresias,
    subsubsection font-shape              = upshape,
    subsubsection color                   = black,
    subsubsection number prefix           = \thesubsection,
    subsubsection number suffix           = ,
    subsubsection numbering               = arabic,
    subsubsection indent                  = 0pt,
    subsubsection beforeskip              = -3.25explus -1ex minus -.2ex,
    subsubsection afterskip               = 1.5ex plus .2ex,
    subsubsection align                   = center,
    subsubsection title align             = center,
    subsubsection number after     =,
    subsubsection border-left-width       = 0pt,
    subsubsection border-right-width      = 0pt,
    subsubsection border-top-width        = 2pt,
    subsubsection border-bottom-width     = 0pt,
    subsubsection padding-left-width      = 0pt,
    subsubsection padding-right-width     = 0pt,
    subsubsection padding-top-width       = 20pt,
    subsubsection padding-bottom-width    = 20pt,
    subsubsection shadow                  = no shadow,
    subsubsection title font-size         = large,
    subsubsection title font-weight       = bfseries,
    subsubsection title font-family       = serif,
    subsubsection title font-shape        = itshape,
    subsubsection title color             = thesubsectiontitlecolor,
    paragraph name                = paragraph,
    paragraph format              = inline,
    paragraph name                = paragraph,
    paragraph font-size           = large,
    paragraph font-weight         = bfseries,
    paragraph font-family         = rmfamily,
    paragraph font-shape          = upshape,
    paragraph numbering           = alpha,
    paragraph number prefix       = \thesubsubsection,
    paragraph align               = flushleft,
    paragraph beforeskip          = 3.25ex plus1ex minus.2ex,
    paragraph afterskip           = -1em,
    paragraph indent              = 0pt,
    paragraph number after        = \quad,
    paragraph color               = bgsexy,
    paragraph background-color    = white,
    paragraph shadow              = no shadow,
    paragraph afterindent         = off
    subparagraph name             = subparagraph,
    subparagraph format           = inline,
    subparagraph name             = subparagraph,
    subparagraph font-size        = large,
    subparagraph font-weight      = bfseries,
    subparagraph font-family      = rmfamily,
    subparagraph font-shape       = upshape,
    subparagraph color            = bgsexy,
    subparagraph background-color = bgsexy,
    subparagraph numbering        = none,
    subparagraph align            = flushleft,
    subparagraph beforeskip       = 3.25ex plus1ex minus .2ex,
    subparagraph afterskip        = -1em,
    subparagraph indent           = 0pt,
    subparagraph number after     = ,
    %subparagraph shadow           = off,
%% toc contents element settings
    toc name               = Table of Contents,
    toc  before            =,
    toc  after             =,
    toc  numwidth          = 0pt,
    toc  color             = thetocname,
    toc  background-color  = bgsexy!20,
    toc  frame-color       = red,
    toc  shadow            = none,
    toc  font-weight       = normal,
    toc  font-family       = rmfamily,
    toc  font-shape        = upshape,
    toc  font-size         = Huge,
    toc  afterskip         = 30pt,
    toc  after             = ,
    toc  align             = left,
    toc  indent            = 0pt,
    toc case               = none,
    toc  page after        = A,
    toc  pagestyle         = headings,
    toc  rmarg             = 4em,
%% TOC part keys
    toc part font-size    = LARGE,
    toc part color        =  black,
    toc part beforeskip   =  1em,
    toc part page before  =,
    toc part indent       =  0pt,
    toc part numwidth     = 1.5em,
    % table of contents defaults
    % toc chapter keys
    toc chapter font-size   = Large,
    toc chapter font-family = rmfamily,
    toc chapter font-weight = normal,
    % the toc chapter color thetocchapter
    % is fetched from the palette define
    % your own color in the palette rather than
    % change this here
    toc chapter color       = thetocchapter,
    toc chapter beforeskip  =1em,
    toc chapter afterskip   = 12pt plus0.2pt minus .2pt,
    toc chapter case        = upper,
    toc chapter numwidth    = 1.5em,
    %  TOC chapter page formatting
    toc chapter page font-size        = Large,
    toc chapter page font-shape       = upshape,
    toc chapter page font-weight      = normal,
    toc chapter page font-family      = rmfamily,
    toc chapter page color            = black,
    toc chapter page background-color = white,
    toc chapter page before           =,
    toc chapter page after            =,
      %TOC section
      % indentation
       toc section indent=1.5em,
       toc section numwidth= 2.3em,
       toc section beforeskip=0pt,
       toc section afterskip=0pt,
      % page number fonts
       toc  section page font-size          = large,
       toc  section page font-shape         = upshape,
       toc  section page font-weight        = normal,
       toc  section page font-family        = rmfamily,
       % page number colors
       toc  section page color              = bgsexy,
       toc  section page background-color   = white, %theblue!10,
       % page number before after elements
       toc  section page before             =,
       toc  section page after              =,
       toc section page after = ,
       toc section page before =,
%%
%% subsection defaults
    toc subsection indent        = 3.8em,
    toc subsection numwidth      = 3.2em,
    toc subsection page before   = {},
    toc subsection page after    = {},
%%
    % List of Figures
    lof name              = List of Figures,
    lof before            =,
    lof after             =,
    lof numwidth          = 0pt,
    lof color             = thelofname,
    lof background-color  = white,
    lof frame-color       = white,
    lof shadow            = none,
    lof font-weight       = normal,
    lof font-family       = rmfamily,
    lof font-shape        = upshape,
    lof font-size         = Huge,
    lof afterskip         = 40pt,
    lof after             = ,
    lof align             = left,
    lof indent            = 0pt,
    lof case              = none,
    lof page after        = ,
   color command     = themacrocolor,
   color environment = theenvironment,
   color key         = thekey,
   color value       = thevalue,
   color color       = black,%leaks to index
   color option      = theoption,
   color meta        = themeta,
   color frame       = theframe,
   % indexing
   index actual  = {@},
   index quote   = {!},
   index level   = {>},
   index doc settings,
  docexample/.style={colframe=ExampleFrame,colback=ExampleBack,fontlower=\footnotesize},
  documentation minted style=,
  documentation minted options={tabsize=2,fontsize=\small},
  english language/.code={\cxset{doclang/.cd,
    color=color,colors=Colors,
    environment content=environment content,
    environment=environment,environments=Environments,
    key=key,keys=Keys,
    index=Index,
    pageshort={P.},
    value=value,values=Values}},
 }
\endinput
%%
%% End of file `phd-documentation-defaults.def'.

\cxset{palette spring onion,
       subsection afterindent=off}

\let\solution\undefined
%\usepackage{fancyvrb-ex}
\usepackage{tasks}
\usepackage{exsheets}
\usepackage{exsheets-listings}               
\usepackage{verbatim}  
\usepackage{helvet}
%\usepackage[osf]{mathpazo}            


\title{A New Implementation of \LaTeX{}'s \\ \texttt{verbatim}
       and \texttt{verbatim*} Environments.}
\author{Rainer Sch\"opf\\
        \and
        Bernd Raichle\\
        \and
        Chris Rowley}

\date{2001/03/12}
\begin{document}
\pagestyle{headings}
\markboth{Verbatim style option}{Verbatim style option}

\maketitle
\chapter{A more flexible and robust method of defining functions with LaTeX3 and xparse}
\label{ch:xparse}

\section{Introduction}

The \latex2e |\newcommand| macro is the most popular user command for creating macros. The command
provides a number of checks and also has the ability to define macros with an optional argument.
For more complex macros, users have to revert to using |\def| or use packages which extend |\newcommand|,
such as \pkg{twoopt}.\footcite{twoopt}

The \LaTeX3 Team developed the package \pkg{xparse} to provide document level 
authors with some powerful commands that extend those such as \cs{newcommand}
of \latexe. The code is been stable and the interface is not expected to change. 
Although targetted at document level, the commands offered can be used effectively to produce code used in packages.\footnote{\protect\url{http://tex.stackexchange.com/questions/98152/always-use-newdocumentcommand-instead-of-newcommand}} The functions offered by the package enable commands with star, or optional arguments to be produced easily.\tcbdocmarginnote{Revised\\ July 2018} A good introduction to the package was published in TUGboat by Joseph Wright.\footcite{wright2010} 


\section{Creating document commands}

\begin{docCommand}{DeclareDocumentCommand}{\marg{function}\marg{argument specification}\marg{code}}
This family of commands are used to create a document-level \emph{function}. The argument
specification for the function is given by \textit{arg spec}, and expanding to be replaced by the
\textit{code}. Unlike \latex's definition commands, all xparse commands take two arguments.
The first one is the \textit{argument specifier}, and the second is the \textit{code.}
\end{docCommand}

\begin{texexample}{DeclareDocumentCommand}{l3:1}
\DeclareDocumentCommand \foo { m o m } { 
    arg 1 = #1, arg 2 = #2,  arg 3 = #3 }
\foo{A}[B]{C}  

\foo{A}{B}    
\end{texexample}

In the example above |{m o m}| is the argument specifier. It tells the function  to expect, two mandatory arguments and one optional denoted by the letter \textbf{o}. There are many more specifiers. For example \textbf{O} takes an parameter as a default value.\index{argument specifier}

\begin{texexample}{DeclareDocumentCommand}{l3:1}
\DeclareDocumentCommand \foo { m O{\ldots} m } { 
    arg 1 = #1, arg 2 = #2,  arg 3 = #3 }
\foo{A}[B]{C}  

\foo{A}{B}    
\end{texexample}

The argument markers can be entered in any order. In the following example we will also add an optional argument in a curly bracket. Although this is frowned upon in certain contexts it is useful. Consider the case of a chapter title that also has a subtitle. \docAuxCommand*{Chapter}, it maybe more natural and useful to have input of the form, as shown in Example~\ref{l3:g}. 

\begin{texexample}{DeclareDocumentCommand}{l3:g}
\DeclareDocumentCommand \MyChapter { o m g } { 
\centering #2\par #3\par }
    
\MyChapter{THIS IS THE MAIN TITLE}{This is a subtitle}  

\MyChapter[]{THIS IS THE MAIN TITLE}{This is a subtitle}        
\end{texexample}

\begin{texexample}{DeclareDocumentCommand}{l4:g}
\DeclareDocumentCommand \MyChapter {s o m g } { 
\IfBooleanTF {#1} {\gdef\fonta{\bfseries\selectfont}}{\gdef\fonta{}}
\IfNoValueTF {#2} {No option\par}{#2}

\centering {\fonta #3}\par #4\par 
  }   
\MyChapter{ THIS IS THE MAIN TITLE}  

\MyChapter*[short title]{THIS IS THE MAIN TITLE}{This is a subtitle}        
\end{texexample}

As you can see, it is fairly easy to produce starred and unstarred versions of commands as well as as any form of optional arguments. Let us now see some of the other command definition functions, before we continue with other specifiers.

\begin{docCommand}{NewDocumentCommand}{\marg{function}\marg{argument specification}\marg{code}}
will issue an error if \meta{function} has already been defined
\end{docCommand}

\begin{docCommand}{RenewDocumentCommand}{\marg{function}\marg{argument specification}\marg{code}}
For changing a definition,
issuing an error message if the macro does
not already exist.
\end{docCommand}


As the \cmd{\DeclareDocumentCommand} always updates a definition, it is used for the examples in this chapter to avoid any errors.

What sets the above commands apart from \latexe \cmd{\newcommand} is the argument specification.



\begin{texexample}{DeclareDocumentCommand}{l3:1}
\DeclareDocumentCommand \teststar {s o m } { 
\IfBooleanTF {#1}
  { \typesetnormalchapter {#2} {#3} }
  { \typesetstarchapter {#3} }
}  
\newcommand\typesetnormalchapter[2][]{
  normal chapter
}
\newcommand\typesetstarchapter[1]{
  #1
}
\teststar{Test}

\teststar*{test}
\end{texexample}    

    
The argument specification \textbf{m o m} in the example enables the function to accept three arguments, two mandatory and one optional. 


\section{Argument specifications}

The basic idea of an argument specification is that each argument is listed as a single letter. 
As the argument specification is a mandatory argument, a function with no arguments still needs an arg spec. The number of letters in the argument specification tells you how many arguments a function takes, while the letters themeselves determine the type of argument. Unlike |\newcommand| or |\def| a function with no arguments still needs to be specified with an empty
argument specification.

\begin{texexample}{Empty arg spec}{}
\DeclareDocumentCommand\atest{}{some text}
\atest
\end{texexample}

There is a wide range of argument specifier letters. Mandatory arguments are created using the letter \textbf{m}.

\begin{marglist}
\item [m] Mandatory. This is a standard mandatory argument, which can either be a single token alone or multiple tokens surrounded by curly braces. Regardless of the input, the argument will
be passed to the internal code surrounded by a brace pair. This is the \pkgname{xparse} type
specifier for a normal \tex argument.
\end{marglist}

\begin{texexample}{Mandatory Values, verbatim}{}
\DeclareDocumentCommand\testverbatim{ v }{
    \ttfamily#1
}
\testverbatim+ \this is a test +

\testverbatim * &^%$#\test *

\testverbatim{\ttfamily \bfseries\normalfont test}
\end{texexample}

The \textbf{l} specifier reads its argument, until it encounters a left brace. It is equivalent to \tex \# argument. Can be used basically for |\hbox| type comands.

\begin{texexample}{Mandatory arguments l-specifier}{}
\DeclareDocumentCommand\myhbox{ l }{
   \hbox to \dimexpr(#1)\relax
}
\fbox{\myhbox 12pt+1em+13ex  {test}}
\end{texexample}

One difference between |xparse| functions and |\newcommand| is that the functions defined are not |\long|. In |xparse| the argument specidfier can b used to allow paragraph tokens or not. This is done by preceding the arg spec letter by |+|:

\begin{texexample}{Long macros with xparse}{ex:xparse1}
\DeclareDocumentCommand \mylorem {m +m }{%
  #1
  #2
}

\mylorem{\hrule}{\lorem\par}
\end{texexample}

If you expect longer texts, it is always a good idea to make the macros |long|. 

\begin{margoptionslist}
\item [o] Optional argument in  []. Returns |-NoValue-| if not present.
\item [O] As for \textbf{o}, but returns \meta{default}, if no value is given. Should be given as |O{default}|.

\item [s] Starred version
\item [v] Verbatim. Reads an argument “verbatim”, between the following character and its next occurrence,
in a way similar to the argument of the LATEX2" command \cmd{\verb}. Thus
a v-type argument is read between two matching tokens, which cannot be any of
\%, \#, \{, \}, \^ or  . The verbatim argument can also be enclosed between braces,
\{ and \}. A command with a verbatim argument will not work when it appears
within an argument of another function.
\item [l] An argument which reads everything up to the first open group token: in standard
\latex this is a left brace.
\item [u] Reads an argument “until \meta{tokens} are encountered, where the desired \meta{tokens}
are given as an argument to the specifier: |u|\meta{tokens}.
\item [d] An optional argument that is delimited. 
\item [D] As for d, but returns \meta{default} if no value is given: D\meta{token1} \meta{token2}\marg{default}.
Internally, the o, d and O types are short-cuts to an appropriated-constructed D
type argument.
\item [t]  An optional \meta{token}, which will result in a value \cs{BooleanTrue} if \meta{token} is 
            present and \cs{BooleanFalse} otherwise. Given as \meta{token}.
\item [g] An optional argument given inside a pair of \tex group tokens (in standard \latex,
              \{ . . . \}, which returns |-NoValue-| if not present.
\item [G] As  for \textbf{g} but returns \meta{default} if no value is given: |G|\marg{default}.
\end{margoptionslist}

\begin{texexample}{Default Values}{}
\DeclareDocumentCommand\testcolor{ O{red} m }{
    \textcolor{#1}{#2}
}
\testcolor{This is typeset in red}
\testcolor[blue]{This is typeset in blue.}
\end{texexample}

\section{Testing special values}

The optional arguments of a function defined using |xparse| use dedicated variables to return
information about the naure of the argument received.

\begin{docCommand}{IfNoValueTF}{\Arg{argument}\Arg{true code}\Arg{false code}}
The function tests if the argument has a value and executes the true of false code, by means
of a |-NoValue-| marker. 
\begin{texexample}{special values}{}
\DeclareDocumentCommand\doccmd{O{red} m}
    {
        \IfNoValueTF{#1}
            {\doccmdnocolor{#1}}
            {\doccmdcolor{\textcolor{#1}{#2}}}
     }
\newcommand\doccmdnocolor[1]{#1}
\newcommand\doccmdcolor[2]{#1 #2}     
This is \doccmd[blue]{text}  and this is \doccmd{text}.   
\end{texexample}
\end{docCommand}

\begin{texexample}{special values}{}
\DeclareDocumentEnvironment{allbold}{o}
    {
        \bfseries 
        \IfNoValueTF{#1}
            {\color{red}}
            {\color{#1}}
    }
    {                 }
\begin{allbold}[magenta]
\lorem
\end{allbold}
\end{texexample}

\begin{texexample}{variants}{ex:variants}
\ExplSyntaxOn
\cs_set:Npn \foo_something:Nn #1#2 {
   \csname\expandafter#1\endcsname{blue}{a a a} 
   { #2}
  }
\cs_generate_variant:Nn \foo_something:Nn { c }
%\meaning\foo_something:cn
\ExplSyntaxOff
\lorem
\end{texexample}

\section{Robustness}

|xparse| craetes functions which are naturally \enquote{robust}. This means that they can be used in section names
and so on without needing to be protected using
|\protect|. This makes using functions created using
xparse much more reliable than using those created
using |\newcommand|, particularly when there are optional
arguments.
xparse is also designed so that optional arguments
can themselves contain optional material. For
example, if you try:

\begin{teXXX}
\newcommand*\foo[2][]{%
% Code
}
\foo[\baz[arg1]{arg2}]{arg3}
\end{teXXX}

you will find that |\foo| will pick up ‘|\baz[arg1|’ as
\#1 and ‘arg2’ as \#2: not what is intended. However,
the same code with xparse

\begin{teXXX}
\DeclareDocumentCommand \foo { o m } {%
% Code
}
\foo[\baz[arg1]{arg2}]{arg3}
\end{teXXX}
will parse ‘|\baz[arg1]{}|’ as \#1 and ‘arg’ as \#2, as
anticipated.\footcite{wright2010}

\section{Using xparse in Packages}

There is no reason not to use \pkg{xparse} in your packages. The convention here is to use it as:

\begin{teX}
\NewDocumentCommand \bibnotemark { o } {
  \IfNoValueTF {#1} 
    {
      \int_gincr:N \g_@@_note_int
      \@@_mark_note:x { \@@_note_name: }
    }
    { \@@_mark_note:x { \@@_insert_refsection: #1 } }
}
\end{teX}

The strange |\@@| will be replaced by l3doc when the document is processed to print the module name and mark the macros as internal. The code used above is from |notes2bib|.

In a package it is advisable to use the NewDocumentCommand in order to ensure that the command is not overwritten by other packages or the author unintentionally.








\chapter{LaTeX3 Key value system}
\label{l3:keys}
The key-value system has been discussed earlier but avoided to cover the |l3keys| module of \latex3 until such time as the basics of the expl3 syntax was discussed. 


The l3keys modules provides general purpose keyval processing for |expl3| code. However, it does not interact with LaTeX2e's package or class option system. For that, you need to load some additional code, which is available in the package l3keys2e. This provides the \docAuxCommand*{ProcessKeysOptionscommand} to parse class/package options and process them using keyvals defined by l3keys.

The reason for this separation is that l3keys is intended to form part of a LaTeX3 kernel, while l3keys2e is tied to the LaTeX2e model for processing options. It seems extremely likely that a stand-alone LaTeX3 kernel will use keyval options 'natively' but with a different underlying implementation.

 The high level functions here are intended as a method to create
 key--value controls. Keys are themselves created using a key--value
 interface, minimising the number of functions and arguments
 required. Each key is created by setting one or more \emph{properties}
 of the key:
 \begin{verbatim}
   \keys_define:nn { mymodule }
     {
       key-one .code:n   = code including parameter #1,
       key-two .tl_set:N = \l_mymodule_store_tl
     }
 \end{verbatim}
 
  At a document level, |\keys_set:nn| will be used within a
 document function, for example
 \begin{verbatim}
   \DeclareDocumentCommand \MyModuleSetup { m }
     { \keys_set:nn { mymodule } { #1 }  }
   \DeclareDocumentCommand \MyModuleMacro { o m }
     {
       \group_begin:
         \keys_set:nn { mymodule } { #1 }
        ... Main code for the macro
       \group_end:
     }
 \end{verbatim}
 
 The process of incorporating a key value system into a macro or a package involves three steps. First the keys are defined then processed to set them to some values and lastly incorporated into a function or package.
 
 It is best to illustrate the process with a small example. Example\ref{ex:keyval1} defines two keys that affect the typesetting of paragraphs |parindent| and |parskip|. These are defined using the |.code|, pretty much the same way that |pgfkeys| that we discussed earlier defines keys. 
 
 \begin{texexample}{Key value}{ex:keyval1}
 \ExplSyntaxOn
 \keys_define:nn {scratch}
   {
      parindent .code:n = \parindent#1,
      parskip     .code:n = \parskip#1
   }
   
\DeclareDocumentCommand \MyModuleSetup { m }
     { \keys_set:nn { scratch } { #1 }  }
     
\DeclareDocumentCommand \MyModuleMacro { o }
     {
       \group_begin:
         \keys_set:nn { scratch } { #1 }
         % Main code for \MyModuleMacro
         \lorem\par
         \lorem\par
       \group_end:
     }
 \ExplSyntaxOff   
 \MyModuleSetup{parindent=1em, parskip=1pt}
 \MyModuleMacro [parindent=10pt, parskip=10pt]
 \end{texexample}
 
 
 The definition of the keys was achieved using the command:
 
\begin{docCommand}{keys_define:nn}{\marg{module}\marg{keyval list}}
The command parses the \meta{keyval list} and defines the keys associated there for \meta{module}. 
\end{docCommand}

The \meta{keyval list} should consist of one or more key names along with an associated
key \emph{property}. The properties of a key determine how it acts. The individual properties
are described in the following text; Note that the properties of the key begin from the dot (|.|) after the key name. The various properties available take no arguments or require one or more. All key definitions are local. 
 
 \begin{margoptionslist}
 \item [ .code:n] Stores the \meta{code} for execution when \meta{key} is used. 
 \item [.default:n] \meta{key} |.default:n| = \meta{default} This creates a \meta{default} value for \meta{key} if no value is given. This will be used if only the key name is given, but not if a blank \meta{value} is given. This behaviour is similar to the |pgfkeys| package.
 \item [.initial:n] \meta {key} |.initial:n| = \meta{value} Initialises the \meta{key} with the \meta{value}, equivalent to
|\keys_set:nn| \meta{module} \meta{key} = \meta{value}
 
 \item [.dim_set:N] \meta{key} |.dim_set:N| = \meta{dimension} Defines \meta{key} to set \meta{dimension} to \meta{value} (which must a dimension expression). If the variable does not exist, it will be created globally at the point that the key is set up.
 \end{margoptionslist}
 
%  \begin{texexample}{Key value}{ex:keyval1}
% \ExplSyntaxOn
% \dim_new:N \l_parskip
% \dim_new:N \l_parindent
% \keys_define:nn {scratch}
%   {
%      parindent .dim_set:N = \l_parindent,
%     % parindent .initial:n = 0pt,
%      parskip     .dim:n = \l_parskip,
%      %parskip     .initial:n = 1pt,
%      
%   }
%   
%\DeclareDocumentCommand \MyModuleSetup { m }
%     { \keys_set:nn { scratch } { #1 }  }
%     
%\DeclareDocumentCommand \MyModuleMacro { o }
%     {
%       \group_begin:
%         \dim_set_eq:NN \parindent \l_parindent
%         \dim_set_equal:NN \parskip \l_parskip
%         \keys_set:nn { scratch } { #1 }
%         % Main code for \MyModuleMacro
%         \lorem\par
%         \lorem\par
%       \group_end:
%     }
% \ExplSyntaxOff   
% 
% \MyModuleMacro [parindent=10pt, parskip=10pt]
% \end{texexample}

 
 \subsection{Choice keys}
 
 One of the most powerful features of modern key value packages is the ability to define and set keys for mutally exclusive values. In the |l3keys| module this can be achieved using the choice key.
 
 \begin{margoptionslist}
 \item [.choice] \meta{key} |.choice| This sets \meta{key} to act as a choice key. Each choice is then created, as discussed below:
 \end{margoptionslist}
 
 
 \begin{texexample}{Some choices}{}
 \ExplSyntaxOn
 \keys_define:nn { scratchi }
 {
    mycolor .choice:,
    mycolor/fire .code:n = {\color{red}},
    mycolor/sky .code:n = {\color{blue}},
    mycolor/orange .code:n = {\color{orange}},
    mycolor/lemon .code:n = {\color{yellow}},
    mycolor/grass .code:n = {\color{green}},
    mycolor .initial:n =sky,
    mycolor .default:n=orange,
    unknown .code:n={\color{red} ERROR},
 }

\keys_set:nn { scratchi } { mycolor=fire }  

\DeclareDocumentCommand \MyModuleSetup { m }
     { \keys_set:nn { scratchi } { #1 }  }

\DeclareDocumentCommand \MyModuleMacro { o +m}
     {
       \group_begin:
         \keys_set:nn { scratchi } { #1 }
         #2
         \group_end:
     }
     
 \ExplSyntaxOff
    
 \MyModuleSetup{mycolor=fire}

 \MyModuleMacro [mycolor=grass]{grass,} 
 
 \MyModuleMacro [mycolor]{default}
 
 \MyModuleMacro [apple]{}
 
 \MyModuleMacro [fire]{Fire}
\end{texexample} 
 
The |.choice|  key is a bit different from how it is used in the |xtemplate| package and |pgf| but probably easier to use and define. Of course our example was trivial and the colors should have been achieved with just one code key, capturing the value. It takes some practice to get used to all the types of keys available and to develop error free code easily, but by using a key value system, truly flexible, modern functions can be developed.
 

\subsection{Handling of unknown keys}
 
 Handling of unknown keys is similar to |pgf| where a key defined as |.unknown| is defined. 
 If a key has not previously been defined (is unknown), |\keys_set:nn| will look for a special
unknown key for the same module, and if this is not defined raises an error indicating that
the key name was unknown. This mechanism can be used for example to issue custom
error texts.

\begin{verbatim}
\keys_define:nn { mymodule }
{
unknown .code:n =
You~tried~to~set~key~’\l_keys_key_tl’~to~’#1’.
}
\end{verbatim}
 
 
 As for |pgf| there are many other key types and these are listed in the |l3keys| manual and are not listed here for brevity. 
 
 
%\chapter{The LaTeX3 l3token package}
\label{ch:l3token}

The \tex concept of tokens is central to its operation. In earlier chapters we discussed extensively the use of category codes and other important aspects of \tex’s tokens. Rememeber a \tex token is either a single character or a control sequence such as a the control sequence |\test|.

A review of all possible tokens is appropriate at this stage, before we examine the module in more detail. We distinguish the meaning of a token which controls the expansion of the token and its effect on \tex’s state,
and its shape, which is used when comparing token lists such as for delimited arguments.
Two tokens of the same shape must have the same meaning, but the converse does not
hold.

A token has one of the following shapes:

\begin{enumerate}
\item A control sequence, characterized by the sequence of characters that constitute its
name: for instance, |\use:n| is a five-letter control sequence.
\end{enumerate}

Now is perhaps a good time to mention some subtleties relating to tokens with
category code 10 (space). Any input character with this category code (normally, space
and tab characters) becomes a normal space, with character code 32 and category code 10.

When a macro takes an undelimited argument, explicit space characters (with character
code 32 and category code 10) are ignored. If the following token is an explicit
character token with category code 1 (begin-group) and an arbitrary character code,
then TEX scans ahead to obtain an equal number of explicit character tokens with category
code 1 (begin-group) and 2 (end-group), and the resulting list of tokens (with outer
braces removed) becomes the argument. Otherwise, a single token is taken as the argument
for the macro: we call such single tokens \enquote{N-type}, as they are suitable to be used
as an argument for a function with the signature :N.



\begin{texexample}{Space tokens}{ex:sptoken}
\ExplSyntaxOn  
 \cs_set:Npn \my_space_token { }
 \token_to_meaning:N \my_space_token\\
 \token_to_meaning:N \c_space_token
 
 % Note that the ~ active character in an ExplSyntaxOn
 % environment has a more complicated definition.
 \token_to_meaning:N ~
\ExplSyntaxOff  
\end{texexample}

The actual definition from the kernel code for the \cs{c_space_token}
\begin{teX}
\use:n { \tex_global:D \tex_let:D \c_space_token = ~ } ~
\end{teX}

\begin{texexample}{makeatletter}{}
\ExplSyntaxOn
\group_begin:
\char_set_catcode_letter:N @
\char_set_catcode_letter:N 1
\def\@store1a{AAAA}
\@store1a\\
\token_to_meaning:N @\\
\token_to_meaning:N 1\\
\char_set_catcode_other:N @
\char_set_catcode_other:N 1
\token_to_meaning:N @\\
\token_to_meaning:N 1\\
\group_end:
\ExplSyntaxOff
\end{texexample}

There are sixteen different commands to set the catcode to any of the predefined groups used by \tex. If you cannot remember the catcode number for a character, try and remember its normal name!

\begin{verbatim}
 \char_set_catcode_escape:N 
 \char_set_catcode_group_begin:N
 \char_set_catcode_group_end:N
 \char_set_catcode_math_toggle:N
 \char_set_catcode_alignment:N
 \char_set_catcode_end_line:N
 \char_set_catcode_parameter:N
 \char_set_catcode_math_superscript:N
 \char_set_catcode_math_subscript:N
 \char_set_catcode_ignore:N
 \char_set_catcode_space:N
 \char_set_catcode_letter:N
 \char_set_catcode_other:N
 \char_set_catcode_active:N
 \char_set_catcode_comment:N
\char_set_catcode_invalid:N
\end{verbatim}

\section{Token predicate functions}

\begin{docCommand}{token_if_macro:NTF} { \meta{token} \marg{true code} \marg{false code}}
tests if the \meta{token} is a \tex macro.
\end{docCommand}

\begin{texexample}{Test if is a macro}{ex:assertt}
\ExplSyntaxOn

% This is a common problem in LaTeXe. 
% \sometest is let tto |\relax| in a csname
 \csname my_sometest\endcsname
 
 % traditional definition using a csname and 
 % \expandafter
 \expandafter\def\csname my_sometesti\endcsname{}
 
 % All tests must pass
 \token_if_macro:NTF \par           { \FAIL } { \PASS } 
 \token_if_macro:NTF \minipage      { \PASS } { \FAIL } 
 \token_if_macro:NTF \my_sometest   { \FAIL } { \PASS }

 \token_if_macro:NTF \my_sometesti  { \PASS } { \FAIL }
 \token_if_macro:NTF Z              { \FAIL } { \PASS } 

 % True was set to relax
 \token_if_eq_meaning:NNTF \my_sometest\relax { \PASS } { \FAIL }
 
 \ExplSyntaxOff
\end{texexample}

Notice the unusual syntax for \cs{ifx} which is named \cs{token_if_eq_meaning:NN}. Also note that Example~\ref{ex:assertt}, uses an assertion style where all tests must return true (\mbox{\PASS}). If you have a lot
of tests in a test file, it is easier to spot what is failing. See below where I redefined the token \enquote{Z} as an active
character and then defined a macro with it. Our test file will then clearly show the test failing. Just a small reminder to turn a character into a macro, you need to set it first to |\active| and then define it. Here is the test file again. 

\begin{texexample}{Test if is a macro}{ex:assertt1}
\ExplSyntaxOn

% \define Z
\group_begin:
\catcode `\Z = \active
\cs_set:Npn  Z {hello~}
% This is a common problem in LaTeXe. 
% \sometest is let tto |\relax| in a csname
 \csname my_sometest\endcsname
 
 % traditional definition using a csname and 
 % \expandafter
 \expandafter\def\csname my_sometesti\endcsname{}
 
 % All tests must pass
 \token_if_macro:NTF \par           { \FAIL } { \PASS } 
 \token_if_macro:NTF \minipage      { \PASS } { \FAIL } 
 \token_if_macro:NTF \my_sometest   { \FAIL } { \PASS }

 \token_if_macro:NTF \my_sometesti  { \PASS } { \FAIL }
 \token_if_macro:NTF Z              { \FAIL } { \PASS } 

 % True was set to relax
 \token_if_eq_meaning:NNTF \my_sometest\relax { \PASS } { \FAIL }
 \group_end:
 \ExplSyntaxOff
\end{texexample}


\bigskip

\begin{question}
It is recommended that you code these exercises as MWEs and try and not refer to the source3 manual,
during your first attempt. Namespace any macros you have to develop as part of the tasks below
with the prefix |yourname|.
\begin{tasks}
\task Define four macros and using \cs{token_if_macro:NTF} typeset a word.
\task Test the meaning of the four macros.
\end{tasks}
\end{question}

\subsection{Test if a control sequence is primitive}

One of the advantages of \latex3 is that it provides new names for all the primitives. This enables
one to check if a primitive has been redefined and to provide suitable tests and replacements.

 If it is a primitive we can find out, using yet another boolean construction \docAuxCommand*{token_if_primitive:NTF}  We can also check its meaning. It is interesting to note that \docAuxCommand*{par} is not a macro. Interestingly we can view what \tex does when we say |\csname somecs\endcsname|. It justs sets it equal to |\relax|. 
 
 Again this is important in parsing and in automating the generation of commands. For example  in the |phd| package, we allow for a key value to be entered either as a control sequence for example, |\Large| or simply as a |large|. A test could be provided before further processing such type of input.

\begin{texexample}{Test if is a macro}{ex:ifprimitive1}
\ExplSyntaxOn
\makeatletter
\token_to_meaning:N \par\\
\token_to_meaning:N \toks
\token_if_primitive:NTF \par       { \PASS } { \FAIL }\\
\token_if_primitive:NTF \@@par     { \PASS } { \FAIL }\\
\token_if_primitive:NTF \tex_par:D { \PASS } { \FAIL }
\makeatother
\ExplSyntaxOff

\end{texexample}

Example~\ref{ex:ifprimitive1} can be used to test if a primitive has been redefined (this can be important for your code and to restore its meaning if necessary or issue an error message.  Another test which is available is to check if a token is a macro. 


\begin{texexample}{Test if a cs is primitive}{ex:primitive}
\ExplSyntaxOn
\group_begin:
\makeatletter
% LaTeX2e normally defines this as @par. 
% Use \par in a group to test.
\def\par{\let\par\@@par\par}
\token_if_primitive:NTF \@par { \PASS } { \FAIL }
\makeatother
\group_end:
\ExplSyntaxOff
\end{texexample}

\subsection{Test for category codes}

The next set of available commands are helper functions equivalent to the output of |\ifcat| 

\begin{docCommand} {token_if_group_begin:NTF} {\meta{token} \marg{true code} \marg{false code}}
Tests if \meta{token} has the category code of a begin group token (\{) when normal TEX
category codes are in force). Note that an explicit begin group token cannot be tested in
this way, as it is not a valid N-type argument. To test it you have to use |\c_group_begin_token|. This is mostly
used in conjuction with |futurelet| type constructions and or parsing.
\end{docCommand}


\begin{texexample} {Test if group begin} {ex:ifgroubbegin}
\ExplSyntaxOn
 \token_if_group_begin:NTF \c_group_begin_token { \PASS } { \FAIL }
 \token_if_group_end:NTF   \c_group_end_token   { \PASS } { \FAIL }\par
 \the\catcode`{
\ExplSyntaxOff
\end{texexample}

Behind the scenes |expl3| uses the |\ifcat| primitive to test the token against the catcode values. Constructions for all categories are available and summarized in the test below rather than described.
\begin{texexample} {Test if group begin} {ex:ifgroubbegin}
\ExplSyntaxOn
 \token_if_group_begin:NTF \c_group_begin_token { \PASS } { \FAIL }
 \token_if_group_end:NTF   \c_group_end_token   { \PASS } { \FAIL }\par
 \token_if_alignment:NTF   \c_alignment_token   { \PASS } { \FAIL }\par
 \token_if_parameter:NTF   \c_parameter_token   { \PASS } { \FAIL }\par
\ExplSyntaxOff
\end{texexample}

Use the constant form of these tokens to avoid errors and to make the code more readable.

The module is feature rich, with too many functions to remember easily. If your code needs to deal
with too many changes of catcodes, lccodes and the like, you will have to study it carefully.



\section{LaTeX3 Futurelet type functions}

In Chapter Futurelet, we spend considerable effort to understand how \tex’s futurelet macro works. There is often a need to look ahead at the next token in the input stream while leaving
it in place. This is handled using the “peek” functions. The generic \docAuxCommand*{peek_after:Nw} is
provided along with a family of predefined tests for common cases. As peeking ahead does
not skip spaces the predefined tests include both a space-respecting and space-skipping
version.

\begin{texexample}{Peek ahead ignoring spaces} {}
\ExplSyntaxOn
\peek_catcode_remove_ignore_spaces:NTF =  
    { 
      \PASS  
      \token_if_letter:NTF
          {l_peek_token ~= ~\token_to_meaning:N \l_peek_token \\  } 
          {   }
    } 
    { \FAIL }  
 = abcde \\
\ExplSyntaxOff
\end{texexample}

Most applications would require to recursively pick up tokens from the input stream and only terminated once a special token is found. This is the most powerful method to parse input strings and create really powerful functions. 

You will understand better if we hide the code in a function.

\begin{texexample}{Peek ahead ignoring spaces} {ex}
\ExplSyntaxOn
\cs_new:Npn \checkletter #1 {
\peek_catcode_remove_ignore_spaces:NTF #1  
    { 
      \PASS  
      \token_if_letter:NTF
          {l_peek_token ~= ~\token_to_meaning:N \l_peek_token \\  } 
          {   }
    } 
    { \FAIL } }

\checkletter {=} =abcde \par
\checkletter {A} Abcde \par
\ExplSyntaxOff
\end{texexample}

\begin{texexample}{Peek ahead ignoring spaces} {}
\ExplSyntaxOn
\cs_set:Npn \check_letter_and_removeall #1 {
\peek_catcode_remove_ignore_spaces:NTF #1  
    { 
      \PASS  
      \removeallaux:w  
    } 
   { \FAIL } 
 }

\cs_set:Npn \removeallaux:w #1; { removed~#1~ }

\check_letter_and_removeall {W}  W 12pt; \par
\ExplSyntaxOff
\end{texexample}

In the next example we will try and remove from the input stream recursively any |;|.
Tests if the next non-space token in the input stream has the same character code as
the test token (as defined by the test \cs{token_if_eq_charcode:NNTF}). Explicit and
implicit space tokens (with character code 32 and category code 10) are ignored and
removed by the test and thehtokeni is removed from the input stream if the test is true.
The function then places either the htrue codei or hfalse codei in the input stream (as
appropriate to the result of the test).
\begin{texexample}{ex:recursivefl}  { }                            
\ExplSyntaxOn

\cs_set:Npn \remove_colon: #1 {
   \peek_charcode_remove_ignore_spaces:NTF#1 
   {
    \TRUE
    \meaning #1 \par
    \peek_charcode_remove_ignore_spaces:NTF#1
   } 
   {
    \meaning#1
    \FALSE
     
   }
}

\remove_colon:;;;;;;;!
\ExplSyntaxOff
\end{texexample}

\subsection{Using higher functions}

\cs{peek_catcode_collect_inline:Nn}\Arg{test token}\Arg{inline code}. Collects and removes tokens from the input stream until finding a token that does not match the \meta{test token}. The colected tokens are passed to the \meta{inline code} as |#1|.   

In the example we collect tokens until we reach the comma (\ExplSyntaxOn\char_value_catcode:n{`\,}\ExplSyntaxOff) character which does not have the same category code as Z ({\ExplSyntaxOn\char_value_catcode:n{`\Z}\ExplSyntaxOff)}. We store the results in the |\grubber|.



\begin{texexample}{Collect tokens}{}
\ExplSyntaxOn

\cs_set:Npn \decorate_and_remove {
    {\space\bfseries \tl_use:N \g_tmpa_tl}
   }

\cs_set:Npn \collect_letters {
  \peek_catcode_collect_inline:Nn Z {\tl_put_right:Nn \g_tmpa_tl {##1}}
}

\cs_set:Npn \collect_others  {
  \peek_catcode_collect_inline:Nn ; {\tl_put_right:Nn \g_tmpb_tl {##1}}
}

\cs_set:Npn \maybe_first_is_surname:w #1   
  {  
    % Clear any contents from the token list 
    \tl_clear:N \g_tmpa_tl
    
    % Collect any letters until catcode is other
    \cs_set:Npn \result {\peek_catcode_collect_inline:Nn Z {\tl_put_right:Nn \g_tmpa_tl {####1}}#1}
    
    \def\removecomma##1##2;;{
      #2
    }
    \removecomma\result;;
    \tl_if_empty:NTF \g_tmpa_tl {\TRUE}
        {}
  }
  
\maybe_first_is_surname:w {Lazarides, Yiannis} 
{\bfseries \tl_use:N \g_tmpa_tl}
    
{\color{red}\tl_use:N \g_tmpb_tl}

%\maybe_first_is_surname:w {Lazarides Yiannis} ;

%\maybe_first_is_surname:n { Lazarides, Yiannis\par }
%
%\maybe_first_is_surname:n { Yiannis;Lazarides   }\par

\ExplSyntaxOff
\end{texexample}






















\end{document}
\cxset{subsubsection title color=black}
\colorlet{thesubsubsectioncolor}{black}

\let\sidenote\footnote
\chapter{PROGRAMMING MACROS}

\addtocimage{-10pt}{-40pt}{../graphics/harnett.jpg}

\pagebreak
\setlength\columnsep{1.5em}

\thispagestyle{plain}
\pagestyle{headings}
{\centering  \includegraphics[width=0.7\linewidth]{./graphics/harnett.jpg}\par}

\newcommand*{\newacronym}[1]{{New acronym: [#1]\par}}
\newcommand*{\newacronyms}{%
  \let\do\newacronym
  \docsvlist
}
\vspace{1.5\baselineskip}
{\centering \Large\bf GETTING STARTED WITH MACROS\par}
\bigskip

\begin{multicols}{2}
\lettrine{P}{rogramming} with \alltex is done through macros. \tex has a macro programming language,
which allows features to be added. The best known
and most widely used collection of \tex macros is \latex.
(This is not quite accurate. Although originally
\latex used \tex, since 2003 it by default uses
e-TEX, which is an extension of TEX. Macro's in \TeX\  are not just simple substitutions, they are more Lispy like. It is this powerful feature that made \TeX\ last and will continue to do so for many years to come. This program that started as a typesetting program, programmed in a variant of what is now an ancient computer language Pascal is testimony to good programming 
practices and a reminder to the programming priesthood that the tool is not important, but what you do with it is. 

A \emph{macro} is a sequence of tokens that has been abbreviated into a control sequence. Statement starting with among others
\cmd{def} are called \textit{macro definitions}. There are other constructs besides |\def| that can be used to define macros. \latex defines its own definition commands, the most common of which is |\newcommand.| The way \tex's macro language is build, you can also define your own. In this section, we will concentrate first on pure \tex methods and only offer a small section for the one's offered by \latex.

\end{multicols}
\clearpage

\section{How to define macros}

Most of \tex programming is learning how to work with macros and control sequences. The workhorse behind all these is a three letter control sequence that can be used to define macros \refCom{def}.
There are other macro defining primitives available, but we will first focus on \refCom{def}. Depending on the way the arguments of the command are specified these are normally divided into two general categories: \emph{delimited macros} and \emph{undelimited}.
 
\subsection{Simple substitution macros}

\begin{docCommand*}{def} { \meta{macro name}\meta{parameter text} \marg{replacement text}}{}
Simple substitution macros, during expansion replace their name with the contents enclosed between the braces. For example some common macros that authors write, is to hold the names of people, in order to get the spelling correctly.
\end{docCommand*}

\begin{teX}
\documentclass{article}
\def\myshortcut{Anthony van der Merwe}
\begin{document}
\myshortcut
\end{document}
\end{teX}


In the above we defined a macro named, |\myshortcut|, will print the name \texttt{Anthony van der Merwe}, every time it is invoked as |\myshortcut|. You will notice, that the macro definition is placed in the preamble. This is not necessary, but it is good practice. Macros can be placed anywhere in the document, in packages and or classes.


If we were writing the macro and compiling it using \tex only, the example can be much shorter and it will also compile much faster. 

\emphasis{def,bye}
\begin{teX}
\def\myshortcut{Anthony van der Merwe}
\myshortcut
\bye
\end{teX}

Macros can use other macro commands. For example if we wanted to store the name of the author of |pdfTeX| we could write,

\emphasize{def	}
\begin{texexample}{example substitution macro}{}
\def\Thanh{%
      H\`an~%
      \texorpdfstring{Th\^e\llap{\raise 0.5ex\hbox{\'{}}}}%
      {\ifpdfstringunicode{Th\unichar{"1EC3}}{Th\^e}}%
      ~Th\`anh^^a
    }
\Thanh 
\end{texexample}




\subsection{Macro parameters.} 

In this Chapter we will spend most of the time with commands available in \tex core, before we move onto commands that are available in \latex. Now, in the example above, we did not use any parameters. \tex allows us to define parameter by adding |#1|..|#9| as parameters to the macro definition. Here is a short example, again using plain \tex. 


\begin{teX}
\def\twonumbers#1#2{(#1,#2)} 
\twonumbers{12,13}
\bye
\end{teX}

\def\twonumbers#1#2{(#1,#2)}
This will print \texttt{\twonumbers{12}{13}}. The macro takes the two arguments 12 and 13  and prints the two numbers in parentheses. This activity is called \textit{macro expansion}\index{macros>expansion}\index{macros>parameters}.

 
\section{Delimited arguments}

As a simple example consider the following:\index{macros>delimited}

\begin{teX}
\def\asentence#1#2;{{\textbf{#1}#2}}
\bye
\end{teX}

\begin{texexample}{delimited examples}{ex:delimited}
\def\asentence#1{\textbf{#1}}
\asentence The whole sentence is printed;

\def\asentence#1;{\textbf{#1}}
\asentence The whole sentence is printed;

\end{texexample}

Example~\ref{ex:delimited} defines a macro with an undelimited first parameter, and a second parameter delimited by a semicolon. The replacement text consists only of |#1| and hence the rest are discarded.

\begin{texexample}{delimited examples}{ex:delimited}
\def\asentence#1 #2;{{\bfseries #1 }#2}
\asentence The whole sentence is printed;
\end{texexample}

Example~\ref{ex:delimited} defines a macro with an undelimited first parameter, and a second parameter delimited by a semicolon. The replacement text consists only of |#1| and hence the rest are discarded.


\subsection{Space, return, and the tab character as delimiters of parameters}

A space can be used to delimit a parameter. The space character, return character and the tab character are all converted into space tokens by \tex. Here is an example,

\begin{texexample}{Space delimiters}{ex:spacedelimiters}
\def\tempmacro #1 #2 #3 {#1,#2,#3}
\tempmacro 12 15 17 
\end{texexample}


\section{Format of a macro definition}

So far we have looked at macros that have no parameters, macros that have parameters and macros that have delimited arguments. A macro definition consists of, in sequence,

\begin{enumerate}
\item any number of \cmd{\global}, \cmd{\long}, and \cmd{\outer}, prefixes
\item a \cmd{\def} control sequence, or anything that has been \cmd{\let} to one,
\item possibly a parameter text specifying among other things how many parameters the macro has,
\item a replacement text enclosed in explicit characters \{\}
\end{enumerate}


\cmd{\global}\cmd{\def}\meta{command}\{\ldots\}

As the name implies global macros define macros that they have a global scope. \TeX, like many other computer languages has scoping rules. We will revisit \tex's scoping rule in the Chapter for Grouping.  Try the following example:


\begin{teX}
\def\sometext{This is some text}
\def\someothertext{%
   \def\sometext{I am in the macro, someothertext.}\par
   \sometext
}
\sometext
\end{teX}

\def\sometext{This is some text}
\def\someothertext{%
   \def\sometext{I am in the macro, someothertext.}\par
   \sometext
}
\sometext
\someothertext

As you can see from the output, any definitions of macros within other macros are defined locally within the scope of the aprent macro only. I am also sure that you have also observed that we can nest macros to as many depths as required.

\def\sometext{This is some text}

\def\someothertext{%
   \gdef\sometext{I am in the macro, someothertext.}\par
   \sometext
}

\sometext

\someothertext

\sometext

\cmd{\long}\cmd{\def}\marg{command}\{\ldots\}

\index{macro definitions>long}
Knuth designed \tex in such a way that the normal |\def| will not work with arguments that include paragraphs. This was so that if you forget to add a brace \enquote{\}} \tex will not continue absorbing tokens until the end of the file or completely full \tex's memory. Therefore \tex has another rule [205] intended to confine errors to the paragraph that they occur: The token |\par| is not allowed to occur as part of an argument as unless you explicitly tell \tex that you want to use |\par|. Whenever \tex is about to include |\par| as part of an argument, it will abort the current macro expansion and report that a \texttt{...runaway argument} has been found.

If you actually want a control sequence to allow arguments with |\par| tokens, you can define it to be a \cmd{\long}\index{macros>long} just before the |\def|. For example the |\bold| macro defined by:


\begin{teX}
\long\def\mybold#1{{\bf#1}} 
\end{teX}

\noindent is capable of setting several paragraphs in boldface type. However, such a macro is not a especially good way to typeset bold text. It would be better to say, e.g.,

\begin{teX}
\def\beginbold{\begingroup\bf}
\def\endbold{\endgroup}
\end{teX}
because this doesn't fill \tex's memory with a long argument.


\begin{docCommand}{edef}{\marg{macro name}\meta{arguments}\marg{replacement text}}
Defines a fully expandable macro.
\end{docCommand}

Another command that can be used to define macros is \cmd{\edef}. You can say |\edef\foo{bar}|. The syntax is the same as |\def|, but the token list in the body is fully expanded (tokens that come from |\the| are not expanded).

You can say |\xdef\foo{bar}|. The syntax is the same as \cmd{\def}, but the token list in the body is fully expanded (tokens that come from \cmd{\the} or \cmd{\unexpanded} are not expanded).

\cmd{\global}\cmd{\edef}

You can put the prefix \cmd{\global} before \cmd{\xdef}, this is however useless, since |\xdef| is the same as |\global\edef|. The following example puts a brace in |\foo|. The |\string| command can be expanded, the value is the name of the command (preceded by a backslash, or whatever the value of the escape character is). Here the assignment to the escape character is local, the assignment to |\foo| is global.


\begin{texexample}{Putting a brace}{ex:put a brace}
{\escapechar=-1 
\xdef\fooleft{\string\{}
\xdef\fooright{\string\}}}
\fooleft A\fooright 

\textbraceright A\textbraceleft
\end{texexample}


\begin{docCmd}{relax}{} 
A control sequence that cannot be expanded, but when it is executed nothing happens.
\end{docCmd}
The control sequence \cmd{relax} cannot be expanded, but when it is executed \textit{nothing happens}.
This statement sounds a bit paradoxical, so consider an example. 


\begin{texexample}{Why use relax}{ex:relax}
\newcount\MyCount
\newcount\MyOtherCount \MyOtherCount=2
\MyCount=1\number\MyOtherCount3\relax4\par

\the\MyCount
\end{texexample}


The command \cmd{\number} is expandable, and \cmd{\relax} is not. When TEX constructs the number that is
to be assigned it will expand all commands, either until a non-digit is found, or until an unexpandable
command is encountered. Thus it reads the 1; it expands the sequence \verb+ \number\MyOtherCount+,
which gives 2; it reads the 3; it sees the \cmd{\relax}, and as this is unexpandable it halts. The number
to be assigned is then 123, and the whole call has been expanded.


\noindent Since the \cmd{\relax} token has no effect when it is executed, the result of this line is that 123 is
assigned to \verb+ \MyCount +, and the digit 4 is printed.



Another example of how \cmd{\relax} can be used to indicate the end of a command is

\verb+ \MyCount=123\relax4+

\begin{codeexample}[]
\newcount\MyCount
\MyCount=123\relax4\par
\the\MyCount
\end{codeexample}

\noindent Since the \cmd{relax} token has no effect when it is executed, the result of this line is that 123 is
assigned to \verb+ \MyCount +, and the digit 4 is printed.

Another example of how \cmd{relax} can be used to indicate the end of a command is


\begin{teX}
\everypar{\hskip 0cm plus 1fil }
\indent Later that day, ...
\end{teX}

\noindent This will be misunderstood: TEX will see

\verb+ \hskip 0cm plus 1fil L+

\noindent and fil L is a valid, if bizarre, way of writing fill (see Chapter 36). One remedy is to write

\verb+ \everypar{\hskip 0cm plus 1fil\relax}+

\section{Spaces after macro calls}

\cmd{\ignorespaces}
The primitive \cmd{\ignorespaces} allows the user to unify the calls of certain macros. Consider the following:

\begin{codeexample}[]
\bgroup
\def\\{A}
\def\xx{..}
\def\yy{...}

\\ABC
\\ ABC
\xx ABC
\yy{1}ABC
\yy{a} ABC
\egroup
\end{codeexample}

As it can be observed from the example spaces after control\textit{symbols} like |\\| are \emph{not ignored}, and therefore the output from line 1 reads ``AABC" and the output from line 2 reads ``X ABC. To bring some uniformity to the treatment of spaces after macro calls (regardless of whether the macro has parameters or not, the \cmd{\ignorespaces} primitive can be used. Including this instruction as the \emph{last} token in the replacement text of a macro causes the space (or any number of space tokens) following the macro call to be ignored.

%\begin{codeexample}[]
\bgroup
\def\\{A\ignorespaces}
\def\yy{...\ignorespaces}

\\ABC
\yy{a}\ignorespaces ABC
\egroup
%\end{codeexample}

Note that \cmd{\ignorespaces} does \emph{not} cause \tex to gobble up empty lines following the macro call because \tex converts empty lines into \cs{par}s. 

\cmd{\ignorespaces} does nothing, if no space token or space tokens follow it.  However, it \emph{does} expand token follow it though to find out whether they contain space tokens or not.

\section{Creating macros on the fly}


One of the more useful ability of \tex is that macros can be created programmatically. This is achieved using \refCom{string} and \refCom{csname}.
\footnote{Most of this discussion is based on an article by Stephan v. Bechtolsheim see \url{http://www.tug.org/TUGboat/Articles/tb10-2/tb24bechtolsheim.pdf}}

This article discusses \cmd{\string} and \cmd{\csname} to
convert back and forth between strings and tokens.
To control loading macro source files in a convenient
way, I will show an application of \cmd{\csname}. I
will also discuss cross referencing which relies on
\cmd{\csname}.


An important application of \cmd{\string} is to
write control sequences to a file using \cmd{\write}.
Any control sequence which should be written
to a file (instead of being expanded) must be
prefixed by \cmd{\string}. The command \cmd{\noexpand} can also be used.

\begin{docCommand}{csname}{}{}
\end{docCommand}

\emphasize{csname,endcsname,}
\begin{texexample}{Example of using \textbackslash csname.}{ex:csname}
\expandafter\def\csname john\endcsname{My name is John.} %(*@\dcircle{2}@*)
\john

\expandafter\def\csname john\endcsname{My name is Johnny.} %(*@\dcircle{2}@*)
\john
\end{texexample}

If we did not use the \docAuxCommand{expandafter}, we would have redefined \docAuxCommand{csname} with disastrous effects. The example defines the macro |john| twice, at \dcircle{1} and \dcircle{2} and acts normally as if we have defined the macros using |\def|. 

The \cmd{\csname} command
is, in a certain sense, the inverse operation of
\docAuxCommand{string}. It converts a sequence of characters into
one token. Observe that I said \enquote{characters} and
not \enquote{letters.} Using \texttt{\string\csname} allows you to build
names for tokens that contain { non-letter characters}
such as digits.\footnote{Normal macro definitions cannot contain any digits, but just alphanumeric characters}

\begin{texexample}{Example of using \textbackslash csname with non-letter arguments}{ex:csname2}
\makeatletter
\expandafter\def\csname john1\endcsname{John Lovelace }
\expandafter\def\csname john2\endcsname{John Smith }

John1 is \csname john1\endcsname
but
John2 is \@nameuse{john2}
\makeatother
\end{texexample}


\begin{docCommand}{string}{}{}
\end{docCommand}



The ordinary way to write control
sequences restricts the user to control words (the
escape character followed by any number of letters,
but letters only) and control symbols (the escape
character followed by one and only one nonletter
character).


\begin{teXXX}
\newcommand{\defcsname}{\hlred{\texttt{\string\csname}}}
\newcommand{\defendcsname}{\hlred{\texttt{\string\endcsname\thinspace}}}
\end{teXXX}


The |\defcsname| control sequence is applied as
follows. After |\defcsname|, list the characters naming
the token. You also may use macros, but only
those which expand to characters. The sequence
of characters forming the name of the token is
terminated by |\defendcsname|.

Here is an example. To name the token


\begin{teX}\?-a*l7 .g\end{teX}

\begin{teX}
   \csname ?-a*l7. g\endcsname
\end{teX}

\begin{docCommand}{expandafter}{}
\end{docCommand}
It is important to stress that|\csname| does not define anything: you need to use the TeX primitive \cmd{\def} to create a definition. This also requires the \cmd{\expandafter} primitive.

\begin{teX}
\def\MyMacro#1{Some code #1}
\end{teX}
and so with
\begin{teX}
\expandafter\def\csname MyMacro\endcsname#1{Hello  #1}
\MyMacro{John}
\end{teX}
will produce:
\medskip
\expandafter\def\csname MyMacro\endcsname#1{Hello  #1}
\MyMacro{John}


As mentioned before it is legal to call a macro
inside a |\def\csname| . . .|\def\endcsname| sequence as long
as the macro expands to characters only. Counter registers
can also be used:


\begin{texexample}{count example}{}
\makeatletter
\bgroup
\count0=5 %
\expandafter\def\csname ZZ-\the\count0\endcsname{outputs: 
ZZ-\the\count0 }

\@nameuse{ZZ-5}
\egroup
\makeatother
\end{texexample}


\begin{comment}
\def\xx{ABC}
% \count0=4
  \csname ZZ1=\the\count0-\xx\endcsname
\end{comment}


This will print |\ZZ1-137-ABC|. This example is equivalent to forming the same
token using. |\csname ZZ-4-ABC\endcsname|. Although all these might not make much sense now, the ability to name macros on the fly, is leveraged by most package authors.

So far we have been using \tex core counters only and now and then we have seen a \latexe command such as |\@nameuse|. In the case study that follows, we will use mostly \latexe definitions. 


\chapter*{CASE STUDY 13}

\noindent We want to define a command that can hold text. The command must have the form |\lorem@i|, we want to automate the production of such commands, so that we can produce them automatically using |csname|.

\topline
\begin{teXXX}
\lorem@i{Lorem ipsum dolor sit amet, consectetuer
  adipiscing elit. Ut purus elit, vestibulum ut, placerat ac,
  adipiscing vitae, felis.. \par}

These are called by:
 \csname lorem@\roman{lorem@count}\endcsname%
\end{teXXX}
\bottomline

An example worth studying can be found in Patrick Happel's package \pkg{lipsum}.

We first define a counter and set it to zero


\begin{texexample}{Setting the counters}{ex:counters}
\makeatletter 
\newcounter{lorem@count}
\setcounter{lorem@count}{0}

% define a new command for default values
\newcommand\lorem@default{1-7}

% allow user to change this default value
% using setlipsumdefault 
\newcommand\setloremdefault[1]{%
  \renewcommand{\lorem@default}{#1}}

% change the defaults example to only two lines
\setloremdefault{1-2}  
\makeatother  
\end{texexample}



\begin{texexample}{Rest of code}{ex:rest}
\makeatletter

% define a new command for default values
\newcommand\lorem@default{1-7}

% allow user to change this default value
% using setlipsumdefault 
\newcommand\setloremdefault[1]{%
  \renewcommand{\lorem@default}{#1}}

% change the defaults example to only two lines
\setloremdefault{1-5}

% This is a bit difficult to grasp
% try it on your own a few times
\newcommand\lorem@dolipsum{%
  \ifnum\value{lorem@count}<\lorem@max\relax%
    \addtocounter{lorem@count}{1}%
%\roman would convert numerals
% to roman numerals all the lipsum paragraphs
% are referenced in roman  
    \csname lorem@\roman{lorem@count}\endcsname%
    \lorem@dolipsum%
  \fi  
}

% lipsum[1-8] would print para 1-8 etc
% this routine defines the command
\newcommand\lorems[1][\lorem@default]{%
  \expandafter\lorem@minmax\expandafter{#1}%
  \setcounter{lorem@count}{\lorem@min}%
  \addtocounter{lorem@count}{-1}%
  \lorem@dolipsum%
}

% define min and max
%this is quite involved
\def\lorem@get#1-#2;{\def\lorem@min{#1}\def\lorem@max{#2}}
\def\lorem@stripmax#1-{\edef\lorem@max{#1}}

\def\lorem@minmax#1{%
  \lorem@get#1-\relax;%
  \edef\lorem@tmpa{\lorem@max}%
  \edef\lorem@relax{\relax}%
  \ifx\lorem@tmpa\lorem@relax\edef\lorem@max{\lorem@min}%
  \else\expandafter\lorem@stripmax\lorem@max\fi%
}

% All the paragraphs are set as commands
% for example
\newcommand\lorem@i{1 Lorem ipsum dolor sit amet, consectetuer
  adipiscing elit. Ut purus elit, vestibulum ut, placerat ac,
  adipiscing vitae, felis.. \par}

\newcommand\lorem@ii{\[ a+ b +c = \frac{a+b}{a+b+c}\]}

\newcommand\lorem@iii{3 Lorem ipsum dolor sit amet, consectetuer
  adipiscing elit. Ut purus elit, vestibulum ut, placerat ac,
  adipiscing vitae, felis.. \par}
  
\newcommand\lorem@iv{4 Lorem ipsum dolor sit amet, consectetuer
  adipiscing elit. Ut purus elit, vestibulum ut, placerat ac,
  adipiscing vitae, felis.. \par}  
  
\newcommand\lorem@v{5 \selectlanguage{ngerman}
This are German quotes \enquote{quoted text}. 
\par}  
  
\newcommand\lorem@vi{
\cxset{paragraph format=hang, paragraph color=yellow,}
\paragraph* {Septic Equation} 
   \begin{gather}ax^7+bx^6+cx^5+dx^4+ex^3+fx^2+gx+h=0\end{gather}%
   }
  
These are called by:
\setcounter{lorem@count}{2}
\csname lorem@\roman{lorem@count}\endcsname%

\makeatother

% author commands
\lorems[1-2]
\lorems[5]
\lorems[6]
\end{texexample}



\section*{CONDITIONAL STATEMENTS}

\begin{multicols}{2}
As Knuth said, when authors start using macros the next thing the ask is conditional statements.
\TeX\  provides a number of  conditional commands that can help you code almost anything you can do with any low level or high level language.

All  control sequences for conditionals begin with \doccmd{if}...,
and they all have a matching \doccmd{fi}. This convention that\doccmd{if}... pairs up
with |fi| makes it easier to see the nesting of conditionals within your program. 

The nesting of \docAuxCommand*{if}$\ldots$\docAuxCommand*{fi}  is independent of the nesting of \{...\}; thus, you can begin or end
a group in the middle of a conditional, and you can begin or end a conditional in the
middle of a group. Knuth notes that

\begin{quotation}
Extensive experience with macros has shown that such independence
is important in applications; but it can also lead to confusion if you aren't careful.
\end{quotation}

\textbf{\textbackslash if constructions.} \quad The first conditional we will review, is |\if| \ldots |\fi|. This is used to compare two unexpandable tokens. \TeX will expand macros following |\if| until two unexpandable tokens are found. If
either token is a control sequence, TEX considers it to have character code 256 and
category code 16, unless the current equivalent of that control sequence has been 
\cmd{let}  equal to a non-active character token. In this way, each token specifes a (character
code, category code) pair. The condition is true if the character codes are equal,
independent of the category codes.

 For example, after 
\end{multicols}


\section{Characters and control sequences}

\begin{docCommand}{if}{\meta{$token_1$}\meta{$token_2$}}{}
Tests for equality of character codes.
\end{docCommand}

\emphasize{if,fi}
\begin{texexample}{Check for Equality of Characters}{ex:if}
\def\a{a}\def\b{a}
\def\TRUE{TRUE} 
\def\FALSE{FALSE}
\if\a\b\TRUE\fi
\end{texexample}


\begin{docCommand}{ifx}{\meta{$token_1$}\meta{$token_2$}}{}
Test equality of macro expansion, or equality of character code and category code
\end{docCommand}

\begin{texexample}{Check for Equality of Characters}{}
\def\a{a}\def\b{a}\def\c{other}
\def\TRUE{TRUE}
\def\FALSE{FALSE}

\ifx\a\b\TRUE\fi

\ifx\a\c\TRUE\else\FALSE\fi

\meaning\a

\meaning\b
\end{texexample}


\section{Numerical Testing}

\begin{docCommand}{ifnum}{\meta{$number_1$}\meta{relation}\meta{$number_2$}}
Testing of a numerical relation, where the relation is a character <, =, or >, of category 12.
\end{docCommand}

\emphasize{ifnum,else,fi}
\begin{texexample}{Testing a relation}{ex:numericaltests}
\edef\a{12}\edef\b{13}
\ifnum\a<\b\TRUE\else\FALSE\fi

\ifnum\a>\b\TRUE\else\FALSE\fi

\ifnum\a=\b\TRUE\else\FALSE\fi
\end{texexample}


\section{\textbackslash ifodd}


The \docAuxCommand{ifodd} construction, checks if a number is odd and you can use it to for example to color 
all the odd rows of a table. 

\begin{texexample}{\textbackslash ifodd construction.}{ex:ifodd}
   \ifodd\thepage  
      this page is odd
   \else
      this page is even   
   \fi
\end{texexample}

\section{Case Statements}


{\textbackslash ifcase.} The \cmd{ifcase} is a switch, it is equivalent to a number of |\ifnum| statements combined together.
Remember for most of \TeX\  constructs you do not use parentheses, just write freely. Like a Turing machine,
just read from the tape and give your result to the next token and so on.

Here is a trivial example:


\begin{teX}
\ifcase 12% 
    I am zero      %   0
   \or I am one    %   1
   \or I am two    %   2
   \or I am three  %   3
   \else 
      I am different 
\fi 
\end{teX}

This will output  \ldots \texttt{I am different}  


Just to become more familiar with the syntax let us see another example. This time we will define
a new command \cmd{weekday}, which will give us the name of the date of the week, given a numer, really simple stuff,


\begin{texexample}{ifcase}{ex:ifcase}
\gdef\weekday#1{
 \ifcase#1
   Sunday          	
   \or Monday    		
   \or Tuesday    	
   \or Wednesday  	
   \or Thursday     
   \or Friday  		
   \or Saturday 		
   \else 
      Error No: 212345, this is not a  weekday!
 \fi\relax} 
\weekday{2}
\end{texexample}



\begin{texexample}{Month and time stamp}{}
\bgroup
\def\monthname{%
  \ifcase\month
    \or Jan\or Feb\or Mar\or Apr\or May\or Jun%
    \or Jul\or Aug\or Sep\or Oct\or Nov\or Dec%
\fi}%

%\def\timestring{\begingroup
%  \count0 = \time \divide\count0 by 60
%  \count2 = \count0 % The hour.
%  \count4 = \time \multiply\count0 by 60
%  \advance\count4 by -\count0 % The minute.
%  \ifnum\count4<10 \toks1 = {0}% Get a leading zero.
%  \else \toks1 = {}%
%  \fi
%
%\ifnum\count2<12 \toks0 = {a.m.}%
%\else \toks0 = {p.m.}%
%\advance\count2 by -12
%\fi
%
%\ifnum\count2=0 \count2 = 12 \fi 
%\number\count2:\the\toks1 \number\count4
%\thinspace \the\toks0
% \endgroup}%
%
%\def\timestamp{\number\day\space\monthname\space
%\number\year\quad\timestring}%
%
%number = \number
%
%day = \day 
%
%year =\year
%
%month = \month
%
%month-name  = \monthname 8
%
%time = \timestring
\monthname{5} \weekday{\day} 
\egroup
\end{texexample}

\section{Box Testing}

\begin{docCommand}{ifvoid}{\marg{8-bit number}}{}

Test a box register if is void
\end{docCommand}

\begin{texexample}{Box registers}{}
% create a new box
\newbox\mybox

% check if it is void
\ifvoid\mybox\TRUE\else\FALSE\fi

% put some material in the box
\setbox\mybox=\hbox{This is some test}

% check if void, returns false
\ifvoid\mybox\TRUE\else\FALSE\fi

% unbox the \mybox, this will empty the box
\box\mybox

% check if void, returns true as using a \box will
% empty it
\ifvoid\mybox\TRUE\else\FALSE\fi

% put some new material in the box
\setbox\mybox=\hbox{new material in the box}

% unbox
\copy\mybox

% check if void, returns false
\ifvoid\mybox\TRUE\else\FALSE\fi

\end{texexample}

\begin{docCommand}{ifhbox}{\meta{8-bit number}}
Returns true if the box register contains a horizontal box.
\end{docCommand}

\begin{docCommand}{ifvbox}{\meta{8-bit number}}
Returns true if the box register contains a vertical box.
\end{docCommand}

\emphasize{ifhbox,ifvbox,else,fi}
\begin{texexample}{ifhbox}{}
% create a new box
\newbox\hororvertbox

% insert some material in the box 
\setbox\hororvertbox=\hbox{This is some test.}

% check if is a horizontal box, returns true
\ifhbox\hororvertbox\TRUE\else\FALSE\fi

% insert some vertical material
\setbox\hororvertbox=\vbox{This is a vertical box}

% check if it is a horizontal box, returns false
\ifhbox\hororvertbox\TRUE\else\FALSE\fi

% check if it is a vertical box, returns true
\ifvbox\hororvertbox\TRUE\else\FALSE\fi
\end{texexample}



\section{Special tests}

The tests \docAuxCommand{iftrue} and \docAuxCommand{iffalse} are always true and false respectively. They are mainly
useful as tools in macros.
For instance, the sequences

|\iftrue{\else}\fi|

and

|\iffalse{\else}\fi|

yield a left and right brace respectively, but they have balanced braces, so they can be used
inside a macro replacement text.
The \docAuxCommand{newif macro}, treated below, provides another use of |\iftrue| and |\iffalse|. On
page 260 of the TEX book these control sequences are also used in an interesting manner.


\begin{texexample}{latex newif}{ex:ltxnewif}
\makeatletter
\newif\if@mytest
\@mytesttrue
\if@mytest\TRUE\else\FALSE\fi
\makeatother 
\end{texexample}

The \docAuxCommand{newif} is not a \tex primitive command, as one would at first expect, but is defined later
in the format.

The definition is:

\begin{teXXX}
%\newif And here's a different sort of allocation: For example, \newif\iffoo creates
%\footrue, \foofalse to go with \iffoo.
\def\newif#1{%
 \count@\escapechar \escapechar\m@ne
 \let#1\iffalse
 \@if#1\iftrue
 \@if#1\iffalse
 \escapechar\count@}
%\@if
\def\@if#1#2{%
 \expandafter\def\csname\expandafter\@gobbletwo\string#1%
 \expandafter\@gobbletwo\string#2\endcsname
 {\let#1#2}}
\end{teXXX}

If the value of \docAuxCommand{escapechar} is negative or larger than 255, then no escape-character will be printed, when using string or writing to a file and in the text printed using |\meaning|. Simplistically \tex uses this to remove the if from the characters and define the two macros. In example \ref{es:escapechar}, we will set the |\escapechar| to -1 and see its effects on |\meaning| and |\string| more clearly. 

The |\newif| does not do any checking for duplicate names. It will simply overwrite the definition? 

\begin{texexample}{Effects of \textbackslash @escapechar}{ex:escapechar}
\makeatletter
\bgroup
\count@\escapechar 

% define \test to hold \string\test
\def\test{\escapechar\m@ne \string\test}

% prints just test
\test

% prints macro:->escapechar m@ne string test 
\meaning\test 

% reset escapechar
\escapechar=`\\

% prints ]test
\string\test

% prints macro:->\escapechar \m@one \string \test
\meaning\test

\egroup
\makeatother
\end{texexample}

Lamport in \latex used these constructs extensively to define switch-on and switch-off macros. For example\\
|\def\@nobreaktrue {\global\let\if@nobreak\iftrue}|\\
is used to used to avoid page breaks caused by |\label| after a section heading, etc.
It should be globally set true after the |\nobreak| and globally set false by
the next invocation of |\everypar|. Commands that reset |\everypar| should globally set it false if appropriate


\begin{texexample}{latex newif}{ex:ltxnewif}
\makeatletter
\if@nobreak\TRUE\else\FALSE\fi

\begin{minipage}{\linewidth}
\if@minipage
  \lorem
\fi

\if@nobreak\TRUE\else\FALSE\fi  
\end{minipage}
\if@minipage
  \lorem
\fi  
\makeatother
\end{texexample}

This brings us to the end of the discussion of \tex's and \latexe's conditionals. They are somewhat different from other
programming languages and it takes sometime to get used to them. If you can store variables, have recursion, you have a complete Turing engine. You can actually program anything---although this can take a bit longer than usual programming using \tex. 

Turing-completeness basically requires three properties:\footnote{For a longer discussion see \url{https://tex.stackexchange.com/questions/58042/are-there-any-disadvantages-of-tex-being-turing-complete}}

\begin{enumerate}
\item Variables (registers) that can be modified by primitive arithmetic: An inc operation (+1) is enough.
\item A test operation (if)
\item An endless loop construct, such as goto or while or recursion. \tex does not have a |goto| but has recursion.
\end{enumerate}




\endinput
\section{Find the lenth of an argument}
% This can be useful standard library routine
% Find the length of a string - but not spaces

\begin{verbatim}
\def\length#1{{\count0=0 \getlength#1\end \number\count0}}

\def\getlength#1{\ifx#1\end \let\next=\relax
\else\advance\count0 by1 \let\next=\getlength\fi \next}

\length{The flying fox said foo !}
\end{verbatim}


This will give us:  \TODO intefering  Just note that this is not the string length, like you will find in a normal programming language, but the length of the arguments i.e. the non-space characters.

\medskip
\verb*+The flying fox said foo!+
\medskip

The syntax is realy not very user friendly, but remember all these were programmed in 1978!


Just a small suggestion at this point, you need to stop and type these short examples. As Knuth says in Exercise~6.1 

\begin{quote} 
Statistics show that only 7.43 of 10 people who read this manual actually type
the story.tex file as recommended, but that those people learn \TeX\  best. So
why don't you join them?\sidenote{answer: laziness and obstinacy}
\end{quote}

\section{Packages}

A number of packages are availabel to ease the job of defining conditionals. One of the first packages was David Carlisle's \pkg{ifthen}

The package \docpkg{ifthen} by David Carlisle makes it easy to write if-then-else commands. 
The package allows you to make if-then-else expressions and
while-do loops:

\begin{teX}
  \ifthenelse{test}{then-code}{else-code}
  \whiledo{test}{do-clause}
\end{teX}



\section{whiledo}

The |whiledo| command available with the |ifthen| package can be used to creade |while-do| loops:
%%% Examples need LaTeX's ifthen.sty package


\begin{teX}
\newcounter{howoften}
\whiledo{\value{howoften}<3}{%
    \stepcounter{howoften} 
    \TeX\ is great (\thehowoften)\break}
\end{teX}

\noindent This will display:
\medskip

{
\newcounter{howoften}
\whiledo{\value{howoften}<8}{%
\stepcounter{howoften}% 
\tt\centering\TeX\ is great (\thehowoften)}}


\begin{teX}
\newcounter{myi}
\newcounter{myj}

\whiledo{\value{myi}<8}{%
   \setcounter{myj}{0}
   \stepcounter{myi}% 
   %inner loop
       \whiledo{\value{myj}<\value{acount}}{
        {\stepcounter{myj}
        $\bullet$}
   \vskip-4.3pt }
}

%needs work
\end{teX}


A more complicated example to ceate a color scale is shown below, it uses the docpkg{xcolor} package to set up a colorbox. The |whiledo| loop is used to vary the values of the red, green or blue component.

\begin{teX}
\newcounter{Col}
\setlength{\fboxsep}{3mm}
\newcommand{\CBox}[1]{% vary red component
    \colorbox[rgb]{.#1,0.,0.}{.#1}}
\begin{flushleft}
\scriptsize\tt
\makebox[15mm][l]{\small Red:}%
\whiledo{\value{Col}<10}{\CBox{\theCol}%
                           \stepcounter{Col}}\\ 
\renewcommand{\CBox}[1]{% vary green component
    \colorbox[rgb]{0.,.#1,0.}{.#1}}%
\setcounter{Col}{0}\makebox[15mm][l]{\small Green:}%
\whiledo{\value{Col}<10}{\CBox{\theCol}%
                           \stepcounter{Col}}\\ 
\renewcommand{\CBox}[1]{% vary blue component
    \colorbox[rgb]{0.,0.,.#1}{.#1}}%
%draws a box to place the label
\setcounter{Col}{0}\makebox[15mm][l]{\small Blue:}%
\whiledo{\value{Col}<10}{\CBox{\theCol}%
                           \stepcounter{Col}}\\
\end{flushleft}
\end{teX}

\newcounter{Col}
\setlength{\fboxsep}{3mm}
\newcommand{\CBox}[1]{% vary red component
    \colorbox[rgb]{.#1,0.,0.}{.#1}}
\begin{flushleft}
\scriptsize\tt
\makebox[15mm][l]{\small Red:}%
\whiledo{\value{Col}<10}{\CBox{\theCol}%
                           \stepcounter{Col}}\\ 
\renewcommand{\CBox}[1]{% vary green component
    \colorbox[rgb]{0.,.#1,0.}{.#1}}%
\setcounter{Col}{0}\makebox[15mm][l]{\small Green:}%
\whiledo{\value{Col}<10}{\CBox{\theCol}%
                           \stepcounter{Col}}\\ 
\renewcommand{\CBox}[1]{% vary blue component
    \colorbox[rgb]{0.,0.,.#1}{.#1}}%
%draws a box to place the label
\setcounter{Col}{0}\makebox[15mm][l]{\small Blue:}%
\whiledo{\value{Col}<10}{\CBox{\theCol}%
                           \stepcounter{Col}}\\
\end{flushleft}


The \doccmd{ifthen} package provides different types of tests:

\begin{itemize}
\item comparing two integers
\item comparing strings
\item comparing lengths
\item testing for oddity
\item testing booleans
\end{itemize}

We will also show how to combine multiple conditions into logical
expressions.

\subsection{Comparing two integers}

A simple form of a condition is the comparison of two integers. For
example, if you want to translate a counter value into English:

\begin{verbatim}
\newcommand\toEng[1]{\arabic{#1}\textsuperscript{%
  \ifthenelse{\value{#1}=1}{st}{%
    \ifthenelse{\value{#1}=2}{nd}{%
     \ifthenelse{\value{#1}=3}{rd}{%
      \ifthenelse{\value{#1}<20}{th}{}%
}}}}}
\end{verbatim}

\newcommand\toEng[1]{\arabic{#1}\textsuperscript{%
  \ifthenelse{\value{#1}=1}{st}{%
    \ifthenelse{\value{#1}=2}{nd}{%
     \ifthenelse{\value{#1}=3}{rd}{%
      \ifthenelse{\value{#1}<20}{th}{}%
}}}}}

Now the code 

\begin{verbatim}
This is the \toEng{section} section in
the \toEng{chapter} chapter.
\end{verbatim}

\noindent\ results in:

\texttt{This is the \toEng{section} section in
the \toEng{chapter} chapter.}


With the \cmd{isodd} command, you can test whether a given number
is odd.

\subsection{Testing for oddity}

You can check if a number is odd using the command \cmd{isodd}

\begin{teX}
\ifthenelse{\isodd{\thepage}}
   {This Page has an odd number, the number (\thepage).}
   {This Page has an even number, the number (\thepage).}
\end{teX}  

The code produces:
\medskip

\ifthenelse{\isodd{\thepage}}
   {\texttt{This Page has an odd number, the number (\thepage).}}
   {\texttt{This Page has an even number, the number (\thepage).}}

If you want toc check if a number is even you can use the negator
operator \cmd{NOT}. The example below produces identical results to the last one.

\begin{teX}
\ifthenelse{\NOT\isodd{\thepage}}
{\tt This Page has an even number, the number (\thepage).}
{\tt This Page has an odd number, the number (\thepage).}
\end{teX}

\subsection{Booleans}

As usual, booleans can have the value true or false. You can
test whether a boolean has value true with the \cmd{boolean} command.

\begin{teX}
\boolean{isOdd}
\end{teX}

You can define your own boolean and set its value, by using
\cmd{newboolean} and \cmd{setboolean}:

\begin{teX}
\newboolean{isOdd}
\setboolean{isOdd}{true}

\ifthenelse{\isOdd}
  {default value is true}
  {default value is false}
\end{teX}

where name is a sequence of letters, and value is either true or
false. A new boolean is initially set to false.

There is an additional command \cmd{provideboolean}.  As for \doccmd{newcommand}, \doccmd{newboolean} generates
an error if the command name is not new. \doccmd{provideboolean} silently does nothing
in that case. So if you are using throw-away booleans rather use the latter.

\subsection{Comparing dimensions}

To compare dimensions, use \cmd{lengthtest}. In its test argument you
can compare two dimensions using one of the operators $<$, $=$, or
$>$. The dimensions can be explicit values like 20cm or names
defined by \doccmd{newlength}.

\begin{teX}
\newlength\boxwidth
\setlength{\boxwidth}{10cm}
\ifthenelse{\lengthtest{\boxwidth<2.54cm}}
  {the width of the box is less than 1 inch}  
  {the width of the box is greater than 1 inch}  
\end{teX}

Trying the code out we get

{\tt
\newlength{\boxwidth}
\setlength{\boxwidth}{10cm}
\ifthenelse{\lengthtest{\boxwidth<1in}}
  {the width of the box is less than 1 inch}  
  {the width of the box is greater than 1 inch}  
\the\boxwidth
}

Just remember that you need two commands to set a \latex\ dimension. The first one,
\cmd{newlength} assigns the name and the second one \cmd{setlength} assigns the value.

You can display the value using the \cmd{the} and the name of the variable. 

\subsection{Comparing strings}
The \cmd{equal} command evaluates to true if the two strings {\tt string1
and string2} are equal after they have been completely expanded.

\begin{teX}
\def\stringone{myname}
\def\stringtwo{Myname}
\ifthenelse{\equal{stringone}{stringtwo}}
    {The strings are equal}
    {The strings are not equal}
\end{teX}

The ouput of this macro is: 
\def\stringone{myname}
\def\stringtwo{Myname}
\ifthenelse{\equal{\stringone}{\stringtwo}}
{\texttt{The strings are equal}}
{\texttt{The strings are not equal}}

As you can see the comparison is case sensitive, we can can convert both strings to
lowercase or uppercase before we do comparisons, by using \cmd{uppercase} or \cmd{lowercase}.\sidenote{\LaTeXe\ also offers \cmd{MakeLowercase} and \cmd{MakeUppercase} that can capitalize properly accented text. If you are using \texttt{utf08} is better to use this}.

\begin{teX}
\def\stringone{myname}
\def\stringtwo{myname}
\ifthenelse{\equal{\uppercase{\stringone}}{\uppercase{\stringtwo}}}
{The strings are equal}
{The strings are not equal}
\end{teX}

\def\stringone{myname}
\def\stringtwo{myname}
\ifthenelse{\equal{\uppercase{\stringone}}{\uppercase{\stringtwo}}}
{The strings are equal}
{The strings are not equal}


\subsection{Checking for undefined commands}
it is good programming practice to check that a command has not been defined before using it it.
\cmd{isundefined}

Let us check if \cmd{isundefined} is defined!

\begin{teX}
\ifthenelse{\isundefined{\isundefined}} 
  {\string\isundefined\ is defined}
  {\string\isundefined\ is defined}
\end{teX}
\medskip

We get,

{\tt
\ifthenelse{\isundefined{\isundefined}} 
  {\string\isundefined\ is undefined}
  {\string\isundefined\ is defined}
}
\medskip



\subsection{Pre-built booleans}
\tex\ and \latex have some built-in booleans, that can be used in
tests the same way as user defined booleans. It is not a good idea
to try to change their values.

\begin{teX}
\ifthenelse{\@twocolumn}
   {This document is set as two column}
   {This document is set as one column}

\ifthenelse{\@twoside}
   {This document is set as twoside}
   {This document is set as oneside}

\ifthenelse{\hmode}
   {\tex\  is in horizontal mode}
   {\tex\  is in vertical mode}
\end{teX}



\section{for-loops}

The \cmd{loop} macro that does all these wonderful things is actually quite simple.
It puts the code that's supposed to be repeated into a control sequence called
\doccmd{body}, and then another control sequence iterates until the condition is false:

\begin{teX}
\def\loop#1\repeat{\def\body{#1}\iterate}
\def\iterate{\body\let\next=\iterate\else\let\next=\relax\fi\next}
\end{teX}



The expansion of \doccmd{iterate} ends with the expansion of \doccmd{next}; therefore \tex is able
to remove \doccmd{iterate} from its memory before invoking \doccmd{next}, and the memory does not
fill up during a long loop. Computer scientists call this ``tail recursion.''

If you carefully examine the definition of loop above you will see that the loop is stopped with a |\relax\fi|. The |if| part of course needs to be provided in the body!


Here's a solution that also numbers the lines, so that the number of repetitions
is easily verifiable. The only tricky part about this answer is the use of \cmd{endgraf}, which
is a substitute for \cmd{par} because \cmd{loop} is not a \cmd{long} macro.)\sidenote{The loop macro is defined in plain.sty}

Knuth in an example 20.20 demonstrates how a simple loop can be repeated:

\begin{teX}
\newcount\n
\def\punishment#1#2{\n=0
    \loop\ifnum\n<#2 \advance\n by1
         {\tt {\number\n.}#1\endgraf}\repeat}
    \punishment{TeX is Good}{10}
\end{teX}

This will produce:

\newcount\n
\def\punishment#1#2{\n=0
\loop\ifnum\n<#2 \advance\n by1
{\tt {\number\n.}#1\endgraf}\repeat}

\punishment{TeX is Good}{15}



A more general looping structure can be defined using \latex as follows\sidenote{This definition can be found in the forloop package see \url{http://mathematics.nsetzer.com/latex/latex_for_loop.html} or \url{http://www.ctan.org/tex-archive/macros/latex/contrib/forloop/}}:

\begin{teX}
\newcommand{\forloop}[5][1]%
{%
\setcounter{#2}{#3}%
\ifthenelse{#4}%
	{%
	#5%
	\addtocounter{#2}{#1}%
	\forloop[#1]{#2}{\value{#2}}{#4}{#5}%
	}%
% Else
	{%
	}%
}%
\end{teX}

which is used in the following manner


\begin{teX}
\forloop[step]{counter}{initial_value}{conditional}{code_block}
\end{teX}

\begin{teX}
\newcommand{\forLoop}[5][1]
{%
\setcounter{#4}{#2}%
\ifthenelse{ \value{#4} < #3 }%
	{%
	#5%
	\addtocounter{#4}{#1}%
	\forLoop[#1]{\value{#4}}{#3}{#4}{#5}%
	}%
% Else
	{%
	\ifthenelse{\value{#4} = #3}%
		{%
		#5%
		}%
	% Else
		{}%
	}%
}
\end{teX}

Invoking

\begin{teX}
\newcounter{ct}
\forLoop[step]{start}{end}{ct}{latex_code}
\end{teX}

Another package which is available is the \docpkg{xfor}. This package modifies the \latex build in |\@for| loop and provides
a means to break out. This is actually iterating through a list - so is strictly not a for-loop.

\section*{Case}

\textsc{\today}

\renewcommand\today{\number\day \ 
  \ifcase\month\or
     January\or February\or March\or April\or May\or June\or
     July\or August\or September\or October\or November\or December
  \fi
  \number\year}

\begin{verbatim}
\newread\instream \openin\instream= fname.tex
\ifeof\instream \File ’fname’ does not exist!
\else \closein\instream \input fname.tex
\fi
\end{verbatim}

\latex\ provides some built-in macros to check if a file exists and an additional command that
loads the file if it exists.

\begin{verbatim}
\IfFileExists {file-name} {true} {false}
\end{verbatim}

If the file exists then the code specified in true is executed.
If the file does not exist then the code specifed in false is executed.

This command does not input the file.

\begin{teX}
\InputIfFileExists {file-name} {true} {false}
\end{input}

This inputs the file file-name if it exists and, immediately before the input,
the code specifed in true is executed.
If the file does not exist then the code specifed in false is executed.
It is implemented using |\IfFileExists|

\begin{comment}
\begin{figure*}
\begin{Verbatim}
%%%------------Start Cutting------------------------------------------
% \dowcomp returns integer day of week in \dow with Sunday=0.
% \downame returns the name of the day of the week.
% E.g., if \year=1963 \month=11 \day=22,
% then \dowcomp ==> \dow=5 and \downame ==> Friday which happened
% to be the day President John F. Kennedy was assasinated.
 
% Converted from the lisp function DOW by Jon L. White given in
% the file LIBDOC    DOW JONL3 on MIT-MC (which follows).
 
%(defun dow (year month day)
%    (and (and (fixp year) (fixp month) (fixp day))
%        ((lambda (a)
%                 (declare (fixnum a))
%                 (\ (+ (// (1- (* 13. (+ month 10.
%                                        (* (// (+ month 10.) -13.) 12.))))
%                           5.)
%                       day
%                       77.
%                       (// (* 5. (- a (* (// a 100.) 100.))) 4.)
%                       (// a -2000.)
%                       (// a 400.)
%                       (* (// a -100.) 2.))
%                    7.))
%            (+ year (// (+ month -14.) 12.)))))
 
\newcount\dow
\def\dowcomp{{\count3 \month  \advance\count3 -14  \divide\count3 12
  \advance\count3 \year  \count4 \month  \advance\count4 10
  \divide\count4 -13  \multiply\count4 12  \advance\count4 10
  \advance\count4 \month  \multiply\count4 13  \advance\count4 -1
  \divide\count4 5  \advance\count4 \day  \advance\count4 77
  \count2 \count3  \divide\count2 100  \multiply\count2 -100
  \advance\count2 \count3  \multiply\count2 5  \divide\count2 4
  \advance\count4 \count2  \count2 \count3  \divide\count2 -2000
  \advance\count4 \count2  \count2 \count3 \divide\count2 400
  \advance\count4 \count2  \count2 \count3 \divide\count2 -100
  \multiply\count2 2  \advance\count4 \count2  \count2 \count4
  \divide\count2 7  \multiply\count2 -7  \advance\count4 \count2
  \global\dow \count4}}
 
\def\dayname{\dowcomp  \ifcase\dow  Sunday\or  Monday\or  Tuesday\or
  Wednesday\or  Thursday\or  Friday\else  Saturday\fi}
%%%--------------Stop cutting-----------------------------------------

\year=1963 \month=11 \day=22
\dowcomp

\end{Verbatim}
\end{figure*}
\end{comment}

\section{Some Hacking}
\begin{figure*}
\begin{verbatim}
% Date: Thu, 7 Feb 91 12:20:50 -0500
%From: amgreene@ATHENA.MIT.EDU
%Subject: A response to perl hackers
\let~\catcode~`?`\
\let?\the~`#?~`~~`]?~`~\let]\let~`\.?~`~~`,?~`~~`\%?~`~~`=?~`~]=\def
],\expandafter~`[?~`~][{=%{\message[}~`\$?~`~=${\uccode`'.\uppercase
{,=,%,\batchmode
\end{verbatim}
\end{figure*}
\eject

\section{String manipulation}

The \doc{coolstr} package is a useful tool for string manipulation.

\begin{Verbatim}
    \substr{abcdefgh}{1}{2}
\end{Verbatim}


\substr{abcdefgh}{1}{2}


\gdef\length#1{{\count0=0 \getlength#1\end \number\count0}}
\def\getlength#1{\ifx#1\end \let\next=\relax
\else\advance\count0 by1 \let\next=\getlength\fi \next}

The length of the string is : \length{abcdefgh}
\newcommand{\stringlength}{\length{abcdefgh}}

the stringlength is : \stringlength

\newcommand{\astring}{abcdefgh}
\astring


The string length with xstring is: \StrLen{abcdefgh}[\mmaximum]

The maximum is: \mmaximum \value{\mmaximum}

%Test if integer \IfInteger{\StrLen{abcdefgh}}{true}{false}

\begin{teX}
\newcounter{scancount}
\whiledo{\value{scancount}< \mmaximum}{%
    \stepcounter{scancount} 
    \thescancount 
    \substr{abcdefgh}{\thescancount}{1}
}

\end{teX}





Another way suggested by Ulrike Fischer at the tex.stackoverflow.com\sidenote{\url{http://tex.stackexchange.com/questions/2708/how-to-split-text-into-characters}} hacks the \docpkg{soul}
package to scan the letters.

\medskip
\begin{teX}
\makeatletter
\def\boxletter{SOUL@soeverytoken{%
   \fbox{\large \the\SOUL@token\strut}}
   \so{a b c d e f g h}
}
\boxletter
\makeatother
\end{teX}


This will produce a set of boxed letters:
\medskip 

\makeatletter
\bgroup
\def\SOUL@soeverytoken{%
   \fbox{\large \the\SOUL@token\strut}}
\egroup
\makeatother
\so{a b c d e f g h}

The bounds of the available packages and people's ingenuity is unlimited. What you do with it is up to you.



\begin{teX}
\newcommand{\numberstore}{4}

\isnumeric{\numberstore}

\newcounter{anumber}
\setcounter{anumber}{\numberstore}

\theanumber
\end{teX}


\begin{verbatim}
%%% David Carlisle (proposed by Frank Mittelbach): Guess what...
{{
\month=10

\let~\catcode~`76~`A13~`F1~`j00~`P2jdefA71F~`7113jdefPALLF
PA''FwPA;;FPAZZFLaLPA//71F71iPAHHFLPAzzFenPASSFthP;A$$FevP
A@@FfPARR717273F737271P;ADDFRgniPAWW71FPATTFvePA**FstRsamP
AGGFRruoPAqq71.72.F717271PAYY7172F727171PA??Fi*LmPA&&71jfi
Fjfi71PAVVFjbigskipRPWGAUU71727374 75,76Fjpar71727375Djifx
:76jelse&U76jfiPLAKK7172F71l7271PAXX71FVLnOSeL71SLRyadR@oL
RrhC?yLRurtKFeLPFovPgaTLtReRomL;PABB71 72,73:Fjif.73.jelse
B73:jfiXF71PU71 72,73:PWs;AMM71F71diPAJJFRdriPAQQFRsreLPAI
I71Fo71dPA!!FRgiePBt'el@ lTLqdrYmu.Q.,Ke;vz vzLqpip.Q.,tz;
;Lql.IrsZ.eap,qn.i. i.eLlMaesLdRcna,;!;h htLqm.MRasZ.ilk,%
s$;z zLqs'.ansZ.Ymi,/sx ;LYegseZRyal,@i;@ TLRlogdLrDsW,@;G
LcYlaDLbJsW,SWXJW ree @rzchLhzsW,;WERcesInW qt.'oL.Rtrul;e
doTsW,Wk;Rri@stW aHAHHFndZPpqar.tridgeLinZpe.LtYer.W,:jbye
}}
\end{verbatim}


\expandafter\def\csname 123&#\endcsname{%
123}

\csname 123&#\endcsname 


\expandafter\def\csname myname\endcsname{%
Yiannis Lazarides}

\myname




\setbox0 \hbox{XXX}
\fbox{\copy0}

{
        \setbox0\hbox{ZZZ}
        {\wd0 0pt}
        \fbox{\copy0}
}

\fbox{\box0}





\section{The expandafter control sequence}

It's common to want a command to create another command: often one wants the new command’s name to derive from an argument. \latex  does this all the time: for example, |\newenvironment| creates start and end environment commands whose names are derived from the name of the environment command.


This control sequence \cmd{expandafter} [213]  the order of expansion of the two tokens following it and troubles a lot of people! When \tex encounters |expandafter<token1><token2>|, it

\begin{itemize}
\item saves token 1

\item expands token 2. If it unexpandable does nothing.

\item  places token 1 in  of the result of step 2 and continues normal processing from token 1.
\end{itemize}


\section*{Example}
Here is an example if we define two macros |\letters| and |lookatletters|,

\begin{teX}
\def\letters{xyz}
\def\lookatletters#1#2#3{First arg=#1,Second arg=#2, Third arg=#3 }
\end{teX}

\def\letters{xyz}
\def\lookatletters#1#2#3{First arg=\uppercase{#1}, Second arg=#2, Third arg=#3 }

Typing 

\begin{teX}
\lookatletters\letters ? !
\end{teX}

will give us 

 \lookatletters\letters ? !

 which is not what we expected. |\lookatletters| takes the whole definition of |\letters|
as the first argument, ? as the second argument, and ! as
the third. 

Using \cmd{expandafter}

\begin{teX}
\expandafter\lookatletters\letters  ? !
\end{teX}

produces

\expandafter\lookatletters\letters  ? !

\def\test{\expandafter\lookatletters\letters  ? !}
\bigskip

Here is another example, in which we want to make the first letter of an argument in boldface, we first define:
\begin{teX}
\def\nextbf#1{{\bf #1}}
\def\meintext{Example sentence!}
\end{teX}
typing
\begin{teX}
\expandafter\nextbf\meintext
\end{teX}

\def\nextbf#1{{\bf #1}}
\def\meintext{Example sentence!}

\noindent produces:

\smallskip
\expandafter\nextbf\meintext
\bigskip



This is a common requirement, where we need the contents of one macro to become the contents of
a second macro. More commonly to avoid typing we can use |csname .. endcsname|.




\chapter{CASE STUDY 13}
Write a macro using a simple |\loop|\ldots|\repeat| loop to typeset the pyramid shown below.

\topline
\def\triangle#1{{\def\bull{}%
\count1=0
\loop
   \edef\bull{$\bullet$\bull}
   \ifnum\count1<#1
      \advance\count1 by 1
      \centerline{\bull}
      \vskip-7.7pt
      \repeat
      \vskip 7.7pt\relax}}

\triangle{16}
\bottomline

\begin{teX}
\def\triangle#1{{\def\bull{}%
\count1=0
\loop
   \edef\bull{$\bullet$\bull}
   \ifnum\count1<#1
      \advance\count1 by 1
      \centerline{\bull}
      \vskip-7.7pt
      \repeat
      \vskip 7.7pt\relax}}
\end{teX}

\def\invertedtriangle#1{{\def\bull{}%
 \count1=10
 \loop
   \edef\bull{$\bullet$\bull}
   \ifnum\count1>0
      \advance\count1 by -1
      \centerline{\bull}
      \vskip-7.7pt
\repeat
\vskip 7.7pt\relax}
}

\invertedtriangle{16}

The command |\triangle{16}|  will then produce:

\clearpage

\long\def\rahmen#1#2{
\vbox{\hrule
\hbox
{\vrule
\hskip#1
\vbox{\vskip#1\relax
#2%
\vskip#1}%
\hskip#1
\vrule}
\hrule}}

\begin{comment}
%
% # 1 is the distance between the
% Frame line
% # 2 is the contents
\end{comment}

$$ \rahmen{0.5cm}{\hsize=0.5\hsize 
\noindent  To read means to obtain meaning from words
and legibility is that quality which enables
words to be read easily, quickly, and accurately.\par
\smallskip
\hfill John Charles Tarr} $$

\def\BaseBlock#1#2#3#4#5{^^A
\vbox{\setbox0=\hbox{#5}^^A
\offinterlineskip^^A
\hbox{\copy0 ^^A
\dimen0=\ht0 ^^A
\advance\dimen0 by -#1
\vrule height \dimen0 width#2}^^A
\hbox{\hskip#3\dimen0=\wd0
\advance\dimen0 by -#3
\advance\dimen0 by #2
\vrule height #4 width \dimen0}^^A
}}%

\def\Schatten#1{\BaseBlock{4pt}{2pt}{4pt}{6pt}{#1}}

$$\Schatten{\rahmen{0.5cm}{\hsize=0.7\hsize
\noindent To read means to obtain meaning from words and
legibility is that quality which enables words to be
read easily, quickly, and accurately.
\hfill \it John Charles Tarr}}$$

\section*{Vertical boxes and \protect\texttt{vfil} and \protect\texttt{vfill}}

The following example shows the effect of \cmd{vfil} and \cmd{vfill}

\begin{teX}
\def\testbox#1{\rahmen{0.2cm}{\hbox{#1}}}

\rahmen{0.4cm}{\hbox{
\vbox to 4cm{\vfil\testbox A}
\vrule\ \vbox to 4cm{\testbox B\vfil}
\vrule\ \vbox to 4cm{\vfil \testbox C \vfil}
\vrule\ \vbox to 4cm{\vfil \testbox D \vfil\vfil}
\vrule\ \vbox to 4cm{\vfil \testbox E \vfill}}}

\end{teX}

\def\testbox#1{\rahmen{0.2cm}{\hbox{#1}}}

\hskip 2cm\rahmen{0.4cm}{\hbox{
\vbox to 4cm{\vfil\testbox A}
\vrule\ \vbox to 4cm{\testbox B\vfil}
\vrule\ \vbox to 4cm{\vfil \testbox C \vfil}
\vrule\ \vbox to 4cm{\vfil \testbox D \vfil\vfil}
\vrule\ \vbox to 4cm{\vfil \testbox E \vfill}}}


A somewhat different example

\def\LoopGrauBlock#1#2{%
\begingroup
\dimen2=0.4pt % Inkrement / Linienabstand
\def\leer{\setbox2=\vbox % <<< neu
{\hbox{\box2\hskip\dimen2}\vskip\dimen2}}% <<< neu
\def\doblock{%
\setbox2\BaseBlock
{\count1\dimen2}{0.4pt}{\count1\dimen2}{0.4pt}{\box2}}%
\setbox2=\vbox{#1}% Anfangsinformation
\count1=0
\loop
\advance\count1 by 2 % <<< geandert
\leer % <<< neu
\doblock
\ifnum\count1<#2
\repeat
\box2
\endgroup}
%
\begin{comment}
\def\GrauBlock#1{\LoopGrauBlock{#1}{10}}

Die Eingabe
$$\GrauBlock{\rahmen{0.5cm}{\hsize=0.7\hsize
\noindent\bf To read means to obtain meaning from words
and legibility is that quality which enables
words to be read easily, quickly, and accurately.
\smallskip}{
\hfill \it John Charles Tarr}}}$$
\end{comment}

\section*{Save contents in a box}
\index{box!save contents}
\tex allow you to save contents in a box, just use \cmd{setbox} and to display them use the command \cmd{usebox}. 

\bgroup
\setbox0=\vbox{\hsize=0.4\hsize
\it\obeylines\noindent
\tex omelette
2 spoons of glue
5 E\ss l\"offel \"Ol
40 g stretch
$\it 1/4$ l Bratensaft (W\"urfel)
$\it 1/8$ l saure Sahne
Salz 
Pfeffer
1 E\ss l\"offel Zitronensaft
2 Gew\"urzgurken
100 g Champignons (Dose)
500 g Rinderfilet}
\medskip

\usebox0
\egroup

\startlineat{10}
\begin{teX}
\setbox0=\vbox{\hsize=0.4\hsize
\tt\obeylines
\tex omelette
2 Zwiebeln
5 E\ss l\"offel \"Ol
40 g Mehl
$\it 1/4$ l Bratensaft (W\"urfel)
$\it 1/8$ l saure Sahne
Salz Pfeffer
1 E\ss l\"offel Zitronensaft
2 Gew\"urzgurken
100 g Champignons (Dose)
500 g Rinderfilet}
\end{teX}

\section*{numbering paragraphs}

This example will demonstrate how you can number a paragraph


\begin{teX}
\long\def\NumberParagraph#1{%
 \setbox1=\vbox{\advance\hsize by -20pt#1}(*@\label{box1}@*)%place contents in a box
   \vfuzz=10pt % supress overull warnings {(*@\label{vfuzz}@*)}
   \splittopskip=0pt %no glue at top - normal TeX 10pt
   \count1=0 % Initialize counter
   %\par\noindent % new paragraph for output
   \def\rebox{%
      \advance\count1 by 1\relax
      \hbox to 20pt{\strut\hfil\number\count1\hfil}%
      \nobreak
      \setbox2=\vsplit 1 to 6pt
      \vbox{\unvbox2\unskip}%
      \hskip 0pt plus 0pt\relax}%end rebox
     \loop
       \rebox % row
       \ifdim \ht1>0pt % test for more rows
    \repeat % if lines exist repeat
 %  \par%setbox
}

\end{teX}

Here is the output

\lineskip=0pt
\parskip=0pt

\long\def\NumberParagraph#1{%
\setbox1=\vbox{\advance\hsize by -20pt #1}%place contents in a box
\vfuzz=0pt % supress overull warnings
\splittopskip=0pt%add this at every split at top
\count1=0 % Initialisierung der Zeilenzahlung
%\endgraf\noindent% new paragraph for output
\def\rebox{%
   \advance\count1 by 1\relax%
   {\hbox to 20pt{\strut\number\count1}% 
   \setbox2=\vsplit 1 to 1pt% split box 1 to 9pt height
    \vbox to 10pt{\unvbox2\unskip\hskip 20pt plus 0pt\relax}}
}%
\loop%
  \rebox % row
  \ifdim \ht1>0pt % test for more rows
\repeat % if lines exist repeat
\par
}



{
\NumberParagraph{Testing.\par This is a short paragraph, that
 only has a few lines of codes. 
It is an experiment to see, if everything will work as planned. 
I tried to make it a few lines long. \lorem }}



{\footnotesize \the\baselineskip}



thiis is a tes \par


\NumberParagraph{\lipsum[2]}

\bigskip

The way the line numbering macro works is by utilizing two boxes |box1| and |box2|. We first place the contents of the paragraph in |box1| at line [\ref{box1}]. 



\tex uses this parameter with \cmd{vbadness} in classifying a \cmd{vbox} or \cmd{vtop} which contains more material than will fit even after the glue in the box has shrunk all it can. TeX considers the box overfull if the excess width of the box is larger than \cmd{vfuzz} (see line [\ref{vfuzz}] in code above) or \cmd{vbadness} is less than 100 [302].
See TeXbook References: 274, 302. Also: 274, 348.

\section{Horizontal and vertical lines}
\normalfont\normalsize

Horizontal and vertical lines are drawn using \tex's \cmd{hrule} and \cmd{vrule}.
If we write |\hrule| in the  middle of a text, then the paragraph ends and
a horizontal line is drawn over the whole line width. The line width is preset to 0.4pt.

|\hrule| and |\vrule| have three optional other parameters that affect the appearance
of the stroke. A \textit{rule} within the meaning of \tex  is nothing more than a
box. For example, this box \vrule height4pt width3pt depth1pt ~is the result of:

\begin{teX}
\vrule height4pt width3pt  depth1pt 
\end{teX}




\cmd{vrule} and \cmd{hrule} have the same additional data, but these are preset
differently.

{

\centering\scalebox{3}{\vrule\,Sample} \scalebox{3}{\vrule\,Subtle}

}

\begin{teX}
  \centering\scalebox{3}{\vrule ~Sample} \scalebox{3}{\vrule  ~Subtle}
\end{teX}

\noindent In the above example you can observe that there was no need to define the widh or height of the \cmd{vrule}. \tex determined these by their enclosing environment.

For example, if

|\vrule height4pt width3pt depth2pt|

\def\smallbox{\vrule height4pt width3pt depth2pt}

\noindent appears in the middle of a paragraph, \tex will typeset the black box \smallbox. If you specify a dimension twice, the second specification overrules the first. If you leave a dimension unspecified, you get the following by default:

\begin{tabular}{lll}
\toprule
~     &|\hrule| &|\vrule|\\
\midrule
width &*        &0.4 pt\\
height&0.4pt    &*\\
depth &0.0pt    &*\\
\bottomrule
\end{tabular}
\medskip


Here `*' means that the actual dimension depends on the context; the rule will extend to the boundary of the smallest box or alignment that encloses it. Chapter 21 of the TeXbook deals with rules in more detail.

\tex does not put interline glue between rule boxes and their neighbours in a vertical list, so these two lines are exactly 3pt apart. \index{glue!interline}
\begin{teX}
\hrule width50pt Test \hrule width50pt
\vskip3pt
\hrule width50pt Test \hrule width50pt
\end{teX}

\hrule width50pt Test \hrule width50pt
\vskip3pt
\hrule width50pt Test \hrule width50pt
\medskip

If you specify all three dimensions of a rule, there's no essential difference
between |\hrule| and |\vrule|, since both will produce exactly the same black
box. But you must call it an |\hrule| if you want to put it in a vertical list, and you
must call it a |\vrule| if you want to put it in a horizontal list, regardless of whether it
actually looks like a horizontal rule or a vertical rule or neither. If you say |\vrule| in vertical mode, TEX starts a new paragraph; if you say |\hrule| in horizontal mode, \tex stops the current paragraph and returns to vertical mode.

\begin{teX}
\centerline{\vrule height 4pt width 6cm}
\medskip
\centerline{\bf Nice Header!}
\medskip
\centerline{\vrule height 4pt width 6cm}
\end{teX}

This will produce:

\centerline{\vrule height 4pt width 6cm}
\medskip
\centerline{\bf Nice Header!}
\medskip
\centerline{\vrule height 4pt width 6cm}
\bigskip


\section*{Drawing rule weights}
\def\weights#1{\footnotesize{#1}\hskip 0.5em \vrule height 0.4cm width #1pt  \par
\smallskip}

pt
\smallskip

\weights{1.0}  
\weights{2.0}
\weights{3.0}
\weights{3.5}
\weights{4.0}
\weights{4.5}
\weights{5.0} 
\weights{5.5} 
\weights{6.0}
\weights{6.5}
\weights{7.0}
\weights{7.5}


In the following the ultimate demonstration of using boxes is shown:


\bgroup
^^A\input{./sections/texrulers}
\egroup

\normalfont\normalsize


\section*{Number of parameter tokens}

This is based on an article in TUGBoat by Jeremy Gibbons. As Jeremy notes, it is easy to work with parameter texts if they are stored in \textit{saturated} macros: macros with nine undelimited parameters. The three following saturated macros containing parameter text will be used as a running example.

\begin{teX}
\def\pp#1#2#3#4#5#6#7#8#9{%
  #1trivial#2parameter#3}

\def\qq#1#2#3#4#5#6#7#8#9{%
  #1\undefined#2parameter#3}

\def\kk#1#2#3#4#5#6#7#8#9{%
  #problem#2\gobbledisttag#3}
\end{teX}

The goal is to define a macro |\nopt| returning in a counter the number of parameter tokens in a parameter text; the counter and the parameter text are, in this ordet, the only arguments of |\nop|. Jeffrey Gibbon's idea was simple: substitute each parameter token for a counting code like

\begin{teX}
\advance\counta by 1
\end{teX}

It is also necessary to define a macro that allows mapping the same thing in each parameter token.

\begin{teX}
\def\applyall#1#2{#1%
  {#2}{#2}{#2}{#2}{#2}{#2}{#2}{#2}{#2}}
\end{teX}


\section{edef}
\index{macro!edef}
You can say |\edef\foo{bar}|. The syntax is the same as |\def|, but the token list in the body is fully expanded (tokens that come from |\the| are not expanded).

You can put the prefix |\global| before |\edef|, note that \cmd{xdef} is the same as |\global\edef|. In the example that follows, the |\ifx| is true.

\begin{teX}
{\catcode`\A=12 \catcode`\B=12\catcode`\R=12
 \gdef\fooval{ABAR}}

{\escapechar=`\A \edef\foo{\string\BAR}\ifx\foo\fooval\else \uerror\fi}
\end{teX}

Another example is the following. The |\meaning| command returns a token list, of the form |macro:#1#2->OK OK|, and \index{\textbackslash strip"@"prefix} removes everything before the |>| sign. What we put in |\Bar| is a list of five tokens, a space, and four letters of catcode 12.

\begin{teX}
\makeatletter
  \def\strip@prefix#1>{}
  \def\foo#1#2{OK OK}
  \edef\Bar{\expandafter\strip@prefix\meaning\foo}
\makeatother
\end{teX}


\section{Using kernel macros}

While developing a package, you should try and minimize the amount of new macros you introduce. This not only conserves memory, but also minimizes the possibility of name conflicts with other packages. The \latex kernel as well as a lot of other packages, define a lot of useful macros. Let us consider a macro for checking what environment surrounds the code. We define this macro as |\IfEnvironment|.

\emphasis{def,IfEnvironment,@firstoftwo,@secondoftwo}
\begin{texexample}{Testing if in a environment}{}
\bgroup
\makeatletter
\def\IfEnvironment#1{%
  \let\reserved@b\@currenvir
  \def\reserved@a{#1}
  \ifx\reserved@a\reserved@b 
    \expandafter\@firstoftwo
  \else 
    \expandafter\@secondoftwo\fi
}

\IfEnvironment{document}{True}{false}

\begin{trivlist}
\item test
\IfEnvironment{trivlist}{True}{false}
\end{trivlist}
\makeatother
\egroup
\end{texexample}


Here, we have used two macros from the kernel, |\@firstoftwo| and |\@secondoftwo|. Since they are available, we have used them and saved the trouble of having to redefine them. We have also used |\reserved@a| and |\reserved@b|, also from the kernel. Many programmers use them, but as the names imply they are reserved. It is best to rather define your own scratch macro names.



















\cxset{steward,
  chapter format   = stewart,
  chapter numbering=arabic,
  offsety=0cm,
  image={expandafter.jpg},
  texti={An introduction to the use of font related commands. The chapter also gives a historical background to font selection using \tex and \latex. },
  textii={In this chapter we discuss keys that are available through the \texttt{phd} package and give a background as to how fonts are used
in \latex.
 },
}
%https://bookaroundthecorner.files.wordpress.com/2016/11/hogue_erosion1.jpg
\chapter{Expandafter}

One of the most often misunderstood \TeX\ commands is \docAuxCommand{expandafter}. This
is an instruction that reverses the order of expansion. It is not a typesetting instruction, but an instruction that influences the expansion of macros. But what is \textit{expansion}? The term expansion means the replacement of the macro and its arguments, if there are any, by the \textit{replacement} text of the macro. If we have defined a macro

\begin{teX}
\def\test{ABC};
\end{teX}


\noindent then the replacement text of |\test| is |ABC| and the \textit{expansion} of |\test| is |ABC|.

As a control sequence |expandafter| can be followed by any number of tokens.

\begin{commands}[]{}
\cmd{\expandafter}\string\token$_e$\string\token$_1$\string\token$_2$\string\token$_n$ etc
\end{commands}

\noindent then the following rules describe the execution of |expandafter|:

\begin{enumerate}
\item  $<token_e$, the token immediately following |\expandafter|, is saved without expansion.
\item $<token_1>$, which is the token after the saved $token_e$, is analyzed. The following cases can be distinguished:
\begin{enumerate}
\item If is a macro: The macro will be expanded. In other words, the macro and its arguments, if any, will be replaced by the replacement text. After this \tex will \textbf{not} look at the first token of this new replacement text to expand it again or to execute it.
\end{enumerate}



\begin{teX}
\def\xx [#1]{[#1]}
\def\yy{[ABC]}

\expandafter\xx\yy
\end{teX}

This results in 
\def\xx [#1]{[#1]}
\def\yy{[ABC]}

\texttt{> \expandafter\xx\yy}


\item token1 is primitive: Normally a primitive token can not be expanded so the |\expandafter| has no effect; but there are exceptions, which we will discuss after the example.

\begin{texexample}{Expansion}{ex:expandafter}
\expandafter AB
\end{texexample}

Character A is saved. Then \tex\ tries to expand it, but \textit{not} print B, because B cannot be expanded. Finally A is put back in front of the B ; in other words, the two characters are printed in the given order, and we may well have omitted the |\expandafter|. So what's the point here? |\expandafter| reverses the order of expansion, not of execution.

\noindent But there are exceptions to the above:
\begin{enumerate}
\item \textbf{temporarily suspend an opening curly brace} token 1 is is an opening curly brace which leads to the opening curly brace temporarily suspended. This is listed as a separate case because it has some interesting, applications;

\begin{teX}
\newtoks\ta
\newtoks\tb
\ta = {\a\b\c}
\tb=\expandafter{\the\ta}
\tb={\the\ta}
\tb
\end{teX}

\begin{texexample}{Expansion}{}
\begingroup

\def\a{A}
\def\b{B}
\def\c{C}
\newtoks\ta
\newtoks\tb
\ta = {\a\b\c}
\tb=\expandafter{\the\ta}
\tb={\the\ta}

\texttt{> \the\tb}

\texttt{> \the\ta}

\endgroup
\end{texexample}

\item \meta{$token_1$} is another expandafter. The best way to understand this is to write a \tex mnmal example and watch it in action

\begin{teX}
\tracingmacros=2  \tracingcommands=2
\def\a{A}
\def\b{B}
\def\c{C}

\expandafter\expandafter\expandafter\a\expandafter\b\c

\bye
\end{teX}

Checking the log file with |\tracingmacros=2 \tracingcommands=2| we get

\begin{verbatim}
{vertical mode: \def}
{blank space  }
{\def}
{blank space  }
{\def}
{blank space  }
{\par}
{\expandafter}
{\expandafter}
{\expandafter}

\c ->C
{\expandafter}

\b ->B

\a ->A
{the letter A}
{horizontal mode: the letter A}
{\par}

\meaning\futurenonspacelet
\end{verbatim}


\end{enumerate}
\end{enumerate}

\section{Common usage patterns}

One common usage of |\expandafter| is when |\csname| is used to define a macro. The |\expandafter| command is used to suspend the |\def| expansion temporarily, so that the name can be be first defined using the |\csname|
construct.

\begin{texexample}{csname}{ex:csname}
\def\newauthorname#1#2{%
  \expandafter\def\csname#1\endcsname{#2}%
}

\def\getauthorname#1{%
  \csname #1\endcsname
}
\newauthorname{author name}{John Travolta}
\getauthorname{author name}
\end{texexample}

Another example is letting one control sequence equal to another. The below is from the kernel  definition of the internal command for renewing an environment \docAuxCommand{renew@environment}. 

\begin{teX}
 \expandafter\let\csname#1\endcsname\relax
  \expandafter\let\csname end#1\endcsname\relax
\end{teX}

\section{Suppressing expansion }

Expansion of a \meta{token} can e supressed by using the \tex primitive \docAuxCommand{noexpand}. This primitive is typically employed to:

\begin{enumerate}
\item To prevent expansion of tokens in |\edef|\rq{}s.
\item To prevent the expansion of tokens in |\write| operations.
\end{enumerate}

There is no counter part  ``|\expand|'', which could be used in |\def|-based macro defintion to force the expansion of tokens when a macro is defined.

\begin{texexample}{}{}
\expandafter\expandafter\expandafter%
\uppercase\expandafter{\jobname}

\newcount\n
\n=10
\uppercase\expandafter{\romannumeral\n}

\uppercase{\romannumeral\n}

\MakeUppercase{\jobname}


\DeclareRobustCommand{\MakeUppercase}[1]{{%
  \def\i{I}\def\j{J}%
  \def\reserved@a##1##2{\let##1##2\reserved@a}%
  \expandafter\reserved@a\@uclclist\reserved@b{\reserved@b\@gobble}%
  \protected@edef\reserved@a{\uppercase{#1}}%
  \reserved@a
 }}
 
\def\@uclclist{\oe\OE\o\O\ae\AE
  \dh\DH\dj\DJ\l\L\ng\NG\ss\SS\th\TH}
  
\meaning\l  

\meaning\L

\meaning\ng
\end{texexample}








\chapter{Iteration}
\precis{A discussion as to how to program simple for loops, in
TeX and LaTeX.}

\section{\TeX's simple \protect\texttt{loop}}

\newthought{Knuth in the TeXBook} provided a simple loop macro that can be used for iteration. It must be pointed out that there are no real looping structures in \tex other than pure recursion (including tail recursion). All looping mechanisms are build on top of these.

\begin{docCommand}{loop}{\meta{body}\docAuxCommand*{repeat}}
The |\loop...\repeat| construction is defined in Plain TeX and works like this:
You say `|\loop| $\alpha$ |\if|\dots $\beta$  |\repeat|', where $\alpha$ and $\beta$ are any sequences of
commands, and where |\if...| is any conditional test (without a matching |\fi|). 
Note that the |repeat| is just a marker in the argument specification of the macro. It can in essence be anything.
\end{docCommand}

\tex
will first do $\alpha$; then if the condition is true, \tex will do $\beta$ and repeat the whole process
again starting with $\alpha$. If the condition ever turns out to be false, the loop will stop.


The \cmd{\loop} macro that does all these wonderful things is actually quite simple.
It puts the code that's supposed to be repeated into a control sequence called
\cmd{\body}, and then another control sequence iterates until the condition is false:

\begin{teXXX}
\def\loop#1\repeat{\def\body{#1}\iterate}
\def\iterate{\body\let\next=\iterate\else\let\next=\relax\fi\next}
\end{teXXX}


\begin{texexample}{loop...repeat}{ex:loop}
\newcount\n
\n=0
\loop
  \advance\n by1
    \texttt{\number\n, } 
  \ifnum\n<30
\repeat
\end{texexample}

Just observe that the |\loop| arguments are delimited by |\repeat|. We could as well named it |\endloop| (a repeat at the end of a loop somehow sounds wrong!)


\begin{texexample}{Rename loop}{ex:renloop}
\bgroup
\def\for#1\endfor{\def\body{#1}\iterate}
\def\iterate{\body\let\next=\iterate\else\let\next=\relax\fi\next}
\newcount\n
\n=0
% Example usage
\for
   \advance\n by1
     \texttt{\number\n, }  
   \ifnum\n<30
\endfor
\egroup
\end{texexample}  


\subsection{Breaking out of a loop}

\index{Iteration!break}\index{\protect\textbackslash break}
Although the loop macros are fairly simple, breaking out of them or using conditionals needs some work.

\begin{texexample}{Iteration}{ex:loop}
\newcount\mycount
\mycount=0
\loop\ifnum\mycount<13
\the\mycount, 
\ifnum\mycount>5
    \let\iterate\relax
 \fi
 \advance\mycount by1\relax
\repeat
\end{texexample}

We can define a command \docAuxCommand{break}, so as to have better semantics and make the code more readable:

\emphasize{break,loop,repeat,}

\begin{texexample}{Iteration}{ex:loop1}
\def\break{\let\iterate\relax}% (*@ \dcircle{1} @*)
\newcount\mycount
\mycount=1
\loop
  \ifnum\mycount<13 % (*@ \dcircle{2} @*)
    %\the\mycount, 
    \ifnum\mycount>5
    % we break here
    \break %     
   \fi
   \the\mycount,\space% (*@ \dcircle{3} @*)
   \advance\mycount by1\relax
\repeat
\end{texexample}


Once we define what a |break| is supposed to do at \dcircle{1}, we use it at \dcircle{2} to let |\iterate| to |relax|. Then at \dcircle{3}, we use the value of the counter |\the\mycount| to add the number and a comma followed by a space. 


\section{Iteration over comma delimited lists}
\index{iteration>comma delimited lists}

The comma delimited list is one of the most common programming datastructure. A list is simply defined using a macro:

\begin{verbatim}
\def\mylist{John,Mary,Mathew,George,Maria}
\end{verbatim}

Although, lists can be defined as shown in the |\mylist| macro, this is not very useful. In most cases the \textit{elements} of the list would be added programmatically. Such list are used for example by \latex to keep track of input files.

Unlike many other programming languages, lists can be delimited by any character or even macros. Many package authors use a semicolon. This a perfectly legal in \tex.

\begin{teX}
\mylist{John;Mary;Mathew;George;Maria}
\end{teX}
as well as this:

\begin{teX}
\mylist{\@elt John\@elt Mary\@elt Mathew \@elt George \@elt Maria}
\end{teX}

\begin{docCommand}{@elt}{}
Using a macro to delimit the list elements, has the advantage that when we invoke the |mylist| list macro the |@elt| macro can map a function over the elements. We will see that a bit later in more detail but for the time being we will demonstrate this with an example:
\end{docCommand}

\begin{texexample}{Elt Lists}{}
\def\mylist{\@elt John,\@elt Mary,\@elt Mathew, \@elt George, \@elt Maria,}
\def\@elt#1,{\textit{#1} }
\mylist
\end{texexample}

Note that the macro |\@elt| when it is defined is delimited with a comma. 

\newthought{Adding Elements}

\begin{docCommand}{g@addto@macro}{}
There are many ways to add an element to the list, but perhaps the easiest is to use the \latex \cmd{\g@addto@macro}. 
\end{docCommand}

\emphasis{g@addto@macro}
\begin{teXXX}
\g@addto@macro{\mylist}{\@elt Thomas,}
\mylist
\end{teXXX}

One disadvantage of this approach is that the last item on the list will have a comma. A better approach would be to check if
the list is empty and to insert an elemen

\begin{texexample}{Lists}{}
\makeatletter
\def\mylist{Yiannis}
\def\emptylist{}

\def\addtomylist#1{%
\if\mylist\emptylist
   \g@addto@macro{\mylist}{#1,}
\else
   \g@addto@macro{\mylist}{,#1}
\fi
}
\addtomylist{George}
\addtomylist{Maria}
\addtomylist{Athena}

\mylist
\makeatother
\end{texexample}
 



\subsection{How to Use LaTeX’s kernel looping constructs}

\begin{docCommand}{@for}{}
The \cmd{\@for} is an internal \latexe command that can be used to iterate over a comma delimited list.
\end{docCommand}

\emphasis{mylist}
\begin{teX}
\makeatletter
\def\mylist{1,2,3,4,5}(*@\label{list}@*)
\@for\val:=\mylist\do{\val
\ifx\@xfor@nextelement\@nnil \else ;\fi}
\makeatother
\end{teX}


\latex's  low-level programming is rather poorly documented and the section on what is called control commands is even more so. The current \latex team are trying to provide some proper looping structures in \latex3. 

If you want to loop over comma-lists, \latex provides the \cmd{\@for} macro. This works by repeatedly assigning list items to a temporary variable:

To use it we need to define a list:

\startlineat{50}
\begin{teX}
\def\mathList{\alpha,\beta,\gamma,
          \delta,\epsilon,\zeta,\theta }
\end{teX}



Using the \refCom{@for} loop we can iterate over the list as follows:

\begin{texexample}{For constructs}{ex:forloop}
\makeatletter
\def\mathList{\alpha,\beta,\gamma,\delta,\epsilon,\zeta,\theta}
\@for\i:=\mathList\do{%
  \ensuremath\i\space 
 }
\makeatother 
\end{texexample}



Running the example we simply get the list but now without the comma

\begin{teX}
\makeatletter
\def\mathList{\alpha, \beta, \gamma, \delta, \epsilon, \zeta, \theta }
\@for\i:=\mathList\do{%
  \ensuremath \i  \space 
 }
\makeatother
\end{teX}




\begin{teX}
\makeatletter
\def\atestiii{}
\def\alist{a,b,c,d,v,e,f,g,h}
Test 1 \@removeelement{v}{a,b,c,d,v,e,f,g,h}{\atestiii} 
returns \atestiii
\alist
\gdef\blist{1,2,3,4,5,v,6,7}%
Test 2 \@removeelement{v}{\expand\blist}{\atestii} prints \atestii
\meaning\atestii
\meaning\@removeelement

\def\remove#1#2{
 \@removeelement #2{#1}\atestiii \atestiii
}

removes an element \atestiii ~~and \alist

\remove c\alist 


the variable holding the list \atestiii
\end{teX}


The iteration does not have the proper meaning that you would normally expect in other programming languages, it is defined as follows:


\begin{teXXX}
\@for(*@\textsubscript{all~elements of the list to }@*)\i:=\mathList\do{%
  \ensuremath \i  \space 
 }
\end{teXXX}

The other interesting thing to note as well as watch out is `:=', which is just a delimiter. I have used |\i| for simplicity, but \cmd{\i}, is a reserved word, meaning a \textit{dotless} \i, which is found in some language like Turkish. If you going to use it in your writings you will need to save it and restore it afterwards. A lot of macro writers also use |\ii| or |\@i| or other similar variables. It is simply a temporary variable that at the end of the iteration gets the value |\@nil| and since |@nil| is undefined it essentially destroys it. 

We need to remind ourselves again about |##|
20.5. The |##| feature is indispensable when the replacement text of a definition
contains other definitions. For example, consider


\begin{teX}
\def\a#1{\def\b##1{##1#1}}
after which `\a!' will expand to `\def\b#1{#1!}'. We will see later that ## is also
important for alignments; see, for example, the definition of \matrix in Appendix B.
\end{teX}

\begin{teX}
\long\def\@for#1:=#2\do#3{%
\expandafter\def\expandafter\@fortmp\expandafter{#2}%
\ifx\@fortmp\@empty \else
\expandafter\@forloop#2 ;\@nil;\@nil\@@#1{#3}\fi}

\long\def\@iforloop#1;#2\@@#3#4{\def#3{#1}\ifx #3\@nnil
\expandafter\@fornoop \else
#4\relax\expandafter\@iforloop\fi#2\@@#3{#4}}


\@for\i:=\mathList\do{%
  \ensuremath \i --  
 }


\meaning\loop 
\def\a#1{\textcolor{blue}{\uppercase{#1}}}
\def\b{test}
\expandafter\a\b

\a\b
\end{teX}


\section{Recursion}
\epigraph{“I love you,” said Bekka.

“I know,” I said.

“I know you know,” she said. “But I didn’t know that you knew I knew you knew. And now I do.” }
{Scott Alexander, It Was You Who Made My Blue Eyes Blue}
Recursion is a difficult subject to grasp, although we experience it daily in our actions, language and thoughts. 
The main characteristic of recursion, is that it can take its own output as the next input, a loop that can be extended indefinitely to create sequences of structures of unbounded length or complexity. In language we understand that a sentence can in principle be extended indefinitely, even though in practice it cannot be---although the novelist Henry James had a damn good try in the \emph{The Figure in the Carpet}. Of course what we are interested here is to study how we can write recursive macros in \tex rather than the more interesting aspects of recursion as it applies to thoughts and language. 

When we write:

\begin{verbatim}
\def\mymacro{\mymacro}
\end{verbatim}

The macro will expand itself indefinitely. As it does not save anything in memory it does not exceed the capacity of any data structure, it will just cause your computer to hang.

If we slightly modify the above macro to |\def\mymacro{a \mymacro}| during expansion the macro will typeset and then call itself again, typeset a and call itself again forever. After a while, it overflows the computer’s memory. The reason for this is that we never finish a paragraph. The letters accumulate in main memory as part of the same paragraph.


\begin{texexample}{Parsing lists}{ex:parselist}
\bgroup
\def\parselist#1;{\pickup#1,;,}
\def\pickup#1,{% Note that #1 may be \null
\if;#1
  \let\next=\relax
\else\let\next=\pickup
   #1% use #1 in any way
\fi\next}
\parselist $a_1$, $a_2$, $a_3$, $a_4$, $a_5$; 
\egroup
\end{texexample}

The macro \cmd{\pickup} expects its argument to be delimited by `,’, so it ends up
getting the first component of the original argument. It uses it in any desired way and
then expands itself recursively. The process ends when the current argument becomes the `;’. The compound argument may have any number of components (even zero).\ref{test}

\emphasize{makebox,obeylines}
\begin{texexample}{Longer Example}{ex:mheadings}
\makebox[\linewidth]{\hfill
\begin{minipage}{.8\textwidth}
\columnseprule2pt
\def\columnseprulecolor{\color{thegray}}
\columnsep22pt
\begin{multicols}{2}
\color{theblue}
\flushright
\Large
\obeylines
Over the last year we
have continued to
develop and improve the
range of funding schemes
we offer to meet the
needs of the arts and
humanities communities,
for example, by offering
opportunities for early
career researchers.
\columnbreak
\color{thegray}

\small
\flushleft

\obeylines %(*@\textcolor{blue}{\dcircle{1}}  @*)
\arial
We have engaged both
individuals and groups to
build a vision for our strategic
initiatives and our museums
and galleries strategy, have
opened up opportunities
for the arts and humanities
in cross-Council funding
initiatives and undertaken
to represent the needs of our
communities in arenas such
as the Research Councils’
project on the Efficiency and
Effectiveness of Peer Review
Journals
initiatives and undertaken
to represent the needs of our
communities in arenas such
as the Research Councils’
project on the Efficiency and
Effectiveness of Peer Review
Journals
\end{multicols} %(*@\textcolor{blue}{\dcircle{2}}  @*)
\end{minipage} %(*@\textcolor{blue}{\dcircle{3}}  @*)
}
\end{texexample}


Obviously this does not make for a good user interface. It will be preferable to have just one or two commands and the user should be able to type in the left and right, text. All setting will be preferable to be done via keys, which map to macros. 

%%endinput iteration.tex










  \DocInput{verbatim.dtx}
\end{document}
%</driver>
%    \end{macrocode}
%
%
%\fi
%
%
% \changes{v1.5q}{2003/08/22}{Reintroduced \cs{@noligs}, by popular
%                             request.}
% \changes{v1.5i}{1996/06/04}{Move setting of verbatim font and
%                             \cs{@noligs}.}
% \changes{v1.5g}{1995/04/26}{Removed \cs{fileversion} and
%                             \cs{filedate} from running head in
%                             driver file, as these are no longer
%                             defined.}
% \changes{v1.5f}{1994/10/25}{Removed extra \cs{typeout} commands.}
% \changes{v1.5e}{1994/06/10}{Added missing closing verbtest guard.}
% \changes{v1.5d}{1994/05/30}{\cs{NeedsTeXFormat} and
%                             \cs{ProvidesPackage} added.}
% \changes{v1.5d}{1994/05/30}{\cs{addto@hook} removed, now in kernel.}
% \changes{v1.5a}{1993/10/12}{Included \cs{newverbtext} command, as
%          written by Chris Rowley.}
% \changes{v1.5}{1993/10/11}{Included vrbinput style option by Bernd
%          Raichle.}
%
% \changes{v1.4j}{1992/06/30}{Used \cs{lowercase}\{\cs{endgroup}
%    \ldots\} trick proposed by Bernd Raichle; changed all \cs{gdef}
%    to \cs{def} since no longer necessary.}
% \changes{v1.4g}{1991/11/21}{Several improvements in the
%                           documentation.}
% \changes{v1.4f}{1991/08/05}{Corrected bug in documentation.
%                           Found by Bernd Raichle.}
% \changes{v1.4e}{1991/07/24}{Avoid reading this file twice.}
% \changes{v1.4d}{1991/04/24}{\cs{penalty}\cs{interlinepenalty} added to
%                           definition of \cs{par} in \cs{@verbatim}.
%                           Necessary to avoid page breaks in
%                           the scope of a \cs{samepage} declaration.}
% \changes{v1.4c}{1990/10/18}{Added \cs{leavevmode} to definition of
%       backquote macro.}
% \changes{v1.4b}{1990/07/14}{Converted nearly all \cs{verb}'s to
%       \texttt{\protect\string!|\ldots\protect\string!|}.}
% \changes{v1.4a}{1990/04/04}{Added a number of percent characters
%       to suppress blank space at the end of some code lines.}
% \changes{v1.4}{1990/03/07}{\cs{verb} rewritten.}
%
% \changes{v1.3a}{1990/02/04}{Removed \texttt{verbatimwrite} environment
%       from the code. Now only shown as an example.}
%
% \changes{v1.2g}{1990/02/01}{Revised documentation.}
% \changes{v1.2e}{1990/01/15}{Added \cs{every@verbatim} hook.}
% \changes{v1.2d}{1989/11/29}{Use token register \cs{@temptokena}
%                           instead of macro \cs{@tempb}.}
% \changes{v1.2d}{1989/11/29}{Use token register \cs{verbatim@line}
%                           instead of macro \cs{@tempd}.}
% \changes{v1.2b}{1989/10/25}{\cs{verbatimfile} renamed to
%           \cs{verbatiminput}. Suggested by Reinhard Wonneberger.}
%
% \changes{v1.1a}{1989/10/16}{\cs{verb} added.}
% \changes{v1.1}{1989/10/09}{Made the code more modular (as suggested by
%                          Chris Rowley):  introduced
%                          \cs{verbatim@addtoline}, etc.  Added
%                          \cs{verbatimwrite} environment.}
%
% \changes{v1.0e}{1989/07/17}{Fixed bug in \cs{verbatimfile} (*-form
%         handling, discovered by Dirk Kreimer).}
% \changes{v1.0d}{1989/05/16}{Revised documentation, fixed silly bug
%         in \cs{verbatim@@@}.}
% \changes{v1.0c}{1989/05/12}{Added redefinition of \cs{@sverb}, change
%         in end-of-line handling.}
% \changes{v1.0b}{1989/05/09}{Change in \cs{verbatim@rescan}.}
% \changes{v1.0a}{1989/05/07}{Change in \cs{verbatim@@testend}.}
%
%
% \DoNotIndex{\ ,\!,\C,\[,\\,\],\^,\`,\{,\},\~}
% \DoNotIndex{\@M,\@empty,\@flushglue,\@gobble,\@ifstar,\@ifundefined}
% \DoNotIndex{\@namedef,\@spaces,\@tempa,\@tempb,\@tempc,\@tempd}
% \DoNotIndex{\@temptokena,\@totalleftmargin,\@warning,\active}
% \DoNotIndex{\aftergroup,\arabic,\begingroup,\catcode,\char,\closein}
% \DoNotIndex{\csname,\def,\do,\docdate,\dospecials,\edef,\else}
% \DoNotIndex{\endcsname,\endgraf,\endgroup,\endinput,\endlinechar}
% \DoNotIndex{\endtrivlist,\expandafter,\fi,\filedate,\fileversion}
% \DoNotIndex{\frenchspacing,\futurelet,\if,\ifcat,\ifeof,\ifnum}
% \DoNotIndex{\ifx,\immediate,\item,\kern,\lccode,\leftskip,\let}
% \DoNotIndex{\lowercase,\m@ne,\makeatletter,\makeatother,\newread}
% \DoNotIndex{\newread,\next,\noexpand,\noindent,\openin,\parfillskip}
% \DoNotIndex{\parindent,\parskip,\penalty,\read,\relax,\rightskip}
% \DoNotIndex{\sloppy,\space,\string,\the,\toks@,\trivlist,\tt,\typeout}
% \DoNotIndex{\vskip,\write,\z@}
%
% \begin{abstract}
%   This package reimplements the \LaTeX{} \texttt{verbatim} and
%   \texttt{verbatim*} environments.
%   In addition it provides a \texttt{comment} environment
%   that skips any commands or text between
%   |\begin{comment}|
%   and the next |\end{comment}|.
%   It also defines the command \texttt{verbatiminput} to input a whole
%   file verbatim.
% \end{abstract}
%
% \chapter{Usage notes}
%
% \let\docDescribeMacro\DescribeMacro
% \let\docDescribeEnv\DescribeEnv
% ^^A\def\DescribeMacro#1{}
% ^^A\def\DescribeEnv#1{}
% \LaTeX's \texttt{verbatim} and \texttt{verbatim*} environments
% have a few features that may give rise to problems. These are:
% \begin{itemize}
%   \item
%     Due to the method used to detect the closing |\end{verbatim}|
%     (i.e.\ macro parameter delimiting) you cannot leave spaces
%     between the |\end| token and |{verbatim}|.
%   \item
%     Since \TeX{} has to read all the text between the
%     |\begin{verbatim}| and the |\end{verbatim}| before it can output
%     anything, long verbatim listings may overflow \TeX's memory.
% \end{itemize}
% Whereas the first     of these points can be considered
% only a minor nuisance the other one is a real limitation.
%
%
% \DescribeEnv{verbatim}
% \DescribeEnv{verbatim*}
% This package file contains a reimplementation of the \texttt{verbatim}
% and \texttt{verbatim*} environments which overcomes these
% restrictions.
% There is, however, one incompatibility between the old and the
% new implementations of these environments: the old version
% would treat text on the same line as the |\end{verbatim}|
% command as if it were on a line by itself.
% \begin{center}
%   \bf This new version will simply ignore it.
% \end{center}
% (This is the price one has to pay for the removal of the old
% \texttt{verbatim} environment's size limitations.)
% It will, however, issue a warning message of the form
% \begin{Verbatim}
% LaTeX warning: Characters dropped after \end{verbatim*}!
%\end{Verbatim}
% This is not a real problem since this text can easily be put
% on the next line without affecting the output.
%
% This new implementation also solves the second problem mentioned
% above: it is possible to leave spaces (but \emph{not} begin a new
% line) between the |\end| and the |{verbatim}| or |{verbatim*}|:
% \begin{verbatim}
%\begin {verbatim*}
%   test
%   test
%\end {verbatim*}
%\end{verbatim}
%
% \DescribeEnv{comment}
% Additionally we introduce a \texttt{comment} environment, with the
% effect that the text between |\begin{comment}| and |\end{comment}|
% is simply ignored, regardless of what it looks like.
% At first sight this seems to be quite different from the purpose
% of verbatim listing, but actually the implementation of these two
% concepts turns out to be very similar.
% Both rely on the fact that the text between |\begin{...}| and
% |\end{...}| is read by \TeX{} without interpreting any commands or
% special characters.
% The remaining difference between \texttt{verbatim} and
% \texttt{comment} is only that the text is to be typeset in the
% first case and to be thrown away in the latter. Note that these
% environments cannot be nested.
%
% \DescribeMacro{\verbatiminput}
% |\verbatiminput| is a command with one argument that inputs a file
% verbatim, i.e.\ the command |verbatiminput{xx.yy}|
% has the same effect as\\[2pt]
%   \hspace*{\MacroIndent}|\begin{verbatim}|\\
%   \hspace*{\MacroIndent}\meta{Contents of the file \texttt{xx.yy}}\\
%   \hspace*{\MacroIndent}|\end{verbatim}|\\[2pt]
% This command has also a |*|-variant that prints spaces as \verb*+ +.
%
%
% \StopEventually{}
%
%
% \section{Interfaces for package writers}
%
% The \texttt{verbatim} environment of \LaTeXe{} does not
% offer a good interface to programmers.
% In contrast, this package provides a simple mechanism to
% implement similar features, the \texttt{comment} environment
% implemented here being an example of what can be done and how.
%
%
% \subsection{Simple examples}
%
% It is now possible to use the \texttt{verbatim} environment to define
% environments of your own.
% E.g.,
%\begin{verbatim}
% \newenvironment{myverbatim}%
%   {\endgraf\noindent MYVERBATIM:%
%    \endgraf\verbatim}%
%   {\endverbatim}
%\end{verbatim}
% can be used afterwards like the \texttt{verbatim} environment, i.e.
% 
% \begin{verbatim}
% \newenvironment{myverbatim}%
%   {\endgraf\noindent MYVERBATIM:%
%    \endgraf\verbatim}%
% \begin {myverbatim}
%   test
%   test
% \end {myverbatim}
% \end{verbatim}
%
% Another way to use it is to write
% \begin{verbatim}
%\let\foo=\comment
%\let\endfoo=\endcomment
%\end{verbatim}
% and from that point on environment \texttt{foo} is the same as the
% comment environment, i.e.\ everything inside its body is ignored.
%
% You may also add special commands after the |\verbatim| macro is
% invoked, e.g.
%\begin{verbatim}
%\newenvironment{myverbatim}%
%  {\verbatim\myspecialverbatimsetup}%
%  {\endverbatim}
%\end{verbatim}
% though you may want to learn about the hook |\every@verbatim| at
% this point.
% \changes{v1.5h}{1995/09/21}{Clarified documentation on use of other
%               environments to define new verbatim-type ones.}
% However, there are still a number of restrictions:
% \begin{enumerate}
%   \item
%     You must not use the |\begin| or the |\end| command inside a
%     definition, e.g.~the following two examples will \emph{not} work:
%\begin{verbatim*}
%\newenvironment{myverbatim}%
%{\endgraf\noindent MYVERBATIM:%
% \endgraf\begin{verbatim}}%
%{\end{verbatim}}
%\newenvironment{fred}
%{\begin{minipage}{30mm}\verbatim}
%{\endverbatim\end{minipage}}
%\end{verbatim*}
%     If you try these examples, \TeX{} will report a
%     ``runaway argument'' error.
%     More generally, it is not possible to use
%     |\begin|\ldots\allowbreak|\end|
%     or the related environments in the definition of the new
%     environment. Instead, the correct way to define this environment
%     would be
%    \begin{verbatim*}
%\newenvironment{fred}
%{\minipage{30mm}\verbatim}
%{\endverbatim\endminipage}
%\end{verbatim*}
%
% \begin{texexample}{}{}
%\newenvironment{fred}
%{\minipage{30mm}\verbatim}
%{\endverbatim\endminipage}
% \begin{fred}
%   some \test some \test some \test 
%   some \test some \test some \test 
% \end{fred}
% \end{texexample}
%   \item
%     You can\emph{not} use the \texttt{verbatim} environment inside
%     user defined \emph{commands}; e.g.,
% \changes{v1.4g}{1991/11/21}{Corrected wrong position of optional
%        argument to \cs{newcommand}. Discovered by Piet van Oostrum.}
%     \begin{verbatim*}
%\newcommand{\verbatimfile}[1]%
%           {\begin{verbatim}\input{#1}\end{verbatim}}
%\end{verbatim*}
%     does \emph{not} work; nor does
%     \begin{verbatim}
%\newcommand{\verbatimfile}[1]{\verbatim\input{#1}\endverbatim}
%\end{verbatim}
%   \item The name of the newly defined environment must not contain
%     characters with category code other than $11$ (letter) or
%     $12$ (other), or this will not work.
% \end{enumerate}
%
%
% \subsection{The interfaces}
%
% \DescribeMacro{\verbatim@font}{}{}
% \begin{docCommand}{verbatim@font} { \meta{void} }
% \end{docCommand}
% Let us start with the simple things.
% Sometimes it may be necessary to use a special typeface for your
% verbatim text, or perhaps the usual computer modern typewriter shape
% in a reduced size.
%
% \begin{texexample}{}{}
%   \meaning\verbatim@font\\
%   \meaning\verbatim@start@font\\
%   \meaning\verbatim@start@start@font\\
%   \meaning\verbatim@start@start@start@font\\
% \end{texexample}
%
% You may select this by redefining the macro |\verbatim@font|.
% This macro is executed at the beginning of every verbatim text to
% select the font shape.
% Do not use it for other purposes; if you find yourself abusing this
% you may want to read about the |\every@verbatim| hook below.
%
% By default, |\verbatim@font| switches to the typewriter font and
% disables the ligatures contained therein.
%
%
% \DescribeMacro{\every@verbatim}
% \DescribeMacro{\addto@hook}
% There is a hook (i.e.\ a token register) called |\every@verbatim|
% whose contents are inserted into \TeX's mouth just before every
% verbatim text.
% Please use the |\addto@hook| macro to add something to this hook.
% It is used as follows:\\[2pt]
% \hspace*{\MacroIndent}|\addto@hook|\meta{name of the hook}^^A
%  |{|\meta{commands to be added}|}|
% \vspace*{2pt}
%
%
%
% \DescribeMacro{\verbatim@start}
% After all specific setup, like switching of category codes, has been
% done, the |\verbatim@start| macro is called.
% This starts the main loop of the scanning mechanism implemented here.
% Any other environment that wants to make use of this feature should
% execute this macro as its last action.
%
% \DescribeMacro{\verbatim@startline}
% \DescribeMacro{\verbatim@addtoline}
% \DescribeMacro{\verbatim@processline}
% \DescribeMacro{\verbatim@finish}
% These are the things that concern the start of a verbatim
% environment.
% Once this (and other) setup has been done, the code in this package
% reads and processes characters from the input stream in the
% following way:
% \begin{enumerate}
%   \item Before the first character of an input line is read, it
%     executes the macro |\verbatim@startline|.
%   \item After some characters have been read, the macro
%     |\verbatim@addtoline| is called with these characters as its only
%     argument.
%     This may happen several times per line (when an |\end| command is
%     present on the line in question).
%   \item When the end of the line is reached, the macro
%     |\verbatim@processline| is called to process the characters that
%     |\verbatim@addtoline| has accumulated.
%   \item Finally, there is the macro |\verbatim@finish| that is called
%     just before the environment is ended by a call to the |\end|
%      macro.
% \end{enumerate}
%
%
% To make this clear let us consider the standard \texttt{verbatim}
% environment.
% In this case the three macros above are defined as follows:
% \begin{enumerate}
%   \item |\verbatim@startline| clears the character buffer
%     (a token register).
%   \item |\verbatim@addtoline| adds its argument to the character
%     buffer.
%   \item |\verbatim@processline| typesets the characters accumulated
%     in the buffer.
% \end{enumerate}
% With this it is very simple to implement the \texttt{comment}
% environment:
% in this case |\verbatim@startline| and |\verbatim@processline| are
% defined to be
% no-ops whereas |\verbatim@addtoline| discards its argument.
%
%
% Let's use this to define a variant of the |verbatim|
% environment that prints line numbers in the left margin.
% Assume that this would be done by a counter called |VerbatimLineNo|.
% Assuming that this counter was initialized properly by the
% environment, |\verbatim@processline| would be defined in this case as
% \begin{verbatim}
%\def\verbatim@processline{%
%  \addtocounter{VerbatimLineNo}{1}%
%  \leavevmode
%  \llap{\theVerbatimLineNo\ \hskip\@totalleftmargin}%
%  \the\verbatim@line\par}
%\end{verbatim}
%
% A further possibility is to define a variant of the |verbatim|
% environment that boxes and centers the whole verbatim text.
% Note that the boxed text should be less than a page otherwise you
% have to change this example.
%
%\begin{verbatim}
%\def\verbatimboxed#1{\begingroup
%  \def\verbatim@processline{%
%    {\setbox0=\hbox{\the\verbatim@line}%
%     \hsize=\wd0
%     \the\verbatim@line\par}}%
%  \setbox0=\vbox{\parskip=0pt\topsep=0pt\partopsep=0pt
%                 \verbatiminput{#1}}%
%  \begin{center}\fbox{\box0}\end{center}%
% \endgroup}
%\end{verbatim}
%
% As a final nontrivial example we describe the definition of an
% environment called \texttt{verbatimwrite}.
% It writes all text in its body to a file whose name is
% given as an argument.
% We assume that a stream number called |\verbatim@out| has already
% been reserved by means of the |\newwrite| macro.
%
% Let's begin with the definition of the macro |\verbatimwrite|.
% \begin{verbatim}
%\def\verbatimwrite#1{%
%\end{verbatim}
% First we call |\@bsphack| so that this environment does not influence
% the spacing.
% Then we open the file and set the category codes of all special
% characters:
% \begin{verbatim}
%  \@bsphack
%  \immediate\openout \verbatim@out #1
%  \let\do\@makeother\dospecials
%  \catcode`\^^M\active
%\end{verbatim}
% The default definitions of the macros
% \begin{verbatim}
%  \verbatim@startline
%  \verbatim@addtoline
%  \verbatim@finish
%\end{verbatim}
% are also used in this environment.
% Only the macro |\verbatim@processline| has to be changed before
% |\verbatim@start| is called:
% \begin{Verbatim}
%  \def\verbatim@processline{%
%    \immediate\write\verbatim@out{\the\verbatim@line}}%
%  \verbatim@start}
%\end{Verbatim}
% The definition of |\endverbatimwrite| is very simple:
% we close the stream and call |\@esphack| to get the spacing right.
% \begin{verbatim}
%\def\endverbatimwrite{\immediate\closeout\verbatim@out\@esphack}
%\end{verbatim}
%
% \section{The implementation}
%
% \let\DescribeMacro\docDescribeMacro
% \let\DescribeEnv\docDescribeEnv
%
% \changes{v1.4e}{1991/07/24}{Avoid reading this file twice.}
% The very first thing we do is to ensure that this file is not read
% in twice. To this end we check whether the macro |\verbatim@@@| is
% defined. If so, we just stop reading this file. The `package'
% guard here allows most of the code to be excluded when extracting
% the driver file for testing this package.
%    \begin{macrocode}
%<*package>
\NeedsTeXFormat{LaTeX2e}
\ProvidesPackage{verbatim}
     [2014/10/28 v1.5q LaTeX2e package for verbatim enhancements]
\@ifundefined{verbatim@@@}{}{\endinput}
%    \end{macrocode}
%
% We use a mechanism similar to the one implemented for the
% |\comment|\ldots\allowbreak|\endcomment| macro in \AmSTeX:
% We input one line at a time and check if it contains the |\end{...}|
% tokens.
% Then we can decide whether we have reached the end of the verbatim
% text, or must continue.
%
%
% \subsection{Preliminaries}
%
% \begin{macro}{\every@verbatim}
%    The hook (i.e.\ token register) |\every@verbatim|
%    is initialized to \meta{empty}.
%    \begin{macrocode}
\newtoks\every@verbatim
\every@verbatim={}
%    \end{macrocode}
% \end{macro}
%
%
% \begin{macro}{\@makeother}
% \changes{v1.1a}{1989/10/16}{\cs{@makeother} added.}
%    |\@makeother| takes as argument a character and changes
%    its category code to $12$ (other).
%    \begin{macrocode}
\def\@makeother#1{\catcode`#112\relax}
%    \end{macrocode}
% \end{macro}
%
%
% \begin{macro}{\@vobeyspaces}
% \changes{v1.5}{1993/10/11}{Changed definition to not use \cs{gdef}.}
% \changes{v1.1a}{1989/10/16}{\cs{@vobeyspaces} added.}
%    The macro |\@vobeyspaces| causes spaces in the input
%    to be printed as spaces in the output.
%    \begin{macrocode}
\begingroup
 \catcode`\ =\active%
 \def\x{\def\@vobeyspaces{\catcode`\ \active\let \@xobeysp}}
 \expandafter\endgroup\x
%    \end{macrocode}
% \end{macro}
%
%
% \begin{macro}{\@xobeysp}
% \changes{v1.1a}{1989/10/16}{\cs{@xobeysp} added.}
%    The macro |\@xobeysp| produces exactly one space in
%    the output, protected against breaking just before it.
%    (|\@M| is an abbreviation for the number $10000$.)
%    \begin{macrocode}
\def\@xobeysp{\leavevmode\penalty\@M\ }
%    \end{macrocode}
% \end{macro}
%
%
% \begin{macro}{\verbatim@line}
% \changes{v1.2d}{1989/11/29}{Introduced token register
%                \cs{verbatim@line}.}
%    We use a newly defined token register called |\verbatim@line|
%    that will be used as the character buffer.
%    \begin{macrocode}
\newtoks\verbatim@line
%    \end{macrocode}
% \end{macro}
%
% The following four macros are defined globally in a way suitable for
% the \texttt{verbatim} and \texttt{verbatim*} environments.
% \begin{macro}{\verbatim@startline}
% \begin{macro}{\verbatim@addtoline}
% \begin{macro}{\verbatim@processline}
%    |\verbatim@startline| initializes processing of a line
%    by emptying the character buffer (|\verbatim@line|).
%    \begin{macrocode}
\def\verbatim@startline{\verbatim@line{}}
%    \end{macrocode}
%    |\verbatim@addtoline| adds the tokens in its argument
%    to our buffer register |\verbatim@line| without expanding
%    them.
%    \begin{macrocode}
\def\verbatim@addtoline#1{%
  \verbatim@line\expandafter{\the\verbatim@line#1}}
%    \end{macrocode}
%    Processing a line inside a \texttt{verbatim} or \texttt{verbatim*}
%    environment means printing it.
% \changes{v1.2c}{1989/10/31}{Changed \cs{@@par} to \cs{par} in
%    \cs{verbatim@processline}.  Removed \cs{leavevmode} and \cs{null}
%    (i.e.\ the empty \cs{hbox}).}
%    Ending the line means that we have to begin a new paragraph.
%    We use |\par| for this purpose.  Note that |\par|
%    is redefined in |\@verbatim| to force \TeX{} into horizontal
%    mode and to insert an empty box so that empty lines in the input
%    do appear in the output.
% \changes{v1.2f}{1990/01/31}{\cs{verbatim@startline} removed.}
%    \begin{macrocode}
\def\verbatim@processline{\the\verbatim@line\par}
%    \end{macrocode}
% \end{macro}
% \end{macro}
% \end{macro}
%
% \begin{macro}{\verbatim@finish}
%    As a default, |\verbatim@finish| processes the remaining
%    characters.
%    When this macro is called we are facing the following problem:
%    when the |\end{verbatim}|
%    command is encountered |\verbatim@processline| is called
%    to process the characters preceding the command on the same
%    line.  If there are none, an empty line would be output if we
%    did not check for this case.
%
%    If the line is empty |\the\verbatim@line| expands to
%    nothing.  To test this we use a trick similar to that on p.\ 376
%    of the \TeX{}book, but with |$|\ldots|$| instead of
%    the |!| tokens.  These |$| tokens can never have the same
%    category code as a |$| token that might possibly appear in the
%    token register |\verbatim@line|, as such a token will always have
%    been read with category code $12$ (other).
%    Note that |\ifcat| expands the following tokens so that
%    |\the\verbatim@line| is replaced by the accumulated
%    characters
% \changes{v1.2d}{1989/11/29}{Changed \cs{ifx} to \cs{ifcat} test.}
% \changes{v1.1b}{1989/10/18}{Corrected bug in if test (found by CRo).}
%    \begin{macrocode}
\def\verbatim@finish{\ifcat$\the\verbatim@line$\else
  \verbatim@processline\fi}
%    \end{macrocode}
% \end{macro}
%
%
% \subsection{The \texttt{verbatim} and \texttt{verbatim*} environments}
%
% \begin{macro}{\verbatim@font}
% \changes{v1.2f}{1990/01/31}{\cs{@lquote} macro removed.}
% \changes{v1.1b}{1989/10/18}{\cs{@noligs} removed.  Code inserted
%                           directly into \cs{verbatim@font}.}
% \changes{v1.1a}{1989/10/16}{\cs{verbatim@font} added.}
% \changes{v1.1a}{1989/10/16}{\cs{@noligs} added.}
% \changes{v1.1a}{1989/10/16}{\cs{@lquote} added.}
%    We start by defining the macro |\verbatim@font| that is
%    to select the font and to set font-dependent parameters.
%    Then we expand |\@noligs| (defined in the \LaTeXe{} kernel). Among
%    possibly other things, it will go through |\verbatim@nolig@list|
%    to avoid certain ligatures.
%    |\verbatim@nolig@list| is a macro defined in the \LaTeXe{} kernel
%    to expand to
%    \begin{verbatim}
%    \do\`\do\<\do\>\do\,\do\'\do\-
%\end{verbatim}
%    All the characters in this list can be part of a ligature in some
%    font or other.
% \changes{v1.2f}{1990/01/31}{\cs{@lquote} macro removed.}
% \changes{v1.4c}{1990/10/18}{Added \cs{leavevmode}.}
% \changes{v1.4k}{1992/07/13}{Replaced Blank after $96$ by \cs{relax}.
%                           (Proposed by Dan Dill.)}
% \changes{v1.5}{1993/10/11}{Definition changed according to new code
%          in latex.tex and to avoid global definition.}
% \changes{v1.5c}{1994/02/07}{Changed to use new font switching
%                           commands.}
% \changes{v1.5m}{2000/01/07}{Disable hyphenation even if the font
%    allows it.}
% \changes{v1.5q}{2003/08/22}{Use \cs{@noligs}, as it is by now properly
%    defined in the \LaTeXe{} kernel.}
%    \begin{macrocode}
\def\verbatim@font{\normalfont\ttfamily
                   \hyphenchar\font\m@ne
                   \@noligs}
%    \end{macrocode}
% \end{macro}
%
%
% \begin{macro}{\@verbatim}
% \changes{v1.1a}{1989/10/16}{\cs{@verbatim} added.}
%    The macro |\@verbatim| sets up things properly.
%    First of all, the tokens of the |\every@verbatim| hook
%    are inserted.
%    Then a \texttt{trivlist} environment is started and its first
%    |\item| command inserted.
%    Each line of the \texttt{verbatim} or \texttt{verbatim*}
%    environment will be treated as a separate paragraph.
% \changes{v1.2e}{1990/01/15}{Added \cs{every@verbatim} hook.}
% \changes{v1.5b}{1994/01/24}{Removed optional argument of \cs{item}.}
%    \begin{macrocode}
\def\@verbatim{\the\every@verbatim
  \trivlist \item \relax
%    \end{macrocode}
% \changes{v1.5b}{1994/01/24}{Set \texttt{@inlabel} switch to false.}
% \changes{v1.5f}{1994/10/25}{Removed setting of \texttt{@inlabel}
%                             switch again.}
% \changes{v1.3c}{1990/02/26}{Removed extra vertical space.
%           Suggested by Frank Mittelbach.}
% \changes{v1.5h}{1995/09/21}{Added the space again, since it is
%           necessary for correct vertical spacing if \texttt{verbatim}
%           is nested inside \texttt{quote}.}
%    The following extra vertical space is for compatibility with the
%    \LaTeX kernel: otherwise, using the |verbatim| package changes
%    the vertical spacing of a |verbatim| environment nested within a
%    |quote| environment.
%    \begin{macrocode}
  \if@minipage\else\vskip\parskip\fi
%    \end{macrocode}
% \changes{v1.4k}{1992/07/13}{Added setting for
%        \cs{@beginparpenalty}. Suggested by Frank Mittelbach.}
%    The paragraph parameters are set appropriately:
%    the penalty at the beginning of the environment,
%    left and right margins, paragraph indentation, the glue to
%    fill the last line, and the vertical space between paragraphs.
%    The latter space has to be zero since we do not want to add
%    extra space between lines.
%    \begin{macrocode}
  \@beginparpenalty \predisplaypenalty
  \leftskip\@totalleftmargin\rightskip\z@
  \parindent\z@\parfillskip\@flushglue\parskip\z@
%    \end{macrocode}
% \changes{v1.1b}{1989/10/18}{Added resetting of \cs{parshape}
%                           if at beginning of a list.
%                           (Problem pointed out by Chris Rowley.)}
%    There's one point to make here:
%    the \texttt{list} environment uses \TeX's |\parshape|
%    primitive to get a special indentation for the first line
%    of the  list.
%    If the list begins with a \texttt{verbatim} environment
%    this |\parshape| is still in effect.
%    Therefore we have to reset this internal parameter explicitly.
%    We could do this by assigning $0$ to |\parshape|.
%    However, there is a simpler way to achieve this:
%    we simply tell \TeX{} to start a new paragraph.
%    As is explained on p.~103 of the \TeX{}book, this resets
%    |\parshape| to zero.
% \changes{v1.1c}{1989/10/19}{Replaced explicit resetting of
%                           \cs{parshape} by \cs{@@par}.}
%    \begin{macrocode}
  \@@par
%    \end{macrocode}
%    We now ensure that |\par| has the correct definition,
%    namely to force \TeX{} into horizontal mode
%    and to include an empty box.
%    This is to ensure that empty lines do appear in the output.
%    Afterwards, we insert the |\interlinepenalty| since \TeX{}
%    does not add a penalty between paragraphs (here: lines)
%    by its own initiative. Otherwise a |verbatim| environment
%    could be broken across pages even if a |\samepage|
%    declaration were present.
%
%    However, in a top-aligned minipage, this will result in an extra
%    empty line added at the top. Therefore, a slightly more
%    complicated construct is necessary.
%    One of the important things here is the inclusion of
%    |\leavevmode| as the first macro in the first line, for example,
%    a blank verbatim line is the first thing in a list item.
% \changes{v1.2c}{1989/10/31}{Definition of \cs{par} added.
%                           Ensures identical behaviour for
%                           verbatim and \cs{verbatiminput}.
%                           Problem pointed out by Chris.}
% \changes{v1.4d}{1991/04/24}{\cs{penalty}\cs{interlinepenalty} added.
%                           Necessary to avoid page breaks in
%                           the scope of a \cs{samepage} declaration.}
% \changes{v1.5b}{1994/01/24}{Improved definition of \cs{par} to work
%                           under all circumstances.}
% \changes{v1.5f}{1994/10/25}{\cs{leavevmode} added for first line.}
%    \begin{macrocode}
  \def\par{%
    \if@tempswa
      \leavevmode\null\@@par\penalty\interlinepenalty
    \else
      \@tempswatrue
      \ifhmode\@@par\penalty\interlinepenalty\fi
    \fi}%
%    \end{macrocode}
%    But to avoid an error message when the environment
%    doesn't contain any text, we redefine |\@noitemerr|
%    which will in this case be called by |\endtrivlist|.
% \changes{v1.4j}{1992/06/30}{Introduced warning instead of error
%        for empty body of verbatim text.
%        Suggested by Nelson Beebe.}
%    \begin{macrocode}
  \def\@noitemerr{\@warning{No verbatim text}}%
%    \end{macrocode}
%    Now we call |\obeylines| to make the end of line character
%    active,
%    \begin{macrocode}
  \obeylines
%    \end{macrocode}
%    change the category code of all special characters,
%    to $12$ (other).
%    \changes{v1.5i}{1996/06/04}{Moved \cs{verbatim@font} after
%           \cs{dospecials}.}
%    \begin{macrocode}
  \let\do\@makeother \dospecials
%    \end{macrocode}
%    and switch to the font to be used.
%    \begin{macrocode}
  \verbatim@font
%    \end{macrocode}
%    To avoid a breakpoint after the labels box, we remove the penalty
%    put there by the list macros: another use of |\unpenalty|!
% \changes{v1.5f}{1994/10/25}{Change to \cs{everypar} added.}
%    \begin{macrocode}
  \everypar \expandafter{\the\everypar \unpenalty}}
%    \end{macrocode}
% \end{macro}
%
%
% \begin{macro}{\verbatim}
% \begin{macro}{\verbatim*}
%    Now we define the toplevel macros.
%    |\verbatim| is slightly changed:
%    after setting up things properly it calls
%    |\verbatim@start|.
% \changes{v1.5l}{1999/12/14}{Added \cs{begingroup} for cases where
%    \cs{verbatim} is used directly, rather than in \cs{begin}: see
%    pr/3115.}
%    This is done inside a group, so that |\verbatim| can be used
%    directly, without |\begin|.
%    \begin{macrocode}
\def\verbatim{\begingroup\@verbatim \frenchspacing\@vobeyspaces
              \verbatim@start}
%    \end{macrocode}
%    |\verbatim*| is defined accordingly.
%    \begin{macrocode}
\@namedef{verbatim*}{\begingroup\@verbatim\verbatim@start}
%    \end{macrocode}
% \end{macro}
% \end{macro}
%
% \begin{macro}{\endverbatim}
% \begin{macro}{\endverbatim*}
%    To end the \texttt{verbatim} and \texttt{verbatim*}
%    environments it is only necessary to finish the
%    \texttt{trivlist} environment started in |\@verbatim| and
%    close the corresponding group.
% \changes{v1.5l}{1999/12/14}{Added \cs{endgroup} for cases where
%    \cs{endverbatim} is used directly, rather than in \cs{end}: see
%    pr/3115.}
% \changes{v1.5n}{2000/08/03}{Added \cs{@endpetrue}: needed when
%    faking such a \cs{end} (pr/3234).}
% \changes{v1.5o}{2000/08/23}{Changed \cs{@endpetrue} to \cs{@doendpe}:
%    see (pr/3234).}
%    \begin{macrocode}
\def\endverbatim{\endtrivlist\endgroup\@doendpe}
\expandafter\let\csname endverbatim*\endcsname =\endverbatim
%    \end{macrocode}
% \end{macro}
% \end{macro}
%
%
% \subsection{The \texttt{comment} environment}
%
% \begin{macro}{\comment}
% \begin{macro}{\endcomment}
% \changes{v1.1c}{1989/10/19}{Added \cs{@bsphack}/\cs{@esphack} to the
%            \texttt{comment} environment.  Suggested by Chris Rowley.}
%    The |\comment| macro is similar to |\verbatim*|.
%    However, we do not need to switch fonts or set special
%    formatting parameters such as |\parindent| or |\parskip|.
%    We need only set the category code of all special characters
%    to $12$ (other) and that of |^^M| (the end of line character)
%    to $13$ (active).
%    The latter is needed for macro parameter delimiter matching in
%    the internal macros defined below.
%    In contrast to the default definitions used by the
%    |\verbatim| and |\verbatim*| macros,
%    we define |\verbatim@addtoline| to throw away its argument
%    and |\verbatim@processline|, |\verbatim@startline|,
%    and |\verbatim@finish| to act as no-ops.
%    Then we call |\verbatim@|.
%    But the first thing we do is to call |\@bsphack| so that
%    this environment has no influence whatsoever upon the spacing.
% \changes{v1.1c}{1989/10/19}{Changed \cs{verbatim@start} to
%                           \cs{verbatim@}.  Suggested by Chris Rowley.}
% \changes{v1.1c}{1989/10/19}{\cs{verbatim@startline} and
%                           \cs{verbatim@finish} are now
%                           also redefined to do nothing.}
%    \begin{macrocode}
\def\comment{\@bsphack
             \let\do\@makeother\dospecials\catcode`\^^M\active
             \let\verbatim@startline\relax
             \let\verbatim@addtoline\@gobble
             \let\verbatim@processline\relax
             \let\verbatim@finish\relax
             \verbatim@}
%    \end{macrocode}
%    |\endcomment| is very simple: it only calls
%    |\@esphack| to take care of the spacing.
%    The |\end| macro closes the group and therefore takes care
%    of restoring everything we changed.
%    \begin{macrocode}
\let\endcomment=\@esphack
%    \end{macrocode}
% \end{macro}
% \end{macro}
%
%
%
% \subsection{The main loop}
%
% Here comes the tricky part:
% During the definition of the macros we need to use the special
% characters |\|, |{|, and |}| not only with their
% normal category codes,
% but also with category code $12$ (other).
% We achieve this by the following trick:
% first we tell \TeX{} that |\|, |{|, and |}|
% are the lowercase versions of |!|, |[|, and |]|.
% Then we replace every occurrence of |\|, |{|, and |}|
% that should be read with category code $12$ by |!|, |[|,
% and |]|, respectively,
% and give the whole list of tokens to |\lowercase|,
% knowing that category codes are not altered by this primitive!
%
% But first we have ensure that
% |!|, |[|, and |]| themselves have
% the correct category code!
% \changes{v1.3b}{1990/02/07}{Introduced \cs{vrb@catcodes} instead
%                  of explicit setting of category codes.}
% To allow special settings of these codes we hide their setting in
% the macro |\vrb@catcodes|.  If it is already defined our new
% definition is skipped.
%    \begin{macrocode}
\@ifundefined{vrb@catcodes}%
  {\def\vrb@catcodes{%
     \catcode`\!12\catcode`\[12\catcode`\]12}}{}
%    \end{macrocode}
% This trick allows us to use this code for applications where other
% category codes are in effect.
%
% We start a group to keep the category code changes local.
%    \begin{macrocode}
\begingroup
 \vrb@catcodes
 \lccode`\!=`\\ \lccode`\[=`\{ \lccode`\]=`\}
%    \end{macrocode}
% \changes{v1.2f}{1990/01/31}{Code for TABs removed.}
%    We also need the end-of-line character |^^M|,
%    as an active character.
%    If we were to simply write |\catcode`\^^M=\active|
%    then we would get an unwanted active end of line character
%    at the end of every line of the following macro definitions.
%    Therefore we use the same trick as above:
%    we write a tilde |~| instead of |^^M| and
%    pretend that the
%    latter is the lowercase variant of the former.
%    Thus we have to ensure now that the tilde character has
%    category code $13$ (active).
%    \begin{macrocode}
 \catcode`\~=\active \lccode`\~=`\^^M
%    \end{macrocode}
%    The use of the |\lowercase| primitive leads to one problem:
%    the uppercase character `|C|' needs to be used in the
%    code below and its case must be preserved.
%    So we add the command:
%    \begin{macrocode}
 \lccode`\C=`\C
%    \end{macrocode}
%    Now we start the token list passed to |\lowercase|.
%    We use the following little trick (proposed by Bernd Raichle):
%    The very first token in the token list we give to |\lowercase| is
%    the |\endgroup| primitive. This means that it is processed by
%    \TeX{} immediately after |\lowercase| has finished its operation,
%    thus ending the group started by |\begingroup| above. This avoids
%    the global definition of all macros.
%    \begin{macrocode}
 \lowercase{\endgroup
%    \end{macrocode}
% \begin{macro}{\verbatim@start}
%    The purpose of |\verbatim@start| is to check whether there
%    are any characters on the same line as the |\begin{verbatim}|
%    and to pretend that they were on a line by themselves.
%    On the other hand, if there are no characters remaining
%    on the current line we shall just find an end of line character.
%    |\verbatim@start| performs its task by first grabbing the
%    following character (its argument).
%    This argument is then compared to an active |^^M|,
%    the end of line character.
%    \begin{macrocode}
    \def\verbatim@start#1{%
      \verbatim@startline
      \if\noexpand#1\noexpand~%
%    \end{macrocode}
%    If this is true we transfer control to |\verbatim@|
%    to process the next line.  We use
%    |\next| as the macro which will continue the work.
%    \begin{macrocode}
        \let\next\verbatim@
%    \end{macrocode}
%    Otherwise, we define |\next| to expand to a call
%    to |\verbatim@| followed by the character just
%    read so that it is reinserted into the text.
%    This means that those characters remaining on this line
%    are handled as if they formed a line by themselves.
%    \begin{macrocode}
      \else \def\next{\verbatim@#1}\fi
%    \end{macrocode}
%    Finally we call |\next|.
%    \begin{macrocode}
      \next}%
%    \end{macrocode}
% \end{macro}
%
% \begin{macro}{\verbatim@}
%    The three macros |\verbatim@|, |\verbatim@@|,
%    and |\verbatim@@@| form the ``main loop'' of the
%    \texttt{verbatim} environment.
%    The purpose of |\verbatim@| is to read exactly one line
%    of input.
%    |\verbatim@@| and |\verbatim@@@| work together to
%    find out whether the four characters
%    |\end| (all with category code $12$ (other)) occur in that
%    line.
%    If so, |\verbatim@@@| will call
%    |\verbatim@test| to check whether this |\end| is
%    part of |\end{verbatim}| and will terminate the environment
%    if this is the case.
%    Otherwise we continue as if nothing had happened.
%    So let's have a look at the definition of |\verbatim@|:
% \changes{v1.1a}{1989/10/16}{Replaced \cs{verbatim@@@} by \cs{@nil}.}
%    \begin{macrocode}
    \def\verbatim@#1~{\verbatim@@#1!end\@nil}%
%    \end{macrocode}
%    Note that the |!| character will have been replaced by a
%    |\| with category code $12$ (other) by the |\lowercase|
%    primitive governing this code before the definition of this
%    macro actually takes place.
%    That means that
%    it takes the line, puts |\end| (four character tokens)
%    and |\@nil| (one control sequence token) as a
%    delimiter behind it, and
%    then calls |\verbatim@@|.
% \end{macro}
%
% \begin{macro}{\verbatim@@}
%    |\verbatim@@| takes everything up to the next occurrence of
%    the four characters |\end| as its argument.
%    \begin{macrocode}
    \def\verbatim@@#1!end{%
%    \end{macrocode}
%    That means: if they do not occur in the original line, then
%    argument |#1| is the
%    whole input line, and |\@nil| is the next token
%    to be processed.
%    However, if the four characters |\end| are part of the
%    original line, then
%    |#1| consists of the characters in front of |\end|,
%    and the next token is the following character (always remember
%    that the line was lengthened by five tokens).
%    Whatever |#1| may be, it is verbatim text,
%    so |#1| is added to the line currently built.
%    \begin{macrocode}
       \verbatim@addtoline{#1}%
%    \end{macrocode}
%    The next token in the input stream
%    is of special interest to us.
%    Therefore |\futurelet| defines |\next| to be equal
%    to it before calling |\verbatim@@@|.
%    \begin{macrocode}
       \futurelet\next\verbatim@@@}%
%    \end{macrocode}
% \end{macro}
%
% \begin{macro}{\verbatim@@@}
% \changes{v1.1a}{1989/10/16}{Replaced \cs{verbatim@@@} by
%                           \cs{@nil} where used as delimiter.}
%    |\verbatim@@@| will now read the rest of the tokens on
%    the current line,
%    up to the final |\@nil| token.
%    \begin{macrocode}
    \def\verbatim@@@#1\@nil{%
%    \end{macrocode}
%    If the first of the above two cases occurred, i.e.\ no
%    |\end| characters were on that line, |#1| is empty
%    and |\next| is equal to |\@nil|.
%    This is easily checked.
%    \begin{macrocode}
       \ifx\next\@nil
%    \end{macrocode}
%    If so, this was a simple line.
%    We finish it by processing the line we accumulated so far.
%    Then we prepare to read the next line.
% \changes{v1.2f}{1990/01/31}{Added \cs{verbatim@startline}.}
%    \begin{macrocode}
         \verbatim@processline
         \verbatim@startline
         \let\next\verbatim@
%    \end{macrocode}
%    Otherwise we have to check what follows these |\end|
%    tokens.
%    \begin{macrocode}
       \else
%    \end{macrocode}
%    Before we continue, it's a good idea to stop for a moment
%    and remember where we are:
%    We have just read the four character tokens |\end|
%    and must now check whether the name of the environment (surrounded
%    by braces) follows.
%    To this end we define a macro called |\@tempa|
%    that reads exactly one character and decides what to do next.
%    This macro should do the following: skip spaces until
%    it encounters either a left brace or the end of the line.
%    But it is important to remember which characters are skipped.
%    The |\end|\meta{optional spaces}|{| characters
%    may be part of the verbatim text, i.e.\ these characters
%    must be printed.
%
%    Assume for example that the current line contains
%    \begin{verbatim*}
%      \end {AVeryLongEnvironmentName}
%\end{verbatim*}
%    As we shall soon see, the scanning mechanism implemented here
%    will not find out that this is text to be printed until
%    it has read the right brace.
%    Therefore we need a way to accumulate the characters read
%    so that we can reinsert them if necessary.
%    The token register |\@temptokena| is used for this purpose.
%
%    Before we do this we have to get rid of the superfluous
%    |\end| tokens at the end of the line.
% \changes{v1.4j}{1992/06/30}{Removed use of \cs{toks@}. Suggested by
%           Bernd Raichle.}
%    To this end we define a temporary macro whose argument
%    is delimited by |\end\@nil| (four character tokens
%    and one control sequence token) to be used below
%    on the rest of the line, after appending a |\@nil| token to it.
%    (Note that this token can never appear in |#1|.)
%    We use the following definition of
%    |\@tempa| to get the rest of the line (after the first
%    |\end|).
%    \begin{macrocode}
         \def\@tempa##1!end\@nil{##1}%
%    \end{macrocode}
%    We mentioned already that we use token register
%    |\@temptokena|
%    to remember the characters we skip, in case we need them again.
%    We initialize this with the |\end| we have thrown away
%    in the call to |\@tempa|.
%    \begin{macrocode}
         \@temptokena{!end}%
%    \end{macrocode}
%    We shall now call |\verbatim@test|
%    to process the characters
%    remaining on the current line.
%    But wait a moment: we cannot simply call this macro
%    since we have already read the whole line.
%    Therefore we have to first expand the macro |\@tempa| to insert
%    them again after the |\verbatim@test| token.
%    A |^^M| character is appended to denote the end of the line.
%    (Remember that this character comes disguised as a tilde.)
% \changes{v1.2}{1989/10/20}{Taken local definition of \cs{@tempa} out
%                          of \cs{verbatim@@@} and introduced
%                          \cs{verbatim@test} instead.}
%    \begin{macrocode}
         \def\next{\expandafter\verbatim@test\@tempa#1\@nil~}%
%    \end{macrocode}
%    That's almost all, but we still have to
%    now call |\next| to do the work.
%    \begin{macrocode}
       \fi \next}%
%    \end{macrocode}
% \end{macro}
%
%
% \begin{macro}{\verbatim@test}
% \changes{v1.2}{1989/10/20}{Introduced \cs{verbatim@test}.}
%    We define |\verbatim@test| to investigate every token
%    in turn.
%    \begin{macrocode}
    \def\verbatim@test#1{%
%    \end{macrocode}
%    First of all we set |\next| equal to |\verbatim@test|
%    in case this macro must call itself recursively in order to
%    skip spaces.
%    \begin{macrocode}
           \let\next\verbatim@test
%    \end{macrocode}
%    We have to distinguish four cases:
%    \begin{enumerate}
%      \item The next token is a |^^M|, i.e.\ we reached
%            the end of the line.  That means that nothing
%            special was found.
%            Note that we use |\if| for the following
%            comparisons so that the category code of the
%            characters is irrelevant.
%    \begin{macrocode}
           \if\noexpand#1\noexpand~%
%    \end{macrocode}
%            We add the characters accumulated in token register
%            |\@temptokena| to the current line.  Since
%            |\verbatim@addtoline| does not expand its argument,
%            we have to do the expansion at this point.  Then we
%            |\let| |\next| equal to |\verbatim@|
%            to prepare to read the next line.
% \changes{v1.2f}{1990/01/31}{Added \cs{verbatim@startline}.}
%    \begin{macrocode}
             \expandafter\verbatim@addtoline
               \expandafter{\the\@temptokena}%
             \verbatim@processline
             \verbatim@startline
             \let\next\verbatim@
%    \end{macrocode}
%      \item A space character follows.
%            This is allowed, so we add it to |\@temptokena|
%            and continue.
%    \begin{macrocode}
           \else \if\noexpand#1
             \@temptokena\expandafter{\the\@temptokena#1}%
%    \end{macrocode}
% \changes{v1.2f}{1990/01/31}{Code for TABs removed.}
%      \item An open brace follows.
%            This is the most interesting case.
%            We must now collect characters until we read the closing
%            brace and check whether they form the environment name.
%            This will be done by |\verbatim@testend|, so here
%            we let |\next| equal this macro.
%            Again we will process the rest of the line, character
%            by character.
% \changes{v1.2}{1989/10/20}{Moved the initialization of
%                          \cs{@tempc} from \cs{verbatim@testend} into
%                          \cs{verbatim@test}.}
%            The characters forming the name of the environment will
%            be accumulated in |\@tempc|.
%            We initialize this macro to expand to nothing.
% \changes{v1.3b}{1990/02/07}{\cs{noexpand} added.}
%    \begin{macrocode}
           \else \if\noexpand#1\noexpand[%
             \let\@tempc\@empty
             \let\next\verbatim@testend
%    \end{macrocode}
%            Note that the |[| character will be a |{| when
%            this macro is defined.
%      \item Any other character means that the |\end| was part
%            of the verbatim text.
%            Add the characters to the current line and prepare to call
%            |\verbatim@| to process the rest of the line.
%  \changes{v1.0f}{1989/10/09}{Fixed \cs{end} \cs{end} bug
%                            found by Chris Rowley}
%    \begin{macrocode}
           \else
             \expandafter\verbatim@addtoline
               \expandafter{\the\@temptokena}%
             \def\next{\verbatim@#1}%
           \fi\fi\fi
%    \end{macrocode}
%    \end{enumerate}
%    The last thing this macro does is to call |\next|
%    to continue processing.
%    \begin{macrocode}
           \next}%
%    \end{macrocode}
% \end{macro}
%
% \begin{macro}{\verbatim@testend}
%    |\verbatim@testend| is called when
%    |\end|\meta{optional spaces}|{| was seen.
%    Its task is to scan everything up to the next |}|
%    and to call |\verbatim@@testend|.
%    If no |}| is found it must reinsert the characters it read
%    and return to |\verbatim@|.
%    The following definition is similar to that of
%    |\verbatim@test|:
%    it takes the next character and decides what to do.
% \changes{v1.2}{1989/10/20}{Removed local definition of \cs{@tempa}
%                          from \cs{verbatim@testend} which now
%                          does the work itself.}
%    \begin{macrocode}
    \def\verbatim@testend#1{%
%    \end{macrocode}
%    Again, we have four cases:
%    \begin{enumerate}
%      \item |^^M|: As no |}| is found in the current line,
%            add the characters to the buffer.  To avoid a
%            complicated construction for expanding
%            |\@temptokena|
%            and |\@tempc| we do it in two steps.  Then we
%            continue with |\verbatim@| to process the
%            next line.
% \changes{v1.2f}{1990/01/31}{Added \cs{verbatim@startline}.}
%    \begin{macrocode}
         \if\noexpand#1\noexpand~%
           \expandafter\verbatim@addtoline
             \expandafter{\the\@temptokena[}%
           \expandafter\verbatim@addtoline
             \expandafter{\@tempc}%
           \verbatim@processline
           \verbatim@startline
           \let\next\verbatim@
%    \end{macrocode}
%      \item |}|: Call |\verbatim@@testend| to check
%            if this is the right environment name.
% \changes{v1.3b}{1990/02/07}{\cs{noexpand} added.}
%    \begin{macrocode}
         \else\if\noexpand#1\noexpand]%
           \let\next\verbatim@@testend
%    \end{macrocode}
%  \changes{v1.0f}{1989/10/09}{Introduced check for
%              {\tt\string\verb!\string|!\string\!\string|} to fix
%              single brace bug found by Chris Rowley}
%      \item |\|: This character must not occur in the name of
%            an environment.  Thus we stop collecting characters.
%            In principle, the same argument would apply to other
%            characters as well, e.g., |{|.
%            However, |\| is a special case, since it may be
%            the first character of |\end|.  This means that
%            we have to look again for
%            |\end{|\meta{environment name}|}|.
%            Note that we prefixed the |!| by a |\noexpand|
%            primitive, to protect ourselves against it being an
%            active character.
% \changes{v1.3b}{1990/02/07}{\cs{noexpand} added.}
%    \begin{macrocode}
         \else\if\noexpand#1\noexpand!%
           \expandafter\verbatim@addtoline
             \expandafter{\the\@temptokena[}%
           \expandafter\verbatim@addtoline
             \expandafter{\@tempc}%
           \def\next{\verbatim@!}%
%    \end{macrocode}
%      \item Any other character: collect it and continue.
%            We cannot use |\edef| to define |\@tempc|
%            since its replacement text might contain active
%            character tokens.
%    \begin{macrocode}
         \else \expandafter\def\expandafter\@tempc\expandafter
           {\@tempc#1}\fi\fi\fi
%    \end{macrocode}
%    \end{enumerate}
%    As before, the macro ends by calling itself, to
%    process the next character if appropriate.
%    \begin{macrocode}
         \next}%
%    \end{macrocode}
% \end{macro}
%
% \begin{macro}{\verbatim@@testend}
%    Unlike the previous macros |\verbatim@@testend| is simple:
%    it has only to check if the |\end{|\ldots|}|
%    matches the corresponding |\begin{|\ldots|}|.
%    \begin{macrocode}
    \def\verbatim@@testend{%
%    \end{macrocode}
%    We use |\next| again to define the things that are
%    to be done.
%    Remember that the name of the current environment is
%    held in |\@currenvir|, the characters accumulated
%    by |\verbatim@testend| are in |\@tempc|.
%    So we simply compare these and prepare to execute
%    |\end{|\meta{current environment}|}|
%    macro if they match.
%    Before we do this we call |\verbatim@finish| to process
%    the last line.
%    We define |\next| via |\edef| so that
%    |\@currenvir| is replaced by its expansion.
%    Therefore we need |\noexpand| to inhibit the expansion
%    of |\end| at this point.
%    \begin{macrocode}
       \ifx\@tempc\@currenvir
         \verbatim@finish
         \edef\next{\noexpand\end{\@currenvir}%
%    \end{macrocode}
%    Without this trick the |\end| command would not be able
%    to correctly check whether its argument matches the name of
%    the current environment and you'd get an
%    interesting \LaTeX{} error message such as:
%    \begin{verbatim}
%! \begin{verbatim*} ended by \end{verbatim*}.
%\end{verbatim}
%    But what do we do with the rest of the characters, those
%    that remain on that line?
%    We call |\verbatim@rescan| to take care of that.
%    Its first argument is the name of the environment just
%    ended, in case we need it again.
%    |\verbatim@rescan| takes the list of characters to be
%    reprocessed as its second argument.
%    (This token list was inserted after the current macro
%    by |\verbatim@@@|.)
%    Since we are still in an |\edef| we protect it
%    by means of|\noexpand|.
%    \begin{macrocode}
                    \noexpand\verbatim@rescan{\@currenvir}}%
%    \end{macrocode}
%    If the names do not match, we reinsert everything read up
%    to now and prepare to call |\verbatim@| to process
%    the rest of the line.
%    \begin{macrocode}
       \else
         \expandafter\verbatim@addtoline
           \expandafter{\the\@temptokena[}%
           \expandafter\verbatim@addtoline
             \expandafter{\@tempc]}%
         \let\next\verbatim@
       \fi
%    \end{macrocode}
%    Finally we call |\next|.
%    \begin{macrocode}
       \next}%
%    \end{macrocode}
% \end{macro}
%
% \begin{macro}{\verbatim@rescan}
%    In principle |\verbatim@rescan| could be used to
%    analyse the characters remaining after the |\end{...}|
%    command and pretend that these were read
%    ``properly'', assuming ``standard'' category codes are in
%    force.\footnote{Remember that they were all read with
%          category codes $11$ (letter) and $12$ (other) so
%          that control sequences are not recognized as such.}
%    But this is not always possible (when there are unmatched
%    curly braces in the rest of the line).
%    Besides, we think that this is not worth the effort:
%    After a \texttt{verbatim} or \texttt{verbatim*} environment
%    a new line in the output is begun anyway,
%    and an |\end{comment}| can easily be put on a line by itself.
%    So there is no reason why there should be any text here.
%    For the benefit of the user who did put something there
%    (a comment, perhaps)
%    we simply issue a warning and drop them.
%    The method of testing is explained in Appendix~D, p.\ 376 of
%    the \TeX{}book. We use |^^M| instead of the |!|
%    character used there
%    since this is a character that cannot appear in |#1|.
%    The two |\noexpand| primitives are necessary to avoid
%    expansion of active characters and macros.
%
%    One extra subtlety should be noted here: remember that
%    the token list we are currently building will first be
%    processed by the |\lowercase| primitive before \TeX{}
%    carries out the definitions.
%    This means that the `|C|' character in the
%    argument to the |\@warning| macro must be protected against
%    being changed to `|c|'.  That's the reason why we added the
%    |\lccode`\C=`\C| assignment above.
%    We can now finish the argument to |\lowercase| as well as the
%    group in which the category codes were changed.
%    \begin{macrocode}
    \def\verbatim@rescan#1#2~{\if\noexpand~\noexpand#2~\else
        \@warning{Characters dropped after `\string\end{#1}'}\fi}}
%    \end{macrocode}
% \end{macro}
%
% \subsection{The \cs{verbatiminput} command}
%
% \begin{macro}{\verbatim@in@stream}
%    We begin by allocating an input stream (out of the 16 available
%    input streams).
%\iffalse
%  Vorstellbar ist auch der Aufruf von |`verbatiminput| innerhalb eines
%  |`verbatiminput| (z.B: wenn man |`input|-Anweisungen im zu lesenden
%  File hat und auch diese Files automatisch lesen will).  Dies kann
%  man jedoch nur ermoeglichen, wenn man einen besseren Mechanismus
%  verwendet als es das simple, statische |`newread| darstellt.
%  Vorstellbar fuer eine neuere \LaTeX-Version ist eine (lokale)
%  Allokation des Streams durch ein |`open| und eine Freigabe des
%  Streams durch |`close| oder Verlassen der Gruppe.
%\fi
%    \begin{macrocode}
\newread\verbatim@in@stream
%    \end{macrocode}
% \end{macro}
%
% \begin{macro}{\verbatim@readfile}
%    The macro |\verbatim@readfile| encloses the main loop by calls to
%    the macros |\verbatim@startline| and |\verbatim@finish|,
%    respectively.  This makes sure
%    that the user can initialize and finish the command when the file
%    is empty or doesn't exist.  The \texttt{verbatim} environment has a
%    similar behaviour when called with an empty text.
%    \begin{macrocode}
\def\verbatim@readfile#1{%
  \verbatim@startline
%    \end{macrocode}
%    When the file is not found we issue a warning.
%    \begin{macrocode}
  \openin\verbatim@in@stream #1\relax
  \ifeof\verbatim@in@stream
    \typeout{No file #1.}%
  \else
%    \end{macrocode}
%    At this point we pass the name of the file to |\@addtofilelist|
%    so that its appears appears in the output of a |\listfiles|
%    command.
% \changes{v1.5j}{1996/09/25}{Add \cs{@addtofilelist} and
%    \cs{ProvidesFile} so that the name of the file
%    read in appears in the \cs{listfiles} output (Omission pointed
%    out by Patrick W.~Daly).}
%    In addition, we use |\ProvidesFile| to make a log entry in the
%    transcript file and to distinguish files read in via
%    |\verbatiminput| from others.
%    \begin{macrocode}
    \@addtofilelist{#1}%
    \ProvidesFile{#1}[(verbatim)]%
%    \end{macrocode}
%    While reading from the file it is useful to switch off the
%    recognition of the end-of-line character.  This saves us stripping
%    off spaces from the contents of the line.
%    \begin{macrocode}
    \expandafter\endlinechar\expandafter\m@ne
    \expandafter\verbatim@read@file
    \expandafter\endlinechar\the\endlinechar\relax
    \closein\verbatim@in@stream
  \fi
  \verbatim@finish
}
%    \end{macrocode}
% \end{macro}
%
% \begin{macro}{\verbatim@read@file}
%    All the work is done in |\verbatim@read@file|.  It reads the input
%    file line by line and recursively calls itself until the end of
%    the file.
%    \begin{macrocode}
\def\verbatim@read@file{%
  \read\verbatim@in@stream to\next
  \ifeof\verbatim@in@stream
  \else
%    \end{macrocode}
%    For each line we call |\verbatim@addtoline| with the contents of
%    the line. \hskip0pt plus 3cm\penalty0\hskip0pt plus -3cm
%    |\verbatim@processline| is called next.
%    \begin{macrocode}
    \expandafter\verbatim@addtoline\expandafter{\next}%
    \verbatim@processline
%    \end{macrocode}
%    After processing the line we call |\verbatim@startline| to
%    initialize all before we read the next line.
%    \begin{macrocode}
    \verbatim@startline
%    \end{macrocode}
%    Without |\expandafter| each call of |\verbatim@read@file| uses
%    space in \TeX's input stack.\footnote{A standard \TeX\ would
%    report an overflow error if you try to read a file with more than
%    ca.\ 200~lines.  The same error occurs if the first line of code
%    in \S 390 of \textsl{``TeX: The Program''\/} is missing.}
%    \begin{macrocode}
    \expandafter\verbatim@read@file
  \fi
}
%    \end{macrocode}
% \end{macro}
%
%
% \begin{macro}{\verbatiminput}
%    |\verbatiminput| first starts a group to keep font and category
%    changes local.
%    Then it calls the macro |\verbatim@input| with additional
%    arguments, depending on whether an asterisk follows.
%    \begin{macrocode}
\def\verbatiminput{\begingroup
  \@ifstar{\verbatim@input\relax}%
          {\verbatim@input{\frenchspacing\@vobeyspaces}}}
%    \end{macrocode}
% \end{macro}
%
% \begin{macro}{\verbatim@input}
% \changes{1.5k}{1997/04/30}{Have \cs{verbatim@input} check for
%    existence of file.}
%    |\verbatim@input| first checks whether the file exists, using
%    the standard macro |\IfFileExists| which leaves the name of the
%    file found in |\@filef@und|.
%    Then everything is set up as in the |\verbatim| macro.
%    \begin{macrocode}
\def\verbatim@input#1#2{%
   \IfFileExists {#2}{\@verbatim #1\relax
%    \end{macrocode}
%    Then it reads in the file, finishes off the \texttt{trivlist}
%    environment started by |\@verbatim| and closes the group.
%    This restores everything to its normal settings.
%    \begin{macrocode}
    \verbatim@readfile{\@filef@und}\endtrivlist\endgroup\@doendpe}%
%    \end{macrocode}
%   If the file is not found a more or less helpful message is
%    printed. The final |\endgroup| is  needed to close the group
%    started in |\verbatiminput| above.
%    \begin{macrocode}
   {\typeout {No file #2.}\endgroup}}
%</package>
%    \end{macrocode}
% \end{macro}
%
%
% \subsection{Getting verbatim text into arguments.}
%
% One way of achieving this is to define a macro (command) whose
% expansion is the required verbatim text.  This command can then be
% used anywhere that the verbatim text is required.  It can be used in
% arguments, even moving ones, but it is fragile (at least, the
% version here is).
%
% Here is some code which claims to provide this.  It is a much revised
% version of something I (Chris) did about 2 years ago.  Maybe it needs
% further revision.
%
% It is only intended as an extension to |\verb|, not to the
% \texttt{verbatim} environment.  It should therefore, perhaps, treat
% line-ends similarly to whatever is best for |\verb|.
%
% \begin{macro}{\newverbtext}
% This is the command to produce a new macro whose expansion is
% verbatim text.  This command itself cannot be used in arguments,
% of course! It is used as follows:
%
% \begin{verbatim}
%    \newverbtext{\myverb}"^%{ &~_\}}@ #"
% \end{verbatim}
%
% The rules for delimiting the verbatim text are the same as those for
% |\verb|.
%
%    \begin{macrocode}
%<*verbtext>
\def \newverbtext {%
  \@ifstar {\@tempswatrue \@verbtext }{\@tempswafalse \@verbtext *}%
}
%    \end{macrocode}
% \end{macro}
%    I think that a temporary switch is safe here: if not, then
%    suitable |\let|s can be used.
%    \changes{v1.5i}{1996/06/04}{Moved processing of
%              \cs{verbatim@nolig@list} after \cs{dospecials}.}
%    \changes{v1.5p}{2001/03/12}{Added missing right brace in
%               definition of \cs{@verbtext} (PR 3314).}
%    \begin{macrocode}
\def \@verbtext *#1#2{%
   \begingroup
     \let\do\@makeother \dospecials
     \let\do\do@noligs \verbatim@nolig@list
     \@vobeyspaces
     \catcode`#2\active
     \catcode`~\active
     \lccode`\~`#2%
     \lowercase
%    \end{macrocode}
%    We use a temporary macro here and a trick so that the definition of
%    the command itself can be done inside the group and be a local
%    definition (there may be better ways to achieve this).
%    \begin{macrocode}
     {\def \@tempa ##1~%
           {\whitespaces
%    \end{macrocode}
%    If these |\noexpand|s were |\noexpand\protect\noexpand|, would
%    this make these things robust?
%    \begin{macrocode}
            \edef #1{\noexpand \@verbtextmcheck
                     \bgroup
                     \if@tempswa
                       \noexpand \visiblespaces
                     \fi
                     \noexpand \@verbtextsetup
                     ##1%
                     \egroup}%
            }%
      \expandafter\endgroup\@tempa
     }
}
%    \end{macrocode}
%    This sets up the correct type of group for the mode: it must not
%    be expanded at define time!
%    \begin{macrocode}
\def \@verbtextmcheck {%
   \relax\ifmmode
           \hbox
         \else
           \leavevmode
           \null
         \fi
}
%    \end{macrocode}
%    This contains other things which should not be expanded during the
%    definition.
%    \begin{macrocode}
\def \@verbtextsetup {%
   \frenchspacing
   \verbatim@font
   \verbtextstyle
}
%    \end{macrocode}
%    The command |\verbtextstyle| is a document-level hook which can be
%    used to override the predefined typographic treatment of commands
%    defined with |\newverbtext| commands.
%
%    |\visiblespaces| and |\whitespaces| are examples of possible values
%    of this hook.
%    \begin{macrocode}
\let \verbtextstyle \relax
\def \visiblespaces {\chardef \  32\relax}
\def \whitespaces {\let \ \@@space}
\let \@@space \ %
%</verbtext>
%    \end{macrocode}
%
%
% \section{Testing the implementation}
%
% For testing the implementation and for demonstration we provide
% an extra file. It can be extracted by using the conditional
% `\textsf{testdriver}'. It uses a small input file called
% `\texttt{verbtest.tst}' that is distributed separately.
% Again, we use individual `+' guards.
%
%    \begin{macrocode}
%<*testdriver>
\documentclass{article}

\usepackage{verbatim}

\newenvironment{myverbatim}%
   {\endgraf\noindent MYVERBATIM:\endgraf\verbatim}%
   {\endverbatim}

\makeatletter

\newenvironment{verbatimlisting}[1]%
 {\def\verbatim@startline{\input{#1}%
    \def\verbatim@startline{\verbatim@line{}}%
    \verbatim@startline}%
  \verbatim}{\endverbatim}

\newwrite\verbatim@out

\newenvironment{verbatimwrite}[1]%
 {\@bsphack
  \immediate\openout \verbatim@out #1
  \let\do\@makeother\dospecials\catcode`\^^M\active
  \def\verbatim@processline{%
    \immediate\write\verbatim@out{\the\verbatim@line}}%
  \verbatim@start}%
 {\immediate\closeout\verbatim@out\@esphack}

\makeatother

\addtolength{\textwidth}{30pt}

\begin{document}

\typeout{}
\typeout{===> Expect ``characters dropped''
         warning messages in this test! <====}
\typeout{}

Text Text Text Text Text Text Text Text Text Text Text
Text Text Text Text Text Text Text Text Text Text Text
Text Text Text Text Text Text Text Text Text Text Text
  \begin{verbatim}
    test
    \end{verbatim*}
    test
    \end{verbatim
    test of ligatures: <`!`?`>
    \endverbatim
    test
    \end  verbatim
    test
    test of end of line:
    \end
    {verbatim}
  \end{verbatim} Further text to be typeset: $\alpha$.
Text Text Text Text Text Text Text Text Text Text Text
Text Text Text Text Text Text Text Text Text Text Text
Text Text Text Text Text Text Text Text Text Text Text
  \begin{verbatim*}
    test
    test
  \end {verbatim*}
Text Text Text Text Text Text Text Text Text Text Text
Text Text Text Text Text Text Text Text Text Text Text
Text Text Text Text Text Text Text Text Text Text Text
  \begin{verbatim*}  bla bla
    test
    test
  \end {verbatim*}
Text Text Text Text Text Text Text Text Text Text Text
Text Text Text Text Text Text Text Text Text Text Text
Text Text Text Text Text Text Text Text Text Text Text
Text Text Text Text Text Text Text Text Text Text Text

First of Chris Rowley's fiendish tests:
\begin{verbatim}
the double end test<text>
\end\end{verbatim}  or even \end \end{verbatim}
%
%not \end\ended??
%\end{verbatim}

Another of Chris' devils:
\begin{verbatim}
the single brace test<text>
\end{not the end\end{verbatim}
%
%not \end}ed??
%\end{verbatim}
Back to my own tests:
  \begin{myverbatim}
    test
    test
  \end {myverbatim} rest of line
Text Text Text Text Text Text Text Text Text Text Text
Text Text Text Text Text Text Text Text Text Text Text
Text Text Text Text Text Text Text Text Text Text Text

Test of empty verbatim:
\begin{verbatim}
\end{verbatim}
Text Text Text Text Text Text Text Text Text Text Text
Text Text Text Text Text Text Text Text Text Text Text
Text Text Text Text Text Text Text Text Text Text Text
  \begin {verbatimlisting}{verbtest.tex}
    Additional verbatim text
      ...
  \end{verbatimlisting}
And here for listing a file:
  \verbatiminput{verbtest.tex}
And again, with explicit spaces:
  \verbatiminput*{verbtest.tex}
Text Text Text Text Text Text Text Text Text Text Text
Text Text Text Text Text Text Text Text Text Text Text
Text Text Text Text Text Text Text Text Text Text Text
  \begin{comment}
    test
    \end{verbatim*}
    test
    \end {comment
    test
    \endverbatim
    test
    \end  verbatim
    test
  \end {comment} Further text to be typeset: $\alpha$.
Text Text Text Text Text Text Text Text Text Text Text
Text Text Text Text Text Text Text Text Text Text Text
Text Text Text Text Text Text Text Text Text Text Text
  \begin{comment}  bla bla
    test
    test
  \end {comment}
Text Text Text Text Text Text Text Text Text Text Text
Text Text Text Text Text Text Text Text Text Text Text
Text Text Text Text Text Text Text Text Text Text Text
Text Text Text Text Text Text Text Text Text Text Text

\begin{verbatimwrite}{verbtest.txt}
asfa<fa<df
sdfsdfasd
asdfa<fsa
\end{verbatimwrite}

\end{document}
%</testdriver>
%    \end{macrocode}
%
%
% \Finale
%
\endinput
%%
