\documentclass[10pt,imperial]{octavo}
%\usepackage[top=10mm,bottom=60mm,left=25mm,right=25mm, a4paper]{geometry}
%\usepackage[pass]{geometry}
%\newgeometry{left=5cm,right=5cm,top=0in,bottom=1.5cm, %left is inner
%    marginparsep=5mm,marginparwidth=30mm,
%    headheight=20mm,headsep=1cm,
%    footskip=1.5cm}

\usepackage{microtype,soul,filecontents,pifont}
\usepackage[usenames,dvipsnames,svgnames]{xcolor}
\usepackage{amsmath,etoolbox}
 \usepackage{pgf}
\usepackage{tikz}
\usepackage{picture}
\usepackage{lettrine,caption,multicol}

\usepackage{lipsum,soul}
%\usepackage{palatino}
\usepackage{calligra}
\usepackage{fourier-orns}
%\usepackage[T1]{fontenc}

\usepackage{eso-pic}
\usepackage{layouts}
\usepackage{alphalph}
\usepackage{caption}
\usepackage{fmtcount}
\usepackage[listings,theorems]{tcolorbox}
%\usepackage[charter]{mathdesign}
% \def\rmdefault{bch} % not scaled
% \def\ttdefault{blg}
\usepackage{filecontents,ragged2e}
\makeatletter
%% decide on fonts
\IfFileExists{ifxetex.sty}{%
  \RequirePackage{ifxetex}}{}
  \ifxetex
     \usepackage{fontspec}
     \defaultfontfeatures{Mapping=tex-text}
     \setmainfont{Minion Pro}
     %\setsansfont{Georgia}
  \else
     \usepackage{mathpazo}
     \usepackage[T1]{fontenc}
  \fi


% Define some shortcut macros for error/warning/info logging.

\newcommand{\athenawarning}[1]{\PackageWarning{athena}{#1}}
\athenawarning{We are starting on an adventure!}
%

%\usetikzlibrary{decorations.markings}
%\usepackage{doc}
%\usetikzlibrary{calc} 
\usetikzlibrary{decorations,decorations.shapes,shapes,fadings,patterns}
     % We need lots of libraries...
        \usetikzlibrary{
          arrows,
          calc,
          fit,
          patterns,
          plotmarks,
          shapes.geometric,
          shapes.misc,
          shapes.symbols,
          shapes.arrows,
          shapes.callouts,
          shapes.multipart,
          shapes.gates.logic.US,
          shapes.gates.logic.IEC,
          circuits.logic.US,
          circuits.logic.IEC,
          circuits.logic.CDH,
          circuits.ee.IEC,
          datavisualization,
          datavisualization.formats.functions,
          er,
          automata,
          backgrounds,
          chains,
          topaths,
          trees,
          petri,
          mindmap,
          matrix,
          calendar,
          folding,
          fadings,
          shadings,
          spy,
          through,
          turtle,
          positioning,
          scopes,
          decorations.fractals,
          decorations.shapes,
          decorations.text,
          decorations.pathmorphing,
          decorations.pathreplacing,
          decorations.footprints,
          decorations.markings,
          shadows,
          lindenmayersystems,
          intersections,
          fixedpointarithmetic,
          fpu,
          svg.path,
          external,
        }



\definecolor{theblue} {rgb}{0.02,0.04,0.48}
\definecolor{thered}  {rgb}{0.65,0.04,0.07}
\definecolor{thegreen}{rgb}{0.06,0.44,0.08}
\definecolor{thelightgreen}{rgb}{0.06,0.44,0.06}
\definecolor{thegrey} {gray}{0.5}
\definecolor{thegray} {gray}{0.5}
\definecolor{thedarkgray} {gray}{0.95}
\definecolor{theshade}{gray}{0.94}
\definecolor{theframe}{gray}{0.75}
\definecolor{thecream}{rgb}{1,0.95,0.4}
\definecolor{spot}{rgb}{0,0.2,0.6}
\definecolor{boxframe}{gray}{0.8}
\definecolor{boxfill}{rgb}{0.95,0.95,0.99}
\definecolor{theoption}{rgb}{0.118,0.546,0.222}
\definecolor{themacro}{rgb}{0.784,0.06,0.176}
\definecolor{ExampleFrame}{rgb}{0.628,0.705,0.942}
\definecolor{ExampleBack}{rgb}{0.963,0.971,0.994}
\definecolor{Hyperlink}{rgb}{0.281,0.275,0.485}
\colorlet{thehyperlink}{theblue}
\newcommand*{\defaultcolor}{\color{black}}
\newcommand*{\spotcolor}{\color{spot}}

\newcommand\lorem{Fusce adipiscing justo nec ante. Nullam in enim.
 Pellentesque felis orci, sagittis ac, malesuada et, facilisis in,
 ligula. Nunc non magna sit amet mi aliquam dictum. In mi. Curabitur
 sollicitudin justo sed quam et quadd. \par}
%http://tex.stackexchange.com/questions/41150/background-baseline-grid-with-output-routine

\iffalse
\AddToShipoutPicture{%
    \begin{tikzpicture}[overlay,remember picture]
        \draw[thick,red]
              (current page.north east)
              rectangle (current page.south west);
        \draw[red!30!white,thin]
             (current page.south west) grid[step=2mm]
             (current page.north east);
        \draw[red!30!white,dotted]
             (current page.south west) grid[step=1cm]
             (current page.north east);
    \end{tikzpicture}%
}
\fi

\lstloadlanguages{[LaTeX]TeX, [primitive]TeX}

% Emphasis
\newcommand\emphasis[2][thered]{\lstset{emph={newcommand,def,gdef,#2},
   emphstyle={\bfseries\ttfamily\textcolor{#1}}}}%

\lstset{language={[LaTeX]TeX},
      escapeinside={{(*@}{@*)}}, 
       numbers=left, gobble=0,
       stepnumber=1,numbersep=5pt, 
       numberstyle={\footnotesize\color{gray}},firstnumber=10,
       breaklines=true,
       framesep=5pt,
       basicstyle=\small\ttfamily,
       showstringspaces=false,
      keywordstyle=\bfseries\ttfamily\textcolor{thegreen},
      stringstyle=\color{orange},
      commentstyle=\color{black},
      rulecolor=\color{theshade},
      breakatwhitespace=true,
     showspaces=false,  % shows spacing symbol
      xleftmargin=0pt,
      xrightmargin=5pt,
      aboveskip=3pt plus1pt minus1pt, % compact the code looks ugly in type
      belowskip=7pt plus1pt minus1pt,  % user responsible to insert any skips
      backgroundcolor=\color{theshade}
}

\graphicspath{{chapters/}{images/}}

\begin{filecontents*}{chapterx.sty}
\ProvidesPackage{chaptersx}[2012/04/07 v0.1 Typesetting chapters]

\def\HUGE{\@setfontsize\Huge{38}{47}}
\def\HHUGE{\@setfontsize\HHUGE{58}{67}}


\def\@words@cx#1{%
  \ifcase#1 zero\or one\or two\or three\or four\or five\or six\or seven\or eight\or nine\or ten\or
   eleven\or twelve\or thirteen\or fourteen\or fifteen\or sixteen\or seventeen\or eighteen\or nineteen \or twenty\or twenty one\or twenty two\or twenty three\or twenty four\or twenty five\or twenty six\or twenty seven \or twenty eight \or twenty nine\or thirty\or thirty one\or thirty two\or thirty three\or thirty four\or thirty five\or thirty six\or thirty seven\or thirty eight\or thirty nine\or forty\or forty one\or forty two \or forty three\or forty four\or forty five \or forty six \or forty seven\or forty eight \or forty nine\or fifty\or fifty on\or fifty two\or fifty three\or fifty four\or fifty five\or fifty six\or fifty seven\or fifty eight\or fifty nine\or sixty \or sixty one \or sixty two
\or sixty three \or sixty four \or \sixty five\else\@ctrerr\fi}



\newenvironment{absquote}
               {\list{}{\leftmargin2cm\rightmargin\leftmargin}%
                \item\relax\footnotesize}
               {\endlist}

\newenvironment{summary}
               {\list{}{\listparindent0pt %
                        \itemindent\listparindent
                        \leftmargin0pt
                        \rightmargin\leftmargin
                        \parsep\z@ \@plus\p@}%
                \item\relax\itshape}
               {\endlist}


% helper macros for rulers
\gdef\thinrule{\rule{\textwidth}{0.4pt}}
\gdef\mediumrule{\rule{\textwidth}{0.8pt}}
\def\thickrule{\leavevmode \leaders \hrule height 3pt \hfill \kern \z@}


%% broad positions
\newif\if@left
\newif\if@right
\newif\if@center
\@leftfalse
\@rightfalse
\@centerfalse
% newifs for number position
\newif\if@lefttitle
\newif\if@righttitle
\newif\if@leftname
\newif\if@rightname
\newif\if@chapterspaceout\@chapterspaceoutfalse
\newif\if@titlespaceout\@chapterspaceoutfalse
\newif\if@sectionspaceout\@sectionspaceoutfalse
\newif\if@openanywhere\@openanywherefalse
%% The chapter command does not allow for openleft pages which we need
%% for some of the designs. 
\newif\if@openleft\@openleftfalse
\newif\if@openany\@openanyfalse
%
% Switch to indicate if an item should be added to the toc or not.
\newif\if@toc  \@toctrue
\newif\if@tocimage \@tocimagefalse

% Boolean set to true if a chapter head includes an author block
\newif\if@authorblock

%
%
\def\cx@optionlist{}


\def\cx@optionlist{}

\def\cxuselibrary#1{\cxset{library/.cd,#1}}
%
% The library is added by inputting the file and setting the path accordingly.
\def\cx@add@library#1#2{%
  \cxset{library/#1/.code={\@ifundefined{cxlibrary@#1@loaded}{\input #2}{}}}%
  \DeclareOption{#1}{\edef\cx@optionlist{\cx@optionlist,#1}}%
}

% Process options need to add




% Keys for styling chapters
% Define a family for chapter styling keys
\pgfkeys{/chapter/.is family}


\def\cxset{\pgfqkeys{/chapter}} %Notice this is pgf q keys


\cxset{%
  name/.code={\gdef\chaptername{#1}},
  chapter name/.code={\gdef\chaptername{#1}},
  color/.store in=\color@cx,
  color/.default=black,
  chapter opening/.is choice,
  chapter opening/right/.code={\@openrighttrue},
  chapter opening/left/.code={\@openlefttrue},
  chapter opening/any/.code={\@openanytrue},
  chapter opening/none/.code={\@openanywheretrue\@openrightfalse
                                              \@openleftfalse\@openanyfalse},
  chapter opening/anywhere/.code={\@openanywheretrue\@openrightfalse
\@openleftfalse\@openanyfalse},
  chapter opening/ifafter/.code={}, %still to come with pagesofar
  chapter font-family/.store in=\chapterfontfamily@cx,
  chapter font-weight/.store in=\chapterfontweight@cx,
  chapter font-size/.store in=\chapterfontsize@cx,
  chapter color/.store in=\chaptercolor@cx,
  chapter before/.store in=\chapterbefore@cx,
  chapter after/.store in=\chapterafter@cx,
  chapter spaceout/.is choice,
  chapter spaceout/soul/.code={\@chapterspaceouttrue},
  chapter spaceout/none/.code={\@chapterspaceoutfalse},
  % title keys
  title font-family/.store in=\titlefontfamily@cx,
  title font-family/.default=\rmfamily,
  title font-weight/.store in=\titlefontweight@cx,
  title font-size/.store in=\titlefontsize@cx,
  title font-color/.store in=\titlefontcolor@cx,
  title font-shape/.store in= \titlefontshape@cx,
  title spaceout/.is choice,
  title spaceout/soul/.code={\@titlespaceouttrue},
  title spaceout/none/.code={\@titlespaceoutfalse},
  title font/.style={title font-family=#1},
  title before/.store in=\titlebefore@cx,
  title after/.store in=\titleafter@cx,
  title beforeskip/.store in=\titlebeforeskip@cx,
  title afterskip/.store in=\titleafterskip@cx,
  position/.is choice,
  position/left/.code={\@lefttrue},
  position/right/.code={\@righttrue},
  position/center/.code={\@centertrue},
% numbering options that are required
  numbering/.is choice,
% better to rename thechapter?
  numbering/roman/.code={\gdef\thechapter{\@roman\c@chapter}},
  numbering/Roman/.code={\gdef\thechapter{\@Roman\c@chapter}},
  numbering/arabic/.code={\gdef\thechapter{\@arabic\c@chapter}},
  numbering/alpha/.code={\gdef\thechapter{\alphalph\c@chapter}},
  numbering/Alpha/.code={\gdef\thechapter{\AlphAlph\c@chapter}},
  numbering/words/.code={\gdef\thechapter{\MakeTextLowercase{\expandafter\@words@cx{\expandafter\@arabic\c@chapter}}}},
%% These proved a bit trouble some and ended up calling the fmtcount package routines
  numbering/WORDS/.code={\gdef\thechapter{\NUMBERstring{chapter}}},
  numbering/Words/.code={\gdef\thechapter{\NumToName{\expandafter\@arabic\c@chapter}}},
%% These is for my own use for EJD type numbering, placed in an mbox as there is something funny
% happening with spaces.
  numbering/padzeroes/.code={\gdef\thechapter{\mbox{EWD -\padzeroes[4]\decimal{chapter}}%
  }%
 },
  numbering/none/.code={\gdef\thechapter{}}, % do not leave empty gives other problems??? 
  number dot/.store in=\numberpunctuation@cx,
  number position/.is choice,
  number position/leftname/.code={\@leftnametrue\@rightnamefalse},
  number position/rightname/.code={\@rightnametrue\@leftnamefalse},
  number position/absolute/.code={},
  number position/righttitle/.code={\@righttitletrue},
  number position/lefttitle/.code={\@lefttitletrue},
  number after/.store in=\numberafter@cx,
  number before/.store in=\numberbefore@cx,
  number color/.store in=\numbercolor@cx,
  number font-size/.store in=\numberfontsize@cx,
  number font-family/.store in=\numberfontfamily@cx,
  number font-weight/.store in=\numberfontweight@cx,
% authorblock
  author block/.is choice,
  author block/true/.code={\@authorblocktrue},
  author block/false/.code={\@authorblockfalse},
  author names/.store in=\authorblock@cx,  
  author block format/.store in=\authorblockformat@cx,
  chapter toc/.is choice,
  chapter toc/true/.code=\@toctrue\cxset{numbering=arabic,name=Chapter},
  chapter toc/false/.code=\@tocfalse,
  chapter toc/none/.code=\@tocfalse,
% sectioning commands
  section font-size/.store in=\sectionfontsize@cx,
  section font-weight/.store in=\sectionfontweight@cx,
  section font-family/.store in=\sectionfontfamily@cx,
  section font-shape/.store in=\sectionfontshape@cx,
  section color/.store in=\sectioncolor@cx,
  section numbering/.is choice,
  section numbering/roman/.code={\gdef\thesection{\thechapter\@roman\c@section}\renewsection},
  section numbering/(roman)/.code={\gdef\thesection{\thechapter(\@roman\c@section)}\renewsection},
  section numbering/arabic/.code={\gdef\thesection{\thechapter.\@arabic\c@section}\renewsection},
  section numbering/numeric/.code={\gdef\thesection{\thechapter.\@arabic\c@section}\renewsection},
  section numbering custom/.code=\gdef\thesection{#1}\renewsection,
  section numbering/none/.code={\gdef\thesection{}\renewsection},
  section align/.store in=\sectionalign@cx,
  section beforeskip/.store in=\sectionbeforeskip@cx,
  section afterskip/.store in=\sectionafterskip@cx,
  section indent/.store in=\sectionindent@cx,
  section spaceout/.is choice,
  section spaceout/soul/.code={\@sectionspaceouttrue},
  section spaceout/none/.code={\@sectionspaceoutfalse},
% subsections
  subsection font-size/.store in=\subsectionfontsize@cx,
  subsection font-weight/.store in=\subsectionfontweight@cx,
  subsection font-family/.store in=\subsectionfontfamily@cx,
  subsection font-shape/.store in=\subsectionfontshape@cx,
  subsection color/.store in=\subsectioncolor@cx,
  subsection numbering/.is choice,
  subsection numbering/arabic/.code={\gdef\thesubsection{\thesection.\@arabic\c@subsection}},
  subsection numbering/custom/.store in=\thesubsection@cx,
  subsection numbering/none/.code={\gdef\thesubsection{}},
  subsection align/.store in=\subsectionalign@cx,
  subsection beforeskip/.store in=\subsectionbeforeskip@cx,
  subsection afterskip/.store in=\subsectionafterskip@cx,
  subsection indent/.store in=\subsectionindent@cx,
%%
%% subsubsection
  subsubsection font-size/.store in=\subsubsectionfontsize@cx,
  subsubsection font-weight/.store in=\subsubsectionfontweight@cx,
  subsubsection font-family/.store in=\subsubsectionfontfamily@cx,
  subsubsection font-shape/.store in=\subsubsectionfontshape@cx,
  subsubsection color/.store in=\subsubsectioncolor@cx,
  subsubsection numbering/.is choice,
  subsubsection numbering/numeric/.code={\gdef\thesubsubsection{\thesubsection.\@arabic\c@subsubsection}},
  subsubsection numbering/custom/.store in=\thesubsubsection@cx,
  subsubsection numbering/none/.code={\gdef\thesubsubsection{}},
  subsubsection align/.store in=\subsubsectionalign@cx,
  subsubsection beforeskip/.store in=\subsubsectionbeforeskip@cx,
  subsubsection afterskip/.store in=\subsubsectionafterskip@cx,
  subsubsection indent/.store in=\subsubsectionindent@cx,
%
% paragraph
%
  paragraph font-size/.store in=\paragraphfontsize@cx,
  paragraph font-weight/.store in=\paragraphfontweight@cx,
  paragraph font-family/.store in=\paragraphfontfamily@cx,
  paragraph font-shape/.store in=\paragraphfontshape@cx,
  paragraph color/.store in=\paragraphcolor@cx,
  paragraph numbering/.is choice,
  paragraph numbering/numeric/.code={\gdef\theparagraph{\thesubsubsection.\@arabic\c@paragraph}},
  paragraph numbering/custom/.store in=\theparagraph@cx,
  paragraph numbering/none/.code={\gdef\theparagraph{}},
  paragraph align/.store in=\paragraphalign@cx,
  paragraph beforeskip/.store in=\paragraphbeforeskip@cx,
  paragraph afterskip/.store in=\paragraphafterskip@cx,
  paragraph indent/.store in=\paragraphindent@cx,
%% subparagraphs
%
  subparagraph font-size/.store in=\subparagraphfontsize@cx,
  subparagraph font-weight/.store in=\subparagraphfontweight@cx,
  subparagraph font-family/.store in=\subparagraphfontfamily@cx,
  subparagraph font-shape/.store in=\subparagraphfontshape@cx,
  subparagraph color/.store in=\subparagraphcolor@cx,
  subparagraph numbering/.is choice,
  subparagraph numbering/numeric/.code={\gdef\thesubparagraph{\theparagraph.\@arabic\c@subparagraph}},
  subparagraph numbering/arabic/.code={\gdef\thesubparagraph{\theparagraph.\@arabic\c@subparagraph}}
  subparagraph numbering/custom/.store in=\thesubparagraph@cx,
  subparagraph numbering/none/.code={\gdef\thesubparagraph{}},
  subparagraph align/.store in=\subparagraphalign@cx,
  subparagraph beforeskip/.store in=\subparagraphbeforeskip@cx,
  subparagraph afterskip/.store in=\subparagraphafterskip@cx,
  subparagraph indent/.store in=\subparagraphindent@cx,
  subparagraph number after/.estore in=\subparagraphnumberafter@cx,
  subparagraph number after/.default=,
  subparagraph number after/.initial=,
%
%% headers and footers
  header style/.store in=\headerstyle@cx,
% general draft rules
  rule /.is choice,
  rule on/.code={\gdef\rulewidth@cx{0.4pt}},
  rule off/.code={\gdef\rulewidth@cx{0pt}},
% headers and footers
  lhead/.code ={\lhead{#1}},
  rhead/.code={\rhead{#1}},
  chead/.code={\chead{#1}},
  lfoot/.code ={\lhead{#1}},
  cfoot/.code={\chead{#1}},
  rfoot/.code={\rhead{#1}},
  headrulewidth/.code={\renewcommand\headrulewidth{#1}},
  footrulewidth/.code={\renewcommand\footrulewidth{#1}},
}



%% Commands to renew sectioning, these unfortunately need to be 
%% called explicitly
\def\renewsection{%
\renewcommand\section{%
\@startsection{section}%
{1}%level
{\sectionindent@cx}%indent
{\sectionbeforeskip@cx}%
{\sectionafterskip@cx}%
{\sectionfontweight@cx\sectionfontfamily@cx%
\sectionfontsize@cx\sectionfontshape@cx\sectionalign@cx}%
}%
}

\def\renewsubsection{%
\renewcommand\subsection{%
 \@startsection{subsection}%
{2}%level
{\subsectionindent@cx}%indent
{\subsectionbeforeskip@cx}%
{\subsectionafterskip@cx}%
{\subsectionfontweight@cx\subsectionfontfamily@cx\subsectionfontshape@cx%
 \subsectionfontsize@cx\color{\subsectioncolor@cx}\subsectionalign@cx}%
}%
}



\def\renewsubsubsection{%
\renewcommand\subsubsection{%
 \@startsection{subsubsection}%
{3}%level
{\subsubsectionindent@cx}%indent
{\subsubsectionbeforeskip@cx}%
{\subsubsectionafterskip@cx}%
{\subsubsectionfontweight@cx\subsubsectionfontfamily@cx%
\subsubsectionfontsize@cx\subsubsectionfontshape@cx\subsubsectionalign@cx}%
}}

\def\renewparagraph{%
\renewcommand\paragraph{%
 \@startsection{paragraph}%
{4}%level
{\paragraphindent@cx}%indent
{\paragraphbeforeskip@cx}%
{\paragraphafterskip@cx}%
{\paragraphfontweight@cx\paragraphfontfamily@cx%
\paragraphfontsize@cx\paragraphfontshape@cx\paragraphalign@cx}%
}}

\def\renewsubparagraph{%
\renewcommand\subparagraph{%
 \@startsection{subparagraph}%
{5}%level
{\subparagraphindent@cx}%indent
{\subparagraphbeforeskip@cx}%
{\subparagraphafterskip@cx}%
{\subparagraphfontweight@cx\subparagraphfontfamily@cx%
\subparagraphfontsize@cx\subparagraphfontshape@cx\subparagraphalign@cx}%
}%
\def\@seccntformat##1{\csname the##1\endcsname\subparagraphnumberafter@cx}%
}

% set some defaults

\cxset{%
  lhead=left head text,
  chead=center text,
  rhead=right text,
  lfoot=left text,
  cfoot=center text,
  rfoot=right text,
  headrulewidth=2pt,
  footrulewidth=0.4pt,
  color=blue,
  position=center,
  name=chapteris,
  title font-family=\rmfamily,
  title font-weight=\bfseries,
  title font-size=\Huge,
  title font-color=\color{olive},
  numbering=arabic,
  number dot=,
  chapter font-family=\sffamily,
  chapter font-weight=\bfseries,
  chapter font-size=\Large,
  chapter color=gray,
  chapter before={\leavevmode\par\hfill},%need to correct for 0pt
  chapter after=\hfill\hfill,
  number position=rightname,
  number color=\color{gray},
  number after=\hspace{20pt},
  number before=\hspace*{20pt},
  number font-size=\huge,
  number font-family=\sffamily,
  number font-weight=\bfseries,
  number before=,
  number after=,
  title before=,
  title after=,
  title afterskip={\vskip50pt},
  title beforeskip={\vskip10pt},
  header style=fancy,
}



\gdef\setdefaults{%
\cxset{%
  lhead=left head text,
  chead=center text,
  rhead=right text,
  lfoot=left text,
  cfoot=center text,
  rfoot=right text,
name=CHAPTER,
title font-family=\rmfamily,
title font-weight=\bfseries,
title font-size=\Huge,
title font-color=\color{purple},
numbering=arabic,
number dot=,
number before=,
number after=,
chapter font-family=\sffamily,
chapter font-weight=\bfseries,
chapter font-size=\large,
chapter spaceout=soul,
chapter color=gray,
chapter before={},%need to correct for 0pt
chapter after=,
number position=rightname,
number color=\color{gray},
number after=\hspace{20pt},
number before=\space,
number font-size=\large,
number font-family=\sffamily,
number font-weight=\bfseries,
title before=,
title after=,
title afterskip={\vskip24pt},
title beforeskip={\vskip10pt},
title font=\rmfamily,
header style=plain,
section font-weight=\bfseries,
section font-family=\rmfamily,
section font-size=\Large,
section font-shape=\upshape,
section align=\raggedright,
title font-shape=\upshape,
subsection color=teal,
}
}

%% STYLE 13
\cxset{style13/.style={
 name={Chapter},
 numbering=arabic,
 number font-size=\HUGE,
 number font-family=\sffamily,
 number font-weight=\bfseries,
 number color=\color{gray!50},
 number before=\par\vspace*{5pt}\hfill\hfill,
 number dot=,
 number after={\hspace*{7pt}\par},
 number position=rightname,
 chapter font-family=\sffamily,
 chapter font-weight=\normalfont,
 chapter font-size=\LARGE,
 chapter before={\thinrule\vspace*{20pt}\par\hfill\hfill},
 chapter after={\vskip0pt\par},
 chapter color={black!50},
 title beforeskip={\vspace*{10pt}},
 title afterskip={\vspace*{50pt}\par},
 title before={\hfill\hfill\raggedleft},
 title after={},
 title font-family=\sffamily,
 title font-color=\color{purple},
 title font-weight=\bfseries,
 title font-size=\huge,
 section indent=-1em,
 section align=\raggedright,
 section numbering=arabic,
 section indent=0pt,
 section beforeskip=0pt,
 section afterskip=\baselineskip,
 subsection align=\raggedright,
 subsection font-family=\sffamily,
 subsection font-weight=\bfseries,
 subsection font-size=\large,
 subsection font-shape=\itshape,
 subparagraph number after=\space,
}
}



% The |\@makechapterhead| is used by LaTeX to typeset the
% Chapter heading. This is a rewrite in order to make it
% more flexible and use the keys.
% #1 spare
% #2 title
\newif\if@special\@specialfalse
\cxset{custom/.code=\gdef\customdesign@cx{\csname#1\endcsname}\@specialtrue,
       fill/.store in=\fill@cx}
\cxset{custom=tikzspecial,
    title font-size=\Large,
    title font-color=\color{white},
    title font-family=\sffamily,
    fill=thegrey,
}

% Helper macro for use with the makechapterheadmacro
\def\printnumber{%% macro for typesetting the chapter number
    \numberbefore@cx
      {%
      \numberfontweight@cx
      \numbercolor@cx
      \numberfontsize@cx
      \numberfontfamily@cx
      \thechapter
      \numberpunctuation@cx
      }
      \numberafter@cx
  }


\renewcommand\@makechapterhead[2][]{%
% The special switch triggers the special macro that format the head
% this is used for TikZ designs etc.
\if@special
    \customdesign@cx{#2}
   \else
  %
% macro for typesetting the chapter name
  \def\printchaptername{%
    {
    \chapterfontfamily@cx
    \chapterfontsize@cx
    \chapterfontweight@cx
    \color{\chaptercolor@cx}%
% Check if the chapter name is spaced out and use the
% soul package. TODO add values for soul parameters
% as a style feature.
    \if@chapterspaceout 
     \expandafter\so\expandafter{\chaptername}
    \else
      \@chapapp\space
    \fi
    }%
  }%
  \def\printauthorblock{
      \authorblockformat@cx\authorblock@cx%
  }%
% set all keys
    {%
    \parindent0pt 
    \normalfont%
    \ifnum \c@secnumdepth>\m@ne%
      \if@mainmatter%
 % we start displaying  the names and any preambles such 
 % as images
 % print chapter name
         \chapterbefore@cx%
         \if@leftname
            \printnumber
         \fi%
         \printchaptername
         \if@rightname
            \printnumber
         \fi%
         \chapterafter@cx  
      \fi%
    \fi%
     %chapter title
    \interlinepenalty\@M%
% if the number is before the title
% if the number prints together with the title they 
% are considered as one indivisible part.
     \titlebeforeskip@cx%
     \if@lefttitle%
       \beforenumber@cx%
       \counterdisplay\c@chapter\afternumber@cx%
     \fi
% Display title
       \titlefontweight@cx
      \titlefontfamily@cx
      \titlefontshape@cx
      \titlefontsize@cx
      \titlefontcolor@cx
      \selectfont
      \titlebefore@cx%
      \if@titlespaceout
         \so{#2}%
      \else
         \texorpdfstring{#2}%title
      \fi%
      \titleafter@cx
    \if@righttitle%
      \afternumber@cx
      \counterdisplay\c@chapter\afternumber@cx%
    \fi
% If there is an author block format it and print it
    \if@authorblock\printauthorblock\fi%
    \par\nobreak%
% skip after title
    \titleafterskip@cx
% headers
   }
\fi
}


%% Special Chapter command. This one just typesets a picture on top of a 
%% Chapter head
\newcommand\specialchapter@cx[2][]{%
\refstepcounter{chapter}
\cxset{image/.store in=\image@cx,
       image caption/.store in=\caption@cx}
\cxset{#1}
\vbox to 0pt{\color{blue}\rule{\paperwidth}{0.4pt}\par\vskip-1.4pt
\rule{0.4pt}{\textheight}\rule{4cm}{0.4pt}}

\vbox to 0pt{\parbox[b]{4.7cm}{%
\raggedright

\leftskip1.5cm
\caption@cx\par
 \expandafter\rule{\rulewidth@cx}{5.8cm}
}\parbox[b]{0.5cm}{\includegraphics[width=0.5cm,height=9.15cm]{./chapters/shadow}}\includegraphics{./chapters/\image@cx}\par}

\vspace{8.2cm}
\hspace*{-3.51cm}\hbox to 0pt{\hspace*{1.01cm}\includegraphics[width=7.7cm,height=3.8cm]{./chapters/genetics-band}
\hspace*{-2.7cm}\sffamily\color{\numbercolor@cx}\HHUGE \raise30pt\hbox{\thechapter}%
\hspace{1.5cm}\raise0.5pt\hbox{\includegraphics{./chapters/chapterconcept}\includegraphics{./chapters/shadow2}}
}

%% Title name
\parbox[b]{0.45\textwidth}{%
  \titlefontsize@cx
  \titlefontweight@cx
  \titlefontfamily@cx
  \leftskip0.5em \color{\titlefontcolor@cx}
  #2
}
%% Concepts
}

\newenvironment{specialchapter}[2][]{%
  \if@openright\cleardoublepage\else\clearpage\fi
    \thispagestyle{plain}%
    \global\@topnum\z@
    \@afterindentfalse
    \specialchapter@cx[#1]{#2}
    \begin{minipage}{0.5\textwidth}%
    \vspace{0.5\baselineskip}
    \raggedright
}{\end{minipage}}

%% Utility command
\providecommand{\cleartoevenpage}[1][\@empty]{%
 \clearpage%
 \ifodd\c@page\hbox{}#1\clearpage\fi}

\end{filecontents*}
\usepackage{fancyhdr}
\usepackage{chapterx}
\usepackage{makeidx}
\makeindex
\author{Y Lazarides}
\title{\parindent0pt A New Approach in\\ Styling Chapters}

\usepackage{textcase}

%% bibliography related
\usepackage[american]{babel}
\usepackage{csquotes}
\usepackage[style=authoryear]{biblatex}

% set-up the hyperef package properly
\usepackage{hyperref}
\hypersetup{%
  colorlinks=true,
  linkcolor=DarkBlue,
  pagecolor=DarkBlue,
  urlcolor=spot,
  bookmarks=true,
  bookmarksopen=true,
  bookmarksnumbered=false,
  plainpages=false,
  pdfpagemode=UseOutlines,
  pdfview=FitH,
  pdfstartview=FitH}

\addbibresource{biblatex-examples.bib}
\setlength{\parindent}{0pt}

% Some generic settings
\def\cmd#1{\texttt{\textbackslash #1}}

\def\file#1{\protect\texttt{\textbackslash #1}}


\newenvironment{bibsample}
  {\trivlist\samepage
   \setlength{\itemsep}{0pt}}
  {\endtrivlist}

%% doccommands
\newcommand*{\marglistfont}{\itshape}
\newcommand*{\margnotefont}{}
\newcommand*{\optionlistfont}{\bfseries}
\newcommand*{\ltxsyntaxfont}{\ttfamily}
\newcommand*{\ltxsyntaxlabelfont}{\bfseries}
\newcommand*{\changelogfont}{\normalfont}
\newcommand*{\changeloglabelfont}{\bfseries}
\newcommand*{\verbatimfont}{\ttfamily}
\newcommand*{\displayverbfont}{\ttfamily}
\renewcommand*{\verbatim@font}{\verbatimfont}
\renewrobustcmd*{\cmd}[1]{\mbox{\verbatimfont\bfseries\textbackslash#1}}
\newrobustcmd*{\env}[1]{\mbox{\verbatimfont\bfseries\textcolor{thegreen}{#1}}}
\newrobustcmd*{\len}[1]{\mbox{\verbatimfont\textbackslash#1}}
\newrobustcmd*{\cnt}[1]{\mbox{\verbatimfont#1}}
\newlength{\marglistsep}
\newlength{\marglistwidth}
\setlength{\marglistwidth}{(\oddsidemargin+1in)*85/100}%
\deflength{\marglistsep}{10pt}
%% This needs thorough checking as to restore previous definitions
%% of parsep we want parsep to be a bit higher than standard enumerated lists.
\global\newlength\oldparsep
\newenvironment*{marglist}
  {\setlength\oldparsep{\parsep}\list{}{%
     \parsep 3.5\p@ \@plus0\p@ \@minus\p@
     \setlength{\labelwidth}{\marglistwidth}%
     \setlength{\labelsep}{\marglistsep}%
     \setlength{\leftmargin}{0pt}%
     \renewcommand*{\makelabel}[1]{\hss\marglistfont##1}}}
  {\endlist\setlength\parsep{\oldparsep}}

\newenvironment*{keymarglist}
  {\marglist
   \setlength{\itemsep}{0pt}%
   \raggedright}
  {\endmarglist}

%% DOCUMENTATION MACROS



% color definitions
\def\colDef#1{\textcolor{themacro}{#1}}
% color for options
\def\colOpt#1{\textcolor{theblue}{#1}}
\newcommand{\option}[1]{\colOpt{#1}}

% The following macros are taken from ltxdoc
 \ifx\l@nohyphenation\undefined
   \newlanguage\l@nohyphenation
 \fi
 \DeclareRobustCommand\meta[1]{%
%Since the old implementation of \meta could be used in math we better ensure
%that this is possible with the new one as well. So we use \ensuremath around
%\langle and \rangle. However this is not enough: if \meta@font@select below
%expands to \itshape it will fail if used in math mode. For this reason we hide
%the whole thing inside an \nfss@text box in that case.
 \ensuremath\langle
 \ifmmode \expandafter \nfss@text \fi
 {%
 \meta@font@select
%Need to keep track of what we changed just in case the user changes font inside
%the argument so we store the font explicitly.
 \edef\meta@hyphen@restore
 {\hyphenchar\the\font\the\hyphenchar\font}%
 \hyphenchar\font\m@ne
 \language\l@nohyphenation
 #1\/%
 \meta@hyphen@restore
 }\ensuremath\rangle
}
\def\meta@font@select{\itshape}
\def\cmd#1{\cs{\expandafter\cmd@to@cs\string#1}}
\def\cmd@to@cs#1#2{\char\number`#2\relax}

%%% stop error
% \newrobustcmd*{\cs}[1]{\mbox{\verbatimfont\bfseries\textbackslash#1}}
% This documentation macro is defined in a few places check if undefined and act accordingly
% override everything
\ifx\cs\undefined
\DeclareRobustCommand\cs[1]{{\color{themacro}\bfseries\verbatimfont\char`\\#1}%
  \index{macros!\textbackslash#1}}%
\else
  \let\cs\relax
  \DeclareRobustCommand\cs[1]{{\color{themacro}\bfseries\verbatimfont\char`\\#1}%
  \index{macros!\textbackslash#1}}
\fi


%
\def\marg#1{%
  {\ttfamily\char`\{}\meta{#1}{\ttfamily\char`\}}}


\def\oarg#1{%
  \colOpt{{\ttfamily[}\meta{#1}{\ttfamily]}}}

\def\pkg#1{\textbf{#1}}
\tcbset{after=\medskip,before=\bigskip}


% Many of the chapter styles we provide have epigraphs incorporated. We
% provide the necessary keys to set parameters. We follow terminology from
% the original package.

\RequirePackage{epigraph}
\makeatletter
\cxset{
  epigraph width/.code={\setlength\epigraphwidth{#1}},
  epigraph font-size/.code={\renewcommand{\epigraphsize}{#1}},
  epigraph beforeskip/.code={\setlength\beforeepigraphskip{#1}},
  epigraph afterskip/.code={\setlength\afterepigraphskip{#1}},
  epigraph align/.is choice,
  epigraph align/center/.code={\renewcommand{\epigraphflush}{center}},
  epigraph align/left/.code={\renewcommand{\epigraphflush}{flushleft}},
  epigraph align/right/.code={\renewcommand{\epigraphflush}{flushright}},
  epigraph source align/.is choice,
  epigraph source align/left/.code={\renewcommand{\sourceflush}{flushleft}},
  epigraph source align/right/.code={\renewcommand{\sourceflush}{flushright}},
  epigraph source align/center/.code={\renewcommand{\sourceflush}{center}},
  epigraph text align/.is choice,
  epigraph text align/left/.code={\renewcommand{\textflush}{flushleft}},
  epigraph text align/right/.code={\renewcommand{\textflush}{flushright}},
  epigraph text align/center/.code={\renewcommand{\textflush}{center}},
  epigraph rule width/.code={\setlength\epigraphrule{#1}},
  epigraph rule/.code={\renewcommand{\@epirule}{\color{orange}\rule[.5ex]{\epigraphwidth}{\epigraphrule}}},
}

\cxset{epigraph width=0.5\linewidth,
       epigraph font-size=\small,
       epigraph rule width=0.4pt,
       epigraph align=right,
       epigraph source align=right,
       epigraph text align=right,
       epigraph rule}

\parindent1em

% documentation macro
\newlength\temp@cx

% \begin{macro}
%    \begin{macrocode}
\long\def\keyval#1#2#3{%
\begingroup
\settowidth\temp@cx{#1}%
\parindent-30pt%\dimexpr\temp@cx+10pt\relax%
\par\leavevmode%
{\color{theblue}\bfseries #1}\thinspace=\thinspace#2 #3%
\par\addvspace{1.5pt}
\endgroup
}
%    \end{macrocode}
% \end{macro}

\newif\if@latexbook
\@latexbookfalse

\if@latexbook
\renewcommand\chapter{\if@openright\cleardoublepage\else\clearpage\fi
                    \thispagestyle{plain}%
%    \end{macrocode}
%    Then we prevent floats from appearing at the top of this page
%    because it looks weird to see a floating object above a chapter
%    title.
%    \begin{macrocode}
                    \global\@topnum\z@
%    \end{macrocode}
%    Then we suppress the indentation of the first paragraph by
%    setting the switch |\@afterindent| to |false|. We use |\secdef|
%    to specify the macros to use for actually setting the chapter
%    title.
%    \begin{macrocode}
                    \@afterindentfalse
                    \secdef\@chapter\@schapter}

\else

%% It also defaults on a pagestyle which we also need to correct
\renewcommand\chapter{%
%  \if@latexbook \expandafter\latexbookchapter
%  \else
    \if@openright\cleardoublepage\fi
    \if@openleft\cleartoevenpage\fi
    \if@openany\clearpage\fi
%% we need to capture the running head keys
%% \headerstyle@cx defaults to empty
    \thispagestyle{\headerstyle@cx}%
    \global\@topnum\z@
    \@afterindentfalse
    \secdef\@chapter\@schapter%must disable \@schapter
 }
\fi


\def\@part[#1]#2{%
    \ifnum \c@secnumdepth >-2\relax
      \refstepcounter{part}%
      \addcontentsline{toc}{part}{\thepart\hspace{1em}#1}%
    \else
      \addcontentsline{toc}{part}{#1}%
    \fi
    \markboth{}{}%
    {\centering
     \interlinepenalty \@M
     \normalfont
     \ifnum \c@secnumdepth >-2\relax
       \huge\bfseries \partname\nobreakspace\thepart
       \par
       \vskip 20\p@
     \fi
     \Huge \bfseries #2\par}%
    \@endpart}


%% Adjusted to get toc parameters in
\DeclareRobustCommand\Rule{{\color{\tocchapternumberfill@cx}\rule[-4.1pt]{13cm}{0.4pt}}}

\def\@chapter[#1]#2{%
    \ifnum \c@secnumdepth >\m@ne
          \if@mainmatter
               \refstepcounter{chapter}%needs checking as to pos
            \if@toc
               \typeout{\@chapapp\space\thechapter.}%
% check if main matter and if toc is set
%             \addcontentsline{toc}{chapter}%
%                                   {\protect\numberline{\thechapter}#1}% might as well add image here
              \addcontentsline{toc}{chapter} {\protect\numberline{%
                                    \protect\colorbox{\protect\tocchapternumberfill@cx}%
                                        {\color{\tocchapternumbercolor@cx}\string\parbox{1.5em} {\protect\tocnumberalign@cx\thechapter\stoptocnumberalign@cx}}%
                   \Rule\kern-13cm\hspace{1em}\bfseries\sffamily}\protect\hspace*{1em}\bfseries#1}%Make Upper case
            \fi%
         \else
                \addcontentsline{toc}{chapter}{#1}%for part???
         \fi
     \else
             \addcontentsline{toc}{chapter}{#1}%????? for part
      \fi
             \chaptermark{#1}%
             \addtocontents{lof}{\protect\addvspace{10\p@}}%
             \addtocontents{lot}{\protect\addvspace{10\p@}}%
%              \if@twocolumn
%                      \@topnewpage[\@makechapterhead{#2}]%
%                \else
                     \@makechapterhead{#2}%
               \@afterheading
%             \fi
   }



% Table of contents key settings
\cxset{
  toc levels/.code={\setcounter{tocdepth}{#1}},
  toc number width/.code = {\def\@pnumwidth{#1}},
  toc chapter number color/.store in=\tocchapternumbercolor@cx,
  toc chapter number fill/.store in=\tocchapternumberfill@cx,
  toc chapter after/.store in=\tocchapterafter@cx,
  toc section indent/.store in=\tocsectionindent@cx,
  toc subsection indent/.store in=\tocsubsectionindent@cx,
  toc subsubsection indent/.store in=\tocsubsubsectionindent@cx,
  toc paragraph indent/.store in=\tocparagraphindent@cx,
  toc subparagraph indent/.store in=\tocsubparagraphindent@cx,
  toc section number width/.store in=\tocsectionnumberwidth@cx,
  toc dots/.is choice,
  toc dots/none/.code=\renewcommand\@dotsep{1000}, %set to high value
  toc dots/true/.code=\renewcommand\@dotsep{4.5},
  toc dotsep/.code=\renewcommand\@dotsep{#1},
%% toc title
  toc chapter name/.store in=\chaptername,
  toc chapter name color/.code=\gdef\tocchapternamecolor@cx{#1},
  toc title color/.store in=\toctitlecolor@cx,
  toc title font-weight/.store in=\toctitlefontweight@cx,
  toc title before/.store in=\toctitlebefore@cx,
  toc title after/.store in=\toctitleafter@cx,
  toc after pagenumber/.store in=\tocafterpagenumber@cx,
  toc right margin/.code=\renewcommand\@tocrmarg{#1},
  toc number after/.store in={\tocnumberafter@cx},
  toc number align/.is choice,
  toc number align/center/.code=\def\tocnumberalign@cx{\centering}\def\stoptocnumberalign@cx{},
  toc number align/left/.code=\def\tocnumberalign@cx{\centering}\def\stoptocnumberalign@cx{},
  toc number align/right/.code=\def\tocnumberalign@cx{\hfill}\def\stoptocnumberalign@cx{},
  toc image/.is if=@tocimage,
  hypersetup linkcolor/.store in=\hypersetuplinkcolor@cx,
}
 
%% SET TOC STYLING
\cxset{chapter toc=true, 
          toc number width=2.7em, 
          toc levels=5,
% Chapter
          toc chapter number color=black,
          toc chapter number fill=cyan,
          toc chapter after=20pt,
% Sectioning
          toc section indent=3.3em,
          toc subsection indent=5.9em,
          toc subsubsection indent=9.2em,
          toc subparagraph indent=9.2em,
          toc paragraph indent=9.2em,
          toc section number width=2.5em,
          toc dots=none,
          toc title font-weight=\bfseries,
          toc right margin=4cm,
          toc number after=\hspace{1em},
          toc title color=thegray,
          toc title font-weight=\fontfamily{ptm}\selectfont\bfseries ,
          toc title before=\hspace*{0.5em},
          toc title after=\hspace{32.2em},
          toc after pagenumber=,
          toc number align=center,
          toc image=true,
          hypersetup linkcolor=black,
          custom={},
  }

%\@dottedtocline{<level>}{<indent>}{<numwidth>}{<title>}{<page>}

%
%\renewcommand*\l@chapter[2]{%
%  %#1 number and title  #2 page number
%   \ifnum \c@tocdepth >\m@ne
%    \addpenalty{-\@highpenalty}%
%    \vskip 1.0em \@plus\p@
%    \setlength\@tempdima{1.5em}%
%    \begingroup
%      \parindent \z@ \rightskip \@pnumwidth
%      \parfillskip -\@pnumwidth
%      \leavevmode \bfseries \color{thegray}
%      \advance\leftskip\@tempdima
%      \hskip -\leftskip
%      (#1)\nobreak\hfil \nobreak\hb@xt@\@pnumwidth{\hss#2\hspace*{3cm}}\par
%      \penalty\@highpenalty
%    \endgroup
%  \fi}


\hypersetup{%
  colorlinks=true,
  linkcolor=\hypersetuplinkcolor@cx}

%\@dottedtocline{<level>}{<indent>}{<numwidth>}{<title>}{<page>}: Macro
%to produce a table of contents line with the following parameters:
\renewcommand*\l@section{\@dottedtocline{1}{\tocsectionindent@cx}{\tocsectionnumberwidth@cx}}
\renewcommand*\l@subsection{\@dottedtocline{2}{\tocsubsectionindent@cx}{3.2em}}
\renewcommand*\l@subsubsection{\@dottedtocline{3}{\tocsubsubsectionindent@cx}{4.1em}}
\renewcommand*\l@paragraph{\@dottedtocline{4}{\tocparagraphindent@cx}{5em}}
\renewcommand*\l@subparagraph{\@dottedtocline{5}{\tocsubparagraphindent@cx}{6em}}
%% How to modify to achieve good boxing? \makebox
%\def\numberline#1{\makebox[2cm]{\textcolor{black}{#1}}}

\def\@dottedtocline#1#2#3#4#5{%
 \hsize12cm
 \ifnum #1>\c@tocdepth \else
 \vskip \z@ \@plus.2\p@
 {\leftskip #2\relax \rightskip \@tocrmarg \parfillskip -\rightskip
 \parindent #2\relax\@afterindenttrue
 \interlinepenalty\@M
 \leavevmode
 \@tempdima #3\relax
 \advance\leftskip \@tempdima \null\nobreak\hskip -\leftskip
%% Changes to definitions to accommodate keys
 % \vbox{\hsize10cm \raggedright\par \hangindent2.5em #4\par}\nobreak%title
{#4}\nobreak%title
 \leaders\hbox{$\m@th
%If a document uses fonts other than computer modern, the use of a dot from math
%can be very disturbing despite the fact that this might be the only place in a
%document that then uses computer modern. Therefore we surround the dot with
%an \hbox to escape to the surrounding text font.
 \mkern \@dotsep mu\hbox{.}\mkern \@dotsep
 mu$}\hfill
 \nobreak
 \hb@xt@\@pnumwidth{\hfil
\normalfont \normalcolor\color{\toctitlecolor@cx} #5}%page
 \par}%
\fi}

% We set a counter for every time we print a chapter 
% toc image.
% The images are saved by convention as tocblocnumber.
% this could also come from a key in fututre versions
\newcounter{img}
\setcounter{img}{1}


% We redefine the command to allow for key values and to enable
% custom designs to co-exist

\renewcommand*\l@chapter[2]{%
  % example of a toc custom design
\def\toccustom##1##2{% CAN BE MOVED TO ADD CONTENTS AT CHAPTER
 \par\vspace*{10pt}
        ##1\toctitleafter@cx##2\par\vspace*{-10pt}
        \hspace*{-2.1cm}\vbox to 0pt{\par\rule{.01pt}{22pt}\par
        \IfFileExists{./chapters/tocblock\theimg.JPG}{%
           \includegraphics[width=2.5cm]{tocblock\theimg}}\par\vspace*{\tocchapterafter@cx}
        }{\vspace*{10pt}}
 \stepcounter{img}
}

  %#1 number and title  #2 page number
  \ifnum \c@tocdepth >\m@ne
    \addpenalty{-\@highpenalty}%
    \vskip 1.0em \@plus\p@
    \setlength\@tempdima{1.5em}%
    \begingroup
      \parindent \z@ %\rightskip \@pnumwidth
      %\parfillskip -\@pnumwidth
%      \leavevmode \bfseries \color{thegray}
%      \advance\leftskip\@tempdima
%      \hskip -\leftskip
      \if@tocimage 
           \toccustom{#1}{#2}%
     \else
         #1\toctitleafter@cx#2\par\vspace*{-10pt}%
        \hspace*{-2.1cm}%
     \fi
     
     \penalty\@highpenalty
    \endgroup
\fi}



%%% HEADER AND FOOTER SETTINGS AND STYLES
\newif\if@headertoprule
\newif\if@headerbottomrule
\cxset{
   chaptermark name color/.store in=\chaptermarknamecolor@cx,
   chaptermark title color/.store in=\chaptermarktitlecolor@cx,
   chaptermark title before/.store in=\chaptermarktitlebefore@cx,
   chaptermark after number/.store in=\chaptermarkafternumber@cx,
   chaptermark name/.store in=\chaptermarkname@cx,
   chaptermark numbering/.is choice,
   leftmark before/.store in=\leftmarkbefore@cx,
   leftmark after/.store in=\leftmarkafter@cx,
   rightmark before/.store in=\rightmarkbefore@cx,
   rightmark after/.store in=\rightmarkafter@cx,
   chaptermark numbering/none/.code=\def\chaptermarknumber{},
   sectionmark name/.is choice,
   sectionmark name/none/.code=\def\sectionmarkname@cx{},
   sectionmark name custom/.code=\def\sectionmarkname@cx{#1},
   sectionmark number/.is choice,
   sectionmark number/none/.code=\def\sectionmarknumber@cx{},
   sectionmark after number/.store in=\sectionmarkafternumber@cx,
   sectionmark name color/.store in=\sectionmarkcolor@cx,
   sectionmark title font/.store in=\sectionmarktitlefont@cx,
   sectionmark title color/.store in=\sectionmarktitlecolor@cx,
   sectionmark before title/.store in=\sectionmarkbeforetitle@cx,
   sectionmark after title/.store in=\sectionmarkaftertitle@cx,
   header offset even/.store in=\headeroffseteven@cx,
   header offset odd/.store in=\headeroffsetodd@cx,
  %
   header top rule/.is if=@headertoprule,
   header bottom rule/.is if=@headerbottomrule,
   author block=false,
}

%% Set defaults
\cxset{ header offset even=0pt,
          header offset odd=0pt,
          rightmark before=,
          rightmark after=,
}

\cxset{pagestyle/.code=\pagestyle{#1}}

\cxset{headings ruled-01/.style={pagestyle=headings,
          header style=headings,
          chaptermark name color=theblue,
          chaptermark after number={\thinspace:\space },
          chaptermark name=,%\@chapapp,
          chaptermark title color=black!80,
          leftmark before=\thepage\hfill, 
          leftmark after=,
          sectionmark name color=theblue,
          sectionmark title color=black!80,
          header offset even=0pt,
          header offset odd=0pt,
          header top rule=false,
          header bottom rule=true}}

\cxset{headings ruled-02/.style={pagestyle=headings,
          header style=headings,
          chaptermark name color=theblue,
          chaptermark after number=,
          chaptermark name=,%\@chapapp,
          chaptermark numbering=none,
          chaptermark title color=black!80,
          sectionmark name=none,
          sectionmark number=none,
          leftmark before=,
          leftmark after=\qquad\quad\thepage,
          rightmark before=\thepage,
          rightmark after=\hfill\hfill,
          sectionmark name color=theblue,
          sectionmark title color=black!80,
          sectionmark after title=,
          sectionmark after number=\qquad,
          header top rule=false,
          header bottom rule=true,
          header offset even=1.5cm,
          header offset odd=-1.5cm,
          header bottom rule=false}}

\cxset{%
          header style=headings,
          chaptermark name color=black,
          chaptermark after number={\thinspace:\space },
          chaptermark title color=black!80,
          sectionmark name color=black,
          sectionmark title color=black!80,
          header top rule=false,
          header bottom rule=true}  


\cxset{headings boxedpagenumber/.style={
          pagestyle=headings,
          header style=headings,
% Chaptermarks 
          chaptermark name color=black,
          chaptermark after number=,
          chaptermark name={\bfseries SHORT BOOK TITLE},
          chaptermark numbering=none,
          chaptermark title color=black!80,
          chaptermark title before=\@gobble,
% Leftmarks
          leftmark before=\colorbox{thegray!50}{\thepage\quad}\quad, %even pages
          leftmark after=\hfill\hfill,
% Right marks influenced by chapter name?
          rightmark before=\colorbox{thegray!50}{\thepage\quad}\quad, %odd pages
          rightmark after=\hfill\hfill,
% Section marks
          sectionmark name custom=\chaptertitle@cx,
          sectionmark number=none,
          sectionmark name color=black,
          sectionmark title color=black!80,
          sectionmark before title=\@gobble, % we do not need the section title
          sectionmark after title=\hfill\hfill,
          sectionmark after number=,
%  rules we remove or inherit
%       header top rule=false,
          header bottom rule=true,
          header offset even=-1.3cm,
          header offset odd=-1.3cm,
          }}

%% QUANTUM FRONTIER
\cxset{headings titlestyle/.style={
          pagestyle=headings,
          header style=headings,
          headings boxedpagenumber,
% Chaptermarks 
          chaptermark name color=black,
          chaptermark after number=,
          chaptermark name={\bfseries The Quantum Frontier},
          chaptermark numbering=none,
          chaptermark title color=black!80,
          chaptermark title before=\@gobble,
% Leftmarks
          leftmark before=\thepage\quad, %even pages
          leftmark after=\hfill\hfill,
% Right marks influenced by chapter name?
          rightmark before=\hfill\hfill, %odd pages
          rightmark after=\thepage,
% Section marks
          sectionmark name custom=\chaptertitle@cx,
          sectionmark number=none,
          sectionmark name color=black,
          sectionmark title color=black!80,
          sectionmark before title=\@gobble, % we do not need the section title
          sectionmark after title=\quad,
          sectionmark after number=,
%  rules we remove or inherit
          header top rule=false,
          header bottom rule=false,
          header offset even=0pt,
          header offset odd=0pt,
          }}

%% EVOLUTION OF THE INSECTS
\cxset{headings style58/.style={
 %         pagestyle=headings,
%          header style=headings,
          headings titlestyle,
% Chaptermarks 
          %chaptermark name color=black,
          %chaptermark after number=,
          chaptermark name={\bfseries EVOLUTION OF THE INSECTS},
          %chaptermark numbering=none,
          %chaptermark title color=black!80,
          %chaptermark title before=\@gobble,
% Leftmarks
          leftmark before=\thepage\quad\hfill\hfill, %even pages
          leftmark after=,
% Right marks influenced by chapter name?
          rightmark before=, %odd pages
          rightmark after=\hfill\hfill\thepage,
% Section marks
%          sectionmark name custom=\chaptertitle@cx,
%          sectionmark number=none,
%          sectionmark name color=black,
%          sectionmark title color=black!80,
%          sectionmark before title=\@gobble, % we do not need the section title
%          sectionmark after title=\quad,
%          sectionmark after number=,
%%  rules we remove or inherit
%          header top rule=false,
%          header bottom rule=false,
%          header offset even=0pt,
%          header offset odd=0pt,
          }}



%% HEADERS AND FOOTERS
\if@twoside
  \def\ps@headings{%
      \let\@oddfoot\@empty
      \def\@oddfoot{\rule{\textwidth}{0.4pt}}
      \let\@evenfoot\@empty
      \def\@evenhead{\parbox{\textwidth}{%
                                   \leavevmode
                                   \if@headertoprule\rule{\textwidth}{0.4pt}%
                                       \vskip2pt plus1pt minus1pt\fi
%typesetter
                                     \hskip\headeroffseteven@cx\hbox to \textwidth{%
                                           \leftmarkbefore@cx
                                           \leftmark
                                           \leftmarkafter@cx
                                     }%
                                     \if@headerbottomrule\vskip-7pt plus1pt minus1pt
                                    \rule{\textwidth}{0.4pt}\fi%
          }% end parbox
       }%
%% Defines the odd head
      \def\@oddhead{
         \parbox{\textwidth}{%
                                   \leavevmode
                                   \if@headertoprule\rule{\textwidth}{0.4pt}%
                                       \vskip2pt plus1pt minus1pt\fi
%typesetter
                                     \hskip\headeroffsetodd@cx\hbox to \textwidth{%
                                           \rightmarkbefore@cx
                                           \rightmark
                                           \rightmarkafter@cx
                                     }%
                                     \if@headerbottomrule\vskip-7pt plus1pt minus1pt
                                    \rule{\textwidth}{0.4pt}\fi%
          }% end parbox
      }%
      \let\@mkboth\markboth
 % chaptermark called by chapter and also by table of contents etc. This is essentially a 
%  leftmark
\def\chaptermark##1{%
     \gdef\chaptertitle@cx{##1}%
      \markboth {%
       \ifnum \c@secnumdepth >\m@ne
          \if@mainmatter%
              \color{\chaptermarknamecolor@cx}%
              \MakeUppercase{\chaptermarkname@cx\ }%
              \chaptermarknumber%
              \chaptermarkafternumber@cx%
          \fi
        \fi
        \color{\chaptermarktitlecolor@cx}%
       % \hfill%
        \MakeUppercase{\chaptermarktitlebefore@cx{##1}}}{}%
}%end chaptermark
% section
  \def\sectionmark##1{%
      \markright {%
        \ifnum \c@secnumdepth >\z@
           {\bfseries\textcolor{\sectionmarkcolor@cx}{\sectionmarkname@cx\sectionmarknumber@cx\sectionmarkafternumber@cx}%
        } %
  \fi
         \color{\sectionmarktitlecolor@cx}\MakeUppercase{\normalfont\sffamily \sectionmarkbeforetitle@cx{##1}\sectionmarkaftertitle@cx}}}}%
\else
  \def\ps@headings{%
    \let\@oddfoot\@empty
    \def\@oddhead{{\slshape\rightmark}\hfil\thepage}%
    \let\@mkboth\markboth
    \def\chaptermark##1{%
      \markright {%
        \ifnum \c@secnumdepth >\m@ne
          \if@mainmatter
            \@chapapp\ \thechapter... \ %
          \fi
        \fi
        ##1}}}
\fi
\def\ps@myheadings{%
    \let\@oddfoot\@empty\let\@evenfoot\@empty
    \def\@evenhead{\thepage\hfil\slshape\leftmark}%
    \def\@oddhead{{\slshape\rightmark}\hfil\thepage}%
    \let\@mkboth\@gobbletwo
    \let\chaptermark\@gobble
    \let\sectionmark\@gobble
 }

% We define some special headers for convenience. This one is for double
% images for the oxford and tufte-styles.

\def\ps@caption{%
    \let\@oddfoot\@empty\let\@evenfoot\@empty
    \def\@evenhead{\begin{picture}(0,0)
           \put(-150,-200){A funny caption}
         \end{picture}}%
    \let\@oddhead\@evenhead
    \let\@mkboth\@gobbletwo
    \let\chaptermark\@gobble
    \let\sectionmark\@gobble
 }

%%%%%%%%%%%%%%%%%%%%%%%%%%% END HEADERS


\@specialfalse
\renewcommand\tableofcontents{%
\cxset{chapter before=\vspace*{-50pt},
       chapter toc=none, 
       numbering=none, 
       chapter opening=right, title font-color=\color{black},
       title font-size=\Large, title font-shape=\upshape,
       name={},
       title before=\hspace*{-2.1cm}\parbox{3.3cm}{\includegraphics{tocblock01}},
}
    \if@twocolumn
      \@restonecoltrue\onecolumn
    \else
      \@restonecolfalse
    \fi
    % \pagestyle{empty}
    \chapter{\contentsname}%had UPPER CASE
     \@starttoc{toc}%
    \if@restonecol\twocolumn\fi
    \cxset{chapter toc=true, toc image=false}
    }


%% PRODUCES THE COVER PAGE OF THE DOCUMENTATION
\newcommand\coverpage[3]{%
\vbox{%
  \vspace*{-2\baselineskip}
  \hskip-3.95cm\includegraphics[width=\paperwidth]{#1}\par
 % \putimage[width=17cm,offsetx=0cm, offsety=3\baselineskip, border=0pt,padding=0pt]{cockfight}\par
  \vspace*{3\baselineskip}
   \hbox to \hsize{\Huge \hfill\hfill{\MakeUppercase{\bfseries \textsf{ A NEW LOOK AT  }}}}
   \vspace*{0.3cm}
   \hbox to \hsize{\Huge \hfill\hfill{\MakeUppercase{\bfseries \textsf{LATEX  book design}}}}
  \vspace*{2\baselineskip}
   \hbox to \hsize{\huge \hfill\hfill\textsf{\hbox{#2}}}
    \vspace*{1.35cm}
   \hbox to \hsize{\huge \hfill\hfill\textsf{\hbox{#3}}}
}
}

%% We define a command for the second page of the documentation
\newcommand\secondpage{\clearpage\null\vfill\vfill
\thispagestyle{empty}
\begin{minipage}[b]{0.9\textwidth}
\includegraphics[width=3cm]{hine02}\par
\raggedright
\textit{Cover image: }
The cover image shows Jo Bodeon, a back-roper in the mule room at Chace Cotton Mill. Burlington, Vermont. This and other similar images in this book were taken by Lewis W. Hine, in the period between 1908-1912. These images as well as social campaigns by many including Hine, helped to formulate America's anti-child labour laws.
\end{minipage}\par
\vspace*{\baselineskip}
\begin{minipage}[b]{0.9\textwidth}
\raggedright
\setlength{\parskip}{0.5\baselineskip}
Copyright \copyright 2012  Dr Yiannis Lazarides\par
Permission is granted to copy, distribute and\slash or modify this document under the terms of the GNU Free Documentation License, version 1.2, with no invariant sections, no front-cover texts, and no back-cover texts.\par
A copy of the license is included in the appendix.\par
This document is distributed in the hope that it will be useful, but without any warranty; without even the implied warranty of merchantability or fitness for a particular purpose.
\end{minipage}
\vspace*{2\baselineskip}
\clearpage
}

%% Some special styles
%     \begin{macrocode}
\IfFileExists{changepage.sty}{%
  \PassOptionsToPackage{strict}{changepage}
  \RequirePackage{changepage}
  }{}
\IfFileExists{rotating.sty}{\RequirePackage{rotating}}{}
%    \end{macrocode}
%
% \begin{macro}{\even@samplepage}

% \begin{macro}{\odd@samplepage}
%    \begin{macrocode}
\def\even@samplepage{%
 \begin{picture}(0,0)
   \put(\Xeven,\Yeven){\turnbox{90}{\Huge \textcolor{\watermark@textcolor}{\watermark@text}}}
\end{picture}
}
%% Define a macro to print SAMPLE PAGE IN THE MARGIN
\def\odd@samplepage{%
 \begin{picture}(0,0)
   \put(\Xodd,\Yodd){\turnbox{90}{\Huge \textcolor{\watermark@textcolor}{\watermark@text}}}
 \end{picture}
}
%    \end{macrocode}
% \begin{macro}{watermarktext}
%  Define the watermark words
%    \begin{macrocode}
\def\watermarktext#1{\gdef\watermark@text{\fontfamily{phv}\selectfont#1}}
\def\watermarktextcolor#1{\gdef\watermark@textcolor{#1}}
\watermarktext{SAMPLE PAGE}
\watermarktextcolor{purple}
%    \end{macrocode}
% \end{macro}
%    \begin{macrocode}
%% redefine LaTeX's plain as myplain for headings
\def\ps@samplepage{\let\@mkboth\@gobbletwo
 \let\@oddhead\odd@samplepage\def\@oddfoot{\reset@font\hfil\thepage}
 \let\@evenhead\even@samplepage\def\@evenfoot{\reset@font\thepage\hfil}}
%%
%
%% We define two macros to position the watermark on the page
\def\Xodd{500}
\def\Xeven{-70}\def\Yeven{-810}
\def\Yeven{-\expandafter\strip@pt\textheight}
\let\Yodd\Yeven


\cxset{blank page text/.store in=\blankpagetext@cx{#1}}

\def\cleardoublepage{\clearpage\if@twoside\ifodd\c@page\else
  \hbox{}
  \vspace*{\fill}
  \begin{center}
     \blankpagetext@cx      
  \end{center}
  \vspace{\fill}
  \thispagestyle{empty}
  \newpage
  \if@twocolumn\hbox{}\newpage\fi\fi\fi}

\parindent1.5em

%% Examples for documentation
\newcounter{texexp}[chapter]
\@addtoreset{c@texexp}{c@chapter}
\def\thetexexp{\thesection.\arabic{texexp}}
\tcbset{
texexp/.style={%
   fonttitle=\small\sffamily\bfseries, fontupper=\small, fontlower=\small},
  example/.code 2 args={\refstepcounter{texexp}\label{#2}}%
  \pgfkeysalso{texexp,title={Example \thetexexp\ #1}},
}


\newenvironment{texexp}[1]{\tcblisting{texexp,#1}}{\endtcblisting}
\newenvironment{example}[3][]{\tcblisting{example={#2}{#3},#1}}{\endtcblisting}
%%rename to avoid clashes with other classes that use it

\newenvironment{texexample}[3][]{\tcblisting{example={#2}{#3},#1}}{\endtcblisting}


% Define a macro to hold the name of this package.
\def\athena{\textcolor{thered}{athena}}
% 
% Be nice to hackers define a boolean to know if the package was loaded
\newif\if@athena
\@athenatrue

% STOLEN FROM TUFTE
% \RaggedRight allows hyphenation

\RequirePackage{ragged2e}
\setlength{\RaggedRightRightskip}{\z@ plus 0.08\hsize}
\setlength{\RaggedRightParindent}{1pc}

% Paragraph indentation and separation for normal text
\newcommand{\@tufte@reset@par}{%
  \setlength{\RaggedRightParindent}{1.0pc}%
  \setlength{\parindent}{1pc}%
  \setlength{\parskip}{0pt}%
}
\@tufte@reset@par

% Paragraph indentation and separation for marginal text
\newcommand{\@tufte@margin@par}{%
  \setlength{\RaggedRightParindent}{0.5pc}%
  \setlength{\parindent}{0.5pc}%
  \setlength{\parskip}{0pt}%
}
% Use Donald Arseneau's improved float parameters for the documentation and 
% to develop sensible values.
\renewcommand{\topfraction}{.5}
\renewcommand{\bottomfraction}{.7}
\renewcommand{\textfraction}{.04}
\renewcommand{\floatpagefraction}{.92} % have a high one don't encourage it
\renewcommand{\dbltopfraction}{.66}
\renewcommand{\dblfloatpagefraction}{.66}
\setcounter{topnumber}{9}
\setcounter{bottomnumber}{9}
\setcounter{totalnumber}{20}
\setcounter{dbltopnumber}{9}

% Some people might prefer setting the author fields as macros
\newcommand\addauthors[1]{%
   \cxset{author names=#1}
}

%%%%%%%%%%%%% CHAPTERS %%%%%%%%%%%%%%%%%%%%%%%%%%%%%%
\cxset{custom/.code=\gdef\customdesign@cx{\csname#1\endcsname}\@specialtrue,
       fill/.store in=\fill@cx}
%\cxset{stefan/.style={fill=purple, title font-color=\color{white},
%          custom=tikzspecials}}


\cxset{steward/.style={
  offsety/.store in=\soffsety,
  image/.store in=\image@cx,
  texti/.store in=\texti@cx,
  textii/.store in=\textii@cx,
  header style=empty,
}}


\newcommand\tikzspecial[2][]{%
     \clearpage

     \begin{tikzpicture}[remember picture,overlay]
     % Main shading block
     \node [xshift=5cm,yshift=-\paperheight] at (current page.north west)
        [text width=0.98\textwidth,text height=\paperheight, fill=thecream!30,above right]
        {};
      \node [xshift=6.5cm,yshift=-1.5cm-\soffsety] at (current page.north west)
         [text width=0.7\textwidth,below right]{\sffamily \bfseries \huge\raggedright #2\par};

        \node [xshift=3cm,yshift=-1.5cm] at (current page.north west)
      [text width=3cm,align=center,minimum height=2.5cm, fill=blue,below right]
	{$$\text{\HHUGE\bfseries\sffamily\color{white}\thechapter}$$
	\par\vspace*{3pt}
	};

	\node [xshift=-0.2cm,yshift=-21.5cm] at (current page.north west)
	[text width=3cm,above right]%
	{\includegraphics[width=1.0\paperwidth]{./chapters/\image@cx}};
	% second box left
	\node [xshift=3cm,yshift=-19.5cm] at (current page.north west)
	[text width=9cm,minimum height=2.5cm,inner sep=0.5em, fill=blue,below right]
	{\color{white}
 	 \bfseries\sffamily \texti@cx
	};
	% Last block
	\node [xshift=6.5cm,yshift=-26cm] at (current page.north west)
	[text width=12cm,above right]
	{\textii@cx};
	\end{tikzpicture}
\par
\thispagestyle{empty}
\clearpage
}%end tikzspecial

\cxset{steward,
  chapter toc=true,
  numbering=arabic,
  custom=tikzspecial,
  offsety=0cm,
  image=hine03,
  texti={Type some text here by setting texti},
  textii={Type some text here by setting textii},
 }


%% EDDOUARD MANET

\cxset{manet/.style={
 chapter opening=anywhere,
 chapter toc=true,
 toc image=false,
 name={},
 numbering=none,
 number font-size=,
 number font-family=,
 number font-weight=,
 number before={\vspace*{-2.5cm}},
 number dot={},
 number after={},
 number position=leftname,
 chapter font-family=,
 chapter font-weight=,
 chapter font-size=,
 chapter before={},
 chapter after={},
 chapter color={black!90},
 number color=\color{purple},
 title beforeskip={},
 title afterskip={},
 title before={\hskip-2.3cm\includegraphics[width=1.25\textwidth]{./chapters/manet}\par
    \par\hfill\hfill{\tiny\bfseries Manet's  \textit{The Barmaid.}}\\
    \par
    \vspace*{\baselineskip}
    \par\hfill},
 title after={\hfill\hfill},
 title font-family=\sffamily,
 title font-color=\color{black!80},
 title font-weight=\bfseries,
 title font-size=\LARGE,
 header style=plain}}

\def\topimage#1{\cxset{title before={\vspace*{\headsep}\hspace*{\dimexpr(-\marginparwidth-2.1cm-\marginparsep)}\includegraphics[width=\paperwidth]{./chapters/#1}\par
\vspace*{\baselineskip}\par}}}
% need to reset 
\cxset{manet, toc image=true}
\begin{document}

\frontmatter
\pagestyle{empty}
%\coverpage{hine02}{Yiannis Lazarides}{Published by Camel Press}savinggrace,vespa

\coverpage{bird-brain}{Yiannis Lazarides}{Published by Camel Press}
\secondpage
\thispagestyle{samplepage}
%% temporary titles
% command to provide stretchy vertical space in proportion
\newcommand\nbvspace[1][1]{\vspace*{\stretch{#1}}}
% allow some slack to avoid under/overfull boxes
\newcommand\nbstretchyspace{\spaceskip0.5em plus 0.25em minus 0.25em}
% To improve spacing on titlepages
\newcommand{\nbtitlestretch}{\spaceskip0.6em}
\pagestyle{empty}
\begin{center}
\bfseries
\nbvspace[1]
\Huge
{\nbtitlestretch\huge
A NEW LOOK AT\\ LATEX BOOK DESIGN}

\nbvspace[1]
\normalsize

TO WHICH IS ADDED MANY USEFUL MACROS\\
AND THE \textbf{CHAPTERX} PACKAGE SO THAT\\
YOU CAN MAKE BEAUTIFUL BOOKS\\
%YOU CAN CODE LIKE A HAWK
\nbvspace[1]
\small BY\\
\Large DR YIANNIS LAZARIDES\\[0.5em]
%\footnotesize AUTHOR OF ``A WORKING ALGEBRA,'' ``WIRELESS TELEGRAPHY,\\
%ITS HISTORY, THEORY AND PRACTICE,'' ETC., ETC.

\nbvspace[2]

\includegraphics[width=1.5in]{pic37}
\nbvspace[3]
\normalsize

DOHA\\
\large
PUBLISHED IN THE WILD
\nbvspace[1]
\end{center}

% Must move from here
\cxset{blank page text=\epigraph{We all agree that your theory is crazy. 
          But is it crazy enough?}{Niels Bohr}}

%%%%%%%%%%%  MAIN MATTER %%%%%%%%%%%%%%%%%%%%%%%%%%%%%%
\mainmatter
\pagestyle{plain}
%%    \begin{macro}
%%    This macro is a helper macro to set the paper height and width
%%    we also save the paper name in its own macro.
%%    \begin{macrocode}
%\gdef\setpapersize@cx#1#2#3{%
%   \gdef\papername{#1}
%   \setlength\paperheight{#2}
%   \setlength\paperwidth{#3}
%   % headheight is common to all so we set it here
%   \setlength\headheight{12\p@}
%  % if pdf we need to set the pageheight and pagewidth
%  \global\pdfpageheight=#2
%  \global\pdfpagewidth=#3
%}
%%    \end{macrocode}
%%    \end{macro}
%%
%%    \begin{macro}
%%    \begin{macrocode}
%\def\setparams@cx#1#2#3{%
%    \def\X{#3}\def\XX{11pt}
%    % 11pt font set it as well
%    \ifx\X\XX
%          \@setfontsize\normalsize\@xipt{13.2}\selectfont%
%          \abovedisplayskip 13.2\p@ \@plus 3\p@ \@minus 3\p@
%          \abovedisplayshortskip \z@ \@plus 3\p@
%           \belowdisplayshortskip 6.6\p@ \@plus 3\p@ \@minus 3\p@
%    \else
%       \def\XX{12pt}
%        \ifx\X\XX
%           \@setfontsize\normalsize\@xiipt\@xivpt\selectfont
%           \abovedisplayskip 14.4\p@ \@plus 3\p@ \@minus 3\p@
%           \abovedisplayshortskip \z@ \@plus 3\p@
%          \belowdisplayshortskip 7.2\p@ \@plus 3\p@ \@minus 3\p@
%       \fi
%    \fi
%    \setlength\headsep{#3}
%    \setlength\footskip{#2}
%    \setlength\topskip{#3}
%    \setlength\maxdepth{0.5\topskip} % need to check
% }
%%    \end{macrocode}
%%    \end{macro}
%%
%%    We now set keys for all the paper sizes  
%\cxset{
%        a4paper/.code=\setpapersize@cx{a4paper}{297mm}{210mm},
%        a5paper/.code=\setpapersize@cx{a5paper}{210mm}{148mm},
%        a6paper/.code=\setpapersize@cx{a6paper}{105mm}{148},
%        b5paper/.code=\setpapersize@cx{b5paper}{250mm}{176mm},
%        letterpaper/.code=\setpapersize@cx{letterpaper}{11n}{8.5in},
%        legalpaper/.code=\setpapersize@cx{legalpaper}{14in}{8.5in},
%        executivepaper/.code=\setpapersize@cx{executivepaper}{10.5in}{7.25in},
%}
%%    the classical dimesions were obtained from the Octavo class
%%    we use mm or in depending on the type of paper standard
%\cxset{foolscap/.code=\setpapersize@cx{foolscap}{171mm}{108mm},
%          crown/.code=\setpapersize@cx{crown}{191mm}{127mm},
%          post/.code=\setpapersize@cx{post}{194mm}{122mm},
%          large post/.code=\setpapersize@cx{large post}{210mm}{137mm},
%          demy/.code=\setpapersize@cx{demy}{222mm}{143mm},
%          medium/.code=\setpapersize@cx{medium}{229mm}{146mm},
%          royal/.code =  \setpapersize@cx{royal}{254mm}{159mm},
%          superroyal/.code=\setpapersize@cx{superroyal}{267mm}{171mm}, 
%          imperial/.code=  \setpapersize@cx{imperial}{279mm}{191mm}}
%%   Lulu paper sizes
%%   http://wepod.wordpress.com/lulu-specs/
%%Manuscript Templates
%%6″ x 9″  US TRADE
%%(15.24cm x 22.86cm)
%%8.5″ x 11″
%%(21.59cm x 27.94cm)
%%Comic, 6.625″ x 10.25″
%%(16.827cm x 26.03cm)
%%Landscape, 9″ x 7″
%%(22.86cm x 17.78cm)
%%Square, 7.5″ x 7.5″
%%(19.05cm x 19.05cm)
%%Pocket Size, 4.25″ x 6.875″
%%(10.8cm x 17.46cm)
%%Royal, 15.6cm x 23.4cm
%%(6.14″ x 9.21″)
%%Crown Quarto, 18.9cm x 24.6cm
%%(7.44″ x 9.68″)
%%A4, 21.0cm x 29.7cm
%%(8.27″ x 11.69″)
%%   Set the parameters that depend on font-sizes
%\cxset{
%        lulu pocketbook/.code=\setpapersize@cx{lulu pocket book}{6.87in}{4.25in},
%	lulu digest/.code=\setpapersize@cx{lulu digest}{8.5in}{5.5in},
%	lulu us trade/.code=\setpapersize@cx{lulu us trade}{9in}{6in},
%	lulu royal/.code=\setpapersize@cx{lulu royal}{9.21in}{6.13in},
%	lulu comic/.code=\setpapersize@cx{lulu comic}{10.25in}{6.625in},
%	lulu crown quarto/.code=\setpapersize@cx{lulu crown}{9.68in}{7.44in},
%	lulu small square/.code=\setpapersize@cx{lulu small}{7.5in}{7.5in},
%	lulu square/.code=\setpapersize@cx{lulu large}{8.5in}{8.5in},
%	lulu landscape/.code=\setpapersize@cx{lulu landscape}{7in}{9in},
%	%lulu large landscape/.code=\setpapersize@cx{lulu large landscape}{}{},
%}
%
%\cxset{
%         10pt/.code=\setparams@cx{6pt}{25pt}{10pt},
%         11pt/.code=\setparams@cx{7pt}{27.5pt}{11pt},
%         12pt/.code=\setparams@cx{8pt}{30pt}{12pt} \@setfontsize\normalsize\@xiipt\@xivpt\selectfont,
%}%
%
%%   we need to set a default size before we determine the
%%   rest of the parameters.
% \cxset{a4paper,10pt}
%
%% does not seem to work
%%\@setfontsize\normalsize\@xiipt\@xivpt\normalsize
%
%%    set a default top margin first
%\def\topmarginauto{%
%\setlength{\topmargin}{0.1\paperheight}
%    \addtolength{\topmargin}{-\headheight}
%    \addtolength{\topmargin}{-\headsep}
%    \addtolength{\topmargin}{-1in}
%}
%
%\topmarginauto
%
%\cxset{topmargin/.code=\setlength{\topmargin}{#1}}
%\cxset{topmargin latex/.code=\topmarginauto}
%\cxset{topmargin latex}
%
%%   \section{Calculation of textwidth}
%%    The calculation of textwidth will depend on the strategy employed to calculate it.
%% \begin{macro}{\textwidth}
%%    Define the width of the text block to 0.7 of the page width, and make
%%    calculations a little easier by adjusting the calculated width to a 
%%    whole number of points.
%%    \begin{macrocode}
%\iffalse
%\setlength{\textwidth}{0.7\paperwidth}
%    \@settopoint\textwidth
%%    \end{macrocode}
%% \end{macro}
%%
%% \begin{macro}{\textheight}
%%    The height of the text block itself is set to 0.7 times the page height. 
%%    This amount is then adjusted to ensure that a whole number of lines makes 
%%    up the text block, and does so exactly.
%%    \begin{macrocode}
%\setlength\@tempdima{0.7\paperheight}
%%    \end{macrocode}
%%    take away the first line, which is a bit shorter than the |\baselineskip|,
%%    \begin{macrocode}
%    \addtolength\@tempdima{-\topskip}
%%    \end{macrocode}
%%    this length may be very close, but just a little too small to accommodate 
%%    one more line, so we add a small amount,
%%    \begin{macrocode}
%    \addtolength\@tempdima{5\p@}
%%    \end{macrocode}
%%    and calculate the number of lines in this length,
%%    \begin{macrocode}
%    \divide\@tempdima\baselineskip
%    \@tempcnta=\@tempdima
%%    \end{macrocode}
%%    The correct textheight comes to the number of lines just calculated, 
%%    multiplied by the height of text lines, |\baselineskip|, and with the 
%%    addition of the |\topskip| we took away initially.
%%    \begin{macrocode}
%    \setlength\textheight{\@tempcnta\baselineskip}
%    \addtolength\textheight{\topskip}
%%    \end{macrocode}
%% \end{macro}
%%
%% \subsubsection{Margin dimensions}
%%     Now that we have set the size of the text block, the amount of space
%%     available for margins is set as well. The remaining white space is divided
%%     in a 1:2 ratio, hence the proportions between margins and text become 1:7:2.
%%
%% \begin{macro}{\evensidemargin}
%% \begin{macro}{\oddsidemargin}
%%    Since we are typesetting books, both even and odd side margins have to be
%%    set.
%%    \begin{macrocode}
%\setlength{\evensidemargin}{0.2\paperwidth}
%\addtolength{\evensidemargin}{-1in}
%\setlength{\oddsidemargin}{0.1\paperwidth}
%\addtolength{\oddsidemargin}{-1in}
%%    \end{macrocode}
%
%\fi
%%% end of octavo algorithm and calculations
%
%%    Define an innermargin to enable easy drawing of parameters
%\newlength\innermargin
%\newlength\lefttrim
%\newlength\bottomtrim
%
%%    The stockheight and stockwidth are used when the paper is to be trimmed
%%    they default to the dimensions for paper width and paper height
%\@ifundefined{stockheight}{\global\newlength\stockheight}{}
%\@ifundefined{stockwidth}{\global\newlength\stockwidth}{}
%\ifdim\stockheight=0pt\addtolength\stockheight{\paperheight}\fi
%   \addtolength\stockheight{0mm}
%%
%\ifdim\stockwidth=0pt\addtolength\stockwidth{\paperwidth}\fi
%   \addtolength\stockwidth{0mm}
%%
%%   We set all the trims to zero to start with.
%\setlength\lefttrim{0mm}
%\setlength\bottomtrim{0mm}
%\setlength\trimtop{0mm}
%\setlength\trimedge{0mm}
%%
%%   
%
%
%%% This is a sidenote without the footnote mark
%%\newcommand\marginnote[2][0pt]{%
%% % \let\cite\@tufte@infootnote@cite%   use the in-sidenote \cite command
%%  %\gdef\@tufte@citations{}%           clear out any old citations
%%  \@tufte@margin@par%                 use parindent and parskip settings for marginal text
%%  \marginpar{\hbox{}\vspace*{#1}\marginparfont@cx\marginparjustification@cx\vspace*{-1\baselineskip}\noindent #2}%
%%  \@tufte@reset@par%                  use parindent and parskip settings for body text
%%  %\@tufte@print@citations%            print any citations
%%  %\let\cite\@tufte@normal@cite%       go back to using normal in-text \cite command
%%}
%
%% This macro has been adapted from the layouts package, it sets the units to be printed
%% in the diagrams.
%\newcommand{\printinunitsof@cx}[1]{%
%  \def\l@yunitperpt{1.0}\def\l@yunits{pt}%
%  \def\l@yta{#1}\def\l@ytb{pt}%
%  \ifx \l@yta\l@ytb
%    \def\l@yunitperpt{1.0}\def\l@yunits{pt}%
%  \else
%    \def\l@ytb{pc}%
%    \ifx \l@yta\l@ytb
%      \def\l@yunitperpt{0.083333}\def\l@yunits{pc}%
%    \else
%      \def\l@ytb{in}%
%      \ifx \l@yta\l@ytb
%        \def\l@yunitperpt{0.013837}\def\l@yunits{in}%
%      \else
%        \def\l@ytb{mm}%
%        \ifx \l@yta\l@ytb
%          \def\l@yunitperpt{0.351459}\def\l@yunits{mm}%
%        \else
%          \def\l@ytb{cm}%
%          \ifx \l@yta\l@ytb
%            \def\l@yunitperpt{0.0351459}\def\l@yunits{cm}%
%          \else
%            \def\l@ytb{bp}%
%            \ifx \l@yta\l@ytb
%              \def\l@yunitperpt{0.996264}\def\l@yunits{bp}%
%            \else
%              \def\l@ytb{dd}%
%              \ifx \l@yta\l@ytb
%                \def\l@yunitperpt{0.9345718}\def\l@yunits{dd}%
%              \else
%                \def\l@ytb{cc}%
%                \ifx \l@yta\l@ytb
%                  \def\l@yunitperpt{0.0778809}\def\l@yunits{cc}%
%%                \else
%%                  \def\l@ytb{PT}%
%%                  \ifx \l@yta\l@ytb
%%                    \def\l@yunitperpt{1.0}\def\l@yunits{PT}% gives problems with pgfmathparse
%%                  \fi
%                \fi
%              \fi
%            \fi
%          \fi
%        \fi
%      \fi
%    \fi
%  \fi
%}
%
%% Define keys to set it
%\cxset{geometry units/.code=\printinunitsof@cx{#1}}
%\cxset{geometry units=pt}
%
%% #1 value in pts
%% default in mm sorry USA.
%% rounding in 1 decimal place
%\def\convert@cx#1{%
%   \pgfmathparse{#1*\l@yunitperpt}
%   %\pgfmathround{\pgfmathresult}
%   \pgfmathresult\thinspace\l@yunits
%}
%
%% Layout related macros to go to separate style file
%\def\aspectratio{\pgfmathparse{\paperheight/\paperwidth} \pgfmathresult}
%
%
%
%
%% Set to true to draw an oddside page. Initially set to false.
%\newcommand\layoutscale@cx{0.4}
%
%\newif\ifoddpagelayout@cx
%   \oddpagelayout@cxtrue
%
%% Set true to draw marginpars on a page
%\newif\ifdrawmarginpars
%   \drawmarginparstrue
%
%% This draws a two page spread
%\newlength\bindingcorrection
%\newlength\oneninth
%\newlength\sixninths
%\setlength\oneninth{\dimexpr(\paperwidth/9)}
%\setlength\sixninths{\dimexpr(\paperwidth*6/9)}
%\let\trytextwidth\sixninths
%
%
%\newcommand{\alphabet}{\normalfont\selectfont\raggedleft abcdefghijklmnopqrstuvwxyz}%82
%
%
%
%\newcommand\charactersperline{%
%  \settowidth{\@tempdima}{\alphabet}
%  \pgfmathparse{\textwidth/\@tempdima*26}
% \pgfmathprintnumber{\pgfmathresult}
%}
%
%\newcommand\alphabetsperline{
%  \settowidth{\@tempdima}{\alphabet}
%  \pgfmathparse{\textwidth/\@tempdima}
%  \pgfmathresult
%}
%
%\newlength\alphlength
%\newcommand\alphabetlength{%
%  \settowidth{\alphlength}{\alphabet}
%  \pgfmathparse{\alphlength}
%  \pgfmathprintnumber{\pgfmathresult}pt
%}
%
%% We need to use the fp package to calculate the ratios, as PGF has problems with large 
%% dimensions or I am making an error
%\newcommand\textarearatio{%
%    \FPmul{\result}{\strip@pt\textwidth}{\strip@pt\textheight}
%    \FPmul{\resulti}{\strip@pt\paperwidth}{\strip@pt\paperheight}
%    \FPdiv{\resultii}{\result}{\resulti}
%    \pgfmathprintnumber{\resultii}
%}
%
%% Calculate the ratio textheight/paperheight
%\newcommand\textheightratio{%
%    \FPdiv{\result}{\strip@pt\textheight}{\strip@pt\paperheight}
%    \FPround{\result}{\result}{2}
%    \result
%}
%
%% Calculate textheight/paperwidth
%
%\newcommand\textheighttopaperwidth{%
%    \pgfmathparse{\textheight/\paperwidth}
%    \pgfkeys{/pgf/number format/.cd,fixed,precision=2}
%    \pgfmathprintnumber{\pgfmathresult}
%}
%
%\newlength\margintop
%
%\newcommand\thetop{%
%   \pgfmathparse{1in+\topmargin+\headheight+\headsep}
%   \pgfmathsetlength{\margintop}{\pgfmathresult}
%}
%
%\thetop
%
%\newlength\marginbottom
%\newcommand\thebottom{%
%   \pgfmathparse{\stockheight-(1in+\topmargin+\headheight+\headsep+\textheight)}
%    \pgfmathsetlength{\marginbottom}{\pgfmathresult}
%  }
%\thebottom
%
%\newcommand\verticalmarginratio{%
%\pgfmathparse{(\paperheight-(1in+\topmargin+\headheight+\headsep+\textheight))/  (\paperheight-(1in+\topmargin+\headheight+\headsep+\textheight))}
%\pgfmathresult
%}
%
%\newcommand\horizontalmarginratio{%
%\pgfmathparse{(\paperwidth-\textwidth-\oddsidemargin)/(1in+\oddsidemargin)}
%\pgfmathresult
%}
%
%\newcommand\numbertextlines{%
%% baselineskip to be corrected
%   \pgfmathparse{(\textheight-\topskip)/(12)-1}\pgfmathresult
%}
%
%\cxset{geometry units=mm}
%
%\def\printgeometryvalues{%
%   \noindent
%   \begin{tabular}{ll}
%   paper name & \papername\\
%   stock height & \convert@cx{\stockheight}\\
%   stock width  & \convert@cx{\stockwidth}\\
%   paperwidth & \convert@cx{\paperwidth}\\
%   paperheight & \convert@cx{\paperheight}\\
%   voffset & \convert@cx{\voffset}\\
%   hoffset & \convert@cx{\hoffset}\\
%   thetextheight & \convert@cx{\textheight}\\
%   thetextwidth  & \convert@cx{\textwidth}\\
%   Top margin   &  \thetop\convert@cx{\the\margintop}\\  % need to correct
%   Bottom margin & \thebottom\\
%   thetopmargin & \convert@cx{\topmargin}\\
%   theheadheight & \convert@cx{\headheight}\\
%   theheadsep & \convert@cx{\headsep}\\
%   theoddsidemargin & \convert@cx{\oddsidemargin}\\
%   theevensidemargin & \convert@cx{\evensidemargin}\\
%   themarginparsep& \convert@cx{\marginparsep}\\
%   themarginparwidth& \convert@cx{\marginparwidth}\\
%   themarginpush& \convert@cx{\marginparpush}\\
%   thevoffset& \convert@cx{\voffset}\\
%   thefootskip& \convert@cx{\footskip}\\
%   aspect ratio \aspectratio\\
%   twoside&  \if@twoside true\else false\fi\\
%   reversemarginpar& \if@mparswitch true \else false\fi\\
%  \end{tabular}
% }
%
%\def\readability{%
%\begin{tabular}{lr}
%  Characters per line &\charactersperline\\
%  Alphabets per line &\alphabetsperline\\
%  Alphabet length &\alphabetlength\\
%  Baselineskip & \the\baselineskip\\
%  Number of text lines &\numbertextlines\\
%  Text area ratio &\textarearatio\\
%  textheight/paperwidth&\textheighttopaperwidth\\
%  Text/page height ratio & \textheightratio\\
%  Vertical margin ratio &\verticalmarginratio\\
%  Horizontal margin ratio &1:\horizontalmarginratio\\
%\end{tabular}}
%
%
%% Note with new geometry paper has to be defined in preamble
%% I do not feel very confident of this
%% Don't understand it fully how is working
% %\@twosidefalse \@mparswitchfalse % one side option
%%\cxset{geometry oxford/.code={
%%\newgeometry{left=74.8mm,top=27.4mm,headsep=2\baselineskip,%
%%marginparsep=8.2mm,marginparwidth=49.4mm,textheight=49\baselineskip,headheight=\baselineskip}
%%\@twosidefalse \@mparswitchfalse % one side option
%%\reversemarginpar
%%}}
%% \@mparswitchfalse
%%\cxset{geometry textwidth/.store in=\textwidth@cx,
%%          geometry textheight/.store in=\textheight@cx,
%%          geometry tufte/.code={
%%             \newgeometry{a4paper,left=24.8mm,top=27.4mm,headsep=2\baselineskip,%
%%             textwidth=107mm,marginparsep=8.2mm,marginparwidth=49.4mm,%
%%             textheight=\textheight@cx\baselineskip,headheight=\baselineskip}
%%            \@twosidefalse \@mparswitchfalse % one side option
%%           %\reversemarginpar
%%    }
%%}
%%
%%
%%\cxset{marginpar push/.store in=\marginparpush@cx,
%%          marginpar font/.store in=\marginparfont@cx,
%%          marginpar justification/.is choice,
%%          marginpar justification/justifying/.code=\gdef\marginparjustification@cx{\justifying},
%%          marginpar justification/raggedright/.code=\gdef\marginparjustification@cx{\raggedright},
%%          marginpar justification/RaggedRight/.code=\gdef\marginparjustification@cx{\RaggedRight},
%%          marginpar justification/RaggedLeft/.code=\gdef\marginparjustification@cx{\RaggedLeft},
%% }
%%%\cxset{marginpar push=10pt,
%%%          marginpar font=\normalfont\footnotesize\sffamily,
%%%          marginpar justification=RaggedLeft}
%%%
%%%
%%%\cxset{style13, geometry textheight=47,
%%%          %geometry tufte,
%%%          watermark text=SAMPLE TUFTE VARIANT,
%%%          watermark text color=thered,
%%%          header style=samplepage}
%%%%%%%%%%%%%%%%%%%%%
%
%%%%%%%%%%%%%%%%%%%%%%%%%%%%%%%%%%%%%%%%%%%%%%%%%%%%%%%%%%%%%%%%%%%%%%%%%%
%%    DRAW THE PAGE ON A TRIAL BASIS
%%
%%%%%%%%%%%%%%%%%%%%%%%%%%%%%%%%%%%%%%%%%%%%%%%%%%%%%%%%%%%%%%%%%%%%%%%%%%%
%
%\cxset{geometry units= in}
%% lots of keys for trial sizes. We default all sizes to the ones defined in
%% by the document class.
%
%% We first set keys for the vertical dimensions
%\newlength\trytextheight@cx
%\newlength\tryheadheight@cx
%\newlength\tryheadsep@cx
%\newlength\tryfootskip@cx
%
%% LaTeX uses a correction to adjust the top margin, which is called topmargin. It does not 
%% represent the top margin though which following geometry we denote as top. It could perhaps
%% better be called top margin correction
%
%\newlength\trytopmargin@cx
%
%% Set keys for all the vertical dimensions and default to the current document settings
%\cxset{try textheight/.code=\global\setlength\trytextheight@cx{#1},
%          try textheight/.default=\textheight,
%          try headheight/.code=\global\setlength\tryheadheight@cx{#1},
%          try headheight/.default=\headheight,
%          try headsep/.code=\global\setlength\tryheadsep@cx{#1},
%          try headsep/.default=\headsep,
%          try footskip/.code=\global\setlength\tryfootskip@cx{#1},
%          try footskip/.default=\footskip,
%          try topmargin/.code=\global\setlength\trytopmargin@cx{#1},
%          try topmargin/.default=\topmargin,
%}
%
%% Set keys for all the trims, different people have different names for them. Normally two trims are
%% specified the top trim and the edge trip. We define two others just in case and to make calculations
%% easier if we have to use a different stock paper from the actual virtual paper width. the virtual
%% paper is called the paperwidth and paperheight.
%
%% We need to pick-up the memoir and koma allowances. TODO!
%\newlength\trimtop@cx
%
%\cxset{try trimtop/.code=\global\setlength\trimtop@cx{#1},
%          try trimtop/.default=\global\setlength\trimtop{0pt},}
%
%% set all the defaults
%
%\cxset{try textheight,
%          try headheight,
%          try headsep,
%          try footskip,
%          try topmargin=0pt, % compensate for trim
%          try trimtop=0pt}
%
%\addtolength\trytopmargin@cx{0pt}
%
%% set horizontal keys
%\newlength\trytextwidth@cx
%\setlength\trytextwidth@cx{0pt}
%\newlength\trytrimedge@cx
%\setlength\trytrimedge@cx{0pt}
%
%\cxset{try textwidth/.code=\global\setlength{\trytextwidth@cx}{#1},
%          try trimedge/.code=\global\setlength{\trytrimedge@cx}{#1},
%}
% 
%\cxset{try textwidth=\textwidth,
%          try trimedge=0pt}
%
%\def\alignedge{%
%% removed parindent from here must add it at the image
%  \checkoddpage%
%%   \ifoddpage \global\setlength\innermargin{\oddsidemargin}
%%          \else \global\setlength\innermargin{\evensidemargin}
%%      \fi%
%%   \if@twoside\setlength\innermargin{\dimexpr(\evensidemargin-\marginparsep)}%
%%             \else\let\innermargin\oddsidemargin\fi
%   \ifoddpage 
%      \innermargin\oddsidemargin
%      \def\innermarginname{oddsidemargin}%
%     \else
%        \innermargin\evensidemargin
%        \def\innermarginname{evensidemargin}%
%  \fi
%  }
%
%\alignedge
%
%
%%\ifoddpage
%%  \addtolength\innermargin{50pt}
%%\else
%%  \addtolength\innermargin{20pt}
%%\fi
%%\addtolength\trytextheight@cx{-20pt}
%%\addtolength\trytextwidth@cx{-24pt}
%%\addtolength\marginparwidth{-24pt}
%
%\reversemarginparfalse
%
%\def\drawlayout{%
%  \checkoddpage
%   \alignedge
%
%\tikzset{dim/.style = {>= latex,color=black}}
%\begin{tikzpicture}[scale=0.45,font={\scriptsize\rmfamily},line width=.8pt,
%       every node={color=black}]
%
%% first we draw stockwidth and stockheight
%\draw [color=gray,fill=thegray] (0,0) rectangle ++(\stockwidth,\stockheight);
%
%% draw the paper 
%\ifoddpage
%  \draw [color=NavyBlue,dashed thick,fill=white]  (0+\lefttrim,\stockheight-\trimtop@cx) rectangle ++ 	(\stockwidth-\lefttrim-\trytrimedge@cx,-\stockheight+\trimtop@cx+\bottomtrim);
%\else
% \draw [color=NavyBlue,dashed thick,fill=white]  (0+\lefttrim+\trytrimedge@cx,\stockheight-\trimtop@cx) rectangle ++ (\stockwidth-\lefttrim-\trytrimedge@cx,-\stockheight+\trimtop@cx+\bottomtrim);
%\fi
%% dimensions one more try
%%\cxset{geometry units=mm}
%% paper width dimensions, better to change to a macro
%% tol is the distance to dimension
%
%% paper width
%\edef\tol{-2.5\baselineskip}
%\coordinate (A) at (0+\lefttrim,\tol);
%\coordinate (B) at (\stockwidth-\trimedge,\tol);
%\coordinate (C) at (0.5\stockwidth,\tol);
%\draw[dim, |<->|] (A) -- (B); 
%\node at (C) [yshift=0.5\baselineskip)]{paper width = \convert@cx{\paperwidth}};
%
%% stockwidth
%\edef\tol{-5\baselineskip}
%\coordinate (BD) at (0,\tol);
%\coordinate (BD2) at (\stockwidth,-5\baselineskip);
%\draw[dim, |<->|] (BD) -- (BD2); 
%\draw (BD) ++ (0.5\stockwidth,0) node [yshift=0.5\baselineskip]{stockwidth=\convert@cx{\stockwidth}} ;
%
%% top dimension at left
%\coordinate (H1) at (-5mm,\stockheight);
%\coordinate (H2) at (-5mm,\stockheight-1in-\trytopmargin@cx-\tryheadsep@cx-\tryheadheight@cx);
%\draw [dim,|<->|] (H1) -- (H2);
%\node[left,text width=1.5cm, text ragged left] at (-5mm,\stockheight-0.5*\margintop){top\\ \convert@cx{\the\margintop}};
%
%% bottom dimension at left
%\coordinate (H3) at (-5mm,0);
%\coordinate (H4) at (-5mm,\marginbottom);
%\draw [dim,|<->|] (H3) -- (H4);
%\node[left] at (-5mm,0.5*\marginbottom){\convert@cx{\the\marginbottom}};
%
%% textheight at left
%\draw[dim,<->]  (-5mm, \marginbottom) -- ++ (0,\trytextheight@cx);
%\node[left,text width=1.5cm,text ragged left] at (-5mm,\marginbottom+0.5\trytextheight@cx){textheight \convert@cx{\trytextheight@cx}};
%
%
%% trimedge
%\ifoddpage
%  \coordinate (D) at (\stockwidth-4\trimedge, 0.10\trytextheight@cx);
%  \coordinate (E) at (\stockwidth,0.10\trytextheight@cx);
%  \draw [dim,->|] (D) -- ++(3\trimedge,0);
%  \draw [dim,|<-|] (E) -- ++(3\trimedge,0) node at ++(0,0) [right,text width=2cm,color=black] {trim edge    \convert@cx{\the\trimedge}};
%\else
%%  \coordinate (D1) at (\trytrimedge@cx, 0);
%%  \coordinate (E1) at ++ (\trytrimedge@cx,\stockheight-\trimtop@cx);
%%  \draw (D1)--(E1);
%\fi
%
%
%% toptrim
%%\ifdim\trimtop>0pt
%  \coordinate (F) at (0.9\stockwidth, \stockheight-\trimtop@cx-8mm);
%  \coordinate (G) at (0.9\stockwidth, \stockheight-\trimtop@cx);
%  \coordinate (H) at (0.9\stockwidth,\stockheight);
%  \draw (F)[dim,->|] -- (G);
%  \draw (H) -- ++ (0,8mm) -- ++ (5mm,0)[|<-|,>=latex] 
%          node [right] at ++ (0,0) {top trim =  \convert@cx{\the\trimtop@cx}};
%%\fi
%
%% 1in offsets
%\draw[dashed,color=gray] (1in,0) -- (1in,\stockheight);
%\draw[dashed,color=gray] (0in,\stockheight-1in)-- ++ (\stockwidth,0);
%
%% oddsidemargin/evensidemargin
%% draw dimension and name based on even or odd page
%\draw[dim,|<->|] (0,0.1\trytextheight@cx) -- ++(1in+\innermargin,0) node[right] at ++ (2ex,0) [text width=2cm] {\innermarginname\  \convert@cx{\the\innermargin}};
%
%% HEADER
%\coordinate (I) at (1in-\lefttrim+\innermargin,\stockheight-1in-\tryheadheight@cx-\trytopmargin@cx+\trimtop@cx);
%\draw (I) rectangle ++ (\textwidth,\tryheadheight@cx);
%
%%\draw[dim,<->] (1.5in\tol,\stockheight) -- ++(0,-1in) node[above right] at ++ (0,0.2in) {1in + yoffset};
%
%% add in inch 
%\draw [dim,|-|] (\stockwidth+3ex,\stockheight-\trimtop@cx)
%      -- ++(0,-1in) node [right] at ++(2ex,0.65in) {offset=\convert@cx{1in}};
%
%%   add topmargin dimension
%\ifdim\topmargin>0pt
%\draw [dim,|-] (\stockwidth+3ex,\stockheight-1in+\trimtop@cx)
%      -- ++(0,-\trytopmargin@cx) node [right] at ++(2ex,0.5\trytopmargin@cx) {topmargin=\convert@cx{\topmargin}};
%\fi
%
%%  add headheight dimension
%\draw [dim,|-|] (\stockwidth+3ex,\stockheight-1in+\trimtop@cx-\trytopmargin@cx)
%        -- ++(0,-\tryheadheight@cx) node [right] at ++(2ex,0.5\tryheadheight@cx) {headheight=\convert@cx{\the\tryheadheight@cx}};
%
%%   add headsep dimension
%\draw [dim,|-] (\stockwidth+3ex,\stockheight-1in+\trimtop-\tryheadsep@cx-\tryheadheight@cx-\trytopmargin@cx)
%          -- ++(0,\tryheadsep@cx) node [below right] at ++(2ex,0){headsep = \convert@cx{\the\tryheadsep@cx}};
%
%% footskip dimension
%\draw [dim,|-|] (\stockwidth+3ex,\stockheight-1in+\trimtop@cx-\tryheadsep@cx-\tryheadheight@cx-\trytopmargin@cx-\trytextheight@cx) -- ++(0,-\tryfootskip@cx) node [right] at ++(2ex,0.5\tryfootskip@cx){footskip=\convert@cx{\the\tryfootskip@cx}};
%
%
%% textarea
%\coordinate (J) at (1in-\lefttrim+\innermargin-\trytrimedge@cx,\stockheight-1in+\trimtop@cx-\tryheadheight@cx-\trytopmargin@cx-\tryheadsep@cx-\trytextheight@cx);
%\draw[fill=lightgray!50] (J) rectangle ++ (\trytextwidth@cx,\trytextheight@cx);
%
%\draw[dim,<->|] (1in-\lefttrim+\innermargin,0.75\trytextheight@cx) -- ++(\trytextwidth@cx, 0)  node at ++(-0.5\trytextwidth@cx,0.5\baselineskip) {textwidth} node at ++ (-0.5\trytextwidth@cx,-\baselineskip) {\convert@cx{\the\trytextwidth@cx}};
%
%\pgfmathsetmacro{\gridx}{12}
%% draw grid
%\draw[xstep=(\paperwidth-\trimedge)/\gridx, ystep=(\stockheight-\trimtop@cx)/\gridx,color=gray,dotted]  (0,0) grid (\paperwidth,\paperheight); 
%%%   add textheight dimension
%%\draw [dim,-] (\stockwidth+3ex,\stockheight-1in+\trimtop-\headsep-\headheight-\topmargin) -- ++(0,-\textheight) node [right] at ++(2ex,0.5\textheight){textheight=\convert@cx{\the\textheight}};
%
%% footer
%\coordinate (I) at (1in-\lefttrim+\innermargin,  \stockheight-1in+\trimtop@cx-\tryheadheight@cx-\trytopmargin@cx-\tryheadsep@cx-\trytextheight@cx-\tryfootskip@cx);
%\draw (I) rectangle ++ (\trytextwidth@cx,\tryheadheight@cx);
%
%
%% marginpar
%\def\leftmarginpar{%
%    \draw [fill=Linen,opacity=0.7] (1in+\innermargin+\trytextwidth@cx+\marginparsep,   \stockheight-1in+\trimtop@cx-\trytopmargin@cx-\tryheadsep@cx-\tryheadheight@cx ) rectangle ++(\marginparwidth,-\trytextheight@cx);
% \draw [dim,|<->|] (1in-\lefttrim+\trytextwidth@cx+\innermargin+\marginparsep+\marginparwidth,0.75\trytextheight@cx) -- ++ (-\marginparwidth,0) node at ++(0.5\marginparwidth,0.5\baselineskip) {marginpar} node at ++(0.5\marginparwidth,-\baselineskip){\convert@cx{\the\marginparwidth}};
%}
%
%\def\rightmarginpar{%
% \draw [color=red] (1in+\innermargin-\marginparsep,\stockheight-1in+\trimtop@cx-\trytopmargin@cx-\tryheadsep@cx-\tryheadheight@cx ) rectangle ++(-\marginparwidth,-\trytextheight@cx);
%     \draw [dim,|<->|] (1in-\lefttrim+\innermargin-\marginparsep-\marginparwidth,0.75\trytextheight@cx) -- ++ (\marginparwidth,0) node at ++(-0.5\marginparwidth,0.5\baselineskip) {marginpar} node at ++(-0.5\marginparwidth,-\baselineskip){\convert@cx{\the\marginparwidth}};
%}
%
%\ifdrawmarginpars
%  \checkoddpage
%  \alignedge
%    \if@twoside
%         \ifoddpage
%            \leftmarginpar
%         \else
%            \rightmarginpar
%        \fi
%   \else
%  % one side paper
%        \leftmarginpar
%    \fi
%\fi
%
%% draw diagonal
%\ifoddpage
%     \draw [color=blue]  (\paperwidth-\trytrimedge@cx,0) -- (0, \stockheight-\trimtop@cx);
%  \else
%    \draw [color=blue] (\trytrimedge@cx,0) -- (\paperwidth,\paperheight-\trimtop@cx);
%\fi  
%\end{tikzpicture}
%}
%
%
%%%%%%%%%%%%%%%%%%%%%%%%%%%%%%%%%%%%%%%%%%%%%%%%%%%%%%%%%%%%%%%%%%
%%                 SPREAD DRAWN AS PER CLASSICAL RULES
%%                 FOR ILLUSTRATION PURPOSE
%%%%%%%%%%%%%%%%%%%%%%%%%%%%%%%%%%%%%%%%%%%%%%%%%%%%%%%%%%%%%%%%%%%
%\newlength\paperwidth@cx
%\newlength\paperheight@cx
%\setlength\paperwidth@cx{6in}
%\setlength\paperheight@cx{9in}
%\setlength\bindingcorrection{0.1in}
%
%\def\spread{%
%   \begin{tikzpicture}[scale=0.5,inner sep=0pt,outer sep=0pt]
%   % draw the two pages
%  
%   \draw[xstep=\paperwidth@cx/9,ystep=\paperheight@cx/9,color=blue] (0,0) rectangle (\paperwidth@cx,\paperheight@cx)  (\paperwidth@cx+\bindingcorrection,0) rectangle ++(\paperwidth@cx,\paperheight@cx);
%
%% draw the binding correction
%\draw[fill=gray, draw] (\paperwidth@cx,0)  rectangle (\paperwidth@cx+\bindingcorrection,\paperheight@cx);
%
%% draw grid
%
%\draw[xstep=(\paperwidth@cx)/9, ystep=(\paperheight@cx)/9,color=gray,]  (0,0) grid (\paperwidth@cx,\paperheight@cx);
%
%\draw[xstep=(\paperwidth@cx)/9, ystep=(\paperheight@cx)/9,color=red]  
%(6.2in,0) grid (12.2in,\paperheight@cx);
%
%% add type areas
%
%\draw[fill=purple] (2\paperwidth@cx/9,2\paperheight@cx/9) rectangle  ++(6/9*\paperwidth@cx,6*\paperheight@cx/9);
%
%\draw[fill=green] (\paperwidth@cx+\paperwidth@cx/9+\bindingcorrection,2\paperheight@cx/9) rectangle ++(6\paperwidth@cx/9,6\paperheight@cx/9);
%
%\ifdim\bindingcorrection>0pt
%\draw[color=white,font={\sffamily\bfseries}] node at (\paperwidth@cx+0.5\bindingcorrection, 0.5\paperheight@cx)[rotate=90,inner sep=0pt,outer sep=0pt] {BINDING CORRECTION};\fi
%
%\node [color=white,font={\sffamily\bfseries}] at (0.5\paperwidth,0.5\paperheight)  {LEFT PAGE};
%\node [color=white,font={\sffamily\bfseries}] at (1.5\paperwidth@cx+\bindingcorrection,0.5\paperheight@cx){RIGHT PAGE};
%
%% draw diagonals
%
%\draw [color=thegreen, line width=1.5pt] (0,0)-- (\paperwidth@cx,\paperheight@cx);
%\draw [color=thegreen, line width=1.5pt] (2\paperwidth@cx+\bindingcorrection,0)-- ++(-\paperwidth@cx,\paperheight@cx);
%
%% draw circles
%
%\draw [color=red] (0.5\paperwidth@cx,5\paperheight@cx/9) circle (0.5\paperwidth@cx);
%
%\end{tikzpicture}
%}
%
%
%
%

\chapter{Geometry and Page Dimensions}
\parindent1.5em

\section{Introduction}

Setting up the page geometry, is normally done by the class or if adjustments need to be made, most authors will use the package geometry. If you need to view the geometry and the values of the document layout you can use the pkg{layouts}. This package offers a set of convenience key values for setting up geometry in order to enable authors to have a comprehensive style sheet.

\section{How to set geometry via this package}

To set the geometry page of the whole document, set the keys in the preamble. To change the page geometry anywhere in the document use the appropriate style or keys where you want the page geometry to change.
Note that the paper zize can only be defined in the preamble. The package is more useful when loaded with predefined styles.

\begin{tcolorbox}
\begin{lstlisting}
\cxset{page geometry=medieval}
\end{lstlisting}
\end{tcolorbox}

In most instances you will want to load the geometry at the style sheet.


\section{Viewing the page geometry}

The package offers a number of keys to set documents either document wide or locally to change page 
parameters or to view the frames. this is very similar to what the layouts and geometry packages offer. We do
however use TikZ for these diagrams.

To incorporate a layouts diagram we offer two macros \cs{printlayout} and \cs{printlayoutvalues}. Both have associated styling keys.
\medskip

\section{The Ideal Page Layout}

Since the invention of writing, typographers, scribes and graphics artists have been on the quest to find the ideal
layout for a page. Figure~\ref{fig:medieval}, shows a probablee geometric method that was used to typeset such books as the Gutenburg bible. Tschischold was a major revivalist of the method. Since most measurements in those times were probably only done using a compass a ruler and possibly a square, dividing the page equally into a nine part grid was done by first drawings the diagonals that are shown in blue in the figure, the intersections were then determined from the red lines thus enabling the typed area to be demarcated. 


\begin{figure}[htbp]
\pgfmathsetmacro\xsteps{9}
\pgfmathsetmacro\ysteps{9}
\cxset{spread scale=0.3}
\drawclassicspread
\caption{The ideal medieval page spread.}
\label{fig:medieval}
\end{figure}

To the modern eye, pages typeset in this manner might look rather empty, so smadjustments are made to such layuots. However, one tries to keep the proportions approximately to those of the classical layouts.
 
\begin{figure}[htbp]
  \includegraphics[width=0.95\textwidth]{tchichold01}
  \caption{\protect\url{http://www.artlebedev.com/everything/izdal/novaya-tipografika/}}
\end{figure}

\section{Technical discussion}
\subsection{The LaTeX standard classes}

LaTeX has pre-build layouts that depend  on two variables, specified by the user: the \textit{paper size} and the \textit{font size}. Appropriate values for the rest of the page layout are then  calculated by the class algorithm or are preset to certain values.

\subsection{Other common classes}

The more generic classes such as memoir and koma-script offer extensive customization and calculation of page parameters. They all use the basic laTeX page terminology which they supplement for additional parameters.

The octavo class offers a set of paper sizes suited for classical layouts printed on classical sizes such as the octavo. Classes such as the tufte-book offer a fixed design and no special commands for parameter manipulation.

\subsection{Paper sizes}

Most people using LaTeX, will print on either a4paper or letterpaper sizes. If you going to bind the work it might be necessary to trip the paper a little bit during binding to make sure that the top and side of the book are not ragged. This is normally called the \textit{trim}. If the document is to be printed by a publishing house this might be done by the printer which will use a different size \textit{stock size}. They might also allow for two additional dimensions called the spinemargin or the foremargin.

\begin{table}[ht]
\caption{North American paper sizes.}
\begin{tabular}{lllll}
\toprule
Size &width (mm)  &Height (mm)  &Width (in) &Height (in)\\
\midrule
US Ledger   &432 &279 & 17.0 &11.0\\
US Tabloid &279 & 432 & 11.0 &17.0\\
US Letter  &216 & 279 & 8.5 &11.0\\
US Legal   &216 &356 & 8.5 & 14.0\\
Government Letter &203 & 267 & 8.0 &10.5\\
Junior Legal &203 & 127 & 8.0 & 5.0\\
\bottomrule
\end{tabular}
\end{table}

\clearpage

\begin{table}[ht]
\caption{A series paper sizes.}
\begin{tabular}{lllll}
\toprule
Size &width (mm)  &Height (mm)  &Width (in) &Height (in)\\
\midrule
A0   &841 &1189 &33.1 & 46.8\\
A1   &594 & 841 &23.4 & 33.1\\
A2   & 420 & 594 &16.5 &23.4\\
A3   &297 & 420 &11.7 &16.5\\
A4   &210 &297 &8.3 &11.7\\ 
A5   &148 & 210 &5.8 & 8.3\\
A6   &105 & 148 & 4.1 & 5.8\\
A7   & 74 & 105 & 2.9 & 4.1\\
A8   &52 & 74 & 2.0 & 2.9\\
A9   &37 & 52 & 1.5 & 2.0\\
A10  & 26 & 37 & 1.0 & 1.5\\
\bottomrule
\end{tabular}
\end{table}


\begin{table}[ht]
\caption{ANSI series paper sizes.}
\begin{tabular}{lllll}
\toprule
Size &width (mm)  &Height (mm)  &Width (in) &Height (in)\\
\midrule
ANSI A &216 &279 &8.5 &11.0\\
ANSI B &279 &432 &11.0 &17.0\\
ANSI C &432 &559 &17.0 &22.0\\
ANSI D &559 &864 &22.0 &34.0\\
ANSI E &864 &1118 &34.0 &44.0\\

\bottomrule
\end{tabular}
\end{table}

\clearpage

\section{Swedish Standard}
The Swedish standard SIS 014711 generalized the ISO system of A, B, and C formats by adding D, E, F, and G formats to it. Its D format sits between a B format and the next larger A format (just like C sits between A and the next larger B). The remaining formats fit in between all these formats, such that the sequence of formats A4, E4, C4, G4, B4, F4, D4, H4, A3 is a geometric progression, in which the dimensions grow by a factor 21/16 from one size to the next. However, the SIS 014711 standard does not define any size between a D format and the next larger A format (called H in the previous example). Of these additional formats, G5 and E5 are popular in Sweden for printing dissertations,but the other formats have not turned out to be particularly useful in practice and they have not caught on internationally.

\begin{table}[ht]
\caption{Swedish Extension}
\begin{tabular}{lllll}
\toprule
Size &width (mm)  &Height (mm)  &Width (in) &Height (in)\\
\midrule
G5 &169 &239 &6.65 &9.41\\
E5  &155 &220 &6.10 &8.66\\

\bottomrule
\end{tabular}
\end{table}


\begin{table}
\centering
\caption{Octavo page layout parameters, influenced by font-size}
\begin{tabular}{llll}
\toprule
                    & 10pt & 11pt &12pt \\
\midrule
\textit{Octavo}              &      &      &\\
headsep        &  6pt  &  7pt &  8pt\\
topskip          & 10pt &  11pt & 12pt\\
texwidth         &0.7paperwidth & &\\
\midrule
\textit{LaTeX}              &      &      &\\
headsep        & .25in   &  .275in & .275in \\
topskip          & 10pt &  11pt & 12pt\\
footskip         &.35in &  .38in & 30pt \\
maxdepth         &.5\textbackslash topskip & &\\
textwidth        & 345pt  & 360pt & 390pt\\
\bottomrule
\end{tabular}
\end{table}

\subsection{The page dimensions}

The page dimensions are shown in figure 1. We tried to cater for the common terminology of all the classes.

\subsection{Texwidth}

The width of the text can only be determined based on the designer's strategy and is inexorably tied also to
the textheight. For example in classical page design, the designer tried to get the textwidth to be the same size like the page width, thus giving an almost squarish look. Another strategy is the 6-9 strategy, where the paper is divided into a grid of 9 equal blocks and the textwidth occupies the 6. 

\subsection{Readability considerations}

An average line that is longer than 40 to 70 characters long -- inluding spaces, is difficult to read. This is generally applicable to European languages and might be different for other languages. In addition the average number of words in one line should also be considered. For the German language Willi Egger (2004) recommends a line consisting of 8 to 12 words as optimal. If this strategy is adopted one can determine the line length based on the number of letters.

The characters per line for this document is \charactersperline. Of course from a readabilty point of view one could keep increasing the font size, but this is poor strategy. A well designed page should allow for good proportions as well as readabilty. In general a tolerance up  to 80 characters on a line should be adequate.

Peter Wilson in the manual for the memoir class refers to equations developed by Morten H{\o}gholm\index{H{\o}gholm, Morten} that has done some curve fitting
to the data. He determined that the expressions
\begin{equation}
L_{65} = 2.042\alpha + 33.41 \label{eq:L65}
\end{equation}
and
\begin{equation}
L_{45} = 1.415\alpha + 23.03 \label{eq:L45}
\end{equation}
fitted aspects of the data, where $\alpha$ is the length of the alphabet
in points, and $L_{i}$ is the suggested width in points, for a line with
$i$ characters (remember that 1pc = 12pt).

Using these equations one could get a first estimate of the textwidth. I am not too sure though if this is a good strategy as one can calculate it fully using TeX. bringhurst and them had to read these values from tables, but we do not; we can easily calculate them. For example to calculate the alphabet length for the bookman font:

\begin{texexample}{}{}
  \bgroup
  \fontfamily{pbk}\selectfont\alphabetlength\\
  \charactersperline\\
  \the\textwidth
  \egroup
\end{texexample}

Table~\ref{tab:alphlengths} adapted from the memoir class, gives alphabet lengths in points for various
fonts. My own recommendation is that for wide paper you should use a wider font and possibly move to 11pt font, rather than the traditional LaTeX default of 10pt.

\begin{table}
\centering
\caption{Lowercase alphabet lengths, in points, for various fonts}\label{tab:alphlengths}
\begin{tabular}{lrrrrrrrr} \toprule
                                            & 8pt & 9pt & 10pt & 11pt & 12pt & 14pt & 17pt & 20pt \\ \midrule
\fontfamily{pbk}\selectfont Bookman         & 113 & 127 & 142 & 155 & 170 & 204 & 245 & 294 \\
\fontfamily{bch}\selectfont Charter         & 102 & 115 & 127 & 139 & 152 & 184 & 221 & 264 \\
\fontfamily{cmr}\selectfont Computer Modern & 108 & 118 & 127 & 139 & 149 & 180 & 202 & 242 \\
\fontfamily{ccr}\selectfont Concrete Roman  & 109 & 119 & 128 & 140 & 154 & 185 & 222 & 266 \\
\fontfamily{pnc}\selectfont New Century Schoolbook     & 108 & 122 & 136 & 149 & 162 & 194 & 234 & 281 \\ 	
\fontfamily{ppl}\selectfont Palatino        & 107 & 120 & 133 & 146 & 160 & 192 & 230 & 276 \\ 	
\fontfamily{ptm}\selectfont Times Roman     &  96 & 108 & 120 & 131 & 143 & 172 & 206 & 247 \\
\fontfamily{put}\selectfont Utopia          & 107 & 120 & 134 & 146 & 161 & 193 & 232 & 277 \\
\fontfamily{pag}\selectfont Avant Garde Gothic  & 113 & 127 & 142 & 155 & 169 & 203 & 243 & 293 \\
\fontfamily{cmss}\selectfont Computer Sans  & 102 & 110 & 120 & 131 & 140 & 168 & 193 & 233 \\
\fontfamily{phv}\selectfont Helvetica       & 102 & 114 & 127 & 139 & 152 & 184 & 220 & 264 \\
\fontfamily{pcr}\selectfont Courier         & 125 & 140 & 156 & 170 & 187 & 224 & 270 & 324 \\
\fontfamily{cmtt}\selectfont Typewriter     & 110 & 122 & 137 & 149 & 161 & 192 & 232 & 277 \\
\bottomrule
%\facesubseeidx{Bookman}\facesubseeidx{Charter}\facesubseeidx{Computer Modern}%
%\facesubseeidx{Concrete Roman}\facesubseeidx{New Century Schoolbook}
%\facesubseeidx{Palatino}\facesubseeidx{Times Roman}\facesubseeidx{Utopia}%
%\facesubseeidx{Avant Garde Gothic}\facesubseeidx{Computer Sans}
%\facesubseeidx{Helvetica}\facesubseeidx{Courier}%
%\facesubseeidx{Computer Typewriter}%
\end{tabular}
\end{table}

\subsection{Textwidth influenced by margin materials}

Many books, including LaTeX allow for margin materials. If this is true then of course margins must by their nature be larger at the paper edges to allow for such material. 

\subsubsection{Simple strategy}
However, despite most of the above typesetting strategies many an author just want to take an approach, where they specify the margins and want to get what they need for example a spine margin of 1cm and an edge margin of 1.5cm. This is also important for screen dimensions.

\begin{lstlisting}
\cxset{
    margin inner= 1in
    margin outer= 2in
    margin top=1in
    margin bottom=2in
}
\end{lstlisting}

If all four margins are specified, the typesetting area can be positioned on the paper block. Life is not this easy though.

\begin{lstlisting}
\cxset{
    textarea proportional={1}{6}{2}  %
    textarea octavo
    textarea latex
    textarea other
}
\end{lstlisting}

\subsection{Auto strategy}

A more involved approach is to combine the strategies. First to get a good margin to type area, you will need to choose a paper that has good ratios. A paper such as \textit{imperial} comes close to an A4 size or imperial size. One can trip the balance of the paper or adjust slightly the ratios for twoside printing. Since we talking about book design any consideration for one side printing is immaterial. For oneside printing one can accept a wider latitude of values. Algorithm follows:

\begin{enumerate}
\item select paper.
\item select font.
\item marginmaterial true or false.
\item financial constraints - minimize number of pages, maximize number of pages.
\item check ideal number of characters at 65 per line.
\item decide on 10pt, 11pt or 12pt and constrain the algorithm.
\item use 0.7 textwidth area and check for max characters, if not iterate to 11pt.
\item recommend trimming values to suit.
\end{enumerate}


\subsection{Textheight}
Normally there is more latitude in choosing the 
proportions\index{proportion!margin} 
of the upper and lower margins, though usually the upper 
margin\index{margin!upper} is less than the lower margin\index{margin!lower}
so the typeblock\index{typeblock!location} is not vertically centered. Many modern books disregard all these rules and in many examples the upper margin is higher than the lower margin.

For text height calculations there are two considerations. One is to select top and bottom margings that are either equal or at a 1:1.5--2.0 ratio and relate to the width of the horizontal margins. The second consideration is that this length must be exactly divisible by baselineskip. When using \cs{flushbottom} LaTeX expects that the \cs{textheight} is such that a number of textlines in the body font will fit exactly into the height. If not, it issues an underfull vbox's message. LaTeX calculates these parameters when loading the class .clo files and sets the number of lines to a round number.

Many modern books have equal upper and lower margins.
\bigskip

\section{Allowing for trims}

Once a book is printed the edges are trimmed a bit in order to ensure a smooth top and right edge. For most desktop publishing you should not worry about such trimming. If you are going to publish the book in a professional publisher get their advice as to any allowances, you need to make in your pdf file. In other possible scenarios is that you may want to use a paper size such as A4 and trim it yourself down to one of the classical sizes such as Royal. 

All calculations are based on selecting a paper size and trimming it down. Unlike some other classes we assume you have selected the paper as stock size and then trimmed. Adding the trims makes no sense. You could simply print them on the larger page with trim marks, which we cater for.

  \begin{align}
   H_p    & = \sum h_1\ldots h_n\\
      h_t  &= H_p -   \sum h_1\ldots h_n - h_b
  \end{align}

The top margin is influenced by the \textit{device margin}, which is set at one inch, which we denote as $C$. If paper is to be trimmed the effective device margin offset will be reduced by the trim amount, $\Delta_t$.

Hence, the top margin is given by
\begin{align}
     h_t = C-\Delta_t+h1+h_2+h_3
\end{align}


\drawtriallayout


\printgeometryvalues
\readability

\newpage

\drawtriallayout

\readability

\newpage



% end of two page spread
\subsection{Top and bottom margins}

Before you follow any advice in places such as the Lulu forums to have your top and bottom margins equal, consider the following quotation by Bernard Shaw:

\begin{quotation}
Every printer can understand regularity: few have studied good looks except in living creatures. Consequently they aim at equal margins; and even when they have learnt that an upper margin must be less than a lower one if it is not to look more, they do not always see that it looks well only when it looks less. The mediaeval manuscript or early printed book, with its very narrow margin at the top and very broad margin at the bottom of the page, with its outer margins broad and its inner ones contracted, so that when the book lies open the two pages seem to make but a single block of letterpress in a single frame, instead of two side by side, has never been improved upon and probably never will be. But I find it almost impossible to persuade a modern printer to make his top margin small enough; and when I at last succeed, he measures it from the running title instead of from the top line of the page.

I saw a book the other day, excellently printed in old faced type, set solid, on a fine light, clean white crusty paper; yet the page was quite spoiled by an exaggerated top margin,like a masher's collar, and by that abomination of desolation, a rule. The only thing that never looks right is a rule. There is not in existence a page with a rule on it that cannot be instantly and obviously improved by taking the rule out.
\end{quotation}

\subsection{Headers and footers}
A page may have two additional items, and usually has at least one of these. They are the
running header and running footer. If the page has a folio then it is located either in the
header or in the footer. The word ‘in’ is used rather lightly here as the folio may not be
actually in the header or footer but is always located at some constant relative position. A
common position for the folio is towards the fore-edge of the page, either in the header or
the footer. This makes it easy to spot when thumbing through the book. It may be placed
at the center of the footer, but unless you want to really annoy the reader do not place it
near the spine.

Often a page header contains the current chapter title, with perhaps a section title on
the opposite header, as aids to the reader in navigating around the book. Some books put
the book title into one of the headers, usually the verso one, but I see little point in that as
presumably the reader knows which particular book he is reading, and the space would
be better used providing more useful signposts.

\subsubsection{Determining the geometry of the headers and footers}

The important parameter in the calculation of the header and footers, is the \cs{headheight} and \cs{headsep}. Most classes tend to have these as fixed parameters, related to font-size as can be seen in Table~\ref{tab:headerparams}.

\begin{table}[htbp]
\centering
\caption{Header and footer parameters settings by common classes.}
\label{tab:headerparams}
\begin{tabular}{llll}
\toprule
                    &headsep                   &headheight &footskip\\
\midrule
LaTeX 10pt    &             &                 &           \\
LaTeX 11pt    &             &                 &           \\
LaTeX 12pt    &             &                 &           \\
Octavo          &             &                 &           \\
tufte-book     &2 \texttt{baselineskip}   & 1 baselineskip           &            \\
\bottomrule
\end{tabular}
\end{table}

\begin{figure}[htbp]
\includegraphics[width=\textwidth]{paradoxicalbrain}

\caption{Modern book approach to footer and header design. From \textit{Paradoxical Brain,} Narinder Kapur \textit{et al.}, Cambridge Univerity Press, 2011. Book is printed on Royal size paper.}
\label{fig:paradoxical}
\end{figure}

\begin{figure}[htbp]
{{\parindent0pt
\begin{tikzpicture}[inner sep=0pt,outer sep=0pt]
  \node (img) {\includegraphics[height=8cm]{paradoxicalbrain}};
  \draw  (img.north east) ++ (5pt,0pt)-- ++ (15pt,0) ++(-15pt, -0.083*8cm) --++ (15pt,0pt) 
            (img.south east) ++ (5pt,0pt) -- ++ (15pt,0pt) ++ (0,0.075*8cm) -- ++ (-15pt,0);
\end{tikzpicture}}}
\caption{Modern book approach to footer and header design. From \textit{Paradoxical Brain,} Narinder Kapur \textit{et al.}, Cambridge Univerity Press, 2011. Book is printed on Royal size paper. The top margin is $1/12$ of the page height and the bottom margin is $1/16$ of page height. No need for apogryphal methods here.}
\label{fig:paradoxical}
\end{figure}

The more modern style tends to shift the headers and footers towards the top edge and bottom edge of the paper respectively, and allows very little space at the top of the paper. Figure~\ref{fig:paradoxical} shows a footer that is very near the bottom of the text and a header that has been shifted upwards. This makes for a more economical design as it increases the amount of text that can be printed in the typed area. For special designs such as this, it is not possible to automate calculations other than specifying a full algorithm for margins and typed area. Margins for the example follow the 10/12 rule for the typed area and inner and outer margins are equal at 1:12 ratio to the trimmed paper width.


\section{Floating parameters}


\section{Summing up}

Although one would ideally like to input some constraints and get out a perfect layout, as the previous discussion shows this is not an easy task, as well 

\begin{figure}[htbp]
\includegraphics[width=0.9\textwidth]{artbook}
\end{figure}

%%\end{document}
%\lipsum[1-4]\marginnote[1pt]{\lorem
%    \lorem}
%
%\lipsum[1-2]

%% Stick the caption in the head might as well place the first picture also
\def\asidecaption{\parbox{4.2cm}{{\bfseries Image \thefigure}\par\lorem}%
  % \addtocontents{lof}{This is image 8}
}
\def\ps@caption{%
     \let\@oddfoot\@empty\let\@evenfoot\@empty%
    \def\@evenhead{%
        \begin{picture}(0,0)%
           \put(-150,-80){\asidecaption\par}%
            \stepcounter{figure}
           \put(-150,-370){\asidecaption}%
        \end{picture}%
      }%
    \let\@oddhead\@evenhead%
    \let\@mkboth\@gobbletwo%
    \let\chaptermark\@gobble%
    \let\sectionmark\@gobble%
 }

\def\ps@bigpicture{%
    \setlength\headheight{19cm}%
    \let\@oddfoot\@empty\let\@evenfoot\@empty%
    \def\@evenhead{%
         \begin{picture}(0,0)%
          \put(-149,0){\includegraphics[width=\dimexpr(\textwidth+150pt)]{stuartpearson}}%
         \end{picture}%
      }%
    \let\@oddhead\@evenhead%
    \let\@mkboth\@gobbletwo%
    \let\chaptermark\@gobble%
    \let\sectionmark\@gobble%
 }



\def\doubletakeimage{%
  \renewcommand{\topfraction}{.95}  % ensure seecond image will not float away
  \begin{figure}[t]
    \thispagestyle{caption}
    \includegraphics[width=\textwidth]{matron}%
  \end{figure}

  \begin{figure}[tp]
   \hspace*{-\marginparwidth}\includegraphics[height=0.9\textheight]{stuartpearson}
 \end{figure}
}




\lipsum[1-4]
\begin{figure}[htp]
\includegraphics[width=0.98\textwidth]{captionspecial}
\centering
\caption{Figure from \textit{Oxford History of Art, Portraiture}, Shearer West, Oxford University Press, 2004. The figures are numbered consecutively and the text in the List of Illustrations have different formatting.}
\end{figure}

\doubletakeimage



%% RESET EVERYTHING AT END OF CHAPTER
\addtocounter{chapter}{-2}

\@toctrue\@specialtrue


\@specialfalse
\tableofcontents
\listoftables
\listoffigures


\cxset{toc image=true}
\part{The Package}


\pagestyle{plain}
\setdefaults

%\input{tikztutorial.tex}
%
%
 \begin{epigraphpage}
 \epigraph{Begin at the beginning,'' the King said, gravely, ``Then
 go till you come to the end; then stop.''}{Lewis Carroll, {\it Alice
 in Wonderland}}

 \epigraph{You can never get a cup of tea large enough or a book long enough to
 suit me''}{C. S. Lewis}
 \end{epigraphpage}

\parindent1em
%\cxset{style13}
%\cxset{title margin bottom=10pt,
%          title beforeskip=1pt}

\chapter{Introduction}
\addtocimage{-12pt}{-20pt}{../images/tocblock-fish}


\epigraph{``Begin at the beginning,'' the king said
"and then go on till you come to the end, then stop."}{
---Lewis Carroll, Alice in Wonderland}

 \parskip3pt plus 5pt 
\noindent This package and its documentation attempts to eliminate some common 
problems encountered when using \LaTeX2e. The first one is the loading of 
recommended packages for a large and perhaps complicated document and 
the second is the re-designing of styles for a document.

 \LaTeX2e, does not provide a standard library, but comes equipped with
 a package mechanism that allows code extensions to be loaded as required.
 This has created a strong vibrant community, hundreds of packages and a 
 headache to both new and seasoned users. What packages are available, when
 to use them and in which order is a common theme for many questions on
 lists and |TX.SE|.

 It is quite common during the writing of a thesis or book
 for the author to keep on adding macros and packages
 at the preamble of the document. In most cases this can
 be satisfactory but in many others it leads to
 incompatibilities and errors. This package aims at
 minimizing one's preamble, by prefetching a number of
 commonly used packages. It also aims at loading them
 in the right order and providing patches for conflicts.
 
 I am hoping that using this package, will lead to less
 frustrations with the intricacies of \LaTeX2e\ packages.

The package code is complicated, but its usage is simple. You first load the package and then
you use one of the available templates:

 \begin{commands}[]{}
 \begin{verbatim}
 \usepackage{phd}
 \usetemplate{style13}
 \end{verbatim}
 \end{commands}

This is what you need to typeset a good looking book or thesis. The rest of this book is a footnote and you can skip them if you want. 

It will be better for the longer projects to just fork the
 package and adapt it to your needs. In this respect, I have
 uploaded the package to |github|.\footnote{\url{https://github.com/yannisl/phd}}

 My goal in selecting the packages and adding a number of 
 commands for the authors was to be able to typeset a 
 document for most common use cases, without the need of
 additional packages. The packages I selected are biased
 towards academic publications, although they can find use
 in almost any fields. The package provides a mechanism via
 PGF keys to provide a settings file. 
 
 Most of the documentation can be found in the implementation part.

Browse any books in a library or bookshop and the striking thing is that their design is very individualistic. They might have similarities but their main features vary. In many respects they resemble people's faces where minor differences have striking effects.

This package arose out of a question at stackexchange. How to redefine chapter heads. Having seen the popularity of the |pgf| package \cite{pkg-pgf} I realized that \latex users prefer this method of styling rather the traditional \latex method.

The user interface can be extended to basically all major packages. The principle is to keep to a minimum changes that can affect the LaTeX core commands. If there are any additions a key setting is provided to be able to revert back to normal LaTeX.

The workflow can be simplified. In addition I want to believe that the interface can provide a useful addition to the open source community and that other people will contribute style libraries, which will be simpler to write. It is also possible
to device an easy and uncomplicated web interface to handle
such a great number of variables.


Most people when they get started with \LaTeX\ will either use one of the standard classes such as the \docFile{book.cls} or one of the generic classes notably koma-script or memoir. Most students will be forced to use on of the many thesis classes available.

\section{The key value concept}

The key-value concept that originated with \LaTeX\ has been extended many times, the last and most serious implementation of it by Tantau in the PGF package. What essentially Tantau developed is a scripting language to script TeX code. The \tikzname and pgfplots packages are two major packaged that use keys effectively. Their popularity is growing and what this package does is to offer a user interface that has been modelled to be similar to that of \texttt{css} (cascade style sheets). 
\smallskip

\begin{scriptexample}{}{}
\textit{chapter number} font-size = Large,\\
\textit{chapter number}     color = theblue
\end{scriptexample}
\smallskip

The main idea behind the package, is that you are configuring a document style by means of \emph{settings} rather than writing macros. In the example above the \emph{number, chapter} can be thought of as class or id names in css style sheets and the |font-size, color| as property settings that apply to the particular element. 


\subsection{Settings}

Settings are activated either by using the command |\cxset|  or by loading a full style sheet. In most cases you will probably import a style sheet and then modify some of the properties using |cxset|.  For example this heading has a dot after the subsection number. This was accomplished by setting,

We can de-activate it for the next and subsequent subsection headings with the setting:

\lorem

\begin{scriptexample}{}{}
\begin{verbatim}
\cxset{subsection number after=\quad}
\end{verbatim}
\end{scriptexample}




\subsection{Cascading}

Most values once set for a higher section will be seen in a cascade by all subsectioning commands in a similar fashion similar to CSS. These include properties such as color, font families and alignment. Best though to specify all of them for maximum flexibility to your users.

\section{On typography}

This package hopefully will assist in improving the typography of books set with \latexe. Any typographical comments on the various styles are just my own ramblingss and not necessarily absolute truths. Like fashion and art typography has opinions rather than absolute truths. In many styles the design is slightly adapted to blend a bit better with this manual. Also I did not select fonts as per the samples but this is left on you the user to decide.



\section{Packages and Fonts}

This manual has been typeset with numerous fonts in order to enable the typsetting of almost all the scripts provided by the Unicode standard. In order to process it from the |.dtx| file, these fonts must be available in your system, otherwise \XeLaTeX\ will have a problem finding the fonts and it will take an awful long time to process. This is especially true for the scripts section, where virtually all the Unicode defined scripts are discussed. You will need a fast computer and a fast hard disk to process the document within a reasonable time. When using \pkg{fontspec} always define your fonts with the \cmd{\newfontfamily} this will speed up processing by an order of magnitude. Compiling from the command prompt will speed up compilation. Average speed 2-3 pages per second.

Many of \tex's parameters are stretched to the limit with a complicated document such as this manual. You will require a full distribution otherwise expect some errors. Important packages is \pkg{morefloats} and \pkg{morewrites}. The package will also expect that you have |e-tex| installed. Ubuntu users are normally one year behind in updates, so you might wish to update manually. It will take upwards of 5 minutes to compile fully on an old laptop and a couple of minutes on a state of the art computer.

The |dtx| should be processed best with its own make file provided for Windows only |phd-lua.bat|. The make file will process the documentation using \lualatex. You can also process the document with \xelatex but is prone to produce errors. Using \latexe the sections on scripts etc will not be printed and a much shorter version of the manual is provided. 

\section{Scripts and Languages}

The package and the documentation offer a full repertoire of font selection keys for different scripts and languages. It hasn't been possible, however hard I tried to compile this section of the documentation with \xelatex, as it kept giving errors of too many files open. This was also not possible even with the \pkg{morewrites} package loaded. With \lualatex the document compiled with no major problems other than the font rendering being of a lower quality to that of XeLaTeX on windows, other than disabling incompatible packages and a number of commands that were redefined. 

Some good news for multi-script typesetting is the |Noto| fonts from Google. These fonts named Noto from "No Tofu" meaning you do not see any little square blocks for undefined glyphs, are fast to load. Disantvantage you need to switch between font commands fairly often.

\section{This book}

When developing the templates, I started using \emph{lorem ipsum} text as samples. Half-way through this
became a jumble mass of uninteresting pages interspersed with code. Headings and the contents of the book
determine both the structure and the selection of fonts, so I went back and wrote narratives  to accompany
the headings. Many of the narratives are semi-autobiographical in nature; others are clustered around books I read and my own interests. Some I stumbled on them accidentally and are mostly there to demonstrate some code.

Besides the templates and the code there is another narrative which is based on notes I kept on \tex and its friends over the years and are offered as a more advanced introduction to coding \latexe and \tex. The whole manual was typeset in a |ltxdoc| class, slightly modified to turn into a book class.

The implementation code is also available and it was mostly for my own benefit. The whole manual with the exception of the |\cxset| introduction, is just a test document. The notes and the “dissection” of the standard \latexe and the standard classes are there to explain the background to the many coding decisions that I took while I was developing the package.

PhD students are notorious for going in all directions and exploring many adjacent fields before they sit down and write their theses. Some become life-time students. To all these new men and women of the Renaissance that slave away to inch knowledge one thesis at a time, I dedicate this book and the name of the package.

\subsection{The TeX hacking sections}

To start programming \tex you need to have a knowldge of \tex basic commands and approach. \latex2015 is a format build on top of \tex to provide a more structured approach. To program \latexe packages you need to understand \latexe concepts, code organization and conventions. To program in \latex3, you need to learn a whole new language and you still need to understand \tex, \latexe and the expl3 language and conventions. To program using LuaTeX, other than the Lua language you need to understand \tex very well.
None of these can be found in one place.  I have gathered a lot of material and put it together. This is not a language you can master easily or quickly, but can teach you a lot about typesetting, computer science and many other interesting topics.


 \section{Version control with Git and Github}
 
 If you are involved with code or a publication that will have frequent changes, you should consider
 some type of version control system. My own recommendation is to use |git| and an online repository such
 as |github|. The latter is currently very fashionable and makes sharing code easier. Note that the |github|
 offers both public as well as private repositories. The general recommendation is that for unpublished work
 such as a thesis or code under development, it is preferable to go for a private repository. 
 
 \lorem\lorem

 \section{Ordering of Packages}
 
One package that normally leads to errors is the 
\pkg{hyperref}. The package which is an outstanding example of software engineering and supported single handledy by Heiko Oberdiek\footcite{hyperref} redefines a a lot of internal commands of the kernel. As a lot of other packages do the same it has to be loaded at the end of the preable with the exception of some packages! 
 
 This manual is typeset according to the conventions of the
 \LaTeX \textsc{docstrip} utility which enables the automatic
 extraction of the \LaTeX{} macro source files~\cite{GOOSSENS94}.

 
 \href{http://tex.stackexchange.com/questions/96350/problem-with-algorithmic-and-hyperref}{problem with algorithmic and hyperref}

 \begin{verbatim}
\usepackage{float}  % load float package first!

\usepackage{hyperref} % let hyperref patch the float package stuff
.
 \usepackage{algorithm} % let algorithm use the patched version of the float package
 \end{verbatim}
 

\section{Known problems}

Perhaps the biggest issue with the package is the speed of
compilation with \XeLaTeX\ or \LuaTeX. This is to be expected, as both engines spend a lot of resources in font management. On demand loading of packages is something I have in the back of my mind. This should be done via document styles i.e., if a book is for the humanities, perhaps only a rudimentary amount of maths packages should be loaded.

\section{Future Directions}

\latexe and \tex usage appears to be increasing. This is mostly by programs that export results with \latexe code rather than authors writing books.  The method adopted here is easier to automate all sorts of reports and automated texts. I would like too develop a web interface for processing such templates and at the same time export into html instead of just producing pdfs. I have already a prototype.   

\section{Tooling}

Some of the scripts on a Windows machine need MSYS\footnote{\url{http://mingw.org/wiki/MSYS}}









%\makeatletter
\cxset{defaults/.style ={% 
    chapter title margin-top-width    =  0cm,
    chapter title margin-right-width  =  1cm,
    chapter title margin-bottom-width = 10pt,
    chapter title margin-left-width   = 0pt,
    chapter align                     = left,
    chapter title align               = left, %checked
    chapter name                      = CHAPTER,
    chapter format                    = block,
    chapter font-size                 = Huge,
    chapter font-weight               = bold,
    chapter font-family               = sffamily,
    chapter font-shape                = upshape,
    chapter background-color          = white,
  % chapter label    
    chapter color               = black,
    chapter number prefix             = ,
    chapter number suffix             = ,
    chapter numbering                 = arabic,
    chapter indent                    = 0pt,
    chapter beforeskip                = -3cm,
    chapter afterskip                 = 30pt,
    chapter afterindent               = off,
    chapter number after              = ,
    chapter arc                       = 0mm,
    chapter label background-color    = white,
    chapter label color               = black,
   % chapter afterindent               = on,
    chapter grow left                 = 0mm,
    chapter grow right                = 0mm,
    chapter rounded corners           = northeast,
    chapter shadow                    = fuzzy halo,
    chapter border-left-width         = 0pt,
    chapter border-right-width        = 0pt,
    chapter border-top-width          = 0pt,
    chapter border-bottom-width       = 0pt,
    chapter padding-left-width        = 0pt,
    chapter padding-right-width       = 10pt,
    chapter padding-top-width         = 10pt,
    chapter padding-bottom-width      = 10pt,
    %  
    chapter number color              = black,
    chapter number background-color   = white,
    chapter number font-size        = huge,
    chapter number font-weight      = bfseries,
    chapter number font-family      = sffamily,
    chapter number font-shape       = upshape,
    chapter number align            = Centering,
    %
    chapter title font-size        = Huge,
     chapter title font-weight      = bold,
     chapter title font-family      = sffamily,
     chapter title font-shape       = upshape,
     chapter title color            = black,
     chapter title background-color = white,
     }%
   }  
\makeatother     
%\makeatletter
%\cxset{toc image=\@empty,
%       chapter toc=true,
%       title beforeskip=1pt}
%
%\@specialfalse
%
%
%\renewcommand\stewart[2][]{%
%\fancypagestyle{fancy}{%
%\lhead{}\rhead{}
%\chead{}
%\cfoot{}
%\lfoot{}
%\rfoot{\thepage}
%\def\footrule#1{{\color{blue}%
%  \hrule width\paperwidth}\vskip3pt
%}
%
%\renewcommand{\headrulewidth}{0pt}
%\renewcommand{\footrulewidth}{0.4pt}}
%
%\clearpage
%
%\begin{tikzpicture}[remember picture,overlay]
%% Main shading block
%\node [xshift=5cm,yshift=-\paperheight] at (current page.north west)
%[text width=0.98\textwidth,text height=\paperheight, fill=thecream!30,rounded corners,above right]
%{};
%\node [xshift=6.5cm,yshift=-1.5cm-\soffsety] at (current page.north west)
%[text width=0.9\textwidth,below right]{\sffamily \bfseries \huge #2};
%
%\node [xshift=3cm,yshift=-1.5cm] at (current page.north west)
%[text width=3cm,align=center,minimum height=2.5cm, fill=blue,below right]
%{\[\text{\HHUGE\bfseries\sffamily\color{white}\thechapter}\]
%\par\vspace*{3pt}
%};
%
%\node [xshift=-0.2cm,yshift=-21.5cm] at (current page.north west)
%[text width=3cm,above right]%
%{\includegraphics[width=1.0\paperwidth]{\image@cx}};
%% second box left
%\node [xshift=3cm,yshift=-19.5cm] at (current page.north west)
%[text width=9cm,minimum height=2.5cm,inner sep=0.5em, fill=blue,below right]
%{\color{white}
%  \bfseries\sffamily \texti@cx
%};
%% Last block
%\node [xshift=6.5cm,yshift=-26cm] at (current page.north west)
%[text width=12cm,above right]
%{\textii@cx
%};
%\end{tikzpicture}
%\par
%\clearpage
%}





\cxset{steward,
  chapter numbering=arabic,
  chapter format = stewart,
  offsety=0cm,
  image= {./images/hine02.jpg},
  texti={When Lamport designed the original \LaTeX\ sectioning commands he did not provide a fully comprehensive interface for modifying their design. With current tools available improvements are much easier to program and this chapter provides the details.},
  textii={\precis{In this chapter we discuss a method that allows the production of fancy chapter headings and formatting, based on a set of key values. Central  to this process is the separation of content from presentation.
We also discuss the basic formatting tools that are available and how one can modify them to mould new book designs.}
 }
}


\chapter{Designing Chapter Headings}
\addtocimage{-12pt}{-20pt}{./images/tocblock-man-01.jpg}

\section*{Introduction}

A \textls*{crowded} first page is as unsightly as a crowded title page, wrote De Vinne in \emph{Modern Methods of Book Composition} in 1904.  Not much has changed since. A new chapter must make a good impression and must give an immediate signal that a different topic is going to be discussed. Traditionally chapter openings in LaTeX are an unimpressive and dry event. Our aim is to brighten it up a bit, while keeping true separation of content from presentation, but avoiding the pit traps of over ornamenting the design. A book is to be read and we should provide minimal ornamentation. \index[phdkeys]{chapter> ornamentation}

% \usepackage{array,tabularx}
%\newcolumntype{Y}{>{\raggedleft\arraybackslash}X}% see tabularx
%\tcbset{enhanced,fonttitle=\bfseries\large,fontupper=\normalsize\sffamily,
%colback=yellow!10!white,colframe=red!50!black,colbacktitle=thecodebackground,
%coltitle=black,center title,
%tabularx={X||Y|Y|Y|Y||Y},% this sets ’before upper’ and ’after upper’
%before upper app={Group & One & Two & Three & Four & Sum\\\hline\hline} }
%
%\begin{tcolorbox}[title=My table]
%Red & 1000.00 & 2000.00 & 3000.00 & 4000.00 & 10000.00\\\hline
%Green & 2000.00 & 3000.00 & 4000.00 & 5000.00 & 14000.00\\\hline
%Blue & 3000.00 & 4000.00 & 5000.00 & 6000.00 & 18000.00\\\hline\hline
%Sum & 6000.00 & 9000.00 & 12000.00 & 15000.00 & 42000.00
%\end{tcolorbox}

\begin{figure}[htbp]
\centering
\parindent=0pt
\fbox{\includegraphics[width=\textwidth]{metropolitan-spread}}
\par
\caption{A chapter opening from the Metropolitan Museum of Art publicaion, \textit{Assyrian Reliefs and Ivories} by Vaughn. E. Crawford et. al., 1980. The spread is simple and the chapters are not numbered. This is a common characteristic of many more recently published books.}
\end{figure}


What is to us now a common occurence with instant book-printing was not always so. The cost of illustrated books was a prime factor and as Tschichold wrote:
\begin{quotation}
In the area of book design, in the last few years a revolution has taken place, until recently recognized by only a few. but which now begins to influence a much wider range of action.
It means placing much greater emphasis on the appearance of the book and a wholly contemporary use of typographic and photographic means. Before the invention of printing, literature of that time was spread around by the mouth of the author himself or by professional bards. The books of the Middle Ages - like the "Mannessische Liederhandschrift" - had
\end{quotation}

The type of book you are writing and its contents will determine an appropriate design for chapter headings and the type of design and numbering if any for subsections. Here we are merely providing a mechanism to produce them. These methods can produce a mastepiece or an ugly piece of work. Some simple suggestions follow (from my observations of styles in books I like). In general you need to think what type of book you are developing. For example a novel, should be sectioned very carefully. Many books avoid marking of sections other than chapters totally, perhaps marking them just with a soft ornament such as three centered asterisks.

\section{Numbering of Sections}


In general books do not number sections beyond subsection. You can avoid them all together, if you are not going to reference the sections extensively. 

In works of fiction, authors sometimes number their chapters eccentrically, often as a metafictional statement. For example:
Seiobo There Below by László Krasznahorkai has chapters numbered according to the Fibonacci sequence.

The Curious Incident of the Dog in the Night-Time by Mark Haddon only has chapters which are prime numbers.

At Swim-Two-Birds by Flann O'Brien has the first page titled Chapter 1, but has no further chapter divisions.

God, A Users' Guide by Seán Moncrieff is chaptered backwards (i.e., the first chapter is chapter 20 and the last is chapter 1). The novel The Running Man by Stephen King also uses a similar chapter numbering scheme.
Every novel in the series A Series of Unfortunate Events by Lemony Snicket has thirteen chapters, except the final instalment (The End), which has a fourteenth chapter formatted as its own novel.

Mammoth by John Varley has the chapters ordered chronologically from the point of view of a non-time-traveler, but, as most of the characters travel through time, this leads to the chapters defying the conventional order.


\begin{pgfpicture}
\pgfpathmoveto{\pgfpointorigin}
\pgfpathlineto{\pgfpoint{1cm}{1cm}}
\pgfpathlineto{\pgfpoint{1cm}{0cm}}
\pgfusepath{fill}
\end{pgfpicture}




\begin{figure}[tbp]
\centering
\parindent=0pt
\fbox{\includegraphics[width=\textwidth]{fantasy-architecture}}
\par
\caption{A chapter opening from the Metropolitan Museum of Art publicaion, \textit{Assyrian Reliefs and Ivories} by Vaughn. E. Crawford et. al., 1980. The spread is simple and the chapters are not numbered. This is a common characteristic of many more recent books.}
\end{figure}


\begin{figure}[tbp]
\centering
\parindent=0pt
\fbox{\includegraphics[width=\textwidth]{fantasy-architecture-02}}
\par
\caption{A chapter opening from the Metropolitan Museum of Art publicaion, \textit{Assyrian Reliefs and Ivories} by Vaughn. E. Crawford et. al., 1980. The spread is simple and the chapters are not numbered. This is a common characteristic of many more recent books.}
\end{figure}


\section*{Use of Color}

The modern books that Tschilchod was discussing have long been overwhelmed by the appearance of larger, coffee book type of books. Our brains our now conditioned by branding and graphic design is everywhere. 

Once you have decided that the book is going to be a bit more colorfull, the choice of color will follow. The decision what to color will be an important one, which brings us to color theory. The history of color is perhaps as colorfull as the rest. Attempts to formalize and recognize order date back to Aristotle (384-322 bce) but began in earnest with Leonardo da Vinci (1452-1519) and have progressed ever since. Leonardo noted that certain colors intensify each other, discovering \textit{contrary} and \textit{complementary} colors. The first color wheel was invented by Britain's Sir Isaac Newton (1642-1727), who split white light into red, orange, yellow, green, blue, indigo and violet beams, then joined the two ends of the spectrum to form a circle showing the natural progression of colors. When Newton created the color wheel, he noticed that mixing two colors from opposite positions produced a neutral or \textit{anonymous} color.


\begin{figure}[htbp]
\parindent=0pt
\centering
\fbox{\includegraphics[width=\textwidth]{line-designs} }
\caption{Spread from \textit{Beautiful Geometry}, Eli Maor and Eugen Jost, Princeton Univeristy Press, 2014. A subtle coloring of the chapter heading, de-emphasizing the chapter number and coloring the chapter title. There is no chapter label. A dropcap with the same color starts the first paragraph. This style is easy to achive with the phd system.}
\end{figure}


\begin{figure}[htbp]
\parindent=0pt
\centering
\fbox{\includegraphics[width=\textwidth]{color-book01.jpg} }
\bigskip

\fbox{\includegraphics[width=\textwidth]{color-book02.jpg} }
\end{figure}

One would expect a book written for the sole purpose of describing color theory and its application to the Graphic Arts, is expected to be colorful. Note the de-emphasizing of the label and number. 

\begin{figure}[htbp]
\parindent=0pt
\centering
\fbox{\includegraphics[width=\textwidth]{color-book-03.jpg} }
The chapter heading label and number are almost invisible. The heading text, is typeset in large bold letters, shouting what is coming next. Not your typical scintific book\ldots
\bigskip

\fbox{\includegraphics[width=\textwidth]{color-book-04.jpg} }
\end{figure}

Advertizing people understand that they need to present the message of an advertizement loud and clear so as to catch the busy eye. A heading's message is the title description. Neither the label not the chapter if any are necessary to convey the message. The chapter heading is analogous to the stop at the end of a sentence. The brain gets a signal to absorb what was written before it and get ready for the next. The heading signals the end of a topic. One must not dwell on it.


\section{Contemporary Chapter Headings}

In the book \textit{China} the designer used both a chapter heading on a spread of two images, as well as repeated the chapter number on the text pages \ref{fig:threepage}. The images distill the message of the chapter, although the chapter subtitle is almost unreadable, dominated by the surrounding text. From a technical perspective, the chapter command must paint the two images, set the right type of heading for each page and then without increasing the counter, change the counter to one that displays the chapter number in words and then continue with typesetting the text. A careful choice of images is necessary for such chapters, as well as cropping the images to match the aspect ratio of the book pages. One also needs to be carefull for \latexe not to place any floats in between the page spreads. 

\begin{figure}[htbp]
\parindent=0pt
\centering
\fbox{\includegraphics[width=\textwidth]{beijing.jpg} }\par
\vfill

\fbox{\includegraphics[width=\textwidth]{beijing-01.jpg} }\par
%\fbox{\includegraphics[width=\textwidth]{pearl-river.jpg} }
\caption{A full page chapter spread.}
\label{fig:threepage}
\end{figure}

\begin{figure}[htbp]
\parindent=0pt
\centering
\fbox{\includegraphics[width=\textwidth]{beijing.jpg} }\par
\vfill

\fbox{\includegraphics[width=\textwidth]{beijing-01.jpg} }\par
%\fbox{\includegraphics[width=\textwidth]{pearl-river.jpg} }
\caption{A full page chapter spread.}
\label{fig:threepage}
\end{figure}


\clearpage



In Figure~\ref{fig:photospread} the bands are black, but position low on the page. The size of the pages are 9.69 \texttimes 11.42. The books sections are not numbered. Text i sbroken through inserts of bigger text. Many of the examples here are from
commercial nude photography books, as they tend to break with tradition. In the 1970s and 1980s, fashion photographers began to present a
new, confrontational image of the female body. The pioneer in this
respect was the German Helmut Newton (1920–2004). Newton’s
photographs of nudes were overtly sexual, with an undertone of
menace, and although his models tended to be depicted as part
of the social elite they were often placed, apparently caught out
in reportage style, in sordid environments engaged in fantasy and
fetish. His work made him highly influential in fashion photography,
though some of it was thought too highly sexual for American
magazines and appeared only in those published in Europe.


\begin{figure}[htbp]
\parindent=0pt
\includegraphics[width=\textwidth]{baetens-01.jpg} \par
\vfill\vfill\vfill\vfill
\includegraphics[width=\textwidth]{baetens-02.jpg}\par
\caption{Chapter spread and first pages after the chapter title which is on the right page of the chapter spread. From \textit{New Photography, Art and the Craft}, Pascal Baetens, DK Publications. }
\label{fig:photospread}
\end{figure}

In the 1980s, Newton undressed the dynamic and independent
female in a series called Big Nudes. In this series the women are
indeed naked and very tall, wearing nothing but makeup and high
heels. The Big Nudes were exhibited in the form of life-size prints
that were intended to provoke the viewer by showing self-confident
women who knew what they wanted and were very aware of their
beauty and sexuality



\chapter{Package Usage}

To use the package include it just like any other package:

\begin{teXXX}
\documentclass{book}
\usepackage{phd}
\cxset{style13}
\begin{document}
\chapter{Introduction}
\end{document}
\end{teXXX}

The command \docAuxCommand{cxset} sets the default style for the example to the style defined as \meta{style13}. The package currently offers  100 templates and numerous keys to manipulate them further. Styles are similar to \enquote{themes} used in web programming; they are a collection of keys that resemble in many ways \texttt{css}. Styles can have any names and I am sure as package usage increases and evolve,they will get better names. 

\section{Background}

Before describing in detail how to specify a new layout for headings, we offer an overview of how the task can be accomplished and the design philosophy behind the approach. 

Irrespective of the technique and tools used, the creation of new layouts can always be divided into the following three tasks: constructing a document from “layout bricks”, which we can term as “blocks” or “elements”; establishing the layout semantics of each block; and finally, creating a layout engine supporting any document constructed from such blocks.

\begin{description}
\item [Canned Layouts] At one end of the spectrum, the most accessible approach consists of picking, a canned layout, such as LaTeX itself and perhaps only provide rudimentary macros to manipulate it.
\item [Constraints] Constraints offer a middle ground between canned layouts and handwritten layout engines. Constraints are arguably the most widespread and successful layout programming technique. For, instance, the foundations of \tex are laid upon constraint. CSS, the ubiquitous web template language, also relies on constraints, although in a more restricted and indirect manner.
\end{description}

\subsection{Blocks and Elements}

We define an \emph{element} as a document block, that cannot be subdivided further. For example the chapter title element, is composed of the text of the chapter title. 

A \emph{block} on the other hand is can contain other blocks and or numerous elements. We can consider the chapter headings as \emph{blocks}, composed of three blocks the chapter, number and title. Each block is then composed of elements. Each element has properties and traits. One of these mandary properties is the name. 

Blocks are either \emph{configured} (all constraints are mandatory), or flexible (there are optional/alternative constraints). By bundling optional constraints, flexible blocks make their specification customizable by non-technical users. 

\subsection{Language semantics}

One of the aims of the syntax of the templates was to offer familiar terminology and to remove the use
of \tex macros as far as possible from templates. 
\medskip

{\parindent0pt

 \textit{section}| font-family=serif,|\\
 \textit{section}| font-size=LARGE,|\\
 \textit{section}| font-weight=bold,|\\
}

The restriction I imposed is problematic when dealing with fractions of linewidths and textwidths. So
at present we allow for example |title text-width=0.5\texwidth| or |title text-width=10cm| or any other valid units. Ideas for improvements can only come from user feedback in the future.

Some experimental ideas incorporated are:

\begin{verbatim}
title text-width = 0.5 text-width,
title text-width = 1.2 text-width,
\end{verbatim}

A better parser will need to be programmed for dimensions, which are all currently handled as etex |dimexpr|. 

The syntax must allows both for microtypography as well as macro-typographical features. The former would deal with mostly fonts, spacing and text justification, where the latter deals with layouts, borders shapes and the positioning of elements on the page and also reletively to other elements or blocks.

An advantage of this approach is that it also opens the possibility of parsing the text with a language other than \tex and translating the document to another format, such as |HTML| or |XML| either fully or partially. Next we will describe both the syntax as well as the usage of the settings.

\section{Chapter opening page}

The standard \latexe classes offer only two options to either open a chapter on an odd page or at any page. This package offers five alternatives:

\begin{docKey}[phd]{chapter opening}{=\meta{any, left, right, anywhere, ifafter}}{default none, initial=any}
For documents that are primarily to be read on the web, use |any| for normal books, use \textit{right}. Some templates that we provide use |any| and the examples use |anywhere| to enable us to display the heading at any position on the page.
\end{docKey}

\begin{decription}
\item [any] Opens a chapter at any page, either \textit{verso} or \textit{recto}.
\item [left] Opens a chapter on an even page
\item [right] Opens a chapter on a right page.
\item [anywhere] Opens a chapter at the point where the \cs{chapter} is typed.
\item [none] Alias for \marg{anywhere}.
\item [ifafter] Opens a chapter at the next page if the page has material that does not exceed a certain portion of \cs{textheight}.
\end{description}

\colorlet{theoption}{bgsexy}

To change a setting you just modify the value of the key \oarg{\option{chapter opening}} to one of the values described earlier. 

\begin{dispListing}
\cxset{chapter opening = anywhere}
\end{dispListing}
 
We use this key to print the many examples typesetting chapter heads that follow (see the example~\ref{ex:anywhere}).  


\begin{texexample}{title=Inline Chapter Example}{ex:anywhere}
\cxset{examplestyle/.style = {chapter format = block,
       chapter opening = anywhere,
       chapter name = CHAPTER, 
       %label
       chapter label font-family      = sffamily,
       chapter label color            = primary,
       chapter label background-color = white,
       % number
       chapter number font-family = sffamily,
       chapter number font-size = HUGE,
       chapter number color     = primary,
       chapter label align = centering,
       chapter number background-color = white,
       %title
       chapter title font-family = rmfamily,
       chapter title align = centering,
       chapter title background-color = bgsexy!15,
       chapter title before background-color=white}}
\cxset{examplestyle}       
\lorem
\chapter{Typography Example}
\lorem
\chapter{Another Chapter Heading}
\lorem
\end{texexample}


%\cxset{toc chapter = true}
\addtocounter{chapter}{-1}

Examples for other types of chapter openings follow in the rest of the documentation.

\subsection{Blank pages before chapters}

In the standard LaTeX book class when the \texttt{openany} option is not given or in the report class when the openright is given, chapters start at odd-numbered pages. This can cause a blank page to be printed. Some book designers prefer this page to be completely empty, without any headers or footers. This cannot be done with \lstinline{\thispagestyle} as this command will have to be issued on the \textit{previous} page. However by a suitable redefinition of the
\lstinline{\clearpage} this can be done automatically.
\medskip

\begin{teXXX}
\makeatletter
\def\cleardoublepage{\clearpage\if@twoside\ifodd\c@page\else
  \hbox{}
  \vspace*{\fill}
  \begin{center}
    This page left intentionally blank.
  \end{center}
  \vspace{\fill}
  \thispagestyle{empty}
  \newpage
  \if@twocolumn\hbox{}\newpage\fi\fi\fi}
\makeatother
\end{teXXX}


This is achieved easily by setting the following options:
\bigskip

\begin{tcolorbox}
\lstinline{chapter blank page=empty}\par
\lstinline{chapter blank page text=Some text.}\par
\lstinline{chapter blank page=plain}\par
\end{tcolorbox}
\medskip



The last one refers to a \lstinline!\thispagestyle{plain}!.
\cxset{chapter opening = right, chapter format = block}
\chapter{Test}

\cxset{defaults, chapter opening= anywhere}



\section*{Keys for chapter head formatting}

A chapter heading can be considered of being constructed of several parts, the \textit{chapter number}, the chapter name typically \textit{chapter} and the \textit{title}. Predefined keys handle all the elements of formatting. Additional keys are defined to handle other elements such as inclusion of images or producing complicated examples with graphics constructed with \texttt{TikZ} and other similar packages.


\bigskip\bigskip\bigskip\bigskip
\let\oldrefkey\refKey
\let\refKey\texttt
\makeatletter
\long\def\demobox#1#2{%
\par\bigskip\bigskip\bigskip
\begin{tcolorbox}[enhanced,left=0pt, top=0pt, bottom=0pt,width=\textwidth,
  enlarge top initially by=1cm,enlarge bottom finally by=1cm,left skip=1cm,right skip=1cm,
  colframe=white,colback=white,
  colbacktitle=red!30!white,colupper=black!7!white,
  code={\appto\kvtcb@shadow{%
    \path[fill=white,draw=yellow!50!black,dashed,line width=0.4pt]
      ([xshift=-1cm,yshift=-1cm]frame.south west) rectangle
      ([xshift=1cm,yshift=1cm]frame.north east);
     \path[fill=blue!20!white, 
              opacity=0.3, draw=yellow!50!black,solid,line width=1pt]
      ([xshift=-2cm,yshift=-2cm]frame.south west) rectangle
      ([xshift=2cm,yshift=2cm]frame.north east);  
    }},
  finish={
  \draw[thick,<->] ([yshift=-1.3cm]frame.north west)-- node[below]{\texttt{#1 width}}
    ([yshift=-1.3cm]frame.north east);
  \draw[thick,<->] ([xshift=-15mm]frame.north east)-- node[above]{\refKey{#1 height}}
    ([xshift=-15mm]frame.south east);
  \draw[thick,<->] (frame.north)-- node[right]{\refKey{#1 padding-top}} +(0,1);
  \draw[thick,<->] ([yshift=1cm]frame.north)-- node[right]{\refKey{#1 margin-top}} +(0,1);
  \draw[thick,<->] (frame.south)-- node[right, align=left]{\refKey{#1 padding-bottom}}+(0,-1);
  %left padding
  \draw[thick,<->] (frame.west)-- node[below right,align=center]{\refKey{#1 padding-left }}+(-1,0);
  %left margin
  \draw[thick,<->] ([xshift=-1cm,yshift=-0.9cm]frame.west)-- node[below right,xshift=-1,align=left]{\refKey{#1 margin-left }\\\refKey{#1 grow to left by}}+(-1,0);
  %right padding
  \draw[thick,<->] (frame.east)-- node[below left,align=center]{\refKey{#1 padding-right}}+(1,0);
 %right margin
  \draw[thick,<->] ([xshift=1cm,yshift=-0.9cm]frame.east)-- node[below left,xshift=1, align=right]{\refKey{#1 margin-right}\\\refKey{#1 grow to right by}}+(1,0);
 \draw[thick,<->] ([yshift=-2cm]frame.south)-- node[right, align=left]{\refKey{#1 margin-bottom},\\ \refKey{#1 after skip}}+(0,1);
  }
    ]
#2%
%\hrule width0pt height4.5cm depth0pt\relax% \vspace*{4.5cm}% \lipsum[1]
\end{tcolorbox}\par
\bigskip\bigskip\bigskip}
\makeatother

\demobox{chapter}{\scalebox{1.17}{\HHHUGE Chapter}}

The number box is again drawn in a box similar to a chapter with all properties generalized.

\demobox{number}{\scalebox{1.15}{\HHHUGE Thirteen}}



All parameters shown in the diagram can be set using the command \cs{cxset}. The property names follow conventions similar to those of |css|, rather than typical conventions of \tikzname that are more widely known to the programming community. The prefix to these properties (in the example \textit{chapter}) can be thought of
as similar to a |class| or |id| name in |css|.  

\begin{docCommand}{cxset}{\marg{options}}
  Sets options for every following \refEnv{tcolorbox} inside the current \TeX\ group.
  By default, this does not apply to nested boxes, see \Vref{subsec:everybox}.\par
  For example, the colors of the boxes may be defined for the whole document by this:
\begin{dispListing}
\cxset{chapter numbering = Roman,
       chapter number color = blue}
\end{dispListing}
\end{docCommand}

\begin{docKey}[]{chapter padding-top}{=\meta{dimension}}{no default, initial value 0pt}
All padding keys take one argument, which is a dimension. The length is also stored in a register
\cmd{\chapterpaddingtop}. In this chapter it was set at %\the\chapterpaddingtop.
\begin{dispListing}
\cxset{colback=red!5!white,colframe=red!75!black, chapter padding-top=2pt}
\end{dispListing}
\end{docKey}



\begin{docKey}[]{chapter padding-right}{=\meta{dimension}}{no default, initial value 0pt}
All padding keys take one argument, which is a dimension. The length is also stored in a register
\cmd{\chapterpaddingright}.  In this chapter it was set at %\the\chapterpaddingright.
\end{docKey}

\begin{docKey}[]{chapter padding-bottom}{=\meta{dimension}}{no default, initial value 0pt}
All padding keys take one argument, which is a dimension. The length is also stored in a register
\cmd{\chapterpaddingbottom}.  In this chapter it was set at %\the\chapterpaddingbottom.
\end{docKey}

\begin{docKey}[]{chapter padding-left}{=\meta{dimension}}{no default, initial value 0pt}
All padding keys take one argument, which is a dimension. The length is also stored in a register
\cmd{\chapterpaddingleft}.  In this chapter it was set at %\the\chapterpaddingleft.
\end{docKey}

%% margin

\begin{docKey}[]{chapter margin-top}{=\meta{dimension}}{no default, initial value 0pt}
All padding keys take one argument, which is a dimension. The length is also stored in a register
\cmd{\chaptermargintop}. In this chapter it was set at .
\end{docKey}

\begin{docKey}[]{chapter margin-right}{=\meta{dimension}}{no default, initial value 0pt}
All padding keys take one argument, which is a dimension. The length is also stored in a register
\cmd{\chapterpaddingright}.  In this chapter it was set at %\the\chapterpaddingright.
\end{docKey}

\begin{docKey}[]{chapter margin-bottom}{=\meta{dimension}}{no default, initial value 0pt}
All padding keys take one argument, which is a dimension. The length is also stored in a register
\cmd{\chapterpaddingbottom}.  In this chapter it was set at %\the\chapterpaddingbottom.
\end{docKey}

\begin{docKey}[]{chapter margin-left}{=\meta{dimension}}{no default, initial value 0pt}
All padding keys take one argument, which is a dimension. The length is also stored in a register
\cmd{\chaptermarginleft}.  In this chapter it was set at %\the\chaptermarginleft.
\end{docKey}

\subsection{Borders}

Border have three properties \emph{width, color} and \emph{style}. They can set individually for
each side of the box or using the shorter key .

\begin{docKey}[]{chapter border-top-width}{ = \meta{dimension}}{no default, initial value 0pt}
All border keys take one argument, which is a dimension.
\end{docKey}

\begin{docKey}[]{chapter border-right-width}{=\meta{dimension}}{no default, initial value 0pt}
All border keys take one argument, which is a dimension.
\end{docKey}

\begin{docKey}[]{chapter border-bottom-width}{ = \meta{dimension}}{no default, initial value 0pt}
All border keys take one argument, which is a dimension.
\end{docKey}

\begin{docKey}[]{chapter border-left-width}{ = \meta{dimension}}{no default, initial value 0pt}
All border keys take one argument, which is a dimension.
\end{docKey}

\subsubsection{Border Colors}

The colors follow the same pattern for |border-width| and again they can be set individually or using
a shorter key to set all of them in one color. 

\begin{docKey}[]{chapter border-top-color}{=\meta{color name}}{no default, initial value black}
All border keys take one argument, which is a dimension.
\end{docKey}

\begin{docKey}[]{chapter border-right-color}{=\meta{color name}}{no default, initial value black}
All border keys take one argument, which is a dimension.
\end{docKey}

\begin{docKey}[]{chapter border-bottom-color}{=\meta{color name}}{no default, initial value black}
All border keys take one argument, which is a dimension.
\end{docKey}

\begin{docKey}[]{chapter border-left-color}{=\meta{color name}}{no default, initial value black}
This key is stored in \cmd{\chapterborderrightcolor} and the value in this chapter is 
%\ExplSyntaxOn \l_phd_chapter_border_right_color_tl.
\ExplSyntaxOff
\end{docKey}



\subsubsection{Border Styles}

Standard |css|  offers four styles \emph{dotted, solid, double, dashed}. We offer almost an unlimited set of styles.

\begin{docKey}[phd]{chapter border-top-style}{=\meta{style name}}{no default, initial value \texttt{none}}
The |border-style| properties take a value, which can be |solid, double, dotted, dashed, asterisk|.
\end{docKey}

\begin{docKey}[phd]{chapter border-right-style}{=\meta{style name}}{no default, initial value \texttt{none}}
The |border-style| properties take a value, which can be |solid, double, dotted, dashed, asterisk|.
\end{docKey}

\begin{docKey}[]{chapter border-bottom-style}{=\meta{style name}}{no default, initial value \texttt{none}}
The |border-style| properties take a value, which can be |solid, double, dotted, dashed, asterisk|.
\end{docKey}

\begin{docKey}[]{chapter border-left-style}{=\meta{style name}}{no default, initial value \texttt{none}}
The |border-style| properties take a value, which can be |solid, double, dotted, dashed, asterisk|.
\end{docKey}

\begin{docKey}[phd]{chapter border-style}{=\meta{style name}}{no default, initial value \texttt{none}}
This key sets all chapter-border-\meta{top,right,bottom,left}-style to a single value.
\end{docKey}

\subsubsection{Fonts and colors}

All font parameters can be set using individual keys. The naming scheme in general follows |css| conventions.

\begin{docKey}[phd]{chapter color}{=\meta{color name}}{no default, initial value \texttt{black}}
This key sets the color for the \textit{chapter element}. The color name is stored in \cmd{\chaptercolor@cx}.
The value in this chapter is% \makeatletter\texttt{\chaptercolor@cx}\makeatother.
\end{docKey}

\begin{docKey}[phd]{chapter font-size}{=\meta{Huge, Large}}{no default, initial value \texttt{Huge}}
This sets the size for rendering the \textit{chapter element}. Use one of the following predefined values.
Note that you can either use a command i.e, |chapter font-size=|\cmd{\huge} 
or the command name i.e., |chapter font-size=huge|. The latter is the recommended method.
\end{docKey}

\begin{marglist}
\item [tiny] renders as {\tiny tiny}.
\item[footnotesize] renders as {\footnotesize footnotesize}
\item [small] Opens a chapter on an even page
\item [large] Opens a chapter on a right page.
\item [LARGE] Opens a chapter at the point where the \cs{chapter} is typed.
\item [huge] Alias for \marg{anywhere}.
\item [Huge] Opens a chapter at the next page if the page has material that does not exceed a certain portion of
 \cs{textheight}.
 \item[HUGE] renders as {\HUGE HUGE}.
 \item[HHUGE] renders as {\HHUGE HUGE}.
\end{marglist}

\begin{texexample}{Sizing settings}{}
\cxset{
          chapter format = block,
          chapter label font-size= HUGE,
          chapter name = Chapter,
          chapter format=block,
          chapter number font-size= HUGE,
          chapter title font-size=LARGE,
         % 
         % chapter padding-top=0pt,
         % chapter padding-bottom=0pt,
         % title margin-top=3pt,
         %
          }
\chapter{Setting font-sizes}          
\lorem

\end{texexample}


\begin{docKey}{chapter font-family}{ = \meta{sffamily, rmfamily etc.}}{no default, initial value \texttt{sffamily}}
The |font-family| key accepts \latexe conventional family names or |css| names such as |serif| and |non-serif|. The
value is stored in \docAuxCommand{chapter_font_family}, in this chapter it is set as {\ExplSyntaxOn\meaning\chapter_font_family\ExplSyntaxOff}
\end{docKey}


\begin{marglist}
\item [sffamily] The \emph{chapter element} is rendered in the document default \cmd{\sffamily}.
\item [rmfamily] The \emph{chapter element} is rendered in the document default \cmd{\rmfamily}.
\end{marglist}

%% Font weights
\begin{docKey}[]{chapter font-weight}{=\meta{mdseries,bfseries,etc.}}{no default, initial value \texttt{bfseries}}
The |font-weight| key accepts \latexe conventional family names or |css| names such as |bold| and |bfseries|. The
value is stored in \cmd{\chapterfontweight@cx}, in this chapter it is set as 
{\ExplSyntaxOn\expandafter\string\l_phd_chapter_label_fontweight_tl\ExplSyntaxOff}

\begin{texexample}{Setting chapter element font-weights}{fontweight}
\cxset{chapter label font-weight=normal}
\chapter{Font-weight is normal}
\cxset{chapter label font-weight= bfseries}
\chapter{Font-weight is bfseries}
\lorem
\end{texexample}
\end{docKey}


\begin{marglist}
\item [normal] The \emph{chapter element} is rendered in the document default \cmd{\sffamily}.
\item [bold] The \emph{chapter element} is rendered in the document default \cmd{\rmfamily}.
\item[bfseries] Renders as bold.
\item[mdseries] renders as medium series.
\item[light] This is an alias for normal.
\item[\upshape\ttfamily\string\bfseries] The command version of the setting.
\item[\upshape\ttfamily\string\mdseries] The command version of the setting.
\end{marglist}



\begin{docKey}[]{chapter font-shape}{=\meta{itshape,upshape,etc.}}{no default, initial value \texttt{upshape}}
The |font-weight| key accepts \latexe conventional family names or |css| names such as |bold| and |bfseries|. The
value is stored in |chapter_font_weight|, in this chapter it is set as %\ExplSyntaxOn \texttt{\chapter_font_shape}\ExplSyntaxOff.
\end{docKey}

In |css| the |font-shape| is named as |font-style| so we alias it as well. 

%\begin{marglist}
%\item[normal] normal font-style, defaults to |upshape|.
%\item[upshape] normal font-style, defaults to |upshape|. 
%\item[italic] italic shape, renders as {\itshape italic}. For some fonts it might not be available.
%\item[itshape] italic shape, alias of |italic|.
%\item[oblique] oblique font, in \latexe is equivalent to \cmd{\slshape} and renders as {\slshape slshape}, which might be slightly different than {\itshape italic}.
%\end{marglist}


\begin{texexample}{Setting up Fonts}{chapterfonts}
\cxset{   chapter format = block,
          chapter opening=anywhere,
          chapter label font-weight=normal,
          chapter label font-shape=upshape,
          %chapter border-width=0pt,
          %chapter border-style=none,
          %chapter padding-top=0pt,
          chapter label font-size=large,
          chapter number font-size=large,
          chapter number color=black,
          %title font-size=large,
          }
\chapter[fonts]{Test Font Weights}
\lorem
\cxset{chapter label font-shape=itshape}
\chapter{Test Italic Shape}
\lorem
\cxset{chapter label font-shape=normal}
\chapter{Test normal font-shape}
\lorem
\end{texexample}



The specification of font families is somewhat problematic. In the web the |css| allows |font-family|  to hold several font names as a ``fallback” system. If the browser does not support the first font, it tries the next font.

There are two types of font family names:

\begin{description}
\item[family-name] The name of a font-family, like “times”, “courier”, “arial”, etc.
\item[generic-family] The name of a generic family, like “serif”, “sans-serif”, “cursive”, “fantasy”, “monospace”.
\end{description}

Generally in the \tex community leaving the choice of font  open to what is available on a user’s computer is frowned upon. Knuth’s original aim to render consistently documents, irrespective of a user’s computer installation has served the community well, and it is possible three decades later to produce documents identical in all respects to the original. 

If this is still a valid requirement for documents is debatable. Current document processing requirements are focusing more in the preservation of content and document structure rather than form. Typeset documents in soft copy are now widely preserved in |pdf| or |postcript|  formats. One can archive the |.tex| file as well as the |pdf| file.  Back to the provision of keys, a key defined in a 
similar fashion to those of |css| could help, but there is also the issue of slow compilation. If a font cannot be
found, with the current code, it can slow down compilation tremendously. I am leaving the choice where it belongs to the user and the package writer. It makes no harm if a more flexible definition is provided. The user can then decide to only provide one or many fonts. 

This avoids complicated and almost unintelligible commands such as:

\begin{dispListing}
\setkomafont{subsection}{\usefont{T1}{fvm}{m}{n}}
\setkomafont{section}{\usefont{T1}{fvs}{b}{n}\Large}
\end{dispListing}

Here are some examples. 

\begin{texexample}{Serif and non-serif}{ex:fontfamily}
\cxset{chapter label font-family=serif, 
       chapter opening=anywhere}
\chapter{Serif font}
\lorem
\end{texexample}


\section{Floating and Alignment} 

This particular key bothered me, as the term \emph{float} has a different meaning in \latexe. However, to
be consistent with |css| terminology I have yielded to the temptation and included it.

\begin{docKey}[]{chapter float}{=\meta{left,center,right,none}}{no default, initial value \texttt{none}}
Key that controls the horizontal alignment of the \emph{chapter element}. I order for the
element to float, its |display| property must be set to |inline|.
\end{docKey}

%\begin{texexample}{Floating}{chapter:float}
%\cxset{chapter opening=anywhere, chapter float=center}
%\chapter{Centered Chapter}
%\lorem
%\cxset{chapter float=left}
%\chapter{Left Aligned}
%\lorem
%\cxset{chapter float=right}
%\chapter{Right Aligned}
%\lorem
%\end{texexample}


\subsection{The display property}

Both the |css| box model as well as the \TeX layout engine provide numerous complex algorithms in managing the floating of elements. This is normally controlled using two properties |display| and |float|.


\makeatletter

\begin{docKey}[phd]{chapter position}{ = \meta{absolute, relative}}{no default, initial value black}
This positioning directive instructs the engine to position the element at an exact position.
\end{docKey}



\tcbox[nobeforeafter]{$box_1$}\tcbox[nobeforeafter]{$box_2$}\tcbox[nobeforeafter]{$box_3$}\dotfill\tcbox[nobeforeafter]{$box_n$}
\tcbox[before skip=0.2cm, after skip=0pt, width=1cm, enlarge left by=10cm,width=5cm,enhanced,show bounding box]{title before element}
\tcbox[before skip=0pt, width=1cm, enlarge left by=10cm,width=5cm,enhanced,show bounding box]{
\tcbox{tb}\tcbox{title}\tcbox[nobeforeafter, width=1cm,]{tb}}
\tcbox[before skip=0pt, after skip=12pt, width=1cm, enlarge left by=10cm,width=5cm,enhanced,show bounding box]{\emph{title after} element \fbox{some}}
\makeatother

\begin{docKey}[phd]{chapter float}{=\meta{left,center,right,none}}{no default, initial value \texttt{none}}
Key that controls the horizontal alignment of the \emph{chapter element}. I order for the
element to float, its |display| property must be set to |inline|.
\end{docKey}
In document preparation systems or web page development the layout is user generated, i.e., the user is expected to type the html and the |css| will then specify as to how the page will be rendered by the browser. In our case for documents we can specify how we want the headings to look. The layout manager for each element, creates other associated elements, as shown for the title here. This way most layouts can be accomplished with the declarative visual language of the \pkgname{phd} package. 

\subsubsection{In-line elements}

When an element is specified as |inline| the rendering algorithm places the boxes after each other. This is widely used in |chapter elements| to render the number inline with the chapter name.
\medskip
\bgroup

\noindent
\tcbox[nobeforeafter,width=3cm, height=1cm]{Chapter}\tcbox[nobeforeafter]{twelve}
 
When the property is set as |block| the elements are stacked below each other.
\medskip

\tcbox{chapter  display=block   CHAPTER}
\tcbox{number display=block    TWELVE}

The elements can be considered to be enclosed in a \emph{ghost} element. If the property is set to float we
\begin{figure}[htbp]
\makeatletter
\parindent0pt\fboxsep0pt
\fbox{\vbox to 0pt{\hbox to \dimexpr(\textwidth)\relax{{\hss\tcbox[capture=minipage,width=5cm, height=2cm, top=0pt]{\raggedright number display=block\\ number float=right }}%
}%
}%
}\par
\vspace*{2cm}
\makeatother
\end{figure}
signalling to the layout engine that the element must be placed to the right of the page, as shown in the figure. 


\begin{figure}[htbp]
\makeatletter
\parindent0pt\fboxsep0pt
\fbox{\vbox to 0pt{\hbox to \dimexpr(\textwidth+2cm)\relax{{\hss\tcbox[capture=minipage,width=5cm, height=2cm, top=0pt]{\raggedright number display=block\\ \emph{element} float=right }
\tcbox[capture=minipage,width=5cm, height=2cm, top=0pt]{\raggedright \emph{element} display=block\\ \emph{element} float=right }
}%
}%
}%
}\par
\vspace*{2cm}
\makeatother
\end{figure}

\subsection{Absolute positioning}

Absolute positioning mode, will place an element at an exact position on the page. They are more difficult to
achieve and inflexible. 

\begin{docKey}{position}{=\meta{absolute},\meta{relative}}{no default, initial none}{}

\end{docKey}



This positioning directive instructs the engine to position the element at an exact position.


\begin{docKey}[]{chapter float}{=\meta{left,center,right,none}}{no default, initial value \texttt{none}}
Key that controls the horizontal alignment of the \emph{chapter element}. In order for the
element to float, its |display| property must be set to |inline|.
\end{docKey}
\egroup



\section{Number Element Keys}


\subsection*{Keys for numbering}

Chapter numbering follows that of the standard \LaTeX\ classes and is extended to cover some additional cases such as fully spelled out numbers. This of course is only good for languages that use the arabic numeralsn. For other languages numerals in different formats can be added with simple keys and without the need of \pkgname{polyglossia} or \pkgname{babel}. 

Note that the package uses Heiko Oberdiek's package \pkgname{alphalph} to allow for alphabetic numbering that extends beyond the normal 26 letters of the alphabet. Examples for numbering can be seen in \ref{ex:romannumbering}


\begin{docKey}[phd]{number numbering}{= \oarg{alph,Alph,roman,Roman,none,WORDS,words,none}}{default arabic}
Style of numbering.
\end{docKey}

\begin{marglist}
\item [arabic] Despite that the Arabs call what the West calls Arabic numbers Indian numbers, we provide the value arabic to have normal numbers printed.
\item [alph] Lowercase alphabetic numbering.
\item [Alph] Uppercase alphabetic numbering.
\item [roman] Lowercase roman numbering.
\item [Roman] Uppercase roman numbering.
\item [words] The number is in lowercase words.
\item [WORDS] The number is in uppercase literal numerals.
\item [Words] Prints the number in words and capitalizes the first letter, for example the number 21 will be printed as `Twenty One'\footnote{Currently limited to the first hundred numbers}.
\index{chapter design>numbering>words}
\item [ordinals] Prints the number as ordinal.
\item [Ordinals] Prints the number as Ordinal.
\item [ORDINALS] Prinst the number as ORDINALS.
\item [none] This is equivalent to using the star version of the command. It does not print any number and does not increment the chapter counter.\footnote{I am ambivalent about this, perhaps it will be better to increment it, as it can give a more general approach.}

\end{marglist}
\begin{texexample}{Literal Numbering}{ex:literal}
\cxset{chapter numbering=WORDS} 
\chapter{Literal numbering}
\lorem
\cxset{chapter numbering=words,chapter name=chapter}
\chapter{Literal numbering} 
\lorem
\end{texexample}




\cxset{chapter opening=anywhere, chapter numbering=Roman, chapter number font-shape=upshape}
\index{chapter design>numbering>roman}

\begin{texexample}{Setting up keys for numbering}{ex:romannumberingx}
\bgroup
\cxset{chapter format = traditional, 
       chapter name = CHAPTER, 
       chapter numbering = Roman,
       chapter label color = bgsexy}
\chapter{Roman numbering}
\lorem
\egroup
\end{texexample}





To emulate some old books we also offer an ordinal numbering scheme.

\begin{texexample}{Literal Numbering}{ex:ordinals}
\cxset{chapter numbering=ORDINALS} 
\chapter{Ordinals numbering}
\lorem
\cxset{chapter numbering=words,chapter name=chapter}
\chapter{Literal numbering} 
\lorem
\end{texexample}

\cxset{chapter numbering=arabic}

\subsection{Fonts and colors}
\begin{docKey}[phd]{number color}{=\meta{color name}}{no default, initial value \texttt{black}}
This key sets the color for the \textit{number element}. The color name is stored in %\cmd{\numbercolor@cx}.
The value in this chapter is %\makeatletter\texttt{\numbercolor@cx}\makeatother.
\end{docKey}

\begin{docKey}[phd]{number font-size}{=\meta{Huge, Large}}{no default, initial value \texttt{Huge}}
This sets the size for rendering the \textit{number element}. Use one of the predefined values, as described
in the section for the \emph{chapter} element.
Note that you can either use a command i.e, |number font-size=|\cmd{\huge} 
or the command name i.e., |number font-size=huge|. The latter is the recommended method.
\end{docKey}

Letter spacing can be achieved using the soul package in a combination with the key |spaceout|.
The following examples illustrate the usage.

\index[phdkeys]{{\ttfamily phd/chapter design test}}

%\begin{texexample}{Letter Spacing}{ex:letterspacing}
%\cxset{numbering=Roman,
%        % number letter-spacing=soul,
%        % chapter spaceout=soul,
%         %title spaceout=soul,
%         title font-size=Large,
%         title font-family=rmfamily,
%         title font-shape=scshape}
%\chapter{Letter Spacing}
%
%\lorem
%\end{texexample}

\begin{docKey}[phd]{chapter number letter-spacing}{=\meta{none, true, etc.}}{no default, initial value \texttt{none}}.
\end{docKey}

\begin{marglist}
\item[none] Default value no tracking is used and the letters are spaced as per the basic font information.
\item[inherit] Inherits the letter-spacing settings from the \emph{chapter} element.
\item[true] Letter spacing is employed, using the |soul| package.
\item[false] Alias for |none|.
\item[soul] The \pkgname{soul} package is used for letter-spacing.
\item[microtype] The \pkgname{microtype} package is used for letter-spacing. When the microtype package is used more fine tuning of parameters is available.
\end{marglist}

The example that follows, explains how the features offered by the \pkgname{microtype} package can be used to
set different tracking options.

\begin{texexample}{Microtypography}{micro}
\bgroup

\SetTracking
 [ no ligatures = {f},
 spacing = {600*,-100*, },
 outer spacing = {450,250,150},
 outer kerning = {*,*} ]
 { encoding = * }
 { 100 }

{\huge \textls{Chapter Twenty}}

\SetTracking
 [ no ligatures = {f},
 spacing = {600*,-100*, },
 outer spacing = {450,250,150},
 outer kerning = {*,*} ]
 { encoding = * }
 { 200 }
 
{\huge \textls{Chapter Twenty}}

\egroup
\end{texexample}


\hbox{\drawfontbox{\huge \upshape\textls(Chapter Twenty}}

\hbox{\drawfontbox{\huge \upshape\textls{Chapter Twenty}}}


\section{Styling the chapter title}

Similarly to the number and chapter styling keys exist for styling the chapter title. We summarize the available standard keys below:

\index{chapter design!labels!letter spacing}
\begin{texexample}{Styling the Title}{ex:title} 
\cxset{chapter numbering=arabic, chapter title font-shape=itshape}
\chapter{Chapter title}
\lorem
\end{texexample}


\begin{docKey}[phd]{chapter title font-family}{=\marg{family}}{no default, initial inherit document font}
Selects a predefined font family
\end{docKey}

\begin{texexample}{Title element font styling}{}
\cxset{chapter title font-family=sffamily}
\chapter{Title font family settings}
\lorem
\cxset{chapter title font-shape=itshape}
\chapter{Title font-style settings}
\lorem
\end{texexample}


\begin{docKey}[phd]{chapter title font-weight}{ = \marg{\cs{bfseries},\cs{normalseries}}} {}
Font weight.
\end{docKey}

\begin{docKey}[phd]{chapter title font-size}{= \marg{large, Large, huge, Huge, HUGE, HHuge}}{}
Font sizing commands or their names. Both \docAuxCommand{\HUGE} and HUGE are allowed to be used as values for the key.
\end{docKey}

\begin{docKey}[phd]{chapter title color} { = \marg{color}} {}
The color of the chapter title letters. This takes any predefined color name. 
\end{docKey}


\begin{docKey}[phd]{chapter title spaceout}{ = \marg{soul,none}} {no default, initial = none}
 This key will space out the title. 
\end{docKey}

\begin{texexample}{Title element spacing}{}
\cxset{chapter name=none,
       chapter numbering=none,
       chapter title font-size=Large,
       chapter title color=black,
       chapter title width=0.6\textwidth,
       %title spaceout=soul,
         }
\chapter{The Prehistoric Period in South-East Asia: 2300 BC--AD 400}        
\lorem 
    
\end{texexample}
\cxset{defaults}


\subsection*{Adding content before and after the title element}

Like all the other elements, the title element can be decorated with additional content,
before and after the text. There are two different forms. 

\begin{docKey}[phd]{title before}{=\marg{code}}{default none}
Contents before the title (vertical material)
\end{docKey}

\begin{docKey}[phd]{title after}{=\marg{code}}{default none}
Contents after the title (vertical material)
\end{docKey}

\begin{docKey}[phd]{title content before}{=\marg{code}}{default none}
Contents before the title (horizontal material)
\end{docKey}

\begin{docKey}[phd]{title content after}{=\marg{code}}{default none}
Contents after the title (horizontal material)
\end{docKey}

The difference between the two type of settings, consider the following situation. Assume you have a title that has a rule at the top and bottom and the text is surrounded by two ornaments. The surrounding ornaments will be inserted using the |title before content|, and the rules using the |title before| form. The |title before| is a full fledged element on its own. 

%{
%\hrule
%\centering
%*** Introduction ***
%\par
%\hrule
%}
%
%{
%\MakePercentComment
%\startlineat{200}
%\lstinputlisting{./styles/style13.tex}
%\MakePercentIgnore
%}



 
\begin{docKey}{/phd/ chapter title before skip}{= \marg{soul,none}}{}
Before title string skip.
\end{docKey}

\begin{docKey}{/phd/ chapter title after skip}{ = \marg{soul,none} }{}
After title string skip.
\end{docKey}

\lorem 
%
%\begin{texexample}{letter spacing the chapter title block}{ex:title3}
%
%\cxset{chapter spaceout=none,
%         numbering=arabic}
%         
%\chapter{Chapter Title Styling}
%\end{texexample}
%
%\end{document}



\cxset{chapter opening=right}
\section{Table of Contents}\index{table of contents!key settings}

Traditionally a chapter will be added to the Table of Contents if the \cs{chapter} command is issued. The starred version will not produce a number and will not add a contents line. Since we have adopted an approach where we use a key value interface we can dispense with the starred version of the command, by setting the \option{chapter toc} option to false. For example if we want to define a command for a ``Foreward'' or ``Epiloque'' without wishing them to be added to the table of contents we can use the following setting.\index{Foreward>definitions}\index{Epilogue>definitions}



\begin{texexample}{changing the chapter label name}{}
\cxset{chapter name=Chapteris, chapter numbering=arabic,}
\chapter{Foreward}
\lorem
\end{texexample}

Note that the key \option{numbering=none} still has to be set.


Please note that when \textbf{numbering=none} the chapter number is not available anymore and yo may have to reset it if required again. Although this might be seen as rather cumbersome than simply using \cs{chapter*} the advantage is consistency in the user interface and the use of appropriate semantic definitions for all sectioning commands thus achieving a bit more separation of context from style.


%\cxset{chapter toc=true}

\section{Defining styles}

Named styles can be defined using the standard \textsc{PGF} conventions. To define a style for the forward above we can use:

\begin{texexample}{}{}
\cxset{foreward/.style={chapter numbering=none,
          chapter name=none,
          chapter title font-size= Large,
          chapter title font-family= sffamily,
          chapter numbering=none}}
\cxset{foreward}
\chapter{Foreward.}
\lorem
\end{texexample}



\cxset{chapter numbering=arabic}
\section{Creating semantic names for commands and environments}

To keep our search for semantic commands and true separation of contents it is prudent to define some macros for typesetting the  `foreward' section.

\bgroup
\begin{texexample}{defining a \textit{Foreward} macro.}{}
\begin{lstlisting}
\cxset{foreward/.style={chapter toc=false,
          name=none,
          title font-size = Large,
          title font-family = sffamily,
          numbering=none}}
\newcommand\forewardname{foreward}
\expandafter\newenvironment\expandafter{\forewardname}{%
\cxset{foreward}\chapter{Foreward}}%
{}
\begin{foreward}
\lorem
\end{foreward}
\end{lstlisting}
\end{texexample}
\egroup

Notice the use of a new command \cmd{\forewardname} to allow for internationlization using Babel or other methods. One is tempted to let the English name, but a better approach perhaps is to define both.

\makeatletter



%\makeatletter\@specialtrue\makeatother

\newcommand{\mf}{{\fontencoding{U}\fontfamily{zmf}\selectfont METAFONT}}

\newcommand{\pcstrut}{\vrule height11pt width0pt}

\newcommand{\sample}{Typographia Ars Artium Omnium Conservatrix}

\newcommand{\thefont}[4][OT1]{%
	\textcolor{thefontname}{#2}&%
	\pcstrut\fontencoding{#1}\fontfamily{#3}\selectfont#4\\}

\newcommand{\fonttitle}[1]{%
	\multicolumn2{p{\columnwidth}}{\vrule height1.5pc width0pt
	\fontseries{b}\selectfont\textcolor{Subheadings}{#1}}\\[3pt]}


\cxset{steward,
  numbering=arabic,
  custom=stewart,
  offsety=0cm,
  image={hine03.jpg},
  texti={An introduction to the use of font related commands. The chapter also gives a historical background to font selection using \tex and \latex. },
  textii={In this chapter we discuss keys that are available through the \texttt{phd} package and give a background as to how fonts are used
in \latex.
 },
 subsubsection indent=0pt,
 section font-family=tiresias,
 subsection font-family=tiresias,
 subsubsection font-family=tiresias,
 }


\chapter{Setting up Fonts}
\label{ch:fonts}
\section{Introduction}
\pagestyle{headings}
\index{fonts>serif}\index{fonts>non-serif}
Selecting the right fonts for a book is a job best left to the book designer. Despise this good advice most \latex authors get their hands dirty trying to play the role of the book designer. A word of advice is that most of them make a royal mess of it. Irrespective of the \tex engine employed, being \tex, \latexe, \lualatex or \xelatex there are two issues in using fonts. How to select them and specify them and what fonts to use. We will dwell on the technical aspects of font selection in this Chapter.

There is another more serious aspect in selecting fonts based on ``physiological’’ considerations. Boris Veytsman in an article in TUGboat \citep{boris2012} reviewed the literature comparing fonts for readbility as well as the ``trustability'' of the text based on different fonts. Experiments carried out by Morris \citep{morris2012a} concluded that fonts affect the reader's attitude towards the text. Baskerville scored the highest and Comic Sans the lowest.
Interestingly Computer Modern, the default typeface of \tex, scored high in the test.  Other tests carried out by \cite{boris2012a} also concluded that there are no noticable differences between serif and non-serif fonts in reading comprehension for cyrillic adult readers and that comprehension and reading speed might be affected by factors other than the font serifs alone. 

Of course the biggest effect on readers is when fonts used for ``branding''. 
Marketers have been brainwashing
consumers for years through the use of fonts. In \textit{Branding
With Type}  by Rogener, Pool, and Packhauser
(1995), a fervent argument is made for unique
but consistent typefaces as a crucial element of
corporate branding. Rogener et al. describe the
fonts used by IBM, Mercedes, Nivea, and
Marlboro as instantly recognisable
internationally, and imply that the significant
investment by such companies in design and
copyright of trademarked fonts is worthwhile. 

For example, \citet*{rogener1995}. discuss the Nivea
Bold typeface developed in 1992 by Gunther
Heinrich at advertising agency \textsc{TBWA} in
Hamburg, Germany, for skincare brand Nivea,
and claim that the Nivea Bold typeface has
effectively embodied the Nivea brand’s `pure
and simple’ product philosophy. They link the
font directly to profitability and Nivea’s
worldwide product category market share of
35\% \cite[p. 91]{rogener1995} (Rogener, Pool \& Packhauser, 1995, p.
91).


{\small
\tiresias
\lorem

}

\section{The Choice of Typesetting Engine}

If you use only |pdfLaTeX| the range of fonts is rather limiting and I would highly recommend for any serious typesetting work to move onto |XeLaTeX| and the use of the package \pkg{fontspec} \citep{fontspec}. Another alternative is to use \lualatex. The latter is becoming more stable and is production ready to a large extend. It is expected to be the successor to pdfTeX.

One of the things I wanted to achieve with the \pkgname{phd} package was  to take care of different \tex engines, and to ensure that the package can be used irrespective of the \TeX\ engine used. 

Before we start outlining the scheme let us start, by demonstrating how to load one of the standard fonts provided by \latexe. We will load the Computer Modern font.\index{Computer Modern (font)} 

\begin{texexample}{How to load a font}{ex:fonts}
\newcommand{\fontdemo}[4][OT1]{
    \leavevmode
    \textcolor{thefontname}{#2}
    \fontencoding{#1}\fontfamily{#3}\selectfont#4 }

\fontdemo{CM}{cmtt}{ \alphabet\par}

\fox
\end{texexample}

In the example we have used a number of convenience commands that are provided by the |phd| package.

\CMDI{\alphabet} Typesets the letters of the English alphabet

\CMDI{\fox} Typesets the fox passage

The example  creates a convenience command to call the |computer modern typewriter| font and to print the alphabet.\footnote{The command \cs{alphabet} is provided by the \texttt{phd} package.} In this case we are asking \latex to load a font from the |cmtt| family. 

To load a font two things are required the encoding scheme [|OT1|] in the example and the somewhat cryptic font family name [|cmtt|].

\section{What is a character? And a glyph?}
\index{glyph}\index{character}
A character is an abstract
concept: the letter “A” is a character, while any
particular drawing of that character is a glyph. In many
cases there is one glyph for each character and one character
for each glyph, but not always.

The glyph used for the Latin letter “A” may also be
used for the Greek letter “Alpha”, while in Arabic writing
most Arabic letters have at least four different glyphs
(often vastly more) depending on what other letters are
around them.

\section{What's a font?}

As the \pkgname{fontinst} manual says: ``Once upon a time, this question was easily answered: a font is a set of type
in one size, style, etc. There used to be no ambiguity, because a font was a
collection of chunks of type metal kept in a drawer, one drawer for each font'' \citet{fontinst}.


With digital typesetting, things are more complicated. What a font
\textit{is} isn't easy to pin down. A typical use of a PostScript font with \latex might
use these elements:

\begin{enumerate}
\item Type 1 printer font file
\item Bitmap screen font file
\item Adobe font metric file (afm file)
\item \tex font metric file (tfm file)
\item Virtual font file (vf file)
\item font definition file (fd file)
\end{enumerate}

Looked at from a particular point of view, each of these files \textit{is} the font. So
what’s going on? Every text font in \latex has five attributes:

\index{encoding schemes>OML}\index{encoding schemes>OMS}\index{encoding schemes>OMX}\index{encoding schemes>U}\index{encoding schemes>OML}
\index{encoding schemes}\index{encoding schemes>OT1}
\begin{description}
\item[Encoding Schemes]
The \textit{encoding} scheme (in the example |OT1|) provides information as to which glyph goes into what slot in a font table. These font tables can be printed using |fonttest.tex|. We show the test for |cmtt10| in Figure~\ref{fig:fonttest}. The
most common values for the font encoding are:
\medskip

\begin{longtable}{ll}
OT1   & TEX text\\
T1     & TEX extended text\\
OML  & TEX math italic\\
OMS  & TEX math symbols\\
OMX  & TEX math large symbols\\
U       & Unknown\\ 
L\meta{xx}  A local encoding\\
\end{longtable}
\medskip

\item[family]\index{fonts>family}\index{fonts>cmr}\index{fonts>cmss}
\index{fonts>cmtt}
The name for a collection of fonts, usually grouped under a common
name by the font foundry. For example, `Adobe Times', `ITC Garamond',
and Knuth's `Computer Modern Roman' are all font families.

There are far too many font families to list them all, but some common ones
are:

\begin{longtable}{rl}
cmr  &Computer Modern Roman\\
cmss &Computer Modern Sans\\
cmtt &Computer Modern Typewriter\\
cmm  &Computer Modern Math Italic\\
cmsy &Computer Modern Math Symbols\\
cmex &Computer Modern Math Extensions\\
ptm  &Adobe Times\\
phv  &Adobe Helvetica\\
pcr  &Adobe Courier\\
\end{longtable}

\item[series] How heavy or expanded a font is. For example, `medium weight', `narrow'
and `bold extended' are all series.

\item[shape] The form of the letters within a font family. For example, `italic',
`oblique' and `upright' (sometimes called `roman') are all font shapes. The most common values for the font shape are:

\begin{longtable}{ll}
n  &Normal (that is `upright' or `roman')\\
it &Italic\\
sl &Slanted (or `oblique')\\
sc &Caps and small caps\\
\end{longtable}

\item[size] The design size of the font, for example `10pt'. If no dimension is specified, `pt' is assumed.
\end{description}

These five parameters specify every \latex
font, for example:

\begin{longtable}{lll}
|LaTeX| specification &Font  &TEX font name\\
|OT1 cmr m n 10|      &Computer Modern Roman 10 point &cmr10\\
|OT1 cmss m sl 1pc|   &Computer Modern Sans Oblique 1 pica &cmssi12\\
|OML cmm m it 10pt|   &Computer Modern Math Italic 10 point &cmmi10\\
|T1 ptm b it 1in|  &Adobe Times Bold Italic 1 inch &ptmb8t at 1in\\
\end{longtable}

When you get a font error or an underfull or overfull box \tex always will print an error with the font specification in full as shown below:

\begin{verbatim}
LaTeX Font Warning: Font shape `EU1/cmr/m/sc' undefined
(Font)              using `EU1/cmr/m/n' instead on input line 160.
\end{verbatim}



\begin{tabbing}
\ttverb\textvisiblespace\quad\=bbbbbbbbbbbbbbbbbbbbbbbbbbbbbbb\=b'b'\=cccccccccccccc\kill
\ttverb\`{}               \>OT1, T1, EU1, EU2\>   \a`{}\> (grave)      \\
\ttverb\'{}               \>OT1, T1, EU1, EU2\>   \a'{}\> (acute)      \\
\ttverb\^{}               \>OT1, T1, EU1, EU2\>   \^{}\>  (circumflex) \\
\ttverb\~{}               \>OT1, T1, EU1, EU2\>   \~{}\>  (tilde)      \\
\ttverb\"{}               \>OT1, T1, EU1, EU2\>   \"{}\>  (umlaut)     \\
\ttverb\H{}               \>OT1, T1, EU1, EU2\>   \H{}\>  (Hungarian umlaut) \\
\ttverb\r{}               \>OT1, T1, EU1, EU2\>   \r{}\>  (ring)       \\
\ttverb\v{}               \>OT1, T1, EU1, EU2\>   \v{}\>  (ha\v{c}ek)  \\
\ttverb\u{}               \>OT1, T1, EU1, EU2\>   \u{}\>  (breve)      \\
\ttverb\t{}               \>OT1, T1, EU1, EU2\>   \t{}\>  (tie)        \\
\ttverb\={}               \>OT1, T1, EU1, EU2\>   \a={}\> (macron)     \\
\ttverb\.{}               \>OT1, T1, EU1, EU2\>   \.{}\>  (dot)        \\
\ttverb\b{}               \>OT1, T1, EU1, EU2\>   \b{}\>  (underbar)   \\
\ttverb\c{}               \>OT1, T1, EU1, EU2\>   \c{}\>  (cedilla)    \\
\ttverb\d{}               \>OT1, T1, EU1, EU2\>   \d{}\>  (dot under)  \\
\ttverb\k{}               \>T1    \>   \k{}\>  (ogonek)     \\
\ttverb\AE                \>OT1, T1, EU1, EU2\>   \AE \>               \\
\ttverb\DH                \>T1    \>   \DH \>               \\
\ttverb\DJ                \>T1    \>   \DJ \>               \\
\ttverb\L                 \>OT1, T1, EU1, EU2\>   \L  \>               \\
\ttverb\NG                \>T1    \>   \NG \>               \\
\ttverb\OE                \>OT1, T1, EU1, EU2\>   \OE \>               \\
\ttverb\O                 \>OT1, T1, EU1, EU2\>   \O  \>               \\
\ttverb\SS                \>OT1, T1, EU1, EU2\>   \SS \>               \\
\ttverb\TH                \>T1    \>   \TH \>               \\
\ttverb\ae                \>OT1, T1, EU1, EU2\>   \ae \>               \\
\ttverb\dh                \>T1    \>   \dh \>               \\
\ttverb\dj                \>T1    \>   \dj \>               \\
\ttverb\guillemotleft     \>T1    \>   \guillemotleft  \> (guillemet) \\
\ttverb\guillemotright    \>T1    \>   \guillemotright \> (guillemet) \\
\ttverb\guilsinglleft     \>T1    \>   \guilsinglleft  \> (guillemet) \\
\ttverb\guilsinglright    \>T1    \>   \guilsinglright \> (guillemet) \\
\ttverb\i                 \>OT1, T1, EU1, EU2\>   \i  \>               \\
\ttverb\j                 \>OT1, T1, EU1, EU2\>   \j  \>               \\
\ttverb\l                 \>OT1, T1, EU1, EU2\>   \l  \>               \\
\ttverb\ng                \>T1    \>   \ng \>               \\
\ttverb\oe                \>OT1, T1, EU1, EU2\>   \oe \>               \\
\ttverb\o                 \>OT1, T1, EU1, EU2\>   \o  \>               \\
\ttverb\quotedblbase      \>T1    \>   \quotedblbase   \>   \\
\ttverb\quotesinglbase    \>T1    \>   \quotesinglbase \>   \\
\ttverb\ss                \>OT1, T1, EU1, EU2\>   \ss \>               \\
\ttverb\textasciicircum   \>OT1, T1, EU1, EU2\>   \textasciicircum \>  \\
\ttverb\textasciitilde    \>OT1, T1, EU1, EU2\>   \textasciitilde  \>  \\
\ttverb\textbackslash     \>OT1, T1, EU1, EU2\>   \textbackslash   \>  \\
\ttverb\textbar           \>OT1, T1, EU1, EU2\>   \textbar         \>  \\
\ttverb\textbraceleft     \>OT1, T1, EU1, EU2\>   \textbraceleft   \>  \\
\ttverb\textbraceright    \>OT1, T1, EU1, EU2\>   \textbraceright  \>  \\
\ttverb\textcompwordmark  \>OT1, T1, EU1, EU2\>   \textcompwordmark\> (invisible) \\
\ttverb\textdollar        \>OT1, T1, EU1, EU2\>   \textdollar      \>  \\
\ttverb\textemdash        \>OT1, T1, EU1, EU2\>   \textemdash      \>  \\
\ttverb\textendash        \>OT1, T1, EU1, EU2\>   \textendash      \>  \\
\ttverb\textexclamdown    \>OT1, T1, EU1, EU2\>   \textexclamdown  \>  \\
\ttverb\textgreater       \>OT1, T1, EU1, EU2\>   \textgreater     \>  \\
\ttverb\textless          \>OT1, T1, EU1, EU2\>   \textless        \>  \\
\ttverb\textquestiondown  \>OT1, T1, EU1, EU2\>   \textquestiondown\>  \\
\ttverb\textquotedbl      \>T1    \>   \textquotedbl    \>  \\
\ttverb\textquotedblleft  \>OT1, T1, EU1, EU2\>   \textquotedblleft\>  \\
\ttverb\textquotedblright \>OT1, T1, EU1, EU2\>   \textquotedblright\> \\
\ttverb\textquoteleft     \>OT1, T1, EU1, EU2\>   \textquoteleft   \>  \\
\ttverb\textquoteright    \>OT1, T1, EU1, EU2\>   \textquoteright  \>  \\
\ttverb\textregistered    \>OT1, T1, EU1, EU2\>   \textregistered  \>  \\
\ttverb\textsection       \>OT1, T1, EU1, EU2\>   \textsection     \>  \\
\ttverb\textsterling      \>OT1, T1, EU1, EU2\>   \textsterling    \>  \\
\ttverb\texttrademark     \>OT1, T1, EU1, EU2\>   \texttrademark   \>  \\
\ttverb\textunderscore    \>OT1, T1, EU1, EU2\>   \textunderscore  \>  \\
\ttverb\textvisiblespace  \>OT1, T1, EU1, EU2\>   \textvisiblespace\>  \\
\ttverb\th                \>T1    \>   \th              \>
\end{tabbing}                        

Do note that when you use the \pkgname{hyperref}, you will get a surprise, all the commands have been converted to "PU" encoding. This is mostly harmless and is  done in order for |hyperref| to mark bookmarks\footnote{http://tex.stackexchange.com/questions/198810/why-does-the-hyperref-package-changes-encoding-of-font-commands} in a safe way.

\begin{texexample}{font encoding}{ex:encoding}
\meaning\textasciitilde\\
\meaning\"\\
\meaning\NG\\
\meaning\k\\
\meaning\alpha
\meaning\printfontparams

\printfontparams
\end{texexample}

A peek at the \docfile{puenc.def} reveals the inner workings
of the encoding mechanism.

\begin{verbatim}
\ProvidesFile{puenc.def}
  [2003/01/20 v6.73l
  Hyperref: PDF Unicode definition (HO)]
\DeclareFontEncoding{PU}{}{}
\DeclareTextCommand{\textLF}{PU}{\80\012} % line feed
\DeclareTextCommand{\textCR}{PU}{\80\015} % carriage return
\DeclareTextCommand{\textHT}{PU}{\80\011} % horizontal tab
\DeclareTextCommand{\textBS}{PU}{\80\010} % backspace
\DeclareTextCommand{\textFF}{PU}{\80\014} % formfeed
\DeclareTextAccent{\`}{PU}{\textgrave}
\DeclareTextAccent{\'}{PU}{\textacute}
\DeclareTextAccent{\^}{PU}{\textcircumflex}
\end{verbatim}

\printfontparams 


\latex uses a number of other files to get to the particular file that contains the font metrics file |cmtt10| and to find the appropriate file. For the original Knuth fonts the filenames have been kept the same, essentially as a request from Knuth that one should not change them.

Most of the difficulty in selecting and using fonts is figuring the encoding scheme and the Karl Berry naming scheme. In the Example~\ref{ex:fonts} we select the \cs{fontfamily} |cmtt| which is computer modern type writer and then we invoke the macros for the shape \cs{itshape} and print the |alphabet|. The macro \cmd{\alphabet} is build-in the |phd| package as we use it in a few places.

\begin{figure}[htbp]
\centering

\hspace*{-2cm}\includegraphics[width=\textwidth]{./images/testfont-output.pdf}

\caption{Output from testfont.tex for cmtt10 font}
\label{fig:fonttest}
\end{figure}



\subsection{The Postscript fonts}

With Adobe reader a number of fonts come pre-packaged and these have been incorporated into \latex2e. These fonts can be found in all \tex distributions. The \textit{Times New Roman} is named |ptm|. 

\begin{texexample}{The Postscript fonts}{ex:postscriptfonts}
\raggedright
\begin{tabular}{@{}>{\sffamily\bfseries}rl}
\fonttitle{The Adobe `LaserWriter 35', 10 typefaces in a total of 35
different styles, standard on all PostScript printers}

\thefont{Avant Garde Book}{pag}{\fontsize{9}{9}\selectfont\sample}
\thefont{Bookman Light}{pbk}{\sample}
\thefont{Courier}{pcr}{\sample}
\thefont{Helvetica}{phv}{\sample}
\thefont{New Century Schoolbook}{pnc}{\sample}
\thefont{Palatino}{ppl}{\sample}
\thefont[U]{Symbol}{psy}{\sample}
\thefont{Times New Roman}{ptm}{\sample}
\thefont{Zapf Chancery Medium Italic}{pzc}{\fontsize{12}{12}\selectfont\itshape\sample}
\thefont[U]{Zapf Dingbats}{pzd}{\sample}
\end{tabular}
\end{texexample}

Using the |phd| package we can come closer to the |fontspec| or LuaTeX way of doing things and use longer font names as those found in the operating system.


ctivating the key will set the font to |pzc| and unless is within a group
will typeset the rest of the document with this typeface.

\makeatletter
\def\fontname@cx{}
\cxset{font name/.is choice,
       font name/Zapf Chancery Medium Italic/.code={\fontfamily{pzc}\selectfont},
 font name/courier/.code={\fontfamily{pcr}\selectfont},
font name/Helvetica/.code={\fontfamily{phv}\selectfont},
font name/helvetica/.code={\fontfamily{phv}\selectfont},
font name/Bookman Light/.code={\fontfamily{pbk}\selectfont},
font name/bookman/.code={\fontfamily{pbk}\selectfont},
font name/Utopia/.code={\fontfamily{put}\selectfont},
font name/Palatino/.code={\fontfamily{put}\selectfont},
font name/Old Standard/.store in=\fontname@cx,
font name/Junicode/.code={
\fontspec{Junicode}\addfontfeature{StylisticSet=2}}
}
\makeatother

\begin{key}{/phd/font name=\marg{Zapf Chancery Medium Italic}}
\cxset{font name=Zapf Chancery Medium Italic}
\bgroup \itshape This is how it is typeset\egroup
\end{key}


\begin{key}{/phd/font name=\marg{Old Standard}}
Setting the key to \texttt{Old Standard} will typeset the next sample in \texttt{OldStandard-Regular}, |Stylistic Set=2|. 

\bgroup
\parindent1em\itshape
^^A\cxset{font name=Old Standard}

\aliceii

abcdefg
\egroup
\end{key}

\begin{key}{/phd/font name=\marg{Junicode}}
Setting the key to \texttt{Junicode} will typeset the next sample in \texttt{Junicode}, \texttt{Stylistic Set=2}. 

\bgroup
\parindent1em\itshape
\cxset{font name=Junicode}

\aliceii

abcdefg
\egroup
\end{key}




\begin{key}{/phd/font name=\marg{Bookman Light or bookman}}
Bookman Light or |bookman|
\end{key}

\bgroup
\cxset{font name=bookman}
\aliceiii
\egroup


\begin{key}{/phd/font name=\marg{Utopia or utopia}}

\end{key}

\bgroup
\cxset{font name=Utopia}

\renewcommand{\LettrineFontHook}{\fontfamily{put}\fontseries{bx}}%
\par\leavevmode

\lettrine[lines=5, lhang=0.1,lraise=0.28,findent=1pt]{g}{oats} are animals found in all sort of places. The paragraph has been set using the font family |utopia|. The comment about the goats was just to get the letter g.
comfortable in mountain areas. I don't recall Alice  They are more
comfortable in mountain areas. I don't recall Alice  They are more
comfortable in mountain areas. I don't recall Alice  They are more
comfortable in mountain areas. I don't recall Alice  They are more
comfortable in mountain areas. I don't recall Alice  They are more
comfortable in mountain areas. I don't recall Alice  They are more
comfortable in mountain areas. I don't recall Alice 


\renewcommand{\LettrineFontHook}{\fontfamily{phv}\fontseries{bx}}%



\par\leavevmode

\lettrine[lines=5, lhang=0.1,lraise=0.28,findent=1pt]{g}{oats} are animals found in all sort of places. The paragraph has been set using the font family |utopia|. The comment about the goats was just to get the letter g.
comfortable in mountain areas. I don't recall Alice  They are more
comfortable in mountain areas. I don't recall Alice   They are more
comfortable in mountain areas. I don't recall Alice   They are more
comfortable in mountain areas. I don't recall Alice  They are more
comfortable in mountain areas. I don't recall Alice  They are more
comfortable in mountain areas. I don't recall Alice  They are more
comfortable in mountain areas. I don't recall Alice 

\medskip



\lettrine{G}{o}ats are among the earliest animals domesticated by humans. The most recent genetic analysis confirms the archaeological evidence that the wild Bezoar ibex of the Zagros Mountains are the likely origin of almost all domestic goats today. Neolithic farmers began to herd wild goats for easy access to milk and meat, primarily, as well as for their dung, which was used as fuel, and their bones, hair, and sinew for clothing, building, and tools. The earliest remnants of domesticated goats dating 10,000 years before present are found in Ganj Dareh in Iran. Goat remains have been found at archaeological sites in Jericho, Choga Mami Djeitun and Çay\"on\"u, dating the domestication of goats in Western Asia at between 8000 and 9000 years ago.\footnote{Text is from wikipedia's article for the domesticated goat.}

\bgroup

\cxset{font name=bookman}

As you have observed we did not change the normal size of paragraphs, but the examples demonstrate that differences in font families also affect the visual size of the typeset text. |Helvetica| is normally scaled down to 0.95 and |Chancery| is scaled a little bit up or we use a larger font size.
\egroup

\everypar{}%FIXME

\subsection{\textsf{Additional free fonts for use with \LaTeX}}

A number of archaic and other fonts are available in the \latexe historical collection. These are very impressive. They also provide in most instances transliterations.

\begin{tabular}{@{}>{\sffamily\bfseries}rl}
\fonttitle{\textit{The Historical Collection}}
\thefont{Cypriot}{cypr}{\fontsize{7}{7}\selectfont\sample}
\thefont{Linear `B'}{linb}{\fontsize{8}{8}\selectfont\sample}
\thefont{Phoenician}{phnc}{\sample}
\thefont{Runic}{fut}{TYPOGRAPHIA ARS ARTIUM OMNIUM CONSERVATRIX}
%\thefont{Rustic}{rust}{\sample}
\thefont[U]{Bard}{zba}{\sample}
\thefont{Uncial}{uncl}{\sample}[-3pt]
\end{tabular}

\subsection{Uncial fonts}

\newcommand{\ABC}{ABCDEFGHIJKLMNOPQRSTUVWXYZ}
%\newcommand{\abc}{abcdefghijkl mnopqrstuvwxyz}
\newcommand{\punct}{.,;:!?`' \&{} () []}
\newcommand{\figs}{0123456789}
\newcommand{\dashes}{- -- ---}
\newcommand{\sentence}{%
this is an example of the uncial font. now is the time for all good
men, and women, to come to the aid of the party while the quick brown fox
jumps over the lazy dog:}


\newcommand{\Sentence}{%
This is an example of the Uncial font. Now is the time for all good
men, and women, to come to the aid of the party while the quick brown fox
jumps over the lazy dog:}

Peter Wilson's \pkgname{uncial} package provides a useful uncial font and is easily used by just including the file. 

\begin{texexample}{Unical fonts example}{}
\begin{center}
The Uncial Huge normal font. \\ \par
{\unclfamily\Huge \ABC\\ \alphabet\\ \punct{}\dashes\\ \figs\\ \par }
\end{center}
\end{texexample}




The following fonts are all selections from Yiannis Haralambous collection and we categorize them as other scripts collection.

\begin{tabular}{@{}>{\sffamily\bfseries}rl}
\fonttitle{\textit{The Other Scripts Collection}}
\thefont{Calligraphic}{zca}{\fontsize{15}{15}\selectfont\sample}
\thefont[U]{Fraktur}{yfrak}{%
	Alle\char'215\ Verg\"angliche ist nur ein Gleichni
	Da\char'215\ Unzul\"angliche hier wird'\char'215\
	Ereigni\char'215;}

\thefont[U]{Schwabacher}{yswab}{%
	Da\char'215\ Unbeschreibliche hier wird'\char'215\ getan / 
	Da\char'215\ Ewig-Weibliche zieht un\char'215\ hinan!}
\thefont[U]{`Gothic'}{ygoth}{If it plese ony man spirituel or temporel
to bye any pye\char'140\ of two and thre comemoraci\~o\char'140}[6pt]
\thefont[U]{Decorative Initials}{yinit}{\fontsize{8}{8}\selectfont
\raisebox{-12pt}{YIANNIS}}
\end{tabular}

\section{Dingbat and Symbol Fonts}

\index{fonts>Zapf Dingbats}

Fonts containing collections of special symbols, which are normally not found in a text font, are called  \textit{dingbats}. One such font, the PostScript font Zapf Dingbats, is available if you use the |pifont| package, originally written by Sebastian Rahtz, and now part of |PSNFSS|. This is loaded automatically by the |phd| package. (See also implementation code at Page \pageref{dingbats}).

The parameter for the \cs{ding} command is an integer that specifies the character to be typeset according to Table~\ref{tbl:dingbats}. For example |\ding{38}| gives \ding{38}.

For Open Type fonts the |Wingdings| family can be found on Windows systems. The advent of Unicode and the universal character set allowed commonly used dingbats to be given their own character codes. Although fonts claiming Unicode coverage will contain glyphs for dingbats \textit{in addition} to alphabetic characters continue to be popular, primarily for ease of input. Such fonts are sometimes known as \textit{pi fonts}.\index{fonts>pi fonts}

\subsection{Unicode Dingbats block}

The Dingbats block |U+2700-U+27BF| was added to the Unicode Standard in June, 1993, with the release of version 1.1. This code block  contains decorative character variants, and other marks of emphasis and non-textual symbolism. Most of its characters were taken from Zapf Dingbats. 

The Ornamental Dingbats block (|U+1F650–U+1F67F|) was added to the Unicode Standard in June 2014 with the release of version 7.0. This code block contains ornamental leaves, punctuation, and ampersands, quilt squares, and checkerboard patterns. It is a subset of dingbat fonts Webdings, Wingdings, and Wingdings 2. \footnote{See \url{http://std.dkuug.dk/jtc1/sc2/wg2/docs/n4115.pdf}}

A font that we will be using for many of the \XeLaTeX examples is |code2000|
and |code2001|. The fonts were designed by James Kas
\footnote{They can be downloaded at \url{http://www.alanwood.net/downloads/index.html}}. They are True type fonts. The fonts contain a respectable collection of more or less exotic Unicode characters both within the Basic Multilingual Plane (BMP). They were designed by James Kass and were freeware. Sadly the website is no longer available, but the files can be downloaded in the links I have provided. I have also included them in the distribution for the |phd| package, as they are such a useful tool.

\index{Unicode}\index{Basic Multilingual Plane}

\CMDI{\codetwothousand} Loads the TrueType font \texttt{code2000.ttf}\index{fonts>code2000}\index{code2001}

\CMDI{\codetwothousandone} Loads the TrueType font \texttt{code2001}

\CMDI{\symbola} Loads the TrueType font \texttt{symbola}\index{fonts>Symbola}

\index{fonts>Symbola}
\index{fonts>code2000}
\index{fonts>code2001}
\begin{verbatim}
\newfontfamily{\codetwothousand}{code2000.ttf}
  \newfontfamily{\codetwothousandone}{code2001.ttf}
  \newfontfamily{\symbola}{symbola.ttf}
\end{verbatim}




\index{fonts>wingdings}
\begin{texexample}{Wingdings}{ex:wingdings}
\ifxetex
   {\codetwothousand \symbol{9742} \symbol{9743}
    Katakana (片仮名, カタカナ)
   \codetwothousandone \symbol{57508}
   \symbola \symbol{9816}
  }
\else
  \ifluatex
  {\codetwothousand \symbol{9742} \symbol{9743}
    Katakana (片仮名, カタカナ)
   \codetwothousandone \symbol{57508}
   \symbola \symbol{9816}
  }
  \else
   Compile the document with XeTeX to see the example
  \fi 
\fi
\end{texexample}

Another useful font for experimenting if you are using a Windows computer is |Arial Unicode MS|.
\index{Arial Unicode MS (font)}\index{fonts>Arial Unicode MS}

\begin{smallverbatim}
\documentclass{article}
\usepackage{fontspec}
\setmainfont{Arial Unicode MS}
\usepackage{multicol}
\setlength{\columnseprule}{0.4pt}
\usepackage{multido}
\setlength{\parindent}{0pt}
\begin{document}

\begin{multicols}{8}
\multido{\i=0+1}{"10000}{^^A from U+0000 to U+FFFF
  \iffontchar\font\i %
    \makebox[3em][l]{\i}%
    \symbol{\i}\endgraf
  \fi
}
\end{multicols}
\centering
\symbol{57352}
\end{document}
\end{smallverbatim} 


%
%\begin{multicols}{8}
%\ExplSyntaxOn
%\bgroup
%\symbola
%\multido{\next=0+1}{"10000}{
%  \iffontchar\font\next %
%     \makebox[3em][l]{\next}%
%    \symbol{\next}\endgraf
%  \fi
%\egroup  
%\ExplSyntaxOff
%}
%\end{multicols}

The |Symbola Font| has many other symbols, including chess and even Mahjong symbols\index{Mahjong}.\footnote{\url{http://users.teilar.gr/~g1951d/Symbola.pdf}}. \person{George Douros} has packaged many of the fonts for archaic languages, but sadly the substitution mechanisms of \latexe do not always map the fonts properly.

With |LuaLaTeX| and |XeLaTeX|, \tex has moved into the twenty-first century and its usefulness can now be extended to many other languages and fields. 

\section{Naming digital fonts}

Commercial and Open Source fonts come as a set of several files. The |.pfb| file and less frequently, a |.pfa| file or other files depending on the type of font and the operating system and provider. The metric information file resides in an associated |.afm| file. Other files, with extensions |.inf| (information) and |.pfm| are irrelevant to \latex and \tex.

Fonts already have names given them by their designers. The problem lies in associating this name with the font files. Restriction of operating systems originally from PC-DOS dates, restricted to the initial part of file names to eight characters.

\subsection{Karl Berry naming scheme}\index{Karl Berry Scheme}\index{fonts>Karl Berry scheme}

The original inspiration for Fontname was Frank Mittelbach and Rainer Schoepf's article in TUGboat 11(2) (June 1990). This led to an article by Karl Berry in TUGboat 11(4) (November pages 512-519).

Karl Berry then suggested a system---with many limitations, but perhaps the best that could have been done in its time, for mapping a lengthy font name into a file name that was eight or fewer characters long. If the font files are renamed accordingly then we can deduce the nature of the font by examining its file name. The scheme did not apply to the original Computer Modern fonts that retained their original names \citep{fontname}.

This scheme assumes that only eight characters or fewer can be available for naming the font. These eight characters look like,

\begin{verbatim}
FNNW[S][V]7V
\end{verbatim}

Some additional comments on this shorthand notation is in order. 
The most common foundry abbreviations are |p| for Adobe (from PostScript), \textbf{b} for BitStream, and \textbf{m} Monotype. A font flouting this scheme will begin with a z.

The next two letters are reserved for the typeface name. The hundreds and hundred of available faces guarantees  that many of these will be cryptic, even for the most common typefaces---Adobe Garamond is |ad|. 



\section{Using fonts with XeTeX based engines}

Depending on the fonts in your system, some features that are described here, might not be available.

\begin{texexample}{}{}
\ifxetex
  %\usepackage{fontspec}
  %\defaultfontfeatures{Mapping=tex-text}
  %\setmainfont{Times New Roman}
  %\setsansfont{Myriad Pro}
  Running XeTeX
\else
  \ifluatex
  %\usepackage{lmodern}
  %\usepackage[T1]{fontenc}
    Running LuaTeX
  \else
    Running pdfLaTeX
  \fi
\fi
\end{texexample}

For free fonts there exist a few resources that can be used with \LaTeX.
\url{http://tex.stackexchange.com/questions/53416/using-a-good-non-default-font}. Integrating them within a new document can be a nightmare but is the job of the class and book designer.

\section{Terminology}

The best source of information for XeTeX is the \ctan{fontspec} manual. It is not an easy read, but if you are going to be resetting a lot of fonts, it is advisable to do so.

Most typesetting systems allow for setting document wide fonts. In \latexe we get the following commands:


\cs{sffamily}

\cs{rmfamily}

\cs{ttfamily}

To be able to use the |phd| package properly you will have to familiarize yourself with the terminology, if you are not.

\index{CSS}
\texttt{CSS} uses a combination of font-family and fallback generic families to achieve this and it is instructive to review it as we will a similar system here.

\begin{tcolorbox}
\begin{lstlisting}
p{font-family:"Times New Roman", Georgia, Serif;}
\end{lstlisting}
\end{tcolorbox}

The font-family property specifies the font for an element.

The font-family property can hold several font names as a "fallback" system. If the browser does not support the first font, it tries the next font.

There are two types of font family names:

family-name - The name of a font-family, like "times", "courier", "arial", etc.

generic-family - The name of a generic-family, like "serif", "sans-serif", "cursive", "fantasy", "monospace".

There is though a fundamental difference that one needs to keep in mind, \TeX\ exists in order to always typeset the same on any machine. CSS endeavours to run in any browser and any system, disregarding typography. Nevertheless I decided to provide the interface so at least as to enable document compilation at all times, well almost all times.


\section{General font selection with fontspec}

\begin{trivlist}
\item [\cs{fontspec}\oarg{font features}\marg{font name}]
\item [\cs{setmainfont}\oarg{font features}\marg{font name}]
\item [\cs{setsansfont}\oarg{font features}\marg{font name}]
\item [\cs{setmonofont}\oarg{font features}\marg{font name}]
\item [\cs{newfontfamily}\marg{cmd}\oarg{font features}\marg{font name}]
\end{trivlist}

These are the main font-selecting commands of this package. The \cs{fontspec}
command selects a font for one-time use; all others should be used to define the
standard fonts used in a document. They will be described later in this section.
The font features argument accepts comma separated \marg{font feature}=\marg{option}
lists; these are described in later:

\ifxetex
\begin{texexample}{}{}
\bgroup
\fontspec{Verdana}
\raggedright
\knutext

\newfontfamily\calibri{Calibri}
  \calibri 


\def\setchapterfont{\calibri\huge}

\textsf{\large \lorem}
\egroup
\end{texexample}
\fi

\begin{verbatim}
\DeclareTextFontCommand{\textsf}{\calibri}
\end{verbatim}

\subsection{fontspec commands to select font families}

In many cases there is only a need to define a new font for specific case, for example only for a chapter head. It is 

\CMDI{\newfontfamily}\marg{font-switch}\oarg{font features}\marg{font name}

For cases when a specific font with a specific feature set is going to be re-used
many times in a document, it is inefficient to keep calling \cs{fontspec} for every use. For this reason, new commands can be created for loading a particular font family

While the \cs{fontspec} command does not define a new font instance after the first
call, the feature options must still be parsed and processed.
\cs{newfontfamily}. The example that follows, defines a new font family to be used only for chapterheads. This is more efficient and also provides a semantic interface for the author.

\begin{texexample}{newfontfamily}{ex:newfontfamily}
 
\newfontfamily\calibri{Calibri}
\def\setchapterfont{%
   \calibri\huge\bfseries}

\bgroup
\setchapterfont CHAPTER 10
\egroup
\end{texexample}

\begin{teX}
15 \DeclareTextFontCommand{\textrm}{\rmfamily}
16 \DeclareTextFontCommand{\textsf}{\sffamily}
17 \DeclareTextFontCommand{\texttt}{\ttfamily}
18 \DeclareTextFontCommand{\textnormal}{\normalfont}
\end{teX}

\subsection{Setting font features}
\index{fontspec>font features}

The \pkgname{fontspec} package enables the selection of font features during run-time; font features are items such as colors, proportional OldStyle numbers and other similar items. Some of the examples that follow have been extracted from the fontspec documentation.

\ifxetex\else\if\luatex
\begin{texexample}{}{}
\fontspec[Numbers={Proportional,OldStyle}]
{TeX Gyre Adventor}
`In 1842, 999 people sailed 97 miles in
13 boats. In 1923, 111 people sailed 54
miles in 56 boats.' \bigskip

\fontspec{TeX Gyre Adventor}
`In 1842, 999 people sailed 97 miles in
13 boats. In 1923, 111 people sailed 54
miles in 56 boats.' \bigskip
\end{texexample}

Accessing Raw Features explicitly is perhaps better suited to people that like to investigate under the hood and can also provide a clearer way to understand what is going on.
 
\begin{texexample}{}{}
\fontspec[RawFeature=+onum;+zero]{TeX Gyre Adventor}
`In 1842, 999 people sailed 97 miles in
13 boats. In 1923, 1110 people sailed 54
miles in 56 boats.' \bigskip

\fontspec[RawFeature=+tnum;+zero]{TeX Gyre Adventor}
00001761 tabular figures \fox \bigskip

\fontspec[RawFeature=+pnum;+zero]{TeX Gyre Adventor}
00001761 proportional figures \bigskip

\fontspec[RawFeature=+onum;+zero]{TeX Gyre Adventor}
00001761 old numerals\bigskip

\fontspec[RawFeature=+lnum;+zero]{TeX Gyre Adventor}
00001761 lining figures\bigskip
\end{texexample}

Not all \OpenType\footnote{See \protect\url{https://www.microsoft.com/typography/otspec/featurelist.htm}} fonts provide all font features and some will only work in combination with others, for example the |onum| feature will only work with the |pnum| number features. Some experimentation and viewing the font features with a utility is essential.

\fi\fi


\section{The phd package interface.}

By design feature options for XeTeX/XeLaTeX have currently been restricted. The reason behind this decision is that I was concerned that I would have added a complicated interface with very little reason as to its use. I opted for a more semantic approach and expect the user to define custom macros to handle anything else.
\medskip

\keyval{mainfont}{\marg{font1,font2,font3}}{A comma separated list of one or more font-names. The main font will be set to the first font found.}
\keyval{chapterfont}{\marg{font1,font2,font3}}{A comma separated list of one or more font-names. The main font will be set to the first font found.}
\keyval{sectionfont}{\marg{font1,font2,font3}}{A comma separated list of one or more font-names. The main font will be set to the first font found.}
\keyval{contentsfont}{\marg{font1,font2,font3}}{A comma separated list of one or more font-names. The main font will be set to the first font found.}
\keyval{bibliographyfont}{\marg{font1,font2,font3}}{A comma separated list of one or more font-names. The main font will be set to the first font found.}

Note that the package will first check if is running under XeTeX. If it does it will execute the commands and load the macros, otherwise it will fall back on pdfLaTeX commands.

\section{Viewing and selecting fonts}

\subsection*{\textsf{\color{Headings}Typefaces that come with the
standard \LaTeX\ distribution}}
{
\raggedright
\begin{tabular}{@{}>{\sffamily\bfseries}rl}
\fonttitle{Computer Modern (CM), \LaTeX's default typeface}
\thefont{CM Roman}{cmr}{\sample}
\thefont{CM Italic}{cmr}{\itshape\sample}
\thefont{CM Slanted (Oblique)}{cmr}{\slshape\sample}
\thefont{CM Bold}{cmr}{\fontseries{b}\selectfont\sample}
\thefont{CM Bold Extended}{cmr}{\bfseries\sample}
\thefont{CM Bold Italic}{cmr}{\itshape\bfseries\sample}
\thefont{CM Bold Slanted}{cmr}{\slshape\bfseries\sample}
\thefont{CM Caps \& Small Caps}{cmr}{\scshape\sample}
\thefont{CM Sans-Serif}{cmss}{\sample}
\thefont{CM Sans-Serif Oblique}{cmss}{\itshape\sample}
\thefont{CM Sans-Serif Bold}{cmss}{\bfseries\sample}
\thefont{CM Typewriter}{cmtt}{\sample}
\thefont{CM Typewriter Italic}{cmtt}{\itshape\sample}
\thefont{CM Typewriter Bold}{cmtt}{\bfseries\sample}
\thefont{CM Typewriter C\&SC}{cmtt}{\scshape\sample}
\thefont[OMS]{CM Mathematics}{cmsy}{$E=mc^2$\qquad}
\thefont{CM `Dunhill'}{cmdh}{\sample}
\thefont{CM `Fibonacci'}{cmfib}{\sample}
\end{tabular}
}\index{fonts>Fibonacci}\index{fonts>Dunhill}
\section{Discussion}


Unfortunately, even with the best will loading fonts will always be a difficult task in TeX. Hopefully the interface provided will result in better separation of presentation from content and offers consistency in the styling of documents. Nothing prevents you from adding normal macros to styles. Each style can be treated as a package in many respects.



\section{XeLaTeX and LuaLaTeX}



\bgroup
\fontspec{Verdana}
\begin{minipage}[t]{.2\linewidth}
\hbox to \linewidth{\hfill\hfill Verdana\hspace{2em}}
\end{minipage}
\begin{minipage}[t]{.75\linewidth}
^^A\addfontfeature{ItalicFeatures={Alternate = 1}}
\noindent\fox\\
\alphabet\\
\punctuation\\
\frogking
\end{minipage}
\egroup

\bgroup
\fontspec{Calibri}
\begin{minipage}[t]{.2\linewidth}
\hbox to \linewidth{\hfill\hfill Calibri\hspace{2em}}
\end{minipage}
\begin{minipage}[t]{.65\linewidth}
^^A\addfontfeature{ItalicFeatures={Alternate = 1}}
\noindent\fox\\
\alphabet\\
\textsc{\alphabet}\\
\punctuation\\
\frogking
Θαμκυαμ πλαθονεμ ραθιονιβυς ναμ ει, δυις περπετυα σιθ αδ, νες ιδ δισυντ σοντεντιωνες. Κυι σινθ μυνδι εα, φιμ αν γραεσω ιυδισαβιτ, εραθ δολορ φιρθυθε υθ δυο. Συ νοσθερ οπθιων ευμ, μει ερος προβο φιερενθ ευ. Ιυς μανδαμυς τωρκυαθος εξπεθενδις ιδ. Σεδ θε νιβχ νονυμυ δελισαθισιμι, φιμ νο νιβχ λαβωραμυς, σεα εα δισο ποσιμ αντιωπαμ.
\end{minipage}
\egroup

\section{Utilities for testing fonts}

The package \pkg{fonttable} is an extension and re-implementation of Donald Knuth’s testfont.tex, which
is available from CTAN. The package was developed by Peter Wilson and currently maintained by Will Robertson \citep{fonttable}. It provides a number of utility commands for typesetting font tables.


The {fonttable}\marg{font} takes the font file as an  argument and typesets it in a nice table. 

% \ifengine
%\ifxetex
% \else
%  \fonttable{pzdr}
%\fi

A great tool to inspect a True Type font on the command line is \texttt{luaotfload-tool}:

\begin{verbatim}
  luaotfload-tool --find="Iwona" --inspect
\end{verbatim}

\section{ \texttt{.tfm } files}


When you tell \tex that you will be using a particular font, it has to find out information about that font. This information is stored in a file with the extension \docfileextension{.tfm}. For example when you say:

|\font a=cmr10|

\noindent \tex looks for  a file named |cmr10.tfm|. If this is not found then an error is issued |Lookup failed on file CMR10.TFM|

Generall speaking, a font's |.tfm| file contains information about the height, width and depth of all the characters in the font plus kerning and ligature information. So, |cmr10.tfm| might say that the lower-case "d" in CMR10 is 5.5 points wide, 6.94 points high, etc. This is the information that \tex uses to make its lowest-level boxes---those around characters. See the \tex manual for information about what \tex does with these boxes. Note the |.tfm| files do not contain any information that is device dependent. Only device-drivers read \tex's |dvi| output files can use that sort of information.


\section{Fonts for Far East Languages}

In internationalization, CJK is a collective term for the Chinese, Japanese, and Korean languages, all of which use Chinese characters and derivatives (collectively, CJK characters) in their writing systems. Occasionally, Vietnamese is included, making the abbreviation CJKV, since Vietnamese historically used Chinese characters as well.
The characters are known as hànzì in Chinese, kanji in Japanese, hanja in Korean, and Chữ Nôm in Vietnamese.\index{kanji}\index{hanja}\index{hànzì}\index{CJK}\index{CJKV}


\subsection{Selecting a font}

The easiest way is with Will Robertson's \pkgname{fontspec} package. In this sample, we have used the \texttt{SimSun} font, which can be found on windows machines:

\begin{verbatim}
\usepackage{fontspec}
\setromanfont{SimSun}
\end{verbatim}
in the preamble, to use a Far Eastern font as the initial default typeface.

\section{Entering CJK text}

You can just enter Unicode text directly in the document: 你好. Don't use legacy \LaTeX\ packages such as \verb|inputenc| or \verb|CJK|, as \XeTeX\ reads the text as Unicode characters, not the separate byte codes of UTF-8 sequences, and passes them directly to the Unicode font. (Actually, it would probably be possible to use \verb|\XeTeXinputencoding "bytes"| and work with legacy \LaTeX\ input encoding support. But then you're pretty much committed to all the old encoding and font machinery, and there's not much point in using the \XeTeX\ engine at all.)

\begin{comment}
\setromanfont{Times New Roman}

\subsection{A CJK environment}

\newenvironment{CJK}{\fontspec[Scale=0.9]{SimSun}}{}

\newcommand{\cjk}[1]{{\fontspec[Scale=0.9]{SimSun}#1}}

Rather than selecting a CJK font as the main document typeface, you might want to define a CJK environment for text fragments used in the midst of a document using a normal Roman font. This allows me to say \verb|\begin{CJK}東光\end{CJK}| to generate \begin{CJK}東光\end{CJK}, without putting the whole paragraph into the Far Eastern font. Or I could define a command that takes the CJK text as an argument, so that \verb|\cjk{北京}| produces \cjk{北京}. It's that easy! Such an environment can easily be set using the \cmd{newfamily} or \cmd{\fontspec}.

\begin{verbatim}
\newenvironment{CJK}{\fontspec[Scale=0.9]{SimSun}}{}
\newcommand{\cjk}[1]{{\fontspec[Scale=0.9]{SimSun}#1}}
\end{verbatim}
\end{comment}


\normalfont

\section{Changing the font size in LaTeX}

\index{fonts>sizing commands}

Changing the font size in LaTeX can be done at two levels, 
affecting the whole document or elements with in it. 
Using a different font size on a
global level will affect all normal-sized text as well
as the sizes of headings, footnotes, etc. By changing
the font size locally, however, a single word, a few
lines of text, a large table, or a heading throughout
the document may be modified. Fortunately, there is
no need for the writer to juggle with numbers when
doing so. \latex provides a set of macros for changing
the font size locally, taking into consideration the
document’s global font size. \citep{thurnherr2012}

A number of packages exist \ctan{moresize} \citep{moresize},
 \ctan{anyfont} \citep{anyfont}. The standard classes, memoir
 KOMA classes and most journals also come with their own
 defined sizing commands.
 
 

















%\makeatletter\@specialfalse
\cxset{custom = stewart}
\cxset{steward,
  numbering=arabic,
  custom=stewart,
  offsety=0cm,
  image={./images/hine03.jpg},
  texti={When Lamport designed the original \LaTeX\ sectioning commands he did not provide a fully comprehensive interface for modifying their design. With current tools available improvements are much easier to program and this chapter provides the details.},
  textii={\precis{In this chapter we discuss a method that allows the production of fancy chapter headings and formatting, based on a set of key values. Central  to this process is the separation of content from presentation.
We also discuss the basic formatting tools that are available and how one can modify them to mould new book designs.}
 }
}




\cxset{section align=left}
\chapter{Epigraphs}\index{epigraphs}
\epigraph{Example is the school of mankind, and they
will learn at no other.}{Letters on a Regicide Peace}



\section{Introduction}

Epigraphs or quotations before or after chapters are quite common in books. Peter Wilson's epigraph package, 
does a good job and we have adapted it where necessary to allow for a key value interface. The command:

\cs{epigraph}\marg{text}\marg{source}. By default the epigraph is placed at the right
hand side of the textblock, and the \marg{source} is typeset at the bottom right of the \marg{text}. 
Numerous settings allow for manipulating the width of the epigraph, the location and other 
variables. If the package is available we use it otherwise we use other internal commands.



\section{Key-value interface}
The key value interface provided by the package is shown below. It mostly follows the 
naming conventions of the epigraph package to make the transition easier for experienced users.
\medskip

\keyval{epigraph align}{\marg{left, center, right}}{A font-size command such as \cs{footnotesize}, 
\cs{small} and other similar commands.}

\keyval{epigraph rule width}{\marg{dim}}{A font-size command such as \cs{footnotesize}, \cs{small} 
and other similar commands.}

\keyval{epigraph font-size}{\marg{dim}}{A font-size command such as \cs{footnotesize}, \cs{small} and 
other similar commands.}

\keyval{epigraph beforeskip}{\marg{dim}}{Space before the epigraph.}
\keyval{epigraph afterskip}{\marg{dim}}{Space after the epigraph.}
\keyval{epigraph source align}{\marg{left, center, right}}{Align the source text to the right, left or center.}
\keyval{epigraph source font-size}{\marg{dim}}{Align the source text to the right, left or center.}
\keyval{epigraph source font-shape}{\marg{dim}}{Align the source text to the right, left or center.}
\keyval{epigraph source font-family}{\marg{dim}}{Align the source text to the right, left or center.}
\keyval{epigraph source font-weight}{\marg{bold,normal}}{Align the source text to the right, left or center.}


\section{Example usage}
To set the style and an example usage is shown in .

\begin{example}{epigraph example}{}
\cxset{epigraph width=0.5\linewidth,
       epigraph font-size=\small,
       epigraph rule width=0.4pt,
       epigraph align=right,
       epigraph source align=right,
       epigraph text align=right}


\epigraph{Example is the school of mankind, and they
will learn at no other.}{Letters on a Regicide Peace}
\end{example}

Another example for a somewhat longer quote:

\begin{example}{epigraph example}{}
\cxset{epigraph width=0.5\linewidth,
          epigraph font-size=\small,
          epigraph rule width=0.4pt,
          epigraph align=left,
          epigraph source align=right,
          epigraph text align=left}

\epigraph{Everything written with vitality expresses that vitality; there are no dull subjects, 
only dull minds.}{Raymond Chandler\\\textit{Letters on a Regicide Peace}}
\end{example}

More usage examples can be found in relevant style examples (See Chapter~\ref{ch:41}) for a rather 
nice example with non-traditional alignment.

\section{Epigraphs on empty pages}

When a chapter open on an odd page sometimes the  previous page is left empty. Some book designers 
add the words ``this page left intentionally blank'' and other might add a quote. To add such a quote use:

\begin{tcolorbox}
\begin{lstlisting}
\cxset{blank page text=\epigraph{The great tragedy of science is the slaying of a beautiful theory
by an ugly fact.}{Thomas Huxley}}
\end{lstlisting}
\end{tcolorbox}

%%%%%%%%%%%%%%%%%%%%%%%%%%%%%%%%%%%%%%%%%%%%%%%%%%
%  SECTIONING COMMANDS
\@specialtrue
\cxset{steward,
  numbering=arabic,
  custom=tikzspecial,
  offsety=0cm,
  image=hine03,
  texti={When Lamport designed the original \LaTeX\ sectioning commands, limitations of computer power forced him to restrict the abstraction of complicated chapter layouts. With current tools available improvements are much easier to program.},
%
  textii={In this chapter we discuss a method that allows the production of fancy chapter headings and formatting, based on a set of key values. Central  to this process is the separation of content from presentation.
We also discuss the basic formatting tools that are available and how one can modify them to mould new book designs.
 }
}


%
\@specialtrue
\cxset{steward,
  numbering=arabic,
  custom=stewart,
  offsety=0cm,
  image=hine03,
  texti={When Lamport designed the original \LaTeX\ sectioning commands, limitations of computer power forced him to restrict the abstraction of complicated chapter layouts. With current tools available improvements are much easier to program.},
%
  textii={In this chapter we discuss a method that allows the production of fancy sectionr headings and formatting, based on a set of key values. Central  to this process is the separation of content from presentation.
We also discuss the basic formatting tools that are available and how one can modify them to mould new book designs.
 }
}



\raggedbottom

\chapter{Lower Level Headings}
\@specialfalse

\section{Introduction}

Good book design dictates that sectioning styles follow that of the general book design and theme. An academic publication for example might have chapters and section numbered in arabic numerals, whereas a high school textbook might have sections marked in colored boxes.

Similarly to the chapter key value interface, the package offers a key value interface to adjust sectioning command parameters.



\cxset{section beforeskip={10pt},
      section indent=0pt}
\cxset{section afterskip={10pt}}
\renewsection

\section{Section styling}

In a similar fashion to the chapter commands the following keys are provided.

\subsection{Fonts and numerals}

Font and numeral keys are shown below.
\medskip

  \keyval{section font-size}{\marg{cmd}}{Font size command such as \cs{large.}}
  \keyval{section font-weight}{\marg{cmd}}{Font weight command such as \cs{bfseries.}}
  \keyval{section font-family}{\marg{cmd}}{Font family command such as \cs{sffamily.}}
  \keyval{section font-shape}{\marg{cmd}}{Font shape command such as \cs{itshape}}
  \keyval{section color}{\marg{color}}{Color of section.}
  \keyval{section numbering}{\marg{arabic|roman|Roman|alph|Alph|words|WORDS}}{Section number style.}
  \begin{marglist}
  \item [arabic] Typesers the section number in arabic numerals.
  \item [roman] Typesets the section number in lowercase roman numerals.
  \item [Roman] Typesets the section number in uppercase roman numerals.
  \item [alph] Typesets the section number in lowercase alphabetic numbering.
  \item [Alph] Typesets the section number in uppercase alphabetic numerals.
  \item [words] Typesets the numbers in words (lowercase).
  \item [WORDS] Typesets the number in words (uppercase).
  \end{marglist}

\subsection{Skip and indentation commands}

The keys for indentaion and above and below skips are shown below.
\medskip

\keyval{section beforeskip}{}{}
\keyval{section afterskip}{}{}
\keyval{section indent}{\marg{dim}}{Indentation from margin as per standard LaTeX class definitions.}
\keyval{section spaceout}{}{}
\begin{marglist}
 \item[soul]
 \item[none]
\end{marglist}

\subsection{align}

\keyval{section align}{\marg{cmd}}{One of the alignment commands centering, ragged right, raggedleft}

\subsection{Hooks}

Hooks for adding material are shown in the following sketch.
\medskip

\fbox{aboveskip}

\fbox{indent} \fbox{number}\fbox{hook}\fbox{title}

\fbox{belowskip}

%\lipsum

\section{Example usage}

\cxset{
 chapter toc=false,
 name=CHAPTER,
 numbering=arabic,
 number font-size=\huge,
 number font-family=\sffamily,
 number font-weight=\bfseries,
 number before=,
 number dot=,
 number after=\hspace{1em},
 number position=rightname,
 chapter opening=anywhere,
 chapter font-family=\sffamily,
 chapter font-weight=\bfseries,
 chapter font-size=\huge,
 chapter before={\vspace*{0.1\textheight}\hfill},
 chapter after={\hfill\hfill\vskip0pt\thinrule\par},
 chapter color={black!90},
 number color=\color{black!90},
 title beforeskip={\vspace*{30pt}},
 title afterskip={\vspace*{30pt}\par},
 title before={\hfill},
 title after={\hfill\hfill},
 title font-family=\sffamily,
 title font-color=\color{black!90},
 title font-weight=\bfseries,
 title font-size=\huge,
%%%%%%%%%% Sections
 section font-size=\LARGE,
 section font-weight=\normalfont,
 section font-family=\sffamily,
 section align=\centering,
 section numbering=arabic,
 section indent=0em,
 section align=\centering,
 section beforeskip=20pt,
 section afterskip=10pt,
 section spaceout=soul,
 section font-shape=\itshape,
}
\cxset{book/.style={
 section numbering=arabic,
 section font-size=\Large,
 section font-weight=\bfseries,
 section font-family=\rmfamily,
 section font-shape=\normalfont,
 section align=\raggedright,
 %section numbering custom=\color{gray}{Section} (\thechapter-\@arabic\c@section),
 subsection font-size=\large
 section indent=0em,
 section beforeskip=-3.5ex \@plus -1ex\@minus -0.2ex,
 section afterskip=2.3ex\@plus.2ex,
 subsection beforeskip=-3.5ex \@plus -1ex\@minus -0.2ex,
 subsection afterskip= 1.5ex \@plus .2ex,
}}


\begin{example}{Adjusting section parameters}{}
\cxset{ section font-size=\LARGE,
 section font-weight=\normalfont,
 section font-family=\sffamily,
 section align=\centering,
 section numbering=(roman),
 section indent=0em,
 section align=\centering,
 section beforeskip=20pt,
 section afterskip=10pt,}
\chapter{A First Look at the Sectioning Keys}
\section{First section}
\lorem
\end{example}

One notable thing to keep in mind is that the numbering of the chapter is independent of that for the section, so if you need to have strange combinations rather define a section numbering custom.\index{section formatting!vertical space}

\cxset{section numbering=arabic}
\subsection{Adjusting vertical spaces}

Perhaps the most important issues we need to consider is the adjusting of vertical spaces; example~\ref{ex:latex}, that follows illustrates settings from the Octavo class and compare them with those of standard the \LaTeXe\ book class. The Octavo class through settings that are based on baselineskip fractions and multiples endeavours to achieve a grid layout. The class also tones down the `loudness' of some of the headings compared to those of the book class.


\cxset{octavo/.style={
 section font-size=\large,
 section font-weight=\normalfont,
 section font-family=\rmfamily,
 section font-shape=\scshape,
 section indent=0em,
 section align=\centering,
 section beforeskip=-1.666\baselineskip\@minus -2\p@,
 section afterskip=0.835\baselineskip \@minus 2\p@,
 subsection numbering=none,
 subsection font-family=\rmfamily,
 subsection font-size=\normalfont,
 subsection font-shape=\scshape,
 subsection font-weight=\normalfont,
 subsection indent=1em,
 subsection align=\raggedright,
 subsection beforeskip=-0.666\baselineskip\@minus -2\p@,
 subsection afterskip=0.333\baselineskip \@minus 2\p@
 }}




\cxset{book/.style={
 section numbering=arabic,
 section font-size=\Large,
 section font-weight=\bfseries,
 section font-family=\rmfamily,
 section font-shape=\normalfont,
 section align=\raggedright,
 %section numbering custom=\color{gray}{Section} (\thechapter-\@arabic\c@section),
 subsection font-size=\large,
 section indent=0em,
 section beforeskip=-3.5ex \@plus -1ex\@minus -0.2ex,
 section afterskip=2.3ex\@plus.2ex,
 subsection font-size=\large,
 subsection font-weight=\bfseries,
 subsection numbering=arabic,
 subsection indent=0pt,
 subsection beforeskip=-3.5ex \@plus -1ex\@minus -0.2ex,
 subsection afterskip= 1.5ex \@plus .2ex,
}}

\cxset{octavo headings/.style={%
 section numbering=none,section font-size=\large,section font-weight=\normalfont,
 section font-family=\rmfamily, section font-shape=\scshape,
 section indent=0em, section align=\centering, section beforeskip=-1.666\baselineskip\@minus -2\p@,
 section afterskip=0.835\baselineskip \@minus 2\p@, subsection numbering=none,
 subsection font-family=\rmfamily, subsection font-size=\normalfont, subsection font-shape=\scshape,
 subsection font-weight=\normalfont, subsection indent=1em, subsection align=\raggedright,
 subsection beforeskip=-0.666\baselineskip\@minus -2\p@,
 subsection afterskip=0.333\baselineskip \@minus 2\p@,
 subsubsection numbering=none,
 subsubsection font-family=\rmfamily,
 subsubsection font-size=\normalfont,
 subsubsection font-shape=\itshape,
 subsubsection font-weight=\normalfont,
 subsubsection indent=1em,
 subsubsection align=\raggedright,
 subsubsection beforeskip=-0.666\baselineskip\@minus -2\p@,
 subsubsection afterskip=0.333\baselineskip \@minus 2\p@,
 paragraph numbering=none,
 paragraph font-family=\rmfamily,
 paragraph font-size=\normalfont,
 paragraph font-shape=\normalfont,
 paragraph font-weight=\normalfont,
 paragraph indent=-1em,
 paragraph align=\raggedright,
 paragraph beforeskip=\z@,
 paragraph afterskip=0\p@,
% subparagraph numbering=none,
% subparagraph font-family=\rmfamily,
% subparagraph font-size=\normalfont,
% subparagraph font-shape=\normalfont,
% subparagraph font-weight=\normalfont,
% subparagraph indent=0em,
% subparagraph align=\raggedright,
% subparagraph beforeskip=\z@,
% subparagraph afterskip=0\p@,
}}
\cxset{octavo headings}
\renewsection\renewsubsection\renewsubsubsection\renewparagraph

\begin{example}{Octavo class headings, settings}{}
\cxset{octavo headings/.style={%
 section numbering=none,section font-size=\large,section font-weight=\normalfont,
 section font-family=\rmfamily, section font-shape=\scshape,
 section indent=0em, section align=\centering, section beforeskip=-1.666\baselineskip\@minus -2\p@,
 section afterskip=0.835\baselineskip \@minus 2\p@, subsection numbering=none,
 subsection font-family=\rmfamily, subsection font-size=\normalfont, subsection font-shape=\scshape,
 subsection font-weight=\normalfont, subsection indent=1em, subsection align=\raggedright,
 subsection beforeskip=-0.666\baselineskip\@minus -2\p@,
 subsection afterskip=0.333\baselineskip \@minus 2\p@,
 subsubsection numbering=none,
 subsubsection font-family=\rmfamily,
 subsubsection font-size=\normalfont,
 subsubsection font-shape=\itshape,
 subsubsection font-weight=\normalfont,
 subsubsection indent=1em,
 subsubsection align=\raggedright,
 subsubsection beforeskip=-0.666\baselineskip\@minus -2\p@,
 subsubsection afterskip=0.333\baselineskip \@minus 2\p@,
 paragraph numbering=none,
 paragraph font-family=\rmfamily,
 paragraph font-size=\normalfont,
 paragraph font-shape=\normalfont,
 paragraph font-weight=\normalfont,
 paragraph indent=-1em,
 paragraph align=\raggedright,
 paragraph beforeskip=\z@,
 paragraph afterskip=0\p@,}}

\cxset{octavo headings}
\renewsection\renewsubsection\renewsubsubsection\renewparagraph
\section{Octavo Class Heading}
\lorem
\subsection{Octavo subsection}
This is some text short text\par
\subsubsection{Octavo sub-subsection}
\lorem
\paragraph{paragraph heading} This is some short text.
\end{example}

\begin{example}{}{}
\cxset{octavo}
\section{Octavo Class Heading}
\lorem
\subsection{Octavo subsection}
\lorem
\subsubsection{Octavo sub-subsection}
\lorem
\paragraph{paragraph heading} This is some short text.
\lorem
\paragraph{paragraph heading} This is some short text.
\lorem
\end{example}



\begin{example}{\LaTeXe\ book class headings settings}{ex:latex}
\cxset{book/.style={
 section numbering=arabic,
 section font-size=\Large,
 section font-weight=\bfseries,
 section font-family=\rmfamily,
 section font-shape=\normalfont,
 section align=\raggedright,
 %section numbering custom=\color{gray}{Section} (\thechapter-\@arabic\c@section),
 subsection font-size=\large,
 section indent=0em,
 section beforeskip=-3.5ex \@plus -1ex\@minus -0.2ex,
 section afterskip=2.3ex\@plus.2ex,
 subsection font-size=\large,
 subsection font-shape=\normalfont,
 subsection font-weight=\bfseries,
 subsection numbering=arabic,
 subsection indent=0pt,
 subsection beforeskip=-3.5ex \@plus -1ex\@minus -0.2ex,
 subsection afterskip= 1.5ex \@plus .2ex,
}}
\cxset{book}
\renewsubsection
\section{LaTeX Book  Class Heading}
\lorem
\subsection{A subsection}
\lorem
\end{example}

\section{Grid example}

One problem sometimes is that the sectioning commands create problems with grid layouts. Example~\ref{ex:grid} shows example settings.

\begin{example}{Section styles from the grid package}{ex:grid}
\cxset{grid/.style={
 section numbering=arabic,
 section font-size=\normalsize,
 section font-weight=\bfseries\mathversion{bold},
 section font-family=\rmfamily,
 section font-shape=\normalfont\bfseries\mathversion{bold},
 section beforeskip=-.999\baselineskip,
 section afterskip=0.001\baselineskip,
 section align=\raggedright,
 %section numbering custom=\color{gray}{Section} (\thechapter-\@arabic\c@section),
 subsection font-size=\normalsize,
 section indent=0em,
% section beforeskip=-3.5ex \@plus -1ex\@minus -0.2ex,
 %section afterskip=2.3ex\@plus.2ex,
 subsection font-shape=,
 subsection font-weight=\bfseries\mathversion{bold},
 subsection numbering=arabic,
 subsection indent=0pt,
 subsection beforeskip=\baselineskip,
 subsection afterskip= -.35\baselineskip,
% subsub section
 subsubsection font-shape=\itshape,
 subsubsection font-weight=\bfseries\mathversion{bold},
 subsubsection numbering=numeric,
 subsubsection indent=0pt,
 subsubsection beforeskip=\baselineskip,
 subsubsection afterskip= -.35\baselineskip,
}}
\cxset{grid}
\renewsubsection
\begin{multicols}{2}
\section{Grid  Class Heading}
\lorem
\subsection{Grid  subsection.}
\lorem
\subsubsection{A subsection grid.}
\lorem
\subsubsection{Another subsection grid.}
\lorem
\end{multicols}
\end{example}



The key \option{\bfseries section numbering custom}=\marg{code} is quite powerfull and can be used to define any type of section number style. Just remember that the numbering so far depends on two counters, the c@chapter and c@section. What the section numbering does, it redefines the macro \cs{thesection} to the new definition provided as argument for the key.

Although the temptation to define a lot of key combinations one would rather define new styles as a more user friendly approach.

\cxset{section numbering=arabic, section align=\raggedright, section font-shape=\upshape, section font-family=\rmfamily}
\section{Handling Other Section Levels}

Other sectioning commands such as \cs{subsubsection}, \cs{paragraph} and \cs{subparagraph} have equivalent keys. Examples can be found in the chapters that follow for specific styles.

\section{Technical discussion}

The standard LaTeX classes, book report and article have sections showing dot leaders, whereas in the article class the sections are shown without the dotted lines, as the l@section macro is redefined for articles.

\index{macros!\textbackslash @seccntformat}

\subsection{Indexing of Lower Section Headings}
\LaTeXe\ offers two pathways in redefining section commands, the first one is @startsection and the second is \cs{@seccntformat} \index{sectioning macros}. It also uses the macro \cs{secdef} to create the starred and unstarred versions of the sectioning commands.

\begin{tcolorbox}{}
\begin{lstlisting}
% \begin{macro}{\l@section}
%    In the article document class the entry in the table of contents
%    for sections looks much like the chapter entries for the report
%    and book document classes.
%
%    First we make sure that if a pagebreak should occur, it occurs
%    \emph{before} this entry. Also a little whitespace is added and a
%    group begun to keep changes local.
% \changes{v1.0h}{1993/12/18}{Replaced -\cs{@secpenalty} by
%    \cs{@secpenalty}.  ASAJ.}
% \changes{v1.2i}{1994/04/28}{Don't print a toc line when the tocdepth
%    counter is less than 1.}
% \changes{v1.4a}{1998/10/12}{we should use \cs{@tocrmarg}; see PR/2881.}
%    \begin{macrocode}
%<*article>
\newcommand*\l@section[2]{%
  \ifnum \c@tocdepth >\z@
    \addpenalty\@secpenalty
    \addvspace{1.0em \@plus\p@}%
%    \end{macrocode}
%
%    The macro |\numberline| requires that the width of the box that
%    holds the part number is stored in \LaTeX's scratch register
%    |\@tempdima|. Therefore we put it there. We begin a group, and
%    change some of the paragraph parameters (see also the remark at
%    \cs{l@part} regarding \cs{rightskip}).
%    \begin{macrocode}
    \setlength\@tempdima{1.5em}%
    \begingroup
      \parindent \z@ \rightskip \@pnumwidth
      \parfillskip -\@pnumwidth
%    \end{macrocode}
%    Then we leave vertical mode and switch to a bold font.
%    \begin{macrocode}
      \leavevmode \bfseries
%    \end{macrocode}
%    Because we do not use |\numberline| here, we have do some fine
%    tuning `by hand', before we can set the entry. We discourage but
%    not disallow a pagebreak immediately after a section entry.
%    \begin{macrocode}
      \advance\leftskip\@tempdima
      \hskip -\leftskip
      #1\nobreak\hfil \nobreak\hb@xt@\@pnumwidth{\hss #2}\par
    \endgroup
  \fi}
%</article>
\end{lstlisting}
\end{tcolorbox}

As you can see the dot leaders are not present in the above definition. Although we can get rid of dot leaders in other section by redefining them, it is not as easy to add them back.

As our aim is to be able to have all the classes used a common denominator we can define a command as follows (using book as a base)

\begin{tcolorbox}{}
\begin{lstlisting}
\def\articlesection{
\newcommand*\l@section[2]{%
  \ifnum \c@tocdepth >\z@
    \addpenalty\@secpenalty
    \addvspace{1.0em \@plus\p@}%
    \setlength\@tempdima{1.5em}%
    \begingroup
      \parindent \z@ \rightskip \@pnumwidth
      \parfillskip -\@pnumwidth
      \leavevmode \bfseries
      \advance\leftskip\@tempdima
      \hskip -\leftskip
      #1\nobreak\hfil \nobreak\hb@xt@\@pnumwidth{\hss #2}\par
    \endgroup
  \fi}
}
\end{lstlisting}
\end{tcolorbox}

%\articlesection

The \cs{@starredsection} macro is one of those locomotive type of commands. It takes 7 required arguments and 2 optional ones and hidden within it are two booleans. The full set looks like this:

\cs{@startsection} \marg{name} \marg{level} \marg{indent} \marg{beforeskip} \marg{afterskip} \marg{style}[*]
  [\marg{altheading}]\marg{heading}.

\begin{marglist}
\item[name] The name of the level command.
\item [level] A number denoting the depth of the section, chapter=1, section=2, etc. A section number will be printed only if \marg{level} is equal or smaller than the value of \textit{secnumdepth}
\item[indent] The indentation of the heading from the left margin.
\item[beforeskip]  The absolute value of this argument is the skip to leave above the heading. If it is negative, then the paragraph indent of the text following the heading is suppressed.
\item [afterskip] If positive, it is the skip to leave below the heading, else it is the skip to the right of a run-in heading.
\item [style] Sets the style of the heading.
\item[\textup{[*]}] When this is missing the heading is numbered and the corresponding counter is incremented.
\item[\textup{[\textit{altheading}]}] Gives an alternative heading to use in the table of contents and in the running heads. This should be present when the * form is used.
\item[heading] The heading of the new section.
\end{marglist}

\begin{example}{Example formatting run-in section}{}
\makeatletter
\bgroup
\renewcommand\section{%
    \@startsection{section}%
    {1}%
    {0em}%
    {-0.8em}%
    {-0.5em}%
    {\large\normalfont\scshape}}
\makeatother
\section[]{test}
\lorem
\egroup
\end{example}

Note we run the example in a group so that we will not influence the formatting of this document.

As mentioned earlier there is an additional way to introduce formatting for sections and this is using the command \cs{@seccntformat}, which is responsible for typesetting the counter part of a section title. The default definition of the command typesets the \cs{the} representation of the section counter.

\begin{example}{}{}
\bgroup
\renewcommand\section{%
    \@startsection{section}%
    {1}%
    {0em}%
    {-0.8em}%
    {-0.5em}%
    {\large\normalfont\scshape}}
\renewcommand\@seccntformat[1]{\fbox
{\csname the#1\endcsname}\hspace{0.5em}}
\makeatother
\section[]{test}\label{sec:ok}
\lorem

See section \ref{sec:ok}.
\egroup
\end{example}

The definition of \cs{@seccntformat} applies to all headings
defined with the \cs{@startsection} command (which is described in the next
section). Therefore, if you wish to use different definitions of \cs{@seccntformat}
for different headings, you must put the appropriate code into every heading
definition.

\begin{tcolorbox}
\begin{lstlisting}
\def\@seccntformat##1{\csname the##1\endcsname{}}
\end{lstlisting}
\end{tcolorbox}

\section{Custom headings}

It is also possible to define section headings without resorting to any of the above. To do this.

\begin{tcolorbox}
\begin{lstlisting}
\newcommand\part{\secdef\cmda\cmdb}
\end{lstlisting}
\end{tcolorbox}

the part and chapter and sometimes appendix are defined this way, but nothing stops us from doing the same for other sections. A generic section command can be defined as follows:

\begin{example}{}{}
\bgroup
\renewcommand\section[2] [?]{% % Complex form:
\refstepcounter{section}% % step counter/ set label
\addcontentsline{toc}{appendix}% % generate toe entry
{\protect\numberline{section-\thesection}#1}%
{\raggedright\large\bfseries section %\appendixname\ % typeset the title
\thesection\par \centering#2\par}% % and number
\sectionmark{#1}% % add to running header
\@afterheading % prepare indentation handling
%\addvspace{\baselineskip}
}
\section{Test}
\lorem
\egroup
\end{example}

Many other strategies can also be implemented that are perhaps easier to grasp.

\begin{example}{}{}
\bgroup
\def\strut{\vrule height12pt depth1pt width0pt}
\renewcommand\section[2] []{% % Complex form:
\refstepcounter{section}% % step counter/ set label
\addcontentsline{toc}{section}% % generate toc entry
{\protect\numberline{\thesection} }%
{\raggedright\large\bfseries\scshape %
\parbox[b]{\dimexpr(\linewidth-0.5\columnsep)}{\colorbox{brown!80}%
{{\vbox{\strut\raise2pt\hbox{#2}}}}}}\vskip0pt% % and number
\sectionmark{#1}% % add to running header
\@afterheading % prepare indentation handling
\vspace{\dimexpr\baselineskip+6pt}%must have a parameter
}
\chapter{Fossil Insects}
\begin{multicols*}{2}\raggedcolumns
\section[Insect Fossilization]{\raggedright \thinspace Insect Fossilization}
\lipsum[1]
\end{multicols*}
\egroup
\end{example}
% To answer http://tex.stackexchange.com/questions/52998/change-title-to-small-caps-but-not-in-toc

Of course some work is needed to center the text properly in the middle of the colour box. For all practical purposes it is lining up as per the sample.

In Chapter we discussed a forward, but this may not apply if there are no chapters or we need to treat these as sections, the example \ref{ex:forwardsection} shows such a method.

\begin{example}{Defining a Foreward Section}{ex:forwardsection}

\newcommand\prematter@sp[1]{% % Complex form:
%\refstepcounter{section}% % step counter/ set label
\addcontentsline{toc}{section}% % generate toe entry
{\protect\numberline{}\textsc{#1}}%
\sectionmark{#1}% % add to running header
{\LARGE\centering\normalfont\sffamily\colorbox{brown!80}{ \textsc{#1}}\par}%
\@afterheading % prepare indentation handling
\addvspace{\baselineskip}
\@afterindentfalse
}

\newenvironment{prematter}[1]{%
   \prematter@sp{#1}}
{}
\begin{multicols}{2}
\label{theok}
\begin{prematter}{Foreward}
\lipsum[1]
\end{prematter}\ref{theok}
\end{multicols}
\end{example}

\section{underlining}

I am aware that some people have no choice but have some sections underlined as dictated by archaic regulations in some establishments for thesis submission. If nobody is forcing you to underline it is best to avoid it. We use Donald Arsenau's ulem package to achieve underlining.

%%%%%%%%%%%%%%%%%%%%%%%%%%%%%%%%%%%%%%%%%%%%%%%%%%%%%%
\cxset{steward,
  chapter toc=true,
  numbering=arabic,
  custom=tikzspecial,
  offsety=0cm,
  image=hine03,
  texti={When Lamport designed the original \LaTeX\ sectioning commands, limitations of computer power forced him to restrict the abstraction of complicated chapter layouts. With current tools available improvements are much easier to program.},
  textii={In this chapter we discuss a method that allows the production of fancy chapter headings and formatting, based on a set of key values. Central  to this process is the separation of content from presentation.
We also discuss the basic formatting tools that are available and how one can modify them to mould new book designs.},
}
\@specialtrue

\cxset{chapter opening=left}
\chapter{Special Designs}
\section{Introduction}

The strength of the package lies in having defined mechanisms to enable easier abstraction of special designs.
We will first outline a simple mechanism for such definitions.

To define any special chapter you need to either redefine a command or create a new one. Let us look at
an example, which simply uses the tikZ package to draw a chapter header at the top of the page. Every time the \cs{chapter} command is called, this command will be indirectly activated at the appropriate point.
What is available for you to use is all the chapter settings information. You can also add additional keys.

\begin{tcolorbox}
\begin{lstlisting}
\renewcommand{\tikzspecial}[2][]{%
\begin{tikzpicture}[remember picture,overlay]
    \node[yshift=-3cm] at (current page.north west)
      {\begin{tikzpicture}[remember picture, overlay]
        \draw[fill=\fill@cx] (0,0) rectangle (\paperwidth,3cm);
        \node[anchor=east,xshift=.9\paperwidth,rectangle,
              rounded corners=10pt,inner sep=11pt,
              fill=thegreen]{%
        \titlefontcolor@cx
        \titlefontsize@cx\bfseries
        \titlefontfamily@cx
        \thechapter\
        \textsc{#2}};
       \end{tikzpicture}
      };
\end{tikzpicture}
\mbox{}
\vspace*{60pt}\par
}
\end{lstlisting}
\end{tcolorbox}

%% new variables for the box
\cxset{
   chapter box color/.store in=\chapterboxcolor@cx,
   chapter band color/.store in=\chapterbandcolor@cx,
}

The strength of the system lies in defining an adequate number of variables to abstract the design. We also need to decide which are the important parameters. Let us for demonstration purposes just add two new keys.
One for the band color and another for the rounded colour.

\begin{tcblisting}{}
\cxset{
   chapter box color/.store in=\chapterboxcolor@cx,
   chapter band color/.store in=\chapterbandcolor@cx,
}
\end{tcblisting}

Note that any design based on tikZ's  \texttt{remember picture, overlay} requires possibly two and sometimes three runs in order to stabilize.

\cxset{custom/.code=\gdef\customdesign@cx{\csname#1\endcsname}\@specialtrue,
       fill/.store in=\fill@cx,
      }
 \cxset{yshift/.store in=\yshift@cx}
\cxset{stefan/.style={fill=purple, title font-color=\color{white},
          custom=tikzspecials,
          yshift=-3cm,
          fill=purple!80,
          }}

\newcommand{\tikzspecials}[2][]{%
\begin{tikzpicture}[remember picture,overlay]
    \node[yshift=\yshift@cx] at (current page.north west)
      {\begin{tikzpicture}[remember picture, overlay]
        \draw[fill=\fill@cx, draw=none] (0,0) rectangle (\paperwidth,3cm);
        \node[anchor=east,xshift=.9\paperwidth,rectangle,
              rounded corners=10pt,inner sep=11pt,
              fill=\fill@cx]{%
        \titlefontcolor@cx
        \titlefontsize@cx\bfseries
        \titlefontfamily@cx
        \thechapter\
        \textsc{#2}};
      \draw [fill=red] (0,10cm) -- (5cm,10cm);
       \end{tikzpicture}
      };
\end{tikzpicture}
\mbox{}
\vspace*{60pt}\par
}

\section{Naming conventions}

When you set the key custom it redefines the command \cs{customdesign@cx} to hold the name of your special macro.  So the only place where you need to add a definition is one macro. You can name your style anything you want, however I recommend that variants are named in two or more words, the second one simulating a theme. For example you can name your theme \option{stephan} and a sub-theme as  \option{stephan blue}.


\section{Themes and styles}

Once you have a design abstracted and its major components defined as keys, you can think of it as a template. A template then can be extended to different \textit{themes}. For example if we name our template as \textit{stefan}, we can have themes as \textit{stefan blue}, \textit{stefan green} or other similar and appropriate names. This is closer to what is currently used in CMS systems on the web.

\begin{tcblisting}{}
\cxset{stefan purple/.style={
         fill=purple,
         title font-color=\color{white},
         custom=tikzspecials}}
\end{tcblisting}



\fancypagestyle{chapterstyle}{\fancyhf{}%
  \fancyhead[RO,LE] {\thepage}%
  \fancyhead[LO,RE] {}%
  \fancyfoot [R] {\scriptsize\today}
  \renewcommand\headrulewidth{1pt}
}

\pagestyle{chapterstyle}

\fancypagestyle{plain}{\fancyhf{}%
  \fancyhead[RO,LE] {\thepage}%
  \fancyhead[LO,RE] {}%
  \fancyfoot [R] {\scriptsize\today}
\renewcommand\headrulewidth{0pt},
\renewcommand\footrulewidth{0pt}
}


\cxset{chapter toc=false, chapter opening=any,
          header style=samplepage}
\pagestyle{chapterstyle}

\@specialtrue
\clearpage
\cxset{stefan}
\chapter{A Chapter Head Drawn with TikZ}

\lipsum[1-3]

\clearpage
%\pagestyle{fancy}


\cxset{header style=samplepage}

\cxset{chapter opening=anywhere}
\chapter{Introduction to TikZ Style Chapters}
\chapter{Introduction to TikZ Style Chapters}
The \lstinline{tikZ} package brings a lot of capabilities to the design of fancy style headings, including shading effects and the like. I expect this type of design to grow in the future. Since tikZ is part of the PGF family it is easy to integrate with the package.

\section{Integrating the code}

Code integration, especially with a document that might have different chapter headings presents a challenge. However, if we do touch the chapter command it might make things easier. We provide a key called special that instead of calling the \string\@make... calls a special
routine to handle the tikz commands (as one would expect that all the code will then be here).

\begin{lstlisting}
\newcommand\chapter{\if@openright\cleardoublepage\else\clearpage\fi
                    \thispagestyle{plain}%
                    \global\@topnum\z@
                    \@afterindentfalse
                    \secdef\@chapter\@schapter}

\@chapter[#1]#2{\ifnum \c@secnumdepth >\m@ne
                    \if@mainmatter
                      \refstepcounter{chapter}%
                         \typeout{\@chapapp\space\thechapter.}%
          %%
              \if@toc
                      \addcontentsline{toc}{chapter}%
                                   {\protect\numberline{\thechapter}#1}\fi%
                       \else
                         \addcontentsline{toc}{chapter}{#1}%
                       \fi
                    \else
                      \addcontentsline{toc}{chapter}{#1}%
                    \fi
                    \chaptermark{#1}%
                    \addtocontents{lof}{\protect\addvspace{10\p@}}%
                    \addtocontents{lot}{\protect\addvspace{10\p@}}%
                    \if@twocolumn
                      \@topnewpage[\@makechapterhead{#2}]%
                    \else
                      \@makechapterhead{#2}%
                      \@afterheading
                    \fi}
\end{lstlisting}

The best approach I could think of was to add some sort of switch in
the @makechapterhead macro, which will then call the special.

First we define a special key.


\begin{lstlisting}
\cxset{custom/.code=\gdef\customdesign@cx{#1}\@specialtrue,
       fill/.store in=\fill@cx}
\cxset{custom=tikzspecial,
       title font-size=\Large,
       title font-color=\color{white}}
\end{lstlisting}

We have assumed that the only value we want to pass is the Chapter title, as the rest can be handled quite easily, by means of key values.

\section{Key management}

When you develop a generic template all the standards keys are available to you. For example the chapter opening commands. However, if you positioning using fixed parameters, the anywhere key cannot work properly, by adding a yshift into the definition of the special and adjusting you can achieve it.

\clearpage

\cxset{chapter opening=anywhere,
          yshift=-12cm,chapter toc=false}
\chapter{A test}
\cxset{chapter opening=anywhere,
          fill=olive,
          yshift=-18cm}
\chapter{A test}

\section{Conclusions}

In this chapter we have seen how to design and code special templates for special openings. In most cases you will use TikZ to produce them, so familiarity with the graphics program is essential. In general I advice that before you embark on a special design to select the method you will used based on the following:

\begin{enumerate}
\item If the template requires positioning of pictures and text at exact positions, you can use the 
        picture environment and the built-in commands provided in this package.
\item If it requires any special graphics, coloured blocks and the like, use the TikZ package or pstricks.
\item If you only manipulating textual information you don't need a special use the key value interface provided by the package (see for example the verso style).
\end{enumerate}
 

















%% DESIGNING HEADERS AND FOOTERS  **************************


\chapter{Running Titles and Paging}

Early printed books had no running title or paging figures. The first attempt to satisfy this need of the reader was to repeat the number of the chapter at the head of each page.\footnote{De Vinne, pg 142.}  As books and styles evolved, if the words of the running title or chapter began appearing together with the page number. Practical considerations regarding the wearing of plates, school-books and all works that were printed frequently had running titles in capitals of light-faced antique. 

\begin{figure}[htp]
\includegraphics[width=\textwidth]{./images/beauty-and-art-spread.jpg}
\end{figure}

Almost every type of design has been adopted by typographers and book designers; sometimes the text is centered and in other cases it is set flush up to the inner or outer margin of the facing pages. The book chapter and the section of the book is sometimes specified in the running title, the chapter name on the left and the section on the right. When the running title consists of the name of the book, it was sometimes divided so that one half only of this name would appear on one page and the other half on the facing page. De Vrinde was highly critical of such practices and remarked `Nor is this a commendable fashion, for a line of many words can seldom be evenly divided; if it is not so divided, one heading will be longer than the other.’  Some modern books that follow in a similar fashion would place the chapter label and number at the left and the chapter title on the right. 

I am unsure if repeating the name of the book in its running title has any benefits to the reader, especially if the name of the book is well known to the reader. This title is most useful when it explains or to some extend defines the matter on the page, and this explanation should refer not to the first but to the last paragraph on that page.  Many authors prefer to not have sections in chapters and in such cases running the book name in the header rather than having left and right headers that just repeat the chapter name is preferable. An example of this is Tufte’s \textit{Beautiful Evidence}.  Tufte’s books do not have any footer material.  Many specialist scientific books have multi-authors, sometimes the running head includes the authors name (See figure from ). This particular illustration also shows the use of rules. Traditionally the rules were applied to protect the top of the block from mechanical wear during printing. 

\begin{figure}[hb]
\includegraphics[width=\textwidth]{./images/headers/header-humidification-odd.jpg}
\includegraphics[width=\textwidth]{./images/headers/header-humidification-even.jpg}
\end{figure}

As a rule,  paging with arabic figures begins with the text of the book. The matter before the text (as the title, preface, introduction, etc., which are printed last of all) is paged with roman lower-case numerals. Appendices, indices and all additions to the text take arabi figures for paging, but publisher’s advertisements at the end of the book should receive their special paging in a figure of a different face. Maps, portraits, and illustrations made on separate pages never receive printed paging, although they may be reckoned as pages in the table of contents or the index. 
\begin{figure}[htb]
\includegraphics[width=\textwidth]{./images/headers/architect.jpg}
\caption{The headers here, have a background shading.}
\end{figure}

\begin{figure}[htb]
\includegraphics[width=\textwidth]{logic.jpg}
\caption{The headers shown here include small dotted rules, running to the outer page end. This type of header can be build by adding properties and inheriting the properties of other headers.}
\end{figure}

\begin{figure}[htb]
\hskip-.1\textwidth\includegraphics[width=1.2\textwidth]{./images/headers/tulip-01.jpg}

\vspace*{1cm}

\hskip-.1\textwidth\hbox to 0pt{\includegraphics[width=1.2\textwidth]{./images/headers/tulip-02.jpg}}
\caption{The headers here, have a background shading.}
\end{figure}

\begin{figure}[htp]
\includegraphics[width=1\textwidth]{./images/headers/small-flash-01.jpg}

\vspace*{1cm}
\includegraphics[width=1\textwidth]{./images/headers/small-flash-02.jpg}

\caption{The headers here, have a background shading.}
\end{figure}


\begin{figure}[htp]
\includegraphics[width=1\textwidth]{./images/headers/power-and-politics-01}

\vspace*{1cm}
\includegraphics[width=1\textwidth]{./images/headers/power-and-politics-02}

\caption{The headers here, have a background shading.}
\end{figure}

\begin{figure}[htp]
\includegraphics[width=1\textwidth]{./images/headers/economic-warfare-01}

\vspace*{1cm}
\includegraphics[width=1\textwidth]{./images/headers/economic-warfare-02}

\caption{The headers here, have a background shading.}
\end{figure}


\section{The Requirements}

The brief discussion above and the examples from various publications can help in definind the final requirements of what we are about to program. The header or the footer for that matter as they are very similar needs to communicate with the page that is currently being processed to obtain the page number and any other marks that need to go into the running head.

\begin{tabular}{>{\raggedright}p{5cm}l}
Access to the page number & \\
Build up string from sections, chapters, titles or subtitles &\\
Distingusih between left and right numbers &\\
Add user data  &\\
Provide an intuitive user interface&\\
\end{tabular}

A more modern approach would be to offer a small templating language to deal with the headers and footers. This is for example, now common in web applications where variables are sent by the server to the web page being build and transformed in templates.

Another approach is to use a graphical language, such as metapost.

Since we have to deal with odd and even pages and a header and or a footer, the minimum variables needed to hold this information is four. 

A graphicablock can also happily contain the necessary information.

The algorithm is described below:

\begin{enumerate}
\item Set the variable headerleft and headerright to indicate one page or two page printing.
\item Define text block templates as macros to set the typesetting to a named style. Each header style
         will have its own name. Standardize parameters to enable easy redefinition of commands. As a 
         final fully flexible approach the key header = custom will provide full capabilities for any user
         defined design.
\item Distinguish how headers and footers will be typeset on title pages, chapter openings, bibliographies, 
         automatically generated pages, such as float pages etc.
\item Hook into LaTeX’s output routine to obtain information about the top and bottom inserts and other marks.         
\item Inherit properties, such as language and directionality.
\item Provide less intrusive ways to define different styles by the user.
\item define block commands to mark start of different headings for example |\mainmatter|. This will define
         the start of the main text of the publication and issue a command to process the pages that follow.
\end{enumerate}

A special type of header is something that will be repeated on every page, say a watermark of some sorts. These are dealt as backgrounds.
 
\section{Traditional LaTeX page style commands}
  
One of the first tasks of any \LaTeXe\ class is to redefine the headers and footers. The format of the running headers or footers in \LaTeX\ terminology is called the \textit{page style}. Each different format is given names like \textit{empty} or \textit{plain} to make it easier to select and remember. 

\begin{figure}[hbt]
\includegraphics[width=\textwidth]{./images/headers/Running-heads-lace.png}
\caption{This last example shows what kind of atmosphere you can create with running heads. Here a bit of lace texture has been softened and graduated, creating a kind of gentle, suggestive frame around these pages. I’ve also used line drawings, logos, and other graphic elements to dress up running heads like these. From the \protect\href{bookk  }{bookdesigner.com}}
\end{figure}


The LaTeX kernel\footnote{In File J file{ltpage.dtx}, page 311.} defines two commands for selecting the running heads:

\begin{lstlisting}
\pagestyle{<style>} : sets the page style of the current and succeeding pages to style
\thispagestyle{<style>} : sets the page style of the current page only to style.
\end{lstlisting}

\section{Traditional LaTeX page style definition}

To define a page style \textit{style}, you must define the \lstinline{\ps@style} to set the page parameters.

\subsection{How a page style makes running heads and feet}
The \lstinline{\ps@}. . . command defines the macros \lstinline{\@oddhead}, \lstinline{\@oddfoot}, \lstinline{\@evenhead},
and \lstinline{\@evenfoot} to define the running heads and feet. (See output routine.) As some headings contain information such as the chapter name or section number these
headings are based on the sectioning commands, which define them. The page style defines the commands




\verb!\chaptermark,\sectionmark!, etc., where

\verb+\chaptermark{<text>}+ is called by \verb+\chapter+ to set a mark. The  ...mark commands and the ...head
macros are defined with the help of the following macros.
%(All the \ ...mark commands should be initialized to no-ops.)



\subsection{marking conventions}

LaTeX produces two kinds of marks a `left' and a `right' mark using the following commands.

markboth

markright



\section{The low level page style interface}

The basic mechanics of defining page styles is provided in the \LaTeXe\ kernel and it  involves defining or redefining four macros:

\begin{marglist}
\item [\cs{oddhead}] For two-sided printing, it generates the header for the odd-numbered
pages; otherwise, it generates the header for all pages.

\item [\cs{oddfoot}] For two-sided printing, it generates the footer for the odd-numbered pages; otherwise, it generates the footer for all pages.

\item [\cs{evenhead}] For two-sided printing, it generates the header of the even-numbered
pages; it is ignored in one-sided printing.

\item [\cs{evenfoot}] For two-sided printing, it generates the footer of the even-numbered
pages; it is ignored in one-sided printing.

\end{marglist}
A named page style, involves the redefinition of these commands stored in a macro \cs{ps@<style>}.
The \cs{pagestyle}\marg{plain} is defined as:



%\begin{tcolorbox}
%\begin{lstlisting}
%\newcommand\ps@plain{%
%  \renewcommand\@oddhead{}%
%  \let\@evenhead\@oddhead
%  \renewcommand\@evenfoot{%
%  {\hfil\normalfont\textrm{\thepage}\hfil}}%
%  \let\@oddfoot\@evenfoot
%}
%\end{lstlisting}
%\end{tcolorbox}

Since the \textit{plain} style treats both the odd and even pages the same way, the \cs{@evenfoot} and \cs{@evenhead} are let to the \cs{@oddhead} and \cs{@oddfoot} commands. The style only prints a page number at the center of the footer.


\subsection{A longer example}

\index{watermark}\index{water mark!sample page style}
\thispagestyle{samplepage}
Consider the case, where we need to print on a page the words \textsc{sample page}, as you might have noticed in some places of this document and at the margin of this page. Sometimes this type of mark is called a \textit{watermark.}

We will call this type of page style \textit{samplepage} and we will activate it on a particular page by typing \cs{thispagestyle}\marg{samplepage}.




%\begin{tcolorbox}
%\begin{lstlisting}
%%% Some special styles
%\IfFileExists{rotating.sty}{\RequirePackage{rotating}}{}
%
%\def\even@samplepage{%
% \begin{picture}(0,0)
%   \put(\Xeven,\Yeven){\turnbox{90}{\Huge \textcolor{\watermark@textcolor}{\watermark@text}}}
%\end{picture}
%}
%
%\def\odd@samplepage{%
% \begin{picture}(0,0)
%   \put(\Xodd,\Yodd){\turnbox{90}{\Huge \textcolor{\watermark@textcolor}{\watermark@text}}}
% \end{picture}
%}
%
%\def\watermarktext#1{\gdef\watermark@text{\fontfamily{phv}\selectfont#1}}
%\def\watermarktextcolor#1{\gdef\watermark@textcolor{#1}}
%\watermarktext{SAMPLE PAGE}
%\watermarktextcolor{purple}
%
%\def\ps@samplepage{\let\@mkboth\@gobbletwo
% \let\@oddhead\odd@samplepage\def\@oddfoot{\reset@font\hfil\thepage}
% \let\@evenhead\even@samplepage\def\@evenfoot{\reset@font\thepage\hfil}}
%
%\def\Xodd{500}
%\def\Xeven{-70}\def\Yeven{-810}
%\def\Yeven{-\expandafter\strip@pt\textheight}
%\let\Yodd\Yeven
%\end{lstlisting}
%\end{tcolorbox}

If you study the code in the example, you will notice that we are using \LaTeXe's \env{picture} environment to
place the text exactly where we need it. This is one way of absolutely positioning text on a page, another way is to use |pgf|’s absolute positioning methods.




\subsection{The key value interface}

The key value interface provides a number of mechanisms to tap into the page styles, enabling consistency in the user interface.

\medskip

\keyval{header style}{\marg{text}}{Triggers a page style for one page only.} The following values can be used.

\begin{marglist}
\item [empty] Standard class empty headers.
\item [plain] Standard class plain headers.
\item [headings] Standard class headings.
\item [fancy] If you use the fancyhdr package any fancy header style.
\item [sample page] Prints sample at the edge of the paper.
\item [preprint] Prints preprint at the edge of the paper.
\item [watermark] Prints a watermark at predefined places.
\end{marglist}

\keyval{watermark}{\marg{true|false}}{Prints a watermark on all pages, defaults to false.}
\keyval{watermark text}{\marg{text}}{The watermark text.}
\keyval{watermark text left}{\marg{text}}{The watermark text on left pages.}
\keyval{watermark text right}{\marg{text}}{The watermark text on right pages.}
\keyval{watermark angle}{\marg{number}}{The rotation angle of the water mark}




%\cxset{ watermark text/.store in=\watermark@text,
%           watermark text color/.store in=\watermark@textcolor,
%           watermark font-size/.store in=\watermarkfontsize@cx,
%           watermark odd x/.store in=\watermarkoddx@cx,
%           watermark even x/.store in=\watermarkevenx@cx,
%           watermark even y/.store in=\watermarkeveny@cx}
%
%\cxset{watermark text= PRE-PRINT,
%          watermark text color=theblue,
%          watermark font-size=\huge,
%          watermark odd x=470,
%          watermark even y=700,
%          watermark even x=60}
%
%\def\Xodd{\watermarkoddx@cx}
%\def\Xeven{-\watermarkevenx@cx}
%\def\Yeven{-\watermarkeveny@cx}
%%\def\Yeven{-\expandafter\strip@pt\textheight}
%\let\Yodd\Yeven
%
%\def\even@samplepage{%
% \begin{picture}(0,0)
%   \put(\Xeven,\Yeven){\turnbox{60}{\watermarkfontsize@cx \textcolor{\watermark@textcolor}{\watermark@text}}}
%\end{picture}
%}
%
%\def\odd@samplepage{%
% \begin{picture}(0,0)
%   \put(\Xodd,\Yodd){\turnbox{90}{\watermarkfontsize@cx\textcolor{\watermark@textcolor}{\watermark@text}}}
% \end{picture}
%}






\subsection{Using the headings as hooks}

Since the headings are added to the page during processing of the output routine, they are sometimes used
to insert material on the page at places other than the head, through the use of a zero width box. For example we
can use this approach to add a watermark on a page. Other approaches to position material at absolute positions
on a page, is to hook at \emph{shipout}. Some packages such as TikZ can also be used through the |remember picture, overlay |  key settings. 

The |phd| package has a predefined style, named samplepage that can be used to typeset some text at the outer margin of a page. The text is configurable and you can set it for example to typeset “PRE-PRINT” rather than the “SAMPLE PAGE” string. 

\begin{tcolorbox}
\begin{lstlisting}
\cxset{
     watermark text= PRE-PRINT,
     watermark text color=theblue,
     watermark font-size=\huge
}
\end{lstlisting}
\end{tcolorbox}

\makeatletter
\cxset{watermark text/.code =\watermarktext }
\makeatother

\watermarktext{PRE-PRINT}
   
\pagestyle{samplepage}


\section{Adding marks}

Most books will have headers that include marks such as the chapter name and number and or other combinations together with section numbers.

The standard book class include two styles one called \textit{headings} and another called \textit{myheadings} that style such headers.




\subsection{Key value interface}
\makeatletter
\cxset{
   chaptermark name color/.store in=\chaptermarknamecolor@cx,
   sectionmark name color/.store in=\sectionmarkcolor@cx,
   sectionmark title font/.store in=\sectionmarktitlefont@cx,
   section title color/.store in=\sectiontitlecolor@cx,
}

\makeatother

\cxset{chaptermark name color=thered,
          sectionmark name color=thered}





\begin{tcolorbox}
\begin{lstlisting}
%% STYLE 57 QUANTUM FRONTIER
\cxset{headings style57/.style={
          headings titlestyle,
% Chaptermarks
          chaptermark name={\bfseries EVOLUTION OF THE INSECTS},
% Leftmarks
          leftmark before=\thepage\quad, %even pages
          leftmark after=\hfill\hfill,
% Right marks influenced by chapter name?
          rightmark before=\hfill\hfill, %odd pages
          rightmark after=\thepage,
% Section marks
          sectionmark name custom=\chaptertitle@cx,
          sectionmark after title=\quad,
%  rules we remove or inherit
          header top rule=false,
          header bottom rule=false,
          header offset even=0pt,
          header offset odd=0pt,
          }}
\end{lstlisting}
\end{tcolorbox}


%\if@twoside
%  \def\ps@headings{%
%      \let\@oddfoot\@empty
%      \def\@oddfoot{\rule{\textwidth}{0.4pt}}
%      \let\@evenfoot\@empty
%      \def\@evenhead{\parbox{\textwidth}{%
%                                   \leavevmode
%                                   \if@headertoprule\rule{\textwidth}{0.4pt}%
%                                       \vskip2pt plus1pt minus1pt\fi
%%typesetter
%                                     \hskip\headeroffseteven@cx\hbox to \textwidth{%
%                                           \leftmarkbefore@cx
%                                           \leftmark
%                                           \leftmarkafter@cx
%                                     }%
%                                     \if@headerbottomrule\vskip-7pt plus1pt minus1pt
%                                    \rule{\textwidth}{0.4pt}\fi%
%          }% end parbox
%       }%
%%% Defines the odd head
%      \def\@oddhead{
%         \parbox{\textwidth}{%
%                                   \leavevmode
%                                   \if@headertoprule\rule{\textwidth}{0.4pt}%
%                                       \vskip2pt plus1pt minus1pt\fi
%%typesetter
%                                     \hskip\headeroffsetodd@cx\hbox to \textwidth{%
%                                           \rightmarkbefore@cx
%                                           \rightmark
%                                           \rightmarkafter@cx
%                                     }%
%                                     \if@headerbottomrule\vskip-7pt plus1pt minus1pt
%                                    \rule{\textwidth}{0.4pt}\fi%
%          }% end parbox
%      }%
%      \let\@mkboth\markboth
% % chaptermark called by chapter and also by table of contents etc. This is essentially a
%%  leftmark
%\def\chaptermark##1{%
%     \gdef\chaptertitle@cx{##1}%
%      \markboth {%
%       \ifnum \c@secnumdepth >\m@ne
%          \if@mainmatter%
%              \color{\chaptermarknamecolor@cx}%
%              \MakeUppercase{\chaptermarkname@cx\ }%
%              \chaptermarknumber%
%              \chaptermarkafternumber@cx%
%          \fi
%        \fi
%        \color{\chaptermarktitlecolor@cx}%
%       % \hfill%
%        \MakeUppercase{\chaptermarktitlebefore@cx{##1}}}{}%
%}%end chaptermark
%% section
%  \def\sectionmark##1{%
%      \markright {%
%        \ifnum \c@secnumdepth >\z@
%           {\bfseries\textcolor{\sectionmarkcolor@cx}{\sectionmarkname@cx\sectionmarknumber@cx\sectionmarkafternumber@cx}%
%        } %
%  \fi
%         \color{\sectionmarktitlecolor@cx}\MakeUppercase{\normalfont\sffamily \sectionmarkbeforetitle@cx{##1}\sectionmarkaftertitle@cx}}}}%
%\else
%  \def\ps@headings{%
%    \let\@oddfoot\@empty
%    \def\@oddhead{{\slshape\rightmark}\hfil\thepage}%
%    \let\@mkboth\markboth
%    \def\chaptermark##1{%
%      \markright {%
%        \ifnum \c@secnumdepth >\m@ne
%          \if@mainmatter
%            \@chapapp\ \thechapter... \ %
%          \fi
%        \fi
%        ##1}}}
%\fi
%\def\ps@myheadings{%
%    \let\@oddfoot\@empty\let\@evenfoot\@empty
%    \def\@evenhead{\thepage\hfil\slshape\leftmark}%
%    \def\@oddhead{{\slshape\rightmark}\hfil\thepage}%
%    \let\@mkboth\@gobbletwo
%    \let\chaptermark\@gobble
%    \let\sectionmark\@gobble
% }

Note that the \cs{markboth} command takes two arguments the left mark and the right mark. It works reasonably well.



%\cxset{headings boxedpagenumber}
%\cxset{headings style58}
%\pagestyle{headings}

\clearpage
\cxset{header style=empty}
\chapter{SAMPLE HEADERS AND FOOTERS}

%\section{Using the fancyvrb package}
%
%Most LaTeX users, when redesigning running headers or footers will use the fancyvrb package. This well established package by Piet van
%Oostrum, allows easy customization of page headers and footers. The default page style provided by fancyhdr is named fancy. It should be activated via
%\lstinline{\pagestyle} after any changes to textwidth are made, as fancyhdr initializes
%the header and footer widths using the current value of this length.
%The look and feel of the fancy page style is determined by six declarations
%Balle 1I1[l'rtace that define the material that will appear on the left, center, and right of the header
%and footer areas. For example, lhead specifies what should show up on the left
%in the header area, while cfoot defines what will appear in the center of the
%footer area. The results of all six declarations are shown in the next example.
%
%\section{Header and footer options of the chaptersx package}
%
%To keep a consistent and familiar interface, the package uses the
%fancyhdr package which it loads by default and defines option keys that are similar to the macros provided by fancyhdr, as well as some additional typesetting keys.
%
%It is also possible to group these in styles, which we have done for the following:
%
%We have endeavoured o include as many keys possible in order to provide a consistent and comprehensive style.
%
%\section{Example header style}


\section{Description of the header engine}

The header engine works by defining three vertical regions to contain a top and bottom rule or other material and a middle layer that contains the text and or graphical material. In its simpler form is an assembly of boxes as shown in figure~\ref{fig:headerengine} .
\medskip

\noindent\vbox{

\framebox[80pt]{toprule}

\fbox{\fbox{offset}$\rightarrow$\fbox{\fbox{name} \fbox{number} \fbox{title} \fbox{page number}}}

\framebox[80pt]{bottomrule}

\captionof{figure}{Typical assembly of boxes for a header}
\label{fig:headerengine}
}

Of course the order of display of the various components can vary, for example in figure~\ref{fig:headerengine01}, we have the page number to the left.

\noindent\fbox{\vbox{%
oddhead\par
\framebox[80pt]{toprule}

\fbox{\fbox{offset}$\rightarrow$\fbox{\fbox{page number}\fbox{name} \fbox{number} \fbox{title} }}

\framebox[80pt]{bottomrule}

\captionof{figure}{Typical assembly of boxes for a header}
\label{fig:headerengine01}
}}

The LaTeX engine as previously described allows for an oddhead and evenhead macros to hold the typesetting information. The typesetting information is obtained asynchrnously,

The chaptermark assembles and stores

\fbox{\fbox{chapter name}$\rightarrow$\fbox{chapter number}$\rightarrow$\fbox{chapter title}}

whereas the sectionamark similarly stores information on section numbering:

\begin{figure}[h]
\fbox{\fbox{section name}\fbox{section number}\fbox{section title}}
\end{figure}

This information of the mark macros is then activated by the leftmark and rightmark commands and when the typesetter calls the oddhead and evenhead the page number is added. So to have a complete definition
we need to define both the marks as well as the oddhead and evenhead macros. Similarly the pagenumber's position can vary in different designs.

The approach we took is to firstly extend the informtion held in each box, for example, the chaptername
looks more like:

\begin{figure}[h]
   \fbox{chaptername before}$\rightarrow$\fbox{chaptername}$\rightarrow$\fbox{chapternameafter}
\end{figure}

Similarly the header elements, \textit{name, number, title} and \textit{pagenumer}  hold information to add material before and after the element. This adds complexity, but generalizes and abstracts the problem nicely.

Now back to leftmarks and rightmarks. The leftmark or rightmark is called by the oddside or evenside macros. We suitably add macros for before and after.

\fbox{before mark}$\rightarrow$\fbox{leftmark or rightmark}$\rightarrow$\fbox{after mark}

\subsection{Discussion}

One would argue that this is a convoluted way of describing the header and footers. However, to have a truly flexible system (with no macro writing for designers and authors) one has to resort to such a long way. It is in many respects simlar to CSS. A looks description language needs as many variables and as much flexibility as possible. The one up on CSS is that the template can be saved and themes developed and invoked very simply.

\subsection{Example settings}
\index{Styles!style56}
\index{Header and footers!example}
To define a style based on the \textit{headings} pagestyle. We will base the style definition on figure~\ref{globalstrategy}. The interesting part of this design is that the chapter numbers are not shown shown in the introduction (which is treated as front matter material). The rest of the chapters are numbered, although the header information and styling remains the same.

\begin{figure}[hp]
\centering
\includegraphics[width=0.8\textwidth]{chapter56}\vspace{0.5\baselineskip}
\includegraphics[width=0.8\textwidth]{chapter56a}
\caption{Example pages from \textit{Global Competitive Strategy}, by Daniel F. Spulber,} Cambridge University Press, 2007.
\label{globalstrategy}
\end{figure}

The interesting part of this design is that the headers as well as the chapter openings vary. On the even page the words ``Golbal Competitive Strategy'' are printed, which is the book title\footnote{Although it is an attractive design, it does not mean that this way of header information is a good way of structuring information.}.

Note that since LaTeX only provides two mark we need to arrange our own typesetting rules, we will abuse the macros to achieve it.

\begin{tcolorbox}
\begin{lstlisting}
\cxset{headings style56/.style={
          pagestyle=headings,
          header style=headings,
% Chaptermarks
          chaptermark name color=black,
          chaptermark after number=,
          chaptermark name=SHORT BOOK TITLE,
          chaptermark numbering=none,
          chaptermark title color=black!80,
          chaptermark title before=\@gobble,
% Leftmarks
          leftmark before=\colorbox{thegray!50}{\thepage\quad}\quad, %even pages
          leftmark after=\hfill\hfill,
% Right marks influenced by chapter name?
          rightmark before=\colorbox{thegray!50}{\thepage\quad}\quad, %odd pages
          rightmark after=\hfill\hfill,
% Section marks
          sectionmark name custom=\chaptertitle@cx,
          sectionmark number=none,
          sectionmark name color=black,
          sectionmark title color=black!80,
          sectionmark before title=\@gobble, % we do not need the section title
          sectionmark after title=\hfill\hfill,
          sectionmark after number=,
%  rules we remove or inherit
%       header top rule=false,
          header bottom rule=true,
          header offset even=-1.3cm,
          header offset odd=-1.3cm,
          }}
\end{lstlisting}
\end{tcolorbox}

The approach we need to take here is to first define the even pages, which contain the book short title. Instead of printing the \cs{chaptername}, we give the value of the title to the \textbf{chaptermark} key.

\begin{tcolorbox}
   chaptermark name = SHORT BOOK TITLE,
\end{tcolorbox}

Offsetting the page numbers is done by using the \textbf{header offset} key. As both the left as well as the right pages have the page on the left we offset both of them by the same amount.

\begin{tcolorbox}
\begin{lstlisting}
    chaptermark name = SHORT BOOK TITLE,
    header offset even = -1.3cm,
    header offset odd  = -1.3cm,
\end{lstlisting}
\end{tcolorbox}

\subsection{Inheriting and transforming styles}
One of the advantages of this method, is that we can inherit styles and transform styles easily. Consider style57, which is shown in figure. This is a simple design and follows trends to include the book title in the header. This is a very similar design to header \textit{style57}.

\begin{tcolorbox}
\begin{lstlisting}
%% STYLE 57 QUANTUM FRONTIER
\cxset{headings style57/.style={
          headings style56,
% Chaptermarks
          chaptermark name={\bfseries The Quantum Frontier},
% Leftmarks
          leftmark before=\thepage\quad, %even pages
          leftmark after=\hfill\hfill,
% Right marks influenced by chapter name?
          rightmark before=\hfill\hfill, %odd pages
          rightmark after=\thepage,
% Section marks
          sectionmark name custom=\chaptertitle@cx,
          sectionmark after title=\quad,
%  rules we remove or inherit
          header top rule=false,
          header bottom rule=false,
          header offset even=0pt,
          header offset odd=0pt,
          }}
\end{lstlisting}
\end{tcolorbox}

A crude form of object inheritance is possible by including a style at the top of the key definitions in this case \texttt{headings style56}, we then only need to redefine the values for the changes. We also set zero all offsets and adjust the widths.

The success of the method is defining an appropriate set of general commands and building a community chest of styles.

This concludes the long excursion into headers and footers. This class and method of styling is brand new and is bound to evolve, as it is being used. Feedback is most welcomed as well as bug reports.
\begin{figure}[tp]
\centering
\includegraphics[width=0.8\textwidth]{chapter57a}\vspace{0.5\baselineskip}
\caption{Example pages from \textit{The quantum frontier: the large hadron collider}, by Don Lincoln, The Johns Hopkins University Press, 2009.  It uses the book short-title at even pages and the chapter title at the odd pages.}
\end{figure}

\section{Another inheritance example}

This example is from the book \textit{Evolution of the Insects} by David Grimaldi and Michael S. Engel and published by the Cambridge University Press in 2005. It is a beautifully typeset book and a fine piece of scientific work. The only difference in the headers of the previous examples is the setting of the page number and header text. This one as well as many other books uses the title of the book and the chapter name.

\begin{tcolorbox}
\begin{lstlisting}
%% EVOLUTION OF THE INSECTS
\cxset{headings style58/.style={
          headings style57,
          chaptermark name={\bfseries EVOLUTION OF THE INSECTS},
          leftmark before=\thepage\quad\hfill\hfill, %even pages
          leftmark after=,
          rightmark before=, %odd pages
          rightmark after=\hfill\hfill\thepage,
 }}
\end{lstlisting}
\end{tcolorbox}

Since most of the information was captured in \textit{style57} we inherit the values and only supply the ones  that are changing. This involves changing six settings, illustrating the strength of the procedure adopted. Just a word of caution I found it difficult to follow some of the terminology, if you confused by what a leftmark and right mark are think of them as holding all the header information except the page number.



\begin{figure}
\centering
\includegraphics[width=0.8\textwidth]{chapter58}\vspace{0.5\baselineskip}
\includegraphics[width=0.8\textwidth]{chapter58a}
\caption{This example is from the book \textit{Evolution of the Insects}, David Grimaldi and Michael S. Engel,  Cambridge University Press, 2005. It is a beautifully typeset book and a fine piece of scientific work. The only difference in the headers of the previous examples is the setting of the page number and header text. This one as well as many other books use the title of the book and the chapter name in headers. The footers are empty with th exception of the chapter page}
\end{figure}


\begin{figure}
\centering
\includegraphics[width=0.8\textwidth]{chapter55}\vspace{0.5\baselineskip}
\includegraphics[width=0.8\textwidth]{chapter55a}
\caption{Example pages from \textit{The Author's Due, Printing and the Prehistory of Copyright}, by Joseph Lowewenstein, The University of Chicago Press, 2002.  It uses the part title on the even page and the chapter title on the odd page. The page number is set in the margin. The footers are clear.}
\end{figure}


\clearpage
\thispagestyle{empty}


%%%%%%%%%% NEW GEOMETRY
\newgeometry{top=1cm,bottom=2cm,left=2cm}

%% GENETICS
\@specialfalse
\cxset{
 custom=,
 name={CHAPTER CONCEPT},
 numbering=none,
 number font-size=,
 number font-family=,
 number font-weight=,
 number color=white,
 numbering=arabic,
 chapter opening=right,
 chapter color={black},
 chapter font-family=\sffamily,
 chapter font-size=\large,
 chapter font-weight=\bfseries,
 title font-family=\sffamily,
 title font-color=teal,
 rule off,
}

\begin{specialchapter}[
     image=genetics-dogs,
     image caption={Labrador retriever\\
         puppies expressing\\
         brown (chocolate),\\
         golden (yellow),\\
         and black\\
         coat colors,\\
         traits controlled\\
         by two gene pairs.}]%
{Extensions\\ of Mendelian\\ Genetics}
\begin{itemize}
\item While alleles are transmitted from parent to   offspring
according to Mendelian principles, they often do not
display the clear-cut dominant/recessive relationship
observed by Mendel.
\item In many cases, in a departure from Mendelian genetics,
two or more genes are known to influence the phenotype
of a single characteristic.
\item Still another exception to Mendelian inheritance occurs
when genes are located on the X chromosome, because one
of the sexes receives only one copy of that chromosome,
eliminating the possibility of heterozygosity.
\item Phenotypes are often the combined result of genetics and
the environment within which genes are expressed.
\item The result of the various exceptions to Mendelian principles
is the occurrence of phenotypic ratios that differ from those
produced by standard monohybrid, dihybrid, and trihybrid
crosses.
  \end{itemize}
\end{specialchapter}


%%%%%%%%%% NEW GEOMETRY
\newgeometry{top=2cm,bottom=2cm,left=2cm}
\clearpage

\begin{tcolorbox}
\begin{lstlisting}
%% Special Chapter command
\newcommand\specialchapter@cx[2][]{%
\refstepcounter{chapter}
\cxset{image/.store in=\image@cx,
       image caption/.store in=\caption@cx}
\cxset{#1}
\vbox to 0pt{\color{blue}\rule{\paperwidth}{0.4pt}\par\vskip-1.4pt
\rule{0.4pt}{\textheight}\rule{4cm}{0.4pt}}

\vbox to 0pt{\parbox[b]{4.7cm}{%
\raggedright

\leftskip1.5cm
\caption@cx\par
 \expandafter\rule{\rulewidth@cx}{5.8cm}
}\parbox[b]{0.5cm}{\includegraphics[width=0.5cm,height=9.15cm]{./chapters/shadow}}\includegraphics{./chapters/\image@cx}\par}

\vspace{8.2cm}
\hspace*{-3.51cm}\hbox to 0pt{\hspace*{1.01cm}\includegraphics[width=7.7cm,height=3.8cm]{./chapters/genetics-band}
\hspace*{-2.7cm}\sffamily\color{\numbercolor@cx}\HHUGE \raise30pt\hbox{\thechapter}%
\hspace{1.5cm}\raise0.5pt\hbox{\includegraphics{./chapters/chapterconcept}\includegraphics{./chapters/shadow2}}
}

%% Title name
\parbox[b]{0.45\textwidth}{%
  \titlefontsize@cx
  \titlefontweight@cx
  \titlefontfamily@cx
  \leftskip0.5em \color{\titlefontcolor@cx}
  #2
}
%% Concepts
}

\newenvironment{specialchapter}[2][]{%
  \if@openright\cleardoublepage\else\clearpage\fi
    \thispagestyle{plain}%
    \global\@topnum\z@
    \@afterindentfalse
    \specialchapter@cx[#1]{#2}
    \begin{minipage}{0.5\textwidth}%
    \vspace{0.5\baselineskip}
    \raggedright
}{\end{minipage}}
\end{lstlisting}
\end{tcolorbox}


The aim of the package is to allow easy styling of chapter heads and extends these to include images and special effects, which are difficult to achieve using traditional methods.

Abstracting the various designs is a non-trivial undertaking due to the hundreds of different possibilities.

\begin{multicols}{2}
\long\def\specialsection#1{\hspace*{0.5em}\vbox{\hsize\columnwidth%
 \vspace{\baselineskip}
\refstepcounter{section}
\parindent0pt
\raggedright
\vbox to 0pt{%
\parindent0pt
\color{red}\rule[-49.6pt]{0.4pt}{50pt}%
\color{red}\rule{0.5in}{0.4pt}\colorbox{teal}{\color{white}{\large \space\thesection\space}}}%
\vskip5pt%
\hspace{-3.5pt}\fbox{.}\par
\vspace*{-45pt}
\hspace*{1em}\vbox{\Large\sffamily#1\par}}
\vspace*{\baselineskip}\par
}

\specialsection{The Ratio of Males to Females\\ in Humans is not 1.0}
\lipsum[2-5]
\specialsection{Variation in Chromosome Number:\\
Terminology and Origin}
\lipsum[1]
\bigskip
\noindent\textcolor{teal}{\large\bfseries\sffamily Monosomy}
\smallskip

\noindent\lipsum[2-3]


\specialsection{Background to the\\
sectioning commands}

The \LaTeX2e\ method of constructing the layout for Chapters is complicated and spread all over the book.cls code. Although not very difficult to customize, customization is not user friendly.

\begin{description}
\item [counters] Counters can be displayed or not. These are constructed using the normal LaTeX method.
   \begin{verbatim}
   \renewcommand \thechapter {\@arabic\c@chapter}
   \end{verbatim}

\item [name] Here we use the term \textit{name} to denote in english the word ``chapter''. This can be typeset differently, depending on the language. It depends on on redefining one macro.
   \begin{verbatim}
     \def\chaptername{Chapter}
   \end{verbatim}
\item [openright] The global option open right, triggers the typesetting of chapter on odd pages only. There are a  couple of layouts that must be typeset on an even pages.

\item [\string\chapter] The chapter command is the main author command and where all the branching starts.
    \begin{verbatim}
\newcommand\chapter{%
  \if@openright\cleardoublepage\else\clearpage\fi
    \thispagestyle{plain}%
    \global\@topnum\z@
    \@afterindentfalse
    \secdef\@chapter\@schapter}
    \end{verbatim}

One limitation for this command is that it always starts a chapter on a new page and the macro needs to be rewritten if for example a new chapter is allowed to start anywhere.

Consider options openright, openleft, continuous.

The pagestyle is also settled here.

secdef will define basic macros for chaapter and starred chapter. What it basically does... this will become unecessary as we are going to find out a bit later on, but first the @chapter.

\item [\string\@chapter] This is the basic routine
\begin{verbatim}
\def\@chapter[#1]#2{
  \ifnum \c@secnumdepth >\m@ne
    \if@mainmatter
      \refstepcounter{chapter}%
      \typeout{\@chapapp\space\thechapter.}%
      \addcontentsline{toc}{chapter}%
         {\protect\numberline{\thechapter}#1}%
      \else
         \addcontentsline{toc}{chapter}{#1}%
    \fi
  \else
    \addcontentsline{toc}{chapter}{#1}%
  \fi
  \chaptermark{#1}%
  \addtocontents{lof}{\protect\addvspace{10\p@}}%
  \addtocontents{lot}{\protect\addvspace{10\p@}}%
  \if@twocolumn
    \@topnewpage[\@makechapterhead{#2}]%
  \else
    \@makechapterhead{#2}%
    \@afterheading
  \fi}
\end{verbatim}
  The important branching command here is makechapterhead,   which is responsible for typesetting the layout.

\end{description}
\end{multicols}


\section{Counters}
\begin{verbatim}
\renewcommand \thepart {\@Roman\c@part}
\renewcommand \thechapter {\@arabic\c@chapter}
\end{verbatim}



\section{major components}

The major components of a chapter opening, is the chapter name, the number and the title. It can be enclosed in boxes rules or other decorative elements.

One peculiarity is how to specify the position of the number.

leftofchaptername rightofchaptername ownline

\subsection{algorithmic approach}

The strategy in abstracting the chapter commands follows closely to that of laTeX.

First the chapter is called with the minimum of redefinitions. This then calls makechapterhead.


\begin{specialchapter}[
     image=genetics-dogs,
     image caption={Labrador retriever\\
         puppies expressing\\
         brown (chocolate),\\
         golden (yellow),\\
         and black\\
         coat colors,\\
         traits controlled\\
         by two gene pairs.}]%
{Sample\\ Chapter\\ Styles}
\begin{itemize}
\item The many permutations of variables affecting a chapter design, necessitate an interface that is easy to use and remember. The package provides an interface that a design can simply be changed by changing one word.
\item Learn how to select from a number of predefined styles, which you can view in the pages that follow.
\item Learn how to design your own styles and incorporate them easily in a new document.
\item The designs that follow have been selected from actual books. They may differ slightly in page geometry, spacing and fonts as I have tried to keep a somewhat unified design across this document.
\item i welcome contributions in terms of libraries and additional styles. What I have provided in this package is only a very small subset of what is possible to achieve.
\item Almost all styles are numbered, as this was the easiest way to incorporate so many designs. The few exceptions are noted in the relevant pages.
  \end{itemize}
\clearpage
\end{specialchapter}

%
\cxset{lineskip/.code=\setlength\lineskip{#1},
       lineskip/.default=1pt,
          normallineskip/.code=\setlength\normallineskip{#1},
          parindent/.code=\setlength\parindent{#1},
          parskip/.code=\setlength\parskip{#1},
          text-indent/.code=\setlength\parindent{#1},
          baselinestretch/.code=\renewcommand\baselinestretch{#1},
          single spacing/.code=\singlespacing,
          single spacing/.default=\singlespacing,
          double spacing/.code=\doublespacing}

\cxset{lineskip=1pt,
          normallineskip=1pt,
          parindent=1em,
          parskip=1pt,
          text-indent=1em,
          baselinestretch={},
          single spacing}

\makeatletter\@specialtrue\makeatother
\cxset{steward,
  numbering=arabic,
  custom=stewart,
  offsety=0cm,
  image={./images/hine05.jpg},
  texti={When Lamport designed the original \LaTeX\ sectioning commands, limitations of computer power forced him to restrict the abstraction of complicated chapter layouts. With current tools available improvements are much easier to program.},
  textii={In this chapter we discuss a method that allows the production of fancy chapter headings and formatting, based on a set of key values. Central  to this process is the separation of content from presentation.
We also discuss the basic formatting tools that are available and how one can modify them to mould new book designs.
 }
}
\cxset{chapter opening=left}

\chapter{General Settings}

\section{Introduction}

Here we define and set general paragraph settings. The parameters which control \TeX's behaviour when typesetting paragraphs can receive a bit of a tweak here. We also describe a set of options to handle parameters that can influence grid typesetting. This is especially important for two or more column typesetting. The commands act only on the text within a grouped environment. They do not affect captions or footnotes. Use anything over \emph{single spacing} with care, as books are meant to be single spaced.  



\section{Controlling inter-line spacing}
\index{line spacing}
Interline spacing traditionally has been controlled using the \pkgname{setspace} or by setting appropriate primitive \tex commands \cite{setspace}. The \pkgname{phd} loads the |setspace| package and then provides parameterized commands for setting styles. 

\begin{key}{/chapter/single spacing} 
	The Lineskip parameter emulates \TeX's \cmd{\parindent} command.
\end{key}
\begin{key}{/chapter/one half spacing} 
	The Lineskip parameter emulates \TeX's \cmd{\parindent} command.
\end{key}
\begin{key}{/chapter/double spacing} 
	Sets the document line-spacing to double.
\end{key}

If you want to use larger inter-line spacing in a document, you can change its value by putting the

\CMDI{\linespread}\meta{factor} Use |\linespread{1.3}| for "one and a half" line spacing, and |\linespread{1.6}| for "double" line spacing. Normally the lines are not spread, so the default line spread factor is~1.

The setspace package allows more fine-grained control over line spacing. To set "one and a half" line spacing document-wide, but not where it is usually unnecessary (e.g. footnotes, captions):

\begin{teXXX}
\usepackage{setspace}
%\singlespacing
\onehalfspacing
%\doublespacing
%\setstretch{1.1}
\end{teXXX}

The |phd| package provides the settings

\begin{key}{/chapter/single spacing}
We use the \pkgname{setspace} to effect the desired line spread effect.
\end{key}


These command offer little value over the normal \TeX\ macros other than keeping the interface, uniform. One can also extend the interface to cover CSS style commands:

\begin{verbatim}
\cxset{text-indent=50pt}

\cxset{double spacing}
\lipsum*[1]

\cxset{single spacing}
\lipsum*[1]
\end{verbatim}



\subsection{Parameters controlling paragraphs}\index{Paragraphs!controlling parameters}
The parameters \cs{lineskip} and \cs{normallineskip} influence \TeX\ when two lines come two close.
\medskip



\begin{key}{/chapter/lineskip=1pt} 
	The Lineskip parameter emulates \TeX's \cmd{\lineskip} command.
\end{key}

\begin{key}{/chapter/normallineskip=\marg{dim}} 
	The normallineskip parameter emulates \TeX's \cmd{\normallineskip} command.
\end{key}

\begin{key}{/chapter/lineskiplimit=\marg{dim}} 
	The Lineskip parameter emulates \TeX's \cmd{\lineskiplimit} command.
\end{key}

\begin{key}{/chapter/parindent=\marg{dim}} 
	The Lineskip parameter emulates \TeX's \cmd{\parindent} command.
\end{key}

\keyval{parindent}{\marg{dim}}{Paragraph indentation.}
\keyval{text-indent}{\marg{dim}}{Alias for \cs{parindent}.}
\keyval{parskip}{\marg{dim}}{Spacing between paragraphs.}


Another advantage, the package offers a few pre-configured styles, just setting a style to latex will revert everything back to latex.

\section{Technical discussion}

Most classes, including the standard \LaTeXe\ classes as well as packages attempting to achieve a grid typesetting try define a text height that is a multiple of \cs{baselineskip}. This way they give little opportunity to TeX to adjust the vertical glue to achieve a flush bottom.

\section{Dropcaps and Lettrines}\index{Lettrine!basic typesetting}

Dropcaps or lettrines are those letters that start paragraphs with a fancy larger letter. The class uses a parameterized version of the lettrine package of Daniel Flipo. Lettrine letters are easily typed and produced, but they are notoriously difficult to get right and no-one seems to agree on settings. These settings depend on the font the sizing of the text and the personal taste of the book interior designer. As I don't profess to be one, I have done what I think Knuth have done (just studied existing sources) allowed programming hooks and provided defaults as close as possible to the originals.



\cxset{steward,
  chapter toc=true,
  toc image=false,
  numbering=arabic,
  custom = stewart,
  offsety=0cm,
  image={./images/hine06.jpg},
  texti={A picture is worth a thousand words, but if you don't add a good description of what it is in a caption, your readers will be left scratching their heads. Here we discuss captions in general as well as the formatting commands available in LaTeX, some common packages and athena.},
  textii={In this chapter we discuss methods that allow the formatting and positioning of captions, based on a set of key values. Central  to this process is the separation of content from presentation.
We also discuss the basic formatting tools that are available and how one can modify them to blend them with the rest of the design.
 }
}
\cxset{section numbering prefix=\thechapter.}
\chapter{Typesetting Captions}
\section{Introduction}

Publications that include figures and tables will normally dictate
the style of captions. Captions, besides normal typography 
requirements such as fonts, can vary in their numbering scheme, can
include a label such as figure or fig they can include a colon or stop
after the label and can be centered hanged or left justified. 
Numbering can also vary; the counters can be reset at every chapter or section or can be continuous. So
there are quite a few options to define in a template.

The formatting commands for the captions key value interface follow the same style of the rest of the package. We use the \pkg{caption} package to provide the interface to the key value settings. To format the captions you just include the appropriate keys in one of the style
files.


\section{Conventions}

All caption keys start with the word |caption|. The float type follows, so |caption figure font-size| refers to the caption of a \textit{figure environment}. If the word \textit{figure} is omitted the style is applicable to both tables and figures. 

As users will probably only have to set these keys once, my recommendation is to use the longer version that can give you finer control. Also your template will be easier to modify in the future.
\medskip

{
\keyval{caption format}{\marg{plain|hang}}{This affects all captions such as tables and figures and will produce either a hang caption or with plain will wrap arund the figure number like a normal paragraph.}

\keyval{caption figure format}{\marg{plain | hang}}{Affects ONLY figure captions such as tables and figures and will produce either a hang caption or with plain will wrap around the figure number like a normal paragraph.}

\keyval{caption figure numbering style}{\marg{auto|continuous|reset on sections|custom}}{}
\keyval{caption figure numbering}{\marg{arabic|alph|Alph|roman|Roman|custom}}{Sets the style of numbering.}
\keyval{caption separator}{\marg{colonsemicolon|none|custom}}{Sets the separator, such as \textbf{:} or a colon or none.}
\keyval{caption label name}{\marg{text}}{Sets the label name such as figure.}
\keyval{caption aboveskip}{\marg{dim}}{Sets the \cs{belowcaptionskip}.}
\keyval{caption belowskip}{\marg{dim}}{Sets the \cs{abovecaptionskip}. You use as simply \texttt{10pt} ot similar. In LaTeX this value is normally set as \texttt{0pt}. Note that below a float normally an additional skip is introduced.}

\keyval{caption font}{\marg{bf|tt|it}}{Sets the font commands. }
\keyval{caption figure name}{Figure}{Sets the figure name}
\keyval{caption defaults}{\marg{true|false}}{Sets all styling back to default styles.}
}

Although it looks a simple piece of text, as you notice there are about
a dozen of variables that one could set. Color can be determined both
from the caption labl colour as well as from hyperlinking if necessary.
More complicated styles can be build in a simila fashion to chapter
heads, by diverting to a custom command \cs{captionspecial}. This
will be provided at the next release of the package.


\cxset{caption format/.code=\captionsetup[figure]{format=#1}} 


\begin{texexample}{}{}
\bgroup
\cxset{caption format = hang}
\includegraphics[width=80pt]{../graphics/sudan.jpg}
\captionof{figure}{This is a very long command to see how all
these can wrap in a hang format, if the text is longer than
a paragraph.}
\egroup

\bgroup
\cxset{caption format = plain}
\captionof{figure}{This is a very long command to see how all
these can wrap in a hang format, if the text is longer than
a paragraph.}
\egroup
\end{texexample}

As you can see from the example, the changes can also be localized if
they are within a group.



\makeatletter
\def\captionlabelfont@cx{bf}
\cxset{caption font/.code = \captionsetup[figure]{font=#1}}
\cxset{caption font={bf}}
\makeatother



\begin{texexample}{}{}
\cxset{caption format = hang}
\cxset{caption font={bf}}
\captionof{figure}{This is a very long command to see how all
these can wrap in a hang format, if the text is longer than
a paragraph.}
\end{texexample}



\section{Technical discussion}

The formatting of the caption, happens in stages like the sectioning commands.  |\@makecaption|  command is responsible for the typesetting and is defined in the standard LaTeX classes. The \cs{caption} and command is defined in the LaTeX kernel in the 
|float.dtx| class. As always we will start our discussion from the user command and follow it through to the typesetting macros.

When the user command \cs{caption} is processed, LaTeX checks if it is outside a float and if it is issues an error message. It then swallows the argument. It then calls \cs{@caption} which does further processing.

\startlineat{5}
\begin{teXXX}
\def\caption{%
  \ifx\@captype\@undefined
   \@latex@error{\noexpand\caption outside float}\@ehd
   \expandafter\@gobble
 \else
   \refstepcounter\@captype
  \expandafter\@firstofone
 \fi
 {\@dblarg{\@caption\@captype}}%
}

\long\def\@caption#1[#2]#3{%
  \par
  \addcontentsline{\csname ext@#1\endcsname}{#1}%
  {\protect\numberline{\csname the#1\endcsname}{\ignorespaces  #2}}%
  \begingroup
        \@parboxrestore
  \if@minipage
     \@setminipage
  \fi
  \normalsize
 \@makecaption{\csname fnum@#1\endcsname}{\ignorespaces #3}
 \par
 \endgroup}
\end{teXXX}


The \cs{@makecaption} is the main typesetting macro and this is
where we need to hook if we want finer grain of control.

\makeatletter
\cxset{label punctuation/.code = \gdef\labelpunctuation@cx{#1}}
\cxset{label space/.code = \gdef\labelhspace@cx{\hskip#1}}
\cxset{caption above skip/.store in= \abovecaptionskip@cx}
\cxset{caption above skip=10pt}
\makeatother

\captionof{figure}{This is a very long command to see how all
these can wrap in a hang format, if the text is longer than
a paragraph.}

\begin{texexample}{}{}
\cxset{caption format = hang}
\cxset{label punctuation=?}
\cxset{label space =1.5em}
\captionof{figure}{This is a very long command to see how all
these can wrap in a hang format, if the text is longer than
a paragraph.}

\end{texexample}



\cxset{label punctuation=?}
\captionof{figure}{This is a very long command to see how all
these can wrap in a hang format, if the text is longer than
a paragraph.}

\cxset{label punctuation=:}
\cxset{label space =.5em}
\def\figurename{\textbf{Figure}}

\makeatletter
\setlength\abovecaptionskip{\abovecaptionskip@cx}
\setlength\belowcaptionskip{0\p@}

\long\def\@makecaption#1#2{%
  \vskip\abovecaptionskip
  \sbox\@tempboxa{#1\labelpunctuation@cx #2}
  \ifdim \wd\@tempboxa >\hsize
    #1\labelpunctuation@cx\labelhspace@cx#2\par
  \else
    \global \@minipagefalse
    \hb@xt@\hsize{\hfil\box\@tempboxa\hfil}%
  \fi
  \vskip\belowcaptionskip}
\makeatother

\begin{teXXX}
\newlength\abovecaptionskip
\newlength\belowcaptionskip
\setlength\abovecaptionskip{10\p@}
\setlength\belowcaptionskip{0\p@}

\long\def\@makecaption#1#2{%
  \vskip\abovecaptionskip
  \sbox\@tempboxa{#1:: #2}
  \ifdim \wd\@tempboxa >\hsize
    #1:: #2\par
  \else
    \global \@minipagefalse
    \hb@xt@\hsize{\hfil\box\@tempboxa\hfil}%
  \fi
  \vskip\belowcaptionskip}
\end{teXXX}



\section{List of Figures}

\begin{docCommand}{listoffigures}{}
The list of figures (lof) is included on a page by using the command \cs{listoffigures}.
\end{docCommand}

The command is not defined in the kernel but rather in the standard classes as shown below. By default it uses the |\chapter| to typeset its heading. Commands like |\tableofcontents| that should set the marks in some page
styles use a |\@mkboth| command, which is |\let| by the pagestyle command |(\ps@...)| to |\markboth| for setting the heading or to |\@gobbletwo| to do nothing.\footnote{See source ltpage.dtx Date: 2000/06/02 Version v1.0k, page311.}

\begin{teXXX}
\newcommand\listoffigures{%
    \if@twocolumn
      \@restonecoltrue\onecolumn
    \else
      \@restonecolfalse
    \fi
    \chapter*{\listfigurename}%
      \@mkboth{\MakeUppercase\listfigurename}%
              {\MakeUppercase\listfigurename}%
    \@starttoc{lof}%
    \if@restonecol\twocolumn\fi
    }
\end{teXXX}



In the |phd| package this is set as a property via a key-value interface and hence we can use a normal chapter. If it need be we can define a special chapter style only for this heading. This way we can control all aspects of the formatting of the head.

\begin{docCommand}{\listfigurename}{}
The \textit{List of Figures} for example in many Social Sciences books is typed as {List of Illustrations} and also adds credits.
\end{docCommand}




\begin{figure}[htp]
\includegraphics[width=\textwidth]{./images/listofillustrations.jpg}
\caption{List of Illustrations extract from \textit{Oxford History of Art, Portraiture}, Shearer West, Oxford University Press, 2004.}
\end{figure}
\begin{figure}[htp]
\includegraphics[width=0.67\textwidth]{./images/titian.jpg}
\centering
\caption{Figure from \textit{Oxford History of Art, Portraiture}, Shearer West, Oxford University Press, 2004. The figures are numbered consecutively and the text in the List of Illustrations have different formatting.}
\end{figure}



\section{Formatting the List of Figures Heading}

LaTeX formats the list of figures heading in a similar manner to that of the Table of Contents. The Title `List of Figures` is obtained from the \cs{listfigurename} and which is also accessible from Babel. It does not add an entry to the ToC.



It is good to know that \cs{captionsetup} has an effect on the current environment only.
So if you want to change settings for the current figure or table only, just place the
\cs{captionsetup} command inside the figure or table right before the \cs{caption}
command.


Many of the caption figures can be changed within \latexe itself. For example to get continuous numbering in the book class.

\begin{teXXX}
\makeatletter
\@removefromreset{table}{chapter}
\renewcommand{\thetable}{\arabic{table}}
\makeatother
\end{teXXX}

\begin{docCommand}{removefromreset}{}
The command \cs{removefromreset} can be found by loading the \pkg{remreset} package. Other combinations are also possible.
\end{docCommand}

\subsection{Caption numbering scheme}

The caption numbering scheme key value interface, provides five
options: 
\medskip

\keyval{caption numbering scheme}{\marg{default|continuous| chapter|section}}{The numbering style either continous or reset per spacing etc...}


\begin{comment}
% Date: Sat, 30 Jul 1994 17:58:55 PST
% From: Donald Arseneau <asnd@erich.triumf.ca>
%
%  |\@removefromreset{FOO}{BAR}| : Removes counter FOO from the list of
%                       counters |\cl@BAR| to be reset when counter BAR
%                       is stepped.  The opposite of |\@addtoreset|.
\end{comment}


\begin{teXXX}

\makeatletter
\setdefaults
\cxset{chapter opening=anywhere,
          chapter font-size=\normalfont,
          title font-size=\large}

\def\@removefromreset#1#2{\let\@tempb\@elt
   \expandafter\let\expandafter\@tempa\csname c@#1\endcsname
   
   \def\@elt##1{\expandafter\ifx\csname c@##1\endcsname\@tempa\else
         \noexpand\@elt{##1}\fi}%
   \expandafter\edef\csname cl@#2\endcsname{\csname cl@#2\endcsname}%
   \let\@elt\@tempb}

\@removefromreset{figure}{chapter}
\renewcommand{\thefigure}{\arabic{figure}}

\@specialfalse\@tocfalse
\gdef\continuousfigures@cx{\@removefromreset{figure}{chapter}
%\gdef{\thefigure}{\arabic{figure}}}

\cxset{caption numbering continuous/.code={\continuousfigures@cx}}


\chapter{This is the First Chapter}

\captionof{figure}{test}

\captionof{figure}{test}

\chapter{This is the Second Chapter}

\captionof{figure}{test}
\captionof{figure}{test}
\makeatother

\end{teXXX}


\begin{figure}[htp]
\includegraphics[width=0.98\textwidth]{./images/captionspecial.jpg}
\centering
\caption{Figure from \textit{Oxford History of Art, Portraiture}, Shearer West, Oxford University Press, 2004. The figures are numbered consecutively and the text in the List of Illustrations have different formatting.}
\end{figure}



 NEED TO RESTORE
%%    \begin{macro}
%%    This macro is a helper macro to set the paper height and width
%%    we also save the paper name in its own macro.
%%    \begin{macrocode}
%\gdef\setpapersize@cx#1#2#3{%
%   \gdef\papername{#1}
%   \setlength\paperheight{#2}
%   \setlength\paperwidth{#3}
%   % headheight is common to all so we set it here
%   \setlength\headheight{12\p@}
%  % if pdf we need to set the pageheight and pagewidth
%  \global\pdfpageheight=#2
%  \global\pdfpagewidth=#3
%}
%%    \end{macrocode}
%%    \end{macro}
%%
%%    \begin{macro}
%%    \begin{macrocode}
%\def\setparams@cx#1#2#3{%
%    \def\X{#3}\def\XX{11pt}
%    % 11pt font set it as well
%    \ifx\X\XX
%          \@setfontsize\normalsize\@xipt{13.2}\selectfont%
%          \abovedisplayskip 13.2\p@ \@plus 3\p@ \@minus 3\p@
%          \abovedisplayshortskip \z@ \@plus 3\p@
%           \belowdisplayshortskip 6.6\p@ \@plus 3\p@ \@minus 3\p@
%    \else
%       \def\XX{12pt}
%        \ifx\X\XX
%           \@setfontsize\normalsize\@xiipt\@xivpt\selectfont
%           \abovedisplayskip 14.4\p@ \@plus 3\p@ \@minus 3\p@
%           \abovedisplayshortskip \z@ \@plus 3\p@
%          \belowdisplayshortskip 7.2\p@ \@plus 3\p@ \@minus 3\p@
%       \fi
%    \fi
%    \setlength\headsep{#3}
%    \setlength\footskip{#2}
%    \setlength\topskip{#3}
%    \setlength\maxdepth{0.5\topskip} % need to check
% }
%%    \end{macrocode}
%%    \end{macro}
%%
%%    We now set keys for all the paper sizes  
%\cxset{
%        a4paper/.code=\setpapersize@cx{a4paper}{297mm}{210mm},
%        a5paper/.code=\setpapersize@cx{a5paper}{210mm}{148mm},
%        a6paper/.code=\setpapersize@cx{a6paper}{105mm}{148},
%        b5paper/.code=\setpapersize@cx{b5paper}{250mm}{176mm},
%        letterpaper/.code=\setpapersize@cx{letterpaper}{11n}{8.5in},
%        legalpaper/.code=\setpapersize@cx{legalpaper}{14in}{8.5in},
%        executivepaper/.code=\setpapersize@cx{executivepaper}{10.5in}{7.25in},
%}
%%    the classical dimesions were obtained from the Octavo class
%%    we use mm or in depending on the type of paper standard
%\cxset{foolscap/.code=\setpapersize@cx{foolscap}{171mm}{108mm},
%          crown/.code=\setpapersize@cx{crown}{191mm}{127mm},
%          post/.code=\setpapersize@cx{post}{194mm}{122mm},
%          large post/.code=\setpapersize@cx{large post}{210mm}{137mm},
%          demy/.code=\setpapersize@cx{demy}{222mm}{143mm},
%          medium/.code=\setpapersize@cx{medium}{229mm}{146mm},
%          royal/.code =  \setpapersize@cx{royal}{254mm}{159mm},
%          superroyal/.code=\setpapersize@cx{superroyal}{267mm}{171mm}, 
%          imperial/.code=  \setpapersize@cx{imperial}{279mm}{191mm}}
%%   Lulu paper sizes
%%   http://wepod.wordpress.com/lulu-specs/
%%Manuscript Templates
%%6″ x 9″  US TRADE
%%(15.24cm x 22.86cm)
%%8.5″ x 11″
%%(21.59cm x 27.94cm)
%%Comic, 6.625″ x 10.25″
%%(16.827cm x 26.03cm)
%%Landscape, 9″ x 7″
%%(22.86cm x 17.78cm)
%%Square, 7.5″ x 7.5″
%%(19.05cm x 19.05cm)
%%Pocket Size, 4.25″ x 6.875″
%%(10.8cm x 17.46cm)
%%Royal, 15.6cm x 23.4cm
%%(6.14″ x 9.21″)
%%Crown Quarto, 18.9cm x 24.6cm
%%(7.44″ x 9.68″)
%%A4, 21.0cm x 29.7cm
%%(8.27″ x 11.69″)
%%   Set the parameters that depend on font-sizes
%\cxset{
%        lulu pocketbook/.code=\setpapersize@cx{lulu pocket book}{6.87in}{4.25in},
%	lulu digest/.code=\setpapersize@cx{lulu digest}{8.5in}{5.5in},
%	lulu us trade/.code=\setpapersize@cx{lulu us trade}{9in}{6in},
%	lulu royal/.code=\setpapersize@cx{lulu royal}{9.21in}{6.13in},
%	lulu comic/.code=\setpapersize@cx{lulu comic}{10.25in}{6.625in},
%	lulu crown quarto/.code=\setpapersize@cx{lulu crown}{9.68in}{7.44in},
%	lulu small square/.code=\setpapersize@cx{lulu small}{7.5in}{7.5in},
%	lulu square/.code=\setpapersize@cx{lulu large}{8.5in}{8.5in},
%	lulu landscape/.code=\setpapersize@cx{lulu landscape}{7in}{9in},
%	%lulu large landscape/.code=\setpapersize@cx{lulu large landscape}{}{},
%}
%
%\cxset{
%         10pt/.code=\setparams@cx{6pt}{25pt}{10pt},
%         11pt/.code=\setparams@cx{7pt}{27.5pt}{11pt},
%         12pt/.code=\setparams@cx{8pt}{30pt}{12pt} \@setfontsize\normalsize\@xiipt\@xivpt\selectfont,
%}%
%
%%   we need to set a default size before we determine the
%%   rest of the parameters.
% \cxset{a4paper,10pt}
%
%% does not seem to work
%%\@setfontsize\normalsize\@xiipt\@xivpt\normalsize
%
%%    set a default top margin first
%\def\topmarginauto{%
%\setlength{\topmargin}{0.1\paperheight}
%    \addtolength{\topmargin}{-\headheight}
%    \addtolength{\topmargin}{-\headsep}
%    \addtolength{\topmargin}{-1in}
%}
%
%\topmarginauto
%
%\cxset{topmargin/.code=\setlength{\topmargin}{#1}}
%\cxset{topmargin latex/.code=\topmarginauto}
%\cxset{topmargin latex}
%
%%   \section{Calculation of textwidth}
%%    The calculation of textwidth will depend on the strategy employed to calculate it.
%% \begin{macro}{\textwidth}
%%    Define the width of the text block to 0.7 of the page width, and make
%%    calculations a little easier by adjusting the calculated width to a 
%%    whole number of points.
%%    \begin{macrocode}
%\iffalse
%\setlength{\textwidth}{0.7\paperwidth}
%    \@settopoint\textwidth
%%    \end{macrocode}
%% \end{macro}
%%
%% \begin{macro}{\textheight}
%%    The height of the text block itself is set to 0.7 times the page height. 
%%    This amount is then adjusted to ensure that a whole number of lines makes 
%%    up the text block, and does so exactly.
%%    \begin{macrocode}
%\setlength\@tempdima{0.7\paperheight}
%%    \end{macrocode}
%%    take away the first line, which is a bit shorter than the |\baselineskip|,
%%    \begin{macrocode}
%    \addtolength\@tempdima{-\topskip}
%%    \end{macrocode}
%%    this length may be very close, but just a little too small to accommodate 
%%    one more line, so we add a small amount,
%%    \begin{macrocode}
%    \addtolength\@tempdima{5\p@}
%%    \end{macrocode}
%%    and calculate the number of lines in this length,
%%    \begin{macrocode}
%    \divide\@tempdima\baselineskip
%    \@tempcnta=\@tempdima
%%    \end{macrocode}
%%    The correct textheight comes to the number of lines just calculated, 
%%    multiplied by the height of text lines, |\baselineskip|, and with the 
%%    addition of the |\topskip| we took away initially.
%%    \begin{macrocode}
%    \setlength\textheight{\@tempcnta\baselineskip}
%    \addtolength\textheight{\topskip}
%%    \end{macrocode}
%% \end{macro}
%%
%% \subsubsection{Margin dimensions}
%%     Now that we have set the size of the text block, the amount of space
%%     available for margins is set as well. The remaining white space is divided
%%     in a 1:2 ratio, hence the proportions between margins and text become 1:7:2.
%%
%% \begin{macro}{\evensidemargin}
%% \begin{macro}{\oddsidemargin}
%%    Since we are typesetting books, both even and odd side margins have to be
%%    set.
%%    \begin{macrocode}
%\setlength{\evensidemargin}{0.2\paperwidth}
%\addtolength{\evensidemargin}{-1in}
%\setlength{\oddsidemargin}{0.1\paperwidth}
%\addtolength{\oddsidemargin}{-1in}
%%    \end{macrocode}
%
%\fi
%%% end of octavo algorithm and calculations
%
%%    Define an innermargin to enable easy drawing of parameters
%\newlength\innermargin
%\newlength\lefttrim
%\newlength\bottomtrim
%
%%    The stockheight and stockwidth are used when the paper is to be trimmed
%%    they default to the dimensions for paper width and paper height
%\@ifundefined{stockheight}{\global\newlength\stockheight}{}
%\@ifundefined{stockwidth}{\global\newlength\stockwidth}{}
%\ifdim\stockheight=0pt\addtolength\stockheight{\paperheight}\fi
%   \addtolength\stockheight{0mm}
%%
%\ifdim\stockwidth=0pt\addtolength\stockwidth{\paperwidth}\fi
%   \addtolength\stockwidth{0mm}
%%
%%   We set all the trims to zero to start with.
%\setlength\lefttrim{0mm}
%\setlength\bottomtrim{0mm}
%\setlength\trimtop{0mm}
%\setlength\trimedge{0mm}
%%
%%   
%
%
%%% This is a sidenote without the footnote mark
%%\newcommand\marginnote[2][0pt]{%
%% % \let\cite\@tufte@infootnote@cite%   use the in-sidenote \cite command
%%  %\gdef\@tufte@citations{}%           clear out any old citations
%%  \@tufte@margin@par%                 use parindent and parskip settings for marginal text
%%  \marginpar{\hbox{}\vspace*{#1}\marginparfont@cx\marginparjustification@cx\vspace*{-1\baselineskip}\noindent #2}%
%%  \@tufte@reset@par%                  use parindent and parskip settings for body text
%%  %\@tufte@print@citations%            print any citations
%%  %\let\cite\@tufte@normal@cite%       go back to using normal in-text \cite command
%%}
%
%% This macro has been adapted from the layouts package, it sets the units to be printed
%% in the diagrams.
%\newcommand{\printinunitsof@cx}[1]{%
%  \def\l@yunitperpt{1.0}\def\l@yunits{pt}%
%  \def\l@yta{#1}\def\l@ytb{pt}%
%  \ifx \l@yta\l@ytb
%    \def\l@yunitperpt{1.0}\def\l@yunits{pt}%
%  \else
%    \def\l@ytb{pc}%
%    \ifx \l@yta\l@ytb
%      \def\l@yunitperpt{0.083333}\def\l@yunits{pc}%
%    \else
%      \def\l@ytb{in}%
%      \ifx \l@yta\l@ytb
%        \def\l@yunitperpt{0.013837}\def\l@yunits{in}%
%      \else
%        \def\l@ytb{mm}%
%        \ifx \l@yta\l@ytb
%          \def\l@yunitperpt{0.351459}\def\l@yunits{mm}%
%        \else
%          \def\l@ytb{cm}%
%          \ifx \l@yta\l@ytb
%            \def\l@yunitperpt{0.0351459}\def\l@yunits{cm}%
%          \else
%            \def\l@ytb{bp}%
%            \ifx \l@yta\l@ytb
%              \def\l@yunitperpt{0.996264}\def\l@yunits{bp}%
%            \else
%              \def\l@ytb{dd}%
%              \ifx \l@yta\l@ytb
%                \def\l@yunitperpt{0.9345718}\def\l@yunits{dd}%
%              \else
%                \def\l@ytb{cc}%
%                \ifx \l@yta\l@ytb
%                  \def\l@yunitperpt{0.0778809}\def\l@yunits{cc}%
%%                \else
%%                  \def\l@ytb{PT}%
%%                  \ifx \l@yta\l@ytb
%%                    \def\l@yunitperpt{1.0}\def\l@yunits{PT}% gives problems with pgfmathparse
%%                  \fi
%                \fi
%              \fi
%            \fi
%          \fi
%        \fi
%      \fi
%    \fi
%  \fi
%}
%
%% Define keys to set it
%\cxset{geometry units/.code=\printinunitsof@cx{#1}}
%\cxset{geometry units=pt}
%
%% #1 value in pts
%% default in mm sorry USA.
%% rounding in 1 decimal place
%\def\convert@cx#1{%
%   \pgfmathparse{#1*\l@yunitperpt}
%   %\pgfmathround{\pgfmathresult}
%   \pgfmathresult\thinspace\l@yunits
%}
%
%% Layout related macros to go to separate style file
%\def\aspectratio{\pgfmathparse{\paperheight/\paperwidth} \pgfmathresult}
%
%
%
%
%% Set to true to draw an oddside page. Initially set to false.
%\newcommand\layoutscale@cx{0.4}
%
%\newif\ifoddpagelayout@cx
%   \oddpagelayout@cxtrue
%
%% Set true to draw marginpars on a page
%\newif\ifdrawmarginpars
%   \drawmarginparstrue
%
%% This draws a two page spread
%\newlength\bindingcorrection
%\newlength\oneninth
%\newlength\sixninths
%\setlength\oneninth{\dimexpr(\paperwidth/9)}
%\setlength\sixninths{\dimexpr(\paperwidth*6/9)}
%\let\trytextwidth\sixninths
%
%
%\newcommand{\alphabet}{\normalfont\selectfont\raggedleft abcdefghijklmnopqrstuvwxyz}%82
%
%
%
%\newcommand\charactersperline{%
%  \settowidth{\@tempdima}{\alphabet}
%  \pgfmathparse{\textwidth/\@tempdima*26}
% \pgfmathprintnumber{\pgfmathresult}
%}
%
%\newcommand\alphabetsperline{
%  \settowidth{\@tempdima}{\alphabet}
%  \pgfmathparse{\textwidth/\@tempdima}
%  \pgfmathresult
%}
%
%\newlength\alphlength
%\newcommand\alphabetlength{%
%  \settowidth{\alphlength}{\alphabet}
%  \pgfmathparse{\alphlength}
%  \pgfmathprintnumber{\pgfmathresult}pt
%}
%
%% We need to use the fp package to calculate the ratios, as PGF has problems with large 
%% dimensions or I am making an error
%\newcommand\textarearatio{%
%    \FPmul{\result}{\strip@pt\textwidth}{\strip@pt\textheight}
%    \FPmul{\resulti}{\strip@pt\paperwidth}{\strip@pt\paperheight}
%    \FPdiv{\resultii}{\result}{\resulti}
%    \pgfmathprintnumber{\resultii}
%}
%
%% Calculate the ratio textheight/paperheight
%\newcommand\textheightratio{%
%    \FPdiv{\result}{\strip@pt\textheight}{\strip@pt\paperheight}
%    \FPround{\result}{\result}{2}
%    \result
%}
%
%% Calculate textheight/paperwidth
%
%\newcommand\textheighttopaperwidth{%
%    \pgfmathparse{\textheight/\paperwidth}
%    \pgfkeys{/pgf/number format/.cd,fixed,precision=2}
%    \pgfmathprintnumber{\pgfmathresult}
%}
%
%\newlength\margintop
%
%\newcommand\thetop{%
%   \pgfmathparse{1in+\topmargin+\headheight+\headsep}
%   \pgfmathsetlength{\margintop}{\pgfmathresult}
%}
%
%\thetop
%
%\newlength\marginbottom
%\newcommand\thebottom{%
%   \pgfmathparse{\stockheight-(1in+\topmargin+\headheight+\headsep+\textheight)}
%    \pgfmathsetlength{\marginbottom}{\pgfmathresult}
%  }
%\thebottom
%
%\newcommand\verticalmarginratio{%
%\pgfmathparse{(\paperheight-(1in+\topmargin+\headheight+\headsep+\textheight))/  (\paperheight-(1in+\topmargin+\headheight+\headsep+\textheight))}
%\pgfmathresult
%}
%
%\newcommand\horizontalmarginratio{%
%\pgfmathparse{(\paperwidth-\textwidth-\oddsidemargin)/(1in+\oddsidemargin)}
%\pgfmathresult
%}
%
%\newcommand\numbertextlines{%
%% baselineskip to be corrected
%   \pgfmathparse{(\textheight-\topskip)/(12)-1}\pgfmathresult
%}
%
%\cxset{geometry units=mm}
%
%\def\printgeometryvalues{%
%   \noindent
%   \begin{tabular}{ll}
%   paper name & \papername\\
%   stock height & \convert@cx{\stockheight}\\
%   stock width  & \convert@cx{\stockwidth}\\
%   paperwidth & \convert@cx{\paperwidth}\\
%   paperheight & \convert@cx{\paperheight}\\
%   voffset & \convert@cx{\voffset}\\
%   hoffset & \convert@cx{\hoffset}\\
%   thetextheight & \convert@cx{\textheight}\\
%   thetextwidth  & \convert@cx{\textwidth}\\
%   Top margin   &  \thetop\convert@cx{\the\margintop}\\  % need to correct
%   Bottom margin & \thebottom\\
%   thetopmargin & \convert@cx{\topmargin}\\
%   theheadheight & \convert@cx{\headheight}\\
%   theheadsep & \convert@cx{\headsep}\\
%   theoddsidemargin & \convert@cx{\oddsidemargin}\\
%   theevensidemargin & \convert@cx{\evensidemargin}\\
%   themarginparsep& \convert@cx{\marginparsep}\\
%   themarginparwidth& \convert@cx{\marginparwidth}\\
%   themarginpush& \convert@cx{\marginparpush}\\
%   thevoffset& \convert@cx{\voffset}\\
%   thefootskip& \convert@cx{\footskip}\\
%   aspect ratio \aspectratio\\
%   twoside&  \if@twoside true\else false\fi\\
%   reversemarginpar& \if@mparswitch true \else false\fi\\
%  \end{tabular}
% }
%
%\def\readability{%
%\begin{tabular}{lr}
%  Characters per line &\charactersperline\\
%  Alphabets per line &\alphabetsperline\\
%  Alphabet length &\alphabetlength\\
%  Baselineskip & \the\baselineskip\\
%  Number of text lines &\numbertextlines\\
%  Text area ratio &\textarearatio\\
%  textheight/paperwidth&\textheighttopaperwidth\\
%  Text/page height ratio & \textheightratio\\
%  Vertical margin ratio &\verticalmarginratio\\
%  Horizontal margin ratio &1:\horizontalmarginratio\\
%\end{tabular}}
%
%
%% Note with new geometry paper has to be defined in preamble
%% I do not feel very confident of this
%% Don't understand it fully how is working
% %\@twosidefalse \@mparswitchfalse % one side option
%%\cxset{geometry oxford/.code={
%%\newgeometry{left=74.8mm,top=27.4mm,headsep=2\baselineskip,%
%%marginparsep=8.2mm,marginparwidth=49.4mm,textheight=49\baselineskip,headheight=\baselineskip}
%%\@twosidefalse \@mparswitchfalse % one side option
%%\reversemarginpar
%%}}
%% \@mparswitchfalse
%%\cxset{geometry textwidth/.store in=\textwidth@cx,
%%          geometry textheight/.store in=\textheight@cx,
%%          geometry tufte/.code={
%%             \newgeometry{a4paper,left=24.8mm,top=27.4mm,headsep=2\baselineskip,%
%%             textwidth=107mm,marginparsep=8.2mm,marginparwidth=49.4mm,%
%%             textheight=\textheight@cx\baselineskip,headheight=\baselineskip}
%%            \@twosidefalse \@mparswitchfalse % one side option
%%           %\reversemarginpar
%%    }
%%}
%%
%%
%%\cxset{marginpar push/.store in=\marginparpush@cx,
%%          marginpar font/.store in=\marginparfont@cx,
%%          marginpar justification/.is choice,
%%          marginpar justification/justifying/.code=\gdef\marginparjustification@cx{\justifying},
%%          marginpar justification/raggedright/.code=\gdef\marginparjustification@cx{\raggedright},
%%          marginpar justification/RaggedRight/.code=\gdef\marginparjustification@cx{\RaggedRight},
%%          marginpar justification/RaggedLeft/.code=\gdef\marginparjustification@cx{\RaggedLeft},
%% }
%%%\cxset{marginpar push=10pt,
%%%          marginpar font=\normalfont\footnotesize\sffamily,
%%%          marginpar justification=RaggedLeft}
%%%
%%%
%%%\cxset{style13, geometry textheight=47,
%%%          %geometry tufte,
%%%          watermark text=SAMPLE TUFTE VARIANT,
%%%          watermark text color=thered,
%%%          header style=samplepage}
%%%%%%%%%%%%%%%%%%%%%
%
%%%%%%%%%%%%%%%%%%%%%%%%%%%%%%%%%%%%%%%%%%%%%%%%%%%%%%%%%%%%%%%%%%%%%%%%%%
%%    DRAW THE PAGE ON A TRIAL BASIS
%%
%%%%%%%%%%%%%%%%%%%%%%%%%%%%%%%%%%%%%%%%%%%%%%%%%%%%%%%%%%%%%%%%%%%%%%%%%%%
%
%\cxset{geometry units= in}
%% lots of keys for trial sizes. We default all sizes to the ones defined in
%% by the document class.
%
%% We first set keys for the vertical dimensions
%\newlength\trytextheight@cx
%\newlength\tryheadheight@cx
%\newlength\tryheadsep@cx
%\newlength\tryfootskip@cx
%
%% LaTeX uses a correction to adjust the top margin, which is called topmargin. It does not 
%% represent the top margin though which following geometry we denote as top. It could perhaps
%% better be called top margin correction
%
%\newlength\trytopmargin@cx
%
%% Set keys for all the vertical dimensions and default to the current document settings
%\cxset{try textheight/.code=\global\setlength\trytextheight@cx{#1},
%          try textheight/.default=\textheight,
%          try headheight/.code=\global\setlength\tryheadheight@cx{#1},
%          try headheight/.default=\headheight,
%          try headsep/.code=\global\setlength\tryheadsep@cx{#1},
%          try headsep/.default=\headsep,
%          try footskip/.code=\global\setlength\tryfootskip@cx{#1},
%          try footskip/.default=\footskip,
%          try topmargin/.code=\global\setlength\trytopmargin@cx{#1},
%          try topmargin/.default=\topmargin,
%}
%
%% Set keys for all the trims, different people have different names for them. Normally two trims are
%% specified the top trim and the edge trip. We define two others just in case and to make calculations
%% easier if we have to use a different stock paper from the actual virtual paper width. the virtual
%% paper is called the paperwidth and paperheight.
%
%% We need to pick-up the memoir and koma allowances. TODO!
%\newlength\trimtop@cx
%
%\cxset{try trimtop/.code=\global\setlength\trimtop@cx{#1},
%          try trimtop/.default=\global\setlength\trimtop{0pt},}
%
%% set all the defaults
%
%\cxset{try textheight,
%          try headheight,
%          try headsep,
%          try footskip,
%          try topmargin=0pt, % compensate for trim
%          try trimtop=0pt}
%
%\addtolength\trytopmargin@cx{0pt}
%
%% set horizontal keys
%\newlength\trytextwidth@cx
%\setlength\trytextwidth@cx{0pt}
%\newlength\trytrimedge@cx
%\setlength\trytrimedge@cx{0pt}
%
%\cxset{try textwidth/.code=\global\setlength{\trytextwidth@cx}{#1},
%          try trimedge/.code=\global\setlength{\trytrimedge@cx}{#1},
%}
% 
%\cxset{try textwidth=\textwidth,
%          try trimedge=0pt}
%
%\def\alignedge{%
%% removed parindent from here must add it at the image
%  \checkoddpage%
%%   \ifoddpage \global\setlength\innermargin{\oddsidemargin}
%%          \else \global\setlength\innermargin{\evensidemargin}
%%      \fi%
%%   \if@twoside\setlength\innermargin{\dimexpr(\evensidemargin-\marginparsep)}%
%%             \else\let\innermargin\oddsidemargin\fi
%   \ifoddpage 
%      \innermargin\oddsidemargin
%      \def\innermarginname{oddsidemargin}%
%     \else
%        \innermargin\evensidemargin
%        \def\innermarginname{evensidemargin}%
%  \fi
%  }
%
%\alignedge
%
%
%%\ifoddpage
%%  \addtolength\innermargin{50pt}
%%\else
%%  \addtolength\innermargin{20pt}
%%\fi
%%\addtolength\trytextheight@cx{-20pt}
%%\addtolength\trytextwidth@cx{-24pt}
%%\addtolength\marginparwidth{-24pt}
%
%\reversemarginparfalse
%
%\def\drawlayout{%
%  \checkoddpage
%   \alignedge
%
%\tikzset{dim/.style = {>= latex,color=black}}
%\begin{tikzpicture}[scale=0.45,font={\scriptsize\rmfamily},line width=.8pt,
%       every node={color=black}]
%
%% first we draw stockwidth and stockheight
%\draw [color=gray,fill=thegray] (0,0) rectangle ++(\stockwidth,\stockheight);
%
%% draw the paper 
%\ifoddpage
%  \draw [color=NavyBlue,dashed thick,fill=white]  (0+\lefttrim,\stockheight-\trimtop@cx) rectangle ++ 	(\stockwidth-\lefttrim-\trytrimedge@cx,-\stockheight+\trimtop@cx+\bottomtrim);
%\else
% \draw [color=NavyBlue,dashed thick,fill=white]  (0+\lefttrim+\trytrimedge@cx,\stockheight-\trimtop@cx) rectangle ++ (\stockwidth-\lefttrim-\trytrimedge@cx,-\stockheight+\trimtop@cx+\bottomtrim);
%\fi
%% dimensions one more try
%%\cxset{geometry units=mm}
%% paper width dimensions, better to change to a macro
%% tol is the distance to dimension
%
%% paper width
%\edef\tol{-2.5\baselineskip}
%\coordinate (A) at (0+\lefttrim,\tol);
%\coordinate (B) at (\stockwidth-\trimedge,\tol);
%\coordinate (C) at (0.5\stockwidth,\tol);
%\draw[dim, |<->|] (A) -- (B); 
%\node at (C) [yshift=0.5\baselineskip)]{paper width = \convert@cx{\paperwidth}};
%
%% stockwidth
%\edef\tol{-5\baselineskip}
%\coordinate (BD) at (0,\tol);
%\coordinate (BD2) at (\stockwidth,-5\baselineskip);
%\draw[dim, |<->|] (BD) -- (BD2); 
%\draw (BD) ++ (0.5\stockwidth,0) node [yshift=0.5\baselineskip]{stockwidth=\convert@cx{\stockwidth}} ;
%
%% top dimension at left
%\coordinate (H1) at (-5mm,\stockheight);
%\coordinate (H2) at (-5mm,\stockheight-1in-\trytopmargin@cx-\tryheadsep@cx-\tryheadheight@cx);
%\draw [dim,|<->|] (H1) -- (H2);
%\node[left,text width=1.5cm, text ragged left] at (-5mm,\stockheight-0.5*\margintop){top\\ \convert@cx{\the\margintop}};
%
%% bottom dimension at left
%\coordinate (H3) at (-5mm,0);
%\coordinate (H4) at (-5mm,\marginbottom);
%\draw [dim,|<->|] (H3) -- (H4);
%\node[left] at (-5mm,0.5*\marginbottom){\convert@cx{\the\marginbottom}};
%
%% textheight at left
%\draw[dim,<->]  (-5mm, \marginbottom) -- ++ (0,\trytextheight@cx);
%\node[left,text width=1.5cm,text ragged left] at (-5mm,\marginbottom+0.5\trytextheight@cx){textheight \convert@cx{\trytextheight@cx}};
%
%
%% trimedge
%\ifoddpage
%  \coordinate (D) at (\stockwidth-4\trimedge, 0.10\trytextheight@cx);
%  \coordinate (E) at (\stockwidth,0.10\trytextheight@cx);
%  \draw [dim,->|] (D) -- ++(3\trimedge,0);
%  \draw [dim,|<-|] (E) -- ++(3\trimedge,0) node at ++(0,0) [right,text width=2cm,color=black] {trim edge    \convert@cx{\the\trimedge}};
%\else
%%  \coordinate (D1) at (\trytrimedge@cx, 0);
%%  \coordinate (E1) at ++ (\trytrimedge@cx,\stockheight-\trimtop@cx);
%%  \draw (D1)--(E1);
%\fi
%
%
%% toptrim
%%\ifdim\trimtop>0pt
%  \coordinate (F) at (0.9\stockwidth, \stockheight-\trimtop@cx-8mm);
%  \coordinate (G) at (0.9\stockwidth, \stockheight-\trimtop@cx);
%  \coordinate (H) at (0.9\stockwidth,\stockheight);
%  \draw (F)[dim,->|] -- (G);
%  \draw (H) -- ++ (0,8mm) -- ++ (5mm,0)[|<-|,>=latex] 
%          node [right] at ++ (0,0) {top trim =  \convert@cx{\the\trimtop@cx}};
%%\fi
%
%% 1in offsets
%\draw[dashed,color=gray] (1in,0) -- (1in,\stockheight);
%\draw[dashed,color=gray] (0in,\stockheight-1in)-- ++ (\stockwidth,0);
%
%% oddsidemargin/evensidemargin
%% draw dimension and name based on even or odd page
%\draw[dim,|<->|] (0,0.1\trytextheight@cx) -- ++(1in+\innermargin,0) node[right] at ++ (2ex,0) [text width=2cm] {\innermarginname\  \convert@cx{\the\innermargin}};
%
%% HEADER
%\coordinate (I) at (1in-\lefttrim+\innermargin,\stockheight-1in-\tryheadheight@cx-\trytopmargin@cx+\trimtop@cx);
%\draw (I) rectangle ++ (\textwidth,\tryheadheight@cx);
%
%%\draw[dim,<->] (1.5in\tol,\stockheight) -- ++(0,-1in) node[above right] at ++ (0,0.2in) {1in + yoffset};
%
%% add in inch 
%\draw [dim,|-|] (\stockwidth+3ex,\stockheight-\trimtop@cx)
%      -- ++(0,-1in) node [right] at ++(2ex,0.65in) {offset=\convert@cx{1in}};
%
%%   add topmargin dimension
%\ifdim\topmargin>0pt
%\draw [dim,|-] (\stockwidth+3ex,\stockheight-1in+\trimtop@cx)
%      -- ++(0,-\trytopmargin@cx) node [right] at ++(2ex,0.5\trytopmargin@cx) {topmargin=\convert@cx{\topmargin}};
%\fi
%
%%  add headheight dimension
%\draw [dim,|-|] (\stockwidth+3ex,\stockheight-1in+\trimtop@cx-\trytopmargin@cx)
%        -- ++(0,-\tryheadheight@cx) node [right] at ++(2ex,0.5\tryheadheight@cx) {headheight=\convert@cx{\the\tryheadheight@cx}};
%
%%   add headsep dimension
%\draw [dim,|-] (\stockwidth+3ex,\stockheight-1in+\trimtop-\tryheadsep@cx-\tryheadheight@cx-\trytopmargin@cx)
%          -- ++(0,\tryheadsep@cx) node [below right] at ++(2ex,0){headsep = \convert@cx{\the\tryheadsep@cx}};
%
%% footskip dimension
%\draw [dim,|-|] (\stockwidth+3ex,\stockheight-1in+\trimtop@cx-\tryheadsep@cx-\tryheadheight@cx-\trytopmargin@cx-\trytextheight@cx) -- ++(0,-\tryfootskip@cx) node [right] at ++(2ex,0.5\tryfootskip@cx){footskip=\convert@cx{\the\tryfootskip@cx}};
%
%
%% textarea
%\coordinate (J) at (1in-\lefttrim+\innermargin-\trytrimedge@cx,\stockheight-1in+\trimtop@cx-\tryheadheight@cx-\trytopmargin@cx-\tryheadsep@cx-\trytextheight@cx);
%\draw[fill=lightgray!50] (J) rectangle ++ (\trytextwidth@cx,\trytextheight@cx);
%
%\draw[dim,<->|] (1in-\lefttrim+\innermargin,0.75\trytextheight@cx) -- ++(\trytextwidth@cx, 0)  node at ++(-0.5\trytextwidth@cx,0.5\baselineskip) {textwidth} node at ++ (-0.5\trytextwidth@cx,-\baselineskip) {\convert@cx{\the\trytextwidth@cx}};
%
%\pgfmathsetmacro{\gridx}{12}
%% draw grid
%\draw[xstep=(\paperwidth-\trimedge)/\gridx, ystep=(\stockheight-\trimtop@cx)/\gridx,color=gray,dotted]  (0,0) grid (\paperwidth,\paperheight); 
%%%   add textheight dimension
%%\draw [dim,-] (\stockwidth+3ex,\stockheight-1in+\trimtop-\headsep-\headheight-\topmargin) -- ++(0,-\textheight) node [right] at ++(2ex,0.5\textheight){textheight=\convert@cx{\the\textheight}};
%
%% footer
%\coordinate (I) at (1in-\lefttrim+\innermargin,  \stockheight-1in+\trimtop@cx-\tryheadheight@cx-\trytopmargin@cx-\tryheadsep@cx-\trytextheight@cx-\tryfootskip@cx);
%\draw (I) rectangle ++ (\trytextwidth@cx,\tryheadheight@cx);
%
%
%% marginpar
%\def\leftmarginpar{%
%    \draw [fill=Linen,opacity=0.7] (1in+\innermargin+\trytextwidth@cx+\marginparsep,   \stockheight-1in+\trimtop@cx-\trytopmargin@cx-\tryheadsep@cx-\tryheadheight@cx ) rectangle ++(\marginparwidth,-\trytextheight@cx);
% \draw [dim,|<->|] (1in-\lefttrim+\trytextwidth@cx+\innermargin+\marginparsep+\marginparwidth,0.75\trytextheight@cx) -- ++ (-\marginparwidth,0) node at ++(0.5\marginparwidth,0.5\baselineskip) {marginpar} node at ++(0.5\marginparwidth,-\baselineskip){\convert@cx{\the\marginparwidth}};
%}
%
%\def\rightmarginpar{%
% \draw [color=red] (1in+\innermargin-\marginparsep,\stockheight-1in+\trimtop@cx-\trytopmargin@cx-\tryheadsep@cx-\tryheadheight@cx ) rectangle ++(-\marginparwidth,-\trytextheight@cx);
%     \draw [dim,|<->|] (1in-\lefttrim+\innermargin-\marginparsep-\marginparwidth,0.75\trytextheight@cx) -- ++ (\marginparwidth,0) node at ++(-0.5\marginparwidth,0.5\baselineskip) {marginpar} node at ++(-0.5\marginparwidth,-\baselineskip){\convert@cx{\the\marginparwidth}};
%}
%
%\ifdrawmarginpars
%  \checkoddpage
%  \alignedge
%    \if@twoside
%         \ifoddpage
%            \leftmarginpar
%         \else
%            \rightmarginpar
%        \fi
%   \else
%  % one side paper
%        \leftmarginpar
%    \fi
%\fi
%
%% draw diagonal
%\ifoddpage
%     \draw [color=blue]  (\paperwidth-\trytrimedge@cx,0) -- (0, \stockheight-\trimtop@cx);
%  \else
%    \draw [color=blue] (\trytrimedge@cx,0) -- (\paperwidth,\paperheight-\trimtop@cx);
%\fi  
%\end{tikzpicture}
%}
%
%
%%%%%%%%%%%%%%%%%%%%%%%%%%%%%%%%%%%%%%%%%%%%%%%%%%%%%%%%%%%%%%%%%%
%%                 SPREAD DRAWN AS PER CLASSICAL RULES
%%                 FOR ILLUSTRATION PURPOSE
%%%%%%%%%%%%%%%%%%%%%%%%%%%%%%%%%%%%%%%%%%%%%%%%%%%%%%%%%%%%%%%%%%%
%\newlength\paperwidth@cx
%\newlength\paperheight@cx
%\setlength\paperwidth@cx{6in}
%\setlength\paperheight@cx{9in}
%\setlength\bindingcorrection{0.1in}
%
%\def\spread{%
%   \begin{tikzpicture}[scale=0.5,inner sep=0pt,outer sep=0pt]
%   % draw the two pages
%  
%   \draw[xstep=\paperwidth@cx/9,ystep=\paperheight@cx/9,color=blue] (0,0) rectangle (\paperwidth@cx,\paperheight@cx)  (\paperwidth@cx+\bindingcorrection,0) rectangle ++(\paperwidth@cx,\paperheight@cx);
%
%% draw the binding correction
%\draw[fill=gray, draw] (\paperwidth@cx,0)  rectangle (\paperwidth@cx+\bindingcorrection,\paperheight@cx);
%
%% draw grid
%
%\draw[xstep=(\paperwidth@cx)/9, ystep=(\paperheight@cx)/9,color=gray,]  (0,0) grid (\paperwidth@cx,\paperheight@cx);
%
%\draw[xstep=(\paperwidth@cx)/9, ystep=(\paperheight@cx)/9,color=red]  
%(6.2in,0) grid (12.2in,\paperheight@cx);
%
%% add type areas
%
%\draw[fill=purple] (2\paperwidth@cx/9,2\paperheight@cx/9) rectangle  ++(6/9*\paperwidth@cx,6*\paperheight@cx/9);
%
%\draw[fill=green] (\paperwidth@cx+\paperwidth@cx/9+\bindingcorrection,2\paperheight@cx/9) rectangle ++(6\paperwidth@cx/9,6\paperheight@cx/9);
%
%\ifdim\bindingcorrection>0pt
%\draw[color=white,font={\sffamily\bfseries}] node at (\paperwidth@cx+0.5\bindingcorrection, 0.5\paperheight@cx)[rotate=90,inner sep=0pt,outer sep=0pt] {BINDING CORRECTION};\fi
%
%\node [color=white,font={\sffamily\bfseries}] at (0.5\paperwidth,0.5\paperheight)  {LEFT PAGE};
%\node [color=white,font={\sffamily\bfseries}] at (1.5\paperwidth@cx+\bindingcorrection,0.5\paperheight@cx){RIGHT PAGE};
%
%% draw diagonals
%
%\draw [color=thegreen, line width=1.5pt] (0,0)-- (\paperwidth@cx,\paperheight@cx);
%\draw [color=thegreen, line width=1.5pt] (2\paperwidth@cx+\bindingcorrection,0)-- ++(-\paperwidth@cx,\paperheight@cx);
%
%% draw circles
%
%\draw [color=red] (0.5\paperwidth@cx,5\paperheight@cx/9) circle (0.5\paperwidth@cx);
%
%\end{tikzpicture}
%}
%
%
%
%

\chapter{Geometry and Page Dimensions}
\parindent1.5em

\section{Introduction}

Setting up the page geometry, is normally done by the class or if adjustments need to be made, most authors will use the package geometry. If you need to view the geometry and the values of the document layout you can use the pkg{layouts}. This package offers a set of convenience key values for setting up geometry in order to enable authors to have a comprehensive style sheet.

\section{How to set geometry via this package}

To set the geometry page of the whole document, set the keys in the preamble. To change the page geometry anywhere in the document use the appropriate style or keys where you want the page geometry to change.
Note that the paper zize can only be defined in the preamble. The package is more useful when loaded with predefined styles.

\begin{tcolorbox}
\begin{lstlisting}
\cxset{page geometry=medieval}
\end{lstlisting}
\end{tcolorbox}

In most instances you will want to load the geometry at the style sheet.


\section{Viewing the page geometry}

The package offers a number of keys to set documents either document wide or locally to change page 
parameters or to view the frames. this is very similar to what the layouts and geometry packages offer. We do
however use TikZ for these diagrams.

To incorporate a layouts diagram we offer two macros \cs{printlayout} and \cs{printlayoutvalues}. Both have associated styling keys.
\medskip

\section{The Ideal Page Layout}

Since the invention of writing, typographers, scribes and graphics artists have been on the quest to find the ideal
layout for a page. Figure~\ref{fig:medieval}, shows a probablee geometric method that was used to typeset such books as the Gutenburg bible. Tschischold was a major revivalist of the method. Since most measurements in those times were probably only done using a compass a ruler and possibly a square, dividing the page equally into a nine part grid was done by first drawings the diagonals that are shown in blue in the figure, the intersections were then determined from the red lines thus enabling the typed area to be demarcated. 


\begin{figure}[htbp]
\pgfmathsetmacro\xsteps{9}
\pgfmathsetmacro\ysteps{9}
\cxset{spread scale=0.3}
\drawclassicspread
\caption{The ideal medieval page spread.}
\label{fig:medieval}
\end{figure}

To the modern eye, pages typeset in this manner might look rather empty, so smadjustments are made to such layuots. However, one tries to keep the proportions approximately to those of the classical layouts.
 
\begin{figure}[htbp]
  \includegraphics[width=0.95\textwidth]{tchichold01}
  \caption{\protect\url{http://www.artlebedev.com/everything/izdal/novaya-tipografika/}}
\end{figure}

\section{Technical discussion}
\subsection{The LaTeX standard classes}

LaTeX has pre-build layouts that depend  on two variables, specified by the user: the \textit{paper size} and the \textit{font size}. Appropriate values for the rest of the page layout are then  calculated by the class algorithm or are preset to certain values.

\subsection{Other common classes}

The more generic classes such as memoir and koma-script offer extensive customization and calculation of page parameters. They all use the basic laTeX page terminology which they supplement for additional parameters.

The octavo class offers a set of paper sizes suited for classical layouts printed on classical sizes such as the octavo. Classes such as the tufte-book offer a fixed design and no special commands for parameter manipulation.

\subsection{Paper sizes}

Most people using LaTeX, will print on either a4paper or letterpaper sizes. If you going to bind the work it might be necessary to trip the paper a little bit during binding to make sure that the top and side of the book are not ragged. This is normally called the \textit{trim}. If the document is to be printed by a publishing house this might be done by the printer which will use a different size \textit{stock size}. They might also allow for two additional dimensions called the spinemargin or the foremargin.

\begin{table}[ht]
\caption{North American paper sizes.}
\begin{tabular}{lllll}
\toprule
Size &width (mm)  &Height (mm)  &Width (in) &Height (in)\\
\midrule
US Ledger   &432 &279 & 17.0 &11.0\\
US Tabloid &279 & 432 & 11.0 &17.0\\
US Letter  &216 & 279 & 8.5 &11.0\\
US Legal   &216 &356 & 8.5 & 14.0\\
Government Letter &203 & 267 & 8.0 &10.5\\
Junior Legal &203 & 127 & 8.0 & 5.0\\
\bottomrule
\end{tabular}
\end{table}

\clearpage

\begin{table}[ht]
\caption{A series paper sizes.}
\begin{tabular}{lllll}
\toprule
Size &width (mm)  &Height (mm)  &Width (in) &Height (in)\\
\midrule
A0   &841 &1189 &33.1 & 46.8\\
A1   &594 & 841 &23.4 & 33.1\\
A2   & 420 & 594 &16.5 &23.4\\
A3   &297 & 420 &11.7 &16.5\\
A4   &210 &297 &8.3 &11.7\\ 
A5   &148 & 210 &5.8 & 8.3\\
A6   &105 & 148 & 4.1 & 5.8\\
A7   & 74 & 105 & 2.9 & 4.1\\
A8   &52 & 74 & 2.0 & 2.9\\
A9   &37 & 52 & 1.5 & 2.0\\
A10  & 26 & 37 & 1.0 & 1.5\\
\bottomrule
\end{tabular}
\end{table}


\begin{table}[ht]
\caption{ANSI series paper sizes.}
\begin{tabular}{lllll}
\toprule
Size &width (mm)  &Height (mm)  &Width (in) &Height (in)\\
\midrule
ANSI A &216 &279 &8.5 &11.0\\
ANSI B &279 &432 &11.0 &17.0\\
ANSI C &432 &559 &17.0 &22.0\\
ANSI D &559 &864 &22.0 &34.0\\
ANSI E &864 &1118 &34.0 &44.0\\

\bottomrule
\end{tabular}
\end{table}

\clearpage

\section{Swedish Standard}
The Swedish standard SIS 014711 generalized the ISO system of A, B, and C formats by adding D, E, F, and G formats to it. Its D format sits between a B format and the next larger A format (just like C sits between A and the next larger B). The remaining formats fit in between all these formats, such that the sequence of formats A4, E4, C4, G4, B4, F4, D4, H4, A3 is a geometric progression, in which the dimensions grow by a factor 21/16 from one size to the next. However, the SIS 014711 standard does not define any size between a D format and the next larger A format (called H in the previous example). Of these additional formats, G5 and E5 are popular in Sweden for printing dissertations,but the other formats have not turned out to be particularly useful in practice and they have not caught on internationally.

\begin{table}[ht]
\caption{Swedish Extension}
\begin{tabular}{lllll}
\toprule
Size &width (mm)  &Height (mm)  &Width (in) &Height (in)\\
\midrule
G5 &169 &239 &6.65 &9.41\\
E5  &155 &220 &6.10 &8.66\\

\bottomrule
\end{tabular}
\end{table}


\begin{table}
\centering
\caption{Octavo page layout parameters, influenced by font-size}
\begin{tabular}{llll}
\toprule
                    & 10pt & 11pt &12pt \\
\midrule
\textit{Octavo}              &      &      &\\
headsep        &  6pt  &  7pt &  8pt\\
topskip          & 10pt &  11pt & 12pt\\
texwidth         &0.7paperwidth & &\\
\midrule
\textit{LaTeX}              &      &      &\\
headsep        & .25in   &  .275in & .275in \\
topskip          & 10pt &  11pt & 12pt\\
footskip         &.35in &  .38in & 30pt \\
maxdepth         &.5\textbackslash topskip & &\\
textwidth        & 345pt  & 360pt & 390pt\\
\bottomrule
\end{tabular}
\end{table}

\subsection{The page dimensions}

The page dimensions are shown in figure 1. We tried to cater for the common terminology of all the classes.

\subsection{Texwidth}

The width of the text can only be determined based on the designer's strategy and is inexorably tied also to
the textheight. For example in classical page design, the designer tried to get the textwidth to be the same size like the page width, thus giving an almost squarish look. Another strategy is the 6-9 strategy, where the paper is divided into a grid of 9 equal blocks and the textwidth occupies the 6. 

\subsection{Readability considerations}

An average line that is longer than 40 to 70 characters long -- inluding spaces, is difficult to read. This is generally applicable to European languages and might be different for other languages. In addition the average number of words in one line should also be considered. For the German language Willi Egger (2004) recommends a line consisting of 8 to 12 words as optimal. If this strategy is adopted one can determine the line length based on the number of letters.

The characters per line for this document is \charactersperline. Of course from a readabilty point of view one could keep increasing the font size, but this is poor strategy. A well designed page should allow for good proportions as well as readabilty. In general a tolerance up  to 80 characters on a line should be adequate.

Peter Wilson in the manual for the memoir class refers to equations developed by Morten H{\o}gholm\index{H{\o}gholm, Morten} that has done some curve fitting
to the data. He determined that the expressions
\begin{equation}
L_{65} = 2.042\alpha + 33.41 \label{eq:L65}
\end{equation}
and
\begin{equation}
L_{45} = 1.415\alpha + 23.03 \label{eq:L45}
\end{equation}
fitted aspects of the data, where $\alpha$ is the length of the alphabet
in points, and $L_{i}$ is the suggested width in points, for a line with
$i$ characters (remember that 1pc = 12pt).

Using these equations one could get a first estimate of the textwidth. I am not too sure though if this is a good strategy as one can calculate it fully using TeX. bringhurst and them had to read these values from tables, but we do not; we can easily calculate them. For example to calculate the alphabet length for the bookman font:

\begin{texexample}{}{}
  \bgroup
  \fontfamily{pbk}\selectfont\alphabetlength\\
  \charactersperline\\
  \the\textwidth
  \egroup
\end{texexample}

Table~\ref{tab:alphlengths} adapted from the memoir class, gives alphabet lengths in points for various
fonts. My own recommendation is that for wide paper you should use a wider font and possibly move to 11pt font, rather than the traditional LaTeX default of 10pt.

\begin{table}
\centering
\caption{Lowercase alphabet lengths, in points, for various fonts}\label{tab:alphlengths}
\begin{tabular}{lrrrrrrrr} \toprule
                                            & 8pt & 9pt & 10pt & 11pt & 12pt & 14pt & 17pt & 20pt \\ \midrule
\fontfamily{pbk}\selectfont Bookman         & 113 & 127 & 142 & 155 & 170 & 204 & 245 & 294 \\
\fontfamily{bch}\selectfont Charter         & 102 & 115 & 127 & 139 & 152 & 184 & 221 & 264 \\
\fontfamily{cmr}\selectfont Computer Modern & 108 & 118 & 127 & 139 & 149 & 180 & 202 & 242 \\
\fontfamily{ccr}\selectfont Concrete Roman  & 109 & 119 & 128 & 140 & 154 & 185 & 222 & 266 \\
\fontfamily{pnc}\selectfont New Century Schoolbook     & 108 & 122 & 136 & 149 & 162 & 194 & 234 & 281 \\ 	
\fontfamily{ppl}\selectfont Palatino        & 107 & 120 & 133 & 146 & 160 & 192 & 230 & 276 \\ 	
\fontfamily{ptm}\selectfont Times Roman     &  96 & 108 & 120 & 131 & 143 & 172 & 206 & 247 \\
\fontfamily{put}\selectfont Utopia          & 107 & 120 & 134 & 146 & 161 & 193 & 232 & 277 \\
\fontfamily{pag}\selectfont Avant Garde Gothic  & 113 & 127 & 142 & 155 & 169 & 203 & 243 & 293 \\
\fontfamily{cmss}\selectfont Computer Sans  & 102 & 110 & 120 & 131 & 140 & 168 & 193 & 233 \\
\fontfamily{phv}\selectfont Helvetica       & 102 & 114 & 127 & 139 & 152 & 184 & 220 & 264 \\
\fontfamily{pcr}\selectfont Courier         & 125 & 140 & 156 & 170 & 187 & 224 & 270 & 324 \\
\fontfamily{cmtt}\selectfont Typewriter     & 110 & 122 & 137 & 149 & 161 & 192 & 232 & 277 \\
\bottomrule
%\facesubseeidx{Bookman}\facesubseeidx{Charter}\facesubseeidx{Computer Modern}%
%\facesubseeidx{Concrete Roman}\facesubseeidx{New Century Schoolbook}
%\facesubseeidx{Palatino}\facesubseeidx{Times Roman}\facesubseeidx{Utopia}%
%\facesubseeidx{Avant Garde Gothic}\facesubseeidx{Computer Sans}
%\facesubseeidx{Helvetica}\facesubseeidx{Courier}%
%\facesubseeidx{Computer Typewriter}%
\end{tabular}
\end{table}

\subsection{Textwidth influenced by margin materials}

Many books, including LaTeX allow for margin materials. If this is true then of course margins must by their nature be larger at the paper edges to allow for such material. 

\subsubsection{Simple strategy}
However, despite most of the above typesetting strategies many an author just want to take an approach, where they specify the margins and want to get what they need for example a spine margin of 1cm and an edge margin of 1.5cm. This is also important for screen dimensions.

\begin{lstlisting}
\cxset{
    margin inner= 1in
    margin outer= 2in
    margin top=1in
    margin bottom=2in
}
\end{lstlisting}

If all four margins are specified, the typesetting area can be positioned on the paper block. Life is not this easy though.

\begin{lstlisting}
\cxset{
    textarea proportional={1}{6}{2}  %
    textarea octavo
    textarea latex
    textarea other
}
\end{lstlisting}

\subsection{Auto strategy}

A more involved approach is to combine the strategies. First to get a good margin to type area, you will need to choose a paper that has good ratios. A paper such as \textit{imperial} comes close to an A4 size or imperial size. One can trip the balance of the paper or adjust slightly the ratios for twoside printing. Since we talking about book design any consideration for one side printing is immaterial. For oneside printing one can accept a wider latitude of values. Algorithm follows:

\begin{enumerate}
\item select paper.
\item select font.
\item marginmaterial true or false.
\item financial constraints - minimize number of pages, maximize number of pages.
\item check ideal number of characters at 65 per line.
\item decide on 10pt, 11pt or 12pt and constrain the algorithm.
\item use 0.7 textwidth area and check for max characters, if not iterate to 11pt.
\item recommend trimming values to suit.
\end{enumerate}


\subsection{Textheight}
Normally there is more latitude in choosing the 
proportions\index{proportion!margin} 
of the upper and lower margins, though usually the upper 
margin\index{margin!upper} is less than the lower margin\index{margin!lower}
so the typeblock\index{typeblock!location} is not vertically centered. Many modern books disregard all these rules and in many examples the upper margin is higher than the lower margin.

For text height calculations there are two considerations. One is to select top and bottom margings that are either equal or at a 1:1.5--2.0 ratio and relate to the width of the horizontal margins. The second consideration is that this length must be exactly divisible by baselineskip. When using \cs{flushbottom} LaTeX expects that the \cs{textheight} is such that a number of textlines in the body font will fit exactly into the height. If not, it issues an underfull vbox's message. LaTeX calculates these parameters when loading the class .clo files and sets the number of lines to a round number.

Many modern books have equal upper and lower margins.
\bigskip

\section{Allowing for trims}

Once a book is printed the edges are trimmed a bit in order to ensure a smooth top and right edge. For most desktop publishing you should not worry about such trimming. If you are going to publish the book in a professional publisher get their advice as to any allowances, you need to make in your pdf file. In other possible scenarios is that you may want to use a paper size such as A4 and trim it yourself down to one of the classical sizes such as Royal. 

All calculations are based on selecting a paper size and trimming it down. Unlike some other classes we assume you have selected the paper as stock size and then trimmed. Adding the trims makes no sense. You could simply print them on the larger page with trim marks, which we cater for.

  \begin{align}
   H_p    & = \sum h_1\ldots h_n\\
      h_t  &= H_p -   \sum h_1\ldots h_n - h_b
  \end{align}

The top margin is influenced by the \textit{device margin}, which is set at one inch, which we denote as $C$. If paper is to be trimmed the effective device margin offset will be reduced by the trim amount, $\Delta_t$.

Hence, the top margin is given by
\begin{align}
     h_t = C-\Delta_t+h1+h_2+h_3
\end{align}


\drawtriallayout


\printgeometryvalues
\readability

\newpage

\drawtriallayout

\readability

\newpage



% end of two page spread
\subsection{Top and bottom margins}

Before you follow any advice in places such as the Lulu forums to have your top and bottom margins equal, consider the following quotation by Bernard Shaw:

\begin{quotation}
Every printer can understand regularity: few have studied good looks except in living creatures. Consequently they aim at equal margins; and even when they have learnt that an upper margin must be less than a lower one if it is not to look more, they do not always see that it looks well only when it looks less. The mediaeval manuscript or early printed book, with its very narrow margin at the top and very broad margin at the bottom of the page, with its outer margins broad and its inner ones contracted, so that when the book lies open the two pages seem to make but a single block of letterpress in a single frame, instead of two side by side, has never been improved upon and probably never will be. But I find it almost impossible to persuade a modern printer to make his top margin small enough; and when I at last succeed, he measures it from the running title instead of from the top line of the page.

I saw a book the other day, excellently printed in old faced type, set solid, on a fine light, clean white crusty paper; yet the page was quite spoiled by an exaggerated top margin,like a masher's collar, and by that abomination of desolation, a rule. The only thing that never looks right is a rule. There is not in existence a page with a rule on it that cannot be instantly and obviously improved by taking the rule out.
\end{quotation}

\subsection{Headers and footers}
A page may have two additional items, and usually has at least one of these. They are the
running header and running footer. If the page has a folio then it is located either in the
header or in the footer. The word ‘in’ is used rather lightly here as the folio may not be
actually in the header or footer but is always located at some constant relative position. A
common position for the folio is towards the fore-edge of the page, either in the header or
the footer. This makes it easy to spot when thumbing through the book. It may be placed
at the center of the footer, but unless you want to really annoy the reader do not place it
near the spine.

Often a page header contains the current chapter title, with perhaps a section title on
the opposite header, as aids to the reader in navigating around the book. Some books put
the book title into one of the headers, usually the verso one, but I see little point in that as
presumably the reader knows which particular book he is reading, and the space would
be better used providing more useful signposts.

\subsubsection{Determining the geometry of the headers and footers}

The important parameter in the calculation of the header and footers, is the \cs{headheight} and \cs{headsep}. Most classes tend to have these as fixed parameters, related to font-size as can be seen in Table~\ref{tab:headerparams}.

\begin{table}[htbp]
\centering
\caption{Header and footer parameters settings by common classes.}
\label{tab:headerparams}
\begin{tabular}{llll}
\toprule
                    &headsep                   &headheight &footskip\\
\midrule
LaTeX 10pt    &             &                 &           \\
LaTeX 11pt    &             &                 &           \\
LaTeX 12pt    &             &                 &           \\
Octavo          &             &                 &           \\
tufte-book     &2 \texttt{baselineskip}   & 1 baselineskip           &            \\
\bottomrule
\end{tabular}
\end{table}

\begin{figure}[htbp]
\includegraphics[width=\textwidth]{paradoxicalbrain}

\caption{Modern book approach to footer and header design. From \textit{Paradoxical Brain,} Narinder Kapur \textit{et al.}, Cambridge Univerity Press, 2011. Book is printed on Royal size paper.}
\label{fig:paradoxical}
\end{figure}

\begin{figure}[htbp]
{{\parindent0pt
\begin{tikzpicture}[inner sep=0pt,outer sep=0pt]
  \node (img) {\includegraphics[height=8cm]{paradoxicalbrain}};
  \draw  (img.north east) ++ (5pt,0pt)-- ++ (15pt,0) ++(-15pt, -0.083*8cm) --++ (15pt,0pt) 
            (img.south east) ++ (5pt,0pt) -- ++ (15pt,0pt) ++ (0,0.075*8cm) -- ++ (-15pt,0);
\end{tikzpicture}}}
\caption{Modern book approach to footer and header design. From \textit{Paradoxical Brain,} Narinder Kapur \textit{et al.}, Cambridge Univerity Press, 2011. Book is printed on Royal size paper. The top margin is $1/12$ of the page height and the bottom margin is $1/16$ of page height. No need for apogryphal methods here.}
\label{fig:paradoxical}
\end{figure}

The more modern style tends to shift the headers and footers towards the top edge and bottom edge of the paper respectively, and allows very little space at the top of the paper. Figure~\ref{fig:paradoxical} shows a footer that is very near the bottom of the text and a header that has been shifted upwards. This makes for a more economical design as it increases the amount of text that can be printed in the typed area. For special designs such as this, it is not possible to automate calculations other than specifying a full algorithm for margins and typed area. Margins for the example follow the 10/12 rule for the typed area and inner and outer margins are equal at 1:12 ratio to the trimmed paper width.


\section{Floating parameters}


\section{Summing up}

Although one would ideally like to input some constraints and get out a perfect layout, as the previous discussion shows this is not an easy task, as well 

\begin{figure}[htbp]
\includegraphics[width=0.9\textwidth]{artbook}
\end{figure}

%%\end{document}
%\lipsum[1-4]\marginnote[1pt]{\lorem
%    \lorem}
%
%\lipsum[1-2]

%% Stick the caption in the head might as well place the first picture also
\def\asidecaption{\parbox{4.2cm}{{\bfseries Image \thefigure}\par\lorem}%
  % \addtocontents{lof}{This is image 8}
}
\def\ps@caption{%
     \let\@oddfoot\@empty\let\@evenfoot\@empty%
    \def\@evenhead{%
        \begin{picture}(0,0)%
           \put(-150,-80){\asidecaption\par}%
            \stepcounter{figure}
           \put(-150,-370){\asidecaption}%
        \end{picture}%
      }%
    \let\@oddhead\@evenhead%
    \let\@mkboth\@gobbletwo%
    \let\chaptermark\@gobble%
    \let\sectionmark\@gobble%
 }

\def\ps@bigpicture{%
    \setlength\headheight{19cm}%
    \let\@oddfoot\@empty\let\@evenfoot\@empty%
    \def\@evenhead{%
         \begin{picture}(0,0)%
          \put(-149,0){\includegraphics[width=\dimexpr(\textwidth+150pt)]{stuartpearson}}%
         \end{picture}%
      }%
    \let\@oddhead\@evenhead%
    \let\@mkboth\@gobbletwo%
    \let\chaptermark\@gobble%
    \let\sectionmark\@gobble%
 }



\def\doubletakeimage{%
  \renewcommand{\topfraction}{.95}  % ensure seecond image will not float away
  \begin{figure}[t]
    \thispagestyle{caption}
    \includegraphics[width=\textwidth]{matron}%
  \end{figure}

  \begin{figure}[tp]
   \hspace*{-\marginparwidth}\includegraphics[height=0.9\textheight]{stuartpearson}
 \end{figure}
}




\lipsum[1-4]
\begin{figure}[htp]
\includegraphics[width=0.98\textwidth]{captionspecial}
\centering
\caption{Figure from \textit{Oxford History of Art, Portraiture}, Shearer West, Oxford University Press, 2004. The figures are numbered consecutively and the text in the List of Illustrations have different formatting.}
\end{figure}

\doubletakeimage



%% RESET EVERYTHING AT END OF CHAPTER
\addtocounter{chapter}{-2}

\@toctrue\@specialtrue

\@specialfalse
\pdfpageheight=\paperheight
\pdfpagewidth=\paperwidth
\cxset{manet, toc image=false}
\cxset{toc image=false},
\topimage{manet}

\chapter{THE BARMAID}
\begin{multicols}{3}
      \leftskip0pt
      \lettrine{I}{psum dolor} sit amet latixeus. \lipsum*[1-2]
      Latinicus porcupinus to fill the line.
      \tikz{\draw[thick] (0,0)--(\columnwidth,0);}
\end{multicols}
\clearpage


%\cxset{manet, toc image=false}
%\cxset{toc image=false},
%\topimage{Alan-MacDonald-Cardinal-Spin-01}
%\chapter{THE BARMAID}
%\begin{multicols}{3}
%      \leftskip0pt
%      \lettrine{I}{psum dolor} sit amet latixeus. \lipsum*[1-2]
%      Latinicus porcupinus to fill the line.
%\end{multicols}
%\clearpage
%
%\restoregeometry
%\@specialfalse


\newgeometry{top=1.35cm,bottom=2cm,left=2cm}
\clearpage
\cxset{manet/.style={
 chapter opening=anywhere,
 chapter toc=true,
 toc image=false,
 name={},
 numbering=none,
 number font-size=,
 number font-family=,
 number font-weight=,
 number before={\vspace*{-2.5cm}},
 number dot={},
 number after={},
 number position=leftname,
 chapter font-family=,
 chapter font-weight=,
 chapter font-size=,
 chapter before=,
 chapter after={},
 chapter color={black!90},
 number color= teal,
 title beforeskip={},
 title afterskip={},
 title before={\hspace*{-2.47cm}\includegraphics[width=1.27\textwidth]{./chapters/manet}%
    \par\hfill\hfill{\tiny\bfseries Manet's  \textit{The Barmaid.}}\\
    \par
    \vspace*{\baselineskip}
    \par\hfill},
 title after={\hfill\hfill},
 title font-family=\sffamily,
 title font-color= black!80,
 title font-weight=\bfseries,
 title font-size=\LARGE}}
\cxset{manet}



\chapter{A New Approach to Designing \LaTeX\ Classes}
\begin{multicols}{3}
      \leftskip0pt
      \lettrine{I}{psum dolor} sit amet latixeus. \lipsum*[1-2]
      Latinicus porcupinus to fill the line.


This particular code, uses the predefined style \textit{manet}. The only difference we have now defined a helper macro to make it easier for such images to be inserted for similar style chapter openings.
If a full book is to be designed using chapter openings in this fashion more keys and styles could be defined to make it even more easy to enter.
\end{multicols}



\def\topimage#1{\cxset{title before={\hskip-2.3cm\includegraphics[width=1.25\textwidth]{./chapters/#1}\par
\vspace*{\baselineskip}\par}}}




The full code to have the chapter typeset is shown below:


\begin{lstlisting}
\cxset{manet}
\topimage{Alan-MacDonald-Cardinal-Spin-01}

\chapter{ALAN MacDONALD}
\begin{multicols}{3}
      \leftskip0pt
      \lettrine{I}{psum dolor} sit amet latixeus. \lipsum*[1-2]
      Latinicus porcupinus to fill the line.
\end{multicols}
\end{lstlisting}
\lipsum[2]

\loadgeometry{std}



%\makeatletter

\long\gdef\versochapter#1{
  \vspace*{3cm}
  \minipage{\textwidth}
  \hfill\includegraphics[width=0.5\textwidth]{\chapterimage@cx}\par
  \vspace*{6pt}
  \hfill\minipage{0.75\textwidth}
  {\HUGE\bfseries\flushright #1\endflushright}
  \endminipage
  \endminipage
  \newpage


\vspace*{10cm}
\@specialfalse
\@openleftfalse
\@openanyfalse
\@openrighttrue
}


\newgeometry{bottom=2.5cm}

\cxset{
   chapter image/.code={\def\chapterimage@cx{#1}},
   chapter opening/.is choice,
   chapter opening/verso/.code={\@specialtrue\@openlefttrue
   \gdef\customdesign@cx##1{\versochapter{##1}}}
}

\cxset{
 custom=versochapter,
 chapter image={vespa.jpg},
 chapter opening=verso,
 name={},
 numbering=none,
 number font-size=LARGE,
 number font-family=rmfamily,
 number font-weight=bfseries,
 number before=,
 number dot={},
 number after=,
 number position=leftname,
 chapter font-family=sffamily,
 chapter font-weight=\normalfont,
 chapter font-size=\Large,
 chapter before={\vspace*{0pt}\par},
 chapter after={\hfill\hfill\par},
 chapter color={black!90},
 number color=purple,
 title beforeskip={\vspace*{0pt}},
 title afterskip={\vspace*{0.4\textheight}\par},
 title before={},
 title after={},
 title font-family=sffamily,
 title font-color=purple,
 title font-weight=bfseries,
 title font-size=LARGE,
 header style=plain,
 pagestyle=plain,
 }

\makeatletter
\@specialtrue
\makeatother



\chapter{Verso Chapters}

\parindent1.5em

{\Huge T}he theme of this template is from a book called 
{ \textit{From Western attitudes toward death from the middle ages to the present}, Philippe Ari\'es. London, 1974.

\begin{figure}
\includegraphics[width=\textwidth]{./chapters/versochapter01.png}
\caption{Chapter opening on verso page.}
\end{figure}

I will quote Oliver Sacks words verbatim, as I cannot ever imagine that I can do it better:

\begin{quote}
A MONTH ago, I felt that I was in good health, even robust health. At 81, I still swim a mile a day. But my luck has run out — a few weeks ago I learned that I have multiple metastases in the liver. Nine years ago it was discovered that I had a rare tumor of the eye, an ocular melanoma. Although the radiation and lasering to remove the tumor ultimately left me blind in that eye, only in very rare cases do such tumors metastasize. I am among the unlucky 2 percent.

I feel grateful that I have been granted nine years of good health and productivity since the original diagnosis, but now I am face to face with dying. The cancer occupies a third of my liver, and though its advance may be slowed, this particular sort of cancer cannot be halted.

It is up to me now to choose how to live out the months that remain to me. I have to live in the richest, deepest, most productive way I can. In this I am encouraged by the words of one of my favorite philosophers, David Hume, who, upon learning that he was mortally ill at age 65, wrote a short autobiography in a single day in April of 1776. He titled it “My Own Life.”

“I now reckon upon a speedy dissolution,” he wrote. “I have suffered very little pain from my disorder; and what is more strange, have, notwithstanding the great decline of my person, never suffered a moment’s abatement of my spirits. I possess the same ardour as ever in study, and the same gaiety in company.”

I have been lucky enough to live past 80, and the 15 years allotted to me beyond Hume’s three score and five have been equally rich in work and love. In that time, I have published five books and completed an autobiography (rather longer than Hume’s few pages) to be published this spring; I have several other books nearly finished.

Hume continued, “I am ... a man of mild dispositions, of command of temper, of an open, social, and cheerful humour, capable of attachment, but little susceptible of enmity, and of great moderation in all my passions.”

Here I depart from Hume. While I have enjoyed loving relationships and friendships and have no real enmities, I cannot say (nor would anyone who knows me say) that I am a man of mild dispositions. On the contrary, I am a man of vehement disposition, with violent enthusiasms, and extreme immoderation in all my passions.

And yet, one line from Hume’s essay strikes me as especially true: “It is difficult,” he wrote, “to be more detached from life than I am at present.”

Over the last few days, I have been able to see my life as from a great altitude, as a sort of landscape, and with a deepening sense of the connection of all its parts. This does not mean I am finished with life.

On the contrary, I feel intensely alive, and I want and hope in the time that remains to deepen my friendships, to say farewell to those I love, to write more, to travel if I have the strength, to achieve new levels of understanding and insight.\footnote{Oliver Sacks, a professor of neurology at the New York University School of Medicine, is the author of many books, including “Awakenings” and “The Man Who Mistook His Wife for a Hat.”}
\end{quote}

\begin{quote}
I have been increasingly conscious, for the last 10 years or so, of deaths among my contemporaries. My generation is on the way out, and each death I have felt as an abruption, a tearing away of part of myself. There will be no one like us when we are gone, but then there is no one like anyone else, ever. When people die, they cannot be replaced. They leave holes that cannot be filled, for it is the fate — the genetic and neural fate — of every human being to be a unique individual, to find his own path, to live his own life, to die his own death.

I cannot pretend I am without fear. But my predominant feeling is one of gratitude. I have loved and been loved; I have been given much and I have given something in return; I have read and traveled and thought and written. I have had an intercourse with the world, the special intercourse of writers and readers.

Above all, I have been a sentient being, a thinking animal, on this beautiful planet, and that in itself has been an enormous privilege and adventure.
\end{quote}

To load the template just type:

\begin{verbatim}
\cxset{%
 custom=versochapter,
 chapter image=vespa.jpg,
 chapter opening=verso}
\end{verbatim}





\makeatletter
\@specialfalse
\makeatother

%\makeatletter\@specialfalse\makeatother
%%%%%%%%%%%%%%%%%%%%%%%%%%%%%%%%%%%%%%%%%%%
%%%%%%  STYLE 01
%%%%%%%%%%%%%%%%%%%%%%%%%%%%%%%%%%%%%%%%%%%


\cxset{
 name={},
 numbering=arabic,
 number font-size=\LARGE,
 number font-family=\rmfamily,
 number font-weight=\bfseries,
 number before=,
 number dot=,
 number after=,
 number position=leftname,
 chapter font-family=\sffamily,
 chapter font-weight=\normalfont,
 chapter font-size=\Large,
 chapter before={\vspace*{20pt}\par},
 chapter after={\hfill\hfill\par},
 chapter color={black!90},
 number color=\color{purple},
 title beforeskip={\vspace*{30pt}},
 title afterskip={\vspace*{40pt}\par},
 title before={},
 title after={},
 title font-family=\sffamily,
 title font-color=\color{purple},
 title font-weight=\bfseries,
 title font-size=\LARGE,
 header style=headings}

\cxset{headings ruled-01}

\chapter{Introduction to Style One}


\begin{summary}
This design is simple and its distinguishing characteristic is a short summary at the beginning of the chapter. This is almost like an abstract typeset in italic font without setting the margins in. We provide a \lstinline{summary} environment for convenience. Note the very simple line in the running head to the left of the page number.
\end{summary}

\medskip
\begin{figure}[ht]
\centering
\includegraphics[width=0.5\textwidth]{./chapters/chapter01}
\end{figure}


%\clearpage
\makeatletter\@debugtrue\makeatother
\cxset{
 chapter toc=true,
 name=CHAPTER,
 chapter numbering=ORDINALS,
 number font-size=Large,
 number font-family=rmfamily,
 number font-weight=bfseries,
 number before=\kern0.5em,
 number dot=,
 number after=\hfill\hfill\par,
 number position=rightname,
 chapter font-family=rmfamily,
 chapter font-weight=bold,
 chapter font-size=Large,
 chapter before={\vspace*{20pt}\par\hfill},
 chapter after=,
 chapter color=black,
 number color=black,
 %
 title margin top=10pt,
 title before=\par\nointerlineskip\hfill,
 title after=\hfill\hfill\par\nointerlineskip,
 title font-family=rmfamily,
 title font-color= black,
 title font-weight=bfseries,
 title font-size=LARGE,
 chapter title width=0.8\textwidth,
 chapter title align=centering,
 title margin-left=0pt,
 author block=false}

\debugtitle
\debugchapter
\chapter[Template 2]{Mondino, the Restorer of Anatomy}

The archive.org is an extraordinary hunting ground  for typographical surprises. On a recent excursion to find some books on Versalius I stubled on a book titled \emph{Andreas Vesalius, the reformer of anatomy} by  Ball, James Moores. It is an old book published in 1910 and has a couple of unusual features. Check the figure below and see if you can identify the challenging feature.

\begin{figure}[ht]
\centering
\includegraphics[width=0.8\textwidth]{versalius}
\caption{J.B. Moore’s \emph{Andreas Versalius, the Reformer of Anatomy} has many unusual features, including chapter numbers using ordinals. }
\end{figure}

\cxset{chapter toc=true,
          chapter opening=anywhere}
          
\chapter{The Template}          
The template is called \emph{Versalius} and is stored under style02. It can be loaded in the normal way using:
\begin{verbatim}
\usepackage[style02]{phd}
\end{verbatim}

I have not reproduced the full extend of the book’s requirements, as some details are quite cumbersome to be automated through \tex. These though can easily be incorporated in a manual way. More about this later.


\section{The Table of Contents}
Another interesting aspect of this book, which is common with many books of its period is the ToC. The ToC shows the full range of the chapter pages, i.e., it is marked as Page 1-16 rather than the common practice nowdays that indicates only the starting page of the chapter. It also has “TABLE OF CONTENTS”  as a heading and not just contents as you would expect from today’s books.

\begin{figure}[ht]
\centering
\includegraphics[width=0.8\textwidth]{versalius-01}
\caption{J.B. Moore’s \emph{Andreas Versalius, the Reformer of Anatomy} has many unusual features, including chapter numbers using ordinals. }
\end{figure}

\section{List of Illustrations}

\begin{figure}[ht]
\centering
\includegraphics[width=0.8\textwidth]{versalius-02}
\caption{J.B. Moore’s \emph{Andreas Versalius, the Reformer of Anatomy} has many unusual features, including chapter numbers using ordinals. }
\end{figure}

\section{The Frontmatter}
As a foreward there is an unumbered chapter called ``Introduction’’. The chapter heading also has a head band.
\begin{figure}[ht]
\centering
\includegraphics[width=0.8\textwidth]{versalius-03}
\caption{J.B. Moore’s \emph{Andreas Versalius, the Reformer of Anatomy} has many unusual features, including chapter numbers using ordinals. }
\label{lettrine}
\end{figure}

\bgroup
\centering
\includegraphics[width=0.7\textwidth]{versalius-headband}

\LARGE\bfseries INTRODUCTION\par
\egroup
\def\dropcapversalius{%
\vbox to 0pt{\vskip6pt\leavevmode\noindent\includegraphics[width=2.39cm]{versalius-dropcap}%
}%
}
\parindent0pt

\hangindent2.6cm \hangafter0
\dropcapversalius \textsc{he dropcap will have to be inserted}, either using the lettrine package or do be achieved via a parshape command and manual entry. You can also write your own macro command using the details we provide under the Paragraphs chapter. On this page I have manually inserted it, as I used an image from the book for the dropcap. If you were to use the template for a full book, it will be then preferable to use

the lettrine package to set the dropcaps. If you observe Figure~\ref{lettrine} carefully, you will notice the first line of theopening paragraph is in small caps. As \tex typesets the full paragraph this is almost an impossible task to achieve through normal \tex commands and in order not to overcomplicate the discussion it can be achieved manually via trial and error. 

\section{Figures}

Most of the figures are wrapped illustrations. A couple are full page figures and bear no caption numbering. One such illustration is shown on page~\pageref{fig:vesalius}. Do note that the List of Illustrations does have the illustrations listed with additional information to that shown in the captions. 

\begin{figure}[p]
\centering
\includegraphics[width=\textwidth]{vesalius}
\centering
ANDREAS VESALIUS\par
(From an old copperplate engraving)\par
\label{fig:vesalius}
\end{figure}







%\newgeometry{top=-10pt,bottom=2cm}

\tcbset{width=\paperwidth,boxrule=0pt,right=3cm,arc=0pt}

\cxset{style03/.style={
 name={},
 numbering=arabic,
 number font-size= HUGE,
 number font-family= rmfamily,
 number font-weight= bfseries,
 number before=\par\vspace*{10pt}\hfill\hfill,
 number dot=.,
 number after=,
 number position=rightname,
 chapter font-family= sffamily,
 chapter font-weight=normalfont,
 chapter font-size= Large,
 chapter before={\hspace*{-2.5cm}\vbox\bgroup\tcolorbox\bgroup\vspace*{20pt}\hfill\hfill},
 chapter after={\par\vspace*{15pt}},
 chapter color=black!90,
 number color= thered,
 title beforeskip={},
 title afterskip={\vspace*{10pt}\par},
 title before={\hfill\hfill},
 title after={\vspace*{60pt}\egroup\endtcolorbox\egroup},
 title font-family=\sffamily,
 title font-color= thered,
 title font-weight=\bfseries,
 title font-size=\Huge}}

\cxset{style03}

\chapter{Introduction Style Three}

This is not an exact reproduction as I am still thinking as to how to use
specials with the package. You can vary it by setting the tcolorbox settings as well as the geometry settings.
\medskip

\begin{figure}[ht]
\centering
\includegraphics[width=0.39\textwidth]{./chapters/chapter03}
\end{figure}

This setting involves changing the geometry of the page as well as adding the chapter name and title in a color box. For this I have used the \lstinline{tcolorbox}. Of course you can use any other shaded environment you feel comfortable with such as mdframed. It is important to set the colorbox parameters.

\begin{lstlisting}
\newgeometry{top=-10pt}
\tcbset{width=\paperwidth,boxrule=0pt,right=3cm,arc=0pt}
\end{lstlisting}

Note that we set the width of the \lstinline{tcolorbox} to \lstinline{\paperwidth} in order for the shading to extend to the full width of the page.

\restoregeometry

%\clearpage
\cxset{style04/.style={
 numbering=Roman,
 number font-size=\Large,
 number font-family=\rmfamily,
 number font-weight=\bfseries,
 number before=,
 number dot=,
 number after=,
 number position=rightname,
 chapter font-family=\rmfamily,
 chapter font-weight=\normalfont,
 chapter font-size=\Large,
 chapter before={\vspace*{20pt}\par\hfill},
 chapter after={\hfill\hfill\par\vspace*{10pt}},
 chapter color={black!90},
 number color=purple,
 title beforeskip={},
 title afterskip={\vspace*{50pt}\par},
 title before={\hfill},
 title after={\hfill\hfill\par},
 title font-family=\rmfamily,
 title font-color= purple,
 title font-weight=\normalfont,
 title font-size=\LARGE,
 section numbering=none,
 section align = center}}

\cxset{style04}

\chapter{INTRODUCTION TO STYLE FOUR}

This is a very simple design applicable perhaps to translations and commentary on older texts.
\medskip
\begin{figure}[ht]
\centering
\includegraphics[width=0.6\textwidth]{./chapters/chapter04.png}
\end{figure}

%
\cxset{style05/.style={
 name={Chapter},
 chapter color = magenta,
 chapter toc = true,
 numbering=arabic,
 number font-size=\Large,
 number font-family=\rmfamily,
 number font-weight=\normalfont\itshape,
 number color= purple,
 number before=\hspace*{-15pt},
 number dot=,
 number after=,
 number position=rightname,
 chapter font-family=sffamily,
 chapter font-weight= \bfseries\itshape,
 chapter font-size=\Large,
 chapter before={\hrule width \columnwidth \kern12.6pt \par\hfill},
 chapter after={\hfill\hfill\par},
 chapter color={magenta},
 chapter spaceout=none,
 title beforeskip={\vspace*{10pt}},
 title afterskip={\vspace*{30pt}\par},
 title before={\hfill},
 title after={\hfill\hfill \vskip12.6pt\hrule width \columnwidth \kern2.6pt },
 title font-family=\rmfamily,
 title font-color=black!90,
 title font-weight=\bfseries,
 title font-size=\huge,
 title font-shape = normal,
 header style= headings}}

\cxset{style05}
\chapter{Introduction to Style Five}\index{ch:style5}

\tcbset{width=\textwidth}
I think this style can be improved with a bit of color. You can experiment with it quite easily. The spacing on top of this style can also be adjusted to suit your typographical taste.
\medskip
\begin{figure}[ht]
\centering
\includegraphics[width=0.6\textwidth]{./chapters/chapter05}
\end{figure}

%\section{General notes on rules}

LaTeX's default rules would normally give problems. Best is to use TeX's primitives to built them.

\index{rules!example color}

\begin{texexample}{}{}
\makeatletter
\hrule width 5cm \kern2.6\p@
AAAAAAAAAAAAAAAAAAAAA
\vskip2.6pt\hrule width 5cm
\medskip

Problem with LaTeX rules.

\rule{5cm}{0.4pt}\par
AAAAAAAAAAAAAAAAAAAAA\par%
\rule[6.5pt]{5cm}{0.4pt}

\def\rule{\@ifnextchar[\@rule{\@rule[\z@]}}
\def\@rule[#1]#2#3{%
 \leavevmode
 \hbox{%
 \setlength\@tempdima{#1}%
 \setlength\@tempdimb{#2}%
 \setlength\@tempdimc{#3}%
 \advance\@tempdimc\@tempdima%
 \vrule\@width\@tempdimb\@height\@tempdimc\@depth-\@tempdima}}

\def\thickrule{\leavevmode \leaders \hrule height 3pt \hfill \kern \z@}

{\color{teal}\hrule width 10.5cm height3pt \kern2.6\p@
    {{\color{black!80}\HUGE CHAPTER TITLE}}\vskip3pt
\hrule width 10.5cm height3pt}
\makeatother
\end{texexample}

%% Requires the package calligra
\newfontfamily{\ovidius}{ovidius demi}
\cxset{style06/.style={%
 chapter opening=anywhere,
 chapter name=Chapter,
 chapter numbering=arabic,
 chapter number font-size=LARGE,
 chapter label font-family =ovidius,
 chapter number font-weight = bfseries,
 chapter number color= black!60,
% chapter number before=\kern3.5pt,
% chapter number dot=,
 %number after=\hfill\hfill\par\offinterlineskip,
% number position=rightname,
 chapter label font-family= ovidius,
 chapter label font-shape=itshape,
 chapter label font-weight=normal,
 chapter label font-size= LARGE,
% chapter before=\vspace*{2pt}\par\hfill,
% chapter after=,
 chapter label color=black!60,
% chapter spaceout=none,
 chapter title margin top=30pt,
 %title before=\hfill\par,
% title after=\hfill,
 chapter title font-family=ovidius,
 chapter title font-color= black!90,
 chapter title font-weight=normalfont,
 chapter title font-size=LARGE,
 %title spaceout=none,
 chapter title width=0.6\textwidth,
 chapter title align=centering,
 }}

\cxset{style06}
\cxset{chapter label font-face= ovidius}
\cxset{chapter format=traditional}

\chapter{THE RITUALS OF THE MONTHS OF THE YEAR}
\renewcommand{\DefaultLhang}{0.1}
\renewcommand{\LettrineFontHook}{\calligra}
\setlength{\DefaultFindent}{9.5pt}
\setlength{\DefaultNindent}{0pt}
\renewcommand{\LettrineFontHook}{\ovidius}
\lettrine[loversize=0.6]{\textcolor{thegray!60}{C}}{}rist\'obal de Molina’s manuscript titled \emph{Account of the Fables and Rites of the Incas (Relación
de las fábulas y ritos de los incas)}, written aroun 1575 records the rituals that were conducted in Cuzco during the last years of the Inca Empire. An excellent translation was published by
\begin{figure}[ht]
\centering
\includegraphics[width=0.45\textwidth]{./chapters/chapter06.png}
\caption{Style 5 sample}
\end{figure}
the University of Texas Press. The translation is by Brian S. Bauer, Vania Smith-Oka and Gabriel E. Cantarutti who did an excellent job. The typesetting attracted my attention by its effective simplicity and I read the book in one evening. The book size is 5.50 x 8.50 in, pages 187. Times Roman, ArnoPro and Ovidius. The Ovidius font was designed by  Thaddeus Szumilas and 
belongs to a family known as eroded fonts. It has found many devoted users especially for book covers.

This template has a lot of potential and I will come back to it and add more key hooks for lettrine settings per letter and font management. They can also come alive with a gold color.

The dropcap in the original book as well as the chapter font is given a worn style (Ovidius Demi font\footnote{Available at the fontpalace website \protect\url{http://www.fontpalace.com/font-download/Ovidius Demi/}}), I guess in order to give it a touch of style reminiscent of a manuscript. It can also look
good using |calligra| and also a bit of color. You can experiment also with many other calligraphic fonts. The example below demonstrates the use of the |calligra| font.
\bigskip

\renewcommand{\LettrineFontHook}{\calligra}
\cxset{chapter label font-family=calligra}
\cxset{chapter number font-family=calligra}
\cxset{chapter number font-weight=calligra}
\cxset{chapter number font-size=Huge}
%\cxset{chapter color=black!90,
\cxset{chapter number color=black!90}
\bigskip

\cxset{chapter opening=any}

\chapter{OF QUIPUS AND INCA YUPANQUI}

\lettrine[loversize=.6]{\textcolor{orange}{T}}{}he book by Crist\'obal de Molina’s manuscript titled \emph{Account of the Fables and Rites of the Incas (Relación
de las fábulas y ritos de los incas)}, written aroun 1575 records the rituals that were conducted in Cuzco during the last years of the Inca Empire. An excellent translation was published by
\medskip

The dropcap looks as good if not better with the |calligra| font and I have given it a colour to stand out. The chapter number has to be increased in height, so I have used |huge|. The new
settings are shown below:

\begin{verbatim}
\renewcommand{\LettrineFontHook}{\calligra}

\cxset{chapter opening=any}
\cxset{chapter label font-family=calligra}
\cxset{chapter number font-family=calligra}
\cxset{chapter number font-weight=calligra}
\cxset{chapter number font-size=Huge}
\cxset{chapter number before=\kern2.5pt}
\cxset{chapter label color=black!90,
      chapter number color=black!90}
\end{verbatim}

\cxset{section align=center,
          section numbering=none,
          section font-weight=normalfont,
          section font-family=rmfamily,
          section font-size=large,
          section color=black,
          section font-shape=scshape}


\section{THE SECTIONS}

The sections are typeset in normal font and are centered.

{\ovidius \lorem}

\bottomline 
%<<<<<<< HEAD

\newgeometry{top=2cm,bottom=2cm,left=3cm,right=3cm}
%%%%%%%%%%%%%%%%%%%%%%%%%%%%%%%%%%%%%%%%%%%
%%%%%%  STYLE 07
%%%%%%%%%%%%%%%%%%%%%%%%%%%%%%%%%%%%%%%%%%%

\cxset{style07/.style={
 name={},
 numbering=arabic,
 number font-size=\Huge,
 number font-family=\rmfamily,
 number font-weight=\bfseries,
 number before=,
 number dot=,
 number color=\color{gray},
 number after=\par,
 number position=rightname,
 chapter font-family=\sffamily,
 chapter font-weight=\normalfont,
 chapter font-size=\Large,
 chapter before={\hfill\hfill\par},
 chapter after={},
 chapter color={black!90},
 title beforeskip={\vspace*{30pt}},
 title afterskip={\vspace*{50pt}\par},
 title before={},
 title after={\par\rule[13pt]{\textwidth}{0.4pt}},
 title font-family=\sffamily,
 title font-color=\color{purple},
 title font-weight=\bfseries,
 title font-size=\LARGE,
 title spaceout=none,
}}

\cxset{style07}
\chapter{Introduction to Style Seven}

\parindent0pt
\lipsum[1]
\medskip
\begin{figure}[ht]
\centering
\includegraphics[width=0.6\textwidth]{./chapters/chapter07}
\end{figure}
\lipsum[1]
=======

\newgeometry{top=2cm,bottom=2cm,left=3cm,right=3cm}
%%%%%%%%%%%%%%%%%%%%%%%%%%%%%%%%%%%%%%%%%%%
%%%%%%  STYLE 07
%%%%%%%%%%%%%%%%%%%%%%%%%%%%%%%%%%%%%%%%%%%

\cxset{style07/.style={
 name={},
 numbering=arabic,
 number font-size=\Huge,
 number font-family=\rmfamily,
 number font-weight=\bfseries,
 number before=,
 number dot=,
 number color=\color{gray},
 number after=\par,
 number position=rightname,
 chapter font-family=\sffamily,
 chapter font-weight=\normalfont,
 chapter font-size=\Large,
 chapter before={\hfill\hfill\par},
 chapter after={},
 chapter color={black!90},
 title beforeskip={\vspace*{30pt}},
 title afterskip={\vspace*{50pt}\par},
 title before={},
 title after={\par\rule[13pt]{\textwidth}{0.4pt}},
 title font-family=\sffamily,
 title font-color=\color{purple},
 title font-weight=\bfseries,
 title font-size=\LARGE,
 title spaceout=none,
}}

\cxset{style07}
\chapter{Introduction to Style Seven}

\parindent0pt
\lipsum[1]
\medskip
\begin{figure}[ht]
\centering
\includegraphics[width=0.6\textwidth]{./chapters/chapter07}
\end{figure}
\lipsum[1]
>>>>>>> merged

%\clearpage

\setdefaults

\cxset{style08/.style={
 name={},
 chapter toc=true,
 numbering=arabic,
 number font-size=\LARGE,
 number font-family=\sffamily,
 number font-weight=\bfseries,
 number color= black!90,
 number before=,
 number dot=,
 number after=,
 number position=rightname,
 chapter font-family=\sffamily,
 chapter font-weight=\normalfont,
 chapter font-size=\Large,
 chapter before={\vspace*{20pt}\hfill},
 chapter after={\vspace{20pt}\par},
 chapter color={black!90},
 title beforeskip={},
 title afterskip={\vspace*{50pt}\par},
 title before={\hfill\hfill\raggedleft},
 title after={},
 title font-family=\sffamily,
 title font-color=black!90,
 title font-weight=\bfseries,
 title font-size=\LARGE,
 author block=true,
 author block format=\par\addvspace{12pt}\normalfont\large\raggedleft,
 author names=Yiannis Lazarides\par Larnaka,
 header style=empty}}

\cxset{style08}
\chapter{Introduction to Chapter Style Eight}

\lipsum[1]
\medskip
\begin{figure}[ht]
\centering
\includegraphics[width=0.5\textwidth]{./chapters/chapter08.png}
\end{figure}
\lipsum[1]

%\cxset{author block=false}
\clearpage

\cxset{
 name={},
 numbering=arabic,
 number font-size=\LARGE,
 number font-family=\rmfamily,
 number font-weight=\bfseries,
 number before=,
 number dot=.,
 number after=\hspace{1em},
 number position=rightname,
 chapter font-family=\sffamily,
 chapter font-weight=\normalfont,
 chapter font-size=\Large,
 chapter before={\vspace*{20pt}\par\hfill},
 chapter after={},
 chapter color={black!90},
 number color= purple,
 title beforeskip={},
 title afterskip={\vspace*{50pt}\par},
 title before={},
 title after={},
 title font-family=\sffamily,
 title font-color= purple,
 title font-weight=\bfseries,
 title font-size=\LARGE}
\chapter{Introduction 09}
\lipsum[1]

\medskip
\begin{figure}[ht]
\centering
\includegraphics[width=0.8\textwidth]{./chapters/chapter09}
\end{figure}

\textit{In preparation. Patience!}

%%\makeatletter
\cxset{
 chapter opening=right,
 chapter toc=false,
 name=CHAPTER,
 numbering= WORDS, %WORDS gives errors
 number font-size=huge,
 number font-family=sffamily,
 number font-weight=bfseries,
 number before=\kern1em,
 number dot=,
 number after=,
 number position=rightname,
 % set chapter fonts 
 chapter font-family=sffamily,
 chapter font-weight=bfseries,
 chapter font-size=huge,
 chapter margin top=5cm,
 chapter margin left=0pt,
 chapter before=\par\hfill,
 chapter after=,
 chapter color=black,
 chapter spaceout=none,
 chapter title align=center,
 chapter afterindent=true,
 number color=black,
% chapter titles
 title margin top=30pt,
 title margin bottom=30pt,
 chapter title width=\textwidth,
 chapter title text-align=center,
 title font-family=sffamily,
 title font-color=black,
 title font-weight=bfseries,
 title font-size=huge,
 title font-shape=upshape,
 title before=,
 title after=,
% sections 
 section font-size=LARGE,
 section font-weight=normalfont,
 section font-family=sffamily,
 section color=black,
 section align=centering,
 section numbering=none,
 section indent=-1em,
 section beforeskip=20pt,
 section afterskip=10pt,
 section spaceout=soul,
 section font-shape=,
 pagestyle = plain,
 subsection color=black,
}

\chapter{Introduction to Style 10}

\addcontentsline{toc}{section}{Template 10 (style10)}

This style is very similar to the |verso chapter| style. I have reproduced it as close as possible to the book that gave me the inspiration titled \emph{Mind Machines}.

\begin{figure}[htb]
\centering
\fboxrule1pt
\fbox{\includegraphics[width=0.8\textwidth]{./chapters/chapter10}}
\caption{Style ten example.}
\end{figure}

Another interesting aspect is that subsections are centered and have a colon at the end of the subsection title. The setting for this is the option \lstinline{numeric=WORDS}. Use either a capital for uppercase or \lstinline{numeric=words} for lowercase number labels.

\cxset{chapter toc=true,
          chapter margin top=0pt}
\makeatother

%%\cxset{style11/.style={
 chapter opening=any,
 name=Chapter,
 numbering=arabic,
 number font-size=LARGE,
 number font-family=rmfamily,
 number font-weight=bfseries,
 number before=,
 number dot=,
 number after=,
 number before=\kern0.5em,
 number display=inline,
 number float=center,
 chapter display=block,
 chapter float=center,
 chapter font-family=rmfamily,
 chapter font-weight=bfseries,
 chapter font-size=LARGE,
 chapter before=,
 chapter after=,
 chapter color=black!90,
 chapter spaceout=none,
 chapter border-width=0pt,
 chapter border-style=none,
 number color=black!90,
 title beforeskip=,
 title afterskip=,
 title before=,
 title after=,
 title font-family=rmfamily,
 title font-color=black!90,
 title font-weight=bfseries,
 title font-size=LARGE,
 chapter title width=\textwidth,
 chapter title align=centering,
 section afterindent=true,
 section align=left,
 section numbering=arabic,
 section numbering prefix=\thechapter.,
 section numbering suffix=\space,
 section indent=0pt,
 section font-family=rmfamily,
 }}
\renewsection\renewsubsection

\cxset{style11}
\chapter{\textit{Elements} II and Babylonian Metric Algebra, Introduction to Style Eleven}

The origins of Greek Mathematics, according to the Greeks is Egypt and according to J\"oran Friberg is Babylonia. This template is based on Friberg's book \emph{Amazing Traces of a Babylonian Origin in Greek Mathematics}. The book was published by World Scientific in 2007. The book size is $5.97\times8.88$ inches and uses a variety of fonts, with the main document font in Times. 

\medskip
\begin{figure}[ht]
\centering
\fbox{\includegraphics[width=0.65\textwidth]{./chapters/chapter11.png}}
\end{figure}
\lipsum[1]

\section{Indentation}

The book follows swedish traditional typography with the paragraphs following subheadings indented. This is achieved in the template using:

\begin{verbatim}
\cxset{section afterindent=true}
\end{verbatim}

\section{Images}
\indent Images and their captions follow a \latexe style and I am sure the book must have been styled using a \latexe xml clone as the book's pdf was produced with iText\footnote{\url{http://itextpdf.com/}}.

\begin{figure}[ht]
\centering
\includegraphics[width=0.8\textwidth]{greekmaths}
\caption{Extract from the \textit{Amazing Traces of Babylonian Influence in Greek Mathematics.} Note the styling of the caption.}
\end{figure}

\testsections

% reset for following chapters
\cxset{section afterindent=false}


%%\cxset{style12/.style={
 chapter name=,
 chapter toc=true,
 chapter numbering=arabic,
 number font-size=\HUGE,
 number font-family=\rmfamily,
 number font-weight=\bfseries,
 number before=,
 number dot=,
 number color= gray,
 number after=\par,
 number position=rightname,
 chapter font-family=\sffamily,
 chapter font-weight=\normalfont,
 chapter font-size=\Huge,
 chapter before={\hfill\hfill\hfill\par},
 chapter after={},
 chapter color={black!90},
 title beforeskip={\vspace*{0pt}},
 title afterskip={\vspace*{50pt}\par},
 title before={},
 title after={\par\vspace{10pt}\rule{\textwidth}{4pt}},
 title font-family=\sffamily,
 title font-color=black!90,
 title font-weight=\bfseries,
 title font-size=\HUGE,
 title font-shape=normal,
 title spaceout=none,
}}

\cxset{style12}
\chapter{Introduction to Style Twelve}

This is a variation of Style 7, with only the lettering and the rule are thicker. In my opinion it looks better with a bit of color, so I have used a purple color with a gray.

\medskip
\begin{figure}[ht]
\centering
\includegraphics[width=0.35\textwidth]{./chapters/chapter12.png}
\end{figure}
\lipsum[1]
\chapter{Second Chapter}

%%%%%%%%%%%%%%%%%%%%%%%%%%%%%%%%%%%%%%%%%%%%%
%%%%%%  STYLE 13
%%%%%%%%%%%%%%%%%%%%%%%%%%%%%%%%%%%%%%%%%%%

\cxset{style13/.style={
 name={Chapter},
 numbering=arabic,
 number font-size=\HUGE,
 number font-family=\sffamily,
 number font-weight=\bfseries,
 number color=\color{gray!50},
 number before=\par\vspace*{5pt}\hfill\hfill,
 number dot=,
 number after={\hspace*{7pt}\par},
 number position=rightname,
 chapter font-family=\sffamily,
 chapter font-weight=\normalfont,
 chapter font-size=\LARGE,
 chapter before={\thickrule\vspace*{20pt}\par\hfill\hfill},
 chapter after={\vskip0pt\par},
 chapter color={black!50},
 title beforeskip={\vspace*{10pt}},
 title afterskip={\vspace*{50pt}\par},
 title before={\hfill\hfill\raggedleft},
 title after={},
 title font-family=\sffamily,
 title font-color=\color{thered},
 title font-weight=\bfseries,
 title font-size=\huge,
 section indent=-1em,
 section align=\raggedright,
 section numbering=arabic,
 section indent=0pt,
 section beforeskip=0pt,
 section afterskip=\baselineskip,
 subsection align=\raggedright,
 subsection font-family=\sffamily,
 subsection font-weight=\bfseries,
 subsection font-size=\large,
 subsection font-shape=\itshape,
 subparagraph number after=\space,
}
}

\def\setstyle#1{\cxset{style#1}%
 \renewsection\renewsubsection\renewsubsubsection%
 \renewparagraph\renewsubparagraph}

\setstyle{13}


\chapter{Introduction to Chapter\\ Style Thirteen}

\section{A Brief History of Biomedical\\ Fluid Mechanics}
\lorem
\medskip
\begin{figure}[ht]
\centering
\includegraphics[width=0.45\textwidth]{./chapters/chapter14}
\includegraphics[width=0.45\textwidth]{./chapters/chapter14a}
\end{figure}
\lorem
 % CHECK FOR ERRORS
%%\cxset{%
 chapter opening=any,
 name=Chapter,
 numbering=arabic,
 number font-size=HHUGE,
 number font-family=rmfamily,
 number font-weight=bfseries,
 number before=,
 number dot=,
 number after=,
 number position= rightname,
 number display=block,
 number float=right,
 chapter display=block,
 chapter float=right,
 chapter font-family=\sffamily,
 chapter font-size=\Large,
 chapter before={\offinterlineskip\hbox{\vrule height2pt width\textwidth}\vskip3.5pt}\hfill,
 chapter after=\vskip3.5pt,
 chapter color=black!90,
 number color=black!90,
 chapter title width={0.5\textwidth},
 title margin-left=0pt,
 chapter title align=right,
 chapter title text-align=raggedleft,
 chapter margin top=0pt,
 chapter margin-left=0pt,
 title margin top=0pt,
 title margin bottom=10pt,
 title before={},
 title after={},
 title font-family=sffamily,
 title font-color=black!90,
 title font-weight=bfseries,
 title font-size=LARGE,
 section numbering prefix=\thechapter.,
 section color=black,
 subsection color=black}
 
\chapter{Review of Basic Fluid Mechanics Concepts}
\thispagestyle{headings}

This book is $5.51\times9.06$ inches and was produced according with the soft copy I have with Acrobat Distiller 5.0 (Windows). It uses a variety of fonts Arial, Century Schoolbook and Helvetica, Times Roman, MathematicalPi-One.
\medskip
\begin{figure}[ht]
\centering
\fbox{%
\includegraphics[width=0.45\textwidth]{./chapters/chapter14}
\includegraphics[width=0.45\textwidth]{./chapters/chapter14a.png}}
\end{figure}
\lipsum[2]

\section{Images}

\begin{figure}[ht]
\centering
\includegraphics[width=0.8\textwidth]{biofluids}
\end{figure}
\lipsum[2]

\lipsum[1]
\section{Examples}
Both full page ad well as block examples exist, these are all in boxes and they are numbered either with subsection counters in a fashion that is continuous from the text. 
\begin{figure}[ht]
\centering
\includegraphics[width=0.8\textwidth]{biofluids-1}
\end{figure}
\lipsum[2-4]

\begin{figure}[!b]
\begin{scriptexample}{This is a test}{}
\subsection{Clinical feature: polycythemia}

Polycythemia refers to a condition in which there is an increase in hemoglobin
above 17.5 g/dL in adult males or above 15.5 g/dL in females
(Hoffbrand and Pettit, 1984). There is usually an icrease in the number of
red blood cells above 6 - 1012 $L{^1}$ in males and 5.5 - 1012 $L^{-1}$ in females.
That is, a sufferer from this condition has a much higher blood viscosity due
to this elevated red blood cell count.
\end{scriptexample}
\end{figure}

Our boxes with the \pkgname{tcolorbox} are more than adequate for the job. The color settings can be changed via normal tcolorbox key settings that I have linked to the phd package keys.

The boxes can be also turned into floating boxes and to force them either to be on a full page or at the bottom in order for them to make a better impact in the layout. Key settings for setting the color of the boxes are provided
as well as a special environment.

\begin{verbatim}
\cxset{example box color=thegray}
\end{verbatim}

These example or sideboxes can easily be modified and you may have to provide your own, if you want anything particularly fancy. See the tcolorbox manual for many setting, but please do not use rounded boxes. 
%\end{document}
%%%%%%%%%%%%%%%%%%%%%%%%%%%%%%%%%%%%%%%%%%%%%
%%%%%%  STYLE 15
%%%%%%%%%%%%%%%%%%%%%%%%%%%%%%%%%%%%%%%%%%%
\newgeometry{left=2cm,right=7cm, marginparsep=15pt, marginparwidth=4.2cm,top=2cm}
\cxset{
 name={},
 numbering=none,
 number font-size=\LARGE,
 number font-family=\rmfamily,
 number font-weight=\bfseries,
 number before=,
 number dot=,
 number after=,
 number position=rightname,
 chapter font-family=\sffamily,
 chapter font-weight=\normalfont,
 chapter font-size=\Large,
 chapter before={},
 chapter after={},
 chapter color={black!90},
 number color=\color{purple},
 title beforeskip={},
 title afterskip={\vspace*{50pt}\par},
 title before={},
 title after={},
 title font-family=\rmfamily,
 title font-color=\color{black!80},
 title font-weight=\normalfont,
 title font-size=\Huge,
 title font-shape=\itshape,
 chapter opening=any,
 watermark text=SAMPLE PAGE,
 header style=samplepage}


\chapter{Introduction to Style Fifteen}

\parindent1em
\def\thefigure{\arabic{chapter}.\arabic{figure}}
\lorem\par

\marginpar{%
 {\centering
 \includegraphics[width=4.2cm]{./chapters/chapter15}\vskip5pt\par}
 {\footnotesize\lorem}
}
\marginpar{%
{\centering
\includegraphics[width=4.2cm]{./chapters/chapter15}\par}
 { \captionof{figure}{\footnotesize\lorem}}
}
This is another marginpar of the same size.

\lorem

\lipsum
\marginpar{%
{\centering
\includegraphics[width=4.2cm]{./chapters/chapter15}\par}
 { \captionof{figure}{\footnotesize\lorem}}
}
%%%% END STYLE %%%%%%%%%%%%%%%%%%%%%%%%%%%%%%%%%

%%
\newgeometry{left=7.5cm,right=2cm, marginparsep=15pt, marginparwidth=4.2cm,top=2cm}


\cxset{style16/.style={
 chapter opening=left,
 name={},
 numbering=arabic,
 number color= thegray,
 number font-size=\HHUGE,
 number font-family=\rmfamily,
 number font-weight=\bfseries,
 number before=\leftskip-4cm\vbox to 0cm\bgroup\vspace*{5cm},
 number after=\egroup\vskip0pt\par,
 number dot=,
 number position=leftname,
 chapter font-family=\sffamily,
 chapter font-weight=\normalfont,
 chapter font-size=\Large,
 chapter before={},
 chapter after={\begin{picture}(0,0)
                          \put(-50pt,\dimexpr-\textheight+\footskip+20pt\relax){\parbox{\marginparwidth}{\textbf{Napoleon}\par\lorem}}%
                        \end{picture}},
 chapter color={black!90},
 title beforeskip={\par\hspace*{4cm}\thinrule\vskip0pt\hspace*{4cm}\vbox\bgroup},
 title afterskip={\vspace*{50pt}\par\egroup},
 title before={},
 title after=,
 title font-color= black!80,
 title font-weight=\normalfont,
 title font-family=\rmfamily,
 title font-shape=\upshape,
 title font-size=\HUGE,
 header style=empty,
 subsection numbering=none,
 subsection color=teal,
 subsection align=teal,
 header style=empty,
 }}

\renewsubsection


\def\fullpageimage{%
       \leavevmode\mbox{}
       \vbox{%
        \parindent0pt
       \vfill
       \hspace*{-1cm}\fbox{\includegraphics[width=25cm]{napoleon}}%
       }%
}


\cleardoublepage

\fullpageimage

\cxset{style16}
\chapter{Victorian England:\\ Introduction 16}
\label{style16}
This design from a Social Sciences book had to be set into two vboxes and negative skips allowed to line
up the numbers. Once I am totally happy with it, I will add parameter adjustments, as well as a bit of automation of length calculations.
\thispagestyle{empty}

\medskip

\begin{figure}[ht]
\centering
\includegraphics[width=0.35\textwidth]{./chapters/chapter16}
\end{figure}
\lipsum[2]

\cxset{geometry marginparsep/.code=\setlength\marginparsep{#1},
          geometry marginparwidth/.code=\setlength{\marginparwidth}{#1}}

\section{Technical notes}

This design looks simple but takes a bit of effort to achieve it, especially due to the tendency of LaTeX and the TeX engine to make decisions for you. Firstly we cannot reset the page geometry between the image and the chapter, as this will either result in unpredictable behaviour or if you use the \cs{newgeometry} macro, it will for certain leave a blank page in between.

\begin{description}
\item [image sizing] The image is set at \textbf{width=paperwidth}. This can vary depending on the page geometry and the image aspect ratio. In general you may need to ensure that your image has the same aspect ratio as the page to avoid problems with placement and the generation of extra blank pages.
\item [image caption] The image caption is placed using the picture environment, so that it can be typeset absolutely, feel free to use TikZ for the same purpose. We also use Heiko Oberdiek's the \pkg{picture} package to make calculations easier by specifying actual dimensions and not needing to strip the point.
\end{description}



\restoregeometry


%%
\newgeometry{left=7cm,right=2cm, marginparsep=15pt, marginparwidth=4.2cm,top=2cm,%
reversemarginpar}
\cxset{style17/.style={
 name={},
 numbering=arabic,
 number font-size=\huge,
 number font-family=\sffamily,
 number font-weight=\bfseries,
 number before=,
 number dot=.,
 number color= teal,
 number after=\thinspace,
 number position=rightname,
 chapter font-family=\sffamily,
 chapter font-weight=\bfseries,
 chapter font-size=\LARGE,
 chapter before={\vspace*{20pt}\par\hfill},
 chapter after={},
 chapter color= teal,
 title beforeskip={},
 title afterskip={\vspace*{70pt}\par},
 title before={},
 title after={},
 title font-family=\sffamily,
 title font-color= teal,
 title font-shape=\upshape,
 title font-weight=\bfseries,
 title font-size=\huge,
 section numbering=none,
 section beforeskip=10pt,
 section afterskip=10pt,
 section font-family= sffamily,
 section font-shape=upshape,
 section color=teal,
}}

\cxset{style17}

\chapter{Style Seventeen}

I tend to favour this design for books that have a lot of pictures. It brings the design into the margins and leaves plentiful white space in the margins. From a programming point of view the chapter is the opposite of openany. It has to open on an odd number.

\marginpar{%
\vspace*{0.2\textheight}
\includegraphics[width=\marginparwidth]{./chapters/chapter17}\par
{\footnotesize\lorem}
}

\section{Use the margins}

Adjustments to the geometry layout can be carried out temporarily or permanently via the use of keys and
the geometry package. These are probably the less problematic and easier to set geometry settings.

\section{Margin notes}

Marginal notes use the same mechanism as
floats to communicate with the \cs{output} routine. Marginal notes are distinguished from
floats by having a negative placement specification. The command
\cs{marginpar}\oarg{left text}\marg{right text} generates a marginal note in a parbox,
using LTEXT if it's on the left and RTEXT if it's on the right.
(Default is RTEXT = LTEXT.) It uses the following parameters.
\cs{marginparwidth}: Width of marginal notes.
\cs{marginparsep}: Distance between marginal note and text.
the page layout to determine how to move the marginal
note into the margin. E.g.,

\begin{tcolorbox}
\begin{lstlisting}
\@leftmarginskip ==\hskip -\marginparwidth \hskip -\marginparsep .
\end{lstlisting}
\end{tcolorbox}

\cs{marginparpush} Minimum vertical separation between \cs{marginpar}'s
Marginal notes are normally put on the outside of the page
if @mparswitch = true, and on the right if @mparswitch = false.
The command \cs{reversemarginpar} reverses the side where they
are put. \cs{normalmarginpar} undoes \cs{reversemarginpar}.
These commands have no effect for two-column output.
\marginpar{\footnotesize \textsc{\bfseries NOTE:} if two marginal notes appear on the same line of
text, then the second one could appear on the next page, in
a funny position.}
\section{Sample text}
\lipsum[2-4]

\restoregeometry

%%\restoregeometry

\cxset{
 name={CHAPTER},
 numbering=arabic,
 number font-size=\Large,
 number font-family=\rmfamily,
 number font-weight=\normalfont,
 number before=\kern0.5em,
 number after=\hfill\hfill\par\vspace*{20pt}\centerline{\decoone}\vspace*{20pt},
 number dot={},
 number position=rightname,
 name=CHAPTER,
 chapter font-family=rmfamily,
 chapter font-weight=mdweight,
 chapter font-size=Large,
 chapter before={\vspace*{20pt}\par\hfill},
 chapter after={},
 chapter color=black!90,
 number color=black!90,
 chapter title align=center,
 chapter title text-align=center,
 title margin-left=0pt,
 title margin bottom=50pt,
 title margin top=30pt,
 title before=,
 title after=,
 title font-family=rmfamily,
 title font-shape=upshape,
 title font-color= black!90,
 title font-weight=\normalfont,
 title font-size=Huge,
 title display=block}

\chapter[Style 18]{Chapter Style Eighteen}

\parindent0pt
This design introduces an ornament. There are a number of packages on ctan that provide ornaments. If you using XeLaTeX it is also possible to use system fonts. The ornament is introduced with the key number after. At this point also we introduced all the vertical skips.
\medskip
\begin{figure}[ht]
\centering
\includegraphics[width=0.45\textwidth]{./chapters/chapter18.png}
\end{figure}

\section{Sections}
\lorem

\subsection{Subsections}
\lorem


\subsubsection{Subsubsections}
\lorem

\parindent3em
\newcommand{\wb}[2]{\fontsize{#1}{#2}\usefont{U}{webo}{xl}{n}}
\newcommand{\showb}[1]{\wb{12}{14}#1}
\newfontfamily{\minion}{MinionPro-Regular.otf}
\def\ornament{{\minion \char"2740}}

\cxset{chapter name=,
          epigraph align=center,
          epigraph text align=center,
          epigraph rule width=0pt,
          title margin top=10pt,
          number font-size=small,
          number after=\hfill\hfill\par\vspace*{5pt}\centerline{\showb{[]}}\vspace*{5pt},
           %number after=\hfill\hfill\par\vspace*{5pt}\centerline{\Large\ornament}\vspace*{5pt},
          }
\chapter{THE IMPRESSIONISTS IN NEW YORK}


\epigraph{\ldots\itshape a pile of unsung treasures \ldots}{}
\minion

\lettrine{O}{n 13 March 1886, Paul Durant-Ruel and his young son Charles were travelling} through the streets of Paris, on their way to Gare Gare Saint-Lazare. In the two decades since Paul had
inherited his father’s business, Paris had been transformed. Haussmann had
realized his dream. The city was only three years away from the
Exposition of 1889 and the erection of the new Eiffel Tower, the symbol
of modern Paris. By 1890, Baron Haussmann would be saying of his newly
created capital of Europe, ‘these days, it’s fashionable to admire old Paris,
which people only know about from books’. Some areas of Paris had
hardly changed: the poor still lived in the shacks of Montmartre or the
shanties of Belleville; there were still cholera, typhoid, deaths in childbirth
and infant mortality. But to the uninitiated, those problems were now
hidden from view. Paris had a new image: the new Republic was
streamlined and stylish, the epitome of healthy living and good taste.
Haussmann’s Paris was architecturally modern, stratified by wealth,
quintessentially urban and, above all, commercially prosperous.

\begin{figure}[ht]
\centering
\fbox{%
\includegraphics[width=0.8\textwidth]{impressionist-lives}}
\caption{Spread from the Book \emph{The Private Lives of the Impressionists} by Sue Rose and published by Harper Collins.}
\end{figure}

The ctan repository has two good packages for ornamental fonts \pkgname{webomints} and \pkgname{fourier-orns}. The one shown in the orgininal publication is from Minion Symbols Pro.

They have been inserted in the template by using the |number after| key and a custom command from the \pkgname{webomints}
\bigskip

\begin{scriptexample}{}{}
\begin{verbatim}
\newcommand{\wb}[2]{\fontsize{#1}{#2}\usefont{U}{webo}{xl}{n}}
\newcommand{\showb}[1]{\wb{12}{14}#1}
\end{verbatim}
\end{scriptexample}


\ornament

\let\oldsection\section
\long\def\section{%
\par\medskip
\addvspace{20pt}
\centerline{{\LARGE *}}%
\addvspace{20pt}}


Cézanne wanted nothing to do with any war. Taking Hortense with him,
he left their garret at 53, rue Notre-Dame-des-Champs, and made for Aix.
Zola, who had recently married, returned to Provence with his wife,
heading from there to Marseilles. Monet, still in Trouville, waited for the
time being to see how events would turn out. Degas, Renoir, Bazille and
Manet, who stayed behind, were all eligible to fight

Cézanne had been working right up to the last minute to meet the 1866
Salon deadline. On the last possible day for submitting, a wheelbarrow
arrived outside the Palais de l’Industrie, pushed and pulled by Cézanne
and Oller, his Cuban friend from Suisse’s. Cézanne rushed to unwrap his
paintings, eager to show them to anyone who wanted to see. But by now
his hopes were not particularly high. When both his paintings were
rejected he was hardly surprised. He headed straight back to Aix,
complaining to Pissarro about the ‘rotten’ family he was being forced to
rejoin, all of them ‘boring beyond measure’. 

\section

Sections are marked with a single asterisk like ornament. This is a common element
in many non-fiction as well as fiction books. Some might have anything from on to three
asterisks. Many books printed in the nineteenth century have very fancy end section ornamentation.
I like the simplicity of the one asterisk.

\let\section\oldsection
\cxset{title display=in-line block}



%%
\cxset{style19/.style={
 name={},
 numbering=arabic,
 number font-size=Huge,
 number font-family=rmfamily,
 number font-weight=bfseries,
 number before=\par\offinterlineskip,
 number after=\kern0.5em,
 number dot={ },
 number position=rightname,
 chapter font-family=rmfamily,
 chapter font-weight=mdseries,
 chapter font-size=Huge,
 chapter before=\par,
 chapter after=\par,
 chapter color=black!90,
 number  color=black!90,
 chapter title width=0.8\textwidth,
 chapter title align=left,
 title   beforeskip=,
% title afterskip={\vspace*{50pt}\par},
 title margin top=0pt,
 title margin bottom=50pt,
 title margin-left=0pt,
 title before=,
 title after=\par,
 chapter title text-align=left,
 title font-family=rmfamily,
 title font-color=black!90,
 title font-weight=bfseries,
 title font-size=Huge}}
 
\parindent1em

\cxset{style19}
\chapter{Introduction to chapter style nineteen}

I first visited the Gulf seven years after its independence. As Simon C. Smith puts it, in his book \emph{Britain’s Revival and Fall in the Gulf} the decolonization of the Gulf was a mere footnote on British history. As I have lived in and I am currently still working in the area in this footnote for about twelve years Smith’s book sheds light to the beginnings of the Gulf.

\medskip
\begin{figure}[ht]
\centering 
\includegraphics[width=0.5\textwidth]{./chapters/chapter19.png}
\end{figure}

The book’s typography is what one expects from an academic publication. The headings are simple and the text is typeset in Times Roman. I was tempted to name this template \emph{plain vanilla} but perhaps it deserves better.
There is a large section of book designers that believe that the typography of a book should be like a crystal glass.

\begin{quote}
Now the man who first chose glass instead of clay or metal to hold his wine was a ``modernist" in the sense in which I am going to use that term. That is, the first he asked of this particular object was not "How should it look?" but ``What must it do?" and to that extent all good typography is modernist.	
\end{quote}

Throughout the essay, Warde argues for the discipline and humility required to create quietly set, ``transparent" book pages.

Now, back to the template one of the difficulties we will face is that the chapter title blocks are set in Once we adjust the title to be anything less than the width of the text block, we will also need to be careful
about words in order to give it some balance.
two or three lines and they do not extend to the full length of the text block.

The main settings are as follows:

\begin{verbatim}
\cxset{reset,
 chapter title width=0.65\textwidth,
 chapter title align=raggedright,}
\end{verbatim}


\cxset{chapter opening=anywhere}
\chapter{The failure of the federal idea in the Gulf, 1950-68}

The book does not have any lower level headings. Another characteristic is a subtitle below the main chapter block on some of the chapters. The subtitle is set in normal weight and is \emph{partially} used as a heading. 

\testsections





%%<<<<<<< HEAD
\colorlet{toprule}{teal}
\colorlet{theblock}{teal}
%%%%%%%%%%%%%%
%%%%%%%%%%%%%%%%%%%%%%%%%%%%%
%%%%%%  STYLE 20
%%%%%%%%%%%%%%%%%%%%%%%%%%%%%%%%%%%%%%%%%%%
\cxset{rule color/.store in={\rulecolor@cx},
          block color/.store in={\blockcolor@cx}}
\cxset{style20/.style={
 rule color=teal!90,
 block color=teal!90,
 name={},
 numbering = arabic,
 number font-size=\HHUGE,
 number color=\color{white},
 number font-family=\sffamily,
 number font-weight=\bfseries,
 number before={\hbox to 0pt{\vbox to -10pt{\colorbox{\blockcolor@cx}{\rule{0pt}{70pt}\HHUGE \color{\blockcolor@cx}1331}}}\vskip1pt\color{\rulecolor@cx}\rule{\textwidth}{5pt}\par\vskip10pt\relax\hspace{2.5em}},
 number after=\hspace{3em},
 number dot={ },
 number position=leftname,
 chapter font-family=\rmfamily,
 chapter font-weight=\normalfont,
 chapter font-size=\huge,
 chapter before={},
 chapter after={\hskip0pt},
 chapter color={black!90},
 title beforeskip={},
 title afterskip={\vspace*{30pt}\par}, % before text
 title before={\hskip0.2em},
 title after={\par\vspace{0pt}\color{\rulecolor@cx}\rule{\textwidth}{5pt}},
 title font-family=\sffamily,
 title font-color=\color{black!90},
 title font-weight=\bfseries,
 title font-size=\HUGE,
 section color=teal,
 section font-family=\sffamily,
 section font-weight=\bfseries,
 section font-shape=\upshape\color{teal},
 section indent=-10pt,
 header style=plain}}
=======
\colorlet{toprule}{teal}
\colorlet{theblock}{teal}
%%%%%%%%%%%%%%
%%%%%%%%%%%%%%%%%%%%%%%%%%%%%
%%%%%%  STYLE 20
%%%%%%%%%%%%%%%%%%%%%%%%%%%%%%%%%%%%%%%%%%%
\cxset{rule color/.store in={\rulecolor@cx},
          block color/.store in={\blockcolor@cx}}
\cxset{style20/.style={
 rule color=teal!90,
 block color=teal!90,
 name={},
 numbering = arabic,
 number font-size=\HHUGE,
 number color=\color{white},
 number font-family=\sffamily,
 number font-weight=\bfseries,
 number before={\hbox to 0pt{\vbox to -10pt{\colorbox{\blockcolor@cx}{\rule{0pt}{70pt}\HHUGE \color{\blockcolor@cx}1331}}}\vskip1pt\color{\rulecolor@cx}\rule{\textwidth}{5pt}\par\vskip10pt\relax\hspace{2.5em}},
 number after=\hspace{3em},
 number dot={ },
 number position=leftname,
 chapter font-family=\rmfamily,
 chapter font-weight=\normalfont,
 chapter font-size=\huge,
 chapter before={},
 chapter after={\hskip0pt},
 chapter color={black!90},
 title beforeskip={},
 title afterskip={\vspace*{30pt}\par}, % before text
 title before={\hskip0.2em},
 title after={\par\vspace{0pt}\color{\rulecolor@cx}\rule{\textwidth}{5pt}},
 title font-family=\sffamily,
 title font-color=\color{black!90},
 title font-weight=\bfseries,
 title font-size=\HUGE,
 section color=teal,
 section font-family=\sffamily,
 section font-weight=\bfseries,
 section font-shape=\upshape\color{teal},
 section indent=-10pt,
 header style=plain}}
>>>>>>> merged

%%%%%%%%%%%%%%%%%%%%%%%%%%%%%%%
%%%%%%  STYLE 20a
%%%%%%%%%%%%%%%%%%%%%%%%%%%%%%%%%%%%%%%%%%%
\cxset{rule color/.store in={\rulecolor@cx},
          block color/.store in={\blockcolor@cx}}
\cxset{style20a/.style={
 rule color=teal!90,
 block color=cyan,
 name=chapter,
 numbering = arabic,
 number font-size=\HHUGE,
 number color=\color{white},
 number font-family=\sffamily,
 number font-weight=\bfseries,
 number before={\hbox to 0pt{\vbox to -10pt{\colorbox{\blockcolor@cx}{\rule{0pt}{70pt}\HHUGE \color{\blockcolor@cx}1331}}}\vskip1pt\vskip10pt\relax\hspace{2.5em}},
 number after=\hspace{3em},
 number dot={ },
 number position=rightname,
 chapter font-family=\rmfamily,
 chapter font-weight=\normalfont,
 chapter font-size=\large,
 chapter before={},
 chapter after={\hskip0pt},
 chapter color={black!90},
 title beforeskip={},
 title afterskip={\vspace*{30pt}\par}, % before text
 title before={\hskip0.2em},
 title after={\par\vspace{0pt}\color{\rulecolor@cx}\rule{\textwidth}{5pt}},
 title font-family=\sffamily,
 title font-color=\color{black!90},
 title font-weight=\bfseries,
 title font-size=\HUGE,
 section color=teal,
 section font-family=\sffamily,
 section font-weight=\bfseries,
 section font-shape=\upshape\color{teal},
 section indent=-10pt,
 header style=plain}}
\cxset{style20}
\parindent1em
\chapter{STYLE 20}

\lettrine{\textcolor{teal}{T}}{his} style is probably useful in some corporate environment. The layout has been defined traditionally using boxes and skips and can perhaps be improved tremendously via TikZ. I selected this layout from a book titled \textit{Manufacturing at Warp Speed}, Eli Schragenheim, H. William Dettmer, 2001. The original book's chapter header is not coloured. We will use this example to define color schemes and themes.
\index{color schemes}\index{themes}.
\medskip
\begin{figure}[ht]
\centering
\includegraphics[width=0.7\textwidth]{chapter20}
\end{figure}

\section{Creating themes}
The strategy we use to define themes, especially if based on color changes is to define commands to generate them. This way one could define a number of themes fairly quickly.

\section{Adding templates and themes to a library}
Templates that have a lot of themes can be considered libraries and can be loaded with the package. (See the section on libraries).

\begin{texexample}{}{}
\cxset{chapter opening=anywhere}
% #1 style to add themes
% #2 name of theme
% #3 key value list
\newcommand\maketheme[3][style20]{%
\cxset{#1 #2/.style={#1,#3 }}}
% create some themes
\maketheme[style20]{black}{rule color=black,block color=black,}
\maketheme[style20]{blue}{rule color=theblue,block color=theblue,}
\maketheme[style20]{blue}{rule color=theblue,block color=theblue,}
\maketheme[style20]{orange}{rule color=orange,block color=orange,}
\cxset{style20 black}
\chapter{A Test}
\cxset{style20 blue}
\chapter{A Test}
\cxset{style20 orange}
\chapter{A Test}
\end{texexample}

\begin{texexample}{}{}
\fboxrule0pt\fboxsep0pt
\colorbox{teal}{\fbox{\parbox[b]{3cm}{%
\vbox to 0pt{\hbox to 3cm{\hfill\large\itshape\color{white} Chapter\hfill}}
\vbox{}%
\hbox to 3cm{\hfill \color{white}\sffamily\bfseries\HHUGE39\rule{0pt}{60pt}}
\hbox to 3cm{\rule{0pt}{40pt}}
}}}\hspace{0.5em}
\fbox{\parbox[b]{13cm}{%
\huge\color{teal} Paradoxical functional facilitation\\[-1pt] and recovery in neurological\\[-1pt]
 and psychiatric conditions\par
\medskip
\vspace*{20pt}
\color{black}
\large Dr Yiannis Lazaridegj
}}

\end{texexample}
\newcommand\allbluechapter[2][]{%
\fboxrule0pt\fboxsep0pt%
\hspace*{-1em}\fbox{\colorbox{theblock}{\fbox{\parbox[b]{3cm}{%
\vbox to 0pt{\hbox to 3cm{\hfill\large\itshape\color{white} Chapter\hfill}}
\vbox{}%
\hbox to 3cm{\hfill \color{white}\sffamily\bfseries\HHUGE\thechapter\rule{0pt}{60pt}}
\hbox to 3cm{\rule{0pt}{40pt}}%
}}}\hspace{1.5em}
\fbox{\parbox[b]{10cm}{%
\huge\color{teal} #2\par
\medskip
\vspace*{10pt}
\color{black}
\large \authorblockformat@cx\authorblock@cx
}}}
\vspace{25pt}

\thispagestyle{fancy}
}
\clearpage

\@specialtrue
\cxset{custom=allbluechapter,
         header style=plain, %check why is not working
         chapter opening=any,
         subsection numbering=none,
         subsection color=teal,
         subsection font-weight=\bfseries,
         subsection font-shape=\upshape,
         subsection indent=-10pt,
         author block=true,
         author block format=\bfseries\normalfont\raggedright,
         author names={Dr Yiannis Lazarides, Maria Lazarides and Athena Lazarides}}
\renewsubsection\renewsection
\parindent1em
\chapter[Paradoxical facilitation]{Paradoxical functional facilitation\\ and recovery in neurological\\
 and psychiatric conditions}

\section{Introduction}
\lorem

\section{Author Block Formatting}

Each chapter of the book carries the names of its authors, which is typeset as shown above. The standard available fields for author blocks are programmed in the special template. The full settings are shown in Example .
\medskip

\noindent\begin{tcolorbox}
\begin{lstlisting}
\@specialtrue
\cxset{custom=allbluechapter,
         header style=plain, %check why is not working
         chapter opening=any,
         author block=true,
         author block format=\bfseries\normalfont\raggedright,
         author names=Dr Yiannis Lazarides, Maria Lazarides and Athena Lazarides}
\chapter{Paradoxical functional facilitation...}
\end{lstlisting}
\end{tcolorbox}

Remember that it is also possible to add the author with the command \cs{addauthor} or its alias macro \cs{addauthors}. Cases where people can make mistakes are normally aliased to avoid common errors.

\section{Key value interface}
\subsection{General keys}
All keys for chapters chapters can be used in the template and toc.
Additional keys are described below.


\subsection{Other formatting hooks}

When a special template is designed it is prudent to provide hooks for minor tweaks. This way it is unecessary to modify the code of a special template for such changes. Keys have been provided for all the struts etc.
All normal keys can be used, such as font selection, spacing etc.

\lipsum
 % depend on 20
%%<<<<<<< HEAD
\@specialfalse
%%%%%%%%%%%%%%%%%%%%%%%%%%%%%%%%%%%%%%%%%%%
%%%%%%  STYLE 21
%%%%%%%%%%%%%%%%%%%%%%%%%%%%%%%%%%%%%%%%%%%
\newgeometry{left=4.5cm,right=2.5cm, marginparsep=15pt, marginparwidth=4.2cm,top=2cm,%
reversemarginpar}
\cxset{
 chapter opening=right,
 name={},
 numbering=none,
 number font-size=\Large,
 number font-family=\rmfamily,
 number font-weight=\bfseries,
 number before=,
 number after=,
 number position=rightname,
 chapter font-family=\sffamily,
 chapter font-weight=\normalfont,
 chapter font-size=\Large,
 chapter before={\vspace*{0.3\textheight}},
 chapter after={\par},
 chapter color={black!90},
 number color=\color{black!90},
 title beforeskip={},
 title afterskip={\par\rule{\textwidth}{3.5pt}\vspace{20pt}},
 title before={},
 title after={},
 title font-family=\sffamily,
 title font-color=\color{black},
 title font-weight=\bfseries,
 title font-size=\Huge,
 author block=false}


\chapter{INTRODUCTION TO STYLE 21}

\lipsum[1]
\medskip
\begin{figure}[ht]
\centering
\fbox{\includegraphics[width=0.65\textwidth]{./chapters/chapter21}}
\end{figure}
\lipsum[1]
\clearpage
=======
\@specialfalse
%%%%%%%%%%%%%%%%%%%%%%%%%%%%%%%%%%%%%%%%%%%
%%%%%%  STYLE 21
%%%%%%%%%%%%%%%%%%%%%%%%%%%%%%%%%%%%%%%%%%%
\newgeometry{left=4.5cm,right=2.5cm, marginparsep=15pt, marginparwidth=4.2cm,top=2cm,%
reversemarginpar}
\cxset{
 chapter opening=right,
 name={},
 numbering=none,
 number font-size=\Large,
 number font-family=\rmfamily,
 number font-weight=\bfseries,
 number before=,
 number after=,
 number position=rightname,
 chapter font-family=\sffamily,
 chapter font-weight=\normalfont,
 chapter font-size=\Large,
 chapter before={\vspace*{0.3\textheight}},
 chapter after={\par},
 chapter color={black!90},
 number color=\color{black!90},
 title beforeskip={},
 title afterskip={\par\rule{\textwidth}{3.5pt}\vspace{20pt}},
 title before={},
 title after={},
 title font-family=\sffamily,
 title font-color=\color{black},
 title font-weight=\bfseries,
 title font-size=\Huge,
 author block=false}


\chapter{INTRODUCTION TO STYLE 21}

\lipsum[1]
\medskip
\begin{figure}[ht]
\centering
\fbox{\includegraphics[width=0.65\textwidth]{./chapters/chapter21}}
\end{figure}
\lipsum[1]
\clearpage
>>>>>>> merged

%%

\restoregeometry

\newgeometry{left=4.5cm,right=2.5cm, marginparsep=15pt, marginparwidth=4.2cm,top=2cm,
reversemarginpar}

\setdefaults
\cxset{style22/.style={
 name={},
 numbering=none,
 number font-size=\Large,
 number font-family=\rmfamily,
 number font-weight=\bfseries,
 number before=,
 number after=,
 number position=rightname,
 chapter font-family=\sffamily,
 chapter font-weight=\normalfont,
 chapter font-size=\Large,
 chapter before=\vspace*{-10pt},
 chapter after={},
 chapter color=black!90,
 number color= black!90,
 title beforeskip= \raggedleft,
 title afterskip={\vspace{70pt}},
 title before=\hspace*{-2cm},
 title after={},
 title font-family=\sffamily,
 title font-color=black,
 title font-weight=\bfseries,
 title font-size=\huge,
 section numbering=none,
 section font-family=\sffamily,
 section font-weight=\bfseries,
 section color=black,
 section indent= 10pt,
 subsection indent = 0pt,
 header style=plain}}


\cxset{style22}
\renewsection\renewsubsection

\chapter{INTRODUCTION TO STYLE TWENTY TWO}\index{style22}\index{lettrine}\index{drop cap}

\section{INTRODUCTION}

\renewcommand{\DefaultLoversize}{0.3}
\renewcommand{\LettrineTextFont}{\fontfamily{Minion Pro}\normalfont\itshape}
\renewcommand{\LettrineFontHook}{%
\fontseries{bx}\fontshape{up}\color{gray}}

\cxset{lettrine lines/.code=\global\setcounter{DefaultLines}{#1}}

\cxset{lettrine lines=5}

\lettrine[lraise=0.0, nindent=0em, slope=-.5em]{Y}{oic} \lipsum[1]

\medskip
\begin{figure}[ht]
\centering
\fbox{\includegraphics[width=0.5\textwidth]{./chapters/chapter22.png}}
\end{figure}

\lipsum[1-2]
\parindent0pt

%%\setcounter{secnumdepth}{6}

\newgeometry{left=4cm,right=4cm,bottom=2cm}

\cxset{style23/.style={
 name={Chapter},
 numbering=arabic,
 number font-size=\Large,
 number font-family=\rmfamily,
 number font-weight=\bfseries,
 number before={},
 number after={},
 number dot=,
 number position=rightname,
 chapter font-family=\sffamily,
 chapter font-weight=\normalfont,
 chapter font-size=\Large,
 chapter before={\hspace*{-50pt}\rule{\dimexpr\textwidth+50pt\relax}{0.4pt}\par\hspace*{-51pt}},
 chapter after={\par},
 chapter color= black!90,
 number color=gray,
 title beforeskip={},
 title afterskip={\vspace{30pt}},
 title before=\hspace*{-50pt},
 title after={\par\vspace*{-10pt}\hspace*{-50pt}\rule{\dimexpr\textwidth+50pt}{0.4pt}\par},
 title font-family=\rmfamily,
 title font-color=black!80,
 title font-weight=\bfseries,
 title font-size=\LARGE,
 section font-family=\rmfamily,
 section font-shape=\itshape,
 section font-weight=\bfseries,
 section numbering=arabic,
 section indent=-49pt,
 section beforeskip=\baselineskip,
 section afterskip=\baselineskip,
 subsection font-shape=\upshape,
 subsection beforeskip=\baselineskip,
 subsection afterskip=10pt,
 subsection indent=-49pt,
 subsection numbering=arabic,
 subsubsection font-family=\rmfamily,
 subsubsection font-shape=\itshape,
 subsubsection font-weight=\bfseries,
 subsubsection font-shape=\itshape,
 subsubsection font-size=\large,
 subsubsection align=,
 subsubsection beforeskip=10pt,
 subsubsection afterskip=\baselineskip,
 subsubsection indent=-49pt,
 subsubsection numbering=numeric,
 paragraph font-family=\rmfamily,
 paragraph font-shape=\itshape,
 paragraph font-weight=\normalfont,
 paragraph font-shape=\itshape,
 paragraph font-size=\large,
 paragraph align=,
 paragraph beforeskip=10pt,
 paragraph afterskip=0pt,
 paragraph indent=-49pt,
 paragraph numbering=numeric,
 subparagraph font-family=\rmfamily,
 subparagraph font-shape=\itshape,
 subparagraph font-weight=\normalfont,
 subparagraph font-shape=\itshape,
 subparagraph font-size=\large,
 subparagraph align=,
 subparagraph beforeskip=10pt,
 subparagraph afterskip=0pt,
 subparagraph indent=-49pt,
 subparagraph numbering=arabic,
 subparagraph number after=\thinspace,
 header style=empty,
 pagestyle=headings,
}}



\cxset{style23}
\chapter{Introduction to style twenty three}

\section{Introduction}

This style requires that the chapter settings as well as the
section headings are set in the margins, leaving the text after the sectioning commands to be indented. We achieve this by using negative skips.
\medskip
\begin{figure}[ht]
\centering
\fbox{\includegraphics[width=0.45\textwidth]{./chapters/chapter23}
\includegraphics[width=0.45\textwidth]{./chapters/chapter23a}}
\end{figure}

\subsection{Subsections}
The same style is applied to the subsectioning commands up to the subsubsection which is not numbered but just uses an italic font. The subsubsection is also indented into the margin. Since the book is about construction claims it follows a style found in legal and construction documents, where all paragraphs are indented with respect to the section headings.
\lipsum[1-2]
\section{Even pages}
\lipsum[2]
\subsection{Setting different margins}
\lipsum[1]
\subsubsection{Setting Subsubsections}
\lipsum[1]
\paragraph{Paragraph level. } \lipsum*[3]\par

\lipsum[1]
\subparagraph{sub-paragraph level. } \lipsum*[3]\par


\restoregeometry

%%\makeatletter
\clearpage
\cxset{
 name=Chapter,
 numbering=arabic,
 number font-size=\Large,
 number font-family=\sffamily,
 number font-weight=\normalfont,
 number before={},
 number after={\space},
 number position=rightname,
 chapter font-family=\sffamily,
 chapter font-weight=\normalfont,
 chapter font-size=\Large,
 number after={},
 number dot=,
 chapter before={},
 chapter after={\par\thinrule\vskip12pt},
 chapter color=black!90,
 number color= black!90,
 chapter spaceout=none,
 title beforeskip={},
 title afterskip={\vspace{30pt}},
 title before=,
 title after={\par},
 title font-family=\sffamily,
 title font-color= black!80,
 title font-weight=\bfseries,
 title font-size=\LARGE,
 title afterskip=\par\vspace*{3cm}\thinrule\par\bigskip\bigskip,
 section indent= 0pt,
 section font-shape=upshape,
 section font-family=upshape}



\chapter{Introduction to style twenty four}


\def\objectives@{%
 \begin{tcolorbox}[width=\linewidth,boxsep=10pt,right=10pt]
\textbf{Learning Objectives}\parindent0pt\leavevmode}
\def\stopobjectives@{\end{tcolorbox}}
\newenvironment{objectives}{\bigskip\objectives@}{\stopobjectives@\bigskip}

\parindent1em

\begin{objectives}
\par
\lipsum[1]
\bigskip\bigskip
\end{objectives}

This design is ideal for scholarly books or notes. It has a nice clean design with a shaded block for the learning objectives. \lipsum*[2-3]
\medskip
\begin{figure}[ht]
\centering
\fbox{\includegraphics[width=0.6\textwidth]{./chapters/chapter24.png}}
\end{figure}



%%\makeatletter
\cxset{author/.store in=\author@cx}
\cxset{author block=true}

\cxset{style25/.style={
 name={CHAPTER},
 numbering=arabic,
 number font-size=huge,
 number font-family=sffamily,
 number font-weight=\normalfont,
 number before=\kern0.5em,
 number position=rightname,
 chapter font-family=sffamily,
 chapter font-weight=normalfont,
 chapter font-size=huge,
 number after=\hfill\hfill\vskip1pt\hrule width\textwidth height1pt\relax\vskip1pt,
 chapter before=\hrule width\textwidth height1pt\relax\vskip1pt\hfill,
 chapter after=,
 chapter color=black!90,
 number color=black!90,
  title afterskip={\vspace{10pt}\author@cx},
 title before=\leavevmode,
 title after=\vskip1pt\hrule width\textwidth height1pt\relax,
 chapter title align=centering,
 chapter title width=.9\textwidth,
 title font-family=sffamily,
 title font-color= black!80,
 title font-weight=bfseries,
 title font-size=huge}}
\makeatother

\cxset{author=\centering\bfseries\upshape\large Yiannis Lazarides and Athena Lazarides\par\vspace{30pt}}

\cxset{style25}
\chapter{INTRODUCTION TO STYLE 25}

The interesting part of this style is that it uses roman numerals to display the counter that is in a different font than that used for the chapter name.
\lipsum*[1-2]
\begin{figure}[ht]
\centering
\fbox{\includegraphics[width=0.5\textwidth]{./chapters/chapter25.png}}
\end{figure}


%%%%%%%%%%%%%%%%%%%%%%%%%%%%%%%%%%%%%%%%%%%%%
%%%%%%  STYLE 26
%%%%%%%%%%%%%%%%%%%%%%%%%%%%%%%%%%%%%%%%%%%

\cxset{
 name={},
 numbering=arabic,
 number font-size=\huge,
 number font-family=\sffamily,
 number font-weight=\bfseries,
 number before={},
 number position=leftname,
 chapter font-family=\sffamily,
 chapter font-weight=\normalfont,
 chapter font-size=\small,
 number after={},
 chapter before={},
 chapter after={\par\vskip12pt},
 chapter color={black!90},
 number color=\color{black!90},
 title beforeskip={},
 title afterskip={\vspace{30pt}},
 title before=,
 title after={\par},
 title font-family=\sffamily,
 title font-color=\color{black!80},
 title font-weight=\bfseries,
 title font-size=\LARGE}
\chapter{Introduction to style twenty five Dr. Yiannis Lazarides and Athena Lazarides}

The interesting part of this style is that it uses roman numerals to display the counter that is in a different font than that used for the chapter name.
\medskip
\begin{figure}[ht]
\centering
\fbox{\includegraphics[width=0.6\textwidth]{./chapters/chapter26}}
\end{figure}
\lipsum[2-3]

%%

%%%%%%%%%%%%%%%%%%%%%%%%%%%%%%%%%%%%%%%%%%%
%%%%%%  STYLE 27
%%%%%%%%%%%%%%%%%%%%%%%%%%%%%%%%%%%%%%%%%%%

\cxset{
 author block=false,
 name={},
 numbering=arabic,
 number font-size=\HUGE,
 number font-family=\sffamily,
 number font-weight=\bfseries,
 number before={},
 number dot=,
 number position=leftname,
 chapter font-family=\sffamily,
 chapter font-weight=\normalfont,
 chapter font-size=\small,
 number after={},
 chapter before={\vspace*{50pt}},
 chapter after={\par\vskip12pt},
 chapter color={black!90},
 number color=\color{black!90},
 title beforeskip={},
 title afterskip={\vspace{30pt}},
 title before=,
 title after={\par},
 title font-family=\sffamily,
 title font-color=\color{black!80},
 title font-weight=\bfseries,
 title font-size=\huge}
\chapter{Introduction to style twenty seven Dr. Yiannis Lazarides and Athena Lazarides}
\lipsum[3]

\medskip
\begin{figure}[ht]
\centering
\fbox{\includegraphics[width=0.5\textwidth]{./chapters/chapter27}}
\end{figure}
\lipsum[2-3]


%%\cxset{
 name={},
 numbering=arabic,
 number font-size=\HUGE,
 number font-family=\sffamily,
 number font-weight=\bfseries,
 number before={},
 number position=leftname,
 chapter font-family=\sffamily,
 chapter font-weight=\normalfont,
 chapter font-size=\small,
 number after={},
 chapter before={},
 chapter after={\hspace*{20pt}},
 chapter color=black!90,
 number color= black!90,
 title beforeskip={},
 title afterskip={\vspace{70pt}},
 title before=,
 title after={\par},
 title font-family=\itshape,
 title font-color= black!80,
 title font-weight=\itshape,
 title font-size=\LARGE,
 author block=false}

\chapter{Introduction to Style Twenty Eight}

The interesting part of this style is that it uses roman numerals to display the counter that is in a different font than that used for the chapter name.
\medskip
\begin{figure}[ht]
\centering
\includegraphics[width=0.6\textwidth]{./chapters/chapter28.png}
\end{figure}
\lipsum[2]

\section{Sectioning Commands}
\lorem

\subsection{Subsectioning commands}
\lorem
%% \makeatletter
\cxset{plain sections/.style={
 chapter name = CHAPTER,
 chapter toc = true,
 chapter color= thegray,
 chapter opening = right, 
 chapter numbering = arabic,
 chapter font-family= sffamily,
 chapter font-weight= bold,
 chapter font-size= LARGE,
 chapter before={\thinrule\vspace*{20pt}\par\hfill\hfill},
 chapter after={\vskip0pt\par},
 chapter spaceout = soul,
 number font-size= Large,
 number font-family= rmfamily,
 number font-weight= bfseries,
 number color=thegray,
 number before=\vspace*{5pt}\hfill\hfill,
 number dot=.,
 number after={\hspace*{7pt}\par},
 title beforeskip={\vspace*{10pt}},
 title afterskip={\vspace*{50pt}\par},
 title before={\hfill\hfill\raggedleft},
 title after={\par\thinrule},
 title font-family=\sffamily,
 title font-color= teal,
 title font-weight=\bfseries,
 title font-family=\sffamily,
 title font-size= Large,
 title font-shape= upshape,
 title spaceout= none,
 title beforeskip={\vspace*{10pt}},
 title afterskip={\vspace*{50pt}\par},
 title before={\hfill\hfill\raggedleft},
%
% numbers
% number font-family=\sffamily,
% number font-weight=\bfseries,
 number color=thelightgray,
 number before=\par\vspace*{5pt}\hfill\hfill,
 number dot=.,
 number after={\hspace*{7pt}\par},
 number position=rightname,
 section color= thered,     
 section beforeskip=15pt,
 section afterskip=15pt,
 section indent=0pt,
 section font-family= sffamily,
 section font-size= LARGE,
 section font-weight= bfseries,
 section font-shape=,
 section align= centering,
 section numbering prefix =,%use \thechapter. for books or add as option
 section numbering= arabic,
 section spaceout=none,
 section number after=ooo,
 subsection color= thered,
       subsection beforeskip=10pt,
       subsection afterskip=10pt,
       subsection indent=0pt,
       subsection font-family= rmfamily,
       subsection font-size= large,
       subsection font-weight= bold,
       subsection font-shape= upshape,
       subsection align= centering,
       subsection numbering prefix=\thesection.,%\S\hairsp,%add . 
       subsection numbering custom =\@arabic\c@subsection,% \two@digits{\@arabic\c@subsection},%
       subsubsection color= gray,
       subsubsection beforeskip=5pt plus3pt minus 2pt,
       subsubsection afterskip=5pt,
       subsubsection indent=0pt,
       subsubsection font-family= rmfamily,
       subsubsection font-size= normalfont,
       subsubsection font-weight= bold,
       subsubsection font-shape= itshape,
       subsubsection align= centering,
       subsubsection numbering prefix =\thesubsection.\@arabic\c@subsubsection,
       subsubsection numbering custom =, %\two@digits{\@arabic\c@subsubsection},
       subsubsection number after =, 
%
       paragraph color= thegrey,
       paragraph beforeskip=,
       paragraph afterskip=-0.5em,
       paragraph indent=0pt,
       paragraph font-family= rmfamily,
       paragraph font-size= large,
       paragraph font-weight= bfseries,
       paragraph font-shape=,
       paragraph align= centering,
       paragraph number after = 0pt,
       paragraph numbering=numeric,
       subparagraph color= thered,
       subparagraph beforeskip=0pt,
       subparagraph afterskip=-.5em,
       subparagraph indent=0pt,
       subparagraph font-family= sffamily,
       subparagraph font-size= large,
       subparagraph font-weight= normalfont,
       subparagraph font-shape= slshape,
       subparagraph align= RaggedRight,
       subparagraph number after =, % can affect all needs checking
       %subsubsection numbering prefix=\S\hairsp\thesection,%add . here if need be
       subparagraph numbering=none,
}
}
\cxset{plain sections}
\cxset{style13/.style={
 name= {\protect\pan अमुकग्रन्थे},
 chapter spaceout = none,
 numbering=arabic,
 number font-size= HUGE,
 number font-family= sffamily,
 number font-weight= bfseries,
 number color= gray!50,
 number before=\par\vspace*{5pt}\hfill\hfill,
 number dot=,
 number after={\hspace*{7pt}\par},
 number position=rightname,
 chapter font-family= sffamily,
 chapter font-weight= bold,
 chapter font-size= LARGE,
 chapter before={\tikzrule\vspace*{20pt}\par\hfill\hfill},
 chapter color= black!50,
 title beforeskip={\vspace*{10pt}},
 title afterskip={\vspace*{50pt}\par},
 title before={\hfill\hfill\raggedleft},
 chapter rule color=spot!50,
 title after=\par\tikzrule,
 title font-family= sffamily,
 title font-color= teal,
 title font-weight= bfseries,
 title font-size= huge,
 section indent=-1em,
 section align= left,
 section numbering= arabic,
 section indent=0pt,
 section beforeskip=0pt,
 section afterskip= 10pt,
 section color=teal,
 subsection align= ,
 subsection font-family= sffamily,
 subsection font-weight= bfseries,
 subsection color = teal,
 subsection font-size= large,
 subsection font-shape=,
 subparagraph number after=,
 subsubsection align=,
}
}
\cxset{style13}

\renewparagraph
\renewsection
\renewsubsection
\renewsubparagraph
\renewsubsubsection

\makeatother
  
 \cxset{style29/.style={
 name={},
 numbering=arabic,
 number font-size=normalsize,
 number font-family=sffamily,
 number font-weight=bfseries,
 number before={\vspace*{30pt}},
 number position=leftname,
 number after=\hrule width\textwidth height1pt\par,
 chapter font-family=sffamily,
 chapter font-weight=,
 chapter font-size=small,
 chapter before={\vskip2.5pt},
 chapter after=,
 chapter color= black!90,
 number color= black!90,
 title beforeskip={},
 title afterskip={\bigskip},
 title before=,
 title after={\par},
 title font-family=rmfamily,
 title font-color= black!80,
 title font-weight=bfseries,
 title font-size=\huge,
 chapter title align=raggedright,
 section indent=0pt,
 section numbering=arabic,
 section font-family=\rmfamily,
 section font-shape=\upshape,
 section numbering prefix=\thechapter.,
section numbering suffix=,
 section color=black,
 subsection number after=\quad,
 subsection number after=\quad}}
\cxset{style29}

\endinput
\renewsection\renewsubsection

\chapter{Reading Systems: An Introduction to Digital Document Processing}
\bigskip\bigskip

\textit{Lambert Schoemacher}
\bigskip\bigskip\bigskip\bigskip

\section{Introduction}

Style 29 comes from the computer world and is representative of conference publications. It is always instructive
to go back and read research undertaken decades ago to understand the present state of the art but also to study how standards emerge and the competitive forces that shape the survivability of computer software. 


\begin{figure}[ht]
\centering
\includegraphics[width=0.5\textwidth]{./chapters/chapter29.png}
\end{figure}

The template is based on a Springer-Verlag London publication dated 2007 from a series on Advance Pattern Recognition. Many of these publications were prepared using \latex itself and templates from this firm are still available and on ctan for its many journals. The introduction by Schoemacher provides background information to anyone interested to understand the evolution of document processing. 

\section{Documents}

The word document comes from the Latin word ‘documentum’, which has the same
stem as the verb ‘doceo’ (meaning ‘to teach’), plus the suffix ‘-umentum’ (indicating
a means for doing something). Hence, it is intended to denote ‘a means for teaching’
(in the same way as ‘instrument’ denotes a means to build, ‘monument’ denotes a
means to warn, etc.). Dictionary definitions of a document are the following [2]:
\begin{enumerate}
\item Proof, Evidence
\item An original or official paper relied on as basis, proof or support of something
\item Something (as a photograph or a recording) that serves as evidence or proof
\item A writing conveying information
\item A material substance (as a coin or stone) having on it a representation of thoughts
by means of some conventional mark or symbol.
\end{enumerate}

The first definition is more general. The second one catches the most intuitive
association of a document to a paper support, while the third one extends the definition
to all other kinds of support that may have the same function. While all
these definitions mainly focus on the bureaucratic, administrative or juridic aspects
of documents, the fourth and fifth ones are more interested in its role of information
bearer that can be exploited for study, research, information. Again, the former
covers the classical meaning of documents as written papers, while the latter extends
it to any kind of support and representation. Summing up, three aspects can be
considered as relevant in identifying a document: its original meaning is captured
by definitions 4 and 5, while definition 1 extends it to underline its importance as
a proof of something, and definitions 2 and 3 in some way formally recognize this
role in the social establishment.

\section {Current Landscape}

While up to recently documents were produced in paper format, and their digital
counterpart was just a possible consequence carried out for specific purposes,
nowadays we face the opposite situation: nearly all documents are produced and
exchanged in digital format, and their transposition on a tangible, human-readable
support has become a successive, in many cases optional, step. Additionally, significant
efforts have been spent in the digitization of previously existing documents


\section{The Shelf-life of Documents}

In a post at \texttt{TX.SX} Barbara Beeton mentioned that one of the advantages of \tex is that documents produced by it results in documents with a very long shelf life and that Mathematics has a long shelf life.
Of course the best way to ensure a long shelf life is to have the document printed out and stored in an archive.
Paper is still the best way to ensure long term survivability. Knuth’s idea when he insisted that \tex cannot be changed was to ensure that a document could be printed always the same way. I am not too sure if this is an absolute necessity, as what is important is for the content to survive.


%%\makeatletter
\cxset{plain sections/.style={
 chapter name = CHAPTER,
 chapter toc = true,
 chapter color= thegray,
 chapter opening = right, 
 chapter numbering = arabic,
 chapter font-family= sffamily,
 chapter font-weight= bold,
 chapter font-size= LARGE,
 chapter before={\thinrule\vspace*{20pt}\par\hfill\hfill},
 chapter after={\vskip0pt\par},
 chapter spaceout = soul,
 number font-size= Large,
 number font-family= rmfamily,
 number font-weight= bfseries,
 number color=thegray,
 number before=\vspace*{5pt}\hfill\hfill,
 number dot=.,
 number after={\hspace*{7pt}\par},
 title beforeskip={\vspace*{10pt}},
 title afterskip={\vspace*{50pt}\par},
 title before={\hfill\hfill\raggedleft},
 title after={\par\thinrule},
 title font-family=\sffamily,
 title font-color= teal,
 title font-weight=\bfseries,
 title font-family=\sffamily,
 title font-size= Large,
 title font-shape= upshape,
 title spaceout= none,
 title beforeskip={\vspace*{10pt}},
 title afterskip={\vspace*{50pt}\par},
 title before={\hfill\hfill\raggedleft},
%
% numbers
% number font-family=\sffamily,
% number font-weight=\bfseries,
 number color=thelightgray,
 number before=\par\vspace*{5pt}\hfill\hfill,
 number dot=.,
 number after={\hspace*{7pt}\par},
 number position=rightname,
 section color= thered,     
 section beforeskip=15pt,
 section afterskip=15pt,
 section indent=0pt,
 section font-family= sffamily,
 section font-size= LARGE,
 section font-weight= bfseries,
 section font-shape=,
 section align= centering,
 section numbering prefix =,%use \thechapter. for books or add as option
 section numbering= arabic,
 section spaceout=none,
 section number after=ooo,
 subsection color= thered,
       subsection beforeskip=10pt,
       subsection afterskip=10pt,
       subsection indent=0pt,
       subsection font-family= rmfamily,
       subsection font-size= large,
       subsection font-weight= bold,
       subsection font-shape= upshape,
       subsection align= centering,
       subsection numbering prefix=\thesection.,%\S\hairsp,%add . 
       subsection numbering custom =\@arabic\c@subsection,% \two@digits{\@arabic\c@subsection},%
       subsubsection color= gray,
       subsubsection beforeskip=5pt plus3pt minus 2pt,
       subsubsection afterskip=5pt,
       subsubsection indent=0pt,
       subsubsection font-family= rmfamily,
       subsubsection font-size= normalfont,
       subsubsection font-weight= bold,
       subsubsection font-shape= itshape,
       subsubsection align= centering,
       subsubsection numbering prefix =\thesubsection.\@arabic\c@subsubsection,
       subsubsection numbering custom =, %\two@digits{\@arabic\c@subsubsection},
       subsubsection number after =, 
%
       paragraph color= thegrey,
       paragraph beforeskip=,
       paragraph afterskip=-0.5em,
       paragraph indent=0pt,
       paragraph font-family= rmfamily,
       paragraph font-size= large,
       paragraph font-weight= bfseries,
       paragraph font-shape=,
       paragraph align= centering,
       paragraph number after = 0pt,
       paragraph numbering=numeric,
       subparagraph color= thered,
       subparagraph beforeskip=0pt,
       subparagraph afterskip=-.5em,
       subparagraph indent=0pt,
       subparagraph font-family= sffamily,
       subparagraph font-size= large,
       subparagraph font-weight= normalfont,
       subparagraph font-shape= slshape,
       subparagraph align= RaggedRight,
       subparagraph number after =, % can affect all needs checking
       %subsubsection numbering prefix=\S\hairsp\thesection,%add . here if need be
       subparagraph numbering=none,
}
}
\cxset{plain sections}
\cxset{style13/.style={
 name= {\protect\pan अमुकग्रन्थे},
 chapter spaceout = none,
 numbering=arabic,
 number font-size= HUGE,
 number font-family= sffamily,
 number font-weight= bfseries,
 number color= gray!50,
 number before=\par\vspace*{5pt}\hfill\hfill,
 number dot=,
 number after={\hspace*{7pt}\par},
 number position=rightname,
 chapter font-family= sffamily,
 chapter font-weight= bold,
 chapter font-size= LARGE,
 chapter before={\tikzrule\vspace*{20pt}\par\hfill\hfill},
 chapter color= black!50,
 title beforeskip={\vspace*{10pt}},
 title afterskip={\vspace*{50pt}\par},
 title before={\hfill\hfill\raggedleft},
 chapter rule color=spot!50,
 title after=\par\tikzrule,
 title font-family= sffamily,
 title font-color= teal,
 title font-weight= bfseries,
 title font-size= huge,
 section indent=-1em,
 section align= left,
 section numbering= arabic,
 section indent=0pt,
 section beforeskip=0pt,
 section afterskip= 10pt,
 section color=teal,
 subsection align= ,
 subsection font-family= sffamily,
 subsection font-weight= bfseries,
 subsection color = teal,
 subsection font-size= large,
 subsection font-shape=,
 subparagraph number after=,
 subsubsection align=,
}
}
\cxset{style13}

\renewparagraph
\renewsection
\renewsubsection
\renewsubparagraph
\renewsubsubsection

\makeatother

\cxset{style30/.style={
 name={},
 numbering=arabic,
 number font-size=HUGE,
 number font-family=sffamily,
 number font-weight=bfseries,
 number before=\rule{\textwidth}{5pt}%
                           \par\vspace*{12pt}%
                           \hspace*{20pt},%
 number after=\hspace{1em},
 number position=leftname,
 chapter font-family=sffamily,
 chapter font-weight=normalfont,
 chapter font-size=small,
 chapter before={},
 chapter after={\hspace*{20pt}},
 chapter color= black!90,
 number color= black!90,
 title beforeskip={},
% title afterskip={\vspace{30pt}},
 title before=,
 title after=,
title margin bottom=30pt,
 title font-family=sffamily,
 title font-color=black!80,
 title font-weight=\normalfont\sffamily,
 title font-size=Huge,
 chapter title align=raggedright,
 chapter title width=.7\textwidth,
 section numbering prefix=\thechapter.,
 section numbering=arabic,
 subsection number after=\quad,
}}


\cxset{style30}

\chapter[Introduction to Style Thirty]{{\language-1 Introduction to Style Thirty with a Somehow Long Title to Illustrate the Example}}

Since we do not know how long a chapter title can end up, it is best to
typeset this using two minipages or parboxes. The number is pushed down slightly although it can look as good with both the number and the text fully aligned on top.
\medskip
\begin{figure}[ht]
\centering
\includegraphics[width=0.6\textwidth]{./chapters/chapter30.png}
\end{figure}

\section{Testing}

\lipsum[1-5]





%%\makeatletter
\cxset{style31/.style={
 name={},
 numbering=arabic,
 number font-size=\HUGE,
 number font-family=\sffamily,
 number font-weight=\bfseries,
 number before=,
 number after={},
 number position=leftname,
 chapter font-family=\sffamily,
 chapter font-weight=\normalfont,
 chapter font-size=\small,
 chapter before={},
 chapter after={\vskip2.5pt{\color{gray}\rule{3cm}{5pt}\rule[3.5pt]{\dimexpr\textwidth-3cm\relax}{0.4pt}}\par},
 chapter color=gray,
 number color=gray,
 title beforeskip={},
 title afterskip={\vspace{30pt}},
 title before=,
 title after={\par{\color{gray}\rule[6pt]{3cm}{0.4pt}}\par},
 title font-family=\itshape,
 title font-color=black,
 title font-weight=itshape,
 title font-shape=itshape,
 title font-size=LARGE}}

\cxset{style31}
\chapter[Evolution of Organizations and the Environment]{The Evolution of Organizations\\ and the Environment}

This is an unusual design by all counts. I did soften the rules a bit to make them a bit less conspicuous.
\medskip
\begin{figure}[ht]
\centering
\includegraphics[width=0.6\textwidth]{./chapters/chapter31.png}
\end{figure}

\lipsum[1-2]

\section{Test}

\lipsum[1]

\lipsum[2]


%%
\newgeometry{left=5cm,right=2cm,bottom=2cm}
\cxset{style32/.style={
 name={},
 numbering=arabic,
 number font-size=\HUGE,
 number font-family=\sffamily,
 number font-weight=\bfseries,
 number before={\hspace*{-10pt}},
 number position=leftname,
 chapter font-family=\sffamily,
 chapter font-weight=\normalfont,
 chapter font-size=\small,
 number after={},
 chapter before={\vspace*{50pt}\par\hspace*{-60pt}},
 chapter after={\hspace*{20pt}},
 chapter color=black!90,
 number color=black!90,
 title beforeskip={},
 title afterskip={\vspace{70pt}},
 title before=,
 title after={\par},
 title font-family=\itshape,
 title font-color=black!80,
 title font-weight=\bfseries,
 title font-shape=\itshape,
 title font-size= Huge,
}}

\cxset{style32}
\chapter{Introduction to Style Thirty Two}

This style has a modern look to it. Its main characteristic is the large chapter number and the fact that it is drawn into the margin. A common style for computer books.
\medskip
\begin{figure}[ht]
\centering
\includegraphics[width=0.6\textwidth]{./chapters/chapter32}
\end{figure}
The example is from Python NLP book.


%%
\restoregeometry
\cxset{style33/.style={
 name=CHAPTER,
 numbering=arabic,
 number font-size= LARGE,
 number font-family= rmfamily,
 number font-weight= \normalfont,
 number before={\par\hfill},
 number after={\hfill\hfill\par},
 number position=leftname,
 name=,
 chapter font-family=\sffamily,
 chapter font-weight=\normalfont,
 chapter font-size=\small,
 chapter before={\vskip10pt},
 chapter after={\vskip10pt\par},
 chapter color= black!90,
 number color= black!90,
 title beforeskip={},
 title afterskip={\vspace{50pt}},
 title before=\hfill,
 title after={\hfill\hfill\par},
 title font-family=\normalfont,
 title font-color= black!80,
 title font-weight=\normalfont,
 title font-shape=\upshape,
 title font-size=\LARGE,
 section numbering=none,
 section font-size=\Large,
 section align= center}}

\cxset{style33}
\chapter{Introduction to Style Thirty Three}

The interesting part of this style is that it uses roman numerals to display the counter that is in a different font than that used for the chapter name.
\medskip
\begin{figure}[ht]
\centering
\includegraphics[width=0.6\textwidth]{./chapters/chapter33}
\end{figure}


%%
\cxset{
 name=CHAPTER,
 numbering=Roman,
 number font-size=\small,
 number font-family=\rmfamily,
 number font-weight=\normalfont,
 number before={},
 number position=rightname,
 chapter font-family=\sffamily,
 chapter font-weight=\normalfont,
 chapter font-size=\small,
 number after={},
 chapter before={},
 chapter after={\par},
 chapter color={black!90},
 number color= black!90,
 title beforeskip={},
 title afterskip={\vspace{50pt}},
 title before=,
 title after={\par},
 title font-family=\normalfont,
 title font-color=\color{black!80},
 title font-weight=\normalfont,
 title font-size=\LARGE}

\section{Basic astronomical phenomena}

The interesting part of this style is that it uses roman numerals to display the counter that is in a different font than that used for the chapter name.
\medskip
\begin{figure}[ht]
\centering
\includegraphics[width=0.6\textwidth]{./chapters/chapter34.png}
\end{figure}

%%\input{style35.tex}  TO DO
%%
%%%%%%%%%%%%%%%%%%%%%%%%%%%%%%%%%%%%%%%%%%%
%%%%%%  STYLE 36
%%%%%%%%%%%%%%%%%%%%%%%%%%%%%%%%%%%%%%%%%%%

\cxset{numbering=none}
%% has errors
\cxset{style36/.style={
 name={},
 numbering={none},
 number font-size=\small,
 number font-family=\rmfamily,
 number font-weight=\normalfont,
 number before={},
 number position=rightname,
 chapter font-family=\sffamily,
 chapter font-weight=\normalfont,
 chapter font-size=\small,
 number after={},
 chapter before={},
 chapter after={},
 chapter color={black!90},
 number color=\color{black!90},
 title beforeskip={},
 title before=,
 title after={\vskip-12.5pt\rule{\columnwidth}{3.5pt}\vspace*{50pt}},
 title afterskip={},
 title font-family=\sffamily,
 title font-color=\color{black!80},
 title font-weight=\bfseries,
 title font-size=\Huge}}

\cxset{style36}
\chapter{Introduction to Style Thirty Six}

The interesting part of this style is that it uses roman numerals to display the counter that is in a different font than that used for the chapter name.
\medskip

\begin{figure}[ht]
\centering
\includegraphics[width=0.6\textwidth]{./chapters/chapter36}
\end{figure}

%%\cxset{style37/.style={
 name=CHAPTER,
 numbering=Roman,
 number font-size=\small,
 number font-family = sffamily,
 number font-weight= bold,
 number before={},
 number position=rightname,
 chapter font-family= sffamily,
 chapter font-weight= bold,
 chapter font-size=\small,
 chapter spaceout=soul,
 number after={},
 chapter before={},
 chapter after={\par},
 chapter color=black!90,
 number color=black!90,
 title beforeskip={},
 title afterskip={\vspace{50pt}},
 title before=,
 title after={\par},
 title font-family= sffamily,
 title font-color= black!80,
 title font-weight= sfseries,
 title font-size=\huge}}

\cxset{style37}
\chapter{Introduction to Style Thirty Seven}

The interesting part of this style is that it uses roman numerals to display the counter that is in a different font than that used for the chapter name.
\medskip

\begin{figure}[ht]
\centering
\includegraphics[width=0.6\textwidth]{./chapters/chapter37}
\end{figure}


%%\cxset{style38/.style={
 name=,
 numbering=arabic,
 number font-size= huge,
 number font-family= rmfamily,
 number font-weight= normalfont,
 number before= \centering,
 number position=leftname,
 number after =\centering,
 chapter font-family= sffamily,
 chapter font-weight= normalfont,
 number after={\vspace*{6.5pt}\par},
 chapter before=\par,
 chapter after=\par,
 chapter color= black!90,
 number color=black!90,
 title beforeskip={},
 title afterskip={\vspace{50pt}},
 title before=,
 title after=\par,
 title font-family= rmfamily,
 title font-color= black!80,
 title font-weight=\normalfont,
 title font-size= huge,
 chapter font-size=,
}}

\cxset{style38,
       author block=true,
       author block format=\centering}
\chapter{STAGES OF INITIATION IN THE \vspace{0pt} ELEUSINIAN AND\\SAMOTHRACIAN MYSTERIES\\STYLE 38}

This style uses rules to enclose both the chapter name and number as well as the title, which necessarily needs to be rather short.
\medskip

\begin{figure}[ht]
\centering
\includegraphics[width=0.6\textwidth]{./chapters/chapter38}
\end{figure}

 % fix author block
%\cxset{
 name=CHAPTER,
 numbering=arabic,
 number font-size=\LARGE,
 number font-family=\sffamily,
 number font-weight=\bfseries,
 number before={},
 number position=rightname,
 chapter font-family=\sffamily,
 chapter font-weight=\bfseries,
 number after={},
 chapter before={\rule{\textwidth}{2pt}\par},
 chapter after={\vskip0pt\vspace*{-8pt}\rule{\textwidth}{.4pt}\vskip-7pt},
 chapter color={black!90},
 number color=black!90,
 title beforeskip={},
 title afterskip={\vspace{50pt}},
 title before=,
 title after={\par\vskip-16.5pt\rule{\textwidth}{0.4pt}\par} ,
 title font-family=\sffamily,
 title font-color=black!80,
 title font-weight=\bfseries,
 title font-size=\LARGE,
 chapter font-size=\LARGE,
 author block=false}

\chapter{STYLE 39}

This style uses rules to enclose both the chapter name and number as well as the title, which necessarily needs to be rather short.
\medskip

\begin{figure}[ht]
\centering
\includegraphics[width=0.6\textwidth]{./chapters/chapter39.png}
\end{figure}
In the picture it does not look very attractive, but in the actual book it does. My observation is that the rule clearances are a bit tight and if you use this type of layout it is better to experiment until you get them right.




%<<<<<<< HEAD

%%%%%%%%%%%%%%%%%%%%%%%%%%%%%%%%%%%%%%%%%%%
%%%%%%  STYLE 40
%%%%%%%%%%%%%%%%%%%%%%%%%%%%%%%%%%%%%%%%%%%

\cxset{
 name=,
 numbering=arabic,
 number font-size=\LARGE,
 number font-family=\sffamily,
 number font-weight=\bfseries,
 number before={},
 number position=rightname,
 chapter font-family=\sffamily,
 chapter font-weight=\bfseries,
 chapter before=\hfill,
 number after=,
 chapter after=\hfill\hfill\vskip0pt ,
 chapter color={black!90},
 number color=\color{black!90},
 title beforeskip=,
 title before=\begin{center},
 title after=\end{center},
 title font-family=\sffamily,
 title font-color=\color{black!80},
 title font-weight=\bfseries,
 title font-size=\LARGE,
 chapter font-size=\LARGE,
 author block=true,
 author block format=\normalfont\Large\centering,
 author names=Karin Wahl-Jorgensen and Thomas Hanitzch}


\chapter{Introduction:\\ On Why and How to Use\\ Chapter Style Forty}

A simple design for sombre multi-author books. This is now common in many publications such as proceedings, conferences and the like. They are actually mostly collections of articles, but formatted as books.

\begin{figure}[ht]
\centering
\includegraphics[width=0.6\textwidth]{./chapters/chapter40}
\end{figure}

One issue with such designs is how to make it easy for the user to add the author block. There are two pathways, the one is to use \cs{cxset} and the other is to make a special command for it.

=======

%%%%%%%%%%%%%%%%%%%%%%%%%%%%%%%%%%%%%%%%%%%
%%%%%%  STYLE 40
%%%%%%%%%%%%%%%%%%%%%%%%%%%%%%%%%%%%%%%%%%%

\cxset{
 name=,
 numbering=arabic,
 number font-size=\LARGE,
 number font-family=\sffamily,
 number font-weight=\bfseries,
 number before={},
 number position=rightname,
 chapter font-family=\sffamily,
 chapter font-weight=\bfseries,
 chapter before=\hfill,
 number after=,
 chapter after=\hfill\hfill\vskip0pt ,
 chapter color={black!90},
 number color=\color{black!90},
 title beforeskip=,
 title before=\begin{center},
 title after=\end{center},
 title font-family=\sffamily,
 title font-color=\color{black!80},
 title font-weight=\bfseries,
 title font-size=\LARGE,
 chapter font-size=\LARGE,
 author block=true,
 author block format=\normalfont\Large\centering,
 author names=Karin Wahl-Jorgensen and Thomas Hanitzch}


\chapter{Introduction:\\ On Why and How to Use\\ Chapter Style Forty}

A simple design for sombre multi-author books. This is now common in many publications such as proceedings, conferences and the like. They are actually mostly collections of articles, but formatted as books.

\begin{figure}[ht]
\centering
\includegraphics[width=0.6\textwidth]{./chapters/chapter40}
\end{figure}

One issue with such designs is how to make it easy for the user to add the author block. There are two pathways, the one is to use \cs{cxset} and the other is to make a special command for it.

>>>>>>> merged

%\makeatletter
\cxset{chapter author/.store in=\chapterauthor@cx}
\cxset{style41/.style={
 color=purple,
 name=CHAPTER,
 numbering=WORDS,
 number font-size=\large,
 number font-family=\rmfamily,
 number font-weight=\normalfont,
 number before={},
 number position=rightname,
 chapter font-family=\rmfamily,
 chapter font-weight=\normalfont,
 chapter before=\hfill,
 chapter spaceout=none,
 number after=,
 chapter after=\hfill\hfill\vskip0pt,
 chapter color = black!90,
 number color= black!90,
 title beforeskip=,
 title before=\begin{center},
 title after=\par\end{center},
 title spaceout=none, 
 title font-family=\rmfamily,
 title font-color= black!80,
 title font-weight=\normalfont,
 title font-size=\LARGE,
 chapter font-size=\large,
 section numbering=none,
 section indent=0pt,
 section align=\centering,
 section font-shape=\upshape,
 section font-weight=\normalfont,
 section font-size=\large,
 section spaceout=soul,
 section beforeskip=10pt,
 section afterskip=10pt,
}}

\cxset{style41,
       chapter author=Yiannis Lazarides,
       epigraph width=0.85\textwidth,
       epigraph text align=left,
       epigraph source align=right,
       epigraph rule width=0pt,
       epigraph afterskip=50pt,
       author block=true,
       author block format=\normalfont\itshape\Large\centering,
}

\renewsection

\addauthors{Dr Yiannis Lazarides}
\chapter{INTRODUCTION TO CHAPTER STYLE FORTY ONE}

\label{ch:41}
\epigraph{The existence of an area of free land, its continuous recession, and the advance of American
settlement westward explain American development.}{Frederick Jackson Turner, \textit{The Significance of the Frontier in American\\ History,} Columbian Exploration, Chicago, July 12, 1893}

A classical style chapter style with finely spaced out letters. A number of books spell out the chapter numbers. In general I find this as a good idea as sometimes the numbers don't blend in very well with the design.

\begin{figure}[ht]
\centering
\fbox{\includegraphics[width=0.5\textwidth]{./chapters/chapter41}}
\end{figure}

This book has different chapters written by different authors and the author's name appear below an ornament. Don't dismiss ornaments as old fashioned as a lot of modern books still use them.

\begin{lstlisting}
\cxset{style41,
         chapter author=Yiannis Lazarides}
\end{lstlisting}

The ornaments I used was from the \texttt{fourier-orns} package. Here is a MWE if you want to experiment with various designs.


\begin{lstlisting}
\documentclass{article}
\usepackage{fourier-orns}
\begin{document}
\Huge
\textxswup\textxswdown
\decoone\decotwo
\decothreeleft\decothreeright
\decofourleft\decofourright
\floweroneleft\floweroneright
\end{document}
\end{lstlisting}

\section{THINGS THAT ARE NOT AUTOMATED}

If you need the title to be spaced out using the soul package and you have a line break, the package will issue the error `reconstruction failed'. In this case it is better to include the spaceout commands in the title (subject to hyperref not breaking up everything).


\begin{verbatim}
\chapter{\so{INTRODUCTION TO}\\ \so{CHAPTER STYLE FORTY ONE}}
\end{verbatim}


%\cxset{style42/.style={
 name=,
 chapter opening=right,
 numbering= arabic,
 number font-size=\huge,
 number before={},
 number position=leftname,
 chapter before=\vspace*{10pt},
 number after=\hfill\hfill,
 chapter after=\hfill\hfill\vskip20pt ,
 number color= gray,
 title font-family=\sffamily,
 title font-color= black!80,
 title font-weight=,
 title font-size=\Huge,
 title before=,
 title after=\par,
 author block=false,
 section numbering=none,
 epigraph width=0.85\textwidth,
 epigraph text align=left,
 epigraph source align=right,
 epigraph rule width=0pt,
 epigraph afterskip=30pt,
}}

\cxset{style42}
\chapter{Introduction to Style Forty Two}

\epigraph{Tell me, O Muse, of that ingenious hero who trawled far and wide after he had
sacked the famous town of Troy. Many cities did he visit, and many were the nations with whose manners
and customs he was acquainted; moreover he suffered much by sea while trying to save his own life and bring
his men safely home \ldots }{Homer, \textit{The Odyssey}}

Style 42 is shown in the following figure:

\begin{figure}[ht]
\centering
\includegraphics[width=0.6\textwidth]{./chapters/chapter42.png}
\end{figure}
The distinguishing characteristics of this chapter are that it has an epigraph and is composed of very simple stylistic elements. The epigraph is placed quite a bit lower than the chapter title. The heading style is just the page number and underlined.




%
%% Chapter 43 Style 43

\cxset{
 name=CHAPTER,
 number dot=,
 numbering=arabic,
 number font-size=\Large,
 number before={},
 number position=rightname,
 chapter color={black!80},
 chapter font-size=\Large,
 chapter before=\par\hfill\hfill,
 number after=,
 chapter after=\vskip20pt ,
 number color=\color{black!80},
 title font-family=,
 title font-color=\color{black!95},
 title font-weight=\itshape,
 title font-size=\LARGE,
 title font-shape=\itshape,
 title spaceout=none,
 title beforeskip=\hfill,
 epigraph width=0.95\textwidth,
 epigraph font-size=\normalfont,
 header style=empty,
 blank page text=THIS PAGE LEFT INTENTIONALLY BLANK,
 author block=false}



\cxset{headings ruled-01}
\cxset{chaptermark name=,
          chaptermark after number=,
          header bottom rule=false,
          header style=empty}

\chapter{Introduction to Style 43}

\epigraph{The Jebel Druse is a country of great feudal chiefs, whose efforts are
directed to preserving the powers by which they live.What we call
progress means in their eyes the loss of their privileges and later on
perhaps the partition of their lands.With regard to the inhabitants,
who are ignorant or unmindful of any better fate, they are deeply rooted
in their serfdom and are as conservative as their masters. They have no
aspirations for a system of greater social justice nor [sic] for a better
communal life.}{---Testimony to the League of Nations Permanent Mandates\\
Commission investigating the Syrian Revolt, Geneva, 1926}

\epigraph{Syrians, remember your forefathers, your history, your heroes, your
martyrs, and your national honor. Remember that the hand of God is
with us and that the will of the people is the will of God. Remember
that civilized nations that are united cannot be destroyed.

The imperialists have stolen what is yours. They have laid hands on
the very sources of your wealth and raised barriers and divided your
indivisible homeland. They have separated the nation into religious
sects and states. They have strangled freedom of religion, thought,
conscience, speech, and action.We are no longer even allowed to move
about freely in our own country.

To arms! Let us realize our national aspirations and sacred hopes.

To arms! Confirm the supremacy of the people and the freedom of
the nation.

To arms! Let us free our country from bondage.}{---Excerpt from a rebel manifesto signed\\ by Sultan
al-Atrash and issued on 23 August 1925}

\lettrine{T}his style is reminiscent of the stylistic elements found in Tufte's books with the chapter title set in italics.

\begin{figure}[ht]
\centering
\includegraphics[width=0.95\textwidth]{chapter43.jpg}\par
\includegraphics[width=0.95\textwidth]{chapter43a}
\end{figure}

I saw this style in the \textit{The Great Syrian Revolution and the Rise of Arab Nationalism} by Michael Provence, published by the University of Texas at Austin (2005). Notably the best part of the first page is taken by epigraphs, but as you can see from the image, the ugly ``This Page intentionally left blank'' is all over the place, perhaps they could have been mover over? The chapter opens on an even page and bear no headers or footers. The large dropcap at the start of the chapter text balances the ragged left elements of the chapter block.
\lipsum

%%\cxset{
 name=,
 numbering=none,
 number font-size=,
 number before=,
 number after=,
 number position=rightname,
 chapter color=black,
 chapter font-size=,
 chapter before=\thinrule,
 chapter after=\vskip20pt ,
 number color= black!80,
 title font-family=\rmfamily,
 title font-color= black!80,
 title font-weight=,
 title font-size=\huge,
 title font-shape=\upshape,
 title beforeskip=,
 title after=\par\vspace*{20pt},
 title afterskip=\vspace*{20pt},
 header style=empty,
 author block format=\normalfont\upshape\LARGE,
 author block=true,
 section align= RaggedRight,
 section indent=0pt,
 section font-weight=\bfseries,
 section font-size=\LARGE}

\addauthors{D.T.Potts}
\chapter{A Short Introduction\\ to Style Forty Five\\ including the addition\\of an author}

\section{Introduction}
This is an unusual book with a rather unique style. The vertical rule is simple and breaks the monotony of a book that is heavy on text.
\begin{figure}[ht]
\centering
\includegraphics[width=0.6\textwidth]{./chapters/chapter45}
\end{figure}


UNITED ARAB EMIRATES
a new perspective
Edited by
IBRAHIM AL ABED
PETER HELLYERTrident Press Ltd
Layout and design,1997, 2001 Trident Press Ltd,UK.





\cxset{
 name=CHAPTER,
 numbering=arabic,
 number font-size= Large,
 number before={},
 number position=rightname,
 chapter color=black!80,
 chapter font-size=Large,
 chapter before=\par\hfill\hfill,
 number after=,
 chapter after=\vskip20pt ,
 number color=black!80,
 title font-family=\itshape,
 title font-color= black!80,
 title font-weight=,
 title font-size=LARGE,
 title beforeskip=\hfill,header style=empty}

\section{Introduction to Style 46}


This is an unusual book with a rather unique style. The vertical rule is simple and breaks the monotony of a book that is heavy on text.
\begin{figure}[ht]
\includegraphics[width=0.45\textwidth]{./chapters/chapter46}
\includegraphics[width=0.45\textwidth]{./chapters/chapter46a}
\end{figure}

Understanding the Arab Culture a cross-cultural guide, second edition, published by How To Content, Dr Jehad Al-Omari,2008.


\def\anornament{
\begin{tikzpicture}[decoration={markings,
  mark=between positions 0 and 1 step 8pt
  with { \draw [fill=black] (0,0) circle [radius=1pt];}}]
\path[postaction={decorate}] (0,0) to (15,0);
\end{tikzpicture}}


\cxset{style46/.style={
 name=CHAPTER,
 numbering=arabic,
 number font-size=\Large,
 number before={},
 number position=rightname,
 chapter color=black!80,
 chapter font-size=\Large,
 chapter before=\par\hfill,
 number after=,
 chapter after=\hfill\hfill\par,
 number color= black!80,
 title font-family=\rmfamily,
 title font-shape=\upshape,
 title font-color= black!80,
 title font-weight=,
 title font-size=\LARGE,
 title before=\par\anornament\par \centering,
 title after= \vskip-10pt\anornament\par\vspace*{10pt},
 title beforeskip=,
 title afterskip=\vspace*{30pt},
 author block format=\normalsize,
 }}

\cxset{style46}
\addauthors{Jonathan Taylor}
\chapter{INTRODUCTION TO STYLE\\ FORTY SIX }

This is an unusual book with a rather unique style. The book is heavy on text and I introduced it to show the possibilities of ornaments with TikZ. The rule is made out of tikz decorations as per an answer on tex.sx.
\begin{figure}[ht]
\includegraphics[width=0.48\textwidth]{./chapters/chapter48}\hfill
\includegraphics[width=0.48\textwidth]{./chapters/chapter48a}
\end{figure}


 LOTS OF ERRORS
%
The Oxford handbook of cuneiform Culture
\makeatletter
\def\@seccntformat#1{\protect\makebox[0pt][l]{\csname the#1\endcsname}}
\makeatother
\cxset{
 name=CHAPTER,
 numbering=arabic,
 number font-size=\LARGE,
 number font-weight=\bfseries,
 number before=\hfill\vrule height15pt width1.5pt\hspace*{5pt},
 number after=,
 number position=rightname,
 chapter color={black!80},
 chapter font-size=\Large,
 name=,
 chapter before=\vbox to -8pt\bgroup\vskip7.5pt\hbox to \textwidth\bgroup,
 chapter after=\egroup\egroup,
 number color= black!80,
 chapter title width=0.7\textwidth,
 chapter title align=raggedright,
 title font-family=arial,
 title font-color=black!80,
 title font-weight=\normalfont,
 title font-size=Huge,
 title font-shape=upcase,
 title before=\hsize\dimexpr\textwidth-50pt\par\raggedleft,
 title after=\par,
 title beforeskip=,
 header style=empty,
 author block=false}



\chapter{Collecting Dreams: Watching the Sleeping Mind}

\epigraph{\rightskip50pt Children begin by loving their parents. After a time they judge them. Rarely if ever they forgive them.}{\rightskip35pt---Oscar Wilde, \textit{A woman of No Importance}}

\lorem

This is an unusual book with a rather unique style. The vertical rule is simple and breaks the monotony of a book that is heavy on text.
\begin{figure}[ht]
\fbox{%
\includegraphics[width=0.48\textwidth]{./chapters/chapter49.png}\hfill
\includegraphics[width=0.48\textwidth]{./chapters/chapter49a.png}}

\caption{Style 49 from the Oxford Handbook of Cuneiform Culture.}
\end{figure}

%<<<<<<< HEAD
I battled a bit to set everything and in retrospect chapters like this should have been programmed as specials. However, as soon as they are set, adjustments are easily done.



%% Style 50

\cxset{style50/.style={
 name=,
 numbering=arabic,
 number font-size=\LARGE,
 number font-weight=\bfseries,
 number before={},
 number position=rightname,
 number dot=.,
 chapter color={black!80},
 chapter font-size=,
 chapter before=,
 number after=,
 chapter after=,
 number color=\color{black!80},
 title font-family=\rmfamily,
 title font-color=\color{black!80},
 title before=,
 title after=\par,
 title font-weight=\bfseries,
 title font-size=\LARGE,
 title beforeskip=\space,
 header style=empty,
 author block=true,
 author names=\textsc{James A. Russel and\\[-1.5pt] Jos\'e Miguel Fernandez-Dols },
 author block format=\normalfont\large,
 epigraph width=0.8\textwidth, epigraph align=left}}

\cxset{style50}
\chapter[Chapter Style Fifty]{Introduction to Chapter \\Style Fifty}

\epigraph{\raggedleft The human face -- in repose and in movement, at the moment of death as in life, in silence and in speech, when seen or seemed from within, in actuality or as recorded in art or recorded by the camera}{F. Ekman}

This is an unusual book with a rather unique style. The vertical rule is simple and breaks the monotony of a book that is heavy on text.
\begin{figure}[ht]
\includegraphics[width=0.48\textwidth]{./chapters/chapter50}\hfill
\includegraphics[width=0.48\textwidth]{./chapters/chapter50a}
\caption{Style 50 from the Oxford Handbook of Cuneiform Culture.}
\end{figure}

This style is very modern and typical of many computer books. The difficulty is in integrating all the page elements to make it work flawlessly.

The psychology of facial
expression
Edited by
James A. Russell
University of British Columbia
Jose Miguel Fernandez-Dols
Universidad Autonoma de Madrid, Cambridge University Press.

=======
I battled a bit to set everything and in retrospect chapters like this should have been programmed as specials. However, as soon as they are set, adjustments are easily done.



%% Style 50

\cxset{style50/.style={
 name=,
 numbering=arabic,
 number font-size=\LARGE,
 number font-weight=\bfseries,
 number before={},
 number position=rightname,
 number dot=.,
 chapter color={black!80},
 chapter font-size=,
 chapter before=,
 number after=,
 chapter after=,
 number color=\color{black!80},
 title font-family=\rmfamily,
 title font-color=\color{black!80},
 title before=,
 title after=\par,
 title font-weight=\bfseries,
 title font-size=\LARGE,
 title beforeskip=\space,
 header style=empty,
 author block=true,
 author names=\textsc{James A. Russel and\\[-1.5pt] Jos\'e Miguel Fernandez-Dols },
 author block format=\normalfont\large,
 epigraph width=0.8\textwidth, epigraph align=left}}

\cxset{style50}
\chapter[Chapter Style Fifty]{Introduction to Chapter \\Style Fifty}

\epigraph{\raggedleft The human face -- in repose and in movement, at the moment of death as in life, in silence and in speech, when seen or seemed from within, in actuality or as recorded in art or recorded by the camera}{F. Ekman}

This is an unusual book with a rather unique style. The vertical rule is simple and breaks the monotony of a book that is heavy on text.
\begin{figure}[ht]
\includegraphics[width=0.48\textwidth]{./chapters/chapter50}\hfill
\includegraphics[width=0.48\textwidth]{./chapters/chapter50a}
\caption{Style 50 from the Oxford Handbook of Cuneiform Culture.}
\end{figure}

This style is very modern and typical of many computer books. The difficulty is in integrating all the page elements to make it work flawlessly.

The psychology of facial
expression
Edited by
James A. Russell
University of British Columbia
Jose Miguel Fernandez-Dols
Universidad Autonoma de Madrid, Cambridge University Press.

>>>>>>> merged

%\cxset{style51/.style={
 chapter opening=any,
 name=CHAPTER,
 numbering=arabic,%change to WORDS
 number font-size=Large,
 number font-weight=normal,
 number font-family=rmfamily,
 number before=\kern0.5em,
 number dot=,
 number position=rightname,
 number border-width=0pt,
 number padding-bottom=0pt,
 number border-style=none,
 chapter border-width=0pt,
 %chapter name
 chapter color=black!80,
 chapter font-size= Large,
 chapter font-weight=normalfont,
 chapter font-family=sffamily,
 chapter before=,
 chapter spaceout=soul,
 number after=,
 chapter after=,
 chapter margin left=2cm,
 number color=black!80,
 chapter padding=0pt,
 %chapter title
 chapter title width=0.8\textwidth,
 title font-family=normalfont,
 title font-color=black!80,
 title font-weight=,
 title font-size=huge,
 chapter title align=none,
 chapter title text-align=left,
 title margin-left=2cm,
 title before=,
 title after=,
 title beforeskip=,
 title afterskip=,
 author block=false,
 section font-size=\Large,
 section font-weight=bfseries,
 section indent=0pt,
 epigraph width=\dimexpr(\textwidth-2cm)\relax,
 epigraph align=left,
 section font-weight=\normalfont,
 header style=empty}}

\cxset{style51}

\chapter[Template 51]{The Secret Meetings\\ of the Executive Committee\\ of the National Security Council\\  Style Fifty One}

\par
\bigskip
\hspace*{2cm}\begin{minipage}{0.7\textwidth}
                     \textbf{\sffamily Tuesday, October 16, 11:50 \textsc{a.m}, Cabinet Room}\par
                 ``How do you know this is a medium-range ballistic missile?''\\President John F. Kennedy.
\end{minipage}
\bigskip

\noindent This is an unusual book with a rather unique style. The vertical rule is simple and breaks the monotony of a book that is heavy on text.
\begin{figure}[ht]
\fbox{%
\includegraphics[width=0.48\textwidth]{./chapters/chapter51.png}\hfill
\includegraphics[width=0.48\textwidth]{./chapters/chapter51a.png}}
\caption{Style 51 Spread.}
\end{figure}

\section{Some notes}

If you observe this style carefully you will notice that the full title block, including the epigraph are set in from the left margin. This is achieved by setting appropriate \cs{hspace} lengths. The epigraph key width, needs to have the right distance and calculate the width. Example~ \ref{ch:style51} demonstrates the technique.

\cxset{chapter opening=anywhere}

\example
\begin{verbatim}
\cxset{style51,
   chapter opening=anywhere,
   epigraph align=right,
   epigraph width=\dimexpr(\textwidth-1.0cm)\relax,
  }
\end{verbatim}

\solution
    
\chapter{Introduction to Chapter Style 51}
\epigraph{\textbf{\sffamily Tuesday, October 16, 11:50 \textsc{a.m}, Cabinet Room}\par
               ``How do you know this is a medium-range ballistic missile?''}{President John F. Kennedy}
\lorem

\cxset{chapter opening=any,}


%\cxset{
 name=CHAPTER,
 numbering=arabic,
 number font-size=large,
 number position=rightname,
 chapter spaceout=soul,
 chapter color=black,
 chapter font-size=large,
 chapter before=\hrule\vskip1pt\hfill,
 chapter after=,
 number before=,
 number after=\hfill\hfill\vskip1pt\hrule\vskip0pt\par,
 chapter margin left=0pt,
 number color=black,
 title font-family=bfseries,
 title font-color=black,
 title font-weight=,
 title font-size=Huge,
 title before=,
 title after=\par,
 title beforeskip=\vspace*{1cm},
 title afterskip=,
 title margin bottom=1.5cm,
 title margin-left=0pt,
 chapter title width=0.8\textwidth,
 chapter title align=centering,
 epigraph width=0.85\textwidth,
 epigraph align=center,
 header style=empty,
 epigraph rule width=0pt,
 section font-weight=bold}

\debugtitle
\chapter{The Chomskyan Revolution, Introduction to Style Fifty Two}

\epigraph{In the late forties \ldots\ it seemed to many that the conquest of syntax finally lay open before the profession. At the beginning of the fifties confidence was running high.}{--H. Allan Gleason}

\section{Looking for Mr. Goodstructure}

\lorem
\begin{figure}[ht]
\centering
\fbox{%
\includegraphics[width=0.35\textwidth]{./chapters/chapter52.png}
\includegraphics[width=0.35\textwidth]{./chapters/chapter52a.png}}

\caption{Style 50 from the Oxford Handbook of Cuneiform Culture.}
\end{figure}

This style is very modern and typical of many computer books. The difficulty is in integrating all the page elements to make it work flawlessly.

%
\cxset{style64/.style={
 name=,
 numbering=Roman,
 number font-size=Large,
 number font-family=rmfamily,
 number before=,
 number after=,
 number dot=,
 number position=rightname,
 number float=center,
 number display=block,
 number color=sweet,
 chapter color=black,
 chapter font-size=Large,
 chapter before=,
 chapter after=,
 chapter margin left=0pt,
 title font-family=rmfamily,
 title font-color=sweet,
 title font-weight=,
 title font-size=LARGE,
 title before=,
 chapter title align=centering,
 chapter title width=0.8\textwidth,
  title beforeskip=,
 title after= {\par\offinterlineskip
                   \vskip20pt
                  \hbox to \textwidth{\hfill\vrule width2cm height1pt\hfill}%
                   \vskip2pt
                   \hbox to \textwidth{\hfill\vrule width2cm height1pt\hfill}
                  \vskip10pt},
 header style=empty}}

\cxset{style64}

\chapter[Style 53]{VAULT OF THE AGES STYLE FIFTY THREE}


The book alleges the discovery of Jesus tomp \cite{style53}. True or not, the book eloquently describes  life in Israel in the early years of the second millenium.\footnote{From \textit{The
Jesus
Family
Tomb,   
The Discovery, the Investigation,
and the Evidence
That Could Change History}, 
Simcha Jacobovici and Charles Pellegrino
Harper Collins.}

Simcha Jacobovici and Charles
Pellegrino have carried out and delivered evidence based on their own investigations to 
connect a first-century Jewish tomb found in Talpiot, Jerusalem,
in 1980 to the tomb of Jesus and his family. They also provide physical evidence from within the tomb says about Jesus, his death, and his relationships with the other family members found in
the same burial site. I read the book and remained unconvinced. 

\begin{figure}[ht]
\fbox{%
\includegraphics[width=0.48\textwidth]{./chapters/chapter53}\hfill
\includegraphics[width=0.48\textwidth]{./chapters/chapter53a}}
\caption{Style 53 Page spread.}
\end{figure}

The book is written for the wide public and although it does provides references 
\lipsum

\section{Illustrations}

Illustrations less than the width of the textblock are set with a sideways caption.

\begin{figure}[ht]
\includegraphics[width=\textwidth]{jesus}
\caption{Style 53 Image pages.}
\end{figure}
\lipsum[1-4]







%
%% Style 54

\cxset{appendix name/.store in=\appendix@cx}
\cxset{
 name=,
 numbering=arabic,
 number font-size=LARGE,
 number before={},
 number after=,
 number dot=,
 number position=rightname,
 number color=blue,
 chapter color=blue,
 chapter font-size=Large,
 chapter before=\par\hfill\hfill,
 chapter after=\hfill,
 title font-family=\rmfamily,
 title font-color=blue,
 title font-weight=,
 title font-size=\LARGE,
 title beforeskip=\hfill,
 title afterskip={\vspace*{20pt}},
 header style=empty}
\chapter{{STYLE FIFTY FOUR}}


This is an unusual book with a rather unique style. The vertical rule is simple and breaks the monotony of a book that is heavy on text.\index{rules!style 54}

\begin{figure}[ht]
\fbox{\includegraphics[width=0.48\textwidth]{./chapters/chapter54.png}}\hfill
\fbox{\includegraphics[width=0.48\textwidth]{./chapters/chapter54a.png}}
\caption{Style 54 from Steward's Calculus.}
\end{figure}

%This style has already been discussed.

\@specialtrue
%\setstyle{53}
\cxset{appendix name/.code=\gdef\appendixname{#1}}

\cxset{steward,
  appendix name=Appendix,
  numbering=arabic,
  custom=tikzspecial,
  offsety=0cm,
  image=rainbow,
  texti={So far we have seen how to reset styles for the common sectioning commands, such as chapters and sections. Other common elements of a book such as an appendices are discussed here.},
  textii={We have already investigated some of the applications of derivatives, but now that we know the differentiation
rules we are in a better position to pursue the applications of differentiation in greater depth. Here
we learn how derivatives affect the shape of a graph of a function and, in particular, how they help us
locate maximum and minimum values of functions. Many practical problems require us to minimize a
cost or maximize an area or somehow find the best possible outcome of a situation. In particular, we will
be able to investigate the optimal shape of a can and to explain the location of rainbows in the skys.}
}

\appendix
\cxset{numbering=Alpha,
       section numbering=numeric,
       section indent=0pt,
       section beforeskip=10pt,
       section afterskip=10pt}

\chapter{STYLING APPENDICES}

\section{Appendix section}

As far as LaTeX is concerned, there is nothing special in styling an appendix. It is either a chapter or a section with a different name. This name in order to allow internationalization is called \lstinline{appendixname}.
\bigskip

\begin{tcolorbox}[width=\linewidth]
\begin{lstlisting}
\newcommand\appendix{\par
  \setcounter{chapter}{0}%
  \setcounter{section}{0}%
  \gdef\@chapapp{\appendixname}(*@\footnote{The actual literal used for   \textbackslash{appendixname} is defined later on, so that you can customize the language}\label{appendixname}@*)
  \gdef\thechapter{\@Alph\c@chapter}
}
\end{lstlisting}
\end{tcolorbox}
\medskip

The code above is only a simplified version of the command. One might need to add more formatting information such as resetting equation numbers, tables and figures and any special floating environments that have their own numbering.

\begin{tcolorbox}[width=\linewidth]
\begin{lstlisting}
\renewcommand\appendix{\par
                \stepcounter{chapter}
                \setcounter{chapter}{0}
                \stepcounter{section}
                \setcounter{section}{0}
                \setcounter{equation}{0}
                \setcounter{figure}{0}
                \setcounter{table}{0}
                \setcounter{footnote}{0}
  \def\@chapapp{\appendixname}%
  \renewcommand\thechapter{\@Alph\c@chapter}}
\end{lstlisting}
\end{tcolorbox}


\section{Usage}

With the \lstinline{classx} package appendices are formatted as chapters.

\begin{tcolorbox}[width=\linewidth]
\begin{lstlisting}
\appendix
\cxset{numbering=Alpha}
\chapter{STYLING APPENDICES}
\end{lstlisting}
\end{tcolorbox}

\subsection{Enhancements}
More enhancements are possible. For one we can get rid of the chapter, which semantically is not very good, one should have followed a similar style to that of the sections and the \lstinline!\appendix{Title}!. I am not sure if this wouldn't be a bit confusing to people.

\subsubsection{Appendices at end of chapters}
Some styles require appendices to be set at the end of each chapter. These type of appendices can also be added. However an appendix counter might need to be defined.

\paragraph{test paragraph}

\subparagraph{test subparagraph}


%\setstyle{13}
\cxset{numbering=Alpha, name=Appendix}
\@specialfalse%required to negate effect of special tikz picture.

\chapter{Another Appendix}
\section{Calling appendix styles}
\lipsum[1-3]


\@specialtrue
\cxset{steward,
  appendix name=Appendix,
  numbering=Alpha,
  custom=tikzspecial,
  offsety=0cm,
  image=rainbow,
  texti={So far we have seen how to reset styles for the common sectioning commands, such as chapters and sections. Other common elements of a book such as an appendices are discussed here.},
  textii={We have already investigated some of the applications of derivatives, but now that we know the differentiation
rules we are in a better position to pursue the applications of differentiation in greater depth. Here
we learn how derivatives affect the shape of a graph of a function and, in particular, how they help us
locate maximum and minimum values of functions. Many practical problems require us to minimize a
cost or maximize an area or somehow find the best possible outcome of a situation. In particular, we will
be able to investigate the optimal shape of a can and to explain the location of rainbows in the sky.}
}

%\chapter{The Special Environments\\Quotation and Quote}

\label{quotations}


\begin{figure}[p]
\centering
\fbox{\includegraphics[width=0.9\linewidth]{quotations-01}}
\caption{Many books have quotes flushed right.}
\label{frightquotation}
\end{figure}

\begin{figure}[p]
\centering
\fbox{\includegraphics[width=0.9\linewidth]{full-width-quotation}}
\caption[Sample quotation.]{Other books have the quotations full width, but in smaller font as shown above. the extract is from \textit{The Essential Turing}, Edited by B. Jack Copeland and  published by the Oxford University Press, 2004. }
\label{fullwidthquotation}
\end{figure}


\section{Quotation}
In the standard \LaTeXe\ classes the quotation and quote environment are defined by making clever use of the list environment. The main difference between the quotation and the quote environment is that the first line of the former is indented. The key value interface for the quotation environment is shown below and a similar one exists for the quotation environment:


\let\quotation\oquotation
\begin{quotation}
\lipsum[1]
\end{quotation}

The standard classes offer a very similar enevironment with the only difference the first line is not indented and is illustrated below:

\begin{quote}
\lipsum[1]
\end{quote}


\section{Key-value interface}\index{quotation!keys}

\keyval{quote above}{\marg{dim}}{Skip dimension for above quotation skip.}
\keyval{quote below}{\marg{dim}}{Skip dimension for below quotation skip.}
\keyval{quote parindent}{\marg{dim}}{Paragraph indentation.}
\keyval{quote parsep}{\marg{dim}}{Paragraph below skip.}
\keyval{quote left margin}{\marg{dim}}{Paragraph below skip.}
\keyval{quote right margin}{\marg{dim}}{Paragraph below skip.}
\vfill

\index{quotation!example}
\begin{tcblisting}{title=Quotation environment example,width=\textwidth}
\setquotation{%
  quotation above=36pt,
  quotation left margin=30pt,
  quotation right margin=0pt,
  quotation parsep=10pt,
  quotation font-size=\small\color{teal},
  quotation parindent=1em,
}
\lorem

\begin{quotation}
\lipsum[2-3]
\end{quotation}
\end{tcblisting}

\section{Quote}
This is the quote environment:
\begin{quote}
\lipsum[1-2]
\end{quote}


\section{Some commonly used styles}

Besides the centered quotation \fref{fullwidthquotation} shows a style
common in Oxford University Publications. This one is from \textit{The Essential Turing}, Edited by B. Jack Copeland and  published by the Oxford University Press, 2004. Perhaps indicative of the efforts to keep costs down quotations are set at full width, but in smaller font. They
both look good and keep the cost down by reducing the amount of
paper required to print the book.

\topline

Von Neumann gave his engineers `On Computable Numbers' to read when, in
1946, he established his own project to build a stored-programme computer at
the Institute for Advanced Study.\textsuperscript{22} Julian Bigelow, von Neumann's chief engineer,
recollected:
\vspace*{-20pt}

\cxset{quotation example/.style={
  quotation above=0pt,
  quotation left margin=0pt,
  quotation right margin=0pt,
  quotation parsep=10pt,
  quotation font-size=\small\color{teal},
  quotation parindent=1em,
}}

\cxset{quotation turing/.style={
  quotation above=0pt,
  quotation left margin=0pt,
  quotation right margin=0pt,
  quotation parsep=10pt,
  quotation font-size=\small,
  quotation parindent=1em,
}}

\cxset{quotation theme/.code = \setquotation{quotation #1},
       quotation style/.code = \setquotation{quotation #1}}



\cxset{quotation theme = example}




\begin{quotation}

The person who really\ldots pushed the whole Weld ahead was von Neumann, because he
understood logically what [the stored-programme concept] meant in a deeper way than
anybody else\ldots The reason he understood it is because, among other things, he understood
a good deal of the mathematical logic which was implied by the idea, due to the
work of A. M. Turing\ldots in 1936-1937\ldots Turing's [universal] machine does not sound
much like a modern computer today, but nevertheless it was. It was the germinal
idea\ldots So\ldots [von Neumann] saw\ldots that \textsc{[ENIAC]} was just the first step, and that great
improvement would come.\textsuperscript{23}
\end{quotation}

\bottomline

Personally I like this style, especially for books that have a lot
of lengthy citations such as typically found in the humanities and
scientific fields.

\section{Theming}

To make things easier for the designer and to enable easy re-use of
styles we defined a theme key. You first define your keys via
the \cs{cxset} command and then you call it normally using the 
theme. You can extend it, if you like to use subthemes, such as 
quotation theme |quotation theme = example teal|. 

\begin{teX}
\cxset{quotation example/.style={
  quotation above=0pt,
  quotation left margin=0pt,
  quotation right margin=0pt,
  quotation parsep=10pt,
  quotation font-size=\small\color{teal},
  quotation parindent=1em,
}}
\cxset{quotation theme = example}
\end{teX}








%%FIX LIST DIAGRAM

\cxset{steward,
  chapter name=chapter,
  chapter format= stewart,
   image={sweepers.jpg},
  texti={Lists are essential elements of any document style and perhaps the most troublesome to get right.
         In this chapter we discuss the construction of lists and offer a key value interface.},
  textii={The Chapter discusses in detail the construction of lists. It reviews the mechanisms offered
          by LaTeX and outlines a key value approach to building lists. We define a standard interface that does not
          interfere with the original commands. The three standard list styles \textit{enumerate, itemize} and \textit{description} are redesigned to accept a key value interface. The photograph is Lewis Hine's which noted: ``Ivey Mill Company, Hickory, N.C. Some doffers and sweepers. Plenty of them.'' Location: Hickory, Catawba County Date: November 1908. Photographs like this were used by Hine to campaign against child labour.
         }
}

\def\storyi{Lists are essential elements of any document style and perhaps the most troublesome to get right.
         In this chapter we discuss the standard lists offered in the LaTeX classes and describe how new lists can be constructed. We review and use:
         
         \begin{enumerate}
           \item enumerate
           \item itemize
           \item description
           \item trivlist
         \end{enumerate}
   
Some commonly used packages are also reviewed.        
         }

\cxset{palette spring onion}
\pagestyle{headings-spring-onion}
\makeatletter
\def\imagewidth@cx{6cm}
\makeatother
\cxset{chapter format=fashion,
       fashion image=fashion-pngtree.png}

\chapter{Standard \LaTeX\ Lists}

\pagebreak

\section{Introduction}

There are four environments for producing formatted lists:\footnote{There are also other environments, such as \emph{quote}, %
\emph{quotation}, \emph{verbatim}, which behind the scenes are also lists.}

\begin{trivlist}
\item |\begin|\marg{trivlist} list text |\end{trivlist}|
\item |\begin|\marg{itemize} list text |\end{itemize}|
\item |\begin|\marg{enumerate} list text |\end{enumerate}|
\item |\begin|\marg{description} list text |\end{description}|
\end{trivlist}

Lists shape their contents so that 
 the \emph{list text} is indented from the left margin
and a label, or marker, is included. What type of label is used depends
on the selected list environment. The command to produce the label is |\item|. Any following paragraphs, i.e., paragraphs types without being prefixed by |\item| are at the same distance from the margin. 

Lists can be nested either mixed or of one type to a depth of four levels. The type of label used depends on the level of nesting. The indentation is always relative to the left margin or right margin of the enclosing list.


\cxset{label itemi = \textbullet,
       label itemii = *}

\begin{itemize}
\item This is the first level of the list.
  \begin{itemize}
     \item This is the second level of the list.
  \end{itemize}
\item And back to the first level.
\end{itemize}

The optional argument of the |\item| can be used to change the label in the itemize and enumerate environments. The optional
argument takes precedence over the standard label. For the enumerate environment, this means that the corresponding counter is not automatically incremented. You will need to do the numbering manually.



\begin{texexample}{Example with manual settings}{ex:settings}
\section{Example of the itemize environment}
\begin{itemize}
\item[---] This is the first level of the list.
  \begin{itemize}
     \item[\textbullet] This is the second level of the list.
  \end{itemize}
\item[---] And back to the first level.
\end{itemize}
\end{texexample}

The optional label appears right justified within the area reserved for
the label. The width of this area is the amount of indentation at that level
less the separation between label and text; this means that the left edge
of the label area is flush with the left margin of the enclosing level.

It is also possible to change the standard labels for all or part of the
document. The labels are generated with the internal commands

\bgroup
\trivlist\item
\cs{labelitemi}, \cs{labelitemii}, \cs{labelitemiii}, \cs{labelitemiv}, 
\cs{labelenumi}, \cs{labelenumii}, \cs{labelenumiii}, \cs{labelenumiv}
\endtrivlist
\egroup

The endings i, ii, iii, and iv refer to the four possible levels.
These commands may be altered with \cs{renewcommand}. For example,
to change the label of the third level of the itemize environment to a checkmark (\ding{51}), we can write:



\begin{quote}\small
|\renewcommand{\labelitemiii}{\ding{51}}|
\end{quote}

The symbol |\ding{31}| is available by loading the \pkg{pifont}\footcite{pifont} or another package that can be used for such symbols such as \pkg{bbding} or create your own using a suitable font such as \pkg{symbola}
or \pkg{fontawesome}. 

\begin{texexample}{Changing the label symbols}{ex:symbols}
\renewcommand{\labelitemiii}{\ding{51}}
\begin{itemize}
\item This is the first level of the list.
  \begin{itemize}
     \item This is the second level of the list.
     \begin{itemize}
     \item This is the third level of the list.
     \end{itemize}
  \end{itemize}
\item And back to the first level.
\end{itemize}

\end{texexample}

We can use symbols, if we need them as in the following example:

\begin{texexample}{A yes and no list}{ex:yesno}
\newcommand{\Yess}{\ding{51}}
\newcommand{\Noo}{\ding{55}}

\begin{enumerate}
 \item[\Yess] Learn about lists.
 \item[\Noo] Learn about the \tex's output routine.
\end{enumerate}
\end{texexample}

As we can see from the example, we can use the argument of an |item| and it can change an enumerated list into an itemized list, provided we type the argument manually. The |enumi| will not be incremented. 



\section{Generalized lists}

Lists such as those in the three environments itemize, enumerate, and
description can be formed in a quite general way. The type of label and
its width, the depth of indentation, spacings for paragraphs and labels,
and so on, may be wholly or partially set by the user by means of the list
environment:

\bgroup
\trivlist\item 
     \cs{begin}\{list\}\marg{std\_label}\marg{list\_declarations} item list \cs{end}\{list\}
\endtrivlist
\egroup


Here item list consists of the text for the listed entries, each of which
begins with an |\item| command that generates the corresponding label.
The \(stnd\_label\) contains the definition of the label to be produced by the
\cs{item} command when the optional argument is missing.  

The first argument in the list environment defines the |standard label|, that
is, the label that is produced by the \cs{item} command when it appears
without an argument. In the case of an unchanging label, such as for the
itemize environment, this is simply the desired symbol. If this is to be a
mathematical symbol, it must be given as \$symbol name\$, enclosed in \$
signs. For example, to select a right arrow symbol ($\Rightarrow$) as the label, 
the \emph{std\_label} must be defined to be |\$\Rightarrow\$|.



\newcounter{steps}
\setcounter{steps}{0}

\begin{list}{\bfseries\upshape Step \arabic{steps}:}
{%
\usecounter{steps}
\setlength{\labelwidth}{2cm}\setlength{\leftmargin}{2.6cm}
\setlength{\labelsep}{0.5cm}\setlength{\rightmargin}{1cm}
\setlength{\parsep}{0.5ex plus0.2ex minus0.1ex}
\setlength{\itemsep}{0ex plus0.2ex minus0pt}\relax \slshape %
}
\item Melt the butter and dark chocolate
\item Prepare the egg and sugar mix
\item Cool the butter and dark chocolate
\item Set out the milk and white chocolate
\item Prepare the brownie tin

\item[]\ldots
    
\item Fold in the chocolate to the eggy mousse\ldots
\item Add the flour and cocoa\ldots
\item Get the tin in the oven.
\end{list}



\begin{texexample}[listing only]{Generalized Lists}{ex:genlists}
% create a new counter for the list
%\newcounter{steps}
%\setcounter{steps}{0}
Continuing with our recipe\ldots

\makeatletter
\def\usecounter#1{\@nmbrlisttrue\def\@listctr{#1}}


\begin{list}{\bfseries\upshape Step \arabic{steps}:}%
{ 
% this has to be on the first line 
\usecounter{steps}
\setlength{\itemsep}{0ex} 
\setlength{\labelwidth}{2cm}
\setlength{\leftmargin}{2.6cm}
\setlength{\labelsep}{0.5cm}
\setlength{\rightmargin}{1cm}
\setlength{\parsep}{0.5ex plus.2pt}
 }
\item Melt the butter and dark chocolate
\item Prepare the egg and sugar mix
\item Cool the butter and dark chocolate
\item Set out the milk and white chocolate
\item Prepare the brownie tin
      \ldots
\item Fold in the chocolate to the eggy mousse\ldots
\item Add the flour and cocoa\ldots
\item Get the tin in the oven.

% end the list
\end{list}
\makeatother


This brings us to the end of our cooking lessons.
\end{texexample}

This brings us to the next step. 

\makeatletter
\gdef\resume{\def\usecounter##1{\@nmbrlisttrue\def\@listctr{##1}}\relax}
\gdef\reset{\def\usecounter##1{\@nmbrlisttrue\def\@listctr{##1}\setcounter{##1}{0}\relax}}
\makeatother

\resume
\begin{list}{\bfseries\upshape A \arabic{steps}:}
{%
\usecounter{steps}
\setlength{\labelwidth}{2cm}\setlength{\leftmargin}{2.6cm}
\setlength{\labelsep}{0.5cm}\setlength{\rightmargin}{1cm}
\setlength{\parsep}{0.5ex plus0.2ex minus0.1ex}
\setlength{\itemsep}{0ex plus0.2ex minus0pt}\relax \slshape %
}
\item Melt the butter and dark chocolate
\item Prepare the egg and sugar mix
\item Cool the butter and dark chocolate
\item Set out the milk and white chocolate
\item Prepare the brownie tin
\end{list}

\reset
\begin{list}{\bfseries\upshape A \arabic{steps}:}
{%
\usecounter{steps}
\setlength{\labelwidth}{2cm}\setlength{\leftmargin}{2.6cm}
\setlength{\labelsep}{0.5cm}\setlength{\rightmargin}{1cm}
\setlength{\parsep}{0.5ex plus0.2ex minus0.1ex}
\setlength{\itemsep}{0ex plus0.2ex minus0pt}\relax \slshape %
}
\item Melt the butter and dark chocolate
\item Prepare the egg and sugar mix
\item Cool the butter and dark chocolate
\item Set out the milk and white chocolate
\item \lorem
\end{list}


Using either |newenvironment| we can

\emphasize{recipe}
\begin{texcode}{Creating a new named List}{ex:newlist}
\newenvironment{recipe}{\list{\bfseries\upshape Step \arabic{steps}:}
{%
\usecounter{steps}
\setlength{\labelwidth}{2cm}\setlength{\leftmargin}{2.6cm}
\setlength{\labelsep}{0.5cm}\setlength{\rightmargin}{1cm}
\setlength{\parsep}{0.5ex plus0.2ex minus0.1ex}
\setlength{\itemsep}{0ex plus0.2ex minus0pt}\relax \slshape %
}}
{\endlist}

\begin{recipe}
\item Have a nice afternoon\ldots
\end{recipe}
\end{texcode}

\newenvironment{recipe}{\list{\bfseries\upshape Step \arabic{steps}:}
{%
\usecounter{steps}
\setlength{\labelwidth}{2cm}\setlength{\leftmargin}{2.6cm}
\setlength{\labelsep}{0.5cm}\setlength{\rightmargin}{1cm}
\setlength{\parsep}{0.5ex plus0.2ex minus0.1ex}
\setlength{\itemsep}{0ex plus0.2ex minus0pt}\relax \slshape %
}}
{\endlist}

\begin{recipe}
\item Have a nice afternoon\ldots
\end{recipe}

% something gone terribly wrong, need to restore the settings
\makeatletter
%\@restorepar\@nobreakfalse\@nmbrlistfalse
\let\par\@@par
\makeatother


\subsection{trivlist}

The simplest of all lists is |\trivlist|. In simple terms the |trivlist| environment turns each item into a paragraph and thus it is easier to apply formatting information to a list of paragraphs or items if you want to think of them this way. As it carries common information to all lists, it is used to build up more complicated structures. A good use of trivlists is to simplify the writing of table heads and produce semantic table environments. They are also used to define teh spacing around the verbatim environments. A simple author use is shown in Example~\ref{ex:trivlists}.

\begin{texexample}{Trivlists}{ex:trivlists}
\newenvironment{name}
  {\trivlist\item
   \tabular{@{}ll@{}}}
  {\endtabular\endtrivlist}
  
% test it  
\begin{name}
   First    & Mary  \\
   Second   & Jones \\
   Nickname & --- \\
\end{name}
 

But whatever you call the comma, is it right or wrong? There’re fair arguments on both sides.  One might be concerned about limiting ambiguity. Alas, including the Oxford comma can lead to ambiguity, but omitting it can lead to ambiguity as well.  Consider (3) and (4):
\begin{trivlist}
\item[(3a)] I own pictures of my friends, Hugh Grant, and Dolly Parton.
\item[(3b)] I own pictures of my friends, Hugh Grant and Dolly Parton.
\item[] 
\item[(4a)] I am writing to my Congresswoman, Alia Shawkat, and Michael Cera.
\item[(4b)] I am writing to my Congresswoman, Alia Shawkat and Michael Cera.
\end{trivlist}
\end{texexample}
           


The general parameters affecting a general list is shown in the  diagram  below\footnote{Produced using the \texttt{layouts} package.}. LaTeX offers three general list structures, enumerate, itemize and description.
%\begin{figure}[hp]
%\listdiagram
%\caption{Layout of an \texttt{enumerate} list} \label{fig:lstenum}
%\end{figure}

\section{The list geometry}

You can draw a list diagram as shown below, using the function \docAuxCmd{drawlistdiagram} from
the \pkgname{xlayouts} package, which is bundled with the \pkgname{phd} package.


\begin{figure}[htbp]
\bgroup
\centering
\cxset{geometry units=mm}
\drawlistdiagram
\caption[List geoemetry]{\latexe list diagram.}
\egroup
\end{figure}


List in \latexe are shaped using |\parshape|. Sometimes as you change parameters things react unintuitevely. An easy wway to remember is that the parameter |\leftmargin| determines the first line of the indentation and |\linewidth| is the
|hsize-leftmargin-rightmargin|



What is important to notice here is that all the standard list parameters are left essentially unchanged. The only item that is affected is \refCom{makelabel}, which is redefined in \lstinline{description} label.

\section{Packages}


 Most journals develped their own lists and hard-wired them. Current packages are:
 \pkg{enumitem}, \pkg{enumerate}, \pkg{paralist}.



\paragraph{Enumerate} This package gives the \refEnv{enumerate} environment an optional argument which determines
 the style in which the counter is printed. An occurence of one of the tokens 
 \textbf{A a I i } or \textbf{1} produces the value of the counter printed with
 (respectively) \cmd{\Alph} \cmd{\alph} \cmd{\Roman} \cmd{\roman} or \cmd{\arabic}. 
 These letters may be surrounded by any string involving any other \tex expressions, 
 however the tokens \textbf{A a I i 1} must be inside a \{\} group if they are
 not to be taken as special.

\begin{texcode}{Example using the package enumerate}{ex:enum}
 \begin{enumerate}[EX i.]
   \item one one one one one one one
         one one one one\label{LA}
   \item two
      \begin{enumerate}[{example} a)]
        \item one of two one of two
          one of two\label{LB}
        \item two of two
       \end{enumerate}
   \item two of two
 \end{enumerate}
 

 \begin{enumerate}[{A}-1]
 \item one\label{LC}
 \item two
 \end{enumerate}
\end{texcode}

 This package minimally changes the original \latexe definitions. It is very convenient when you
 want now and then to change labels in a document. The central idea behind the |phd| group of
 packages and eponymous classes, is that there is a distinction between the author, template designer and 
 programmer. The packages at the level of the author always use a key value interface, that
 limits the involvement of the author and to a large extend the template designer to that of 
 setting a number of keys. 

\paragraph{Babel} The babel package\footcite{babel} will redefine enumerate for a number of languages such as french and greek. It is best in these cases, if you still need to use a different list to create a new list with one of the other packages such as enumitem or using the phd-lists package.
\bgroup
\selectlanguage{french}
\extrasfrench

\lorem
\begin{frenchenumerate}
\item \lorem
      \lorem
        \begin{frenchenumerate}
         \item Something
         \item \lorem
               
               \lorem
        \end{frenchenumerate}
\item \lorem
\end{frenchenumerate}

\lorem


\egroup
\paragraph{phd-lists} The package, which is described in the next chapter uses a key value approach in setting new lists
and adjusting their parameters.

\begin{texexample}{Example using phd-lists}{ex:phd-lists}
\cxset{enumerate numberingi   = Alpha,
       enumerate numberingii  = alpha,
       enumerate numberingiii = Roman,}

 
 \begin{enumerate}
   \item This is the top level. \lorem
      \begin{enumerate}
        \item This is the second level. \lorem
          \begin{enumerate}
            \item This is the third level. \lorem
          \end{enumerate}. 
      \end{enumerate}
 \end{enumerate}
\end{texexample}

\vfill
\endinput

\section{Creating new description like environments}

The macro \docAuxCmd*{NewDescriptionEnvironment} can be used to redefine new description like environments, using a key value interface.

%\begin{texexample}{Define a new description list environment}{ex:newdesclist}
%% define the orange description environment
%\NewDescriptionEnvironment[description centered]{orangedescription}
%
%% Sample
%The \texttt{orangedescription} environment in action.
%
%\begin{orangedescription}
%
% \item[One] \lorem
% \item[Two] \lorem
% \item[Three] \lorem
%
%\end{orangedescription}
%
%\lorem
%
%\makeatother
%\end{texexample}



\section{Example: redefining a description list}
We will now develop a description environment, that can be useful for the documentation of packages to describe options. We will use a description list as the basis of the environment. We define the following key values.
|\itemindent-\leftmargin|

\section{Enumerated lists}


\begin{enumerate}
\item one
\item two
\item three
\end{enumerate}

Enumerated (numbered) list environments are characterized by numbering. They use a variety of fields and counters as shown in table.

\subsection{Vertical skips}

By default LaTeX adds vertical skips, as shown in figure 1. The definition of these skips is influenced by the font size and are defined in the \texttt{bk10.clo} files, hence hard to find and change. Each level of the list has its own definition as \lstinline{\@listi}.

\bigskip
\tcbset{width=\linewidth,arc=1mm,before=\bigskip,left=8mm}

\begin{teXXX}
\def\@listi{\leftmargin\leftmargini
            \parsep 40pt plus20pt minus0pt
            \topsep 80pt plus20pt minus40pt
            \itemsep40pt plus20pt minus0pt}
\let\@listI\@listi
\@listi

\def\@listii {\leftmargin\leftmarginii
              \labelwidth\leftmarginii
              \advance\labelwidth-\labelsep
              \topsep    40pt plus20pt minus0pt
              \parsep    20pt plus0pt  minus0pt
              \itemsep   \parsep}
\def\@listiii{\leftmargin\leftmarginiii
              \labelwidth\leftmarginiii
              \advance\labelwidth-\labelsep
              \topsep    20pt plus0ptminus0pt
              \parsep    1pt
              \partopsep 0pt plus\z@ minus0pt
              \itemsep   \topsep}
\def\@listiv {\leftmargin\leftmarginiv
              \labelwidth\leftmarginiv
              \advance\labelwidth-\labelsep}
\def\@listv  {\leftmargin\leftmarginv
              \labelwidth\leftmarginv
              \advance\labelwidth-\labelsep}
\def\@listvi {\leftmargin\leftmarginvi
              \labelwidth\leftmarginvi
              \advance\labelwidth-\labelsep}
\end{teXXX}


\begin{texexample}{Compact Styles}{ex:compact2}
\makeatletter
\cxset{enumerate compact/.style={%
  enumerate numberingi=alpha,
  enumerate numberingii=roman,
  enumerate numberingiii=alpha,
  enumerate numberingiv=roman,
  enumerate labeli punctuation=,
  enumerate label left=(,
  enumerate label right=),
  enumerate leftmargini=2.2em,
  enumerate leftmarginii=2.1em,
  enumerate leftmarginiii=1.5em,
  enumerate leftmarginiv=2em,
  listi topsep=8pt plus2pt minus0pt,
  listi itemsep=0pt plus2pt minus0pt,
  listi parsep=0pt plus2pt minus0pt,
  listi parindent=0pt,
  listii parindent=1em,
  listiii parindent=1em,
  listii topsep=0pt plus2pt minus0pt,
  listii itemsep=0pt plus2pt minus0pt,
  listii parsep=0pt plus2pt minus0pt,
  listiii topsep=0pt plus2pt minus0pt,
  listiii itemsep=0pt plus2pt minus0pt,
  listiv parindent=0pt,
  listiv parsep=0pt plus2pt minus0pt,
  listiv parsep=0pt plus2pt minus0pt,
  listiv topsep=0pt plus2pt minus0pt,
  listiv itemsep=0pt plus2pt minus0pt,
  listiv parsep=0pt plus2pt minus0pt,
}}
\makeatother


\begin{enumerate}
\item Does this project actually merit the use of the Minor Works Form or Intermediate Form instead of their `grown up' relatives?
\item Do the number of PC or prime cost items mean that it would be more desirable to use a re-measurable form?
\item Is this a contract which merits the production of full scale bills
of quantities or is something more standardised going to suffice?
\end{enumerate}
\end{texexample}



As you will observe the numbering in the above example has been enclosed in round brackets, using:


\begin{texcode}{Bracketting a numeral}{ex:brackets}
  enumerate label left=(,
  enumerate label right=),
\end{texcode}


The next example is from the \textit{LaTeX Companion}. In example~\ref{ex:companion}, the first-level list elements are decorated with the section sign (\S) as a prefix and a period as a suffix (omitted in references). We will
define this as a style named \textit{paragraphsymbol} for the lack of any better name. This style can sometimes be found in legal texts.

\begin{texexample}{Paragraph symbols in enumerate}{ex:companion}
\cxset{paragraphsymbol/.style={%
  enumerate numberingi=arabic,
  enumerate labeli punctuation=.,
  enumerate label left=\S,
  enumerate label right=,
}}

\setenumerate{paragraphsymbol}
\begin{enumerate}
\item \lorem
\item \lorem
\item \lorem
\end{enumerate}
\end{texexample}


%\section{Creating enumerated environments}
%
%New enumerated environments cab be created by using the macro \lstinline{\newenumeratedenvironment}. Keys are set as either styles or individually.
%
%%% verbatim from latex
%
%
%\begin{texexample}{An enumerated list factory}{ex:listfactory}
%
%
%\makeatletter
%\newenvironment#1#2{
%#2\enumerate}{\endenumerate}
%\makeatother
%
%\newenumeratedenvironment{paragraphsymbol}{
%  enumerate numberingi=alpha,
%  enumerate numberingii=roman,
%  enumerate numberingiii=alpha,
%  enumerate numberingiv=roman,
%  enumerate labeli punctuation=,
%  enumerate label left=(,
%  enumerate label right=),
%  enumerate leftmargini=2.2em,
%  enumerate leftmarginii=2.1em,
%  enumerate leftmarginiii=1.5em,
%  enumerate leftmarginiv=2em,
%  listi topsep=8pt plus2pt minus0pt,
%  listi itemsep=0pt plus2pt minus0pt,
%  listi parsep=0pt plus2pt minus0pt,
%  listi parindent=0pt,
%  listii parindent=1em,
%  listiii parindent=1em,
%  listii topsep=0pt plus2pt minus0pt,
%  listii itemsep=0pt plus2pt minus0pt,
%  listii parsep=0pt plus2pt minus0pt,
%  listiii topsep=0pt plus2pt minus0pt,
%  listiii itemsep=0pt plus2pt minus0pt,
%  listiv parindent=0pt,
%  listiv parsep=0pt plus2pt minus0pt,
%  listiv parsep=0pt plus2pt minus0pt,
%  listiv topsep=0pt plus2pt minus0pt,
%  listiv itemsep=0pt plus2pt minus0pt,
%  listiv parsep=0pt plus2pt minus0pt,
%  enumerate numberingi=alpha,
%  enumerate labeli punctuation=.,
%  enumerate label left={\P},
%  enumerate label right=,
%}
%
%
%\begin{paragraphsymbol}
%\item \lorem
%\item \lorem
%\item \lorem
%\end{paragraphsymbol}
%
%\end{texexample}

\section{The description list environment}

You can use the description list environment as you would normally use it with \latexe.

\begin{docEnv} {description} {}
\end{docEnv}

However, a number of settings are available to modify the styling of the environment. These 
include settings for fonts and color, as well as spacing and margins.

\begin{docKey} {description label font-face} { = \meta{font shape} } {initial, default=inherit}
\end{docKey}

\begin{docKey} {description label font-family} { = \meta{font shape} } {initial, default=inherit}
\end{docKey}

\begin{docKey} {description label font-size} { = \meta{font size} } {initial, default=inherit}
\end{docKey}

\begin{docKey} {description label font-weight} { = \meta{font weight} } {initial, default=inherit}
\end{docKey}

\begin{docKey} {description label font-shape} { = \meta{font shape} } {initial, default=inherit}
\end{docKey}

\begin{docKey} {description label color} { = \meta{color name} } {initial, default=thedescriptionlabelcolor}
  Setts the description label color
\end{docKey}

\begin{docKey} {description label sep} { = \meta{dim} } {initial, default = 1em}
\end{docKey}

\begin{docKey} {description label width} { = \meta{dim} } {initial, default = 1em}
\end{docKey}

\begin{docKey} {description margin left} { = \meta{dim} } {initial, default = 0em}
\end{docKey}

\begin{docKey} {description margin right} { = \meta{dim} } {initial, default = 0em}
\end{docKey}

\begin{docKey} {description item indent} { = \meta{dim} } {initial, default = 0em}
\end{docKey}

Unlike the enumerate and itemize environment, the description list environment is defined in the book class.
The environment is defined as:

\refCom{descriptionlabel} sets the typesetting of the description label.
\section{Itemized lists}

The itemized \LaTeX\ lists are similar to those for the enumerated lists. However they are somehow simpler as there is no need for counters.

\bigskip
\begin{tcolorbox}[width=\linewidth,arc=2mm,title=Default \LaTeX\ parameters for itemized lists]
\begin{lstlisting}
\newcommand\labelitemi{\textbullet}
\newcommand\labelitemii{\normalfont\bfseries \textendash}
\newcommand\labelitemiii{\textasteriskcentered}
\newcommand\labelitemiv{\textperiodcentered}
\end{lstlisting}
\end{tcolorbox}




\cxset{red/.style={
 labelitemi={{\color{green}\ding{'64}}},
 labelitemii=\color{red}\textendash,
 labelitemiii=\textasteriskcentered,
 labelitemiv=\textperiodcentered,
}}

Now that we have managed to abstract the itemized environment we can generate a new environment factory.

\makeatletter
\def\newitemizedenvironment#1#2{
\@itemdepth=0
\expandafter\def\csname#1\endcsname{%
 \cxset{#2}%
 \ifnum \@itemdepth >\thr@@\@toodeep\else
 \advance\@itemdepth\@ne
 \edef\@itemitem{labelitem\romannumeral\the\@itemdepth}%
 \expandafter
 \list
 \csname\@itemitem\endcsname
 {\def\makelabel####1{\hss\llap{####1}}}%
 \fi}
 \expandafter\let\csname end#1\endcsname=\endlist
}
\makeatother

%\newitemizedenvironment{reditemize}{black}
%
%
%\begin{reditemize}
%\item Test.
%   \begin{reditemize}
%    \item test.
%   \end{reditemize}
%\end{reditemize}
%
%\begin{itemize}
%\item Level i
%      \begin{itemize}
%       \item Level ii
%          \begin{itemize}
%            \item Level iii
%              \begin{itemize}
%                \item Level iv. \lipsum*[1]
%              \end{itemize}
%          \end{itemize}
%      \end{itemize}
%\end{itemize}


\section{Itemized lists with ding symbols}

So far we have used both standard symbols as well as those provided by the pifont that offers numerous,
dingbang symbols. The pifont package also offers environments to do that more easily.


\begin{texcode}{dinglist}{ex:dinglists}
\begin{dinglist}{"E4}
\item The first item. \item The second
item in the list.
\end{dinglist}

\end{texcode}

This will give us:

\begin{dinglist}{"E4}
\item The first item. \item The second
item in the list.
\end{dinglist}



%\begin{dingautolist}{'30}
%\item The first item in the list.\label{lst:a}
%\item The second item in the list.\label{lst:b}
%\item The third item in the list.\label{lst:c}
%\item The fourth item in the list.\label{lst:d}
%\end{dingautolist}
%
%\newenvironment{steps}{\dingautolist{'30}}{\enddingautolist}
%
%\begin{steps}
%\item The first item in the list.\label{lst:a}
%\item The second item in the list.\label{lst:b}
%\item The third item in the list.\label{lst:c}
%\item The fourth item in the list.\label{lst:d}
%\end{steps}


\endinput

\def\start@SFBbox{\@next\@currbox\@freelist{}{}%
 \global\setbox\@currbox
 \vbox\bgroup
  \hsize \textwidth
  \@parboxrestore
}
\def\finish@SFBbox{\par\vskip -\dbltextfloatsep
  \egroup
  \global\count\@currbox\tw@
  \global\@dbltopnum\@ne
  \global\@dbltoproom\maxdimen\@addtodblcol
  \global\vsize\@colht
  \global\@colroom\@colht
}

\newif\ifSFB@keywords
\def\keywords{\if@twocolumn
  \start@SFBbox\@keywords
 \else
  \@keywords
 \fi
}
\def\@keywords{\list{}{%
    \labelwidth\z@ \labelsep\z@
    \leftmargin 11pc\rightmargin\z@  % was 11pc\right....
    \parsep 0pt plus 1pt}\item[]\reset@font\large{\bf Key words: }%
}
\def\endkeywords{\if@twocolumn
  \endlist\addvspace{37pt plus 0.5\baselineskip}\finish@SFBbox
 \else
  \endlist
\fi










%\makeatletter

\thispagestyle{plain}

\cxset{image={./images/breakerboys.jpg},
       subsection font-shape= upshape,}

\usemintedstyle{friendly}

\chapter{Listings styles}  

Many users of \latex require to typeset formatted code. There are two packages that
can be used the more conventional \pkgname{listings}\footcite{listings} and \pkgname{minted}\footcite{minted}. The
|minted| package is a more powerful and flexible package than listing, since it uses
an external program |Pygments| which is written in |Python|\footnote{See \protect\url{http://pygments.org/} for more details. You can also review and contribute to the code at \protect\url{https://bitbucket.org/birkenfeld/pygments-main}}. My recommendation to you is
to use the |minted package|. The two packages can happily co-exist and each one has
its own advantages and disantvantages. The listings package has all its color
parameters configurable via its \latex key value settings, whereas the pygments program
has its own way of setting these styles, which are only accessible through \latex
as a set of fixed styles. To create a new color scheme, you will need to write some
simple |python|, register it as a plugin or drop it at the folder holding the styles.\footnote{See documentation at \protect\url{http://pygments.org/docs/styles/}.}

For me
it is the only limiting factor of pygments and which there are ways around it. However, this might be also an advantage
as users are more likely to be familiar with such code coloring schemes in their language.

\section{Using minted}

Since minted makes calls to the outside world (that is, Pygments), you need to
tell the \latex processor about this by passing it the |-shell-escape| option or it
won’t allow such calls. In effect, instead of calling the processor like this:

\usemintedstyle{friendly}
\begin{minted}[fontsize=\footnotesize,style=vim]{bash}
$ latex input
you need to call it like this:
$ latex -shell-escape input
\end{minted}

The same holds for other processors, such as pdf\latex or \xelatex.
You should be aware that using -shell-escape allows \latex to run potentially
arbitrary commands on your system. It is probably best to use -shell-escape
only when you need it, and to use it only with documents from trusted sources.

\subsection{A minimal example}   
 
The minted package is loaded like any other package (with or without options). 
You can then use the \docAuxEnv{minted} environment with the language we want to use
as the first argument. The environment also takes an optional argument where the numerous
settings of the package can be specified.
 
\begin{phdverbatim}[basicstyle=\small\ttfamily]
\documentclass{article}
\usepackage{minted}
\begin{document}
\begin{minted}[fontsize=\footnotesize,style=friendly]{javascript}
if (Meteor.isClient) {
  // This code only runs on the client
  Template.body.helpers({
    tasks: [
      { text: "This is task 1" },
      { text: "This is task 2" },
      { text: "This is task 3" }
    ]
  });
}
\end{minted}
\end{document}
\end{phdverbatim}

This will produce an output as:
\medskip

\begin{minted}[fontsize=\footnotesize,style=friendly]{javascript}
if (Meteor.isClient) {
  // This code only runs on the client
  Template.body.helpers({
    tasks: [
      { text: "This is task 1" },
      { text: "This is task 2" },
      { text: "This is task 3" }
    ]
  });
}
\end{minted}

If we do not need a style, the |style=default| setting will typeset as,

\begin{minted}[fontsize=\footnotesize,style=trac]{javascript}
if (Meteor.isClient) {
  // This code only runs on the client
  Template.body.helpers({
    tasks: [
      { text: "This is task 1" },
      { text: "This is task 2" },
      { text: "This is task 3" }
    ]
  });
}
\end{minted}

You can also set the style for the whole document using:

\begin{minted}[fontsize=\footnotesize,style=trac]{TeX}
\usemintedstyle{<name>}
\end{minted}
where you can get <name> by typing

\begin{minted}[fontsize=\footnotesize,style=bw]{bash}
$ pygmentize -L styles
\end{minted}
at the command prompt/terminal. For example, the minted documentation itself uses the |trac| style.

\begin{minted}[fontsize=\footnotesize,style=friendly]{html}
<!-- First set the doctype -->
<!DOCTYPE html>
    <html>
      <head>
        <title>Canvas</title>
        <meta charset="UTF-8" />
        <style>
          #square {
            border: 1px solid black;
                    transform: scale(10) rotate(3deg) translateX(0px);
                    -moz-transform: scale(10) rotate(3deg) translateX(0px);
          }

          .box {              
                    transition-duration: 2s;
                    transition-property: transform;
                    transition-timing-function: linear;
          }
        </style>
      </head>
      <body>
        <canvas id="square" width="200" height="200"></canvas>
        <script>
                var canvas = document.createElement('canvas');
                canvas.width = 200;
                canvas.height = 200;

                var image = new Image();
                image.src = 'images/card.png';
                image.width = 114;
                image.height = 158;
                image.onload = window.setInterval(function() {
                    rotation();
                }, 1000/60);
       </script>
      </body>
    </html>
\end{minted} 
   



\begin{lstlisting}
int main() {
printf("hello, world");V\colorbox{green}{**}V
return 0;
}
\end{lstlisting}



\chapter{Documentation Macros}


\section{Documentation macros}

When developing this package the need arose to define a number of documentation macros. I~have used heavily macros and ideas present in the \pkg{doc} package, \pkg{pgf} documentation, \pkg{biblatex} documentation  and \pkg{tcolorbox} and for which I am grateful to their respective authors. The major change was to adopt the macros to use different fonts and colors and to use these from a list of key values defined at document level. More about this later. General package user documentation as opposed to package documentation that can be achieved using the |doc/docstrip| system requires that macros and environments be developed for the following:

\begin{enumerate}
\item Macros for command documentation.
\item Environments for commands and options.
\item Latex examples that need to be executed within the document as well as described.
\end{enumerate}


\section{Commands and Styles for Documenting macros}

The most commonly used commands for documenting macros are |\cs|, |\cmd|, |\meta|, |\marg|, |\oarg|. These commands have been defined by many authors and perhaps the best implementation can be found in the \pkg{doc}. Many package authors have redefined them in their documention, some if just to add a bit of colour, others to have them add the command to an index. As we also had a target to allow for
the package to be used in both normal documents as well as documentation
of packages and classes that use the \pkg{doc} and \pkg{docstrip} combination we provided many compatible macros.

\begin{environment}{macro} The environment macro is made available in this
package. 
\end{environment}

\DescribeMacro{macro} The environment macro is made available in this package. 

\begin{macro}{\cmd} The command \cmd{\cmd} typesets its argument in
  verbatim. Typing |\cmd{\cmd}| typsets \cmd{\cmd}. If the class
  |ltxdoc| is loaded the command is defined there. We have modified
  it to accept a colour and changes to the verbatim font 
  for consistency.
\end{macro}

\begin{macro}{meta}
The macro \cs{meta} is normally used to build other commands. On its own it can be used to typeset
examples of the argument of macros, typing |meta{Aristotle}| will typeset meta{Aristotle}. The command provides a hook to set the font via a macro |\meta@font@select|. 
\end{macro}


|\def\meta@font@select{\upshape\color{black}}|


\subsection{Color management}
One of the first requirements for redefining some of the standard doc commands is the need to use color easily, hence we will try and define a certain amount of keys for colors.

Just a bit of a refresher, to define colors we use, either the \cs{definecolor} or the \cs{colorlet} commands.

\emphasis{definecolor,colorlet}

\begin{minted}[fontsize=\small]{TeX}
\definecolor{Hyperlink}{rgb}{0.281,0.275,0.485}
\colorlet{thehyperlink}{theblue}
\end{minted}


We use a semantic approach, where the colors are first defined with a mnemonic command such as {\bfseries\textcolor{theblue}{theblue}} and then we define a semantic command such as the\cs{option} that lets the color to the option command. This sort of double entry has proved useful in navigating through the dozen of the commands that I needed for this documentation.


\subsection{Semantic color names}
\begin{marglist}
\item [\option{theoption}] Coloring of options in margin lists.
\item [\option{themacro}] Coloring of command macros \cs{foo}.
\item [\option{hyperlink}] If we use the \texttt{hyperref} package a number of colors need to be defined for links.
\end{marglist}

\subsection{Named colors}
Standard colors that we provide are:
\begin{marglist}
\item [\textcolor{theblue}{theblue}] This color is used mainly for options.
\item [\textcolor{thered}{thered}] The color mostly used for macro commands and keys.
\item [\textcolor{thegreen}{thegreen}] used for environments.
\item [\textcolor{thelightgreen}{thelightgreen}] Used for margin lists.
\item [\textcolor{thegray}{thegray}] Used as a background to the listings.
\item [\colorbox{thegrey}{\color{white}thegrey}] Alias for the gray to satisfy both sides of the Atlantic and as I sometimes don't remeber which is which.
\item [\colorbox{theshade}{theshade}] Another slightly lighter shade.
\end{marglist}



\begin{marglist}
\item [\cs{cs}] \cs{cs} text Prints a command.
\item [\cs{cmd}] Prints a command.
\end{marglist}




\section{Lists for documentation}



The environment \env{marglist}
\begin{marglist}
\item[testing]\lorem
\item [test]\lorem
\end{marglist}

\env{keymarginlist}This environment is suitable for listing keys, set-in the margin.

\begin{keymarglist}
\item[bibliography] The term <bibliography>, also available as \cmd{\bibname}.
\item[references] The term <references>, also available as \cmd{refname}.
\item[shorthands] The term <list of shorthands> or <list of abbreviations>, also available as \cmd{losname}.
\end{keymarglist}


\env{argumentlist} This environment is suitable for listing macro arguments and their explanations.



\section{Breakable Boxes}

The \pkg{mdframed} as well as the newer versions of \pkg{tcolorbox}
offer breakable boxes.


\begin{tcolorbox}[enhanced, breakable,
  colback=blue!5!white,colframe=blue!75!black,title=Breakable box,
  watermark color=white,watermark text=\Roman{tcbbreakpart}]
  \lipsum[1-3]
\end{tcolorbox}

\section{PGF Style Code Boxes}

\begin{codeexample}[]
\begin{tikzpicture}
  \node[place,label=above:$p_1$,tokens=2]        (p1) {};
  \node[place,label=below:$p_2\ge1$,right=of p1] (p2) {};
\end{tikzpicture}
\end{codeexample}






%\input{tikztutorial.tex}
%\chapter{Bibliography Management} 

\begin{figure}[p]
\includegraphics[width=\textwidth]{./images/ammar.jpg}
\caption{Wilson, Digital Collage, L. Ammar \protect\url{http://daliahammar.com/post/49217473452/wilson-digital-collage}}
\end{figure}
 
\precis{In this chapter we outline a number of experimental keys that been defined to handle Table of Contents (ToC) formatting. These keys are currently experimental.}
\addtocimage{-12pt}{-20pt}{./images/tocblock-man-02.jpg}

       
\def\bibtex{\texttt{bibTeX\xspace}}

For any academic/research writing, incorporating references into a document is an important task. Fortunately, \latex provides  a variety of features that make dealing with references much simpler, including built-in support for citing references. However, a much more powerful and flexible solution is achieved thanks to an auxiliary tool called \bibtex and if your \latex  distribution does not include it is obtainable from \url{http://www.bibtex.org}.


The style of this book places all citations to the side margin. For example, the command  \verb+\cite{Abrahams2003}+, will produce the citation \cite{Abrahams2003}. I find this type of style (suggested by \cite{Tufte1997}) more clear and relevant.

Notes in text for many centuries, before printed books were a common feature. The author picking up a different thread and not wishing to divert immediate attention away from the main body of his work. With printing, the costs of books were high and printers started placing citations and footnotes at the bottom of the page. You are not limited though to use only this style, by using |\cite{Bringhurst2005}|, \citet{Bringhurst2005}.

You can also use, the following code to get a within the text full citation:


\bibentry{Bringhurst2005}



\bibtex provides for the storage of all references in an external, flat-file database. This database can be linked to any \latex document, and citations made to any reference that is contained within the file. This is often more convenient than embedding them at the end of every document written. There is now a centralized bibliography source that can be linked to as many documents as desired (write once, read many!). 

Of course, bibliographies can be split over as many files as one wishes, so there can be a file containing references concerning General Relativity and another about Quantum Mechanics. When writing about Quantum Gravity (QG), which tries to bridge the gap between these two theories, both of these files can be linked into the document, in addition to references specific to QG.

\section{Citations}

To actually cite a given document is very easy. Go to the point where you want the citation to appear, and use the following: cite cite key, where the cite key is that of the bibitem you wish to cite. When LaTeX processes the document, the citation will be cross-referenced with the bibitems and replaced with the appropriate number citation. The advantage here, once again, is that LaTeX looks after the numbering for you. If it were totally manual, then adding or removing a reference would be a real chore, as you would have to re-number all the citations by hand.

Instead of WYSIWYG editors, typesetting systems like TeX or LaTeX \citep{lamport2004} can be used. cite{Abut1990}

\section{Referring to specific pages}

Sometimes you want to refer to a certain page, figure or theorem in a text book. For that you can use the arguments to the 

\begin{texexample}{Citation Example}{}
\cs{cite} command:
\cite[p. 215]{Mittelbach2004}
\end{texexample}

The argument, "p. 215", will show up inside the same brackets

\section{BibTeX}

I have previously introduced the idea of embedding references at the end of the document, and then using the \cs{cite} command to cite them within the text. In this tutorial, I want to do a little better than this method, as it's not as flexible as it could be. Which is why I wish to concentrate on using BibTeX.

A BibTeX database is stored as a .bib file. It is a plain text file, and so can be viewed and edited easily. The structure of the file is also quite simple. An example of a BibTeX entry:

\begin{verbatim}
@article{greenwade93,
    author  = "George D. Greenwade",
    title   = "The {C}omprehensive {T}ex {A}rchive {N}etwork ({CTAN})",
    year    = "1993",
    journal = "TUGBoat",
    volume  = "14",
    number  = "3",
    pages   = "342--351"
}
\end{verbatim}

Each entry begins with the declaration of the reference type, in the form of @type. BibTeX knows of practically all types you can think of, common ones are: book, article, and for papers presented at conferences, there is inproceedings. In this example, I have referred to an article within a journal.\sidenote{\obeylines 
book,
article,
conference
}

After the type, you must have a left curly brace '\{' to signify the beginning of the reference attributes. The first one follows immediately after the brace, which is the citation key. This key must be unique for all entries in your bibliography. It is this identifier that you will use within your document to cross-reference it to this entry. It is up to you as to how you wish to label each reference, but there is a loose standard in which you use the author's surname, followed by the year of publication. This is the scheme that I use in this tutorial.

Next, it should be clear that what follows are the relevant fields and data for that particular reference. The field names on the left are BibTeX keywords. They are followed by an equals sign (=) where the value for that field is then placed. BibTeX expects you to explicitly label the beginning and end of each value. I personally use quotation marks ("), however, you also have the option of using curly braces \verb+('{', '}')+. But as you will soon see, curly braces have other roles, within attributes, so I prefer not to use them for this job as they can get more confusing. 

A notable exception is when you want to use characters with umlauts (ü, ö, etc), since their notation is in the format \verb+\"{o}+, and the quotation mark will close the one opening the field, causing an error in the parsing of the reference.

Remember that each attribute must be followed by a comma to delimit one from another. You do not need to add a comma to the last attribute, since the closing brace will tell BibTeX that there are no more attributes for this entry, although you won't get an error if you do.

It can take a while to learn what the reference types are, and what fields each type has available (and which ones are required or optional, etc). So, look at this entry type reference and also this field reference for descriptions of all the fields. It may be worth bookmarking or printing these pages so that they are easily at hand when you need them.

\section{Authors}

BibTeX can be quite clever with names of authors. It can accept names in forename surname or surname, forename. I personally use the former, but remember that the order you input them (or any data within an entry for that matter) is customizable and so you can get BibTeX to manipulate the input and then output it however you like. If you use the forename surname method, then you must be careful with a few special names, where there are compound surnames, for example "John von Neumann". In this form, BibTeX assumes that the last word is the surname, and everything before is the forename, plus any middle names. You must therefore manually tell BibTeX to keep the 'von' and 'Neumann' together. This is achieved easily using curly braces. So the final result would be "John {von Neumann}". This is easily avoided with the surname, forename, since you have a comma to separate the surname from the forename.

Secondly, there is the issue of how to tell BibTeX when a reference has more than one author. This is very simply done by putting the keyword |and| in between every author. As we can see from another example:


\section{The natbib package}

Using the standard \latex bibliography support, you will see that each reference is numbered and each citation corresponds to the numbers. The numeric style of citation is quite common in scientific writing. In other disciplines, the author-year style, e.g., (Roberts, 2003), such as Harvard is preferred, and is in fact becoming increasingly common within scientific publications. A discussion about which is best will not occur here, but a possible way to get such an output is by the natbib package. In fact, it can supersede LaTeX's own citation commands, as |natbib| allows the user to easily switch between Harvard or numeric \docpkg{natbib}\citep{natbib2009}.


The first job is to add the following to your preamble:

\begin{verbatim}
\usepackage{natbib}
\end{verbatim}


The bibliography |.bib| file is still typed using the normal format as for example:---

\begin{verbatim}
@book{goossens93,
    author    = "Michel Goossens and Frank Mittlebach and Alexander Samarin",
    title     = "The LaTeX Companion",
    year      = "1993",
    publisher = "Addison-Wesley",
    address   = "Reading, Massachusetts"
}
\end{verbatim}



Also, you need to change the bibliography style file to be used, so edit the appropriate line at the bottom of the file so that it reads: |\bibliographystyle{plainnat}|. Once done, it is basically a matter of altering the existing \texttt{cite} commands to display the type of citation you want.


The main commands simply add a (t)  for 'textual' or (p) for 'parenthesized', to the basic \cs{cite} command. You will also notice how Natbib by default will compress references with three or more authors to the more concise 1st surname et al version. By adding an asterisk (*), you can override this default and list all authors associated with that citation. There are some other less common commands that Natbib supports, listed in the table here.

Using |natbib|, can satisfy every style required by a stern and difficult editor.

\begin{table}
\begin{tabular}{ll}
\toprule
Citation command	&Output\\
\midrule
\verb+ \citet{goossens93}+	&\citep{goossens93}\\
\verb+ \citep{goossens93}+	&\citep{goossens93}\\
\verb+ \citet*{goossens93}+	&\citet*{goossens93}\\
\verb+ \citep*{goossens93}+	&\citep*{goossens93}\\
\verb+ \citeauthor{goossens93}+	&\citeauthor{goossens93} \\
\verb+ \citeauthor*{goossens93}+	&\citeauthor*{goossens93}\\
\verb+ \citeyear{goossens93}+	&\citeyear{goossens93}\\
\verb+ \citeyearpar{goossens93}+	&\citeyearpar{goossens93}\\
\verb+ \citealt{goossens93}+	&\citealt{goossens93}\\
\verb+ \citealp{goossens93}+	&\citealp{goossens93}\\
\bottomrule
\end{tabular}
\caption{Natbib package commands}
\end{table}

When changing the bibliography style, sometimes natbib is upset because it can't interpret the data correctly.

In any case, after changing the argument to |\bibliographystyle| a run of LaTeX and one of BibTeX are necessary to get back in sync. Removing the |.bbl| and |.aux| files before those run is recommended, in order to avoid spurious error messages that might corrupt the .aux file currently being generated.\footnote{\url{http://tex.stackexchange.com/questions/54480/package-natbib-error-bibliography-not-compatible-with-author-year-citations}}

\section{Including URLs in bibliography}

As you can see, there is no field for URLs. One possibility is to include Internet addresses in howpublished field of @misc or note field of |@techreport|, |@article|,|@book|:

\begin{lstlisting}[language={[common]TeX},% 
                           alsolanguage={[LaTeX]TeX},% 
                           alsolanguage={[primitive]TeX},%
                           ]
howpublished = "\url{http://www.example.com}"
\end{lstlisting}

Note the usage of \cs{url} command to ensure proper appearance of URLs.
Another way is to use special field url and make bibliography style recognise it.

\begin{lstlisting}[language={[common]TeX},% 
                           alsolanguage={[LaTeX]TeX},% 
                           alsolanguage={[primitive]TeX},%
                           ]
URL = "http://www.example.com"
\end{lstlisting}

You need to use \texttt{usepackage{url}} in the first case or \texttt{usepackage{hyperref}} in the second case.
Styles provided by Natbib (see below) handle this field, other styles can be modified using |urlbst| program. Modifications of three standard styles (|plain|, |abbrv| and |alpha|) are provided with |urlbst|.

If you need more help about URLs in bibliography, visit FAQ of UK List of TeX.


\section{changing punctuation}

When I started using natbib I kept getting square barackets. Use
\begin{lstlisting}[language={[common]TeX},% 
                           alsolanguage={[LaTeX]TeX},% 
                           alsolanguage={[primitive]TeX},%
                           ]
    \bibpunct{(}{)}{;}{a}{,}{,}
    \bibliographystyle{plainnat}
\end{lstlisting}

\section{Error Checking}

You can check the file for errors by runing it through |bibTeX|. This will point database errors etc. 


\subsection{Entry Types}

Bibliography entries included in a .bib file are split by types. The following types are understood by virtually all |BibTeX| styles:

\subsubsection*{article}
  An article from a journal or magazine.

  Required fields: author, title, journal, year

  Optional fields: volume, number, pages, month, note, key

\emph{book}
   A book with an explicit publisher.
   Required fields: author/editor, title, publisher, year
   Optional fields: volume, series, address, edition, month, note, key

\emph{booklet}
   A work that is printed and bound, but without a named publisher or sponsoring institution.
   Required fields: title
   Optional fields: author, howpublished, address, month, year, note, key

\emph{conference}
   The same as inproceedings, included for Scribe compatibility.
   Required fields: author, title, booktitle, year
   Optional fields: editor, pages, organization, publisher, address, month, note, key

\emph{inbook}

    A part of a book, usually untitled. May be a chapter (or section or whatever) and/or a range of pages.
    Required fields: author/editor, title, chapter/pages, publisher, year
    Optional fields: volume, series, address, edition, month, note, key

\emph{incollection}

    A part of a book having its own title.
    Required fields: author, title, booktitle, year
    Optional fields: editor, pages, organization, publisher, address, month, note, key

\emph{inproceedings}

An article in a conference proceedings.
Required fields: author, title, booktitle, year
Optional fields: editor, series, pages, organization, publisher, address, month, note, key

\emph{manual}

Technical documentation.
Required fields: title
Optional fields: author, organization, address, edition, month, year, note, key

\emph{mastersthesis}

A Master's thesis.
Required fields: author, title, school, year
Optional fields: address, month, note, key

\emph{misc}

For use when nothing else fits.

Required fields: none
Optional fields: author, title, howpublished, month, year, note, key

\emph{phdthesis}

A Ph.D. thesis.

Required fields: |author|, |title|, |school|, |year|\\
Optional fields: |address|, |month|, |note|, |key|
proceedings
The proceedings of a conference.
Required fields: title, year
Optional fields: editor, publisher, organization, address, month, note, key
techreport
A report published by a school or other institution, usually numbered within a series.
Required fields: author, title, institution, year
Optional fields: type, number, address, month, note, key
unpublished
A document having an author and title, but not formally published.
Required fields: author, title, note
Optional fields: month, year, key

\section{The bibentry package}

 This package allows one to be able to place bibliographic entries anywhere
 in the text. It is to be used to produce annotated bibliographies, such as
 \begin{quote}
   For an intoduction to this topic, see Jones, J.~R., Basics on this topic,
   {\it J.\ Last Resorts}, \textbf{13}, 234--254, 1994. For more advanced
   information, see \dots.
 \end{quote}

 The idea is that the full reference is used, not just the citation Jones
 [1994].

 \section{Invoking the Package}
 The macros in this package are included in the main document
 with the |\usepackage| command of \LaTeXe,
 \begin{quote}
 |\documentclass[..]{...}|\\
 |\usepackage{|\texttt{\filename}|}|
 \end{quote}

 \section{Usage}

 \newcommand\btx{\textsc{Bib}\TeX}
 This package must be used with \btx, not with a hand-written
 \texttt{thebibliography} environment.

 More precisely, there must be a \texttt{.bbl} file external to the \LaTeX\
 file; whether this is written by hand or by BibTeX is unimportant.

| \nobibliography|
 The bibliography entries are stored with the command
 |\nobibliography| |\marg{bibfiles}|, which is like the usual
 |\bibliography| |\marg{bibfiles}| except no bibliography is printed. The
 \texttt{.bbl} file is read in as usual but the \texttt{thebibliography} is
 redefined so that all the entries are stored, not printed.


 The text of the entries may be printed with the command
 \begin{quote}
    |\bibentry| |\marg{key}|
 \end{quote}

 These commands may only be issued after |\nobibliography|, for otherwise
 the reference texts are not known.

 The final period of the original text will be missing, so that one can add
 punctuation as one pleases.

 Regular |\cite| (or the \texttt{natbib} versions) may be issued anywhere as
 usual.

|\nobibliography*|
 If a regular list of references is to be given too, with the
 |\bibliography|\sidenote{bibfiles} command, issue the starred version
 |\nobibliography*| (without argument) in order to store the bib entry texts.
 This will load the same \texttt{.bbl} file as |\bibliography|, but will avoid
 messages from BibTeX about multiple |\bibdata| commands and warnings from
 \LaTeX\ about multiply defined citations.

 The processing procedure is as usual:
 \begin{enumerate}
  \item \LaTeX\ the file;
  \item Run \btx;
  \item \LaTeX\ the file twice.
 \end{enumerate}

 \noindent
 \textbf{Note:} it is highly recommended to make use of the \docpkg{url}
 package, which will nicely format both |url| and |doi| addresses; in particular,
 they will break at convenient locations without a hyphen.\index{bibliography>doi}
\index{bibliography>url}




Here are some useful references about \LaTeX. They are
available in every worthy bookshop. Many other good documentations
might be found on the web (the FAQ of \textsf{comp.text.tex} for
instance).


\begin{verbatim}
\bibitem[GMS93]{companion} Michel Goossens, Franck Mittelbach and Alexander
Samarin, \emph{The \LaTeX{} Companion}, Addison Wesley, 1993.
\bibitem[Lam97]{lamport} Leslie Lamport, \emph{\LaTeX: A Document Preparation
System}, Addison Wesley, 1997.
\end{verbatim}

This is the main matter of the document, mentioning
[\ref{doc1}] and [\ref{doc2}], for instance.




% Index
\printindex


\end{document}
https://docs.google.com/open?id=0B75n0wtctw00RDhYSkIwOTBRRHU5Rl93Q19lalRlQQ