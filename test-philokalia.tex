\documentclass{book}
\usepackage{phd}
\usepackage{philokalia}

\begin{document}


{\newfontfamily\plk{Philokalia-Regular}
\plk
\newfontfamily\PHtitl[Script=Greek,RawFeature=+titl;grek]{Philokalia-Regular}
 %\font\PHtitl="[Philokalia-Regular]/ICU:script=grek,+titl"
 \newsavebox{\mybox}\savebox{\mybox}{\PHtitl Π}
 \lettrine[lines=3]{\usebox{\mybox}}{ερὶ} ποιητικῆς αὐτῆς τε καὶ τῶν εἰδῶν αὐτῆς, ἥν τινα δύναμιν ἕκαστον ἔχει, 
καὶ πῶς δεῖ συνίστασθαι τοὺς μύθους  εἰ μέλλει καλῶς ἕξειν ἡ ποίησις, ἔτι δὲ ἐκ πόσων καὶ ποίων 
ἐστὶ μορίων, ὁμοίως δὲ καὶ περὶ τῶν ἄλλων ὅσα τῆς αὐτῆς ἐστι μεθόδου, λέγωμεν ἀρξάμενοι κατὰ φύσιν 
πρῶτον ἀπὸ τῶν πρώτων.
 
Ἐποποιία δὴ καὶ ἡ τῆς τραγῳδίας ποίησις ἔτι δὲ κωμῳδία καὶ ἡ διθυραμβοποιητικὴ καὶ τῆς αὐλητικῆς 
ἡ πλείστη καὶ κιθαριστικῆς πᾶσαι τυγχάνουσιν οὖσαι μιμήσεις τὸ σύνολον· διαφέρουσι δὲ ἀλλήλων τρισίν, 
ἢ γὰρ τῷ ἐν ἑτέροις μιμεῖσθαι ἢ τῷ ἕτερα ἢ τῷ ἑτέρως καὶ μὴ τὸν αὐτὸν τρόπον. 

Ὥσπερ γὰρ καὶ χρώμασι καὶ σχήμασι πολλὰ μιμοῦνταί τινες ἀπεικάζοντες (οἱ μὲν [20] διὰ τέχνης οἱ δὲ διὰ συνηθείας),
ἕτεροι δὲ διὰ τῆς φωνῆς, οὕτω κἀν ταῖς εἰρημέναις τέχναις ἅπασαι μὲν ποιοῦνται τὴν μίμησιν ἐν ῥυθμῷ καὶ λόγῳ καὶ
ἁρμονίᾳ, τούτοις δ᾽ ἢ χωρὶς ἢ μεμιγμένοις· οἷον ἁρμονίᾳ μὲν καὶ ῥυθμῷ χρώμεναι μόνον ἥ τε αὐλητικὴ καὶ ἡ κιθαριστικὴ
κἂν εἴ τινες [25] ἕτεραι τυγχάνωσιν οὖσαι τοιαῦται τὴν δύναμιν, οἷον ἡ τῶν συρίγγων, αὐτῷ δὲ τῷ ῥυθμῷ [μιμοῦνται]
χωρὶς ἁρμονίας ἡ τῶν ὀρχηστῶν (καὶ γὰρ οὗτοι διὰ τῶν σχηματιζομένων ῥυθμῶν μιμοῦνται καὶ ἤθη καὶ πάθη καὶ πράξεις)· 
 }
\end{document}