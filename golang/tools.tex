\chapter{Tools}

\section{gofmt}

Some of Go's concept can be considered controversial. One such tool is |gofmt|. The Go project has 
a canonical program format, implemented by a program that will pick up any program and reformat it.
Every program in the Go source tree has been formatted with |gofmt|. The code review and revision control 
tools check that any new code is gofmt-formatted too. 

The most obvious benefit of using |gofmt| is that when you open an unfamiliar Go program, your brain doesn't get distracted, even subconsciously, about why that brace is in the wrong place; you can focus on the code, not the formatting. But there are many more interesting uses for gofmt. Gofmt can take any file in the Go source tree, parse it into an internal representation, and then write exactly the same bytes back out to the file. There are two reasons this works: the primary reason is that a lot of effort went into gofmt, but an important secondary reason is that gofmt only has to worry about one formatting convention, and we've agreed to accept that as the official one. There is an interesting blog post by Russ Cox at \href{http://research.swtch.com/gofmt}{Gofmt}, 
discussing the tool.

\begin{teX}
gofmt
from
for{
fmt.Println("    I feel pretty." );
       }
\end{teX}
       
to

\begin{teX}
for {
    fmt.Println("I feel pretty.")
}
\end{teX}

One final note. Being able to pick up a program and write it back out, preserving comments, is not an easy task in any language. Robert Griesemer deserves all the credit for the huge effort to make |gofmt| handle real programs so well.

\section{godoc}


\footnote{\protect\url{https://talks.golang.org/2014/hammers}}

