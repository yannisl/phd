\newcommand\gotype[1]{\index{go types>#1}{\color{bgsexy}\texttt{#1}}}

\chapter{Types}

\section{Numbers}

Numbers in Go are split into integers and floating-point numbers.

\subsection{Integers}

Go's integer types are: \gotype{uint8}, \gotype{uint16}, \gotype{uint32}, \gotype{uint64}, \gotype{int8}, \gotype{int16}, \gotype{int32} and \gotype{int64}. 8, 16, 32 and 64 tell us how many bits each of the types use. uint means “unsigned integer” while int means “signed integer”. Unsigned integers only contain positive numbers (or zero). In addition there two alias types: byte which is the same as \gotype{uint8} and rune which is the same as int32. Bytes are an extremely common unit of measurement used on computers (1 byte = 8 bits, 1024 bytes = 1 kilobyte, 1024 kilobytes = 1 megabyte, …) and therefore Go's byte data type is often used in the definition of other types. There are also 3 machine dependent integer types: \gotype{uint}, \gotype{int} and \gotype{uintptr}. They are machine dependent because their size depends on the type of architecture you are using.

Generally if you are working with integers you should just use the int type.

\subsection{Floating Point Numbers}

Floating point numbers are numbers that contain a decimal component (real numbers). (1.234, 123.4, 0.00001234, 12340000) Their actual representation on a computer is fairly complicated and not really necessary in order to know how to use them. So for now we need only keep the following in mind:

Floating point numbers are inexact. Occasionally it is not possible to represent a number. For example computing 1.01 - 0.99 results in 0.020000000000000018 – A number extremely close to what we would expect, but not exactly the same.

Like integers floating point numbers have a certain size (32 bit or 64 bit). Using a larger sized floating point number increases it's precision. (how many digits it can represent)

In addition to numbers there are several other values which can be represented: “not a number” (|NaN|, for things like |0/0|) and positive and negative infinity. (+∞ and −∞)

Go has two floating point types: \gotype{float32} and \gotype{float64} (also often referred to as single precision and double precision respectively) as well as two additional types for representing complex numbers (numbers with imaginary parts): \gotype{complex64} and \gotype{complex128}. 

\begin{minted}{go}
package main

import "fmt"

func main() {
  fmt.Println("1 + 1 =", 1 + 1)
}
\end{minted}

\section{Strings}

Before we start discussing strings it will be instructive to start with some basics.

In Go a string is in effect a read-only slice of bytes. It is important to emphasize
that a string holds \textit{arbitrary} bytes. It is not required to hold Unicode text,
UTF-8 text, or any other predefined format. As far as the content of a string is concerned,
it is exactly equivalent to a slice of bytes.

Here is a string literal (more about those soon) that uses the |\xNN| notation to define a string constant holding some peculiar byte values. (Of course, bytes range from hexadecimal values 00 through FF, inclusive.)

\begin{lstlisting}
    const sample = "\xbd\xb2\x3d\xbc\x20\xe2\x8c\x98"
\end{lstlisting}

\subsection{Printing Strings}

\subsection{UTF-8 and string literals}

Now as we saw ain the above example, when we index a string we get bytes, not characters:a string
is just a bunch of bytes. This means that when we store a string, we store its byte-at-a-time representation. Let us look at a more controlled example to see hpow that happens.

Here's a simple program that prints a string constant with a single character three different ways, once as a plain string, once as an ASCII-only quoted string, and once as individual bytes in hexadecimal. To avoid any confusion, we create a \enquote{raw string}, enclosed by back quotes, so it can contain only literal text. (Regular strings, enclosed by double quotes, can contain escape sequences as we showed above.)

\begin{minted}[escapeinside=VV,fontfamily=tt]{go}
func main() {
    const placeOfInterest = `⌘`
    fmt.Printf("plain string: ")
    fmt.Printf("%s", placeOfInterest)
    fmt.Printf("\n")

    fmt.Printf("quoted string: ")
    fmt.Printf("%+q", placeOfInterest)
    fmt.Printf("\n")

    fmt.Printf("hex bytes: ")
    for i := 0; i < len(placeOfInterest); i++ {
        fmt.Printf("%x ", placeOfInterest[i])
    }
    fmt.Printf("\n")
}
\end{minted}
\hbox to \linewidth {\hfill \uline{\href{http://play.golang.org/p/WLzmLpkZHK}{playground}}\hbox to 1cm{}}


\noindent{\panunicode
plain string: ⌘\\
quoted string: "|\u2318|"\\
hex bytes: e2 8c 98
}

\noindent which reminds us that the Unicode character value U+2318, the "Place of Interest" symbol {\panunicode ⌘}, is represented by the bytes e2 8c 98, and that those bytes are the |UTF-8| encoding of the hexadecimal value 2318.

It may be obvious or it may be subtle, depending on your familiarity with UTF-8, but it's worth taking a moment to explain how the UTF-8 representation of the string was created. The simple fact is: it was created when the source code was written.

Source code in Go is defined to be UTF-8 text; no other representation is allowed. That implies that when, in the source code, we write the text

`⌘`
the text editor used to create the program places the UTF-8 encoding of the symbol ⌘ into the source text. When we print out the hexadecimal bytes, we're just dumping the data the editor placed in the file.

In short, Go source code is UTF-8, so the source code for the string literal is UTF-8 text. If that string literal contains no escape sequences, which a raw string cannot, the constructed string will hold exactly the source text between the quotes. Thus by definition and by construction the raw string will always contain a valid UTF-8 representation of its contents. Similarly, unless it contains UTF-8-breaking escapes like those from the previous section, a regular string literal will also always contain valid UTF-8.

Some people think Go strings are always UTF-8, but they are not: only string literals are UTF-8. As we showed in the previous section, string values can contain arbitrary bytes; as we showed in this one, string literals always contain UTF-8 text as long as they have no byte-level escapes.

To summarize, strings can contain arbitrary bytes, but when constructed from string literals, those bytes are (almost always) UTF-8.

\subsection{Code points, characters, and runes}



\subsection{Libraries}

Go's standard library provides strong support for UTF-8 text. If a for range loop isn't sufficient for your purposes, chances are the facility you need is provided by a package in the library.

The most important such package is \pkgname{unicode/utf8}, which contains helper routines to validate, disassemble, and reassemble UTF-8 strings. Here is a program equivalent to the for range example above, but using the |DecodeRuneInString| function from that package to do the work. The return values from the function are the rune and its width in |UTF-8-|encoded bytes.

{\panunicode
U+65E5 '日' starts at byte position 0\\
U+672C '本' starts at byte position 3\\
U+8A9E '語' starts at byte position 6\\
}


\subsection{strings package}


\subsubsection{func Title}
\begin{lstlisting}
package main

import (
	"fmt"
	"strings"
)

func main() {
	fmt.Println(strings.Title("the chapter title a.b.c!"))
}

(*@\uline{\href{http://play.golang.org/p/Z3TqF_q4zY}{try in playground}}@*)
\end{lstlisting}

\subsubsection*{funct join}


\def\playground#1{\uline{\href{#1}{playground}}}
\begin{lstlisting}
package main

import (
	"fmt"
	"strings"
)

func main() {
	s := []string{"foo", "bar", "baz"}
	fmt.Println(strings.Join(s, "_"))
}
(*@\playground{http://play.golang.org/p/Ioeopc7I3r}@*)
\end{lstlisting}


\subsection{Checking if a string is empty}

\begin{lstlisting}
package main

import (
	"fmt"
	"strings"
)

func main() {
	stringOne := "merpflakes"
	stringTwo := "   "
	stringThree := ""

	if len(strings.TrimSpace(stringOne)) == 0 {
		fmt.Println("String is empty!")
	}

	if len(strings.TrimSpace(stringTwo)) == 0 {
		fmt.Println("String two is empty!")
	}

	if len(stringTwo) == 0 {
		fmt.Println("String two is still empty!")
	}

	if len(strings.TrimSpace(stringThree)) == 0 {
		fmt.Println("String three is empty!")
	}
}

(*@\playground{http://play.golang.org/p/2AQ0x5P_QE}@*)
\end{lstlisting}





\subsection{Summary}

Strings as we have discussed briefly are built from bytes so indexing  them yields bytes, not characters. A string might not even hold characters. In fact, the definition of \enquote{character} is ambiguous and it would be a mistake to try to resolve the ambiguity by defining that strings are made of characters.

There's much more to say about Unicode, UTF-8, and the world of multilingual text processing, but it can wait for another post. For now, we hope you have a better understanding of how Go strings behave and that, although they may contain arbitrary bytes, UTF-8 is a central part of their design.


\chapter{Booleans}
A boolean type is a one 1 bit integer type used to represent true and false (or on and off). Three logical operators are used with boolean values:

\begin{tabular}{cc}
\&\& & and\\
\textbar\textbar & or\\
!    & not \\
\end{tabular}


\section{Structs}

One of the most useful type of Go is \gotype{struct}. A struct is a collection of fields.

(And a type declaration does what you'd expect.)

\begin{lstlisting}
package main

import "fmt"

type Point struct {
	X int
	Y int
	Z int
}

func main() {
	fmt.Println(Point{1, 2, 3})
}
\end{lstlisting}

This outputs |{1 2 3}|

\subsection{Accessing fields}

Fields are accessed using the dot notation (similar to objects in other languages).

\begin{lstlisting}
package main

import "fmt"

type Point struct {
	X int
	Y int
	Z int
}

func main() {
	v := Point{10, 20, 30}
	v.X = 5
	fmt.Println(v.X)
}
\end{lstlisting}

\section{Arrays}

The type [n]T is an array of n values of type T.

The expression

var a [10]int
declares a variable a as an array of ten integers.

An array's length is part of its type, so arrays cannot be resized. This seems limiting, but as we will see alter on Go provides a convenient way of working with arrays.

\begin{lstlisting}
package main

import "fmt"

func main() {
	var a [2]string
	a[0] = "Hello"
	a[1] = "World"
	fmt.Println(a[0], a[1])
	fmt.Println(a)
}
>Hello World
>[Hello World]
\end{lstlisting}

We have declared that the array |a| will have two element of type |string|.


\section{Slices}

Arrays are inflexible, so we don't see them very often in Go code. Slices, though, are everywhere. They build on arrays to provide great power and convenience.\footnote{See also article by Andrew Gerrand\protect\url{https://blog.golang.org/go-slices-usage-and-internals}}.

The type specification for a slice is []T, where T is the type of the elements of the slice. Unlike an array type, a slice type has no specified length.

A slice literal is declared just like an array literal, except you leave out the element count:

\begin{lstlisting}
letters := []string{"a", "b", "c", "d"}
\end{lstlisting}







