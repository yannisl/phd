% \iffalse meta-comment
%<*internal>
\iffalse
%</internal>
%<*readme>
----------------------------------------------------------------
phd --- a package to shorten preambles
E-mail: yannislaz@gmail.com
Released under the LaTeX Project Public License v1.3c or later
See http://www.latex-project.org/lppl.txt
----------------------------------------------------------------
This file provides a phd for defining a class.
%</readme>
%<*readmemd>
###The `phd` LaTeX2e package

The `phd` latex package and the class with the same name provide
convenient methods to create new styles for books, reports
and articles. It also loads the most commonly used packages 
and resolves conflicts.

This work consists of the file  `phd.dtx`,
and the derived files   `phd.ins`,  `phd.pdf`, and `phd.sty`.

###Installation

run
          
           pdflatex phd.dtx
           makeindex -s gind.ist -g phd 

If you have any difficulties with the package come and join us at
http://tex.stackexchange.com and post a new question or
add a comment at http://tex.stackexchange.com/a/45023/963.
or send me a message at  yannislaz at gmail.com

### Documentation

The package was written using the `doc` and `docscript` packages,
so that it is self documented in a literary programming style. 
The .pdf is a fat document, providing over fifty book styles (the
equivalent of classes) plus there is a lot of write-up on the inner
workings of TeX and LaTeX2e. However, you don't need to know much
to use it.

      \usepackage{phd}
      %%%%%%%%%%%%%%%%%%%%%%%%%%%%%%%%%%%%%%%%%%%
%%%%%%  STYLE 13
%%%%%%%%%%%%%%%%%%%%%%%%%%%%%%%%%%%%%%%%%%%

\cxset{style13/.style={
 name={Chapter},
 numbering=arabic,
 number font-size=\HUGE,
 number font-family=\sffamily,
 number font-weight=\bfseries,
 number color=\color{gray!50},
 number before=\par\vspace*{5pt}\hfill\hfill,
 number dot=,
 number after={\hspace*{7pt}\par},
 number position=rightname,
 chapter font-family=\sffamily,
 chapter font-weight=\normalfont,
 chapter font-size=\LARGE,
 chapter before={\thickrule\vspace*{20pt}\par\hfill\hfill},
 chapter after={\vskip0pt\par},
 chapter color={black!50},
 title beforeskip={\vspace*{10pt}},
 title afterskip={\vspace*{50pt}\par},
 title before={\hfill\hfill\raggedleft},
 title after={},
 title font-family=\sffamily,
 title font-color=\color{thered},
 title font-weight=\bfseries,
 title font-size=\huge,
 section indent=-1em,
 section align=\raggedright,
 section numbering=arabic,
 section indent=0pt,
 section beforeskip=0pt,
 section afterskip=\baselineskip,
 subsection align=\raggedright,
 subsection font-family=\sffamily,
 subsection font-weight=\bfseries,
 subsection font-size=\large,
 subsection font-shape=\itshape,
 subparagraph number after=\space,
}
}

\def\setstyle#1{\cxset{style#1}%
 \renewsection\renewsubsection\renewsubsubsection%
 \renewparagraph\renewsubparagraph}

\setstyle{13}


\chapter{Introduction to Chapter\\ Style Thirteen}

\section{A Brief History of Biomedical\\ Fluid Mechanics}
\lorem
\medskip
\begin{figure}[ht]
\centering
\includegraphics[width=0.45\textwidth]{./chapters/chapter14}
\includegraphics[width=0.45\textwidth]{./chapters/chapter14a}
\end{figure}
\lorem


All choices, are made via an extended key-value interface. 
Although not a compliment, it resembles CSS and the keys are a bit verbose but
attributes are easy to change and have a consistent and easy to remember interface.

To set or add a key we only use one command:

      \cxset{chapter name font-size: Huge,
             chapter number font-size: HUGE} 

### Future Development

This is still an experimental version, but I will retain the
interface in future releases. There is a large amount of
work still to be carried out to improve the template styles
provided, to test it more thoroughly and to add a number of
improvements in the special designs.

__The package as it stands is not production stable.__ 


%</readmemd>
%
%<*TODO>
add tcolorbox support fix pgf codeexample has issues with integration
too 
%</TODO>
%<*internal>
\fi
\def\nameofplainTeX{plain}
\ifx\fmtname\nameofplainTeX\else
  \expandafter\begingroup
\fi
%</internal>
%<*install>
\input docstrip.tex
\keepsilent
\askforoverwritefalse
\preamble
----------------------------------------------------------------
phd --- A package to beautify documents.
E-mail: yannislaz@gmail.com
Released under the LaTeX Project Public License v1.3c or later
See http://www.latex-project.org/lppl.txt
----------------------------------------------------------------

\endpreamble

%\BaseDirectory{C:/users/admin/my documents/github/phd}
%\usedir{MWE}
\generate{\file{\jobname.sty}{\from{\jobname.dtx}{package}}}
\generate{
  \file{MWE-02.tex}{\from{\jobname.dtx}{MWE-02}}
  \file{MWE-03.tex}{\from{\jobname.dtx}{MWE-03}}
}
\generate{
  \file{test-tufte.tex}{\from{\jobname.dtx}{test-tufte}}
  \file{test-memoir.tex}{\from{\jobname.dtx}{test-memoir}}
  \file{test-scrartcl.tex}{\from{\jobname.dtx}{test-scrartcl}}
  \file{test-algorithms.tex}{\from{\jobname.dtx}{test-algorithms}}
  \file{test-hyphenation.tex}{\from{\jobname.dtx}{test-hyphenation}}
  \file{settings.tex}{\from{\jobname.dtx}{settings}}
  \file{test-spacing.tex}{\from{\jobname.dtx}{test-spacing}}
 }
%\nopreamble\nopostamble
%\generate{
%  \file{hhiero.lua}{\from{\jobname.dtx}{hhiero}}
%}
%</install>

%<install>\endbatchfile
%<*internal>
%\usedir{tex/latex/phd}
\generate{
  \file{\jobname.ins}{\from{\jobname.dtx}{install}}
}
\nopreamble\nopostamble

\generate{
	\file{README.txt}{\from{\jobname.dtx}{readme}}
  }

\generate{
  \file{README.md}{\from{\jobname.dtx}{readmemd}}
}
\generate{
  \file{TODO.tex}{\from{\jobname.dtx}{TODO}}
}
\generate{
  \file{MWE-01.tex}{\from{\jobname.dtx}{MWE-01}}
}

\ifx\fmtname\nameofplainTeX
  \expandafter\endbatchfile
\else
  \expandafter\endgroup
\fi
 
\immediate\write18{makeindex -s gglo.ist -g phd.gls phd.glo}  %needs checking from trivfloat
\immediate\write18{makeindex -s gind.ist -g phd.ind phd.idx} %needs checking from Joseph’s trivfloat
%</internal>
%<*driver>
\listfiles
\documentclass[oneside,11pt,a4paper]{ltxdoc}
\makeatletter
\gdef\@notprerr{text command only in preamble} %supress error for commands only in preamble
%\def\@eha{}
\makeatother
%
\usepackage[bottom=2cm]{geometry}
\savegeometry{std}
% Font management is perhaps the most complicated
% part for using the three engines we are supporting
% 
% \usepackage[style=mla]{biblatex}
\usepackage{phd}
\usepackage{Acorn, AnnSton, ArtNouv, ArtNouvc, Carrickc, Eichenla, Eileen, EileenBl, Elzevier, GotIn, GoudyIn, Kinigcap, Konanur, Kramer, MorrisIn, Nouveaud, Romantik, Rothdn, Royal, Sanremo, Starburst, Typocaps, Zallman}
\RequirePackage{morewrites}
\ifluatex
  \newfontfamily\hiero{NotoSansEgyptianHieroglyphs-Regular.ttf}
  \newfontfamily\meitei{Noto Sans Meetei Mayek}
  \newfontfamily\yi{Microsoft Yi Baiti}
  \newfontfamily\sundanese{SundaneseUnicode-1.0.5.ttf}
  \newfontfamily\tailue{NotoSansNewTaiLue-Regular.ttf}
  \newfontfamily\myanmar{Padauk}
  \newfontfamily\hebrew{Miriam}
  \newfontfamily\arabian
    [Script=Arabic,        % to get correct arabic shaping
     Scale=1.2]            % make the arabic font bigger, a matter of taste
    {Scheherazade}         % whatever Arabic font you like
  \newcommand{\textarabic}[1] % Arabic inside LTR
           {\bgroup\luatextextdir TRT\arabian #1\egroup}
  \newcommand{\narabic}         [1] % for digits inside Arabic text
           {\bgroup\luatextextdir TLT #1\egroup}
  \newcommand{\afootnote} [1] % Arabic footnotes
           {\footnote{\textarabic{#1}}}
  \newenvironment{Arabic}     % Arabic paragraph
           {\luatextextdir TRT\luatexpardir TRT\arabicfont}{}
  \newfontfamily\arabicfont[Script=Arabic]{Amiri}
  \newfontfamily\cherokee{Digohweli_1.ttf}%
  \newfontfamily\telugufam{code2000.ttf}
 
  \newfontfamily\gujarati[Scale=1.0]{code2000.ttf}
  \newfontfamily\bengal[Script=Bengali,Scale=1]{Shonar Bangla}
  \newfontfamily\russianfonttt{DejaVu Sans Mono}
  \newfontfamily\russianfont{Arial}
  \newfontfamily\armenianfont[Script=Armenian,Scale=MatchLowercase]{FreeSans}
  %%\newfontfamily\titus[Scale=1.05]{TITUSCBZ.ttf}
  %\newfontfamily\noto{NotoSerif-Regular.ttf}
  \newfontfamily\brahmi{Noto Sans Brahmi}
  %\newfontfamily\arial{Arial Unicode MS}
  \newfontfamily\pan{code2000.ttf}
  \let\cjk\pan
  \let\mongolian\pan
  \newfontfamily\aegean{Aegean.ttf}
   % Thai font
  \newfontfamily\thai[Scale=1.0,Script=Thai]{IrisUPC}
  % balinese
  \newfontfamily\balinese{Aksara Bali}
  \let\balinese\pan
  % lao
  \newfontfamily\lao[Scale=1.1]{code2000.ttf}
  \let\lao\pan
  \let\indicative\pan
  \def\defaulttext{\arial }
  \linespread{1.05}
  \frenchspacing
\fi
\ifxetex
\newfontfamily\arabian
    [Script=Arabic,        % to get correct arabic shaping
     Scale=1.2]            % make the arabic font bigger, a matter of taste
    {Scheherazade}         % whatever Arabic font you like
\newcommand{\textarabic}[1] % Arabic inside LTR
           {\bgroup\luatextextdir TRT\arabian #1\egroup}
\newcommand{\narabic}         [1] % for digits inside Arabic text
           {\bgroup\luatextextdir TLT #1\egroup}
\newcommand{\afootnote} [1] % Arabic footnotes
           {\footnote{\textarabic{#1}}}
\newenvironment{Arabic}     % Arabic paragraph
           {\luatextextdir TRT\luatexpardir TRT\arabicfont}{}
\newfontfamily\arabicfont[Script=Arabic]{Amiri}
  \newfontfamily%
  \arabicfonttt[Script=Arabic,Scale=.75]{DejaVu   Sans Mono}
  \newfontfamily\telugufam{code2000.ttf}
  \newfontfamily\gujarati[Scale=1.0]{code2000.ttf}
  \newfontfamily\bengal[Script=Bengali,Scale=1]{Shonar Bangla}
  \newfontfamily\russianfonttt{DejaVu Sans Mono}
  \newfontfamily%
    \russianfont{Arial}
  \newfontfamily%
    \armenianfont[Script=Armenian,Scale=MatchLowercase]
{FreeSans}
  \newfontfamily\titus[Scale=1.05]{TITUSCBZ.ttf}

  \newfontfamily\noto{NotoSerif-Regular.ttf}
  \newfontfamily\brahmi{Noto Sans Brahmi}
  \newfontfamily\arial{Arial Unicode MS}
  \newfontfamily\pan{code2000.ttf}
  \let\mongolian\pan
  %\newfontfamily\aegean{Aegean.ttf}
   % Thai font
  \newfontfamily\thai[Scale=1.0,Script=Thai]{IrisUPC}
  % balinese
  \newfontfamily\balinese{Aksara Bali}
  \let\balinese\pan
  % lao
  \newfontfamily\lao[Scale=1.1]{code2000.ttf}
  \let\lao\pan
  \let\indicative\pan
  \def\defaulttext{\arial }
  \linespread{1.05}
  \frenchspacing
\fi

\setcounter{secnumdepth}{6}

\sethyperref


%\makeatletter
\cxset{plain sections/.style={
 chapter name = CHAPTER,
 chapter toc = true,
 chapter color= thegray,
 chapter opening = right, 
 chapter numbering = arabic,
 chapter font-family= sffamily,
 chapter font-weight= bold,
 chapter font-size= LARGE,
 chapter before={\thinrule\vspace*{20pt}\par\hfill\hfill},
 chapter after={\vskip0pt\par},
 chapter spaceout = soul,
 number font-size= Large,
 number font-family= rmfamily,
 number font-weight= bfseries,
 number color=thegray,
 number before=\vspace*{5pt}\hfill\hfill,
 number dot=.,
 number after={\hspace*{7pt}\par},
 title beforeskip={\vspace*{10pt}},
 title afterskip={\vspace*{50pt}\par},
 title before={\hfill\hfill\raggedleft},
 title after={\par\thinrule},
 title font-family=\sffamily,
 title font-color= teal,
 title font-weight=\bfseries,
 title font-family=\sffamily,
 title font-size= Large,
 title font-shape= upshape,
 title spaceout= none,
 title beforeskip={\vspace*{10pt}},
 title afterskip={\vspace*{50pt}\par},
 title before={\hfill\hfill\raggedleft},
%
% numbers
% number font-family=\sffamily,
% number font-weight=\bfseries,
 number color=thelightgray,
 number before=\par\vspace*{5pt}\hfill\hfill,
 number dot=.,
 number after={\hspace*{7pt}\par},
 number position=rightname,
 section color= thered,     
 section beforeskip=15pt,
 section afterskip=15pt,
 section indent=0pt,
 section font-family= sffamily,
 section font-size= LARGE,
 section font-weight= bfseries,
 section font-shape=,
 section align= centering,
 section numbering prefix =,%use \thechapter. for books or add as option
 section numbering= arabic,
 section spaceout=none,
 section number after=ooo,
 subsection color= thered,
       subsection beforeskip=10pt,
       subsection afterskip=10pt,
       subsection indent=0pt,
       subsection font-family= rmfamily,
       subsection font-size= large,
       subsection font-weight= bold,
       subsection font-shape= upshape,
       subsection align= centering,
       subsection numbering prefix=\thesection.,%\S\hairsp,%add . 
       subsection numbering custom =\@arabic\c@subsection,% \two@digits{\@arabic\c@subsection},%
       subsubsection color= gray,
       subsubsection beforeskip=5pt plus3pt minus 2pt,
       subsubsection afterskip=5pt,
       subsubsection indent=0pt,
       subsubsection font-family= rmfamily,
       subsubsection font-size= normalfont,
       subsubsection font-weight= bold,
       subsubsection font-shape= itshape,
       subsubsection align= centering,
       subsubsection numbering prefix =\thesubsection.\@arabic\c@subsubsection,
       subsubsection numbering custom =, %\two@digits{\@arabic\c@subsubsection},
       subsubsection number after =, 
%
       paragraph color= thegrey,
       paragraph beforeskip=,
       paragraph afterskip=-0.5em,
       paragraph indent=0pt,
       paragraph font-family= rmfamily,
       paragraph font-size= large,
       paragraph font-weight= bfseries,
       paragraph font-shape=,
       paragraph align= centering,
       paragraph number after = 0pt,
       paragraph numbering=numeric,
       subparagraph color= thered,
       subparagraph beforeskip=0pt,
       subparagraph afterskip=-.5em,
       subparagraph indent=0pt,
       subparagraph font-family= sffamily,
       subparagraph font-size= large,
       subparagraph font-weight= normalfont,
       subparagraph font-shape= slshape,
       subparagraph align= RaggedRight,
       subparagraph number after =, % can affect all needs checking
       %subsubsection numbering prefix=\S\hairsp\thesection,%add . here if need be
       subparagraph numbering=none,
}
}
\cxset{plain sections}
\cxset{style13/.style={
 name= {\protect\pan अमुकग्रन्थे},
 chapter spaceout = none,
 numbering=arabic,
 number font-size= HUGE,
 number font-family= sffamily,
 number font-weight= bfseries,
 number color= gray!50,
 number before=\par\vspace*{5pt}\hfill\hfill,
 number dot=,
 number after={\hspace*{7pt}\par},
 number position=rightname,
 chapter font-family= sffamily,
 chapter font-weight= bold,
 chapter font-size= LARGE,
 chapter before={\tikzrule\vspace*{20pt}\par\hfill\hfill},
 chapter color= black!50,
 title beforeskip={\vspace*{10pt}},
 title afterskip={\vspace*{50pt}\par},
 title before={\hfill\hfill\raggedleft},
 chapter rule color=spot!50,
 title after=\par\tikzrule,
 title font-family= sffamily,
 title font-color= teal,
 title font-weight= bfseries,
 title font-size= huge,
 section indent=-1em,
 section align= left,
 section numbering= arabic,
 section indent=0pt,
 section beforeskip=0pt,
 section afterskip= 10pt,
 section color=teal,
 subsection align= ,
 subsection font-family= sffamily,
 subsection font-weight= bfseries,
 subsection color = teal,
 subsection font-size= large,
 subsection font-shape=,
 subparagraph number after=,
 subsubsection align=,
}
}
\cxset{style13}

\renewparagraph
\renewsection
\renewsubsection
\renewsubparagraph
\renewsubsubsection

\makeatother

%\usepackage[verbose]{backref}  not eith biblatex
%\backrefsetup{verbose=false}
% gives error
%
%% PACKAGES AFTER HYPERREF
%\usepackage{arydshln}
%\usepackage{cleveref}
\usepackage{expl3}
\usepackage{xparse}
%\usepackage{pagenote}
%\usepackage{pkgindoc}
%xfrac loads xtemplate?
\usepackage{subcaption}
\usepackage{calligra} 
%\usedictionary{pages}
%Fix overfful hboxes automatically
\tolerance=2000
\emergencystretch=10pt
\makepagenote %????
%\EnableCrossrefs
% One of the two commands below
%\CodelineIndex

\PageIndex
\RecordChanges
\usepackage{makeidx}
\makeindex

\DeclareFloatingEnvironment[fileext=plate,
                                             listname=List of Plates,
                                             name=Plate,
                                             placement=htbp,
                                             within=none]{plate}
                                             
\DeclareFloatingEnvironment[fileext=painting,
                                             listname=Paintings,
                                             name=Painting,
                                             placement=htbp,
                                             within=none]{painting}                                             
            

% Try getting errors
%\usepackage{opcit}  
%\let\citep\cite
%\let\citet\cite
%\usepackage{glossaries} conflict with acronym
%\makeglossaries %experimental
\makeatletter\@debugfalse\makeatother
%\let\oldnobreakspace\nobreakspace
%\usepackage{ctib}
%\let\nobreakspace\oldnobreakspace
\newfontfamily{\codetwothousand}{code2000.ttf}
  \newfontfamily{\codetwothousandone}{code2001.ttf}
  \newfontfamily{\symbola}{symbola.ttf}
\begin{document}
\mainmatter
\cxset{blank page text=}

\DocInput{\jobname.dtx}
\raggedright
%%\def\chaptername{Chapter}
%\makeatletter
%\cxset{style13}
\cxset{style87a/.style={
 chapter opening=any,
 name=Chapter,
 % positioning and float - inline is 0
 %  float right is 2
 number display=block,
 number float=right,
 number shape=starburst,
 chapter numbering=arabic,
 number spaceout=none,
 number font-size=huge,
 number font-weight=mdseries,
 number font-family=sffamily,
 number font-shape=upshape,
 number before=,
 number display=inline,
 number float=none,
% 
 number border-top-width=0pt,
 number border-right-width=0pt,
 number border-bottom-width=0pt,
 number border-left-width=0pt,
 number border-width=0pt,
%  
 number padding-left=0em,
 number padding-right=0.5em,
 number padding-top=0em,
 number padding-bottom=0pt,
  %number margin-top=, to do
 %number margin-left=0pt,  to create
 %
 number after=,
 number dot=,
 number position=rightname,
 number color=black,
 number background-color=white,
 %chapter name
 chapter display=block,
 chapter float=left,
 chapter shape=ellipse,
 chapter color=white,
 chapter background-color=sweet,
 chapter font-size= Huge,
 chapter font-weight=mdseries,
 chapter font-family=sffamily,
% chapter font-shape=upshape,
 chapter before=,
 chapter spaceout=none,
 chapter after=,
 chapter margin left=0cm,
 chapter margin top=0pt,
 %
 chapter border-width=0pt,
 chapter border-top-width=0pt,
 chapter border-right-width=0pt,
 chapter border-bottom-width=0pt,
 chapter border-left-width=0pt,
% 
 chapter padding-left=0pt,
 chapter padding-right=0pt,
 chapter padding-top=0pt,
 chapter padding-bottom=0pt,
  %chapter title
 title font-family=sffamily,
 title font-color=black!80,
 title font-weight=bfseries,
 title font-size=huge,
 chapter title align=none,
 title margin-left=1cm,
 title margin bottom=1.3cm,
 title margin top=25pt,
 % title borders
 title border-width=0pt,
 title padding=0pt,
 title border-color=black!80,
% title border-top-color=spot!50,
% title border-top-width=20pt,
 title border-left-color=black!80,
 title border-left-width=2pt,
 title border-color=black!80,
 title padding-top=10pt,
 title padding-bottom=10pt,
 title padding-left=10pt,
 title padding-right=0pt,
% title border-right-color=spot!50,
% title border-right-width=20pt,
% title border-bottom-color=spot!50,
% title border-bottom-width=20pt,
 %
 chapter title align=left,
 chapter title text-align=left,
 chapter title width=0.8\textwidth,
 title before=0pt,
 title after=,
 title display=block,
 title beforeskip=,
 title afterskip=,
 author block=false,
 section font-family=rmfamily,
 section font-size=LARGE,
 section font-weight=bfseries,
 section indent=0pt,
 epigraph width=\dimexpr(\textwidth-2cm)\relax,
 epigraph align=center,
 epigraph text align=center,
 section color=spot!50,
 section font-weight=bfseries,
 section align=left,
 section number after=\hskip10pt,
 section font-family=sffamily,
 section numbering prefix=\@arabic\c@chapter.,
 epigraph rule width=0pt,
 header style=plain}}
 \makeatother
 
\cxset{style87a}


%\cxset{name=Chapter, chapter toc=true,
%         }
\pagenumbering{gobble}
\setcounter{page}{0}
\cxset{number font-style=upshape, number color=black, number padding-left=0.5em,
         chapter numbering=Words, number spaceout=none,chapter spaceout=none, section align=left,
         section font-family=sffamily, section font-weight=bfseries,chapter opening=any}

\chapter{Summary MEP Progress Report for St Regis Hotel, Habtoor City}
\pagenumbering{arabic}
\thispagestyle{plain}
\section{Current Status}

We have started flushing of the Chilled Water system on the 7 April 2015, as planned and we anticipate to be in a position to progressively provide \emph{wild air} before the end of April, ahead of the scheduled date of the 7 May 2015. In the Basements and in the Guest rooms we have started final fix works, where possible. The Main Plantrooms at Technical Floors 1 and Podium 6, are in the main completed, except final ductwork connections where they impede access. BMS DDC Panels are expected to arrive by the 22 April 2015 and installation expected to be completed within 25-30 days to ensure that by end May we can provide controlled conditions.

Delays have been experienced in the receipt of Electrical panels, such as DBs (delayed due to late deliveries of components by Legrand) and others that were subjected to numerous changes, as described later on.  Other delays were due to late instructions as briefly detailed in Section~\ref{delays}. 

The current outstanding works for the St Regis Hotel are as follows:

\subsection{St Regis Basement}

\begin{description}
\item[Kitchen Corridors] Some kitchen corridors cable pulling is still under progress. Expected to complete by 30 Apr 2015.
\item[Main Electrical Room] Delays experienced due to the failure of cable trays during cable pulling and also due to the some of the MDBs being returned to the factory for modifications, as they failed QA/QC Inspections.
\item[Fan Rooms] Fans scheduled to be delivered 23 Apr 2015.
\item[BMS] DDC Panels still to be delivered.
\item[Sump Pumps] Expected to be delivered by 10 May 2015. 
\item[Others] There are still closure related works, for areas currently inaccessible, such as the new ramp areas, store and office areas. 
\end{description}

\subsection{Ground Floor}
\begin{description}
\item[Ballroom] This area is still under scaffolding being used by the Main Contractor to erect walk-ways in the ceiling. Once the scaffolding is dropped and we are given access to the lower level, we have to install another layer of services, give ceiling grid clearances and upon construction of the ceiling grid we can then install final sprinkler droppers and give clearances for final boarding.
\item[Banquet Hall] This area has been delayed due to the Iridium Spa delays in Design and appointment of subcontractors. As this area is above the Banquet Hall, coring for drainage pipes delayed the works. This coring is now complete and we expect to ask the Main Contractor to lower the scaffolding and start with the rest of the services.
\end{description}
\subsection{Mezzanine}
\begin{description}
\item[Festival Dining Restaurant] Currently this area is under nomination, there is no ID Design and final details are still awaited. 
\item[Security Room] The design for this room has recently changed. The room as shown in the new designs is different from what has been constructed on site and has no space for CCUs. 
\item[AV Room] Expected to be completed 30 Apr 2015.
\item[Furniture Store] Expected to be completed 25 Apr 2015.
\item[Balance Corridors] Expected to be completed 25 Apr 2015.
\end{description}

\subsection{Podium 1}

\begin{description}
\item[Banquet A/V Technician] We have no access. This is currently being used as a store.
\item[Service Corridor] Plan to release for ceiling grid on 23 Apr 2015.
\item[St Regis Main Kitchen and Corridor] Plan to release on 30 Apr 2015.
\item[Property Store] Currently no access. If access provided we can release by 30 Apr 2015.
\item[Steak House Kitchen] Plan to release by 30 Apr 2014.
\end{description}

\subsection{Podium 2}
\begin{description}
\item[Iridium Spa and related areas] We are currently working in the area, which was delayed by late appointment of Finishing Contractor. Still some ID Shop Drawings not available. We expect to catch-up with delays by end May 2015. We plan to complete final fix by 10 June 2015 and Testing and Commissioning by 20 Jul 2015.
\item[Other Areas] All other areas will be released for closure by 26 Apr 2015.
\end{description}

\subsection{Podium 3-6}

All guestrooms have been handed over for ceiling closures with the exception of some of the suites, where information and access was provided late. These are the following:

\begin{description}
\item[Ambassador Suite] Co-ordination ongoing. Expect resolution and final clearances 25 May 2015.
\item[Bentley Suite] Incomplete information. Completion targets uncertain at this stage.
\item[Royal Suite] Co-ordination on-going. Expect resolution and final clearances 25 May 2015.
\end{description}

\subsection{Floor 1}

\begin{description}
\item[Kitchen 4 and Kitchen 6] Works for walls are progressing, insufficient detail information. Can complete by 15 May 2015, provided all Kitchen Subcontactor’s drawings become available and unimpeded access.
\end{description}

\section{Delays in Target Dates}
\label{delays}
This is a brief summary of recent selected instructions for additional works that have impacted  MEP Progress. 
In addition to these additional works another critical factor that affected progress was the congestion of services and the numerous RFIs and responses we had to raise in order to resolve them.

\begin{itemize}
\item Relocation of Kitchen Extract ducting Ground Floor, Mezzanine and Podium BOH areas.
\item  Additional AV points in all public areas.
\item  Additional telephone, data and CCTV points in all Public Areas.
\item  Motorized curtains Meeting Rooms.
\item Lighting Control System. 
\item Emergency Lighting System. (see details Chapter~\ref{emergencylights})
\item Changes to Electrical DBs, SMDBs due to late receipt of DEWA approved drawings. (See Chapter~\ref{electrical})
\end{itemize}

We have reacted as fast as possible to all instructions and as soon they were received we have added resources to mitigate delays. Where days slipped these are only by a few days and we are confident that by end of this month all physical installations will be completed with the exception of the English Pub, Banquet and Royal Suite. 

\subsection{Back of the House Areas}

All back of the House Areas experienced delays, due to the lack of primary co-ordination at design stage. This caused delays until solutions were found enabling us to install the services. 

The allowable ceiling height in this area was impossible to be achieved and the kitchen extract duct eventually was split in two sections and distributed through two different routes in order to avoid passing it through the corridors which could not accomodate it.

In addition a new roller shutter window was introduced, that made it impossible to install the fresh air ducts feeding the kitchen. After several attempts by |K&A| to find an acceptable solution the roller shutter  window was abandoned as per the instructions of the Client Representative. 

\subsection{Basement Kitchen and Related Areas at B1}

Please note that these areas (with the exception of the corridor) have been cleared for ceiling grid closures in most areas and the balances are as per target to close by the 15 April 2015, including additional works. The additional works were mostly for additional ELV points on walls and for which we have received drawings on the 29 March 2015. We have instituted overtime and added additional crews to complete the works as fast as possible. Most rooms in the area have been affected. 

\begin{comment}
\chapter{Busbar System}

As per the approved Baseline Program we expected to place the busbar order for all three hotels on 27 February 2014. However, HLS DSE-JV were unable to place any orders due to the events that are outlined below, with finality on all busbars only achieved in April 2015. 

\begin{enumerate}
\item On the 23 December 2013 we were requested to change the specification for some busbars via HLG transmittal Ref. No. HLG-626-DT-HLS-0628 dated 23 Decemeber 2013 \textit{Fire Resistance Bus Bar Specification}.

\item On the 25 February 2014 we were issued revised designs via tranmittal Ref. No. HLG-626-DT-HLS-0873 \textit{Revised Electrical Drawings}.

\end{enumerate}


\chapter{Generators}

\section{Generator Ventilation}

\subsection{Background}

The original tender drawings indicated the Generator Ventilation to be by means of Louvres. When such an approach is taken normally the ventilation openings are dictated by the size of the generators.


HLS DSE-JV have submitted as early as 2014 RFIs outlining concerns regarding the adequacy of the ventilation openings and sizing of Generator rooms in the basements.

On the 25 March 2015, we were instructed to proceed with the purchase of additional fans from Systemaire. We issued the order request on the ..... and the order placed on the ......  without formal approval of the amounts in order to speed up the purchase. This affected the commissioning of the generators.

\chapter{Transformer Room Ventilation}

\subsection{Background}

\subsection{Design Errors}
\end{comment}


\chapter{Additional Works Due to Revised Electrical Drawings}
\label{electrical}
Additional works as per letter \texttt{HLG/626/2.05/YE/es/7312/15} dated 6 April 2015. These changes relate to late approval of DEWA drawings. These changes affected all the hotels.

\section{St. Regis}
The following changes were instructed via the above letter and were based on drawing number |EM3300|.
\begin{description}
\item[SMDB-H1-1PLBPR] The works adds outgoing cables feeding |ADD-SS-01| and for |DBP-H1-1PLBPR1|  the cable size was changed from 4c:10mm2 XLPE to 4c:16 mm2 XLPE. The breaker size was changed to 60A MCCB.

\item[SMDB-H1-1TEFCWF] The instruction requests the changing of 15A breaker to 20A for eight CP-H1-1-TEWF/05 T.C.L.-1kW and one CP-H1-1TEWF/09 T.C.L.-1kW.

\item[SMDB-H1-2PL] The instruction requests the following changes:
   \begin{enumerate}
      \item DBP-H1-2PL MCCB 60A change to 80A and cable size 4c:16mm2 XLPE change to 4c:25mm2 XLPE.
      \item BPN-PN-16 and 18 30mA ELCB added.
   \end{enumerate}

\item[SMDB-H1-2PSPA] The instruction requests the following changes:
    \begin{enumerate}
      \item Male and female Jacuzzi bath MCCB 15A change to 20A TCL-3kW.
      \item Female steam room cable size changed (4c:10mm2 XLPE to 4c:16mm2 XLPE).
    \end{enumerate}


\item[SMDB-H1-6PL] The instruction requests the following changes:
   \begin{enumerate}
      \item Additional outgoing feeders for EC-01B, EC-02B, WET-PN-011, WET-PN-017 and WET-PN-020.
      \item 40ATP MCCB removed for FP-H1-1FL
   \end{enumerate}

\end{description}

The following changes were due to drawing No:EM3301

\begin{description}
\item [MDB-H1-B1R1] The instruction requests the following changes:
    \begin{enumerate}
       \item SMDB-H1-GR2 MCCB 200A change to 225A and cable size 4c:95mm2. XLPE change to 4c: 120mm2 XLPE (TCL 116.8kW).
       \item UPS MCCB 60A change to 80A.
    \end{enumerate}
\item[MDB-H1-GR2] The following changes were instructed:
    \begin{enumerate}
       \item Incomer MCCB 200A TP change to 225A TP.
       \item Additional outgoing for WPN-PA-012, WPN-PA-032.
    \end{enumerate}
\end{description}

The following changes were due to drawing No:EM3302

\begin{description}
\item[SMDB-H1-GLSTBR] The following changes were requested:
   \begin{enumerate}
      \item Additional outgping for St Regis, Special Event, St Regis BR.
      \item DBP-H1-GLSTBR MCCB 60A change to 80A.
   \end{enumerate}
\item[SMDB-H1-1PLMK] The following changes were requested:
      \begin{enumerate}
        \item Additional space.
        \item 60A TP MCCB removed.
      \end{enumerate}  
\item[SMDB-H1-2PGSC] The following changes were requested:
     \begin{enumerate}
        \item CAF-SS-01 cable and MCCB size changed from 4c:70mm2 XLPE and 150A TP to 4c:XLPE and 30A TP (TCL-6.5kW).
     \end{enumerate}
\end{description}

The following changes were detailed on drawing No:EM3303

\begin{description}
\item[MDB-H1-B1LR2] SMDB-H1-GL MCCB80A change to 100A.
\item[SMDB-H1-GL] DBP-H1-GLPFA MCCB 60A change 80A and cable size 4c:16mm2 XLPE change 4c:25mm2 XLPE.
\item[SMDB-H1-GLBP1] Additional outgoing for BOQ-KIT-016.
\end{description}

The following changes were detailed on drawing No:EM3304.

\begin{description}
\item[EMDB-H1-B1]  The following changes were requested:
   \begin{enumerate}
      \item ESMDP-H1-GR2 MCCB 80A change to 150A and cable size 4c:35mm2 XLPE change to 4c:70mm2 XLPE.
      \item ESMDB-H1-6PMS1 MCCB 400ATP change to 500ATP.
   \end{enumerate}
\item[EMDB-H1-6PMS1] The following changes were requested:
    \begin{enumerate}
       \item Incomer MCCB 400ATP change to 500ATP.
       \item Additional outgoing for EC-01A,B and future load.
       \item ESMDB-H1-RS cable size changed from 4c:70mm2 XLP (125A TP to 4c:95mm2 XLPE (TCL-55kW).
    \end{enumerate}
\item[ESMDB-H1-GL]
\item[ESMDB-H1-6PMS2]  The incomer to MCCB was changed from 700A TP to 800A TP.
\item[ESMDB-H1-6PL] An additional outgoing cable was requested for EC-02A. For LIFT-H1-SL05 and LIFT-H1-SL06 the cable size was requested to be changed to 4c:35mm2 MGT/XLPE.
\item[ESMDB-H1-2PL] CAF-SK-012, EC-01B MCCB and cable size changed from 30A SP 2c:16mm2 XLPE to 20ASP and 2c:4mm2 PVC (T.C.L.-2.6kW and 0.8kW).
\end{description}

\subsection{St Regis Basement Areas}
The following changes were detailed on drawing No:EM3200.
\begin{description}
\item[SMDB-BP-1BS1] Additional outgoing circuits were requested for DB-LS-SR2, DB-LS-SR3.
\item[SMDB-BP-1BS3]  An additional outgoing circuit was instructed for DB-LS-SR5.
\item[SMDB-BP-1BS5] An additional outgoing circuit was requested for DB-LS-SR6.
\end{description}

The following changes were detailed on drawing No:EM3201.

\begin{description}
\item[EMDB-BP-1B3] ESMDB-BP-1BS7 MCCB 40A change to 80A and cable size 4c:10mm2 XLPE change 4c:16mm2 XLPE (TCL-17.8kW).
\item[ESMDB-BP-1BS9] cable size 4c:35mm2 XLPE change to 4c:70mm2 XLPE (TCL-44.4kW).
\item[ESMDB-BP-1B3]  The following changes affected this panel:
     \begin{enumerate}
        \item ESMDB-BP-1BS7 MCCB 40A change to 80A and cable size 4c:10mm2 XLPE change to 4c:16mm2 XLPE    (TCL-17.8kW). 
        \item ESMDB-BP-1BS9 cable size 4c:35mm2 XLPE change to 4c:70mm2 XLPE (TCL-44.4kW). 
        \item ESMDB-BP1BS10 cable size 4c:240mm2 MGT change to 4c:300mm2 MGT(TCL-120kW).
     \end{enumerate}
\item[ESMDB-BP-1BS2]  3 Nos CP-BP-1BE/F1 cable size 4c:16mm2 MGT change to 4c:25mm2 MGT (TCL-17kW).
\item[ESMDB-BP-1BS3]  20ATP, pulse meter, 10mm2 MGT removed for SPCP-BP-1B12.
\item[ESMDB-BP-1BBPA] Incomer MCCB 80A TP change to 100A TP.
\item[ESMDB-BP-1BCOM1] Additional outgoing for COM-IC-001, COM-IC-002, COM-IC-003, COM-IC-006.
\item[USMDB-BP-1BS] UDB-BP-1BS4 and UDB-BP-1BS5 cable size 4c:10mm2 XLPE change to 4c:16mm2 XLPE (TCL-8.8kW and TCL-7.6kw).
\item[ESMDB-BP-1BS1] Incomer MCCB 200A TP change to 250A TP.
\item[ESMDB-BP-1B] ESMDB-BP-1BBPA MCCB 80A change to 100A and cable size 4c:50mm2 XLPE change to 4c:70mm2 XLPE(TCL-44.8kW).
\end{description}

The following changes were due to additional works detailed on drg No: EM3204.

\begin{description}
\item[MDB-BP-2BMEC]
   \begin{enumerate}
     \item Incomer MCCB 80A TP change 100A TP.
     \item FPCP-H1-2B2 cable size 4c:10mm2 XLPE change to 4c:16mm2 XLPE.
     \item FPCP-H1-2B1 cable size 4c:6mm2 XLPE change to 4c:6mm2 XLPE change to 4c:10mm2 XLPE (TCL-5.5kW).
   \end{enumerate}
\item[SMDB-FB-2BMEC]
\end{description}

The following changes were due to drawing No: EM3206.
\begin{description}
\item[MDB-BP-1B2] 
    \begin{enumerate}
       \item MDB-BP-1BCOM Additional outgoings for COM-MP-041.
       \item SMDB-BP-1BS6 MCCB 400A change to 500A and cable size 2x4c:120mm2 XLPE change to 2x4c:150mm2 XLPE (TCL-221kW).
       \item 400A TP+2x4c:120mm2 XLPE removed for FFP-3.
    \end{enumerate}
\item[SMDB-BP-1BS6] Additional outgoing for DB-LS-SR4.
\item[SMDB-BP-1BS10] Additional works were requested as follows:
    \begin{enumerate}
      \item DB-H3-1BSS2 cable size change to 2c:10mm2 XLPE change to 2c:16mm2 XLPE (TCL-1.2kW).
      \item CP-BP-1BTE/F4 cable size change to 4c:16mm2 XLPE change to 4c:25mm2 XLPE MCCB 40A TP Change to 60A TP (TCL-25kW).
      \item CP-BP-1BTF/F2 and CP-BP-1BTE/F2 MCCB 60A TP change to 80A TP (TCL-37kW).
      \item CP-BP-1BTF/F3 and CP-BP-1BTE/F3 MCCB 40A TP change to 60A TP (TCL-22kW and 25kW).
    \end{enumerate}
\end{description}


\chapter{Emergency Lighting System}
\label{emergencylights}
The Emergency Lighting System was finalized on the 22 February 2015. This is impacting on the final fix and commissioning of the Hotel’s Central Battery and Emergency Lighting System. 

\begin{enumerate}
\item As per the approved Baseline Program, we were planning to submit the Material Submission of the Emergency Lighting System by the 25 Feb 2014.
\item On the 25 Nov 2013, we raised RFI \texttt{HLS-DSE/142 JV-RFI-MEP-E028} requesting full details of the Emergency Lights as well as the capacity of the central battery system in order to proceed with Technical Submittals, design of containment system and procurement of equipment.
\item On the 12 Dec 2013 we received an insufficient reply to the above mentioned RFI. We have notified you that the repsonse was insufficient via letter \texttt{HLS DSE/JV/HLG/YL1181} dated 14 Jan 2014, clearly stating that we were unable to proceed further with the submission of the Central Battery System, until the requested information was provided. In our letter we had requested that all details such as diffuser details, base type, IP rating and lamp characteristics are provided. We have also provided details as to Civil Defence requirements.
\item The above concerns were forwarded to the Engineer by the Main Contractor on the 20 Jan 2014. The Engineer instructed us to follow the current design dawings until the completion of the Lighting Consultant’s works.

\item On 10 Feb 2014, we had responded via letter \texttt{HLS-DSE/JVHLG/YL/1227} stating that the information provided by the Engineer, as response to RFI HLS-DSE/142 MEP-E028 was inadequate to produce Shop Drawings and to proceed with material procurement or calculations.
\item On 19 March 2014, once again we responded via letter HLS DSE/JV/626/2.05/YE/nd/2609/14 dated 4 Mar 2014 stating that the inforamtion was inadequate.
\item On 16 April 2014 we sent a clear notification that the lack of information was expected to delay the works via letter \texttt{HLS DSE/JV/HC/L/YL/1322} stating that we were unable to proceed with this portion of the works.

\item On the 20 August 2014 we received via an email instructions to proceed based on a generalized scheme.
\item We raised RFI-MEP-E249 dated 21 Sep 2014, requesting more details on locations and quantities of Emergency Light Fittings. The RFI response was received on 13 Oct 2014 with the response to follow the latest issued Guest Room drawings. 
\item Engineer’s letter \texttt{DU1211/DU/L20054/14} dated 15 Sep 2014, confirmed that due to several ID Design issues the above details were no longer applicable.
\item On 30 Sep 2014 we served notices regarding additional works due to revisions of the Emergency Lighting System for all three hotels.
\item On 15 Nov 2014, we raised concerns due to late finalization of the Central Battery System for W and Westin Hotels. 
\item On the 20 Dec 2014 the we received instructions from the Engineer and Client requesting us to revert back to the original K\&A designs.
\item On the 22 Feb 2015, the Engineer instructed us to procure and install all the Front of House exit lights. We confirmed receipt of the instruction via letter \texttt{YL/1935} dated 24 Mar 2014 once all final details and samples were finalized.
\end{enumerate}











%


\def\programming{%
  \part{PROGRAMMING}
  \MakePercentComment
\chapter{PROGRAMMING MACROS}
\addtocimage{-10pt}{-40pt}{../graphics/harnett.jpg}
%\minitoc
\pagebreak
\setlength\columnsep{1.5em}

\thispagestyle{plain}
{\centering  \includegraphics[width=0.7\linewidth]{./graphics/harnett.jpg}\par}

\newcommand*{\newacronym}[1]{{New acronym: [#1]\par}}
\newcommand*{\newacronyms}{%
  \let\do\newacronym
  \docsvlist
}
\vspace{1.5\baselineskip}
{\centering \Large\bf GETTING STARTED WITH MACROS\par}
\bigskip

\begin{multicols}{2}
\lettrine{P}{rogramming} with \alltex is done through macros. \tex has a macro programming language,
which allows features to be added. The best known
and most widely used \tex macro package is \latex.
(This is not quite accurate. Although originally
\latex used \tex, since 2003 it by default uses
e-TEX, which is an extension of TEX. Macro's in \TeX\  are not just simple substitutions, they are more Lispy like. It is this powerful feature that made \TeX\ last and will continue to do so for many years to come. This program that started as a typesetting program, programmed in a variant of what is now an ancient computer language Pascal is a manifest to good programming and a reminder to the programming priesthood that the tool is not important, but what you do with it is. A macro is a sequence of tokens that has been abbreviated into a control sequence. Statement starting with among others
\cmd{def} are called \textit{macro definitions}. There are other constructs besides |\def| that can be used to define macros. \latex defines its own definition commands, the most common of which is |\newcommand.| The way \tex's macro language is build, you can also define your own. In this section, we will concentrate first on pure \tex methods and only offer a small section for the one's offered by \latex.

\end{multicols}
\clearpage

\section{Simple substitution macros}

\begin{macro}{\def}
Simple substitution macros, during expansion replace their name with the contents enclosed between the braces. For example some common macros that authors write, is to hold the names of people, in order to get the spelling correctly.
\end{macro}

\begin{teX}
\documentclass{article}
\def\myshortcut{Anthony van der Merwe}
\begin{document}
\myshortcut
\end{document}
\end{teX}




In the above we are defining a macro named, |\myshortcut|, will print the name \texttt{Anthony van der Merwe}, every time it is invoked as |\myshortcut|. You will notice, that the macro definition is placed in the preamble. This is not necessary, but it is good practice. Macros can be placed anywhere in the document, in packages and or classes.

If we were writing the macro and compiling it using \tex only, the example can be much shorter.

\emphasis{def}
\begin{teX}
\def\myshortcut{Anthony van der Merwe}
\myshortcut
\bye
\end{teX}

Macros can use other macro commands. For example if we wanted to store the name of the author of |pdfTeX| we could write,

\begin{texexample}{example substitution macro}{}
\def\Thanh{^^A
      H\`an~%
      \texorpdfstring{Th\^e\llap{\raise 0.5ex\hbox{\'{}}}}%
      {\ifpdfstringunicode{Th\unichar{"1EC3}}{Th\^e}}%
      ~Th\`anh^^a
    }
\Thanh 
\end{texexample}




\subsection{Macro parameters.} 

In this Chapter we will spend most of the time with commands available in TeX core, before we move onto commands that are available in \latex. Now, in the example above, we did not use any parameters. \tex allows us to define parameter by adding |\#1|..|\#9| as parameters to the macro definition. Here is a short example, again using plain \tex. 


\begin{teX}
\def\twonumbers#1#2{(#1,#2)} 
\twonumbers{12,13}
\bye
\end{teX}
\def\twonumbers#1#2{(#1,#2)}
This will print \texttt{\twonumbers{12}{13}}. The macro takes the two arguments 12 and 13  and prints the two numbers in parentheses. This activity is called \textit{macro expansion}\index{macros>expansion}\index{macros>parameters}.

 
\section{Delimited arguments}

As a simple example consider the following:\index{macros>delimited}

\begin{teX}
\def\asentence#1#2;{{#1#2}}
\bye
\end{teX}

\begin{texexample}{delimited examples}{delimited}
\def\asentence#1#2;{{#1#2}}

{\asentence The whole sentence is printed;}\par
{\asentence The whole sentence is printed;}\par
{\asentence The whole sentence is printed;}\par
\end{texexample}


Example~\ref{delimited} defines a macro with an undelimited first parameter, and a second parameter delimited by a
semicolon.

\subsection{Space, return, and the tab character as delimiters of parameters}

A space can be used to delimit a parameter. The space character, return character and the tab character are all converted into space tokens by \tex. Here is an example,

\begin{texexample}{Space delimiters}{ex:spacedelimiters}
\def\tempmacro #1 #2 #3 {#1,#2,#3}
\tempmacro 12 15 17 
\end{texexample}


\section{Format of a macro definition}

So far we have looked at macros that have no parameters, macros that have parameters and macros that have delimited arguments. A macro definition consists of, in sequence,

\begin{enumerate}
\item any number of \cmd{\global}, \cmd{\long}, and \cmd{\outer}, prefixes
\item a \cmd{\def} control sequence, or anything that has been \cmd{\let} to one,
\item possibly a parameter text specifying among other things how many parameters the macro has,
\item a replacement text enclosed in explicit characters \{\}
\end{enumerate}


\CMDI{\global}\cmd{\def}\meta{command}\{\ldots\}

As the name implies global macros define macros that they have a global scope. \TeX, like many other computer languages has scoping rules. We will revisit \tex's scoping rule in the Chapter for Grouping.  Try the following example:


\begin{teX}
\def\sometext{This is some text}
\def\someothertext{%
   \def\sometext{I am in the macro, someothertext.}\par
   \sometext
}
\sometext
\end{teX}

\def\sometext{This is some text}
\def\someothertext{%
   \def\sometext{I am in the macro, someothertext.}\par
   \sometext
}
\sometext
\someothertext

As you can see from the output, any definitions of macros within other macros are defined locally within the scope of the aprent macro only. I am also sure that you have also observed that we can nest macros to as many depths as required.

\def\sometext{This is some text}

\def\someothertext{%
   \gdef\sometext{I am in the macro, someothertext.}\par
   \sometext
}

\sometext

\someothertext

\sometext

\CMDI{\long}\cmd{\def}\marg{command}\{\ldots\}

\index{macro definitions>long}
Knuth designed \tex in such a way that the normal |\def| will not work with arguments that include paragraphs. This was so that if you forget to add a brace '\}' \tex will not continue absorbing tokens until the end of the file or completely full \tex's memory. Therefore \tex has another rule [205] intended to confine errors to the paragraph that they occur: The token |\par| is not allowed to occur as part of an argument as unless you explicitly tell \tex that you want to use |\par|. Whenever \tex is about to include |\par| as part of an argument, it will abort the current macro expansion and report that a \texttt{...runaway argument} has been found.

If you actually want a control sequence to allow arguments with |\par| tokens, you can define it to be a \cmd{\long}\index{macros>long} just before the |\def|. For example the |\bold| macro defined by:


\begin{teX}
\long\def\bold#1{{\bf#1}}
\end{teX}

\noindent is capable of setting several paragraphs in boldface type. However, such a macro is not a especially good way to typeset bold text. It would be better to say, e.g.,

\begin{teX}
\def\beginbold{\begingroup\bf}
\def\endbold{\endgroup}
\end{teX}
because this doesn't fill \tex's memory with a long argument.


\CMDI{\edef}
\index{macro definitions>\string\edef=\texttt{\string\edef}}

Another command that can be used to define macros is \cmd{\edef}. You can say |\edef\foo{bar}|. The syntax is the same as |\def|, but the token list in the body is fully expanded (tokens that come from |\the| are not expanded).

You can say |\xdef\foo{bar}|. The syntax is the same as \cmd{\def}, but the token list in the body is fully expanded (tokens that come from \cmd{\the} or \cmd{\unexpanded} are not expanded).

\CMDI{\global}\cmd{\edef}

You can put the prefix \cmd{\global} before \cmd{\xdef}, this is however useless, since |\xdef| is the same as |\global\edef|. The following example puts a brace in |\foo|. The |\string| command can be expanded, the value is the name of the command (preceded by a backslash, or whatever the value of the escape character is). Here the assignment to the escape character is local, the assignment to |\foo| is global.


\begin{teX}
{\escapechar=-1 \xdef\foo{\string\}}}
\end{teX}


\CMDI{\relax}\quad 

The control sequence \cmd{relax} cannot be expanded, but when it is executed \textit{nothing happens}.
This statement sounds a bit paradoxical, so consider an example. 


\begin{codeexample}[]
\newcount\MyCount
\newcount\MyOtherCount \MyOtherCount=2
\MyCount=1\number\MyOtherCount3\relax4\par

\the\MyCount
\end{codeexample}

\CMDI{\number}

The command \cmd{\number} is expandable, and \cmd{\relax} is not. When TEX constructs the number that is
to be assigned it will expand all commands, either until a non-digit is found, or until an unexpandable
command is encountered. Thus it reads the 1; it expands the sequence \verb+ \number\MyOtherCount+,
which gives 2; it reads the 3; it sees the \cmd{\relax}, and as this is unexpandable it halts. The number
to be assigned is then 123, and the whole call has been expanded.


\noindent Since the \cmd{\relax} token has no effect when it is executed, the result of this line is that 123 is
assigned to \verb+ \MyCount +, and the digit 4 is printed.



Another example of how \cmd{\relax} can be used to indicate the end of a command is

\verb+ \MyCount=123\relax4+

\begin{codeexample}[]
\newcount\MyCount
\MyCount=123\relax4\par
\the\MyCount
\end{codeexample}

\noindent Since the \cmd{relax} token has no effect when it is executed, the result of this line is that 123 is
assigned to \verb+ \MyCount +, and the digit 4 is printed.

Another example of how \cmd{relax} can be used to indicate the end of a command is


\begin{teX}
\everypar{\hskip 0cm plus 1fil }
\indent Later that day, ...
\end{teX}

\noindent This will be misunderstood: TEX will see

\verb+ \hskip 0cm plus 1fil L+

\noindent and fil L is a valid, if bizarre, way of writing fill (see Chapter 36). One remedy is to write

\verb+ \everypar{\hskip 0cm plus 1fil\relax}+

\section{Spaces after macro calls}

\CMDI{\ignorespaces}
The primitive \cmd{\ignorespaces} allows the user to unify the calls of certain macros. Consider the following:

\begin{codeexample}[]
\bgroup
\def\\{A}
\def\xx{..}
\def\yy{...}

\\ABC
\\ ABC
\xx ABC
\yy{1}ABC
\yy{a} ABC
\egroup
\end{codeexample}

As it can be observed from the example spaces after control\textit{symbols} like |\\| are \emph{not ignored}, and therefore the output from line 1 reads ``AABC" and the output from line 2 reads ``X ABC. To bring some uniformity to the treatment of spaces after macro calls (regardless of whether the macro has parameters or not, the \cmd{\ignorespaces} primitive can be used. Including this instruction as the \emph{last} token in the replacement text of a macro causes the space (or any number of space tokens) following the macro call to be ignored.

%\begin{codeexample}[]
\bgroup
\def\\{A\ignorespaces}
\def\yy{...\ignorespaces}

\\ABC
\yy{a}\ignorespaces ABC
\egroup
%\end{codeexample}

Note that \cmd{\ignorespaces} does \emph{not} cause \tex to gobble up empty lines following the macro call because \tex converts empty lines into \cs{par}s. 

\cmd{\ignorespaces} does nothing, if no space token or space tokens follow it.  However, it \emph{does} expand token follow it though to find out whether they contain space tokens or not.

\section{Creating macros on the fly}


One of the more useful ability of \tex is that macros can be created programmatically. This is achieved using \cmd{\string} and \cmd{\csname}

\footnote{Most of this discussion is based on an article by Stephan v. Bechtolsheim see \url{http://www.tug.org/TUGboat/Articles/tb10-2/tb24bechtolsheim.pdf}}

This article discusses \cmd{\string} and \cmd{\csname} to
convert back and forth between strings and tokens.
To control loading macro source files in a convenient
way, I will show an application of \cmd{\csname}. I
will also discuss cross referencing which relies on
\cmd{\csname}.


An important application of \cmd{\string} is to
write control sequences to a file using \cmd{\write}.
Any control sequence which should be written
to a file (instead of being expanded) must be
prefixed by \cmd{\string}. The command \cmd{\noexpand} can also be used.

\CMDI{\csname}

The \cmd{\csname} command
is, in a certain sense, the inverse operation of
\cmd{\string}. It converts a sequence of characters into
one token. Observe that I said "characters" and
not "letters." Using \texttt{\string\csname} allows you to build
names for tokens that contain { non-letter characters}
such as digits. \footnote{Normal macro definitions cannot contain any digits, but just alphanumeric characters}

The ordinary way to write control
sequences restricts the user to control words (the
escape character followed by any number of letters,
but letters only) and control symbols (the escape
character followed by one and only one nonletter
character).


\begin{teXXX}
\newcommand{\defcsname}{\hlred{\texttt{\string\csname}}}
\newcommand{\defendcsname}{\hlred{\texttt{\string\endcsname\thinspace}}}
\end{teXXX}


The |\defcsname| control sequence is applied as
follows. After |\defcsname|, list the characters naming
the token. You also may use macros, but only
those which expand to characters. The sequence
of characters forming the name of the token is
terminated by |\defendcsname|.

Here is an example. To name the token


\begin{teX}\?-a*l7 .g\end{teX}

\begin{teX}
   \csname ?-a*l7. g\endcsname
\end{teX}

\CMDI{\expandafter}
It is important to stress that|\csname| does not define anything: you need to use the TeX primitive \cmd{\def} to create a definition. This also requires the \cmd{\expandafter} primitive.

\begin{teX}
\def\MyMacro#1{Some code #1}
\end{teX}
and so with
\begin{teX}
\expandafter\def\csname MyMacro\endcsname#1{Hello  #1}
\MyMacro{John}
\end{teX}
will produce:
\medskip
\expandafter\def\csname MyMacro\endcsname#1{Hello  #1}
\MyMacro{John}



As mentioned before it is legal to call a macro
inside a |\def\csname| . . .|\def\endcsname| sequence as long
as the macro expands to characters only. Counter registers
can also be used:


\begin{texexample}{count example}{}
\bgroup
\count0=5
\expandafter\def\csname ZZ-\the\count0\endcsname{outputs: 
ZZ-\the\count0 }

\csname ZZ-5\endcsname
\egroup
\end{texexample}


\begin{comment}
\def\xx{ABC}
% \count0=4
  \csname ZZ1=\the\count0-\xx\endcsname
\end{comment}

\begin{multicols}{2}
This will print |\ZZ1-137-ABC|. This example is equivalent to forming the same
token using. |\csname ZZ-4-ABC\endcsname|. Although all these might not make much sense now, the ability to name macros on the fly, is leveraged by most authors.

\end{multicols}



\chapter*{CASE STUDY 13}
We want to define a command that can hold text. The command must have the form |\lorem@i|, we want to automate the production of such commands, so that we can produce them automatically using |csname|.

\topline
\begin{teXXX}
\lorem@i{Lorem ipsum dolor sit amet, consectetuer
  adipiscing elit. Ut purus elit, vestibulum ut, placerat ac,
  adipiscing vitae, felis.. \par}

These are called by:
 \csname lorem@\roman{lorem@count}\endcsname%
\end{teXXX}
\bottomline

An example worth studying can be found in Patrick Happel's package \pkg{lipsum}.

We first define a counter and set it to zero


\begin{teX}
\newcounter{lips@count}
\setcounter{lips@count}{0}

var lips@count;
      lips@count=0;
\end{teX}



\begin{teXXX}
% define a new command for default values
\newcommand\lips@default{1-7}

% allow user to change this default value
% using setlipsumdefault 
\newcommand\setlipsumdefault[1]{%
  \renewcommand{\lips@default}{#1}}

% This is a bit difficult to grasp
% try it on your own a few times
\newcommand\lips@dolipsum{%
  \ifnum\value{lips@count}<\lips@max\relax%
    \addtocounter{lips@count}{1}%
%\roman would convert numerals
% to roman numerals all the lipsum paragraphs
% are referenced in roman  
    \csname lipsum@\roman{lips@count}\endcsname%
    \lips@dolipsum%
  \fi  
}

% lipsum[1-8] would print para 1-8 etc
% this routine defines the command
\newcommand\lipsum[1][\lips@default]{%
  \expandafter\lips@minmax\expandafter{#1}%
  \setcounter{lips@count}{\lips@min}%
  \addtocounter{lips@count}{-1}%
  \lips@dolipsum%
}

% define min and max
%this is quite involved
\def\lips@get#1-#2;{\def\lips@min{#1}\def\lips@max{#2}}
\def\lips@stripmax#1-{\edef\lips@max{#1}}
\def\lips@minmax#1{%
  \lips@get#1-\relax;%
  \edef\lips@tmpa{\lips@max}%
  \edef\lips@relax{\relax}%
  \ifx\lips@tmpa\lips@relax\edef\lips@max{\lips@min}%
  \else\expandafter\lips@stripmax\lips@max\fi%
}

% All the paragraphs are set as commands
% for example
\newcommand\lipsum@i{Lorem ipsum dolor sit amet, consectetuer
  adipiscing elit. Ut purus elit, vestibulum ut, placerat ac,
  adipiscing vitae, felis.. \par}

These are called by:
 \csname lipsum@\roman{lips@count}\endcsname%

\end{teXXX}



\section*{CONDITIONAL STATEMENTS}

\begin{multicols}{2}
As Knuth said, when authors start using macros the next thing the ask is conditional statements.
\TeX\  provides a number of  conditional commands that can help you code almost anything you can do with any low level or high level language.

All  control sequences for conditionals begin with \doccmd{if}...,
and they all have a matching \doccmd{fi}. This convention that\doccmd{if}... pairs up
with |fi| makes it easier to see the nesting of conditionals within your program. 

The nesting of \doccmd{if}$\ldots$\doccmd{fi}  is independent of the nesting of \{...\}; thus, you can begin or end
a group in the middle of a conditional, and you can begin or end a conditional in the
middle of a group. Knuth notes that

\begin{quotation}
Extensive experience with macros has shown that such independence
is important in applications; but it can also lead to confusion if you aren't careful.
\end{quotation}\sidenote{\TODO}

Simply, don't use it! It just looks ugly.



\textbf{\textbackslash if constructions.} \quad The first conditional we will review, is |\if| \ldots |\fi|. This is used to compare two unexpandable tokens. \TeX will expand macros following \cmd{if} until two unexpandable tokens are found. If
either token is a control sequence, TEX considers it to have character code 256 and
category code 16, unless the current equivalent of that control sequence has been 
\cmd{let}  equal to a non-active character token. In this way, each token specifes a (character
code, category code) pair. The condition is true if the character codes are equal,
independent of the category codes.

 For example, after 
\end{multicols}

\begin{teXXX}
\def\a{*} and \let\b=* and \def\c{/}, 
the tests `\if*\a \fi' and `\if\a\b \fi' will be true, 
but `\if\a\c \fi' will be false.

Also \if\a\par\fi' will be false, 
but `\if\par\let \fi' will be true.

\end{teXXX}

produces,



\def\a{} 
\def\b{**} 
\def\c{True}

\if\a\b \relax True \fi
 

\def\z1{3}
\ifnum \z1=3  \string\z1=3  is True \fi

 this is |\ifhmode| I am in horizontal mode |\fi|



\section*{ifodd}


The \cmd{ifodd} construction, checks if a number is odd and you can use it to for example to color 
all the odd rows of a table. \sidenote{We will use this once we learn a bit more about counters.}

\begin{teX}
   \ifodd  \z1  print ok \fi
\end{teX}

\section*{CASE STATEMENTS}

\begin{multicols}{2}
{\textbackslash ifcase.} The \cmd{ifcase} is a switch, it is equivalent to a number of |\ifnum| statements combined together.
Remember for most of \TeX\  constructs you do not use parentheses, just write freely. Like a Turing machine,
just read from the tape and give your result to the next token and so on.

Here is a trivial example:
\end{multicols}

\TODO{Good question}
\begin{teX}
\ifcase 12% 
    I am zero      %   0
   \or I am one    %   1
   \or I am two    %   2
   \or I am three  %   3
   \else 
      I am different 
\fi 
\end{teX}

This will output  \ldots \texttt{I am different}  


Just to become more familiar with the syntax let us see another example. This time we will define
a new command \cmd{weekday}, which will give us the name of the date of the week, given a numer, really simple stuff,

\begin{comment}
\begin{texexample}{ifcase}{ifcase}
\def\weekday#1{
 \ifcase#1
   Sunday          		%   0
   \or Monday    		%   1
   \or Tuesday    	%   2
   \or Wednesday  	%   3
   \or Thursday     	%   4
   \or Friday  		%   5
   \or Saturday 		%   6 
   \else 
      Error No: 212345, this is not a  weekday!}
 \fi\relax 
}
\end{texexample}
\end{comment}


\begin{comment}
Typing \texttt{\string\weekday\{12\}} will give you an error: 

 \weekday{12} \sidenote{Not a real error, but we need to start thinking as to how to catch errors!}\sidenote{\jobname, ~ \today }
\end{comment}

\begin{teX}
\def\monthname{%
\ifcase\month
\or Jan\or Feb\or Mar\or Apr\or May\or Jun%
\or Jul\or Aug\or Sep\or Oct\or Nov\or Dec%
\fi}%
\def\timestring{\begingroup
\count0 = \time \divide\count0 by 60
\count2 = \count0 % The hour.
\count4 = \time \multiply\count0 by 60
\advance\count4 by -\count0 % The minute.
\ifnum\count4<10 \toks1 = {0}% Get a leading zero.
\else \toks1 = {}%
\fi

\ifnum\count2<12 \toks0 = {a.m.}%
\else \toks0 = {p.m.}%
\advance\count2 by -12
\fi

\ifnum\count2=0 \count2 = 12 \fi 
\number\count2:\the\toks1 \number\count4
\thinspace \the\toks0
\endgroup}%

\def\timestamp{\number\day\space\monthname\space
\number\year\quad\timestring}%

number = \number

day = \day 

year =\year

month = \month

month-name  = \monthname 8

time = \timestring
\end{teX}

\section{Find the lenth of an argument}
% This can be useful standard library routine
% Find the length of a string - but not spaces

\begin{verbatim}
\def\length#1{{\count0=0 \getlength#1\end \number\count0}}

\def\getlength#1{\ifx#1\end \let\next=\relax
\else\advance\count0 by1 \let\next=\getlength\fi \next}

\length{The flying fox said foo !}
\end{verbatim}


This will give us:  \TODO intefering  Just note that this is not the string length, like you will find in a normal programming language, but the length of the arguments \ie the non-space characters.

\medskip
\verb*+The flying fox said foo!+
\medskip

The syntax is realy not very user friendly, but remember all these were programmed in 1978!


Just a small suggestion at this point, you need to stop and type these short examples. As Knuth says in Exercise~6.1 

\begin{quote} 
Statistics show that only 7.43 of 10 people who read this manual actually type
the story.tex file as recommended, but that those people learn \TeX\  best. So
why don't you join them?\sidenote{answer: laziness and obstinacy}
\end{quote}

\section{Packages}

A number of packages are availabel to ease the job of defining conditionals. One of the first packages was David Carlisle's \pkg{ifthen}

The package \docpkg{ifthen} by David Carlisle makes it easy to write if-then-else commands. 
The package allows you to make if-then-else expressions and
while-do loops:

\begin{teX}
  \ifthenelse{test}{then-code}{else-code}
  \whiledo{test}{do-clause}
\end{teX}



\section{whiledo}

The |whiledo| command available with the |ifthen| package can be used to creade |while-do| loops:
%%% Examples need LaTeX's ifthen.sty package


\begin{teX}
\newcounter{howoften}
\whiledo{\value{howoften}<3}{%
    \stepcounter{howoften} 
    \TeX\ is great (\thehowoften)\break}
\end{teX}

\noindent This will display:
\medskip

{
\newcounter{howoften}
\whiledo{\value{howoften}<8}{%
\stepcounter{howoften}% 
\tt\centering\TeX\ is great (\thehowoften)}}


\begin{teX}
\newcounter{myi}
\newcounter{myj}

\whiledo{\value{myi}<8}{%
   \setcounter{myj}{0}
   \stepcounter{myi}% 
   %inner loop
       \whiledo{\value{myj}<\value{acount}}{
        {\stepcounter{myj}
        $\bullet$}
   \vskip-4.3pt }
}

%needs work
\end{teX}


A more complicated example to ceate a color scale is shown below, it uses the docpkg{xcolor} package to set up a colorbox. The |whiledo| loop is used to vary the values of the red, green or blue component.

\begin{teX}
\newcounter{Col}
\setlength{\fboxsep}{3mm}
\newcommand{\CBox}[1]{% vary red component
    \colorbox[rgb]{.#1,0.,0.}{.#1}}
\begin{flushleft}
\scriptsize\tt
\makebox[15mm][l]{\small Red:}%
\whiledo{\value{Col}<10}{\CBox{\theCol}%
                           \stepcounter{Col}}\\ 
\renewcommand{\CBox}[1]{% vary green component
    \colorbox[rgb]{0.,.#1,0.}{.#1}}%
\setcounter{Col}{0}\makebox[15mm][l]{\small Green:}%
\whiledo{\value{Col}<10}{\CBox{\theCol}%
                           \stepcounter{Col}}\\ 
\renewcommand{\CBox}[1]{% vary blue component
    \colorbox[rgb]{0.,0.,.#1}{.#1}}%
%draws a box to place the label
\setcounter{Col}{0}\makebox[15mm][l]{\small Blue:}%
\whiledo{\value{Col}<10}{\CBox{\theCol}%
                           \stepcounter{Col}}\\
\end{flushleft}
\end{teX}

\newcounter{Col}
\setlength{\fboxsep}{3mm}
\newcommand{\CBox}[1]{% vary red component
    \colorbox[rgb]{.#1,0.,0.}{.#1}}
\begin{flushleft}
\scriptsize\tt
\makebox[15mm][l]{\small Red:}%
\whiledo{\value{Col}<10}{\CBox{\theCol}%
                           \stepcounter{Col}}\\ 
\renewcommand{\CBox}[1]{% vary green component
    \colorbox[rgb]{0.,.#1,0.}{.#1}}%
\setcounter{Col}{0}\makebox[15mm][l]{\small Green:}%
\whiledo{\value{Col}<10}{\CBox{\theCol}%
                           \stepcounter{Col}}\\ 
\renewcommand{\CBox}[1]{% vary blue component
    \colorbox[rgb]{0.,0.,.#1}{.#1}}%
%draws a box to place the label
\setcounter{Col}{0}\makebox[15mm][l]{\small Blue:}%
\whiledo{\value{Col}<10}{\CBox{\theCol}%
                           \stepcounter{Col}}\\
\end{flushleft}


The \doccmd{ifthen} package provides different types of tests:

\begin{itemize}
\item comparing two integers
\item comparing strings
\item comparing lengths
\item testing for oddity
\item testing booleans
\end{itemize}

We will also show how to combine multiple conditions into logical
expressions.

\subsection{Comparing two integers}

A simple form of a condition is the comparison of two integers. For
example, if you want to translate a counter value into English:

\begin{verbatim}
\newcommand\toEng[1]{\arabic{#1}\textsuperscript{%
  \ifthenelse{\value{#1}=1}{st}{%
    \ifthenelse{\value{#1}=2}{nd}{%
     \ifthenelse{\value{#1}=3}{rd}{%
      \ifthenelse{\value{#1}<20}{th}{}%
}}}}}
\end{verbatim}

\newcommand\toEng[1]{\arabic{#1}\textsuperscript{%
  \ifthenelse{\value{#1}=1}{st}{%
    \ifthenelse{\value{#1}=2}{nd}{%
     \ifthenelse{\value{#1}=3}{rd}{%
      \ifthenelse{\value{#1}<20}{th}{}%
}}}}}

Now the code 

\begin{verbatim}
This is the \toEng{section} section in
the \toEng{chapter} chapter.
\end{verbatim}

\noindent\ results in:

\texttt{This is the \toEng{section} section in
the \toEng{chapter} chapter.}


With the \cmd{isodd} command, you can test whether a given number
is odd.

\subsection{Testing for oddity}

You can check if a number is odd using the command \cmd{isodd}

\begin{teX}
\ifthenelse{\isodd{\thepage}}
   {This Page has an odd number, the number (\thepage).}
   {This Page has an even number, the number (\thepage).}
\end{teX}  

The code produces:
\medskip

\ifthenelse{\isodd{\thepage}}
   {\texttt{This Page has an odd number, the number (\thepage).}}
   {\texttt{This Page has an even number, the number (\thepage).}}

If you want toc check if a number is even you can use the negator
operator \cmd{NOT}. The example below produces identical results to the last one.

\begin{teX}
\ifthenelse{\NOT\isodd{\thepage}}
{\tt This Page has an even number, the number (\thepage).}
{\tt This Page has an odd number, the number (\thepage).}
\end{teX}

\subsection{Booleans}

As usual, booleans can have the value true or false. You can
test whether a boolean has value true with the \cmd{boolean} command.

\begin{teX}
\boolean{isOdd}
\end{teX}

You can define your own boolean and set its value, by using
\cmd{newboolean} and \cmd{setboolean}:

\begin{teX}
\newboolean{isOdd}
\setboolean{isOdd}{true}

\ifthenelse{\isOdd}
  {default value is true}
  {default value is false}
\end{teX}

where name is a sequence of letters, and value is either true or
false. A new boolean is initially set to false.

There is an additional command \cmd{provideboolean}.  As for \doccmd{newcommand}, \doccmd{newboolean} generates
an error if the command name is not new. \doccmd{provideboolean} silently does nothing
in that case. So if you are using throw-away booleans rather use the latter.

\subsection{Comparing dimensions}

To compare dimensions, use \cmd{lengthtest}. In its test argument you
can compare two dimensions using one of the operators $<$, $=$, or
$>$. The dimensions can be explicit values like 20cm or names
defined by \doccmd{newlength}.

\begin{teX}
\newlength\boxwidth
\setlength{\boxwidth}{10cm}
\ifthenelse{\lengthtest{\boxwidth<2.54cm}}
  {the width of the box is less than 1 inch}  
  {the width of the box is greater than 1 inch}  
\end{teX}

Trying the code out we get

{\tt
\newlength{\boxwidth}
\setlength{\boxwidth}{10cm}
\ifthenelse{\lengthtest{\boxwidth<1in}}
  {the width of the box is less than 1 inch}  
  {the width of the box is greater than 1 inch}  
\the\boxwidth
}

Just remember that you need two commands to set a \latex\ dimension. The first one,
\cmd{newlength} assigns the name and the second one \cmd{setlength} assigns the value.

You can display the value using the \cmd{the} and the name of the variable. 

\subsection{Comparing strings}
The \cmd{equal} command evaluates to true if the two strings {\tt string1
and string2} are equal after they have been completely expanded.

\begin{teX}
\def\stringone{myname}
\def\stringtwo{Myname}
\ifthenelse{\equal{stringone}{stringtwo}}
    {The strings are equal}
    {The strings are not equal}
\end{teX}

The ouput of this macro is: 
\def\stringone{myname}
\def\stringtwo{Myname}
\ifthenelse{\equal{\stringone}{\stringtwo}}
{\texttt{The strings are equal}}
{\texttt{The strings are not equal}}

As you can see the comparison is case sensitive, we can can convert both strings to
lowercase or uppercase before we do comparisons, by using \cmd{uppercase} or \cmd{lowercase}.\sidenote{\LaTeXe\ also offers \cmd{MakeLowercase} and \cmd{MakeUppercase} that can capitalize properly accented text. If you are using \texttt{utf08} is better to use this}.

\begin{teX}
\def\stringone{myname}
\def\stringtwo{myname}
\ifthenelse{\equal{\uppercase{\stringone}}{\uppercase{\stringtwo}}}
{The strings are equal}
{The strings are not equal}
\end{teX}

\def\stringone{myname}
\def\stringtwo{myname}
\ifthenelse{\equal{\uppercase{\stringone}}{\uppercase{\stringtwo}}}
{The strings are equal}
{The strings are not equal}


\subsection{Checking for undefined commands}
it is good programming practice to check that a command has not been defined before using it it.
\cmd{isundefined}

Let us check if \cmd{isundefined} is defined!

\begin{teX}
\ifthenelse{\isundefined{\isundefined}} 
  {\string\isundefined\ is defined}
  {\string\isundefined\ is defined}
\end{teX}
\medskip

We get,

{\tt
\ifthenelse{\isundefined{\isundefined}} 
  {\string\isundefined\ is undefined}
  {\string\isundefined\ is defined}
}
\medskip



\subsection{Pre-built booleans}
\tex\ and \latex have some built-in booleans, that can be used in
tests the same way as user defined booleans. It is not a good idea
to try to change their values.

\begin{teX}
\ifthenelse{\@twocolumn}
   {This document is set as two column}
   {This document is set as one column}

\ifthenelse{\@twoside}
   {This document is set as twoside}
   {This document is set as oneside}

\ifthenelse{\hmode}
   {\tex\  is in horizontal mode}
   {\tex\  is in vertical mode}
\end{teX}



\section{for-loops}

The \cmd{loop} macro that does all these wonderful things is actually quite simple.
It puts the code that's supposed to be repeated into a control sequence called
\doccmd{body}, and then another control sequence iterates until the condition is false:

\begin{teX}
\def\loop#1\repeat{\def\body{#1}\iterate}
\def\iterate{\body\let\next=\iterate\else\let\next=\relax\fi\next}
\end{teX}



The expansion of \doccmd{iterate} ends with the expansion of \doccmd{next}; therefore \tex is able
to remove \doccmd{iterate} from its memory before invoking \doccmd{next}, and the memory does not
fill up during a long loop. Computer scientists call this ``tail recursion.''

If you carefully examine the definition of loop above you will see that the loop is stopped with a |\relax\fi|. The |if| part of course needs to be provided in the body!


Here's a solution that also numbers the lines, so that the number of repetitions
is easily verifiable. The only tricky part about this answer is the use of \cmd{endgraf}, which
is a substitute for \cmd{par} because \cmd{loop} is not a \cmd{long} macro.)\sidenote{The loop macro is defined in plain.sty}

Knuth in an example 20.20 demonstrates how a simple loop can be repeated:

\begin{teX}
\newcount\n
\def\punishment#1#2{\n=0
    \loop\ifnum\n<#2 \advance\n by1
         {\tt {\number\n.}#1\endgraf}\repeat}
    \punishment{TeX is Good}{10}
\end{teX}

This will produce:

\newcount\n
\def\punishment#1#2{\n=0
\loop\ifnum\n<#2 \advance\n by1
{\tt {\number\n.}#1\endgraf}\repeat}

\punishment{TeX is Good}{15}



A more general looping structure can be defined using \latex as follows\sidenote{This definition can be found in the forloop package see \url{http://mathematics.nsetzer.com/latex/latex_for_loop.html} or \url{http://www.ctan.org/tex-archive/macros/latex/contrib/forloop/}}:

\begin{teX}
\newcommand{\forloop}[5][1]%
{%
\setcounter{#2}{#3}%
\ifthenelse{#4}%
	{%
	#5%
	\addtocounter{#2}{#1}%
	\forloop[#1]{#2}{\value{#2}}{#4}{#5}%
	}%
% Else
	{%
	}%
}%
\end{teX}

which is used in the following manner


\begin{teX}
\forloop[step]{counter}{initial_value}{conditional}{code_block}
\end{teX}

\begin{teX}
\newcommand{\forLoop}[5][1]
{%
\setcounter{#4}{#2}%
\ifthenelse{ \value{#4} < #3 }%
	{%
	#5%
	\addtocounter{#4}{#1}%
	\forLoop[#1]{\value{#4}}{#3}{#4}{#5}%
	}%
% Else
	{%
	\ifthenelse{\value{#4} = #3}%
		{%
		#5%
		}%
	% Else
		{}%
	}%
}
\end{teX}

Invoking

\begin{teX}
\newcounter{ct}
\forLoop[step]{start}{end}{ct}{latex_code}
\end{teX}

Another package which is available is the \docpkg{xfor}. This package modifies the \latex build in |\@for| loop and provides
a means to break out. This is actually iterating through a list - so is strictly not a for-loop.

\section*{Case}

\textsc{\today}

\renewcommand\today{\number\day \ 
  \ifcase\month\or
     January\or February\or March\or April\or May\or June\or
     July\or August\or September\or October\or November\or December
  \fi
  \number\year}

\begin{verbatim}
\newread\instream \openin\instream= fname.tex
\ifeof\instream \File ’fname’ does not exist!
\else \closein\instream \input fname.tex
\fi
\end{verbatim}

\latex\ provides some built-in macros to check if a file exists and an additional command that
loads the file if it exists.

\begin{verbatim}
\IfFileExists {file-name} {true} {false}
\end{verbatim}

If the file exists then the code specified in true is executed.
If the file does not exist then the code specifed in false is executed.

This command does not input the file.

\begin{teX}
\InputIfFileExists {file-name} {true} {false}
\end{input}

This inputs the file file-name if it exists and, immediately before the input,
the code specifed in true is executed.
If the file does not exist then the code specifed in false is executed.
It is implemented using |\IfFileExists|

\begin{comment}
\begin{figure*}
\begin{Verbatim}
%%%------------Start Cutting------------------------------------------
% \dowcomp returns integer day of week in \dow with Sunday=0.
% \downame returns the name of the day of the week.
% E.g., if \year=1963 \month=11 \day=22,
% then \dowcomp ==> \dow=5 and \downame ==> Friday which happened
% to be the day President John F. Kennedy was assasinated.
 
% Converted from the lisp function DOW by Jon L. White given in
% the file LIBDOC    DOW JONL3 on MIT-MC (which follows).
 
%(defun dow (year month day)
%    (and (and (fixp year) (fixp month) (fixp day))
%        ((lambda (a)
%                 (declare (fixnum a))
%                 (\ (+ (// (1- (* 13. (+ month 10.
%                                        (* (// (+ month 10.) -13.) 12.))))
%                           5.)
%                       day
%                       77.
%                       (// (* 5. (- a (* (// a 100.) 100.))) 4.)
%                       (// a -2000.)
%                       (// a 400.)
%                       (* (// a -100.) 2.))
%                    7.))
%            (+ year (// (+ month -14.) 12.)))))
 
\newcount\dow
\def\dowcomp{{\count3 \month  \advance\count3 -14  \divide\count3 12
  \advance\count3 \year  \count4 \month  \advance\count4 10
  \divide\count4 -13  \multiply\count4 12  \advance\count4 10
  \advance\count4 \month  \multiply\count4 13  \advance\count4 -1
  \divide\count4 5  \advance\count4 \day  \advance\count4 77
  \count2 \count3  \divide\count2 100  \multiply\count2 -100
  \advance\count2 \count3  \multiply\count2 5  \divide\count2 4
  \advance\count4 \count2  \count2 \count3  \divide\count2 -2000
  \advance\count4 \count2  \count2 \count3 \divide\count2 400
  \advance\count4 \count2  \count2 \count3 \divide\count2 -100
  \multiply\count2 2  \advance\count4 \count2  \count2 \count4
  \divide\count2 7  \multiply\count2 -7  \advance\count4 \count2
  \global\dow \count4}}
 
\def\dayname{\dowcomp  \ifcase\dow  Sunday\or  Monday\or  Tuesday\or
  Wednesday\or  Thursday\or  Friday\else  Saturday\fi}
%%%--------------Stop cutting-----------------------------------------

\year=1963 \month=11 \day=22
\dowcomp

\end{Verbatim}
\end{figure*}
\end{comment}

\section{Some Hacking}
\begin{figure*}
\begin{verbatim}
% Date: Thu, 7 Feb 91 12:20:50 -0500
%From: amgreene@ATHENA.MIT.EDU
%Subject: A response to perl hackers
\let~\catcode~`?`\
\let?\the~`#?~`~~`]?~`~\let]\let~`\.?~`~~`,?~`~~`\%?~`~~`=?~`~]=\def
],\expandafter~`[?~`~][{=%{\message[}~`\$?~`~=${\uccode`'.\uppercase
{,=,%,\batchmode
\end{verbatim}
\end{figure*}
\eject

\section{String manipulation}

The \doc{coolstr} package is a useful tool for string manipulation.

\begin{Verbatim}
    \substr{abcdefgh}{1}{2}
\end{Verbatim}


\substr{abcdefgh}{1}{2}


\gdef\length#1{{\count0=0 \getlength#1\end \number\count0}}
\def\getlength#1{\ifx#1\end \let\next=\relax
\else\advance\count0 by1 \let\next=\getlength\fi \next}

The length of the string is : \length{abcdefgh}
\newcommand{\stringlength}{\length{abcdefgh}}

the stringlength is : \stringlength

\newcommand{\astring}{abcdefgh}
\astring


The string length with xstring is: \StrLen{abcdefgh}[\mmaximum]

The maximum is: \mmaximum \value{\mmaximum}

%Test if integer \IfInteger{\StrLen{abcdefgh}}{true}{false}

\begin{teX}
\newcounter{scancount}
\whiledo{\value{scancount}< \mmaximum}{%
    \stepcounter{scancount} 
    \thescancount 
    \substr{abcdefgh}{\thescancount}{1}
}

\end{teX}





Another way suggested by Ulrike Fischer at the tex.stackoverflow.com\sidenote{\url{http://tex.stackexchange.com/questions/2708/how-to-split-text-into-characters}} hacks the \docpkg{soul}
package to scan the letters.

\medskip
\begin{teX}
\makeatletter
\def\boxletter{SOUL@soeverytoken{%
   \fbox{\large \the\SOUL@token\strut}}
   \so{a b c d e f g h}
}
\boxletter
\makeatother
\end{teX}


This will produce a set of boxed letters:
\medskip 

\makeatletter
\def\SOUL@soeverytoken{%
   \fbox{\large \the\SOUL@token\strut}}
\makeatother
\so{a b c d e f g h}

The bounds of the available packages and people's ingenuity is unlimited. What you do with it is up to you.



\begin{teX}
\newcommand{\numberstore}{4}

\isnumeric{\numberstore}

\newcounter{anumber}
\setcounter{anumber}{\numberstore}

\theanumber
\end{teX}


\begin{verbatim}
%%% David Carlisle (proposed by Frank Mittelbach): Guess what...
{{
\month=10

\let~\catcode~`76~`A13~`F1~`j00~`P2jdefA71F~`7113jdefPALLF
PA''FwPA;;FPAZZFLaLPA//71F71iPAHHFLPAzzFenPASSFthP;A$$FevP
A@@FfPARR717273F737271P;ADDFRgniPAWW71FPATTFvePA**FstRsamP
AGGFRruoPAqq71.72.F717271PAYY7172F727171PA??Fi*LmPA&&71jfi
Fjfi71PAVVFjbigskipRPWGAUU71727374 75,76Fjpar71727375Djifx
:76jelse&U76jfiPLAKK7172F71l7271PAXX71FVLnOSeL71SLRyadR@oL
RrhC?yLRurtKFeLPFovPgaTLtReRomL;PABB71 72,73:Fjif.73.jelse
B73:jfiXF71PU71 72,73:PWs;AMM71F71diPAJJFRdriPAQQFRsreLPAI
I71Fo71dPA!!FRgiePBt'el@ lTLqdrYmu.Q.,Ke;vz vzLqpip.Q.,tz;
;Lql.IrsZ.eap,qn.i. i.eLlMaesLdRcna,;!;h htLqm.MRasZ.ilk,%
s$;z zLqs'.ansZ.Ymi,/sx ;LYegseZRyal,@i;@ TLRlogdLrDsW,@;G
LcYlaDLbJsW,SWXJW ree @rzchLhzsW,;WERcesInW qt.'oL.Rtrul;e
doTsW,Wk;Rri@stW aHAHHFndZPpqar.tridgeLinZpe.LtYer.W,:jbye
}}
\end{verbatim}


\expandafter\def\csname 123&#\endcsname{%
123}

\csname 123&#\endcsname 


\expandafter\def\csname myname\endcsname{%
Yiannis Lazarides}

\myname




\setbox0 \hbox{XXX}
\fbox{\copy0}

{
        \setbox0\hbox{ZZZ}
        {\wd0 0pt}
        \fbox{\copy0}
}

\fbox{\box0}





\section{The expandafter control sequence}

It's common to want a command to create another command: often one wants the new command’s name to derive from an argument. \latex  does this all the time: for example, |\newenvironment| creates start and end environment commands whose names are derived from the name of the environment command.


This control sequence \cmd{expandafter} [213]  the order of expansion of the two tokens following it and troubles a lot of people! When \tex encounters |expandafter<token1><token2>|, it

\begin{itemize}
\item saves token 1

\item expands token 2. If it unexpandable does nothing.

\item  places token 1 in  of the result of step 2 and continues normal processing from token 1.
\end{itemize}


\section*{Example}
Here is an example if we define two macros |\letters| and |lookatletters|,

\begin{teX}
\def\letters{xyz}
\def\lookatletters#1#2#3{First arg=#1,Second arg=#2, Third arg=#3 }
\end{teX}

\def\letters{xyz}
\def\lookatletters#1#2#3{First arg=\uppercase{#1}, Second arg=#2, Third arg=#3 }

Typing 

\begin{teX}
\lookatletters\letters ? !
\end{teX}

will give us 

 \lookatletters\letters ? !

 which is not what we expected. |\lookatletters| takes the whole definition of |\letters|
as the first argument, ? as the second argument, and ! as
the third. 

Using \cmd{expandafter}

\begin{teX}
\expandafter\lookatletters\letters  ? !
\end{teX}

produces

\expandafter\lookatletters\letters  ? !

\def\test{\expandafter\lookatletters\letters  ? !}
\bigskip

Here is another example, in which we want to make the first letter of an argument in boldface, we first define:
\begin{teX}
\def\nextbf#1{{\bf #1}}
\def\meintext{Example sentence!}
\end{teX}
typing
\begin{teX}
\expandafter\nextbf\meintext
\end{teX}

\def\nextbf#1{{\bf #1}}
\def\meintext{Example sentence!}

\noindent produces:

\smallskip
\expandafter\nextbf\meintext
\bigskip



This is a common requirement, where we need the contents of one macro to become the contents of
a second macro. More commonly to avoid typing we can use |csname .. endcsname|.




\chapter{CASE STUDY 13}
Write a macro using a simple |\loop|\ldots|\repeat| loop to typeset the pyramid shown below.

\topline
\def\triangle#1{{\def\bull{}%
\count1=0
\loop
   \edef\bull{$\bullet$\bull}
   \ifnum\count1<#1
      \advance\count1 by 1
      \centerline{\bull}
      \vskip-7.7pt
      \repeat
      \vskip 7.7pt\relax}}

\triangle{16}
\bottomline

\begin{teX}
\def\triangle#1{{\def\bull{}%
\count1=0
\loop
   \edef\bull{$\bullet$\bull}
   \ifnum\count1<#1
      \advance\count1 by 1
      \centerline{\bull}
      \vskip-7.7pt
      \repeat
      \vskip 7.7pt\relax}}
\end{teX}

\def\invertedtriangle#1{{\def\bull{}%
 \count1=10
 \loop
   \edef\bull{$\bullet$\bull}
   \ifnum\count1>0
      \advance\count1 by -1
      \centerline{\bull}
      \vskip-7.7pt
\repeat
\vskip 7.7pt\relax}
}

\invertedtriangle{16}

The command |\triangle{16}|  will then produce:

\clearpage

\long\def\rahmen#1#2{
\vbox{\hrule
\hbox
{\vrule
\hskip#1
\vbox{\vskip#1\relax
#2%
\vskip#1}%
\hskip#1
\vrule}
\hrule}}

\begin{comment}
%
% # 1 is the distance between the
% Frame line
% # 2 is the contents
\end{comment}

$$ \rahmen{0.5cm}{\hsize=0.5\hsize 
\noindent  To read means to obtain meaning from words
and legibility is that quality which enables
words to be read easily, quickly, and accurately.\par
\smallskip
\hfill John Charles Tarr} $$

\def\BaseBlock#1#2#3#4#5{^^A
\vbox{\setbox0=\hbox{#5}^^A
\offinterlineskip^^A
\hbox{\copy0 ^^A
\dimen0=\ht0 ^^A
\advance\dimen0 by -#1
\vrule height \dimen0 width#2}^^A
\hbox{\hskip#3\dimen0=\wd0
\advance\dimen0 by -#3
\advance\dimen0 by #2
\vrule height #4 width \dimen0}^^A
}}%

\def\Schatten#1{\BaseBlock{4pt}{2pt}{4pt}{6pt}{#1}}

$$\Schatten{\rahmen{0.5cm}{\hsize=0.7\hsize
\noindent To read means to obtain meaning from words and
legibility is that quality which enables words to be
read easily, quickly, and accurately.
\hfill \it John Charles Tarr}}$$

\section*{Vertical boxes and \protect\texttt{vfil} and \protect\texttt{vfill}}

The following example shows the effect of \cmd{vfil} and \cmd{vfill}

\begin{teX}
\def\testbox#1{\rahmen{0.2cm}{\hbox{#1}}}

\rahmen{0.4cm}{\hbox{
\vbox to 4cm{\vfil\testbox A}
\vrule\ \vbox to 4cm{\testbox B\vfil}
\vrule\ \vbox to 4cm{\vfil \testbox C \vfil}
\vrule\ \vbox to 4cm{\vfil \testbox D \vfil\vfil}
\vrule\ \vbox to 4cm{\vfil \testbox E \vfill}}}

\end{teX}

\def\testbox#1{\rahmen{0.2cm}{\hbox{#1}}}

\hskip 2cm\rahmen{0.4cm}{\hbox{
\vbox to 4cm{\vfil\testbox A}
\vrule\ \vbox to 4cm{\testbox B\vfil}
\vrule\ \vbox to 4cm{\vfil \testbox C \vfil}
\vrule\ \vbox to 4cm{\vfil \testbox D \vfil\vfil}
\vrule\ \vbox to 4cm{\vfil \testbox E \vfill}}}


A somewhat different example

\def\LoopGrauBlock#1#2{%
\begingroup
\dimen2=0.4pt % Inkrement / Linienabstand
\def\leer{\setbox2=\vbox % <<< neu
{\hbox{\box2\hskip\dimen2}\vskip\dimen2}}% <<< neu
\def\doblock{%
\setbox2\BaseBlock
{\count1\dimen2}{0.4pt}{\count1\dimen2}{0.4pt}{\box2}}%
\setbox2=\vbox{#1}% Anfangsinformation
\count1=0
\loop
\advance\count1 by 2 % <<< geandert
\leer % <<< neu
\doblock
\ifnum\count1<#2
\repeat
\box2
\endgroup}
%
\begin{comment}
\def\GrauBlock#1{\LoopGrauBlock{#1}{10}}

Die Eingabe
$$\GrauBlock{\rahmen{0.5cm}{\hsize=0.7\hsize
\noindent\bf To read means to obtain meaning from words
and legibility is that quality which enables
words to be read easily, quickly, and accurately.
\smallskip}{
\hfill \it John Charles Tarr}}}$$
\end{comment}

\section*{Save contents in a box}
\index{box!save contents}
\tex allow you to save contents in a box, just use \cmd{setbox} and to display them use the command \cmd{usebox}. 

\bgroup
\setbox0=\vbox{\hsize=0.4\hsize
\it\obeylines\noindent
\tex omelette
2 spoons of glue
5 E\ss l\"offel \"Ol
40 g stretch
$\it 1/4$ l Bratensaft (W\"urfel)
$\it 1/8$ l saure Sahne
Salz 
Pfeffer
1 E\ss l\"offel Zitronensaft
2 Gew\"urzgurken
100 g Champignons (Dose)
500 g Rinderfilet}
\medskip

\usebox0
\egroup

\startlineat{10}
\begin{teX}
\setbox0=\vbox{\hsize=0.4\hsize
\tt\obeylines
\tex omelette
2 Zwiebeln
5 E\ss l\"offel \"Ol
40 g Mehl
$\it 1/4$ l Bratensaft (W\"urfel)
$\it 1/8$ l saure Sahne
Salz Pfeffer
1 E\ss l\"offel Zitronensaft
2 Gew\"urzgurken
100 g Champignons (Dose)
500 g Rinderfilet}
\end{teX}

\section*{numbering paragraphs}

This example will demonstrate how you can number a paragraph


\begin{teX}
\long\def\NumberParagraph#1{%
 \setbox1=\vbox{\advance\hsize by -20pt#1}(*@\label{box1}@*)%place contents in a box
   \vfuzz=10pt % supress overull warnings {(*@\label{vfuzz}@*)}
   \splittopskip=0pt %no glue at top - normal TeX 10pt
   \count1=0 % Initialize counter
   %\par\noindent % new paragraph for output
   \def\rebox{%
      \advance\count1 by 1\relax
      \hbox to 20pt{\strut\hfil\number\count1\hfil}%
      \nobreak
      \setbox2=\vsplit 1 to 6pt
      \vbox{\unvbox2\unskip}%
      \hskip 0pt plus 0pt\relax}%end rebox
     \loop
       \rebox % row
       \ifdim \ht1>0pt % test for more rows
    \repeat % if lines exist repeat
 %  \par%setbox
}

\end{teX}

Here is the output

\lineskip=0pt
\parskip=0pt

\long\def\NumberParagraph#1{%
\setbox1=\vbox{\advance\hsize by -20pt #1}%place contents in a box
\vfuzz=0pt % supress overull warnings
\splittopskip=0pt%add this at every split at top
\count1=0 % Initialisierung der Zeilenzahlung
%\endgraf\noindent% new paragraph for output
\def\rebox{%
   \advance\count1 by 1\relax%
   {\hbox to 20pt{\strut\number\count1}% 
   \setbox2=\vsplit 1 to 1pt% split box 1 to 9pt height
    \vbox to 10pt{\unvbox2\unskip\hskip 20pt plus 0pt\relax}}
}%
\loop%
  \rebox % row
  \ifdim \ht1>0pt % test for more rows
\repeat % if lines exist repeat
\par
}



{
\NumberParagraph{Testing.\par This is a short paragraph, that
 only has a few lines of codes. 
It is an experiment to see, if everything will work as planned. 
I tried to make it a few lines long. \lorem }}



{\footnotesize \the\baselineskip}



thiis is a tes \par


\NumberParagraph{\lipsum[2]}

\bigskip

The way the line numbering macro works is by utilizing two boxes |box1| and |box2|. We first place the contents of the paragraph in |box1| at line [\ref{box1}]. 



\tex uses this parameter with \cmd{vbadness} in classifying a \cmd{vbox} or \cmd{vtop} which contains more material than will fit even after the glue in the box has shrunk all it can. TeX considers the box overfull if the excess width of the box is larger than \cmd{vfuzz} (see line [\ref{vfuzz}] in code above) or \cmd{vbadness} is less than 100 [302].
See TeXbook References: 274, 302. Also: 274, 348.

\section{Horizontal and vertical lines}
\normalfont\normalsize

Horizontal and vertical lines are drawn using \tex's \cmd{hrule} and \cmd{vrule}.
If we write |\hrule| in the  middle of a text, then the paragraph ends and
a horizontal line is drawn over the whole line width. The line width is preset to 0.4pt.

|\hrule| and |\vrule| have three optional other parameters that affect the appearance
of the stroke. A \textit{rule} within the meaning of \tex  is nothing more than a
box. For example, this box \vrule height4pt width3pt depth1pt ~is the result of:

\begin{teX}
\vrule height4pt width3pt  depth1pt 
\end{teX}




\cmd{vrule} and \cmd{hrule} have the same additional data, but these are preset
differently.

{

\centering\scalebox{3}{\vrule\,Sample} \scalebox{3}{\vrule\,Subtle}

}

\begin{teX}
  \centering\scalebox{3}{\vrule ~Sample} \scalebox{3}{\vrule  ~Subtle}
\end{teX}

\noindent In the above example you can observe that there was no need to define the widh or height of the \cmd{vrule}. \tex determined these by their enclosing environment.

For example, if

|\vrule height4pt width3pt depth2pt|

\def\smallbox{\vrule height4pt width3pt depth2pt}

\noindent appears in the middle of a paragraph, \tex will typeset the black box \smallbox. If you specify a dimension twice, the second specification overrules the first. If you leave a dimension unspecified, you get the following by default:

\begin{tabular}{lll}
\toprule
~     &|\hrule| &|\vrule|\\
\midrule
width &*        &0.4 pt\\
height&0.4pt    &*\\
depth &0.0pt    &*\\
\bottomrule
\end{tabular}
\medskip


Here `*' means that the actual dimension depends on the context; the rule will extend to the boundary of the smallest box or alignment that encloses it. Chapter 21 of the TeXbook deals with rules in more detail.

\tex does not put interline glue between rule boxes and their neighbours in a vertical list, so these two lines are exactly 3pt apart. \index{glue!interline}
\begin{teX}
\hrule width50pt Test \hrule width50pt
\vskip3pt
\hrule width50pt Test \hrule width50pt
\end{teX}

\hrule width50pt Test \hrule width50pt
\vskip3pt
\hrule width50pt Test \hrule width50pt
\medskip

If you specify all three dimensions of a rule, there's no essential difference
between |\hrule| and |\vrule|, since both will produce exactly the same black
box. But you must call it an |\hrule| if you want to put it in a vertical list, and you
must call it a |\vrule| if you want to put it in a horizontal list, regardless of whether it
actually looks like a horizontal rule or a vertical rule or neither. If you say |\vrule| in vertical mode, TEX starts a new paragraph; if you say |\hrule| in horizontal mode, \tex stops the current paragraph and returns to vertical mode.

\begin{teX}
\centerline{\vrule height 4pt width 6cm}
\medskip
\centerline{\bf Nice Header!}
\medskip
\centerline{\vrule height 4pt width 6cm}
\end{teX}

This will produce:

\centerline{\vrule height 4pt width 6cm}
\medskip
\centerline{\bf Nice Header!}
\medskip
\centerline{\vrule height 4pt width 6cm}
\bigskip


\section*{Drawing rule weights}
\def\weights#1{\footnotesize{#1}\hskip 0.5em \vrule height 0.4cm width #1pt  \par
\smallskip}

pt
\smallskip

\weights{1.0}  
\weights{2.0}
\weights{3.0}
\weights{3.5}
\weights{4.0}
\weights{4.5}
\weights{5.0} 
\weights{5.5} 
\weights{6.0}
\weights{6.5}
\weights{7.0}
\weights{7.5}


In the following the ultimate demonstration of using boxes is shown:


\bgroup
^^A\input{./sections/texrulers}
\egroup

\normalfont\normalsize


\section*{Number of parameter tokens}

This is based on an article in TUGBoat by Jeremy Gibbons. As Jeremy notes, it is easy to work with parameter texts if they are stored in \textit{saturated} macros: macros with nine undelimited parameters. The three following saturated macros containing parameter text will be used as a running example.

\begin{teX}
\def\pp#1#2#3#4#5#6#7#8#9{%
  #1trivial#2parameter#3}

\def\qq#1#2#3#4#5#6#7#8#9{%
  #1\undefined#2parameter#3}

\def\kk#1#2#3#4#5#6#7#8#9{%
  #problem#2\gobbledisttag#3}
\end{teX}

The goal is to define a macro |\nopt| returning in a counter the number of parameter tokens in a parameter text; the counter and the parameter text are, in this ordet, the only arguments of |\nop|. Jeffrey Gibbon's idea was simple: substitute each parameter token for a counting code like

\begin{teX}
\advance\counta by 1
\end{teX}

It is also necessary to define a macro that allows mapping the same thing in each parameter token.

\begin{teX}
\def\applyall#1#2{#1%
  {#2}{#2}{#2}{#2}{#2}{#2}{#2}{#2}{#2}}
\end{teX}


\section{edef}
\index{macro!edef}
You can say |\edef\foo{bar}|. The syntax is the same as |\def|, but the token list in the body is fully expanded (tokens that come from |\the| are not expanded).

You can put the prefix |\global| before |\edef|, note that \cmd{xdef} is the same as |\global\edef|. In the example that follows, the |\ifx| is true.

\begin{teX}
{\catcode`\A=12 \catcode`\B=12\catcode`\R=12
 \gdef\fooval{ABAR}}

{\escapechar=`\A \edef\foo{\string\BAR}\ifx\foo\fooval\else \uerror\fi}
\end{teX}

Another example is the following. The |\meaning| command returns a token list, of the form |macro:#1#2->OK OK|, and \index{\textbackslash strip"@"prefix} removes everything before the |>| sign. What we put in |\Bar| is a list of five tokens, a space, and four letters of catcode 12.

\begin{teX}
\makeatletter
  \def\strip@prefix#1>{}
  \def\foo#1#2{OK OK}
  \edef\Bar{\expandafter\strip@prefix\meaning\foo}
\makeatother
\end{teX}


\section{Using kernel macros}

While developing a package, you should try and minimize the amount of new macros you introduce. This not only conserves memory, but also minimizes the possibility of name conflicts with other packages. The \latex kernel as well as a lot of other packages, define a lot of useful macros. Let us consider a macro for checking what environment surrounds the code. We define this macro as |\IfEnvironment|.

\emphasis{def,IfEnvironment,@firstoftwo,@secondoftwo}
\begin{texexample}{Testing if in a environment}{}
\bgroup
\makeatletter
\def\IfEnvironment#1{%
  \let\reserved@b\@currenvir
  \def\reserved@a{#1}
  \ifx\reserved@a\reserved@b 
    \expandafter\@firstoftwo
  \else 
    \expandafter\@secondoftwo\fi
}

\IfEnvironment{document}{True}{false}

\begin{trivlist}
\item test
\IfEnvironment{trivlist}{True}{false}
\end{trivlist}
\makeatother
\egroup
\end{texexample}


Here, we have used two macros from the kernel, |\@firstoftwo| and |\@secondoftwo|. Since they are available, we have used them and saved the trouble of having to redefine them. We have also used |\reserved@a| and |\reserved@b|, also from the kernel. Many programmers use them, but as the names imply they are reserved. It is best to rather define your own scratch macro names.
\MakePercentIgnore

















%
  
\chapter{GROUPING AND SCOPING RULES}
\index{Grouping}
\label{ch:grouping}

Like most computer languages \tex\ has a scoping mechanism that is able to confine most changes to a particular locality. This chapter explains what sort of actions can be local, and how groups are formed.
\medskip

\begin{docCommand}{bgroup}{}
Implicit beginning of group character.
\end{docCommand}

\begin{docCommand}{egroup}{}
 Implicit end of group character.
 \end{docCommand}

\begin{docCommand}{begingroup}{}
 Open a group that must be closed with |\endgroup|.
\end{docCommand}

\begin{docCommand}{endgroup}{} 
Close a group that was opened with |\begingroup|.
\end{docCommand}

\begin{docCommand}{aftergroup}{} 
Save the next token for insertion after the current group ends.
\end{docCommand}

\begin{docCommand}{global}{}
 Make assignments, macro definitions, and arithmetic global.
\end{docCommand} 

\begin{docCommand}{globaldefs}{}
 Parameter for overriding |\global| prefixes. IniTEX default: 0.
\end{docCommand}



The grouping mechanism can be thought of a bit like scope in other programming languages, with the
exception that in \tex the mechanism is much more Pascal-like. Most assignments made inside a group are local to that group
unless explicitly indicated otherwise, and outside the group old values are restored (pretty much like in Pascal). 

The most common way to group a portion of your program is to use braces. If we type the following  example:

\begin{texexample}{}{}
\def\i{42} 

{
  \def\i{43}
  \def\b{2}
}

The value of the \textbackslash i is now \i

\def\x{a}
\let\y\x
\bgroup
  \def\x{b}
  Within group \x\par
\egroup
  Outside group \x
\end{texexample}
We get   \texttt{The value of the \textbackslash i is now 42}. Due to the way \tex scoping rules work, the old program state
will be restored \textit{completely} after returning from the local group. Neither the change to |\i| nor the definition of |\b| will survive. This is also true for register changes or other assignments.



\section{Local and global assignments}

An assignment or macro definition is usually made global by prefixing it with \cs{global}, but nonzero
values of the integer parameter |globaldefs| override |doccmd{global}|
is positive every assignment is implicitly prefixed with \docAuxCommand{global}, and if |\globaldefs| is negative,
|\global| is ignored. Ordinarily this parameter is zero. It has very
limited use and even in the \latex\ kernel we can only find 3-4 uses when defining math fonts.\footnote{In file \texttt{ltfssbas.dtx}.}


Some assignment are always global: the \marg{global} assignments are:

\begin{description}
\item[font assignment] assignments involving \cs{fontdimen}, \cs{hyphenchar}, and \cs{skewchar}.

\item[hyphenation] assignment \cs{hyphenation} and \cs{patterns} commands.

\item[hbox size assignment] altering box dimensions with \cs{ht}, \cs{dp}, and \cs{wd} 

\item[interaction mode assignment] run modes for a \tex job.

\item[intimate assignment] assignments to a special integer or special dimen
\end{description}

\section{Braces}

The most common way to group is to use braces. They are used for two purposes:

\begin{enumerate}
\item to indicate the start and end of a group. For example |{\small here is some text}|.

\item to indicate that a string of tokens should be treated as one unit. For example in |\def\abc{...}| the braces are used
to delimit the argument.
\end{enumerate}

It is important to note that the characters `\{', `\}' are not hardwired in \tex. Any tokens with catcodes 1 and 2 can be used.
The plain format starts [343] by defining:

\begin{teX}
\catcode`\{ =1
\catcode `} = 2
\end{teX}

Tokens with catcodes 1 and 2 are called \emph{explicit braces}. An \emph{implicit} brace is a control sequence whose replacement text is an explicit brace. Thus the two |plain| control sequences 
|\bgroup| and |\egroup| are implicit braces. 

There is also a low-level \tex operator pair for creating groups. It works
just as the braces. A group is started with \cs{begingroup} and ended with
\cs{endgroup}. These operators may be freely mixed with braces but pairs
should be properly matched. So |{ \begingroup \endgroup }| is allowed
but |{ \begingroup } \endgroup| is not.

\begin{teX}
\let\bgroup={
let\egroup=}
\end{teX}

They can be used where unbalanced braces are needed.

Salomon gives an example to typeset a number of paragraphs with a negative indentation\footnote{This style can sometimes be found in old books.}:

\begin{teX}
\def\negIndent{\brgoup\parindent=-20pt}
\def\endIndent{\par\egroup}

\negIndent
  \small\lipsum[1]
\endIndent
\end{teX}

This will typeset:

\def\beginindent{\bgroup\parindent=-20pt}
\def\endindent{\par\egroup}

\beginindent
  \small\lipsum[1-3]
\endindent

\section{Forming Groups Using \textbackslash begingroup and \textbackslash endgroup} 

The other two primitives \docAuxCommand{begingroup} and \docAuxCommand{endgroup} can also be used to define a group. However a group that starts with a |\begingroup| must end with an |\endgroup|. This provides a mechanism for error checking, which \tex's parsing routines can easily catch.

Note that |\begingroup| and |\endgroup| can only be used to define a group, not to delimit a string. You can say:

\begin{teX}
\begingroup
  \it abc
\endgroup
\end{teX}

but the following will get \tex to complain about missing braces

\begin{teX}
\hbox\begingroup\it abc\endgroup
\end{teX}

It should be pointed out that |\begingroup| and |\endgroup| do not really
add any new grouping functionality that could not be provided by curly braces
or |\bgroup| and |\egroup|. On the other hand, these two instructions are very
useful in nested groups of complicated structures, where one wants to make sure
that a certain "begin group instruction" is matched by a certain "end group
instruction." For this pair of grouping instructions, and this pair only, use |\begingroup|
and |\endgroup|. In case a |\begingroup| is not matched by a |\endgroup|,
an error is generated by \tex.\footcite{bechto1993} 

The case when not to use |begingroup| is clear. However, if one should use it for cases where
|\bgroup| is possible, is a subject with different opinions.\footnote{See \url{https://tex.stackexchange.com/questions/1930/when-should-one-use-begingroup-instead-of-bgroup/1932\#1932}.} Unless you are using |mathmode| or have deeply nested structures, |bgroup| is fine to use. In all
other cases it is preferable to use |\begingroup|.

\section*{Examples}
From the TexBook Exercise 7.4

Suppose that the commands
\begin{texexample}{}{}
{\catcode`\<=1 \catcode`\>=2
 \bfseries test
>
 test
\end{texexample}

appear near the beginning of a group that begins with |{| these specifications instruct
TEX to treat |<| and |>| as group delimiters. According to \tex's rules of locality, the
characters |<| and |>| will revert to their previous categories when the group ends. But
should the group end with |}| or with |>| ?

It ends with either |>| or |}| or any character of category 2; then the effects of all
\cs{catcode} definitions within the group are wiped out, except those that were global.
\tex  doesn't have any built-in knowledge about how to pair up particular kinds of
grouping characters. New category codes take effect as soon as a |\catcode| assignment
has been digested. For example,

\begin{teX}
{\catcode`\>=2 >
\end{teX}

is a complete group. But without the space after |2|  it would not be complete, since TEX
would have read the |>|  and converted it to a token before knowing what category code
was being specified; \tex always reads the token following a constant before evaluating
that constant.

\topline

\textbf{Example}: \textsc{Adjusting the spacing of a font} An interesting example that illustrates some of the concepts that were discussed so far is to try and change the \textit{inter word spacing} of text using the \cs{fontdimen2} parameter. The interesting aspect of this example is that
we want to change the spacing, but since the font changes are global, we want to revert back to the original font at the end of the group. Although there are many other ways of achieving this we will use the \cs{aftergroup}.

\begin{teX}
\font \roman=cmr10
\font\specroman=cmr10
%% Next, the special registers
\newdimen\savedvalue
\savedvalue=\fontdimen2\roman
\newdimen\specialvalue
\specialvalue=13.0pt
%% Finally, definitions.
\def \rm{%
  \fontdimen2\roman=\savedvalue }
\def\specrm{%
  \aftergroup\restoredimen
  \fontdimen2\specroman=\specialvalue
  \specroman  }
\def\restoredimen{%
\fontdimen2\roman=\savedvalue }
\end{teX}
{
%% First, fonts.
\font \roman=cmr10
\font\specroman=cmr10
%% Next, the special registers
\newdimen\savedvalue
\savedvalue=\fontdimen2\roman
\newdimen\specialvalue
\specialvalue=13.0pt
%% Finally, definitions.
\def \rm{%
  \fontdimen2\roman=\savedvalue }
\def\specrm{%
  \aftergroup\restoredimen
  \fontdimen2\specroman=\specialvalue
  \specroman  }
\def\restoredimen{%
\fontdimen2\roman=\savedvalue }


{\bf Spaced Out Text} 
\medskip
{\specrm \lorem} dimension2 the interword   value \the\fontdimen2\font


{\bf  Back to Normal}
\medskip

\rm
\lorem

}

\section{\textbackslash aftergroup}

The \cs{aftergroup} control sequence saves a token for insertion after the current group. Several
tokens can be set aside by this command, and they are inserted in the left-to-right order in which
they were stated.

\begin{texexample}{}{}
\def\x#1;{#1}
\def\y{15}
{\globaldefs1
\bgroup
   \def\y{0}
   \aftergroup\x\aftergroup\y\aftergroup;
   \aftergroup}
\egroup
\y


\globaldefs0

\def\z{1}
{\def\z{0}
\z
}

\z

\end{texexample}

\begin{texexample}{}{}
{ \def\z{1}
  {\def\z{0}\globaldefs1
     \z
    {
	\z
    }
   \z
  }
 \z
}
\end{texexample}
\section{afterassignment}

An interesting primitive is \docAuxCommand{afterassignment}. The primitive saves the token immediately following it without
expansion. Nothing happens until after the next assignment; immediately after the next assignment the saved token is expanded.

\begin{texexample}{Aftergroup}{ex:aftergroup}
\def\yy{%
  \afterassignment\yyb
  \let\yyDiscard = 
}

\def\yyb{%
 ``%
 \bgroup
 \itshape
 \aftergroup\yyc
}
\def\yyc{%
  ''%
}

\yy{This is a test}  
\end{texexample}

The above example is not a very common or idiomatic way of writing macros. So what is |\afterassignment| good for? Its main use is to write macros with \enquote{arguments} similar to the way \tex assigns registers. Afterassignment allow you to define macros which avoid curly braces to enclose arguments.

The most common use of |\afterassignment| is in a macro whose parameter is glue or dimen. Consider the definition of a macro such as:
\begin{quote}
 |\def\myglue#1{\leftskip=#1 \rightskip=#1}|
\end{quote}

Such a macro can be called as |\myglue{3pt plus5pt minus3pt}|, but if we want to keep the same conventions as \tex we might prefer to have the ability to call it as |\myglue 3pt plus5pt minus3pt|. To achieve this we can do:

\begin{texexample}{Afterassignment}{ex:afterassignment}
\bgroup
\font\larger=cmr10 scaled\magstep1
\larger
\newskip\tempskip
\def\myglue{\afterassignment\myglueaux \tempskip}
\def\myglueaux{\leftskip=\tempskip \rightskip=\tempskip}
\myglue=30pt plus1pt minus1pt
\lorem\par
\egroup
\lorem
\end{texexample}



\section{Scoping Rules for boxes}

The scoping rules for boxes work similarly to those for other command sequences, since they are just macros defined by \latex or |plain|. In the example below, we define a box |\mybox| and we save a sentence both in global scope as well as local scope.

\begin{teX}
\documentclass{article}
\begin{document}
  \newsavebox{\mybox}
  \savebox{\mybox}{Outside scope}
  \usebox\mybox
  \begin{minipage}{5cm}
    \sbox{\mybox}{from first minipage}(*@ \label{global} @*)
    \usebox\mybox
  \end{minipage}
  \usebox{\mybox}
\end{document}
\end{teX}


This will typeset:
\medskip

\newsavebox{\myboxi}
\savebox{\myboxi}{\tt > Outside scope}

\noindent\usebox\myboxi

\noindent\begin{minipage}{5cm}
\sbox{\myboxi}{\tt > from first minipage}
\noindent\usebox\myboxi
\end{minipage}

\noindent\usebox{\myboxi}


\medskip 
Changing line [\ref{global}] to |\global\sbox| will make the definition of |\mybox| within the minipage environment global and would change the output to:
\medskip


To save memory space, box registers become empty by using them: \tex assumes
that after you have inserted a box by calling |\boxnn| in some mode, you do not need the contents of that register any more and empties it. In case you do need the contents of a box register more
than once, you can |\copy| it. Calling |\copynn| is equivalent to |\boxnn| in all respects except that the register is not cleared.


There are 256 box registers, numbered 0–255. Either a box register is empty (‘void’), or it contains
a horizontal or vertical box. This section discusses specifically box registers; the sizes of boxes,
and the way material is arranged inside them, is treated below.




\newbox\MyBox

\setbox\MyBox=\hbox{\hfil Test\hfill}

\unhbox\MyBox


\noindent\unhbox\MyBox

\noindent{\hfill Test \hfill}



\framebox{\parbox{\linewidth}{\color{theblue}
\textbf{\textcolor{purple}{\textsf{CAUTION}}}
\begin{enumerate}
\itemsep-5pt
\item \latex will not empty a box as it uses the \cs{copy} command in the definition of the \cs{newsavebox}.
\item It is better to use \LaTeX\ commands rather than \tex primitives, when defining boxes, as \latex tests for duplication of names - which is very important if a user uses a lot of different packages.
\item Give always preferences to local definitions rather than global. Globals always create maintenance problems in programming.
\end{enumerate}
}}


\section{Implicit Grouping}

There are  instances where grouping is \textit{implicit}. What this means is that \text starts and ends a group automatically and without any action by the user. There are two major cases where this happens:

\begin{enumerate}
\item The text inside a box such as |\hbox|, |\vbox|, |\vtop|, |\vcenter| etc. is automatically treated by \tex as a group.  For example |\hbox{\bf My Heading}|, will print  \hbox{\bf My Heading}  and it will not continue with the bold font once outside the group. All these commands have curly brackets and these curly brackets form implicit groups.
\item In five cases \tex forms implicit groups. In some of these cases not even curly braces are involved.
\end{enumerate}

\begin{enumerate}
\item The text inside math mode is treated as a group. This is true both for inline math as well as display math.
\item Matching |\left| and |\right| primitives treat the formula in between them as a group.
\item Fractions are treated as a group.
\item The execution of an ouput routine is implicitly enclosed in a group.
\item Columns in |\halign| based tables are local.
\end{enumerate} 

\subsection{\texttt{afterssignment and grouping}}

\begin{macro}{\afterassignment}
The primitive |\afterasignment| does not follow grouping in that it does not save the definition of a token when |\afterassignment| is executed. Consider the following example:
\end{macro}

Define the two macros |\xx| and |\yy|.

\begin{texexample}{afterassignment}{}
\def\xx{\string\xx\ executed\par }

\def\yy{\string\yy\ executed\par }

\afterassignment\xx
\end{texexample}

We start a group, where we have two definitions of |\xx| and |\yy|

\begin{texexample}{afterassignment}{}
\def\yy{42}
{
  \def\xx{\string\xx executed inside a group\par}

  \def\yy{\string\yy executed inside a group\par}

The second afterassignment is execute

  \afterassignment\yy

The group is ended

}
\end{texexample}

Note \cs{afterassignment} saves the token following \cs{afterassignment} without expanding it. Nothing happens until after the next assignment; immediately after the next assignment the saved token is expanded. This is a bit of a tricky part and you can go over it to make sure you understand it well.
\footnote{\url{http://tug.org/TUGboat/tb32-2/tb101grunewald.pdf}}
\footnote{\url{http://tex.stackexchange.com/questions/65462/plain-tex-theory-afterassignment}}


\begin{texexample}{Combining bgroup and begingroup}{}
\begingroup
\newbox\savedparbox

\def\saveparbox{\par\begingroup
  \def\par{\egroup\endgroup}
  \global\setbox\savedparbox\vbox\bgroup}

Ordinary paragraph.
\saveparbox
This paragraph will be saved in \string\box\string\savedparbox.
If you wish, you can unpack the box and do all kinds of processing on it.
In this demo, I won't do any processing.
Look in the log file to examine the box contents.

Another ordinary paragraph.
\endgroup
\end{texexample}

































%
  \chapter{Iteration}
\precis{A discussion as to how to program simple for loops, in
TeX and LaTeX.}

\section{\TeX's simple \protect\texttt{loop}}

\newthought{Knuth in the TeXBook} provided a simple loop macro that can be used for iteration. It must be pointed out that there are no real looping structures in \tex other than pure recursion (including tail recursion). All looping mechanisms are build on top of these.

\begin{docCommand}{loop}{\meta{body}\docAuxCommand*{repeat}}
The |\loop...\repeat| construction is defined in Plain TeX and works like this:
You say `|\loop| $\alpha$ |\if|\dots $\beta$  |\repeat|', where $\alpha$ and $\beta$ are any sequences of
commands, and where |\if...| is any conditional test (without a matching |\fi|). 
Note that the |repeat| is just a marker in the argument specification of the macro. It can in essence be anything.
\end{docCommand}

\tex
will first do $\alpha$; then if the condition is true, \tex will do $\beta$ and repeat the whole process
again starting with $\alpha$. If the condition ever turns out to be false, the loop will stop.


The \cmd{\loop} macro that does all these wonderful things is actually quite simple.
It puts the code that's supposed to be repeated into a control sequence called
\cmd{\body}, and then another control sequence iterates until the condition is false:

\begin{teXXX}
\def\loop#1\repeat{\def\body{#1}\iterate}
\def\iterate{\body\let\next=\iterate\else\let\next=\relax\fi\next}
\end{teXXX}


\begin{texexample}{loop...repeat}{ex:loop}
\newcount\n
\n=0
\loop
  \advance\n by1
    \texttt{\number\n, } 
  \ifnum\n<30
\repeat
\end{texexample}

Just observe that the |\loop| arguments are delimited by |\repeat|. We could as well named it |\endloop| (a repeat at the end of a loop somehow sounds wrong!)


\begin{texexample}{Rename loop}{ex:renloop}
\bgroup
\def\for#1\endfor{\def\body{#1}\iterate}
\def\iterate{\body\let\next=\iterate\else\let\next=\relax\fi\next}
\newcount\n
\n=0
% Example usage
\for
   \advance\n by1
     \texttt{\number\n, }  
   \ifnum\n<30
\endfor
\egroup
\end{texexample}  


\subsection{Breaking out of a loop}

\index{Iteration!break}\index{\protect\textbackslash break}
Although the loop macros are fairly simple, breaking out of them or using conditionals needs some work.

\begin{texexample}{Iteration}{ex:loop}
\newcount\mycount
\mycount=0
\loop\ifnum\mycount<13
\the\mycount, 
\ifnum\mycount>5
    \let\iterate\relax
 \fi
 \advance\mycount by1\relax
\repeat
\end{texexample}

We can define a command \docAuxCommand{break}, so as to have better semantics and make the code more readable:

\emphasize{break,loop,repeat,}

\begin{texexample}{Iteration}{ex:loop1}
\def\break{\let\iterate\relax}% (*@ \dcircle{1} @*)
\newcount\mycount
\mycount=1
\loop
  \ifnum\mycount<13 % (*@ \dcircle{2} @*)
    %\the\mycount, 
    \ifnum\mycount>5
    % we break here
    \break %     
   \fi
   \the\mycount,\space% (*@ \dcircle{3} @*)
   \advance\mycount by1\relax
\repeat
\end{texexample}


Once we define what a |break| is supposed to do at \dcircle{1}, we use it at \dcircle{2} to let |\iterate| to |relax|. Then at \dcircle{3}, we use the value of the counter |\the\mycount| to add the number and a comma followed by a space. 


\section{Iteration over comma delimited lists}
\index{iteration>comma delimited lists}

The comma delimited list is one of the most common programming datastructure. A list is simply defined using a macro:

\begin{verbatim}
\def\mylist{John,Mary,Mathew,George,Maria}
\end{verbatim}

Although, lists can be defined as shown in the |\mylist| macro, this is not very useful. In most cases the \textit{elements} of the list would be added programmatically. Such list are used for example by \latex to keep track of input files.

Unlike many other programming languages, lists can be delimited by any character or even macros. Many package authors use a semicolon. This a perfectly legal in \tex.

\begin{teX}
\mylist{John;Mary;Mathew;George;Maria}
\end{teX}
as well as this:

\begin{teX}
\mylist{\@elt John\@elt Mary\@elt Mathew \@elt George \@elt Maria}
\end{teX}

\begin{docCommand}{@elt}{}
Using a macro to delimit the list elements, has the advantage that when we invoke the |mylist| list macro the |@elt| macro can map a function over the elements. We will see that a bit later in more detail but for the time being we will demonstrate this with an example:
\end{docCommand}

\begin{texexample}{Elt Lists}{}
\def\mylist{\@elt John,\@elt Mary,\@elt Mathew, \@elt George, \@elt Maria,}
\def\@elt#1,{\textit{#1} }
\mylist
\end{texexample}

Note that the macro |\@elt| when it is defined is delimited with a comma. 

\newthought{Adding Elements}

\begin{docCommand}{g@addto@macro}{}
There are many ways to add an element to the list, but perhaps the easiest is to use the \latex \cmd{\g@addto@macro}. 
\end{docCommand}

\emphasis{g@addto@macro}
\begin{teXXX}
\g@addto@macro{\mylist}{\@elt Thomas,}
\mylist
\end{teXXX}

One disadvantage of this approach is that the last item on the list will have a comma. A better approach would be to check if
the list is empty and to insert an elemen

\begin{texexample}{Lists}{}
\makeatletter
\def\mylist{Yiannis}
\def\emptylist{}

\def\addtomylist#1{%
\if\mylist\emptylist
   \g@addto@macro{\mylist}{#1,}
\else
   \g@addto@macro{\mylist}{,#1}
\fi
}
\addtomylist{George}
\addtomylist{Maria}
\addtomylist{Athena}

\mylist
\makeatother
\end{texexample}
 



\subsection{How to Use LaTeX’s kernel looping constructs}

\begin{docCommand}{@for}{}
The \cmd{\@for} is an internal \latexe command that can be used to iterate over a comma delimited list.
\end{docCommand}

\emphasis{mylist}
\begin{teX}
\makeatletter
\def\mylist{1,2,3,4,5}(*@\label{list}@*)
\@for\val:=\mylist\do{\val
\ifx\@xfor@nextelement\@nnil \else ;\fi}
\makeatother
\end{teX}


\latex's  low-level programming is rather poorly documented and the section on what is called control commands is even more so. The current \latex team are trying to provide some proper looping structures in \latex3. 

If you want to loop over comma-lists, \latex provides the \cmd{\@for} macro. This works by repeatedly assigning list items to a temporary variable:

To use it we need to define a list:

\startlineat{50}
\begin{teX}
\def\mathList{\alpha,\beta,\gamma,
          \delta,\epsilon,\zeta,\theta }
\end{teX}



Using the \refCom{@for} loop we can iterate over the list as follows:

\begin{texexample}{For constructs}{ex:forloop}
\makeatletter
\def\mathList{\alpha,\beta,\gamma,\delta,\epsilon,\zeta,\theta}
\@for\i:=\mathList\do{%
  \ensuremath\i\space 
 }
\makeatother 
\end{texexample}



Running the example we simply get the list but now without the comma

\begin{teX}
\makeatletter
\def\mathList{\alpha, \beta, \gamma, \delta, \epsilon, \zeta, \theta }
\@for\i:=\mathList\do{%
  \ensuremath \i  \space 
 }
\makeatother
\end{teX}




\begin{teX}
\makeatletter
\def\atestiii{}
\def\alist{a,b,c,d,v,e,f,g,h}
Test 1 \@removeelement{v}{a,b,c,d,v,e,f,g,h}{\atestiii} 
returns \atestiii
\alist
\gdef\blist{1,2,3,4,5,v,6,7}%
Test 2 \@removeelement{v}{\expand\blist}{\atestii} prints \atestii
\meaning\atestii
\meaning\@removeelement

\def\remove#1#2{
 \@removeelement #2{#1}\atestiii \atestiii
}

removes an element \atestiii ~~and \alist

\remove c\alist 


the variable holding the list \atestiii
\end{teX}


The iteration does not have the proper meaning that you would normally expect in other programming languages, it is defined as follows:


\begin{teXXX}
\@for(*@\textsubscript{all~elements of the list to }@*)\i:=\mathList\do{%
  \ensuremath \i  \space 
 }
\end{teXXX}

The other interesting thing to note as well as watch out is `:=', which is just a delimiter. I have used |\i| for simplicity, but \cmd{\i}, is a reserved word, meaning a \textit{dotless} \i, which is found in some language like Turkish. If you going to use it in your writings you will need to save it and restore it afterwards. A lot of macro writers also use |\ii| or |\@i| or other similar variables. It is simply a temporary variable that at the end of the iteration gets the value |\@nil| and since |@nil| is undefined it essentially destroys it. 

We need to remind ourselves again about |##|
20.5. The |##| feature is indispensable when the replacement text of a definition
contains other definitions. For example, consider


\begin{teX}
\def\a#1{\def\b##1{##1#1}}
after which `\a!' will expand to `\def\b#1{#1!}'. We will see later that ## is also
important for alignments; see, for example, the definition of \matrix in Appendix B.
\end{teX}

\begin{teX}
\long\def\@for#1:=#2\do#3{%
\expandafter\def\expandafter\@fortmp\expandafter{#2}%
\ifx\@fortmp\@empty \else
\expandafter\@forloop#2 ;\@nil;\@nil\@@#1{#3}\fi}

\long\def\@iforloop#1;#2\@@#3#4{\def#3{#1}\ifx #3\@nnil
\expandafter\@fornoop \else
#4\relax\expandafter\@iforloop\fi#2\@@#3{#4}}


\@for\i:=\mathList\do{%
  \ensuremath \i --  
 }


\meaning\loop 
\def\a#1{\textcolor{blue}{\uppercase{#1}}}
\def\b{test}
\expandafter\a\b

\a\b
\end{teX}


\section{Recursion}
\epigraph{“I love you,” said Bekka.

“I know,” I said.

“I know you know,” she said. “But I didn’t know that you knew I knew you knew. And now I do.” }
{Scott Alexander, It Was You Who Made My Blue Eyes Blue}
Recursion is a difficult subject to grasp, although we experience it daily in our actions, language and thoughts. 
The main characteristic of recursion, is that it can take its own output as the next input, a loop that can be extended indefinitely to create sequences of structures of unbounded length or complexity. In language we understand that a sentence can in principle be extended indefinitely, even though in practice it cannot be---although the novelist Henry James had a damn good try in the \emph{The Figure in the Carpet}. Of course what we are interested here is to study how we can write recursive macros in \tex rather than the more interesting aspects of recursion as it applies to thoughts and language. 

When we write:

\begin{verbatim}
\def\mymacro{\mymacro}
\end{verbatim}

The macro will expand itself indefinitely. As it does not save anything in memory it does not exceed the capacity of any data structure, it will just cause your computer to hang.

If we slightly modify the above macro to |\def\mymacro{a \mymacro}| during expansion the macro will typeset and then call itself again, typeset a and call itself again forever. After a while, it overflows the computer’s memory. The reason for this is that we never finish a paragraph. The letters accumulate in main memory as part of the same paragraph.


\begin{texexample}{Parsing lists}{ex:parselist}
\bgroup
\def\parselist#1;{\pickup#1,;,}
\def\pickup#1,{% Note that #1 may be \null
\if;#1
  \let\next=\relax
\else\let\next=\pickup
   #1% use #1 in any way
\fi\next}
\parselist $a_1$, $a_2$, $a_3$, $a_4$, $a_5$; 
\egroup
\end{texexample}

The macro \cmd{\pickup} expects its argument to be delimited by `,’, so it ends up
getting the first component of the original argument. It uses it in any desired way and
then expands itself recursively. The process ends when the current argument becomes the `;’. The compound argument may have any number of components (even zero).\ref{test}

\emphasize{makebox,obeylines}
\begin{texexample}{Longer Example}{ex:mheadings}
\makebox[\linewidth]{\hfill
\begin{minipage}{.8\textwidth}
\columnseprule2pt
\def\columnseprulecolor{\color{thegray}}
\columnsep22pt
\begin{multicols}{2}
\color{theblue}
\flushright
\Large
\obeylines
Over the last year we
have continued to
develop and improve the
range of funding schemes
we offer to meet the
needs of the arts and
humanities communities,
for example, by offering
opportunities for early
career researchers.
\columnbreak
\color{thegray}

\small
\flushleft

\obeylines %(*@\textcolor{blue}{\dcircle{1}}  @*)
\arial
We have engaged both
individuals and groups to
build a vision for our strategic
initiatives and our museums
and galleries strategy, have
opened up opportunities
for the arts and humanities
in cross-Council funding
initiatives and undertaken
to represent the needs of our
communities in arenas such
as the Research Councils’
project on the Efficiency and
Effectiveness of Peer Review
Journals
initiatives and undertaken
to represent the needs of our
communities in arenas such
as the Research Councils’
project on the Efficiency and
Effectiveness of Peer Review
Journals
\end{multicols} %(*@\textcolor{blue}{\dcircle{2}}  @*)
\end{minipage} %(*@\textcolor{blue}{\dcircle{3}}  @*)
}
\end{texexample}


Obviously this does not make for a good user interface. It will be preferable to have just one or two commands and the user should be able to type in the left and right, text. All setting will be preferable to be done via keys, which map to macros. 

%%endinput iteration.tex









%
  \chapter{Expandafter}

One of the most often misunderstood \TeX\ commands is \cmd{\expandafter}
expandafter is an instruction that reverses the order of expansion. It is not a typesetting instruction, but an instruction that influences the expansion of macros. But what is \textit{expansion}? The term expansion means the replacement of the macro and its arguments, if there are any, by the \textit{replacement} text of the macro. If we have defined a macro

\begin{teX}
\def\test{ABC};
\end{teX}


\noindent then the replacement text of |\test| is |ABC| and the \textit{expansion} of |\test| is |ABC|.

As a control sequence |expandafter| can be followed by any number of tokens.

\begin{commands}[]{}
\cmd{\expandafter}\string\token$_e$\string\token$_1$\string\token$_2$\string\token$_n$ etc
\end{commands}

\noindent then the following rules describe the execution of |expandafter|:

\begin{enumerate}
\item  $<token_e$, the token immediately following |\expandafter|, is saved without expansion.
\item $<token_1>$, which is the token after the saved $token_e$, is analyzed. The following cases can be distinguished:
\begin{enumerate}
\item If is a macro: The macro will be expanded. In other words, the macro and its arguments, if any, will be replaced by the replacement text. After this \tex will \textbf{not} look at the first token of this new replacement text to expand it again or to execute it.
\end{enumerate}



\begin{teX}
\def\xx [#1]{[#1]}
\def\yy{[ABC]}

\expandafter\xx\yy
\end{teX}

This results in 
\def\xx [#1]{[#1]}
\def\yy{[ABC]}

\texttt{> \expandafter\xx\yy}


\item token1 is primitive: Normally a primitive token can not be expanded so the |\expandafter| has no effect; but there are exceptions, which we will discuss after the example.

\begin{texexample}{Expansion}{}
\expandafter AB
\end{texexample}

Character A is saved. Then \tex\ tries to expand it, but \textit{not} print B, because B cannot be expanded. Finally A is put back in front of the B ; in other words, the two characters are printed in the given order, and we may well have omitted the |\expandafter|. So what's the point here? |\expandafter| reverses the order of expansion, not of execution.

\noindent But there are exceptions to the above:
\begin{enumerate}
\item \textbf{temporarily suspend an opening curly brace} token 1 is is an opening curly brace which leads to the opening curly brace temporarily suspended. This is listed as a separate case because it has some interesting, applications;

\begin{teX}
\newtoks\ta
\newtoks\tb
\ta = {\a\b\c}
\tb=\expandafter{\the\ta}
\tb={\the\ta}
\tb
\end{teX}

\begin{texexample}{Expansion}{}
\begingroup

\def\a{A}
\def\b{B}
\def\c{C}
\newtoks\ta
\newtoks\tb
\ta = {\a\b\c}
\tb=\expandafter{\the\ta}
\tb={\the\ta}

\texttt{> \the\tb}

\texttt{> \the\ta}

\endgroup
\end{texexample}

\item \meta{$token_1$} is another expandafter. The best way to understand this is to write a \tex mnmal example and watch it in action

\begin{teX}
\tracingmacros=2  \tracingcommands=2
\def\a{A}
\def\b{B}
\def\c{C}

\expandafter\expandafter\expandafter\a\expandafter\b\c

\bye
\end{teX}

Checking the log file with |\tracingmacros=2 \tracingcommands=2| we get

\begin{verbatim}
{vertical mode: \def}
{blank space  }
{\def}
{blank space  }
{\def}
{blank space  }
{\par}
{\expandafter}
{\expandafter}
{\expandafter}

\c ->C
{\expandafter}

\b ->B

\a ->A
{the letter A}
{horizontal mode: the letter A}
{\par}

\meaning\futurenonspacelet
\end{verbatim}


\end{enumerate}


\section{Defining Macros on the fly}

This is a very common requirement. 

\begin{texexample}{csname}{}
\def\newtest#1#2{
  \expandafter\def\csname#1\endcsname{#2}%
}
\newtest{letters}{test for letters}
\letters
\end{texexample}





\end{enumerate}










  \chapter{Futurelet}
\precis{A discussion on one of the most esoteric commands of \protect\tex, with examples as to how to write macros with optional arguments.}
\addtocimage{-12pt}{-20pt}{../images/tocblock-futurelet.jpg}
\epigraph{Life can only be understood backwards; but it must be lived forwards.}{
---S Kierkegaard}

The \cmd{\futurelet} primitive deserves its own chapter, as most people have difficulty in understanding the command. The instruction allows the user to \textit{look ahead}. By look ahead we mean that \tex will look at a future token\footnote{remember that a token is either a single character or a macro command} without absorbing it, i.e, without removing that token from the token list. This operation allows the programmer to perform a test to check what token is 'coming'. You can read a couple of articles about it for example \citep{Eijkhout2001}, but generally they are difficult to follow. The information about the command is also very sparse in the TeXBook.  Another TUGboat article is \citep{bechto88}, which gives pretty much the same example as we describe below. 

The token looked at through
|\futurelet| will be removed later, typically as part
of an argument of a later macro call as we will see
shortly. It is not removed by the action of the
|\futurelet| primitive.

Let us be more precise now; the |\futurelet|
instruction has the following format:


\begin{teX}
\futurelet (tokenl) (token2) (token3)
\end{teX}


\begin{enumerate}
\item  \tex will execute a \cmd{\let}\meta{tokenl}=\meta{token3}.
We therefore have generated a copy of (token3)
stored under the name of (tokenl).\label{lettoken}


\item  removes (tokenl) from the main token list.

\item \tex expands (token2). This token is for all
practical purposes a macro with the following
properties:

(a) The macro will use (tokenl), which is a
copy of (token3), to find out what (token3)
is, in other words what token is to be
expected later.
(b) It will cause another macro to be expanded
which will ultimately absorb (token3).

This other macro ordinarily depends on
what $<token_l>$ is.

\end{enumerate}

The description above, is a bit of a mouthful and it is better to describe it with an example. In Example~\ref{futurelet} we will try and find if the next token is the opening square bracket `['. We then according to the definition in \ref{lettoken} this should be stored in \cs{tokenone}. We verify this by peeking at its meaning.

\begin{texexample}{futurelet}{futurelet}
\def\tokentwo#1{}
\futurelet\tokenone\tokentwo[
\meaning\tokenone
\end{texexample}

The second token \cs{tokentwo} we have defined it, so that it justs absorbs its next argument and does nothing for the time being. As you can see its meaning is \texttt{the character [}. Now what happens if there was a space between the \cs{tokentwo} and the `['?

\begin{texexample}{futurelet second}{futurelet2}
\def\tokentwo#1{}
\futurelet\tokenone\tokentwo     [
\meaning\tokenone
\end{texexample}

As you can see so far the spaces have been absorbed, but let us now change the definition of \cs{tokentwo}.

\begin{texexample}{futurelet second}{futurelet2}
\def\tokentwo#1{}
\futurelet\tokenone\tokentwo     
\meaning\tokenone
\end{texexample}



\begin{texexample}{futurelet}{futurelet}
\def\tokentwo#1{%
   \ifx\tokenone[ true [\else false\fi
}
\futurelet\tokenone\tokentwo[
\meaning\tokenone
\end{texexample}

We try again with spaces,

\emphasis{tokentwo,[}
\begin{texexample}{futurelet}{futurelet}
\def\tokentwo#1{%
   \ifx\tokenone[ true [\else false\fi
}
\futurelet\tokenone\tokentwo     [
\meaning\tokenone
\end{texexample}

As you can see from the examples we cannot capture the spaces. This might present a problem, if we enclose everything in other macros as \tex might leave extra spaces in the stream. Better to absorb them. We will see how later, using LaTeX. 


\section{Applications}

There are many applications of |\futurelet|.
will here present only one example, although
we will present it in quite some detail so the user
will know how to apply |\futurelet| in different
circumstances.

\subsection{Using \textbackslash futurelet in Macros with Optional
Arguments}

A typical application of |\futurelet| is the handling
of macros with optional arguments\cite{Becht1988} as they are used,
for instance, in \latex. By "optional argument" we
mean an argument which in most cases is omitted,
and is provided only occasionally in macro calls.\footnote{See also the discussion at \url{http://tex.stackexchange.com/questions/4557/how-to-use-futurelet-to-define-optional-parameters}}

\textbf{Defining the Problem}

Let us give a specific example: we would like to
define a macro \cmd{xx}, which can be called in two
different ways:

\begin{enumerate}
\item With optional argument as in |\xx [opt]{arg}|
where opt is the optional argument enclosed
in square brackets and \meta{arg} is the mandatory argument
argument.

\item Without optional argument as in |\xx{arg}|
where \meta{arg} is again the regular argument.

\end{enumerate}


Before we discuss how this can be done in \tex,
observe that we do not really have to use an
optional argument. We could simply define two
different macros \cmd{xxwithoptions} for the case where an
optional argument is given, and \cmd{xxnooptions} for the
case where no optional argument is given:


\begin{texexample}{two macros}{ex:twomacros}
\def\xxWithOpt [#1]#2{...}
\def\xxNoOpt #1{...}
\def\xxWithOpt (#1)#2{\fbox{#2}}
\xxWithOpt (box){Testing}
\end{texexample}

How we can use |\futurelet| to find out
whether an optional argument was given or not?

We will define a macro |\xx| whose only function is
to check whether there is an opening square bracket
(optional argument is present) or not (no optional
argument). The |\xx| macro will, after this has been
determined, cause the |\xxWithOpt| macro to be invoked
when there is an optional argument, and the
|\xxNoOpt| macro to be called if there is no opening
bracket. In other words the macros |\xxWithOpt|
and |\xxNoOpt| do the "real work while the only
purpose of the |\xx| macro is to decide which of the
two macros should be invoked.


Here is the completely worked out example.


\begin{teX}
\def \xxWithOpt [#1] #2{...}
\def\xxNoOpt #2{...}

\def\xx {%
\futurelet\xxLookedAtToken
    \xxDecide
}

% (3) The \xxDecide macro, based on
% the lookahead of \xx, calls
% either \xxWithOpt or \xxNoOpt .
\def\xxDecide {%
 \ifx\xxLookedAtToken [%
\let\next = \xxWithOpt
\else
 \let\next = \xxNoOpt
 \fi
\next
}
\end{teX}

\section{Other Applications in the LaTeX kernel}

\begin{teX}
\def\elidebefore[#1]#2{[$\ldots$] #2}
\def\elideafter#1{#1$\ldots$}

\def\elide {%
\futurelet\ifoptions
    \choosemacro
}

\elide{Lorem Ipsum}

\elide[b]{Lorem ipsum}
\end{teX}

\begin{comment}
% The \choosemacro, based on
% the lookahead of \elide, calls
% either \elidebefore or \elideafter 
\end{comment}

\begin{teX}
\def\choosemacro{%
 \ifx\ifoptions [%
     \let\choice = \elidebefore 
 \else
    \let\choice = \elideafter
 \fi
\choice
}
\end{teX}



\begin{teX}
\elide{Lorem Ipsum}

\elide[b]{Lorem ipsum}

\end{teX}

\begin{teX}
\def \xxWithOpt [#1] #2{...}
\def\xxNoOpt #2{...}

\def\xx {%
\futurelet\xxLookedAtToken
    \xxDecide
}

% (3) The \xxDecide macro, based on
% the lookahead of \xx, calls
% either \xxWithOpt or \xxNoOpt .
\def\xxDecide {%
 \ifx\xxLookedAtToken [%
\let\next = \xxWithOpt
\else
 \let\next = \xxNoOpt
 \fi
\next
}
\end{teX}



To build a command with any optional parameter, as you find in many of LaTeX's commands, you will need two things:

\begin{itemize}
\item a macro with delimited parameters

\item a way to grab the first non-space token that follows the command
\end{itemize}


The first part is fairly easy using delimited argument macros, for example we can say

\begin{verbatim}
\def\test(#1)#2#3{#1, #2, #3}
\end{verbatim}

We can then call this macro as:

\begin{verbatim}
\test(a){b}{c}
\end{verbatim}


resulting in a,b,c

To define the |()| as an optional parameter, we effectively need to define the macro as a conditional a sort of a "yes-no" switch. If \tex finds the "(" bracket the "yes-code" will be called and if it finds only the normal arguments the "no-code" will be executed.

For this we can use the |\@ifnextchar| macro from the LaTeX kernel.
You can say |@ifnextchar{char}{yes-code}{no-code}| to test for |(|. The result then will depend on the token that follows. If this token is the same as the first argument, then the "yes-code" is executed, otherwise the "no-code" is executed. The first argument should be a single token (for instance a character). Spaces are ignored. 

As for example we can redefine the LaTeX code for `rule` to accept an optional parameter in round brackets, rather than the traditional square brackets.

\begin{texexample}{Using ifnextchar}{}
\makeatletter
\def\Rule{\@ifnextchar(\@Rule%
        {\@Rule(\z@)}}
\def\@Rule(#1)#2#3{%
 \leavevmode
 \hbox{%
 \setlength\@tempdima{#1}%
 \setlength\@tempdimb{#2}%
 \setlength\@tempdimc{#3}%
 \advance\@tempdimc\@tempdima
 \vrule\@width\@tempdimb\@height\@tempdimc\@depth-\@tempdima}}
\makeatother

A test \Rule(6.5pt){100pt}{1pt}

Another test \Rule{100pt}{3pt}

Not that difficult but you will need to , but why on earth do you need this?
\end{texexample}




\begin{comment}
\def\elidebefore[#1]#2{[$\ldots$] #2}
\def\elideafter#1{#1$\ldots$}

\def\elide {%
\futurelet\ifoptions
    \choosemacro
}

% The \choosemacro, based on
% the lookahead of \elide, calls
% either \elidebefore or \elideafter 


\def\choosemacro{%
 \ifx\ifoptions [%
     \let\choice = \elidebefore 
 \else
    \let\choice = \elideafter
 \fi
\choice
}

Testing \elide[b]{Lorem ipsum}

\elide{Lorem Ipsum}

\elide[b]{Lorem ipsum}

\end{comment}


\section{Using LaTeX \protect\textbackslash @ifnextchar}

\latex defines the |\@ifnextchar| kernel command that is used effectively to
determine the token that follows the command. It is used in the definitions
of macros with optional arguments amongst other things.

\begin{teXXX}
\@ifnextchar]{true}{false}] 
\@ifnextchar[{true}{false}[
\end{teXXX}
The result would both be true,

\begin{texexample}{Example ifnextchar}{ifnextchar}
\makeatletter
\@ifnextchar]{true ]}{false} ] %notice ]
\@ifnextchar[{true [}{false} [ %notice [
\makeatother
\end{texexample}






















  \chapter{Colors}

\newthought{The figure below, shows the wavelengths} in nm of the visible light. It has been drawn using the \docpkg{xcolor} package and the native \latex environment \cmd{picture}. The colors can be typest using the wavelength of light.

\smallskip

\begin{texexample}{}{}
  \hbox{\color{thered} A TesT}
\end{texexample}




\newcount\WL \unitlength.75pt

\begin{figure}
\hskip-3pt\scalebox{0.9}{
\noindent

\begin{picture}(460,60)(355,-10)
\sffamily \tiny \linethickness{1.25\unitlength} \WL=360
\multiput(360,0)(1,0){456}%
{{\color[wave]{\the\WL}\line(0,1){50}}\global\advance\WL1}
\linethickness{0.25\unitlength}\WL=360
\multiput(360,0)(20,0){23}%
{\picture(0,0)
\line(0,-1){5} \multiput(5,0)(5,0){3}{\line(0,-1){2.5}}
\put(0,-10){\makebox(0,0){\the\WL}}\global\advance\WL20
\endpicture}
\end{picture}}
\caption{The visible spectrum nm}
\end{figure}

The |xcolor| package provides numerous macros for typesetting colors, using a variety of methods and color schemes. For example we can use the command \cs{color} to print a text sample in color.
\newlength\pull

\def\colorSample#1{%
\leavevmode
\parindent0pt
   \def\colorRule{\color[wave]{#1}\rule{\textwidth}{0.4pt}} 
   \colorRule
%% set to the width of the box
   \settowidth\pull{\framebox{\Large #1 nm}}
%% pull by one em
   \addtolength\pull{1em}
   \hskip -\pull{\color[wave]{#1}{{\framebox{\Large #1 nm}}}}%
   %% add story
   \hskip1em\noindent\onepar\par
   \colorRule
}
\bgroup
\colorSample{385}
\colorSample{809}
\egroup


\section{Specifying colors by name}

The easier way to specify colors is to use the pre-build names available
with the package drivers.











  \let\sidenote\footnote
\let\citep\footcite
\chapter{How to Develop your Own Class or Package}

\cxset {epigraph width=0.67\textwidth}

\epigraph{First there was one user and I took a lot of time to satisfy myself. Then I had 10 users, and a whole new level of difficulties arose. Then I had a hundred users and another level of things happened. I had a thousand users, I had ten thousand each of those were special phases in the development, important. I couldn't have gone with ten thousand until I'd done
it with a thousand. But each time a new wave of
changes came along, the idea was to have \tex get
better, and not get more diverse as it needed to handle
new things.}{Donald Knuth}

\parindent1em

\section{Introduction}


To \emph{make} a book is an interesting and somewhat involved process\footcite{town}. The text is set in type and printed on pages, the pages are gathered and folded into signatures and these are gathered and folded into signatures and these are then bound and covered. Many of the aspects of this process that has passed down to us by previous generations is discussed extensively in other sections of this book.  Class authors have to distill this knowledge in a set of typographical rules to be described in a class file. The first thing such an author must do is to describe the \emph{rationale} of developing such a class. The \docClass{octavo}\citep{octavo} class was developed to enable printing books in dimensions that follow traditional styles. The \citep{memoir}  class to offer a flexible system on which other classes could be based and so does \citep{koma}. The |tufte-book| and |tufte-handout| classes to provide a style that resembles those found in Tufte books. Many Universities offer \emph{Thesis} classes to standardize the way these are produced. Many of these Universities, translated the styles previously typed and the results are a typographical disaster, only mitigated by the ability to display beautiful mathematics. As these are printed on standard \emph{photocopy paper} one cannot do much with the layout. 
\section{What is a class?}
A class is simply a file with the extension \docExtension{cls} containg a set of macros. 
A class can load another class.
\section{Identifying your class}

The first thing a class must do is to identify any other formats it needs and to announce
its name. This is accomplished using the two commands 
\refCom{NeedsTeXFormat} and \refCom{ProvidesClass}.

The following example, delares the version of \LaTeXe\ that it requires and then
gives the class name. It can be found in the preable of most well written classes. You should also put some remarks to identify you as the author, the version number and other similar details. These are discussed in more detail in the next Chapter, where you will see how to automate documentation for your class.

\begin{teX}
\NeedsTeXFormat{LaTeX2e}[1994/06/01]
\ProvidesClass{myclass-book}[2010/12/11 v3.5.0 myclass-book]
\end{teX}

The above syntax must be followed exactly so that this information can be
used by \texttt{LoadClass} or \texttt{documentclass} (for classes) or \docAuxCommand{RequirePackage}
 or\cmd{usepackage} (for packages) to test that the release is not too old.
The whole of this $<release-info>$ information is displayed by \docAuxCommand{listfiles} and
should therefore not be too long.

\begin{teX}
% Load the common style elements
\input{myclass-common.def}
\end{teX}


Another command that can be used is \docAuxCommand{ProvidesFile}. 
This is similar to the two previous commands except that here the fullname,
including the extension, must be given. It is used for declaring any files other
than main class and package files.

This is useful, if you decide to have your main definitions in a separate file.

\section{Class Options}

Before we see in detail how to add options to a class, we need to review a package called
\pkgname{xkeyval}. Unless you are in the business of re-discovering wheels, this is an absolute must
for developing, readable and maintenable code and your class is to provide many options. 
\begin{teX}
\usepackage[textcolor=red,font=times]{mypack}
\end{teX}

Class options are best set by using booleans\docAuxCommand{newboolean}.

We first set a new boolean that we |name@myclass@afourpaper.| This is used using the package
\texttt{ifthen}\sidenote{The ifthen package was developed by 
David Carlisle, can be downloaded at \url{ http://www.ifi.uio.no/it/latex-links/ifthen.pdf }} 
Then we can |DecalareOptionX| and we set the boolean to default to true. If the user then types

myclass[a4paper]

The a4paper options will be set. This is a much better and concise way of defining options.
\cmd{newboolean}


\begin{teX}
\newboolean{@myclass@afourpaper}
\DeclareOptionX[myclass]<common>{a4paper}
  {
   \setboolean{@myclass@afourpaper}
   {true}
  }
\end{teX}
\medskip

Note that the command provide by \texttt{ifthen} \docAuxCommand{setboolean} takes true or false, as \#2, and sets \#1 accordingly. In the above code we set the option as true. 


It is much easier and most programmers use the \texttt{ifthen} package to check
for option booleans

\begin{teX}
\ifthenelse{\boolean{@myclass@afourpaper}}
  {\geometry{
        a4paper,
        left=24.8mm,
        top=27.4mm,
        headsep=2\baselineskip,
        textwidth=107mm,
        marginparsep=8.2mm,
        marginparwidth=49.4mm,
        textheight=49\baselineskip,
        headheight=\baselineskip
    }
  }
 {}
\end{teX}

\section{Set-up the font sizes}

LaTeX does not provide definitions of all the font-sizes. Unless you are
extending an existing class, this is one of the first tasks you need to 
do in your new class.

Normally class authors will define all the commonly defined size commands,
such as  \cmd{small}, \cmd{normalsize} and other similar commands.

In the example shown below, we first start by defining the \cmd{normalsize} font
size. In this book the \cmd{\normalsize}  is defined as 14pt. We also define the vertical
spaces that we need to have abovedisplay and belowdisplayskip. These are all very difficult to
remember and once you have something you are happy with, just copy from class to class
or even define a samll definition file to keep them all together.


{\fontfamily{phv}\selectfont Helvetica looks like this}
and {\fontencoding{OT1}\fontfamily{ppl} Palatino looks like this}.


 The user has access to a number of commands which change the size of
 the fount, relative to the `main' size used for the bulk of the text.


 These \cmd{size} commands issue a \cmd{@setfontsize}\index{Latex kernel!@setfontsize} 
 command.

\begin{teX}
  \@setfontsize\size\font-size{baselineskip} where:
\end{teX}



  \begin{description}
    \item {font-size} The absolute size of the fount to use from
        now on.
    \item{baselineskip} The normal value of \cmd{baselineskip}
        for the size of the fount selected. (The actual value will be
       % |\baselinestretch| * \meta{baselineskip}.)
    \end{description}

A number of commands, defined in the \LaTeX  kernel, shorten the
following  definitions and are used throughout. These are:

    \begin{center}
    \begin{tabular}{ll@{\qquad}ll@{\qquad}ll}
    \verb=\@vpt= & 5 & \verb=\@vipt= & 6 & \verb=\@viipt= & 7 \\
    \verb=\@viiipt= & 8 & \verb=\@ixpt= & 9 & \verb=\@xpt= & 10 \\
    \verb=\@xipt= & 10.95 & \verb=\@xiipt= & 12 & \verb=\@xivpt= & 14.4\\
    \ldots
    \end{tabular}
    \end{center}


\subsection{Setting up the normalsize}
 The user command to obtain the `main' size is \cmd{normalsize}. \LaTeX\
 uses \cmd{@normalsize} \index{Latex kernel!@normalsize} when referring to the main size and maintains this
 value even if \docAuxCommand{normalsize} is redefined. The \docAuxCommand{normalsize} macro also
  sets values for \cmd{abovedisplayskip}, \cmd{abovedisplayshortskip} and 
\cmd{belowdisplayshortskip}.



\begin{teX}
%%
% Set the font sizes and baselines to match Tufte's books
% normalsize
%%
\renewcommand\normalsize{%
   \@setfontsize\normalsize\@xpt{14}%
   \abovedisplayskip 10\p@ \@plus2\p@ \@minus5\p@
   \abovedisplayshortskip \z@ \@plus3\p@
   \belowdisplayshortskip 6\p@ \@plus3\p@ \@minus3\p@
   \belowdisplayskip \abovedisplayskip
   \let\@listi\@listI}

\normalbaselineskip=14pt
\normalsize
\end{teX}



\begin{teX}
\renewcommand\small{%
   \@setfontsize\small\@ixpt{12}%
   \abovedisplayskip 8.5\p@ \@plus3\p@ \@minus4\p@
   \abovedisplayshortskip \z@ \@plus2\p@
   \belowdisplayshortskip 4\p@ \@plus2\p@ \@minus2\p@
   \def\@listi{\leftmargin\leftmargini
               \topsep 4\p@ \@plus2\p@ \@minus2\p@
               \parsep 2\p@ \@plus\p@ \@minus\p@
               \itemsep \parsep}%
   \belowdisplayskip \abovedisplayskip
}
\renewcommand\footnotesize{%
   \@setfontsize\footnotesize\@viiipt{10}%
   \abovedisplayskip 6\p@ \@plus2\p@ \@minus4\p@
   \abovedisplayshortskip \z@ \@plus\p@
   \belowdisplayshortskip 3\p@ \@plus\p@ \@minus2\p@
   \def\@listi{\leftmargin\leftmargini
               \topsep 3\p@ \@plus\p@ \@minus\p@
               \parsep 2\p@ \@plus\p@ \@minus\p@
               \itemsep \parsep}%
   \belowdisplayskip \abovedisplayskip
}
\renewcommand\scriptsize{\@setfontsize\scriptsize\@viipt\@viiipt}
\renewcommand\tiny{\@setfontsize\tiny\@vpt\@vipt}
\renewcommand\large{\@setfontsize\large\@xipt{15}}
\renewcommand\Large{\@setfontsize\Large\@xiipt{16}}
\renewcommand\LARGE{\@setfontsize\LARGE\@xivpt{18}}
\renewcommand\huge{\@setfontsize\huge\@xxpt{30}}
\renewcommand\Huge{\@setfontsize\Huge{24}{36}}

%% Define a HUGE for fun
\newcommand\HUGE{\@setfontsize\Huge{38}{47}}  
\end{teX}


\section{Adjusting paragraph parameters}

 The parameters which control \TeX 's behaviour when typesetting
 paragraphs receive a bit of a tweak here. Contrary to the usual
 behaviour of modifying the grid with glue when difficulties are
 encountered with vertical space, here we shall try to counteract
 these tendencies and enforce as much as possible uniformity of the 
 grid of lines.

A good value for paragraph indentation is \texttt{parindent 0.5pt}, for vertical spacing between
paragraphs that are indented use 0pt. At this point if you are using any marginals it is a good idea
to allow hyphenation with the \docpkg{ragged2e} package. Since marginals use very narrow paragraphs you may
get a very funny looking marginal text. Using the package, adjustments can be made to hyphenate
the marginal text.

\begin{teXXX}
%%
% \RaggedRight allows hyphenation

\RequirePackage{ragged2e}
\setlength{\RaggedRightRightskip}{\z@ plus 0.08\hsize}
\setlength{\RaggedRightParindent}{1pc}

% Paragraph indentation and separation for normal text
\newcommand{\@tufte@reset@par}{%
  \setlength{\RaggedRightParindent}{1.0pc}%
  \setlength{\parindent}{1pc}%
  \setlength{\parskip}{0pt}%
}
\@tufte@reset@par

% Paragraph indentation and separation for marginal text
\newcommand{\@tufte@margin@par}{%
  \setlength{\RaggedRightParindent}{0.5pc}%
  \setlength{\parindent}{0.5pc}%
  \setlength{\parskip}{0pt}%
}
\end{teXXX}


\section{Formatting Chapters and Sections}

The section on Chapters etc, has more on this, but we will touch on it briefly.
Most recent class developerss use the \pkg{titlesec} and \pkg{titletoc} package to handle the complexity 
of these commands. With the |phd| package this is unecessary. 

\begin{teXXX}
\titleformat{\subsection}%
  [hang]% shape
  {\normalfont\large}% format applied to label+text removed \itshape
  {\thesubsection}% label
  {1em}% horizontal separation between label and title body
  {}% before the title body
  []% after the title body
\end{teXXX}

These are normally followed by the ``titlespacing" commands to define the space around these sections.

\begin{teXXX}
%% We set the titlespacing using the package titlesec and titletoc
%
\titlespacing*{\chapter}{0pt}{20pt}{40pt}
\titlespacing*{\section}{0pt}{3.5ex plus 1ex minus .2ex}{2.3ex plus .2ex}
\titlespacing*{\subsection}{0pt}{3.25ex plus 1ex minus .2ex}{1.5ex plus.2ex}
\end{teXXX}

\section{Adjusting the Index}

For classes representing books, the index is treated like a chapter whereas for others it is normally
treated like a section. Whatever your document ends up like, indices are best done in a multi-column environment.
One possibility is shown below, using the package "multcol".

\begin{teXXX}
\RequirePackage{multicol}
\renewenvironment{theindex}
  {\begin{fullwidth}%
    \small%
    \ifthenelse{\equal{\@tufte@class}{book}}%
      {\chapter{\indexname}}%
      {\section*{\indexname}}%
    \parskip0pt%
    \parindent0pt%
    \let\item\@idxitem%
    \begin{multicols}{3}%
  }
  {\end{multicols}%
    
\renewcommand\@idxitem{\par\hangindent 2em}
\renewcommand\subitem{\par\hangindent 3em\hspace*{1em}}
\renewcommand\subsubitem{
    \par\hangindent 4em\hspace*{2em}
}
\renewcommand\indexspace{
    \par\addvspace{
       1.0\baselineskip plus 0.5ex minus 0.2ex}\relax
    }%
%we now  swallow the letter heading in the index
\newcommand{\lettergroup}[1]{}

\end{teXXX}

The code, renews the "theindex" environment, with minor tweaks and defines it as a three column
layout at "fullwidth".

\section{Provide some hooks}
It is useful at the end of the class to allow for localization of the class
by importing a local file. This is easily achieved by checking if the file exists
and then loading it.  If there is a |myclass-book-local.sty|  file, load it.

\begin{teX}
\IfFileExists{myclass-book-local.tex}
  {input{myclass-book-local}
   \MyClassInfoNL{Loading myclass-book-local.tex}}
  {}
\end{teX}

If you intent to publish your class, you may also want to consider adding a hook for a patch-file.


\section{The final act of kindness to your users}
Many common classes, such as the |memoir| use such a tactic to avoid breaking old code.\index{IfFileExists}

\begin{teX}
 \IfFileExists{mypatch.sty}{%
 \RequirePackage{mypatch}}{}
\end{teX}


\parindent1em
\chapter{How to Package Your Class}

In the previous chapter we have outlined the main sections that you probably need
to define in your class. In the examples we have used we just typed the examples
as |example.cls| or |package.sty|.

In this chapter we will go over the packaging of the class
and automating the generation of user documentation, using the |doc| and \pkg{DocStrip}\footcite{docstrip}
programs in files with an extension |.dtx|. The DocStrip program is an amazing piece of code that was originally
created by Frank Mittelbach to accompany the |doc| package. The idea behind it was to remove comment lines
in order to reduce the execution time of the program. Having created the DocStrip program to remove comment lines from  programs it became feasible to do more than just strip comments.
Wouldn't it be nice to have a way to include parts of the code only when some
condition is set true? Wouldn't it be as nice to have the possibility to split the
source of a \tex program into several smaller files and combine them later into
one `executable'? Both these wishes have been implemented in the DocStrip program.



You should also be
familiar with ``LaTeX2e'' for Class and Package Writers”, which is available
from CTAN (\url{http://www.ctan.org}) and comes with most LaTeX2e" distributions
in a file called clsguide.dvi.\footcite{pakin2004}  Finally, you should know how to
install packages that are shipped as a \texttt{.dtx} file plus a \texttt{.ins} file.

style (.sty) file is primarily a collection of macro and
environment definitions. One or more style files (e.g., a main style file that
\cs{input}  or \cs{RequirePackages} multiple helper files) is called a package.
Packages are loaded into a document with \cs{usepackage}\marg{main .sty fille}.
In the rest of this document, we use the notation \meta{package} to represent
the name of your package.


Motivation The important parts of a package are the code, the documentation
of the code, and the user documentation. Using the \docpkg{Doc}  and
DocStrip programs, it’s possible to combine all three of these into a single,
documented LATEX(.dtx) file. The primary advantage of a .dtx file is that
it enables you to use arbitrary LATEX constructs to comment your code.
Hence, macros, environments, code stanzas, variables, and so forth can be
explained using tables, figures, mathematics, and font changes. Code can
be organized into sections using LATEX’s sectioning commands. Doc even
facilitates generating a unified index that indexes both macro definitions (in
the LATEX code) and macro descriptions (in the user documentation). 

This emphasis on writing verbose, nicely typeset comments for code—essentially
treating a program as a book that describes a set of algorithms—is known
as literate programming \cite{literate} and has been in use since the early days of \tex\ .

Furthermore,
this tutorial shows how to write a single file that serves as both documentation
and driver file, which is a more typical usage of the \texttt{Doc} system than
using separate files.

\subsection{The .ins file}

The first step in preparing a package for distribution is to write an installer
(|.ins|) file. An installer file extracts the code from a |.dtx| file, uses \pkg{docstrip}
to strip off the comments and documentation, and outputs a |.sty| file. The
good news is that a |.ins| file is typically fairly short and doesn’t change
significantly from one package to another.

\paragraph{License} The |ins| files usually start with comments specifying the copyright and license
information:

\begin{minted}{latex}
%%
%% Copyright (C) year by your name %%
%% This file may be distributed and/or modified under the
%% conditions of the LaTeX Project Public License, either
%% version 1.2 of this license or (at your option) any later
%% version. The latest version of this license is in:
%%
%% http://www.latex-project.org/lppl.txt
%%
%% and version 1.2 or later is part of all distributions of
%% LaTeX version 1999/12/01 or later.
%%
\end{minted}

The LATEX Project Public License (LPPL) is the license under which most
packages—and LATEX itself—are distributed. Of course, you can release your
package under any license you want; the LPPL is merely the most common
license for LATEX packages. The LPPL specifies that a user can do whatever
he wants with your package—including sell it and give you nothing in return.
The only restrictions are that he must give you credit for your work, and
he must change the name of the package if he modifies anything to avoid
versioning confusion.
The next step is to load DocStrip:

\begin{teXXX}
%%\input docstrip.tex
%%\keepsilent
\end{teXXX}



By default, DocStrip gives a line-by-line account of its activity. These messages
aren’t terribly useful, so most people turn them off, by using the command \docAuxCommand{keepsilent}:

\begin{teXXX}
\keepsilent
\end{teXXX}


A system administrator can specify the base directory under which all
TEX-related files should be installed, e.g., \texttt{/usr/share/texmf}. (See
\cmd{\BaseDirectory} in the DocStrip manual.) The |ins| file specifies where
its files should be installed relative to that. The following is typical:

\begin{teXXX}
\usedir{tex/latex/packagename}
\preamble
htexti \endpreamble
\end{teXXX}



The next step is to specify a preamble, which is a block of commentary that
will be written to the top of every generated file:

\begin{minted}{latex}
\preamble
----------------------------------------------------------------
phddoc --- A class to typeset LaTeX code.
E-mail: yannislaz@gmail.com
Released under the LaTeX Project Public License v1.3c or later
See http://www.latex-project.org/lppl.txt
----------------------------------------------------------------
\endpreamble
\end{minted}


The preceding preamble would cause |package.sty|  to begin as follows:

\begin{minted}{latex}
%%
%% This is file `phddoc.cls',
%% generated with the docstrip utility.
%%
%% The original source files were:
%%
%% phddoc.dtx  (with options: `class')
%% ----------------------------------------------------------------
%% phddoc --- A class to typeset LaTeX code.
%% E-mail: yannislaz@gmail.com
%% Released under the LaTeX Project Public License v1.3c or later
%% See http://www.latex-project.org/lppl.txt
%% ----------------------------------------------------------------
\end{minted}

We now reach the most important part of a .ins file: the specification of
what files to generate from the |.dtx| file. The following tells DocStrip to
generate hpackagei.sty from hpackagei.dtx by extracting only those parts
marked as `package'  in the .dtx file. (Marking parts of a .dtx file is
described later on.)

\begin{teXXX}
\generate{\file{<package>.sty}{\from{<package>.dtx}{package}}}
\end{teXXX}

\cmd{\generate} can extract any number of files from a given .dtx file. It can
even extract a single file from multiple |.dtx| files. See the DocStrip manual
for details.

Personally I also generate README.md files in |markdown| format as well, so that
when they get uploaded to |github| they can be rendered nicely.

\begin{minted}{latex}
\generate{\file{\jobname.md}{\from{\jobname.dtx}{readmemd}}}
\end{minted}

The text has to be wriiten using `guards' with the tag |readmd|

\begin{minted}{latex}
%<*readmemd>
# The `phddoc` LaTeX2e class

The `phd` latex package and the class with the same name provide
convenient methods to create new styles for books, reports
and articles. It also loads the most commonly used packages 
and resolves conflicts.
%</readmemd>
\end{minted}

\subsection{Generating messages} 

The next part of a |.ins| file consists of commands to output a message to
the user, telling him what files need to be installed and reminding him how
to produce the user documentation. The following set of \cmd{Msg} commands is
typical:

\begin{minted}[
frame=lines,
framesep=2mm,
baselinestretch=1.2,
fontsize=\footnotesize,
linenos
]{latex}
\obeyspaces
\Msg{****************************************************}
\Msg{* *}
\Msg{* To finish the installation you have to move the *}
\Msg{* following file into a directory searched by TeX: *}
\Msg{* *}
\Msg{* packagei.sty *}
\Msg{* *}
\Msg{* To produce the documentation run the file *}
\Msg{* package.dtx through LaTeX. *}
\Msg{* *}
\Msg{* Happy TeXing! *}
\Msg{* *}
\Msg{****************************************************}
Note the use of \obeyspaces to inhibit \tex from collapsing multiple spaces
into one.
\endbatchfile
\end{minted}


Appendix A.1 lists a complete, skeleton .ins file. Appendix A.2 is similar
but contains slight modifications intended to produce a class (|.cls|) file
instead of a style (|.sty|) file

\section{What to put in a  .dtx file}
We started describing the |.ins| install file first. The next file we will describe is
the |.dtx| file. This holds both the code definitions as well as the user documentation.

A |dtx|\ file contains both the commented source code and the user documentation
for the package. Running a |dtx|  file through |latex| typesets the
user documentation, which usually also includes a nicely typeset version of
the commented source code.

Due to some Doc trickery, a |dtx|  file is actually evaluated twice. The first
time, only a small piece of \latex\  driver code is evaluated. The second time,
comments in the |dtx|  file are evaluated, as if there were no `\%'  preceding
them. This can lead to a good deal of confusion when writing |dtx|  files
and occasionally leads to some awkward constructions. Fortunately, once
the basic structure of a |dtx|  file is in place, filling in the code is fairly
straightforward.

\paragraph{Guards} If you open any .dtx file you will notice that the lines either start with a \%
sign or sometimes with a percentage sign and |<|\textit{guard}|>|. The latter is called a guard and they are in a way
like html tags. They have a starting and an ending tag. In the example below there are two different guards
|<*10pt>...</10pt>| and |<*11pt></11pt>|. Unlike html tags guards are boolean expressions! You can use:
\begin{quote}
\textbar  ! \&  
\end{quote}

The \textbar stands for disjunction (OR), the \& stands for conjunction (AND) and the ! (NOT) stands for
negation. The terminal is any sequence of letters and evaluates to true iff it
occurs in the list of options that have to be included.

\begin{minted}{latex}
%<*10pt|11pt|12pt>
... code
%</10pt|11pt|12pt>
\end{minted}

A longer example from KOMA shows the concept better.

\fvset{gobble=0}
\begin{minted}[
frame=lines,
framesep=2mm,
baselinestretch=1.2,
fontsize=\footnotesize,
linenos
]{latex}
%    \begin{macrocode}
\def\normalsize{%
%<*10pt>
  \@setfontsize\normalsize\@xpt\@xiipt
  \abovedisplayskip 10\p@ \@plus2\p@ \@minus5\p@
  \abovedisplayshortskip \z@ \@plus3\p@
  \belowdisplayshortskip 6\p@ \@plus3\p@ \@minus3\p@
%</10pt>
%<*11pt>
  \@setfontsize\normalsize\@xipt{13.6}%
  \abovedisplayskip 11\p@ \@plus3\p@ \@minus6\p@
  \abovedisplayshortskip \z@ \@plus3\p@
  \belowdisplayshortskip 6.5\p@ \@plus3.5\p@ \@minus3\p@
%</11pt>
... 
%    end{macrocode}
\end{minted}

If the guards only contain a one line of text, then a short form is provided as |<10pt>|. It is unecessary to provide a closing tag and the `*' is omitted. The example below from the KOMA classes shows a quite ingenious way of writing the |\ProvidesFile| macro in
the different files; one for each tag. 
Two kinds of optional code are supported: one can either have optional code
that is on one line of tex code.

To distinguish both kinds of optional code the `guard modier' has been introduced. 
The `guard modifier' is one character that immediately follows the < of
the guard. It can be either * for the beginning of a block of code, or / for the end
of a block of code. The beginning and ending guards for a block of code have to
be on a line by themselves.

When a block of code is not included, any guards that occur within that block
are not evaluated.


\begin{minted}{latex}
%    \begin{macrocode}
\ProvidesFile{%
%<10pt>  scrsize10pt.clo%
%<11pt>  scrsize11pt.clo%
%<12pt>  scrsize12pt.clo%
}[\KOMAScriptVersion\space font size class option %
%<10pt>  (10pt)%
%<11pt>  (11pt)%
%<12pt>  (12pt)%
]
%    \end{macrocode}
\end{minted}

In the |.ins| file one could write to generate the various |.clo| files.:

\begin{minted}{latex}
\generate{\usepreamble\defaultpreamble
  \file{scrsize10pt.clo}{%
    \from{scrkernel-version.dtx}{clo,10pt}%
    \from{scrkernel-fonts.dtx}{clo,10pt}%
    \from{scrkernel-paragraphs.dtx}{clo,10pt}%
  }%
  \file{scrsize11pt.clo}{%
    \from{scrkernel-version.dtx}{clo,11pt}%
    \from{scrkernel-fonts.dtx}{clo,11pt}%
    \from{scrkernel-paragraphs.dtx}{clo,11pt}%
  }%
  \file{scrsize12pt.clo}{%
    \from{scrkernel-version.dtx}{clo,12pt}%
    \from{scrkernel-fonts.dtx}{clo,12pt}%
    \from{scrkernel-paragraphs.dtx}{clo,12pt}%
  }%
}%
\end{minted}

Becareful not to introduce spurious empy lines in your generated files by having empty lines in no-man's land, that is between tags.\footnote{In the phd package, I automatically generate the default settings from the |.dtx| files. In this case pgf will complain.}

\begin{minted}{latex}
%</install>

%<install>\endbatchfile
\end{minted}

\paragraph{The character table check } The second mechanism that Doc uses to ensure that a |dtx|  file is uncorrupted
is a character table. If you put the following command verbatim into
your |dtx|  file, then \pkg{Doc} will ensure that no unexpected character translation
took place in transport:

\begin{minted}[
frame=lines,
framesep=2mm,
baselinestretch=1.2,
fontsize=\footnotesize,
linenos,gobble=0,
]{latex}
% \CharacterTable
% {Upper-case \A\B\C\D\E\F\G\H\I\J\K\L\M\N\O\P\Q\R\S\T\U\V\W\X\Y\Z
% Lower-case \a\b\c\d\e\f\g\h\i\j\k\l\m\n\o\p\q\r\s\t\u\v\w\x\y\z
% Digits \0\1\2\3\4\5\6\7\8\9
% Exclamation \! Double quote \" Hash (number) \#
% Dollar \$ Percent \% Ampersand \&
% Acute accent \’ Left paren \( Right paren \)
% Asterisk \* Plus \+ Comma \,
% Minus \- Point \. Solidus \/
% Colon \: Semicolon \; Less than \<
% Equals \= Greater than \> Question mark \?
% Commercial at \@ Left bracket \[ Backslash \\
% Right bracket \] Circumflex \^ Underscore \_
% Grave accent \‘ Left brace \{ Vertical bar \|
% Right brace \} Tilde \~}
A success message looks like this:
***************************
* Character table correct *
***************************

and an error message looks like this:
! Package doc Error: Character table corrupted.
\end{minted}

\paragraph{DoNotIndex} When producing an index, \pkg{doc} normally indexes every control sequence
(i.e., backslashed word or symbol) in the code. The problem with this level
of automation is that many control sequences are uninteresting from the
perspective of understanding the code. For example, a reader probably
doesn’t want to see every location where \cs{if} is used—or \cs{the} or \cs{let} or
\cs{begin} or any of numerous other control sequences.

As its name implies, the \cs{DoNotIndex} command gives |Doc| a list of control
sequences that should not be indexed. \cs{DoNotIndex} can be used any
number of times, and it accepts any number of control sequence names per
invocation:

\begin{minted}[
frame=lines,
framesep=2mm,
baselinestretch=1.2,
bgcolor=white,
fontsize=\footnotesize,
linenos
]{latex}
\DoNotIndex{\#,\$,\%,\&,\@,\\,\{,\},\^,\_,\~,\ }
\DoNotIndex{\@ne}
\DoNotIndex{\advance,\begingroup,\catcode,\closein}
\DoNotIndex{\closeout,\day,\def,\edef,\else,\empty, \endgroup}
\end{minted}


\subsection{User documentation}

We can finally start writing the user documentation. A typical beginning
looks like this:

\begin{minted}[
frame=lines,
framesep=2mm,
baselinestretch=1.2,
bgcolor=white,
fontsize=\footnotesize,
linenos
]{latex}
% \title{The \textsf{package} package\thanks{This document
% corresponds to \textsf{package}~\fileversion,
% dated~\filedate.}}
% \author{your name \\ \texttt{your e-mail address}}
%
% \maketitle
\end{minted}


The title can certainly be more creative, but note that it’s common for
package names to be typeset with \docAuxCommand{textsf} and for \docAuxCommand{thanks} to be used to
specify the package version and date. This yields one of the advantages
of literate programming: Whenever you change the package version (the
optional second argument to \docAuxCommand{ProvidesPackage}), the user documentation
is updated accordingly. Of course, you still have to ensure manually that
the user documentation accurately describes the updated package.

Write the user documentation as you would any \latexe document, except
that you have to precede each line with a |\%|. Note that the |ltxdoc| document
class is derived from article, so the top-level sectioning command is
|\section|, not |\chapter|.



\section{General tips for defining a Class}

Evaluate, if there is a class that is nearer to what you wish to achive. If not do a set of
requirements.

Book structure - start with book or |Octavo| if you need to hack extensively. If not use memoir, |koma| or |tufte-book|.

Paragraph looks

Lists

Figures

Bibliography and citations

Footnotes

Index

Title pages

Book Cover

Language support

Mathematics

Graphs and figures

Typography - fonts, indentations fontsize etc

headers and footers


\section{Declaring Options}

Most classes or packages will have a good deal of options. These are declared using the
\docAuxCommand{DeclareOption} command. In this part no package loading should take place.

\begin{docCommand} {DeclareOption} { \marg{option} \marg{code}}
  The argument option is the name of the option being declared and the \marg{code} is the
  code that will execute if this option is requested.
\end{docCommand}


\begin{docCommand}{DeclareOption*} { \marg{code}}
  The argument \meta{code} in the star version of the command specifies the action to be 
  taken if an unknown option is specified. Within this argument the \docAuxCommand{CurrentOption}
  refers to the name of the option in question. 
  
\end{docCommand}

For example one could pass all such options
  to another package, using:
  \begin{verbatim}
  \DeclareOption*{\PassOptionsToPackage{\CurrentOption}{A}}
  \end{verbatim}


\section{Executing Options}

Normally after the options have been defined, one would need to provide default values and 
the options need to be executed. 

\begin{docCmd} {ExecuteOptions} { \marg{option list}}
  
\end{docCmd}

You can also |\ExecuteOptions| when declaring other options. There is one caveat. This command
can only be executed prior to executing the |\ProcessOptions| command because, as one of
its last actions, the latter command reclaims all of the memory taken up by the code for
the declared options.

\begin{docCmd} {ProcessOptions*} {}
\end{docCmd}

For some packages it is preferable or essential to process options in the order they
appear in the |usepackage| commands rather than using the order given through the
sequence of the \refCmd{DeclareOption} commands. In this case it one has to use
the star version of the command, i.e, |\ProcessOptions*| rather than |\ProcessOptions|.

\section{Special Commands for class files}

It is sometimes preferable to define a new class based on another and hence to extend it.
To support this concept the \latexe kernel provides two commands, \docAuxCommand{LoadClass} and
\docAuxCommand{PassOptionsToClass}. These two commands can then be used to develop a new class, by adding and extending the functionality of the loaded class.

\begin{docCommands}
\refCom{LoadClass}{ \oarg{option list}\marg{class}\oarg{release}}
\end{docCommands}  
  
For example the |ltxdoc| class loads the standard |article| class. The \pkg{tufte-book} class loads
the |book| class. The best way to understand the concepts discussed here is to
study these classes.

\section{A minimal class}

\begin{texexample}{Model Class}{ex:modelclass}

\begin{filecontents}{phdexampleclass.cls}
\NeedsTeXFormat{LaTeX2e}
\ProvidesClass{phdexampleclass}[2015/07/07]
\renewcommand\normalsize{\fontsize{}{10pt}{12pt}\selectfont}
\setlength\textwidth{6.5in}
\setlength\textheight{5in}
\pagenumbering{arabic}
\end{filecontents}

\end{texexample}

\vfill
\endinput





















  
\makeatletter\@specialtrue\makeatother
\cxset{steward,
  numbering=arabic,
  custom=stewart,
  offsety=0cm,
  image={elevendays.jpg},
  texti={An introduction to the use of font related commands. Thee chapter also gives a historical background to font selection using \tex and \latex. },
  textii={In this chapter we discuss keys that are available through the \texttt{phd} package. The image is William Hogarth's painting (c. 1755) which is the main source for `Give us our Eleven Days'.
 },
}
\chapter{Handling Dates and Time}
\label{dates}\label{ch:dates}

\parindent1.5em

\section{Problems with time and date}

\tex and \latex do not offer\cite{Thanh:TB18-4-249} any sophisticated support for date and time routines.
One can get the current system date using \cmd{\today}
Typing |\today| we get \texttt{\today}. Normally the format of |\today| would vary from class to class, as this is one of the first things class authors style. The |\today| command is build using three other commands.\footnote{It appears that there is also a time=now in IniTeX} Another issue with such commands is the fact that they are dependent on the language used and the prevalent conventions.

\begin{texexample}{Basic date example}{}
\the\day

\the\month

\the\year

\meaning\today
\end{texexample}

{
\makeatletter
|\the\month| \the\month

|\the\day| \the\day

|\the\time| \two@digits{\the\count@}:\two@digits{\the\count2}
\makeatother}

\tex offers only one macro \cmd{\time} which is the time in hours since midnight.


The code below is from the \latex kernel and can be found in the \docfile{ltdirchk.dtx}

\startlineat{126}
\begin{teX}
\count@\time
\divide\count@ 60
\count2=-\count@
\multiply\count2 60
\advance\count2 \time

\edef\today{%
  \the\year/\two@digits{\the\month}/\two@digits{\the\day}:%
  \two@digits{\the\count@}:\two@digits{\the\count2}
 }
\end{teX}

\begin{texexample}{Time in LaTeX}{}
\makeatletter
\count@\time
\divide\count@ 60
\count2=-\count@
\multiply\count2 60
\advance\count2 \time

\edef\today{%
\the\year/\two@digits{\the\month}/\two@digits{\the\day}:%
\two@digits{\the\count@}:\two@digits{\the\count2}}


\today:   \the\count2:  \the\count@

the time \the\time
\makeatother
\end{texexample}




\section{Getting the time}

\tex has a primitive register that contains “the number of minutes since midnight”; with this knowledge it’s a moderately simple programming job to print the time (one that no self-respecting Plain \tex user would bother with anyone else’s code for).

However, \latex provides no primitive for “time”, so the non-programming LaTeX user needs help.


\section*{Getting the time using pdf internal commands}

One of the problems with \tex's |\time| is that it is not possible to count seconds. One way to by-pass this is to use the pdfLaTeX or pdfTeX macro
\cmd{pdfcreationdate}.


\texttt{> \textbackslash pdfcreationdate}

\texttt{\pdfcreationdate}

As you can observe from the above, the pdf has a special format and it even includes infromation about the timezone.

PDF defines a standard date format, which closely follows that of the international standard ASN.1 (Abstract Syntax Notation One), defined in ISO/IEC 8824 (see the Bibliography). A date is a string of the form

|(D:YYYYMMDDHHmmSSOHH'mm')|

where

\begin{teX}
YYYY is the year
MM is the month
DD is the day (01-31)
HH is the hour (00-23)
mm is the minute (00-59)
SS is the second (00-59)
\end{teX}


O is the relationship of local time to Universal Time (UT), denoted by one of the characters +, -, or Z (see below)
HH followed by ' is the absolute value of the offset from UT in hours (00–23)
mm followed by ' is the absolute value of the offset from UT in minutes (00–59)

The quotation mark character (') after HH and mm is part of the syntax. All fields after the year are optional. (The prefix D:, although also optional, is strongly recommended.) The default values for MM and DD are both 01; all other numerical fields default to zero values. A plus sign (+) as the value of the O field signifies that local time is later than UT, a minus sign (-) that local time is earlier than UT, and the letter Z that local time is equal to UT. If no UT information is specified, the relationship of the specified time to UT is considered to be unknown. Whether or not the time zone is known, the rest of the date should be specified in local time.

For example, December 23, 1998, at 7:52 PM, U.S. Pacific Standard Time, is represented by the string,


|D:199812231952-08'00'|


Two packages are available, both providing ranges of ways of printing the date, as well as of the time: this question will concentrate on the time-printing capabilities, and interested users can investigate the documentation for details about dates.


\section*{Using \protect\texttt{datetime}}

The \pkg{datetime} package defines two time-printing functions: \cmd{\xxivtime} (for 24-hour time), \cmd{\ampmtime} (for 12-hour time) and \cmd{\oclock} (for time-as-words, albeit a slightly eccentric set of words).

\emphasis{xxivtime,ampmtime,oclock}

\begin{texexample}{Using DateTime}{ex:datetime}
The time is \xxivtime
The time is \ampmtime
The time is \oclock

The time is \xxivtime

The time is \ampmtime

The time is \oclock
\end{texexample}


\section{Using scrtime}

The \pkg{scrtime} package (part of the compendious KOMA-Script bundle) takes a package option (12h or 24h) to specify how times are to be printed. The command \cmd{\thistime} then prints the time appropriately (though there's no am or pm in 12h mode). The \cmd{\thistime} command also takes an optional argument, the character to separate the hours and minutes.


\begin{texexample}{Example scrtime}{ex:scrtime}
The time is \thistime
The time is \thistime[h]
\end{texexample}

\label{datesend}


The time is \thistime[ hours ] minutes 

{> The time is \thistime*[:] } 

|\thistime*| works in almost the same way as |\thistime|. The only
diffrence is that unlike with |\thistime|, with |\thistime*| the value of
the minute field is not preceded by a zero when its value is less than 10.
Thus, with |\thistime| the minute field has always two places.



\begin{comment}
%% Hack to get the time zone
%% This is based on a macro at http://tex.stackexchange.com/questions/8612/write-date-time-and-time-zone by Will Robertson



\pdfcreationdate
\newcounter{temp}
\setcounter{temp}{1}
\let\Box=\boxed

\def\Box#1{\fbox{\strut\textbf{#1}$\scriptscriptstyle\,_{\thetemp}$\stepcounter{temp} }}

\Box{D}\Box{:} \Box{\color{red}2}\Box{\color{red}0}\Box{\color{red}1}%
\Box{\color{red}1}
 \Box{0}\Box{1}
\Box{\color{purple}1} \Box{\color{purple}1}\Box{0}

\newtoks\tyear
\newtoks\tmonth
\newtoks\tday
\newtoks\thour
\newtoks\tminutes
\newtoks\tseconds
\newtoks\UTCh

\def\grabtimezone #1#2#3#4#5#6#7#8{
\tyear={#3#4#5#6}%
\tmonth{#7#8}%
\grabtimezoneB}

\def\grabtimezoneB #1#2#3#4#5#6#7#8{
  \tday={#1#2}%
  \thour={#3#4}%
  \tminutes={#5#6}%
  \tseconds={#7#8}%
\grabtimezoneC}

%\def\grabtimezoneC #1#2#3'#4'{\UTCh={sign:#1  hr: #2#3 min: #4}}
%\expandafter \grabtimezone\pdfcreationdate
%
%
%%\@namedef{timezone+0930}{CST}
%%\@namedef{timezone+1000}{EST}
%%\@namedef{timezone+1030}{CST'}
%
%\the\tyear 
%
%\the\tmonth
%
%\the\tday
%
%\the\thour
%
%\the\tminutes
%
%\the\tseconds
%
%\the\UTCh
\end{comment}

\section*{Day of the Week}
The day of the week can be calculated using the |dow| macro that 
has been around for a while

\begin{comment}
\def\DayOfWeekLong{%
%
% 	Calculate day of the week, return "Sunday", etc.
%
  \newcount\dow				% Gets day of the week
  \newcount\leap			% Leap year fingaler
  \newcount\x				% Temp register
  \newcount\y 				% Another temp register
%		leap = year + (month - 14)/12;
  \leap=\month \advance\leap by -14 \divide\leap by 12
  \advance\leap by \year
%		dow = (13 * (month + 10 - (month + 10)/13*12) - 1)/5
  \dow=\month \advance\dow by 10
  \y=\dow \divide\y by 13 \multiply\y by 12
  \advance\dow by -\y \multiply\dow by 13 \advance\dow by -1 \divide\dow by 5
%		dow += day + 77 + 5 * (leap % 100)/4
  \advance\dow by \day \advance\dow by 77
  \x=\leap \y=\x \divide\y by 100 \multiply\y by 100 \advance\x by -\y
  \multiply\x by 5 \divide\x by 4 \advance\dow by \x
%		dow += leap / 400
  \x=\leap \divide\x by 400 \advance\dow by \x
%		dow -= leap / 100 * 2;
%		dow = (dow % 7)
  \x=\leap \divide\x by 100 \multiply\x by 2 \advance\dow by -\x
  \x=\dow \divide\x by 7 \multiply\x by 7 \advance\dow by -\x
  \ifcase\dow Sunday\or Monday\or Tuesday\or Wednesday\or
	Thursday\or Friday\or Saturday\fi
}

\def\DayOfWeekShort{%
%
% 	Calculate day of the week, return "Sunday", etc.
%
  \newcount\dow				% Gets day of the week
  \newcount\leap			% Leap year fingaler
  \newcount\x				% Temp register
  \newcount\y 				% Another temp register
%		leap = year + (month - 14)/12;
  \leap=\month \advance\leap by -14 \divide\leap by 12
  \advance\leap by \year
%		dow = (13 * (month + 10 - (month + 10)/13*12) - 1)/5
  \dow=\month \advance\dow by 10
  \y=\dow \divide\y by 13 \multiply\y by 12
  \advance\dow by -\y \multiply\dow by 13 \advance\dow by -1 \divide\dow by 5
%		dow += day + 77 + 5 * (leap % 100)/4
  \advance\dow by \day \advance\dow by 77
  \x=\leap \y=\x \divide\y by 100 \multiply\y by 100 \advance\x by -\y
  \multiply\x by 5 \divide\x by 4 \advance\dow by \x
%		dow += leap / 400
  \x=\leap \divide\x by 400 \advance\dow by \x
%		dow -= leap / 100 * 2;
%		dow = (dow % 7)
  \x=\leap \divide\x by 100 \multiply\x by 2 \advance\dow by -\x
  \x=\dow \divide\x by 7 \multiply\x by 7 \advance\dow by -\x
  \ifcase\dow Sun\or Mon\or Tue\or Wed\or
	Thur\or Fri\or Sat\fi
}


\DayOfWeekLong

\DayOfWeekShort
\end{comment}

\makeatother








The \pkg{datenumber} has been developed by J\"org-Michael Schr\"oder and provides commands to convert a date into a number. Turned around a date can be calculated also by a number. Additionally there are commands for incrementing and decrementing a date. Leap years and the Gregorian calendar reform are considered.
\index{dates}\index{dates!leap year}\index{dates! Gregorian calendar}

\section{Start year}

The start of the counting is determined with \verb+\setstartyear{year}+ (standard 1800). The first day of the start year gets the number 1. The value of \texttt{startyear} must be greater 0. It may not be larger than the year of a date to be calculated. If the difference of date and \texttt{startyear} is large, the calculation can last for a long time. The correct setting of the weekdays is guaranteed only if the value of \texttt{startyear} is 1800, 1900 or 2000.


\section{Counters}
There are five counters defined \doccmd{datenumber}, \doccmd{dateyear}, \doccmd{datemonth}

\begin{description}
\item[\texttt{datenumber}:] number of the day
\item[\texttt{dateyear}:] year
\item[\texttt{datemonth}:] month
\item[\texttt{dateday}:] day
\item[\texttt{datedayname}:] weekday: 1--7 (Monday--Sunday)
\end{description}


\section{Macros}
\subsection{Macros which operate with defined counters\label{macro}}
All counters specified above are updated by these macros. \verb+\datedayname+ and \verb+\datemonthname+ are also updated.

\begin{description}
\item[\texttt{\textbackslash setdatenumber\{year\}\{month\}\{day\}}:] Sets the counter \texttt{datenumber} to a value, which corresponds to the date.
\item[\texttt{\textbackslash setdatebynumber\{number\}}:] Sets the counters \texttt{dateyear}, \texttt{datemonth}, and \texttt{dateday} to values, which corresponds to the number.
\item[\texttt{\textbackslash nextdate}:] Sets the counters \texttt{dateyear}, \texttt{datemonth}, and \texttt{dateday} to the next date.
\item[\texttt{\textbackslash prevdate}:] Sets the counters \texttt{dateyear}, \texttt{datemonth}, and \texttt{dateday} to the previous date.
\item[\texttt{\textbackslash setdate\{year\}\{month\}\{day\}}:] Sets the counters \texttt{dateyear}, \texttt{datemonth}, and \texttt{dateday} to \texttt{year}, \texttt{month}, and \texttt{day}.
\item[\texttt{\textbackslash setdatetoday}:] Sets the counters \texttt{dateyear}, \texttt{datemonth}, and \texttt{dateday} to the current date.
\item[\texttt{\textbackslash datemonthname}:] typesets the month (see section \ref{monthname}).
\item[\texttt{\textbackslash datedayname}:] typesets the weekday (see section \ref{dayname}).
\item[\texttt{\textbackslash datedate}:] typesets the date, corresponding to the counters \texttt{dateyear}, \texttt{datemonth}, \texttt{dateday}.
\end{description}


\subsection{Macros which operate with your own counters}
Only the counters you specified are updated by these macros. \verb+\datedayname+ and \verb+\datemonthname+ are not updated.
\begin{description}\sloppypar
\item[\texttt{\textbackslash setmydatenumber\{numbercount\}\{year\}\{month\}\{day\}}:] Sets the counter \texttt{numbercount} to a value, which corresponds to the date.
\item[\texttt{\textbackslash setmydatebynumber\{number\}\{yearcount\}\{monthcount\}\{daycount\}}:] Sets the counters \texttt{yearcount}, \texttt{monthcount}, and \texttt{daycount} to values, which corresponds to the number.
\item[\texttt{\textbackslash mynextdate\{yearcount\}\{monthcount\}\{daycount\}}:] Sets the counters \texttt{yearcount}, \texttt{monthcount}, and \texttt{daycount} to the next date.
\item[\texttt{\textbackslash mynextdate\{yearcount\}\{monthcount\}\{daycount\}}:]Sets the counters \texttt{yearcount}, \texttt{monthcount}, and \texttt{daycount} to the previous date.
\end{description}



\subsection{Month\label{monthname}}
The command \verb+\datemonthname+ typesets the month. It is updated by macros described in section \ref{macro}. You can do this by your own saying \verb+\setmonthname{number}+.

\subsection{Weekday\label{dayname}}
To typeset the weekday say \verb+\datedayname+. This command is updated by macros described in section \ref{macro}.
You can do this by your own saying \verb+\setmonthname{number}+ (1 for Monday and 7 for Sunday). You can also write \verb+\setdaynamebynumber{number}+, were \verb+number+ is the number of a date. If \texttt{startyear} is set to 1800, 1900 or 2000 the calculation of the weekday will work.

\section{Language}

The language options \texttt{english}, \texttt{USenglish} (standard), \texttt{french}, \texttt{spanish}, \texttt{german}, and \texttt{ngerman} are supported. Say \verb+\dateselectlanguage{language}+ to select a language. For other languages: Create a file \texttt{datenumbermylanguage.ldf}. Copy the text from \texttt{datenumberdummy.ldf}. Replace every ``dummy'' with ``mylanguage'' and change the months and weekdays. Say \verb+\usepackage{datenumber}+ \verb+\input{datenumbermylanguage.ldf}+ in your document.

\section{Examples}

\begin{teX}
\setdate{2002}{1}{1}
\thedatenumber
\end{teX}

\setdate{2000}{1}{1}



\begin{verbatim}
\setdatetoday
\addtocounter{datenumber}{10}%
\setdatebynumber{\thedatenumber}%
In 10 days is \datedate
\end{verbatim}

\setdatetoday
\addtocounter{datenumber}{10}%
\setdatebynumber{\thedatenumber}%

Result: In 10 days is \datedate


We can now find the days to Christmas

\begin{teX}
\newcounter{dateone}\newcounter{datetwo}%

\newcommand{\daydifftoday}[3]{%
  \setmydatenumber{dateone}{\the\year}{\the\month}{\the\day}%
  \setmydatenumber{datetwo}{#1}{#2}{#3}%
  \addtocounter{datetwo}{-\thedateone}%
  \thedatetwo
}
\end{teX}
\newcounter{dateone}%
\newcounter{datetwo}%
\newcommand{\daydifftoday}[3]{%
  \setmydatenumber{dateone}{\the\year}{\the\month}{\the\day}%
  \setmydatenumber{datetwo}{#1}{#2}{#3}%
  \addtocounter{datetwo}{-\thedateone}%
  \thedatetwo}

There is still \daydifftoday{\the\year}{12}{25} days to Christmas.


Result: There is still \daydifftoday{\the\year}{12}{25} days to Christmas.


\newcommand{\sd}{%
\ifcase\thedatedayname \or
    Mon\or Tue\or Wed\or Thu\or
    Fri\or Sat\or Sun\fi
}%

\newcommand{\pnext}{%
\thedateyear/%
\ifnum\value{datemonth}<10 0\fi
\thedatemonth/%
\ifnum\value{dateday}<10 0\fi
\thedateday%
\nextdate
}



\begin{verbatim}
\setdate{2001}{9}{29}%
\[\begin{tabular}{lll}
\sd & \pnext & Abc\\
\sd & \pnext & Def\\
\sd & \pnext & Ghi\\
\sd & \pnext & Jkl\\
\end{tabular}\]
\end{verbatim}


Result: \setdate{2001}{9}{29}%

\[\begin{tabular}{lll}
\sd & \pnext & Abc\\
\sd & \pnext & Def\\
\sd & \pnext & Ghi\\
\sd & \pnext & Jkl\\
\end{tabular}\]


\newthought{Get your age calculated}

\begin{teXXX}
\documentclass{article}
\usepackage{datenumber,fp}
\begin{document}
\newcounter{dateone}%
\newcounter{datetwo}%
\setmydatenumber{dateone}{1989}{08}{01}%
\setmydatenumber{datetwo}{\the\year}{\the\month}{\the\day}%
\FPsub\result{\thedatetwo}{\thedateone}
\FPdiv\myage{\result}{365.25} 
\FPround\myage{\myage}{0}\myage\ years old
\end{document}
\end{teXXX}


\subsection{Other}

Because of the Protestant Reformation, however, many Western European countries did not initially follow the Gregorian reform, and maintained their old-style systems. Eventually other countries followed the reform for the sake of consistency, but by the time the last adherents of the Julian calendar in Eastern Europe (Russia and Greece) changed to the Gregorian system in the 20th century, they had to drop 13 days from their calendars, due to the additional accumulated difference between the two calendars since 1582.

The leapyear \index{dates>leapyear} can be tested using
\cmd{\leapyear} and the date can be checked for validity using
\cmd{\ifvaliddate}. The examples below show such tests

\begin{itemize}
\item leap year test
\begin{quote}
\begin{verbatim}
The year 2012 is
\ifleapyear{2012} a \else no \fi leap year.
\end{verbatim}
Result: The year |2012| is \ifleapyear{2012} a \else no \fi leap year.
\end{quote}
\item date test
\begin{quote}
\begin{verbatim}
The 29.2.1900 is
\ifvaliddate{1900}{2}{29} a \else no \fi valid date.
\end{verbatim}


Result: The 29.2.1900 is \ifvaliddate{1900}{2}{29} a \else no \fi valid date.%
\end{quote}
\end{itemize}

\section*{Calculating the week number}
\begin{figure}%
  \centering
  \includegraphics[width=1.1\linewidth]{./graphics/babylonianmaps.jpg}
  \caption[Babylonian Imago Mundi]{\protect\footnotesize \protect\raggedright The Babylonian Imago Mundi, dated to the 6th century BC (Neo-Babylonian Empire). The map shows Babylon on the Euphrates, surrounded by a circular landmass showing Assyria, Armenia and several cities, in turn surrounded by a `bitter river' (Oceanus), with seven islands arranged around it so as to form a seven-pointed star.}
  \label{fig:eleven days}
\end{figure}

I an attempt to produce gantt charts (see Section \ref{ganttcharts}) that follow Tufte's ideas of simplicity, I came across the need to define a week number. The ISO week date system is a leap week calendar system that is part of the ISO 8601 date and time standard. The system is used (mainly) in government and business for fiscal years, as well as in timekeeping.

The system uses the same cycle of 7 weekdays as the Gregorian calendar. Weeks start with Monday. ISO week-numbering years have a year numbering which is approximately the same as the Gregorian years, but not exactly (see below). An ISO week-numbering year has 52 or 53 full weeks (364 or 371 days). The extra week is here called a leap week (ISO 8601 does not use the term).



A date is specified by the ISO week-numbering year in the format YYYY, a week number in the format ww prefixed by the letter W, and the weekday number, a digit d from 1 through 7, beginning with Monday and ending with Sunday. For example, |2006-W52-7| (or in compact form |2006W527|) is the Sunday of the 52nd week of 2006. In the Gregorian system this day is called 31 December 2006.

The system has a 400-year cycle of 146 097 days (20 871 weeks), with an average year length of exactly 365.2425 days, just like the Gregorian calendar. In every 400 years there are 71 years with 53 weeks.

\textsc{The first week of a year is the week that contains the first Thursday of the year.}

Based on this a calculation can be made using routines available from the above packages. However, how many weeks are included in a typical month it is still a problem.


\section{Summary}

This rather long chapter discussed the various options and packages available to deal with dates. The best way so far, for pdfLaTeX and pdfTeX users is to use the \cmd{pdfcreation} to access system time. Once the information made available by this command is parsed the rest of the routines can be developed. And now we have dates. Next we are going to try and develop some scheduling routines for gantt charts.

\section{phd package Internationalization of dates and time}

The phd package currently offers a range of date modules for the internationalization of dates and other strings. See the Chapter on internationalization.

























































  \chapter{Key Value Interfaces}

The key value system greatly simplifies the \tex interface for authors. As \cite{joseph2009} wrote this ease of use was not transferred into settting up key-value systems for authors of pre-packaged \tex code. This Chapter and the one that follows that focus specifically on the \pkg{pgfkeys} package provides an overview and describes some of the more difficult areas. The TUGboat article referenced earlier and written by Joseph Wright \textit{et.al} has an excellent introduction to the available packages and some longer examples for comparison. Chapter~\ref{ch:l3keys}
\nameref{ch:l3keys} discusses the |expl3| key-value functions.

\section{keyval}

The \pkgname{keyval} written by David Carlisle is still widely used by package authors to provide the means for users to easily specify numerous optional arguments for macros \cite{keyval}. The main advantages of using keyval are that  (1) the number of optional arguments is no longer limited to 9 and that (2) the arguments are named, and hence there is less chance of confusion about the syntax of a macro.

\section{xkeyval}

A more recent package, \pkgname{xkeyval} provides improvements for programming keys and  also
provides a more advanced interface for the namespacing of keys and families.
Before you start experimenting with the xkeyval package, I suggest that you load the package \pkg{xkview}. This is part of the \ctan{xkeyval}  bundle and can help you to view key value parameters in various ways. The \pkg{xkeyval} package was developed by Hendri Adriaens and Uwe Kern \citep{xkeyval}. This package is an extension of the well-known |keyval| package. The package provides more flexible commands and syntax enhancements as well as newer option processing mechanism.

The main change of the |xkeyval| package is that it provides a means to namespace the keys, which all have the form |\KV@family@keyname|, where the KV is a literal prefix to avoid collisions. They take one argument to handle user input.

The main commands of the package are the same as those of keyval. 

\begin{texexample}{xkeyval }{}
\makeatletter
\define@key{phd}{pi}{\setlength{\parindent}{#1}}
\setkeys{phd}{pi=50pt}
\makeatother
\lorem\par
\setkeys{phd}{pi=0pt}
\lorem\par
\end{texexample}

Defining a default key, i.e., a key that can be used as |indent| or |indent=30pt| will stretch your memory, as it has an optional parameter as its third argument. 

\begin{verbatim}
\define@key{family}{key}[none]{The input is: #1}
\end{verbatim}

\begin{texexample}{xkeyval }{}
\makeatletter

\define@key{phd}{pi}[30pt]{\setlength{\parindent}{#1}}

\setkeys{phd}{pi}

\lorem

or \setkeys{phd}{pi=0pt}

\lorem
\makeatother
\end{texexample}

\section{Ordinary Keys}
\makeatletter
\define@key{phd}{pi}[1em]{\setlength{\parindent}{#1}}
\makeatother


Ordinary keys are keys that have values such as \texttt{animal=elephant} and your macro can be called like \texttt{animals[animal=elephant]\{14\}}.

   

\section{Keys and values in package options}

First of all, the package supplies macros to declare class or package options, execute them and process
them. The macros are available under the usual
\latex names, but all with the suffix \textbf{X}, namely

\begin{docCommand}{DeclareOptionX}{}
\begin{docCommand}{DeclareOptionX*}{}
\begin{docCommand}{ExecuteOptionsX}{}
\begin{docCommand}{ProcessOptionsX}{}
These commands allow the user to assign a value to
an option just like when using |\setkeys|. The first
macro is based on |\define@key| and the final two
are based on |\setkeys|. Supposing that a package
|mypack| is set up with these commands, a user could
for instance do
\end{docCommand}
\end{docCommand}
\end{docCommand}
\end{docCommand}

\begin{verbatim}
\usepackage[textcolor=red,font=times]{mypack}
\end{verbatim}

These |xkeyval|macros are fully compatible with the \latex option conventions. They will allow packages to copy global options specified in the |\documentclass| command, to pass options to other classes or packages and to update the list of unused global options that will be displayed by \latex in the log file. 


\section{kvoptions}

Another package \pkgname{kvoptions} by Heiko Oberdiek is used extensively in the large suite of packages
developed by Heiko \cite{kvoptions}. The package originally formed part of the \pkgname{hyperref} and later branched into
an independent package. The package provides a number of additional commands to those found in the \latexe kernel and a comparison of the commands is shown in the table below. It is a good alternative to single purpose
package writers. An important feature of the package is its ability to process options both globally as well as locally avoiding conflicts when options are specified both globally as well as locally. Heiko provides an example from his bookmark package \cite{bookmark}, which provides the option \option{open}
that specifies whether the bookmarks are opened or closed initially. It’s values are
true or false. Since KOMA-Script version 3.00 the KOMA classes also introduces
option open with values right and any and a complete different meaning.
Such conflicts can be resolved by marking all or part of options as local by
|\DeclareLocalOption| or |\DeclareLocalOptions|. Then the packages ignores
global occurences of these options



\input{./sections/pgfmanual-en-pgfkeys}











%
 }
%
\def\fontsandsymbols{%
   \part{FONTS}
   \chapter{Unicode}

Unicode is an encoding of \textit{characters}, and it is the first encoding that took the trouble to define what a
\textit{character} is. The distinction between a character and a \textit{glyph} has come to be of interest to philosophers with the Japanese philosopher Shigeki Moro to say that Unicode’s approach is Aristotelian essentialist. 

In this book we adopt the practical definition given by Spyropoulos in his book Unicode \& Encodings. 

\begin{itemize}

\item A glyph is the image of a symbol used in a writing system (in an alphabet, a syllabary, a set of ideographs, etc.) or in a notational system (like music, mathematics, cartography etc.)

\item A \textit{character} is the simple description, primarily linguistic or logical, of an equivalence class of glyphs.
\end{itemize}

\section{Unicode's Principles.}

Unicode subscribes to ten principles.

\medskip
\begin{tabular}{ll}
Universal repertoire &Logical order\\
Efficiency &Unification\\
Characters, not glyphs &Dynamic composition\\
Semantics &Stability\\
Plain Text &Convertibility\\
\end{tabular}
\medskip


The character sets of many existing international, national and corporate standards are incorporated within the Unicode Standard. For example, its first 256 characters are taken from the widely used Latin-1 character set.

Duplicate encoding of characters is avoided by unifying characters within scripts across languages; characters that are equivalent in form are given a single code. Chinese/Japanese/Korean (CJK) consolidation is achieved by assigning a single code for each ideograph that is common to more than one of these languages. This is instead of providing a separate code for the ideograph each time it appears in a different language. (These three languages share many thousands of identical characters because their ideograph sets evolved from the same source.)

The Unicode Standard specifies an algorithm for the presentation of text with bidirectional behavior, for example, Arabic and English. Characters are stored in logical order. The Unicode Standard includes characters to specify changes in direction when scripts of different directionality are mixed. For all scripts Unicode text is in logical order within the memory representation, corresponding to the order in which text is typed on the keyboard.


\section{Unicode Character Database}

Unicode provides all the raw data in its database in the form of specially formatted text files.\footnote{\url{http://www.unicode.org/reports/tr44/tr44-20.html\#Unicode_10.0.0}}

\newfontfamily\panuni{aegean}

\section{Character Data}

Each character is defined by a unique codepoint. Unicode does not care about how it looks, but how it is described, so each character is defined by a unique codepoint.  In addition every character has a number of properties associated with it that defines how a character is to be used by various processes. Below we illustrate some of these properties by parsing a single unicode point, using a custom Lua script.

\bgroup
\parindent=0pt
\begin{multicols}{2}
\small
\panuni
\luadirect{
   dofile("./i18n/parseunicode.lua")
}
\end{multicols}
\egroup


Among the properties that each character has are:

\begin{enumerate}
\item The character’s code-point value and name.

\item The character’s general category. All of the characters in Unicode are grouped into 30 categories,
17 of which are considered normative. The category tells you things like whether the character is
a letter, numeral, symbol, whitespace character, control code, etc.

\item The character’s decomposition, along with whether it’s a canonical or compatibility
decomposition, and for compatibility composites, a tag that attempts to indicate what data is lost
when you convert to the decomposed form.

\item The character’s case mapping. If the character is a cased letter, the database includes the mapping
from the character to its counterpart in the opposite case.

\item For characters that are considered numerals, the database includes the character’s numeric value.
(That is, the numeric value the character represents, not the character’s code point value.)

\item The character’s directionality. (e.g., whether it’s left-to-right, right-to-left, or takes on the
directionality of the surrounding text). The Unicode Bidirectional Layout Algorithm uses this
property to determine how to arrange characters of different directionalities on a single line of
text.

\item The character’s mirroring property. This says whether the character take on a mirror-image glyph
shape when surrounded by right-to-left text.

\item The character’s combining class. This is used to derive the canonical representation of a character
with more than one combining mark attached to it (it’s used to derive the canonical ordering of
combining characters that don’t interact with each other).

\item The character’s line-break properties. This is used by text rendering processes to help figure out
where line divisions should go.

\item  Many more… 
\end{enumerate}


Most of the information for each character is obtained from UnicodeData.txt. This file contains most of the  
Unicode Character Database. As the database has grown, and as supplementary information has been
added to the database, various pieces of it have been split out into separate files, but the most
important parts of the standard are still in UnicodeData.txt. 

\section{Code Point Ranges}

A range of code points is specified by the form |"X..Y"|.
Each code point in a range has the associated property value specified on a data file. For example (from \docFile{Blocks.txt}),


\begin{dispListing}
0000..007F; Basic Latin
0080..00FF; Latin-1 Supplement
\end{dispListing}

Block ranges are different from Scripts which are defined in the \docFile{Scripts.txt} file. A block range is defined by a range of hexadecimal codepoints. 

All block ranges start with a value where |(cp MOD 16) = 0|,
  and end with a value where |(cp MOD 16) = 15|. In other words,
  the last hexadecimal digit of the start of range is |...0|
  and the last hexadecimal digit of the end of range is |...F.|
  This constraint on block ranges guarantees that allocations
  are done in terms of whole columns, and that code chart display
  never involves splitting columns in the charts.

  All code points not explicitly listed for Block
  have the value |No_Block|.

The advantage for providing the database in specially formatted text files, is that they can be parsed into any computer language easily. In our case I have parsed the files both using Lua, as well as modified variants using \latexe. 
The only frustration is to make sure the files are somewhere where |texlua| can find them. 

The list of all the blocks can be obtained and typeset using the |phdlua| modules and is shown next.

{\parindent0pt
\leavevmode\luadirect{dofile("./i18n/parseunicodeblocks.lua")}
}

We can use the same module to determine the current total unicode points and blocks defined by UNICODE.




   \makeatletter\@specialtrue\makeatother

\newcommand{\mf}{{\fontencoding{U}\fontfamily{zmf}\selectfont METAFONT}}

\newcommand{\pcstrut}{\vrule height11pt width0pt}

\newcommand{\sample}{Typographia Ars Artium Omnium Conservatrix}

\newcommand{\thefont}[4][OT1]{%
	\textcolor{thefontname}{#2}&%
	\pcstrut\fontencoding{#1}\fontfamily{#3}\selectfont#4\\}

\newcommand{\fonttitle}[1]{%
	\multicolumn2{p{\columnwidth}}{\vrule height1.5pc width0pt
	\fontseries{b}\selectfont\textcolor{Subheadings}{#1}}\\[3pt]}


\cxset{steward,
  numbering=arabic,
  custom=stewart,
  offsety=0cm,
  image={hine03.jpg},
  texti={An introduction to the use of font related commands. The chapter also gives a historical background to font selection using \tex and \latex. },
  textii={In this chapter we discuss keys that are available through the \texttt{phd} package and give a background as to how fonts are used
in \latex.
 },
 subsubsection indent=0pt,
 section font-family=tiresias,
 subsection font-family=tiresias,
 subsubsection font-family=tiresias,
 }


\chapter{Setting up Fonts}
\label{ch:fonts}
\section{Introduction}
\pagestyle{headings}
\index{fonts>serif}\index{fonts>non-serif}
Selecting the right fonts for a book is a job best left to the book designer. Despise this good advice most \latex authors get their hands dirty trying to play the role of the book designer. A word of advice is that most of them make a royal mess of it. Irrespective of the \tex engine employed, being \tex, \latexe, \lualatex or \xelatex there are two issues in using fonts. How to select them and specify them and what fonts to use. We will dwell on the technical aspects of font selection in this Chapter.

There is another more serious aspect in selecting fonts based on ``physiological’’ considerations. Boris Veytsman in an article in TUGboat \citep{boris2012} reviewed the literature comparing fonts for readbility as well as the ``trustability'' of the text based on different fonts. Experiments carried out by Morris \citep{morris2012a} concluded that fonts affect the reader's attitude towards the text. Baskerville scored the highest and Comic Sans the lowest.
Interestingly Computer Modern, the default typeface of \tex, scored high in the test.  Other tests carried out by \cite{boris2012a} also concluded that there are no noticable differences between serif and non-serif fonts in reading comprehension for cyrillic adult readers and that comprehension and reading speed might be affected by factors other than the font serifs alone. 

Of course the biggest effect on readers is when fonts used for ``branding''. 
Marketers have been brainwashing
consumers for years through the use of fonts. In \textit{Branding
With Type}  by Rogener, Pool, and Packhauser
(1995), a fervent argument is made for unique
but consistent typefaces as a crucial element of
corporate branding. Rogener et al. describe the
fonts used by IBM, Mercedes, Nivea, and
Marlboro as instantly recognisable
internationally, and imply that the significant
investment by such companies in design and
copyright of trademarked fonts is worthwhile. 

For example, \citet*{rogener1995}. discuss the Nivea
Bold typeface developed in 1992 by Gunther
Heinrich at advertising agency \textsc{TBWA} in
Hamburg, Germany, for skincare brand Nivea,
and claim that the Nivea Bold typeface has
effectively embodied the Nivea brand’s `pure
and simple’ product philosophy. They link the
font directly to profitability and Nivea’s
worldwide product category market share of
35\% \cite[p. 91]{rogener1995} (Rogener, Pool \& Packhauser, 1995, p.
91).


{\small
\tiresias
\lorem

}

\section{The Choice of Typesetting Engine}

If you use only |pdfLaTeX| the range of fonts is rather limiting and I would highly recommend for any serious typesetting work to move onto |XeLaTeX| and the use of the package \pkg{fontspec} \citep{fontspec}. Another alternative is to use \lualatex. The latter is becoming more stable and is production ready to a large extend. It is expected to be the successor to pdfTeX.

One of the things I wanted to achieve with the \pkgname{phd} package was  to take care of different \tex engines, and to ensure that the package can be used irrespective of the \TeX\ engine used. 

Before we start outlining the scheme let us start, by demonstrating how to load one of the standard fonts provided by \latexe. We will load the Computer Modern font.\index{Computer Modern (font)} 

\begin{texexample}{How to load a font}{ex:fonts}
\newcommand{\fontdemo}[4][OT1]{
    \leavevmode
    \textcolor{thefontname}{#2}
    \fontencoding{#1}\fontfamily{#3}\selectfont#4 }

\fontdemo{CM}{cmtt}{ \alphabet\par}

\fox
\end{texexample}

In the example we have used a number of convenience commands that are provided by the |phd| package.

\CMDI{\alphabet} Typesets the letters of the English alphabet

\CMDI{\fox} Typesets the fox passage

The example  creates a convenience command to call the |computer modern typewriter| font and to print the alphabet.\footnote{The command \cs{alphabet} is provided by the \texttt{phd} package.} In this case we are asking \latex to load a font from the |cmtt| family. 

To load a font two things are required the encoding scheme [|OT1|] in the example and the somewhat cryptic font family name [|cmtt|].

\section{What is a character? And a glyph?}
\index{glyph}\index{character}
A character is an abstract
concept: the letter “A” is a character, while any
particular drawing of that character is a glyph. In many
cases there is one glyph for each character and one character
for each glyph, but not always.

The glyph used for the Latin letter “A” may also be
used for the Greek letter “Alpha”, while in Arabic writing
most Arabic letters have at least four different glyphs
(often vastly more) depending on what other letters are
around them.

\section{What's a font?}

As the \pkgname{fontinst} manual says: ``Once upon a time, this question was easily answered: a font is a set of type
in one size, style, etc. There used to be no ambiguity, because a font was a
collection of chunks of type metal kept in a drawer, one drawer for each font'' \citet{fontinst}.


With digital typesetting, things are more complicated. What a font
\textit{is} isn't easy to pin down. A typical use of a PostScript font with \latex might
use these elements:

\begin{enumerate}
\item Type 1 printer font file
\item Bitmap screen font file
\item Adobe font metric file (afm file)
\item \tex font metric file (tfm file)
\item Virtual font file (vf file)
\item font definition file (fd file)
\end{enumerate}

Looked at from a particular point of view, each of these files \textit{is} the font. So
what’s going on? Every text font in \latex has five attributes:

\index{encoding schemes>OML}\index{encoding schemes>OMS}\index{encoding schemes>OMX}\index{encoding schemes>U}\index{encoding schemes>OML}
\index{encoding schemes}\index{encoding schemes>OT1}
\begin{description}
\item[Encoding Schemes]
The \textit{encoding} scheme (in the example |OT1|) provides information as to which glyph goes into what slot in a font table. These font tables can be printed using |fonttest.tex|. We show the test for |cmtt10| in Figure~\ref{fig:fonttest}. The
most common values for the font encoding are:
\medskip

\begin{longtable}{ll}
OT1   & TEX text\\
T1     & TEX extended text\\
OML  & TEX math italic\\
OMS  & TEX math symbols\\
OMX  & TEX math large symbols\\
U       & Unknown\\ 
L\meta{xx}  A local encoding\\
\end{longtable}
\medskip

\item[family]\index{fonts>family}\index{fonts>cmr}\index{fonts>cmss}
\index{fonts>cmtt}
The name for a collection of fonts, usually grouped under a common
name by the font foundry. For example, `Adobe Times', `ITC Garamond',
and Knuth's `Computer Modern Roman' are all font families.

There are far too many font families to list them all, but some common ones
are:

\begin{longtable}{rl}
cmr  &Computer Modern Roman\\
cmss &Computer Modern Sans\\
cmtt &Computer Modern Typewriter\\
cmm  &Computer Modern Math Italic\\
cmsy &Computer Modern Math Symbols\\
cmex &Computer Modern Math Extensions\\
ptm  &Adobe Times\\
phv  &Adobe Helvetica\\
pcr  &Adobe Courier\\
\end{longtable}

\item[series] How heavy or expanded a font is. For example, `medium weight', `narrow'
and `bold extended' are all series.

\item[shape] The form of the letters within a font family. For example, `italic',
`oblique' and `upright' (sometimes called `roman') are all font shapes. The most common values for the font shape are:

\begin{longtable}{ll}
n  &Normal (that is `upright' or `roman')\\
it &Italic\\
sl &Slanted (or `oblique')\\
sc &Caps and small caps\\
\end{longtable}

\item[size] The design size of the font, for example `10pt'. If no dimension is specified, `pt' is assumed.
\end{description}

These five parameters specify every \latex
font, for example:

\begin{longtable}{lll}
|LaTeX| specification &Font  &TEX font name\\
|OT1 cmr m n 10|      &Computer Modern Roman 10 point &cmr10\\
|OT1 cmss m sl 1pc|   &Computer Modern Sans Oblique 1 pica &cmssi12\\
|OML cmm m it 10pt|   &Computer Modern Math Italic 10 point &cmmi10\\
|T1 ptm b it 1in|  &Adobe Times Bold Italic 1 inch &ptmb8t at 1in\\
\end{longtable}

When you get a font error or an underfull or overfull box \tex always will print an error with the font specification in full as shown below:

\begin{verbatim}
LaTeX Font Warning: Font shape `EU1/cmr/m/sc' undefined
(Font)              using `EU1/cmr/m/n' instead on input line 160.
\end{verbatim}



\begin{tabbing}
\ttverb\textvisiblespace\quad\=bbbbbbbbbbbbbbbbbbbbbbbbbbbbbbb\=b'b'\=cccccccccccccc\kill
\ttverb\`{}               \>OT1, T1, EU1, EU2\>   \a`{}\> (grave)      \\
\ttverb\'{}               \>OT1, T1, EU1, EU2\>   \a'{}\> (acute)      \\
\ttverb\^{}               \>OT1, T1, EU1, EU2\>   \^{}\>  (circumflex) \\
\ttverb\~{}               \>OT1, T1, EU1, EU2\>   \~{}\>  (tilde)      \\
\ttverb\"{}               \>OT1, T1, EU1, EU2\>   \"{}\>  (umlaut)     \\
\ttverb\H{}               \>OT1, T1, EU1, EU2\>   \H{}\>  (Hungarian umlaut) \\
\ttverb\r{}               \>OT1, T1, EU1, EU2\>   \r{}\>  (ring)       \\
\ttverb\v{}               \>OT1, T1, EU1, EU2\>   \v{}\>  (ha\v{c}ek)  \\
\ttverb\u{}               \>OT1, T1, EU1, EU2\>   \u{}\>  (breve)      \\
\ttverb\t{}               \>OT1, T1, EU1, EU2\>   \t{}\>  (tie)        \\
\ttverb\={}               \>OT1, T1, EU1, EU2\>   \a={}\> (macron)     \\
\ttverb\.{}               \>OT1, T1, EU1, EU2\>   \.{}\>  (dot)        \\
\ttverb\b{}               \>OT1, T1, EU1, EU2\>   \b{}\>  (underbar)   \\
\ttverb\c{}               \>OT1, T1, EU1, EU2\>   \c{}\>  (cedilla)    \\
\ttverb\d{}               \>OT1, T1, EU1, EU2\>   \d{}\>  (dot under)  \\
\ttverb\k{}               \>T1    \>   \k{}\>  (ogonek)     \\
\ttverb\AE                \>OT1, T1, EU1, EU2\>   \AE \>               \\
\ttverb\DH                \>T1    \>   \DH \>               \\
\ttverb\DJ                \>T1    \>   \DJ \>               \\
\ttverb\L                 \>OT1, T1, EU1, EU2\>   \L  \>               \\
\ttverb\NG                \>T1    \>   \NG \>               \\
\ttverb\OE                \>OT1, T1, EU1, EU2\>   \OE \>               \\
\ttverb\O                 \>OT1, T1, EU1, EU2\>   \O  \>               \\
\ttverb\SS                \>OT1, T1, EU1, EU2\>   \SS \>               \\
\ttverb\TH                \>T1    \>   \TH \>               \\
\ttverb\ae                \>OT1, T1, EU1, EU2\>   \ae \>               \\
\ttverb\dh                \>T1    \>   \dh \>               \\
\ttverb\dj                \>T1    \>   \dj \>               \\
\ttverb\guillemotleft     \>T1    \>   \guillemotleft  \> (guillemet) \\
\ttverb\guillemotright    \>T1    \>   \guillemotright \> (guillemet) \\
\ttverb\guilsinglleft     \>T1    \>   \guilsinglleft  \> (guillemet) \\
\ttverb\guilsinglright    \>T1    \>   \guilsinglright \> (guillemet) \\
\ttverb\i                 \>OT1, T1, EU1, EU2\>   \i  \>               \\
\ttverb\j                 \>OT1, T1, EU1, EU2\>   \j  \>               \\
\ttverb\l                 \>OT1, T1, EU1, EU2\>   \l  \>               \\
\ttverb\ng                \>T1    \>   \ng \>               \\
\ttverb\oe                \>OT1, T1, EU1, EU2\>   \oe \>               \\
\ttverb\o                 \>OT1, T1, EU1, EU2\>   \o  \>               \\
\ttverb\quotedblbase      \>T1    \>   \quotedblbase   \>   \\
\ttverb\quotesinglbase    \>T1    \>   \quotesinglbase \>   \\
\ttverb\ss                \>OT1, T1, EU1, EU2\>   \ss \>               \\
\ttverb\textasciicircum   \>OT1, T1, EU1, EU2\>   \textasciicircum \>  \\
\ttverb\textasciitilde    \>OT1, T1, EU1, EU2\>   \textasciitilde  \>  \\
\ttverb\textbackslash     \>OT1, T1, EU1, EU2\>   \textbackslash   \>  \\
\ttverb\textbar           \>OT1, T1, EU1, EU2\>   \textbar         \>  \\
\ttverb\textbraceleft     \>OT1, T1, EU1, EU2\>   \textbraceleft   \>  \\
\ttverb\textbraceright    \>OT1, T1, EU1, EU2\>   \textbraceright  \>  \\
\ttverb\textcompwordmark  \>OT1, T1, EU1, EU2\>   \textcompwordmark\> (invisible) \\
\ttverb\textdollar        \>OT1, T1, EU1, EU2\>   \textdollar      \>  \\
\ttverb\textemdash        \>OT1, T1, EU1, EU2\>   \textemdash      \>  \\
\ttverb\textendash        \>OT1, T1, EU1, EU2\>   \textendash      \>  \\
\ttverb\textexclamdown    \>OT1, T1, EU1, EU2\>   \textexclamdown  \>  \\
\ttverb\textgreater       \>OT1, T1, EU1, EU2\>   \textgreater     \>  \\
\ttverb\textless          \>OT1, T1, EU1, EU2\>   \textless        \>  \\
\ttverb\textquestiondown  \>OT1, T1, EU1, EU2\>   \textquestiondown\>  \\
\ttverb\textquotedbl      \>T1    \>   \textquotedbl    \>  \\
\ttverb\textquotedblleft  \>OT1, T1, EU1, EU2\>   \textquotedblleft\>  \\
\ttverb\textquotedblright \>OT1, T1, EU1, EU2\>   \textquotedblright\> \\
\ttverb\textquoteleft     \>OT1, T1, EU1, EU2\>   \textquoteleft   \>  \\
\ttverb\textquoteright    \>OT1, T1, EU1, EU2\>   \textquoteright  \>  \\
\ttverb\textregistered    \>OT1, T1, EU1, EU2\>   \textregistered  \>  \\
\ttverb\textsection       \>OT1, T1, EU1, EU2\>   \textsection     \>  \\
\ttverb\textsterling      \>OT1, T1, EU1, EU2\>   \textsterling    \>  \\
\ttverb\texttrademark     \>OT1, T1, EU1, EU2\>   \texttrademark   \>  \\
\ttverb\textunderscore    \>OT1, T1, EU1, EU2\>   \textunderscore  \>  \\
\ttverb\textvisiblespace  \>OT1, T1, EU1, EU2\>   \textvisiblespace\>  \\
\ttverb\th                \>T1    \>   \th              \>
\end{tabbing}                        

Do note that when you use the \pkgname{hyperref}, you will get a surprise, all the commands have been converted to "PU" encoding. This is mostly harmless and is  done in order for |hyperref| to mark bookmarks\footnote{http://tex.stackexchange.com/questions/198810/why-does-the-hyperref-package-changes-encoding-of-font-commands} in a safe way.

\begin{texexample}{font encoding}{ex:encoding}
\meaning\textasciitilde\\
\meaning\"\\
\meaning\NG\\
\meaning\k\\
\meaning\alpha
\meaning\printfontparams

\printfontparams
\end{texexample}

A peek at the \docfile{puenc.def} reveals the inner workings
of the encoding mechanism.

\begin{verbatim}
\ProvidesFile{puenc.def}
  [2003/01/20 v6.73l
  Hyperref: PDF Unicode definition (HO)]
\DeclareFontEncoding{PU}{}{}
\DeclareTextCommand{\textLF}{PU}{\80\012} % line feed
\DeclareTextCommand{\textCR}{PU}{\80\015} % carriage return
\DeclareTextCommand{\textHT}{PU}{\80\011} % horizontal tab
\DeclareTextCommand{\textBS}{PU}{\80\010} % backspace
\DeclareTextCommand{\textFF}{PU}{\80\014} % formfeed
\DeclareTextAccent{\`}{PU}{\textgrave}
\DeclareTextAccent{\'}{PU}{\textacute}
\DeclareTextAccent{\^}{PU}{\textcircumflex}
\end{verbatim}

\printfontparams 


\latex uses a number of other files to get to the particular file that contains the font metrics file |cmtt10| and to find the appropriate file. For the original Knuth fonts the filenames have been kept the same, essentially as a request from Knuth that one should not change them.

Most of the difficulty in selecting and using fonts is figuring the encoding scheme and the Karl Berry naming scheme. In the Example~\ref{ex:fonts} we select the \cs{fontfamily} |cmtt| which is computer modern type writer and then we invoke the macros for the shape \cs{itshape} and print the |alphabet|. The macro \cmd{\alphabet} is build-in the |phd| package as we use it in a few places.

\begin{figure}[htbp]
\centering

\hspace*{-2cm}\includegraphics[width=\textwidth]{./images/testfont-output.pdf}

\caption{Output from testfont.tex for cmtt10 font}
\label{fig:fonttest}
\end{figure}



\subsection{The Postscript fonts}

With Adobe reader a number of fonts come pre-packaged and these have been incorporated into \latex2e. These fonts can be found in all \tex distributions. The \textit{Times New Roman} is named |ptm|. 

\begin{texexample}{The Postscript fonts}{ex:postscriptfonts}
\raggedright
\begin{tabular}{@{}>{\sffamily\bfseries}rl}
\fonttitle{The Adobe `LaserWriter 35', 10 typefaces in a total of 35
different styles, standard on all PostScript printers}

\thefont{Avant Garde Book}{pag}{\fontsize{9}{9}\selectfont\sample}
\thefont{Bookman Light}{pbk}{\sample}
\thefont{Courier}{pcr}{\sample}
\thefont{Helvetica}{phv}{\sample}
\thefont{New Century Schoolbook}{pnc}{\sample}
\thefont{Palatino}{ppl}{\sample}
\thefont[U]{Symbol}{psy}{\sample}
\thefont{Times New Roman}{ptm}{\sample}
\thefont{Zapf Chancery Medium Italic}{pzc}{\fontsize{12}{12}\selectfont\itshape\sample}
\thefont[U]{Zapf Dingbats}{pzd}{\sample}
\end{tabular}
\end{texexample}

Using the |phd| package we can come closer to the |fontspec| or LuaTeX way of doing things and use longer font names as those found in the operating system.


ctivating the key will set the font to |pzc| and unless is within a group
will typeset the rest of the document with this typeface.

\makeatletter
\def\fontname@cx{}
\cxset{font name/.is choice,
       font name/Zapf Chancery Medium Italic/.code={\fontfamily{pzc}\selectfont},
 font name/courier/.code={\fontfamily{pcr}\selectfont},
font name/Helvetica/.code={\fontfamily{phv}\selectfont},
font name/helvetica/.code={\fontfamily{phv}\selectfont},
font name/Bookman Light/.code={\fontfamily{pbk}\selectfont},
font name/bookman/.code={\fontfamily{pbk}\selectfont},
font name/Utopia/.code={\fontfamily{put}\selectfont},
font name/Palatino/.code={\fontfamily{put}\selectfont},
font name/Old Standard/.store in=\fontname@cx,
font name/Junicode/.code={
\fontspec{Junicode}\addfontfeature{StylisticSet=2}}
}
\makeatother

\begin{key}{/phd/font name=\marg{Zapf Chancery Medium Italic}}
\cxset{font name=Zapf Chancery Medium Italic}
\bgroup \itshape This is how it is typeset\egroup
\end{key}


\begin{key}{/phd/font name=\marg{Old Standard}}
Setting the key to \texttt{Old Standard} will typeset the next sample in \texttt{OldStandard-Regular}, |Stylistic Set=2|. 

\bgroup
\parindent1em\itshape
^^A\cxset{font name=Old Standard}

\aliceii

abcdefg
\egroup
\end{key}

\begin{key}{/phd/font name=\marg{Junicode}}
Setting the key to \texttt{Junicode} will typeset the next sample in \texttt{Junicode}, \texttt{Stylistic Set=2}. 

\bgroup
\parindent1em\itshape
\cxset{font name=Junicode}

\aliceii

abcdefg
\egroup
\end{key}




\begin{key}{/phd/font name=\marg{Bookman Light or bookman}}
Bookman Light or |bookman|
\end{key}

\bgroup
\cxset{font name=bookman}
\aliceiii
\egroup


\begin{key}{/phd/font name=\marg{Utopia or utopia}}

\end{key}

\bgroup
\cxset{font name=Utopia}

\renewcommand{\LettrineFontHook}{\fontfamily{put}\fontseries{bx}}%
\par\leavevmode

\lettrine[lines=5, lhang=0.1,lraise=0.28,findent=1pt]{g}{oats} are animals found in all sort of places. The paragraph has been set using the font family |utopia|. The comment about the goats was just to get the letter g.
comfortable in mountain areas. I don't recall Alice  They are more
comfortable in mountain areas. I don't recall Alice  They are more
comfortable in mountain areas. I don't recall Alice  They are more
comfortable in mountain areas. I don't recall Alice  They are more
comfortable in mountain areas. I don't recall Alice  They are more
comfortable in mountain areas. I don't recall Alice  They are more
comfortable in mountain areas. I don't recall Alice 


\renewcommand{\LettrineFontHook}{\fontfamily{phv}\fontseries{bx}}%



\par\leavevmode

\lettrine[lines=5, lhang=0.1,lraise=0.28,findent=1pt]{g}{oats} are animals found in all sort of places. The paragraph has been set using the font family |utopia|. The comment about the goats was just to get the letter g.
comfortable in mountain areas. I don't recall Alice  They are more
comfortable in mountain areas. I don't recall Alice   They are more
comfortable in mountain areas. I don't recall Alice   They are more
comfortable in mountain areas. I don't recall Alice  They are more
comfortable in mountain areas. I don't recall Alice  They are more
comfortable in mountain areas. I don't recall Alice  They are more
comfortable in mountain areas. I don't recall Alice 

\medskip



\lettrine{G}{o}ats are among the earliest animals domesticated by humans. The most recent genetic analysis confirms the archaeological evidence that the wild Bezoar ibex of the Zagros Mountains are the likely origin of almost all domestic goats today. Neolithic farmers began to herd wild goats for easy access to milk and meat, primarily, as well as for their dung, which was used as fuel, and their bones, hair, and sinew for clothing, building, and tools. The earliest remnants of domesticated goats dating 10,000 years before present are found in Ganj Dareh in Iran. Goat remains have been found at archaeological sites in Jericho, Choga Mami Djeitun and Çay\"on\"u, dating the domestication of goats in Western Asia at between 8000 and 9000 years ago.\footnote{Text is from wikipedia's article for the domesticated goat.}

\bgroup

\cxset{font name=bookman}

As you have observed we did not change the normal size of paragraphs, but the examples demonstrate that differences in font families also affect the visual size of the typeset text. |Helvetica| is normally scaled down to 0.95 and |Chancery| is scaled a little bit up or we use a larger font size.
\egroup

\everypar{}%FIXME

\subsection{\textsf{Additional free fonts for use with \LaTeX}}

A number of archaic and other fonts are available in the \latexe historical collection. These are very impressive. They also provide in most instances transliterations.

\begin{tabular}{@{}>{\sffamily\bfseries}rl}
\fonttitle{\textit{The Historical Collection}}
\thefont{Cypriot}{cypr}{\fontsize{7}{7}\selectfont\sample}
\thefont{Linear `B'}{linb}{\fontsize{8}{8}\selectfont\sample}
\thefont{Phoenician}{phnc}{\sample}
\thefont{Runic}{fut}{TYPOGRAPHIA ARS ARTIUM OMNIUM CONSERVATRIX}
%\thefont{Rustic}{rust}{\sample}
\thefont[U]{Bard}{zba}{\sample}
\thefont{Uncial}{uncl}{\sample}[-3pt]
\end{tabular}

\subsection{Uncial fonts}

\newcommand{\ABC}{ABCDEFGHIJKLMNOPQRSTUVWXYZ}
%\newcommand{\abc}{abcdefghijkl mnopqrstuvwxyz}
\newcommand{\punct}{.,;:!?`' \&{} () []}
\newcommand{\figs}{0123456789}
\newcommand{\dashes}{- -- ---}
\newcommand{\sentence}{%
this is an example of the uncial font. now is the time for all good
men, and women, to come to the aid of the party while the quick brown fox
jumps over the lazy dog:}


\newcommand{\Sentence}{%
This is an example of the Uncial font. Now is the time for all good
men, and women, to come to the aid of the party while the quick brown fox
jumps over the lazy dog:}

Peter Wilson's \pkgname{uncial} package provides a useful uncial font and is easily used by just including the file. 

\begin{texexample}{Unical fonts example}{}
\begin{center}
The Uncial Huge normal font. \\ \par
{\unclfamily\Huge \ABC\\ \alphabet\\ \punct{}\dashes\\ \figs\\ \par }
\end{center}
\end{texexample}




The following fonts are all selections from Yiannis Haralambous collection and we categorize them as other scripts collection.

\begin{tabular}{@{}>{\sffamily\bfseries}rl}
\fonttitle{\textit{The Other Scripts Collection}}
\thefont{Calligraphic}{zca}{\fontsize{15}{15}\selectfont\sample}
\thefont[U]{Fraktur}{yfrak}{%
	Alle\char'215\ Verg\"angliche ist nur ein Gleichni
	Da\char'215\ Unzul\"angliche hier wird'\char'215\
	Ereigni\char'215;}

\thefont[U]{Schwabacher}{yswab}{%
	Da\char'215\ Unbeschreibliche hier wird'\char'215\ getan / 
	Da\char'215\ Ewig-Weibliche zieht un\char'215\ hinan!}
\thefont[U]{`Gothic'}{ygoth}{If it plese ony man spirituel or temporel
to bye any pye\char'140\ of two and thre comemoraci\~o\char'140}[6pt]
\thefont[U]{Decorative Initials}{yinit}{\fontsize{8}{8}\selectfont
\raisebox{-12pt}{YIANNIS}}
\end{tabular}

\section{Dingbat and Symbol Fonts}

\index{fonts>Zapf Dingbats}

Fonts containing collections of special symbols, which are normally not found in a text font, are called  \textit{dingbats}. One such font, the PostScript font Zapf Dingbats, is available if you use the |pifont| package, originally written by Sebastian Rahtz, and now part of |PSNFSS|. This is loaded automatically by the |phd| package. (See also implementation code at Page \pageref{dingbats}).

The parameter for the \cs{ding} command is an integer that specifies the character to be typeset according to Table~\ref{tbl:dingbats}. For example |\ding{38}| gives \ding{38}.

For Open Type fonts the |Wingdings| family can be found on Windows systems. The advent of Unicode and the universal character set allowed commonly used dingbats to be given their own character codes. Although fonts claiming Unicode coverage will contain glyphs for dingbats \textit{in addition} to alphabetic characters continue to be popular, primarily for ease of input. Such fonts are sometimes known as \textit{pi fonts}.\index{fonts>pi fonts}

\subsection{Unicode Dingbats block}

The Dingbats block |U+2700-U+27BF| was added to the Unicode Standard in June, 1993, with the release of version 1.1. This code block  contains decorative character variants, and other marks of emphasis and non-textual symbolism. Most of its characters were taken from Zapf Dingbats. 

The Ornamental Dingbats block (|U+1F650–U+1F67F|) was added to the Unicode Standard in June 2014 with the release of version 7.0. This code block contains ornamental leaves, punctuation, and ampersands, quilt squares, and checkerboard patterns. It is a subset of dingbat fonts Webdings, Wingdings, and Wingdings 2. \footnote{See \url{http://std.dkuug.dk/jtc1/sc2/wg2/docs/n4115.pdf}}

A font that we will be using for many of the \XeLaTeX examples is |code2000|
and |code2001|. The fonts were designed by James Kas
\footnote{They can be downloaded at \url{http://www.alanwood.net/downloads/index.html}}. They are True type fonts. The fonts contain a respectable collection of more or less exotic Unicode characters both within the Basic Multilingual Plane (BMP). They were designed by James Kass and were freeware. Sadly the website is no longer available, but the files can be downloaded in the links I have provided. I have also included them in the distribution for the |phd| package, as they are such a useful tool.

\index{Unicode}\index{Basic Multilingual Plane}

\CMDI{\codetwothousand} Loads the TrueType font \texttt{code2000.ttf}\index{fonts>code2000}\index{code2001}

\CMDI{\codetwothousandone} Loads the TrueType font \texttt{code2001}

\CMDI{\symbola} Loads the TrueType font \texttt{symbola}\index{fonts>Symbola}

\index{fonts>Symbola}
\index{fonts>code2000}
\index{fonts>code2001}
\begin{verbatim}
\newfontfamily{\codetwothousand}{code2000.ttf}
  \newfontfamily{\codetwothousandone}{code2001.ttf}
  \newfontfamily{\symbola}{symbola.ttf}
\end{verbatim}




\index{fonts>wingdings}
\begin{texexample}{Wingdings}{ex:wingdings}
\ifxetex
   {\codetwothousand \symbol{9742} \symbol{9743}
    Katakana (片仮名, カタカナ)
   \codetwothousandone \symbol{57508}
   \symbola \symbol{9816}
  }
\else
  \ifluatex
  {\codetwothousand \symbol{9742} \symbol{9743}
    Katakana (片仮名, カタカナ)
   \codetwothousandone \symbol{57508}
   \symbola \symbol{9816}
  }
  \else
   Compile the document with XeTeX to see the example
  \fi 
\fi
\end{texexample}

Another useful font for experimenting if you are using a Windows computer is |Arial Unicode MS|.
\index{Arial Unicode MS (font)}\index{fonts>Arial Unicode MS}

\begin{smallverbatim}
\documentclass{article}
\usepackage{fontspec}
\setmainfont{Arial Unicode MS}
\usepackage{multicol}
\setlength{\columnseprule}{0.4pt}
\usepackage{multido}
\setlength{\parindent}{0pt}
\begin{document}

\begin{multicols}{8}
\multido{\i=0+1}{"10000}{^^A from U+0000 to U+FFFF
  \iffontchar\font\i %
    \makebox[3em][l]{\i}%
    \symbol{\i}\endgraf
  \fi
}
\end{multicols}
\centering
\symbol{57352}
\end{document}
\end{smallverbatim} 


%
%\begin{multicols}{8}
%\ExplSyntaxOn
%\bgroup
%\symbola
%\multido{\next=0+1}{"10000}{
%  \iffontchar\font\next %
%     \makebox[3em][l]{\next}%
%    \symbol{\next}\endgraf
%  \fi
%\egroup  
%\ExplSyntaxOff
%}
%\end{multicols}

The |Symbola Font| has many other symbols, including chess and even Mahjong symbols\index{Mahjong}.\footnote{\url{http://users.teilar.gr/~g1951d/Symbola.pdf}}. \person{George Douros} has packaged many of the fonts for archaic languages, but sadly the substitution mechanisms of \latexe do not always map the fonts properly.

With |LuaLaTeX| and |XeLaTeX|, \tex has moved into the twenty-first century and its usefulness can now be extended to many other languages and fields. 

\section{Naming digital fonts}

Commercial and Open Source fonts come as a set of several files. The |.pfb| file and less frequently, a |.pfa| file or other files depending on the type of font and the operating system and provider. The metric information file resides in an associated |.afm| file. Other files, with extensions |.inf| (information) and |.pfm| are irrelevant to \latex and \tex.

Fonts already have names given them by their designers. The problem lies in associating this name with the font files. Restriction of operating systems originally from PC-DOS dates, restricted to the initial part of file names to eight characters.

\subsection{Karl Berry naming scheme}\index{Karl Berry Scheme}\index{fonts>Karl Berry scheme}

The original inspiration for Fontname was Frank Mittelbach and Rainer Schoepf's article in TUGboat 11(2) (June 1990). This led to an article by Karl Berry in TUGboat 11(4) (November pages 512-519).

Karl Berry then suggested a system---with many limitations, but perhaps the best that could have been done in its time, for mapping a lengthy font name into a file name that was eight or fewer characters long. If the font files are renamed accordingly then we can deduce the nature of the font by examining its file name. The scheme did not apply to the original Computer Modern fonts that retained their original names \citep{fontname}.

This scheme assumes that only eight characters or fewer can be available for naming the font. These eight characters look like,

\begin{verbatim}
FNNW[S][V]7V
\end{verbatim}

Some additional comments on this shorthand notation is in order. 
The most common foundry abbreviations are |p| for Adobe (from PostScript), \textbf{b} for BitStream, and \textbf{m} Monotype. A font flouting this scheme will begin with a z.

The next two letters are reserved for the typeface name. The hundreds and hundred of available faces guarantees  that many of these will be cryptic, even for the most common typefaces---Adobe Garamond is |ad|. 



\section{Using fonts with XeTeX based engines}

Depending on the fonts in your system, some features that are described here, might not be available.

\begin{texexample}{}{}
\ifxetex
  %\usepackage{fontspec}
  %\defaultfontfeatures{Mapping=tex-text}
  %\setmainfont{Times New Roman}
  %\setsansfont{Myriad Pro}
  Running XeTeX
\else
  \ifluatex
  %\usepackage{lmodern}
  %\usepackage[T1]{fontenc}
    Running LuaTeX
  \else
    Running pdfLaTeX
  \fi
\fi
\end{texexample}

For free fonts there exist a few resources that can be used with \LaTeX.
\url{http://tex.stackexchange.com/questions/53416/using-a-good-non-default-font}. Integrating them within a new document can be a nightmare but is the job of the class and book designer.

\section{Terminology}

The best source of information for XeTeX is the \ctan{fontspec} manual. It is not an easy read, but if you are going to be resetting a lot of fonts, it is advisable to do so.

Most typesetting systems allow for setting document wide fonts. In \latexe we get the following commands:


\cs{sffamily}

\cs{rmfamily}

\cs{ttfamily}

To be able to use the |phd| package properly you will have to familiarize yourself with the terminology, if you are not.

\index{CSS}
\texttt{CSS} uses a combination of font-family and fallback generic families to achieve this and it is instructive to review it as we will a similar system here.

\begin{tcolorbox}
\begin{lstlisting}
p{font-family:"Times New Roman", Georgia, Serif;}
\end{lstlisting}
\end{tcolorbox}

The font-family property specifies the font for an element.

The font-family property can hold several font names as a "fallback" system. If the browser does not support the first font, it tries the next font.

There are two types of font family names:

family-name - The name of a font-family, like "times", "courier", "arial", etc.

generic-family - The name of a generic-family, like "serif", "sans-serif", "cursive", "fantasy", "monospace".

There is though a fundamental difference that one needs to keep in mind, \TeX\ exists in order to always typeset the same on any machine. CSS endeavours to run in any browser and any system, disregarding typography. Nevertheless I decided to provide the interface so at least as to enable document compilation at all times, well almost all times.


\section{General font selection with fontspec}

\begin{trivlist}
\item [\cs{fontspec}\oarg{font features}\marg{font name}]
\item [\cs{setmainfont}\oarg{font features}\marg{font name}]
\item [\cs{setsansfont}\oarg{font features}\marg{font name}]
\item [\cs{setmonofont}\oarg{font features}\marg{font name}]
\item [\cs{newfontfamily}\marg{cmd}\oarg{font features}\marg{font name}]
\end{trivlist}

These are the main font-selecting commands of this package. The \cs{fontspec}
command selects a font for one-time use; all others should be used to define the
standard fonts used in a document. They will be described later in this section.
The font features argument accepts comma separated \marg{font feature}=\marg{option}
lists; these are described in later:

\ifxetex
\begin{texexample}{}{}
\bgroup
\fontspec{Verdana}
\raggedright
\knutext

\newfontfamily\calibri{Calibri}
  \calibri 


\def\setchapterfont{\calibri\huge}

\textsf{\large \lorem}
\egroup
\end{texexample}
\fi

\begin{verbatim}
\DeclareTextFontCommand{\textsf}{\calibri}
\end{verbatim}

\subsection{fontspec commands to select font families}

In many cases there is only a need to define a new font for specific case, for example only for a chapter head. It is 

\CMDI{\newfontfamily}\marg{font-switch}\oarg{font features}\marg{font name}

For cases when a specific font with a specific feature set is going to be re-used
many times in a document, it is inefficient to keep calling \cs{fontspec} for every use. For this reason, new commands can be created for loading a particular font family

While the \cs{fontspec} command does not define a new font instance after the first
call, the feature options must still be parsed and processed.
\cs{newfontfamily}. The example that follows, defines a new font family to be used only for chapterheads. This is more efficient and also provides a semantic interface for the author.

\begin{texexample}{newfontfamily}{ex:newfontfamily}
 
\newfontfamily\calibri{Calibri}
\def\setchapterfont{%
   \calibri\huge\bfseries}

\bgroup
\setchapterfont CHAPTER 10
\egroup
\end{texexample}

\begin{teX}
15 \DeclareTextFontCommand{\textrm}{\rmfamily}
16 \DeclareTextFontCommand{\textsf}{\sffamily}
17 \DeclareTextFontCommand{\texttt}{\ttfamily}
18 \DeclareTextFontCommand{\textnormal}{\normalfont}
\end{teX}

\subsection{Setting font features}
\index{fontspec>font features}

The \pkgname{fontspec} package enables the selection of font features during run-time; font features are items such as colors, proportional OldStyle numbers and other similar items. Some of the examples that follow have been extracted from the fontspec documentation.

\ifxetex\else\if\luatex
\begin{texexample}{}{}
\fontspec[Numbers={Proportional,OldStyle}]
{TeX Gyre Adventor}
`In 1842, 999 people sailed 97 miles in
13 boats. In 1923, 111 people sailed 54
miles in 56 boats.' \bigskip

\fontspec{TeX Gyre Adventor}
`In 1842, 999 people sailed 97 miles in
13 boats. In 1923, 111 people sailed 54
miles in 56 boats.' \bigskip
\end{texexample}

Accessing Raw Features explicitly is perhaps better suited to people that like to investigate under the hood and can also provide a clearer way to understand what is going on.
 
\begin{texexample}{}{}
\fontspec[RawFeature=+onum;+zero]{TeX Gyre Adventor}
`In 1842, 999 people sailed 97 miles in
13 boats. In 1923, 1110 people sailed 54
miles in 56 boats.' \bigskip

\fontspec[RawFeature=+tnum;+zero]{TeX Gyre Adventor}
00001761 tabular figures \fox \bigskip

\fontspec[RawFeature=+pnum;+zero]{TeX Gyre Adventor}
00001761 proportional figures \bigskip

\fontspec[RawFeature=+onum;+zero]{TeX Gyre Adventor}
00001761 old numerals\bigskip

\fontspec[RawFeature=+lnum;+zero]{TeX Gyre Adventor}
00001761 lining figures\bigskip
\end{texexample}

Not all \OpenType\footnote{See \protect\url{https://www.microsoft.com/typography/otspec/featurelist.htm}} fonts provide all font features and some will only work in combination with others, for example the |onum| feature will only work with the |pnum| number features. Some experimentation and viewing the font features with a utility is essential.

\fi\fi


\section{The phd package interface.}

By design feature options for XeTeX/XeLaTeX have currently been restricted. The reason behind this decision is that I was concerned that I would have added a complicated interface with very little reason as to its use. I opted for a more semantic approach and expect the user to define custom macros to handle anything else.
\medskip

\keyval{mainfont}{\marg{font1,font2,font3}}{A comma separated list of one or more font-names. The main font will be set to the first font found.}
\keyval{chapterfont}{\marg{font1,font2,font3}}{A comma separated list of one or more font-names. The main font will be set to the first font found.}
\keyval{sectionfont}{\marg{font1,font2,font3}}{A comma separated list of one or more font-names. The main font will be set to the first font found.}
\keyval{contentsfont}{\marg{font1,font2,font3}}{A comma separated list of one or more font-names. The main font will be set to the first font found.}
\keyval{bibliographyfont}{\marg{font1,font2,font3}}{A comma separated list of one or more font-names. The main font will be set to the first font found.}

Note that the package will first check if is running under XeTeX. If it does it will execute the commands and load the macros, otherwise it will fall back on pdfLaTeX commands.

\section{Viewing and selecting fonts}

\subsection*{\textsf{\color{Headings}Typefaces that come with the
standard \LaTeX\ distribution}}
{
\raggedright
\begin{tabular}{@{}>{\sffamily\bfseries}rl}
\fonttitle{Computer Modern (CM), \LaTeX's default typeface}
\thefont{CM Roman}{cmr}{\sample}
\thefont{CM Italic}{cmr}{\itshape\sample}
\thefont{CM Slanted (Oblique)}{cmr}{\slshape\sample}
\thefont{CM Bold}{cmr}{\fontseries{b}\selectfont\sample}
\thefont{CM Bold Extended}{cmr}{\bfseries\sample}
\thefont{CM Bold Italic}{cmr}{\itshape\bfseries\sample}
\thefont{CM Bold Slanted}{cmr}{\slshape\bfseries\sample}
\thefont{CM Caps \& Small Caps}{cmr}{\scshape\sample}
\thefont{CM Sans-Serif}{cmss}{\sample}
\thefont{CM Sans-Serif Oblique}{cmss}{\itshape\sample}
\thefont{CM Sans-Serif Bold}{cmss}{\bfseries\sample}
\thefont{CM Typewriter}{cmtt}{\sample}
\thefont{CM Typewriter Italic}{cmtt}{\itshape\sample}
\thefont{CM Typewriter Bold}{cmtt}{\bfseries\sample}
\thefont{CM Typewriter C\&SC}{cmtt}{\scshape\sample}
\thefont[OMS]{CM Mathematics}{cmsy}{$E=mc^2$\qquad}
\thefont{CM `Dunhill'}{cmdh}{\sample}
\thefont{CM `Fibonacci'}{cmfib}{\sample}
\end{tabular}
}\index{fonts>Fibonacci}\index{fonts>Dunhill}
\section{Discussion}


Unfortunately, even with the best will loading fonts will always be a difficult task in TeX. Hopefully the interface provided will result in better separation of presentation from content and offers consistency in the styling of documents. Nothing prevents you from adding normal macros to styles. Each style can be treated as a package in many respects.



\section{XeLaTeX and LuaLaTeX}



\bgroup
\fontspec{Verdana}
\begin{minipage}[t]{.2\linewidth}
\hbox to \linewidth{\hfill\hfill Verdana\hspace{2em}}
\end{minipage}
\begin{minipage}[t]{.75\linewidth}
^^A\addfontfeature{ItalicFeatures={Alternate = 1}}
\noindent\fox\\
\alphabet\\
\punctuation\\
\frogking
\end{minipage}
\egroup

\bgroup
\fontspec{Calibri}
\begin{minipage}[t]{.2\linewidth}
\hbox to \linewidth{\hfill\hfill Calibri\hspace{2em}}
\end{minipage}
\begin{minipage}[t]{.65\linewidth}
^^A\addfontfeature{ItalicFeatures={Alternate = 1}}
\noindent\fox\\
\alphabet\\
\textsc{\alphabet}\\
\punctuation\\
\frogking
Θαμκυαμ πλαθονεμ ραθιονιβυς ναμ ει, δυις περπετυα σιθ αδ, νες ιδ δισυντ σοντεντιωνες. Κυι σινθ μυνδι εα, φιμ αν γραεσω ιυδισαβιτ, εραθ δολορ φιρθυθε υθ δυο. Συ νοσθερ οπθιων ευμ, μει ερος προβο φιερενθ ευ. Ιυς μανδαμυς τωρκυαθος εξπεθενδις ιδ. Σεδ θε νιβχ νονυμυ δελισαθισιμι, φιμ νο νιβχ λαβωραμυς, σεα εα δισο ποσιμ αντιωπαμ.
\end{minipage}
\egroup

\section{Utilities for testing fonts}

The package \pkg{fonttable} is an extension and re-implementation of Donald Knuth’s testfont.tex, which
is available from CTAN. The package was developed by Peter Wilson and currently maintained by Will Robertson \citep{fonttable}. It provides a number of utility commands for typesetting font tables.


The {fonttable}\marg{font} takes the font file as an  argument and typesets it in a nice table. 

% \ifengine
%\ifxetex
% \else
%  \fonttable{pzdr}
%\fi

A great tool to inspect a True Type font on the command line is \texttt{luaotfload-tool}:

\begin{verbatim}
  luaotfload-tool --find="Iwona" --inspect
\end{verbatim}

\section{ \texttt{.tfm } files}


When you tell \tex that you will be using a particular font, it has to find out information about that font. This information is stored in a file with the extension \docfileextension{.tfm}. For example when you say:

|\font a=cmr10|

\noindent \tex looks for  a file named |cmr10.tfm|. If this is not found then an error is issued |Lookup failed on file CMR10.TFM|

Generall speaking, a font's |.tfm| file contains information about the height, width and depth of all the characters in the font plus kerning and ligature information. So, |cmr10.tfm| might say that the lower-case "d" in CMR10 is 5.5 points wide, 6.94 points high, etc. This is the information that \tex uses to make its lowest-level boxes---those around characters. See the \tex manual for information about what \tex does with these boxes. Note the |.tfm| files do not contain any information that is device dependent. Only device-drivers read \tex's |dvi| output files can use that sort of information.


\section{Fonts for Far East Languages}

In internationalization, CJK is a collective term for the Chinese, Japanese, and Korean languages, all of which use Chinese characters and derivatives (collectively, CJK characters) in their writing systems. Occasionally, Vietnamese is included, making the abbreviation CJKV, since Vietnamese historically used Chinese characters as well.
The characters are known as hànzì in Chinese, kanji in Japanese, hanja in Korean, and Chữ Nôm in Vietnamese.\index{kanji}\index{hanja}\index{hànzì}\index{CJK}\index{CJKV}


\subsection{Selecting a font}

The easiest way is with Will Robertson's \pkgname{fontspec} package. In this sample, we have used the \texttt{SimSun} font, which can be found on windows machines:

\begin{verbatim}
\usepackage{fontspec}
\setromanfont{SimSun}
\end{verbatim}
in the preamble, to use a Far Eastern font as the initial default typeface.

\section{Entering CJK text}

You can just enter Unicode text directly in the document: 你好. Don't use legacy \LaTeX\ packages such as \verb|inputenc| or \verb|CJK|, as \XeTeX\ reads the text as Unicode characters, not the separate byte codes of UTF-8 sequences, and passes them directly to the Unicode font. (Actually, it would probably be possible to use \verb|\XeTeXinputencoding "bytes"| and work with legacy \LaTeX\ input encoding support. But then you're pretty much committed to all the old encoding and font machinery, and there's not much point in using the \XeTeX\ engine at all.)

\begin{comment}
\setromanfont{Times New Roman}

\subsection{A CJK environment}

\newenvironment{CJK}{\fontspec[Scale=0.9]{SimSun}}{}

\newcommand{\cjk}[1]{{\fontspec[Scale=0.9]{SimSun}#1}}

Rather than selecting a CJK font as the main document typeface, you might want to define a CJK environment for text fragments used in the midst of a document using a normal Roman font. This allows me to say \verb|\begin{CJK}東光\end{CJK}| to generate \begin{CJK}東光\end{CJK}, without putting the whole paragraph into the Far Eastern font. Or I could define a command that takes the CJK text as an argument, so that \verb|\cjk{北京}| produces \cjk{北京}. It's that easy! Such an environment can easily be set using the \cmd{newfamily} or \cmd{\fontspec}.

\begin{verbatim}
\newenvironment{CJK}{\fontspec[Scale=0.9]{SimSun}}{}
\newcommand{\cjk}[1]{{\fontspec[Scale=0.9]{SimSun}#1}}
\end{verbatim}
\end{comment}


\normalfont

\section{Changing the font size in LaTeX}

\index{fonts>sizing commands}

Changing the font size in LaTeX can be done at two levels, 
affecting the whole document or elements with in it. 
Using a different font size on a
global level will affect all normal-sized text as well
as the sizes of headings, footnotes, etc. By changing
the font size locally, however, a single word, a few
lines of text, a large table, or a heading throughout
the document may be modified. Fortunately, there is
no need for the writer to juggle with numbers when
doing so. \latex provides a set of macros for changing
the font size locally, taking into consideration the
document’s global font size. \citep{thurnherr2012}

A number of packages exist \ctan{moresize} \citep{moresize},
 \ctan{anyfont} \citep{anyfont}. The standard classes, memoir
 KOMA classes and most journals also come with their own
 defined sizing commands.
 
 

















   %% Check on why fonttable gives problems in Index
\parindent1em
\chapter{Symbols}

\section{Introduction}
\label{ch:comprehensivesymbols}

The \pkgname{phd} package, preloads a number of packages, to provide as
many symbols as possible irrespective of the \tex engine used. Many of these
symbols can easily be replaced by the use of \pkgname{fontspec} and if
the package is set to unicode math with the use of and suitable fonts Open Type Fonts. What follows has been largely copied from \emph{The Comprehensive List of \latexe Symbols}, which has been the authoritative publication, using almost all available symbols. 
The publication lists many symbols which I have dropped due to having exceeded the number of math alphabets allowed by \tex. 


With the newer engines \xetex \luatex you can now use any symbol you can imagine, but there is still room and advantages for using commands. It is at least for me faster than looking up a symbol's unicode character or trying it out with a screen keyboard (unless of course you are using a foreign language keyboard). 

The sections that follow describe the commands and packages that are
available, by simply including the \pkgname{phd} package. Most of the conflicts have been resolved and I am hoping that in the next version we will add some more symbols. 
Currently these are over 1500 as commands and in excess of 60,000 unicode glyphs, provided you have access to the fonts.\footnote{This document has been compiled using \luatex.}
  

\subsection{Reserved Symbols}
\tex has a number of symbols that need to be escaped, as they have 
special meanings during processing see Table~\vref{special-escapable} and also Chapter~\vref{ch:characters}.


\begin{symtable}{\latexe{} Escapable ``Special'' Characters}
\index{special characters=``special'' characters}
\index{escapable characters}
\index{underline}
\label{special-escapable}
\begin{tabular}{*6{ll@{\qqquad}}ll}
\K\$   & \K\%   & \K\_$\,^*$  & \Kp\}  & \K\&   & \K\#   & \Kp\{   \\
\end{tabular}
\end{symtable}

The \latexe kernel command \refCom{@sanitize} changes the catcode of these characters so they can be included in commands such as |\index|. In text just escape them with a (\textbackslash).


\begin{longsymtable}{Predefined \latexe{} Text-mode Commands}
\index{inequalities}
\index{tilde}
\index{underline}
\index{copyright}
\idxboth{dot}{symbols}
\index{dots (ellipses)} \index{ellipses (dots)}
\idxboth{legal}{symbols}
\label{text-predef}
\begin{longtable}{lll@{\qqquad}lll}
\indexTextcomp\textasciicircum$^*$    					& \indexTextcomp\textless                             \\
\indexTextcomp\textasciitilde$^*$     						& \indexTextcomp[\ltextordfeminine]\textordfeminine   \\
\indexTextcomp\textasteriskcentered   					& \indexTextcomp[\ltextordmasculine]\textordmasculine \\
\indexTextcomp{\textbackslash}          				    & \indexTextcomp\textparagraph$^\dag$                 \\
texbar                                              & \indexTextcomp\textperiodcentered                   \\
\indexTextcomp{textbraceleft}           $^\dag$   & \indexTextcomp\textquestiondown                     \\
\indexTextcomp\textbraceright$^\dag$  & \indexTextcomp\textquotedblleft                     \\
\indexTextcomp\textbullet             & \indexTextcomp\textquotedblright                    \\
\indexTextcomp[\ltextcopyright]\textcopyright$^\dag$
                          & \indexTextcomp\textquoteleft                        \\
\indexTextcomp\textdagger$^\dag$      & \indexTextcomp\textquoteright                       \\
\indexTextcomp\textdaggerdbl$^\dag$   & \indexTextcomp[\ltextregistered]\textregistered     \\
\indexTextcomp\textdollar$^\dag$      & \indexTextcomp\textsection$^\dag$                   \\
\indexTextcomp\textellipsis$^\dag$    & \indexTextcomp\textsterling$^\dag$                  \\
\indexTextcomp\textemdash             & \indexTextcomp[\ltexttrademark]\texttrademark       \\
\indexTextcomp\textendash             & \indexTextcomp\textunderscore$^\dag$                \\
\indexTextcomp\textexclamdown         & \indexTextcomp\textvisiblespace                     \\
\indexTextcomp\textgreater                                                      \\
\end{longtable}

\bigskip
\twosymbolmessage

\bigskip
\begin{tablenote}[*]
  \docAuxCommand{^} and
%  \cmdI[\string\~{}]{\~{}}\verb|{}| can be used instead of
  \docAuxCommand{textasciicircum} and \docAuxCommand{textasciitilde}.  See the
  discussion of ``\texttt{\textasciitilde}'' \vpageref[below]{page:tildes}.
\end{tablenote}

\bigskip
\usetextmathmessage[\dag]
\end{longsymtable}



\begin{symtable}{\latexe{} Commands Defined to Work in Both Math and Text Mode}
\index{dots (ellipses)} \index{ellipses (dots)}
\index{copyright}
\idxboth{legal}{symbols}
\label{math-text}
\begin{tabular}{*3{lll@{\qqquad}}lll}
\indexTextcomp\$ & \indexTextcomp\_              & \indexTextcomp\ddag    & \Vp\{ \\
\indexTextcomp\P & \indexTextcomp[\ltextcopyright]\copyright
                         & \indexTextcomp\dots    & \Vp\} \\
 & \indexTextcomp\dag            & \indexTextcomp\pounds          \\%V\S removed
\end{tabular}

\bigskip
\twosymbolmessage
\end{symtable}

\begin{symtable}{AMS Commands Defined to Work in Both Math and Text Mode}
\index{check marks}
\label{ams-math-text}
\begin{tabular}{*2{ll@{\qquad}}ll}
\X\checkmark & \X\circledR & \X\maltese
\end{tabular}
\end{symtable}


\begin{symtable}{Non-ASCII Letters (Excluding Accented Letters)}
\index{letters>non-ASCII} %\K\l to fix
\index{ASCII}
\label{non-ascii}
\begin{tabular}{*4{ll@{\qqquad}}ll}
\K\aa      & \Ks\DH     & \Ks\L      & \K\o       & \K\ss                   \\
\K\AA      & \Ks\dh     & &          & \K\O       & \K\SS                   \\
\K\AE      & \Ks\DJ     & \Ks\NG     & \K\OE      & \Ks\TH                  \\
\K\ae      & \Ks\dj     & \Ks\ng     & \K\oe      & \Ks\th                  \\
\end{tabular}

\bigskip
\begin{tablenote}[*]
  Not available in the OT1 \fntenc[OT1].  Use the \pkgname{fontenc}
  package to select an alternate \fntenc[T1], such as T1.
\end{tablenote}
\end{symtable}

\section{Punctuation marks}

\begin{longsymtable}{Punctuation Marks Not Found in OT1}
\index{punctuation}
\label{punc-no-OT1}
\begin{longtable}{*8l}
\Kt\guillemotleft  & \Kt\guilsinglleft & \Kt\quotedblbase & \Kt\textquotedbl \\
\Kt\guillemotright & \Kt\guilsinglright & \Kt\quotesinglbase \\
\end{longtable}
\end{longsymtable}


\begin{longsymtable}[PI]{\PI\ Decorative Punctuation Marks}
\index{punctuation}
\label{pi-punctuation}
\begin{longtable}{*5{ll}}
\indexDing{123} & \indexDing{125} & \indexDing{161} & \indexDing{163} \\
\indexDing{124} & \indexDing{126} & \indexDing{162} \\
\end{longtable}
\medskip
\begin{tablenote}
  To get these symbols, use the \pkgname{fontenc} package to select an
  alternate \fntenc[T1], such as~T1.
\end{tablenote}

\end{longsymtable}

\section{Accents}
\begin{symtable}{Text-mode Accents}
\index{accents}
\index{accents>acute=acute (\blackacchack\')}   
\index{accents>arc=arc (\blackacchack\newtie)}
\index{accents>breve=breve (\blackacchack\u)}   
\index{accents>caron=caron (\blackacchack\v)} 
\index{accents>cedilla=cedilla (\blackacc\c)} 
\index{accents>circumflex=circumflex (\blackacchack\^)}  
\index{accents>diaeresis=di\ae{}resis (\blackacchack\")}  
\index{accents>dot=dot (\blackacchack\. or \blackacc\d)} 
\index{accents>double acute=double acute (\blackacchack\H)}  
\index{accents>grave=grave (\blackacchack\`)}  
\index{accents>ogonek=ogonek (\encone{\blackacc\k})} 
\index{accents>ring=ring (\blackacchack\r)} 
\label{text-accents}
\begin{tabular}{*3{ll@{\qqquad}}ll}
\Q\"                                & \Q\`         & \Q\d         & \Q\r        \\
\Q\'                                & \QivBAR\ddag & \Qiv\G\ddag  & \Q\t        \\
\Q\.                                & \Q\~         & \Qv\h\S      &        \\ %Q/u removed
\Qe[\magicequal][\magicequalname]\= & \Q\b         & \Q\H         & \Qiv\U\ddag \\
\Q\^                                & \Q\c         & \Qt\k$^\dag$ & \Q\v        \\
\end{tabular}
\par\medskip
\begin{tabular}{ll@{\qqquad}ll}
\Q\newtie$^*$ & \Qc\textcircled
\end{tabular}

\bigskip
\begin{tablenote}[*]
  Requires the \TC\ package.
\end{tablenote}

\medskip
\begin{tablenote}[\dag]
  Not available in the OT1 \fntenc[OT1].  Use the \pkgname{fontenc}
  package to select an alternate \fntenc[T1], such as T1.
\end{tablenote}

\medskip
\begin{tablenote}[\ddag]
  Requires the T4 \fntenc[T4], provided by the \FC\ package.
\end{tablenote}

\medskip
\begin{tablenote}[\S]
  Requires the T5 \fntenc[T5], provided by the \VIET\ package.
\end{tablenote}

\bigskip
\begin{tablenote}
  \index{dotless i=dotless $i~(\imath)$>text mode} \index{dotless
  j=dotless $j~(\jmath)$>text mode} Also note the existence of
  \docAuxCommand{i} and \docAuxCommand{j}, which produce dotless versions of ``i'' and
  ``j'' (viz., ``\i'' and ``\j'').  These are useful when the accent
  is supposed to replace the dot in encodings that need to
  composite\index{composited accents} (i.e.,~combine) letters and
  accents.  For example, ``\verb|na\"{\i}ve|'' always produces a
  correct ``na\"{\i}ve'', while ``\verb|na\"{i}ve|'' yields the rather
  odd-looking na\"{i}ve
  \makeatletter
  ``na\add@accent{127}{i}ve''\index{i=\add@accent{127}{i}}
  \makeatother
  when using the OT1 \fntenc[OT1] and older versions of \latex.  Font
  encodings other than OT1 and newer versions of \latex properly
  typeset ``\verb|na\"{i}ve|'' as ``na\"{\i}ve''.
\end{tablenote}
\end{symtable}

\section{Diacritics and Accents}

Again the most convenient way to get diagritics is to use the
\pkgname{textcomp}. The \TC\ package defines all of the above as ordinary characters,
  and not as accents. Of course with Unicode and True Type fonts, the worlds accents and
  diagritics, make these tables pale in comparison. 

\begin{longsymtable}{\TC\ Diacritics}
\index{accents}
\index{accents>acute=acute (\blackacchack\')}   
\index{accents>breve=breve (\blackacchack\u)}  
\index{accents>caron=caron (\blackacchack\v)}  
\index{accents>diaeresis=di\ae{}resis (\blackacchack\")} 
\index{accents>double acute=double acute (\blackacchack\H)}
\index{accents>grave=grave (\blackacchack\`)}  
\index{diacritics}
  
\label{tc-accent-chars}
\begin{longtable}{*3{ll}}
\K\textacutedbl      & \K\textasciicaron    & \K\textasciimacron \\
\K\textasciiacute    & \K\textasciidieresis & \K\textgravedbl    \\
\K\textasciibreve    & \K\textasciigrave                         \\
\end{longtable}
\end{longsymtable}


\begin{longsymtable}{\TC\ Currency Symbols}
\idxboth{currency}{symbols}
\idxboth{monetary}{symbols}
\index{euro signs}
\label{tc-currency}
\begin{longtable}{*4{ll}}
\K\textbaht          & \K\textdollar$^*$     & \K\textguarani  & \K\textwon \\
\K\textcent          & \K\textdollaroldstyle & \K\textlira     & \K\textyen \\
\K\textcentoldstyle  & \K\textdong           & \K\textnaira    \\
\K\textcolonmonetary & \K\texteuro           & \K\textpeso     \\
\K\textcurrency      & \K\textflorin         & \K\textsterling$^*$ \\
\end{longtable}
\end{longsymtable}

\begin{symtable}[MARV]{\MARV\ Currency Symbols}
\idxboth{currency}{symbols}
\idxboth{monetary}{symbols}
\index{euro signs}
\label{marv-currency}
\begin{tabular}{*4{ll}ll}
\K\Denarius   & \K\EUR    & \K\EURdig   & \K\EURtm      & \K\Pfund      \\
\K\Ecommerce  & \K\EURcr  & \K\EURhv    & \K\EyesDollar & \K\Shilling   \\
{\arial \char"20AC}                      &                &                   &                      &                   \\
\end{tabular}

\bigskip

\begin{tablenote}
  The different euro signs are meant to be visually compatible with
  different fonts---\PSfont{Courier} (\texttt{\string\EURcr}),
  \PSfont{Helvetica} (\texttt{\string\EURhv}), \PSfont{Times Roman}
  (\texttt{\string\EURtm}), and the \MARV\ digits listed in
  \ref{marv-digits} (\texttt{\string\EURdig}).
%  
%
%\ifMDES
%  The \MDES\ package redefines \cmdI[\MDEStexteuro]{\texteuro} to be
%  visually compatible with one of three additional fonts:
%  \PSfont{Utopia}~({\usefont{TS1}{mdput}{m}{n}\char"BF}),
%  \PSfont{Charter}~({\usefont{TS1}{mdbch}{m}{n}\char"BF}), or
%  \PSfont{Garamond}~({\usefont{TS1}{mdugm}{m}{n}\char"BF}).
%\fi
%
\end{tablenote}
\end{symtable}


\begin{symtable}[WASY]{\WASY\ Currency Symbols}
\idxboth{currency}{symbols}
\idxboth{monetary}{symbols}
\label{wasy-currency}
\begin{tabular}{ll@{\qquad}ll}
\K\cent & \K\currency \\
\end{tabular}
\end{symtable}

There is another package providing Euro related signs the \pkgname{eurosym}. The package provides the commands, \docAuxCommand{geneuro}, \docAuxCommand{geneuronarrow}, \docAuxCommand{geneurowide} and \cmd{\officialeuro}. You can read more at \url{http://www.theiling.de/eurosym.html}. \texttt{eurosym}  provides a new symbol to be used for the European currency, the Euro. The specifications were taken from a picture in the c't magazine 11/98 p.211 and from Encyclopaedia Britannica, Book of the Year 2002 (thanks to Dr. Werner Gans).

\texttt{eurosym}'s Euro symbol is implemented in \texttt{MetaFont}, and thus fits smoothly into a \texttt{LaTeX} installation. It is now part of major Linux distributions, including Debian, Suse, Mandrake and probably others.

Apart from the official form, the eurosym package provides some generalisations that fit non-roman font faces better.

\ifEUSYM
\begin{symtable}[EUSYM]{\EUSYM\ Euro Signs}
\idxboth{currency}{symbols}
\idxboth{monetary}{symbols}
\index{euro signs}
\label{eurosym-euros}
\begin{tabular}{*4{ll}}
\K\geneuro & \K\geneuronarrow & \K\geneurowide & \K\officialeuro \\
\end{tabular}

\bigskip

\begin{tablenote}
  \cmd{\euro} is automatically mapped to one of the above---by
  default, \docAuxCommand{officialeuro}---based on a \EUSYM\ package option.
  \seedocs{\EUSYM}.  The \verb|\geneuro|\dots{} characters are
  generated from the current body font's ``C'' character and therefore
  may not appear exactly as shown.
\end{tablenote}

\begin{tablenote}
To use the symbol with fontspec see the package documentation.
\end{tablenote}
\end{symtable}
\fi



\begin{symtable}[CHINA]{\CHINA\ Currency Symbols}
\idxboth{currency}{symbols}
\idxboth{monetary}{symbols}
\index{euro signs}
\label{china-euro}
\begin{tabular}{ll@{\qquad}ll}
  \K\Euro & \K\Pound \\
\end{tabular}
\end{symtable}



\begin{symtable}{\TC\ Legal Symbols}
\index{copyright}
\idxboth{legal}{symbols}
\label{tc-legal}
\begin{tabular}{*2{lll@{\qquad}}lll}
\indexTextcomp\textcircledP & \indexTextcomp[\ltextcopyright]\textcopyright   
&\indexTextcomp\textservicemark \\

\indexTextcomp\textcopyleft 
& \indexTextcomp[\ltextregistered]\textregistered 
& \indexTextcomp[\ltexttrademark]\texttrademark \\
\end{tabular}

\bigskip
\twosymbolmessage
\medskip
\begin{tablenote}
  \hspace*{15pt}%
  See \url{http://www.tex.ac.uk/cgi-bin/texfaq2html?label=tradesyms}
  for solutions to common problems that occur when using these symbols
  (e.g.,~getting a~``\textcircled{r}'' when you expected to get
  a~``\textregistered'').
\end{tablenote}
\end{symtable}


\begin{symtable}[CCLIC]{\CCLIC\ Creative Commons License Icons}
\index{Creative Commons licenses}
\index{copyright}
\idxboth{legal}{symbols}
\label{creativecommons}
\begin{tabular}{*4{ll@{\qqquad}}ll}
\K\cc & \K\ccby & \K\ccnc$^*$ & \K\ccnd & \K\ccsa$^*$ \\
\end{tabular}

\bigskip
\begin{tablenote}[*]
  These symbols utilize the \pkgname{rotating} package and therefore
  display improperly in some DVI\index{DVI} viewers.
\end{tablenote}
\end{symtable}


\begin{symtable}{\TC\ Old-style Numerals}
\idxboth{old-style}{digits}
\index{numerals>old style}
\label{old-style-nums}
\begin{tabular}{*3{ll}}
\K\textzerooldstyle  & \K\textfouroldstyle  & \K\texteightoldstyle \\
\K\textoneoldstyle   & \K\textfiveoldstyle  & \K\textnineoldstyle  \\
\K\texttwooldstyle   & \K\textsixoldstyle   \\
\K\textthreeoldstyle & \K\textsevenoldstyle \\
\end{tabular}

\bigskip
\begin{tablenote}
  Rather than use the bulky \cmd{\textoneoldstyle},
  \cmd{\texttwooldstyle}, etc.\ commands shown above, consider using
  \docAuxCommand{oldstylenums}\verb|{|$\ldots$\verb|}| to typeset an old-style eg. abcde{\oldstylenums 123456789}fgh. These type of
symbols and commands become redundant with the correct font, as the old style numbers are a feature of the font. Not all fonts provide old style numbers.
\end{tablenote}
\end{symtable}

\section{Miscellaneous Symbols}

\begin{longsymtable}{Miscellaneous \TC\ Symbols}
\idxboth{musical}{symbols}
\index{tilde}
\label{tc-misc}
\begin{longtable}{lll@{\qquad}lll}
\indexTextcomp\textasteriskcentered & \indexTextcomp[\ltextordfeminine]\textordfeminine   \\
\indexTextcomp\textbardbl           & \indexTextcomp[\ltextordmasculine]\textordmasculine \\
\indexTextcomp\textbigcircle        & \indexTextcomp\textparagraph$^*$                    \\
\indexTextcomp\textblank            & \indexTextcomp\textperiodcentered                   \\
\indexTextcomp\textbrokenbar        & \indexTextcomp\textpertenthousand                   \\
\indexTextcomp\textbullet           & \indexTextcomp\textperthousand                      \\
\indexTextcomp\textdagger$^*$       & \indexTextcomp\textpilcrow                          \\
\indexTextcomp\textdaggerdbl$^*$    & \indexTextcomp\textquotesingle                      \\
\indexTextcomp\textdblhyphen        & \indexTextcomp\textquotestraightbase                \\
\indexTextcomp\textdblhyphenchar    & \indexTextcomp\textquotestraightdblbase             \\
\indexTextcomp\textdiscount         & \indexTextcomp\textrecipe                           \\
\indexTextcomp\textestimated        & \indexTextcomp\textreferencemark                    \\
\indexTextcomp\textinterrobang      & \indexTextcomp\textsection$^*$                      \\
\indexTextcomp\textinterrobangdown  & \indexTextcomp\textthreequartersemdash              \\
\indexTextcomp\textmusicalnote      & \indexTextcomp\texttildelow                         \\
\indexTextcomp\textnumero           & \indexTextcomp\texttwelveudash                      \\
\indexTextcomp\textopenbullet                                                 \\
\end{longtable}

\bigskip
\twosymbolmessage

\bigskip
\usetextmathmessage[*]

\end{longsymtable}
%
%\begin{symtable}[WASY]{Miscellaneous \WASY\ Text-mode Symbols}
%\label{wasy-text}
%\begin{tabular}{ll}
%\K\permil \\
%\end{tabular}
%\end{symtable}
%\idxbothend{body-text}{symbols}



\section{Mathematical symbols}
\label{math-symbols}
\idxbothbegin{mathematical}{symbols}


Most, but not all, of the symbols in this section are math-mode only.
That is, they yield a ``\texttt{Missing~\$ inserted}''\index{Missing
\$ inserted=``\texttt{Missing~\$ inserted}''} error message if not
used within \verb|$|$\ldots$\verb|$|, \verb|\[|$\ldots$\verb|\]|, or
another math-mode environment.  Operators marked as ``variable-sized''
are taller in displayed formulas, shorter in in-text formulas, and
possibly shorter still when used in various levels of superscripts or
subscripts.

% The following definition is used both in the discussion of disjoint
% union and in the "Joining and overlapping existing symbols" section.

\newcommand{\dotcup}{\ensuremath{\mathaccent\cdot\cup}}


Alphanumeric symbols (e.g., $\mathscr{L}$, and
|\varmathbb{Z}|) are usually produced using one of the math
alphabets in \ref{alphabets} rather than with an explicit symbol
command.  Look there first if you need a symbol for a transform,
number set, or some other alphanumeric.

Although there have been many requests on \ctt for a
contradiction\idxboth{contradiction}{symbols} symbol, the ensuing
discussion invariably reveals innumerable ways to represent
contradiction in a proof, including ``|\blitza|''~(\cmd{\blitza}),
``$\Rightarrow\Leftarrow$''~(\docAuxCommand{Rightarrow}\docAuxCommand{Leftarrow}),
``$\bot$''~(\docAuxCommand{bot}),
``$\nleftrightarrow$''~(\docAuxCommand{nleftrightarrow}), and
%``\textreferencemark''~(\docAuxCommand{textreferencemark}).  Because of the
%lack of notational consensus, it is probably better to spell out
%``Contradiction!''\ than to use a symbol for this purpose.  Similarly,
%discussions on \ctt have revealed that there are a variety of ways to
%indicate the mathematical notion of ``is
%defined\idxboth{definition}{symbols} as''.  Common candidates include
%``$\triangleq$''~(\docAuxCommand{triangleq}), ``$\equiv$''~(\docAuxCommand{equiv}),
%``$\coloneqq$''~(\emph{various}\footnote{In \TX, \PX, and \MTOOLS\ the
%symbol is called \docAuxCommand{coloneqq}.  In |\ABX\| and MNS\footnote{Do not use it uses too many aplhabets} it's called
%\cmdI[$\string\ABXcoloneq$]{\coloneq}.  In \CEQ\ it's called
%colonequals}.}), and ``$\stackrel{\text{\tiny
%def}}{=}$''~(\cmd{\stackrel}\verb|{|\cmd{\text}\verb|{\tiny|
%\verb|def}}{=}|).  See also the example of \cmd{\equalsfill}
%\vpageref[below]{equalsfill-ex}.  Depending upon the context,
%disjoint\index{disjoint union} union may be represented as
%``$\coprod$''~(\docAuxCommand{coprod}), ``$\sqcup$''~(\docAuxCommand{sqcup}),
%``$\dotcup$''~(\docAuxCommand{dotcup}), ``$\oplus$''~(\docAuxCommand{oplus}), or any
%of a number of other symbols.\footnote{\person{Bob}{Tennent} listed
%these and other disjoint-union symbol possibilities in a November~2007
%post to \ctt.}  Finally, the average\index{average} value of a
%variable~$x$ is written by some people as
%``$\overline{x}$''~(\verb|\overline{x}|)\incsyms\indexaccent[$\string\blackacc{\string\overline}$]{\overline},
%by some people as ``$\langle x \rangle$''~(\docAuxCommand{langle} \texttt{x}
%\docAuxCommand{rangle}), and by some people as ``$\diameter x$'' or
%``$\varnothing x$''~(\docAuxCommand{diameter} \texttt{x} or \docAuxCommand{varnothing}
%\texttt{x}).  The moral of the story is that you should be careful
%always to explain your notation to avoid confusing your readers.



\bigskip

\begin{symtable}{Math-Mode Versions of Text Symbols}
\index{underline}
\label{math-text-vers}
\begin{tabular}{*3{ll}}
\X\mathdollar   & \X\mathparagraph & \X\mathsterling   \\
\X\mathellipsis & \X\mathsection   & \X\mathunderscore \\
\end{tabular}

\bigskip
\usetextmathmessage

\end{symtable}

\subsection{CMLL}
The \pkgname{cmll} defines a handful of symbols useful in linear logic and not found in other
%font packages \cite{cmll}. The package defines unary operators, binary operators, large operators, binary relations and letter-like symbols |\Bot| $\Bot$ and $\simbot$
%and |\simbot|.

\begin{symtable}[CMLL]{\CMLL\ Unary Operators}
\idxboth{unary}{operators}
\idxboth{linear logic}{symbols}
\label{cmll-unary}
\begin{tabular}{*2{ll@{\qquad}}ll}
\K[!]\oc$^*$         & \K[\CMLLshneg]\shneg & \K[?]\wn$^*$ \\
\K[\CMLLshift]\shift & \K[\CMLLshpos]\shpos &              \\
\end{tabular}

\bigskip

\begin{tablenote}[*]
  \docAuxCommand{oc} and \docAuxCommand{wn} differ from~``!''  and~``?'' in
  terms of their math-mode spacing: \verb|$A=!B$| produces ``$A=!B$'',
  for example, while \verb|$A=\oc B$| produces ``$A=\mathord{!}B$''.
\end{tablenote}
\end{symtable}


`Linear implication' is not included in the grammar of connectives, but is definable in CLL using linear negation and multiplicative disjunction, by $A⊸B:=A^{{\pan ⊥}}$.


\begin{symtable}{Binary Operators}
\idxboth{binary}{operators}
\index{division}
\idxboth{linear logic}{symbols}
\label{bin}
\begin{tabular}{*4{ll}}
\X\amalg           & \X\cup          & \X\oplus    & \X\times           \\
\X\ast             & \X\dagger       & \X\oslash   & \X\triangleleft    \\
\X\bigcirc         & \X\ddagger      & \X\otimes   & \X\triangleright   \\
\X\bigtriangledown & \X\diamond      & \X\pm       & \X\unlhd$^*$       \\
\X\bigtriangleup   & \X\div          & \X\rhd$^*$  & \X\unrhd$^*$       \\
\X\bullet          & \X\lhd$^*$      & \X\setminus & \X\uplus           \\
\X\cap             & \X\mp           & \X\sqcap    & \X\vee             \\
\X\cdot            & \X\odot         & \X\sqcup    & \X\wedge           \\
\X\circ            & \X\ominus       & \X\star     & \X\wr              \\
\end{tabular}

\bigskip
\notpredefinedmessage
\end{symtable}


\begin{symtable}{AMS Binary Operators}
\idxboth{binary}{operators}
\index{semidirect products}
\label{ams-bin}
\begin{tabular}{*3{ll}}
\X\barwedge        & \X\circledcirc     & \X\intercal$^*$    \\
\X\boxdot          & \X\circleddash     & \X\leftthreetimes  \\
\X\boxminus        & \X\Cup             & \X\ltimes          \\
\X\boxplus         & \X\curlyvee        & \X\rightthreetimes \\
\X\boxtimes        & \X\curlywedge      & \X\rtimes          \\
\X\Cap             & \X\divideontimes   & \X\smallsetminus   \\
\X\centerdot       & \X\dotplus         & \X\veebar          \\
\X\circledast      & \X\doublebarwedge  \\
\end{tabular}

\bigskip

\begin{tablenote}[*]
  \newcommand{\trpose}{{\mathpalette\raiseT{\intercal}}}
  \newcommand{\raiseT}[2]{\raisebox{0.25ex}{$#1#2$}}
%
  Some people use a superscripted \docAuxCommand{intercal} for matrix
  transpose\index{transpose}: ``\verb|A^\intercal|''~$\mapsto$
  ``$A^\intercal$''.  (See the May~2009 \ctt thread, ``raising math
  symbols'', for suggestions about altering the height of the
  superscript. and se.tex question \footnote{\url{http://tex.stackexchange.com/questions/30619/what-is-the-best-symbol -for-vector-matrix-transpose}})  \docAuxCommand{top} (\vref*{letter-like}), \verb|T|, and
  \verb|\mathsf{T}| are other popular choices: ``$A^\top$'',
  ``$A^T$'', ``$A^{\text{\textsf{T}}}$''.
\end{tablenote}

\end{symtable}



\subsection{St Mary Road Binary Operators}

%\begin{symtable}[ST]{\ST\ Binary Operators}
\idxboth{binary}{operators}
\idxboth{linear logic}{symbols}
\label{st-bin}
\captionof{table}{\ST\ Binary Operators}
\begin{longtable}{*3{ll}}
\X\baro                & \X\interleave          & \X\varoast             \\
\X\bbslash             & \X\leftslice           & \X\varobar             \\
\X\binampersand        & \X\merge               & \X\varobslash          \\
\X\bindnasrepma        & \X\minuso              & \X\varocircle          \\
\X\boxast              & \X\moo                 & \X\varodot             \\
\X\boxbar              & \X\nplus               & \X\varogreaterthan     \\
\X\boxbox              & \X\obar                & \X\varolessthan        \\
\X\boxbslash           & \X\oblong              & \X\varominus           \\
\X\boxcircle           & \X\obslash             & \X\varoplus            \\
\X\boxdot              & \X\ogreaterthan        & \X\varoslash           \\
\X\boxempty            & \X\olessthan           & \X\varotimes           \\
\X\boxslash            & \X\ovee                & \X\varovee             \\
\X\curlyveedownarrow   & \X\owedge              & \X\varowedge           \\
\X\curlyveeuparrow     & \X\rightslice          & \X\vartimes            \\
\X\curlywedgedownarrow & \X\sslash              & \X\Ydown               \\
\X\curlywedgeuparrow   & \X\talloblong          & \X\Yleft               \\
\X\fatbslash           & \X\varbigcirc          & \X\Yright              \\
\X\fatsemi             & \X\varcurlyvee         & \X\Yup                 \\
\X\fatslash            & \X\varcurlywedge       \\
\end{longtable}


%\end{symtable}


\begin{symtable}[WASY]{\WASY\ Binary Operators}
\idxboth{binary}{operators}
\label{wasy-bin}
\begin{tabular}{*4{ll}}
\X\lhd & \X\ocircle & \X\RHD   & \X\unrhd \\
\X\LHD & \X\rhd     & \X\unlhd            \\
\end{tabular}
\end{symtable}


\section{Unicode Binary Operators}
 \subsection{Binary operators}
 \index{binary operators}
 \begin{multicols}{2}
% \showsymbolbin+{000B}{}
 \showsymbolbin\pm{00B1}{}
 \showsymbolbin\cdotp{00B7}{}%?, \cmd\centerdot
 \showsymbolbin\times{00D7}{}
 \showsymbolbin\div{00F7}{}
 \showsymbolbin\dagger{2020}{}
 \showsymbolbin\ddagger{2021}{}
 \showsymbolbin\smblkcircle{2022}{}
 \showsymbolbin\fracslash{2044}{}
 \showsymbolbin\upand{214B}{}
% \showsymbolbin-{000D}{}
 \showsymbolbin\mp{2213}{}
 \showsymbolbin\dotplus{2214}{}
 \showsymbolbin\smallsetminus{2216}{}
 \showsymbolbin\ast{2217}{}
 \showsymbolbin\vysmwhtcircle{2218}{}
 \showsymbolbin\vysmblkcircle{2219}{}, {\small\cmd\bullet}
 \showsymbolbin\wedge{2227}{}, \cmd\land
 \showsymbolbin\vee{2228}{}, \cmd\lor
 \showsymbolbin\cap{2229}{}
 \showsymbolbin\cup{222A}{}
 \showsymbolbin\dotminus{2238}{}
 \showsymbolbin\invlazys{223E}{}
 \showsymbolbin\wr{2240}{}
 \showsymbolbin\cupleftarrow{228C}{}
 \showsymbolbin\cupdot{228D}{}
 \showsymbolbin\uplus{228E}{}
 \showsymbolbin\sqcap{2293}{}
 \showsymbolbin\sqcup{2294}{}
 \showsymbolbin\oplus{2295}{}
 \showsymbolbin\ominus{2296}{}
 \showsymbolbin\otimes{2297}{}
 \showsymbolbin\oslash{2298}{}
 \showsymbolbin\odot{2299}{}
 \showsymbolbin\circledcirc{229A}{}
 \showsymbolbin\circledast{229B}{}
 \showsymbolbin\circledequal{229C}{}
 \showsymbolbin\circleddash{229D}{}
 \showsymbolbin\boxplus{229E}{}
 \showsymbolbin\boxminus{229F}{}
 \showsymbolbin\boxtimes{22A0}{}
 \showsymbolbin\boxdot{22A1}{}
 \showsymbolbin\intercal{22BA}{}
 \showsymbolbin\veebar{22BB}{}
 \showsymbolbin\barwedge{22BC}{}
 \showsymbolbin\barvee{22BD}{}
 \showsymbolbin\diamond{22C4}{}, \cmd\smwhtdiamond
 \showsymbolbin\cdot{22C5}{*}
 \showsymbolbin\star{22C6}{}
 \showsymbolbin\divideontimes{22C7}{}
 \showsymbolbin\ltimes{22C9}{}
 \showsymbolbin\rtimes{22CA}{}
 \showsymbolbin\leftthreetimes{22CB}{}
 \showsymbolbin\rightthreetimes{22CC}{}
 \showsymbolbin\curlyvee{22CE}{}
 \showsymbolbin\curlywedge{22CF}{}
 \showsymbolbin\Cap{22D2}{}, \cmd\doublecap
 \showsymbolbin\Cup{22D3}{}, \cmd\doublecup
 \showsymbolbin\varbarwedge{2305}{*}
 \showsymbolbin\vardoublebarwedge{2306}{*}
 \showsymbolbin\obar{233D}{}
 \showsymbolbin\triangle{25B3}{}, \cmd\bigtriangleup
 \showsymbolbin\lhd{22B2}{}
 \showsymbolbin\rhd{22B3}{}
 \showsymbolbin\unlhd{22B4}{}
 \showsymbolbin\unrhd{22B5}{}
 \showsymbolbin\mdlgwhtcircle{25CB}{*}
 \showsymbolbin\boxbar{25EB}{*}
 \showsymbolbin\veedot{27C7}{*}
 \showsymbolbin\wedgedot{27D1}{*}
 \showsymbolbin\lozengeminus{27E0}{*}
 \showsymbolbin\concavediamond{27E1}{*}
 \showsymbolbin\concavediamondtickleft{27E2}{*}
 \showsymbolbin\concavediamondtickright{27E3}{*}
 \showsymbolbin\whitesquaretickleft{27E4}{*}
 \showsymbolbin\whitesquaretickright{27E5}{*}
 \showsymbolbin\typecolon{2982}{*}
 \showsymbolbin\circlehbar{29B5}{*}
 \showsymbolbin\circledvert{29B6}{}
 \showsymbolbin\circledparallel{29B7}{}
 \showsymbolbin\obslash{29B8}{}
 \showsymbolbin\operp{29B9}{*}
 \showsymbolbin\olessthan{29C0}{}
 \showsymbolbin\ogreaterthan{29C1}{}
 \showsymbolbin\boxdiag{29C4}{}
 \showsymbolbin\boxbslash{29C5}{}
 \showsymbolbin\boxast{29C6}{}
 \showsymbolbin\boxcircle{29C7}{}
 \showsymbolbin\boxbox{29C8}{*}
 \showsymbolbin\triangleserifs{29CD}{*}
 \showsymbolbin\hourglass{29D6}{*}
 \showsymbolbin\blackhourglass{29D7}{*}
 \showsymbolbin\shuffle{29E2}{*}
 \showsymbolbin\mdlgblklozenge{29EB}{*}
 \showsymbolbin\setminus{29F5}{*}
 \showsymbolbin\dsol{29F6}{*}
 \showsymbolbin\rsolbar{29F7}{*}
 \showsymbolbin\doubleplus{29FA}{*}
 \showsymbolbin\tripleplus{29FB}{*}
 \showsymbolbin\tplus{29FE}{*}
 \showsymbolbin\tminus{29FF}{*}
 \showsymbolbin\ringplus{2A22}{}
 \showsymbolbin\plushat{2A23}{}
 \showsymbolbin\simplus{2A24}{}
 \showsymbolbin\plusdot{2A25}{}
 \showsymbolbin\plussim{2A26}{}
 \showsymbolbin\plussubtwo{2A27}{}
 \showsymbolbin\plustrif{2A28}{*}
 \showsymbolbin\commaminus{2A29}{*}
 \showsymbolbin\minusdot{2A2A}{}
 \showsymbolbin\minusfdots{2A2B}{}
 \showsymbolbin\minusrdots{2A2C}{*}
 \showsymbolbin\opluslhrim{2A2D}{*}
 \showsymbolbin\oplusrhrim{2A2E}{*}
 \showsymbolbin\vectimes{2A2F}{*}
 \showsymbolbin\dottimes{2A30}{}
 \showsymbolbin\timesbar{2A31}{}
 \showsymbolbin\btimes{2A32}{}
 \showsymbolbin\smashtimes{2A33}{*}
 \showsymbolbin\otimeslhrim{2A34}{*}
 \showsymbolbin\otimesrhrim{2A35}{*}
 \showsymbolbin\otimeshat{2A36}{*}
 \showsymbolbin\Otimes{2A37}{*}
 \showsymbolbin\odiv{2A38}{*}
 \showsymbolbin\triangleplus{2A39}{*}
 \showsymbolbin\triangleminus{2A3A}{*}
 \showsymbolbin\triangletimes{2A3B}{*}
 \showsymbolbin\intprod{2A3C}{*}
 \showsymbolbin\intprodr{2A3D}{*}
 \showsymbolbin\fcmp{2A3E}{*}
 \showsymbolbin\amalg{2A3F}{}
 \showsymbolbin\capdot{2A40}{*}
 \showsymbolbin\uminus{2A41}{*}
 \showsymbolbin\barcup{2A42}{*}
 \showsymbolbin\barcap{2A43}{*}
 \showsymbolbin\capwedge{2A44}{*}
 \showsymbolbin\cupvee{2A45}{*}
 \showsymbolbin\cupovercap{2A46}{*}
 \showsymbolbin\capovercup{2A47}{*}
 \showsymbolbin\cupbarcap{2A48}{*}
 \showsymbolbin\capbarcup{2A49}{*}
 \showsymbolbin\twocups{2A4A}{*}
 \showsymbolbin\twocaps{2A4B}{*}
 \showsymbolbin\closedvarcup{2A4C}{*}
 \showsymbolbin\closedvarcap{2A4D}{*}
 \showsymbolbin\Sqcap{2A4E}{*}
 \showsymbolbin\Sqcup{2A4F}{*}
 \showsymbolbin\closedvarcupsmashprod{2A50}{*}
 \showsymbolbin\wedgeodot{2A51}{*}
 \showsymbolbin\veeodot{2A52}{*}
 \showsymbolbin\Wedge{2A53}{*}
 \showsymbolbin\Vee{2A54}{*}
 \showsymbolbin\wedgeonwedge{2A55}{*}
 \showsymbolbin\veeonvee{2A56}{*}
 \showsymbolbin\bigslopedvee{2A57}{*}
 \showsymbolbin\bigslopedwedge{2A58}{*}
 \showsymbolbin\wedgemidvert{2A5A}{*}
 \showsymbolbin\veemidvert{2A5B}{*}
 \showsymbolbin\midbarwedge{2A5C}{*}
 \showsymbolbin\midbarvee{2A5D}{*}
 \showsymbolbin\doublebarwedge{2A5E}{}
 \showsymbolbin\wedgebar{2A5F}{*}
 \showsymbolbin\wedgedoublebar{2A60}{*}
 \showsymbolbin\varveebar{2A61}{*}
 \showsymbolbin\doublebarvee{2A62}{*}
 \showsymbolbin\veedoublebar{2A63}{}
 \showsymbolbin\dsub{2A64}{*}
 \showsymbolbin\rsub{2A65}{*}
 \showsymbolbin\eqqplus{2A71}{}
 \showsymbolbin\pluseqq{2A72}{}
 \showsymbolbin\interleave{2AF4}{}
 \showsymbolbin\nhVvert{2AF5}{}
 \showsymbolbin\threedotcolon{2AF6}{}
 \showsymbolbin\trslash{2AFB}{}
 \showsymbolbin\sslash{2AFD}{}
 \showsymbolbin\talloblong{2AFE}{}
 \end{multicols}





\begin{symtable}{Variable-sized Math Operators}
\idxboth{variable-sized}{symbols}
\idxboth{linear logic}{symbols}
\index{integrals}
\label{op}
\renewcommand{\arraystretch}{1.75}  
\begin{tabular}{*3{l@{$\:$}ll@{\qquad}}l@{$\:$}ll}
\R\bigcap    & \R\bigotimes & \R\bigwedge  & \R\prod      \\
\R\bigcup    & \R\bigsqcup  & \R\coprod    & \R\sum       \\
\R\bigodot   & \R\biguplus  & \R\int       \\
\R\bigoplus  & \R\bigvee    & \R\oint      \\
\end{tabular}
\end{symtable}




\begin{symtable}[AMS]{\AmS Variable-sized Math Operators}
\idxboth{variable-sized}{symbols}
\index{integrals}
\label{ams-large}
\renewcommand{\arraystretch}{2.5}  
\begin{tabular}{l@{$\:$}ll@{\qquad}l@{$\:$}ll}
% removed optional \R[\AMSiint]
\R\iint     & \R\iiint       \\
\R\iiiint & \R\idotsint \\
\end{tabular}
\end{symtable}


\begin{symtable}[ST]{\ST\ Variable-sized Math Operators}
\idxboth{variable-sized}{symbols}
\label{st-large}
\renewcommand{\arraystretch}{1.75} 
\begin{tabular}{*2{l@{$\:$}ll@{\qquad}}l@{$\:$}ll}
\R\bigbox        & \R\biginterleave & \R\bigsqcap                            \\
\R\bigcurlyvee   & \R\bignplus      & \R[\STbigtriangledown]\bigtriangledown \\
\R\bigcurlywedge & \R\bigparallel   & \R[\STbigtriangleup]\bigtriangleup     \\
\end{tabular}
\end{symtable}


\begin{symtable}[WASY]{\WASY\ Variable-sized Math Operators}
\idxboth{variable-sized}{symbols}
\index{integrals}
\label{wasy-large}
\renewcommand{\arraystretch}{2.5}  
\begin{tabular}{*2{l@{$\:$}ll@{\qquad}}l@{$\:$}ll}
\R[\varint]\int$^\dag$ & \R\iint        & \R\iiint \\
\R\varint$^*$          & \R\varoint$^*$ & \R\oiint \\
\end{tabular}

\bigskip
\begin{tablenote}
  None of the preceding symbols are defined when \WASY\ is passed the
  \optname{wasysym}{nointegrals} option.
\end{tablenote}

\medskip
\begin{tablenote}[*]
  Not defined when \WASY\ is passed the \optname{wasysym}{integrals} option.
\end{tablenote}

\medskip
\begin{tablenote}[\dag]
  Defined only when \WASY\ is passed the \optname{wasysym}{integrals}
  option.  Otherwise, the default \latex \docAuxCommand{int} glyph (as shown
  in \ref{op}) is used.
\end{tablenote}
\end{symtable}

\begin{symtable}{Negated Binary Relations}
\index{binary relations>negated}
\index{relational symbols>negated binary}
\label{ams-nrel}
\begin{tabular}{*3{ll}}
\X\ncong     & \X\nshortparallel & \X\nVDash      \\
\X\nmid      & \X\nsim           & \X\precnapprox \\
\X\nparallel & \X\nsucc          & \X\precnsim    \\
\X\nprec     & \X\nsucceq        & \X\succnapprox \\
\X\npreceq   & \X\nvDash         & \X\succnsim    \\
\X\nshortmid & \X\nvdash                          \\
\end{tabular}
\end{symtable}


%\begin{symtable}[ST]{\ST\ Binary Relations}
%\index{binary relations}
%\index{relational symbols>binary}
%\label{st-rel}
%\begin{tabular}{*2{ll}}
%\X\inplus & \X\niplus \\
%\end{tabular}
%\end{symtable}


\begin{symtable}[WASY]{\WASY\ Binary Relations}
\index{binary relations}
\index{relational symbols>binary}
\label{wasy-rel}
\begin{tabular}{*3{ll}}
\X\invneg & \X\leadsto & \X\wasypropto \\
\X\Join   & \X\logof                   \\
\end{tabular}
\end{symtable}


\begin{symtable}[CMLL]{\CMLL\ Binary Relations}
\index{binary relations}
\index{relational symbols>binary}
\idxboth{linear logic}{symbols}
\label{cmll-rel}
\begin{tabular}{ll@{\hspace*{2em}}ll}
\K[\CMLLcoh]\coh     & \K[\CMLLscoh]\scoh     \\
\K[\CMLLincoh]\incoh & \K[\CMLLsincoh]\sincoh \\
\end{tabular}
\end{symtable}



\begin{symtable}{Subset and Superset Relations}
\index{binary relations}
\index{relational symbols>binary}
\index{subsets}
\index{supersets}
\index{symbols>subset and superset}
\label{subsets}
\begin{tabular}{*3{ll}}
\X\sqsubset$^*$ & \X\sqsupseteq & \X\supset   \\
\X\sqsubseteq   & \X\subset     & \X\supseteq \\
\X\sqsupset$^*$ & \X\subseteq                 \\
\end{tabular}

\bigskip
\notpredefinedmessageABX
\end{symtable}

\section{Inequalities}
\begin{symtable}{Inequalities}
\index{binary relations}\index{relational symbols>binary}
\index{inequalities}
\label{inequal-rel}
\begin{tabular}{*5{ll}}
\X\geq & \X\gg & \X\leq & \X\ll & \X\neq \\
\end{tabular}
\end{symtable}
\begin{symtable}{ Subset and Superset Relations}
\index{binary relations}
\index{relational symbols>binary}
\index{subsets}
\index{supersets}
\index{symbols>subset and superset}
\label{ams-subsets}
\begin{tabular}{*3{ll}}
\X\nsubseteq  & \X\subseteqq  & \X\supsetneqq    \\
\X\nsupseteq  & \X\subsetneq  & \X\varsubsetneq  \\
\X\nsupseteqq & \X\subsetneqq & \X\varsubsetneqq \\
\X\sqsubset   & \X\Supset     & \X\varsupsetneq  \\
\X\sqsupset   & \X\supseteqq  & \X\varsupsetneqq \\
\X\Subset     & \X\supsetneq                     \\
\end{tabular}
\end{symtable}


%\begin{symtable}[ST]{\ST\ Subset and Superset Relations}
%\index{binary relations}
%\index{relational symbols>binary}
%\index{subsets}
%\index{supersets}
%\index{symbols>subset and superset}
%\label{st-subsets}
%\begin{tabular}{*2{ll}}
%\X\subsetplus   & \X\supsetplus   \\
%\X\subsetpluseq & \X\supsetpluseq \\
%\end{tabular}
%\end{symtable}


\begin{symtable}[WASY]{\WASY\ Subset and Superset Relations}
\index{binary relations}
\index{relational symbols>binary}
\index{subsets}
\index{supersets}
\index{symbols>subset and superset}
\label{wasy-subset}
\begin{tabular}{*2{ll}}
\X\sqsubset & \X\sqsupset \\
\end{tabular}
\end{symtable}


\begin{symtable}{AMS Triangle Relations}
\index{triangle relations}\index{relational symbols>triangle}
\label{ams-triangle-rel}
\begin{tabular}{*3{ll}}
\X\blacktriangleleft  & \X\ntriangleright    & \X\trianglerighteq  \\
\X\blacktriangleright & \X\ntrianglerighteq  & \X\vartriangleleft  \\
\X\ntriangleleft      & \X\trianglelefteq    & \X\vartriangleright \\
\X\ntrianglelefteq    & \X\triangleq         &                     \\
\end{tabular}
\end{symtable}


%\begin{symtable}[ST]{\ST\ Triangle Relations}
%\index{triangle relations}\index{relational symbols>triangle}
%\label{st-triangle-rel}
%\begin{tabular}{*2{ll}}
%\X\trianglelefteqslant  & \X\trianglerighteqslant  \\
%\X\ntrianglelefteqslant & \X\ntrianglerighteqslant \\
%\end{tabular}
%\end{symtable}




\begin{symtable}{Arrows}
\index{arrows}
\label{arrow}
\begin{tabular}{*3{ll}}
\X\Downarrow          & \X\longleftarrow      & \X\nwarrow     \\
\X\downarrow          & \X\Longleftarrow      & \X\Rightarrow  \\
\X\hookleftarrow      & \X\longleftrightarrow & \X\rightarrow  \\
\X\hookrightarrow     & \X\Longleftrightarrow & \X\searrow     \\
\X\leadsto$^*$        & \X\longmapsto         & \X\swarrow     \\
\X\leftarrow          & \X\Longrightarrow     & \X\uparrow     \\
\X\Leftarrow          & \X\longrightarrow     & \X\Uparrow     \\
\X\Leftrightarrow     & \X\mapsto             & \X\updownarrow \\
\X\leftrightarrow     & \X\nearrow$^\dag$     & \X\Updownarrow \\
\end{tabular}

\bigskip
\notpredefinedmessage

\bigskip
\begin{tablenote}[\dag]
  See the note beneath \ref{extensible-accents} for information
  about how to put a diagonal arrow across a mathematical expression%
%\ifhavecancel
%  ~(as in ``$\cancelto{0}{\nabla \cdot \vec{B}}\quad$'')
%\fi
.
\end{tablenote}
\end{symtable}


\begin{symtable}{Harpoons}
\index{harpoons}
\label{harpoons}
\begin{tabular}{*3{ll}}
\X\leftharpoondown   & \X\rightharpoondown  & \X\rightleftharpoons \\
\X\leftharpoonup     & \X\rightharpoonup                           \\
\end{tabular}
\end{symtable}


\begin{symtable}{\TC\ Text-mode Arrows}
\index{arrows}
\label{tc-arrows}
\begin{tabular}{*2{ll}}
\K\textdownarrow & \K\textrightarrow \\
\K\textleftarrow & \K\textuparrow    \\
\end{tabular}
\end{symtable}


\begin{symtable}{AmS Arrows}
\index{arrows}
\label{ams-arrows}
\begin{tabular}{*3{ll}}
\X\circlearrowleft    & \X\leftleftarrows          & \X\rightleftarrows   \\
\X\circlearrowright  & \X\leftrightarrows       & \X\rightrightarrows  \\
\X\curvearrowleft   & \X\leftrightsquigarrow & \X\rightsquigarrow   \\
\X\curvearrowright & \X\Lleftarrow              & \X\Rsh               \\
\X\dashleftarrow     & \X\looparrowleft        & \X\twoheadleftarrow  \\
\X\dashrightarrow  & \X\looparrowright      & \X\twoheadrightarrow \\
\X\downdownarrows   & \X\Lsh                   & \X\upuparrows        \\
\X\leftarrowtail       & \X\rightarrowtail        &                      \\
\end{tabular}
\end{symtable}


\begin{symtable}{\AmS Negated Arrows}
\index{arrows>negated}
\label{ams-narrows}
\begin{tabular}{*3{ll}}
\X\nLeftarrow       & \X\nLeftrightarrow  & \X\nRightarrow     \\
\X\nleftarrow       & \X\nleftrightarrow   & \X\nrightarrow     \\
\end{tabular}
\end{symtable}


\begin{symtable}{\AmS Harpoons}
\index{harpoons}
\label{ams-harpoons}
\begin{tabular}{*3{ll}}
\X\downharpoonleft  & \X\leftrightharpoons   & \X\upharpoonleft  \\
\X\downharpoonright & \X\rightleftharpoons & \X\upharpoonright \\
\end{tabular}
\end{symtable}




\section{Log-like Symbols}
\begin{symtable}{Log-like Symbols}
\idxboth{log-like}{symbols}
\index{atomic math objects}
\index{limits}
\label{log}
\begin{tabular}{*8l}
\Z\arccos & \Z\cos  & \Z\csc & \Z\exp & \Z\ker    & \Z\limsup & \Z\min & \Z\sinh \\
\Z\arcsin & \Z\cosh & \Z\deg & \Z\gcd & \Z\lg     & \Z\ln     & \Z\Pr  & \Z\sup  \\
\Z\arctan & \Z\cot  & \Z\det & \Z\hom & \Z\lim    & \Z\log    & \Z\sec & \Z\tan  \\
\Z\arg    & \Z\coth & \Z\dim & \Z\inf & \Z\liminf & \Z\max    & \Z\sin & \Z\tanh
\end{tabular}

\bigskip
\begin{tablenote}
  Calling the above ``symbols'' may be a bit
  misleading.\footnotemark{} Each log-like symbol merely produces the
  eponymous textual equivalent, but with proper surrounding spacing.
  See \ref{math-spacing} for more information about log-like
  symbols.  As \cmd{\bmod} and \cmd{\pmod} are arguably not symbols we
  refer the reader to the Short Math Guide for
  \latex~\cite{Downes:smg} for samples.
\end{tablenote}
\end{symtable}
\footnotetext{Michael\index{Downes, Michael J.} J. Downes prefers the
more general term, ``atomic\index{atomic math objects} math objects''.}


\begin{symtable}{AMS Log-like Symbols}
\idxboth{log-like}{symbols}
\index{atomic math objects}
\index{limits}
\label{ams-log}
\renewcommand{\arraystretch}{1.5} 
\begin{tabular}{*2{ll@{\qquad}}ll}
\X\injlim     & \X\varinjlim  & \X\varlimsup  \\
\X\projlim    & \X\varliminf  & \X\varprojlim
\end{tabular}

\bigskip
\begin{tablenote}
  Load the \pkgname{amsmath} package to get these symbols.  See
  \ref{math-spacing} for some additional comments regarding
  log-like symbols.  As \cmd{\mod} and \cmd{\pod} are arguably not
  symbols we refer the reader to the Short Math Guide for
  \latex~\cite{Downes:smg} for samples.
\end{tablenote}
\end{symtable}



\section{Greek Letters}
   
  For usage see also Chapter \vref{ch:maths}. Greek letters are fundamental
  for most mathematical documents and the control sequences to use them are shown in 
  Table~\vref{greek}.
   
  The remaining Greek majuscules\index{majuscules} can be produced
  with ordinary Latin letters.  The symbol ``M'', for instance, is
  used for both an uppercase ``m'' and an uppercase ``$\mu$''.

  See \ref{bold-math} for examples of how to produce bold Greek
  letters.\index{Greek>bold}

  The symbols in this table are intended to be used in mathematical
  typesetting.  Greek body text can be typeset using the
  \pkgname{babel} package's \optname{babel}{greek} (or
  \optname{babel}{polutonikogreek}\idxboth{polytonic}{Greek})
  option---and, of course, a font that provides the glyphs for the
  Greek alphabet.

\begin{longsymtable}{Greek Letters}
\index{Greek}\index{alphabets>Greek}
\label{greek}
\begin{longtable}{*8l}
\X\alpha        &\X\theta       &\X o           &\X\tau         \\
\X\beta         &\X\vartheta    &\X\pi          &\X\upsilon     \\
\X\gamma        &\X\iota        &\X\varpi       &\X\phi         \\
\X\delta        &\X\kappa       &\X\rho         &\X\varphi      \\
\X\epsilon      &\X\lambda      &\X\varrho      &\X\chi         \\
\X\varepsilon   &\X\mu          &\X\sigma       &\X\psi         \\
\X\zeta         &\X\nu          &\X\varsigma    &\X\omega       \\
\X\eta          &\X\xi                                          \\
                                                                \\
\X\Gamma        &\X\Lambda      &\X\Sigma       &\X\Psi         \\
\X\Delta        &\X\Xi          &\X\Upsilon     &\X\Omega       \\
\X\Theta        &\X\Pi          &\X\Phi
\end{longtable}
\end{longsymtable}


\begin{symtable}[AMS]{\AmS\ Greek Letters}
\index{Greek}\index{alphabets>Greek}
\label{ams-greek}
\begin{tabular}{*4l}
\X\digamma      &\X\varkappa
\end{tabular}
\end{symtable}



\section{Hebrew letters}
\begin{symtable}{AMS Hebrew Letters}
\index{Hebrew}\index{alphabets>Hebrew}
\label{ams-hebrew}
\begin{tabular}{*6l}
\X\beth & \X\gimel & \X\daleth
\end{tabular}

\bigskip
\begin{tablenote}
\docAuxCommand{aleph}~($\aleph$) appears in \vref{ord}.
\end{tablenote}
\end{symtable}

\section{Letter-like Symbols}

\begin{symtable}{Letter-like Symbols}
\idxboth{letter-like}{symbols}
\index{tacks}
\idxboth{linear logic}{symbols}
\label{letter-like}
\begin{tabular}{*5{ll}}
\X\bot    & \X\forall & \X\imath & \X\ni      & \X\top \\
\X\ell    & \X\hbar   & \X\in    & \X\partial & \X\wp  \\
\X\exists & \X\Im     & \X\jmath & \X\Re               \\
\end{tabular}
\end{symtable}


\begin{symtable}{\AmS Letter-like Symbols}
\idxboth{letter-like}{symbols}
\label{ams-letter-like}
\begin{tabular}{*3{ll}}
\X\Bbbk       & \X\complement & \X\hbar    \\
\X\circledR   & \X\Finv       & \X\hslash  \\
\X\circledS   & \X\Game       & \X\nexists \\
\end{tabular}
\end{symtable}


\section{Variable-sized delimiters}

\begin{symtable}{Variable-sized Delimiters}
\index{delimiters}
\index{delimiters>variable-sized}
\label{dels}
\renewcommand{\arraystretch}{1.75} 
\begin{tabular}{lll@{\qquad}lll@{\hspace*{1.5cm}}lll@{\qquad}lll}
  \N\downarrow & \N\Downarrow &               & \N[\magicrbrack]{. } \\
  \N\langle         & \N\rangle         & \Np[\vert][\magicvertname]|
                                                                          & \Np[\Vert][\magicVertname]\| \\
  \N\lceil            & \N\rceil             & \N\uparrow      & \N\Uparrow          \\
  \N\lfloor          & \N\rfloor           & \N\updownarrow  & \N\Updownarrow      \\
  \N(                  & \N)                   & \Np\{           & \Np\}               \\
  \N/                  & \N\backslash                                         \\
\end{tabular}

\bigskip
\begin{tablenote}
  When used with \cmd{\left} and \cmd{\right}, these symbols expand to
  the height of the enclosed math expression.  Note that \docAuxCommand{vert}
  is a synonym for \verb+|+\index{_=\magicvertname{} ($\vert$)}, and
  \docAuxCommand{Vert} is a synonym for \verb+\|+\index{_=\magicVertname{}
  ($\Vert$)}.

  $\varepsilon$-\TeX{}\index{e-tex=$\varepsilon$-\TeX} provides a
  \cmd{\middle} analogue to \cmd{\left} and \cmd{\right}.
  \cmd{\middle} can be used, for example, to make an internal ``$\vert$''
  expand to the height of the surrounding \cmd{\left} and \cmd{\right}
  symbols.  (This capability is commonly needed when typesetting
  adjacent bras\index{bra} and kets\index{ket} in Dirac\index{Dirac
  notation} notation: ``$\langle\phi\vert\psi\rangle$'').  A similar
  effect can be achieved in conventional \latex using the
  \pkgname{braket} package.
\end{tablenote}
\end{symtable}



\begin{symtable}[ST]{\ST\ Variable-sized Delimiters}
\index{delimiters}
\index{delimiters>variable-sized}
\index{semantic valuation}
\label{st-var-del}
\begin{tabular}{lll@{\qquad}lll}
\N\llbracket & \N\rrbracket
\end{tabular}
\end{symtable}

\begin{symtable}{\TC\ Text-mode Delimiters}
\index{delimiters}
\index{delimiters>text-mode}
\label{tc-delimiters}
\begin{tabular}{*2{ll}}
\K\textlangle    & \K\textrangle    \\
\K\textlbrackdbl & \K\textrbrackdbl \\
\K\textlquill    & \K\textrquill    \\
\end{tabular}
\end{symtable}




%%problematic skip for the moment
\section{Math-mode Accents}
%
%
%\begin{symtable}{Math-mode Accents}
%\index{accents}
%\index{accents>acute=acute (\blackacchack\')}   
%\index{accents>breve=breve (\blackacchack\u)}   
%\index{accents>caron=caron (\blackacchack\v)}   
%\index{accents>circumflex=circumflex (\blackacchack\^)}   
%\index{accents>diaeresis=di\ae{}resis (\blackacchack\")} 
%\index{accents>dot=dot (\blackacchack\. or \blackacc\d)}  
%\index{accents>grave=grave (\blackacchack\`)}   
%  
%\index{accents>ring=ring (\blackacchack\r)}     
%\index{tilde}
%\label{math-accents}
%\begin{tabular}{*4{ll}}
%\W\acute{a}    & \W\check{a}    & \W\grave{a}    & \W\tilde{a} \\
%\W\bar{a}      & \W\ddot{a}     & \W\hat{a}      & \W\vec{a}   \\
%\W\breve{a}    & \W\dot{a}      & \W\mathring{a}               \\
%\end{tabular}
%
%
%\bigskip
%
%\begin{tablenote}
%  \index{dotless i=dotless $i~(\imath)$>math mode}
%  \index{dotless j=dotless $j~(\jmath)$>math mode}
%  Also note the existence of \docAuxCommand{imath} and \docAuxCommand{jmath}, which
%  produce dotless versions of ``\textit{i}'' and ``\textit{j}''.  (See
%  \vref{ord}.)  These are useful when the accent is supposed to
%  replace the dot.  For example, ``\verb|\hat{\imath}|'' produces a
%  correct ``$\,\hat{\imath}\,$'', while ``\verb|\hat{i}|'' would yield
%  the rather odd-looking ``\,$\hat{i}\,$''.
%\end{tablenote}
%\end{symtable}
%
%
\begin{symtable}{AMS Math-mode Accents}
\index{accents}
\label{ams-math-accents}
\begin{tabular}{ll@{\hspace*{2em}}ll}
\W\dddot{a}    & \W\ddddot{a} \\
\end{tabular}

\bigskip

\begin{tablenote}
  These accents are also provided by the ABX and \pkgname{accents}
  packages and are redefined by the MDOTS package if the
  \pkgname{amsmath} and \pkgname{amssymb} packages have previously
  been loaded.  All of the variations except for the original AMS
  ones tighten the space between the dots%


\end{tablenote}
\end{symtable}

%
%
\subsection{Extensible Accents}
%
\begin{longsymtable}{Extensible Accents}
\index{accents}
\idxboth{extensible}{accents}
\idxboth{extensible}{arrows}
\index{underline}
\index{tilde}
\index{tilde>extensible}
\index{extensible tildes}
\index{symbols>extensible}
\index{accents>circumflex=circumflex (\blackacchack\^)}  
\label{extensible-accents}
\renewcommand{\arraystretch}{1.5}
\begin{longtable}{*4l}
\W\widetilde{abc}$^*$         & \W\widehat{abc}$^*$    \\
\W\overleftarrow{abc}$^\dag$  & \W\overrightarrow{abc}$^\dag$ \\
\W\overline{abc}              & \W\underline{abc}      \\
\W\overbrace{abc}             & \W\underbrace{abc}     \\[5pt]
\W\sqrt{abc}$^\ddag$                                   \\
\end{longtable}

\bigskip

\begin{tablenote}
  \def\longdivsign{%
    \ensuremath{\overline{\vphantom{)}%
      \hbox{\smash{\raise3.5\fontdimen8\textfont3\hbox{$)$}}}%
      abc}}}

  \index{long division|(}
  \index{division|(}
  \index{polynomial division|(}

  As demonstrated in a 1997 TUGboat\index{TUGboat} article about
  typesetting long-division problems~\cite{Gibbons:longdiv}, an
  extensible long-division sign (``\,\longdivsign\,'') can be faked by
  putting a ``\verb|\big)|'' in a \texttt{tabular} environment with an
  \verb|\hline| or \verb|\cline| in the preceding row.  The article
  also presents a piece of code (uploaded to CTAN as
  \texttt{longdiv.tex}%
  \index{longdiv=\textsf{longdiv} (package)}%
  \index{packages>\textsf{longdiv}}) that automatically solves and
  typesets---by putting an \docAuxCommand{overline} atop ``\verb|\big)|'' and
  the desired text---long-division problems.  Of course now we have
  a not so good unicode character for it \texttt{U+27cc} {{\pan3\char"27CC 123456}},
  which you can use with a font that supports it. 
  See also the
  \pkgname{polynom} package, which automatically solves and typesets
  polynomial-division problems in a similar manner.

  \index{long division|)}
  \index{division|)}
  \index{polynomial division|)}
\end{tablenote}

\bigskip

\begin{tablenote}[*]
  These symbols are made more extensible by the MNS package and even
  more extensible by the \pkgname{yhmath} package.
\end{tablenote}

\bigskip

\begin{tablenote}[\dag]
  If you're looking for an extensible \emph{diagonal} line or arrow to
  be used for canceling or reducing mathematical
  subexpressions\index{arrows>diagonal, for reducing subexpressions}
\ifhavecancel
  %(e.g.,~``$\cancel{x + -x}$'' or ``$\cancelto{5}{3+2}\quad$'')
\fi
  then consider using the \pkgname{cancel} package.
\end{tablenote}

\bigskip

\begin{tablenote}[\ddag]
  With an optional argument, \verb|\sqrt| typesets nth roots.  For
  example, ``\verb|\sqrt[3]{abc}|'' produces~``$\!\sqrt[3]{abc}$\,''
  and ``\verb|\sqrt[n]{abc}|'' produces~``$\!\sqrt[n]{abc}$\,''.
\end{tablenote}
\end{longsymtable}


The \pkgname{ymath} package provides some very wide and extensible accents, as well as the |\widetriangle{XYZ}| triangular hat. The latter is used in France to show that the notation $ABC$ where $A,B,C$ are three points means a triangle $\widetriangle{ABC}$ and not an 
angle $\wideparen{ABC}$ \citep{ymath}. 
\index{triangular hat accent}\index{wide triangle accent} 


\medskip
\bgroup
%\begin{longsymtable}[YH]{yhmath Extensible Accents}
\idxboth{extensible}{accents}
\index{symbols>extensible}
\index{accents>arc=arc (\blackacchack\newtie)} 
\label{yhmath-extensible-accents}
\renewcommand{\arraystretch}{1.5}
\begin{longtable}{*4l}
\W\wideparen{ABC}    & \W\widetriangle{ABC} \\[5pt]
\W\widering{ABC}     & \W\wideparen {ABC}      \\
%\W\widebar{ABC}
\end{longtable}
\captionof{table}{yhmath Extensible Accents}
\egroup
\medskip

Yiannis Haralambous stated that he called the |widering| 
because it plays the r\^ole of a wide
 symbol (and since the ring can't be widened, a parenthesis is used).
 
Here are some more examples from the documentation (the first one coded as |\ring{A}|):
 
\begin{texexample}{The ymath package} {ex:ymath}
 $$
 \ring{A},
 \widering{AB},
 \widering{ABC},
 \widering{ABCD},
 \widering{ABCDE},
 \widering{ABCDEF},
 \widering{ABCDEFG},
 $$
\end{texexample} 
 
%In this paper we give a Clifford bundle motivated approach to the wave equation of a free spin $1/2$ fermion in the de Sitter manifold, a brane with topology $M=\mathrm{S0}(4,1)/\mathrm{S0}(3,1)$ living in the bulk spacetime $\mathbb{R}^{4,1}=(\mathring{M}=\mathbb{R}^{5},\bm{\mathring{g}})$ and equipped with a metric field $\bm{g:=-i}^{\ast}\bm{\mathring{g}}$ with $\bm{i}:M\rightarrow\mathring{M}$ being the inclusion map. To obtain the analog of Dirac equation in Minkowski spacetime in the structure $\mathring{M}$ we appropriately factorize the two Casimir invariants $C_{1}$ and $C_{2}$ of the Lie algebra of the de Sitter group \ldots.

 \begin{gather}
 \begin{pmatrix} a & b\\ c & d\end{pmatrix}
 \begin{pmatrix} a & b & c\\ d & e & f\\ g & h & i\end{pmatrix}
 \begin{pmatrix} a & b & c & d\\ e & f & g & h\\ i & j & k & l\\
 m & n & o & p\end{pmatrix}
 \\
 \begin{pmatrix} a & b & c & d & e\\ f & g & h & i & j\\
 k & l & m & n & o\\ p & q & r & s & t\\ u & v & w & x & y\end{pmatrix}
 \begin{pmatrix} a & b & c & d & e & f \\ g & h & i & j & k & l \\
 m & n & o & p & q & r \\ s & t & u & v & w & x \\ y & z & \alpha &
 \beta & \gamma & \delta\end{pmatrix}
 \end{gather}

%A Kakeya set is a subset of ${\mathbb R}^d$ that contains a unit line segment in every direction. Let $\mathring S^{d-1}$ denote the unit sphere in ${\mathbb R}^d$ with antipodal points identified. We encode a Kakeya set in ${\mathbb R}^d$ as a bounded map $f:\mathring S^{d-1}\to{\mathbb R}^d$, where $f(x)$ gives the centre of the unit line segment orientated in the $x$ direction. Denoting by $B(\mathring S^{d-1})$ the collection of all such maps equipped with the supremum norm, we show that (i) for a dense set of $f$ the corresponding Kakeya set has positive Lebesgue measure and (ii) the set of those $f$ for which the corresponding Kakeya set has maximal upper box-counting (Minkowski) dimension $d$ is a residual subset of $B(\widering S^{d-1})$. We also give a very simple proof that the lower box-counting dimension of any Kakeya set is at least $d/2$.


\begin{symtable}[MTOOLS]{\MTOOLS\ Extensible Accents}
\idxboth{extensible}{accents}
\index{symbols>extensible}
\label{mathtools-extensible-accents}
\renewcommand{\arraystretch}{1.5}
\begin{tabular}{ll@{\qquad}ll}
\W[\MTOOLSoverbrace]\overbrace{abc}         & \W[\MTOOLSunderbrace]\underbrace{abc}         \\
\W[\MTOOLSoverbracket]\overbracket{abc}$^*$ & \W[\MTOOLSunderbracket]\underbracket{abc}$^*$ \\
\end{tabular}

\bigskip

\begin{tablenote}[*]
  \verb|\overbracket| and \verb|\underbracket| accept optional
  arguments that specify the bracket height and thickness.
  \seedocs{\MTOOLS}.
\end{tablenote}
\end{symtable}




\subsection{Extensible Arrows}

\begin{symtable}{AMS Extensible Arrows}
\index{arrows}
\idxboth{extensible}{arrows}
\index{symbols>extensible}
\label{ams-extensible-arrows}
\begin{tabular}{ll@{\qquad}ll}
\W\xleftarrow{abc} & \W\xrightarrow{abc} \\
\end{tabular}
\end{symtable}



\section{Dots}

%\begin{symtable}{Dots}
%\idxboth{dot}{symbols}
%\index{dots (ellipses)} \index{ellipses (dots)}
%\label{dots}
%\begin{tabular}{*{3}{ll@{\hspace*{1.5cm}}}ll}
%\X\cdotp & \X\colon$^*$    & \X\ldotp & \X\vdots$^\dag$ \\
%\X\cdots & \X\ddots$^\dag$ & \X\ldots                   \\
%\end{tabular}
%
%\bigskip
%
%\begin{tablenote}[*]
%  While ``\texttt{:}'' is valid in math mode, \cmd{\colon} uses
%  different surrounding spacing.  See \ref{math-spacing} and the
%  Short Math Guide for \latex~\cite{Downes:smg} for more information on
%  math-mode spacing.
%\end{tablenote}
%
%\bigskip
%
%\begin{tablenote}[\dag]
% \ifMDOTS
%    \let\mdcmdX=\cmdX
%  \else
%    \let\mdcmdX=\cmd
%  \fi
% The \MDOTS\ package redefines \docAuxCommand{ddots} and \docAuxCommand{vdots} to
%  make them scale properly with font size.  (They normally scale
%  horizontally but not vertically.)  \mdcmdX{\fixedddots} and
%  \mdcmdX{\fixedvdots} provide the original, fixed-height
%  functionality of \latexe's \docAuxCommand{ddots} and \docAuxCommand{vdots} macros.
%\end{tablenote}
%\end{symtable}
%
%
%
%\begin{symtable}{\AmS Dots}
%\idxboth{dot}{symbols}
%\index{dots (ellipses)} \index{ellipses (dots)}
%\label{ams-dots}
%\begin{tabular}{*{2}{ll@{\hspace*{1.5cm}}}ll}
%\X\because$^*$   & \X[\cdots]\dotsi & \X\therefore$^*$ \\
%\X[\cdots]\dotsb & \X[\cdots]\dotsm &                  \\
%\X[\ldots]\dotsc & \X[\ldots]\dotso &                  \\
%\end{tabular}
%
%\bigskip
%
%\begin{tablenote}[*]
%  \docAuxCommand{because} and \docAuxCommand{therefore} are defined as binary
%  relations and therefore also appear in \vref{ams-rel}.
%\end{tablenote}
%
%\bigskip
%
%\begin{tablenote}
%  The \AmS \verb*|\dots| symbols are named
%  according to their intended usage: \cmdI[$\string\cdots$]{\dotsb}
%  between pairs of binary operators/relations,
%  \cmdI[$\string\ldots$]{\dotsc} between pairs of commas,
%  \cmdI[$\string\cdots$]{\dotsi} between pairs of integrals,
%  \cmdI[$\string\cdots$]{\dotsm} between pairs of multiplication
%  signs, and \cmdI[$\string\ldots$]{\dotso} between other symbol
%  pairs.
%\end{tablenote}
%\end{symtable}
%


%\begin{symtable}{WASY Dots}
%\idxboth{dot}{symbols}
%\label{wasy-dots}
%\begin{tabular}{ll}
%\K\wasytherefore
%\end{tabular}
%\end{symtable}



\begin{symtable}{Miscellaneous \latexe{} Math Symbols}
\idxboth{miscellaneous}{symbols}
\index{card suits}
\index{diamonds (suit)}
\index{hearts (suit)}
\index{clubs (suit)}
\index{spades (suit)}
\idxboth{musical}{symbols}
\index{dots (ellipses)}
\index{ellipses (dots)}
\index{null set}
\index{dotless i=dotless $i~(\imath)$>math mode}
\index{dotless j=dotless $j~(\jmath)$>math mode}
\index{angles}
\label{ord}
\AMSfalse
\ifAMS
  \def\AMSfn{$^\ddag$}
\else
  \def\AMSfn{}
\fi
\begin{tabular}{*4{ll}}
\X\aleph          & \X\Diamond$^*$    & \X\infty   & \X\prime     \\
\X\angle          & \X\diamondsuit    & \X\mho$^*$ & \X\sharp     \\
\X\backslash      & \X\emptyset\AMSfn & \X\nabla   & \X\spadesuit \\
\X\Box$^{*,\dag}$ & \X\flat           & \X\natural & \X\surd      \\
\X\clubsuit       & \X\heartsuit      & \X\neg     & \X\triangle  \\
\end{tabular}

\bigskip
\begin{tablenote}[*]
  Not predefined in \latexe.  Use one of the packages
  \pkgname{latexsym}, \pkgname{amsfonts}, \pkgname{amssymb},
  \pkgname{txfonts}, \pkgname{pxfonts}, or \pkgname{wasysym}.  Note,
  however, that \pkgname{amsfonts} and \pkgname{amssymb} define
  \docAuxCommand{Diamond} to produce the same glyph as
  the other packages produce a squarer \docAuxCommand{Diamond} as depicted above.
\end{tablenote}

\bigskip
\begin{tablenote}[\dag]
  To use \docAuxCommand{Box}---or any other symbol---as an end-of-proof
  (Q.E.D\@.)\index{Q.E.D.}\index{end of proof}\index{proof, end of}
  marker, consider using the \pkgname{ntheorem} package, which
  properly juxtaposes a symbol with the end of the proof text.
\end{tablenote}
\end{symtable}



\subsection{Miscellaneous Text-mode Math Symbols}

\subsection{Biological Symbols}
\begin{symtable}[MARV]{\MARV\ Biological Symbols}
\idxboth{biological}{symbols}
\index{male}
\index{female}
\label{marv-bio}
\begin{tabular}{*3{ll}ll}
\K\Female        & \K\FemaleMale    & \K\MALE          & \K\Neutral       \\
\K\FEMALE        & \K\Hermaphrodite & \K\Male          \\
\K\FemaleFemale  & \K\HERMAPHRODITE & \K\MaleMale      \\
\end{tabular}
\end{symtable}

\begin{symtable}[WASY]{\WASY\ Biological Symbols}
\index{male}
\index{female}
\label{wasy-bio}
\begin{tabular}{*2{ll}}
\K\female & \K\male \\
\end{tabular}
\end{symtable}

\begin{symtable}[MARV]{\MARV\ Safety-related Symbols}
\idxboth{safety-related}{symbols}
\label{marv-safety}
\begin{tabular}{*3{ll}ll}
\K\Biohazard     & \K\CEsign        & \K\Explosionsafe & \K\Radioactivity \\
\K\BSEfree       & \K\Estatically   & \indexlinearb\Laserbeam     & \K\Stopsign      \\
\end{tabular}
\end{symtable}

\idxbothend{scientific}{symbols}
\idxbothend{technological}{symbols}


\section{Dingbats}
\idxbothbegin{dingbat}{symbols}

Dingbats are symbols such as stars, arrows, and geometric shapes.
They are commonly used as bullets in itemized lists or, more
generally, as a means to draw attention to the text that follows.

The \PI\ dingbat package warrants special mention.  Among other
capabilities, \PI\ provides a \latex\ interface to the \PSfont{Zapf
Dingbats} font (one of the standard~35 \postscript\index{PostScript
fonts} fonts).  However, rather than name each of the dingbats
individually, \PI\ merely provides a single \cmd{\ding} command, which
outputs the character that lies at a given position in the font.  The
consequence is that the \PI\ symbols can't be listed by name in this
document's index, so be mindful of that fact when searching for a
particular symbol.

\bigskip


\begin{symtable}[DING]{\DING\ Arrows}
\label{bbding-arrows}
\begin{tabular}{*3{ll}}
\K\ArrowBoldDownRight    & \K\ArrowBoldRightShort  & \K\ArrowBoldUpRight \\
\K\ArrowBoldRightCircled & \K\ArrowBoldRightStrobe \\
\end{tabular}
\end{symtable}


\begin{symtable}[PI]{\PI\ Arrows}
\index{arrows}
\idxboth{fletched}{arrows}
\label{pi-arrows}
\begin{tabular}{*5{ll}}
\indexDing{212} & \indexDing{221} & \indexDing{230} & \indexDing{239} & \indexDing{249} \\
\indexDing{213} & \indexDing{222} & \indexDing{231} & \indexDing{241} & \indexDing{250} \\
\indexDing{214} & \indexDing{223} & \indexDing{232} & \indexDing{242} & \indexDing{251} \\
\indexDing{215} & \indexDing{224} & \indexDing{233} & \indexDing{243} & \indexDing{252} \\
\indexDing{216} & \indexDing{225} & \indexDing{234} & \indexDing{244} & \indexDing{253} \\
\indexDing{217} & \indexDing{226} & \indexDing{235} & \indexDing{245} & \indexDing{254} \\
\indexDing{218} & \indexDing{227} & \indexDing{236} & \indexDing{246} \\
\indexDing{219} & \indexDing{228} & \indexDing{237} & \indexDing{247} \\
\indexDing{220} & \indexDing{229} & \indexDing{238} & \indexDing{248} \\
\end{tabular}
\end{symtable}



\begin{symtable}[MARV]{\MARV\ Scissors}
\index{scissors}
\label{marv-scissors}
\begin{tabular}{*3{ll}}
\K\Cutleft       & \K\Cutright      & \indexlinearb\Leftscissors  \\
\K\Cutline       & \K\Kutline       & \K\Rightscissors \\
\end{tabular}
\end{symtable}


\begin{symtable}[DING]{\DING\ Scissors}
\index{scissors}
\label{scissors}
\begin{tabular}{*2{ll}}
\K\ScissorHollowLeft        & \K\ScissorLeftBrokenTop     \\
\K\ScissorHollowRight       & \K\ScissorRight             \\
\K\ScissorLeft              & \K\ScissorRightBrokenBottom \\
\K\ScissorLeftBrokenBottom  & \K\ScissorRightBrokenTop    \\
\end{tabular}
\end{symtable}


\begin{symtable}[PI]{\PI\ Scissors}
\index{scissors}
\label{pi-scissors}
\begin{tabular}{*4{ll}}
\indexDing{33} & \indexDing{34} & \indexDing{35} & \indexDing{36} \\
\end{tabular}
\end{symtable}

\begin{symtable}[DING]{\DING\ Pencils and Nibs}
\index{pencils}
\index{nibs}
\label{pencils-nibs}
\begin{tabular}{*3{ll}}
\K\NibLeft         & \K\PencilLeft      & \K\PencilRightDown \\
\K\NibRight        & \K\PencilLeftDown  & \K\PencilRightUp   \\
\K\NibSolidLeft    & \K\PencilLeftUp    \\
\K\NibSolidRight   & \K\PencilRight     \\
\end{tabular}
\end{symtable}


\begin{symtable}[PI]{\PI\ Pencils and Nibs}
\index{pencils}
\index{nibs}
\label{pi-pencils}
\begin{tabular}{*5{ll}}
\indexDing{46} & \indexDing{47} & \indexDing{48} & \indexDing{49} & \indexDing{50} \\
\end{tabular}
\end{symtable}

\begin{symtable}[DING]{\DING\ Fists}
\index{fists}
\label{hands}
\begin{tabular}{*3{ll}}
\K\HandCuffLeft    & \K\HandCuffRightUp & \K\HandPencilLeft  \\
\K\HandCuffLeftUp  & \K\HandLeft        & \K\HandRight       \\
\K\HandCuffRight   & \K\HandLeftUp      & \K\HandRightUp     \\
\end{tabular}
\end{symtable}


\begin{symtable}[PI]{\PI\ Fists}
\index{fists}
\label{pi-hands}
\begin{tabular}{*4{ll}}
\indexDing{42} & \indexDing{43} & \indexDing{44} & \indexDing{45} \\
\end{tabular}
\end{symtable}

\begin{symtable}[DING]{\DING\ Crosses and Plusses}
\index{crosses}
\index{plusses}
\index{crucifixes}
\label{crosses-plusses}
\begin{tabular}{*3{ll}}
\K[\dingCross]\Cross  & \K\CrossOpenShadow    & \K\PlusOutline        \\
\K\CrossBoldOutline   & \K\CrossOutline       & \K\PlusThinCenterOpen \\
\K\CrossClowerTips    & \K\Plus               \\
\K\CrossMaltese       & \K\PlusCenterOpen     \\
\end{tabular}
\end{symtable}


\begin{symtable}[PI]{\PI\ Crosses and Plusses}
\index{symbols>crosses}
\index{symbols>plusses}
\index{symbols>crucifixes}
\label{pi-crosses-plusses}
\begin{tabular}{*4{ll}}
\indexDing{57} & \indexDing{59} & \indexDing{61} & \indexDing{63} \\
\indexDing{58} & \indexDing{60} & \indexDing{62} & \indexDing{64} \\
\end{tabular}
\end{symtable}


\begin{symtable}[DING]{\DING\ Xs and Check Marks}
\index{symbols>check marks}
\index{symbols>Xs}
\label{ding-check-marks}
\begin{tabular}{*3{ll}}
\K\Checkmark     & \K\XSolid        & \K\XSolidBrush   \\
\K\CheckmarkBold & \K\XSolidBold    \\
\end{tabular}
\end{symtable}


\begin{symtable}[PI]{\PI\ Xs and Check Marks}
\index{check marks}
\index{Xs}
\label{pi-check-marks}
\begin{tabular}{*3{ll}}
\indexDing{51} & \indexDing{53} & \indexDing{55} \\
\indexDing{52} & \indexDing{54} & \indexDing{56} \\
\end{tabular}
\end{symtable}


\begin{symtable}[WASY]{\WASY\ Xs and Check Marks}
\index{check marks}
\index{Xs}
\label{wasy-check-marks}
\begin{tabular}{*6l}
\K\CheckedBox & \K\Square & \K\XBox \\
\end{tabular}
\end{symtable}


\begin{symtable}[PI]{\PI\ Circled Numbers}
\index{circled numbers}
\index{numbers>circled}
\label{circled-numbers}
\begin{tabular}{*4{ll}}
\indexDing{172} & \indexDing{182} & \indexDing{192} & \indexDing{202} \\
\indexDing{173} & \indexDing{183} & \indexDing{193} & \indexDing{203} \\
\indexDing{174} & \indexDing{184} & \indexDing{194} & \indexDing{204} \\
\indexDing{175} & \indexDing{185} & \indexDing{195} & \indexDing{205} \\
\indexDing{176} & \indexDing{186} & \indexDing{196} & \indexDing{206} \\
\indexDing{177} & \indexDing{187} & \indexDing{197} & \indexDing{207} \\
\indexDing{178} & \indexDing{188} & \indexDing{198} & \indexDing{208} \\
\indexDing{179} & \indexDing{189} & \indexDing{199} & \indexDing{209} \\
\indexDing{180} & \indexDing{190} & \indexDing{200} & \indexDing{210} \\
\indexDing{181} & \indexDing{191} & \indexDing{201} & \indexDing{211} \\
\end{tabular}

\bigskip

\begin{tablenote}
  \PI\ (part of the \pkgname{psnfss} package) provides a
  \cmd{dingautolist} environment which resembles \texttt{enumerate}
  but uses circled numbers as bullets.\footnotemark{}
  \seedocs{\pkgname{psnfss}}.
\end{tablenote}
\end{symtable}
\footnotetext{In fact, \cmd{\dingautolist} can use any set of
  consecutive \PSfont{Zapf Dingbats} symbols.}


\begin{symtable}[WASY]{\WASY\ Stars}
\index{stars}
\index{Jewish star}\index{Star of David}
\label{wasy-stars}
\begin{tabular}{*6l}
\K\davidsstar & \K\hexstar & \K\varhexstar
\end{tabular}
\end{symtable}


\begin{symtable}[DING]{\DING\ Stars, Flowers, and Similar Shapes}
\index{asterisks}
\index{clovers}
\index{flowers}
\index{ornaments}
\index{sparkles}
\index{snowflakes}
\index{stars}
\index{Jewish star}\index{Star of David}
\label{star-like}
\begin{tabular}{*3{ll}}
\K\Asterisk                & \K\FiveFlowerPetal      & \K\JackStar                  \\
\K\AsteriskBold            & \K\FiveStar             & \K\JackStarBold              \\
\K\AsteriskCenterOpen      & \K\FiveStarCenterOpen   & \K\SixFlowerAlternate        \\
\K\AsteriskRoundedEnds     & \K\FiveStarConvex       & \K\SixFlowerAltPetal         \\
\K\AsteriskThin            & \K\FiveStarLines        & \K\SixFlowerOpenCenter       \\
\K\AsteriskThinCenterOpen  & \K\FiveStarOpen         & \K\SixFlowerPetalDotted      \\
\K\DavidStar               & \K\FiveStarOpenCircled  & \K\SixFlowerPetalRemoved     \\
\K\DavidStarSolid          & \K\FiveStarOpenDotted   & \K\SixFlowerRemovedOpenPetal \\
\K\EightAsterisk           & \K\FiveStarOutline      & \K\SixStar                   \\
\K\EightFlowerPetal        & \K\FiveStarOutlineHeavy & \K\SixteenStarLight          \\
\K\EightFlowerPetalRemoved & \K\FiveStarShadow       & \K\Snowflake                 \\
\K\EightStar               & \K\FourAsterisk         & \K\SnowflakeChevron          \\
\K\EightStarBold           & \K\FourClowerOpen       & \K\SnowflakeChevronBold      \\
\K\EightStarConvex         & \K\FourClowerSolid      & \K\Sparkle                   \\
\K\EightStarTaper          & \K\FourStar             & \K\SparkleBold               \\
\K\FiveFlowerOpen          & \K\FourStarOpen         & \K\TwelweStar                \\
\end{tabular}
\end{symtable}

\begin{symtable}[WASY]{\WASY\ Geometric Shapes}
\index{polygons}
\index{geometric shapes}
\label{wasy-geometrical}
\begin{tabular}{*8l}
\K\hexagon & \K\octagon & \K\pentagon & \K\varhexagon
\end{tabular}
\end{symtable}

\begin{symtable}[DING]{\DING\ Geometric Shapes}
\index{circles}
\index{diamonds}
\index{ellipses (ovals)}
\index{geometric shapes}
\index{ovals}
\index{rectangles}
\index{squares}
\index{triangles}
\label{ding-geometrical}
\begin{tabular}{*3{ll}}
\K\CircleShadow    & \K\Rectangle                   & \K\SquareShadowTopLeft     \\
\K\CircleSolid     & \K\RectangleBold               & \K\SquareShadowTopRight    \\
\K\DiamondSolid    & \K\RectangleThin               & \K\SquareSolid             \\
\K\Ellipse         & \K[\dingSquare]\Square         & \K\TriangleDown            \\
\K\EllipseShadow   & \K\SquareCastShadowBottomRight & \K\TriangleUp              \\
\K\EllipseSolid    & \K\SquareCastShadowTopLeft     \\
\K\HalfCircleLeft  & \K\SquareCastShadowTopRight    \\
\K\HalfCircleRight & \K\SquareShadowBottomRight     \\
\end{tabular}
\end{symtable}


\begin{symtable}[PI]{\PI\ Geometric Shapes}
\index{circles}
\index{diamonds}
\index{geometric shapes}
\index{rectangles}
\index{squares}
\index{triangles}
\label{pi-geometrical}
\begin{tabular}{*5{ll}}
\indexDing{108} & \indexDing{111} & \indexDing{114} & \indexDing{117} & \indexDing{121} \\
\indexDing{109} & \indexDing{112} & \indexDing{115} & \indexDing{119} & \indexDing{122} \\
\indexDing{110} & \indexDing{113} & \indexDing{116} & \indexDing{120} \\
\end{tabular}
\end{symtable}\begin{symtable}[DING]{Miscellaneous \DING\ Dingbats}
\idxboth{miscellaneous}{symbols}
\index{envelopes}
\label{bbding-misc}
\begin{tabular}{*4{ll}}
\K\Envelope             & \K\Peace & \K\PhoneHandset & \K\SunshineOpenCircled \\
\K\OrnamentDiamondSolid & \K\Phone & \K\Plane        & \K\Tape                \\
\end{tabular}
\end{symtable}


\begin{symtable}[PI]{Miscellaneous \PI\ Dingbats}
\idxboth{miscellaneous}{symbols}
\index{card suits}
\index{diamonds (suit)}
\index{hearts (suit)}
\index{clubs (suit)}
\index{spades (suit)}
\index{fleurons}
\index{leaves}
\index{ornaments}
\label{pi-misc}
\begin{tabular}{*5{ll}}
\indexDing{37} & \indexDing{40}  & \indexDing{164} & \indexDing{167} & \indexDing{171} \\
\indexDing{38} & \indexDing{41}  & \indexDing{165} & \indexDing{168} & \indexDing{169} \\
\indexDing{39} & \indexDing{118} & \indexDing{166} & \indexDing{170} \\
\end{tabular}
\end{symtable}
\idxbothend{dingbat}{symbols}

\begin{symtable}{\TC\ Genealogical Symbols}
\idxboth{genealogical}{symbols}
\label{genealogical}
\begin{tabular}{*3{ll}}
\K\textborn     & \K\textdivorced & \K\textmarried  \\
\K\textdied     & \K\textleaf     \\
\end{tabular}
\end{symtable}


\begin{symtable}[WASY]{\WASY\ General Symbols}
\index{symbols>general}
\index{smiley faces}
\index{frowny faces}
\index{faces}
\idxboth{clock}{symbols}
\index{check marks}
\label{wasy-general}
\begin{tabular}{*4{ll}}
\K\ataribox    & \K[\WASYclock]\clock & \indexlinearb\LEFTarrow  & \K\smiley      \\
\K\bell        & \K\diameter          & \K\lightning  & \K\sun         \\
\K\blacksmiley & \K\DOWNarrow         & \K\phone      & \K\UParrow     \\
\K\Bowtie      & \K\frownie           & \K\pointer    & \K\wasylozenge \\
\K\brokenvert  & \K\invdiameter       & \K\recorder                    \\
\K\checked     & \K\kreuz             & \K\RIGHTarrow                  \\
\end{tabular}
\end{symtable}


\begin{symtable}[WASY]{\WASY\ Circles}
\index{circles}
\label{wasy-circles}
\begin{tabular}{*8l}
\K\CIRCLE         & \indexlinearb\LEFTcircle     & \K\RIGHTcircle    & \K\rightturn      \\
\K\Circle         & \indexlinearb\Leftcircle     & \K\Rightcircle    \\
\indexlinearb\LEFTCIRCLE     & \K\RIGHTCIRCLE    & \K\leftturn       \\
\end{tabular}
\end{symtable}


\begin{symtable}[WASY]{\WASY\ Musical Symbols}
\idxboth{musical}{symbols}
\label{wasy-music}
\begin{tabular}{*{10}l}
\K\eighthnote & \K\halfnote    & \K\twonotes &
\K\fullnote   & \K\quarternote \\
\end{tabular}

\bigskip
\begin{tablenote}
  See also \docAuxCommand{flat}, \docAuxCommand{sharp}, and \docAuxCommand{natural}
  (\vref*{ord}).
\end{tablenote}
\end{symtable}

\begin{symtable}[MARV]{\MARV\ Navigation Symbols}
\idxboth{navigation}{symbols}
\label{marv-navigation}
\begin{tabular}{*3{ll}ll}
\K\Forward        & \K\MoveDown  & \K\RewindToIndex  & \K\ToTop \\
\K\ForwardToEnd   & \K\MoveUp    & \K\RewindToStart  \\
\K\ForwardToIndex & \K\Rewind    & \K\ToBottom       \\
\end{tabular}
\end{symtable}


\begin{symtable}[MARV]{\MARV\ Laundry Symbols}
\idxboth{laundry}{symbols}
\label{marv-laundry}
\begin{tabular}{*3{ll}}
\K\AtForty            & \K\Handwash           & \K\ShortNinetyFive    \\
\K\AtNinetyFive       & \K\IroningI           & \K\ShortSixty         \\
\K\AtSixty            & \K\IroningII          & \K\ShortThirty        \\
\K\Bleech             & \K\IroningIII         & \K\SpecialForty       \\
\K\CleaningA          & \K\NoBleech           & \K\Tumbler            \\
\K\CleaningF          & \K\NoChemicalCleaning & \K\WashCotton         \\
\K\CleaningFF         & \K\NoIroning          & \K\WashSynthetics     \\
\K\CleaningP          & \K\NoTumbler          & \K\WashWool           \\
\K\CleaningPP         & \K\ShortFifty         \\
\K\Dontwash           & \K\ShortForty         \\
\end{tabular}
\end{symtable}


\begin{symtable}[MARV]{\MARV\ Information Symbols}
\idxboth{information}{symbols}
\index{check marks}
\index{Xs}
\idxboth{clock}{symbols}
\label{marv-info}
\begin{tabular}{*3{ll}ll}
\K\Bicycle      & \K\Football     & \K\Pointinghand \\
\K\Checkedbox   & \K\Gentsroom    & \K\Wheelchair   \\
\K\Clocklogo    & \K\Industry     & \K\Writinghand  \\
\K\Coffeecup    & \K\Info         \\
\K\Crossedbox   & \indexlinearb\Ladiesroom   \\
\end{tabular}
\end{symtable}


\begin{symtable}[MARV]{Other \MARV\ Symbols}
\idxboth{miscellaneous}{symbols}
\index{crosses}
\index{crucifixes}
\index{smiley faces}
\index{frowny faces}
\index{faces}
\index{man}
\index{woman}
\index{globe}
\index{world}
\label{marv-other}
\begin{tabular}{*4{ll}}
\K\Ankh        & \K\Cross        & \K\Heart       & \K\Smiley      \\
\K\Bat         & \K\FHBOlogo     & \K\MartinVogel & \K\Womanface   \\
\K\Bouquet     & \K\FHBOLOGO     & \K\Mundus      & \K\Yinyang     \\
\K\Celtcross   & \K\Frowny       & \K\MVAt                         \\
\K\CircledA    & \K\FullFHBO     & \K\MVRightarrow                 \\
\end{tabular}
\end{symtable}

\section{Alphabets}

\begin{symtable}[CYPR]{\CYPR\ Cypriot Letters}
\index{Cypriot}
\index{alphabets>Cypriot}
\label{cypriot}
\begin{tabular}{*5{ll@{\qquad}}ll}
\Kcyp[{\Ca}]\Ca   & \Kcyp[{\Cku}]\Cku & \Kcyp[{\Cmu}]\Cmu & \Kcyp[{\Cpo}]\Cpo & \Kcyp[{\Cso}]\Cso & \Kcyp[{\Cwi}]\Cwi \\
\Kcyp[{\Ce}]\Ce   & \Kcyp[{\Cla}]\Cla & \Kcyp[{\Cna}]\Cna & \Kcyp[{\Cpu}]\Cpu & \Kcyp[{\Csu}]\Csu & \Kcyp[{\Cwo}]\Cwo \\
\Kcyp[{\Cga}]\Cga & \Kcyp[{\Cle}]\Cle & \Kcyp[{\Cne}]\Cne & \Kcyp[{\Cra}]\Cra & \Kcyp[{\Cta}]\Cta & \Kcyp[{\Cxa}]\Cxa \\
\Kcyp[{\Ci}]\Ci   & \Kcyp[{\Cli}]\Cli & \Kcyp[{\Cni}]\Cni & \Kcyp[{\Cre}]\Cre & \Kcyp[{\Cte}]\Cte & \Kcyp[{\Cxe}]\Cxe \\
\Kcyp[{\Cja}]\Cja & \Kcyp[{\Clo}]\Clo & \Kcyp[{\Cno}]\Cno & \Kcyp[{\Cri}]\Cri & \Kcyp[{\Cti}]\Cti & \Kcyp[{\Cya}]\Cya \\
\Kcyp[{\Cjo}]\Cjo & \Kcyp[{\Clu}]\Clu & \Kcyp[{\Cnu}]\Cnu & \Kcyp[{\Cro}]\Cro & \Kcyp[{\Cto}]\Cto & \Kcyp[{\Cyo}]\Cyo \\
\Kcyp[{\Cka}]\Cka & \Kcyp[{\Cma}]\Cma & \Kcyp[{\Co}]\Co   & \Kcyp[{\Cru}]\Cru & \Kcyp[{\Ctu}]\Ctu & \Kcyp[{\Cza}]\Cza \\
\Kcyp[{\Cke}]\Cke & \Kcyp[{\Cme}]\Cme & \Kcyp[{\Cpa}]\Cpa & \Kcyp[{\Csa}]\Csa & \Kcyp[{\Cu}]\Cu   & \Kcyp[{\Czo}]\Czo \\
\Kcyp[{\Cki}]\Cki & \Kcyp[{\Cmi}]\Cmi & \Kcyp[{\Cpe}]\Cpe & \Kcyp[{\Cse}]\Cse & \Kcyp[{\Cwa}]\Cwa &                         \\
\Kcyp[{\Cko}]\Cko & \Kcyp[{\Cmo}]\Cmo & \Kcyp[{\Cpi}]\Cpi & \Kcyp[{\Csi}]\Csi & \Kcyp[{\Cwe}]\Cwe &                         \\
\end{tabular}

\bigskip
\begin{tablenote}
  \usefontcmdmessage{}{\cyprfamily}.  Single-character
  shortcuts are also supported: Both
  ``\verb+{\Cpa\Cki\Cna}+'' and ``\verb+{pcn}+''
  produce ``{pcn}'', for example.  \seedocs{\CYPR}.
\end{tablenote}
\end{symtable}


\begin{symtable}[PRSN]{\PRSN\ Cuneiform Letters}
\index{cuneiform}
\index{alphabets>Old Persian (cuneiform)}
\label{oldprsn}
\begin{tabular}{*4{ll@{\qquad}}ll}
\indexoldpersian[\textcopsn{\Oa}]\Oa     & \indexoldpersian[\textcopsn{\Oga}]\Oga   & \indexoldpersian[\textcopsn{\Ola}]\Ola   & \indexoldpersian[\textcopsn{\Oru}]\Oru   & \indexoldpersian[\textcopsn{\Ovi}]\Ovi   \\
\indexoldpersian[\textcopsn{\Oba}]\Oba   & \indexoldpersian[\textcopsn{\Ogu}]\Ogu   & \indexoldpersian[\textcopsn{\Oma}]\Oma   & \indexoldpersian[\textcopsn{\Osa}]\Osa   & \indexoldpersian[\textcopsn{\Oxa}]\Oxa   \\
\indexoldpersian[\textcopsn{\Oca}]\Oca   & \indexoldpersian[\textcopsn{\Oha}]\Oha   & \indexoldpersian[\textcopsn{\Omi}]\Omi   & \indexoldpersian[\textcopsn{\Osva}]\Osva & \indexoldpersian[\textcopsn{\Oya}]\Oya   \\
\indexoldpersian[\textcopsn{\Occa}]\Occa & \indexoldpersian[\textcopsn{\Oi}]\Oi     & \indexoldpersian[\textcopsn{\Omu}]\Omu   & \indexoldpersian[\textcopsn{\Ota}]\Ota   & \indexoldpersian[\textcopsn{\Oza}]\Oza   \\
\indexoldpersian[\textcopsn{\Oda}]\Oda   & \indexoldpersian[\textcopsn{\Oja}]\Oja   & \indexoldpersian[\textcopsn{\Ona}]\Ona   & \indexoldpersian[\textcopsn{\Otha}]\Otha &                            \\
\indexoldpersian[\textcopsn{\Odi}]\Odi   & \indexoldpersian[\textcopsn{\Oji}]\Oji   & \indexoldpersian[\textcopsn{\Onu}]\Onu   & \indexoldpersian[\textcopsn{\Otu}]\Otu   &                            \\
\indexoldpersian[\textcopsn{\Odu}]\Odu   & \indexoldpersian[\textcopsn{\Oka}]\Oka   & \indexoldpersian[\textcopsn{\Opa}]\Opa   & \indexoldpersian[\textcopsn{\Ou}]\Ou     &                            \\
\indexoldpersian[\textcopsn{\Ofa}]\Ofa   & \indexoldpersian[\textcopsn{\Oku}]\Oku   & \indexoldpersian[\textcopsn{\Ora}]\Ora   & \indexoldpersian[\textcopsn{\Ova}]\Ova   &                            \\
\end{tabular}

\bigskip
\begin{tablenote}
  \usefontcmdmessage{\textcopsn}{\copsnfamily}.  Single-character
  shortcuts are also supported: Both
  ``\verb+\textcopsn{\Opa\Oka\Ona}+'' and ``\verb+\textcopsn{pkn}+''
  produce ``\textcopsn{pkn}'', for example.  \seedocs{\PRSN}.
\end{tablenote}
\end{symtable}


\begin{symtable}[PRSN]{\PRSN\ Cuneiform Numerals}
\index{cuneiform}
\index{numerals>cuneiform}
\label{oldprsn-nums}
\begin{tabular}{*4{ll@{\qquad}}ll}
\indexoldpersian[\textcopsn{\Oone}]\Oone & \indexoldpersian[\textcopsn{\Otwo}]\Otwo & \indexoldpersian[\textcopsn{\Oten}]\Oten & \indexoldpersian[\textcopsn{\Otwenty}]\Otwenty & \indexoldpersian[\textcopsn{\Ohundred}]\Ohundred \\
\end{tabular}

\bigskip
\begin{tablenote}
  \usefontcmdmessage{\textcopsn}{\copsnfamily}.
\end{tablenote}
\end{symtable}


\begin{symtable}[PRSN]{\PRSN\ Cuneiform Words}
\index{cuneiform}
\label{oldprsn-objs}
\begin{tabular}{*3{ll@{\qquad}}ll}
\indexoldpersian[\textcopsn{\OAura}]\OAura         & \indexoldpersian[\textcopsn{\Ocountrya}]\Ocountrya & \indexoldpersian[\textcopsn{\Ogod}]\Ogod           &                                      \\
\indexoldpersian[\textcopsn{\OAurb}]\OAurb         & \indexoldpersian[\textcopsn{\Ocountryb}]\Ocountryb & \indexoldpersian[\textcopsn{\Oking}]\Oking         &                                      \\
\indexoldpersian[\textcopsn{\OAurc}]\OAurc         & \indexoldpersian[\textcopsn{\Oearth}]\Oearth       & \indexoldpersian[\textcopsn{\Owd}]\Owd             &                                      \\
\end{tabular}

\bigskip
\begin{tablenote}
  \usefontcmdmessage{\textcopsn}{\copsnfamily}.
\end{tablenote}
\end{symtable}

\subsection{Ugaritic}

\begin{symtable}[UGAR]{\UGAR\ Cuneiform Letters}
\index{cuneiform}
\index{alphabets>Ugarite (cuneiform)}
\label{ugarite}
\begin{tabular}{*4{ll@{\qquad}}ll}
\indexugar[\textcugar{\Arq}]\Arq & \indexugar[\textcugar{\Az}]\Az   & \indexugar[\textcugar{\Am}]\Am   & \indexugar[\textcugar{\Asd}]\Asd & \indexugar[\textcugar{\Au}]\Au   \\
\indexugar[\textcugar{\Ab}]\Ab   & \indexugar[\textcugar{\Ahd}]\Ahd & \indexugar[\textcugar{\Adb}]\Adb & \indexugar[\textcugar{\Aq}]\Aq   & \indexugar[\textcugar{\Asg}]\Asg \\
\indexugar[\textcugar{\Ag}]\Ag   & \indexugar[\textcugar{\Atd}]\Atd & \indexugar[\textcugar{\An}]\An   & \indexugar[\textcugar{\Ar}]\Ar   & \indexugar[\textcugar{\Awd}]\Awd \\
\indexugar[\textcugar{\Ahu}]\Ahu & \indexugar[\textcugar{\Ay}]\Ay   & \indexugar[\textcugar{\Azd}]\Azd & \indexugar[\textcugar{\Atb}]\Atb &                          \\
\indexugar[\textcugar{\Ad}]\Ad   & \indexugar[\textcugar{\Ak}]\Ak   & \indexugar[\textcugar{\As}]\As   & \indexugar[\textcugar{\Agd}]\Agd &                          \\
\indexugar[\textcugar{\Ah}]\Ah   & \indexugar[\textcugar{\Asa}]\Asa & \indexugar[\textcugar{\Alq}]\Alq & \indexugar[\textcugar{\At}]\At   &                          \\
\indexugar[\textcugar{\Aw}]\Aw   & \indexugar[\textcugar{\Al}]\Al   & \indexugar[\textcugar{\Ap}]\Ap   & \indexugar[\textcugar{\Ai}]\Ai   &                          \\
\end{tabular}

\bigskip
\begin{tablenote}
  \usefontcmdmessage{\textcugar}{\cugarfamily}.  Single-character
  shortcuts and various aliases are also supported:
  ``\verb+\textcopsn{\Ap\Aq\An}+'',
  ``\verb+\textcopsn{\Ape\Aqoph\Anun}+'', and
  ``\verb+\textcopsn{pqn}+'' all produce ``\textcopsn{pqn}'', for
  example.  \seedocs{\UGAR}.
\end{tablenote}
\end{symtable}


\begin{longsymtable}[SARAB]{\SARAB\ South Arabian Letters}
\index{South Arabian alphabet}
\index{alphabets>South Arabian}
\label{sarabian}
\begin{longtable}{*4{ll@{\qquad}}ll}
\indexsoutharabian[\textsarab{\SAa}]\SAa   & \indexsoutharabian[\textsarab{\SAz}]\SAz   & \indexsoutharabian[\textsarab{\SAm}]\SAm   & \indexsoutharabian[\textsarab{\SAsd}]\SAsd & \indexsoutharabian[\textsarab{\SAdb}]\SAdb \\
\indexsoutharabian[\textsarab{\SAb}]\SAb   & \indexsoutharabian[\textsarab{\SAhd}]\SAhd & \indexsoutharabian[\textsarab{\SAn}]\SAn   & \indexsoutharabian[\textsarab{\SAq}]\SAq   & \indexsoutharabian[\textsarab{\SAtb}]\SAtb \\
\indexsoutharabian[\textsarab{\SAg}]\SAg   & \indexsoutharabian[\textsarab{\SAtd}]\SAtd & \indexsoutharabian[\textsarab{\SAs}]\SAs   & \indexsoutharabian[\textsarab{\SAr}]\SAr   & \indexsoutharabian[\textsarab{\SAga}]\SAga \\
\indexsoutharabian[\textsarab{\SAd}]\SAd   & \indexsoutharabian[\textsarab{\SAy}]\SAy   & \indexsoutharabian[\textsarab{\SAf}]\SAf   & \indexsoutharabian[\textsarab{\SAsv}]\SAsv & \indexsoutharabian[\textsarab{\SAzd}]\SAzd \\
\indexsoutharabian[\textsarab{\SAh}]\SAh   & \indexsoutharabian[\textsarab{\SAk}]\SAk   & \indexsoutharabian[\textsarab{\SAlq}]\SAlq & \indexsoutharabian[\textsarab{\SAt}]\SAt   & \indexsoutharabian[\textsarab{\SAsa}]\SAsa \\
\indexsoutharabian[\textsarab{\SAw}]\SAw   & \indexsoutharabian[\textsarab{\SAl}]\SAl   & \indexsoutharabian[\textsarab{\SAo}]\SAo   & \indexsoutharabian[\textsarab{\SAhu}]\SAhu & \indexsoutharabian[\textsarab{\SAdd}]\SAdd \\
\end{longtable}

\bigskip
\begin{tablenote}
  \usefontcmdmessage{\textsarab}{\sarabfamily}.  Single-character
  shortcuts are also supported: Both
  ``\verb+\textsarab{\SAb\SAk\SAn}+'' and ``\verb+\textsarab{bkn}+''
  produce ``\textsarab{bkn}'', for example.  \seedocs{\SARAB}.
\end{tablenote}
\end{longsymtable}

\begin{longsymtable}[LINA]{\LINA\ Linear~A Script}
\index{Linear A}
\index{alphabets>Linear A}
\label{linearA}
\begin{longtable}{*3{ll@{\quad}}ll}
\multicolumn{8}{l}{\small\textit{(continued from previous page)}} \\[1ex]
\endhead
\endfirsthead
\\[3ex]
\multicolumn{8}{r}{\small\textit{(continued on next page)}}
\endfoot
\endlastfoot
\indexlinearb\LinearAI           & \indexlinearb\LinearAXCIX        & \indexlinearb\LinearACXCVII      & \indexlinearb\LinearACCXCV       \\
\indexlinearb\LinearAII          & \indexlinearb\LinearAC           & \indexlinearb\LinearACXCVIII     & \indexlinearb\LinearACCXCVI      \\
\indexlinearb\LinearAIII         & \indexlinearb\LinearACI          & \indexlinearb\LinearACXCIX       & \indexlinearb\LinearACCXCVII     \\
\indexlinearb\LinearAIV          & \indexlinearb\LinearACII         & \indexlinearb\LinearACC          & \indexlinearb\LinearACCXCVIII    \\
\indexlinearb\LinearAV           & \indexlinearb\LinearACIII        & \indexlinearb\LinearACCI         & \indexlinearb\LinearACCXCIX      \\
\indexlinearb\LinearAVI          & \indexlinearb\LinearACIV         & \indexlinearb\LinearACCII        & \indexlinearb\LinearACCC         \\
\indexlinearb\LinearAVII         & \indexlinearb\LinearACV          & \indexlinearb\LinearACCIII       & \indexlinearb\LinearACCCI        \\
\indexlinearb\LinearAVIII        & \indexlinearb\LinearACVI         & \indexlinearb\LinearACCIV        & \indexlinearb\LinearACCCII       \\
\indexlinearb\LinearAIX          & \indexlinearb\LinearACVII        & \indexlinearb\LinearACCV         & \indexlinearb\LinearACCCIII      \\
\indexlinearb\LinearAX           & \indexlinearb\LinearACVIII       & \indexlinearb\LinearACCVI        & \indexlinearb\LinearACCCIV       \\
\indexlinearb\LinearAXI          & \indexlinearb\LinearACIX         & \indexlinearb\LinearACCVII       & \indexlinearb\LinearACCCV        \\
\indexlinearb\LinearAXII         & \indexlinearb\LinearACX          & \indexlinearb\LinearACCVIII      & \indexlinearb\LinearACCCVI       \\
\indexlinearb\LinearAXIII        & \indexlinearb\LinearACXI         & \indexlinearb\LinearACCIX        & \indexlinearb\LinearACCCVII      \\
\indexlinearb\LinearAXIV         & \indexlinearb\LinearACXII        & \indexlinearb\LinearACCX         & \indexlinearb\LinearACCCVIII     \\
\indexlinearb\LinearAXV          & \indexlinearb\LinearACXIII       & \indexlinearb\LinearACCXI        & \indexlinearb\LinearACCCIX       \\
\indexlinearb\LinearAXVI         & \indexlinearb\LinearACXIV        & \indexlinearb\LinearACCXII       & \indexlinearb\LinearACCCX        \\
\indexlinearb\LinearAXVII        & \indexlinearb\LinearACXV         & \indexlinearb\LinearACCXIII      & \indexlinearb\LinearACCCXI       \\
\indexlinearb\LinearAXVIII       & \indexlinearb\LinearACXVI        & \indexlinearb\LinearACCXIV       & \indexlinearb\LinearACCCXII      \\
\indexlinearb\LinearAXIX         & \indexlinearb\LinearACXVII       & \indexlinearb\LinearACCXV        & \indexlinearb\LinearACCCXIII     \\
\indexlinearb\LinearAXX          & \indexlinearb\LinearACXVIII      & \indexlinearb\LinearACCXVI       & \indexlinearb\LinearACCCXIV      \\
\indexlinearb\LinearAXXI         & \indexlinearb\LinearACXIX        & \indexlinearb\LinearACCXVII      & \indexlinearb\LinearACCCXV       \\
\indexlinearb\LinearAXXII        & \indexlinearb\LinearACXX         & \indexlinearb\LinearACCXVIII     & \indexlinearb\LinearACCCXVI      \\
\indexlinearb\LinearAXXIII       & \indexlinearb\LinearACXXI        & \indexlinearb\LinearACCXIX       & \indexlinearb\LinearACCCXVII     \\
\indexlinearb\LinearAXXIV        & \indexlinearb\LinearACXXII       & \indexlinearb\LinearACCXX        & \indexlinearb\LinearACCCXVIII    \\
\indexlinearb\LinearAXXV         & \indexlinearb\LinearACXXIII      & \indexlinearb\LinearACCXXI       & \indexlinearb\LinearACCCXIX      \\
\indexlinearb\LinearAXXVI        & \indexlinearb\LinearACXXIV       & \indexlinearb\LinearACCXXII      & \indexlinearb\LinearACCCXX       \\
\indexlinearb\LinearAXXVII       & \indexlinearb\LinearACXXV        & \indexlinearb\LinearACCXXIII     & \indexlinearb\LinearACCCXXI      \\
\indexlinearb\LinearAXXVIII      & \indexlinearb\LinearACXXVI       & \indexlinearb\LinearACCXXIV      & \indexlinearb\LinearACCCXXII     \\
\indexlinearb\LinearAXXIX        & \indexlinearb\LinearACXXVII      & \indexlinearb\LinearACCXXV       & \indexlinearb\LinearACCCXXIII    \\
\indexlinearb\LinearAXXX         & \indexlinearb\LinearACXXVIII     & \indexlinearb\LinearACCXXVI      & \indexlinearb\LinearACCCXXIV     \\
\indexlinearb\LinearAXXXI        & \indexlinearb\LinearACXXIX       & \indexlinearb\LinearACCXXVII     & \indexlinearb\LinearACCCXXV      \\
\indexlinearb\LinearAXXXII       & \indexlinearb\LinearACXXX        & \indexlinearb\LinearACCXXVIII    & \indexlinearb\LinearACCCXXVI     \\
\indexlinearb\LinearAXXXIII      & \indexlinearb\LinearACXXXI       & \indexlinearb\LinearACCXXIX      & \indexlinearb\LinearACCCXXVII    \\
\indexlinearb\LinearAXXXIV       & \indexlinearb\LinearACXXXII      & \indexlinearb\LinearACCXXX       & \indexlinearb\LinearACCCXXVIII   \\
\indexlinearb\LinearAXXXV        & \indexlinearb\LinearACXXXIII     & \indexlinearb\LinearACCXXXI      & \indexlinearb\LinearACCCXXIX     \\
\indexlinearb\LinearAXXXVI       & \indexlinearb\LinearACXXXIV      & \indexlinearb\LinearACCXXXII     & \indexlinearb\LinearACCCXXX      \\
\indexlinearb\LinearAXXXVII      & \indexlinearb\LinearACXXXV       & \indexlinearb\LinearACCXXXIII    & \indexlinearb\LinearACCCXXXI     \\
\indexlinearb\LinearAXXXVIII     & \indexlinearb\LinearACXXXVI      & \indexlinearb\LinearACCXXXIV     & \indexlinearb\LinearACCCXXXII    \\
\indexlinearb\LinearAXXXIX       & \indexlinearb\LinearACXXXVII     & \indexlinearb\LinearACCXXXV      & \indexlinearb\LinearACCCXXXIII   \\
\indexlinearb\LinearAXL          & \indexlinearb\LinearACXXXVIII    & \indexlinearb\LinearACCXXXVI     & \indexlinearb\LinearACCCXXXIV    \\
\indexlinearb\LinearAXLI         & \indexlinearb\LinearACXXXIX      & \indexlinearb\LinearACCXXXVII    & \indexlinearb\LinearACCCXXXV     \\
\indexlinearb\LinearAXLII        & \indexlinearb\LinearACXL         & \indexlinearb\LinearACCXXXVIII   & \indexlinearb\LinearACCCXXXVI    \\
\indexlinearb\LinearAXLIII       & \indexlinearb\LinearACXLI        & \indexlinearb\LinearACCXXXIX     & \indexlinearb\LinearACCCXXXVII   \\
\indexlinearb\LinearAXLIV        & \indexlinearb\LinearACXLII       & \indexlinearb\LinearACCXL        & \indexlinearb\LinearACCCXXXVIII  \\
\indexlinearb\LinearAXLV         & \indexlinearb\LinearACXLIII      & \indexlinearb\LinearACCXLI       & \indexlinearb\LinearACCCXXXIX    \\
\indexlinearb\LinearAXLVI        & \indexlinearb\LinearACXLIV       & \indexlinearb\LinearACCXLII      & \indexlinearb\LinearACCCXL       \\
\indexlinearb\LinearAXLVII       & \indexlinearb\LinearACXLV        & \indexlinearb\LinearACCXLIII     & \indexlinearb\LinearACCCXLI      \\
\indexlinearb\LinearAXLVIII      & \indexlinearb\LinearACXLVI       & \indexlinearb\LinearACCXLIV      & \indexlinearb\LinearACCCXLII     \\
\indexlinearb\LinearAXLIX        & \indexlinearb\LinearACXLVII      & \indexlinearb\LinearACCXLV       & \indexlinearb\LinearACCCXLIII    \\
\indexlinearb\LinearAL           & \indexlinearb\LinearACXLVIII     & \indexlinearb\LinearACCXLVI      & \indexlinearb\LinearACCCXLIV     \\
\indexlinearb\LinearALI          & \indexlinearb\LinearACXLIX       & \indexlinearb\LinearACCXLVII     & \indexlinearb\LinearACCCXLV      \\
\indexlinearb\LinearALII         & \indexlinearb\LinearACL          & \indexlinearb\LinearACCXLVIII    & \indexlinearb\LinearACCCXLVI     \\
\indexlinearb\LinearALIII        & \indexlinearb\LinearACLI         & \indexlinearb\LinearACCXLIX      & \indexlinearb\LinearACCCXLVII    \\
\indexlinearb\LinearALIV         & \indexlinearb\LinearACLII        & \indexlinearb\LinearACCL         & \indexlinearb\LinearACCCXLVIII   \\
\indexlinearb\LinearALV          & \indexlinearb\LinearACLIII       & \indexlinearb\LinearACCLI        & \indexlinearb\LinearACCCXLIX     \\
\indexlinearb\LinearALVI         & \indexlinearb\LinearACLIV        & \indexlinearb\LinearACCLII       & \indexlinearb\LinearACCCL        \\
\indexlinearb\LinearALVII        & \indexlinearb\LinearACLV         & \indexlinearb\LinearACCLIII      & \indexlinearb\LinearACCCLI       \\
\indexlinearb\LinearALVIII       & \indexlinearb\LinearACLVI        & \indexlinearb\LinearACCLIV       & \indexlinearb\LinearACCCLII      \\
\indexlinearb\LinearALIX         & \indexlinearb\LinearACLVII       & \indexlinearb\LinearACCLV        & \indexlinearb\LinearACCCLIII     \\
\indexlinearb\LinearALX          & \indexlinearb\LinearACLVIII      & \indexlinearb\LinearACCLVI       & \indexlinearb\LinearACCCLIV      \\
\indexlinearb\LinearALXI         & \indexlinearb\LinearACLIX        & \indexlinearb\LinearACCLVII      & \indexlinearb\LinearACCCLV       \\
\indexlinearb\LinearALXII        & \indexlinearb\LinearACLX         & \indexlinearb\LinearACCLVIII     & \indexlinearb\LinearACCCLVI      \\
\indexlinearb\LinearALXIII       & \indexlinearb\LinearACLXI        & \indexlinearb\LinearACCLIX       & \indexlinearb\LinearACCCLVII     \\
\indexlinearb\LinearALXIV        & \indexlinearb\LinearACLXII       & \indexlinearb\LinearACCLX        & \indexlinearb\LinearACCCLVIII    \\
\indexlinearb\LinearALXV         & \indexlinearb\LinearACLXIII      & \indexlinearb\LinearACCLXI       & \indexlinearb\LinearACCCLIX      \\
\indexlinearb\LinearALXVI        & \indexlinearb\LinearACLXIV       & \indexlinearb\LinearACCLXII      & \indexlinearb\LinearACCCLX       \\
\indexlinearb\LinearALXVII       & \indexlinearb\LinearACLXV        & \indexlinearb\LinearACCLXIII     & \indexlinearb\LinearACCCLXI      \\
\indexlinearb\LinearALXVIII      & \indexlinearb\LinearACLXVI       & \indexlinearb\LinearACCLXIV      & \indexlinearb\LinearACCCLXII     \\
\indexlinearb\LinearALXIX        & \indexlinearb\LinearACLXVII      & \indexlinearb\LinearACCLXV       & \indexlinearb\LinearACCCLXIII    \\
\indexlinearb\LinearALXX         & \indexlinearb\LinearACLXVIII     & \indexlinearb\LinearACCLXVI      & \indexlinearb\LinearACCCLXIV     \\
\indexlinearb\LinearALXXI        & \indexlinearb\LinearACLXIX       & \indexlinearb\LinearACCLXVII     & \indexlinearb\LinearACCCLXV      \\
\indexlinearb\LinearALXXII       & \indexlinearb\LinearACLXX        & \indexlinearb\LinearACCLXVIII    & \indexlinearb\LinearACCCLXVI     \\
\indexlinearb\LinearALXXIII      & \indexlinearb\LinearACLXXI       & \indexlinearb\LinearACCLXIX      & \indexlinearb\LinearACCCLXVII    \\
\indexlinearb\LinearALXXIV       & \indexlinearb\LinearACLXXII      & \indexlinearb\LinearACCLXX       & \indexlinearb\LinearACCCLXVIII   \\
\indexlinearb\LinearALXXV        & \indexlinearb\LinearACLXXIII     & \indexlinearb\LinearACCLXXI      & \indexlinearb\LinearACCCLXIX     \\
\indexlinearb\LinearALXXVI       & \indexlinearb\LinearACLXXIV      & \indexlinearb\LinearACCLXXII     & \indexlinearb\LinearACCCLXX      \\
\indexlinearb\LinearALXXVII      & \indexlinearb\LinearACLXXV       & \indexlinearb\LinearACCLXXIII    & \indexlinearb\LinearACCCLXXI     \\
\indexlinearb\LinearALXXVIII     & \indexlinearb\LinearACLXXVI      & \indexlinearb\LinearACCLXXIV     & \indexlinearb\LinearACCCLXXII    \\
\indexlinearb\LinearALXXIX       & \indexlinearb\LinearACLXXVII     & \indexlinearb\LinearACCLXXV      & \indexlinearb\LinearACCCLXXIII   \\
\indexlinearb\LinearALXXX        & \indexlinearb\LinearACLXXVIII    & \indexlinearb\LinearACCLXXVI     & \indexlinearb\LinearACCCLXXIV    \\
\indexlinearb\LinearALXXXI       & \indexlinearb\LinearACLXXIX      & \indexlinearb\LinearACCLXXVII    & \indexlinearb\LinearACCCLXXV     \\
\indexlinearb\LinearALXXXII      & \indexlinearb\LinearACLXXX       & \indexlinearb\LinearACCLXXVIII   & \indexlinearb\LinearACCCLXXVI    \\
\indexlinearb\LinearALXXXIII     & \indexlinearb\LinearACLXXXI      & \indexlinearb\LinearACCLXXIX     & \indexlinearb\LinearACCCLXXVII   \\
\indexlinearb\LinearALXXXIV      & \indexlinearb\LinearACLXXXII     & \indexlinearb\LinearACCLXXX      & \indexlinearb\LinearACCCLXXVIII  \\
\indexlinearb\LinearALXXXV       & \indexlinearb\LinearACLXXXIII    & \indexlinearb\LinearACCLXXXI     & \indexlinearb\LinearACCCLXXIX    \\
\indexlinearb\LinearALXXXVI      & \indexlinearb\LinearACLXXXIV     & \indexlinearb\LinearACCLXXXII    & \indexlinearb\LinearACCCLXXX     \\
\indexlinearb\LinearALXXXVII     & \indexlinearb\LinearACLXXXV      & \indexlinearb\LinearACCLXXXIII   & \indexlinearb\LinearACCCLXXXI    \\
\indexlinearb\LinearALXXXVIII    & \indexlinearb\LinearACLXXXVI     & \indexlinearb\LinearACCLXXXIV    & \indexlinearb\LinearACCCLXXXII   \\
\indexlinearb\LinearALXXXIX      & \indexlinearb\LinearACLXXXVII    & \indexlinearb\LinearACCLXXXV     & \indexlinearb\LinearACCCLXXXIII  \\
\indexlinearb\LinearALXXXX       & \indexlinearb\LinearACLXXXVIII   & \indexlinearb\LinearACCLXXXVI    & \indexlinearb\LinearACCCLXXXIV   \\
\indexlinearb\LinearAXCI         & \indexlinearb\LinearACLXXXIX     & \indexlinearb\LinearACCLXXXVII   & \indexlinearb\LinearACCCLXXXV    \\
\indexlinearb\LinearAXCII        & \indexlinearb\LinearACLXXXX      & \indexlinearb\LinearACCLXXXVIII  & \indexlinearb\LinearACCCLXXXVI   \\
\indexlinearb\LinearAXCIII       & \indexlinearb\LinearACXCI        & \indexlinearb\LinearACCLXXXIX    & \indexlinearb\LinearACCCLXXXVII  \\
\indexlinearb\LinearAXCIV        & \indexlinearb\LinearACXCII       & \indexlinearb\LinearACCLXXXX     & \indexlinearb\LinearACCCLXXXVIII \\
\indexlinearb\LinearAXCV         & \indexlinearb\LinearACXCIII      & \indexlinearb\LinearACCXCI       & \indexlinearb\LinearACCCLXXXIX   \\
\indexlinearb\LinearAXCVI        & \indexlinearb\LinearACXCIV       & \indexlinearb\LinearACCXCII      &                       \\
\indexlinearb\LinearAXCVII       & \indexlinearb\LinearACXCV        & \indexlinearb\LinearACCXCIII     &                       \\
\indexlinearb\LinearAXCVIII      & \indexlinearb\LinearACXCVI       & \indexlinearb\LinearACCXCIV      &                       \\
\end{longtable}
\end{longsymtable}

\begin{longsymtable}[LINB]{\LINB\ Linear~B Basic and Optional Letters}
\index{Linear B}
\index{alphabets>Linear B}
\label{linearB}
\begin{longtable}{*5{ll@{\qquad}}ll}
\indexlinearb[\textlinb{\Ba}]\Ba         & \indexlinearb[\textlinb{\Bja}]\Bja       & \indexlinearb[\textlinb{\Bmu}]\Bmu       & \indexlinearb[\textlinb{\Bpte}]\Bpte     & \indexlinearb[\textlinb{\Broii}]\Broii   & \indexlinearb[\textlinb{\Bto}]\Bto       \\
\indexlinearb[\textlinb{\Baii}]\Baii     & \indexlinearb[\textlinb{\Bje}]\Bje       & \indexlinearb[\textlinb{\Bna}]\Bna       & \indexlinearb[\textlinb{\Bpu}]\Bpu       & \indexlinearb[\textlinb{\Bru}]\Bru       & \indexlinearb[\textlinb{\Btu}]\Btu       \\
\indexlinearb[\textlinb{\Baiii}]\Baiii   & \indexlinearb[\textlinb{\Bjo}]\Bjo       & \indexlinearb[\textlinb{\Bne}]\Bne       & \indexlinearb[\textlinb{\Bpuii}]\Bpuii   & \indexlinearb[\textlinb{\Bsa}]\Bsa       & \indexlinearb[\textlinb{\Btwo}]\Btwo     \\
\indexlinearb[\textlinb{\Bau}]\Bau       & \indexlinearb[\textlinb{\Bju}]\Bju       & \indexlinearb[\textlinb{\Bni}]\Bni       & \indexlinearb[\textlinb{\Bqa}]\Bqa       & \indexlinearb[\textlinb{\Bse}]\Bse       & \indexlinearb[\textlinb{\Bu}]\Bu         \\
\indexlinearb[\textlinb{\Bda}]\Bda       & \indexlinearb[\textlinb{\Bka}]\Bka       & \indexlinearb[\textlinb{\Bno}]\Bno       & \indexlinearb[\textlinb{\Bqe}]\Bqe       & \indexlinearb[\textlinb{\Bsi}]\Bsi       & \indexlinearb[\textlinb{\Bwa}]\Bwa       \\
\indexlinearb[\textlinb{\Bde}]\Bde       & \indexlinearb[\textlinb{\Bke}]\Bke       & \indexlinearb[\textlinb{\Bnu}]\Bnu       & \indexlinearb[\textlinb{\Bqi}]\Bqi       & \indexlinearb[\textlinb{\Bso}]\Bso       & \indexlinearb[\textlinb{\Bwe}]\Bwe       \\
\indexlinearb[\textlinb{\Bdi}]\Bdi       & \indexlinearb[\textlinb{\Bki}]\Bki       & \indexlinearb[\textlinb{\Bnwa}]\Bnwa     & \indexlinearb[\textlinb{\Bqo}]\Bqo       & \indexlinearb[\textlinb{\Bsu}]\Bsu       & \indexlinearb[\textlinb{\Bwi}]\Bwi       \\
\indexlinearb[\textlinb{\Bdo}]\Bdo       & \indexlinearb[\textlinb{\Bko}]\Bko       & \indexlinearb[\textlinb{\Bo}]\Bo         & \indexlinearb[\textlinb{\Bra}]\Bra       & \indexlinearb[\textlinb{\Bswa}]\Bswa     & \indexlinearb[\textlinb{\Bwo}]\Bwo       \\
\indexlinearb[\textlinb{\Bdu}]\Bdu       & \indexlinearb[\textlinb{\Bku}]\Bku       & \indexlinearb[\textlinb{\Bpa}]\Bpa       & \indexlinearb[\textlinb{\Braii}]\Braii   & \indexlinearb[\textlinb{\Bswi}]\Bswi     & \indexlinearb[\textlinb{\Bza}]\Bza       \\
\indexlinearb[\textlinb{\Bdwe}]\Bdwe     & \indexlinearb[\textlinb{\Bma}]\Bma       & \indexlinearb[\textlinb{\Bpaiii}]\Bpaiii & \indexlinearb[\textlinb{\Braiii}]\Braiii & \indexlinearb[\textlinb{\Bta}]\Bta       & \indexlinearb[\textlinb{\Bze}]\Bze       \\
\indexlinearb[\textlinb{\Bdwo}]\Bdwo     & \indexlinearb[\textlinb{\Bme}]\Bme       & \indexlinearb[\textlinb{\Bpe}]\Bpe       & \indexlinearb[\textlinb{\Bre}]\Bre       & \indexlinearb[\textlinb{\Btaii}]\Btaii   & \indexlinearb[\textlinb{\Bzo}]\Bzo       \\
\indexlinearb[\textlinb{\Be}]\Be         & \indexlinearb[\textlinb{\Bmi}]\Bmi       & \indexlinearb[\textlinb{\Bpi}]\Bpi       & \indexlinearb[\textlinb{\Bri}]\Bri       & \indexlinearb[\textlinb{\Bte}]\Bte       &                               \\
\indexlinearb[\textlinb{\Bi}]\Bi         & \indexlinearb[\textlinb{\Bmo}]\Bmo       & \indexlinearb[\textlinb{\Bpo}]\Bpo       & \indexlinearb[\textlinb{\Bro}]\Bro       & \indexlinearb[\textlinb{\Bti}]\Bti       &                               \\
\end{longtable}

\bigskip
\begin{tablenote}
  \usefontcmdmessage{\textlinb}{\linbfamily}.  Single-character
  shortcuts are also supported: Both
  ``\verb+\textlinb{\Bpa\Bki\Bna}+'' and ``\verb+\textlinb{pcn}+''
  produce ``\textlinb{pcn}'', for example.  \seedocs{\LINB}.
\end{tablenote}
\end{longsymtable}


\begin{symtable}[LINB]{\LINB\ Linear~B Numerals}
\index{Linear B}
\index{numerals>Linear B}
\index{tally markers}
\label{linearB-nums}
\begin{tabular}{*4{ll@{\qquad}}ll}
\indexlinearb[\textlinb{\BNi}]\BNi       & \indexlinearb[\textlinb{\BNvii}]\BNvii   & \indexlinearb[\textlinb{\BNxl}]\BNxl     & \indexlinearb[\textlinb{\BNc}]\BNc       & \indexlinearb[\textlinb{\BNdcc}]\BNdcc   \\
\indexlinearb[\textlinb{\BNii}]\BNii     & \indexlinearb[\textlinb{\BNviii}]\BNviii & \indexlinearb[\textlinb{\BNl}]\BNl       & \indexlinearb[\textlinb{\BNcc}]\BNcc     & \indexlinearb[\textlinb{\BNdccc}]\BNdccc \\
\indexlinearb[\textlinb{\BNiii}]\BNiii   & \indexlinearb[\textlinb{\BNix}]\BNix     & \indexlinearb[\textlinb{\BNlx}]\BNlx     & \indexlinearb[\textlinb{\BNccc}]\BNccc   & \indexlinearb[\textlinb{\BNcm}]\BNcm     \\
\indexlinearb[\textlinb{\BNiv}]\BNiv     & \indexlinearb[\textlinb{\BNx}]\BNx       & \indexlinearb[\textlinb{\BNlxx}]\BNlxx   & \indexlinearb[\textlinb{\BNcd}]\BNcd     & \indexlinearb[\textlinb{\BNm}]\BNm       \\
\indexlinearb[\textlinb{\BNv}]\BNv       & \indexlinearb[\textlinb{\BNxx}]\BNxx     & \indexlinearb[\textlinb{\BNlxxx}]\BNlxxx & \indexlinearb[\textlinb{\BNd}]\BNd       &                               \\
\indexlinearb[\textlinb{\BNvi}]\BNvi     & \indexlinearb[\textlinb{\BNxxx}]\BNxxx   & \indexlinearb[\textlinb{\BNxc}]\BNxc     & \indexlinearb[\textlinb{\BNdc}]\BNdc     &                               \\
\end{tabular}

\bigskip
\begin{tablenote}
  \usefontcmdmessage{\textlinb}{\linbfamily}.
\end{tablenote}
\end{symtable}


\begin{symtable}[LINB]{\LINB\ Linear~B Weights and Measures}
\index{Linear B}
\label{linearB-weights}
\begin{tabular}{*4{ll@{\qquad}}ll}
\indexlinearb[\textlinb{\BPtalent}]\BPtalent & \indexlinearb[\textlinb{\BPvolb}]\BPvolb     & \indexlinearb[\textlinb{\BPvolcf}]\BPvolcf   & \indexlinearb[\textlinb{\BPwtb}]\BPwtb       & \indexlinearb[\textlinb{\BPwtd}]\BPwtd       \\
\indexlinearb[\textlinb{\BPvola}]\BPvola     & \indexlinearb[\textlinb{\BPvolcd}]\BPvolcd   & \indexlinearb[\textlinb{\BPwta}]\BPwta       & \indexlinearb[\textlinb{\BPwtc}]\BPwtc       &                                   \\
\end{tabular}

\bigskip
\begin{tablenote}
  \usefontcmdmessage{\textlinb}{\linbfamily}.
\end{tablenote}
\end{symtable}


\begin{symtable}[LINB]{\LINB\ Linear~B Ideograms}
\index{Linear B}
\index{arrows}
\index{animals}
\label{linearB-objs}
\begin{tabular}{*3{ll@{\qquad}}ll}
\indexlinearb[\textlinb{\BPamphora}]\BPamphora       & \indexlinearb[\textlinb{\BPchassis}]\BPchassis       & \indexlinearb[\textlinb{\BPman}]\BPman               & \indexlinearb[\textlinb{\BPwheat}]\BPwheat           \\
\indexlinearb[\textlinb{\BParrow}]\BParrow           & \indexlinearb[\textlinb{\BPcloth}]\BPcloth           & \indexlinearb[\textlinb{\BPnanny}]\BPnanny           & \indexlinearb[\textlinb{\BPwheel}]\BPwheel           \\
\indexlinearb[\textlinb{\BPbarley}]\BPbarley         & \indexlinearb[\textlinb{\BPcow}]\BPcow               & \indexlinearb[\textlinb{\BPolive}]\BPolive           & \indexlinearb[\textlinb{\BPwine}]\BPwine             \\
\indexlinearb[\textlinb{\BPbilly}]\BPbilly           & \indexlinearb[\textlinb{\BPcup}]\BPcup               & \indexlinearb[\textlinb{\BPox}]\BPox                 & \indexlinearb[\textlinb{\BPwineiih}]\BPwineiih       \\
\indexlinearb[\textlinb{\BPboar}]\BPboar             & \indexlinearb[\textlinb{\BPewe}]\BPewe               & \indexlinearb[\textlinb{\BPpig}]\BPpig               & \indexlinearb[\textlinb{\BPwineiiih}]\BPwineiiih     \\
\indexlinearb[\textlinb{\BPbronze}]\BPbronze         & \indexlinearb[\textlinb{\BPfoal}]\BPfoal             & \indexlinearb[\textlinb{\BPram}]\BPram               & \indexlinearb[\textlinb{\BPwineivh}]\BPwineivh       \\
\indexlinearb[\textlinb{\BPbull}]\BPbull             & \indexlinearb[\textlinb{\BPgoat}]\BPgoat             & \indexlinearb[\textlinb{\BPsheep}]\BPsheep           & \indexlinearb[\textlinb{\BPwoman}]\BPwoman           \\
\indexlinearb[\textlinb{\BPcauldroni}]\BPcauldroni   & \indexlinearb[\textlinb{\BPgoblet}]\BPgoblet         & \indexlinearb[\textlinb{\BPsow}]\BPsow               & \indexlinearb[\textlinb{\BPwool}]\BPwool             \\
\indexlinearb[\textlinb{\BPcauldronii}]\BPcauldronii & \indexlinearb[\textlinb{\BPgold}]\BPgold             & \indexlinearb[\textlinb{\BPspear}]\BPspear           &                                           \\
\indexlinearb[\textlinb{\BPchariot}]\BPchariot       & \indexlinearb[\textlinb{\BPhorse}]\BPhorse           & \indexlinearb[\textlinb{\BPsword}]\BPsword           &                                           \\
\end{tabular}

\bigskip
\begin{tablenote}
  \usefontcmdmessage{\textlinb}{\linbfamily}.
\end{tablenote}
\end{symtable}


\begin{longsymtable}[LINB]{\LINB\ Unidentified Linear~B Symbols}
\index{Linear B}
\label{linearB-unknown}
\begin{longtable}{*4{ll@{\qquad}}ll}
\indexlinearb[\textlinb{\BUi}]\BUi       & \indexlinearb[\textlinb{\BUiv}]\BUiv     & \indexlinearb[\textlinb{\BUvii}]\BUvii   & \indexlinearb[\textlinb{\BUx}]\BUx       & \indexlinearb[\textlinb{\Btwe}]\Btwe     \\
\indexlinearb[\textlinb{\BUii}]\BUii     & \indexlinearb[\textlinb{\BUv}]\BUv       & \indexlinearb[\textlinb{\BUviii}]\BUviii & \indexlinearb[\textlinb{\BUxi}]\BUxi     &                               \\
\indexlinearb[\textlinb{\BUiii}]\BUiii   & \indexlinearb[\textlinb{\BUvi}]\BUvi     & \indexlinearb[\textlinb{\BUix}]\BUix     & \indexlinearb[\textlinb{\BUxii}]\BUxii   &                               \\
\end{longtable}

\bigskip
\begin{tablenote}
  \usefontcmdmessage{\textlinb}{\linbfamily}.
\end{tablenote}
\end{longsymtable}

\section{Magical Staves}

\begin{longsymtable}[STAVE]{\STAVE\ Magical Staves}
\index{symbols>staves}
\index{symbols>magical signs}
\index{magical signs}
\index{staves}
\index{Icelandic staves}
\label{staves}
\small
\begin{longtable}{*2{ll@{\qqquad}}ll}
\multicolumn{6}{l}{\small\textit{(continued from previous page)}} \\[3ex]
\endhead
\endfirsthead
\\[3ex]
\multicolumn{6}{r}{\small\textit{(continued on next page)}}
\endfoot
\endlastfoot
\Kstav\staveI     & \Kstav\staveXXIV    & \Kstav\staveXLVII  \\
\Kstav\staveII    & \Kstav\staveXXV     & \Kstav\staveXLVIII \\
\Kstav\staveIII   & \Kstav\staveXXVI    & \Kstav\staveXLIX   \\
\Kstav\staveIV    & \Kstav\staveXXVII   & \Kstav\staveL      \\
\Kstav\staveV     & \Kstav\staveXXVIII  & \Kstav\staveLI     \\
\Kstav\staveVI    & \Kstav\staveXXIX    & \Kstav\staveLII    \\
\Kstav\staveVII   & \Kstav\staveXXX     & \Kstav\staveLIII   \\
\Kstav\staveVIII  & \Kstav\staveXXXI    & \Kstav\staveLIV    \\
\Kstav\staveIX    & \Kstav\staveXXXII   & \Kstav\staveLV     \\
\Kstav\staveX     & \Kstav\staveXXXIII  & \Kstav\staveLVI    \\
\Kstav\staveXI    & \Kstav\staveXXXIV   & \Kstav\staveLVII   \\
\Kstav\staveXII   & \Kstav\staveXXXV    & \Kstav\staveLVIII  \\
\Kstav\staveXIII  & \Kstav\staveXXXVI   & \Kstav\staveLIX    \\
\Kstav\staveXIV   & \Kstav\staveXXXVII  & \Kstav\staveLX     \\
\Kstav\staveXV    & \Kstav\staveXXXVIII & \Kstav\staveLXI    \\
\Kstav\staveXVI   & \Kstav\staveXXXIX   & \Kstav\staveLXII   \\
\Kstav\staveXVII  & \Kstav\staveXL      & \Kstav\staveLXIII  \\
\Kstav\staveXVIII & \Kstav\staveXLI     & \Kstav\staveLXIV   \\
\Kstav\staveXIX   & \Kstav\staveXLII    & \Kstav\staveLXV    \\
\Kstav\staveXX    & \Kstav\staveXLIII   & \Kstav\staveLXVI   \\
\Kstav\staveXXI   & \Kstav\staveXLIV    & \Kstav\staveLXVII  \\
\Kstav\staveXXII  & \Kstav\staveXLV     & \Kstav\staveLXVIII \\
\Kstav\staveXXIII & \Kstav\staveXLVI    &                \\
\end{longtable}

\bigskip

\begin{tablenote}
  The meanings of these symbols are described on the Web site for the
  Museum of Icelandic Sorcery and Witchcraft\index{Museum of Icelandic
  Sorcery and Witchcraft} at
  \url{http://www.galdrasyning.is/index.php?option=com_content&task=category&sectionid=5&id=18&Itemid=60}
  (TinyURL: \url{http://tinyurl.com/25979m}).  For example,
  \docAuxCommand{staveL}~(``\staveL'') is intended to ward off
  ghosts\index{ghosts} and evil\index{evil spirits} spirits.
\end{tablenote}
\end{longsymtable}


\subsection{Resizing symbols}
\label{resizing-symbols}
\index{symbols>resize}

Mathematical symbols listed in this document as
``variable-sized\idxboth{variable-sized}{symbols}'' are designed to
stretch vertically.  Each
variable-sized\idxboth{variable-sized}{symbols} symbol comes in one or
more basic sizes plus a variation comprising both stretchable and
nonstretchable segments.  Table \vref{var-sized-syms} presents the
symbols %\docAuxCommand{}}
 and \docAuxCommand{uparrow} in their default size, in their
\cmd{\big}, \cmd{\Big}, \cmd{\bigg}, and \cmd{\Bigg} sizes, in an even
larger size achieved using \cmd{\left}\slash\cmd{\right}, and---for
contrast---in a large size achieved by changing the font size using
\latexe's \cmd{\fontsize} command.  Because the symbols shown belong
to the \PSfont{Computer Modern} family, the \pkgname{type1cm} package
needs to be loaded to support font sizes larger than 24.88\,pt.

\begin{nonsymtable}{Sample resized delimiters}
\idxboth{variable-sized}{symbols}
\label{var-sized-syms}
\newcommand{\maketall}[1]{\ensuremath{\left.\rule{0pt}{1.5cm}\right#1}}
\newcommand{\makebig}[1]{\fontsize{3cm}{3cm}\selectfont\ensuremath{#1}}

\begin{tabular}{@{}*8c@{}}
  \toprule
  Symbol &
  Default size &
  \cmd{\big} &
  \cmd{\Big} &
  \cmd{\bigg} &
  \cmd{\Bigg} &
  \cmd{\left}\,/\,\cmd{\right} &
  \cmd{\fontsize} \\
  \midrule

  \verb|\}| &
  $\}$ &
  $\big\}$ &
  $\Big\}$ &
  $\bigg\}$ &
  $\Bigg\}$ &
  \maketall\} &
  \makebig\} \\

  \verb|\uparrow| &
  $\uparrow$ &
  $\big\uparrow$ &
  $\Big\uparrow$ &
  $\bigg\uparrow$ &
  $\Bigg\uparrow$ &
  \maketall\uparrow &
  \makebig\uparrow \\
  \bottomrule
\end{tabular}
\end{nonsymtable}

All variable-sized delimiters are defined (by the corresponding
\texttt{.tfm} file) in terms of up to five segments, as illustrated by
\vref{extensible-brace}.  The top, middle, and bottom segments
are of a fixed size.  The top-middle and middle-bottom segments (which
are constrained to be the same character) are repeated as many times
as necessary to achieve the desired height.

\begin{figure}[htbp]
\centering
\renewcommand{\arraystretch}{2}
\newcommand{\cmexchar}{\usefont{OMX}{cmex}{m}{n}\selectfont\char}
\newlength{\braceheight}
\setlength{\braceheight}{6.5\baselineskip}
\begin{tabular}{@{}ccl@{}}
  \multirow{5}*{$\left.\rule{0pt}{\braceheight}\right\} \longrightarrow$}
  & \cmexchar'71 & top \\
  & \cmexchar'76 & top-middle (extensible) \\
  & \cmexchar'75 & middle \\
  & \cmexchar'76 & middle-bottom (extensible) \\
  & \cmexchar'73 & bottom \\
  \\
\end{tabular}
\index{symbols>extensible}
\caption{Implementation of variable-sized delimiters}
\label{extensible-brace}
\end{figure}

  
\subsubsection{Reflecting and rotating existing symbols}

 
  \index{symbols>reversed|(}
  \index{symbols>rotated|(}
  \index{symbols>upside-down|(}
  \index{symbols>inverted|(}
  \index{reversed symbols|(}
  \index{rotated symbols|(}
  \index{upside-down symbols|(}
  \index{inverted symbols|(}
  
  
  \begin{texexample}{Create an Irony mark}{}
  \DeclareRobustCommand{\irony}{\textsuperscript{\reflectbox{?}}}
  \end{texexample}
  \DeclareRobustCommand{\irony}{\textsuperscript{\reflectbox{?}}}
  A common request on \ctt is for a reversed or rotated version of an
  existing symbol.  As a last resort, these effects can be achieved
  with the \pkgname{graphicx} (or \pkgname{graphics}) package's
  \cmd{\reflectbox} and \cmd{\rotatebox} macros.
  \newcommand{\definitedescription}{\rotatebox[origin=c]{180}{$\iota$}}
  For example, \verb|\textsuperscript{\reflectbox{?}}| produces an
  irony\index{irony mark=irony mark (\irony)} mark~(``\,\irony\,'';
  cf.~\url{http://en.wikipedia.org/wiki/Irony_mark}), and
  \verb|\rotatebox[origin=c]{180}{$\iota$}| produces the
  definite-description\index{definite-description operator
  (\definitedescription)}\index{iota, upside-down}
  operator~(``\rotatebox[origin=c]{180}{$\iota$}'').  
  
  The disadvantage
  of the \pkgname{graphicx}/\pkgname{graphics} approach is that not
  every \tex backend handles graphical transformations.\footnote{As an
  example, Xdvi\index{Xdvi} ignores both \cmd{\reflectbox} and
  \cmd{\rotatebox}.}  Far better is to find a suitable font that
  contains the desired symbol in the correct orientation.  For
  instance, if the PHON package is available, then
  \verb|\textit{\riota}| will yield a
  backend-independent~``\textit{\cmd{\riota}}''.
  Similarly,\label{page:such-that} \TIPA's
  \docAuxCommand{textrevepsilon}~(``\textrevepsilon'') or \WIPA's
  \docAuxCommand{textrevepsilon}~(``\textrevepsilon'') may be used to express the
  mathematical notion of ``such\index{such that} that'' in a cleaner
  manner than with \cmd{\reflectbox} or
  \cmd{\rotatebox}.\footnote{More common symbols for representing
  ``such\index{such that} that'' include ``\texttt{\textbar}'',
  ``\texttt{:}'', and ``\texttt{s.t.}''.}
  \index{symbols>reversed|)}
  \index{symbols>rotated|)}
  \index{symbols>upside-down|)}
  \index{symbols>inverted|)}
  \index{reversed symbols|)}
  \index{rotated symbols|)}
  \index{inverted symbols|)}



\begin{texexample}{Enlarging Delimiters}{ex:type1cm}
\newcommand{\makeBIG}[1]{\fontsize{1cm}{1cm}\selectfont\ensuremath{#1}}
  \makeBIG\>

\end{texexample}

\subsection{Where can I find the symbol for~\dots?}

\label{combining-symbols}

An easy way to find a symbol is to use \url{http://detexify.com}. This is a website service that you can use to identify a symbol by drawing it. The menu always adds the necessary package to a symbol after presenting possible matches to what you have drawn. But you also can click on the ``symbols'' button and enter the command. Here, too, the necessary package is added. 

If you can't find some symbol you're looking for in this document, there
are a few possible explanations:

\begin{itemize}
  \item The symbol isn't intuitively named.  As a few examples, the
  \IFS\ command to draw dice\index{dice} is
  ``\docAuxCommand{Cube}''; a plus sign with a circle around it
  (``exclusive or''\index{exclusive or} to computer engineers) is
  ``\docAuxCommand{oplus}''; and lightning bolts in fonts designed by German
  speakers may have ``blitz'' in their names as in the
  ULSY package.  The moral of the story is to be creative with
  synonyms when searching the index.

  \item The symbol is defined by some package that I overlooked (or
  deemed unimportant).  

  \item The symbol isn't defined in any package whatsoever.
\end{itemize}


  Even in the last case, all is not lost.  Sometimes, a symbol exists
  in a font, but there is no \latex{} binding for it.  For example,
  the \postscript \PSfont{Symbol} font contains a
  ``\Pisymbol{psy}{191}''\index{arrows} symbol, which may be useful
  for representing a carriage\index{carriage return} return, but there
  is no package (as far as I know) for accessing that symbol.  To
  produce an unnamed symbol, you need to switch to the font explicitly
  with \latexe's low-level font commands~\cite{fntguide} and use
  \tex's primitive \cmd{\char} command~\cite{Knuth:ct-a} to request a
  specific character number in the font.\footnote{\pkgname{pifont}
  defines a convenient \cmd{\Pisymbol} command for accessing symbols
  in \postscript\index{PostScript fonts} fonts by number.  For example,
  ``\cmd{\Pisymbol}\texttt{\string{psy\string}\string{191\string}}''
  produces ``\Pisymbol{psy}{191}''.}
   
  In fact, \cmd{\char} is not strictly necesssary; the character can
  often be entered symbolically.
 

  For example, the symbol for an impulse train or Tate-Shafarevich
  group (``{|\string\fontencoding{OT2}\string\selectfont SH|}'') is actually an
  uppercase \textit{sha} in the Cyrillic\index{alphabets>Cyrillic}
  alphabet.  (Cyrillic is supported by the OT2 \fntenc[OT2], for
  instance).  While a \textit{sha} can be defined numerically as
  
  it may be more intuitive to use the OT2 \fntenc[OT2]'s ``SH''
  ligature:
  
 
The \pkgname{slashed} package \citep{slashed}, although originally designed for
producing Feynman\index{Feynman slashed character notation}
slashed-character\idxboth{slashed}{letters} notation, in fact
facilitates the production of \emph{arbitrary} overlapped symbols.
\ifhaveslashed
  \newcommand{\rqm}{{\declareslashed{}{\text{-}}{0.04}{0}{I}\slashed{I}}}
  The default behavior is to overwrite a given character with ``$/$''.
  For example, \cmd{\slashed}\verb|{D}| produces ``$\slashed{D}$''.
  However, the \cmd{\declareslashed} command provides the flexibility
  to specify the mathematical context of the composite character
  (operator, relation, punctuation, etc., as will be discussed in
  \ref{math-spacing}), the overlapping symbol, horizontal and
  vertical adjustments in symbol-relative units, and the character to
  be overlapped.  Consider, for example, the symbol for reduced
  quadrupole moment~(``$\rqm$'').  This can be declared as follows:

\begin{verbatim}
    \newcommand{\rqm}{{%
      \declareslashed{}{\text{-}}{0.04}{0}{I}\slashed{I}}}
\end{verbatim}

  \noindent
  \newcommand{\curlyarg}{\texttt{\char`\{}$\cdot$\texttt{\char`\}}}%

  \cmd{\declareslashed}\curlyarg\curlyarg\curlyarg\curlyarg\verb|{I}|
  affects the meaning of all subsequent \cmd{\slashed}\verb|{I}|
  commands in the same scope.  The preceding definition of \docAuxCommand{rqm}
  therefore uses an extra set of curly braces to limit that scope to a
  single \cmd{\slashed}\verb|{I}|.  In addition, \docAuxCommand{rqm} uses
  \pkgname{amstext}'s \cmd{\text} macro
  (described~\vpageref[below]{text-macro}) to make
  \cmd{\declareslashed} use a text-mode hyphen~(``-'') instead of a
  math-mode minus sign~(``$-$'') and to ensure that the hyphen scales
  properly in size in subscripts and superscripts.
\fi  

See \pkgname{slashed}'s documentation (located in
\docfilename{slashed.sty} itself) for a detailed usage description of the
\cmd{\slashed} and \cmd{\declareslashed} commands.

Somewhat simpler than \pkgname{slashed} is the \pkgname{centernot}
package.  \pkgname{centernot} provides a single command,
\cmd{\centernot}, which, like \cmd{\not}, puts a slash over the
subsequent mathematical symbol.  However, instead of putting the slash
at a fixed location, \cmd{\centernot} centers the slash over its
argument.  \cmd{\centernot} might be used, for example, to create a
``does\index{does not imply} not imply'' symbol%

\ifhavecenternot
%   \begin{center}
%    \renewcommand{\arraystretch}{1.25}%
%    \begin{tabular}{cl}
%      $\not\Longrightarrow$       & \verb|\not\Longrightarrow| \\
%      \multicolumn{2}{c}{vs.} \\
%      $\centernot\Longrightarrow$ & \verb|\centernot\Longrightarrow| \\
%    \end{tabular}
%  \end{center}
\else
  .
\fi   
\seedocs{\pkgname{centernot}}


\subsection{How to make new symbols work in superscripts and subscripts}

\index{subscripts>new symbols used in|(}
\index{superscripts>new symbols used in|(}


To make composite symbols work properly within subscripts and
superscripts, you may need to use \tex's \cmd{\mathchoice} primitive.
\cmd{\mathchoice} evaluates one of four expressions, based on whether
the current math style is display, text, script, or scriptscript.
(See \TeXbook for a more complete description.)  For example, the
following \latex code---posted to \ctt by
\person{Torsten}{Bronger}---composes a sub/superscriptable
``\cmd{\topbot}'' symbol out of \docAuxCommand{top} and \docAuxCommand{bot} (``$\top$''
and ``$\bot$''):



\indexcommand{\displaystyle}%
\indexcommand{\textstyle}%
\indexcommand{\scriptstyle}%
\indexcommand{\scriptscriptstyle}%
\label{code:topbot}%

\begin{verbatim}
   \def\topbotatom#1{\hbox{\hbox to 0pt{$#1\bot$\hss}$#1\top$}}
   \newcommand*{\topbot}{\mathrel{\mathchoice{\topbotatom\displaystyle}
                                    {\topbotatom\textstyle}
                                    {\topbotatom\scriptstyle}
                                    {\topbotatom\scriptscriptstyle}}}
\end{verbatim}
\index{superscripts>new symbols used in|)}
\index{subscripts>new symbols used in|)}

\begin{texexample}{mathchoice}{ex:mathchoice}
\bgroup
\def\topbotatom#1{\hbox{\hbox to 0pt{$#1\bot$\hss}$#1\top$}}
   \def\topbot{\mathrel{\mathchoice{\topbotatom\displaystyle}
                                    {\topbotatom\textstyle}
                                    {\topbotatom\scriptstyle}
                                    {\topbotatom\scriptscriptstyle}}}
\[ a_{\topbot} + b^{\topbot} \]
\egroup
\end{texexample}


\subsection{Modifying \latex-generated symbols}

\index{dots (ellipses)|(}
\index{ellipses (dots)|(}
\index{dot symbols|(}
\index{symbols>dot|(}

Oftentimes, symbols composed in the \latexe source code can be
modified with minimal effort to produce useful variations.  For
example, \fontdefdtx composes the \docAuxCommand{ddots} symbol (see
\vref{dots}) out of three periods, raised~7\,pt., 4\,pt., and
1\,pt., respectively:

\begin{verbatim}
   \def\ddots{\mathinner{\mkern1mu\raise7\p@
       \vbox{\kern7\p@\hbox{.}}\mkern2mu
       \raise4\p@\hbox{.}\mkern2mu\raise\p@\hbox{.}\mkern1mu}}
\end{verbatim}

\noindent
\cmd{\p@} is a \latexe{} shortcut for ``\texttt{pt}'' or
``\texttt{1.0pt}''.  The remaining commands are defined in \TeXbook.
To\label{revddots} draw a version of \docAuxCommand{ddots} with the dots going
along the opposite diagonal, we merely have to reorder the
\verb|\raise7\p@|, \verb|\raise4\p@|, and \verb|\raise\p@|:

\begin{texexample}{revddots}{ex:revddots}
\makeatletter
   \def\revddots{\mathinner{\mkern1mu\raise\p@
      \vbox{\kern7\p@\hbox{.}}\mkern2mu
       \raise4\p@\hbox{.}\mkern2mu\raise7\p@\hbox{.}\mkern1mu}}
\makeatother

\[\revddots \]
\end{texexample}


    \makeatletter
      \def\revddots{\mathinner{\mkern1mu\raise\p@
        \vbox{\kern7\p@\hbox{.}}\mkern2mu
        \raise4\p@\hbox{.}\mkern2mu\raise7\p@\hbox{.}\mkern1mu}}
    \makeatother
\indexcommand[$\string\revddots$]{\revddots}

\noindent
\docAuxCommand{revddots} is essentially identical to the \MDOTS\
package's
\ifMDOTS
  \docAuxCommand{iddots}
\else
  \cmd{\iddots}
\fi
command or the \YH\ package's
%\ifYH
%  \docAuxCommand{adots}
%\else
  \cmd{\adots}
%\fi
command.
\index{symbols>dot|)}
\index{dot symbols|)}
\index{ellipses (dots)|)}
\index{dots (ellipses)|)}




\section{ASCII and Latin~1 quick reference}
\label{ascii-quickref}

\index{ASCII|(}

\vref{ascii-table} amalgamates data from various other tables in this
document into a convenient reference for \latexe typesetting of \texttt{ascii}
characters, i.e., the characters available on a typical U.S. computer
keyboard.  The first two columns list the character's \texttt{ascii} code in
decimal and hexadecimal.  The third column shows what the character
looks like.  The fourth column lists the \latexe command to typeset
the character as a text character.  And the fourth column lists the
\latexe command to typeset the character within a
\verb|\texttt{|$\ldots$\verb|}| command (or, more generally, when
\verb|\ttfamily| is in effect).


\index{ASCII|)}

\begin{nonsymtable}{\latexe ASCII Table}
  \index{ASCII>table}
  \label{ascii-table}
  ^^A Define an equivalent of \vdots that's the height of a "9".
  \newlength{\digitheight}
  \settoheight{\digitheight}{9}
  \newcommand{\digitvdots}{\raisebox{-1.5pt}[\digitheight]{$\vdots$}}

 ^^A Replace all glyphs in a row with vertical dots.
  \makeatletter
  \newcommand{\skipped}{%
    \settowidth{\@tempdima}{99} \makebox[\@tempdima]{\digitvdots} &
    \settowidth{\@tempdima}{99} \makebox[\@tempdima]{\digitvdots} &
    \digitvdots &
    \digitvdots &
    \digitvdots \\
  }
  \makeatother

  ^^A Typesetting a symbol by prefixing it with a "\".
  \newcommand{\bscommand}[1]{#1 & \cmd{#1} & \cmd{#1}}

  \begin{tabular}[t]{@{}*2{>{\ttfamily}r}c*2{>{\ttfamily}l}l@{}} \\ \toprule
    \multicolumn{1}{@{}c}{Dec} &
    \multicolumn{1}{c}{Hex} &
    \multicolumn{1}{c}{Char} &
    \multicolumn{1}{c}{Body text} &
    \multicolumn{1}{c@{}}{\ttfamily\string\texttt} \\ \midrule

    33 & 21 & ! & ! & ! \\
    34 & 22 & {\fontencoding{T1}\selectfont\textquotedbl} &
      \string\textquotedbl & " \\      ^^A Not available in OT1
    35 & 23 & \bscommand{\#} \\
    36 & 24 & \bscommand{\$} \\
    37 & 25 & \bscommand{\%} \\
    38 & 26 & \bscommand{\&} \\
    39 & 27 & ' & ' & ' \\
    40 & 28 & ( & ( & ( \\
    41 & 29 & ) & ) & ) \\
    42 & 2A & * & * & * \\
    43 & 2B & + & + & + \\
    44 & 2C & , & , & , \\
    45 & 2D & - & - & - \\
    46 & 2E & . & . & . \\
    47 & 2F & / & / & / \\
    48 & 30 & 0 & 0 & 0 \\
    49 & 31 & 1 & 1 & 1 \\
    50 & 32 & 2 & 2 & 2 \\
    \skipped
    57 & 39 & 9 & 9 & 9 \\
    58 & 3A & : & : & : \\
    59 & 3B & ; & ; & ; \\
    60 & 3C & \textless & \docAuxCommand{textless} & < \\       ^^A Or $<$
    61 & 3D & = & = & = \\ \bottomrule
  \end{tabular}
  \hfil
  \begin{tabular}[t]{@{}*2{>{\ttfamily}r}c*2{>{\ttfamily}l}l@{}} \\ \toprule
    \multicolumn{1}{@{}c}{Dec} &
    \multicolumn{1}{c}{Hex} &
    \multicolumn{1}{c}{Char} &
    \multicolumn{1}{c}{Body text} &
    \multicolumn{1}{c@{}}{\ttfamily\string\texttt} \\ \midrule

    62 & 3E & \textgreater & \docAuxCommand{textgreater} & > \\   
    63 & 3F & ? & ? & ? \\
    64 & 40 & @ & @ & @ \\
    65 & 41 & A & A & A \\
    66 & 42 & B & B & B \\
    67 & 43 & C & C & C \\
    \skipped
    90 & 5A & Z & Z & Z \\
    91 & 5B & [ & [ & [ \\
    92 & 5C & \textbackslash & \docAuxCommand{textbackslash} &
      \verb|\char`\\| \\   ^^A \textbackslash works in non-OT1
    93 & 5D & ] & ] & ] \\
    94 & 5E & \^{} & \verb|\^{}| & \verb|\^{}| \\   ^^A Or \textasciicircum
    95 & 5F & \_ & \verb|\_| & \verb|\char`\_| \\   ^^A \_ works in non-OT1
    96 & 60 & ` & ` & ` \\
    97 & 61 & a & a & a \\
    98 & 62 & b & b & b \\
    99 & 63 & c & c & c \\
    \skipped
   122 & 7A & z & z & z \\
   123 & 7B & \{ & \verb|\{| & \verb|\char`\{| \\   
   124 & 7C & \textbar & \docAuxCommand{textbar} & \textbar \\    
   125 & 7D & \} & \verb|\}| & \verb|\char`\}| \\   
   126 & 7E & \~{} & \verb|\~{}| & \verb|\~{}| \\   
   \\
   \bottomrule
  \end{tabular}
\end{nonsymtable}

The following are some additional notes about the contents of
\ref{ascii-table}:

\begin{itemize}
  \item
  ``\indexcommand[\string\encone{\string\textquotedbl}]{\textquotedbl}{\encone{\textquotedbl}}''
  is not available in the OT1 \fntenc[OT1].

  \item \ref{ascii-table} shows a close quote for character~39 for
    consistency with the open quote shown for character~96.  A
    straight quote can be typeset using \docAuxCommand{textquotesingle}
    (cf.~\ref{tc-misc}).

  \item
  The\label{upside-down}\index{symbols>upside-down|(}\index{upside-down
  symbols|(} characters ``\texttt{<}'', ``\texttt{>}'', and
  ``\texttt{\textbar}'' do work as expected in math mode, although they
  produce, respectively, ``<'', ``>'', and ``\textbar'' in text mode when
  using the OT1 \fntenc[OT1].\footnote{Donald\index{Knuth, Donald E.}
  Knuth didn't think such symbols were important outside of
  mathematics so he omitted them from his text fonts.} The following
  are some alternatives for typesetting ``\textless'',
  ``\textgreater'', and ``\textbar'':

  \begin{itemize}
    \item Specify a document \fntenc{} other than OT1 (as
    described~\vpageref[above]{altenc}).

    \item Use the appropriate symbol commands from
    \vref{text-predef}, viz.~\docAuxCommand{textless},
    \docAuxCommand{textgreater}, and \docAuxCommand{textbar}.

    \item Enter the symbols in math mode instead of text mode,
    i.e.,~\verb+$<$+, \verb+$>$+, and \verb+$|$+.
  \end{itemize}

  \noindent
  Note that for typesetting metavariables many people prefer
  \docAuxCommand{textlangle} and \docAuxCommand{textrangle} to \docAuxCommand{textless} and
  \docAuxCommand{textgreater}; i.e., ``\meta{filename}'' instead of
  ``$<$\textit{filename}$>$''.\index{symbols>upside-down|)}\index{upside-down
  symbols|)}

  \item Although ``\texttt{/}'' does not require any special
  treatment, \latex additionally defines a \docAuxCommand{slash} command which
  outputs the same glyph but permits a line~break afterwards.  That
  is, ``\texttt{increase/decrease}'' is always typeset as a single
  entity while ``\verb|increase\slash{}decrease|'' may be typeset with
  ``increase/'' on one line and ``decrease'' on the next.

  \item \label{page:tildes} \index{tilde|(} \docAuxCommand{textasciicircum}
  can be used instead of 
 % \cmdI[\string\^{}]{\^{}}\verb|{}|, 
  and
  \docAuxCommand{textasciitilde} can be used instead of
%  \cmdI[\string\~{}]{\~{}}\verb|{}|.  Note that
  \docAuxCommand{textasciitilde} and 
  %\cmdI[\string\~{}]{\~{}}\verb|{}|
  produce raised, diacritic tildes.  ``Text''
  (i.e.,~vertically\index{tilde>vertically centered} centered)
  tildes can be generated with either the math-mode \docAuxCommand{sim}
  command (shown in \vref{rel}), which produces a somewhat wide
  ``$\sim$'', or the \TC\ package's \docAuxCommand{texttildelow} (shown in
  \vref{tc-misc}), which produces a vertically centered
  ``{\fontfamily{ptm}\selectfont\texttildelow}'' in most fonts but a
  baseline-oriented ``\texttildelow'' in \PSfont{Computer Modern},
  \TX, \PX, and various other fonts originating from the
  \tex\ world.  If your goal is to typeset tildes in URLs or Unix
  filenames, your best bet is to use the \pkgname{url} package,
  which has a number of nice features such as proper line-breaking
  of such names.\index{tilde|)}

  \item The various \cmd{\char} commands within \verb|\texttt| are
  necessary only in the OT1 \fntenc[OT1].  In other encodings
  (e.g.,~T1)\index{font encodings>T1}, commands such as 
%  \cmdIp{\{},
  %\cmdIp{\}}, \
  %docAuxCommand{_}, 
  and \docAuxCommand{textbackslash} all work properly.

  \item The code\index{code page 437} page~437 (IBM~PC\index{IBM PC})
  version of \texttt{ascii} characters~1 to~31 can be typeset
  using the \ASCII\ package.
\ifASCII
  See \vref{ibm-ascii}.
\fi

  \item To replace~``\verb|`|'' and~``\verb|'|'' with the more
  computer-like (and more visibly distinct) ``\texttt{\char18}''
  and~``\texttt{\char13}'' within a \texttt{verbatim} environment,
  use the \pkgname{upquote} package.  Outside of \texttt{verbatim},
  you can use \cmd{\char}\texttt{18} and \cmd{\char}\texttt{13} to
  get the modified quote characters.  (The former is actually a
  grave accent.)
\end{itemize}





\subsection{Unicode characters}
\label{unicode-chars}

\index{Unicode|(}

\href{http://www.unicode.org/}{Unicode} is a ``universal character
set''---a standard for encoding (i.e.,~assigning unique numbers to)
the symbols appearing in many of the world's languages.  While \texttt{ascii}
can represent 128 symbols and Latin~1 can represent 256 symbols,
Unicode can represent an astonishing 1,114,112 symbols.

Because \tex and \latex{} predate the Unicode standard and Unicode
fonts by almost a decade, support for Unicode has had to be added to
the base \tex{} and \latex{} systems.  Note first that \latex{}
distinguishes between \emph{input} encoding---the characters used in
the \texttt{.tex} file---and \emph{output} encoding---the characters
that appear in the generated \texttt{.dvi}, \texttt{.pdf}, etc.\ file.
For a discusiion on Unicode for Mathematics see \citep{beetona}.

\begin{texexample}{How to add symbols}{unicodesymbols}
\ifxetex
  \newfontfamily{\codetwothousand}{code2000.ttf}
  \codetwothousand\char"1F050 \char"2603\char"2617
  \symbol{9825}
  \newfontfamily{\codetwothousandone}{code2001.ttf}
  \newfontfamily{\symbola}{symbola.ttf}
  {\codetwothousand \symbol{9742} \symbol{9743}
    Katakana (片仮名, カタカナ)
   \codetwothousandone \symbol{57508}
   \symbola \symbol{9816}
   
  }
\else
   Compile the document with XeTeX to see the example
\fi
\end{texexample}

\subsubsection{Inputting Unicode characters}

To include Unicode characters in a \texttt{.tex} file, load the
\pkgname{ucs} package and load the \pkgname{inputenc} package with the
\optname{inputenc}{utf8x} (``\utfviii extended'')
option.\footnote{\utfviii is the 8-bit Unicode Transformation Format,
  a popular mechanism for representing Unicode symbol numbers as
  sequences of one to four bytes.}  These packages enable \latex{} to
translate \utfviii sequences to \latex{} commands, which are
subsequently processed as normal.  For example, the \utfviii text
``\texttt{Copyright~\textcopyright\ \the\year}''---``\texttt{\textcopyright}''
is not an \texttt{ascii} character and therefore cannot be input directly
without packages such as \pkgname{ucs}/\pkgname{inputenc}---is
converted internally by \pkgname{inputenc} to ``\texttt{Copyright}
\verb+\textcopyright{}+ \texttt{\the\year}'' and therefore typeset as
``Copyright~\textcopyright\ \the\year''.

The \pkgname{ucs}\slash\pkgname{inputenc} combination supports only a
tiny subset of Unicode's million-plus symbols.  Additional symbols can
be added manually using the \cmd{\DeclareUnicodeCharacter} command.
\cmd{\DeclareUnicodeCharacter} takes two arguments: a Unicode number
and a \latex{} command to execute when the corresponding Unicode
character is encountered in the input.  For example, the Unicode
character ``degree celsius''~(``\,\textcelsius\,'') appears at
character position U+2103.\footnote{The Unicode convention is to
  express character positions as ``U+\meta{hexadecimal number}''.}
However, ``\,\texttt{\textcelsius}\,'' is not one of the characters
that \pkgname{ucs} and \pkgname{inputenc} recognize.  

The following
document shows how to use \cmd{\DeclareUnicodeCharacter} to tell
\latex{} that the ``\,\texttt{\textcelsius}\,'' character should be
treated as a synonym for \docAuxCommand{textcelsius}:

\begin{verbatim}
   \documentclass{article}
   \usepackage{ucs}
   \usepackage[utf8x]{inputenc}
   \usepackage{textcomp}

   \DeclareUnicodeCharacter{"2103}{\textcelsius} % Enable direct input of U+2103.
\end{verbatim}
\noindent
\verb|   \begin{document}| \\
\verb|   |\texttt{It was a balmy 21\textcelsius.} \\
\verb|   \end{document}|

\medskip

\noindent
which produces

\begin{quotation}
  It was a balmy 21\textcelsius.
\end{quotation}

\seedocs{\pkgname{ucs}} and for descriptions of the various options that control \pkgname{ucs}'s behavior.


\subsection{Outputting Unicode characters}

Orthogonal to the ability to include Unicode characters in a
\latex\ input file is the ability to include a given Unicode character
in the corresponding output file.  By far the easiest approach is to
use \xelatex instead of pdf\LaTeX\index{pdfLaTeX=pdf\LaTeX} or
ordinary \latex.  \xelatex handles Unicode input and output natively
and can utilize system fonts directly without having to expose them
via \texttt{.tfm}, \texttt{.fd}, and other such files.  To output a
Unicode character, a \xelatex document can either include that
character directly as \utfviii text or use \tex's \cmd{\char}
primitive, which \xelatex extends to accept numbers larger than~255.

\DeclareRobustCommand{\trafficsign}{\includegraphics[height=10pt]{./images/traffic-sign-01.png}
}
Suppose we need to declare a traffic sign \trafficsign and for which we have some images ready.


\newfontfamily{\codetwothousand}{code2000.ttf}
\newfontfamily{\codetwothousandone}{code2001.ttf}
  \newfontfamily{\symbola}{symbola.ttf}
  
\DeclareRobustCommand{\versicle}{%
  \raisebox{-2.2bp}{\includegraphics{./images/versicle.jpg}}\kern-1pt}
\DeclareRobustCommand{\response}{%
  \raisebox{-1.2bp}{\includegraphics{./images/response.jpg}}\kern-1pt}
\newcommand{\versicleIDX}{\index{versicle=versicle (\versicle)}}
\newcommand{\responseIDX}{\index{response=response (\response)}}

Suppose we want to output the symbols for
versicle\versicleIDX~(``\versicle'') and
response\responseIDX~(``\response'') in a document.  The Unicode
charts list ``versicle\versicleIDX'' at position~U+2123 ({\codetwothousand\char"2123}) and
``response\responseIDX'' at position~U+211F ({\codetwothousand\char"211F}).  We therefore need to
install a font that contains those characters at their proper
positions.  One such font that is freely available from CTAN\idxCTAN{}
is Junicode Regular (\docfilename{Junicode-Regular.ttf}) from the
\pkgname{junicode} package.  

The \pkgname{fontspec} package makes it
easy for a \xelatex or \lualatex document to utilize a system font.  The following
example defines a \texttt{\string\textjuni} command that uses
\pkgname{fontspec} to typeset its argument in Junicode Regular:

\begin{verbatim}
   \documentclass{article}
   \usepackage{fontspec}

   \newcommand{\textjuni}[1]{{\fontspec{Junicode-Regular}#1}}

   \begin{document}
   We use ``\textjuni{\char"2123}'' for a versicle
   and ``\textjuni{\char"211F}'' for a response.
   \end{document}
\end{verbatim}

\noindent
which produces

\begin{quotation}
  We use ``\versicle'' for a versicle\versicleIDX\ and ``\response''
  for a response\responseIDX.
\end{quotation}

\noindent
(Typesetting the entire document in Junicode Regular would be even
easier.  \seedocs{\pkgname{fontspec}} regarding font selection.)  Note
how the preceding example uses \cmd{\char} to specify a Unicode
character by number.  The double quotes before the number indicate
that the number is represented in hexadecimal instead of decimal.

\index{Unicode|)}

\section{XeLaTeX and fontspec}

\index{maths>fontspec}
The best option so far for math fonts using XeLaTeX and \pkgname{fontspec} is to use the option |no-math|. When typesetting this document for example there were numerous problems with accents (I lost the |ring| accent, until I used this option. Unicode math fonts are not available in large numbers.



% Because the Math Alphabets table is a bit different from the symbol
% tables in this document we start it on its own page to emphasize it
% and to include enough room for some of the table notes.
\clearpage

\begin{symtable}{Math Alphabets}
\idxboth{math}{alphabets}
\label{alphabets}
\begin{tabular}{@{}*3l@{}}
\toprule
Font sample & Generating command & Required package           \\
\midrule
\Wf\mathrm{ABCabc123}    & \textit{none}                      \\
\Ww\textit\mathit{ABCabc123}    & \textit{none}               \\
\Wf\mathnormal{ABCabc123}& \textit{none}                      \\
|\Ww\CMcal\mathcal{ABC}|   & \textit{none}                      \\

\ifx\mathscr\undefined\else
\Wf\mathscr{ABC}         & \pkgname{mathrsfs} \\
\multicolumn{1}{r@{}}{\emph{or}}
        &\verb|\mathcal{ABC}|
                         & \pkgname{calrsfs} \\
\fi
%
%\ifEU
%\Wf\mathcal{ABC}         & \pkgname{euscript} with the
%                           \optname{euscript}{mathcal} option \\
%\multicolumn{1}{r@{}}{\emph{or}}
%        &\verb|\mathscr{ABC}|
%                         & \pkgname{euscript} with the
%                           \optname{euscript}{mathscr} option \\
%\fi

\ifx\mathpzc\undefined\else
\Wf\mathpzc{ABCdef123}   & \textit{none}; manually defined$^*$    \\
\fi

\ifx\mathbb\undefined\else
\Wf\mathbb{ABC}          & \pkgname{amsfonts},%
                           \ifx\MSYMmathbb\undefined\else$^\S$~\fi
                           \pkgname{amssymb}, \pkgname{txfonts}, or
                           \pkgname{pxfonts} \\
\fi

\ifx\varmathbb\undefined\else
\Wf\varmathbb{ABC}       & \pkgname{txfonts} or \pkgname{pxfonts} \\
\fi

%\ifx\BBmathbb\undefined\else
%\Ww\BBmathbb\mathbb{ABCdef123}
%                         & \pkgname{bbold} or \pkgname{mathbbol}$^\dag$  \\
%\fi
%
%\ifx\MBBmathbb\undefined\else
%\Ww\MBBmathbb\mathbb{ABCdef123}
%                         & \pkgname{mbboard}$^\dag$              \\
%\fi

%\ifx\mathbbm\undefined\else
%\Wf\mathbbm{ABCdef12}    & \pkgname{bbm}                         \\
%\Wf\mathbbmss{ABCdef12}  & \pkgname{bbm}                         \\
%\Wf\mathbbmtt{ABCdef12}  & \pkgname{bbm}                         \\
%\fi

\ifx\mathds\undefined\else
\Wf\mathds{ABC1}         & \pkgname{dsfont}                      \\
\Ww\mathdsss\mathds{ABC1}
                         & \pkgname{dsfont} with the
                           \optname{dsfont}{sans} option         \\
\fi

\ifx\symA\undefined\else
\symA\symB\symC & \docAuxCommand{symA}\docAuxCommand{symB}\docAuxCommand{symC}
                         & \pkgname{china2e}$^\ddag$             \\
\fi

\ifx\mathfrak\undefined\else
\Wf\mathfrak{ABCdef123}  & \pkgname{eufrak}                      \\
\fi

\ifx\textfrak\undefined\else
\Wf\textfrak{ABCdef123}  & \pkgname{yfonts}$^\P$                 \\
\Wf\textswab{ABCdef123}  & \pkgname{yfonts}$^\P$                 \\
\Wf\textgoth{ABCdef123}  & \pkgname{yfonts}$^\P$                 \\
\fi
\bottomrule
\end{tabular}
\end{symtable}
\unskip



\begin{center}
\ifx\mathpzc\undefined\else
\bigskip
\begin{tablenote}[*]
  Put ``\verb|\DeclareMathAlphabet{\mathpzc}{OT1}{pzc}{m}{it}|'' in your
  document's preamble to make \verb|\mathpzc| typeset its argument in
  \PSfont{Zapf Chancery}.
\ifx\textcalligra\undefined\else
  As a similar trick, you can typeset the \PSfont{Calligra} font's
  script ``{\Large\textcalligra{r}\,}'' (or other calligraphic symbols)
  in math mode by loading the \pkgname{calligra} package and putting
  ``\verb|\DeclareMathAlphabet{\mathcalligra}{T1}{calligra}{m}{n}|''
  in your document's preamble to make \verb|\mathcalligra| typeset its
  argument in the \PSfont{Calligra} font.  (You may also want to
  specify
  ``\verb|\DeclareFontShape{T1}{calligra}{m}{n}{<->s*[2.2]callig15}{}|''
  to set \PSfont{Calligra} at 2.2~times its design size for a better
  blend with typical body fonts.)
\fi
\end{tablenote}
\fi

\ifx\BBmathbb\undefined\else
\bigskip
\begin{tablenote}[\dag]
  The \pkgname{mathbbol} package defines some additional blackboard bold
  characters: parentheses, square brackets, angle brackets, and---if
  the \optname{mathbbol}{bbgreekl} option is passed to
  \pkgname{mathbbol}---Greek\index{Greek>blackboard bold} letters.  For
  instance,
%  ``$\BBmathbb{\char`<\char`[\char`(\char"0B\char"0C\char"0D\char`)\char`]\char`>}$''
%  is produced by
%  ``\cmd{\mathbb}\verb|{|\docAuxCommand{Langle}\linebreak[1]%
%  \docAuxCommand{Lbrack}\linebreak[1]\docAuxCommand{Lparen}\linebreak[1]%
%  \docAuxCommand{bbalpha}\linebreak[1]\docAuxCommand{bbbeta}\linebreak[1]%
%  \docAuxCommand{bbgamma}\linebreak[1]\docAuxCommand{Rparen}\linebreak[1]%
%  \docAuxCommand{Rbrack}\linebreak[1]\docAuxCommand{Rangle}\verb|}|''.

  \ifx\MBBmathbb\undefined
    \pkgname{mbboard} extends the blackboard bold symbol set
    significantly further.  It supports not only the
    Greek\index{Greek>blackboard bold}\index{alphabets>Greek}
    alphabet---including ``Greek-like'' symbols such as
    \cmd{\bbnabla}---but also \emph{all} punctuation marks, various
    currency\idxboth{currency}{symbols}\idxboth{monetary}{symbols}
    symbols such as \cmd{\bbdollar} and \cmd{\bbeuro},\index{euro
    signs>blackboard bold} and the
    Hebrew\index{Hebrew}\index{alphabets>Hebrew} alphabet.
  \else
    \pkgname{mbboard} extends the blackboard bold symbol set
    significantly further.  It supports not only the
    Greek\index{Greek>blackboard bold}\index{alphabets>Greek}
    alphabet---including ``Greek-like'' symbols such as
    \docAuxCommand{bbnabla}~(``\bbnabla'')---but also \emph{all} punctuation
    marks, various
    currency\idxboth{currency}{symbols}\idxboth{monetary}{symbols}
    symbols such as \docAuxCommand{bbdollar}~(``\bbdollar'') and
    \docAuxCommand{bbeuro}~(``\bbeuro''),\index{euro signs>blackboard bold}
    and the Hebrew\index{Hebrew}\index{alphabets>Hebrew}
    alphabet~(e.g.,~``\docAuxCommand{bbfinalnun}\linebreak[1]\docAuxCommand{bbyod}%
    \linebreak[1]\docAuxCommand{bbqof}\linebreak[1]\docAuxCommand{bbpe}''~$\rightarrow$
    ``\bbfinalnun\bbyod\bbqof\bbpe'').
  \fi    t
\end{tablenote}
\fi

\ifx\symA\undefined\else
\bigskip
\begin{tablenote}[\ddag]
  The \verb|\sym|\dots\ commands provided by the \pkgname{package} are
  actually text-mode commands.  They are included in \ref{alphabets}
  because they resemble the blackboard-bold symbols that appear in the
  rest of the table.  In addition to the 26 letters of the English
  alphabet, \CHINA\ provides three umlauted%
  \index{accents>diaeresis=di\ae{}resis (\blackacchack\")}  % 
  blackboard-bold letters:
  \docAuxCommand{symAE}~(``\symAE''), \docAuxCommand{symOE}~(``\symOE''), and
  \docAuxCommand{symUE}~(``\symUE'').  Note that \CHINA\ does provide
  math-mode commands for the most common number-set symbols.  These
  are presented in \vref{china-numsets}.
\end{tablenote}
\fi

\ifx\textfrak\undefined\else
\bigskip
\begin{tablenote}[\P]
  As their \verb|\text|\dots{} names imply, the fonts provided by the
  \pkgname{yfonts} package are actually text fonts.  They are
  included in \ref{alphabets} because they are frequently used
  in a mathematical context.
\end{tablenote}
\fi

\ifx\MSYMmathbb\undefined\else
\bigskip
\begin{tablenote}[\S]
  An older (i.e.,~prior to~1991) version of the \AMS's fonts rendered
  $\mathbb{C}$, $\mathbb{N}$, $\mathbb{R}$, $\mathbb{S}$,
  and~$\mathbb{Z}$ as $\MSYMmathbb{C}$, $\MSYMmathbb{N}$,
  $\MSYMmathbb{R}$, $\MSYMmathbb{S}$, and~$\MSYMmathbb{Z}$.  As some
  people prefer the older glyphs---much to the \AMS's surprise---and
  because those glyphs fail to build under modern versions of
  \metafont, \person{Berthold}{Horn} uploaded \postscript fonts for
  the older blackboard-bold glyphs to CTAN\idxCTAN{}, to the
  \texttt{fonts/msym10} directory.  As of this writing, however, there
  are no \latexE packages for utilizing the now-obsolete glyphs.
\end{tablenote}
\fi
\end{center}


\idxbothend{mathematical}{symbols}


\bgroup
\renewcommand\arraystretch{1.4}
\newcommand\leg[1]{{\tiny\tt\char92#1}}
\newcommand\sho[1]{{\large #1}}
\begin{tabular}{|*{10}{c}|} \hline
\leg{Pickup} &
\leg{Letter} & 
\leg{Mobilefone} &
\leg{Telefon} &
\leg{fax} &
\leg{FAX} &
\leg{Faxmachine} &
\leg{Email} &
\leg{Lightning} &
\leg{EmailCT} \\
\sho{\Pickup} &
\sho{\Letter} &
\sho{\Mobilefone} &
\sho{\Telefon} &
\sho{\fax} &
\sho{\FAX} &
\sho{\Faxmachine} &
\sho{\Email} &
\sho{\Lightning} &
\sho{\EmailCT} \\
\hline
\end{tabular}

\begin{tabular}{|*{8}{c}|} \hline
\leg{Beam} &
\leg{Bearing} &
\leg{LooseBearing} &
\leg{FixedBearing} &
\leg{LeftTorque} &
\leg{RightTorque} &
\leg{Lineload} &
\leg{MVArrowDown} \\
\sho{\Beam} &
\sho{\Bearing} &
\sho{\LooseBearing} &
\sho{\FixedBearing} &
\sho{\LeftTorque} &
\sho{\RightTorque} &
\sho{\Lineload} &
\sho{\MVArrowDown} \\
\hline
\leg{OktoSteel} &
\leg{HexaSteel} &
\leg{SquareSteel} & 
\leg{RectSteel} &
\leg{CircSteel} &
\leg{SquarePipe} &
\leg{RectPipe} &
\leg{CircPipe}
\\
\sho{\OktoSteel} &
\sho{\HexaSteel} &
\sho{\SquareSteel} &
\sho{\RectSteel} &
\sho{\CircSteel} &
\sho{\SquarePipe} &
\sho{\RectPipe} &
\sho{\CircPipe}
\\ \hline
\leg{LSteel} &
\leg{RoundedLSteel} &
\leg{TSteel} &
\leg{RoundedTSteel} &
\leg{TTsteel} &
\leg{RoundedTTSteel} &
\leg{FlatSteel} &
\leg{Valve}
\\
\sho{\LSteel} &
\sho{\RoundedLSteel} &
\sho{\TSteel} &
\sho{\RoundedTSteel} &
\sho{\TTSteel} &
\sho{\RoundedTTSteel} &
\sho{\FlatSteel} &
\sho{\Valve}
\\ \hline
\end{tabular}

\subsection{Information}

\begin{tabular}{|*{8}{c}|} \hline
\leg{Industry} &
\leg{Coffeecup} &
\leg{LeftScissors} &
\leg{CuttingLine} &
\leg{RightScissors} &
\leg{Football} &
\leg{Bicycle} & \\
\sho{\Industry} &
\sho{\Coffeecup} &
\sho{\LeftScissors} &
\sho{\CuttingLine} &
\sho{\RightScissors} &
\sho{\Football} &
\sho{\Bicycle} & \\
\hline
\leg{Info} &
\leg{ClockLogo} &
\leg{CutRight} &
\leg{CutLineine} &
\leg{CutLeft} &
\leg{Wheelchair} &
\leg{Gentsroom} &
\leg{Ladiesroom} \\
\sho{\Info} &
\sho{\ClockLogo} &
\sho{\CutRight} &
\sho{\CutLine} &
\sho{\CutLeft} &
\sho{\Wheelchair} &
\sho{\Gentsroom} &
\sho{\Ladiesroom} \\
\hline
\leg{Checkedbox} &
\leg{CrossedBox} &
\leg{HollowBox} &
\leg{PointingHand} &
\leg{WritingHand} &
\leg{MineSign} &
\leg{Recycling} &
\leg{PackingWaste} \\
\sho{\Checkedbox} &
\sho{\CrossedBox} &
\sho{\HollowBox} &
\sho{\PointingHand} &
\sho{\WritingHand} &
\sho{\MineSign} &
\sho{\Recycling} &
\sho{\PackingWaste} \\
\hline
\end{tabular}

\subsection{Laundry}

\begin{tabular}{|*{8}{c}|} \hline
\leg{WashCotton} &
\leg{WashSynthetics} &
\leg{WashWool} &
\leg{HandWash} &
\leg{NoWash} &
\leg{Tumbler} &
\leg{NoTumbler} &
\leg{NoChemicalCleaning} \\
\sho{\WashCotton} &
\sho{\WashSynthetics} &
\sho{\WashWool} &
\sho{\HandWash} &
\sho{\NoWash} &
\sho{\Tumbler} &
\sho{\NoTumbler} &
\sho{\NoChemicalCleaning} \\
\hline
\leg{Bleech} &
\leg{NoBleech} &
\leg{CleaningA} &
\leg{CleaningP} &
\leg{CleaningPP} &
\leg{CleaningF} &
\leg{CleaningFF} & \\
\sho{\Bleech} &
\sho{\NoBleech} &
\sho{\CleaningA} &
\sho{\CleaningP} &
\sho{\CleaningPP} &
\sho{\CleaningF} &
\sho{\CleaningFF} & \\
\hline
\leg{IroningI} &
\leg{IroningII} &
\leg{IroningIII} &
\leg{NoIroning} &
\leg{AtNinetyFive} &
\leg{ShortNinetyFive} &
\leg{AtSixty} &
\leg{ShortSixty} \\
\sho{\IroningI} &
\sho{\IroningII} &
\sho{\IroningIII} &
\sho{\NoIroning} &
\sho{\AtNinetyFive} &
\sho{\ShortNinetyFive} &
\sho{\AtSixty} &
\sho{\ShortSixty} \\
\hline
\leg{ShortFifty} &
\leg{AtForty} &
\leg{ShortForty} &
\leg{SpecialForty} &
\leg{ShortThirty} &&& \\
\sho{\ShortFifty} &
\sho{\AtForty} &
\sho{\ShortForty} &
\sho{\SpecialForty} &
\sho{\ShortThirty} &&& \\
\hline
\end{tabular}

\subsection{Currency}

\begin{tabular}{|*{11}{c}|} \hline
\leg{EUR} &
\leg{EURdig} &
\leg{EURhv} &
\leg{EURcr} &
\leg{EURtm} &
\leg{Ecommerce} &
\leg{Shilling} &
\leg{Denarius} &
\leg{Pfund} &
\leg{EyesDollar} &
\leg{Florin} \\
 &
\leg{EurDig} &
\leg{EurHv} &
\leg{EurCr} &
\leg{EurTm} &
\leg{EstimatedSign} &
 &
\leg{Deleatur} &
 &
 &
 \\
\sho{\EUR} &
\sho{\EurDig} &
\sho{\EurHv} &
\sho{\EurCr} &
\sho{\EurTm} &
\sho{\EstimatedSign} &
\sho{\Shilling} &
\sho{\Deleatur} &
\sho{\Pfund} &
\sho{\EyesDollar} &
\sho{\Florin} \\
\hline
\end{tabular}
\label{currencysymbols} 

\subsection{Safety}

\begin{tabular}{|*{8}{c}|} \hline
\leg{Stopsign} &
\leg{CESign} &
\leg{Estatically} &
\leg{Explosionsafe} &
\leg{Laserbeam} &
\leg{Biohazard} &
\leg{Radioactivity} &
\leg{BSEFree} \\
\sho{\Stopsign} &
\sho{\CESign} &
\sho{\Estatically} &
\sho{\Explosionsafe} &
\sho{\Laserbeam} &
\sho{\Biohazard} &
\sho{\Radioactivity} &
\sho{\BSEFree} \\
\hline
\end{tabular}

\subsection{Navigation}

\begin{tabular}{|*{10}{c}|} \hline
\leg{RewindToIndex} &
\leg{RewindToStart} &
\leg{Rewind} &
\leg{Forward} &
\leg{ForwardToEnd} &
\leg{ForwardToIndex} &
\leg{MoveUp} &
\leg{MoveDown} &
\leg{ToTop} &
\leg{ToBottom} \\
\sho{\RewindToIndex} &
\sho{\RewindToStart} &
\sho{\Rewind} &
\sho{\Forward} &
\sho{\ForwardToEnd} &
\sho{\ForwardToIndex} &
\sho{\MoveUp} &
\sho{\MoveDown} &
\sho{\ToTop} &
\sho{\ToBottom} \\
\hline
\end{tabular}

\subsection{Computers}

\begin{tabular}{|*{6}{c}|} \hline
\leg{ComputerMouse} &
\leg{SerialInterface} &
\leg{Keyboard} &
\leg{SerialPort} &
\leg{ParallelPort} &
\leg{Printer} \\
\sho{\ComputerMouse} &
\sho{\SerialInterface} &
\sho{\Keyboard} &
\sho{\SerialPort} &
\sho{\ParallelPort} &
\sho{\Printer} \\
\hline
\end{tabular}

\subsection{Numbers}

\begin{tabular}{|*{10}{c}|} \hline
\leg{MVZero} &
\leg{MVOne} &
\leg{MVTwo} &
\leg{MVThree} &
\leg{MVFour} &
\leg{MVFive} &
\leg{MVSix} &
\leg{MVSeven} &
\leg{MVEight} &
\leg{MVNine} \\
\sho{\MVZero} &
\sho{\MVOne} &
\sho{\MVTwo} &
\sho{\MVThree} &
\sho{\MVFour} &
\sho{\MVFive} &
\sho{\MVSix} &
\sho{\MVSeven} &
\sho{\MVEight} &
\sho{\MVNine} \\
\hline
\end{tabular}

\subsection{Maths}

\begin{tabular}{|*{8}{c}|} \hline
\leg{MVLeftBracket} &
\leg{MVRightBracket} &
\leg{MVComma} &
\leg{MVPeriod} &
\leg{MVMinus} &
\leg{MVPlus} &
\leg{MVDivision} &
\leg{MVMultiplication} \\
\sho{\MVLeftBracket} &
\sho{\MVRightBracket} &
\sho{\MVComma} &
\sho{\MVPeriod} &
\sho{\MVMinus} &
\sho{\MVPlus} &
\sho{\MVDivision} &
\sho{\MVMultiplication} \\
\hline
% \end{tabular}
% 
% \begin{tabular}{|*{10}{c}|} \hline
\leg{Conclusion} &
\leg{Equivalence} &
\leg{barOver} &
\leg{BarOver} &
\leg{arrowOver} &
\leg{ArrowOver} &
\leg{StrikingThrough} &
\leg{MultiplicationDot} \\
\sho{\Conclusion} &
\sho{\Equivalence} &
\sho{\barOver} &
\sho{\BarOver} &
\sho{\arrowOver} &
\sho{\ArrowOver} &
\sho{\StrikingThrough} &
\sho{\MultiplicationDot} \\
\hline
% \end{tabular}
 
% \begin{tabular}{|*{10}{c}|} \hline
\leg{LessOrEqual} &
\leg{LargerOrEqual} &
\leg{AngleSign} &
\leg{Corresponds} &
\leg{Congruent} &
\leg{NotCongruent} &
\leg{Divides} &
\leg{DividesNot} \\
\sho{\LessOrEqual} &
\sho{\LargerOrEqual} &
\sho{\AngleSign} &
\sho{\Corresponds} &
\sho{\Congruent} &
\sho{\NotCongruent} &
\sho{\Divides} &
\sho{\DividesNot} \\
\hline
\end{tabular}

 \subsection{Biology}
 
 \begin{tabular}{|*{10}{c}|} \hline
 \leg{Neutral} &
 \leg{Male} &
 \leg{Hermaphrodite} &
 \leg{Female} &
 \leg{MALE} &
 \leg{HERMAPHRODITE} &
 \leg{FEMALE} &
 \leg{MaleMale} &
 \leg{FemaleFemale} &
 \leg{FemaleMale} \\
 \sho{\Neutral} &
 \sho{\Male} &
 \sho{\Hermaphrodite} &
 \sho{\Female} &
 \sho{\MALE} &
 \sho{\HERMAPHRODITE} &
 \sho{\FEMALE} &
 \sho{\MaleMale} &
 \sho{\FemaleFemale} &
 \sho{\FemaleMale} \\
 \hline
 \end{tabular}

\subsection{Biology}

\begin{tabular}{|*{4}{c}|} \hline
\leg{Female} &
\leg{Male} &
\leg{Hermaphrodite} &
\leg{Neutral} \\
\sho{\Female} &
\sho{\Male} &
\sho{\Hermaphrodite} &
\sho{\Neutral} \\
\hline
\leg{FEMALE} &
\leg{MALE} &
\leg{HERMAPHRODITE} & \\
\sho{\FEMALE} &
\sho{\MALE} &
\sho{\HERMAPHRODITE} & \\
\hline
\leg{FemaleFemale} &
\leg{MaleMale} &
\leg{FemaleMale} & \\
\sho{\FemaleFemale} &
\sho{\MaleMale} &
\sho{\FemaleMale} & \\
\hline
\end{tabular}

\subsection{Astronomy}

\begin{tabular}{|*{11}{c}|} \hline
\leg{Sun} &
\leg{Moon} &
\leg{Mercury} &
\leg{Venus} &
\leg{Mars} &
\leg{Jupiter} &
\leg{Saturn} &
\leg{Uranus} &
\leg{Neptune} &
\leg{Pluto} &
\leg{Earth} \\
\sho{\Sun} &
\sho{\Moon} &
\sho{\Mercury} &
\sho{\Venus} &
\sho{\Mars} &
\sho{\Jupiter} &
\sho{\Saturn} &
\sho{\Uranus} &
\sho{\Neptune} &
\sho{\Pluto} &
\sho{\Earth} \\
\hline
\end{tabular}

\subsection{Astrology}



\begin{tabular}{|*{12}{c}|} \hline
\leg{Aries} &
\leg{Taurus} &
\leg{Gemini} &
\leg{Cancer} &
\leg{Leo} &
\leg{Virgo} &
\leg{Libra} &
\leg{Scorpio} &
\leg{Sagittarius} &
\leg{Capricorn} &
\leg{Aquarius} &
\leg{Pisces} \\
\sho{\Aries} &
\sho{\Taurus} &
\sho{\Gemini} &
\sho{\Cancer} &
\sho{\Leo} &
\sho{\Virgo} &
\sho{\Libra} &
\sho{\Scorpio} &
\sho{\Sagittarius} &
\sho{\Capricorn} &
\sho{\Aquarius} &
\sho{\Pisces} \\
\hline
\end{tabular}

\subsection{Others}

\begin{tabular}{|*{10}{c}|} \hline
\leg{YinYang} &
\leg{MVRightArrow} &
\leg{MVAt} &
\leg{BOLogo} &
\leg{BOLogoL} &
\leg{BALogoP} &
\leg{Mundus} &
\leg{Cross} &
\leg{CeltCross} &
\leg{Ankh} \\
\sho{\YinYang} &
\sho{\MVRightArrow} &
\sho{\MVAt} &
\sho{\BOLogo} &
\sho{\BOLogoL} &
\sho{\BOLogoP} &
\sho{\Mundus} &
\sho{\Cross} &
\sho{\CeltCross} &
\sho{\Ankh} \\
\hline
\leg{Heart} &
\leg{CircledA} &
\leg{Bouquet} &
\leg{Frowny} &
\leg{Smiley} &
\leg{PeaceDove} &
\leg{Bat} &
\leg{WomanFace} &
\leg{ManFace} & \\
\sho{\Heart} &
\sho{\CircledA} &
\sho{\Bouquet} &
\sho{\Frowny} &
\sho{\Smiley} &
\sho{\PeaceDove} &
\sho{\Bat} &
\sho{\WomanFace} &
\sho{\ManFace} & \\
\hline
\end{tabular}


\egroup

\thetotalsymbols











 





























   \chapter{Characters}


\normalsize

\tex\ works internally by translating characters into character codes. The way characters are encoded in a computer
may differ from system to system.\index{characters>encoding}


There are 256 characters that \tex\  might encounter at
each step, in a file or in a line of text typed directly on your terminal. These
256 characters are classified into 16 categories numbered 0 to 15:\index{characters>catcodes}\index{catcodes}


\arial
\begin{table}[htbp]
\centering
\begin{tabular}{rll}
\toprule
Code & Description & Representation\\
\midrule
0 & Escape character & (\textbackslash in this book)\\
1 & Beginning of group & (|{| in this book)\\
2 & End of group & (|\}| in this book )\\
3 & Math shift & (|\$| in this book)\\
4 & Alignment tab & (|\&| in this book)\\
5 & End of line &(return in this book)\\
6 & Parameter &(|\#| in this book\\
7 & Supescript &(|\^| in this book)\\
8 & Subscript &(|\_| in this bookl)\\
9 & Ignored character &(null in this manual)\\
10 & Space &(\textvisiblespace in this book)\\
11 &Letter &(A,\ldots,Z and a,\ldots z)\\
12 &Other character &(none of the above or below)\\
13 &Active character &(|\~| in this manual)\\
14 &Comment character &(|\%| in this book)\\
15 &Invalid character &(delete in this book)\\
\bottomrule
\end{tabular}
^^A\captionof{table}{Character Codes}
\end{table}
\medskip

When \tex reads a line of text from a file, or a line of text that you entered
directly on your keyboard, it converts that text into a list of \cmd{\tokens}. A
token is either (a) a single character with an attached category code, or (b) a control
sequence. For example, if the normal conventions of plain \tex  are in force, the text
\verb*+ `{\hskip 36 pt}'+  is converted into a list of \textit{eight} tokens:
\medskip

$ \{_{1}$ hskip $3_{12}~~6_{12}~~\_{10}~~p_{11}~~t_{11}~~\}_2 $

\medskip
The subscripts here are the category codes, as listed earlier:
\begin{itemize}
\item[1] for beginning of group,
\item[12] for |other| character," and so on. The |hskip| doesn't get a subscript, because it
represents a control sequence token instead of a character token. Notice that the space
after \cmd{hskip} does not get into the token list, because it follows a control word.
\end{itemize}

Knuth in the \texbook notes that:

\begin{quotation}

It is important to understand the idea of token lists, if you want to gain a
thorough understanding of \tex, and it is convenient to learn the concept by
thinking of \tex as if it were a living organism. The individual lines of input in your
files are seen only by \tex's \textit{eyes} and \textit{mouth}; but after that text has been gobbled
up, it is sent to \tex's \textit{stomach} in the form of a token list, and the digestive processes
that do the actual typesetting are based entirely on tokens. As far as the stomach is
concerned, the input 
flows in as a stream of tokens, somewhat as if your \tex manuscript
had been typed all on one extremely long line.
\end{quotation}

\section{Control sequences for characters}

\DescribeMacro{\char}
There are a number of ways in which a control sequence can denote a character. The \cmd{\char} command
specifies a character to be typeset; the \cmd{\let} command introduces a synonym for a character
token, that is, the combination of character code and category code.

\section{Denoting characters to be typeset: \texttt{char}}

\index{\string\char}
Characters can be denoted numerically by, for example, \verb+ \char98 +. This command tells \tex to add
character number 98 of the current font to the horizontal list currently under construction.

Instead of decimal notation, it is often more convenient to use octal or hexadecimal notation. For
octal the single quote is used: \verb+ \char’142+; hexadecimal uses the double quote: \verb+ \char"62+. Note that

\begin{texexample}{Characters}{ex:chars}
\bgroup
\ttfamily

\char65

\char`b

\char`\b

\char"70

\egroup
\end{texexample}

\verb+ \char`'62+  is incorrect; the process that replaces two quotes by a double quote works at a later
stage of processing (the visual processor) than number scanning (the execution processor).

Because of the explicit conversion to character codes by the back quote character it is also possible
to get a ‘b’ – provided that you are using a font organized a bit like the ASCII table – with \verb+ \char‘b+
or \verb+ \char‘\b+.

The \cmd{\char} command looks superficially a bit like the \verb+  ^^+ substitution mechanism.

Both mechanisms access characters without directly denoting them. However, the \verb+ ^^+ mechanism
operates in a very early stage of processing (in the input processor of \tex, but before category
code assignment); the \cmd{\char} command, on the other hand, comes in the final stages of processing.
In effect it says ‘typeset character number so-and-so’.

\CMDI{\Uchar} The LuaTeX expandable command \cmd{\Uchar} reads a number between 0 and 1,114,111 and expands to the
associated Unicode character. 

\DescribeMacro{\chardef}
There is a construction to let a control sequence stand for some character code: the \cmd{\chardef}
command. The syntax of this is\\
\cs{chardef}\meta{control sequence}=\meta{number},\\
where the number can be an explicit representation or a counter value, but it can also be a character
code obtained using the left quote command (see above; the full definition of hnumberi is
given in Chapter 7). In the plain format the latter possibility is used in definitions such as

\verb+ \chardef\%=‘\%+

or 

\verb+ \chardef\%=37   +

command to typeset character 37 (usually the per cent character).\index{characters!percent character}

A control sequence that has been defined with a \cmd{\chardef} command can also be used as a hnumberi.
This fact is used in allocation commands such as \verb+ \char{newbox}+ (see Chapters 7 and 31). Tokens defined
with \verb+ \char{mathchardef}+ can also be used this way.


But \tex\ actually provides another kind of number that makes it unnecessary
for you to know texttt{ASCII} at all! The token `12 (left quote), when followed by
any character token or by any control sequence token whose name is a single character,
stands for \tex's internal code for the character in question. For example, \verb+\char`b+ and
\verb+ \char`\b+ are also equivalent to \verb+ \char98+. If you look in Appendix B to see how \verb+ \%+ is
defined, you'll notice that the definition is

\verb+\def\%{\char`\%}+

instead of \verb+ \char37+  as claimed above.

\section{Special notation for invisible characters}

\tex has a standard way to refer to the invisible characters of |ASCII|: 

Code 0 can be typed as the sequence of three characters \verb+ ^^@+, code 1 can be typed
\verb+ ^^A+, and so on up to code 31, which is \verb+ ^^_  +(see Appendix C). If the character following
\verb+ ^^+ has an internal code between 64 and 127, TEX subtracts 64 from the code; if the
code is between 0 and 63, \tex adds 64. 

Hence code 127 can be typed \verb+^^?+, and
the dangerous bend sign can be obtained by saying \verb+{\manual^^?}+. However, you must
change the category code of character 127 before using it, since this character ordinarily
has category 15 (invalid); say, e.g., 

\verb+ \catcode`\^^?=12 +

The \verb+ ^^+ notation is different from
\cmd{\char}, because \verb+ ^^+ combinations are like single characters; for example, it would not
be permissible to say \verb+ \catcode`\char127+, but \verb+^^+ symbols can even be used as letters within control words.

\begin{texexample}{Special notation}{ex:texbook1}
\def\^^zz{test}
\^^zz
\end{texexample}


Most of the \verb+ ^^+ codes are unimportant except in unusual applications. But
\verb+ ^^M+ is particularly noteworthy because it is code 13, the |ASCII| return that
\tex normally places at the right end of every line of your input file. By changing the
category of \verb+ ^^M+  you can obtain useful special effects, as we shall see later.

\section{Upper and Lowercase characters}

\verb*+\lccode +the character code for the lowercase form of a letter (p. 103)

\DescribeMacro{\lowercase}
\DescribeMacro{\uppercase}
The twin operations \cmd{\uppercase}\marg{token list} and \cmd{\lowercase}\marg{token listi}
go through a given token list and convert all of the character tokens to their
\cmd{\uppercase}  or \cmd{lowercase} equivalents.

\begin{texexample}{Uppercase and Lowercase}{ex:lowercase} 
\uppercase{abcdefgh} 
\lowercase{ABCDEFGH}
\end{texexample}

Here's how: Each of the 256 possible characters
has two associated values called the \cmd{\uccode} and the \cmd{lccode}; these values are
changeable just as a \cmd{\catcode} is. Conversion to uppercase means that a character
is replaced by its \cmd{\uccode} value, unless the \cmd{\uccode} value is zero (when no change
is made). Conversion to lowercase is similar, using the
\verb+  \lccode+. The category codes
aren't changed. 

When INITEX begins, all \verb+ \uccode+ and \verb+ \lccode+ values are zero except
that the letters a to z and A to Z have \verb+\uccode+ values A to Z and \verb+\lccode+ values a to z.

These tow control sequences are used to build a hash table, mapping all the capital and lowercase letters to their respective character codes.
(see pg 41 TeXbook)

\section{Some Practical Examples}

If you are typesetting anything that has to do with \tex\ or \latex\ you are bound to have to typeset a lot of commands. This short code below will change the category code of the \texttt{"} (double quote) to be the active command. This way anything between double quotes will be  typed out verbatim and in a Maroon color. By mainipulating the \cmd{catcode} of characters we can achieve this.

\begin{teX}
%% Code to catch commands
\def\Meaningless#1>{}
\catcode`\"=\active
\def\startV{\leavevmode\begingroup
  \ifdim 0pt=\lastskip\penalty200 \fi
  \catcode`\{11 \catcode`\}11 \catcode`\%11
  \moreV}
\long\def\moreV#1"{%
  \def\LtxCode{#1}%
  \ignorespaces
      \expandafter\Meaningless\meaning\LtxCode
      \unskip%
  \endgroup}
\let"\startV

\bgroup
\catcode`\<=\active
\def<#1>{\ensuremath{\langle\mbox{\textsl{#1}}\rangle}}
\end{teX}

\begin{comment}
\bgroup
\def\Meaningless#1>{}
\catcode`\"=\active
\def\startV{\leavevmode\begingroup
  \ifdim 0pt=\lastskip\penalty200 \fi
  \catcode`\{11 \catcode`\}11 \catcode`\%11
  \moreV}
\long\def\moreV#1"{%
  \def\LtxCode{#1}%
  \ignorespaces
      \expandafter\Meaningless\meaning\LtxCode
      \unskip%
  \endgroup}
\let"\startV

\catcode`\<=\active
\def<#1>{\ensuremath{\langle\mbox{\textsl{#1}}\rangle}}

\noindent Testing it out with a few commands we get 
"\catcode", "\char" ,"\def" etc. We will revert back to this short example later on in our book, when you have learned a bit more about macros and programming \tex\. Note that this also affects "quotes".

\egroup
\end{comment}

A more complex example is the \pkg{shortvrb} package code.

\begin{teX}
%% Copyright (C) 1989-1999 Frank Mittelbach, all rights reserved.
\def\MakeShortVerb{%
  \@ifstar
    {\def\@shortvrbdef{\verb*}\@MakeShortVerb}%
    {\def\@shortvrbdef{\verb}\@MakeShortVerb}}

\def\@MakeShortVerb#1{%
  \expandafter\ifx\csname cc\string#1\endcsname\relax
    \@shortvrbinfo{Made }{#1}\@shortvrbdef
    \add@special{#1}%
    \expandafter
    \xdef\csname cc\string#1\endcsname{\the\catcode`#1}%
    \begingroup
      \catcode`\~\active  \lccode`\~`#1%
      \lowercase{%
      \global\expandafter\let
         \csname ac\string#1\endcsname~%
      \expandafter\gdef\expandafter~\expandafter{\@shortvrbdef~}}%
    \endgroup
    \global\catcode`#1\active
  \else
    \@shortvrbinfo\@empty{#1 already}{\@empty\verb(*)}%
  \fi}
\def\DeleteShortVerb#1{%
  \expandafter\ifx\csname cc\string#1\endcsname\relax
    \@shortvrbinfo\@empty{#1 not}{\@empty\verb(*)}%
  \else
    \@shortvrbinfo{Deleted }{#1 as}{\@empty\verb(*)}%
    \rem@special{#1}%
    \global\catcode`#1\csname cc\string#1\endcsname
    \global \expandafter\let \csname cc\string#1\endcsname \relax
    \ifnum\catcode`#1=\active
      \begingroup
        \catcode`\~\active   \lccode`\~`#1%
        \lowercase{%
          \global\expandafter\let\expandafter~%
          \csname ac\string#1\endcsname}%
      \endgroup \fi \fi}
\def\@shortvrbinfo#1#2#3{%
  \PackageInfo{shortvrb}{%
     #1\expandafter\@gobble\string#2 a short reference
                                          for \expandafter\string#3}}
\def\add@special#1{%
  \rem@special{#1}%
  \expandafter\gdef\expandafter\dospecials\expandafter
    {\dospecials \do #1}%
  \expandafter\gdef\expandafter\@sanitize\expandafter
    {\@sanitize \@makeother #1}}
\def\rem@special#1{%
  \def\do##1{%
    \ifnum`#1=`##1 \else \noexpand\do\noexpand##1\fi}%
  \xdef\dospecials{\dospecials}%
  \begingroup
    \def\@makeother##1{%
      \ifnum`#1=`##1 \else \noexpand\@makeother\noexpand##1\fi}%
    \xdef\@sanitize{\@sanitize}%
  \endgroup}
\endinput
%%
%% End of file `shortvrb.sty'.
\end{teX}

We will spent the rest of the book in trying to understand and write code like this. My ultimate aim is  to be able to produce \tex\ code like any other program. 

\section{Example}

In this example we wish to redefine some of the active codes to act as text only:

\begin{teX}
\newenvironment{plaintext}{%
        \catcode`\$12
        \def\&{&}%
        \catcode`\&12
        \def\_{_}%
        \catcode`\_12
        \def\^{^}%
        \catcode`\^12
        \catcode`\#12
        \catcode`\%12
        \let\~~%
        \catcode`\~12
}{}
\end{teX}

\newenvironment{plaintext}{%
        \catcode`\$12
        \def\&{&}%
        \catcode`\&12
        \def\_{_}%
        \catcode`\_12
        \def\^{^}%
        \catcode`\^12
        \catcode`\#12
        \catcode`\%12
        \let\~~%
        \catcode`\~12
}{}
Use it like

\begin{plaintext}
Here is some test text % ^ & _ $ # &.

How about some math \(x\_y\^z\). You're still out of luck with braces
though.
\end{plaintext}

\begingroup
\catcode`\{=11 
\catcode`\}=11
\catcode`\[=1
\catcode`\]=2

{This is a test}

\endgroup


\section{Checking to see the meaning of a control sequence
}
Finding out just what a control sequence has been defined to be with |\let| can be done using

%\meaning: the sequence

\begin{teXXX}
\let\x=3 \meaning\x
\end{teXXX}
\graybox{
gives 'the character 3'.}




 }   
%
%

\def\mathdocs{
\part{MATHEMATICS}
%\chapter{maths}
%\label{ch:maths}
%
%
%                      
%
%\parindent1em
%
%
%
%\starttemplate{kroll}
%\thispagestyle{empty}
%    \begin{leftcolumn}
%         {{\centering \huge  THE ART OF \\
%       TYPESETTING\\
%       MATHS\\}}
%      \medskip
%       {\justifying \small Perhaps some day a typesetting language will become standardized to the 
%point where papers can be submitted to the American Mathematical Society 
%from computer to computer via telephone lines. Galley proofs will not be 
%necessary, but referees  and / or copy editors could send suggested changes to 
%the author, and he could insert these into the manuscript, again via telephone.\par
%\hfill \textit{Donald Knuth}}
%\medskip
%       \putimage[width=1.0\linewidth]{./images/halmos.png}\par
%       \aheader{shows Kroll at 59. Says he. ``Painting is fascinating'' even when motif my own mug.}
%   \end{leftcolumn}
%   \begin{rightcolumn}
%       \putimage[width=0.9\linewidth]{./images/themathematician.jpg}
%       \onelinecaption{{\resizebox{0.9\linewidth}{5.5pt}{\bfseries The Mathematician (1918) an oil painting by the Mexican artist Diego Rivera. }}\par}
%
%     \vspace*{1.5\baselineskip}
%
%  \centerline{\onelineheader{TYPESETTING MATHEMATICS}}
%      \begin{multicols}{2}
%     
%      \lettrine{M}{ost people} discover, \alltex when they are faced with the production of a thesis or paper that includes lots of
%mathematical text. It is the rais\`on detrait for \tex. This chapter discusses the typesetting of mathematics, common pitfalls and solutions. We will also review some typographical questions.
%\parindent1em
%
%You can start writing mathematical text, without any additional loading of packages, either in plain \tex or \latex. The reality is that most institutions have developed their own styles and common packages used by AMS have found widespread use. We will discussing these extensively, but first let us look at how to enter mathematical text. 
%
%To enter mathematical text in plain \tex you just use the |$|\ldots|$| for inline math and |$$|\ldots|$$| for displayed math. \latex uses |\(|
%and |)| for inline and |[\]|.
%      \end{multicols}
%   \end{rightcolumn}
%\stoptemplate
%\end{group}
%
%\newgeometry{left=2.9cm,right=2.9cm, bottom=2cm}
%\pagestyle{headings}
\large
\chapter{Mathematics}

\section{Introduction}
\epigraph{Do or do not. There is no try.}{Yoda in \textit{The Empire Strikes Back}}

Our purpose is to study the typesetting of mathematics using \latexe by providing somewhat longer examples than those provided in normal \latex tutorials, starting with a faster and perhaps steeper learning curve, before we delve into the subtleties of the typesetting macros. Mathematics is described using symbols and texts. We start by the definition of the ordinary \textit{integers}, i.e., the positive and negative whole numbers and zero:

\[\ldots,-3,-2,-1,0,2,3\dots.\]

To typeset the above we enclose the equation using |\[ ... \]|. This is called a display formula, as compared to inline math. Numbers are typeset just by typing the numbers |-3,-2,-1,0,2,3| and the elision using |\ldots | or |\dots|. 

\begin{texexample}{Basic Math Typesetting}{ex:typesetting}
\[\ldots,-3,-2,-1,0,2,3\dots.\]
\end{texexample}


The acute reader will notice the difference in typesetting the first ellipsis ($\ldots, -3$) from the last ($3\dots$). There is no general consensus on the typesetting of dots. Within maths the AmS will use its own defaults if we use |\dots|. 

\[ 1 + 2 +\dots + n  \]

If what follows the dots is ambiguous as to the
choice, the specific form of the command can be used. However, by using the semantically
oriented commands
\begin{description}
 \item[\cs{dotsc}] for \enquote{dots with commas}
 \item[\cs{dotsb}] for \enquote{dots with binary operators/relations}
 \item[\cs{dotsm}] for \enquote{multiplication dots}
 \item[\cs{dotsi}] for \enquote{dots with integrals}
 \item[\cs{dotso}] for \enquote{other dots} (none of the above)
\end{description}
we can adapt our mathematical paper to the standard of any Journal.

Mathematical induction can be used to prove that the following statement, $P(n)$, holds for all natural numbers.

\begin{gather}
0 + 1 +2 + \dots +n = \frac{n(n+1)}{2}.
\end{gather}

I have now introduced another macro |\frac|, this takes two arguments, one for the donominator and another for the numerator. This time we also get an equation number as well by enclosing the formula in an environment.

\begin{texexample}{Display equation with gather}{ex:gather0}
\begin{gather}
0 + 1 +2 + \dots +n = \frac{n(n+1)}{2}.
\end{gather}
\end{texexample}

There are many environments available with subtle differences. Getting back with our writing, we need to explain our method of proofs by induction. We now write,

\textit{Base case}: Show that the statement holds for $n=0$ (taking 0 as a natural number).

$P(0)$ is easily seen to be true:

\[0 = \frac{0\times(0+1)}{2}\].

I have now introduced the |\times| multiplication $(\times)$ symbol and we have not numbered the equation. When multiplying numerals we can use the |\times| operator for multiplication or use a |\cdot| $(\cdot)$ to typeset the equation as follows:

\[0 = \frac{0\cdot(0+1)}{2}\].


\textit{Inductive step}: Show that if $P(k)$ holds, then also $P(k + 1)$ holds. This can be done as follows.

Assume $P(k)$ holds (for some unspecified value of $k$). It must then be shown that $P(k + 1)$ holds, that is:

\begin{gather}
\underbrace{(0+1+2+\dots+k)}_\text{terms to be replaced by $\frac{k(k+1)}{2}$} + (k+1) = \frac{(k+1)((k+1) +1 ) }{2} \label{eq:underbrace}
\end{gather}



Using the induction hypothesis that $P(k)$ holds, the left-hand side can be rewritten to:

\[ \frac{k(k+1)}{2} + (k+1) \]

\begin{align}
\frac{k(k + 1)}{2} + (k+1) & = \frac {k(k+1)+2(k+1)} 2 \\
& = \frac{(k+1)(k+2)}{2} \\
& = \frac{(k+1)((k+1) + 1)}{2}
\end{align}
thereby showing that indeed  $P(k + 1)$ holds.

Since both the base case and the inductive step have been performed, by mathematical induction the statement $P(n)$ holds for all natural numbers $n$.

The alignment of the equations was achieved by enclosing the equations in another environment |align|. This comes in a star and an unstarred version, the latter will typeset the equations without a number. It is very similar to a |tabular| environment and uses |&| as the tabulator symbol and |\\| as the carriage return.

\begin{texexample}{Aligning equations}{ex:align0}
\begin{align}
\frac{k(k + 1)}{2} + (k+1) & = \frac {k(k+1)+2(k+1)} 2 \\
& = \frac{(k+1)(k+2)}{2} \\
& = \frac{(k+1)((k+1) + 1)}{2}
\end{align}
\end{texexample}

Although we so far we managed to typeset beautiful looking mathematics if we blindly apply induction reasoning we are on shaking ground. Consider the statement:

$f(n) = n^2 - n +41$ is a prime number for every $n\ge1$. 

As we evaluate $f(n)$ for $n = 1,2,3,4,\dots,40,$ we obtain the following values:

\begin{gather*}
41,43,47,53,61,71,83,97,113,131, \\
151,173,197,223,251,281,313,347,383,421, \\
461,503,547,593,641,691,743,797,853,911, \\
971,1033,1097,1163,1231,1301,1373,1447,1523,1601. \\
\end{gather*}

It is tedious but not difficult to prove that every one of these 
numbers is prime. Can we now conclude that all the numbers of the form $f(n)$ 
are prime? For example, is the next number $f(41) = 1681$ prime? The answer is no: $f(41) = 41^2 -41 + 41 = 41^2$ , which obviously factors, and hence 
$f(41)$ is not prime. 




P\'olya in \textit{How To Solve It, A New Aspect of Mathematical Method} gave a simple problem: to number the pages of a bulky volume, the printer 
used 2989 digits. How many pages has the volume? 

A volume of 999 numbered pages needs\\ 
$$9 + 2 \times 90 + 3 \times 900 = 2889$$ \\
digits, If the bulky volume in question has $x$ pages \\
$$2889 + 4(x-999) = 2989 $$
$$x = 1024 $$
This problem may teach us that a preliminary estimate 
of the unknown may be useful (or even necessary, as in the present case). 

Pólya was born in Budapest, Austria-Hungary to Anna Deutsch and Jakab Pólya, Hungarian Jews who had converted to the Roman Catholic faith in 1886.[4] Although his parents were religious and he was baptized into the Roman Catholic Church, George Pólya grew up to be an agnostic.[5] He was a professor of mathematics from 1914 to 1940 at ETH Zürich in Switzerland and from 1940 to 1953 at Stanford University. He remained Stanford Professor Emeritus for the rest of his life and career. He worked on a range of mathematical topics, including series, number theory, mathematical analysis, geometry, algebra, combinatorics, and probability.[6] 

For further insights into problem solving, I highly recommend his books.


\begin{Axiom}[Least Integer Axiom]\footnote{An axiom or postulate is a statement that is taken to be true, to serve as a premise or starting point for further reasoning and arguments. The word comes from the Greek axíōma (ἀξίωμα) 'that which is thought worthy or fit' or `that which commends itself as evident.'} Every nonempty collection $C$ of positive integers has a smallest number in it.
\end{Axiom}

\begin{theorem}[Least Criminal] 
Let $S(n)$ be a family of statements, where n 
varies over some nonempty collection of positive integers. If some of these 
statements are false, then there is a first false statement. 

\begin{proof}
Let $C$ be the collection of all those positive integers $n$ for which $S(n)$ 
is false; by hypothesis, $C$ is nonempty. The Least Integer Axiom provides a 
smallest number $m$ in $C$, and $S(m)$ is the first false statement. 
\end{proof}
\end{theorem}

Here are two comments before giving more applications of induction. First, one must verify both the base step and the inductive step; verification of only one of them is inadequate\ldots




\begin{Exercise}
\upshape
This example uses both inductive reasoning and mathematical induction. We seek a formula for the sum of the first $n$ odd numbers: $1 + 3 + \dots+(2n-1)$. A list of the sums for $n =1,2,3,4,5$ is $1,4,9,16,25$. These are perfect squares; we can rewrite them better as $1^2,2^2,3^2,4^2,5^2$. Inductive reasoning suggests the \emph{guess}
\[ S(n): 1 + 3 +5 +\dots+(2n-1)=n^2\]

We have discovered a formula. We now use mathematical induction to prove that this guess is always true. The base step S(1) has already been checked. For the inductive step we must prove 

\[S(n+1): [1+3+\dots+(2n-1)] + (2n+1) = (n+1)^2 \].
\end{Exercise}


Most mathematical texts, will have theorems, proofs and the like, so I would like to introduce them early. These are actually LaTeX environments, supplemented by local definitions and packages.
So if we wanted to print the Fundamental Theorem of Arithmetic, perhaps the most important theorem we say:

\begin{texexample}{Theorems}{ex:theor}
\begin{theorem}[Fundamental theorem of arithmetic] For each natural
number $n$ there is a unique factorization
\[n = p_1^{a_1} p_2^{a_2}\ldots p_k^{a_k},\]
where exponents $a_i$ are positive integers and $p_1<p_2\ldots<p_k$ are primes.
\end{theorem}
\end{texexample}

To typeset subscripts and superscripts we use |_| and |^|.

The next step towards the modern concept of number was made by the invention of the decimal point $(\middot)$ and decimal notation. This made it possible to represent \textit{irrational numbers} to a very high accuracy. For example:\\
the square root $\sqrt{2}\sim 1.4142135623$\\[0.3ex]
correct to 10 decimal places. The symbol $\sim$ means \enquote{is approximately equal to}. This expression is not exact: its square is actually:\\[0.3ex]
\directlua{tex.print(1.4142135623*1.4142135623)}

LaTeX and numerous packages provide commands to typeset an incredible amount of mathematical symbols, which are divided into different classes, such as operators, ordinals, delimiters and the like. We will look at them in more detail later on.

In number theory and related mathematics, the main number systems are normally denoted using a blackboard font as follows. For example $\mathbb{N}$, is typeset using |\mathbb{N}|.
\begin{align*}
\mathbb{N} & = \text{the set of all natural numbers }0,1,2,3,\dots \\
\mathbb{Z} & = \text{the set of all integers }, -3,-2,-1,0,1,2,3,\dots\\
\mathbb{Q} & = \text{the set of all rational numbers}\\
\mathbb{R} & = \text{the set of all real numbers}\\
\mathbb{C} & = \text{the set of all complex numbers}\\
\end{align*}


At this point we need to digress and discuss, the various types of fonts that can be used in \latex. The LaTeX kernel defines several alphabets in \docFile{fontmath.ltx}. The |\math...| commands provide macros to typeset mathematics in these different fonts.
\begin{description}
\item[\cs{mathrm}] is the normal upright Roman font.
\item[\cs{mathnormal}] is the normal math italic font: |$\mathnormal{a}$| and |$a$| give the same result.
\item[\cs{mathcal}] is the special calligraphic font. This can be used only for uppercase letters.
\item[\cs{mathbf}] gives the upright Roman boldface letters.
\item[\cs{mathsf}] gives upright sans serif letters
\item[\cs{mathit}] gives text italic letters: $different\ne\mathit{different}.$
\item[\cs{mathtt}] gives upright letters from the typewriter type font.
\end{description} 

We should mention that the argument to each of these commands is typeset in \emph{math mode}, so spaces are ignored and hyphens become minus signs. Using those commands with arguments not consisting only of normal letters can give unexpected (and sometimes bizarre) results.

No package has to be loaded for being able to use those alphabets.

When we load the AMSfonts package, which by the way is loaded automatically by the amssymb package we get access also to

\begin{description}
\item[\cs{mathfrak}] for Fraktur (aka Gothic) letters, upper and lower case
\item[\cs{mathbb}] for \enquote{blackboard bold} uppercase letters
\end{description}



Now that we have a method to typeset various types of symbols with distinct fonts, we recast
the fundamental theorem of arithmetic with set notation. Most of the symbols for set have
intuitive names, like |\in| for $\in$. 

Let $\mathbb{N} = \{0,1,2,3, \dots \}$ be the set of {\bf natural numbers}.
A number $p \in \mathbb{N}, p>1$ is {\bf prime} if $p$ has no factors different from $1$ and $p$. 
With a {\bf prime factorization} $n=p_1 \dots p_n$, we understand the prime factors $p_j$
of $n$ to be ordered as $p_i \leq p_{i+1}$. The {\bf fundamental theorem of arithmetic} is
\index{natural numbers}
\index{primes}
\index{prime factors} 
\index{prime factorization}

\begin{theorem}
Every $n \in \mathbb{N}, n>1$ has a unique prime factorization.
\end{theorem}

As professor Stewart says, these systems fit inside each other like Russian dolls.\\
$\mathbb{N}\subset \mathbb{Z} \subset \mathbb{Q} \subset \mathbb{R} \subset \mathbb{C}$\\
where the set theory symbol |\subset| ($\subset$) means is \textit{contained in}.

Let $\vert X\vert$ denote the {\bf cardinality} of a finite set $X$. This means that
$\vert X\vert$ is the number of elements in $X$. A function $f$ from a set $X$ to a set $Y$
is called {\bf injective} if $f(x)=f(y)$ implies $x=y$. The {\bf pigeonhole principle} tells:

%% Division
\begin{Definition}
If $a$ and $b$ are integers, $a \neq 0$, and if there is an integer $c$ such that $b = ac$, then we say that $a$ divides $b$, and we write $a\vert b$. If $a$ does not divide $b$, then we write $a \nmid b$.
\end{Definition}

Although $\frac{7}{5} = 1.4$ the quotient is not an integer and thus $5 \nmid 7$. Other examples are
\[2\vert 18, 1\vert 42, 3\vert(6), -6\nmid 31. \]


\begin{theorem} Every square integer is of the form $4k$ or $4k+1$, where $k$ is an integer.
Since $x^2$ and $y^2$ must be of the form 4k or 4k + 1, x2 þ y2, the sum of two
squares, can only be of the form 4k, 4k þ 1 , or 4k þ 2 and we have
established the next result.
\end{theorem}



There are many more number representations, but at this stage I want to introduce the Hebrew alphabet letter which is used normally as a notation for infinities |\infty| ($\infty$), such as Cantor's ordinal and cardinal infinities of set theory.  Georg Cantor developed a system of transfinite numbers, in which the first transfinite cardinal is aleph-null ($\aleph_0$), the cardinality of the set of natural numbers. This modern mathematical conception of the quantitative infinite developed in the late nineteenth century from work by Cantor, Gottlob Frege, Richard Dedekind and others, using the idea of collections, or sets.

In set theory, the cardinality of the continuum is the cardinality or \enquote{size} of the set of real numbers {$\mathbb {R} $ $\mathbb {R}$ , sometimes called the continuum. It is an infinite cardinal number and is denoted by ${\displaystyle |\mathbb {R} |}$ 

\[\mathfrak c = 2^{\mathcal N_0}>\mathcal N_0.\]

\section*{Notation}
 
At this stage one should start developing a notation table. Since we dealing with number theory we will give an example of a notation system

\begin{longtable}{lp{6cm}} 
$d,k,m,n$ & positive integers (unless otherwise indicated) \\
$d\vert n$ & $d$ divides $n$\\ 
$(m,n)$ & greatest common divisor of $m,n$. If $(m,n)=1,m$ and $n$ are called relatively prime or coprime.\\
$(d_1,\dots,d_n)$ &greatest common divisor of $d_1,\dots,d_n$\\
$\sum_{d\vert n}, \prod_{d\vert n}$ & sum, product taken over divisors of $n$.\\
$p, p_1, p_2,\dots $ & prime numbers (or primes): integers $(>1)$ with only two positive integer 
divisors, $1$ and the number itself.\\
$\sum_p, \prod_p$ & sum, product extended over all primes. \\ 
$x,y$ & real numbers.\\
$\log x$ & natural logarithm of $x$, written as $\ln x$ in other chapters.\\
$\zeta(s)$ & Riemann zeta function.\\
$(n\vert P)$ & Jacobi symbol.\\
$(n\vert p) & Legendre symbol\\
 \end{longtable}


\idxboth{binary}{operators}
\index{semidirect products}
\label{ams-bin}
\begin{tabular}{*3{ll}}
\X\barwedge        & \X\circledcirc     & \X\intercal$^*$    \\
\X\boxdot          & \X\circleddash     & \X\leftthreetimes  \\
\X\boxminus        & \X\Cup             & \X\ltimes          \\
\X\boxplus         & \X\curlyvee        & \X\rightthreetimes \\
\X\boxtimes        & \X\curlywedge      & \X\rtimes          \\
\X\Cap             & \X\divideontimes   & \X\smallsetminus   \\
\X\centerdot       & \X\dotplus         & \X\veebar          \\
\X\circledast      & \X\doublebarwedge  \\
\end{tabular}

\bigskip

\begin{tablenote}[*]
  \newcommand{\trpose}{{\mathpalette\raiseT{\intercal}}}
  \newcommand{\raiseT}[2]{\raisebox{0.25ex}{$#1#2$}}
%
  Some people use a superscripted \docAuxCommand{intercal} for matrix
  transpose\index{transpose}: ``\verb|A^\intercal|''~$\mapsto$
  ``$A^\intercal$''.  (See the May~2009 \ctt thread, ``raising math
  symbols'', for suggestions about altering the height of the
  superscript. and se.tex question \footnote{\url{http://tex.stackexchange.com/questions/30619/what-is-the-best-symbol -for-vector-matrix-transpose}})  \docAuxCommand{top} (\vref*{letter-like}), \verb|T|, and
  \verb|\mathsf{T}| are other popular choices: ``$A^\top$'',
  ``$A^T$'', ``$A^{\text{\textsf{T}}}$''.
\end{tablenote}


\begin{Rule} Visit arxiv.com and read some of the papers in your field. Gather a set of symbols that you will need into a symbols list first. Keep the list updated as you expand your writings.
\end{Rule}

\begin{Rule} Choose your notation carefully from that available in the field you are writing about. Use one of the math alphabets to distinguish sets, constants etc. 
\end{Rule}

\begin{Rule} Use a combination of text and mathematics to describe the work. Use amsmath standard packages at first while you learning (from day one).
\end{Rule}


%|\mathbb {R} | or {\displaystyle {\mathfrak {c}}} {\mathfrak {c}} (a lowercase fraktur script "c").

%The real numbers $\mathbb {R}$  are more numerous than the natural numbers $\mathbb {N}$. Moreover,  $\mathbb {R}$  has the same number of elements as the power set of {\displaystyle \mathbb {N} } \mathbb N. Symbolically, if the cardinality of {\displaystyle \mathbb {N} } \mathbb N is denoted as {\displaystyle \aleph _{0}} \aleph _{0}, the cardinality of the continuum is


Consider the typesetting of Euclid's theorem for the existance of infinitely many primes. In books you have probably read something like.

\begin{theorem}[Euclid] There exist infinitely many primes.
\end{theorem}
\begin{proof} Assume that the primes are a finite in number, and denote by $p$ the largest.

\[
n=2\cdot3\cdot5\cdots p+1
\] 
\end{proof}


Continuing with our study of primes we define the nth \textit{primorial} number as  the product of consecutive primes numbers, starting with the first prime, namely 2. One distinguishes between the n, nth primorial number and the primorial of a natural number $n$.

\begin{gather}
p_n\# := \prod_{i=1}^{n} p_i,\quad n \ge 0, \,
\end{gather}

A002110 The primorial numbers, $p_n\#,\ n \,\ge\, 0. \,$

\begin{aligned}
 p_n\# =  1, 2, 6, 30, 210, 2310, \numprint{30030}, \numprint{510510}, \numprint{9699690}, \numprint{223092870}, \numprint{6469693230}, \\
 \numprint{200560490130}, \numprint{7420738134810}, \numprint{304250263527210}, \numprint{13082761331670030},\\
\numprint{614889782588491410}, \numprint{32589158477190044730}, \ldots 
\end{aligned}


To avoid confusion large numbers are typeset either without a comma separator or with spaces. We avoid justification and if need be we manually break the line.


\section{Inline and Display Math}
Mathematical typesetting involves either inline math formulae, which are part of a paragraph of text or display math which are a block of mathematical material. \tex will typeset inline math by enclosing the material between  \$\ldots\$. See Example~\ref{ex:inline}.
\bigskip

\begin{texexample}{Inline and Display Math}{ex:inline}
This is an inline equation \(a^2+b^2=c^2\)

And this equation is a display equation:
\[a^2+b^2=c^2\]
\end{texexample}


The above code, is valid in \latex and its variants as well. However in certain cases, especially for displayed math, this can introduce some blank lines. Lamport redefined
the \$\$. For inline math use |\(|\ldots|\)| and for display math |\[|\ldots|\] |. The above example can be written as:
\bigskip

\begin{texexample}{The Math Environment}{ex:mathenvironment}
 This is an inline equation \(a^2+b^2=c^2\)

 And this equation is a display equation:
 \[a^2+b^2=c^2\]

 \begin{math}
 \sum_{i=1}^{n}i=\frac{1}{2}n\cdot(n+1)
 \end{math}
\end{texexample}

As you can observe the output remains the same.  Spaces within  maths environment are ignored, so use this to your advantage when writing maths. As mentioned earlier, we can use |\[...\]| for display math.


\section{Displayed Equations}


Displayed equations, can either be displayed \emph{flushed left} or \emph{centered}, depending on the option loaded with the standard classes, for example in article use,

\begin{verbatim}
\documentclass[imperial,11pt,openany, twoside,fleqn]{article}
\end{verbatim}

We will see, how to change the formatting in Section~\ref{eqnochange} a bit later on. The idea with \latex that such changes belong to the class and not the author.

\section{Greek letters}

One of the most celebrated equations is Euler's 

\begin{gather}
\frac{1}{1^2} + \frac{1}{2^2} + \frac{1}{3^2} +\dots=\frac{\pi^2}{6}.
\end{gather}
Although the symbol $\pi$ was used earlier, Euler's formula was of such importance that this notation has been adopted ever since. The curious number 1.644934\ldots turns out to be $\tfrac{1}{6}\pi^2$ an astonishing result that did much to enhance Euler's growing reputation.


All the Greek letters are available in \tex{}. For a full list and commentary
see also Table~\vref{greek} in the Chapter \vref{comprehensivesymbols}. 

\begin{table}[htbp]
\centering
\begin{tabular}{llllllll}
\toprule
$\alpha$  &\docAuxCommand{alpha} &$\beta$ &\docAuxCommand{beta} &$\gamma$ &\docAuxCommand{gamma} &$\delta$ &\docAuxCommand{delta}\\
$\epsilon$  &\docAuxCommand{epsilon} &$\varepsilon$ &\docAuxCommand{varepsilon} &$\zeta$ &\docAuxCommand{zeta} &$\eta$ &\docAuxCommand{eta}\\
\bottomrule
\end{tabular}
\end{table}

%Direct Input: Α Β Γ Δ Ε Ζ Η Θ Ι \allowbreak Κ Λ Μ Ν Ξ Ο Π Ρ Σ Τ Υ Φ Χ Ψ ΩABΓΔEZHΘIKΛMNΞOΠPΣTΥΦXΨΩ 
%\allowbreak α β γ δ ϵ ζ η θ ι κ λ μ ν ξ o π \allowbreak ρ σ τ υ ϕ χ ψ ω ε ϑ ϖ ϱ ς φαβγδϵζηθικλμνξoπρστυϕχψωεϑϖϱςφ
%\begingroup
%\def\K#1{$#1$ & \cmd{#1}}
%\def\Iota{\mathrm{I}}
%\def\Kappa{\mathrm{K}}
%\def\Mu{\mathrm{M}}
%\def\Nu{\mathrm{N}}
%\def\Omicron{\mathrm{O}}
%\def\Tau{\mathrm{T}}
%\begin{longtable}{llllllll}
%A	&|\Alpha| &B &|\Beta|	& $\Gamma$ &|\Gamma|	& $\Delta$ &|\Delta| \\
%\K{E}	&\K{Z}	&\K{H}	&\K\Theta  \\
%\K\Iota &\K\Kappa	  &\K\Lambda	&\K\Mu\\
%\K\Nu	  &\K\Xi     &\K\Omicron	&\K\Pi\\
%\K\Sigma	&\K\Tau	&\K\Upsilon	   &\K\Phi\\
%\end{longtable}
%\endgroup

%

%
%\ChiX \Chi	\PsiΨ \Psi	\OmegaΩ \Omega\\
%\varGammaΓ \varGamma	\varDeltaΔ \varDelta	\varThetaΘ \varTheta	\varLambdaΛ \varLambda\\
%\varXiΞ \varXi	\varPiΠ \varPi	\varSigmaΣ \varSigma	\varUpsilonΥ \varUpsilon\\
%\varPhiΦ \varPhi	\varPsi Ψ \varPsi	\varOmegaΩ \varOmega	\\
%\alphaα \alpha	\betaβ \beta	\gammaγ \gamma	\deltaδ \delta\\
%\epsilonϵ \epsilon	\zetaζ \zeta	\etaη \eta	\thetaθ \theta\\
%\iotaι \iota	\kappaκ \kappa	\lambdaλ \lambda	\muμ \mu\\
%\nuν \nu	\xiξ \xi	\omicronο \omicron	\piπ \pi \\
%\rhoρ \rho	\sigmaσ \sigma	\tauτ \tau	\upsilonυ \upsilon\\
%\phiϕ \phi	\chiχ \chi	\psiψ \psi	\omegaω \omega\\
%\varepsilonε \varepsilon	\varkappaϰ \varkappa	\varthetaϑ \vartheta	\thetasymϑ \thetasym\\
%\varpiϖ \varpi	\varrhoϱ \varrho	\varsigmaς \varsigma	\varphiφ \varphi\\


Sometimes accents are put above or below symbols. The control words used for accents
in mathematics are different from those used for normal text. The normal text control words
may not be used for mathematics and vice-versa.\footnote{\texbook 135-136 }. See also
\href{mathmode.pdf}{http://www.tex.ac.uk/tex-archive/info/math/voss/mathmode/Mathmode.pdf}

\section{Fractions}

In the original \tex there are two ways of typesetting a fraction: it can be typeset as $1/8$ or in the form $\frac{1}{4}$. The first case is entered with no special control characters that is,  \verb+ $1/8$+. The second case is just entered with the control word \cmd{\over}.  Hence\verb+ ${1} \over {8}$+ gives ${1} \over {8}$. 
Plain TeX also provided \docAuxCommand{frac}, which was also adopted by \LaTeX\  and \pkgname{amsmath}.
The |\over| command led to numerous problems and its usage is frowned upon. So do not use it.

A more complex example,

\begin{texexample}{Fractions}{ex:fractions}
\[
\dfrac{a +\gamma}{\delta^2}
\]

\end{texexample}



One complication with using maths in-line with text is that of using different type of fonts. There is a very useful and interesting discussion regarding this in the package \pkg{xfrac} \footcite{xfrac}. 

\begin{latexquotation}
 One of the first exercises in \emph{The \TeX Book} is to design a
 macro for split level fractions. The solution presented is fairly
  simple, using a \emph{virgule} (a slash) for separating the two
  components. It looks okay because the text font and math font of
  Computer Modern look almost identical.\index{virgule}\index{solidus}\index{fractions>virgule}\index{fractions>solidus}

  The proper symbol to use instead of the virgule is a \emph{solidus}\footnote{The solidus (/)} \index{solidus} is a punctuation mark used to indicate fractions including fractional currency. It may also be called a shilling mark, an in-line fraction bar, or a fraction slash. Its Unicode encoding is \texttt{U+2044}.
\end{latexquotation}

The solidus is similar to another punctuation mark, the slash, which is found on standard keyboards; the slash is closer to being vertical than the solidus. These are two distinct symbols that traditionally have entirely different uses. However, many people no longer distinguish between them, and when there is no alternative it is acceptable to use the slash in place of the solidus.
Both the ISO and Unicode designate the solidus as \texttt{FRACTION SLASH U+2044} and the slash as \texttt{SOLIDUS U+002F}. This contradicts long-established English typesetting terminology (See Bringhurst).
  which does not exist in Computer Modern. It is however available in
  the European Computer Modern fonts, but I'll get back to that.

Wills also notes in the |xfrac| documentation package the limitations
of the |nicefrac| package:

\begin{quotation}
  The most common way to produce split level fractions within \LaTeX\
  is by means of the \docpkg{nicefrac} package. Part of the reason it
  has found widespread use is due to the strange design of the
  built-in text fractions of the EC fonts, which look like this:
  \textonehalf. 
\end{quotation}

The package is very simple to use but there are a few
issues:

 \begin{itemize}
  \item It uses the virgule instead of the solidus.
  \item Font size of numerator and denominator is bigger than in the
    built-in symbol. Compare Palatino: \switch{ppl}{\nicefrac{1}{2}}
    vs. \switch{ppl}{\textonehalf }. (\sfrac{1}{2})

  \item It doesn't correct for fonts using text figures such as in the
    \docpkg{eco} package. Compare \switch{cmor}{\nicefrac{1}{2}} and
    \switch{cmor}{\nicefrac{8}{9}}.
  \item In math mode, it doesn't always pick up the correct math
    alphabet.
 \end{itemize}

In short: \pkg{nicefrac} doesn't attempt to be the answer to
everything and so this is not a criticism of the package. It works
quite well for Computer Modern which was pretty much what was widely
available at the time it was developed. Users these days, however,
have a choice of many fonts when they write their documents.

\subsection{Inline fractions, changing fonts}

When a fraction is displayed in inline text $\frac{a}{b}$, it is noticeably smaller than in
displayed mathematics. The \cs{tfrac} and \cs{sfrac} force the use of the respective styles |\textstyle| and |\displaystyle| $\sfrac{a}{b}$ and $\tfrac{a}{b}$. 

With |xfrac| we can switch fonts easily

 ``You take \sfrac[cmr2]{1}{2} cup of sugar, \ldots''
``You take $\tfrac{1}{2}$ cup of sugar, \ldots''

This is more of an issue if you are going to use |XeLaTeX| and 
you are after specific fonts.

An early guide to the typography of fractions appeared in the Notices of the Royal Astronomical Society, Volume 8, Issue 6, 1909. 

\begin{quotation}
First, figures or letters of a smaller size than those to which they are appended have to be set as indices or suffixes; and consequently, except when the expressiosn are of such frequent occurence as to make it worthhwile
to have them cast upon type of the various bodies with which they are used, it becomes necessary to fit these smaller types in their proper positions by special methods. This process, which is called \enquote{justification,}
consists in filling up the difference between the bodies of the larger and smaller types with suitable pieces of metal.

The second difficulty
\end{quotation}

Some guidelines

Instead of $\dfrac{x}{3}$ write $\frac{1}{3}x$

Instead of \(\dfrac{a+b}{2}\) write \(\frac{1}{2}(a+b)\).

\section{Continued fractions}

At first glance nothing seems simpler than writing a number, for example $\frac{9}{7}$ in the form

\[ \frac{9}{7} = 1 + \frac{2}{7} = 1 + \frac{1}{\frac{7}{2}} = 1 + \frac{1}{3 + \frac{1}{2}}=1+\frac{1}{3+\frac{1}{1+\frac{1}{1}}}\]

It is not known when, and by whom continued fractions were first used, but the notion seems to be quite old. Euclid's algorithm for finding the greatest common divisor of two integers in effect converts a fraction into a terminating continued fraction. Continued fractions were used in India in the 12th century by Bhaskaracharya for solving the `Pell' equation. Rafael Bombelli (1526--1572) and Pietro Antonio Cataldi (1548--1626), both of Bologna (Italy), gave continuous fractions for $\sqrt{13}$ and $\sqrt{18}$ respectively. 

The most common notation for continued fractions was introduced in 1898 by Alfred Pringsheim but previously many other notations were used. We can write a continued fraction of this form using the \cs{cfrac} command from the \pkg{amsmath} package. 

\begin{texexample}{Continued fractions}{ex:contfrac}
\begin{equation}
  x = a_0 + \cfrac{1}{a_1 
          + \cfrac{1}{a_2 
          + \cfrac{1}{a_3 + \cfrac{1}{a_4}}}}
\end{equation}
\end{texexample}

\begin{texexample}{Continued Fractions}{ex:cfrac}
\begin{luacode}
function cfrac(a,b,finite)
  local b0,b1,b2,b3 = "b_0","b_1","b_2", "b_3" 
  if type(b)=="number" then 
    b0, b1, b2, b3 = tostring(b),tostring(b),tostring(b),tostring(b)
  end  
  local ddots="\\ddots"
  if finite then 
    ddots = "\\frac{1}{a_n}"
  end  
  local str = a..
    "_0 + \\cfrac{" .. b0 .. "}{"..
     a..
     "_1 + \\cfrac{" .. b1 .."}{" .. 
     a .. 
     "_2 + \\cfrac{" .. b2 .."}{" ..
     a..
     "_3 + \\cfrac{"..b3.."}{"..
     a..
     "_4 + " .. ddots .."}}}}"
  return str         
end
tex.print("\\begin{gather}")
tex.print(cfrac("a","b",true))
tex.print("\\end{gather}")
\end{luacode}
\end{texexample}

The real discoverer of continued fractions, the Italian mathematician Pietro  Cataldi\footnote{Bologna, 1548 - Bologna, 11.2.1626)} used, in 1613, the  notation

{\arraystretch{0}\setlength\arraycolsep{2pt}
\[
\begin{array}{llll}
4 & .  &               &\\
  & \& & \dfrac{2}{8.}  &\\
  &    &               & \& \dfrac{2}{8.}\\
\end{array}
\]
}
which for convenience, he modified to become

\[4.\&\frac{2}{8.}\&\frac{2}{8.}\]
The point signified that the following fraction is a fraction of the denominator. Later in his book, Cataldi placed the point on the same line as the sign \&. 

This can be seen in his approximation of \sqrt{18}.

{\def\arraystretch{3}
\def\temp{.\&\dfrac{2}{8}}
\[
\begin{array}{rl} 
\text{Di 18. la $R$ sia }\quad  & 4\temp\temp\temp \temp \temp\temp\\
\text{\`o vogliamo dire }\quad  & 4\temp\temp\temp\temp\temp.\&\dfrac{1}{4}\\
\text{cio\`e}\quad              & 4\temp\temp\temp\temp.\&\dfrac{8}{33}\\
\text{vogliamo dire}\quad       & 4\temp\temp\temp.\&\dfrac{1}{4}\dfrac{4}{33}\\
\text{che \`e}\quad             & 4\temp\temp\temp.\&\dfrac{33}{136}\\
\text{chio\`e}\quad             & 4\temp\temp.\&\dfrac{272}{1121}\\
\text{\`o vogliamo dire}\quad   & 4\temp.\&\dfrac{1}{4}\dfrac{136}{1121}\\
\text{che \`e}\quad             & 4\temp.\&\dfrac{1121}{4620}\\
\text{chio\`e}\quad             & 4.\&\dfrac{9240}{3801}.''
\end{array}
\]
}
Except for the \enquote{\&} in place of \enquote{+} \textsc{CATALDI}'s notation is quite similar to one of our modern notations.


In 1655, John Wallis used

\begin{gather}
\setlength\arraycolsep{1pt}
\sqrt{18} =  \begin{array}[c]{c}
\dfrac{a}{\alpha}\\ 
\end{array}
\begin{array}[c]{c}
 \\[1em] \dfrac{b}{\beta} \\ 
\end{array}
\begin{array}[c]{c}
 \\[2em] \dfrac{c}{\gamma}\\
\end{array}
\begin{array}[c]{c}
 \\[3em]  \dfrac{d}{\delta}\\
\end{array}
\begin{array}[c]{c}
 \\[4em] \dfrac{e}{\epsilon}\:\:\text{etc.}\\
\end{array}
\end{gather}

To typeset this form of a continuous fraction I have used the |array| environment. We could also replace the arrays with box commands or even use plainTeX |halign| etc.




Leibniz, in a letter to Johan I Bernoulli dated Decmeber 28, 1696 used
\begin{gather}
\setlength\arraycolsep{1pt}
a + \begin{array}[c]{c}
\dfrac{1}{b}
\end{array}
\begin{array}[c]{c}
\underset{\raisebox{-.6em}{+}}{}\\ 
\end{array}
\begin{array}[c]{c}
 \\ \dfrac{1}{c} \\ 
\end{array}
\begin{array}[c]{c}
\underset{\raisebox{-2em}{+}}{}\\ 
\end{array}
\begin{array}[c]{c}
 \\[1.5em] \dfrac{1}{d}\\
\end{array}
\begin{array}[c]{c}
\underset{\raisebox{-3.5em}{+}}{}\\ 
\end{array}
\begin{array}[c]{c}
 \\[3em]  \dfrac{1}{e}\\
\end{array}
\begin{array}[c]{c}
 \\[4.6em]  {+}\:\:\text{etc.}\\
\end{array}
\end{gather}

Christiaan Huygens wrote in a posthumous text published in 1698

\newlength\deltaplus
\newlength\deltaterm

\begin{gather}\label{eq:huygens}
\setlength\arraycolsep{1pt}\def\arraystretch{0}
\setlength\deltaplus{1.3em}
\setlength\deltaterm{1.1em}
\def\ARRPLUS{\begin{array}[c]{c}
\\[\the\deltaplus]{+}\\            %+1 em
\end{array}%
\addtolength\deltaplus{1em}}
\def\ARRTERM#1#2{%
   \ARRPLUS
  \begin{array}[c]{c}
   \\[\the\deltaterm] \dfrac{#1}{#2}  
  \end{array}
  \addtolength\deltaterm{1em}
}%
\def\ARRLASTTERM#1#2{%
  \ARRPLUS
  \begin{array}[c]{c}
   \\[\the\deltaterm] \dfrac{#1}{#2}\:\:\text{etc}  
  \end{array}
}%
%
%
\dfrac{77708431}{2640858} = 29 + \dfrac{1}{2}  
\ARRTERM{1}{2}
\ARRTERM{1}{2} 
\ARRTERM{1}{5}                             
\ARRTERM{1}{1}
\ARRLASTTERM{1}{4}
\end{gather}

In 1834, Maritz Abraham Stern designates the finite continued fraction

\def\cfracc#1#2{\dfrac{#1\vert}{\vert#2}}

\begin{gather}
a + \dfrac{b_1\vert}{\vert a_1} + \cfracc{b_2}{a_2} +\dots+ \cfracc{b_m}{am}
\end{gather}

This is one of the easiest notation to typeset with \tex. But there is one catch. We have a multitude of choices as to what to use as the vertical bar. We have |\vert|, |\mid|, |\lvert|, |\rvert|, |\lVert|, |\rVert|.
The |\vert| is defined in source2e whereas the rest by the amsmath or similar packages. If we read the amsldoc manual, by texdoc amsldoc, we read:

\begin{quote}
The amsmath package provides commands |\lvert|, |\rvert|, |\lVert|, |\rVert| (compare |\langle|, |\rangle|) to address the problem of overloading for the vert bar character $\vert$. This character is currently used in LaTeX documents to represent a wide variety of mathematical objects [\ldots]. The multiplicity of uses in itself is not so bad; what is bad, however, is that fact that not all of the uses take the same typographical treatment, and that the complex discriminatory powers of a knowledgeable reader cannot be replicated in computer processing of mathematical documents. It is recommended therefore that there should be a one-to-one correspondence in any given document between the vert bar character $\vert$ and a selected mathematical notation, and similarly for the double-bar command \cs{|}. This immediately rules out the use of $\vert$ and \cs{|} for delimiters, because left and right delimiters are distinct usages that do not relate in the same way to adjacent symbols; recommended practice is therefore to define suitable commands in the document preamble for any paired-delimiter use of vert bar symbols:
\end{quote}
In short the general advice is to create our own command.
In the next example I define the macro |\cfracc| to type it. The macro takes two parameters, one for the numerator and another for the denominator.\footnote{I have used the prefix \cs{cf} to prefix commands in this package as is a good mnemonic for continued fraction.}

\begin{texexample}{Stern's notation}{ex:stern}
\def\cfvertsymbol{\lvert}
\def\cfracc#1#2{\dfrac{#1\cfvertsymbol}{\cfvertsymbol#2}}
\[
a + \dfrac{b_1\vert}{\vert a_1} + \cfracc{b_2}{a_2} +\dots+ \cfracc{b_m}{am}
\]
\end{texexample}



\textsc{Knopff} in \emph{Theory and application of Infinite Series} used the notation for infinite continued fractions,
\newcommand{\cfK}{%
  \mathop{ % we want it to be an operator
    \mathchoice{\docfK\Huge}
               {\docfK\Huge}
               {\docfK\Huge}
               {\docfK\huge}
    }\displaylimits % not necessary, but added for clarity
}

\newcommand{\docfK}[1]{%
  \vcenter{#1\kern.2ex\hbox{$\mathrm{K}$}\kern.2ex}}
  
\begin{gather}
b_0 + \cfK_{n=1}^{\infty} \dfrac{a_n}{b_n}
\end{gather}

He wrote,

Here the sequence $(x_n)$ under examination is fomred by means of two other sequences $(a_1,a_2,\dots)$ and
$b_0,b_2,\dots$, by writing:

\[x_0=b_0, x_1=b_0 +\dfrac{a_1}{b_1}, \]

\[b_0 + \cfracc{a_1}{b_1}\]

An expression of the form
\begin{luacode}
tex.print("\\begin{gather}") 
local str = cfrac("a",1)
tex.print(str)
tex.print("\\end{gather}")
\end{luacode}
is called a \textit{continued fraction}. In general, the numbers $a_1,a_2,a_3,\ldots$,$b_1,b_2,b_3,\ldots$, may be any real or complex numbers, and the number of terms may be finite or infinite.

Here we will only deal with \textit{simple continued fractions.} These have the form
\begin{luacode}
tex.print("\\begin{gather}") 
local str = cfrac("a",1, true)
tex.print(str)
tex.print("\\end{gather}")
\end{luacode}

Some authors write continued fractions as 
\[ \frac{67}{29} = [2,3,4,2] \]

A more complicated way to write continued fractions is the old style found in some books. To use such a style
we need to use |\underset| and define a couple of commands for convenience.
\begin{teX}
\def\bottomplus{%
 \underset{\raisebox{.4ex}{ + }}{ }%
}
\def\bottomdots{%
 \underset{\raisebox{.35ex}{\ldots}}{ }%
}
\end{teX}

\def\bottomplus{%
 \underset{\raisebox{.4ex}{ + }}{ }%
}
\def\bottomdots{%
 \underset{\raisebox{.3ex}{\ldots}}{ }%
}


\begin{gather}
  \sqrt{18} = 4 + \frac{2}{8} \bottomplus \frac{2}{8} \bottomplus \frac{2}{8} \bottomplus \bottomdots
\end{gather}

Another method is using |\over|.

\begin{texexample}{Continued fraction notation}{ex:contfrac}
or in another related notation as

\[ x = a_0 + {1 \over a_1 + {}} {1 \over a_2 + {}} {1 \over a_3 {}}. \]

Sometimes angle brackets are used, like this:

\[ x = \left \langle a_0; a_1, a_2, a_3 \right \rangle. \]

\end{texexample}


\section{Square roots}

Symbols used for square roots go back to the time of the Egyptians. 

The principal symbolisms for the designation of roots, which have been developed since the influx of Arabic learning in Europe in the twelfth century, fall into four groups. 

The sign {\panunicode ℞} came to be used very extensively for \enquote{route} but occassionally it stood also for the first power of the unknown quantity, $x$.

\subsection*{Napier's line notation}

John Napier prepared a manuscript on algebra that was not printed until 1839. He made use of Stifel's notation for radicals, but at the same time devised a new scheme of his own. His notation was derived from this figure.

\[\arrayrulewidth1pt
\begin{array}{c|c|c}
1&2&3\\
\hline
4&5&6\\
\hline
7&8&9\\
\end{array}
\]


He took this simple combination
of equal \scalebox{0.5}{\begingroup\arrayrulewidth2pt$\begin{array}{c|c|c}
\Zi&\Zi&\Zi\\
\hline
\Zi&\Zi&\Zi\\
\hline
\Zi&\Zi&\Zi\\
\end{array}$\endgroup} intersecting each other at equal
intervals. In the nine compartments, which this figure
presents, he inserted the nine numerals,---for the sake, he
says, of assisting the memory, From the natural arrangement of the figures within it is easy to remember the number appertaining to each compartment. Separate the
compartments thus, \raisebox{-.35\height}{\includegraphics[scale=0.5]{napier-sqrt}} and still the memory will retain
the relation of each to the original form, even when the
numerals are withdrawn. By this simple process there is
actually obtained an equivalent for the nine significant digits
of Arithmetic, susceptible of the same combinations, yet perfectly
distinct in character.

Note that I have included an inline image in the above paragraph. In the Chapter for images you can get more details as to how to include images and adjust them relatively to the baseline. There are many methods, but perhaps the easiest is to include it in a tabular environment such as an |array| or |tabular| as they automatically center their content.
Back to our description of Napier's methods, if we are to progress further we need to define macros for the roots. 

\newcommand\napierroot{{\setlength\arraysep{0pt}
\renewcommand\arraystretch{1.5}
\arrayrulewidth1pt
\scriptsize
\mbox{\begin{array}{|c|}
\phantom{2}\\
\hline
\end{array}\:}
}}

\newcommand\napierrootv{{\setlength\arraysep{0pt}
\renewcommand\arraystretch{1.5}
\scriptsize
\mbox{\begin{array}{|c|}
\hline
\phantom{2}\\
\hline
\end{array}}
}}

Napier used $^2\napierroot\,\dfrac{22}{4}$ next to a number to indicate its root. The fifth root denoted by $\napierrootv x$.

We define a symbol to denote it $+5\mathcal{t}\mathcal{T}$.
\begin{texexample}{Napier's line symbolism}{ex:napierroot}
\def\napierrootv{{\setlength\arraysep{0pt}
\renewcommand\arraystretch{1.5}
\scriptsize
\mbox{\begin{array}[t]{|c|}
\hline
\phantom{2}\\
\hline
\end{array}}
}}
The fifth root using Napier's line symbolism $\napierrootv$
\end{texexample}

It is instructive to stop for a moment and review the various settings for an array environment. 


The symbol $\surd$ we use today, originated in Germany. Four manuscripts algebras have been available for the study of this. We can also use unicode symbols but they will be more difficult to match with fonts. {\panunicode √52}  


Another notation is 
$\begin{array}{c}
\mbox{\includegraphics[scale=0.5]{sqrt}}
\end{array}
$ and $\begin{array}{c}
\mbox{\includegraphics[scale=0.35]{radix}}
\end{array}
$

\subsection*{John Wallis (1656)}

John Wallis\footcite{wallis} used the symbol $\surd c$, $\surd qq$ for cube root, fourth root. Later in Proposition 73 he changes to the notation given as $\surd^3, \surd^4$


\begin{gather}
\begin{array}{@{}llll l l@{}}
\text{the series }          & \surd^2 0a,        &\surd^2 1a,         &\surd^2 2a,         &\surd^2 3a,           &\text{etc.}\\
\text{by the series }       & \surd^5 0b,        &\surd^5 1b,         &\surd^5 2b,         &\surd^5 3b,           &\text{etc.}\\
\text{that is, the series } & \surd^{10}0a^5,    & \surd^{10} 1a^5,   &\surd^{10}32a^5,    &\surd^{10} 243a^5,    &\text{etc.}\\
\text{by the series }       & \surd^{10}0b^2,    &\surd^{10} 1b^2,    &\surd^{10}4b^2,     &\surd^{10} 9b^2,      &\text{etc.}\\
\text{it will produce}      & \surd^{10}0a^5b^2, &\surd^{10} 1a^5b^2, &\surd^{10}128^5b^2, &\surd^{10}2187a^5b^2, &\text{etc.}\\ 
\end{array}
\end{gather}
And this holds similar in other multiplications of this kind.

\subsection*{John Kersey's notation (1673)}

A curious notation, was used by Kersey. He used $\surd(2)\!\colon\!\!\!\overline{\frac{1}{2}r - \sqrt{\frac{1}{4}rr-s}}\colon$  for 
$\sqrt{%
  \frac{1}{2}r-\sqrt{%
                     \frac{1}{4}r^2-s%
                    }%
      }$. 

We observe here the superposition of two notations for aggregation, the
Oughtredian colon placed before and after the binomial, and the vinculum. Either of these without the other would have been sufficient



The LaTeX/TeX for the |\sqrt|  is simple enough and intuitive. $\surd12 \sqrt{12}$.

\begin{texexample}{Square Roots}{}
$\sqrt{x^2+y^2}$
\end{texexample}




Notice that \tex takes care of the placement of
symbols and the height and length of the radical. To make cube or other roots, the control
words \cs{root} and \cs{of} are used. These two commands are \tex primitives. \latexe
\cs{sqrt} takes an optional argument for the root.

\begin{texexample}{nth roots}{ex:nroot}
   \[\root n \of  {1+x^n}\]
   
   \[\sqrt[n]{1+x^n}\]
\end{texexample}

\begin{docCommand}{surd}{}
 A possible alternative is to use the control word \cs{surd}; the input  \verb+ \( \surd 2 \)+ will
produce \(\surd 2\) instead of \(\sqrt{2}\). When typesetting square roots care should be taken, to use struts appropriately to get the sizing right. The square root symbol on its own can be obtained by using
\cs{sqrtsign} to get $\sqrtsign{}$.
\end{docCommand}

When typesetting roots, sometimes there are issues with heights. The following example
from \citetitle{mathmode}\footcite{mathmode} illustrates the point.

\begin{equation}
 \sqrt{a}\,%
 \sqrt{T}\,%
 \sqrt{2\alpha k_{B_1}T^i}\label{eq:root1}
\end{equation}

This can be corrected using \cs{vphantom}. 

\begin{texexample}{Correcting height issues}{ex:sqrtheights}
\begin{equation}\label{eq:root2}
 \sqrt{a\vphantom{k_{B_1}T^i}}\,%
 \sqrt{T\vphantom{k_{B_1}T^i}}\,%
 \sqrt{2\alpha k_{B_1}T^i}
\end{equation}

\begin{equation}
x = \sqrt[3]{6+\sqrt[3]{6+\sqrt[3]{6+\sqrt[3]{6+\cdots}}}}
\end{equation}
\end{texexample}

Using \pkg{amsmath} \docAuxCommand{smash} can be used for even better results when
using inline or displayed roots. It must be noted that \cs{smash} in \latexe is defined
without such an optional argument.

\makeatletter
\renewcommand{\smash}[1][tb]{%
\def\mb@t{\ht}\def\mb@b{\dp}\def\mb@tb{\ht\z@\z@\dp}%
\edef\finsm@sh{\csname mb@#1\endcsname\z@\z@ \box\z@}%
\ifmmode \@xp\mathpalette\@xp\mathsm@sh
\else \@xp\makesm@sh
\fi
}

\makeatother
This is a test $\sqrt{\lambda_{ki}}$ and $\smash[tb]{\sqrt{\lambda_{ki}}} $ 
\meaning\smash

\refCom{smash}{ \oarg{position}\marg{argument} }

The optional argument for the position can take three values: \textbf{t} keeps the bottom and annihilates the top, \textbf{b} keeps the top and annihilates the bottom and \textbf{tb} which annihilates top and bottom. The latter is the default.


The \pkgname{mathtools} interferes with some of these?

\subsection{Other spacing adjustments}

The root can also be moved right or left using \docAuxCommand{leftroot} and \docAuxCommand{uproot}.



\section{Trigonometric and other Functions}

There are several types of functions that appear frequently in mathematical text. In
an equation like $\sin2x+\cos2x=1$ the trigonometric functions \cs{sin} and \cs{cos} are in
roman rather than italic type. This is the usual mathematical convention to indicate that
it is a function being described and not the product of three variables. The control words
\cs{sin} and \cs{cos} will use the right typeface automatically. 
Here is a table of these and some other special functions:

\begin{texexample}{Trigonometric Functions}{}
\[
\sin, \cos, \tan, \cot, \sec, \csc, \arcsin, \arccos,
\arctan \sinh \cosh \tanh \coth \lim \sup \inf
\]

\begin{math}
\limsup \liminf \log \ln \lg \exp \det \deg
\dim \hom \ker \max \min \arg \gcd \Pr
\end{math}

\begin{math}
\cos(2\theta) = 2 \cos^{2}2 \theta-1
\end{math}
\end{texexample}

Functions that are missing from a basic installations can be either defined or one can use one of the many packages that supplement the above.

The distinction between algebra and geometry is an artificial one; both are parts of the same subject.

\subsection*{Bhaskara I sine formula}

In mathematics, Bhaskara I's sine approximation formula is a rational expression in one variable for the computation of the approximate values of the trigonometric sines discovered by Bhaskara~I (c. 600 – c. 680), a seventh-century Indian mathematician. This formula is given in his treatise titled Mahabhaskariya. It is not known how Bhaskara~I arrived at his approximation formula. However, several historians of mathematics have put forward different hypotheses as to the method Bhaskara might have used to arrive at his formula. The formula is elegant, simple and enables one to compute reasonably accurate values of trigonometric sines without using any geometry whatsoever.

In modern mathematical notations, for an angle $x$ in degrees, this formula gives

\begin{gather} \sin x^\circ \approx \frac{4 x (180-x)}{40500 - x(180-x)} \end{gather}


Bhaskara I's sine approximation formula can be expressed using the \emph{radian} measure of angles as follows.

\begin{gather}\sin x \approx \frac{16x (\pi - x)}{5\pi^2 - 4x (\pi - x)}.\end{gather}



Trigonometric ratios can be expressed in terms of each other:

\begin{alignat}{2}
\sin \theta &= \frac{MP}{OP}  &= \frac{s}{1}=s,        &                       \\
\cos \theta &= \frac{OM}{OP} &= \sqrt{1-s^2}           &=\sqrt{1-\sin^2\theta},\\
\tan \theta &= \frac{MP}{OM} &= \frac{s}\sqrt{1-s^2}   &=\frac{\sin \theta}{\sqrt{1-sin^2\theta}}\\
\cot \theta &= \frac{OM}{MP} & = \frac{\sqrt{1-s^2}}{s}&=\frac{\sqrt{1-\sin^2\theta}}{\sin\theta}\\
\operatorname{cosec}\theta  &= \frac{OP}{MP} & = \frac{1}{s}=\frac{1}{\sin\theta}, &
\end{alignat}

A variant environment alignat allows the horizontal space between equations
to be explicitly specified. This environment takes one argument, the number
of equation columns (the number of pairs of right-left aligned columns; the
argument is the number of pairs): count the maximum number of \&s in any row,
add 1 and divide by 2.

[expand on all environments]

[Analytical trigonometry]




Let $(a,b,c)$ be a Pythagorean triple. Dividing both sides of the defiing equation

\begin{gather}
  a^2 + b^2 = c^2
\end{gather}
by $c^2$ gives

\begin{gather}
(a/c)^2 + (b/c)^2 = 1,
\end{gather}
so that the triple gives an ordered pair of positive \emph{rational numbers}

\begin{lemma} Let (g,h) be a point other than (-1,0) lying on the unit circle, 
and let l be the line joining these two points. Then f has the equation y == 
t(x + 1), where t == h/(g + 1), 

\begin{align}
  g &= \frac{1-t^2}{1+t^2}\\
\intertext{and}  
 h  &= \frac{2t}{1+t^2}
\end{align}

\end{lemma}

Trigonometric functions first arose over 3000 years ago in relation to 
right triangles (the word trigonometry means \enquote{triangle measure}). 
Let $\triangle ABC$ be a right triangle with side lengths $a, b$ and $c$, and let $\alpha$ 
be one of its acute angles. 



\begin{Rule} Use \texttt{PGFPlots} for plotting functions.
\end{Rule}

\begin{Rule} For the degree symbol use the angle (|\ang|) command from the |siunitx| package. This is provides a semantic command.\footnote{See \protect\url{https://tex.stackexchange.com/questions/384873/what-is-the-degree-symbol} for a discussion.}
\end{Rule}

\begin{Exercise}
Plot $z = sin(x) \cdot sin(y)$ in the range $\ang{0}$ to $\ang{360}$
\end{Exercise}

\begin{tikzpicture}
\begin{axis}
\addplot3[
surf,
domain=0:360,
samples=40,
] {sin(x)*sin(y)};
\end{axis}
\end{tikzpicture}
\begin{tikzpicture}
\begin{axis}
\addplot3[
surf,
domain=0:360,
samples=40,
] {cos(x)*cos(y)};
\end{axis}
\end{tikzpicture}

\begin{tikzpicture}
\begin{axis}
\addplot3[
surf,
domain=0:360,
samples=40,
] {sin(x)+cos(y)};
\end{axis}
\end{tikzpicture}
\begin{tikzpicture}
\begin{axis}
\addplot3[
surf,
domain=0:180,
samples=40,
] {sin(x*x)+cos(y*y)};
\end{axis}
\end{tikzpicture}

\begin{tikzpicture}
\begin{axis}
\addplot gnuplot [
id=sin,
] {sin(x)};
\end{axis}
\end{tikzpicture}


\begin{tikzpicture}
\begin{axis}[
xmin=-2,
ymin=-2,
xlabel= $x$,
ylabel= $y$,
grid=major,
]
\addplot[
domain=-1.5:2,
samples=40,
] {e^x-1};
\end{axis}
\end{tikzpicture}

\begin{tikzpicture}
\begin{axis}[
axis line=center,
xmin=-5.5,
ymin=-4,
xlabel= $x$,x label style={right},
ylabel= $y$,y label style= {above},
grid=major,
]
\addplot[
domain=-5.5:3,
samples=100,
] {1+x+x^2/2+x^3/6+x^4/24};
\end{axis}
\end{tikzpicture}


\begin{tikzpicture}
\datavisualization [school book axes, visualize as smooth line]
data [format=function] {
var x : interval [-1.3:1.3];
func y = \value x*\value x*\value x;
};
\end{tikzpicture}


\section{Symbols, Shortcuts and Semantic Aliasing}

Although Journals recommend not to alias LaTeX commands the harsh reality is that each topic and paper has different needs for notations and things can get messy.

Consider papers written for computations of floating algorithms. We might find in preambles definitions like, 
{\parskip0pt\large
\begin{verbatim}
%% Floating-point machine operations.
\newcommand{\mop}{\boxcircle}
\newcommand{\madd}{\boxplus}
\newcommand{\msub}{\boxminus}
\newcommand{\mmul}{\boxdot}
\newcommand{\mdiv}{\boxdiag}
\end{verbatim}
}
\newcommand{\mop}{\boxcircle}
\newcommand{\madd}{\boxplus}
\newcommand{\msub}{\boxminus}
\newcommand{\mmul}{\boxdot}
\newcommand{\mdiv}{\boxdiag}

I consider aliasing of non-semantic commands to semantic commands, such as the machine operation |\mop| from |\boxcircle| as good practice. 

\begin{texexample}{Example with macro aliasing}{ex;aliasing}
\begin{Definition} \summary{(Absorption.)}
Let $x, y, z \in \Fset$ with $y, z \in \Rset$,
let $\mathord{\mop}$ be any IEEE~754 floating-point operator,
and let $x = y \mop z$.
Then $y \mop z$ gives rise to \emph{absorption} if
\begin{itemize}
\item
$\mathord{\mop} = \mathord{\madd}$
and either $x = y$ and $z \neq 0$, or $x = z$ and $y \neq 0$;
\item
$\mathord{\mop} = \mathord{\msub}$
and either $x = y$ and $z \neq 0$, or $x = -z$ and $y \neq 0$;
\item
$\mathord{\mop} = \mathord{\mmul}$
and either $x = \pm y$ and $z \neq \pm 1$, or $x = \pm z$ and $y \neq \pm 1$;
\item
$\mathord{\mop} = \mathord{\mdiv}$,
$x = \pm y$ and $z \neq \pm 1$.
\end{itemize}
\end{Definition}
\end{texexample}


The most obvious shortcuts should be for calligraphic or fraktur math alphabets.


\begin{texexample}{Calligraphic Alphabet}{ex:callalphabet}
\providecommand*{\cA}{\ensuremath{\mathcal{A}}}
\providecommand*{\cB}{\ensuremath{\mathcal{B}}}
\providecommand*{\cC}{\ensuremath{\mathcal{C}}}
\providecommand*{\cD}{\ensuremath{\mathcal{D}}}
\providecommand*{\cE}{\ensuremath{\mathcal{E}}}
\providecommand*{\cF}{\ensuremath{\mathcal{F}}}
\providecommand*{\cG}{\ensuremath{\mathcal{G}}}
\providecommand*{\cH}{\ensuremath{\mathcal{H}}}
\providecommand*{\cI}{\ensuremath{\mathcal{I}}}
\providecommand*{\cJ}{\ensuremath{\mathcal{J}}}
\providecommand*{\cK}{\ensuremath{\mathcal{K}}}
\providecommand*{\cL}{\ensuremath{\mathcal{L}}}
\providecommand*{\cM}{\ensuremath{\mathcal{M}}}
\providecommand*{\cN}{\ensuremath{\mathcal{N}}}
\providecommand*{\cO}{\ensuremath{\mathcal{O}}}
\providecommand*{\cP}{\ensuremath{\mathcal{P}}}
\providecommand*{\cQ}{\ensuremath{\mathcal{Q}}}
\providecommand*{\cR}{\ensuremath{\mathcal{R}}}
\providecommand*{\cS}{\ensuremath{\mathcal{S}}}
\providecommand*{\cT}{\ensuremath{\mathcal{T}}}
\providecommand*{\cU}{\ensuremath{\mathcal{U}}}
\providecommand*{\cV}{\ensuremath{\mathcal{V}}}
\providecommand*{\cW}{\ensuremath{\mathcal{W}}}
\providecommand*{\cX}{\ensuremath{\mathcal{X}}}
\providecommand*{\cY}{\ensuremath{\mathcal{Y}}}
\providecommand*{\cZ}{\ensuremath{\mathcal{Z}}}
\begin{luacode}
local alphabet = {"A","B","C","D","E","F","G","H","I","J","K","L",
                  "M","N","O","P","Q","R","S","T","U",
                  "V","W","X","Y","Z"}
for k,v in pairs(alphabet) do
	tex.print("\\c"..v..", ")
end
\end{luacode}
\end{texexample}


\section{Calculus}



\subsection*{Differentiation}
Leibnitz invented the transmutation theorem for finding the area under a curve. Finding areas beneath curves
was 


\subsection*{Integration}

\subsection*{Limits}

\marginpar{\includegraphics[width=3cm]{bolzano}
{\scriptsize\RaggedLeft Bolzano begins his work by explaining what he means by theory of science, and the relation between our knowledge, truths and sciences. Human knowledge, he states, is made of all truths (or true propositions) that men know or have known. This is, however, only a very small fraction of all the truths that exist, although still too much for one human being to comprehend. Therefore, our knowledge is divided into more accessible parts. Such a collection of truths is what Bolzano calls a science (Wissenschaft). It is important to note that not all true propositions of a science have to be known to men; hence, this is how we can make discoveries in a science.}}

\textbf{Bolzano} Although implicit in the development of calculus of the 17th and 18th centuries, the modern idea of the limit of a function goes back to Bolzano who, in 1817, introduced the basics of the epsilon-delta technique to define continuous functions. However, his work was not known during his lifetime.\footcite{coolidge} The first publication in English had to wait for almost 150 years and it was published by Coolidge in 1949 in his popular work \emph{The Mathematetics of Great Amateurs.} Grattan-Guiness\footcite{grattan-guiness} falls short though of accusing Cauchy of plagiarizing the work of Bolzano and other contemporaries.


\textbf{Cauchy} discussed variable quantities, infinitesimals, and limits and defined continuity of {$\displaystyle y=f(x)$} $y=f(x)$ by saying that an infinitesimal change in $x$ necessarily produces an infinitesimal change in $y$ in his 1821 book \emph{Cours d'analyse}, while (Grabiner 1983) claims that he only gave a verbal definition. 

\textbf{Weierstrass} first introduced the epsilon-delta definition of limit in the form it is usually written today. He also introduced the notations $\lim$ and $\lim x\rightarrow x_0$. The notation which has been used widely at the time in Europe was
$\operatorname{Lim}_{x=a}$, to express the \enquote{the limit as $x$ approaches $a$.} It is found in papers of Weierstrass, who in 1841 wrote $\lim$ and in 1854 \enquote{$\underset{n=\infty}{\operatorname{Lim.}}\:p_n = \infty $}

In later publications we find $\operatorname{Lim}$ without the point, but still capitalized. I particularly like the way the equations were grouped and the limit indicated by cases and shown in fig.~\ref{weis}.

\begin{figure}[htbp]
\centering
\includegraphics[width=0.8\textwidth]{weistrass}
\caption{Extract from Weierstrass publication.}
\label{weis}

\end{figure}

To create a new operator symbol we can use \cs{operatorname}. The command will take care of spacing, whereas if we use only
|\mathrm| it will not. If we were to use it frequently we could also define a macro using \cs{DeclareMathOperator} from amsmath. 

\begin{texexample}{operatorname}{ex:onames}

$$\operatorname{Lim.}p_n$$

$$\operatorname{Lim}.p_n$$

$$\mathrm{Lim.}p_n$$ 
\end{texexample}

By the early 1900's the notation started approaching more to what is commonly used in today's publications.

\textbf{Hardy} The modern notation of placing the arrow below the limit symbol is due to Hardy in his book \emph{A Course of Pure Mathematics} in 1908, however in the preface to the book he credits Leathem and Bromwich\footcite{bromwich}.\footnote{Hardy wrote ... there are two respects in which I have diverted ...} 


\begin{gather}
\begin{aligned}
f(a) &= f\left[\lim_{k\rightarrow\infty} x_k \right]=\lim_{k} f(x_k)\le 0 & \text{ and }\\
f(a) &= f\left[\lim_{k\rightarrow\infty} X_k \right]=\lim_{k} f(X_k)\ge 0. & \\  
\end{aligned}
\end{gather}

In Cauchy's words, these inequalities established that ``the quantity
$f(a)$ \ldots cannot differ from zero.'' He had thus proved the existence of a
number a between $x$ and $X$ for which $f(a) = O$. The general version of the
intermediate value theorem, namely that a continuous function takes all
values between $f(x_0)$ and $f(X)$, follows as an easy corollary.
This was a remarkable achievement. Cauchy had, for the most part,
succeeded in demonstrating a "self-evident" principle by analytic methods.


Apart from open intervals, limits can be defined for functions on arbitrary subsets of R, as follows.  Let $f$ be a real-valued function defined on a subset $S$ of the real line.  Let $p$ be a limit point of ''S'' that is, ''p'' is the limit of some sequence of elements of ''S'' distinct from p.  The limit of ''f'', as ''x'' approaches ''p'' from values in ''S'', is ''L'' if, for every {{nowrap|''ε'' > ''0''}}, there exists a δ > ''0'' such that {{nowrap|0 < {{abs|''x'' − ''p''}} < ''δ''}} and {{nowrap|''x'' ∈ ''S''}} implies {{nowrap|{{abs|''f''(''x'') − ''L''}} < $\epsilon$.

This limit is often written

\[
L = \underset{x\in S}{\lim_{x\to p}} f(x).
\]

The condition that $f$ be defined on $S$ is that $S$ be a subset of the domain of $f$.  This generalization includes as special cases limits on an interval, as well as left-handed limits of real-valued functions (e.g., by taking ''S'' to be an open interval of the form $(-\infty,a)$, and right-handed limits (e.g., by taking ''S'' to be an open interval of the form $(a,\infty)$. It also extends the notion of one-sided limits to the included endpoints of (half-)closed intervals, so the Square root function $f(x)=\sqrt{x}$ can have limit 0 as $x$ approaches 0 from above.




\section*{Fraktur letters}

Individual Fraktur letters are sometimes used in mathematics, which often denotes associated or parallel concepts by the same letter in different fonts. For example, a Lie group is often denoted by ''G'', while its associated Lie algebra is $\mathfrak{g}$. 
A ring ideal might be denoted by $\mathfrak{a}$ (or $\mathfrak{p}$ if a prime ideal) while an element is $a \in \mathfrak{a}$. The Fraktur $\mathfrak c$ is also sometimes used to denote the cardinality of the continuum, that is, the cardinality of the real line. In model theory, $\mathfrak{A}$ is used to denote an arbitrary model, with ''A'' as its universe. Fraktur is also used in other ways at the discretion of the author.


\begin{luacode}
local alphabet = {"A","B","C","D","E","F","G","H","I","J","K","L",
                  "M","N","O","P","Q","R","S","T","U","V","W","X","Y","Z"}
for k,v in pairs(alphabet) do
	tex.print("\\mathfrak{"..v.."}, ")
end
\end{luacode}

Special symbols such as dingbats loaded using the \pkg{pifont} package need to be enclosed within an \docAuxCommand{mbox} in order to be able to display the glyph properly.
\bigskip

\begin{texexample}{Using special dingbat symbols}{}
\label{e14}
 We start by showing that the function $f(x)=x^2$ is
continuous over the set $X_2$\label{p:X2} defined as the interval
$[0,1]$ where numerals $\frac{i}{\mbox{\ding{172}}}, 0 \le i \le
\mbox{\ding{172}},$ are used to express its points in units $\mu$.
First of all, note that the set $X_2$ is continuous in   $\mu$
because its points are equidistant with the distance
$d=\mbox{\ding{172}}^{-1}$. Since this function is strictly
increasing,  to show its continuity it is sufficient to check the
difference $f(x)-f(x^{-})$ at the point $x=1$. In this case,
$x^{-}=1-\mbox{\ding{172}}^{-1}$ and we have
\[
 f(1)-f(1-\mbox{\ding{172}}^{-1})=
 1-(1-\mbox{\ding{172}}^{-1})^2 =
 2\mbox{\ding{172}}^{-1}(-1)\mbox{\ding{172}}^{-2}.
\]
This number is infinitesimal, thus $f(x)=x^2$ is continuous over
the set $X_2$. \hfill $\Box$
\end{texexample}

\cs{Box}
Notice the use of the \docAuxCommand{Box} command to draw a square for  end of proof symbol. This is from the amsmath package. The box is placed at the end of the line using |\hfill $\Box$|.

%
%\begin{Rule}
%Alias common symbols to semantic macro names that follow their mathematical definitions.
%\end{Rule}
%
%\begin{Rule}
%Create shorcut commands with moderation for alphabets and algebras and alias if necessary. Good practice, sets etc.
%\end{Rule}



\section{Partial derivatives}

Partial derivatives can be typeset using the \latex{} command \cs{partial}

\begin{texexample}{Partial Derivatives}{ex:partial}
\[
dS = \frac{\partial S}{\partial x}\, dx
   + \frac{\partial S}{\partial y}\, dy
   = \left(y - \frac{2V}{x^2}\right) dx
   + \left(x - \frac{2V}{y^2}\right) dy.
\]
\meaning\partial
\end{texexample}


This is using a certain style. For other styles it maybe more difficult.

\section*{Binomial Coefficients}

To typeset binomial coefficients or similar structures, use the command
\cs{binom} from \pkg{amsmath}\index{amsmath>binom}

\begin{texexample}{Binomial}{ex:binomial0}
\[
\begin{align}
(x+y)^3 & = x^3 + 3x^2y + 3xy^2 + y^3, \\[8pt]
(x+y)^4 & = x^4 + 4x^3y + 6x^2y^2 + 4xy^3 + y^4, \\[8pt]
(x+y)^5 & = x^5 + 5x^4y + 10x^3y^2 + 10x^2y^3 + 5xy^4 + y^5, \\[8pt]
(x+y)^6 & = x^6 + 6x^5y + 15x^4y^2 + 20x^3y^3 + 15x^2y^4 + 6xy^5 + y^6, \\[8pt]
(x+y)^7 & = x^7 + 7x^6y + 21x^5y^2 + 35x^4y^3 + 35x^3y^4 + 21x^2y^5 + 7xy^6 + y^7.
\end{align}
\]
\end{texexample}


Pascal's rule can be typeset as:


\begin{texexample}{Binomial Distribution}{ex:binomial}
\[
\binom{n}{k} =\binom{n-1}{k}
+ \binom{n-1}{k-1}
\]
\end{texexample}



\section[Symbols and Abbreviations]{Symbols and Abreviations}
\label{math:abbreviations}

\label{abbr}\label{symbols}%

\begin{texexample}{Creating symbols}{ex:parallelogram}
\newlength{\dentwidth}\setlength{\dentwidth}{\textwidth}
\addtolength{\dentwidth}{-\parindent}

\makeatletter
\gdef\@parallelogram#1{%
  \textnormal{\setbox\z@\hbox{#1/}\dimen@\wd\z@
   \@tempdima 2.45\dimen@
   \vbox{\offinterlineskip
      \hbox{\kern.8\dimen@\vrule\@width\@tempdima\@height.4\p@}%
      \kern-.0\p@
      \hbox to\@tempdima{#1/\hfil\rlap/}%
      \kern-.5\p@
      \hbox{\kern.1\dimen@\vrule\@width\@tempdima\@height.4\p@}}}}
      
 \gdef\Par{%
   \mathchoice
      {\@parallelogram\scriptsize}%
      {\@parallelogram\scriptsize}%
      {\@parallelogram\tiny}%
      {\@parallelogram\tiny}}

            



\begin{tabular*}{\dentwidth}{rl@{\extracolsep{\fill}}l@{\extracolsep{0pt}}@{\dots}l}

$>$ & is (or are) greater than. & Def. & definition. \\
$<$ & is (or are) less than. & Ax. & axiom. \\
$\Bumpeq$ & is (or are) equivalent to. & Hyp. & hypothesis. \\
$\therefore$ & therefore. & Cor. & corollary. \\
$\perp$ & perpendicular. & Scho. & scholium. \\
$\perp_s$ & perpendiculars. & Ex. & exercise. \\
$\parallel$ & parallel.\qquad $\parallel_s$ parallels. & Adj. & adjacent. \\
$\angle$ & angle.\qquad $\angle_s$ angles. & Iden. & identical. \\
$\triangle$ & triangle.\qquad $\triangle_s$ triangles. & Const. & construction. \\
$\Par$ & parallelogram. & Sup. & supplementary. \\
$\Par_s$ & parallelograms. & Ext. & exterior. \\
$\odot$ & circle.\qquad $\odot_s$ circles. & Int. & interior. \\
rt. & right.\qquad  st.\ straight. & Alt. & alternate. \\
\end{tabular*}


\hbox{\qed\ stands for quod erat demonstrandum, \emph{which was to be proved}.\hss}

\newcommand{\qef}{\textsc{q.e.f.}}
\qef\ stands for quod erat faciendum, \emph{which was to be done.}

The signs $+$, $-$, $\times$, $\div$, $=$, have the same meaning as in Algebra.

\makeatother   
\end{texexample}
   
\section{Afixing symbols to other symbols}

\latex provides \docAuxCommand{stackrel} for placing a superscript above a binary relation. In the \pkg{amsmath} package there are somewhat more general commands \docAuxCommand{overset} and \docAuxCommand{underset}, that can be used to place one symbol above or below another symbol, whether is a relation or something else. Oberdiek's package \pkg{stackrel}\footcite{stackrel} extends the syntax by adding an optional argument for the subscript position. It follows the syntax of extensible arrows of the packages |amsmath| and |mathtools|.

\begin{docCommand}{stackrel}{\oarg{subscript}\marg{superscript}{\marg{rel}}}
  Typeset a subscript or superscript above a symbol.
\end{docCommand}

\ifSTACKREL
\begin{texexample}{Example of stackrel and stackbin}{ex:stackrel}
\[
 A \stackbin[\text{and}]{}{+} B \stackrel[x]{!}{=} C
\] 
\end{texexample}
\else
\begin{texcode}{Example of \cs{stackrel} and \cs{stackbin}}{ex:stackrel}
  Example cannot be shown as the package stackrel is not loaded.
\end{texcode}
\fi


\chapter{Matrices and Mathematical Environments}
\label{matrices}

\section{Matrices}

Matrices have a long history of application in solving linear equations but they were known as arrays until the 1800s. The Chinese text The Nine Chapters on the Mathematical Art written in 10th–2nd century BCE is the first example of the use of array methods to solve simultaneous equations,[102] including the concept of determinants. In 1545 Italian mathematician Gerolamo Cardano brought the method to Europe when he published Ars Magna.[103] The Japanese mathematician Seki used the same array methods to solve simultaneous equations in 1683.[104] 


The Dutch statesman and mathematician Johan (Jan) de Witt represented transformations resembling arrays in his 1659 book Elements of Curves (1659). He is also perhaps the second Mathematician to be lynched\footnote{Hypatia, was torn to bits by a lynching Christian mob in the streets of Alexandria in \textsc{AD} 415.} by a mob and had parts of their bodies eaten! [\ldots]The brothers were shot and then left to the mob. Their naked, mutilated bodies were strung up on the nearby public gibbet, while the Orangist mob partook of their roasted livers in a cannibalistic frenzy. Throughout it all, a remarkable discipline was maintained by the mob, according to contemporary observers, making one doubt the spontaneity of the event.\footcite{israel1995} \textit{Elementa Curvarum Linearum} has been described as the first textbook in analytic geometry.\footnote{The savage murder of a man that history has judged a highly competent leader is regarded by the Dutch as one of the most shameful episodes in their history.}


Between 1700 and 1710 Gottfried Wilhelm Leibniz publicized the use of arrays for recording information or solutions and experimented with over 50 different systems of arrays.[103] Cramer presented his rule in 1750.

The term "matrix" (Latin for "womb", derived from mater—mother[106]) was coined by James Joseph Sylvester in 1850,[107] who understood a matrix as an object giving rise to a number of determinants today called minors, that is to say, determinants of smaller matrices that derive from the original one by removing columns and rows. In an 1851 paper, Sylvester explains:

\begin{quote}
I have in previous papers defined a \enquote{Matrix} as a rectangular array of terms, out of which different systems of determinants may be engendered as from the womb of a common parent.[108]
\end{quote}

Arthur Cayley published a treatise on geometric transformations using matrices that were not rotated versions of the coefficients being investigated as had previously been done. Instead he defined operations such as addition, subtraction, multiplication, and division as transformations of those matrices and showed the associative and distributive properties held true. Cayley investigated and demonstrated the non-commutative property of matrix multiplication as well as the commutative property of matrix addition.[103] Early matrix theory had limited the use of arrays almost exclusively to determinants and Arthur Cayley's abstract matrix operations were revolutionary. He was instrumental in proposing a matrix concept independent of equation systems. In 1858 Cayley published his A memoir on the theory of matrices[109][110] in which he proposed and demonstrated the Cayley–Hamilton theorem.[103] Cayley introduced a particular notation for fencing the array as shown in fig~\ref{fig:cayley}, which used () for the first row and $\vert$ for subsequent rows.

\begin{figure}[htbp]
\centering
\includegraphics[width=0.6\textwidth]{cayley-notation}

\caption{Extract from Caley's paper \textit{A memoir on the theory of matrices.}}
\label{fig:cayley}

\end{figure}

An English mathematician named Cullis was the first to use modern bracket notation for matrices in 1913 and he simultaneously demonstrated the first significant use of the notation $A = [a_{i,j}]$ to represent a matrix where $a_{i,j}$ refers to the i$^$th row and the $j$th column.

Cayley's introductory paper in matrix theory was written in French and published in a German periodical. In this paper, matrices are introduced to simplify the notation which arises in simultaneous linear equations.  

\begin{gather}
\begin{aligned}
\xi  &= \alpha x + \beta y + \gamma z + \dots\\
\eta  &= \alpha' x + \beta' y + \gamma' z + \dots\\
\zeta &= \alpha'' x + \beta'' y + \gamma'' z + \dots
\end{aligned}
\end{gather}

The set of equations is written as:

\begin{gather}
(\xi, \eta, \zeta,\dots ) = (\alpha,\beta,\gamma,\dots)(x,y,z,\dots)
\end{gather}

The same article also introduces, although quite sketchily, the ideas of inverse matrix and of matrix multiplication, or \enquote{compounding} as Cayley called it. The above basic properties are expanded in a second expository article\footcite{fieldman1962} which
also lists many additional properties of matrices. In this important paper, Cayley works mostly with square matrices with
nine elements. He represents the zero matrix,\footcite[This is an interesting development]{fieldman1962}

\[
\begin{vmatrix}
0,0,0\\
0,0,0\\
0,0,0
\end{vmatrix}
\]
by \enquote{0} and the \enquote{matrix unity,}

\[
\begin{vmatrix}
1,0,0\\
0,1,0\\
0,0,1
\end{vmatrix}
\]
by \enquote{1}

Caley then introduces the algebra of matrices by defining the addition of two matrices by

\begin{gather}
(\begin{cayleymatrix}{3}
a&b&c\\
a' &b' & c'\\
a''&b'' & c''\\
a''' & b''' & c'''\\
\end{cayleymatrix}) +
(\begin{cayleymatrix}{3}
\alpha & \beta & \gamma\\
\alpha & \beta & \gamma\\
\alpha & \beta & \gamma\\
\end{cayleymatrix})\:\: = \:\:
(\begin{cayleymatrix}{3}
a+a  &b+b  &c+\gamma\\
a''+\alpha' & b'' + \beta'' & c'' +\gamma'' \\
a'''+\alpha''' & b''' + \beta''' & c''' +\gamma''' \\
\end{cayleymatrix})
\end{gather}
Caley stated but without proof, that matrices are commutative and associative under addition. Two types
of multiplication are exhibited. The first type is designated as \enquote{scalar multiplication,} that is

\[
m(\begin{cayleymatrix}{3}
a & b & c\\
a' & b' & c'\\
a'' & b'' & c''
\end{cayleymatrix}) 
= 
(\begin{cayleymatrix}{3}
ma & mb & mc\\
ma' & mb' & mc'\\
ma'' & mb'' & mc''
\end{cayleymatrix})
\]

The notation used for the matrix (\begin{cayleymatrix}{3}
a&b\\
c&d\\
e&f
\end{cayleymatrix}) is defined later on. It is a special environment.

The second type of multiplication is called \enquote{compounding,} according to the following scheme:

\begin{multline}
(\begin{cayleymatrix}{3}
a  &b &c\\
a' &b' &c'\\
a'' &b'' &c''
\end{cayleymatrix}\between \begin{cayleymatrix}{3}
\alpha & \beta & \gamma\\
\alpha'' & \beta'' & \gamma''\\
\alpha'' & \beta'' & \gamma''\\
\end{cayleymatrix})
\\=
 (\begin{cayleymatrix}{3} \bigl(a,b,c\between a, a',a''\bigr) & (a,b,c\between (\beta, \beta', \beta'') & (a,b,c\between \gamma, \gamma', \gamma'')\\
(a &b &c\\
(a &b &c
\end{cayleymatrix})\label{caleyprod}
\end{multline}

The product equation is complex to type, but very clear to understand its definition.\footnote{Uses multline environment.}


Later developments are described by Feldmann\footcite[This is the second paper that appeared in the \textit{The Mathematics Teacher relating to matrices.}][]{feldmann1963}. 

In the 1930s books on matrices written in English started to appear. The two leading ones were C.C. MacDuffee \textit{Theory of Matrices,} Springer (1933) and J. H. M. Wedderburn \textit{Lectures on Matrices}\footnote{The book was published by the American Mathematical Society and bears reseblance to a publication printed using TeX.} (1934). They both wrote matrices with double vertical lines as in

\[
\begin{Vmatrix}
a_{11} & a_{12} & \dots & a_{1n}\\
a_{11} & a_{12} & \dots & a_{1n}\\
\vdots & \vdots & \vdots & \vdots\\
a_{n1} & a_{n2} & \dots &a_{nn}
\end{Vmatrix}
\]
The use of double vertical lines was easy to typeset as well as to handwrite. It could also relate to the single vertical lines which are used to denote determinants.




%%%%%%%%%%%%%%%%%%%%%%%%%%%%%%%%%%%%%%%%%%%%%
%%%%%%%%  TYPOGRAPHY %%%%%%%%%%%%%%%%%%%%%%%%
\section{Typography of Matrices}

Matrices are mostly typed the way tabular environements are types, i.e., you need to use the tabulator sign ``\&''.
Mathematical environments are provided both by \latexe as well as amsmath. The latter are to be preferred.

\begin{docEnvironment}{array}{\marg{specifier}}
\end{docEnvironment}

The |array| environment is a \latex2e provided environment and is identical to tabular, but works automatically with mathematics. It has to be enclosed in a mathematical environment.


\begin{texexample}{Matrices}{ex:matrices}
\[
\mathbf{X} = \left(
\begin{array}{ccc}
x_1 & x_2 & \ldots \\
x_3 & x_4 & \ldots \\
\vdots & \vdots & \ddots
\end{array} \right)
\]
\end{texexample}

\section{AMS Math matrices}

The \pkg{amsmath} package provides environments that go beyond the basic \docAuxEnvironment{array} of \latex2e. The \docAuxEnvironment{pmatrix}, \docAuxEnvironment{Bmatrix}, \docAuxEnvironment{vmatrix} and \docAuxEnvironment{Vmatrix}. 
There is also a \docAuxEnvironment{smallmatrix} that is more suitable for inline text display. 

\begin{texexample}{Using \textbackslash smallmatrix}{ex:smallmatrix}
This is a small matrix $\bigl(\begin{smallmatrix}
a&b\\
c&d\\
\end{smallmatrix}\bigr)$ environment. \lorem
\end{texexample}


\begin{docEnvironment}{bmatrix}{}
The example \ref{ex:bmatrix} illustrates the typesetting of a bracketted matrix hence \ul{b}matrix. If you want the matrix equation to be numbered enclose it withn an |equation| or |gather| environment. In this documentation we have let the |equation| environment to |gather| hence it will make no difference which ever is used. (amsmath)
\end{docEnvironment}

\begin{texexample}{Bmatrix}{ex:bmatrix}
\begin{equation}
\begin{matrix}
1 & 2 \\
3 & 4
\end{matrix} \qquad
\begin{bmatrix}
p_{11} & p_{12} & \ldots & p_{1n} \\
p_{21} & p_{22} & \ldots & p_{2n} \\
\vdots & \vdots & \ddots & \vdots \\
p_{m1} & p_{m2} & \ldots & p_{mn}
\end{bmatrix}
\end{equation}
\end{texexample}

The \refEnv{bmatrix} is from the amsmath package, see also \vref{bmatrix} and \nameref{bmatrix} and \pageref{bmatrix}. 


One issue to be aware is that the \refEnv{bmatrix} does not allow more than 10 tab stops. If you need to use more, you will have to \docCounter{MaxMatrixCols} to a higher number. The |phd| package sets this automatically at 20, so you will not have to worry about it.

\begin{teX}
\setcounter{MaxMatrixCols}{20}
\end{teX}

\begin{texexample}{Large Matrices}{ex:largematrices}
\[
\begin{bmatrix}
1 & 0 & 0 & -1 & 0  & 0  & 1 & -1 & 1  & -1 & 0 \\
0 & 1 & 0 & 0  & -1 & 0  & 1 & -1 & 0  & 1  & -1 \\
0 & 0 & 1 & 0  & 0  & -1 & 1 & -1 & -1 & 0  & 1 
\end{bmatrix}
\]
\end{texexample}



\section{vmatrix}

\begin{texexample}{vmatrix}{ex:vmatrix}

\begin{gather}
\begin{vmatrix}
aa' + bb' + cc' & ea' + fb' + gc' \\
ae' + bf' + cg' & ee' + ff' + gg'
\end{vmatrix}
{} = \begin{vmatrix}
a & b \\
e & f
\end{vmatrix}  \begin{vmatrix}
a' & b' \\
e' & f'
\end{vmatrix} + \begin{vmatrix}
a & c \\
e & g
\end{vmatrix}  \begin{vmatrix}
a' & c' \\
e' & g'
\end{vmatrix}.
\end{gather}
\end{texexample}





\section{Single equations that are too long}

In many cases equations need to be written over two or more lines. The \pkgname{amsmath} package, provides environments that are suitable for this:


\begin{texexample}{The multiline amsmath environment}{ex:multiline}
\begin{multline}
   a + b + c + d + e + f+ g + h + i  + k + l + m + n + o + p\\
              = j + k + l + m + n +\cos^{2}-1
\end{multline}
\end{texexample}



\section{array environment}

This is simply the same as the |eqnarray| environment only with the possibility of
variable rows and columns and the fact, that the whole formula has only one
equation number and that the array environment can only be part of another math
environment, like the equation environment or the displaymath environment. With
@{} before the first and after the last column the additional space |\arraycolsep| is
not used, which maybe important when using left aligned equations.

\begin{texexample}{array environment}{ex:array2}
\begin{eqnarray}
  a & = & b + c \\
    & = & d + e + f + g + h + i
               + j + k + l \nonumber \\
    &   & +\: m + n + o \\
    & = & p + q + r + s
\end{eqnarray}
\end{texexample}

The equations
to be aligned are entered with each one terminated by \cs{cr}. In each equation there should be
one alignment symbol \& to indicate where the alignment should take place. This is usually
done at the equal signs, although it is not necessary to do so. For example


\begin{texexample}{The array environment}{ex:array}
Thus to change $\frac34$ to a decimal divide $4$ into $3$
and we get $.75$ as a result, thus:
\[
\begin{array}{r@{}r@{}}
4 \; & \vline \; 3.00 \\\cline{2-2}
     &            .75
\end{array}
\]

To find the square root of a four-figure number
such as our example calls for, work it out in the
following manner:
\[
\arraycolsep=0em
\begin{array}{cccccccccccc}
\multicolumn{3}{c}{\text{2d pair}} &\qquad&\qquad&
\multicolumn{3}{c}{\text{1st pair}}&\qquad&\qquad&
\multicolumn{2}{c}{\text{square root}}\\
 & \overbrace{\quad}&\ZZZ&&&\ZZZ&\overbrace{\quad}&\ZZZ\\
 & 42 &&&&& 25 &&&&\vline\;65&(answer)\\\cline{11-11}
 & 36 &&&&& \\\cline{2-2}
\multirow{2}{*}{125\:} & \vline\hfill \Zi6 \hfill&&&&& 25\\
 & \vline\hfill \Zi6 \hfill&&&&& 25\\\cline{2-7}
\end{array}
\]
\end{texexample}


\subsection{Array environment in game theory}

This example \ref{ex:gamearray} is from \footnote{From determinacy to Nash equilibrium,St\'ephane Le Roux, TU Darmstadt }

\begin{texexample}{array environment in game theory}{ex:gamearray}
The game $\langle\{a,b,c\},\{1,2,3,4\}^3,\{0,1,2,3,4\},v,(<_d)_{d\in\{a,b,c\}}
\rangle$ is represented below, where player $a$ chooses the row, $b$ the column, and $c$ the array. 
\[\begin{array}{c@{\hspace{1cm}}c@{\hspace{1cm}}c@{\hspace{1cm}}c}
\begin{array}{|c|c|c|c|}
\hline 1 & 1 & 1 & 1\\
\hline 1 & 1 & 1 & 1\\
\hline 1 & 1 & 1 & 1\\
\hline 4 & 1 & 1 & 1\\
\hline
\end{array}
&
\begin{array}{|c|c|c|c|}
\hline 1 & 2 & 1 & 1\\
\hline 2 & 2 & 2 & 2\\
\hline 1 & 2 & 1 & 1\\
\hline 4 & 2 & 1 & 1\\
\hline
\end{array}
&
\begin{array}{|c|c|c|c|}
\hline 1 & 1 & 3 & 1\\
\hline 1 & 1 & 3 & 1\\
\hline 3 & 3 & 3 & 3\\
\hline 4 & 1 & 3 & 1\\
\hline
\end{array}
&
\begin{array}{|c|c|c|c|}
\hline 2 & 4 & 4 & 4\\
\hline 4 & 3 & 4 & 4\\
\hline 4 & 4 & 4 & 4\\
\hline 0 & 0 & 0 & 0\\
\hline
\end{array}
\end{array}
\]
Let us show that the game $\langle\{a,b,c\},\{1,\dots,n\}^3,\{0,\dots,n\},v,(<_d)_{d\in\{a,b,c\}}
\rangle$ witnesses the claim. First, the preferences are linear orders indeed. Second, let us show that there is no Nash equilibrium by case-splitting below. 
\end{texexample}



\section{The AMSmath Package}

\index{maths environment>align}
\index{maths environment>align}
\index{maths environments}{falign}
\index{maths environments}{xalignat}
\index{maths environments}{xxalignat}
\index{maths environments}{eqnarray}

The \pkg{amsmath} package offers five different align environments, |align|, |alignat|, |falign|, |xalignat| and |xxalignat|. 

In difference to the \refEnv{eqnarray} environment from standard \latex the ``three'' parts of one equation expr.-symbol-expr. are divided by only one ampersand in two parts. In general the ampersand should be before the symbol to get the right spacing, for example y \&= x. 

\subsection{The align environment}

The |align| environment is considered an improvement over \latex's |eqnarray| environment. It is very similar to a tabular
environment and is aligned at the |&|. 
It requires one |&| less and produces a tighter equation. Mathematicians 
are very particular in not using |eqnarray| and the package 
\pkgname{onlyamsomath} if used will warn if it finds that it has been used in the document. 
\index{eqnarray (environment)>spacing}
\index{eqnarray (environment)>alternatives}
\index{math environments>align}
\index{math environmenta>eqnarray}


\begin{texexample}{Comparison between |align| and |eqnarray|}{ex:eqnarraycomp}
% First example with eqnarray
\begin{eqnarray}
a & = & b + c \\
  & = & d + e + f + g + h + i
        + j + k + l \nonumber \\
  &   & +\: m + n + o \\
  & = & p + q + r + s
\end{eqnarray}

% Second example with align
\begin{align}
a & =  b + c \\
  & =  d + e + f + g + h + i
       + j + k + l \nonumber \\
  & +\: m + n + o \\
  & =  p + q + r + s
\end{align}
\end{texexample}


You can observe in Example~\ref{eqnarraycomp} the essential differences between the two, which we re-iterate, it requires one less |&| and produces a tighter display.

\begin{texexample}{The align environment}{ex:align}
\begin{align}
         y & = d\label{eq:IntoSection}\\
         y & = cx+d\\
    y_{12} & = bx^{2}+cx+d\\
     y(x)  & = ax^{3}+bx^{2}+cx+d
 \end{align}

%\begin{align*}
%\therefore (13 - x_1) + (13 - x_2) + \dotsb + (13 - x_p) + r &= 52\,,\\
%\therefore 13p - (x_1 + x_2 + \dotsb + x_p) + r              &= 52\,,\\
%\therefore x_1 + x_2 + \dotsb + x_p                          &= 13p - 52 + r\\
%                                                          &= 13 (p - 4) + r\,.
%\end{align*}

whence we conclude that $\gamma$ is a primitive root modulo $p$. But
\begin{align*}
\gamma^{p-1}-1 &=
     g^{p-1} - 1 + \frac{p-1}{1!}g^{p-2}xp +
        \frac{(p-1)(p-2)}{2!}g^{p-3}x^2p^2 + \ldots \\
  &= p\left(kp + \frac{p-1}{1!}g^{p-2}x +
        \frac{(p-1)(p-2)}{2!}g^{p-3}x^2p + \ldots\right).
\end{align*}
\end{texexample}

As you can observe from example\ref{ex:align}, each line is numbered if the unstarred version of the command is used. The |aligned| environment remedies this.
 


\subsection{The aligned environment}

The aligned environment allows more than one horizontal alignment but has only one equation number.

%\newcommand{\dotsbsmall}{\ldot\!\ldot\!\ldot}
%\newcommand{\ldot}{\mathbin{.}}			% dot with math spacing
%\newcommand{\nobf}[1]{\no \textbf{#1}}		% no with bold number
%
%\begin{texexample}{aligned environment example}{}
%\begin{equation}
%\begin{aligned}
%  &\:C_1x^{r_1}\,[\varphi_{r_1 0} \,+ \varphi_{r_1 1}\log x \,+ \dotsb + \varphi_{r_1 \alpha_1}(\log x)^{\alpha_1}]\\
%+ &\:C_2x^{r_2}\,[\varphi_{r_2 0} \,+ \varphi_{r_2 1}\log x \,+ \dotsb + \varphi_{r_2 \alpha_2}(\log x)^{\alpha_2}]\\
%+ &\multispan{1}{\:\dotfill}\\
%+ &\:C_nx^{r_n}[\varphi_{r_n 0} + \varphi_{r_n 1}\log x + \dotsb + \varphi_{r_n \alpha_n}(\log x)^{\alpha_n}],
%\end{aligned}
%\end{equation}
%
%\[
%\tag{98}
%\left\{\qquad
%\begin{aligned}
%T_1 &= T_2 = T_3 (=T)\\
%p_1 &= p_2 = p_3\\
%s_1-s_2 &= \frac{(u_1-u_2)+p_1(v_1-v_2)}{T}\\
%s_2-s_3 &= \frac{(u_2-u_3)+p_2(v_2-v_3)}{T}.
%\end{aligned}
%\right.
%\]
%\end{texexample}
 




\subsection{How to interrupt a display}

\begin{docCommand}{intertext}{}
 In many instances you will want to interrupt a display with some text. This can be accomplished using the control sequence |\intertext|.
 \end{docCommand}
 
 

\begin{texexample}{Using \cs{intertext}}{ex:intertext}
\begin{gather}
\begin{aligned}
U &= M u = M(c_v T + b)\\
S &= M(c_v  \log T + \frac{R}{m}  \log v + a),\\
\intertext{and } 
F &= M \left\{T(c_v - a - c_v \log T) - \frac{RT}{m} \log v + b \right\}.
\end{aligned}
\end{gather}
\end{texexample}

As the command \docAuxCommand{intertext} can only come after a |\\|  command we place it accordingly at {1}. Its function is to preserve the alignment after the text is typeset. This is a common requirement in many mathematical 
structures and the command can be used in all of amsmath aligning environments.







\subsection{The alignat environment}

\begin{docEnvironment}{alignat}{}
The alignat environment means \emph{align at} and can be used to align a set of equations vertically at more than one place. The star version of the environment omits the equation numbering.
\end{docEnvironment}


%\begin{texexample}{The alignat environment}{}
%\renewcommand{\dotsb}{\ldots}			% use lower dots after +-
%\renewcommand{\dotsbsmall}{\ldot\!\ldot\!\ldot}
%\renewcommand{\ldot}{\mathbin{.}}			% dot with math spacing
%\renewcommand{\nobf}[1]{\no \textbf{#1}}	
%
%\begin{alignat*}{5}
%  &p_{i+1}\dfrac{d^{m-i-1}y_1}{dx^{m-i-1}} &&+ \dotsbsmall +p_{m}y_1
%  && = -\Big(\dfrac{d^{m}y_1}{dx^{m}} &&+p_1\dfrac{d^{m-1}y_1}{dx^{m-1}}
%  &&+ \dotsbsmall +p_{i}\dfrac{d^{m-i}y_1}{dx^{m-i}}\Big), \\
%\multispan{10}{\makebox[36em]{\dotfill},}\\
% &p_{i+1}\dfrac{d^{m-i-1}y_{m-i}}{dx^{m-i-1}} &&+ \dotsbsmall +p_{m}y_{m-i}\!
% &&= -\Big(\dfrac{d^{m}y_{m-i}}{dx^{m}} &&+ p_1\dfrac{d^{m-1}y_{m-i}}{dx^{m-1}}
% &&+ \dotsbsmall +p_{i}\dfrac{d^{m-i}y_{m-i}}{dx^{m-i}}\Big).
%\end{alignat*}
%\end{texexample}



Remember that when using one of the align environments, there should be no |\\| at the end of the
last line, otherwise you will get another equation number for this ``empty''  line.


\subsection{Multline}

\begin{docEnvironment}{multline}{\meta{contents}}
\end{docEnvironment}
The |multline| environment is another attempt at displaying long equations. It will set the first line flush left and the last one flush right. It can be quite useful when one has very long equations. The line break is marked with |\\|. It is good typographical practice to have the first line shorter than the last line and not the other way around.

\begin{texexample}{Multiline Equations}{mult}
Example unumbered
\begin{multline*}
x^{\rho}f(x, \rho) = x^{\rho} \Big [ u_{m}x^{m}\frac{\rho(\rho-1)\ldots (\rho-m+1)}{x^{m}} \\
                   + u_{m-1}x^{m-1}\frac{\rho(\rho-1)\ldots (\rho-m+2)}{x^{m-1}}+ \ldots
                   + u_{2}x^{2}\frac{\rho(\rho-1)}{x^2}+u_{1}x\frac{\rho}{x}+u_0 \Big ].
\end{multline*}
Example  numbered
\begin{multline}
M \left[\delta u - \left(T_1\, \frac{dp_{12}}{dT_{12}} - p_1\right) \delta v\right] \\
= \delta T_{12} \left[M_{12}\, \frac{du_{12}}{dT_{12}} + M_{21}\, \frac{du_{21}}{dT_{12}}
  - \left(T_1\, \frac{dp_{12}}{dT_{12}} - p_1\right)
    \left(M_{12}\, \frac{dv_{12}}{dT_{12}}
        + M_{21}\, \frac{dv_{21}}{dT_{12}}\right)\right].
\end{multline}
\end{texexample}



\section{gathered}

The |gathered| environment is like the |aligned| or |alignat| environment. They use
only so much horizontal space as the widest line needs. In difference to the gather
environment it must be itself inside math mode.

\begin{docEnvironment}{gathered}{\meta{contents}}
\end{docEnvironment}

\emphasize{cases}
\begin{texexample}{The gathered environment}{exe:gathered}
\[
  \left .
   \begin{gathered}
    \left [ \frac{\alpha}{p} \right ] +
    \left [ \frac{\alpha}{p^2} \right ] +
    \left [ \frac{\alpha}{p^3} \right ] +
    \ldots \\
    \left [ \frac{\beta}{p} \right ] +
    \left [ \frac{\beta}{p^2} \right ] +
    \left [ \frac{\beta}{p^3} \right ] +
    \ldots \\
      \vdots \\
    \left [ \frac{\lambda}{p} \right ] +
    \left [ \frac{\lambda}{p^2} \right ] +
    \left [ \frac{\lambda}{p^3} \right ] +
    \ldots
   \end{gathered}
  \right \} \tag{B}
\]
\end{texexample}

\section{The cases environment}

The \refEnv{cases} environment renders multiple lines with an extensible left curly-brace. It can be used for piecewise-defined functions. For this to work, you must have |\usepackage{amsmath}| in the preamble.

\newlength{\boxla}
\newlength{\boxlb}
\newlength{\boxlc}
\setlength{\boxla}{1.15in}
\setlength{\boxlb}{1.7in}
\setlength{\boxlc}{1.6in}
\newcommand{\boxa}[1]{\makebox[\boxla]{\small #1\dotfill}}
\newcommand{\boxb}[1]{\makebox[\boxlb]{\small #1\dotfill}}

\begin{docEnvironment}{cases}{}
\end{docEnvironment}


\begin{texexample}{The cases environment}{ex:cases}
\begin{align*}
\boxa{DOYEN} & \quad
\parbox{3.4in}{\small MM. \\
MILNE EDWARDS, Professeur. Zoologie, Anatomie, \\
\hspace*{1.5in} Physiologie compare.}
\\
\parbox[b]{\boxla}{\small PROFESSEURS\\HONORAIRES\dotfill} &
\begin{cases}
\text{\small DUMAS.}\\
\text{\small PASTEUR.}
\end{cases}
\\
\boxa{PROFESSEURS} &
\begin{cases}
\boxb{CHASLES}\text{\small Gomtrie suprieure.} \\
\boxb{P. DESAINS}\text{\small Physique.} \\
\boxb{PUISEUX}\text{\small Astronomie.} \\
\boxb{JAMIN}\text{\small Physique.} \\
\boxb{O. BONNET}\text{\small Astronomie.}
\end{cases}
\\
\boxa{AGROGES} &
\begin{cases}
\parbox{\boxlb}{%
\small BERTRAND\dotfill\\
J. VIEILLE\dotfill}\bigg\} \text{\small Sciences mathematiques.} \\
\boxb{PELIGOT}\text{\small Sciences physiques.}
\end{cases}\\
\boxa{SECRETAIRE} & \quad \text{\small PHILIPPON.}
\end{align*}
\end{texexample}


\begin{texexample}{}{}
non plus orthogonale mais telle que
\[
{\sum_{i}}' a_{pi} a_{qi}
  = \begin{cases}
    0 & \text{ si } p \gtrless q \\
    1 & \text{ si } p = q
    \end{cases}
\]
alors on a aussi
\[
{\sum_{i}}' a_{pi} a_{iq}
  = \begin{cases}
    0 & \text{ si } p \gtrless q \\
    1 & \text{ si } p = q
    \end{cases}
\]
\end{texexample}



\section{flalign}

\emphasize{falign}
\begin{docEnvironment}{falign}{\meta{contents}}
\end{docEnvironment}

%\cxset{
%         tag left bracket =[,
%         tag right bracket =],
%         tag font-weight=\textbf,
%      } 
%\newtagform{squarebrackets}{[}{]}
%\usetagform{squarebrackets}

\begin{texexample}{flalign}{ex:flalign}
\begin{flalign}
&&
\chi\omega  &= \omega - S \omega\, \nabla \centerdot \sigma\, dt, &&\\
&\text{whence}&
\chi'^{-1} \omega &= \omega + \nabla_1 S \omega \sigma_1\, dt, &&
\end{flalign}
\end{texexample}
%\newtagform{roundbrackets}{(}{)}
%\usetagform{roundbrackets}



\section{Sums}

\newcommand\reverseprop{\rotatebox[origin=c]{180}{$\propto$}}

Of the various special kinds of numbers used in analysis, there is hardly a species that 
that is so important and so generally applicable as the Bernoulli  
Numbers. Their numerous properties and applications have caused the creation 
of an extensive literature on the subject which still continues to attract the 
attention of scholars. The first statement of the properties of these numbers 
was given to the world by their inventor Jacques (1) Bernoulli (1654-1705) in 
his posthumously printed work, \emph{Ars Conjectandi} (Basel, 1713), pages 95 to 
98. 

Earlier scholars produced similar results the most important being Faulhaber and Remmelin of Ulm, Wallis, Mercator, in his \emph{Logarithmotechnia} and others.

Bernoulli's original notation is unfamiliar to modern mathematics. He used three notations, first the
summation symbol he used was $\smallint$ which followed Leibnitz. He also used 2 dots to indicate 
grouping. For \ldots he use an {\panunicode Ӿ}  with a bar in the middle. 
Note that Bernoulli writes $n.n-1$ where we would write $n(n-1)$. he also uses Descartes equality symbol, which in our modern notation is a mirrored proportionality symbol \(\reverseprop\) that indicates the equal sign.%
\footnote{This equality symbol was widely used in France and Holland during the latter part of the seventeeth and early eighteenth centuries, but it never attained a substantial foothold in other countries.} His use of $nn$ instead of $n^2$ should also be noted.

\def\xellipsis{\begingroup\mkern\thinmuskip \mbox{\panunicode Ӿ}\mkern\thinmuskip\endgroup}
\begin{longtable}{lll}
\smallint    & |\smallint| & Summation (modern notation  $\sum$)\\
\reverseprop & |\reverseprop| & Equality sign i.e, $=$\\
$\xellipsis$   & |\xellipsis|   & Ellipsis i.e., \ldots\\
  $n.n$        & |n.n-1|          & brackets $n(n-1)$\\
\end{longtable}

\[\smallint \overline{n-1}\]



\[\frac{n.n-1}{1.2} = \frac{nn-n}{2}\]

{\def\arraystretch{1.5}
\setlength\arraysep{3.5pt}
\begin{align}
\smallint\! n   = & \frac{1}{2}nn + \frac{1}{2}n,\\
\smallint\! nn  = & \frac{1}{3}n^3 + \frac{1}{2} nn + \frac{1}{6}n,\\
\smallint\! n^3 = & \frac{1}{4}n^4 + \frac{1}{2} n^3 + \frac{1}{4}nn,\\
\smallint\! n^4 = & \\
\smallint\! n^5 = & \frac{1}{5}n + \frac{1}{2}n^5 + \frac{5}{12}n^4 \xellipsis - \frac{1}{12}nn\\
\smallint\! n^6 = & \\
\smallint\! n^7 = & \\
\smallint\! n^8 = & \\
\smallint\! n^9 = & \\
\smallint\! n^{10} =& \frac{1}{11}n^{11} + \frac{1}{2}n^{10}+\frac{5}{6}n^9 \xellipsis -1n^7 \xellipsis 1n^5 \xellipsis -\frac{1}{2}n^3 \xellipsis  \frac{5}{66}n
\end{align}
}



\[a_1 + a_2 + \dots +a_n,\]
where each $a_k$ is a number that has been defined somehow. 
Each element $a_k$ of a sum is called a \emph{term}. 

The threedots notation has many uses, but it can be ambiguous and a bit long-winded. Another alternative, is the delimited form.

\begin{gather}
\sum_{k=1}^n a_k,
\end{gather}
which is called the Sigma-notation because it uses the upper case Greek letter sigma $\Sigma$.



\begin{equation*}
P = \frac{\displaystyle{
\sum_{i=1}^n (x_i- x)
(y_i- y)}}
{\displaystyle{\left[
\sum_{i=1}^n(x_i-x)^2
\sum_{i=1}^n(y_i- y)^2
\right]^{1/2}}}
\end{equation*}


\section{Math accents}

Mathematical accents are a bit different that the ones used for normal text in order to cater, firstly for the exotic taste in diagritics taste by mathematicians and secondly to cater for the fact that mathematics is styled in italics.
This is a short summary of what is available. 
\bigskip

%\begin{tabular}{llllll}
%\toprule
%$\hat{a}$    & \docCommand{hat\{a\}} & $\check{a}$ & \docCommand{check\{a\}} &$\tilde{a}$&\docCommand{tilde\{a\}}\\
%$\grave{a}$ &\docCommand{grave\{a\}}    & $\dot{a}$ &\docCommand{dot\{a\}} &$\ddot{a}$ &\docCommand{ddot\{a\}}\\
% $\bar{a}$ &\docCommand{bar\{a\}} & $\vec{a}$ &\docCommand{vec\{a\}} & $\widehat{AAA}$ &\docCommand{widehat\{AAA\}}\\
%$\acute{a}$ &\docCommand{acute\{a\}} &$\breve{a}$  &\docCommand{breve\{a\}} &$\widetilde{AAA}$ &\docCommand{widetilde\{AAA\}}\\
% & & & & &\\%CHEXK mathring gives problems
%\bottomrule
%\end{tabular}

\section{Binary Relations}


%\begin{tabular}{llllll}
%\toprule
%$<$ &$<$  &$>$ &$>$ &$=$ &$=$\\
%$\le$  &\docCommand{leq} or \docCommand{le}  &$\geq$ &\docCommand{geq} or \docCommand{ge} &$\equiv$ &\docCommand{equiv}\\
%$\ll$  &\docCommand{ll}   &$\gg$  &\docCommand{gg}   &$\doteq$  &\docCommand{doteq} \\
%$\prec$ &\docCommand{prec} &$\succ$  &\docCommand{succ} &$\sim$ &\docCommand{sim}\\
%$\preceq$ &\docCommand{preceq} &$\succeq$  &\docCommand{succeq} &$\simeq$ &\docCommand{simeq}\\
%$\subset$ &\docCommand{subset}  &$\supset$ &\docCommand{supset} &$\approx$ &\docCommand{approx}\\
%$\subseteq$ &\docCommand{subseteq} &$\supseteq$  &\docCommand{supseteq} &$\cong$  &\docCommand{cong} \\
%$\sqsubset$  &\docCommand{sqsubset}  &$\sqsupset$  &\docCommand{sqsupset}  &$\Join$  &\docCommand{Join}\\
%$\sqsubseteq$   &\docCommand{sqsubseteq}   &$\sqsupseteq$ &\docCommand{sqsupseteq}   &$\bowtie$ &\docCommand{bowtie} \\
%$\in$ &\docCommand{in}  &$\ni$ &\docCommand{ni}, \docCommand{owns} &$\propto$ &\docCommand{propto}\\
%$\vdash$ &\docCommand{vdash}  &$\dashv$ &\docCommand{dashv} &$\models$ &\docCommand{models}\\
%
%\bottomrule
%\end{tabular}



\section{Brackets, braces and parentheses}

In addition  to the previous commands \cmd{Bigg} and \cmd{Biggm} can be used to add a bit more horizontal space.

\[3\Big\downarrow 
\Big\Downarrow\]


\[3\Big\updownarrow
\Big\Updownarrow\]

Another way to typeset the big separators is to split them over a line as shown below

{\arraycolsep=2pt
 \begin{equation}
 \begin{array}{rcl}
 \frac{1}{2}\Delta(f_{ij}f^{ij}) & = & 2\Bigg({\displaystyle
 \sum_{i<j}}\chi_{ij}(\sigma_{i}-\sigma_{j})^{2}+f^{ij}%
 \nabla_{j}\nabla_{i}(\Delta f)+\\
 & & +\nabla_{k}f_{ij}\nabla^{k}f^{ij}+f^{ij}f^{k}[2
 \nabla_{i}R_{jk}-\nabla_{k}R_{ij}]\Bigg)
 \end{array}
 \end{equation}

This is achieved by typing

\begin{teX}
{\arraycolsep=2pt
 \begin{equation}
 \begin{array}{rcl}
 \frac{1}{2}\Delta(f_{ij}f^{ij}) & = & 2\Bigg({\displaystyle
 \sum_{i<j}}\chi_{ij}(\sigma_{i}-\sigma_{j})^{2}+f^{ij}%
 \nabla_{j}\nabla_{i}(\Delta f)+\\
 & & +\nabla_{k}f_{ij}\nabla^{k}f^{ij}+f^{ij}f^{k}[2
 \nabla_{i}R_{jk}-\nabla_{k}R_{ij}]\Bigg)
 \end{array}
 \end{equation}

\end{teX}



\chapter{Maths Typography}

\texttt{
Handbook of Typography for the\\
Mathematical Sciences\\
Steven G. Krantz\\
January 21, 2003}\par

\url{http://www.faqorama.net/tecno/[LaTeX]%20Handbook%20of%20Typography%20for%20the%20Mathematical%20Sciences%20-%20S.G.Krantz%20(2003).pdf}


Ellen Swanson’s book Mathematics into Type is a unique and important contribution to the literature of technical typesetting. It set a
standard for how mathematics should be translated from a handwritten
manuscript to a printed book or document. While Swanson’s book was
intended primarily as a resource for technical typesetters, it was also important to mathematical and other technical authors who wanted to take
an active role in ensuring that their work reached print in an attractive
and accurate form.
The landscape has now changed considerably. With the advent and
wide availability of \tex,
most mathematicians can take a more active
role in producing typeset versions of their work. Indeed, many mathematicians currently use TEX to write preliminary versions of their work
that are very similar (in many respects) to what will ultimately appear
in print.

While the output from \tex has a more typeset appearance than that
from most word processors, the TEX product is not automatically (without human intervention) \enquote{ready to go to press}. There are still \enquote{post processing} typesetting issues that must be addressed before a work
actually appears in print. 

The style and format of running heads, section headings and other titles, the formatting of theorems and other
enunciations, the text at the bottom of the page, page break issues, and
the fonts and spacing used in all of these go under the name of “page design”. These are often customized for a particular book or journal. The
index and table of contents must be designed and typeset. Graphics,
and sometimes new fonts, must be integrated. Additional questions of
style in the formatting of equations and superscripts and subscripts can
also arise. Most TEX users do not know how to handle the questions just
listed, which is why most publishers currently send \tex documents for
books or journal articles to a third-party \tex consultant. The purpose
of the present work is to serve as a touchstone for those who want to
learn to make typesetting decisions themselves.


\def\smsqr#1#2{\sqrt{{#1}^2 + {#2}^2} + \frac{1}{{#1}^2 + {#2}^2}}

\[ \smsqr{a}{c} \]

There are other aspects of consistency about which many authors
are blissfully unaware: spacing above and below a displayed equation,
spacing above and below a theorem,6
space after a proof, the mark at
the end of a proof (QED, or the Halmos "tombstone" |\qed|, for example).\footnote{ "The symbol is definitely not my invention — it appeared in popular magazines (not mathematical ones) before I adopted it, but, once again, I seem to have introduced it into mathematics. It is the symbol that sometimes looks like \(\boxed{\thinspace}\), and is used to indicate an end, usually the end of a proof. It is most frequently called the 'tombstone', but at least one generous author referred to it as the 'halmos'.", Paul R. Halmos, I Want to Be a Mathematician: An Automathography, 1985, p. 403.}

Again, a good macro can be invaluable in addressing these issues; but
awareness of the problem is also a great asset.

\begin{latexquotation}
You make everyone's
life easier if you eschew the eccentric and stick to the most basic constructions. This advice is valid for the Plain \tex user, for the \latex
user, for the Microsoft Word user, and for every other user of electronic
tools.
\end{latexquotation}



\section{Choose your notation carefully}
\index{maths>typography>notation}

Some believe that mathematics is created others that it is discovered, \emph{notation} is certainly created and
a matter that has occupied the minds of many mathematicians. Peterson\footcite{peterson2009} \footcite{peterson2009}  went as far as to claim  `that notation can direct the course of mathematics’ and perhaps rightly so.

Bad notation can make good exposition bad and bad exposition worse; ad hoc decisions about notation, made mid-sentence in the heat of composition, are almost certain to result in bad notation. Good notation has a kind of alphabetical harmony and avoids dissonance.

Leibniz had a lading role in the development of mathematical notations. He made a prolonged study of matters of notation: 
(538)

Leonhard Euler was one of the most prolific mathematicians in history, and also a prolific inventor of canonical notation. His contributions include his use of e to represent the base of natural logarithms. It is not known exactly why {$\displaystyle e$} e was chosen, but it was probably because the four letters of the alphabet were already commonly used to represent variables and other constants. Euler used {$\displaystyle \pi$ }  to represent pi consistently. The use of {$\displaystyle \pi$ }   was suggested by William Jones, who used it as shorthand for perimeter. Euler used {$\displaystyle i$}  to represent the square root of negative one,[note 41] although he earlier used it as an infinite number. [note 42][note 43] For summation, Euler used sigma, Σ.[note 44] For functions, Euler used the notation {$\displaystyle f(x)$}  to represent a function of {$\displaystyle x$} . In 1730, Euler wrote the gamma function.[note 45] In 1736, Euler produces his paper on the Seven Bridges of Königsberg[72] initiating the study of graph theory.


\begin{figure}[htbp]
\includegraphics[width=\textwidth]{shot8}
\caption{The equation of love from \emph{Rites of Love and Math}. The equation which has been shown as a tatoo  on Kayshonne Insixieng May, first appeared in a 100-page paper \emph{Instantons Beyond Topological Theory I} \cite{frenkel2012,FLN}}
\end{figure}


\subsection{One symbol, one letter}
\index{maths>typography>symbols}

A mathematical symbol is usually indicated by \emph{one} letter, not two or three. If for example we want to suggest that the \textit{factor of safety} is equal to three, we should write
\[F_{\mathrm{s}}=3\]
and not
\[F_{\mathrm{safetyfactor}}=3\]
or worse
\[F_{\mathrm{sf}}=3\]
typesetting the subscript in \textit{italic} font is also wrong
\[F_{s}=3\]
as it does not represent a mathematical symbol, but is just an abbreviation for safety factor.

Sometimes the use of the one symbol one letter rule cannot be applied, without the notation becoming complex
\medskip

{
\narrower\narrower
The static friction force \(F_{\mathrm{sf}}\) will exactly oppose forces applied to an object parallel to a surface contact up to the limit specified by the [[coefficient of static friction]] \(\mu_{\mathrm{sf}}\) multiplied by the normal force \(F_N\). In other words the magnitude of the static friction force satisfies the inequality:

\[ \le F_{\mathrm{sf}} \le \mu_{\mathrm{sf}} F_\mathrm{N}. \]

The kinetic friction force \(F_{\mathrm{kf}}\) is independent of both the forces applied and the movement of the object. Thus, the magnitude of the force equals:

\[F_{\mathrm{kf}} = \mu_{\mathrm{kf}} F_\mathrm{N}\]

where \(\mu_{\mathrm{kf}}\) is the coefficient of kinetic friction.
}




\newthought{Do not start a sentence with an equation}

\newthought{Display math}

In general mathematics typeset better when they are displayed. Use in-line maths only for the simplest of equations and for explanations of symbols and the like. Watch out for inconsistent spacing before and after displayed math.

\subsection{Correct badly sized math}

Briggs worked out the table from scratch. Starting with $\log 10 = 1$, he calculated
successively $\sqrt{10}$, $\sqrt{\sqrt{10}}$, $\sqrt{\sqrt{\sqrt{10}}}$, $\cramped{\sqrt{\sqrt{\sqrt{\sqrt{10}}}}}$ \ldots , until after 54 such root extractions he reached a number very close to 1.


$\sqrtsign{a12}$

\meaning\sqrtsign


\begin{quote}
The 2005 Euro\TeX{} ... 16th ($\cramped{2^{2^2}}$)($2^{2^2}$) ...
\end{quote}

Some \tex constructions typeset rather badly, consider for example this:

\[
\sqrt{\frac{\beta}{\gamma}} = \sqrt{X} + \sqrt{y}
\]

\noindent or this,

\[
\surd{\frac{\beta}{\gamma}} = \surd{X} + \surd{y}
\]


You can remedy this by using a \cs{mathstrut}.


\begin{texexample} {Correcting square roots} {ex:sqroot}
\[
\sqrt{\mathstrut a}=\sqrt{\mathstrut X}+\sqrt{\mathstrut y}
\sqrt{\mathstrut a}=\sqrt{\mathstrut X}+\sqrt{\mathstrut y_{i,j}}
\]
\end{texexample}



\newthought{Multiplication}

One of the most common errors is to use the ``dot'' to indicate multiplication between scalars\footnote{\url{http://www.tug.org/TUGboat/Articles/tb29-2/tb92guiggiani.pdf}}. For example the folowing formul\ae
\[a\cdot x^2+b\cdot x+c=0\]
should be written as
\[ax^2+bx+c=0\]

In fact, for the the sake of simplicity, the standard multiplication between letters, or between letters, or between a number and a letter, does not require any symbol. If, on the other hand, the multiplication is between two numbers, the $\times$ or $\cdot$ symbols are required to avoid ambiguity.
For example you should write

\[2\times 3=6 \text{ and not } 2\thickspace 3=6 \]


\paragraph{Using the right font}

Matching the text font with the mathematical font is the job of the class and
style designer. Ideally these should be selected by a specialist at the publisher.

This is now complicated a bit in that there are a limited number of unicode fonts, as well as unicode defined symbols. Nevertheless at least the \docFont{stix} family of fonts is a good start.

The Euler equation involves the five most important mathematical constants. First we typeset it with no space corrections\footnote{\texttt{\textbackslash eu\^\,\{\textbackslash iu\textbackslash pi\}}},
% The number `e'
\providecommand*{\eu}%
{\ensuremath{\mathrm{e}}}
% The imaginary unit
\providecommand*{\iu}%
{\ensuremath{\mathrm{j}}}
\[\scalebox{3}{$\eu^{\iu\pi}$}\]
a small correction to the space should be added

\[\scalebox{3}{$\eu^{\,\iu\pi}$}\]

\subsection{Differential operators}
A peculiar defnition is required to properly
write the differential symbol. It is in fact an operator that has a space only on its left. In Beccari (2007b) the following solution is proposed:

\bigskip


\clearpage
\section{tikz}
\begin{texexample}{Using TikZ with Maths}{ex:tikzmaths}
because of the periodicity of the Jacobi theta functions involved in the construction of the vectors.
The height difference between starting point and endpoint of the path is thus $Lp$ as shown in figure \ref{fig:path}. Moreover, because of the periodicity of the theta functions it is sufficient to restrict the initial height to $\ell_1=0,1,\dots,L-1$ in this case.

{  \centering
  \begin{tikzpicture}[>=stealth]
     \draw[scale=0.5,thick] (0,0)--(2,2)--(3,1)--(5,3)--(6,2)--(7,3);
           
     \draw[<->] (0,2) -- (0,-0.5) -- (4,-0.5);
     \foreach \x in {0.25,0.75,...,3.25}
       \draw[xshift=\x cm,yshift=-0.5cm] (0,-0.075)--(0,0.075);
     
    \foreach \y in {-0.5,0,...,1.5}
       \draw[yshift=\y cm] (-0.075,0)--(0.075,0);
 
    \draw (0.25,-0.5) node[below] {$1$};
    \draw (0.75,-0.5) node[below] {$2$};
    \draw (2,-0.5) node[below] {$\cdots$};
    \draw (3.25,-0.5) node[below] {$N$};
    \draw (4,-0.5) node[below] {$j$};
    \draw (0,0) node [left] {$\ell$};
    \draw (0,1.5) node [left] {$\ell+Lp$};
    \draw (0,2) node [right] {$\ell_j$};
    \clip[scale=0.5] (0,-1.5) rectangle (7.5,3.5);
    \draw[scale=0.5,dotted] (-1,-2) grid (8,5);
  \end{tikzpicture}
}
  
\end{texexample}




%\begin{texexample}{Example}{}
%\newcommand{\ud}{\ensuremath \mathop{}\!\mathrm{d}}
%\(z=2\sin x\mathrm{d}x\) and \(z=2\sin x\ud x\)
%
%
%\newcommand{\ud}{\mathop{}\!\mathrm{d}}
%\bigskip
%
%It uses an empty operator and eliminates the space
%on its left with |\!|.
%
%Note the difference between
%
%\[z=2\sin x\mathrm{d}x  \]
%
%\[z=2\sin x\ud x\]
%
%where the diffrential is obtained respectively with
%|\mathrm{d}| and |\ud|.
%\end{texexample}




\subsubsection{God is in the details}

Sometimes you will be faced with small decisions for which the Journal style manual might not have an answer for you or the journal Editor might have a different opinion to yours. One such question is if one needs to insert the thousand separator in coefficients.

\[
\operatorname{erf}^{-1}(z)=\tfrac{1}{2}\sqrt{\pi}\left (z+\frac{\pi}{12}z^3+\frac{7\pi^2}{480}z^5+\frac{127\pi^3}{40320}z^7+\frac{4369\pi^4}{5806080}z^9+\frac{34807\pi^5}{182476800}z^{11}+\cdots\right )
\]


\[
\operatorname{erf}^{-1}(z)=\tfrac{1}{2}\sqrt{\pi}\left (z+\frac{\pi}{12}z^3+\frac{7\pi^2}{480}z^5+\frac{127\pi^3}{40,320}z^7+\frac{4,369\pi^4}{5,806,080}z^9+\frac{34,807\pi^5}{182{,}476,800}z^{11}+\cdots\right )
\]

\[
\operatorname{erf}^{-1}(z)=\tfrac{1}{2}\sqrt{\pi}\left (z+\frac{\pi}{12}z^3+\frac{7\pi^2}{480}z^5+\frac{127\pi^3}{40{,}320}z^7+\frac{4{,}369\pi^4}{5{,}806{,}080}z^9+\frac{34{,}807\pi^5}{182{,}476{,}800}z^{11}+\cdots\right )
\]



\[
\operatorname{erf}^{-1}(z)=\tfrac{1}{2}\sqrt{\pi}\left (z+\frac{\pi}{12}z^3+\frac{7\pi^2}{480}z^5+\frac{127\pi^3}{40\thinspace 320}z^7+\frac{4\thinspace 369\pi^4}{5\thinspace 806\thinspace 080}z^9+\frac{34\thinspace 807\pi^5}{182\thinspace 476\thinspace 800}z^{11}+\cdots\right )
\]


If you type the commas watch out to put them in |{,}| otherwise \tex's algorithm will leave a space. See the difference below:

\[182{,}476{,}800 \]
\[182,476,800\]

It is interesting to note that Knuth believes that in equations this is unnecessary.
He is quoted in Typesetting Mathematics.


\begin{quotation}
But where Don wrote 1000000 they substituted
1,000,000. Don objected that although this might be justifed in text, his use is perfectly OK in a formula. Well then, they replied, write \(10^6\).
Fine, said, Don, but what do I do 
when the number is 1234567? The IEEE standard here is to insert spaces, thus: 1 234 567.
Don doesn't like this in formulae, but agrees that it may be useful in a high precision context, such as numerical tables. 
\end{quotation}



The following are extracts from his paper \textit{Johann Faulhaber and Sums of Powers}:\footnote{\url{http://www-cs-faculty.stanford.edu/~uno/papers/jfsp.tex.gz}}

{
\[\vcenter{\halign{$#$\hfil\ &$#$\hfil\cr
\Sigma n^{11}&=39916800{n+6\choose 12}+
19958400{n+5\choose 10}+3160080{n+4\choose 8}
+168960{n+3\choose 6}\cr
\noalign{\smallskip}
&\qquad\null+2046{n+2\choose 4}+{n+1\choose 2}\,;\cr
\noalign{\smallskip}
\Sigma n^{13}&=6227020800{n+7\choose 14}+3632428800{n+6\choose 12}+
726485760{n+5\choose 10}\cr
\noalign{\smallskip}
&\qquad\null+57657600{n+4\choose 8}
+1561560{n+3\choose 6}+8190{n+2\choose 4}+\binom{n+1}{2}\,.\cr}}\]
}

Also, note in the last equation the use of a period at the end. 
This is something that evokes strong opinions and flaming wars in fora. 
I am not too sure if I agree on the last one, but the way that Knuth writes 
is very clear and his equations in a way are paragraphs. 
In this case the use of a period is recommended.


\subsection{Punctuation}
\index{maths>typography>punctuation}

There are two schools of thought when it comes to punctuation, that is punctuation in display style formulae. Some authors (Beccari, 2007) argue it is not necessary, others that it is necessary and essential \footcite{guiggiani2008}. Mermin \footcite{mermin89} strongly argued in his third rule that:  `The
Math Is Prose rule simply says: End
a displayed equation with a punctuation
mark. It is implicit in this
statement that the absence of a punctuation
mark is itself a degenerate
form of punctuation that, like periods,
commas or semicolons, can be used
provided it makes sense.''

The authors of this article believe that equations, both in display and text style, are part of the argumentation
and so punctuation should be used to help the reader. An example of good use of punctuation is:


Since
\[ a=b \]
and
\[ b=c,\]
it is proven that
\[ a =c. \]

It is extremely unusual to find an equation end with a question mark but here is one. What is 
\[ a = d^2\mathrm{?} \]

If you do put a question mark or an exclamation mark and you are using unicode fonts,
you will need to use |\mathrm{?}| not to confuse it with the relevant symbols.

Most journals require that equations be punctuated, like normal text. Even if the author of the manuscript disagrees, probably the journal editor will add the punctuation.

Read your mathematical text aloud and introduce punctuation as if it was spelled in words rather than mathematical symbols. 

\subsection{Numbering Equations}
\index{maths>typography>equation numbering}

One question that you may face is the numbering of display equations. Early books used numbering sparingly, whereas many authors go overboard and number all the equations.

According to Knuth et al:\footnote{\url{http://tex.loria.fr/typographie/mathwriting.pdf}}
Numbering all displayed formulas is usually a bad idea; number the important ones only.

Halmos\footnote{\url{http://www.math.uh.edu/~tomforde/Books/Halmos-How-To-Write.pdf}} offers pretty much the same good advice,

\begin{latexquotation}
What about ``inequality (*)", or ``equation (7)", or ``formula (iii)"; should all displays be labelled or numbered? My answer is no. Reason: just as you shouldn't mention irrelevant assumptions or name irrelevant concepts, you also shouldn't attach irrelevant labels. Some small part of the reader's attention is attracted to the label, and some small part of his mind will wonder why the label is there. If there is a reason, then the wonder serves a healthy purpose by way of preparation, with no fuss, for a future reference to the same idea; if there is no reason, then the attention and the wonder were wasted.
\end{latexquotation}

Mermin's argues in his Good samaritan Rule: that it is distressing to having to hunt for an equation back in a manuscript for Eq. (2.46) not because your subsequent progress requires
you to inspect it in detail, but merely to find out what it is about so
you may know the principles that go into the construction of Eq. (7.38).
The Good Samaritan rule says: When referring to an equation identify it by
a phrase as well as a number. No compassionate and helpful person
would herald the arrival of Eq. (7.38) by saying "inserting (2.47) and (3.51)
into (5.13)..." when it is possible to say "inserting the form (2.47) of the
electric field $E$ and the Lindhard form (3.51) of the dielectric function $e$ into
the constitutive equation (5.13). To be sure, it's longer this way but much 
clearer\ldots'

For those that want only the equations that are referenced numbered the package \pkgname{autoref} can automate this. This is not included in this package due to conflicts with \pkgname{mathtools} and \pkgname{amsmath}.
\footnote{See also discussion at \protect\url{http://tex.stackexchange.com/questions/29267/which-equations-should-be numbered/49080\#49080}}

Now if you wish to argue about this is fine.

\section{Mathmode}

\tex is in \textit{mathmode} when it is reading mathematics. The |ifmmode| can be used to find out if \tex is in math mode. It denotes the start of an if-then-else control structure that tests whether \tex is currently in either math mode or display math mode. The |\else| part is optional. <TeX code 1> is processed if TeX is in one of the math modes, otherwise it is ignored. 
If the |\else| section is included and TeX is not in one of the math modes then \meta{TeX code 2} is processed; otherwise it is ignored.


\begin{texexample}{Calligraphic fonts}{ex:cal}

\newcommand{\Acal}{\ifmmode \mathcal{Acal} \else \(\mathcal{Acal}\) \fi}
The commandd efines a macro |\Acal| that can be used both in and out of math mode to typeset a calligraphy script A. 

This is a calligraphic {\Acal} or ({\Acal}).
\end{texexample}


\section{Useful packages}

Besides the main packages that we have discussed so far and which should be in everyone's toolbox, there are a number of other packages that you may find useful. One such package is the \pkgname{multienum}, which although not really a packaged specializing in mathematical typesetting, it provides an environment to set multiple equations, as in an exercise or exam.



\subsection{the multienum package}

The \docpkg{multienum} enables  the typestting of multiple equations on one line and numbering them, either with roman, arabic or alpha letters.

\emphasis{usepackage, multienum,begin,end,multienumerate}
\begin{teXXX}
\documentclass{article}
\usepackage{multienum}
\renewcommand{\regularlisti}{\setcounter{multienumi}{0}%
  \renewcommand{\labelenumi}%
  {\addtocounter{multienumi}{1}\alph{multienumi})}}
\begin{document}
\begin{multienumerate}[oddlist]
\mitemxxx{\(x^2 + y^2 = 1\)}{\(a + b = c\)}{\(r-x = y+z\)}
\mitemxxx{\(f - y = z\)}{\(a - b = 2d\)}{\(r+x = 2y-3z\)}
\end{multienumerate}
\end{document}
\end{teXXX}


\begin{multienumerate}[oddlist]
\mitemxxx{\(x^2 + y^2 = 1\)}{\(a + b = c\)}{\(r-x = y+z\)}
\end{multienumerate}
\begin{multienumerate}[evenlist]
\mitemxxx{\(f - y = z\)}{\(a - b = 2d\)}{\(r+x = 2y-3z\)}
\end{multienumerate}


\hrule

\bigskip

We can also enumerate the items using an even-only or odd only
counter.
\subsection*{Answers to Even-Numbered Exercises}
\begin{multienumerate}[evenlist]
\mitemxxxx{Not}{Linear}{Not}{Quadratic}
\mitemxxxo{Not}{Linear}{No; if $x=3$, then $y=-2$.}
\mitemxx{$(x_1,x_2)=(2+\frac{1}{3}t,t)$ or
$(s,3s-6)$}{$(x_1,x_2,x_3)=(2+\frac{5}{2}s-3t,s,t)$}
\mitemx{$(x_1,x_2,x_3,x_4)= (\frac{1}{4}+\frac{5}{4}s+\frac{3}{4}t-u,s,t,u)$
or $(s,t,u,\frac{1}{4}-s+\frac{5}{4}t+\frac{3}{4}u)$}
\mitemxxxx{$(2,-1,3)$}{None}{$(2,1,0,1)$}{$(0,0,0,0)$}
\end{multienumerate}
\bigskip



\newcommand\thecasestudylabel{Case Study}
\newenvironment{casestudy}[2][]{%
   \clearpage\par\leavevmode
   \addcontentsline{toc}{section}{\thecasestudylabel:  #1}
    \topline\vskip1.5pt
   {\noindent\large CASE STUDY\par}\vspace{3.5pt}
   \noindent\textsc{\large#1}\par
   \bigskip
   {#2}
   \medskip
   \topline
}{%
\vfill\bottomline}

\begin{casestudy}[The Riemann hypothesis.]{%
Typeset the text and the equations, shown below. Use a standard minimal to achieve it. Note the fraktur fonts. Text must all be as one paragraph.}

It is well known that the Riemann zeta function $\zeta(s)$ of a complex variable $s=\sigma+it$ is defined by
\[
\zeta(s)=\sum_{n=1}^{\infty}\frac{1}{n^{s}}
\]
for the real part $\mathfrak{R}(s)>1$ and its analytic continuation in the half plane $\sigma>0$ is
\begin{equation}\label{func:zeta}
\zeta(s)=\sum_{n=1}^{N}\frac{1}{n^{s}}-\frac{N^{1-s}}{1-s}-\frac{1}{2}N^{-s}
+s\int_{N}^{\infty}\frac{\frac{1}{2}-\{x\}}{x^{s+1}}dx
\end{equation}
for any integer $N\geq1$ and $\mathfrak{R}(s)>0$.
It extends to an analytic function in the whole complex plane except for having a simple pole at $s=1$. Trivially, $\zeta(-2n)=0$ for all positive integers. All other zeros of the Riemann zeta functions are called its nontrivial zeros.
\bottomline

\begin{teX}
It is well known that the Riemann zeta function $\zeta(s)$ of a complex variable $s=\sigma+it$ is defined by
\[
\zeta(s)=\sum_{n=1}^{\infty}\frac{1}{n^{s}}
\]
for the real part $\mathfrak{R}(s)>1$ and its analytic continuation in the half plane $\sigma>0$ is
\begin{equation}\label{func:zeta}
\zeta(s)=\sum_{n=1}^{N}\frac{1}{n^{s}}-\frac{N^{1-s}}{1-s}-\frac{1}{2}N^{-s}
+s\int_{N}^{\infty}\frac{\frac{1}{2}-\{x\}}{x^{s+1}}dx
\end{equation}
for any integer $N\geq1$ and $\mathfrak{R}(s)>0$.
It extends to an analytic function in the whole complex plane except for having a simple pole at $s=1$. Trivially, $\zeta(-2n)=0$ for all positive integers. All other zeros of the Riemann zeta functions are called its nontrivial zeros.
\end{teX}

Please note that the maths and the text, are typed as a single block. Do not leave any spaces in between. We have used |\mathfrak| for the fraktur font. We have also used $it$ for the imaginary part. This would depend on the style used in your field. 
\end{casestudy}

\clearpage
\section{Gather}

\begin{docEnvironment}{gather}{}
This is like a multi line environment with no special horizontal alignment. All rows
are centered and can have an own equation number:
\end{docEnvironment}

\begin{texexample}{Gather}{ex:gather}
\begingroup
\def\O{\mathcal{O}}
\begin{gather}
 \O,\O(E_4),\O(E_2),\O(H-E_3-E_5),\O(H-E_3),\O(H-E_5),\\ 
\O(2H-E_1-E_3-E_5-E_6),\O(2H-E_1-E_3-E_5),\O(2H-E_3-E_5-E_6).
\end{gather}

So lautet der Beweis des Satzes $2 \times 2 = 4$:
\begin{gather}
(\Omega^{\nu})^{\mu}{}'x = \Omega^{\nu \times \mu}{}'x \text{ Def.}\\
%\begin{split}
\Omega^{2 \times 2}{}'x = (\Omega^{2})^{2}{}'x = (\Omega^{2})^{1 + 1}{}'x = \Omega^{2}{}'\Omega^{2}{}'x = \Omega^{1 + 1}{}'\Omega^{1 + 1}{}'x\nonumber \\
= (\Omega'\Omega)'(\Omega'\Omega)'x = \Omega'\Omega'\Omega'\Omega'x = \Omega^{1 + 1 + 1 + 1}{}'x = \Omega^{4}{}'x.
%\end{split}
\end{gather}


\begin{gather*}
  x = \Omega^{0}{}' x \text{ Def.\ and}\\
  \Omega'\Omega^{\nu}{}'x = \Omega^{\nu+1}{}'x \text{ Def.}
\end{gather*}
\begin{equation}
  x = \Omega^{0}{}' x \text{ Def.\ and}\\
\Omega'\Omega^{\nu}{}'x = \Omega^{\nu+1}{}'x \text{ Def.}
\end{equation}
\endgroup
\end{texexample}


\section{How to number equations either to the left or right?}
\label{eqnochange}

\latexe uses Plain TeX \docAuxCommand{eqno} and \docAuxCommand{leqno} to place the equation number either to the left or the right of the equation. The placement is done automatically and the recommended way is to set this through the documentclass declaration at the start of the document. This can also be set manually later on through the above control sequences, as shown in the Example~\ref{ex:eqno}.

\begin{texexample}{Left or right numbering}{ex:eqno}
\makeatletter
\[ a = b + x^2 \eqno \@eqnnum \]
or at left
\[ a = b + x^2 \leqno \@eqnnum \]
\makeatother
\end{texexample}


%\meaning\text

%\chapter{Unicode Math}
\tcbdocmarginnote{N 29-06-2018}
Unicode contains separate codepoints for most if not all variations of alphabet
shape one may wish to use in mathematical notation. The complete list is shown
in table 5. Some of these have been covered in the previous sections.
The math font switching commands do not nest; therefore if you want sans
serif bold, you must write |\mathbfsf{...}| rather than |\mathbf{\mathsf{...}}|.
This may change in the future.

\section{Unicode maths font setup}

The promise of Unicode is that all symbols and alphabetic variants are in one font. The \pkgname{unicode-math}
maps all the available unicode math characters of a math font to respective \latex commands. If you have patience you can actually input them directly from the keyboard rather than in commands.

The best advice that I can give you is to read the \pkgname{unicode-math} carefully. 

\begin{docCommand} {setmathfont} { \oarg{range=\meta{unicode range}, \meta{font features } } \marg{font name} }
In many cases using one font might not be adequate. Specific Unicode ranges can be assigned to separate fonts.
\end{docCommand}

\subsection{Control over maths alphabets}

\subsection{Math `versions'}

\subsection{Maths input}

\subsection{Math `style'}

\subsubsection{Bold style}

Similar as in the previous section, ISO standards differ somewhat to \tex’s conventions
(and classical typesetting) for ‘boldness’ in mathematics. In the past, it has
been customary to use bold upright letters to denote things like vectors and matrices.


$$\boldsymbol{\omega} \times \mathbf{T} = \mathbf{T'}$$

$$\mathbf{\omega} = {1\over 2}\kappa \mathbf{B} + {1\over 2}(\kappa \mathbf{B} + \tau \mathbf{T}) + {1\over 2}\tau \mathbf{T} = \kappa \mathbf{B} + \tau \mathbf{T}
$$

\[
\mathbf{e} = \frac{\mathbf{A}}{m k} = \frac{1}{m k}(\mathbf{p} \times \mathbf{L})
\]

\subsubsection{Sans serif style}

\subsubsection{Blackboard or double-struck}



\subsubsection{Caligraphic and Script variants}



\section{Growing and non-growing accents}

This are the most problematic with Unicode fonts.

%%%%%%%% INPUT INTEGRAL FILES %%%%%%%%%%
%%%%%%%%%%%%%%%%%%%%%%%%%%%%%%%
 %% Too many errors here need to revisit.
 \subsection{Delimiters}
 \begin{multicols}{2}
% \showmbrace/{002F}{}
% \showlbrace({0028}{}
% \showlbrace[{005B}{}
% \showlbrace\lbrace{007B}{}
% \showmbrace\backslash{005C}{}
% \showrbrace){0029}{}
% \showrbrace]{005D}{}
% \showrbrace\rbrace{007D}{}
% \showlbrace\lceil{2308}{}
% \showlbrace\lfloor{230A}{}
% \showlbrace\lmoustache{23B0}{*}
% \showlbrace\lbrbrak{2772}{*}
% \showlbrace\lBrack{27E6}{*}
% \showlbrace\langle{27E8}{}, \cmd<
% \showlbrace\lAngle{27EA}{*}
% \showlbrace\lgroup{27EE}{*}
% \showlbrace\lBrace{2983}{*}
% \showlbrace\lParen{2985}{*}
% \showrbrace\rceil{2309}{}
 %\showrbrace\rfloor{230B}{}
% \showrbrace\rmoustache{23B1}{*}
% \showrbrace\rbrbrak{2773}{*}
% \showrbrace\rBrack{27E7}{*}
% \showrbrace\rangle{27E9}{}, \cmd>
% \showrbrace\rAngle{27EB}{*}
% \showrbrace\rgroup{27EF}{*}
 %\showrbrace\rBrace{2984}{*}
% \showrbrace\rParen{2986}{*}
 \end{multicols}

 \begin{multicols}{2}
% \showmbrace\vert{007C}{}, \cmd|
 \showmbrace\Vert{2016}{*}, \cmd\|
 \showmbrace\Vvert{2980}{}
 \showmbrace\uparrow{2191}{}
 \showmbrace\downarrow{2193}{}
 \showmbrace\updownarrow{2195}{}
 \showmbrace\Uparrow{21D1}{}
 \showmbrace\Downarrow{21D3}{}
 \showmbrace\Updownarrow{21D5}{}
% \showmbrace\Uuparrow{290A}{*}
% \showmbrace\Ddownarrow{290B}{*}
% \showmbrace\UUparrow{27F0}{*}
% \showmbrace\DDownarrow{27F1}{*}
% \showmbrace\arrowvert{XXXX}{}
% \showmbrace\Arrowvert{XXXX}{}
% \showmbrace\bracevert{XXXX}{*}
 \end{multicols}

 \subsection{Other bracess}
 \begin{multicols}{2}
 \showsymbol\ulcorner{231C}{*}
 \showsymbol\urcorner{231D}{*}
 \showsymbol\llcorner{231E}{*}
 \showsymbol\lrcorner{231F}{*}
 \showsymbol\Lbrbrak{27EC}{*}
 \showsymbol\Rbrbrak{27ED}{*}
 \showsymbol\llparenthesis{2987}{*}
 \showsymbol\rrparenthesis{2988}{*}
 \showsymbol\llangle{2989}{*}
 \showsymbol\rrangle{298A}{*}
 \showsymbol\lbrackubar{298B}{*}
 \showsymbol\rbrackubar{298C}{*}
 \showsymbol\lbrackultick{298D}{*}
 \showsymbol\rbracklrtick{298E}{*}
 \showsymbol\lbracklltick{298F}{*}
 \showsymbol\rbrackurtick{2990}{*}
 \showsymbol\langledot{2991}{*}
 \showsymbol\rangledot{2992}{*}
 \showsymbol\lparenless{2993}{*}
 \showsymbol\rparengtr{2994}{*}
 \showsymbol\Lparengtr{2995}{*}
 \showsymbol\Rparenless{2996}{*}
 \showsymbol\lblkbrbrak{2997}{*}
 \showsymbol\rblkbrbrak{2998}{*}
 \showsymbol\lvzigzag{29D8}{*}
 \showsymbol\rvzigzag{29D9}{*}
 \showsymbol\Lvzigzag{29DA}{*}
 \showsymbol\Rvzigzag{29DB}{*}
 \showsymbol\lcurvyangle{29FC}{*}
 \showsymbol\rcurvyangle{29FD}{*}
 \showsymbol\lbrbrak{2772}{*}
 \showsymbol\rbrbrak{2773}{*}
 \showsymbol\lbag{27C5}{*}
 \showsymbol\rbag{27C6}{*}
 \showsymbol\Lbrbrak{27EC}{*}
 \showsymbol\Rbrbrak{27ED}{*}
 \end{multicols}

 \section{Alphabetics}
 \begin{multicols}{2}
\showsymbolalpha\Gamma{0393}{}
\showsymbolalpha\Delta{0394}{}
\showsymbolalpha\Theta{0398}{}
\showsymbolalpha\Lambda{039B}{}
\showsymbolalpha\Xi{039E}{}
\showsymbolalpha\Pi{03A0}{}
\showsymbolalpha\Sigma{03A3}{}
\showsymbolalpha\Upsilon{03A5}{}
\showsymbolalpha\Phi{03A6}{}
\showsymbolalpha\Psi{03A8}{}
\showsymbolalpha\Omega{03A9}{}
\showsymbolalpha\alpha{03B1}{}
\showsymbolalpha\beta{03B2}{}
\showsymbolalpha\gamma{03B3}{}
\showsymbolalpha\delta{03B4}{}
\showsymbolalpha\epsilon{03B5}{}
\showsymbolalpha\zeta{03B6}{}
\showsymbolalpha\eta{03B7}{}
\showsymbolalpha\theta{03B8}{}
\showsymbolalpha\iota{03B9}{}
\showsymbolalpha\kappa{03BA}{}
\showsymbolalpha\lambda{03BB}{}
\showsymbolalpha\mu{03BC}{}
\showsymbolalpha\nu{03BD}{}
\showsymbolalpha\xi{03BE}{}
\showsymbolalpha\pi{03C0}{}
\showsymbolalpha\rho{03C1}{}
\showsymbolalpha\sigma{03C3}{}
\showsymbolalpha\tau{03C4}{}
\showsymbolalpha\upsilon{03C5}{}
\showsymbolalpha\phi{03D5}{}
\showsymbolalpha\chi{03C7}{}
\showsymbolalpha\psi{03C8}{}
\showsymbolalpha\omega{03C9}{}
\showsymbolalpha\varepsilon{03F5}{}
\showsymbolalpha\vartheta{03D1}{}
\showsymbolalpha\varpi{03D6}{}
\showsymbolalpha\varrho{03F1}{}
\showsymbolalpha\varsigma{03C2}{}
\showsymbolalpha\varphi{03C6}{}
\showsymbolalpha\nabla{2207}{}
\showsymbolalpha\partial{2202}{}
\showsymbolalpha\imath{1D6A4}{}
\showsymbolalpha\jmath{1D6A5}{}
 \end{multicols}
% \subsection{Accents}
 \begin{multicols}{2}
 \showaccent\grave{0300}{}
 \showaccent\acute{0301}{}
 \showaccent\hat{0302}{}
 \showaccent\tilde{0303}{}
 \showaccent\bar{0304}{}
 \showaccent\breve{0306}{}
 \showaccent\dot{0307}{}
 \showaccent\ddot{0308}{}
 \showaccent\ovhook{0309}{}
 \showaccent\mathring{030A}{}
 \showaccent\check{030C}{}
 \showaccent\candra{0310}{}
 \showaccent\oturnedcomma{0312}{}
 \showaccent\ocommatopright{0315}{}
 \showaccent\droang{031A}{}
 \showaccent\leftharpoonaccent{20D0}{}
 \showaccent\rightharpoonaccent{20D1}{}
 %\showaccent\leftarrowaccent{20D6}{}
 \showaccent\vec{20D7}{}, \cmd\rightarrowaccent
 %\showaccent\leftrightarrowaccent{20E1}{}
 \showaccent\dddot{20DB}{}
 \showaccent\ddddot{20DC}{}
 \showaccent\annuity{20E7}{}
 \showaccent\widebridgeabove{20E9}{}
 \showaccent\asteraccent{20F0}{}
 \end{multicols}

 \begin{multicols}{2}
 \showwideaccent\widehat{0302}{*}
 \showwideaccent\widetilde{0303}{*}
% \showwideaccent\widecheck{030C}{*}
 \showwideaccent\overleftarrow{20D6}{}
 \showwideaccent\overrightarrow{20D7}{}
 \showwideaccent\underrightarrow{20EF}{}
 \showwideaccent\underleftarrow{20EE}{}
 \showwideaccent\overleftrightarrow{20E1}{}
 \showwideaccent\underleftrightarrow{034D}{}
% \showwideaccent\overleftharpoon{20D0}{}
% \showwideaccent\overrightharpoon{20D1}{}
% \showwideaccent\underleftharpoon{20EC}{}
% \showwideaccent\underrightharpoon{20ED}{}
 \end{multicols}

 OpenType STIX fonts include a number of under accents that can be used in
 math mode, but \TeX\ does not support under accents natively so such glyphs
 can not be used directly. Under accents can be set using regular accents and
 commands like |\underaccent| from the \pkgname{accents} package, for example
 gives |\(\underaccent{\hat}{X}\)|. The
 \pkgname{undertilde} package provides |\utilde| for extensible under tilde
 accent \citep{undertilde}.
 
 
 % \subsection{Over and under brackets}
 \begin{multicols}{2}
 %\showover\overbracket{23B4}{} conflict mathtools consider remove mathtools for UNICODE
 \showover\overparen{23DC}{}
 \showover\overbrace{23DE}{}
 \showover\underbracket{23B5}{} %conflict mathtools?
 \showover\underparen{23DD}{}
 \showover\underbrace{23DF}{}
 \end{multicols}

 \subsection{Radicals}
 \begin{multicols}{2}
 \showaccent\sqrt{221A}{}
 \showaccent\longdivision{27CC}{*}
 \end{multicols}
 
 
 
 
 
 
 \subsection{Big operators}
 \begin{multicols}{2}
 \showop\Bbbsum{2140}{}
 \showop\prod{220F}{}
 \showop\coprod{2210}{}
 \showop\sum{2211}{}
 \showop\bigwedge{22C0}{}
 \showop\bigvee{22C1}{}
 \showop\bigcap{22C2}{}
 \showop\bigcup{22C3}{}
 \showop\leftouterjoin{27D5}{*}
 \showop\rightouterjoin{27D6}{*}
 \showop\fullouterjoin{27D7}{*}
 \showop\bigbot{27D8}{*}
 \showop\bigtop{27D9}{*}
 \showop\xsol{29F8}{*}
 \showop\xbsol{29F9}{*}
 \showop\bigodot{2A00}{*}
 \showop\bigoplus{2A01}{*}
 \showop\bigotimes{2A02}{*}
 \showop\bigcupdot{2A03}{*}
 \showop\biguplus{2A04}{*}
 \showop\bigsqcap{2A05}{*}
 \showop\bigsqcup{2A06}{*}
 \showop\conjquant{2A07}{*}
 \showop\disjquant{2A08}{*}
 \showop\bigtimes{2A09}{*}
 \showop\modtwosum{2A0A}{*}
 \showop\Join{2A1D}{*}
 \showop\bigtriangleleft{2A1E}{*}
 \showop\zcmp{2A1F}{*}
 \showop\zpipe{2A20}{*}
 \showop\zproject{2A21}{*}
 \showop\biginterleave{2AFC}{}
 \showop\bigtalloblong{2AFF}{*}
 \end{multicols}
 
 \section{General manipulations test.}
 
 Here are some examples using big operators

$$\sum_{n=s}^t C\cdot f(n) = C\cdot \sum_{n=s}^t f(n),$$ where ``'C'' is a constant

\[\sum_{n=s}^t f(n) + \sum_{n=s}^{t} g(n) = \sum_{n=s}^t \left[f(n) + g(n)\right]\] 

\[\sum_{n=s}^t f(n) - \sum_{n=s}^{t} g(n) = \sum_{n=s}^t \left[f(n) - g(n)\right]\] 

\[\sum_{n=s}^t f(n) = \sum_{n=s+p}^{t+p} f(n-p) \] 

\[\sum_{n\in B} f(n) = \sum_{m\in A} f(\sigma(m)),\] for a bijection σ from a finite set ``$A$'' onto a finite set ``$B$''; this generalizes the preceding formula.

\[\sum_{n=s}^j f(n) + \sum_{n=j+1}^t f(n) = \sum_{n=s}^t f(n)\] 

\[\sum_{i=k_0}^{k_1}\sum_{j=l_0}^{l_1} a_{i,j} = \sum_{j=l_0}^{l_1}\sum_{i=k_0}^{k_1} a_{i,j}\] 

\[\sum_{k\le j \le i\le n} a_{i,j} = \sum_{i=k}^n\sum_{j=k}^i a_{i,j} = \sum_{j=k}^n\sum_{i=j}^n a_{i,j}\] 


\[\sum_{n=0}^t f(2n) + \sum_{n=0}^t f(2n+1) = \sum_{n=0}^{2t+1} f(n)\] 

\[\sum_{n=0}^t \sum_{i=0}^{z-1} f(z\cdot n+i) = \sum_{n=0}^{z\cdot t+z-1} f(n)\] 

\[\sum_{i=s}^m\sum_{j=t}^n {a_i}{c_j} = \sum_{i=s}^m a_i \cdot \sum_{j=t}^n c_j\] 

\[\sum_{n=s}^t \ln f(n) = \ln \prod_{n=s}^t f(n)\] 

\[c^{\left[\sum_{n=s}^t f(n) \right]} = \prod_{n=s}^t c^{f(n)}\] 
 
 
 \section{Convergence criteria}
 
The product of positive real numbers

\[\prod_{n=1}^{\infty} a_n\]
converges to a nonzero real number if and only if the sum
\[\sum_{n=1}^{\infty} \log(a_n)\]
converges. This allows the translation of convergence criteria for infinite sums into convergence criteria for infinite products. The same criterion applies to products of arbitrary complex numbers (including negative reals) if log is understood as a fixed Complex logarithm which satisfies $\log(1) = 0$, with the proviso that the infinite product diverges when infinitely many ''$a_n$'' fall outside 
the domain of log, whereas finitely many such ''$a_n$'' can be ignored in the sum.

For products of reals in which each $a_n\ge1$, written as, for instance, $a_n=1+p_n$,
where $p_n\ge 0$, the bounds

\[1+\sum_{n=1}^{N} p_n \le \prod_{n=1}^{N} \left( 1 + p_n \right) \le \exp \left( \sum_{n=1}^{N}p_n \right)\]

show that the infinite product converges precisely if the infinite sum of the $p_n$ converges. This relies on the Monotone convergence theorem. More generally, the convergence of $\prod_{n=1}^\infty(1+p_n)$ is equivalent to the convergence of $\sum_{n=1}^\infty p_n$ 
%if ''p<sub>n</sub>'' are real or complex numbers such that <math>\sum_{n=1}^\infty|p_n|^2<+\infty\], since <math>\log(1+x)=x+O(x^2)\] in a neighbourhood of 0.
%
%If the series ''p''<sub>''n''</sub> diverges, then the sequence of partial products converges to zero as a sequence.  The infinite product is said to '''diverge to zero'''.
% 
 
 
 
 
 
 
 
 
%% Relations symbols from STIX font
%% 
  \subsection{Relations}
 \begin{multicols}{2}
% \showrelsymbol*{002A}{}, \cmd\ast
% \showrelsymbol:{003A}{}
 \showrelsymbol{\less}{003C}{}, \cmd\less
 \showrelsymbol{\equal}{003D}{}, \cmd\equal
 \showrelsymbol{\greater}{003E}{}, \cmd\greater
 \showrelsymbol\closure{2050}{*}
 %\showrelsymbol\vertoverlay{20D2}{}
 \showrelsymbol\leftarrow{2190}{}, \cmd\gets
 \showrelsymbol\uparrow{2191}{}
 \showrelsymbol\rightarrow{2192}{}, \cmd\to
 \showrelsymbol\downarrow{2193}{}
 \showrelsymbol\leftrightarrow{2194}{}
 \showrelsymbol\updownarrow{2195}{}
 \showrelsymbol\nwarrow{2196}{}
 \showrelsymbol\nearrow{2197}{}
 \showrelsymbol\searrow{2198}{}
 \showrelsymbol\swarrow{2199}{}
 \showrelsymbol\nleftarrow{219A}{}
 \showrelsymbol\nrightarrow{219B}{}
 \showrelsymbol\leftwavearrow{219C}{}
 \showrelsymbol\rightwavearrow{219D}{}
 \showrelsymbol\twoheadleftarrow{219E}{}
 \showrelsymbol\twoheaduparrow{219F}{}
 \showrelsymbol\twoheadrightarrow{21A0}{}
 \showrelsymbol\twoheaddownarrow{21A1}{}
 \showrelsymbol\leftarrowtail{21A2}{}
 \showrelsymbol\rightarrowtail{21A3}{}
 \showrelsymbol\mapsfrom{21A4}{}
 \showrelsymbol\mapsup{21A5}{}
 \showrelsymbol\mapsto{21A6}{}
 \showrelsymbol\mapsdown{21A7}{}
 \showrelsymbol\hookleftarrow{21A9}{}
 \showrelsymbol\hookrightarrow{21AA}{}
 \showrelsymbol\looparrowleft{21AB}{}
 \showrelsymbol\looparrowright{21AC}{}
 \showrelsymbol\leftrightsquigarrow{21AD}{}
 \showrelsymbol\nleftrightarrow{21AE}{}
 \showrelsymbol\downzigzagarrow{21AF}{}
 \showrelsymbol\Lsh{21B0}{}
 \showrelsymbol\Rsh{21B1}{}
 \showrelsymbol\Ldsh{21B2}{}
 \showrelsymbol\Rdsh{21B3}{}
 \showrelsymbol\curvearrowleft{21B6}{}
 \showrelsymbol\curvearrowright{21B7}{}
 \showrelsymbol\circlearrowleft{21BA}{}
 \showrelsymbol\circlearrowright{21BB}{}
 \showrelsymbol\leftharpoonup{21BC}{}
 \showrelsymbol\leftharpoondown{21BD}{}
 \showrelsymbol\upharpoonright{21BE}{}, \cmd\restriction
 \showrelsymbol\upharpoonleft{21BF}{}
 \showrelsymbol\rightharpoonup{21C0}{}
 \showrelsymbol\rightharpoondown{21C1}{}
 \showrelsymbol\downharpoonright{21C2}{}
 \showrelsymbol\downharpoonleft{21C3}{}
 \showrelsymbol\rightleftarrows{21C4}{}
 \showrelsymbol\updownarrows{21C5}{}
 \showrelsymbol\leftrightarrows{21C6}{}
 \showrelsymbol\leftleftarrows{21C7}{}
 \showrelsymbol\upuparrows{21C8}{}
 \showrelsymbol\rightrightarrows{21C9}{}
 \showrelsymbol\downdownarrows{21CA}{}
 \showrelsymbol\leftrightharpoons{21CB}{}
 \showrelsymbol\rightleftharpoons{21CC}{}
 \showrelsymbol\nLeftarrow{21CD}{}
 \showrelsymbol\nLeftrightarrow{21CE}{}
 \showrelsymbol\nRightarrow{21CF}{}
 \showrelsymbol\Leftarrow{21D0}{}
 \showrelsymbol\Uparrow{21D1}{}
 \showrelsymbol\Rightarrow{21D2}{}
 \showrelsymbol\Downarrow{21D3}{}
 \showrelsymbol\Leftrightarrow{21D4}{}
 \showrelsymbol\Updownarrow{21D5}{}
 \showrelsymbol\Nwarrow{21D6}{}
 \showrelsymbol\Nearrow{21D7}{}
 \showrelsymbol\Searrow{21D8}{}
 \showrelsymbol\Swarrow{21D9}{}
 \showrelsymbol\Lleftarrow{21DA}{*}
 \showrelsymbol\Rrightarrow{21DB}{*}
 \showrelsymbol\leftsquigarrow{21DC}{}
 \showrelsymbol\rightsquigarrow{21DD}{}, \cmd\leadsto
 \showrelsymbol\barleftarrow{21E4}{*}
 \showrelsymbol\rightarrowbar{21E5}{*}
 \showrelsymbol\circleonrightarrow{21F4}{*}
 \showrelsymbol\downuparrows{21F5}{}
 \showrelsymbol\rightthreearrows{21F6}{*}
 \showrelsymbol\nvleftarrow{21F7}{*}
 \showrelsymbol\nvrightarrow{21F8}{*}
 \showrelsymbol\nvleftrightarrow{21F9}{*}
 \showrelsymbol\nVleftarrow{21FA}{*}
 \showrelsymbol\nVrightarrow{21FB}{*}
 \showrelsymbol\nVleftrightarrow{21FC}{*}
 \showrelsymbol\leftarrowtriangle{21FD}{*}
 \showrelsymbol\rightarrowtriangle{21FE}{*}
 \showrelsymbol\leftrightarrowtriangle{21FF}{*}
 \showrelsymbol\in{2208}{}
 \showrelsymbol\notin{2209}{}
 \showrelsymbol\smallin{220A}{}
 \showrelsymbol\ni{220B}{}, \cmd\owns
 \showrelsymbol\nni{220C}{}
 \showrelsymbol\smallni{220D}{}
 \showrelsymbol\propto{221D}{}
 \showrelsymbol\varpropto{221D}{}
 \showrelsymbol\mid{2223}{}
 \showrelsymbol\shortmid{2223}{}
 \showrelsymbol\nmid{2224}{}
 \showrelsymbol\nshortmid{2224}{*}
 \showrelsymbol\parallel{2225}{}
 \showrelsymbol\shortparallel{2225}{*}
 \showrelsymbol\nparallel{2226}{}
 \showrelsymbol\nshortparallel{2226}{*}
 \showrelsymbol\Colon{2237}{}
 \showrelsymbol\dashcolon{2239}{}
 \showrelsymbol\dotsminusdots{223A}{}
 \showrelsymbol\kernelcontraction{223B}{}
 \showrelsymbol\sim{223C}{}
 \showrelsymbol\thicksim{223C}{}
 \showrelsymbol\backsim{223D}{}
 \showrelsymbol\nsim{2241}{}
 \showrelsymbol\eqsim{2242}{}
 \showrelsymbol\simeq{2243}{}
 \showrelsymbol\nsime{2244}{}
 \showrelsymbol\cong{2245}{}
 \showrelsymbol\simneqq{2246}{}
 \showrelsymbol\ncong{2247}{}
 \showrelsymbol\approx{2248}{}
 \showrelsymbol\thickapprox{2248}{}
 \showrelsymbol\napprox{2249}{}
 \showrelsymbol\approxeq{224A}{}
 \showrelsymbol\approxident{224B}{}
 \showrelsymbol\backcong{224C}{}
 \showrelsymbol\asymp{224D}{}
 \showrelsymbol\Bumpeq{224E}{}
 \showrelsymbol\bumpeq{224F}{}
 \showrelsymbol\doteq{2250}{}
 \showrelsymbol\Doteq{2251}{}, \cmd\doteqdot
 \showrelsymbol\fallingdotseq{2252}{}
 \showrelsymbol\risingdotseq{2253}{}
 \showrelsymbol\coloneq{2254}{}
 \showrelsymbol\eqcolon{2255}{}
 \showrelsymbol\eqcirc{2256}{}
 \showrelsymbol\circeq{2257}{}
 \showrelsymbol\arceq{2258}{}
 \showrelsymbol\wedgeq{2259}{}
 \showrelsymbol\veeeq{225A}{}
 \showrelsymbol\stareq{225B}{}
 \showrelsymbol\triangleq{225C}{}
 \showrelsymbol\eqdef{225D}{}
 \showrelsymbol\measeq{225E}{}
 \showrelsymbol\questeq{225F}{}
 \showrelsymbol\ne{2260}{}, \cmd\neq
 \showrelsymbol\equiv{2261}{}
 \showrelsymbol\nequiv{2262}{}
 \showrelsymbol\Equiv{2263}{}
 \showrelsymbol\leq{2264}{}, \cmd\le
 \showrelsymbol\geq{2265}{}, \cmd\ge
 \showrelsymbol\leqq{2266}{}
 \showrelsymbol\geqq{2267}{}
 \showrelsymbol\lneqq{2268}{}
 \showrelsymbol\lvertneqq{2268}{}
 \showrelsymbol\gneqq{2269}{}
 \showrelsymbol\gvertneqq{2269}{}
 \showrelsymbol\ll{226A}{}
 \showrelsymbol\gg{226B}{}
 \showrelsymbol\between{226C}{}
 \showrelsymbol\nasymp{226D}{}
 \showrelsymbol\nless{226E}{}
 \showrelsymbol\ngtr{226F}{}
 \showrelsymbol\nleq{2270}{}
 \showrelsymbol\ngeq{2271}{}
 \showrelsymbol\lesssim{2272}{}
 \showrelsymbol\gtrsim{2273}{}
 \showrelsymbol\nlesssim{2274}{}
 \showrelsymbol\ngtrsim{2275}{}
 \showrelsymbol\lessgtr{2276}{}
 \showrelsymbol\gtrless{2277}{}
 \showrelsymbol\nlessgtr{2278}{}
 \showrelsymbol\ngtrless{2279}{}
 \showrelsymbol\prec{227A}{}
 \showrelsymbol\succ{227B}{}
 \showrelsymbol\preccurlyeq{227C}{}
 \showrelsymbol\succcurlyeq{227D}{}
 \showrelsymbol\precsim{227E}{}
 \showrelsymbol\succsim{227F}{}
 \showrelsymbol\nprec{2280}{}
 \showrelsymbol\nsucc{2281}{}
 \showrelsymbol\subset{2282}{}
 \showrelsymbol\supset{2283}{}
 \showrelsymbol\nsubset{2284}{}
 \showrelsymbol\nsupset{2285}{}
 \showrelsymbol\subseteq{2286}{}
 \showrelsymbol\supseteq{2287}{}
 \showrelsymbol\nsubseteq{2288}{}
 \showrelsymbol\nsupseteq{2289}{}
 \showrelsymbol\subsetneq{228A}{}
 \showrelsymbol\varsubsetneq{228A}{*}
 \showrelsymbol\supsetneq{228B}{}
 \showrelsymbol\varsupsetneq{228B}{*}
 \showrelsymbol\sqsubset{228F}{}
 \showrelsymbol\sqsupset{2290}{}
 \showrelsymbol\sqsubseteq{2291}{}
 \showrelsymbol\sqsupseteq{2292}{}
 \showrelsymbol\vdash{22A2}{}
 \showrelsymbol\dashv{22A3}{}
 \showrelsymbol\assert{22A6}{}
 \showrelsymbol\models{22A7}{}
 \showrelsymbol\vDash{22A8}{}
 \showrelsymbol\Vdash{22A9}{}
 \showrelsymbol\Vvdash{22AA}{}
 \showrelsymbol\VDash{22AB}{}
 \showrelsymbol\nvdash{22AC}{}
 \showrelsymbol\nvDash{22AD}{}
 \showrelsymbol\nVdash{22AE}{}
 \showrelsymbol\nVDash{22AF}{}
 \showrelsymbol\prurel{22B0}{}
 \showrelsymbol\scurel{22B1}{}
 \showrelsymbol\vartriangleleft{22B2}{}
 \showrelsymbol\vartriangleright{22B3}{}
 \showrelsymbol\trianglelefteq{22B4}{}
 \showrelsymbol\trianglerighteq{22B5}{}
 \showrelsymbol\origof{22B6}{}
 \showrelsymbol\imageof{22B7}{}
 \showrelsymbol\multimap{22B8}{}
 \showrelsymbol\bowtie{22C8}{}
 \showrelsymbol\backsimeq{22CD}{}
 \showrelsymbol\Subset{22D0}{}
 \showrelsymbol\Supset{22D1}{}
 \showrelsymbol\pitchfork{22D4}{}
 \showrelsymbol\equalparallel{22D5}{}
 \showrelsymbol\lessdot{22D6}{}
 \showrelsymbol\gtrdot{22D7}{}
 \showrelsymbol\lll{22D8}{}, \cmd\llless
 \showrelsymbol\ggg{22D9}{}, \cmd\gggtr
 \showrelsymbol\lesseqgtr{22DA}{}
 \showrelsymbol\gtreqless{22DB}{}
 \showrelsymbol\eqless{22DC}{}
 \showrelsymbol\eqgtr{22DD}{}
 \showrelsymbol\curlyeqprec{22DE}{}
 \showrelsymbol\curlyeqsucc{22DF}{}
 \showrelsymbol\npreccurlyeq{22E0}{}
 \showrelsymbol\nsucccurlyeq{22E1}{}
 \showrelsymbol\nsqsubseteq{22E2}{}
 \showrelsymbol\nsqsupseteq{22E3}{}
 \showrelsymbol\sqsubsetneq{22E4}{*}
 \showrelsymbol\sqsupsetneq{22E5}{*}
 \showrelsymbol\lnsim{22E6}{}
 \showrelsymbol\gnsim{22E7}{}
 \showrelsymbol\precnsim{22E8}{}
 \showrelsymbol\succnsim{22E9}{}
 \showrelsymbol\nvartriangleleft{22EA}{}
 \showrelsymbol\nvartriangleright{22EB}{}
 \showrelsymbol\ntrianglelefteq{22EC}{}
 \showrelsymbol\ntrianglerighteq{22ED}{}
 \showrelsymbol\vdots{22EE}{}
 \showrelsymbol\adots{22F0}{}
 \showrelsymbol\ddots{22F1}{}
 \showrelsymbol\disin{22F2}{*}
 \showrelsymbol\varisins{22F3}{*}
 \showrelsymbol\isins{22F4}{*}
 \showrelsymbol\isindot{22F5}{*}
 \showrelsymbol\varisinobar{22F6}{}
 \showrelsymbol\isinobar{22F7}{*}
 \showrelsymbol\isinvb{22F8}{*}
 \showrelsymbol\isinE{22F9}{*}
 \showrelsymbol\nisd{22FA}{*}
 \showrelsymbol\varnis{22FB}{*}
 \showrelsymbol\nis{22FC}{*}
 \showrelsymbol\varniobar{22FD}{}
 \showrelsymbol\niobar{22FE}{*}
 \showrelsymbol\bagmember{22FF}{*}
 \showrelsymbol\frown{2322}{}
 \showrelsymbol\smallfrown{2322}{*}
 \showrelsymbol\smile{2323}{}
 \showrelsymbol\smallsmile{2323}{*}
 \showrelsymbol\APLnotslash{233F}{}
 \showrelsymbol\vartriangle{25B5}{*}
 \showrelsymbol\perp{27C2}{*}
 \showrelsymbol\bsolhsub{27C8}{}
 \showrelsymbol\suphsol{27C9}{}
 \showrelsymbol\upin{27D2}{*}
 \showrelsymbol\pullback{27D3}{*}
 \showrelsymbol\pushout{27D4}{*}
 \showrelsymbol\DashVDash{27DA}{*}
 \showrelsymbol\dashVdash{27DB}{*}
 \showrelsymbol\multimapinv{27DC}{*}
 \showrelsymbol\vlongdash{27DD}{*}
 \showrelsymbol\longdashv{27DE}{*}
 \showrelsymbol\cirbot{27DF}{*}
 \showrelsymbol\UUparrow{27F0}{*}
 \showrelsymbol\DDownarrow{27F1}{*}
 \showrelsymbol\acwgapcirclearrow{27F2}{*}
 \showrelsymbol\cwgapcirclearrow{27F3}{*}
 \showrelsymbol\rightarrowonoplus{27F4}{*}
 \showrelsymbol\longleftarrow{27F5}{*}
 \showrelsymbol\longrightarrow{27F6}{*}
 \showrelsymbol\longleftrightarrow{27F7}{*}
 \showrelsymbol\Longleftarrow{27F8}{*}
 \showrelsymbol\Longrightarrow{27F9}{*}
 \showrelsymbol\Longleftrightarrow{27FA}{*}
 \showrelsymbol\longmapsfrom{27FB}{*}
 \showrelsymbol\longmapsto{27FC}{*}
 \showrelsymbol\Longmapsfrom{27FD}{*}
 \showrelsymbol\Longmapsto{27FE}{*}
 \showrelsymbol\longrightsquigarrow{27FF}{*}
 \showrelsymbol\nvtwoheadrightarrow{2900}{*}
 \showrelsymbol\nVtwoheadrightarrow{2901}{*}
 \showrelsymbol\nvLeftarrow{2902}{*}
 \showrelsymbol\nvRightarrow{2903}{*}
 \showrelsymbol\nvLeftrightarrow{2904}{*}
 \showrelsymbol\twoheadmapsto{2905}{*}
 \showrelsymbol\Mapsfrom{2906}{*}
 \showrelsymbol\Mapsto{2907}{*}
 \showrelsymbol\downarrowbarred{2908}{*}
 \showrelsymbol\uparrowbarred{2909}{*}
 \showrelsymbol\Uuparrow{290A}{*}
 \showrelsymbol\Ddownarrow{290B}{*}
 \showrelsymbol\leftbkarrow{290C}{*}
 \showrelsymbol\rightbkarrow{290D}{*}
 \showrelsymbol\leftdbkarrow{290E}{*}, \cmd\dashleftarrow
 \showrelsymbol\dbkarow{290F}{*}, \cmd\dashrightarrow
 \showrelsymbol\drbkarow{2910}{*}
 \showrelsymbol\rightdotarrow{2911}{*}
 \showrelsymbol\baruparrow{2912}{*}
 \showrelsymbol\downarrowbar{2913}{*}
 \showrelsymbol\nvrightarrowtail{2914}{*}
 \showrelsymbol\nVrightarrowtail{2915}{*}
 \showrelsymbol\twoheadrightarrowtail{2916}{*}
 \showrelsymbol\nvtwoheadrightarrowtail{2917}{*}
 \showrelsymbol\nVtwoheadrightarrowtail{2918}{*}
 \showrelsymbol\lefttail{2919}{*}
 \showrelsymbol\righttail{291A}{*}
 \showrelsymbol\leftdbltail{291B}{*}
 \showrelsymbol\rightdbltail{291C}{*}
 \showrelsymbol\diamondleftarrow{291D}{*}
 \showrelsymbol\rightarrowdiamond{291E}{*}
 \showrelsymbol\diamondleftarrowbar{291F}{*}
 \showrelsymbol\barrightarrowdiamond{2920}{*}
 \showrelsymbol\nwsearrow{2921}{*}
 \showrelsymbol\neswarrow{2922}{*}
 \showrelsymbol\hknwarrow{2923}{*}
 \showrelsymbol\hknearrow{2924}{*}
 \showrelsymbol\hksearow{2925}{*}
 \showrelsymbol\hkswarow{2926}{*}
 \showrelsymbol\tona{2927}{*}
 \showrelsymbol\toea{2928}{*}
 \showrelsymbol\tosa{2929}{*}
 \showrelsymbol\towa{292A}{*}
 \showrelsymbol\rightcurvedarrow{2933}{*}
 \showrelsymbol\leftdowncurvedarrow{2936}{*}
 \showrelsymbol\rightdowncurvedarrow{2937}{*}
 \showrelsymbol\cwrightarcarrow{2938}{*}
 \showrelsymbol\acwleftarcarrow{2939}{*}
 \showrelsymbol\acwoverarcarrow{293A}{*}
 \showrelsymbol\acwunderarcarrow{293B}{*}
 \showrelsymbol\curvearrowrightminus{293C}{*}
 \showrelsymbol\curvearrowleftplus{293D}{*}
 \showrelsymbol\cwundercurvearrow{293E}{*}
 \showrelsymbol\ccwundercurvearrow{293F}{*}
 \showrelsymbol\acwcirclearrow{2940}{*}
 \showrelsymbol\cwcirclearrow{2941}{*}
 \showrelsymbol\rightarrowshortleftarrow{2942}{*}
 \showrelsymbol\leftarrowshortrightarrow{2943}{*}
 \showrelsymbol\shortrightarrowleftarrow{2944}{*}
 \showrelsymbol\rightarrowplus{2945}{*}
 \showrelsymbol\leftarrowplus{2946}{*}
 \showrelsymbol\rightarrowx{2947}{*}
 \showrelsymbol\leftrightarrowcircle{2948}{*}
 \showrelsymbol\twoheaduparrowcircle{2949}{*}
 \showrelsymbol\leftrightharpoonupdown{294A}{*}
 \showrelsymbol\leftrightharpoondownup{294B}{*}
 \showrelsymbol\updownharpoonrightleft{294C}{*}
 \showrelsymbol\updownharpoonleftright{294D}{*}
 \showrelsymbol\leftrightharpoonupup{294E}{*}
 \showrelsymbol\updownharpoonrightright{294F}{*}
 \showrelsymbol\leftrightharpoondowndown{2950}{*}
 \showrelsymbol\updownharpoonleftleft{2951}{*}
 \showrelsymbol\barleftharpoonup{2952}{*}
 \showrelsymbol\rightharpoonupbar{2953}{*}
 \showrelsymbol\barupharpoonright{2954}{*}
 \showrelsymbol\downharpoonrightbar{2955}{*}
 \showrelsymbol\barleftharpoondown{2956}{*}
 \showrelsymbol\rightharpoondownbar{2957}{*}
 \showrelsymbol\barupharpoonleft{2958}{*}
 \showrelsymbol\downharpoonleftbar{2959}{*}
 \showrelsymbol\leftharpoonupbar{295A}{*}
 \showrelsymbol\barrightharpoonup{295B}{*}
 \showrelsymbol\upharpoonrightbar{295C}{*}
 \showrelsymbol\bardownharpoonright{295D}{*}
 \showrelsymbol\leftharpoondownbar{295E}{*}
 \showrelsymbol\barrightharpoondown{295F}{*}
 \showrelsymbol\upharpoonleftbar{2960}{*}
 \showrelsymbol\bardownharpoonleft{2961}{*}
 \showrelsymbol\leftharpoonsupdown{2962}{*}
 \showrelsymbol\upharpoonsleftright{2963}{*}
 \showrelsymbol\rightharpoonsupdown{2964}{*}
 \showrelsymbol\downharpoonsleftright{2965}{*}
 \showrelsymbol\leftrightharpoonsup{2966}{*}
 \showrelsymbol\leftrightharpoonsdown{2967}{*}
 \showrelsymbol\rightleftharpoonsup{2968}{*}
 \showrelsymbol\rightleftharpoonsdown{2969}{*}
 \showrelsymbol\leftharpoonupdash{296A}{*}
 \showrelsymbol\dashleftharpoondown{296B}{*}
 \showrelsymbol\rightharpoonupdash{296C}{*}
 \showrelsymbol\dashrightharpoondown{296D}{*}
 \showrelsymbol\updownharpoonsleftright{296E}{*}
 \showrelsymbol\downupharpoonsleftright{296F}{*}
 \showrelsymbol\rightimply{2970}{*}
 \showrelsymbol\equalrightarrow{2971}{*}
 \showrelsymbol\similarrightarrow{2972}{*}
 \showrelsymbol\leftarrowsimilar{2973}{*}
 \showrelsymbol\rightarrowsimilar{2974}{*}
 \showrelsymbol\rightarrowapprox{2975}{*}
 \showrelsymbol\ltlarr{2976}{*}
 \showrelsymbol\leftarrowless{2977}{*}
 \showrelsymbol\gtrarr{2978}{*}
 \showrelsymbol\subrarr{2979}{*}
 \showrelsymbol\leftarrowsubset{297A}{*}
 \showrelsymbol\suplarr{297B}{*}
 \showrelsymbol\leftfishtail{297C}{*}
 \showrelsymbol\rightfishtail{297D}{*}
 \showrelsymbol\upfishtail{297E}{*}
 \showrelsymbol\downfishtail{297F}{*}
 \showrelsymbol\rtriltri{29CE}{*}
 \showrelsymbol\ltrivb{29CF}{*}
 \showrelsymbol\vbrtri{29D0}{*}
 \showrelsymbol\lfbowtie{29D1}{*}
 \showrelsymbol\rfbowtie{29D2}{*}
 \showrelsymbol\fbowtie{29D3}{*}
 \showrelsymbol\lftimes{29D4}{*}
 \showrelsymbol\rftimes{29D5}{*}
 \showrelsymbol\dualmap{29DF}{*}
 \showrelsymbol\lrtriangleeq{29E1}{*}
 \showrelsymbol\eparsl{29E3}{*}
 \showrelsymbol\smeparsl{29E4}{*}
 \showrelsymbol\eqvparsl{29E5}{*}
 \showrelsymbol\gleichstark{29E6}{*}
 \showrelsymbol\ruledelayed{29F4}{*}
 \showrelsymbol\veeonwedge{2A59}{*}
 \showrelsymbol\eqdot{2A66}{}
 \showrelsymbol\dotequiv{2A67}{}
 \showrelsymbol\equivVert{2A68}{*}
 \showrelsymbol\equivVvert{2A69}{*}
 \showrelsymbol\dotsim{2A6A}{}
 \showrelsymbol\simrdots{2A6B}{*}
 \showrelsymbol\simminussim{2A6C}{*}
 \showrelsymbol\congdot{2A6D}{}
 \showrelsymbol\asteq{2A6E}{}
 \showrelsymbol\hatapprox{2A6F}{}
 \showrelsymbol\approxeqq{2A70}{}
 \showrelsymbol\eqqsim{2A73}{}
 \showrelsymbol\Coloneq{2A74}{*}
 \showrelsymbol\eqeq{2A75}{*}
 \showrelsymbol\eqeqeq{2A76}{*}
 \showrelsymbol\ddotseq{2A77}{*}
 \showrelsymbol\equivDD{2A78}{*}
 \showrelsymbol\ltcir{2A79}{*}
 \showrelsymbol\gtcir{2A7A}{*}
 \showrelsymbol\ltquest{2A7B}{*}
 \showrelsymbol\gtquest{2A7C}{*}
 \showrelsymbol\leqslant{2A7D}{}
 \showrelsymbol\geqslant{2A7E}{}
 \showrelsymbol\lesdot{2A7F}{*}
 \showrelsymbol\gesdot{2A80}{*}
 \showrelsymbol\lesdoto{2A81}{*}
 \showrelsymbol\gesdoto{2A82}{*}
 \showrelsymbol\lesdotor{2A83}{*}
 \showrelsymbol\gesdotol{2A84}{*}
 \showrelsymbol\lessapprox{2A85}{*}
 \showrelsymbol\gtrapprox{2A86}{*}
 \showrelsymbol\lneq{2A87}{}
 \showrelsymbol\gneq{2A88}{}
 \showrelsymbol\lnapprox{2A89}{}
 \showrelsymbol\gnapprox{2A8A}{}
 \showrelsymbol\lesseqqgtr{2A8B}{*}
 \showrelsymbol\gtreqqless{2A8C}{*}
 \showrelsymbol\lsime{2A8D}{*}
 \showrelsymbol\gsime{2A8E}{*}
 \showrelsymbol\lsimg{2A8F}{*}
 \showrelsymbol\gsiml{2A90}{*}
 \showrelsymbol\lgE{2A91}{*}
 \showrelsymbol\glE{2A92}{*}
 \showrelsymbol\lesges{2A93}{*}
 \showrelsymbol\gesles{2A94}{*}
 \showrelsymbol\eqslantless{2A95}{}
 \showrelsymbol\eqslantgtr{2A96}{}
 \showrelsymbol\elsdot{2A97}{*}
 \showrelsymbol\egsdot{2A98}{*}
 \showrelsymbol\eqqless{2A99}{*}
 \showrelsymbol\eqqgtr{2A9A}{*}
 \showrelsymbol\eqqslantless{2A9B}{*}
 \showrelsymbol\eqqslantgtr{2A9C}{*}
 \showrelsymbol\simless{2A9D}{}
 \showrelsymbol\simgtr{2A9E}{}
 \showrelsymbol\simlE{2A9F}{*}
 \showrelsymbol\simgE{2AA0}{*}
 \showrelsymbol\Lt{2AA1}{*}
 \showrelsymbol\Gt{2AA2}{*}
 \showrelsymbol\partialmeetcontraction{2AA3}{*}
 \showrelsymbol\glj{2AA4}{*}
 \showrelsymbol\gla{2AA5}{*}
 \showrelsymbol\ltcc{2AA6}{*}
 \showrelsymbol\gtcc{2AA7}{*}
 \showrelsymbol\lescc{2AA8}{*}
 \showrelsymbol\gescc{2AA9}{*}
 \showrelsymbol\smt{2AAA}{*}
 \showrelsymbol\lat{2AAB}{*}
 \showrelsymbol\smte{2AAC}{*}
 \showrelsymbol\late{2AAD}{*}
 \showrelsymbol\bumpeqq{2AAE}{*}
 \showrelsymbol\preceq{2AAF}{}
 \showrelsymbol\npreceq{XXXX}{*}
 \showrelsymbol\succeq{2AB0}{}
 \showrelsymbol\nsucceq{XXXX}{*}
 \showrelsymbol\precneq{2AB1}{*}
 \showrelsymbol\succneq{2AB2}{*}
 \showrelsymbol\preceqq{2AB3}{*}
 \showrelsymbol\succeqq{2AB4}{*}
 \showrelsymbol\precneqq{2AB5}{*}
 \showrelsymbol\succneqq{2AB6}{*}
 \showrelsymbol\precapprox{2AB7}{*}
 \showrelsymbol\succapprox{2AB8}{*}
 \showrelsymbol\precnapprox{2AB9}{*}
 \showrelsymbol\succnapprox{2ABA}{*}
 \showrelsymbol\Prec{2ABB}{*}
 \showrelsymbol\Succ{2ABC}{*}
 \showrelsymbol\subsetdot{2ABD}{}
 \showrelsymbol\supsetdot{2ABE}{}
 \showrelsymbol\subsetplus{2ABF}{*}
 \showrelsymbol\supsetplus{2AC0}{*}
 \showrelsymbol\submult{2AC1}{*}
 \showrelsymbol\supmult{2AC2}{*}
 \showrelsymbol\subedot{2AC3}{*}
 \showrelsymbol\supedot{2AC4}{*}
 \showrelsymbol\subseteqq{2AC5}{}
 \showrelsymbol\nsubseteqq{XXXX}{*}
 \showrelsymbol\supseteqq{2AC6}{}
 \showrelsymbol\nsupseteqq{XXXX}{*}
 \showrelsymbol\subsim{2AC7}{*}
 \showrelsymbol\supsim{2AC8}{*}
 \showrelsymbol\subsetapprox{2AC9}{*}
 \showrelsymbol\supsetapprox{2ACA}{*}
 \showrelsymbol\subsetneqq{2ACB}{}
 \showrelsymbol\varsubsetneqq{2ACB}{*}
 \showrelsymbol\supsetneqq{2ACC}{}
 \showrelsymbol\varsupsetneqq{2ACC}{*}
 \showrelsymbol\lsqhook{2ACD}{}
 \showrelsymbol\rsqhook{2ACE}{}
 \showrelsymbol\csub{2ACF}{}
 \showrelsymbol\csup{2AD0}{}
 \showrelsymbol\csube{2AD1}{}
 \showrelsymbol\csupe{2AD2}{}
 \showrelsymbol\subsup{2AD3}{}
 \showrelsymbol\supsub{2AD4}{}
 \showrelsymbol\subsub{2AD5}{}
 \showrelsymbol\supsup{2AD6}{}
 \showrelsymbol\suphsub{2AD7}{}
 \showrelsymbol\supdsub{2AD8}{}
 \showrelsymbol\forkv{2AD9}{}
 \showrelsymbol\topfork{2ADA}{}
 \showrelsymbol\mlcp{2ADB}{}
 \showrelsymbol\forks{2ADC}{}
 \showrelsymbol\forksnot{2ADD}{}
 \showrelsymbol\shortlefttack{2ADE}{}
 \showrelsymbol\shortdowntack{2ADF}{}
 \showrelsymbol\shortuptack{2AE0}{}
 \showrelsymbol\vDdash{2AE2}{}
 \showrelsymbol\dashV{2AE3}{}
 \showrelsymbol\Dashv{2AE4}{}
 \showrelsymbol\DashV{2AE5}{}
 \showrelsymbol\varVdash{2AE6}{}
 \showrelsymbol\Barv{2AE7}{}
 \showrelsymbol\vBar{2AE8}{}
 \showrelsymbol\vBarv{2AE9}{}
 \showrelsymbol\barV{2AEA}{}
 \showrelsymbol\Vbar{2AEB}{}
 \showrelsymbol\Not{2AEC}{}
 \showrelsymbol\bNot{2AED}{}
 \showrelsymbol\revnmid{2AEE}{}
 \showrelsymbol\cirmid{2AEF}{}
 \showrelsymbol\midcir{2AF0}{}
 \showrelsymbol\nhpar{2AF2}{}
 \showrelsymbol\parsim{2AF3}{}
 \showrelsymbol\lllnest{2AF7}{}
 \showrelsymbol\gggnest{2AF8}{}
 \showrelsymbol\leqqslant{2AF9}{}
 \showrelsymbol\geqqslant{2AFA}{}
 \showrelsymbol\circleonleftarrow{2B30}{*}
 \showrelsymbol\leftthreearrows{2B31}{*}
 \showrelsymbol\leftarrowonoplus{2B32}{*}
 \showrelsymbol\longleftsquigarrow{2B33}{*}
 \showrelsymbol\nvtwoheadleftarrow{2B34}{*}
 \showrelsymbol\nVtwoheadleftarrow{2B35}{*}
 \showrelsymbol\twoheadmapsfrom{2B36}{*}
 \showrelsymbol\twoheadleftdbkarrow{2B37}{*}
 \showrelsymbol\leftdotarrow{2B38}{*}
 \showrelsymbol\nvleftarrowtail{2B39}{*}
 \showrelsymbol\nVleftarrowtail{2B3A}{*}
 \showrelsymbol\twoheadleftarrowtail{2B3B}{*}
 \showrelsymbol\nvtwoheadleftarrowtail{2B3C}{*}
 \showrelsymbol\nVtwoheadleftarrowtail{2B3D}{*}
 \showrelsymbol\leftarrowx{2B3E}{*}
 \showrelsymbol\leftcurvedarrow{2B3F}{*}
 \showrelsymbol\equalleftarrow{2B40}{*}
 \showrelsymbol\bsimilarleftarrow{2B41}{*}
 \showrelsymbol\leftarrowbackapprox{2B42}{*}
 \showrelsymbol\rightarrowgtr{2B43}{*}
 \showrelsymbol\rightarrowsupset{2B44}{*}
 \showrelsymbol\LLeftarrow{2B45}{*}
 \showrelsymbol\RRightarrow{2B46}{*}
 \showrelsymbol\bsimilarrightarrow{2B47}{*}
 \showrelsymbol\rightarrowbackapprox{2B48}{*}
 \showrelsymbol\similarleftarrow{2B49}{*}
 \showrelsymbol\leftarrowapprox{2B4A}{*}
 \showrelsymbol\leftarrowbsimilar{2B4B}{*}
 \showrelsymbol\rightarrowbsimilar{2B4C}{*}
 \showrelsymbol\ngeqq{XXXX}{}
 \showrelsymbol\ngeqslant{XXXX}{}
 \showrelsymbol\nleqslant{XXXX}{}
 \showrelsymbol\nleqq{XXXX}{}
% \showrelsymbol\ncongdot{XXXX}{}
% \showrelsymbol\napproxeqq{XXXX}{}
% \showrelsymbol\nll{XXXX}{}
% \showrelsymbol\ngg{XXXX}{}
% \showrelsymbol\nsqsubset{XXXX}{}
% \showrelsymbol\nsqsupset{XXXX}{}
% \showrelsymbol\nBumpeq{XXXX}{}
% \showrelsymbol\nbumpeq{XXXX}{}
%\showrelsymbol\neqsim{XXXX}{}
 %\showrelsymbol\nvarisinobar{XXXX}{}
% \showrelsymbol\nvarniobar{XXXX}{}
% \showrelsymbol\neqslantless{XXXX}{}
 %\showrelsymbol\neqslantgtr{XXXX}{}
 \showrelsymbol\lhook{XXXX}{}
 \showrelsymbol\rhook{XXXX}{}
 \showrelsymbol\relbar{XXXX}{}
 \showrelsymbol\Relbar{XXXX}{}
 %\showrelsymbol\Rrelbar{XXXX}{*}
% \showrelsymbol\RRelbar{XXXX}{*}
 \showrelsymbol\mapsfromchar{XXXX}{}
 \showrelsymbol\mapstochar{XXXX}{}
 \end{multicols}
 %%% TODO HOW TO HANDLE BOTH LOCALLY
 %%%
 \subsection{Integrals}
 \label{integrals}
 Integrals come in two styles, the slanted versions shown below (|$\intsl$|,
 etc.)\ and right versions such as~|$\int$|. By default, the symbol names
 listed below will give you the slanted style, but if you specify the |int|
 package option, they will give you the corresponding right symbols.

 It is highly recommended that authors stick to the names below and use the
 |int| package option to choose a style globally for their document.
 However, in recognition of the fact that it might occasionally be necessary
 to mix the two styles, alternative names have been provided for all
 integrals. Append |sl| or || to the names below to request either the
 \emph{sl}anted or the \emph{}right variant. Thus, |$\intsl$| will always
 yield~|$\intsl$| and |$\int$| will always yield~|$\int$|, and similarly for
 the other integrals.
%
 \begin{multicols}{2}
 \integralsymbol\int{222B}{}
 \integralsymbol\iint{222C}{}
 \integralsymbol\iiint{222D}{}
 \integralsymbol\oint{222E}{}
 \integralsymbol\oiint{222F}{}
 \integralsymbol\oiiint{2230}{}
 \integralsymbol\intclockwise{2231}{}
 \integralsymbol\varointclockwise{2232}{}
 \integralsymbol\ointctrclockwise{2233}{}
 \integralsymbol\sumint{2A0B}{}
 \integralsymbol\iiiint{2A0C}{}
 \integralsymbol\intbar{2A0D}{}
 \integralsymbol\intBar{2A0E}{}
 \integralsymbol\fint{2A0F}{}
 \integralsymbol\cirfnint{2A10}{}
 \integralsymbol\awint{2A11}{}
 \integralsymbol\rppolint{2A12}{}
 \integralsymbol\scpolint{2A13}{}
 \integralsymbol\npolint{2A14}{}
 \integralsymbol\pointint{2A15}{}
 \integralsymbol\sqint{2A16}{}
 \integralsymbol\intlarhk{2A17}{}
 \integralsymbol\intx{2A18}{}
% \integralsymbol\intcapup{2A19}{}
 %\integralsymbol\intc{2A1A}{}
% \integralsymbol\intcup{2A1B}{}
 \integralsymbol\lowint{2A1C}{}
 \end{multicols}

\endinput



%%%%%%%%%%%%%%%%%%%%%%%%%%%%%%%
 \subsection{Ordinary symbols}
 \begin{multicols}{2}
 %\showsymbol\#{0023}{}
 %\showsymbol\mathdollar{0024}{}
 % \showsymbol\%{0025}{}
 %\showsymbol\&{0026}{}
% \showsymbol.{002E}{}
% \showsymbol/{002F}{}
% \showsymbol?{003F}{}
% \showsymbol@{0040}{}
 \showsymbol\backslash{005C}{}
 \showsymbol\mathsterling{00A3}{}
 \showsymbol\mathsection{00A7}{}
 \showsymbol\neg{00AC}{}, \cmd\lnot
 \showsymbol\mathparagraph{00B6}{}
 \showsymbol\eth{00F0}{}
 \showsymbol\Zbar{01B5}{*}
 \showsymbol\digamma{03DD}{}
 \showsymbol\varkappa{03F0}{}
 \showsymbol\backepsilon{03F6}{}
 \showsymbol\upbackepsilon{03F6}{}
 \showsymbol\enleadertwodots{2025}{}
 \showsymbol\mathellipsis{2026}{}
 \showsymbol\prime{2032}{}
 \showsymbol\dprime{2033}{}
 \showsymbol\trprime{2034}{}
 \showsymbol\backprime{2035}{}
 \showsymbol\backdprime{2036}{}
 \showsymbol\backtrprime{2037}{}
 \showsymbol\caretinsert{2038}{}
 \showsymbol\Exclam{203C}{}
 \showsymbol\hyphenbullet{2043}{*}
 \showsymbol\Question{2047}{}
 \showsymbol\qprime{2057}{}
 \showsymbol\enclosecircle{20DD}{}\indexmathcmd[Circles]{\enclosecircle}
 \showsymbol\enclosesquare{20DE}{*}
 \showsymbol\enclosediamond{20DF}{*}
 \showsymbol\enclosetriangle{20E4}{}
 \showsymbol\Eulerconst{2107}{}
 \showsymbol\hbar{210F}{*}
 \showsymbol\hslash{210F}{}
 \showsymbol\Im{2111}{}
 \showsymbol\ell{2113}{}
 \showsymbol\wp{2118}{}
 \showsymbol\Re{211C}{}
 \showsymbol\mho{2127}{}
 \showsymbol\turnediota{2129}{}
 \showsymbol\Angstrom{212B}{}
 \showsymbol\Finv{2132}{}
 \showsymbol\aleph{2135}{}
 \showsymbol\beth{2136}{}
 \showsymbol\gimel{2137}{}
 \showsymbol\daleth{2138}{}
 \showsymbol\Game{2141}{*}
 \showsymbol\sansLturned{2142}{*}
 \showsymbol\sansLmirrored{2143}{*}
 \showsymbol\Yup{2144}{*}
 \showsymbol\PropertyLine{214A}{*}
 \showsymbol\updownarrowbar{21A8}{}
 \showsymbol\linefeed{21B4}{}
 \showsymbol\carriagereturn{21B5}{}
 \showsymbol\barovernorthwestarrow{21B8}{}
 \showsymbol\barleftarrowrightarrowbar{21B9}{}
 \showsymbol\acwopencirclearrow{21BA}{}\indexmathcmd[Circles]{\acwopencirclearrow}
 \showsymbol\cwopencirclearrow{21BB}{}\indexmathcmd[Circles]{\cwopencirclearrow}
 \showsymbol\nHuparrow{21DE}{*}
 \showsymbol\nHdownarrow{21DF}{*}
 \showsymbol\leftdasharrow{21E0}{*}
 \showsymbol\updasharrow{21E1}{*}
 \showsymbol\rightdasharrow{21E2}{*}
 \showsymbol\downdasharrow{21E3}{*}
 \showsymbol\leftwhitearrow{21E6}{}
 \showsymbol\upwhitearrow{21E7}{}
 \showsymbol\rightwhitearrow{21E8}{}
 \showsymbol\downwhitearrow{21E9}{}
 \showsymbol\whitearrowupfrombar{21EA}{}
 \showsymbol\forall{2200}{}
 \showsymbol\complement{2201}{}
 \showsymbol\exists{2203}{}
 \showsymbol\nexists{2204}{}
 \showsymbol\varnothing{2205}{}
 \showsymbol\emptyset{2205}{}
 \showsymbol\increment{2206}{}
 \showsymbol\QED{220E}{*}
 \showsymbol\infty{221E}{}
 \showsymbol\rightangle{221F}{}
 \showsymbol\angle{2220}{}
 \showsymbol\measuredangle{2221}{}
 \showsymbol\sphericalangle{2222}{}
 \showsymbol\therefore{2234}{}
 \showsymbol\because{2235}{}
 \showsymbol\sinewave{223F}{}
 \showsymbol\top{22A4}{}
 \showsymbol\bot{22A5}{}
 \showsymbol\hermitmatrix{22B9}{}
 \showsymbol\measuredrightangle{22BE}{}
 \showsymbol\varlrtriangle{22BF}{}
 %\showsymbol\cdots{22EF}{} % TO FIX
 \showsymbol\diameter{2300}{*}
 \showsymbol\house{2302}{}
 \showsymbol\invnot{2310}{}
 \showsymbol\sqlozenge{2311}{*}
 \showsymbol\profline{2312}{*}
 \showsymbol\profsurf{2313}{*}
 \showsymbol\viewdata{2317}{*}
 \showsymbol\turnednot{2319}{}
 \showsymbol\varhexagonlrbonds{232C}{*}
 \showsymbol\conictaper{2332}{*}
 \showsymbol\topbot{2336}{}
 \showsymbol\APLnotbackslash{2340}{*}
 \showsymbol\APLboxupcaret{2353}{*}
 \showsymbol\APLboxquestion{2370}{*}
 \showsymbol\rangledownzigzagarrow{237C}{*}
 \showsymbol\hexagon{2394}{*}
 \showsymbol\bbrktbrk{23B6}{}
 \showsymbol\varcarriagereturn{23CE}{*}
 \showsymbol\obrbrak{23E0}{}
 \showsymbol\ubrbrak{23E1}{}
 \showsymbol\trapezium{23E2}{*}
 \showsymbol\benzenr{23E3}{*}
 \showsymbol\strns{23E4}{*}
 \showsymbol\fltns{23E5}{*}
 \showsymbol\accurrent{23E6}{*}
 \showsymbol\elinters{23E7}{*}
 \showsymbol\mathvisiblespace{2423}{}
 \showsymbol\circledR{24C7}{}
 \showsymbol\circledS{24C8}{}
 \showsymbol\mdlgblksquare{25A0}{*}, \cmd\blacksquare
 \showsymbol\mdlgwhtsquare{25A1}{*}, \cmd\square, \cmd\Box
 \showsymbol\squoval{25A2}{*}
 \showsymbol\blackinwhitesquare{25A3}{*}
 \showsymbol\squarehfill{25A4}{*}
 \showsymbol\squarevfill{25A5}{*}
 \showsymbol\squarehvfill{25A6}{*}
 \showsymbol\squarenwsefill{25A7}{*}
 \showsymbol\squareneswfill{25A8}{*}
 \showsymbol\squarecrossfill{25A9}{*}
 \showsymbol\smblksquare{25AA}{*}
 \showsymbol\smwhtsquare{25AB}{*}
 \showsymbol\hrectangleblack{25AC}{*}
 \showsymbol\hrectangle{25AD}{*}
 \showsymbol\vrectangleblack{25AE}{*}
 \showsymbol\vrectangle{25AF}{*}
 \showsymbol\parallelogramblack{25B0}{*}
 \showsymbol\parallelogram{25B1}{*}
 \showsymbol\bigblacktriangleup{25B2}{*}
 \showsymbol\blacktriangle{25B4}{*}
 \showsymbol\blacktriangleright{25B6}{*}
 \showsymbol\smallblacktriangleright{25B8}{*}
 \showsymbol\smalltriangleright{25B9}{*}
 \showsymbol\blackpointerright{25BA}{*}
 \showsymbol\whitepointerright{25BB}{*}
 \showsymbol\bigblacktriangledown{25BC}{*}
 \showsymbol\bigtriangledown{25BD}{}
 \showsymbol\blacktriangledown{25BE}{*}
 \showsymbol\triangledown{25BF}{*}
 \showsymbol\blacktriangleleft{25C0}{*}
 \showsymbol\smallblacktriangleleft{25C2}{*}
 \showsymbol\smalltriangleleft{25C3}{*}
 \showsymbol\blackpointerleft{25C4}{*}
 \showsymbol\whitepointerleft{25C5}{*}
 \showsymbol\mdlgblkdiamond{25C6}{*}
 \showsymbol\mdlgwhtdiamond{25C7}{*}
 \showsymbol\blackinwhitediamond{25C8}{*}
 \showsymbol\fisheye{25C9}{*}
 \showsymbol\mdlgwhtlozenge{25CA}{}, \cmd\lozenge, \\ \cmd\Diamond
 \showsymbol\dottedcircle{25CC}{*}
 \showsymbol\circlevertfill{25CD}{*}
 \showsymbol\bullseye{25CE}{*}
 \showsymbol\mdlgblkcircle{25CF}{*}
 \showsymbol\circlelefthalfblack{25D0}{*}
 \showsymbol\circlerighthalfblack{25D1}{*}
 \showsymbol\circlebottomhalfblack{25D2}{*}
 \showsymbol\circletophalfblack{25D3}{*}
 \showsymbol\circleurquadblack{25D4}{*}
 \showsymbol\blackcircleulquadwhite{25D5}{*}
 \showsymbol\blacklefthalfcircle{25D6}{*}
 \showsymbol\blackrighthalfcircle{25D7}{*}
 \showsymbol\inversebullet{25D8}{*}
 \showsymbol\inversewhitecircle{25D9}{*}
 \showsymbol\invwhiteupperhalfcircle{25DA}{*}
 \showsymbol\invwhitelowerhalfcircle{25DB}{*}
 \showsymbol\ularc{25DC}{*}
 \showsymbol\urarc{25DD}{*}
 \showsymbol\lrarc{25DE}{*}
 \showsymbol\llarc{25DF}{*}
 \showsymbol\topsemicircle{25E0}{*}
 \showsymbol\botsemicircle{25E1}{*}
 \showsymbol\lrblacktriangle{25E2}{*}
 \showsymbol\llblacktriangle{25E3}{*}
 \showsymbol\ulblacktriangle{25E4}{*}
 \showsymbol\urblacktriangle{25E5}{*}
 \showsymbol\circ{25E6}{}, \cmd\smwhtcircle
 \showsymbol\squareleftblack{25E7}{*}
 \showsymbol\squarerightblack{25E8}{*}
 \showsymbol\squareulblack{25E9}{*}
 \showsymbol\squarelrblack{25EA}{*}
 \showsymbol\trianglecdot{25EC}{}
 \showsymbol\triangleleftblack{25ED}{*}
 \showsymbol\trianglerightblack{25EE}{*}
 \showsymbol\lgwhtcircle{25EF}{*}
 \showsymbol\squareulquad{25F0}{*}
 \showsymbol\squarellquad{25F1}{*}
 \showsymbol\squarelrquad{25F2}{*}
 \showsymbol\squareurquad{25F3}{*}
 \showsymbol\circleulquad{25F4}{*}
 \showsymbol\circlellquad{25F5}{*}
 \showsymbol\circlelrquad{25F6}{*}
 \showsymbol\circleurquad{25F7}{*}
 \showsymbol\ultriangle{25F8}{*}
 \showsymbol\urtriangle{25F9}{*}
 \showsymbol\lltriangle{25FA}{*}
 \showsymbol\mdwhtsquare{25FB}{*}
 \showsymbol\mdblksquare{25FC}{*}
 \showsymbol\mdsmwhtsquare{25FD}{*}
 \showsymbol\mdsmblksquare{25FE}{*}
 \showsymbol\lrtriangle{25FF}{*}
 \showsymbol\bigstar{2605}{*}
 \showsymbol\bigwhitestar{2606}{*}
 \showsymbol\astrosun{2609}{}
 \showsymbol\danger{2621}{}
 \showsymbol\blacksmiley{263B}{}
 \showsymbol\sun{263C}{}
 \showsymbol\rightmoon{263D}{}
 \showsymbol\leftmoon{263E}{}
 \showsymbol\female{2640}{}
 \showsymbol\male{2642}{}
 \showsymbol\spadesuit{2660}{*}
 \showsymbol\heartsuit{2661}{*}
 \showsymbol\diamondsuit{2662}{*}
 \showsymbol\clubsuit{2663}{*}
 \showsymbol\varspadesuit{2664}{}
 \showsymbol\varheartsuit{2665}{}
 \showsymbol\vardiamondsuit{2666}{}
 \showsymbol\varclubsuit{2667}{}
 \showsymbol\quarternote{2669}{}
 \showsymbol\eighthnote{266A}{}
 \showsymbol\twonotes{266B}{}
 \showsymbol\flat{266D}{}
 \showsymbol\natural{266E}{}
 \showsymbol\sharp{266F}{}
 \showsymbol\acidfree{267E}{*}
 \showsymbol\dicei{2680}{}
 \showsymbol\diceii{2681}{}
 \showsymbol\diceiii{2682}{}
 \showsymbol\diceiv{2683}{}
 \showsymbol\dicev{2684}{}
 \showsymbol\dicevi{2685}{}
 \showsymbol\circledrightdot{2686}{}
 \showsymbol\circledtwodots{2687}{}
 \showsymbol\blackcircledrightdot{2688}{}
 \showsymbol\blackcircledtwodots{2689}{}
 \showsymbol\Hermaphrodite{26A5}{}
 \showsymbol\mdwhtcircle{26AA}{}
 \showsymbol\mdblkcircle{26AB}{}
 \showsymbol\mdsmwhtcircle{26AC}{}
 \showsymbol\neuter{26B2}{}
 \showsymbol\checkmark{2713}{}
 \showsymbol\maltese{2720}{}
 \showsymbol\circledstar{272A}{}
 \showsymbol\varstar{2736}{}
 \showsymbol\dingasterisk{273D}{}
 \showsymbol\draftingarrow{279B}{*}
 \showsymbol\threedangle{27C0}{*}
 \showsymbol\whiteinwhitetriangle{27C1}{*}
 \showsymbol\subsetcirc{27C3}{*}
 \showsymbol\supsetcirc{27C4}{*}
 \showsymbol\diagup{27CB}{*}
 \showsymbol\diagdown{27CD}{*}
 \showsymbol\diamondcdot{27D0}{*}
 \showsymbol\rdiagovfdiag{292B}{*}
 \showsymbol\fdiagovrdiag{292C}{*}
 \showsymbol\seovnearrow{292D}{*}
 \showsymbol\neovsearrow{292E}{*}
 \showsymbol\fdiagovnearrow{292F}{*}
 \showsymbol\rdiagovsearrow{2930}{*}
 \showsymbol\neovnwarrow{2931}{*}
 \showsymbol\nwovnearrow{2932}{*}
 \showsymbol\uprightcurvearrow{2934}{*}
 \showsymbol\downrightcurvedarrow{2935}{*}
 \showsymbol\mdsmblkcircle{2981}{*}
 \showsymbol\fourvdots{2999}{*}
 \showsymbol\vzigzag{299A}{*}
 \showsymbol\measuredangleleft{299B}{*}
 \showsymbol\rightanglesqr{299C}{*}
 \showsymbol\rightanglemdot{299D}{*}
 \showsymbol\angles{299E}{*}
 \showsymbol\angdnr{299F}{*}
 \showsymbol\gtlpar{29A0}{*}
 \showsymbol\sphericalangleup{29A1}{*}
 \showsymbol\turnangle{29A2}{*}
 \showsymbol\revangle{29A3}{*}
 \showsymbol\angleubar{29A4}{*}
 \showsymbol\revangleubar{29A5}{*}
 \showsymbol\wideangledown{29A6}{*}
 \showsymbol\wideangleup{29A7}{*}
 \showsymbol\measanglerutone{29A8}{*}
 \showsymbol\measanglelutonw{29A9}{*}
 \showsymbol\measanglerdtose{29AA}{*}
 \showsymbol\measangleldtosw{29AB}{*}
 \showsymbol\measangleurtone{29AC}{*}
 \showsymbol\measangleultonw{29AD}{*}
 \showsymbol\measangledrtose{29AE}{*}
 \showsymbol\measangledltosw{29AF}{*}
 \showsymbol\revemptyset{29B0}{*}
 \showsymbol\emptysetobar{29B1}{*}
 \showsymbol\emptysetocirc{29B2}{*}
 \showsymbol\emptysetoarr{29B3}{*}
 \showsymbol\emptysetoarrl{29B4}{*}
 \showsymbol\obot{29BA}{*}
 \showsymbol\olcross{29BB}{*}
 \showsymbol\odotslashdot{29BC}{*}
 \showsymbol\uparrowoncircle{29BD}{*}
 \showsymbol\circledwhitebullet{29BE}{*}
 \showsymbol\circledbullet{29BF}{*}
 \showsymbol\cirscir{29C2}{*}
 \showsymbol\cirE{29C3}{*}
 \showsymbol\boxonbox{29C9}{*}
 \showsymbol\triangleodot{29CA}{*}
 \showsymbol\triangleubar{29CB}{*}
 \showsymbol\triangles{29CC}{*}
 \showsymbol\iinfin{29DC}{*}
 \showsymbol\tieinfty{29DD}{*}
 \showsymbol\nvinfty{29DE}{*}
 \showsymbol\laplac{29E0}{*}
 \showsymbol\thermod{29E7}{*}
 \showsymbol\downtriangleleftblack{29E8}{*}
 \showsymbol\downtrianglerightblack{29E9}{*}
 \showsymbol\blackdiamonddownarrow{29EA}{*}
 \showsymbol\blacklozenge{29EB}{}
 \showsymbol\circledownarrow{29EC}{*}
 \showsymbol\blackcircledownarrow{29ED}{*}
 \showsymbol\errbarsquare{29EE}{*}
 \showsymbol\errbarblacksquare{29EF}{*}
 \showsymbol\errbardiamond{29F0}{*}
 \showsymbol\errbarblackdiamond{29F1}{*}
 \showsymbol\errbarcircle{29F2}{*}
 \showsymbol\errbarblackcircle{29F3}{*}
 \showsymbol\perps{2AE1}{}
 \showsymbol\topcir{2AF1}{}
 \showsymbol\squaretopblack{2B12}{}
 \showsymbol\squarebotblack{2B13}{}
 \showsymbol\squareurblack{2B14}{}
 \showsymbol\squarellblack{2B15}{}
 \showsymbol\diamondleftblack{2B16}{}
 \showsymbol\diamondrightblack{2B17}{}
 \showsymbol\diamondtopblack{2B18}{}
 \showsymbol\diamondbotblack{2B19}{}
 \showsymbol\dottedsquare{2B1A}{}
 \showsymbol\lgblksquare{2B1B}{}
 \showsymbol\lgwhtsquare{2B1C}{}
 \showsymbol\vysmblksquare{2B1D}{}
 \showsymbol\vysmwhtsquare{2B1E}{}
 \showsymbol\pentagonblack{2B1F}{}
 \showsymbol\pentagon{2B20}{}
 \showsymbol\varhexagon{2B21}{}
 \showsymbol\varhexagonblack{2B22}{}
 \showsymbol\hexagonblack{2B23}{}
 \showsymbol\lgblkcircle{2B24}{}
 \showsymbol\mdblkdiamond{2B25}{}
 \showsymbol\mdwhtdiamond{2B26}{}
 \showsymbol\mdblklozenge{2B27}{}
 \showsymbol\mdwhtlozenge{2B28}{}
 \showsymbol\smblkdiamond{2B29}{}
 \showsymbol\smblklozenge{2B2A}{}
 \showsymbol\smwhtlozenge{2B2B}{}
 \showsymbol\blkhorzoval{2B2C}{}
 \showsymbol\whthorzoval{2B2D}{}
 \showsymbol\blkvertoval{2B2E}{}
 \showsymbol\whtvertoval{2B2F}{}
 \showsymbol\medwhitestar{2B50}{}
 \showsymbol\medblackstar{2B51}{}
 \showsymbol\smwhitestar{2B52}{}
 \showsymbol\rightpentagonblack{2B53}{}
 \showsymbol\rightpentagon{2B54}{}
 \showsymbol\postalmark{3012}{}
 \showsymbol\hzigzag{3030}{}
 \showsymbol\Bbbk{1D55C}{}
 \showsymbol\bracevert{XXXX}{*}
 \end{multicols}

 \subsection{Binary operators}
 \index{binary operators}
 \begin{multicols}{2}
% \showsymbolbin+{000B}{}
 \showsymbolbin\pm{00B1}{}
 \showsymbolbin\cdotp{00B7}{}%?, \cmd\centerdot
 \showsymbolbin\times{00D7}{}
 \showsymbolbin\div{00F7}{}
 \showsymbolbin\dagger{2020}{}
 \showsymbolbin\ddagger{2021}{}
 \showsymbolbin\smblkcircle{2022}{}
 \showsymbolbin\fracslash{2044}{}
 \showsymbolbin\upand{214B}{}
% \showsymbolbin-{000D}{}
 \showsymbolbin\mp{2213}{}
 \showsymbolbin\dotplus{2214}{}
 \showsymbolbin\smallsetminus{2216}{}
 \showsymbolbin\ast{2217}{}
 \showsymbolbin\vysmwhtcircle{2218}{}
 \showsymbolbin\vysmblkcircle{2219}{}, {\small\cmd\bullet}
 \showsymbolbin\wedge{2227}{}, \cmd\land
 \showsymbolbin\vee{2228}{}, \cmd\lor
 \showsymbolbin\cap{2229}{}
 \showsymbolbin\cup{222A}{}
 \showsymbolbin\dotminus{2238}{}
 \showsymbolbin\invlazys{223E}{}
 \showsymbolbin\wr{2240}{}
 \showsymbolbin\cupleftarrow{228C}{}
 \showsymbolbin\cupdot{228D}{}
 \showsymbolbin\uplus{228E}{}
 \showsymbolbin\sqcap{2293}{}
 \showsymbolbin\sqcup{2294}{}
 \showsymbolbin\oplus{2295}{}
 \showsymbolbin\ominus{2296}{}
 \showsymbolbin\otimes{2297}{}
 \showsymbolbin\oslash{2298}{}
 \showsymbolbin\odot{2299}{}
 \showsymbolbin\circledcirc{229A}{}
 \showsymbolbin\circledast{229B}{}
 \showsymbolbin\circledequal{229C}{}
 \showsymbolbin\circleddash{229D}{}
 \showsymbolbin\boxplus{229E}{}
 \showsymbolbin\boxminus{229F}{}
 \showsymbolbin\boxtimes{22A0}{}
 \showsymbolbin\boxdot{22A1}{}
 \showsymbolbin\intercal{22BA}{}
 \showsymbolbin\veebar{22BB}{}
 \showsymbolbin\barwedge{22BC}{}
 \showsymbolbin\barvee{22BD}{}
 \showsymbolbin\diamond{22C4}{}, \cmd\smwhtdiamond
 \showsymbolbin\cdot{22C5}{*}
 \showsymbolbin\star{22C6}{}
 \showsymbolbin\divideontimes{22C7}{}
 \showsymbolbin\ltimes{22C9}{}
 \showsymbolbin\rtimes{22CA}{}
 \showsymbolbin\leftthreetimes{22CB}{}
 \showsymbolbin\rightthreetimes{22CC}{}
 \showsymbolbin\curlyvee{22CE}{}
 \showsymbolbin\curlywedge{22CF}{}
 \showsymbolbin\Cap{22D2}{}, \cmd\doublecap
 \showsymbolbin\Cup{22D3}{}, \cmd\doublecup
 \showsymbolbin\varbarwedge{2305}{*}
 \showsymbolbin\vardoublebarwedge{2306}{*}
 \showsymbolbin\obar{233D}{}
 \showsymbolbin\triangle{25B3}{}, \cmd\bigtriangleup
 \showsymbolbin\lhd{22B2}{}
 \showsymbolbin\rhd{22B3}{}
 \showsymbolbin\unlhd{22B4}{}
 \showsymbolbin\unrhd{22B5}{}
 \showsymbolbin\mdlgwhtcircle{25CB}{*}
 \showsymbolbin\boxbar{25EB}{*}
 \showsymbolbin\veedot{27C7}{*}
 \showsymbolbin\wedgedot{27D1}{*}
 \showsymbolbin\lozengeminus{27E0}{*}
 \showsymbolbin\concavediamond{27E1}{*}
 \showsymbolbin\concavediamondtickleft{27E2}{*}
 \showsymbolbin\concavediamondtickright{27E3}{*}
 \showsymbolbin\whitesquaretickleft{27E4}{*}
 \showsymbolbin\whitesquaretickright{27E5}{*}
 \showsymbolbin\typecolon{2982}{*}
 \showsymbolbin\circlehbar{29B5}{*}
 \showsymbolbin\circledvert{29B6}{}
 \showsymbolbin\circledparallel{29B7}{}
 \showsymbolbin\obslash{29B8}{}
 \showsymbolbin\operp{29B9}{*}
 \showsymbolbin\olessthan{29C0}{}
 \showsymbolbin\ogreaterthan{29C1}{}
 \showsymbolbin\boxdiag{29C4}{}
 \showsymbolbin\boxbslash{29C5}{}
 \showsymbolbin\boxast{29C6}{}
 \showsymbolbin\boxcircle{29C7}{}
 \showsymbolbin\boxbox{29C8}{*}
 \showsymbolbin\triangleserifs{29CD}{*}
 \showsymbolbin\hourglass{29D6}{*}
 \showsymbolbin\blackhourglass{29D7}{*}
 \showsymbolbin\shuffle{29E2}{*}
 \showsymbolbin\mdlgblklozenge{29EB}{*}
 \showsymbolbin\setminus{29F5}{*}
 \showsymbolbin\dsol{29F6}{*}
 \showsymbolbin\rsolbar{29F7}{*}
 \showsymbolbin\doubleplus{29FA}{*}
 \showsymbolbin\tripleplus{29FB}{*}
 \showsymbolbin\tplus{29FE}{*}
 \showsymbolbin\tminus{29FF}{*}
 \showsymbolbin\ringplus{2A22}{}
 \showsymbolbin\plushat{2A23}{}
 \showsymbolbin\simplus{2A24}{}
 \showsymbolbin\plusdot{2A25}{}
 \showsymbolbin\plussim{2A26}{}
 \showsymbolbin\plussubtwo{2A27}{}
 \showsymbolbin\plustrif{2A28}{*}
 \showsymbolbin\commaminus{2A29}{*}
 \showsymbolbin\minusdot{2A2A}{}
 \showsymbolbin\minusfdots{2A2B}{}
 \showsymbolbin\minusrdots{2A2C}{*}
 \showsymbolbin\opluslhrim{2A2D}{*}
 \showsymbolbin\oplusrhrim{2A2E}{*}
 \showsymbolbin\vectimes{2A2F}{*}
 \showsymbolbin\dottimes{2A30}{}
 \showsymbolbin\timesbar{2A31}{}
 \showsymbolbin\btimes{2A32}{}
 \showsymbolbin\smashtimes{2A33}{*}
 \showsymbolbin\otimeslhrim{2A34}{*}
 \showsymbolbin\otimesrhrim{2A35}{*}
 \showsymbolbin\otimeshat{2A36}{*}
 \showsymbolbin\Otimes{2A37}{*}
 \showsymbolbin\odiv{2A38}{*}
 \showsymbolbin\triangleplus{2A39}{*}
 \showsymbolbin\triangleminus{2A3A}{*}
 \showsymbolbin\triangletimes{2A3B}{*}
 \showsymbolbin\intprod{2A3C}{*}
 \showsymbolbin\intprodr{2A3D}{*}
 \showsymbolbin\fcmp{2A3E}{*}
 \showsymbolbin\amalg{2A3F}{}
 \showsymbolbin\capdot{2A40}{*}
 \showsymbolbin\uminus{2A41}{*}
 \showsymbolbin\barcup{2A42}{*}
 \showsymbolbin\barcap{2A43}{*}
 \showsymbolbin\capwedge{2A44}{*}
 \showsymbolbin\cupvee{2A45}{*}
 \showsymbolbin\cupovercap{2A46}{*}
 \showsymbolbin\capovercup{2A47}{*}
 \showsymbolbin\cupbarcap{2A48}{*}
 \showsymbolbin\capbarcup{2A49}{*}
 \showsymbolbin\twocups{2A4A}{*}
 \showsymbolbin\twocaps{2A4B}{*}
 \showsymbolbin\closedvarcup{2A4C}{*}
 \showsymbolbin\closedvarcap{2A4D}{*}
 \showsymbolbin\Sqcap{2A4E}{*}
 \showsymbolbin\Sqcup{2A4F}{*}
 \showsymbolbin\closedvarcupsmashprod{2A50}{*}
 \showsymbolbin\wedgeodot{2A51}{*}
 \showsymbolbin\veeodot{2A52}{*}
 \showsymbolbin\Wedge{2A53}{*}
 \showsymbolbin\Vee{2A54}{*}
 \showsymbolbin\wedgeonwedge{2A55}{*}
 \showsymbolbin\veeonvee{2A56}{*}
 \showsymbolbin\bigslopedvee{2A57}{*}
 \showsymbolbin\bigslopedwedge{2A58}{*}
 \showsymbolbin\wedgemidvert{2A5A}{*}
 \showsymbolbin\veemidvert{2A5B}{*}
 \showsymbolbin\midbarwedge{2A5C}{*}
 \showsymbolbin\midbarvee{2A5D}{*}
 \showsymbolbin\doublebarwedge{2A5E}{}
 \showsymbolbin\wedgebar{2A5F}{*}
 \showsymbolbin\wedgedoublebar{2A60}{*}
 \showsymbolbin\varveebar{2A61}{*}
 \showsymbolbin\doublebarvee{2A62}{*}
 \showsymbolbin\veedoublebar{2A63}{}
 \showsymbolbin\dsub{2A64}{*}
 \showsymbolbin\rsub{2A65}{*}
 \showsymbolbin\eqqplus{2A71}{}
 \showsymbolbin\pluseqq{2A72}{}
 \showsymbolbin\interleave{2AF4}{}
 \showsymbolbin\nhVvert{2AF5}{}
 \showsymbolbin\threedotcolon{2AF6}{}
 \showsymbolbin\trslash{2AFB}{}
 \showsymbolbin\sslash{2AFD}{}
 \showsymbolbin\talloblong{2AFE}{}
 \end{multicols}










 \makeatletter
\newcommand\QEDit{\hspace{6pt}\textit{Q.~E.~D.}\quad}
\newcommand\QEFit{\hspace{6pt}\textit{Q.~E.~F.}\quad}
\newcommand\QEIit{\hspace{6pt}\textit{Q.~E.~I.}\quad}
\newcommand\QEDup{\hspace{6pt}Q.~E.~D.\quad}
\newcommand\QEFup{\hspace{6pt}Q.~E.~F.\quad}
\newcommand\QEIup{\hspace{6pt}Q.~E.~I.\quad}
\newcommand\QEOup{\hspace{6pt}Q.~E.~O.\quad}
\newcounter{wrapwidth}
\newcount \Zw
\newcount \Zh


\newcommand\pngright[4]{%
    \Zw=#2 \divide \Zw by 10
    \Zh=#3 \divide \Zh by 120  \advance\Zh by 1
    \setcounter{wrapwidth}{\Zw}
\begin{wrapfigure}[\Zh]{r}{\value{wrapwidth}pt}%
\begin{center}
\vspace{#4pt}%
\includegraphics*[width=\Zw pt]{images/#1}%
\end{center}
\end{wrapfigure}}

\newcommand\propnopage[1]{
\begin{center}{\large #1}\end{center}}

\parindent1em

\cxset{toc image=\@empty}
\chapter{PERICULA}

\noindent\textsc{Case Study: } We will now typeset a section, from Isaac Newton's \textit{Philosophi\ae\  Naturalis Principia Mathematica}. The typeset example is shown below.

\bottomline
\bgroup
\small

\cxset{section align=center,
         section numbering=none}

\section{{SECT}$\cdot$ VIII$\cdot$}

\begin{center}{\textit{De Motu per Fluida propagato.}}\end{center}

\makeatletter
%\propnopage{Prop.\ XLI\@. Theor.\ XXXI.}
\meaning\@
\makeatother

\textit{Pressio non propagatur per Fluidum secundum lineas rectas, nisi
ubi particul{\ae} Fluidi in directum jacent.}

Si jaceant particul{\ae} $a$, $b$, $c$, $d$, $e$ in linea recta, potest quidem
pressio directe

\begin{wrapfigure}[8]{O}[1pt]{0.3\textwidth}
  \vspace{-17pt}
  \includegraphics[width=0.27\textwidth]{images/362.png}
\end{wrapfigure}

\noindent propagari  ab $a$ ad $e$; at
particula $e$ urgebit particulas oblique positas
$f$ \& $g$ oblique, \& particul{\ae} ill{\ae} $f$ \& $g$
non sustinebunt pressionem illatam, nisi fulciantur
a particulis ulterioribus $h$ \& $k$;
quatenus autem fulciuntur, premunt particulas
fulcientes; \& h{\ae} non sustinebunt pressionem nisi fulciantur
ab ulterioribus $l$ \& $m$ easque premant, \& sic deinceps in infinitum.
Pressio igitur, quam primum propagatur ad particulas
qu{\ae} non in directum jacent, divaricare incipiet \& oblique propagabitur
in infinitum; \& postquam incipit oblique propagari, si
inciderit in particulas ulteriores, qu{\ae} non in directum jacent, iterum
divaricabit; idque toties, quoties in particulas non accurate
in directum jacentes inciderit. \QEDit

\topline

\vspace*{-\baselineskip}
\captionof{figure}{Example of a typeset page from Principi\ae.}
\egroup
\smallskip
Figure~\ref{fig:principia}, shows a scan of the actual page. We will not reproduce, the fonts and the page geometry exactly, but rather we will attempt to extract and reproduce the typographical rules employed in the printing of the \textit{Principi\ae}.

We begin by typesetting the section number and its heading. The use of roman numbers creates better harmony between the text and the heading

\begin{teX}
\sectpage{VIII$\middot$}
\begin{center}{\textit{De Motu per Fluida propagato.}}\end{center}
\end{teX}
The proposition and theorem line, has its own command
\begin{teX}
\makeatletter
\propnopage{Prop.\ XLI. Theor.\ XXXI.}
\makeatother
\end{teX}

\propnopage{\color{gray}Prop.\ XLI\@. Theor.\ XXXI.}
\vspace*{-37pt}
\propnopage{Prop. XLI. Theor. XXXI.}


Notice the small differences in the spacing with the commands as shown and with the black text, without them. The rest is based on normal \LaTeX\ commands.

\textit{Pressio non propagatur \ldots particul{\ae}\ldots}

\pngright{362.png}{709}{603}{-24}

Si jaceant particul{\ae} $a$, $b$, $c$, $d$,
$e$ in linea recta, potest quidem
pressio directe propagari ab $a$ ad $e$; at


Remember that it is important to start a new paragraph after the 
|pngright| command. The |wrapfig| package works by using |everypar| to insert the hanging indentation.

\begin{figure}[p]
\centering
\includegraphics[scale=1]{./images/page354.png}
\caption{Page 354 from Isaac Newton's \textit{Philosophi\ae\  Naturalis Principia Mathematica}. Image was obtained from Google's copy, available at Google Books.}
\label{fig:principia}
\end{figure}

}
  
\def\bibandindex{%
  \part{BIBLIOGRAPHIES, INDICES and GLOSSARIES}
  \chapter{Bibliography Management} 

\begin{figure}[p]
\includegraphics[width=\textwidth]{./images/ammar.jpg}
\caption{Wilson, Digital Collage, L. Ammar \protect\url{http://daliahammar.com/post/49217473452/wilson-digital-collage}}
\end{figure}
 
\precis{In this chapter we outline a number of experimental keys that been defined to handle Table of Contents (ToC) formatting. These keys are currently experimental.}
\addtocimage{-12pt}{-20pt}{./images/tocblock-man-02.jpg}

       
\def\bibtex{\texttt{bibTeX\xspace}}

For any academic/research writing, incorporating references into a document is an important task. Fortunately, \latex provides  a variety of features that make dealing with references much simpler, including built-in support for citing references. However, a much more powerful and flexible solution is achieved thanks to an auxiliary tool called \bibtex and if your \latex  distribution does not include it is obtainable from \url{http://www.bibtex.org}.


The style of this book places all citations to the side margin. For example, the command  \verb+\cite{Abrahams2003}+, will produce the citation \cite{Abrahams2003}. I find this type of style (suggested by \cite{Tufte1997}) more clear and relevant.

Notes in text for many centuries, before printed books were a common feature. The author picking up a different thread and not wishing to divert immediate attention away from the main body of his work. With printing, the costs of books were high and printers started placing citations and footnotes at the bottom of the page. You are not limited though to use only this style, by using |\cite{Bringhurst2005}|, \citet{Bringhurst2005}.

You can also use, the following code to get a within the text full citation:


\bibentry{Bringhurst2005}



\bibtex provides for the storage of all references in an external, flat-file database. This database can be linked to any \latex document, and citations made to any reference that is contained within the file. This is often more convenient than embedding them at the end of every document written. There is now a centralized bibliography source that can be linked to as many documents as desired (write once, read many!). 

Of course, bibliographies can be split over as many files as one wishes, so there can be a file containing references concerning General Relativity and another about Quantum Mechanics. When writing about Quantum Gravity (QG), which tries to bridge the gap between these two theories, both of these files can be linked into the document, in addition to references specific to QG.

\section{Citations}

To actually cite a given document is very easy. Go to the point where you want the citation to appear, and use the following: cite cite key, where the cite key is that of the bibitem you wish to cite. When LaTeX processes the document, the citation will be cross-referenced with the bibitems and replaced with the appropriate number citation. The advantage here, once again, is that LaTeX looks after the numbering for you. If it were totally manual, then adding or removing a reference would be a real chore, as you would have to re-number all the citations by hand.

Instead of WYSIWYG editors, typesetting systems like TeX or LaTeX \citep{lamport2004} can be used. cite{Abut1990}

\section{Referring to specific pages}

Sometimes you want to refer to a certain page, figure or theorem in a text book. For that you can use the arguments to the 

\begin{texexample}{Citation Example}{}
\cs{cite} command:
\cite[p. 215]{Mittelbach2004}
\end{texexample}

The argument, "p. 215", will show up inside the same brackets

\section{BibTeX}

I have previously introduced the idea of embedding references at the end of the document, and then using the \cs{cite} command to cite them within the text. In this tutorial, I want to do a little better than this method, as it's not as flexible as it could be. Which is why I wish to concentrate on using BibTeX.

A BibTeX database is stored as a .bib file. It is a plain text file, and so can be viewed and edited easily. The structure of the file is also quite simple. An example of a BibTeX entry:

\begin{verbatim}
@article{greenwade93,
    author  = "George D. Greenwade",
    title   = "The {C}omprehensive {T}ex {A}rchive {N}etwork ({CTAN})",
    year    = "1993",
    journal = "TUGBoat",
    volume  = "14",
    number  = "3",
    pages   = "342--351"
}
\end{verbatim}

Each entry begins with the declaration of the reference type, in the form of @type. BibTeX knows of practically all types you can think of, common ones are: book, article, and for papers presented at conferences, there is inproceedings. In this example, I have referred to an article within a journal.\sidenote{\obeylines 
book,
article,
conference
}

After the type, you must have a left curly brace '\{' to signify the beginning of the reference attributes. The first one follows immediately after the brace, which is the citation key. This key must be unique for all entries in your bibliography. It is this identifier that you will use within your document to cross-reference it to this entry. It is up to you as to how you wish to label each reference, but there is a loose standard in which you use the author's surname, followed by the year of publication. This is the scheme that I use in this tutorial.

Next, it should be clear that what follows are the relevant fields and data for that particular reference. The field names on the left are BibTeX keywords. They are followed by an equals sign (=) where the value for that field is then placed. BibTeX expects you to explicitly label the beginning and end of each value. I personally use quotation marks ("), however, you also have the option of using curly braces \verb+('{', '}')+. But as you will soon see, curly braces have other roles, within attributes, so I prefer not to use them for this job as they can get more confusing. 

A notable exception is when you want to use characters with umlauts (ü, ö, etc), since their notation is in the format \verb+\"{o}+, and the quotation mark will close the one opening the field, causing an error in the parsing of the reference.

Remember that each attribute must be followed by a comma to delimit one from another. You do not need to add a comma to the last attribute, since the closing brace will tell BibTeX that there are no more attributes for this entry, although you won't get an error if you do.

It can take a while to learn what the reference types are, and what fields each type has available (and which ones are required or optional, etc). So, look at this entry type reference and also this field reference for descriptions of all the fields. It may be worth bookmarking or printing these pages so that they are easily at hand when you need them.

\section{Authors}

BibTeX can be quite clever with names of authors. It can accept names in forename surname or surname, forename. I personally use the former, but remember that the order you input them (or any data within an entry for that matter) is customizable and so you can get BibTeX to manipulate the input and then output it however you like. If you use the forename surname method, then you must be careful with a few special names, where there are compound surnames, for example "John von Neumann". In this form, BibTeX assumes that the last word is the surname, and everything before is the forename, plus any middle names. You must therefore manually tell BibTeX to keep the 'von' and 'Neumann' together. This is achieved easily using curly braces. So the final result would be "John {von Neumann}". This is easily avoided with the surname, forename, since you have a comma to separate the surname from the forename.

Secondly, there is the issue of how to tell BibTeX when a reference has more than one author. This is very simply done by putting the keyword |and| in between every author. As we can see from another example:


\section{The natbib package}

Using the standard \latex bibliography support, you will see that each reference is numbered and each citation corresponds to the numbers. The numeric style of citation is quite common in scientific writing. In other disciplines, the author-year style, e.g., (Roberts, 2003), such as Harvard is preferred, and is in fact becoming increasingly common within scientific publications. A discussion about which is best will not occur here, but a possible way to get such an output is by the natbib package. In fact, it can supersede LaTeX's own citation commands, as |natbib| allows the user to easily switch between Harvard or numeric \docpkg{natbib}\citep{natbib2009}.


The first job is to add the following to your preamble:

\begin{verbatim}
\usepackage{natbib}
\end{verbatim}


The bibliography |.bib| file is still typed using the normal format as for example:---

\begin{verbatim}
@book{goossens93,
    author    = "Michel Goossens and Frank Mittlebach and Alexander Samarin",
    title     = "The LaTeX Companion",
    year      = "1993",
    publisher = "Addison-Wesley",
    address   = "Reading, Massachusetts"
}
\end{verbatim}



Also, you need to change the bibliography style file to be used, so edit the appropriate line at the bottom of the file so that it reads: |\bibliographystyle{plainnat}|. Once done, it is basically a matter of altering the existing \texttt{cite} commands to display the type of citation you want.


The main commands simply add a (t)  for 'textual' or (p) for 'parenthesized', to the basic \cs{cite} command. You will also notice how Natbib by default will compress references with three or more authors to the more concise 1st surname et al version. By adding an asterisk (*), you can override this default and list all authors associated with that citation. There are some other less common commands that Natbib supports, listed in the table here.

Using |natbib|, can satisfy every style required by a stern and difficult editor.

\begin{table}
\begin{tabular}{ll}
\toprule
Citation command	&Output\\
\midrule
\verb+ \citet{goossens93}+	&\citep{goossens93}\\
\verb+ \citep{goossens93}+	&\citep{goossens93}\\
\verb+ \citet*{goossens93}+	&\citet*{goossens93}\\
\verb+ \citep*{goossens93}+	&\citep*{goossens93}\\
\verb+ \citeauthor{goossens93}+	&\citeauthor{goossens93} \\
\verb+ \citeauthor*{goossens93}+	&\citeauthor*{goossens93}\\
\verb+ \citeyear{goossens93}+	&\citeyear{goossens93}\\
\verb+ \citeyearpar{goossens93}+	&\citeyearpar{goossens93}\\
\verb+ \citealt{goossens93}+	&\citealt{goossens93}\\
\verb+ \citealp{goossens93}+	&\citealp{goossens93}\\
\bottomrule
\end{tabular}
\caption{Natbib package commands}
\end{table}

When changing the bibliography style, sometimes natbib is upset because it can't interpret the data correctly.

In any case, after changing the argument to |\bibliographystyle| a run of LaTeX and one of BibTeX are necessary to get back in sync. Removing the |.bbl| and |.aux| files before those run is recommended, in order to avoid spurious error messages that might corrupt the .aux file currently being generated.\footnote{\url{http://tex.stackexchange.com/questions/54480/package-natbib-error-bibliography-not-compatible-with-author-year-citations}}

\section{Including URLs in bibliography}

As you can see, there is no field for URLs. One possibility is to include Internet addresses in howpublished field of @misc or note field of |@techreport|, |@article|,|@book|:

\begin{lstlisting}[language={[common]TeX},% 
                           alsolanguage={[LaTeX]TeX},% 
                           alsolanguage={[primitive]TeX},%
                           ]
howpublished = "\url{http://www.example.com}"
\end{lstlisting}

Note the usage of \cs{url} command to ensure proper appearance of URLs.
Another way is to use special field url and make bibliography style recognise it.

\begin{lstlisting}[language={[common]TeX},% 
                           alsolanguage={[LaTeX]TeX},% 
                           alsolanguage={[primitive]TeX},%
                           ]
URL = "http://www.example.com"
\end{lstlisting}

You need to use \texttt{usepackage{url}} in the first case or \texttt{usepackage{hyperref}} in the second case.
Styles provided by Natbib (see below) handle this field, other styles can be modified using |urlbst| program. Modifications of three standard styles (|plain|, |abbrv| and |alpha|) are provided with |urlbst|.

If you need more help about URLs in bibliography, visit FAQ of UK List of TeX.


\section{changing punctuation}

When I started using natbib I kept getting square barackets. Use
\begin{lstlisting}[language={[common]TeX},% 
                           alsolanguage={[LaTeX]TeX},% 
                           alsolanguage={[primitive]TeX},%
                           ]
    \bibpunct{(}{)}{;}{a}{,}{,}
    \bibliographystyle{plainnat}
\end{lstlisting}

\section{Error Checking}

You can check the file for errors by runing it through |bibTeX|. This will point database errors etc. 


\subsection{Entry Types}

Bibliography entries included in a .bib file are split by types. The following types are understood by virtually all |BibTeX| styles:

\subsubsection*{article}
  An article from a journal or magazine.

  Required fields: author, title, journal, year

  Optional fields: volume, number, pages, month, note, key

\emph{book}
   A book with an explicit publisher.
   Required fields: author/editor, title, publisher, year
   Optional fields: volume, series, address, edition, month, note, key

\emph{booklet}
   A work that is printed and bound, but without a named publisher or sponsoring institution.
   Required fields: title
   Optional fields: author, howpublished, address, month, year, note, key

\emph{conference}
   The same as inproceedings, included for Scribe compatibility.
   Required fields: author, title, booktitle, year
   Optional fields: editor, pages, organization, publisher, address, month, note, key

\emph{inbook}

    A part of a book, usually untitled. May be a chapter (or section or whatever) and/or a range of pages.
    Required fields: author/editor, title, chapter/pages, publisher, year
    Optional fields: volume, series, address, edition, month, note, key

\emph{incollection}

    A part of a book having its own title.
    Required fields: author, title, booktitle, year
    Optional fields: editor, pages, organization, publisher, address, month, note, key

\emph{inproceedings}

An article in a conference proceedings.
Required fields: author, title, booktitle, year
Optional fields: editor, series, pages, organization, publisher, address, month, note, key

\emph{manual}

Technical documentation.
Required fields: title
Optional fields: author, organization, address, edition, month, year, note, key

\emph{mastersthesis}

A Master's thesis.
Required fields: author, title, school, year
Optional fields: address, month, note, key

\emph{misc}

For use when nothing else fits.

Required fields: none
Optional fields: author, title, howpublished, month, year, note, key

\emph{phdthesis}

A Ph.D. thesis.

Required fields: |author|, |title|, |school|, |year|\\
Optional fields: |address|, |month|, |note|, |key|
proceedings
The proceedings of a conference.
Required fields: title, year
Optional fields: editor, publisher, organization, address, month, note, key
techreport
A report published by a school or other institution, usually numbered within a series.
Required fields: author, title, institution, year
Optional fields: type, number, address, month, note, key
unpublished
A document having an author and title, but not formally published.
Required fields: author, title, note
Optional fields: month, year, key

\section{The bibentry package}

 This package allows one to be able to place bibliographic entries anywhere
 in the text. It is to be used to produce annotated bibliographies, such as
 \begin{quote}
   For an intoduction to this topic, see Jones, J.~R., Basics on this topic,
   {\it J.\ Last Resorts}, \textbf{13}, 234--254, 1994. For more advanced
   information, see \dots.
 \end{quote}

 The idea is that the full reference is used, not just the citation Jones
 [1994].

 \section{Invoking the Package}
 The macros in this package are included in the main document
 with the |\usepackage| command of \LaTeXe,
 \begin{quote}
 |\documentclass[..]{...}|\\
 |\usepackage{|\texttt{\filename}|}|
 \end{quote}

 \section{Usage}

 \newcommand\btx{\textsc{Bib}\TeX}
 This package must be used with \btx, not with a hand-written
 \texttt{thebibliography} environment.

 More precisely, there must be a \texttt{.bbl} file external to the \LaTeX\
 file; whether this is written by hand or by BibTeX is unimportant.

| \nobibliography|
 The bibliography entries are stored with the command
 |\nobibliography| |\marg{bibfiles}|, which is like the usual
 |\bibliography| |\marg{bibfiles}| except no bibliography is printed. The
 \texttt{.bbl} file is read in as usual but the \texttt{thebibliography} is
 redefined so that all the entries are stored, not printed.


 The text of the entries may be printed with the command
 \begin{quote}
    |\bibentry| |\marg{key}|
 \end{quote}

 These commands may only be issued after |\nobibliography|, for otherwise
 the reference texts are not known.

 The final period of the original text will be missing, so that one can add
 punctuation as one pleases.

 Regular |\cite| (or the \texttt{natbib} versions) may be issued anywhere as
 usual.

|\nobibliography*|
 If a regular list of references is to be given too, with the
 |\bibliography|\sidenote{bibfiles} command, issue the starred version
 |\nobibliography*| (without argument) in order to store the bib entry texts.
 This will load the same \texttt{.bbl} file as |\bibliography|, but will avoid
 messages from BibTeX about multiple |\bibdata| commands and warnings from
 \LaTeX\ about multiply defined citations.

 The processing procedure is as usual:
 \begin{enumerate}
  \item \LaTeX\ the file;
  \item Run \btx;
  \item \LaTeX\ the file twice.
 \end{enumerate}

 \noindent
 \textbf{Note:} it is highly recommended to make use of the \docpkg{url}
 package, which will nicely format both |url| and |doi| addresses; in particular,
 they will break at convenient locations without a hyphen.\index{bibliography>doi}
\index{bibliography>url}




Here are some useful references about \LaTeX. They are
available in every worthy bookshop. Many other good documentations
might be found on the web (the FAQ of \textsf{comp.text.tex} for
instance).


\begin{verbatim}
\bibitem[GMS93]{companion} Michel Goossens, Franck Mittelbach and Alexander
Samarin, \emph{The \LaTeX{} Companion}, Addison Wesley, 1993.
\bibitem[Lam97]{lamport} Leslie Lamport, \emph{\LaTeX: A Document Preparation
System}, Addison Wesley, 1997.
\end{verbatim}

This is the main matter of the document, mentioning
[\ref{doc1}] and [\ref{doc2}], for instance.



  \parskip1ex 
\newfontfamily\tibetan{TibMachUni.ttf}
\def\deva{{\protect\tibetan\symbol{"0F7C} xx ༃}}

\chapter{Indices}
\pagebreak

\starttemplate{kroll}
\thispagestyle{empty}
    \begin{leftcolumn}
       \begin{center} 
          \huge \noindent PREPARING\\
                   INDEXES
       \end{center}
     
      \medskip

       {\justifying \small\noindent And in such indexes, although small pricks\\
To their subsequent volumes, there is seen\\
The baby figure of the giant mass\\
Of things to come at large. \par
\hfill \textit{--Shakespeare}\par
\hfill\hfill{ \RaggedRight from \textit{Troilus and Cressida}}}
\medskip
       \putimage[width=1.0\linewidth]{./images/animalium01.jpg}\par
       \aheader{Handwritten cards compiled by  Sherborn for his publication \textit{Index Animalium}}
   \end{leftcolumn}
   \begin{rightcolumn}
       \putimage[width=\linewidth]{./images/animalium.jpg}
       \onelinecaption{{\resizebox{\linewidth}{5.5pt}{\bfseries Sherborn’s `Index  Animalium'. Credit Natural History Museum, London\footnote{This monumental publication became the basis for all zoological nomenclature work having gathered together all the relevant data in one place, just as an online database does today.}}}\par}
%  \centerline{\onelineheader{TYPESETTING AN INDEX}}
      \begin{multicols}{2}
        
\parindent1em      \lettrine{T}{he} first English Language index, appeared in Christopher Marlowe's \textit{Hero and Leander} in 1593. At that period, as often as not, by an ``index to a book'' was meant what we should now call a table of contents. Among the first indexes---in the modern sense---to a book in the English language was one in Plutarch's Parallel Lives, in Sir Thomas North's 1595 translation\footnote{Borko, Harold \& Bernier, Charles L. (1978). \textit{Indexing Concepts and Methods}, ISBN 0-12-118660-1.}.  

A section entitled ``An Alphabetical Table of the most material contents of the whole book'' may be found in Henry Scobell's Acts and Ordinances of Parliament of 1658. This section comes after ``An index of the general titles comprised in the ensuing Table''. Both of these indexes predate the index to Alexander Cruden's Concordance (1737) see \citep{farrow96}, which is erroneously held to be the earliest index found in an English book.

      \end{multicols}
   \end{rightcolumn}
\stoptemplate

\pagestyle{headings}


\index{Quantum Mechanics>History|(}
\setlength{\columnsep}{2em}
\begin{multicols}{2}

\section{Preparing an index}

In order to produce an index, we need to load the
package \pkg{makeidx}  and immediately issue the command \cmd{\makeindex}.\footnote{You will also need to at least run the file once using |MakeIndex| on |MikTex|. Check your distribution if you getting problems.}
At the place where
we want the index to be printed,we use \cmd{\printindex}.
\parindent1em

In \latex, we include a word
in the index by using the command \cmd{\index}\meta{arg}, so if the word Kroll should be included in
the index, we should use the command |\index{Kroll}|.

If the word is to be printed in bold, we use

% : = |
% = = @
% \catcode `|
\bgroup
 \catcode`|=11
\gdef\idxmain#1{%
   \def\idxmainentryi##1##2##3;{%
      \index{Kr0=\textbf{#1}|textbf}
    }
   \idxmainentryi#1;    
}  


\egroup

\idxmain{Kroll}

\index{Kroll=\textbf{Kroll}>Leon Kroll}


\emphasis{usepackage,makeidx,makeindex,index,printindex}
\begin{teXXX}
\documentclass{article}
\usepackage{makeidx}
\makeindex
\begin{document}
   To prepare an index, just include the
   package\index{package}

\printindex
\end{document}
\end{teXXX}


There is a  special character |\@| is used to denote that what appears on its left side must be
typeset as it appears on its right side.  the first occurrence of perl will also be used
by the sorting algorithm. is is very useful since what is used for sorting and what
will be printed may be different! For example,we may want to have the name 
\texttt{Donald Knuth}  under the letter K., we should write


\verb+\index{Knuth@{Donald Knuth}}+

Another thing we may want to change is the way that the page number is typeset.
If we want, for example, to have the page number in bold, we would write
\verb+\index{perl|textbf}+.  Notice that we wrote "textbf"  without the backslash. Of course,
the above can be combined with the  command


\verb+\index{perl@\textbf{perl}|textit}+


will print the word perl in the index (the entry will be typeset in boldface type) sorted
as "perl",’ and its page number will be italic. A common application of this is through
the command . If we want to send the reader to another index entry, say, to send
the reader from the \verb+$\omega$+ to the \verb+$\Omega$+ command, we can write

\verb+\index{omega@$\omega$|see{$\Omega$}}+


Here, we ask for the entry to be sorted according to the word omega and, in its place,
the program must use 

\begin{teX}
$\omega$|see{$\Omega$}
\end{teX}

If a word is used repeatedly in a range of pages and we want to have this range
in the index, we do not write the relative \texttt{index} command all of the time. Instead,
we write \verb+\index\{convex\|(\}+  at the place where we have the first occurrence and
\verb+\index\{convex|)\}+  at the place where we have the last occurrence. This will produce a
page range in the index for the word "convex".

\subsection{Subindices}
Subindices can be produced using an exclamation mark. If we want the word `Zeus'
to appear in the category of Greek which is in the category of Gods, we will write:

\verb+\index{Gods!Greek!Zeus}+

The actual symbol used will depend on the |.ist| file used when the file is compiled. For example in the |ltxdoc| class this is redefined as |>|.
\DeclareRobustCommand\textat{%
  \bgroup\makeatother \egroup
}


\robustify{\ttdefault}
\robustify{\ttfamily}
\robustify{\color}

 \makeatletter
\def\IDX#1#2{%
   \def\xx{\expandafter\@gobble\string#2}
   \index{\bgroup\small\texttt{\textbackslash #1}>\small\texttt{\textbackslash \xx}\egroup}
}
 
 \makeatother

\IDX{chapter}{\@introduction}
\IDX{chapter}{\intro@duction}
\IDX{abr}{\@some@other}

\meaning\ttfamily \\
\meaning\ttdefault\\

\section{Multiple Pages}

To perform multi-page indexing, add a |( and |) to the end of the \cmd{\index} command, as in 
\index{Indexing>multi-page}.

{\small
\verb+\index{Quantum Mechanics!History|(}+

\narrower\narrower
In 1901, Max Planck released his theory of radiation dependant 
on quantized energy. While this explained the ultraviolet catastrophe
 in the spectrum of blackbody radiation, this had far larger consequences 
as the beginnings of quantum mechanics.\ldots

\verb+\index{Quantum Mechanics!History|)}+
}

\end{multicols}

\index{Quantum Mechanics>History|)}


\section{Summary of commands}

\begin{tabular}{lll}
\toprule
Example	&Index Entry	&Comment\\
\midrule
\textbackslash index\{hello\}	          &hello, 1	&Plain entry\\
\textbackslash index\{hello!Peter\}	      &Peter, 3	&Subentry under 'hello'\\
\textbackslash index\{Sam@\textbackslash textsl\{Sam\}\}	&Sam, 2	&Formatted entry\\
\textbackslash index\{Lin@\textbackslash textbf\{Lin\}\}	&\textbf{Lin}, 7	&Same as above\\
\textbackslash index\{Jennytextbf\}	     &Jenny, 3	&Formatted page number\\
\textbackslash index\{Joe textit\}	&Joe, 5	          &Same as above\\
\textbackslash index\{ecole@\'ecole\}	&école, 4	&Handling of accents\\
\textbackslash index\{Peter see\{hello\}\}	&Peter, see hello	&Cross-references\\
\textbackslash index\{Jen see also\{Jenny\}\}	&Jen, see also Jenny	 &Same as above\\
\bottomrule
\end{tabular}

\section{Indexing Class Documentation}

\index{Indexing=\textbf{Indexing}}
\index{Indexing>general}
\index{Indexing>doc}

Indexing \latex2e classes is normally achieved using the |doc| and |docstrip| program, which they are not so clear with examples, if you need to deviate from the standard methods. The important thing here to remember is that you need to use different characters |=| |>| |*|.

\begin{tabular}{ll}
\toprule
normal    & doc \\
\midrule
\string @ & \texttt{=} \\
\string ! & \texttt{>}\\
\bottomrule
\end{tabular}

\begin{verbatim}
\index{Indexing=\textbf}
\index{Indexing>general}
\index{Indexing>doc}
\end{verbatim}

This manual, was build using a large |ltxdoc| class and these problems appeared while I was developing it. As normal with such problems, they were very time consuming to debug. There are still issues in some parts and one day, I am hoping to come back and correct them. One needs at this point to query the need to use the |doc| and |docstrip| method of documenting macros and if it shouldn't have a pre-processor written in a higher language to ease development. 

\section{Customization}\index{Indexing>customizing}

When creating an index with \pkgname{makeindex} one can create a \docfile{sample.ist} file that can be used together with the |makeidx| program to customize the way the index will look.

\begin{verbatim}
heading_prefix "{\\bfseries\\hfil "
heading_suffix "\\hfil}\\nopagebreak\n"
headings_flag 1
delim_0 "\\dotfill"
delim_1 "\\dotfill"
delim_2 "\\dotfill"
\end{verbatim}

This will write the first alphabet symbol in bold font, and uses dots as delimiters. This file is generally  used jointly with \texttt{makeindex} using

\verb|makeindex -s sample.ist filename.idx|

where |filename.idx| has been craeted by executing |latex| or one of the other engine commands such as |pdflatex| on |filename.tex|.


According to \citep{gabora}, you may use

\begin{verbatim}
sort_rule "\." "\b\."
sort_rule "\:" "\b\:"
sort_rule "\," "\b\,"
\end{verbatim}


\section{Writing custom indexing commands}

For complex documents it is easier to write a number of macros to assist with indexing and to also provide consistency. For example if you want to index the Devanagari alphabet we might need to get quite creative as to how to both index it as well as get the symbols in the index.
\DeclareRobustCommand\ta{{\tibetan ༃ }}

%\begin{texexample}{Writing Indexing Commands}{ex:zs}
%\gdef\ZZs#1{\incsyms%
%   \indexcommand[\string#1]{#1}%
%   #1}
%\DeclareRobustCommand\ta{{\tibetan ༃}\xparse}
%\ta
%\end{texexample}
%
%The command |\ZZs{\ta}| typesets the command in the document, as (\ZZs{\ta}) and also adds it to the index and typesets the symbol.
%
%
%\begin{phdverbatim}
%\def\ZZs#1{\incsyms%
%   \indexcommand[\string#1]{#1}
%   \string#1}
%\end{phdverbatim}
























 }  
 


\def\ttreport{
	%\def\chaptername{Chapter}
%\makeatletter
%\cxset{style13}
\cxset{style87a/.style={
 chapter opening=any,
 name=Chapter,
 % positioning and float - inline is 0
 %  float right is 2
 number display=block,
 number float=right,
 number shape=starburst,
 chapter numbering=arabic,
 number spaceout=none,
 number font-size=huge,
 number font-weight=mdseries,
 number font-family=sffamily,
 number font-shape=upshape,
 number before=,
 number display=inline,
 number float=none,
% 
 number border-top-width=0pt,
 number border-right-width=0pt,
 number border-bottom-width=0pt,
 number border-left-width=0pt,
 number border-width=0pt,
%  
 number padding-left=0em,
 number padding-right=0.5em,
 number padding-top=0em,
 number padding-bottom=0pt,
  %number margin-top=, to do
 %number margin-left=0pt,  to create
 %
 number after=,
 number dot=,
 number position=rightname,
 number color=black,
 number background-color=white,
 %chapter name
 chapter display=block,
 chapter float=left,
 chapter shape=ellipse,
 chapter color=white,
 chapter background-color=sweet,
 chapter font-size= Huge,
 chapter font-weight=mdseries,
 chapter font-family=sffamily,
% chapter font-shape=upshape,
 chapter before=,
 chapter spaceout=none,
 chapter after=,
 chapter margin left=0cm,
 chapter margin top=0pt,
 %
 chapter border-width=0pt,
 chapter border-top-width=0pt,
 chapter border-right-width=0pt,
 chapter border-bottom-width=0pt,
 chapter border-left-width=0pt,
% 
 chapter padding-left=0pt,
 chapter padding-right=0pt,
 chapter padding-top=0pt,
 chapter padding-bottom=0pt,
  %chapter title
 title font-family=sffamily,
 title font-color=black!80,
 title font-weight=bfseries,
 title font-size=huge,
 chapter title align=none,
 title margin-left=1cm,
 title margin bottom=1.3cm,
 title margin top=25pt,
 % title borders
 title border-width=0pt,
 title padding=0pt,
 title border-color=black!80,
% title border-top-color=spot!50,
% title border-top-width=20pt,
 title border-left-color=black!80,
 title border-left-width=2pt,
 title border-color=black!80,
 title padding-top=10pt,
 title padding-bottom=10pt,
 title padding-left=10pt,
 title padding-right=0pt,
% title border-right-color=spot!50,
% title border-right-width=20pt,
% title border-bottom-color=spot!50,
% title border-bottom-width=20pt,
 %
 chapter title align=left,
 chapter title text-align=left,
 chapter title width=0.8\textwidth,
 title before=0pt,
 title after=,
 title display=block,
 title beforeskip=,
 title afterskip=,
 author block=false,
 section font-family=rmfamily,
 section font-size=LARGE,
 section font-weight=bfseries,
 section indent=0pt,
 epigraph width=\dimexpr(\textwidth-2cm)\relax,
 epigraph align=center,
 epigraph text align=center,
 section color=spot!50,
 section font-weight=bfseries,
 section align=left,
 section number after=\hskip10pt,
 section font-family=sffamily,
 section numbering prefix=\@arabic\c@chapter.,
 epigraph rule width=0pt,
 header style=plain}}
 \makeatother
 
\cxset{style87a}


\cxset{ 
           %chapter toc=true,
           chapter numbering=arabic,
           chapter number color=black,
           chapter number font-shape=upshape,
           subsubsection numbering=none,
           subsubsection font-family=itshape,
           subsubsection color=black,
           subsection number after=\quad,
          section number after=\quad,
          section color=black,
    }
\def\thesubsubsection{}         
%\pagenumbering{gobble}
\def\JV{HLS DSE-JV\xspace}
\def\letter#1{\texttt{HLSDSEJV/HC/L/YL/#1}\xspace}
\def\KA{K\&A}
\def\DT#1{HLG Transmittal Ref. No.: \texttt{HLG-626-DT-HLS-#1}\xspace}
\def\idxbusbar#1{\index{Busbar Delays>#1}}
\def\idxwestin#1{\index{Westin Delays>#1}}
\def\idxstregis#1{\index{St. Regis Delays>#1}}
\def\idxahu#1{\index{Air Handling Unit Delays>#1}}
\let\idxahus\idxahu
\def\CAR#1{\index{Cost Adjustment Requests>CAR-#1}{\texttt{CAR-#1}}\xspace}
\def\idxbasement#1{\index{Basement delays>#1}}
\let\basement\idxbasement
\def\idxdewa#1{\index{Dewa Approvals>#1}}


\mainmatter
\pagestyle{plain}
\cxset{chapter name=,
          chapter numbering=none}
\chapter{Executive Summary}
\thispagestyle{empty}

This short report provides background information related to  the Habtoor City Project MEP works and the steps taken by the \JV to accelerate the works, under the instructions of the Client, Engineer and Main Contractor.  We mobilized to the Project late August 2013. At the time construction was on-going, with the basements structures mostly completed. On mobilization the only K\&A MEP designs available were those provided with the tender package---which was issued in March~2013. Besides procurement and some engineering activities, the \JV  construction activities were mainly focused on builder's works and remaining underground services until March 2014. 

We started receiving design drawings in March and April 2014. The design was issued piecemeal and in out of sequence fashion for the works to progress as planned and according to the agreed Baseline Program . This enabled us to proceed with works only in the Car Parking Areas of the Basements.  The first partially workable set of design drawings received to enable construction in other areas were the drawings received in September 2014 (Mechanical) and December 2014 (Electrical).


\medskip
		
\paragraph{Delayed Incomplete and Unworkable MEP Designs} The general issue of drawings in September~14, provided general design concepts without concerns for physical plant and ceiling constraints. The Plant rooms at T1 and PD6, as designed were not constructible, as the allocated headroom and space was inadequate. We assisted the Engineer by providing 3D and other drawings to at least fit the equipment in the available space. Fans had to be relocated in ceilings at Podium 1, and ducting was re-routed over the same ceiling void. This delayed finalization of Shop Drawings for essentially all the public areas.


The K\&A \enquote{design} mechanical design for St. Regis was only partially completed in September 2014. This design was deficient in many respects, especially in areas such the Technical floors, and as it stood the design was not constructible. This design was inadequate to close equipment orders for long delivery plant, such as AHU, fans and pumps, as calculations for static pressures could not progress. However, we took the initiative to finalize orders based on estimates and released orders before design finalization. We also assisted the Engineer with solving many of the design issues in order to progress with the works.  In addition the Electrical works suffered because of the designs issued in September 2014, as they have not been co-ordinated with the requirements of the Mechanical plant, Kitchen Contractors etc. \par

The delays  to the completion of the final Project requirements are still on-going with many areas of the Hotels still under design development and without related subcontractors appointed on time.

\begin{table}[ht]
\centering

\begin{tabular}{l l p{3cm}  l l}
\toprule
        &Area         &\raggedright Design required as per baseline program & Design Issued & Delay\\
\midrule        
\inc  &First Floor &16 Apr 14  &5 Jan 15  &  8 months\\
\inc  &Attic Floor & 8 Apr 14  &5 Jan 15  & 8 months \\
\inc  &Podium 6  &1 Apr 14   &6 Sep 14  & 5 months \\
\inc  &Podium 5  &29 Mar 14 &6 Sep 14  & 5 months\\
\inc &Podium 4   &20 Mar 14 &6 Sep 14  & 5.5 months\\
\inc &Podium 3  &12 Mar 14  &6 Sep 14  & 5.5 months\\
\inc &Technical 1 &6 Feb 14   &6Sep 14    &7 months\\
\inc &Mezzanine &26 Dec 13  &6 Sep 14  &9 months\\ 
\bottomrule
\end{tabular}
\caption{Design delays for St. Regis}

\end{table}

The MEP Good for Engineering Designs as received from K\&A enabled part of the Engineering and Procurement activities to start bu the design as it stood was  proceed, they are not sufficient to install MEP services. Drawings from ID Consultants, Lighting Consultants, Kitchen Consultant, ELV Consultants and subcontractor Shop Drawings for the same are necessary. These were mostly unavailable.

\paragraph{Instruction to accelerate the works}
Under this background we received the instruction to  accelerate the works (July 2014). We wrote to to the Main Contractor, requesting that a plan be first agreed as to how program recovery could be achieved and \emph{then} agree to a plan to accelerate the works further, so as to bring the Contract Completion dates forward. The request was to accelerate the St. Regis Hotel first with a Target Completion date of 30 March 2014.

At the time approximately 40\% of the slabs  for St. Regis were incomplete. This included critical plant areas at the two technical floors. Not only the structure had to be completed, but also the technical floor, had to have floating floors casted. The T1 floor was partially handed over to us end October and the PD6 floor in January 2015. As is also evident from the subsequently issued Design MEP Drawings, ID Drawings, Lighting Consultant and ELV Consultant drawings issued, the Professional Team was not ready with their Designs. 

As MEP works are closely interlinked with other trades it is important to note that the Structure Cabling, Kitchen Subcontractors, AV and CCTV Subcontractors were not appointed. 
\medskip

 
   


\label{acceleration}
\index{acceleration>manpower}\index{manpower>acceleration}
\paragraph{JV actions taken to accelerate the works.} Once the information started flowing, we reinforced our Engineering and Site Teams. We also added technicians as areas opened to us for work.
\medskip

\noindent\textit{Workforce}
\medskip

\noindent The \JV upon receipt of the instructions to accelerate, and under the impression that designs and appointments of other subcontractors would be accelerated as well, doubled the workforce in July~2014 and subsequently added technicians and other staff until it is at its current level of approximately 3000 personnel. The deployment of personnel is shown in the table below.

\begin{table}[hbp]
\begin{tabular}{c c c c c c c c c}
\toprule
Item &Sep 13 &Feb 14 &Mar 14 & Jul-14 & Aug-14 &Oct-14 & Jan-15 & Mar-15\\
\midrule
 Site Labour   & 48      &610      & 634     & 1212   &  1300     & 1845   &2 781   & 2 731 \\
\bottomrule
\end{tabular}
\end{table}

Although issues prohibited us from fully handing over areas and ceiling closures, the quantum of the work achieved in this short time can be gauged from the gross claimed amount of close to AED~280,000,000.00. (April~14-April~15). 
\medskip

\noindent\textit{Air-freighting of equipment}
\medskip

\noindent In addition to adding personnel we proceeded to air-freight the following equipment, without which the program recovery would have failed:

\begin{enumerate}
\item Chilled water pumps. The chilled water pumps were necessary to be delivered as early as possible in order to enable piping to be connected and for providing wild air as possible. The first submittal for pumps was made on the 25 February 2014. This was returned on the 26 March 2014. The pumps were again resubmitted in 23 April 2014, after revisions to match changes in equipment. They were returned after 40 days, despite the fact that at the time the Engineer was asking us to accelerate the works. Third and fourth submittals followed and the pumps finally approved on 3 July 2014. Pump heads were reverified to meet new layouts and the order place in August, after opening LCs and finalizing prices with Supplier. Cost AED 50,000.00. 
\index{airfreight>chilled water pumps}
\index{chilled water pumps>air freight costs}

\item First fan coil units deliveries for St Regis. These were subjected to similar delays and 388 fan coil units were air-freighted from Thailand at a cost of 196,539.60~AED. 

\item Air Separator for the St Regis Plantroom was air-freighted at a cost of AED~14,185.00.
\item All fans for St Regis. Many of these fans were to be installed in ceiling voids. These were air-freighted at a cost of AED~221,772.00. This also included air-freighting charges for fire rated motors to be air-freighted from Brazil to the Nuaire Wales factory.

\item The above secured the St Regis Hotel plant room areas.

\item Air-freighting of ECUs and Basement fans was stopped after Client Representative wrote us a letter that they would not consider paying for the above costs. \index{Ecology Units>air freight} \index{air freight costs}

These were sea-freighted, with a consequent further compression in the program of works and delaying completion of the following areas:

\begin{enumerate}
\item Basement areas
\item PD6 St Regis Plantroom
\item Kitchens
\end{enumerate}
\end{enumerate}


These are also expected to delay commissioning of kitchen areas in the basement and the Car Park Ventilation System.

\paragraph{Focus of the claim}

The claim should focus on the following:

\begin{enumerate}
\item Establishing the time extension claim. This should not be too difficult given the delays in design information. Also the casting on all buildings had considerable delays (recorded in the weekly  reports). The biggest delays in casting occured in the "W" hotel. Considerable delays in the issue of provisional sums information was another source of delay. These are listed in the last section of this report. Engineering has all the details as to when the information was released to us. I have recorded the dates we required the information in order not to incur delays.

\item Establishing disruption. Unless this element can be argued successfully, the monetary claim will be insignificant. We should at least try and recover 20\% on the labour component.

\item Establishing acceleration. The Site Team needs to provide the Claim Consultant's an accurate status of the Project and details as to what is still incomplete. Please give attention to teh fact that there are two  non-binding contractual milestones. the completion of basement works and teh completion of the St. Regis Hotel by July 2015. These milestones need to be achieved in order to validate the acceleration part of the claim.
\end{enumerate}

Between the above three components we should be able to Claim in excess of  AED 30 million, which hopefully would recover the higher labour costs incurred.

The Sections that follow are general outlines and an incomplete list of what can be claimed. 
Full information is available with the Engineering Team and the Commercial Team.

\setcounter{chapter}{0}





%\chapter{Summary MEP Progress Report for St Regis Hotel, Habtoor City}
%\pagenumbering{arabic}
%\thispagestyle{plain}
%\section{Current Status}
%
%We have started flushing of the Chilled Water system on the 7 April 2015, as planned and we anticipate to be in a position to progressively provide \emph{wild air} before the end of April, ahead of the scheduled date of the 7 May 2015. In the Basements and in the Guest rooms we have started final fix works, where possible. The Main Plantrooms at Technical Floors 1 and Podium 6, are in the main completed, except final ductwork connections where they impede access. BMS DDC Panels are expected to arrive by the 22 April 2015 and installation expected to be completed within 25-30 days to ensure that by end May we can provide controlled conditions.
%
%Delays have been experienced in the receipt of Electrical panels, such as DBs (delayed due to late deliveries of components by Legrand) and others that were subjected to numerous changes, as described later on.  Other delays were due to late instructions as briefly detailed in Section~\ref{delays}. 
%
%The current outstanding works for the St Regis Hotel are as follows:
%
%\subsection{St Regis Basement}
%
%\begin{description}
%\item[Kitchen Corridors] Some kitchen corridors cable pulling is still under progress. Expected to complete by 30 Apr 2015.
%\item[Main Electrical Room] Delays experienced due to the failure of cable trays during cable pulling and also due to the some of the MDBs being returned to the factory for modifications, as they failed QA/QC Inspections.
%\item[Fan Rooms] Fans scheduled to be delivered 23 Apr 2015.
%\item[BMS] DDC Panels still to be delivered.
%\item[Sump Pumps] Expected to be delivered by 10 May 2015. 
%\item[Others] There are still closure related works, for areas currently inaccessible, such as the new ramp areas, store and office areas. 
%\end{description}
%
%\subsection{Ground Floor}
%\begin{description}
%\item[Ballroom] This area is still under scaffolding being used by the Main Contractor to erect walk-ways in the ceiling. Once the scaffolding is dropped and we are given access to the lower level, we have to install another layer of services, give ceiling grid clearances and upon construction of the ceiling grid we can then install final sprinkler droppers and give clearances for final boarding.
%\item[Banquet Hall] This area has been delayed due to the Iridium Spa delays in Design and appointment of subcontractors. As this area is above the Banquet Hall, coring for drainage pipes delayed the works. This coring is now complete and we expect to ask the Main Contractor to lower the scaffolding and start with the rest of the services.
%\end{description}
%\subsection{Mezzanine}
%\begin{description}
%\item[Festival Dining Restaurant] Currently this area is under nomination, there is no ID Design and final details are still awaited. 
%\item[Security Room] The design for this room has recently changed. The room as shown in the new designs is different from what has been constructed on site and has no space for CCUs. 
%\item[AV Room] Expected to be completed 30 Apr 2015.
%\item[Furniture Store] Expected to be completed 25 Apr 2015.
%\item[Balance Corridors] Expected to be completed 25 Apr 2015.
%\end{description}
%
%\subsection{Podium 1}
%
%\begin{description}
%\item[Banquet A/V Technician] We have no access. This is currently being used as a store.
%\item[Service Corridor] Plan to release for ceiling grid on 23 Apr 2015.
%\item[St Regis Main Kitchen and Corridor] Plan to release on 30 Apr 2015.
%\item[Property Store] Currently no access. If access provided we can release by 30 Apr 2015.
%\item[Steak House Kitchen] Plan to release by 30 Apr 2014.
%\end{description}
%
%\subsection{Podium 2}
%\begin{description}
%\item[Iridium Spa and related areas] We are currently working in the area, which was delayed by late appointment of Finishing Contractor. Still some ID Shop Drawings not available. We expect to catch-up with delays by end May 2015. We plan to complete final fix by 10 June 2015 and Testing and Commissioning by 20 Jul 2015.
%\item[Other Areas] All other areas will be released for closure by 26 Apr 2015.
%\end{description}
%
%\subsection{Podium 3-6}
%
%All guestrooms have been handed over for ceiling closures with the exception of some of the suites, where information and access was provided late. These are the following:
%
%\begin{description}
%\item[Ambassador Suite] Co-ordination ongoing. Expect resolution and final clearances 25 May 2015.
%\item[Bentley Suite] Incomplete information. Completion targets uncertain at this stage.
%\item[Royal Suite] Co-ordination on-going. Expect resolution and final clearances 25 May 2015.
%\end{description}
%
%\subsection{Floor 1}
%
%\begin{description}
%\item[Kitchen 4 and Kitchen 6] Works for walls are progressing, insufficient detail information. Can complete by 15 May 2015, provided all Kitchen Subcontactor’s drawings become available and unimpeded access.
%\end{description}
%
%\section{Delays in Target Dates}
%\label{delays}
%This is a brief summary of recent selected instructions for additional works that have impacted  MEP Progress. 
%In addition to these additional works another critical factor that affected progress was the congestion of services and the numerous RFIs and responses we had to raise in order to resolve them.
%
%\begin{itemize}
%\item Relocation of Kitchen Extract ducting Ground Floor, Mezzanine and Podium BOH areas.
%\item  Additional AV points in all public areas.
%\item  Additional telephone, data and CCTV points in all Public Areas.
%\item  Motorized curtains Meeting Rooms.
%\item Lighting Control System. 
%\item Emergency Lighting System. (see details Chapter~\ref{emergencylights})
%\item Changes to Electrical DBs, SMDBs due to late receipt of DEWA approved drawings. (See Chapter~\ref{electrical})
%\end{itemize}
%
%We have reacted as fast as possible to all instructions and as soon they were received we have added resources to mitigate delays. Where days slipped these are only by a few days and we are confident that by end of this month all physical installations will be completed with the exception of the English Pub, Banquet and Royal Suite. 
%
%\subsection{Back of the House Areas}
%
%All back of the House Areas experienced delays, due to the lack of primary co-ordination at design stage. This caused delays until solutions were found enabling us to install the services. 
%
%The allowable ceiling height in this area was impossible to be achieved and the kitchen extract duct eventually was split in two sections and distributed through two different routes in order to avoid passing it through the corridors which could not accomodate it.
%
%In addition a new roller shutter window was introduced, that made it impossible to install the fresh air ducts feeding the kitchen. After several attempts by |K&A| to find an acceptable solution the roller shutter  window was abandoned as per the instructions of the Client Representative. 
%
%\subsection{Basement Kitchen and Related Areas at B1}
%
%Please note that these areas (with the exception of the corridor) have been cleared for ceiling grid closures in most areas and the balances are as per target to close by the 15 April 2015, including additional works. The additional works were mostly for additional ELV points on walls and for which we have received drawings on the 29 March 2015. We have instituted overtime and added additional crews to complete the works as fast as possible. Most rooms in the area have been affected. 

\begin{comment}
\chapter{Busbar System}

As per the approved Baseline Program we expected to place the busbar order for all three hotels on 27 February 2014. However, HLS DSE-JV were unable to place any orders due to the events that are outlined below, with finality on all busbars only achieved in April 2015. 

\begin{enumerate}
\item On the 23 December 2013 we were requested to change the specification for some busbars via HLG transmittal Ref. No. HLG-626-DT-HLS-0628 dated 23 Decemeber 2013 \textit{Fire Resistance Bus Bar Specification}.

\item On the 25 February 2014 we were issued revised designs via tranmittal Ref. No. HLG-626-DT-HLS-0873 \textit{Revised Electrical Drawings}.

\end{enumerate}


\chapter{Generators}

\section{Generator Ventilation}

\subsection{Background}

The original tender drawings indicated the Generator Ventilation to be by means of Louvres. When such an approach is taken normally the ventilation openings are dictated by the size of the generators.


HLS DSE-JV have submitted as early as 2014 RFIs outlining concerns regarding the adequacy of the ventilation openings and sizing of Generator rooms in the basements.

On the 25 March 2015, we were instructed to proceed with the purchase of additional fans from Systemaire. We issued the order request on the ..... and the order placed on the ......  without formal approval of the amounts in order to speed up the purchase. This affected the commissioning of the generators.

\chapter{Transformer Room Ventilation}

\subsection{Background}

\subsection{Design Errors}
\end{comment}


\cxset{chapter name=Section,
          chapter numbering=arabic}
\chapter{Emergency Lighting System}
\label{emergencylights}
The Emergency Lighting System was finalized on the 22 February 2015. This is impacting on the final fix and commissioning of the Hotel’s Central Battery and Emergency Lighting System. 

\begin{enumerate}
\item As per the approved Baseline Program, we were planning to submit the Material Submission of the Emergency Lighting System by the 25 Feb 2014.
\item On the 25 Nov 2013, we raised RFI \texttt{HLS-DSE/142 JV-RFI-MEP-E028} requesting full details of the Emergency Lights as well as the capacity of the central battery system in order to proceed with Technical Submittals, design of containment system and procurement of equipment.
\item On the 12 Dec 2013 we received an insufficient reply to the above mentioned RFI. We have notified you that the repsonse was insufficient via letter \texttt{HLS DSE/JV/HLG/YL1181} dated 14 Jan 2014, clearly stating that we were unable to proceed further with the submission of the Central Battery System, until the requested information was provided. In our letter we had requested that all details such as diffuser details, base type, IP rating and lamp characteristics are provided. We have also provided details as to Civil Defence requirements.
\item The above concerns were forwarded to the Engineer by the Main Contractor on the 20 Jan 2014. The Engineer instructed us to follow the current design dawings until the completion of the Lighting Consultant’s works.

\item On 10 Feb 2014, we had responded via letter \texttt{HLS-DSE/JVHLG/YL/1227} stating that the information provided by the Engineer, as response to RFI HLS-DSE/142 MEP-E028 was inadequate to produce Shop Drawings and to proceed with material procurement or calculations.
\item On 19 March 2014, once again we responded via letter HLS DSE/JV/626/2.05/YE/nd/2609/14 dated 4 Mar 2014 stating that the inforamtion was inadequate.
\item On 16 April 2014 we sent a clear notification that the lack of information was expected to delay the works via letter \texttt{HLS DSE/JV/HC/L/YL/1322} stating that we were unable to proceed with this portion of the works.

\item On the 20 August 2014 we received via an email instructions to proceed based on a generalized scheme.
\item We raised RFI-MEP-E249 dated 21 Sep 2014, requesting more details on locations and quantities of Emergency Light Fittings. The RFI response was received on 13 Oct 2014 with the response to follow the latest issued Guest Room drawings. 
\item Engineer’s letter \texttt{DU1211/DU/L20054/14} dated 15 Sep 2014, confirmed that due to several ID Design issues the above details were no longer applicable.
\item On 30 Sep 2014 we served notices regarding additional works due to revisions of the Emergency Lighting System for all three hotels.
\item On 15 Nov 2014, we raised concerns due to late finalization of the Central Battery System for W and Westin Hotels. 
\item On the 20 Dec 2014 the we received instructions from the Engineer and Client requesting us to revert back to the original K\&A designs.
\item On the 22 Feb 2015, the Engineer instructed us to procure and install all the Front of House exit lights. We confirmed receipt of the instruction via letter \texttt{YL/1935} dated 24 Mar 2014 once all final details and samples were finalized.
\end{enumerate}



















	\chapter{Dewa Approval}
\label{ch:dewa}

As per the approved Baseline Program, we were expected to receive Dewa approved drawings on the 28th November 2013. However, HLS-DSE JV received the LV approved drawings on 15th July 2014, as per HLG transmittal reference No. HLG-626-DT-HLS-1397 dated 15th July 2014. This delayed finalization of orders and progress on site.

 In particular:
 
 \begin{enumerate}
 \item Cables cannot be ordered until such time as approved single line diagrams are available. Once these become available  Shop drawings are prepared and main panels can also be finalized.
 \item MDBs and SMDBs can be finalized and ordered.
 \item Completion of Generator Rooms.
 \item Completion of Transformer and LV Rooms.
  \end{enumerate} 
  
\section{Action by the HLS DSE-JV}
  
Given the enormous task at hand and the instructions received to accelerate the works, we added an Electrical Engineering Manager to assist the Team with the task at hand. We also added additional CAD Operators.

\section{Design Deficiencies}

The Dewa drawings were out of step with the latest revisions of other drawings in terms of architectural, HVAC, Kitchen requirements and other equipment. They also underestimated both the main power required by 2.5MW, as well as the stand-by power required, leading to revisions to the Generator Plant. The Generator Plant is handled under delays of Electrical equipment.

Normally once drawings are submitted and approved by Dewa, the design can be considered complete, however, many areas remained incomplete.

\begin{enumerate}
\item On 20th August 2014, we requested by letter HLSDSEJV/HC/L/YL/1502 to be issued officially a number of revisions we received via email correspondence for the St Regis Hotel. 

\item On 21st August 2014 we confirmed receipt of revised Electrical Drawings from Ground to First Floor via letter ref. no. HLSDSEJV/HC/L/YL/1524. (\CAR{0076})\idxdewa{21 August revisions}

\item On 1 September 2014, we issued delay notice for revised electrical drawings, received by email for St. Regis via letter ref. no.: HLSDSEJV/HC/L/YL/1533 (CAR 83).\CAR{0083}

\item On 11 September, 2014 we confirmed via letter 

\item On 8 September 2014 we received further changes to Electrical Drawings for Westin via HLG Transmittal Ref. No. HLG-626-DT-HLS-1671 dated 8 September 2014.

\item As there was uncertainty over which drawings were to be used, HLG issued us a letter from the Engineer dated 11 September 2004, confirming the following:

      \begin{enumerate}
    	\item  St. Regis Hotel - Electrical Design drawings to be followed as per 1 September 2014 issue drawings (DU/L/18451/14).
    	\item Westin Hotel - Electrical Design drawings to be followed as per the 4 September issued drawings (DU/L/18896/14).
    	\item W Hotel - Electrical Design drawings will be issued after incorporating new Restaurant and ID drawings.
	  \end{enumerate}
\end{enumerate}

\section{30 December 2014 Dewa approved drawings issue}
\label{electrical}

On the 31 December 2014 final Dewa revised drawings were issued. This incorporated further revisons to electrical panels, additional SMDBs, changes to cable sizes, breaker sizes etc. Delay notice was served via letter Ref: YL/1796 date 19/1/2015. Changes affected all areas, including basements, St Regis, Westin and W Hotels. We wrote to the Engineer with suggestions to minimize the impact via letter Ref 25 January 2015 and recording the changes. For the W \& Westin Hotel we did the same via letter ref YL/1843 dated 3 February 2015 (\CAR {0136}).\CAR{0126}

The letters remained unanswered and we issued reminder letter related to these changes via letter ref YL/1907. We also confirmed that the works ere put on hold until such time as we had received confirmation from the Engineer.

Additional works as per letter \texttt{HLG/626/2.05/YE/es/7312/15} dated 6 April 2015. These changes relate to late approval of DEWA drawings. These changes affected all the hotels.

These revisions to the electrical design obstructed us from finalizing and ordering the Electrical Panels including MDBs, MCC, SMDB and electrical cables. The final impact of these changes is described below.

\section{St. Regis}
The following changes were instructed via the above letter and were based on drawing number |EM3300|.
\begin{description}
\item[SMDB-H1-1PLBPR] The works adds outgoing cables feeding |ADD-SS-01| and for |DBP-H1-1PLBPR1|  the cable size was changed from 4c:10mm2 XLPE to 4c:16 mm2 XLPE. The breaker size was changed to 60A MCCB.

\item[SMDB-H1-1TEFCWF] The instruction requests the changing of 15A breaker to 20A for eight CP-H1-1-TEWF/05 T.C.L.-1kW and one CP-H1-1TEWF/09 T.C.L.-1kW.

\item[SMDB-H1-2PL] The instruction requests the following changes:
   \begin{enumerate}
      \item DBP-H1-2PL MCCB 60A change to 80A and cable size 4c:16mm2 XLPE change to 4c:25mm2 XLPE.
      \item BPN-PN-16 and 18 30mA ELCB added.
   \end{enumerate}

\item[SMDB-H1-2PSPA] The instruction requests the following changes:
    \begin{enumerate}
      \item Male and female Jacuzzi bath MCCB 15A change to 20A TCL-3kW.
      \item Female steam room cable size changed (4c:10mm2 XLPE to 4c:16mm2 XLPE).
    \end{enumerate}


\item[SMDB-H1-6PL] The instruction requests the following changes:
   \begin{enumerate}
      \item Additional outgoing feeders for EC-01B, EC-02B, WET-PN-011, WET-PN-017 and WET-PN-020.
      \item 40ATP MCCB removed for FP-H1-1FL
   \end{enumerate}

\end{description}

The following changes were due to drawing No:EM3301

\begin{description}
\item [MDB-H1-B1R1] The instruction requests the following changes:
    \begin{enumerate}
       \item SMDB-H1-GR2 MCCB 200A change to 225A and cable size 4c:95mm2. XLPE change to 4c: 120mm2 XLPE (TCL 116.8kW).
       \item UPS MCCB 60A change to 80A.
    \end{enumerate}
\item[MDB-H1-GR2] The following changes were instructed:
    \begin{enumerate}
       \item Incomer MCCB 200A TP change to 225A TP.
       \item Additional outgoing for WPN-PA-012, WPN-PA-032.
    \end{enumerate}
\end{description}

The following changes were due to drawing No:EM3302

\begin{description}
\item[SMDB-H1-GLSTBR] The following changes were requested:
   \begin{enumerate}
      \item Additional outgping for St Regis, Special Event, St Regis BR.
      \item DBP-H1-GLSTBR MCCB 60A change to 80A.
   \end{enumerate}
\item[SMDB-H1-1PLMK] The following changes were requested:
      \begin{enumerate}
        \item Additional space.
        \item 60A TP MCCB removed.
      \end{enumerate}  
\item[SMDB-H1-2PGSC] The following changes were requested:
     \begin{enumerate}
        \item CAF-SS-01 cable and MCCB size changed from 4c:70mm2 XLPE and 150A TP to 4c:XLPE and 30A TP (TCL-6.5kW).
     \end{enumerate}
\end{description}

The following changes were detailed on drawing No:EM3303

\begin{description}
\item[MDB-H1-B1LR2] SMDB-H1-GL MCCB80A change to 100A.
\item[SMDB-H1-GL] DBP-H1-GLPFA MCCB 60A change 80A and cable size 4c:16mm2 XLPE change 4c:25mm2 XLPE.
\item[SMDB-H1-GLBP1] Additional outgoing for BOQ-KIT-016.
\end{description}

The following changes were detailed on drawing No:EM3304.

\begin{description}
\item[EMDB-H1-B1]  The following changes were requested:
   \begin{enumerate}
      \item ESMDP-H1-GR2 MCCB 80A change to 150A and cable size 4c:35mm2 XLPE change to 4c:70mm2 XLPE.
      \item ESMDB-H1-6PMS1 MCCB 400ATP change to 500ATP.
   \end{enumerate}
\item[EMDB-H1-6PMS1] The following changes were requested:
    \begin{enumerate}
       \item Incomer MCCB 400ATP change to 500ATP.
       \item Additional outgoing for EC-01A,B and future load.
       \item ESMDB-H1-RS cable size changed from 4c:70mm2 XLP (125A TP to 4c:95mm2 XLPE (TCL-55kW).
    \end{enumerate}
\item[ESMDB-H1-GL]
\item[ESMDB-H1-6PMS2]  The incomer to MCCB was changed from 700A TP to 800A TP.
\item[ESMDB-H1-6PL] An additional outgoing cable was requested for EC-02A. For LIFT-H1-SL05 and LIFT-H1-SL06 the cable size was requested to be changed to 4c:35mm2 MGT/XLPE.
\item[ESMDB-H1-2PL] CAF-SK-012, EC-01B MCCB and cable size changed from 30A SP 2c:16mm2 XLPE to 20ASP and 2c:4mm2 PVC (T.C.L.-2.6kW and 0.8kW).
\end{description}

\subsection{St Regis Basement Areas}
The following changes were detailed on drawing No:EM3200.\idxdewa{basements}\idxbasement{SMDB revisions}
\begin{description}
\item[SMDB-BP-1BS1] Additional outgoing circuits were requested for DB-LS-SR2, DB-LS-SR3.
\item[SMDB-BP-1BS3]  An additional outgoing circuit was instructed for DB-LS-SR5.
\item[SMDB-BP-1BS5] An additional outgoing circuit was requested for DB-LS-SR6.
\end{description}

The following changes were detailed on drawing No:EM3201.

\begin{description}
\item[EMDB-BP-1B3] ESMDB-BP-1BS7 MCCB 40A change to 80A and cable size 4c:10mm2 XLPE change 4c:16mm2 XLPE (TCL-17.8kW).\idxbasement{EMDB revisions}\idxbasement{ESMDB revisions}
\item[ESMDB-BP-1BS9] cable size 4c:35mm2 XLPE change to 4c:70mm2 XLPE (TCL-44.4kW).
\item[ESMDB-BP-1B3]  The following changes affected this panel:
     \begin{enumerate}
        \item ESMDB-BP-1BS7 MCCB 40A change to 80A and cable size 4c:10mm2 XLPE change to 4c:16mm2 XLPE    (TCL-17.8kW). 
        \item ESMDB-BP-1BS9 cable size 4c:35mm2 XLPE change to 4c:70mm2 XLPE (TCL-44.4kW). 
        \item ESMDB-BP1BS10 cable size 4c:240mm2 MGT change to 4c:300mm2 MGT(TCL-120kW).
     \end{enumerate}
\item[ESMDB-BP-1BS2]  3 Nos CP-BP-1BE/F1 cable size 4c:16mm2 MGT change to 4c:25mm2 MGT (TCL-17kW).
\item[ESMDB-BP-1BS3]  20ATP, pulse meter, 10mm2 MGT removed for SPCP-BP-1B12.
\item[ESMDB-BP-1BBPA] Incomer MCCB 80A TP change to 100A TP.
\item[ESMDB-BP-1BCOM1] Additional outgoing for COM-IC-001, COM-IC-002, COM-IC-003, COM-IC-006.
\item[USMDB-BP-1BS] UDB-BP-1BS4 and UDB-BP-1BS5 cable size 4c:10mm2 XLPE change to 4c:16mm2 XLPE (TCL-8.8kW and TCL-7.6kw).
\item[ESMDB-BP-1BS1] Incomer MCCB 200A TP change to 250A TP.
\item[ESMDB-BP-1B] ESMDB-BP-1BBPA MCCB 80A change to 100A and cable size 4c:50mm2 XLPE change to 4c:70mm2 XLPE(TCL-44.8kW).
\end{description}

The following changes were due to additional works detailed on drg No: EM3204.

\begin{description}
\item[MDB-BP-2BMEC]
   \begin{enumerate}
     \item Incomer MCCB 80A TP change 100A TP.
     \item FPCP-H1-2B2 cable size 4c:10mm2 XLPE change to 4c:16mm2 XLPE.
     \item FPCP-H1-2B1 cable size 4c:6mm2 XLPE change to 4c:6mm2 XLPE change to 4c:10mm2 XLPE (TCL-5.5kW).
   \end{enumerate}
\item[SMDB-FB-2BMEC]
\end{description}

The following changes were due to drawing No: EM3206.
\begin{description}
\item[MDB-BP-1B2] 
    \begin{enumerate}
       \item MDB-BP-1BCOM Additional outgoings for COM-MP-041.
       \item SMDB-BP-1BS6 MCCB 400A change to 500A and cable size 2x4c:120mm2 XLPE change to 2x4c:150mm2 XLPE (TCL-221kW).
       \item 400A TP+2x4c:120mm2 XLPE removed for FFP-3.
    \end{enumerate}
\item[SMDB-BP-1BS6] Additional outgoing for DB-LS-SR4.
\item[SMDB-BP-1BS10] Additional works were requested as follows:
    \begin{enumerate}
      \item DB-H3-1BSS2 cable size change to 2c:10mm2 XLPE change to 2c:16mm2 XLPE (TCL-1.2kW).
      \item CP-BP-1BTE/F4 cable size change to 4c:16mm2 XLPE change to 4c:25mm2 XLPE MCCB 40A TP Change to 60A TP (TCL-25kW).
      \item CP-BP-1BTF/F2 and CP-BP-1BTE/F2 MCCB 60A TP change to 80A TP (TCL-37kW).
      \item CP-BP-1BTF/F3 and CP-BP-1BTE/F3 MCCB 40A TP change to 60A TP (TCL-22kW and 25kW).
    \end{enumerate}
\end{description}








	
\chapter{Delays in Finalizing Requirements for the Busbar System}

As per the approved Baseline Program\footnote{Issued 4 Jan 14 and approved 9 Jan 14, as per HLG letter Ref: HLG/626/2.5/SO/nd/1862/14},  we were planning to order the Busbar on 27 February 2014. The \JV was unable to finalize the Bus Bar Material Submittal due to the numerous revisions issued and the lack of Dewa approved drawings.

\begin{enumerate}
\item On 23 December 2013 we received HLG transmittal ref: no. HLG-626-DT-HLS-0628 dated 23 December 2013 ``Fire Resistance Bus Bar Specification'', instructing us to change some of the busbars to fire rated busbars.
\idxbusbar{change in specification}\idxbusbar{fire rated}
\label{fireratedbusbar}

\item HLG transmittal Ref: No.: HLG-626-DT-HLS-0797 dated 10 February 2014 titled ``Electrical Updated Coordinated Drawings for Basements". (\CAR{0004}).

\item HLG Transmittal Ref. No.: HLG-626-DT-HLS-0873 dated 25 February 2014 ``Revised Electrical Drawings''.

\item On 18 and 20 March 2014 via \DT{0930\&939} we were issued updated drawings for three Hotel (\CAR{0036}).  

\item \DT {10127} dated 10 April 2014 ``Revised Electrical Drawings''. 

\item On 16 June 2014 we were instructed to stop any works on Westin and W Hotel Bus bars due to ``significant comments on Dewa LV approval''. This instruction was received via \KA letter ref. no. DU1211/DU/L/13086. At the time we had completed LV Schematic drawings and had the bus bar isometrics completed. 

\item The approved Dewa LV drawings were received on 15 July 2014 as per \DT {1397} dated 15 July 2014. One month later than the instruction to hold the orders and the works.

\item On 20 August 2014, we requested that we be issued officially the St Regis Hotel revised drawings that we were being send piecemeal by email. \letter{1515}.

\item On 1 September 2014 we had responded to HLG letter ref. no. HLG/626/1.12.AMM/es/4109/14 expressing concerns as to delays in receiving receiving workable design drawings (\letter{1533} (\CAR{0083}).

\item We received revised Electrical Drawings for Westin via \DT {1671} dated 8 September 2014.

\item On 17 September 2014 we received Electrical Drawings for W Hotel via \DT{1728}. 

\item On 27 September we served notice for additional costs and time due to revised electrical drawings related to Mechanical Equipment. The additional loads were received as response to RFI/MEP/E295 for all three Hotels (\CAR{0083}, \CAR{0086}\& \CAR{0087}).

\item On 2 October 2014 we served notices for additional works due to revised Electrical Drawings for Food and Beverage areas. (\CAR{0093}) {\CAR{0093}}

\item We served notice for drawings issued to us for W Hotel from 24 to 27 floor (\CAR{0112}).

\item On 19 February 2015 we issued estimated costs at the Engineer's request for the fire rated bus bars. On 25 March 2015 we were instructed to change all fire rated bus bars to normal bus bars. Accordingly the procurement of this particular bus bars took from 23 December 2013 until 19 February 2015 to be concluded and delayed the works.  (See \ref{fireratedbusbar} referred to when the first instruction was received. )


\end{enumerate}
	\chapter{Delay in Finalization of Air Handling Unit, Fans and Fan Coil Units}
\label{ch:ahus}

As per the Approved Baseline Program we planned to place the HVAC equipment orders on 20 March 2014. Changes to all equipment continued well into 2015. \idxahu{general delays}

\begin{enumerate}
\item On 17 September 2013, we were issued via \DT{0320} updated Mechanical Drawings. \CAR{0002}. This was the first major drawing issue, kick-starting the Engineering and Procurement process.

\item On 10 December 2013 we received updated Mechanical Drawings which revised many of the previously issued drawings and equipment schedules. (See \CAR{0001} and \CAR{0002})

\item On 11 February 2014, via  \DT{0813} we were issued updated Mechanical Drawings for Basements which led to  abortive Engineering works and precluded the finalization of any detailed calculations and the procurement of equipment. (\CAR{0026})\basement{HVAC}

\item On 27 February 2014, we received via \DT{0878} updated Fire Fighting and HVAC drawings (28 Nos) for Basements which superceded some of the previously issued drawings (See \CAR{0031}.)\basement{HVAC}

\item On 8 March 2014 we received instructions via \DT{0898} and updated Mechanical drawings for all three hotels \CAR{0031}. 

\item On 16 March 2014 we received instructions via \DT{920} and updated Mechanical drawings for all three hotels.
\idxwestin{AHU}\idxwestin{fans}\idxwestin{fcu}.

\item On 10 April 2014 we received instructions via \DT{1027} ``Revised Mechanical Drawings'' with again more changes and revisions to the design.\idxahu{10 April 2014}

\item On 28 May 2014 we received more revisions via \DT{1194} ``Revised Mechanical Drawings for St Regis Hotel''.

\item On 13 July 2014 we received major changes to equipment schedules via email. 

\item On 4 August 2014 we issued letter \letter{1489} which was affecting the ordering of fans for the St. Regis Hotel and Basement areas.\basement{fans}

\item On 13 August 2014 \idxahus{13 Aug 2014 revisions} we received updated Mechanical Drawings again with equipment revisions. We reserved our rights via \letter {1504}. (See \CAR{0077})

\item On 1st September  2014 we received advance copies of further changes to the HVAC works. These were issued to us officially on 10 September 2014 and our confirmation followed by letter \letter{1549} and \CAR{0085}. 

\begin{enumerate}
  \item Additional AHUs were introduced.
  \item Additional fcus were introduced.
  \item Additional fans were introduced.
  \item Additional CAVs were introduced.
  \item Additional ductwork and revisions to duct sizes
  \item Additional piping.
  \item The revisions impacted on related Electrical works, Fire Alarm, Fire Detection and BMS systems. 
\end{enumerate}

The above resulted to numerous abortive works on site and voiding previous Engineering works.

\item On 25 September 2014 we reserved our rights to further revisions that we received via RFI replies. RFI/MEP/M307. \CAR{0085}\idxahus{25 Sept revisions}

\item On 2 October 2014, we confirmed additional works due to revised Mechanical Equipment Schedules. These were received via RFI/MEP/M299 for the St Regis Hotel and Basement levels \CAR{0085}.\idxbasement{equipment revisions}

\item Engineer requested us to upgrade one AHU for W Hotel and this was confirmed via letter \letter{2014}.\CAR{0085}.
\end{enumerate}

The above changes seriously impacted on the ordering, but most importantly on the finalization of critical equipment and ductwork for the whole development. In order to minimize the impact we employed a full time Senior Engineer just to handle these continuous changes and to manage the calculations for static pressures.












	

\chapter{Delays in Engineering, Procurement and Construction due to Frequent  Design Changes}

As per the approved Baseline programme the HLS-DSE JV were expecting to receive Good for Construction (GFC) drawings on 14 September 2013 for all areas. This would have ensured that the Contract dates could have been met. GFC drawings are an industry practice by which the Engineer signals the completion of the design and the avoidance of errors omissions and delays by using drawings such as those marked as GFE (Good for Engineering). 

The Consultant has been unable to finalize the design on time and drawings and designs were provided mostly reactively to requests and notices by the Contractor. This has subsequently caused disruption to the \JV Engineering, Submittals, Procurement and Work Progress activities.

\section{Design Changes}

Many design changes were as a response to the \JV RFIs. As of today more than 1300 RFIs have been issued. The events described below are more or less in reverse chronological order from the more recent to the earliest.

Design changes can in general be grouped in the following categories:

\begin{enumerate}
\item As responses to RFIs to resolve, space constraints. The Engineer's design was not coordinated with the basic architectural and structural design. This was most acute in the St Regis Hotel, where large beams and inadequate floor to ceiling height resulted in congested  areas. This was not evident during the tender process and has resulted in additional costs to the Contractor.

\item As responses to RFIs to provide further information due to lack of completeness of the design.

\item Errors and omissions, which either the Engineer corrected or as responses to RFIs.
\end{enumerate}

\begin{enumerate}
\item Variation notification for HVAC works was served under \letter{YL/1922 \& 1931} dated 15 March 2015 and 23 March 2015 (CAR 085). This variations to the works related to:
  \begin{enumerate}
	\item HVAC provisions for Electrical and Telecommunication rooms.
	\item Modifications to duct sizes in St. Regis Mezzanine Floor.
	\item Wow suite toilet exhaust requirements.
	\item Fan air flow changes.
	\item Updates AHUs serving F35 Westin Hotel.
	\item Modifications to duct sizes (W Hotel)
	\item Additional exhaust fan.
  \end{enumerate}
  
 \item Updated electrical drawings (38) were issued on 7 Mar 2015. We served notification as to the time impact in studying the changes (as there were no cloud revisions) and to advice impacts. Costs were reserved under CAR 146. Notifications were served via \letter{1913} and \letter{1920} dated 11 Mar 2015 and 15 Mar 2015.
 
 \item We raised concerns for missing design information for Kitchen equipment revised SLD/power drawings for W Hotel areas in order to proceed with Shop drawings preparation \letter{YL/1856} dated 5 Feb 2015.
 
 \item We raised variation notification due to additional HVAC works \letter{YL/1850} dated 5 Feb 2015.
       \begin{enumerate}
          \item BTU meters for the chilled water system.
          \item Seasonal Taste air distribution details.
          \item Relocation of AHUs at Basement-2
          \item AHU orientation, piping connection and duct re-arrangement.
          \item Additional motorized smoke dampers for standby smoke fans.
       \end{enumerate}

    
\item We requested Specialist subcontractor drawing for Data Centre and Security rooms  to proceed with MEP Shop drawings (affected Mezzanine St. Regis) \letter{YL/1813} dated 20 Jan 2015.\footnote{Security room finalization still pending as of 20 May 2015}. 

\item On 16 December 2014, we advised that MEP delays to the accelerated program were attributed to design delays, civil work delays in the casting of slabs, lack of primary co-ordination by \KA\ delays in appointing subcontractors affecting MEP works (kitchens, ELV works, IT and Audio-visual) and failure of ID designers \letter{YL/1741} dated 16 November 2014.
  
\item We received instructions to provide adaptors and sanitaryware accessories for all Hotels via \letter{Du/26882/14}  dated 16 Nov 2014. 

\item We recorded our concerns due to revisions to the design we received via sketches showing cross-contamination between fresh air intakes and exhaust from kitchens via letter ref: \letter{YL/1760} dated 24 December 2014.

\item Additional fire-fighting works for all hotels were confirmed via letter ref: \letter{YL/1765} dated 28 December 2014.

\item Revised RCP drawings were issued to the \JV which lead to abortive works at St. Regis Ground Floor to Level-2. This was confirmed by \letter{YL/1732} dated 4 December 2014.

\item We received instructions to install earthing system for all structured cabling Telecommunications rooms for all hotels (CAR 130) and \letter{YL/1728} dated 3 Dec 2014.

\item Additional UPS was added to serve the St Regis Hotel Data Center. This was confirmed by letter on 30 October 2014. (CAR 111).

\item Additional floor clean-outs were requested by the Engineer during inspections (CAR-116).

\item FCU type changed from decorative to ducted in W-hotel (CAR-115).

\item Changes to drainage services at the St Regis Hotel due to floor sinks passing over post tension slabs. (CAR 114).

\item Additional Control Panels in St. Regis Hotel Areas. (CAR-113).

\item Revisions to air outlets (CAR-117)

\item Revised Electrical works at St. Regis Hotel (CAR 83) submitted on 4/11/2014.

\item Additional Fans/AHUs that were instructed via RFI responses (CAR 85).

\item  Revised HVAC works  in St Regis Hotel (CAR 116).

\item Additional Electrical Works to St Regis Hotel Data Center (CAR 118)

\item Revised CCTV layouts for W \& Westin Hotel Areas (CAR 087)
  
\item Delays in the finalization of the Central Battery System for W \& Westin Hotels. We raised concerns via letter \letter{YL/1688}  dated 15 October 2014.

\item On 25 September 2014 we confirmed additional works for Drainage Service for one additional toilet at basement. (CAR 92).

\item On 25 September 2014, we confirmed additional works for water supply supply points for pantry at P4 in the St Regis Hotel.

\item On 25 September 2014, we confirmed additional works due to revisions to equipment schedules received via \letter{RF/MEP/M307}  (CAR 85).

\item On 25 September 2014 we confirmed additional works due to revised electrical drawings for Westin Hotel. (CAR 86).

\item On 1 October 2014 we confirmed instructions for additional electrical works due to missing power feeders at the Banquet Hall. received via RFI/MEP306 \& 309 for St Regis Hotel. (CAR 83).

\item On 2 October 2014, we confirmed additional works due to revisions of Mechanical Electrical loads. This was received as a reply to RFI/MEP/E295 for all three Hotels (CAR 83, 86 \&87). 

\item On 2 October 2014 we confirmed additional works due to revised Electrical drawings for Food and Beverage areas (CAR 93).

\item On 2 October 2014, we confirmed additional works due to Revised Mechanical Equipment schedules. This was received via RFI/MEP/M309 for all three Hotels (CAR 85).

\item On 8 October 2014, we were instructed by letter HLG/626/2.05/YE/nd/4327/14 attaching \KA letter Ref. No.: DU1211/DU/L/20844/14 dated 7 September 2004 confirming additional containment for the Access Control System.

\item On 13 October 2014, we confirmed additional works due to the addition of Electrical Heaters to some of the AHUs. This was received via RFI/MEP/M309 for all three hotels (CAR 94).

\item On 6 September 2014, we expressed concerns as to the constructibility and maintainability of the St Regis Technical 1 Plant room and issued proposals to minimize impacts. \letter{1538}. In our letter we requested \KA to resolve the issues latest within 7 days in order to minimize the impact on the accelerated target dates. Further updates were issued via letter \letter{1545} dated 9 September 2014 and \letter{1566} dated 17 September 2014 and \letter{1729} dated 4 December 2014.


\item On 7 September 2014, we requested details for the Access Control System via \letter{1542}.

\item On 9 September 2014, we expressed concerns as to changes to the Architectural design of the St. Regis Fire Pump Room which remained incomplete via \letter{1550} impeding installation of Fire Pumps.

\item On 10 September 2014, we confirmed receipt of revised HVAC drawing via \letter{1549} (CAR 85).

\item On 11 September 2014, we confirmed receipt of revised Electrical drawings for St. Regis Hotel via letter \letter{1555} (CAR 83).

\item On 22 September 2014, we issued notification for additional works related to the Domestic Cold Water system via \letter{1572} (CAR 89).

\item We received revised electrical drawings for Westin via HLG Transmittal Ref. No: HLG-626-DT-HLS-1671 dated 8 September 2014. (CAR 86)

\item We received revised electrical drawings for W Hotel via HLG Transmittal Ref. No.: HLG-626-DT-HLS-1728 dated 17 September 2014.
\end{enumerate}

The above do not record fully the method and lack of detail in issuing design information to the \JV. The general drawing issued to us did not contain adequate information to develop Shop Drawings. Clarifications and proposals were sent to the Engineer for missing information  via RFIs. 




	\makeatletter
\long\def\hlshadi#1{\hl{#1}}
\cxset{enumerate numberingi/.is choice,
  enumerate numberingi/.code={\renewcommand\theenumi {\csname#1\endcsname{enumi}}},
  enumerate numberingii/.code={\renewcommand\theenumii {\csname#1\endcsname{enumii}}},
  enumerate numberingiii/.code={\renewcommand\theenumiii {\csname#1\endcsname{enumiii}}},
  enumerate numberingiv/.code={\renewcommand\theenumiv {\csname#1\endcsname{enumiv}}},
  enumerate labeli punctuation/.store in=\enumeratepunctuationi@cx,
  enumerate labeli/.is choice,
  enumerate labeli/brackets/.code={\renewcommand\labelenumi{(\theenumi\enumeratepunctuationi@cx)}},
  enumerate labeli/square brackets/.code={\renewcommand\labelenumi{[\theenumi\enumeratepunctuationi@cx]}},
  enumerate labeli/right bracket/.code={\renewcommand\labelenumi{\theenumi\enumeratepunctuationi@cx)}},
  enumerate label left/.store in=\enumeratelabelleft@cx,
  enumerate label right/.code=\renewcommand\labelenumi{\enumeratelabelleft@cx\theenumi\enumeratepunctuationi@cx#1},
  enumerate leftmargini/.code={\setlength\leftmargini{#1}},
  enumerate leftmarginii/.code={\setlength\leftmarginii{#1}},
  enumerate leftmarginiii/.code={\setlength\leftmarginiii{#1}},
  enumerate leftmarginiv/.code={\setlength\leftmarginiv{#1}},
  listi topsep/.store in=\listitopsep@cx,
  listi partopsep/.store in=\listipartopsep@cx,
  listi itemsep/.store in=\listiitemsep@cx,
  listi parsep/.store in=\listiparsep@cx,
  listii topsep/.store in=\listiitopsep@cx,
  listii partopsep/.store in=\listiipartopsep@cx,
  listii itemsep/.store in=\listiiitemsep@cx,
  listii parsep/.store in=\listiiparsep@cx,
  listiii topsep/.store in=\listiiitopsep@cx,
  listiii partopsep/.store in=\listiiipartopsep@cx,
  listiii itemsep/.store in=\listiiiitemsep@cx,
  listiii parsep/.store in=\listiiiparsep@cx,
}
\cxset{compact1/.style={%
  enumerate numberingi=arabic,
  enumerate numberingii=alph,
  enumerate numberingiii=alph,
  enumerate numberingiv=roman,
  enumerate labeli punctuation=.,
  enumerate label left=,
  enumerate label right=,
  enumerate leftmargini=2.2em,
  enumerate leftmarginii=2.1em,
  enumerate leftmarginiii=1.5em,
  enumerate leftmarginiv=2em,
  listi topsep=8\p@ \@plus2\p@ \@minus\p@,
  listi itemsep=0\p@ \@plus2\p@ \@minus\p@,
  listi parsep=0\p@ \@plus2\p@ \@minus\p@,
  listii topsep=0\p@ \@plus2\p@ \@minus\p@,
  listii itemsep=0\p@ \@plus2\p@ \@minus\p@,
  listii parsep=0\p@ \@plus2\p@ \@minus\p@,
  listiii topsep=0\p@ \@plus2\p@ \@minus\p@,
  listiii itemsep=0\p@ \@plus2\p@ \@minus\p@,
  listiii parsep=0\p@ \@plus2\p@ \@minus\p@,
}}
\cxset{compact2/.style={%
  enumerate numberingi=alph,
  enumerate numberingii=roman,
  enumerate numberingiii=alph,
  enumerate numberingiv=roman,
  enumerate labeli punctuation=,
  enumerate label left=(,
  enumerate label right=),
  enumerate leftmargini=2.2em,
  enumerate leftmarginii=2.1em,
  enumerate leftmarginiii=1.5em,
  enumerate leftmarginiv=2em,
  listi topsep   = 8\p@ \@plus2\p@ \@minus\p@,
  listi itemsep = 0\p@ \@plus2\p@ \@minus\p@,
  listi parsep   = 0\p@ \@plus2\p@ \@minus\p@,
  listii topsep  = 0\p@ \@plus2\p@ \@minus\p@,
  listii itemsep= 0\p@ \@plus2\p@ \@minus\p@,
  listii parsep  = 0\p@ \@plus2\p@ \@minus\p@,
  listiii topsep = 0\p@ \@plus2\p@ \@minus\p@,
  listiii itemsep= 0\p@ \@plus2\p@ \@minus\p@,
  listiii parsep  = 0\p@ \@plus2\p@ \@minus\p@,
}}

\ExplSyntaxOn
\def\setenumerate#1{
\cxset{#1}
\def\@listi{%
           \leftmargin\leftmargini
            \parsep\listiparsep@cx
            \topsep\listitopsep@cx\relax
            \itemsep\listiitemsep@cx}
            
\def\@listii{\leftmargin\leftmarginii
            \parsep\listiiparsep@cx
            \topsep\listiitopsep@cx\relax
            \itemsep\listiiitemsep@cx}
            
\def\@listiii{\leftmargin\leftmarginiii
            \parsep\listiiiparsep@cx
            \topsep\listiiitopsep@cx\relax
            \itemsep\listiiiitemsep@cx}
}
\ExplSyntaxOff

\setenumerate{compact1}
\makeatother
\def\delay{\textcolor{red}{\Fire}}

\ExplSyntaxOn
% Holds master fields for all ODBs
\clist_new:c {DBSMASTER}

% New DBS
\clist_new:c {DBS}

%% Create the DB. The DB can have any name
%% 
\NewDocumentCommand {\CreateDB} { m }
  {
    \clist_new:c {#1}
    \clist_gput_left:cn {DBS} {#1}
  }   

% add only the number, and this only at the right,
% expecting the user to type it in ascending order and thus make sorting 
% easier
% Note the elements are stores as 1,3,4,68,112 etc.
% #1 DB name
% #2 field index - integer
% 
\NewDocumentCommand \addtoDB { m m  }
  {
    \clist_gput_right:cx { #1 } { #2 }
  }
 
%% Generate some variants
%%

\cs_generate_variant:Nn \clist_sort:Nn {cn}   

%% generalized sort DB
\NewDocumentCommand \SortDB { m }
{
  % remove any duplicates before sorting out
   \clist_remove_duplicates:c { #1 }
   % make variant here
   \clist_sort:cn {#1}
     {
       \int_compare:nNnTF { ##1 } < { ##2 }
        { \sort_reversed: }
        { \sort_ordered: }
    }
}

% #1 DB #2 Suffix
%
 \NewDocumentCommand \printRFI { m m }
  {
  % sort the list in numerical order
    \SortDB { #1 }
     
 % map and print only for category (one to many also possible here)   
     \clist_map_inline:cn { #1 }
      {  
       \cs_if_exist:cT {RFI##1-#2}
          {
            \cs:w RFI##1-#2\cs_end:
            \index {RFI~Mechanical>RFI-M-##1-#2}
          }
       
      }
  }
\ExplSyntaxOff    


 %#1 Number
%#2 Hotel
%#3 Impact
%#4 Description
%#5 Area

\newenvironment {RFI} [3] {%
  \vspace*{12pt}
  \parindent0pt
  \mbox{\bfseries\color{red!80!black}\textsf{RFI-M-#1}}
        \textbf{#2} \textbf{#3} \par
  \begin{enumerate}}
 {\end{enumerate}}

% Optional DB letters W - W hotel
% WE - Westing
% SR  - St Regis

\NewDocumentCommand {\addRFI} { O{SR} +m +m +m +m +g  +g } 
{%
   \expandafter\gdef\csname RFI#2-#1\endcsname
   {%
      \begin{RFI}{#2} {#3} {#4} 
      
        #5
      
      \end{RFI}
       
      \IfNoValueTF{ #6 }{ #6 }{ } 
       
      \IfNoValueTF{ #7 }{ #7 }{ }
    } 
    \addtoDB {MRFI} {#2}
   \par 
}

%% Create DB 
\CreateDB {MRFI}   




\chapter{Mechanical RFIs Related to St Regis}



\addRFI {497} {7 Dec 14 received 19 Jan 2015} {St Regis, Podium 1, Main Kitchen}
{
\item The RFI referred to previous response to RFI, where the Engineer directed the Contractor 
         to their comments on Shop Drawing AHC-HLS-SRH-SDM-AC-P1-0. Contractor re-iterated
         that achievable ceiling in Kitchen area could only be 2300mm and at the kitchen extract hood 2m.
\item This area was almost impossible to complete given the congestion to achieve this ceiling height.         
}


\addRFI {537} {18 Jan 2015 received 22 Jan 2015} {St Regis, Basement 1}%
 {
   \item Engineer issued instruction via drgs DU-1211-DU-00564 on 12 Jan 2014. 
    Engineer issued revised layout for AC-4020.
   \item Suggested re-arrangement blocked access to shaft 35-36/D-F.
   \item No chilled water piping and valving was not shown.
   \item Length of AHU-B2-02 is in excess of 5 meters plus a 500 mm plenum. 
 }



\addRFI{657}
{20-Apr-2015, responded 27 April 2015} {St Regis, Mezzanine, Security Room} 
{
\index{CCU>Security Room}\index{Primary co-ordination>CCTV}
\index{St Regis>Mezzanine>Security Room}

\item The layouts were not matching with latest Architectural drawings.
\item Layout not matching with HVAC Design.
        \begin{enumerate}
        \item The new layouts did not allow space for the CCUs.
        \item The new layouts did not show any access flooring.
        \end{enumerate}
\item Location of CCU 
\item Clear height not achievable.
\item Kitchen extract passing through security room, clashing with Monitor/LED display.
\item Uncertainty if there is access floor or not.
\item Engineer responded to swap store room with security room. The response did not adequately cover all the 
queries. The kitchen headroom or the access flooring was not responded to.
\item The response was followed by workshop meetings. As of May 20 2015, the Main Contractor did not carry the architectural changes required. \delay\delay\delay
}


\addRFI{662} {12-Apr-2014, responded 13-Apr-2014} {St Regis Podium 1 Winter Garden} 
{
\item This RFI dealt with incomplete issues regarding ceilings in Podium 1 Winter Garden.
\item The RFI requested details of return air grilles.
}


\addRFI {0646}{23 April 2015 responded 5 April 2015}{St Regis, Spa Area Podium 2}%
 {
 \index{St Regis>Spa>RFI-0646}
 \item Contractor requested clarifications for discrepancies between BARR \& WRAY requirement drawings and MEP design layout.
 \item Engineer responded to all queries and confirmed requirements.
 \item Area affected Spa. \delay \hl{Shadi to report on nature of delay}
 }{
 The Spa area subcontractor was appointed late. The area at Podium 2, was one of the first areas to be
 completed. The public areas around it, as well as the French Courtyard were designed very late.
 
 The Spa area subcontractor was appointed late. The area at Podium 2, was one of the first areas to be
 completed. The public areas around it, as well as the French Courtyard were designed very late.
}{
 Appointment took place only in January 15.
}

\addRFI [B] {0222}{10 Jun 2014 received 17 Jun 2014 }{Air and Dirt Separators, Boiler Room}
{
\item The Contractor requested clarification on additional air separators in the Boiler Room.
\item Relevant drawings Ref: DU1211-WS2173 Westin Hotel \& DU1211-WS2024 for Basement-1
   \begin{enumerate}
      \item Air and dirt separators are shown in B-1 Boiler plant room for the central water heating. This was not shown on Tender drawings.
      \item Air and Dirt separator is shown in Westin Hotel TE-3 plant room.
      \item Contractor requested verification and noted that the additional equipment shown would lead to additional costs to the Client.
      \item Engineer responded that they should be provided as shown on the revised drawings.
   \end{enumerate}
\item   Later Contractor made arrangements for submittals and for air-freighting later on these equipment. 
}{}{}

\addRFI [B] {0215}{2 Jun 2014 replied 12 Jun 2014} {Basement Duct clashes with Water Feature Plant Room}
{
\item Issues arose due to co-ordination of HV cables tray and ducting.
\item Engineer instructed that the duct be re-sized and re-routed.
}{}{}

\addRFI [WE] {0247} {25 June 2014  returned 5 July 2014} {Westin, Ground Floor, Insufficient ceiling void}
{
   \item Contractor advised that the ID drawings show a gap between the ceiling and the duct of 60mm and that this was going to impede return air.
   \item Please refer to attached modified ceiling levels.
}

\addRFI [B] {0655} {16 Apr 15 received 4 May 15 } {Basement 1, MEP Services in loading Bay}
{
  \item Contractor requested confirmation to Engineer's reply RFI: No. HLG-626-RFI-ME-0618 to proceed with capacity as mentioned in the RFI with one set of fans, one duty and one standby (total 2 fans). 
  \item Engineer responded to refer to RFI reply dated 21 April 2015.
}

\addRFI [W] {0661} {12 Apr 2015 received 22 Apr 2015} {W Mezzanine Floor, Male and Female Accessible Toilets RCP}
  {
    \item Contractor requested co-ordinated RCP layout with diffuser size type and location for male/female and accessible toilet to finalize Shop Drawing.
   \item Engineer responded by providing CAD and PDF files. 
   
   \hlshadi{Shadi to confirm if the drawings received were adequate. (Drawings were received by HLG Feb 15 2015.
     They seem to me they were for a Tender package. We should request Subcontractor Shop Drawings from
     Main Contractor.
    } 
  }
  
\addRFI[B] {0664}  {21 Apr 2015 received 28 Apr 2015} {Basement 1, Loading Bay Area}
 {
   \item Contractor requested clarifications and made proposals for additional fans.
   \item Engineer responded with detailed reply, including tag-number. 
 }
 
\addRFI {0666} {28 April 2015 received 14 May 2015 } {St Regis, Festival Dining Kitchen Podium~1 }
 {
   \item Contractor highlighted that the drainage for this area are not available and would cause problems in finalizing Mezzanine below.
   \item This was responded by CKP Who advised that designs have still not been received from AHG (FNB Division). These are AHG spaces (Specialty Restaurants). \delay\delay\delay
 }


\addRFI [WE] {0676} {7 May 2015 received 18 May 2015} {No access to shaft MR1 at Technical~2 Westin Hotel}
{
  \item Westin Hotel Technical 2 core wall penetration Builder's Works drawings (rev0 and rev1) we proposed 1700 x 950   opening. This was approved by the Engineer. This size has been revised due to the comment, changes were then incorporated in rev 1 layout. At Site the available opening is 2440 x 600. Contractor included an Annexure with proposals.
  \item Engineer responded with proposal to reduce space further between pipes. \hl{Shadi to confirm what eventually was followed.}
}

\printRFI {MRFI} {SR} 

%% Typeset all Chapters
%% 
\chapter {Mechanical RFIs Related to Westin Hotel}
\printRFI {MRFI} {WE}


\chapter{Mechanical RFIs Related to `W'~Hotel}
\printRFI {MRFI} {W}

\chapter{Mechanical RFIs Related to `Basements'  }
\printRFI {MRFI} {B}
  



	
\def\omar#1{\hl{#1}}

\ExplSyntaxOn
 
\edef\aprefix{ERFI}
% #1 DB #2 Suffix
%
 \NewDocumentCommand \printERFI { m m }
  {
  % sort the list in numerical order
    \SortDB { #1 }
     
 % map and print only for category (one to many also possible here)   
     \clist_map_inline:cn { #1 }
      {  
       \cs_if_exist:cT {\aprefix##1-#2}
          {
      %     \PASS RFI-E##1-#2
           \cs:w \aprefix##1-#2\cs_end:
           \index {RFI~Electrical>RFI-E-##1-#2}
           }
         % {\FAIL ##1-#2}\par
      }
  }
  
\ExplSyntaxOff    


 %#1 Number
%#2 Hotel
%#3 Impact
%#4 Description
%#5 Area

\newenvironment {ERFI} [3] {%
  \vspace*{12pt}
  \parindent0pt
  \mbox{\bfseries\color{red!80!black}\textsf{RFI-E-#1}}
        \textbf{#2} \textbf{#3} \par
  \begin{enumerate}}
  {\end{enumerate}}

% Optional DB letters W - W hotel
% WE - Westing
% SR  - St Regis

\NewDocumentCommand {\addERFI} { O{SR} +m +m +m +m +g  +g } 
{%
   \expandafter\gdef\csname \aprefix#2-#1\endcsname
   {%
      \begin{ERFI}{#2} {#3} {#4} 
        #5
     \end{ERFI}
       
      \IfNoValueTF{ #6 }{ #6 }{ } 
       
      \IfNoValueTF{ #7 }{ #7 }{ }
    } 
    \addtoDB {ERFIDB} {#2}
   \par 
}

%% Create DB 
\CreateDB {ERFIDB}   
  
\addERFI[W] {703} {10 May 2015 received 24 May 2015}{Westin and W Guestroom Floors}
{\index{W RFI>RFI-E-703}
   \item Contractor issued query based on HLG letter Ref: HLG/626/2.05/AMM/es/7764/15 dated 4 May 2015
      to request how the lighting in guestroom corridors would be controlled. The HLG letter referred to the Client instructing that lighting control in corridors be cancelled. 
   \item Engineer answered, 'grid switch to be provided in each electrical room for each floor, no of gangs to be as per circuits requirement and to proceed futher as per site conditions.  
   \hl{Omar and PMs to report time implications. Does this involve redrafting of any drawings? If yes please specify}   
}{}{}

\addERFI[WE] {703} {10 May 2015 received 24 May 2015}{Westin and W Guestroom Floors}
{
   \item Contractor issued query based on HLG letter Ref: HLG/626/2.05/AMM/es/7764/15 dated 4 May 2015
      to request how the lighting in guestroom corridors would be controlled. The HLG letter referred to the Client instructing that lighting control in corridors be cancelled. 
   \item Engineer answered, 'grid switch to be provided in each electrical room for each floor, no of gangs to be as per circuits requirement and to proceed futher as per site conditions.  
   \omar{Omar and PMs to report time implications. Does this involve redrafting of any drawings? If yes please specify}   
}{}{}


\addERFI [B] {702} {18 May 2015 received 22 Jan 2015} {St Regis, Basements 1, 2 and 3, Confirmation for lighting circuits for Executive Lift Lobby (B1) and Banquet suite Lift Lobby (B2 and B3) }%
 {
  \item Contractor observed that no lighting design was available for the Executive Lift Lobby (B1) and Banquet suite Lift Lobbies (B2 \& B3). Contractor added circuit references to the light points shown on RCP drawings which were received via \DT{2885}{} dated 5 May 2015 and \DT{2896}{} dated 10 May 2015.
  \item Co-ordinate light location/distribution with ID/Fit Out drawings. Submit RCP for review and approval. Update load schedule based on similar loads.
 }{}{}

\addERFI [W] {701} {12 May 2015 received 23 May 2015} {W Hotel Podium 2, Kitchen, missing circuit references}
 {
   \item Contractor advised of missing DB references and circuits for kitchen areas.
   \item \KA responded to be connected to nearest light DB, advisable DB-FB-2PR2 which has an available spare. \KA also requested for the load schedule to be submitted for approval.
   \omar{Omar I thought we have submitted load schedules, for W?}
 }
 {}{}

\addERFI [B] {696} {7 May 2015 received 19 May 2015} {Basement 1 Janitor Room and Staff Toilets }
{
  \item Contractor noted that the Janitor Room and Staff Toilet lighting and power design is not available.
  \item \KA responded to update the drawing and to be resubmitted for approval!
   \omar{Omar Did we submit?}   
}

\addERFI [SR] {695} {7 May 2015 received 12 May 2015 } {St Regis, PD3, PD4, PD5}
  {
    \item Contractor attached drawing, showing power supply to door requiring not shown in the Electrical design. Contractor requested feeder details for the power socket.
    \item \KA advised to connect to nearby motorized damper.
    \hl{Kyriacos, Omar, how does this affected site. This is clearly a disruptive activity.}
  }

\addERFI [W] {693} {27 Apr 2014 received 30 Apr 2014} {Westin Hotel, discrepancy between Fino RCP drawings and approved shop drawing, Podium 2, Meeting Rooms}
 {
    \item Contractor highlighted conflict between Contractor's approved Shop Drawing and Fino's latest issued RCP drawings. 
    \item \KA respondedn to follow RCP from Fino Int'l for location and number of downlights.
    \omar{Does this mean we need to go back and modify works on site?}
 }

\addERFI {692} {26 Apr 2015 received 2 May 2015 } {All levels having Sofia drawings}
  {
    \item  Contractor referred to nmerous RFI replies   and updated SLD attached.
    \item \KA responded to submit the SLD, for review assesment considering voltage drop, available NOC load etc. \omar{Did we submit? We are not responsible for Sofia loads and hence KA should not be asking us to stay with submitted Dewa loads. If these were exceeded please advise them}
  }

\addERFI [B] {691} {26 April 2015 received 28 April 2015 }{ Basement Loading and unloading area. Missing height of the isolators for scissor lifts.}
 {
   \item Contractor requested information as to the height of required scissor lift isolators.
   \item \KA instructed HLG to co-ordinate with Otis and advise.
   \omar{There are cables in the air in that vicinity, has this been resolved?}  
 }

\addERFI [W] {690} {7 May 2015 received 23 May 2015} {W Hotel, Mezzanine, discrepancy between FA design and architectural drawing.}
 {
 	\item Contractor advised conflicts between architectural drawings and FA design drawings.
 	\item \KA advised to update fire alarm drawings in co-ordination with FA Supplier and UAE code. Shop Drawing to be submitted for approval.
 }
 
\addERFI [SR] {689} {7 May 2015 received 13 May 2015}{St Regis, Podium 1 and TEchnical 1, relocation of amplifier. }
 {
	\item Contractor requested confirmation of location of amplifier rack from Podium-1 to Technical-1.
	\item \KA confirmed and noted routing of containment to be co-ordinated with other services.
	\omar{What amplifier is this? Was containment missed earlier, otherwise this is a disruptive activity.}
 } 

\addERFI [W] {688} {3 May 15 returned 18 May 2015} {W Hotel Ground Floor }
 {
	\item Contractor advised conflicts between current architectural drawings and Approved Fire Alarm Shop drawings.
	\item \KA replied to revise Shop Drawings according to UAE Fire regulations and resubmit.
	\omar {Did we resubmit?}
 }
 
 \addERFI [B] {687} {} {Basement 1 Clashing location of telephone and power sockets with lockers.}
  {
    \item We noticed that the location of power socket and telephone is clashing with locker location and requested clarifications.
    \item \KA provided Mediatech's response to move to nearest available free wall space.
    \omar {Why did we query this? Was it oicked up during an inspection?}
  }

 \addERFI [B] {686} {} {Basement 1 Fire Command and BMS Requirement}
  {
    \item Engineer made comments on \JV Shop drawing. Contractor requested clarifications.
    \item Provision for monitoring of BMS to be provided in the Fire Command Room \JV \& BMS Specialist to confirm that 1 data point and 1 socket is required only.
  }
  
\addERFI [B] {685} {28 Apr 2015 received 6 May 2015} {Basement 1 Fire Command Center - Elevator Panel Details} 
 {
 	\item Contractor requested information for the following:
 	\item \KA replied HLG shall lead the co-ordination with all subcontractors and conclude the requested information as all the submittals are approved. 
 	
 	\delay\delay\delay
 	
 	\omar{Has this now been received?}
 }
 
\addERFI [B] {684} {28 Apr 2015 received 6 May 2015} {Basement 1 Fire Command Center} 
 {
	\item Contractor requested the actual furniture layout for fire conrol room, as commented by Engineer on Shop drawings.
	\item \KA\ instructed HLG to co-ordinate with Furniture Supplier to submit proposal for Operator Approval.
	
	\omar {Zeljko, Rabih/Omar/Rahul we need to finish and get out, request furniture officially from HLG to install computers. } 
	\delay\delay\delay
 }

\addERFI [B] {683}{}{BMS}
 {
 	\item BMS DDC Panel relocation, became necessary due to space constraints.
 	\item \KA responded negatively.
 }

\addERFI [B] {682} { } {Du}
 {
 	\item \lorem
 }


 %%
\chapter{Electrical RFIs St~Regis Hotel}
\printERFI {ERFIDB} {SR}
%
\chapter {Electrical Basements}
\printERFI {ERFIDB} {B}
%
\chapter {Westin Hotel}
\printERFI {ERFIDB} {WE}
%
\chapter {W Hotel}
\printERFI {ERFIDB} {W}











	\def\hot{{\color{red}\scalebox{1.5}{\Fire}} delayed works. }
\def\ghot{{\color{green!80!black}\raggedright\scalebox{1.5}{\Fire}} delayed works, but completed. }
\def\phot{{\color{green!80!black}\raggedright\scalebox{1.5}{\Fire}} delayed works. partially completed. }

\def\check{{$\color{green!80!black}\check$}}
\def\partiald {50\% complete}
\def\unavailable{\hot Design unavailable}
\cxset{subsection color=black}

\chapter{Late Finalization of Provisional Sum Works}

In April 2014, we wrote to the Main Contractor informing them of cut-off dates for providing information related to provisional sums and cut-off dates for their release based on the Baseline Program. The dates varied with the latest one being middle May 2014. Some basic designs were available that enabled some embedded first fix activities to start. However, none of the information was issued on time and according to the baseline program of works. 

The state of the design upon our appointment can be gauged by the value of the provisional works which was in the vicity of 60,000,000.00~AED. This represented approximately 20\% of the base contract value. The sections that follow detail the delays that occured in receiving information.\footnote{A \textcolor{red}{\Fire} indicates the designs arrived late and a \ghot}

\section{Mechanical}

The Total Mechanical Provisionals 19,600,000.00 AED.  Supply, installation, connection, testing and commissioning of all mechanical services including Kitchen hood-make up unit, hood extract ecology unit, AHU, ducting, insulation, air outlets, grills \& diffusers, cold water \& hot water piping, valves, drainage piping, floor drains, gas piping, firefighting piping, sprinklers, fire extinguishers etc (hood fire suppression system by others) to the approval of the Engineer for food services facilities for
\medskip

\bgroup
\raggedright\small
\setcounter{step}{0}
\begin{longtable}{lp{3.4cm}rllp{3.5cm}}
\toprule
\textbf{Item}	& \textbf{Description}	 &\textbf{Amount}&\textbf{Remarks}	&\textbf{1st Fix Start}	&\textbf{K\&A Design}\\
\midrule
1 & Food Service Facilities, Main Receiving Dock and Waste Control \& Centralized Commissary and Stores
& 1,100,000.00 
& B1-SR
&1-Mar-14
& \ghot\\
\midrule
2&Food Service Facilities for St. Regis Hotel	 &1,700,000.00 &&&\\
&&&B2	&30-Mar-14	& \ghot\\
&&&B1	&30-Mar-14	& \ghot\\
&&&GF	&30-Mar-14	& \hot partially still unavailable \\
&&&P1	&10-Apr-14	& \ghot\\
&&&P2	&10-Apr-14	& \ghot\\
\midrule
3 &Food Service Facilities for W. Hotel	 &1,500,000.00 &&&\\
&&&GF	&30-Mar-14	&\hot \\
&&&P1	&30-Mar-14	&\hot \\
&&&P2	&30-Mar-14	&\hot \\
&&&P4	&30-Mar-14	&\hot \\
&&&1st Flr	&30-Mar-14	&\hot \\
&&&24th Flr	&10-May-14	&\hot \\
&&&25th Flr	&10-May-14	&\hot \\
&&&26th Flr	&10-May-14	&\hot \\
\midrule
4&Food Service Facilities for Westin Hotel	 &1,300,000.00 &&&\\
&&&GF	&30-Mar-14	&\hot \partiald  \\
&&&P1	&30-Mar-14	&\hot \\
&&&P2	&30-Mar-14	&\hot \\
&&&1st Flr	&30-Mar-14	&\hot \\
&&&35th Flr	&30-Mar-14	&\hot\\
\midrule
5&Food Service Facilities for Client's Areas	 &1,000,000.00&&&\ghot\\ 
\midrule
6&Laudry Facilities& 1,500,000.00 	&B1-SR		&& \ghot \\
 \bottomrule
\end{longtable}
\egroup

Supply, installation, connection, testing and commissioning of all mechanical services including ducting, insulation, air outlets, grills \& diffusers, chilled water piping, cold water \& hot water piping, valves, drainage piping, floor drains, etc. to the approval of the Engineer for:				
\medskip

\bgroup
\raggedright\small
\setcounter{step}{0}
\begin{longtable}{lp{3.4cm}rlll}
\toprule
\textbf{Item}	& \textbf{Description}	 &\textbf{Amount}&\textbf{Remarks}	&\textbf{1st Fix Start}	&\textbf{K\&A Design}\\
\midrule
7	&Final Fix-Mechanical	 &3,000,000.00 		&&&\hot \\

\multicolumn{2}{c}{\textbf{Total Pro. Sums - Mechanical}}	 &\textbf{19,600,000.00} &&&			\\

\end{longtable}
\egroup

\section{Electrical}

\raggedright\small
\setcounter{step}{0}
\begin{longtable}{cp{3.4cm}rllp{3.5cm}}
\toprule
\textbf{Item}	& \textbf{Description}	 &\textbf{Amount}&\textbf{Remarks}	&\textbf{1st Fix Start}	&\textbf{K\&A Design}\\
\midrule
1	&\hcancel{Supply and fix of ID works for guest elevator cars}	 &\hcancel{2,335,000.00} 	&All	&	& \\
2	&\hcancel[red]{Obstruction lighting}	 &\hcancel[red]{280,000.00} 	&\hcancel{All}	&30-Apr-14	&Cancelled\\
3	&Containment - structured cabling system	 &1,350,000.00 	&All	&15-Apr-14	& \phot \\ 
4	&Containment - CCTV system	 &600,000.00 	&All	&15-Apr-14	& \phot \\
5	&Containment - access control system	 &450,000.00 &All	&15-Apr-14	& \phot \\
6	&Containment - A/V system	 &1,300,000.00 	&All	&15-Apr-14	& \phot \\
7	&Containment - Room Management System	 &1,100,000.00 	&All	&15-Apr-14	& \phot \\
8	&Lighting Control System	 &5,000,000.00 	&All	&30-Apr-14	& \phot\par "BOH partially completed.\\ 
9	&Façade Lighting 	 &1,100,000.00 	&All	&30-Apr-14	& \phot \\
10	&landscape lighting 	& 540,000.00 	&All	&30-Apr-14	& \ghot \\
\midrule
	&\textbf{Total for P.S. Electrical}	 &14,055,000.00 &&&\\

\bottomrule
\end{longtable}


Provisional Sums - Electrical
				
Supply, installation and connection of electrical works including lighting outlets, lighting switches, lighting control, socket outlets, telephone outlet, TV outlet, speakers, internal conduits and wiring, and all necessary accessories				

\newenvironment{pstable}{
\raggedright\small
\setcounter{step}{0}
\begin{longtable}{c>{\raggedright}p{3.4cm}rl p{2cm} p{3.5cm}}
\toprule
\textbf{Item}	& \textbf{Description}	 &\textbf{Amount}&\textbf{Remarks}	&\textbf{Design Cut-off date}
             	&\textbf{K\&A}\\
\midrule}{\bottomrule
\end{longtable}}

\begin{pstable}
a) &Basement/Common Facilities Valets Lifts Lobby 	 &7,865.00 	&B3		& & \ghot\\
b) &Common Facilities: Public Lifts Lobby 2nd Basement Level	 &26,483.00 	&B2	&	& \ghot \\
c) &Common Facilities: Parking Lifts Lobby 2nd Basement Level	 &8,712.00 	&B2		&& \ghot \\
d)	&Common Facilities: Banquet Lifts Lobby 2nd Basement Level	 &16,781.00 	&B2 &		& \ghot \\
e)	&Common Facilities: Staff Cafeteria 2nd Basement Level	 &732,325.00 	&B2	 &	& \ghot\\
   & \textbf{St Regis}                                              &              &     & &\\
\end{pstable}

\begin{pstable}   
f)	&Common Facilities: Valets Lifts Lobby 1st Basement Level	 &7,260.00 	&B1  && \ghot \\
g)	&Common Facilities: Parking Lifts Lobby 1st Basement Level	 &10,868.00 	&B1  && \ghot\\
h)	&Common Facilities: Parking Lifts Lobby 1st Basement Level	 &8,712.00 	&B1  && \ghot \\
i)	&Common Facilities: Royal Suite Lift Lobby 1st Basement Level	 &26,208.00 	&B1 && \ghot \\
j)	&Common Facilities: Exec. Suite Lift Lobby 1st Basement Level	 &8,833.00 	&B1 && \ghot \\
k)	&Common Facilities: Valets Lifts Lobby 1st Basement Level	 &9,504.00 	&B1&& \ghot \\
l)	&Common Facilities: St. Regis Housekeeping 1st Basement Level	 &71,525.00 	&B1 &	1-Apr-14&\ghot \\
m)	&Common Facilities: Maintenance/Engineering Workshop 1st Basement Level	 &267,025.00 	&B1	&1-Apr-14 &\hot \\
n)	&Common Facilities: Laundry 1st Basement Level	 &483,460.00 	&B1	 &1-Apr-14	& \ghot \\
o)	&Common Facilities: Kitchen 1st Basement Level	 &263,495.00 	&B1	 &1-Apr-14	&\ghot \\
p)	&Common Facilities: Chillers/Freezers 1st Basement Level	 &232,220.00 	&B1 &1-Apr-14	& \ghot \\
q)	&Common Facilities: Kitchen Stores 1st Basement Level	 &134,500.00 	&B1 &1-Apr-14	&\ghot \\
r)	&Common Facilities: Office 1st Basement Level	 &13,629.00 	&B1	 &1-Apr-14	&\ghot \\
s)	&Common Facilities: Chillers 1st Basement	 &31,573.00 	&B1	 &1-Apr-14	&\ghot \\
t)	&Common Facilities: Cold Rooms Compressor Room 1st Basement	 &23,216.00 	&B1	 &1-Apr-14	&\ghot \\
u)	&Common Facilities: Toilet 1st Basement Level	 &9,567.00 	&B1	 &1-Apr-14	&\ghot \\
v)	&Common Facilities: Waste Handling Area 1st Basement Level	 &259,105.00 	&B1 &1-Apr-14	&\ghot \\
\midrule
	   &\textbf{Total  P Sums - Electrical}	 &2,652,866.00 &&\\
\end{pstable}

\bigskip

\subsection{Electrical St.Regis Hotel}

Supply, installation and connection of electrical works including lighting outlets, lighting switches, lighting control, socket outlets, telephone outlet, TV outlet, speakers, internal conduits and wiring, and all necessary accessories

\begin{pstable}
1	&St. Regis Hotel Meeting Room 1 Ground Level	 &49,753.00 	&GF	  &27-Apr-14	&\ghot \\
2	&St. Regis Hotel Meeting Room 2 Ground Level	 &26,158.00 	&GF	  &27-Apr-14	&\ghot \\
3	&St. Regis Hotel: Prefunction Area Ground Level	 &326,150.00 	&GF	 &27-Apr-14	&\ghot \\
4	&St. Regis Hotel: Male/Female Toilets Ground Level	 &39,758.00 	&GF	 &27-Apr-14	&\ghot \\
5	&St. Regis Hotel: Boardroom Ground Level	 &43,758.00 	&GF	       &27-Apr-14	&\ghot \\
\end{pstable}

\begin{pstable}
6	&St. Regis Hotel: Lift's Lobby Ground Level	 &42,669.00 	&GF	    &27-Apr-14	&\ghot \\
7	&St. Regis Hotel: Lobby/Sculpture/Reception Ground Level	 &500,429.00 &GF	&27-Apr-14 &\ghot \\
8	&St. Regis Hotel: Pantry Ground Level	 &12,078.00 	&GF	    &27-Apr-14	&\ghot \\
9	&St. Regis Hotel: Lift's Lobby Ground Level	 &47,163.00 	&GF	  &27-Apr-14	&\ghot \\
10	&St. Regis Hotel: Shop Ground Level	 &9,851.00 	&GF	 &27-Apr-14	& \ghot \\
11	&St. Regis Hotel: Valet Parking Cor. Ground Level	 &9,636.00 	&GF	  &27-Apr-14	&\hot \\
12	&St. Regis Hotel: Suite Lift's Lobby Ground Level	 &5,357.00 	&GF	  &27-Apr-14	&\ghot \\
13	&St. Regis Hotel: St. Regis Restaurant Ground Level	 &89,029.00 	&GF	  &27-Apr-14	&\ghot \\
14	&St. Regis Hotel: Cigar Room Ground Level	 &44,798.00 	&GF	       &27-Apr-14	&\ghot \\
15	&St. Regis Hotel: Female Toilet Ground Level	 &16,929.00 	&GF	      &27-Apr-14	&\ghot \\
16	&St. Regis Hotel: Male Toilet Ground Level	 &13,928.00 	&GF	      &27-Apr-14	&\ghot \\
17	&St. Regis Hotel: Corridor Ground Level	 &84,909.00 	&GF	         &27-Apr-14	   &\ghot \\
18	&St. Regis Hotel: Male Toilet Ground Level	 &19,679.00 	&GF      &27-Apr-14   &\ghot \\
19	&St. Regis Hotel: Prefunction Hall Ground Level	 &379,049.00 	&GF	 &27-Apr-14	&\ghot \\
20	&St. Regis Hotel: Pub Kitchen Ground Level	 &40,855.00 	&GF	   &27-Apr-14	&\ghot \\
21	&St. Regis Hotel: St. Regis Ballroom Ground Level	 &455,637.00 	&GF	      &28-Apr-14	&\hot \\
22	&St. Regis Hotel: Banquet Hall Ground Level	 &851,532.00 	&GF	   &24-May-14	&\hot \\
23	&St. Regis Hotel: Banquet Pantry Ground Level	 &160,287.00 &GF	&15-May-14	&\ghot \\

24	&St. Regis Hotel: St. Regis Restaurant Podium 1 Level	 &308,165.00 	&P1	 &15-May-14	&\ghot \\
25	&St. Regis Hotel: Café/Lounge Podium 1 Level	 &417,373.00 	&P1	 &15-May-14	&\ghot \\

26	&St. Regis Hotel: Male/Female Toilets Podium 1 Level	 &43,457.00 	&P1	 &15-May-14	&\ghot \\
27	&St. Regis Hotel: Champagne Bar Podium 1 Level	 &90,418.00 	&P1	 &15-May-14	&\ghot \\

28	&St. Regis Hotel: Pantry Podium 1 Level	 &23,672.00 	&P1	 &15-May-14	&\ghot \\

29	&St. Regis Hotel: Lift's Lobbies Podium 1 Level	 &180,950.00 	&P1	 &15-May-14	&\ghot \\

30	&St. Regis Hotel: St. Regis Main Kitchen Podium 1 Level	 &169,054.00 	&P1	 &15-May-14	&\ghot \\

31	&St. Regis Hotel: Italian Restaurant Podium 1 Level	 &215,254.00 	&P1 &15-May-14	&\ghot \\

32	&St. Regis Hotel: Kitchen Podium 1 Level	 &42,335.00 	&P1	 &15-May-14	&\ghot \\

33	&St. Regis Hotel: Boardroom Podium 1 Level	 &40,798.00 	&P1	 &1-May-14	&\ghot \\

34	&St. Regis Hotel: Break-Out Area Podium 1 Level	 &96,443.00 	&P1  &1-May-14&\ghot \\
35	&St. Regis Hotel: Meeting Room Podium 1 Level	 &25,815.00 	&P1	 & 1-May-14&\ghot \\
36	&St. Regis Hotel: Meeting Room Podium 1 Level	 &52,712.00 	&P1	 &1-May-14&\ghot \\
37	&St. Regis Hotel: Meeting Room Podium 1 Level	 &52,712.00 	&P1	 &1-May-14&\ghot \\
38	&St. Regis Hotel: Meeting Room Podium 1 Level	 &36,452.00 	&P1	 &1-May-14&\ghot \\
39	&St. Regis Hotel: Male Toilet Podium 1 Level	    &14,756.00 	&P1	 &1-May-14&\ghot \\
40	&St. Regis Hotel: Female Toilet Podium 1 Level	 &12,794.00 	&P1	 &1-May-14&\ghot \\

41	&St. Regis Hotel: Gourmet Shop/Cafe Podium 2 Level	 &267,790.00 	&P2	 &1-May-14&\ghot \\
\end{pstable}

\begin{pstable}
42	&St. Regis Hotel: Spa Podium 2 Level	 &541,352.00 	&P2	 &24-Apr-14	&\ghot \\
43	&St. Regis Hotel: Business Centre Podium 2 Level	 &197,690.00 	&P2	 &24-Apr-14	&\ghot \\

44	&St. Regis Hotel: Deluxe Suite Podium 2 Level	 &114,930.00 	&P2 &24-Apr-14	&\ghot \\
45	&St. Regis Hotel: Junior Suites Podium 2 Level	 &100,752.00 	&P2 &24-Apr-14	&\ghot \\

46	&St. Regis Hotel: Junior Suites Podium 2 Level	 &144,084.00 	&P2	 &24-Apr-14	&\hot \\

47	&St. Regis Hotel: Ambassador Suite Podium 2 Level	 &175,242.00 	&P2 &24-Apr-14	&\hot \\
48	&St. Regis Hotel: Executive Suite Podium 2 Level	 &203,292.00 	&P2 &24-Apr-14	&\hot \\

49	&St. Regis Hotel: Water Fountain Podium 2 Level	 &409,475.00 	&P2	 &8-May-14	&\hot \\
50	&St. Regis Hotel: Corridor Podium 2 Level	 &349,470.00 	&P2   &8-May-14	&\hot \\

51	&St. Regis Hotel: Guest Corridor Podium 2 Level	 &447,436.00 	&P2	 &8-May-14	&\hot \\
52	&St. Regis Hotel: Lift's Lobby Podium 2 Level	 &25,342.00 	&P2	 &8-May-14	&\hot \\

53	&St. Regis Hotel: Deluxe Suite Podium 3 Level	 &114,930.00 	&P3	 &4-May-14	&\hot \\
54	&St. Regis Hotel: Ambassador Suite Podium 3 Level	 &175,242.00 	&P3 &4-May-14	&\hot \\

55	&St. Regis Hotel: Junior Suite 1 Podium 3 Level	 &50,340.00 	&P3 &4-May-14	&\hot \\
56	&St. Regis Hotel: Junior Suite 2 Podium 3 Level	 &53,682.00 	&P3	 &4-May-14	&\hot \\

57	&St. Regis Hotel: Junior Suite 3 Podium 3 Level	 &72,042.00 	&P3	 &4-May-14	&\hot \\
58	&St. Regis Hotel: Junior Suite 4 Podium 3 Level	 &50,376.00 	&P3	 &4-May-14	&\hot \\
59	&St. Regis Hotel: Executive Suite Podium 3 Level	 &135,528.00 	&P3	 &4-May-14	&\hot \\

60	&St. Regis Hotel: Executive Suite 1 Podium 3 Level	 &132,060.00 	&P3	 &4-May-14	&\hot  \\

61	&St. Regis Hotel: Corridor Podium 3 Level	 &412,704.00 	&P3	 &18-May-14	&\hot \\
62	&St. Regis Hotel: Deluxe Suite Podium 4 Level	 &114,930.00 	&P4	 &15-May-14	&\hot \\

63	&St. Regis Hotel: Designer Suite Podium 4 Level	 &175,242.00 	&P4	 &15-May-14	&\hot \\
64	&St. Regis Hotel: Junior Suite 1 Podium 4 Level	 &50,340.00 	&P4	 &15-May-14	&\ghot \\

65	&St. Regis Hotel: Junior Suite 2 Podium 4 Level	 &53,682.00 	&P4	 &15-May-14	&\ghot \\

66	&St. Regis Hotel: Junior Suite 3 Podium 4 Level	 &72,042.00 	&P4	 &15-May-14	&\ghot \\

67	&St. Regis Hotel: Junior Suite 4 Podium 4 Level	 &50,376.00 	&P4	 &15-May-14	&\ghot \\

68	&St. Regis Hotel: Executive Suite 2 Podium 4 Level	 &264,120.00 	&P4	 &15-May-14	&\ghot \\
69	&St. Regis Hotel: Corridor Podium 4 Level	 &412,704.00 	&P4	 &30-May-14	&\ghot \\
70	&St. Regis Hotel: Presidential Suite Podium 5 Level	 &145,536.00 	&P5 &22-May-14	&\hot \\
71	&St. Regis Hotel: Designer Suite Podium 5 Level	 &174,918.00 	&P5	 &22-May-14	&\hot \\

72	&St. Regis Hotel: Executive Suite Podium 5 Level	 &271,056.00 	&P5	 &22-May-14	&\hot \\
73	&St. Regis Hotel: Junior Suite 1 Podium 5 Level	 &50,340.00 	&P5	 &22-May-14	&\ghot \\
74	&St. Regis Hotel: Junior Suite 2 Podium 5 Level	 &53,682.00 	&P5	 &22-May-14	&\ghot \\
75	&St. Regis Hotel: Junior Suite 3 Podium 5 Level	 &72,042.00 	&P5	 &22-May-14	&\ghot \\
76	&St. Regis Hotel: Junior Suite 4 Podium 5 Level	 &50,376.00 	&P5	 &22-May-14	&\ghot \\
77	&St. Regis Hotel: Corridor Podium 5 Level	 &461,461.00 	&P5	   &10-May-14	&\ghot \\
78	&St. Regis Hotel: Presidential Suite Podium 6 Level	 &145,536.00 	&P6	 &3-May-14	&\ghot \\
79	&St. Regis Hotel: Royal Suite Podium 6 Level	 &521,004.00 	&P6	 &3-May-14	&\hot \\
80	&St. Regis Hotel: Junior Suite 1 Podium 6 Level	 &72,042.00 	&P6	 &3-May-14	&\ghot \\
81	&St. Regis Hotel: Junior Suite 2 Podium 6 Level	 &50,376.00 	&P6	 &3-May-14	&\ghot \\
82	&St. Regis Hotel: Corridor Podium 6 Level	 &166,150.00 	&P6	 &26-May-14	&\ghot \\
83	&St. Regis Hotel: Special Cabanas F01 Level	 &42,878.00 	&F01	&9-May-14	&\hot \\
84	&St. Regis Hotel: Male \& Female Changing Room F01 Level	 &25,553.00 	&F01	&9-May-14	&\hot \\
85	&St. Regis Hotel: Royal Private Swimming Pool F01 Level	 &105,096.00 	&F01	&9-May-14	&\hot \\
86	&St. Regis Hotel: Pantry F01 Level	 &18,216.00 	                 &F01	&9-May-14	&\hot \\
87	&St. Regis Hotel: Special Cabanas F01 Level	 &42,895.00 	&F01	&9-May-14	&\hot \\
88	&St. Regis Hotel: Kids Play Room F01 Level	 &148,500.00 	&F01	&9-May-14	&\hot \\
89	&St. Regis Hotel: Male \& Female Changing Room F01 	 &52,008.00 	&F01	&9-May-14	&\hot \\
90	&St. Regis Hotel: Pantry F01 Level	           &25,889.00 	&F01	&9-May-14	&\hot \\
91	&St. Regis Hotel: Bar Pagoda F01 Level	 &69,603.00 	&F01	 &9-May-14	&\hot \\
92	&St. Regis Hotel: Lift's Lobbies F01 Level	 &41,250.00 	&F01	&2-May-14	&\hot \\
\midrule
   &\textbf{Total for PS - Electrical}	 &\textbf{13,266,154.00} &&&\\

\end{pstable}

\subsection{Electrical Westin Hotel}

Supply, installation and connection of electrical works including lighting outlets, lighting switches, lighting control, socket outlets, telephone outlet, TV outlet, speakers, internal conduits and wiring, and all necessary accessories

\begin{pstable}
1	&W. Hotel: W. Hotel Entrance Lobby Ground Level	 &191,100.00 	&GF	 &17-Apr-14	&\hot \\
2	&W. Hotel: Theatre Entrance Lobby Mezzanine Level	 &224,437.00 	&MZ	 &17-Apr-14	&\hot \\

3	&W. Hotel: Public Lifts Mezzanine Level	 &35,959.00 	&MZ	 &17-Apr-14	&\hot \\

4	&W. Hotel: Corridor Mezzanine Level	 &10,368.00 	&MZ	 &17-Apr-14	&\hot \\

5	&W. Hotel: Male/Female Toilets Mezzanine Level	 &61,785.00 	&MZ	 &17-Apr-14	&\hot \\


6	&W. Hotel: Kitchen Podium 1 Level	 &62,743.00 	&P1	 &19-Apr-14	& \hot \\

7	&W. Hotel: Kitchen Podium 1 Level	 &44,945.00 	&P1	 &19-Apr-14	& \hot \\

8	&W. Hotel: Kitchen Tech 1 Level	    &29,695.00 	&T1	 &19-Apr-14	&\hot \\

9	&W. Hotel: Festival Dining Restaurant 3 Area 276m2 PL1	 &133,480.00 	&P1	 &30-Apr-14	&\hot \\

10	&W. Hotel: Public Lifts Podium 1 Level	 &146,898.00 	&P1	 &30-Apr-14	& \hot \\

11	&W. Hotel: Male/Female Toilets Tech 1 Level	 &19,929.00 	&T1	 &30-Apr-14	& \hot \\

12	&W. Hotel: Lifts Lobby Tech 1 Level	 &191,363.00 	&T1	 &30-Apr-14	& \hot \\

13	&W. Hotel: KitchenPodium 2 Level	 &38,590.00 	&P2 &22-Apr-14	& \hot \\

14	&W. Hotel: Jazz Room Podium 2 Level	 &306,873.00 	&P2	 &3-Apr-14	& \hot \\

15	&W. Hotel: Public Lift Lobby Podium 2 Level	 &229,092.00 	&P2	 &3-Apr-14	& \hot \\

16	&W. Hotel: Female WC Podium 2 Level	 &8,140.00 	&P2	 &3-Apr-14	& \hot \\

17	&W. Hotel: Male WC Podium 2 Level	 &13,140.00 	&P2	 &3-Apr-14	&\hot \\

18	&W. Hotel: W-Great Room Podium 4 Level	 &434,605.00 	&P4	 &13-Apr-14	&\hot \\

19	&W. Hotel: Great Room Pantry Podium 4 Level	 &83,182.00 	&P4	 &13-Apr-14	&\hot \\

20	&W. Hotel: Business Centre Podium 4 Level	 &50,479.00 	&P4	 &13-Apr-14	&\hot \\

21	&W. Hotel: Male \& Female WC Podium 4 Level	 &37,752.00 	&P4	 &13-Apr-14	&\hot \\

22	&W. Hotel: Lift Lobby Podium 4 Level	 &22,517.00 	&P4	 &13-Apr-14	& \hot \\

23	&W. Hotel: Pre Function Hall Podium 4 Level	 &157,487.00 	&P4	 &13-Apr-14	& \hot \\

24	&W. Hotel: W-Studio 1 Podium 5 Level	 &59,004.00 	&P5	 &23-Apr-14	& \hot \\

25	&W. Hotel: Pre Function Podium 5 Level	 &101,954.00 	&P5	 &23-Apr-14	&\hot \\

26	&W. Hotel: Lift Lobby Podium 5 Level	 &17,226.00 	&P5	 &23-Apr-14	&\hot \\

27	&W. Hotel: W-Studio 4 Podium 5 Level	 &22,154.00 	&P5	 &23-Apr-14	& \hot \\

28	&W. Hotel: W-Studio 5 Podium 5 Level	 &19,399.00 	&P5 &23-Apr-14	& \hot \\

29	&W. Hotel: Male \& Female WC Podium 5 Level	 &40,018.00 	&P5	 &23-Apr-14	& \hot \\

30	&W. Hotel: W-Studio 3 Podium 5 Level	 &28,666.00 	&P5	 &23-Apr-14	& \hot \\

31	&W. Hotel: W-Studio 2 Podium 5	 &28,160.00 	&P5	 &23-Apr-14	& \hot \\

32	&W. Hotel: W Specialty Restaurant F01 Level	 &597,564.00 	&F01	&12-Apr-14	& \hot \\

33	&W. Hotel: WOW Lifts Lobby F01 Level	 &42,884.00 	&F01	&12-Apr-14	& \hot \\

34	&W. Hotel: Fantastic Suite F02 Level	 &51,126.00 	&F02	&8-Apr-14	& \hot \\

35	&W. Hotel: WOW/Parlour Suite F02 Level	 &53,598.00 	&F02	&8-Apr-14	& \hot \\

36	&W. Hotel: Studio Suite F02 Level	 &41,382.00 	&F02	&8-Apr-14	& \hot \\
37	&W. Hotel: Fabulous Room F02 Level	 &29,233.00 	&F02	&8-Apr-14	& \hot \\

38	&W. Hotel: Party Room F02 Level	 &27,005.00 	&F02	&8-Apr-14	& \hot \\

39	&W. Hotel: Corridor F02 Level	 &100,007.00 	&F02	&8-Apr-14	&\hot \\

40	&W. Hotel: Fantastic Suite TP1 Level	 &64,542.00 	&TP1	&15-Apr-14	& \hot \\

41	&W. Hotel: WOW/Parlour Suite TP1 Level	 &67,938.00 	&TP1	&15-Apr-14	& \hot \\

42	&W. Hotel: Studio Suite TP1 Level	 &42,306.00 	&TP1	&15-Apr-14	& \hot \\

43	&W. Hotel: Fabulous Room TP1 Level	 &29,233.00 	&TP1	&15-Apr-14	&\hot \\

44	&W. Hotel: Party Room TP1 Level	 &27,005.00 	&TP1	&15-Apr-14	& \hot \\
45	&W. Hotel Corridor TP1 Level	 &95,348.00 	&TP1	&15-Apr-14	& \hot \\

46	&W. Hotel: Fantastic Suite TP2 Level	 &64,884.00 	&TP2	&12-Apr-14	& \hot \\

47	&W. Hotel: WOW/Parlour Suite TP2 Level	 &69,612.00 	&TP2	&12-Apr-14	& \hot \\

48	&W. Hotel: Studio Suite TP2 Level	 &42,678.00 	&TP2	&12-Apr-14	& \hot \\

49	&W. Hotel: Fabulous Room TP2 Level	 &29,233.00 	&TP2	&12-Apr-14	& \hot \\

50	&W. Hotel: Party Room TP2 Level	 &27,005.00 	&TP2	&12-Apr-14	& \hot \\

51	&W. Hotel: Corridor TP2 Level	 &95,348.00 	&TP2	&12-Apr-14	& \hot \\

52	&W. Hotel: WOW Suite F18 Level	 &124,062.00 	&F18	&12-May-14	&\hot \\

53	&W. Hotel: WOW Suite F18 Level	 &98,436.00 	&F18	&12-May-14	& \hot \\

54	&W. Hotel: Studio Suite F18 Level	 &45,396.00 	&F18	&12-May-14	& \hot \\

55	&W. Hotel: Fabulous Suite F18 Level	 &39,204.00 	&F18	&12-May-14	& \hot \\
56	&W. Hotel: Emergency Lifts Lobby F18 Level	 &19,173.00 	&F18	&12-May-14	& \hot \\
57	&W. Hotel: Corridor F18 Level	 &93,918.00 	&F18	&12-May-14	& \hot \\

58	&W. Hotel: WOW Suite F19 Level	 &124,062.00 	&F19	&25-May-14	& \hot \\

59	&W. Hotel: WOW Suite 2 F19 Level	 &98,436.00 	&F19	&25-May-14	& \hot \\

60	&W. Hotel: Studio Suite F19 Level	 &45,396.00 	&F19	&25-May-14	& \hot \\

61	&W. Hotel: Fabulous Suite F19 Level	 &31,890.00 	&F19	&25-May-14	& \hot \\

62	&W. Hotel: Party Room F19 Level	 &27,005.00 	&F19	&25-May-14	& \hot \\

63	&W. Hotel: Corridor F19 Level	 &93,918.00 	&F19	&25-May-14	& \hot \\

64	&W. Hotel: Extreme WOW Suite F20 Level	 &198,558.00 	&F20	&4-May-14	& \hot \\

65	&W. Hotel: Studio Suite F20 Level	 &45,396.00 	&F20	&4-May-14	 & \hot \\

66	&W. Hotel: Fabulous Suite F20 Level	 &39,204.00 	&F20	&4-May-14	 & \hot \\

67	&W. Hotel: Emergency Lifts Lobby F20 Level	 &19,173.00 	&F20	&4-May-14	& \hot \\

68	&W. Hotel: Corridor F20 Level	 &102,707.00 	&F20	&4-May-14	& \hot \\

69	&W. Hotel: Extreme WOW Suite TP3 Level	 &139,116.00 	&TP3	&15-May-14	& \hot \\
70	&W. Hotel: Studio Suite TP3 Level	 &45,396.00 	&TP3	&15-May-14	& \hot \\

71	&W. Hotel: Fabulous Suite TP3 Level	 &31,890.00 	&TP3	&15-May-14	& \hot \\

72	&W. Hotel: Party Room TP3 Level	 &27,005.00 	&TP3	&15-May-14	& \hot \\

73	&W. Hotel: Corridor TP3 Level	 &91,933.00 	&TP3	&15-May-14	&\hot \\
74	&W. Hotel: Extreme WOW Suite F22 Level	 &198,558.00 	&F22	&25-May-14	&\hot \\
75	&W. Hotel: Studio Suite F22 Level	 &45,396.00 	&F22	&25-May-14	&\hot \\
76	&W. Hotel: Fabulous Suite F22 Level	 &31,890.00 	&F22	&25-May-14	& \hot \\
77	&W. Hotel: Party Room F22 Level	 &27,005.00 	&F22	&25-May-14	& \hot \\
78	&W. Hotel: Corridor F22 Level	 &91,933.00 	&F22	&25-May-14	& \hot \\
79	&W. Hotel: Back of House Pantry F24 (F25) Level	 &32,570.00 	&F25	&12-May-14	&\hot \\
80	&W. Hotel: W-Living Room F24 (F25) Level	 &519,261.00 	&F25	&25-May-14	&\hot \\
81	&W. Hotel: Male \& Female WC F24 (F25) Level	 &52,707.00 	&F25	&25-May-14	&\hot \\
82	&W. Hotel: Lifts Lobby F24 (F25) Level	 &22,077.00 	&F25	&25-May-14	&\hot \\
83	&W. Hotel: Deputy GM F24 (F25) Level	 &25,823.00 	&F25	&25-May-14	&\hot \\
84	&W. Hotel: Lifts Lobby 2 F24 (F25) Level	 &32,791.00 	&F25	&25-May-14	&\hot \\
85	&W. Hotel: Back of House Kitchen F25L (F26) Level	 &24,165.00 	&F26	&16-May-14	&\hot \\
86	&W. Hotel: Restaurant F25L (F26) Level	 &430,694.00 	&F26	&29-May-14	&\hot \\
87	&W. Hotel: Male \& Female WC F25L (F26) Level	 &52,707.00 	&F26	&29-May-14	&\hot \\
88	&W. Hotel: Lifts Lobby 1 F25L (F26) Level	 &21,978.00 	&F26	&29-May-14	&\hot \\
89	&W. Hotel: Lifts Lobby 2 F25L (F26) Level	 &13,211.00 	&F26	&29-May-14	&\hot \\
90	&W. Hotel: Destination Restaurant F26L (F28) Level	 &568,524.00 	&F28	&15-May-14	&\hot \\
91	&W. Hotel: BOH Pantry F26L (F28) Level	 &15,862.00 	&F28	&15-May-14	&\hot \\
92	&W. Hotel: Lift Lobby F26L (F28) Level	 &22,077.00 	&F28	&15-May-14	&\hot \\
93	&W. Hotel: Male \& Female WC F26L (F28) Level	 &36,872.00 	&F28	&15-May-14	&\hot \\
94	&W. Hotel: Entrance/WOW Lift Lobby F26L (F28) Level	 &22,471.00 	&F28	&15-May-14	&\hot \\
95	&W. Hotel: VIP's Deck F27 (F30) Level	 &175,373.00 	&F30	&15-May-14	&\hot \\
\midrule
	&\textbf{Total for Provisional Sums - Electrical}	 &8,494,290.00 &&&\\			

\end{pstable}
\bigskip

\section{Westin Electrical PS}

Supply, installation and connection of electrical works including lighting outlets, lighting switches, lighting control, socket outlets, telephone outlet, TV outlet, speakers, internal conduits and wiring, and all necessary accessories

\begin{pstable}
1	&Westin Hotel: Westin Bar Ground Level	 &139,700.00 &GF	&3-Mar-14	&\hot  \\
2	&Westin Hotel: Westin Bar Kitchen Ground Level	 &26,500.00 	&GF	 &3-Mar-14 &\hot \\
3	&Westin Hotel: Lobby/Reception/Entrance Ground Level	 &676,000.00 	&GF	 &3-Mar-14 &\hot \\
4	&Westin Hotel: Wash Rooms Ground Level	 &41,250.00 	&GF &3-Mar-14	&\hot \\

5	&Westin Hotel: Retail Ground Level	 &13,750.00 	&GF &3-Mar-14 &\hot \\
6	&Westin Hotel: Cocktail Room Ground Level	 &123,200.00 	&Gf	 &3-Mar-14	&\hot \\

7	&Westin Hotel: Concierge Ground Level	 &11,550.00 	&GF	 &3-Mar-14 &\hot \\
8	&Westin Hotel: Pantry Ground Level	 &24,750.00 	&GF	 &3-Mar-14	&\hot \\

9	&Westin Hotel: Westin Bar Upper Level Mezzanine Level	 &112,750.00 	&MZ &3-Mar-14 &\hot \\

10	&Westin Hotel: Food Stations Podium 1 Level	 &853,600.00 	&P1	 &22-Apr-14	& \unavailable \\

11	&Westin Hotel: Coffee Pantry Podium 1 Level	 &57,200.00 	&P1	 &22-Apr-14	& \hot \\
12	&Westin Hotel: Toilets Podium 1 Level	 &42,750.00 	&P1 &22-Apr-14	&\hot \\

13	&Westin Hotel: Kitchen Podium 1 Level	 &148,000.00 	&P1	 &10-Apr-14	&\hot \\
14	&Westin Hotel: Corridor Podium 2 Level	 &71,500.00 	&P2	 &11-Apr-14	& \hot \\

15	&Westin Hotel: Meeting Rooms Podium 2 Level	 &521,400.00 	&P2 &11-Apr-14	&\ghot \\
16	&Westin Hotel: Corridor of Meeting Rooms Podium 2 Level	 &386,100.00 	&P2 &11-Apr-14	&\ghot \\

17	&Westin Hotel: Toilets Podium 2 Level	 &47,250.00 	&P2 &11-Apr-14	&\ghot \\
18	&Westin Hotel: Kitchen Podium 2 Level	 &52,500.00 	&P2	 &29-Apr-14	&\hot \\

19	&Westin Hotel: Administration Offices Podium 4 Level	 &1,037,300.00 	&P4	 &23-Apr-14	&\ghot \\
20	&Westin Hotel: Toilets Podium 4 Level	 &34,200.00 	&P4	 &23-Apr-14	&\ghot \\

21	&Westin Hotel: W/Westin Gym/Spa Lower Level Podium 6 Level &1,125,850.00 	&P6	 &12-Apr-14	&\unavailable \\
22	&Westin Hotel: Corridor \& Lobby F01 Level	 &289,250.00 &F01	&19-Apr-14	&\unavailable \\

23	&Westin Hotel: W/Westin Gym/Spa Upper Level F01 Level	 &676,500.00 &F01	&19-Apr-14 &\unavailable \\
24	&Westin Hotel: W-Sweat Fitness Room F01 Level	 &112,750.00 	&F01	&19-Apr-14	&\unavailable \\

25	&Westin Hotel: Corridor \& Lobby for each floor in F02 to F31 Level	 &5,206,500.00 	&F31	&15-Apr-14& Completed up to 22 Floor as of 20 may 2015.	\\

26	&Westin Hotel: Presidential Suite F32 Level	 &241,800.00 	&F32	&3-May-14	 &\unavailable \\
27	&Westini Hotel: Corridor \& Lobby F32 Level	 &192,400.00 	&F32	&3-May-14	 & \unavailable\\
28	&Westin Hotel: Executive Lounge F33 Level	 &391,600.00 	&F33	&18-May-14	&\unavailable \\

29	&Westin Hotel: Corridor \& Lobby F33 Level	 &189,150.00 	&F33	&18-May-14	&\unavailable \\
30	&Westin Hotel: Royal Suite F34 Level	 &267,600.00 	&F34	&26-May-14	&\unavailable  \\

31	&Westin Hotel: Corridor \& Lobby F34 Level	 &228,150.00 	&F34	&26-May-14	&\unavailable \\
32	&Westin Hotel: Lounge and Restaurant F35 Level	 &482,350.00 	&F35	&25-May-14	&\unavailable \\
33	&Westin Hotel: Outdoor Deck F35 Level	 &102,850.00 	&F35	&25-May-14	&\unavailable \\
34	&Westin Hotel: Specialty Restaurant F35 Level	 &517,000.00 	&F35	&25-May-14	&\unavailable \\
35	&Westin Hotel: Outdoor Deck F35 Level	 &102,850.00 	&F35	&25-May-14	&\unavailable\\
\midrule
	&Total for Provisional Sums - Electrical	 &14,547,850.00 	&&&\\		
\end{pstable}




	\normalsize
}



\def\latexiiidocs{
    \parindent1em
\chapter{File Input and Output using Primitive TeX Commands}

\tex provides commands for writing and reading of streams
either from a file or a keyboard. Both the available commands as well as a limitation in the number of files that can be allocated, shows \tex's age. For more complicated programming one needs to escape to the shell and use a scripting language or LuaTeX (see \nameref{ch:luaio} and \nameref{ch:l3files} for \latex3 i/o handling.)

An example from the \latexe kernel, can illustrate better than
words the mechanism. The example is the definition of
the command \cs{bibliography}. Bibliographic information is written first to the |.aux| file, and then at the second run is read from a file the extension |.bbl|.

\begin{codeexample}[code only,vbox]
   \def\bibliography#1{%
   \if@filesw
     \immediate\write\@auxout{\string\bibdata{#1}}%
   \fi
   \@input@{\jobname.bbl}}
\end{codeexample}

\tex\ and \latex\  ability to read and write to external filesmakes it possible to produce
a Table of Contents or a List of Figures. 


Table , summarizes the available \tex\ commands. 



\begin{docCommand*}{input} {\marg{file name}}
The command \cs{input} inputs the specified file as \tex input and is perhaps the most widely  i/o command used by authors. 
\end{docCommand*}

\begin{texexample}{}{}
\meaning\input
\end{texexample}



\begin{docCommand}{endinput}{}
\cs{endinput} Terminate inputting the current file after the current line. This is recommended to be inserted at the
end of packages and other files in order to avoid spurious spaces after the file is included.
\end{docCommand}

\begin{docCommand*}{pausing}{}
\cs{pausing}  Specify that TEX should pause after each line that is read from a file.
\end{docCommand*}

\begin{docCommand*}{inputlineno}{}
\cs{inputlineno} Number of the current input line.
\end{docCommand*}

\begin{docCommand*}{message}{}
\cs{message} Write a message to the terminal. This is used widely in the kernel which is sprinklered with code such as:
\end{docCommand*}

\begin{teXXX}
\message{registers,}
\end{teXXX}

\begin{docCommand*}{write} {}
\cs{write} write  text to the terminal or to a file. 
\end{docCommand*}

\begin{docCommand*} {read} {}
\cs{read} Read a line from a stream into a control sequence.
\end{docCommand*}

\begin{docCommand*}{newwrite}{}
\cs{newread} \cs{newwrite} Macro for allocating a new input/output stream.
\end{docCommand*}

\begin{docCommand*}{openin}{\meta{4-bit number} = \meta{file name}}
\cs{openin} \cs{closein} Open/close an input stream. The command opens a file for input. The file may then 
be read line by line using |\read| or a test can be made to check if a file actually exixist or not. The number must be between 0 and 15. 
\end{docCommand*}

\begin{docCommand*}{openout}{}
\cs{openout} \cs{closeout} Open/close an output stream.
\end{docCommand*}

\begin{docCommand*}{ifeof}{}
\cs{ifeof} Test whether a file has been fully read, or does not exist.\\
\end{docCommand*}

\begin{docCommand*}{immediate}{}
\cs{immediate} Prefix to have output operations executed right away.\\
\end{docCommand*}

\begin{docCommand*}{escapechar}{}
\cs{escapechar} Number of the character that is used when control sequences are being converted into character tokens. IniTEX default: 92.\\
\end{docCommand*}

\begin{docCommand*}{newlinechar}{}
\cs{newlinechar} Number of the character that triggers a new line in \cs{write} and \cs{message}
statements.
\end{docCommand*}

\section{Opening and closing files}

It is easy to write and read text files from inside a \tex\  document. The \cmd{\input} is well known and is commonly used
to break a long document into smaller--and more logical parts. In addition the \cmd{\read} makes it  possible to
read a file record by record. New files can be created by \tex\ and data written on them record by record. There can be
a maximum of 16 input and 16 output files open at any given time. Each file is identified  internally by means of a file number. 


The \cmd{\newread} and \cmd{\newwrite} generate the next available file number. 



Output is done by write or \cmd{\immediate}\cmd{\write}. Input is done either by \refCom{read} to or 
\verb+ \input<filename>+. Each write creates a record on the file, whose maximum size is only limited by the operating system, so these can be quite large.

\subsection{Writing control sequences to files}

An important feature of file output is that expandable tokens are expanded during a \refCom{write}. If the
name of a control sequence, rather than its expansion, should be written on a file, either
\cmd{\noexpand} or \cmd{\string} should be used to inhibit the expansion. If a control sequence is unexpandable,
its name is written on the file. If it is undefined an error message is issued when \tex\ tries to expand it during
the \cmd{\write}. It is also possible to avoid expansion during a \\cs{write} by changing the catcode of `\textbackslash'.
This way, anything that starts with a backslash is no longer considered a control sequence.

To complicate matters more, the actual write is deferred until the current page is shipped out. The reason for that is that the user may want to write the page number on the file (this is common when a table-of-contents file or index file is generated) and this number is only known inside the Output Routine. If no page numbers are involved, the user can force the record to be written on the file immediately by using |\immediate\write|:

\begin{teXXX}
  \immediate\write
\end{teXXX}

\section{Interaction with the user}

File numbers are between 0 and 15. File numbers outside this range refer to the standard I/O devices. If you 
write

\begin{teXXX}
  \read-1 to \note
\end{teXXX}

will read from the keyboard without a prompt into \cmd{\note}. The quantity \cmd{\note} does not need to be predefined
or declared. Once input is read into it, \\cs{note}  can be expanded like a macro.
The |filenumber| 16 displays information to the screen.

\begin{teXXX}
  \write16{...}
\end{teXXX}


\section{Writing Arbitrary Strings on a File}

We start with a short review of \cmd{\edef}. In |\edef\abc{\xyz \kern1em}|, the control 
sequence |\xyz| is expanded immediately (when |\abc| is defined), but the 
\cmd{\kern} is only executed later, when |\abc| is expanded.


\section{Writing to standard latex files}

You can write to the aux file with

|\write\@auxout{hello}|

or

|\immediate\write\@auxout{hello2}|

or

|\protected@write\@auxout{}{hello3}|

Depending on requirements.

|\immediate\write| writes to the specified file at that point, expanding the supplied tokens (like |\edef|) so fragile commands will do the wrong thing.

\cs{write} does not write at that point it puts a write node into the current vertical or horizontal list and if that list is shipped out to make a page then the write happens. This is needed to get page numbers correct. (If the write is inside a box and that box is never used on the main page then nothing is written to the file.)


|\protected@write| is a LaTeX-defined macro that uses |\write| but arranges that |\protect| works as required in LaTeX to protect fragile commands. The extra argument unused above allows you to locally insert extra definitions to make more commands be safe or have special definition in the write, see for example the definition of |\index| or |\addtocontents|.


\subsection{Writing to the auxiliary file }

It is safe to write to the aux file, however you have to be aware that the file will be read back at least at the begin and end of the document, so you need to write lines that are safe in that context.

If you want to write to your own file then you just need to do

\begin{teXXX}
\newwrite\myfile
\immediate\openout\myfile=\jobname.foo
\end{teXXX}

in the preamble and then replace |\@auxout| by |\myfile| when writing.

Have a look at the way |\tableofcontents| or |\listoftables| or |\listoffigures| work in latex.ltx or documented in source2e. They basically all use

\begin{teXXX}
\def\@starttoc#1{%
  \begingroup
    \makeatletter
    \@input{\jobname.#1}%
    \if@filesw
      \expandafter\newwrite\csname tf@#1\endcsname
      \immediate\openout \csname tf@#1\endcsname \jobname.#1\relax
    \fi
    \@nobreakfalse
  \endgroup}
\end{teXXX}

\section{What to do when you run out of files}

The limitation on the number of files that can be open for writing is discussed further in Chapter~\ref{ch:l3files}.  In this package we have overcome it, by using the \pkgname{morewrites}. Do note that any packages that check for validity of newwrite (such as filecontents) will produce errors when used in conjuction with \pkgname{morewrites}. In this package we have patched the filecontents  package with an internal version.



































    \chapter{Expl3 File Operations}
\label{ch:l3files}


 
\tex provides only some basic primitive control sequences for dealing with files. \tex is also limited to 16 input streams and 16 output streams making it considerably difficult to manipulate too many files. In most of the examples here we have used, streams allocated by the \latexe kernel for temporary operations and hence we can re-use them, but with extreme care. 

Checking for the existence of a file is simple and we can use the |\file_if_exist:nTF| function. 

\section{Creating streams}

Before you can use a file in \tex you need to allocate a stream. With \latex3 you can use \docAuxCommand*{ior_new:N} or \docAuxCommand*{iow_new:N} depending if is a stream for write or read. All I/O operations are global: streams are declared with global names and treated accordingly.   

As one can run out of handles very quickly, the \pkgname{phd} package loads the \pkgname{morewrites} and also patches the \pkgname{filecontents} package to enable us to use it. This was renamed to |phdfilecontents|. In Example~\ref{ex:createstreams} 

\begin{texexample}{File streams}{ex:createstreams}
\ExplSyntaxOn
\iow_new:N \scratch_filea
\iow_new:N \scratch_fileb
\iow_new:N \scratch_filec
\iow_new:N \scratch_filed
\ExplSyntaxOff
\end{texexample}

Another way used widely by \latexe and package authors is to use \latexe's |\@inputcheck| file handle. This is a file handle used by |filecontents| as well as internally by the \latexe kernel for  and for building the |\IfFileExists| control sequence and hence its name. 


\begin{texexample}{LaTeXe examples}{ex:inputcheck}
\makeatletter
\bgroup
\ttfamily \meaning\@inputcheck\\
\number\@inputcheck %
\egroup
\makeatother
\end{texexample}

Stream management goes back to \tex and Plain\tex which use an allocation meachanism to assign the the names to numbers and then . This mechanism was then moved onto \latex2.  \latex3 uses a slightly different mechanism but the basic logic is still the same. With \latex3 the allocations are kept in a property list. 

\section{File Operations}
\begin{texexample}{File operations}{ex:fileops}
\ExplSyntaxOn
% Check if a file exists 
  \file_if_exist:nTF { filetest.txt } { \PASS } { \FAIL }
% Check for another one
  \file_if_exist:nTF { filetest }     { \PASS } { \FAIL }
  \g_file_current_name_tl
\ExplSyntaxOff
\end{texexample}

\subsection{Handling paths}

The |l3files| module provides a mechanism to add paths to the search path used to search for 
a file. This is pretty much similar to |\graphicspath|.

\begin{docCommand*}{file_path_include:n} {\marg{path}}
Adds \meta{path} to the list of those used to search when reading files. The assignment is local.
The \meta{path} is processed in the same way as a \meta{file name}, i.e., with x-type expansion
except active characters. Spaces are not allowed in the \meta{path}.
\end{docCommand*}

In Example~\ref{ex:corpora1} we add to the search path the directory, where our corpora data is residing. Then we check to see if the file |female.txt| exist. This is a large text file (extension |.txt|), containing common female names
in the US. We will use this file later on for some for examples.

\begin{texexample}{File operations}{ex:corpora1}
\ExplSyntaxOn
% Add path
\file_path_include:n {./corpora/}
\file_if_exist:nTF {female.txt} {\PASS}{\FAIL}
\ExplSyntaxOff
\end{texexample}
\index{path}\index{file operations>path}

\subsection{Loading files}

Loading a full file can of course be achieved with the |\input| command. Under the experimental section of the |\expl3| there is also a function that can load a file on condition that it exists. 

\begin{docCommand} {file_if_exist_input:n} { \marg{file name} }
Searches for \meta{file name} using the current TEX search path and the additional paths
controlled by |\file_path_include:n|). If found, inserts the \meta{true code} then reads in
the file as additional \latex source as described for |\file_input:n|. Note that 
|\file_if_exist_input:n| does not raise an error if the file is not found, in contrast to |\file-input:n|.
\end{docCommand}

\begin{texexample}{Loading a file only if it exists}{}
\ExplSyntaxOn
\file_if_exist_input:n {filetest.txt}
\ExplSyntaxOff
\end{texexample}

Since the file is loaded within the expl3 block, it will cause all spaces to be removed from the output, we can overcome this by following the |expl3| design pattern of declaring a function with |xparse| and calling it outside the block.

\begin{texexample}{Loading a file only if it exists}{}
\ExplSyntaxOn
\DeclareDocumentCommand \CorporaInput{ m }
  {
    \file_if_exist_input:n { #1 }
  }
\ExplSyntaxOff
\CorporaInput {filetest.txt}  
\end{texexample}

As is usual with |expl3| functions the |nTF| signature form of the |\file_if_exist_input:| is also available. 

\begin{texexample}{Loading a file only if it exists}{}
\ExplSyntaxOn
\DeclareDocumentCommand \CorporaInput{ m }
  {
     \file_if_exist_input:nTF {#1}
  }
\ExplSyntaxOff
\CorporaInput {filetest.txt}{ \PASS } { \FAIL }
\CorporaInput { filetest.txt }{ \PASS } { \FAIL }
\end{texexample}

The above code fails if we leave spaces between the \{\verb*+ filetest.txt +\}, we can easily remove them using the  |>{ \TrimSpaces }| argument processor.

\begin{texexample}{Loading a file only if it exists}{ex:trimspaces}
\ExplSyntaxOn
\DeclareDocumentCommand \CorporaInput{  >{  \TrimSpaces } m }
  {
     \file_if_exist_input:nTF {#1}
  }
\ExplSyntaxOff
\CorporaInput {filetest.txt}{ \PASS } { \FAIL }
\CorporaInput { filetest.txt }{ \PASS } { \FAIL }
\makeatletter
%\@ifpackageloaded{test}{\PASS test loaded}{\FAIL}
\makeatother
\end{texexample}

In the next example, we will try and consolidate some of the skills we have been developing so far.
In the \pkgname{phd} package, we are loading over 70 packages through the package manager. We wanted
to automatically keep track of which packages we loaded and which we did not (as they might
not have been in our distribution). \latexe provides a useful macro \docAuxCommand{@ifpackageloaded}
that can check if the package has been loaded or not. 

\begin{texexample}{Loading a Package}{ex:loadpackage}
\ExplSyntaxOn

\clist_new:N \g_packages_loaded_clist
\clist_new:N \g_packages_failed_clists
\clist_new:N \g_packages_loaded_by_others_clist

\makeatletter
%\DeclareDocumentCommand \requirepackage{  >{  \TrimSpaces } m m m }
%  {
%     \file_if_exist:nTF {#1.sty} 
%       { 
%         \@ifpackageloaded{#1} 
%              {
%                   \clist_put_left:Nn \g_packages_loaded_by_others_clist  {#1} 
%               }
%               {
%                   \clist_put_left:Nn \g_packages_loaded_clist  {#1}  
%                 #2
%              }
%        } 
%        { 
%          \clist_put_left:Nn \g_packages_failed_clist  {#1}
%          #3 
%        }
%  }
 

%\@ifpackageloaded{xcolor}{true}{false}
%\@ifpackageloaded{lettrine}{\PASS}{\FAIL}
%\@ifpackagewith{ragged2e}{}{\PASS}{\FAIL}
%\@ifpackagewith{soul}{}{\PASS}{\FAIL}
%\@ifpackagewith{siunitx}{fixed-exponent,scientific-notation}{\PASS}{\FAIL siunitx}
% \makeatother
% 
%\par
%\requirepackage {xcolor}    {\PASS } { \FAIL }
%\requirepackage {soul}      {\PASS } { \FAIL }
%\requirepackage {calligra}  {\PASS } { \FAIL }
%\requirepackage {hyperref}  {\PASS } { \FAIL }
%\requirepackage {layouts}   {\PASS } { \FAIL }
%\par
%
%\clist_map_inline:Nn \g_packages_loaded_by_others_clist 
%  {
%    loaded~by~others~\ldots~ #1,~
%  }
\makeatother
\ExplSyntaxOff
\end{texexample}



\section{Reading and writing to streams}

Typical file operations are reading, writing and appending. Common file management operations are creating, deleting, opening, closing, copying and renaming.

\begin{texexample}{File operations}{ex:fileops}
\edef\someheading{Another test}
\ExplSyntaxOn
\iow_open:Nn \tempstream { filetest.txt }
\iow_now:Nx \tempstream {\someheading}
\iow_close:N \tempstream
\let\getfile\file_input:n
%\file_input:n {filetest.txt}
\ExplSyntaxOff
\getfile {filetest.txt}
%\getfile{./corpora/female.txt}
\end{texexample}   

\section{Input-output streams}

Reading one line at a time from a file, uses the \tex primitive |\read|. One important item to watch is that there are different commands for read and write you need to use the |\io|\textcolor{thered}{\texttt{r}}, rather than |iow_|

Streams  are precious in \tex  as we only have 16 available , so when reading from a file, when we use \LaTeXe, we can use some of \LaTeX build-in streams, so we will be using |\@inputcheck|

\begin{texexample}{File operations}{ex:fileops}
\ExplSyntaxOn

\makeatletter
\global\let\ltx_scratch_stream \@inputcheck
\makeatother
\ior_open:Nn \ltx_scratch_stream {male-a.txt}
\ior_get:NN \ltx_scratch_stream \l_tmpa_tl
\tl_use:N \l_tmpa_tl\par
\ior_get:NN \ltx_scratch_stream \l_tmpa_tl
\tl_use:N \l_tmpa_tl
\ior_close:N \ltx_scratch_stream

\ExplSyntaxOff
\end{texexample}   


Reading line by line is not very useful. What we need is the ability to read all he lines
recursively until the end of the file. 

\begin{texexample}{File operations}{ex:fileops}
\ExplSyntaxOn
% add the file path to the name
\file_path_include:n {./corpora/}

% open the stream
\ior_open:Nn \ltx_scratch_stream {male-a.txt}

% define a macro so we can do recursion
\cs_set:Npn \read_loop {
  \if_eof:w \ltx_scratch_stream
    \ior_close:N \ltx_scratch_stream
    \let\next\relax
 \else:
   \ior_get:NN \ltx_scratch_stream \tmpa
   \tl_use:N \tmpa,~
   \let\next\read_loop
 \fi:      
 \next 
}

% read the file and typeset the words with a comma
\read_loop
\ExplSyntaxOff
\RaggedRight
\end{texexample}   

In the following example we do some more changes to the example. This time instead of typesetting the names,
we will add them to a clist. We will also read a different file that contains an alphabetical list of male names. Then we will check if the name Zacharias is in the list. 

\begin{texexample}{File operations}{ex:fileops}
\ExplSyntaxOn
% \clist
\clist_new:N \males
% open the stream
\file_path_include:n {./corpora/}
\ior_open:Nn \ltx_scratch_stream {male.txt}

% define a macro so we can do recursion
\cs_set:Npn \read_loop {
  \ior_if_eof:NTF \ltx_scratch_stream
    {
      \ior_close:N \ltx_scratch_stream
      \cs_set_eq:NN \next\relax
    }
    {  
      \ior_get:NN \ltx_scratch_stream \tmpa
      \clist_put_right:Nx \males {\tl_use:N \tmpa}
      \cs_set_eq:NN \next\read_loop
   } 
 \next 
}

% read the file and parse the words as a clist \males
\read_loop

% check if Zacharias or Mary are in the list
\clist_if_in:NnTF\males {Zacharias} {\PASS} {\FAIL}
\clist_if_in:NnTF\males {Mary} {\PASS} {\FAIL}
\ExplSyntaxOff
\end{texexample}   

\section{Appending to a file}

To append to a file, first we need to read the contents of the file into a |tl_var| then apend the material and close the file. We then open it again in write mode and write the contents of the |tl_var|. Let us try it out.


    
\section{Writing to the log or aux files}    

There are some constant input-output streams. There is a somewhat different programming philosophy here are these are normally via messages and not directly as shown in the example here.  These are handled in the chapter dealing with messages.

\begin{texexample}{Writing to log and terminal}{ex:log}
\ExplSyntaxOn
\iow_term:x {Something}
\ExplSyntaxOff
\end{texexample}
      
Armed with all these it maybe time to review again our database functions that we have created in the earlier chapter on clists.

                
          
            
              
                
                  
                      




	\chapter{A more flexible and robust method of defining functions with LaTeX3 and xparse}
\label{ch:xparse}

The \LaTeX3 Team developed the package \pkg{xparse} to provide document level 
authors with some powerful commands that extend those such as \cs{newcommand}
of \latexe. The code is been stable and the interface is not expected to change. 
Although targetted at document level, the commands offered can be used effectively to produce code used in packages.\footnote{\protect\url{http://tex.stackexchange.com/questions/98152/always-use-newdocumentcommand-instead-of-newcommand}} The functions offered by the package enable commands with star, or optional arguments to be produced easily. 

\begin{docCommand}{DeclareDocumentCommand}{\marg{function}\marg{argument specification}\marg{code}}
This family of commands are used to create a document-level \emph{function}. The argument
specification for the function is given by \textit{arg spec}, and expanding to be replaced by the
\textit{code}. Unlike \latex's definition commands, all xparse commands take two arguments.
The first one is the \textit{argument specifier}, and the second is the \textit{code.}
\end{docCommand}

\begin{texexample}{DeclareDocumentCommand}{l3:1}
\DeclareDocumentCommand \foo { m o m } { 
    arg 1 = #1, arg 2 = #2,  arg 3 = #3 }
\foo{A}[B]{C}  

\foo{A}{B}    
\end{texexample}

In the example above |{m o m}| is the argument specifier. It tells the function  to expect, two mandatory arguments and one optional denoted by the letter \textbf{o}. There are many more specifiers. For example \textbf{O} takes an parameter as a default value.\index{argument specifier}

\begin{texexample}{DeclareDocumentCommand}{l3:1}
\DeclareDocumentCommand \foo { m O{\ldots} m } { 
    arg 1 = #1, arg 2 = #2,  arg 3 = #3 }
\foo{A}[B]{C}  

\foo{A}{B}    
\end{texexample}

The argument markers can be entered in any order. In the following example we will also add an optional argument in a curly bracket. Although this is frowned upon in certain contexts it is useful. Consider the case of a chapter title that also has a subtitle. \docAuxCommand*{Chapter}, it maybe more natural and useful to have input of the form, as shown in Example~\ref{l3:g}. 

\begin{texexample}{DeclareDocumentCommand}{l3:g}
\DeclareDocumentCommand \MyChapter { o m g } { 
\centering #2\par #3\par }
    
\MyChapter{THIS IS THE MAIN TITLE}{This is a subtitle}  

\MyChapter[]{THIS IS THE MAIN TITLE}{This is a subtitle}        
\end{texexample}

\begin{texexample}{DeclareDocumentCommand}{l4:g}
\DeclareDocumentCommand \MyChapter {s o m g } { 
\IfBooleanTF {#1} {\gdef\fonta{\bfseries\selectfont}}{\gdef\fonta{}}
\IfNoValueTF {#2} {No option\par}{#2}

\centering {\fonta #3}\par #4\par 
  }   
\MyChapter{ THIS IS THE MAIN TITLE}  

\MyChapter*[short title]{THIS IS THE MAIN TITLE}{This is a subtitle}        
\end{texexample}

As you can see, it is fairly easy to produce starred and unstarred versions of commands as well as as any form of optional arguments. Let us now see some of the other command definition functions, before we continue with other specifiers.

\begin{docCommand}{NewDocumentCommand}{\marg{function}\marg{argument specification}\marg{code}}
will issue an error if \meta{function} has already been defined
\end{docCommand}

\begin{docCommand}{RenewDocumentCommand}{\marg{function}\marg{argument specification}\marg{code}}
For changing a definition,
issuing an error message if the macro does
not already exist.
\end{docCommand}


As the \cmd{\DeclareDocumentCommand} always updates a definition, it is used for the examples in this chapter to avoid any errors.

What sets the above commands apart from \latexe \cmd{\newcommand} is the argument specification.



\begin{texexample}{DeclareDocumentCommand}{l3:1}
\DeclareDocumentCommand \teststar {s o m } { 
\IfBooleanTF {#1}
  { \typesetnormalchapter {#2} {#3} }
  { \typesetstarchapter {#3} }
}  
\newcommand\typesetnormalchapter[2][]{
  normal chapter
}
\newcommand\typesetstarchapter[1]{
  #1
}
\teststar{Test}

\teststar*{test}
\end{texexample}    

    
The argument specification \textbf{m o m} in the example enables the function to accept three arguments, two mandatory and one optional. 


\section{Argument specifications}

The basic idea of an argument specification is that each argument is listed as a single letter. 
As the argument specification is a mandatory argument, a function with no arguments still needs an arg spec.

\begin{texexample}{Empty arg spec}{}
\DeclareDocumentCommand\atest{}{some text}
\atest
\end{texexample}

Manadatory arguments are created using the letter \textbf{m}.

\begin{marglist}
\item [m] Mandatory. This is a standard mandatory argument, which can either be a single token alone or multiple tokens surrounded by curly braces. Regardless of the input, the argument will
be passed to the internal code surrounded by a brace pair. This is the \pkgname{xparse} type
specifier for a normal \tex argument.
\end{marglist}

\begin{texexample}{Mandatory Values, verbatim}{}
\DeclareDocumentCommand\testverbatim{ v }{
    \ttfamily#1
}
\testverbatim+ \this is a test +

\testverbatim * &^%$#\test *

\testverbatim{\ttfamily \bfseries\normalfont test}
\end{texexample}

The \textbf{l} specifier reads its argument, until it encounters a left brace. It is equivalent to \tex \# argument. Can be used basically for |\hbox| type comands.

\begin{texexample}{Mandatory arguments l-specifier}{}
\DeclareDocumentCommand\myhbox{ l }{
   \hbox to \dimexpr(#1)\relax
}
\fbox{\myhbox 12pt+1em+13ex  {test}}
\end{texexample}



\begin{marglist}
\item [o] Optional argument in  []. Returns |-NoValue-| if not present.
\item [O] As for \textbf{o}, but returns \meta{default}, if no value is given. Should be given as |O{default}|.

\item [s] Starred version
\item [v] Verbatim. Reads an argument “verbatim”, between the following character and its next occurrence,
in a way similar to the argument of the LATEX2" command \cmd{\verb}. Thus
a v-type argument is read between two matching tokens, which cannot be any of
\%, \#, \{, \}, \^ or  . The verbatim argument can also be enclosed between braces,
\{ and \}. A command with a verbatim argument will not work when it appears
within an argument of another function.
\item [l] An argument which reads everything up to the first open group token: in standard
\latex this is a left brace.
\item [u] Reads an argument “until \meta{tokens} are encountered, where the desired \meta{tokens}
are given as an argument to the specifier: |u|\meta{tokens}.
\item [d] An optional argument that is delimited. 
\item [D] As for d, but returns \meta{default} if no value is given: D\meta{token1} \meta{token2}\marg{default}.
Internally, the o, d and O types are short-cuts to an appropriated-constructed D
type argument.
\item [t]  An optional \meta{token}, which will result in a value \cs{BooleanTrue} if \meta{token} is 
            present and \cs{BooleanFalse} otherwise. Given as \meta{token}.
\item [g] An optional argument given inside a pair of \tex group tokens (in standard \latex,
              \{ . . . \}, which returns |-NoValue-| if not present.
\item [G] As  for \textbf{g} but returns \meta{default} if no value is given: |G|\marg{default}.
\end{marglist}

\begin{texexample}{Default Values}{}
\DeclareDocumentCommand\testcolor{ O{red} m }{
    \textcolor{#1}{#2}
}
\testcolor{This is typeset in red}
\testcolor[blue]{This is typeset in blue.}
\end{texexample}

\section{Testing special values}

The optional arguments of a function defined using |xparse| use dedicated variables to return
information about the naure of the argument received.

\begin{docCommand}{IfNoValueTF}{\marg{argument}\marg{true code}\marg{false code}}
The function tests if the argument has a value and executes the true of false code, by means
of a |-NoValue-| marker. 
\begin{texexample}{special values}{}
\DeclareDocumentCommand\doccmd{O{red} m}
    {
        \IfNoValueTF{#1}
            {\doccmdnocolor{#1}}
            {\doccmdcolor{\textcolor{#1}{#2}}}
     }
\newcommand\doccmdnocolor[1]{#1}
\newcommand\doccmdcolor[2]{#1 #2}     
This is \doccmd[blue]{text}  and this is \doccmd{text}.   
\end{texexample}
\end{docCommand}

\begin{texexample}{special values}{}
\DeclareDocumentEnvironment{allbold}{o}
    {
        \bfseries 
        \IfNoValueTF{#1}
            {\color{red}}
            {\color{#1}}
    }
    {                 }
\begin{allbold}[magenta]
\lorem
\end{allbold}
\end{texexample}

\begin{texexample}{variants}{ex:variants}
\ExplSyntaxOn
\cs_set:Npn \foo_something:Nn #1#2 {
   \csname\expandafter#1\endcsname{blue}{a a a} 
   { #2}
  }
\cs_generate_variant:Nn \foo_something:Nn { c }
%\meaning\foo_something:cn
\ExplSyntaxOff
\lorem


\end{texexample}



	\makeatletter\@specialfalse\makeatother
\parindent1em
\chapter{The Basic LaTeX3 Syntax and Approach}
 \label{ch:l3intro}
 
 \epigraph{A final hint: listen carefully to what language users say they
want, until you have an understanding of what they really want. Then
find some way of achieving the latter at a small fraction of the cost
of the former. This is the test of success in language design, and
of progress in programming methodology. Perhaps these two are the same
subject anyway.}{C.A.R. Hoare, 1973}

		
\epigraph{Frank, in case you needed encouragement, please bear this in mind: I'm very much down at the blunt end of (La)TeX -- almost a total end-user. Following an earlier recommendation in this Q\&A, I visited the expl3 manual and was scared witless... Hope you can understand that---it's not a complaint, just an indication of the intellectual/experience distance from here to there.}{---Brent.Longborough Mar 2 '12 at 9:02 at \href{https://tex.stackexchange.com/questions/45838/what-can-i-do-to-help-the-latex3-project/46427\#46427}{SX.TX}}
 
Niklaus Wirth, the developer of the Pascal language long back in the 70’s wrote a paper titled \emph{On the Design of Programming Languages}. In his paper Wirth advocated that an important aspect of language design is \emph{simplicity}. He later on described the lessons learnt from his own works as:\footnote{\protect\url{http://chrisposkitt.com/tag/wirth/}}:

\begin{enumerate}
\item Writing a program is difficult.
\item Writing a correct program is even more so.
\item Writing a publishable program is exacting.
\item Programs are not written. They grow!
\item Controlling growth needs much discipline.
\item Reducing size and complexity is the triumph.
\item Programs must not be regarded as code for computers, but as literature for humans.
\end{enumerate}

The LaTeX3 syntax can only be described with some awe as `different’, although it retains some remnants of 
\tex’s syntax retaining the backslash, it is so different that many developers and package writers have resisted its adoption irrespective of the fact that it offers some solid code. 

Resistance to the language is understandable and noticed early by Computer Science pioneers. Hoare wrote:

\cxset{quotation font-size=\normalsize}
\begin{quotation}
A necessary condition for the achievement of any of these objectives
is the utmost simplicity in the design of the language. Without simplicity,
even the language designer himself cannot evaluate the consequences of his
design decisions. Without simplicity, the compiler writer cannot achieve
even reliability, and certainly cannot construct compact, fast and
- efficient compilers. But the main beneficiary of simplicity is the user
of the language. In all spheres of human intellectual and practical
activity, from carpentry to golf, from sculpture to space travel, the
true craftsman is the one who thoroughly understands his tools. And this
applies to programmers too. A programmer who fully understands his
language can tackle more complex tasks, and complete them quicker and
more satisfactorily than if he did not. In fact, a programmer's need
for an understanding of his language is so great, that it is almost
impossible to persuade him to change to a new one. No matter what the
deficiencies of his current language, he has learned to live with them;
he has learned how to mitigate their effects by discipline and documentation,
and even to take advantage of them in ways which would be impossible
in a new and cleaner language which avoided the deficiency.

It therefore seems especially necessary in the design of a new
programming language, intended to attract programmers away from their
current high level language, to pursue the goal of simplicity to an
extreme, so that a programmer can readily learn and remember all its
features, can select the best facility for each of his purposes, can
fully understand the effects and consequences of each decision, and can
then concentrate the major part of his intellectual effort to understanding
his problem and his programs rather than his tool.
\end{quotation}

I have been programming for many years and have a disdain for languages that---as Hoare
put it--- I cannot remember ``all its features’’.  LaTeX3 has not achieved the level of simplicity required in its core. As a tool it fails the simplicity test and effortful learning is necessary to use it effectively. Currently there are probably less than twenty developers that understand it fully. 

Where, \latex3 excels is its architecture, overall plan and direction and modularizing the code to an extend that the required tools reside in logically set modules or classes in \latex’s terminology. What I can promise you, once you master it, there is no looking back. 

\section{Is it stable?}

One question that often arises is the stability of the current \latex~3 code base. Of course the degree to which software are “stable enough” depends on the requirements. Joseph Wright, answering a question on the SX.TX Q\&A site wrote:

\begin{latexquotation}
If you want 'will never change again', then plain TeX is probably your best bet. Knuth does still fix bugs periodically, but most things are now likely to be regarded as 'features' rather than bugs and so it's extremely likely that a document written in plain today will still work totally unchanged in tens of years (assuming TeX systems continue to be available).

The LaTeX2e kernel is also very unlikely to change further, and so is almost if not quite as stable as TeX itself. The team do fix bugs and do allow a bit more leeway than Knuth does, but even so it's extremely unlikely anything will change with LaTeX2e at the kernel level in a way that would require changes in documents.

There are some LaTeX packages one could reasonably decide to use which are also very stable and unlikely to see changes, either because they are no longer being actively developed or because the authors are careful to only change code related to genuine bugs or new, non-breaking, features. Obvious candidates are keyval, graphicx, etc.: probably there is actually quite a decent list, depending on your requirements.

In the case of the LaTeX3 packages l3kernel and l3packages, 'stable' does not extend as far as 'you will never have to make a change to a document using them', at least at this stage. What it means is that the team will not be making 'arbitrary' changes and will document/announce when this happens. Most of l3kernel is 'done', with the plans primarily focused on addition of new functionality rather than altering existing code. However there are a few places where we know some change may be required, and that will be announced on the LaTeX-L mailing list and documented. Even within these changes, 'breaking' (non-back-compatible) alterations will be small in number, but there is at least one of them we still need to do.

In the case of xparse, \docAuxCommand*{DeclareDocumentCommand} and so on are 'stable' in the sense that they will only be augmented, not removed, but there could be some changes on the more esoteric functions (for example, there are questions centred on the \textbf{g} argument type).

Thus 'stable enough' depends on your use case. If you can live with 'will have to make very occasional changes based on documented and scheduled updates' then expl3 is entirely usable. (I and others use if routinely in packages.) On the other hand, if you want 'this code must work with no changes with all future releases of support code' then we are not quite there yet.
\end{latexquotation}

\section{Getting started}

Other than the obvious of making sure you have the latest distribution from the LaTeX3 repository, 
the first step is to understand the conventions used by the \LaTeX3 developers. Macros are termed \meta{functions} and \meta{variables}. Macro names in general use the underscore and the colon in their names.
This is by design and to be honest is part of what many developers are unhappy about. It does cut down on the readability of the code and the longer names are more difficult to remember. This type of naming convention is similar to Hungarian notation, in which the name of a variable or function indicates its type  or its intended use and it does not have a lot of friends.


Consider the \tex primitive \docAuxCommand*{meaning}. In \latex3 it has been remapped to \docAuxCommand*{token_to_meaning:N}. Similarly \docAuxCommand*{scan_stop:} has been let to \docAuxCommand*{relax}.

\begin{texexample}{Getting started}{ex:meaning}
\ExplSyntaxOn
\def\somevar{one}
\token_to_meaning:N \scan_stop:  \\
\meaning\somevar \\
\token_to_meaning:N \somevar \\
\token_to_meaning:N \token_to_meaning:N
\ExplSyntaxOff
\end{texexample}

The part that comes after the colon is termed the \emph{function signature}. For example in |token_to_meaning:N|, the function signature is the \textbf{N}. The individual letter “N” is termed the argument specifier. Another important part is the prefix of the functions. There are some exceptions but the prefix normally indicates the module where the macro has been defined. So |\token_to_meaning:N|  can be found in the |l3token| package.\footnote{The term module and package are used interchangeably by the \latex3 Team.}

Consider the definition of a simple function  |\phd_print_xy:nn| that accepts two values $x,y$ and prints them. This can be defined by one of the |cs_| type functions.

One way we could have defined the macro using  \tex would be:

\begin{teXXX}
\def\phdprint #1#2{x#1 y#2}
\end{teXXX}

Using \latexe we would have probably used |\newcommand| and if the definition was internal to a package used an |@|. 

\begin{teXXX}
\makeatletter
\newcommand\phd@print [2] {x#1 y#2}
\makeatother
\end{teXXX}

In \latex3 we would use |\cs_set_no_par:Npn|.

\begin{teXXX}
\cs_set_nopar:Npn \phd_print_xy:nn #1#2 { x #1 y #2 }
\end{teXXX}

So what is this mysterious |\cs_set_nopar:Npn|? We can find out by peeking at its meaning. This is shown in Example~\ref{ex:somemeaning}. As you can see behind the new dress is Knuth’s same old |\def|.

\begin{texexample}{The meaning of a command}{ex:somemeaning}
\ExplSyntaxOn
\token_to_meaning:N  \cs_set_nopar:Npn
\ExplSyntaxOff
\end{texexample}

But first let us examine the |:Npn| part of the |\cs_set_no_par:Npn| more carefully. What this means is the macro has three arguments. The first one is N-type which is a \tex token. The second one is p-type, which denotes normal \tex parameters such as |#1#2|. Lastly the n-type can be either a single token or a bracketted parameter. 

There are many more argument specifiers. Functions can be found with different argument specifiers and these are termed \emph{variants}. Recall that a macro can be defined using |\def|, |\edef| or a |\csname| construct. The argument specifier to the |\cs_setnopar| can be varied to achieve it. 

\begin{texexample}{The meaning of a command}{ex:somemeaning}
\ExplSyntaxOn
\token_to_meaning:N  \cs_set_nopar:Npx\\
\token_to_meaning:N  \cs_set_nopar:cpx\\
\ExplSyntaxOff
\end{texexample}

Consider the use  of a |\csname| construct to define our |\phd_print_xy:nn| macro. The example that follows

\begin{texexample}{ex:csname}{ex:csname}

\ExplSyntaxOn
\expandafter\def\csname phd_print_xy:nn\endcsname #1 #2{x#1 y#2}

\token_to_meaning:N \phd_print_xy:nn\\

\cs_set_nopar:cpx {phd_print_xy:nn} #1 #2 {x#1 y#2}
\token_to_meaning:N \phd_print_xy:nn\\
\ExplSyntaxOff
\end{texexample} 

By using \latex3 functions, we do not need to use the |\expandafter| macro. The macros are generally longer but the overall code is shorter.

So far we have used the |\token_to_meaning:N|. \latex3 offers similar commands to get the argument specification, the prefix and the replacement specification. When we specify a macro in \latex3 we can capture all its constituent parts and handle them individually if we want.

\begin{texexample}{Dissecting a macro}{}
\ExplSyntaxOn
\cs_set_nopar:Npn \phd_print_xy:nn #1#2! { x #1 y #2 }
\token_to_meaning:N \phd_print_xy:nn \\
\token_get_arg_spec:N \phd_print_xy:nn  \\
\token_get_prefix_spec:N \phd_print_xy:nn\\
\token_get_replacement_spec:N \phd_print_xy:nn\\
\ExplSyntaxOff
\end{texexample}


Some argue that the syntax is not syntactic sugar but syntactic cyanide that changes the look and feel both of \latexe and \tex command macros. You should think of |expl3| as a new computer language. It does introduce consistency and offers a full repertoire of tools. The syntactic strangeness of the language does introduce barriers to mastering it, but the advantages far outweigh the difficulties of the language.


The eye tends to miss the argument specifier, it is important to note that the macro
name is \cmd{\test\_something:nn} and not \cmd{\test\_something} and the factory command is |\cs_new:Npn| and not |\cs_new|. If you have been programming using traditional macros this is a common mistake that you will accidentally make and you will get an |error unknown| message.

\section{Where from here}

The chapters of this book follow a logical sequence for learning the language, although most of them can be read as stand alone. 

The steps in learning any computer language require a logical sequence of study:

\begin{enumerate}
\item Understanding the syntax
\item Variables and datatypes
\item Numbers and assignments
\item Control Structures
\item Functions
\item Data structures
\item Ecosystem
\end{enumerate}

In the next chapter we would study the creation of functions in more detail. This is the most important skill to master before you proceed with the rest of the programming constructs, such as iteration, arithmetic operations etc.



\chapter{Defining Functions and Variables}

\section{Defining functions}
There are two main methods to define functions. In the first method you are required to use parameter tex, whereas in the second this can be left out, as it can be inferred from the argument specification of the function being defined. The functions used to create other functions can be found in both forms. For example:

\begin{texexample}{Using parameter text}{}
\ExplSyntaxOn
\cs_set_nopar:Npn \phd_print:n #1 {#1}
\token_to_meaning:N \phd_print:n\\

\cs_set_nopar:Nn  \phd_print:n  {#1}


\token_to_meaning:N \phd_print:n\\
\ExplSyntaxOff
\end{texexample}



 Functions can be created with no requirement that they are declared
 first (in contrast to variables, which must always be declared).\footnote{This primarily refers to variables that require a \tex register.}
 Declaring a function before setting up the code means that the name
 chosen will be checked and an error raised if it is already in use.
 The name of a function can be checked at the point of definition using
 the \docAuxCommand*{cs_new}\ldots functions: this is recommended for all
 functions which are defined for the first time.

 There are three primary ways to define new functions, using |new|, |set| or |gset| variations.  The first one is similar to the \latexe |\newcommand|, and produces macros that will generate an error if there is an attempt to redefine them. The other two are variations of the |\def or \edef| and |\gdef or \xdef| \tex commands.
 
 All classes define a function to expand to the substitution text.
 Within the substitution text the actual parameters are substituted
 for the formal parameters (|#1|, |#2|, \ldots).
 
 \begin{description}
   \item[\texttt{new}]
     Create a new function with the \texttt{new} scope,
     such as \docAuxCommand* {cs_new:Npn}.  The definition is global and will result in
     an error if it is already defined.
   \item[\texttt{set}]
     Create a new function with the \texttt{set} scope,
     such as \docAuxCommand* {cs_set:Npn}. The definition is restricted to the current
     \TeX{} group and will not result in an error if the function is already
     defined.
   \item[\texttt{gset}]
     Create a new function with the \texttt{gset} scope,
     such as \docAuxCommand* {cs_gset:Npn}. The definition is global and
     will not result in an error if the function is already defined.
 \end{description}

  Finally, the functions in
 Subsections~\ref{sec:l3basics:defining-new-function-1}~and
 \ref{sec:l3basics:defining-new-function-2} are primarily meant to define
 \emph{base functions} only. Base functions can only have the following
 argument specifiers:
 \begin{description}
   \item[|N| and |n|] No manipulation.
   \item[|T| and |F|] Functionally equivalent to |n| (you are actually
     encouraged to use the family of |\prg_new_conditional:| functions
     described in Section~\ref{sec:l3prg:new-conditional-functions}).
   \item[|p| and |w|] These are special cases.
 \end{description}



 Within each set of scope there are different ways to define a function.
 The differences depend on restrictions on the actual parameters and
 the expandability of the resulting function.
 \begin{description}
   \item[\texttt{nopar}]
      Create a new function with the \texttt{nopar} restriction,
      such as \docAuxCommand*{cs_set_nopar:Npn}. The parameter may not contain
      \docAuxCommand*{par} tokens.
   \item[\texttt{protected}]
      Create a new function with the \texttt{protected} restriction,
      such as \docAuxCommand*{cs_set_protected:Npn}. The parameter may contain
      \docAuxCommand*{par} tokens but the function will not expand within an
      \texttt{x}-type expansion.
 \end{description}
 
 
\subsection{Defining new functions using parameter text}

Theses function are \TeX ish in style, as compared to those functions that use the signature to automatically detect the number of parameters and are more \LaTeX-like. They are mainly used with the |:Npn| signature specification.

\begin{texexample}{Using parameter text}{}
\ExplSyntaxOn
\cs_new:Npn \phd_print:n #1 {#1}

\token_to_meaning:N \cs_new:Npn\\
\token_to_meaning:N \phd_print:n\\
\ExplSyntaxOff
\end{texexample}

\begin{docCommand}{cs_new:Npn} {\meta{function} \meta{parameters} \marg{code}}
Creates \meta{function} to expand to \meta{code} as replacement text. Within the \meta{code}, the
\meta{parameters} (\#1, \#2, etc.) will be replaced by those absorbed by the function. The
definition is \textbf{global} and an error will result if the \meta{function} is already defined.
Variants with |cpn,Npx,cpx| are predefined by the kernel.
\end{docCommand}

The |:Npn| form can also be used even if there is no parameter text. However this is considered a constant variable and is preferred to be coded as a |tl| such.

\begin{texexample}{Usage of the macro}{ex:csnew}
\ExplSyntaxOn
  \cs_new:Npn \copyrightfootnote: 
    {
      \footnotetext{Copyright~(2014-2015)~of~Yiannis~Lazarides,~distributed~
      under~the~\LaTeX{}~Project~Public~License~(LPPL).}
    }
  \copyrightfootnote:
\ExplSyntaxOff
\end{texexample}

An important point to note is if you use the function signature type you will get an error if the trailing |:| is not used in the macro name. 

\begin{teXXX}
\cs_new:Nn \copyrightafootnote 
  {
    ...
  }
\copyrightafootnote
\end{teXXX}

This will produce an error:\ExplSyntaxOn\copyrightfootnote:\ExplSyntaxOff

\begin{smallverbatim}
! LaTeX error: "kernel/missing-colon"
! Function '\copyrightafootnote' contains no ':'.
! See the LaTeX3 documentation for further information.
! For immediate help type H <return>.
\end{smallverbatim}

If the function is redefined, it will produce an error, similar to \latexe |\newcommand|. However, do note that the |set| family of commands can silently overwrite it. 

\begin{texexample}{Usage of the macro \protect\string\cs\_gset:Npn}{ex:csnew}
\ExplSyntaxOn
\cs_gset:Npn \copyrightfootnote: {\footnotetext{Copyright~(2014-2015)~of~Yiannis~Lazarides,~distributed~
under~the~\LaTeX{}~Project~Public~License~(LPPL).}}
\copyrightfootnote:
\ExplSyntaxOff
\end{texexample}

\begin{docCommand}{cs_new_nopar:Npn} {\meta{function} \meta{parameters} \marg{code}}
Creates \meta{function} to expand to \meta{code} as replacement text. Within the \meta{code}, the
\meta{parameters} (\#1, \#2, etc.) will be replaced by those absorbed by the function. When the
\meta{function} is used the hparametersi absorbed cannot contain \par tokens. The definition
is global and an error will result if the \meta{function} is already defined.
\end{docCommand}

\begin{texexample}{Meaning}{}
\ExplSyntaxOn
\token_to_meaning:N \cs_new_nopar:Npn
\ExplSyntaxOff
\end{texexample}

\begin{docCommand}{cs_new_protected:Npn}{\meta{function} \meta{parameters} \marg{code}}
Creates \meta{function} to expand to \meta{code} as replacement text. Within the hcodei, the
hparametersi (\#1, \#2, etc.) will be replaced by those absorbed by the function. The
\meta{function} will not expand within an x-type argument. The definition is global and an
error will result if the hfunctioni is already defined.
\end{docCommand}

\begin{docCommand}{cs_new_protected_nopar:Npn}{\meta{function} \meta{parameters} \marg{code}}
Creates \meta{function} to expand to \meta{code} as replacement text. 
When the \meta{function} is used the \meta{parameters} absorbed cannot contain \docAuxCommand*{par} tokens. The hfunctioni
will not expand within an x-type argument. The definition is global and an error will
result if the \meta{function} is already defined.
\end{docCommand}

This brings us to the end of the |new| type functions that can be used for function definitions. They all have variants of the form |cpn| and |cpx| and the base function for edef also is available. You can consult the manual for more definitions.

\subsubsection{The set type functions}

The rest of the commands are variations using the |set| form of function creating macros. These do not issue a 
warning if redefined.

 \begin{docCommand}{cs_set:Npn} {\meta{function} \meta{parameters} \marg{code}}
   Sets \meta{function} to expand to \meta{code} as replacement text.
   Within the \meta{code}, the \meta{parameters} (|#1|, |#2|,
   \emph{etc.}) will be replaced by those absorbed by the function.
   The assignment of a meaning to the \meta{function} is restricted to
   the current \TeX{} group level.
\end{docCommand}

\begin{texexample}{Meaning}{}
\ExplSyntaxOn
\token_to_meaning:N \cs_set:Npn
\ExplSyntaxOff
\end{texexample}

As can be seen from the example this is |\protected \long \def|. The |\cs_set_nopar:Npn| in the maeaning in the example is described next and is simply an equivalent function to |\def|.

 \begin{docCommand} {cs_set_nopar:Npn}{\meta{function} \meta{parameters} \marg{code}}
   Sets \meta{function} to expand to \meta{code} as replacement text.
   Within the \meta{code}, the \meta{parameters} (|#1|, |#2|,
   \emph{etc.}) will be replaced by those absorbed by the function.
   When the \meta{function} is used the \meta{parameters} absorbed
   cannot contain \cs{par} tokens. The assignment of a meaning
   to the \meta{function} is restricted to the current \TeX{} group
   level.
 \end{docCommand}
 
 \begin{texexample}{Meaning \textbackslash cs\_set\_nopar:Npn}{}
\ExplSyntaxOn
\token_to_meaning:N \cs_set_nopar:Npn
\ExplSyntaxOff
\end{texexample}
 

\begin{docCommand}{cs_set_protected:Npn} {\meta{function} \meta{parameters} \marg{code}}
   Sets \meta{function} to expand to \meta{code} as replacement text.
   Within the \meta{code}, the \meta{parameters} (|#1|, |#2|,
   \emph{etc.}) will be replaced by those absorbed by the function.
   The assignment of a meaning to the \meta{function} is restricted to
   the current \TeX{} group level. The \meta{function} will
   not expand within an \texttt{x}-type argument.
 \end{docCommand}
 \begin{texexample}{Meaning \textbackslash cs\_set\_protected:Npn}{}
 \ExplSyntaxOn
 \token_to_meaning:N \cs_set_protected:Npn
\ExplSyntaxOff
\end{texexample}
 


\begin{docCommand}{cs_set_protected_nopar:Npn}{\meta{function} \meta{parameters} \marg{code}}
   Sets \meta{function} to expand to \meta{code} as replacement text.
   Within the \meta{code}, the \meta{parameters} (|#1|, |#2|,
   \emph{etc.}) will be replaced by those absorbed by the function.
   When the \meta{function} is used the \meta{parameters} absorbed
   cannot contain \cs{par} tokens. The assignment of a meaning
   to the \meta{function} is restricted to the current \TeX{} group
   level. The \meta{function} will not expand within an
   \texttt{x}-type argument.
\end{docCommand}
\begin{texexample}{Meaning \textbackslash cs\_set\_protected\_nopar:Npn}{}
\ExplSyntaxOn
\token_to_meaning:N \cs_set_protected_nopar:Npn
\ExplSyntaxOff
\end{texexample}
 
Next the above are made available by the \latex3 kernel but all in the |global| form of the command. The syntax is identical except they use |cs_gset|.


\begin{docCommand} {cs_gset:Npn}{\meta{function} \meta{parameters} \marg{code}}
   Globally sets \meta{function} to expand to \meta{code} as replacement
   text. Within the \meta{code}, the \meta{parameters} (|#1|, |#2|,
  \emph{etc.}) will be replaced by those absorbed by the function.
  The assignment of a meaning to the \meta{function} is \emph{not}
   restricted to the current \TeX{} group level: the assignment is
   global.
\end{docCommand}
\begin{texexample}{Meaning \textbackslash cs\_gset:Npn}{}
\ExplSyntaxOn
\token_to_meaning:N \cs_gset:Npn
\ExplSyntaxOff
\end{texexample}

\begin{docCommand}{cs_gset_nopar:Npn} {\meta{function} \meta{parameters} \marg{code}}
   Globally sets \meta{function} to expand to \meta{code} as replacement
   text. Within the \meta{code}, the \meta{parameters} (|#1|, |#2|,
   \emph{etc.}) will be replaced by those absorbed by the function.
   When the \meta{function} is used the \meta{parameters} absorbed
   cannot contain \cs{par} tokens. The assignment of a meaning to the
   \meta{function} is \emph{not} restricted to the current \TeX{}
   group level: the assignment is global.
\end{docCommand}
\begin{texexample}{Meaning \textbackslash cs\_gset\_nopar:Npn}{}
\ExplSyntaxOn
\token_to_meaning:N \cs_gset_nopar:Npn
\ExplSyntaxOff
\end{texexample}


\begin{docCommand} {cs_gset_protected:Npn} {\meta{function} \meta{parameters} \marg{code}}
   Globally sets \meta{function} to expand to \meta{code} as replacement
   text. Within the \meta{code}, the \meta{parameters} (|#1|, |#2|,
   \emph{etc.}) will be replaced by those absorbed by the function.
   The assignment of a meaning to the \meta{function} is \emph{not}
   restricted to the current \TeX{} group level: the assignment is
   global. The \meta{function} will not expand within an
   \texttt{x}-type argument.
\end{docCommand}
\begin{texexample}{Meaning \textbackslash cs\_gset\_protected:Npn}{}
\ExplSyntaxOn
\token_to_meaning:N \cs_gset_protected:Npn
\ExplSyntaxOff
\end{texexample}

\begin{docCommand}{cs_gset_protected_nopar:Npn} {\meta{function} \meta{parameters} \marg{code}}
   Globally sets \meta{function} to expand to \meta{code} as replacement
   text. Within the \meta{code}, the \meta{parameters} (|#1|, |#2|,
   \emph{etc.}) will be replaced by those absorbed by the function.
   When the \meta{function} is used the \meta{parameters} absorbed
   cannot contain \cs{par} tokens. The assignment of a meaning to the
   \meta{function} is \emph{not} restricted to the current \TeX{}
   group level: the assignment is global. The \meta{function} will
   not expand within an \texttt{x}-type argument.
\end{docCommand}
\begin{texexample}{Meaning \textbackslash cs\_gset\_protected\_nopar:Npn}{}
\ExplSyntaxOn
\token_to_meaning:N \cs_gset_protected_nopar:Npn
\ExplSyntaxOff
\end{texexample}

This brings us to the end of the functions available to the developer for defining macros. It’s a lot of them. In the next section some more functions are defined, this time using the signature of the function the function are created automatically without the need to type in the parameter text.


\subsection{Defining new functions using the signature}

The functions outlined below have a simpler form in that they create other commands without the need to specify their arguments. The number of parameters is detected automatically from the function signature. Which method is the best is obvious up to the user preferences.\footnote{See discussion at SX.TX \protect{\url{http://tex.stackexchange.com/questions/240675/differences-in-latex3-function-generation-methods}}} 


\begin{docCommand}{cs_new:Nn}{\meta{function}\marg{code}}
Creates \meta{function} to expand to \meta{code} as replacement text. A nice feature is that within the \meta{code}
the number of parameters is detected automatically from the function signature. These \meta{parameters} (\#1, \#2, etc.) will be replaced by those absorbed by the function. The definition is global and an error will result if the \meta{function} is already defined.\footnote{The definitions of the commands have been taken mostly verbatim from the documentation of the package.}


\begin{texexample}{Signature}{ex:signature}
\ExplSyntaxOn
\cs_new:Nn \exampleone:nn {}
\cs_new:Nn \exampletwo:nn{#1 #2}
\exampleone:nn {one}{two}

\exampletwo:nn{one }{two}

\texttt\textbackslash\cs_to_str:N\exampleone:nn
\ExplSyntaxOff
\end{texexample}
\end{docCommand}

 
 
 
\begin{docCommand}{cs_new_nopar:Nn}{\meta{function} \marg{code}}
   Creates \meta{function} to expand to \meta{code} as replacement text.
   Within the \meta{code}, the number of \meta{parameters} is detected
   automatically from the function signature. These \meta{parameters}
   (|#1|, |#2|, \emph{etc.}) will be replaced by those absorbed by the
   function.  When the \meta{function} is used the \meta{parameters}
   absorbed cannot contain \docAuxCommand*{par} tokens. The definition is global and
   an error will result if the \meta{function} is already defined.
 \end{docCommand}

\begin{docCommand}{cs_new_protected:Nn}{\meta{function} \marg{code}}
   Creates \meta{function} to expand to \meta{code} as replacement text.
   Within the \meta{code}, the number of \meta{parameters} is detected
   automatically from the function signature. These \meta{parameters}
   (|#1|, |#2|, \emph{etc.}) will be replaced by those absorbed by the
   function. The \meta{function} will not expand within an \texttt{x}-type
   argument. The definition is global and
   an error will result if the \meta{function} is already defined.
\end{docCommand}


%
% \begin{function}
%   {
%     \docAuxCommand*_new_protected_nopar:Nn, \docAuxCommand*_new_protected_nopar:cn,
%     \docAuxCommand*_new_protected_nopar:Nx, \docAuxCommand*_new_protected_nopar:cx
%   }
%   \begin{syntax}
%     \docAuxCommand*{cs_new_protected_nopar:Nn} \meta{function} \Arg{code}
%   \end{syntax}
%   Creates \meta{function} to expand to \meta{code} as replacement text.
%   Within the \meta{code}, the number of \meta{parameters} is detected
%   automatically from the function signature. These \meta{parameters}
%   (|#1|, |#2|, \emph{etc.}) will be replaced by those absorbed by the
%   function.  When the \meta{function} is used the \meta{parameters}
%   absorbed cannot contain \docAuxCommand*{par} tokens. The \meta{function} will not
%   expand within an \texttt{x}-type argument. The definition is global and
%   an error will result if the \meta{function} is already defined.
% \end{function}

Similarly to the |cs_new| commands the |cs_set| functions create other commands, this time
with a local scope. This pattern is followed right through the kernel.

 \begin{docCommand}{cs_set:Nn}{\meta{function}\marg{code}}
   Sets \meta{function} to expand to \meta{code} as replacement text.
   Within the \meta{code}, the number of \meta{parameters} is detected
   automatically from the function signature. These \meta{parameters}
   (|#1|, |#2|, \emph{etc.}) will be replaced by those absorbed by the
   function.
   The assignment of a meaning to the \meta{function} is restricted to
   the current \TeX{} group level.
 \end{docCommand}

\begin{docCommand}{cs_set_nopar:Nn}{\meta{function}\marg{code}}
   Sets \meta{function} to expand to \meta{code} as replacement text.
   Within the \meta{code}, the number of \meta{parameters} is detected
   automatically from the function signature. These \meta{parameters}
   (|#1|, |#2|, \emph{etc.}) will be replaced by those absorbed by the
   function.  When the \meta{function} is used the \meta{parameters}
   absorbed cannot contain \docAuxCommand*{par} tokens.
   The assignment of a meaning to the \meta{function} is restricted to
   the current \TeX{} group level. This is the \tex primitive \docAuxCommand*{def}
\end{docCommand}

\begin{teXXX}
\tex_let:D \cs_set_nopar:Npn \tex_def:D
748 \tex_let:D \cs_set_nopar:Npx \tex_edef:D
749 \etex_protected:D \cs_set_nopar:Npn \cs_set:Npn
750                     { \tex_long:D \cs_set_nopar:Npn }
751 \etex_protected:D \cs_set_nopar:Npn \cs_set:Npx
752                   { \tex_long:D \cs_set_nopar:Npx }
753 \etex_protected:D \cs_set_nopar:Npn \cs_set_protected_nopar:Npn
754 { \etex_protected:D \cs_set_nopar:Npn }
755 \etex_protected:D \cs_set_nopar:Npn \cs_set_protected_nopar:Npx
756 { \etex_protected:D \cs_set_nopar:Npx }
757 \cs_set_protected_nopar:Npn \cs_set_protected:Npn
758 { \etex_protected:D \tex_long:D \cs_set_nopar:Npn }
759 \cs_set_protected_nopar:Npn \cs_set_protected:Npx
760 { \etex_protected:D \tex_long:D \cs_set_nopar:Npx }
\end{teXXX}
\ExplSyntaxOn
\meaning\cs_new:Npn
\ExplSyntaxOff


\begin{docCommand}{cs_set_protected:Nn}{\meta{function}\marg{code}}
   Sets \meta{function} to expand to \meta{code} as replacement text.
   Within the \meta{code}, the number of \meta{parameters} is detected
   automatically from the function signature. These \meta{parameters}
   (|#1|, |#2|, \emph{etc.}) will be replaced by those absorbed by the
   function. The \meta{function} will not expand within an \texttt{x}-type
   argument.
   The assignment of a meaning to the \meta{function} is restricted to
   the current \TeX{} group level.
 \end{docCommand}

\begin{docCommand}{cs_set_protected_nopar:Nn}{ \meta{function} \marg{code}}
   Sets \meta{function} to expand to \meta{code} as replacement text.
   Within the \meta{code}, the number of \meta{parameters} is detected
   automatically from the function signature. These \meta{parameters}
   (|#1|, |#2|, \emph{etc.}) will be replaced by those absorbed by the
   function.  When the \meta{function} is used the \meta{parameters}
   absorbed cannot contain \docAuxCommand*{par} tokens. The \meta{function} will not
   expand within an \texttt{x}-type argument.
   The assignment of a meaning to the \meta{function} is restricted to
   the current \TeX{} group level.
 \end{docCommand}

The next commands create functions with global scope.

 \begin{docCommand}{cs_gset:Nn}{ \meta{function} \marg{code}}
   Sets \meta{function} to expand to \meta{code} as replacement text.
   Within the \meta{code}, the number of \meta{parameters} is detected
   automatically from the function signature. These \meta{parameters}
   (|#1|, |#2|, \emph{etc.}) will be replaced by those absorbed by the
   function.
   The assignment of a meaning to the \meta{function} is  global.
 \end{docCommand}

 \begin{docCommand}{cs_gset_nopar:Nn}{ \meta{function} \marg{code}}
   Sets \meta{function} to expand to \meta{code} as replacement text.
   Within the \meta{code}, the number of \meta{parameters} is detected
   automatically from the function signature. These \meta{parameters}
   (|#1|, |#2|, \emph{etc.}) will be replaced by those absorbed by the
   function.  When the \meta{function} is used the \meta{parameters}
   absorbed cannot contain \docAuxCommand*{par} tokens.
   The assignment of a meaning to the \meta{function} is global.
 \end{docCommand}
 

\section{Copying control sequences}

Control sequences (not just functions as defined above) can be set to have the same
meaning using the functions described here. Making two control sequences equivalent
means that the second control sequence is a copy of the first (rather than a pointer to
it). Thus the old and new control sequence are not tied together: changes to one are not
reflected in the other. These are syntactic replacements for |\let|.

\begin{texexample}{Let}{}
\ExplSyntaxOn
\cs_set_nopar:Nn \testa: {AAA}
\cs_set_eq:NN\testb: \testa:
\token_to_meaning:N \testa:  \\
\cs_set_nopar:Nn \testa: {BBBB}
\testb:  \\
\token_to_meaning:N \testb:  \\
\token_to_meaning:N \testa:  \\
\testa:\\
\testb: \\
\meaning\cs_set_eq:NN

% check if equal to \let
\token_to_meaning:N \let\\
\token_to_meaning:N \cs_set_equal:NN
\ExplSyntaxOff
\end{texexample}

 In the following text \enquote{cs} is used as an abbreviation for
 \enquote{control sequence}.

 \begin{docCommand}{cs_new_eq:NN} {\meta{cs1} \meta{cs2}}
   Globally creates \meta{control sequence 1} and sets it to have the same
   meaning as \meta{control sequence 2} or |<token>|.
   The second control sequence may
   subsequently be altered without affecting the copy.
\end{docCommand}


\begin{docCommand}{cs_set_eq:NN} {\meta{cs1} \meta{cs2}}
   Sets \meta{control sequence1} to have the same meaning as
   \meta{control sequence2} (or |<token>|).
   The second control sequence may subsequently be
   altered without affecting the copy. The assignment of a meaning
   to the \meta{control sequence1} is restricted to the current
   \TeX{} group level.
 \end{docCommand}


\begin{docCommand} {cs_gset_eq:NN} {\meta{cs1} \meta{cs2}}
   Globally sets \meta{control sequence1} to have the same meaning as
   \meta{control sequence2} (or |<token>|).
   The second control sequence may subsequently be
   altered without affecting the copy. The assignment of a meaning to
   the \meta{control sequence1} is \emph{not} restricted to the current
   \TeX{} group level: the assignment is global.
\end{docCommand}

\section{Undefining control sequences}

There are occasions where control sequences need to be deleted. This is handled in a
very simple manner by the use of 
|\cs_undefine:N| \meta{control sequence},
which sets \meta{control sequence} to be globally |undefined|.

\begin{texexample}{Undefining control sequences}{ex:undefine}
\ExplSyntaxOn
\cs_set_nopar:Npn \testa: {AAA}
\cs_set_nopar:cpn {testb} {AAA}

\cs_undefine:N \testa:
\cs_undefine:c {testb}
\token_to_meaning:N \cs_undefine:N\\

\token_to_meaning:N \testa:\\
\token_to_meaning:c {testb}\\
\token_to_meaning:N \token_to_meaning:c
\ExplSyntaxOff
\end{texexample}

The function would simply set the command to the \tex primitive |undefine|, as can be seen from the example.
There is another group of commands associated with constructor functions.

\section{Converting to and from control sequences}

\begin{docCommand}{cs_if_exist_use:N} {\meta{control sequence}}
Tests whether the \meta{control sequence} is currently defined (whether as a function or another
control sequence type), and if it does inserts the \meta{control sequence} into the input stream.
\end{docCommand}

\begin{docCommand}{cs_if_exist_use:NTF} {\meta{control sequence}}
Tests whether the \meta{control sequence} is currently defined (whether as a function or another
control sequence type), and if it does inserts the \meta{control sequence} into the input stream
followed by the \meta{true code}.
\end{docCommand}

\begin{texexample}{Converting to and from control sequences}{ex:ifexists}
\ExplSyntaxOn
\cs_if_exist_use:NTF \test {}{\FALSE}
\ExplSyntaxOff
\end{texexample}

Note that numerous times, I have typed |\cs_if_exists_use:NTF| rather than the more grammatical  |\cs_if_exist_use:NTF| with consequent errors. Grammar is hardwired in the brain and it requires mental effort to write ungrammatical commands. This is an issue that needs to be addressed by the \latex3 developers. 

The famous |\csname| is mapped in this section of the module as well. Unpredictably, it got a shorter name, but a weird suffix |w|! It deserves both as it is the workhorse of \tex. The remapped commands are formally described in the manual as shown below:

\begin{docCommand}{cs:w} {\meta{control sequence name} \texttt{cs\_end:}}
Converts the given \meta{control sequence name} into a single control sequence token. This
process requires one expansion. The content for \meta{control sequence name} may be literal
material or from other expandable functions. The \meta{control sequence name} must, when
fully expanded, consist of character tokens which are not active: typically, they will be
of category code 10 (space), 11 (letter) or 12 (other), or a mixture of these.
\end{docCommand}


\section{User Commands}

All the commands above are at the programming level. For the development of user commands the \pkgname{xparse} package provides some extremely useful commands. These are dealt under \nameref{ch:xparse}
on page \pageref{ch:xparse}.

\begin{teXXX}
\NewDocumentCommand{\kant}{s>{\SplitArgument{1}{-}}O{1-7}}
  {
   \group_begin:
   \IfBooleanTF{#1} (*@\label{starargument}@*)
     { \cs_set_eq:NN \kgl_par: \kgl_star: }
     { \cs_set_eq:NN \kgl_par: \kgl_nostar: }
     \kgl_process:nn #2
    \kgl_print:
   \group_end:
  }
\end{teXXX}

In Line~\ref{starargument} we test for the star version of the command and then we continue examining the optional argument |O{1-7}|, but first and here is the magic, we have passed the argument through a pre-processing macro named |\SplitArgument|, which has captured the splitted argument and placed it, into two braced macros. It then passes it to a second macro |\getwords| that expects two mandatory aruguments and which handles the typesetting of the two words.
    
\begin{texexample}{Split Argument}{}    
\NewDocumentCommand{\separatewords}{>{\SplitArgument{1}{-}}m}{\getwords#1}
\NewDocumentCommand{\getwords}{ m m }{First word:#1  Second~Word:#2}
\separatewords{mail-coach}

\separatewords{mail}

\end{texexample}    

A similar example see TX.SX.\footnote{\protect{\url{http://tex.stackexchange.com/questions/154941/new-command-in-tex-for-fraction/154950\#154950}}}


\chapter{LaTeX3 Control Structures}
 \section{The boolean data type}

 This section describes a boolean data type which is closely
 connected to conditional processing as sometimes you want to
 execute some code depending on the value of a switch
 (\emph{e.g.},~draft/final) and other times you perhaps want to use it as a
 predicate function in an |if_predicate:w| test. The problem of the
 primitive \docAuxCommand*{if_false:} and \docAuxCommand*{if_true:} tokens is that it is not
 always safe to pass them around as they may interfere with scanning
 for termination of primitive conditional processing. In \latex3
 two canonical booleans ar employed: \docAuxCommand*{c_true_bool} or
\docAuxCommand{c_false_bool}. Besides preventing problems as described above. This also let
to the implementation of  a simple boolean parser supporting the
 logical operations And, Or, Not, \emph{etc.}\ which can then be used on
 both the boolean type and predicate functions.

 All conditional |\bool_| functions except assignments are expandable
 and expect the input to also be fully expandable (which will generally
 mean being constructed from predicate functions, possibly nested).
 
Before a boolean can be used it needs to be created with \docAuxCommand{bool_new:N}, but first let us make sure we understand what a boolean is. A Boolean data type is a data type, having two values (usually denoted \emph{true} and \emph{false}), intended to represent the truth values of logic and Boolean algebra. It is named after George Boole, who first defined an algebraic system of logic in the mid 19th century. 

So how does \latex3 construct a boolean? If we examine the code, which we will in a small example, we can see that a boolean variable is just another macro that either stores 0 or 1. If the value is odd then the boolean is \emph{true} else the boolean is \emph{false}. 

\begin{teXXX}
\tex_chardef:D \c_true_bool = 1 ~
\tex_chardef:D \c_false_bool = 0 ~
\end{teXXX}

\begin{teXXX}
 \cs_new_protected:Npn \bool_new:N #1 { \cs_new_eq:NN #1 \c_false_bool }
 \cs_generate_variant:Nn \bool_new:N { c }
\end{teXXX}

When a new boolean is constructed it is always set to false, as is evident from its code. 

Here is the formal syntax of the |\bool_new:N| function.

 \begin{docCommand}{bool_new:N}{\meta{boolean}}
   Creates a new \meta{boolean} or raises an error if the
   name is already taken. The declaration is global. The
   \meta{boolean} will initially be \texttt{false}. Once the boolean is created
   it can be set to logical true or false using \docAuxCommand*{bool_set_false:N} and \docAuxCommand*{bool_set_true:N}.
 \end{docCommand}
 
\begin{texexample}{Booleans}{}
\ExplSyntaxOn
\bool_new:N \mybool
\bool_set_false:N \mybool
\bool_if:NTF\mybool { \PASS } { \FAIL }
\ExplSyntaxOff
\end{texexample}
 

 
The real strength of the \latex~3 macros are the convenience of providing for |Or| and |And|
operations, negation etc.  and for its ability to evaluate fully boolean expressions. 

\begin{docCommand}{bool_if:nTF}{\marg{boolean expression} \marg{true code} \marg{false code}}
   Tests the current truth of \meta{boolean expression}, and
   continues expansion based on this result. The
   \meta{boolean expression} should consist of a series of predicates
   or boolean variables with the logical relationship between these
   defined using |&&| (\enquote{And}), \verb"||" (\enquote{Or}),
   |!| (\enquote{Not}) and parentheses. Minimal evaluation is used
   in the processing, so that once a result is defined there is
   not further expansion of the tests. 
\end{docCommand}   



\begin{texexample}{Booleans}{}
\ExplSyntaxOn
\bool_new:N\chapterfloat
\bool_new:N\numberfloat
\bool_set_false:N\chapterfloat
\bool_set_true:N\numberfloat

\bool_if:nTF {\chapterfloat || \numberfloat}  { \TRUE }{ \FALSE }

\bool_if:nTF {\chapterfloat && \numberfloat}  { \TRUE }{ \FALSE }

\ExplSyntaxOff
\end{texexample}

\subsection{\textbackslash if\_meaning}

The primitive |ifx| conditional has an equivalent in \latex3. This is called more semantically \docAuxCommand*{if_meaning:w}. This compares two tokens based on their meaning.



\begin{texexample}{Test ifx}{}
\ExplSyntaxOn
\group_begin:
  \cs_set_nopar:Npn \a: {BBB}
  \cs_set_nopar:Npn \b: {BBB~}
  \cs_set_nopar:Npn \c: {B~BB}
  
  \if_meaning:w \a:\b: \PASS \else: \FAIL \fi:
  \if_meaning:w \a:\c: \PASS \else: \FAIL \fi:
  
  \token_to_meaning:N \b:\\
  \token_to_meaning:N \a:  
\group_end:  
\ExplSyntaxOff
\end{texexample}

\begin{texexample}{LaTeX2e booleans}{}
\makeatletter
\ExplSyntaxOn
\if@mainmatter
     in~main~text
   \else
    not~in~main~text  
\fi    

 \meaning\@mainmattertrue\\
\bool_new:N \phd_mainmatter_bool 
\meaning\phd_mainmatter_bool
\ExplSyntaxOff
\makeatother  
\end{texexample}

\section{Predicate functions}

Predicate functions are one of the more powerful features of |expl3|. What are predicate functions? They are macros that test a predicate (\emph{true} or \meta{false}) and branch to either a true or false branch or just a single branch depending on the signature of the function. The |expl3| package has numerous such functions for example:

\begin{teXXX}
 \str_if_eq:nnT {}{}{}
\end{teXXX}

accepts two strings and if true does something. The expl3 package, provides a function that can generate such predicate functions fairly easily.

\begin{docCommand}{prg_set_conditional:Npnn}{\meta {function name}: \meta{arg spec} \meta{parameters} \marg{conditions code}}

These functions create a family of conditionals using the same \meta{code} to perform the
test created. Those conditionals are expandable if \meta{code} is. The new versions will
check for existing definitions and perform assignments globally (cf. |\cs_new:Npn|) whereas
the set versions do no check and perform assignments locally (cf. |\cs_set:Npn|). The
conditionals created are dependent on the comma-separated list of \meta{conditions}, which
should be one or more of p, T, F and TF.
\end{docCommand}

\begin{teXXX}
\prg_set_conditional:Npnn \cs_if_exist:N #1 { p , T , F , TF }
 {
 \if_meaning:w #1 \scan_stop:
   \prg_return_false:
     \else:
        \if_cs_exist:N #1
           \prg_return_true:
        \else:
          \prg_return_false:
      \fi:
 \fi:
}
2556 \prg_new_conditional:Npnn \token_if_eq_meaning:NN #1#2 { p , T , F , TF }
2557 {
2558 \if_meaning:w #1 #2
2559 \prg_return_true: \else: \prg_return_false: \fi:
2560 }

2201 \prg_new_conditional:Npnn \mode_if_math: { p , T , F , TF }
2202 { \if_mode_math: \prg_return_true: \else: \prg_return_false: \fi: }

\prg_new_conditional:Npnn \int_if_even:n #1 { p , T , F , TF}
3321 {
3322 \if_int_odd:w \__int_eval:w #1 \__int_eval_end:
3323 \prg_return_false:
3324 \else:
3325 \prg_return_true:
3326 \fi:
3327 }
\end{teXXX}

\begin{texexample}{isEven}{}
\ExplSyntaxOn
\prg_new_conditional:Npnn \isEven:n #1 { p, T, F, TF}
{
 \if_int_odd:w \__int_eval:w #1 \__int_eval_end:
    \prg_return_false:
 \else:
    \prg_return_true:
 \fi:
}

\isEven:nTF {2045679}{\PASS}{\FAIL}
\isEven:nTF {1000000}{\PASS}{\FAIL}
\ExplSyntaxOff
\end{texexample}

A common need for programmers is the testing of an integer or real for positiveness  with expl3 we can use predicate functions. In Example~\ref{ex:positive} we define predicate functions \docAuxCommand*{isPositive:nTF} to test an integer expression and feed the results to a true or false branch or according to the function signature. 

\begin{texexample}{isPositive} {ex:positive}
\ExplSyntaxOn
\prg_new_conditional:Npnn \isPositive:n #1 { p, T, F, TF}
{
\if_int_compare:w  \__int_eval:w #1 \__int_eval_end: >\__int_eval:w 0 \__int_eval_end:
     \prg_return_true:
\else:
    \prg_return_false:
\fi:          
}

\prg_new_conditional:Npnn \isNegative:n #1 { p, T, F, TF}
{
\if_int_compare:w  \__int_eval:w #1 \__int_eval_end: >\__int_eval:w 0 \__int_eval_end:
     \prg_return_false:
\else:
    \prg_return_true:
\fi:          
}
\cs_new:Npn \assert_is_positive:n #1 
   {
     \isPositive:nTF {#1} {\PASS #1} {\FAIL #1}
   }  
\cs_new:Npn \assert_is_negative:n #1 
   {
     \isNegative:nTF {#1} {\PASS #1} {\FAIL #1}
   } 
\assert_is_positive:n {2059+23-1245}
\assert_is_positive:n {-2059+23-1245}
\assert_is_negative:n {2059+23-1245}
\assert_is_negative:n {-2059+23-1245}
\ExplSyntaxOff
\end{texexample}

In the next example  we will use a common \tex trick to determine if a number is an integer or not. When \tex tries to convert a number to roman it will not scan past a minus sign .

\begin{texexample}{isInteger} {ex:isinteger}
\ExplSyntaxOn
\prg_set_conditional:Npnn \isInteger:n #1 { p, T, F, TF}
{
   \tl_if_blank:oTF {#1}{\prg_return_false:}
    {
     \tl_if_blank:oTF {  \__int_to_roman:w -\__int_eval:w #1 \__int_eval_end: }
		   {
		     \prg_return_true:
		   }
		   {
		     % not a number, but can be a negative number
		     \prg_return_false:
	         }
   }   
}

\cs_new:Npn \assert_is_integer:n #1 
   {
     \isInteger:nTF {#1} {\PASS\ ~~ #1} {\FAIL\ ~~ #1}\par
   }  
\assert_is_integer:n { }   
\assert_is_integer:n { 12}
\assert_is_integer:n {2059+1}
\assert_is_integer:n {-2059}
\assert_is_integer:n {2059}
%\assert_is_integer:n {ABC-1245}
\ExplSyntaxOff
\end{texexample}

The tests will pass provided even if we pass a  |numexpr|, but the assertion will fails if the number is negative. 
What we should have done was to test first if the head of the string was a (-) and then send it for further processing. 

\begin{texexample}{Testing the head of a string for the minus sign}{ex:string}
\ExplSyntaxOn
\cs_set:Npn \test:#1#2;{
   \str_if_eq:nnTF {-}{#1}{\PASS\par }{\FAIL\par }
   \str_if_eq:nnTF {-}{#1#2}{\PASS\par }{\FAIL\par }
}
\test:-;
\test:-12356;
\test:1234;
\ExplSyntaxOff
\end{texexample}

This passes all the comparison correctly, so we will have to re-write our function to test for the minus sign, before we send it to the main function. The reason I wrote the two tests above, is that a minus sign cannot be considered a number. 

\chapter{LaTeX3 Boxes}

\epigraph{If you go far enough back, your genome connects you with bacteria, butterflies, and barracuda---the great chain of being linked together through DNA.}{---Spencer Wells}

The \pkgname{l3box} package, provides numerous commands that deal with boxes. Before you delve in the code you should be familiar with \tex’s concepts of boxes such as \docAuxCommand*{hbox} and \docAuxCommand*{vbox}. The full repertoire of commands is available, as well as additional helper functions to reduce the number of commands necessary when storing content in boxes. There is also an additional package for handling the \latexe type |\fbox| and |\makebox| commands, still in experimental stage called \pkgname{xbox}. The latter also is attempting to provide some integration with the \pkgname{xcoffins} package which is an entirely new concept for box manipulation in \latex3. Most of the commands are just syntactic translations of the \latexe macros. 

Do they offer any advantage? I am not too sure if they do at this stage. When it comes to boxes, which is such a fundamental typographic concept users expect much more than these basic commands, however one needs to build up from more basic commands and these have to be re-defined to keep up with the spirit of \latex3.

\section{Storing content in boxes}

\tex’s concept of storing content in boxes is fundamental to any programming effort, where the dimensions of typeset material needs to be determined before further processing.

\subsection{Creating and initializing boxes}
\begin{docCommand}{box_new:N} {  \meta{box}}
   Creates a new \meta{box} or raises an error if the name is
   already taken. The declaration is global. The \meta{box} will
   initially be void.
\end{docCommand}
\begin{docCommand}{box_new:c}{\meta{box}}
   Creates a new \meta{box} or raises an error if the name is
   already taken. The declaration is global. The \meta{box} will
   initially be void.
\end{docCommand}

Normally three operations are involved. Creating an empty box or using one of the available temporary one, setting the contents in a horizontal or vertical or a combination of both of them, measuring them if necessary 
and then 

\begin{texexample}{Storing content in boxes}{l3box}
\ExplSyntaxOn
\box_new:c { textbox }
\hbox_set_to_wd:cnn { textbox } { 6cm } 
  {
    \tex_hsize:D 5cm
    \colorbox{spot!10}{\vbox:n  
       { \lorem }}
  }
\box_use:c {textbox}
\ExplSyntaxOff
\end{texexample}

The naming schemes are a bit unintuitive but this is inherited from \tex itself. To restrict the |\vbox| you need to set the |\hsize|.  

 \begin{docCommand}{box_move_right:nn}{\docAuxCommand*{box_move_right:nn} \marg{dimexpr} \marg{box function}}
 This function operates in vertical mode, and inserts the
  material specified by the \meta{box function}
  such that its reference point is displaced horizontally by the given
   \meta{dimexpr} from the reference point for typesetting, to the right
   or left as appropriate. The \meta{box function} should be
   a box operation such as |\box_use:N \<box>| or a \enquote{raw}
   box specification such as |\vbox:n { xyz }|.
 \end{docCommand}

 \begin{docCommand}{box_move_up:nn}{\docAuxCommand*{box_move_up:nn} \marg{dimexpr} \marg{box function}}
   This function operates in horizontal mode, and inserts the
   material specified by the \meta{box function}
   such that its reference point is displaced vertical by the given
   \meta{dimexpr} from the reference point for typesetting, up
   or down as appropriate. The \meta{box function} should be
   a box operation such as |\box_use:N \<box>| or a \enquote{raw}
   box specification such as |\vbox:n { xyz }|.
 \end{docCommand}
 
\begin{texexample}{Moving Boxes up or down}{l3boxdown}
\ExplSyntaxOn
\vbox_set:cn{textbox}{abcd}
A \box_move_down:nn{10pt}{\box_use:c {textbox}} 
\ExplSyntaxOff
\end{texexample}

 \section{Measuring and setting box dimensions}

\begin{docCommand}{box_dp:N}{\docAuxCommand*{box_dp:N} \meta{box}}
   Calculates the depth (below the baseline) of the \meta{box}
   in a form suitable for use in a \meta{dimension expression}.
\end{docCommand}

\begin{docCommand}{box_ht:N}{\docAuxCommand*{box_ht:N} \meta{box}}
   Calculates the height (above the baseline) of the \meta{box}
   in a form suitable for use in a \meta{dimension expression}.
  This is the \TeX{} primitive \docAuxCommand*{ht}.
 \end{docCommand}

% \begin{function}{\box_wd:N, \box_wd:c}
%   \begin{syntax}
%     \docAuxCommand*{box_wd:N} \meta{box}
%   \end{syntax}
%   Calculates the width of the \meta{box} in a form
%   suitable for use in a \meta{dimension expression}.
%   \begin{texnote}
%     This is the \TeX{} primitive \tn{wd}.
%   \end{texnote}
% \end{function}

\section{Horizontal Boxes}
\label{l3:hboxes}

So far we have discussed the boxing, unboxing and measuring of box dimensions. In the examples we have used
the \latex3 form of |\hbox| and |\vbox|.  Now time to lose our  beloved |source2e| favoured command \docAuxCommand*{hb@xt@} and friends. 

 \begin{docCommand}{hbox:n}{\docAuxCommand*{hbox:n} \marg{contents}}
   Typesets the \meta{contents} into a horizontal box of line width and then includes this box in the current list for typesetting.
   This is the \TeX{} primitive \docAuxCommand*{hbox}.
 \end{docCommand}

\begin{texexample}{Natural width boxes}{l3:hbox}
\ExplSyntaxOn
\hbox:n{\includegraphics[width=0.8\textwidth]{latex3}}
\ExplSyntaxOff
\end{texexample}

\begin{texexample}{Natural width boxes}{l3:hbox}
\ExplSyntaxOn
\hbox:n{\includegraphics[width=0.8\textwidth]{latex3}}
\ExplSyntaxOff
\end{texexample}


\begin{docCommand}{hbox_to_wd:nn}{\docAuxCommand*{hbox_to_wd:nn} \marg{dimexpr} \marg{contents}}
   Typesets the \meta{contents} into a horizontal box of width
   \meta{dimexpr} and then includes this box in the current list for
   typesetting.
\end{docCommand}

\begin{texexample}{Natural width boxes}{l3:hbox}
\ExplSyntaxOn
\DeclareDocumentCommand\PutImage{o m}{
  \IfNoValueTF{#1}
      {\putimage{#2}}
      {\putimage{#1}{#2}}
}

\noindent\hbox_to_wd:nn{0.3\textwidth}{\includegraphics[width=0.3\textwidth]{amato}}
\hbox_to_wd:nn{0.3\textwidth}{\includegraphics[width=0.3\textwidth]{amato}}
\ExplSyntaxOff

\noindent\hbox to 0.3\textwidth{\includegraphics[width=0.3\textwidth]{amato}}%
\hbox to 0.3\textwidth{\includegraphics[width=0.3\textwidth]{amato}}
\end{texexample}

Having set our goodbyes to |\hb@xt@| we also don’t feel very sorry for not having to type \% to eliminate wandering spaces. As we delve further into the intricacies of \latex3 we can also start appreciating its advantages.

% \begin{function}{\hbox_to_zero:n}
%   \begin{syntax} 
%     \docAuxCommand*{hbox_to_zero:n} \Arg{contents}
%   \end{syntax}
%   Typesets the \meta{contents} into a horizontal box of zero width
%   and then includes this box in the current list for typesetting.
% \end{function}

\section{Vertical Boxes}
The vertical box equivalents to \tex’s |\vbox|, |\vtop| are provided, as well as helper functions to store contents in a box typeset zero width boxes or lap them left, right or center. The commands are mostly syntactic sugar to the primitive commands. 

\begin{docCommand}{vbox:n}{\marg{contents}}
Typesets the \meta{contents} into a vertical box of natural height and includes this box in the current list for typesetting.
\end{docCommand}

\begin{docCommand}{vbox_to_ht:nn}{\marg{dimexpr}\marg{contents}}
Typesets the \meta{contents} into a vertical box of height \meta{dimexpr} and includes this box in the current list for typesetting.
\end{docCommand}

\begin{docCommand}{vbox_to_zero:n}{\marg{contents}}
Typesets the \meta{contents} into a vertical box of zero height and includes this box in the current list for typesetting.
\end{docCommand}

%\tcbset{listing options={
%              firstnumber=10, stepnumber=1, belowskip=0pt, 
%              escapeinside={(*@}{@*)},
%              backgroundcolor=\color{graphicbackground}
%              }}
\begin{texexample}{vboxes in LaTeX3}{l3:boxes}
\ExplSyntaxOn
    \fbox{\vbox:n{\lorem}}\par
    \fbox{\vbox_to_ht:nn {1.5cm}{\lorem}}\par
    \fbox{\vbox_to_zero:n {\lorem}}
\ExplSyntaxOff
\vspace*{1cm}
\end{texexample}

In Example~\ref{l3:boxes} we use \docAuxCommand*{vbox_to_ht:nn} and \docAuxCommand*{vbox_to_zero:n} to set text in two vertical boxes. The first one is typeset in a vertical box of 2cm height, whereas the second one in a box of zero height. The macro
|\fbox| which we discussed earlier in the \latexe boxes chapter, is also available in \latex3 but as part of the still under trial package \pkgname{xbox}.\footnote{To make matters more complicated, the version used in this document has been redefined further!} 



%\tcbset{listing options={
%              firstnumber=last, stepnumber=1, belowskip=0pt, 
%              escapeinside={(*@}{@*)},
%              backgroundcolor=\color{graphicbackground},
%              upquote=true,
%          }}
              
\begin{texexample}{vboxes in LaTeX3}{l3:boxes}
\ExplSyntaxOn
    \fbox{\vbox:n{\lorem}}\par
    \fbox{\vbox_to_ht:nn {1.5cm}{\lorem}}\par
    \fbox{\vbox_to_zero:n {\lorem}}
\ExplSyntaxOff
\vspace*{1cm}
\end{texexample}

\chapter{LaTeX3 xcoffins, special boxes for special typesetting}

\epigraph{The history of that name (as I remember it at least) goes way back to a stroll in some town in the UK sometime in the last century, probably 1997 (may have been Nottingham, but I don't remember) with David Carlisle and Chris Rowley and perhaps a few others on which we discussed those ideas about boxes with handles and somehow somebody came up with "rather like a coffin" and that is how it got born. And no, I don't remember whether it was David, Chris or myself.

Somehow the name stuck; initially as a working title when we first implemented a prototype, but later I must confess I rather liked it -- a bit morbit for sure, but also catchy :-) ... and it made for few a great lines in my talk in San Francisco, such as: Now in 2010 coffins are back – exhumed, cleaned up – and ready for display
what else can you hope for?}{---Frank Mittelbach}


%\tcbset{listing options={
%              firstnumber=10, stepnumber=1, belowskip=0pt, 
%              escapeinside={(*@}{@*)},
%              backgroundcolor=\color{graphicbackground},
%              upquote=true,
%          }}
          
 In \LaTeX3 terminology, a \enquote{coffin} is a box containing
 typeset material.\footnote{The term `coffin’ was probably coined by Frank Mittelbach (see \protect\url{http://tex.stackexchange.com/questions/147738/origin-of-the-latex3-term-coffin})} Along with the box itself, the coffin structure
 includes information on the size and shape of the box, which makes
 it possible to align two or more coffins easily. This is achieved
 by providing a series of `poles' for each coffin. These
 are horizontal and vertical lines through the coffin at defined
 positions, for example the top or horizontal centre. The points
 where these poles intersect are called \enquote{handles}. Two
 coffins can then be aligned by describing the relationship between
 a handle on one coffin with a handle on the second. In words, an
 example might then read
 \begin{quote}
   Align the top-left handle of coffin A with the bottom-right
   handle of coffin B.
 \end{quote}

 The locations of coffin handles are much easier to understand
 visually. Figure~\ref{fgr:handles} shows the standard handle
 positions for a coffin typeset in horizontal mode (left) and in
 vertical mode (right). Notice that the later case results in a greater
 number of handles being available. As illustrated, each handle
 results from the intersection of two poles. For example, the centre
 of the coffin is marked |(hc,vc)|, \emph{i.e.}~it is the
 point of intersection of the horizontal centre pole with the
 vertical centre pole. New handles are generated automatically when
 poles are added to a coffin: handles are \enquote{dynamic} entities.
 \NewCoffin \ExampleCoffin
\begin{figure}[htbp]
   \hfil
    \fboxsep2pc
     \colorbox{black}{\color{white}\begin{minipage}{0.4\textwidth}
     \SetHorizontalCoffin\ExampleCoffin
       {\color{white}\rule{1 in}{1 in}}
  \DisplayCoffinHandles\ExampleCoffin{yellow}
   \end{minipage}}
   \hfil
   \begin{minipage}{0.4\textwidth}
     \SetVerticalCoffin\ExampleCoffin{1 in}
       {\color{black!10!white}\rule{1 in}{1 in}}
     \DisplayCoffinHandles\ExampleCoffin{red}
   \end{minipage}
   \hfil
   \caption{Standard coffin handles: left, horizontal coffin; right,
     vertical coffin}
   \label{fgr:handles}
 \end{figure}


All coffin operations are local to the current \tex group with the exception
of coffin creation. Coffins are also “color safe”: in contrast to the code-level \docAuxCommand*{box_}\ldots.
functions there is no need to add additional grouping to coffins when dealing with color.

The user interface for the command is somewhat complicated. This is an area where the package
can be enhanced in the future and the sole reason is being kept under the \emph{experimental}
branch of \latex3.

\section{Getting Started}

Before a \meta{coffin} can be used, it must be allocated using \docAuxCommand*{NewCoffin}.

\begin{docCommand}{NewCoffin}{\meta{coffin}}
Before a \meta{coffin} can be used, it must be allocated using \docAuxCommand*{NewCoffin}. The name of the
hcoffini should be a control sequence (starting with the escape character, usually \textbackslash ), for
example

\begin{verbatim}
\NewCoffin\MyCoffin
\end{verbatim}

Coffins are allocated globally, and an error will be raised if the name of the \meta{coffin} is
not globally-unique.
\end{docCommand}

\begin{texexample}{Coffins}{ex:coffins}
  \NewCoffin \AnExampleCoffin
\end{texexample}

 \begin{docCommand}{SetHorizontalCoffin}{\docAuxCommand*{SetHorizontalCoffin} \meta{coffin} \marg{material}}
   Typesets the \meta{material} in horizontal mode, storing the result
   in the \meta{coffin}. The standard poles for the \meta{coffin} are
   then set up based on the size of the typeset material.
 \end{docCommand}

 \begin{docCommand}{SetVerticalCoffin}{\docAuxCommand*{SetVerticalCoffin} \meta{coffin} \marg{width} \marg{material}}
   Typesets the \meta{material} in vertical mode constrained to the
   given \meta{width} and stores the result in the \meta{coffin}. The
   standard poles for the \meta{coffin} are then set up based on the
   size of the typeset material.
 \end{docCommand}

In Example~\ref{ex:coffins2} we will create a horizontal coffin and then typeset it. 
 
%\tcbset{listing options={
%              firstnumber=last, stepnumber=1, belowskip=0pt, 
%              escapeinside={(*@}{@*)},
%              backgroundcolor=\color{graphicbackground},
%              upquote=true,
%          }}
          
\begin{texexample}{Creating coffins}{ex:coffins2}
\SetHorizontalCoffin\ExampleCoffin
   {\color{red}\rule{4cm}{1pc}}  
  This is a coffin\hspace{0.9cm}\DisplayCoffinHandles\ExampleCoffin{black}\hspace{0.9cm}!
\end{texexample}
  
The rule was created using \latexe |\rule|  macro and then it was saved in a coffin box named |\ExampleCoffin|. The typesetting was done using |\DisplayCoffinHandles| 

In the next example, we will create a second rule and then demonstrate the joining operation. We will need two more coffins, one to hold the results and the other to hold the material for the second box.

\begin{texexample}{Joining Coffins}{ex:coffins3}
\NewCoffin\ExampleCoffinTwo
\NewCoffin\Result
\SetHorizontalCoffin\ExampleCoffin
   {\color{red}\rule{3cm}{1pc}} 
\SetHorizontalCoffin\ExampleCoffinTwo
   {\color{green}\rule{3cm}{1pc}}    
\JoinCoffins\Result\ExampleCoffin   
\JoinCoffins \Result[\ExampleCoffin-t,\ExampleCoffin-r] \ExampleCoffinTwo [b,l](0pt,2mm)
\TypesetCoffin\Result
\end{texexample}
 
The interesting, but complicated command is |\JoinCoffins|. This takes two arguments, the coffins to be joined, which in turn have optional commands, specifying how the coffins are joined at their poles. 
This is the key operation for coffins,  joining coffins to each other. This
 is always carried out such that the first coffin is the
 \enquote{parent}, and is updated by the alignment. The second
 \enquote{child} coffin is not altered by the alignment process.

 \begin{docCommand}{JoinCoffins}{ \docAuxCommand*{JoinCoffins} *
     ~~\meta{coffin1} [ \meta{coffin1-pole1} , \meta{coffin1-pole2} ]
     ~~\meta{coffin2} [ \meta{coffin2-pole1} , \meta{coffin2-pole2} ]
     ~~( \meta{x-offset} , \meta{y-offset} )}
   Joining of two coffins is carried out by the \docAuxCommand*{JoinCoffins}
   function, which takes two mandatory arguments: the \enquote{parent}
   \meta{coffin1} and the \enquote{child} \meta{coffin2}. All of the
   other arguments shown are optional.
 \end{docCommand}

   The standard \docAuxCommand*{JoinCoffins} functions joins \meta{coffin2} to
   \meta{coffin1} such that the bounding box of \meta{coffin1} after the
   process will expand. The new bounding box will be the smallest
   rectangle covering the bounding boxes of the two input coffins.
   When the starred variant of \docAuxCommand*{JoinCoffins} is used, the bounding
   box of \meta{coffin1} is not altered, \emph{i.e.}~\meta{coffin2} may
   protrude outside of the bounding box of the updated \meta{coffin1}.
   The difference between the two forms of alignment is best illustrated
   using a visual example. In Figure~\ref{fgr:alignment}, the two
   processes are contrasted. In both cases, the small red coffin has been
   aligned with the large grey coffin. In the left-hand illustration,
   the \docAuxCommand*{JoinCoffins} function was used, resulting in an expanded
   bounding box. In contrast, on the right \docAuxCommand*{AttachCoffin} was used,
   meaning that the bounding box does not include the area of the
   smaller coffin.
   
\begin{texexample}{Joining Coffins}{ex:coffins4}
\SetHorizontalCoffin\ExampleCoffin
   {\color{red}\rule{3cm}{1pc}} 
\SetHorizontalCoffin\ExampleCoffinTwo
   {\color{green}\rule{3cm}{1pc}}    
\JoinCoffins\Result\ExampleCoffin   
\JoinCoffins*\Result[\ExampleCoffin-l,\ExampleCoffin-b] \ExampleCoffinTwo [t,l](0pt,2mm)
\TypesetCoffin\Result
\end{texexample}   
   
\section{Controlling coffin poles}

 A number of standard poles are automatically generated when the coffin
 is set or an alignment takes place. The standard poles for all coffins
 are:
 \begin{marglist}
   \item[l] a pole running along the left-hand edge of the bounding
     box of the coffin;
   \item[hc] a pole running vertically through the centre of the coffin
     half-way between the left- and right-hand edges of the bounding
       box (\emph{i.e.}~the \enquote{horizontal centre});
   \item[r] a pole running along the right-hand edge of the bounding
     box of the coffin;
   \item[b] a pole running along the bottom edge of the bounding
     box of the coffin;
   \item[vc] a pole running horizontally through the centre of the
     coffin half-way between the bottom and top edges of the bounding
     box (\emph{i.e.}~the \enquote{vertical centre});
   \item[t] a pole running along the top edge of the bounding
     box of the coffin;
   \item[H] a pole running along the baseline of the typeset material
     contained in the coffin.
 \end{marglist}
 In addition, coffins containing vertical-mode material also
 feature poles which reflect the richer nature of these systems:
 \begin{itemize}
   \item[B] a pole running along the baseline of the material at the
     bottom of the coffin.
   \item[T] a pole running along the baseline of the material at the top
     of the coffin.
 \end{itemize}  
 
\section{A larger example}

Consider the book cover of Judy Estrin’s book, \emph{Closing the Innovation Gap} shown in Example~\ref{ex:covers}. The title elements have been carefully placed by the book designer. This sort
of cover page is within the possibilities of what can be programmed via \latex~3 and the package \pkgname{xcoffins}.

\begin{texexample}{Typesetting Cover Pages}{ex:covers}  
\bgroup
\parindent0pt
% For each element declare a new  coffin
\NewCoffin\ci
\NewCoffin\cii
\NewCoffin\ciii
\NewCoffin\civ

% Always better to give semantic names!
\NewCoffin\slogan
\NewCoffin\ImageCoffin
\NewCoffin\AuthorCoffin

% A convenient commant to set font a
\DeclareDocumentCommand\fonta{}
  {
      \color{white}\LARGE\bfseries\sffamily
  }

% Similar command for font b    
\DeclareDocumentCommand\fontb{}
  {
      \color{white}\large\bfseries\sffamily
  }  
\SetHorizontalCoffin\Result{}
\SetHorizontalCoffin\ci{\fonta\space CLOSING} 
\SetHorizontalCoffin\cii{\fontb THE}
\SetHorizontalCoffin\ciii{\fonta INNOVATION}
\SetHorizontalCoffin\civ{\fonta GAP}

\SetVerticalCoffin\slogan{\CoffinWidth\ciii+30pt}{\vspace*{25pt}\centering
\small\sffamily REIGNITING THE SPARK OF
THE GLOBAL ECONOMY\par}

% set the image coffin
\SetHorizontalCoffin\ImageCoffin{\space\space
  \includegraphics[width=100pt]{./images/innovation-book-cover.jpg}}
  
% set the author  
\SetHorizontalCoffin\AuthorCoffin{\fontb\centering JUDY ESTRIN\par}

% Now join all the coffins check the manual for the handles!    
\JoinCoffins\Result\ci
\JoinCoffins\Result[hc,b]    \cii[hc,t](0pt,-2mm)%the
\JoinCoffins\Result[l,b]       \ciii[l,t](15pt,-2mm)%innovation
\JoinCoffins\Result[\ciii-hc,\ciii-b] \civ[l,t](0pt,-2mm)
\JoinCoffins\Result[l,b]      \slogan[l,t](0pt,-2mm)
\JoinCoffins\Result[hc,b]   \AuthorCoffin[hc,t](0pt,-4mm)
\JoinCoffins\Result[r,b]      \ImageCoffin[l,b](0pt, 0pt)
   \fboxsep1pc
  \colorbox{black}{\color{white}\TypesetCoffin\Result}

% close the group we opened     
\egroup
\end{texexample}

Of course my general advice to anyone programming \latex is to always get professional advice on designing a book cover. Mathematicians, programmers and scientists are not the best of people to design book covers. They can come up with the code, but hardly succeed with the graphics aspects. There are also other methods to design and typeset book covers. An excellent package using \tikzname is \pkgname{bookcover} by Tibor Tómács. 


One tends to forget if the syntax requires to type \textit{t}, \textit{l} or \textit{l}, \textit{t} and this is a common issue with this type of commands. As we said before \latex stresses one’s memory to the limit. It can also be a bit confusing, as to when one needs to use a vertical rather than horizontal coffin.
    
If you stydy the code in Example~\ref{ex:covers} you will notice that the last box, has a width that was set using
\docAuxCommand*{CoffinWidth}. The package provides commands that provide the value of the coffin dimensions. These are described in the next section that together with some other auxiliary helper functions concludes our discussion of the package.

 \section{Measuring coffins}

 There are places in the design process where it is useful to be able to
 measure coffins outside of pole-setting procedures.

 \begin{docCommand}{CoffinDepth}{ \docAuxCommand*{CoffinDepth} \meta{coffin}}
   Calculates the depth (below the baseline) of the \meta{coffin}
   in a form suitable for use in a \meta{dimension expression}, for example
   |\setlength{\mylength}{\CoffinDepth\ExampleCoffin}|.
 \end{docCommand}

 \begin{docCommand}{CoffinHeight}{\docAuxCommand*{CoffinHeight} \meta{coffin}}
   Calculates the height (above the baseline) of the \meta{coffin}
   in a form suitable for use in a \meta{dimension expression}, for example
   |\setlength{\mylength}{\CoffinHeight\ExampleCoffin}|.
 \end{docCommand}

 \begin{docCommand}{CoffinTotalHeight}{\docAuxCommand*{CoffinTotalHeight} \meta{coffin}}
   Calculates the total height of the \meta{coffin}
   in a form suitable for use in a \meta{dimension expression}, for example
   |\setlength{\mylength}{\CoffinTotalHeight\ExampleCoffin}|.
 \end{docCommand}

 \begin{docCommand}{CoffinWidth}{\docAuxCommand*{CoffinWidth} \meta{coffin}}
   Calculates the width of the \meta{coffin} in a form
   suitable for use in a \meta{dimension expression}, for example
   |\setlength{\mylength}{\CoffinWidth\ExampleCoffin}|.
 \end{docCommand} 
    
\section{Debugging}

Debugging code that includes |coffin| functions is made easier when you can view information on the
poles. The pakage provides commands for both printing the information as well as viewing it on the screen.

\begin{docCommand}{DisplayCoffinHandles}{\meta{coffin}meta{color}}
This function first calculates the intersections between all of the hpolesi of the \meta{coffin} to
give a set of \meta{handles}. It then prints the \meta{coffin} at the current location in the source,
with the position of the \meta{handles} marked on the coffin. The \meta{handles} will be labelled
as part of this process: the locations of the \meta{handles} and the labels are both printed in
the \meta{color} specified. This is similar to the |\TypesetCoffin| function, except the former will also print
the handles. 
\end{docCommand}
  
\begin{docCommand}{MarkCoffinHandle}{\meta{coffin}\oarg[\meta{pole1}, \meta{pole2}] \marg{color}}  
This function first calculates the \meta{handle} for the \meta{coffin} as defined by the intersection
of \meta{pole1} and \meta{pole2}. It then marks the position of the \meta{handle} on the \meta{coffin}. The
\meta{handle} will be labelled as part of this process: the location of the \meta{handle} and the
label are both printed in the \meta{color} specified. If no \meta{poles} are give, the default (H,l) is
used.
\end{docCommand}
  
   \begin{figure}
     \hfil
     \SetHorizontalCoffin\ExampleCoffin
       {%
         \color{black!10!white}\rule{0.5 in}{1 in}^^A
         \color{black!20!white}\rule{0.5 in}{1 in}^^A
       }
     \begin{minipage}{0.4\textwidth}
       \DisplayCoffinHandles\ExampleCoffin{blue}
     \end{minipage}
     \hfil
     \begin{minipage}{0.4\textwidth}
       \RotateCoffin\ExampleCoffin{45}
       \DisplayCoffinHandles\ExampleCoffin{red!50!black}
     \end{minipage}
     \hfil
     \caption{Coffin rotation: left, unrotated; right, rotated by
       $45$\textdegree.}
     \label{fgr:rotation}
   \end{figure}
   
%\newpage 
%\newgeometry{margin=5pt,}  
%\null
%
%\newcommand\cbox[2][.8]{{\setlength\fboxsep{0pt}\colorbox[gray]{#1}{#2}}}
%
%
%  \NewCoffin \result
%  \NewCoffin \aaa
%  \NewCoffin \bbb
%  \NewCoffin \ccc
%  \NewCoffin \ddd
%  \NewCoffin \eee
%  \NewCoffin \fff
%  \NewCoffin \rulei
%  \NewCoffin \ruleii
%  \NewCoffin \ruleiii
%
%\SetHorizontalCoffin \result {}
%\SetHorizontalCoffin \aaa {\fontsize{52}{50}\sffamily\bfseries mep progress}
%\SetHorizontalCoffin \bbb {\fontsize{52}{50}\sffamily\bfseries habtoor city}%typographische}habtoor city
%\SetHorizontalCoffin \ccc {\fontsize{12}{10}\sffamily 
%                      \quad zeitschrift des bildungsverbandes der
%                      deutschen buchdrucker leipzig 
%                     \textbullet{} oktoberheft 1925}
%\SetHorizontalCoffin \ddd {\fontsize{28}{20}\sffamily report}%sonderheft}
%\SetVerticalCoffin \eee {180pt}
%                 {\raggedleft\fontsize{31}{36}\sffamily\bfseries 
%                      elementare\\
%                      typographie}
%\SetVerticalCoffin \fff {140pt}
%                 {\raggedright \fontsize{13}{14}\sffamily\bfseries 
%                       yannis lazarides \\
%                       nasser khalf \\
%                       kyriacos savva \\
%                       max burchartz \\
%                       el lissitzky \\
%                       ladislaus moholy-nagy \\
%                       moln\'ar f.~farkas \\
%                       johannes molzahn \\
%                       kurt schwitters \\
%                       mart stam \\
%                       ivan tschichold}
%
%\RotateCoffin \bbb {90}
%\RotateCoffin \ccc {270}
%
%\SetHorizontalCoffin \rulei  {\color{red}\rule{6.5in}{1pc}}
%\SetHorizontalCoffin \ruleii {\color{red}\rule{1pc}{20.5cm}}
%\SetHorizontalCoffin \ruleiii{\color{black}\rule{10pt}{152pt}}
%
%
%\JoinCoffins \result                \aaa 
%\JoinCoffins \result[\aaa-t,\aaa-r] \rulei   [b,r](0pt,2mm)
%\JoinCoffins \result[\aaa-b,\aaa-l] \bbb     [B,r](2pt,0pt)
%\JoinCoffins \result[\bbb-t,\bbb-r] \ruleii  [t,r](-2mm,0pt)
%\JoinCoffins \result[\aaa-B,\aaa-r] \ccc     [B,l](66pt,14pc)
%\JoinCoffins \result[\bbb-l,\ccc-B] \fff     [t,r](-2mm,0pt)
%\JoinCoffins \result[\fff-b,\fff-r] \ruleiii [b,l](2mm,0pt)
%\JoinCoffins \result[\ccc-r,\fff-l] \eee     [B,r]
%\JoinCoffins \result[\eee-T,\eee-r] \ddd     [B,r](0pt,4pc)
%
%
%
%\TypesetCoffin \result
%
%\restoregeometry


\chapter{LaTeX3 String Manipulation and other Goodies}

 \TeX{} associates each character with a category code: as such, there is no
 concept of a \enquote{string} as commonly understood in many other
 programming languages. However, there are places where we wish to manipulate
 token lists while in some sense \enquote{ignoring} category codes: this is
 done by treating token lists as strings in a \TeX{} sense.

 A \TeX{} string (and thus an \pkg{expl3} string) is a series of characters
 which have category code $12$ (\enquote{other}) with the exception of
 space characters which have category code $10$ (\enquote{space}). Thus
 at a technical level, a \TeX{} string is a token list with the appropriate
 category codes. In this documentation, these will simply be referred to as
 strings: note that they can be stored in token lists as normal.

 The functions documented here take literal token lists,
 convert to strings and then carry out manipulations. Thus they may
 informally be described as \enquote{ignoring} category code. Note that
 the functions \docAuxCommand*{cs_to_str:N}, \docAuxCommand*{tl_to_str:n}, \docAuxCommand*{tl_to_str:N} and
 \docAuxCommand*{token_to_str:N} (and variants) will generate strings from the appropriate
 input: these are documented in \pkg{l3basics}, \pkg{l3tl} and \pkg{l3token},
 respectively.

 \section{The first character from a string}

 \begin{docCommand}{str_head:n}{\docAuxCommand*{str_head:n} \marg{token list}}
   Converts the \meta{token list} into a string, as described for
   \docAuxCommand*{tl_to_str:n}. The \docAuxCommand*{str_head:n} function then leaves
   the first character of this string in the input stream.
   The \docAuxCommand*{str_tail:n} function leaves all characters except
   the first in the input stream. The first character may be
   a space. If the \meta{token list} argument is entirely empty,
   nothing is left in the input stream.
 \end{docCommand}

\begin{texexample}{Strings}{ex:strings}
\ExplSyntaxOn
\DeclareDocumentCommand\asentence{ m }{
  \str_head:n {#1}\par}
  
\asentence{This is something}  

\str_head:n{\This~is~something}\par
\str_tail:n{\This~is~something}
\ExplSyntaxOff


\end{texexample}

 \subsection{Tests on strings}

The package provides some very powerful commands that can be used in string comparisons. Internally the comparisons are carried out using |\pdfstrcmp|. This has some complications in LuaTeX. 

 \begin{docCommand}{str_if_eq_x:nnTF}{\docAuxCommand*{str_if_eq_p:nn} \marg{tl1} \marg{tl2}}
%     \docAuxCommand*{str_if_eq:nnTF} \Arg{tl_1} \Arg{tl_2} \Arg{true code} \Arg{false code}
%   \end{syntax}
   Compares the two \meta{token lists} on a character by character
   basis, and is \texttt{true} if the two lists contain the same
   characters in the same order. Thus for example
   \begin{verbatim}
     \str_if_eq_p:no { abc } { \tl_to_str:n { abc } }
   \end{verbatim}
   is logically \texttt{true}.
\end{docCommand}


\begin{texexample}{String comparisons}{ex:test}
\ExplSyntaxOn
\let\abc\empty
\str_if_eq_x:nnTF{abc}{abc}{\TRUE}{\FALSE}\par
\str_if_eq_x:nnTF{\abc}{\abc}{\TRUE}{\FALSE}
\ExplSyntaxOff
\end{texexample}

 \section{String manipulation}

 \begin{docCommand}{str_lower_case:n}{\marg{tokens}}
%      \str_lower_case:n, \str_lower_case:f, 
%      \str_upper_case:n, \str_upper_case:f
%   }
%   \begin{syntax}
%     \docAuxCommand*{str_lower_case:n} \Arg{tokens}
%     \docAuxCommand*{str_upper_case:n} \Arg{tokens}
%   \end{syntax}
   Converts the input \meta{tokens} to their string representation, as
   described for \docAuxCommand*{tl_to_str:n}, and then to the lower or upper
   case representation using a one-to-one mapping as described by the
   Unicode Consortium file |UnicodeData.txt|.
   
   These functions are intended for case changing programmatic data in
   places where upper/lower case distinctions are meaningful. One example
   would be automatically generating a function name from user input where
   some case changing is needed. In this situation the input is programmatic,
   not textual, case does have meaning and a language-independent one-to-one
   mapping is appropriate. For example
%   \begin{verbatim}
%     \docAuxCommand*_new_protected:Npn \myfunc:nn #1#2
%       {
%         \docAuxCommand*_set_protected:cpn
%           {
%             user
%             \str_upper_case:f { \tl_head:n {#1} }
%             \str_lower_case:f { \tl_tail:n {#1} }
%           }
%           { #2 }
%       }
%   \end{verbatim}
%   would be used to generate a function with an auto-generated name consisting
%   of the upper case equivalent of the supplied name followed by the lower
%   case equivalent of the rest of the input.
%   
%   These functions should \emph{not} be used for
%   \begin{itemize}
%     \item Caseless comparisons: use \docAuxCommand*{str_fold_case:n} for this
%       situation (case folding is district from lower casing).
%     \item Case changing text for typesetting: see the \docAuxCommand*{tl_lower_case:n(n)},
%       \docAuxCommand*{tl_upper_case:n(n)} and \docAuxCommand*{tl_mixed_case:n(n)} functions which
%       correctly deal with context-dependence and other factors appropriate
%       to text case changing.
%   \end{itemize}
%
%   \begin{texnote}
%     As with all \pkg{expl3} functions, the input supported by
%     \docAuxCommand*{str_fold_case:n} is \emph{engine-native} characters which are or
%     interoperate with \textsc{utf-8}. As such, when used with \pdfTeX{}
%     \emph{only} the Latin alphabet characters A--Z will be case-folded
%     (\emph{i.e.}~the \textsc{ascii} range which coincides with
%     \textsc{utf-8}). Full \textsc{utf-8} support is available with both
%     \XeTeX{} and \LuaTeX{}, subject only to the fact that \XeTeX{} in
%     particular has issues with characters of code above hexadecimal
%     $0\mathrm{xFFF}$ when interacting with \docAuxCommand*{tl_to_str:n}.
%   \end{texnote}
 \end{docCommand}
 
 A common programming task is to convert strings to either uppercase or lowercase equivalents.v
 \begin{texexample}{Converting strings to lower and uppercase}{ex:cases}%TOFIX 
 \ExplSyntaxOn
    \tl_tail:n {TEST} 
   
      \cs_new_protected:Npn \myfunc:nn #1#2
       {
         \cs_set_protected:cpn
           {
             user
             \str_upper_case:f { \tl_head:n {#1} }
             \str_lower_case:f { \tl_tail:n {#1} }
           }
           { #2 }
       }
\docAuxCommand*_new_protected:cpn {yiannis}{Lazarides}
 \ExplSyntaxOff
 \end{texexample}


\chapter{LaTeX3 properties}


 \LaTeX3 implements a \enquote{property list} data type, which contain
 an \emph{unordered list} of entries each of which consists of a \meta{key} and
 an associated \meta{value}. The \meta{key} and \meta{value} may both be
 any \meta{balanced text}. It is possible to map functions to property lists
 such that the function is applied to every key--value pair within
 the list.

 Each entry in a property list must have a unique \meta{key}: if an entry is
 added to a property list which already contains the \meta{key} then the new
 entry will overwrite the existing one. The \meta{keys} are compared on a
 string basis, using the same method as \docAuxCommand*{str_if_eq:nn}.

 Property lists are intended for storing key-based information for use within
 code.  This is in contrast to key--value lists, which are a form of
 \emph{input} parsed by the \pkgname{keys} module.

 \section{Creating and initialising property lists}

 \begin{docCommand}{prop_new:N or :c}{\meta{property list}}
   Creates a new \meta{property list} or raises an error if the name is
   already taken. The declaration is global. The \meta{property list} will
   initially contain no entries.
 \end{docCommand}

 \begin{docCommand}{prop_clear:N (:n:c) }{ \meta{property list}}
   Clears all entries from the \meta{property list}.
 \end{docCommand}

 \section{Adding entries to property lists}

 \begin{docCommand}{prop_put:Nnn}{ \meta{property list} \marg{key} \marg{value}}
 
   Adds an entry to the \meta{property list} which may be accessed
   using the \meta{key} and which has \meta{value}. Both the \meta{key}
   and \meta{value} may contain any \meta{balanced text}. The \meta{key}
   is stored after processing with \docAuxCommand*{tl_to_str:n}, meaning that
   category codes are ignored. If the \meta{key} is already present
   in the \meta{property list}, the existing entry is overwritten
   by the new \meta{value}.
 \end{docCommand}
 
 \begin{texexample}{Property lists}{ex:proplits}
 \ExplSyntaxOn
 \prop_new:N \proptemp
 \prop_put:Nnn \proptemp {symbolic}{true}
 \ExplSyntaxOff
\end{texexample}

 \section{Recovering values from property lists}

   \begin{docCommand}{prop_get:NnN}{ \meta{property list} \marg{key} \meta{tl var}}
   Recovers the \meta{value} stored with \meta{key} from the
   \meta{property list}, and places this in the \meta{token list
   variable}. If the \meta{key} is not found in the
   \meta{property list} then the \meta{token list variable} will
   contain the special marker \docAuxCommand*{q\_no\_value}. The \meta{token list
     variable} is set within the current \TeX{} group. See also
   \docAuxCommand*{prop_get:NnNTF}.
  \end{docCommand}



	\chapter{Expansion and LaTeX3}

Expansion and variants are central to the concept of \latex3. The module| l3expan| provides generic methods for expanding \tex arguments in a systematic manner. The functions in the module all have prefix |exp|.

The module provides functions to produce \enquote{variants}. This is one of the most fundamental concepts of \latex3 and is good before we proceed further to recap on some of the \latex3 concepts.

\begin{description}
\item [naming conventions] The naming convention for command in \latex3 (expl3)  structures for command names is:

\textbackslash \meta{module}\textunderscore \meta{description}: \meta{arg-specifiers}

\textit{module} identifies the (main) type of data the function manipulates or use (for example, int (integers), prop (property lists), etc., or it might be the name of a package or some specific concept. 

\textit{description} says what is being done, e.g., |put_left|, |get|, |clear|, |count|, etc. If it makes sense the same descriptions are reused, but for special tasks there can, of course, be some that are used only once.

\textit{arg-specifiers} finally describe what arguments the function has and how they should be treated (more on this below).


\item[Base functions] \lorem 

\item[Variant functions]  Any command that uses one or more of these \emph{arg-specifiers} is called a \emph{variant} of the corresponding \emph{base function}. What these functions do is that they modify the argument one way or another and \textbf{only} then pass it to the underlying base function. For example:

\begin{teXXX}
\foo_bar:cVno {cmd} \VAR {text} \CMD
\end{teXXX}

would

\begin{itemize}
\item generate from the string |cmd| the command name \cs{cmd}
\item look up the value of the variable \cs{VAR}
\item leave text alone
\item expand \cs{CMD} once and surround the result with braces
\end{itemize}

It is important to stress that the variants do not produce aliases for the functions, they are also not overloading them. They just expand the base function arguments in a different way. 

\begin{teXXX}
 \foo_bar:Nnnn \cmd {<value-of-\VAR>} {text} {<one-level-expansion-of-\CMD>}
\end{teXXX}

Now ideally we want any possible variant of a base function automatically available for a programmer. Unfortunately, this can only be reliably done if all variants have all been predefined (as TeX doesn't offer you to trap the \enquote{undefined csname} error and do something on the fly).

Given the number of arg-specifier and the possible permutations predefining all variants, of which 90\% would never be needed, is not realistic. As Fank Mittelbach wrote on TEX.SX Q\&A site the \latex3 Team adopted the following strategy:

\begin{enumerate}
\item conceptually all variants are available and everybody can assume this is the case

\item in reality the kernel only defines a small subset that is often needed

\item any variant not defined by the kernel needs to be defined by the programmer using \docAuxCommand*{cs_generate_variant:Nn}

\item \docAuxCommand*{cs_generate_variant:Nn} has been designed in such a way that it doesn't matter if it is called several times: if the variant already exists it will do nothing. So if two programmers define the same variant in their packages it doesn't hurt, the first one executed will define the variant the second one will simply be ignored (with very little overhead).
\end{enumerate}

If some variants are used fairly often they may eventually get defined already in the kernel. Because of the last point it doesn't hurt if some packages still define the variant, i.e., there is no need for programmers to modify their packages in that case.

So in summary: Whenever you need a variant that is not predefined, define it at the beginning of your code. This is even sensible if you need the variant only once, because the code using the variant will be much more readable than any manual preprocessing of the argument and the speed difference is close to zero.

\item[The exp\_args:N.. functions]

Technicically speaking a variant defined via |\cs_generate_variant:Nn| has a very simple definition: |\foo_bar:cVno| above would simply expand to
\end{description}

\section{How to define variants}

The workhorse function used to define variants is:

\begin{docCommand}{cs_generate_variant:Nn} { \meta{parent control sequence} \marg{variant arg-spec}}

\end{docCommand}




 
	\chapter{LaTeX3 quarks and recursion}
\label{ch:quarks}

\section{What are quarks?}
Quarks and recursion are central to the expl3 language. Quarks are a weird concept and is inherited from \tex’s way of scanning macro arguments.

But before we delve into the details of |expl3|'s quarks let us review \tex's delimited functions with an example. Consider the following example where we delimit the arguments of a macro |\test| with the control sequence |\texquark|. We do not need to define the |\texquark| and as we discussed in the section on macros it can even consist of the macro name itself. \tex will scan the input until the marker is found. It will also absorb the marker and do nothing about it.

\begin{texexample}{TeX quarks!}{}
\def\test#1\texquark{#1}
\test 123456\texquark \\
\def\test#1\test{#1}
\test 123456\test
\end{texexample}

In \latex2e macro delimiters are found all over the place, mostly in the form of \docAuxCommand{@nil}, \docAuxCommand{@nni}  or \docAuxCommand{@@}. See for example, how the \latex2e kernel defines lists.

 In \latex3 these have been termed \enquote{quarks} and \enquote{scan marks}. By convention all constants of type quark start out with |\q_| and scan marks start with |\s_|. Scan marks are reserved for internal use by the kernel and you should avoid using them in your code.\index{scan marks}\index{quarks}

They differ from the simple case above with the \tex example, in that they are used mostly indirectly. The \latex3 quarks, are defined so that they expand to themselves. As such they should never be executed directly in the code. This would cause and endless loop and cause either the program or even your computer to crash. The reason they hold a value, is that they can be tested, using |\ifx| which compares the meaning of two macros without expanding them. The equivalent construction in |expl3| is |\if_meaning:w|. We can use it at the next example. Note I gave used |\def| in th example to make it clearer, but one of course can use |\cs_set:Npn| or an equivalent function.

\begin{texexample}{Checking if is a quark}{ex:quarks}
\ExplSyntaxOn
\def\quark{\quark}
\cs_set_nopar:Npn \b {\quark}
\if_meaning:w  \quark\b
   \PASS
\else:
  \FAIL 
\fi:
\ExplSyntaxOff
\end{texexample}

This ingenious technique employed in Example~\ref{ex:quarks} depends on \tex’s ability to carry out comparisons without expanding the macros being compared. This way semantic definitions can be made for quarks and employed in generic recursive functions. 
Normally, you wouldn’t need to define your own quarks, as the ones made available by |expl3| are adequate for most tasks. If you have to create one, it can be created using:

\begin{docCommand}{quark_new:N}{ \meta{quark}}
Creates a new \meta{quark} which expands only to \meta{quark}. The \meta{quark} is defined globally, and an error message will be raised if the name was already taken.
\end{docCommand}

For example, the kernel defines two flavours of quarks to be used specifically for recursion and which we will use in the next section.

\begin{teXXX}
\quark_new:N \q_recursion_tail
\quark_new:N \q_recursion_stop
\end{teXXX}

Other flavours are lsited in the manual and summarized below:

\begin{docCommand}{q_stop}{ \meta{quark}}
Used as a marker for delimited arguments such as:
\begin{verbatim}
\cs_set:Npn \tmp:w #1#2 \q_stop {#1}
\end{verbatim}
\end{docCommand}





\section{Recursion}
 
One of the problem areas in programming recursion is to have a uniform interface to intercepting and terminating loops when one is doing recursion. \latex3 provides the building blocks.

First let us see an example:

\begin{texexample}{Recursion}{ex:l3recursion}
\ExplSyntaxOn

\cs_new:Npn \__my_decoration_fn:nn #1  {
  \str_if_eq:nnTF{e}{#1}
    {[{\bfseries\color{red}#1}]}
    {[#1]}
}

\cs_new:Npn \mymain #1 
{
      \__my_map:n #1 \q_recursion_tail\q_recursion_stop
}

\cs_new:Npn \__my_map:n #1 
  {
    \quark_if_recursion_tail_stop:n {#1}
    \__my_decoration_fn:nn  {#1} 
    \__my_map:n
  }
\ExplSyntaxOff
 
\mymain {abcdefgh}
\end{texexample}

The main function, will first call a mapping function leaving in the stream the following:
\medskip

\texttt{bcdefgh} {\hl{\textbackslash q\_recursion\_tail} \hl{\textbackslash q\_recursion\_stop}}
\medskip

On the second iteration the stream will be reduced by one token (b) and the remaing value will be:
\medskip

\texttt{cdefgh} {\hl{\textbackslash q\_recursion\_tail} \hl{\textbackslash q\_recursion\_stop}}
\medskip

This is repeated, until the quark is captured which causes the recursion to terminate. The termination is achieved by
the macro |\quark_if_recursion_tail_stop:n|. This will also absorb the |\_recursion_stop| quark. 

While the function is recursing we send the captured letter to a function to decorate and typeset it. This function can be programmed to do whatever you want to achieve. Note in the example it can only accept one argument.

Now what happens, if you wanted to capture two letters at a time or three letters at a time? The program would have to be modified as follows:

\begin{texexample}{Recursion}{ex:l3recursion}
\ExplSyntaxOn
\cs_new:Npn \__my_second_decoration_function:nn #1#2{
   {\color{red}
   [#1#2]}  
}
\cs_set:Npn \mymainother #1
{
 
   \__my_map_other:nn #1  \q_recursion_tail\q_recursion_tail\q_recursion_stop
}

\cs_new:Npn \__my_map_other:nn #1#2
  {
    \quark_if_recursion_tail_stop:n {#1}
    \quark_if_recursion_tail_stop:n {#2}
    \__my_second_decoration_function:nn  {#1}{#2} 
    \__my_map_other:nn
  }

\ExplSyntaxOff 
 
\mymainother {abcdefgh}
\end{texexample}

What just happened, we modified the custom function to accept two arguments, as well as |\_my_map_other:nn|. I also changed  their names to avoid clashes in this document.

In the next example we will iterate through two lists recursively. The first list will provide a string, which we will have to check if it consists of valid character. The valid characters are provided by the second argument of the main macro.


\begin{texexample}{Recursion}{ex:l3recursion}
\ExplSyntaxOn
\cs_new:Npn \ylcompare #1#2
  {
     \__yl_compare_auxi:nN {#2} #1 \q_recursion_tail \q_recursion_stop
  }
  
  
\cs_new:Npn \__yl_compare_auxi:nN #1#2
  {
    \quark_if_recursion_tail_stop:N #2
    \__yl_compare_auxii:nN {#1} #2
    \__yl_compare_auxi:nN {#1}
  }
 
 
\cs_new:Npn \__yl_compare_auxii:nN #1#2
  {
    \__yl_compare_auxiii:NN #2 #1 \q_recursion_tail \q_recursion_stop
  }
\cs_new:Npn \__yl_compare_auxiii:NN #1#2
  {
  % if found not found stop and print
    \quark_if_recursion_tail_stop_do:Nn #2 { \FAIL\  #1 }
  % if not the list end  
    \str_if_eq:nnT {#1} {#2}
      {
        \use_i_delimit_by_q_recursion_stop:nw { \PASS\  #1 }
      }
  % recurse     
    \__yl_compare_auxiii:NN #1
  }
\ExplSyntaxOff

\ylcompare{1234567890AAA}{-1234567890)(}
\ylcompare{text}{abcdefghijklmnopqrst} 
\end{texexample}

How would one modify the above to provide a boolean value if the string is made up only of valid characters? For example for a vowel or alphabet string. This is easy as we can define a boolean, so instead of printing the assertion we would set the boolean at false if it fails. For a number proving string, our method will fail, as we need to test for cases such as |-12345-567|, which is not a valid string also we need to think if we want to allow any spaces. This would probably have to be programmed as a special macro.

\section{Lower level functions}

As we have seen in the section for \tex iteration, one can build almost anything given patience and skills. Many examples can be found in the |expl3| package |fp|. Example~\ref{ex:fp1} is taken from the |fp| package and is a macro to trim leading zeros from a token representing a real number. All the |\@@_| are used in packages to add a prefix when processed through the doc/docstrip system, in this case it will add |fp_|.

\begin{texexample}{Weirds}{ex:fp1}
\makeatletter
\ExplSyntaxOn
 \cs_new:Npn \@@_trim_zeros:w #1 ;
  {
    \@@_trim_zeros_loop:w #1
      ; \@@_trim_zeros_loop:w 0; \@@_trim_zeros_dot:w .; \s__stop
  }
  
\cs_new:Npn \@@_trim_zeros_loop:w #1 0; #2 { #2 #1 ; #2 }

\cs_new:Npn \@@_trim_zeros_dot:w #1 .; { \@@_trim_zeros_end:w #1 ; }

\cs_new:Npn \@@_trim_zeros_end:w #1 ; #2 \s__stop { #1 }
 
 
\@@_trim_zeros:w  121200010.000; 
\ExplSyntaxOff
\end{texexample}


I have removed the |@@_| and replaced them with the |fp_| prefix to make the code more concise and readable.
The main function is delimited with a semi-colon |;| delimited function. Within the macro this is passed onto
|\fp_trim_zeros_loop:w| for further processing. 

\begin{texexample}{Weird}{ex:fp1}
\makeatletter
\ExplSyntaxOn

 \cs_new:Npn \fp_trim_zeros:w #1 ;
  {
    \fp_trim_zeros_loop:w #1;\fp_trim_zeros_loop:w 0; \fp_trim_zeros_dot:w .; \s__stop
  }
  
\cs_new:Npn \fp_trim_zeros_loop:w #1 0; #2 { #2 #1 ; #2 }

\cs_new:Npn \fp_trim_zeros_dot:w #1 .; { \fp_trim_zeros_end:w #1 ; }

\cs_new:Npn \fp_trim_zeros_end:w #1 ; #2 \s__stop { #1 }

 
\fp_trim_zeros:w  131.200010000 ; 
 \ExplSyntaxOff
\end{texexample}

This function is looking for two variables |#1 0; #2| It will scan until its end and then rescan again. The secon time it will absorb |\fp_trim_zeros_dot:w| as its second argument and then continue expanding this function.

\begin{teXXX}
\fp_trim_zeros_loop:w #1;\fp_trim_zeros_loop:w 0; {second macro}
\end{teXXX} 

Weird but wonderful functional programming.

\section{Summary}

This has brought us to almost the end of the |expl3| structures and language. There is much more to cover, but once you become proficient with the syntax and basic usage of its modules, you can pick up the rest through the documentation. 









	\chapter{LaTeX3 Key value system}
\label{l3:keys}
The key-value system has been discussed earlier but avoided to cover the |l3keys| module of \latex3 until such time as the basics of the expl3 syntax was discussed. 


The l3keys modules provides general purpose keyval processing for |expl3| code. However, it does not interact with LaTeX2e's package or class option system. For that, you need to load some additional code, which is available in the package l3keys2e. This provides the \docAuxCommand*{ProcessKeysOptionscommand} to parse class/package options and process them using keyvals defined by l3keys.

The reason for this separation is that l3keys is intended to form part of a LaTeX3 kernel, while l3keys2e is tied to the LaTeX2e model for processing options. It seems extremely likely that a stand-alone LaTeX3 kernel will use keyval options 'natively' but with a different underlying implementation.

 The high level functions here are intended as a method to create
 key--value controls. Keys are themselves created using a key--value
 interface, minimising the number of functions and arguments
 required. Each key is created by setting one or more \emph{properties}
 of the key:
 \begin{verbatim}
   \keys_define:nn { mymodule }
     {
       key-one .code:n   = code including parameter #1,
       key-two .tl_set:N = \l_mymodule_store_tl
     }
 \end{verbatim}
 
  At a document level, |\keys_set:nn| will be used within a
 document function, for example
 \begin{verbatim}
   \DeclareDocumentCommand \MyModuleSetup { m }
     { \keys_set:nn { mymodule } { #1 }  }
   \DeclareDocumentCommand \MyModuleMacro { o m }
     {
       \group_begin:
         \keys_set:nn { mymodule } { #1 }
        ... Main code for the macro
       \group_end:
     }
 \end{verbatim}
 
 The process of incorporating a key value system into a macro or a package involves three steps. First the keys are defined then processed to set them to some values and lastly incorporated into a function or package.
 
 It is best to illustrate the process with a small example. Example\ref{ex:keyval1} defines two keys that affect the typesetting of paragraphs |parindent| and |parskip|. These are defined using the |.code|, pretty much the same way that |pgfkeys| that we discussed earlier defines keys. 
 
 \begin{texexample}{Key value}{ex:keyval1}
 \ExplSyntaxOn
 \keys_define:nn {scratch}
   {
      parindent .code:n = \parindent#1,
      parskip     .code:n = \parskip#1
   }
   
\DeclareDocumentCommand \MyModuleSetup { m }
     { \keys_set:nn { scratch } { #1 }  }
     
\DeclareDocumentCommand \MyModuleMacro { o }
     {
       \group_begin:
         \keys_set:nn { scratch } { #1 }
         % Main code for \MyModuleMacro
         \lorem\par
         \lorem\par
       \group_end:
     }
 \ExplSyntaxOff   
 \MyModuleSetup{parindent=1em, parskip=1pt}
 \MyModuleMacro [parindent=10pt, parskip=10pt]
 \end{texexample}
 
 
 The definition of the keys was achieved using the command:
 
\begin{docCommand}{keys_define:nn}{\marg{module}\marg{keyval list}}
The command parses the \meta{keyval list} and defines the keys associated there for \meta{module}. 
\end{docCommand}

The \meta{keyval list} should consist of one or more key names along with an associated
key \emph{property}. The properties of a key determine how it acts. The individual properties
are described in the following text; Note that the properties of the key begin from the dot (|.|) after the key name. The various properties available take no arguments or require one or more. All key definitions are local. 
 
 \begin{margoptionslist}
 \item [ .code:n] Stores the \meta{code} for execution when \meta{key} is used. 
 \item [.default:n] \meta{key} |.default:n| = \meta{default} This creates a \meta{default} value for \meta{key} if no value is given. This will be used if only the key name is given, but not if a blank \meta{value} is given. This behaviour is similar to the |pgfkeys| package.
 \item [.initial:n] \meta {key} |.initial:n| = \meta{value} Initialises the \meta{key} with the \meta{value}, equivalent to
|\keys_set:nn| \meta{module} \meta{key} = \meta{value}
 
 \item [.dim_set:N] \meta{key} |.dim_set:N| = \meta{dimension} Defines \meta{key} to set \meta{dimension} to \meta{value} (which must a dimension expression). If the variable does not exist, it will be created globally at the point that the key is set up.
 \end{margoptionslist}
 
%  \begin{texexample}{Key value}{ex:keyval1}
% \ExplSyntaxOn
% \dim_new:N \l_parskip
% \dim_new:N \l_parindent
% \keys_define:nn {scratch}
%   {
%      parindent .dim_set:N = \l_parindent,
%     % parindent .initial:n = 0pt,
%      parskip     .dim:n = \l_parskip,
%      %parskip     .initial:n = 1pt,
%      
%   }
%   
%\DeclareDocumentCommand \MyModuleSetup { m }
%     { \keys_set:nn { scratch } { #1 }  }
%     
%\DeclareDocumentCommand \MyModuleMacro { o }
%     {
%       \group_begin:
%         \dim_set_eq:NN \parindent \l_parindent
%         \dim_set_equal:NN \parskip \l_parskip
%         \keys_set:nn { scratch } { #1 }
%         % Main code for \MyModuleMacro
%         \lorem\par
%         \lorem\par
%       \group_end:
%     }
% \ExplSyntaxOff   
% 
% \MyModuleMacro [parindent=10pt, parskip=10pt]
% \end{texexample}

 
 \subsection{Choice keys}
 
 One of the most powerful features of modern key value packages is the ability to define and set keys for mutally exclusive values. In the |l3keys| module this can be achieved using the choice key.
 
 \begin{margoptionslist}
 \item [.choice] \meta{key} |.choice| This sets \meta{key} to act as a choice key. Each choice is then created, as discussed below:
 \end{margoptionslist}
 
 
 \begin{texexample}{Some choices}{}
 \ExplSyntaxOn
 \keys_define:nn { scratchi }
 {
    mycolor .choice:,
    mycolor/fire .code:n = {\color{red}},
    mycolor/sky .code:n = {\color{blue}},
    mycolor/orange .code:n = {\color{orange}},
    mycolor/lemon .code:n = {\color{yellow}},
    mycolor/grass .code:n = {\color{green}},
    mycolor .initial:n =sky,
    mycolor .default:n=orange,
    unknown .code:n={\color{red} ERROR},
 }

\keys_set:nn { scratchi } { mycolor=fire }  

\DeclareDocumentCommand \MyModuleSetup { m }
     { \keys_set:nn { scratchi } { #1 }  }

\DeclareDocumentCommand \MyModuleMacro { o +m}
     {
       \group_begin:
         \keys_set:nn { scratchi } { #1 }
         #2
         \group_end:
     }
     
 \ExplSyntaxOff
    
 \MyModuleSetup{mycolor=fire}

 \MyModuleMacro [mycolor=grass]{grass,} 
 
 \MyModuleMacro [mycolor]{default}
 
 \MyModuleMacro [apple]{}
 
 \MyModuleMacro [fire]{Fire}
\end{texexample} 
 
The |.choice|  key is a bit different from how it is used in the |xtemplate| package and |pgf| but probably easier to use and define. Of course our example was trivial and the colors should have been achieved with just one code key, capturing the value. It takes some practice to get used to all the types of keys available and to develop error free code easily, but by using a key value system, truly flexible, modern functions can be developed.
 

\subsection{Handling of unknown keys}
 
 Handling of unknown keys is similar to |pgf| where a key defined as |.unknown| is defined. 
 If a key has not previously been defined (is unknown), |\keys_set:nn| will look for a special
unknown key for the same module, and if this is not defined raises an error indicating that
the key name was unknown. This mechanism can be used for example to issue custom
error texts.

\begin{verbatim}
\keys_define:nn { mymodule }
{
unknown .code:n =
You~tried~to~set~key~’\l_keys_key_tl’~to~’#1’.
}
\end{verbatim}
 
 
 As for |pgf| there are many other key types and these are listed in the |l3keys| manual and are not listed here for brevity. 
 
 
	\cxset{section align=left,
       section font-weight=bold}
       
\chapter{The LaTeX3 l3msg Module and how to use it for Error, Warning and other Messages}
\index{messaging>error}\index{messaging>warning}
\section{Introduction}

Messages need to be passed to the user by modules, either when errors occur or to indicate
how the code is proceeding. The l3msg module provides a consistent method for doing
this (as opposed to writing directly to the terminal or log).

The system used by l3msg to create messages divides the process into two distinct
parts. Named messages are created in the first part of the process; at this stage, no
decision is made about the type of output that the message will produce. The second
part of the process is actually producing a message. At this stage a choice of message
class has to be made, for example error, warning or info.

By separating out the creation and use of messages, several benefits are available.
First, the messages can be altered later without needing details of where they are used
in the code. This makes it possible to alter the language used, the detail level and so
on. Secondly, the output which results from a given message can be altered. This can be
done on a message class, module or message name basis. In this way, message behaviour
can be altered and messages can be entirely suppressed.

\section{Creating messages}

Messages \emph{must} be created before they can be used. This has the advantage that they can be used
over and over and also one could start thinking of internationilizing the package.

Messages can be created as either new or set and there is also a TF to check if the message exists:

\begin{docCommand}{msg_new:nnnn}{ \marg{module} \marg{message} \marg{text} \marg{more text} }
Creates a hmessagei for a given hmodulei. The message will be defined to first give htexti
and then \meta{more text} if the user requests it. If no \meta{more text} is available then a standard
text is given instead. Within htexti and more text four parameters (\#1 to \#4) can be
used: these will be supplied at the time the message is used. An error will be raised if
the \meta{message} already exists.
\end{docCommand}

\section{Messaging classes}

Messages are divided into categories termed message classes. 
These are according to \emph{severity}, \emph{fatal}, \emph{critical}, \emph{error}, \emph{warning} and \emph{info}. Each one has its own set of creation functions.

\begin{docCommand*}{msg_fatal:nnnnnn}{\marg{module} \marg{message} \marg{arg one} \marg{arg two} \marg{arg three}
\meta{arg four}}
Issues \meta{module} error \meta{message}, passing \meta{arg one} to \meta{arg four} to the text-creating
functions. After issuing a fatal error the \tex run will halt.
\end{docCommand*}



\begin{texexample}{Typical package error setup}{ex:errors}
% Error message example
%
% simulate LaTeX2e \fmversion
\def\fmversion{2000/11/12}
\makeatletter
\ExplSyntaxOn 

% create error boolean
\bool_new:N \l_mypackage_error_bool
 
% redirect package errors here  %(*@\label{l:warning}@*)
\cs_new_protected:Npn \mypackage_warning:nxx #1 #2 #3 
  {
    \bool_set_true:N \l_mypackage_error_bool
    \msg_info:nnxx { mypackage } { #1 } { #2 } { #3 }
  }
 
% define some error messages
\msg_set:nnnn { mypackage } { old-version }
  { LaTeX~source~files~more~than~5~year~old.~ Is~dated~(year:#1~date:#2-#3) }
  { Please~update~your~distribution~visit~ctan } %(*@\label{test}@*)
   

% check version number   	   	
\cs_new:Npn \mypackage_check_version:n #1 
  {
    \exp_after:wN \l_mypackage_check_version_aux:w #1\q_stop
  }

% check version number auxiliary 
% #1 relation
% #2 true code
% #3 false code  
\cs_new:Npn \l_mypackage_check_version_aux:w #1/#2/#3\q_stop 
  {
    \int_compare:nNnTF { ( \tex_year:D-#1 )*12 + (\tex_month:D-#2) } > { 65 }
      { \FAIL \mypackage_warning:nxx { old-version } { #1 } { #1 / #2 / #3 } }
      { \PASS } 
  }
  
\mypackage_check_version:n \fmversion 
\mypackage_check_version:n \fmtversion 
  
\ExplSyntaxOff
\end{texexample}

In Example~\ref{ex:errors} Line \ref{test} we define our own package command for issuing an error message, rather than typing |\msg_error:nnxx| directly. This is considered good practice and we avoid typing in \enquote{mypackage} all the time. 

\section{How to translate strings}
\index{internationalization>expl3}

I posted similar code to the |TX.SX| Q\&A site to elicit comments from other users, as to recommended best practices.
%http://tex.stackexchange.com/questions/246810/latex3-l3msg-best-practices 
At this moment in time I am not too sure if a LaTeX3 approach to internationalization is appropriate. If one looks at the complexities, it is preferable to use Lua. 






	\parindent1em
\chapter{LaTeX3 counters and registers}

 \section{Introduction}
 
 This \latex3 module is dealing with integer arithmetic as well as utilizing these to create counter data type structures. It can be used to develop counters in a similar fashion to \latexe, although the user level functions are missing.  All control squences in this module are prefixed with |\int|. Some of the examples are prefixed as |phd_counter| and they emulate \latex’s counter commands. In this chapter also we describe a rather long example to typeset a table of numbers in various notations, such as roman, literal, arabic etc.
 
 \section{Integer expressions}

 Calculation and comparison of integer values can be carried out
 using literal numbers, \texttt{int} registers, constants and
 integers stored in token list variables. The standard operators
 \texttt{+}, \texttt{-}, \texttt{/} and \texttt{*} and
 parentheses can be used within such expressions to carry
 arithmetic operations. This module carries out these functions
 on \emph{integer expressions} (\enquote{\texttt{intexpr}}).

 \begin{docCommand}{int_eval:n} {\marg{integer expression}}
    Evaluates the \meta{integer expression}, expanding any
   integer and token list variables within the \meta{expression}
   to their content (without requiring \docAuxCommand*{int_use:N}/\docAuxCommand*{tl_use:N})
   and applying the standard mathematical rules. For example both
 \end{docCommand}
   
   \begin{verbatim}
     \int_eval:n { 5 +  4 * 3 - ( 3 + 4 * 5 ) }
   \end{verbatim}
   and
   \begin{verbatim}
     \tl_new:N  \l_my_tl
     \tl_set:Nn \l_my_tl { 5 }
     \int_new:N  \l_my_int
     \int_set:Nn \l_my_int { 4 }
    \int_eval:n { \l_my_tl +  \l_my_int * 3 - ( 3 + 4 * 5 ) }
   \end{verbatim}
   both evaluate to \( -6 \). The  \marg{integer expression} may
   contain the operators \texttt{+}, \texttt{-}, \texttt{*} and
   \texttt{/}, along with parenthesis \texttt{(} and \texttt{)}.
   Any functions within the expressions should expand to an
   \meta{integer denotation}: a sequence of a sign and digits matching
   the regex |\-?[0-9]+|).
   After expansion \docAuxCommand*{int_eval:n} yields an  \meta{integer denotation}
   which is left in the input stream.
 
  
     Exactly two expansions are needed to evaluate \docAuxCommand*{int eval:n}.
     The result is \emph{not} an \meta{internal integer}, and therefore
     requires suitable termination if used in a \TeX{}-style integer
     assignment.
   
 
  
  If you familiar with e-tex’s |numexpr|, this is equivalent code. 
 
  \begin{texexample}{Integer Evaluation}{ex:numexpr}
  \ExplSyntaxOn
   \int_eval:n { 5 +  4 * 3 - ( 3 + 4 * 5 )+2*2 }\par
   \int_to_arabic:n { ( 2+7 ) / 3 }\par
   \int_to_alph:n { 2+7 } \par
   \int_to_Alph:n { 6 * 2 }\par
   \int_to_roman:n { 9 } \par
   \int_to_Roman:n { 21 } \par
  \ExplSyntaxOff 
  \end{texexample} 
  

 
 \section{Creating and initializing integers} 
  
These are \latex’s equivalents of counters. Numerous commands are provided by the \latex3 kernel and these in my opinion offer a much better interface to lower level commands. A common question is that what one does if it is required to access a \latexe counter. The more-or-less “official” answer was provided by Joseph Wright at the Stack Exchange\footnote{\protect\url{http://tex.stackexchange.com/questions/167094/manipulate-a-latex2e-counter-with-latex3}} Q\&A site with the recommendation that: `mixing up the two interaces is asking for trouble, and while we are working on several areas we’ve not got a “user level” counter approach at yet. (Indeed, the entire question of how variables at the document-level should be handled is open.)’ 
  
  So you have it you keep on using commands such as \docAuxCommand*{setcounter} here is that |\c@..|. is an internal for LaTeX2e, and the entire point of expl3 is to have clear interfaces and internals. There's no reason to abuse the interfaces here (no functionality gain), so stick with them. In my opinion also it is not a good idea to mix |\@| notation with |expl3| notation. When there is a need to use both it is best to clearly separate the two and create an interface if they must somehow share information.

 

  Before we go further with the code is instructive to peek at the \latex3 kernel and understand what is an integer. An integer is simply a \tex \docAuxCommand*{newcount}, as shown from the code below.
  
  \begin{teXXX}
  \cs_new_protected:Npn \int_new:N #1
  {
    \__chk_if_free_cs:N #1(*@\label{allocation}@*)
    \cs:w newcount \cs_end: #1
  }
 \end{teXXX} 
  

   
 \begin{texexample}{allocations}{ex:counter allocations} 
 \ExplSyntaxOn
 \newcount\somecounter
 \meaning\somecounter\par
 
 \int_new:N\someothercounter
 \meaning\someothercounter\par
 
 \tl_map_inline:nn {
    \somecounter
    \someothercounter
  }
  { \cs_undefine:N #1 }
  
  \meaning\somecounter
 \ExplSyntaxOff
 \end{texexample}
 
 As you can observe from the example, using |\int_new:N| to create a counter is identical to the \latexe |\newcount|. It is instructive to keep this in mind later on and in your code, during debugging. Line~\ref{allocation} checks if the |\count| is available and then allocates the counter to the command sequence.
 Once we typeset the example, I have used |\cs_undefined| in a sequence to free the register. This is always good practice.  You must be careful not to get confused here with the terminology, we are dealing with \tex’s primitive |\count| registers.\footnote{To be more specific e-\tex.} Although the original \tex came only with 256 registers the new engines allow up to 65535 count registers. 
  
 \begin{docCommand}{int_new:N}{\meta{integer}}
   Creates a new \meta{integer} or raises an error if the name is
   already taken. The declaration is global. The \meta{integer} will
   initially be equal to $0$.
 \end{docCommand}
 
 In most instances counters involve a three step operation:
 
 \begin{enumerate}
 \item Creating the counter
 \item Adding values
 \item Typesetting the value or using it in another expression
 \end{enumerate}
 

 
 \begin{texexample}{Counters}{ex:l3counters}
 \ExplSyntaxOn
 \int_new:N \exercise
 \int_add:Nn \exercise {12+15}
 \int_to_roman:n \exercise \\
 
 \int_use:N \c_max_register_int
 \ExplSyntaxOff
 \end{texexample}

The \docAuxCommand*{int_use:N} is the \tex primitive \docAuxCommand*{the}. This is one of several \latex3 names of the primitive.


 \begin{docCommand}{int_const:Nn}{\meta{integer} \marg{integer expression}}
   Creates a new constant \meta{integer} or raises an error if the name
   is already taken. The value of the \meta{integer} will be set
   globally to the \meta{integer expression}.
 \end{docCommand}
 
The next commands can be used for round off, absolute functions etc.

 \begin{docCommand}{int_abs:n}{\marg{integer expression}}
   Evaluates the \meta{integer expression} as described for
   \docAuxCommand*{int_eval:n} and leaves the absolute value of the result in
   the input stream as an \meta{integer denotation} after two
   expansions.
 \end{docCommand}
 
 \begin{docCommand}{int_div_round:nn}{\marg{intexpr1} \marg{intexpr2}}
   Evaluates the two \meta{integer expressions} as described earlier,
   then divides the first value by the second, and rounds the result
   to the closest integer.  Ties are rounded away from zero.
   Note that this is identical to using
   |/| directly in an \meta{integer expression}. The result is left in
   the input stream as an \meta{integer denotation} after two expansions.
 \end{docCommand}
 

 \begin{docCommand}{int_div_truncate:nn}{ \marg{intexpr1} \marg{intexpr2}}
   Evaluates the two \meta{integer expressions} as described earlier,
   then divides the first value by the second, and rounds the result
   towards zero.  Note that division using |/|
   rounds the result. The result is left in the input stream as an
   \meta{integer denotation} after two expansions.
 \end{docCommand}
 
 \begin{texexample}{Truncating}{ex:truncate}
 \ExplSyntaxOn
 
 \int_div_round:nn  {10}{3}~~
 \int_div_truncate:nn  {13}{3}
 
 \ExplSyntaxOff
 \end{texexample}




\begin{docCommand}{int_max:nn}{ \marg{intexpr1} \marg{intexpr2}}
   Evaluates the \meta{integer expressions} as described for
   \docAuxCommand*{int_eval:n} and leaves either the larger or smaller value
   in the input stream as an \meta{integer denotation} after two
   expansions. The minimum of two numbers an be fund using |\int_min:nn|
\end{docCommand}

 \begin{texexample}{Finding minima and maxima}{ex:maxima}
 \ExplSyntaxOn
 
 \int_max:nn  {10}{3}~~
 \int_min:nn  {13}{3}
 
 \ExplSyntaxOff
 \end{texexample}


  \begin{docCommand}{int_mod:nn}{ \marg{intexpr1} \marg{intexpr2}}
   Evaluates the two \meta{integer expressions} as described earlier,
   then calculates the integer remainder of dividing the first
   expression by the second.  This is obtained by subtracting
   \docAuxCommand*{int_div_truncate:nn} \marg{intexpr1} \marg{intexpr2} times
   \meta{intexpr2} from \meta{intexpr1}.  Thus, the result has the
   same sign as \meta{intexpr1} and its absolute value is strictly
   less than that of \meta{intexpr2}.  The result is left in the input
   stream as an \meta{integer denotation} after two expansions.
   (See example~\ref{ex:mod}).
   
 \end{docCommand}
  
 \begin{texexample}{Modulus}{ex:mod}
 \ExplSyntaxOn
     \int_mod:nn  {10+13}{3+3}~~
 \ExplSyntaxOff
 \end{texexample}

\subsection{Setting and incrementing integers}

\begin{docCommand}{int_add:Nn}{\meta{integer} \marg{integer expression}}
   Adds the result of the \meta{integer expression} to the current
   content of the \meta{integer}.
 \end{docCommand}

\begin{docCommand}{int_incr:Nn}{\meta{integer} \marg{integer expression}}
 Increases the value stored in \meta{integer} by $1$.
 \end{docCommand}


\begin{docCommand}{int_decr:Nn}{\meta{integer} \marg{integer expression}}
  Decreases the value stored in \meta{integer} by $1$. 
 \end{docCommand}
 
 \begin{docCommand}{int_set:Nn}{ \meta{integer} \marg{integer expression}}
   Sets \meta{integer} to the value of \meta{integer expression},
   which must evaluate to an integer.
 \end{docCommand}
 
 \begin{docCommand}{int_sub:Nn} {\meta{integer} \marg{integer expression}}
   Subtracts the result of the \meta{integer expression} from the
   current content of the \meta{integer}.
 \end{docCommand}  

  \section{Using integers}
  
 Although we have already used \docAuxCommand*{int_use:N} to recover and typeset the value of a counter
 we now give its formal definition and an example of usage. As this is the primitive |\the| some care needs to 
 be  taken with expansion.
 

 \begin{docCommand}{int_use:N}{ \meta{integer}}
   Recovers the content of an \meta{integer} and places it directly
   in the input stream. An error will be raised if the variable does
   not exist or if it is invalid. Can be omitted in places where an
   \meta{integer} is required (such as in the first and third arguments
   of \docAuxCommand*{int_compare:nNnTF}).
 \end{docCommand}
 
 If we are to follow \latexe’s paradigm counters are names and not command sequences at the user level.
 With \latex3 of course we can define them as both. In Example~\ref{ex:intuse}, we use the |:c| variant of the commands to define a counter \meta{somecounter}, add globally an integer expression and then retrieve and typeset its value. 
 
 \begin{texexample}{More on retrieving values}{ex:intuse}
 \ExplSyntaxOn
   \int_new:c {somecounter}
   \int_gadd:cn {somecounter}{263+223/23}
   \int_use:c {somecounter}
 \ExplSyntaxOff
 \end{texexample}
 
\subsection{Longer example}

We wish to create a table that would list numbers and their literal equivalents.

 \begin{texexample}{Creating a numbered table}{ex:inc}
 \ExplSyntaxOn
 \int_gzero_new:N \phd_step_int
 \cs_new:Nn \g_phd_step_counter:n {
     \int_gincr:N\phd_step_int 
     \int_use:N \phd_step_int
 }
 
 \DeclareDocumentCommand\Inc{}{
    \g_phd_step_counter:n
 }
 \begin{tabular}{ll}
 \Inc  & One \\
 \Inc  & Two\\
 \Inc  & Three\\
 \end{tabular}
 
  \ExplSyntaxOff
 \end{texexample}
 
 With all these syntactic changes to the \latex code and conventions, perhaps we should retrace our steps to Knuth’s original terminology, Lamport’s structural documents concepts and a more simplistic language as expounded by my own philosophy in the phd package.

So what do we need, first we need to think that all these functions and programming are to produce documents, so our higher level macros should be document focused. 

The intention of the design layer is to provide interfaces that allow specifying layout and typography styles in a declarative way. On the implementation side there are a number of prototype implementations (most notably xtemplate and the recent reimplementation of the ldb). Those need to get unified into a common model which requires some more experimentation and proably also some further thoughts.

But the real importance of this layer is not the implementation of its interfaces but the conceptual view of it: provisioning a rich declarative method (or methods) to describe design and layout. I.e., enabling a designer to think not in programs but in visual representations and relationships.

 \begin{texexample}{Counter concepts}{ex:inc2}
 \makeatletter
 \ExplSyntaxOn

  \DeclareDocumentCommand\NewCounter{ m } {
     \int_gzero_new:c {#1}
     % create auxiliary functions
     \cs_set:Nn \g_phd_stepcounter:n {
         \int_gincr:c {#1} 
         \int_use:c {#1}
      }
  }
  
 \DeclareDocumentCommand\StepCounter{ m } { 
    \g_phd_stepcounter:n {#1} 
}  
  
 \DeclareDocumentCommand\SetCounter { m m } {
    \int_gset:cn {#1}{#2}
 }

    
\NewCounter{phd_temp_counter} 

 \DeclareDocumentCommand\Inc{}{
    \StepCounter{phd_temp_counter}
 }
 
 \SetCounter{phd_temp_counter}{12}
 
 \DeclareDocumentCommand\CounterToAlpha{ m }{
     \edef\x{\int_use:c{#1}}
     \int_to_Alph:n {\x}
}   
 \DeclareDocumentCommand\CounterToalpha{ m }{
     \edef\x{\int_use:c{#1}}
     \int_to_alph:n {\x}
}   

\DeclareDocumentCommand\CounterToRoman{ m }{
     \edef\x{\int_use:c{#1}}
     \int_to_Roman:n {\x}
}
\DeclareDocumentCommand\CounterToroman{ m }{
     \edef\x{\int_use:c{#1}}
     \int_to_roman:n {\x}
}
\DeclareDocumentCommand\IncA{}{
    \CounterToAlpha{phd_temp_counter}
}
\DeclareDocumentCommand\Inca{}{
    \CounterToalpha{phd_temp_counter}
}

\DeclareDocumentCommand\IncR{}{
    \CounterToRoman{phd_temp_counter}
}
\DeclareDocumentCommand\Incr{}{
    \CounterToroman{phd_temp_counter}
}
\DeclareDocumentCommand\IncW{}{
    \edef\x{\int_use:c{phd_temp_counter}}
    \expandafter\Words@cx{\x}
}
\DeclareDocumentCommand\Incw{}{
    \edef\x{\int_use:c{phd_temp_counter}}
    \expandafter\words@cx{\x}
}

 \begin{tabular}{c c c c c c c}
 \toprule
 Number & Literal & literal &Alpha &alpha &Roman &roman\\
 \midrule
 \Inc  & \IncW &\Incw &\IncA &\Inca &\IncR &\Incr\\
 \Inc  & \IncW &\Incw &\IncA &\Inca &\IncR &\Incr\\
 \Inc  & \IncW &\Incw &\IncA &\Inca &\IncR &\Incr\\
 \Inc  & \IncW &\Incw &\IncA &\Inca &\IncR &\Incr\\
 \bottomrule
 \end{tabular}
 
 \ExplSyntaxOff
 \makeatother
 \end{texexample}

What have just happened in our example, we have emulated most of \latexe counter macros in a kind of a different way, but there is an important part of the counter macros that is missing. This is the ability to reset counters when a master counter is changed. For example when a chapter counter is incremented the section counters in \latexe will reset and start counting from one again.



\docAuxCommand*{cl@foo} List of counters to be reset when foo stepped. This has a  format
\begin{verbatim}
   \@elt{countera}\@elt{counterb}\@elt{counterc}
\end{verbatim}

\textbf{Adding a prefix to the counters} So what we need to do first decide on some prefixes for counters or another similar convention. I will use a prefix |__counter_| to make the code readable. Although tempting to use |c@|  to make our counters compatible with \latexe counters, as stated earlier this would be against the central philosophy of \latex3 of keeping a onsistent syntax and function aiming conventions. All we have to do is add the prefixes and create the new counter to hold the rest counters and provide helper commands. We can also do some error capturing. 


\begin{teXXX}
 \DeclareDocumentCommand\NewCounter{ o m } {
  % create new counter if it does not exist and also its resets counters
   \int_if_exist:NTF {__counter_#2}
     {
         \msg_error:nn { counters } { counter exists and cannot be created }
     }
     { 
         \int_gzero_new:c {__counter_#2}
         \int_gzero_new:c {__counter_resets_#2}
     }
 
  % handle the reset   
  \IfNoValueF{#1}
     {
         \int_if_exist:NTF {__counter_resets_#1} 
             {add to reset} {false code}
     }    
 
  % create auxiliary functions  to be added later   
       
   }
\end{teXXX}
    
This is a rough outline of the code we need to develop. It is considered bad practice to mix too many low level commands with high level commands and we should replace these with auxiliary functions. The auxiliary functions will create automated functions such as |thechapter| in \latexe.

\begin{teXXX}
\cs_new:Nn \phd_create_new_counter:n {
    \int_if_exist:NTF {__counter_#2}{error}
         { 
             \int_gzero_new:c {__counter_#2}
             \int_gzero_new:c {__counter_resets_#2}
         }
}
\end{teXXX}

We continue our code development, this time we will add a prefix before the counter name. The code is shown
in Example~\ref{ex:createcounters}



\begin{texexample}{Auxiliary constructor function}{ex:createcounters}
\ExplSyntaxOn
\makeatletter
\cs_gset:Nx  \counter_prefix: {c@}
\cs_gset:Nx  \counter_resets_prefix: {__counter_resets_prefix_}

\cs_new:Nn  \phd_create_new_counter:n {
    \int_if_exist:cTF {\counter_prefix:#1}{ERROR~counter~exists}
        { 
            \int_gzero_new:c {\counter_prefix:#1}
            \seq_new:c {\counter_resets_prefix:#1}
        }
}

\phd_create_new_counter:n {test}

\phd_create_new_counter:n {test2}

\cs_new:Nn\stepacounter:n {
  \int_gincr:c{\counter_prefix: #1}
}

\cs_new:Nn\setacounter:cn {
  \int_set:cn {\counter_prefix: #1}{#2}
}

\cs_gset:Nn  \countervalue:n {
    \the\cs:w\counter_prefix: #1\cs_end:\relax
}

\newcommand{\makeauxiliaries}[1]{\mycommandaux#1\relax}
    \def\mycommandaux#1#2\relax{%
       \uppercase{
       \expandafter\gdef\csname #1}
       #2Value
       \endcsname
       {\the\cs:w\counter_prefix: #1#2\cs_end:\relax}
    }

\makeauxiliaries {test}

\setacounter:cn {test}{18}

\countervalue:n {test}\par


\makeauxiliaries {section}
first~test~\TestValue\par 
\stepacounter:n {test}
\stepacounter:n {test}

second~test~\TestValue\par 

section~counter~first~test:~\SectionValue\par
section~counter~with~\docAuxCommand*{thesection}:~\thesection\par
\stepacounter:n {section}
section~counter~second~test:~\SectionValue
\ExplSyntaxOff
\makeatother
\end{texexample}



\ExplSyntaxOn
\newcommand{\makeauxiliaryfunctions}[1]{\mycommandaux#1\relax}
\def\mycommandaux#1#2\relax{%
       \uppercase{\expandafter\gdef\cs:w #1}#2Value\cs_end:
       {\tex_the:D\cs:w\counter_prefix: #1#2\cs_end:\relax}%
    }
\ExplSyntaxOff
    
The example above creates a function |\phd_create_new_counter:n| and then tests it. The function uses a conditional to test if the counter exists  and then if it has not been defined earlier it creates the two counters and sets them to zero. If it exists it will typeset an error message. This is of course so that we can view the error in the document, normally this would display an error on the screen. I have not covered the messaging part of the code so far for displaying errors and warnings. These are created with the \pkgname{l3msg} package, which is bundled with |expl3|. We will cover this later on in the book. 

\textbf{Adding the counter to the reset list of another} We now go on to develop a function to add to the reset list of another.

\begin{texexample}{Adding to the reset}{ex:addtoreset}
\ExplSyntaxOn
\cs_gset:Nn \addtoreset:nn {
    \exp_args:Nf\seq_put_left:cn {\counter_resets_prefix:#1}{#2}
    Added~ to~the~ #1 ~ resets~ #2.~The~resets~list~is~now~\seq_use:cn {\counter_resets_prefix:#1}{,}
 }

\ExplSyntaxOff
\end{texexample}

The |\stepcounter:c| will be used to step a counter and to reset all subsidiary counters. 


\begin{texexample}{Adding to the reset}{ex:addtoreset}
\ExplSyntaxOn
\makeatletter
\cs_gset:Nn \resetcounter:c 
  {
    \int_gset:cn {\counter_prefix: #1}{0}
  }
  
\cs_gset:Nn \stepcounter:c {
  \int_gincr:c {\counter_prefix: #1}
  \seq_set_eq:Nc \tempa {__counter_resets_prefix_#1}
  \seq_map_inline:Nn \tempa {\resetcounter:c{##1}}
}      

\stepcounter:c {test}

% Test that the value is captured
\int_use:c {\counter_prefix: test}

\makeatother
\ExplSyntaxOff
\end{texexample}

Now that we have developed the code for |\addtoreset:nn| we are ready to modify and finalize our counter creation macro to the following:

\begin{texexample}{Refactor creation macro}{ex:refactor}
\ExplSyntaxOn
 \DeclareDocumentCommand \NewCounter{ o m } {
    \phd_create_new_counter:n {#2}
    \IfNoValueF{#1}
      {
         \int_if_exist:cT {\counter_resets_prefix:#1} 
             {\addtoreset:nn{#1}{#2}} 
      }    
    \makeauxiliaryfunctions {#1}
 }
%\NewCounter{Chapter}
%\NewCounter[Chapter]{Section}\par
%\NewCounter[Chapter]{Figure}
%\stepcounter:c{Chapter}
%\int_use:c {\counter_prefix: Chapter} 
%\int_use:c {\counter_prefix: Section}
%\countervalue:n{Chapter}
%\ChapterValue
\ExplSyntaxOff
\end{texexample}

%% Rewrite as the examples are in a group and they cannot leak out

\ExplSyntaxOn
\NewDocumentCommand\NewCounter{ o m } {
    \phd_create_new_counter:n {#2}
    \IfNoValueF{#1}
      {
         \int_if_exist:cT {\counter_resets_prefix:#1} 
             {\addtoreset:nn{#1}{#2}} 
      }    
    \makeauxiliaryfunctions {#1}
}
\ExplSyntaxOff

All we have to now do is to write some tests. The \latex3 Team provide testing routines with the tests being run using a Lua script. In our case we can run all our tests within the documentation here. These tests are shown in Example~\ref{ex:countertests}. 


\begin{texexample}{Counter module tests }{ex:countertests}
\ExplSyntaxOn
\NewCounter{Chapter}
\NewCounter[Chapter]{Section}\par
\NewCounter[Chapter]{Figure}\par
\NewCounter[Chapter]{Problems}\par
\stepcounter:c{Chapter}
\stepcounter:c{Chapter}
Value~ of~ Chapte~ counter:~ \int_use:c {\counter_prefix: Chapter}\par 
Value~of~Section~ counter:~\int_use:c {\counter_prefix: Section}\par
Value~of~Chapter~ counter~ using \docAuxCommand*{countervalue:n}:~ \countervalue:n{Chapter}\par
Value~of~Chapter~ counter~ using \docAuxCommand*{ChapterValue}:~ \ChapterValue\par
\ExplSyntaxOff
\end{texexample}


\begin{docCommand}{ChapterCounter}{}
   Typesets the Chapter counter. This is equivalent to |\thechapter|. Decorating the actual counter value, should
   all be based on a key-value system and the author has no access to this function. It is the template designer’s 
   job to define it.
   
  \begin{verbatim}
   chapter numbering = arabic
   chapter font-size = huge
  \end{verbatim} 
  
\end{docCommand}

\begin{docCommand}{ChapterCounterValue}{}
   Typesets the Chapter counter in arabic. This form can also be used in other expressions This is equivalent to \latexe’s \docAuxCommand*{c@chapter}, but not equal, i.e, there is no interface to \latexe counters.
\end{docCommand}
  
  
 \subsection{Integer conditionals}

Comparing the values of two counters can be achieved with the use of conditional expressions. There are numerous commands provided for this purpose and we outline some of the most important ones. Do consult the manual to view others. The first one we will examine is \docAuxCommand*{int_compare:nNnTF}. This function evaluates each of two expressions and branches to the true or false code. 

\begin{docCommand}{int_compare:nNnTF}{\marg{intexpr1} \meta{relation} \marg{intexpr2}
\marg{true code} \marg{false code}}
   This function first evaluates each of the \meta{integer expressions}
   as described for \docAuxCommand*{int_eval:n}. The two results are then
   compared using the \meta{relation}:
   \begin{center}
     \begin{tabular}{ll}
       Equal                 & |=| \\
       Greater than      & |>| \\
       Less than           & |<| \\
     \end{tabular}
   \end{center}
 \end{docCommand}
 
 Consider two counters \docAuxCommand*{counteri} and \docAuxCommand*{counterii} that we need to compare their values and branch to either false or true code.
 
 \begin{texexample}{Integer conditionals}{ex:intconditionals}
 \ExplSyntaxOn
 \int_new:N  \counteri
 \int_new:N  \counterii
 
 \int_compare:nNnTF {\counteri} = {\counterii}
     {true~code\par}{false~ code\par}
     
 \int_gadd:Nn \counterii {15+12}    
 
  \int_compare:nNnTF {\counteri} = {\counterii}
     {true~code~\par}{false~ code\par}
     
\ifnum\counteri<\counterii primitive~ifnum~true\else primitive~false\fi\par
   
 \ExplSyntaxOff
 \end{texexample}
 
 A common error is to include the \meta{relation} code in curly brackets. This leads to errors during parsing. 
 
   
 
 
  \begin{docCommand}{int_case:nnTF} {\marg{test integer expression}
      \{ 
     \marg{intexpr case1} \marg{code case1} 
     \marg{intexpr case2} \marg{code case2} 
     \ldots 
     \marg{intexpr casen} \marg{code casen} 
     \} 
     \marg{true code}
     \marg{false code}}
     
   This function evaluates the \meta{test integer expression} and
   compares this in turn to each of the
   \meta{integer expression cases}. If the two are equal then the
   associated \meta{code} is left in the input stream. If any of the
   cases are matched, the \meta{true code} is also inserted into the
   input stream (after the code for the appropriate case), while if none
   match then the \meta{false code} is inserted. The function
   \docAuxCommand*{int_case:nn}, which does nothing if there is no match, is also
   available. For example
   \begin{texexample}{Case example}{ex:case}
   \makeatletter
   \ExplSyntaxOn
     \int_case:nnF
       { 2 * 5 }
       {
         { 5 }       { Small }
         { 4 + 6 }   { Medium }
         { -2 * 10 } { Negative }
       }
       { No idea! }
       
   
  \cs_set:Npn \@fnsymbol #1
   {
    \int_case:nnF {#1}
     {
      {0} {}
      {1} { \TextOrMath \textasteriskcentered* }
      {2} { \TextOrMath \textdagger\dagger }
      {3} { \TextOrMath \textdaggerdbl\ddagger }
      {4} { \TextOrMath \textsection\mathsection }
      {5} { \TextOrMath \textparagraph\mathparagraph }
      {6} { \TextOrMath \textbardbl\| }
      {7} { \TextOrMath {\textasteriskcentered\textasteriskcentered}{**} }
      {8} { \TextOrMath {\textdagger\textdagger}{\dagger\dagger} }
      {9} { \TextOrMath {\textdaggerdbl\textdaggerdbl}{\ddagger\ddagger} }
     }
     { \@ctrerr }
   }
   
   
   \@fnsymbol {3}
     \ExplSyntaxOff  
     \makeatother
    \end{texexample}
    
 \end{docCommand}
 
 The next case example is from \href{https://github.com/wspr/fontspec/blob/master/fontspec.dtx}{fontspec.dtx}. It just wraps the official \latexe definition from \pkgname{fixltx2e} into the |expl| language. If you are going to utilize code from \latexe verbatim, it is always best to use it as is and only using a wrapper.

 \begin{texexample}{Case example}{ex:case2}
   \makeatletter
   \ExplSyntaxOn
   
  \cs_set:Npn \@fnsymbol #1
   {
    \int_case:nnF {#1}
     {
      {0} {}
      {1} { \TextOrMath \textasteriskcentered* }
      {2} { \TextOrMath \textdagger\dagger }
      {3} { \TextOrMath \textdaggerdbl\ddagger }
      {4} { \TextOrMath \textsection\mathsection }
      {5} { \TextOrMath \textparagraph\mathparagraph }
      {6} { \TextOrMath \textbardbl\| }
      {7} { \TextOrMath {\textasteriskcentered\textasteriskcentered}{**} }
      {8} { \TextOrMath {\textdagger\textdagger}{\dagger\dagger} }
      {9} { \TextOrMath {\textdaggerdbl\textdaggerdbl}{\ddagger\ddagger} }
     }
     { \@ctrerr }
 }
 
   
 \@fnsymbol {3}
 \ExplSyntaxOff  
 \makeatother
 \end{texexample}
    

%
% \begin{function}[updated = 2013-01-13, EXP, pTF]{\int_compare:n}
%   \begin{syntax} 
%     \docAuxCommand*{int_compare_p:n} \\
%     ~~\{ \\
%     ~~~~\meta{intexpr_1} \meta{relation_1} \\
%     ~~~~\ldots{} \\
%     ~~~~\meta{intexpr_N} \meta{relation_N} \\
%     ~~~~\meta{intexpr_{N+1}} \\
%     ~~\} \\
%     \docAuxCommand*{int_compare:nTF}
%     ~~\{ \\
%     ~~~~\meta{intexpr_1} \meta{relation_1} \\
%     ~~~~\ldots{} \\
%     ~~~~\meta{intexpr_N} \meta{relation_N} \\
%     ~~~~\meta{intexpr_{N+1}} \\
%     ~~\} \\
%     ~~\Arg{true code} \Arg{false code}
%   \end{syntax}
%   This function evaluates the \meta{integer expressions} as described
%   for \docAuxCommand*{int_eval:n} and compares consecutive result using the
%   corresponding \meta{relation}, namely it compares \meta{intexpr_1}
%   and \meta{intexpr_2} using the \meta{relation_1}, then
%   \meta{intexpr_2} and \meta{intexpr_3} using the \meta{relation_2},
%   until finally comparing \meta{intexpr_N} and \meta{intexpr_{N+1}}
%   using the \meta{relation_N}.  The test yields \texttt{true} if all
%   comparisons are \texttt{true}.  Each \meta{integer expression} is
%   evaluated only once, and the evaluation is lazy, in the sense that
%   if one comparison is \texttt{false}, then no other \meta{integer
%     expression} is evaluated and no other comparison is performed.
%   The \meta{relations} can be any of the following:
%   \begin{center}
%     \begin{tabular}{ll}
%       Equal                    & |=| or |==| \\
%       Greater than or equal to & |>=|        \\
%       Greater than             & |>|         \\
%       Less than or equal to    & |<=|        \\
%       Less than                & |<|         \\
%       Not equal                & |!=|        \\
%     \end{tabular}
%   \end{center}
% \end{function}
%

%
% \begin{function}[EXP,pTF]{\int_if_even:n, \int_if_odd:n}
%   \begin{syntax}
%     \docAuxCommand*{int_if_odd_p:n} \Arg{integer expression}
%     \docAuxCommand*{int_if_odd:nTF} \Arg{integer expression}
%     ~~\Arg{true code} \Arg{false code}
%   \end{syntax}
%   This function first evaluates the \meta{integer expression}
%   as described for \docAuxCommand*{int_eval:n}. It then evaluates if this
%   is odd or even, as appropriate.
% \end{function}
 \subsection{Integer expression loops}
 
 Integer expression loops, bring \latex nearer to the functionality of other computer languages. In Example~\ref{ex:dowhile} a \emph{do}\ldots \emph{while} loop is constructed to print all even numbers from |0..16|. Many variations to looping structures are also provided and these are discussed after the example.
 
 \begin{texexample}{Integer Expression loops}{ex:dowhile}
 \ExplSyntaxOn
 \int_new:N \l_tempa_int
 \int_zero:N \l_tempa_int 
 \int_do_while:nn {\l_tempa_int <= 10 + 6 } {
     \int_use:N \l_tempa_int,~ 
     \int_add:Nn \l_tempa_int {2}
 }
 \ExplSyntaxOff
 \end{texexample}
 
 
\begin{texexample}{Step function} {}
\ExplSyntaxOn
\cs_set:Npn \my_func:n #1 {test #1}
\int_step_function:nnnN {1} {1} {5} {\my_func:n}
\ExplSyntaxOff
\end{texexample}
 
\section{Summing up} 

This has been a long chapter, and we both deserve some coffee and a break. We have discussed the creation of integer expressions, their use as counters and typesetting commands for counters. We have also examined in depth conditionals associated with integers and also some looping structures that are very robust. I have spent more time than I expected on this module, as it wraps up a lot of the concepts we have been discussing in other chapters and I thought a thorough review and some longer examples would be beneficial. 

By now, if you have been running the examples on your own, you should be more or less start thinking in
\latex3 speak. It takes a while for the syntax and the concepts to sink in. From my own experience, you need to spend at least 2-3 weeks just programming in the |expl| language and you should avoid the temptation to use \latexe macros or \tex primitives. Easier said that done.
 

 
	\chapter{The xtemplate package of LaTeX3 and how to use it effectively}

Back in 1999 Frank Mittelbach together with David Carlisle and Chris Rowley published a paper in TUGboat describing their ideas of  \enquote{New Interfaces for \latex Class Design.} 
 
 \begin{latexquote}
 Traditional \latex class files typically implement one
fixed design via ad hoc, and often low-level, \latex
code. This style of implementation makes it much
harder than is either desirable or necessary to produce
classes that implement a specific visual design.
Moreover, the construction of such classes typically
involves a lot of work that is essentially programming
and thus does not live easily with the declarative
kind of design specification for a document (or
range of documents) that would be produced by a
professional typographic designer.
\end{latexquote}

The \emph{declarative kind} of design specification for a document, mentioned by the authors has been the holy grail of \latex for sometime. With the proliferation of key value packages it came closer to fruition and my own work in the |phd| package had this goal as one of its primary objectives. The \pkgname{xtemplate} is at a much lower level than the phd package and I have struggled in my head as to how to integrate the two, so far unsuccessfully. There are very few articles on |xtemplate| but a good introductory one is \emph{Some notes on templates} by  Lars Hellström’s and which was published in TUGboat. The \pkgname{xgalley} still under develpment makes use of templates extensively and is worth to have a good look at the code.

\section{Objects, templates and instances}

\subsection{Object types}

An \emph{object type} sometimes termed \enquote{object} is an abstract idea of a document element that has a fixed number of arguments corresponding to the information from the document author that it is representing.  A sectioning object, for example, might take three inputs: \enquote{title}, \enquote{short title}, and \enquote{label}.

\begin{docCommand} {DeclareObjectType} { \meta{object type} \meta{no of args}}
This function defines an \meta{object type} taking \meta{number of arguments}, where the \meta{object type} is an abstraction as discussed above. For example:
   \begin{verbatim}
     \DeclareObjectType{chapter}{3}
   \end{verbatim}
This would create an object type \enquote{sectioning}, where each use of that object type will need three arguments.   
\end{docCommand}

The object type doesn’t do much when it is declared. It just records the name and the number of arguments in a property store, as can be seen in the code below, extracted from the |xtemplate| package:

\begin{teXXX}
\cs_new_protected:Npn \@@_declare_object_type:nn #1#2
  {
    \int_set:Nn \l_@@_tmp_int {#2}
    \bool_if:nTF
      {
        \int_compare_p:nNn {#2} > \c_nine ||
        \int_compare_p:nNn {#2} < \c_zero
      }
      {
        \msg_error:nnxx { xtemplate } { bad-number-of-arguments }
          {#1} { \exp_not:V \l_@@_tmp_int }
      }
      {
        \msg_info:nnxx { xtemplate } { declare-object-type }
          {#1} {#2}
        \prop_gput:NnV \g_@@_object_type_prop {#1}
          \l_@@_tmp_int
      }
  }
\end{teXXX}

    
\subsection{Templates}

Once an object is created a \emph{template} can be used to generalize a design solution for representing the information of a specific object type. A template has a name and a parent object. There are two important parts to a template:

\begin{enumerate}
\item The parameters it takes to vary the design it is producing.
\item The implementation of the design.
\end{enumerate}

The template definition is split into two parts using \cs{DeclareTemplateInterface} and \cs{DeclareTemplateCode}.
We will first examine \docAuxCommand*{DeclareTemplateInterface}.

\begin{docCommand} {DeclareTemplateInterface} {\meta{object} \marg{key value list}}
The key value list is of the form:
\begin{verbatim}

    key1 : key type1,
    key2 : key type2,
    key3 : key type3  = default3,
    key4 : key type4  = default4,
\end{verbatim}

An important item to note is that spaces in key names are ignored so writing |my key| and |mykey| is one and the same. 

Essentially the |DeclareTemplateInterface| is a command that initializes the list of key values applicable to the object type. The key list must be the same as declared for |object|. I didn’t know the \latex guys were fans of Java. The code pattern here is very similar. You declare an object and then its interface. Once this is done then the code can be developed. The key types available are shown in Table~\ref{tab:key-types}, which has been extracted from the documentation.

\end{docCommand}

\begin{docKey}{boolean} { boolean type for template interface}{\meta{true or false}}
a true or false value
\end{docKey}

   \begin{table}
     \centering
     \begin{tabular}{>{\ttfamily}ll}
       \toprule
       \multicolumn{1}{l}{Key-type} & Description of input \\
       \midrule
       boolean    & \texttt{true} or \texttt{false}            \\
       choice\marg{choices}
         & A list of pre-defined \meta{choices} \\
       code
         & Generalised key type: use |#1| as the input to the key \\
       commalist  & A comma-separated list                        \\
       function\marg{$N$}
         & A function definition with $N$ arguments
          ($N$ from $0$ to $9$) \\
       instance\marg{name}
                      & An instance of type \meta{name} \\
       integer    & An integer or integer expression            \\
       length     & A fixed length                              \\
       muskip    & A math length with shrink and stretch components \\
       real         & A real (floating point) value               \\
       skip         & A length with shrink and stretch components \\
       tokenlist  & A token list: any text or commands          \\
       \bottomrule
     \end{tabular}
     \caption{Key-types for defining template interfaces with
       \cs{DeclareTemplateInterface}.}
     \label{tab:key-types}
   \end{table}
   
\begin{texexample}{xtemplate short example}{}
\DeclareObjectType{obj}{0}
\DeclareTemplateInterface{obj}{tmpt1}{0}
{
  section-name: tokenlist = section,
  section-numbering: tokenlist =Roman,
  section-color: tokenlist = blue,
}
\end{texexample}

The |\DeclareTemplateInterface| part of the code is just a macro, whose fourth argument is written in a funny way.
We could have just written it as:

\begin{teXXX}
\DeclareTemplateInterface{obj}{tmp1}{0}{section-numbering:tokenlist=arabic, ... }
\end{teXXX}

A confusing aspect of the |templates| package is how the code part is defined. Here for each
key declared in the \docAuxCommand{DeclareTemplateInterface} you will need to allocate it an appropriate
macro. This works like in normal \latex keys. 

\begin{texexample}{The template code}{}
\ExplSyntaxOn
\DeclareTemplateCode{obj}{tmpt1}{0}
{
  section-name         = \sectionname,
  section-numbering = \numberingtype, 
  section-color = \colorname,
  }
{
% the implementation part
\AssignTemplateKeys
  \sectionname\ ~ 
  {\cs:w\numberingtype\cs_end: {section}\scan_stop:}\\
}

\DeclareInstance {obj}{inst} {tmpt1}{section-numbering = roman}
\UseInstance{obj}{inst}
\DeclareInstance {obj}{inst2}{tmpt1}{section-name=SECTION,
                                                      section-numbering = arabic}
\UseInstance{obj}{inst2}
\ExplSyntaxOff
\end{texexample}

The implementation part is the part that starts with |\AssignTemplateKeys|. Here we can use the values stored in the key functions to do something useful. Again here, remember, we are using macros and |\AssignTemplateKeys| is a macro with five arguments. This is defined by the package as:

 \begin{verbatim}
   \@@_declare_template_code:nnnnn {#1} {#2} {#3} {#4} {#5}
\end{verbatim}

Looking back at our simple example, the formatting of the section number in |arabic| or |roman| did not make any particular checks for validity. This would have been better programmed as a |choice| key with all the choice words allowed hardcoded in the implementation part. 

\begin{teXXX}
 section-numbering  : choice { arabic, Roman, roman, words, Words, alph, Alph } = arabic
\end{teXXX}                                 

The |choice| key type implements multiple choice input. At the interface level only the list of valid choice is needed:

\begin{teXXX}
\DeclareTemplateInterface{ foo }{ bar }{ 0 }
    { key-name : choice { A, B, C } }
\end{teXXX}

Note that the choices are given in a comma delimited list (which must therefore be wrapped in braces). A default value can also be given:


\begin{teXXX}
\DeclareTemplateInterface{ foo }{ bar }{ 0 }
    { key-name : choice { A, B, C } = A }
\end{teXXX}

\begin{teXXX}
 section-numbering      =
      {
        roman =
          \cs_set_nopar:Npn \numberingtype:
            {
              ... code
            },
        roman  =
          \cs_set_nopar:Npn \numberingtype:
            {
              ... code 
            }
      },
\end{teXXX}

\begin{texexample}{The template code}{ex:unknownkey}
\ExplSyntaxOn
\DeclareTemplateInterface{obj}{section}{0}
{
  section-name: tokenlist = section,
  section-numbering: choice  {arabic, Roman, roman}=roman,
  section-color: tokenlist = blue,
}
\DeclareTemplateCode{obj}{section}{0}
{
  section-name         = \sectionname,
  section-numbering = 
     {
       roman     =   \cs_set_nopar:Npn  \numberingtypei: { \roman{section} \scan_stop: },
       Roman     =  \cs_set_nopar:Npn   \numberingtypei: { \Roman{section} \scan_stop: },
       arabic      =   \cs_set_nopar:Npn  \numberingtypei: { \arabic{section} \scan_stop: },
       unknown =   \cs_set_nopar:Npn  \numberingtypei: { ERROR~unknown~key }
     },  
  section-color = \colorname,
  }
{
% the implementation part

\AssignTemplateKeys
  \sectionname\ ~ 
  \numberingtypei: \par
}

\DeclareInstance {obj}{inst} {section}{section-numbering = roman}
\UseInstance{obj}{inst}
\DeclareInstance {obj}{inst2}{section}
    {
        section-name=SECTION,
        section-numbering = arabic
     }
     
\DeclareInstance {obj}{inst3}{section}
    {
        section-name=SECTION,
        section-numbering = Arabic
     }  
                                                          
\UseInstance{obj}{inst2}
\UseInstance{obj}{inst3}
\ExplSyntaxOff
\meaning\numberingtypei
\end{texexample} 

In Example~\ref{ex:unknownkey} we have introduced the |choice| type key. This also takes an option
\option{unknown}. If a value is given that has not been previously been defined, then it essentially acts as
an |else| branch to the code and executes the definition given, in our example just typesets an 
error message. The code in the example at this stage is very simplistic and it has not been abstracted properly. The example is simply here to demonstrate the various types of keys available. The |length| and the |skip| keys 
accept dimensions or skips and are simply coded. The |function| type of key can be very useful in many situations. 

In the next example we will add some skips before and after the section, as well introduce a boolean to choose betwen a block heading or an inline heading. 


\begin{texexample}{The template code}{ex:unknownkey}
\ExplSyntaxOn
\DeclareTemplateInterface{obj}{headings}{0}
{
  name: tokenlist = section,
  numbering: choice  {arabic, Roman, roman,none} = roman,
  color: tokenlist = blue,
  display: boolean = true,
  aboveskip: skip=10pt,
  belowskip: skip=10pt,
  }
 
\bool_new:N \l_display_bool

\DeclareTemplateCode{obj}{headings}{0}
{
  name         = \sectionname,
  numbering = 
     {
       roman     =  \cs_set_nopar:Npn \numberingtypei: {\roman{section}},
       Roman     = \cs_set_nopar:Npn \numberingtypei: {\Roman{section} },
       arabic      =  \cs_set_nopar:Npn \numberingtypei: {\arabic{section}},
       none       =  \cs_set_nopar:Npn \numberingtypei: {},
       unknown =  \cs_set_nopar:Npn\numberingtypei: {ERROR~unknown~key }
     },  
  color = \colorname,
  display = \l_display_bool,
  aboveskip = \l_tmpa_skip,
  belowskip = \l_tmpb_skip,
 }
 {
% the implementation part
  \AssignTemplateKeys
  \par\skip_vertical:N  \l_tmpa_skip
  \sectionname\ ~ 
  \numberingtypei: \par
}

\DeclareInstance {obj}{part} {headings}
  { name      = PART,
    numbering = Roman
  }
  
\DeclareInstance {obj}{section}{headings}
  { name      = SECTION,
    numbering = arabic
  }
  
\DeclareInstance {obj}{chapter}{headings}
  {
    name      = CHAPTER,
    numbering = Roman, 
    aboveskip = 5pt
  }    
                                                                                                                   
\UseInstance{obj}{part}
\UseInstance{obj}{section}
\UseInstance{obj}{chapter}

\ExplSyntaxOff

\end{texexample}   

Although the |xtemplate| manual recommends that booleans should be preferred over 
|choice| keys, but from a user interface point of view |choice| keys are more powerful. One can define
key variations such as (true, false, on, off, none) and other similar values. 

Another few notes for readers coming from \latexe. The  |\skip_vertical:N| is the
\tex |\vskip|.  There are also some questions arising from the approach, which can affect the
coding. The format and the flexibility of the final settings offered for the user. From a programmer’s
perspective the view is different.  We could view the three basic elements of a heading at a more
elementary level, consisting of a number, a label and a title. Consider the |HTML| element |<span>|,
how can we make an equivalent in \latex3? There are many approaches one could think of, but this time
having covered the basics of how to program templates, we will start from the Designer Level. The Designer
wishes to define commands that are normally used inline and are used for different type of purposes. Such functions can typeset words that are emphasized, others that represent computer code and are typeset verbatim, acronyms and abbreviations. These also can automatically add themselves to an index etc.

Of course we don’t want to offer the user a command called |\span| where he needs to type |\span[emph]|. What we want to offer the user is a series of commands. However at the Design Level, these can be created by means of templates.

\begin{texexample}{A template for spans}{ex:span}
\ExplSyntaxOn
\DeclareObjectType{inlineobj}{1}
\DeclareTemplateInterface{inlineobj}{span}{1}
{
  font-face: tokenlist,
  font-shape: choice {italic, slanted, normal},
  font-weight: choice {bold, normal},
  font-color: tokenlist,
  quote: function 1,
}
\cs_set_nopar:Npn \quote_format:n#1 {\enquote{#1}}
\cs_set_nopar:Npn \quote_format_none:n#1 {#1}

\DeclareTemplateCode{inlineobj}{span}{1}
{
  font-face         =  \l_font_tl,
  font-shape = {
     italic     = \cs_set_nopar:Nn \afontshape: {\itshape},
     slanted = \cs_set_nopar:Nn \afontshape: {\itshape},
     normal = \cs_set_nopar:Nn \afontshape: {\upshape}
  },
  font-weight = {
     bold    = \cs_set_nopar:Nn \afontseries: {\bfseries},
     normal =\cs_set_nopar:Nn \afontseries: {\mdseries}
   },
  font-color = \l_tmpa_tl,  
  quote = \quote_format:n,
}
{
% the implementation part
  \AssignTemplateKeys
  \group_begin:
  \color\l_tmpa_tl
   \cs:w \l_font_tl \cs_end: 
   \afontshape:
   \afontseries: 
       \quote_format:n{\detokenize{#1}} 
   \group_end:
 }
 
\ExplSyntaxOff

\DeclareInstance {inlineobj}{docFunction}{span}
    {
        font-face=arial,
        font-shape=normal, 
        font-weight=bold,
        font-color=green!40!black
     }

\DeclareDocumentCommand\docFunction{ m }{
   \IfInstanceExistTF {inlineobj}{docFunction} 
     {\UseInstance{inlineobj}{docFunction}{#1}}
     {ERROR                                                   }
}

\DeclareInstance {inlineobj}{tn}{span}
    {
        font-face=ttfamily,
        font-shape=normal, 
        font-weight=normal,
        font-color=green!40!black
     }

\DeclareDocumentCommand\tn{ m }{
   \IfInstanceExistTF {inlineobj}{tn} 
     {\UseInstance{inlineobj}{tn}{#1}}
     {ERROR                                                   }
} 
   
The function \docFunction {get_string ( )} is used throughout to get a string in LuaTeX, where macros in text paragraphs are shown as \docFunction\mymacro in green.
\end{texexample}
\ExplSyntaxOn
\DeclareObjectType{inlineobj}{1}
\DeclareTemplateInterface{inlineobj}{span}{1}
{
  font-face: tokenlist,
  font-shape: choice {italic, slanted, normal},
  font-weight: choice {bold, normal},
  font-color: tokenlist,
  quote: function 1,
}
\cs_set_nopar:Npn \quote_format:n#1 {\enquote{#1}}

\DeclareTemplateCode{inlineobj}{span}{1}
{
  font-face         =  \l_font_tl,
  font-shape = {
     italic     = \cs_set_nopar:Nn \afontshape: {\itshape},
     slanted = \cs_set_nopar:Nn \afontshape: {\itshape},
     normal = \cs_set_nopar:Nn \afontshape: {\upshape}
  },
  font-weight = {
     bold    = \cs_set_nopar:Nn \afontseries: {\bfseries},
     normal =\cs_set_nopar:Nn \afontseries: {\mdseries}
   },
  font-color = \l_tmpa_tl,  
  quote = \quote_format:n,
}
{
% the implementation part
  \AssignTemplateKeys
  \group_begin:
  \color\l_tmpa_tl
   \cs:w \l_font_tl \cs_end: 
   \afontshape:
   \afontseries: 
       \quote_format:n{\detokenize{#1}} 
   \group_end:
 }
 
\ExplSyntaxOff

\DeclareInstance {inlineobj}{docFunction}{span}
    {
        font-face=arial,
        font-shape=normal, 
        font-weight=bold,
        font-color=green!40!black
     }

\DeclareDocumentCommand\docFunction{ m }{
   \IfInstanceExistTF {inlineobj}{docFunction} 
     {\UseInstance{inlineobj}{docFunction}{#1}}
     {ERROR}
}

\DeclareInstance {inlineobj}{tn}{span}
    {
        font-face=ttfamily,
        font-shape=normal, 
        font-weight=normal,
        font-color=green!40!black
     }

\DeclareDocumentCommand\tn{ m }{
   \IfInstanceExistTF {inlineobj}{tn} 
     {\UseInstance{inlineobj}{tn}{#1}}
     {ERROR}
} 
   
With the last example I have introduced also the conditional \docAuxCommand*{IfInstanceTF} that provides a test if the template exist. In our case typesets |ERROR| if the instance does not exist.

\begin{docCommand}{IfInstanceExistTF}{\marg{object type} \marg{instance} \marg{true code} \marg{false code}}
Tests if the named \meta{instance} of an \meta{object type} exists, and then inserts the appropriate code into the input stream. 
\end{docCommand}


\section{Summary}

\latex3’s \pkgname{xtemplate} offers a flexible and robust way to enable  declarative
setting of typographical parameters for a document. For the package writer it has one major advantage. It can be used to expose an API through which users communicate with the package's important commands. I would go as far as to say that packages should only expose an API and no settings should occur during loading. This can reduce both errors during package loading with different key values, as well as perhaps stop the race at the |AtBeginDocument|. 

If you want to study a longer non-trivial example you can have a look at the \pkgname{xfrac} package. In this package Will Robertson used |xtemplate| extensively. He also used some of the more esoteric commands of the package and is worth studying the code, before you start using |xtemplate| in your package.

All the functionality made available by the package can easily be provided by |pgfkeys| and the creation of some custom commands. This will remain as a competitor to the package until some of the limitations of |xparse| are addressed. The main limitation currently from my point of view is the addition of custom types  in a similar fashion to pgfkeys \emph{handlers}, although the \tn{code} and the \tn{function} types can be used in this respect.




	\chapter{Sequence lists}

\epigraph{``Where did we get that (equation) from? Nowhere. It is not possible to derive it from anything you know. It came out of the mind of Schrödinger.''}{---Richard Feynman}

One very useful data type, which is incorporated in \latex3 is the ``sequence'' data type. This contains a list of entries which may contain any \meta{balanced text}. One of the most powerful features of lists is that t is possible to map functions to sequences such that the function is applied to every item in the sequence.

Sequences are also used to implement stack functions in \latex3. This is achieved using a number of dedicated stack functions.

\section{Creating sequences}

Like most of the modules new sequences are created using the prefix for the module and the word ``new''.

\begin{docCommand}{seq_new:N}{}
Creates a new \meta{sequence} or raises an error if the name is already taken. The declaration
is global. The \meta{sequence} will initially contain no items.
\end{docCommand}

First let us create and examine the meaning of a simple example of the use of sequences. 

\begin{texexample}{Creating sequences}{ex:seq}
\ExplSyntaxOn
\seq_new:N \g_scratch_seq 
\token_to_meaning:N \g_scratch_seq
\ExplSyntaxOff
\end{texexample}

Examining the meaning of the sequence we created with \refCom{seq_new:N} we observe that there is no magic involved, it is just another macro that holds two others. So let us add some material and see what happens next.

\begin{texexample}{Creating sequences}{ex:seq}
\ExplSyntaxOn
\gdef\tempa {AAA}
\gdef\tempb {BBB}

% Add some material left and right
\seq_gput_left:Nn \g_scratch_seq \tempa
\seq_gput_right:Nn \g_scratch_seq \tempb

% examine the meaning of the \scratch_seq:N
% and the marker at the beginning
\token_to_meaning:N \g_scratch_seq\\
\token_to_meaning:N \s__seq\\
\token_to_meaning:N \s__seq_item:n
\ExplSyntaxOff
\end{texexample}

We have used two more functions that by now you are familiar to put material both at the left and at the right of the sequence, and again examined its meaning. We also examined the meaning of |s__seq| which is the internal command at the start of the list. The concept is very similar to |\@elt| lists

\begin{texexample}{Creating sequences}{ex:seq}
\ExplSyntaxOn
% examine the meaning of the \g_scratch_seq:N
% and the marker at the beginning again
\token_to_meaning:N \g_scratch_seq\\
\token_to_meaning:N \s__seq\\
\token_to_meaning:N \s__seq_item:n\\

% pop the left of the sequence
% store in \l_tmpa
\seq_pop_left:NN \g_scratch_seq \l_tmpa_tl

% typeset contents of left cell
\tl_use:N \l_tmpa_tl

\ExplSyntaxOff
\end{texexample}



\begin{texexample}{Creating sequences}{ex:seq}
\ExplSyntaxOn
\def\urlctan   {\url{\http:ctan.org}}
\def\urlgithub {\url{http:github.org}}

% clearing the sequence
\seq_clear:N \g_scratch_seq 

\seq_gput_left:Nn \g_scratch_seq \urlctan
\seq_gput_left:Nn \g_scratch_seq \urlgithub
% typeset contents of left cell
% pop the left of the sequence
% store in \l_tmpa
\seq_gpop_left:NN \g_scratch_seq \l_tmpa_tl

% typeset contents of left cell
\tl_use:N \l_tmpa_tl

\ExplSyntaxOff
\end{texexample}

\begin{texexample}{An equation database}{ex:seq}
 % #1 name
 % #2 equation
 
\ExplSyntaxOn   
\cs_gset:Npn \addEquation #1#2
  {
    \cs_set:cpn {-#1} {{\bfseries#1}\begin{gather}#2\end{gather}}
    \seq_gput_right:Nn \g_scratch_seq {#1}
  }

\cs_gset:Npn \typesetEquations 
  {
    \seq_map_inline:Nn \g_scratch_seq 
      {
        \cs:w-##1\cs_end:
      }
  }
  
  
% clearing the sequence
\seq_clear:N \g_scratch_seq 

\ExplSyntaxOff

\addEquation {quadratic} 
  {
    ax^2 + bx + c =0
  }
\addEquation {linear}    
  {
    x = \frac{b}{a}
  }
\addEquation {cubic}    
  {
    x^3 + 2x^2 + 10x = 20
  }
    
\typesetEquations

\end{texexample}

By the way Leonardo de Pisa, also known as Fibonacci (1170–1250), was able to find the positive solution to the cubic equation \( x^3 + 2x^2 + 10x = 20\), using the Babylonian numerals. He gave the result as \(1,22,7,42,33,4,40\) (equivalent to \(1 + 22/60 + 7/602 + 42/603 + 33/604 + 4/605 + 40/606)\), which differs from the correct value by only about three trillionths.

But let us fill our little database with the quartic, quintic, sextic and septic functions,
so we can have a few more data in our sequence. Also I suggest you try and run some of the examples on your own to get used to the language, solving syntax errors for typos and the like.

%\tcbset{texexmp/.style={ 
%        colback = white,% background
%        colframe=white, 
%        %bottombox=ignored,   
%        listing options={
%          backgroundcolor=\color{white},
%          keywordstyle=\color{black},
%          breaklines=true,
%          breakatwhitespace=true,
%          commentstyle=\color{thelightgray},
%          emph={addEquation, typesetEquations},
%          emphstyle=\color{thegreen},
%         },
%        }}%

\emphasize{addEquation,typesetEquations}
\begin{texexample}{An equation database}{ex:seq}
% add equations to db
\addEquation {quartic} 
  {
    f(x)=ax^4+bx^3+cx^2+dx+e
  }
  
\addEquation {quintic}    
  {
    g(x)=ax^5+bx^4+cx^3+dx^2+ex+f
  }
  
\addEquation {sextic}    
  {
    ax^6+bx^5+cx^4+dx^3+ex^2+fx+g=0
  }

\addEquation {septic}    
  {
    ax^7+bx^6+cx^5+dx^4+ex^3+fx^2+gx+h=0
  } 
  
\addEquation {BBGKY}
  {\scriptstyle  
   \frac{\partial f_s}{\partial t} + \sum_{i=1}^s \dot{\mathbf{q}}_i \frac{\partial f_s}{\partial \mathbf{q}_i} + \sum_{i=1}^s \left( - \frac{\partial \Phi_i^{ext}}{\partial \mathbf{q}_i} - \sum_{j=1}^s \frac{\partial \Phi_{ij}}{\partial \mathbf{q}_i} \right) \frac{\partial f_s}{\partial \mathbf{p}_i} = (N-s) \sum_{i=1}^s \frac{\partial}{\partial \mathbf{p}_i} \int \frac{\partial \Phi_{is+1}}{\partial \mathbf{q}_i}\cdot f_{s+1} \,d\mathbf{q}_{s+1} d\mathbf{p}_{s+1}.
  }
  
 \addEquation {Borda-Carnot}
  {
    \Delta E\, =\, \frac12\, \rho\, \left( v_3\, -\, v_2 \right)^2\,
           =\, \frac12\, \rho\, \left( \frac{1}{\mu}\, -\, 1 \right)^2\, v_2^2\,
           =\, \frac12\, \rho\, \left( \frac{1}{\mu}\, -\, 1 \right)^2\, \left( \frac{A_1}{A_2} \right)^2\, v_1^2.} %(*@\label{borda}@*)

% typeset all equations in db                      
\typesetEquations 

\end{texexample}

If you observe the last example we have hit a small problem, we had to reduce the size of the display
equation to fit it in. We would have been better to have displayed the equation in a \docAuxEnvironment{multline}
or a \docAuxEnv{brqew environment}, as shown below. 



\begin{multline}
\frac{\partial f_s}{\partial t} + \sum_{i=1}^s \dot{\mathbf{q}}_i \frac{\partial f_s}{\partial \mathbf{q}_i} + \sum_{i=1}^s \left( - \frac{\partial \Phi_i^{ext}}{\partial \mathbf{q}_i} - \sum_{j=1}^s \frac{\partial \Phi_{ij}}{\partial \mathbf{q}_i} \right) \frac{\partial f_s}{\partial \mathbf{p}_i} =\\
 (N-s) \sum_{i=1}^s \frac{\partial}{\partial \mathbf{p}_i} \int \frac{\partial \Phi_{is+1}}{\partial \mathbf{q}_i}\cdot f_{s+1} \,d\mathbf{q}_{s+1} d\mathbf{p}_{s+1}.
\end{multline}

Recall how we defined |\addEquation|,

\begin{teXXX}
\cs_gset:Npn \addEquation #1#2{
  \expandafter\gdef\csname-#1\endcsname 
    {
      {\bfseries#1}\begin{gather}#2\end{gather}
    }
  \seq_gput_right:Nn \g_scratch_seq {#1}
 }
\end{teXXX}

We can change the function to accept an optional argument and a starred or unstarred version to allow the user to add a field to the input that can determine the output. This is fairly easy with \pkgname{xparse}.


\begin{texexample}{An equation database}{ex:seq}
\addEquation {quartic} 
  {
    f(x)=ax^4+bx^3+cx^2+dx+e
  }
\addEquation {quintic}    
  {
    g(x)=ax^5+bx^4+cx^3+dx^2+ex+f
  }
\addEquation {sextic}    
  {
    ax^6+bx^5+cx^4+dx^3+ex^2+fx+g=0
  }

\addEquation {septic}    
  {
    ax^7+bx^6+cx^5+dx^4+ex^3+fx^2+gx+h=0
  } 
\addEquation {BBGKY}
  {\scriptstyle  
   \frac{\partial f_s}{\partial t} + \sum_{i=1}^s \dot{\mathbf{q}}_i \frac{\partial f_s}{\partial \mathbf{q}_i} + \sum_{i=1}^s \left( - \frac{\partial \Phi_i^{ext}}{\partial \mathbf{q}_i} - \sum_{j=1}^s \frac{\partial \Phi_{ij}}{\partial \mathbf{q}_i} \right) \frac{\partial f_s}{\partial \mathbf{p}_i} = (N-s) \sum_{i=1}^s \frac{\partial}{\partial \mathbf{p}_i} \int \frac{\partial \Phi_{is+1}}{\partial \mathbf{q}_i}\cdot f_{s+1} \,d\mathbf{q}_{s+1} d\mathbf{p}_{s+1}.
  }
 \addEquation {Borda-Carnot}
  {
    \Delta E\, =\, \frac12\, \rho\, \left( v_3\, -\, v_2 \right)^2\,
           =\, \frac12\, \rho\, \left( \frac{1}{\mu}\, -\, 1 \right)^2\, v_2^2\,
           =\, \frac12\, \rho\, \left( \frac{1}{\mu}\, -\, 1 \right)^2\, \left( \frac{A_1}{A_2} \right)^2\, v_1^2.}
\typesetEquations 
\end{texexample}





\begin{texexample}{Sequence}{ex:sequence}
\ExplSyntaxOn
\def\exception{}
\NewDocumentCommand\SplitDemo { +m m } 
  {
    \my_seq_split:nn { #1 }{#2}
  }

\tl_new:N \l_first_word_tl

\cs_new_protected:Npn \my_seq_split:nn #1 #2
  { 
    
    \seq_set_split:Nnn \l_tmpa_seq { #2 } { #1 }
    \seq_use:Nn   \l_tmpa_seq {\par}
    \seq_get_left:NN \l_tmpa_seq \l_first_word_tl
    %\textcolor{blue} { \tl_use:N \l_first_word_tl  }
  }
      
\ExplSyntaxOff

\SplitDemo { This is one sentence. 
             This is a second one. 
             This is the third sentence. }{ . }\par
\SplitDemo { The \exception{A.B.C.} corporation. Another sentence. }{ . }
\SplitDemo { The \exception{A.B.C.} corporation. Another sentence. }{~}
\end{texexample}


\paragraph{Capitalization and l3} \latex3 provides capitalization related functions in three modules: str, tl and token. Each module handles different cases. These are still listed under l3candidates and have not as yet been moved into the main modules. We normally have three cases for capitalization, changing a word to all capitals, lowercase or the first letter is capitalized and the rest are lowercased. In |l3| this is termed mixed case (my preference is naming it ucfirst), as mixed case should also include camel case, such as |CamelCase|. In example~\ref{ex:capitalization}, we use the three available functions to see how they are working.

\begin{texexample}{Change Lowercase Letters to Capitals}{ex:capitalization}
\ExplSyntaxOn
\tl_set:Nn \l_tmpa_tl { Hello~WORLD}
Input:~ \tl_use:N \l_tmpa_tl\par

Uppercase:~ \tl_upper_case:n { \l_tmpa_tl }\par

Mixedcase:~ \tl_mixed_case:n {\l_tmpa_tl}\par

Lowercase:~ \tl_lower_case:n {\l_tmpa_tl}\par
\ExplSyntaxOff
\end{texexample}




In the next example we will consider a difficult problem for machines, but an easy problem for humans, the capitalization rules for titles.words that are not normally capitalized in a title.


\begin{texexample}{Sequence}{ex:sequence}
\ExplSyntaxOn
\clist_gset:Nn \title_words_not_capitalized_en 
 {a, an, the, at, by, for, in, of, on, to, up, and, as, but, it, or, nor, do, for, this, be, A, An, The, At, By, For, In, Of, On, To, Up, And, As, But, It, Or, Nor, Do, For, This, Be,abaft, aboard, about, above, absent, across, afore, after, against, along, alongside, amid, amidst, among, amongst, an, anenst, apropos, apud, around, as, aside, astride, at, athwart, atop, barring, before, behind, below, beneath, beside, besides, between, beyond, but, by, circa, concerning, despite, down, during, except, excluding, failing, following, for, forenenst, from, given, in, including, inside, into, lest, like, mid, midst, minus, modulo, near, next, notwithstanding, of, off, on, onto, opposite, out, outside, over, pace, past,  per, plus, pro, qua, regarding, round, sans, save, since, than, through, throughout, till, times, to, toward, towards, under, underneath, unlike, until, unto, up, upon, Versus, versus, via, vice, with, within, without, worth
}

\clist_gset:Nn \abbreviations 
 {
  A.B.C.,iTunes
 }

\clist_gset:Nn \acronyms
  {
    NATO,UN,US
  }  						
    
\cs_new:Npn \ucfirst_aux:w #1#2 \q_stop { \tl_upper_case:n { #1 } #2 }

\cs_new:Npn \ucfirst #1 
	{
		\exp_after:wN \ucfirst_aux:w #1 \q_stop
	}

\cs_new:Npn \lowerfirst #1 
	{
		\tl_lower_case:n {#1}
 	}

\NewDocumentCommand\UppercaseTitle {s +m }
  {
	  \IfBooleanTF { #1 } { {\bfseries {#2} } }
      {     
       	\tex_hyphenpenalty:D = 10000
       	  % split on space
	       \seq_set_split:Nnn \g_tmpa_seq {~} {#2}
	       
	       \seq_use:Nn   \g_tmpa_seq {~}\\
	       
	       % pop the left into a temporary token list
	       % the first letter must always be capitalized 
	       \seq_pop_left:NN \g_tmpa_seq \l_tmpa_tl  
	       
	       % typeset the first letter
	       {\bfseries\ucfirst \l_tmpa_tl \space} 
	       
	       % map it in line
	       \seq_map_inline:Nn \g_tmpa_seq 
	       	{
	         	\clist_if_in:NnTF \title_words_not_capitalized_en { ##1 }
	           { {\bfseries \lowerfirst {##1}~}} { {\bfseries \ucfirst{##1}~ } }    
	            
	         }            
      }    
	} 
	
\ExplSyntaxOff    

 \UppercaseTitle {Top ten things To do in Paris}\\
 \UppercaseTitle {How to use LaTeX sequence lists effectively}\\
 \UppercaseTitle {Senate Votes to Confirm Elena Kagan For U.S. Supreme Court}\\
 \UppercaseTitle {what would be a ``correct'' capitalization for the title of this question?}\\
 \UppercaseTitle* {How about {$e=mc^2$}? }\\
\end{texexample}

The code we just wrote suffers from a lot of deficiencies.


\ExplSyntaxOn

\NewDocumentCommand\UppercaseTitle {s +m }
    {
      \IfBooleanTF { #1 } { {\bfseries {#2} } }
        {     \tex_hyphenpenalty:D = 10000
	        \seq_set_split:Nnn \g_tmpa_seq {~} {#2}
	        \seq_use:Nn   \g_tmpa_seq {~}\\
	        
	        \seq_pop_left:NN \g_tmpa_seq \l_tmpa_tl  
	       
	        {\bfseries\ucfirst \l_tmpa_tl \space} 
	        \seq_map_inline:Nn \g_tmpa_seq 
	           {
	              \clist_if_in:NnTF \title_words_not_capitalized_en { ##1 }
	              { {\bfseries \lowerfirst {##1}~}} { {\bfseries \ucfirst{##1}~ } }    
	            
	           }            
	       
       }    
    } 
\ExplSyntaxOff   
The rules for capitalization of titles varies from publication to publication and from department to department. A look at \href{http://arxiv.org/pdf/1505.04095v1.pdf}{arxiv} yielded a number of papers that do not follow the above rules. This will remain an unsolved problem, but at least we have moved forward. 

\begin{texexample}{Uppercase Title Issues}{}
\UppercaseTitle {Measuring Political Polarization: Twitter shows the two sides of Venezuela}\\
\UppercaseTitle {The Directed Dominating Set Problem: Generalized Leaf Removal and Belief Propagation}\\
\UppercaseTitle {Cities through the Prism of People's Spending Behavior}\\
\UppercaseTitle {On the p-th root of a p-adic number}\\
\UppercaseTitle {Planetary Formation Scenarios Revistied: Core-Accretion Versus Disk Instability}\\
\UppercaseTitle {de Haas-van Alphen effect versus Integer Quantum Hall effect}\\
\UppercaseTitle {A Simple Desultory Philippic (or How I Was Robert McNamara'd into Submission)}
\end{texexample}


The first title, shows a rule in many style manuals that words more than five characters should be capitalized, a rule broken by the third in the list above, although it is a preposition and many style books dictate that all prepositions be lowercase. It would make sense to add prepositions to our list. 

The last example is the Dutch and Afrikaans preposition \emph{de} meaning  \enquote{of} or \enquote{from}. This would make an exception on the first word of the sentence but not the last. The prefix von is not capitalised in German-speaking countries. The Duden dictionary recommends capitalizing the prefix von at the beginning of the sentence, but not in its abbreviated form, in order to avoid confusion with an abbreviated first name: \enquote{Von Humboldt kam später.} and \enquote{v. Humboldt kam später.} (Von Humboldt came later.) The Swiss Neue Zürcher Zeitung, however, recommends omitting the von completely at the beginning of the sentence: \enquote{Humboldt kam später.}


\begin{description}
\item [First and last words] These are always capitalized. There is a general agreement for this one by all guides and editors, however there are exceptions.

\item [Prepositions] Do not capitalize English prepositions in the body of the title, but capitalize them if they are the first word.

\item [Foreign language prepositions] These have their own rules and are discussed later.
\end{description}

Let us now try and improve our code. In Example~\ref{ex:sequence}, we ensured that the first word is always capitalized by popping out the first word from the list and capitalizing it by using:

\begin{teXXX}
\seq_pop_left:NN \g_tmpa_seq \l_tmpa_tl 
\end{teXXX}

The first suggestion that comes to mind is to change is to add a pop left operation and perhaps to add both  to an auxiliary function so that we can later on add exceptions for foreign language names, as desribed in the specification earlier.

\begin{texexample}{Renew \textbackslash UppercaseTitle}{}
\ExplSyntaxOn
%\cs_new:Npn \ucfirst_aux:w #1#2 \q_stop { \tl_upper_case:n { #1 } #2 }
%
%\cs_new:Npn \ucfirst #1 {
%     \exp_after:wN \ucfirst_aux:w #1 \q_stop
%}
%
%\cs_new:Npn \lowerfirst #1 {
%       \tl_lower_case:n {#1}
% }

% Main command
\RenewDocumentCommand\UppercaseTitle {s +m }
    {
      \IfBooleanTF { #1 } { {\bfseries {#2} } }
        {     \tex_hyphenpenalty:D = 10000
	        \seq_set_split:Nnn \g_tmpa_seq {~} {#2}

	        % type splitted sequence	        
	        \seq_use:Nn   \g_tmpa_seq {~}\\
	        
	        % 
	        \pop_first:N  \g_tmpa_seq
	        \seq_pop_right:NN  \g_tmpa_seq \l_tmpb_tl
	        %
	        \seq_map_inline:Nn \g_tmpa_seq 
	           {
	              \clist_if_in:NnTF \title_words_not_capitalized_en { ##1 }
	              { {\bfseries \lowerfirst {##1}~}} { {\bfseries \ucfirst{##1}~ } }    
	            
	           }            
	    {{ \bfseries \ucfirst{\l_tmpb_tl} }}
       }    
    } 

 % Function to pop the first item and decorate it    
\cs_new_nopar:Npn \pop_first:N #1 {
 	        \seq_pop_left:NN \g_tmpa_seq\l_tmpa_tl
	        {\bfseries\ucfirst \l_tmpa_tl \space} 
  }
  
 % Function to pop last word and decorate it 
\cs_new_nopar:Npn \pop_last:N #1 {
	        \seq_pop_right:NN \g_tmpa_seq \l_tmpa_tl
	        {\bfseries\ucfirst  \l_tmpa_tl} 
}        
        
\ExplSyntaxOff    
\UppercaseTitle {What to do with Versus~}\\
\UppercaseTitle {What to do with versus?~}\\
\UppercaseTitle {What to do with versus: Versus or versus~?~}\\
\end{texexample}

What just happened is that we have also created two new auxiliaries one to pop the first word and another to pop the last word. We are now closer to a final solution, but the decoration of the words, needs to be taken care of as well. These are always better to be functions of their own and we can do it quite easily. By decoration we mean adding fonts colors and the like. We do not consider capitalization as decoration. 


\begin{question}
\begin{tasks}(1)
\task Develop a function to detect if a token is composed of all capital letters.
\task Develop a boolean \cs{is_uppecase:nTF} that can return true or false if the text is uppercase.
\task Develop a boolean \cs{is_lowercase:nTF} that can return true or false if the text is all lowercase.
\task Develop a boolean \cs{is_mixedcase:nTF} that can return true or false if the text consists of upper first and then lowercase.
\end{tasks}
\end{question}

\begin{texexample}{Check if uppercase}{ex:seqif}
\ExplSyntaxOn
% create a new sequence
\seq_new:N \alphabet_en_seq

% split the sequence at empty spaces, use gset to pick in the next
% example as well
\seq_gset_split:Nnn \alphabet_en_seq {~} {A~B~C~D~E~F~G~H~I~J~K~L~M~N~O~P~Q~R~S~T~U~V~W~X~Y~Z} 

% create a predicate to check if the letter is uppercase
\prg_new_protected_conditional:Npnn \is_uppercase:n #1  {TF, T, F}{
   \seq_if_in:NnTF \alphabet_en_seq 
   {#1} 
   {\prg_return_true:}
   {\prg_return_false:}
}

\prg_generate_conditional_variant:Nnn \is_uppercase:n {o}{TF, T, F}


Z~is~uppercase~ \is_uppercase:nTF {Z} {\TRUE}{\FALSE}\\

v~is~not~uppercase \is_uppercase:nTF {v} {\TRUE}{\FALSE}
\ExplSyntaxOff 
\end{texexample}

Now that we have the basic definitions, what we have to do is iterate through all the letters of a single
token and carry out the tasks of determining if a string is fully capitalized. Fully capitalized strings,
will be assumed to be acronyms or specific words that the user wishes to have fully capitalized and left in the title as is. This will obviously fail if the title is fully capitalized and we wish to change it to one of the other cases, so first we need to determine that we have a title that is composed of mixed cases.

\begin{texexample}{}{}
\ExplSyntaxOn
% check we have not lost the sequence we defined earlier
%\seq_set_split:Nnn \alphabet_en_seq {~} {A~B~C~D~E~F~G~H~I~J~K~L~M~N~O~P~Q~R~S~T~U~V~W~X~Y~Z} 
\seq_use:Nn \alphabet_en_seq {,}
\ExplSyntaxOff
\end{texexample}

We assume that the title will be split into words and we need to determine if a word is fully
capitalized:

\begin{texexample}{Determine if a word consists of all capital letters}{ex:allcaps}
\ExplSyntaxOn
\tl_gset:Nn \l_my_tl {ALLCAPsa}

\tl_map_inline:Nn \l_my_tl
{
% Do something useful
  \is_uppercase:nTF {#1}{#1~\TRUE}{\FALSE \tl_map_break:}
}
\ExplSyntaxOff
\end{texexample}

We are now on thr right track. We map the word letter by letter and break out at the first occurence of
a lowercase letter. Notice the \cs{l_my_tl} stores |ALLCAPSsa| we broke out using |tl_map_break:| and never typeset the letter. All is left is to define a number of predicates using the macros we have just developed.

\begin{texexample}{Determine the case of a word}{ex:allcaps2}
% ... continued
\ExplSyntaxOn
% create a predicate to check if the letter is uppercase
\bool_new:N \word_allcaps_bool
\bool_new:N \word_mixed_bool
\bool_new:N \word_alllower_bool

\prg_new_protected_conditional:Npnn \if_is_word_uppercase:n #1  {TF, T, F}
  {  
    % Reset all the booleans
     \bool_gset_false:N \word_allcaps_bool
     \bool_gset_false:N \word_alllower_bool
     \bool_gset_false:N \word_mixed_bool
     
      \tl_map_inline:nn {#1}
      {
        \is_uppercase:nTF {##1}
           {\bool_gset_true:N \word_allcaps_bool}
           {
              \bool_gset_true:N \word_alllower_bool
           }
      }
      
    \bool_if:NTF\word_allcaps_bool
       { 
         % Check for mixed case
         \bool_if:NTF \word_alllower_bool
           {
             % If a word has activated all caps 
             % it can still conatin lowercase letters
             % in this case set the mixed case true
             \bool_gset_true:N \word_mixed_bool
             \prg_return_false:
           }
           {\prg_return_true:}
       }
       {
         \prg_return_false:
       }
   }


% Tests true
\if_is_word_uppercase:nTF {ALL}{ALL \TRUE}{\FALSE}\\

% Tests false
\if_is_word_uppercase:nTF {All}{\TRUE}{All \FALSE}\\

% Since the last test is a mixed case check that the boolean is set
\bool_if:NTF\word_mixed_bool {\TRUE}{All \FALSE}\\

% One more test
\if_is_word_uppercase:nTF {N-time}{ALL \TRUE}{\FALSE}\\
\bool_if:NTF\word_mixed_bool {\TRUE}{All \FALSE}

\ExplSyntaxOff
\end{texexample}

There are many more tests that one should include in a comprehensive package, with many more tests for edge cases. For example what to do after punctuation etc. One day this partial code will become a proper
package and integrated into the phd-lowerlevelheadings package.

\subsection{Final approach}

Given the rules above, some words can have three different ways of capitalization depending on their position in the sentence i.e, first, middle or end.

I posted some of the above code on |TEX.SX| and I had some amazing response from two of the developers of |expl3|. Under development there is a version of a |\title_case:n| command, which follows more or less the approach described in the example above.  I  have included the example in the chapter to demonstrate some of the aproaches to programming.

\robustify\url
\robustify\href
\robustify\textbf
\ExplSyntaxOn
\DeclareDocumentCommand \arxiv {g g}
{
  \IfNoValueTF {#1} {\href{http://arxiv.org}{{\color{blue}arxiv}}\xspace}
 {\href{http://arxiv.org/#1}{{\color{blue}#2}}\xspace}
}
\ExplSyntaxOff

\paragraph{The exceptions to the rules}
The article \arxiv{abs/1505.05148}{ALMA maps the Star-Forming Regions in a Dense Gas Disk at z\char`\~3 } has also a problematic title.  Here the first word is an acronym and is left as is,  and the last word is an abbreviation also left as is. Exclusion list, abbreviation lists perhaps need to be build over time similar to hyphenation lists.  Another paper
\arxiv{abs/1505.05156}{Statistics of Measuring Neutron Star Radii: The Bayesian vs. The Frequentist Approach} shows some of the problems with abbreviations, such as \enquote{vs.}, so filtering through a list of exclusions before capitalization is unavoidable. 
\acrodef{QGP}{quark-gluon plasma}

The title  ``\arxiv{abs/1505.04994}{Viscosities of Gluon Dominated QGP Model within Relativistic Non-Abelian Hydrodynamics}'', will only be capitalized properly, except for the  \ac{QGP} acronym, which must be present in an exclusion list.

\paragraph{Problems with NLP}
Some of the problems described in rationalizing an algorithmic approach fall in what is desribed as Natural Language Processing and such problems are not easy to solve. A solution that maybe can be satisfactory for
most cases, would involve at least a couple of passes and it will also assume that the author is somehow correct in typing the text, such a sleaving spaces after a stop. Using a second pass with regular expressions, it maybe possible to start filtering some of the abbreviations and acronyms, as well as do sentence detection.

Thanks to the \latex3 Team some problems taht would seem almost impossible to solve previously, at least now we can contemplate solution.

We will revisit the code in the chapter on Regular Expressions.

\begin{enumerate}
\item Write or ask for a specification. This can clarify requirements and avoid too many iterations of the code development. Have many examples of usage for testing.  Write tests for your code and always test against them. I understood the rules of title casing better from collecting titles from the \arxiv website.

\item Search for similar code and packages before you start developing.

\item Don't be afraid to ask the experts, most of the time they are more than willing to help.

\item Open source development is great. Consider contibuting to it. It is great for you and it is great for the community. The old concept of ``commons'' has more or less disappeared except in programming. Foster it and take care of it.
\end{enumerate}


\section{Summary}

In this chapter we reviewed the basics of the data type \enquote{Sequence lists} and have managed to produce some useful code in our final example. We have also reviewed some of the general concepts behind programming and have even managed to get two of the \latex3 developers to contribute code.

The code with some modifications is included in the \pkgname{phd} to provide title casing for headings and titles. The credit goes to Will Robertson and Joseph Wright. 

This chapter also brings us to the end of the list structures of |expl3|. Lua has its tables which are used to develop any data type and data structure required and similarly |expl3|'s lists can be used to develop and data structure you can imagine. One can think of link lists and tree data structures.

\endinput
A link list can easily be developed as it consists of elements that point to the next element only.  

\begin{verbatim}
\expandafter\def \csname link_item_1_next\endcsname {end list marker}
\end{verbatim}

At creation the link list item will expand to a marker. When the second item is added, the previous item will point to the second element and so on. The advantages of a link list is that if we want to delete an item or insert, we do not have to iterate through the whole list. Of course, if we had to delete it we could just simply mark it as undefined.

All the lists and parsing described in this book depend on one amazing fact, which is what \tex does when scanning the argument specification of a macro (between |\def\acommand | and either the opening bracket |{|. Leverage this fact as much as you can in your parsers. 

During mapping this could be detected and not used. \tex like any language has its own paradigms and we need not   follow other language patterns, but is good to know that we truly have a highly flexible and Turing-complete language available (even if it is a macro language). A macro language is still a language.



	\chapter{The LaTeX3 l3token package}
\label{ch:l3token}

The \tex concept of tokens is central to its operation. In earlier chapters we discussed extensively the use of category codes and other important aspects of \tex’s tokens. Rememeber a \tex token is either a single character or a control sequence such as a the control sequence |\test|.

\begin{texexample}{makeatletter}{}
\ExplSyntaxOn
\group_begin:
\char_set_catcode_letter:N @
\char_set_catcode_letter:N 1
\def\@store1a{AAAA}
\@store1a\\
\token_to_meaning:N @\\
\token_to_meaning:N 1\\
\char_set_catcode_other:N @
\char_set_catcode_other:N 1
\token_to_meaning:N @\\
\token_to_meaning:N 1\\
\group_end:
\ExplSyntaxOff
\end{texexample}

There are sixteen different commands to set the catcode to any of the predefined groups used by \tex. If you cannot remember the catcode number for a tab character, try and remember its command!

\begin{verbatim}
\char_set_catcode_escape:N 
\char_set_catcode_group_begin:N
\char_set_catcode_group_end:N
\char_set_catcode_math_toggle:N
\char_set_catcode_alignment:N
\char_set_catcode_end_line:N
\char_set_catcode_parameter:N
\char_set_catcode_math_superscript:N
\char_set_catcode_math_subscript:N
\char_set_catcode_ignore:N
\char_set_catcode_space:N
\char_set_catcode_letter:N
\char_set_catcode_other:N
\char_set_catcode_active:N
\char_set_catcode_comment:N
\char_set_catcode_invalid:N
\end{verbatim}

\section{Token predicate functions}

\begin{docCommand}{token_if_macro:NTF} { \meta{token} \marg{true code} \marg{false code}}
tests if the \meta{token} is a \tex macro.
\end{docCommand}

\begin{texexample}{Test if is a macro}{}
\ExplSyntaxOn
\csname sometest\endcsname
\expandafter\def\csname sometesti\endcsname{}
\token_if_macro:NTF \par { \PASS } { \FAIL }
\token_if_macro:NTF \minipage { \PASS } {\FAIL }
\token_if_macro:NTF \sometest { \PASS } {\FAIL }
\token_if_macro:NTF \sometesti { \PASS } {\FAIL }
\par
\meaning\sometest
\ExplSyntaxOff
\end{texexample}

 If it is a primitive we can find out, using yet another boolean construction \docAuxCommand*{token_if_primitive:NTF}  We can also check its meaning. It is interesting to note that \docAuxCommand*{par} is not a macro. Interestingly we can view what \tex does when we say |\csname somecs\endcsname|. It justs sets it equal to |\relax|. 
 
 Again this is important in parsing and in automating the generation of commands. For example  in the |phd| package, we allow for a key value to be entered either as a control sequence for example, |\Large| or simply as a |large|. A test could be provided before further processing such type of input.

\begin{texexample}{Test if is a macro}{ex:ifprimitive1}
\ExplSyntaxOn
\token_to_meaning:N \par\\
\token_to_meaning:N \toks
\token_if_primitive:NTF \par { \PASS } { \FAIL }\\
\ExplSyntaxOff
\end{texexample}

Example~\ref{ex:ifprimitive1} can be used to test if a primitive has been redefined (this can be important for your code and to restore its meaning if necessary or issue an error message.  Another test which is available is to check if a token is a macro. 

%% Dangerous??
\begin{texexample}{Test if is a macro}{ex:active}
\ExplSyntaxOn
\group_begin:
\token_if_cs:NTF \char_set_catcode_active:N  { \PASS } { \FAIL }
\group_end:
\ExplSyntaxOff
\end{texexample}

The next set of available commands are helper functions equivalent to the output of |\ifcat| 

\begin{docCommand} {token_if_group_begin:NTF} {\meta{token} \marg{true code} \marg{false code}}
Tests if \meta{token} has the category code of a begin group token (\{) when normal TEX
category codes are in force). Note that an explicit begin group token cannot be tested in
this way, as it is not a valid N-type argument. To test it you have to use |\c_group_begin_token|. This is mostly
used in conjuction with |futurelet| type constructions and or parsing.
\end{docCommand}


\begin{texexample} {Test if group begin} {ex:ifgroubbegin}
\ExplSyntaxOn
 \token_if_group_begin:NTF \c_group_begin_token { \PASS }{ \FAIL }
 \token_if_group_end:NTF    \c_group_end_token { \PASS }{ \FAIL }\par
 \the\catcode`{
\ExplSyntaxOff
\end{texexample}

Behind the scenes |expl3| uses the |\ifcat| primitive to test the token against the catcode values. Constructions for all categories are available and summarized in the test below rather than described.
\begin{texexample} {Test if group begin} {ex:ifgroubbegin}
\ExplSyntaxOn
 \token_if_group_begin:NTF \c_group_begin_token { \PASS }{ \FAIL }
 \token_if_group_end:NTF    \c_group_end_token { \PASS }{ \FAIL }\par
 \token_if_alignment:NTF     \c_alignment_token { \PASS }{ \FAIL }\par
 \token_if_parameter:NTF    \c_parameter_token { \PASS }{ \FAIL }\par
\ExplSyntaxOff
\end{texexample}


\section{LaTeX3 Futurelet type functions}

In Chapter Futurelet, we spend considerable effort to understand how \tex’s futurelet macro works. There is often a need to look ahead at the next token in the input stream while leaving
it in place. This is handled using the “peek” functions. The generic \docAuxCommand*{peek_after:Nw} is
provided along with a family of predefined tests for common cases. As peeking ahead does
not skip spaces the predefined tests include both a space-respecting and space-skipping
version.

\begin{texexample}{Peek ahead ignoring spaces} {}
\ExplSyntaxOn
\peek_catcode_remove_ignore_spaces:NTF =  
    { 
      \PASS  
      \token_if_letter:NTF
          {l_peek_token ~= ~\token_to_meaning:N \l_peek_token \\  } 
          {   }
    } 
    { \FAIL }  
 = abcde \\
\ExplSyntaxOff
\end{texexample}

Most applications would require to recursively pick up tokens from the input stream and only terminated once a special token is found. This is the most powerful method to parse input strings and create really powerful functions. 

You will understand better if we hide the code in a function.

\begin{texexample}{Peek ahead ignoring spaces} {ex}
\ExplSyntaxOn
\cs_new:Npn \checkletter #1 {
\peek_catcode_remove_ignore_spaces:NTF #1  
    { 
      \PASS  
      \token_if_letter:NTF
          {l_peek_token ~= ~\token_to_meaning:N \l_peek_token \\  } 
          {   }
    } 
    { \FAIL } }

\checkletter {=} =abcde \par
\checkletter {A} Abcde \par
\ExplSyntaxOff
\end{texexample}

\begin{texexample}{Peek ahead ignoring spaces} {}
\ExplSyntaxOn
\cs_set:Npn \check_letter_and_removeall #1 {
\peek_catcode_remove_ignore_spaces:NTF #1  
    { 
      \PASS  
      \removeallaux:w  
    } 
   { \FAIL } 
 }

\cs_set:Npn \removeallaux:w #1; { removed~#1~ }

\check_letter_and_removeall {W}  W 12pt; \par
\ExplSyntaxOff
\end{texexample}


\begin{texexample}{ex:recursivefl}  { }                            
\lorem 
\ExplSyntaxOn
\def\recurse {
   \peek_catcode_remove_ignore_spaces:NTF ; 
   {TRUE} 
   {
     \recurse
   } abcdouiop;
}
\ExplSyntaxOff
\end{texexample}
















	\chapter{LaTeX3 Token Lists}

 \TeX{} works with tokens, and \LaTeX3 therefore provides a number of
 functions to deal with lists of tokens.  Token lists may be present
 directly in the argument to a function:
 \begin{verbatim}
   \foo:n { a collection of \tokens }
 \end{verbatim}
 or may be stored in a so-called \enquote{token list variable}, which
 have the suffix \texttt{tl}: a token list variable can also be used as
 the argument to a function, for example
 \begin{verbatim}
   \foo:N \l_some_tl
 \end{verbatim}
 In both cases, functions are available to test an manipulate the lists
 of tokens, and these have the module prefix \texttt{tl}.
 
 
 In many cases, function which can be applied to token list variables
 are paired with similar functions for application to explicit lists
 of tokens: the two \enquote{views} of a token list are therefore collected
 together here.

 A token list (explicit, or stored in a variable) can be seen either
 as a list of \enquote{items},
 or a list of \enquote{tokens}. An item is whatever \docAuxCommand*{use:n} would
 grab as its argument: a single non-space token or a brace group,
 with optional leading explicit space characters (each item is thus
 itself a token list). A token is either a normal \texttt{N} argument,
 or \verb*| |, |{|, or |}| (assuming normal \TeX{} category codes).
 Thus for example
 \begin{verbatim}
   { Hello } ~ world
 \end{verbatim}
 contains six items (\texttt{Hello}, \texttt{w}, \texttt{o}, \texttt{r},
 \texttt{l} and \texttt{d}), but thirteen tokens (|{|, \texttt{H}, \texttt{e},
 \texttt{l}, \texttt{l}, \texttt{o}, |}|, \verb*| |, \texttt{w}, \texttt{o},
 \texttt{r}, \texttt{l} and \texttt{d}).
 Functions which act on items are often faster than their analogue acting
 directly on tokens.
%
% ^^A todo: perhaps move to another module, l3token or l3basics?
% \begin{texnote}
   When \TeX{} fetches an undelimited argument from the input stream,
   explicit character tokens with character code $32$ (space) and
   category code $10$ (space), which we here call \enquote{explicit
     space characters}, are ignored.  If the following token is an
   explicit character token with category code $1$ (begin-group) and an
   arbitrary character code, then \TeX{} scans ahead to obtain an equal
   number of explicit character tokens with category code $1$
   (begin-group) and $2$ (end-group), and the resulting list of tokens
   (with outer braces removed) becomes the argument.  Otherwise, a
   single token is taken as the argument for the macro: we call such
   single tokens \enquote{N-type}, as they are suitable to be used as
   an argument for a function with the signature~\texttt{:N}.

   When \TeX{} reads a character of category code $10$ for the first
   time, it is converted to an explicit space character, with character
   code $32$, regardless of the initial character code.
   \enquote{Funny} spaces with a different category code, can be
   produced using \docAuxCommand*{tl_to_lowercase:n} or \docAuxCommand*{tl\_to\_uppercase:n}.
   Explicit space characters are also produced as a result of
   \docAuxCommand*{token\_to\_str:N}, \docAuxCommand*{tl\_to\_str:n}, etc.
% \end{texnote}

 \section{Creating and initialising token list variables}

Before a token list can be used it needs to be created. It is important to recall that a token list is a \tex macro that holds tokens. This is achieved by using the e-\tex primitive \docAuxCommand*{unexpanded} inside a \tex |\edef| it is possible to store any tokens, including \#, in this way. The |\unexpanded| macro has been mapped to |\exp_not:n|.  This has made the token list registers (|\toks|) provided by \tex more or less redundant. Hence in |expl3| there are no |toks| registers.

 \begin{docCommand}{tl_new:N}{\meta{tl~var}}
    Creates a new \meta{tl~var} or raises an error if the
   name is already taken. The declaration is global. The
   \meta{tl~var} will initially be empty.
\end{docCommand}

In Example~\ref{ex:tl1}, line~\ref{lin:newlist} we will first create a new token list variable and then examine its meaning.
To examine its meaning, we use another |expl3| module function |\token_to_meaning:N|. This is the \tex primitive camouflaged in the new lingua. As we can see from the example a \meta{token list} is just a macro. 

\begin{texexample}{Create a token list variable}{ex:tl1}
\ExplSyntaxOn
% create the token list (*@\label{lin:newlist}@*)
\tl_new:N \g_mymodule_tl 

% examine its meaning
\token_to_meaning:N \g_mymodule_tl
\ExplSyntaxOff
\end{texexample}

If you have a basic knowledge of \tex programming, sometimes just peeking at the meaning of a function can give you a good understanding of what is happenning behind the scenes. In many of the example, I have included a line or two of code to examine the meaning of commands.


 \section{Adding data to token list variables}

Data can be added to the token list variable, as a once operation or added at left or right of the token list. The list can also be let to another one or emptied.

 \begin{docCommand}{tl_set:Nn}{\meta{tl~var} \marg{tokens}}
   Sets \meta{tl~var} to contain \meta{tokens},
   removing any previous content from the variable. The global |\tl_gset:Nn|, as well as different signature combinations are availabel such as \texttt{NV,Nv,No,Nf,Nx,cn,cV,cv,co,cf,cx}.
 \end{docCommand}

\begin{texexample}{Adding data}{ex:tl2}
\ExplSyntaxOn
% add something to the list
\tl_gput_left:Nn \g_mymodule_tl {This~is~something}
\token_to_meaning:N \tl_put_left:Nn
\ExplSyntaxOff
\end{texexample}

Let us put some more material and then view it again. This time we will add both to the left, as well as the right of the token list.

\begin{texexample}{Adding data}{ex:tl3}
\ExplSyntaxOn
% add something to the list
\tl_gput_left:Nn \g_mymodule_tl {LEFT~MATERIAL~}
\tl_gput_right:Nn \g_mymodule_tl {~RIGHT~MATERIAL}
\tl_gput_left:Nn \g_mymodule_tl {\bfseries}

% Use the token list
\g_mymodule_tl\\

% Another way to use it
\tl_use:N \g_mymodule_tl
\ExplSyntaxOff
\end{texexample}

The above are not the best of examples, but they illustrate the concepts well.

\begin{docCommand} {tl_put_left:Nn} {\meta{tl var} \marg{tokens}}
Prepends \meta{tokens} to the left side of the current content of \meta{tlvar}
\end{docCommand}

\begin{docCommand} {tl_put_right:Nn} {\meta{tl var} \marg{tokens}}
Appends \meta{tokens} to the right side of the current content of \meta{tlvar}
\end{docCommand}

\begin{texexample}{Append and prepend functions}{ex:prepend}
\ExplSyntaxOn
\tl_new:N  \phd_tempa_tl
\tl_new:N  \phd_tempb_tl
\tl_new:N  \phd_tempc_tl

\tl_put_left:Nn \phd_tempa_tl {First}
\tl_put_left:Nn \phd_tempa_tl {\bgroup\bfseries}
\tl_put_right:Nn \phd_tempa_tl {~Second\egroup}

\tl_set:Nn \phd_tempb_tl {\fbox{\tl_use:N \phd_tempa_tl }}

\tl_use:N \phd_tempb_tl

\tl_put_left:Nn \phd_tempb_tl { \fboxsep=3pt \fboxrule=0.1pt }

\tl_use:N \phd_tempb_tl

\phd_tempb_tl
\ExplSyntaxOff
\end{texexample}

The token list can be used simply by typing |\phd_tempb_tl| or by using the |\tl_use:N| function for example  |\tl_use:N \phd_tempb_tl|. The latter is preferred as it checks for naming errors. If for example the list name was mistyped it will give a |bad variable error|. It is also a good programming pattern to use, as it follows logically to \emph{construct} the list, \emph{manipulate} it and then \emph{use} it,  to typeset the contents. 



\subsection{Concatenation}

Two lists can be concatenated together by using a third list and adding the contents of the other two together.

\begin{texexample}{Concatenation and other functions}{ex:concat}
\ExplSyntaxOn
\tl_new:N  \tl_phd_temp_a
\tl_new:N  \tl_phd_temp_b
\tl_new:N  \tl_phd_temp_c
%%\tl_concat:NNN \tl_phd_temp_a \tl_phd_temp_b \tl_phd_temp_c
\ExplSyntaxOff
\end{texexample}


\begin{texexample}{Concatenation and other function}{ex:concat}
\ExplSyntaxOn
\cs_new:Nn\whatever:n{[{#1}]}

\cs_set:Nn\l_phd_test_helper: {
       \tl_set:Nn  \tl_phd_temp_a {Yiannis Mary John}
       \tl_put_left:Nn \tl_phd_temp_a {Dr.~}
       \tl_set:Nn  \tl_phd_temp_b {{Lazarides}{Lou}{Smith}}
       \tl_concat:NNN \tl_phd_temp_c \tl_phd_temp_a \tl_phd_temp_b
       \tl_map_inline:Nn\tl_phd_temp_c{\whatever:n{##1}}
}
\DeclareDocumentCommand\Test{ }
    {
      \l_phd_test_helper:
    }   
\ExplSyntaxOff 
\Test
\end{texexample}

The syntax of the concatenation function is shown below.

 \begin{docCommand}{ tl_concat:NNN}{ \meta{tl~var1} \meta{tl~var2} \meta{tl~var3}}
   Concatenates the content of \meta{tl~var2} and \meta{tl~var3}
   together and saves the result in \meta{tl~var1}. The \meta{tl~var2}
   will be placed at the left side of the new token list.
\end{docCommand}


 \begin{docCommand}{tl_const:Nn}{\meta{tl~var} \marg{token list}}
   Creates a new constant \meta{tl~var} or raises an error
   if the name is already taken. The value of the
   \meta{tl~var} will be set globally to the
   \meta{token list}.
 \end{docCommand}




\begin{texexample}{Concatenation and other function}{ex:concat}
\ExplSyntaxOn
\tl_concat:NNN \tl_phd_temp_a \tl_phd_temp_b \tl_phd_temp_c
\ExplSyntaxOff
\end{texexample}

\subsection{Search and replace}

It is an amazing feat that the \latexe Team has managed to provide replacement routines. These so far have not found meaningful application but I have an idea which I am going to share it with you in a moment.

\begin{texexample}{Replacement}{ex:replacement}
\ExplSyntaxOn
\tl_set:Nn \tempa {This~is~the~LaTeXe~logo.~The~LaTeXe~logo.}
\tl_replace_once:Nnn \tempa {LaTeXe} {\latexe}
\tl_use:N \tempa\par
\tl_replace_all:Nnn \tempa {LaTeXe} {\latexe}
\tl_use:N \tempa\par
\ExplSyntaxOff
\end{texexample}


\begin{texexample}{Replacement}{ex:replacement}
\long\def\putimage#1]]{%
   \bgroup
   \fboxsep=3pt
   \fboxrule=1pt
   \fbox{\includegraphics[width=3cm]{amato}}\hspace{2pt}%
   \egroup
}
\long\def\putsomecaption#1]]{
  \captionof{figure}{#1}
}

\ExplSyntaxOn
   \tl_set:Nn \tempai {
      \centering
      [[img amato]]
      [[cap This~is~the~first~caption]]
      [[img amato]]
      [[cap This~is~the~second~caption]]
      [[img amato.jpg]] 
      [[cap This~is~the~third~caption]]
   }
   \tl_replace_all:Nnn\tempai {[[img}{\putimage}
   \tl_replace_all:Nnn\tempai {[[cap}{\putsomecaption}
   \tl_use:N \tempai
\ExplSyntaxOff   
\end{texexample}

The code is still not mature and settled enough to provide an extensive search and replace, although a regex module has been coded by Bruno. However, simple templates can be developed such as the above. 

Of course in our first image example, we simplified the code in order to explain the concepts more clearly. In our second example we will add code so that the images are enclosed in a |minipage| and the captions set underneath the images. We will also center both the individual images as well as the three images overall.

\begin{texexample}{Replacement}{ex:replacement}
\long\def\putimage#1!!{%
    \includegraphics[width=\linewidth]{amato}%
 }
\long\def\putsomecaption#1!!{
  \captionof{figure}{#1}
  \par\endminipage\hfill
}


\ExplSyntaxOn
   \tl_set:Nn \tempai {
      \centering
      !!img amato!!
      !!cap This~is~the~first~caption!!
      
      !!img amato!!
      !!cap This~is~the~second~caption!!
      
      !!img amato.jpg!!
      !!cap This~is~the~third~caption!!
      
      !!img amato!!
      !!cap This~is~the~fourth~caption!!
      
      !!img amato!!
      !!cap This~is~the~fifth~caption!!
      
      !!img amato!!
      !!cap This~is~the~sixth~caption!!
 }
   \tl_replace_all:Nnn\tempai {!!img}{\minipage{3.6cm}\centering\putimage}
   \tl_replace_all:Nnn\tempai {!!cap}{\putsomecaption}
   \tl_use:N \tempai
\ExplSyntaxOff   
\end{texexample}

One might argue that by using |!!  !!| as a delimiter is not easier or clearer than say |\img |. In the final example I will propose another idea. But first let us create the environment. The best way to collect the body of the environment is to use the \pkgname{environ} package. 


When I posted the code at |TX.SX| egreg’s idea was to extend the code using a key value input and to be able to type:


\latex3 keys are discussed in the Chapter LaTeX3 key value and the example is reused but with key values. The second batch of images, shows some of the other challenges with the code. It will be best to have all the images measured and adjust the spacing in-between to be equal. 
\newpage


\ExplSyntaxOn
\NewEnviron{multiimages}[1][]
 {
  \keys_set:nn { yl/multiimages } { #1 }
  \tl_use:N \l_yl_multi_start_tl
  \dim_set:Nn \parindent { 0pt }
  \skip_set:Nn \leftskip { 0pt plus 1fil }
  \skip_set:Nn \rightskip { 0pt plus -1fil }
  \skip_set:Nn \lineskip { \l_yl_multi_skip_skip }
  \yl_multiimages:V \BODY
  \tl_use:N \l_yl_multi_end_tl
 }

\tl_new:N \l_yl_multi_start_tl
\tl_new:N \l_yl_multi_end_tl

\keys_define:nn { yl/multiimages }
 {
  env .choice:,
  env/none .code:n =
   \tl_set:Nn \l_yl_multi_start_tl { \par\addvspace{\topsep} }
   \tl_set:Nn \l_yl_multi_end_tl { \par\addvspace{\topsep} },
  env/figure .code:n =
   \tl_set:Nn \l_yl_multi_start_tl { \__yl_multi_beginfigure:V \l_yl_multi_pos_tl }
   \tl_set:Nn \l_yl_multi_end_tl { \end{figure} },
  env .initial:n = none,

  pos .tl_set:N = \l_yl_multi_pos_tl,
  pos .initial:n = { htp },

  outer .dim_set:N = \l_yl_multi_outer_dim,
  outer .initial:n = 3.6cm,
  inner .dim_set:N = \l_yl_multi_inner_dim,
  inner .initial:n = 3.6cm,

  skip .dim_set:N = \l_yl_multi_skip_skip,
  skip .initial:n = \lineskip,

  last .choice:,
  last/fill .code:n = 
   \tl_set:Nn \l_yl_multi_last_tl { { \parfillskip=0pt\par } },
  last/center .code:n =
   \tl_set:Nn \l_yl_multi_last_tl { { \parfillskip=0pt plus 2fil\par } },
  last .initial:n = fill,
 }

\cs_new_protected:Npn \__yl_multi_beginfigure:n #1
 {
  \begin{figure}[#1]
 }
\cs_generate_variant:Nn \__yl_multi_beginfigure:n { V }

\cs_new_protected:Npn \yl_multiimages:n #1
 {
  \tl_set:Nn \l_tmpa_tl { #1 }
  \tl_remove_all:Nn \l_tmpa_tl { \par }
  \tl_replace_all:Nnn \l_tmpa_tl { !!img ~ } { \__yl_multi_img:w }
  \tl_replace_all:Nnn \l_tmpa_tl { !!cap ~ } { \__yl_multi_cap:w }
  \tl_use:N \l_tmpa_tl \tl_use:N \l_yl_multi_last_tl
 }
\cs_generate_variant:Nn \yl_multiimages:n { V }

\cs_new_protected:Npn \__yl_multi_img:w #1 !!
 {
  \begin{minipage}{\l_yl_multi_outer_dim}\centering
  \includegraphics[width=\l_yl_multi_inner_dim]{ #1 }
 }
\cs_new_protected:Npn \__yl_multi_cap:w #1 !!
 {
  \captionof{figure}{#1}
  \end{minipage}\hspace{1pc plus 3pc}
 }
\ExplSyntaxOff   


\begin{multiimages}[last=center, env=figure,pos=ht]
  !!img 1975!! 
  !!cap This is the first caption!!

  !!img 1976!!
  !!cap This is the second caption!!

  !!img 1977!!
  !!cap This is the third caption!!

  !!img 1978!!
  !!cap This is the fourth caption!!

  !!img 1979!!
  !!cap This is the fifth caption!!

\end{multiimages}


\begin{multiimages}[last=center]
  !!img example-image-a!! 
  !!cap This is the first caption!!

  !!img example-image-a!!
  !!cap This is the second caption!!

  !!img example-image-a!!
  !!cap This is the third caption!!

  !!img example-image-a!!
  !!cap This is the fourth caption!!

  !!img example-image-a!!
  !!cap This is the fifth caption!!

\end{multiimages}

\begin{multiimages}[last=center,env=figure,pos=p,inner=3cm,skip=10ex]
  !!img example-image-a!! 
  !!cap This is the first caption!!

  !!img example-image-a!!
  !!cap This is the second caption!!

  !!img example-image-a!!
  !!cap This is the third caption!!

  !!img example-image-a!!
  !!cap This is the fourth caption!!

  !!img example-image-a!!
  !!cap This is the fifth caption!!

\end{multiimages}

\subsection{Token list conditionals}

The token list conditionals follow the same pattern than for other lists and are somewhat easier to remember.
There are conditional to check if the token list is empty, blank or equal to another. We just list the functions and provide an example at the end.

\begin{docCommand}{tl_if_blank:nTF}{\marg{token list} \marg{true code} \marg{false code}}
  Tests if the \meta{token list} consists only of blank spaces (i.e. contains no item). The test is
  true if \meta{token list} contains zero or more explicit space characters (explicit tokens with character
  code 32 and category code 10), and is false otherwise.
\end{docCommand}


\begin{texexample}{Token list conditionals}{ex:tlconditionals}
\ExplSyntaxOn
  \tl_set:Nn \l_tempa_tl {}
  \tl_if_blank:nTF \l_tempa_tl {\TRUE}{\FALSE}
\ExplSyntaxOff
\end{texexample}

\begin{docCommand}{tl_if_empty:NTF}{\marg{token list} \marg{true code} \marg{false code}}
  Tests if the \meta{token list variable} is entirely empty (i.e. contains no tokens at all). If the token list variable contains even a single space it returns false.
\end{docCommand}

\begin{texexample}{Token list conditionals}{ex:tlconditionals}
\ExplSyntaxOn
  \tl_set:Nn \l_tempa_tl { }
  \tl_if_empty:nTF \l_tempa_tl {\TRUE}{\FALSE}
  \tl_set:Nn \l_tempb_tl {}
  \tl_if_empty:nTF \l_tempb_tl {\TRUE}{\FALSE}
\ExplSyntaxOff
\end{texexample}

The |tl_case:NnTF| checks the token list variable against a series of one or more variables.

\begin{texexample}{Case}{tl}
\ExplSyntaxOn
\tl_set:Nn \l_tempa_tl {apple}
\tl_set:Nn \l_tempb_tl {apple}
\tl_set:Nn \l_tempc_tl {orange}

\tl_case:NnTF \l_tempa_tl
{
  { \l_tempb_tl }{is~apple~} 
  {\l_tempb_tl}{is~orange~}
}
{\TRUE}
{\FALSE}

\tl_set:Nn \l_tempb_tl {orange}

\tl_case:NnTF \l_tempc_tl
{
  { \l_tempb_tl }{is~orange~} 
  { \l_tempa_tl }{is~other~}
}
{\TRUE}
{\FALSE}

\ExplSyntaxOff
\end{texexample}

Another useful function tests if a token list var is in another token list. 



 
	\chapter{LaTeX 3 clists module}

\epigraph{``I bet the human brain is a kludge.’’ }{---Marvin Minsky}

\precis{This chapter explores the expl3 comma delimited lists. It provides numerous working examples to demonstrate the use of the numerous available functions, provided by the module.}

\section{Introduction}

One of the most common data structure that computer languages provide are comma delimited lists.
 Comma lists contain ordered\footnote{Ordered does not mean sorted. It means they keep the order they were entered.} data where items can be added to the left
 or right end of the list. The resulting ordered list can then
 be mapped over using \docAuxCommand*{clist_map_function:NN}. Several items can
 be added at once, and spaces are removed from both sides of each item
 on input. Hence,
 \begin{verbatim}
   \clist_new:N \l_my_clist
   \clist_put_left:Nn \l_my_clist { ~ a ~ , ~ {b} ~ }
   \clist_put_right:Nn \l_my_clist { ~ { c ~ } , d }
 \end{verbatim}
 results in \docAuxCommand*{l_my_clist} containing |a,{b},{c~},d|.
 Comma lists cannot contain empty items, thus
 \begin{verbatim}
   \clist_clear_new:N \l_my_clist
   \clist_put_right:Nn \l_my_clist { , ~ , , }
   \clist_if_empty:NTF \l_my_clist { true } { false }
 \end{verbatim}
 will leave \texttt{true} in the input stream. To include an item
 which contains a comma, or starts or ends with a space,
 surround it with braces.  The sequence data type should be preferred
 to comma lists if items are to contain |{|, |}|, or |#| (assuming the
 usual \TeX{} category codes apply).

Implementation of list data structure normally provide the minimum following operations:

\begin{enumerate}
\item a constructor for creating an empty list;
\item an operation for testing whether or not a list is empty;
\item an operation for prepending an entity to a list
\item an operation for appending an entity to a list
\item an operation for determining the first component (or the "head") of a list
\item an operation for referring to the list consisting of all the components of a list except for its first (this is called the "tail" of the list.)
\end{enumerate}

 \section{Creating and initialising comma lists}

 \begin{docCommand}{clist_new:N}{ \meta{comma list}}
   Creates a new \meta{comma list} or raises an error if the name is
   already taken. The declaration is global. The \meta{comma list} will
   initially contain no items.
 \end{docCommand}


\begin{docCommand}{clist_const:Nn}{ \meta{clist~var} \marg{comma list}}
   Creates a new constant \meta{clist~var} or raises an error
   if the name is already taken. The value of the
   \meta{clist~var} will be set globally to the
   \meta{comma list}.
 \end{docCommand}

\begin{docCommand}{clist_clear:N}{ \meta{comma list}}
   Clears all items from the \meta{comma list}.
\end{docCommand}

 \section{Adding data to comma lists}

Adding data to a comma delimited list, is normally done through the use of helper functions and user commands.
If it is to be done once for example at the beginning of a document then it is a once off operation and we can use one of the \docAuxCommand*{clist_set} variants shown below. If the items are to be added programmatically or by the user in more than one place, then one of the functions \docAuxCommand*{clist_put_left} or \docAuxCommand*{clist_out_right} should be used. These prepend or append to the list, so goodbye \docAuxCommand*{@cdr}, \docAuxCommand*{@car} and their friends. 



% \begin{function}[added = 2011-09-06]
%   {
%     \clist_set:Nn,  \clist_set:NV,
%     \clist_set:No,  \clist_set:Nx,
%     \clist_set:cn,  \clist_set:cV,
%     \clist_set:co,  \clist_set:cx,
%     \clist_gset:Nn, \clist_gset:NV,
%     \clist_gset:No, \clist_gset:Nx,
%     \clist_gset:cn, \clist_gset:cV,
%     \clist_gset:co, \clist_gset:cx
%   }
  \begin{docCommand}{clist_set:Nn}{%
      \meta{comma list} \{
      \meta{item$_1$},
      \ldots,
      \meta{item$_n$} 
      \}
      }
   Sets \meta{comma list} to contain the \meta{items},
   removing any previous content from the variable.
   Spaces are removed from both sides of each item. Variants for : Nv,Nx,cV,cx,NV,cV exist.
 \end{docCommand}

\begin{texexample}{Creating a Comma delimited list}{ex:clists}
\ExplSyntaxOn
\clist_gset:Nn \phd_test_clist {one, two, three, four, five}
\phd_test_clist
\clist_gput_right:Nn \phd_test_clist{six, seven, eight}
\cs_new:Nn \nine: {9}
\clist_gput_right:Nn \phd_test_clist\nine:
\clist_gput_right:Nn \phd_test_clist\nine:
\par\phd_test_clist
\par\clist_if_in:NnTF\phd_test_clist {eight} {true} {false}
\ExplSyntaxOff 
\end{texexample}


%   \begin{syntax}
%     \docAuxCommand*{clist_put_left:Nn} \meta{comma list} \meta{item 1},\ldots{},\meta{item n}}
%   \end{syntax}
%   Appends the \meta{items} to the left of the \meta{comma list}.
%   Spaces are removed from both sides of each item.
% \end{function}
%
% \begin{function}[updated = 2011-09-05]
%   {
%     \clist_put_right:Nn,  \clist_put_right:NV,
%     \clist_put_right:No,  \clist_put_right:Nx,
%     \clist_put_right:cn,  \clist_put_right:cV,
%     \clist_put_right:co,  \clist_put_right:cx,
%     \clist_gput_right:Nn, \clist_gput_right:NV,
%     \clist_gput_right:No, \clist_gput_right:Nx,
%     \clist_gput_right:cn, \clist_gput_right:cV,
%     \clist_gput_right:co, \clist_gput_right:cx
%   }
 \begin{docCommand} {clist_put_right:Nn}  {\meta{comma list} \{\meta{item 1},\ldots{},\meta{item n}\}}
   Appends the \meta{items} to the right of the \meta{comma list}.
   Spaces are removed from both sides of each item.
\end{docCommand}

\begin{texexample}{Adding content to the list}{}
\ExplSyntaxOn
\clist_new:N \l_my_clist
\clist_put_right:Nn \l_my_clist{\square, \Diamond, \diamond, d, e, f}
\clist_put_left:Nn \l_my_clist{1,2,3,4,5,6,7,8,9,\hfill, 0}
\clist_use:Nn \l_my_clist{~}
\ExplSyntaxOff
\end{texexample}



\begin{texexample}{Adding content to the list}{}
\ExplSyntaxOn
\clist_put_right:Nn \l_my_clist {\alpha, \beta, \gamma, \delta, \epsilon}
\clist_put_left:Nn \l_my_clist {{\alpha\ldots}}
\[ \clist_use:Nnnn \l_my_clist {,} {,} {,} \]
\ExplSyntaxOff
\end{texexample}

\section{Mapping to comma lists}

\begin{texexample}{Mapping}{ex:longimages}
\ExplSyntaxOn
\clist_set:Nn \imgdb:n {fig145,fig161,fig162,fig163,fig164,fig165,fig166}
\clist_map_inline:Nn \imgdb:n {\includegraphics[width=1.5cm]{./images-01/#1}}
\ExplSyntaxOff
\end{texexample}

If the inline code is long, it might be preferable to use the function version of the map. This callback function should accept one parameter. Note that the mapping command format does not need the \#1 you only provide it with the function name.

\begin{texexample}{Mapping}{ex:longimages}
\ExplSyntaxOn
\cs_set:Npn \put_graphic:n #1 
   {
     \includegraphics[width=1.5cm]{./images-01/#1}
   }
\cs_set:Npn \put_graphic_with_space:n #1 
   {
     \put_graphic:n {#1}
     \hspace{5pt}
   }   
\clist_set:Nn \imgdb:n {fig145,fig161,fig162,fig163,fig164,fig165,fig166}

  \clist_map_function:NN \imgdb:n \put_graphic:n\par  
  \clist_map_function:NN \imgdb:n \put_graphic_with_space:n
\ExplSyntaxOff
\end{texexample}

The inline version is obviously a bit faster, as it does less work, but personally I prefer the callback style as it produces more readable code. Of course we could have used the |clist_if_in:NnTF| conditional. There are numerous conditionals and these are discussed later on.

The mapping function definitions are shown below,

\begin{docCommand}{clist_map_function:NN}{\meta{comma list} \meta{function}}
Applies a callback function to each item stored in the comma list. The function will receive one argument for each iteration. The items are returned from left to right. 
\end{docCommand}

\begin{docCommand}{clist_map_inline:NN}{\meta{comma list} \meta{inline function}}
Applies \meta{inline function} to every \meta{item} stored within the \meta{comma list}. The \meta{inline function} should consist of code which will receive the \meta{item} as \#1. One inline mapping can be nested inside another. The items are returned from left to right.
\end{docCommand}

There are is a third type of mapping function available where each entry in the list is passed to a variable which is then used in a function.

\begin{docCommand}{clist_map_variable:NNn}{\meta{comma list} \meta{tl. var} \marg{inline function}}
Stores each entry in the \meta{comma list} in turn in the \meta{tl var} and applies \meta{function} using \meta{tl var}. the function will usually consist of code making use of the \meta{t var}, but this is not enforced. One variable mapping can be nested inside another. the \meta{items} are returned from left to right.
\end{docCommand}

\subsection{Terminating mapping functions}

All lists in |expl3| can be terminated using a break function. A clist breaks by using the \docAuxCommand*{clist_map_break:n} function. 

\begin{texexample}{Mapping}{ex:longimages}
\ExplSyntaxOn
\cs_set:Npn \put_graphic:n # 1 
   { 
     \includegraphics[width=1.48cm]{./images-01/#1}
   }
\cs_set:Npn \put_graphic_with_space:n #1 
   {
      \parbox[b]{1.52cm}{\put_graphic:n {#1}\par\centering #1}
     \hspace{5pt}
   }   
   
\clist_set:Nn \imgdb:n {fig145,fig161,fig162,fig163,fig164,fig165,fig166}

 \clist_map_function:NN \imgdb:n \put_graphic_with_space:n\par
 \clist_map_inline:Nn \imgdb:n
     {
         \str_if_eq:nnTF {#1} {fig166}
         {\clist_map_break:n { \PASS~ \put_graphic:n {#1} ~#1} }
         {
          \FAIL #1
         }
     }
     
\ExplSyntaxOff
\end{texexample}

What just happened have added \docAuxCommand*{clist_map_inline:Nn} which iterates through all the elements in a list until a search string is found. As you can see as a search function it will be slow as it has to iterate through all the elements of a list. A more efficient way would have been to use \tex’s scanning mechanism of delimited functions to find the item. 

I have named the |clist| in the above examples as |imgdb| as one can easily extend the functions to store other information besides the filename. This can be done in many ways for example using the \meta{property} module of |expl3| or using |\csname|. In Chapter~\ref{ch:longfifgures} \nameref{ch:longfigures} we have used traditional techniques to typeset a lot of figures, in a similar fashion to a long table. Here we provide a similar example using |expliii|.

let us consider the simple case of a record for a person.

\begin{texexample}{Person record}{}
\ExplSyntaxOn
% create a new clist
\clist_new:N \personDB 

% auxiliary function to typeset an image
\cs_gset:Npn \put_graphic:n #1 
   {
     \includegraphics[height=3cm]{#1}
   }

% auxiliary function to enclose the image in a minipage      
\cs_gset:Npn \put_graphic_with_space:n #1 
   {
       \begin{minipage}[b]{3cm}
             \centering
             \put_graphic:n {#1}\par
             \csname#1_name\endcsname\\
             \csname#1_occupation\endcsname\\
      \end{minipage}\hspace{5pt}  
   }   

% helper function to add a person record to the clist (*@\label{lin:personrecord}@*)
\cs_gset:Npn \addtodb:nn #1#2#3
    {
        \cs_gset:cpn { #1_name } { #2 }
        \cs_gset:cpn { #1_occupation } { #3 }
        \clist_gput_left:Nn \personDB { #1 }
    }
    
%        
\addtodb:nn {turner}  {Ted~Turner}  {tycoon}
\addtodb:nn {britney} {Britney Spears} {actress}
\addtodb:nn {che} {Che ~Guevara} {revolutionary}
\clist_map_function:NN \personDB \put_graphic_with_space:n\par
\ExplSyntaxOff
\end{texexample}

The Line~\ref{lin:personrecord} creates to macros, one that will hold the name of the person and another that will hold the occupation. Note that the code would have normally used a |\csname| construction. Here |expl3| takes care of both the |expandafter| as well as the |\csname| construct, simply by using |:cpn| version of |gset|.

What is different with this example, we have added the \docAuxCommand*{addtodb:nn} to add the person names to the list. This still has to be done as an author interface, but as it is just an example, I want to keep the code short. 

\textbf{Automating the addition of fields and records} We have named our database |imageDB| and we have called it a database, but it is so far very unfriendly and all the fields are hard wired in the next Chapter we will create a more appropriate record database.

\textbf{Create a new data base} First we concern ourselves with creating a new database. This is the very first activity we need to define. We store the names of the databases in a master list which we have named |\g_DB_dbs_clist|. We will prefix all our functions and variables with |DB| and we will use this as our module name. 



\textbf{Specifying the database meta data} Our databases will be also records or objects if you want to use an inexactitude name and will also hold information, this is termed \emph{meta data}:

\begin{tabular}{ll}
  name   & \meta{database name} \\
  fields & \meta{list containing the fields as fieldnames}\\
  status & \meta{active or not active}\\
  number of records &\\
  tables & \meta{list}\\
  views  & \\
\end{tabular}

\def\paragraph#1{{\par\leavevmode\bfseries#1}}

\paragraph {Create the master database list} All databases that we will create will be stored
as meta data into another list. This is used only internally at this stage, so we give it an |expl3| sexy name \docAuxCommand{g_db_dbs_clist}.

\begin{texexample}{Creating a database package}{ex:master DB list}
\ExplSyntaxOn
% already defined no need to have it in the example
 \clist_new:N \g_db_dbs_clist
\ExplSyntaxOff
\end{texexample}

\paragraph{Constructor function} Next we create a function that is called when we
need to create a new database.

\begin{texexample}{Continued..}{}
\ExplSyntaxOn
% constructor function
\cs_gset:Npn \g_db_construct_clist:n #1
  { 
% create new DB
  \clist_new:c {#1} 
	% add to master
  \clist_put_left:Nn \g_db_dbs_clist { #1 }
% create meta table
  \g_construct_metatable:n { # 1}
  }
\ExplSyntaxOff
\end{texexample}
		
\paragraph{Creating tables} So far we have created the functions that we need to create a new database. Next we can start writing functions for creating tables for a database. In reality, I called them tables, but this is a misnomer as they hold other stuff as well. 
		
\begin{texexample}{...continued}{ex:db4}		
\ExplSyntaxOn
% persons metatable
% PERSONS-METANAME
% PERSONS-STATUS
% PERSONS-TABLES
\cs_gset:Npn \g_construct_metatable:n #1 
  {
    \cs_gset:cpn   {#1-METANAME  } {   #1    }
    \cs_gset:cpn   {#1-METASTATUS} {-NoValue-}
    \clist_gset:cn {#1-METATABLES} {-NoValue-}
  }		
  
% PERSONS-TABLE-TABLENAME 
% PERSONS-TABLE-TABLENAME-FIELDS (list)		

\cs_gset:Npn \g_construct_table:cc #1 #2 
  {
    \cs_gset:cpn   {#1-TABLE-#2-NAME      } {#2}
    \cs_gset:cpn   {#1-TABLE-#2-STATUS    } {}
    \clist_gset:cn {#1-TABLE-#2-FIELDNAMES} {}
    
    % index key as edef
    \tl_gset:cx  {#1#2-} {name}
    
    % data holding list
    \clist_gset:cn { #1 #2 } { } %(*@\label{lin:personsfamous}@*)
  }
\ExplSyntaxOff  
\end{texexample}

The interesting part is line~\ref{lin:personsfamous} which is a comma delimited list
that will hold all the index keys.

\begin{texexample}{Databases...continued}{ex:fields}
\ExplSyntaxOn
% adds a fieldname to fieldnames
% PERSONS-TABLE-TABLENAME-FIELDNAMES
\cs_gset:Npn \add_fieldname #1 #2 #3
  {
    \clist_gput_left:cx {#1-TABLE-#2-FIELDNAMES} {#3}

  }
%

\cs_gset:Npx \create_index_field #1#2#3#4
  {
    \clist_gput_left:cx {#1#2} {#4}
    
  }  
% create DB table FAMOUS 
\cs_gset:Npx \add_data_index #1#2#3#4
  {
    \clist_gput_left:cx {#1#2} {#4}
    
  } 
  
% add data if is index goes onto clist  
% PERSONS-FAMOUS-ID-SURNAME-VALUE
%   
\cs_gset:Npn \add_field_data #1#2#3#4#5 
  {
   \cs_gset:cpn {#1#2#3#4} 
    { #5    }
  } 
  



% read a field        
\cs_gset:Npn \get_field #1#2#3#4
  { 
    \cs:w #1#2#3#4\cs_end:  
  }
                                
 % create DB PERSONS   
\g_db_construct_clist:n {PERSONS}
\g_construct_table:cc {PERSONS}{FAMOUS}                                
%
\gdef\AddPerson#1#2#3#4{
	\add_data_index {PERSONS} {FAMOUS} {name} {#1}
	\add_field_data {PERSONS} {FAMOUS}{#1} {firstname   } {#1}
	\add_field_data {PERSONS} {FAMOUS}{#1} {surname   } {#2}
	\add_field_data {PERSONS} {FAMOUS}{#1} {occupation} {#3} 
	\add_field_data {PERSONS} {FAMOUS}{#1} {photo} {#4}
}
%
%\get_field {PERSONS} {FAMOUS} {Iggy}  {photo} 
\gdef\PrintImages#1#2{
  \centering 
  \clist_map_inline:cn {#1#2}
    {
      \includegraphics[height=3cm]
      {./martin-schoeller/
        \get_field {#1}{#2}{##1}{photo}
      }\hskip1sp
    }
}
\ExplSyntaxOff
\end{texexample}

What just happened is that we have created two lists one to hold DBs metadata as a simple list and a second |PERSONS|. We have also created the meta-data record. 
 
\begin{texexample}{Database ...continued}{ex:db2}
\AddPerson {Barack} {Obama} {Actor} {barack_obama_2004}
\AddPerson {Iggy} {Pop} {Actor} {iggy_pop_2001}
\AddPerson {Henry} {Kissinger} {Arsehole} {henry_kissinger_2007}
\AddPerson {Frankie} {Velilla} {Student} {frankie_velilla_2001}
\AddPerson {Cindy} {Sheman} {Queen} {cindy_sheman_2000}
\AddPerson {Joe} {Namath} {Tough} {joe_namath_2006}
\AddPerson {Christopher} {Walken} {Tough} {christopher_walken_2000}
\AddPerson {Xiakababoi} {Xiakababoi} {Tough} {xiakababoi_2005}
\AddPerson {Jack} {Nicholson} {Tough} {jack_nicholson_2002}
\AddPerson {Robert} {Deniro} {Actor} {robert_DeNiro_2006}
\PrintImages{PERSONS}{FAMOUS}
\end{texexample}


Now what happens if we decide that we want to add another field in 
our database of famous people, say their biography? we would need to add
another document level command |\AddPersonBio|

\begin{texexample}{adding a bio field}{ex:bio}
\ExplSyntaxOn
\long\gdef\AddPersonBio #1#2 {
   \add_field_data {PERSONS} {FAMOUS} {#1} {bio} {#2}
}
\ExplSyntaxOff
\end{texexample}

Let us add some data for some of the person records we have in our database.

\ExplSyntaxOn
\DeclareDocumentCommand \GetBio {m} {
  \get_field {PERSONS}{FAMOUS}{#1}{bio}
}
\DeclareDocumentCommand \GetPhoto {m} {
  \includegraphics[width=0.8\linewidth] {./martin-schoeller/
    \get_field {PERSONS} {FAMOUS} {#1} {photo}} 
  }
\DeclareDocumentCommand \GetFullName {m} {
    \get_field {PERSONS} {FAMOUS} {#1} {firstname}
    \space 
    \get_field {PERSONS} {FAMOUS} {#1} {surname}
} 
\ExplSyntaxOff


\begin{texexample}{add bio to some records}{ex:bio1}
\ExplSyntaxOn
\DeclareDocumentCommand \GetBio {m} {
  \get_field {PERSONS}{FAMOUS}{#1}{bio}
}
\DeclareDocumentCommand \GetPhoto {m} {
  \includegraphics[width=0.8\linewidth] {./martin-schoeller/
    \get_field {PERSONS} {FAMOUS} {#1} {photo}} 
  }
\DeclareDocumentCommand \GetFullName {m} {
    \get_field {PERSONS} {FAMOUS} {#1} {firstname}
    \space 
    \get_field {PERSONS} {FAMOUS} {#1} {surname}
}    
\ExplSyntaxOff
\end{texexample}
\begin{texexample}{add some more declarations}{ex:2}
\AddPersonBio{Robert}{
  Robert De Niro (/dəˈnɪroʊ/; born August 17, 1943) is an American actor and producer   who has starred in over 90 films. His first major film roles were in the sports drama Bang the Drum Slowly (1973) and Martin Scorsese's crime film Mean Streets (1973). In 1974, after being turned down for the role of Sonny Corleone in the crime film The Godfather (1972), he was cast as the young Vito Corleone in The Godfather Part II (1974), a role for which he won the Academy Award for Best Supporting Actor.

De Niro's longtime collaboration with Scorsese later earned him an Academy Award for Best Actor for his portrayal of Jake LaMotta in the 1980 film Raging Bull. He also earned nominations for the psychological thrillers Taxi Driver (1976) and Cape Fear (1991), both directed by Scorsese. De Niro received additional Academy Award nominations for Michael Cimino's Vietnam war drama The Deer Hunter (1978), Penny Marshall's drama Awakenings (1990), and David O. Russell's romantic comedy-drama Silver Linings Playbook (2012). His portrayal of gangster Jimmy Conway in Scorsese's crime film Goodfellas (1990) earned him a BAFTA nomination in 1990.[1] De Niro has earned four nominations for the Golden Globe Award for Best Actor – Motion Picture Musical or Comedy, for his work in the musical drama New York, New York (1977), opposite Liza Minnelli, the action comedy Midnight Run (1988), the gangster comedy Analyze This (1999), and the comedy Meet the Parents (2000). He has also simultaneously directed and starred in films such as the crime drama A Bronx Tale (1993) and the spy film The Good Shepherd (2006). De Niro has also received the AFI Life Achievement Award in 2003 and the Golden Globe Cecil B. DeMille Award in 2010.}

\AddPersonBio {Jack}{
John Joseph "Jack" Nicholson (born April 22, 1937) is an American actor and filmmaker. Throughout his career, Nicholson has portrayed unique and challenging roles, many of which include dark portrayals of excitable, neurotic and psychopathic characters. Nicholson's 12 Academy Award nominations make him the most nominated male actor in the Academy's history.

Nicholson has won the Academy Award for Best Actor twice, one for the drama One Flew Over the Cuckoo's Nest (1975) and the other for the romantic comedy As Good as It Gets (1997). He also won the Academy Award for Best Supporting Actor for the comedy-drama Terms of Endearment (1983). Nicholson is tied with Walter Brennan and Sir Daniel Day-Lewis as one of three male actors to win three Academy Awards. In 1988 Nicholson won a Grammy Award for Best Album for Children for The Elephant's Child. He is well known for playing Frank Costello in the Martin Scorsese-directed crime drama The Departed (2006), Jack Torrance in the Stanley Kubrick–directed psychological horror film The Shining and the Joker in Batman (1989).

Nicholson is one of only two actors to be nominated for an Academy Award for acting in every decade from the 1960s to the 2000s; the other was Michael Caine. He has won six Golden Globe Awards, and received the Kennedy Center Honor in 2001. In 1994, he became one of the youngest actors to be awarded the American Film Institute's Life Achievement Award. Other notable films in which he has starred include the road movie Easy Rider (1969), the drama Five Easy Pieces (1970), the comedy-drama film The Last Detail (1973), the neo-noir mystery film Chinatown (1974), the drama The Passenger (1975), the epic film Reds (1981), the romantic horror film Wolf (1994), the legal drama A Few Good Men (1992), the Sean Penn-directed mystery film The Pledge (2001), and the comedy-drama About Schmidt (2002).
}
\end{texexample}

Finally continuing our example we will now define a \docAuxCommand{PrintBio} that can be used
to finally extract the data and present typeset it.

  
\begin{texexample}{Printing the Bios}{ex:bio3}
\long\gdef\PrintBio#1{%
\par
 {\pagebreak
 \leavevmode 
 \Huge
 \bfseries
 \centerline{\GetFullName {#1}}}
 \par
 \vspace{20pt}

 {\centering
  \GetPhoto {#1}\par
  \vspace{20pt}}

 \parindent1em
 \GetBio {#1}
 \vfill
}

\PrintBio{Robert}

\PrintBio{Jack}
\end{texexample}

There are a lot of improvements that we can do to the code. Firstly we have not done any error checking. The idea of pre-packaged code is that the finer details can be handled. Error checking should be done for example to verify that an image is available on disk. Also not to hard wired any directories. Sorting is still an issue. Our indexing is also inadequate. What happens if we have Robert DeNiro and Robert Williams? We indexed on the Robert. We would have been better off to add an index key automatically or index by using both name and surname. All these are issues that need to be incorporated. 



\chapter{Queues}
\section{Queue Fundamentals}

A queue is an ordered list in which all insertions are made at one end, called the rear end, while all deletions are made at the other end, called the front end. Given a queue $Q=(a_1,a_2,\dots,a_n)$ with $a_1$ as the front element and $a_n$ as the rear element, we say that $a_{i+1}$ is behind $a_1$ $1 \leq i <n$.

\section{Operations on a Queue}

The operations which are carried on queue are similar to these which are carried on a stack, except their semantics are different. The operations are:

\begin{enumerate}
\item To create a queue
\item To insert an element into the queue
\item To delete an element from the queue
\item To check which element is in the front of  the queue
\item To check whether a queue is empty or not.
\end{enumerate}

\begin{figure}[htbp]
\centering
\includegraphics[width=0.5\textwidth]{queue}
\end{figure}

Since this is a book about typesetting, the next example will create a queue structure that will typeset the operations of a queue and provide diagrams to illustrate the algorithmic steps involved.


\begin{docCommand}{CreateQueue}{\meta{queue name}}
Creates an empty queue .
\end{docCommand}

\def\anitem{{\color{blue}\vrule height1.5cm width0.4cm}\thinspace}
\DeclareDocumentCommand\anitem{O{blue}}{%
{\color{#1}\vrule height1.5cm width0.4cm}\thinspace
}
\NewDocumentCommand\EnqueueString{s}{
  \IfBooleanTF #1
     {Enqueue $\rightarrow$}
     {Enqueue \phantom{$\rightarrow$}}
}

\NewDocumentCommand\DequeueString{s}{
  \IfBooleanTF #1
     {\phantom{$\rightarrow$} Dequeue $\rightarrow$ }
     {\phantom{$\rightarrow$} Dequeue \phantom{$\rightarrow$}}
}
\begin{enumerate}
\item The conventions we will use is that when an item is enqueued it will be typeset in red as shown below, when it enters the front end.

Enqueue $\rightarrow$ \anitem[red]\DequeueString* 

\item When another item is added the above procedure is repeated, but this time the elements not in the front are shown in blue.

\EnqueueString* \anitem[red]\anitem \phantom{$\rightarrow$} Dequeue

\item Enqueue one more item will change the diagram to the following:

\EnqueueString \anitem \anitem \anitem \anitem \anitem \phantom{$\rightarrow$} Dequeue \hfill \anitem[blue!30] 

\item Enqueue one more item will change the diagram to the following:

\EnqueueString \anitem \anitem \anitem \anitem  \DequeueString \hfill \anitem[blue!30] \anitem[blue!30]

\item To summarize the typeset diagram represents the three states of the queue, enqueue, status and dequeue. If a right arrow is shown it either enqueued or dequeued an element. If none is shown it represents the status of the system
\end{enumerate}

I have specifically made the example a bit more complicated, in order to reinforce some of the concepts discussed in other chapters.

The example requires that when a dequeuing command is entered it is indicated with a right arrow (|dequeue \rightarrow|) the arrow is not shown when the enqueuing operation takes place. To keep the length of the diagram spaced properly it requires that a phantom command is used for the enqueing operation.

\subsection{Coding auxiliary macros}

We will need two auxiliary macros to typese the enqueue and dequeue strings with or without arrows. We will use the \pkgname{xparse} package to create the commands. We will use the star version of the command as a toggle to show the \docAuxCommand*{rightarrow} or not. If the command is enetered with a star it will leave the right amount of space to the right of the string, so that all diagrams line nicely to the left. This is achieved using the  \docAuxCommand*{phantom} command that we have encountered earlier.

\emphasis{IfBooleanTF}
\begin{teXXX}
\NewDocumentCommand\EnqueueString{s}{
  \IfBooleanTF #1
     {Enqueue $\rightarrow$ }
     {Enqueue \phantom{$\rightarrow$}}
}
\NewDocumentCommand\DequeueString{s}{
  \IfBooleanTF #1
     {\phantom{$\rightarrow$} Dequeue $\rightarrow$ }
     {\phantom{$\rightarrow$} Dequeue \phantom{$\rightarrow$}}
}
\end{teXXX}

\subsection{Creating the Queue Macros}

In order to typeset the diagrams we will use two queues. One to store the main queue and a second one to store the dequeue items. Before we code the actual functions it will be nice to think of the  commands we want to offer our users. This will also dictate to an extend the code we require.

\begin{verbatim}
\DrawQueStatus
\Enque
\Deque
\DrawEnque
\DrawDeque
\end{verbatim}

\begin{texexample}{Adding content to the sequence}{}
\ExplSyntaxOn
\seq_new:N \g_qlisti
\seq_new:N \g_qlistii
\seq_gpush:Nn \g_qlisti{\anitem[blue]}
\seq_gpush:Nn \g_qlisti{\anitem[blue]}

\seq_push:Nn \g_qlistii{\anitem[blue!30]}

\DeclareDocumentCommand\Enque{O{red}}
   {
      \seq_gpush:Nn \g_qlisti{\anitem[#1]}
   }
   
 \DeclareDocumentCommand\Deque{O{blue!30}}
   {
      \seq_gpop_left:NN \g_qlisti \@tempa
      \seq_gpush:Nn\g_qlisti{\anitem[blue]}
      \seq_gpush:Nn \g_qlistii{\anitem[#1]}
   }  
   
\Enque\Enque\Enque\Enque

\EnqueueString\seq_use:Nn \g_qlisti {} 

\DequeueString*\hfill\seq_use:Nn \g_qlistii{}
\ExplSyntaxOff
%%%%
\end{texexample}
\ExplSyntaxOn
\DeclareDocumentCommand\Enque{O{red}}
   {
      \seq_gpush:Nn \g_qlisti{\anitem[#1]}
   }
   
 \DeclareDocumentCommand\Deque{O{blue!30}}
   {
      \seq_gpop_left:NN \g_qlisti \@tempa
      \seq_gpush:Nn\g_qlisti{\anitem[blue]}
      \seq_gpop_right:NN \g_qlisti \@tempa
      \seq_gpush:Nn \g_qlistii{\anitem[#1]}
   }  
\ExplSyntaxOff
   
One of the characteristics of the programming process is that it is like painting. Some programmers come up with  excellent code on their first attempt, whereas most of us will \emph{refactor} the code over several passes either to improve it, optimize it or catch possible errors.

A subtle issue with the above code is if we enqueue a number of items and then dequeue only the first item will change from red to blue the rest will be still in the que as red. What we will have to do is modify the \docAuxCommand*{Enque} to check if the list is not empty to remove the head item and replace it with a blue box, before effecting the enque operation. This will also give us a chance to use the sequence conditional functions for emptiness. We should also add the conditional in the \docAuxCommand*{Deque} function as well as the author typesetting commands \docAuxCommand*{DrawEnque} and \docAuxCommand*{DrawDeque}.

\begin{texexample}{The drawing functions}{ex:drgfunctions}
\ExplSyntaxOn
\DeclareDocumentCommand\DrawDeque{ O{blue!30} }
  { 
   \EnqueueString  \seq_use:Nn \g_qlisti {} 
   \DequeueString*  \hfill  \seq_use:Nn \g_qlistii{}    
  }
\Deque\Deque\Deque
\DrawDeque
\ExplSyntaxOff  
\end{texexample}

The \docAuxCommand*{DrawQues}, draws the two queues. This is very similar to the other two \docAuxCommand*{Draw}{\meta{deque}} or \meta{enque} functions. It just does not draw the arrows.

\begin{texexample}{The drawing functions}{ex:drgfunctions}
\ExplSyntaxOn
\DeclareDocumentCommand\DrawQues{ O{blue!30} }
  { 
   \EnqueueString  \seq_use:Nn \g_qlisti {} 
   \DequeueString  \hfill  \seq_use:Nn \g_qlistii{}    
  }
\Deque
\DrawQues
\ExplSyntaxOff  
\end{texexample}


 \chapter{Using Comma lists as stacks}
 
 In this chapter, we will look at one common Abstract Data Type (ADT), the stack. A stack is a \emph{collection}, meaning that it is a data structure that contains multiple elements. Other collections we have seen include dictionaries and lists. An ADT is defined by the operations that can be performed on it, which is called an interface. The interface for a stack consists of these operations:

\begin{description}
\item [init] Initialize a new empty stack.
\item [push]
Add a new item to the stack.
\item [pop]
Remove and return an item from the stack. The item that is returned is always the last one that was added.

\item [emptiness] Check whether the stack is empty.
\end{description}

A stack is sometimes called a “Last in, First out” or LIFO data structure, because the last item added is the first to be removed.

 Comma lists can be used as stacks, where data is pushed to and popped
 from the top of the comma list. (The left of a comma list is the top, for
 performance reasons.) The stack functions for comma lists are not
 intended to be mixed with the general ordered data functions detailed
 in the previous section: a comma list should either be used as an
 ordered data type or as a stack, but not in both ways.
 
 \begin{figure}[htbp]
 \hspace*{3cm}%optically center it
 \scalebox{0.7}{\begin{drawstack}
  \startframe
  \cell{First cell}
  \cell{Second cell}
  \finishframe{Some stack frame}
  \cell{Not interesting}
  \startframe
  \cell{Next stack frame}
  \cell{Next stack frame}
  \finishframe{Another stack frame}
\end{drawstack}}
\caption{A stack drawn with the \pkgname{drawstack} package. The package can be used to draw different stacks and their frames.}
\end{figure}

To construct a new empty stack, use the same functions as for a clist or sequence data structure. They are identical and calling them a stack is just syntactic sugar.

 \begin{docCommand}{clist_get:NN}{ \meta{comma list} \meta{token list variable}}
   Stores the left-most item from a \meta{comma list} in the
   \meta{token list variable} without removing it from the
   \meta{comma list}. The \meta{token list variable} is assigned locally.
   If the \meta{comma list} is empty the \meta{token list variable} will
   contain the marker value \docAuxCommand*{q_no_value}.
 \end{docCommand}
\makeatletter 
 \global\let\clistsort\lst@BubbleSort
\makeatother 
\begin{texexample}{Sequence}{ex:sequence}
\makeatletter
\ExplSyntaxOn
\clist_gset:Nn \title_words_not_capitalized_en 
 { 
  a,an,the,at,by,for,in,of,on,to,up,and,as,but,it,or, 
  nor,do,for,this,be,A,An,The,At,By,For,In,Of,On,To,Up, 
  And,As,But,It,Or,Nor,Do,For,This,Be, 
  abaft,aboard,about,above,absent,across,afore,after,against,along,
  alongside,amid,amidst,among,amongst,an,anenst,apropos,apud,around,
  as,aside,astride,at,athwart,atop,barring,before,behind,below,beneath,
  beside,besides,between,beyond,but,by,circa,concerning,despite,down,
  during,except,excluding,failing,following,for,forenenst,from,given,in,
  including,inside,into,lest,like,mid,midst,minus,modulo,near,next,
  notwithstanding,of,off,on,onto,opposite,out,outside,over,pace,past,
  per,plus,pro,qua,regarding,round,sans,save,since,than,through,
  throughout,till,times,to,toward,towards,under,underneath,unlike,
  until,unto,up,upon,Versus,versus,via,vice,with,within,without,worth
}
\clist_gset:Nn \abbreviations 
 {
   A.B.C.,iTunes
 }

\clist_gset:Nn \acronyms
 {
   NATO,UN,US,Scuba,Laser
 }  
\cs_new:Npn \addacronym #1 
 {
   \clist_put_left:Nn \acronyms {#1}
   \lst@BubbleSort\acronyms
   \clist_remove_duplicates:N\acronyms
 }  

\addacronym {EU}
\meaning\acronyms\\
\addacronym {AA}
\acronyms
\ExplSyntaxOff
\makeatother
\end{texexample}


\endinput
\end{document}

\section{Moving items from one list to another}{}

\begin{texexample}{Moving items from one stack to another.}{ex:stacks}
\ExplSyntaxOn
\clist_new:N \phd_stack_a
\clist_new:N \phd_stack_b
\token_to_meaning:N \phd_stack_a
\Expl_SyntaxOff
\end{texexample}

If we examine the meaning of the stacks at this stage, they are just empty macros, not holding any values.

\begin{texexample}{put something into the stacks}{}
\ExplSyntaxOn
\clist_gset:Nn \phd_stack_a {3,4,5,6,}
\clist_gpush:Nn\phd_stack_a {\ldots}

STACK a:~\phd_stack_a\par
\ExplSyntaxOff
\end{texexample}


Let us continue by popping and pushing some more values
\begin{texexample}{Continue}{}
\ExplSyntaxOn
\clist_gpop:NN\phd_stack_a\@tempa
\clist_gpush:Nx\phd_stack_b\@tempa
%% Pop a value
\clist_gpop:NN\phd_stack_a\@tempa
\clist_gpush:Nx\phd_stack_b\@tempa
\clist_gpop:NN\phd_stack_a\@tempa
\clist_gpush:Nx\phd_stack_b\@tempa
\clist_gpop:NN\phd_stack_a\@tempa
\clist_gpush:Nx\phd_stack_b\@tempa
stack a:~\phd_stack_a\par
stack b:~\phd_stack_b
\ExplSyntaxOff
%%%%%%%%%%%%%
\end{texexample}

The much promised freedom from having to deal with \tex expansion has not arrived---although we can save some frustration and typing. When moving items from |stacka| to |stack b| we have used the |:Nx| form of the command
so that the temporary token list variable is expanded. If we do not do that the second stack values will only store the last value of |@tempa|.

In practical applications the second stack is normally used as an array to just store the values. The symbol items before being popped are examined and if for example is a |+| sign the items will be summed up and placed again in the second stack to keep tally of our totals.

Our next step is to refactor the code in our example to recursively empty the first stack.

\begin{texexample}{Moving items from one stack to another}{ex:stacks}
\ExplSyntaxOn
\fboxsep=2pt
\fboxrule=0.4pt
\clist_gset:Nn \phd_stack_a {1,2,3,4,5,6,\ldots}
\clist_gset:Nn \phd_stack_b {}
original~stack a:~
\cs_gset:Nn\recurse:
 {
   \clist_gpop:NNTF\phd_stack_a\@tempa{\clist_gpush:Nx\phd_stack_b\@tempa
      \fbox{\@tempa}~
      \recurse:}{empty~stack\par}
 }  
\recurse: 
stack b:~\phd_stack_b
\ExplSyntaxOff
%%%%%%%%%%%%%
\end{texexample}

\begin{texexample}{Moving items from one stack to another}{ex:stacks}
\ExplSyntaxOn
\fboxsep=2pt
\fboxrule=0.4pt
\cs_gset:Nn\recurseb:
 {
   \clist_gpop:NNTF\phd_stack_b\@tempa{
      \if +\@tempa\relax\else\framebox[1.5em]{\strut\@tempa}\\ \fi
      \recurseb:}{empty~stack\par}
 }  
\recurseb: 
stack b:~\phd_stack_b
\ExplSyntaxOff
%%%%%%%%%%%%%
\end{texexample}

So far so good. We have managed to construct two stacks and to typeset their content in nice boxes. Hopefully, by now if you have been following the examples, you have the rudimentary skills to build our next, more complicate example that would parse a sequence of algebraic expressions and tokenize them. 



\begin{docCommand}{clist_get:NNTF}{ \meta{comma list} \meta{token list variable} \marg{true code} \marg{false code}}
 
   If the \meta{comma list} is empty, leaves the \meta{false code} in the
   input stream.  The value of the \meta{token list variable} is
   not defined in this case and should not be relied upon.  If the
   \meta{comma list} is non-empty, stores the top item from the
   \meta{comma list} in the \meta{token list variable} without removing it
   from the \meta{comma list}. The \meta{token list variable} is assigned
   locally.
 \end{docCommand}


  \begin{docCommand}{clist_pop:NN}{ \meta{comma list} \meta{token list variable}}
   Pops the left-most item from a \meta{comma list} into the
   \meta{token list variable}, \emph{i.e.}~removes the item from the
   comma list and stores it in the \meta{token list variable}.
   Both of the variables are assigned locally.
 \end{docCommand}


  \begin{docCommand}{clist_gpop:NN}{ \meta{comma list} \meta{token list variable}}
   Pops the left-most item from a \meta{comma list} into the
   \meta{token list variable}, \emph{i.e.}~removes the item from the
   comma list and stores it in the \meta{token list variable}.
   The \meta{comma list} is modified globally, while the assignment of
   the \meta{token list variable} is local. Also available as :cN
 \end{docCommand}

 \begin{docCommand}{clist_pop:NNTF}{ \meta{sequence} \meta{token list variable} \marg{true code} \marg{false code}}
   If the \meta{comma list} is empty, leaves the \meta{false code} in the
   input stream.  The value of the \meta{token list variable} is
   not defined in this case and should not be relied upon.  If the
   \meta{comma list} is non-empty, pops the top item from the
   \meta{comma list} in the \meta{token list variable}, \emph{i.e.}~removes
   the item from the \meta{comma list}. Both the \meta{comma list} and the
   \meta{token list variable} are assigned locally.
 \end{docCommand}


   \begin{docCommand}{clist_gpop:NNTF}{\meta{comma list} \meta{token list variable} \marg{true code} \marg{false code}}
     If the \meta{comma list} is empty, leaves the \meta{false code} in the
   input stream.  The value of the \meta{token list variable} is
   not defined in this case and should not be relied upon.  If the
   \meta{comma list} is non-empty, pops the top item from the
   \meta{comma list} in the \meta{token list variable}, \emph{i.e.}~removes
   the item from the \meta{comma list}. The \meta{comma list} is modified
   globally, while the \meta{token list variable} is assigned locally.
 \end{docCommand}

% \begin{function}
%   {
%     \clist_push:Nn,  \clist_push:NV,  \clist_push:No,  \clist_push:Nx,
%     \clist_push:cn,  \clist_push:cV,  \clist_push:co,  \clist_push:cx,
%     \clist_gpush:Nn, \clist_gpush:NV, \clist_gpush:No, \clist_gpush:Nx,
%     \clist_gpush:cn, \clist_gpush:cV, \clist_gpush:co, \clist_gpush:cx
%   }
 \begin{docCommand}{clist_push:Nn}{ \meta{comma list} \marg{items}}
   Adds the \marg{items} to the top of the \meta{comma list}.
   Spaces are removed from both sides of each item.
 \end{docCommand}
%
% \section{Using a single item}
%
% \begin{function}[added = 2014-07-17, EXP]
%   {\clist_item:Nn, \clist_item:cn, \clist_item:nn}
%   \begin{syntax}
%     \docAuxCommand*{clist_item:Nn} \meta{comma list} \Arg{integer expression}
%   \end{syntax}
%   Indexing items in the \meta{comma list} from~$1$ at the top (left), this
%   function will evaluate the \meta{integer expression} and leave the
%   appropriate item from the comma list in the input stream. If the
%   \meta{integer expression} is negative, indexing occurs from the
%   bottom (right) of the comma list. When the \meta{integer expression}
%   is larger than the number of items in the \meta{comma list} (as
%   calculated by \docAuxCommand*{clist_count:N}) then the function will expand to
%   nothing.
%   \begin{texnote}
%     The result is returned within the \tn{unexpanded}
%     primitive (\docAuxCommand*{exp_not:n}), which means that the \meta{item}
%     will not expand further when appearing in an \texttt{x}-type
%     argument expansion.
%   \end{texnote}
% \end{function}


}


\def\luadocs{%
  \part{LuaTeX}
  \parindent=1em
  \parindent1em


\chapter{Internationalization and Globalization}


\section{Introduction}

In this Chapter we discuss the requirements for localization of software and how this can be applied to \latex. In a way this chapter overlaps the one on languages. However, here we focus mostly on LuaTeX solutions. We also extend the discussion to calendric and solar calculations.

Internationalization is the process of designing a software application so that it can potentially be adapted to various languages and regions without engineering changes. Localization is the process of adapting internationalized software for a specific region or language by adding locale-specific components and translating text. Localization (which is potentially performed multiple times, for different locales) uses the infrastructure or flexibility provided by internationalization (which is ideally performed only once, or as an integral part of ongoing development).\index{internationalization}\index{globalization}

The development of routines for software internationalization and globalization has been an ongoing effort for many years. Currently the accepted method for building such software is the use of i18n. This is an abbreviation of the first letter and last letter of the word internationalization and the 18 is the number of characters in the word.

Internationalization based on i18n is not an easy task for \LaTeX. To an extend some of the issues have been removed with the use of Babel and Polyglossia that provide translation strings for many of the worlds scripts. The de facto resource for internationalization is the Unicode Consortium’s \href{http://cldr.unicode.org/}{CLDR} project.\index{i18n}

\section{Enforcing local styles}

To understand the magnitude of the problem let us look at some of the easier parts of localizing. Consider the Greek days of the week.
\medskip
\begin{trivlist}\item[]\panunicode
\begin{tabular}{llll}
\toprule
Day &Normal Form &Abbreviation &Narrow\\
Monday &Δευτέρα &Δευ. &Δ. \\
\midrule
\end{tabular}
\end{trivlist}

In Greek the abbreviated form, is always capitalized and a stop is provided. The same is true for the month. The narrow form can give problems, unless it is for calendars, where the content is clear. This is because "{\panunicode Π}" are the initials for both ``{\panunicode Πέμπτη}" (Thursday), and ``{\panunicode Παρασκευή}" (Friday). 

In date formats with long month format, that do not include the day, the full month form should be used.
In date formats with long month format, that also include the day, the long date format should be used.

If limited space is available, it is possible to omit the period in the abbreviated form of months, but this should be used only when there is a serious technical restriction

Ultimately, we are aiming at providing the necessary rules to build an automated style that can be used by the system.
                

\section{Locales}
\index{locale}

In computing, a \emph{locale} is a set of parameters that defines the user's language, country and any special variant preferences that the user wants to see in their user interface. Usually a locale identifier consists of at least a \textit{languag}e identifier and a \textit{region} identifier.

On POSIX platforms such as Unix, Linux and others, locale identifiers are defined similar to the BCP 47 definition of language tags, but the locale variant modifier is defined differently, and the character set is included as a part of the identifier. It is defined in this format: |[language[_territory][.codeset][@modifier]]|. (For example, Australian English using the UTF-8 encoding is en\_AU.UTF-8.)

For \latex these ``locales'' can be thought of as the settings of language keys through Babel and Polyglossia. These settings have served the community well for many years, but a litany of duct taping through other packages are a testimony to their limitations. Packages for dates, time and number formatting have been developed to assist. Here is my attempt to put the solution on a better footing and to start providing mechanisms via LuaTeX for a 'plugin'
architecture to find improve solutions. 

\section{Common Locale Data Repository}

The Common Locale Data Repository Project, is a project of the Unicode Consortium to provide locale data in the XML format for use in computer applications. CLDR contains locale specific information that an operating system will typically provide to applications. CLDR is written in LDML (Locale Data Markup Language). The information is currently used in International Components for Unicode, Apple's Mac OS X, OpenOffice.org, and IBM's AIX, among other applications and operating systems

\begin{enumerate}
\item Translations for language names.
\item Translations for territory and country names.
\item Translations for currency names, including singular/plural modifications.
\item Translations for weekday, month, era, period of day, in full and abbreviated forms.
\item Translations for timezones and example cities (or similar) for timezones.
\item Translations for calendar fields. This is useful especially in conjuction with PGF presentational forms.
\item Patterns for formatting/parsing dates or times of day.
\item Examplar sets of characters used for writing the language.
\item Patterns for formatting/parsing numbers.
\item Rules for language adapted collation. \label{collation}
\item Rules for formatting numbers in traditional numeral systems (like Roman numerals, Armenian numerals, ...).
\item Rules for spelling out numbers as words.
\item Rules for transliteration between scripts. A lot of it is based on BGN/PCGN romanization.
\item Rules for \emph{delimiters} such as quotations and question marks.
\end{enumerate}

Currently the consortium’s distribution make the data available in both json and xml formats.  These files hold data for a specific \emph{locale}. Sadly missing are any document sectioning information that would have enabled the incorporation of the above into LaTeX and overcoming some of the Babel and Polyglossia limitations.

We do not need many of the files provided by the CLDR unicode consortium and others we are missing. Take for example the |delimiters| file. 

\bgroup
\scriptsize
\begin{phdverbatim}
  "main" = {
    "ff": {
      "identity": {
        "version": {
          "_cldrVersion": "26",
          "_number": "$Revision: 10739 $"
        },
        "generation": {
          "_date": "$Date: 2014-08-07 12:54:13 -0500 (Thu, 07 Aug 2014) $"
        },
        "language": "ff"
      },
      "delimiters": {
        "quotationStart": "„",
        "quotationEnd": "”",
        "alternateQuotationStart": "‚",
        "alternateQuotationEnd": "’"
      }
    }
  }
}
\end{phdverbatim}
\egroup

Of course the |Json| format as it is, is not readable by Lua a format such as:

\begin{verbatim}
delimiters = {
        quotationStart = "«",
        quotationEnd = "»",
        alternateQuotationStart = "\"",
        alternateQuotationEnd = "\""
      }
\end{verbatim}

\begin{texexample}{i18n}{i18-1}

\panunicode
\begin{luacode*}
-- mock the delimiters from the json
-- file
greekname = 'el'
delimiters = {
        quotationStart = "«",
        quotationEnd = "»",
        alternateQuotationStart = [["]],
        alternateQuotationEnd = [["]]
      }
tex.print(delimiters.quotationStart .. 'test' .. delimiters.quotationEnd)
tex.print ([[\gdef\]] .. greekname .. [[quote#1{\directlua{tex.sprint(delimiters.quotationStart .. '#1' .. delimiters.quotationEnd)}}]])
\end{luacode*}

\def\elquote#1{%
  \directlua {tex.sprint(delimiters.quotationStart .. '#1' .. delimiters.quotationEnd)}
}
\end{texexample}



This is of course a much more simplified way of what one needs to program for a full system. The advantage
of producing the \tex definition also through LuaTeX is that we can keep all the code in one place and econd, we can avoid |\csname| costructs.
\begin{texexample}{elquote}{}
\elquote{This is some longer text in Greek quotes.}
\end{texexample}

I have opted to incorporate these files in the |json| format and provide routines for interfacing via the \pkgname{phd} package.  The reason for opting for a json format, is my other attempts to interface the package with |couchdb|.  My preference for a Nosql type of database, is that  they are better suited in handling data that is commonly  found in documents and also many of the routines will be interchangeable for web applications. I am also hoping that the collation information (see \ref{collation}), will eventually lead to better indices, a subject left untouched in the current distribution.\index{json}

\section{The package phd approach}

The package |phd| packge takes an approach to use only json resource files for the provision of language dependent information, rather than TeX commands alone, as is done by Babel and Polyglossia. 

\section{Language and Region Tags}
\index{tags>regions}\index{tags>language}

Languages are represented by tags such as "en"  for English or "el" for Greek. Other languages have no significant variation and are represented by a language subtag such as "en-US".  The names are mostly intuitive, but in many case bear no relationship to their English names, for example Armenian is coded as \textbf{hy}. There is a useful utility at the SIL website for viewing these codes.\footnote{\protect\url{http://www-01.sil.org/iso639-3/codes.asp?order=reference_name&letter=\%25}.} Note that the CLDR database does not cover all the languages listed in the ISO-639.\footcite{iso639} \index{ISO-639}

The language tags are based on the BGN which is mapped to languages based on ISO-639-1.

ISO 639-2 is the alpha-3 code in Codes for the representation of names of languages-- Part 2. There are 21 languages that have alternative codes for bibliographic or terminology purposes. In those cases, each is listed separately and they are designated as "B" (bibliographic) or "T" (terminology). In all other cases there is only one ISO 639-2 code. Multiple codes assigned to the same language are to be considered synonyms. ISO 639-1 is the alpha-2 code.

We will describe the tables using the English language, which is normally the default and Greek as a second language, as the script is distinctive enough to demonstrate their use. We will also explain Lua routines available via the \pkgname{phd} that are provided as alternatives to Babel and Polyglossia.

{layout.lua}

{layout.orientation.characterOrder} = |left_to_right| or |right_to_left|

layout.orientation.lineOrder = |top_to_bottom|

Example \ref{i18-1} loads the Greek internationalization file |layout| and prints the two fields. Before we send it to
the TeX typesetter we sanitize the string underscores using |gsub|. For illustration purposes we have used |gsub| both as an object method and as a function.

\begin{texexample}{i18n}{i18-1}
\begin{luacode}
local c = require("i18n.el.layout")
local s1 = string.gsub(c.el.layout.orientation.characterOrder, '_', '\\textunderscore ')
local s2 = c.el.layout.orientation.lineOrder:gsub('_', '\\textunderscore ')
tex.print('typeof :', type(c))
tex.print(s1, '\\par', s2)
\end{luacode}
\end{texexample}

Of course for Greek the above information is hardly necessary, but at the level of Lua programming, if we are automating the switching of text direction Greek text might signal a change in direction. Let us have another try using the same code for arabic text. All we have to change is the \textbf{el} to \textbf{ar}.

\begin{texexample}{i18n}{i18-2}
\begin{luacode}
local c = require("i18n.ar.layout")
local s1 = string.gsub(c.ar.layout.orientation.characterOrder, '_', '\\textunderscore ')
local s2 = c.ar.layout.orientation.lineOrder:gsub('_', '\\textunderscore ')
tex.print('typeof :', type(c), '\\par')
tex.print(s1, '\\par', s2)
\end{luacode}
\end{texexample}

In the next example we get the string for the first month of the year in the ``abbreviated'' style. I have changed the json
strings directly to Lua for this file to speed up processing.

\begin{texexample}{i18n}{i18-2}
\begin{luacode}
local c = require("i18n.el.cagregorian")
local months = c.el.dates.calendars.gregorian.months.formats
local days = c.el.dates.calendars.gregorian.days.formats

tex.print("\\begin{tabular}{ll} ")
for i=1,12 do
  tex.sprint(i.." &"..months.wide[i].."\\\\ ")
end
tex.print("\\end{tabular}")
\end{luacode}
\end{texexample}

Printing directly to the document has many benefits but does slow developemnt, both of the code as well as the document. Another distraction is transferring arguments from \tex to Lua and vice versa.

Similarly we can print the months in the Italian language by loading the \textbf{i18n.italian} module and iterating through the month strings. I am still thinking about the interface and the best way forward to provide an easy to use and remember interface. 


Let us now develop a longer example. We will load a number of languages and typeset a table for the different months.
Since we are running the example directly in the document, some patience is required. 

\bigskip 

\begin{texexample}{Month string in various languages}{ex:transl}
\bgroup
\parindent0pt
\newfontfamily\langtable{code2000}
\langtable
\scriptsize
\begin{luacode} 

c = require("i18n.irish")
d = require("i18n.russian")
e = require("i18n.latin")
f = require("i18n.german")
g = require("i18n.kannada")
h = require("i18n.lao")
j = require("i18n.turkish")
k = require("i18n.albanian")

local count=0

local months_irish = c.irish.months
local months_russian = d.russian.months
local months_latin = e.latin.months 
local months_german = f.german.months
local months_kannada = g.kannada.months
local months_lao    = h.lao.months
local months_turkish = j.turkish.months
local months_albanian = k.albanian.months
local centering = function()
                     tex.print("\\centering")
end

local par = function()
               tex.print("\\par")   
end

local tabular = function() 
	tex.print("\\begin{tabular}{clllllll}")
	tex.sprint("\\toprule")
end	


local endtabular = function()
	tex.print("\\bottomrule")
	tex.print("\\end{tabular}")
	tex.print("\\medskip")
end

local eol = function()
  return("\\\\")
end


-- center the table
centering()
tabular()
tex.sprint("Month","&Irish", "&Russian", "&Latin", "&Kannada", "&Lao","&Turkish","&Albanian", eol())
tex.sprint("\\midrule")
for i = 1,12 do
   count = i
   tex.sprint(i.."&", months_irish[i],
                 "&",months_russian[i], 
                 "&",months_latin[i], 
                 "&", months_kannada[i], 
                 "&"..months_lao[i], 
                 "&"..months_turkish[i],
                 "&"..months_albanian[i],
                 eol() )
end  
endtabular()
par()

\end{luacode} 
 
\egroup
\end{texexample}

Now some explanation for the code. We started by loading the necessary libraries for the languages that we wanted to print the month strings and allocated them to local variables.

We then iterated through the twelve months of the gregorian table and typeset them. We could have put the languages in a Lua table and iterated over them. I haven't done it so that the code is clearer. I tried to keep the API functions separate as much as possible. We also defined a font using \docAuxCommand{newfontfamily} of the \pkg{fontspec} package to ensure that we can print the Asian and Cyrillic scripts.

The long javanesque object notations make it difficult to work, but once they are set in functions and locals, development is fast. After the detour to explore the i18n tables and available information, we are now ready to tackle the production of multi-lingual calendars and to complete are library on internationalization. Before we do that a detour to understand
the complexity of calendrical calculations and some historical information is required.
\vfill




  \chapter{LuaTeX}
\addtocimage{-10pt}{-40pt}{./images/tocblock-lua.JPG}

\section{What is LuaTeX}

Although \tex and \latexe can be used to program complex commands, and the proof is this book, they are not  general programming languages. Many tasks that can easily be carried out
in other programming languages, are extrememely complicated or very slow when done with \tex. Due to this limitation many auxiliary programs have been developed to assist in common tasks, such as |bibtex| or |biber| that are used to build bibliographies and |makeindex| or |xindy| to generate indexes. In both cases, sorting a list alphabetically is a relatively easy task, but very complicated using \tex.


Another limitation is \tex's math capabilities. One can just marvel at the efforts of \tex macro authors in achieving what they have achieved. A cursory look at the code in the \pkgname{fp} package \footfullcite{fp} can give you an idea of the difficulties and complexities involved. The similarly named package by the \latex3 Team is another gigantic and amazing production.

To address the need to do more complex functions within \TeX, an extension of \TeX{} called Lua\TeX{} started a few years ago.  
(The leaders of the project and main developers are Taco Hoekwater, Hartmut Henkel and Hans Hagen.) The idea was to enhance \TeX{} with a previously existing general purpose programming language. After a careful evaluation of possible candidates, the language chosen was Lua (see \href{http://www.lua.org/}{lua.org}), a powerful, fast, lightweight, embeddable scripting language that has,  of course, a free license suitable to be used with \TeX. It is a pity that at the time |Go| was not available, as it would have been a perfect languge to interface with \tex and perhaps eventually rre-write the original program into a new language.

Lua is considered an easy to learn and use language. Anyone with basic programming skills can use it without difficulty. (Many examples of Lua code can be found in this book, and also in \href{http://rosettacode.org/wiki/Category:Lua}{rosettacode.com},
and \url{http://lua-users.org/}).

Lua\TeX{} is not \TeX{}, but an extension of \TeX{}, in the same way that pdf\TeX{} or \XeTeX{} are also extensions.
In fact, Lua\TeX{} includes pdf\TeX{} (it is an extension of pdf\TeX{}, and offers backward compatibility), 
and also has many of the features of \XeTeX.

Lua\TeX{} is now in a beta stage, but the current versions are usable (the first public beta was launched in 2007,
and when this paper was written on \today, the release used was version 1.\the\luatexversion). 

\section{Is LuaTeX for you?}

The choice of TeX engine should in a way be immaterial to you, if you are just thinking about typesetting normal books. If you are writing a thesis or you are involved with a longer and more complex document, \lualatex might be for you. 

\section{Embedding Lua in a TeX document}

Short Lua code can be embedded in a document fairly easily with 
\begin{docCommand}{directlua}{\marg{code}}
Although it is recommended to put Lua code in Lua files, from time to time one may want or need to go Lua in the middle of a document. To this end, LuaTeX has two commands: \cmd{\directlua} and \cmd{\latelua}. They work the same, except \cmd{\latelua} is processed when the page where it appears is shipped out, whereas \cmd{\directlua} is processed at once; the distinction is immaterial here, and what is said of \cmd{\directlua} also applies to \cmd{\latelua}.
\end{docCommand}

\section{From Lua to TeX}

There are a number of commands that can be used to pass code and values from Tex to Lua and vice-versa.

\begin{docCommand}{tex.print}{\meta{material}}
\end{docCommand}

Inside Lua code, one can pass strings to be processed by \TeX\ with the functions |tex.print()|, |tex.sprint()| and |tex.tprint()|. All such calls are processed at the end of a \refCom{directlua} call, even though they might happen in the middle of the code. This behavior is worth noting because it might be surprising in some cases, although it is generally harmless.

The function can also can also take an optional number as its first argument; it is interpreted as referring to a catcode table (as defined by \docAuxCommand{initcatcodetable} and \docAuxCommand{savecatcodetable}), and each line is processed by TeX with that catcode regime. For instance (note that with such a minimal catcode table, braces don't even have their usual values):

\begin{texexample}{catcodes}{lua:catcodes}
\directlua {tex.enableprimitives('',tex.extraprimitives()) }
\bgroup
\initcatcodetable1
\catcode`\_=0
\savecatcodetable1
\egroup
\directlua{tex.print(1, "_TeX")}
\luadirect{tex.print(1, "_TeX")}
\end{texexample}

The string will be read with |_| as an escape character, and thus interpreted as the command commonly known as |\TeX|. The catcode regime holds only for the strings passed to |tex.print()| and the rest of the document isn't affected.

If the optional number is -1, or points to an invalid (i.e. undefined) catcode table, then the strings are processed with the current catcodes, as if there was no optional argument. If it is -2, then the strings are read as if the result of \cmd{\detokenize}: all characters have catcode 12 (i.e. `other', characters that have no function beside representing themselves), except space, which has catcode 10 (as usual).


Note the |\directlua {tex.enableprimitives('',tex.extraprimitives()) }|, without this directive the extra primitives are not loaded, and the example will produce errors. this is rather hidden in the documention in version 0.\the\luatexversion running under \formatname.


\section{Checking for LuaTeX}

The first thing you will need to program, when developing a \luatex\ package, is to check that the engine is in fact using \luatex. This can be done using the \cs{ifluatex} from the package \pkgname{ifluatex}. 

\begin{texexample}{Checking for Lua}{}
\ifxetex
  Using XeTeX
\else
  \ifluatex
   Using LuaTeX
  \else
   Using LaTeX
  \fi
\fi
\end{texexample}


%---------------
\begin{figure}
\begin{luacode*}
-- Fourier series
function partial_sum(n,x)
    partial = 0;
    for k = 1, n, 1 do 
        partial = partial + math.sin(k*x)/k 
    end;
    return partial
end
-- Code to write PGFplots data as coordinates
function print_partial_sum(n,xMin,xMax,npoints,option)
    local delta = (xMax-xMin)/(npoints-1)
    local x = xMin
    if option~=[[]] then
        tex.sprint("\\addplot["..option.."] coordinates{")
    else
        tex.sprint("\\addplot coordinates{")
    end
    for i=1, npoints do
        y = partial_sum(n,x)
        tex.sprint("("..x..","..y..")")
        x = x+delta
    end
    -- We can write "};" and then it is not necessary to put ";" when used
    tex.sprint("}") 
end
\end{luacode*}
\newcommand\addLUADEDplot[5][]{\directlua{print_partial_sum(#2,#3,#4,#5,[[#1]])}}
\centering
\pgfplotsset{width=15cm, height=7cm}  
\begin{tikzpicture}\small
\begin{axis}[xmin=-0.2, xmax=31.6, ymin=-1.85, ymax=1.85, 
  xtick={0,5,10,15,20,25,30},
  ytick={-1.5,-1.0,-0.5,0.5,1.0,1.5},
  minor x tick num=4,
  minor y tick num=4,
  axis lines=middle,
  axis line style={-}
  ] 
%%\addplot[color=red] {pi/2-x/2};
% SYNTAX: Partial sum 30, from x = 0 to 10*pi, and sampled in 1000 points
\addLUADEDplot[color=blue,smooth]{30}{0}{10*math.pi}{1000};
\end{axis} 
\end{tikzpicture}
\caption{The partial sum $\sum_{k=1}^{30} \frac{\sin(kx)}{k}$ of the Fourier series  of $f(x)=(\pi-x)/2$ illustrating the Gibbs phenomenon.}\label{fig:Gibbs}
\end{figure}

\begin{figure}[htp]
\begin{scriptexample}[]{}
\begin{luacode*}
-- Fourier series
function partial_sum(n,x)
    partial = 0;
    for k = 1, n, 1 do 
        partial = partial + math.sin(k*x)/k 
    end;
    return partial
end
-- Code to write PGFplots data as coordinates
function print_partial_sum(n,xMin,xMax,npoints,option)
    local delta = (xMax-xMin)/(npoints-1)
    local x = xMin
    if option~=[[]] then
        tex.sprint("\\addplot["..option.."] coordinates{")
    else
        tex.sprint("\\addplot coordinates{")
    end
    for i=1, npoints do
        y = partial_sum(n,x)
        tex.sprint("("..x..","..y..")")
        x = x+delta
    end
    tex.sprint("}") -- We can write "};" and then it is not necessary to put ";" when used
end
\end{luacode*}
\newcommand\addLUADEDplot[5][]{\directlua{print_partial_sum(#2,#3,#4,#5,[[#1]])}}
\centering
\pgfplotsset{width=15cm, height=7cm}  
\begin{tikzpicture}\small
\begin{axis}[xmin=-0.2, xmax=31.6, ymin=-1.85, ymax=1.85, 
  xtick={0,5,10,15,20,25,30},
  ytick={-1.5,-1.0,-0.5,0.5,1.0,1.5},
  minor x tick num=4,
  minor y tick num=4,
  axis lines=middle,
  axis line style={-}
  ] 
%%\addplot[color=red] {pi/2-x/2};
% SYNTAX: Partial sum 30, from x = 0 to 10*pi, and sampled in 1000 points
\addLUADEDplot[color=blue,smooth]{30}{0}{10*math.pi}{1000};
\end{axis} 
\end{tikzpicture}
\begin{verbatim}
\begin{luacode*}
-- Fourier series
function partial_sum(n,x)
    partial = 0;
    for k = 1, n, 1 do 
        partial = partial + math.sin(k*x)/k 
    end;
    return partial
end
-- Code to write PGFplots data as coordinates
function print_partial_sum(n,xMin,xMax,npoints,option)
    local delta = (xMax-xMin)/(npoints-1)
    local x = xMin
    if option~=[[]] then
        tex.sprint("\\addplot["..option.."] coordinates{")
    else
        tex.sprint("\\addplot coordinates{")
    end
    for i=1, npoints do
        y = partial_sum(n,x)
        tex.sprint("("..x..","..y..")")
        x = x+delta
    end
    tex.sprint("}") -- We can write "};" and then it is not necessary to put ";" when used
end
\end{luacode*}
\end{verbatim}
\end{scriptexample}
\end{figure}

\section{Getting started}

For an introduction to the most important gotchas of \docAuxCommand{directlua}, see the guide \texttt{lualatex-doc.pdf}, which is available at ctan or possibly in your distribution.

Before presenting the tools in this package, let me insist that the best way to manage a nontrivial piece of Lua code is preferable to use an external file and source it from Lua, as explained in the cited document.

\LuaTeX communicates with TeX with only a few commands. We will explore these commands first.

\subsection{Basic print statements}

\begin{teXXX}
\directlua{tex.print}
\end{teXXX}

To print you can use the Lua command \lstinline{tex.print}.

The package \pkg{luacode}\cite{luacode} provides some further environments and commands catering for
variations in catcodes. 



The |\directlua| is useful to call functions that have been defined using Lua.

\begin{texexample}{luacode}{ex:luacode}
\begin{luacode}
function LineBreaks(T1,T2)
 local a = T1 .. "\\\\" .. T2 .. "\\par" -- T1\\\\T2\\par
 local b = [[ T1\\T2 ]]                  -- "T1"\\\\"T2"
 local c = T1 .. "\\newline " .. T2      -- T1\\newline T2
 local d = [[ T1\string\newline T2 ]]    -- "T1"\\newline "T2"
 tex.print(a,b,c,d)
 end
\end{luacode}

\def\testlinebreaks#1#2{%
\bgroup\parindent=0pt
    \directlua{LineBreaks("#1","#2")}%
\egroup}
\testlinebreaks{one}{two}
\testlinebreaks{three}{four}
\end{texexample}

The code in the example ensures that the code is escaped properly firstly for \tex as it will try and expand the
strings and then also not to allow any strings that need escaping in Lua. This is a double edge sword that caught me a number of times.


Lua can handle subtraction, negative numbers, numbers with decimal points, multiplication (using *),
division (using /), exponentiation (using |^|), and combinations of these. Here are some examples:
\emphasis{directlua, tex,print}


\section{Numbers}

The number type represents real (double-precision floating-point) numbers. Lua has no integer type, as it does not need it. There is a widespread misconception about floating-point arithmetic errors and some people fear that even a simple increment can go weird with floating-point numbers. The fact is that, when you use a double to represent an integer, there is no rounding error at all (unless the number is greater than 100,000,000,000,000). Specifically, a Lua number can represent any long integer without rounding problems.
We can write numeric constants with an optional decimal part, plus an optional decimal exponent. 

Examples of valid numeric constants are:
\begin{verbatim}
    4     0.4     4.57e-3     0.8e12     5e+20
\end{verbatim}

\begin{texexample}{Formatting numbers}{ex:fmtn}
\begin{luacode}
local str = string.format("%.2f",0.8e12)
print(str)
\end{luacode}
\end{texexample}



\subsection{Scientific notation}

You can also write numbers using the \textit{scientific notation}, where the prt before the upper multiplied by 10 to the power after $\textup{e}$

\begin{texexample}{Numbers scientific notation}{ex:scientific}

Some big numbers \directlua{tex.print(1.2193263111264E17 / 987654321)}

\end{texexample}


\section{Hexadecimal numbers}
Lua also understands hexadecimal (base 16) numbers, using the letters
a-f (or A-F) to represent 10 through 15. Hexadecimal numbers should start with \texttt{0x} or \texttt{0X}.

\begin{texexample}{Hexadecimal numbers}{ex:hex}
\directlua{tex.print(0XA)}
\end{texexample}


\section{Lua variables}

Lua variables are similar to those used by other computer languages, such as Python, Javascript and similar modern languages. They are untyped, which means they can easily produce subtle errors.

Programmers generally choose names for their variables that are meaningful—they document what the variable is used for.

Variable names can be as long as you like. They can contain almost any characters  but they can’t begin with a number. It is legal to use uppercase letters, but it is conventional to use only lower case for variable names.

\begin{texexample}{Lua Variables}{}
\luadirect{
  A = 2^0.5
  B = 42 
 local  pi = 3.141617
 tex.print(A+B-pi)
}
\end{texexample}

Variables unless they are preceded with a |local| are in the document global space. We can continue our example from above as we did not define A and B again, they are available in the next example.

\begin{texexample}{Lua Variables}{}
\luadirect{
   tex.print(A+B)
   
   local A,B = 1,1
   tex.print(A+B) 
}
\end{texexample}

\begin{texexample}{Lua Variables}{}
\luadirect{
   tex.print(A+B)
   
   local A,B = 1,1
   tex.print(A+B) 
}

\def\abadd#1#2{%
  \luadirect{
     A = #1 + #2
     tex.print(A)}
}

\abadd{100}{925.56}
\end{texexample}


Lua is case sensitive, which means |NUM|, |Num| and |num| are different variables.

\begin{texexample}{Lua Variables case sensitivity}{}
\directlua{
   NUM=1
   Num=2 
   num=3
   tex.print(NUM+Num+num)
}
\end{texexample}


\chapter{Statements}

Lua supports an almost conventional set of statements, similar to those in C,
Pascal or Go. Conventional statements include assignment, control structures,
and procedure calls. Lua also supports some not so conventional statements,
such as multiple assignments and local variable declarations

\section{Assignment}

Assignment is the basic means of changing the value of a variable or a table field.
\begin{verbatim}
a = "hello".."Lua"
b = 100.231
\end{verbatim}


\subsection{Multiple Assignment}

You can assign multiple values to multiple variables at the same time. You can also print multiple values
at the same time. The comma is used for both. Here's how you use the comma for multiple assignment, and for printing multiple values at the same time:

\begin{texexample}{Another example}{}
\directlua{
  Product, Exponent = 10*10, 999
  A,B=1,2
  tex.print(Product, Exponent)
  tex.print(A,B)
}
\directlua{
    A,B=1,2
    tex.sprint("A=" .. A, "B =" .. B)
}
\end{texexample}


\begin{texexample}{Multiple assignments}{ex:massign}
\begin{luacode}
local x1, x2, x3 = 16, 50, 33
local sum = x1^3 + x2^3 + x3^3
tex.print("sum = " .. sum)
\end{luacode}
\end{texexample}

Lua can be very useful in calculating and presenting the results in a nice way. In (\ref{ex:massign2}) we add additional \luacmd{tex.print} lines to calculate the sum.
Remember that we need to escape all backslasheses using (|\\|).

\begin{texexample}{Multiple assignments}{ex:massign2}
\begin{luacode}
local x1, x2, x3 = 16, 50, 33
local sum = x1^3 + x2^3 + x3^3
tex.print("\\[")
tex.print("16^3 + 50^3 + 33^3 = "..sum)
tex.print("\\]")
\end{luacode}
\end{texexample}
If you notice that the sum digits are made up of the digits of the numbers we have cubed.

Let us practice a bit more and get the next number in the series.

\begin{texexample}{Multiple assignments}{ex:massign2}
\begin{luacode}
local x1, x2, x3 = 16, 50, 33
local sum = x1^3 + x2^3 + x3^3
local sum1 = 166^3 + 500^3 + 333^3 
local sum2 = 1666^3 + 5000^3 + 3333^3
local sum3 = 16666^3 + 50000^3 + 33333^3
tex.print("\\begin{align}")
tex.print("16^3 + 50^3 + 33^3 &= "..sum.."\\\\")
tex.print("166^3 + 500^3 + 333^3 &= "..sum1.."\\\\")
tex.print("1666^3 + 5000^3 + 3333^3 &= "..sum2.."\\\\")
tex.print("16666^3 + 50000^3 + 33333^3 &= "..string.format("%.0f",sum3))
tex.print("\\end{align}")
\end{luacode}
\end{texexample}

The above pattern can be extended forever and the proof is available in 34. Writing it this way is not very efficient. In the next chapter we will write a function to make it easier to typeset and calculate these numbers. If you notice the last answer is in scientific notation. 

Multiple assignments are not faster than the equivalent single assignments. Sometimes they can be useful for swapping values such as |x, y = y, x|. A more frequent use is to collect multiple returns from function calls. For instance, in the assignmen |x,y = f()| the call to |f| returns two results: |a| gets the first and |b| is assigned the second.

\section{Local Variables and Blocks}

Lua provides a small and conventional set of control structures, with if for conditional
execution and while, repeat, and for for iteration. All control structures
have an explicit terminator: |end| terminates |if|, |for| and |while| structures; |until|
terminates |repeat| structures.

The condition expression of a control structure can result in any value. Remember
that Lua treats as true all values different from false and nil. (In
particular, Lua treats both zero and the empty string as true.)


\section{Conditionals}



\subsection*{if then else}

The \textit{if–then} construct (sometimes called \textit{if–then–else}) is common across many programming languages.
 
\begin{texexample}{Conditionals}{ex:conditionals}
\directlua{
 local a=3
 local b=12^2
 if a<b then 
   tex.sprint("$a<b^2$ "..a..", "..b) 
 else
 end
}
\end{texexample}

To write nested \textbf{if}s you can use \textbf{elseif}. It is similar to \luacmd{else} followed by \luacmd{if}, but it avoids the need for multiple \luacmd{end}s.

Lua does not have a switch statement and is common to find \luacmd{elseif} in expressions.

To test whether a variable is not |nil| in a conditional, it is terser to just write the variable name rather than explicitly compare against |nil|. Lua treats |nil| and |false| as |false| (and all other values as |true|) in a conditional:

\begin{texcode}{Idiomatic Lua conditionals}{ex:lua-condinional}
local line = io.read()
if line then  -- instead of line ~= nil
  ...
end
...
if not line then  -- instead of line == nil
  ...
end
\end{texcode}

However, if the variable tested can ever contain false as well, then you will need to be explicit if the two conditions must be differentiated: line == nil v.s. line == false.

and and or may be used for terser code:

\begin{texexample}{idiomatic Lua conditionals}{ex:cond}
\luadirect{
local function test(x)
  x = x or "idunno"
    -- rather than if x == false or x == nil then x = "idunno" end
  tex.print("YES!")
    -- rather than if x == "yes" then tex.print("YES!") else tex.print(x) end
end
  test()
}
\end{texexample}


\subsection*{while}

As the name implies, a \luacmd{while} loop repeats its body while a condition is true. As
usual, Lua first tests the while condition; if the condition is false, then the loop
ends; otherwise, Lua executes the body of the loop and repeats the process.

\begin{texexample}{while loop}{ex:loop}
\begin{luacode}
   counter = 1
   while counter <= 100 do
      tex.print(counter)
      counter = counter + 1
   end
\end{luacode}
\end{texexample}


\let\exec\directlua


\section{for}

The |for| statements has two variants: the \emph{numeric} |for| and the \emph{generic} |for|. The 
numeric |for| has the following syntax:
\begin{verbatim}
for var = exp1, exp2, exp3 do
  code...
end  
\end{verbatim}

This loop will execute the code for each value of |var| from from |exp1| to |exp2|, using
|exp3| as the step to increment |var|. This third expression is optional; when
absent, Lua assumes 1 as the step value. A typical example is shown in Example~\ref{ex:forloop}

\begin{texexample}{for loop}{ex:forloop}
\begin{luacode}
for n = 1, 59 do
   tex.print("\\(\\overline{"..n.."}\\), ")
end
\end{luacode}
\end{texexample}

The example is interesting, as it shows a notation used sometimes in Mathematical texts to denote numbers in sexagesimal systems. A sexagesimal system is an numeral system based on a base of 60, originally used by the Sumerians and then became known as Babylonian numbers. We still use it today to tell time and measure angles. The |\overline|, is just a notation to show it is not a decimal number but something else. In this case it represents one of the digits of a base 60 numeral system. More modern texts do not use the overline but simply the number. 

Let us print the example, this time grouping the numbers in units of 12. We will use the mod|%| operator and simply print a linebreak.

\begin{texexample}{for loop}{ex:forloop}
\begin{luacode}
for n = 1, 59 do
   tex.print("\\(\\overline{"..n.."}\\), ")
   if n%12==0 then 
      tex.print("\\\\")
   end 
end
\end{luacode}
\end{texexample}


\subsection{repeat until}

As the name implies, a repeat–until statement repeats its body until its condition
is true. This statement does the test after the body, so that it always
executes the body at least once.

\let\exec\directlua

\begin{texexample}{repeat until}{ex:repeat0}
\begin{luacode}
local  n, z = 1, 21
local  f = 1
  repeat
    tex.print(n + z)
    n = n + 1
    f = f + 10
  until f>=99
\end{luacode}
\end{texexample}

Unlike in most other languages, in Lua the scope of a local variable declared
inside the loop includes the condition:

\begin{texexample}{repeat-until}{ex:repeat1}
\begin{luacode}
local x = 19.7899
local sqr = x/2
local count = 0
repeat
   sqr = (sqr + x/sqr)/2
   local error = math.abs(sqr^2 - x)
   count = count + 1
   print("count = "..count.." square root = "..sqr.." error = "..error)
until error < x/10000 or count > 100-- local 'error' still visible here
\end{luacode}
\end{texexample}


\subsection{The break and do Statements}
\begin{texexample}{Break and do}{ex:breakanddo}
\exec{
for N = 1, 10 do
 if N > 5 then
 break
 end
 tex.print(N)
 end
}
\end{texexample}

\chapter{Functions}

\cxset{epigraph width=0.6\textwidth,
       epigraph text align=left}
\epigraph{“A computer is like a violin. You can imagine a novice trying fist a phonograph and then a violin. The latter, he says, sounds terrible. That is the argument we have heard from our humanists and most of our computer scientists. Computer programs are good, they say, for particular purposes, but they aren't flexible. Neither is a violin, or a typewriter, until you learn how to use it.” }{Marvin Minsky}


\section{What are functions}

The main mechanism for abstraction of statements and expressions in Lua are \textit{functions}. Function carry out specific tasks. In other languages sometimes they are called a \textit{procedure} or a \textit{subroutine}. Functions are first class citizens in Lua. They can be used both as statements or as expressions. All functions in Lua are anonymous. This is not immediately clear in the standard syntax for defining a function
\begin{verbatim}
function add (x,y)
  return x + y
end  
\end{verbatim}
Nevertheless, this syntax is just syntactic sugar for an assignment of an anonymous function to a variable\footnote{Since Lua version 3.1}
\begin{verbatim}
add = function (c, y)
  return x + y
end  
\end{verbatim}
That is, the code creates an anonymous function and assigns it to the global
variable add. Therefore, all functions in Lua are anonymous, like in Scheme.
A \enquote{name} for a function is actually the name for a variable that holds that
function.

\begin{texexample}{Functions used in expressions}{ex:expressions}
\begin{luacode}
  a = math.sin(2) + math.cos(10)
  tex.print(a)
\end{luacode}
\end{texexample}

The arguments are enclosed in parentheses denoting a call. If the function does not have any arguments, we must still call it with an empty list () to denote the call.

Let us return to our example with the numbers. Using a function we will try and generalize the problem.

Given three integer values typeset a curious number.

\begin{texexample}{Curious Numbers}{ex:curfunction}
\begin{luacode}
function curious(x1, x2, x3)
   local sum = x1^3 + x2^3 + x3^3
   local str = tostring(x1).."^3 + "..tostring(x2).. "^3 + ".. 
               tostring(x3) .."^3& =".. 
               string.format("%.0f",sum) 
   return str
end

print("\\begin{align}")
print(curious(1666,5000,3333))
print("\\end{align}")
\end{luacode}
\end{texexample}


\section{Fixed Parameters}

The default convention for passing parameters to a function is using fixed parameters. These tell
the function how many parameters there will be and the order of the parameters. When using fixed
parameters , you know what to expect from the function, and based on the values of the parameters
passed, you can take appropriate action.

\begin{texexample}{An example}{}
\begin{luacode}
print=tex.print
function listParams(one, two, three)
print( "We got :", one, two, three)
if one=="Q" then
  tex.print("\\textls{The letter Q}")
else
  tex.print("Other letter")
end
end
listParams("Q", "Q", "R")
\end{luacode}
\end{texexample}


\section{Multiple Results}

An unconventional, feature of Lua is that functions can return multiple results. Many predefined functions in Lua return multiple values. 

Example \ref{ex:functions} has two parameters and returns the average of these numbers.

\begin{texexample}{Functions}{ex:functions}
\exec{
function Average(Num1, Num2)
    return (Num1 + Num2) / 2
end
tex.print(Average(10,12))
}
\end{texexample}

Here is one that prints the Ackerman function. This is an interesting function as the value grows very quickly and can cause a stack overflow. It can also slow compilation quite a bit for higher values. ack(4,1) will cause a stack overflow. Whereas ack(3,11) compiles giving a value of 16381 ack(3,12) causes a stack overflow.  

\[
\begin{matrix}
   a\uparrow\uparrow b & = {\ ^{b}a}  = & \underbrace{a^{a^{{}^{.\,^{.\,^{.\,^a}}}}}} & 
   = & \underbrace{a\uparrow (a\uparrow(\dots\uparrow a))} 
\\  
    & & b\mbox{ copies of }a
    & & b\mbox{ copies of }a
  \end{matrix}
\]

\[
\begin{bmatrix}
   a\times b & = & \underbrace{a+a+\dots+a} \\
   & & b\mbox{ copies of }a
\end{bmatrix} 
\]


\begin{texexample}{The ackerman function}{ex:ackerman}
\exec{
function ack(M,N)
    if M == 0 then return N + 1 end
    if N == 0 then return ack(M-1,1) end
    return ack(M-1,ack(M, N-1))
end
tex.sprint(ack(3,11))
}
\end{texexample}


\subsection{Returning multiple values}

This return an error if Arg1 is a number

\begin{texexample}{Return multiple arguments}{ex:multargs}
\exec{
function ReturnArgs(Arg1, Arg2, Arg3)
  return Arg1, Arg2, Arg3
end
tex.sprint(ReturnArgs("aaaa", 2, 3))
   
function Fact(N)
  local Ret = 1
  for I = 1, N do
    Ret = Ret * I
  end
  return Ret
  end
tex.print(Fact(10))
}
\end{texexample}


\begin{texexample}{Factorial}{ex:luafac}
\exec{
function Fact(N)
  local Ret = 1
  for I = 1, N do
    Ret = Ret * I
  end
  return Ret
  end
tex.print(Fact(10))
}
\end{texexample}

\subsection{Anonymous Functions}

Anonymous functions do not have a function name and can be assigned to a variable. 

\begin{texexample}{Anonymous Functions}{ex:anonymous}
\exec{
(function(A, B)
   tex.print(A + B^2)
 end)(2, 3)
}
\end{texexample}

\section{Using Library Functions and Methods}

\begin{texexample}{Big Numbers}{ex:bignum}
\begin{luacode}
require("BigNum")
ret = BigNum.new("1")
for i=1, 60 do
  ret = ret*i
end
local res = BigNum.mt.tostring(ret)

local str,token = "",""
for token in string.gmatch(res, (".")) do
   str = str..token.."\\hskip1sp "
end

print("63! = "..str)
\end{luacode}
\end{texexample}

Palindromic numbers that read the same both ways (left-to-right and right-to-left), 
e.g., 12321 or 9210129 are number theory curiosities.

All numbers in any base with one digit are palindromic. The number of palindromic numbers with two digits is nine.

\begin{gather}
\{11,22,33,44,55,66,77,88,99\}
\end{gather}

Let us find some of them.

\begin{texexample}{Palindromic Numbers}{ex:pal}
\begin{luacode}
require("BigNum")

function pallindrome(n,topower)
  local n1 = BigNum.new(n)
  n2 = n1^topower
  return n2
end


tpallindrome3 = {101,111,10101,11011,110011,1001001,2201}
tpallindrome2 = {26,264,307,836} 

print("\\begin{align}")
for _,v in pairs(tpallindrome2) do
  pl = pallindrome(v,2)
   print(v.."^".."2".." &= "..BigNum.mt.tostring(pl).."\\\\")
end 
for k,v in pairs(tpallindrome3) do
  pl = pallindrome(v,3)
  if next(tpallindrome3,k)==nil then
    str=""
  else
    str="\\\\"
  end  
  print(v.."^".."3".." &= "..BigNum.mt.tostring(pl)..str)
end 
print("\\end{align}")

-- largest non-palindromic number whose square is a palindrome
local lp = pallindrome("831775153121251039203514",2)
print(lp)

\end{luacode}
\end{texexample}

The construction
\begin{verbatim}
if next(tpallindrome3,k)==nil then
    str=""
  else
    str="\\\\"
  end 
\end{verbatim}
is to ensure that we do not have a carriage return |\\| at the last equation of align. 

The number 2,201 is truly amazing as it is the only non-palindromic number that its cube is palindromic and is shown in our example as having the value \num{10662526601}. This number appeared in a number of Martin Gardner's books.





\section{First Class Functions}

A language has first-class functions if it can do for each of the following without recursively invoking a compiler or interpreter. In languages with first-class functions, the names of functions do not have any special status; they are treated like ordinary variables with a function type.[3] The term was coined by Christopher Strachey in the context of \enquote{functions as first-class citizens} in the mid-1960s.

\begin{enumerate}
\item Create new functions from pre-existing functions at run-time.
\item Store functions in collections.
\item Use functions as arguments to other functions.
\item Use functions as return values of other functions.
\end{enumerate}

\section*{Primes}

Continuing our exploration of Lua and number theory we will use a longer example using \textit{prime numbers}. My basic reference for this is Stewart's \textit{Theory of Numbers}\footcite{stewart1964}.

In 1876 Lucas was able to show that the number $M_{127}$ is a prime and for a very long time it remained the largest number to be a prime. For more extended numerical investigations in pre-computer era two books by D.N.Lehmer provded
tables of primes to around 10,000,000.00.\footnote{\textit{Factor table for the first ten millions containing the smaller factor of every number not divisible by 2,3,5 and 7 between the limits 0 and 10,017,000.} Carnegie Institution of Washington Publication 105, 1909.\par \textit{List of prime numbers from 1 to 10,006,721.} Carnegie Institution of Washington Publication 165, 1914.} Useful  as they were, such tables are, of course, inadequate to handle problems like the one proposed by Lucas 1867.

In Example~\ref{ex:primesmore} we will first write a function that formats a number with commas and then calculate 
M127.

\begin{texexample}{Lucas M127}{ex:primesmore}
\begin{luacode}
function formatnum(n) -- credit http://richard.warburton.it
	local left,num,right = string.match(n,'^([^%d]*%d)(%d*)(.-)$')
	return left..(num:reverse():gsub('(%d%d%d)','%1,'):reverse())..right
end

local n1 = BigNum.new(2)
local bnum = BigNum.mt.tostring
n2 = n1^127-1
print(formatnum(bnum(n2)).."\\\\")

str = string.format("%.0f", tostring(2^127))
print(formatnum(str))
\end{luacode}
\end{texexample}

The most widely used method for preparing alist of prime numbers less than a given limit is a device ascribed to Eratosthenes (276-194 B.C.). One starts  by writing down a list of the integes from 2 to that limit and then in a sytematic way eliminating all the composite integers. The method as described by Stewart is as follows:

\begin{quotation}
For example, with a limit of n = 100, we first set down a list of the 
integers from 2 to 100. Recognizing that 2 is a prime, but that all proper 
multiples of 2 are composite, we cross out 4,6,8,..., 100. The next number 
not crossed out is 3, which must be a prime for the only possible proper 
factor is 2, and 3 is not a multiple of 2 else it would have been crossed out. 
Recognizing that all proper multiples of 3 are composite, we cross out 
6, 9, 12,..., 99—although it is not actually necessary to cross out 6, 12, 
18,..., 96 again, since they are already crossed out, being multiples of 2. 
The next number not crossed out is 5; this number must be a prime, for if 
it were composite, it would have to have as a proper factor a prime less 
than 5, namely, either 2 or 3; but since 5 is not crossed out, it is not a 
multiple of 2 or 3. Crossing out all multiples of 5, not previously crossed 
out, namely: 25, 35, 55, 65, 85, 95, we find, by the same reasoning as 
before, that the next number not crossed out must be a prime; it is 7. 
The only multiples of 7, not previously crossed out, are 49, 77, 91, and 
these we now cancel. Now, unless we have been analyzing the sieve process 
carefully, we are due for a surprise—all the remaining numbers which 
have survived the sieve are primes! The sieve appears as follows: 
\end{quotation}

Our first concern is to first develop functions to typeset such tables and once the
typesetting is out of the way to then implement the sieve. In current computer terminology
a sieve is a filter.

We will represent a single number in this table as a tuple (n, f) where $f$ is a boolean indicating if the
number is to be crossed out or not.

\begin{texexample}{Eratosthenes Sieve I}{ex:eratosthenes}
\begin{luacode}
local T = {}
local num = {}
function eratosthenes (n)
   for i=1,n do
     num[i]=false
     table.insert(T,num)
   end
end

eratosthenes(100)

function print_eratosthenes()
  for k,v in pairs(T) do
    if k>1 then print("\\mbox{"..k.."} ") 
    else
      print("\\mbox{~~~} ")
    end
    if k%20==0 then print("\\\\") end
  end
end  
print_eratosthenes()
\end{luacode}
\end{texexample}

Our first skeleton program is shown above and consists of two functions, |eratosthenes| and |print_eratosthenes|. The first at this stage just initializes the table with all the elements having a false value, and to the second function we delegate the printing. The result did not come out very well due to using |mbox|. It will be preferable to use a fixed width box or we could use a LaTeX tabular environment or even draw the cells using \tikzname or use a packae such as tcolorbox. At this stage we also need to decide if we need to use a strike to indicate the numbers eliminated or simply use color. It would be better to rather write anothe function to decorate the cell and provide an option.

In the next example we add the function |mbox| and also we use the \pkg{cancel}\footcite{cancel} to strike through the
numbers. This will produce identical results to the previous example with all the numbers cancelled. In the final
example we will implement the sieve.

\begin{texexample}{Sieve}{ex:sieve2}
\begin{luacode}
local T = {}
local num = {}
function eratosthenes (n)
   for i=1,n do
     num[i]=true
     table.insert(T,num)
   end
end

eratosthenes(100)

function mbox(n,tf)
  local nstr = "$"..tostring(n).."$"
  if tf then nstr="$\\cancel{"..n.."}$" end
  return "\\mbox{"..nstr.."}"
end

function print_eratosthenes(f)
  for k,v in pairs(T) do
    if k>1 then print(f(k,v[1])) 
    else
      print("\\mbox{~~~} ")
    end
    if k%20==0 then print("\\\\") end
  end
end  


print_eratosthenes(mbox)
\end{luacode}
\end{texexample}

Note than |print_eratosthenes| now takes one argument |f| which is a function.


\begin{texexample}{Sieve}{ex:sieve2}
\begin{luacode}
local T = {}
local num = {}
function eratosthenes (n)
   for i=1,n do
     num[i]=true
     table.insert(T,num)
   end
end

eratosthenes(100)

function mbox(n,tf)
  local nstr = "$"..tostring(n).."$"
  if tf then nstr="$\\cancel{"..n.."}$" end
  return "\\mbox{"..nstr.."}"
end

function tcbox(n,tf)
  local nstr = "$"..tostring(n).."$"
  if tf then nstr="$\\cancel{"..n.."}$" end
  return "\\tcbox[nobeforeafter,boxsep=0.5pt,boxsep=0.5pt,right=1pt,left=1pt,width=3em,]{\\hbox to 1.5em{"..nstr.."}}"
end

function print_eratosthenes(f)
  for k,v in pairs(T) do
    if k>1 then print(f(k,v[1])) 
    else
      print(f(1,v[1]))
    end
    if k%20==0 then print("\\\\") end
  end
end  


print_eratosthenes(mbox)

print_eratosthenes(tcbox)
\end{luacode}
\end{texexample}


On my computer in takes about 3s to compute the primes up to 10 million and 55s to compute to 100 million.

\section*{Mersenne Numbers}

A number is called a Mersenne number if it is in the form:

\begin{gather}
 M_p = 2^p-1
\end{gather}
where $p$ is a prime . If a Mersenne number $M_p = 2^p-1$ is a prime. then it is
called a \textit{Mersenne prime}.



\section{Summary}




\chapter{Working with Lua Tables}

This section explores a new data type called a table. It's a data structure, which means that it lets
you combine other values. Because of its flexibility, it is Lua’s only data structure. (It is possible to
create other, special-purpose data structures in C.)
\begin{phdverbatim}
\begin{filecontents*}{test.lua}
  hash = {}
  hash[ "key-1" ] = "val1"
  hash[ "key-2" ] = 1
  hash[ "key-3" ] = {}
  tex.sprint(hash["key-2"])
\end{filecontents*}
\directlua{dofile("test.lua")}
\end{phdverbatim}


\section{Using dofile}

Although using \lstinline{directlua} has been a useful function so far, this is limiting in many respects.

\begin{verbatim}
\begin{filecontents*}{test.lua}
 function test(N)
  return N
end
tex.print(test(5))
\end{filecontents*}
\end{verbatim}

\begin{texexample}{Working with lua file}{ex:dofile}
 function test(N)
   return N
 end
 tex.print(test(5))
\end{texexample}

As we want to keep ourselves within a document, rather than a separate file, we
use the \verb+filecontents+ package to save a lua file on disc and then use it with
\lstinline{dofile}.

\begin{phdverbatim}
\begin{filecontents*}{test.lua}
hash = {}
hash[ "key-1" ] = "val1"
hash[ "key-2" ] = 1
hash[ "key-3" ] = {}
hash[" key-3"] = {}
\end{filecontents*}
\directlua{dofile("test.lua")}
\end{phdverbatim}

The keys can include spaces. If you have to automate the key generation, ensure that the keys are trimmed.

\directlua{dofile("test.lua")}
%\begin{tcblisting}{}
%\begin{filecontents*}{test.lua}
%Squares = {} -- A table constructor can be empty.
% for I = 1, 5 do
%   Squares[I] = I ^ 2
% end
% for I = 1, 5 do
%   tex.sprint(I .. " squared is " .. Squares[I].."\\\\")
% end
%\end{filecontents*}
%\directlua{dofile("test.lua")}
%\end{tcblisting}
%
%
%\begin{tcblisting}{}
%\begin{filecontents*}{test01.lua}
% NameToInstr = {John = "rhythm guitar",
% Paul = "bass guitar",
% George = "lead guitar",
% Ringo = "drumkit"}
% for Name, Instr in ipairs(NameToInstr) do
%   tex.sprint(Name)
% end
%\end{filecontents*}
%\directlua{dofile("test01.lua")}
%\end{tcblisting}


Using tables that contain functions is a handy way to organize functions, and Lua keeps many of its
built-in functions in tables, indexed by strings. For example, the table found in the global variable table
contains functions useful for working with tables. If you assign another value to table, or to one of the
other global variables used to store built-in functions, the functions won’t be available anymore unless
you put them somewhere else beforehand. If you do this accidentally, just restart the interpreter



\subsection{table.sort}
\begin{texexample}{Sorting}{lua:sort}
\begin{luacode}
Names = {"Scarlatti", "Telemann", "Corelli", "Purcell",
   "Vivaldi", "Handel", "Bach","Yiannis"}
table.sort(Names)
 for I, Name in ipairs(Names) do
   tex.sprint(I.." ".. Name.."\\par")
 end
\end{luacode}
\end{texexample}


%\endinput
%
%\subsection{table.concat}
%\subsection{table.sort}
%\begin{texexample}{Lua Sort}{}
%\begin{luacode}
%require "lualibs"
%-- require "luasql.mysql"
%-- commas (and spaces):
% function CommaSeparate(Arr)
%   return table.concat(Arr, ", ")
% end
% tex.print(CommaSeparate({"a", "bc", "d"}))
% FileStr = os.tmpname()
%    tex.sprint("\\string"..os.tmpname().." \\% temporary file name")
% Hnd = io.open(FileStr, "w")
% if Hnd then
%   tex.sprint("Opened temporary file ", FileStr, " for writing\n")
%   Hnd:write("Line 1\nLine 2\nLine 3\n")
%Hnd:close()
%for Str in io.lines(FileStr) do
%  io.write(Str, "\n")
%end
%  os.remove(FileStr)
%else
%   io.write("Error opening ", FileStr, " for writing\n")
%end
%\end{texexample}



\chapter{Packages}
The \pkg{luacode} package provides commands to make entering of code from a \TeX
file easier. It provides a better escape mechanism

\emphasis{luaexec,luacode}
\begin{texexample}{luaexec}{ex:exec}
\def\foo{156789.0001}
\(
 \luaexec{
 texio.write_nl("Special chars: _ ^ & $ { } ~ working.\string\n"
 .. "Backslashes still need a bit of care.\string\n"
 .. "Single sharps are easier now: \#")
  % a tex comment: we also get a % below
 tex.sprint("\\pi \\neq ", tostring(math.pi):gsub('\%.', '+'))
 % we can use TeX macros
tex.sprint("-", math.sqrt(\foo))
 }
\)
\end{texexample}


\section{Using \texttt{luacode}}

One very useful environment is \refEnv{luacode}.

\begin{docEnvironment}{luacode}{}
The luacode environment is similar to luaexec. You need to be careful to pass a macro that is fully expandable in order to work properly.
\end{docEnvironment}

\begin{texexample}{luacode environment}{ex:luacode}
\edef\foo{156789.0001}
\[
 \begin{luacode}
    tex.sprint("a=", math.sqrt(\foo),"+ z^3")
 \end{luacode}
\]
\end{texexample}

\section{Creating new environments}
\emphasis{luacode,endluacode,tex}
If you need to create new environment you will have to use the \lstinline!\luacode...\endluacode! form  rather than the begin...end.




\chapter{Finding where everything is}

One of the most frustrating things when you starting with a new language, is to find out where
everything is and if the various paths are correct. The snippet below does just that. It tests for a file using \lstinline!kpse.find_file!
\bigskip

%\emphasis{kpse,find}
%\begin{tcblisting}{}
%\begin{filecontents}{atest.tex}
%\end{filecontents}
%\begin{luacode}
%  -- kpse.set_program_name("lualatex")
%  -- uncomment if having problems
%  local file = "phd.dtx"
%  local path = kpse.find_file(file, '.dtx')
%    if path==nil then
%     tex.tprint({"\\color{red}"},{"File not found!"})
%   else
%     tex.tprint({path})
%   end
%\end{luacode}
%\end{tcblisting}

In the example, we created a small file on the fly, using \lstinline!filecontents! package. Then we test using \lstinline!kpse.find_file!. It returns true and we print the full path.

\subsection{the os library}
Getting the operating system type and name
\emphasis{kpse,find,require,os,name,lualibs,type}
\begin{tcblisting}{}
\begin{filecontents}{atest.tex}
\end{filecontents}
\begin{luacode}
  -- get the os name
  require("lualibs-os")
  local G,Y=os.name,os.type
  tex.tprint({G.."\\par"},{Y.."\\par"})
\end{luacode}
\end{tcblisting}

Gettting the runtime of a program

\begin{texexample}{}{}
\begin{filecontents}{atest.tex}
\end{filecontents}
\begin{luacode}
  -- get the os name
  require("lualibs-os")
  tex.print (">> ",os.runtime()," ms")
  tex.print({-2,os.uuid()})
  tex.print({-2,os.tmpname()})
\end{luacode}
\end{texexample}



\subsection{uuid}
\emphasis{os,uuid}
Getting a uuid or a slight variant of it we use \lstinline!os.uuid()!



\section{Reading from a file}

|\usepackage{luatextra}|


\begin{filecontents}{testdata10.dat}
  A  B
  1.0 20
  1.1 21
  1.2 22
\end{filecontents}
%
%\begin{texexample}{Reading from files}{ex:readfile}
%\begin{luacode*}
%function readtextfile()
%    file = io.open("testdata10.dat", "r")
%    for line1 in file:lines() do
%      --print(line1)
%      tex.print(line1)
%    end
%    file:close()
%end
%\end{luacode*}
%\directlua{readtextfile()}
%\end{texexample}
%
%\end{document}

\section{Further reading}

\begin{enumerate}
\item  Really nice introductory article ar \url{https://lwn.net/Articles/731581/}
\item   \url{https://lwn.net/Articles/640302/}
\item  News on Lua \url{https://luadigest.immortalin.com/}
\end{enumerate}











  \makeatletter
\long\def\auxm#1(#2);{%
  \def\Xtemp{#1}
  \def\Ytemp{#2}
  \parindent=0pt
  \par\leavevmode
  \hangafter=1\relax   \hangindent=1em\relax
  \bgroup  
   \bfseries\sffamily\color{red}\Xtemp\,\color{black}(\texttt{\Ytemp})\hskip0.8em
  \egroup
}

   
\newenvironment{docLpeg}[1]{
  \auxm#1;
  }  
{%
\@@par
\smallskip\parindent=1em }
\makeatother

\chapter{Lua LPeg}

\section{Pattern Matching}
\parskip6pt

LPeg is a powerful notation for matching text data, which is more capable than Lua string patterns and standard regular expressions. However, like any language you need to know the basic words and how to combine them. These patterns like their  cousins the regular expressions in languages like Perl or Javascript have a steep learning curve. The Lpeg name derives from the acronym PEG which stands for Parsing Expression Grammars. 
See also \href{http://www.inf.puc-rio.br/~roberto/lpeg/}{Parsing Expression Grammars For Lua}. 

The |lpeg| library provides functions whose names start with the prefix |lpeg|. such as |lpeg.P|, |lpeg.R|, |lpeg.S|. We can avoid having to write these prefixes by using local variables. In Lua it is better to use local variables wherever possible. RiscLua has the extra facility that we can make statements like

|   local P,R,S = lpeg.P, lpeg.R, lpeg.S|

so that P,R,S become local variables standing for lpeg.P,lpeg.R,lpeg.S. Such statements help to make code less verbose and more efficient. All our examples of code will presume that these and similar abbreviations have occurred previously.

The whole tangle of magic characters and unreadable syntax, which bedevils RegExp notation, arises from the elementary error of not distinguishing between different datatypes. For lpeg, patterns are objects in their own right, and are not strings. 

One of the recurring themes in the development of programming has been that trouble usually comes from conflating things that are not the same. Booleans are not really integers. Floating point numbers and whole numbers are different beasts. We are used to distinguishing between the number 123 and the string "123". So now we have to smarten up and recognize the difference between these and the patterns P(123) and P("123"). The former is a pattern that matches any string that is at least 123 characters long, and the latter is a pattern that matches any string that starts with the prefix "123". The function lpeg.P is an all-purpose converter of things into patterns. For example, P(true)is a pattern that matches anything, whereas P(false) matches nothing. P can convert functions and tables, too, if they are of an appropriate kind. It converts patterns to themselves. But what are patterns? That they are an abstract datatype is an answer that will probably not satisfy your curiosity. The best way to understand them is to see examples, understand how they can be combined, and see them in action the self running examples here.

Given a string $s$ and a pattern $p$ we say that p \textit{succeeds} on s if s begins with a substring that matches p. Otherwise we say that p fails on s. The expression |p:match(s)| returns a non-nil value if p succeeds on s and nil if it fails. The non-nil value depends on what sort of pattern is being used, in particular on what the pattern captures. The default situation is that the non-nil value is one plus the index of the last character matched. In the case of literal string patterns like |P"Poly"|, the value returned will be the index of the character following the matching prefix.


\subsection{String Equality}

The simplest pattern we can write is one that checks if a string is equal to another string. If there is no match L

\begin{texexample}{String Equality}{ex:strequality}
\bgroup
\catcode`\#=11
\begin{luacode}
if type(tex) == 'table' then print = tex.print end
local temp = (lpeg.P("hello"):match("world")) 
if temp then
   print(z)
   else print('no match returned nil') 
end   
local temp1 = (lpeg.P("+hello"):match("+hello world")) 
print(temp1)
\end{luacode}
\egroup
\end{texexample}


Always when we use an |lpeg| expression and pass it to the \tex engine, we need to check for a nil value. We can handle errors
in a more sophisticated way later on.


\emphasis{lpeg, match,P}
\startnumberat{10}
\begin{texexample}{}{}
\begin{luacode}
   local match = lpeg.match   
   local P = lpeg.P           
   tex.print(match(P('Za'), 'Zaza'))
   if match(P('za'), 'Zaza') then 
      tex.print(match(P('za'), 'Zaza')) 
   else tex.print('nil') 
   end
\end{luacode}
\end{texexample}

To match a pattern against a block of text, we need to first define a \emph{pattern} or as it is sometimes referred in the literature a needle, as in `find the needle in haystack'. We first start by defining some shorthands in line~\ref{match}-\ref{match1}



The \luacmd{lpeg.P} is an \emph{operator} and operates on the string, whereas \luacmd{lpeg.match} is a function. The LPeg patterns are first class objects. These patterns are regular Lua values (represented by userdata). In the example above the match matches a string exactly. The verbosity of the code must not discourage you as the aptterns are composable. All the operators are represented by single letters.  The above pattern matches at the beginning of the string and returns the end position of the string. 

To capture a range of characters, for example the lower case letters of the English alphabet  ['a-z'] we have to use \luacmd{lpeg.R}. Notice the \textbf{R} stands for range.

\emphasis{R}
\begin{texexample}{Matching Patterns lpeg.P}{}
\begin{luacode}
   local match = lpeg.match 
   local R = lpeg.R 
   local P = lpeg.P
   tex.print(match(R'az'^1 * -1, 'abcdefgi'))
   if match(R'ac', 'abcdefgh') then 
      tex.print(match(R'ac', 'abcdefgh')) 
   else 
      tex.print('nil') 
   end
   
local str =  (P'Poly'):match("Polymath")  
tex.print(str)  --> 5   
\end{luacode}
\end{texexample}

What happened in the example, we defined the letters |R'az\^1 * -1|. Lots of symbols but they will start making more sense in a minute. The  |(\^1)| means a sequence of at least one. Unlike with Lua patterns or regular expressions, you don't have to worry about escaping 'magic' characters - every character in a string stands for itself: '(','+','*', '\%', etc are just their ASCII equivalents. 

Earlier I have mentioned that patterns are composable. The + operator means \emph{either} one or the other pattern. There are numerous operators, covering all the cases as shown in Table~\ref{tbl:peg}

\medskip

\bgroup
\captionof{table}{Basic operations for creating patterns}\label{tbl:peg}
\begin{longtable}[c]{lp{8cm}}
\toprule
Operator	          &Description\\
\midrule
|lpeg.P(string)|	  &Matches string literally\\
|lpeg.P(n)|	       &Matches exactly \meta{n} characters\\
|lpeg.S(string)|  &Matches any character in string (Set)\\
|lpeg.R("xy")|	   &Matches any character between \meta{x} and \meta{y} (Range)\\
|patt^n|	         &Matches at least \meta{n} repetitions of |patt|\\
|patt^-n|	       &Matches at most n repetitions of |patt|\\
|patt1 * patt2|	 &Matches |patt1| followed by |patt2|\\
|patt1 + patt2|	 &Matches |patt1| or |patt2| (ordered choice)\\
|patt1 - patt2|	 &Matches |patt1| if |patt2| does not match\\
|-patt|	          &Equivalent to ("" - patt)\\
|#patt|	          &Matches patt but consumes no input\\
|lpeg.B(patt)|	 &Matches patt behind the current position, consuming no input\\
\bottomrule
\end{longtable}
\egroup

\begin{longtable}{ll}
P & literal\\
S & set \\
R & Range \\
* & And\\
+ & Or\\
\^& At least\\
1 & Any single\\
- & Except\\
\end{longtable}

Composition of patterns

Lua identifier

\begin{teX}
local identifier = (lpeg.R('az', 'AZ') + '_') * (lpeg.R('az','AZ','09') + '_')^0
\end{teX}

\section{Functions}

\begin{docLpeg}{lpeg.match (pattern, subject [, init])}
The matching function. It attempts to match the given pattern against the subject string. If the match succeeds, returns the index in the subject of the first character after the match, or the captured values (if the pattern captured any value).
\end{docLpeg}

An optional numeric argument init makes the match start at that position in the subject string. As usual in Lua libraries, a negative value counts from the end.

Unlike typical pattern-matching functions, match works only in anchored mode; that is, it tries to match the pattern with a prefix of the given subject string (at position init), not with an arbitrary substring of the subject. So, if we want to find a pattern anywhere in a string, we must either write a loop in Lua or write a pattern that matches anywhere. This second approach is easy and quite efficient; see examples.


\begin{docLpeg}{lpeg.type (value)}
If the given value is a pattern, returns the string "pattern". Otherwise returns nil.
\end{docLpeg}

\begin{docLpeg}{lpeg.version ()}
Returns a string with the running version of LPeg.
\end{docLpeg}

\begin{docLpeg}{lpeg.setmaxstack (max)}
Sets a limit for the size of the backtrack stack used by LPeg to track calls and choices. (The default limit is 400.) Most well-written patterns need little backtrack levels and therefore you seldom need to change this limit; before changing it you should try to rewrite your pattern to avoid the need for extra space. Nevertheless, a few useful patterns may overflow. Also, with recursive grammars, subjects with deep recursion may also need larger limits.
\end{docLpeg}


\begin{texexample}{Matching Patterns lpeg.P}{}
\begin{luacode}
local P, match = lpeg.P, lpeg.match
local print = tex.print
       
local either_ab = (P'a' + P'b')^1
        tex.print(match(either_ab,  'ababcdefg'))
     str = "Fetchez; la vache!"
print(lpeg.anywhere("a" ):match(str))
print(lpeg.anywhere("la"):match(str))
print(lpeg.anywhere("ac"):match(str))  



 str = "aaababaab,a;"
rep1 = {
    { "a",  "\\textcolor{red}{x}" },
    { "aa", "y" },
}

rep2 = {
    { "aa", "\\textcolor{blue}{y}" },
    { "a",  "x" },
    {lpeg.patterns.semicolon, "\\textcolor{purple}{semicolon;}"},
    {lpeg.patterns.comma, "\\textcolor{purple}{comma,}"}
}

print(lpeg.replacer(rep1):match(str),'\\par')

print(lpeg.replacer(rep2):match(str), '\\par') 

str ='+12.267'
print(lpeg.match(lpeg.patterns.sign, str))
\end{luacode}
\end{texexample}


It attempts to match the given pattern against the subject string. If the match succeeds, returns the index in the subject of the first character \emph{after} the match, or the values of captured values (if the pattern captured any value).

An optional numeric argument init makes the match starts at that position in the subject string. As usual in Lua libraries, a negative value counts from the end.

Unlike typical pattern-matching functions, match works only in anchored mode; that is, it tries to match the pattern with a prefix of the given subject string (at position init), not with an arbitrary substring of the subject. So, if we want to find a pattern anywhere in a string, we must either write a loop in Lua or write a pattern that matches anywhere. This second approach is easy and quite efficient.

The lualibs library which we load by default when LuaLaTeX is detected comes with dozens of predefined patterns and a number of utilities.

\section[\textbackslash lpegreplace]{\textbackslash lpegreplace(table, text)}

\begin{docLpeg}{lpeg.replacer(table)}
Accepts a list of pairs and returns a pattern that substitutes any first elements of a given pair by its second element. The latter can be a string, a hashtable, or a function (whatever is permissible with |lpegs.Cs|).
The order of the elements of a table is important, as lpeg's match the leftmost element of a disjunction first.
\end{docLpeg}


\section{Basic Captures}

Getting the index after a match is all very well, and you can then use \luacmd{string.sub} to extract the strings. But there are ways of explicitly asking for captures:

\begin{phdverbatim}
> C = lpeg.C  -- captures a match
> Ct = lpeg.Ct -- a table with all captures from the pattern
\end{phdverbatim}


\begin{texexample}{Matching Patterns lpeg.P}{ex:peg0}
\begin{luacode}
 local C = lpeg.C
 Ct = lpeg.Ct
 digit = lpeg.R'09'   --  anything from '0' to '9'
 digits = digit^1     --  a sequence of at least one digit
 cdigits= C(digits)   --  capture digits
 letters = lpeg.R'az'^1 --  captures letters
 cletters= C(letters)  --  captures letters
 tex.print(cdigits:match '123A9AAAA')
 tex.print(cletters:match 'onlylowercaseAAAA')
\end{luacode}
\end{texexample}

In Example~\ref{ex:peg} we define two functions $tobold(n)$ and $toitalic(char)$, which will be
mapped to the capture. 

\begin{texexample}{Matching Patterns lpeg.P}{ex:peg}
\begin{luacode}
local tobold = function(n)
         return "\\textbf{"..tostring(n).."}%\n" 
      end
      
local toitalic = function(a)      
        return "\\textit{"..a.."}%\n"
      end  
      
local C  = lpeg.C
local Cf = lpeg.Cf
local Ct = lpeg.Ct
 
digit = lpeg.R'09'/tobold   --  anything from '0' to '9'
digits = digit              --  a sequence of at least one digit
letters = lpeg.R'AZ'/toitalic
cdigits= (Cf(digits, tobold) + Cf(letters, toitalic))^1  --  capture digits
 
 local function test(str)
    if (cdigits:match(str)) then
      print(cdigits:match(str))
    else
      print("No capture")
    end
  end
  
test("123456HAMBURGERVIS")
\end{luacode}
\end{texexample}


A capture is a pattern that produces values (the so called semantic information) according to what it matches. LPeg offers several kinds of captures, which produces values based on matches and combine these values to produce new values. Each capture may produce zero or more values.

The following table summarizes the basic captures:
\captionof{table}{Basic operations for creating patterns}\label{tbl:peg}
\nobreak
\begin{longtable}[c]{|l|p{8cm}|}
\hline
Operation	   &What it Produces\\
\hline

lpeg.Cf(patt, func)	&a folding of the captures from patt\\
lpeg.Cg(patt [, name])	&the values produced by patt, optionally tagged with name\\
lpeg.Cp()	&the current position (matches the empty string)\\
lpeg.Cs(patt)	&the match for patt with the values from nested captures replacing their matches\\
lpeg.Ct(patt)	&a table with all captures from patt\\
patt / string	&string, with some marks replaced by captures of patt\\
patt / number	&the n-th value captured by patt, or no value when number is zero.\\
patt / table	&table[c], where c is the (first) capture of patt\\
patt / function	&the returns of function applied to the captures of patt\\
lpeg.Cmt(patt, function)	&the returns of function applied to the captures of patt; the application is done at match time\\
\hline
\end{longtable}





\begin{docLpeg}{lpeg.C(patt)}
the match for |patt| plus all captures made by patt
\end{docLpeg} 

\begin{docLpeg}{lpeg.Carg(n)}	
The value of the nth extra argument to lpeg.match (matches the empty string).
\end{docLpeg}

\begin{docLpeg}{lpeg.Cb(name)}
The values produced by the previous group capture named name (matches the empty string)
\end{docLpeg}

\begin{docLpeg}{lpeg.Cc(values)}
The given values (matches the empty string)
\end{docLpeg}

\begin{docLpeg}{lpeg.Cf(patt, func)}
a folding of the captures from patt
\end{docLpeg}

\begin{docLpeg}{lpeg.Cg(patt [, name])}
the values produced by patt, optionally tagged with name
\end{docLpeg}

\begin{docLpeg}{lpeg.Cp()}
the current position (matches the empty string)
\end{docLpeg}

\begin{docLpeg}{lpeg.Cs(patt)}
the match for patt with the values from nested captures replacing their matches
\end{docLpeg}


\section{Built in patterns}

The following is an extract from the |ConTeXt| guide. The example captures all words in square brackets. 

\begin{texexample}{Matching Patterns lpeg.P}{}

\begin{luacode}
if type(tex) == 'table' then print = tex.print end
local P, R, C, Ct = lpeg.P, lpeg.R, lpeg.C, lpeg.Ct
local pattern = Ct((P("[") * C(R("az")^0) * P(']') + P(1))^0)
local words = lpeg.match(pattern,"a [ first ] and [second] word, some utf [third]")
tex.print(words)  

 
\end{luacode}
\end{texexample}




\footnote{\protect\url{http://www.inf.puc-rio.br/~roberto/lpeg/lpeg.html\#intro}}

\endinput

local alpha, cntrl, digit, graph, lower, punct, space, upper, alnum, xdigit =
   lpeg.alpha, lpeg.cntrl, lpeg.digit, lpeg.graph, lpeg.lower, lpeg.punct,
   lpeg.space, lpeg.upper, lpeg.alnum, lpeg.xdigit

local pattern = lpeg.Ct(lpeg.upper)   
local upwords = lpeg.match(pattern "Some words") 

http://leafo.net/guides/parsing-expression-grammars.html#what-is-a-peg 
  \chapter{Working with Lua Tables}

This section explores Lua's data type called a table. It's a data structure, which means that it lets
you combine other values. Because of its flexibility, it is Lua’s only data structure. (It is possible to
create other, special-purpose data structures in C.) We will first outline the general concepts for creating tables and then examine the use of the library |table| and also some extra utilities available in distributions. The |phd| package also provides some additional help with the module \textbf{phd.util.tablex}.

\section{Table Construction}


\begin{texexample}{}{}
\begin{luacode}
hash = {}
hash[ 'key-1' ] = "val1"
hash[ 'key-2' ] = 1
hash[ 'key-3' ] = {}
tex.sprint(hash['key-2'])
\end{luacode}
\end{texexample}

For example we can declare tables within tables, font holding tables and even whole databases.

\begin{texexample}{Defining tables}{}
\begin{luacode}
local t = {
    document_type = {'article'},
    geometry = {
       paper = 'a4'
     }
} 
tex.print(t.geometry.paper)
\end{luacode}
\end{texexample}

One trick that we will be using early on, is to load some utilities. These take the form of modules which we describe in more detail later on. 

\emphasis{type}
\begin{texexample}{Defining tables}{}
\begin{luacode}
if type(tex) == 'table' then print = tex.print end
local t = {
    document_type = {'article'},
    geometry = {
       paper = 'a4'
     }
} 
print(t.geometry.paper)
\end{luacode}
\end{texexample}

As you can see from the example, |tex.print()| function is also stored in a table. Tables are versatile enough especially when combined with metatables and metamethods that they can be used in defining classes, objects,  stacks, lists, tuples or any other data structure you can imagine. 

One useful construct is to detect if we are running in LuaTeX or another Lua binary. This enables us to use the normal Lua function |print|  enabling our Lua programs to run under normal Lua binaries as well as LuaTeX. This can speed up development and also enable debugging using test environments.

\begin{scriptexample}{}{}
\begin{verbatim}
  if type(tex) == 'table' then print = tex.print end
\end{verbatim}
\end{scriptexample}

To test we have used the Lua function |type()|, which returns the type of the variable, in this case a 'table'. Other types are |number|, |string|, |userdata| and |nil|. 

\section{Accessing keys and values}

You can access any key using the dot notation, as we have used in the above example or using square brackets. The former is considered better programming style than the latter.

\emphasis{type}
\begin{texexample}{Defining tables}{}
\begin{luacode}
if type(tex) == 'table' then print = tex.print end
local t = {
    document_type = {'article'},
    geometry = {
       paper = 'a4'
     }
  } 
 print(t.geometry['paper'])
\end{luacode}
\end{texexample}

\section{Iterate over the keys of a table}

\begin{texexample}{Defining tables}{}
\begin{luacode}
if type(tex) == 'table' then print = tex.print end
local paper = {'a4', 'a3', 'a2', 'a1', 'a5', 'a0' }
for k,v in pairs(paper) do
   print(v)
end
\end{luacode}
\end{texexample}


\section{The Table Library}
The Lua |table| library comprises auxiliary functions to manipulate tables as arrays.
It provides functions to insert and remove elements from lists, to sort the elements of an array, and to concatenate all strings in an array.

\subsection{table.insert(aTable,[pos],value)}
This function is used to insert a value into an array table, as indicated by a Table. The value
is inserted at the position indicated by pos; if not value is indicated for pos, then it is assigned
the default value of the length of the table plus 1. (i.e., it is inserted at the end of the table).

\emphasis{insert}
\begin{texexample}{Insert}{ex:luainsert}
\begin{luacode}
   local someTable = {5,10,22,23,26}
   table.insert(someTable,2,13)
   tex.print(someTable)
\end{luacode}  

\end{texexample}



\subsection{Sorting}

Sorting is one issue, which cannot be handled easily with \latexe or \latex3. So far I have been using
a buble sort for lstdoc where I had to to sort through a list.

Here is one area, where Lua can shine. Consider a list of paper sizes.

\emphasis{table, sort}
\begin{texexample}{Sorting tables}{}
\long\def\clisttest#1{
  \directlua{
    if type(tex) == 'table' then print = tex.print end 
      local paper = {#1}
      table.sort(paper)
      for k,v in pairs(paper) do
      print(v.."paper, ")
    end
}}

\ExplSyntaxOn
\clist_new:N \clist_values

\DeclareDocumentCommand\ClistTest { m } {%
  \clist_set:Nn \clist_values {#1}
  \clisttest{\clist_values}
}
\ExplSyntaxOff

\ClistTest{'a1','a3', 'a2', 'a4','a5'}
\end{texexample}

One word of explanation regarding line \ref{totable}. We test if |type(tex)| is a table and then we set print to be equal to |tex.print|
This convention, is useful, if you want to run longer code in an IDE rather than through LuaTeX. It is easier and faster
for debugging.

One caveat, is that tables that contain |nil| values cannot be sorted reliably. The table library provides a function |table.sort|, which receives a table and sort its elements. Such functions normally provide an optional parameter that provides variations in the sorting order, such as \emph{ascending} or \emph{descending}, numeric or alphabetical. Instead of trying to provide all kinds of options |sort| provides a single optional parameter, which is the \emph{order function}: a function that receives two elements and returns whether the first must come before the second in the 
sorted list. Suppose we have a table of records such as this:

\begin{quote}
\begin{verbatim}
local students = {
  {name = 'John', avg = '80',
  {name = 'Mary', avg = '66',
  {name = 'David', avg = '99', 
  {name = 'Patel', avg ='98',
}
\end{verbatim}
\end{quote}

If we want to sort the table by the field name, in reverse alphabetical order, we
just call the {\color{thered}\ttfamily table.sort}  funcion with an optional sorting function.

\begin{texexample}{Sorting tables}{}
\begin{luacode}
if type(tex) == 'table' then print = tex.print end
local students = {
  {name = 'John', avg = '80'},
  {name = 'Mary', avg = '66'},
  {name = 'David', avg = '99'}, 
  {name = 'Patel', avg ='98'},
  {name = 'Anne', avg = '52'},
}
table.sort(students, function (a, b) return (a.name < b.name) end)
for k, v in pairs(students) do
  print(v.name)
end
\end{luacode}
\end{texexample}

Of course typing the function is not always necessary, as we can define two variables to hold the functions:

\begin{scriptexample}{}{}
\begin{verbatim}
local descending = function (a, b) return (a.name > b.name) end
local ascending = function (a, b) return (a.name < b.name) end
\end{verbatim}
\end{scriptexample}

We can sort by marks as well or any other key. In the next example we just change the sorting function to achieve this.


\begin{teX}
\begin{luacode}
if type(tex) == 'table' then print = tex.print end
local students = {
  {name = 'Peter', avg = '52'},  (*@\label{peter}  @*)
  {name = 'John', avg = '80'},
  {name = 'Mary', avg = '66'},
  {name = 'David', avg = '99'}, 
  {name = 'Patel', avg ='98'},
  {name = 'Anne', avg = '52'},
  {name = 'An', avg = '52'},
  {name = 'Zavier', avg = '87'} 
}

local descending = function (a, b) return (a.avg > b.avg) end
local ascending = function (a, b) return (a.avg < b.avg) end

table.sort(students, descending)   

print('\\begin{tabular}{ll}')
for _, v in pairs(students) do
   print(v.name ..'&'..  v.avg .. '\\\\')
end
print('\\end{tabular}')
\end{luacode}
\end{teX}

What just happened, not only we have sorted the table, but we have also managed to typeset it using LaTeX tabular environment. Using the function \emph{descending} we first sorted the table and then used an iterator to print the table.

One note in the Lua manual states that the sort function is not stable (see line \ref{peter}). That does not mean that the sort algorithm is still under development. This is a computer science term that means that given two equal marks in our example the order of the sort for these marks might be different that the one entered. This is better understood by the image, which is from the relevant wikipedia article. To achieve stable sorting one has to implement \emph{radix} sorting.\footnote{See \url{https://github.com/kennyledet/Algorithm-Implementations/blob/master/Radix_Sort/Lua/Yonaba/radix_sort.lua} for an implementation.}

\begin{figure}[htbp]
\centering

\includegraphics[width=0.4\textwidth]{./images/stable-sort.png}
\caption{Differences between stable and unstable sorts.}
\end{figure}

\begin{scriptexample}{}{}
To summarize. A Lua table always contains an unordered set of key-value pairs. Lua provides the tools but you always you need to provide the batteries.
\end{scriptexample}

\subsection{Inserting and removing elements}

The table.insert function inserts an element in a given position of an array,
moving up other elements to open space. For instance, if t is the array
{10,20,30}, after the call table.insert(t,1,15) t will be {15,10,20,30}. As
a special (and frequent) case, if we call insert without a position, it inserts the
element in the last position of the array (and, therefore, moves no elements).

\emphasis{insert}

\begin{texexample}{Inserting and Deleting}{}
\begin{luacode}
if type(tex) == 'table' then print = tex.print end
local students = {
  {name = 'Peter', avg = '52'}, 
  {name = 'John', avg = '80'},
  {name = 'Mary', avg = '66'},
  {name = 'David', avg = '99'}, 
  {name = 'Patel', avg ='98'},
  {name = 'Anne', avg = '52'},
  {name = 'An', avg = '52'},
  {name = 'Zavier', avg = '87'} 
}

table.insert(students, {name='Annabel', avg='52'})

local descending = function (a, b) return (a.avg > b.avg) end
local ascending = function (a, b) return (a.avg < b.avg) end

--table.sort(students, descending)   

print('\\begin{tabular}{ll}')
for _, v in pairs(students) do
   print(v.name ..'&'..  v.avg .. '\\\\')
end
print('\\end{tabular}')
\end{luacode}

\end{texexample}

The |insert| function takes another argument to specify the position we want the element to be inserted in the array |table.insert(students, 5, {name='Annabel', avg='52'})|. If it is left out it will insert it at the head of the list.

\section{An i18n table}

In the Chapter \nameref{ch:languages} we discuss loacalization and internationalization issues. In the next example we will use tables holding language strings (captions in Babel terminology) for months. 

\begin{texexample}{Language captions}{ex:lcaptions}
\begin{luacode}
  local l1, l2 = require("i18n.spanish"), require("i18n.german")
  tex.print(l2.german.months.january)
  tex.print(l1.spanish.months[2])
\end{luacode}
\end{texexample}














  \chapter{Metatables and Metamethods}

\def\textdblunderscore#1{{\bfseries \textunderscore\hskip1pt\textunderscore#1}}

Every language has its own peculiarities and Lua is no exception with its metatables and metamethods. Lua’s metatables allow us to change the behaviour of a value when confronted with an undefined operation. For example when using metatables, we can define how lua computes the expression a+b, where a and b are tables. Whenever Lua tries to add two tables, it checks whether either of them has a \emph{metatable} and whether ths metatable has an \textdblunderscore{add} field. If Lua finds this field, it calls the corresponding value---the so-called \emph{metamethod}, which should be a function---to compute the sum.

Lua by default always creates new tables without metatables. We can use \luacmd{setmetatable} to set or change the metatable of any table.

\begin{texexample}{Metatable}{}
\begin{luacode}
require"lualibs-dir"
local t, t1 = {}, {}
  print = tex.print
  setmetatable (t, t1)

-- print(getmetatable(t)==t1)

local result = dir.found("./lua/")

if result then tex.print(-2, result) else tex.print('not found')  end

local image_fixed = img.scan({filename = "./images/korea-01.jpg"})

local image     = img.copy(image_fixed)
local halfimage = img.copy(image_fixed)

halfimage.height = halfimage.height *0.3
halfimage.width  = halfimage.width  *0.3

--node.write(img.node(image))


local h=node.hpack(img.node(halfimage), img.node(halfimage))

node.write(h)

tex.print((halfimage.width/65000)..'pt')

\end{luacode}
\end{texexample}
\includegraphics[width=0.3\textwidth,decodearray={0.2 0.2 1 0 1 0.8 1 0}]{./images/korea-01.jpg}

From Lua we can access the metatables only of tables; for any other values we must use C code. 
 
  
\let \handlegroupnormalbefore \relax
\let \handlegroupnormalafter  \relax

\protected \def \handlegroupnormal #1#2{^^A
  \bgroup
  \def \handlegroupbefore {#1}^^A
  \def \handlegroupafter  {#2}^^A
  \afterassignment \handlegroupnormalbefore
  \let \next =^^A
}

\def \handlegroupnormalbefore {^^A
  \bgroup 
  \handlegroupbefore
  \bgroup 
  \aftergroup \handlegroupnormalafter%
}

\def \handlegroupnormalafter {%
  \handlegroupafter
  \egroup 
  \egroup 
}

\let \groupedcommand \handlegroupnormal 

\def \definehighlight [#1][#2]{%
  \ifcsname #1\endcsname\else
    \expandafter\def\csname #1\endcsname{%
      \leavevmode
      \groupedcommand {#2}\empty%
    }
  \fi%
}



\def\restoreunderscore{\catcode`\_=12\relax}

\definehighlight         [fileent][\ttfamily\restoreunderscore]         %% files, dirs
\definehighlight        [texmacro][\sffamily\itshape\textbackslash]     %% cs
\definehighlight     [luafunction][\sffamily\itshape\restoreunderscore] %% lua identifiers
\definehighlight      [identifier][\sffamily]                           %% names
\definehighlight          [abbrev][\rmfamily\scshape]                   %% acronyms
\definehighlight        [emphasis][\rmfamily\slshape]                   %% level 1 emph

\definehighlight       [Largefont][\Large]                              %% font size
\definehighlight       [smallcaps][\sc]                                 %% font feature
\definehighlight [nonproportional][\tt]                                 %% font switch
%\let \inlinecode \lstinline
\protected \def \inlinecode {\lstinline}

\chapter{LuaTeX and Fonts}
\label{c:luatexfonts}


\section{Introduction}
With \LuaTeX/\LuaLaTeX a completely different mechanism to that of \XeLaTeX or \pdfLaTeX is used to load fonts.

Font management and installation has always been painful with \tex.  A
lot of files are needed for one font ({tfm},{pfb},
{map}, {fd}, {vf}), and due to the 8-Bit encoding
each font is limited to 256 characters.

But the font world has evolved since the original \tex, and new
typographic systems have appeared, most notably the so called
\emph{smart font} technologies like \OpenType fonts ({otf}).

These fonts can contain many more characters than \tex fonts, as well
as additional functionality like ligatures, old-style numbers, small
capitals, etc., and support more complex writing systems like Arabic
and Indic
  Unfortunately, \pkgname{luaotfload} doesn‘t support many Indic
  scripts right now.
scripts.

In \identifier{luaotfload}, the canonical syntax for font requests
requires a \emphasis{prefix}:
%
\begin{quote}
  \nonproportional{\string\font\string\fontname\space= }%
  \meta{prefix}%
  \nonproportional{:}%
  \meta{fontname}%
  \dots
\end{quote}
%
where \meta{prefix} is either \inlinecode{file:} or \inlinecode {name:}.
  The development version also knows two further prefixes,
  \inlinecode {kpse:} and \inlinecode {my:}.
  %
  A \inlinecode {kpse} lookup is restricted to files that can be found by
  \identifier{kpathsea} and
  will not attempt to locate system fonts.
  %
  This behavior can be of value when an extra degree of encapsulation is
  needed, for instance when supplying a customized tex distribution.

  The \inlinecode {my} lookup takes this a step further: it lets you define
  a custom resolver function and hook it into the 

   \verb+\luafunction{resolve_font}+

  callback.
  %
  This ensures full control over how a file is located.
  %
  For a working example see the
  \url {https://bitbucket.org/phg/lua-la-tex-tests/src/5f6a535d/pln-lookup-callback-1.tex}.

%
It determines whether the font loader should interpret the request as
a \emphasis{file name} or
  \emphasis{font name}, respectively,
which again influences how it will attempt to locate the font.
%
Examples for font names are
            “Latin Modern Italic”,
            “GFS Bodoni Rg”, and
            “PT Serif Caption”
-- they are the human readable identifiers
usually listed in drop-down menus and the like.\footnote{%
  Font names may appear like a great choice at first because they
  offer seemingly more intuitive identifiers in comparison to arguably
  cryptic file names:
  %
  “PT Sans Bold” is a lot more descriptive than \fileent{PTS75F.ttf}.
  On the other hand, font names are quite arbitrary and there is no
  universal method to determine their meaning.
  %
  While \identifier{luaotfload} provides fairly sophisticated heuristic
  to figure out a matching font style, weight, and optical size, it
  cannot be relied upon to work satisfactorily for all font files.
  %
  For an in-depth analysis of the situation and how broken font names
  are, please refer to
  \hyperlink [this post]{http://www.ntg.nl/pipermail/ntg-context/2013/073889.html}
  by Hans Hagen, the author of the font loader.
  %
  If in doubt, use filenames.
  %
  \fileent{luaotfload-tool} can perform the matching for you with the
  option \inlinecode {--find=<name>}, and you can use the file name it returns
  in your font definition.}

%

\subsection {Compatibility Layer}

In addition to the regular prefixed requests, \identifier{luaotfload}
accepts loading fonts the \XETEX way.
%
There are again two modes: bracketed and unbracketed.
A bracketed request looks as follows.

\begin{quote}
  \nonproportional{\string\font\string\fontname\space = [}%
  \meta{/path/to/file}%
  \nonproportional{]}
\end{quote}

\noindent
Inside the square brackets, every character except for a closing
bracket is permitted, allowing for specifying paths to a font file.
%
Naturally, path-less file names are equally valid and processed the
same way as an ordinary \inlinecode {file:} lookup.

\begin{quote}
  \nonproportional{\string\font\string\fontname\space= }%
  \meta{font name}
  \ldots
\end{quote}

Unbracketed (or, for lack of a better word: \emphasis{anonymous})
font requests resemble the conventional \TEX syntax.
%
However, they have a broader spectrum of possible interpretations:
before anything else, \identifier{luaotfload} attempts to load a
traditional \TEX Font Metric (\abbrev{tfm} or \abbrev{ofm}).
%
If this fails, it performs a \inlinecode {name:} lookup, which itself will
fall back to a \inlinecode {file:} lookup if no database entry matches
\meta{font name}.

Furthermore, \identifier{luaotfload} supports the slashed (shorthand)
font style notation from \XETEX.

\begin{quote}
  \nonproportional{\string\font\string\fontname\space= }%
  \meta{font name}%
  \nonproportional{/}%
  \meta{modifier}
  \dots
\end{quote}





\texttt{OpenType} fonts are widely deployed and available for all modern
operating systems.

As of 2014 they have become the de facto standard for advanced text
layout.

However, until recently the only way to use them directly in the \tex
world was with the \XeTeX engine.

Unlike \XeTeX, \LUATEX has no built-in support for \OpenType or
technologies other than the original \tex fonts.

Instead, it provides hooks for executing LUA code during the TEX run
that allow implementing extensions for loading fonts and manipulating
how input text is processed without modifying the underlying engine.

This is where \pkgname{luaotfload} comes into play:
Based on code from \CONTEXT, it extends \LUATEX with functionality necessary
for handling \OpenType fonts.

Additionally, it provides means for accessing fonts known to the operating system conveniently by indexing the metadata. The |luaotfload| package is an adaptation of the \CONTEXT font loading system, allowing for loading \OpenType fonts with an extended syntax and adds support for a variety of font features. With current developments of moving \xetex to \luatex it is the expected way of \latex to evolve.

\section{Loading Fonts}

|luaotfload| supports an extended syntax for font loading, similar to that used by \xetex.



\CMDI{\font}=\meta{prefix}:\meta{font name}:\meta{font features}\meta{\tex features}

The curly brackets are optional and escape the spaces in the enclosed
font name. Alternatively, double quotes serve the same purpose.\footnote{A surprising feature is with LuaTeX capitalization is not significant, the font name can be typed in lower or upper case and will still find the file. See \protect\url{http://tex.stackexchange.com/questions/223236/why-is-setmainfont-case-sensitive-with-xelatex-but-not-with-lualatex}.}

\section{Below the Hood}

Most \tex users like to have full control and are curious to understand how things work. Using |luaotfload| we can have a look at the |otf| tables, and although a bit error prone can 
see how the fonts store information. The first example we will examine is a short programme to print 50 of the glyph names. In Example \ref{ex:symbola}, we use the built-in  \luacmd{fontloader} function. Many font symbolic names have underscores or other problematic characters. We change the underscore using \luacmd{string.gsub} from the |strings| module, which is loaded automatically with LuaTeX.

\newfontfamily{\symbola}{Symbola.ttf}

\begin{texexample}{Getting the name of Glyphs}{ex:symbola}
\bgroup
\begin{luacode}
tex.print("\\footnotesize")
f=fontloader.open("c:/windows/fonts/Symbola.ttf")
tex.print (f.fontname,'\\par\\symbola')
local i = 65
while (i < 100) do  
local g = f.glyphs[i]
if g then
  local s = string.gsub(g.name, '%_', '\\textunderscore  ')
  tex.print(s .. "(\\symbola\\char" .. g.unicode ..")" )
end
  i = i + 1
end
fontloader.close(f)
\end{luacode}
\egroup
\end{texexample}

If you are compiling this document in Windows, the above example would probably find your fonts. However, this is not a very good way to search for fonts. The TeX world has settled over many years on two principles when distributing files. The TeX Directory Structure (TDS) which is a directory hierarchy for macros, fonts, and the other implementation-independent TeX system files and the kpathsea utility for locating these files and for running the many scripts that are necessary while typesetting.

\begin{texexample}{Getting the name of Glyphs}{}
\begin{luacode}
local filename = "Symbola.ttf"
local fullname = kpse.find_file(filename, 'truetype fonts') 
if fullname then tex.print(fullname) else tex.print("font ".. filename .. " not found") end
\end{luacode}
\end{texexample}

The various fontloader programs go at great lengths to ensure that all possible directory paths are searched and the main reason, why when using a font for the first time it takes so long to process. With Lua once the
 file is located, we can read it and manipulate the data in any way we want.






  

\newcommand{\be}{\begin{equation}}
\newcommand{\ee}{\end{equation}}
\newcommand{\vv}[1]{\mathbf #1}						% 3-vector
\newcommand{\la}[1]{\label{#1}}
\newcommand{\Eq}[1]{Eq.~(\ref{#1})}
\newcommand{\Eqs}[2]{Eqs.~(\ref{#1},\ref{#2})}
\newcommand{\bea}{\begin{eqnarray}}
\newcommand{\eea}{\end{eqnarray}}
\newcommand{\Sec}[1]{Sec.~\ref{#1}}
\newcommand{\Secs}[2]{Secs.~\ref{#1}, \ref{#2}}

\newcommand{\Fig}[1]{Fig.~\ref{#1}}
\def\nn{\nonumber \\ }

\chapter{Solar Position}
\label{ch:solar}

In Chapter i18n, we primarily dealt with the necessary translation strings and we briefly touched on 
calendric calculations. In this Chapter we will describe equations that describe the position of the sun, as seen from observers on earth and develop some interesting typographical sidelines. The main object of the Chapter is to outline the usage of LuaTeX in the preparation of complex documents.

\section{Spherical coordinates for the sun}

\begin{figure} [tbh]
\begin{center}
	\includegraphics[width=0.37 \textwidth]{./images/solar/spherical.pdf}
\end{center}
\caption{\small In spherical coordinates, a three-dimensional vector $\vv r$ is expressed in terms of a radial distance $r$, a polar angle $\theta$, and an azimuthal angle $\phi$.  The vector {\boldmath $\rho$} is the projection of $\vv r$ onto the $x$-$y$ plane.  In terms of the rectangular coordinates, $x = r \sin \theta \cos \phi$, $y = r \sin \theta \sin \phi$, and $z = r \cos \theta$.\la{fig:spherical}}
\end{figure}
Any point in three-dimensional space can be defined by the spherical co-ordinates $(r, \theta, \phi)$, where $r$ is the distance, $\theta$ the polar angle, and $\phi$ is the azimuthal angle. 

In terms of the rectangular coordinates $(x, y, z)$, we have
\be
\vv r = \left( \begin{array}{c} x \\ y \\ z \end{array} \right) = \left( \begin{array}{c} r \sin \theta \cos \phi  \\ r \sin \theta \sin \phi \\ r \cos \theta \end{array} \right)~,
\la{eq:spherical}
\ee

The {\it celestial sphere} is an imaginary spherical surface, sharing a center with the Earth's globe, and with a very large, indefinite radius.  The positions of the stars, planets, and other heavenly bodies are characterized by their radial projection onto this surface.  The largeness of the radius of the celestial sphere, compared to the radius of the Earth, allows us, when convenient, to picture it as centered at the position of an observer standing on the Earth's surface, rather than at the center of the Earth.

For simplicity, we take the radius $r$ of the celestial sphere to be equal to 1 (in undetermined units).  We shall use a subscript $\odot$ (the astronomical symbol for the Sun) to indicate that a vector or a coordinate thereof refers to the position of the Sun.

\subsection{Ecliptic frame}
\la{sec:ecliptic}

From the Earth, the Sun appears to move, against the background of the distant stars, along a great circle on the celestial sphere called the {\it ecliptic}.\footnote{The ecliptic is sometimes defined as the plane of the Earth's orbit around the Sun.  The circle that we call the ``ecliptic'' is the intersection of that plane with the celestial sphere.}  We will therefore start by working in an ``ecliptic frame,'' in which the position of the distant stars is fixed,\footnote{For this reason the distant stars, which form the constellations, are also referred to as the ``fixed stars.''} and in which the polar angle of the Sun is always $\theta_\odot = \pi / 2$, whereas the azimuthal angle $\phi_\odot$ varies over the course of the year, as shown in \Fig{fig:ecliptic}.  If the Earth's orbit were perfectly circular, then $\phi_\odot$ would increase at a constant rate, completing a full revolution in a year.  In \Sec{sec:center} we will see how to account for the fact that the Earth's orbit is slightly elliptical, but for now we will simply express the azimuthal angle of the Sun as a function of the time $t$.  We therefore express the position of the Sun, in the ecliptic frame of reference, as:
\be
\vv r_\odot (t) = \left( \begin{array}{c} \cos \phi_\odot (t) \\ \sin \phi_\odot (t) \\ 0 \end{array} \right)~.
\la{eq:ecliptic}
\ee

\begin{figure} [tbh]
\begin{center}
	\includegraphics[width=0.47 \textwidth]{./images/solar/ecliptic.pdf}
\end{center}
\caption{\small The Sun moves along the ecliptic during the course of the year.  In the ecliptic frame of reference, the Sun's polar angle is fixed, $\theta_\odot = \pi / 2$, while the azimuthal angle $\phi_\odot$ increases with time at an approximately constant rate of $2 \pi$ per year.\la{fig:ecliptic}}
\end{figure}

\subsection{Equatorial frame}
\la{sec:equatorial}

The axis of rotation of the Earth is tilted with respect to the ecliptic frame by an angle of obliquity
\be
\varepsilon = 23.44^\circ = 0.4091 ~.
\la{eq:obliquity}
\ee
It is therefore convenient to change coordinates to an ``equatorial frame,'' by rotating about the $x$-axis by an angle $\varepsilon$, as shown in \Fig{fig:equatorial}(a), so that the new $z'$-axis coincides with the Earth's axis of rotation.  The motion of the Sun in this equatorial frame is illustrated in \Fig{fig:equatorial}(b), in which the celestial north pole is labelled $P$ and the celestial south pole $\overline P$.  The ecliptic intersects the celestial equator at two points, $e$ and $\bar e$, known as the {\it equinoxes}.  At $e$ the Sun crosses the equator from south to north, and this is therefore known as the northward equinox (or ``first equinox'', since it occurs first in the calendar year).  Conversely, $\bar e$ is known as the southward, or second equinox.  The points of maximum displacement between the position of the Sun and the celestial equator are known as {\it solstices}, and are marked in \Fig{fig:equatorial}(b) by $s$ and $\bar s$.

The position of Sun in the equatorial coordinate frame is given by:
\bea
\vv r'_\odot =
\left( \begin{array}{c} x'_\odot \\ y'_\odot \\ z'_\odot \end{array} \right) &=&
\left( \begin{array}{c} \sin \theta'_\odot \cos \phi'_\odot \\ \sin \theta'_\odot \sin \phi'_\odot \\ \cos \theta'_\odot \end{array} \right) \nn
&=&
\left( \begin{array}{r r r}
1 & 0 & 0 \\
0 & \cos \varepsilon & - \sin \varepsilon \\
0 & \sin \varepsilon & \cos \varepsilon
\end{array} \right)
\left( \begin{array}{c} \cos \phi_\odot  \\ \sin \phi_\odot \\ 0 \end{array} \right) = 
\left( \begin{array}{c} \cos \phi_\odot  \\ \cos \varepsilon \sin \phi_\odot \\ \sin \varepsilon \sin \phi_\odot \end{array} \right) ~.
\la{eq:equatorial}
\eea
The polar angle for the Sun in this equatorial reference frame is therefore
\be
\theta'_\odot = \arccos z'_\odot = \arccos \left( \sin \varepsilon \sin \phi_\odot \right)~.
\la{eq:polar-sun}
\ee

In the astronomical literature, the value of $\phi_\odot$, measured with respect to the first equinox $e$, is called the Sun's ``ecliptic longitude.''  The corresponding $\phi'_\odot$ in the equatorial frame is called the Sun's ``right ascension,'' while $\pi / 2 - \theta'_\odot$ is its ``declination.''\footnote{It is common to approximate the Sun's declination as \hbox{$\pi / 2 - \theta'_\odot \simeq \varepsilon \sin \phi_\odot$}; see, e.g., \cite{Khavrus}.  This is accurate to within $0.26^\circ$ for the value of $\varepsilon$ in \Eq{eq:obliquity}, but rather obscures the geometry involved, as Sproul comments in \cite{Sproul}.}  For a full discussion of the celestial sphere and of the coordinate systems that astronomers use to characterize points on it, see \cite{celestial-sphere}.

\begin{figure} [htb]
\begin{center}
	\includegraphics[width=0.3 \textwidth]{./images/solar/obliquity.pdf} \hskip 2.5 cm
	\includegraphics[width=0.42 \textwidth]{./images/solar/equinoxes.pdf}
\end{center}
\caption{\small (a): We may transform from the ecliptic to the equatorial reference frame by rotating along the $x$-axis by an angle equal to the obliquity $\varepsilon$, given in \Eq{eq:obliquity}, so that the new axis $z'$-axis is also the axis of the Earth's rotation.  (b):  The motion of the Sun along the ecliptic, as seen in the new equatorial reference frame.  Point $P$ marks the celestial north pole and $\overline P$ the celestial south pole.  The northward and southward equinoxes are marked by $e$ and $\bar e$, respectively.  The northern and southern solstices are indicated by $s$ and $\bar s$, respectively.\la{fig:equatorial}}
\end{figure}

\subsection{Terrestrial frame}
\la{sec:terrestrial}

Seen from the Earth, objects in the sky rotate azimuthally in the equatorial frame (i.e., about the $z'$-axis), with constant angular velocity
\be
\omega = \frac{2\pi}{23.9345 ~\hbox{hours}}~,
\la{eq:omega}
\ee
where $23.9345$ hours is the duration of the ``sidereal day,'' equal to the amount of time that it takes the Earth to complete one rotation about its axis (and therefore also for a distant star to return to the same position in the sky).  This is slightly less than the ``mean solar day'' of 24 hours, because of the Sun's motion along the ecliptic during the course of one sidereal day.

To characterize the position of the Sun, as seen from a point on the surface of the Earth, we must also adjust for the geographic latitude $L$.  We can achieve this by rotating about the $x$-axis by an angle equal to the co-latitude $\pi / 2 - L$.  The transformation from the equatorial frame to the terrestrial frame therefore gives: 
\bea
\vv r''_\odot =
\left( \begin{array}{c} x''_\odot \\ y''_\odot \\ z''_\odot \end{array} \right) &=&
\left( \begin{array}{c} \sin \theta''_\odot \cos \phi''_\odot \\ \sin \theta''_\odot \sin \phi''_\odot \\ \cos \theta''_\odot \end{array} \right) \nn
&=& \left( \begin{array}{r r r}
1 & 0 & 0 \\
0 & \sin L & - \cos L \\
0 & \cos L & \sin L
\end{array} \right)
\left( \begin{array}{r r r}
\cos \left[ \omega (t - t_0) \right] & \sin \left[ \omega (t - t_0) \right] & 0 \\
-\sin \left[ \omega (t - t_0) \right] & \cos \left[ \omega (t - t_0) \right] & 0 \\
0 & 0 & 1
\end{array} \right)
\left( \begin{array}{c} \cos \phi_\odot  \\ \cos \varepsilon \sin \phi_\odot \\ \sin \varepsilon \sin \phi_\odot \end{array} \right)~,
\la{eq:terrestrial}
\eea
where $t - t_0$ is the interval during which the Earth has rotated, measured with respect to a reference time $t_0$.\footnote{Note that the signs of the off-diagonal $\pm \sin[\omega(t-t_0)]$ entries in the corresponding rotation matrix in \Eq{eq:terrestrial} reflect the fact that the Earth's rotation displaces the Sun in an azimuthal direction {\it opposite} to that of the Sun's yearly motion along the ecliptic.  This is the reason why the mean solar day of 24 hours is {\it longer} than the sidereal day of 23.9345 hours: the extra 4 minutes of rotation are need to compensate for the change in $\phi_\odot$ in \Eq{eq:ecliptic}.}  We will discuss how to choose the value of $t_0$ (which will depend on the geographic longitude $\ell$) in \Sec{sec:longitude}.

By \Eq{eq:terrestrial}, the altitude (or ``elevation'') of the Sun above the horizon, as a function of the latitude and the time $t$, is
\bea
\alpha_\odot (L, t) = \frac{\pi}{2} - \theta''_\odot (L, t) &=& \arcsin [z''_\odot (L, t)] \nn
&=& \arcsin \left( - \cos L \cdot \cos [\phi_\odot (t)] \cdot \sin \left[ \omega (t - t_0) \right]
+ \cos L \cdot \cos \varepsilon \cdot \sin [\phi_\odot (t)] \cdot \cos \left[ \omega (t - t_0) \right] \right. \nn
&&  \hskip 1.5 cm \left. +  \sin L \cdot \sin \varepsilon \cdot \sin [\phi_\odot (t)] \right)~.
\la{eq:altitude}
\eea
When $\alpha_\odot = 0$, the Sun is either rising or setting.  When $\alpha_\odot = \pi / 2$, the Sun is directly overhead, at the ``zenith'' (this can occur only at tropical latitudes $-\varepsilon \leq L \leq \varepsilon$).  

Meanwhile, the azimuthal angle $\phi''_\odot$ can be computed from \Eq{eq:terrestrial}, using the relation
\be
\tan \phi''_\odot = \frac{y''_\odot}{x''_\odot}~.
\la{eq:terr-azimuth}
\ee
In \Sec{sec:longitude} we will work out the relation between this $\phi''_\odot$ and the cardinal directions (North, East, South, and West).

\section{Astronomical adjustments}
\la{sec:astronomical}

For some purposes, it may be acceptable to approximate the angle $\phi_\odot (t)$ as increasing linearly and completing a full revolution in one year (as do the authors of \cite{Khavrus}). A more precise expression can be obtained from Kepler's first and second laws of planetary motion, which state that the Earth moves along an ellipse, with the Sun at a focus, while the line segment from the Sun to the Earth sweeps out equal areas in equal times.

\subsection{Equation of the center}
\la{sec:center}

The angle subtended by the line from the Sun to the Earth, with respect to the major axis of the elliptical orbit, is known to astronomers as the ``true anomaly'' and is usually represented by the letter $v$.  Finding $v$ as a function of time has no exact analytic solution,\footnote{Newton offered a rigorous proof that no analytic solution could exist, using concepts now associated with topology, long before topology was invented.  This fascinating proof (the first impossibility proof since the ancient Greeks) is discussed in \cite{Arnold,Chandra}.} but an expansion can be obtained, which converges rapidly for small orbital eccentricity $e$, known as the ``equation of the center:''
\be
v = M + 2 e \sin M + \frac{5}{4} e^2 \sin 2M + \frac{1}{12} e^3 \left( 13 \sin 3M - 3 \sin M \right) + {\cal O}(e^4)~.
\la{eq:center}
\ee
The ``mean anomaly'' in \Eq{eq:center} can be expressed as
\be
M = M_0 + M_1 t~,
\la{eq:mean}
\ee
with constant $M_{0,1}$; it would be equal to the angle $v$ for a perfectly circular orbit ($e=0$) of equal area to the true elliptical orbit; see \cite{Danby}.  The values of $v$ and $M$ in \Eq{eq:center} are measured with respect to the {\it perihelion}, which is the point of closest approach between the Earth and the Sun, as shown in \Fig{fig:orbit}.  The value of $2 \pi / M_1$ is slightly greater than one calendar year because of the slow precessions of the equinoxes and the perihelion, which we will discuss in \Sec{sec:precession}. 

\begin{figure} [htb]
\begin{center}
	\includegraphics[width=0.6 \textwidth]{./images/solar/orbit.pdf}
\end{center}
\caption{\small Diagram of the Earth's orbit around the Sun.  The eccentricity is exaggerated for clarity.  The ``true anomaly'' $v$ is measured from perihelion, which is the point of closest approach between the Sun and the Earth.  The point of greatest separation between Sun and Earth is called the ``aphelion.''  We measure the ecliptic azimuthal angle of the Sun, $\phi_\odot$, from the Earth's position at the time of the first equinox.  Therefore $\phi_\odot = v - v_0$, where $v_0$ is the angular displacement between the perihelion and the equinox.\la{fig:orbit}}
\end{figure}

For our purposes it will be convenient to measure the angle $\phi_\odot$ from the first equinox of the year.  Therefore we let
\be
\phi_\odot = v - v_0 ~,
\la{eq:v0}
\ee
where $v_0$ is the angular displacement between the perihelion and the first equinox, as shown in \Fig{fig:orbit}.  Using the current astronomical data for the parameters $M_0$, $M_1$, $e$, and $v_0$ \cite{orbit-parameters}, we can write the equation of the center for the Earth as:
\be
M(t) = -0.0410 + 0.017202 \, t
\la{eq:mean-numbers}
\ee
and
\be
\phi_\odot (t) = - 1.3411 + M(t) + 0.0334 \sin [M(t)] + 0.0003 \sin [2M(t)]~,
\la{eq:center-numbers} 
\ee
where $t = 0$ corresponds to 1 January 2013, 0:00, Universal Coordinated Time (UTC), and $t$ is measured in mean solar days of 24 hours.

We may see from \Eq{eq:center-numbers} that the correction to $\phi_\odot$ introduced by the eccentricity of the Earth's orbit is small: less than $2^\circ$ at any given time of the year.  For a planet like Mercury, whose orbit is more eccentric and whose rotation is slower than the Earth's, the motion of the Sun in the sky is qualitatively different, as discussed in \citep{Mercury}.

\subsection{Precession of equinoxes and perihelion}
\la{sec:precession}

In the second century BCE, the Greek astronomer Hipparchos of Nicaea found that the positions of the equinoxes moved along the ecliptic (i.e., with respect to the distant stars) by about $1^\circ$ per century (the modern estimate is $1.38^\circ$ per century).  Newton correctly explained this as due to the tidal forces that the Moon and the Sun exert on the Earth, which is not perfectly spherical.  If the Earth did not spin, those tidal forces would pull the Earth's equatorial bulge onto the orbital plane of the corresponding perturbing body (i.e., of either the Moon or the Sun).  The Earth's spinning turns the action of that tidal torque into a {\it precession}, so that the axis of the Earth's rotation describes a cone, and the position of the celestial north pole therefore moves slowly along a circle, with respect to the constellations.\footnote{This slow change of the positions of the poles, equinoxes, and solstices, relative to the distant stars, implies that the signs of the Zodiac are not fixed with respect to the solar calendar.  For example, the ``Tropic of Cancer'' was so named because the position of the Sun at the time of the northern solstice used to lie within the constellation of Cancer, but today the northern solstice actually lies in Taurus.  The first equinox, which used to lie in Aries when the ancient Babylonians developed the calendar, has since shifted to Pisces and will move into Aquarius around the year 2,600.  This last circumstance has been the source of much mystical twaddle about the ``dawning of the Age of Aquarius.''}

The period of the precession of the Earth's axis is about 26,000 years.  Since the recurrence of the seasons depends on the periodicity of the equinoxes, rather than on the actual time it takes the Earth to go once around the Sun, the modern calendar is based on the ``mean tropical year,'' which is shorter than the sidereal year by about $20$ minutes (i.e., 1/26,000 of a sidereal year).

The position of the perihelion with respect to the distant stars also varies, but more slowly, with a period of about 112,000 years, which is equivalent to a displacement of about $0.32^\circ$ per century.  This precession results from perturbations to the motion of the Earth around the Sun caused by the gravitational pull of the Moon and the other planets, and to a lesser extent also by relativistic corrections to Newtonian gravity.\footnote{One of the most convincing early demonstrations of the validity of Einstein's theory of general relativity was that it explained the anomalous precession of the perihelion for the orbit of Mercury, which astronomers had until then failed to account for by the gravitational influence of the known planets; see \cite{dInverno} for a detailed discussion.}

The respective precessions of the equinox and the perihelion proceed in opposite directions along the ecliptic, causing the value of $v_0$ in \Eq{eq:v0} to {\it decrease} by about $1.7^\circ$ per century.\footnote{The quantity $2\pi - v_0$ is known to astronomers the ``longitude of perihelion.''}  Though for our purposes such precision is hardly justified, if we wished to take into account those precessions, we could make $v_0$ in \Eq{eq:v0} a time-dependent parameter.

\section{Duration of daylight}
\la{sec:daylight}

The computation only up to \Eq{eq:polar-sun} suffices to obtain a good estimate of the number of hours of daylight for a given day of the year, if we do not care for the precise time of sunrise and sunset.  Here the main approximation is that that the azimuthal angle of the Sun in the ecliptic frame, $\phi_\odot$, will be taken to be fixed during a given calendar date $d$.  For definiteness, let us say that $\phi_\odot$ is computed at noon for the date and location of interest, the corresponding time being translated to Universal Coordinated Time (UTC), for use in Eqs.~(\ref{eq:mean-numbers}) and (\ref{eq:center-numbers}).

\begin{figure} [t]
\begin{center}
	\includegraphics[width=0.5 \textwidth]{./images/solar/daylight-sphere.pdf} \hskip 1.5 cm
	\includegraphics[width=0.325 \textwidth]{./images/solar/daylight-circle.pdf}
\end{center}
\caption{\small (a): Cross-section of the celestial sphere along the Earth's axis of rotation $P \overline P$, centered at the position $O$ of an observer at geographic latitude $L$.  The point $m$ corresponds to the maximum altitude of the Sun, and $\overline m$ to the minimum altitude.  (b): Cross-section of the celestial sphere, centered at point $a$ and perpendicular to the Earth's axis of rotation.  The point $c$ corresponds to sunrise and $\bar c$ to sunset.  The arrows show the direction in which the celestial sphere rotates with respect to the observer at $O$.\la{fig:daylight}}
\end{figure}

Figure \ref{fig:daylight}(a) shows a cross-section of the celestial sphere, parallel to the Earth's axis of rotation $P \overline P$.  As the sphere rotates about the observer at point $O$, the celestial pole $P$ maintains a fixed altitude, equal to the observer's geographic latitude $L$.\footnote{If we take $L$ to be positive for points on the northern hemisphere of the Earth, then $P$ is the north celestial pole, and $\overline P$ is the south celestial pole.  The opposite convention would be more convenient for observers in the southern hemisphere.}  Point $m$ marks the maximum altitude of the Sun, while point $\overline m$ marks its minimum altitude.

The path of the Sun in the sky corresponds to the circle $am$, shown in \Fig{fig:daylight}(b) (again, as long as we neglect the change in $\phi_\odot$, and therefore also in $\theta'_\odot$, during the course of one day).  This circle is a cross-section of the celestial sphere, perpendicular to the axis $P \overline P$ and parallel to the line $m \overline m$.

In terms of the angle $\delta$ in \Fig{fig:daylight}(b),\footnote{Astronomers call $\delta$ the Sun's ``local hour angle'' at the times of rising and setting.  See, e.g., \cite{Meeus-rising}.} the number of hours of daylight is simply
\be
H = 24 \left( 1 - \frac{\delta}{\pi} \right)~,
\ee
since the Sun moves uniformly along the circle $am$, with a period of 24 hours.\footnote{By making the period of rotation of the Sun about the celestial poles in \Fig{fig:daylight} equal to the mean solar day of 24 hours, rather than the sidereal day of 23.9345 hours, we are taking into account the average change in $\phi_\odot$ during the course of one day.}  Examining Figs.~\ref{fig:daylight}(a) and (b), we see that
\be
\delta = \arccos \frac{ab}{am} = \arccos \left( \tan L \cot \theta'_\odot \right)~.
\la{eq:delta}
\ee
Therefore we can express the number of hours of daylight as a function of geographic latitude and day of the year in the form:
\bea
H(L,d) &=& 24 \left[ 1 - \frac{\arccos \left( \tan L \cot \theta'_\odot (d) \right)}{\pi} \right] \nn
&=& 24 \left[ 1 - \frac{1}{\pi} \arccos \left( \tan L \frac{\sin \varepsilon \sin [\phi_\odot (d)]}
{\sqrt{1 - \sin^2 \varepsilon \sin^2 [\phi_\odot (d)]}} \right) \right]
\la{eq:hours}
\eea
(which agrees with the expression obtained in \cite{Khavrus}).

\begin{figure} [t]
\begin{center}
	\includegraphics[width=0.5 \textwidth]{./images/solar/daylight-hours.pdf}
\end{center}
\caption{\small Number of hours of continuous daylight $H$, as a function of the day of the year $d$ (starting on 1 January), computed using \Eq{eq:hours}, for: the latitude of Cartagena de Indias, Colombia, $10^\circ 24'$ N (red curve); the latitude of Boston, Massachusetts, USA, $42^\circ 21'$ N (blue curve); the latitude of Stockholm, Sweden, $59^\circ 20'$ N (green curve); and the Arctic Circle, $66^\circ 34'$ N (dashed black curve).\la{fig:daylight-hours}}
\end{figure}

Figure \ref{fig:daylight-hours} shows plots of $H$ as a function of the day of the year $d$, at the latitudes of Cartagena de Indias (Colombia), Boston (USA), Stockholm (Sweden), and the Arctic Circle, all in the northern hemisphere.  Note that, for the northern hemisphere, the midyear solstice (which occurs around 21 June, or $d=171$) is always the longest day, whereas it is the shortest day everywhere in the southern hemisphere.  Conversely, the year-end solstice (around 21 December, or $d = 354$) is always the longest day in the southern hemisphere and the shortest in the northern hemisphere.
 
\section{Calculations}

With the theoretical part now behind us, we will revisit and extend our module for the calendar
to include the solar position calculations. We first load the module and then assign a local variable \textbf{doy}, day of year to get the day of the year in a leap year. We will also get the same for a normal year to compare and test our function.

\begin{texexample}{Loading the Lua module}{}
\begin{luacode}
local c = require("i18n.calendar") 
local doy = c.calcDayOfYear(11, 13, true)
tex.print("Day of year for 13 Nov in a leap year", doy, "\\par")
doy = c.calcDayOfYear(11, 13, false)
tex.print("Day of year for 13 Nov in a normal year", doy)
\end{luacode}
\end{texexample}
As expected in a leap year the day number was higher by one day.  This function will be the starting point for most of the calculations that follow. Similarly to the day number the year needs to be expressed fractionally before we can use it in other equations. This is given in radians.

\begin{texexample}{Calculate fractional year}{}
\begin{luacode}
local c = require("i18n.calendar") 
local doy = c.calcDayOfYear(12, 13, false)
local fy = c.fractionalYear(doy, 12)
tex.print("Day of year for 13 Dec in a normal year", doy, "\\par")
tex.print("Fractional year at doy = ",doy, fy.. "rad", math.deg(fy))
\end{luacode}
\end{texexample}

The next two calculations calculate the Equation of Time and the declineation angle.

\begin{texexample}{Calculate equation of time and declneation}{}
\begin{luacode}
local c = require("i18n.calendar") 
local doy = c.calcDayOfYear(12, 13, false)
local fy = c.fractionalYear(doy, 12)
local decl = c.declineation(fy)
local ET   = c.equation_of_time(fy)

tex.print("Day of year for 13 Dec in a normal year", doy, "\\par")
tex.print("Fractional year at doy = ", doy, fy.. " rad", math.deg(fy))
tex.print("Declineation", decl, math.deg(decl))
tex.print("Equation of time", ET)
\end{luacode}
\end{texexample}

\begin{texexample}{Calculate equation of time and declneation}{}
\begin{luacode}
local c = require("i18n.calendar") 
local ET, doy   

for i = 1, 30 do 
    doy = c. calcDayOfYear(i, 0, false)
  ET= c.equation_of_time(doy)
  tex.print(ET .. ', ')
end 
\end{luacode}
\end{texexample}

Now that we have generated all these points, we might make a small diversion and plot them. We will use pgfplots to typeset the chart for us. To plot the equation of time against the day of the year, we first generate the data within a 
\cmd{\luadirect}. 
\bigskip

\begin{scriptexample}{test}{}
\begin{verbatim}
\edef\equationoftime{\luadirect{
  local c = require("i18n.calendar") 
  local ET   
  for fy = 1, 365, 1  do 
    ET= c.equation_of_time(fy)
    tex.sprint('(',fy ..', '.. ET .. ') ')
  end 
}}
\end{verbatim}
\end{scriptexample}
\bigskip

The next step is to write the pgfplots code. We will keep it simple and without a lot of formatting at first.
\edef\equationoftime{\luadirect{
local c = require("i18n.calendar") 
local ET   
for fy = 1, 365, 1  do 
  ET= c.equation_of_time(fy)
  tex.sprint('(',fy ..', '.. ET .. ') ')
end 
}}

\begin{scriptexample}{}{}
\begin{verbatim}
\begin{tikzpicture}
\begin{axis}[height=5cm,
                  width = 0.6\textwidth,
                  xlabel = day number,
                  ylabel = equation of time (min),
                  ]
\addplot[smooth] coordinates {
\equationoftime
};
\end{axis}
\end{tikzpicture}
\end{verbatim}
\end{scriptexample}

Combining the code and inserting it in our document we obtain a nice plot of the equation of time, possibly correct to 12 decimal places.

\begin{figure}[htb]
\begin{scriptexample}{}{}
\bgroup
\centering
\begin{tikzpicture}
\begin{axis}[height=5cm,
                  width = 0.6\textwidth,
                  xmin = 1, xmax=365,
                  ymax = 15,
                  xlabel = day number,
                  enlarge y limits = true,
                  enlarge x limits = true,
                  ylabel = equation of time (min),
                  grid = major
                  ]
\addplot[smooth] coordinates {
\equationoftime
};
\end{axis}
\end{tikzpicture}

\egroup
\caption{Equation of time, generated via LuaTeX mark-up.}
\end{scriptexample}
\end{figure}

The graph of course still needs a lot of polishing. The axes can have better numbering, we may want to add a title and other enhancements. The one modification which we will do is to make the chart a bit wider and get the x-axis numbering a bit better.  

\begin{figure}[htb]
\begin{scriptexample}{}{}
\bgroup
\centering
\begin{tikzpicture}
\begin{axis}[height=5cm,
                  width = \textwidth,
                  xmin = 30, xmax=365,
                  ymax = 15,
                  xtick = {30,60,...,360},
                  xlabel = day number,
                  enlarge y limits = true,
                  enlarge x limits = true,
                  ylabel = equation of time (min),
                  grid = major,
                  grid style ={color = black!50},
                  ]
\addplot[smooth] coordinates {
\equationoftime
};
\end{axis}
\end{tikzpicture}

\egroup
\caption{Equation of time, generated via LuaTeX mark-up.}
\end{scriptexample}
\end{figure}

The advantage of having the data the program and the documentation in one document, has major advantages. The generated plots guaranteed that our methods and functions have been tested properly. Of course once we have such a powerful tool in our disposal we can generate from the same routines tabular figures. 



The problem we need to solve is that we need to print the values in rows, one row for each value of the day number for all the months in a year. We also need to cater for all of TeX's mark-up without breaking up anything. Although at first this looks like a difficult task, now that we have sorted the rads from degrees, we need to go back to our module and handle the table from there. It is always easier to generate TeX commands from Lua rather than working from TeX and exporting and importing to Lua.

But how do we organize the code? 


\begin{scriptexample}{}{}
\bgroup
\scriptsize
\begin{luacode}
local c = require("i18n.calendar") 
c.equation_of_time_table ()
\end{luacode}
\captionof{table}{Equation of Time (ET) values for a normal year. }
\egroup
\end{scriptexample}

Modifying the routines to cater for a leap year is fairly simple, we can introduce a variable leap and modify the \luacmd{equation\_of\_time\_table}. 


\bgroup
\footnotesize
\begin{luacode}
local c = require("i18n.calendar") 
c.debug = true
c.equation_of_time_table (true, 12)
\end{luacode}
\captionof{table}{Equation of Time (ET) values for a leap year. }
\egroup


The coding of the tables (for the normal year and the leap year) although somewhat unexciting, is important to test that the routines are performing as required, as they provide an easy way to examine the output. As you can observe the day number for both the years are correct. 

It is good practice, that when developing code, you include testing routines and if appropriate also logging routines. The leap table was produced using |c.debug = true|  to view the day number, which was programmed as a subscript to the numerical values of the equation of time.

\subsection{declination}

The declineation angle also takes the day of the year parameter and returns the angle. For this we have already
added a function

\begin{scriptexample}{}{}
\begin{verbatim}
\begin{luacode}
local c = require("i18n.calendar")
c.printdeclination()
\end{luacode}
\end{verbatim}
\end{scriptexample}

\subsection{Calculating the Solar Azimuth and Altitude}

Now that we have most of the programming behind us, we will calculate the solar azimuth and altitude angles. These require effectively the calling of most of the functions we have coded so far. Once these are calculated then the sunset, zenith and sunrise can be calculated. These will be required later on to build a model for the solar gains and losses of buildings. 

\subsection{Assembling the Model}

The model specification required that from a given location, defined by its latitude and longitude at a particular month, day and time of the year, we should be able to determine the location of the sun on the celestial sphere.

We define an object |location = {latitude, longitude, name}| 












  \chapter{Variadic Functions}

Variadic functions are present in most modern languages and Lua is no exception. For example, the following function returns the summation of all its arguments\footnote{There is a difference from the way the |arg| was accessed in Lua 5.1 to Lua 5.2}:

\begin{texexample}{Variadic Functions}{ex:variadic}
\begin{luacode*}
function add (...)
	local s = 0
	for i, v in ipairs{...} do
   		s = s + v
	end
	return s
end
tex.print(add(3, 4 ,7,9,10))
\end{luacode*}
\end{texexample}

The interesting part of the function is the three dots notation |(...)| in the parameter list.

The three dots (...) in the parameter list indicate that the function is variadic.
When we call this function, Lua collects all its arguments internally; we call
these collected arguments the extra arguments of the function. A function can
access its extra arguments using again the three dots, now as an expression.
In our example, the expression {...} results in an array with all collected
arguments. The function then traverses the array to add its elements.

This type of  expression |...| is called  a \emph{vararg} expression. It behaves like a multiple
return function, returning all extra arguments of the current function. For
instance, the command \texttt{print(...)} prints all extra arguments of the function. The three dots is just a s

Lua allows to have any number of fixed arguments before the dots. Lua assigns the fixed arguments first and the rest as extra arguments.

\begin{texexample}{Named Arguments}{ex:namedargs}
\begin{luacode*}
function Call(...)
   local arg = {n=select('#',...),...}
  -- arg.n is the real size
  for i = 1,arg.n do
    tex.print(arg[i])
  end
 end

Call("sth", 1, 2, 6, "yo", "test" )	

\end{luacode*}
\end{texexample}


\section{Named Arguments}

The Lua mentality is that it provides tools and not solutions. In languages like |python| or |php| one can create functions with named parameters for example: \texttt{myfunction(a=1,b=10,c=20)}, where the values of the arguments are default values, in case the function is called without them. This is very useful also when functions can have many optional parameters.









  \chapter{Strings}

Strings in Lua represent text. They can contain a single letter or an entire book. Programs that manipulate 1 M characters are not unusual in Lua. \cite{roberto}

\tex never provided any mechanisms for the manipulation of strings and neither did \latex. This is a serious limitation for libraries that have to deal mostly with the manipulation and typesetting of textual data and one of the reasons to use \LUA. 

The double quote characters (") mark the beginning and end of the string. Marking the beginning and
end is all they do; they are not actually part of the string, which is why print doesn't print them, as in
this example:

\begin{texexample}{Strings}{lua:strings}
\directlua{
    tex.sprint("This is a string")
}
\end{texexample}

Like numbers, strings are values, which means they can be assigned to variables. Here's an example

\begin{texexample}{Strings are values}{lua:svalues}
\directlua{
  local  name, phone = "Jane X. Doe", "248-555-5898"
  tex.sprint(name, phone)
}
\end{texexample}

\subsection{Quoting Strings with Square Brackets}

\begin{texexample}{Multiline strings}{lua:multilinestrings}
\directlua{
 tex.sprint([[
    There are some 
    funky  characters in this
    string. 
]])
}
\end{texexample}

Now what do you do if you want to print to square brackets together?

\begin{texexample}{Comments}{}
\ttfamily \directlua{
 tex.sprint([=[
    [[There]] are some 
    funky  characters in this
    string. 
]=])
}
\end{texexample}

\subsection{Escaping Characters}

\begin{texexample}{Escaping characters}{ex:esc1}
\ttfamily \directlua{
 tex.sprint([=[
    [[There]] are some 
    funky \string\\ characters in this \string\\
    string. 
]=])
}
\end{texexample}


\begin{texexample}{Escaping}{ex:esc}
\ttfamily \directlua{
 tex.sprint("2"+21)
}
\end{texexample}

\subsection{Concatenation operator}

\begin{texexample}{Concatenation operator}{ex:concatop}
\ttfamily \directlua{
 tex.sprint("99" ..  21) -- this is a comment
}
\end{texexample}

\section{Comments}
\begin{texexample}{comments}{ex:comment}
\ttfamily \directlua{
 tex.sprint("99"..21) -- this is a comment
}
\end{texexample}


\section{The kpse library}

This library provides two separate, but nearly identical interfaces to the kpathsea file search
functionality: there is a \enquote{normal} procedural interface that shares its kpathsea instance with
LuaTEX itself, and an object oriented interface that is completely on its own.

\subsection{find\textunderscore file}

The most often used function in the librar is \textsf{find\_file}:

\chapter{Lua Strings Library}

\section{Introduction}


The \LUA string library in \LUA\tex has been extended to provide a number of additional functions: a very useful function is \luacmd{string.explode} which returns a table with the exploded string. 


\begin{docLua}{string.len(string)} 
can be used to find the length of a string.
The |string.len| function is not unicode aware and the slunicode library which be used for cases where utf characters above unicode codepoint 127 is required.
\end{docLua}


\begin{docLua}{string.upper(argument)}
Returns a capitalized representation of the argument.
\end{docLua}
\begin{texexample}{Finding the length of a string}{}
We can get the length of a string using the length operator (denoted by \#):
\begin{luacode}
a = "hello"
tex.print(#a) --> 5
tex.print(#"good bye") --> 8
\end{luacode}
\end{texexample}



\begin{docLua}{string.lower(argument)}
Returns a lower case representation of the argument.
\end{docLua}

\begin{docLua}{string.gsub(\meta{mainString},findString,replaceString)}
Returns a string by replacing occurrences of findString with replaceString.
\end{docLua}

\begin{docLua}{string.reverse(arg)}
Returns a string by reversing the characters of the passed string.
\end{docLua}

\begin{docLua}{string.char(arg)} 
and string.byte(arg)
Returns internal numeric and character representations of input argument.
\end{docLua}

\begin{docLua}{string.rep(string, n))}
Returns a string by repeating the same string n number times.
\end{docLua}



\begin{docLua}{string.format(format,string,...,stringn)}
Returns a formatted string.
\end{docLua}


\begin{texexample}{Formatting strings}{}
\begin{luacode}
string1 = "Lua"
string2 = "Tutorial"
number1 = 10
number2 = 20
-- Basic string formatting
print(string.format("Basic formatting %s %s",string1,string2))

-- Date formatting
date = 2; month = 1; year = 2014
print(string.format("Date formatting %02d/%02d/%03d", date, month, year))

-- Decimal formatting
print(string.format("%.4f",1/3))
\end{luacode}
\end{texexample}




\begin{texexample}{Exploding strings}{}
\begin{luacode}
local z = string.explode("abcd", " ")
local   writeln = function (v) 
   return   tex.print("key = ", v, "\\par")
end

for _,v in ipairs (z) do
  writeln(v)
end

tex.print(unicode.utf8.len("résumé"))
\end{luacode}
\end{texexample}

If you notice in the example above the space at the end of "d" was captured and inserted into the return table. The string.trim trims a string and removes and spaces from the beginning and the end of a string.

\begin{texexample}{Trimming strings}{}
\begin{luacode*}


string.trim = function (s)
   return (string.gsub(s, "^%s*(.-)%s*$", "%1"))
end

-- string.trim = trim

local  writeln = function (v) 
   return   tex.print("key = ", v, "\\par")
end

local str = " a b c d e "

tex.print(str:trim(str).."zzzz\\par")

local z = string.explode(str, " ")
  
for _,v in ipairs (z) do
  writeln(v)
end
\end{luacode*}
\end{texexample}

One of the advantages of Lua, is that any library is just a table and hence can be extended easily. In the example above we extended the string library by adding the \textbf{line string.trim = trim}.  




\begin{texexample}{Finding the length  of a string}{}
\begin{luacode*}
 local z = "ąΒβ"
 tex.print(string.len(z),"\\par")
 tex.print(unicode.utf8.len(z))
 str="äöABCäö"
 i = font.id("arial")
 tex.print(i, font.current(), str)
\end{luacode*}
\end{texexample}

For more details on the rest of the available methods, see the Lua and LuaLaTeX manuals.
    
  \chapter{Pattern Matching}

The |string| library of Lua, offers a number of pattern matching functions. Unlike many other languages it does not offer a full regular expression library. One can take this at first to be a disantvantage, but the reasons are explained in \textit{Programming in Lua} by Roberto Ierusalimschy, clearly:

\begin{quote}
Unlike several other scripting languages, Lua uses neither \texttt{POSIX} regex nor
Perl regular expressions for pattern matching. The main reason for this decision
is size: a typical implementation of \texttt{POSIX} regular expressions takes more than
4000 lines of code. This is about the size of all Lua standard libraries together.
In comparison, the implementation of pattern matching in Lua has less than
600 lines. Of course, pattern matching in Lua cannot do all that a full POSIX
implementation does. Nevertheless, pattern matching in Lua is a powerful tool,
and includes some features that are difficult to match with standard POSIX
implementations.
\end{quote}

\CMDI{string.find()}
searches for a pattern inside a given subject string. The simplest form of a \textit{pattern} is a word, which matches only a copy of itself.

For instance, the pattern ‘Lua’ will search for the substring “Lua” inside the
subject string. When |find| finds its pattern, it returns two values: the \textit{index} where the match begins and the index where the match ends. If it does not find a match, it returns |nil|. Example~\ref{ex:patterns1} demonstrates the |find| function. We use the |luacode| environment to avoid dealing with catcodes and we format the results in maths mode.

\begin{texexample}{Pattern Matching with find}{ex:patterns1}
\begin{luacode*}
s = "hello Lua world"
i, j = string.find(s,"Lua")
tex.print("indices $i="..i..",".."j="..j.."$")
\end{luacode*}
\end{texexample}

\CMDI{string.sub}
When a match succeeds, we can call string.sub with the values returned by
string.find to get the part of the subject string that matched the pattern. For
simple patterns, this is the pattern itself.
The string.find function has an optional third parameter: an index that
tells where in the subject string to start the search. This parameter is useful
when we want to process all the indices where a given pattern appears: we
search for a new match repeatedly, each time starting after the position where
we found the previous one. As an example, the following code makes a table
with the positions of all newlines in a string:

We will see later a simpler way to write such loops, using the string.gmatch
iterator.

\begin{texexample}{Iterating over all matches}{}
\edef\tempstring{\string\\par..is this is a is test \string\\par and is this another?}
\begin{luacode}
local s = "\tempstring".."This is is a string and this is another"
local t = {}
local i= 0
while true do
    i = string.find(s, "\\par", i+1)
    if i == nil then break end
    t[#t + 1] = i
    tex.print(i)
end
   
\end{luacode}
\end{texexample}

\section{Patterns}

Patterns can be more useful with \textit{character classes}. 

\begin{longtable}{>{\color{blue}}ll}
. & all characters\\
\%a & letters\\
\%c &control characters\\
\%d &digits\\ 
\%g &printable characters except spaces\\
\%l &lower-case letters\\
\%p &punctuation characters\\
\%s &space characters\\
\%u  &upper-case letters\\
\%w  &alphanumeric characters\\
\%x  &hexadecimal digits\\
\end{longtable}




 
  \newfontfamily\hiero{NotoSansEgyptianHieroglyphs-Regular.ttf}
\directlua{Start = 0}

           
\chapter{Lua Modules}

\epigraph{People think that computer science is the art of geniuses but the actual reality is the opposite, just many people doing things that build on each other, like a wall of mini stones.}{---Donald Knuth}


\section{Modularity}

Modularity is a fundamental requirement for any computer language, as it enables the development of programs using sound software engineering principles, but most importantly it allows different people to work on the codebase simultaneously. Lua is no exception. Luas modules are easy to develop and when they are all related, they are disributed as a package \textit{package}.

\section{The require Function}

In Lua the \luacmd{require} function treats a module as any code that defines some values, such as table or functions. 

To load  a module, we simply call \textbf{require"modulename"}. The first step of the \textbf{require} function is to check in table \textbf{package.loaded} whether the module is already loaded.

If the module is not loaded Lua will then search for a file name with the module name. If it finds it, it loads it with |loadfile|. 


\begin{texexample}{Findind a module}{ex:modulesearch}
\begin{luacode*}
s = string.gsub(package.cpath,"\\", "\\textbackslash ")
s = string.gsub(s,";", ";\\par ")

tex.print(s)
\end{luacode*}
\end{texexample}

In the example we use the |string.gsub| to replace windows \textbackslash to a harmless version, so we do not need to worry about catcodes, a quick and dirty solution. 

\section{Writing Modules in Lua}

The simplest way to create a module in Lua is really very simple: create a table and put all functions you want to export inside it, and return this table.

\begin{texexample}{Writing a Module}{ex:wmodule}
\begin{filecontents*}{chicken.lua}
local M = {}

function M.chicken ()
  return "chicken"
end

function M.chickens ()
  return "many chickens" 
end

function M.ancient_chickens ()
  return "\\bgroup\\hiero\\char\"13171 \\egroup"
end

return M
\end{filecontents*}

\begin{luacode*}
local c = require "chicken"
      tex.print(c.chicken(), c.chickens(), c.ancient_chickens())
\end{luacode*}
\end{texexample}

A second and in my opinion much better way is to return the list of functions you want to export. This way\footnote{This is very similar to Javascript modules.} your code will be much more cleaner and easier to maintain. We rewrite our module |chicken|  to |chickens|

\begin{texexample}{Writing a Module}{ex:wmodule}
\begin{filecontents*}{chickens.lua}
local M = {}

function M.chicken ()
  return "chicken"
end

function M.chickens ()
  return "many chickens" 
end

function M.ancient_chickens ()
  return "\\bgroup\\hiero\\HUGE\\char\"13171 \\egroup"
end

return {chicken           = chicken,
        chickens          = chickens,
        ancient_chickens  = ancient_chickens}
\end{filecontents*}

\begin{luacode*}
   local c = require "chicken"
   tex.print(c.chicken(), c.chickens(),"\\par","ancient chickens", "{\\Huge" .. c.ancient_chickens() .. "}")
\end{luacode*}
\end{texexample}



The Lua Manual also describes a way of eliminating the return statement, by assigning it directly into |package.loaded|:

\begin{verbatim}
local M = {}
package.loaded[...] = M
\end{verbatim}

writing the final return always results in clearer code and is preferable.

\section{Using Environments}

One drawback of those basic methods for creating modules is that it is all too easy to pollute the global name space, for instance by forgetting a local in a private declaration.

\section{Submodules and Packages}

Lua allows module names to be hierarchical, using a dot to separate name levels. For instance, a module named |mod.sub| is a \textit{submodule} of |mod|. A \textit{package} is a complete tree of modules; it is the unit of distribution in Lua. 

From the point of view of Lua, submodules in the same package have no explicit relationship. requiring a module a does not automatically load any of its submodules; similarly, requiring |a.b| does not automatically load |a|. Of course, the package author can create these links as she wants. For example, a particular module may start by explicitly requiring one or all of its submodules.

In LuaTeX things are much simpler if everything falls into place and the LuaTeX searchers find the modules. I experimented a bit, before getting the paths right on a MikTeX installation. Normal modules need to be located in a place where 

\begin{texexample}{Lua Modules and Paths}{}
\begin{filecontents*}{test.config}
    # test.config
    # Read timeout in seconds
    read.timeout=10
    # Write timeout in seconds
    write.timeout=5
   #acceptable ports
   ports = 1002, 1003, 1004
\end{filecontents*}
\begin{luacode}
  -- readconfig.lua
  
local config       = require 'pl.config'
local t              = config.read 'test.config'

tex.print("read time out = " .. t.write_timeout, "\\par",
            "ports = " .. "ports" .. "\\par")

 local Map = luatexbase.require_module("pl.Map")   
 Map = require("pl.Map")
   m = Map{one=1, two=2}
   m:update{three=3, four=4,two=20}
   for k,v in pairs (m) do
       tex.print(k .. " = " .. v .. "\\par")
   end 
   
 function search (name)
    altname = string.gsub(name, '.', '\\')
    filename = kpse.find_file(altname, 'lua')	
    if not filename then
      filename = kpse.find_file(name, 'lua')
    end
    if not filename then
      return string.format("[kpse lua searcher] file not found: %s", name)
    else
      return "found" -- loadfile(filename)
    end
end
local z = search ("Map")

tex.print(z)
local r = kpse.show_path("lua")
tex.print(r)
\end{luacode}
\end{texexample}



LuaTeX uses the kpse library to search for files and naturally it is at home, when the files are located somewhere wher kpse can find them. In the example below, I have downloaded the penlight library into the scripts folder. The Map file can be found in:

\begin{texexample}{lookup for files}{}
\begin{luacode}
local f = kpse.lookup("pl/Map", {format = "lua"})

local f1 = kpse.lookup("Map", {format = "lua"})
tex.print("\\par",f)
tex.print("\\par",f1)
\end{luacode}
\end{texexample}

However, most Lua libraries use the dot notation as described above.

\section{The luatexbase package}

Lua's standard function require() is similar to TEX’s \cmd{\input} primitive but is somehow more
evolved in that it makes a few checks to avoid loading the same module twice. In the TEX
world, this needs to be taken care of by macro packages; in the LATEX world this is done by
\cmd{\usepackage.}

But \usepackage also takes care of many other things. Most notably, it implements a
complex option system, and does some identification and version checking. The present package
doesn’t try to provide anything for options, but implements a system for identification and
version checking similar to LATEX’s system.

It is important to note that Lua’s unction module() is deprecated in Lua 5.2 and should be
avoided. For examples of good practices for creating modules, see section 1.4. Chapter 15 of
Programming in Lua, 3rd ed. discusses various methods for managing package

The package which was developed by Heiko Oberdiek Élie Roux, Manuel Pégourié-Gonnard, Philipp Gesang∗
provides the function 

|luatexbase.require_module(name [, required date])|

which can be used as a replacement to \luacmd{require()}. The only difference between them being that the luatexbase.require()  will check that the module properly identifies itself.

\begin{texexample}{luatexbase module}{}
\begin{filecontents*}{dice.lua}
local err, warn, info, log = luatexbase.provides_module({
     name      = 'dice',
     date       = '2014/11/01',
     version   = '0.0',
     description = 'simulating a die',
     author      = 'Y Lazarides',
     license      =  'LPPL v1.3+' ,
})

local dice =  {}    -- is this better to be local?

function dice.txprint()
    return 'dice'
end    

return dice         -- return the table returns {} 
\end{filecontents*} 

\begin{luacode}   
local  f = require("dice")           
        tex.print('return from dice module = ', f.txprint())
\end{luacode}

\end{texexample}

Working through a document is a hostile environment, and please do not follow my example. It is best to
load the files and develop them individually. However, it is prone with errors either way and one can easily get confused with local and global variables. Perseverance and patience are required, and the frustrations of TeX will diminish.

\section{Installing the penlight package}

The penlight package can provides some very useful routines and we will use it as an example
to understand the module system.

The easiest way to get the package is to use git, as the package can be found at Github. Use |git clone https|  to clone the directory into |C:/Program Files/MikTeX 2.9/scripts/|. The |scripts| folder is used for normal packages, whereas a similar path exists for |.dlls|, which are placed in a bin directory. The path system is a bit of a mess and it requires a few attempts before everything falls into place. 

\begin{texexample}{Using the penlight library}{}
\begin{luacode}
local Map = require("pl.Map")
       m = Map{document="article", 
                     page = 2}
       m:update {font = "Calibri", fontsize = "12pt"}               
       tex.print("\\{\\par")
       for k,v in pairs (m)  do
           tex.print('~~~~',k,' = ', v..',', "\\par")
       end   
       tex.print("\\par\\}\\par") 
\end{luacode}
\end{texexample}



\section{Setting up a Database}

Apache foundation gave to the world many gifts and one of the best is couchdb. The database is especially suited for documents and this is my attempt to integrate it with LuaTeX. Installing couchdb is normally non-eventfull and I also suggest that you get an account at iriscouch.db in order to synchronize your local installation with a remote. Computing is converging to a paradigm where all software can talk to each other remotely. Another possibility is to use squlite for a more structured approach. Before we can use the couchdb we will need  a number of libraries to communicate with the db. First we will load the prerequisites, which are |logging| and a json library. 

\subsection{The logging package}

For windows navigate to |C:\Program Files\MiKTeX 2.9\scripts| and clone the repository for the logging module into logging. 

|git clone https://github.com/Neopallium/lualogging.git  logging|


 
\begin{texexample}{Using the logging module}{}
\begin{luacode}
local logging = require "logging"

local socket = require"logging.socket"

local logger = logging.new(function(self, level, message)
                             tex.print(level, message)
                             return true
                           end)
                           
logger:setLevel (logging.WARN)
logger:log(logging.INFO, "sending email")

logger:info("trying to contact server")
logger:warn("server did not responded yet")
logger:error("server unreachable")

-- dump a table in a log message
local tab = { a = 1, b = 2 }
logger:debug(tab)

-- use string.format() style formatting
logger:info("val1='%s', val2=%d", "string value", 1234)

-- complex log formatting.
local function log_callback(val1, val2)
	-- Do some complex pre-processing of parameters, maybe dump a table to a string.
	return string.format("val1='%s', val2=%d", val1, val2)
end
-- function 'log_callback' will only be called if the current log level is "DEBUG"
logger:debug(log_callback, "string value", 1234)

x = os.getenv('PATH')
s = string.gsub(x,"\\", "\\textbackslash ")

tex.print(s)

require("lualibs.lua")

local libloaded = require("lualibs-util-jsn")

local tmp = [[ { "a" : true, "b" : [ 123 , 456E-10, { "a" : true, "b" : [ 123 , 456 ] } ] } ]]

 tmp = utilities.json.tolua(tmp)
 
 
 -- use our own table prettifier
 
 local utils = require("phd-utils")
 
 utils.inspect(tmp)
 
 for k,v in pairs (tmp) do
    tex.print(k, " \\par ")
 end
 
inspect(tmp)
tex.print(inspect(temp))

tmp = utilities.json.tostring(tmp)

tex.print(tmp, " \\par ")


-- tmp = json.tolua(tmp)
-- inspect(tmp)
-- tmp = json.tostring(tmp)
-- inspect(tmp)

-- inspect(json.tostring(true))


local dim =   require("lualibs-util-dim")

local dimstr = string.todimen("10.0mm")

local txprint =function (v)
     				tex.print (v, '\\par')
                  end

dimstr = dimen "10pt" + dimen "20pt" + dimen "200pt" - dimen "100sp" / 10 + "20pt" + "0pt"

txprint(dimstr)

dimstr = dimen "10pt" + dimen "20pt" + dimen "200pt" - dimen "100sp" 

txprint(dimstr)

require("lualibs-gzip") 

str ="Some text that we have put in a file and zipped it."

gzip.save ("test.gz", str)

str  = gzip.load("test.gz")


txprint(str)
\end{luacode}

\end{texexample}

\section{Libraries from ConTeXt}


\subsection*{Dir}

The dir library uses functions of the lfs library that is linked into LuaTEX.
current

This returns the current directory:

\subsection*{dir.current()}

\directlua{Stop = os.runtime()
             diff = Start - Stop 
             tex.print(Stop, os.clock())

tex.print(os.which("ps2pdf"))}


\def\function#1{\leavevmode\noindent{\color{teal}
\parindent0pt\leavevmode\par \bfseries #1 }}
 
 
\function{os.which(\meta{filename})}  This function returns the path of a file and emulates the kpse library function, with similar results. However, the function looks up the local environment path, so if a file is on the path in can find it.

\function{os.where()} This is an alias for \textbf{os.which()}

\begin{filecontents*}{phd-utils.lua}
-- presents nicely a table 

local M = M or {}

local rep, write = string.rep, tex.print

function M.inspect (tab, offset)
   local openbracket, closebracket, par = "\\{", "\\mbox{..}\\}", "\\par"
   
    offset = offset or ""
    for k, v in pairs (tab) do
        local newoffset = offset .. "\\mbox{~~}"
        if type(v) == "table" then
           write(offset .. k .. " = " .. openbracket .. par)
           M.inspect(v, newoffset)
           write(offset .. closebracket .. par)
        else
         if k~="data" then write(offset..k.." =  ".. tostring(v), "\\par") 
           else
                 write(offset.."k = char data ")
           end
       end
    end
end

return M
\end{filecontents*}

\section{phd package utilities}

\function{inspect()} Typesets a Lua table. 

\begin{texexample}{inspecting tables}{}
\begin{luacode}
local utils = require("phd-utils")
local inspect = utils.inspect

inspect({a="b", c={d="man", e ="woman"}})
\end{luacode}
\end{texexample}



  \chapter{Lua Objects}
\label{c:luaobjects}

\section{Class-like tables}
Lua is similar to JavaScript in that the concept of class is not directly supported by the language. In fact, Lua has a very general mechanism for extending the behaviour of tables which makes it straightforward to implement classes. A table’s behaviour is controlled by its metatable. If that metatable has a \cmd{__index} function or table, this will handle looking up anything which is not found in the original table. A class is just a table with an \cmd{__index} key pointing to itself. Creating an object involves making a table and setting its metatable to the class; then when handling obj.fun, Lua first looks up fun in the table obj, and if not found it looks it up in the class. obj:fun(a) is just short for obj.fun(obj,a). So with the metatable mechanism and this bit of syntactic sugar, it is straightforward to implement classic object orientation.

Lua doesn't have any built-in mechanism for classes. They are created through the use of tables
and metatables. A class works as a mold for the creation of objects. Most object-oriented languages
offer the concept of class. In such languages, each object is an instance
of a specific class. Lua does not have the concept of class; each object defines
its own behavior and has a shape of its own. Nevertheless, it is not difficult to
emulate classes in Lua, following the lead from prototype-based languages like
Self and NewtonScript. In these languages, objects have no classes. Instead,
each object may have a prototype, which is a regular object where the first object
looks up any operation that it does not know about. To represent a class in
such languages, we simply create an object to be used exclusively as a prototype
for other objects (its instances). Both classes and prototypes work as a place to
put behavior to be shared by several objects
\emphasis{Animal}
\begin{texexample}{Class-like}{}
\begin{luacode}
local Animal = {}                                  -- 1. 

function Animal:new()                         -- 2.
  newObj = {sound = 'woof'}               -- 3.
  self.__index = self                      -- 4.
  return setmetatable(newObj, self)     -- 5.
end

function Animal:makeSound()             -- 6.
  tex.print('I say ' .. self.sound)
end

Dog = Animal:new()                            -- 7.
Dog:makeSound()  -- 'I say woof'         -- 8.

function Dog:makesound()
  s = self.sound .. ' '
  tex.print(s .. s .. s )
end

Dog:makesound()
  
\end{luacode}
\end{texexample}

Unsurprisingly Lua uses tables to create class-like objects. In the example we will create a class Animal and then an instance of a Dog. The function \luacmd{tablename:fn()} is the same as \luacmd{tablename.fn(self,...)}. The ‘:’ just adds a first argument to the function called \textit{self}.  The \luacmd{newObj} will be an instance of class \textbf{Animal}. The self = Animal, but inheritance ca change it. 


If needed, a subclass's new() is like the bases's:

\begin{verbatim}
function LoudDog:new()
  newObj = {}
  -- set up newObj
  self.__index = self
  return setmetatable(newObj, self)
end
\end{verbatim}

\begin{texexample}{A shapes librayy}{}
\begin{luacode}
-- Meta class

Shape = {area = 0}
-- Base class method new

function Shape:new (o, side)
  o = o or {}
  setmetatable(o, self)
  self.__index = self
  side = side or 0
  self.area = side*side;
  return o
end

-- Base class method printArea
function Shape:printArea ()
  tex.print("The area is ", self.area)
end

-- We can extend the shape to a square class as shown below.

Square = Shape:new()
-- Derived class method new

function Square:new (o,side)
  o = o or Shape:new(o,side)
  setmetatable(o, self)
  self.__index = self
  return o
end

local square = Square:new(nil, 8)
square:printArea()
\end{luacode}
\end{texexample}

A common error when calling  methods is to forget to call them with (:) syntax. 

\section{Naming conventions and style}

Classes have names that start with a capital letter. Methods should be preferably all lower case. 

\section{Using a library}

Many libraries, such as penlight offer convenient modules to create classes and avoid developing them from scratch. The example that follows is from its documentation.

\begin{texexample}{Using penlight}{}
\begin{luacode}

-- animal.lua

class = require 'pl.class'

class.Animal()

function Animal:_init(name)
    self.name = name
end

function Animal:__tostring()
  return self.name..': '..self:speak()
end

class.Dog(Animal)

function Dog:speak()
  return 'bark'
end

class.Cat(Animal)

function Cat:_init(name,breed)
    self:super(name)  -- must init base!
    self.breed = breed
end

function Cat:speak()
  return 'meow'
end

class.Lion(Cat)

function Lion:speak()
  return 'roar'
end

fido = Dog('Fido')
felix = Cat('Felix','Tabby')
leo = Lion('Leo','African')

tex.print(fido:speak(), leo.breed, fido.name)
\end{luacode}
\end{texexample}










  \def\lstlua#1{\bgroup\noindent
     \color{blue}
     \textsf{#1}\par
     \egroup}
\chapter{TeX without TeX}

Many developers that are programming LuaTeX routines have a preference to  build the output using Lua rather than commands such as |tex.print|. This is a much more difficult task than simply using \tex, but as it opens the internals of the \tex building procedures it is much more powerful.

To successfully carry out such tasks, a good knowledge of \tex internals is necessary as well as a good understanding of Lua. It took me a while to become familiar with both, and a good way to start is to first visualize what is going on. TeX is building boxes all the time and it really does not care what is going inside them. All these boxes and boxes withing boxes as well as other information is stored in \textit{linked lists}, holding records. This type of data structures are very fast while processing them and was the standard way to program in Pascal or C. This approach when first used it appears as if we are using a sledge-hammer to crack a nut\footnote{\url{http://tug.org/TUGboat/tb33-1/tb103gundlach.pdf}}. The core concept of such programming is to create the fundamental data structures \tex uses internally for representing a glyph, a rule, a glue, a |whatsit| and all other items described in the \texbook. For example the digit `0' could be represented by a table with these entries:
\bigskip

\begin{center}
\begin{tabular}{ll}
\toprule
entry & value\\
\midrule
id  &37\\
char &47\\
font &15\\
lang &0\\
\bottomrule
\end{tabular}
\end{center}

There are other optional entries in that table, but only the \textit{prev} and \textit{next} entries are required for building more complex data structures.

\bigskip
Patrick Gundlach has written a number of nice tutorials describing the use of nodes. We will start with a simple example.  We can visualize these nodes, in a very roundabout way, by using a utility that has been provided by Patrick Gundlach's\footnote{\protect\url{https://gist.github.com/pgundlach/556247/}} code and the GraphViz programme. For example  |\setbox0\hbox{a}|. If you are familiar with link lists, these will be familiar to you. 


\begin{figure}[ht]
\centering
\includegraphics[scale=.6]{./images/single-node.pdf}
\caption{A single node representation, holding a glyph.}
\end{figure}

To construct a horizontal box with a glyph a call to |node.hpack()| is sufficient:
\begin{verbatim}
 hbox = node.hpack(n)
\end{verbatim}

The above code is the same as if we have written |\hbox{0}| except that at this point it is kept in \tex's memory and is not typeset yet. It gets more complex if you want to put more than one item and place them in a box. We will need to create the nodes and chain them together. As you can observe from Figure every node has a |prev|  and a |next| table entries


It might puzzle you as to why the language is shown, but \tex ships this information with every box while building a horizontal list. More information can also be gleaned from the package \pkgname{lua-visual-debug}, also developed by Patrick Gundlach. 

\begin{figure}[ht]
\centering
\includegraphics[width=0.9\textwidth]{./images/Nodes_in_sample_paragraph.png}
\caption{Nodes in a sample paragraph. From  \protect\href{http://wiki.luatex.org/images/Nodes_in_sample_paragraph.png}{luatex wiki}}
\end{figure}



\begin{texexample}{Building nodes programmatically}{}
\begin{luacode*}
local g = node.new("glyph")
g.font = font.current()
g.lang = tex.language
g.char = 86 -- V

local hbox = node.hpack(g)
local vbox = node.vpack(hbox)

node.write(vbox)
\end{luacode*}
\end{texexample}

\begin{figure}[ht]
\centering
\includegraphics[scale=.6]{./images/double-node.pdf}
\caption{A single node representation, holding a glyph.}
\end{figure}


If we want to put two glyphs in the horizontal list we can use the |next| and previous pointers to do it.

\begin{texexample}{Building nodes programmatically}{}
\begin{luacode*}
local g1 = node.new("glyph")
g1.font = font.current()
g1.lang = tex.language
g1.char = 86

local g2 = node.new("glyph")
g2.font = font.current()
g2.lang = tex.language
g2.char = 97

g1.next = g2
g2.prev = g1

local hbox = node.hpack(g1)
local vbox = node.vpack(hbox)

node.write(vbox)
\end{luacode*}
\end{texexample}

The glyphs $g1$ and $g2$ are chained together by setting the next and prev pointer to each other. If they were not connected, only glyph g1 gets into the |hbox|.

If you take a close look at the PDF, you see that the two glyphs are too far away, a (negative) kern should be inserted. This can be done by inserting a kern-node manually or you can ask TeX to do that for you. The last lines of the example above should then read:

Once the nodes were build they were appended onto \tex's \textit{current list}. Note that according to the LuaTeX manual these are deep copied and there is no error checking.

\lstlua{node.write(n)}

The nodes were created using:

\lstlua{n = node.new(id)}

The |id| is a number or it can be matched to a  string such as "glyph"

\section{What type of nodes are available?}




\tex's nodes are represented in \LUA as userdata object with a variable set of fields. In the following
syntax tables, such the type of such a userdata object is represented as \meta{node}.

The current return value of node.types() is: hlist (0), vlist (1), rule (2), ins (3), mark (4),
adjust (5), disc (7), whatsit (8), math (9), glue (10), kern (11), penalty (12), unset (13), style
(14), choice (15), noad (16), op (17), bin (18), rel (19), open (20), close (21), punct (22), inner
(23), radical (24), fraction (25), under (26), over (27), accent (28), vcenter (29), fence (30),
math\_char (31), sub\_box (32), sub\_mlist (33), math\_text\_char (34), delim (35), margin\_kern
(36), glyph (37), align\_record (38), pseudo\_file (39), pseudo\_line (40), page\_insert (41),
split\_insert (42), expr\_stack (43), nested\_list (44), span (45), attribute (46), glue\_spec
(47), attribute\_list (48), action (49), temp (50), align\_stack (51), movement\_stack (52),
if\_stack (53), unhyphenated (54), hyphenated (55), delta (56), passive (57), shape (58), fake
(100),.

NOTE: The |\lastnodetype| primitive is ε-\tex compliant. The valid range is still -1 .. 15 and glyph
nodes have number 0 (used to be char node) and ligature nodes are mapped to 7. That way macro
packages can use the same symbolic names as in traditional ε-TEX. Keep in mind that the internal node
numbers are different and that there are more node types than 15.

\setbox0=\hbox attr0 = 5 attr10 = 45 {contents}

\directlua{
  local attr = tex.box[0].attr.next
  while attr do
    texio.write_nl(attr.number .. " = " .. attr.value)
    attr = attr.next
  end
}

A very useful article on understanding how paragraphs are build is Paul Isambert's article \textit{What it takes to make a paragraph.}

\luadirect{
local GLYF = node.id("glyph")
function show_nodes (head)
    local nodes = ""
    for item in node.traverse(head) do
    local i = item.id
if i == GLYF then
   i = unicode.utf8.char(item.char)
end
nodes = nodes .. i .. "x "
end
tex.print(nodes)
end
}

\begin{luacode*}
local nodenew = node.new
local nodecopy = node.copy
local nodetail = node.tail
local nodeinsertbefore = node.insert_before
local nodeinsertafter = node.insert_after
local noderemove = node.remove
local nodeid = node.id
local nodetraverseid = node.traverse_id
local nodeslide = node.slide

Hhead = nodeid("hhead")
RULE = nodeid("rule")
GLUE = nodeid("glue")
WHAT = nodeid("whatsit")
COL = node.subtype("pdf_colorstack")
GLYPH = nodeid("glyph")
color_push = nodenew(WHAT,COL)
color_pop = nodenew(WHAT,COL)
color_push.stack = 0
color_pop.stack = 0
color_push.command = 1
color_pop.command = 2
chicken_pagenumbers = true

boustrophedon = function(head)
  rot = node.new(8,8)
  rot2 = node.new(8,8)
  odd = true
    for line in node.traverse_id(0,head) do
      --if odd == false then
        w = line.width/65536*0.99625 -- empirical correction factor (?)
        rot.data  = "-1 0 0 1 "..w.." 0 cm"
        rot2.data = "-1 0 0 1 "..-w.." 0 cm"
        line.head = node.insert_before(line.head,line.head,nodecopy(rot))
        nodeinsertafter(line.head,nodetail(line.head),nodecopy(rot2))
        odd = true
      --else
        --odd = false
      --end
    end
  return head
end
\end{luacode*}

\def\boustrophedon{
  \directlua{luatexbase.add_to_callback("post_linebreak_filter",boustrophedon,"boustrophedon")}}
\def\unboustrophedon{
  \directlua{luatexbase.remove_from_callback("post_linebreak_filter","boustrophedon")}}

\boustrophedon

\lipsum[1-2]

\unboustrophedon

\lorem


  
  \chapter{Lua i/o}
\label{ch:luaio}


The I/O library is used for reading and manipulating files in Lua. Like most programming languages there are two kinds of file operations in Lua namely implicit file descriptors and explicit file descriptors.

For the following examples, we will use a sample file test.lua as shown below.

\begin{phdverbatim}
-- sample test.lua
-- sample2 test.lua
\end{phdverbatim}

A simple file open operation uses the following statement.


\begin{docLua}{io.open (filename [, mode])}
A simple file open operation uses the following statement.\\
file = io.open (filename [, mode])
\end{docLua}

The various file modes are listed in the following table.

\begin{longtable}{ll}
Mode	& Description\\
\enquote{r}	& Read-only mode and is the default mode where an existing file is opened.\\
"w"	& Write enabled mode that overwrites the existing file or creates a new file.\\
"a"	& Append mode that opens an existing file or creates a new file for appending.\\
"r+"	& Read and write mode for an existing file.\\
"w+"	& All existing data is removed if file exists or new file is created with read write permissions.\\
"a+"	& Append mode with read mode enabled that opens an existing file or creates a new file.\\
\end{longtable}

\section{Implicit file descriptors}

Implicit file descriptors use the standard input/ output modes or using a single input and single output file. A sample of using implicit file descriptors is shown below.

\begin{texexample}{I/O Operations}{ex:iolua}
\begin{luacode}
fileHandle =io.open("file.txt", "w+")
tex.print(io.type(fileHandle).."\\par")
fileHandle:write("Hello world")
io.close(fileHandle)
tex.print(io.type(fileHandle).."\\par")
\end{luacode}
\end{texexample}

\begin{docLua}{io.flush()}
This function runs |file:flush()| on the default file. When a file is in buffered mode, the data is not
written to the file immediately; it remains in the buffers and is written when the buffer nears getting
full. This function forces the buffers to write to file and clear up.
\end{docLua}

\begin{docLua}{io.input([file])}
This function returns the current default input file when called without any parameters. The parameter
passed can be either a filename or a file handle. When the function is called with a filename, it sets
the handle to this named file as the default input file. If it is called with a file handle, it just sets the
default input file handle to this passed file handle.
\end{docLua}


\begin{docLua}{io.tmpfile()}
This function returns a handle for a temporary file; the file is opened in update mode and removed
automatically when the program ends.
\end{docLua}

The use of a temporary file, is useful in many situations, where we need to write something temporarily
and then read it later on in the program. 

\begin{texexample}{Temporary files}{}
\begin{luacode}
if type(tex)==table then print=print.tex end
fh = io.tmpfile()
fh:write("some sample data on page \\thepage ")
fh:flush()
-- Now let's read the data we just wrote
fh:seek("set", 0)
content = fh: read("*a")
print("We got: ", content)
\end{luacode}
\end{texexample}

\begin{docLua}{io.type(obj)}
This function returns the result if the handle specified by obj is a valid file handle. It returns the string
``file'' if |obj| is an open file handle and the string ``closed file'' if |obj| is a closed file handle. It
returns |nil| if |obj| is not a file handle.
\end{docLua}

\begin{texexample}{Checking the type}{}
\begin{luacode}
if type(tex)==table then print=print.tex end
fh = io.input()
if fh then
    tex.print(io.type(fh)) 
  else
    tex.print("nil")
end
\end{luacode}
\end{texexample} 

 
\begin{filecontents*}{input.txt}
line 1
line 2
line 3
line 4
line 5
\end{filecontents*}


\section*{How to read a file line by line}

Reading a file line by line is easy using an iterator.

\begin{texexample}{Read a file line by line}{}
\begin{luacode}
if type(tex)=='table' then print=tex.print end
for line in io.lines("input.txt") do
  print(line, "\\par")
end
\end{luacode}
\end{texexample}
\endinput
%\newfontfamily\hiero{NotoSansEgyptianHieroglyphs-Regular.ttf}

\section{A more practical example}
In the sections of the documentation, where we describe
the use of unicode for scripts, we provide files with almost
all unicode blocks we describe. For example the file
|hieroglyphics.txt| provides the details for the first
hieroglyphic characters |U+13000-130FF|.

This is parsed using \tex, but could equally well and perhaps easier and more robustly be parsed using Lua.  


\begin{texexample}{Reading a file line by line}{}
\begin{luacode*}
filename = "./languages/hieroglyphics.txt"
fp = io.open( filename, "r" )
for line in fp:lines() do
    tex.print(line..", ")
   end
fp:close()
\end{luacode*}
\end{texexample}

In the second example, we start building our programme
by defining functions to get a glyph using a particular font
command and to typeset it.

The function |getglyph()| takes two arguments, the first is the
name of a command for example for the control sequence \cmd{\hiero} the argument is  ``hiero" and a unicode codepoint in hexadecimal digits (as a string).

\begin{texexample}{Glyph functions}{}
\begin{luacode*}
function getglyph(cmd, codepoint)
  local texstring = "\\Large\\"..cmd.." \\char".."\""..codepoint
  return texstring
end

function printglyph(cmd, codepoint)
  tex.print(getglyph(cmd,codepoint))
end

printglyph("hiero","13000")
printglyph("hiero","13050")
\end{luacode*}
\end{texexample}

You could of course argue, that this is easily be done so far using \tex programming, but soon we will get into areas of the parser that it would be difficult to achieve using \tex only. With Lua also it is easier to add error detection to make our parser more robust.

Back to our parser and hieroglyphics another function we need to develop is a function that can take care of the writing direction. Hieroglyphics, like every written language,  needed conventions to keep writings consistent and readable. For instance English is always read left to right $\rightarrow$. 

Hieroglyphic writing was written in columns or rows. Reading direction is determined by the direction that human and animal figures faced. Reading starts from the direction that figures face and continues in the opposite direction. If the figures look left the direction is left to right and vice versa.

\begin{center}
\bgroup
$\rightarrow$

\scalebox{2}[2]{\hiero\char"13000\char"13011\char"13020}

$\leftarrow$

\scalebox{-2}[2]{\hiero\char"13000\char"13011\char"13020}

\scalebox{2}[2]{\hiero
\char"13000
\char"13011
\char"13020}
\egroup
\end{center}

Vertical texts were written from top to bottom, like the figure shown below. As these were composed from single hieroglyphs the direction the figures are facing is immaterial, other than for symmetry.

\begin{center}
{\hiero
\makebox[3em]{\scalebox{-2}[2]{\hiero
\char"13001}\hss}
\vskip3pt
\makebox[3em]{\scalebox{-2}[2]{\hiero
\char"13006}\hss}
\vskip3pt
\makebox[3em]{\scalebox{-2}[2]{\hiero
\char"13007}\hss}
}
\end{center}

Columns were read down as we would read lines down a page. The Egyptians liked symmetry. If hieroglyphs were inscribed in a column, they would often inscribe the same text in the opposite column, except with the writing reversed. When written vertical they were accompanied by drawings to illustrate the words, pretty much like we do today in publications.

\begin{figure}[hb]
\centering
\includegraphics[width=0.7\textwidth]{./images/senet.jpg}
\caption{Nefertari playing Senet. Painting in tomb of Egyptian Queen Nefertari (1295–1255 BC). \href{http://en.wikipedia.org/wiki/Senet}{Wikimedia}}
\end{figure}



\begin{docCommand}{scalebox} {\oarg{[]} \oarg{[]} \marg{box contents}}
It is much easier to use \tex to achieve the glyph rotation. The package \pkgname{graphicx} provides the command \cmd{\scalebox} which with suitable parameters can achieve this (\seedocs{graphicx}).
\end{docCommand}

In our Lua script we add the functions |getglyphRL()| and |printglyphRL()| to cater for direction. We can also add a helper function for top to bottom scripts:

\begin{verbatim}
function getglyphRL(cmd, codepoint)
  local texstring = "\\scalebox{-5}[5]{\\"..cmd.." \\char".."\""..codepoint.."}"
  return texstring
end

function printglyphRL(cmd, codepoint)
  tex.print(getglyphRL(cmd,codepoint))
end
\end{verbatim}

\section{Parsing transliteration strings}

The input of hieroglyphic texts presents a challenge to computer science. One cannot make a keyboard of hundreds of characters and which in any case would be difficult to use. 
Most Egyptologists tranliterate hieroglyphics, using a scheme that is described in more detail in the Chapter \textit{Egyptian Hieroglyphics}.

A transliteration script consists at is basic form of alphanumeric numbers separated by a "-" to separate the glyphs. It uses the Gardiner classification for numbering the glyphs.

\begin{center}
\texttt{A1*-R2*-R3*-F1*!}
\end{center}

Other symbols such as ":,\^\_*!" are used to denote how the scripts are assembled etc.

\begin{description}
\item [dash]
The dash (-) is used to indicate the end of a glyph. In reality many people in the field use a space which is a good as any symbol. 
\item [Exclamation mark] The exclamation mark indicates a line break. Two exclamation marks(!!) indicate a page break.

\item [Asterisk] an asterisk (*) indicates that the glyph is placed on the same row as the preceding one.

\item[Colon] the colon (:) indicates that the glyph is stacked under the preceding one; glyphs will continue to be stacked under until a stopping command is found (- or ! or !!).

\end{description}

Getting the TeX side of things right is also important, as we may need to scale the glyphs. There are many ways to scale the glyphs; we can put them in boxes and use scalebox. Another good option is to use font sizing commands.

\begin{center}

{\hiero\fontsize{2cm}{0.5em}\selectfont \char"13000}
{\hiero\fontsize{1cm}{0.25em}\selectfont 
\parbox[b]{1cm}{\hiero\char"13000\\ \char"13080}
}
{
\hiero\fontsize{0.66cm}{0.5em}\selectfont 
\fbox{\parbox[b]{1cm}{\hiero\char"13030\\ \char"13080\\ \char"133E5}}
\hiero\fontsize{2cm}{0.5em}\selectfont \char"13000
}



\end{center}

Once the font sizing commands are ready, it is a simple matter to use a stacking mechanism to stack them. If you recall from a parbox can be displayed inline--as far as \tex is concerned it is just another box to be typeset.  

\section{boxes}

\begin{texexample}{Reading values of a box}{}
\bgroup
\setbox0=\hbox{A\hskip 5pt B\hskip 10pt C}
\noindent Using \TeX{} code, box 0 has width \number\wd0\relax \space sp\par
\noindent We can also use Lua and call one of Lua\TeX's functions to get the same
information.\vskip10mm
\noindent From Lua code, box 0 has width 

\directlua{
local boxwidth = tex.box[0].width
tex.print(boxwidth.." sp")
} which, of course, is identical to the value obtained from \TeX{} code.

\egroup
\end{texexample}





}


\def\kernel{%
  \part{THE LaTeX2e KERNEL}
     \part{The \LaTeX\ standard class}
\parindent0pt
\setlength\columnsep{2em}
\def\Paragraph#1{{\bf #1}\quad}
\cxset{chapter toc=true}
\chapter{The book.cls}
\index{classes>standard}
\index{book>class}

\clearpage

\includegraphics[width=\textwidth]{./graphics/anatomy.jpg}

\vspace{2\baselineskip}

\textbf{\Large DISSECTING THE BOOK CLASS}
\thispagestyle{plain}
\begin{multicols}{2}
This appendix describes the listing of the book class as defined by \latex. It is described here with extra commentary in order to enable you to understand, how it all works.

\lipsum[1-3]
\end{multicols}

\section{General}
\pagestyle{headings}

The book class starts with declaring the version of \latex, required
and naming the class it provides. The class choices are always checked for backward compatibility with the earlier version of \latex. All commands that need to be modified between two column and one column layouts, check the setting and branch accordingly. Another primary choice is if the book is to be printed on both sides or only on one side.


\begin{teX}
\NeedsTeXFormat{LaTeX2e}[1995/12/01]
\ProvidesClass{book}
              [2007/10/19 v1.4h
 Standard LaTeX document class]
\end{teX}

\begin{macro}{\cs{@ptsize}}
This is set to an empty value at start up. The original idea here was by modifying the value one could scale the
text later on. I am not aware of any such usage.
\end{macro}

\begin{teX}
\newcommand\@ptsize{}
\newif\if@restonecol
\newif\if@titlepage \@titlepagetrue
\newif\if@openright
\newif\if@mainmatter \@mainmattertrue
\end{teX}


\Paragraph{Paper size.} After checking for compatibilty with older versions the code branches to define the different standard paper sizes! The options that are declared are, |a4paper|, |a5paper|, |b5paper|, |letterpaper|, |legalpaper| and  |executivepaper|. The class will then later on process the options and set the default to |letterpaper|. \footnote{The package \texttt{geometry}, some classes such as the |Octavo| and |KOMA| classes add additional sizes to cater for other standards.}

\begin{teX}
\if@compatibility\else
\DeclareOption{a4paper}
   {\setlength\paperheight {297mm}%
    \setlength\paperwidth  {210mm}}
\DeclareOption{a5paper}
   {\setlength\paperheight {210mm}%
    \setlength\paperwidth  {148mm}}
\DeclareOption{b5paper}
   {\setlength\paperheight {250mm}%
    \setlength\paperwidth  {176mm}}
\DeclareOption{letterpaper}
   {\setlength\paperheight {11in}%
    \setlength\paperwidth  {8.5in}}
\DeclareOption{legalpaper}
   {\setlength\paperheight {14in}%
    \setlength\paperwidth  {8.5in}}
\DeclareOption{executivepaper}
   {\setlength\paperheight {10.5in}%
    \setlength\paperwidth  {7.25in}}
\end{teX}

\begin{multicols}{2}
\Paragraph{Paper orientation.} the paper orientation is set based on the |landscape| option. If it is declared it stores the |\paperheight| into one of the \latex kernel scratch registers, |\@tempdima| and then reverses the length with the |\paperwidth|.
\end{multicols}

\begin{teX}
\DeclareOption{landscape}
   {\setlength\@tempdima   {\paperheight}%
    \setlength\paperheight {\paperwidth}%
    \setlength\paperwidth  {\@tempdima}}
\fi
\end{teX}

\begin{multicols}{2}
\Paragraph{Font sizing} The class provides three font sizes |10pt|, |11pt| and |12pt|. It default to ten point text.
\end{multicols}

\begin{teX}
\if@compatibility
  \renewcommand\@ptsize{0}
\else
\DeclareOption{10pt}{\renewcommand\@ptsize{0}}
\fi
\DeclareOption{11pt}{\renewcommand\@ptsize{1}}
\DeclareOption{12pt}{\renewcommand\@ptsize{2}}
\end{teX}

\begin{multicols}{2}
\Paragraph{Recto and verso pages.} The class provides the |oneside| and |twoside| options for switching between one side printing or two side printing.  It sets the booleans |\if@twoside| and |\if@mparswitch| accordingly. These booleans are used later on for setting other variables.
\end{multicols}
\begin{teX}
\if@compatibility\else
\DeclareOption{oneside}{\@twosidefalse \@mparswitchfalse}
\fi
\DeclareOption{twoside}{\@twosidetrue  \@mparswitchtrue}
\end{teX}

\begin{multicols}{2}
\Paragraph{Draft and final options.} The options draft and final, just set the |\overfullrule| to either 1pt or 0pt. The |\overfullrule| is a \tex command and simply prints a small vertical line to indicate overfull boxes for the attention of the author. 
\end{multicols}

\begin{teX}
\DeclareOption{draft}{\setlength\overfullrule{5pt}}
\if@compatibility\else
\DeclareOption{final}{\setlength\overfullrule{0pt}}
\fi
\end{teX}

\begin{multicols}{2}
\Paragraph{Title page option.} If the book class, needed such an option is debatable. The |titlepage| option is normally set as true and results in the title being on its own page. The |notitlepage| will omit the page break and display the title on the same page with that of the opening text. Highly unlikely for any author to use it for a book. It is useful for the article class.
\end{multicols}

\begin{teX}
\DeclareOption{titlepage}{\@titlepagetrue}
\if@compatibility\else
\DeclareOption{notitlepage}{\@titlepagefalse}
\fi
\end{teX}

\begin{multicols}{2}
\Paragraph{Display of chapters.} Chapters can be set to start only on an even page or any page. The class provides the options |openright| and |openany|.
\end{multicols}


\begin{teX}
\if@compatibility
\@openrighttrue
\else
\DeclareOption{openright}{\@openrighttrue}
\DeclareOption{openany}{\@openrightfalse}
\fi
\end{teX}


\begin{teX}
\if@compatibility\else
\DeclareOption{onecolumn}{\@twocolumnfalse}
\fi
\DeclareOption{twocolumn}{\@twocolumntrue}
\DeclareOption{leqno}{\input{leqno.clo}}
\DeclareOption{fleqn}{\input{fleqn.clo}}
\DeclareOption{openbib}{%
  \AtEndOfPackage{%
   \renewcommand\@openbib@code{%
      \advance\leftmargin\bibindent
      \itemindent -\bibindent
      \listparindent \itemindent
      \parsep \z@
      }%
   \renewcommand\newblock{\par}}%
}
\end{teX}

We now execute the options and process them.
\begin{teX}
\ExecuteOptions{letterpaper,10pt,twoside,onecolumn,final,openright}
\ProcessOptions
\end{teX}

\begin{multicols}{2}
\textbf{The .clo files}\quad The book class now inputs the file .clo etc that defines the fontsizes
 for anything specific to the 10pt. These files hold quite a bit of information and size related commands for the
standard sizes provided by \latex. The |.clo| files also set many other parameters for page sizing, lists, paper sectioning, such as margins, marginpars and the like.
\end{multicols}

\begin{teX}
\input{bk1\@ptsize.clo}
\setlength\lineskip{1\p@}
\setlength\normallineskip{1\p@}
\renewcommand\baselinestretch{}
\setlength\parskip{0\p@ \@plus \p@}
\end{teX}

\begin{multicols}{2}
\Paragraph{Penalties.} Here the following penalties are set.
\end{multicols}

\begin{teX}
\@lowpenalty   51
\@medpenalty  151
\@highpenalty 301
\end{teX}

\begin{multicols}{2}
\textbf{Float control parameters.}\quad The allowable number of floats on a page are controlled by a number of parameters. These are set here. Many users overwrite these parameters in order to have more control on the placement of floats.
\end{multicols}


\begin{teX}
\setcounter{topnumber}{2}
\renewcommand\topfraction{.7}
\setcounter{bottomnumber}{1}
\renewcommand\bottomfraction{.3}
\setcounter{totalnumber}{3}
\renewcommand\textfraction{.2}
\renewcommand\floatpagefraction{.5}
\setcounter{dbltopnumber}{2}
\renewcommand\dbltopfraction{.7}
\renewcommand\dblfloatpagefraction{.5}
\end{teX}

\begin{multicols}{2}
\textbf{Running head and foot.}\quad A page header or simply header in typography is text which is separated from the main body of text and appears at the top of a printed page. Word processing programs usually provide for the creation and maintenance of page headers, which are often the same from page to page, with merely small differences in information, such as page number.

In publishing, the page header (or ``pagehead'') is often referred to as the running head. Typical running heads in a book might consist of the book title on the left-hand (verso) page, and the chapter title on the right-hand (recto) page, or chapter title on the verso and subsection title on the recto.
\end{multicols}


\begin{teX}
\if@twoside
  \def\ps@headings{%
      \let\@oddfoot\@empty\let\@evenfoot\@empty
      \def\@evenhead{\thepage\hfil\slshape\leftmark}%
      \def\@oddhead{{\slshape\rightmark}\hfil\thepage}%
      \let\@mkboth\markboth
 % chapter
  \def\chaptermark##1{%
      \markboth {\MakeUppercase{%
        \ifnum \c@secnumdepth >\m@ne
          \if@mainmatter
            \@chapapp\ \thechapter. \ %
          \fi
        \fi
        ##1}}{}}%
% section
    \def\sectionmark##1{%
      \markright {\MakeUppercase{%
        \ifnum \c@secnumdepth >\z@
          \thesection. \ %
        \fi
        ##1}}}}
\else
  \def\ps@headings{%
    \let\@oddfoot\@empty
    \def\@oddhead{{\slshape\rightmark}\hfil\thepage}%
    \let\@mkboth\markboth
    \def\chaptermark##1{%
      \markright {\MakeUppercase{%
        \ifnum \c@secnumdepth >\m@ne
          \if@mainmatter
            \@chapapp\ \thechapter. \ %
          \fi
        \fi
        ##1}}}}
\fi
\def\ps@myheadings{%
    \let\@oddfoot\@empty\let\@evenfoot\@empty
    \def\@evenhead{\thepage\hfil\slshape\leftmark}%
    \def\@oddhead{{\slshape\rightmark}\hfil\thepage}%
    \let\@mkboth\@gobbletwo
    \let\chaptermark\@gobble
    \let\sectionmark\@gobble
    }
\end{teX}

\begin{multicols}{2}\index{headings!plain}
Please note the \textit{plain} headings are not defined in the class. These are defined in the \latex kernel\footnote{See \texttt{File J: ltpage.dtx}, page 312.}

\begin{teX}
\ps@plain The plain page style: No head, centred page number in foot.
13 \def\ps@plain{\let\@mkboth\@gobbletwo
14 \let\@oddhead\@empty\def\@oddfoot{\reset@font\hfil\thepage
15 \hfil}\let\@evenhead\@empty\let\@evenfoot\@oddfoot}
\end{teX}



\textbf{Title pages.}\quad Title pages are defined between a conditional, that handle the option |titlepage|
and. The commands just take mostly of the typography. If you use the option |notitlepage| in the book class, the title will be similar for all practical purposes to that of an |article| and it will appear on the top of the first page.

The |\maketitle| sets the |\footnotesise|, the |\footnoterule| and the |\footnote|.
\end{multicols}

\begin{teX}
 \if@titlepage
  \newcommand\maketitle{\begin{titlepage}%
  \let\footnotesize\small
  \let\footnoterule\relax
  \let \footnote \thanks
  \null\vfil
  \vskip 60\p@
  \begin{center}%
    {\LARGE \@title \par}%
    \vskip 3em%
    {\large
     \lineskip .75em%
      \begin{tabular}[t]{c}%
        \@author
      \end{tabular}\par}%
      \vskip 1.5em%
    {\large \@date \par}%       % Set date in \large size.
  \end{center}\par
  \@thanks
  \vfil\null
  \end{titlepage}%
  \setcounter{footnote}{0}%
  \global\let\thanks\relax
  \global\let\maketitle\relax
  \global\let\@thanks\@empty
  \global\let\@author\@empty
  \global\let\@date\@empty
  \global\let\@title\@empty
  \global\let\title\relax
  \global\let\author\relax
  \global\let\date\relax
  \global\let\and\relax
}
\else
\newcommand\maketitle{\par
  \begingroup
    \renewcommand\thefootnote{\@fnsymbol\c@footnote}%
    \def\@makefnmark{\rlap{\@textsuperscript{\normalfont\@thefnmark}}}%
    \long\def\@makefntext##1{\parindent 1em\noindent
            \hb@xt@1.8em{%
                \hss\@textsuperscript{\normalfont\@thefnmark}}##1}%
    \if@twocolumn
      \ifnum \col@number=\@ne
        \@maketitle
      \else
        \twocolumn[\@maketitle]%
      \fi
    \else
      \newpage
      \global\@topnum\z@   % Prevents figures from going at top of page.
      \@maketitle
    \fi
    \thispagestyle{plain}\@thanks
  \endgroup
  \setcounter{footnote}{0}%
  \global\let\thanks\relax
  \global\let\maketitle\relax
  \global\let\@maketitle\relax
  \global\let\@thanks\@empty
  \global\let\@author\@empty
  \global\let\@date\@empty
  \global\let\@title\@empty
  \global\let\title\relax
  \global\let\author\relax
  \global\let\date\relax
  \global\let\and\relax
}
\def\@maketitle{%
  \newpage
  \null
  \vskip 2em%
  \begin{center}%
  \let \footnote \thanks
    {\LARGE \@title \par}%
    \vskip 1.5em%
    {\large
      \lineskip .5em%
      \begin{tabular}[t]{c}%
        \@author
      \end{tabular}\par}%
    \vskip 1em%
    {\large \@date}%
  \end{center}%
  \par
  \vskip 1.5em}
\fi
\end{teX}


\Paragraph{Section counters.}\quad 
In LaTeX all defaults all document section are numbered by default. These numbers are kept in counters, named after the section name. A series of commands are provided to access these numbers.
All the counters are in arabic numerals, with the exception of "part", which is in Roman.


\begin{teX}
\newcommand*\chaptermark[1]{}
\setcounter{secnumdepth}{2}
\newcounter {part}
\newcounter {chapter}
\newcounter {section}[chapter]
\newcounter {subsection}[section]
\newcounter {subsubsection}[subsection]
\newcounter {paragraph}[subsubsection]
\newcounter {subparagraph}[paragraph]
\renewcommand \thepart {\@Roman\c@part}
\renewcommand \thechapter {\@arabic\c@chapter}
\renewcommand \thesection {\thechapter.\@arabic\c@section}
\renewcommand\thesubsection   {\thesection.\@arabic\c@subsection}
\renewcommand\thesubsubsection{\thesubsection.\@arabic\c@subsubsection}
\renewcommand\theparagraph    {\thesubsubsection.\@arabic\c@paragraph}
\renewcommand\thesubparagraph {\theparagraph.\@arabic\c@subparagraph}
\newcommand\@chapapp{\chaptername}
\end{teX}

\begin{multicols}{2}
\textbf{Frontmatter, mainmatter and backmatter.} These are author command to set mostly, the page numbering and the clearing of pages for two page layouts. Front matter has lower roman pages numbering and the main matter has arabic numerals.
\end{multicols}

\emphasis{frontmatter,mainmatter,backmatter}
\begin{teXXX}
\newcommand\frontmatter{%
    \cleardoublepage
  \@mainmatterfalse
  \pagenumbering{roman}}

\newcommand\mainmatter{%
    \cleardoublepage
  \@mainmattertrue
  \pagenumbering{arabic}}

\newcommand\backmatter{%
  \if@openright
    \cleardoublepage
  \else
    \clearpage
  \fi
  \@mainmatterfalse}
\end{teXXX}

\begin{multicols}{2}
\Paragraph{Part.}The is the definition of part. The Part is displayed with a plain header and the it goes into the secdef. If the section depth is greater or equal -2, the start counter is increased and the part is added to the toc, using |\addcontentsline|. 
The partname i.e., default 'Part' gets printer either way except for the star version of the command.
\end{multicols}

\emphasis{@part,@spart,secdef}
\begin{teXXX}
\newcommand\part{%
  \if@openright
    \cleardoublepage
  \else
    \clearpage
  \fi
  \thispagestyle{plain}%
  \if@twocolumn
    \onecolumn
    \@tempswatrue
  \else
    \@tempswafalse
  \fi
  \null\vfil
  \secdef\@part\@spart}
\end{teXXX}

\begin{multicols}{2}
 The important command to remember here
is |\secdef|. This is defined in the kernel and not in the classes \texttt{ltsect.dtx}. Essentially in the code |\@part| calls the unstar command and the @spart calls the starred command. We copy the definition from the kernel for convenience.
\end{multicols}

\begin{teXXX}
is \secdef{unstarcmds}{unstarcmds}{starcmds}
When defining a \chapter or \section command without using \@startsection,
you can use \secdef as follows:
1. \def\chapter{ . . . \secdef \starcmd \unstarcmd}
2. \def\hstarcmdi[#1]#2{ . . . } % Command to define \chapter[. . . ]{. . . }
3. \def\unstarcmd#1{ . . . } % Command to define \chapter*{. . . }
125 \def\secdef#1#2{\@ifstar{#2}{\@dblarg{#1}}}
\end{teXXX}

The |@part| starts now,

\begin{teXXX}
\def\@part[#1]#2{%
    \ifnum \c@secnumdepth >-2\relax
      \refstepcounter{part}%
      \addcontentsline{toc}{part}{\thepart\hspace{1em}#1}%
    \else
      \addcontentsline{toc}{part}{#1}%
    \fi
    \markboth{}{}%
    {\centering
     \interlinepenalty \@M
     \normalfont
     \ifnum \c@secnumdepth >-2\relax
       \huge\bfseries \partname\nobreakspace\thepart
       \par
       \vskip 20\p@
     \fi
     \Huge \bfseries #2\par}%
    \@endpart}
\end{teXXX}

\begin{multicols}{2}
The starred version of the command is provided next. The difference the name `Part'' is not displayed. However the parameter provided by the user is displayed. A normal font is provided. Final settings depending on @openright and header styles are set and the code macro is completed.
\end{multicols}

\begin{teXXX}
\def\@spart#1{%
    {\centering
     \interlinepenalty \@M
     \normalfont
     \Huge \bfseries #1\par}%
    \@endpart}

\def\@endpart{\vfil\newpage
              \if@twoside
               \if@openright
                \null
                \thispagestyle{empty}%
                \newpage
               \fi
              \fi
              \if@tempswa
                \twocolumn
              \fi}
\end{teXXX}

\begin{multicols}{2}
\Paragraph{Chapter}. The chapter definition follows, the same pattern as that of the part definitions. It calls secdef and defines commands for the starred and unstarred versions.
\end{multicols}

\begin{teXXX}
\newcommand\chapter{\if@openright\cleardoublepage\else\clearpage\fi
                    \thispagestyle{plain}%
                    \global\@topnum\z@
                    \@afterindentfalse
                    \secdef\@chapter\@schapter}
\end{teXXX}

\Paragraph{Unstarred version}

\begin{teXXX}
\def\@chapter[#1]#2{\ifnum \c@secnumdepth >\m@ne
                       \if@mainmatter
                         \refstepcounter{chapter}%
                         \typeout{\@chapapp\space\thechapter.}%
                         \addcontentsline{toc}{chapter}%
                                   {\protect\numberline{\thechapter}#1}%
                       \else
                         \addcontentsline{toc}{chapter}{#1}%
                       \fi
                    \else
                      \addcontentsline{toc}{chapter}{#1}%
                    \fi
                    \chaptermark{#1}%
                    \addtocontents{lof}{\protect\addvspace{10\p@}}%
                    \addtocontents{lot}{\protect\addvspace{10\p@}}%
                    \if@twocolumn
                      \@topnewpage[\@makechapterhead{#2}]%
                    \else
                      \@makechapterhead{#2}%
                      \@afterheading
                    \fi}
\end{teXXX}

\begin{multicols}{2}
\Paragraph{Defining the looks of the Chapter heading.}

Good practice dictates, that when you change the chapterhead layout for the numbered version, you also change it for the star version of the command. You can do that by using two different macros, although at first glance it might be difficult to see where the difference is.
\end{multicols}


\emphasis{@makeschapterhead,@makechapterhead}
\begin{teXXX}
\def\@makechapterhead
\def\@makeschapterhead
\end{teXXX}

\begin{teXXX}
\def\@makechapterhead#1{%
  \vspace*{50\p@}%
  {\parindent \z@ \raggedright \normalfont
    \ifnum \c@secnumdepth >\m@ne
      \if@mainmatter
        \huge\bfseries \@chapapp\space \thechapter
        \par\nobreak
        \vskip 20\p@
      \fi
    \fi
    \interlinepenalty\@M
    \Huge \bfseries #1\par\nobreak
    \vskip 40\p@
  }}
\end{teXXX}

\begin{multicols}{2}
Finally the starred version of the command is called. This now checks for twocolumn or one column via an if sttaement and executes, the makeschapterhead. Another mysterious and wonderful command appears again from the LaTeX source2e.\cmd{\@afterheading}. This command 
is just a hook for custom headings? (Needs to be reviewed again).
\end{multicols}

\begin{teXXX}
\def\@schapter#1{\if@twocolumn
                   \@topnewpage[\@makeschapterhead{#1}]%
                 \else
                   \@makeschapterhead{#1}%
                   \@afterheading
                 \fi}
\end{teXXX}

And finally the |\@makeschapterhead| (remember \textbf{s} for \textbf{s}tar).

\begin{teXXX}
\def\@makeschapterhead#1{%
  \vspace*{50\p@}%
  {\parindent \z@ \raggedright
    \normalfont
    \interlinepenalty\@M
    \Huge \bfseries  #1\par\nobreak
    \vskip 40\p@
  }}
\end{teXXX}

All sorts of variations of the above two commands can be found in different classes, such as |KOMA|, |memoir| and others. The example which follows, typesets the headings as shown in \fref{fig:chapterhead-17}. The |@makechapterhead| command is modified to produce a centered heading which is displayed between two heavy rules. This style can be found in quite a number of books.

\begin{figure*}[htbp]
\includegraphics[width=\linewidth]{./graphics/chapterhead-17.png}
\caption{Modifying the way the chapterhead looks can be achieved by redefining the \texttt{\textbackslash @makechapterhead} and \texttt{\textbackslash @makeschapterhead} commands.}
\label{fig:chapterhead-17}
\end{figure*}

\section*{Full working example}

\begin{teX}
\documentclass[oneside]{book}
\usepackage[english]{babel}
\usepackage{lipsum}
\makeatletter
\def\thickhrule{\leavevmode \leaders \hrule height 1ex \hfill \kern \z@}

%% Note the difference between the commands the one is 
%% make and the other one is makes
\renewcommand{\@makechapterhead}[1]{%
  \vspace*{10\p@}%
  {\parindent \z@ \centering \reset@font
        {\Huge \scshape  \thechapter }
        \par\nobreak
        \vspace*{10\p@}%
        \interlinepenalty\@M
        \thickhrule
        \par\nobreak
        \vspace*{2\p@}%
        {\Huge \bfseries #1\par\nobreak}
        \par\nobreak
        \vspace*{2\p@}%
        \thickhrule
    \vskip 40\p@
    \vskip 100\p@
  }}

%% This is makes
\def\@makeschapterhead#1{%
  \vspace*{10\p@}%
  {\parindent \z@ \centering \reset@font
        {\Huge \scshape \vphantom{\thechapter}}
        \par\nobreak
        \vspace*{10\p@}%
        \interlinepenalty\@M
        \thickhrule
        \par\nobreak
        \vspace*{2\p@}%
        {\Huge \bfseries #1\par\nobreak}
        \par\nobreak
        \vspace*{2\p@}%
        \thickhrule
    \vskip 100\p@
  }}
\begin{document}
\chapter{The Real Numbers}
\lipsum[1-2]
\chapter*{The Imaginary Numbers}
\lipsum[1-2]
\end{document}
\end{teX}

\begin{multicols}{2}
\Paragraph{The sections.}
In this section, all the document elements besides the Chapter and the Part are Defined. They use the mother of all commands from the kernel
ltsection.dtx, named |\@startsection|. This is just a call to the kernel command. No other settings are done here. In order to remember what it does we refer to its definition in the kernel. Of interest is the sixth argument which sets the font style.

The parameter takes eight parameters, some of them optional. We discuss this command in more detail in the kernel chapter.

\end{multicols}


\begin{teX}
\newcommand\section{\@startsection {section}{1}{\z@}%
                                   {-3.5ex \@plus -1ex \@minus -.2ex}%
                                   {2.3ex \@plus.2ex}%
                                   {\normalfont\Large\bfseries}}
\newcommand\subsection{\@startsection{subsection}{2}{\z@}%
                                     {-3.25ex\@plus -1ex \@minus -.2ex}%
                                     {1.5ex \@plus .2ex}%
                                     {\normalfont\large\bfseries}}
\newcommand\subsubsection{\@startsection{subsubsection}{3}{\z@}%
                                     {-3.25ex\@plus -1ex \@minus -.2ex}%
                                     {1.5ex \@plus .2ex}%
                                     {\normalfont\normalsize\bfseries}}
\newcommand\paragraph{\@startsection{paragraph}{4}{\z@}%
                                    {3.25ex \@plus1ex \@minus.2ex}%
                                    {-1em}%
                                    {\normalfont\normalsize\bfseries}}
\newcommand\subparagraph{\@startsection{subparagraph}{5}{\parindent}%
                                       {3.25ex \@plus1ex \@minus .2ex}%
                                       {-1em}%
                                      {\normalfont\normalsize\bfseries}}
\if@twocolumn
  \setlength\leftmargini  {2em}
\else
  \setlength\leftmargini  {2.5em}
\fi
\leftmargin  \leftmargini
\setlength\leftmarginii  {2.2em}
\setlength\leftmarginiii {1.87em}
\setlength\leftmarginiv  {1.7em}
\if@twocolumn
  \setlength\leftmarginv  {.5em}
  \setlength\leftmarginvi {.5em}
\else
  \setlength\leftmarginv  {1em}
  \setlength\leftmarginvi {1em}
\fi
\setlength  \labelsep  {.5em}
\setlength  \labelwidth{\leftmargini}
\addtolength\labelwidth{-\labelsep}
\@beginparpenalty -\@lowpenalty
\@endparpenalty   -\@lowpenalty
\@itempenalty     -\@lowpenalty
\renewcommand\theenumi{\@arabic\c@enumi}
\renewcommand\theenumii{\@alph\c@enumii}
\renewcommand\theenumiii{\@roman\c@enumiii}
\renewcommand\theenumiv{\@Alph\c@enumiv}
\newcommand\labelenumi{\theenumi.}
\newcommand\labelenumii{(\theenumii)}
\newcommand\labelenumiii{\theenumiii.}
\newcommand\labelenumiv{\theenumiv.}
\renewcommand\p@enumii{\theenumi}
\renewcommand\p@enumiii{\theenumi(\theenumii)}
\renewcommand\p@enumiv{\p@enumiii\theenumiii}
\newcommand\labelitemi{\textbullet}
\newcommand\labelitemii{\normalfont\bfseries \textendash}
\newcommand\labelitemiii{\textasteriskcentered}
\newcommand\labelitemiv{\textperiodcentered}
\newenvironment{description}
               {\list{}{\labelwidth\z@ \itemindent-\leftmargin
                        \let\makelabel\descriptionlabel}}
               {\endlist}
\newcommand*\descriptionlabel[1]{\hspace\labelsep
                                \normalfont\bfseries #1}
\end{teX}

\begin{multicols}{2}
\textbf{The verse environment}\quad \latex's verse environment, can only serve for the incidental use of a few stanzas. It leaves most of the formatting to the author.  It redefines the line break |\\| to a |\centercr|.
\end{multicols}

\begin{teX}
\newenvironment{verse}
    {\let\\\@centercr(*@\protect\footnote{This is defined in ltmiscen.dtx}@*)
     \list{}{\itemsep  \z@
             \itemindent   -1.5em%
             \listparindent\itemindent
             \rightmargin  \leftmargin
             \advance\leftmargin 1.5em}%
       \item\relax}
    {\endlist}
\end{teX}
\makeatletter
\newenvironment{Verse}
    {\let\\\@centercr%
     \list{}{\itemsep1pt
             \itemindent-1.5em%
             \listparindent\itemindent
             \rightmargin\leftmargin
             \advance\leftmargin 1.5em}%
       \item\relax}
    {\endlist}
\makeatother
\begin{teX}
  \begin{Verse}
     My mobile test\\
     this is other\\
     this is last\\
  \end{Verse}
\end{teX}

The environment doesn't really do much, the way I see it but just move the poem a couple of ems inwards 
to much the definition of lists. Most people will want more from a poem environment.
\begin{Verse}
     My mobile test\\
      this is other\\
       this is last\\
\end{Verse}

The simplest thing we can add to this environment if we want to modify it, is a hook. This we can do using the |blckcntrl| package. \sidenote{From the \url{http://www.ifi.uio.no/it/latex-links/blkcntrl.pdf}}.

\begin{teX}
\renewenvironment{verse}
50 {\let\\\@centercr
51 \relax\list{}{\setlength{\itemsep}{\z@}%
52 \setlength{\itemindent}{-1.5em}%
53 \setlength{\listparindent}{\itemindent}%
54 \setlength{\rightmargin}{\leftmargin}%
55 \addtolength{\leftmargin}{1.5em}}%
56 \item\relax\PreVerse\relax}
57 {\endlist}
\end{teX}

Using the command |\PreVerse|, we can add a block at the beginning of the block. For example some code to make a poem title and insert it later on. The setting of the rightmargin to the leftmargin here is curious. It might for example give us problems with |tufte-latex| classes.

\begin{multicols}{2}
\textbf{The quote and quotation environments.}\quad The environments |quote| and |quotation| are defined next. Again they are defined using the general |\list| environment. Again the general |\list|, is used in the definition. The |listparindent| is set to 1.5 em.
\end{multicols}

\begin{teX}
\newenvironment{quotation}
               {\list{}{\listparindent 1.5em%
                        \itemindent\listparindent
                        \rightmargin\leftmargin
                        \parsep\z@ \@plus\p@}%
                \item\relax}
               {\endlist}
\end{teX}

\begin{teX}
\newenvironment{quote}
               {\list{}{\rightmargin\leftmargin}%
                \item\relax}
               {\endlist}




\section{The \protect\texttt{titlepage} environment}
\if@compatibility
\newenvironment{titlepage}
    {%
      \cleardoublepage
      \if@twocolumn
        \@restonecoltrue\onecolumn
      \else
        \@restonecolfalse\newpage
      \fi
      \thispagestyle{empty}%
      \setcounter{page}\z@
    }%
    {\if@restonecol\twocolumn \else \newpage \fi
    }
\else
\newenvironment{titlepage}
    {%
      \cleardoublepage
      \if@twocolumn
        \@restonecoltrue\onecolumn
      \else
        \@restonecolfalse\newpage
      \fi
      \thispagestyle{empty}%
      \setcounter{page}\@ne
    }%
    {\if@restonecol\twocolumn \else \newpage \fi
     \if@twoside\else
        \setcounter{page}\@ne
     \fi
    }
\fi
\end{teX}



\begin{multicols}{2}
\includegraphics[width=\linewidth]{./graphics/appendix.png}

\Paragraph{The Appendix.}
Similarly to the chapter sectioning commands, the Appendix is not defined as a section. It simply sets the chapter and section counters to zero and sets the name of the section. All the relevant counters and uses letters for the numbering of the following chapters etc. If you closely follow the code, it is all based on the chapter command, except that it defaults to Alphanumeric counting.

\begin{teX}
\newcommand\appendix{\par
  \setcounter{chapter}{0}%
  \setcounter{section}{0}%
  \gdef\@chapapp{\appendixname}(*@\sidenote{The actual literal used for \textbackslash{appendixname} is defined later on, so that you can customize the language}\label{appendixname}@*)
  \gdef\thechapter{\@Alph\c@chapter}}
\end{teX}

An Appendix page has the same looks and feel to that of a Chapter. For all practical purposes, it is a chapter, with different labels and roman numbering.
\end{multicols}

\begin{multicols}{2}
\Paragraph{General Settings.} Here, some general settings are set. These include settings for framed boxes, tabbing separators and array column separators.

\end{multicols}
\begin{teX}
\setlength\arraycolsep{5\p@}
\setlength\tabcolsep{6\p@}
\setlength\arrayrulewidth{.4\p@}
\setlength\doublerulesep{2\p@}
\setlength\tabbingsep{\labelsep}
\skip\@mpfootins = \skip\footins
\setlength\fboxsep{3\p@}
\setlength\fboxrule{.4\p@}
\end{teX}

\begin{multicols}{2}
\Paragraph{Equation numbering}
The equation counter is reset according to the chapter counter, using the \latex kernel command |\@addtoreset|. 

\end{multicols}
\begin{teX}
\@addtoreset {equation}{chapter}
\renewcommand\theequation
  {\ifnum \c@chapter>\z@ \thechapter.\fi \@arabic\c@equation}
\end{teX}

\section*{FIGURE AND TABLE ENVIRONMENTS}

\begin{multicols}{2}
\Paragraph{Figure Environment} The figure environment is defined using commands that have been provided by the kernel.  The command |\thefigure| is first redefined to display the combination of the chapter dot figure counter, all in arabic numerals. The extension for the list of figures and finally the floats for single column and double column.

\end{multicols}

\label{book:figure}
\begin{teX}
\newcounter{figure}[chapter]
\renewcommand \thefigure
     {\ifnum \c@chapter>\z@ \thechapter.\fi \@arabic\c@figure}
\def\fps@figure{tbp}
\def\ftype@figure{1}
\def\ext@figure{lof}
\def\fnum@figure{\figurename\nobreakspace\thefigure}
\newenvironment{figure}
               {\@float{figure}}
               {\end@float}
\newenvironment{figure*}
               {\@dblfloat{figure}}
               {\end@dblfloat}
\end{teX}

\begin{multicols}{2}
\Paragraph{Table Environment} Table floats are defined the same way like the figures with their respective counters and names.  
\end{multicols}

\begin{teX}
\newcounter{table}[chapter]
\renewcommand \thetable
     {\ifnum \c@chapter>\z@ \thechapter.\fi \@arabic\c@table}
\def\fps@table{tbp}
\def\ftype@table{2}
\def\ext@table{lot}
\def\fnum@table{\tablename\nobreakspace\thetable}


\newenvironment{table}
               {\@float{table}}
               {\end@float}

\newenvironment{table*}
               {\@dblfloat{table}}
               {\end@dblfloat}
\end{teX}

\begin{multicols}{2}
\Paragraph{Captions}
The captioning macros are rather short but need a bit of explanation. First
some lengths are defined. The lengths are for |abovecaptionskip| and |belowcaptionskip| are set equal to a default of 10pt as for the font-size, but the length |belowcaptionskip| is set to |opt|.
\end{multicols}

\begin{teX}
\newlength\abovecaptionskip
\newlength\belowcaptionskip
\setlength\abovecaptionskip{10\p@}
\setlength\belowcaptionskip{0\p@}

\long\def\@makecaption#1#2{%
  \vskip\abovecaptionskip
  \sbox\@tempboxa{#1: #2}
  \ifdim \wd\@tempboxa >\hsize
    #1: #2\par
  \else
    \global \@minipagefalse
    \hb@xt@\hsize{\hfil\box\@tempboxa\hfil}%
  \fi
  \vskip\belowcaptionskip}
\end{teX}

\begin{multicols}{2}
The |@makecaption| macro is also interesting. Firstly note in line  the use of a colon (:). So if you do not like to have this you know where you need to go and change it. The contents of the caption are first saved into a box. If the box is greater than |hsize| then they are written like a paragraph otherwise, they are centered. Note that the centering is done using |\hfil\box\@tempoxa\hfil|. The mysterious command |\hb@xt| is defined in the kernel and
is equivalent to |\hbox to|
\end{multicols}

\begin{teXXX}
  \hb@xt@ The next one is another 100 tokens worth.
  16 \def\hb@xt@{\hbox to}
\end{teXXX}

\begin{multicols}{2}
It is simply an abbreviation of |\hbox to|. There are many short-cut commands like this, so the command just again sets the caption in a  horizontal box. There is more to the story later on. 
\end{multicols}


\section*{Defining the old style font commands}

\begin{teX}
\DeclareOldFontCommand{\rm}{\normalfont\rmfamily}{\mathrm}
\DeclareOldFontCommand{\sf}{\normalfont\sffamily}{\mathsf}
\DeclareOldFontCommand{\tt}{\normalfont\ttfamily}{\mathtt}
\DeclareOldFontCommand{\bf}{\normalfont\bfseries}{\mathbf}
\DeclareOldFontCommand{\it}{\normalfont\itshape}{\mathit}
\DeclareOldFontCommand{\sl}{\normalfont\slshape}{\@nomath\sl}
\DeclareOldFontCommand{\sc}{\normalfont\scshape}{\@nomath\sc}
\DeclareRobustCommand*\cal{\@fontswitch\relax\mathcal}
\DeclareRobustCommand*\mit{\@fontswitch\relax\mathnormal}
\end{teX}


\section*{Table of contents}

\begin{multicols}{2}
Firstly we define the width of the box that the page number is set. Use ems so that it does not need to be redefined for every change in font size.
Toc entries are treated as rectangular areas where the text
and probably a filler will be written. Let's draw such an
area (of course, the lines themselves are not printed):
\end{multicols}


\setlength{\unitlength}{1cm}
\begin{center}
\begin{picture}(8,2.2)
\put(1,1){\line(1,0){6}}
\put(1,2){\line(1,0){6}}
\put(1,1){\line(0,1){1}}
\put(7,1){\line(0,1){1}}
\put(0,.7){\vector(1,0){1}}
\put(8,.7){\vector(-1,0){1}}
\put(0,.2){\makebox(1,.5)[b]{\textit{left}}}
\put(7,.2){\makebox(1,.5)[b]{\textit{right}}}
\end{picture}
\end{center}

The space between the left page margin and the left edge of
the area will be named |<left>|; similarly we have |<right>|.
You are allowed to modify the beginning of the first line and
the ending of the last line. For example by ``taking up'' both
places with |\hspace*{2pc}| the area becomes:
\begin{center}
\begin{picture}(8,2.2)
\put(1,1){\line(1,0){5.5}}
\put(6.5,1){\line(0,1){.5}}
\put(6.5,1.5){\line(1,0){.5}}
\put(1.5,2){\line(1,0){5.5}}
\put(1,1.5){\line(1,0){.5}}
\put(1.5,1.5){\line(0,1){.5}}
\put(1,1){\line(0,1){.5}}
\put(7,1.5){\line(0,1){.5}}
\put(0,.7){\vector(1,0){1}}
\put(8,.7){\vector(-1,0){1}}
\put(0,.2){\makebox(1,.5)[b]{\textit{left}}}
\put(7,.2){\makebox(1,.5)[b]{\textit{right}}}
\end{picture}
\end{center}
And by ``clearing'' space in both places with |\hspace*{-2pc}|
the area becomes:
\begin{center}
\begin{picture}(8,2.2)
\put(1,1){\line(1,0){6.5}}
\put(7.5,1){\line(0,1){.5}}
\put(7.5,1.5){\line(-1,0){.5}}
\put(.5,2){\line(1,0){6.5}}
\put(1,1.5){\line(-1,0){.5}}
\put(.5,1.5){\line(0,1){.5}}
\put(1,1){\line(0,1){.5}}
\put(7,1.5){\line(0,1){.5}}
\put(0,.7){\vector(1,0){1}}
\put(8,.7){\vector(-1,0){1}}
\put(0,.2){\makebox(1,.5)[b]{\textit{left}}}
\put(7,.2){\makebox(1,.5)[b]{\textit{right}}}
\end{picture}
\end{center}

\begin{multicols}{2}
If you have seen tocs, the latter should be familiar to you--
the label at the very beginning, the page at the very end:
\columnbreak

\topline
\begin{verbatim}
    3.2  This is an example showing that toc
         entries fits in that scheme . . . .   4
\end{verbatim}
\bottomline
\end{multicols}


\begin{teX}
\newcommand\@pnumwidth{1.55em}%Width of box in which page number is set.
\end{teX}

We then define the margin and the dotsep. We also set the toc counter to whatever is require (don't go too deep especially if you have an index).

\begin{teX}
\newcommand\@tocrmarg{2.55em}%Right margin indentation for all but last line of multiple-line entries.
\newcommand\@dotsep{4.5}%Separation between dots, in mu units. Should be \def'd to a number like
2 or 1.7
\end{teX}

\begin{macro}{\tableofcontents}
\Paragraph{Defining the  contents table.} The author is provided with the author command |\tableofcontents|. All format information is provided at this point.
\end{macro}

\begin{teX}
\setcounter{tocdepth}{2}
\newcommand\tableofcontents{%
    \if@twocolumn
      \@restonecoltrue\onecolumn
    \else
      \@restonecolfalse
    \fi
    \chapter*{\contentsname
        \@mkboth{%
           \MakeUppercase\contentsname}{\MakeUppercase\contentsname}}%
    \@starttoc{toc}%
    \if@restonecol\twocolumn\fi
    }
\end{teX}

\begin{teX}
\newcommand*\l@part[2]{%
  \ifnum \c@tocdepth >-2\relax
    \addpenalty{-\@highpenalty}%
    \addvspace{2.25em \@plus\p@}%
    \setlength\@tempdima{3em}%
    \begingroup
      \parindent \z@ \rightskip \@pnumwidth
      \parfillskip -\@pnumwidth
      {\leavevmode
       \large \bfseries #1\hfil \hb@xt@\@pnumwidth{\hss #2}}\par
       \nobreak
         \global\@nobreaktrue
         \everypar{\global\@nobreakfalse\everypar{}}%
    \endgroup
  \fi}

\newcommand*\l@chapter[2]{%
  \ifnum \c@tocdepth >\m@ne
    \addpenalty{-\@highpenalty}%
    \vskip 1.0em \@plus\p@
    \setlength\@tempdima{1.5em}%
    \begingroup
      \parindent \z@ \rightskip \@pnumwidth
      \parfillskip -\@pnumwidth
      \leavevmode \bfseries
      \advance\leftskip\@tempdima
      \hskip -\leftskip
      #1\nobreak\hfil \nobreak\hb@xt@\@pnumwidth{\hss #2}\par
      \penalty\@highpenalty
    \endgroup
  \fi}
\end{teX}


The five remaining levels (entry in \latex terminology, are defined next). This is done with the general \latex kernel command 

\begin{teX}
\@dottedtocline{<level>}{<indent>}{<numwidth>}{<title>}{<page>}: Macro
to produce a table of contents line with the following parameters:
\end{teX}

The commands for the remaining sections are defined as follows:

\begin{teX}
\newcommand*\l@section{\@dottedtocline{1}{1.5em}{2.3em}}
\newcommand*\l@subsection{\@dottedtocline{2}{3.8em}{3.2em}}
\newcommand*\l@subsubsection{\@dottedtocline{3}{7.0em}{4.1em}}
\newcommand*\l@paragraph{\@dottedtocline{4}{10em}{5em}}
\newcommand*\l@subparagraph{\@dottedtocline{5}{12em}{6em}}
\end{teX}

So where are the last two parameters? These are just zeroed here!


I can assure that the |dotted| type of section bothers a lot of people. Most new books will both compact the table of contents as well as remove the dots. You can use the |titlesec| and |titletoc| to do this rather than redefining the kernel commands or the standard classes styles.

\subsection*{List of figures, tables etc}
\begin{teX}
\newcommand\listoffigures{%
    \if@twocolumn
      \@restonecoltrue\onecolumn
    \else
      \@restonecolfalse
    \fi
    \chapter*{\listfigurename}%
      \@mkboth{\MakeUppercase\listfigurename}%
              {\MakeUppercase\listfigurename}%
    \@starttoc{lof}%
    \if@restonecol\twocolumn\fi
    }
\end{teX}
The interesting command here is the |@starttoc{lof}|. This simply does all the housekeeping to open a file. as you can see it is not too difficult to have file extension names other than the standard ones.

The |l@| commands for the Table of Contents are defined as per the rest of the sectioning commands.

\begin{teX}
\newcommand*\l@figure{\@dottedtocline{1}{1.5em}{2.3em}}
\newcommand\listoftables{%
    \if@twocolumn
      \@restonecoltrue\onecolumn
    \else
      \@restonecolfalse
    \fi
    \chapter*{\listtablename}%
      \@mkboth{%
          \MakeUppercase\listtablename}%
         {\MakeUppercase\listtablename}%
    \@starttoc{lot}%
    \if@restonecol\twocolumn\fi
    }
\let\l@table\l@figure
\end{teX}

\section*{Bibliographies}

\begin{multicols}{2}
\latex provides some basic bibliographic commands. Every entry is defined to be displayed in a block. It starts by defining a new length |\bibindent|. Entries are displayed using the  |\list|. The commands here are mainly to set parameters for macros already provide by the kernel.
\end{multicols}
\index{book!environments!thebibliography}

\begin{teX}
\newdimen\bibindent
\setlength\bibindent{1.5em}
\newenvironment{thebibliography}[1]
     {\chapter*{\bibname}%
      \@mkboth{\MakeUppercase\bibname}{\MakeUppercase\bibname}%
      \list{\@biblabel{\@arabic\c@enumiv}}%
           {\settowidth\labelwidth{\@biblabel{#1}}%
            \leftmargin\labelwidth
            \advance\leftmargin\labelsep
            \@openbib@code
            \usecounter{enumiv}%
            \let\p@enumiv\@empty
            \renewcommand\theenumiv{\@arabic\c@enumiv}}%
      \sloppy
      \clubpenalty4000
      \@clubpenalty \clubpenalty
      \widowpenalty4000%
      \sfcode`\.\@m}
     {\def\@noitemerr
       {\@latex@warning{Empty `thebibliography' environment}}%
      \endlist}
\newcommand\newblock{\hskip .11em\@plus.33em\@minus.07em}
\let\@openbib@code\@empty
\end{teX}

\section*{The Index Environment}
This is a short environment definition for styling the Index. It defines in line [\ref{idxitem}] the 
|@idxidtem|, which is then used to define \cmd{subitem} and \cmd{subsubitem} styling.

\begin{teX}
\newenvironment{theindex}
   {\if@twocolumn
      \@restonecolfalse
      \else
         \@restonecoltrue
      \fi
      \twocolumn[\@makeschapterhead{\indexname}]%
      \@mkboth{\MakeUppercase\indexname}%
              {\MakeUppercase\indexname}%
                \thispagestyle{plain}\parindent\z@
                \parskip\z@ \@plus .3\p@\relax
                \columnseprule \z@
                \columnsep 35\p@
                \let\item\@idxitem}
      {\if@restonecol\onecolumn\else\clearpage\fi}
\newcommand\@idxitem{\par\hangindent 40\p@} (*@\label{idxitem}@*)
\newcommand\subitem{\@idxitem \hspace*{20\p@}}
\newcommand\subsubitem{\@idxitem \hspace*{30\p@}}
\newcommand\indexspace{\par \vskip 10\p@ \@plus5\p@ \@minus3\p@\relax}
\end{teX}

\section{Footnotes}
\label{book:footnotes}
\index{footnotes>\textbackslash footnoterule}

\begin{macro}{\footnoterule}
\begin{macro}{\@makefntext}
\begin{macro}{\@makefnmark}
\Paragraph{Footnote rules.} Footnote rules are defined by renewing the command |\footnoterule|. Counters for footnotes are reset based on the chapter counters. The footnote command |\@makefntext| provides the formatting. It also gives the user the ability to use these to insert footnotes, in difficult places.
\end{macro}
\end{macro}
\end{macro}

\begin{teX}
\renewcommand\footnoterule{%
  \kern-3\p@
  \hrule\@width.4\columnwidth 
  \kern2.6\p@}

\@addtoreset{footnote}{chapter}

\newcommand\@makefntext[1]{%
    \parindent 1em%
    \noindent
    \hb@xt@1.8em{\hss\@makefnmark}#1}
\end{teX}

\section*{Catering for Other Languages}

\begin{multicols}{2}
\textbf{Structural element names.}\quad \latex does not provide by itself the means to change the structural element names to a language other than English. Howerer, their names are defined  in a series of commands, that make it easier to be overwritten to change them to another language. As they are separate from the macros that use them, it is easy to overwrite them, in order to use another language. This is what the Babel package does. Note the Section, is not defined here.
\end{multicols}

\begin{teX}
\newcommand\contentsname{Contents}
\newcommand\listfigurename{List of Figures}
\newcommand\listtablename{List of Tables}
\newcommand\bibname{Bibliography}
\newcommand\indexname{Index}
\newcommand\figurename{Figure}
\newcommand\tablename{Table}
\newcommand\partname{Part}
\newcommand\chaptername{Chapter}
\newcommand\appendixname{Appendix}
\end{teX}

\textbf{Dates} Not much of a use but the month names are also defined here in an |\ifcase| statement. Again they can be overwritten by Babel.


\begin{teX}
\def\today{\ifcase\month\or
  January\or February\or March\or April\or May\or June\or
  July\or August\or September\or October\or November\or December\fi
  \space\number\day, \number\year}
\end{teX}

\Paragraph{\bf Multicolumn gutter and rule.}\quad Here two lengths are set. The distance between two columns of text and the width of the separating rule.

\begin{teX}
\setlength\columnsep{10\p@}
\setlength\columnseprule{0\p@}
\end{teX}


\section{Final}

\begin{teX}
\pagestyle{headings}
\pagenumbering{arabic}
\if@twoside
\else
  \raggedbottom
\fi
\if@twocolumn
  \twocolumn
  \sloppy
  \flushbottom
\else
  \onecolumn
\fi
\endinput
%%
%% End of file `book.cls'.

\end{teX}

\section{Ending remarks}

It is to the credit of Lamport and his associates that he was the first one to produce a system of mark-up that structured documents, using the TeX typographical engine. The class is widely used and many variants exist. One area that can be improved is to provide more `hooks' to enable programmers to redefine classes more easily.

Since the class has been published new packages have established themselves as the `de facto` standards of defining portions of the class. For example the no-new class will attempt to define all the papers as Lamport did, but would rather use the |geometry| package to do so. Top and bottom headings are defined using the |fancyverb|. 

\begin{quotation}
It was when the code was written, but is not now (in my opinion). The current LaTeX2e kernel was release in 1992 and carries forward a lot of material from LaTeX2.09. Even with these optimizations and the old 'autoload' system, there were a lot of systems that LaTeX was too big for on release. So looked at in the early 1990s this was entirely sensible.

I'd say this is no longer needed as in most LaTeX documents today there are a lot of tokens used by things like pgf which make the modest saving in optimisation pretty meaningless. One of the things we're doing in LaTeX3 is trying to move to more logical constructs at the expense of efficiency in tokens, at least at a higher level. (Right at the core of expl3 there is still a need to watch the number of expansions, etc., and this is an area where we may yet need some more optimisation.)

\end{quotation}

You can think of the \latex classes working at three levels. 

\begin{enumerate}
\item Selecting paper sizes and defining main page elements.
\item They define how the document is section. I have called this sectioning by referring to it as structural commands.
\item It provides the typesetting of these structural elements.
\end{enumerate}

Unfortunately, they are not separated in a way that makes it easy for them to be modified. A plethora of packages assists the author in modifying every type of sectioning and formatting decisions of Lamport. Most authors will focus on the formatting commands. Some will add a bit of structure, perhaps some special sections for questions and answers. If you have used the titlesec package for modifying the sections, the caption package for modifying the way captions are displayed, the fancyhdr for headers, the titletoc for the way table of contents are displayed, one of the bibliography packages what begs to be question is what remains? Very little. You might as well at this point decide on a new class. It will be more efficient and you will have better control. Separation of structure from presentational decisions is important. Some common structural elements that are missing should be integrated in. The KOMA classes and memoir went totally overboard, in that they try to be everything to everybody. A system that is nearer to defining a structural template and then decorate it with a selction of fonts, colors, spacing and the like would have been more appropriate.

\begin{teX}
%%
%% This is file `bk10.clo',

\ProvidesFile{bk10.clo}
              [2007/10/19 v1.4h
      Standard LaTeX file (size option)]
\end{teX}

\begin{teX}
\renewcommand\normalsize{%
   \@setfontsize\normalsize\@xpt\@xiipt
   \abovedisplayskip 10\p@ \@plus2\p@ \@minus5\p@
   \abovedisplayshortskip \z@ \@plus3\p@
   \belowdisplayshortskip 6\p@ \@plus3\p@ \@minus3\p@
   \belowdisplayskip \abovedisplayskip
   \let\@listi\@listI}
\normalsize
\newcommand\small{%
   \@setfontsize\small\@ixpt{11}%
   \abovedisplayskip 8.5\p@ \@plus3\p@ \@minus4\p@
   \abovedisplayshortskip \z@ \@plus2\p@
   \belowdisplayshortskip 4\p@ \@plus2\p@ \@minus2\p@
   \def\@listi{\leftmargin\leftmargini
               \topsep 4\p@ \@plus2\p@ \@minus2\p@
               \parsep 2\p@ \@plus\p@ \@minus\p@
               \itemsep \parsep}%
   \belowdisplayskip \abovedisplayskip
}
\newcommand\footnotesize{%
   \@setfontsize\footnotesize\@viiipt{9.5}%
   \abovedisplayskip 6\p@ \@plus2\p@ \@minus4\p@
   \abovedisplayshortskip \z@ \@plus\p@
   \belowdisplayshortskip 3\p@ \@plus\p@ \@minus2\p@
   \def\@listi{\leftmargin\leftmargini
               \topsep 3\p@ \@plus\p@ \@minus\p@
               \parsep 2\p@ \@plus\p@ \@minus\p@
               \itemsep \parsep}%
   \belowdisplayskip \abovedisplayskip
}
\newcommand\scriptsize{\@setfontsize\scriptsize\@viipt\@viiipt}
\newcommand\tiny{\@setfontsize\tiny\@vpt\@vipt}
\newcommand\large{\@setfontsize\large\@xiipt{14}}
\newcommand\Large{\@setfontsize\Large\@xivpt{18}}
\newcommand\LARGE{\@setfontsize\LARGE\@xviipt{22}}
\newcommand\huge{\@setfontsize\huge\@xxpt{25}}
\newcommand\Huge{\@setfontsize\Huge\@xxvpt{30}}
\end{teX}

\begin{multicols}{2}
\Paragraph{Indentation.} Paragraph indentation is controlled by the \tex command |parindent|. It is set narrower in two column text, to avoid problems with hyphenation that can result in overfull boxes.\index{Typography rules! paragraph!parindent}

1em rule \rule{1em}{1ex}  and 15pt rule \rule{15pt}{1ex} and 1.5em \rule{1.5em}{1ex}
\end{multicols}

\begin{teX}
\if@twocolumn
  \setlength\parindent{1em}
\else
  \setlength\parindent{15\p@}
\fi

\setlength\smallskipamount{3\p@ \@plus 1\p@ \@minus 1\p@}
\setlength\medskipamount{6\p@ \@plus 2\p@ \@minus 2\p@}
\setlength\bigskipamount{12\p@ \@plus 4\p@ \@minus 4\p@}
\setlength\headheight{12\p@}
\setlength\headsep   {.25in}
\setlength\topskip   {10\p@}
\setlength\footskip{.35in}
\if@compatibility \setlength\maxdepth{4\p@} \else
\setlength\maxdepth{.5\topskip} \fi
\if@compatibility
  \if@twocolumn
    \setlength\textwidth{410\p@}
  \else
    \setlength\textwidth{4.5in}
  \fi
\else
  \setlength\@tempdima{\paperwidth}
  \addtolength\@tempdima{-2in}
  \setlength\@tempdimb{345\p@}
  \if@twocolumn
    \ifdim\@tempdima>2\@tempdimb\relax
      \setlength\textwidth{2\@tempdimb}
    \else
      \setlength\textwidth{\@tempdima}
    \fi
  \else
    \ifdim\@tempdima>\@tempdimb\relax
      \setlength\textwidth{\@tempdimb}
    \else
      \setlength\textwidth{\@tempdima}
    \fi
  \fi
\fi
\if@compatibility\else
  \@settopoint\textwidth
\fi
\if@compatibility
  \setlength\textheight{41\baselineskip}
\else
  \setlength\@tempdima{\paperheight}
  \addtolength\@tempdima{-2in}
  \addtolength\@tempdima{-1.5in}
  \divide\@tempdima\baselineskip
  \@tempcnta=\@tempdima
  \setlength\textheight{\@tempcnta\baselineskip}
\fi
\addtolength\textheight{\topskip}
\if@twocolumn
 \setlength\marginparsep {10\p@}
\else
  \setlength\marginparsep{7\p@}
\fi
\setlength\marginparpush{5\p@}
\if@compatibility
   \setlength\oddsidemargin   {.5in}
   \setlength\evensidemargin  {1.5in}
   \setlength\marginparwidth {.75in}
  \if@twocolumn
     \setlength\oddsidemargin  {30\p@}
     \setlength\evensidemargin {30\p@}
     \setlength\marginparwidth {48\p@}
  \fi
\else
  \if@twoside
    \setlength\@tempdima        {\paperwidth}
    \addtolength\@tempdima      {-\textwidth}
    \setlength\oddsidemargin    {.4\@tempdima}
    \addtolength\oddsidemargin  {-1in}
    \setlength\marginparwidth   {.6\@tempdima}
    \addtolength\marginparwidth {-\marginparsep}
    \addtolength\marginparwidth {-0.4in}
  \else
    \setlength\@tempdima        {\paperwidth}
    \addtolength\@tempdima      {-\textwidth}
    \setlength\oddsidemargin    {.5\@tempdima}
    \addtolength\oddsidemargin  {-1in}
    \setlength\marginparwidth   {.5\@tempdima}
    \addtolength\marginparwidth {-\marginparsep}
    \addtolength\marginparwidth {-0.4in}
    \addtolength\marginparwidth {-.4in}
  \fi
  \ifdim \marginparwidth >2in
     \setlength\marginparwidth{2in}
  \fi
  \@settopoint\oddsidemargin
  \@settopoint\marginparwidth
  \setlength\evensidemargin  {\paperwidth}
  \addtolength\evensidemargin{-2in}
  \addtolength\evensidemargin{-\textwidth}
  \addtolength\evensidemargin{-\oddsidemargin}
  \@settopoint\evensidemargin
\fi
\end{teX}

\begin{multicols}{2}
\Paragraph{Top margin} Next the top margin is calculated.  In earlier versions the |\topmargin| was a fixed number. In this class, it is automatically calculated form the |\paperheight| (as the user only inputs the papersize through one of the paper selection options).
\end{multicols}

\begin{teX}
\if@compatibility
  \setlength\topmargin{.75in}
\else
  \setlength\topmargin{\paperheight}
  \addtolength\topmargin{-2in}
  \addtolength\topmargin{-\headheight}
  \addtolength\topmargin{-\headsep}
  \addtolength\topmargin{-\textheight}
  \addtolength\topmargin{-\footskip}     % this might be wrong! (previously set at 0.35in)
  \addtolength\topmargin{-.5\topmargin}
  \@settopoint\topmargin
\fi
\end{teX}

The lists settings follow. Similarly all values are hard-coded based on the font size.
\begin{teX}
\setlength\footnotesep{6.65\p@}
\setlength{\skip\footins}{9\p@ \@plus 4\p@ \@minus 2\p@}
\setlength\floatsep    {12\p@ \@plus 2\p@ \@minus 2\p@}
\setlength\textfloatsep{20\p@ \@plus 2\p@ \@minus 4\p@}
\setlength\intextsep   {12\p@ \@plus 2\p@ \@minus 2\p@}
\setlength\dblfloatsep    {12\p@ \@plus 2\p@ \@minus 2\p@}
\setlength\dbltextfloatsep{20\p@ \@plus 2\p@ \@minus 4\p@}
\setlength\@fptop{0\p@ \@plus 1fil}
\setlength\@fpsep{8\p@ \@plus 2fil}
\setlength\@fpbot{0\p@ \@plus 1fil}
\setlength\@dblfptop{0\p@ \@plus 1fil}
\setlength\@dblfpsep{8\p@ \@plus 2fil}
\setlength\@dblfpbot{0\p@ \@plus 1fil}
\setlength\partopsep{2\p@ \@plus 1\p@ \@minus 1\p@}
\def\@listi{\leftmargin\leftmargini
            \parsep 4\p@ \@plus2\p@ \@minus\p@
            \topsep 8\p@ \@plus2\p@ \@minus4\p@
            \itemsep4\p@ \@plus2\p@ \@minus\p@}
\let\@listI\@listi
\@listi
\def\@listii {\leftmargin\leftmarginii
              \labelwidth\leftmarginii
              \advance\labelwidth-\labelsep
              \topsep    4\p@ \@plus2\p@ \@minus\p@
              \parsep    2\p@ \@plus\p@  \@minus\p@
              \itemsep   \parsep}
\def\@listiii{\leftmargin\leftmarginiii
              \labelwidth\leftmarginiii
              \advance\labelwidth-\labelsep
              \topsep    2\p@ \@plus\p@\@minus\p@
              \parsep    \z@
              \partopsep \p@ \@plus\z@ \@minus\p@
              \itemsep   \topsep}
\def\@listiv {\leftmargin\leftmarginiv
              \labelwidth\leftmarginiv
              \advance\labelwidth-\labelsep}
\def\@listv  {\leftmargin\leftmarginv
              \labelwidth\leftmarginv
              \advance\labelwidth-\labelsep}
\def\@listvi {\leftmargin\leftmarginvi
              \labelwidth\leftmarginvi
              \advance\labelwidth-\labelsep}
\endinput
%%
%% End of file `bk10.clo'.

\end{teX}












     \input{./sections/latexkernel}
     \chapter{Float Class}
\label{ch:ltfloat}

|\ProvidesFile{ltfloat.dtx}[2002/10/01 v1.1v LaTeX Kernel (Floats)]|

 \section{Float types}

  The different types of floats are identified by a \meta{type} name,
  which is the name of the counter for that kind of float.  For
  example, figures are of type \emph{figure} and tables are of type \emph{table}.
  Each \meta{type} has associated a positive \meta{type number}, which
  is a \textbf{power of two}  e.g.,\\
  figures might be have type number~1, tables type number~2, programs
  type number~4, etc.

  The locations where a float can go are specified by a
  \meta{placement specifier}, which is a list of the possible
  locations, each denoted by a letter as follows:
  
  \begin{center}
    \begin{tabular}{l@{ : }l@{ --- }l}
     h & here   & at the current location in the text.\\
     t & top    & at the top of a text page.\\
     b & bottom & at the bottom of a text page.\\
     p & page   & on a separate float page
    \end{tabular}
  \end{center}
  
  In addition, in conjunction with these, you can use `!' which means
  that the current values of the float positioning parameters are
  ignored for this float. (Has no effect on `p', float page
  positioning.)
  For example, `pht' specifies that the float can appear in any of
  three locations: page, here or top.


 \subsection{Floating Environments}
    \begin{teX}
\message{floats,}
    \end{teX}


 Where floats may appear on a page, and how many may appear there
 are specified by the following float placement parameters.  The
 numbers are named like counters so the user can set them with
 the ordinary counter-setting commands.
 
 \begin{description}
  \item[\string\c@topnumber] Number of floats allowed at the top of a column.
  
  \item[\string\topfraction] Fraction of column that can be devoted to floats.
  
  \item[\string\c@dbltopnumber, \string\dbltopfraction]  Same as above, but for double-column floats.

  \item[\string\c@bottomnumber, \string\bottomfraction ]
                     Same as above for bottom of page.
 
 \item[\string\c@totalnumber]    Number of floats allowed in a single column,
                          including in-text floats.
 
 \item[\string\textfraction]     Minimum fraction of column that must contain text.

  \item[\string\floatpagefraction] Minimum fraction of page that must be taken
                          up by float page.
                          
\item[\string\dblfloatpagefraction]   Same as above, for double-column floats.
\end{description}

 The document style must define the following.

   | \fps@TYPE|   : The default placement specifier for floats of type
                  TYPE.  see \ref{book:figure}

    |\ftype@TYPE|  The type number for floats of type TYPE. see \ref{book:figure}

    |\ext@TYPE|   The file extension indicating the file on which the
                  contents list for float type TYPE is stored.

  For example,  |\ext@figure = 'lof'|.

   | \fnum@TYPE|  A macro to generate the figure number for a caption.
                  For example, |\fnum@TYPE == Figure \thefigure.|

   | \@makecaption{NUM}{TEXT} |
              A macro to make a caption, with NUM the value
              produced by |\fnum@..|. and TEXT the text of the caption.
              It can assume it's in a |\parbox| of the appropriate width.

  |\@float{TYPE}[PLACEMENT]| : This macro begins a float environment for a
     single-column float of type TYPE with PLACEMENT as the placement
     specifier.  The default value of PLACEMENT is defined by
     |\fps@TYPE|.  The environment is ended by |\end@float|.
     E.g., |\figure == \@float{figure}, \endfigure == \end@float|.

%  \@float{TYPE}[PLACEMENT] ==
%   BEGIN
%     if hmode then \@bsphack
%                   \@floatpenalty := -10002
%              else \@floatpenalty := -10003
%     fi
%     \@captype ==L TYPE
%     \@dblflset
%     \@fps     ==L PLACEMENT
%     \@onelevel@sanitize \@fps 
%     add default PLACEMENT if at most ! in PLACEMENT == \@fpsadddefault
%     if inner
%       then LaTeX Error: 'Not in outer paragraph mode.'
%            \@floatpenalty := 0
%       else if \@freelist nonempty
%              then \@currbox  :=L head of \@freelist
%                   \@freelist :=G tail of \@freelist
%                   \count\@currbox :=G 32*\ftype@TYPE + 
%                                          bits determined by PLACEMENT
%              else \@floatpenalty := 0
%                   LaTeX Error: 'Too many unprocessed floats'
%            fi
%     fi
%     \@currbox :=G   \color@vbox
%                       \normalcolor
%                         \vbox{
%                          %% 15 Dec 87 --
%                          %% removed \boxmaxdepth :=L 0pt
%                          %% that made box 0 depth because it screwed
%                          %% things up. Instead, added \vskip0pt at end
%                               \hsize = \columnwidth
%                               \@parboxrestore
%                               \@floatboxreset
%   END
%
%  \caption ==
%    BEGIN
%     \refstepcounter{\@captype}
%     \@dblarg{\@caption{\@captype}}
%    END
%
% In following definition, \par moved from after \addcontentsline to
% before \addcontentsline because the \write could cause
% an extra blank line to be added to the paragraph above the
% caption.  (Change made 12 Jun 87)
%
%  \@caption{TYPE}[STEXT]{TEXT} ==
%   BEGIN
%     \par
%     \addcontentsline{\ext@TYPE}{TYPE}{\numberline{\theTYPE}{STEXT}}
%     \begingroup
%       \@parboxrestore
%       \@normalsize
%       \@makecaption{\fnum@TYPE}{TEXT}
%       \par
%     \endgroup
%   END
%
%
%  \@dblfloat{TYPE}[PLACEMENT] : Macro to begin a float environment for
%     a double-column float of type TYPE with PLACEMENT as the placement
%     specifier.  The default value of PLACEMENT is 'tp'
%     The environment is ended by \end@dblfloat.
%     E.g., \figure* == \@dblfloat{figure}, 
%           \endfigure* == \end@dblfloat.
%
%  \@dblfloat{TYPE}[PLACEMENT] ==
%     Identical to \@float{TYPE}[PLACEMENT] except \hsize and \linewidth
%     are set to \textwidth.
% \end{oldcomments}
%
 \begin{docCommand}{@floatpenalty}{}
    \begin{teX}
\newcount\@floatpenalty
    \end{teX}
 \end{docCommand}
%
%
 \begin{docCommand}{caption}{}

    This is set to be an error message outside a float since no
    |\@captype| is defined there; this may need to be changed by some 
    classes.

    \begin{teX}
\def\caption{%
   \ifx\@captype\@undefined
     \@latex@error{\noexpand\caption outside float}\@ehd
     \expandafter\@gobble
   \else
     \refstepcounter\@captype
     \expandafter\@firstofone
   \fi
   {\@dblarg{\@caption\@captype}}%
} 
    \end{teX}
 \end{docCommand}
%
 \begin{docCommand}{@caption}{}
% \changes{v1.0b}{1994/03/28}
%     {Use \cs{normalsize} not \cs{@normalsize}}
% \changes{v1.1r}{1996/12/06}
%     {Call \cs{@setminpage} if needed. latex/2318}
    \begin{teX}
\long\def\@caption#1[#2]#3{%
  \par
  \addcontentsline{\csname ext@#1\endcsname}{#1}%
    {\protect\numberline{\csname the#1\endcsname}{\ignorespaces #2}}%
  \begingroup
    \end{teX}

 The paragraph setting parameters are normalised at this point, however
 |\@parboxrestore| resets |\everypar| which is not correct in this
 context so |\@setminipage| is called if needed.

 The float mechanism, like minipage, sets the flag |@minipage| true
 before executing the user-supplied text. Many \LaTeX\ constructs
 test for this flag and do not add vertical space when it is true.
 The intention is that this emulates \TeX's `top of page' behaviour.
 The flag must be set false at the start of the first paragraph. This
 is achieved by a redefinition of |\everypar|, but the call to
 |\@parboxrestore| removes that redefinition, so it is re-inserted 
 if needed. If the flag is already false then the |\caption| was not
 the first entry in the float, and so some other paragraph has already
 activated the special |\everypar|. In this case no further action is
 needed.
 \begin{teX}
    \@parboxrestore
    \if@minipage
      \@setminipage
    \fi
  \end{teX}
%
    \begin{teX}
    \normalsize
    \@makecaption{\csname fnum@#1\endcsname}{\ignorespaces #3}\par
  \endgroup}
    \end{teX}
 \end{docCommand}
%
 \begin{docCommand}{@float} {}
 \begin{docCommand}{@dblflset} {}
% \changes{v1.1a}{1994/10/31}{Macro added}
% \changes{v1.1g}{1994/12/10}{Macro removed temporarily}
% \changes{v1.1a}{1994/10/31}
%    {Major changes to parameter parsing, setting of local variables,
%    etc; two-column and one-column cases merged; space hacks moved}
% \changes{v1.1c}{1994/11/05}
%         {Add compatibility with old version of \cs{@xfloat}.}
% \changes{v1.1g}{1994/12/10}{Old version reinstated temporarily}
%
   \begin{teX}
\def\@float#1{%
  \@ifnextchar[%
    {\@xfloat{#1}}%
    {\edef\reserved@a{\noexpand\@xfloat{#1}[\csname fps@#1\endcsname]}%
     \reserved@a}}
  \end{teX}
%    
 \end{docCommand}
 \end{docCommand}
%
  \begin{docCommand}{@dblfloat}{}
% \changes{v1.1a}{1994/10/31}
%     {Major changes since two-column and one-column cases merged}
% \changes{v1.1g}{1994/12/10}{Old version reinstated temporarily}
%
    \begin{teX}
\def\@dblfloat{%
  \if@twocolumn\let\reserved@a\@dbflt\else\let\reserved@a\@float\fi
  \reserved@a}
    \end{teX}
  \end{docCommand}
%
%    
  \begin{docCommand}{fps@dbl}{}
% \changes{v1.1a}{1994/10/31}{Macro added}
% \changes{v1.1g}{1994/12/10}{Macro removed temporarily}
  Note that all double floats have default fps `tp'.
  \end{docCommand}
%  
  \begin{docCommand}{@setfps} {}
% \changes{v1.1a}{1994/10/31}{Macro added}
% \changes{v1.1c}{1994/11/05}
%         {Add compatibility with old version of \cs{@xfloat}.}
% \changes{v1.1g}{1994/12/10}{Macro removed temporarily}
    This sets the fps, dealing with error conditions by adding
    the default.
%    
  \end{docCommand}
%
  \begin{docCommand}{@xfloat} {}
% \changes{LaTeX2e}{1993/12/05}{Command changed}
% \changes{LaTeX2e}{1994/01/21}{Added missing percent characters.}
% \changes{v1.1a}{1994/10/31}
%     {Major changes, removing setting of local variables, space hacks
%     etc; two-column and one-column cases merged}
% \changes{v1.1c}{1994/11/05}
%         {Add compatibility with old version of \cs{@xfloat}: but the
%         arguments, provided at exorbitant cost, are now completely
%         ignored}
% \changes{v1.1f}{1994/11/21}
%     {Missing percents reinserted after 4, 8: these are not numbers.}
% \changes{v1.1g}{1994/12/10}{Old version reinstated temporarily}
% \changes{v1.1g}{1994/12/10}{Sanitisation added temporarily}
     The first part of this sets the count register that stores all
     the information about the type and fps of the float.
%
    We assume here that the default specifiers already contain no
    active characters.
%
    It may be better to store the defaults as numbers, rather than
    symbol strings.
%
% \changes{v1.1p}{1996/10/24}{Added \cs{@nodocument} to trap
%                  floats in the preamble}
    \begin{teX}
\def\@xfloat #1[#2]{%
  \@nodocument % traps floats in preamble 
  \def \@captype {#1}%
   \def \@fps {#2}%
   \@onelevel@sanitize \@fps 
   \def \reserved@b {!}%
   \ifx \reserved@b \@fps
     \@fpsadddefault
   \else
     \ifx \@fps \@empty
       \@fpsadddefault
     \fi
   \fi
   \ifhmode
     \@bsphack
     \@floatpenalty -\@Mii
   \else
     \@floatpenalty-\@Miii
   \fi
  \ifinner
     \@parmoderr\@floatpenalty\z@
  \else
    \@next\@currbox\@freelist
      {%
       \@tempcnta \sixt@@n
       \expandafter \@tfor \expandafter \reserved@a
         \expandafter :\expandafter =\@fps 
         \do
          {%
           \if \reserved@a h%
             \ifodd \@tempcnta
             \else
               \advance \@tempcnta \@ne
             \fi
           \fi
           \if \reserved@a t%
             \@setfpsbit \tw@
           \fi
           \if \reserved@a b%
             \@setfpsbit 4%
           \fi
           \if \reserved@a p%
             \@setfpsbit 8%
           \fi
           \if \reserved@a !%
             \ifnum \@tempcnta>15
               \advance\@tempcnta -\sixt@@n\relax
             \fi
           \fi
           }%
       \@tempcntb \csname ftype@\@captype \endcsname
       \multiply \@tempcntb \@xxxii
       \advance \@tempcnta \@tempcntb
       \global \count\@currbox \@tempcnta
       }%
    \@fltovf
  \fi
    \end{teX}

    The remainder sets up the box in which the float is typeset, and
    the typesetting environment to be used.  It is essential to have
    the extra box to avoid the unwanted space that would otherwise
    often be put at the top of the float.

    It ends with a hook; not sure how useful this is but it is needed
    at present to deal with double-column floats.
% \task{CAR?}{Sort out hooks}
% \changes{v1.0a}{1994/03/07}
%     {(DPC) Extra group for colour}
% \changes{v1.0c}{1994/03/14}
%     {(DPC) Use \cs{color@begingroup}}
% \changes{v1.0g}{1994/05/13}
%     {(DPC) Use \cs{normalcolor}}
% \changes{v1.1a}{1994/10/31}
%     {(DPC/CAR) Extra box added to remove colour resetting from vmode}
% \changes{v1.1a}{1994/10/31}{Reset hook added}
% \changes{v1.1c}{1994/11/05}
%         {Use new \cs{color@hbox} concept.}
% \changes{v1.1f}{1994/11/21}
%         {Changed to \cs{color@vbox} so that large floats overflow
%          at the bottom}
% \changes{v1.1f}{1994/11/21}{Use \cs{@setnobreak}}
% \changes{v1.1f}{1994/11/21}{Added \cs{@setminipage}}
% \changes{v1.1f}{1994/11/21}{Added resetting of size and font}
% \changes{v1.1m}{1995/05/25}{(CAR) Resettings moved to hook}
    \begin{teX}
  \global \setbox\@currbox
    \color@vbox
      \normalcolor
      \vbox \bgroup
        \hsize\columnwidth
        \@parboxrestore
        \@floatboxreset
}
    \end{teX}
  \end{docCommand}
%  
  \begin{docCommand}{@floatboxreset} {}
% \changes{v1.1a}{1994/10/31}{Macro added}
%    
 The rational for allowing these normally global flags to be set
 locally here, via |\@parboxrestore|, was stated originally by
 Donald Arseneau and extended by Chris Rowley.
 It is because these flags are only set globally to
 true by section commands, and these should never appear within
 marginals or floats or, indeed, in any group; and they are only ever
 set globally to false when they are definitely true.

 If anyone is unhappy with this argument then both flags should be
 treated as in |\set@nobreak|; otherwise this command will be
 redundant. 
% \changes{v1.1p}{1996/10/24}
%     {Added local settings of flags: dangerous!!}
    \begin{teX}
\def \@floatboxreset {%
        \reset@font
        \normalsize
        \@setminipage
}
    \end{teX}
  \end{docCommand}
%  
  \begin{docCommand}{@setnobreak} {}
% \changes{v1.1f}{1994/11/21}{Macro added}
% \changes{v1.1n}{1996/07/26}{remove unecessary \cs{global} before
%                 \cs{@nobreak...}}   
    \begin{teX}
\def \@setnobreak{%
  \if@nobreak
    \let\outer@nobreak\@nobreaktrue
    \@nobreakfalse
  \fi
}
    \end{teX}
  \end{docCommand}
%
  \begin{docCommand}{@setminipage} {}
% \changes{v1.1f}{1994/11/21}{Macro added}
% \changes{v1.1n}{1996/07/26}{remove unecessary \cs{global} before
%                 \cs{@minipage...}}   
    \begin{teX}
\def \@setminipage{%
  \@minipagetrue
  \everypar{\@minipagefalse\everypar{}}%
}
    \end{teX}
  \end{docCommand}
%
 \begin{docCommand}{end@float} {}
% \changes{v1.0f}{1994/05/03}
%     {(CAR) Added \cs{@largefloatcheck}}
% \changes{v1.1n}{1995/10/25}{(CAR) unify code for double and
%    single versions}
    \begin{teX}
\def\end@float{%
  \@endfloatbox
  \ifnum\@floatpenalty <\z@
    \end{teX}
    
 We make sure that we never exceed |\textheight|, otherwise float
 will never get typeset (91/03/15 FMi).
    
    \begin{teX}
    \@largefloatcheck
    \@cons\@currlist\@currbox
    \ifnum\@floatpenalty <-\@Mii
      \penalty -\@Miv
    \end{teX}

  Saving and restoring |\prevdepth| added 26 May 87 to prevent extra
 vertical space when used in vertical mode.

        \begin{teX}
      \@tempdima\prevdepth
      \vbox{}%
      \prevdepth\@tempdima
   \end{teX}

 \begin{teX}
      \penalty\@floatpenalty
 \end{teX}
% \changes{LaTeX2.09}{1992/03/18}
%         {(RmS) changed \cs{@esphack} to \cs{@Esphack}}
 \begin{teX}
    \else
      \vadjust{\penalty -\@Miv \vbox{}\penalty\@floatpenalty}\@Esphack
    \fi
  \fi
}
    \end{teX}
 \end{docCommand}
%
% \begin{docCommand}{end@dblfloat}
% \changes{v1.0f}{1994/05/03}{\cs{@largefloatcheck} added}
% \changes{v1.1n}{1995/10/25}{(CAR) unify code for double and
%    single versions}
    \begin{teX}
\def\end@dblfloat{%
\if@twocolumn
  \@endfloatbox
  \ifnum\@floatpenalty <\z@
%    \end{teX}
% We make sure that we never exceed |\textheight|, otherwise float
% will never get typeset (91/03/15 FMi).
    \begin{teX}
    \@largefloatcheck
    \@cons\@dbldeferlist\@currbox
  \fi
    \end{teX}
% RmS 92/03/18 changed |\@esphack| to |\@Esphack|.
    \begin{teX}
    \ifnum \@floatpenalty =-\@Mii \@Esphack\fi
\else
  \end@float
\fi
}
    \end{teX}
% \end{docCommand}
% 
 \begin{docCommand}{@endfloatbox} {}
% \changes{v1.1n}{1995/10/25}{(CAR) macro added: to unify code for
%    double and single versions}
%    This macro is not intended to be a hook; it is designed to help
%    maintain the integrity of this code, which is used twice and, as
%    can be seen, is subject to frequent changes.
    \begin{teX}
\def \@endfloatbox{%
      \par\vskip\z@skip      %% \par\vskip\z@ added 15 Dec 87
   \end{teX}
% \changes{v1.0a}
%     {1994/03/07}{(DPC) Extra group for colour}
% \changes{v1.0c}{1994/03/14}
%     {(DPC) Use \cs{color@endgroup}}
% \changes{v1.0h}{1994/05/20}{Restore outer value of @nobreak switch.}
% \changes{v1.1a}{1994/10/31}
%     {(DPC/CAR) Extra box added to remove colour resetting from vmode}
% \changes{v1.1c}{1994/11/05}
%         {Use new \cs{color@hbox} concept.}
% \changes{v1.1f}{1994/11/21}{Corrected position of \cs{outer@nobreak}}
% \changes{v1.1f}{1994/11/21}{Added reset of minipage flag}
% \changes{v1.1n}{1996/07/26}{remove unecessary \cs{global} before
%                 \cs{@minipage...}}   
 \begin{teX}
      \@minipagefalse   
      \outer@nobreak
    \egroup                  %% end of vbox
  \color@endbox
}
\end{teX}
\end{docCommand}
% 
 \begin{docCommand}{outer@nobreak} {}
% \changes{v1.0h}{1994/05/20}{Macro added: default is to do nothing.}
    \begin{teX}
\let\outer@nobreak\@empty
    \end{teX}
 \end{docCommand}
% 
%
  \begin{docCommand}{@largefloatcheck} {}
% \changes{v1.0e}{1994/04/25}{Command added}
% 
    This calculates by how much a float is oversize for the page and
    prints this in a warning message.
%    
    \begin{teX}  
\def \@largefloatcheck{%
  \ifdim \ht\@currbox>\textheight
    \@tempdima -\textheight
    \advance \@tempdima \ht\@currbox
    \end{teX}
% \changes{v1.0e}{1994/04/25}{Changed warning message to give more
% info}
    \begin{teX}
    \@latex@warning {Float too large for page by \the\@tempdima}%
    \ht\@currbox \textheight
  \fi
}
    \end{teX}
  \end{docCommand}
%
%  \begin{docCommand}{@dbflt}
%  \begin{docCommand}{@xdblfloat}
% \changes{v1.1a}{1994/10/31}
%     {Macros removed: \cs{@dbflt}, \cs{@xdblfloat}}
% \changes{v1.1g}{1994/12/10}{Macros reinserted temporarily}
%    
%
    \begin{teX}
\def\@dbflt#1{\@ifnextchar[{\@xdblfloat{#1}}{\@xdblfloat{#1}[tp]}}
\def\@xdblfloat#1[#2]{%
  \@xfloat{#1}[#2]\hsize\textwidth\linewidth\textwidth}
    \end{teX}
%  \end{docCommand}
%  \end{docCommand}
%
%
% Moved to ltoutput 93/12/16
    \begin{teX}
%\newcount\c@topnumber
%\newcount\c@dbltopnumber
%\newcount\c@bottomnumber
%\newcount\c@totalnumber
    \end{teX}
%
% An analysis of |\@floatplacement|:
%
% This should be called whenever |\@colht| has been set.
    \begin{teX}
\def\@floatplacement{\global\@topnum\c@topnumber
    % Textpage bit, global:
   \global\@toproom \topfraction\@colht
   \global\@botnum  \c@bottomnumber
   \global\@botroom \bottomfraction\@colht
   \global\@colnum  \c@totalnumber
    % Floatpage bit, local:
   \@fpmin   \floatpagefraction\@colht}
    \end{teX}
%
  \begin{docCommand}{@dblfloatplacement} {}
% \changes{LaTeX2e}{1993/12/05}{Command changed}
% 
%     This should be called only within a group.  Now changed to
%     provide extra checks in |\@addtodblcol|, needed when processing a
%     BANG float.
%    
    \begin{teX}  
\def \@dblfloatplacement {%
    \end{teX}
%    Textpage bit: global, but need not be.
    \begin{teX}  
  \global \@dbltopnum \c@dbltopnumber
  \global \@dbltoproom \dbltopfraction\@colht
    \end{teX}
%   This new bit uses |\@textmin| to locally store the amount of extra
%   room in the column.   
    \begin{teX}
  \@textmin \@colht
  \advance \@textmin -\@dbltoproom
    \end{teX}
%    Floatpage bit: must be local.
    \begin{teX}
  \@fpmin \dblfloatpagefraction\textheight
  \@fptop \@dblfptop
  \@fpsep \@dblfpsep
  \@fpbot \@dblfpbot
}
    \end{teX}
  \end{docCommand}
%
%
% \begin{oldcomments}
\subsection{Marginal Notes}
\label{ltx:marginalnotes}\index{marginpar>latex definitions}

   Marginal notes use the same mechanism as floats to communicate
   with the |\output|  routine.  Marginal notes are distinguished from
   floats by having a negative placement specification.  The command
  \CMDI{\marginpar}[LTEXT]{RTEXT} generates a marginal note in a parbox,
    using LTEXT if it's on the left and RTEXT if it's on the right.
   (Default is RTEXT = LTEXT.)  It uses the following parameters.
%
%   \marginparwidth : Width of marginal notes.
%   \marginparsep   : Distance between marginal note and text.
%        the page layout to determine how to move the marginal
%        note into the margin.   E.g., \@leftmarginskip ==
%        \hskip -\marginparwidth \hskip -\marginparsep .
%   \marginparpush  :  Minimum vertical separation between \marginpar's
%
%  Marginal notes are normally put on the outside of the page
%  if @mparswitch = true, and on the right if @mparswitch = false.
%  The command \reversemarginpar reverses the side where they
%  are put.  \normalmarginpar undoes \reversemarginpar.
%  These commands have no effect for two-column output.
%
%  SURPRISE: if two marginal notes appear on the same line of
%  text, then the second one could appear on the next page, in
%  a funny position.
%
%
%  \marginpar [LTEXT]{RTEXT} ==
%   BEGIN
%     if hmode then \@bsphack
%                   \@floatpenalty := -10002
%              else \@floatpenalty := -10003
%     fi
%     if inner
%       then LaTeX Error: 'Not in outer paragraph mode.'
%            \@floatpenalty := 0
%       else if \@freelist has two elements:
%              then get \@marbox, \@currbox  from \@freelist
%                   \count\@marbox :=G -1
%              else \@floatpenalty := 0
%                   LaTeX Error: 'Too many unprocessed floats'
%                   \@currbox, \@marbox := \@tempboxa    %%use \def
%            fi
%     fi
%     if optional argument
%       then %% \@xmpar ==
%            \@savemarbox\@marbox{LTEXT}
%            \@savemarbox\@currbox{RTEXT}
%       else %% \@ympar ==
%            \@savemarbox\@marbox{RTEXT}
%            \box\@currbox :=G \box\@marbox
%    fi
%    \@xympar 
%   END
%
% \reversemarginpar == BEGIN \@mparbottom   :=G 0
%                            @reversemargin :=G true
%                      END
%
% \normalmarginpar  == BEGIN \@mparbottom   :=G 0
%                            @reversemargin :=G false
%                      END
%
% \end{oldcomments}
%
 \begin{docCommand}{marginpar} {}
    \begin{teX}
\def\marginpar{%
  \ifhmode
    \@bsphack
    \@floatpenalty -\@Mii
  \else
    \@floatpenalty-\@Miii
  \fi
  \ifinner
    \@parmoderr
    \@floatpenalty\z@
  \else
    \@next\@currbox\@freelist{}{}%
    \@next\@marbox\@freelist{\global\count\@marbox\m@ne}%
       {\@floatpenalty\z@
        \@fltovf\def\@currbox{\@tempboxa}\def\@marbox{\@tempboxa}}%
  \fi
  \@ifnextchar [\@xmpar\@ympar}
    \end{teX}
 \end{docCommand}
%
 \begin{docCommand}{@xmpar} {}
    \begin{teX}
\long\def\@xmpar[#1]#2{%
  \@savemarbox\@marbox{#1}%
  \@savemarbox\@currbox{#2}%
  \@xympar}
    \end{teX}
 \end{docCommand}
%
 \begin{docCommand}{@ympar} {}
    \begin{teX}
\long\def\@ympar#1{%
  \@savemarbox\@marbox{#1}%
  \global\setbox\@currbox\copy\@marbox
  \@xympar}
    \end{teX}
 \end{docCommand}
% 
 \begin{docCommand}{@savemarbox} {}
% \changes{v1.0b}{1994/03/12}
%     {(DPC) Extra group for colour}
% \changes{v1.0c}{1994/03/14}
%     {(DPC) Use \cs{color@begingroup}}
% \changes{v1.0d}{1994/04/18}
%     {(DPC) Remove Colour support}
% \changes{v1.1a}{1994/10/31}
%     {(DPC/CAR) Extra box added for colour}
% \changes{v1.1c}{1994/11/05}
%         {Use new \cs{color@hbox} concept.}
% \changes{v1.1f}{1994/11/21}{Changed to \cs{color@vbox} }
% \changes{v1.1f}{1994/11/21}{Use \cs{@setnobreak} etc}
% \changes{v1.1f}{1994/11/21}{Added \cs{@setminipage} etc}
% \changes{v1.1f}{1994/11/21}{Added resetting of size and font}
% \changes{v1.1m}{1995/05/25}{(CAR) Resettings moved to hook}
% \changes{v1.1n}{1996/07/26}{remove unecessary \cs{global} before
%                 \cs{@minipage...}}   
The command saves a margin box. The macros \CMDI{\color@vbox} and \CMDI{\color@endbox} are to
allow the \pkgname{color} to save colors while the box can move. See page 6 of the |color| manual \cite{color}. In
the kernel they are all initially set to \cmd{\relax}.
    \begin{teX}
\long\def \@savemarbox #1#2{%
  \global\setbox #1%
    \color@vbox
      \vtop{%
        \hsize\marginparwidth
        \@parboxrestore 
        \@marginparreset
        #2%
        \@minipagefalse   
        \outer@nobreak
        }%
    \color@endbox
}
    \end{teX}
 \end{docCommand}
% 
  \begin{docCommand}{@marginparreset} {}
% \changes{v1.1f}{1994/11/21}{Macro added}
%
% The rational for allowing these normally global flags to be set
% locally here, via |\@parboxrestore| was stated originally by
% Donald Arsenau and extended by Chris Rowley.
% It is because these flags are only set globally to
% true by section commands, and these should never appear within
% marginals or floats or, indeed, in any group; and they are only ever
% set globally to false when they are definitely true.
%
% If anyone is unhappy with this argument then both flags should be
% treated as in |\set@nobreak|; otherwise this command will be
% redundant. 
% \changes{v1.1p}{1996/10/24}
%     {Added local settings of flags: dangerous!!}
    \begin{teX}
\def \@marginparreset {%
        \reset@font
        \normalsize
%        \let\if@nobreak\iffalse
%        \let\if@noskipsec\iffalse
%        \@setnobreak
        \@setminipage
}
    \end{teX}
  \end{docCommand}
%
 \begin{docCommand}{@xympar} {}
  % \changes{LaTeX2.09}{1992/03/18}
%     {(RmS) added \cs{global}\cs{@ignorefalse}}
% \changes{v1.0b}{1994/03/12}
%     {(DPC) Extra bgroup for colour}
% \changes{1.0c}{1994/03/14}
%     {(DPC) Use \cs{color@begingroup}}
% \changes{v1.1a}{1994/10/31}
%     {(DPC/CAR) Extra box added since needed for floats}
% \changes{v1.1c}{1994/11/05}
%         {Use new \cs{color@hbox} concept.}
% \changes{v1.1f}{1994/11/21}{Changed to \cs{color@vbox} }
%     Setting the box here is done only because the code
%     uses \cs{end@float}; it will be empty and gets discarded. 
% \changes{v1.1o}{1996/08/02}{Remove \cs{global} before \cs{@ignore...}}
    \begin{teX}
\def \@xympar{%
  \ifnum\@floatpenalty <\z@\@cons\@currlist\@marbox\fi
  \setbox\@tempboxa
    \color@vbox
      \vbox \bgroup
  \end@float
  \@ignorefalse
  \@esphack
}
    \end{teX}
 \end{docCommand}
%
 \begin{docCommand}{reversemarginpar} {}
 \begin{docCommand}{normalmarginpar} {}
 The \CMDI{\reversemarginpar} is used to reverse the placement of the marginpar from
 the outer to the inner margins. To reset it use \CMDI{\normalmarginpar}
    \begin{teX}
\def\reversemarginpar{\global\@mparbottom\z@ \@reversemargintrue}
\def\normalmarginpar{\global\@mparbottom\z@ \@reversemarginfalse}
    \end{teX}
 \end{docCommand}
 \end{docCommand}
%
\section{Footnotes}

    \begin{teX}
\message{footnotes,}
    \end{teX}
%
 \subsection{Footnotes}
%
This section of the |float| class handles the administrative macros for setting up footnotes
and how these are inputted by the user.


%   \footnote{NOTE}       : User command to insert a footnote.
%
%   \footnote[NUM]{NOTE}: User command to insert a footnote numbered
%                       NUM, where NUM is a number -- 1, 2,
%                       etc.  For example, if footnotes are numbered
%                       *, **, etc. within pages, then \footnote[2]{...}
%                       produces footnote '**'.  This command does not
%                       step the footnote counter.
%
%   \footnotemark[NUM] : Command to produce just the footnote mark in
%                        the text, but no footnote.  With no argument,
%                        it steps the footnote counter before generating
%                        the mark.
%
%   \footnotetext[NUM]{TEXT} : Command to produce the footnote but
%                              no mark.  \footnote is equivalent to
%                              \footnotemark \footnotetext .
%
%   As in PLAIN, footnotes use \insert\footins, and the following
%   parameters: 
%
%   \footnotesize   : Size-changing command for footnotes.
%
%   \footnotesep    : The height of a strut placed at the beginning of
%                     every footnote.
%   \skip\footins   : Space between main text and footnotes.  The rule
%                     separating footnotes from text occurs in this
%                     space. This space lies above the strut of height
%                     \footnotesep which is at the beginning of the
%                     first footnote.
%   \footnoterule   : Macro to draw the rule separating footnotes from
%                     text. It is executed right after a \vspace of
%                     \skip\footins. It should take zero vertical
%                     space--i.e., it should to a negative skip to
%                     compensate for any positive space it occupies.
%                     (See PLAIN.TEX.)
%
%   \interfootnotelinepenalty : Interline penalty for footnotes.
%
%   \thefootnote : In usual LaTeX style, produces the footnote number.
%                  If footnotes are to be numbered within pages, then
%                  the document style file must include an \@addtoreset
%                  command to cause the footnote counter to be reset
%                  when the page counter is stepped.  This is not a good
%                  idea, though, because the counter will not always be
%                  reset in time to ensure that the first footnote on a
%                  page is footnote number one.
%
%   \@thefnmark : Holds the current footnote's mark--e.g., \dag or '1'
%                 or 'a'. 
%
%   \@mpfnnumber  : A macro that generates the numbers for \footnote
%                  and \footnotemark commands. It == \thefootnote
%                  outside a minipage environment, but can be
%                  changed inside to generate numbers for
%                  \footnote's.
%
%   \@makefnmark : A macro to generate the footnote marker from
%                 \@thefnmark The default definition was
%                 \hbox{$^\@thefnmark$}.
%
%                 This is now replaced by
%                 \textsuperscript{\@thefnmark}
%
%   \@makefntext{NOTE} :
%        Must produce the actual footnote, using \@thefnmark as the mark
%        of the footnote and NOTE as the text.  It is called when
%        effectively  inside a \parbox, with \hsize = \columnwidth.
%          For example, it might be as simple as
%               $^{\@thefnmark}$  NOTE
%

 In a minipage environment, \footnote and \footnotetext are redefined
 so that
 
    (a) they use the counter mpfootnote\\
    (b) the footnotes they produce go at the bottom of the minipage.\\
The switch is accomplished by letting |\@mpfn == footnote| or mpfootnote
 and |\thempfn == \thefootnote| or |\thempfootnote|, and by redefining
| \@footnotetext| to be |\@mpfootnotetext| in the minipage.
 
%
% \footnote{NOTE}  ==
%  BEGIN
%    \stepcounter{\@mpfn}
%    begingroup
%       \protect == \noexpand
%       \@thefnmark :=G eval (\thempfn)
%    endgroup
%    \@footnotemark
%    \@footnotetext{NOTE}
%  END
%
% \footnote[NUM]{NOTE} ==
%  BEGIN
%    begingroup
%       \protect == \noexpand
%       counter \@mpfn :=L NUM
%       \@thefnmark :=G eval (\thempfn)
%    endgroup
%    \@footnotemark
%    \@footnotetext{NOTE}
%  END
%
% \footnotemark      ==
%  BEGIN \stepcounter{footnote}
%        begingroup
%           \protect == \noexpand
%           \@thefnmark :=G eval(\thefootnote)
%        endgroup
%        \@footnotemark
%  END
%
% \footnotemark[NUM] ==
%   BEGIN
%       begingroup
%         footnote counter :=L NUM
%         \protect == \noexpand
%        \@thefnmark :=G eval(\thefootnote)
%       endgroup
%       \@footnotemark
%   END
%
% \@footnotemark ==
%   BEGIN
%    \leavevmode
%    IF hmode THEN \@x@sf := \the\spacefactor FI
%    \@makefnmark          % put number in main text
%    IF hmode THEN \spacefactor := \@x@sf FI
%   END
%
% \footnotetext      ==
%    BEGIN begingroup \protect == \noexpand
%                     \@thefnmark :=G eval (\thempfn)
%          endgroup
%          \@footnotetext
%    END
%
% \footnotetext[NUM] ==
%    BEGIN begingroup  counter \@mpfn :=L NUM
%                      \protect == \noexpand
%                      \@thefnmark :=G eval (\thempfn)
%          endgroup
%          \@footnotetext
%    END
%
% \end{oldcomments}
%
%
%
 \begin{docCommand}{footins}{}
  \LaTeX\ does use the same insert for footnotes as Plain.
    \begin{teX}
\newinsert\footins
    \end{teX}
%
 \LaTeX\ leaves these initializations for the |\footins| insert. The |\count\footins| ensures that
 the characters are represented at the correct magnification.

    \begin{teX}
\skip\footins=\bigskipamount % space added when footnote is present
\count\footins=1000 % footnote magnification factor (1 to 1)
\dimen\footins=8in % maximum footnotes per page
    \end{teX}
 \end{docCommand}
%
%
 \begin{docCommand}{footnoterule}{}
 
 This macro is responsible to typeset the footnote rule. Lamport used the exact
 definition, as that of Knuth’s plaintex.  The definition here is a default definition
 as the standard classes will redefine it, rather than set it up. (See \pageref{book:footnotes}).
In the standard classes for example the width is defined in terms of |\columnwidth|.

    \begin{teX}
\def\footnoterule{\kern-3\p@
  \hrule \@width 2in \kern 2.6\p@} % the \hrule is .4pt high
    \end{teX}
 \end{docCommand}

Next the footnote counter is defined. The counter initialization is then left for the standard
classes to set. Footnotes are also set in arabic numerals as default.

 \begin{docCommand}{thefootnote}{}
    \begin{teX}
\@definecounter{footnote}
\def\thefootnote{\@arabic\c@footnote}
    \end{teX}
 \end{docCommand}

 \begin{docCommand}{thempfootnote}{}
% \changes{v1.1j}{1995/05/18}{Added \cs{itshape}.}
% \changes{v1.1v}{2002/10/01}{Use braces around \cs{itshape} 
%    to keep font change local (pr/3460).}
%    The default display for the footnote counter in minipages is to
%    use italic letters. We use |\itshape| not |\textit| as the latter
%    would add an italic correction.
    \begin{teX}
\@definecounter{mpfootnote}
\def\thempfootnote{{\itshape\@alph\c@mpfootnote}}
    \end{teX}
 \end{docCommand}
%
 \begin{docCommand}{@makefnmark}{}
% \changes{v1.1i}{1995/05/16}{Now use \cs{textsuperscript}.}
% \changes{v1.1j}{1995/05/18}{Added \cs{normalfont}.}
% \changes{v1.1k}{1995/05/20}{Moved \cs{normalfont} to
%                             \cs{textsuperscript}}
% \changes{v1.1k}{1995/05/20}{Moved \cs{normalfont} back
%                        and use \cs{@textsuperscript}}
%    Default definition.
    \begin{teX}
%\def\@makefnmark{\hbox{$^{\@thefnmark}\m@th$}}
\def\@makefnmark{\hbox{\@textsuperscript{\normalfont\@thefnmark}}}
    \end{teX}
 \end{docCommand}
%

 \begin{docCommand}{textsuperscript} {}
This is a surprising macro and if \CMDI{\textsuperscript} deserved to be defined
in a float class is doubtful. Since it is necessary to make it robust it
is defined using \cmd{DeclareRobustCommand}. It’s twin is defined in the
\pkgname{fixltx2e} which contains a number of kernel fixes. (see line 379) in the |fixltx2e| 
documentation. 
%
% \changes{v1.1i}{1995/05/16}{Command added./pr1503}
% \changes{v1.1k}{1995/05/20}{Use \cs{normalfont}.}
% \changes{v1.1l}{1995/05/24}{Use \cs{@textsuperscript}}
%    This command provides superscript characters in the current text
%    font. It's implementation might change!!!
    \begin{teX}
\DeclareRobustCommand*\textsuperscript[1]{%
  \@textsuperscript{\selectfont#1}}
    \end{teX}
  \end{docCommand}

In normal Lamport style, the definition is split into a main macro and an auxiliary
macro \CMDI{\textsuperscript}
%
  \begin{docCommand}{@textsuperscript} {}
% \changes{v1.1l}{1995/05/24}{Command added.}
% \changes{v1.1n}{1995/12/05}{Use \cs{ensuremath} for latex/1984.}
% \changes{v1.1m}{1995/12/07}
%      {Move \cs{m@th} out of the \cs{ensuremath} for latex/1984.}
    This command should not be used directly, but may be used to define
    other commands |\textsuperscript|, |\@makefnmark|. |#1| should
   always start with a font selection command, to activate the font
   size switch. See for example usage at the |\maketitle| in |book.cls|.
   
    \begin{teX}
\def\@textsuperscript#1{%
  {\m@th\ensuremath{^{\mbox{\fontsize\sf@size\z@#1}}}}}
    \end{teX}
  \end{docCommand}
%
The \CMDI{\footnotesep} is normally handled at the class level, but it is defined
here. Again here I am not too sure if it was not best left for a class that would have
defined all the page parameters and dimensions.

 \begin{docCommand}{footnotesep} {}
    \begin{teX}
\newdimen\footnotesep
    \end{teX}
 \end{docCommand}
%
We now come at the actual definition of the \CMDI{footnote}
 \begin{docCommand}{footnote} {}
% \changes{LaTeX2.09}{1991/11/01}
%         {(RmS) Added \cs{let}\cs{protect}\cs{noexpand} in
%          \cs{footnote}, \cs{footnotemark},
%               and \cs{footnotetext}, since \cs{xdef} is used}
% \changes{LaTeX2.09}{1991/11/22}
%         {(RmS) Added \cs{let}\cs{protect}\cs{noexpand} in
%             \cs{@xfootnote}, \cs{@xfootnotemark},
%               and \cs{@xfootnotetext}}
% \changes{LaTeX2.09}{1992/11/26}
%         {(RmS) Changed all to 
%             `def`protect\string{`noexpand`protect`noexpand\string}}
% \changes{v1.1b}{1994/11/26}
%         {(ASAJ) Added \cs{protected@xdef}.}
%
    \begin{teX}
\def\footnote{\@ifnextchar[\@xfootnote{\stepcounter\@mpfn
     \protected@xdef\@thefnmark{\thempfn}%
     \@footnotemark\@footnotetext}}
    \end{teX}
 \end{docCommand}
%
 \begin{docCommand}{@xfootnote} {}
    \begin{teX}
\def\@xfootnote[#1]{%
   \begingroup 
     \csname c@\@mpfn\endcsname #1\relax
     \unrestored@protected@xdef\@thefnmark{\thempfn}%
   \endgroup
   \@footnotemark\@footnotetext}
    \end{teX}
 \end{docCommand}

 \begin{docCommand}{@footnotetext} {}
% \changes{LaTeX2.09}{1991/09/29}
%     {(RmS) added \cs{reset@font}}
% \changes{LaTeX2.09}{1992/11/26}
%     {(RmS) added protection for \cs{edef}}
% \changes{v1.0a}{1994/03/07}
%     {(DPC) Extra group for colour}
% \changes{v1.0c}{1994/03/14}
%     {(DPC) Use \cs{color@begingroup}, add \cs{endgraf}}
% \changes{v1.0d}{1994/04/18}
%     {(DPC) Remove Colour support}
% \changes{v1.0g}{1994/05/13}
%     {(DPC) Add new style colour support: \cs{normalcolor}}
% \changes{v1.0g}{1994/05/13}
%     {(DPC) Use \cs{@finalstrut}}
% \changes{v1.1a}{1994/10/31}
%     {(DPC/CAR) Move colour setting to output routine}
% \changes{v1.1b}{1994/11/04}
%     {(ASAJ) Added \cs{protected@edef}.}
% \changes{v1.1c}{1994/11/05}
%         {Removed \cs{normalcolor} (again)}
% \changes{v1.1t}{1997/11/19}
%         {Missing percent, again}
    \begin{teX}
\long\def\@footnotetext#1{\insert\footins{%
    \reset@font\footnotesize
    \interlinepenalty\interfootnotelinepenalty
    \splittopskip\footnotesep
    \splitmaxdepth \dp\strutbox \floatingpenalty \@MM
    \hsize\columnwidth \@parboxrestore
    \protected@edef\@currentlabel{%
       \csname p@footnote\endcsname\@thefnmark
    }% 
    \color@begingroup
      \@makefntext{%
        \rule\z@\footnotesep\ignorespaces#1\@finalstrut\strutbox}%
    \color@endgroup}}%
    \end{teX}
 \end{docCommand}
%
 \begin{docCommand}{footnotemark} {}
% \changes{v1.1b}{1994/11/04}{Added \cs{protected@xdef} to
    \cs{footnotemark}. 
    \begin{teX}
\def\footnotemark{%
   \@ifnextchar[\@xfootnotemark
     {\stepcounter{footnote}%
      \protected@xdef\@thefnmark{\thefootnote}%
      \@footnotemark}}
    \end{teX}
 \end{docCommand}
%
 \begin{docCommand}{@xfootnotemark} {}
    \begin{teX}
\def\@xfootnotemark[#1]{%
   \begingroup 
      \c@footnote #1\relax
      \unrestored@protected@xdef\@thefnmark{\thefootnote}%
   \endgroup
   \@footnotemark}
    \end{teX}
 \end{docCommand}

 \begin{docCommand}{@footnotemark} {}
% \changes{v1.1h}{1995/05/12}
         {Add \cs{nobreak} to allow hyphenation. latex/1605}
    \begin{teX}
\def\@footnotemark{%
  \leavevmode
  \ifhmode\edef\@x@sf{\the\spacefactor}\nobreak\fi
  \@makefnmark
  \ifhmode\spacefactor\@x@sf\fi
  \relax}
    \end{teX}
 \end{docCommand}
%
 \begin{docCommand}{footnotetext}{}
    \begin{teX}
\def\footnotetext{%
     \@ifnextchar [\@xfootnotenext
       {\protected@xdef\@thefnmark{\thempfn}%
    \@footnotetext}}
    \end{teX}
 \end{docCommand}
%
 \begin{docCommand}{@xfootnotenext}{}
    \begin{teX}
\def\@xfootnotenext[#1]{%
  \begingroup 
     \csname c@\@mpfn\endcsname #1\relax
     \unrestored@protected@xdef\@thefnmark{\thempfn}%
  \endgroup
  \@footnotetext}
    \end{teX}
 \end{docCommand}
%
 \begin{docCommand}{thempfn}{}
 \begin{docCommand}{@mpfn}{}
    \begin{teX}
\def\@mpfn{footnote}
\def\thempfn{\thefootnote}
    \end{teX}
 \end{docCommand}
 \end{docCommand}

This brings us to the end of the float class. Note that the actual page building mechanism and the moving
of floats is handled at the |output| routine level. So it is best if you have read so far to delve straight into the
output routine.

The macros described by this class are redefined by a number of packages and classes. The placement of footnotes in multicolumn texts was discussed extensively by \citeyearpar{mittelbach1990}. The \pkgname{ftnright} also by
Mittelbach \citeyearpar{mittelbach2014} also modifes the kernel commands for double column placement of footnotes. See also the chapters on multicolumn texts as well as the chapter on endnotes and footnotes.


     \chapter{ltoutput.dtx}

 In \autoref{ch:OTR}, we described the mechanics of output routines both
 as found in Plain \tex and in \LaTeXe. This is
 a longer treatise of the subject and includes commentary on the
 actual listing as found in \LaTeXe. The Output Routine (OR) or (OTR) as is sometimes denoted in the literature, is the procedure
by which \LaTeXe\ assembles the material that makes a page by combining
text and floats, adding any other inserts such as footnotes, headers and
footers and then ships out the page to produce a |.dvi| or with some \tex engines to
be translated straight into |.pdf| output. It is a very complex process
as it has to keep a lot of different material in different lists and boxes.

The output routine as defined in the kernwl covers a lot of functionality.

\begin{enumerate}
\item Defines page gemetry parameters.
\item Positions floats
\item Adds headers and footers
\item Adds hooks
\end{enumerate}


 \section{Output Routine}

 \subsection{Floats}


\section{Page Layout Parameters}

\begin{tabular}{lp{6cm}}
   |\topmargin|      & Extra space added to top of page.\\
   |\@twoside|       & boolean.  T if two-sided printing\\
   |\oddsidemargin|  & IF @twoside = T
                       THEN extra space added to left of odd-numbered
                            pages.
                       ELSE extra space added to left of all pages.\\
   |\evensidemargin| & IF @twoside = T
                       THEN extra space added to left of even-numbered
                            pages.\\
   |\headheight|     & height of head\\
   |\headsep|        & separation between head and text\\
   |\footskip|       & distance separation between baseline of last
                     line of text and baseline of foot.
                     Note difference between |\footSKIP| and |\headSEP|.\\

   |\textheight|     & height of text on page, excluding head and foot\\

   |\textwidth|      & width of printing on page\\
   |\columnsep|      & IF @twocolumn = T
                       THEN width of space between columns\\

   |\columnseprule|  & IF @twocolumn = T
                       THEN width of rule between columns (0 if none).\\

   |\columnwidth|    & IF @twocolumn = T
                       THEN |(\textwidth - \columnsep)|/2
                       ELSE |\textwidth|
                     It is set by the |\twocolumn| and
                     |\onecolumn| commands.\\

   |\@textbottom|    & Command executed at bottom of vbox holding text of
                     page (including figures).  The |\raggedbottom|
                     command almost |\let|'s this to |\vfil| (actually sets
                     it to |\vskip \z@| plus.0001fil).
                     Should have depth 0pt.\\

   |\@texttop|       & Command executed at top of vbox holding text of
                     page (including figures).  Used by letter style;
                     can also be used to produce centered pages.
                     Let to |\relax| by |\raggedbottom| and |\flushbottom|.\\
\end{tabular}


   Page layout must initialize |\@colht| and |\@colroom| to |\textheight|.

\section{Page Style Parameters}

\begin{tabular}{p{3.5cm}p{6cm}}
   |\floatsep|       & Space left between floats.\\
   |\textfloatsep|   & Space between last top float or first bottom float
                     and the text.\\
   |\topfigrule|     & Command to place rule (or whatever) between floats
                     at top of page and text.  Executed in inner
                     vertical mode right before the |\textfloatsep| skip
                     separating the floats from the text.  Must occupy
                     zero vertical space.  (See |\footnoterule|.)\\
   |\botfigrule|     & Same as |\topfigrule|, but put after the
                     |\textfloatsep| skip separating text from the
                     floats at bottom of page.\\
   |\intextsep|      & Space left on top and bottom of an in-text float.\\
   |\dblfloatsep|    & Space between double-column floats.\\
   |\dbltextfloatsep| & Space between top double-column floats
                      and text.\\
   |\dblfigrule|     & Similar to |\topfigrule|, but for double-column
                     floats.\\
   |\@fptop|         & Glue to go at top of float column -- must be 0pt +
                     stretch\\
   |\@fpsep|         & Glue to go between floats in a float column.\\
   |\@fpbot|         & Glue to go at bottom of float column
                       -- must be 0pt +
                     stretch\\
   |\@dblfptop|, |\@dblfpsep|, |\@dblfpbot|
                   & Analogous for double-column float page in
                     two-column format.\\

   |@twocolumn|      & Boolean.  T if two columns per page globally.\\


   |\@oddhead|        & IF @twoside = T
                           THEN macro to generate head of odd-numbered
                                pages.
                           ELSE macro to generate head of all pages.\\
   |\@evenhead|      & IF @twoside = T
                           THEN macro to generate head of even-numbered
                                pages.\\
   |\@oddfoot|        & IF @twoside = T
                           THEN macro to generate foot of odd-numbered
                                pages.
                           ELSE macro to generate foot of all pages.\\
   |\@evenfoot|       & IF @twoside = T
                           THEN macro to generate foot of even-numbered
                                pages.\\
   |@specialpage|     & boolean.  T if current page is to have a special
                               format.\\
  |\@specialstyle|   & If its value is  foo then
                     IF @specialpage = T
                       THEN the command |\ps@foo| is executed to
                            temporarily reset the page style parameters
                            before composing the current page.
                            This command should execute only |\def|'s and
                            |\edef|'s, making only local definitions.\\
\end{tabular}



\section{Float placement parameters}

 The following parameters are set by the macro |\@floatplacement|.
 When |\@floatplacement| is called,
 |\@colht| is the height of the page or column being built.  I.e.:

         * For single-column page it equals |\textheight|.\\
         * For double-column page it equals |\textheight| - height
           of double-column floats on page.

 Note that some are set globally and some locally:

\begin{description}

  \item[\cs{@topnum}] = G Maximum number of floats allowed on the top of a
                  column.
  \item [\cs{@toproom}] :=G Maximum amount of top of column devoted to floats--
                  excluding |\textfloatsep| separation below the floats
                  and |\floatsep| separation between them.  For
                  two-column output, should be computed as a function
                  of |\@colht|.
\end{description}


\begin{tabular}{p{3.5cm}p{6cm}}
    |\@botnum|, |\@botroom|
                & Analogous to above.\\
    |\@colnum|  & G Maximum number of floats allowed in a column,
                  including in-text floats.\\
    |\@textmin| & L Minimum amount of text (excluding footnotes) that
                  must appear on a text page.
                  It is used locally in processing double
                  floats.\\
    |\@fpmin|   & L Minimum height of floats in a float column.\\
\end{tabular}


 The macro \cs{@dblfloatplacement} sets the following parameters.

\begin{tabular}{p{3.5cm}p{6cm}}
    |\@dbltopnum|  &G Maximum number of double-column floats allowed at
                     the top of a two-column page.\\
    |\@dbltoproom|  &G Maximum height of double-column floats allowed at
                     top of two-column page.\\
    |\@fpmin|      &L Minimum height of floats in a float column.\\
\end{tabular}

 It should also perform the following local assignments where necessary
 -- i.e., where the new value differs from the old one:

\begin{tabular}{p{3.5cm}p{6cm}}
     |\@fptop|    & L |\@dblfptop|\\
     |\@fpsep|      & L |\@dblfpsep|\\
     |\@fpbot|      &L |\@dblfpbot|\\
\end{tabular}


\section{Output Routine Variables}


\begin{tabular}{p{3.5cm}p{6cm}}
  |\@colht| & The total height of the current column.  In single column
            style, it equals |\textheight|.  In two-column style, it is
            |\textheight| minus the height of the double-column floats
            on the current page.  MUST BE INITIALIZED TO |\textheight|.\\

  |\@colroom| & The height available in the current column for text and
              footnotes.  It equals |\@colht| minus the height of all
              floats committed to the top and bottom of the current
              column.\\

  |\@textfloatsheight| & The total height of in-text floats on the
                       current page.\\

  |\footins| & Footnote insertion number.\\

  |\@maxdepth| & Saved value of TeX's |\maxdepth|.  Must be set
               when any routine sets |\maxdepth|.\\
\end{tabular}



\section{Calling the output routine}

 The output routine is called either by TeX's normal page-breaking
 mechanism, or by a macro putting a penalty \(\le -10000\) in the output
 list.  In the latter case, the penalty indicates why the output
 routine was called, using the following code.

\begin{table}[ht]
\centering
\begin{tabular}{lp{6cm}}
\toprule
   Penalty   & Reason\\
\midrule
   -10000    & \cs{pagebreak}\\
             & \cs{newpage}\\
   -10001    & \cs{clearpage} (\cs{penalty} -10000 \cs{vbox}|{}| \cs{penalty} -10001)\\
   -10002    & float insertion, called from horizontal mode\\
   -10003    & float insertion, called from vertical mode.\\
   -10004    & float insertion.\\
\bottomrule
\end{tabular}
\caption{Penalties when calling the output routine.}
\end{table}
Note that a float or marginpar puts the following sequence in the output
list:
\begin{enumerate} 
  \item a penalty of -10004,
  \item a null |\vbox|
  \item a penalty of -10002 or -10003.
\end{enumerate}

This solves two special problems:

\begin{enumerate}
  \item If the float comes right after a \cs{newpage} or \cs{clearpage},
        then the first penalty is ignored, but the second one
       invokes the output routine.
 \item If there is a split footnote on the page, the second 'page'
       puts out the rest of the footnote.
\end{enumerate}
             

            
\section{Functions used in the output routine}

\begin{macro}{\@outputpage}
 \cs{@outputpage} : Produces an output page with the contents of box
              |\@outputbox| as the text part.
              Also sets |\@colht| :=G |\textheight|.
              The page style is determined as follows.
              \begin{algorithm}[H]
               \Begin{
                \If{\cs{@thispagestyle} = true}{
                   use \cs{thispagestyle} style}{
                   use ordinary page style.}}
              \end{algorithm}
\end{macro}


\begin{macro}{\@trycolumn}
\begin{description}
 \item[\cs{@tryfcolumn}\cs{FLIST}]  Tries to form a float column composed of floats
         from |\FLIST| (if nonempty) with the following parameters:
 
           \begin{tabular}{p{3cm}l}
                |\@colht|  & height of box\\
                |\@fpmin| & minimum height of floats in the box\\
                |\@fpsep|  & interfloat space\\
                |\@fptop | & glue at top of box\\
                |\@fpbot | & glue at bottom of box.\\
          \end{tabular}


  If it succeeds, then it does the following:

         \begin{tabular}{p{3cm}l}
                 |\@outputbox| & L the composed float box.\\
                 |@fcolmade|   & G true\\
                 |\FLIST|          & G |\FLIST| - floats put in box\\
                 |\@freelist|     & G |\@freelist| + floats put in box\\
         \end{tabular}

              If it fails, then:

         \begin{tabular}{ll}
                |@fcolmade| & G false\\
        \end{tabular}
           NOTE: BIT MUST BE A SINGLE TOKEN!
\end{description}
\end{macro}




\begin{macro}{\@makefcolumn}
 |\@makefcolumn \FLIST| is similar to |\@tryfcolumn| except that it
             fails to make a float column only if |\FLIST| is empty.
             Otherwise, it makes a float column containing at least
             the first box in |\FLIST|, disregarding |\@fpmin|.
\end{macro}

\begin{macro}{\@startcolumn}

\begin{description}
\item[ \cs{@startcolumn} ]
       Calls |\@tryfcolumn\@deferlist|.  If |\@tryfcolumn| returns with
       (globally set) @fcolmade = false, then:

\item           Globally sets |\@toplist| and |\@botlist| to floats
                  from |\@deferlist| to go at top and bottom of column,
                  deleting them from |\@deferlist|.  It does
                  this using |\@colht| as the total height, the page
                  style parameters |\@floatsep| and |\@textfloatsep|, and
                  the float placement parameters |\@topnum|, |\@toproom|,
                  |\@botnum|, |\@botroom|, |\@colnum| and |\textfraction|.

\item          Globally sets |\@colroom| to |\@colht| minus the height
                  of the added floats.
\end{description}
\end{macro}




\begin{macro}{\@startdblcolumn }

      Calls |\@tryfcolumn\@dbldeferlist{8}|.  If |\@tryfcolumn| returns
      with (globally set) @fcolmade = false, then:

               * Globally sets |\@dbltoplist| to floats from
                 |\@dbldeferlist| to go at top and bottom of column,
                 deleting them from |\@dbldeferlist|.
                 It does this using |\textheight| as the
                 total height, and the parameters |\@dblfloatsep|, etc.

               * Globally sets |\@colht| to |\textheight| minus the height
                 of the added floats.

\end{macro}

\begin{macro}{\@combinefloats}
 \cs{@combinefloats} Combines the text from box
          \cs{@outputbox} with the floats from \cs{@toplist} and \cs{@botlist},
          putting the new box in \cs{@outputbox}.  It uses \cs{floatsep}
          and \cs{textfloatsep} for the appropriate separations.
          It puts the elements of \cs{TOPLIST} and \cs{BOTLIST} onto
          \cs{@freelist}, and makes those lists null.

\end{macro}

\begin{macro}{\@makecol}
|\@makecol| Makes the contents of |\box255| plus the accumulated
              footnotes, plus the floats in |\@toplist| and |\@botlist|,
              into a single column of height |\@colht| (unless the page
              height has been locally changed), which it puts
              into box |\@outputbox|.  It puts boxes in |\@midlist| back
              onto |\@freelist| and restores |\maxdepth|.
\end{macro}



\begin{macro}{\@opcol}
 \cs{@opcol} Outputs a column whose text is in box \cs{@outputbox}

\begin{algorithm}
 \cs{\@opcol}==\Begin{%    
\eIf{@twocolumn = false}{\cs{@outputpage}\\
  \cs{@colht} :=G \cs{textheight}\\
  \cs{@floatplacement}}{\eIf{@firstcolumn = true}{puts box \cs{@outputbox}
      into \cs{@leftcolumn}\\
      @firstcolumn :=G false.}{puts out the current two-column page\\
      any possible two-column float pages,\\
      determine \cs{@dbltoplist} for the next page.}}
}
\end{algorithm}
\end{macro}


\section{User commands  that call  affect the output routine}

\begin{macro}{\newpage}
 \newpage == BEGIN \par\vfil\penalty -10000 END
\end{macro}

\begin{macro}{\clearpage}
\begin{verbatim}
              == BEGIN \newpage
                     \write -1{}    % Part of hack to make sure no
                     \vbox{}        % \write's get lost.
                     \penalty -10001
               END
\end{verbatim}
\begin{verbatim}
 \cleardoublepage == BEGIN \clearpage
                           if @twoside = true and c@page is even
                             then \hbox{} \newpage fi
                     END

  
 \twocolumn[BOX] : starts a new page, changing to twocolumn setting
     and puts BOX in a parbox of width \textwidth across the top.
     Useful for full-width titles for double-column pages.
     SURPRISE: The stretch from \@dbltextfloatsep will be inserted
               between the BOX and the top of the two columns.
\end{verbatim}



\section{Float-handling mechanisms}

 The float environment obtains an insertion number B from the
 |\@freelist| (see below for a description of list manipulation), puts
 the float into box B and sets |\count| B to a FLOAT SPECIFIER.  For
 a normal (not double-column) float, it then causes a page break
 in one of the following two ways:

   - In outer hmode: |\vadjust{\penalty -10002}|
   - In vmode :      |\penalty -10003.|

 For a double-column float, it puts B onto the |\@dbldeferlist|.
 The float specifier has two components:

    * A PLACEMENT SPECIFICATION, describing where the float may
      be placed.

    * A TYPE, which is a power of two--e.g., figures might be
      type 1 floats, tables type 2 floats, programs type 4 floats, etc.
 The float specifier is encoded as follows, where bit 0 is the least
 significant bit.
\medskip


\begin{tabular}{ll}
\toprule
  Bit    & Meaning\\
\midrule
   0     & 1 if the float may go where it appears in the text.\\
   1     & 1 if the float may go on the top of a page.\\
   2     & 1 if the float may go on the bottom of a page.\\
   3     & 1 if the float may go on a float page.\\
   4     & 1 unless the PLACEMENT incluses a !\\
   5     &  if a type 1 float\\
   6     & 1 if a type 2 float
          etc.\\
\bottomrule
\end{tabular}
\medskip

A negative float specifier is used to indicate a marginal note.



\section{Macros and data structures for processing floats}

  A \textit{float list} consisting of the floats in boxes |\boxa ... \boxN| has
  the form:

  \begin{verbatim}
         \@elt \boxa ... \@elt \boxN
  \end{verbatim}

  where  |\boxI| is defined by:

  \begin{verbatim}
         \newinsert\boxI
  \end{verbatim}

  Normally, |\@elt| is |\let| to |\relax|.  A test can be performed on the
  entire float list by locally |\def|'ing |\@elt| appropriately and
  executing the list.

  This is a lot more efficient than looping through the list.

  The following macros are used for manipulating float lists.

  \begin{verbatim}
  \@next \CS \LIST {NONEMPTY}{EMPTY} ==  %% NOTE: ASSUME \@elt = \relax
    BEGIN  assume that \LIST == \@elt \B1 ... \@elt \Bn
           if n = 0
             then  EMPTY
             else  \CS    :=L \B1
                   \LIST  :=G \@elt \B2 ... \@elt \Bn
                   NONEMPTY
           fi
    END
  \end{verbatim}



\begin{verbatim}
%
%
%  \@bitor\NUM\LIST : Globally sets switch @test to the disjunction for
%         all I of bit  log2 \NUM of the float specifiers of all the
%         floats in \LIST.

%         I.e., @test is set to true iff there is at least one
%         float in \LIST having bit  log2 \NUM  of its float specifier
%         equal to 1.
%
%  Note: log2 [(\count I)/32] is the bit number corresponding to the
%  type of float I.  To see if there is any float in \LIST having
%  the same type as float I, you run \@bitor with
%
%    \NUM = [(\count I)/32] * 32.
%
% \@bitor\NUM\LIST ==
%   BEGIN
%      @test :=G false
%      { \@elt \CTR ==  if \NUM <> 0 then
%                          if \count\CTR / \NUM is odd
%                             then  @test := true       fi fi
%        \LIST
%      }
%   END
%
%
% \@cons\LIST\NUM : Globally sets \LIST := \LIST * \@elt \NUM
%
% \@cons\LIST\NUM ==
%   BEGIN {  \@elt == \relax
%            \LIST :=G \LIST \@elt \NUM
%         }

\end{verbatim}



\section{Box lists for float-placement algorithms}


\begin{tabular}{p{3cm}p{6cm}}
    |\@freelist|     & List of empty boxes for placing new floats.\\
    |\@toplist|      & List of floats to go at top of current column.\\
    |\@midlist|      & List of floats in middle of current column.\\
    |\@botlist|      & List of floats to go at bottom of current column.\\
    |\@deferlist|    & List of floats to go after current column.\\
    |\@dbltoplist|   & List of double-col. floats to go at top of current
                     page.\\
    |\@dbldeferlist| & List of double-column floats to go on subsequent
                     pages.\\
\end{tabular}



\section{Float-Placement algorithms}


\begin{macro}{\@addtobot}  Tries to put insert |\@currbox| on |\@botlist|.
                     Called only when:

                  * |\ht BOX < \@colroom|\\
                  * type of |\@currbox| not on |\@deferlist|\\
                  * |\@colnum > 0|\\
                  * @insert = false\\
\end{macro}



               If it succeeds, then:
\begin{trivlist}
                  \item  sets @insert true\\
                  \item  decrements |\@botroom| by |\ht| BOX\\
                  \item  decrements |\@botnum| and |\@colnum| by 1\\
                  \item decrements |\@colroom| by |\ht| BOX + either |\floatsep|
                    or |\textfloatsep|, as appropriate.\\
                 \item sets |\maxdepth| to 0pt\\
\end{trivlist}



\begin{macro}{\@addtotoporbot}

 Tries to put insert |\@currbox| on |\@toplist| or
                    |\@botlist|.

                    Called only under same conditions as |\@addtobot|.

                    If it succeeds, then:
                       * sets @insert true
                       * decrements |\@toproom| or |\@botroom| by |\ht| BOX
                       * decrements |\@colnum| and either |\@topnum| or
                         |\@botnum| by 1
                       * decrements |\@colroom| by |\ht| BOX + |\floatsep|
                         or |\textfloatsep|, as appropriate.
\end{macro}

\begin{macro}{\@addtocurcol}  Tries to add |\@currbox| to current column, setting
                 @insert true if it succeeds, false otherwise.
                 It will add |\@currbox| to top only if bit 0 of
                |\count \@currbox| is 0, and to the bottom only if
                 bit 0 = 0 or an earlier float of the same type is
                 put on the bottom.

                 If the float is put in the text, then
                 |\penalty\interlinepenalty| is put
                 right after the float, before the following |\vskip|,
                 and 
                     
                         |\outputpenalty :=L 0.|
\end{macro}

\begin{macro}{\@addtonextcol} Tries to add |\@currbox| to the next column, setting
                  @insert true if it succeeds, false otherwise.
\end{macro}

\begin{macro}{\@addtodblcol} Tries to add |\@currbox| to the next double-column page,
                 adding it to |\@dbltoplist| if it succeeds and
                 |\@dbldeferlist| if it fails.
\end{macro}


\begin{algorithm}
  \cs{@addmarginpar} ==\\
   \Begin{
     \eIf{\cs{@currlist} nonempty}{
        remove \cs{@marbox} from \cs{@currlist}\\
        add \cs{@marbox} and \cs{@currbox} to \cs{@freelist}\\
         NOTE: \cs{@currbox} = left box}{
         LaTeX error: ?  \\
     }
     \cs{@tempcnta} := 1\\     %% 1 = right, -1 = left
     \eIf{@twocolumn = true}{
       then if @firstcolumn = true
              then \cs{@tempcnta} := -1
            fi}{
            \eI{@mparswitch = true}{
               \If{count0 odd}{}{
                   \cs{@tempcnta} := -1
               }
            }
            \If{@reversemargin = true}{
               \cs{@tempcnta} := -\cs{@tempcnta}
            }
     }

     \If{\cs{@tempcnta} < 0}{\cs{box}\cs{@marbox} :=G \cs{box}\cs{@currbox}}
     
     \cs{@tempdima}   :=L maximum(\cs{@mparbottom} - \cs{@pageht}
                                           + ht of \cs{@marbox}, 0)\\

     \If{\cs{@tempdima} > 0}{LaTeX warning: 'marginpar moved'}

     \cs{@mparbottom} :=G \cs{@pageht} + \cs{@tempdima} + depth of \@cs{marbox}
                          + \cs{marginparpush}\\

     \cs{@tempdima}   :=L \cs{@tempdima} - ht of \@cs{marbox}\\

     \cs{box}\cs{@marbox} :=G \cs{box}\cs{@currbox}\\
                                \cs{vbox}\{ \cs{vskip}\cs{@tempdima}\\
                                        \cs{box}\cs{@marbox}\\
                                       \}\\
     height of \cs{@marbox} :=G depth of \cs{@marbox} :=G 0\\
     \cs{kern} -\cs{@pagedp}\\
     \cs{nointerlineskip}\\
     
     hbox\{\eIf{@tempcnta > 0}{hskip columnwidth\\
                              hskip marginparsep}{
                             hskip -marginparsep\\
                             hskip -marginparwidth}
             \cs{box}\cs{@marbox}\cs{hss}
          \}\\
     \cs{nobreak}\\
     \cs{nointerlineskip}\\
     \cs{hbox}\{\cs{vrule} height=0 width=0 depth=\cs{@pagedp}\}
  }
\end{algorithm}


   Floats and marginpars add a lot of dead cycles.
    \begin{teX}
\maxdeadcycles = 100
    \end{teX}

    \begin{teX}
\let\@elt\relax
    \end{teX}

    \begin{teX}
\def\@next#1#2#3#4{\ifx#2\@empty #4\else
   \expandafter\@xnext #2\@@#1#2#3\fi}
    \end{teX}

    \begin{teX}
\def\@xnext \@elt #1#2\@@#3#4{\def#3{#1}\gdef#4{#2}}
    \end{teX}


%    \begin{teX}
\def\@testfalse{\global\let\if@test\iffalse}
\def\@testtrue {\global\let\if@test\iftrue}
\@testfalse
%    \end{teX}
%
% \changes{v1.1v}{1996/07/26}{remove \cs{global} before \cs{@test...}}
%    \begin{teX}
\def\@bitor#1#2{\@testfalse {\let\@elt\@xbitor
   \@tempcnta #1\relax #2}}
%    \end{teX}
%    RmS 91/11/22: Added test for |\count#1 = 0|.
%                  Suggested by Chris Rowley.
%
%
% \changes{v1.1v}{1996/07/26}{remove \cs{global} before \cs{@test...}}
%    \begin{teX}
\def\@xbitor #1{\@tempcntb \count#1
   \ifnum \@tempcnta =\z@
   \else
     \divide\@tempcntb\@tempcnta
     \ifodd\@tempcntb \@testtrue\fi
   \fi}
%    \end{teX}
%
\section{Definition of Float Boxes}

    \begin{teX}
\newinsert\bx@A
\newinsert\bx@B
\newinsert\bx@C
\newinsert\bx@D
\newinsert\bx@E
\newinsert\bx@F
\newinsert\bx@G
\newinsert\bx@H
\newinsert\bx@I
\newinsert\bx@J
\newinsert\bx@K
\newinsert\bx@L
\newinsert\bx@M
\newinsert\bx@N
\newinsert\bx@O
\newinsert\bx@P
\newinsert\bx@Q
\newinsert\bx@R
    \end{teX}

    \begin{teX}
\gdef\@freelist{\@elt\bx@A\@elt\bx@B\@elt\bx@C\@elt\bx@D\@elt\bx@E
               \@elt\bx@F\@elt\bx@G\@elt\bx@H\@elt\bx@I\@elt\bx@J
                \@elt\bx@K\@elt\bx@L\@elt\bx@M\@elt\bx@N
                \@elt\bx@O\@elt\bx@P\@elt\bx@Q\@elt\bx@R}
    \end{teX}

    \begin{teX}
\gdef\@toplist{}
\gdef\@botlist{}
\gdef\@midlist{}
\gdef\@currlist{}
\gdef\@deferlist{}
\gdef\@dbltoplist{}
\gdef\@dbldeferlist{}
    \end{teX}

\section{Page layout parameters}
    \begin{teX}
\newdimen\topmargin
\newdimen\oddsidemargin
\newdimen\evensidemargin
\let\@themargin=\oddsidemargin
\newdimen\headheight
\newdimen\headsep
\newdimen\footskip
\newdimen\textheight
\newdimen\textwidth
\newdimen\columnwidth
\newdimen\columnsep
\newdimen\columnseprule
\newdimen\marginparwidth
\newdimen\marginparsep
\newdimen\marginparpush
    \end{teX}




 \begin{macro}{\AtBeginDvi}

    We use a box register in which to put
    stuff that must appear before anything else in the
    |.dvi| file.

    The stuff in the box should not add any typeset material to the
    page when it is unboxed.
    \begin{teX}
\newbox\@begindvibox
\def \AtBeginDvi #1{%
  \global \setbox \@begindvibox
    \vbox{\unvbox \@begindvibox #1}%
}                             
    \end{teX}
 \end{macro}
 \end{macro}
  
  \begin{macro}{\@maxdepth}
    This is not the right place to set this; it needs to be set in a
    class/style file when |\maxdepth| is set.

    Also, many settings to |\maxdepth| should be to |\@maxdepth|,
    probably?     

    \begin{teX}  
\newdimen\@maxdepth
\@maxdepth = \maxdepth
    \end{teX}
  \end{macro}
 \begin{macro}{\paperheight}
 \begin{macro}{\paperwidth}
    New |\paper|\ldots\ registers.
    \begin{teX}
\newdimen\paperheight
\newdimen\paperwidth
    \end{teX}
 \end{macro}
 \end{macro}


 \begin{macro}{\if@insert}
 \begin{macro}{\if@fcolmade}
 \begin{macro}{\if@specialpage}
 \begin{macro}{\if@firstcolumn}
 \begin{macro}{\if@twocolumn}
 \begin{macro}{\if@twoside}
 \begin{macro}{\if@reversemarginpar}
 \begin{macro}{\if@mparswitch}
 \begin{macro}{\col@number}
    Local switches first:
    \begin{teX}
\newif \if@insert
    \end{teX}
    These should definitely be global:
    \begin{teX}
\newif \if@fcolmade
\newif \if@specialpage \@specialpagefalse
    \end{teX}
    These should be global but are not always set globally in other
    files. 
    \begin{teX}
\newif \if@firstcolumn \@firstcolumntrue
\newif \if@twocolumn   \@twocolumnfalse
    \end{teX}
    Not sure about these: two questions.
    Should things which must apply to a whole doument be local or
    global (they probably should be `preamble only' commands)?
    Are these three such things?
    \begin{teX}
\newif \if@twoside     \@twosidefalse
\newif \if@reversemargin \@reversemarginfalse
\newif \if@mparswitch  \@mparswitchfalse
    \end{teX}
    This counter has been imported from `multicol'.
    \begin{teX}
\newcount \col@number
\col@number \@ne
    \end{teX}
 \end{macro}
 \end{macro}
 \end{macro}
 \end{macro}
 \end{macro}
 \end{macro}
 \end{macro}
 \end{macro}
 \end{macro}


% INTERNAL REGISTERS
%
%    \begin{teX}
\newcount\@topnum
\newdimen\@toproom
\newcount\@dbltopnum
\newdimen\@dbltoproom
\newcount\@botnum
\newdimen\@botroom
\newcount\@colnum
\newdimen\@textmin
\newdimen\@fpmin
\newdimen\@colht
\newdimen\@colroom
\newdimen\@pageht
\newdimen\@pagedp
\newdimen\@mparbottom \@mparbottom\z@
\newcount\@currtype
\newbox\@outputbox
\newbox\@leftcolumn
\newbox\@holdpg
%    \end{teX}
%
%    \begin{teX}
\def\@thehead{\@oddhead} % initialization
\def\@thefoot{\@oddfoot}
%    \end{teX}


  \begin{macro}{\clearpage}
 The tests at the beginning are an experimental attempt to avoid a
 completely empty page after a |\twocolumn[...]|.  This prevents the
 text from the argument vanishing into a float box, never to be seen
 again.  We hope that it does not produce wrong formatting in other
 cases.
    \begin{teX}
\def\clearpage{%
  \ifvmode
    \ifnum \@dbltopnum =\m@ne
      \ifdim \pagetotal <\topskip
        \hbox{}%
      \fi
    \fi
  \fi
  \newpage
  \write\m@ne{}%
  \vbox{}%
  \penalty -\@Mi
}
    \end{teX}
 \end{macro}

  \begin{macro}{\cleardoublepage}
  
    \begin{teX}
\def\cleardoublepage{\clearpage\if@twoside \ifodd\c@page\else
    \hbox{}\newpage\if@twocolumn\hbox{}\newpage\fi\fi\fi}
    \end{teX}
 \end{macro}

  \begin{macro}{\onecolumn}
    \begin{teX}
\def\onecolumn{%
  \clearpage
  \global\columnwidth\textwidth
  \global\hsize\columnwidth
  \global\linewidth\columnwidth
  \global\@twocolumnfalse
  \col@number \@ne
  \@floatplacement}
    \end{teX}
 \end{macro}

  \begin{macro}{\newpage}
    The two checks at the beginning ensure that an item label or
    run-in section title immediately before a |\newpage| get printed
    on the correct page, the one before the page break.

    All three tests are largely to make error processing more robust;
    that is why they all reset the flags explicitly, even when it
    would appear that this would be done by a |\leavevmode|. 
    \begin{teX}
\def \newpage {%
  \if@noskipsec 
    \ifx \@nodocument\relax
      \leavevmode
      \global \@noskipsecfalse 
    \fi
  \fi
  \if@inlabel
    \leavevmode
    \global \@inlabelfalse 
  \fi
  \if@nobreak \@nobreakfalse \everypar{}\fi
  \par
  \vfil
  \penalty -\@M}
    \end{teX}
  \end{macro}
  
  \begin{macro}{\@emptycol}
    It may be better to use an invisible rule rather than an empty
    box here.  
    \begin{teX}
\def \@emptycol {\vbox{}\penalty -\@M}
    \end{teX}
  \end{macro}

  \begin{macro}{\twocolumn}
  \begin{macro}{\@topnewpage}
    There are several bug fixes to the two-column stuff here.
    \begin{teX}
\def \twocolumn {%
  \clearpage
  \global\columnwidth\textwidth
  \global\advance\columnwidth-\columnsep
  \global\divide\columnwidth\tw@
  \global\hsize\columnwidth
  \global\linewidth\columnwidth
  \global\@twocolumntrue
  \global\@firstcolumntrue
  \col@number \tw@
    \end{teX}
    There is no reason to put a |\@dblfloatplacement| here since
    |\@topnewpage| ignores these settings.
    The |\@floatplacement| is needed in case this comes after some
    changes.

    \begin{teX}
  \@ifnextchar [\@topnewpage\@floatplacement
}
    \end{teX}
    
    Note that here, getting a box from the freelist can assume
    success since this comes just after a |\clearpage|.
    \begin{teX}
\long\def \@topnewpage [#1]{%
  \@nodocument
  \@next\@currbox\@freelist{}{}%
  \global \setbox\@currbox
    \color@vbox 
      \normalcolor
      \vbox {%
        \hsize\textwidth
        \@parboxrestore
        \col@number \@ne
        #1%
        \vskip -\dbltextfloatsep
             }%
    \color@endbox
    \end{teX}
    Added size test and warning message; perhaps we should use
    an error message.

%    \begin{teX}
  \ifdim \ht\@currbox>\textheight
    \ht\@currbox \textheight
  \fi
%    \end{teX}
%
%    This next line is not essential but it is more robust to make this
%    value non-zero, in case of weird errors.
%
%    This next bit is what is needed from |\@addtodblcol|, plus some
%    extra checks for error trapping.
%    \begin{teX}
  \global \count\@currbox \tw@
  \@tempdima -\ht\@currbox
  \advance \@tempdima -\dbltextfloatsep
  \global \advance \@colht \@tempdima
  \ifx \@dbltoplist \@empty
  \else
    \@latexerr{Float(s) lost}\@ehb
    \let \@dbltoplist \@empty
  \fi
  \@cons \@dbltoplist \@currbox
%    \end{teX}
%    This setting of |\@dbltopnum| is used only to change the
%    typesetting in\\ |\@combinedblfloats|.
%    \begin{teX}
  \global \@dbltopnum \m@ne
%<*trace>
    \tr@ce{dbltopnum set to -1 (= \the \@dbltopnum) (topnewpage)}%
%</trace>
%    \end{teX}

    At points such as this we need to check that there is still a
    minimal amount of room left on the page; this uses an arbitrary
    small value at present; but note that this value is larger than
    that used when checking that page is too full of normal floats.
   
    If there is little room left we just force a page-break, OK?
    This involves producing two empty columns.  The second empty
    column may be produced by |\output|, in which case an extra,
    misleading, warning will be generated, OK?  (This happens only
    when there is too little room left on the page for any float.)
    Otherwise (\ie if the size is such that it is allowed as a normal
    float) the extra |\@emptycol| will be invoked in the second
    column by the conditional code guarded by the |\if@firstcolumn|
    test.
    
    I now think that the cut-off point here should be |3\baselineskip|,
    but we make it a bit less so that 3 lines of text will be
    allowed, OK?

    Since this happens only when there is nothing on the page but the
    `top-box', the empty box should not cause any problem other than
    some overfull box messages, which is not entirely misleading.

    Here we need two page-ends since both columns need to be empty.

    \begin{teX}
  \ifdim \@colht<2.5\baselineskip
    \@latex@warning@no@line {Optional argument of \noexpand\twocolumn
                too tall on page \thepage}%
    \@emptycol
    \if@firstcolumn
    \else
      \@emptycol
    \fi
  \else
    \global \vsize \@colht
    \global \@colroom \@colht
    \@floatplacement
  \fi
}
    \end{teX}
  \end{macro}
  \end{macro}


  \begin{macro}{\output}
    This needs some small adjustments.  We cannot
    guarantee that the float mechanism will interact correctly with
    this stuff, but that mechanism does not always work properly
    with footnotes already.

    RmS 91/09/29:

    added reset of |\par| to the output routine.
    This avoids problems when the output routine is
    called within a list where |\par| may be a no-op.

    \begin{teX}
\output {%
  \let \par \@@par
  \ifnum \outputpenalty<-\@M
    \@specialoutput
  \else
    \@makecol
    \@opcol
    \@startcolumn
    \@whilesw \if@fcolmade \fi
      {\@opcol\@startcolumn}%
  \fi
  \ifnum \outputpenalty>-\@Miv
    \ifdim \@colroom<1.5\baselineskip
      \ifdim \@colroom<\textheight  
        \@latex@warning@no@line {Text page \thepage\space
                               contains only floats}%
        \@emptycol
      \else
        \global \vsize \@colroom
      \fi
    \else
      \global \vsize \@colroom
    \fi
  \else
    \global \vsize \maxdimen
  \fi
}
    \end{teX}
\end{macro}


% \begin{oldcomments}

% CHANGES TO \@specialoutput:
% * \penalty\z@ changed to \penalty\interlinepenalty so \samepage
%   works properly with figure and table environments.
%   (Changed 23 Oct 86)
%
% * Definition of \@specialoutput changed 26 Feb 88 so \@pageht and
%   \@pagedp aren't changed for a marginal note.
%   (Change suggested by Chris Rowley.)
% \end{oldcomments}
%
%    \begin{teX}
\gdef\@specialoutput{%
   \ifnum \outputpenalty>-\@Mii
     \@doclearpage
   \else
     \ifnum \outputpenalty<-\@Miii
       \ifnum \outputpenalty<-\@MM \deadcycles \z@ \fi
       \global \setbox\@holdpg \vbox {\unvbox\@cclv}%
     \else
%    \end{teX}
%
%    Note that |\boxmaxdepth| should not be set here since we wish to
%    record the natural depth of the holdpg box.
%    
%    This is changed so as to not lose anything, such as writes
%    and marks, which may get into box 255 and should be returned to
%    the list.  This should only happen when the first penalty in the
%    mechanism is discarded and therefore |\@holdpg| should always be
%    void in this case.  This can happen because a penalty is
%    discarded whenever there is no box on the list.
%
%    It was just: |\setbox\@tempboxa \box \@cclv|.
%    
%    The last box which is removed is the box put there by the
%    double-penalty mechanism.  The |\unskip| then removes the
%    |\topskip| which is put there since the box is the first on the
%    page.
%    \begin{teX}
       \global \setbox\@holdpg \vbox{%
                      \unvbox\@holdpg
                      \unvbox\@cclv
%    \end{teX}
%    We must now remove the box added by the float mechanism and the
%    |\topskip| glue therefore added above it by \TeX.
%    \begin{teX}
                      \setbox\@tempboxa \lastbox
                      \unskip
                                     }%
%    \end{teX}
%    These two are needed as separate dimensions only by
%    |\@addmarginpar|; for other purposes we put the whole size into
%    |\@pageht| (see below).
%    \begin{teX}
       \@pagedp \dp\@holdpg
       \@pageht \ht\@holdpg
       \unvbox \@holdpg
       \@next\@currbox\@currlist{%
         \ifnum \count\@currbox>\z@
%    \end{teX}
%    Putting the whole size into |\@pageht| (see above).
%    \begin{teX}
           \advance \@pageht \@pagedp
           \ifvoid\footins \else
             \advance \@pageht \ht\footins
             \advance \@pageht \skip\footins
             \advance \@pageht \dp\footins
           \fi
           \ifvbox \@kludgeins
%    \end{teX}
%    We want to make the adjustment due to this insert only if the
%    non-star form is used.  The *-form will probably not work with
%    floats, but maybe it still could make some adjustment here even
%    so?
%    \begin{teX}
             \ifdim \wd\@kludgeins=\z@
               \advance \@pageht \ht\@kludgeins
             \fi
           \fi
%    \end{teX}
%    This version puts the inserts back just before the additional
%    material; it could be moved earlier, before unboxing the
%    page-so-far.  Neither is guaranteed not to put things on the wrong
%    page.  This version is similar to the original version.
%    \begin{teX}
           \@reinserts
           \@addtocurcol
         \else
           \@reinserts
           \@addmarginpar
         \fi
         }\@latexbug
%    \end{teX}
%    A 2e change: use |\addpenalty| instead of |\penalty| here.  Some 
%    penalty is needed to create a potential break-point immediately
%    after the reinserts (or the marginal).  Otherwise there can be no
%    possibility to break here and this can cause the reinserts or the
%    marginal to appear on the next page (which is often incorrect).
%    However, if the nobreak flag is true, a |\nobreak| must be
%    correct.
%    \begin{teX}
       \ifnum \outputpenalty<\z@
         \if@nobreak
           \nobreak
         \else
           \addpenalty \interlinepenalty
         \fi
       \fi
     \fi
   \fi
}
%</2ekernel|def1|autoload|fltrace>
%    \end{teX}
%  \end{macro}
%  \end{macro}
%
  \begin{macro}{\@doclearpage}
    This is a very much an emergency action, just dumping everything:
    footnotes first then floats.  A more sophisticated version is
    needed; but even more urgent is a bug-free version (see, for
    example, pr/3528).

    Also, it puts any left-over non-boxes (writes, specials, etc.) back
    after any float pages created: this is a very bad bug since,
    for example, a kludge insert will be in quite the wrong place
    and, worse, be irremovable and uncancelable.
    
    \begin{teX}
\def \@doclearpage {%
     \ifvoid\footins
    \end{teX}
%    We empty any left over kludge insert box here; this is a temporary fix.
%    It should perhaps be applied to one page of cleared floats, but
%    who cares?  The whole of this stuff needs completely redoing for
%    many such reasons.
%    \begin{teX}
       \ifvbox\@kludgeins
         {\setbox \@tempboxa \box \@kludgeins}%
       \fi 
       \setbox\@tempboxa\vsplit\@cclv to\z@ \unvbox\@tempboxa
       \setbox\@tempboxa\box\@cclv
       \xdef\@deferlist{\@toplist\@botlist\@deferlist}%
       \global \let \@toplist \@empty
       \global \let \@botlist \@empty
       \global \@colroom \@colht
       \ifx \@currlist\@empty
       \else
          \@latexerr{Float(s) lost}\@ehb
          \global \let \@currlist \@empty
       \fi
       \@makefcolumn\@deferlist
       \@whilesw\if@fcolmade \fi{\@opcol\@makefcolumn\@deferlist}%
       \if@twocolumn
         \if@firstcolumn
           \xdef\@dbldeferlist{\@dbltoplist\@dbldeferlist}%
           \global \let \@dbltoplist \@empty
           \global \@colht \textheight
           \begingroup
              \@dblfloatplacement
              \@makefcolumn\@dbldeferlist
              \@whilesw\if@fcolmade \fi{\@outputpage
                                        \@makefcolumn\@dbldeferlist}%
           \endgroup
         \else
           \vbox{}\clearpage
         \fi
       \fi
     \else
       \setbox\@cclv\vbox{\box\@cclv\vfil}%
       \@makecol\@opcol
       \clearpage
     \fi
}
    \end{teX}
 \end{macro}
%
%  \begin{macro}{\@opcol}
%
%    \begin{teX}
\def \@opcol {%
  \if@twocolumn
    \@outputdblcol
  \else
    \@outputpage
  \fi
%    \end{teX}
%    These do not need to be done every time |\@opcol| is used: they
%    should be grouped together since they all need to be done at the
%    end of the non-special output routine, or at the end of a clearpage
%    one.
%    \begin{teX}
  \global \@mparbottom \z@ \global \@textfloatsheight \z@
  \@floatplacement
}
%    \end{teX}
%  \end{macro}
%
%
%  \begin{macro}{\@makecol}
% \changes{v0.1c}{1993/11/23}{Command changed}
% \changes{v1.0b}{1993/11/29}{\cs{@makespecialcolbox} added}
%    We must rewrite this macro to alllow for variations in page-makeup
%    required by changes in page-length.
%     
%    This uses a different macro if a special-length column is being
%    produced.
%
%    \begin{teX}
%<*2ekernel|def1|autoload>
\gdef \@makecol {%
   \ifvoid\footins
     \setbox\@outputbox \box\@cclv
   \else
     \setbox\@outputbox \vbox {%
%    \end{teX}
%    This |\boxmaxdepth| setting is to ensure that  deep footnotes
%    do not overwrite the footer (on account of the negative skip
%    added later): it should use |\@maxdepth| otherwise the change is
%    pointless when there are footnotes.
% \task{CAR}{Investigate providing an option to put the footnotes
%    below the bottom floats.}
%
%    But see also its use when combining floats.
% \changes{v1.0l}{1994/03/15}{\cs{maxdepth} changed to \cs{@maxdepth}}
%    \begin{teX}
       \boxmaxdepth \@maxdepth                   
       \unvbox \@cclv
       \vskip \skip\footins
       \color@begingroup
         \normalcolor
         \footnoterule
         \unvbox \footins
       \color@endgroup
       }%
   \fi
%    \end{teX}
%    The h floats have now been finally committed to this page so we
%    can reset their list.  The top and bottom floats are then added
%    to the page.
%
%    \begin{teX}
   \let\@elt\relax
   \xdef\@freelist{\@freelist\@midlist}%
   \global \let \@midlist \@empty
   \@combinefloats
%    \end{teX}
%    The variations start here in case |\enlargethispage| has
%    been used.
%    \begin{teX}
   \ifvbox\@kludgeins
     \@makespecialcolbox
   \else
%    \end{teX}
%    This extra reboxing is only needed to add the
%    |\@texttop| and |\@textbotttom| but this could be done earlier,
%    when the floats are added.
%    
%    The |\boxmaxdepth| resetting here will have no effect unless
%    |\@textbottom| ends with a box or rule.  So is this (or possibly
%    |\@maxdepth|) the correct value?
%
%    The |\vskip -\dimen@|
%    ensures that the visible depth of the box does not
%    affect the placement of anything on the page.
%    Thus very deep pages will overprint the footer; but these should
%    have been prevented by suitable settings of the maxdepths at
%    appropriate times.
%    
%    If |\@textbottom| ends with a box or rule of non-zero depth
%    then this skip adjustemnt should be done again after it.
%    
%    I think that the final boxing of the main text page could have a
%    common ending which may make it simpler to see what is going on.
%    
%    This needs further investigation, especially in the `special
%    case'.
%
%    Also, the |\boxmaxdepth| setting here affects what happens wthin
%    |\@texttop| and |\@textbottom|, should it?  Is it needed at all? 
%    
%    \begin{teX}
     \setbox\@outputbox \vbox to\@colht {%
       \@texttop
       \dimen@ \dp\@outputbox
       \unvbox \@outputbox
       \vskip -\dimen@
       \@textbottom
       }%
   \fi
   \global \maxdepth \@maxdepth
}
%    \end{teX}
%  \end{macro}
%
%  \begin{macro}{\@reinserts}
%    This is the code which reinserts the inserts.  It puts them all
%    in one place; this can make some of them come out on the wrong
%    page.
%    It has been put into a separate macro to expedite experimentation.
%    \begin{teX}
\gdef \@reinserts{%
  \ifvoid\footins\else\insert\footins{\unvbox\footins}\fi
}
%    \end{teX}
%  \end{macro}
%
%
%
%  \begin{macro}{\@makespecialcolbox}
%    This implements certain variations in page-makeup.
%    \begin{teX}
\gdef \@makespecialcolbox {%
%    \end{teX}

%    First we find the natural height of the column.
%    See above for discussion of what is happening here.
%    This needs further investigation, especially in this `special
%    case'. 
%    \begin{teX}
   \setbox\@outputbox \vbox {%
     \@texttop
     \dimen@ \dp\@outputbox
     \unvbox\@outputbox
     \vskip-\dimen@
     }%
   \@tempdima \@colht
   \ifdim \wd\@kludgeins>\z@
%    \end{teX}
%    Note that in this case (the *-version), the height of the
%    |\@kludgeins| box is not used since its value is somewhat
%    arbitrary: it need only be big enough to ensure that the
%    page-break is not taken prematurely.
%
%    Here we calculate how much vertical space needs to be added in
%    order to enable the column to fit into a box of size |\@colht|
%    using the best information we have about the amount of shrink
%    available (another thing which is known internally about a box,
%    but cannot be accessed at the \TeX{} level!).
%
%    This needs \TeX3 otherwise |\pageshrink| is zero anyway; it may
%    not be exactly the figure we wish as it is the total available
%    from the all the material collected before the page-break
%    decision is made.  It will, we think, always be an overestimate
%    of the actual shrink in the box; therefore this should always
%    force the shortest possible column with the possibility of an
%    overfull box.
%
%    This should work for bothe flush- and ragged-bottom setting since
%    it makes the contents no smaller than the size (|\@colht|) of the
%    box into which they are put.
%
%    Their should perhaps be an upper limit, of 0pt?, on the extra
%    space added to force shrinking.
%    \task{CAR}{Further investigation of kludge-* space}
%
%    See above for a discussion of the |\boxmaxdepth| setting here.
%    
%    \begin{teX}
     \advance \@tempdima -\ht\@outputbox
     \advance \@tempdima \pageshrink
     \setbox\@outputbox \vbox to \@colht {%
       \unvbox\@outputbox
       \vskip \@tempdima
       \@textbottom
       }%
%    \end{teX}
%    For the unstarred version, the final size of the page is
%    precisely specified.  Therefore, at least for the flush-bottom
%    case, we need to ensure that, visually, it has this size exactly.
%
%    Thus we calculate this size and set the material in a box of this
%    size, which is then put into a box of size |\@colht| with |\vss|
%    at the bottom.
%    \begin{teX}
   \else
     \advance \@tempdima -\ht\@kludgeins
%    \end{teX}
%    This type of final packaging could be done always; this may
%    simplify all of this page-makeup.
%
%    It is not necessary to set |\boxmaxdepth| here since the
%    |\@outputbox| ends with glue.
%
%    \begin{teX}
     \setbox \@outputbox \vbox to \@colht {%
       \vbox to \@tempdima {%
         \unvbox\@outputbox
         \@textbottom}%
       \vss}%
   \fi
%    \end{teX}
%    Finally we need to explicitly make the insert box void.
%    \begin{teX}
   {\setbox \@tempboxa \box \@kludgeins}%
%    \end{teX}
%  \end{macro}

The following macros are just hooks and can be set to add top and
bottom glue on all pages. They are hardly used by anyone.\footnote{http://tex.stackexchange.com/questions/40469/use-of-texttop-and-textbottom-for-vertical-positioning} 
  \begin{macro}{\@texttop}
  \begin{macro}{\@textbottom}
    These do nothing as a default.
    \begin{teX}
\let \@texttop \relax
\let \@textbottom \relax
    \end{teX}
  \end{macro}
  \end{macro}

%  \begin{macro}{\@resetactivechars}
%  \begin{macro}{\@activechar@info}
%
% added hook to protect against certain active characters in the
% output routine. Default checks are for active space and end-of-line.
%
%    \begin{teX}
\def\@activechar@info #1{%
      \@latex@info@no@line {Active #1 character found while
                            output routine is active  
                            \MessageBreak
                            This may be a bug in a package file
                            you are using}%
}
%    \end{teX}
%    
%    Do not put any spaces in this next bit!
%    \begin{teX}
\begingroup
\obeylines\obeyspaces%
\catcode`\'\active%
\gdef\@resetactivechars{%
\def^^M{\@activechar@info{EOL}\space}%
\def {\@activechar@info{space}\space}%
\let'\active@math@prime}%
\endgroup
%    \end{teX}
%  \end{macro}
%  \end{macro}
%
  \begin{macro}{\@outputpage}
         
    The |\color@hbox| hooks here are used to avoid putting just a
    colour special into an otherwise empty box (in a header or
    footer).  These boxes are often set to be completely empty and so
    adding a special produces a very underfull box message.
    
    There has been extensive tidying up of the old code here;
    including the removal of a level of grouping.
 
    The setting of |\protect| immediately before the |\shipout|
    is needed so that protected commands within |\write|s are
    handled correctly.
 
    Within shipout's vbox it is reset to its default value, |\relax|.
 
    Resetting it to its default value after the shipout has been 
    completed (and the contents of the writes have been expanded)
    must be done by use of |\aftergroup|.
    This is because it must have the value |\relax|
    before macros coming from other uses of |\aftergroup| within
    this box are expanded.

    Putting this into the |\aftergroup| token list does not affect
    the definition used in expanding the |\write|s because the
    aftergroup token list is only constructed when popping the
    save-stack, it is not expanded until after the shipout is
    completed.

    Question: should things from an |\aftergroup| within the shipped
    out box be executed in the environment set up for the writes, or
    after it finishes?

    A lot of this code has been in-lined tp prevent mis-use of
    internal commands as hooks.
    \begin{teX}
\def\@outputpage{%
\begingroup           % the \endgroup is put in by \aftergroup
    \end{teX}
    Now all the set-up stuff has been in-lined for Frank.

    First the stuff for the writes.
    
    From here \ldots\ was in the command |\@writesetup|. 
    \begin{teX}
  \let \protect \noexpand
    \end{teX}

    RmS 93/08/19: Redefined accents to allow changes in font encoding; 
    but exactly why was this needed?
 
    The |\catcode`\ = 10| was removed as it was considered useless 
    (presumably because nothing gets tokenised during shipout).
    
    This was put in as some error produced active spaces in a mark, I 
    think.
    
    Why was the hyphen reset?
    
    \begin{teX}
  \@resetactivechars
    \end{teX}
%    If a page break happens between the start of a list and its first
%    item the |@newlist| will be true and this will mess up any list
%    that is used in the header or footer of the page. So we have to
%    reset that flag.
%    \begin{teX}
  \global\let\@@if@newlist\if@newlist
  \global\@newlistfalse
%    \end{teX}
%     with the new encoding setup they can use \cs{let}.
%     It could also use the new internal commands?}
%    This next hook replaces the following:
%    \begin{verbatim}
%      \let\-\@dischyph
%      \let\'\@acci\let\`\@accii\let\=\@acciii
%      \let\\\@normalcr
%      \let\par\@@par %% 15 Sep 87 (this was once inside the box)
%    \end{verbatim}
%    and it does more than they did; in particular it sets:
%    \begin{verbatim}
%      \parindent\z@
%      \parskip\z@skip
%      \everypar{}%
%      \leftskip\z@skip
%      \rightskip\z@skip
%      \parfillskip\@flushglue
%      \lineskip\normallineskip
%      \baselineskip\normalbaselineskip
%      \sloppy
%    \end{verbatim}
%    
%    \begin{teX}
  \@parboxrestore
%    \end{teX}
%    \ldots\ to here was in the command |\@writesetup|. 
%    \begin{teX}
  \shipout \vbox{%
    \set@typeset@protect
    \aftergroup \endgroup
    \aftergroup \set@typeset@protect
                                % correct? or just restore by ending
                                % the group?
%    \end{teX}
%    This first bit has been moved inside the shipped out box.
%    
    Now the setup inside the shipped out box; this should conatin all 
    the stuff that could only affect typesetting; other stuff may need 
    to be reset for the writes also.
    
    From here \ldots\ was in the command |\@shipoutsetup|. 
    \begin{teX}
  \if@specialpage
    \global\@specialpagefalse\@nameuse{ps@\@specialstyle}%
  \fi
  \if@twoside
    \ifodd\count\z@ \let\@thehead\@oddhead \let\@thefoot\@oddfoot
         \let\@themargin\oddsidemargin
    \else \let\@thehead\@evenhead
       \let\@thefoot\@evenfoot \let\@themargin\evensidemargin
    \fi
  \fi
    \end{teX}
    
    The rest was always inside the box.

    \begin{teX}
  \reset@font 
    \end{teX}
    RmS 93/08/06 Added |\lineskiplimit=0pt| to guard against it being
              nonzero: e.g. by |\offinterlineskip| being in effect.
    
    There are probably lots of other things that may need resetting.
    
    \begin{teX}
  \normalsize
    \end{teX}
 Reset the space factors.
     {Call \cs{normalsfcodes} (from patch file) latex/2404}
    \begin{teX}
  \normalsfcodes
    \end{teX}

 Reset these here (previously reset separately for head and foot)
    \begin{teX}
  \let\label\@gobble
  \let\index\@gobble
  \let\glossary\@gobble
    \end{teX}

    \begin{teX}
  \baselineskip\z@skip \lineskip\z@skip \lineskiplimit\z@
    \end{teX}
    \ldots\ to here was in the command |\@shipoutsetup|. 
    \begin{teX}
    \@begindvi
    \vskip \topmargin
    \moveright\@themargin \vbox {%
      \setbox\@tempboxa \vbox to\headheight{%
        \vfil
        \color@hbox
          \normalcolor
          \hb@xt@\textwidth{\@thehead}%
        \color@endbox
        }%                        %% 22 Feb 87
      \dp\@tempboxa \z@
      \box\@tempboxa
      \vskip \headsep
      \box\@outputbox
      \baselineskip \footskip
      \color@hbox
        \normalcolor
        \hb@xt@\textwidth{\@thefoot}%
      \color@endbox
      }%
    }%
    \end{teX}
   |\endgroup| now inserted by |\aftergroup|

 Restore |\if@newlist|
    \begin{teX}
  \global\let\if@newlist\@@if@newlist
    \end{teX}

    \begin{teX}
  \global \@colht \textheight
  \stepcounter{page}%
    \end{teX}
%    It is now clear that this does something useful, thanks to Piet
%    van Oostrum.  It is needed because a float page is made without
%    using TeX's page-builder; thus the output routine is never called
%    so the marks are not updated.
%    \begin{teX}
  \let\firstmark\botmark
}
%    \end{teX}
%  \end{macro}
%  \end{macro}
%  \end{macro}
%
% \begin{macro}{\@begindvi}
%    This unboxes stuff that must appear before anything else in the
%    |.dvi| file, then returns that box register to the free list and
%    cancels itself.
%
%    The stuff in the box should not add any typeset material to the
%    page. 
%    \begin{teX}
\def \@begindvi{%
  \unvbox \@begindvibox
  \global\let \@begindvi \@empty
}
%    \end{teX}
% \end{macro}


% \begin{macro}{\@combinefloats}
% \begin{macro}{\@cflb}
%    The |\boxmaxdepth| setting here was not made local to
%    a box so was dangerous.  It is needed only within the box made
%    by |\@cflt| (and not normally even there), so it has been
%    moved there; this also agrees with the original pseudcode.
%
%    \begin{teX}
\def \@combinefloats {%
    \ifx \@toplist\@empty \else \@cflt \fi
    \ifx \@botlist\@empty \else \@cflb \fi
}
%    \end{teX}
%  
%    \begin{teX}
\def \@cflt{%
    \let \@elt \@comflelt
    \setbox\@tempboxa \vbox{}%
    \@toplist
    \setbox\@outputbox \vbox{%
                             \boxmaxdepth \maxdepth
                             \unvbox\@tempboxa
                             \vskip -\floatsep
                             \topfigrule
                             \vskip \textfloatsep
                             \unvbox\@outputbox
                             }%
    \let\@elt\relax
    \xdef\@freelist{\@freelist\@toplist}%
    \global\let\@toplist\@empty
}
%    \end{teX}
%
%    \begin{teX}
\def \@cflb {%
    \let\@elt\@comflelt
    \setbox\@tempboxa \vbox{}%
    \@botlist
    \setbox\@outputbox \vbox{%
                             \unvbox\@outputbox
                             \vskip \textfloatsep
                             \botfigrule
                             \unvbox\@tempboxa
                             \vskip -\floatsep
                             }%
    \let\@elt\relax
    \xdef\@freelist{\@freelist\@botlist}%
    \global \let \@botlist\@empty
}
%    \end{teX}
%  \end{macro}
%  \end{macro}
%  \end{macro}
%
% \begin{macro}{\@comflelt}
% \begin{macro}{\@comdblflelt}
% \begin{macro}{\@combinedblfloats}
% 
%    \begin{teX}
\def\@comflelt#1{\setbox\@tempboxa
      \vbox{\unvbox\@tempboxa\box #1\vskip\floatsep}}
%    \end{teX}
%
%    \begin{teX}
\def\@comdblflelt#1{\setbox\@tempboxa
      \vbox{\unvbox\@tempboxa\box #1\vskip\dblfloatsep}}
%    \end{teX}
%
%    \begin{teX}
\def \@combinedblfloats{%
  \ifx \@dbltoplist \@empty
  \else
    \setbox\@tempboxa \vbox{}%
    \let \@elt \@comdblflelt
    \@dbltoplist
    \let \@elt \relax 
    \xdef \@freelist {\@freelist\@dbltoplist}%
    \global\let \@dbltoplist \@empty
    \setbox\@outputbox \vbox to\textheight
%    \end{teX}
%
%    The setting of |\boxmaxdepth| here has no effect since the
%    |\@outputbox| should already have depth zero.  Even so, it would
%    have no effect on the layout of the page.
% \changes{v1.0l}{1994/03/15}{Removed boxmaxdepth setting.}
%    \begin{teX}
      {%\boxmaxdepth\maxdepth   %% probably not needed, CAR
       \unvbox\@tempboxa\vskip-\dblfloatsep
%    \end{teX}
%    Here we need different typesetting if the top float comes from
%    |\@topnewpage|. 
% \changes{v1.0n}{1994/04/30}{Removed rule in topnewpage case}
%    \begin{teX}
       \ifnum \@dbltopnum>\m@ne
         \dblfigrule
       \fi
       \vskip \dbltextfloatsep
       \box\@outputbox
       }%
  \fi
}
%    \end{teX}
%  \end{macro}
%  \end{macro}
%  \end{macro}
%
%
%  \begin{macro}{\@startcolumn}
% 
%    We could combine (most of) these two into |\@startcol <list>|.
%    Note that |\@xstartcol| was only used once (\ie in
%    |\@startcolumn|); it has therefore been removed.  This is not quite
%    as efficient but it now has the same structure as
%    |\@startdblcolumn|.
%
%    The empty-list test has been moved to |\@tryfcolumn|.
%
%    \begin{teX}
%<*2ekernel|autoload|fltrace>
\def \@startcolumn {%
  \global \@colroom \@colht
  \@tryfcolumn \@deferlist
  \if@fcolmade
%<*trace>
    \tr@ce{PAGE: float \if@twocolumn column \else page \fi
                completed}%
%</trace>
  \else
%    \end{teX}
% \changes{v1.0h}{1993/12/12}{defs changed to lets}
%    \begin{teX}
    \begingroup
      \let \reserved@b \@deferlist
      \global \let \@deferlist \@empty
      \let \@elt \@scolelt
      \reserved@b
    \endgroup
  \fi
}
%    \end{teX}
%
%    This one does not need to set |\@colht|.
%
%    \begin{teX}
\def \@startdblcolumn {%
%    \end{teX}
% Not needed since this always comes after |\@outputpage|:
%    \begin{teX}
% \global \@colht \textheight
  \@tryfcolumn \@dbldeferlist
  \if@fcolmade
%<*trace>
    \tr@ce{PAGE: double float page completed}%
%</trace>
  \else
%    \end{teX}
% \changes{v1.0h}{1993/12/12}{defs changed to lets}
%    \begin{teX}
    \begingroup
      \let \reserved@b \@dbldeferlist
      \global \let \@dbldeferlist \@empty
      \let \@elt \@sdblcolelt
      \reserved@b
    \endgroup
  \fi
}
%    \end{teX}
%  \end{macro}
%  \end{macro}
%
%  \begin{macro}{\@tryfcolumn}
% \changes{v1.0f}{1993/12/05}{Command changed}
%    Now tests if its list is empty before any further exertion.
%
%    \begin{teX}
\def \@tryfcolumn #1{%
  \global \@fcolmadefalse
  \ifx #1\@empty
  \else
%<*trace>
     \tr@ce{PAGE: try float \if@twocolumn column/page\else page\fi
                  ---\string #1}%
     \tr@ce{----- \string #1: #1}%
%</trace>
%    \end{teX}
% \changes{v1.0h}{1993/12/12}{defs changed to lets}
%    \begin{teX}
    \xdef\@trylist{#1}%
    \global \let \@failedlist \@empty
    \begingroup
      \let \@elt \@xtryfc \@trylist
    \endgroup
    \if@fcolmade
      \@vtryfc #1%
    \fi
  \fi
}
%    \end{teX}
%
%  \end{macro}
%
%
% \begin{macro}{\@scolelt}
%    \begin{teX}
\def\@scolelt#1{\def\@currbox{#1}\@addtonextcol}
%    \end{teX}
% \end{macro}
%
% \begin{macro}{\@sdblcolelt}
%    \begin{teX}
\def\@sdblcolelt#1{\def\@currbox{#1}\@addtodblcol}
%    \end{teX}
% \end{macro}
%
% \begin{macro}{\@vtryfc}
%    \begin{teX}
\def\@vtryfc #1{%
  \global\setbox\@outputbox\vbox{}%
  \let\@elt\@wtryfc
  \@flsucceed
  \global\setbox\@outputbox \vbox to\@colht{%
    \vskip \@fptop
    \vskip -\@fpsep
    \unvbox \@outputbox
    \vskip \@fpbot}%
  \let\@elt\relax
  \xdef #1{\@failedlist\@flfail}%
  \xdef\@freelist{\@freelist\@flsucceed}}
%    \end{teX}
% \end{macro}
%
% \begin{macro}{\@wtryfc}
%    \begin{teX}
\def\@wtryfc #1{%
  \global\setbox\@outputbox\vbox{%
    \unvbox\@outputbox
    \vskip\@fpsep
    \box #1}}
%    \end{teX}
% \end{macro}
%
% \begin{macro}{\@xtryfc}
%    \begin{teX}
\def\@xtryfc #1{%
  \@next\reserved@a\@trylist{}{}%
  \@currtype \count #1%
  \divide\@currtype\@xxxii
  \multiply\@currtype\@xxxii
  \@bitor \@currtype \@failedlist
  \@testfp #1%
  \ifdim \ht #1>\@colht
    \@testtrue
  \fi
  \if@test
    \@cons\@failedlist #1%
  \else
    \@ytryfc #1%
  \fi}
%    \end{teX}
% \end{macro}
%
% \begin{macro}{\@ytryfc}
%    \begin{teX}
\def\@ytryfc #1{%
  \begingroup
    \gdef\@flsucceed{\@elt #1}%
    \global\let\@flfail\@empty
    \@tempdima\ht #1%
    \let\@elt\@ztryfc
    \@trylist
    \ifdim \@tempdima >\@fpmin
      \global\@fcolmadetrue
    \else
      \@cons\@failedlist #1%
    \fi
  \endgroup
  \if@fcolmade
    \let\@elt\@gobble
  \fi}
%    \end{teX}
% \end{macro}
%
% \begin{macro}{\@ztryfc}
%    \begin{teX}
\def\@ztryfc #1{%
  \@tempcnta \count#1%
  \divide\@tempcnta\@xxxii
  \multiply\@tempcnta\@xxxii
  \@bitor \@tempcnta {\@failedlist \@flfail}%
  \@testfp #1%
  \@tempdimb\@tempdima
  \advance\@tempdimb \ht#1%
  \advance\@tempdimb\@fpsep
  \ifdim \@tempdimb >\@colht
    \@testtrue
  \fi
  \if@test
    \@cons\@flfail #1%
  \else
    \@cons\@flsucceed #1%
    \@tempdima\@tempdimb
  \fi}
%    \end{teX}
% \end{macro}
%
%
% The major changes for float suppression and the changes to the float
% mechanism to make it conform to the documentation are in these next
% macros.
%
%  \begin{macro}{\@addtobot}
%
%    \begin{teX}
%<*2ekernel|autoload|fltrace>
\def \@addtobot {%
   \@getfpsbit 4\relax
   \ifodd \@tempcnta
     \@flsetnum \@botnum
     \ifnum \@botnum>\z@
       \@tempswafalse
       \@flcheckspace \@botroom \@botlist
       \if@tempswa
%    \end{teX}
%    This next line means that this page is produced with box 255 
%    having depth zero, rather than the normal maxdepth: is this
%    needed, useful? 
%    \begin{teX}
         \global \maxdepth \z@
         \@flupdates \@botnum \@botroom \@botlist
         \@inserttrue
       \fi
%<*trace>
     \else
       \tr@ce{Fail: botnum = \the \@botnum:
                                  fpstype \the \@fpstype=ORD?}%
       \ifnum \@fpstype<\sixt@@n
         \tr@ce{ERROR: !b float not successful (addtobot)}%
       \fi
%</trace>
     \fi
   \fi
}
%    \end{teX}
%  \end{macro}
%
%  \begin{macro}{\@addtotoporbot}
%    Lots of changes.
%
%    \begin{teX}
\def \@addtotoporbot {%
%<*trace>
   \tr@ce{***Start addtotoporbot}%
%</trace>
   \@getfpsbit \tw@
%<*trace>
   \tr@ce{fpstype \ifodd \@tempcnta OK \else not \fi top:
                                                     \the \@fpstype}%
%</trace>
   \ifodd \@tempcnta
     \@flsetnum \@topnum
     \ifnum \@topnum>\z@
       \@tempswafalse
       \@flcheckspace \@toproom \@toplist
       \if@tempswa
         \@bitor\@currtype{\@midlist\@botlist}%
%<*trace>
           \tr@ce{(mid+bot)list: \@midlist, \@botlist:
                              (addtotoporbot-before)}%
%</trace>
         \if@test
%<*trace>
           \tr@ce{type already on list: mid or bot---sent to addtobot}%
%</trace>
         \else
          \@flupdates \@topnum \@toproom \@toplist
%<*trace>
          \tr@ce{colroom (after-top) = \the \@colroom}%
          \tr@ce{colnum (after-top) = \the \@colnum}%
          \tr@ce{topnum (after-top) = \the \@topnum}%
          \tr@ce{***Success: top}%
%</trace>
          \@inserttrue
         \fi
       \fi
%<*trace>
     \else
       \tr@ce{Fail: topnum = \the \@topnum: fpstype
                                            \the \@fpstype=ORD?}%
       \ifnum \@fpstype<\sixt@@n
         \tr@ce{ERROR: !t float not successful (addtotoporbot)}%
       \fi
%</trace>
     \fi
   \fi
   \if@insert
   \else
%<*trace>
     \tr@ce{sent to addtobot (addtotoporbot)}%
%</trace>
     \@addtobot
   \fi
}
%</2ekernel|autoload|fltrace>
%    \end{teX}
%  \end{macro}
%
%  \begin{macro}{\@addtocurcol}
% \changes{v1.0f}{1993/12/05}{Command changed}
% \task{CAR}{Add rules around h floats for FMi}
% \task{CAR}{Investigate pagebreak option possibilities}
%    Lots of changes.
%
%    \begin{teX}
%<*2ekernel|autoload|fltrace|flafter>
\def \@addtocurcol {%
%<*trace>
  \tr@ce{***Start addtocurcol}%
%</trace>
   \@insertfalse
   \@setfloattypecounts
   \ifnum \@fpstype=8
%<*trace>
     \tr@ce{fpstype !p only (addtocurcol): \the \@fpstype = 8?}%
%</trace>
   \else
     \ifnum \@fpstype=24
%<*trace>
       \tr@ce{fpstype p only (addtocurcol): \the \@fpstype = 24?}%
%</trace>
     \else
       \@flsettextmin
%    \end{teX}
% This is a new adjustment which is quite a major change in
% functionality; but it implements the documentation.
% Note that |\@reqcolroom| will include the whole of the
% page-so-far, and hence includes |\@textfloatsheight| of floats,
% so before comparing it with |\@textmin|, we add this to
% |\@textmin| also.
%    \begin{teX}
%<*trace>
       \tr@ce{textfloatsheight (before) = \the \@textfloatsheight}%
%</trace>
       \advance \@textmin \@textfloatsheight
       \@reqcolroom \@pageht
%    \end{teX}
% This line must be removed since |\@specialoutput| changed.
%    \begin{teX}
%       \advance \@reqcolroom \@pagedp
%<*trace>
       \tr@ce{textmin + textfloatsheight: \the \@textmin}%
       \tr@ce{page-so-far: \the \@reqcolroom}%
%</trace>
       \ifdim \@textmin>\@reqcolroom
         \@reqcolroom \@textmin
%<*trace>
         \tr@ce{ORD? textmin being used}%
%</trace>
       \fi
       \advance \@reqcolroom \ht\@currbox
%<*trace>
       \tr@ce{float size = \the \ht \@currbox (addtocurcol)}%
       \tr@ce{colroom = \the \@colroom (addtocurcol)}%
       \tr@ce{reqcolroom = \the \@reqcolroom (addtocurcol)}%
%</trace>
       \ifdim \@colroom>\@reqcolroom
         \@flsetnum \@colnum
         \ifnum \@colnum>\z@
           \@bitor\@currtype\@deferlist
%<*trace>
           \tr@ce{deferlist: \@deferlist: (addtocurcol-before)}%
%</trace>
           \if@test
%<*trace>
             \tr@ce{type already on list: defer (addtocurcol)}%
%</trace>
           \else
             \@bitor\@currtype\@botlist
%<*trace>
           \tr@ce{botlist: \@botlist: (addtocurcol-before)}%
%</trace>
             \if@test
%<*trace>
               \tr@ce{type already on list: bot---sent to addtobot}%
%</trace>
               \@addtobot
             \else
%<*trace>
               \tr@ce{fpstype \ifodd \@tempcnta OK \else not \fi
                      here: \the \@fpstype}%
%</trace>
               \ifodd \count\@currbox
                 \advance \@reqcolroom \intextsep
                 \ifdim \@colroom>\@reqcolroom
                   \global \advance \@colnum \m@ne
                   \global \advance \@textfloatsheight \ht\@currbox
%    \end{teX}
% This may sometimes give an overestimate.
%    \begin{teX}
                   \global \advance \@textfloatsheight 2\intextsep
                   \@cons \@midlist \@currbox
%<*trace>
                   \tr@ce{***Success: here}%
                   \tr@ce{textfloatsheight (after-here) =
                        \the \@textfloatsheight}%
                   \tr@ce{colnum (after-here) = \the \@colnum}%
%</trace>
%    \end{teX}
% 
% CHANGE TO |\@addtocurcol|:
% 
% |\penalty\z@| changed to |\penalty\interlinepenalty| so |\samepage|
% works properly with figure and table environments.
% (Changed 23 Oct 86)
%
% There is also an |\addpenalty\interlinepenalty| above.
%
% Since in 2e |\samepage| is no longer supported, these could be
% removed.
%
% Although it is best to use |\addvspace| in case two h floats come
% together, this makes other spacing more difficult to adjust; whereas
% if a user specifies two h floats together then they can more easily
% get the spacing correct by ad hoc commands.
%
% It is necessary to adjust for the addition of |\parskip| here in
% case the float is added betweeen paragraphs (\ie when in vertical
% mode).
%
% If the nobreak switch is true we need to reset it and clear
% |\everypar| sionce the float may not reset the flag and cannot reset
% the |\everypar| globally.
% \changes{v1.0l}{1994/03/15}{Changed \cs{addvspace} to \cs{vskip}}
% \changes{v1.1i}{1994/11/21}
%   {Added \cs{if@nobreak} test before float box}
% \changes{v1.1z}{1996/10/24}{Added \cs{nobreak}, etc as appropriate}
% 
% Typesetting starts here (we are in vertical mode).
%    \begin{teX}
                   \if@nobreak
                     \nobreak
                     \@nobreakfalse
                     \everypar{}%
                   \else
                     \addpenalty \interlinepenalty
                   \fi
                   \vskip \intextsep
                   \box\@currbox
                   \penalty\interlinepenalty
                   \vskip\intextsep
                   \ifnum\outputpenalty <-\@Mii \vskip -\parskip\fi
%    \end{teX}
% Typesetting ends here.
%    \begin{teX}
                   \outputpenalty \z@
                   \@inserttrue
%<*trace>
                 \else
                   \tr@ce{Fail---no room at 2nd test of colroom
                                 (addtocorcol \string\intextsep)}%
%</trace>
                 \fi
               \fi
               \if@insert
               \else
%<*2ekernel|autoload|fltrace>
%<*trace>
                 \tr@ce{not here: sent to addtotoporbot}%
%</trace>
                 \@addtotoporbot
%</2ekernel|autoload|fltrace>
%<*!2ekernel&!autoload&!fltrace>
%<*trace>
                 \tr@ce{not here: sent to addtobot}%
%</trace>
                 \@addtobot
%</!2ekernel&!autoload&!fltrace>
               \fi
             \fi
           \fi
%<*trace>
         \else
           \tr@ce{Fail: colnum = \the \@colnum:
                        fpstype \the \@fpstype=ORD?}%
           \ifnum \@fpstype<\sixt@@n
             \tr@ce{ERROR: BANG float not successful (addtocurcol)}%
           \fi
%</trace>
         \fi
%<*trace>
       \else
         \tr@ce{Fail---no room: fl box ht: \the \ht \@currbox
                                                     (addtocurcol)}%
%</trace>
       \fi
     \fi
   \fi
   \if@insert
   \else
     \@resethfps
%<*trace>
     \tr@ce{put on deferlist (addtocurcol)}%
%</trace>
     \@cons\@deferlist\@currbox
%<*trace>
     \tr@ce{deferlist: \@deferlist: (addtocurcol-after)}%
%</trace>
   \fi
}
%</2ekernel|autoload|fltrace|flafter>
%    \end{teX}
%  \end{macro}
%
%  \begin{macro}{\@addtonextcol}
% \changes{v1.0f}{1993/12/05}{Command changed}
%    Lots of changes.
%
%    \begin{teX}
%<*2ekernel|autoload|fltrace>
\def\@addtonextcol{%
  \begingroup
%<*trace>
   \tr@ce{***Start addtonextcol}%
%</trace>
   \@insertfalse
   \@setfloattypecounts
   \ifnum \@fpstype=8
%<*trace>
     \tr@ce{fpstype not curcol: \the \@fpstype = 8?}%
%</trace>
   \else
     \ifnum \@fpstype=24
%<*trace>
       \tr@ce{fpstype not curcol: \the \@fpstype = 24?}%
%</trace>
     \else
       \@flsettextmin
%<*trace>
       \tr@ce{text-so-far: 0pt (top of col)}%
%</trace>
       \@reqcolroom \ht\@currbox
%<*trace>
       \tr@ce{float size: \the \@reqcolroom (addtonextcol)}%
%</trace>
       \advance \@reqcolroom \@textmin
%<*trace>
       \tr@ce{colroom = \the \@colroom (addtonextcol)}%
       \tr@ce{reqcolroom = \the \@reqcolroom (addtonextcol)}%
%</trace>
       \ifdim \@colroom>\@reqcolroom
         \@flsetnum \@colnum
         \ifnum\@colnum>\z@
            \@bitor\@currtype\@deferlist
%<*trace>
            \tr@ce{deferlist: \@deferlist: (addtonextcol-before)}%
%</trace>
            \if@test
%<*trace>
              \tr@ce{type already on list: defer (addtonextcol)}%
%</trace>
            \else
%<*trace>
              \tr@ce{sent to addtotoporbot (addtonextcol)}%
%</trace>
              \@addtotoporbot
            \fi
         \fi
%<*trace>
       \else
         \tr@ce{Fail---no room: fl box ht: \the \ht \@currbox
                                                  (addtonextcol)}%
%</trace>
       \fi
     \fi
   \fi
   \if@insert
   \else
%<*trace>
     \tr@ce{put back on deferlist (addtonextcol)}%
%</trace>
     \@cons\@deferlist\@currbox
%<*trace>
     \tr@ce{deferlist: \@deferlist: (addtonextcol-after)}%
%</trace>
   \fi
%<*trace>
   \tr@ce{End of addtonextcol -- locally counts:}%
   \tr@ce{ col: \the \@colnum. top: \the \@topnum. bot: \the \@botnum.}%
%</trace>
  \endgroup
%<*trace>
  \tr@ce{End of addtonextcol -- globally counts:}%
  \tr@ce{col: \the \@colnum. top: \the \@topnum. bot: \the \@botnum.}%
%</trace>
}
%    \end{teX}
%  \end{macro}
%
%  \begin{macro}{\@addtodblcol}
% \changes{v1.0f}{1993/12/05}{Command changed}
%    Lots of changes.
%
%    \begin{teX}
\def\@addtodblcol{%
  \begingroup
%<*trace>
  \tr@ce{***Start addtodblcol}%
%</trace>
   \@insertfalse
   \@setfloattypecounts
   \@getfpsbit \tw@
%<*trace>
   \tr@ce{fpstype \ifodd \@tempcnta OK \else not \fi dbltop:
                                                     \the \@fpstype}%
%</trace>
   \ifodd\@tempcnta
     \@flsetnum \@dbltopnum
     \ifnum \@dbltopnum>\z@
       \@tempswafalse
       \ifdim \@dbltoproom>\ht\@currbox
         \@tempswatrue
%<*trace>
         \tr@ce{Space OK: \@dbltoproom =
                \the \@dbltoproom > \the \ht \@currbox
                                         (dbltoproom)}%
%</trace>
       \else
%<*trace>
         \tr@ce{fpstype: \the \@fpstype (addtodblcol)}%
%</trace>
         \ifnum \@fpstype<\sixt@@n
%<*trace>
           \tr@ce{BANG float ignoring \@dbltoproom}%
           \tr@ce{\@spaces \@dbltoproom = \the \@dbltoproom.
                           Ht float: \the \ht \@currbox-BANG}%
%</trace>
%    \end{teX}
% Need to check that there is room on the page, using the local value
% of |\@textmin| to make the necessary adjustment to |\@dbltoproom|.
%    \begin{teX}
           \advance \@dbltoproom \@textmin
%<*trace>
           \tr@ce{Local value of texmin: \the\@textmin}%
           \tr@ce{\@spaces space on page = \the \@dbltoproom.
                           Ht float: \the \ht \@currbox-BANG}%
%</trace>
           \ifdim \@dbltoproom>\ht\@currbox
             \@tempswatrue
%<*trace>
             \tr@ce{Space OK BANG: space on page = \the \@dbltoproom >
                                              \the \ht \@currbox}%
           \else
             \tr@ce{fpstype: \the \@fpstype}%
             \tr@ce{Fail---no room dbltoproom-BANG?:}%
             \tr@ce{\@spaces space on page = \the \@dbltoproom.
                           Ht float: \the \ht \@currbox}%
%</trace>
           \fi
           \advance \@dbltoproom -\@textmin
%<*trace>
         \else
           \tr@ce{fpstype: \the \@fpstype}%
           \tr@ce{Fail---no room dbltoproom-ORD?:}%
           \tr@ce{\@spaces \@dbltoproom = \the \@dbltoproom.
                           Ht float: \the \ht \@currbox}%
%</trace>
         \fi
       \fi
       \if@tempswa
           \@bitor \@currtype \@dbldeferlist
%<*trace>
           \tr@ce{dbldeferlist: \@dbldeferlist: (before)}%
%</trace>
           \if@test
%<*trace>
              \tr@ce{type already on list: dbldefer}%
%</trace>
           \else
              \@tempdima -\ht\@currbox
              \advance\@tempdima
                -\ifx \@dbltoplist\@empty \dbltextfloatsep \else
                                          \dblfloatsep \fi
              \global \advance \@dbltoproom \@tempdima
              \global \advance \@colht \@tempdima
              \global \advance \@dbltopnum \m@ne
              \@cons \@dbltoplist \@currbox
%<*trace>
              \tr@ce{dbltopnum (after) = \the \@dbltopnum}%
              \tr@ce{***Success: dbltop}%
%</trace>
              \@inserttrue
           \fi
       \fi
%<*trace>
     \else
       \tr@ce{Fail: dbltopnum = \the \@dbltopnum: fpstype
                                                  \the \@fpstype=ORD?}%
       \ifnum \@fpstype<\sixt@@n
         \tr@ce{ERROR: !t float not successful (addtodblcol)}%
       \fi
%</trace>
     \fi
   \fi
   \if@insert
   \else
%<*trace>
     \tr@ce{put on dbldeferlist}%
%</trace>
     \@cons\@dbldeferlist\@currbox
%<*trace>
     \tr@ce{dbldeferlist: \@dbldeferlist: (after)}%
%</trace>
   \fi
%<*trace>
   \tr@ce{End of addtodblcol -- locally count:}%
   \tr@ce{ dbltop: \the \@dbltopnum.}%
%</trace>
  \endgroup
%<*trace>
  \tr@ce{End of addtodblcol -- globally count:}%
  \tr@ce{dbltop: \the \@dbltopnum.}%
%</trace>
}
%</2ekernel|autoload|fltrace>
%    \end{teX}
%  \end{macro}
%
%
%
% \begin{macro}{\@addmarginpar}
%    \begin{teX}
%<*2ekernel|autoload>
\def\@addmarginpar{\@next\@marbox\@currlist{\@cons\@freelist\@marbox
    \@cons\@freelist\@currbox}\@latexbug\@tempcnta\@ne
    \if@twocolumn
        \if@firstcolumn \@tempcnta\m@ne \fi
    \else
      \if@mparswitch
         \ifodd\c@page \else\@tempcnta\m@ne \fi
      \fi
      \if@reversemargin \@tempcnta -\@tempcnta \fi
    \fi
    \ifnum\@tempcnta <\z@  \global\setbox\@marbox\box\@currbox \fi
    \@tempdima\@mparbottom
    \advance\@tempdima -\@pageht
    \advance\@tempdima\ht\@marbox
    \ifdim\@tempdima >\z@
      \@latex@warning@no@line {Marginpar on page \thepage\space moved}%
    \else
      \@tempdima\z@
    \fi
    \global\@mparbottom\@pageht
    \global\advance\@mparbottom\@tempdima
    \global\advance\@mparbottom\dp\@marbox
    \global\advance\@mparbottom\marginparpush
    \advance\@tempdima -\ht\@marbox
%    \end{teX}
% Putting box movement inside the `marbox':
%    \begin{teX}
    \global\setbox \@marbox
                   \vbox {\vskip \@tempdima
                          \box \@marbox}%
    \global \ht\@marbox \z@
    \global \dp\@marbox \z@
%    \end{teX}
% Sticking (rather than gluing:-) the `marbox' to the line above,
% changed vskip to kern:
%    \begin{teX}
    \kern -\@pagedp
    \nointerlineskip
    \hb@xt@\columnwidth
      {\ifnum \@tempcnta >\z@
          \hskip\columnwidth \hskip\marginparsep
       \else
          \hskip -\marginparsep \hskip -\marginparwidth
       \fi
       \box\@marbox \hss}%
%    \end{teX}
%    For this reason the following code can vanish:
%\begin{verbatim}
%    \nobreak             %% No longer needed.  CAR92/12
%    \vskip -\@tempdima   %% No longer needed.  CAR92/12
%\end{verbatim}
%    \begin{teX}
    \nointerlineskip
    \hbox{\vrule \@height\z@ \@width\z@ \@depth\@pagedp}}
%</2ekernel|autoload>
%    \end{teX}
% \end{macro}
%
% \subsubsection{Kludgeins}
%
% This part of the file is part of the implementation of the following
% two new commands for \LaTeX2e{}.
%
%
% \begin{verbatim}
% \enlargethispage{<dim>}
% \end{verbatim}
%
% Adds |<dim>| to the height of the current column only. On the printed
% page the bottom of this column is extended downwards by exactly
% |<dim>| without having any effect on the placement of the footer; this
% may result in an overprinting.
%
% \begin{verbatim}
% \enlargethispage*{<dim>}
% \end{verbatim}
%
% Similar to |\enlargethispage| but it tries to squeeze the column to
% be printed in as small a space as possible, ie it uses any
% shrinkability in the column. If the column was not explicitly broken
% (\eg with |\pagebreak|) this may result in an overfull box message but
% execpt for this it will come out as expected (if you know what to
% expect).
%
% The star form of this command is dedicated to Leslie Lamport, the
% other we need for ourselves (FMi, CAR).
%
% These commands may well have unwanted effects if used soon
% before a |\clearpage|: please give keep them clear of such places.
%
%  \begin{macro}{\@kludgeins}
% \changes{v0.1c}{1993/11/23}{Insert added}
%    The insert which makes \TeX{} do a lot of the necessary work.
%    All we need to put into it is the amount by which the pagegoal
%    should be changed.
%    \begin{teX}
%<*2ekernel|def1>
\newinsert \@kludgeins
\global\dimen\@kludgeins \maxdimen
\global\count\@kludgeins 1000
%</2ekernel|def1>
%    \end{teX}
%  \end{macro}
%
%
%  \begin{macro}{\enlargethispage}
%  \begin{macro}{\enlargethispage*}
% \changes{v0.1c}{1993/11/23}{Commands added}
%    The user command.
%    \begin{teX}
%<*2ekernel|def1>
\gdef \enlargethispage {%
   \@ifstar
     {%
%<*trace>
      \tr@ce{Enlarging page height * }%
%</trace>
      \@enlargepage{\hbox{\kern\p@}}}%
     {%
%<*trace>
      \tr@ce{Enlarging page height exactly---}%
%</trace>
      \@enlargepage\@empty}%
}
%</2ekernel|def1>
%<*autoload>
\def\enlargethispage{\@autoload{out1}\enlargethispage}
%</autoload>
%    \end{teX}
%  \end{macro}
%  \end{macro}
%
%
%  \begin{macro}{\@enlargepage}
% \changes{v0.1c}{1993/11/23}{Command added}
%    This actually inserts the insert, after checking for extreme
%    values of the change.
%    \begin{teX}
%<*2ekernel|def1>
\gdef\@enlargepage#1#2{%
%<*trace>
   \tr@ce{\@spaces\@spaces by #2}%
%</trace>
   \@tempskipa#2\relax
   \ifdim \@tempskipa>.5\maxdimen
     \@latexerr{Suggested\space extra\space height\space
                (\the\@tempskipa)\space dangerously\space
                large}\@eha
   \else
     \ifdim \vsize<.5\maxdimen
%<*trace>
       \tr@ce {Kludgeins added--pagegoal before: \the\pagegoal}%
%</trace>
       \@bsphack
         \insert\@kludgeins{#1\vskip-\@tempskipa}%
       \@esphack
%    \end{teX}
%    This next bit is for tracing only:
%    \begin{teX}
%<*trace>
       \ifvmode \par
         \tr@ce {Kludgeins added--pagegoal after: \the \pagegoal}%
       \fi
%</trace>
     \else
       \@latexerr{Page\space height\space already\space 
                  too\space large}\@eha
     \fi
   \fi
}
%</2ekernel|def1>
%    \end{teX}
%  \end{macro}
%
% \subsubsection{Float control}
%
% This part implements controllable floats and other changes
% to the float mechanism.
%
% It provides, at the doument level, the following command for
% inclusion in \LaTeX2e{}.
%
% \begin{verbatim}
%     \suppressfloats
% \end{verbatim}
%
% This suppresses all further floats on the current page.
%
% With an optional argument it suppresses only floats only in certain
% positions on the current page.
% \begin{quote}
%  |[t]|\quad suppresses only floats at the top of the page
%  |[b]|\quad suppresses only floats at the bottom of the page
% \end{quote}
%
% It also enables the use of an extra specifier, {\tt !}, in the
% location optional argument of a float.  If this is present then,
% just for this particular float, whenever it is processed by the float
% mechanism the followinhg are ignored:
%
% \begin{itemize}
% \item  all restrictions on the number of floats which can appear;
% \item  all explicit restrictions on the amount of space which should
%   (not) be occupied by floats and/or text.
% \end{itemize}
%
% The mechanism will still attempt to ensure that pages are not
% overfull.
%
% These specifiers override, for the single float, the suppression
% commands described above.
%
%
% In its current form, it also suplies a reasonably exhaustive, and
% somewhat baroque, means of tracing some aspects of the float
% mechanism.
%
% More tracing.
%  \begin{macro}{\tr@ce}
%  \begin{macro}{\notrace}
%  \begin{macro}{\tracefloats}
%  \begin{macro}{\@traceval}
%  \begin{macro}{\tracefloatvals}
%  \begin{macro}{\@tracemessage}
% \changes{v1.0c}{1993/11/30}{Commands added}
% \changes{v1.0h}{1993/12/12}{Commands changed}
% \changes{v1.0j}{1993/12/17}{tracefloatvals made a document command}
%    Set-up tracing for floats independent of other tracing as it
%    produces mega-output.  Default is no tracing.
% \changes{v1.1j}{1995/04/24}
%   {Do not add to kernel unless `trace' specified}
% \task{???}{Make proper tracing module}
%
%    \begin{teX}
%<*trace>
\def \@tracemessage #1{\typeout{LaTeX2e: #1}}
\def \tracefloats{\let \tr@ce \@tracemessage}
\def \notrace {\let \tr@ce \@gobble}
\notrace
\def \@traceval #1{\tr@ce{\string #1 = \the #1}}
\def \tracefloatvals{%
  \@dblfloatplacement
  \@floatplacement
  \@traceval\@colnum
  \@traceval\@colroom
  \@traceval\@topnum
  \@traceval\@toproom
  \@traceval\@botnum
  \@traceval\@botroom
  \@traceval\@fpmin
  \tr@ce{\string\textfraction = \textfraction}%
  \@traceval\@dbltopnum
  \@traceval\@dbltoproom
}
%</trace>
%<*flafter>
\providecommand\tr@ce[1]{}
%</flafter>
%    \end{teX}
%  \end{macro}
%  \end{macro}
%  \end{macro}
%  \end{macro}
%  \end{macro}
%  \end{macro}
%
%  \begin{macro}{\suppressfloats}
%  \begin{macro}{\@flstop}
% \changes{v1.0f}{1993/12/05}{Commands added}
% Float suppression commands: these set the relevant counter
% globally to zero.  Thus they are overridden for a particular float
% by an ! specifier.
%
%    \begin{teX}
%<*2ekernel|autoload>
\def \suppressfloats {%
   \@ifnextchar [%
     \@flstop
    {\global \@colnum \z@}%
}
%    \end{teX}
% Maybe this should be a loop over |#1|?
%    \begin{teX}
\def \@flstop [#1]{%
   \if t#1%
     \global \@topnum \z@
   \fi
   \if b#1%
     \global \@botnum \z@
   \fi
}
%    \end{teX}
%  \end{macro}
%  \end{macro}
%
%
% Manipulation of float placement and type; both their strings and the
% corresponding count registers.
%
%  \begin{macro}{\@fpstype}
%  \begin{macro}{\@reqcolroom}
%  \begin{macro}{\@textfloatsheight}
% \changes{v1.0f}{1993/12/05}{Commands added}
% First a new count register to go with |\@currtype|.
%
% Then a new skip register, for information needed to remove the
% |\@maxsep| conservatism: it is possible that this could use a
% temporary register.
%
% Finally a dimension register to hold the total height of in-text
% floats on the current page.  This is needed to implement a
% major change in the functionality of |\@addtocurcol| which is,
% nevertheless, a bug fix.
% It is not local and therefore cannot be a temporary register.
%
%    \begin{teX}
\newcount \@fpstype
\newdimen \@reqcolroom
\newdimen \@textfloatsheight
%</2ekernel|autoload>
%    \end{teX}
%  \end{macro}
%  \end{macro}
%  \end{macro}
%
%  \begin{macro}{\@fpsadddefault}
% \changes{v1.0f}{1993/12/05}{Command added}
% Adds the default placement to what is already there.
% 
% Should not need to change this, but could do it as follows:
% \begin{verbatim}
%\def \@fpsadddefault {%
%   \@temptokena \expandafter\expandafter\expandafter
%                {\csname fps@\@captype \endcsname}%
%   \edef \reserved@a {\the\@temptokena}%
%   \@onelevel@sanitize \reserved@a
%   \edef \@fps {\@fps\reserved@a}%
%}
% \end{verbatim}
%
%    \begin{teX}
%<*2ekernel|autoload|fltrace>
\def \@fpsadddefault {%
%<*trace>
   \tr@ce{fps changed from: \@fps}%
%</trace>
   \edef \@fps {\@fps\csname fps@\@captype \endcsname}%
   \@latex@warning {%
     No positions in optional float specifier.\MessageBreak
     Default added (so using `\@fps')}%
}
%    \end{teX}
%  \end{macro}
%
%  \begin{macro}{\@setfloattypecounts}
% \changes{v1.0f}{1993/12/05}{Command added}
% Sets counters |\@fpstype| and |\@currtype|.
%
% BANG $==$ bit4 of $|\count\@currbox| = 0$.
%
%    \begin{teX}
\def \@setfloattypecounts {%
  \@currtype \count\@currbox
  \@fpstype \count\@currbox
  \divide\@currtype\@xxxii \multiply\@currtype\@xxxii
  \advance \@fpstype -\@currtype
%<*trace>
  \tr@ce{(mod 32) fpstype: \the \@fpstype}%
  \tr@ce{(mult of 32) currtype: \the \@currtype}%
% Tracing only: but some should be changed into real errors/warnings?
  \ifnum \@fpstype<\sixt@@n
    \ifnum \@fpstype=\z@
      \tr@ce{ERROR: no PLACEMENT, fpstype = \the \@fpstype = 0?}%
    \fi
    \ifnum \@fpstype=\@ne
      \tr@ce{WARNING: only h, fpstype = \the \@fpstype = 1?}%
    \fi
    \tr@ce{BANG float}%
  \else
    \ifnum \@fpstype=\sixt@@n
      \tr@ce{ERROR: no PLACEMENT, fpstype = \the \@fpstype = 16?}%
    \fi
    \ifnum \@fpstype=17
      \tr@ce{WARNING: only h, fpstype = \the \@fpstype = 17?}%
    \fi
    \tr@ce{ORD float}%
  \fi
%</trace>
}
%</2ekernel|autoload|fltrace>
%    \end{teX}
%  \end{macro}
%
% Macros for getting, testing and setting bits of the fps.
%
%
%  \begin{macro}{\@getfpsbit}
% Sets |\@tempcnta| to required bit of |\count\@currbox|.
%
%    \begin{teX}
\def \@getfpsbit {%
   \@boxfpsbit \@currbox
}
%    \end{teX}
%  \end{macro}
%
%
%  \begin{macro}{\@boxfpsbit}
%    Used above.
%    \begin{teX}
\def \@boxfpsbit #1#2{%
   \@tempcnta \count#1%
   \divide \@tempcnta #2\relax
}
%    \end{teX}
%  \end{macro}
%
%  \begin{macro}{\@testfp}
% New definition of the float page test.
%    \begin{teX}
\def \@testfp #1{%
   \@boxfpsbit #18\relax % Really `#1 8' for human readers!
   \ifodd \@tempcnta
   \else
     \@testtrue
   \fi
}
%    \end{teX}
%  \end{macro}
%
%
%  \begin{macro}{\@setfpsbit}
% \changes{v1.0f}{1993/12/05}{Command added}
% Sets required bit of |\@tempcnta| (to 1).
%
%    \begin{teX}
\def \@setfpsbit #1{%
   \@tempcntb \@tempcnta
   \divide \@tempcntb #1\relax
   \ifodd \@tempcntb
   \else
     \advance \@tempcnta #1\relax
   \fi
}
%</2ekernel|autoload>
%    \end{teX}
%  \end{macro}
%
%
%  \begin{macro}{\@resethfps}
% Globally adds t as a possible location for an h or !h only placement:
% this must be done using the count.
%
% Although it will leave |\@fpstype| set to 17 even if it was
% originally 1, this does not matter since it is the last thing in
% |\@addtocurcol|. 
%    \begin{teX}
\def \@resethfps {%
   \let\reserved@a\@empty
   \ifnum \@fpstype=\@ne
      \def \reserved@a {!}%
      \@fpstype 17
   \fi
   \ifnum \@fpstype=17
     \global \advance \count\@currbox \tw@
     \@latex@warning@no@line {%
       `\reserved@a h' float specifier changed to `\reserved@a ht'}%
   \fi
}
%    \end{teX}
%  \end{macro}
%
%
% Special stuff for BANG floats.
%
%  \begin{macro}{\@flsetnum}
%
% Ignores any zero float counter value in case BANG.
%
% It uses a local assignment to the normally global counter: a bit
% naughty, perhaps?
%
% These assgnments are safe so long as the counter involved is only
% consulted once (\ie only for the `bang float') with the changed value.
% This is the case within |\@addtocurcol| because it is used only
% once within a call of the output routine (which forms a group).
%
% For |\@addtonextcol| this is achieved by putting a group around its
% code; this is needed because it is called (by |\@startcolumn|) for
% each float which was on the deferlist.  Almost identical
% considerations pertain to |\@addtodblcol|.  There may be more
% efficient ways to handle this, but the group seems to be the simplest.
%
% \changes{v1.0n}{1994/04/30}{Rogue space removed}
%    \begin{teX}
\def \@flsetnum #1{%
%<*trace>
   \tr@ce{fpstype: \the \@fpstype (flsetnum \string#1)}%
%</trace>
   \ifnum \@fpstype<\sixt@@n
     \ifnum #1=\z@
%<*trace>
       \tr@ce{BANG float resetting \string#1 to 1}%
%</trace>
       #1\@ne
     \fi
   \fi
%<*trace>
   \tr@ce{#1 (before) = \the #1}%
%</trace>
}
%    \end{teX}
%  \end{macro}
%
%
%  \begin{macro}{\@flsettextmin}
% \changes{v1.0f}{1993/12/05}{Command added}
% This ignores |\textfraction| space restriction in case BANG.
%
%    \begin{teX}
\def \@flsettextmin {%
%<*trace>
   \tr@ce{fpstype: \the \@fpstype (flsettextmin)}%
%</trace>
   \ifnum \@fpstype<\sixt@@n
%<*trace>
     \tr@ce{BANG ignoring textmin}%
%</trace>
     \@textmin \z@
   \else
     \@textmin \textfraction\@colht
%<*trace>
     \tr@ce{ORD textmin = \the \@textmin}%
%</trace>
   \fi
}
%    \end{teX}
%  \end{macro}
%
%
%  \begin{macro}{\@flcheckspace}
% This ignores space restriction in case BANG; this is still slightly
% conervative since it does not allow for the fact that, if there is
% no text in the column then |\textfloatsep| is not needed.
% Sets |@tempswa| true if there is room for |\@currbox|.
%
%    \begin{teX}
\def \@flcheckspace #1#2{%
   \advance \@reqcolroom
     \ifx #2\@empty \textfloatsep \else \floatsep \fi
%<*trace>
   \tr@ce{colroom = \the \@colroom (flcheckspace \string#1 \string#2)}%
   \tr@ce{reqcolroom = \the \@reqcolroom
                                   (flcheckspace \string#1 \string#2)}%
%</trace>
   \ifdim \@colroom>\@reqcolroom
     \ifdim #1>\ht\@currbox
       \@tempswatrue
%<*trace>
       \tr@ce{Space OK: #1 = \the #1 > \the \ht \@currbox
                                   (flcheckspace \string#1 \string#2)}%
%</trace>
     \else
%<*trace>
       \tr@ce{fpstype: \the \@fpstype
                                   (flcheckspace \string#1 \string#2)}%
%</trace>
       \ifnum \@fpstype<\sixt@@n
%<*trace>
         \tr@ce{BANG float ignoring #1
                                   (flcheckspace \string#1 \string#2):}%
         \tr@ce{\@spaces #1 = \the #1.  Ht float: \the \ht \@currbox
                                                          BANG}%
%</trace>
         \@tempswatrue
%<*trace>
       \else
         \tr@ce{Fail---no room (flcheckspace \string#1 \string#2)
                       (fpstype \the \@fpstype=ORD?):}%
         \tr@ce{\@spaces #1 = \the #1.  Ht float: \the \ht \@currbox
                                                          ORD?}%
%</trace>
       \fi
     \fi
%<*trace>
   \else
     \tr@ce{Fail---no room at 2nd test of colroom
                   (flcheckspace \string#1 \string#2)}%
%</trace>
   \fi
}
%    \end{teX}
%  \end{macro}
%
%
%  \begin{macro}{\@flupdates}
%    This updates everything when a float is placed.
%
%    \begin{teX}
\def \@flupdates #1#2#3{%
   \global \advance #1\m@ne
   \global \advance \@colnum \m@ne
   \@tempdima -\ht\@currbox
   \advance \@tempdima
     -\ifx #3\@empty \textfloatsep \else \floatsep \fi
   \global \advance #2\@tempdima
   \global \advance \@colroom \@tempdima
   \@cons #3\@currbox
}
%    \end{teX}
%  \end{macro}
%
%
 Interesting facts about float mechanisms past and present, together
 with a summary of various features, some unresolved:

 \begin{enumerate}
%   \item  The value |\textfraction| does not affect the processing
%     of doublecol floats: this seems sensible, but should be
%     documented.
%   \item |\twocolumn| floatplacement was wrong: dbl not needed, ord
%     needed.
%   \item |\@floatplacement| was not called after |\@startdblcol|
%     or |\@topnewpage|.  This has been changed; it is clearly a bug
%     fix.
%   \item The use |\@topnewpage| when |\dblfigrule| is non-trivial
%   produced a rule in the wrong place.  This has been fixed by not
%   using |\dblfigrule| when processing the `float' from
%   |\@topnewpage|. 
%   \item  If the specifier was just h and the float could not be put
%     here, it went on the deferlist and stayed there until a clearpage.
%     It now gets changed to a `th': this is only an error-recovery
%     action, putting just h or !h should be deprecated.   
%   \item |\@dblmaxsep| was `the maximum of |\dblfloatsep| and
%     |\dbltexfloatsep|'. But it was never used!  Now gone completely,
%     like |\@maxsep|.
%   \item After an h float is put on a page, it was counted as text when
%     applying the |\textfraction| test; this is possibly too big a
%     change although it is a bug fix?
%   \item  Two consecutive h floats are separated by twice |\intextsep|:
%     this could be changed to one by use of |\addvspace|, OK?
%     Note that it would also mean that less space is put in if an h
%     float  immediaiely follows other spaces.  This is also possibly
%     too big a change, at least for compatibility mode?
%     Or it may be simply wrong!  It has not been changed.
%   \item Now |\@addtocurcol| checks first for just p fps.  I think
%     that this is an increase in efficiency, but maybe the coding
%     should be made even more efficient.
%   \item |\@tryfcolumn| now tests if the list is empty first, otherwise
%     lots of wasted time!  Thus this test has been removed from
%     |\@startcolumn|.
%     As Frank pointed out, this makes |\@startcolumn| less
%     efficient. But it is now the same as |\@startdblcolumn|: I can
%     see no reason why they should be different, but which is best?
%   \item Why is |\@colroom| set in |\@doclearpage|?
%   \item  Footnotes. Check what |\clearpage| does when footnotes are
%     left over.  Footnotes are not put on float pages and, also,
%     |\@addtonextcol| ignores the existence of held-over footnotes
%     in deciding what floats can go on the page.  Not changed.
%   \item  |\clearpage| can still lose non-boxes, at least when floats
%     are involved.  It also moves some to the `wrong page', but this
%     may be a coding problem.
%   \item  The ! option makes it necessary to check in |\output| that
%     there is enough room left on the page after adding a float.  (This
%     would have been necessary anyway if anyone set |\@textmin| too
%     close to zero!  A similar danger existed also if the text in a
%     |\twocolumn[text]| entity gets too large.)
%     The current implementation of this also makes the normal case a
%     little less efficient, OK?
%     Not enough room means, at present, less than  |\baselineskip|,
%     with a warning: is this OK?  Should it be made generic (another
%     parameter)?
%   \item  There are four possibilities for supporting this:
%
%     |\twocolumn[\maketitle more text]|
%
%     One is to change
%     |\maketitle| slightly to allow this.  Another is to change
%     |\@topnewpage| so that more than one |\twocolumn[]| command is
%     allowed; in this case |\maketitle\twoclumn[more text]| will work.
%     The former is more robust from the user's viewpoint, but makes the
%     code for |\maketitle| rather ad hoc (maybe it is already?).
%     Another is to misuse the global twocolumn flag locally within
%     |\@topnewpage|.
%     Yet another is to move the column count register from the multicol
%     package into the kernel.  This has beeen done.
%   \item  Where should the reinserts be put to maximise the
%     probability that footmotes come out on the correct page?
%     Or should we go for as much compatibility as possible (but see
%     next item)?
%   \item  Should we continue to support (as much as possible)
%     |\samepage|?  Some of its intended functionality is now advertised
%     as being provided by |\enlargethispage|.  Use of either is likely
%     to result in wrongly placed footnotes, marginals, etc.
%     Which should have priority: obeying the pagination instructions,
%     or correct placement of notes/marginalia?
%   \item  Is the adjustment of space to cause shrinking in the
%      kludge-* case correct?  Should it be limited to 0pt?
%   \item  Is the setting of |\boxmaxdepth| in makecol and friends
%     needed?  It only has any effect if |\@textbottom| ends with a box
%     or rule, in which case the vskip to allow for its depth should
%     also be added.  If it is kept, it should probably be the last
%     thing in the box.  It has now been removed.
%     
%     It would perhaps be better to document that |\@textbottom|
%     and |\@texttop| must have natural height 0pt.
%   \item  I cannot see why the vskip adjustement for the depth
%     is needed if boxmaxdepth is used to ensure that there is never
%     a too deep box.
%   \item  The value of |\boxmaxdepth| should be explicitly set
%     whenever necessary: it is too risky to assume that it has any
%     particular value.  Care is needed in deciding what to set it to.
%
%     It is interesting to note that the value of |\boxmaxdepth| is
%     unique in being read before the local settings for the box group
%     are reset; all other parameter settings which affect the box
%     construction use their values outside the box group.
%   \item  Should |\@maxdepth| store the setting of |\maxdepth| from
%     lplain?  Or should we provide a proper interface to class files
%     for setting these? 
% \end{enumerate}
%
% An analysis of various other macros.
%
%    |\@opcol| should do |\@floatplacement|, but where?  Right at the
%    end, since it always occurs at the start of a column.
% \begin{verbatim}
% \def\@opcol{%
%   % Why is this done first?
%   \global \@mparbottom \z@
%   \if@twocolumn
%     \@outputdblcol
%   \else
%     \@outputpage
%     % This is not needed since it is done at the end of
%     %   |\@outputpage|:
%     \global \@colht \textheight
%   \fi}
% \end{verbatim}
%
% Only tracing has been added to these.
%
%    \begin{teX}
\def\@makefcolumn #1{%
  \begingroup
    \@fpmin \z@
    \let \@testfp \@gobble
    \@tryfcolumn #1%
  \endgroup
}
%    \end{teX}
% This will line up the last baselines in the two
% columns provided they are constructed in the normal way: \ie ending
% in a skip of minus the original depth, with |\@textbottom| adding
% nothing. 
%
% Thus again it is essential for |\@textbottom| to have depth 0pt.
% \changes{1.2g}{2000/07/12}{Ensure that rule is in \cs{normalcolor}}
%    \begin{teX}
\def\@outputdblcol{%
  \if@firstcolumn
    \global \@firstcolumnfalse
    \global \setbox\@leftcolumn \box\@outputbox
%<*trace>
    \tr@ce{PAGE: first column boxed}%
%</trace>
  \else
    \global \@firstcolumntrue
    \setbox\@outputbox \vbox {%
                         \hb@xt@\textwidth {%
                           \hb@xt@\columnwidth {%
                             \box\@leftcolumn \hss}%
                           \hfil
                           {\normalcolor\vrule \@width\columnseprule}%
                           \hfil
                           \hb@xt@\columnwidth {%
                             \box\@outputbox \hss}%
                                             }%
                              }%
%<*trace>
    \tr@ce{PAGE: second column also boxed}%
%</trace>
    \@combinedblfloats
    \@outputpage
%<*trace>
    \tr@ce{PAGE: two column page completed}%
%</trace>
    \begingroup
      \@dblfloatplacement
      \@startdblcolumn
%    \end{teX}
%    This loop could be replaced by an |\expandafter| tail
%    recursion in\\ |\@startdblcolumn|.
%    \begin{teX}
      \@whilesw\if@fcolmade \fi
        {\@outputpage
%<*trace>
      \tr@ce{PAGE: double float page completed}%
%</trace>
         \@startdblcolumn}%
    \endgroup
  \fi
}

%    \end{teX}
%
 \subsubsection{Float placement parameters}

%
% The main purpose of this section is to ensure that all the
% float-placement parameters which need to be set in a class file or
% package have been declared.  It also describes their use and sets
% values for them which are reasonable for typical douments using
% US letter or A4 sized paper.
% 
% \paragraph{Limits for the placement of floating objects}
%
% \begin{macro}{\c@topnumber}
%    This counter holds the maximum number of
%    floats that can appear at the top of a text page or column.
%    \begin{teX}
%<*2ekernel|autoload>
\newcount\c@topnumber
\setcounter{topnumber}{2}
%    \end{teX}
% \end{macro}
%
% \begin{macro}{\topfraction}
%    This macro holds the maximum proportion (as a decimal number) of
%    a text page or column that can be occupied by floats at the top.
%    \begin{teX}
\newcommand\topfraction{.7}
%    \end{teX}
% \end{macro}
%
% \begin{macro}{\c@bottomnumber}
%    This counter holds the maximum number of
%    floats that can appear at the bottom of a text page or column.
%    \begin{teX}
\newcount\c@bottomnumber
\setcounter{bottomnumber}{1}
%    \end{teX}
% \end{macro}
%
% \begin{macro}{\bottomfraction}
%    This macro holds the maximum proportion (as a decimal number) of
%    a text page or column that can be occupied by floats at the bottom.
%    \begin{teX}
\newcommand\bottomfraction{.3}
%    \end{teX}
% \end{macro}
%
% \begin{macro}{\c@totalnumber}
%    This counter holds the maximum number of floats that can appear on
%    any text page or column.
%    \begin{teX}
\newcount\c@totalnumber
\setcounter{totalnumber}{3}
%    \end{teX}
% \end{macro}
%
% \begin{macro}{\textfraction}
%    This macro holds the minimum proportion (as a decimal number) of
%    a text page or column that must be occupied by text.
%    \begin{teX}
\newcommand\textfraction{.2}
%    \end{teX}
% \end{macro}
%
% \begin{macro}{\floatpagefraction}
%    This macro holds the minimum proportion (as a decimal number) of
%    a page or column that must be occupied by floating objects before a
%    `float page' is produced.
%    \begin{teX}
\newcommand\floatpagefraction{.5}
%    \end{teX}
% \end{macro}
%
% \begin{macro}{\c@dbltopnumber}
%    This counter holds the maximum number of double-column floats that
%    can appear on the top of a two-column text page.
%    \begin{teX}
\newcount\c@dbltopnumber
\setcounter{dbltopnumber}{2}
%    \end{teX}
% \end{macro}
%
% \begin{macro}{\dbltopfraction}
%    This macro holds the maximum proportion (as a decimal number) of
%    a two-column text page that can be occupied by double-column floats
%    at the top.
%    \begin{teX}
\newcommand\dbltopfraction{.7}
%    \end{teX}
% \end{macro}
%
% \begin{macro}{\dblfloatpagefraction}
%    This macro holds the minimum proportion (as a decimal number) of
%    a page that must be occupied by double-column floating objects
%    before a `double-column float page' is produced.
%    \begin{teX}
\newcommand\dblfloatpagefraction{.5}
%    \end{teX}
% \end{macro}
%
 \paragraph{Floats on a text page}

% \begin{macro}{\floatsep}
% \begin{macro}{\textfloatsep}
% \begin{macro}{\intextsep}
%    When a floating object is placed on a page with text, these
%    parameters control the seperation between the float and the other
%    objects on the page. These parameters are used for both
%    one-column mode and single-column floats in two-column mode.
%    They are all rubber lengths.
%
%    |\floatsep| is the space between adjacent floats that are placed
%    at the top or bottom of the text page or column.
%
%    |\textfloatsep| is the space between the main text and floats
%    at the top or bottom of the page or column.
%
%    |\intextsep| is the space between in-text floats and the text.
%    \begin{teX}
\newskip\floatsep
\newskip\textfloatsep
\newskip\intextsep
\setlength\floatsep    {12\p@ \@plus 2\p@ \@minus 2\p@}
\setlength\textfloatsep{20\p@ \@plus 2\p@ \@minus 4\p@}
\setlength\intextsep   {12\p@ \@plus 2\p@ \@minus 2\p@}
%    \end{teX}
% \end{macro}
% \end{macro}
% \end{macro}
%
% \begin{macro}{\dblfloatsep}
% \begin{macro}{\dbltextfloatsep}
%    When double-column floats (floating objects that span the whole
%    |\textwidth|) are placed at the top of a text page in two-column
%    mode, the separation between the float and the text is controlled
%    by |\dblfloatsep| and |\dbltextfloatsep|.  They are rubber lengths.
%
%    |\dblfloatsep| is the space between adjacent double-column floats
%    placed at the top of the text page.
%
%    |\dbltextfloatsep| is the space between the main text and
%    double-column floats at the top of the page.
%    \begin{teX}
\newskip\dblfloatsep
\newskip\dbltextfloatsep
\setlength\dblfloatsep    {12\p@ \@plus 2\p@ \@minus 2\p@}
\setlength\dbltextfloatsep{20\p@ \@plus 2\p@ \@minus 4\p@}
%    \end{teX}
% \end{macro}
% \end{macro}
%
 \paragraph{Floats on their own page or column}
%
% \begin{macro}{\@fptop}
% \begin{macro}{\@fpsep}
% \begin{macro}{\@fpbot}
%    When floating objects are placed on a seperate page or column,
%    called a `float page', the layout of the page is controlled by
%    these parameters, which are rubber lengths.
%    
%    At the top of the page |\@fptop| is inserted;
%    typically this supplies some stretchable whitespace.
%    At the bottom of the page |\@fpbot| is inserted.
%    Between adjacent floats |\@fpsep| is inserted.
%
%    These parameters are used for all floating objects on a
%    `float page' in one-column mode, and for single-column
%    floats in two-column mode.
%
%    Note that at least one of the two parameters |\@fptop| and
%    |\@fpbot| should contain a |plus ...fil| so as to fill the
%    remaining empty space.
%    \begin{teX}
\newskip\@fptop
\newskip\@fpsep
\newskip\@fpbot
\setlength\@fptop{0\p@ \@plus 1fil}
\setlength\@fpsep{8\p@ \@plus 2fil}
\setlength\@fpbot{0\p@ \@plus 1fil}
%    \end{teX}
% \end{macro}
% \end{macro}
% \end{macro}
%
% \begin{macro}{\@dblfptop}
% \begin{macro}{\@dblfpsep}
% \begin{macro}{\@dblfpbot}
%    Double-column `float pages' in two-column mode use similar
%    parameters.
%    \begin{teX}
\newskip\@dblfptop
\newskip\@dblfpsep
\newskip\@dblfpbot
\setlength\@dblfptop{0\p@ \@plus 1fil}
\setlength\@dblfpsep{8\p@ \@plus 2fil}
\setlength\@dblfpbot{0\p@ \@plus 1fil}
%    \end{teX}
% \end{macro}
% \end{macro}
% \end{macro}
% 
% \begin{macro}{\topfigrule}
% \begin{macro}{\botfigrule}
% \begin{macro}{\dblfigrule}
%    The macros can be used to put in rules between floats and text;
%    whatever they insert should be vertical mode material which takes
%    up zero space.
% \task{CAR}{Add more rules (for Frank in addtocurcol)}
%    \begin{teX}
\let\topfigrule=\relax
\let\botfigrule=\relax
\let\dblfigrule=\relax
%    \end{teX}
% \end{macro}
% \end{macro}
% \end{macro}
% 
% 


     %% LaTeX2e file `./manual/kernel-N-ltlength.tex'
%% generated by the `filecontents' environment
%% from source `photo-book-class-final-test' on 2011/12/21.
%%
\label{kernel:lengths}
\index{LaTeX kernel classes!File n  ltlength.dtx}
\section{File n, lengths and the ltlength.dtx}

This class defines a number of user coomands for manipulating lengths. the code is straightforward. The |\newlength| command allocates a new internal skip register using the |\newskip| command from the allocations,
class.

\let\bs\textbackslash
\index{\bs newlength}\index{\bs setlength}\index{\bs addtolength}\index{\bs settowidth}\index{\bs settoheight}
\index{\bs settodepth}
\medskip
\begin{tabular}{ll}
\verb+\newlength+  &  Declare \#1 to be a new length command.\\
\verb+\setlength+    &  Set the length command, \#1, to the value \#2.\\
|\addtolength| & Increase the value of the length command, \#1, by the value \#2.\\
|\settowidth|   & Set the length, \#1 to the width of a box containing \#2. \\
|\settoheight|  & Set the length, \#1 to the height of a box containing \#2.\\
|\settodepth|   & Set the length, \#1 to the depth of a box containing \#2.\\
|\@settodim|   & internal macro\\
|\@settopoint| & internal macro\\
\end{tabular}
\medskip

\begin{Code}
3 \def\newlength#1{\@ifdefinable#1{\newskip#1}}
4 \def\setlength#1#2{#1#2\relax}
5 \def\addtolength#1#2{\advance#1 #2\relax}
\end{Code}
\medskip

The |setto| commands use a temporary box to store the contents and then measure them using the internal macro |\@settodim|,

\medskip
\begin{Code}
6 \def\@settodim#1#2#3{\setbox\@tempboxa\hbox{{#3}}#2#1\@tempboxa
%  Clear the memory afterwards (which might be a lot).
7       \setbox\@tempboxa\box\voidb@x}
8 \def\settoheight{\@settodim\ht}
9 \def\settodepth {\@settodim\dp}
10 \def\settowidth {\@settodim\wd}
\end{Code}
\medskip

The |\@settopoint| macro takes the contents of the skip register that is supplied as its argument
and removes the fractional part to make it a whole number of points. This can be
used in class files to avoid values like 45.455pt when calulating a dimension. The method of
rounding is interesting. Also it is interesting that this macro, is not used in the kernel at all, but is defined
here for use with the standard classes (it is used to round off dimensions for page calculations).

\medskip
\begin{Code}
11 \def\@settopoint#1{\divide#1\p@\multiply#1\p@}
\end{Code}
\medskip

Example usage:
\medskip

\begin{Code}
\makeatletter
\newlength\test
\setlength\test{19.5pt}
\@settopoint{\test}

\the\test
\makeatother
\end{Code}

produces: \texttt{19pt}.

%  \section{Paragraphs \texttt{File h: ltpar.dtx}}

\index{LaTeX kernel classes!\texttt{File h ltpar.dtx}}
\begin{multicols}{2}
This section of the kernel declares the commands used to set |\par| and |\everypar|
when ever their function needs to be changed for a long time.

As the kernel authors note, There are two situations in which |\par| may be changed:

\begin{enumerate}[1.]
\item  Long-term changes, in which the new value is to remain in effect until the
current environment is left. The environments that change |\par| in this way
are the following:
 All list environments (itemize, quote, etc.)
 Environments that turn |\par| into a noop: tabbing, array and tabular.

\item Temporary changes, in which |\par| is restored to its previous value the next
time it is executed. The following are all such uses.
|\end| when preceded by |\@endparenv|, which is called by |\endtrivlist|

 The mechanism for avoiding page breaks and getting the spacing right
after section heads.
\end{enumerate}

\Paragraph{\textbackslash @setpar} This initializes a long-term change to |\par|. The default definition of |\@par| will ensure that if |\@restorepar| defines |\par|
to execute \@par it will redefine itself to the primitive |\@@par| after one iteration.

\begin{Code}
3 \def\@setpar#1{\def\par{#1}\def\@par{#1}}
4 \def\@par{\let\par\@@par\par}
\end{Code}

\Paragraph{\textbackslash @restorepar} Restore from a short-term change to |\par|.

\begin{Code}
6 \def\@restorepar{\def\par{\@par}}
\end{Code}
\end{multicols}

     \input{./sections/kernel-ltspace}
     \chapter{ltxref.dtx}

 \section{Cross Referencing}
  The user writes  |\label|\marg{foo}  to define the following
  cross-references:

   |\ref|\marg{foo}: value of most recently incremented referencable
             counter. in the current environment. (Chapter, section,
             theorem and enumeration counters counters are
             referencable, footnote counters are not.)

   |\pageref|\marg{foo}: page number at which |\label{foo}|  command
             appeared.  where  foo  can be any string of characters not
             containing  `|\|', `|{|' or `|}|'.

  Note: The scope of the |\label| command is delimited by environments,
  so\\
  |\begin{theorem} \label{foo} ... \end{theorem} \label{bar}|\\
  defines |\ref{foo}| to be the theorem number and |\ref{bar}| to be
  the current section number.

  Note: |\label| does the right thing in terms of spacing -- i.e.,
  leaving a space on both sides of it is equivalent to leaving
  a space on either side.



 \subsection{Cross Referencing}
%
%    \begin{teX}
\message{x-ref,}
%    \end{teX}
%
  This is implemented as follows.  A referencable counter  CNT  is
  incremented by the command  \cs{refstepcounter}\marg{cnt} , which sets
  \cs{@currentlabel} == \marg{CNT}\marg{eval(\cs{p@cnt}\cs{theCNT})}.   The command
  \cs{label}\marg{FOO} then writes the following on file \cs{@auxout} :
        \cs{newlabel}\marg{FOO}\{{eval(\@currentlabel)}{eval(\thepage)}}}|

\end{document}
%  \ref{FOO} ==
%    BEGIN
%      if \r@foo undefined
%        then  @refundefined := G T
%              ??
%              Warning: 'reference foo on page ... undefined'
%        else  \@car \eval(\r@FOO)\@nil
%      fi
%    END
%
%  \pageref{foo} =
%    BEGIN
%      if \r@foo undefined
%        then  @refundefined := G T
%              ??
%              Warning: 'reference foo on page ... undefined'
%        else  \@cdr \eval(\r@FOO)\@nil
%      fi
%    END
%




%
%  \begin{macro}{\G@refundefinedtrue}
% \changes{v1.1i}{1995/12/07}{Renamed (back) from \cs{G@refundefined}}
%  \begin{macro}{\@refundefined}
% \changes{v1.1h}{1995/10/24}{Switch for refundefined replaced}
%    This does not save on name-space (since \cs{G@refundefinedfalse}
%    was never needed) but it does make the implmentation of such
%    one-way switches more consistent. The extra macro to make the
%    change is used since this change appears several times.
%
%    \textbf{Note} despite its name, |\G@refundefinedtrue| does
%    \emph{not} correspnd to an |\if| command, and there is no
%    matching \ldots|false|. It would be more natural to call the
%    command |\G@refundefined| (as inspection of the change log will
%    reveal) but unfortunately such a change would break any package
%    that had defined a  |\ref|-like command that mimicked the
%    definition of |\ref|, calling |\G@refundefinedtrue|. Inspection
%    of the \TeX\ archives revealed several such packages, and so this
%    command has been named \ldots|true| so that the definition of
%    |\ref| need not be changed, and the packages will work without
%    change.
%    \begin{teX}
% \newif\ifG@refundefined
% \def\G@refundefinedtrue{\global\let\ifG@refundefined\iftrue}
% \def\G@refundefinedfalse{\global\let\ifG@refundefined\iffalse}
\def\G@refundefinedtrue{%
  \gdef\@refundefined{%
    \@latex@warning@no@line{There were undefined references}}}
\let\@refundefined\relax
%    \end{teX}
%  \end{macro}
%  \end{macro}
%
%    \begin{teX}
%    \end{teX}
%  \begin{macro}{\ref}
% \changes{LaTeX2e}{1993/12/11}{Macro reimplemented}
%  \begin{macro}{\pageref}
% \changes{LaTeX2e}{1993/12/11}{Macro reimplemented}
%  \begin{macro}{\@setref}
% \changes{LaTeX2e}{1993/12/11}{Macro added}
% \changes{v1.1h}{1995/10/24}{Switch for refundefined renamed}
% \changes{v1.1i}{1995/12/07}{Switch for refundefined restored}
%    Referencing a |\label|.
% RmS 91/10/25: added a few extra |\reset@font|,
%               as suggested by Bernd Raichle
%               
% RmS 92/08/14: made |\ref| and |\pageref| robust
% 
% RmS 93/09/08: Added setting of refundefined switch.
%    \begin{teX}
\def\@setref#1#2#3{%
  \ifx#1\relax
   \protect\G@refundefinedtrue
   \nfss@text{\reset@font\bfseries ??}%
   \@latex@warning{Reference `#3' on page \thepage \space
             undefined}%
  \else
   \expandafter#2#1\null
  \fi}
\def\ref#1{\expandafter\@setref\csname r@#1\endcsname\@firstoftwo{#1}}
\def\pageref#1{\expandafter\@setref\csname r@#1\endcsname
                                   \@secondoftwo{#1}}
%    \end{teX}
%  \end{macro}
%  \end{macro}
%  \end{macro}
%
%
%  \begin{macro}{\newlabel}
% \changes{v1.1b}{1994/05/21}{Use new warning commands}
% \changes{v1.1e}{1995/04/24}{Make \cs{@onlypreamble} for /1388.}
% \changes{v1.1e}{1995/06/19}
%      {Use \cs{@newl@bel} to share code with \cs{bibcite}}
% \changes{v1.1g}{1995/07/14}
%   {Remove \cs{@onlypreamble} so still defined in new \cs{enddocument}}
%    This command will be written to the \texttt{.aux} file to
%    pass label information from one run to another.
%  \begin{macro}{\@newl@bel}
%    The internal form of |\newlabel| and |\bibcite|. Note that this
%    macro does it's work inside a group. That way the local
%    assignments it needs to do don't clutter the save stack. This
%    prevents large documents with many labels to run out of save
%    stack.
% \changes{v1.1h}{1995/10/24}{Switch for multiplelabels replaced by
%    inline code}
% \changes{v1.1k}{2001/02/16}{Added an extra grouplevel (PR3250), jlb}
%    \begin{teX}
\def\@newl@bel#1#2#3{{%
  \@ifundefined{#1@#2}%
    \relax
    {\gdef \@multiplelabels {%
       \@latex@warning@no@line{There were multiply-defined labels}}%
     \@latex@warning@no@line{Label `#2' multiply defined}}%
  \global\@namedef{#1@#2}{#3}}}
%    \end{teX}
%  
%    \begin{teX}
\def\newlabel{\@newl@bel r}
%    \end{teX}
%  
%    \begin{teX}
\@onlypreamble\@newl@bel
%    \end{teX}
%  \end{macro}
%  \end{macro}
%
%  \begin{macro}{\if@multiplelabels}
% \changes{v1.1h}{1995/10/24}{Macro removed}
%  \begin{macro}{\@multiplelabels}
% \changes{v1.1h}{1995/10/24}{Switch for multiplelabels removed}
%    This is redefined to produce a warning if at least one label is
%    defined more than once. It is executed by the |\enddocument|
%    command.
%    \begin{teX}
\let \@multiplelabels \relax
%    \end{teX}
%  \end{macro}
%  \end{macro}
%
%  \begin{macro}{\label}
% \changes{v1.1d}{1994/11/04}{(ASAJ)Added \cs{protected@write}}
%  \begin{macro}{\refstepcounter}
% \changes{v1.1d}{1994/11/04}{(ASAJ)Added \cs{protected@edef}}
%    The commands |\label| and |\refstepcounter| have been changed to
%    allow |\protect|'ed commands to work properly.  For example,
%\begin{verbatim}
%   \def\thechapter{\protect\foo{\arabic{chapter}.\roman{section}}}
%\end{verbatim}
%    will cause a |\label{bar}| command to define |\ref{bar}| to expand
%    to something like |\foo{4.d}|.  Change made 20 Jul 88.
%
%    \begin{teX}
\def\label#1{\@bsphack
  \protected@write\@auxout{}%
         {\string\newlabel{#1}{{\@currentlabel}{\thepage}}}%
  \@esphack}
%    \end{teX}
%
%    \begin{teX}
\def\refstepcounter#1{\stepcounter{#1}%
    \protected@edef\@currentlabel
       {\csname p@#1\endcsname\csname the#1\endcsname}%
}
%    \end{teX}
%  \end{macro}
%  \end{macro}
%
%
%  \begin{macro}{\@currentlabel}
% For |\label| commands that come before any environment
%
%    \begin{teX}
\def\@currentlabel{} 
%    \end{teX}
%  \end{macro}
%
%    \begin{teX}
%</2ekernel>
%    \end{teX}
%
% \subsection{An extension of counter referencing}
%
%
% At the moment a reference to a counter |foo| will generate the
% equivalent of |\p@foo\thefoo| although not quite in this form.  For
% some applications it would be nice of one could have |\thefoo| being
% an argument to |\p@foo| to be able to put material before and after
% the number generated by |\thefoo|. This can be easily achieved with
% a small change to one of the kernel commands as follows:
%
%\begin{verbatim}
%\def\refstepcounter#1{\stepcounter{#1}%
%    \protected@edef\@currentlabel
%       {\csname p@#1\expandafter\endcsname\csname the#1\endcsname}%
%}
%\end{verbatim}
%
% The trick is to ensure that |\csname the#1\endcsname| is turned into
% a single token before |\p@...| is expanded further. This way, if the
% |\p@...| command is a macro with one argument it will receive
% |\the...|. With the kernel code (i.e., without the |\expandafter|)
% it will instead pick up |\csname| which would be disastrous.
%
% Using |\expandafter| instead of braces delimiting the argument is
% better because, assuming that the |\p@...| command is not defined as
% a macro with one argument, the braces will stay and prohibit kerning
% that might otherwise happen between the glyphs generated by
% |\the...| and surrounding glyphs.
%
% We have refrained from making this change in the kernel code
% although for exisiting documents it would be 100\% backward
% compatible. The reason being that any class or package making use of
% this functionality would then horribly fail with older \LaTeX{}
% installations.
%
% Instead we suggest that people who are interested in using this
% functionality in a document class or package add the redefinition to
% the class file. To ensure that this redefinition is properly applied
% they might want to test for the original definition first, e.g.
%
%\begin{verbatim}
%\CheckCommand*\refstepcounter[1]{\stepcounter{#1}%
%    \protected@edef\@currentlabel
%       {\csname p@#1\endcsname\csname the#1\endcsname}%
%}
%\renewcommand*\refstepcounter[1]{\stepcounter{#1}%
%    \protected@edef\@currentlabel
%       {\csname p@#1\expandafter\endcsname\csname the#1\endcsname}%
%}
%\end{verbatim}
%
% \Finale
%

     
\chapter{lterror.dtx}
\label{kernel:lterror}

 \section{Error handling}

 This section defines \LaTeXe's error commands. Most of the error
 messages are defined here, although now and then some error
 messages are defined in other |.dtx| files.

 The `2ekernel' code ensures that a |\usepackage{autoerr}| is
 essentially ignored if a `full' format is being used that has
 the error messages already in the format.
    \begin{teX}
    \expandafter\let\csname ver@autoerr.sty\endcsname\fmtversion
    \end{teX}


 \subsection{General commands}

 \begin{docCommand}{MessageBreak}{}
    This command prints a new-line inside a message, followed by a 
    continuation line beginning with |\@msg@continuation|.  Normally it is 
    defined to be |\relax|, but inside messages, it is let to 
    \refCom{@message@break} for example in the definition of \refCom{GenericWarning}.
 \end{docCommand}
    \begin{teX}
\let\MessageBreak\relax
    \end{teX}
  

 \begin{docCommand}{GenericInfo}{}
    This takes two arguments: a continuation and a message, and sends 
    the result to the log file.
    \begin{teX}
\DeclareRobustCommand{\GenericInfo}[2]{%
   \begingroup
      \def\MessageBreak{^^J#1}%
      \set@display@protect
      \immediate\write\m@ne{#2\on@line.}%
   \endgroup
}
    \end{teX}
 \end{docCommand}



 \begin{docCommand}{GenericWarning}{}
    This takes two arguments: a continuation and a message, and sends 
    the result to the screen.
    \begin{teX}
\DeclareRobustCommand{\GenericWarning}[2]{%
   \begingroup
      \def\MessageBreak{^^J#1}%
      \set@display@protect
      \immediate\write\@unused{^^J#2\on@line.^^J}%
   \endgroup
}
    \end{teX}
  \end{docCommand}

 \begin{docCommand}{GenericError}{}
 This macro takes four arguments: a continuation,
 an error message, where to go for further information, and the help
 information.  It displays the error message, and sets the error help
 (the result of typing |h| to the prompt), and does a horrible hack
 to turn the last context line (which by default is the only context
  line) into just three dots.  This could be made more efficient.

    \begin{teX}
\def\GenericError{\@autoerr\GenericError}

\bgroup
\lccode`\@=`\ %
\lccode`\~=`\ %
\lccode`\}=`\ %
\lccode`\{=`\ %
\lccode`\T=`\T%
\lccode`\H=`\H%
\catcode`\ =11\relax%
\lowercase{%
\egroup%
    \end{teX}

 Unfortunately \TeX\ versions older than 3.141 have a bug which means
 that |^^J| does not force a linebreak in |\message| and |\errmessage|
 commands. So for these old \TeX's we use |\typeout| to produce the
 message, and then have an empty |\errmessage| command. This causes an
 extra line of the form
\begin{verbatim}
 ! .
\end{verbatim}
 To appear on the terminal, but if you do not like it, you can always
 upgrade your \TeX! In order for your format to use this version, you
 must define the macro |\@TeXversion| to be the version number, e.g.,
 3.14 of the underlying \TeX. See the comments in
 \texttt{ltdircheck.dtx}.
    \begin{teX}
\dimen@\ifx\@TeXversion\@undefined4\else\@TeXversion\fi\p@%
\ifdim\dimen@>3.14\p@%
    \end{teX}

 First the `standard case'.
    \begin{teX}
\DeclareRobustCommand{\GenericError}[4]{%
\begingroup%
\immediate\write\@unused{}%
\def\MessageBreak{^^J}%
\set@display@protect%
\edef%
%    %<-------------------do not delete this space!------------------->%
\@err@                                                                 %
{{#4}}%
\errhelp
%    %<-------------------do not delete this space!------------------->%
\@err@                                                                 %
\let
%    %<-------------------do not delete this space!------------------->%
\@err@                                                                 %
\@empty
\def\MessageBreak{^^J#1}%
\def~{\errmessage{%
#2.^^J^^J%
#3^^J%
Type  H <return>  for immediate help%
%    %<-------------------do not delete this space!------------------->%
\@err@                                                                 %
}}%
~%
\endgroup}%
    \end{teX}

    \begin{teX}
\else%
    \end{teX}

 Secondly the version for old \TeX's.
    \begin{teX}
\DeclareRobustCommand{\GenericError}[4]{%
\begingroup%
\immediate\write\@unused{}%
\def\MessageBreak{^^J}%
\set@display@protect%
\edef%
    %<-------------------do not delete this space!------------------->%
\@err@                                                                 %
{{#4}}%
\errhelp
    %<-------------------do not delete this space!------------------->%
\@err@                                                                 %
\let
    %<-------------------do not delete this space!------------------->%
\@err@                                                                 %
\errmessage
\def\MessageBreak{^^J#1}%
\def~{\typeout{! %
#2.^^J^^J%
#3^^J%
Type  H <return>  for immediate help.}%
    %<-------------------do not delete this space!------------------->%
\@err@                                                                 %
{}}%
~%
\endgroup}%
    \end{teX}

    \begin{teX}
\fi}%

    \end{teX}
 \end{docCommand}

\subsection{Package and Class error messages}
 
 
 
 
 \begin{docCommand}{PackageWarningNoLine}{}
 \end{docCommand}
 
 \begin{docCommand}{PackageInfo}{}
 \end{docCommand}
 
 \begin{docCommand}{ClassError}{}
  \end{docCommand}

 \begin{docCommand}{ClassWarning}{}
 \begin{docCommand}{ClassWarningNoLine}{}
 \begin{docCommand}{ClassInfo}{}
  These commands are intended for use by package and class writers, to
  give information to authors.  The syntax is:
    \begin{quote}
       |\PackageError{|\meta{package}|}{|^^A
          \meta{error}|}{|\meta{help}|}| \\
       |\PackageWarning{|\meta{package}|}{|\meta{warning}|}| \\
       |\PackageWarningNoLine{|\meta{package}|}{|\meta{warning}|}| \\
       |\PackageInfo{|\meta{package}|}{|\meta{info}|}|
    \end{quote}
    and similarly for classes.  The |Error| commands print the
    \meta{error} message, and present the interactive prompt; if the
    author types |h|, then the \meta{help} information is displayed.
    The |Warning| commands produce a warning but do not present the
    interactive prompt.  The |WarningNoLine| commands do the same,
    but don't print the input line number.  The |Info| commands write
    the message to the
    |log| file.  Within the messages, the command 
    |\MessageBreak| can be used to
    break a line, |\protect| can be used to protect command names,
    and |\space| is a space, for example:
 \begin{verbatim}
    \newcommand{\foo}{FOO}
    \PackageWarning{ethel}{%
       Your hovercraft is full of eels,\MessageBreak
       and \protect\foo\space is \foo}
 \end{verbatim}
    produces:
 \begin{verbatim}
    Package ethel warning: Your hovercraft is full of eels,
    (ethel)                and \foo is FOO on input line 54.
 \end{verbatim}
 \end{docCommand}
 \end{docCommand}
 \end{docCommand}
 
 The definition of the commands are fairly straight forward.  They all use the \refCmd{GenericError} command to display the error on the terminal. The new
 expl3 module for \latex3e, has expanded on this work tremendously, so that
 there is no real need for using these macros any longer.
  
\begin{docCommand}{PackageError}{ \marg{package} \marg{error} \marg{help}}
 \end{docCommand}
    \begin{teX}
\gdef\PackageError#1#2#3{%
   \GenericError{%
      (#1)\@spaces\@spaces\@spaces\@spaces
   }{%
      Package #1 Error: #2%
   }{%
      See the #1 package documentation for explanation.%
   }{#3}%
}
    \end{teX}
    
    
\begin{docCommand}{PackageWarning}{ \meta{package} \meta{warning}}
 \end{docCommand}
    \begin{teX}
\def\PackageWarning#1#2{%
   \GenericWarning{%
      (#1)\@spaces\@spaces\@spaces\@spaces
   }{%
      Package #1 Warning: #2%
   }%
}
\def\PackageWarningNoLine#1#2{%
   \PackageWarning{#1}{#2\@gobble}%
}
\def\PackageInfo#1#2{%
   \GenericInfo{%
      (#1) \@spaces\@spaces\@spaces
   }{%
      Package #1 Info: #2%
   }%
}
    \end{teX}

    \begin{teX}

\gdef\ClassError#1#2#3{%
   \GenericError{%
      (#1) \space\@spaces\@spaces\@spaces
   }{%
      Class #1 Error: #2%
   }{%
      See the #1 class documentation for explanation.%
   }{#3}%
}
    \end{teX}

    \begin{teX}
\def\ClassWarning#1#2{%
   \GenericWarning{%
      (#1) \space\@spaces\@spaces\@spaces
   }{%
      Class #1 Warning: #2%
   }%
}
\def\ClassWarningNoLine#1#2{%
   \ClassWarning{#1}{#2\@gobble}%
}
\def\ClassInfo#1#2{%
   \GenericInfo{%
      (#1) \space\space\@spaces\@spaces
   }{%
      Class #1 Info: #2%
   }%
}
    \end{teX}



Some errors are marked to be generated only in the \latex2e kernel. This
is where we see the reference to the LaTeX Manual or the LaTeX Companion
for help is mentioned.
 \begin{docCommand}{@latex@error}{}
 \begin{docCommand}{@latex@warning}{}
 \begin{docCommand}{@latex@warning@no@line}{}
 \begin{docCommand}{@latex@info}{}
 \begin{docCommand}{@latex@info@no@line}{}
     \begin{teX}
\gdef\@latex@error#1#2{%
   \GenericError{%
      \space\space\space\@spaces\@spaces\@spaces
   }{%
      LaTeX Error: #1%
   }{%
      See the LaTeX manual or LaTeX Companion for explanation.%
   }{#2}%
}
    \end{teX}

    \begin{teX}
\def\@latex@warning#1{%
   \GenericWarning{%
      \space\space\space\@spaces\@spaces\@spaces
   }{%
      LaTeX Warning: #1%
   }%
}
    \end{teX}

    \begin{teX}
\def\@latex@warning@no@line#1{%
   \@latex@warning{#1\@gobble}}
    \end{teX}

    \begin{teX}
\def\@latex@info#1{%
   \GenericInfo{%
      \@spaces\@spaces\@spaces
   }{%
      LaTeX Info: #1%
   }%
}
    \end{teX}

    \begin{teX}
\def\@latex@info@no@line#1{%
  \@latex@info{#1\@gobble}}
    \end{teX}

    |\@font@warning| and |\@font@info| are defined later since they
    have to be redefined by the \texttt{tracefnt} package.
\begin{verbatim}
\def\@font@warning#1{%
   \GenericWarning{%
      {(font)\@spaces\@spaces}%
      {Font Warning: #1}%
 }
\def\@font@info#1{%
   \GenericInfo{%
      (font)\space\@spaces
   }{%
      Font Info: #1%
   }%
 }
\end{verbatim}
 \end{docCommand}
 \end{docCommand}
 \end{docCommand}
 \end{docCommand}
 \end{docCommand}

 \begin{docCommand}{c@errorcontextlines}{}
 \changes{LaTeX2e}{1993/11/22}{Macro added}
  |\errorcontextlines| as a \LaTeX\ counter, so that it may be be
  manipulated with |\setcounter| (once it is defined :-)
    \begin{teX}
\let\c@errorcontextlines\errorcontextlines
\c@errorcontextlines=-1
    \end{teX}
 \end{docCommand}

 \changes{v1.0d}{1994/03/28}
     {Remove test for \cs{inputlineno} undefined.}
 \begin{docCommand}{on@line}{}
    The message ` on input line~$n$', if possible.
    \begin{teX}
\ifnum\inputlineno=\m@ne
  \let\on@line\@empty
\else
  \def\on@line{ on input line \the\inputlineno}
\fi
    \end{teX}
 \end{docCommand}

  \begin{docCommand}{@warning}{}
  \begin{docCommand}{@@warning}{}
  \begin{docCommand}{@latexerr}{}
     Older \LaTeX{} messages.  For the moment, these
     |\let| to the new message commands.  They may be changed later,
     once only obsolete packages and classes contain them.
 \changes{v1.0b}{1993/12/03}{Set \cs{c@errorcontextlines} to -1}
 \changes{v1.0e}{1993/04/09}{Mention The Companion}
 \changes{v1.0f}{1993/04/11}{Remove setting of errorcontextlines}
 \changes{v1.0k}{1994/05/01}{(CAR) Added draft \cs{@latexinfo}.}
 \changes{v1.0n}{1994/05/10}{(ASAJ) Added extra blank lines to
           \cs{@latexerr}.}
 \changes{v1.0o}{1994/05/11}
     {(ASAJ) Removed one of the extra blank lines to \cs{@latexerr}.}
    \begin{teX}
\let\@warning\@latex@warning
\let\@@warning\@latex@warning@no@line
</2ekernel|autoload>
\global\let\@latexerr\@latex@error
    \end{teX}
  \end{docCommand}
  \end{docCommand}
  \end{docCommand}

 \begin{docCommand}{@spaces}{}
    Four spaces.
    \begin{teX}
\def\@spaces{\space\space\space\space}
    \end{teX}
 \end{docCommand}

 \subsection{Specific errors}

 The four commands
\docAuxCmd{@eha},
\docAuxCmd{@ehb},
\docAuxCmd{@ehc}, 
and \docAuxCmd{@ehd}, 
 expand to specific errors.   The more common error help messages are defined by macros named as a series of |\@eh|\meta{letter}. Obviously short for errorhelp
    \begin{teX}
\gdef\@eha{%
  Your command was ignored.\MessageBreak
  Type \space I <command> <return> \space to replace it %
  with another command,\MessageBreak
  or \space <return> \space to continue without it.}
\gdef\@ehb{%
  You've lost some text. \space \@ehc}
\gdef\@ehc{%
  Try typing \space <return> %
  \space to proceed.\MessageBreak
  If that doesn't work, type \space X <return> \space to quit.}
\gdef\@ehd{%
  You're in trouble here.  \space\@ehc}
    \end{teX}
 As |\latex@error| triggers the autoload, these definitions
 should not be needed in the autoload format, but just to be safe\ldots
    \begin{teX}
%<*autoload>
\let\@eha\@empty\let\@ehb\@empty\let\@ehc\@empty\let\@ehd\@empty
%</autoload>
    \end{teX}


 Here are most of the error message-generating commands of \LaTeX.
 \begin{docCommand}{@autoerr}{}
 Make this autoload command robust, as it may be read in at
 unpredicatble times.
    \begin{teX}
<autoload>\def\@autoerr{\protect\@autoload{err}\protect}
    \end{teX}
 \end{docCommand}

 \begin{docCommand}{@notdefinable}{}
    Error message generated in |\@ifdefinable| from calls
    to one of the commands |\newcommand|, |\newlength| or |\newtheorem|
    specifying an already-defined command name or one that begins
    |\end...|.
 \changes{v1.2n}{1998/05/28}{Added message re `end...' pr/1555}
    \begin{teX}
\gdef\@notdefinable{%
    \end{teX}
 \end{docCommand}

 \begin{docCommand}{@nolnerr}{}
 Generated by |\newline| and |\\| when called in vertical mode.
    \begin{teX}
\gdef\@nolnerr{%
<!autoload>  \@latex@error{There's no line here to end}\@eha}
<autoload>  \@autoerr\@nolnerr}
    \end{teX}
 \end{docCommand}

 \begin{docCommand}{@nocounterr}{}
  Generated by |\setcounter|, |\addtocounter| or
  |\newcounter| if applied to an undefined counter \meta{cnt}.

 \begin{docCommand}{@nocnterr}{}
 Obsolete error message generated in \LaTeX2.09 by
 |\setcounter|, |\addtocounter| or |\newcounter|
 for undefined counter.
 DO NOT use for \LaTeXe\ it MIGHT vanish!
 Use |\@nocounterr|\marg{cnt} instead.
    \begin{teX}
\gdef\@nocounterr#1{%
<!autoload>  \@latex@error{No counter '#1' defined}\@eha}
<autoload>  \@autoerr\@nocounterr}
\gdef\@nocnterr{\@nocounterr?}
    \end{teX}
 \end{docCommand}
 \end{docCommand}

 \begin{docCommand}{@ctrerr}{}
 Called when trying to print the value of a counter
 numbered by letters that's greater than 26.
    \begin{teX}
\gdef\@ctrerr{%
<!autoload>  \@latex@error{Counter too large}\@ehb}
<autoload>  \@autoerr\@ctrerr}
    \end{teX}
 \end{docCommand}


 \begin{docCommand}{@nodocument}
 Error produced if paragraphs are typeset in the preamble.
 \changes{v1.2m}{1996/11/04}{Always define \cs{@nodocument}
           in kernel, so that it can be cleared by \cs{document}.}
    \begin{teX}
<!def>\gdef\@nodocument{%
<!def>  \@latex@error{Missing \protect\begin{document}}\@ehd}
    \end{teX}
 \end{docCommand}

 \begin{docCommand}{@badend}{}
 Called by |\end| that doesn't match its |\begin|.
 RmS 1992/08/24: added code to |\@badend| to display position of
               non-matching |\begin|.
 FMi 1993/01/14: missing space added.
    \begin{teX}
\gdef\@badend#1{%
<!autoload>  \@latex@error{\protect\begin{\@currenvir}\@currenvline
<!autoload>                     \space ended by \protect\end{#1}}\@eha}
<autoload>  \@autoerr\@badend}
    \end{teX}
 \end{docCommand}

 \begin{docCommand}{@badmath}{}
 Called by |\[|, |\]|, |\(| or |\)| when used in wrong mode.
    \begin{teX}
\gdef\@badmath{%
<!autoload>  \@latex@error{Bad math environment delimiter}\@eha}
<autoload>  \@autoerr\@badmath}
    \end{teX}
 \end{docCommand}

 \begin{docCommand}{@toodeep}{}
 Called by a list environment nested more than six levels
 deep, or an enumerate or itemize nested more than four levels.
    \begin{teX}
\gdef\@toodeep{%
<!autoload>  \@latex@error{Too deeply nested}\@ehd}
<autoload>  \@autoerr\@toodeep}
    \end{teX}
 \end{docCommand}

 \begin{docCommand}{@badpoptabs}{}
 Called by |\endtabbing| when not enough |\poptabs| have
 occurred, or by |\poptabs| when too many have occurred.
    \begin{teX}
\gdef\@badpoptabs{%
<!autoload>  \@latex@error{\protect\pushtabs\space and \protect\poptabs
<!autoload>      \space don't match}\@ehd}
<autoload>  \@autoerr\@badpoptabs}
    \end{teX}
 \end{docCommand}

 \begin{docCommand}{@badtab}{}
 Called by |\>|, |\+| , |\-| or |\<| when stepping to an undefined tab.
    \begin{teX}
\gdef\@badtab{%
<!autoload> \@latex@error{Undefined tab position}\@ehd}
<autoload>  \@autoerr\@badtab}
    \end{teX}
 \end{docCommand}

 \begin{docCommand}{@preamerr}{}
 \changes{v1.2k}{1995/10/24}
         {Modify autoload support}
    This error is special: it appears in places where we normally have
    to |\protect| expansions. However, to prevent a protection of
    the error message itself (which would result in the message
    getting printed not issued on the terminal) we need to locally
    reset |\protect| to |\relax|.
    \begin{teX}
\gdef\@preamerr#1{%
  \begingroup
    \let\protect\relax
<*!autoload>
    \@latex@error{\ifcase #1 Illegal character\or
     Missing @-exp\or Missing p-arg\fi\space
     in array arg}\@ehd
</!autoload>
<autoload>  \@autoerr\@preamerr{#1}%
  \endgroup}
    \end{teX}
 \end{docCommand}

 \begin{docCommand}{@badlinearg}{}
 Occurs in |\line| and |\vector| command when a bad slope
 argument is encountered.
    \begin{teX}
\gdef\@badlinearg{%
<!autoload>  \@latex@error{%
<!autoload>       Bad \protect\line\space or \protect\vector 
<!autoload>       \space argument}\@ehb}
<autoload>  \@autoerr\@badlinearg}
    \end{teX}
 \end{docCommand}

 \begin{docCommand}{@parmoderr}{}
 Occurs in a float environment or a |\marginpar| when
 encountered in inner vertical mode.
    \begin{teX}
\gdef\@parmoderr{%
<!autoload>  \@latex@error{Not in outer par mode}\@ehb}
<autoload>  \@autoerr\@parmoderr}
    \end{teX}
 \end{docCommand}

 \begin{docCommand}{@fltovf}{}
 Occurs in float environment or |\marginpar| when there
 are no more free boxes for storing floats.
    \begin{teX}
\gdef\@fltovf{%
<!autoload>  \@latex@error{Too many unprocessed floats}\@ehb}
<autoload>  \@autoerr\@fltovf}
    \end{teX}
 \end{docCommand}

 \begin{docCommand}{@latexbug}{}
 Occurs in output routine.  This is bad news.
    \begin{teX}
\gdef\@latexbug{%
<!autoload>  \@latex@error{This may be a LaTeX bug}{Call for help}}
<autoload>  \@autoerr\@latexbug}
    \end{teX}
 \end{docCommand}

 \begin{docCommand}{@badcrerr}{}
    This error was removed and replaced by |\@nolnerr|.
 \changes{v1.0m}{1994/05/04}{Error message removed}
    \begin{teX}
\def\@badcrerr {\@latex@error{Bad use of \protect\\}\@ehc}
    \end{teX}
 \end{docCommand}

 \begin{docCommand}{@noitemerr}{}
 |\addvspace| or |\addpenalty| was called when not in
  vmode. Probably caused by a missing |\item|.
    \begin{teX}
\gdef\@noitemerr{%
<!autoload>  \@latex@error{Something's wrong--perhaps a missing %
<!autoload>      \protect\item}\@ehc}
<autoload>  \@autoerr\@noitemerr}
    \end{teX}
 \end{docCommand}

 \begin{docCommand}{@notprerr}{}
 A command that can be used only in the preamble
 appears after the command |\begin{document}|.
    \begin{teX}
\gdef\@notprerr{%
<!autoload>  \@latex@error{Can be used only in preamble}\@eha}
<autoload>  \@autoerr\@notprerr}
    \end{teX}
 \end{docCommand}

  \begin{docCommand}{@inmatherr}{}
 \changes{v1.0j}{1994/04/28}{Macro added}
 \changes{v1.1c}{1994/04/28}{Replaced \cs{noexpand} with \cs{protect}.}
    Issued by commands that don't work correctly within math (like
    |\item|). There is no real error recovery happening, e.g., the
    user might get additional errors afterwards.
    \begin{teX}
\gdef\@inmatherr#1{%
   \relax
   \ifmmode
<!autoload> \@latex@error{Command \protect#1 invalid in math mode}\@ehc
<autoload>  \@autoerr\@inmatherr#1%
   \fi}
    \end{teX}
  \end{docCommand}
%
% \begin{docCommand}@invalidchar}
% \changes{LaTeX2.09}{1993/09/19}
%     {(RmS) Error message for invalid input characters.}
% \changes{v1.0d}{1994/03/28}
%     {(DPC) Comment out (use catcode15 instead)}
%    An error for use with invalid characters.  This is commented
%    out, since we decided to use chatcode 15 instead.
%    \begin{teX}
%\def\@invalidchar{\@latex@error{Invalid character in input}\@ehc}
%    \end{teX}
% \end{docCommand}
 
 As well as the above error commands some error messages are directly
 coded to save space. The Messages alrerady present in \LaTeX2.09
 included: 

 |Environment --- undefined|\\
 Issued by |\begin| for undefined environment.

 |tab overflow|\\
  Occurs in |\= when| maximum number of tabs exceeded.

 |\< in mid line|\\
 Occurs in |\<| when it appears in middle of line.

 |Float(s) lost|\\
 In output routine, caused by a float environment or
 |\marginpar| occurring in inner vertical mode.



      \chapter{LaTeXe Kernel Definitions Module}
 \normalfont
\label{ch:ltxdefinitions}

\section{Introduction}

 This section contains a number of commands used in defining other macros, as well as some useseful commands,
 that can be used by package authors. Some of these commands such as \cmd{\@height} were defined to
 save tokens and hence memory tokens and by now they do not add much value to a modern \latexe installation.
 However, removing them would break backward compatibility, but perhaps in your own packages you may choose
 to improve the program readability by not using them.
\medskip

 \begin{tabular}{ll}
 \refCom{two@digits} & prefix a number less than 10 with `0’\\
 |\typeout|   & display something on the terminal\\
 |\newlinechar| & newline character\\
 |\@height| & height\\
 |\@width|  & width\\
 |\@depth|  &depth\\
 \end{tabular}

  \medskip  

\begin{docCommand}{two@digits}{\marg{number}}
Prefix a number less than 10 with `0'.
\end{docCommand}

\begin{teX}
\def\two@digits#1{\ifnum#1<10 0\fi\number#1}
\end{teX}



\begin{docCommand}{typeout} {\meta{message}} 
    Display something on the terminal.
\end{docCommand}
   
\begin{teX}
\def\typeout#1{\begingroup\set@display@protect
    \immediate\write\@unused{#1}\endgroup}
\end{teX}
 
\begin{docCommand} {newlinechar}{ }
    A char to be used as new-line in output to files.
\end{docCommand}

\begin{teX}
\newlinechar`\^^J
\end{teX}
 \section{Saved versions of \TeX{} primitives}


\begin{docCommand}{@@par}{}
The TeX primitive |\foo| is saved as |\@@foo|.
The following primitives are handled in this way:
\end{docCommand}

\begin{teX}
\let\@@par=\par
\let\@@input=\input    %%% moved earlier
\let\@@end=\end        %%%
\end{teX}

\begin{docCommand}{@@hyph}{ }
\begin{docCommand}{-}{ }
  The following comment was added when these commands were first set
  up, 19 April 1986:
  the |\-| command is redefined to allow it to work in the |\ttfamily|
  type style, where automatic hyphenation is suppressed by setting
  |\hyphenchar| to~$-1$. The original primitive \TeX{} definition is
  saved as |\@@hyph| just in case anyone needs it.

  There is a need for a robust command for a discretionary hyphen
  since its exact representation depends on the glyphs available in
  the current font.  For example, with suitable fonts and the
  \texttt{T1} font encoding it is possible to use hanging hyphens.

  A suitable robust definition that allows for many possible types of
  font and encoding may be as follows:
  \begin{verbatim}
  \DeclareRobustCommand {\-}{%
    \discretionary {%
      \char \ifnum\hyphenchar\font<\z@
              \defaulthyphenchar
            \else
              \hyphenchar\font
            \fi
                    }{}{}%
  }
  \end{verbatim}

  The redefinition (via |\let|) of |\-| within tabbing also makes the
  use of a robust command advisable since then any redefinition
  of |\-| via |\DeclareRobustCommand| will not cause a conflict.

  Therefore, macro writers should be hereby warned that
  these internals will probably change! It is likely that a future
  release of \LaTeX{} will make |\-| effectively an encoding specific
  text command.

\begin{teX}
\let\@@hyph=\-        % Save original primitive definition
\def\-{\discretionary{-}{}{}}
\end{teX}
\end{docCommand}
\end{docCommand}
%
\begin{docCommand}{@dischyph}{ }

%         {Define \cs{@dischyph}, was previously in ltboxes.dtx}
\begin{teX}
\let\@dischyph=\-
\end{teX}
\end{docCommand}
%
  \begin{docCommand}{@@italiccorr}{}
 Save the original italic correction.
  \end{docCommand}
%
\begin{teX}
\let\@@italiccorr=\/
\end{teX}
 

%\endinput
\begin{docCommand}{@height} { }
\end{docCommand}

\begin{docCommand}{@depth} { }
\end{docCommand}

\begin{docCommand}{@width} { }
\end{docCommand}

\begin{docCommand}{@minus} { }
\end{docCommand}

\begin{docCommand}{@plus} { }
\end{docCommand}



The following definitions save token space  e.g., using
|\@height| instead of height saves 5 tokens at the cost in time
of one macro expansion.
    
\begin{teX}
\def\@height{height} \def\@depth{depth} \def\@width{width}
\def\@minus{minus}
\def\@plus{plus}
\end{teX}
\begin{docCommand}{hb@xt@} { }
The next one is another 100 tokens worth.
\end{docCommand}
\begin{teX}
\def\hb@xt@{\hbox to}
\end{teX}

\begin{teX}
\message{hacks,}
\end{teX}

 \subsection{Command definitions}



 This section defines the following commands:

 \DescribeMacro
  {\@namedef}\marg{NAME}\\ Expands to |\def\|\marg{NAME},
   except name can contain any characters.

 \DescribeMacro
  {\@nameuse}\marg{NAME}\\
   Expands to |\|\marg{NAME}.

 \DescribeMacro
  {\@ifnextchar} X\marg{YES}\marg{NO}\\
    Expands to \meta{YES} if next character is an `X',
          and to \meta{NO} otherwise.
          (Uses |\reserved@a|--|\reserved@c|.)
          NOTE: GOBBLES ANY SPACE FOLLOWING IT.

 \DescribeMacro
  {\@ifstar}\marg{YES}\marg{NO}\\
          Gobbles following spaces and then tests if next the
          character is a '*'.  If it is, then it gobbles the
   `*' and expands to \meta{YES}, otherwise it expands to \meta{NO}.

 \DescribeMacro
  {\@dblarg}\marg{CMD}\marg{ARG}\\
     Expands to |\|\marg{CMD}\oarg{ARG}\marg{ARG}.  Use
          |\@dblarg\CS| when |\CS| takes arguments |[ARG1]{ARG2}|,
          where default is| ARG1| = |ARG2|.

 \DescribeMacro
  {\@ifundefined}\marg{NAME}\marg{YES}\marg{NO}\\
          : If \cs{NAME} is undefined then it executes \meta{YES},
            otherwise it executes \meta{NO}.  More precisely,
            true if \cs{NAME} either undefined or = |\relax|.

 \DescribeMacro
  {\@ifdefinable}|\NAME|\marg{YES}
       Executes \meta{YES} if the user is allowed to define |\NAME|,
            otherwise it gives an error.  The user can define |\NAME|
            if |\@ifundefined{NAME}| is true, '|NAME|' $\neq$ '|relax|'
            and the first three letters of '|NAME|' are not
           '|end|', and if |\endNAME| is not defined.

 \DescribeMacro
  \newcommand|*|\marg{\cs{FOO}}\oarg{i}\marg{TEXT}\\
         User command to define |\FOO| to be a macro with
            i arguments (i = 0 if missing) having the definition
            \meta{TEXT}.  Produces an error if |\FOO| already
            defined.

            Normally the command is defined to be |\long| (ie it may
            take multiple paragraphs in its argument). In the
            star-form, the command is not defined as |\long| and a
            blank line in any argument to the command would generate
            an error.

 \DescribeMacro
  \renewcommand|*|\marg{\cs{FOO}}\oarg{i}\marg{TEXT}\\
  Same as |\newcommand|, except it checks if |\FOO| already defined.

 \DescribeMacro
  \newenvironment|*|\marg{FOO}\oarg{i}\marg{DEF1}\marg{DEF2}\\
         equivalent to:\\
         |\newcommand{\FOO}[i]{DEF1}| |\def{\endFOO}{DEF2}|\\
 (or the appropriate star forms).

 \DescribeMacro
  \renewenvironment\\ Obvious companion to |\newenvironment|.


 \DescribeMacro
  \@cons : See description of |\output| routine.

 \DescribeMacro{\@car}
  |\@car T1 T2 ... Tn\@nil| == |T1|  (unexpanded)

 \DescribeMacro{\@cdr}
  |\@cdr T1 T2 ... Tn\@ni|l == |T2 ... Tn|     (unexpanded)

 \DescribeMacro
  \typeout\marg{message}\\ Produces a warning message on the terminal.

 \DescribeMacro
  \typein\marg{message}\\
        Types message, asks the user to type in a command, then
            executes it

 \DescribeMacro
  \typein\oarg{\cs{CS}}\marg{MSG}\\
  Same as above, except defines |\CS| to be the input
                      instead of executing it.

 \changes{LaTeX209}{1992/03/18}
  {(RMS) changed input channel from 0 to \cs{@inputcheck} to avoid
     conflicts with other channels allocated by \cs{newread}}

\begin{docCommand}{typein}{}

\begin{teX}
\def\typein{%
  \let\@typein\relax
  \@testopt\@xtypein\@typein}
\end{teX}

\begin{teX}
\def\@xtypein[#1]#2{%
  \typeout{#2}%
  \advance\endlinechar\@M
  \read\@inputcheck to#1%
  \advance\endlinechar-\@M
  \@typein}
\end{teX}
\end{docCommand}

\subsection{csname abstractions }

\begin{docCommand}{@namedef}{}
The \CMDI{\@namedef} and \CMDI{\@nameuse} are used to define or use \cs{csname}. You need to watch
that the arguments are \emph{names} i.e, they are not preceded by a backslash or other active character.
(See example~\ref{ex:namedef}).

\end{docCommand}
\begin{docCommand}{@nameuse}{}
\end{docCommand}

\begin{teX}
\def\@namedef#1{\expandafter\def\csname #1\endcsname}
\def\@nameuse#1{\csname #1\endcsname}
\end{teX}


\begin{texexample}{Using \string\@namedef}{ex:namedef}
\makeatletter
\@namedef{test1}{This is the first test}
\@nameuse{test1}
\makeatother
\end{texexample}

\section{List helper macros}

\begin{docCommand}{@cons} {}
The command \cmd{\@cons} is used for the construction of lists. Its unique
characteristic being that it constsructs \emph{elt} lists. What this means is that each entry in the list is
preceded by another as yet undefined command |\@elt|. This can be later used to grub the element as
its argument and do something about it.

\begin{teX}
\def\@cons#1#2{\begingroup\let\@elt\relax\xdef#1{#1\@elt #2}\endgroup}
\end{teX}
\end{docCommand}

Assume we have a list \cmd{\mylist} which has been defined as |\mylist{\@elt one \@elt two ...}|.
We can use |\@cons| to append more elements as shown in the next example,

\begin{texexample}{Usage of cons}{ex:cons}
\makeatletter
\def\mylist{}
\@cons\mylist{one}
\@cons\mylist{two}
\def\@elt{\space}
\mylist
\makeatother
\end{texexample}


\begin{docCommand}{@car}{}
\end{docCommand}
\begin{docCommand}{@cdr}{}
The next two macros \cmd{\@car} and \cmd{\@cdr} capture either the first element of a list or the rest of the elements
except the first. They are both delimited by |\@nil|.
\end{docCommand}

\begin{teX}
\def\@car#1#2\@nil{#1}
\def\@cdr#1#2\@nil{#2}
\end{teX}



\begin{docCommand}{@carcube}{}
Given a list $T_1\dots T_n$ \cmd{\@carcube} will grub $T_1 T_2 T_3$ for |T1| |T2| |T3| , $n > 3$. If you are familiar with
lisp |car|, |cdr| and |cons| are fundamental functions. The |cons|  function is used to construct lists, and the |car|
 and |cdr| functions are used to take them apart.\footnote{The name of the cons function is not unreasonable: it is an abbreviation of the word `construct'. The origins of the names for |car| and |cdr|, on the other hand, are esoteric: |car| is an acronym from 
 the phrase `Contents of the Address part of the Register'; and cdr (pronounced `could-er') is an acronym from the phrase 
 `Contents of the Decrement part of the Register'. These phrases refer to specific pieces of hardware on 
 the very early computer on which the original Lisp was developed. Besides being obsolete, the phrases have been completely irrelevant for more than 35 years to anyone thinking about Lisp. Nonetheless, although a few brave scholars have begun to 
 use more reasonable names for these functions, the old terms are still in use. In particular, since the terms 
 are still used in the Emacs Lisp source code.}
\end{docCommand}
\begin{teX}
\def\@carcube#1#2#3#4\@nil{#1#2#3}
\end{teX}



\begin{docCommand}{@onlypreamble}{}
    This macro adds its argument to the list of commands stored in
    |\@preamblecmds| to be
    disabled after |\begin{document}|. These commands are redefined
    to generate |\@notprerr| at this point
\end{docCommand}
\begin{docCommand}{@preamblecmds}{}
\end{docCommand}

.
    
\begin{teX}
\def\@preamblecmds{}
\def\@onlypreamble#1{%
  \expandafter\gdef\expandafter\@preamblecmds\expandafter{%
       \@preamblecmds\do#1}}
       
\@onlypreamble\@onlypreamble
\@onlypreamble\@preamblecmds
\end{teX}


\section{Command Building}

\begin{docCommand}{@star@or@long} { \meta{parameter text} }
This is an interesting command in that it checks if \#1 is a star then sets \refCom{l@ngrel@x} to either |\long| or |\relax|. It is used for building \refCom{newcommand}, which is always |\long| except its star version.
\end{docCommand}

\begin{teX}
\def\@star@or@long#1{%
  \@ifstar
   {\let\l@ngrel@x\relax#1}%
   {\let\l@ngrel@x\long#1}}
\end{teX}

%

\begin{docCommand}{l@ngrel@x}{ }
 This is either |\relax| or |\long| depending on whether the |*|-form
 of a definition command is being executed.
\end{docCommand}
\begin{teX}
\let\l@ngrel@x\relax
\end{teX}

%
\begin{docCommand}{newcommand}{ \marg{cmd name} \oarg{number of arguments} \oarg{default for optional argument} \marg{replacement text}}
 User level |\newcommand|.%
 \begin{teX}
\def\newcommand{\@star@or@long\new@command}
\end{teX}
The command simply checks if it is long or short and the calls \refCom{new@command}.
\end{docCommand}
%
\begin{docCommand}{new@command}{} 
% \changes{v1.2u}{1995/10/16}{(DPC) Use \cs{@testopt} /1911}
\begin{teX}
\def\new@command#1{%
  \@testopt{\@newcommand#1}0}
\end{teX}


%
%
\begin{docCommand}{@newcommand}{ }
% \changes{LaTeX2e}{1993/11/23}{Macro added}
\end{docCommand}

\begin{docCommand}{@argdef}{}
% \changes{LaTeX2e}{1993/11/23}{Macro added}
\end{docCommand}
\begin{docCommand}{@xargdef}{}
\end{docCommand}
% \changes{LaTeX2e}{1993/11/23}{Macro interface changed}
% \changes{v1.1g}{2004/01/23}{Use kernel version of
%                             \cs{@ifnextchar} (pr/3501)}
%    Handling arguments for |\newcommand|.
\begin{teX}
\def\@newcommand#1[#2]{%
  \kernel@ifnextchar [{\@xargdef#1[#2]}%
                {\@argdef#1[#2]}}
\end{teX}
    Define |#1| if it is definable.

    Both here and in |\@xargdef| the replacement text is absorbed as
    an argument because if we are not allowed to make the definition
    we have to get rid of it completely.
    
\begin{teX}
\long\def\@argdef#1[#2]#3{%
   \@ifdefinable #1{\@yargdef#1\@ne{#2}{#3}}}
\end{teX}
%
% \changes{v1.2q}{1995/10/02}
%     {New implementation, using \cs{@test@opt}}
%    Handle the second optional argument.
\begin{teX}
\long\def\@xargdef#1[#2][#3]#4{%
  \@ifdefinable#1{%
\end{teX}

    Define the actual command to be:\\
    |\def\foo{\@protected@testopt\foo\\foo{default}}|\\
    where |\\foo| is a csname generated from applying |\csname| and
    |\string| to |\foo|, ie the actual name contains a backslash and
    therefore can't clash easily with existing command names.
    ``Default'' is the contents of the second optional argument of
    |(re)newcommand|.
    
    \begin{texexample}{newcommand internal form}{ex:newcom}
       \newcommand\mytest[2][other material] {some text}
       \meaning\mytest
    \end{texexample}
% The |\aut@global| command below is only used in the autoload
% format. If it is |\global| then a global definition will be made.
% \changes{v1.2q}{1995/10/02}
%     {Add \cs{aut@global} in autoload version}
% \changes{v1.2z2}{1998/03/04}
%     {Unnecessary \cs{expandafter} removed: pr/2758}
\begin{teX}
%<autoload>\aut@global
     \expandafter\def\expandafter#1\expandafter{%
          \expandafter
          \@protected@testopt
          \expandafter
          #1%
          \csname\string#1\endcsname
          {#3}}%
\end{teX}
    Now we define the internal macro ie |\\foo| which is supposed to
    pick up all arguments (optional and mandatory).
\begin{teX}
       \expandafter\@yargdef
          \csname\string#1\endcsname
           \tw@
           {#2}%
           {#4}}}
\end{teX}
\end{docCommand}


%
\begin{docCommand}{@testopt}{ }
 This macro encapsulates the most common call to |\@ifnextchar|, saving
 several tokens each time it is used in the definition of a command
 with an optional argument.
 
 |#1| The code to execute in the case that there is a |[| need not be
 a single token but can be any sequence of commands that `expects' to
 be followed by |[|. If this command were only used in |\newcommand|
 definitions then |#1| would be a single token and the braces could
 be omitted from |{#1}| in the definition below, saving a bit of
 memory.
\end{docCommand}
\begin{teX}
\long\def\@testopt#1#2{%
  \kernel@ifnextchar[{#1}{#1[{#2}]}}
\end{teX}


\begin{docCommand}{@protected@testopt}{ }
 Robust version of |\@testopt|. The extra argument (|#1|) must be a
 single token. If protection is needed the call expands to |\protect|
 applied to this token, and the 2nd and 3rd arguments are
 discarded (by |\@x@protect|). Otherwise |\@testopt| is called on
 the 2nd and 3rd arguments.

 This method of making commands robust avoids the need for using up
 two csnames per command, the price is the extra expansion time
 for the |\ifx| test.
 
\begin{teX}
\def\@protected@testopt#1{%%
  \ifx\protect\@typeset@protect
    \expandafter\@testopt
  \else
    \@x@protect#1%
  \fi}
\end{teX}
\end{docCommand}
%
\begin{docCommand}{@yargdef}{ }
\begin{docCommand}{@yargd@f}{}
\end{docCommand}
\end{docCommand}

    These generate a primitive argument specification, from a
    \LaTeX\ |[|\meta{digit}|]| form; in fact \meta{digit} can be
    anything such that |\number|~\meta{digit} is single digit.

    Reorganised slightly so that |\renewcommand{\reserved@a}[1]{foo}|
    works.  I am not sure this is worth it, as a following
    |\newcommand| would over-write the definition of |\reserved@a|.

    Recall that \LaTeX2.09 goes into an infinite loop with
    |\renewcommand[1]{\@tempa}{foo}| (DPC 6 October 93).

    Reorganised again (DPC 1999). Rather than make a loop to
    construct the argument spec by counting, just extract the
    required argument spec by using a delimited argument (delimited
    by the digit).  This is faster and uses less tokens. The coding
    is slightly odd to preserve the old interface (using |#2| =
    |\tw@| as the flag to surround the first argument with |[]|.  But
    the new method did not allow for the number of arguments |#3| not
    being given as an explicit digit; hence (further expansion of
    this argument and use of) |\number| was added later in 1999.

    It is not clear why these are still |\long|.

\begin{teX}
\long \def \@yargdef #1#2#3{%
  \ifx#2\tw@
    \def\reserved@b##11{[####1]}%
  \else
    \let\reserved@b\@gobble
  \fi
  \expandafter
    \@yargd@f \expandafter{\number #3}#1%
}
\end{teX}
%
% The |\aut@global| command below is only used in the autoload
% format. If it is |\global| then a global definition will be made.
% \changes{v1.2q}{1995/10/02}
%     {Add \cs{aut@global} in autoload version}
\begin{teX}
\long \def \@yargd@f#1#2{%
  \def \reserved@a ##1#1##2##{%
%<autoload>\aut@global
    \expandafter\def\expandafter#2\reserved@b ##1#1%
    }%
  \l@ngrel@x \reserved@a 0##1##2##3##4##5##6##7##8##9###1%
}
\end{teX}
%

%
\begin{docCommand}{@reargdef}{}
%  \changes{LaTeX2e}{1993/12/20}
%                {Kept old version of \cs{@reargdef}, for array.sty}
% \changes{v1.0b}{1994/03/12}{New defn, in terms of \cs{@yargdef}}
%  \changes{v1.2y}{1996/07/26}{third arg picked up by \cs{@yargdef}}
\begin{teX}
\long\def\@reargdef#1[#2]{%
  \@yargdef#1\@ne{#2}}
\end{teX}
\end{docCommand}
%
\begin{docCommand}{renewcommand} { }
    Check the command name is already used.  If not give an error
    message. Then temporarily
    disable |\@ifdefinable| then call |\newcommand|. (Previous
    version |\let#1=\relax| but this does not work too well if |#1|
\end{docCommand}
%    is |\@temp|\emph{a--e}.)
% \changes{LaTeX2e}{1993/11/23}{Macro reimplemented and extended}
% \changes{v1.1f}{1994/05/2}{Removed surplus \cs{space} in error}
\begin{teX}
\def\renewcommand{\@star@or@long\renew@command}
\end{teX}
%
\begin{docCommand}{renew@command} { }
\end{docCommand}
%  \changes{v1.2y}{1996/07/26}{use \cs{relax} in place of empty arg}
%  \changes{v1.2y}{1996/07/26}{use \cs{noexpand} instead of \cs{string}}
% \changes{v1.2z1}{1997/10/21}{Use \cs{begingroup}/\cs{endgroup} rather
%    than braces for grouping, to avoid generating empty math atom.}
\begin{teX}
\def\renew@command#1{%
  \begingroup \escapechar\m@ne\xdef\@gtempa{{\string#1}}\endgroup
  \expandafter\@ifundefined\@gtempa
     {\@latex@error{\noexpand#1undefined}\@ehc}%
     \relax
  \let\@ifdefinable\@rc@ifdefinable
  \new@command#1}
\end{teX}




\subsection{Checking if a command is definable}
%\begin{docCommand}{\@@ifdefinable}
\begin{docCommand}{@ifdefinable}{ }
 Test if user is allowed to define a command.
\end{docCommand}

\begin{docCommand}{@rc@ifdefinable}{}
   
\end{docCommand}
\begin{teX}
\long\def\@ifdefinable #1#2{%
      \edef\reserved@a{\expandafter\@gobble\string #1}%
     \@ifundefined\reserved@a
         {\edef\reserved@b{\expandafter\@carcube \reserved@a xxx\@nil}%
          \ifx \reserved@b\@qend \@notdefinable\else
            \ifx \reserved@a\@qrelax \@notdefinable\else
              #2%
            \fi
          \fi}%
         \@notdefinable}
\end{teX}
%    Saved definition of |\@ifdefinable|.

\begin{teX}
\let\@@ifdefinable\@ifdefinable
\end{teX}

    Version of |\@ifdefinable| for use with |\renewcommand|.  Does
    not do the check this time, but restores the normal definition.

\begin{teX}
\long\def\@rc@ifdefinable#1#2{%
  \let\@ifdefinable\@@ifdefinable
  #2}
\end{teX}

%\end{docCommand}

This command is not a general command for package builders but it has its uses. If the command
has been defined then it issues an error message via |\@notdefinable|. In the example
below we save the |\@notdefinable| in order to redirect the error message to the example.
We check if the counter |c@chapter| can be defined, since it has it will activate the |\@notdefinable| and
print an error. We then restore the command to its previous meaning.

\begin{texexample}{ifdefinable}{ex:ifdefinable}
\makeatletter
\let\save@notdefinable\@notdefinable
\def\@notdefinable{Not definable}
\@ifdefinable{c@chapter}{true}
\let\@notdefinable\save@notdefinable
\makeatother
\end{texexample}

\section{Environment Building commands}

\begin{docCommand}{newenvironment} {}
 Define a new user environment.
    |#1| is the environment name. |#2#| Grabs all the tokens up to
    the first |{|. These will be any optional arguments. They are not
    parsed at this point, but are just passed to |\@newenv| which
    will eventually call |\newcommand|. Any optional arguments will
    then be parsed by |\newcommand| as it defines the command that
    executes the `begin code' of the environment.

    This |#2#| trick removed with version 1.2i as it fails if a |{|
    occurs in the optional argument. Now use |\@ifnextchar| directly.

\end{docCommand}   
\begin{teX}
\def\newenvironment{\@star@or@long\new@environment}
\end{teX}
%
\begin{docCommand}{new@environment} {}

\begin{teX}
\def\new@environment#1{%
  \@testopt{\@newenva#1}0}
\end{teX}
\end{docCommand}
%


\begin{docCommand}{@newenva}{}

\begin{teX}
\def\@newenva#1[#2]{%
   \kernel@ifnextchar [{\@newenvb#1[#2]}{\@newenv{#1}{[#2]}}}
\end{teX}
\end{docCommand}
%
\begin{docCommand}{@newenvb}{}

\begin{teX}
\def\@newenvb#1[#2][#3]{\@newenv{#1}{[#2][{#3}]}}
\end{teX}

\end{docCommand}
%
%
\begin{docCommand}{renewenvironment}{}
    Redefine an environment.
    For |\renewenvironment| disable |\@ifdefinable| and then call
    |\newenvironment|.  It is OK to |\let| the argument to |\relax|
    here as there should not be a |@temp|\ldots\ environment.

\begin{teX}
\def\renewenvironment{\@star@or@long\renew@environment}
\end{teX}

\begin{docCommand}{renew@environment}{}
As a |csname| is used the name of an envronment can contain non-letters, such as dashes etc.
\begin{teX}
\def\renew@environment#1{%
  \@ifundefined{#1}%
     {\@latex@error{Environment #1 undefined}\@ehc
     }\relax
  \expandafter\let\csname#1\endcsname\relax
  \expandafter\let\csname end#1\endcsname\relax
  \new@environment{#1}}
\end{teX}
\end{docCommand}
\end{docCommand}


\begin{docCommand}{@newenv}{ }
\end{docCommand}
    The internal version of |\newenvironment|.

    Call |\newcommand| to define the \meta{begin-code} for the
    environment.  |\def| is used for the \meta{end-code} as it does
    not take arguments. (but may contain |\par|s)

 Make sure that an attempt to define a `graf' or `group' environment
 fails.
\begin{teX}
\long\def\@newenv#1#2#3#4{%
  \@ifundefined{#1}%
    {\expandafter\let\csname#1\expandafter\endcsname
                         \csname end#1\endcsname}%
    \relax
  \expandafter\new@command
     \csname #1\endcsname#2{#3}%
\end{teX}

%     {Add \cs{aut@global} in autoload version}
\begin{teX}
%<autoload>\aut@global
     \l@ngrel@x\expandafter\def\csname end#1\endcsname{#4}}
\end{teX}


\section{newif}

The \latex kernel provides the \refCom{newif} command that can be used to define boolean switches. It uses a couple of tricks to remove the |if| part of the command being defined and to |\let| the switches to true or false.
\begin{docCommand}{newif}{\marg{cmd} }
This is as the notes in the kernel mention a different type of allocation. A macro
to define boolean switches

 For example,
 |\newif\if@foo| creates |\@footrue|, |\@foofalse| to go with |\if@foo|.\footnote{Customarily these command use the `@' to make the definitions internal and also more readable.}
 
 Before we describe how this is achieved we make a small digression to test a not so widely
 known trick to get the \cmd{\string} not to print the backslash, by setting the |\escapechar|
 to -1. This is really an esoteric trick.\index{hacks}
 \end{docCommand}
 
 \begin{texexample}{escapechar}{ex:escapechar}
 \bgroup
 \def\test{This is a test}
 \makeatletter
 \escapechar\m@ne
 \test
 
 \string\test
 
 test
 \makeatother
 \egroup
 \end{texexample}
 
The \cmd{\newif} is defined in a very clever way.  

\begin{teXXX}
\def\newif#1{% 
  % save original definition 
  \count@\escapechar 
  % allocate to -1
  \escapechar\m@ne
  % set to false as a default
    \let#1\iffalse
  % define \@footrue and \@foofalse 
    \@if#1\iftrue
    \@if#1\iffalse
  \escapechar\count@}
\end{teXXX}

\begin{docCommand}{@if}{\meta{name of newif}\meta{|ifalse or iftrue|} }
The auxiliary macro for the |\newif| command.

\begin{phdverbatim}
\def\@if#1#2{% 
  \expandafter\def\csname\expandafter\@gobbletwo\string#1%
                    \expandafter\@gobbletwo\string#2\endcsname
                       {\let#1#2}}
\end{phdverbatim}
\end{docCommand}

%\begin{texexample}{Newif}{ex:newif}
%\def\Newif#1{%  \iftest 
%  \count@\escapechar \escapechar\m@ne
%    \let#1\iffalse
%    \@If#1\iftrue
%    \@If#1\iffalse
%  \escapechar\count@}
%\def\@If#1#2{% (*@\label{atif1} @*)
%  \expandafter\def\csname\expandafter\@gobbletwo\string#1%
%                    \expandafter\@gobbletwo\string#2\endcsname
%                       {\let#1#2}}  
%\end{texeexample}


This is a mind twisting way to define the |\newif| and  I am sure a method  applauded by every Byzantine
General. It also resolved a long standing query in my head as to why there was never a |@gobblethree| command in the kernel  (see \autoref{gobble}).

\section{Provide version of commands}

\begin{docCommand}{providecommand}{}

 |\providecommand| takes the same arguments as |\newcommand|, but
 discards them if |#1| is already defined, Otherwise it just acts like
 |\newcommand|. This implementation currently leaves any discarded
 definition in |\reserved@a| (and possibly |\\reserved@a|) this
 wastes a bit of space, but it will be reclaimed as soon as these
 scratch macros are redefined.


\begin{teX}
\def\providecommand{\@star@or@long\provide@command}
\end{teX}

\begin{docCommand}{provide@command}{}

\begin{teX}
\def\provide@command#1{%
  \begingroup
    \escapechar\m@ne\xdef\@gtempa{{\string#1}}%
  \endgroup
  \expandafter\@ifundefined\@gtempa
    {\def\reserved@a{\new@command#1}}%
    {\def\reserved@a{\renew@command\reserved@a}}%
   \reserved@a}%
\end{teX}
\end{docCommand}
\end{docCommand}

\begin{docCommand}{CheckCommand}{}

    |\CheckCommand| takes the same arguments as |\newcommand|. If
    the command already exists, with the same definition, then
    nothing happens, otherwise a warning is issued. Useful for
    checking the current state befor a macro package starts
    redefining things.  Currently two macros are considered to have
    the same definition if they are the same except for different
    default arguments.  That is, if the old definition was:
    |\newcommand\xxx[2][a]{(#1)(#2)}| then
    |\CheckCommand\xxx[2][b]{(#1)(#2)}| would \emph{not} generate a
    warning, but, for instance |\CheckCommand\xxx[2]{(#1)(#2)}|
    would.
\begin{teX}
\def\CheckCommand{\@star@or@long\check@command}
\end{teX}
    |\CheckCommand| is only available in the preamble part of the
    document.
\begin{teX}
\@onlypreamble\CheckCommand
\end{teX}

\begin{docCommand}{check@command}{}
\begin{teX}
\def\check@command#1#2#{\@check@c#1{#2}}
\@onlypreamble\check@command
\end{teX}
\end{docCommand}
\end{docCommand}

\begin{docCommand}{@check@c}{}
    |\CheckCommand| itself just grabs all the arguments we need,
    without actually looking for |[| optional argument forms.  Now
    define |\reserved@a|. If |\\reserved@a| is then defined, compare it
    with the ``|\#1|' otherwise compare |\reserved@a| with |#1|.
\begin{teX}
\long\def\@check@c#1#2#3{%
  \expandafter\let\csname\string\reserved@a\endcsname\relax
  \renew@command\reserved@a#2{#3}%
  \@ifundefined{\string\reserved@a}%
   {\@check@eq#1\reserved@a}%
   {\expandafter\@check@eq
           \csname\string#1\expandafter\endcsname
           \csname\string\reserved@a\endcsname}}
\@onlypreamble\@check@c
\end{teX}
\end{docCommand}
%
\begin{docCommand}{@check@eq}{}
     Complain if |#1| and |#2| are not |\ifx| equal.
\begin{teX}
\def\@check@eq#1#2{%
  \ifx#1#2\else
     \@latex@warning@no@line
               {Command \noexpand#1 has
                changed.\MessageBreak
                Check if current package is valid}%
  \fi}
\@onlypreamble\@check@eq
\end{teX}
\end{docCommand}

\subsection{Argument gobbling command utilities}
\label{gobble}

The \docAuxCommand{@gobble} macro is used to get rid of its argument. Similarly \docAuxCommand{@gobbletwo} and \docAuxCommand{@gobblefour} eat two or four arguments respectively.
\begin{teX}
\long\def \@gobble #1{}
\long\def \@gobbletwo #1#2{}
\long\def \@gobblefour #1#2#3#4{}
\end{teX}


\section{Other argument grabbers}

The commands \docAuxCommand{@firstofone}, \docAuxCommand{@firstoftwo} and \docAuxCommand{@secondoftwo} are amongst the most widely used commands in the kernel. They only grab the argumeny indicated by their name.

%    Some argument-grabbers.
\begin{teX}
\long\def\@firstofone#1{#1}
\long\def\@firstoftwo#1#2{#1}
\long\def\@secondoftwo#1#2{#2}
\end{teX}

\begin{docCommand}{@iden}{}
    |\@iden| is another name for |\@firstofone| for
    compatibility reasons.
\begin{teX}
\let\@iden\@firstofone
\end{teX}
\end{docCommand}
%
\begin{docCommand}{@thirdofthree}{}
%    Another grabber now used in the encoding specific
%    section.

\begin{teX}
\long\def\@thirdofthree#1#2#3{#3}
\end{teX}
\end{docCommand}
%
%
\begin{docCommand}{@expandtwoargs}{}
 A macro to totally expand two arguments to another macro
   
\begin{teX}
\def\@expandtwoargs#1#2#3{%
   \edef\reserved@a{\noexpand#1{#2}{#3}}\reserved@a}
\end{teX}
\end{docCommand}

\begin{texexample}{Expand two arguments}{ex:2args}
\makeatletter
\bgroup
  \def\test#1#2{#1, #2} 
  \def\xx{first argument}
  \def\yy{second argument}
  \@expandtwoargs\test{\xx}{\yy}
\egroup
\makeatother
\end{texexample}

\begin{docCommand}{@backslashchar}{}
A category code 12 backslash. See also \refCom{textbackslash}  which is a better alternative.
\begin{teX}
\edef\@backslashchar{\expandafter\@gobble\string\\ }
\end{teX}
\end{docCommand}


 \section{Robust commands and protect}

 Fragile and robust commands are one of the thornier issues in
 \LaTeX's commands.  Whilst typesetting documents, \LaTeX{} makes use
 of many of \TeX's features, such as arithmetic, defining macros, and
 setting variables.  However, there are (at least) three different
 occasions when these commands are not safe.  These are called
 `moving arguments' by \LaTeX, and consist of:
 \begin{itemize}
 \item writing information to a file, such as indexes or tables of
    contents.
 \item writing information to the screen.
 \item inside an |\edef|, |\message|, |\mark|, or other command which
    evaluates its argument fully.
 \end{itemize}
 The method \LaTeX{} uses for making fragile commands robust is to
 precede them with |\protect|.  This can have one of five possible
 values:
 
 \begin{itemize}
 \item |\relax|, for normal typesetting.  So |\protect\foo| will
    execute |\foo|.
 \item |\string|, for writing to the screen.  So |\protect\foo| will
    write |\foo|.
 \item |\noexpand|, for writing to a file.  So |\protect\foo| will
    write |\foo| followed by a space.
 \item |\@unexpandable@protect|, for writing a moving argument to a
    file.  So |\protect\foo| will write |\protect\foo| followed by a
    space.  This value is also used inside |\edef|s, |\mark|s and
    other commands which evaluate their arguments fully.
 \item |\@unexpandable@noexpand|, for performing a deferred write
    inside an |\edef|.  So |\protect\foo| will write |\foo| followed
    by a space.  If you want |\protect\foo| to be written, you should
    use |\@unexpandable@protect|. (Removed as never used).
 \end{itemize}

\begin{docCommand}{@unexpandable@protect}{}
\begin{docCommand}{@unexpandable@noexpand}{}

%    These commands are used for setting |\protect| inside |\edef|s.
\begin{teX}
\def\@unexpandable@protect{\noexpand\protect\noexpand}
%\def\@unexpandable@noexpand{\noexpand\noexpand\noexpand}
\end{teX}
\end{docCommand}
\end{docCommand}
%
% \changes{v1.2e}{1994/11/04}{Rewrote protected short commands
%    using \cs{x@protect}. ASAJ.}
%
\begin{docCommand}{DeclareRobustCommand}{}
\begin{docCommand}{declare@robustcommand}{}
    This is a package-writers command, which has the same syntax as
    |\newcommand|, but which declares a protected command.  It does
    this by having\\
    |\DeclareRobustCommand\foo|\\
    define |\foo| to be
    |\protect\foo<space>|,\\
    and then use |\newcommand\foo<space>|.\\
    Since the internal command is |\foo<space>|, when it is written
    to an auxiliary file, it will appear as |\foo|.

    We have to be a
    bit cleverer if we're defining a short command, such as |\_|, in
    order to make sure that the auxiliary file does not include a
    space after the command, since |\_ a| and |\_a| aren't the same.
    In this case we define |\_| to be:
\begin{verbatim}
    \x@protect\_\protect\_<space>
\end{verbatim}
    which expands to:
\begin{verbatim}
    \ifx\protect\@typeset@protect\else
       \@x@protect@\_
    \fi
    \protect\_<space>
\end{verbatim}
    Then if |\protect| is |\@typeset@protect| (normally |\relax|)
    then we just perform |\_<space>|, and otherwise
    |\@x@protect@| gobbles everything up and expands to
    |\protect\_|.

    \emph{Note}: setting |\protect| to any value other than |\relax|
    whilst in `typesetting' mode will cause commands to go into an
    infinite loop!  In particular, setting |\relax| to |\@empty| will
    cause |\_| to loop forever.  It will also break lots of other
    things, such as protected |\ifmmode|s inside |\halign|s.  If you
    really really have to do such a thing, then please set
    |\@typeset@protect| to be |\@empty| as well.  (This is what the
    code for |\patterns| does, for example.)

    More fun with |\expandafter| and |\csname|.
\begin{teX}
\def\DeclareRobustCommand{\@star@or@long\declare@robustcommand}
\end{teX}
%
\begin{teX}
\def\declare@robustcommand#1{%
   \ifx#1\@undefined\else\ifx#1\relax\else
      \@latex@info{Redefining \string#1}%
   \fi\fi
   \edef\reserved@a{\string#1}%
   \def\reserved@b{#1}%
   \edef\reserved@b{\expandafter\strip@prefix\meaning\reserved@b}%
\end{teX}
% \changes{v1.2s}{1995/10/06}
%     {Add \cs{aut@global} in autoload version}
\begin{teX}
%<autoload>\aut@global
   \edef#1{%
      \ifx\reserved@a\reserved@b
         \noexpand\x@protect
         \noexpand#1%
      \fi
      \noexpand\protect
      \expandafter\noexpand\csname
         \expandafter\@gobble\string#1 \endcsname
   }%
   \let\@ifdefinable\@rc@ifdefinable
   \expandafter\new@command\csname
      \expandafter\@gobble\string#1 \endcsname
}
\end{teX}
\end{docCommand}
\end{docCommand}
%
%
\begin{docCommand}{@x@protect}{}
\begin{docCommand}{x@protect}{}
%
\begin{teX}
\def\x@protect#1{%
   \ifx\protect\@typeset@protect\else
      \@x@protect#1%
   \fi
}
\def\@x@protect#1\fi#2#3{%
   \fi\protect#1%
}
\end{teX}
\end{docCommand}
\end{docCommand}
%
\begin{docCommand}{@typeset@protect}{}
%
\begin{teX}
\let\@typeset@protect\relax
\end{teX}
\end{docCommand}
%
% \changes{v1.2e}{1994/11/04}{Added commands for setting and restoring
%    \cs{protect}.  ASAJ.}
%
\begin{docCommand}{set@display@protect}{}
\begin{docCommand}{set@typeset@protect}{}
%    These macros set |\protect| appropriately for typesetting or
%    displaying.
% \changes{v1.2o}{1995/07/03}{Use \cs{@typeset@protect} for init}
\begin{teX}
\def\set@display@protect{\let\protect\string}
\def\set@typeset@protect{\let\protect\@typeset@protect}
\end{teX}
\end{docCommand}
\end{docCommand}
%
\begin{docCommand}{protected@edef}{}
\begin{docCommand}{protected@xdef}{}
\end{docCommand}
\end{docCommand}
\begin{docCommand}{unrestored@protected@xdef}{}
\begin{docCommand}{restore@protect}
    The commands |\protected@edef| and |\protected@xdef| perform
    `safe' |\edef|s and |\xdef|s, saving and restoring |\protect|
    appropriately.  For cases where restoring |\protect| doesn't
    matter, there's an `unsafe' |\unrestored@protected@xdef|, useful
    if you know what you're doing!
    
\begin{teX}
\def\protected@edef{%
   \let\@@protect\protect
   \let\protect\@unexpandable@protect
   \afterassignment\restore@protect
   \edef
}
\def\protected@xdef{%
   \let\@@protect\protect
   \let\protect\@unexpandable@protect
   \afterassignment\restore@protect
   \xdef
}
\def\unrestored@protected@xdef{%
   \let\protect\@unexpandable@protect
   \xdef
}
\def\restore@protect{\let\protect\@@protect}
\end{teX}

\end{docCommand}
\end{docCommand}
%
%
\begin{docCommand}{protect}{}
%    The normal meaning of |\protect|
% \changes{v1.2j}{1995/04/29}{Init \cs{protect} here}
\begin{teX}
\set@typeset@protect
\end{teX}
\end{docCommand}
%
\subsection{Internal defining commands}
%
% These commands are used internally to define other \LaTeX{}
% commands.
\begin{docCommand}{@ifundefined}{}
% \changes{LaTeX2e}{1993/11/23}{Redefined to remove a trailing \cs{fi}}
%    Check if first arg is undefined or \cs{relax} and execute second or
%    third arg depending,
\begin{teX}
\def\@ifundefined#1{%
  \expandafter\ifx\csname#1\endcsname\relax
    \expandafter\@firstoftwo
  \else
    \expandafter\@secondoftwo
  \fi}
\end{teX}
\end{docCommand}
%
%
\begin{docCommand}{@qend}{}
\begin{docCommand}{@qrelax}{}
% The following define |\@qend| and |\@qrelax| to be the strings
% `|end|' and `|relax|' with the characters |\catcode|d 12.
\begin{teX}
\edef\@qend{\expandafter\@cdr\string\end\@nil}
\edef\@qrelax{\expandafter\@cdr\string\relax\@nil}
\end{teX}
\end{docCommand}
\end{docCommand}


\begin{docCommand}{@ifnextchar}{}

  |\@ifnextchar| peeks at the following character and compares it
  with its first argument. If both are the same it executes its
  second argument, otherwise its third.
\begin{teX}
\long\def\@ifnextchar#1#2#3{%
  \let\reserved@d=#1%
  \def\reserved@a{#2}%
  \def\reserved@b{#3}%
  \futurelet\@let@token\@ifnch}
\end{teX}
\end{docCommand}

\begin{docCommand}{kernel@ifnextchar}{}

    This macro is the kernel version of |\@ifnextchar| which is used
    in a couple of places to prevent the AMS variant from being used
    since in some places this produced chaos (for example
    if an \texttt{fd} file
    is loaded in a random place then the optional argument to
    |\ProvidesFile| could get printed there instead of being written
    only in the log file.  This happened
    when there was a space or a newline between the mandatory and
    optional arguments! It should really be fixed in the
    \texttt{amsmath} package one day, but\ldots

    Note that there may be other places in the kernel where this version
    should be used rather than the original, but variable, version.

\begin{teX}
\let\kernel@ifnextchar\@ifnextchar
\end{teX}
\end{docCommand}
%
%
\begin{docCommand}{@ifnch}{}
    |\@ifnch| is a tricky macro to skip any space tokens that may
    appear before the character in question. If it encounters a space
    token, it calls |\@xifnch|.\footnote{There is an error in the kernel at this point in the documentation, as the
    verbatim guards have been omitted.}

\begin{teX}
\def\@ifnch{%
  \ifx\@let@token\@sptoken
    \let\reserved@c\@xifnch
  \else
    \ifx\@let@token\reserved@d
      \let\reserved@c\reserved@a
    \else
      \let\reserved@c\reserved@b
    \fi
  \fi
  \reserved@c}
\end{teX}
\end{docCommand}
%
\begin{docCommand}{@sptoken}{}
    The following code makes |\@sptoken| a space token. It is
    important here that the control sequence |\:| consists of
    a non-letter only, so that the following whitespace is
    significant. Together with the fact that the equal sign
    in a |\let| may be followed by only one optional space
    the desired effect is achieved.
    NOTE: the following hacking must precede the definition of |\:|
    as math medium space.
\begin{teX}
\def\:{\let\@sptoken= } \:  % this makes \@sptoken a space token
\end{teX}
\end{docCommand}
%
\begin{docCommand}{@xifnch}{}
    In the following definition of |\@xifnch|, |\:| is again used
    to get a space token as delimiter into the definition.
\begin{teX}
\def\:{\@xifnch} \expandafter\def\: {\futurelet\@let@token\@ifnch}
\end{teX}
\end{docCommand}

The following two commands are widely used in preambles to change the \refCom{catcode} of |@| to 11 or 12.

\begin{docCommand}{makeatletter}{}
Make internal control sequences accessible and inaccessible (\docAuxCommand{makeatother}).
\begin{teX}
\def\makeatletter{\catcode`\@11\relax}
\def\makeatother{\catcode`\@12\relax}
\end{teX}
\end{docCommand}


This |\@ifstar| command is another useful 
\begin{docCommand}{@ifstar} {\marg{true code} \marg{false code}}
 The new implementation below avoids passing the \meta{true code}
 Through one more \refCom{def} than the \meta{false code}, which previously
 meant that |#| had to be written as |####| in one argument, but |##|
 in the other. The |*| is gobbled by |\@firstoftwo|.
\begin{teX}
\def\@ifstar#1{\@ifnextchar *{\@firstoftwo{#1}}}
\end{teX}
\end{docCommand}

\begin{docCommand}{@dblarg}{}
\begin{docCommand}{@xdblarg}{}

\begin{teX}
\long\def\@dblarg#1{\kernel@ifnextchar[{#1}{\@xdblarg{#1}}}
\long\def\@xdblarg#1#2{#1[{#2}]{#2}}
\end{teX}
\end{docCommand}
\end{docCommand}
%

\begin{docCommand}{@sanitize}{}
 The command |\@sanitize| changes the catcode of all \emph{special characters}
 except for braces to `other'.  It can be used for commands like
 |\index| that want to write their arguments verbatim.  Needless to
 say, this command should only be executed within a group, or chaos
 will ensue.
\end{docCommand}

\begin{teX}
\def\@sanitize{\@makeother\ \@makeother\\\@makeother\$\@makeother\&%
\@makeother\#\@makeother\^\@makeother\_\@makeother\%\@makeother\~}
\end{teX}
%
\begin{docCommand}{@onelevel@sanitize}{}
% \changes{v1.2c}{1994/10/30}{Macro added}

    This makes the whole ``meaning'' of |#1| (its one-level
    expansion) into catcode 12 tokens: it could be used in
    |\DeclareRobustCommand|.

    If it is to be used on default float specifiers, this should be
    done when they are defined.
\begin{teX}
\def \@onelevel@sanitize #1{%
  \edef #1{\expandafter\strip@prefix
           \meaning #1}%
}
\end{teX}
\end{docCommand}

 \subsection{Commands for Autoloading}


\begin{docCommand}{aut@global}{}
% \changes{v1.2q}{1995/10/02}
%     {Macro added}
% This command is only defined in the `autoload' format. It is
% normally |\relax| but may be set to |\global|, in which case
% |\newif| and the commands based on |\newcommand| will all make
% global definitions.
\begin{teX}
\let\aut@global\relax
\end{teX}
\end{docCommand}
%
\begin{docCommand}{@autoload}{}

Use \cs{@@input} not \cs{input} to save string space and
      stops autoload files appearing in \cs{listfiles}

 This macro is only defined in the `autoload' format. It inputs a
 package `|auto#1.sty|' within a local group, and with normalised
 catcodes. |\aut@global| is set to |\global| so that |\newif|
 |\newcommand| and related commands make global definitions.
\begin{teX}
\def\@autoload#1{%
  \begingroup
  \makeatletter
  \let\aut@global\global
  \nfss@catcodes
  \catcode`\ =10
  \let\@latex@e@error\@gobble
  \@@input auto#1.sty\relax
  \endgroup}
\end{teX}
\end{docCommand}

This been a rather long discussion, but this is almost the heart of the kernel commands that are most useful
for package writers. Spend time to understand this section well, as many of the hacks described here are useful
and are used in most of the other sections of the kernel.



     \chapter{Bibliography Kernel Class}
 \label{kernel:biblio} 
  This section of the kernel deals with citations and bibliographies. With the exception of some journals and
  short papers and packages, it is unlikely that some package such as \pkgname{natbib} or \pkgname{biblatex}
  will be loaded.
  
  The class was wrritten and maintained by Johannes Braams,
  David Carlisle,
  Alan Jeffrey,
  Leslie Lamport,
  Frank Mittelbach,
  Chris Rowley,
  Rainer Sch\"opf.



 \section{Bibliography Generation}

  A bibliography is created by the |thebibliography| environment, which
  generates a title such as ``References'', and a list of entries.
  The BIB\TeX{} program will create a file containing such an
  environment, which will be read in by the |\bibliography| command.
  With BIB\TeX, the following commands will be used.

 \begin{docCommand*}{bibliography}{\marg{file1, file2, file3, \ldots, filen}}
  |\bibliography|\marg{file1,file2, \ldots ,filen} : specifies
     the bibdata files.  Writes a |\bibdata| entry on the |.aux| file
     and tries to read in |mainfile.bbl|.
\end{docCommand*}

 \DescribeMacro{\bibliographystyle}
  |\bibliographystyle|\marg{style} :
     Writes a |\bibstyle| entry on the |.aux| file.

 \DescribeEnv{thebibliography}
  The |thebibliography| environment is a list environment.  To save the
  use of an extra counter, it should use  |enumiv|  as the item
  counter.
  Instead of using |\item|, items in the bibliography are produced by
  the  following commands:\\
    |\bibitem|\marg{name}    : Produces a numbered entry cited as
    \meta{name}.\\
    |\bibitem|\oarg{label}\marg{name} : Produces an entry labeled by
    \meta{Label} and cited by \meta{name}.

  The former is used for bibliographies with citations like [1], [2],
  etc.;
  the latter is used for citations like [Knuth82].

  The document class must define the thebibliography environment.  This
  environment has a single argument, which is the widest bibliography
  label-- e.g., if the [Knuth67] is the widest entry, then this
  argument will be Knuth67.  The |\thebibliography| command must begin
  a list  environment, which the |\endthebibliography| command ends.

 \DescribeMacro{\cite}
  Entries are cited by the command |\cite|\marg{name}.

 \DescribeMacro{\nocite}
 |\nocite|\marg{citations}
 puts information on the |.aux| file that causes
 \BibTeX{} to include the \marg{citations} list in the bibliography,
 but puts nothing in the text.

 |\nocite{*}| is special: it tells \BibTeX{} to put the whole of a
 collection of references into the bibiography.

\begin{teX}

\message{bibliography,}
\end{teX}
%
%
 \begin{teX}
  PARAMETERS

   \@cite   : A macro such that \@cite{LABEL1,LABEL2}{NOTE}
              produces the output for a \cite[NOTE]{FOO1,FOO2} command,
              where entry FOOi is defined by \bibitem[LABELi]{FOOi}.
              The switch @tempswa is true if the optional NOTE argument
              is present.
              The default definition is :
                \@cite{LABELS}{NOTE} ==
                   BEGIN [LABELS
                         IF @tempswa = T THEN , NOTE FI
                         ]
                   END

   \@biblabel : A macro to produce the label in the bibliography
                entry.  For \bibitem[LABEL]{NAME}, the label is
                generated by \@biblabel{LABEL}.  It has the default
                definition \@biblabel{LABEL} -> [LABEL].
  CONVENTION

  \b@FOO : The name or number of the reference created by \cite{FOO}
           E.g., if \cite{FOO} -> [17] , then \b@FOO -> 17.

 \end{teX}


 \begin{docCommand}{bibitem}{}
       \begin{teX}
\def\bibitem{\@ifnextchar[\@lbibitem\@bibitem}
       \end{teX}
 \end{docCommand}


\begin{docCommand}{@lbibitem}{}
       \begin{teX}
\def\@lbibitem[#1]#2{\item[\@biblabel{#1}\hfill]\if@filesw
      {\let\protect\noexpand
       \immediate
       \write\@auxout{\string\bibcite{#2}{#1}}}\fi\ignorespaces}
       \end{teX}
 \end{docCommand}
%
%
 \begin{docCommand}{@bibitem}{}
       \begin{teX}
\def\@bibitem#1{\item\if@filesw \immediate\write\@auxout
       {\string\bibcite{#1}{\the\value{\@listctr}}}\fi\ignorespaces}
       \end{teX}
  \end{docCommand}
%
 \begin{docCommand}{bibcite}{}
% \changes{v1.1f}{1995/04/24}
%   {Make \cs{@onlypreamble} /1388.}
% \changes{v1.1h}{1995/06/19}
%   {Call \cs{@newl@bel} so repeated keys produce better warning.}
% \changes{v1.1i}{1995/07/14}
%   {Remove \cs{@onlypreamble} so still defined in new \cs{enddocument}}
       \begin{teX}
\def\bibcite{\@newl@bel b}
       \end{teX}
  \end{docCommand}
%
\begin{docCommand}{citation}{}
       \begin{teX}
\let\citation\@gobble
       \end{teX}
  \end{docCommand}
%
\begin{docCommand}{cite}{}
% \changes{v1.1j}{1995/10/16}{(DPC) Make robust}
       \begin{teX}
\DeclareRobustCommand\cite{%
  \@ifnextchar [{\@tempswatrue\@citex}{\@tempswafalse\@citex[]}}
       \end{teX}
  \end{docCommand}
%
\begin{docCommand}{@citex}{}
% |\penalty\@m| added to definition of |\@citex| to allow a line
% break after the `,' in citations like [Jones80,Smith77]
% (Added 23 Oct 86)
%
% space added after the `,' (21 Nov 87)
%
% \changes{LaTeX2.09}{1991/10/25}
%      {added \cs{reset@font}, suggested by Bernd Raichle.}
% \changes{LaTeX2.09}{1991/11/06}
%     {added code to remove a leading blank}
% \changes{LaTeX2.09}{1992/08/14}
%      {added missing argument braces around \cs{hbox},
%               found by Ed Sznyter}
% \changes{LaTeX2.09}{1992/08/17}
%       {simplified code for removing leading blanks in
%               citation key (proposed by Frank Jensen and
%               Kresten Krab Thorup)}
% \changes{LaTeX2.09}{1993/08/06}
%  {Moved writing to .aux file in loop over citation keys
%               so that leading blanks are removed there as well.}
% \changes{v1.0c}{1994/05/05}{Set switch for warning and end of run.}
% \changes{v1.1e}{1995/04/24}
%   {Add \cs{mbox} to undefined case: latex/1239.}
% \changes{v1.1g}{1995/05/08}{Use \cs{@firstofone}}
% \changes{v1.1k}{1995/10/20}{Removed refundefined flag}
% \changes{v1.1l}{1995/12/07}{Restored name of \cs{G@refundefinedtrue}}
% \changes{v1.1m}{1997/04/24}{\cs{@empty} to avoid primitive
%    error on empty cite keys. latex/2432}
% \changes{v1.1n}{2002/12/13}{Added \cs{leavevmode} in case citation
%    is at start of paragraph (pr/3486)}
       \begin{teX}
\def\@citex[#1]#2{\leavevmode
  \let\@citea\@empty
  \@cite{\@for\@citeb:=#2\do
    {\@citea\def\@citea{,\penalty\@m\ }%
     \edef\@citeb{\expandafter\@firstofone\@citeb\@empty}%
     \if@filesw\immediate\write\@auxout{\string\citation{\@citeb}}\fi
       \end{teX}
       
    Using |\hbox| instead of |\mbox| is fine because of the
    |\leavevmode| above. In fact the use of a box around the citation
    contents is more than questionable in my view (FMi), but within
    2e I have to keep that for compatibility reasons as it would
    probably change too many existing documents. Its main reason is
    to avoid hyphenation of labels such as [FOOB89] into [FOO- B89]
    so in certain styles it makes sense; but, for example, in author
    year citations it becomes more than questionable.

    So Chris added yet another hook here, as suggested by, at least,
    Donald Arsenau.  Note that this one is inside the first argument
    of the |\@cite| hook.
    This decouples the top-level typesetting of the citation from
    the details of the other business conducted here.  All this really
    needs a complete rethink to get the right modularity.

       \begin{teX}
     \@ifundefined{b@\@citeb}{\hbox{\reset@font\bfseries ?}%
       \G@refundefinedtrue
       \@latex@warning
         {Citation `\@citeb' on page \thepage \space undefined}}%
       {\@cite@ofmt{\csname b@\@citeb\endcsname}}}}{#1}}
       \end{teX}
  \end{docCommand}
%
%
\begin{docCommand}{bibdata}{}
\begin{docCommand}{bibstyle}{}
       \begin{teX}
\let\bibdata=\@gobble
\let\bibstyle=\@gobble
       \end{teX}
  \end{docCommand}
  \end{docCommand}


 \begin{docCommand}{bibliography}{}
% \changes{LaTeX2e}{1994/01/18}
%         {Use \cs{@input@} so include files are listed.}
       \begin{teX}
\def\bibliography#1{%
  \if@filesw
    \immediate\write\@auxout{\string\bibdata{#1}}%
  \fi
  \@input@{\jobname.bbl}}
       \end{teX}
 \end{docCommand}
%

 \begin{docCommand}{bibliographystyle}{}
       \begin{teX}
\def\bibliographystyle#1{%
  \ifx\@begindocumenthook\@undefined\else
    \expandafter\AtBeginDocument
  \fi
    {\if@filesw
       \immediate\write\@auxout{\string\bibstyle{#1}}%
     \fi}}
       \end{teX}
 \end{docCommand}
%
%
  \begin{docCommand}{nocite}{}
%    (Added 14 Jun 85)
% \changes{v1.1c}{1994/11/10}{Fix \cs{nocite}\texttt{\char`\{*\char`\}}}
%
%    This puts information on the |.aux| file that causes
%    \BibTeX{} to include the citation list in the bibliography,
%    but puts nothing in the text.
%
% RmS 93/08/06: Made loop for |\nocite| like that for |\@citex|,
%               to get rid of leading spaces.
% \changes{v1.0b}{1994/05/03}{Make \cs{nocite} issue a warning
%            for an undefined citation key.}
% \changes{v1.0c}{1994/05/05}{Do not write page number in
%            \cs{nocite} warning message.}
% \changes{v1.0c}{1994/05/05}{Set switch for warning and end of run.}
% \changes{v1.1g}{1995/05/08}{Use \cs{@firstofone}}
% \changes{v1.1k}{1995/10/20}{Removed refundefined flag}
       \begin{teX}
\def\nocite#1{\@bsphack
       \end{teX}
%    With the implementation designed already in \LaTeX\,2.09 the
%    |\nocite| command will not work before |\begin{document}| since
%    it tries to write to the |.aux| file which is not open before
%    that point. As a result the ``reference'' will appear on the
%    terminal and nothing else will happen.
%
%    This would be easy to fix, but then a document using the fix will
%    silently fail on an older release of \LaTeX{}, missing all
%    citations done with |\nocite|. Thus we do only generate an error
%    message and leave the fix for a \LaTeXe{} successor.
%
% \changes{v1.1o}{2003/05/18}{Check if we are after \cs{document}}
% \changes{v1.1p}{2004/01/04}{Changed error message}
       \begin{teX}
  \ifx\@onlypreamble\document
       \end{teX}
%    Since we are after |\begin{document}| we can do the citations:
       \begin{teX}
    \@for\@citeb:=#1\do{%
      \edef\@citeb{\expandafter\@firstofone\@citeb}%
      \if@filesw\immediate\write\@auxout{\string\citation{\@citeb}}\fi
      \@ifundefined{b@\@citeb}{\G@refundefinedtrue
          \@latex@warning{Citation `\@citeb' undefined}}{}}%
  \else
       \end{teX}
%    But before |\begin{document}| we raise an error message:
       \begin{teX}
    \@latex@error{Cannot be used in preamble}\@eha
       \end{teX}
%    Without the compatibility problems we could fix the problem as follows:
       \begin{teX}
    % \AtBeginDocument{\nocite{#1}}
  \fi
  \@esphack}
       \end{teX}
%    Since |\nocite{*}| should not produce a warning about undefined
%    citation keys (seee PR 557), we need to set the control sequence
%    `|\b@*|' to something other than |\relax|. As a result |\cite{*}|
%    will not warn either (but that never worked with \BibTeX{} in the
%    first place).
       \begin{teX}
\expandafter\let\csname b@*\endcsname\@empty
       \end{teX}
  \end{docCommand}
%
%
 \subsection{Default definitions}

    This hook determines the `relative formatting' of the two logical
    parts of a citation with comment.
    
\begin{docCommand}{@cite}{}
       \begin{teX}
\def\@cite#1#2{[{#1\if@tempswa , #2\fi}]}
       \end{teX}
  \end{docCommand}
%
 \begin{docCommand}{@cite@ofmt}{}
% \changes{v1.1q}{2004/02/15}{Added hook with default value \cs{hbox}}
%    This is, in general, a command that appears to have one argument
%    whose value is, in the kernel, a single cs whose name is the
%    expansion of |b@\@citeb|; the expansion of this cs will
%    typically be some hmode material that produces the detailed
%    typeset form of just the citations themselves.
       \begin{teX}
\let\@cite@ofmt\hbox
       \end{teX}
  \end{docCommand}

 \begin{docCommand}{@biblabel}{}
       \begin{teX}
\def\@biblabel#1{[#1]}
       \end{teX}
  \end{docCommand}

     \chapter{LaTeX kernel sectioning class}
\label{kernel:ltsect}

  Johannes Braams,
  David Carlisle,
  Alan Jeffrey,
  Leslie Lamport,
  Frank Mittelbach,
  Chris Rowley,
  Tobias Oetiker (Tobi updated
              the comments to `doc' conventions)
  Rainer Sch\"opf


 \section{Sectioning Commands}

 This file defines the declarations such as |\author| which are used
 by |\maketitle|. |\maketitle| itself is defined by each class, not
 in the \LaTeX{} kernel.

 The second part of the file defines the generic commands used for
 defining sectioning commands such as |\chapter|. Again the actual
 document level commands are defined in the class files, in terms of
 these commands.


    \begin{teX}
\message{title,}
    \end{teX}

 \subsection{The Title}

 \DescribeMacro{\title}
 The user defines the title and  author by the declarations
 |\title|\marg{name},
 \DescribeMacro{\author}
 |\author|\marg{name}

 \DescribeMacro{\date}
 Similarly the date is declared with
 |\date|\marg{date}.

 \DescribeMacro{\thanks}
 Inside these, the |\thanks|\marg{footnote text} command may be used
 to make acknowledgements, notice of address, etc.\ in a footnote.
 \DescribeMacro{\and}
 If there are multiple authors, they have to be separated with the
 |\and| command.

 \DescribeMacro{\maketitle}
 And finally, the |\maketitle| command produces the actual title,
 using the information previously saved with the other commands.

  \begin{docCommand}{title}{}
  \begin{docCommand}{@title}{}
% \changes{LaTeX2e}{1993/12/11}{Added default}
% |\title| for use in |\maketitle|. If not given |\maketitle| will
% produce an error message.
       \begin{teX}
\def\title#1{\gdef\@title{#1}}
\def\@title{\@latex@error{No \noexpand\title given}\@ehc}
       \end{teX}
  \end{docCommand}
  \end{docCommand}
%
  \begin{docCommand}{author}{}
  \begin{docCommand}{@author}{}
  |\author| for use in |\maketitle|. If not given |\maketitle| will
  produce a warning message.

       \begin{teX}
\def\author#1{\gdef\@author{#1}}
\def\@author{\@latex@warning@no@line{No \noexpand\author given}}
       \end{teX}
  \end{docCommand}
  \end{docCommand}
%
  \begin{docCommand}{date}{}
  \begin{docCommand}{@date}{}
%    |\date| for use in |\maketitle|. If not given |\maketitle| will
%    produce |\today| as the default.
       \begin{teX}
\def\date#1{\gdef\@date{#1}}
\gdef\@date{\today}
       \end{teX}
  \end{docCommand}
  \end{docCommand}
%
% \changes{1.0h}{1994/11/04}{(ASAJ) Added \cs{protected@xdef} to
%    \cs{thanks}.}
\begin{docCommand}{thanks}{}
       \begin{teX}
\def\thanks#1{\footnotemark
    \protected@xdef\@thanks{\@thanks
        \protect\footnotetext[\the\c@footnote]{#1}}%
}
       \end{teX}
\end{docCommand}
%
\begin{docCommand}{@thanks}{}
       \begin{teX}
\let\@thanks\@empty
       \end{teX}
\end{docCommand}


\begin{docCommand}{and}{}
 The \cmd{\and} is used to join the names of authors and to place them in a tabular!
       \begin{teX}
\def\and{%                  % \begin{tabular}
  \end{tabular}%
  \hskip 1em \@plus.17fil%
  \begin{tabular}[t]{c}}%   % \end{tabular}
       \end{teX}
\end{docCommand}
%
       \begin{teX}
\message{sectioning,}
       \end{teX}

 \subsection{Sectioning}

\begin{docCommand}{@secpenalty}{}
       \begin{teX}
\newcount\@secpenalty
\@secpenalty = -300
       \end{teX}
\end{docCommand}

\begin{docCommand}{if@noskipsec}{}
\begin{docCommand}{@noskipsectrue}{}
% \changes{1.0w}{1996/09/29}{Added documentation}
% Way back in 1991 (08/26) FMi \& RmS set the |\@noskipsec| switch
% to true for the preamble and to false in |\document|.
% This was done to trap lists and related text in the preamble but it
% does not catch everything.
       \begin{teX}
\newif\if@noskipsec \@noskipsectrue
       \end{teX}
\end{docCommand}
\end{docCommand}
%
\begin{docCommand}{@startsection}

 The |\@startsection{|\meta{name}|}{|\meta{level}|}{|%
       \meta{indent}|}{|\meta{beforeskip}|}|\\
     |{|\meta{afterskip}|}{|\meta{style}|}*[|\meta{altheading}%
     |]{|\meta{heading}|}|
 command is the mother of all the user level sectioning commands.
 The part after the |*|, including the |*| is optional.

 \begin{description}
 \item[name:] e.g., 'subsection'
 \item[level:] a number, denoting depth of section -- e.g., chapter=1,
                 section = 2, etc.
 \item[indent:] Indentation of heading from left margin
 \item[beforeskip:] Absolute value = skip to leave above the heading.
                If negative, then paragraph indent of text following
                heading is suppressed.
 \item[afterskip:] if positive, then skip to leave below heading, else
                negative of skip to leave to right of run-in heading.
 \item[style:] Commands to set style. Since June 1996 release the
               \emph{last} command in this argument may be a command
                such as |\MakeUppercase| or |\fbox| that takes an
                argument. The section heading will be supplied as the
                argument to this command. So setting |#6| to, say,
                |\bfseries\MakeUppercase| would produce bold,
                uppercase headings.
 \end{description}

  If `|*|' is  missing, then increment the counter.  If it is
  present, then there should be no |[|\meta{altheading}|]| argument.
  The command uses the counter 'secnumdepth'. It contains a pointer
  to the highest section level that is to be numbered.

  \textbf{Warning:}
  The |\@startsection| command should be at the same or higher
  grouping level as the text that follows it.  For example, you should
  \emph{not} do something like
  \begin{verbatim}
      \def\foo{ \begingroup ...
                   \paragraph{...}
                 \endgroup}
  \end{verbatim}
%
% Pseudocode for the |\@startsection| command
% \begin{oldcomments}
% \@startsection {NAME}{LEVEL}{INDENT}{BEFORESKIP}{AFTERSKIP}{STYLE} ==
%    BEGIN
%     IF  @noskipsec = T  THEN  \leavevmode  FI
%                              % true if previous section had no body.
%
%     \par
%     \@tempskipa  := BEFORESKIP
%     @afterindent := T
%     IF \@tempskipa < 0  THEN  \@tempskipa  := -\@tempskipa
%                               @afterindent := F
%     FI
%     IF @nobreak = true
%       THEN \everypar == null
%       ELSE \addpenalty{\@secpenalty}
%            \addvspace{\@tempskipa}
%     FI
%     IF * next
%       THEN \@ssect{INDENT}{BEFORESKIP}{AFTERSKIP}{STYLE}
%       ELSE \@dblarg{\@sect
%                {NAME}{LEVEL}{INDENT}
%                {BEFORESKIP}{AFTERSKIP}{STYLE}}
%     FI
% END
% \end{oldcomments}
%
\numberlineat{22}
       \begin{teX}
\def\@startsection#1#2#3#4#5#6{%
  \if@noskipsec \leavevmode \fi
  \par
  \@tempskipa #4\relax
  \@afterindenttrue
  \ifdim \@tempskipa <\z@
    \@tempskipa -\@tempskipa \@afterindentfalse
  \fi
  \if@nobreak
    \everypar{}%
  \else
    \addpenalty\@secpenalty\addvspace\@tempskipa
  \fi
  \@ifstar
    {\@ssect{#3}{#4}{#5}{#6}}%
    {\@dblarg{\@sect{#1}{#2}{#3}{#4}{#5}{#6}}}}
       \end{teX}
\end{docCommand}
%

\begin{docCommand}{@sect}{}
% Pseudocode for the |\@sect| command
% \begin{oldcomments}
% \@sect{NAME}{LEVEL}{INDENT}{BEFORESKIP}{AFTERSKIP}{STYLE}[ARG1]{ARG2}
%           ==
%   BEGIN
%    IF LEVEL > \c@secnumdepth
%      THEN \@svsec :=L null
%      ELSE \refstepcounter{NAME}
%           \@svsec :=L BEGIN \@seccntformat{#1}\relax END
%    FI
%    IF AFTERSKIP > 0
%      THEN \begingroup
%              STYLE
%              \@hangfrom{\hskip INDENT\@svsec}
%              {\interlinepenalty 10000 ARG2\par}
%           \endgroup
%           \NAMEmark{ARG1}
%           \addcontentsline{toc}{NAME}
%              { IF  LEVEL > \c@secnumdepth
%                  ELSE \protect\numberline{\theNAME}  FI
%                ARG1 }
%      ELSE \@svsechd == BEGIN  STYLE
%                               \hskip INDENT\@svsec
%                               ARG2
%                               \NAMEmark{ARG1}
%                               \addcontentsline{toc}{NAME}
%                                  { IF  LEVEL > \c@secnumdepth
%                                      ELSE
%                                        \protect\numberline{\theNAME}
%                                      FI
%                                    ARG1 }
%                        END
%    FI
%    \@xsect{AFTERSKIP}
% END
% \end{oldcomments}
%
% \changes{LaTeX2.09}{1992/08/25}
%         {(FMi) replaced explicit setting of \cs{@svsec}
%               by call to \cs{@seccntformat}}
% \changes{LaTeX2.09}{1993/08/05}
%         {(RmS) Made sure that \cs{protect} works correctly in
%               expansion of \cs{the} counter}
% \changes{1.0h}{1994/11/04}
%         {(ASAJ) Added \cs{protected@edef}.}
\numberlineat{38}
       \begin{teX}
\def\@sect#1#2#3#4#5#6[#7]#8{%
  \ifnum #2>\c@secnumdepth
    \let\@svsec\@empty
  \else
    \refstepcounter{#1}%
       \end{teX}

    Apparently since |\@seccntformat| might end with an improper |\hskip| which
    is scanning forward for |plus| or |minus| we end the definition
    of |\@svsec| with |\relax| as a precaution.

\numberlineat{43}
       \begin{teX}
    \protected@edef\@svsec{\@seccntformat{#1}\relax}%
  \fi
  \@tempskipa #5\relax
  \ifdim \@tempskipa>\z@
    \begingroup
       \end{teX}
% \changes{v1.0s}{1996/05/21}
%         {(DPC) Moved brace to allow commands like
%           \cs{MakeUppercase} in 6th argument.
%            Changed \cs{par} to \cs{endgraf} to allow non-long
%            commands. internal/2148}
% \changes{v1.0t}{1996/06/10}
%         {(DPC) Changed \cs{endgraf} to \cs{@@par}}
% This |{| used to be after the argument to |\@hangfrom| but was moved
% here to allow commands such as |\MakeUppercase| to be used at the end
% of |#6|.
\startlineat{48}
       \begin{teX}
      #6{%
        \@hangfrom{\hskip #3\relax\@svsec}%
          \interlinepenalty \@M #8\@@par}%
    \endgroup
    \csname #1mark\endcsname{#7}%
    \addcontentsline{toc}{#1}{%
      \ifnum #2>\c@secnumdepth \else
        \protect\numberline{\csname the#1\endcsname}%
      \fi
      #7}%
  \else
%
    \def\@svsechd{%
      #6{\hskip #3\relax
      \@svsec #8}%
      \csname #1mark\endcsname{#7}%
      \addcontentsline{toc}{#1}{%
        \ifnum #2>\c@secnumdepth \else
          \protect\numberline{\csname the#1\endcsname}%
        \fi
        #7}}%
  \fi
  \@xsect{#5}}
       \end{teX}
\end{docCommand}
%
\begin{docCommand}{@xsect}{}%
% Pseudocode for the |\@xsect| command
% \begin{oldcomments}
% \@xsect{AFTERSKIP} ==
%  BEGIN
%    IF AFTERSKIP > 0
%      THEN \par \nobreak
%           \vskip AFTERSKIP
%           \@afterheading
%      ELSE @nobreak :=G F
%           @noskipsec :=G T
%           \everypar{ IF @noskipsec = T
%                        THEN @noskipsec :=G F
%                             \clubpenalty :=G 10000
%                             \hskip -\parindent
%                             \begingroup
%                               \@svsechd
%                             \endgroup
%                             \unskip
%                             \hskip -AFTERSKIP \relax
%                                           %% relax added 14 Jan 91
%                        ELSE \clubpenalty :=G \@clubpenalty
%                             \everypar := NULL
%                      FI
%                    }
%    FI
%
%   END
% \end{oldcomments}

\startlineat{70}
       \begin{teX}
\def\@xsect#1{%
  \@tempskipa #1\relax
  \ifdim \@tempskipa>\z@
       \end{teX}
%    Why not combine |\@sect| and |\@xsect| and save doing the
%    same test twice? It is not possible to change this now as these
%    have become hooks!
%
%    This |\par| seems unnecessary.
       \begin{teX}
    \par \nobreak
    \vskip \@tempskipa
    \@afterheading
  \else
    \@nobreakfalse
    \global\@noskipsectrue
    \everypar{%
      \if@noskipsec
        \global\@noskipsecfalse
       {\setbox\z@\lastbox}%
        \clubpenalty\@M
        \begingroup \@svsechd \endgroup
        \unskip
        \@tempskipa #1\relax
        \hskip -\@tempskipa
      \else
        \clubpenalty \@clubpenalty
        \everypar{}%
      \fi}%
  \fi
  \ignorespaces}
       \end{teX}
\end{docCommand}
%
\begin{docCommand}{@seccntformat}{}
    This command formats the section number including the space
    following it. This command has proved a bit problematic when I parameterized
    the sectioning comands, as it is uniform for all the sectioning commands. For example
    after an inline paragraph command we might want to leave a bit more space.

\startlineat{94}
       \begin{teX}
\def\@seccntformat#1{\csname the#1\endcsname\quad}
       \end{teX}
\end{docCommand}
%
% Pseudocode for the |\@ssect| command
% \begin{oldcomments}
% \@ssect{INDENT}{BEFORESKIP}{AFTERSKIP}{STYLE}{ARG} ==
%   BEGIN
%    IF AFTERSKIP > 0
%      THEN \begingroup
%             STYLE
%             \@hangfrom{\hskip INDENT}{\interlinepenalty 10000 ARG\par}
%           \endgroup
%      ELSE \@svsechd == BEGIN STYLE
%                              \hskip INDENT
%                              ARG
%                        END
%    FI
%    \@xsect{AFTERSKIP}
%   END
% \end{oldcomments}
%
% Pseudocode for the |\@afterheading| command
% \begin{oldcomments}
% \@afterheading ==
%  BEGIN
%    @nobreak :=G true
%    \everypar := BEGIN  IF @nobreak = T
%                          THEN @nobreak  :=G false
%                               \clubpenalty :=G 10000
%                               IF @afterindent = F
%                                 THEN remove \lastbox
%                               FI
%                          ELSE \clubpenalty :=G \@clubpenalty
%                               \everypar := NULL
%                       FI
%                 END
%  END
% \end{oldcomments}
%
%
\begin{docCommand}{@ssect}{}

\startlineat{95}
       \begin{teX}
\def\@ssect#1#2#3#4#5{%
  \@tempskipa #3\relax
  \ifdim \@tempskipa>\z@
    \begingroup
       \end{teX}
% This |{| used to be after the argument to |\@hangfrom| but was moved
% here to allow commands such as |\MakeUppercase| to be used at the end
% of |#4|.
       \begin{teX}
      #4{%
        \@hangfrom{\hskip #1}%
          \interlinepenalty \@M #5\@@par}%
    \endgroup
  \else
    \def\@svsechd{#4{\hskip #1\relax #5}}%
  \fi
  \@xsect{#3}}
       \end{teX}
\end{docCommand}

\begin{docCommand}{if@afterindent}{}
\begin{docCommand}{@afterindenttrue}{}
The boolean \cmd{\@afterindent} is used to control the indentation after a sectioning command.

       \begin{teX}
\newif\if@afterindent \@afterindenttrue
       \end{teX}
\end{docCommand}\end{docCommand}
%
\begin{docCommand}{@afterheading}{}
  This hook is used in setting up custom-built headings in classes.dtx. See for example the book.dtx at
  \autoref{book:afterheading}.

\startlineat{108}
       \begin{teX}
\def\@afterheading{%
  \@nobreaktrue
  \everypar{%
    \if@nobreak
      \@nobreakfalse
      \clubpenalty \@M
      \if@afterindent \else
        {\setbox\z@\lastbox}%
      \fi
    \else
      \clubpenalty \@clubpenalty
      \everypar{}%
    \fi}}
       \end{teX}
\end{docCommand}
%
%
\begin{docCommand}{@hangfrom}{}

 |\@hangfrom{|\meta{text}|}| : Puts \meta{text} in a box, and makes a
 hanging indentation of the following material up to the first
 |\par|. Should be used in vertical mode.

       \begin{teX}
\def\@hangfrom#1{\setbox\@tempboxa\hbox{{#1}}%
      \hangindent \wd\@tempboxa\noindent\box\@tempboxa}
       \end{teX}
\end{docCommand}
%
\begin{docCommand}{c@secnumdepth}{}
\begin{docCommand}{c@tocdepth}{}

       \begin{teX}
\newcount\c@secnumdepth
\newcount\c@tocdepth
       \end{teX}
\end{docCommand}\end{docCommand}
%
\begin{docCommand}{secdef}{}
This is a user command to define star and unstarred commands without having to 
use the \refCom{@startsection}.

 |\secdef{|\meta{unstarcmds}|}{|\meta{unstarcmds}|}{|%
           \meta{starcmds}|}|\\
 When defining a |\chapter| or |\section| command without using
 |\@startsection|, you can use |\secdef| as follows:
 
 \begin{enumerate}
 \item |\def\chapter{| \ldots  |\secdef|
                |\|\meta{starcmd} |\|\meta{unstarcmd} |}|
 \item |\def\|\meta{starcmd}|[#1]#2{| \ldots |}|
            |%| Command to define |\chapter[|\ldots|]{|\ldots|}|
 \item |\def\|\meta{unstarcmd}|#1{| \ldots |}|
   |%| Command to define |\chapter*{|\ldots|}|
 \end{enumerate}

       \begin{teX}
\def\secdef#1#2{\@ifstar{#2}{\@dblarg{#1}}}
       \end{teX}
\end{docCommand}
%
 \subsubsection{Initializations}
\begin{docCommand}{sectionmark}{}
\begin{docCommand}{subsectionmark}{}
\begin{docCommand}{subsubsectionmark}{}
\end{docCommand}
\end{docCommand}
\end{docCommand}
\begin{docCommand}{paragraphmark}{}
\begin{docCommand}{subparagraphmark}{}

       \begin{teX}
\let\sectionmark\@gobble
\let\subsectionmark\@gobble
\let\subsubsectionmark\@gobble
\let\paragraphmark\@gobble
\let\subparagraphmark\@gobble
       \end{teX}
\end{docCommand}
\end{docCommand}

%
       \begin{teX}
\message{contents,}
       \end{teX}
%
 \subsection{Table of Contents etc.}

 \subsubsection{Convention}
% |\tf@|\meta{foo} = file number for output for table foo.
%       The file is opened only if |@filesw| = |true|.
%
 \subsubsection{Commands}


  A |\l@|\meta{type}|{|\meta{entry}|}{|\meta{page}|}| Macro needs to
  defined by document style for making an entry of type \meta{type}
  in a table of contents, etc.  E.g., the document style
  should define |\l@chapter|, |\l@section|, etc.

  \textbf{Note:} When the |\protect| command is
  used in the \meta{entry} or \meta{text} of one of the commands
  below, it causes the following control sequence to be written on
  the file without being expanded.  The sequence will be expanded
  when the table of contents entry is processed.

  \textbf{Surprise:} Inside an |\addcontentsline| or |\addtocontents|
  command argument, the commands: |\index|, |\glossary|,  and |\label|
  are  no-ops .  This could cause a problem if the user puts an
  |\index| or |\label| into one of the commands he writes, or into the
  optional `short version' argument of a |\section| or |\caption|
  command.

\begin{docCommand}{@starttoc}{}
 The |\@starttoc|\marg{ext} command is used to define the commands:\\
 |\tableofcontents|, |\listoffigures|, etc.

 For example:
 |\@starttoc{lof}| is used in |\listoffigures|.  This command
 reads the |.|\meta{ext} file and sets up to write the new
 |.|\meta{ext} file.

% \begin{oldcomments}
% \@starttoc{EXT} ==
%   BEGIN
%     \begingroup
%        \makeatletter
%        read file \jobname.EXT
%        IF @filesw = true
%          THEN  open \jobname.EXT as file \tf@EXT
%        FI
%        @nobreak :=G FALSE  %% added 24 May 89
%     \endgroup
%   END
% \end{oldcomments}

       \begin{teX}
\def\@starttoc#1{%
  \begingroup
    \makeatletter
    \@input{\jobname.#1}%
    \if@filesw
      \expandafter\newwrite\csname tf@#1\endcsname
      \immediate\openout \csname tf@#1\endcsname \jobname.#1\relax
    \fi
    \@nobreakfalse
  \endgroup}
       \end{teX}
\end{docCommand}

  \begin{docCommand}{addcontentsline}{}
  The |\addcontentsline{|\meta{table}|}{|\meta{type}|}{|%
  \meta{entry}|}| command allows the user to  add
  his/her own entry to a table of contents, etc. The command adds the
  entry |\contentsline{|\meta{type}|}{|\meta{entry}|}{|\meta{page}|}|
  to the |.|\meta{table} file.

  This macro is implemented as an application of |\addtocontents|.
  Note that |\thepage| is not expandable during |\protected@write|
  therefore one gets the page number at the time of the |\shipout|.

       \begin{teX}
\def\addcontentsline#1#2#3{%
  \addtocontents{#1}{\protect\contentsline{#2}{#3}{\thepage}}}
       \end{teX}
  \end{docCommand}

\begin{docCommand}{addtocontents}{}

   The |\addtocontents{|\meta{table}|}{|\meta{text}|}| command
   adds \meta{text} to the |.|\meta{table} file, with no
   page number.

       \begin{teX}
\long\def\addtocontents#1#2{%
  \protected@write\@auxout
      {\let\label\@gobble \let\index\@gobble \let\glossary\@gobble}%
      {\string\@writefile{#1}{#2}}}
       \end{teX}
\end{docCommand}
%
\begin{docCommand}{contentsline}{}
% The |\contentsline{|\meta{type}|}{|\meta{entry}|}{|\meta{page}|}|
% macro produces a \meta{type} entry in a table of contents, etc.
% It will appear in the |.toc| or other file.  For example,
% The entry for subsection 1.4.3 in the table of contents for
% example, might be produced by:
%
%  \begin{verbatim}
%       \contentsline{subsection}
%           {\makebox{30pt}[r]{1.4.3} Gnats and Gnus}{22}
%  \end{verbatim}
%
%  The |\protect| command causes command sequences to be written
%  without expanding them.
%
       \begin{teX}
\def\contentsline#1{\csname l@#1\endcsname}
       \end{teX}
\end{docCommand}

\begin{docCommand}{@dottedtocline}{\meta{level}\meta{indent}\meta{numwidth}}%
        |\meta{title}\meta{page}|:
       
   Macro to produce a table of contents line with the following
   parameters:
\end{docCommand}
   
   \begin{marglist}
   \item[level] If \meta{level} $>$ |\c@tocdepth|, then no line
                produced.
   \item[indent] Total indentation from the left margin.
   \item[numwidth] Width of box for number if the \meta{title} has a
                |\numberline| command.
                As of 25 Jan 1988, this is also the amount of extra
                indentation added to second and later lines of a
                multiple line entry.
   \item[title] Contents of entry.
   \item[page] Page number.
  \end{marglist}

  Uses the following parameters, which must be set by the document
  style. They should be defined with |\def|'s.
  
  \begin{marglist}
  \item[\cs{@pnumwidth}]  Width of box in which page number is set.
  \item[\cs{@tocrmarg}] Right margin indentation for all but last line
        of multiple-line entries.
  \item[\texttt\cs{@dotsep}] Separation between dots, in mu units.
                  Should be |\def|'d to a number like 2 or 1.7
  \end{marglist}
%
\begin{docCommand}{@dottedtocline}{}

       \begin{teX}
\def\@dottedtocline#1#2#3#4#5{%
  \ifnum #1>\c@tocdepth \else
    \vskip \z@ \@plus.2\p@
    {\leftskip #2\relax \rightskip \@tocrmarg \parfillskip -\rightskip
     \parindent #2\relax\@afterindenttrue
     \interlinepenalty\@M
     \leavevmode
     \@tempdima #3\relax
       \end{teX}
% \changes{v1.0z}{1996/12/20}{Added \cs{nobreak} for latex/2343}
       \begin{teX}
     \advance\leftskip \@tempdima \null\nobreak\hskip -\leftskip
     {#4}\nobreak
     \leaders\hbox{$\m@th
       \end{teX}
    If a document uses fonts other than computer modern, the use of a
    dot from math can be very disturbing despite the fact that this
    might be the only place in a document that then uses computer
    modern.
    Therefore we surround the dot with an |\hbox| to escape to the
    surrounding text font.
% \changes{v1.0k}{1995/04/25}{Added \cs{hbox} around dots.}
% \changes{v1.0l}{1995/05/02}{Don't reset to \cs{rmfamily}}
       \begin{teX}
        \mkern \@dotsep mu\hbox{.}\mkern \@dotsep
        mu$}\hfill
     \nobreak
     \hb@xt@\@pnumwidth{\hfil\normalfont \normalcolor #5}%
     \par}%
  \fi}
       \end{teX}
\end{docCommand}
%
% \textbf{Note:} |\nobreak|'s added 7 Jan 86 to prevent bad line break
% that left the page number dangling by itself at left edge of a new
% line.
%
% Changed 25 Jan 88 to use |\leftskip| instead of |\hangindent| so
% leaders of multiple-line contents entries would line up properly.
\begin{docCommand}{numberline}{\meta{number}}
 For use in a |\contentsline| command.
   It puts \meta{number} flushleft in a box of width |\@tempdima|
       \begin{teX}
\def\numberline#1{\hb@xt@\@tempdima{#1\hfil}}
%</2ekernel>
       \end{teX}
\end{docCommand}
%

\section{Packages and developments}

Many package authors have redefined the kernel commands to add hooks or styles to the definitions in order to make them more flexible. The \pkgname{tocstyle} \cite{tocstyle} which is still in alpha provides redefinitions so that one can specify toc styles using akey value system. It also provides a mechanism for the rdefinition of
styles. Our own \pkgname{phd} provides a flexible sytem with all the parameters provided as key value pairs based on pgf keys. By far the most popular package is Javier Bezos \pkgname{titlesec}. This forms part of a suite of
packages \pkgname{titlesec}, \pkgname{titleps} and \pkgname{titletoc}. Bezos used a different philosophy to that
used in the kernel and provides more generic commands.

\begin{verbatim}
\titleformat{\section}[runin]
{\normalfont\bfseries}
{\S\ \thesection.}{.5em}{}[.---]
\titlespacing{\section}
{\parindent}{1.5ex plus .1ex minus .2ex}{0pt}
\end{verbatim}








  ^^A
\chapter{ltlists.dtx}
         
 \section{List, and related environments}

 The generic commands for creating an indented environment --
 |enumerate|, |itemize|, |quote|, etc -- are:
 \begin{quote}
        |\list|\marg{LABEL}\marg{COMMANDS} ... |\endlist|
 \end{quote}

 which can be invoked by the user as the list environment.  The LABEL
 argument specifies item labeling.  COMMANDS contains commands for
 changing the horizontal and vertical spacing parameters.

 Each item of the environment is begun by the command
 |\item[|ITEMLABEL|]|
 which produces an item labeled by ITEMLABEL.  If the argument is
 missing, then the LABEL argument of the |\list| command is used as the
 item label.

 The label is formed by putting |\makelabel|\marg{ITEMLABEL} in an hbox
 whose width is either its natural width or else |\labelwidth|,
 whichever is larger.  The |\list| command defines |\makelabel| to have
 the default  definition:
 \begin{quote}
     |\makelabel|\marg{ARG} == BEGIN |\hfil| ARG END
 \end{quote}
 which, for a label of width less than |\labelwidth|, puts the label
 flushright, |\labelsep| to the left of the item's text.  However,
 |\makelabel| can be |\let| to another command by the |\list|'s
 COMMANDS argument.

 A |\usecounter|\marg{foo} command in the second argument causes the
 counter \emph{foo} to be initialized to zero, and stepped by every
 |\item| command without an argument.  (|\label| commands within the
 list refer to this counter.)

 When you leave a list environment, returning either to an enclosing
 list or normal text mode, LaTeX begins a new paragraph if and only if
 you leave a blank line after the |\end| command.  This is accomplished
 by the |\@endparenv| command.

 Blank lines are ignored every other reasonable place--i.e.:
 \begin{itemize}
  \item  Between the |\begin{list}| and the first |\item|,
  \item  Between the |\item| and the text of that item.
  \item Between the end of the last item and the |\end{list}|.
 \end{itemize}

 For an environment like quotation, in which items are not labeled,
 the entire environment is a single item.  It is defined by
 letting |\quotation| == |\list{}{...}\item\relax|.  (Note the
 |\relax|, there in case the first character in the environment is a
 '['.)  The spacing parameters provide a great deal of flexability in
 designing the format, including the ability to let the indentation of
 the first paragraph be different from that of the subsequent ones.

 The trivlist environment is equivalent to a list environment
 whose second argument sets the following parameter values:
 \begin{description}
 \item[\cs{leftmargin} = 0:] causes no indentation of left margin
 \item[\cs{labelwidth} = 0:] see below for precise effect this has.
 \item[\cs{itemindent} = 0:] with a null label, makes first paragraph
        have no indentation.  Succeeding paragraphs have |\parindent|
        indentation.  To give first paragraph same indentation, set
        |\itemindent| = |\parindent| before the |\item[]|.
 \end{description}

 Every |\item| in a trivlist environment must have an argument---in
 many cases, this will be the null argument (|\item[]|).  The trivlist
 environment is mainly used for paragraphing environments, like
 verbatim, in which there is no margin change.  It provides the same
 vertical spacing as the list environment, and works reasonably well
 when it occurs immediately after an |\item| command in an enclosing
 list.



 \subsection{List and Trivlist}


 The following variables are used inside a list environment:
 \begin{description}
 \item[\cs{@totalleftmargin}] The distance that the prevailing left
     margin is indented from the outermost left margin,
 \item[\cs{linewidth}] The width of the current line.  Must be
     initialized to |\hsize|.
 \item[\cs{@listdepth}] A count for holding current list nesting depth.
 \item[\cs{makelabel}] A macro with a single argument, used to
   generate the label from the argument (given or implied)
   of the |\item| command. Initialized to |\@mklab| by the |\list|
   command.  This command must produce  some stretch---i.e., an
   |\hfil|.
 \item[\cs{@inlabel}] A switch that is false except between the time
   an |\item| is encountered and the time that \TeX{}
   actually enters horizontal mode.  Should be tested by commands
   that can be messed up by the list environment's use of |\everypar|.
 \item[\cs{box}\cs{@labels}] When |@inlabel = true|, it holds the labels
   to be put out by |\everypar|.
 \item[\texttt{@noparitem}] A switch set by |\list| when
   |@inlabel = true|.
   Handles the case of a |\list| being the first thing in an item.
 \item[\texttt{@noparlist}] A switch set true for a list that begins an
   item.  No |\topsep| space is added before or after |\item|'s such a
   list.
 \item[\texttt{@newlist}] Set true by |\list|, set false by the first
   text (by |\everypar|).
 \item[\texttt{@noitemarg}]  Set true when executing an |\item| with no
   explicit argument.  Used to save space. To save time, make two
   separate  |\@item| commands.
 \item[\texttt{@nmbrlist}] Set true by |\usecounter| command, causes
   list to be numbered.
 \item[\cs{@listctr}] |\def|'ed by |\usecounter| to name of counter.
 \item[\cs{@noskipsec}] A switch set true by a sectioning command when
    it is creating an in-text heading with |\everypar|.
 \end{description}


 Throughout a list environment, |\hsize| is the width of the current
 line, measured from the outermost left margin to the outermost right
 margin.  Environments like tabbing should use |\linewidth| instead of
 |\hsize|.

 Here are the parameters of a list that can be set by commands in
 the |\list|'s COMMANDS argument.  These parameters are all TeX
 skips or dimensions (defined by |\newskip| or |\newdimen|), so the
 usual \TeX\ or \LaTeX\ commands can be used to set them.  The
 commands will be executed in vmode if and only if the |\list| was
 preceded by a |\par| (or something like an |\end{list}|), so the
 spacing parameters can be set according to whether the list is
 inside a paragraph or is its own paragraph.


 \subsection{Vertical Spacing (skips)}
 \begin{description}
 \item[\cs{topsep}:]  Space between first item and preceding paragraph.
 \item[\cs{partopsep}:] Extra space added to \cs{topsep} when
        environment starts a new paragraph (is called in vmode).
 \item[\cs{itemsep}:] Space between successive items.
 \item[\cs{parsep}:] Space between paragraphs within an item -- the
                 \cs{parskip} for this environment.
 \end{description}

 \subsection{Penalties}
 \begin{description}

 \item[\cs{@beginparpenalty}:] put at the beginning of a list
 \item[\cs{@endparpenalty}:] put at end of list
  \item[\cs{@itempenalty}:] put between items.
  \end{description}

 \subsection{Horizontal Spacing (dimens)}
 \begin{description}
 \item[\cs{leftmargin}:] space between left margin of enclosing
   environment (or of page if top level list) and left margin of
                     this list.  Must be nonnegative.
  \item[\cs{rightmargin}:] analogous.
  \item[\cs{listparindent}:] extra indentation at beginning of every
     paragraph of a list except the one started by the \cs{item}
                      command.  May be negative!  Usually, labeled
                       lists have \cs{listparindent} equal to zero.
   \item[\cs{itemindent}:] extra indentation added right BEFORE an item
                      label.
  \item[\cs{labelwidth}:] nominal width of box that contains the label.
                      If the natural width of the
                         label $< =$ \cs{labelwidth},
                      then the label is flushed right inside a box
                      of width \cs{labelwidth} (with an \cs{hfil}).
                      Otherwise,
                      a box of the natural width is employed, which
                       causes an indentation of the text on that line.
     \item[\cs{labelsep}:] space between end of label box and text of
                      first item.
  \end{description}





 \subsection{Default Values}
 
 Defaults for the list environment are set as follows.
 First, \cs{rightmargin}, \cs{listparindent} and \cs{itemindent}
 are set
      to 0pt.  Then, one of the commands
      \cs{@listi}, \cs{@listii}, ... , \cs{@listvi}
      is called, depending upon the current level of the list.
      The \cs{@list} \ldots commands should be defined by the document
      style.  A convention that the document style should follow is
      to set \cs{leftmargin} to
      \cs{leftmargini},\ldots, \cs{leftmarginvi} for
      the appropriate level.  Items that aren't changed may be left
      alone, but everything that could possibly be changed must be
      reset.


\LinesNumbered
\begin{algorithm}
\caption{The \cs{list} environment}
  \cs{list}\marg{LABEL}\marg{COMMANDS} ==\\
   \Begin{
     \eIf{\cs{@listdepth} > 5}{
        LaTeX error: 'Too deeply nested'}{
        \cs{@listdepth} :=G \cs{@listdepth} + 1\\
     }
     \cs{rightmargin}     := 0pt\\
     \cs{listparindent}   := 0pt\\
     \cs{itemindent}      := 0pt\\
     eval(@list \cs{romannumeral}\cs{the}\cs{@listdepth})\\  
     \cs{@itemlabel}      :=L LABEL\\
     \cs{makelabel}       == \cs{@mklab}\\
     @nmbrlist        :=L false\\
     COMMANDS\\
     \cs{@trivlist}\\  
     \cs{parskip}          :=L \cs{parsep}\\
     \cs{parindent}        :=L \cs{listparindent}\\
     \cs{linewidth}        :=L \cs{linewidth} - \cs{rightmargin} -\cs{leftmargin}\\
     \cs{@totalleftmargin} :=L \cs{@totalleftmargin} + \cs{leftmargin}\\
     \cs{parshape} 1 \cs{@totalleftmargin} \cs{linewidth}\\
     \cs{ignorespaces} 
   }
\end{algorithm}

Th \cs{endlist} simply adjusts the listdepth and ends the \cs{trivlist}.

\begin{algorithm}
 \cs{endlist} == \\
  \Begin{
    \cs{@listdepth} :=G \cs{@listdepth} -1\\
    \cs{endtrivlist}\\
  }
\end{algorithm}

The \cs{@trivlist} is define as,

\begin{algorithm}
 \cs{@trivlist} ==\\
  \Begin{
    \If{@newlist = T}{\cs{@noitemerr}}
     This command removed for some forgotten reason.\\
     \cs{@topsepadd} :=L \cs{topsep}\\
     \If{@noskipsec}{leave vertical mode}
     \eIf{vertical mode}{
        \cs{@topsepadd} :=L \cs{@topsepadd} + \cs{partopsep}}{
        \cs{unskip} \cs{par}}
     \eIf{@inlabel = true}{
         @noparitem :=L true
         @noparlist :=L true}{
         @noparlist :=L false
             \cs{@topsep}   :=L \cs{@topsepadd}}
     \cs{@topsep}      :=L \cs{@topsep} + \cs{parskip}\\
      Restore paragraphing parameters\\
     \cs{leftskip}     :=L 0pt\\  
     \cs{rightskip}    :=L \cs{@rightskip}\\
     \cs{parfillskip}     :=L 0pt + 1fil\\
   NOTE: \cs{@setpar} called on every \cs{list} in case \cs{par} has been\\
   temporarily  munged before the \cs{list} command.\\
     \cs{@setpar}{if @newlist = false then {\@@par} fi}\\
     \cs{@newlist}         :=G T\\
     \cs{@outerparskip}    :=L \cs{parskip}\\
 }
\end{algorithm}

\begin{algorithm}
 \cs{trivlist}  ==\\
 \Begin{
  \cs{parsep} := \cs{parskip}\\
   @nmbrlist := F\\
  \cs{@trivlist}\\
  \cs{labelwidth} := 0\\
  \cs{leftmargin} := 0\\
  \cs{itemindent} := \cs{parindent}\\
  \cs{@itemlabel} :=L "empty"\\ 
  \cs{makelabel}\marg{LABEL} == LABEL\\
 }
\end{algorithm}



\begin{algorithm}
 \cs{endtrivlist} ==\\
 \Begin{
     \If{@inlabel = T}{\cs{indent}}
     \If{horizontal mode}{\cs{unskip} \cs{par}}
     \eIf{@noparlist = true}{}{
        \If{\cs{lastskip} > 0}{
              \cs{@tempskipa} := \cs{lastskip}
              \cs{vskip} - \cs{lastskip}
              \cs{vskip} \cs{@tempskipa} -\cs{@outerparskip} + \cs{parskip}
             }
           \cs{@endparenv}
     }
   }
\end{algorithm}


\begin{algorithm}
 \cs{@endparenv} ==
   \Begin{
    \cs{addpenalty}\marg{@endparpenalty}\\
    \cs{addvspace}\marg{\cs{@topsepadd}}\\
     ends the \cs{begin} command's \cs{begingroup}\\
    \cs{endgroup}\\
    \cs{par}  ==  \Begin{%
                  \cs{@restorepar}\\
                  \cs{everypar}|{}|
                  \cs{par}
                  }
    \cs{everypar} == \Begin{remove \cs{lastbox} \cs{everypar}|{}|}
   to match the \cs{end} commands \cs{endgroup}\\
    \cs{begingroup}  
   }
\end{algorithm}

\index{kernel>lists>\textbackslash item}
The definition of item, is fairly simple deferring the complexity
to |\@item| which follows.

\begin{algorithm}[htbp]
 \cs{item} == \Begin{
    \If{math mode}{issue warning}
    \eIf{ next char = {\tt [}}{
             \cs{@item}}{
              \cs{@noitemarg} := true\\
             \cs{@item}[\cs{@itemlabel}]}
  }
\caption{The algorithm for \textbackslash item}
\end{algorithm}


\begin{algorithm}
 \cs{@item}[LAB] ==
    \Begin{
     \eIf{@noparitem = true}{
        @noparitem := false
             % NOTE: then clause  hardly every taken,\\
             %  so made a macro \cs{@donoparitem}\\
            \cs{box}\cs{@labels} :=G\\
             \cs{hbox} \Begin{\cs{hskip} -\cs{leftmargin}\\
                                   \cs{box}\cs{@labels}\\
                                   \cs{hskip} \cs{leftmargin}}
            \If{@minipage = false}{
               \cs{@tempskipa} := \cs{lastskip}\\
               \cs{vskip} -\cs{lastskip}\\
               \cs{vskip} \cs{@tempskipa} + \cs{@outerparskip} - \cs{parskip}}
            }{
          \If{@inlabel = true}{
              then \cs{indent} \cs{par}   % previous item empty.
           }
           \If{hmode}{then 2 \cs{unskip}'s\\
                           % To remove any space at end of prev.\\
                           % paragraph that could cause a blank line.\\
                     \cs{par}\\
           }
           \eIf{if @newlist = T}{
                \eIf{@nobreak = T}{ 
                      % Kludge if list follows \cs{section}\\
                      \cs{addvspace}\marg{\cs{@outerparskip} - \cs{parskip}}}{
                       \cs{addpenalty}\marg{\cs{@beginparpenalty}}\\
                       \cs{addvspace}\marg{\cs{@topsep}}\\
                       \cs{addvspace}\marg{-\cs{parskip}}\\  
                    }}{
                \cs{addpenalty}\marg{\cs{@itempenalty}}\\
                \cs{addvspace}\marg{\cs{itemsep}}\\
            }

            @inlabel :=G true\\
     }

     \cs{everypar}\{ @minipage :=G F\\
                @newlist :=G F\\
                \If{@inlabel = true}{
                    @inlabel :=G false
                       \cs{hskip} -\cs{parindent}\\
                       \cs{box}\cs{@labels}\\
                       \cs{penalty} 0\\
                       \cs{box}\cs{@labels} :=G null\\
                }
                \cs{everypar}\{\} \}\\
     @nobreak :=G false\\
     \If{@noitemarg = true}{
        @noitemarg := false\\
            \If{@nmbrlist}{
               \cs{refstepcounter}\{\cs{@listctr}\}
            }
     }
     \cs{@tempboxa}   :=L \cs{hbox}\marg{\cs{makelabel}\marg{LAB}}\\
     \cs{box}\cs{@labels} :=G \cs{@labels}\cs{hskip}\cs{itemindent}\\
                       \cs{hskip} - (\cs{labelwidth} + \cs{labelsep})\\
                 \eIf{\cs{wd}\cs{@tempboxa} > \cs{labelwidth}}{
                          \cs{box}\cs{@tempboxa}}{
                          \cs{hbox} to \cs{labelwidth}\{\cs{unhbox}\cs{@tempboxa}\}\\
                 } 
      \cs{hskip}\cs{labelsep}\\
     \cs{ignorespaces} %gobble space up to text
  }
\end{algorithm}




\begin{algorithm}
 \cs{makelabel}\marg{LABEL} == ERROR\\
    default to catch lonely \cs{item}

 \cs{usecounter}\marg{CTR} == \Begin{
                              \cs{@nmbrlist} :=L true\\
                              \cs{@listctr} == CTR\\
                              \cs{setcounter}\marg{CTR}\marg{0}
                             }
\end{algorithm}



 DEFINE \cs{dimen}'s and \cs{count}

 \begin{macro}{\topskip}
 \begin{macro}{\partopsep}
 \begin{macro}{\itemsep}
 \begin{macro}{\parsep}
 \begin{macro}{\@topsep}
 \begin{macro}{\@topsepadd}
 \begin{macro}{\outerparskip}
    \begin{teX}
\newskip\topsep
\newskip\partopsep
\newskip\itemsep
\newskip\parsep
\newskip\@topsep
\newskip\@topsepadd
\newskip\@outerparskip
    \end{teX}
 \end{macro}\end{macro}\end{macro}\end{macro}\end{macro}\end{macro}
 \end{macro}
 \begin{macro}{\leftmargin}\begin{macro}{\rightmargin}
 \begin{macro}{\listparindent}\begin{macro}{\itemindent}
 \begin{macro}{\labelwidth}\begin{macro}{\labelsep}
 \begin{macro}{\@totalleftmargin}
    \begin{teX}
\newdimen\leftmargin
\newdimen\rightmargin
\newdimen\listparindent
\newdimen\itemindent
\newdimen\labelwidth
\newdimen\labelsep
\newdimen\linewidth
\newdimen\@totalleftmargin \@totalleftmargin=\z@
    \end{teX}
 \end{macro}\end{macro}\end{macro}\end{macro}\end{macro}
 \end{macro}\end{macro}

 \begin{macro}{\leftmargini}
 \begin{macro}{\leftmarginii}
 \begin{macro}{\leftmarginiii}
 \begin{macro}{\leftmarginiv}
 \begin{macro}{\leftmarginv}
 \begin{macro}{\leftmarginvi}
    \begin{teX}
\newdimen\leftmargini
\newdimen\leftmarginii
\newdimen\leftmarginiii
\newdimen\leftmarginiv
\newdimen\leftmarginv
\newdimen\leftmarginvi
    \end{teX}
 \end{macro}\end{macro}\end{macro}
 \end{macro}\end{macro}\end{macro}

 \begin{macro}{\@listdepth}\begin{macro}{\@itempenalty}
 \begin{macro}{\@beginparpenalty}\begin{macro}{\@endparpenalty}
    \begin{teX}
\newcount\@listdepth \@listdepth=0
\newcount\@itempenalty
\newcount\@beginparpenalty
\newcount\@endparpenalty
    \end{teX}
 \end{macro}\end{macro}\end{macro}\end{macro}

 \begin{macro}{\@labels}
    \begin{teX}
\newbox\@labels
    \end{teX}
 \end{macro}

 \begin{macro}{\if@inlabel}
 \begin{macro}{\@inlabelfalse}
 \begin{macro}{\@inlabeltrue}
    \begin{teX}
\newif\if@inlabel \@inlabelfalse
    \end{teX}
 \end{macro}\end{macro}\end{macro}

 \begin{macro}{\if@newlist}
 \begin{macro}{\@newlistfalse}
 \begin{macro}{\@newlisttrue}
    \begin{teX}
\newif\if@newlist   \@newlistfalse
    \end{teX}
 \end{macro}\end{macro}\end{macro}

 \begin{macro}{\if@noparitem}
 \begin{macro}{\@noparitemfalse}
 \begin{macro}{\@noparitemtrue}
    \begin{teX}
\newif\if@noparitem \@noparitemfalse
    \end{teX}
 \end{macro}\end{macro}\end{macro}

 \begin{macro}{\if@noparlist}
 \begin{macro}{\@noparlistfalse}
 \begin{macro}{\@noparlisttrue}
    \begin{teX}
\newif\if@noparlist \@noparlistfalse
    \end{teX}
 \end{macro}\end{macro}\end{macro}

 \begin{macro}{\if@noitemarg}
 \begin{macro}{\@noitemargfalse}
 \begin{macro}{\@noitemargtrue}
    \begin{teX}
\newif\if@noitemarg \@noitemargfalse
    \end{teX}
 \end{macro}\end{macro}\end{macro}

 \begin{macro}{\if@newlist}
 \begin{macro}{\@newlistfalse}
 \begin{macro}{\@newlisttrue}
    \begin{teX}
\newif\if@nmbrlist  \@nmbrlistfalse
    \end{teX}
 \end{macro}\end{macro}\end{macro}

 \begin{macro}{\list}
\index{kernel>lists>\textbackslash list}
 List takes two arguments and is an author command for building
other lists.
    \begin{teX}
\def\list#1#2{%
  \ifnum \@listdepth >5\relax
    \@toodeep
  \else
    \global\advance\@listdepth\@ne
  \fi
  \rightmargin\z@
  \listparindent\z@
  \itemindent\z@
  \csname @list\romannumeral\the\@listdepth\endcsname
  \def\@itemlabel{#1}%
  \let\makelabel\@mklab
  \@nmbrlistfalse
  #2\relax
  \@trivlist
  \parskip\parsep
  \parindent\listparindent
  \advance\linewidth -\rightmargin
  \advance\linewidth -\leftmargin
  \advance\@totalleftmargin \leftmargin
  \parshape \@ne \@totalleftmargin \linewidth
  \ignorespaces}
    \end{teX}
 \end{macro}

 \begin{macro}{\par@deathcycles}
    \begin{teX}
\newcount\par@deathcycles
    \end{teX}
 \end{macro}

 \begin{macro}{\@trivlist}

 Because |\par| is sometimes made a no-op it is possible for a missing
 |\item| to produce a loop that does not fill memory and so never gets
 trapped by \TeX.  We thus need to trap this here by seting |\par| to
 count the number of times a paragraph ii is called with no progress
 being made started.
    \begin{teX}
\def\@trivlist{%
  \if@noskipsec \leavevmode \fi
  \@topsepadd \topsep
  \ifvmode
    \advance\@topsepadd \partopsep
  \else
    \unskip \par
  \fi
  \if@inlabel
    \@noparitemtrue
    \@noparlisttrue
  \else
    \if@newlist \@noitemerr \fi
    \@noparlistfalse
    \@topsep \@topsepadd
  \fi
  \advance\@topsep \parskip
  \leftskip \z@skip
  \rightskip \@rightskip
  \parfillskip \@flushglue
  \par@deathcycles \z@
  \@setpar{\if@newlist
             \advance\par@deathcycles \@ne
             \ifnum \par@deathcycles >\@m
               \@noitemerr
               {\@@par}%
             \fi
           \else
             {\@@par}%
           \fi}%
  \global \@newlisttrue
  \@outerparskip \parskip}
    \end{teX}
 \end{macro}

 
 \begin{macro}{\trivlist}
    \begin{teX}
\def\trivlist{%
  \parsep\parskip
  \@nmbrlistfalse
  \@trivlist
  \labelwidth\z@
  \leftmargin\z@
  \itemindent\z@
    \end{teX}

    We initialise |\@itemlabel| so that a \texttt{trivlist} with
    an |\item| not having an optional argument doesn't produce an
    error message.
 \changes{latex2e}{1993/12/13}{Initialised \cs{@itemlabel}}
    \begin{teX}
  \let\@itemlabel\@empty
  \def\makelabel##1{##1}}
    \end{teX}
 \end{macro}

 \begin{macro}{\endlist}
    \begin{teX}
\def\endlist{%
  \global\advance\@listdepth\m@ne
  \endtrivlist}
    \end{teX}
 \end{macro}

    The definition of \cs{trivlist} used to be in ltspace.dtx 
    so that other commands could be `let to it'.  
    They now use \cs{def}.

 \begin{macro}{\endtrivlist}
 \changes{v1.2b ltspace}{1994/11/12}{Changed order of tests to make
 \cs{@noitemerror} correct: end of an era.}
 \changes{v1.0i}{1995/05/25}{Macros moved from ltspace.dtx}
 \changes{v1.0n}{1996/10/25}{Change \cs{indent} to \cs{leavevmode}}
 \changes{v1.0n}{1996/10/25}{Reset flags explicitly}
 \changes{v1.0o}{1996/10/26}{Correct typo}
    \begin{teX}
\def\endtrivlist{%
  \if@inlabel
    \leavevmode
    \global \@inlabelfalse
  \fi
  \if@newlist
    \@noitemerr
    \global \@newlistfalse
  \fi
  \ifhmode\unskip \par
    \end{teX}
    We also check if we are in math mode and issue an error message
    if so (hoping that |\@currenvir| resolves suitably). Otherwise
    the usual ``perhaps a missing item'' error will get triggered
    later which is confusing.
 \changes{v1.0s}{2002/10/28}{Check for math mode (pr/3437)}
    \begin{teX}
  \else
    \@inmatherr{\end{\@currenvir}}%
  \fi
  \if@noparlist \else
    \ifdim\lastskip >\z@
      \@tempskipa\lastskip \vskip -\lastskip
      \advance\@tempskipa\parskip \advance\@tempskipa -\@outerparskip
      \vskip\@tempskipa
    \fi
    \@endparenv
  \fi
}
    \end{teX}
 \end{macro}
 

 
 \begin{macro}{\@endparenv}
 \begin{macro}{\@doendpe}
 To suppress the paragraph indentation in text immediately following
 a paragraph-making environment, \cs{everypar} is changed to remove the
 space, and \cs{par} is redefined to restore \cs{everypar}.  Instead of
 redefining \cs{par} and \cs{everypar}, \cs{@endparenv} was changed to 
 set the @endpe switch, letting \cs{end} redefine \cs{par} and 
 \cs{everypar}.  

 This allows paragraph-making environments to work right when called 
 by other environments. (Changed 27 Oct 86)
    \begin{teX}
\def\@endparenv{%
  \addpenalty\@endparpenalty\addvspace\@topsepadd\@endpetrue}
    \end{teX}

    \begin{teX}
\def\@doendpe{\@endpetrue
     \def\par{\@restorepar\everypar{}
          \par\@endpefalse}\everypar
    \end{teX}
    
    Use |\setbox0=\lastbox| instead of   |\hskip -\parindent|   
    so that a \cs{noindent} becomes a no-op when used before 
    a line immediately following a list environment(23 Oct 86).
 \changes{v1.0k}{1995/11/07}{Enclosed \cs{setbox0} assignment by a
 group so that it leaves the contents of box $0$ intact.
    } 
    \begin{teX}
               {{\setbox\z@\lastbox}\everypar{}\@endpefalse}}
    \end{teX}
 \end{macro}
 \end{macro}

 
 \begin{macro}{\if@endpe}
 \begin{macro}{\@endpefalse}
 \begin{macro}{\@endpeltrue}
    \begin{teX}
\newif\if@endpe
\@endpefalse
    \end{teX}
 \end{macro}\end{macro}\end{macro}

 
 \begin{macro}{\@mklab}
    \begin{teX}
\def\@mklab#1{\hfil #1}
    \end{teX}
 \end{macro}

 \changes{LaTeX2.09}{1992/09/18}
     {(RmS) Added warning if \cs{item} is used in math mode}
 \changes{v1.0c}{1994/04/28}
     {Replaced \cs{@ltxnomath} by \cs{@inmatherr}}
 \changes{v1.0d}{1994/05/03}
     {Removed superfluous braces}
 \begin{macro}{\item}
    \begin{teX}
\def\item{%
  \@inmatherr\item
  \@ifnextchar [\@item{\@noitemargtrue \@item[\@itemlabel]}}
    \end{teX}
 \end{macro}
 \begin{macro}{\@donoparitem}
    \begin{teX}
\def\@donoparitem{%
  \@noparitemfalse
  \global\setbox\@labels\hbox{\hskip -\leftmargin
                               \unhbox\@labels
                                \hskip \leftmargin}%
  \if@minipage
    \else
      \@tempskipa\lastskip
      \vskip -\lastskip
      \advance\@tempskipa\@outerparskip
      \advance\@tempskipa -\parskip
      \vskip\@tempskipa
  \fi}
    \end{teX}
 \end{macro}

 \begin{macro}{\@item}
 \changes{v1.0l}{1996/07/26}{Remove unecessary \cs{global} before
                 \cs{@minipage...}}
    \begin{teX}
\def\@item[#1]{%
  \if@noparitem
    \@donoparitem
  \else
    \if@inlabel
      \indent \par
    \fi
    \ifhmode
      \unskip\unskip \par
    \fi
    \if@newlist
      \if@nobreak
        \@nbitem
      \else
        \addpenalty\@beginparpenalty
        \addvspace\@topsep
        \addvspace{-\parskip}%
      \fi
    \else
      \addpenalty\@itempenalty
      \addvspace\itemsep
    \fi
    \global\@inlabeltrue
  \fi
  \everypar{%
    \@minipagefalse
    \global\@newlistfalse
    \end{teX}
    This |\if@inlabel| check is needed in case an item starts of
    inside a group so that |\everypar| does not become empty
    outside that group. 
 \@nobreakfalse, etc etc.
    \begin{teX}
    \if@inlabel
      \global\@inlabelfalse
    \end{teX}
    The paragraph indent is now removed by using |\setbox...| since
    this makes |\noindent| a no-op here, as it should be. Thus the
    following comment is redundant but is left here for the sake of
    future historians:
    this next command was changed from an hskip to a kern to avoid
    a break point after the parindent box: the skip could cause a
    line-break if a very long label occurs in raggedright setting.
 \changes{v1.0d}{1994/05/03}{\cs{hskip} changed to \cs{kern}}
 \changes{v1.0m}{1996/10/23}{\cs{kern...} changed to \cs{setbox...}}
 \changes{v1.0r}{1997/02/21}
    {\cs{ifvoid} check added for \cs{noindent}. latex/2414}
 If |\noindent| was used after |\item| want to cancel the |\itemindent|
 skip. This case can be detected as the indentation box will be void.
    \begin{teX}
      {\setbox\z@\lastbox
       \ifvoid\z@
         \kern-\itemindent
       \fi}%
    \end{teX}

    \begin{teX}
      \box\@labels
      \penalty\z@
    \fi
    \end{teX}
    This code is intended to prevent a page break after the first
    line of an item that comes immediately after a section title. It
    may be sensible to always forbid a page break after one line of
    an item?  As with all such settings of |\clubpenalty| it is local
    so will have no effect if the item starts in a group.

    Only resetting |\@nobreak| when it is true is now
    essential since now it is sometimes set locally.
 \changes{v1.0m}{1996/10/23}{Added setting of \cs{clubpenalty} and
    set \cs{@nobreakfalse} only when necessary}
    \begin{teX}
    \if@nobreak
      \@nobreakfalse
      \clubpenalty \@M
    \else
      \clubpenalty \@clubpenalty
      \everypar{}%
    \fi}%
    \end{teX}
 \changes{v1.0l}{1996/07/26}{Remove unecessary \cs{global} before
                 \cs{@nobreak...}}
 \changes{v1.0m}{1996/10/23}{\cs{@nobreak...} moved into the
          \cs{everypar} and not executed unconditionally, see above} 
    \begin{teX}
  \if@noitemarg
    \@noitemargfalse
    \if@nmbrlist
    \end{teX}
 \changes{v1.0g}{1995/05/17}{Removed surplus braces}
    \begin{teX}
      \refstepcounter\@listctr
    \fi
  \fi
    \end{teX}
    We use |\sbox| to support colour commands.
 \changes{LaTeX2e}{1993/12/08}{use \cs{sbox} to support colour}
    \begin{teX}
  \sbox\@tempboxa{\makelabel{#1}}%
  \global\setbox\@labels\hbox{%
    \unhbox\@labels
    \hskip \itemindent
    \hskip -\labelwidth 
    \hskip -\labelsep
    \ifdim \wd\@tempboxa >\labelwidth
      \box\@tempboxa
    \end{teX}
 \changes{LaTeX2.09}{1991/11/22}
         {(RmS) Changed second call to \cs{makelabel} to
           \cs{unhbox}\cs{@tempboxa}.
          Avoids problems with side effects in \cs{makelabel} and is
               more efficient.}
    \begin{teX}
    \else
      \hbox to\labelwidth {\unhbox\@tempboxa}%
    \fi
    \hskip \labelsep}%
  \ignorespaces}
    \end{teX}
 \end{macro}

 \begin{macro}{\makelabel}
 \changes{LaTeX2.09}{1991/11/04}
         {(RmS) added default definition for \cs{makelabel},
               to produce an error message.}
    \begin{teX}
\def\makelabel#1{%
  \@latex@error{Lonely \string\item--perhaps a missing
        list environment}\@ehc}
    \end{teX}
 \end{macro}

 \begin{macro}{\@nbitem}
 \changes{v1.0g}{1995/05/17}{Removed surplus braces}
    \begin{teX}
\def\@nbitem{%
  \@tempskipa\@outerparskip
  \advance\@tempskipa -\parskip
  \addvspace\@tempskipa}
    \end{teX}
 \end{macro}

 \begin{macro}{\usecounter}
    \begin{teX}
\def\usecounter#1{\@nmbrlisttrue\def\@listctr{#1}\setcounter{#1}\z@}
    \end{teX}
 \end{macro}


 \subsection{Itemize and Enumerate}

  Enumeration is done with four counters: |enumi|, |enumii|, |enumiii|
  and |enumiv|, where |enum|N controls the numbering of the Nth level
  enumeration.  The label is generated by the commands
  \cs{labelenumi} \ldots{} \cs{labelenumiv}, which should be defined
  by the document style.
  Note that \cs{p@enum}N\cs{theenum}N defines the output
  of a \cs{ref} command.  A typical definition might be:
 \begin{verbatim}
     \def\theenumii{\alph{enumii}}
     \def\p@enumii{\theenumi}
     \def\labelenumii{(\theenumii)}
 \end{verbatim}
 which will print the labels as `(a)', `(b)', \ldots
 and print a \cs{ref} as `3a'.

 The item numbers are moved to the right of the label box, so they are
 always a distance of \cs{labelsep} from the item.

 \cs{@enumdepth} holds the current enumeration nesting depth.

 Itemization is controlled by four commands: \cs{labelitemi},
 \cs{labelitemii},
 \cs{labelitemiii}, and \cs{labelitemiv}.
 To cause the second-level list to be
 bulleted, you just define \cs{labelitemii}
 to be $\bullet$.  \cs{@itemspacing}
 and \cs{@itemdepth} are the analogs of \cs{@enumspacing} and
 \cs{@enumdepth}.

 \begin{teX}
 \enumerate ==
   BEGIN
     if \@enumdepth > 3
       then errormessage: ``Too deeply nested''.
       else \@enumdepth :=L \@enumdepth + 1
            \@enumctr :=L eval(enum@\romannumeral\the\@enumdepth)
            \list{\label(\@enumctr)}
                 {\usecounter{\@enumctr}
                  \makelabel{LABEL} ==  \hss \llap{LABEL}}
     fi
   END

 \endenumerate == \endlist
 \end{teX}

 \begin{macro}{\@enumdepth}
    \begin{teX}
\newcount\@enumdepth \@enumdepth = 0
    \end{teX}
 \end{macro}

 \begin{macro}{\c@enumi}
 \begin{macro}{\c@enumii}
 \begin{macro}{\c@enumii}
 \begin{macro}{\c@enumiv}
    \begin{teX}
\@definecounter{enumi}
\@definecounter{enumii}
\@definecounter{enumiii}
\@definecounter{enumiv}
    \end{teX}
 \end{macro}
 \end{macro}
 \end{macro}
 \end{macro}

 \begin{environment}{enumerate}
  The enumerate environment enumerates the list. The macro
  is written very efficiently and defines the basic structure. The
  typesetting parameters are left for the class files such as book
  to define them.
     \begin{teX}
\def\enumerate{%
  \ifnum \@enumdepth >\thr@@\@toodeep\else
    \advance\@enumdepth\@ne
    \edef\@enumctr{enum\romannumeral\the\@enumdepth}%
      \expandafter
      \list
        \csname label\@enumctr\endcsname
        {\usecounter\@enumctr\def\makelabel##1{\hss\llap{##1}}}%
  \fi}
    \end{teX}

    \begin{teX}
\let\endenumerate =\endlist
    \end{teX}
 \end{environment}


 \begin{teX}
  \itemize ==
    BEGIN
      if \@itemdepth > 3
        then  errormessage: 'Too deeply nested'.
        else \@itemdepth :=L \@itemdepth + 1
             \@itemitem  == eval(labelitem\romannumeral\the\@itemdepth)
             \list{\@nameuse{\@itemitem}}
                   {\makelabel{LABEL} ==  \hss \llap{LABEL}}
      fi
    END

  \enditemize ==  \endlist

 \end{teX}

 \begin{macro}{\@itemdepth}
    \begin{teX}
\newcount\@itemdepth \@itemdepth = 0
    \end{teX}
 \end{macro}

 \begin{environment}{itemize}
     \begin{teX}
\def\itemize{%
  \ifnum \@itemdepth >\thr@@\@toodeep\else
    \advance\@itemdepth\@ne
    \edef\@itemitem{labelitem\romannumeral\the\@itemdepth}%
    \end{teX}
    
 \footnote{changes v1.0j 1995/07/09 Use \textbackslash expandafter. Hard to believe the team missed it!}
    \begin{teX}
    \expandafter
    \list
      \csname\@itemitem\endcsname
      {\def\makelabel##1{\hss\llap{##1}}}%
  \fi}
    \end{teX}

    \begin{teX}
\let\enditemize =\endlist
    \end{teX}
 \end{environment}




}


 %\ttreport
\def\textdocs{
  \parindent1em

\chapter{Paragraphs and Lists}

\epigraph{The paragraph is essentially a unit of thought, not a length}{H.W.Fowler (1858-1933)}

\noindent Paragraphs represent a distinct logical step within the whole argument expounded in section of a document. The \texttt{Tufte-book} class has a control sequence that is named \cmd{\newthought} to reinforce the idea that a paragraph must start with a new thought or argument. How does this particular paragraph contribute to the argument? 
What logical step does it make? Where does it fit in the overall chain?

\section{Historical Notes}

The oldest mark of punctuation in Greek manuscripts is the paragraph. It first occurs as a horizontal stroke (sometimes with a dot over it), placed at the beginning of a line, just beneath the first two or three letters.

This was followed by the paragph mark the pilcrow\footnote{See also \protect\url{http://www.smithsonianmag.com/arts-culture/the-origin-of-the-pilcrow-aka-the-strange-paragraph-symbol-8610683/?no-ist}} (\S). 

After the establishment of indentation the method of marking paragraphs becomes essentially what we find today. At first the old mark was still use for emphasis. But this custom was short-lived.

In the eighteenth century it was a printer’s custom to print the first word of each paragraph in capitals. 

It remains to consider the origin of the so-called section 
mark [\S], called on the continent, \emph{paragraphe}. The genesis of 
this mark has been explained in two different ways. The first 
of these is equally ingenious and ingenuous. It is thus 
expressed in an American treatise on composition and rhetoric . 
" The Section [\S], the mark for which seems to be a combina- 
tion of two s's, standing for \emph{signum sectionis}, the sign of the 
section." The theory is still more definitely expounded  in 
`Quackenbos, Course of Composition and Rhetoric, p. 145'. 


\section{Typesetting paragraphs}

Typesetting paragraphs with \tex does not require any particular effort from the user, other than leaving a blank line to separate a paragraph from other page elements.

\begin{texexample}{Paragraph marking}{} 

In olden times when wishing
still helped one, there lived a
king whose daughters were all
beautiful, but the youngest was so
beautiful that the sun itself,
which has seen so much, was
astonished whenever it shone in
her face. 

Close by the king's
castle lay a great dark forest,
and under an old lime-tree in the
forest was a well, and when
the day was very warm, the
king's child went out into the 
forest and sat down by the side
of the cool fountain, and when she was bored she
took a golden ball, and threw it up on a high and caught it, and this
ball was her favorite plaything.

 This is a paragraph with Maths,
 \[d=a+b+c\]
 where $d=sum$.
\end{texexample}



\section{First line indentation and paragraph separation}

Good typography dictates that the first line of a paragraph is indented. \tex provides two commands that can be used for first line indentation. The first one is \cs{parindent} which is a length expressed normally in |ems|. The \cs{noindent} does what its name implies. The paragraph indentation is sometimes resisted by newcomers to \tex, however most professionally printed material in English typically does not indent the first paragraph, but indents those that follow. For example, Robert Bringhurst states that we should "Set opening paragraphs flush left."\footnote{Bringhurst, Robert (2005). \textit{The Elements of Typographic Style}. Vancouver: Hartley and Marks. p. 39. ISBN 0-88179-206-3.} Bringhurst explains as follows.

\enquote{The function of a paragraph is to mark a pause, setting the paragraph apart from what precedes it. If a paragraph is preceded by a title or subhead, the indent is superfluous and can therefore be omitted.}

The Elements of Typographic Style states that \enquote{at least one en [space]} should be used to indent paragraphs after the first, noting that that is the \enquote{practical minimum}. An em space is the most commonly used paragraph indent. Miles Tinker,\footcite{tinker1963} in his book Legibility of Print, concluded that indenting the first line of paragraphs increases readability by 7\%, on the average. Where longer lines of text are used it is not uncommon to indent the first paragraph line by at least two ems.

\begin{docCommand}{parindent} { \meta{dim} }
\end{docCommand}
\begin{docCommand}{parskip} {\meta{dim}}
\end{docCommand}
\begin{docCommand}{noindent}{}
\LaTeXe has basic parameters that control the appearance of normal paragraphs,
\cs{parindent} and  \cs{parskip}.
The length \cs{parindent}  is the indentation of the first line of a paragraph and the length
parskip is the vertical spacing between paragraphs, as illustrated in \ref{fig:paragaraph}. The
value of \cs{parskip} is usually 0pt, and \texttt{parindent} is usually defined in terms of \textit{ems}
so that the actual indentation depends on the font being used. If \texttt{parindent} is set to a
negative length, then the first line of the paragraphs will be \textit{outdented} into the lefthand
margin.
\end{docCommand}



\subsection{Block paragraph}

A block paragraph is obtained by setting \cs{parindent} to |0em|; \cs{parskip} should be set to
some positive value so that there is some space between paragraphs to enable them to be
identified. Most typographers heartily dislike block paragraphs, not only on aesthetical
grounds but also on practical considerations. Consider what happens if the last line of a
block paragraph is full and also is the last line on the page. The following block paragraph

It is important to know that \latex typesets paragraph by paragraph. For example, the
\cs{baselineskip} that is used for a paragraph is the value that is in effect at the end of the
paragraph, and the font size used for a paragraph is according to the size declaration (e.g.,
large or normalsize or small) at the end of the paragraph, and the raggedness or
otherwise of the whole paragraph depends on the declaration (e.g., \texttt{centering}) in effect
at the end of the paragraph. If a pagebreak occurs in the middle of a paragraph TeX will
not reset the part of the paragraph that goes onto the following page, even if the textwidths
on the two pages are different.


\subsection{Hanging paragraphs}
 
 \begin{docCommand}{hangafter}{\meta{number of lines}}
\begin{docCommand}{hangindent}{\meta{dim}}
A hanging paragraph is one where the length of the first few lines differs from the length
of the remaining lines. A normal indented paragraph may be considered to be a special case of a hanging paragraph where few is one. There are two commands controlling this \cs{hangafter} and \cs{hangindent}, which are both provided by \tex.
\end{docCommand}
\end{docCommand}


These commands can be used - very carefully to arrange wrapping figures
and drop capitals. 

\begin{texexample}{Hanging paragraphs}{}
\hangindent 8em  \hangafter 3  \footnotesize
Adeste hendecasyllabi. quot estis 
omnes. undique quotquot estis omnes. 
iocum me putat esse moecha turpis. 
et negat mihi nostra reddituram 
pugillaria si pati potestis. 
persequamur eam. et reflagitemus. 
quae sit quaeritis. illa quam uidetis 
turpe incedere mimice ac moleste 
ridentem catuli ore Gallicani. 
circumsistite eam. et reflagitate. 
moecha putida. redde codicillos. 
redde putida moecha codicillos. 
non assis facis. o lutum. lupanar, 
aut si perditius potest quid esse. 
sed non est tamen hoc satis putandum 
quod si non aliud potest ruborem 
ferreo canis exprimamus ore. 
conclamate iterum altiore uoce. 
moecha putide. redde codicillos. 
redde putida moecha moecha codicillos. 
sed nil proficimus. nihil mouetur. 
mutanda est ratio modusque uobis 
siquid proficere amplius potestis. 
pudica et proba. redde codicillos.

\end{texexample}


As you probably have guessed, this can be used to wrap figures into the text, although this is hardly necessary. 

Using \cs{hangindent} at the start of a paragraph will cause the paragraph to be hung.
If the length \meta{indent} is positive the lefthand end of the lines will be indented but
if it is negative the righthand ends will be indented by the specified amount. If the
number $num$, say $N$, is negative the first $N$ lines of the paragraph will be indented while
if $N$ is positive the $N+1$ the lines onwards will be indented. 

There should be no space between the  command and
the start of the paragraph. 


\section{Centering lines}

Lines can be centered using the \docAuxCommand{centerline} command. We can use it to center a small phrase commonly found
in typography \footnote{"The quick brown fox jumps over the lazy dog" is an English-language pangram (a phrase that contains all of the letters of the alphabet). It has been used to test typewriters and computer keyboards, and in other applications involving all of the letters in the English alphabet. Owing to its shortness and coherence, it has become widely known and is often used in visual arts.}.

\noindent\centerline{\small\fox}

We can achieve the same effect using \TeX\  primitives \cs{hfil} and writing \verb+\hfil\small\fox\hfil+
\medskip

{\hfil\small\fox\hfil}


This will give a slightly different center?

\section{Flush and Rugged}

Flushleft text has the lefthand end of the lines aligned vertically at the lefthand margin
and flushright text has the righthand end of the lines aligned vertically at the righthand
margin. The opposites of these are raggedleft text where the lefthand ends are not aligned
and raggedright where the righthand end of lines are not aligned. LaTeX normally typesets
flushleft and flushright.



{\small \begin{flushleft} \lorem \end{flushleft}}

{\small \begin{flushright} \lorem \end{flushright}}


\section{Centered text}
\latex provides an environment for centering blocks of text, as well as a single command \docAuxEnvironment{centering}. 

\begin{texexample}{Centering Text}{}
\begin{center}
In the beginning\\
Then God created Newton,\\
And objects at rest tended to remain at rest,\\
And objects in motion tended to remain in motion,\\
And energy was conserved and momentum was conserved and\\
matter was conserved\\
And God saw that it was conservative.\\
\end{center}
\end{texexample}


Text in a flushleft environment is typeset flushleft and raggedright, while in a
flushright environment is typeset raggedleft and flushright. In a center environment
the text is set raggedleft and raggedright, and each line is centered. A small vertical space
is put before and after each of these environments.


\section{Other paragraph styles}

\tex's paragraph builder can be accessed in \tex or \latex derived formats by boxing and unboxing the text and using the command \cmd{\lastbox} to manipulate the contents as an example consider the following problem:

\def\weirdtitle#1{%
       \bgroup
       \setbox0=\vbox{\bf\noindent #1}%
       \setbox1=\vbox{%
            \unvbox0
            \setbox2=\lastbox
            \hbox to \linewidth{\hfill\unhbox2 \hfill}%
       }%
       \unvbox1
      \egroup
  }%

\def\wavelast#1{%
       \bgroup
       \setbox0=\vbox{\bf\noindent #1}%
       \setbox1=\vbox{%
            \unvbox0
            \setbox2=\lastbox
            \hbox to \linewidth{\hfill\uwave{\unhbox2}\hfill}%
       }%
       \unvbox1
      \egroup
  }%
  
\begin{scriptexample}{example}{}
\weirdtitle{A Dialogue between the Landlady, and Susan the Chambermaid, proper to be
read by all Innkeepers, and their Servants; with the Arrival, and
affable Behaviour of a beautiful young Lady; which may teach Persons of
Condition how they may acquire the Love of the whole World.}
\end{scriptexample}

The example was from an old question at \tex{}MAG.\footnote{\url{http://dante.ctan.org/tex-archive/info/digests/tex-mag/v2.n2}.} The solutions offered varied but the one used here, is what was considered to be the most elegant. 

\begin{teXXX}
\def\weirdtitle#1{%
       \bgroup
       \setbox0=\vbox{\bf\noindent #1}%
       \setbox1=\vbox{%
            \unvbox0
            \setbox2=\lastbox
            \hbox to \linewidth{\hfill\unhbox2 \hfill}%
       }%
       \unvbox1
      \egroup
  }%
\end{teXXX}

The solution is to put the paragraph in a box |\box0| and then manipulate the contents in a second box |\box1|. In |box 1| we unvbox the box (causing it to be typeset) and then in yet a third box we pick the last line (\cmd{\lastbox}). This is then placed in a horizontal list and using appropriate glue we center the text. 

\subsection{Underlining the last line of the text}

\epigraph{“For pity’s sake, Laura,
don’t talk about anything so deadly as the bulbs}{\textsc{E.M. DELAFIELD}, \textit{The Way Things Are (1927)}}

The next example appears in a book by \citeauthor{tulipmania}.\footcite{tulipmania} This book is about the tulip market in the late 1630s, the meteoric rising in prices for tulips and the inevitable collapse that followed. Besides the craze for tulips at the time it became fashionable to wear the ruff. The ruff, which was worn by men, women and children, evolved from the small fabric ruffle at the drawstring neck of the shirt or chemise. They served as changeable pieces of cloth that could themselves be laundered separately while keeping the wearer's doublet or gown from becoming soiled at the neckline. The stiffness of the garment forced upright posture, and their impracticality led them to become a symbol of wealth and status. I digressed, just to give you a taste of what I think the book designer had in mind, when he decided to use wavy lines to underline portions of the text, in headings and captions. He also enclosed numbered pages in curly brackets, like so \{13\}. I found this a brilliant idea and if you have an opportunity borrow the book from your library and have a look. It is also well written and captivating. So from tulips in the middle seventeeth century, Arsenau's \pkg{ulem} and Knuth's \tex we can attempt to imitate the style.\index{paragraph>last line}

The last line of the caption is centered and a wavy line drawn underneath it.

\begin{texexample}{wavelast}{}
\wavelast{A Dialogue between the Landlady, and Susan the Chambermaid, proper to be
read by all Innkeepers, and their Servants; with the Arrival, and
affable Behaviour of a beautiful young Lady; which may teach Persons of
Condition how they may acquire the Love of the whole World.}
\end{texexample}

\emphasis{uwave}
\begin{teX}
\def\wavelast#1{%
       \bgroup
       \setbox0=\vbox{\bf\noindent #1}%
       \setbox1=\vbox{%
            \unvbox0
            \setbox2=\lastbox
            \hbox to \linewidth{\hfill\uwave{\unhbox2}\hfill}(*@\label{lin:uwave}@*)%
       }%
       \unvbox1
      \egroup
  }%
\end{teX}

What just happened is that in line [\ref{lin:uwave}] we unboxed the last line and then centered it, by using |\hfill| glue on each side. We then undelined it using a modified version of the command \cmd{\uwave} from the \pkgname{ulem} package.

In the book the last line is not really underlined, rather the underline is a ruler the width of the last line of the paragraph above it. I will come back to this example in the section for boxes, where we can measure the box and then be able to draw the wavy line. We can also probably get a better ruler by using \tikzname to draw the line.

\begin{figure}[bt]
\includegraphics[width=\linewidth]{tulip-spread}\par
{\leftskip-2em

\caption{Extract from Tulipmania \protect\fullcite{tulipmania}. Note the ruff worn by the couple and the wavy rules, separating the captions. The wavy rulers have a width equal to the width of the last line of the caption text. The figure names are denoted as \textsc{plates} and the captions are centered. We discuss how to achieve such captions later on in this book. Note also the captions allign with the bottom of the page. A \cmd{\vfill} can be placed in between the figure and the caption to achieve this. }\par}

\end{figure}

The |\hbox| can easily be changed to use \enquote{Russian style} last lines in paragraphs.

\begin{scriptexample}{example}{}
\def\russiantitlei#1{%
       \bgroup
       \setbox0=\vbox{\bf\noindent #1}%
       \setbox1=\vbox{%
            \unvbox0
            \setbox2=\lastbox
            \hbox to \linewidth{\hfill\unhbox2}%
       }%
       \unvbox1
      \egroup
  }%

\russiantitlei{A Dialogue between the Landlady, and Susan the Chambermaid, proper to be
read by all Innkeepers, and their Servants; with the Arrival, and
affable Behaviour of a beautiful young Lady; which may teach Persons of
Condition how they may acquire the Love of the whole World.}

\russiantitlei{При велит абхорреант ид, еи яуи вирис утрояуе импердиет. Ат хас утрояуе цивибус. Примис постеа вих еу, оптион еуисмод пер ин, модус фастидии ет мел. Вих дицта нецесситатибус ад, тота видиссе молестиае вис те. Иус ех нибх праесент}

\end{scriptexample}

\section{everypar}

\tex performs another action when it starts a paragraph:
it inserts whatever is currently the contents of the \emph{token
list} \cs{everypar}. Usually you will not notice this, because
the token list is empty in plain TEX (the TEX book [3]
gives only a simple example, and the exhortation  \enquote{if you
let your imagination run you will think of better applications} ).
\latex, however, makes regular use of
\cs{everypar}. Some mega-trickery with \cs{everypar}
can be found in \cite{Lamport1994}. 

When \tex enters horizontal mode, it will interrupt its normal scanning to read
tokens that were predefined by the command everypar={token list}. For
example, suppose you have said `everypar={A}'. If you type `B' in vertical mode, TEX
will shift to horizontal mode (after contributing parskip glue to the current page),
and a horizontal list will be initiated by inserting an empty box of width |parindent|.

Then \tex will read \enquote{AB}, since it reads the everypar tokens before getting back to the
`B' that triggered the new paragraph. Of course, this is not a very useful illustration of
\cs{everypar}; but if you let your imagination run you will think of better applications.

Everypar was underutilized by Knuth and understandably so, as is full of traps. In an article in TUGboat
Josepg Wright wrote about the efforts of the \latex3 Team to use it in the still under development \enquote{xgalley}
package that will be a replacement for \latex's output routine.\footcite{joseph2015}




\begin{texexample}{everypar}{ex:everypar}
\def\makefirstwordbold#1 #2 #3.{\textbf{#1 #2} #3}
\everypar{\makefirstwordbold}
This is the first paragraph.\par
This is the second paragraph.\par
\everypar{}
\end{texexample}


We can use \cs{everypar} to add bullets to all paragraphs or a symbol such as the paragraph symbol.
\medskip

\verb+\everypar={$\bullet\quad$}+

\begin{texexample}{everypar add bullets}{}
\everypar={$\bullet\quad$}

This is a test

This is a test

\everypar={}
\end{texexample}


\subsection{Everypar trickery}

Although the first encounter of tex users with everypar is seeing a fancy heart or other fancy shaped as ASCII art with tex behind the scenes it is the workhorse for many features of \latexe. Such examples include most of the list environments. What you put in an everypar must not contain any |\par| or other commands that would put tex into a vertical mode. Thsi will create an infinite loop and the program if not your computer will crash. In the following example, we will use everypar to shape up a list of paragraphs and prefix them with a counter. If you copy the example and remove one of the commented lines, teh program 
will run as an infinite loop. We catch it and exit by using a counter within the parshape. 

\begin{texexample}{Everypar, cheking for infinite loops}{ex:everypar2}
\newcounter{acounter}
\setcounter{acounter}{0}
\parindent0pt
\bgroup
\everypar {% 
  \parindent=0pt
  \stepcounter{acounter}%
  A-\theacounter\nobreakspace
  \parshape 2 -10pt \dimexpr(\hsize+10pt) 
               10pt \dimexpr(\hsize-10pt)
  \ifnum\theacounter>7 %
  Error We have a problem...\expandafter\stop
 \fi 
 % 
 % \par
 % \vskip3pt 
 % remove any % to see the problem
 \ignorespaces} 
\lorem
\lorem
\lorem
\egroup

\lorem
\end{texexample}

\section{Double spacing}

Some people---especially those of control of formatting Theses---like documents to be \textit{double spaced}, such Gestapo type imposition of one's own taste of design normally result in making these documents harder to read but perhaps that is the intention or as \cite{Abrahams2003, Wilson2009} they have `\ldots shares in papermills and lumber companies'. As an Engineer I had countless encounters with overzealous Consultants which actually specified in Construction Specifications that arial had to be used, text had to be doublespacedg in 11pt and other superfluous requirements. 

\begin{docCommand}{onehalfspacing}{}
\end{docCommand}
The package \pkg{setspace}\footcite{setspace} can be used to make life easier, just include the package and use \cs{onehalfspacing} or \cs{doublespacing}.

\begin{docCommand}{doublespacing}{}
\end{docCommand}

\section{Controlling the width of a paragraph}

Another common requirement is controlling the width of paragraphs. For example one might want quoted text to be typeset with a smaller width than that used in paragraphs. Both TeX and \latexe provide such methods.

\subsection{Minipages}
\begin{minipage}{6.7cm}
\parindent=0pt 
{\obeylines

adeste hendecasyllabi. quot estis 
omnes. undique quotquot estis omnes. 
iocum me putat esse moecha turpis. 
et negat mihi nostra reddituram 
pugillaria si pati potestis. 
persequamur eam. et reflagitemus. 
quae sit quaeritis. illa quam uidetis 
turpe incedere mimice ac moleste 
ridentem catuli ore Gallicani. 
circumsistite eam. et reflagitate. 
moecha putida. redde codicillos. 
redde putida moecha codicillos. 
non assis facis. o lutum. lupanar, 
aut si perditius potest quid esse. 
sed non est tamen hoc satis putandum 
quod si non aliud potest ruborem 
ferreo canis exprimamus ore. 
conclamate iterum altiore uoce. 
moecha putide. redde codicillos. 
redde putida moecha moecha codicillos. 
sed nil proficimus. nihil mouetur. 
mutanda est ratio modusque uobis 
siquid proficere amplius potestis. 
pudica et proba. redde codicillos.

\hfil Catullus\par}
\end{minipage}
\hspace{0.8em}
\begin{minipage}{8cm}
{\obeylines
Come here, nasty words, so many I can hardly 
tell where you all came from. 
That ugly slut thinks I'm a joke 
and refuses to give us back 
the poems, can you believe this shit? 
Lets hunt her down , and demand them back! 
Who is she, you ask? That one, who you see 
strutting around, with ugly clown lips, 
laughing like a pesky French poodle. 
Surround her, ask for them again! 
"Rotten slut, give my poems back! 
Give 'em back, rotten slut, the poems!" 
Doesn't give a shit? Oh, crap. Whorehouse. 
Or if anything's worse, you're it. 
But I've not had enough thinking about this. 
If nothing else, lets make that 
pinched bitch turn red-faced. 
All together shout, once more, louder: 
"Rotten slut, give my poems back! 
Give 'em back, rotten slut, the poems!" 
But nothing helps, nothing moves her. 
A change in your methods is cool, 
if you can get anything more done. 
"Sweet thing, give my poems back!"\par

\hfil Catullus\par}
\end{minipage}


\section{obeylines}

\begin{docCommand}{obeylines}{}
You may have several consecutive lines of input for which you want the output
to appear line-for-line in the same way. One solution is to type \cs{par} at the
end of each input line; but that's somewhat of a nuisance, so plain TEX provides the
abbreviation `obeylines', which causes each end-of-line in the input to be like \cs{par}.
After you say obeylines you will get one line of output per line of input, unless an
input line ends with `\%' or unless it is so long that it must be broken. For example, you
probably want to use obeylines if you are typesetting a poem. 
\end{docCommand}

Be sure to enclose
obeylines in a group, unless you want this \textit{poetry} mode to continue to the end of
your document.  You can also use \cs{break} to break a paragraph at a specific point.  \footnote{but why would you want to do so?}\footnote{See source2e File b: ltplain.dtx Date: 2005/09/27 Version v1.1y 17 for the definition of \cs{obeylines}}

\begin{texexample}{obeylines}{ex:obeylines}
\obeylines
Roses are red, 
\quad Violets are blue; 
Rhymes can be typeset
\quad With boxes and glue. \footnote{From page 94 of the TeXBook} 

\end{texexample}

If you are familiar with with |HTML|, you can redefine the obeylines command to \cs{pre}, I find it easier to remember. Strictly speaking it should be the verbatim enevironment.

{\small
\begin{verbatim}
\newcommand{\pre}{\obeylines}
{\pre \small \em \smallskip
Roses are red,
\quad Violets are blue;
Rhymes can be typeset
\quad With boxes and glue.
\smallskip}
\end{verbatim}
}



{\obeylines
{\Large\bf  Catullus 42 \footnote{For a translation of the poem see \url{http://www.obscure.org/obscene-latin/catullus-42.html}}}

adeste hendecasyllabi. quot estis 
omnes. undique quotquot estis omnes. 
iocum me putat esse moecha turpis. 
et negat mihi nostra reddituram 
pugillaria si pati potestis. 
persequamur eam. et reflagitemus. 
quae sit quaeritis. illa quam uidetis 
turpe incedere mimice ac moleste 
ridentem catuli ore Gallicani. 
circumsistite eam. et reflagitate. 
moecha putida. redde codicillos. 
redde putida moecha codicillos. 
non assis facis. o lutum. lupanar, 
aut si perditius potest quid esse. 
sed non est tamen hoc satis putandum 
quod si non aliud potest ruborem 
ferreo canis exprimamus ore. 
conclamate iterum altiore uoce. 
moecha putide. redde codicillos. 
redde putida moecha moecha codicillos. 
sed nil proficimus. nihil mouetur. 
mutanda est ratio modusque uobis 
siquid proficere amplius potestis. 
pudica et proba. redde codicillos.


\hfil Catullus\par}


\bigskip
Another way to use |\obeylines| is in combination with |\everypar|. In Example~\ref{ex:everypar1}
we define everypar to insert an |\hfill| at the start of every paragraph. This will cause the
poem to be typeset at the end of the lines.

\begin{texexample}{everypar and obeylines}{ex:everypar1}
{\obeylines\everypar{\hfill}\parindent=0pt
Mademoiselle from Armentires, Parlez-vous,
Mademoiselle from Armentires, Parlez-vous,
Mademoiselle from Armentires,
She hasn't been kissed for forty years.
Hinky-dinky parlez-vous.

Oh Mademoiselle from Armentires, Parlez-vous,
Mademoiselle from Armentires, Parlez-vous,
She got the palm and the croix de guerre,
For washin' soldiers' underwear,

Hinky-dinky parlez-vous.
\hfil World War I Army Song\par}
\end{texexample}

Roughly speaking, \TeX breaks paragraphs into lines in the following
way: Breakpoints are inserted between words or after hyphens so as to produce
lines whose badnesses do not exceed the current \cs{tolerance}. If there's no way
to insert such breakpoints, an overfull box is set. Otherwise the breakpoints are
chosen so that the paragraph is mathematically optimal, i.e., best possible, in
the sense that it has no more \cs{demerits} than you could obtain by any other
sequence of breakpoints. Demerits are based on the badnesses of individual lines
and on the existence of such things as consecutive lines that end with hyphens,
or tight lines that occur next to loose ones.  \footnote{Perhaps a still unsurpassed algorithm, by other software.}

In the TeXBook, Knuth gives this exercises for the reader. 

\begin{latexquotation}
Since \tex reads an entire paragraph before it makes any decisions about
line breaks, the computer's memory capacity might be exceeded if you are typesetting
the works of some philosopher or modernistic novelist who writes 200-line paragraphs.
Suggest a way to cope with such authors. \footnote{Assuming that the author is deceased and/or set in his or her ways, the remedy
is to insert {\cs{parfillskip=0pt} \cs{par} \cs{parskip=0pt} \cs{noindent}} in random places, after
each 50 lines or so of text. (Every space between words is usually a feasible breakpoint,
when you get sufficiently far from the beginning of a paragraph.)}
\end{latexquotation}

This brings almost to the end the discussion on paragraphs. A simple paragraph and so much to experiment with. If you writing for e-readers, perhaps we also need to redefine how often we use paragraphs. They should be much shorter to cater for shorter attention spans and scanning of text by users, but this is a discussion for another time


\section{Narrowing paragraphs}

\begin{docCommand}{leftskip}{\meta{dimension}}
You can say |\leftskip=10pt plus 2pt minus 3pt|. This explains to \tex that it should put |10pt| (maybe up to 2pt more, maybe up to |3pt| less) of glue on the start of each line. This is not generally recommended to be used directly in text (you should use environments like \docAuxEnvironment{quote} or \docAuxEnvironment{center} instead). 
\end{docCommand}

\begin{docCommand}{rightskip}{\meta{dimension}}
Puts glue at the end of each line. Has the opposite effect of |\leftskip|
\end{docCommand}

A plain \tex command |narrower| can be used to narrow a paragraph. Again \latex's lists are better as they can apply to more than one paragraph. They also are aware of their environment and react accordingly, as far as spacing is concerned.

\begin{docCommand}{narrower}{}
You can use the command \cs{narrower} to indent paragraphs both sides by  an amount equal to the
\cs{parindent} value.
\end{docCommand}

\startlineat{214}
\begin{teXXX}
 \def\narrower{%
   \advance\leftskip\parindent
   \advance\rightskip\parindent}
\end{teXXX}

\begin{texexample}{narrower text}{}
\bgroup
\parindent=2em
\small
\onepar


\narrower

\onepar\par
\egroup
\end{texexample}

We can even make paragraphs doubly narrow by using \cs{narrower} \cs{narrower} in example \refCom{narrow}.

\begin{texexample}{narrowing both sides}{narrow}
\begingroup
The sentence \fox. has been typeset with normal paragraph settings.

\parindent3em
\narrower \narrower\small 
The sentence \fox has been typeset with larger left skips.
\medskip
\endgroup
\end{texexample}

\section{Shaping paragraph}
\label{sec:shapingpar}

\epigraph{Soon the two pages would be filled with colors and shapes, the sheet would become a kind of
reliquary, glowing with gems studded in what would then be the devout text of the writing.}{Umberto Eco}

\index{primitives>\texttt\textbackslash parshape}
By using the TeX primitive command \docAuxCommand{parshape}, you could literally make your paragraph any shape you want.
This is applied as follows:

|\parshape|$=n i_1l_1 i_2 l_2 \ldots i_n l_n$

where $n>\geq1$ is an integer, and all $i_k$ and $l_k(1\leq k)$ are \textit{dimensions}. 

If there are more than $n$ lines then the specification
for the last line ($i_n l_n$) is used for the rest of the
lines in the paragraph.


\begin{texexample}{\textbackslash parshape}{ex:parshape}
\parindent = 0pt
\parshape = 10
   0.5cm .7\linewidth %1
   0.6cm .7\linewidth %2
   0.7cm .7\linewidth %3
   0.8cm .7\linewidth %4
   0.9cm .7\linewidth %5
   1.0cm .7\linewidth %6
   1.1cm .7\linewidth %7
   1.2cm .7\linewidth %8
   1.3cm .7\linewidth %9
   1.4cm .7\linewidth %10
\onepar   
\end{texexample}

This is very interesting but its cumbersomeness index is proportional to the cube of the number of lines one has to type! 

Let us look at something more interesting. Figure~\ref{fig:photospread2} shows a nice layout for a page opening after a fancy chapter opening that essentially takes four pages. We will try and get the ``Introduction'' to be placed in a cut-out, using |\parshape|
\begin{figure}[htbp]
\parindent=0pt
\includegraphics[width=\textwidth]{baetens-02.jpg}\par
\caption{Chapter spread and first pages after the chapter title which is on the right page of the chapter spread. From \textit{New Photography, Art and the Craft}, Pascal Baetens, DK Publications. }
\label{fig:photospread2}
\end{figure}

\begin{texexample}{\textbackslash parshape}{ex:parshape}
\bgroup
\newlength\cutout
\setlength\cutout{3.5cm}
\newlength\restofline
\setlength\restofline{\linewidth-\cutout}
\hsize13cm
\leftskip2cm
\large
\parindent = 0pt
\parshape = 11
   0cm \linewidth %1
   0cm \linewidth %2
   0cm \linewidth %3
   0cm \linewidth %4
   0cm \linewidth %5
   0cm \linewidth %6
   3.5cm \restofline %7
   3.5cm \restofline %8
   3.5cm \restofline %9
   3.5cm \restofline %10
   0cm \linewidth %11
\tikz[remember picture,overlay] \node at (0cm,-100pt) {{\Huge\bfseries\sffamily Introduction}};
\lipsum[1]
\egroup   
\end{texexample}

Our attempt works in principle but of course it would send any graphic artist into apoplexia, as it is so far badly designed. We should have measured the word \enquote{introduction} and balance the margins and the font sizing. 

So how to we insert the word \enquote{introduction}? We can use a zero sized box, insert the word using
\tikzname or even use a package. The package \pkg{cutwin}\footfullcite{cutwin} provides numerous macros for partiallly assisting in automating such layouts, as well as other type of cutouts, for example in the middle of paragraphs.\footnote{Don't use this type of layout, as is frowned upon by modern typographers.} Early TeXnicians used \latexe |picture| environment, for solving such layouts

\section{Creating a cutout in a paragraph}

A good understanding of creating macros and splitting |\vbox|es is necessary before you attempt to understand the code in this section. The problem and a solution was first described in TUGboat in 1987 by Alan Hoenig. Alan wrote:

\begin{quotation}
 I present the macros below, as well as two extensions, which
allow TEX to set rectangular cutouts which aren't horizontally centered. and which force \tex to set cutouts
of arbitrary shape. I do make several limiting assumptions: the cutout fits entirely within a single paragraph,
and the |\baselineskip| remains constant within that paragraph. I believe you can modify these macros
with little additional work, however. There is one known bug, which I was unable to fix in time to meet the
submission date. When the ratio of baselineskip to design font size reaches decreases to a certain critical
value, the cutout is not properly formed. You'll be okay if you keep the baselineskip at least 2 points greater
than the design size. 
\end{quotation}

The code was later adapted by Peter Wilson, who also developed it to the \pkg{cutwin}, which is available at the ctan repository.

Hoening named the parts of this shape as the lintel for the top part, window for the cutout and sill for the bottom part. He then used the command |\parshape| to create an odd-shaped paragraph consisting of a top portion identical to the lintel, a bottom portion identical to the sill, and a lengthy and narrow middle portion with the width of the side text. Then, take this typeset text, and slice it like a roast beef. These "slices " will contain lines of text in |\vboxes| which we rearrange to get the text we want, cutout and all. In figure 4, you see the intermediate position of some text before and after this rearrangement. 

\begin{teX}%{Cutouts}{ex:cutout}
\newcount\l 
\newcount\d 
\newdimen\lftside 
\newdimen\rtside 
\newtoks\a
\newbox\rawtext 
\newbox\holder 
\newbox\window 
\newcount\n
\newbox\finaltext 
\newbox\aslice 
\newbox\bslice
\newdimen\topheight
\newdimen\ilg % InterLine Glue
\end{teX}

\begin{teX}
\def\openwindow\down#l\in#2\for#3\lines{%
% #1 is an integer---no. of lines down from par top
% #2 is a dimension---amount from left where window begins
% #3 is an integer---no. of lines for which window opening
% persists.
\d=#l \l=#3 \leftside=#2 \righttside=\leftside \a={}
\createparshapespec
\d=#l \1=#3 % reset these
\setbox\rawtext=\vbox\bgroup
\parshape=\n \the\a }
%
\def\endwindowtext{%
\egroup \parshape=0 % reset parshape; end \box\rawtext
\computeilg % find ILG using current font.
\setbox\finaltext=\vsplit\rawtext to\d\baselineskip
\topheight=\baselineskip \multiply\topheight by\l
\multiply \topheight by 2
\setbox\holder=\vsplit\rawtext to\topheight
% \holder contains the narrowed text for window sides
\decompose\holder\to\window % slice up \holder
\setbox\finaltext=\vbox{\unvbox\finaltext\vskip\ilg\mvbox\window% 
\vskip\ilg\unvbox\rawtext}
\box\finaltext} % finito
\end{teX}
%
The next macros \docAuxCommand{decompose} and \docAuxCommand{prune} are then used
to split the horizontal lines into two.
\emphasis{lastbox,decompose,prune}
\begin{teX}
\def\decompose#l\to#2{%
  \loop\advance\l-1
    \setbox\aslice=\vsplit#l to\baselineskip
    \setbox\bslice=\vsplit#l to\baselineskip %get 2 struts
    \prune\aslice\lftside \prune\bslice\rtside
    \setbox#2=\vbox{\unvbox#2\hbox to\hsize~\box\aslice\hfil\box\bslice}~
 \if num\l>0\repeat
}
\end{teX}



\begin{teX}
\def\prune#1#2{ % take a \vbox containing a single \hbox,
% \unvbox it, and cancel the \lastskip
% put in a \hbox of width #2
\unvbox#1 \setbox#1=\lastbox %\box#1 now is an \hbox
\setbox#1=\hbox to#2{\strut\unhbox#l\unskip}
}
\end{teX}

The createshapespec creates the parshape specification, which is specified in pairs. This
is a parameterless macro as all the parameters are in registers. 
\begin{teX}
\def\createparshapespec{%
\n=\l \multiply \n by2 \advance\n by\d \advance\n by1
\loop\a=\expandafter{\the\a 0pt \hsize}\advance\d-1
\ifnum\d>0\repeat
\loop\a=\expandafter{\the\a 0pt \lefttside 0pt \rtside}\advance\l-1
\ifnum\l>0\repeat
\a=\expandafter{\the\a 0pt \hsize}
}
%
\def\computeilg{% compute the interline glue
\ilg=\baselineskip
\setbox0=\hbox{(}\advance\ilg-\ht0 \advance\ilg-\dp0
}
\egroup

\end{teX}


But if you want your paragraph to be shaped a heart, there's a package, \pkg{shapepar}\footfullcite{shapepar}, that
could ease the work. The package provides a few predefined shapes that you could call
up by using \cs{diamondpar}, \cs{squarepar}, and \cs{heartpar}

The size is adjusted automatically so that the entire shape is filled with text. There may not be displayed maths or \verb+ €˜\vadjust +  material (no \verb+\vspace+) in the argument of shapepar. The macros work for both LaTeX and plain TeX. For LaTeX, specify usepackage{shapepar}; for Plain, input shapepar.sty.
shapepar works in terms of user-defined shapes, though the package does provide some predefined shapes: so you can form any paragraph into the form of a heart by putting heartpar{sometext...} inside your document. The tedium of creating these polygon definitions may be alleviated by using the shapepatch extension to transfig which will convert xfig output to shapepar polygon form.
The author is Donald Arseneau. The package is Copyright  © 1993,2002,2006 Donald Arseneau.



\newcommand{\abc}{abcdefghijklmnopqrstuvwxyz}

\fbox{\begin{minipage}{2cm}%
 \smallskip \baselineskip=7pt\tiny
\noindent \hfuzz 0.1pt
\parshape 30 0pt 120pt 1pt 118pt 2pt 116pt 3pt 112pt 6pt
108pt 9pt 102pt 12pt 96pt 15pt 90pt 19pt 84pt 23pt 77pt
27pt 68pt 30.5pt 60pt 35pt 52pt 39pt 45pt 43pt 36pt 48pt
27pt 51.5pt 21pt 53pt 16.75pt 53pt 16.75pt 53pt 16.75pt 53pt
16.75pt 53pt 16.75pt 53pt 16.75pt 53pt 16.75pt 53pt 16.75pt
53pt 14.6pt 48pt 28pt 45pt 30.67pt 36.5pt 51pt 23pt 76.3pt
The wines of France and California may be the best
known, but they are not the only fine wines. Spanish
wines are often underestimated, and quite old ones may
be available at reasonable prices. For Spanish wines
the vintage is not so critical, but the climate of the
Bordeaux region varies greatly from year to year. Some
vintages are not as good as others,
so these years ought to be
s\kern -.1pt p\kern -.1pt e\kern -.1pt c\hfil ially
n\kern .1pt o\kern .1pt t\kern .1pt e\kern .1pt d\hfil:
1962, 1964, 1966. 1958, 1959, 1960, 1961, 1964,
1966 are also good California vintages.
Good luck finding them!
\label{fig:parshape}
\end{minipage}}

\section{Summary}
\tex's main blocks are paragraphs. It treats all words as tokens and applies an algorithm of using glue and boxes to typeset it. Commands are available  to modify the display of all elements of paragraphs. We have not discussed {\em boxes} and {\em glue} yet. This is still yet to come once we delve a bit more in the programming side of things.

   
\section{Linenumbers}

In many cases especially those that involve scholarly critical editions we may want to number paragraphs. This seemingly easy task, is extremely difficult to achieve with \tex and \latex, unless you use a pre-existing package. In this case we can use the \docpkg{lineno} package, that can produce numbered paragraphs as shown below.\TODO{clashes with fp}



\section{Dangerous bends}
\gdef\tstory{There are cries, sobs, confusion among the people, and
     at that moment the cardinal himself, the Grand Inquisitor, passes by the
     cathedral. He is an old man, almost ninety, tall and erect, with a
     withered face and sunken eyes, in which there is still a gleam of light.
     He is not dressed in his brilliant cardinal's robes, as he was the day
     before, when he was burning the enemies of the Roman Church~\char144
     \kern2em\hfill---Fyodor Dostoyevsky}
     % The example for several primitives uses \tstory.

\begingroup
     \hsize=2.5in
     \setbox0=\vbox{\adjdemerits=0
     \doublehyphendemerits=100000
     \finalhyphendemerits=900000
     \tstory\par}
     \setbox1=\vbox{\adjdemerits=1000000
     \doublehyphendemerits=100000
     \finalhyphendemerits=900000
     \tstory\par}
     \hbox{\box0\kern0.25in\box1}

\endgroup

The command \cs{doublehyphendemerits} is used by \tex when is breaking a paragraph into lines, this value is added to the demerits calculated for a line if the line and the previous one end with discretionary breaks [98].

\bgroup
\hsize=2.5in%                
   
  \setbox0=\vbox{\tstory\par}%  holds the definition of \tstory

     \setbox1=\vbox{\adjdemerits=0
        \doublehyphendemerits=200000
     \tstory\par}

     \hbox{\box0\kern0.25in\box1}






\egroup
\bigskip
\begingroup
\hsize=2.5in
     \setbox0=\vbox{\adjdemerits=0
     \doublehyphendemerits=100000
     \tstory\par}% 
     \setbox1=\vbox{\adjdemerits=0
     \doublehyphendemerits=100000
     \finalhyphendemerits=900000
     \tstory\par}
     \hbox{\box0\kern0.25in\box1}
\endgroup
\bigskip

You can adjust the \cs{looseness} of a paragraph by adjusting the looseness value. The command |\looseness=l| tells \tex to try and make the current paragraph l lines longer (if loosenessl is $> 0$) or l lines shorter (if $ l < 0$) while maintaining the general tolerances used to typeset the ms. IF $ l is > 0$, a tie `\~' is often placed between the last two words in the paragraph to prevent a short last line [103-104]. The parameter is reset to zero at the end of every paragraph or when internal vertical mode is started [349].

{
\hsize=4.5in
     \tstory\par
     \vskip6pt
     {\looseness=-1
     \tstory\par}
}




\section{The \textbackslash parfillskip}


\begin{texexample}{parfillskip}{ex:parfillskip}



\hfill\hbox to 5.5cm {\hsize 5.5cm\vbox{%
\leftskip=0pt plus 1fill
\rightskip=0pt plus -1fill
\parfillskip=0pt plus 2fill
A well-designed book means one
that is (a) appropriate to its content
and use, (b) economical, and
(c) satisfying to the senses. It is
not a "pretty" book in the superficial
sense and it is not necessarily
more elaborate than usual.\par
}}\hskip1.5cm
\hbox to 5.5cm {\hsize 5.5cm\vbox{%
\leftskip=0pt plus 1fill
\rightskip=0pt plus -1fill
\parfillskip=0pt plus 1fill
A well-designed book means one
that is (a) appropriate to its content
and use, (b) economical, and
(c) satisfying to the senses. It is
not a ``pretty'' book in the superficial
sense and it is not necessarily
more elaborate than usual.\par
}}\hfill\hfill

\end{texexample}

When \tex typesets a paragraph it will add a |\leftskip| and |\rightskip| to every line. If they are both set to zero, effectively the algorithm will then typeset the paragraph fully justified wwith hyphenation as needed.

\begin{texexample}{Flush glue}{ex:parfillskip2}

% we are in vertical mode
\vbox{\hsize 4.8cm\vbox{%
\bgroup
\catcode`\@=11 

\leftskip=\z@skip 
\rightskip=\z@skip
\parfillskip=\z@skip
\fussy


\frogking

\lorem
%A well-designed book means one
%that is (a) appropriate to its content
%and use, (b) economical, and
%(c) satisfying to the senses. It is
%not a ``pretty'' book in the superficial
%sense and it is not necessarily
%more elaborate than usual.\par

\scriptsize
\the\parfillskip\\
\the\leftskip\\
\the\tolerance\\

\the\z@skip\relax

\catcode`\@=12 
\egroup
}}
%\fbox{\hsize 5.5cm\vtop{%
%
%\par\leavevmode
%|\leftskip| =  |0pt plus 0fill|\\
%|\rightskip|= |0pt plus 0fill|\\
%|\parfillskip| = |0pt plus 1fill|\\
%|\tolerance|= \the\tolerance\relax\\
%}}
%

\end{texexample}
%\@flushglue

The command \docAuxCommand{z@skip} is from the LaTeX kernel. It is defined as: 
\begin{quote}
|\z@skip=0pt plus0pt minus0pt|
\end{quote}
As we go along,since this is a book about programming \tex and  \latex I will be introducing both \tex as well as \latex commands.





\section{French spacing}

Before we conclude this chapter it remains to discuss, spacing after punctuation. The \frenchspacing declaration tells \latex not to insert extra space at the end of sentences. This style is common in non-English languages, such as French and hence its name.

\index{frenchspacing}\index{junkspacing}\index{nonfrenchspacing}
Each character in a font has a space factor \index{ space factor} code that is an integer between 0 and 32767. The code is used to adjust the space factor in a horizontal list. The uppercase letters A-Z have space factor code 999. Most other characters have code 1000 [76]. However, Plain TeX makes `)', `'', and `]' have space factor code 0. Also, the \cs{frenchspacing} and \cs{nonfrenchspacing} modes in Plain \tex work by changing the \cs{sfcode} for: `.', `?', `!', `:', `;', and `,' [351].

\begingroup
\def\junkspacing{\sfcode`\.32767 \sfcode`\?6000 \sfcode`\!3000
    \sfcode`\:2500 \sfcode`\;2000 \sfcode`\,1500}

\def\nonfrenchspacing{\sfcode`\.3000 \sfcode`\?3000 \sfcode`\!3000
   \sfcode`\:2000 \sfcode`\;1500 \sfcode`\,1250}

\def\frenchspacing{\sfcode`\.1000 \sfcode`\?1000 \sfcode`\!1000
   \sfcode`\:1000 \sfcode`\;1000 \sfcode`\,1000}

 % Quotes are intentionally omitted in the following story:

 \let\tstory\frogking

\bigskip

\noindent\rule{\linewidth}{0.4pt}

\medskip
\narrower
{\hfill\hfill\small \texttt{\textbackslash junkspacing}}
\medskip


 \junkspacing Once upon a time, there was a naughty squirrel. Where shall I eat
     today? it asked. There were three options: a distant oak tree; a nearby 
    walnut tree; and a freshly-stocked bird feeder. I think\par

\bigskip

\smallskip
{\hfill\hfill\small \texttt{\textbackslash nonfrenchspacing}}

\medskip
\raggedright
     \nonfrenchspacing Once upon a time, there was a naughty squirrel. Where shall I eat
     today? it asked. There were three options: a distant oak tree; a nearby 
    walnut tree; and a freshly-stocked bird feeder. I think\par \par

\bigskip

\smallskip
{\hfill\hfill\noindent\small \texttt{\textbackslash frenchspacing}}

\medskip
     \frenchspacing Once upon a time, there was a naughty squirrel. Where shall I eat
     today? it asked. There were three options: a distant oak tree; a nearby 
    walnut tree; and a freshly-stocked bird feeder. I think\par \par

\medskip
\noindent\rule{\linewidth}{0.4pt}
\endgroup


One other item we need to examine is what happens at the end of an abbreviation or if you end a sentence with a capital letter? Probably not much, especially if you are using the microtype package. Example~\ref{bs} demonstrates its use. The last example is from Barbara Beeton's example at |TX.SX|.\footnote{\url{http://tex.stackexchange.com/questions/22561/what-is-the-proper-use-of-i-e-backslash-at}.}

\begin{texexample}{spacing after abbreviations}{bsat}
\makeatother
The name of the corporation is A.B.C.What happens to spacing after the last stop? 

The name of the corporation is A.B.C. What happens to spacing after the last stop?

The name of the corporation is A.B.C\@. What happens to spacing after the last stop?

Languages like JS, HTML, etc.\ were not used by King Henry III\@. This is Barbara Beeton's example.
\end{texexample}

\section{Code tables}

Table~\ref{tab:coded} 
shows the `numeric' coded commands and the corresponding
glyphs. 

Table~\ref{tab:alpha} 
shows the alphabetic coding (in both single
character and command form) and the corresponding glyphs together with their
transliterations. Note that not every glyph has a transliteration.

\begin{comment}

\begin{center}
  \Large\cartouche{\pmglyph{K:l-i-o-p-a-d:r-a}}
\end{center}
\end{comment}

\chapter{Hyphenation}
\label{ch:hyphenation}

\epigraph{Although it does not find all possible division points in a word, it very rarely makes an error. Tests on a pocket dictionary word list indicate that about 40\% of the allowable hyphen points are found, with 1\% error (relative to the total number of hyphen points). The algorithm requires 4K 36-bit words of code, including the exception dictionary.}{--Franklin Mark Liang \footcite{liang83}}

\label{ch:hyphenation} \index{hyphenation}
It is said that George Bernard Shaw would examine galley proofs of his work and recast sentences, or even whole pages, in order to avoid unsightly word breaks, excessive white space caused by justification, and other typographical difficulties. Of course, he was published at a time when typesetters were sent to the block for committing the abominations above.\cite{Major1991} 

\section{A war over hyphens}

\index{Hyphen War}
In 1989-1990 there was a conflict called The Hyphen War (in Czech, Pomlčkov\'a v\' alka; in Slovak, Pomlkov¡ vojna €”literally `'Dash War'') was the tongue-in-cheek name given to the conflict over what to call Czechoslovakia after the fall of the Communist government. The Communist system in Czechoslovakia fell in November 1989. But in 1990, the official name of the country was still the "Czechoslovak Socialist Republic" (in Czech and in Slovak Československá socialistick¡ republika, or ČSSR). President clav Havel proposed merely dropping the word "Socialist" from the name, but Slovak politicians wanted a second change. They demanded that the country's name be spelled with a hyphen (e.g. "Republic of Czecho-Slovakia" or "Federation of Czecho-Slovakia"), as it was spelled from Czechoslovak independence in 1918 until 1920, and again in 1938 and 1939. President Havel then changed his proposal to "Republic of Czecho-Slovakia" \footnote{See discussion at \url{http://en.wikipedia.org/wiki/Hyphen_War}}. 

The use of the hyphen in English compound nouns and verbs has, in general, been steadily declining. Compounds that might once have been hyphenated are increasingly left with spaces or are combined into one word. In 2007, the sixth edition of the \textit{Shorter Oxford English Dictionary} removed the hyphens from 16,000 entries, such as fig-leaf (now fig leaf), pot-belly (now pot belly) and pigeon-hole (now pigeonhole). The advent of the Internet and the increasing prevalence of computer technology have given rise to a subset of common nouns that may in the past have been hyphenated (e.g. \textit{toolbar}, \textit{hyperlink}, \textit{pastebin}).

Despite decreased usage, hyphenation remains the norm in certain compound modifier constructions and, amongst some authors, with certain prefixes. Hyphenation is also routinely used to avoid unsightly spacing in justified texts (for example, in newspaper columns). 

\section{Hyphenation of common words}
With the advent of computers hyphenating justified text automatically became a challenge.


In \TeX78 a rule-driven algorithm for English   \index{TEX78}
was built-in by Liang and Knuth. Their algorithm
found 40\% of the allowable hyphens, with
about 1\% error. Although authors
claimed that these results are quite good, Liang
continued working on the generalization of the idea
of rules expressed by hyphenating and inhibiting
patterns. In his dissertation \footcite{liang83} he describes
a method, which is used in TEX82, based
on the generalization of the prefix, suffix and the
vowel-consonant-consonant-vowel rules. He wrote
(in \texttt{WEB}) the program \texttt{PATGEN} (Liang and Breitenlohner,
1991) to automate the process of pattern \index{hyphenation!patterns}
generation from a set of already hyphenated words.

He started with the 1966 edition of Webster's Pocket\cite{websters1961}
Dictionary that included hyphenated words and inflections
 (about 50 000 entries in total). In the early
stages, testing the algorithm on a 115 000 word dictionary
from the publisher, 10 000 errors in words
not occurring in the pocket dictionary were found.

Most of these were specialized technical terms that
we decided not to worry about, but a few hundred
were embarrasing enough that we decided to add
them to the word list. (Liang, 1983, p. 30). He
reports the following figures: 89,3\% permissible hyphens
found in the input word-list with 4447 patterns
with 14 exceptions.

Liang's method is described by Knuth (1986b,
Appendix H) and was later adopted in many programs
such as troff (Emerson and Paulsell, 1987)
and Lout, and in localizations of today's WYSIWYG
DTP systems such as QuarkXPress, Ventura,
etc. Although specialized dictionaries such
as Allen's (1990) by Oxford University Press separate
possible word-division points into at least two
categories (preferred and less recommended), we
have not seen any program that incorporates the
possibility of taking into account these classes of
hyphenation points so far.



\section{Liang's hyphenation algorithm}

Franklin M. Liang's hyphenation algorithm is based
on what is termed \emph{competing hyphenation patterns}.\index{hyphenation>competing hyphenation patterns} 

Liang experimented with hyphenation and came with the idea of \textit{hyphenation patterns}.
These are simply strings of letters that, when they match a word, tell us how to hyphenate at some point in the pattern.
For example, the pattern |tion| tell us that we can hyphenate between the |t|. Or when the pattern 'cc' appears in a word, we can ususally hyphenate between the c's. Liang gives some good hyphenating patterns\cite{Liang1981}.

\begin{teX}
.in-d  .in-t  .un-d  b-s -cia
\end{teX}

\noindent (The character '.' matches the beginning or end of a word).


The patterns can
give excellent compression for a hyphenation dictionary,
and using these patterns the fast hyphenator algorithm
can also correctly hyphenate unknown (non-dictionary)
words most of the time. Liang's work
covers also the machine learning of the hyphenation
patterns and exceptions by |PatGen|  pattern generator.
The hyphenation patterns can allow and prohibit
hyphenation breaks on multiple levels. Figure \ref{fig:patterns} shows
the pattern matching on the word |algorithm|. 

\begin{figure}

\mbox{\fbox{\strut.}\fbox{\strut a}\fbox{\strut  l}\fbox{\strut g}\fbox{\strut o}\fbox{\strut  r}\fbox{ \strut i}\fbox{\strut t}\fbox{\strut h}\fbox{\strut m}\fbox{\strut .}}
   4l1g4
     l g o3
    1g o
            2i t h
               4h1m

\mbox{\fbox{\strut 4}\fbox{\strut 1}\fbox{\strut 4}\fbox{\strut 3}\fbox{\strut 2}\fbox{\strut 0}\fbox{\strut 4}\fbox{\strut 1}}

\mbox{\strut\fbox{a}\fbox{l}\fbox{-}\fbox{g}\fbox{o}\fbox{-}\fbox{r}\fbox{i}\fbox{t}\fbox{h}\fbox{-}\fbox{m}}

\caption{\tex hyphenation of the word algorithm.}
\label{fig:patterns}
\end{figure}


The patterns consist of strings of letters and digits. Digits
indicates a `hyphenation value'\index{hyphenation!hyphenation value} for some intercharacter position.  For
example, the pattern \texttt{\.{3t2ion}} specifies that if the string \texttt{\.{tion}}
occurs in a word, we should assign a hyphenation value of 3 to the
position immediately before the \.{t}, and a value of 2 to the position
between the \.{t} and the \.{i}.

To hyphenate a word, we find all patterns that match within the word and
determine the hyphenation values for each intercharacter position.  If
more than one pattern applies to a given position, we take the maximum of
the values specified (i.e., the higher value takes priority).  If the
resulting hyphenation value is odd, this position is a feasible
breakpoint; if the value is even or if no value has been specified, we are
not allowed to break at this position.

In order to find quickly the patterns that match in a given word and to
compute the associated hyphenation values, the patterns generated by this
program are compiled by \.{INITEX} into a compact version of a finite
state machine.  For further details, see the \TeX 82 source.


The
\tex English hyphenation patterns 4l1g4, lgo3, 1go,
2ith and 4h1m match this word and determine its
hyphenation. Only odd numbers mean hyphenation
breaks. If two (or more) patterns have numbers in
the same place, the highest number wins. The \texttt{algo-
rith-m} hyphenation is bad, but the last one-letter
hyphenation is suppressed by \tex, so we end up with
the correct \texttt{al-go-rithm}.(See also the Section on |\hyphenminleft| and |hyphenminright| for more details
how this parameters are adjusted in \tex.

One of the most notable features of this pattern based
hyphenation is the human-readable format of
the knowledge database, in contrast to an equivalent
finite state machine or a similarly good artificial neural
network. This format is good for manual checking and
corrections.



\section{How to tinker hyphenation}

TeX will not insert a hyphen before the number of letters specified by \docAuxCommand{lefthyphenmin},
nor after the number of letters specified by \docAuxCommand{righthyphenmin}. For U.S. English,
|\lefthyphenmin=2| and |\righthyphenmin=3|. 

\index{hyphenation>penalty>\textbackslash hyphenpenalty}
\begin{docCommand}{hyphenpenalty}{}
The best way to examine the effects of the various hyphenation parameters
is to put the words in a narrow minipage (much quicker and visual rather than examining TeX's output. The first parameter setting command is we will examine is \cs{hyphenpenalty}.
\end{docCommand}

\begin{texexample}{}{}
\fbox{
\begin{minipage}{1.3cm}
\hyphenpenalty=-2000
photographer and hyphenation. \par
potographer and hyphenation. \par
unhelpful\par
\end{minipage}}
\end{texexample}



The \cs{hyphenpenalty} can be used to adjust the hyphenation of paragraphs. 
This example typesets a paragraph with three different values of \cs{hyphenpenalty}. 

There is no difference in using the Plain TeX value of 50 and using 0. 
Increasing |\hyphenpenalty| to 200 eliminates all hyphenated words in the paragraph. 
Decreasing |\hyphenpenalty|  to -2000 results in two addition hyphenated words.

\begin{texexample}{}{}
\hsize2.5in
\long\def\testhyphenpenalty#1%
    {\par\leavevmode
       \hyphenpenalty=#1 %
        \tstory% 
        \par
        \vskip2\baselineskip
     }

\testhyphenpenalty{50} 
\testhyphenpenalty{200}
\testhyphenpenalty{-2000}
\end{texexample}





\section{\textbackslash lefthyphenmin}

\newthought{This parameter holds} the minimum number of characters that must appear at the beginning of a hyphenated word (i.e., before the `-'). In particular, \tex will not hyphenate words with fewer than the sum of \cs{lefthyphenmin} and \cs{righthyphenmin} characters [454]. The |whatsit\mkindex{whatsit`5`language}|  made by a change to \cs{language} includes the current value of |\lefthyphenmin|.

Changes made to |\lefthyphenmin| are \textit{local} to the group containing the change.

\begin{teX}
\def\tstoryA{There are cries, sobs, confusion among the people, and at
   that moment the cardinal himself, the Grand Inquisitor, passes by the
   cathedral. He is an old man \ldots\par}

   {\language255\hyphenation{m-oment}}
   \count0=\lefthyphenmin
   \setbox0=\vbox{\hsize=4.3in\language255 \tstoryA}
   \setbox1=\vbox{\hsize=4.3in\language255\lefthyphenmin=1\righthyphenmin=2 \tstoryA}
   \medskip\par
   \hbox to \hsize{\box0}
   \medskip
   \hbox to \hsize{\box1}
\end{teX}

This will produce:

\noindent{\color{orange}\rule{5cm}{1pt}\hfill\hfill\par}

\begingroup
\overfullrule=0.5pt
\def\tstoryA{There are cries, sobs, confusion among the people, and at
   that moment the cardinal himself, the Grand Inquisitor, passes by the
   cathedral.\par}


   {\language255\hyphenation{m-oment}}
   \count0=\lefthyphenmin
   \setbox0=\vbox{\hsize=4.3in\language255 \tstoryA}
   \setbox1=\vbox{\hsize=4.3in\language255\lefthyphenmin=1\righthyphenmin=2 \tstoryA}
   \medskip\par
   \hbox to \hsize{\box0}
   \medskip
   \hbox to \hsize{\box1}
\medskip
\endgroup


{\hfill\hfill\color{orange}\rule{5cm}{1pt}\par}
\hfill\hfill{\raise6pt\hbox{\small}


\section{Hyphenation exceptions}

The command \cs{-}  inserts a discretionary hyphen into a word. This also becomes the only point where hyphenation is allowed in this word. This command is especially useful for words containing special characters (e.g., accented characters), because LaTeX does not automatically hyphenate words containing special characters. A list of hyphenation exceptions has been kept and updated
by Barbara Beeton for many years.\footcite{beeton2015} 

\begin{teX}
\begin{minipage}{2in}
I think this is: su\-per\-cal\-%
i\-frag\-i\-lis\-tic\-ex\-pi\-%
al\-i\-do\-cious
\end{minipage}
\end{teX}
\bigskip



\noindent{\color{orange}\rule{5cm}{1pt}\hfill\hfill\par}
\begin{center}
\par
\begin{minipage}{2in}
I think this is: su\-per\-cal\-%
i\-frag\-i\-lis\-tic\-ex\-pi\-%
al\-i\-do\-cious
\par
\end{minipage}
\end{center}
{\hfill\hfill\color{orange}\rule{5cm}{1pt}\par}
\hfill\hfill{\raise6pt\hbox{\small}
\bigskip

This can be quite cumbersome if one has many words that contain a dash like electromagnetic-endioscopy. One alternative to this is using the \cs{hyp} command of the \docpkg{hyphenat} package. This command typesets a hyphen and allows full automatic hyphenation of the other words forming the compound word. One would thus write

\begin{teX}
electromagnetic\hyp{}endioscopy
\end{teX}


Several words can be kept together on one line with the command

\begin{teX}
\mbox{text}
\end{teX}

It causes its argument to be kept together under all circumstances. 
For example when we are typesetting phone numbers,

\begin{teX}
My phone number will change soon to be |\mbox{0116 291 2319}|.
\end{teX}

\noindent |\fbox| is similar to |\mbox|, but in addition there will be a visible box drawn around the content.

To avoid hyphenation altogether, the penalty for hyphenation can be set to an extreme value:

\begin{teX}
\hyphenpenalty=100000
\end{teX}



The following sample texts from \textit{The frog king}\cite{frogking} have been traditionally used for testing hyphenation algorithms as they
include both short as well as long words. We have varied the |hyphenpenalty| as shown. It is a tribute to \tex that even with no hyphenation present (the last column) the text still looks very presentable with virtually no visible large spaces.

\long\def\sampletext{%
\hskip1em In olden times when wishing
still helped one, there lived a
king whose daughters were all
beautiful, but the youngest was so
beautiful that the sun\hl{ itself},
which has seen so much, was
astonished whenever it shone in
her face. Close by the king's
castle lay a great dark forest,
and under an old lime-tree in the
forest was a well, and when
the day was very warm, the
king's child went out into the 
forest and sat down by the side
of the cool fountain, and when she was bored she
took a golden ball, and threw it up on a high and caught it, and this
ball was her favorite plaything. \par}

\overfullrule=0.1pt

\begin{minipage}{1.9in}
 \hyphenpenalty=0\sampletext
\end{minipage}\hspace{.8cm}
\begin{minipage}{1.9in}
 \hyphenpenalty=100\sampletext
\end{minipage}\hspace{.8cm}
\begin{minipage}{1.9in}
 \hyphenpenalty=100000 \sampletext
\end{minipage}


\def\samplerivers{%
\hskip1em Repeated repeated repeated repeated
repeated repeated repeated repeated
repeated repeated repeated repeated
repeated repeated repeated repeated
repeated repeated repeated repeated
repeated repeated repeated repeated
repeated repeated repeated repeated
repeated repeated repeated repeated
repeated repeated repeated repeated
repeated repeated repeated repeated
repeated repeated repeated repeated
repeated repeated repeated repeated
repeated.}

\overfullrule=0.1pt

\begin{minipage}{1.9in}
 \looseness=-1 \hyphenpenalty=0\samplerivers
\end{minipage}\hspace{.8cm}
\begin{minipage}{1.9in}
  \hyphenpenalty=100\samplerivers
\end{minipage}\hspace{.8cm}
\begin{minipage}{1.9in}
 \hyphenpenalty=100000 \samplerivers
\end{minipage}




%%% Code from GIT posted by Wilson

\frenchspacing
\fussy

\makeatletter

\newbox\trialbox
\newbox\linebox
\newcount\maxbad
\newcount\linebad
\newcount\bestbad
\newcount\worstbad
\newcount\overfulls
\newcount\currenthbadness


\def\trypar#1\par{%
  \showtrybox{\linewidth}{#1\par}%
}

\newcommand\showtrybox[2]{%
  \currenthbadness=\hbadness
  \maxbad=0\relax
  \setbox\trialbox=\vbox{%
    \hsize#1\relax#2%
    \hbadness=10000000\relax
    \eatlines
  }%
  \hbadness=10000000\relax
  \setbox\trialbox=\vbox{%
    \hsize#1\relax#2%
    \printlines
  }%
  \noindent\usebox\trialbox\par
  \hbadness=\currenthbadness
}

\newcommand\trybox[2]{%
  \currenthbadness=\hbadness
  \maxbad=0\relax
  \setbox\trialbox=\vbox{%
    \hsize#1\relax#2\par
    \hbadness=10000000\relax
    \eatlines
  }%
  \hbadness=\currenthbadness
}

\def\eatlines{%
  \begingroup
  \setbox\linebox=\lastbox
  \setbox0=\hbox to \hsize{\unhcopy\linebox\hss}%
  \linebad=\the\badness\relax
  \ifnum\linebad>\maxbad\relax \global\maxbad=\linebad\relax \fi
  \ifvoid\linebox\else
    \unskip\unpenalty\eatlines
  \fi
  \endgroup
}

\def\printlines{%
  \begingroup
  \setbox\linebox=\lastbox
  \setbox0=\hbox to \hsize{\unhcopy\linebox}%
  \linebad=\the\badness\relax
  \ifvoid\linebox\else
    \unskip\unpenalty\printlines
    \ifhmode\newline\fi\noindent\box\linebox\showbadness
  \fi
  \endgroup
}

\def\showbadness{%
  \makebox[0pt][l]{%
    \ifnum\currenthbadness<\linebad\relax
      \ifnum\linebad=1000000\relax\expandafter\@gobble\fi
      {\quad\color{red}\rule{\overfullrule}{\overfullrule}~{\footnotesize\sffamily(\the\linebad)}}%
    \fi
  }%
}

\makeatother



\begin{minipage}{5cm}

\trypar
There is no just ground, therefore, for the charge brought against me by~
certain ignoramuses---that I have never written a moral tale, or, in more
precise words, a tale with a moral. They are not the critics predestined
to bring me out, and \emph{develop} my morals:---that is the secret. By and by
the ``North American Quarterly Humdrum'' will make them ashamed of their
stupidity. In the meantime, by way of staying execution---by way of
mitigating the accusations against me---I offer the sad history appended,---
a history about whose obvious moral there can be no question whatever,
since he who runs may read it in the large capitals which form the title
of the tale. I should have credit for this arrangement---a far wiser one
than that of La Fontaine and others, who reserve the impression to be
conveyed until the last moment, and thus sneak it in at the fag end of
their fables.\par
\end{minipage}

\the\hbadness


\hbadness=2000 
\begin{minipage}{5cm}
\trypar \hyphenpenalty=-50
There is no just ground, therefore, for the charge brought against me by~
certain ignoramuses---that I have never written a moral tale, or, in more
precise words, a tale with a moral. They are not the critics predestined
to bring me out, and \emph{develop} my morals:---that is the secret. By and by
the ``North American Quarterly Humdrum'' will make them ashamed of their
stupidity. In the meantime, by way of staying execution---by way of
mitigating the accusations against me---I offer the sad history appended,---
a history about whose obvious moral there can be no question whatever,
since he who runs may read it in the large capitals which form the title
of the tale. I should have credit for this arrangement---a far wiser one
than that of La Fontaine and others, who reserve the impression to be
conveyed until the last moment, and thus sneak it in at the fag end of
their fables.\par
\end{minipage}
\bigskip
\clearpage

\section{Testing badness}
The following text displays the badness as calculated by the linebreaking algorithm.
\begin{figure*}[htb]
\fussy
\hbadness=-1 
\begin{minipage}[t]{4.5cm}
\mbox{}
\trypar\hyphenpenalty=-500\looseness=1
In olden times when wishing
still helped one, there lived a
king whose daughters were all
beautiful, but the youngest was so
beautiful that the sun itself,
which has seen so much, was
astonished whenever it shone in
her face. Close by the king's
castle lay a great dark forest,
and under an old lime-tree in the
forest was a well, and when
the day was very warm, the
king's child went out into the 
forest and sat down by the side
of the cool fountain, and when she was bored she
took a golden ball, and threw it up on a high and caught it, and this
ball was her favorite plaything. \par
\end{minipage}
\hspace{2cm}
\begin{minipage}[t]{4.5cm}
\mbox{}
\trypar\hyphenpenalty=10000
In olden times when wishing
still helped one, there lived a
king whose daughters were all
beautiful, but the youngest was so
beautiful that the sun itself,
which has seen so much, was
astonished whenever it shone in
her face. Close by the king's
castle lay a great dark forest,
and under an old lime-tree in the
forest was a well, and when
the day was very warm, the
king's child went out into the 
forest and sat down by the side
of the cool fountain, and when she was bored she
took a golden ball, and threw it up on a high and caught it, and this
ball was her favorite plaything. \par
\end{minipage}
\caption{Comparison of two sample texts. The left has a hyphenpenalty=-500 and the right has a hyphenpenenalty=10000. Both look acceptable. The text is set at 4.5cm textwidth}
\end{figure*}

\begin{figure*}[htb]
\fussy
\hbadness=-1 
\begin{minipage}[t]{4.5cm}
\mbox{}
\trypar\hyphenpenalty=500\looseness=1
In olden times when wishing
still helped one, there lived a
king whose daughters were all
beautiful, but the youngest was so
beautiful that the sun itself,
which has seen so much, was
astonished whenever it shone in
her face. Close by the king's
castle lay a great dark forest,
and under an old lime-tree in the
forest was a well, and when
the day was very warm, the
king's child went out into the 
forest and sat down by the side
of the cool fountain, and when she was bored she
took a golden ball, and threw it up on a high and caught it, and this
ball was her favorite plaything. \par
\end{minipage}
\hspace{2cm}
\begin{minipage}[t]{4.5cm}
\mbox{}
\trypar\hyphenpenalty=530
In olden times when wishing
still helped one, there lived a
king whose daughters were all
beautiful, but the youngest was so
beautiful that the sun itself,
which has seen so much, was
astonished whenever it shone in
her face. Close by the king's
castle lay a great dark forest,
and under an old lime-tree in the
forest was a well, and when
the day was very warm, the
king's child went out into the 
forest and sat down by the side
of the cool fountain, and when she was bored she
took a golden ball, and threw it up on a high and caught it, and this
ball was her favorite plaything. \par
\end{minipage}
\caption{Comparison of two sample texts. The left has a hyphenpenalty=-500 and the right has a hyphenpenenalty=10000. Both look acceptable. The text is set at 4.5cm textwidth}
\end{figure*}


\cxset{chapter opening=any}


\chapter{The line breaking problem}

\epigraph{Psychologically bad breaks are not easy to define; we just know they are bad. When
the eye journeys from the end of one line to the beginning of another, in the presence
of a bad break, the second word often seems like an anticlimax, or isolated from
its context. Imagine turning the page between the words ‘Chapter’ and ‘8’ in some
sentence; you might well think that the compositor of the book you are reading should
not have broken the text at such an illogical place}{Donald Knuth}

In the days of typewriters once the end of line was reached a bell rung to tell the author that the end of the line was reached. The author then had the choice to press the carriage return to start a new line or to extend the line by a couple of characters.

Consider the following short text, 

\begin{scriptexample}{}{}
In olden times when wishing
\end{scriptexample}

 \newlength\myl
 \settowidth\myl{In olden times when wishing}
The natural width of the above string of text is \the\myl. Consider again the same string but this time all white space removed.

\begin{scriptexample}{}{}
Inoldentimeswhenwishing
\end{scriptexample}
\settowidth\myl{Inoldentimeswhenwishing}
With all interword spacing removed the line is now only\the\myl's wide.  

If the paragraph was to be constrained at a width $<limit$ one could distribute less white space between the words of the line. If the width of the paragraph was less than limit there would be no choice but to move a word to the line below it. TeX optimizes the justification of a full paragraph rather than optimize the looks of a single line in order to produce high quality typesetting.


Breaking a paragraph into lines consists of selecting the break points in a paragraph. To determine how much space a line takes, each character (or rather glyph) in the paragraph is modelled as a box with a specific width, height and elevation from the baseline. These properties are determined by the font being used and the font size.


Sometimes a broken line may not completely fit into the available space between the left and the right margin. To make it fit, space may be distributed among the spaces in the line, or some whitespace may be taken out, if the text exceeds the 
line width. We assume that there is a single target width for the complete paragraph and the paragraph is adjusted. 
An indent that applies to the line of the paragraph can be modeled using a fixed width unbreakable space (possibly with a negative value).

\section{Formalizing the problem}

Now that we have a good idea of the problem we can formalize it. A paragraph $p$ is a sequence of $n>0$ characters $c_i$,  $i\leq i_i \leq n$.
A \textit{breakpoint candidate}\index{paragraph!breakpoint candidate} in $p$ is an index of a character in $p$ for which it is allowed to break a line. Typical break point candidates are space and hyphen characters and hyphenation points in words.

The \textit{line breaking problem} for a paragraph $p$ for a desired \textit{text width} and \textit{indent} is finding a set of break points of $p$ that look `nice'. We will consider a fixed \textit{text width}, $t_w$, with the exception of the first line which can possibly have a negative indent $in$. What looks nice is obviously a subjective term. Another typography factor is the grayness of the paragraph. 

\section{Greedy algorithm}

An easy way to break a paragraph into lines is to use the greedy algorithm. This algorithm basically puts as many 
words on the line as it can contain, repeating the process for each line until there are no more words in the paragraph. The greedy algorithm is a line-by-line process. During the execution of the algorithm each line is handled independently. \footcite{elyaakoubi}



\section{Knuth Pratt algorithm}
\tex's line breaking algorithm optimizes line breaks on the level of a paragraph. \tex determines character widths taking kerning and ligatures in account and uses a three phased process. 

\begin{description}
\item[Phase 1] In this phase no hyphenation takes place. Only white spaces are considered  for line breaking. For a paragraph broken in $k$ lines, for each line $j=k\ldots k$  a \textit{badness} $b_i$ is calculated using:

$$b_j=100\left(\frac{nlw_{sj}-t_w}{nsp_{sj}\dot f}\right)^{3}$$

where 

$nlw{sj} = \text{the natural width}$

Why this penaly function is a power of three was never explained properly, obviously is to use a penalty function which is non-linear. 

Depending on the value of $b_j$, the line is classified as \textit{tight}, \textit{loose} or  \textit{very loose}. If none of the line's badness exceed a \textit{pretolerance}  limit the paragraph is accepted and no further processing takes place.

\item[Phase 2] In phase 2 the hyphenation points are calculated for the paragraph and a new set of breakpoints  determined. If all of the line's badness is below a a tolerance level  (which differs from the pretolerance level) a \textit{demerit} for the paragraph is calculated as a combination of line badness, roughly as

\begin{equation}
\sum_{j=1}^{n}  \left(c+b_j \right)^{2} + p \cdot \vert p\vert 
\end{equation}

where $c$ is a constant line penalty and $p$ a penalty term. a tight or loose line increases the penalty $p$. If two consecutive lines are hyphenated or if the last line is hyphenated, the penalty is increased. A paragraph with a demerit below a threshold is accepted.

\item[Phase 3] In phase 3 the steps in phase 2 are repeated but now with an \textit{emergency stretch} that allows lines to shrink or expand more. If this last phase does not succeed, \tex outputs overly long lines, due to the thresholds in the algorithm.
\end{description}

The description above is very broad, as Knuth introduced more variables that control for example, what makes a good hyphenation point and what it doesn’t.

There are also rules as to spacing after punctuation, how kerning and similar aspects are considered.  For mathematical typesetting a different algorithm is used. 


\section{Patents and other research}

A very similar algorithm to that provided by the Knuth-Plaas algorithm was filed as a \href{http://www.freepatentsonline.com/6510441.pdf}{patent}
by Adobe \cite{adobepatent}. Another more recent attempt is by Holkner\cite{Holkner2006}. Holkner attempts to optimize over multiple objectives but as he writes:

\begin{quote}
Also surprising is the change in performance as more objective functions are added. When
$\sigma$Looseness is added to $\mu$Looseness and $\Sigma\text{hyphen}$ performance improves --- by more than 10 times
for the wider two columns. On the other hand, when Looseness is added to Looseness and $\Sigma\text{River}$
performance degrades so badly that some of the tests had to be stopped when they took more than six
hours to complete.

\end{quote}

What is interesting though is the author's conclusions towards the end of his thesis. Having defined some new metrics, which we will discuss soon, he writes:

\begin{quote}
It has become apparent through our research that while \tex returns the optimal paragraph according
to its weighted sum measurement, this measurement is not optimal with respect to our metrics.
Having shown that our metrics represent real-world typographic qualities, we can say that \tex can be
improved upon, and that our method does this.

\end{quote}

You can get more info at \url{http://yallara.cs.rmit.edu.au/~aholkner/presentation2.pdf}


\section{Typesetting with varying letter widths}
A totally different approach to hyphenation and line breaking was the work of early typographers including
Gutenburg, where varying width of glyphs were used to fit lines into exact widths. Such an approach was
also used by the \so{hz} software\cite{hz}\index{line breaking!hz-program}\index{hz-program}

An attempt to apply the techniques used by Gutenberg and the |hz| program to 
was done by Miroslava Mis\'akov\'a \cite{Miroslava1998}. The author demonstrated the potential of a simple
method that allows typesetting with varying letter widths implemented by font expansion
to attain better interword spacing of composition. The results were very interesting,
even though the method was not flexible enough for practical use. (this is different than \so{letterspacing}, which
is just a typographic way favoured by some for emphasizing text without affecting the grayness of the page.

These and other methods are discussed under the Chapter for microtypography. 
\section{The strangeness of TeX}

{
 \everypar{\def\indent{1}}
\indent 3 i s a prime number.
and

 \everypar{\def\vrule{1}}
\vrule 3 is  a prime number.
}




\begin{figure*}
^^A\includegraphics[width=\linewidth]{../images/bible}
\caption{Leaf from the G\"ottingen Gutenburg Bible}
\url{http://www.gutenbergdigital.de/gudi/eframes/bibelsei/frmlms/frms.htm}
\label{fig:bible}
\end{figure*}

\section{The Gutenberg Bible}

The Gutenberg Bible (also known as the 42-line Bible, the Mazarin Bible or the B42) was the first major book printed with a movable type printing press, marking the start of the "Gutenberg Revolution" and the age of the printed book. Widely hailed for its high aesthetic and artistic qualities,\cite{Martin1996} the book has iconic status in the West. It is an edition of the Vulgate, printed by Johannes Gutenberg, in Mainz, Germany in the 1450s. Only twenty-one complete copies survive, and they are considered by many sources to be the most valuable books in the world, even though a completed copy has not been sold since 1978.

The 36-line Bible is also sometimes referred to as a Gutenberg Bible, but is possibly the work of another printer.

In appearance the Gutenberg Bible closely resembles the large manuscript Bibles that were being produced at the time. The Giant Bible of Mainz, probably produced in Mainz in 1452-3, has been suggested as the particular model Gutenberg used.[4] Around this time large Bibles, designed to be read from a lectern, were returning to popularity for the first time since the twelfth century. In the intervening period, small hand-held Bibles had been usual.[5] The text of the Gutenberg Bible is traditional, falling within the Paris Vulgate group of texts.[6] Manuscript Bibles all had texts that differed slightly, and the copy used by Gutenberg as the exemplar for his Bible has not been discovered.[7]

The Bible was not Gutenberg's first work.\cite{Man2002} Preparation of it probably began soon after 1450, and the first finished copies were available in 1454 or 1455.[10] However, it is not known exactly how long the Bible took to print.

Gutenberg made three significant changes during the printing process.[11] The first sheets were rubricated by being passed twice through the printing press, using black and then red ink. This was soon abandoned, with spaces being left for rubrication to be added by hand.

Some time later, after more sheets had been printed, the number of lines per page was increased from 40 to 42, presumably to save paper. Therefore, pages 1 to 9 and pages 256 to 265, presumably the first ones printed, have 40 lines each. Page 10 has 41, and from there on the 42 lines appear. The increase in line number was achieved by decreasing the interline spacing, rather than increasing the printed area of the page.
Finally, the print run was increased, probably to 180 copies, necessitating resetting those pages which had already been printed. The new sheets were all reset to 42 lines per page. Consequently, there are two distinct settings in folios 1-32 and 129-158 of volume I and folios 1-16 and 162 of volume II.[12][13]

The most reliable information about the Bible's date comes from a letter. In March 1455, future Pope Pius II wrote that he had seen pages from the Gutenberg Bible, being displayed to promote the edition, in Frankfurt.[14].
It is believed that in total 180 copies of the Bible were produced, 135 on paper and 45 on vellum.[15]

The production process: 'Das Werk der B\"acher'

In a legal paper, written after completion of the Bible, Gutenberg refers to the process as 'Das Werk der Bücher': The work of the books. He had invented the printing press and was the first European to print with movable type[16]. But his greatest achievement was arguably demonstrating that the whole process of printing actually produced books.

Many book-lovers have commented on the high standards achieved in the production of the Gutenberg Bible, some describing it as one of the most beautiful volumes ever printed. The quality of both the ink and other materials and the printing itself have been noted. [1]

Paper and vellum

A single complete copy of the Gutenberg Bible has 1,272 pages; with 4 pages per folio-sheet, 318 sheets of paper are required per copy. The 45 copies printed on vellum required 11,130 sheets. The 135 copies on paper required 49,290 sheets of paper. The handmade paper used by Gutenberg was of fine quality and was imported from Italy. Each sheet contains a watermark, which may be seen when the paper is held up to the light, left by the papermold.

The paper size is 'double folio', with two pages printed on each side (making a total of four pages per sheet). After printing the paper is folded once to the size of a single page. Typically, five of these folded sheets (carrying 10 leaves, or 20 printed pages) were combined to a single physical section, called a quinternion, that could then be bound into a book. Some sections, however, carried as few as 4 leaves or as many as 12 leaves.[17] It is possible that some sections were printed in a larger number, especially those printed later in the publishing process, and sold unbound. The pages were not numbered. This whole technique of course was not new, since it was used already to make white-paper books to be written afterwards. New was the necessity to determine beforehand the right place and orientation of each page on the five sheets, so as to end up in the right reading sequence. Also new was the technique of getting the printed area correctly located on each page.

The folio size, 307 x 445 mm, has the ratio of 1.45. The printed area had the same ratio, and was shifted out of the middle to leave a 2:1 white margin, both horizontally and vertically. Historian John Man writes that the ratio was chosen because of being close to the golden ratio of 1.61.[9] To reach this ratio more closely the vertical size should be 338 mm, but there is no reason why Gutenberg would leave this non-trivial difference of 8 mm go by in such a detailed work in other aspects.

Ink

Gutenberg had to develop a new kind of ink, an oil-based one (as compared with the traditional water-based ink used in manuscripts), so that it would stick better to the metal types. His ink was based on carbon, with high metallic content, including copper, lead, and titanium.

Type style

The Gutenberg Bible is printed in the blackletter type styles that would become known as Textualis (Textura) and Schwabacher. The name texture refers to the texture of the printed page: straight vertical strokes combined with horizontal lines, giving the impression of a woven structure. Gutenberg already used the technique of justification, that is, creating a vertical, not indented, alignment at the left and right-hand sides of the column. To do this, he used various methods, including using characters of narrower widths, adding extra spaces around punctuation, and varying the widths of spaces around words.[19][20] On top of this, he subsequently let punctuation marks go beyond that vertical line, thereby using the massive black characters to make this justification stronger to the eye.


Rubrication, illumination and binding

Copies left the Gutenberg workshop unbound, without decoration, and for the most part without rubrication.
Initially the rubrics -- the headings before each book of the Bible -- were printed, but this experiment was quickly abandoned, and gaps were left for rubrication to be added by hand. A guide of the text to be added to each page, printed for use by rubricators, survives.\cite{Kapr1996}

The spacious margin allowed for illuminated decoration to be added by hand. The amount of decoration presumably depended on how much each buyer could or would pay for. Some copies were never decorated.[22] The place of decoration can be known or inferred for about 30 of the surviving copies. Perhaps 13 of these received their decoration in Mainz, but others were worked on as far away as London.[4] The vellum Bibles were more expensive and perhaps for this reason tend to be more highly decorated, although the vellum copy in the British Library is completely undecorated.[23] There has been speculation that the Master of the Playing Cards was partly responsible for the illumination of the Princeton copy, though all that can be said for certain is that the same model book was used for some of the illustrations in this copy and for some of the Master's playing cards.\cite{Buren1974}

Although many Gutenberg Bibles have been rebound over the years, 9 copies retain fifteenth-century bindings. Most of these copies were bound in either Mainz or Erfurt.[4] Most copies were divided into two volumes, the first volume ending with The Book of Psalms. Copies on vellum were heavier and for this reason were sometimes bound in three or four volumes.\cite{Martin1996}

\section{Life is not always simple}
 This document provides some samples of archaic fonts. They are
available from CTAN in the \texttt{fonts/archaic} directory. The fonts
form a set that display how the Latin alphabet and script evolved from the
initial Proto-Semitic script until Roman times.

    The fonts tend to consist of letters only --- punctuation had not 
been invented during this period except for a word-divider in some cases.
Some of the scripts had signs for numbers but in others either letters
doubled as numbers or the numbers were spelt out. The fonts are all
single-cased. Upper- and lower-case letters were again only invented after
the end of this period.

    Other fonts are also available for some scripts that were not on the
main alphabetic tree. The period covered by the scripts is from about 
3000~BC to the Middle Ages.

    For some of the scripts transliterations into the Latin alphabet can
be automatically generated by the accompanying LaTeX packages.

  The vowels (a, e, i, o, u) are: \textcypr{\Ca{} \Ce{} \Ci{} \Co{} \Cu}.


\section{Remaining Limitations}

Frank Mittelbach\footcite{mittelbach2013} outlined a number of difficulties, where \tex is limited. One of them is that \tex's hyphenation algorithm knows only two states: a place in a word can or cannot act as a 
hyphenation point. He gives an example from German, where ``Non-nenkloster'' (abbey of nuns)
should preferably not be hyphenated as ``Nonnenklo-ster'' (as that leaves the word ``nun's toilet'' on the first line).

Liang’s pattern-based approach works very well for
languages for which the hyphenation rules can be
expressed as patterns of adjacent characters next to
hyphenation points. Such patterns may not be easy
to detect but once determined they will hyphenate
reasonably well. For the approach to be usable, the
necessary set of patterns should be be reasonably
small, as each discrepancy needs one or more exception
patterns with the result that the pattern set
would either become very large or the hyphenation
results would have many errors.

To improve the situation for the latter type of
languages one would need to implement and potentially
first develop other types of approaches. For
now Liang’s algorithm is hardwired in all engines,
though in theory LuaTEX offers possibilities of dropping
in some replacement.


























  \chapter{Characters}


\normalsize

\tex\ works internally by translating characters into character codes. The way characters are encoded in a computer
may differ from system to system.\index{characters>encoding}


There are 256 characters that \tex\  might encounter at
each step, in a file or in a line of text typed directly on your terminal. These
256 characters are classified into 16 categories numbered 0 to 15:\index{characters>catcodes}\index{catcodes}


\arial
\begin{table}[htbp]
\centering
\begin{tabular}{rll}
\toprule
Code & Description & Representation\\
\midrule
0 & Escape character & (\textbackslash in this book)\\
1 & Beginning of group & (|{| in this book)\\
2 & End of group & (|\}| in this book )\\
3 & Math shift & (|\$| in this book)\\
4 & Alignment tab & (|\&| in this book)\\
5 & End of line &(return in this book)\\
6 & Parameter &(|\#| in this book\\
7 & Supescript &(|\^| in this book)\\
8 & Subscript &(|\_| in this bookl)\\
9 & Ignored character &(null in this manual)\\
10 & Space &(\textvisiblespace in this book)\\
11 &Letter &(A,\ldots,Z and a,\ldots z)\\
12 &Other character &(none of the above or below)\\
13 &Active character &(|\~| in this manual)\\
14 &Comment character &(|\%| in this book)\\
15 &Invalid character &(delete in this book)\\
\bottomrule
\end{tabular}
^^A\captionof{table}{Character Codes}
\end{table}
\medskip

When \tex reads a line of text from a file, or a line of text that you entered
directly on your keyboard, it converts that text into a list of \cmd{\tokens}. A
token is either (a) a single character with an attached category code, or (b) a control
sequence. For example, if the normal conventions of plain \tex  are in force, the text
\verb*+ `{\hskip 36 pt}'+  is converted into a list of \textit{eight} tokens:
\medskip

$ \{_{1}$ hskip $3_{12}~~6_{12}~~\_{10}~~p_{11}~~t_{11}~~\}_2 $

\medskip
The subscripts here are the category codes, as listed earlier:
\begin{itemize}
\item[1] for beginning of group,
\item[12] for |other| character," and so on. The |hskip| doesn't get a subscript, because it
represents a control sequence token instead of a character token. Notice that the space
after \cmd{hskip} does not get into the token list, because it follows a control word.
\end{itemize}

Knuth in the \texbook notes that:

\begin{quotation}

It is important to understand the idea of token lists, if you want to gain a
thorough understanding of \tex, and it is convenient to learn the concept by
thinking of \tex as if it were a living organism. The individual lines of input in your
files are seen only by \tex's \textit{eyes} and \textit{mouth}; but after that text has been gobbled
up, it is sent to \tex's \textit{stomach} in the form of a token list, and the digestive processes
that do the actual typesetting are based entirely on tokens. As far as the stomach is
concerned, the input 
flows in as a stream of tokens, somewhat as if your \tex manuscript
had been typed all on one extremely long line.
\end{quotation}

\section{Control sequences for characters}

\DescribeMacro{\char}
There are a number of ways in which a control sequence can denote a character. The \cmd{\char} command
specifies a character to be typeset; the \cmd{\let} command introduces a synonym for a character
token, that is, the combination of character code and category code.

\section{Denoting characters to be typeset: \texttt{char}}

\index{\string\char}
Characters can be denoted numerically by, for example, \verb+ \char98 +. This command tells \tex to add
character number 98 of the current font to the horizontal list currently under construction.

Instead of decimal notation, it is often more convenient to use octal or hexadecimal notation. For
octal the single quote is used: \verb+ \char’142+; hexadecimal uses the double quote: \verb+ \char"62+. Note that

\begin{texexample}{Characters}{ex:chars}
\bgroup
\ttfamily

\char65

\char`b

\char`\b

\char"70

\egroup
\end{texexample}

\verb+ \char`'62+  is incorrect; the process that replaces two quotes by a double quote works at a later
stage of processing (the visual processor) than number scanning (the execution processor).

Because of the explicit conversion to character codes by the back quote character it is also possible
to get a ‘b’ – provided that you are using a font organized a bit like the ASCII table – with \verb+ \char‘b+
or \verb+ \char‘\b+.

The \cmd{\char} command looks superficially a bit like the \verb+  ^^+ substitution mechanism.

Both mechanisms access characters without directly denoting them. However, the \verb+ ^^+ mechanism
operates in a very early stage of processing (in the input processor of \tex, but before category
code assignment); the \cmd{\char} command, on the other hand, comes in the final stages of processing.
In effect it says ‘typeset character number so-and-so’.

\CMDI{\Uchar} The LuaTeX expandable command \cmd{\Uchar} reads a number between 0 and 1,114,111 and expands to the
associated Unicode character. 

\DescribeMacro{\chardef}
There is a construction to let a control sequence stand for some character code: the \cmd{\chardef}
command. The syntax of this is\\
\cs{chardef}\meta{control sequence}=\meta{number},\\
where the number can be an explicit representation or a counter value, but it can also be a character
code obtained using the left quote command (see above; the full definition of hnumberi is
given in Chapter 7). In the plain format the latter possibility is used in definitions such as

\verb+ \chardef\%=‘\%+

or 

\verb+ \chardef\%=37   +

command to typeset character 37 (usually the per cent character).\index{characters!percent character}

A control sequence that has been defined with a \cmd{\chardef} command can also be used as a hnumberi.
This fact is used in allocation commands such as \verb+ \char{newbox}+ (see Chapters 7 and 31). Tokens defined
with \verb+ \char{mathchardef}+ can also be used this way.


But \tex\ actually provides another kind of number that makes it unnecessary
for you to know texttt{ASCII} at all! The token `12 (left quote), when followed by
any character token or by any control sequence token whose name is a single character,
stands for \tex's internal code for the character in question. For example, \verb+\char`b+ and
\verb+ \char`\b+ are also equivalent to \verb+ \char98+. If you look in Appendix B to see how \verb+ \%+ is
defined, you'll notice that the definition is

\verb+\def\%{\char`\%}+

instead of \verb+ \char37+  as claimed above.

\section{Special notation for invisible characters}

\tex has a standard way to refer to the invisible characters of |ASCII|: 

Code 0 can be typed as the sequence of three characters \verb+ ^^@+, code 1 can be typed
\verb+ ^^A+, and so on up to code 31, which is \verb+ ^^_  +(see Appendix C). If the character following
\verb+ ^^+ has an internal code between 64 and 127, TEX subtracts 64 from the code; if the
code is between 0 and 63, \tex adds 64. 

Hence code 127 can be typed \verb+^^?+, and
the dangerous bend sign can be obtained by saying \verb+{\manual^^?}+. However, you must
change the category code of character 127 before using it, since this character ordinarily
has category 15 (invalid); say, e.g., 

\verb+ \catcode`\^^?=12 +

The \verb+ ^^+ notation is different from
\cmd{\char}, because \verb+ ^^+ combinations are like single characters; for example, it would not
be permissible to say \verb+ \catcode`\char127+, but \verb+^^+ symbols can even be used as letters within control words.

\begin{texexample}{Special notation}{ex:texbook1}
\def\^^zz{test}
\^^zz
\end{texexample}


Most of the \verb+ ^^+ codes are unimportant except in unusual applications. But
\verb+ ^^M+ is particularly noteworthy because it is code 13, the |ASCII| return that
\tex normally places at the right end of every line of your input file. By changing the
category of \verb+ ^^M+  you can obtain useful special effects, as we shall see later.

\section{Upper and Lowercase characters}

\verb*+\lccode +the character code for the lowercase form of a letter (p. 103)

\DescribeMacro{\lowercase}
\DescribeMacro{\uppercase}
The twin operations \cmd{\uppercase}\marg{token list} and \cmd{\lowercase}\marg{token listi}
go through a given token list and convert all of the character tokens to their
\cmd{\uppercase}  or \cmd{lowercase} equivalents.

\begin{texexample}{Uppercase and Lowercase}{ex:lowercase} 
\uppercase{abcdefgh} 
\lowercase{ABCDEFGH}
\end{texexample}

Here's how: Each of the 256 possible characters
has two associated values called the \cmd{\uccode} and the \cmd{lccode}; these values are
changeable just as a \cmd{\catcode} is. Conversion to uppercase means that a character
is replaced by its \cmd{\uccode} value, unless the \cmd{\uccode} value is zero (when no change
is made). Conversion to lowercase is similar, using the
\verb+  \lccode+. The category codes
aren't changed. 

When INITEX begins, all \verb+ \uccode+ and \verb+ \lccode+ values are zero except
that the letters a to z and A to Z have \verb+\uccode+ values A to Z and \verb+\lccode+ values a to z.

These tow control sequences are used to build a hash table, mapping all the capital and lowercase letters to their respective character codes.
(see pg 41 TeXbook)

\section{Some Practical Examples}

If you are typesetting anything that has to do with \tex\ or \latex\ you are bound to have to typeset a lot of commands. This short code below will change the category code of the \texttt{"} (double quote) to be the active command. This way anything between double quotes will be  typed out verbatim and in a Maroon color. By mainipulating the \cmd{catcode} of characters we can achieve this.

\begin{teX}
%% Code to catch commands
\def\Meaningless#1>{}
\catcode`\"=\active
\def\startV{\leavevmode\begingroup
  \ifdim 0pt=\lastskip\penalty200 \fi
  \catcode`\{11 \catcode`\}11 \catcode`\%11
  \moreV}
\long\def\moreV#1"{%
  \def\LtxCode{#1}%
  \ignorespaces
      \expandafter\Meaningless\meaning\LtxCode
      \unskip%
  \endgroup}
\let"\startV

\bgroup
\catcode`\<=\active
\def<#1>{\ensuremath{\langle\mbox{\textsl{#1}}\rangle}}
\end{teX}

\begin{comment}
\bgroup
\def\Meaningless#1>{}
\catcode`\"=\active
\def\startV{\leavevmode\begingroup
  \ifdim 0pt=\lastskip\penalty200 \fi
  \catcode`\{11 \catcode`\}11 \catcode`\%11
  \moreV}
\long\def\moreV#1"{%
  \def\LtxCode{#1}%
  \ignorespaces
      \expandafter\Meaningless\meaning\LtxCode
      \unskip%
  \endgroup}
\let"\startV

\catcode`\<=\active
\def<#1>{\ensuremath{\langle\mbox{\textsl{#1}}\rangle}}

\noindent Testing it out with a few commands we get 
"\catcode", "\char" ,"\def" etc. We will revert back to this short example later on in our book, when you have learned a bit more about macros and programming \tex\. Note that this also affects "quotes".

\egroup
\end{comment}

A more complex example is the \pkg{shortvrb} package code.

\begin{teX}
%% Copyright (C) 1989-1999 Frank Mittelbach, all rights reserved.
\def\MakeShortVerb{%
  \@ifstar
    {\def\@shortvrbdef{\verb*}\@MakeShortVerb}%
    {\def\@shortvrbdef{\verb}\@MakeShortVerb}}

\def\@MakeShortVerb#1{%
  \expandafter\ifx\csname cc\string#1\endcsname\relax
    \@shortvrbinfo{Made }{#1}\@shortvrbdef
    \add@special{#1}%
    \expandafter
    \xdef\csname cc\string#1\endcsname{\the\catcode`#1}%
    \begingroup
      \catcode`\~\active  \lccode`\~`#1%
      \lowercase{%
      \global\expandafter\let
         \csname ac\string#1\endcsname~%
      \expandafter\gdef\expandafter~\expandafter{\@shortvrbdef~}}%
    \endgroup
    \global\catcode`#1\active
  \else
    \@shortvrbinfo\@empty{#1 already}{\@empty\verb(*)}%
  \fi}
\def\DeleteShortVerb#1{%
  \expandafter\ifx\csname cc\string#1\endcsname\relax
    \@shortvrbinfo\@empty{#1 not}{\@empty\verb(*)}%
  \else
    \@shortvrbinfo{Deleted }{#1 as}{\@empty\verb(*)}%
    \rem@special{#1}%
    \global\catcode`#1\csname cc\string#1\endcsname
    \global \expandafter\let \csname cc\string#1\endcsname \relax
    \ifnum\catcode`#1=\active
      \begingroup
        \catcode`\~\active   \lccode`\~`#1%
        \lowercase{%
          \global\expandafter\let\expandafter~%
          \csname ac\string#1\endcsname}%
      \endgroup \fi \fi}
\def\@shortvrbinfo#1#2#3{%
  \PackageInfo{shortvrb}{%
     #1\expandafter\@gobble\string#2 a short reference
                                          for \expandafter\string#3}}
\def\add@special#1{%
  \rem@special{#1}%
  \expandafter\gdef\expandafter\dospecials\expandafter
    {\dospecials \do #1}%
  \expandafter\gdef\expandafter\@sanitize\expandafter
    {\@sanitize \@makeother #1}}
\def\rem@special#1{%
  \def\do##1{%
    \ifnum`#1=`##1 \else \noexpand\do\noexpand##1\fi}%
  \xdef\dospecials{\dospecials}%
  \begingroup
    \def\@makeother##1{%
      \ifnum`#1=`##1 \else \noexpand\@makeother\noexpand##1\fi}%
    \xdef\@sanitize{\@sanitize}%
  \endgroup}
\endinput
%%
%% End of file `shortvrb.sty'.
\end{teX}

We will spent the rest of the book in trying to understand and write code like this. My ultimate aim is  to be able to produce \tex\ code like any other program. 

\section{Example}

In this example we wish to redefine some of the active codes to act as text only:

\begin{teX}
\newenvironment{plaintext}{%
        \catcode`\$12
        \def\&{&}%
        \catcode`\&12
        \def\_{_}%
        \catcode`\_12
        \def\^{^}%
        \catcode`\^12
        \catcode`\#12
        \catcode`\%12
        \let\~~%
        \catcode`\~12
}{}
\end{teX}

\newenvironment{plaintext}{%
        \catcode`\$12
        \def\&{&}%
        \catcode`\&12
        \def\_{_}%
        \catcode`\_12
        \def\^{^}%
        \catcode`\^12
        \catcode`\#12
        \catcode`\%12
        \let\~~%
        \catcode`\~12
}{}
Use it like

\begin{plaintext}
Here is some test text % ^ & _ $ # &.

How about some math \(x\_y\^z\). You're still out of luck with braces
though.
\end{plaintext}

\begingroup
\catcode`\{=11 
\catcode`\}=11
\catcode`\[=1
\catcode`\]=2

{This is a test}

\endgroup


\section{Checking to see the meaning of a control sequence
}
Finding out just what a control sequence has been defined to be with |\let| can be done using

%\meaning: the sequence

\begin{teXXX}
\let\x=3 \meaning\x
\end{teXXX}
\graybox{
gives 'the character 3'.}




  \parindent=1.5em


\newacro{SPQR}{Senatus Populusque Romanus}
\newacro{URL}{uniform resource locator}
\newacro{OUP}{Oxford University Press}

\chapter{Acronyms and Abbreviations}


In this section we will discuss the use and typesetting of symbols, abbreviations and acronyms. The |phd| package loads a number of packages and also offers a number of commands in managing symbols, abbreviations and acronyms. The main package we use to manage acronyms is \pkgname{acronym} \cite{acronym}. We also use some build-in commands for abbreviations and to assist in enforcing in-house style guides.  

\section{General Principles}

Abbreviations and symbols represent, through a variety of means, a
shortened form of a word or words. Abbreviations fall into three categories:
only the first of these is technically an abbreviation, though the
term loosely covers them all, and guidelines for their use overlap.

\begin{itemize}
\item \textit{Abbreviations} are formed by omitting the end of a word or words (VCR, lbw, Lieut.).
\item \textit{Contractions} are formed by omitting the middle of a word or words (I've,
mustn't, ne'er-do-well
\item \textit{Acronyms} are formed from the initial letters of words (SALT, Nazi, radar), the results being pronounced as words themselves.
\end{itemize}


\section{Acronyms}

An acronym is distinguished from other abbreviated forms by being a series of letters or 
syllables pronounced as a complete word: \textsc{NATO}
and UEFA are acronyms, but MI6 and BBC are not. Acronyms take no
points, whether all in caps (NAAFI, SALT, WASP), in initial capitals with
upper and lower case (Aga, Fiat, Sogat), or entirely in lower case (derv,
laser, scuba). Since they perform as words they can begin sentences, with
lower-case forms being capitalized normally, such as \textit{Laser treatment}. 

Any all-capital proper-name acronym is, in some house styles, fashioned
with a single initial capital if it exceeds four letters (Basic, Unesco, Unicef).

The \textit{Oxford Guide} suggests that editors should avoid this rule, useful though it is, where the result runs
against the common practice of a discipline (CARPE, SSHRCC, WYSIWYG), or where similar terms would be treated dissimilarly based on length alone.


Acronyms are not new language inventions, they were used well back in antiquity.  For example, the official name for the Roman Empire, and the Republic before it, was abbreviated as \ac{SPQR}. Inscriptions dating from antiquity, both on stone and on coins, use a lot of abbreviations and acronyms to save room and work. For example, Roman first names, of which there was only a small set, were almost always abbreviated. Common terms were abbreviated too, such as writing just "F" for "filius", meaning "son of", a very common part of memorial inscriptions mentioning people. Grammatical markers were abbreviated or left out entirely if they could be inferred from the rest of the text.\ac{SPQR}

So called \textit{Nomina Sacra} were used in many Greek biblical manuscripts. The common words "God" (Θεός), "Jesus" (Ιησούς), "Christ" (Χριστός), and some others, would be abbreviated by their first and last letters, marked with an overline. This was just one of many kinds of conventional scribal abbreviation, used to reduce the time-consuming workload of the scribe and save on valuable writing materials. The same convention is still commonly used in the inscriptions on religious icons and the stamps used to mark the eucharistic bread in eastern churches.

The early Christians in Rome, most of whom were Greek rather than Latin speakers, used the image of a fish as a symbol for Jesus in part because of an acronym—fish in Greek is ΙΧΘΥΣ (ichthys), which was said to stand for Ἰησοῦς Χριστός Θεοῦ Υἱός Σωτήρ (Iesous CHristos THeou hUios Soter: Jesus Christ, God's Son, Savior). Evidence of this interpretation dates from the 2nd and 3rd centuries and is preserved in the catacombs of Rome. And for centuries, the Church has used the inscription INRI over the crucifix, which stands for the Latin \textit{Iesus Nazarenus Rex Iudaeorum} (``Jesus the Nazarene, King of the Jews'').

The Hebrew language has a long history of formation of acronyms pronounced as words, stretching back many centuries. The Hebrew Bible ("Old Testament") is known as "Tanakh", an acronym composed from the Hebrew initial letters of its three major sections: Torah (five books of Moses), Nevi'im (prophets), and K'tuvim (writings). Many rabbinical figures from the Middle Ages onward are referred to in rabbinical literature by their pronounced acronyms, such as Rambam (aka Maimonides, from the initial letters of his full Hebrew name (Rabbi Moshe ben Maimon) and Rashi (Rabbi Shlomo Yitzkhaki).


The main package we load to assist with acronyms and abbreviations is |acronym|, developed by Tobias Oetiker \citeyearpar{acronym}. The package offers a number of useful commands to help with managing acronyms and to produce lists of acronyms and abbreviations. The package works by offering commands that you use to define an acronym as well as an environment serving the same purpose.

\begin{docCommand}{ac}{\meta{short version of the acronym}}
    To enter an acronym inside the text, use the |\ac{NATO}|
\end{docCommand}
    
    \begin{quote}
     |\ac{|\meta{acronym}|}|
    \end{quote}
    command. The first time you use an acronym, the full name of the
    acronym along with the acronym in brackets will be printed. If you
    specify the |footnote| option while loading the package, the full
    name of the acronym is printed as a footnote.
    The next time you access the acronym only the acronym will
    be printed.

\section{Symbols}

Symbols or signs, are a shorthand notation signifying a word or concept, and are frequent features of scientific and technical writing. The distinction between abbreviation and symbol may be blurred when, lie an abbreviation, a symbol is derived directly from a word or words (\textit{Ag} from \textit{argentum}, \textit{Pa} from \textit{pascal}, \textit{U} from \textit{uranium}), and in setting they are often treated similarly. Unlike abbreviations, however, symbols never take points, even if a single letter, or used alone or in conjuction with figures or words: \textit{F}  for \textit{false}, \textit{fluorine}, \textit{phenylalanine}.

Abstract, purely typographical symbols follow similar rules, being either close up (\ding{38}\ding{33}\ding{43})or spaced (\ding{38} \ding{33} \ding{43}). In \latex you can insert a non-breaking space if you want or a |hairsp|.

|\ding{38}~\ding{33}~\ding{43}|

Personally for the example I would prefer not to split them and the non-breaking space is a better option in this instance.

Symbols' uses can differ between disciplines. For example, in philological
works an asterisk (\textasteriskcentered) prefixed to a word signifies a reconstructed
form; in grammatical works it signifies an incorrect or nonstandard
form. A dagger (\textdagger) may signify an obsolete word, or 'deceased' when
placed before a person's name (this convention should be used only in
relation to Christians). In German a double dagger (\textdaggerdbl) follows the name
and signifies `killed in battle', \emph{gefallen} or  \gtrsymKilled.

A full set of these genealogical symbols can be found in the \pkgname{genealogytree} package  developed by  \person{Thomas F. Sturm} \citeyearpar{genealogytree} and are shown below,

\begin{scriptexample}[]{}{}
\textsl{\gtrSymbolsFullLegend[english]}
\end{scriptexample}

The package is loaded automatically by the |phd| package. Besides these symbols numerous other symbols
are loaded and described in the Chapter for Symbols.
\section{Abbreviations}

\subsection{Time Designations}

Most style guides recommend that you spell out the names of the months in the text but abbreviate them in the list  of works cited, except for May, June and July \cite{MLA}. The same manual suggests that words denoting units of time are also spelled out in the text (\textit{second}, \textit{minute}, \textit{week}, \textit{month}, \textit{year}, \textit{century}, some time designations are used only in the abbreviated form (\textit{a.m., p.m., AD, BC, BC, BCE, CE}). The |phd| package provides some assistance by loading the \pkg{datetime} package; more information on using it and of date and time formatting as well as calculations in Handling Dates and Time can be found in Pages~\pageref{ch:dates}--\pageref{datesend}.
\medskip

\begin{longtable}{lp{8cm}}
AD & after the birth of Christ (from the Latin \textit{anno Domini} `in the year of the Lord'; used before numerals ["\AD 14"] and after references for centuries ["twelfth century \AD"]\\
a.m. & before noon (from the Latin \textit{ante meridiem})\\
Apr. &April\\
Aug. &August\\
BC   &before Christ (used after numerals [``18 BC''] and referenced to centuries [``sixth century BC'']\\
BCE &before the common era (used after numerals and references to centuries)\\
CE  &common era (used after numerals and references to centuries)\\
cent. &century\\
Dec. &December\\
Feb  &February\\
Fri. &Friday\\
hr. &hour\\
Jan. &January\\
Mar. &March\\
min. &minute\\
mo. &month\\
Mon. &Monday\\
Nov. &November\\
Oct. &October\\
p.m. &after noon (from the Latin \textit{post meridiem})\\
Sat. &Saturday\\
sec. &second\\
Sept.&September\\
Sun. &Sunday\\
Thurs. &Thursday\\
Tues. &Tuesday\\
Wed. &Wednesday\\
wk. &week\\
yr. &year\\
\end{longtable}

Tables such as the one above, if not provided by the Publisher, can be very helpful, if you develop them on your own and refer back to them for consistency.

\subsection{Geographic Names}

\subsection{Common Scholarly Abbreviations and Reference Words}

\begin{figure}[tp]
\centering
\fbox{\includegraphics[width=1.0\textwidth]{./images/abbreviations.pdf}}
\caption{A typical Abbreviations page. This has been extracted from \protect\cite{bacchae}.}
\end{figure}


\section{The indefinite article with abbreviations}

The choice between \emph{a} and \emph{an} before an abbreviation depends on pronunciation,
not spelling. Use a before abbreviations beginning with a
consonant sound, including an aspirated h and a vowel pronounced with the sound of w or y.

\begin{scriptexample}
a BA degree a KLM flight a BBC announcer
a Herts, address a hilac demonstration a YMCA bed
a SEATO delegate a U-boat captain a UNICEF card
\end{scriptexample}

Use \emph{an} before abbreviations beginning with a vowel sound, including
unaspirated \emph{h}:

an AB degree an MCC ruling an FA cup match
an H-bomb an IOU an MP
an MA an RAC badge an SOS signal

This distinction assumes the reader will pronounce the sounds of the
letters, rather than the words they stand for (a Football Association cup
match, a hydrogen bomb). MS for manuscript is normally pronounced as
the full word, manuscript, and so takes a; MS for multiple sclerosis is
often pronounced em-ess, and so takes an. Likewise 'R.' for rabbi is
pronounced as rabbi ('a R. Shimon wrote'), but 'R' for a restricted classification
is normally pronounced as arr ('an R film').

The difference between sounding and spelling letters is equally important
when choosing the article for abbreviations that are acronyms and
for those that are not: a NASA launch but an NAMB award. The same holds
for names of symbols, which can vary: in America a hash symbol (\#) is a
'number sign' or, more formally, an \textit{octothorp}; in linguistic use an
asterisk may be called a 'star' and in mathematics an exclamation
mark called a 'factorial', 'shriek', or 'bang', so the correct forms are a *
and a ! rather than an * and an !. 

As abbreviated terms enter the
language there can be a period of confusion as to how they are pronounced:
in computing, for example, \ac{URL} is
pronounced by some as an abbreviation (you-are-ell) and others as an
acronym (earl), with the result that some write it as a URL and others as
an URL. Until a single pronunciation becomes generally accepted, the
best practice is simply to ensure consistency within a given work.

\section{Latin abbreviations}
\normalfont

Do not confuse 'e.g.' (\emph{exempli gratia}), meaning 'for example', with 'i.e.'
{id est), meaning 'that is'. Compare hand tools, e.g. hammer and screwdriver
with hand tools, \ie those able to be held in the user's hands. Print both lower-case roman, with two points and no spaces, and preceded by a
comma. In OUP style 'e.g.' and 'i.e.' are not followed by commas, to avoid
double punctuation; commas are often used in US practice.

Although many people employ 'e.g.' and 'i.e.' quite naturally in speech
as well as writing, prefer 'for example' and 'that is' in running text.
(Since 'e.g.' and 'i.e.' are prone to overuse in text, this convention helps
to limit their profusion.) Conversely, adopt 'e.g.' and 'i.e.' within parentheses
or notes, since abbreviations are preferred there. A sentence in text cannot begin with 'e.g.' or 'i.e.'; however, a note can, in which case
they—exceptionally—remain lower case. The \textit{Oxford Guide} gives an example of exception to the rule

The package offers two commands:

\begin{verbatim}
\newcommand{\ie}{\textit{i.\hairsp{}e.}\xspace}
\newcommand{\eg}{\textit{e.\hairsp{}g.}\xspace}
\end{verbatim}

The commands handle the spacing and if they are to be in italics or not. Renew the commands to set the style you want.

\section{Units}
\label{units}

Most users of \latex will have a need for specifying units in  mathematical or text contexts. We load the \pkgname{siunitx} package. The package was developed by Joseph Wright\cite{siunitx}. The correct application of units of measurement is very important in technical applications. For this reason, carefully-crafted definitions of a coherent units system have been
laid down by the \textit{Conférence Générale des Poids} et Mesures (CGPM): this has resulted in
the \textit{Système International d’Unités} (SI). At the same time, typographic conventions for
correctly displaying both numbers and units exist to ensure that no loss of meaning
occurs in printed matter.

|siunitx| aims to provide a unified method for \latex users to typeset numbers and
units correctly and easily. The design philosophy of |siunitx| is to follow the agreed rules
by default, but to allow variation through option settings. In this way, users can use
|siunitx| to follow the requirements of publishers, co-authors, universities, etc. without needing to alter the input at all.



\begin{ddanger}
Angles can be typeset using the \cs{ang} command.  The
 \meta{angle} can be given either as a decimal number or as a
 semi-colon separated list of degrees, minutes and seconds, which
 is called \enquote{arc format} in this document. The numbers which
 make up an angle are processed using the same system as other numbers.
\end{ddanger}

  \chapter{Electronic Documents}
\label{ch:hyperlinks}

One of the reasons this package was developed was to handle the many conflicts of packages with the
\pkgname{hyperref} package. This package developed by Sebastian Rahtz and Heiko Oberdiek \citeyearpar{hyperref}
has brought the capabilities of the |PDF| format to the \tex world. For a good introductory article
see \cite{garcia}.

\section{Special Effects}

When the package is loaded a number of things happen without further intervention.  The items in the table of contents, the list of
figures, and the list of tables, will be links: when
the reader clicks on them, the cursor will jump
to the corresponding target.

\begin{enumerate}

\item  The superscript that calls for a footnote will be
a link to the footnote itself.
\item Bibliographical references through \cmd{\cite} will
create links to the entries in the final bibliography
list.
\item All pairs of  |\label-\ref| will also produce links
(the result of a |\ref| leading by mouse click to
the corresponding |\label|).

\end{enumerate}

\begin{macro}{\pageref}
\begin{macro}{\nameref}
\label{hyperref01}
The macro \CMDI{\pageref} prints the number of the \emph{page} where the referenced elements appears (rather
than the element itself). The command |\pageref{label}|  \pageref{hyperref01} shows the number of this page.

A third command \CMDI{\nameref} will typeset the name of the chapter or section. This is a much better way than
just printing the number of the chapter.
\end{macro}
\end{macro}

In Example~\ref{ex:hyperlinks} we demonstrate the use of the hyperlinking capabilities of the \cmd{\nameref}
command.  We
also demonstrate the built-in command \CMDI{\pkgname} that we use to link package names to their |ctan| repository.

\begin{texexample}{Hyperlinks with hyperref}{ex:hyperlinks}
In Chapter \ref{ch:hyperlinks} we added the |\label{ch:hyperlinks}|. In the Chapter \nameref{ch:hyperlinks}, we discuss
the use of the \pkgname{hyperref} package.
\end{texexample}




\section{Arbitrary cross references}

\begin{macro}{\hypertarget}\marg{key}\marg{text} will link the target with the key and text.

\end{macro}

The macro \CMDI{\hyperlink}\marg{key}\marg{text}


\hypertarget{linktest}{This is a link}.   The \hyperlink{linktest}{test} when clicked will take you to the text this is a
link.

\def\sectionautorefname{Section}
\def\exampleautorefname{Example}
\def\refexample#1{Example~\ref{#1}\xspace}
\let\exampleref\refexample
\section{Automating referencing}



\label{sec:references}

In \autoref{sec:references} we typed the \cmd{\label}\marg{sec:references}. When we use the \CMDI{\autoref}\marg{sec:references} we get a  reference back to the label. 
The command automatically adds the refname of the sectioning command, appropriately.


\begin{texexample}{Auto referencing}{ex:autoref}
\begin{teX}
The \autoref{ex:autoref} creates a link to this example. 
The \refexample{ex:autoref} also uses a built-in command (*@ \label{refex} @*)
provided by the \pkgname{phd}
\end{teX}
\end{texexample}

As you can observe the Example (see line \ref{refex}), could not be automatically detected by the \pkgname{hyperref} package, so we have build another command \CMDI{refexample} to achieve the effect.

\section{Options}

\begin{macro}{\hypersetup}
The |hyperref| package comes with over 60 options. It is not possible to cover them all here, but we have included
them in the phd package. The options can be set using the \CMDI{\hypersetup}, which provides a convenient
API to hook into options after the loading of the package. 
\end{macro}

\section{How to Link to the Web}

\begin{macro}{\url}
\begin{macro}{\href}
The command \CMDI{\url} takes one argument---the destinations \textsc{URL} address---and creates a link to it. For example, |\url{www.tug.org}}| will open the default system browser at the url address. There is also a related
command \CMDI{\href} that creates a hypelink to the address. This can also be used to indicate an email
address using |\href{mailto:}|\meta{email address}\meta{link text}.
\end{macro}
\end{macro}

\section{Including PDFs in Documents}

The \pkgname{pdfpages}  developed by Andreas Mattias \citeyearpar{pdfpages} can be used to insert external
multi-page PDFs or PS documents. It supports pdfTeX, VTeX and XeTeX. One of the limitations of the current
engines is that any included pdfs will be inserted without any links. 

\begin{macro}{\includepdf}
The main command of the package is \CMDI{\includepdf}\oarg{key=val}\marg{filename}. The key-value list
is used to specify a list of options. The filename is any name that can be found on a TeXpath. (See the \nameref{ch:graphics} for details on paths). 
\end{macro}

On of the pitfalls of the package is if you use a \CMDI{\pagecolor} to set the background of pages. The first
\cmd{\pagecolor}  must precede the loading of the package. With VTeX---and since LuaLaTeX is using VTEX---this is not a limitation and hence we have avoided any routines to by-pass the limitation.








   \chapter{Margin Material}

\section{Introduction}

Since the beginning of writing margins were always used to add notes and other marginal material such as citations, notes and even figures. The \pkgname{marginnote} developed by \citeyearpar{marginnote} does not float the marginal
notes and for a good reason, as most of the time they end up in the wrong place in any case. This chapter describes
the usage of marginpar and not the technicalities of the definitions. The definitions are described in the |float.dtx| in 
\nameref{ltx:marginalnotes}

\section{How to Switch Margin paragraphs}

Marginal notes use the same mechanism as floats to communicate with the \cmd{\output} routine.
Marginal notes are distinguished from floats by having a negative placement specification. [372-373] The \CMDI{\marginpar}\oarg{left text}\marg{right text} generates a marginal note in parbox, using either the left 
text or the right text, depending on the placement. It default to righttext=lefttext. 

\begin{figure}
\includegraphics[width=\textwidth]{marginpar-01}
\end{figure}

Most designers appear to prefer to use only the one side of the page, when margin materials form part of the design of the book, such as the tufte-class. 

Marginal notes are normally put on the outside of the page if the switch \CMDI{\@mparswitch} is true, and on the right if |@mparswitch=false|. The command \CMDI{\reversemarginpar} reverses the side where they are put. \CMDI{\normalmarginpar} undoes |\reversemarginpar|. These commands have no effect in two-colum output.


\begin{figure}
\includegraphics[width=\textwidth]{marginpar-02}
\end{figure}


\lipsum
   \chapter{The Special Environments\\Quotation and Quote}

\label{quotations}


\begin{figure}[p]
\centering
\fbox{\includegraphics[width=0.9\linewidth]{quotations-01}}
\caption{Many books have quotes flushed right.}
\label{frightquotation}
\end{figure}

\begin{figure}[p]
\centering
\fbox{\includegraphics[width=0.9\linewidth]{full-width-quotation}}
\caption[Sample quotation.]{Other books have the quotations full width, but in smaller font as shown above. the extract is from \textit{The Essential Turing}, Edited by B. Jack Copeland and  published by the Oxford University Press, 2004. }
\label{fullwidthquotation}
\end{figure}


\section{Quotation}
In the standard \LaTeXe\ classes the quotation and quote environment are defined by making clever use of the list environment. The main difference between the quotation and the quote environment is that the first line of the former is indented. The key value interface for the quotation environment is shown below and a similar one exists for the quotation environment:


\let\quotation\oquotation
\begin{quotation}
\lipsum[1]
\end{quotation}

The standard classes offer a very similar enevironment with the only difference the first line is not indented and is illustrated below:

\begin{quote}
\lipsum[1]
\end{quote}


\section{Key-value interface}\index{quotation!keys}

\keyval{quote above}{\marg{dim}}{Skip dimension for above quotation skip.}
\keyval{quote below}{\marg{dim}}{Skip dimension for below quotation skip.}
\keyval{quote parindent}{\marg{dim}}{Paragraph indentation.}
\keyval{quote parsep}{\marg{dim}}{Paragraph below skip.}
\keyval{quote left margin}{\marg{dim}}{Paragraph below skip.}
\keyval{quote right margin}{\marg{dim}}{Paragraph below skip.}
\vfill

\index{quotation!example}
\begin{tcblisting}{title=Quotation environment example,width=\textwidth}
\setquotation{%
  quotation above=36pt,
  quotation left margin=30pt,
  quotation right margin=0pt,
  quotation parsep=10pt,
  quotation font-size=\small\color{teal},
  quotation parindent=1em,
}
\lorem

\begin{quotation}
\lipsum[2-3]
\end{quotation}
\end{tcblisting}

\section{Quote}
This is the quote environment:
\begin{quote}
\lipsum[1-2]
\end{quote}


\section{Some commonly used styles}

Besides the centered quotation \fref{fullwidthquotation} shows a style
common in Oxford University Publications. This one is from \textit{The Essential Turing}, Edited by B. Jack Copeland and  published by the Oxford University Press, 2004. Perhaps indicative of the efforts to keep costs down quotations are set at full width, but in smaller font. They
both look good and keep the cost down by reducing the amount of
paper required to print the book.

\topline

Von Neumann gave his engineers `On Computable Numbers' to read when, in
1946, he established his own project to build a stored-programme computer at
the Institute for Advanced Study.\textsuperscript{22} Julian Bigelow, von Neumann's chief engineer,
recollected:
\vspace*{-20pt}

\cxset{quotation example/.style={
  quotation above=0pt,
  quotation left margin=0pt,
  quotation right margin=0pt,
  quotation parsep=10pt,
  quotation font-size=\small\color{teal},
  quotation parindent=1em,
}}

\cxset{quotation turing/.style={
  quotation above=0pt,
  quotation left margin=0pt,
  quotation right margin=0pt,
  quotation parsep=10pt,
  quotation font-size=\small,
  quotation parindent=1em,
}}

\cxset{quotation theme/.code = \setquotation{quotation #1},
       quotation style/.code = \setquotation{quotation #1}}



\cxset{quotation theme = example}




\begin{quotation}

The person who really\ldots pushed the whole Weld ahead was von Neumann, because he
understood logically what [the stored-programme concept] meant in a deeper way than
anybody else\ldots The reason he understood it is because, among other things, he understood
a good deal of the mathematical logic which was implied by the idea, due to the
work of A. M. Turing\ldots in 1936-1937\ldots Turing's [universal] machine does not sound
much like a modern computer today, but nevertheless it was. It was the germinal
idea\ldots So\ldots [von Neumann] saw\ldots that \textsc{[ENIAC]} was just the first step, and that great
improvement would come.\textsuperscript{23}
\end{quotation}

\bottomline

Personally I like this style, especially for books that have a lot
of lengthy citations such as typically found in the humanities and
scientific fields.

\section{Theming}

To make things easier for the designer and to enable easy re-use of
styles we defined a theme key. You first define your keys via
the \cs{cxset} command and then you call it normally using the 
theme. You can extend it, if you like to use subthemes, such as 
quotation theme |quotation theme = example teal|. 

\begin{teX}
\cxset{quotation example/.style={
  quotation above=0pt,
  quotation left margin=0pt,
  quotation right margin=0pt,
  quotation parsep=10pt,
  quotation font-size=\small\color{teal},
  quotation parindent=1em,
}}
\cxset{quotation theme = example}
\end{teX}








   \chapter{Quotations and Other Intrusions}
\normalfont

\epigraph{“What is a quote? A quote (cognate with quota) is a cut, a section, a slice of someone else’s orange. You suck the slice, toss the rind, skate away. Part of what you enjoy in a documentary technique is the sense of banditry. To loot someone else’s life or sentences and make off with a point of view, which is called “objective” because you can make anything into an object by treating it this way, is exciting and dangerous.”}{--- Anne Carson, \textit{Decreation}}

\newthought{For centuries quotations were not used in books}. In the earliest printed books, a quotation was marked merely by naming the speaker.

Three forms of quotation mark are still in common use. Inverted and raised commas --- ``quote'' and
`quote' --- are generally favoured in Britain and North America. But baseline and inverted commas --
are still widely used in Germany. Many typographers prefer them to take the shape of
sloped primes ('---") instead of tailed commas. 

\DescribeMacro{\guillemotleft}
\DescribeMacro{\guillemotright}
\guillemotleft quote\guillemotright\ and <<quote>>. \sidenote{Use the \texttt{\textbackslash{guillemotleft}} and \texttt{\textbackslash{guillemotright}} commands. You can also use the \protect\index{csquote} csquote package}Guillemets or otherwise known as duck foot quotation marks, chevrons, or angle quotes - <<quote>> and <quote> - are the normal form in France and Italy and are widely used in the rest of Europe. German typographers set their guillemets the 
>>opposite way<<. In either case, thin spaces are customary between the guillemets and the text they enclose. Other languages and scripts can use a plethora of different characters to denote quotation marks. For example in Chinese you can find {\arial 「」︰單引號 (Mandarin: dān yǐn hào, Jyutping: daan1 jan5 hou6, lit: "Single quotation mark")
『』︰雙引號 (Mandarin: shuāng yǐn hào, Jyutping: soeng1 jan5 hou6, lit: "Double quotation mark"}). 

\bgroup
\LARGE
\noindent\arial
﹁\\
︰\\
﹂\\
\egroup

When quotation marks (including guillemets) are used, the question remains, how many should be there?
The usual British practice is to use single quotes first, and doubles with singles.


\begin{scriptexample}[]{Poliphilus}

And after shee sayde, Poliphilus lette vs goe and ascende vp this mount nexte the Garden, and Thelemia remayning at the stayre foote, wee ascended vp to the playne toppe. Where shee shewed vnto mee, with a heauenly eloquence, a Garden of a large compasse, made in the forme of an intricate Laborynth allyes and wayes, not to bee troden, but sayled about, for insteade of allyes to treade vppon, there were ryuers of water.

\end{scriptexample}


\section{The apostrophe}

The most common error in text is to use |'s| for plurals of numbers, or for multiple letters. This is unecessary, use the \emph{2010s} or the ABCs.

It is normal to avoid the period after metric units and other self-evident abbreviations. Set 11.3 m and 520 cm but 36 in. or 36", and in bibliographical references, p 36f, or pp 306-314. You can also use the \pkg{siunitx} to give you a consistent set of units in scientific texts.

\section{Parentheses}

I used to introduce a lot of parentheses in my writings until I was shock by the AP Manual of Style:
{\emph The perceived need for parentheses is an indication that your sentence is becoming concorted}. If you do use a parentheses, follow these guidelines:\index{style!parentheses}\index{style!brackets}

\begin{itemize}
\item If the material is inside a sentence, place the period outside the parentheses.
\item If the whole sentence is within brackets, put the full stop inside. (Please remember this.)
\end{itemize}

According to Robert Bringhurst's \index{Robert Bringhurst} \textit{Elements of Typographic Style}, the details of typesetting ellipsis depend on the character and size of the font being set and the typographer's preference. Bringhurst writes that a full space between each dot is "another Victorian eccentricity." In most contexts, the Chicago ellipsis is much too wide"—he recommends using flush dots, or thin-spaced dots (up to one-fifth of an em), or the prefabricated ellipsis character (Unicode \unicodenumber{U+2026} ({\pan \char"2026}), Latin entity \&hellip;) \citep{Bringhurst2005}.\index{ellipis>unicode}\index{ellipsis>\protect\string \ldots}

Bringhurst suggests that normally an ellipsis should be spaced fore-and-aft to separate it from the text, but when it combines with other punctuation, the leading space disappears and the other punctuation follows. He provides the following examples:
i\ldots j	k\ldots.	l\ldots l	l, \ldots l	m\ldots?	n\ldots!

[\dots]\lorem

$[\ldots]$\lorem

...\lorem



This all makes for nice-looking output, but it unfortunately adds a bit
of a burden to your job as a typist, because \tex's rule for determining the end of
a sentence doesn't always work. The problem is that a period sometimes comes
in the middle of a sentence \dots like when it is used (as here) to make an ellipsis" of three dots.

Moreover, if you try to specify \ldots by typing three periods in a row,
you get `...' the dots are too close together. One way to handle this is to go
into mathematics mode, using the |\ldots| control sequence defined in plain TEX
format. For example, if you type

Hmmm |$\ldots$| I wonder why?

the result is `Hmmm $\ldots$ I wonder why?'. This works because math formulas are
exempt from the normal text spacing rules.


\begin{teXXX}
\mathchardef\ldotp="613A % ldot as a punctuation mark
\def\ldots{\mathinner{\ldotp\ldotp\ldotp}}
\end{teXXX}




\section{Foreign Words and Romanization}

Foreign words and phrases used in an English text should be italicised (no
inverted commas) and should have the appropriate accents, e.g. \textit{inter alia,
raison d'\^{e}tre}.

Exceptions: words and phrases now in common use and/or considered part of
the English language, e.g. role, ad hoc, per capita, per se, etc.

\begin{enumerate}
\item Personal names should retain their original accents, e.g. Grybauskait\.{e},
Potočnik, Wallstr\"{o}m. Not to forget as Smith tells us to use the diaresis `where the dividing of two vowels makes two different vowels together may be taken for a dipthong, and make the verse fall short of its measure; as might have happened to the lines underneath, had no di\ae resis been used to prevent it; viz.

{\hskip3cm \narrower\narrower\it

 The Swans that in C\"ayster's water burn.\\
 In flames C\"aicus, Peneus, Alpheus, roll'd.\\
 The Tan\"ais smokes amid the boiling wave.\\

}

\item Quotations. Place verbatim quotations in foreign languages in quotation marks
without italicising the text.

\item Latin. Avoid obscure Latin phrases if writing for a broad readership. When
faced with such phrases as a translator, check whether they have the same
currency and meaning when used in English.

\item The expression `per diem' (`daily allowance') and many others have English
equivalents, which should be preferred e.g. `a year' or `per year' rather than `per annum'.

In general Greek, Cyrillic, Chinese or Arabic scripts should be transliterated, except in specialist texts where the author is sure that his audience has knowledge of the language.

\end{enumerate}

The European Commission Directorate-General for Translation has an English Style Guide that deals in detail with foreign words and phrases in english text and romanization systems.




















    % !TEX TS-program = pdflatex
% !TEX encoding = UTF-8 Unicode
% arara: pdflatex: { synctex: true }
%%% cfr-initials.tex
%%% Copyright 2015 Clea F. Rees
%%
%% This work may be distributed and/or modified under the
%% conditions of the LaTeX Project Public License, either version 1.3
%% of this license or (at your option) any later version.
%% The latest version of this license is in
%%   http://www.latex-project.org/lppl.txt
%% and version 1.3 or later is part of all distributions of LaTeX
%% version 2005/12/01 or later.
%%
%% This work has the LPPL maintenance status `maintained'.
%%
%% The Current Maintainer of this work is Clea F. Rees.
%%
%% This work consists of all files listed in manifest.txt.
%\listfiles
%\documentclass[11pt,british,a4paper]{article}
%\usepackage{babel}
%\usepackage[utf8]{inputenc}
%\usepackage[T1]{fontenc}
%\usepackage{textcomp,cfr-lm}
%\usepackage{fancyhdr,pageslts}
%\usepackage{longtable,verbatim}
%\usepackage{lettrine}
%\usepackage{booktabs,url}
%\usepackage{microtype}
%\usepackage[headheight=14pt,vscale=.71]{geometry}	% use 14pt for 11pt text, 15pt for 12pt text
%\usepackage{parskip}
%\usepackage{Acorn, AnnSton, ArtNouv, ArtNouvc, Carrickc, Eichenla, Eileen, EileenBl, Elzevier, GotIn, GoudyIn, Kinigcap, Konanur, Kramer, MorrisIn, Nouveaud, Romantik, Rothdn, Royal, Sanremo, Starburst, Typocaps, Zallman}
%
%\title{cfr-initials}
%\author{Clea F.\ Rees\footnote{reesc21 <at> cardiff <dot> ac <dot> uk}}
%\def\dyddiad{2015--04--06}
%\def\fyversion{Version 1.01}
%\date{\fyversion\ --- \dyddiad}
%\pagestyle{fancy}
%\fancyhf[lh]{\itshape \fyversion}
%\fancyhf[rh]{\itshape \dyddiad}
%\fancyhf[ch]{\itshape cfr-initials}
%\fancyhf[lf]{}
%\fancyhf[rf]{}
%\fancyhf[cf]{\itshape --- \thepage~of~\lastpageref*{LastPage} ---}

\chapter {Decorating Text}

\section{Initials}


In a written or published work, an initial is a letter at the beginning of a word, a chapter, or a paragraph that is larger than the rest of the text. The word is derived from the Latin initialis, which means standing at the beginning. An initial often is several lines in height and in older books or manuscripts, sometimes ornately decorated.

In illuminated manuscripts, initials with images inside them, such as those illustrated here, are known as historiated initials. They were an invention of the Insular art of the British Isles in the eighth century. Initials containing, typically, plant-form spirals with small figures of animals or humans that do not represent a specific person or scene are known as "inhabited" initials. Certain important initials, such as the B of Beatus vir... at the opening of Psalm 1 at the start of a vulgate Latin psalter, could occupy a whole page of a manuscript.

These specific initials, in an illuminated manuscript, also were called Initiums.

\section{Brief history of the initial}

\dropcap{A}{set} of sixteenth-century initial capitals, which is missing a few letters
The classical tradition was late to use capital letters for initials at all; in surviving Roman texts it often is difficult even to separate the words as spacing was not used either. In the Late Antique period both came into common use in Italy, the initials usually were set in the left margin (as in the third example below), as though to cut them off from the rest of the text, and about twice as tall as the other letters. The radical innovation of insular manuscripts was to make initials much larger, not indented, and for the letters immediately following the initial also to be larger, but diminishing in size (called the "diminuendo" effect, after the musical notation). Subsequently they became larger still, coloured, and penetrated farther and farther into the rest of the text, until the whole page might be taken over. The decoration of insular initials, especially large ones, was generally abstract and geometrical, or featured animals in patterns. Historiated initials were an Insular invention, but did not come into their own until the later developments of Ottonian art, Anglo-Saxon art, and the Romanesque style in particular. After this period, in Gothic art large paintings of scenes tended to go in rectangular framed spaces, and the initial, although often still historiated, tended to become smaller again.

In the very early history of printing the typesetters would leave blank the necessary space, so that the initials could be added later by a scribe or miniature painter. Later initials were printed using separate blocks in woodcut or metalcut techniques.


\ExplSyntaxOn
\clist_new:N\dropcapslist
\clist_gset:Nn \dropcapslist
    {\Royal,\Romantik,\EileenBlfamily,\Zallmanfamily,\Konanurfamily,\Starburstfamily,\Typocapsfamily, \ArtNouvcfamily,\Kramerfamily,\GotInfamily,\Sanremofamily,\ArtNouvfamily,}
\ExplSyntaxOff

\setcounter{DefaultLines}{3}%

\makeatletter

\long\def\lettrinetest#1{%
\renewcommand\LettrineFontHook{\color{bgsexy}#1}
\leavevmode
\dropcap{G}{e} any dedicated reader can clearly see, the Ideal of practical reason is a representation of, as far as I know, the things in themselves; as I have shown elsewhere, the phenomena should only be
used as a canon for our understanding. This is of course some nonsense text to see what is goinf wrong, with some calculations.
\texttt{\string#1} \the\@tempcnta, \the\@tempcntb

\drawfontbox{{\Huge\color{thegreen}#1Q}}
\par
}
	

 
\ExplSyntaxOn
 \clist_map_inline:Nn\dropcapslist {
     \lettrinetest{#1}\relax
        }
\ExplSyntaxOff
	
\makeatother


\begin{latexquote}
  \hspace*{-\parindent}\pkgname{cfr-initials} by Clea F.\ Rees is a set of 23 tiny packages designed to make it easier to use the decorative and ornamental initials provided by \pkgname{initials} in \LaTeX.
  \pkgname{initials} provides 23 such fonts in type 1 format, together with the support files required to use them.
  They cannot be used straightforwardly in \LaTeX, however, because the lack of package files providing ready-to-use commands is complicated by the non-standard naming of the font definition files.
  \pkgname{cfr-initials} is designed to make good that deficit.|http://www.1001fonts.com/users/steffmann/|
\end{latexquote}

\section{Using the fonts}\label{sec:usage}

To access the fonts, you simply load the relevant package in your preamble.
For example:
\begin{verbatim}
  \usepackage{Zallman}
\end{verbatim}

Each package provides two new commands.
The first is a font \emph{switch}.
It switches to the relevant set of initials until the end of the group, or until another command switches to a different family.
You will rarely wish to use these commands directly but they are useful in the definitions of macros.
For example, these are the commands you will need if using the fonts with the \pkgname{lettrine} package.
(See section \ref{sec:lettrine} for examples.)

The second takes a single, mandatory argument and typesets that argument in the appropriate font.

\begin{longtable}{llllll}
  \toprule
  \bfseries Family	& \bfseries Package	&	\bfseries Switch	&	\bfseries Command	& \verb|ABC| & \verb|abc|	\\\midrule\endhead
  \bottomrule\endfoot
  Acorn & \pkgname{Acorn} & \verb|\Acornfamily| & \verb|\acorn{}| & \acorn{ABC} & \acorn{abc} \\
  AnnSton & \pkgname{AnnSton} & \verb|\AnnStonfamily| & \verb|\astone{}| & \astone{ABC} & \astone{abc} \\
  ArtNouv & \pkgname{ArtNouv} & \verb|\ArtNouvfamily| & \verb|\artnouv{}| & \artnouv{ABC} & --- \\
  ArtNouvc & \pkgname{ArtNouvc} & \verb|\ArtNouvcfamily| & \verb|\artnouvc{}| & \artnouvc{ABC} & \artnouvc{abc} \\
  Carrickc & \pkgname{Carrickc} & \verb|\Carrickcfamily| & \verb|\carr{}| & \carr{ABC} & \carr{abc} \\
  Eichenla & \pkgname{Eichenla} & \verb|\Eichenlafamily| & \verb|\eichen{}| & \eichen{ABC} & \eichen{abc} \\
  Eileen & \pkgname{Eileen} & \verb|\Eileenfamily| & \verb|\eileen{}| & \eileen{ABC} & \eileen{abc} \\
  EileenBl & \pkgname{EileenBl} & \verb|\EileenBlfamily| & \verb|\eileenbl{}| & \eileenbl{ABC} & \eileenbl{abc} \\
  Elzevier & \pkgname{Elzevier} & \verb|\Elzevier| & \verb|\elz{}| & \elz{ABC} & \elz{abc} \\
  GotIn & \pkgname{GotIn} & \verb|\GotInfamily| & \verb|\gotin{}| & \gotin{ABC} & --- \\
  GoudyIn & \pkgname{GoudyIn} & \verb|\GoudyInfamily| & \verb|\goudyin{}| & \goudyin{ABC} & --- \\
  Kinigcap & \pkgname{Kinigcap} & \verb|\Kinigcapfamily| & \verb|\kinig{}| & \kinig{ABC} & \kinig{abc} \\
  Konanur & \pkgname{Konanur} & \verb|\Konanurfamily| & \verb|\konanur{}| & \konanur{ABC} & \konanur{abc} \\
  Kramer & \pkgname{Kramer} & \verb|\Kramerfamily| & \verb|\kramer{}| & \kramer{ABC} & \kramer{abc} \\
  MorrisIn & \pkgname{MorrisIn} & \verb|\MorrisInfamily| & \verb|\morrisin{}| & \morrisin{ABC} & \morrisin{abc} \\
  Nouveaud & \pkgname{Nouveaud} & \verb|\Nouveaudfamily| & \verb|\nouvd{}| & \nouvd{ABC} & \nouvd{abc} \\
  Romantik & \pkgname{Romantik} & \verb|\Romantik| & \verb|\romantik{}| & \romantik{ABC} & \romantik{abc} \\
  Rothdn & \pkgname{Rothdn} & \verb|\Rothdnfamily| & \verb|\roth{}| & \roth{ABC} & \roth{abc} \\
  Royal & \pkgname{Royal} & \verb|\Royal| & \verb|\royal{}| & \royal{ABC} & \royal{QFR}\\
  Sanremo & \pkgname{Sanremo} & \verb|\Sanremofamily| & \verb|\sanremo{}| & \sanremo{ABC} & \sanremo{abc} \\
  Starburst & \pkgname{Starburst} & \verb|\Starburstfamily| & \verb|\starburst{}| & \starburst{ABC} & \starburst{abc} \\
  Typocaps & \pkgname{Typocaps} & \verb|\Typocapsfamily| & \verb|\typocap{}| & \typocap{ABC} & \typocap{abc} \\
  Zallman & \pkgname{Zallman} & \verb|\Zallmanfamily| & \verb|\zall{}| & \zall{ABC} & \zall{abc} \\
\end{longtable}
\clearpage

\section{Sample lettrines}\label{sec:lettrine}

\lettrinetest{\Royal,\Romantik,\ArtNouvfamily,\EileenBlfamily,\Zallmanfamily,\Konanurfamily,\Starburstfamily,\Typocapsfamily,\ArtNouvcfamily}%,\Kinigcapfamily,\Kramerfamily,\GotInfamily}


\endinput


%\chapter{Poetry}

\epigraph{"Carpe diem. Seize the day, boys. Make your lives extraordinary."}{---John Keating, Dead Poets Society (1989)}

\newcommand{\garden}{
I used to love my garden \\
But now my love is dead \\
For I found a bachelor's button \\
In black-eyed Susan's bed.
}
Typographical standards require poetry to be `'centered on the longest line, unless such line is disproportionately long, in which case optical centring''. \textit{The oxford Dictionary for Writers and Editors}, which presents the house style of
the Oxford University Press). 


\section{Using the verse package}



\begin{lstlisting}[language={[common]TeX},% 
                           alsolanguage={[LaTeX]TeX},% 
                           alsolanguage={[primitive]TeX},%
                           alsolanguage={extras}]
\renewcommand{\poemtoc}{subsection}
\poemtitle{A Limerick}
\settowidth{\versewidth}{There was an old party of Lyme}
\begin{verse}[\versewidth]
    There was an old party of Lyme \\
    Who married three wives at one time. \\
    \vin When asked: `Why the third?' \\
    \vin He replied: `One's absurd, \\
    And bigamy, sir, is a crime.'
\end{verse}
\end{lstlisting}


which gets typeset as below. The default \doccmd{poemtoc}  is redefined to subsection so
the title is entered into the ToC as an unnumbered subsection.

\newlength{\aaaa}
\settowidth{\aaaa}{ZZZZ}

\renewcommand{\poemtoc}{subsection}
\poemtitle{A Limerick}
\settowidth{\versewidth}{There was an old party of Lyme}
\begin{verse}[\versewidth]
    There was an old party of Lyme \\
     Who married three wives at one time. \\
    \vin When asked: `Why the third?' \\
    \ZZZZ He replied: `One's absurd, \\
And bigamy, sir, is a crime.'
\end{verse}

Next is the Garden verse within the altverse environment. It is titled and
centered.

\settowidth{\versewidth}{But now my love is dead}
\poemtitle{Love's lost}
\begin{verse}[\versewidth]
\begin{altverse}
\garden
\end{altverse}
\end{verse}


which produces:

\settowidth{\versewidth}{But now my love is dead}
\poemtitle{Love's lost}
\begin{verse}[\versewidth]
    \begin{altverse}
    \garden
\end{altverse}
\end{verse}


It is left up to you how you might want to add information about the author
of a poem. Here is one example of a macro for this:

\begin{lstlisting}[language={[common]TeX},% 
                           alsolanguage={[LaTeX]TeX},% 
                           alsolanguage={[primitive]TeX},%
                           alsolanguage={extras}]
\newcommand{\attrib}[1]{%
\nopagebreak{\raggedleft\footnotesize #1\par}}
\end{lstlisting}

\newcommand{\attrib}[1]{%
\nopagebreak{\raggedleft\footnotesize #1\par}}
This can be used as in the next bit of doggeral.

\poemtitle{Fleas}
\settowidth{\versewidth}{What a funny thing is a flea}
\begin{verse}[\versewidth]
What a funny thing is a flea. \\
You can't tell a he from a she. \\
 But he can. And she can. \\
 Whoopee!
\end{verse}
\attrib{Anonymous}


Here is an example of line wrapping.
\poemtitle{In the beginning}
\settowidth{\versewidth}{And objects at rest tended to remain at rest}
\begin{verse}[\versewidth]
Then God created Newton, \\*
And objects at rest tended to remain at rest, \\*
And objects in motion tended to remain in motion, \\*
And energy was conserved
and momentum was conserved
and matter was conserved \\*
And God saw that it was conservative.
\end{verse}
\attrib{Possibly from \textit{Analog}, circa 1950}



Here is one with a forced line break and a slightly different title style.

\begin{lstlisting}[language={[common]TeX},% 
                           alsolanguage={[LaTeX]TeX},% 
                           alsolanguage={[primitive]TeX},%
                           alsolanguage={extras}]
\renewcommand{\poemtitlefont}{\normalfont\large\itshape\centering}
\poemtitle{Mathematics}
\settowidth{\versewidth}{Than Tycho Brahe, or Erra Pater:}
\begin{verse}[\versewidth]
    In mathematics he was greater \\
    Than Tycho Brahe, or Erra Pater: \\
    For he, by geometric scale, \\
   Could take the size of pots of ale;\\ \settowidth{\versewidth}{Resolve by}
   Resolve, by sines \\>[\versewidth] and tangents straight, \\
   If bread or butter wanted weight; \\
   And wisely tell what hour o' the day \\
   The clock does strike, by Algebra.
\end{verse}
\attrib{Samuel Butler (1612--1680)}
\end{lstlisting}

The typesetting now is slightly different but still not what is probably required in a poetry book

\begin{verse}[\versewidth]
    In mathematics he was greater \\
    Than Tycho Brahe, or Erra Pater: \\
    For he, by geometric scale, \\
   Could take the size of pots of ale;\\ \settowidth{\versewidth}{Resolve by}
   Resolve, by sines \\>[\versewidth] and tangents straight, \\
   If bread or butter wanted weight; \\
   And wisely tell what hour o' the day \\
   The clock does strike, by Algebra.
\end{verse}
\attrib{Samuel Butler (1612--1680)}
\bigskip


Another limerick, but this time taking advantage of the patverse environment
and numbering every third line.

\begin{lstlisting}[language={[common]TeX},% 
                           alsolanguage={[LaTeX]TeX},% 
                           alsolanguage={[primitive]TeX},%
                           alsolanguage={extras}]
\settowidth{\versewidth}{There was a young lady of Ryde}
\poemtitle{The Young Lady of Ryde}
\begin{verse}[\versewidth]
\poemlines{3}
\indentpattern{00110}
\begin{patverse}
There was a young lady of Ryde \\
Who ate some apples and died. \\
The apples fermented \\
Inside the lamented \\
And made cider inside her inside.
\end{patverse}
\poemlines{0}
\end{verse}
\end{lstlisting}


The  poem on the next page is a bit more involved. Here we use a bigskip between the verses of the poem. Also note
the use of the emdash, which is commonly found in poetry books. The command \doccmd{vin} is from the verse class
and it justs sets the second line in. The figure which is from the same publication was set with a marginfigure and the
vertical height was manually adjusted. Poetry typesetting is highly unlikely to be done automatically. Each poem is
special and would normally be typset to suit.

\begin{lstlisting}[language={[common]TeX},% 
                           alsolanguage={[LaTeX]TeX},% 
                           alsolanguage={[primitive]TeX},%
                           alsolanguage={extras}]
\begin{verse}[\versewidth]
   See the beetle that crawls in your way,\\
  \vin And runs to escape from your feet;\\
   His house is a hole in the clay,\\
   \vin And the bright morning dew is his meat.\\
\bigskip
   But if you more closely behold\\
   \vin This insect you think is so mean,\\
   You will find him all spangled with gold,\\
  \vin And shining with crimson and green.\\
\end{Verse}
\end{lstlisting}

\clearpage
\poemtitle{THE BEETLE}
\marginpar{%
\hspace*{-30pt}\hbox to 0pt{\includegraphics[width=100pt]{./images/beetle.jpg}}
  \label{fig:beetle}
}

\begin{verse}[\versewidth]
See the beetle that crawls in your way,\\
\vin And runs to escape from your feet;\\
His house is a hole in the clay,\\
\vin And the bright morning dew is his meat.\\
\bigskip
But if you more closely behold\\
\vin This insect you think is so mean,\\
You will find him all spangled with gold,\\
\vin And shining with crimson and green.\\
\bigskip
Tho' the peacock's bright plumage we prize,\\
\vin As he spreads out his tail to the sun,\\
The beetle we should not despise,\\
\vin Nor over him carelessly run.\\
\bigskip
They both the same Maker declare---\\
\vin They both the same wisdom display,\\
The same beauties in common they share---\\
\vin Both are equally happy and gay.\\
\bigskip
And remember that while you would fear\\
\vin The beautiful peacock to kill,\\
You would tread on the poor beetle here,\\
\vin And think you were doing no ill.\\
\bigskip
But though 'tis so humble, be sure,\\
\vin As mangled and bleeding it lies,\\
A pain as severe 'twill endure,\\
\vin As if 'twere a giant that dies\sidenote{Anonymous, \textit{Illustrared London Book} 1851}.\\
\end{verse}

\clearpage
\marginpar{%
 \includegraphics[width=80pt]{./images/byron.jpg}
  \label{fig:beetle}
}
\poemtitle{ON JORDAN'S BANKS}
\begin{verse}[\versewidth]
On Jordan's banks the Arab camels stray,\\
On Sion's hill the False One's votaries pray---\\
The Baal-adorer bows on Sinai's steep;\\
Yet there---even there---O God! thy thunders sleep:\\
\bigskip
There, where thy finger scorch'd the tablet stone;\\
There, where thy shadow to thy people shone---\\
Thy glory shrouded in its garb of fire\\
(Thyself none living see and not expire).\\
\bigskip
Oh! in the lightning let thy glance appear---\\
Sweep from his shiver'd hand the oppressor's spear!\\
How long by tyrants shall thy land be trod?\\
How long thy temple worshipless, O God!\\
\end{verse}



\begin{comment}
 \clearpage
 \poemtitle{Mouse's Tale}
 \settowidth{\versewidth}{a mouse that morning}
 \indentpattern{0135554322112346898779775545653222345544456688778899}
 \begin{verse}[\versewidth]
 \setlength{\vgap}{1em}
 \begin{patverse}
 \large Fury said to \\
   a mouse, That \\
   he met \\
   in the \\
   house, \\
 \normalsize `Let us \\
   both go \\
   to law: \\
   \emph{I} will \\
   prosecute \\
   \textit{you.} --- \\ 
   Come, I'll \\
 \small take no \\
   denial; \\
   We must \\
   have a \\
   trial: \\
   For \\
 \footnotesize really \\
   this \\
   morning \\
   I've \\
   nothing \\
   to do.' \\
   Said the \\
   mouse to \\
 \scriptsize the cur, \\
   Such a \\
   trial, \\
   dear sir, \\
   With no \\
   jury or \\
   judge, \\
   would be \\
   wasting \\
   our breath.' \\
 \tiny  `I'll be \\
   judge, \\
   I'll be \\
   jury.' \\
   Said \\
   cunning \\
   old Fury; \\
   `I'll try \\
   the whole \\
   cause \\
   and \\
   condemn \\
   you \\
   to \\
   death.'  \par
 \end{patverse}
 \end{verse}
 \attrib{Lewis Carrol, \textit{Alice's Adventures in Wonderland}, 1865}
 
\clearpage
\end{comment}


 Using the |alltt| environment you can put in the spacing via ordinary
 spaces. That is, this

\begin{texexample}{}{}
 \begin{alltt}\normalfont
 There was an old party of Lyme
 Who married three wives at one time.
       When asked: `Why the third?' 
       He replied: `One's absurd, 
 And bigamy, sir, is a crime.'
 \end{alltt}
\end{texexample}

 is typeset as

 \begin{alltt}
 \normalfont
 There was an old party of Lyme
 Who married three wives at one time.
       When asked: `Why the third?' 
       He replied: `One's absurd, 
 And bigamy, sir, is a crime.'
 \end{alltt}


\begin{codeexample}[]
\begin{tikzpicture}[scale=0.5]
\def \n {5}
\def \radius {3cm}
\def \margin {12} 

\foreach \s[count=\xi from 0] in {1,...,\n}
{
  \node[draw, circle] at ({360/\n * (\s - 1)}:\radius) {$\xi$};
  \draw[->, >=latex] ({360/\n * (\s - 1)+\margin}:\radius) 
    arc ({360/\n * (\s - 1)+\margin}:{360/\n * (\s)-\margin}:\radius);
}
\end{tikzpicture}
\end{codeexample}








  \makeatletter

\thispagestyle{plain}

\cxset{image={./images/breakerboys.jpg},
       subsection font-shape= upshape,}

\usemintedstyle{friendly}

\chapter{Listings styles}  

Many users of \latex require to typeset formatted code. There are two packages that
can be used the more conventional \pkgname{listings}\footcite{listings} and \pkgname{minted}\footcite{minted}. The
|minted| package is a more powerful and flexible package than listing, since it uses
an external program |Pygments| which is written in |Python|\footnote{See \protect\url{http://pygments.org/} for more details. You can also review and contribute to the code at \protect\url{https://bitbucket.org/birkenfeld/pygments-main}}. My recommendation to you is
to use the |minted package|. The two packages can happily co-exist and each one has
its own advantages and disantvantages. The listings package has all its color
parameters configurable via its \latex key value settings, whereas the pygments program
has its own way of setting these styles, which are only accessible through \latex
as a set of fixed styles. To create a new color scheme, you will need to write some
simple |python|, register it as a plugin or drop it at the folder holding the styles.\footnote{See documentation at \protect\url{http://pygments.org/docs/styles/}.}

For me
it is the only limiting factor of pygments and which there are ways around it. However, this might be also an advantage
as users are more likely to be familiar with such code coloring schemes in their language.

\section{Using minted}

Since minted makes calls to the outside world (that is, Pygments), you need to
tell the \latex processor about this by passing it the |-shell-escape| option or it
won’t allow such calls. In effect, instead of calling the processor like this:

\usemintedstyle{friendly}
\begin{minted}[fontsize=\footnotesize,style=vim]{bash}
$ latex input
you need to call it like this:
$ latex -shell-escape input
\end{minted}

The same holds for other processors, such as pdf\latex or \xelatex.
You should be aware that using -shell-escape allows \latex to run potentially
arbitrary commands on your system. It is probably best to use -shell-escape
only when you need it, and to use it only with documents from trusted sources.

\subsection{A minimal example}   
 
The minted package is loaded like any other package (with or without options). 
You can then use the \docAuxEnv{minted} environment with the language we want to use
as the first argument. The environment also takes an optional argument where the numerous
settings of the package can be specified.
 
\begin{phdverbatim}[basicstyle=\small\ttfamily]
\documentclass{article}
\usepackage{minted}
\begin{document}
\begin{minted}[fontsize=\footnotesize,style=friendly]{javascript}
if (Meteor.isClient) {
  // This code only runs on the client
  Template.body.helpers({
    tasks: [
      { text: "This is task 1" },
      { text: "This is task 2" },
      { text: "This is task 3" }
    ]
  });
}
\end{minted}
\end{document}
\end{phdverbatim}

This will produce an output as:
\medskip

\begin{minted}[fontsize=\footnotesize,style=friendly]{javascript}
if (Meteor.isClient) {
  // This code only runs on the client
  Template.body.helpers({
    tasks: [
      { text: "This is task 1" },
      { text: "This is task 2" },
      { text: "This is task 3" }
    ]
  });
}
\end{minted}

If we do not need a style, the |style=default| setting will typeset as,

\begin{minted}[fontsize=\footnotesize,style=trac]{javascript}
if (Meteor.isClient) {
  // This code only runs on the client
  Template.body.helpers({
    tasks: [
      { text: "This is task 1" },
      { text: "This is task 2" },
      { text: "This is task 3" }
    ]
  });
}
\end{minted}

You can also set the style for the whole document using:

\begin{minted}[fontsize=\footnotesize,style=trac]{TeX}
\usemintedstyle{<name>}
\end{minted}
where you can get <name> by typing

\begin{minted}[fontsize=\footnotesize,style=bw]{bash}
$ pygmentize -L styles
\end{minted}
at the command prompt/terminal. For example, the minted documentation itself uses the |trac| style.

\begin{minted}[fontsize=\footnotesize,style=friendly]{html}
<!-- First set the doctype -->
<!DOCTYPE html>
    <html>
      <head>
        <title>Canvas</title>
        <meta charset="UTF-8" />
        <style>
          #square {
            border: 1px solid black;
                    transform: scale(10) rotate(3deg) translateX(0px);
                    -moz-transform: scale(10) rotate(3deg) translateX(0px);
          }

          .box {              
                    transition-duration: 2s;
                    transition-property: transform;
                    transition-timing-function: linear;
          }
        </style>
      </head>
      <body>
        <canvas id="square" width="200" height="200"></canvas>
        <script>
                var canvas = document.createElement('canvas');
                canvas.width = 200;
                canvas.height = 200;

                var image = new Image();
                image.src = 'images/card.png';
                image.width = 114;
                image.height = 158;
                image.onload = window.setInterval(function() {
                    rotation();
                }, 1000/60);
       </script>
      </body>
    </html>
\end{minted} 
   



\begin{lstlisting}
int main() {
printf("hello, world");V\colorbox{green}{**}V
return 0;
}
\end{lstlisting}



\chapter{Documentation Macros}


\section{Documentation macros}

When developing this package the need arose to define a number of documentation macros. I~have used heavily macros and ideas present in the \pkg{doc} package, \pkg{pgf} documentation, \pkg{biblatex} documentation  and \pkg{tcolorbox} and for which I am grateful to their respective authors. The major change was to adopt the macros to use different fonts and colors and to use these from a list of key values defined at document level. More about this later. General package user documentation as opposed to package documentation that can be achieved using the |doc/docstrip| system requires that macros and environments be developed for the following:

\begin{enumerate}
\item Macros for command documentation.
\item Environments for commands and options.
\item Latex examples that need to be executed within the document as well as described.
\end{enumerate}


\section{Commands and Styles for Documenting macros}

The most commonly used commands for documenting macros are |\cs|, |\cmd|, |\meta|, |\marg|, |\oarg|. These commands have been defined by many authors and perhaps the best implementation can be found in the \pkg{doc}. Many package authors have redefined them in their documention, some if just to add a bit of colour, others to have them add the command to an index. As we also had a target to allow for
the package to be used in both normal documents as well as documentation
of packages and classes that use the \pkg{doc} and \pkg{docstrip} combination we provided many compatible macros.

\begin{environment}{macro} The environment macro is made available in this
package. 
\end{environment}

\DescribeMacro{macro} The environment macro is made available in this package. 

\begin{macro}{\cmd} The command \cmd{\cmd} typesets its argument in
  verbatim. Typing |\cmd{\cmd}| typsets \cmd{\cmd}. If the class
  |ltxdoc| is loaded the command is defined there. We have modified
  it to accept a colour and changes to the verbatim font 
  for consistency.
\end{macro}

\begin{macro}{meta}
The macro \cs{meta} is normally used to build other commands. On its own it can be used to typeset
examples of the argument of macros, typing |meta{Aristotle}| will typeset meta{Aristotle}. The command provides a hook to set the font via a macro |\meta@font@select|. 
\end{macro}


|\def\meta@font@select{\upshape\color{black}}|


\subsection{Color management}
One of the first requirements for redefining some of the standard doc commands is the need to use color easily, hence we will try and define a certain amount of keys for colors.

Just a bit of a refresher, to define colors we use, either the \cs{definecolor} or the \cs{colorlet} commands.

\emphasis{definecolor,colorlet}

\begin{minted}[fontsize=\small]{TeX}
\definecolor{Hyperlink}{rgb}{0.281,0.275,0.485}
\colorlet{thehyperlink}{theblue}
\end{minted}


We use a semantic approach, where the colors are first defined with a mnemonic command such as {\bfseries\textcolor{theblue}{theblue}} and then we define a semantic command such as the\cs{option} that lets the color to the option command. This sort of double entry has proved useful in navigating through the dozen of the commands that I needed for this documentation.


\subsection{Semantic color names}
\begin{marglist}
\item [\option{theoption}] Coloring of options in margin lists.
\item [\option{themacro}] Coloring of command macros \cs{foo}.
\item [\option{hyperlink}] If we use the \texttt{hyperref} package a number of colors need to be defined for links.
\end{marglist}

\subsection{Named colors}
Standard colors that we provide are:
\begin{marglist}
\item [\textcolor{theblue}{theblue}] This color is used mainly for options.
\item [\textcolor{thered}{thered}] The color mostly used for macro commands and keys.
\item [\textcolor{thegreen}{thegreen}] used for environments.
\item [\textcolor{thelightgreen}{thelightgreen}] Used for margin lists.
\item [\textcolor{thegray}{thegray}] Used as a background to the listings.
\item [\colorbox{thegrey}{\color{white}thegrey}] Alias for the gray to satisfy both sides of the Atlantic and as I sometimes don't remeber which is which.
\item [\colorbox{theshade}{theshade}] Another slightly lighter shade.
\end{marglist}



\begin{marglist}
\item [\cs{cs}] \cs{cs} text Prints a command.
\item [\cs{cmd}] Prints a command.
\end{marglist}




\section{Lists for documentation}



The environment \env{marglist}
\begin{marglist}
\item[testing]\lorem
\item [test]\lorem
\end{marglist}

\env{keymarginlist}This environment is suitable for listing keys, set-in the margin.

\begin{keymarglist}
\item[bibliography] The term <bibliography>, also available as \cmd{\bibname}.
\item[references] The term <references>, also available as \cmd{refname}.
\item[shorthands] The term <list of shorthands> or <list of abbreviations>, also available as \cmd{losname}.
\end{keymarglist}


\env{argumentlist} This environment is suitable for listing macro arguments and their explanations.



\section{Breakable Boxes}

The \pkg{mdframed} as well as the newer versions of \pkg{tcolorbox}
offer breakable boxes.


\begin{tcolorbox}[enhanced, breakable,
  colback=blue!5!white,colframe=blue!75!black,title=Breakable box,
  watermark color=white,watermark text=\Roman{tcbbreakpart}]
  \lipsum[1-3]
\end{tcolorbox}

\section{PGF Style Code Boxes}

\begin{codeexample}[]
\begin{tikzpicture}
  \node[place,label=above:$p_1$,tokens=2]        (p1) {};
  \node[place,label=below:$p_2\ge1$,right=of p1] (p2) {};
\end{tikzpicture}
\end{codeexample}






  %FIX LIST DIAGRAM

\cxset{steward,
  chapter name=chapter,
  chapter format= stewart,
   image={sweepers.jpg},
  texti={Lists are essential elements of any document style and perhaps the most troublesome to get right.
         In this chapter we discuss the construction of lists and offer a key value interface.},
  textii={The Chapter discusses in detail the construction of lists. It reviews the mechanisms offered
          by LaTeX and outlines a key value approach to building lists. We define a standard interface that does not
          interfere with the original commands. The three standard list styles \textit{enumerate, itemize} and \textit{description} are redesigned to accept a key value interface. The photograph is Lewis Hine's which noted: ``Ivey Mill Company, Hickory, N.C. Some doffers and sweepers. Plenty of them.'' Location: Hickory, Catawba County Date: November 1908. Photographs like this were used by Hine to campaign against child labour.
         }
}

\def\storyi{Lists are essential elements of any document style and perhaps the most troublesome to get right.
         In this chapter we discuss the standard lists offered in the LaTeX classes and describe how new lists can be constructed. We review and use:
         
         \begin{enumerate}
           \item enumerate
           \item itemize
           \item description
           \item trivlist
         \end{enumerate}
   
Some commonly used packages are also reviewed.        
         }

\cxset{palette spring onion}
\pagestyle{headings-spring-onion}
\makeatletter
\def\imagewidth@cx{6cm}
\makeatother
\cxset{chapter format=fashion,
       fashion image=fashion-pngtree.png}

\chapter{Standard \LaTeX\ Lists}

\pagebreak

\section{Introduction}

There are four environments for producing formatted lists:\footnote{There are also other environments, such as \emph{quote}, %
\emph{quotation}, \emph{verbatim}, which behind the scenes are also lists.}

\begin{trivlist}
\item |\begin|\marg{trivlist} list text |\end{trivlist}|
\item |\begin|\marg{itemize} list text |\end{itemize}|
\item |\begin|\marg{enumerate} list text |\end{enumerate}|
\item |\begin|\marg{description} list text |\end{description}|
\end{trivlist}

Lists shape their contents so that 
 the \emph{list text} is indented from the left margin
and a label, or marker, is included. What type of label is used depends
on the selected list environment. The command to produce the label is |\item|. Any following paragraphs, i.e., paragraphs types without being prefixed by |\item| are at the same distance from the margin. 

Lists can be nested either mixed or of one type to a depth of four levels. The type of label used depends on the level of nesting. The indentation is always relative to the left margin or right margin of the enclosing list.


\cxset{label itemi = \textbullet,
       label itemii = *}

\begin{itemize}
\item This is the first level of the list.
  \begin{itemize}
     \item This is the second level of the list.
  \end{itemize}
\item And back to the first level.
\end{itemize}

The optional argument of the |\item| can be used to change the label in the itemize and enumerate environments. The optional
argument takes precedence over the standard label. For the enumerate environment, this means that the corresponding counter is not automatically incremented. You will need to do the numbering manually.



\begin{texexample}{Example with manual settings}{ex:settings}
\section{Example of the itemize environment}
\begin{itemize}
\item[---] This is the first level of the list.
  \begin{itemize}
     \item[\textbullet] This is the second level of the list.
  \end{itemize}
\item[---] And back to the first level.
\end{itemize}
\end{texexample}

The optional label appears right justified within the area reserved for
the label. The width of this area is the amount of indentation at that level
less the separation between label and text; this means that the left edge
of the label area is flush with the left margin of the enclosing level.

It is also possible to change the standard labels for all or part of the
document. The labels are generated with the internal commands

\bgroup
\trivlist\item
\cs{labelitemi}, \cs{labelitemii}, \cs{labelitemiii}, \cs{labelitemiv}, 
\cs{labelenumi}, \cs{labelenumii}, \cs{labelenumiii}, \cs{labelenumiv}
\endtrivlist
\egroup

The endings i, ii, iii, and iv refer to the four possible levels.
These commands may be altered with \cs{renewcommand}. For example,
to change the label of the third level of the itemize environment to a checkmark (\ding{51}), we can write:



\begin{quote}\small
|\renewcommand{\labelitemiii}{\ding{51}}|
\end{quote}

The symbol |\ding{31}| is available by loading the \pkg{pifont}\footcite{pifont} or another package that can be used for such symbols such as \pkg{bbding} or create your own using a suitable font such as \pkg{symbola}
or \pkg{fontawesome}. 

\begin{texexample}{Changing the label symbols}{ex:symbols}
\renewcommand{\labelitemiii}{\ding{51}}
\begin{itemize}
\item This is the first level of the list.
  \begin{itemize}
     \item This is the second level of the list.
     \begin{itemize}
     \item This is the third level of the list.
     \end{itemize}
  \end{itemize}
\item And back to the first level.
\end{itemize}

\end{texexample}

We can use symbols, if we need them as in the following example:

\begin{texexample}{A yes and no list}{ex:yesno}
\newcommand{\Yess}{\ding{51}}
\newcommand{\Noo}{\ding{55}}

\begin{enumerate}
 \item[\Yess] Learn about lists.
 \item[\Noo] Learn about the \tex's output routine.
\end{enumerate}
\end{texexample}

As we can see from the example, we can use the argument of an |item| and it can change an enumerated list into an itemized list, provided we type the argument manually. The |enumi| will not be incremented. 



\section{Generalized lists}

Lists such as those in the three environments itemize, enumerate, and
description can be formed in a quite general way. The type of label and
its width, the depth of indentation, spacings for paragraphs and labels,
and so on, may be wholly or partially set by the user by means of the list
environment:

\bgroup
\trivlist\item 
     \cs{begin}\{list\}\marg{std\_label}\marg{list\_declarations} item list \cs{end}\{list\}
\endtrivlist
\egroup


Here item list consists of the text for the listed entries, each of which
begins with an |\item| command that generates the corresponding label.
The \(stnd\_label\) contains the definition of the label to be produced by the
\cs{item} command when the optional argument is missing.  

The first argument in the list environment defines the |standard label|, that
is, the label that is produced by the \cs{item} command when it appears
without an argument. In the case of an unchanging label, such as for the
itemize environment, this is simply the desired symbol. If this is to be a
mathematical symbol, it must be given as \$symbol name\$, enclosed in \$
signs. For example, to select a right arrow symbol ($\Rightarrow$) as the label, 
the \emph{std\_label} must be defined to be |\$\Rightarrow\$|.



\newcounter{steps}
\setcounter{steps}{0}

\begin{list}{\bfseries\upshape Step \arabic{steps}:}
{%
\usecounter{steps}
\setlength{\labelwidth}{2cm}\setlength{\leftmargin}{2.6cm}
\setlength{\labelsep}{0.5cm}\setlength{\rightmargin}{1cm}
\setlength{\parsep}{0.5ex plus0.2ex minus0.1ex}
\setlength{\itemsep}{0ex plus0.2ex minus0pt}\relax \slshape %
}
\item Melt the butter and dark chocolate
\item Prepare the egg and sugar mix
\item Cool the butter and dark chocolate
\item Set out the milk and white chocolate
\item Prepare the brownie tin

\item[]\ldots
    
\item Fold in the chocolate to the eggy mousse\ldots
\item Add the flour and cocoa\ldots
\item Get the tin in the oven.
\end{list}



\begin{texexample}[listing only]{Generalized Lists}{ex:genlists}
% create a new counter for the list
%\newcounter{steps}
%\setcounter{steps}{0}
Continuing with our recipe\ldots

\makeatletter
\def\usecounter#1{\@nmbrlisttrue\def\@listctr{#1}}


\begin{list}{\bfseries\upshape Step \arabic{steps}:}%
{ 
% this has to be on the first line 
\usecounter{steps}
\setlength{\itemsep}{0ex} 
\setlength{\labelwidth}{2cm}
\setlength{\leftmargin}{2.6cm}
\setlength{\labelsep}{0.5cm}
\setlength{\rightmargin}{1cm}
\setlength{\parsep}{0.5ex plus.2pt}
 }
\item Melt the butter and dark chocolate
\item Prepare the egg and sugar mix
\item Cool the butter and dark chocolate
\item Set out the milk and white chocolate
\item Prepare the brownie tin
      \ldots
\item Fold in the chocolate to the eggy mousse\ldots
\item Add the flour and cocoa\ldots
\item Get the tin in the oven.

% end the list
\end{list}
\makeatother


This brings us to the end of our cooking lessons.
\end{texexample}

This brings us to the next step. 

\makeatletter
\gdef\resume{\def\usecounter##1{\@nmbrlisttrue\def\@listctr{##1}}\relax}
\gdef\reset{\def\usecounter##1{\@nmbrlisttrue\def\@listctr{##1}\setcounter{##1}{0}\relax}}
\makeatother

\resume
\begin{list}{\bfseries\upshape A \arabic{steps}:}
{%
\usecounter{steps}
\setlength{\labelwidth}{2cm}\setlength{\leftmargin}{2.6cm}
\setlength{\labelsep}{0.5cm}\setlength{\rightmargin}{1cm}
\setlength{\parsep}{0.5ex plus0.2ex minus0.1ex}
\setlength{\itemsep}{0ex plus0.2ex minus0pt}\relax \slshape %
}
\item Melt the butter and dark chocolate
\item Prepare the egg and sugar mix
\item Cool the butter and dark chocolate
\item Set out the milk and white chocolate
\item Prepare the brownie tin
\end{list}

\reset
\begin{list}{\bfseries\upshape A \arabic{steps}:}
{%
\usecounter{steps}
\setlength{\labelwidth}{2cm}\setlength{\leftmargin}{2.6cm}
\setlength{\labelsep}{0.5cm}\setlength{\rightmargin}{1cm}
\setlength{\parsep}{0.5ex plus0.2ex minus0.1ex}
\setlength{\itemsep}{0ex plus0.2ex minus0pt}\relax \slshape %
}
\item Melt the butter and dark chocolate
\item Prepare the egg and sugar mix
\item Cool the butter and dark chocolate
\item Set out the milk and white chocolate
\item \lorem
\end{list}


Using either |newenvironment| we can

\emphasize{recipe}
\begin{texcode}{Creating a new named List}{ex:newlist}
\newenvironment{recipe}{\list{\bfseries\upshape Step \arabic{steps}:}
{%
\usecounter{steps}
\setlength{\labelwidth}{2cm}\setlength{\leftmargin}{2.6cm}
\setlength{\labelsep}{0.5cm}\setlength{\rightmargin}{1cm}
\setlength{\parsep}{0.5ex plus0.2ex minus0.1ex}
\setlength{\itemsep}{0ex plus0.2ex minus0pt}\relax \slshape %
}}
{\endlist}

\begin{recipe}
\item Have a nice afternoon\ldots
\end{recipe}
\end{texcode}

\newenvironment{recipe}{\list{\bfseries\upshape Step \arabic{steps}:}
{%
\usecounter{steps}
\setlength{\labelwidth}{2cm}\setlength{\leftmargin}{2.6cm}
\setlength{\labelsep}{0.5cm}\setlength{\rightmargin}{1cm}
\setlength{\parsep}{0.5ex plus0.2ex minus0.1ex}
\setlength{\itemsep}{0ex plus0.2ex minus0pt}\relax \slshape %
}}
{\endlist}

\begin{recipe}
\item Have a nice afternoon\ldots
\end{recipe}

% something gone terribly wrong, need to restore the settings
\makeatletter
%\@restorepar\@nobreakfalse\@nmbrlistfalse
\let\par\@@par
\makeatother


\subsection{trivlist}

The simplest of all lists is |\trivlist|. In simple terms the |trivlist| environment turns each item into a paragraph and thus it is easier to apply formatting information to a list of paragraphs or items if you want to think of them this way. As it carries common information to all lists, it is used to build up more complicated structures. A good use of trivlists is to simplify the writing of table heads and produce semantic table environments. They are also used to define teh spacing around the verbatim environments. A simple author use is shown in Example~\ref{ex:trivlists}.

\begin{texexample}{Trivlists}{ex:trivlists}
\newenvironment{name}
  {\trivlist\item
   \tabular{@{}ll@{}}}
  {\endtabular\endtrivlist}
  
% test it  
\begin{name}
   First    & Mary  \\
   Second   & Jones \\
   Nickname & --- \\
\end{name}
 

But whatever you call the comma, is it right or wrong? There’re fair arguments on both sides.  One might be concerned about limiting ambiguity. Alas, including the Oxford comma can lead to ambiguity, but omitting it can lead to ambiguity as well.  Consider (3) and (4):
\begin{trivlist}
\item[(3a)] I own pictures of my friends, Hugh Grant, and Dolly Parton.
\item[(3b)] I own pictures of my friends, Hugh Grant and Dolly Parton.
\item[] 
\item[(4a)] I am writing to my Congresswoman, Alia Shawkat, and Michael Cera.
\item[(4b)] I am writing to my Congresswoman, Alia Shawkat and Michael Cera.
\end{trivlist}
\end{texexample}
           


The general parameters affecting a general list is shown in the  diagram  below\footnote{Produced using the \texttt{layouts} package.}. LaTeX offers three general list structures, enumerate, itemize and description.
%\begin{figure}[hp]
%\listdiagram
%\caption{Layout of an \texttt{enumerate} list} \label{fig:lstenum}
%\end{figure}

\section{The list geometry}

You can draw a list diagram as shown below, using the function \docAuxCmd{drawlistdiagram} from
the \pkgname{xlayouts} package, which is bundled with the \pkgname{phd} package.


\begin{figure}[htbp]
\bgroup
\centering
\cxset{geometry units=mm}
\drawlistdiagram
\caption[List geoemetry]{\latexe list diagram.}
\egroup
\end{figure}


List in \latexe are shaped using |\parshape|. Sometimes as you change parameters things react unintuitevely. An easy wway to remember is that the parameter |\leftmargin| determines the first line of the indentation and |\linewidth| is the
|hsize-leftmargin-rightmargin|



What is important to notice here is that all the standard list parameters are left essentially unchanged. The only item that is affected is \refCom{makelabel}, which is redefined in \lstinline{description} label.

\section{Packages}


 Most journals develped their own lists and hard-wired them. Current packages are:
 \pkg{enumitem}, \pkg{enumerate}, \pkg{paralist}.



\paragraph{Enumerate} This package gives the \refEnv{enumerate} environment an optional argument which determines
 the style in which the counter is printed. An occurence of one of the tokens 
 \textbf{A a I i } or \textbf{1} produces the value of the counter printed with
 (respectively) \cmd{\Alph} \cmd{\alph} \cmd{\Roman} \cmd{\roman} or \cmd{\arabic}. 
 These letters may be surrounded by any string involving any other \tex expressions, 
 however the tokens \textbf{A a I i 1} must be inside a \{\} group if they are
 not to be taken as special.

\begin{texcode}{Example using the package enumerate}{ex:enum}
 \begin{enumerate}[EX i.]
   \item one one one one one one one
         one one one one\label{LA}
   \item two
      \begin{enumerate}[{example} a)]
        \item one of two one of two
          one of two\label{LB}
        \item two of two
       \end{enumerate}
   \item two of two
 \end{enumerate}
 

 \begin{enumerate}[{A}-1]
 \item one\label{LC}
 \item two
 \end{enumerate}
\end{texcode}

 This package minimally changes the original \latexe definitions. It is very convenient when you
 want now and then to change labels in a document. The central idea behind the |phd| group of
 packages and eponymous classes, is that there is a distinction between the author, template designer and 
 programmer. The packages at the level of the author always use a key value interface, that
 limits the involvement of the author and to a large extend the template designer to that of 
 setting a number of keys. 

\paragraph{Babel} The babel package\footcite{babel} will redefine enumerate for a number of languages such as french and greek. It is best in these cases, if you still need to use a different list to create a new list with one of the other packages such as enumitem or using the phd-lists package.
\bgroup
\selectlanguage{french}
\extrasfrench

\lorem
\begin{frenchenumerate}
\item \lorem
      \lorem
        \begin{frenchenumerate}
         \item Something
         \item \lorem
               
               \lorem
        \end{frenchenumerate}
\item \lorem
\end{frenchenumerate}

\lorem


\egroup
\paragraph{phd-lists} The package, which is described in the next chapter uses a key value approach in setting new lists
and adjusting their parameters.

\begin{texexample}{Example using phd-lists}{ex:phd-lists}
\cxset{enumerate numberingi   = Alpha,
       enumerate numberingii  = alpha,
       enumerate numberingiii = Roman,}

 
 \begin{enumerate}
   \item This is the top level. \lorem
      \begin{enumerate}
        \item This is the second level. \lorem
          \begin{enumerate}
            \item This is the third level. \lorem
          \end{enumerate}. 
      \end{enumerate}
 \end{enumerate}
\end{texexample}

\vfill
\endinput

\section{Creating new description like environments}

The macro \docAuxCmd*{NewDescriptionEnvironment} can be used to redefine new description like environments, using a key value interface.

%\begin{texexample}{Define a new description list environment}{ex:newdesclist}
%% define the orange description environment
%\NewDescriptionEnvironment[description centered]{orangedescription}
%
%% Sample
%The \texttt{orangedescription} environment in action.
%
%\begin{orangedescription}
%
% \item[One] \lorem
% \item[Two] \lorem
% \item[Three] \lorem
%
%\end{orangedescription}
%
%\lorem
%
%\makeatother
%\end{texexample}



\section{Example: redefining a description list}
We will now develop a description environment, that can be useful for the documentation of packages to describe options. We will use a description list as the basis of the environment. We define the following key values.
|\itemindent-\leftmargin|

\section{Enumerated lists}


\begin{enumerate}
\item one
\item two
\item three
\end{enumerate}

Enumerated (numbered) list environments are characterized by numbering. They use a variety of fields and counters as shown in table.

\subsection{Vertical skips}

By default LaTeX adds vertical skips, as shown in figure 1. The definition of these skips is influenced by the font size and are defined in the \texttt{bk10.clo} files, hence hard to find and change. Each level of the list has its own definition as \lstinline{\@listi}.

\bigskip
\tcbset{width=\linewidth,arc=1mm,before=\bigskip,left=8mm}

\begin{teXXX}
\def\@listi{\leftmargin\leftmargini
            \parsep 40pt plus20pt minus0pt
            \topsep 80pt plus20pt minus40pt
            \itemsep40pt plus20pt minus0pt}
\let\@listI\@listi
\@listi

\def\@listii {\leftmargin\leftmarginii
              \labelwidth\leftmarginii
              \advance\labelwidth-\labelsep
              \topsep    40pt plus20pt minus0pt
              \parsep    20pt plus0pt  minus0pt
              \itemsep   \parsep}
\def\@listiii{\leftmargin\leftmarginiii
              \labelwidth\leftmarginiii
              \advance\labelwidth-\labelsep
              \topsep    20pt plus0ptminus0pt
              \parsep    1pt
              \partopsep 0pt plus\z@ minus0pt
              \itemsep   \topsep}
\def\@listiv {\leftmargin\leftmarginiv
              \labelwidth\leftmarginiv
              \advance\labelwidth-\labelsep}
\def\@listv  {\leftmargin\leftmarginv
              \labelwidth\leftmarginv
              \advance\labelwidth-\labelsep}
\def\@listvi {\leftmargin\leftmarginvi
              \labelwidth\leftmarginvi
              \advance\labelwidth-\labelsep}
\end{teXXX}


\begin{texexample}{Compact Styles}{ex:compact2}
\makeatletter
\cxset{enumerate compact/.style={%
  enumerate numberingi=alpha,
  enumerate numberingii=roman,
  enumerate numberingiii=alpha,
  enumerate numberingiv=roman,
  enumerate labeli punctuation=,
  enumerate label left=(,
  enumerate label right=),
  enumerate leftmargini=2.2em,
  enumerate leftmarginii=2.1em,
  enumerate leftmarginiii=1.5em,
  enumerate leftmarginiv=2em,
  listi topsep=8pt plus2pt minus0pt,
  listi itemsep=0pt plus2pt minus0pt,
  listi parsep=0pt plus2pt minus0pt,
  listi parindent=0pt,
  listii parindent=1em,
  listiii parindent=1em,
  listii topsep=0pt plus2pt minus0pt,
  listii itemsep=0pt plus2pt minus0pt,
  listii parsep=0pt plus2pt minus0pt,
  listiii topsep=0pt plus2pt minus0pt,
  listiii itemsep=0pt plus2pt minus0pt,
  listiv parindent=0pt,
  listiv parsep=0pt plus2pt minus0pt,
  listiv parsep=0pt plus2pt minus0pt,
  listiv topsep=0pt plus2pt minus0pt,
  listiv itemsep=0pt plus2pt minus0pt,
  listiv parsep=0pt plus2pt minus0pt,
}}
\makeatother


\begin{enumerate}
\item Does this project actually merit the use of the Minor Works Form or Intermediate Form instead of their `grown up' relatives?
\item Do the number of PC or prime cost items mean that it would be more desirable to use a re-measurable form?
\item Is this a contract which merits the production of full scale bills
of quantities or is something more standardised going to suffice?
\end{enumerate}
\end{texexample}



As you will observe the numbering in the above example has been enclosed in round brackets, using:


\begin{texcode}{Bracketting a numeral}{ex:brackets}
  enumerate label left=(,
  enumerate label right=),
\end{texcode}


The next example is from the \textit{LaTeX Companion}. In example~\ref{ex:companion}, the first-level list elements are decorated with the section sign (\S) as a prefix and a period as a suffix (omitted in references). We will
define this as a style named \textit{paragraphsymbol} for the lack of any better name. This style can sometimes be found in legal texts.

\begin{texexample}{Paragraph symbols in enumerate}{ex:companion}
\cxset{paragraphsymbol/.style={%
  enumerate numberingi=arabic,
  enumerate labeli punctuation=.,
  enumerate label left=\S,
  enumerate label right=,
}}

\setenumerate{paragraphsymbol}
\begin{enumerate}
\item \lorem
\item \lorem
\item \lorem
\end{enumerate}
\end{texexample}


%\section{Creating enumerated environments}
%
%New enumerated environments cab be created by using the macro \lstinline{\newenumeratedenvironment}. Keys are set as either styles or individually.
%
%%% verbatim from latex
%
%
%\begin{texexample}{An enumerated list factory}{ex:listfactory}
%
%
%\makeatletter
%\newenvironment#1#2{
%#2\enumerate}{\endenumerate}
%\makeatother
%
%\newenumeratedenvironment{paragraphsymbol}{
%  enumerate numberingi=alpha,
%  enumerate numberingii=roman,
%  enumerate numberingiii=alpha,
%  enumerate numberingiv=roman,
%  enumerate labeli punctuation=,
%  enumerate label left=(,
%  enumerate label right=),
%  enumerate leftmargini=2.2em,
%  enumerate leftmarginii=2.1em,
%  enumerate leftmarginiii=1.5em,
%  enumerate leftmarginiv=2em,
%  listi topsep=8pt plus2pt minus0pt,
%  listi itemsep=0pt plus2pt minus0pt,
%  listi parsep=0pt plus2pt minus0pt,
%  listi parindent=0pt,
%  listii parindent=1em,
%  listiii parindent=1em,
%  listii topsep=0pt plus2pt minus0pt,
%  listii itemsep=0pt plus2pt minus0pt,
%  listii parsep=0pt plus2pt minus0pt,
%  listiii topsep=0pt plus2pt minus0pt,
%  listiii itemsep=0pt plus2pt minus0pt,
%  listiv parindent=0pt,
%  listiv parsep=0pt plus2pt minus0pt,
%  listiv parsep=0pt plus2pt minus0pt,
%  listiv topsep=0pt plus2pt minus0pt,
%  listiv itemsep=0pt plus2pt minus0pt,
%  listiv parsep=0pt plus2pt minus0pt,
%  enumerate numberingi=alpha,
%  enumerate labeli punctuation=.,
%  enumerate label left={\P},
%  enumerate label right=,
%}
%
%
%\begin{paragraphsymbol}
%\item \lorem
%\item \lorem
%\item \lorem
%\end{paragraphsymbol}
%
%\end{texexample}

\section{The description list environment}

You can use the description list environment as you would normally use it with \latexe.

\begin{docEnv} {description} {}
\end{docEnv}

However, a number of settings are available to modify the styling of the environment. These 
include settings for fonts and color, as well as spacing and margins.

\begin{docKey} {description label font-face} { = \meta{font shape} } {initial, default=inherit}
\end{docKey}

\begin{docKey} {description label font-family} { = \meta{font shape} } {initial, default=inherit}
\end{docKey}

\begin{docKey} {description label font-size} { = \meta{font size} } {initial, default=inherit}
\end{docKey}

\begin{docKey} {description label font-weight} { = \meta{font weight} } {initial, default=inherit}
\end{docKey}

\begin{docKey} {description label font-shape} { = \meta{font shape} } {initial, default=inherit}
\end{docKey}

\begin{docKey} {description label color} { = \meta{color name} } {initial, default=thedescriptionlabelcolor}
  Setts the description label color
\end{docKey}

\begin{docKey} {description label sep} { = \meta{dim} } {initial, default = 1em}
\end{docKey}

\begin{docKey} {description label width} { = \meta{dim} } {initial, default = 1em}
\end{docKey}

\begin{docKey} {description margin left} { = \meta{dim} } {initial, default = 0em}
\end{docKey}

\begin{docKey} {description margin right} { = \meta{dim} } {initial, default = 0em}
\end{docKey}

\begin{docKey} {description item indent} { = \meta{dim} } {initial, default = 0em}
\end{docKey}

Unlike the enumerate and itemize environment, the description list environment is defined in the book class.
The environment is defined as:

\refCom{descriptionlabel} sets the typesetting of the description label.
\section{Itemized lists}

The itemized \LaTeX\ lists are similar to those for the enumerated lists. However they are somehow simpler as there is no need for counters.

\bigskip
\begin{tcolorbox}[width=\linewidth,arc=2mm,title=Default \LaTeX\ parameters for itemized lists]
\begin{lstlisting}
\newcommand\labelitemi{\textbullet}
\newcommand\labelitemii{\normalfont\bfseries \textendash}
\newcommand\labelitemiii{\textasteriskcentered}
\newcommand\labelitemiv{\textperiodcentered}
\end{lstlisting}
\end{tcolorbox}




\cxset{red/.style={
 labelitemi={{\color{green}\ding{'64}}},
 labelitemii=\color{red}\textendash,
 labelitemiii=\textasteriskcentered,
 labelitemiv=\textperiodcentered,
}}

Now that we have managed to abstract the itemized environment we can generate a new environment factory.

\makeatletter
\def\newitemizedenvironment#1#2{
\@itemdepth=0
\expandafter\def\csname#1\endcsname{%
 \cxset{#2}%
 \ifnum \@itemdepth >\thr@@\@toodeep\else
 \advance\@itemdepth\@ne
 \edef\@itemitem{labelitem\romannumeral\the\@itemdepth}%
 \expandafter
 \list
 \csname\@itemitem\endcsname
 {\def\makelabel####1{\hss\llap{####1}}}%
 \fi}
 \expandafter\let\csname end#1\endcsname=\endlist
}
\makeatother

%\newitemizedenvironment{reditemize}{black}
%
%
%\begin{reditemize}
%\item Test.
%   \begin{reditemize}
%    \item test.
%   \end{reditemize}
%\end{reditemize}
%
%\begin{itemize}
%\item Level i
%      \begin{itemize}
%       \item Level ii
%          \begin{itemize}
%            \item Level iii
%              \begin{itemize}
%                \item Level iv. \lipsum*[1]
%              \end{itemize}
%          \end{itemize}
%      \end{itemize}
%\end{itemize}


\section{Itemized lists with ding symbols}

So far we have used both standard symbols as well as those provided by the pifont that offers numerous,
dingbang symbols. The pifont package also offers environments to do that more easily.


\begin{texcode}{dinglist}{ex:dinglists}
\begin{dinglist}{"E4}
\item The first item. \item The second
item in the list.
\end{dinglist}

\end{texcode}

This will give us:

\begin{dinglist}{"E4}
\item The first item. \item The second
item in the list.
\end{dinglist}



%\begin{dingautolist}{'30}
%\item The first item in the list.\label{lst:a}
%\item The second item in the list.\label{lst:b}
%\item The third item in the list.\label{lst:c}
%\item The fourth item in the list.\label{lst:d}
%\end{dingautolist}
%
%\newenvironment{steps}{\dingautolist{'30}}{\enddingautolist}
%
%\begin{steps}
%\item The first item in the list.\label{lst:a}
%\item The second item in the list.\label{lst:b}
%\item The third item in the list.\label{lst:c}
%\item The fourth item in the list.\label{lst:d}
%\end{steps}


\endinput

\def\start@SFBbox{\@next\@currbox\@freelist{}{}%
 \global\setbox\@currbox
 \vbox\bgroup
  \hsize \textwidth
  \@parboxrestore
}
\def\finish@SFBbox{\par\vskip -\dbltextfloatsep
  \egroup
  \global\count\@currbox\tw@
  \global\@dbltopnum\@ne
  \global\@dbltoproom\maxdimen\@addtodblcol
  \global\vsize\@colht
  \global\@colroom\@colht
}

\newif\ifSFB@keywords
\def\keywords{\if@twocolumn
  \start@SFBbox\@keywords
 \else
  \@keywords
 \fi
}
\def\@keywords{\list{}{%
    \labelwidth\z@ \labelsep\z@
    \leftmargin 11pc\rightmargin\z@  % was 11pc\right....
    \parsep 0pt plus 1pt}\item[]\reset@font\large{\bf Key words: }%
}
\def\endkeywords{\if@twocolumn
  \endlist\addvspace{37pt plus 0.5\baselineskip}\finish@SFBbox
 \else
  \endlist
\fi










   \cxset{style13,
         chapter toc = true}
\chapter{Tables}


\parindent1em

Bringhurst \cite{Bringhurst2005} while describing tables admonishes us that we should  `edit tables with the same attention given to text, and set them as text to be read'.

Tables are difficult to typeset and notoriously time-consuming. Newcomers to \latex\
If the table is not planned properly, like text they can go awry when approached on a
purely technical basis.

In general, tables are quite easy and straightforward in \latex 
Part of the price paid for that ease is that very complicated
tables may take excessive ingenuity and thought. There
are several issues here, but it is often a good idea to take a less
mechanistic approach to tables. What is the function of a table?
Why are we bothering to create one in the first place?
Sadly, many tables which appear in reports and books would
be better left out. They are included to lend a (spurious) air
of legitimacy and erudition. They ought to be there as part
of the development of the theme or argument. Sometimes
tables are an archival mechanism, so that others have all the
information needed to confirm or refute the arguments contained.
But too often the tables are merely included to confirm
that a great deal of worthy work was done, work which
should receive some recognition. Most tables are probably
unnecessary. Having said that, they will not go away.

The other point to make about tables is that they are very
strongly visual. In a broad sense, \latex tries to separate the
content of a document from its presentation, but with tables
this is probably problematic in general, and to look at the
specifics, choosing how columns are handled (or that the material
is presented as a column rather than a row), is surely
specifying many of the presentational aspects too. The same
content may be presented several different ways in a table.
Some ways will be confusing and difficult to understand. We
have only to consider the example of a railway timetable to
see just how difficult an apparently simple problem may be
to solve. We can therefore expect that a table may require
adjustment before it ‘works’.

All text should be horizontal, or in rare cases oblique. Setting the column heads vertically 
as a space-saving measure is quite feasble if the text is in Japanese or Chinese, 
but not if it is written in the Latin alphabet (see Table \ref{tbl:rotated}).

Provide a minimum amount of furniture (rules, boxes, dots and other guiderails for travelling through typographic space) and a maximum amount of information. There are two other rules, sometimes referred to as Fear's rules - after the name of the author of the \pkg{booktabs} package, the first one is that you should never ever use any vertical rules and the second one is to never use double rules\cite{booktabs}.


 Fear gives some more good advice,   units belong in the column heading and not the body of the table. Precede a decimal point by a digit; thus 0.1 not just .1, and do not use ‘ditto’ signs or any other such convention to repeat a previous value. In many circumstances a blank will serve just as well. If it won’t,
then repeat the value. You will need to search hard for cases where these rules need to ne broken, one such rule perhaps being the accounting tables of the Auditor General. 

A rule located at the edge of a table separating the first or final column from the adjacent empty space, ordinarily serves no function.


\section{The tabular environment}

Keeping these simple rules in mind before you delve into the code, the technicalities and the intricacies of the code. The \latex workhorse for typestting table sis the |tabular| environment, which we will illustrate with a short example:


\clearpage


To typesat a  table as shown on the right, we first need to enclose the table values within the tabular environment, columns are separated by an ampersand |&| and the end of the rows terminated with |\\|.
In this example we also introduce some rules from the |booktabs| package, as it is the right way to provide rules and we might as well introduce them early. We denote the justification of the columns, by using the directives \textbf{rl}, the \textbf{r}, denotes that the first column must be justified right, whereas the \textbf{l} instructs \latex to typeset the column, left justified. Most of the tabular formatting takes place here.


\begin{tabular}{rl}
\toprule
 guru 			& measure \\
\midrule
Morison         		& 10--12 words \\
Karl Treebus  		& 10--12 words or\\
                      		& 60--70 letters \\
John Miles      		& 60--65 characters \\
Leslie Lamport 		& less than 76 characters \\
Linotype            	& 7--10 words or\\
& 50--65 letters\\
\bottomrule   %[1.1pt]
\end{tabular}


\topline
\emphasis{begin,end,tabular,rl,table,htp,htbp}
\begin{teX}
\begin{tabular}{rl}
\toprule
  guru                & measure \\
\midrule
  Morison             & 10-12 words \\
  Karl Treebus        & 10-12 words or\\
                      & 60-70 letters \\
  John Miles          & 60-65 characters \\
  Leslie Lamport      & less than 76 characters \\
  Linotype            & 7-10 words or\\
                      & 50-65 letters\\
\bottomrule
\end{tabular}
\end{teX}
\medskip

\bottomline






\textbf{Floating environment.}\quad While the example we just described, is fine typographically, it is inadequate for most publications, as it does not have a caption and was inserted exactly where we typed it. Floating environments are described in more detail later on, but all we had to float the table is to enclose it within a |table| environment.

\begin{teXXX}
\begin{table}[htp]
\begin{tabular}{rl}
.
.
.
\end{tabular}
\caption{This is the caption...}
\end{table}
\end{teXXX}

\begin{tabular}{rl}
\toprule
  guru                  & measure \\
\midrule
  Morison             & 10-12 words \\
  Karl Treebus      & 10-12 words or\\
                          & 60-70 letters \\
  John Miles         & 60-65 characters \\
  Leslie Lamport   & less than 76 characters \\
  Linotype            & 7-10 words or\\
                         & 50-65 letters\\
\bottomrule
\end{tabular}
\captionof{table}{Example table}
\medskip
Notice that the |table| environment also uses directives, \textbf{htp}, meaning you can place the table \textbf{h}ere or at the \textbf{t}op of a page or on a floats \textbf{p}age.

\clearpage


\texttt{\bf tabular*}

Besides the normal tabular command sequence, there is also a 
starred version of the tabular:

\begin{tabular*}{\linewidth}{|r|c|c|}
\hline
1971 & 41--54 & \$2.60 \\
\hline
\end{tabular*}
\bigskip

This looks particularly ugly. Later we'll see how to manipulate
this more elegantly. At the moment, it perhaps best
avoided. The use of \cmd{linewidth} requires some elaboration.
In a multicolumn environment or class, it takes the value of
the text within a column. In ‘normal’, single column setting,
it will also take a value corresponding to the current text
width. However, any valid expression of a dimension, as defined
in Chapter 3, is suitable.



\section{More tabular esoterica}

\textbf{Aligning at a decimal point.}\quad One recurrent theme with tables is the desire to align numerical
information on the decimal point. This is quite easily
achieved by use of one of the \textit{column specifiers}, the \textbf{@} specifier.
This allows text to be inserted at a given position in a
column (in every column), and suppresses the space which
is normally added between adjacent column entries. In this
context the suppression of space is `a good thing'. We can
specify the column formats like this:

\begin{teXXX}
\begin{tabular}{r@{.}lr@{.}l}
\end{teXXX}



to create two pairs of columns (that is, four columns in all).
Each pair is really a column aligned on a decimal point. We
therefore write the contents of the table like:

\begin{teX}
\begin{tabular}{r@{.}lr@{.}l}
\toprule
12  &3         & 156 &80 \\
0    &12345  &       &90 \\
\bottomrule
\end{tabular}
\end{teX}

which produces:
\medskip

\begin{tabular}{r@{.}lr@{.}l}
\toprule
 12  &3      & 156 &80 \\
     0   &12345  &     &90 \\
\bottomrule
\end{tabular}

\bigskip
\noindent in order to make |12.3| and |0.12345| align correctly. Note that
we do not use the decimal point at all. We give the alignment
position, and we have already specified the extra text
to be added, the decimal point. 

The penalty we have paid
is to introduce extra columns. This means that we have to
be careful in specifying the column headings (or stubs), since
they should span two columns\footnote{This is a source of endless edits and mistakes in documents.}.


\textbf{Multicolumn cells.}\quad In many tables there is a need of merging two or more cells together. This can be achieved by the using the |\multicolumn| command. 

The \cmd{multicolumn} instruction makes this an easy problem
to solve: a full table might therefore look like \ref{tbl:aligned}:

\begin{flushleft}
\begin{tabular}{rr@{.}lr@{ = }lr@{.}l}
\toprule
$\lambda_{ij}$&\multicolumn{2}{c}{$L_{R-I}$}&
\multicolumn{2}{c}{$DAR$}&
\multicolumn{2}{c}{$L_{R-P}$}\\
\midrule
70&4&60&6&80&5&10\\
80&10&70&12&10&11&20\\
\bottomrule
\end{tabular}
\end{flushleft}
\captionof{table}{A simple table with the numbers aligned at the decimal point}
\label{tbl:aligned}


\begin{teX}
\begin{tabular}{r r@{.} r@{.} r@{.} }
\hline
$\lambda_{ij}$&\multicolumn{2}{c}{$L_{R-I}$}&
\multicolumn{2}{c}{$DAR$}&
\multicolumn{2}{c}{$L_{R-P}$}\\
\hline
70&4&60&6&80&5&10\\
80&10&70&12&10&11&20\\
\hline
\end{tabular}
\end{teX}
\medskip



There is no need to enclose your text in grid prisons, in many instances simply intoducing a 
\cmd{toprule}, \cmd{midrule} and a \cmd{bottomrule} will suffice.
\medskip



\section{The \texttt{table} environment}

The tabular environment can typeset tabular information neatly. The \texttt{table} environment creates \textit{floating} tables. The table placement is controlled with an optional argument. Inside the table we can use the |\caption| to typeset a caption for the table. The starred version of the table produces an unumbered table, which is not listed in the list of tables.
\section{Long Tables}

Long tables offer problems if you are using \latex\. The package 
\pkg{longtable} offers macros to help with long tables. The table below shows such an example.

To use the |longtable| package, just include it normally as any other package. In lieu of typing
|\begin{table} ... \end{table}|, you just type begin{longtable} ... {longtable}. The package has numerous other useful commands, including command that break the page across pages horizontally. I would not recommend you do this. Such tables are virtually unreadable.


\section{tabular}

The tabular environment can also align two tables side by side (See \ref{sidebyside}).


\begin{tabular}[t]{|c|}
\hline A \\ \hline
\end{tabular}
\begin{tabular}[t]{|c|}
\hline A \\ B \\ \hline
\end{tabular}
\begin{tabular}[t]{|c|}
\hline A \\ B \\ C\\\hline
\end{tabular}
\captionof{table}{Tables side by side}
\label{sidebyside}


There is also an option argument (in square braces), which
is another ‘positional’ argument, indicating where the table is
in a vertical sense. By default a table is aligned horizontally
along its centre, but you can align it on its top row, or its bottom
row (using t and b). Why should you want to do this?
Usually when you have several tables side by side. For the
sake of an example,

\begin{teX}
\begin{tabular}[t]{|c|}
\hline A \\ \hline
\end{tabular}
\begin{tabular}[t]{|c|}
\hline A \\ B \\ \hline
\end{tabular}
\end{teX}




\subsection{Restraining the width of some of the columns}

The \texttt{tabularx} takes the same arguments
as \texttt{tabular}, but modifes the widths of certain columns, rather than
the inter column space, to set a table with the requested total width. The
columns that may stretch are marked with the token \texttt{X} in the preamble
argument.



This package requires the \pkg{array} package.


\begin{teX}
\begin{tabularx}{200pt}{|c|X|c|}
   one & \lipsum[1] &three \\
\end{tabularx}
\end{teX}


{\scriptsize

\begin{tabular}{|c|p{3.0cm}|c|}
one & \lipsum[1] &three\\
\end{tabular}

}



The arguments of |tabularx| are essentially the same as those of the standard
|tabular*| environment. However rather than adding space between the columns
to achieve the desired width, it adjusts the widths of some of the columns. The
columns which are affected by the |tabularx| environment should be denoted with
the letter X in the preamble argument. The X column specification will be converted
to p{$<some value>$} once the correct column width has been calculated.

Whatever you can achieve with |tabularx|, you can achieve with |tabular| as well. The package uses
the |p{}| to calculate the widths. Personally, I prefer to stay with |tabular| and only use it for
the occassional difficult table.


\section{Using maths in tabular environments}


Sometimes the material to be displayed especially if it is of a mathematical nature can best be displayed using one of the maths environments available. Consider \fref{fig:align}. This is from a question and answers assignment. We need to align all the numbers including the numbers 9,8 and 7 in the last three rows.
To do this we use the following code:
\medskip
\emphasis{Z,begin,end,align,*}
\begin{teXXX}
\begin{align*}
  1 +  12 &= 13\\
  2 +  11 &= 13\\
  3 +  10 &= 13\\
  4 + \Z9 &= 13\\
  5 + \Z8 &= 13\\
  6 + \Z7 &= 13
\end{align*}
\end{teXXX}
\medskip

Notice the use of |\Z|, which we will define below. This is used as a |\phantom|
command to push the numeral to the right. The use of the |align*| environment, is in many respects much more easier than use of the normal tabular environment.

    

\begin{align*}
1 +  12 &= 13\\
2 +  11 &= 13\\
3 +  10 &= 13\\
4 + \Z9 &= 13\\
5 + \Z8 &= 13\\
6 + 7 + 12&= 25
\end{align*}
\captionof{table}{Equations displayed using the \textbackslash align* environment.}
\label{fig:align}
\medskip


\section{Using the array package.} Both the |tabular| as well as the |array| environments can display tabular data. The latter should be used for tables that are predominantly requiring mathmode.

Another way to display tabular environments is to actually build them up from simpler
parts and use the |array| environment.  Table \tref{tbl:aliquot}  from \textit{Short Cuts in Figures} by A. Frederick Collins, shows such an example.

Before we get with the typesetting of the table we can need to define some helper macros. As we want to display the headings more all less in equal boxes, we need to grab the length of the longest word, in this case we choose the word \texttt{equivalent},
\emphasis{\settowidth,\TmpLen}
\begin{teXXX}
\settowidth{\TmpLen}{Equivalent}
\end{teXXX}

\newlength\TmpLen
\settowidth{\TmpLen}{Equivalent}

Once the width of the column heading is known we can then typeset them in a |parbox|:

\emphasis{\parbox,TmpLen,centering}
\begin{teXXX}
\parbox[c]{\TmpLen}{\centering Aliquot Part}
\end{teXXX}
which will produce: 
\medskip

\colorbox{gray!10}{\parbox[c]{\TmpLen}{\centering Aliquot Part}\hspace{0.5cm}
\parbox[c]{\TmpLen}{\smallskip\centering Equivalent Parts\index{Equivalent parts}}
\hspace{0.5cm}\parbox[c]{\TmpLen}{\centering Whole Number}
}

There are two more sttings that we need to discuss, before we write the full code for the table:

\begin{teXXX}
\renewcommand\arraystretch{1.2}
\setlength\arraycolsep{0.5em}
\end{teXXX}

The final code for the tabel is then:

\begin{teX}
\begin{array}{ccccc}
\parbox[c]{\TmpLen}{...}
\parbox[c]{\TmpLen}{...} 
\parbox[c]{\TmpLen}{\ldots}\\
\Z2 & \text{is} & \frac{1}{50} & \text{of} & 100\\
.
.
.
\end{array}
\end{teX}

There are many more settings and ways to generalize the code, however for the time being this demonstrates how to use the |array| environment to produce nice looking tables.


\renewcommand\arraystretch{1.2}
\setlength\arraycolsep{0.5em}
\[
\begin{array}{ccccc}
\hline
\parbox[c]{\TmpLen}{\centering Aliquot Part}
&
&\parbox[c]{\TmpLen}{\smallskip\centering Equivalent Parts\index{Equivalent parts}\smallskip} 
&
&\parbox[c]{\TmpLen}{\centering Whole Number}\\
%
\hline
\Z2 & \text{is} & \frac{1}{50} & \text{of} & 100\\
\Z4 & \text{is} & \frac{1}{25} & \text{of} & 100\\
\Z5 & \text{is} & \frac{1}{20} & \text{of} & 100\\
 10 & \text{is} & \frac{1}{10} & \text{of} & 100\\
 20 & \text{is} & \frac{1}{5}  & \text{of} & 100\\
 25 & \text{is} & \frac{1}{4}  & \text{of} & 100\\
 50 & \text{is} & \frac{1}{2}  & \text{of} & 100\\
\hline
\end{array}
\]

\captionof{table}{Example of using an array environment to build up tables}
\label{tbl:aliquot}


\textbf{\Large USING LEADERS IN TABLES}
\setlength{\columnsep}{2em}


\parindent1em
Before we call on this code we need to build some new utility commands 

\emphasis{phantom,TmpLen}
\begin{teXXX}
\newcommand\Z{\phantom{0}}
\newcommand\ZZ{\phantom{00}}
\newcommand\ZZZ{\phantom{000}}
\newcommand\ZZZZ{\phantom{0000}}
\newcommand\E{\mathrel{\phantom{=}}}
\end{teXXX}


The |\Z| commands are used to add some phantom space.\cmd{phantom} Any phantom object will occupy exactly the same space as if it were normally there, except it won't. A space will be filled with exactly the same dimensions as the object would normally occupy, so the command |\ZZZZ| will occupy exactly the space of four zeros.


Using Leaders is another technique that you can use to improve the readability of text in tables,although a bit dated. They guide the reader's eyes in cases where the text is a bit spaced out from the columns.



\begin{center}
\normalfont\normalsize\centering Table XIII, Comparison of Weights

\index{Comparison of English weights}
\index{Weights, comparison of English}
\begin{tabular}{p{9em}l@{\qquad}l@{\qquad}c}
\toprule
\multicolumn{1}{c}{\textit{Kind}}&\multicolumn{1}{c@{\qquad}}{\textit{Pound}}
 &\multicolumn{1}{c@{\qquad}}{\textit{Ounce}}&\multicolumn{1}{c}{\textit{Grain}}\\
\midrule
Avoirdupois\dotfill   &  7000 gr. & $437\frac{1}{2}$           gr. & 1\\
Apothecaries'\dotfill &  5760 gr. & 480\phantom{$\frac{1}{2}$} gr. & 1\\
Troy\dotfill          &  5760 gr. & 480\phantom{$\frac{1}{2}$} gr. & 1\\
\bottomrule
\end{tabular}
\end{center}

\medskip


\emphasis{begin,end,tabular,\dotfill}
\begin{teXXX}
\begin{tabular}{p{9em}l@{\qquad}l@{\qquad}c}
    \multicolumn{1}{c}{\textit{Kind}}
   &\multicolumn{1}{c@{\qquad}}{\textit{Pound}}
   &\multicolumn{1}{c@{\qquad}}{\textit{Ounce}}
   &\multicolumn{1}{c}{\textit{Grain}}\\
Avoirdupois\dotfill   &  7000 gr. & $437\frac{1}{2}$           gr. & 1\\
Apothecaries\dotfill  &  5760 gr. & 480\phantom{$\frac{1}{2}$} gr. & 1\\
Troy\dotfill          &  5760 gr. & 480\phantom{$\frac{1}{2}$} gr. & 1
\end{tabular}
\end{teXXX}
\medskip


The only new command here is the \cmd{dotfill}, used to provide the leaders after the text in the {\it boxhead}.
You can redefine dotfill for better semantics on longer tables as \cmd{DotRow}. This is easier to do by
calculation using the \cmd{linewidth}.


\begin{teXXX}
\newlength{\TmpLen}
\newcommand{\DotRow}[2]{%
  \settowidth{\TmpLen}{#2}%
  \parbox[c]{\linewidth-\TmpLen}{#1\dotfill}#2\break%
}
\end{teXXX}

This is used as:

\begin{teX}
\DotRow{\qquad in feet}{$20,926,062$.}
\end{teX}



\begin{comment}

\begingroup
\centering\small
\fboxsep1em
\fbox{%
\begin{minipage}{0.8\linewidth}

\index{Geographical constants}%
\index{Clarke, A. R.}%
  \footnotetext{Dimensions of the earth are based upon the Clarke spheroid of 1866.}

\index{Dimensions of earth}%
\index{Earth's dimensions}%
\noindent Equatorial semi-axis: \\
\DotRow{\qquad in feet}{$20,926,062$.}
\DotRow{\qquad in meters}{$6,378,206.4$}
\DotRow{\qquad in miles}{$3,963.307$}

\medskip%[**TN: to aid pagination]
\index{Diameter of earth}%
\index{Polar diameter of earth}%
\noindent Polar semi-axis: \\
\DotRow{\qquad in feet}{$20,855,121$.}
\DotRow{\qquad in meters}{$6,356,583.8$}
\DotRow{\qquad in miles}{$3,949.871$}
\DotRow{Oblateness of earth}{$1 294.9784$}
\DotRow{Circumference of equator (in miles)}{$24,901.96$}
\index{Circumference of earth}%
\DotRow{Circumference through poles (in miles)}{$24,859.76$}
\DotRow{Area of earth's surface, square miles}{$196,971,984$.}
\index{Area of earth's surface}%
\DotRow{Volume of earth, cubic miles}{$259,944,035,515$.}
\index{Volume of earth}%
\DotRow{Mean density (Harkness)}{$5.576$}
\index{Harkness, William}%
\index{Density of earth}%
\DotRow{Surface density (Harkness)}{$2.56$}
\DotRow{Obliquity of ecliptic}{$23  27' 4.98$~s.}
\DotRow{Sidereal year}{$365$~d.\ $6$~h.\ $9$~m.\ $8.97$~s.\ or $365.25636$~d.}
\index{Sidereal, clock!year}%
\index{Year}%
\DotRow{Tropical year}{$365$~d.\ $5$~h.\ $48$~m.\ $45.51$~s.\ or $365.24219$~d.}
\DotRow{Sidereal day}{$23$~h.\ $56$~m.\ $4.09$~s.\ of mean solar time.}
\DotRow{Distance of earth to sun, mean (in miles)}{$92,800,000$.}
\DotRow{Distance of earth to moon, mean (in miles)}{$238,840$.}
\index{Distances, of planets}%
\end{minipage}
}

\endgroup
\clearpage

\medskip

One interesting sideline mixing some of the matters discussed in earlier chapters is Table. This table is interesting in that it is not easy to be reproduced without a little bit of tinkering. The table shown in table \ref{tbl:romannumerals} uses some characters that are not easily found in font sets. We need to create these on the fly.


  \centerline{\includegraphics[width=0.8\linewidth]{reverseC}}

  \captionof{figure}{This inscription found in Rome reads 1583. The use of the backwards C is very clear.} 
  \label{fig:romannumerals}



\newcommand\nbrotC{\rotatebox[origin=c]{180}{C}\xspace}

 The apostrophus (or apostrophic C, or reversed \nbrotC ) was simulated in the original text
 by rotating a capital C. The following command was used.


\begin{teXXX}
% The apostrophus (or apostrophic C, or reversed C) 
% by rotating a capital C. 
\def\nbrotC{\rotatebox[origin=c]{180}{C}\xspace}
\end{teXXX}

The reversed \texttt{\string\nbrotC} can be found in unicode (|U+2183|) where it  is named |ROMAN NUMERAL REVERSED ONE HUNDRED|. It is also one of the Claudian letters. The overlines are produced using |$\overline{\text{X}}$|. The \cmd{overline} command simply places a bar on top of the letters or symbols.


\def\nbrotC{\rotatebox[origin=c]{180}{C}\xspace}



\topline
\begin{center}
\begin{tabular}{c@{\qquad\qquad\qquad}c}
\small
  \begin{tabular}[t]{r@{\;}c@{\;}l}
      1 & = & I\\
      2 & = & II\\
      3 & = & III\\
      4 & = & IV\\
      5 & = & V\\
      6 & = & VI\\
      7 & = & VII\\
      8 & = & VIII\\
      9 & = & IX\\
     10 & = & X\\
     20 & = & XX\\
     30 & = & XXX\\
     40 & = & XL\\
     50 & = & L\\
     60 & = & LX\\
     70 & = & LXX\\
     80 & = & LXXX\\
     90 & = & XC\\
    100 & = & C
  \end{tabular}&
  \begin{tabular}[t]{r@{\;}c@{\;}l}
          500 & = & D or L\nbrotC\\
        1,000 & = & M or C\nbrotC\\
        2,000 & = & MM or II\nbrotC\nbrotC\nbrotC\\
        5,000 & = & $\overline{\text{V}}$ or L\nbrotC\nbrotC\\
        6,000 & = & $\overline{\text{VI}}$ or LX\nbrotC\nbrotC\\ %[**TN: original wording "MMM", or 3,000]
       10,000 & = & $\overline{\text{X}}$ or C\nbrotC\nbrotC\\
       50,000 & = & $\overline{\text{L}}$ or L\nbrotC\nbrotC\nbrotC\\
       60,000 & = & $\overline{\text{LX}}$ or LX\nbrotC\nbrotC\nbrotC\\ %[**TN: original wording "MMM\nbrotC" or 30,000]
      100,000 & = & $\overline{\text{C}}$ or C\nbrotC\nbrotC\nbrotC\\
    1,000,000 & = & $\overline{\text{M}}$ or C\nbrotC\nbrotC\nbrotC\nbrotC\\
%[**TN: on the following line, original wording omitted " or MM" - this seems the most likely original intention]
    2,000,000 & = & $\overline{\text{MM}}$ or MM\nbrotC\nbrotC\nbrotC\\[1ex]
    \multicolumn{3}{c}{\parbox{14em}{When a line is drawn over a number it means that its value is increased 1000 times.}}
  \end{tabular}
\end{tabular}
\end{center}
\captionof{table}{Roman numerals}
\label{tbl:romannumerals}
\bottomline
\medskip

\end{comment}


Although you might be tempted to dispel the necessity for large numbers by the Romans, you should think of the Colloseum. 

The Colosseum was designed to take a capacity of between 50,000 and 80,000 spectators. The admission to the Colosseum was completely free but everyone had to have a ticket - tickets assisted in crowd control. An exciting event would attract all of the million people who lived in Ancient Rome - without tickets there would have been chaos. 

Outside the Colosseum there was a barrier consisting of chains between 160 bollards to keep people out before the opening of the games. The Colosseum had something that resembled a seating chart. Each ticket was marked with was marked with a seat number, a tier number and a sector number which indicated the correct entrance gate. It was therefore imperative to ensure that the massive crowds who flocked to the Colosseum were seated quickly and efficiently.

The Tickets to the Colosseum were completely free to the Ancient Romans. However, they did have to be acquired in advance or face standing in line on the day of the games and perhaps obtaining a ticket for standing room only. The areas of seating reflected the social status of the Romans. There were four tiers of seating. The closer you were to the action in the arena, the higher was your status in Rome. If you were a 'Pleb' there was no way that you would have access to the first and second tiers which were strictly reserved for the most important people of Rome. Different classes of people would be recognised by their clothing and who they arrived with. The Emperor, his family, noble Patricians, senators, politicians, magistrates and visiting dignitaries.



\section{Alternative solutions}


Many writers struggle at first with wrapping figures with text. The first attempt is to use a tabular
environment to place them. Consider the example shown below.


\section*{Rules for Measuring Surfaces and Solids}
\index{Area}
\index{Measuring surfaces and solids, rule for}
\index{Rules for measuring surfaces and solids}
\index{Solids, rules for measuring}
\index{Surfaces, rules for measuring}
{\setlength\intextsep{0pt}
\setlength\columnsep{1.5em}
\begin{wrapfigure}[10]{r}[-1.5em]{0pt}
\includegraphics[width=2.0in]{p93.pdf}
\end{wrapfigure}

\quad

\textbf{Parallelogram.}\quad%
\index{Area of parallelogram, to find}%
\index{Parallelogram, to find area of}%
---To find the area of a parallelogram,
multiply its
length, or \textit{base} as it is
called, by its height, or
\textit{altitude} as it is called, or
expressed in the simple
form of an algebraic
equation.---
\[
A = b \times  h
\]

\begin{wrapfigure}[10]{l}[-1.5em]{0pt}
\includegraphics[width=1.5in]{p94a.pdf}
\end{wrapfigure}

\textbf{Triangle.}\quad%
\index{Area of triangle, to find}%
\index{Triangle, to find area of}%
---To find the area of a triangle when
the base and altitude are given,
multiply its base by its altitude
and divide by $2$, or
\[
A=\frac{bh}{2}
\]
}
\vspace{2.0cm}


It is preferable to use the \pkg{wrapfig} to do the work for you and you will
prefer the final look of the page in the end.




\begin{teX}
{\setlength\intextsep{0pt}
\setlength\columnsep{1.5em}
\begin{wrapfigure}[10]{r}[-1.5em]{0pt}
\includegraphics[width=2.0in]{p93.pdf}
\end{wrapfigure}

\quad  % dummy to let wrapfig start before 
           % actual paragraph

\mathsubparagraph{Parallelogram.}%
\index{Area of parallelogram, to find}%
\index{Parallelogram, to find area of}%
---To find the area of a parallelogram,
multiply its
length, or \textit{base} as it is
called, by its height, or
\textit{altitude} as it is called, or
expressed in the simple
form of an algebraic
equation.---
\[
A = b \times  h
\]

\begin{wrapfigure}[9]{l}[-1.5em]{0pt}
\includegraphics[width=1.5in]{p94a.pdf}
\end{wrapfigure}

\mathsubparagraph{Triangle.}%
\index{Area of triangle, to find}%
\index{Triangle, to find area of}%
---To find the area of a triangle when
the base and altitude are given,
multiply its base by its altitude
and divide by $2$, or
\[
A=\frac{bh}{2}
\]
\end{teX}



\clearpage

{\Large RULES}

We can specify different types of rules, using the \pkg{booktabs}. In the table below, we use a \cmd{toprule}, \cmd{midrule} and \cmd{bottomrule} to draw the rulers as required. Under the month we use \cmd{cmidrule} in order not to draw the line fully.

{    \centering
    \captionof{table}{Usage of toprule, cmidrule and bottomrule}
    \label{tab:linien}
    \begin{tabular}{@{}l*{4}{l}@{}}
      \toprule
        Month & 1965 & 1966 & 1967 & 1968 \\
      \cmidrule(r){1-1}\cmidrule(lr){2-2}\cmidrule(lr){3-3}\cmidrule(lr){4-4}%
        \cmidrule(l){5-5}
        September & 2000 & 1700 & 2300 & 1900 \\
        Oktober   & 1500 & 1800 & 1900 & 3000 \\
        November  & 2500 & 2800 & 4700 & 3200 \\
        Dezember  & 2300 & 2000 & 3600 & 2700 \\
      \bottomrule
    \end{tabular}
}
  
\bigskip
 \pagebreak


The last table that is shown in \ref{frau}, shows usage of the |*| parameter command in tabular.


{    \centering
     \captionof{table}{Mehrspaltiger Reihensatz}
    \label{tab:mehrspaltig}
    \begin{tabular}{*{4}{l}}
      die Frau & der Frau   & der Frau  & die Frau \\
      der Mann & des Mannes & dem Manne & den Mann \\
      das Kind & des Kindes & dem Kinde & das Kind
    \end{tabular}
  \label{frau}
 }

\begin{teX}
   \begin{tabular}{*{4}{l}}
\end{teX}


{    \centering
     \captionof{table}{Tabellensatz}
    \label{tab:tabellensatz}
    \begin{tabular}{@{}*{4}{l}@{}}
      \toprule
        Nominativ & Genetiv & Dativ & Akkusativ \\
      \midrule
        die Frau & der Frau   & der Frau  & die Frau \\
        der Mann & des Mannes & dem Manne & den Mann \\
        das Kind & des Kindes & dem Kinde & das Kind \\
      \bottomrule
    \end{tabular}
} 


\clearpage

\onelineheader{LANDSCAPE TABLES}
\medskip


Sometime tables can be very wide. In this case we can use the |rotate| package or the |landscape| package to implement a sideways environment.

\begin{teX}
\begin{landscape}
  table code
\end{landscape}
\end{teX}




\onelineheader{VERICAL ALIGNMENT: MULTIROWS}


The \pkg{multirow} package developed by Jerry Leichter  automates the procedure of constructing tables with several rows merged into one. It does so by defining a command \textbackslash multirow. You can fine-tune the command by specifying optional parameters. The full format for the command is as follows:





\emphasis{multirow}
\begin{teXXX}
\multirow{nrow}[njot]{width}[vmove]{contents}
\end{teXXX}




In some instances vertical centering might not come out as desired. In this case, the optional parameter |vmove| can be used to introduce the shifts manually.


\index{Tables!vertical alignment}
\index{Tables!multirow package}
\index{Tables!merge}


\emphasis{multirow}
\vbox{
\noindent\rule{\linewidth}{0.4pt}
\begin{minipage}{5cm}
\begin{teX}
\usepackage{multirow}
\begin{tabular}{|l|l|}
\hline
\multirow{4}{25mm}{Common text in column 1}
  &cell 1a \\\cline{2-2} & cell 1b\\\cline{2-2}
  &cell 1c \\\cline{2-2} & cell 1d\\\cline{2-2}
\end{tabular}
\end{teX}
\end{minipage}
\hfill
\begin{minipage}{4cm}
\hfill\begin{tabular}{|l|l|}
\hline
  \multirow{4}{25mm}{Common text in column 1}
  &cell 1a \\\cline{2-2} & cell 1b\\\cline{2-2}
  &cell 1c \\\cline{2-2} & cell 1d\\\hline
\end{tabular}\hfill\hfill
\end{minipage}\hfill\hfill

\medskip
\noindent\rule{\linewidth}{0.4pt}
}

\captionof{figure}{Using the multirow package. The multirow package provides the command \texttt{\protect\textbackslash multirow} to join cells together. }


  
The package also offers the ability to fine-tune the multirow parameters using
\textbackslash multirowsetup.



\subsection{Coloring multirows}
If you use |\multirow| with the \pkg{colortbl} package you have to take precautions if you want to
colour the column that has the |\multirow| in it. |colortbl| works by colouring each cell separately.
So if you use |\multirow| with a positive |nrows| value, |colortbl| will first color the top cell, then
|\multirow| will typeset nrows cells starting with this cell, and later colortbl will color the other
cells, effectively hiding the text in that area. This can be solved by putting the |\multirow| in
the last row with a negative |nrows| value. See, for example:

\medskip

\emphasis{multirow,columncolor}
\vbox{
\noindent\rule{\linewidth}{0.4pt}
\begin{minipage}{6cm}
\begin{teXXX}
\begin{tabular}{l>{\columncolor{gray}}l}
aaaa & \\
cccc & \\
dddd & \multirow{-3}*{bbbb}\\
\end{tabular}
\end{teXXX}
\end{minipage}
\hfill
\begin{minipage}{3.5cm}
\begin{tabular}{l>{\columncolor{gray}}l}
aaaa & \\
cccc & \\
dddd & \multirow{-3}*{\parbox{2.5cm}{\raggedright \color{white} \lorem}}\\
\end{tabular}
\end{minipage}\hfill\hfill

\medskip
\noindent\rule{\linewidth}{0.4pt}
}

  
We have covered quite a bit of ground here. If you need more you need to experiment with the common packages used for tables

Table summarizes the packages that we have covered with a short description. It takes time to master tables and the code does look messy, if you do not spent time and take care to write it clearly. The end product is almost unbeatable by other modern software:


\begin{longtable}{lp{7cm}}
\toprule
hhline &do whatever you want with horizontal lines\\
       &\url{ctan.org/latex/packages/hhline}\\

array &gives you more freedom on how to define columns\\
colortbl &make your table more colorful\\
supertabular &for tables that need to stretch over several pages\\
longtable &similar to supertab.\\
          &\small Footnotes do not work properly in a normal tabular environment. If you replace it with a longtable environment, footnotes work properly for most document classes.\\

xtab &Yet another package for tables that need to span many pages\\

tabulary &modified tabular* allowing width of columns set for equal heights\\

arydshln &creates dashed horizontal and vertical lines\\

ctable &allows for footnotes under table and properly spaced caption above (incorporates booktabs package)\\

slashbox &create 2D tables with the first cell containing a description for both axes\\
\bottomrule
\end{longtable}


\clearpage

\section{ADDING EXTRA HEIGHT TO ROWS}

You can add extra height to a row as follows:

\emphasis{extrarowheight, begin,end,tabular,rowcolor}
\begin{teXXX}
\setlength\extrarowheight{2pt}
\begin{tabular}{|l|r|c|p{1.75cm}|}\hline
  Links & Rechts & Zentriert & Box\\\hline
  \rowcolor{cyan!40}
  l & r & c & p\{1.75cm\}\\\hline
\end{tabular}
\end{teXXX}

\bigskip

\setlength\extrarowheight{2pt}
\begin{tabular}{|l|r|c|p{1.75cm}|}\hline
Left & Right & Center & Box\\\hline
\rowcolor{cyan!40}
l & r & c & p\{1.75cm\}\\\hline
\end{tabular}

\clearpage

\section{Full width tables in multicolumn text}


\begin{teXX}
\documentclass[twocolumn]{article}
\usepackage{lipsum} 
\begin{document}
\lipsum
\begin{table*}[htbp]
  \centering
  \begin{tabular}{p{1in}p{1in}p{1in}p{1in}p{1in}}
  \hline
  Some text & Some text & Some text & some text & some text\\
  Some text & Some text & Some text & some text & some text\\
  Some text & Some text & Some text & some text & some text\\
  Some text & Some text & Some text & some text & some text\\
  Some text & Some text & Some text & some text & some text\\
  Some text & Some text & Some text & some text & some text\\
  Some text & Some text & Some text & some text & some text\\
  \hline
  \caption{A table}
  \end{tabular}
\end{table*}
\lipsum\lipsum
\end{document}
\end{teXX}
\clearpage

\onelineheader{TABLES IN \TeX}

  
We have covered a lot of ground with \latex tables, without mentioning \tex's way of handling tables. This was done on purpose, as many people consider the usage of |\halign|, as difficult. In my opinion it is only difficult, as the packages are not always available, but the greatest difficulty is the provision of rules. This normally clutters the table and hides the data.

\parindent1em

Use LaTeX tables first, before you move on to typesetting them using only TeX commands is preferred. The time though has come to discuss them in more detail.

\textbf{The \textbackslash halign command.}\quad Tables in \tex are typeset using mainly the commands \cmd{halign},\cmd{omit} and \cmd{span}. The TeX command for horizontal alignment is |\halign|. It must be followed by a group of commands enclosed by braces. This block contains all rows of the table with the cells separated by the character "\&" and the rows separated by |\cr|. The rows containing the text actually to be printed are preceded by a special one called the preamble. It is a template into which the content of the table cells are put before being typeset. "\#" characters serve as placeholders for the cell entries. (Inside macro definitions, "\#\#" is used to avoid confusion with the macro arguments denoted by |#1|, |#2| etc.) One |#| must occur between every two |&|s. 
%http://www.volkerschatz.com/tex/halign.html

\begin{teXXX}
\halign{\hfil \it # & # \hfil \cr
3 cl & cream \cr
2 cl & beer \cr
1.5 cl & orang-utan soup \cr}
\end{teXXX}

The text in the first column is printed in italics (\cmd{it}) and aligned right. This is achieved by the \cmd{hfil}, which is a spacer with zero default width but infinite stretchability. Whenever something is aligned right, left or centred, this is achieved with such fillers. (They come in three powers: |\hfil|, |\hfill|, and |\hfilll|. Each is infinitely more stretchable than its predecessor. The last one is not predefined in LaTeX, but can be emulated by writing |\hskip 0pt plus 1filll|.) Likewise, the second column is aligned left. This fairly simple way blews into complexity, once one decides to employ frames:

\begin{teXXX}
{
\offinterlineskip
\tabskip=0pt
\halign{ 
\vrule height2.75ex depth1.25ex width 0.6pt #\tabskip=1em &
\hfil 0.#\hfil &\vrule # & \qquad$0.#\,\pi$\hfil &\vrule # &
\hfil 0.#\hfil &#\vrule width 0.6pt \tabskip=0pt\cr
\noalign{\hrule height 0.6pt}
& \omit$\alpha_s$ &&\omit star angle && \omit diquark size [fm] & \cr
\noalign{\hrule}
& 3 && 22 && 34 &\cr
& 4 && 14 && 22 &\cr
& 5 && 095 && 15 &\cr
\noalign{\hrule height 0.6pt}
}}
\end{teXXX}



\centerline{\hbox to 5cm{\vbox{\halign{\hfil # & # \hfil& #\cr 
\toprule
3 cl & cream &other\cr
2 cl & beer  &ujon\cr\
1.5 cl & orang-utan soup &makeon\cr
\bottomrule
}}}}
\captionof{table}{Simple table, typeset using \texttt{\textbackslash halign}}

\clearpage


  
\textbf{Centering tables.}\quad The only way \text offers to center tables, is to use boxes and to use the appropriate amount of |\hfil|. Normally you will have to define a command for this:


\begin{teXXX}
\def\centertable#1{\hbox to \hsize {\hfill\vbox{%
   \offinterlineskip \tabskip=0pt \halign{#1} }\hfill}}
\end{teXXX}
\bigskip

\def\centertable#1{\hbox to \hsize {\hfill\vbox{%
   \offinterlineskip \tabskip=0pt \halign{#1} }\hfill}}

{
\offinterlineskip
\tabskip=0pt
\centertable{% 
\vrule height2.75ex depth1.25ex width 0.6pt #\tabskip=1em &
\hfil 0.#\hfil &\vrule # & \qquad$0.#\,\pi$\hfil &\vrule # &
\hfil 0.#\hfil &#\vrule width 0.6pt \tabskip=0pt\cr
\noalign{\hrule height 0.6pt}
& \omit$\alpha_s$ &&\omit star angle && \omit diquark size [fm] & \cr
\noalign{\hrule}
& 3 && 22 && 34 &\cr
& 4 && 14 && 22 &\cr
& 5 && 095 && 15 &\cr
\noalign{\hrule height 0.6pt}
}}
\captionof{table}{Example \tex table, using rules.}

  
\parindent1em
If you want to programmatically, produce tables, in most cases it will be easier to revert back to using \tex directly, rather than going the long route of packages and \latex.

Consider the table shown below, that has been drawn by D.E.~Knuth, in one of his papers\footnote{D.E. Knuth, \textit{Johann Faulhaber and Sums of Powers}}. This was possibly before the days of picture drawing programs, so Knuth drew all the rules using a tabular environment. This is now so much more easy with packages like TikZ and even \latex's picture environment can do better. 
$$\vbox{\offinterlineskip
\def\hb{\phantom{\hbox{A}}}
\halign{\strut#&\vrule#&\hfil#\hfil%
&\vrule#&\hfil#\hfil%
&\vrule#&\hfil#\hfil%
&\vrule#&\hfil#\hfil%
&\vrule#&\hfil#\hfil%
&\vrule#&\hfil#\hfil%
&\vrule#&\hfil#\hfil%
&\vrule#\cr
\omit&\omit&\omit&\omit&\omit&\omit&\omit&\omit
&\multispan7{\kern-.4pt\hrulefill\kern-.4pt}\cr
\omit&\omit&\omit&\omit&\omit&\omit&\omit&\omit&&\hb&&\hb&&\hb&&\cr
\omit&\omit&\omit&\omit&\omit&\omit
&\multispan9{\kern-.4pt\hrulefill\kern-.4pt}\cr
\omit&\omit&\omit&\omit&\omit&\omit&&\hb&&\hb&&\hb&&\cr
\omit&\omit&\omit&\omit&\multispan9{\hrulefill}\cr
\omit&\omit&\omit&\omit&\omit&\hb&&\hb&&\hb&&\cr
\omit&\omit&\multispan9{\kern-.4pt\hrulefill\kern-.4pt}\cr
\omit&\omit&\omit&\hb&&\hb&&\hb&&\cr
\multispan9{\kern-.4pt\hrulefill\kern-.4pt}\cr
\omit&\hb&&\hb&&\hb&&\cr
\multispan7{\kern-.4pt\hrulefill\kern-.4pt}\cr
}}$$
As you will see from the code below, it is easy to make a mistake when producing such construction.
   

\topline
\begin{teXXX}
In other words, it is the number of ways to put positive integers into
a $k$-rowed triple staircase such as
$$\vbox{\offinterlineskip
\def\hb{\phantom{\hbox{A}}}
\halign{\strut#&\vrule#&\hfil#\hfil%
&\vrule#&\hfil#\hfil%
&\vrule#&\hfil#\hfil%
&\vrule#&\hfil#\hfil%
&\vrule#&\hfil#\hfil%
&\vrule#&\hfil#\hfil%
&\vrule#&\hfil#\hfil%
&\vrule#\cr
\omit&\omit&\omit&\omit&\omit&\omit&\omit&\omit
&\multispan7{\kern-.4pt\hrulefill\kern-.4pt}\cr
\omit&\omit&\omit&\omit&\omit&\omit&\omit&\omit&&\hb&&\hb&&\hb&&\cr
\omit&\omit&\omit&\omit&\omit&\omit
&\multispan9{\kern-.4pt\hrulefill\kern-.4pt}\cr
\omit&\omit&\omit&\omit&\omit&\omit&&\hb&&\hb&&\hb&&\cr
\omit&\omit&\omit&\omit&\multispan9{\hrulefill}\cr
\omit&\omit&\omit&\omit&\omit&\hb&&\hb&&\hb&&\cr
\omit&\omit&\multispan9{\kern-.4pt\hrulefill\kern-.4pt}\cr
\omit&\omit&\omit&\hb&&\hb&&\hb&&\cr
\multispan9{\kern-.4pt\hrulefill\kern-.4pt}\cr
\omit&\hb&&\hb&&\hb&&\cr
\multispan7{\kern-.4pt\hrulefill\kern-.4pt}\cr
}}$$
\end{teXXX}
\bottomline

\clearpage
\onelineheader{ALIGNING VERTICAL MATERIAL: \textbackslash valign}


Just as |\halign| creates an alignment by specifying a prototype row, |\valign| creates an alignment by specifying a prototype column. Inside a |\valign|, |&| specifies the end of a row in a column, and |\cr| means "end-of-column"; each cell and column is typeset in (internal) vertical mode and the whole alignment is then passed to the paragraph builder (in horizontal mode). 
An example from David Bausum's book
\medskip

\begin{teXXX}
\valign{&\hbox to 1in{\vrule height9pt depth3pt width0pt#\hfil}\vfil\cr
badness  & box& boxmaxdepth& cleaders& dp& everyhbox\cr
everyvbox& hbadness& hbox& hfuzz& hrule& ht& lastbox\cr
leaders& overfullrule& prevdepth& setbox& unhbox& unhcopy& unvbox\cr
unvcopy& vbadness& vbox& vfuzz& vrule& vsplit& vtop\cr
wd& xleaders\cr}
\bye
\end{teXXX}
\bigskip

\topline

\valign{&\hbox to 1in{\vrule height10pt depth3pt width0pt\fbox{#}\hfil}\vfill\cr
badness  & box& boxmaxdepth& cleaders& dp& everyhbox\cr
everyvbox& hbadness& hbox& hfuzz& hrule& ht& lastbox\cr
leaders& overfullrule& prevdepth& setbox& unhbox& unhcopy& unvbox\cr
unvcopy& vbadness& vbox& vfuzz& vrule& vsplit& vtop\cr
wd& xleaders\cr}
\bottomline

From the same source, a slightly different preamble was offered by TH. This one redefined |\cr|

\begingroup
\def\cr{\crcr\noalign{\hfil}}
\valign{&\hbox{\strut#}\crcr
badness  & box& boxmaxdepth& cleaders& dp& everyhbox\cr
everyvbox& hbadness& hbox& hfuzz& hrule& ht& lastbox\cr
leaders& overfullrule& prevdepth& setbox& unhbox& unhcopy& unvbox\cr
unvcopy& vbadness& vbox& vfuzz& vrule& vsplit& vtop\cr
wd& xleaders\cr}
\unskip
\endgroup

\section{Shining TeX examples}

Applying  a macro at each cell or specific cells.


\def\mymacro#1{\lowercase{#1}\space }

\halign{&\mymacro{#}\cr
HELLO&WORLD&TEST\cr
TEST&123&OTHER\cr}

In \latex one can use the collect cell approach. Solution by martin, author of the |collcell| package.

\begin{teXXX}
\documentclass{article}
\usepackage{collcell}
\usepackage{array}
\newcommand*{\mymacro}[1]{\fbox{#1}}
\newcolumntype{C}{>{\collectcell\mymacro}c<{\endcollectcell}}
\begin{document}
\begin{tabular}{CC}
  TestA  & A longer test cell \\
  \empty & The new version supports 'verb'! \\
\end{tabular}
\end{document}
\end{teXXX}

\begingroup

Before text \lower 30pt\hbox{\valign{&\hbox to 2cm{\fbox{#}\hfil}\vfill\cr
one& two& three& four& five& six\cr}} after text.
\endgroup





   \chapter{Chemistry}

\section{Introduction}

A number of \latex2e packages are available for typesetting chemical formulae: \pkgname{chemfig}, \pkgname{ochem}, \pkgname{streetex}, \pkgname{xymtex}, \pkgname{chemformula} and  \pkgname{mhchem}. By far the most intuitive is |mhchem|. The package has been developed by Martin Hensl and is currently at version |v3.17|.

\begin{texexample}{Chemical formula with mhchem}{}
\ce{$A$ <->T[{Enclose spaces!}] $A’$}
\ce{Zn^2+
<=>[\ce{+ 2OH-}][\ce{+ 2H+}]
$\underset{\text{amphoteres Hydroxid}}{\ce{Zn(OH)2 v}}$
<=>C[+2OH-][{+ 2H+}]
$\underset{\text{Hydroxozikat}}{\ce{[Zn(OH)4]^2-}}$}
\end{texexample}

The \pkgname{chemfig} is a package for drawing chemical formula and can also be used for simpler applications. The package has been developed by Christian Tellechea and is using TikZ for the actual drawing operations. We cannot do justice to the many commands and settings and as the manual (which comes both in French as well as English) is excellent we refer you to it.

\begin{texexample}{Using the chemfig Package}{ex:chemfig}
\definesubmol\Me[H_3C]{CH_3}
\chemfig{*6((-!\Me)=(-!\Me)-(-!\Me)=(-!\Me)-(-!\Me)=(-!\Me)-)}
\end{texexample}

The phd package also provides a small contribution to chemistry typesetting with the Lua module \luacmd{molarmass}. This has been adapted from the Mediawiki, as a demonstration of interfacing TeX with Lua. To use the module, load it with the \luacmd{require} function. The molar mass of simple formulae can then be calculated and typeset. The typesetting is achieved using the |mhchem| package.

\begin{texexample}{}{}
\begin{luacode}
   require("molarmass")
   test_mm("NaCl")
   test_mm("NaOH")
   test_mm("CaCO3")
   test_mm("H2SO4")
   test_mm("C10H8")
   test_mm("CO2")
   test_mm("Mo")
   test_mm("HCl")
   test_mm("Si(OH)4")
   test_mm("CuSO4(H20)5")
   test_mm("H2O")
   test_mm("SB2O3")
\end{luacode}
\end{texexample}

Have a look at the module and extend it to use functions for typesetting atomic weights and also to accept isotopes, as an exercise.






  \makeatletter\@specialfalse
\cxset{custom = stewart}
\cxset{steward,
  numbering=arabic,
  custom=stewart,
  offsety=0cm,
  image={./images/hine03.jpg},
  texti={When Lamport designed the original \LaTeX\ sectioning commands he did not provide a fully comprehensive interface for modifying their design. With current tools available improvements are much easier to program and this chapter provides the details.},
  textii={\precis{In this chapter we discuss a method that allows the production of fancy chapter headings and formatting, based on a set of key values. Central  to this process is the separation of content from presentation.
We also discuss the basic formatting tools that are available and how one can modify them to mould new book designs.}
 }
}




\cxset{section align=left}
\chapter{Epigraphs}\index{epigraphs}
\epigraph{Example is the school of mankind, and they
will learn at no other.}{Letters on a Regicide Peace}



\section{Introduction}

Epigraphs or quotations before or after chapters are quite common in books. Peter Wilson's epigraph package, 
does a good job and we have adapted it where necessary to allow for a key value interface. The command:

\cs{epigraph}\marg{text}\marg{source}. By default the epigraph is placed at the right
hand side of the textblock, and the \marg{source} is typeset at the bottom right of the \marg{text}. 
Numerous settings allow for manipulating the width of the epigraph, the location and other 
variables. If the package is available we use it otherwise we use other internal commands.



\section{Key-value interface}
The key value interface provided by the package is shown below. It mostly follows the 
naming conventions of the epigraph package to make the transition easier for experienced users.
\medskip

\keyval{epigraph align}{\marg{left, center, right}}{A font-size command such as \cs{footnotesize}, 
\cs{small} and other similar commands.}

\keyval{epigraph rule width}{\marg{dim}}{A font-size command such as \cs{footnotesize}, \cs{small} 
and other similar commands.}

\keyval{epigraph font-size}{\marg{dim}}{A font-size command such as \cs{footnotesize}, \cs{small} and 
other similar commands.}

\keyval{epigraph beforeskip}{\marg{dim}}{Space before the epigraph.}
\keyval{epigraph afterskip}{\marg{dim}}{Space after the epigraph.}
\keyval{epigraph source align}{\marg{left, center, right}}{Align the source text to the right, left or center.}
\keyval{epigraph source font-size}{\marg{dim}}{Align the source text to the right, left or center.}
\keyval{epigraph source font-shape}{\marg{dim}}{Align the source text to the right, left or center.}
\keyval{epigraph source font-family}{\marg{dim}}{Align the source text to the right, left or center.}
\keyval{epigraph source font-weight}{\marg{bold,normal}}{Align the source text to the right, left or center.}


\section{Example usage}
To set the style and an example usage is shown in .

\begin{example}{epigraph example}{}
\cxset{epigraph width=0.5\linewidth,
       epigraph font-size=\small,
       epigraph rule width=0.4pt,
       epigraph align=right,
       epigraph source align=right,
       epigraph text align=right}


\epigraph{Example is the school of mankind, and they
will learn at no other.}{Letters on a Regicide Peace}
\end{example}

Another example for a somewhat longer quote:

\begin{example}{epigraph example}{}
\cxset{epigraph width=0.5\linewidth,
          epigraph font-size=\small,
          epigraph rule width=0.4pt,
          epigraph align=left,
          epigraph source align=right,
          epigraph text align=left}

\epigraph{Everything written with vitality expresses that vitality; there are no dull subjects, 
only dull minds.}{Raymond Chandler\\\textit{Letters on a Regicide Peace}}
\end{example}

More usage examples can be found in relevant style examples (See Chapter~\ref{ch:41}) for a rather 
nice example with non-traditional alignment.

\section{Epigraphs on empty pages}

When a chapter open on an odd page sometimes the  previous page is left empty. Some book designers 
add the words ``this page left intentionally blank'' and other might add a quote. To add such a quote use:

\begin{tcolorbox}
\begin{lstlisting}
\cxset{blank page text=\epigraph{The great tragedy of science is the slaying of a beautiful theory
by an ugly fact.}{Thomas Huxley}}
\end{lstlisting}
\end{tcolorbox}

}  
 


\def\phddoc{%
    \begin{epigraphpage}
 \epigraph{Begin at the beginning,'' the King said, gravely, ``Then
 go till you come to the end; then stop.''}{Lewis Carroll, {\it Alice
 in Wonderland}}

 \epigraph{You can never get a cup of tea large enough or a book long enough to
 suit me''}{C. S. Lewis}
 \end{epigraphpage}

\parindent1em
%\cxset{style13}
%\cxset{title margin bottom=10pt,
%          title beforeskip=1pt}

\chapter{Introduction}
\addtocimage{-12pt}{-20pt}{../images/tocblock-fish}


\epigraph{``Begin at the beginning,'' the king said
"and then go on till you come to the end, then stop."}{
---Lewis Carroll, Alice in Wonderland}

 \parskip3pt plus 5pt 
\noindent This package and its documentation attempts to eliminate some common 
problems encountered when using \LaTeX2e. The first one is the loading of 
recommended packages for a large and perhaps complicated document and 
the second is the re-designing of styles for a document.

 \LaTeX2e, does not provide a standard library, but comes equipped with
 a package mechanism that allows code extensions to be loaded as required.
 This has created a strong vibrant community, hundreds of packages and a 
 headache to both new and seasoned users. What packages are available, when
 to use them and in which order is a common theme for many questions on
 lists and |TX.SE|.

 It is quite common during the writing of a thesis or book
 for the author to keep on adding macros and packages
 at the preamble of the document. In most cases this can
 be satisfactory but in many others it leads to
 incompatibilities and errors. This package aims at
 minimizing one's preamble, by prefetching a number of
 commonly used packages. It also aims at loading them
 in the right order and providing patches for conflicts.
 
 I am hoping that using this package, will lead to less
 frustrations with the intricacies of \LaTeX2e\ packages.

The package code is complicated, but its usage is simple. You first load the package and then
you use one of the available templates:

 \begin{commands}[]{}
 \begin{verbatim}
 \usepackage{phd}
 \usetemplate{style13}
 \end{verbatim}
 \end{commands}

This is what you need to typeset a good looking book or thesis. The rest of this book is a footnote and you can skip them if you want. 

It will be better for the longer projects to just fork the
 package and adapt it to your needs. In this respect, I have
 uploaded the package to |github|.\footnote{\url{https://github.com/yannisl/phd}}

 My goal in selecting the packages and adding a number of 
 commands for the authors was to be able to typeset a 
 document for most common use cases, without the need of
 additional packages. The packages I selected are biased
 towards academic publications, although they can find use
 in almost any fields. The package provides a mechanism via
 PGF keys to provide a settings file. 
 
 Most of the documentation can be found in the implementation part.

Browse any books in a library or bookshop and the striking thing is that their design is very individualistic. They might have similarities but their main features vary. In many respects they resemble people's faces where minor differences have striking effects.

This package arose out of a question at stackexchange. How to redefine chapter heads. Having seen the popularity of the |pgf| package \cite{pkg-pgf} I realized that \latex users prefer this method of styling rather the traditional \latex method.

The user interface can be extended to basically all major packages. The principle is to keep to a minimum changes that can affect the LaTeX core commands. If there are any additions a key setting is provided to be able to revert back to normal LaTeX.

The workflow can be simplified. In addition I want to believe that the interface can provide a useful addition to the open source community and that other people will contribute style libraries, which will be simpler to write. It is also possible
to device an easy and uncomplicated web interface to handle
such a great number of variables.


Most people when they get started with \LaTeX\ will either use one of the standard classes such as the \docFile{book.cls} or one of the generic classes notably koma-script or memoir. Most students will be forced to use on of the many thesis classes available.

\section{The key value concept}

The key-value concept that originated with \LaTeX\ has been extended many times, the last and most serious implementation of it by Tantau in the PGF package. What essentially Tantau developed is a scripting language to script TeX code. The \tikzname and pgfplots packages are two major packaged that use keys effectively. Their popularity is growing and what this package does is to offer a user interface that has been modelled to be similar to that of \texttt{css} (cascade style sheets). 
\smallskip

\begin{scriptexample}{}{}
\textit{chapter number} font-size = Large,\\
\textit{chapter number}     color = theblue
\end{scriptexample}
\smallskip

The main idea behind the package, is that you are configuring a document style by means of \emph{settings} rather than writing macros. In the example above the \emph{number, chapter} can be thought of as class or id names in css style sheets and the |font-size, color| as property settings that apply to the particular element. 


\subsection{Settings}

Settings are activated either by using the command |\cxset|  or by loading a full style sheet. In most cases you will probably import a style sheet and then modify some of the properties using |cxset|.  For example this heading has a dot after the subsection number. This was accomplished by setting,

We can de-activate it for the next and subsequent subsection headings with the setting:

\lorem

\begin{scriptexample}{}{}
\begin{verbatim}
\cxset{subsection number after=\quad}
\end{verbatim}
\end{scriptexample}




\subsection{Cascading}

Most values once set for a higher section will be seen in a cascade by all subsectioning commands in a similar fashion similar to CSS. These include properties such as color, font families and alignment. Best though to specify all of them for maximum flexibility to your users.

\section{On typography}

This package hopefully will assist in improving the typography of books set with \latexe. Any typographical comments on the various styles are just my own ramblingss and not necessarily absolute truths. Like fashion and art typography has opinions rather than absolute truths. In many styles the design is slightly adapted to blend a bit better with this manual. Also I did not select fonts as per the samples but this is left on you the user to decide.



\section{Packages and Fonts}

This manual has been typeset with numerous fonts in order to enable the typsetting of almost all the scripts provided by the Unicode standard. In order to process it from the |.dtx| file, these fonts must be available in your system, otherwise \XeLaTeX\ will have a problem finding the fonts and it will take an awful long time to process. This is especially true for the scripts section, where virtually all the Unicode defined scripts are discussed. You will need a fast computer and a fast hard disk to process the document within a reasonable time. When using \pkg{fontspec} always define your fonts with the \cmd{\newfontfamily} this will speed up processing by an order of magnitude. Compiling from the command prompt will speed up compilation. Average speed 2-3 pages per second.

Many of \tex's parameters are stretched to the limit with a complicated document such as this manual. You will require a full distribution otherwise expect some errors. Important packages is \pkg{morefloats} and \pkg{morewrites}. The package will also expect that you have |e-tex| installed. Ubuntu users are normally one year behind in updates, so you might wish to update manually. It will take upwards of 5 minutes to compile fully on an old laptop and a couple of minutes on a state of the art computer.

The |dtx| should be processed best with its own make file provided for Windows only |phd-lua.bat|. The make file will process the documentation using \lualatex. You can also process the document with \xelatex but is prone to produce errors. Using \latexe the sections on scripts etc will not be printed and a much shorter version of the manual is provided. 

\section{Scripts and Languages}

The package and the documentation offer a full repertoire of font selection keys for different scripts and languages. It hasn't been possible, however hard I tried to compile this section of the documentation with \xelatex, as it kept giving errors of too many files open. This was also not possible even with the \pkg{morewrites} package loaded. With \lualatex the document compiled with no major problems other than the font rendering being of a lower quality to that of XeLaTeX on windows, other than disabling incompatible packages and a number of commands that were redefined. 

Some good news for multi-script typesetting is the |Noto| fonts from Google. These fonts named Noto from "No Tofu" meaning you do not see any little square blocks for undefined glyphs, are fast to load. Disantvantage you need to switch between font commands fairly often.

\section{This book}

When developing the templates, I started using \emph{lorem ipsum} text as samples. Half-way through this
became a jumble mass of uninteresting pages interspersed with code. Headings and the contents of the book
determine both the structure and the selection of fonts, so I went back and wrote narratives  to accompany
the headings. Many of the narratives are semi-autobiographical in nature; others are clustered around books I read and my own interests. Some I stumbled on them accidentally and are mostly there to demonstrate some code.

Besides the templates and the code there is another narrative which is based on notes I kept on \tex and its friends over the years and are offered as a more advanced introduction to coding \latexe and \tex. The whole manual was typeset in a |ltxdoc| class, slightly modified to turn into a book class.

The implementation code is also available and it was mostly for my own benefit. The whole manual with the exception of the |\cxset| introduction, is just a test document. The notes and the “dissection” of the standard \latexe and the standard classes are there to explain the background to the many coding decisions that I took while I was developing the package.

PhD students are notorious for going in all directions and exploring many adjacent fields before they sit down and write their theses. Some become life-time students. To all these new men and women of the Renaissance that slave away to inch knowledge one thesis at a time, I dedicate this book and the name of the package.

\subsection{The TeX hacking sections}

To start programming \tex you need to have a knowldge of \tex basic commands and approach. \latex2015 is a format build on top of \tex to provide a more structured approach. To program \latexe packages you need to understand \latexe concepts, code organization and conventions. To program in \latex3, you need to learn a whole new language and you still need to understand \tex, \latexe and the expl3 language and conventions. To program using LuaTeX, other than the Lua language you need to understand \tex very well.
None of these can be found in one place.  I have gathered a lot of material and put it together. This is not a language you can master easily or quickly, but can teach you a lot about typesetting, computer science and many other interesting topics.


 \section{Version control with Git and Github}
 
 If you are involved with code or a publication that will have frequent changes, you should consider
 some type of version control system. My own recommendation is to use |git| and an online repository such
 as |github|. The latter is currently very fashionable and makes sharing code easier. Note that the |github|
 offers both public as well as private repositories. The general recommendation is that for unpublished work
 such as a thesis or code under development, it is preferable to go for a private repository. 
 
 \lorem\lorem

 \section{Ordering of Packages}
 
One package that normally leads to errors is the 
\pkg{hyperref}. The package which is an outstanding example of software engineering and supported single handledy by Heiko Oberdiek\footcite{hyperref} redefines a a lot of internal commands of the kernel. As a lot of other packages do the same it has to be loaded at the end of the preable with the exception of some packages! 
 
 This manual is typeset according to the conventions of the
 \LaTeX \textsc{docstrip} utility which enables the automatic
 extraction of the \LaTeX{} macro source files~\cite{GOOSSENS94}.

 
 \href{http://tex.stackexchange.com/questions/96350/problem-with-algorithmic-and-hyperref}{problem with algorithmic and hyperref}

 \begin{verbatim}
\usepackage{float}  % load float package first!

\usepackage{hyperref} % let hyperref patch the float package stuff
.
 \usepackage{algorithm} % let algorithm use the patched version of the float package
 \end{verbatim}
 

\section{Known problems}

Perhaps the biggest issue with the package is the speed of
compilation with \XeLaTeX\ or \LuaTeX. This is to be expected, as both engines spend a lot of resources in font management. On demand loading of packages is something I have in the back of my mind. This should be done via document styles i.e., if a book is for the humanities, perhaps only a rudimentary amount of maths packages should be loaded.

\section{Future Directions}

\latexe and \tex usage appears to be increasing. This is mostly by programs that export results with \latexe code rather than authors writing books.  The method adopted here is easier to automate all sorts of reports and automated texts. I would like too develop a web interface for processing such templates and at the same time export into html instead of just producing pdfs. I have already a prototype.   

\section{Tooling}

Some of the scripts on a Windows machine need MSYS\footnote{\url{http://mingw.org/wiki/MSYS}}









   \makeatletter
\cxset{defaults/.style ={% 
    chapter title margin-top-width    =  0cm,
    chapter title margin-right-width  =  1cm,
    chapter title margin-bottom-width = 10pt,
    chapter title margin-left-width   = 0pt,
    chapter align                     = left,
    chapter title align               = left, %checked
    chapter name                      = CHAPTER,
    chapter format                    = block,
    chapter font-size                 = Huge,
    chapter font-weight               = bold,
    chapter font-family               = sffamily,
    chapter font-shape                = upshape,
    chapter background-color          = white,
  % chapter label    
    chapter color               = black,
    chapter number prefix             = ,
    chapter number suffix             = ,
    chapter numbering                 = arabic,
    chapter indent                    = 0pt,
    chapter beforeskip                = -3cm,
    chapter afterskip                 = 30pt,
    chapter afterindent               = off,
    chapter number after              = ,
    chapter arc                       = 0mm,
    chapter label background-color    = white,
    chapter label color               = black,
   % chapter afterindent               = on,
    chapter grow left                 = 0mm,
    chapter grow right                = 0mm,
    chapter rounded corners           = northeast,
    chapter shadow                    = fuzzy halo,
    chapter border-left-width         = 0pt,
    chapter border-right-width        = 0pt,
    chapter border-top-width          = 0pt,
    chapter border-bottom-width       = 0pt,
    chapter padding-left-width        = 0pt,
    chapter padding-right-width       = 10pt,
    chapter padding-top-width         = 10pt,
    chapter padding-bottom-width      = 10pt,
    %  
    chapter number color              = black,
    chapter number background-color   = white,
    chapter number font-size        = huge,
    chapter number font-weight      = bfseries,
    chapter number font-family      = sffamily,
    chapter number font-shape       = upshape,
    chapter number align            = Centering,
    %
    chapter title font-size        = Huge,
     chapter title font-weight      = bold,
     chapter title font-family      = sffamily,
     chapter title font-shape       = upshape,
     chapter title color            = black,
     chapter title background-color = white,
     }%
   }  
\makeatother     
%\makeatletter
%\cxset{toc image=\@empty,
%       chapter toc=true,
%       title beforeskip=1pt}
%
%\@specialfalse
%
%
%\renewcommand\stewart[2][]{%
%\fancypagestyle{fancy}{%
%\lhead{}\rhead{}
%\chead{}
%\cfoot{}
%\lfoot{}
%\rfoot{\thepage}
%\def\footrule#1{{\color{blue}%
%  \hrule width\paperwidth}\vskip3pt
%}
%
%\renewcommand{\headrulewidth}{0pt}
%\renewcommand{\footrulewidth}{0.4pt}}
%
%\clearpage
%
%\begin{tikzpicture}[remember picture,overlay]
%% Main shading block
%\node [xshift=5cm,yshift=-\paperheight] at (current page.north west)
%[text width=0.98\textwidth,text height=\paperheight, fill=thecream!30,rounded corners,above right]
%{};
%\node [xshift=6.5cm,yshift=-1.5cm-\soffsety] at (current page.north west)
%[text width=0.9\textwidth,below right]{\sffamily \bfseries \huge #2};
%
%\node [xshift=3cm,yshift=-1.5cm] at (current page.north west)
%[text width=3cm,align=center,minimum height=2.5cm, fill=blue,below right]
%{\[\text{\HHUGE\bfseries\sffamily\color{white}\thechapter}\]
%\par\vspace*{3pt}
%};
%
%\node [xshift=-0.2cm,yshift=-21.5cm] at (current page.north west)
%[text width=3cm,above right]%
%{\includegraphics[width=1.0\paperwidth]{\image@cx}};
%% second box left
%\node [xshift=3cm,yshift=-19.5cm] at (current page.north west)
%[text width=9cm,minimum height=2.5cm,inner sep=0.5em, fill=blue,below right]
%{\color{white}
%  \bfseries\sffamily \texti@cx
%};
%% Last block
%\node [xshift=6.5cm,yshift=-26cm] at (current page.north west)
%[text width=12cm,above right]
%{\textii@cx
%};
%\end{tikzpicture}
%\par
%\clearpage
%}





\cxset{steward,
  chapter numbering=arabic,
  chapter format = stewart,
  offsety=0cm,
  image= {./images/hine02.jpg},
  texti={When Lamport designed the original \LaTeX\ sectioning commands he did not provide a fully comprehensive interface for modifying their design. With current tools available improvements are much easier to program and this chapter provides the details.},
  textii={\precis{In this chapter we discuss a method that allows the production of fancy chapter headings and formatting, based on a set of key values. Central  to this process is the separation of content from presentation.
We also discuss the basic formatting tools that are available and how one can modify them to mould new book designs.}
 }
}


\chapter{Designing Chapter Headings}
\addtocimage{-12pt}{-20pt}{./images/tocblock-man-01.jpg}

\section*{Introduction}

A \textls*{crowded} first page is as unsightly as a crowded title page, wrote De Vinne in \emph{Modern Methods of Book Composition} in 1904.  Not much has changed since. A new chapter must make a good impression and must give an immediate signal that a different topic is going to be discussed. Traditionally chapter openings in LaTeX are an unimpressive and dry event. Our aim is to brighten it up a bit, while keeping true separation of content from presentation, but avoiding the pit traps of over ornamenting the design. A book is to be read and we should provide minimal ornamentation. \index[phdkeys]{chapter> ornamentation}

% \usepackage{array,tabularx}
%\newcolumntype{Y}{>{\raggedleft\arraybackslash}X}% see tabularx
%\tcbset{enhanced,fonttitle=\bfseries\large,fontupper=\normalsize\sffamily,
%colback=yellow!10!white,colframe=red!50!black,colbacktitle=thecodebackground,
%coltitle=black,center title,
%tabularx={X||Y|Y|Y|Y||Y},% this sets ’before upper’ and ’after upper’
%before upper app={Group & One & Two & Three & Four & Sum\\\hline\hline} }
%
%\begin{tcolorbox}[title=My table]
%Red & 1000.00 & 2000.00 & 3000.00 & 4000.00 & 10000.00\\\hline
%Green & 2000.00 & 3000.00 & 4000.00 & 5000.00 & 14000.00\\\hline
%Blue & 3000.00 & 4000.00 & 5000.00 & 6000.00 & 18000.00\\\hline\hline
%Sum & 6000.00 & 9000.00 & 12000.00 & 15000.00 & 42000.00
%\end{tcolorbox}

\begin{figure}[htbp]
\centering
\parindent=0pt
\fbox{\includegraphics[width=\textwidth]{metropolitan-spread}}
\par
\caption{A chapter opening from the Metropolitan Museum of Art publicaion, \textit{Assyrian Reliefs and Ivories} by Vaughn. E. Crawford et. al., 1980. The spread is simple and the chapters are not numbered. This is a common characteristic of many more recently published books.}
\end{figure}


What is to us now a common occurence with instant book-printing was not always so. The cost of illustrated books was a prime factor and as Tschichold wrote:
\begin{quotation}
In the area of book design, in the last few years a revolution has taken place, until recently recognized by only a few. but which now begins to influence a much wider range of action.
It means placing much greater emphasis on the appearance of the book and a wholly contemporary use of typographic and photographic means. Before the invention of printing, literature of that time was spread around by the mouth of the author himself or by professional bards. The books of the Middle Ages - like the "Mannessische Liederhandschrift" - had
\end{quotation}

The type of book you are writing and its contents will determine an appropriate design for chapter headings and the type of design and numbering if any for subsections. Here we are merely providing a mechanism to produce them. These methods can produce a mastepiece or an ugly piece of work. Some simple suggestions follow (from my observations of styles in books I like). In general you need to think what type of book you are developing. For example a novel, should be sectioned very carefully. Many books avoid marking of sections other than chapters totally, perhaps marking them just with a soft ornament such as three centered asterisks.

\section{Numbering of Sections}


In general books do not number sections beyond subsection. You can avoid them all together, if you are not going to reference the sections extensively. 

In works of fiction, authors sometimes number their chapters eccentrically, often as a metafictional statement. For example:
Seiobo There Below by László Krasznahorkai has chapters numbered according to the Fibonacci sequence.

The Curious Incident of the Dog in the Night-Time by Mark Haddon only has chapters which are prime numbers.

At Swim-Two-Birds by Flann O'Brien has the first page titled Chapter 1, but has no further chapter divisions.

God, A Users' Guide by Seán Moncrieff is chaptered backwards (i.e., the first chapter is chapter 20 and the last is chapter 1). The novel The Running Man by Stephen King also uses a similar chapter numbering scheme.
Every novel in the series A Series of Unfortunate Events by Lemony Snicket has thirteen chapters, except the final instalment (The End), which has a fourteenth chapter formatted as its own novel.

Mammoth by John Varley has the chapters ordered chronologically from the point of view of a non-time-traveler, but, as most of the characters travel through time, this leads to the chapters defying the conventional order.


\begin{pgfpicture}
\pgfpathmoveto{\pgfpointorigin}
\pgfpathlineto{\pgfpoint{1cm}{1cm}}
\pgfpathlineto{\pgfpoint{1cm}{0cm}}
\pgfusepath{fill}
\end{pgfpicture}




\begin{figure}[tbp]
\centering
\parindent=0pt
\fbox{\includegraphics[width=\textwidth]{fantasy-architecture}}
\par
\caption{A chapter opening from the Metropolitan Museum of Art publicaion, \textit{Assyrian Reliefs and Ivories} by Vaughn. E. Crawford et. al., 1980. The spread is simple and the chapters are not numbered. This is a common characteristic of many more recent books.}
\end{figure}


\begin{figure}[tbp]
\centering
\parindent=0pt
\fbox{\includegraphics[width=\textwidth]{fantasy-architecture-02}}
\par
\caption{A chapter opening from the Metropolitan Museum of Art publicaion, \textit{Assyrian Reliefs and Ivories} by Vaughn. E. Crawford et. al., 1980. The spread is simple and the chapters are not numbered. This is a common characteristic of many more recent books.}
\end{figure}


\section*{Use of Color}

The modern books that Tschilchod was discussing have long been overwhelmed by the appearance of larger, coffee book type of books. Our brains our now conditioned by branding and graphic design is everywhere. 

Once you have decided that the book is going to be a bit more colorfull, the choice of color will follow. The decision what to color will be an important one, which brings us to color theory. The history of color is perhaps as colorfull as the rest. Attempts to formalize and recognize order date back to Aristotle (384-322 bce) but began in earnest with Leonardo da Vinci (1452-1519) and have progressed ever since. Leonardo noted that certain colors intensify each other, discovering \textit{contrary} and \textit{complementary} colors. The first color wheel was invented by Britain's Sir Isaac Newton (1642-1727), who split white light into red, orange, yellow, green, blue, indigo and violet beams, then joined the two ends of the spectrum to form a circle showing the natural progression of colors. When Newton created the color wheel, he noticed that mixing two colors from opposite positions produced a neutral or \textit{anonymous} color.


\begin{figure}[htbp]
\parindent=0pt
\centering
\fbox{\includegraphics[width=\textwidth]{line-designs} }
\caption{Spread from \textit{Beautiful Geometry}, Eli Maor and Eugen Jost, Princeton Univeristy Press, 2014. A subtle coloring of the chapter heading, de-emphasizing the chapter number and coloring the chapter title. There is no chapter label. A dropcap with the same color starts the first paragraph. This style is easy to achive with the phd system.}
\end{figure}


\begin{figure}[htbp]
\parindent=0pt
\centering
\fbox{\includegraphics[width=\textwidth]{color-book01.jpg} }
\bigskip

\fbox{\includegraphics[width=\textwidth]{color-book02.jpg} }
\end{figure}

One would expect a book written for the sole purpose of describing color theory and its application to the Graphic Arts, is expected to be colorful. Note the de-emphasizing of the label and number. 

\begin{figure}[htbp]
\parindent=0pt
\centering
\fbox{\includegraphics[width=\textwidth]{color-book-03.jpg} }
The chapter heading label and number are almost invisible. The heading text, is typeset in large bold letters, shouting what is coming next. Not your typical scintific book\ldots
\bigskip

\fbox{\includegraphics[width=\textwidth]{color-book-04.jpg} }
\end{figure}

Advertizing people understand that they need to present the message of an advertizement loud and clear so as to catch the busy eye. A heading's message is the title description. Neither the label not the chapter if any are necessary to convey the message. The chapter heading is analogous to the stop at the end of a sentence. The brain gets a signal to absorb what was written before it and get ready for the next. The heading signals the end of a topic. One must not dwell on it.


\section{Contemporary Chapter Headings}

In the book \textit{China} the designer used both a chapter heading on a spread of two images, as well as repeated the chapter number on the text pages \ref{fig:threepage}. The images distill the message of the chapter, although the chapter subtitle is almost unreadable, dominated by the surrounding text. From a technical perspective, the chapter command must paint the two images, set the right type of heading for each page and then without increasing the counter, change the counter to one that displays the chapter number in words and then continue with typesetting the text. A careful choice of images is necessary for such chapters, as well as cropping the images to match the aspect ratio of the book pages. One also needs to be carefull for \latexe not to place any floats in between the page spreads. 

\begin{figure}[htbp]
\parindent=0pt
\centering
\fbox{\includegraphics[width=\textwidth]{beijing.jpg} }\par
\vfill

\fbox{\includegraphics[width=\textwidth]{beijing-01.jpg} }\par
%\fbox{\includegraphics[width=\textwidth]{pearl-river.jpg} }
\caption{A full page chapter spread.}
\label{fig:threepage}
\end{figure}

\begin{figure}[htbp]
\parindent=0pt
\centering
\fbox{\includegraphics[width=\textwidth]{beijing.jpg} }\par
\vfill

\fbox{\includegraphics[width=\textwidth]{beijing-01.jpg} }\par
%\fbox{\includegraphics[width=\textwidth]{pearl-river.jpg} }
\caption{A full page chapter spread.}
\label{fig:threepage}
\end{figure}


\clearpage



In Figure~\ref{fig:photospread} the bands are black, but position low on the page. The size of the pages are 9.69 \texttimes 11.42. The books sections are not numbered. Text i sbroken through inserts of bigger text. Many of the examples here are from
commercial nude photography books, as they tend to break with tradition. In the 1970s and 1980s, fashion photographers began to present a
new, confrontational image of the female body. The pioneer in this
respect was the German Helmut Newton (1920–2004). Newton’s
photographs of nudes were overtly sexual, with an undertone of
menace, and although his models tended to be depicted as part
of the social elite they were often placed, apparently caught out
in reportage style, in sordid environments engaged in fantasy and
fetish. His work made him highly influential in fashion photography,
though some of it was thought too highly sexual for American
magazines and appeared only in those published in Europe.


\begin{figure}[htbp]
\parindent=0pt
\includegraphics[width=\textwidth]{baetens-01.jpg} \par
\vfill\vfill\vfill\vfill
\includegraphics[width=\textwidth]{baetens-02.jpg}\par
\caption{Chapter spread and first pages after the chapter title which is on the right page of the chapter spread. From \textit{New Photography, Art and the Craft}, Pascal Baetens, DK Publications. }
\label{fig:photospread}
\end{figure}

In the 1980s, Newton undressed the dynamic and independent
female in a series called Big Nudes. In this series the women are
indeed naked and very tall, wearing nothing but makeup and high
heels. The Big Nudes were exhibited in the form of life-size prints
that were intended to provoke the viewer by showing self-confident
women who knew what they wanted and were very aware of their
beauty and sexuality



\chapter{Package Usage}

To use the package include it just like any other package:

\begin{teXXX}
\documentclass{book}
\usepackage{phd}
\cxset{style13}
\begin{document}
\chapter{Introduction}
\end{document}
\end{teXXX}

The command \docAuxCommand{cxset} sets the default style for the example to the style defined as \meta{style13}. The package currently offers  100 templates and numerous keys to manipulate them further. Styles are similar to \enquote{themes} used in web programming; they are a collection of keys that resemble in many ways \texttt{css}. Styles can have any names and I am sure as package usage increases and evolve,they will get better names. 

\section{Background}

Before describing in detail how to specify a new layout for headings, we offer an overview of how the task can be accomplished and the design philosophy behind the approach. 

Irrespective of the technique and tools used, the creation of new layouts can always be divided into the following three tasks: constructing a document from “layout bricks”, which we can term as “blocks” or “elements”; establishing the layout semantics of each block; and finally, creating a layout engine supporting any document constructed from such blocks.

\begin{description}
\item [Canned Layouts] At one end of the spectrum, the most accessible approach consists of picking, a canned layout, such as LaTeX itself and perhaps only provide rudimentary macros to manipulate it.
\item [Constraints] Constraints offer a middle ground between canned layouts and handwritten layout engines. Constraints are arguably the most widespread and successful layout programming technique. For, instance, the foundations of \tex are laid upon constraint. CSS, the ubiquitous web template language, also relies on constraints, although in a more restricted and indirect manner.
\end{description}

\subsection{Blocks and Elements}

We define an \emph{element} as a document block, that cannot be subdivided further. For example the chapter title element, is composed of the text of the chapter title. 

A \emph{block} on the other hand is can contain other blocks and or numerous elements. We can consider the chapter headings as \emph{blocks}, composed of three blocks the chapter, number and title. Each block is then composed of elements. Each element has properties and traits. One of these mandary properties is the name. 

Blocks are either \emph{configured} (all constraints are mandatory), or flexible (there are optional/alternative constraints). By bundling optional constraints, flexible blocks make their specification customizable by non-technical users. 

\subsection{Language semantics}

One of the aims of the syntax of the templates was to offer familiar terminology and to remove the use
of \tex macros as far as possible from templates. 
\medskip

{\parindent0pt

 \textit{section}| font-family=serif,|\\
 \textit{section}| font-size=LARGE,|\\
 \textit{section}| font-weight=bold,|\\
}

The restriction I imposed is problematic when dealing with fractions of linewidths and textwidths. So
at present we allow for example |title text-width=0.5\texwidth| or |title text-width=10cm| or any other valid units. Ideas for improvements can only come from user feedback in the future.

Some experimental ideas incorporated are:

\begin{verbatim}
title text-width = 0.5 text-width,
title text-width = 1.2 text-width,
\end{verbatim}

A better parser will need to be programmed for dimensions, which are all currently handled as etex |dimexpr|. 

The syntax must allows both for microtypography as well as macro-typographical features. The former would deal with mostly fonts, spacing and text justification, where the latter deals with layouts, borders shapes and the positioning of elements on the page and also reletively to other elements or blocks.

An advantage of this approach is that it also opens the possibility of parsing the text with a language other than \tex and translating the document to another format, such as |HTML| or |XML| either fully or partially. Next we will describe both the syntax as well as the usage of the settings.

\section{Chapter opening page}

The standard \latexe classes offer only two options to either open a chapter on an odd page or at any page. This package offers five alternatives:

\begin{docKey}[phd]{chapter opening}{=\meta{any, left, right, anywhere, ifafter}}{default none, initial=any}
For documents that are primarily to be read on the web, use |any| for normal books, use \textit{right}. Some templates that we provide use |any| and the examples use |anywhere| to enable us to display the heading at any position on the page.
\end{docKey}

\begin{decription}
\item [any] Opens a chapter at any page, either \textit{verso} or \textit{recto}.
\item [left] Opens a chapter on an even page
\item [right] Opens a chapter on a right page.
\item [anywhere] Opens a chapter at the point where the \cs{chapter} is typed.
\item [none] Alias for \marg{anywhere}.
\item [ifafter] Opens a chapter at the next page if the page has material that does not exceed a certain portion of \cs{textheight}.
\end{description}

\colorlet{theoption}{bgsexy}

To change a setting you just modify the value of the key \oarg{\option{chapter opening}} to one of the values described earlier. 

\begin{dispListing}
\cxset{chapter opening = anywhere}
\end{dispListing}
 
We use this key to print the many examples typesetting chapter heads that follow (see the example~\ref{ex:anywhere}).  


\begin{texexample}{title=Inline Chapter Example}{ex:anywhere}
\cxset{examplestyle/.style = {chapter format = block,
       chapter opening = anywhere,
       chapter name = CHAPTER, 
       %label
       chapter label font-family      = sffamily,
       chapter label color            = primary,
       chapter label background-color = white,
       % number
       chapter number font-family = sffamily,
       chapter number font-size = HUGE,
       chapter number color     = primary,
       chapter label align = centering,
       chapter number background-color = white,
       %title
       chapter title font-family = rmfamily,
       chapter title align = centering,
       chapter title background-color = bgsexy!15,
       chapter title before background-color=white}}
\cxset{examplestyle}       
\lorem
\chapter{Typography Example}
\lorem
\chapter{Another Chapter Heading}
\lorem
\end{texexample}


%\cxset{toc chapter = true}
\addtocounter{chapter}{-1}

Examples for other types of chapter openings follow in the rest of the documentation.

\subsection{Blank pages before chapters}

In the standard LaTeX book class when the \texttt{openany} option is not given or in the report class when the openright is given, chapters start at odd-numbered pages. This can cause a blank page to be printed. Some book designers prefer this page to be completely empty, without any headers or footers. This cannot be done with \lstinline{\thispagestyle} as this command will have to be issued on the \textit{previous} page. However by a suitable redefinition of the
\lstinline{\clearpage} this can be done automatically.
\medskip

\begin{teXXX}
\makeatletter
\def\cleardoublepage{\clearpage\if@twoside\ifodd\c@page\else
  \hbox{}
  \vspace*{\fill}
  \begin{center}
    This page left intentionally blank.
  \end{center}
  \vspace{\fill}
  \thispagestyle{empty}
  \newpage
  \if@twocolumn\hbox{}\newpage\fi\fi\fi}
\makeatother
\end{teXXX}


This is achieved easily by setting the following options:
\bigskip

\begin{tcolorbox}
\lstinline{chapter blank page=empty}\par
\lstinline{chapter blank page text=Some text.}\par
\lstinline{chapter blank page=plain}\par
\end{tcolorbox}
\medskip



The last one refers to a \lstinline!\thispagestyle{plain}!.
\cxset{chapter opening = right, chapter format = block}
\chapter{Test}

\cxset{defaults, chapter opening= anywhere}



\section*{Keys for chapter head formatting}

A chapter heading can be considered of being constructed of several parts, the \textit{chapter number}, the chapter name typically \textit{chapter} and the \textit{title}. Predefined keys handle all the elements of formatting. Additional keys are defined to handle other elements such as inclusion of images or producing complicated examples with graphics constructed with \texttt{TikZ} and other similar packages.


\bigskip\bigskip\bigskip\bigskip
\let\oldrefkey\refKey
\let\refKey\texttt
\makeatletter
\long\def\demobox#1#2{%
\par\bigskip\bigskip\bigskip
\begin{tcolorbox}[enhanced,left=0pt, top=0pt, bottom=0pt,width=\textwidth,
  enlarge top initially by=1cm,enlarge bottom finally by=1cm,left skip=1cm,right skip=1cm,
  colframe=white,colback=white,
  colbacktitle=red!30!white,colupper=black!7!white,
  code={\appto\kvtcb@shadow{%
    \path[fill=white,draw=yellow!50!black,dashed,line width=0.4pt]
      ([xshift=-1cm,yshift=-1cm]frame.south west) rectangle
      ([xshift=1cm,yshift=1cm]frame.north east);
     \path[fill=blue!20!white, 
              opacity=0.3, draw=yellow!50!black,solid,line width=1pt]
      ([xshift=-2cm,yshift=-2cm]frame.south west) rectangle
      ([xshift=2cm,yshift=2cm]frame.north east);  
    }},
  finish={
  \draw[thick,<->] ([yshift=-1.3cm]frame.north west)-- node[below]{\texttt{#1 width}}
    ([yshift=-1.3cm]frame.north east);
  \draw[thick,<->] ([xshift=-15mm]frame.north east)-- node[above]{\refKey{#1 height}}
    ([xshift=-15mm]frame.south east);
  \draw[thick,<->] (frame.north)-- node[right]{\refKey{#1 padding-top}} +(0,1);
  \draw[thick,<->] ([yshift=1cm]frame.north)-- node[right]{\refKey{#1 margin-top}} +(0,1);
  \draw[thick,<->] (frame.south)-- node[right, align=left]{\refKey{#1 padding-bottom}}+(0,-1);
  %left padding
  \draw[thick,<->] (frame.west)-- node[below right,align=center]{\refKey{#1 padding-left }}+(-1,0);
  %left margin
  \draw[thick,<->] ([xshift=-1cm,yshift=-0.9cm]frame.west)-- node[below right,xshift=-1,align=left]{\refKey{#1 margin-left }\\\refKey{#1 grow to left by}}+(-1,0);
  %right padding
  \draw[thick,<->] (frame.east)-- node[below left,align=center]{\refKey{#1 padding-right}}+(1,0);
 %right margin
  \draw[thick,<->] ([xshift=1cm,yshift=-0.9cm]frame.east)-- node[below left,xshift=1, align=right]{\refKey{#1 margin-right}\\\refKey{#1 grow to right by}}+(1,0);
 \draw[thick,<->] ([yshift=-2cm]frame.south)-- node[right, align=left]{\refKey{#1 margin-bottom},\\ \refKey{#1 after skip}}+(0,1);
  }
    ]
#2%
%\hrule width0pt height4.5cm depth0pt\relax% \vspace*{4.5cm}% \lipsum[1]
\end{tcolorbox}\par
\bigskip\bigskip\bigskip}
\makeatother

\demobox{chapter}{\scalebox{1.17}{\HHHUGE Chapter}}

The number box is again drawn in a box similar to a chapter with all properties generalized.

\demobox{number}{\scalebox{1.15}{\HHHUGE Thirteen}}



All parameters shown in the diagram can be set using the command \cs{cxset}. The property names follow conventions similar to those of |css|, rather than typical conventions of \tikzname that are more widely known to the programming community. The prefix to these properties (in the example \textit{chapter}) can be thought of
as similar to a |class| or |id| name in |css|.  

\begin{docCommand}{cxset}{\marg{options}}
  Sets options for every following \refEnv{tcolorbox} inside the current \TeX\ group.
  By default, this does not apply to nested boxes, see \Vref{subsec:everybox}.\par
  For example, the colors of the boxes may be defined for the whole document by this:
\begin{dispListing}
\cxset{chapter numbering = Roman,
       chapter number color = blue}
\end{dispListing}
\end{docCommand}

\begin{docKey}[]{chapter padding-top}{=\meta{dimension}}{no default, initial value 0pt}
All padding keys take one argument, which is a dimension. The length is also stored in a register
\cmd{\chapterpaddingtop}. In this chapter it was set at %\the\chapterpaddingtop.
\begin{dispListing}
\cxset{colback=red!5!white,colframe=red!75!black, chapter padding-top=2pt}
\end{dispListing}
\end{docKey}



\begin{docKey}[]{chapter padding-right}{=\meta{dimension}}{no default, initial value 0pt}
All padding keys take one argument, which is a dimension. The length is also stored in a register
\cmd{\chapterpaddingright}.  In this chapter it was set at %\the\chapterpaddingright.
\end{docKey}

\begin{docKey}[]{chapter padding-bottom}{=\meta{dimension}}{no default, initial value 0pt}
All padding keys take one argument, which is a dimension. The length is also stored in a register
\cmd{\chapterpaddingbottom}.  In this chapter it was set at %\the\chapterpaddingbottom.
\end{docKey}

\begin{docKey}[]{chapter padding-left}{=\meta{dimension}}{no default, initial value 0pt}
All padding keys take one argument, which is a dimension. The length is also stored in a register
\cmd{\chapterpaddingleft}.  In this chapter it was set at %\the\chapterpaddingleft.
\end{docKey}

%% margin

\begin{docKey}[]{chapter margin-top}{=\meta{dimension}}{no default, initial value 0pt}
All padding keys take one argument, which is a dimension. The length is also stored in a register
\cmd{\chaptermargintop}. In this chapter it was set at .
\end{docKey}

\begin{docKey}[]{chapter margin-right}{=\meta{dimension}}{no default, initial value 0pt}
All padding keys take one argument, which is a dimension. The length is also stored in a register
\cmd{\chapterpaddingright}.  In this chapter it was set at %\the\chapterpaddingright.
\end{docKey}

\begin{docKey}[]{chapter margin-bottom}{=\meta{dimension}}{no default, initial value 0pt}
All padding keys take one argument, which is a dimension. The length is also stored in a register
\cmd{\chapterpaddingbottom}.  In this chapter it was set at %\the\chapterpaddingbottom.
\end{docKey}

\begin{docKey}[]{chapter margin-left}{=\meta{dimension}}{no default, initial value 0pt}
All padding keys take one argument, which is a dimension. The length is also stored in a register
\cmd{\chaptermarginleft}.  In this chapter it was set at %\the\chaptermarginleft.
\end{docKey}

\subsection{Borders}

Border have three properties \emph{width, color} and \emph{style}. They can set individually for
each side of the box or using the shorter key .

\begin{docKey}[]{chapter border-top-width}{ = \meta{dimension}}{no default, initial value 0pt}
All border keys take one argument, which is a dimension.
\end{docKey}

\begin{docKey}[]{chapter border-right-width}{=\meta{dimension}}{no default, initial value 0pt}
All border keys take one argument, which is a dimension.
\end{docKey}

\begin{docKey}[]{chapter border-bottom-width}{ = \meta{dimension}}{no default, initial value 0pt}
All border keys take one argument, which is a dimension.
\end{docKey}

\begin{docKey}[]{chapter border-left-width}{ = \meta{dimension}}{no default, initial value 0pt}
All border keys take one argument, which is a dimension.
\end{docKey}

\subsubsection{Border Colors}

The colors follow the same pattern for |border-width| and again they can be set individually or using
a shorter key to set all of them in one color. 

\begin{docKey}[]{chapter border-top-color}{=\meta{color name}}{no default, initial value black}
All border keys take one argument, which is a dimension.
\end{docKey}

\begin{docKey}[]{chapter border-right-color}{=\meta{color name}}{no default, initial value black}
All border keys take one argument, which is a dimension.
\end{docKey}

\begin{docKey}[]{chapter border-bottom-color}{=\meta{color name}}{no default, initial value black}
All border keys take one argument, which is a dimension.
\end{docKey}

\begin{docKey}[]{chapter border-left-color}{=\meta{color name}}{no default, initial value black}
This key is stored in \cmd{\chapterborderrightcolor} and the value in this chapter is 
%\ExplSyntaxOn \l_phd_chapter_border_right_color_tl.
\ExplSyntaxOff
\end{docKey}



\subsubsection{Border Styles}

Standard |css|  offers four styles \emph{dotted, solid, double, dashed}. We offer almost an unlimited set of styles.

\begin{docKey}[phd]{chapter border-top-style}{=\meta{style name}}{no default, initial value \texttt{none}}
The |border-style| properties take a value, which can be |solid, double, dotted, dashed, asterisk|.
\end{docKey}

\begin{docKey}[phd]{chapter border-right-style}{=\meta{style name}}{no default, initial value \texttt{none}}
The |border-style| properties take a value, which can be |solid, double, dotted, dashed, asterisk|.
\end{docKey}

\begin{docKey}[]{chapter border-bottom-style}{=\meta{style name}}{no default, initial value \texttt{none}}
The |border-style| properties take a value, which can be |solid, double, dotted, dashed, asterisk|.
\end{docKey}

\begin{docKey}[]{chapter border-left-style}{=\meta{style name}}{no default, initial value \texttt{none}}
The |border-style| properties take a value, which can be |solid, double, dotted, dashed, asterisk|.
\end{docKey}

\begin{docKey}[phd]{chapter border-style}{=\meta{style name}}{no default, initial value \texttt{none}}
This key sets all chapter-border-\meta{top,right,bottom,left}-style to a single value.
\end{docKey}

\subsubsection{Fonts and colors}

All font parameters can be set using individual keys. The naming scheme in general follows |css| conventions.

\begin{docKey}[phd]{chapter color}{=\meta{color name}}{no default, initial value \texttt{black}}
This key sets the color for the \textit{chapter element}. The color name is stored in \cmd{\chaptercolor@cx}.
The value in this chapter is% \makeatletter\texttt{\chaptercolor@cx}\makeatother.
\end{docKey}

\begin{docKey}[phd]{chapter font-size}{=\meta{Huge, Large}}{no default, initial value \texttt{Huge}}
This sets the size for rendering the \textit{chapter element}. Use one of the following predefined values.
Note that you can either use a command i.e, |chapter font-size=|\cmd{\huge} 
or the command name i.e., |chapter font-size=huge|. The latter is the recommended method.
\end{docKey}

\begin{marglist}
\item [tiny] renders as {\tiny tiny}.
\item[footnotesize] renders as {\footnotesize footnotesize}
\item [small] Opens a chapter on an even page
\item [large] Opens a chapter on a right page.
\item [LARGE] Opens a chapter at the point where the \cs{chapter} is typed.
\item [huge] Alias for \marg{anywhere}.
\item [Huge] Opens a chapter at the next page if the page has material that does not exceed a certain portion of
 \cs{textheight}.
 \item[HUGE] renders as {\HUGE HUGE}.
 \item[HHUGE] renders as {\HHUGE HUGE}.
\end{marglist}

\begin{texexample}{Sizing settings}{}
\cxset{
          chapter format = block,
          chapter label font-size= HUGE,
          chapter name = Chapter,
          chapter format=block,
          chapter number font-size= HUGE,
          chapter title font-size=LARGE,
         % 
         % chapter padding-top=0pt,
         % chapter padding-bottom=0pt,
         % title margin-top=3pt,
         %
          }
\chapter{Setting font-sizes}          
\lorem

\end{texexample}


\begin{docKey}{chapter font-family}{ = \meta{sffamily, rmfamily etc.}}{no default, initial value \texttt{sffamily}}
The |font-family| key accepts \latexe conventional family names or |css| names such as |serif| and |non-serif|. The
value is stored in \docAuxCommand{chapter_font_family}, in this chapter it is set as {\ExplSyntaxOn\meaning\chapter_font_family\ExplSyntaxOff}
\end{docKey}


\begin{marglist}
\item [sffamily] The \emph{chapter element} is rendered in the document default \cmd{\sffamily}.
\item [rmfamily] The \emph{chapter element} is rendered in the document default \cmd{\rmfamily}.
\end{marglist}

%% Font weights
\begin{docKey}[]{chapter font-weight}{=\meta{mdseries,bfseries,etc.}}{no default, initial value \texttt{bfseries}}
The |font-weight| key accepts \latexe conventional family names or |css| names such as |bold| and |bfseries|. The
value is stored in \cmd{\chapterfontweight@cx}, in this chapter it is set as 
{\ExplSyntaxOn\expandafter\string\l_phd_chapter_label_fontweight_tl\ExplSyntaxOff}

\begin{texexample}{Setting chapter element font-weights}{fontweight}
\cxset{chapter label font-weight=normal}
\chapter{Font-weight is normal}
\cxset{chapter label font-weight= bfseries}
\chapter{Font-weight is bfseries}
\lorem
\end{texexample}
\end{docKey}


\begin{marglist}
\item [normal] The \emph{chapter element} is rendered in the document default \cmd{\sffamily}.
\item [bold] The \emph{chapter element} is rendered in the document default \cmd{\rmfamily}.
\item[bfseries] Renders as bold.
\item[mdseries] renders as medium series.
\item[light] This is an alias for normal.
\item[\upshape\ttfamily\string\bfseries] The command version of the setting.
\item[\upshape\ttfamily\string\mdseries] The command version of the setting.
\end{marglist}



\begin{docKey}[]{chapter font-shape}{=\meta{itshape,upshape,etc.}}{no default, initial value \texttt{upshape}}
The |font-weight| key accepts \latexe conventional family names or |css| names such as |bold| and |bfseries|. The
value is stored in |chapter_font_weight|, in this chapter it is set as %\ExplSyntaxOn \texttt{\chapter_font_shape}\ExplSyntaxOff.
\end{docKey}

In |css| the |font-shape| is named as |font-style| so we alias it as well. 

%\begin{marglist}
%\item[normal] normal font-style, defaults to |upshape|.
%\item[upshape] normal font-style, defaults to |upshape|. 
%\item[italic] italic shape, renders as {\itshape italic}. For some fonts it might not be available.
%\item[itshape] italic shape, alias of |italic|.
%\item[oblique] oblique font, in \latexe is equivalent to \cmd{\slshape} and renders as {\slshape slshape}, which might be slightly different than {\itshape italic}.
%\end{marglist}


\begin{texexample}{Setting up Fonts}{chapterfonts}
\cxset{   chapter format = block,
          chapter opening=anywhere,
          chapter label font-weight=normal,
          chapter label font-shape=upshape,
          %chapter border-width=0pt,
          %chapter border-style=none,
          %chapter padding-top=0pt,
          chapter label font-size=large,
          chapter number font-size=large,
          chapter number color=black,
          %title font-size=large,
          }
\chapter[fonts]{Test Font Weights}
\lorem
\cxset{chapter label font-shape=itshape}
\chapter{Test Italic Shape}
\lorem
\cxset{chapter label font-shape=normal}
\chapter{Test normal font-shape}
\lorem
\end{texexample}



The specification of font families is somewhat problematic. In the web the |css| allows |font-family|  to hold several font names as a ``fallback” system. If the browser does not support the first font, it tries the next font.

There are two types of font family names:

\begin{description}
\item[family-name] The name of a font-family, like “times”, “courier”, “arial”, etc.
\item[generic-family] The name of a generic family, like “serif”, “sans-serif”, “cursive”, “fantasy”, “monospace”.
\end{description}

Generally in the \tex community leaving the choice of font  open to what is available on a user’s computer is frowned upon. Knuth’s original aim to render consistently documents, irrespective of a user’s computer installation has served the community well, and it is possible three decades later to produce documents identical in all respects to the original. 

If this is still a valid requirement for documents is debatable. Current document processing requirements are focusing more in the preservation of content and document structure rather than form. Typeset documents in soft copy are now widely preserved in |pdf| or |postcript|  formats. One can archive the |.tex| file as well as the |pdf| file.  Back to the provision of keys, a key defined in a 
similar fashion to those of |css| could help, but there is also the issue of slow compilation. If a font cannot be
found, with the current code, it can slow down compilation tremendously. I am leaving the choice where it belongs to the user and the package writer. It makes no harm if a more flexible definition is provided. The user can then decide to only provide one or many fonts. 

This avoids complicated and almost unintelligible commands such as:

\begin{dispListing}
\setkomafont{subsection}{\usefont{T1}{fvm}{m}{n}}
\setkomafont{section}{\usefont{T1}{fvs}{b}{n}\Large}
\end{dispListing}

Here are some examples. 

\begin{texexample}{Serif and non-serif}{ex:fontfamily}
\cxset{chapter label font-family=serif, 
       chapter opening=anywhere}
\chapter{Serif font}
\lorem
\end{texexample}


\section{Floating and Alignment} 

This particular key bothered me, as the term \emph{float} has a different meaning in \latexe. However, to
be consistent with |css| terminology I have yielded to the temptation and included it.

\begin{docKey}[]{chapter float}{=\meta{left,center,right,none}}{no default, initial value \texttt{none}}
Key that controls the horizontal alignment of the \emph{chapter element}. I order for the
element to float, its |display| property must be set to |inline|.
\end{docKey}

%\begin{texexample}{Floating}{chapter:float}
%\cxset{chapter opening=anywhere, chapter float=center}
%\chapter{Centered Chapter}
%\lorem
%\cxset{chapter float=left}
%\chapter{Left Aligned}
%\lorem
%\cxset{chapter float=right}
%\chapter{Right Aligned}
%\lorem
%\end{texexample}


\subsection{The display property}

Both the |css| box model as well as the \TeX layout engine provide numerous complex algorithms in managing the floating of elements. This is normally controlled using two properties |display| and |float|.


\makeatletter

\begin{docKey}[phd]{chapter position}{ = \meta{absolute, relative}}{no default, initial value black}
This positioning directive instructs the engine to position the element at an exact position.
\end{docKey}



\tcbox[nobeforeafter]{$box_1$}\tcbox[nobeforeafter]{$box_2$}\tcbox[nobeforeafter]{$box_3$}\dotfill\tcbox[nobeforeafter]{$box_n$}
\tcbox[before skip=0.2cm, after skip=0pt, width=1cm, enlarge left by=10cm,width=5cm,enhanced,show bounding box]{title before element}
\tcbox[before skip=0pt, width=1cm, enlarge left by=10cm,width=5cm,enhanced,show bounding box]{
\tcbox{tb}\tcbox{title}\tcbox[nobeforeafter, width=1cm,]{tb}}
\tcbox[before skip=0pt, after skip=12pt, width=1cm, enlarge left by=10cm,width=5cm,enhanced,show bounding box]{\emph{title after} element \fbox{some}}
\makeatother

\begin{docKey}[phd]{chapter float}{=\meta{left,center,right,none}}{no default, initial value \texttt{none}}
Key that controls the horizontal alignment of the \emph{chapter element}. I order for the
element to float, its |display| property must be set to |inline|.
\end{docKey}
In document preparation systems or web page development the layout is user generated, i.e., the user is expected to type the html and the |css| will then specify as to how the page will be rendered by the browser. In our case for documents we can specify how we want the headings to look. The layout manager for each element, creates other associated elements, as shown for the title here. This way most layouts can be accomplished with the declarative visual language of the \pkgname{phd} package. 

\subsubsection{In-line elements}

When an element is specified as |inline| the rendering algorithm places the boxes after each other. This is widely used in |chapter elements| to render the number inline with the chapter name.
\medskip
\bgroup

\noindent
\tcbox[nobeforeafter,width=3cm, height=1cm]{Chapter}\tcbox[nobeforeafter]{twelve}
 
When the property is set as |block| the elements are stacked below each other.
\medskip

\tcbox{chapter  display=block   CHAPTER}
\tcbox{number display=block    TWELVE}

The elements can be considered to be enclosed in a \emph{ghost} element. If the property is set to float we
\begin{figure}[htbp]
\makeatletter
\parindent0pt\fboxsep0pt
\fbox{\vbox to 0pt{\hbox to \dimexpr(\textwidth)\relax{{\hss\tcbox[capture=minipage,width=5cm, height=2cm, top=0pt]{\raggedright number display=block\\ number float=right }}%
}%
}%
}\par
\vspace*{2cm}
\makeatother
\end{figure}
signalling to the layout engine that the element must be placed to the right of the page, as shown in the figure. 


\begin{figure}[htbp]
\makeatletter
\parindent0pt\fboxsep0pt
\fbox{\vbox to 0pt{\hbox to \dimexpr(\textwidth+2cm)\relax{{\hss\tcbox[capture=minipage,width=5cm, height=2cm, top=0pt]{\raggedright number display=block\\ \emph{element} float=right }
\tcbox[capture=minipage,width=5cm, height=2cm, top=0pt]{\raggedright \emph{element} display=block\\ \emph{element} float=right }
}%
}%
}%
}\par
\vspace*{2cm}
\makeatother
\end{figure}

\subsection{Absolute positioning}

Absolute positioning mode, will place an element at an exact position on the page. They are more difficult to
achieve and inflexible. 

\begin{docKey}{position}{=\meta{absolute},\meta{relative}}{no default, initial none}{}

\end{docKey}



This positioning directive instructs the engine to position the element at an exact position.


\begin{docKey}[]{chapter float}{=\meta{left,center,right,none}}{no default, initial value \texttt{none}}
Key that controls the horizontal alignment of the \emph{chapter element}. In order for the
element to float, its |display| property must be set to |inline|.
\end{docKey}
\egroup



\section{Number Element Keys}


\subsection*{Keys for numbering}

Chapter numbering follows that of the standard \LaTeX\ classes and is extended to cover some additional cases such as fully spelled out numbers. This of course is only good for languages that use the arabic numeralsn. For other languages numerals in different formats can be added with simple keys and without the need of \pkgname{polyglossia} or \pkgname{babel}. 

Note that the package uses Heiko Oberdiek's package \pkgname{alphalph} to allow for alphabetic numbering that extends beyond the normal 26 letters of the alphabet. Examples for numbering can be seen in \ref{ex:romannumbering}


\begin{docKey}[phd]{number numbering}{= \oarg{alph,Alph,roman,Roman,none,WORDS,words,none}}{default arabic}
Style of numbering.
\end{docKey}

\begin{marglist}
\item [arabic] Despite that the Arabs call what the West calls Arabic numbers Indian numbers, we provide the value arabic to have normal numbers printed.
\item [alph] Lowercase alphabetic numbering.
\item [Alph] Uppercase alphabetic numbering.
\item [roman] Lowercase roman numbering.
\item [Roman] Uppercase roman numbering.
\item [words] The number is in lowercase words.
\item [WORDS] The number is in uppercase literal numerals.
\item [Words] Prints the number in words and capitalizes the first letter, for example the number 21 will be printed as `Twenty One'\footnote{Currently limited to the first hundred numbers}.
\index{chapter design>numbering>words}
\item [ordinals] Prints the number as ordinal.
\item [Ordinals] Prints the number as Ordinal.
\item [ORDINALS] Prinst the number as ORDINALS.
\item [none] This is equivalent to using the star version of the command. It does not print any number and does not increment the chapter counter.\footnote{I am ambivalent about this, perhaps it will be better to increment it, as it can give a more general approach.}

\end{marglist}
\begin{texexample}{Literal Numbering}{ex:literal}
\cxset{chapter numbering=WORDS} 
\chapter{Literal numbering}
\lorem
\cxset{chapter numbering=words,chapter name=chapter}
\chapter{Literal numbering} 
\lorem
\end{texexample}




\cxset{chapter opening=anywhere, chapter numbering=Roman, chapter number font-shape=upshape}
\index{chapter design>numbering>roman}

\begin{texexample}{Setting up keys for numbering}{ex:romannumberingx}
\bgroup
\cxset{chapter format = traditional, 
       chapter name = CHAPTER, 
       chapter numbering = Roman,
       chapter label color = bgsexy}
\chapter{Roman numbering}
\lorem
\egroup
\end{texexample}





To emulate some old books we also offer an ordinal numbering scheme.

\begin{texexample}{Literal Numbering}{ex:ordinals}
\cxset{chapter numbering=ORDINALS} 
\chapter{Ordinals numbering}
\lorem
\cxset{chapter numbering=words,chapter name=chapter}
\chapter{Literal numbering} 
\lorem
\end{texexample}

\cxset{chapter numbering=arabic}

\subsection{Fonts and colors}
\begin{docKey}[phd]{number color}{=\meta{color name}}{no default, initial value \texttt{black}}
This key sets the color for the \textit{number element}. The color name is stored in %\cmd{\numbercolor@cx}.
The value in this chapter is %\makeatletter\texttt{\numbercolor@cx}\makeatother.
\end{docKey}

\begin{docKey}[phd]{number font-size}{=\meta{Huge, Large}}{no default, initial value \texttt{Huge}}
This sets the size for rendering the \textit{number element}. Use one of the predefined values, as described
in the section for the \emph{chapter} element.
Note that you can either use a command i.e, |number font-size=|\cmd{\huge} 
or the command name i.e., |number font-size=huge|. The latter is the recommended method.
\end{docKey}

Letter spacing can be achieved using the soul package in a combination with the key |spaceout|.
The following examples illustrate the usage.

\index[phdkeys]{{\ttfamily phd/chapter design test}}

%\begin{texexample}{Letter Spacing}{ex:letterspacing}
%\cxset{numbering=Roman,
%        % number letter-spacing=soul,
%        % chapter spaceout=soul,
%         %title spaceout=soul,
%         title font-size=Large,
%         title font-family=rmfamily,
%         title font-shape=scshape}
%\chapter{Letter Spacing}
%
%\lorem
%\end{texexample}

\begin{docKey}[phd]{chapter number letter-spacing}{=\meta{none, true, etc.}}{no default, initial value \texttt{none}}.
\end{docKey}

\begin{marglist}
\item[none] Default value no tracking is used and the letters are spaced as per the basic font information.
\item[inherit] Inherits the letter-spacing settings from the \emph{chapter} element.
\item[true] Letter spacing is employed, using the |soul| package.
\item[false] Alias for |none|.
\item[soul] The \pkgname{soul} package is used for letter-spacing.
\item[microtype] The \pkgname{microtype} package is used for letter-spacing. When the microtype package is used more fine tuning of parameters is available.
\end{marglist}

The example that follows, explains how the features offered by the \pkgname{microtype} package can be used to
set different tracking options.

\begin{texexample}{Microtypography}{micro}
\bgroup

\SetTracking
 [ no ligatures = {f},
 spacing = {600*,-100*, },
 outer spacing = {450,250,150},
 outer kerning = {*,*} ]
 { encoding = * }
 { 100 }

{\huge \textls{Chapter Twenty}}

\SetTracking
 [ no ligatures = {f},
 spacing = {600*,-100*, },
 outer spacing = {450,250,150},
 outer kerning = {*,*} ]
 { encoding = * }
 { 200 }
 
{\huge \textls{Chapter Twenty}}

\egroup
\end{texexample}


\hbox{\drawfontbox{\huge \upshape\textls(Chapter Twenty}}

\hbox{\drawfontbox{\huge \upshape\textls{Chapter Twenty}}}


\section{Styling the chapter title}

Similarly to the number and chapter styling keys exist for styling the chapter title. We summarize the available standard keys below:

\index{chapter design!labels!letter spacing}
\begin{texexample}{Styling the Title}{ex:title} 
\cxset{chapter numbering=arabic, chapter title font-shape=itshape}
\chapter{Chapter title}
\lorem
\end{texexample}


\begin{docKey}[phd]{chapter title font-family}{=\marg{family}}{no default, initial inherit document font}
Selects a predefined font family
\end{docKey}

\begin{texexample}{Title element font styling}{}
\cxset{chapter title font-family=sffamily}
\chapter{Title font family settings}
\lorem
\cxset{chapter title font-shape=itshape}
\chapter{Title font-style settings}
\lorem
\end{texexample}


\begin{docKey}[phd]{chapter title font-weight}{ = \marg{\cs{bfseries},\cs{normalseries}}} {}
Font weight.
\end{docKey}

\begin{docKey}[phd]{chapter title font-size}{= \marg{large, Large, huge, Huge, HUGE, HHuge}}{}
Font sizing commands or their names. Both \docAuxCommand{\HUGE} and HUGE are allowed to be used as values for the key.
\end{docKey}

\begin{docKey}[phd]{chapter title color} { = \marg{color}} {}
The color of the chapter title letters. This takes any predefined color name. 
\end{docKey}


\begin{docKey}[phd]{chapter title spaceout}{ = \marg{soul,none}} {no default, initial = none}
 This key will space out the title. 
\end{docKey}

\begin{texexample}{Title element spacing}{}
\cxset{chapter name=none,
       chapter numbering=none,
       chapter title font-size=Large,
       chapter title color=black,
       chapter title width=0.6\textwidth,
       %title spaceout=soul,
         }
\chapter{The Prehistoric Period in South-East Asia: 2300 BC--AD 400}        
\lorem 
    
\end{texexample}
\cxset{defaults}


\subsection*{Adding content before and after the title element}

Like all the other elements, the title element can be decorated with additional content,
before and after the text. There are two different forms. 

\begin{docKey}[phd]{title before}{=\marg{code}}{default none}
Contents before the title (vertical material)
\end{docKey}

\begin{docKey}[phd]{title after}{=\marg{code}}{default none}
Contents after the title (vertical material)
\end{docKey}

\begin{docKey}[phd]{title content before}{=\marg{code}}{default none}
Contents before the title (horizontal material)
\end{docKey}

\begin{docKey}[phd]{title content after}{=\marg{code}}{default none}
Contents after the title (horizontal material)
\end{docKey}

The difference between the two type of settings, consider the following situation. Assume you have a title that has a rule at the top and bottom and the text is surrounded by two ornaments. The surrounding ornaments will be inserted using the |title before content|, and the rules using the |title before| form. The |title before| is a full fledged element on its own. 

%{
%\hrule
%\centering
%*** Introduction ***
%\par
%\hrule
%}
%
%{
%\MakePercentComment
%\startlineat{200}
%\lstinputlisting{./styles/style13.tex}
%\MakePercentIgnore
%}



 
\begin{docKey}{/phd/ chapter title before skip}{= \marg{soul,none}}{}
Before title string skip.
\end{docKey}

\begin{docKey}{/phd/ chapter title after skip}{ = \marg{soul,none} }{}
After title string skip.
\end{docKey}

\lorem 
%
%\begin{texexample}{letter spacing the chapter title block}{ex:title3}
%
%\cxset{chapter spaceout=none,
%         numbering=arabic}
%         
%\chapter{Chapter Title Styling}
%\end{texexample}
%
%\end{document}



\cxset{chapter opening=right}
\section{Table of Contents}\index{table of contents!key settings}

Traditionally a chapter will be added to the Table of Contents if the \cs{chapter} command is issued. The starred version will not produce a number and will not add a contents line. Since we have adopted an approach where we use a key value interface we can dispense with the starred version of the command, by setting the \option{chapter toc} option to false. For example if we want to define a command for a ``Foreward'' or ``Epiloque'' without wishing them to be added to the table of contents we can use the following setting.\index{Foreward>definitions}\index{Epilogue>definitions}



\begin{texexample}{changing the chapter label name}{}
\cxset{chapter name=Chapteris, chapter numbering=arabic,}
\chapter{Foreward}
\lorem
\end{texexample}

Note that the key \option{numbering=none} still has to be set.


Please note that when \textbf{numbering=none} the chapter number is not available anymore and yo may have to reset it if required again. Although this might be seen as rather cumbersome than simply using \cs{chapter*} the advantage is consistency in the user interface and the use of appropriate semantic definitions for all sectioning commands thus achieving a bit more separation of context from style.


%\cxset{chapter toc=true}

\section{Defining styles}

Named styles can be defined using the standard \textsc{PGF} conventions. To define a style for the forward above we can use:

\begin{texexample}{}{}
\cxset{foreward/.style={chapter numbering=none,
          chapter name=none,
          chapter title font-size= Large,
          chapter title font-family= sffamily,
          chapter numbering=none}}
\cxset{foreward}
\chapter{Foreward.}
\lorem
\end{texexample}



\cxset{chapter numbering=arabic}
\section{Creating semantic names for commands and environments}

To keep our search for semantic commands and true separation of contents it is prudent to define some macros for typesetting the  `foreward' section.

\bgroup
\begin{texexample}{defining a \textit{Foreward} macro.}{}
\begin{lstlisting}
\cxset{foreward/.style={chapter toc=false,
          name=none,
          title font-size = Large,
          title font-family = sffamily,
          numbering=none}}
\newcommand\forewardname{foreward}
\expandafter\newenvironment\expandafter{\forewardname}{%
\cxset{foreward}\chapter{Foreward}}%
{}
\begin{foreward}
\lorem
\end{foreward}
\end{lstlisting}
\end{texexample}
\egroup

Notice the use of a new command \cmd{\forewardname} to allow for internationlization using Babel or other methods. One is tempted to let the English name, but a better approach perhaps is to define both.

\makeatletter



    
\@specialtrue
\cxset{steward,
  numbering=arabic,
  custom=stewart,
  offsety=0cm,
  image=hine03,
  texti={When Lamport designed the original \LaTeX\ sectioning commands, limitations of computer power forced him to restrict the abstraction of complicated chapter layouts. With current tools available improvements are much easier to program.},
%
  textii={In this chapter we discuss a method that allows the production of fancy sectionr headings and formatting, based on a set of key values. Central  to this process is the separation of content from presentation.
We also discuss the basic formatting tools that are available and how one can modify them to mould new book designs.
 }
}



\raggedbottom

\chapter{Lower Level Headings}
\@specialfalse

\section{Introduction}

Good book design dictates that sectioning styles follow that of the general book design and theme. An academic publication for example might have chapters and section numbered in arabic numerals, whereas a high school textbook might have sections marked in colored boxes.

Similarly to the chapter key value interface, the package offers a key value interface to adjust sectioning command parameters.



\cxset{section beforeskip={10pt},
      section indent=0pt}
\cxset{section afterskip={10pt}}
\renewsection

\section{Section styling}

In a similar fashion to the chapter commands the following keys are provided.

\subsection{Fonts and numerals}

Font and numeral keys are shown below.
\medskip

  \keyval{section font-size}{\marg{cmd}}{Font size command such as \cs{large.}}
  \keyval{section font-weight}{\marg{cmd}}{Font weight command such as \cs{bfseries.}}
  \keyval{section font-family}{\marg{cmd}}{Font family command such as \cs{sffamily.}}
  \keyval{section font-shape}{\marg{cmd}}{Font shape command such as \cs{itshape}}
  \keyval{section color}{\marg{color}}{Color of section.}
  \keyval{section numbering}{\marg{arabic|roman|Roman|alph|Alph|words|WORDS}}{Section number style.}
  \begin{marglist}
  \item [arabic] Typesers the section number in arabic numerals.
  \item [roman] Typesets the section number in lowercase roman numerals.
  \item [Roman] Typesets the section number in uppercase roman numerals.
  \item [alph] Typesets the section number in lowercase alphabetic numbering.
  \item [Alph] Typesets the section number in uppercase alphabetic numerals.
  \item [words] Typesets the numbers in words (lowercase).
  \item [WORDS] Typesets the number in words (uppercase).
  \end{marglist}

\subsection{Skip and indentation commands}

The keys for indentaion and above and below skips are shown below.
\medskip

\keyval{section beforeskip}{}{}
\keyval{section afterskip}{}{}
\keyval{section indent}{\marg{dim}}{Indentation from margin as per standard LaTeX class definitions.}
\keyval{section spaceout}{}{}
\begin{marglist}
 \item[soul]
 \item[none]
\end{marglist}

\subsection{align}

\keyval{section align}{\marg{cmd}}{One of the alignment commands centering, ragged right, raggedleft}

\subsection{Hooks}

Hooks for adding material are shown in the following sketch.
\medskip

\fbox{aboveskip}

\fbox{indent} \fbox{number}\fbox{hook}\fbox{title}

\fbox{belowskip}

%\lipsum

\section{Example usage}

\cxset{
 chapter toc=false,
 name=CHAPTER,
 numbering=arabic,
 number font-size=\huge,
 number font-family=\sffamily,
 number font-weight=\bfseries,
 number before=,
 number dot=,
 number after=\hspace{1em},
 number position=rightname,
 chapter opening=anywhere,
 chapter font-family=\sffamily,
 chapter font-weight=\bfseries,
 chapter font-size=\huge,
 chapter before={\vspace*{0.1\textheight}\hfill},
 chapter after={\hfill\hfill\vskip0pt\thinrule\par},
 chapter color={black!90},
 number color=\color{black!90},
 title beforeskip={\vspace*{30pt}},
 title afterskip={\vspace*{30pt}\par},
 title before={\hfill},
 title after={\hfill\hfill},
 title font-family=\sffamily,
 title font-color=\color{black!90},
 title font-weight=\bfseries,
 title font-size=\huge,
%%%%%%%%%% Sections
 section font-size=\LARGE,
 section font-weight=\normalfont,
 section font-family=\sffamily,
 section align=\centering,
 section numbering=arabic,
 section indent=0em,
 section align=\centering,
 section beforeskip=20pt,
 section afterskip=10pt,
 section spaceout=soul,
 section font-shape=\itshape,
}
\cxset{book/.style={
 section numbering=arabic,
 section font-size=\Large,
 section font-weight=\bfseries,
 section font-family=\rmfamily,
 section font-shape=\normalfont,
 section align=\raggedright,
 %section numbering custom=\color{gray}{Section} (\thechapter-\@arabic\c@section),
 subsection font-size=\large
 section indent=0em,
 section beforeskip=-3.5ex \@plus -1ex\@minus -0.2ex,
 section afterskip=2.3ex\@plus.2ex,
 subsection beforeskip=-3.5ex \@plus -1ex\@minus -0.2ex,
 subsection afterskip= 1.5ex \@plus .2ex,
}}


\begin{example}{Adjusting section parameters}{}
\cxset{ section font-size=\LARGE,
 section font-weight=\normalfont,
 section font-family=\sffamily,
 section align=\centering,
 section numbering=(roman),
 section indent=0em,
 section align=\centering,
 section beforeskip=20pt,
 section afterskip=10pt,}
\chapter{A First Look at the Sectioning Keys}
\section{First section}
\lorem
\end{example}

One notable thing to keep in mind is that the numbering of the chapter is independent of that for the section, so if you need to have strange combinations rather define a section numbering custom.\index{section formatting!vertical space}

\cxset{section numbering=arabic}
\subsection{Adjusting vertical spaces}

Perhaps the most important issues we need to consider is the adjusting of vertical spaces; example~\ref{ex:latex}, that follows illustrates settings from the Octavo class and compare them with those of standard the \LaTeXe\ book class. The Octavo class through settings that are based on baselineskip fractions and multiples endeavours to achieve a grid layout. The class also tones down the `loudness' of some of the headings compared to those of the book class.


\cxset{octavo/.style={
 section font-size=\large,
 section font-weight=\normalfont,
 section font-family=\rmfamily,
 section font-shape=\scshape,
 section indent=0em,
 section align=\centering,
 section beforeskip=-1.666\baselineskip\@minus -2\p@,
 section afterskip=0.835\baselineskip \@minus 2\p@,
 subsection numbering=none,
 subsection font-family=\rmfamily,
 subsection font-size=\normalfont,
 subsection font-shape=\scshape,
 subsection font-weight=\normalfont,
 subsection indent=1em,
 subsection align=\raggedright,
 subsection beforeskip=-0.666\baselineskip\@minus -2\p@,
 subsection afterskip=0.333\baselineskip \@minus 2\p@
 }}




\cxset{book/.style={
 section numbering=arabic,
 section font-size=\Large,
 section font-weight=\bfseries,
 section font-family=\rmfamily,
 section font-shape=\normalfont,
 section align=\raggedright,
 %section numbering custom=\color{gray}{Section} (\thechapter-\@arabic\c@section),
 subsection font-size=\large,
 section indent=0em,
 section beforeskip=-3.5ex \@plus -1ex\@minus -0.2ex,
 section afterskip=2.3ex\@plus.2ex,
 subsection font-size=\large,
 subsection font-weight=\bfseries,
 subsection numbering=arabic,
 subsection indent=0pt,
 subsection beforeskip=-3.5ex \@plus -1ex\@minus -0.2ex,
 subsection afterskip= 1.5ex \@plus .2ex,
}}

\cxset{octavo headings/.style={%
 section numbering=none,section font-size=\large,section font-weight=\normalfont,
 section font-family=\rmfamily, section font-shape=\scshape,
 section indent=0em, section align=\centering, section beforeskip=-1.666\baselineskip\@minus -2\p@,
 section afterskip=0.835\baselineskip \@minus 2\p@, subsection numbering=none,
 subsection font-family=\rmfamily, subsection font-size=\normalfont, subsection font-shape=\scshape,
 subsection font-weight=\normalfont, subsection indent=1em, subsection align=\raggedright,
 subsection beforeskip=-0.666\baselineskip\@minus -2\p@,
 subsection afterskip=0.333\baselineskip \@minus 2\p@,
 subsubsection numbering=none,
 subsubsection font-family=\rmfamily,
 subsubsection font-size=\normalfont,
 subsubsection font-shape=\itshape,
 subsubsection font-weight=\normalfont,
 subsubsection indent=1em,
 subsubsection align=\raggedright,
 subsubsection beforeskip=-0.666\baselineskip\@minus -2\p@,
 subsubsection afterskip=0.333\baselineskip \@minus 2\p@,
 paragraph numbering=none,
 paragraph font-family=\rmfamily,
 paragraph font-size=\normalfont,
 paragraph font-shape=\normalfont,
 paragraph font-weight=\normalfont,
 paragraph indent=-1em,
 paragraph align=\raggedright,
 paragraph beforeskip=\z@,
 paragraph afterskip=0\p@,
% subparagraph numbering=none,
% subparagraph font-family=\rmfamily,
% subparagraph font-size=\normalfont,
% subparagraph font-shape=\normalfont,
% subparagraph font-weight=\normalfont,
% subparagraph indent=0em,
% subparagraph align=\raggedright,
% subparagraph beforeskip=\z@,
% subparagraph afterskip=0\p@,
}}
\cxset{octavo headings}
\renewsection\renewsubsection\renewsubsubsection\renewparagraph

\begin{example}{Octavo class headings, settings}{}
\cxset{octavo headings/.style={%
 section numbering=none,section font-size=\large,section font-weight=\normalfont,
 section font-family=\rmfamily, section font-shape=\scshape,
 section indent=0em, section align=\centering, section beforeskip=-1.666\baselineskip\@minus -2\p@,
 section afterskip=0.835\baselineskip \@minus 2\p@, subsection numbering=none,
 subsection font-family=\rmfamily, subsection font-size=\normalfont, subsection font-shape=\scshape,
 subsection font-weight=\normalfont, subsection indent=1em, subsection align=\raggedright,
 subsection beforeskip=-0.666\baselineskip\@minus -2\p@,
 subsection afterskip=0.333\baselineskip \@minus 2\p@,
 subsubsection numbering=none,
 subsubsection font-family=\rmfamily,
 subsubsection font-size=\normalfont,
 subsubsection font-shape=\itshape,
 subsubsection font-weight=\normalfont,
 subsubsection indent=1em,
 subsubsection align=\raggedright,
 subsubsection beforeskip=-0.666\baselineskip\@minus -2\p@,
 subsubsection afterskip=0.333\baselineskip \@minus 2\p@,
 paragraph numbering=none,
 paragraph font-family=\rmfamily,
 paragraph font-size=\normalfont,
 paragraph font-shape=\normalfont,
 paragraph font-weight=\normalfont,
 paragraph indent=-1em,
 paragraph align=\raggedright,
 paragraph beforeskip=\z@,
 paragraph afterskip=0\p@,}}

\cxset{octavo headings}
\renewsection\renewsubsection\renewsubsubsection\renewparagraph
\section{Octavo Class Heading}
\lorem
\subsection{Octavo subsection}
This is some text short text\par
\subsubsection{Octavo sub-subsection}
\lorem
\paragraph{paragraph heading} This is some short text.
\end{example}

\begin{example}{}{}
\cxset{octavo}
\section{Octavo Class Heading}
\lorem
\subsection{Octavo subsection}
\lorem
\subsubsection{Octavo sub-subsection}
\lorem
\paragraph{paragraph heading} This is some short text.
\lorem
\paragraph{paragraph heading} This is some short text.
\lorem
\end{example}



\begin{example}{\LaTeXe\ book class headings settings}{ex:latex}
\cxset{book/.style={
 section numbering=arabic,
 section font-size=\Large,
 section font-weight=\bfseries,
 section font-family=\rmfamily,
 section font-shape=\normalfont,
 section align=\raggedright,
 %section numbering custom=\color{gray}{Section} (\thechapter-\@arabic\c@section),
 subsection font-size=\large,
 section indent=0em,
 section beforeskip=-3.5ex \@plus -1ex\@minus -0.2ex,
 section afterskip=2.3ex\@plus.2ex,
 subsection font-size=\large,
 subsection font-shape=\normalfont,
 subsection font-weight=\bfseries,
 subsection numbering=arabic,
 subsection indent=0pt,
 subsection beforeskip=-3.5ex \@plus -1ex\@minus -0.2ex,
 subsection afterskip= 1.5ex \@plus .2ex,
}}
\cxset{book}
\renewsubsection
\section{LaTeX Book  Class Heading}
\lorem
\subsection{A subsection}
\lorem
\end{example}

\section{Grid example}

One problem sometimes is that the sectioning commands create problems with grid layouts. Example~\ref{ex:grid} shows example settings.

\begin{example}{Section styles from the grid package}{ex:grid}
\cxset{grid/.style={
 section numbering=arabic,
 section font-size=\normalsize,
 section font-weight=\bfseries\mathversion{bold},
 section font-family=\rmfamily,
 section font-shape=\normalfont\bfseries\mathversion{bold},
 section beforeskip=-.999\baselineskip,
 section afterskip=0.001\baselineskip,
 section align=\raggedright,
 %section numbering custom=\color{gray}{Section} (\thechapter-\@arabic\c@section),
 subsection font-size=\normalsize,
 section indent=0em,
% section beforeskip=-3.5ex \@plus -1ex\@minus -0.2ex,
 %section afterskip=2.3ex\@plus.2ex,
 subsection font-shape=,
 subsection font-weight=\bfseries\mathversion{bold},
 subsection numbering=arabic,
 subsection indent=0pt,
 subsection beforeskip=\baselineskip,
 subsection afterskip= -.35\baselineskip,
% subsub section
 subsubsection font-shape=\itshape,
 subsubsection font-weight=\bfseries\mathversion{bold},
 subsubsection numbering=numeric,
 subsubsection indent=0pt,
 subsubsection beforeskip=\baselineskip,
 subsubsection afterskip= -.35\baselineskip,
}}
\cxset{grid}
\renewsubsection
\begin{multicols}{2}
\section{Grid  Class Heading}
\lorem
\subsection{Grid  subsection.}
\lorem
\subsubsection{A subsection grid.}
\lorem
\subsubsection{Another subsection grid.}
\lorem
\end{multicols}
\end{example}



The key \option{\bfseries section numbering custom}=\marg{code} is quite powerfull and can be used to define any type of section number style. Just remember that the numbering so far depends on two counters, the c@chapter and c@section. What the section numbering does, it redefines the macro \cs{thesection} to the new definition provided as argument for the key.

Although the temptation to define a lot of key combinations one would rather define new styles as a more user friendly approach.

\cxset{section numbering=arabic, section align=\raggedright, section font-shape=\upshape, section font-family=\rmfamily}
\section{Handling Other Section Levels}

Other sectioning commands such as \cs{subsubsection}, \cs{paragraph} and \cs{subparagraph} have equivalent keys. Examples can be found in the chapters that follow for specific styles.

\section{Technical discussion}

The standard LaTeX classes, book report and article have sections showing dot leaders, whereas in the article class the sections are shown without the dotted lines, as the l@section macro is redefined for articles.

\index{macros!\textbackslash @seccntformat}

\subsection{Indexing of Lower Section Headings}
\LaTeXe\ offers two pathways in redefining section commands, the first one is @startsection and the second is \cs{@seccntformat} \index{sectioning macros}. It also uses the macro \cs{secdef} to create the starred and unstarred versions of the sectioning commands.

\begin{tcolorbox}{}
\begin{lstlisting}
% \begin{macro}{\l@section}
%    In the article document class the entry in the table of contents
%    for sections looks much like the chapter entries for the report
%    and book document classes.
%
%    First we make sure that if a pagebreak should occur, it occurs
%    \emph{before} this entry. Also a little whitespace is added and a
%    group begun to keep changes local.
% \changes{v1.0h}{1993/12/18}{Replaced -\cs{@secpenalty} by
%    \cs{@secpenalty}.  ASAJ.}
% \changes{v1.2i}{1994/04/28}{Don't print a toc line when the tocdepth
%    counter is less than 1.}
% \changes{v1.4a}{1998/10/12}{we should use \cs{@tocrmarg}; see PR/2881.}
%    \begin{macrocode}
%<*article>
\newcommand*\l@section[2]{%
  \ifnum \c@tocdepth >\z@
    \addpenalty\@secpenalty
    \addvspace{1.0em \@plus\p@}%
%    \end{macrocode}
%
%    The macro |\numberline| requires that the width of the box that
%    holds the part number is stored in \LaTeX's scratch register
%    |\@tempdima|. Therefore we put it there. We begin a group, and
%    change some of the paragraph parameters (see also the remark at
%    \cs{l@part} regarding \cs{rightskip}).
%    \begin{macrocode}
    \setlength\@tempdima{1.5em}%
    \begingroup
      \parindent \z@ \rightskip \@pnumwidth
      \parfillskip -\@pnumwidth
%    \end{macrocode}
%    Then we leave vertical mode and switch to a bold font.
%    \begin{macrocode}
      \leavevmode \bfseries
%    \end{macrocode}
%    Because we do not use |\numberline| here, we have do some fine
%    tuning `by hand', before we can set the entry. We discourage but
%    not disallow a pagebreak immediately after a section entry.
%    \begin{macrocode}
      \advance\leftskip\@tempdima
      \hskip -\leftskip
      #1\nobreak\hfil \nobreak\hb@xt@\@pnumwidth{\hss #2}\par
    \endgroup
  \fi}
%</article>
\end{lstlisting}
\end{tcolorbox}

As you can see the dot leaders are not present in the above definition. Although we can get rid of dot leaders in other section by redefining them, it is not as easy to add them back.

As our aim is to be able to have all the classes used a common denominator we can define a command as follows (using book as a base)

\begin{tcolorbox}{}
\begin{lstlisting}
\def\articlesection{
\newcommand*\l@section[2]{%
  \ifnum \c@tocdepth >\z@
    \addpenalty\@secpenalty
    \addvspace{1.0em \@plus\p@}%
    \setlength\@tempdima{1.5em}%
    \begingroup
      \parindent \z@ \rightskip \@pnumwidth
      \parfillskip -\@pnumwidth
      \leavevmode \bfseries
      \advance\leftskip\@tempdima
      \hskip -\leftskip
      #1\nobreak\hfil \nobreak\hb@xt@\@pnumwidth{\hss #2}\par
    \endgroup
  \fi}
}
\end{lstlisting}
\end{tcolorbox}

%\articlesection

The \cs{@starredsection} macro is one of those locomotive type of commands. It takes 7 required arguments and 2 optional ones and hidden within it are two booleans. The full set looks like this:

\cs{@startsection} \marg{name} \marg{level} \marg{indent} \marg{beforeskip} \marg{afterskip} \marg{style}[*]
  [\marg{altheading}]\marg{heading}.

\begin{marglist}
\item[name] The name of the level command.
\item [level] A number denoting the depth of the section, chapter=1, section=2, etc. A section number will be printed only if \marg{level} is equal or smaller than the value of \textit{secnumdepth}
\item[indent] The indentation of the heading from the left margin.
\item[beforeskip]  The absolute value of this argument is the skip to leave above the heading. If it is negative, then the paragraph indent of the text following the heading is suppressed.
\item [afterskip] If positive, it is the skip to leave below the heading, else it is the skip to the right of a run-in heading.
\item [style] Sets the style of the heading.
\item[\textup{[*]}] When this is missing the heading is numbered and the corresponding counter is incremented.
\item[\textup{[\textit{altheading}]}] Gives an alternative heading to use in the table of contents and in the running heads. This should be present when the * form is used.
\item[heading] The heading of the new section.
\end{marglist}

\begin{example}{Example formatting run-in section}{}
\makeatletter
\bgroup
\renewcommand\section{%
    \@startsection{section}%
    {1}%
    {0em}%
    {-0.8em}%
    {-0.5em}%
    {\large\normalfont\scshape}}
\makeatother
\section[]{test}
\lorem
\egroup
\end{example}

Note we run the example in a group so that we will not influence the formatting of this document.

As mentioned earlier there is an additional way to introduce formatting for sections and this is using the command \cs{@seccntformat}, which is responsible for typesetting the counter part of a section title. The default definition of the command typesets the \cs{the} representation of the section counter.

\begin{example}{}{}
\bgroup
\renewcommand\section{%
    \@startsection{section}%
    {1}%
    {0em}%
    {-0.8em}%
    {-0.5em}%
    {\large\normalfont\scshape}}
\renewcommand\@seccntformat[1]{\fbox
{\csname the#1\endcsname}\hspace{0.5em}}
\makeatother
\section[]{test}\label{sec:ok}
\lorem

See section \ref{sec:ok}.
\egroup
\end{example}

The definition of \cs{@seccntformat} applies to all headings
defined with the \cs{@startsection} command (which is described in the next
section). Therefore, if you wish to use different definitions of \cs{@seccntformat}
for different headings, you must put the appropriate code into every heading
definition.

\begin{tcolorbox}
\begin{lstlisting}
\def\@seccntformat##1{\csname the##1\endcsname{}}
\end{lstlisting}
\end{tcolorbox}

\section{Custom headings}

It is also possible to define section headings without resorting to any of the above. To do this.

\begin{tcolorbox}
\begin{lstlisting}
\newcommand\part{\secdef\cmda\cmdb}
\end{lstlisting}
\end{tcolorbox}

the part and chapter and sometimes appendix are defined this way, but nothing stops us from doing the same for other sections. A generic section command can be defined as follows:

\begin{example}{}{}
\bgroup
\renewcommand\section[2] [?]{% % Complex form:
\refstepcounter{section}% % step counter/ set label
\addcontentsline{toc}{appendix}% % generate toe entry
{\protect\numberline{section-\thesection}#1}%
{\raggedright\large\bfseries section %\appendixname\ % typeset the title
\thesection\par \centering#2\par}% % and number
\sectionmark{#1}% % add to running header
\@afterheading % prepare indentation handling
%\addvspace{\baselineskip}
}
\section{Test}
\lorem
\egroup
\end{example}

Many other strategies can also be implemented that are perhaps easier to grasp.

\begin{example}{}{}
\bgroup
\def\strut{\vrule height12pt depth1pt width0pt}
\renewcommand\section[2] []{% % Complex form:
\refstepcounter{section}% % step counter/ set label
\addcontentsline{toc}{section}% % generate toc entry
{\protect\numberline{\thesection} }%
{\raggedright\large\bfseries\scshape %
\parbox[b]{\dimexpr(\linewidth-0.5\columnsep)}{\colorbox{brown!80}%
{{\vbox{\strut\raise2pt\hbox{#2}}}}}}\vskip0pt% % and number
\sectionmark{#1}% % add to running header
\@afterheading % prepare indentation handling
\vspace{\dimexpr\baselineskip+6pt}%must have a parameter
}
\chapter{Fossil Insects}
\begin{multicols*}{2}\raggedcolumns
\section[Insect Fossilization]{\raggedright \thinspace Insect Fossilization}
\lipsum[1]
\end{multicols*}
\egroup
\end{example}
% To answer http://tex.stackexchange.com/questions/52998/change-title-to-small-caps-but-not-in-toc

Of course some work is needed to center the text properly in the middle of the colour box. For all practical purposes it is lining up as per the sample.

In Chapter we discussed a forward, but this may not apply if there are no chapters or we need to treat these as sections, the example \ref{ex:forwardsection} shows such a method.

\begin{example}{Defining a Foreward Section}{ex:forwardsection}

\newcommand\prematter@sp[1]{% % Complex form:
%\refstepcounter{section}% % step counter/ set label
\addcontentsline{toc}{section}% % generate toe entry
{\protect\numberline{}\textsc{#1}}%
\sectionmark{#1}% % add to running header
{\LARGE\centering\normalfont\sffamily\colorbox{brown!80}{ \textsc{#1}}\par}%
\@afterheading % prepare indentation handling
\addvspace{\baselineskip}
\@afterindentfalse
}

\newenvironment{prematter}[1]{%
   \prematter@sp{#1}}
{}
\begin{multicols}{2}
\label{theok}
\begin{prematter}{Foreward}
\lipsum[1]
\end{prematter}\ref{theok}
\end{multicols}
\end{example}

\section{underlining}

I am aware that some people have no choice but have some sections underlined as dictated by archaic regulations in some establishments for thesis submission. If nobody is forcing you to underline it is best to avoid it. We use Donald Arsenau's ulem package to achieve underlining.

   \cxset{style87/.style={
 chapter opening=any,
 name=Chapter,
 % positioning and float - inline is 0
 %  float right is 2
 number display=block,
 number float=right,
 number shape=starburst,
 numbering=Words,
 number spaceout=none,
 number font-size=huge,
 number font-weight=bold,
 number font-family=rmfamily,
 number font-shape=normal,
 number before=,
 number display=inline,
 number float=none,
% 
 number border-top-width=0pt,
 number border-right-width=0pt,
 number border-bottom-width=0pt,
 number border-left-width=0pt,
 number border-width=0pt,
%  
 number padding-left=0em,
 number padding-right=0.5em,
 number padding-top=0em,
 number padding-bottom=0pt,
  %number margin-top=, to do
 %number margin-left=0pt,  to create
 %
 number after=\par,
 number dot=,
 number position=rightname,
 number color=sweet,
 number background-color=white,
 %chapter name
 chapter display=block,
 chapter float=left,
 chapter shape=ellipse,
 chapter color=black,
 chapter background-color=sweet,
 chapter font-size= Huge,
 chapter font-weight=bfseries,
 chapter font-family=itshape,
 chapter before=,
 chapter spaceout=none,
 chapter after=,
 chapter margin-left=0cm,
 chapter margin-top=0pt,
 %
 chapter border-width=2pt,
 chapter border-top-width=1pt,
 chapter border-right-width=1pt,
 chapter border-bottom-width=1pt,
 chapter border-left-width=4pt,
% 
 chapter padding-left=20pt,
 chapter padding-right=20pt,
 chapter padding-top=20pt,
 chapter padding-bottom=10pt,
  %chapter title
 title font-family=rmfamily,
 title font-color=black!80,
 title font-weight=bfseries,
 title font-size=huge,
 chapter title align=none,
 title margin-left=1cm,
 title margin bottom=1.3cm,
 title margin top=30pt,
 % title borders
 title border-width=0pt,
 title padding=0pt,
 title border-color=black!80,
% title border-top-color=spot!50,
% title border-top-width=20pt,
 title border-left-color=black!80,
 title border-left-width=2pt,
 title border-color=black!80,
 title padding-top=10pt,
 title padding-bottom=10pt,
 title padding-left=10pt,
 title padding-right=0pt,
% title border-right-color=spot!50,
% title border-right-width=20pt,
% title border-bottom-color=spot!50,
% title border-bottom-width=20pt,
 %
 chapter title align=left,
 chapter title text-align=left,
 chapter title width=0.8\textwidth,
 title before=,
 title after=,
 title display=block,
 title beforeskip=12pt,
 title afterskip=12pt,
 author block=false,
 section font-family=rmfamily,
 section font-size=LARGE,
 section font-weight=bfseries,
 section indent=0pt,
  section font-weight=mdseries,
 section align=left,
 subsubsection font-family=tiresias,
 subsubsection font-shape=upshape,
 subsubsection font-weight=mdseries,
 subsubsection align=flushleft,
 epigraph width=\dimexpr(\textwidth-2cm)\relax,
 epigraph align=center,
 epigraph text align=center,
 epigraph rule width=0pt,
 header style=plain}}
 
\cxset{style87}
\renewsection\renewsubsection\renewsubsubsection

\makeatletter
\cxset{enumerate numberingi/.is choice,
  enumerate numberingi/.code={\renewcommand\theenumi {\csname#1\endcsname{enumi}}},
  enumerate numberingii/.code={\renewcommand\theenumii {\csname#1\endcsname{enumii}}},
  enumerate numberingiii/.code={\renewcommand\theenumiii {\csname#1\endcsname{enumiii}}},
  enumerate numberingiv/.code={\renewcommand\theenumiv {\csname#1\endcsname{enumiv}}},
  enumerate labeli punctuation/.store in=\enumeratepunctuationi@cx,
  enumerate labeli/.is choice,
  enumerate labeli/brackets/.code={\renewcommand\labelenumi{(\theenumi\enumeratepunctuationi@cx)}},
  enumerate labeli/square brackets/.code={\renewcommand\labelenumi{[\theenumi\enumeratepunctuationi@cx]}},
  enumerate labeli/right bracket/.code={\renewcommand\labelenumi{\theenumi\enumeratepunctuationi@cx)}},
  enumerate label left/.store in=\enumeratelabelleft@cx,
  enumerate label right/.code=\renewcommand\labelenumi{\enumeratelabelleft@cx\theenumi\enumeratepunctuationi@cx#1},
  enumerate leftmargini/.code={\setlength\leftmargini{#1}},
  enumerate leftmarginii/.code={\setlength\leftmarginii{#1}},
  enumerate leftmarginiii/.code={\setlength\leftmarginiii{#1}},
  enumerate leftmarginiv/.code={\setlength\leftmarginiv{#1}},
  listi topsep/.store in=\listitopsep@cx,
  listi partopsep/.store in=\listipartopsep@cx,
  listi itemsep/.store in=\listiitemsep@cx,
  listi parsep/.store in=\listiparsep@cx,
  listii topsep/.store in=\listiitopsep@cx,
  listii partopsep/.store in=\listiipartopsep@cx,
  listii itemsep/.store in=\listiiitemsep@cx,
  listii parsep/.store in=\listiiparsep@cx,
  listiii topsep/.store in=\listiiitopsep@cx,
  listiii partopsep/.store in=\listiiipartopsep@cx,
  listiii itemsep/.store in=\listiiiitemsep@cx,
  listiii parsep/.store in=\listiiiparsep@cx,
}
\cxset{compact1/.style={%
  enumerate numberingi=arabic,
  enumerate numberingii=alph,
  enumerate numberingiii=alph,
  enumerate numberingiv=roman,
  enumerate labeli punctuation=.,
  enumerate label left=,
  enumerate label right=,
  enumerate leftmargini=2.2em,
  enumerate leftmarginii=2.1em,
  enumerate leftmarginiii=1.5em,
  enumerate leftmarginiv=2em,
  listi topsep=8\p@ \@plus2\p@ \@minus\p@,
  listi itemsep=0\p@ \@plus2\p@ \@minus\p@,
  listi parsep=0\p@ \@plus2\p@ \@minus\p@,
  listii topsep=0\p@ \@plus2\p@ \@minus\p@,
  listii itemsep=0\p@ \@plus2\p@ \@minus\p@,
  listii parsep=0\p@ \@plus2\p@ \@minus\p@,
  listiii topsep=0\p@ \@plus2\p@ \@minus\p@,
  listiii itemsep=0\p@ \@plus2\p@ \@minus\p@,
  listiii parsep=0\p@ \@plus2\p@ \@minus\p@,
}}
\cxset{compact2/.style={%
  enumerate numberingi=alph,
  enumerate numberingii=roman,
  enumerate numberingiii=alph,
  enumerate numberingiv=roman,
  enumerate labeli punctuation=,
  enumerate label left=(,
  enumerate label right=),
  enumerate leftmargini=2.2em,
  enumerate leftmarginii=2.1em,
  enumerate leftmarginiii=1.5em,
  enumerate leftmarginiv=2em,
  listi topsep   = 8\p@ \@plus2\p@ \@minus\p@,
  listi itemsep = 0\p@ \@plus2\p@ \@minus\p@,
  listi parsep   = 0\p@ \@plus2\p@ \@minus\p@,
  listii topsep  = 0\p@ \@plus2\p@ \@minus\p@,
  listii itemsep= 0\p@ \@plus2\p@ \@minus\p@,
  listii parsep  = 0\p@ \@plus2\p@ \@minus\p@,
  listiii topsep = 0\p@ \@plus2\p@ \@minus\p@,
  listiii itemsep= 0\p@ \@plus2\p@ \@minus\p@,
  listiii parsep  = 0\p@ \@plus2\p@ \@minus\p@,
}}

\ExplSyntaxOn
\def\setenumerate#1{
\cxset{#1}
\def\@listi{%
           \leftmargin\leftmargini
            \parsep\listiparsep@cx
            \topsep\listitopsep@cx\relax
            \itemsep\listiitemsep@cx}
            
\def\@listii{\leftmargin\leftmarginii
            \parsep\listiiparsep@cx
            \topsep\listiitopsep@cx\relax
            \itemsep\listiiitemsep@cx}
            
\def\@listiii{\leftmargin\leftmarginiii
            \parsep\listiiiparsep@cx
            \topsep\listiiitopsep@cx\relax
            \itemsep\listiiiitemsep@cx}
}


\setenumerate{compact1}
\ExplSyntaxOff
\makeatother
   \cxset{style87} 
    \cxset{section align=left}
    \cxset{section font-weight=bold}
    \cxset{section font-family=sffamily} 
}

\cxset{chapter numbering=arabic}

\def\graphicsdocs{%
  \part{GRAPHICS}
  \chapter{Drawing pictures and graphs}
\epigraph{Dear God\break If I have but one hour remaining to live, please allow me to spend this time
in a mathematics class so that it will seem to last forever.}{\textit{---A bored student's prayer}}


\begin{figure}%
  \centering
  \includegraphics[width=0.3\linewidth]{./graphics/pic37.png}
  \caption{During the early days of typography fonts were designed to emulate the looks of calligraphic texts.}
  \label{fig:marginfig1}
\end{figure}

\parindent=0em
\parskip=0.25\baselineskip plus .25pt minus .25pt\relax

\section{Inserting figures}

In order to insert figures, the \pkgname{graphicx} package has to included in the preamble (before the |\begin{document}|-command) of your LaTeX-document:

\begin{verbatim}
\usepackage{graphicx}
\end{verbatim}

Originally only EPS-figures could be inserted with the \pkgname{graphic}package. This has now been developed into the  \pkgname{graphicx}, which allows almost any common format to be inserted. 

The simplest way of including a graphic looks like this:


\CMDI{\includegraphics}\marg{filename}


If the image is not located in the same folder as the tex-file, you will have to specify the path relative to the tex-file.

\begin{verbatim}
\includegraphics{./images/filename}
\end{verbatim}


\subsection{Scaling and resizing images}

If you want the image to appear in a different size, you can specifiy the size as a parameter of the |\includegraphics|-command::

\begin{commands}[]{ex:graphics}
\cmd{\includegraphics}\oarg{width=3.9cm}\marg{filename}
\end{commands}

This will scale the image to the width of 3.9 centimeters. 

Use |\textwidth| command if you don't want to specify an absolute size but rather want the actual size to depend on the text width of the page. You can use any of the normal \tex units such as \texttt{em, pt, cm, in}:

\begin{commands}[]{ex:graphics}
\cmd{\includegraphics}\oarg{width=0.5\string\textwidth}\marg{filename}
\end{commands}

\noindent will scale the image to half of the text width. The images in the
figure below were produced by three |\includegraphics| commands. You can have as many as you like and the \tex engine will treat them the same way as text. If you a leave a space between the commands, they will be positioned vertically as they are treated as paragraphs.

\medskip

\begin{commands}[]{}
\begingroup

\centering
\includegraphics[width=0.3\textwidth]{./graphics/amato.jpg}
\includegraphics[width=0.3\textwidth]{./graphics/amato.jpg}
\includegraphics[width=0.3\textwidth]{./graphics/amato.jpg}

\endgroup

\begin{verbatim}
\begingroup

\centering
\includegraphics[width=0.3\textwidth]{./graphics/amato.jpg}
\includegraphics[width=0.3\textwidth]{./graphics/amato.jpg}
\includegraphics[width=0.3\textwidth]{./graphics/amato.jpg}

\endgroup
\end{verbatim}
\captionof{figure}{Images aligned horizontally.}
\end{commands}

The three photos were centered using the |\centering| command, within a group. The |\begingroup..\endgroup| is necessary to limit the effect of centering to
the group only, otherwise \tex would center everything from this point onwards.

\subsection{Controlling the aspect ratio}

You can control the picture aspect ratio by using the command:

\begin{commands}[]{}
\cmd{\includegraphics}\marg{keepaspectratio,width=3cm, height=3cm}\oarg{filename}
\end{commands}

If the key |keepaspectratio| is set to true then specifying 
both |width| and |height| (or |totalheight|) does not distort the figure but 
scales such that neither of the specified dimensions is exceeded.

\medskip
\begin{commands}[]{}
\begingroup

\centering
\includegraphics[width=0.3\textwidth, height=5cm]{./images/amato.jpg}
\includegraphics[keepaspectratio=true,width=4cm, height=5cm]{./images/amato.jpg}
\includegraphics[width=3cm]{./images/amato.jpg}

\endgroup

\begin{verbatim}
\begingroup

\centering
\includegraphics[width=0.3\textwidth, height=5cm]{./images/amato.jpg}
\includegraphics[keepaspectratio=true,width=4cm, height=5cm]{./images/amato.jpg}
\includegraphics[width=3cm]{./images/amato.jpg}

\endgroup

\end{verbatim}
\captionof{figure}{Controlling the aspect ratio.}
\end{commands}

This can be very useful if you have images shown side by side with different
aspect ratios. 


\subsection{Paths and file types}

For larger projects you will probably find it more convenient to have 
images in different folders. You can specify default paths using:


\CMDI{\graphicspath}\marg{dir-list}

This optional declaration may be used to specify a list of directories in which to
search for graphics files. The format is the same as for the \latexe primitive
|\input@path|. A list of directories, each in a \{\} group (even if there is only one
in the list). For example:


\graybox{\texttt{\textbackslash graphicspath\{\{eps/\}\{tiff/\}\}}}


The default image formats can be declared using:

\CMDI{\DeclareGraphicsExtensions}\marg{png, jpg}

This specifies the behaviour of the system when no file extension is specified in 
the argument to |\includegraphics|. \texttt{\{ext-list\}} should be a comma separated 
list of file extensions. (White space is ignored between the entries.) A file name
is produced by appending one extension from the list. If a file is found, the
system acts as if that extension had been specified. If not, the next extension
in \texttt{ext-list} is tried.



\subsection{The figure environment}

You use the figure-environment to let your image appear in a \emph{floating} environment, that is \latex will place it at the right position of a page and even on the next page:

\begin{teX}
\begin{figure}
  \includegraphics{filename.jpg}
  \caption{title of your figure}
  \label{labelname}
\end{figure}
\end{teX}

Here |\caption{...}| defines the title of the figure which will appear beneath the figure. |\label{..}| defines the label which can be used inside the document in order to insert references to the figure:

The figure

|\ref{labelname} on page \pageref{labelname} ..|

The|\label-command| inside the |\figure|-envirnonment hast to appear just after the|\caption|-command.
placing figures

If figures reside inside a |\figure|-environment, this will cause LaTeX to choose the actual location of the figure inside the document. There are different parameters for the placement strategy:

\begin{description}
\item[h (here)] Try to place the figure just where the command is located.

\item [t (top)] Try to place the figure at the top of the page.

\item[b (bottom)] Try to place the figure at the bottom of the page.

\item [p (float page)] Try to place the figure on a page which contains only floating elements.
\end{description}

The order of these parameters doesn't matter since placement is always tried in the order \textbf{h, t, b, p,} if these parameters are present:

If no parameter is present, the default order is  \texttt{[tbp]}.


The command for a figure-environment might for example look like this:

\begin{teX}
\begin{figure}[htbp]
...
\end{figure}
\end{teX}



\subsection{Table of figures}
\index{figures!Table of figures}
A table of figures is inserted (where you place the command) using the command


\begin{teX}
   \listoffigures
\end{teX}

The caption given in the \cmd{caption} command is also used in the list of figures. 
If you want to use different captions, you may add a parameter to the |\caption| command:
|\caption[caption for listoffigures]{caption inside the document}|


\subsection{Figures with a border}

Although drawing frames around tables should be discouraged, if you find the need
to draw them there are  two possible ways to achieve it: either only the figure itself is bordered or there is a border around the figure and its caption. You place a border around the figure using the \cmd{\fbox} command or the \cmd{\framebox}.

\emphasis{fbox,minipage}
\begin{teX}
\begin{figure}[htbp]
  \centering
  \fbox{
    \includegraphics{filename}
  }
  \caption{caption}
  \label{Labelname}
\end{figure}
\end{teX}

Placing a border around the figure and its title is a little more tricky: You need to place the figure and the title in a |\minipage| environment which is bordered again with the |\fbox| command:

\begin{figure}[htbp]
\begin{commands}[]{}
\centering
  \fbox{
    \begin{minipage}{.95\linewidth}
      \mbox{}
      \centering
      
      \includegraphics[width=.9\linewidth]{./images/asia.jpg}
      \caption{How to place a border around an image. }
      \label{labelname}
    \end{minipage}}
 
\begin{verbatim}
\begin{figure}[htbp]
  \centering
    \begin{minipage}{width=.8\linewidth}
     \centering
     
      \includegraphics[.9\linewidth]{filename}
      \caption{caption}
      \label{labelname}
    \end{minipage}
 \end{figure}
\end{verbatim}
\end{commands}
\end{figure}



Unfortunately the width of the border cannot be determined automatically. It has to be specified as a parameter of the |\minipage| environment. However, you may be bale to develop a macro to do this,  based on the ImageSize routines we developed in section.


\section{Complex Layouts}
\label{looting}
In reality most professionally typeset books will have their own style for image pages. In Figure~\ref{complex}
three images are set in a non-symmetrical layout. This type of setting is difficult to automate and manual intervention is possible.

This layout will require four minipages. Two for the top figure (one for the image and one for the caption) and two for the two bottom figures. The rightmost bottom figure will have to be put in a zero height box to let it overflow to the top. The figure has been reproduced from an Oriental Institute publication \emph{Catastrophe! The Looting and Destruction of Iraq’s Past} \cite{looting}. The book is interesting both for its contents as well as its simple but effective typography and appropriate for the topic. The volume has been pblished in conjuction with the exhibition titled as the name of the book, that described the loss of Iraq’s archaeological past to looters and to the war. 

\begin{figure}[p]
\centering
\includegraphics[height=0.8\textheight]{oriental}
\caption{More complex layouts. \emph{Copyright the Oriental Institute of the University of Chicago.}}
\label{complex}
\end{figure}

The style is reproduced in a \pkgname{phd} template (style 56) and both code and details can be found in the relevant pages.





\section{Side by side figures}

You might want to place to figures side by side but to use only one caption. This is achieved by placing both figures in its own |\minipage| which reside in the same |\figure|.

if only one |\caption| command is used, both figures will have a common title:

\medskip
\begin{verbatim}
\begin{figure}[htbp]
  \centering
  \begin{minipage}[b]{5 cm}
    \includegraphics{filename 1}  
  \end{minipage}
  \begin{minipage}[b]{5 cm}
    \includegraphics{filename 2}  
  \end{minipage}
  \caption{common caption}
  \label{Labelname}
\end{figure}
\end{verbatim}
\medskip

The first parameter of the |\minipage| environment determines how both graphics are aligned to each other. b (bottom) aligns the bottom borders of the figures, \textbf{t} (top) aligns the top borders and \textbf{c} aligns the centers.

If you want distinct titles for the two figures you will only have to supply a |\caption| command for both |\minipage|environments:

\begin{teX}
\begin{figure}[htbp]
  \centering
  \begin{minipage}[b]{5 cm}
    \includegraphics{filename 1} 
    \caption{caption 1}
    \label{labelname 1}
  \end{minipage}
  \begin{minipage}[b]{5 cm}
    \includegraphics{filename 2}  
    \caption{caption 2}
    \label{labelname 2}
  \end{minipage}
\end{figure}
\end{teX}


If you want to have subfigures with distinct caption, you use the |\subfig| package:


You can put as many figures as you like on a page, but a word of warning, you may need to make some manual adjustments before you get it right. The package provides support for the manipulation and reference of small or ‘sub’ floats within a single floating (e.g., figure or table) environment1 It is convenient to use this
package when your sub-floats are to be separately captioned, referenced, or when such
sub-captions are to be included on a List-of-Floats page.

The package is a replacement for the subfigure package, from which it was derived.
However, the new subfig package is not completely backward compatible.
Therefore, a new name was called for. The newer package is smaller and easier to use
than the older package, however, it now uses the following additional packages, 
caption (required), 
everysel (optional), 
keyval (required), 
ragged2e (optional).

It will work without the \pkgname{ragged2e} and \pkgname{everysel} packages if you do not use the following
justification options: ‘Center’, ‘RaggedRight’ and ‘RaggedLeft’. The other justification
options ‘center’, ‘raggedright’ and ‘raggedleft’ will work without the above two packages. If the ragged2e package is present, than the caption package will load it and it
will, in turn, load the everysel package. This happens whether or not you will be using
the justification options that require it. If it cannot find the ragged2e package, than the
caption package will print a message that ‘RaggedRight’, etc. will not be available.


\begin{figure}[htb]
\includegraphics[height=5cm]{dotty}
\includegraphics[height=5cm]{bette}
\includegraphics[height=5cm]{dotty}
\end{figure}

 A low bottle-shaped vase, of yellowish ware, with flaring rim and somewhat flattened body. Height, 5 inches; width 5 inches. \ref{fig:one}

A well-made bottle shaped vase, with low neck and globular body, somewhat conical above. Color dark brownish. 7½ inches in height. Shown in \ref{fig:two}


\begin{figure}
  \centering
  \includegraphics[width=0.7\linewidth]{./graphics/fig175.jpg}
   \centerline{From the tomb of a Pull\= arius.}
  \label{fig:marginfig1}
\end{figure}

The above figure is an effigy vase of the dark ware. The body is globular. A kneeling human figure forms the neck. The mouth of the vessel occurs at the back of the head—a rule in this class of vessels. Is is finely made and symmetrical. 9¾ inches high and 7 inches in diameter. being larger than the above two it is preferable to scale it to give the reader an indication. Based on the figure width, you may also need to adjust the distance between the figures to ensure that the whitespace is just about right. For screen reading this can be increased and for printed works you may wish to make it less.



\section{The wrapfig package}


\captionsetup[wrapfigure]{margin=10pt,font=small,labelfont=bf, name=Fig.} % [wrapfigure]{name=Fig.}


Donald Arseneau has created the \pkg{wrapfig} package to allow people to place figures or
tables at the side of a page and wrap text around them. The package provides the
environments wrapfigure and wraptable. Both environments have two required and
two optional arguments. You can see an example taht uses the package to wrap a picture into such a paragraph of text.

\begin{figure}[htbp]
   \includegraphics[width=\linewidth]{./graphics/cyprus.jpg} 
   \caption{\small Cyprian limestone group of Phoenician dancers, about 6½ in. high. There is a somewhat similar group, also from Cyprus, in the British Museum. The dress, a hooded cowl, appears to be of great antiquity.}
\end{figure}

\begin{wrapfigure}[20]{l}{3.8cm}
\centering\small
\includegraphics[width=\linewidth]{./graphics/egyptdance.jpg}  
\caption{\small The hieroglyphics describe the dance.}
\end{wrapfigure}
Amongst the earliest representations that are comprehensible, we have certain Egyptian paintings, and some of these exhibit postures that evidently had even then a settled meaning, and were a phrase in the sentences of the art. Not only were they settled at such an early period (B.C. 3000, fig. 1) but they appear to have been accepted and handed down to succeeding generations (fig. 2), and what is remarkable in some countries, even to our own times. The accompanying illustrations from Egypt and Greece exhibit what was evidently a traditional attitude. The hand-in-hand dance is another of these.

The earliest accompaniments to dancing appear to have been the clapping of hands, the pipes,[1] the guitar, the tambourine, the castanets, the cymbals, the tambour, and sometimes in the street, the drum.

The following account of Egyptian dancing is from Sir Gardiner Wilkinson's "Ancient Egypt"[2]:—
\begin{figure}
   \includegraphics[width=0.3\linewidth]{./graphics/lotus.jpg} 
   \caption{\small Cyprian limestone group of Phoenician dancers, about 6½ in. high. There is a somewhat similar group, also from Cyprus, in the British Museum. The dress, a hooded cowl, appears to be of great antiquity.}
\end{figure}
"The dance consisted mostly of a succession of figures, in which the performers endeavoured to exhibit a great variety of gesture. Men and women danced at the same time, or in separate parties, but the latter were generally preferred for their superior grace and elegance. Some danced to slow airs, adapted to the style of their movement; the attitudes they assumed frequently partook of a grace not unworthy of the Greeks; and some credit is due to the skill of the artist who represented the subject, which excites additional interest from its being in one of the oldest tombs of Thebes (B.C. 1450, Amenophis II.). Others preferred a lively step, regulated by an appropriate tune; and men sometimes danced with great spirit, bounding from the ground, more in the manner of Europeans than of Eastern people. On these occasions the music was not always composed of many instruments, and here we find only the cylindrical maces and a woman snapping her fingers in the time, in lieu of cymbals or castanets.

\begin{figure}
   \includegraphics[width=0.3\linewidth]{./graphics/patera.jpg} 
   \caption{\small Cyprian limestone group of Phoenician dancers, about 6½ in. high. There is a somewhat similar group, also from Cyprus, in the British Museum. The dress, a hooded cowl, appears to be of great antiquity.}
\end{figure}

"Graceful attitudes and gesticulations were the general style of their dance, but, as in all other countries, the taste of the performance varied according to the rank of the person by whom they were employed, or their own skill, and the dance at the house of a priest differed from that among the uncouth peasantry, etc.

"It was not customary for the upper orders of Egyptians to indulge in this amusement, either in public or private assemblies, and none appear to have practised it but the lower ranks of society, and those who gained their livelihood by attending festive meetings.

"Many of these postures resembled those of the modern ballet, and the pirouette delighted an Egyptian party 3,500 years ago.
\medskip

The wrapped figure is positioned using the \texttt{wrapfigure} environment, as shown below:

\begin{teX}
\begin{wrapfigure}[nlines]{placement}[overhang ]{width }
   \includegraphics[width=3.8cm]{./graphics/egyptdance} 
   \caption{\small The hieroglyphics describe the dance.}
\end{wrapfigure}
\end{teX}

The parameter |nlines|  is the number of narrow lines, and placement is one of r, l, i, o, R, L, I, or
O for right, le, inside, and outside, respectively. The uppercase placement specifiers
differ from their lowercase counterparts in that they force \latex to put the float \emph{here},
whereas the lowercase placement specifiers just give a hint to \latex to place them
\texttt{here}. The \meta{width} argument is the width of the figure or table that appears in the body
of the environment. Finally, \texttt{overhang} tells \latex how much the figure should hang out
into the margin of the page. Here is how one may create dangerous paragraphs bends!

The |wrapfig| package is compatible with the |caption| package. You can set the caption parameters using:---

\begin{teX}
\captionsetup[wrapfigure]{<options>}
\end{teX}

If you are probably wondering how |wrapfig| achieves this, you should read the package code. It basically uses \refCom{everypar}, and hence the limitations with |\par|. Here is an extract from the class.

\begin{teX}

% Subvert \everypar to float fig and do wrapping.  
% Also for non-float.
\def\WF@startfloating{%
 \WF@everypar\expandafter{\the\everypar}\let\everypar\WF@everypar
 \WF@@everypar{\ifvoid\WF@box\else\WF@floathand\fi \the\everypar
 \WF@wraphand
}}
\end{teX}

Moving now to a more scientific example that the previous ones, we will place two figures
one on top of each other and give them individual, sub-captions as shown in \ref{fig:honey}.
 
\captionsetup[figure]{margin=10pt,font=small,labelfont=bf,format=hang}%

\begin{figure}[htbp]
\centering
  \begin{subfigure}[b]{0.5\textwidth}
  \includegraphics[width=\linewidth]{./graphics/honey.png}
  \caption{Taylor instability in the surface of the honey in an inverted honey jar.}\label{fig:honey}
    \hspace{1cm}
  \end{subfigure}

  \begin{subfigure}[b]{0.9\textwidth}
     \centering
     \includegraphics[width=9cm]{./graphics/honeydrops.png}
     \caption{Taylor instability in the interface of the water condensing on the underside of a small water pipe.}
  \end{subfigure}  
  \caption{Two examples of Taylor instabilities that are commonly found.}%
    \label{fig:Athird}%
\end{figure}

The figures are from \textit{A Heat Transfer Textbook}, by J.H.Lienhard, which incidentally was typeset using
\tex . It is a McGrawHill publication. 

\begin{teX}
\begin{figure}[htbp]
    \captionsetup[figure]{margin=10pt}%
    \subfloat[Taylor instability...]
     {{\includegraphics[width=8cm]{./graphics/honey}}}
    \hspace{1cm}
     \subfloat[Taylor instability in the...]%
      {\includegraphics[width=9cm]{./graphics/honeydrops}}  
     \\[-10pt]
   \caption{Taylor instability in...}%
    \label{fig:Afirst}%
    \caption{Two examples of... }%
    \label{fig:honey}%
\end{figure}
\end{teX}


The text can have more than one paragraph. It is also possible to include figures
generated by |TikZ/pgf|, as shown in the next example, drawn from real code
in the book.


\begin{wrapfigure}[14]{l}{3.0cm}
\pgfplotsset{width=5.0cm,compat=1.3}
\begin{tikzpicture}
\begin{axis}[minor y tick num=4, 
minor x tick num=4, 
xmin=0,xmax=300,
ymin=0,ymax=60,
xlabel=\textsf{liquidus ($l/s$)},
ylabel=\textsf{capitis ($m$)}, 
ytick={0,15,30,45,60,75},
xtick={0,100,200,300}
]
\addplot[color=blue,mark=x, smooth] coordinates {
(0,44)
(50,43)
(100,42)
(150,40)
(200,33)
(220,29)
};

\end{axis}
\end{tikzpicture}
\caption{Pump headum and flowm}
\end{wrapfigure}


\providecommand\addcredit[1]{%
 \vspace*{-6.5pt}
 \scriptsize%
 \flushright%
 \textit{Credit: #1}%
}

\begin{figure}[htp]
\centering

\captionsetup{name=Photo., labelsep=period}%
   \begin{minipage}[t]{0.48\textwidth}
      \includegraphics[width=\textwidth]{./graphics/movingup.jpg}%
      \addcredit{U.S. DoD.}%
     \caption{The effects of the credit going past the edge of the figure. This can be corrected by adding a minipage to hold both commands. }
\end{minipage}\hfill\hfill
\begin{minipage}[t]{0.48\textwidth}
      \includegraphics[width=\textwidth]{./graphics/survivors.jpg}%
      \addcredit{U.S. DoD.}%
    {\footnotesize Marines awaiting resting before moving on to Japan. }
\end{minipage}

% \begin{minipage}[t]{0.48\textwidth}
%      \includegraphics[width=\textwidth]{./graphics/img009.jpg}%
%      \addcredit{U.S. DoD.}%
%     \caption{Engineer Construction Troops in Liberia, July 1942.}
%\end{minipage}\hfill\hfill
%\begin{minipage}[t]{0.48\textwidth}
%      \includegraphics[width=\textwidth]{./graphics/survivors.jpg}%
%      \addcredit{U.S. DoD.}%
%     \caption{The effects of the credit going past the edge of the figure. This can be corrected by adding a minipage to hold both commands. }
%\end{minipage}
% \begin{minipage}[t]{0.48\textwidth}
%      \includegraphics[width=\textwidth]{./graphics/img126.jpg}%
%      \addcredit{U.S. DoD.}%
%     \caption{Marine Reinforcements.
%A light machine gun squad of 3d Battalion, 1st Marines, arrives during the battle for ``Boulder City.'' }
%\end{minipage}\hfill\hfill
%\begin{minipage}[t]{0.48\textwidth}
%      \includegraphics[width=\textwidth]{./graphics/img124.jpg}%
%      \addcredit{U.S. DoD.}%
%     \caption{Brothers Under the Skin, inductees at Fort Sam Houston, Texas, 1953. }
%\end{minipage}
\end{figure}
\newpage


Armed with all these packages you can help the Gutenburg organization to transcribe
some of the old books that they have online. 

\clearpage








  \chapter{Wrapped Illustrations}
\label{ch:wrapped}
\parindent2em
\let\onepar\lorem

Wrapped figures are not in vogue and most users of \latex avoid them.
If you are planning to have a more traditional book design wrapped figures might be more appropriate. Traditional typographers used
all sorts of styles to achieve wrapped figures which conserved paper. 
The best way to achieve it is to use Donald Arseneau's |wrafig| package \citep{wrapfig}.

\begin{wrapfigure}{l}{3.2cm}
    \includegraphics[width=3cm]{./images/amato.jpg}
    \caption{\footnotesize Wrapped figures}
\end{wrapfigure}

Get prepared to do a lot of manual adjustments, see your figures disappear on page refreshes and reruns. It is also recommended that you do your final adjustments once you are happy with the contents of your document and these final adjustments will not start jumping around. 
After a while though you get the hang of it and by minor adjustments you can really achieve great results. The manual uses \verb+everypar+ to insert commands for the shaping of the paragraphs that \emph{follow} the wrapped figure.

The package provides the environments \pkg{wrapfigure} and \pkg{wraptable} for typesetting a
narrow float at the edge of the text, and making the text wrap around it. The |wrapfigure|
and |wraptable| environments interact properly with the \verb+\caption+ command to produce
proper numbering, but they are not regular floats like \textit{figure} and \textit{table}, so be aware to do manual adjustments. If you do not take care 
they may also be printed out of sequence with the regular floats.

The |wrapfigure| environment  provides one of those monster locomotive type commands that stresses one's memory as it provides for four parameters.
 
The four param
for \verb+\begin{wrapfigure}+, two optional and two required, plus the text of the figure, with a caption perhaps.

\begin{macro}{wrapfigure}
\end{macro}

|\begin{wrapfigure}[12]{r}[34pt]{5cm}\meta{figure}\end{wrapfigure}|

  \begin{tikzpicture}[xshift=-15pt]
    \node (number) at (0mm, 0mm) {\oarg{number of narrow lines}};
    \node (placement) at (36mm, 0mm) {\marg{placement}};
    \node (overhang) at (60mm, 0mm) {\oarg{overhang}};
    \node (width) at (81mm, 0mm) {\marg{width}};
    \begin{scope}[->]
    \draw (number) -- (16mm, 17mm);
    \draw (placement) -- (24mm, 17mm);
    \draw (overhang) -- (35mm, 17mm);
    \draw (width) -- (47mm, 17mm);
    \end{scope}
  \end{tikzpicture}


First we will look at placing the figure without the use of optional commands.


\begin{verbatim}
\begin{wrapfigure}{r}{.4\textwidth}
    \includegraphics[width=.4\textwidth]{./path/file}
    \caption{\footnotesize Wrapped figures}
\end{wrapfigure}
\end{verbatim}

From the four parameters the first one indicates if the figure is to be typeset left or right.

\begin{verbatim}
\begin{wrapfigure}{l}{\imagewidth}
    \includegraphics[width=\imagewidth



]{./graphics/parasol-01}
    \caption{\footnotesize Wrapped figures}
\end{wrapfigure}
\end{verbatim}


\begin{wrapfigure}[18]{I}[0.1pt]{85pt}
    \captionsetup{name=Fig.}
    \vskip-10.5pt plus 2pt minus 2pt\relax
    \includegraphics[width=83pt]{./images/parasol-01.jpg}
    \caption{Wrapped figures, parameters set at \texttt\{l\}\{90pt\}.}
\end{wrapfigure}

Changing the parameters to suit we now have the illustration floating to the left. Allowing for the figure to be approximately two point  wider than the actual graphic, will leave a bit more margin. If the figure is end low in the page you need to be careful, that it does not disappear, as you will not get any warning.

The first parameter we are going to use an optional parameter is the one that determines the number of narrow lines. The format is \verb+[narrowlines]{l}{90pt}+. Think of this parameter as a fine tuning parameter and do not touch it until after your final draft is ready. If you see indented lines at the beginning of the page that follows the wrapped figure, reduce the number of lines, until you get satisfactory results.

The second optional parameter, comes after the \texttt{\{r\}[overhang]} parameter.

The second optional parameter (\#3) tells how much the figure should hang out into
the margin. The default overhang is given by the length \verb+\wrapoverhang+, which is 0pt
normally but can be changed using the command |\setlength|. For example, to have all wrapped figures you can 
use the space reserved for marginal notes,

\begin{verbatim}
\setlength{\wrapoverhang}{\marginparwidth}
\addtolength{\wrapoverhang}{\marginparsep}
\end{verbatim}

Again not recommended. The best approach is to specify the figures with \textbf{O} or \textbf{I}, let them float and if the results are
not very good then make manual adjustments. Get prepared to spend at least 5-10 minutes fiddling with the final result.

When you do specify the overhang explicitly for a particular figure, you can use a
special unit called \string\width meaning the width of the figure. For example, [0.5\string\width]
makes the center of the figure sit on the edge of the text, and [\string\width] puts the figure
entirely in the margin (and the adjacent text is indented by just \string\columnsep). This
\texttt{\string\width} is the actual width of the wrapfigure, which may be greater than the declared
width.

\begin{figure}[tb]
\includegraphics[width=\textwidth]{./graphics/chiefs.jpg}
\caption{Chiefs of Kelau or Kelaou.}
\label{fig:chiefs}
\end{figure}

\begin{figure}[p]
\centering

\includegraphics[width=0.8\textwidth,height=0.9\textheight, keepaspectratio]{./images/parasol-01.jpg}
\caption{Chiefs of Kelau or Kelaou.}
\label{fig:parasol-01}
\end{figure}

\section{Balancing the illustrations}

Illustrations come in various sizes, but in general they need to flow with the text. Place figures on top of the page and figures that would dominate the text on their own page. For example Figure~\ref{fig:chiefs} was allowed to float to the top of a page whereas Figure~\ref{fig:parasol-01} was placed on its own page, as I thought it will overwhelm the text if shown in a large size. However the same figure seems perfectly alright as a wrapped figure.

\begin{texexample}{}{}
\begin{wrapfigure}{I}{0pt}
    \includegraphics[width=75pt]{./images/parasol-01.jpg}
 \end{wrapfigure}
\lipsum[1-2]
\begin{wrapfigure}{l}{0pt}
    \includegraphics[width=75pt]{./images/parasol-01.jpg}
 \end{wrapfigure}
\lipsum[1-2]
\end{texexample}



\begin{texexample}{}{}
\begin{wrapfigure}{l}{0pt}
    \includegraphics[width=70pt]{./images/parasol-01.jpg}
    \includegraphics[width=70pt]{./images/parasol-01.jpg}
 \end{wrapfigure}

\lipsum[1-3]\lorem
\end{texexample}




\begin{texexample}{}{}
\begin{wrapfigure}[13]{L}{0pt}
    \includegraphics[width=100pt]{./graphics/conicalbasket.png}
\end{wrapfigure}

\onepar\onepar\onepar

\end{texexample}




  \chapter{Subfigures}

So far we have been using the |caption| package to add captions to multiple figures, that are numbered individually, but how about if you want to have only one caption and number the subfigures alphabetically. If you want to have |subfigures| with distinct caption, you use the |subfig| package \citep{subfigure}. A newer package \ctan{subcaption} is also now available with the |caption| suite and we will discuss this also. The two packages are incompatible and the recommendation is to use the |subcaption| package. In the |phd| class we load the |caption| package which also loads the |subcaption| package. The latter is to be preferred as it integrates better both with captions as well as the |hyperref| package.

\begin{figure}[h]
\centering
\begin{minipage}[b]{.3\linewidth}
\includegraphics[width=4cm]{./graphics/pic37.png}\hspace{1em}
\subcaption{First fighting elephant}\label{fig:1a}
\end{minipage}\hspace{1em}
\begin{minipage}[b]{.3\linewidth}
\includegraphics[width=4cm]{./graphics/pic37.png}\hspace{1em}
\subcaption{Second fighting  elephant}\label{fig:1b}
\end{minipage}\hspace{1em}
\begin{minipage}[b]{.3\linewidth}
\includegraphics[width=4cm]{./graphics/pic37.png}\hspace{1em}
\subcaption{Third fighting elephant}\label{fig:1c}
\end{minipage}
\caption{Three fighting elephants example}
\end{figure}

You can put as many figures as you like on a page, but a word of warning, you may need to make some manual adjustments before you get it right. The package provides support for the manipulation and reference of small or \enquote{sub} floats within a single floating (e.g., figure or table) environment It is convenient to use this
package when your sub-floats are to be separately captioned, referenced, or when such
sub-captions are to be included on a List of Floats page.

The package is a replacement for the |subfigure| package, from which it was derived.
However, the new |subfig| package is not completely backward compatible.
Therefore, a new name was called for. The newer package is smaller and easier to use
than the older package, however, it now uses the following additional packages,  |caption| (required),  |everysel| (optional),
keyval (required),  |ragged2e| (optional). All these packages are included with the |phd| auto package manager.

It will work without the |ragged2e| and |everysel| packages if you do not use the following
justification options: \enquote{Center}, \enquote{RaggedRight} and \enquote{RaggedLeft}. The other justification
options \enquote{center}, \enquote{raggedright} and \enquote{raggedleft} will work without the above two packages. If the ragged2e package is present, than the caption package will load it and it
will, in turn, load the everysel package. This happens whether or not you will be using
the justification options that require it. If it cannot find the ragged2e package, than the
caption package will print a message that \enquote{RaggedRight}, etc. will not be available.

\section{Subcaption environments}

The |subcaption| package offers an environment for subfigures, which are essentially minipages. Within the environment, the normal caption command can be used rather than the \cmd{\subcaption}.

\begin{figure}%
    \centering
    \captionsetup[figure]{margin=3pt}%
    \begin{subfigure}[b]{.35\linewidth}
    \includegraphics[scale=0.65]{./graphics/fig155.jpg} 
    \label{fig:one}
    \caption{First Caption}
    \end{subfigure}\hspace{2em}
    \begin{subfigure}[b]{.35\linewidth}
    \includegraphics[scale=0.65]{./graphics/fig156.jpg} 
    \caption{First Caption}
    \end{subfigure}
    \caption{Two subfigures side by side.}
    \label{fig:two}
\end{figure}

The sub-figures can be referenced the same way as normal referencing.

\begin{quote}
 A low bottle-shaped vase, of yellowish ware, with flaring rim and somewhat flattened body. Height, 5 inches; width 5 inches. \ref{fig:one}


A well-made bottle shaped vase, with low neck and globular body, somewhat conical above. Color dark brownish. $7\frac{1}{2}$ inches in height. Shown in Figure~\ref{fig:two}.
\end{quote}

\begin{figure}[htp]
  \centering
  \includegraphics[width=0.5\linewidth]{./graphics/fig175.jpg}
  \vspace{3\baselineskip}

   \centerline{\textsc{From the tomb of a Pull\= arius.}}
  \label{fig:marginfig1}
  \caption{ effigy vase of the dark ware. The body is globular. A kneeling human figure forms the neck. The mouth of the vessel occurs at the back of the head—a rule in this class of vessels. Is is finely made and symmetrical. 9.75 inches high and 7 inches in diameter. being larger than the above two it is preferable to scale it to give the reader an indication.}
\end{figure}

The above figure is an Based on the figure width, you may also need to adjust the distance between the figures to ensure that the whitespace is just about right. For screen reading this can be increased and for printed works you may wish to make it less.

\begin{teXXX}
\begin{figure}[htb]
\begin{subfigure}[b]{.5\linewidth}
\centering\large A
\captionsetup{skip=3pt}
\caption{A subfigure}\label{fig:1a}
\end{subfigure}
\end{figure}
\end{teXXX}

\begin{comment}
\begin{figure}[htp]%
    \captionsetup[figure]{margin=3pt}%
    \subfloat[One subone.\label{fig:one}]
     {{\includegraphics[scale=0.65]{./graphics/fig155.jpg}}}
    \hspace{1cm}
    \subfloat[One subtwo.\label{fig:two} --- but this one has a
     very very long caption.  So long that it continues over into
     other lines so that we can test the list-of line settings.]%
      {\includegraphics[scale=0.65]{./graphics/fig156.jpg}}
     \\[-10pt]
    \caption{First figure --- but this one has a very very long caption.
     So long that it continues over into a second line so that we can
     test the margin setting and centering of the caption command in the
     full page mode.}%
    \label{fig:Afirst}%
    \caption{Typical pottery from Oklahoma (\emph{Smithsonian}).}%
    \label{fig:Athird}%
\end{figure}
\end{comment}

The figures have been placed using the code below:

\begin{verbatim}
\begin{figure}%
    \captionsetup[figure]{margin=3pt}%
    \subfloat[One subone.\label{fig:one}]
     {{\includegraphics[scale=0.65]{./graphics/fig155.jpg}}}
    \hspace{1cm}
    \subfloat[One subtwo.\label{fig:two} --- but this one has a
     very very long caption.  So long that it continues over into
     other lines so that we can test the list-of line settings.]%
      {\includegraphics[scale=0.65]{./graphics/fig156.jpg}}
     \\[-10pt]
 \caption{First figure but this one has a very very long caption.
 So long that it continues over into a second line so that we can
 test the margin setting and centering of the caption command in the
 full page mode.}%
 \label{fig:Afirst}%
 \caption{Typical pottery from Oklahoma (\emph{Smithsonian}).}%
 \label{fig:Athird}%
\end{figure}
\end{verbatim}

As you can observe, the |subcaption| package treats the two figures as one and places them side by side. Its trickery is to get them to line up, nicely and to provide all the necessary parameters for the captions. It is a feature-rich package and we will spent some time to explore it. The command |subfloat|, is used to denote the |subfigure|. The rest are self-explanatory. Note that the use of |\hspace{1cm}| to make these two figures come closer together. In the previous listings, |\hfill| was used to space them out as wide as possible. The command |captionsetup| is used to let the package know that we are captioning figures and not tables. (In this book all captions are placed in the side-margins, where God meant them to be! If you use the same code in another package they will be placed underneath the figures.




  \begin{comment}
\documentclass[imperial, justified]{octavo}
\usepackage{caption}
\usepackage{natbib}
\usepackage{lstdoc}
\usepackage{lipsum}
\usepackage{graphicx}
\usepackage{overpic}
\usepackage{url}
\global\setlength\parindent{1em}
\newif\ifdebug
\debugfalse
\ifdebug  
  \setlength\fboxsep{1pt}
\else
  \setlength\fboxsep{0pt}
  \setlength\fboxrule{0pt}
\fi

%% temporary titles
% command to provide stretchy vertical space in proportion
\newcommand\nbvspace[1][1]{\vspace*{\stretch{#1}}}

% allow some slack to avoid under/overfull boxes
\newcommand\nbstretchyspace{\spaceskip0.5em plus 0.25em minus 0.25em}

% To improve spacing on titlepages
\newcommand{\nbtitlestretch}{\spaceskip0.6em}

% temporary length used for some tables
\newlength{\TmpLen}

\begin{document}
\clearpage
\pagestyle{empty}
\begin{center}
\bfseries

\nbvspace[1]
\Huge
{\nbtitlestretch\huge
 TYPESETTING  
WITH  \TeX\ AND SX.TX FRIENDS  \\
}

\nbvspace[2]
\normalsize
TO WHICH IS ADDED MANY USEFUL MACROS
AND CODE WRITTEN SO THAT HE WHO RUNS MAY HACK

\nbvspace[1]
\small BY\\
\nbvspace[1]
\Large THE STACKEXCHANGE COMMUNITY {\large\textsc{}}\\[0.5em]
%\footnotesize AUTHOR OF ``A WORKING ALGEBRA,'' ``WIRELESS TELEGRAPHY,\\
%ITS HISTORY, THEORY AND PRACTICE,'' ETC., ETC.

\nbvspace[4]

%\includegraphics[width=0.8in]{ejc.pdf}
\includegraphics[width=1.5in]{./images/fig176}
\par
\nbvspace[2]
\normalsize
%DOHA$\cdot$BERLIN$ \cdot$ WILD

\nbvspace[10]
\Large
PUBLISHED IN THE WILD
%
\end{center}


\long\def\secondpage{\clearpage\null\vfill\vfill
\pagestyle{empty}
\begin{minipage}[b]{0.9\textwidth}
\footnotesize\raggedright
\setlength{\parskip}{0.5\baselineskip}
Copyright \copyright 2010--\the\year\ Dr Yiannis Lazarides\par
Permission is granted to copy, distribute and\slash or modify this document under the terms of the GNU Free Documentation License, version 1.2, with no invariant sections, no front-cover texts, and no back-cover texts.\par
A copy of the license is included in the appendix.\par
This document is distributed in the hope that it will be useful, but without any warranty; without even the implied warranty of merchantability or fitness for a particular purpose.
\end{minipage}
\vspace*{2\baselineskip}}

\secondpage

\backmatter
\tableofcontents
\listoffigures

\chapter{PREFACE}
This small booklet aims to describe some of the common problems encountered with the 
placement of figures in books. It also tries to provide techniques for storing them within TeX.

\mainmatter
\end{comment}

\chapter{How to Typeset a lot of Figures}
\precis{In this chapter we develop a primitive database for storing graphics and then typesetting them.}
\addtocimage{-12pt}{-20pt}{../images/tocblock-men.jpg}


If you have a lot of figures, it is a lot of work to have to maintain them, as well as
to remember all the file names. The figures are from an old Catalogue of the Smithsonian Institution \citep{holmes1884}. 

\section{A long table for figures}

We are familiar with longtable for tables, this is an equivalent technique for lots of figures.
\smallskip

\def\figurename{\textbf{Plate}}

\DeclareRobustCommand\putgraphic[1]{%
\fboxsep0pt\fboxrule0pt
\fbox{%
\begin{minipage}[b]{2.0cm}%
 \centering
 \vspace{3.8pt}\fbox{%
 \includegraphics[width=0.98\linewidth,
                 height=2.3cm,
                 keepaspectratio]{./images-01/#1.jpg}}%
  \vspace{0.2cm} #1%
  \vspace{0.2cm}%
  \end{minipage}}\hfil
}

\long\def\putcaption#1{\captionof{figure}{#1}}

\makeatletter
{\centering

\gdef\alist{fig145,fig161,fig162,fig163,fig164,fig165,fig166,fig167,^^A,
fig168,fig169,fig170,fig171,fig172,fig173,^^A
fig174,fig175,fig176,fig177,fig180,fig181,fig182,fig183,fig185,fig186,fig187,fig188,fig189}
\@for \i:=\alist\do{^^A
\expandafter\putgraphic{\i}
}
\putcaption{Weaving and pottery artifacts from Arizona.}}

\medskip

The code leverages \tex's ability to create macro names with any character using the |\csname...\endcsname| construct. We first put the
images in a list. The images have been saved as |fig145| etc on the disk and hence what we simply do is just enclose them in a comma delimited list. They do not need to be numbered consequentially in the list.

\begin{verbatim}
\gdef\alist{fig145,fig161,fig162,..,fig187,fig188,fig189}
\end{verbatim}

We then loop over the |\alist| and get the output as shown in Example~\ref{ex:blist}. 

\begin{texexample}{Looping over the list}{ex:blist}
\def\blist{fig189,fig145,fig161,fig162}
\@for \i:=\blist\do{%
  \expandafter\putgraphic{\i}%
}
\end{texexample}


\section{More on figures and looping}

We can extend our macros and try and save some information for each image. To do this we
need to have a way to associate information with the figure number so we will create a number of commands
for each figure.

The \TeX\ way of defining commands on the fly that include non-letters is to use \verb+\csname+
\begin{verbatim}
\expandafter\def\csname fig170\endcsname#1{#1}
\@nameuse{fig170}{Pottery found in Apache%
    lands in Texas.}
\end{verbatim}

\@nameuse{fig170}{Pottery found in Apache %
    lands in Texas.}

This is not very useful, as it is. It is preferable to actually create a little command factory, that can create these
commands.

\begin{texexample}{}{factory command}
\bgroup
\gdef\commandfactory#1#2{
   \expandafter\def\csname #1\endcsname{#1}
   \expandafter\def\csname #1@caption\endcsname{#2}
}
\commandfactory{fig170}{Test}
\centering

\putgraphic{\csname fig170\endcsname}
\putcaption{\@nameuse{fig170@caption}}

\egroup
\end{texexample}

Since we are going to have to type a lot of information into a database to hold information for our images, we might as
well type it straight into our text.

Out of consideration for our users we may want to provide a short command for this.

\begin{verbatim}
\let\img\commandfactory
\end{verbatim}

\let\img\commandfactory

\img{fig171}{Testing again for something.}

We may also want to save the use of the curly brackets, that would visually distruct. We can redefine the Command factory to be a delimited macro. There is a lot of information on delimited macros. One of them is in such a place, hiding on \texttt{tex.sx}.

\def\commandfactory#1|#2|{
   \expandafter\def\csname #1\endcsname{#1}
   \expandafter\def\csname #1@caption\endcsname{#2}
}

\commandfactory fig172|This is figure 172|

\commandfactory fig173|This is figure 173|

\texttt{\@nameuse{fig172@caption}}

\texttt{\@nameuse{fig173@caption}}

Now that we have figured a way to define an efficient way to store information for our figures, we need to build some routines to sort them print them and other similar housekeeping routines.

\section{Sorting}

\global\setlength\parindent{1em}
I have still to find a better sorting routine other than the one available in the listings documentation. I did try my hands with LuaTeX but I am not very fond of jumping in and out of LaTeX. It can also create problems with updates and users that might not have LuaTeX installed.

We will store the record index in a macro that is essentially a comma delimited list. Don't be frighten about speed
I have used this method to store over 4000 figures and there was no problem either with the processing speed or with TeX'es memory.

We call this macro \verb+dbartifacts+, giving it a non-generic name. But first let us see, how we can add items in
and out of the macro. We start from an empty macro.

\begin{verbatim}
\def\dbartifacts{ }
\end{verbatim}
\let\dbartifacts\empty

We can use \LaTeX's \verb+\g@addto@macro+ to then add the items to the \verb+\dbartifacts+ macro.
\begin{verbatim}
\g@addto@macro{\dbartifacts}{fig172,}%
\g@addto@macro{\dbartifacts}{fig173,}%
\end{verbatim}

\g@addto@macro{\dbartifacts}{fig172,}%
\g@addto@macro{\dbartifacts}{fig173,}%


Testing it by just typing \verb+\texttt{\dbartifacts}+ we get: \texttt{\dbartifacts}. This of course is not very convenient and we would rather define a macro to save all the typing and have a more user friendly command.

\begin{verbatim}
\def\addtodb#1#2{%
  \g@addto@macro#1{#2,}%
}
\end{verbatim}
\def\addtodb#1#2{%
  \g@addto@macro#1{#2,}%
  \lst@BubbleSort\dbartifacts%
}

\clearpage

There are many other ways to manipulate the list, including using token registers, elt lists etc, but for such constructions as the ones described here, this is by far the simpler and the easiest.
We can now use this macro, when required:

\begin{verbatim}
\addtodb{\dbartifacts}{fig170}%
\addtodb{\dbartifacts}{fig171}%
\end{verbatim}

Testing again we get \texttt{\dbartifacts} an as you can see it works nicely. This method of trying out your code bit by bit, I call the water painting technique. So now that we have almost got all the routines we want, we can now look at sorting. This we achieve by adding \verb+ \lst@BubbleSort\dbartifacts+. Every time we add a record, the file will be sorted. Intuituitevely, this might not  be very efficient, especially if you are adding a lot of records at one time, but we can add more helper routines later for this.

\begin{verbatim}
\def\addtodb#1#2{%
  \g@addto@macro#1{#2,}%
  \lst@BubbleSort\dbartifacts%
}
\end{verbatim}

\def\figurename{\textbf{Figure}}
\begin{figure}
\vspace*{1cm}
\centering
\includegraphics[scale=0.6]{./images/fig172.jpg}
\caption{Textiles from Arizona. }
\end{figure}

\section{Adding some more user helper macros}

It is expected that the user will produce a file, either through some automatic means or by typing it to hold the data. Deletion and insertion is simply via editing this file through a text editor. However, for completeness, we will write a few macros to help with maintenace of the database. These include macros for delete and modify record etc.

Another set of macros that one can use is to typet the records in lists and or tabulat forms, if required. Early books on archaelogy for example listed all the items in the following format, interspersed with comments and figures.
\smallskip


\hangindent3em
2520. (39510). A double globe jar or canteen. White ground, with ornamentations in black, as seen in Fig. 649. Depression in the center is probably designed to receive a band or cord to carry it with.
\smallskip

Although one is tempted to produce a list for these, the next item from such a book points otherwise:
\smallskip

\hangindent3em
2677-2678. 2677, (39617), and 2678, (39618). With flared and notched rim.
\smallskip

Before extending the database for such forms of descriptions, we can develop the typesetting part. I am sure that Lamport would have used a list, possibly due memory and space limitations and just re-use the \verb+\item+ command, in our case it is better to rather define a small macro
to cater for such items. The indentation can easily be achieved using \verb+\hangindent3em+ or a similar amount of measure.

\begin{verbatim}
\long\def\catno#1\par{
\par%
\hangindent3em\noindent
#1
}
\end{verbatim}


\def\catno#1#2{%
    \@hangfrom{#1. }#2
}


\DescribeMacro{\@hangfrom}\marg{text}   
\LaTeX\ provides a macro named \verb+\@hangfrom{<text>}+, that puts \marg{text} in a box, and makes a hanging indentation of the following material up to the first \verb+\par+. This Should be used in vertical mode.\footnote{See source2e, \texttt{ltsect.dtx}, pg 287.}

\begin{verbatim}
121 \def\@hangfrom#1{\setbox\@tempboxa\hbox{{#1}}%
122 \hangindent \wd\@tempboxa\noindent\box\@tempboxa}
\end{verbatim}

\medskip

\catno{289}{(39914). Fig. 397. Red ware, with white lines on the lower globe and decorations in black on the upper, with orifice in each globe.}

\catno{1289}{(39914). Fig. 397. Red ware, with white lines on the lower globe and decorations in black on the upper, with orifice in each globe.}


\makeatother

\section{Epiloque}

We have managed to write a database, sort it, typeset its contents in a structured or freeform manner
and on the way we have documented the code using a form of \textit{literate programing.} On top which
other language expects you to code your own ifs and for? 
The amount of code we wrote was very minimal and competes well with modern computer languages. 

Hope you had fun. Go and make beautiful books. 

\begin{figure}[htp]
\centering
{\color{thegray}
\fbox{\includegraphics[width=1\linewidth]{./images//pottery-figures.pdf}}}
\caption{Many books in the humanities have figure pages, with many different styles and numbering schemes. This page extract is from \textit{The Cypro-Phoenician pottery of the Iron Age. }  \citep{schreiber1971}}
\end{figure}


\begin{figure}[htp]
\centering
{\color{thegray}
\fbox{\includegraphics[width=1\linewidth]{./images/sample-tof.pdf}}}
\caption{Many books in the humanities have figure pages, with many different styles and numbering schemes. This page extract is from \textit{The Cypro-Phoenician pottery of the Iron Age. }  \citep{schreiber1971} and shows a specific way of numbering subfigures, including references.}
\end{figure}
\clearpage



 \subsection{Acknowledgements}

 Octavo is a modification of \texttt{classes.dtx} written by Leslie Lamport (1992),
 Frank Mittelbach (1994-97) and Johannes Braams (1994-97). As can be seen
 from the code, my own input is restricted to a tweaking of some parameters
 and true credit is due to Lamport, Mittelbach and Braams for their
 monumental efforts.



\begin{comment}
 \begin{thebibliography}{16}

 \bibitem{knuth98} Knuth,~D. 1998. \emph{Digital Typography}. CSLI 
 Publications, Stanford.

 \bibitem{rosarivo61} Rosarivo,~R. 1961. \emph{Divina proportio typographica}. 
 Scherpe, Krefeld.

 \bibitem{taylor98} Taylor,~P. 1998. \emph{Book design for \TeX\ users, Part 1: 
 Theory.} TUGBoat, 19:65--74.

 \bibitem{taylor99} Taylor,~P. 1999. \emph{Book design for \TeX\ users, Part 2:
 Practice.} TUGBoat, 20:378--389.

 \bibitem{town} Town,~L. \emph{Bookbinding by hand.} Faber \& Faber, London.

 \bibitem{tschichold87} Tschichold,~ J. 1987. \emph{Ausgew\"{a}hlte Aufs\"{a}tze
 \"{u}ber Fragen der Gestalt des Buches und der Typographie}. Birkh\"{a}user
 Verlag, Basel.

 \bibitem{williamson66} Williamson,~H. 1966. \emph{Methods of book design.} Oxford 
 University Press, Oxford.

 \bibitem{wilson01} Wilson,~P. 2001. \emph{The Memoir class for configurable
 typesetting.} CTAN. \url{macros\\latex\\contrib\\memoir} 

 \end{thebibliography}
\end{comment}





\chapter[Overflowing Figures into Margins]{OVERFLOWING FIGURES INTO MARGINS}

Most users of \TeX\ are accustomed to let the system position images, either on top or bottom of the page and occasionally use the [h] positioning directive to place the image at the exact location it appears in the text. Traditional typography placed the image in many different positions. It also occasionally overflowed the image into the margins. The image below, copied from the \textit{American Antiquarian}, was placed in the original publication as such. Tufte advocates the use of such techniques in displaying not only information, but also other material such as tables. The Tufte class is discussed extensively in other sections. It has almost a religious following attached to it and I have personally used it for business reports.\citep{seraphini}

\begin{figure}[htbp]
\leftskip-.07\textwidth\includegraphics[width=1.14\textwidth]{./images/elephant-long.jpg}\par

\begin{multicols}{4}
\myanmar

လူတိုင်းသည် တူညီ လွတ်လပ်သော ဂုဏ်သိက္ခါဖြင့် လည်းကောင်း၊ တူညီလွတ်လပ်သော အခွင့်အရေးများဖြင့် လည်းကောင်း၊ မွေးဖွားလာသူများ ဖြစ်သည်။ ထိုသူတို့၌ ပိုင်းခြား ဝေဖန်တတ်သော ဉာဏ်နှင့် ကျင့်ဝတ် သိတတ်သော စိတ်တို့ရှိကြ၍ ထိုသူတို့သည် အချင်းချင်း မေတ္တာထား၍ ဆက်ဆံကျင့်သုံးသင့်၏။
\end{multicols}
\centerline{\protect\textsc{Codex Seraphinianus, Mystery Procession \protect\citep{seraphini}}.}
\end{figure}

Almost as a matter of rule, the caption for these images was in small caps. Using small caps brought the caption into the easy attention of the reader, but it did not distract from the other elements of the page.
The image is not necessarily positioned symmetrically in the page, you can offset it to suit your taste, but in general, unless the image has any particular features that would make it look better offset rather than centered, is best positioned symmetrically. This can be automated, by writing a macro that measures the dimensions of the image and introduces a \verb+\leftskip+ so that the image can be shifted accordingly. A macro to achieve this is now described.


The image can be included by simply using a \verb+\skip-1.2cm+ or \verb+\leftskip-1.2cm+ :

\begin{verbatim}
\begin{figure}[htbp]
\leftskip-1.2cm\includegraphics{image}\par
\centerline{\textsc{Copper Sheath}}
\end{figure}
\end{verbatim}

\long\def\imghangleft#1#2{%
     \figure
     \leftskip-#2\textwidth\includegraphics[width=#1\textwidth]{./images/elephant-long.jpg}\par
     \centerline{\textsc{Codex Serafinianus}}
    \endfigure
}

\imghangleft{1.14}{.07}






  \section{SIDEWAYS PICTURES}
Figures can be rotated as shown in Figure~\ref{fig:sideways}  a landscape mode using the \texttt{rotating} package. A package for rotated objects in LATEX
Robin Fairbairns, Sebastian Rahtz, Leonor Barroca.


The code uses the \verb!sideways! environment. In this particular example, we use footnotes, in the caption and hence we add some code to achieve this.  Note that the package defaults take care of verso and recto page display so that you do not need to   worry about rotating the image clockwise or counterclockwise. The package rotates by default clockwise.

\begin{tcolorbox}
\begin{lstlisting}
\begin{sidewaysfigure}

\includegraphics[height=0.5\textheight, width=0.9\textwidth, keepaspectratio]{nudewithapple}
\captionof{figure}{The package sets the\protect\footnotemark[1] footnotes\protect\footnotemark[2] of a single-column document in two columns;
the package offers a range of parameters to determine\protect\footnotemark[3] the exact appearance\protect\footnotemark[4] of the two columns.}
\vspace{3\baselineskip}
\footnoterule\footnotesize
\begin{minipage}[t]{0.4\linewidth}
\textsuperscript{1} This is the first footnote. And here comes some nonsense text
                    to show that the linebreaks works \par
\textsuperscript{2} This is the second footnote.\par
\end{minipage}\hfill
\begin{minipage}[t]{0.4\linewidth}
\textsuperscript{3} This is the third footnote. \par
\textsuperscript{4} This is the fourth footnote.\par
\textsuperscript{5} This is the fourth footnote.\par
\textsuperscript{6} See \url{http://tex.stackexchange.com/questions/8174/how-to-achieve-a-multi-column-layout-for-footnotes}\par
\end{minipage}
\end{sidewaysfigure}
\end{lstlisting}
\end{tcolorbox}


\begin{sidewaysfigure}

\centering
\includegraphics[height=0.5\textheight, width=0.9\textwidth, keepaspectratio]{bathers-01}
\captionof{figure}{The package sets the\protect\footnotemark[1] footnotes\protect\footnotemark[2] of a single-column document in two columns;
the package offers a range of parameters to determine\protect\footnotemark[3] the exact appearance\protect\footnotemark[4] of the two columns.}
\vspace{3\baselineskip}
\footnoterule\footnotesize
\begin{minipage}[t]{0.49\linewidth}
\textsuperscript{1} This is the first footnote. And here comes some nonsense text
                    to show that the linebreaks works \par
\textsuperscript{2} This is the second footnote.\par
\end{minipage}\hfill
\begin{minipage}[t]{0.49\linewidth}
\textsuperscript{3} This is the third footnote. \par
\textsuperscript{4} This is the fourth footnote.\par
\textsuperscript{5} This is the fourth footnote.\par
\textsuperscript{6} See \url{http://tex.stackexchange.com/questions/8174/how-to-achieve-a-multi-column-layout-for-footnotes}\par
\end{minipage}
\end{sidewaysfigure}


\begin{sidewaysfigure}

\centering
\includegraphics[height=0.5\textheight, width=0.9\textwidth, keepaspectratio]{nudewithapple}
\captionof{figure}{The package sets the\protect\footnotemark[1] footnotes\protect\footnotemark[2] of a single-column document in two columns;
the package offers a range of parameters to determine\protect\footnotemark[3] the exact appearance\protect\footnotemark[4] of the two columns.}
\vspace{3\baselineskip}
\footnoterule\footnotesize
\begin{minipage}[t]{0.49\linewidth}
\textsuperscript{1} This is the first footnote. And here comes some nonsense text
                    to show that the linebreaks works \par
\textsuperscript{2} This is the second footnote.\par
\end{minipage}\hfill
\begin{minipage}[t]{0.49\linewidth}
\textsuperscript{3} This is the third footnote. \par
\textsuperscript{4} This is the fourth footnote.\par
\textsuperscript{5} This is the fourth footnote.\par
\textsuperscript{6} See \url{http://tex.stackexchange.com/questions/8174/how-to-achieve-a-multi-column-layout-for-footnotes}\par
\end{minipage}
\end{sidewaysfigure}




\clearpage

\begin{figure}

\centering
\includegraphics[height=\textheight, width=\textwidth, keepaspectratio]{julesbache}
\end{figure}

\begin{figure}

\centering
\includegraphics[height=\textheight, width=\textwidth, keepaspectratio]{goya01}
\end{figure}

\begin{figure}

\centering
\includegraphics[height=\textheight, width=\textwidth, keepaspectratio]{goya-sideways}
\end{figure}

\begin{figure}

\centering
\includegraphics[height=\textheight, width=\textwidth, keepaspectratio]{goya-sideways01}
\end{figure}

\pagebreak







  
\parindent0pt

\begin{minipage}{1.05\textwidth}
\vspace{\baselineskip}
\parindent0pt
\fboxrule0pt
{
\centering
\fbox{\centering
\begin{minipage}[t]{0.89\textwidth}
\centering
\begin{minipage}[t]{0.41\textwidth}
\includegraphics[width=1\textwidth]{./images/threewomen01.png}\vspace*{-8pt}%
\captionof*{figure}{\noindent\footnotesize\textbf{WALDO PEIRCE}, a famous painting in his own right,
turned model for Bellows, posed for this impressive portrait in New York studio in 1920.}
\end{minipage}\hspace{0.5cm}
\begin{minipage}[t]{0.4\textwidth}
   \includegraphics[width=1\textwidth]{./images/threewomen02.png}\vspace*{-8pt}
    \captionof*{figure}{\noindent\footnotesize\textbf{MRS KATHERINE ROSEN,}
                 the daughter of Charles Rosen, he was an artist and neighbor of bellows, 
                 posed for this  meditative study in 1921.}
\end{minipage}
\end{minipage}
}}

\medskip

\fbox{\hskip-0.3cm\includegraphics[width=1.03\textwidth]{./images/twowomen-03.png}}\\[-27.5pt]
\setlength{\linewidth}{.95\textwidth}
\setlength{\columnsep}{8pt}
\begin{multicols}{2}
\noindent \footnotesize\textbf{TWO WOMEN,} portrays a professional model dressed and undressed. The range and richness of colors is unusual among Bellows' pictures. Bellows always had a horror of studio pictures and ``pretty nudes,'' rarely worked from professional models and never painted a still life.
\end{multicols}
\vfill

\captionof{figure}{Balancing three images on a page. Should the larger image be at the top or at the bottom?}
\end{minipage}

\newcommand\articleheading[1]{%
    \par
    \vspace*{2\baselineskip}
    \bgroup
    \LARGE\bf\textsf{\noindent #1}
    \egroup
   \vskip2\baselineskip
}
\clearpage

\begin{minipage}{\textwidth}
\includegraphics[width=\textwidth]{./images/yaleartschool.png}

\articleheading{TRADITION AND TECHNIQUE AT YALE'S SCHOOL OF  FINE ARTS}

\end{minipage}
\begin{multicols}{3}
        \lettrine{A}{t Yale}\lorem \lipsum[1-3]
        \par
\end{multicols}

\newgeometry{top=0pt, left=0pt, right=0pt, top=0pt, bottom=2cm}
\pagebreak

\begin{minipage}{\textwidth}
\includegraphics[width=\textwidth]{./images/sculpture-lesson.jpg}\par
\vspace{\baselineskip}

\centerline{\HUGE\bfseries SCULPTURE LESSON}
\vspace{0.5\baselineskip}

\centerline{\LARGE\bfseries Noted arist shows how adventurous amateurs can model with clay }

\end{minipage}

{
\leftskip1cm\rightskip1cm\columnsep-1.3cm\par\leavevmode

\begin{multicols}{3}
        \lettrine{A}{t Yale} \lorem \lorem \lorem \lorem
        
\end{multicols}
}

\newgeometry{top=1.5cm,left=2cm,right=2cm,bottom=2cm}

\pagebreak





\lipsum[1]
\includegraphics[height=0.8\textheight, width=\textwidth\relax]{./images/nino.png}

This is a short caption test and this one is a long caption test.

\includegraphics[width=\textheight, width=\textwidth]{./images/woman.png}
Donna Velata.

\clearpage
\raggedbottom


\noindent\includegraphics[width=\textwidth]{./images/odalisque.png}^^A
This is a short caption test and this one is a long caption test.
\vspace*{2\baselineskip}


\begin{minipage}[t]{0.3\textwidth}
\vbox to -6cm{\noindent\includegraphics[width=0.98\textwidth]{./images/ginerva.png}
This is a short caption test and this one is a long caption test.}
\end{minipage}%
\begin{minipage}[t]{.7\textwidth}%
\noindent\textbf{\Huge \hfill Kathleen Gilje\hskip0.1em\hfill}\\[2\baselineskip]
\end{minipage}


\leftskip0.41\textwidth

Lorem ipsum dolor sit amet, consectetur adipiscing elit. Etiam eu nunc dolor. Nam arcu nisi, hendrerit at facilisis et, aliquet sit amet massa. Aenean ullamcorper mi dolor. Sed ut urna vitae elit tristique varius tempus vitae orci. Maecenas tristique lectus vel enim posuere congue. Aliquam pellentesque nisl vel nunc iaculis dictum. Sed luctus, orci vehicula blandit rutrum, risus justo aliquet elit, id venenatis est libero nec sem. Sed varius molestie ante non fringilla.

Vestibulum ut mollis odio. Vivamus ut risus eu dolor laoreet viverra. Nullam elit erat, congue at placerat ut, posuere non diam. Suspendisse eget dui et mi varius bibendum at non orci. Morbi justo arcu, posuere non tempus at, vestibulum sit amet lorem. Class aptent taciti sociosqu ad litora torquent per conubia nostra, per inceptos himenaeos. Donec tempor dignissim tellus, vitae vestibulum tellus hendrerit tempus. Nullam varius justo sit amet risus semper non semper eros placerat. Integer eleifend ligula in est gravida ornare tincidunt velit tristique.


Donec vel erat a ipsum condimentum volutpat vel non odio. Vivamus non justo orci. Pellentesque ligula ipsum, vestibulum at molestie vel, mollis sed odio. Donec rhoncus, sem in auctor tincidunt, libero quam scelerisque urna, et volutpat purus magna ac nulla. Cras vel quam nec urna viverra ornare eu et nibh. Pellentesque tincidunt leo non odio varius vitae sollicitudin neque adipiscing. 

\section{Full Page Images}

\leftskip0pt\parindent1em

In euismod, enim a dictum pharetra, libero nibh tempor enim, vel fermentum justo justo eget sem. Integer convallis massa nec turpis volutpat tristique. Quisque fringilla volutpat sem porta elementum. Donec vel metus quis nisl venenatis vehicula ac quis est. Maecenas vulputate lacinia lacus quis porttitor. Aliquam consectetur consectetur metus eu bibendum. Lorem ipsum dolor sit amet, consectetur adipiscing elit. In sem mauris, mollis nec pulvinar posuere, facilisis quis turpis. Quisque vel laoreet mauris.

Quisque ultrices dignissim odio at malesuada. Duis euismod tellus nec ante porta vel ullamcorper orci semper. Vivamus in eros est. Etiam et pellentesque nisi. Sed faucibus dictum tortor vitae accumsan. Donec ante risus, ornare et iaculis eget, cursus at metus. Maecenas neque urna, rutrum sit amet lacinia non, accumsan nec tortor. Proin tempor dictum porta. Morbi luctus nulla et sapien elementum aliquam ut eget neque. Quisque lobortis eleifend lorem adipiscing semper. Quisque molestie magna lorem, non mollis est. Mauris urna arcu, pretium sed dignissim id, tempor accumsan massa



\noindent\includegraphics[width=\textwidth]{./images/napoleon.jpg}
This is a short caption test and this one is a long caption test.



\clearpage
\newenvironment{kathleen}[1][b]{\def\placement{#1}\parindent0pt
}{}

\cxset{kathleen align/.is choice,
       kathleen align/top/.code=\xdef\kathleenplacement@cx{t},
       kathleen align/bottom/.code=\xdef\kathleenplacement@cx{b},
       kathleen align/center/.code=\xdef\kathleenplacement@cx{c},
       kathleen imagei/.code=\def\imagei{\includegraphics[width=\textwidth]{#1}\par},
 kathleen imageii/.code=\def\imageii{\includegraphics[width=\textwidth]{#1}\par},
kathleen imageiii/.code=\def\imageiii{\includegraphics[width=\textwidth]{#1}\par},
kathleen imageiv/.code=\def\imageiv{\includegraphics[width=\textwidth]{#1}\par},
kathleen imagev/.code=\def\imagev{\includegraphics[width=\textwidth]{#1}\par},
kathleen captioni/.code=\def\captioni{\captionof{figure}{#1}},
kathleen captionii/.code=\def\captionii{\captionof{figure}{#1}},
kathleen captioniii/.code=\def\captioniii{\captionof{figure}{#1}},
kathleen scale/.store in=\kathleenscale@cx
}

\long\def\printkathleen{\begin{kathleen}[t]
\begin{minipage}{\kathleenscale@cx\textwidth}
\begin{minipage}[\kathleenplacement@cx]{0.3\textwidth}
\vbox{}
\imagei
\captioni
\imageii
\captionii
\imageiii
\captioniii
\end{minipage}\hspace{1cm}
\begin{minipage}[\kathleenplacement@cx]{0.46\textwidth}
\vbox{}
\imageiv
\captionof{figure}{This is a short caption test and this one is a long caption test.}\par
\imagev
\captionof{figure}{This is a short caption test and this one is a long caption test.}
\end{minipage}
\end{minipage}
\end{kathleen}}

\begin{figure}
\cxset{kathleen align = top,
       kathleen imagei = {./images/ladyagnew.png},
       kathleen imageii = {./images/etta.png},
       kathleen imageiii = {./images/etta.png},
       kathleen imageiv = {./images/ladyagnew.png},
       kathleen imagev  = {./images/etta.png},
       kathleen captioni = {Al contrario di quanto si pensi, Lorem Ipsum non \`e semplicemente una sequenza casuale di caratteri. Risale ad un classico della letteratura latina del 45 AC.}, 
       kathleen captionii = {Finibus Bonorum et Malorum di Cicerone. Questo testo un trattato su teorie di etica, molto popolare nel Rinascimento. La prima riga del Lorem Ipsum.},
       kathleen captioniii= This is a short caption.,
       kathleen scale = 1.1,
} 

\printkathleen

\caption{The Kathleen template page. It consists of five images and their caption text. Parameters can be set via a key value interface.}
\end{figure}
\clearpage

\cxset{kathleen align = top,
       kathleen imagei = {./images/ladyagnew.png},
       kathleen imageii = {./images/etta.png},
       kathleen imageiii = {./images/etta.png},
       kathleen imageiv = {./images/ladyagnew.png},
       kathleen imagev  = {./images/etta.png},
       kathleen captioni = {Al contrario di quanto si pensi, Lorem Ipsum non \`e semplicemente.}, 
       kathleen captionii = {Finibus Bonorum et Malorum di Cicerone. Questo testo  un trattato.},
       kathleen captioniii= This is a short caption.,
       kathleen scale = 0.7
} 

\cxset{kathleen align=bottom}




\begin{center}\printkathleen\par\label{kathleen}\end{center}

\newpage

\section{The Kathleen template} 

A lot of pages in image rich books have complicated settings for images.
These are difficult to manipulate and we provide here what we hope is
a better method. For example the Figure~\ref{kathleen} shows such a complex layout. This can be achieved by only filling in the template
values as shown below.

\begin{tcolorbox}
\begin{lstlisting}
\cxset{kathleen align = top,
       kathleen imagei = ladyagnew,
       kathleen imageii = etta,
       kathleen imageiii = etta,
       kathleen imageiv = ladyagnew,
       kathleen imagev  = etta,
       kathleen captioni = {Al contrario di quanto si pensi, Lorem Ipsum non \`e semplicemente una sequenza casuale di caratteri. Risale ad un classico della letteratura latina del 45 AC.}, 
       kathleen captionii = {Finibus Bonorum et Malorum di Cicerone. Questo testo un trattato su teorie di etica, molto popolare nel Rinascimento. La prima riga del Lorem Ipsum.},
       kathleen captioniii= This is a short caption.,} 
\cxset{kathleen align=bottom,
       kathleen scale=.5}

\printkathleen

\end{lstlisting}
\end{tcolorbox}


\newgeometry{top=0pt,left=1cm,right=1cm,marginparsep=0pt}

\clearpage


\parindent0pt
\pagestyle{empty}

\fboxsep0pt
\fboxrule0pt

\vspace*{-1cm}
\begin{minipage}{1.05\textwidth}
\hskip-0.9cm\includegraphics[width=1.03\textwidth]{./images/parasol-05.jpg}\\[-27.5pt]
\setlength{\linewidth}{0.95\textwidth}
\setlength{\columnsep}{10pt}
\begin{multicols}{2}
\noindent \footnotesize\textbf{DESIGNED FOR CONTRAST} with the wearer's ensemble, these plaid  tafetta and green rayon parasols, are best sellers at Maey's in New York. Set of matching parasol and shoes, or
gloves, scarves or bags, are also available to give simple dresses
a custom appearance.
\end{multicols}
\vspace{-0.25cm}
\rule{1.5cm}{0pt}\fbox{
\begin{minipage}[t]{0.87\textwidth}
\begin{minipage}[t]{0.41\textwidth}
\includegraphics[width=1.03\textwidth]{./images/parasol-06.jpg}\par%
\noindent \footnotesize\textbf{CHERRY ORNAMENTS} adorn handle and tip of this parasol, made by Jane Derby to go with the afternoon dress. Straight handles are very popular.
\end{minipage}\hspace{0.5cm}
\begin{minipage}[t]{0.4\textwidth}
   \includegraphics[width=1\textwidth]{./images/parasol-07.jpg}\par
\noindent \footnotesize\textbf{MATCHING SETS} of afternoon dress
and parasol, and four-piece polka dot weekend dress and parasol,
both designed by Briganne.
\end{minipage}
\end{minipage}
}

\vfill

\captionof{figure}{Balancing three images on a page. Should the larger image be at the top or at the bottom?}
\end{minipage}




\begin{minipage}{\textwidth}
\begin{minipage}[b][\textheight][b]{.47\linewidth}
\vspace*{2cm}

\includegraphics[width=\linewidth]{./images/parasol-03.jpg}\par
\vspace{2\baselineskip}

\centerline{\bfseries\Huge Parasols}
\vspace{2\baselineskip}

\begin{quote}
\lipsum[2]
\end{quote}

\vfill

\textbf{SHOES AND PARASOL SET} in pink are here combined with a dress, one of whose skirts is pink. Parasol is from New York's ``Uncle Sam'' parasol shop.
\end{minipage}\hspace*{1cm}
\begin{minipage}[b]{.53\linewidth}
\mbox{}
\includegraphics[width=\linewidth]{./images/parasol-01.jpg}\par
\end{minipage}
\end{minipage}

\newgeometry{top=1.5cm,bottom=3cm,left=3.5cm,right=3.5cm}

\clearpage
  \cxset{toc image=botticelli-34}

\chapter{Image Pages}

\lipsum[1-5]

\clearpage

{
\parindent0pt
\pagestyle{empty}

\fboxsep0pt
\fboxrule0pt

\vspace*{-1cm}
\begin{minipage}{1.05\textwidth}
\hskip-0.9cm\includegraphics[width=1.03\textwidth]{twowomen-03}\\[-27.5pt]
\setlength{\linewidth}{0.95\textwidth}
\setlength{\columnsep}{10pt}
\begin{multicols}{2}
\noindent \footnotesize\textbf{TWO WOMEN,} portrays a professional model dressed and undressed. The range and richness of colors is unusual among Bellows' pictures. Bellows always had a horror of studio pictures and ``pretty nudes.'' He rarely worked from professional models and never painted a still life. This painting was published in Life Magazine.
\end{multicols}
\vspace{-0.25cm}
\rule{1.5cm}{0pt}\fbox{
\begin{minipage}[t]{0.87\textwidth}
\begin{minipage}[t]{0.41\textwidth}
\includegraphics[width=1.03\textwidth]{threewomen01}\par\vspace*{-8pt}%
\captionof*{figure}{\noindent\footnotesize\textbf{WALDO PEIRCE}, a famous painting in his own right,
turned model for Bellows, posed for this impressive portrait in New York studio in 1920.}
\end{minipage}\hspace{0.5cm}
\begin{minipage}[t]{0.4\textwidth}
   \includegraphics[width=1\textwidth]{threewomen02}\vspace*{-8pt}
    \captionof*{figure}{\noindent\footnotesize\textbf{Mrs Katherine Rosen,}
the daughter of Charles Rosen, he was an artist and neighbor of bellows, posed for this meditative study in 1921.}
\end{minipage}
\end{minipage}
}

\vfill

\captionof{figure}{Balancing three images on a page. Should the larger image be at the top or at the bottom?}
\end{minipage}
}


   
%
\newgeometry{top=2cm, bottom=1cm, left=1cm, right=1cm,
               marginparsep=0cm, marginpar=0pt}
\makeatletter
\cxset{kroll scale/.store in = \scalekroll@cx,
       kroll left column width/.store in = \krollleftcolumnwidth@cx,
       kroll imagei/.store in = \krollimagei@cx,
       kroll imagei caption/.store in = \krollimageicaption@cx,
       kroll imageii/.store in = \krollimageii@cx,
       kroll imageii caption/.store in = \krollimageiicaption@cx,
       kroll left header/.store in = \krollleftheader@cx,
       kroll header/.store in = \krollheader@cx}

\cxset{kroll scale = 1,
       kroll left column width = .3\textwidth,
       kroll left header = Leon\\[15pt] Kroll,
       kroll imagei = krollportrait,
       kroll imagei caption = shows Kroll at 59. Says he. ``Painting is 
             fascinating'' even when motif my own mug.,
       kroll imageii = nudeback,
       kroll imageii caption = {NUDE  BACK  SHOWS   A  DANCER  WHOSE  BACK  SAYS  KROLL,  HAS  BEAUTIFUL  PLANES},
       kroll header = \scalebox{.97}{THE DEAN OF US NUDE-PAINTERS}
    }

\newenvironment{kroll}{%
\renewenvironment{leftcolumn}{%
   \minipage[b]{\krollleftcolumnwidth@cx}%
  }{\endminipage}\hspace*{0cm}%
 \renewenvironment{rightcolumn}{%
   \minipage[b]{.62\textwidth}%
  }{\endminipage}\hspace*{0cm}% 
\begin{minipage}{\scalekroll@cx\textwidth}%
 \noindent
  \begin{leftcolumn}%
   \MainHeader{\krollleftheader@cx}%
   \putimage[width=0.5\linewidth]{\krollimagei@cx}\par
   \aheader{\krollimageicaption@cx}%
\end{leftcolumn}\hfill%
\begin{rightcolumn}%
 \includegraphics[width=\linewidth]{\krollimageii@cx}%
 \onelinecaption{{\resizebox{\linewidth}{5.5pt}{\bfseries   \krollimageiicaption@cx}}\par}%
 \onelineheader{\krollheader@cx}%
 \begin{multicols}{2}}
{%
   \end{multicols}%
   \end{rightcolumn}%
   \end{minipage}} 
\makeatother



\begin{kroll}
 \lettrine{A}{t the} age of 63 when businessmen are thinking of retiring leon Kroll according to Life Magazine was having the busiest time of his life, just doing what comes naturally.  \lorem
\end{kroll}

\cxset{kroll scale = 1,
       kroll left column width = .3\textwidth,
       kroll left header = Cooling\\Water\\ Systems\vskip5pt
                          {\bfseries \Large \lorem},
       kroll imagei = industrial,
       kroll imagei caption = shows Kroll at 59. Says he. ``Painting is 
                                    fascinating'' even when motif my own mug.,
        kroll imageii = industrial,
       kroll imageii caption = {NUDE  BACK  SHOWS   A  DANCER  WHOSE  BACK  SAYS  KROLL,  HAS  BEAUTIFUL  PLANES},
       kroll header = \scalebox{1}{\hfill HVAC CHILLED WATER SYSTEMS \hfill}
    }


\begin{kroll}
 \lettrine{A}{t the} age of 63 when businessmen are thinking of retiring leon Kroll according to Life Magazine was having the busiest time of his life, just doing what comes naturally.  \lorem \the\pagetotal
\end{kroll}

\restoregeometry
  % 
%
\newgeometry{top=1cm, bottom=1cm, left=1cm, right=1cm,
               marginparsep=0cm, marginpar=0pt}
\newpage

\makeatletter
\cxset{bache scale/.store in = \scalebache@cx,
    bache left column width/.store in = \bacheleftcolumnwidth@cx,
    bache imagei/.store in = \bacheimagei@cx,
    bache imagei caption/.store in = \bacheimageicaption@cx,
    bache imageii/.store in = \bacheimageii@cx,
    bache imageii caption/.store in = \bacheimageiicaption@cx,
    bache left header/.store in = \bacheleftheader@cx,
    bache header/.store in = \bacheheader@cx}%
\cxset{bache scale = 1,
    bache left column width = {\dimexpr\textwidth-.4\textwidth\relax},
    bache left header =,
    bache imagei = bache-01,
    bache imagei caption ={\begin{multicols}{2}\lorem\lorem\end{multicols}},
    bache imageii = nudeback,
    bache imageii caption = {JULES BACHE},
    bache header = \scalebox{.97}{THE DEAN OF US NUDE-PAINTERS}
    }%
\newenvironment{bache}{%
\parindent0pt
\renewenvironment{leftcolumn}{%
   \minipage[t]{\bacheleftcolumnwidth@cx}%
   \leavevmode   
  }{\endminipage}\hspace*{0cm}%
 \renewenvironment{rightcolumn}{%
   \minipage[t][\textheight-45pt][t]{.37\textwidth}%
   \mbox{}%
  }{\endminipage}\hspace*{0cm}% 
\begin{minipage}[t][\textheight][t]{\scalebache@cx\textwidth}%
\resizebox{\textwidth}{!}{\Large\bfseries\sffamily JULES BACHE GIVES HIS \$20,000,000 ART COLLECTION TO NEW YORK}\par%
\begin{leftcolumn}%
\mbox{}%ncessesary to line on top
\par\leavevmode\includegraphics[width=\linewidth]{\bacheimagei@cx}\par
\bacheimageicaption@cx%
\end{leftcolumn}\hfill%
\begin{rightcolumn}%
\mbox{}%
\intextsep0pt
\@afterindentfalse\parindent1em
\begin{wrapfigure}{l}{0pt}
 \includegraphics[width=.37\linewidth]{bache-02}
\caption*{\bfseries\sffamily \bacheimageiicaption@cx}
  \end{wrapfigure}\ignorespaces
 }
{\end{rightcolumn}%
\end{minipage}%
} %
%

\begin{bache}
This layout has a dominant left column image. It is important to
ensure that the image has an aspect ratio to suit. Unfortunately
it is very difficult to crop and scale an image via \tex so a bit
of experimentation is appropriate.

It is also important to ensure that you add an adequate amount
of text during editing, otherwise the layout will not look very good. The right
column has two images (it really looks better when it has two images rather than
one and the bottom image is really a filler, if you have more or less
text you may have to go back and crop the image to suit. Any extra space on the right column is used as glue. The template also has a manual mode, where one can adjust the lengths and writing
a bit more accurately. \label{bache}

\vfill
\includegraphics[width=\linewidth]{bache-03}
\end{bache}

\restoregeometry


\section{The bache template}
The bache template, named after the dominant photograph in the sample template
is an adaptation of a layout from a Life magazine. The basic layout is shown below and
a full page sample is shown on page~\pageref{bache}.

{\begin{center}

\cxset{bache scale=.7}
\fboxsep0pt
\resizebox{\scalebache@cx\textwidth}{!}{\begin{bache}
This layout has a dominant left column image. It is important to
ensure that the image has an aspect ratio to suit. Unfortunately
it is very difficult to crop and scale an image via \tex so a bit
of experimentation is appropriate.

It is also important to ensure that you add an adequate amount
of text during editing, otherwise the layout will not look very good. The right
column has two images (it really looks better when it has two images rather than
one and the bottom image is really a filler

\includegraphics[width=\linewidth]{bache-03}
\end{bache}}

\end{center}
}

\begin{lstlisting}
\cxset{bache scale = 1,
    bache left column width = {\dimexpr\textwidth-.4\textwidth\relax},
    bache left header =,
    bache imagei = bache-01,
    bache imagei caption ={\begin{multicols}{2}\lorem\lorem\end{multicols}},
    bache imageii = nudeback,
    bache imageii caption = {JULES BACHE},
    bache header = \scalebox{.97}{THE DEAN OF US NUDE-PAINTERS}
    }%
\end{lstlisting}

Keeping simplicity in mind, we only require the user to fill the above template and to type only
a short piece of code and the text. It is preferable to write the last piece of
text, rather than insert this type of writing in the template, as one may need to iterate a 
couple of times to get the right amount of text.
\begin{lstlisting}
\begin{bache}
    text body on right column.
\end{bache}
\end{lstlisting}

I believe that filling a few lines of information in a template and then a short environment, is
the simplest way possible. A more complicated way is to set the template on manual and
build it piece by piece with commands.





\restoregeometry
  \newpage
\makeatletter
\@specialtrue
\makeatother

\tikzset{dim/.style={color=black!25,thick,>=stealth,}}%
\def\labelit{%
{\tikz[remember picture]\draw[dim,<-|,overlay] (0,0)--++(0,0.8)--++(1,0) node[right,fill=blue!15,text=black] {textii};}%
}%
\def\labelitt#1{%
\hbox to 0pt{{\tikz[remember picture]\draw[dim,<-|,overlay] (0,0)(0,0.3)--++(0.0,0.8)--++(0.5,0) node[right,fill=blue!15,text=black] {\footnotesize\texttt{#1}};}
}}%
\cxset{custom = genetics,
 title font-size=\Huge,
 title font-weight=\bfseries,
 title font-family=\bfseries,
 image = {./images/greco-02.jpg},
 image caption={\labelitt{image caption}EL GRECO},
 textiii={\labelit 
 \begin{itemize}
\large
\item How to set-up special chapter environments.
\item How to define special field variables.
\item How to set text styles.
  \end{itemize}
}}

\chapter{Grego}

\restoregeometry

\cxset{greco image/.store in=\grecoimage@cx,
       greco heading/.store in=\grecoheading@cx}
\cxset{greco image={./images/julio.jpg},
       greco heading=EL GRECO,
       chapter toc=true}

\newenvironment{greco}{%
\cxset{section align=center,
       section numbering= Roman}
\thispagestyle{plain}%
\checkoddpage
\ifoddpage
  \def\offsetx{0cm}%
\else
  \def\offsetx{-1.9cm}%
\fi%
%\labelitt{figure}

\hskip\offsetx\begin{minipage}[t]{\textwidth}%
\includegraphics[width=\textwidth+1.9cm]{\grecoimage@cx}%

\hbox to 1.15\textwidth{\hss{\tiny\textbf{FROM A PAINTING BY EL GRECO.}}}

\end{minipage}
\vspace*{2\baselineskip}

\section{\grecoheading@cx}
\offsetx
\begin{multicols}{3}
\parindent1em
}{\end{multicols}}



\cxset{greco heading= JULIO CLOVIO}
\begin{greco}

\noindent \lettrine{G}{iulio Clovio} was born in Croatia. He was a native of Griane, a village near the town of Modru.[4] It is not known where he had his early training, but he may have studied art with monks at Fiume of Novi Bazar when he was young. [5]

He moved to Italy at age 18 and entered the household Cardinal Marino Grimani where he was trained as a painter. Between 1516 and ca 1523 Clovio may have lived with Marino in the residence of the latter’s uncle Cardinal Domenico Grimani in Rome. [6] Clovio studied under Giulio Romano during this early period. [7]

While a protege of Cardinal Domenico Grimani Clovio engraved medals and seals for him, as well as the Grimani Commentary Ms., an important early illuminated book (now Sir John Soane's Museum, London).

By 1524 Clovio was at Buda, at the Hungarian court of King Louis II, for whom he painted the ``Judgment of Paris'' and ``Lucretia''. After Louis' death in the Battle of Mohács, Clovio travelled to Rome where he continued his career.[8]

After 1527 he visited several monasteries of the Canons Regular of St. Augustine. In 1534 Clovio returned to the household of Cardinal Marino Grimani.[8] A year later Clovio may have followed Marino when the latter was appointed as a papal legate to Perugia, where Clovio is thought to have worked on illustrations for the Soane Manuscript written by Marino Grimani around that time. Clovio likely returned to Rome by the end of 1538 when he is known to have met with the writer Francisco de Hollanda.

\end{greco}
\clearpage


\cxset{greco heading= EL GRECO}
\cxset{greco image={./images/greco-02.jpg}}
\begin{greco}
\noindent \lettrine{G}{iulio Clovio} was born in Croatia. He was a native of Griane, a village near the town of Modru.[4] It is not known where he had his early training, but he may have studied art with monks at Fiume of Novi Bazar when he was young. [5] 

He moved to Italy at age 18 and entered the household Cardinal Marino Grimani where he was trained as a painter. Between 1516 and ca 1523 Clovio may have lived with Marino in the residence of the latter’s uncle Cardinal Domenico Grimani in Rome. [6] Clovio studied under Giulio Romano during this early period. [7]

While a protege of Cardinal Domenico Grimani Clovio engraved medals and seals for him, as well as the Grimani Commentary Ms., an important early illuminated book (now Sir John Soane's Museum, London).
By 1524 Clovio was at Buda, at the Hungarian court of King Louis II, for whom he painted the ``Judgment of Paris'' and ``Lucretia''. After Louis' death in the Battle of Mohács, Clovio travelled to Rome where he continued his career.[8]

After 1527 he visited several monasteries of the Canons Regular of St.~Augustine. In 1534 Clovio returned to the household of Cardinal Marino Grimani.[8] A year later Clovio may have followed Marino when the latter was appointed as a papal legate to Perugia, where Clovio is thought to have worked on illustrations for the Soane Manuscript written by Marino Grimani around that time. Clovio likely returned to Rome by the end of 1538 when he is known to have met with the writer Francisco de Hollanda.
\end{greco}

\section{Developing special styles for image pages}

\makeatother





  \makeatletter

\long\def\makedoublehead{%
  \leavevmode\hbox{%
  \vtop{\hsize.35\textwidth
  \bfseries\Large\sffamily\noindent  
  \headi@cx
  \raggedright}}%
  \hfill\leavevmode\hbox{\hsize0.4\textwidth
  \vtop{\bfseries\Large\sffamily  \headii@cx \raggedleft}}
  \addcontentsline{toc}{section}{CIRCULAR DATESTAMPS}
  \addcontentsline{toc}{subsection}{Page \thepage}
}

\cxset{ap mainhead/.store in=\mainhead@cx,
       ap headi/.store in=\headi@cx,
       ap headii/.store in = \headii@cx,
       ap image/.store in = \apimage@cx,
       ap image title/.store in=\imagetitle@cx,
       ap image ex/.store in=\imageex@cx,
       ap image caption/.store in=\imagecaption@cx,
       ap frame/.is choice,
       ap frame/on/.code = \fboxrule1pt\fboxsep-1pt,
       }

\cxset{ap mainhead= CYPRUS,
       ap headi =1. Circular Datestamps,
       ap headii= {ap headii},
       ap image = 1948-1,
       ap image title= set image title,
       ap image ex = Goldblatt,
       ap image caption = \lipsum[1],
       ap frame= on,
}


\cxset{ap image title = ADVERTIZED AND UNCLAIMED MARK}


\long\gdef\albumpagegeometry{%
 \newgeometry{top=.5in,bottom=1in,left=1cm,right=1cm}
}

\albumpagegeometry


\newenvironment{singlepage}{%
\bgroup
\parindent0pt
\thispagestyle{empty}
\sffamily
\pagestyle{empty}
\mbox{}
\centerline{\huge\bfseries \mainhead@cx}
\vskip2\baselineskip
\makedoublehead
\vfill
\centering
\lipsum[1]
\vfill
\begin{minipage}{1\textwidth}
\leavevmode\centering
\bfseries\large \imagetitle@cx
\end{minipage}
\vfil
\includegraphics[width=\textwidth]{./images/1948-1.jpg}\par
\hfill \imageex@cx\par 
\vfill\leavevmode
\centering
\leftskip.1\textwidth
\rightskip.1\textwidth
\imagecaption@cx\par 
\egroup
}{\newpage}

\cxset{ap mainhead= mainhead,
       ap headi = ap headi,
       ap headii= {ap headii},
       ap image title= set image title,
       ap image ex = Goldblatt,
       ap image caption =,
       ap frame= on,
}
\begin{singlepage}
\lorem
\end{singlepage}


\cxset{ap mainhead= CHINA,
       ap headi = INFLATION PERIOD,
       ap headii= {Domestic Rates (1946-1947)},
       ap image = 1948-1,
       ap image title = China Inflation Period Cover,
       ap image ex = Goldblatt,
       ap image caption =\lorem,
       ap frame= on,
}
\begin{singlepage}
\lorem
\end{singlepage}











  \begin{commands}[]{}
\bgroup
\makeatletter
\fboxrule0px
\fboxsep-0px
\parindent0pt
\sffamily
\mbox{}%
\centerline{\huge\bfseries \mainhead@cx}%
\vskip2\baselineskip
\makedoublehead%
\hfill\hfill\fbox{\begin{minipage}[t]{0.55\textwidth}%
\null%
\includegraphics[width=1.01\linewidth]{./images/1948-02.jpg}%
\vspace*{20pt}
\end{minipage}}\hfil%
\fbox{\begin{minipage}[t]{.4\linewidth}%
\fbox{\begin{minipage}[t]{\textwidth}%
\null
\lipsum[1]
\end{minipage}}%
\par
\leavevmode\vskip0pt\rule{1px}{30px}\vskip0pt
\fbox{\begin{minipage}[t]{\textwidth}%
\null
\lipsum[1]
\end{minipage}}%
\end{minipage}}%\hfill
\par
\leavevmode
\bfseries The bottom part of the description, would normally go here. This is textbox3 and the text can just be entered by typing in the environment. Any language can be used and also footnote as for example\footnote{Footnote example.}.
\egroup
\end{commands}

\begin{commands}[]{}
\begin{verbatim}
\bgroup
\makeatletter
\fboxrule0px
\fboxsep0px
\parindent0pt
\sffamily
\mbox{}%
\centerline{\huge\bfseries \mainhead@cx}%
\vskip2\baselineskip
\makedoublehead
\centering
\hfill\hfill\fbox{\begin{minipage}[t]{0.55\textwidth}%
\null
\centering
\includegraphics[width=1.03\linewidth]{./images/1948-02.jpg}%
\vspace*{20pt}
\end{minipage}}\hfil%
\fbox{\begin{minipage}[t]{.4\linewidth}%
\fbox{\begin{minipage}[t]{\textwidth}%
\null
\lipsum[1]
\end{minipage}}%
\par
\leavevmode\vskip0pt\rule{1px}{30px}\vskip0pt
\fbox{\begin{minipage}[t]{\textwidth}%
\null
\lipsum[1]
\end{minipage}}%
\end{minipage}}%\hfill
\par
\leavevmode
\bfseries The bottom part of the description, would normally go here... 
 example\footnote{Footnote example.}.
\egroup
\end{verbatim}
\end{commands}
  \restoregeometry
  \chapter{Captions}

\parindent1em

\section{Setting the caption options}

Captions are very visual and both the text as well as its typography need careful consideration. Most readers will read the captions of figures, before reading the text. We will now in the sections that follow use the caption package to change all the parameters of the caption. This is achieved mainly through one macro, with key value styles.


%\DeclareRobustCommand\acaption{\protect\RaggedRight Lorem ipsum caption \protect\ldots.}
\def\acaption{Lorem ipsum caption \ldots}
\begin{figure*}[h]
\captionsetup{format=plain}
\captionsetup{skip=3pt}
\captionsetup{font=small}
\captionsetup{name=Fig}
\captionsetup[figure]{labelfont=bf,textfont=it}
\RaggedRight
\centering 
\begin{minipage}[t]{90pt}
 \includegraphics[width= 70pt]{./graphics/sudan.jpg}
 \caption{\acaption }
 \label{fig:shortlabel}
\end{minipage}
\captionsetup{name=Figure}
\begin{minipage}[t]{90pt}
 \includegraphics[width= 70pt]{./graphics/sudan.jpg}
 \caption{\acaption }
\end{minipage}
\captionsetup{name=Fig,labelsep=space}
\begin{minipage}[t]{90pt}
 \includegraphics[width= 70pt]{./graphics/sudan.jpg}
 \caption{\acaption }
\end{minipage}
\end{figure*}


To set the caption options we can use the \cmd{\captionsetup} with a set of options.
\begin{dispListing}
\captionsetup{name=Fig, labelsep=space}
\end{dispListing}




It is highly recommended to use the \texttt{caption} package to setup the captions of figures. This package developed by Axel Sommerfeldt offers customization of captions in floating environments such
figure and table and cooperates with many other packages. Most classes provide build-in options and commands for customizing captions. 

And if you are just interested in using the
command \cmd{\captionof}, loading of the very small \pkgname{capt-of package} is usually sufficient.

For wrapped figures the label name is preferable to be shorter, otherwise it leads to text that is either underfull or overfull. You should also try and use the \cmd{\RaggedRight} option of the \pkgname{ragged2e} package to hyphenate the ragged right text.

Figure~\ref{fig:shortlabel}, has its label shortened by using ``Fig'' rather than "Figure". I have done this as the space available is narrow. The setup is achieved using the \texttt{caption} package's \verb+\captiosetup+ command. We will use this command to specify, the fonts, numbering, labels, separators and other parameters of the captions.

\subsection{Adjusting the label}%%

The \emph{label} is the name of the figure. Sometimes it is abbreviated, sometimes it is not. Adjusting the label, is achieved by setting the key parameter |name| in \cmd{\captionsetup}. 

\begin{commands}[]{}
\cmd{\captionsetup}\marg{name=Figure}
\end{commands}



The figures were typeset by using a different setup style. The first one displays the  label fully, the second uses an abbreviation and the third has a new line, before the caption text is displayed.

\subsection{Fonts}

There are three font options which affects different parts of the caption: One affecting the
whole caption (font), one which only affects the caption label and separator (labelfont) and at least one which only affects the caption text (textfont). You set them up using the options shown in the table below:

\begin{table}[htp]
\centering
\smaller
\caption{Key values for fonts, using the caption package}
\begin{tabular}{ll}
\toprule
normalfont &Normal shape\\
up &Upright shape\\
it &Italic shape \\
sl &Slanted shape\\
sc & \textsc{Small Caps Shape}\\
md &Medium series\\
bf &Bold series\\
rm &Roman family\\
sf &Sans Serif family\\
tt &Typewriter family\\
\bottomrule
\end{tabular}
\end{table}

\emphasis{captionsetup,captionof}
\begin{teXXX}
\captionsetup{name=Figure.}
\captionof{figure}{\acaption}
\end{teXXX}


\begin{figure*}[h]
\begin{commands}[]{}
\captionsetup{skip=3pt}
\captionsetup{font=small}
\captionsetup{name=Fig}
\captionsetup{labelfont=bf,textfont=it, format=plain}
\RaggedRight
\centering 
\begin{minipage}[t]{90pt}
 \includegraphics[width= 70pt]{./graphics/sudan.jpg}
 \caption{\acaption }
\end{minipage}
\captionsetup{name=Figure}
\begin{minipage}[t]{90pt}
 \includegraphics[width= 70pt]{./graphics/sudan.jpg}
 \caption{\acaption }
\end{minipage}
\captionsetup{name=Fig,labelsep=space}
\begin{minipage}[t]{90pt}
 \includegraphics[width= 70pt]{./graphics/sudan.jpg}
 \caption{\acaption }
\end{minipage}
 \caption{Three boys example (changing the figure name).}
 \end{commands}
\end{figure*}


\section{Adjusting the Separator}


The separator can be adjusted in a similar manner. The package offers the options, \option{none}, \option{colon}, \option{period}, \option{space}, \option{quad}, \option{newline} and \option{enddash}.  The various options are illustrated
in \hbox{Figures~18-23}.


\section{Adjusting spacing before and after the figure}

Skips are the amount of vertical space between the caption and the figure. The caption package offers the option
\option{skip=amount}.\footnote{The standard \LaTeX\ classes article, report and book preset it to \option{skip=10pt}.} We will now make some recommendations as to how to adjust this spacing.

\medskip

\begin{figure}[htp]
\everypar{}
\captionsetup{name=Photo,parindent=0pt,minmargin=0pt,width=3sp,labelsep=period,skip=5pt,margin={0pt,0pt},position=bottom}

\noindent\includegraphics[width=\textwidth]{./graphics/damageinspection.jpg}

\noindent\caption{Damage Inspection.A squadron operations officer of the 332d Fighter Group points out a cannon hole to ground crew, Italy, 1945.}\par
\end{figure}

\medskip

The space between the image and the caption should be approximately half the point size of the text. The photo above had the following settings:


\begin{teX}

\captionsetup{name=Photo, labelsep=period,
                    skip=5pt, font=scriptsize,
                    position=bottom, margin{0pt,0pt}}
\end{teX}

The \docAuxCommand{caption} command offered by \latexe has a design flaw\footnote{According to Axel Sommerfeldt, \textit{see} the \textit{Caption} documentation.}: The command does not
know if it stands on the beginning of the figure or table, or at the end. Therefore it does
not know where to put the space separating the caption from the content of the figure
or table. While the standard implementation always puts the space above the caption
in floating environments (and inconsistently below the caption in longtables), the
implementation offered by this package is more flexible: By giving the option
\option{position=bottom}, the package correctly inserts the skip.  You can also try the \option{position=auto}.
\medskip

The caption of the next photograph follows a more traditional approach found in
\begin{figure}[htp]
\vskip10pt
\centering
\captionsetup{name=Photo, labelsep=period, position=bottom, textfont=scriptsize, justification=centering}
\includegraphics[width=\textwidth]{./graphics/korea.jpg}

\caption*{\textsc{25th Division Troops Unload Trucks and Equipment}\par
\textit{at Sasebo Railway Station, Japan, for transport to Korea, 1950.}}
\vskip10pt
\end{figure}
many books where, there is no label or number and the text is split into two lines. The first line is a photograph heading and the second line is printed in italics with some explanatory stuff about the photo.

To achieve this result we need to firstly use the \emph{starred} form of the caption command and override the formatting commands of the caption.

\begin{teX}
\begin{figure}[htp]
\vskip10pt
\centering
\captionsetup{...}
\includegraphics[width=\textwidth]{filename}
\caption*{\textsc{25th Division Troops Unload Trucks and Equipment}\par
\textit{at Sasebo Railway Station, Japan, for transport to Korea, 1950.}}
\vskip10pt
\end{figure}
\end{teX}

You will notice that the photograph is between the lines of the paragraph, so I have added some small skips to arrange proper spacing around it.


To my knowledge, you cannot customize the caption package to get the heading for the caption text. You can define your own command to do so:
\begin{phdverbatim}
\newcommand\captionx[2]{\par%
     \leavevmode 
     \caption*{\textsc{#1}\par%
     \textit{#1}}%
}
\end{phdverbatim}

\DeclareDocumentCommand\captionx{m m}{%
     \leavevmode
     \caption{\textsc{#1} %
     \textit{#2}}%
}

With photographs you need sometimes to add a "credit" to credit the photographer or even a copyright notice. This is necessary, especially if you have licensed images from an agency. For this I would prefer a simple solution where we
just define an \verb+addcredit+ macro. More customization might be possible, as well as a few setup macros. As an exercise have a look at some publications and see how they handle this type of photographs.

\begin{teX}
\newcommand\addcredit[1]{%
   \vspace*{-10.5pt}%
   \scriptsize
   \hfill\hfill
   \textit{Credit: #1}%
}
\end{teX}

\providecommand\addcredit[1]{%
 \scriptsize%
 \vspace*{-10.5pt}%
 \hfill\hfill\textit{Credit: #1}%
 \vspace{10pt}
}

The results of the code so farm can be seen in the photograph that follows. The credit has been added and
the text has been centered and styled as required.

The full code is now shown below:

\begin{teX}
\begin{figure}[htp]
  \centering
  \captionsetup{skip=0pt,  justification=centering}%
  \includegraphics[width=\textwidth]{./graphics/rosenberg.jpg}%
  \addcredit{U.S. DoD.}%
  \captionx{Assistant Secretary Rosenberg}{talks ...}
\end{figure}
\end{teX}

\begin{figure}[htp]
  \centering
  \captionsetup{name=Photo, labelsep=period, skip=0pt, position=top, textfont=scriptsize,    justification=centering}%
\includegraphics[width=\textwidth]{./graphics/rosenberg.jpg}%
\addcredit{U.S. DoD.}%
\captionx{Assistant Secretary Rosenberg}{talks with men of the 140th Medium Tank Battalion during a Far East tour.}
\vspace{10pt}
\end{figure}

It all looks perfect, but there is a snag. If the photo is narrower, there will be nothing to stop it floating past the edge of the photo. This can be corrected by enclosing the commands within a minipage.


\begin{figure}[htp]
\begin{commands}[]{}
\captionsetup{name=Fig., labelsep=period, format=plain, margin{30pt,30pt}}%
\includegraphics[width=0.97\textwidth]{./graphics/movingup.jpg}%
\addcredit{U.S. DoD.}%
\caption{The effects of the credit going past the edge of the figure. This can be corrected by adding a minipage to hold both the include graaphics, as well as the addcredit command. }

\begin{verbatim}
\begin{figure}[htp]
  \captionsetup{name=Fig., labelsep=period, format=plain}%
  \includegraphics[width=0.97\textwidth]{./graphics/movingup.jpg}%
  \addcredit{U.S. DoD.}%
  \caption{The effects of the credit going past the edge of the figure. This can be corrected by adding a minipage to hold both the include graaphics, as well as the addcredit command. }
\end{figure}
\end{verbatim}
\end{commands}
\end{figure}

\section{Presentation}

Presentation of a lot of figures can influence the appearance of a book tremendously. Like sectioning commands and text styling, magazines and books can be recognized from the styling of their pictures. In figures we imitated the appearance of photographs in Life Magazine. Life in the forties was in the front with the troops and had some great photographers.  It had a style still very hard to improve on.

\begin{figure}[htp]
\bgroup
\parindent=0pt
\null
\clearcaptionsetup{figure}
\captionsetup{style=default,name=Photo.,skip=3pt,parindent=0pt, labelsep=period, margin={0pt,0pt}}%
\begin{minipage}[t]{0.48\textwidth}%
      \includegraphics[width=\textwidth]{./graphics/movingup.jpg}%
      \addcredit{U.S. DoD.}\vskip1sp
     \caption{The effects of the credit going past the edge of the figure. This can be corrected by adding a minipage to hold both commands.}%
\end{minipage}\hfill\hfill
\begin{minipage}[t]{0.48\textwidth}
\includegraphics[width=\linewidth]{./graphics/survivors.jpg}%
%      \addcredit{U.S. DoD.}%
\caption{The effects of the credit going past the edge of the figure. This can be corrected by adding a minipage to hold both commands. }
    
\end{minipage}

 \begin{minipage}[t]{0.48\linewidth}
      \includegraphics[width=\linewidth]{./graphics/img009.jpg}%
      \addcredit{U.S. DoD.}%
     \caption{Engineer Construction Troops in Liberia, July 1942.}
\end{minipage}\hfill\hfill
\begin{minipage}[t]{0.48\textwidth}
      \includegraphics[width=\textwidth]{./graphics/survivors.jpg}%
      \addcredit{U.S. DoD.}%
     \caption{The effects of the credit going past the edge of the figure. This can be corrected by adding a minipage to hold both commands. }
\end{minipage}
\begin{minipage}[t]{0.48\textwidth}
      \includegraphics[width=\textwidth]{./graphics/img126.jpg}%
      \addcredit{U.S. DoD.}%
     \caption{Marine Reinforcements.
A light machine gun squad of 3d Battalion, 1st Marines, arrives during the battle for ``Boulder City.'' }
\end{minipage}\hfill\hfill
\begin{minipage}[t]{0.48\textwidth}
      \includegraphics[width=\textwidth]{./graphics/img124.jpg}%
      \addcredit{U.S. DoD.}%
     \caption{Brothers Under the Skin, inductees at Fort Sam Houston, Texas, 1953. }
\end{minipage}
\egroup
\end{figure}
\newpage


\endinput
  \cxset{steward,
  chapter toc=true,
  toc image=false,
  numbering=arabic,
  custom = stewart,
  offsety=0cm,
  image={./images/hine06.jpg},
  texti={A picture is worth a thousand words, but if you don't add a good description of what it is in a caption, your readers will be left scratching their heads. Here we discuss captions in general as well as the formatting commands available in LaTeX, some common packages and athena.},
  textii={In this chapter we discuss methods that allow the formatting and positioning of captions, based on a set of key values. Central  to this process is the separation of content from presentation.
We also discuss the basic formatting tools that are available and how one can modify them to blend them with the rest of the design.
 }
}
\cxset{section numbering prefix=\thechapter.}
\chapter{Typesetting Captions}
\section{Introduction}

Publications that include figures and tables will normally dictate
the style of captions. Captions, besides normal typography 
requirements such as fonts, can vary in their numbering scheme, can
include a label such as figure or fig they can include a colon or stop
after the label and can be centered hanged or left justified. 
Numbering can also vary; the counters can be reset at every chapter or section or can be continuous. So
there are quite a few options to define in a template.

The formatting commands for the captions key value interface follow the same style of the rest of the package. We use the \pkg{caption} package to provide the interface to the key value settings. To format the captions you just include the appropriate keys in one of the style
files.


\section{Conventions}

All caption keys start with the word |caption|. The float type follows, so |caption figure font-size| refers to the caption of a \textit{figure environment}. If the word \textit{figure} is omitted the style is applicable to both tables and figures. 

As users will probably only have to set these keys once, my recommendation is to use the longer version that can give you finer control. Also your template will be easier to modify in the future.
\medskip

{
\keyval{caption format}{\marg{plain|hang}}{This affects all captions such as tables and figures and will produce either a hang caption or with plain will wrap arund the figure number like a normal paragraph.}

\keyval{caption figure format}{\marg{plain | hang}}{Affects ONLY figure captions such as tables and figures and will produce either a hang caption or with plain will wrap around the figure number like a normal paragraph.}

\keyval{caption figure numbering style}{\marg{auto|continuous|reset on sections|custom}}{}
\keyval{caption figure numbering}{\marg{arabic|alph|Alph|roman|Roman|custom}}{Sets the style of numbering.}
\keyval{caption separator}{\marg{colonsemicolon|none|custom}}{Sets the separator, such as \textbf{:} or a colon or none.}
\keyval{caption label name}{\marg{text}}{Sets the label name such as figure.}
\keyval{caption aboveskip}{\marg{dim}}{Sets the \cs{belowcaptionskip}.}
\keyval{caption belowskip}{\marg{dim}}{Sets the \cs{abovecaptionskip}. You use as simply \texttt{10pt} ot similar. In LaTeX this value is normally set as \texttt{0pt}. Note that below a float normally an additional skip is introduced.}

\keyval{caption font}{\marg{bf|tt|it}}{Sets the font commands. }
\keyval{caption figure name}{Figure}{Sets the figure name}
\keyval{caption defaults}{\marg{true|false}}{Sets all styling back to default styles.}
}

Although it looks a simple piece of text, as you notice there are about
a dozen of variables that one could set. Color can be determined both
from the caption labl colour as well as from hyperlinking if necessary.
More complicated styles can be build in a simila fashion to chapter
heads, by diverting to a custom command \cs{captionspecial}. This
will be provided at the next release of the package.


\cxset{caption format/.code=\captionsetup[figure]{format=#1}} 


\begin{texexample}{}{}
\bgroup
\cxset{caption format = hang}
\includegraphics[width=80pt]{../graphics/sudan.jpg}
\captionof{figure}{This is a very long command to see how all
these can wrap in a hang format, if the text is longer than
a paragraph.}
\egroup

\bgroup
\cxset{caption format = plain}
\captionof{figure}{This is a very long command to see how all
these can wrap in a hang format, if the text is longer than
a paragraph.}
\egroup
\end{texexample}

As you can see from the example, the changes can also be localized if
they are within a group.



\makeatletter
\def\captionlabelfont@cx{bf}
\cxset{caption font/.code = \captionsetup[figure]{font=#1}}
\cxset{caption font={bf}}
\makeatother



\begin{texexample}{}{}
\cxset{caption format = hang}
\cxset{caption font={bf}}
\captionof{figure}{This is a very long command to see how all
these can wrap in a hang format, if the text is longer than
a paragraph.}
\end{texexample}



\section{Technical discussion}

The formatting of the caption, happens in stages like the sectioning commands.  |\@makecaption|  command is responsible for the typesetting and is defined in the standard LaTeX classes. The \cs{caption} and command is defined in the LaTeX kernel in the 
|float.dtx| class. As always we will start our discussion from the user command and follow it through to the typesetting macros.

When the user command \cs{caption} is processed, LaTeX checks if it is outside a float and if it is issues an error message. It then swallows the argument. It then calls \cs{@caption} which does further processing.

\startlineat{5}
\begin{teXXX}
\def\caption{%
  \ifx\@captype\@undefined
   \@latex@error{\noexpand\caption outside float}\@ehd
   \expandafter\@gobble
 \else
   \refstepcounter\@captype
  \expandafter\@firstofone
 \fi
 {\@dblarg{\@caption\@captype}}%
}

\long\def\@caption#1[#2]#3{%
  \par
  \addcontentsline{\csname ext@#1\endcsname}{#1}%
  {\protect\numberline{\csname the#1\endcsname}{\ignorespaces  #2}}%
  \begingroup
        \@parboxrestore
  \if@minipage
     \@setminipage
  \fi
  \normalsize
 \@makecaption{\csname fnum@#1\endcsname}{\ignorespaces #3}
 \par
 \endgroup}
\end{teXXX}


The \cs{@makecaption} is the main typesetting macro and this is
where we need to hook if we want finer grain of control.

\makeatletter
\cxset{label punctuation/.code = \gdef\labelpunctuation@cx{#1}}
\cxset{label space/.code = \gdef\labelhspace@cx{\hskip#1}}
\cxset{caption above skip/.store in= \abovecaptionskip@cx}
\cxset{caption above skip=10pt}
\makeatother

\captionof{figure}{This is a very long command to see how all
these can wrap in a hang format, if the text is longer than
a paragraph.}

\begin{texexample}{}{}
\cxset{caption format = hang}
\cxset{label punctuation=?}
\cxset{label space =1.5em}
\captionof{figure}{This is a very long command to see how all
these can wrap in a hang format, if the text is longer than
a paragraph.}

\end{texexample}



\cxset{label punctuation=?}
\captionof{figure}{This is a very long command to see how all
these can wrap in a hang format, if the text is longer than
a paragraph.}

\cxset{label punctuation=:}
\cxset{label space =.5em}
\def\figurename{\textbf{Figure}}

\makeatletter
\setlength\abovecaptionskip{\abovecaptionskip@cx}
\setlength\belowcaptionskip{0\p@}

\long\def\@makecaption#1#2{%
  \vskip\abovecaptionskip
  \sbox\@tempboxa{#1\labelpunctuation@cx #2}
  \ifdim \wd\@tempboxa >\hsize
    #1\labelpunctuation@cx\labelhspace@cx#2\par
  \else
    \global \@minipagefalse
    \hb@xt@\hsize{\hfil\box\@tempboxa\hfil}%
  \fi
  \vskip\belowcaptionskip}
\makeatother

\begin{teXXX}
\newlength\abovecaptionskip
\newlength\belowcaptionskip
\setlength\abovecaptionskip{10\p@}
\setlength\belowcaptionskip{0\p@}

\long\def\@makecaption#1#2{%
  \vskip\abovecaptionskip
  \sbox\@tempboxa{#1:: #2}
  \ifdim \wd\@tempboxa >\hsize
    #1:: #2\par
  \else
    \global \@minipagefalse
    \hb@xt@\hsize{\hfil\box\@tempboxa\hfil}%
  \fi
  \vskip\belowcaptionskip}
\end{teXXX}



\section{List of Figures}

\begin{docCommand}{listoffigures}{}
The list of figures (lof) is included on a page by using the command \cs{listoffigures}.
\end{docCommand}

The command is not defined in the kernel but rather in the standard classes as shown below. By default it uses the |\chapter| to typeset its heading. Commands like |\tableofcontents| that should set the marks in some page
styles use a |\@mkboth| command, which is |\let| by the pagestyle command |(\ps@...)| to |\markboth| for setting the heading or to |\@gobbletwo| to do nothing.\footnote{See source ltpage.dtx Date: 2000/06/02 Version v1.0k, page311.}

\begin{teXXX}
\newcommand\listoffigures{%
    \if@twocolumn
      \@restonecoltrue\onecolumn
    \else
      \@restonecolfalse
    \fi
    \chapter*{\listfigurename}%
      \@mkboth{\MakeUppercase\listfigurename}%
              {\MakeUppercase\listfigurename}%
    \@starttoc{lof}%
    \if@restonecol\twocolumn\fi
    }
\end{teXXX}



In the |phd| package this is set as a property via a key-value interface and hence we can use a normal chapter. If it need be we can define a special chapter style only for this heading. This way we can control all aspects of the formatting of the head.

\begin{docCommand}{\listfigurename}{}
The \textit{List of Figures} for example in many Social Sciences books is typed as {List of Illustrations} and also adds credits.
\end{docCommand}




\begin{figure}[htp]
\includegraphics[width=\textwidth]{./images/listofillustrations.jpg}
\caption{List of Illustrations extract from \textit{Oxford History of Art, Portraiture}, Shearer West, Oxford University Press, 2004.}
\end{figure}
\begin{figure}[htp]
\includegraphics[width=0.67\textwidth]{./images/titian.jpg}
\centering
\caption{Figure from \textit{Oxford History of Art, Portraiture}, Shearer West, Oxford University Press, 2004. The figures are numbered consecutively and the text in the List of Illustrations have different formatting.}
\end{figure}



\section{Formatting the List of Figures Heading}

LaTeX formats the list of figures heading in a similar manner to that of the Table of Contents. The Title `List of Figures` is obtained from the \cs{listfigurename} and which is also accessible from Babel. It does not add an entry to the ToC.



It is good to know that \cs{captionsetup} has an effect on the current environment only.
So if you want to change settings for the current figure or table only, just place the
\cs{captionsetup} command inside the figure or table right before the \cs{caption}
command.


Many of the caption figures can be changed within \latexe itself. For example to get continuous numbering in the book class.

\begin{teXXX}
\makeatletter
\@removefromreset{table}{chapter}
\renewcommand{\thetable}{\arabic{table}}
\makeatother
\end{teXXX}

\begin{docCommand}{removefromreset}{}
The command \cs{removefromreset} can be found by loading the \pkg{remreset} package. Other combinations are also possible.
\end{docCommand}

\subsection{Caption numbering scheme}

The caption numbering scheme key value interface, provides five
options: 
\medskip

\keyval{caption numbering scheme}{\marg{default|continuous| chapter|section}}{The numbering style either continous or reset per spacing etc...}


\begin{comment}
% Date: Sat, 30 Jul 1994 17:58:55 PST
% From: Donald Arseneau <asnd@erich.triumf.ca>
%
%  |\@removefromreset{FOO}{BAR}| : Removes counter FOO from the list of
%                       counters |\cl@BAR| to be reset when counter BAR
%                       is stepped.  The opposite of |\@addtoreset|.
\end{comment}


\begin{teXXX}

\makeatletter
\setdefaults
\cxset{chapter opening=anywhere,
          chapter font-size=\normalfont,
          title font-size=\large}

\def\@removefromreset#1#2{\let\@tempb\@elt
   \expandafter\let\expandafter\@tempa\csname c@#1\endcsname
   
   \def\@elt##1{\expandafter\ifx\csname c@##1\endcsname\@tempa\else
         \noexpand\@elt{##1}\fi}%
   \expandafter\edef\csname cl@#2\endcsname{\csname cl@#2\endcsname}%
   \let\@elt\@tempb}

\@removefromreset{figure}{chapter}
\renewcommand{\thefigure}{\arabic{figure}}

\@specialfalse\@tocfalse
\gdef\continuousfigures@cx{\@removefromreset{figure}{chapter}
%\gdef{\thefigure}{\arabic{figure}}}

\cxset{caption numbering continuous/.code={\continuousfigures@cx}}


\chapter{This is the First Chapter}

\captionof{figure}{test}

\captionof{figure}{test}

\chapter{This is the Second Chapter}

\captionof{figure}{test}
\captionof{figure}{test}
\makeatother

\end{teXXX}


\begin{figure}[htp]
\includegraphics[width=0.98\textwidth]{./images/captionspecial.jpg}
\centering
\caption{Figure from \textit{Oxford History of Art, Portraiture}, Shearer West, Oxford University Press, 2004. The figures are numbered consecutively and the text in the List of Illustrations have different formatting.}
\end{figure}




  \pagestyle{headings}
 
}

 \def\docboxing{%
    \part{BASIC TeX}
    %\chapter{Characters}


\normalsize

\tex\ works internally by translating characters into character codes. The way characters are encoded in a computer
may differ from system to system.\index{characters>encoding}


There are 256 characters that \tex\  might encounter at
each step, in a file or in a line of text typed directly on your terminal. These
256 characters are classified into 16 categories numbered 0 to 15:\index{characters>catcodes}\index{catcodes}


\arial
\begin{table}[htbp]
\centering
\begin{tabular}{rll}
\toprule
Code & Description & Representation\\
\midrule
0 & Escape character & (\textbackslash in this book)\\
1 & Beginning of group & (|{| in this book)\\
2 & End of group & (|\}| in this book )\\
3 & Math shift & (|\$| in this book)\\
4 & Alignment tab & (|\&| in this book)\\
5 & End of line &(return in this book)\\
6 & Parameter &(|\#| in this book\\
7 & Supescript &(|\^| in this book)\\
8 & Subscript &(|\_| in this bookl)\\
9 & Ignored character &(null in this manual)\\
10 & Space &(\textvisiblespace in this book)\\
11 &Letter &(A,\ldots,Z and a,\ldots z)\\
12 &Other character &(none of the above or below)\\
13 &Active character &(|\~| in this manual)\\
14 &Comment character &(|\%| in this book)\\
15 &Invalid character &(delete in this book)\\
\bottomrule
\end{tabular}
^^A\captionof{table}{Character Codes}
\end{table}
\medskip

When \tex reads a line of text from a file, or a line of text that you entered
directly on your keyboard, it converts that text into a list of \cmd{\tokens}. A
token is either (a) a single character with an attached category code, or (b) a control
sequence. For example, if the normal conventions of plain \tex  are in force, the text
\verb*+ `{\hskip 36 pt}'+  is converted into a list of \textit{eight} tokens:
\medskip

$ \{_{1}$ hskip $3_{12}~~6_{12}~~\_{10}~~p_{11}~~t_{11}~~\}_2 $

\medskip
The subscripts here are the category codes, as listed earlier:
\begin{itemize}
\item[1] for beginning of group,
\item[12] for |other| character," and so on. The |hskip| doesn't get a subscript, because it
represents a control sequence token instead of a character token. Notice that the space
after \cmd{hskip} does not get into the token list, because it follows a control word.
\end{itemize}

Knuth in the \texbook notes that:

\begin{quotation}

It is important to understand the idea of token lists, if you want to gain a
thorough understanding of \tex, and it is convenient to learn the concept by
thinking of \tex as if it were a living organism. The individual lines of input in your
files are seen only by \tex's \textit{eyes} and \textit{mouth}; but after that text has been gobbled
up, it is sent to \tex's \textit{stomach} in the form of a token list, and the digestive processes
that do the actual typesetting are based entirely on tokens. As far as the stomach is
concerned, the input 
flows in as a stream of tokens, somewhat as if your \tex manuscript
had been typed all on one extremely long line.
\end{quotation}

\section{Control sequences for characters}

\DescribeMacro{\char}
There are a number of ways in which a control sequence can denote a character. The \cmd{\char} command
specifies a character to be typeset; the \cmd{\let} command introduces a synonym for a character
token, that is, the combination of character code and category code.

\section{Denoting characters to be typeset: \texttt{char}}

\index{\string\char}
Characters can be denoted numerically by, for example, \verb+ \char98 +. This command tells \tex to add
character number 98 of the current font to the horizontal list currently under construction.

Instead of decimal notation, it is often more convenient to use octal or hexadecimal notation. For
octal the single quote is used: \verb+ \char’142+; hexadecimal uses the double quote: \verb+ \char"62+. Note that

\begin{texexample}{Characters}{ex:chars}
\bgroup
\ttfamily

\char65

\char`b

\char`\b

\char"70

\egroup
\end{texexample}

\verb+ \char`'62+  is incorrect; the process that replaces two quotes by a double quote works at a later
stage of processing (the visual processor) than number scanning (the execution processor).

Because of the explicit conversion to character codes by the back quote character it is also possible
to get a ‘b’ – provided that you are using a font organized a bit like the ASCII table – with \verb+ \char‘b+
or \verb+ \char‘\b+.

The \cmd{\char} command looks superficially a bit like the \verb+  ^^+ substitution mechanism.

Both mechanisms access characters without directly denoting them. However, the \verb+ ^^+ mechanism
operates in a very early stage of processing (in the input processor of \tex, but before category
code assignment); the \cmd{\char} command, on the other hand, comes in the final stages of processing.
In effect it says ‘typeset character number so-and-so’.

\CMDI{\Uchar} The LuaTeX expandable command \cmd{\Uchar} reads a number between 0 and 1,114,111 and expands to the
associated Unicode character. 

\DescribeMacro{\chardef}
There is a construction to let a control sequence stand for some character code: the \cmd{\chardef}
command. The syntax of this is\\
\cs{chardef}\meta{control sequence}=\meta{number},\\
where the number can be an explicit representation or a counter value, but it can also be a character
code obtained using the left quote command (see above; the full definition of hnumberi is
given in Chapter 7). In the plain format the latter possibility is used in definitions such as

\verb+ \chardef\%=‘\%+

or 

\verb+ \chardef\%=37   +

command to typeset character 37 (usually the per cent character).\index{characters!percent character}

A control sequence that has been defined with a \cmd{\chardef} command can also be used as a hnumberi.
This fact is used in allocation commands such as \verb+ \char{newbox}+ (see Chapters 7 and 31). Tokens defined
with \verb+ \char{mathchardef}+ can also be used this way.


But \tex\ actually provides another kind of number that makes it unnecessary
for you to know texttt{ASCII} at all! The token `12 (left quote), when followed by
any character token or by any control sequence token whose name is a single character,
stands for \tex's internal code for the character in question. For example, \verb+\char`b+ and
\verb+ \char`\b+ are also equivalent to \verb+ \char98+. If you look in Appendix B to see how \verb+ \%+ is
defined, you'll notice that the definition is

\verb+\def\%{\char`\%}+

instead of \verb+ \char37+  as claimed above.

\section{Special notation for invisible characters}

\tex has a standard way to refer to the invisible characters of |ASCII|: 

Code 0 can be typed as the sequence of three characters \verb+ ^^@+, code 1 can be typed
\verb+ ^^A+, and so on up to code 31, which is \verb+ ^^_  +(see Appendix C). If the character following
\verb+ ^^+ has an internal code between 64 and 127, TEX subtracts 64 from the code; if the
code is between 0 and 63, \tex adds 64. 

Hence code 127 can be typed \verb+^^?+, and
the dangerous bend sign can be obtained by saying \verb+{\manual^^?}+. However, you must
change the category code of character 127 before using it, since this character ordinarily
has category 15 (invalid); say, e.g., 

\verb+ \catcode`\^^?=12 +

The \verb+ ^^+ notation is different from
\cmd{\char}, because \verb+ ^^+ combinations are like single characters; for example, it would not
be permissible to say \verb+ \catcode`\char127+, but \verb+^^+ symbols can even be used as letters within control words.

\begin{texexample}{Special notation}{ex:texbook1}
\def\^^zz{test}
\^^zz
\end{texexample}


Most of the \verb+ ^^+ codes are unimportant except in unusual applications. But
\verb+ ^^M+ is particularly noteworthy because it is code 13, the |ASCII| return that
\tex normally places at the right end of every line of your input file. By changing the
category of \verb+ ^^M+  you can obtain useful special effects, as we shall see later.

\section{Upper and Lowercase characters}

\verb*+\lccode +the character code for the lowercase form of a letter (p. 103)

\DescribeMacro{\lowercase}
\DescribeMacro{\uppercase}
The twin operations \cmd{\uppercase}\marg{token list} and \cmd{\lowercase}\marg{token listi}
go through a given token list and convert all of the character tokens to their
\cmd{\uppercase}  or \cmd{lowercase} equivalents.

\begin{texexample}{Uppercase and Lowercase}{ex:lowercase} 
\uppercase{abcdefgh} 
\lowercase{ABCDEFGH}
\end{texexample}

Here's how: Each of the 256 possible characters
has two associated values called the \cmd{\uccode} and the \cmd{lccode}; these values are
changeable just as a \cmd{\catcode} is. Conversion to uppercase means that a character
is replaced by its \cmd{\uccode} value, unless the \cmd{\uccode} value is zero (when no change
is made). Conversion to lowercase is similar, using the
\verb+  \lccode+. The category codes
aren't changed. 

When INITEX begins, all \verb+ \uccode+ and \verb+ \lccode+ values are zero except
that the letters a to z and A to Z have \verb+\uccode+ values A to Z and \verb+\lccode+ values a to z.

These tow control sequences are used to build a hash table, mapping all the capital and lowercase letters to their respective character codes.
(see pg 41 TeXbook)

\section{Some Practical Examples}

If you are typesetting anything that has to do with \tex\ or \latex\ you are bound to have to typeset a lot of commands. This short code below will change the category code of the \texttt{"} (double quote) to be the active command. This way anything between double quotes will be  typed out verbatim and in a Maroon color. By mainipulating the \cmd{catcode} of characters we can achieve this.

\begin{teX}
%% Code to catch commands
\def\Meaningless#1>{}
\catcode`\"=\active
\def\startV{\leavevmode\begingroup
  \ifdim 0pt=\lastskip\penalty200 \fi
  \catcode`\{11 \catcode`\}11 \catcode`\%11
  \moreV}
\long\def\moreV#1"{%
  \def\LtxCode{#1}%
  \ignorespaces
      \expandafter\Meaningless\meaning\LtxCode
      \unskip%
  \endgroup}
\let"\startV

\bgroup
\catcode`\<=\active
\def<#1>{\ensuremath{\langle\mbox{\textsl{#1}}\rangle}}
\end{teX}

\begin{comment}
\bgroup
\def\Meaningless#1>{}
\catcode`\"=\active
\def\startV{\leavevmode\begingroup
  \ifdim 0pt=\lastskip\penalty200 \fi
  \catcode`\{11 \catcode`\}11 \catcode`\%11
  \moreV}
\long\def\moreV#1"{%
  \def\LtxCode{#1}%
  \ignorespaces
      \expandafter\Meaningless\meaning\LtxCode
      \unskip%
  \endgroup}
\let"\startV

\catcode`\<=\active
\def<#1>{\ensuremath{\langle\mbox{\textsl{#1}}\rangle}}

\noindent Testing it out with a few commands we get 
"\catcode", "\char" ,"\def" etc. We will revert back to this short example later on in our book, when you have learned a bit more about macros and programming \tex\. Note that this also affects "quotes".

\egroup
\end{comment}

A more complex example is the \pkg{shortvrb} package code.

\begin{teX}
%% Copyright (C) 1989-1999 Frank Mittelbach, all rights reserved.
\def\MakeShortVerb{%
  \@ifstar
    {\def\@shortvrbdef{\verb*}\@MakeShortVerb}%
    {\def\@shortvrbdef{\verb}\@MakeShortVerb}}

\def\@MakeShortVerb#1{%
  \expandafter\ifx\csname cc\string#1\endcsname\relax
    \@shortvrbinfo{Made }{#1}\@shortvrbdef
    \add@special{#1}%
    \expandafter
    \xdef\csname cc\string#1\endcsname{\the\catcode`#1}%
    \begingroup
      \catcode`\~\active  \lccode`\~`#1%
      \lowercase{%
      \global\expandafter\let
         \csname ac\string#1\endcsname~%
      \expandafter\gdef\expandafter~\expandafter{\@shortvrbdef~}}%
    \endgroup
    \global\catcode`#1\active
  \else
    \@shortvrbinfo\@empty{#1 already}{\@empty\verb(*)}%
  \fi}
\def\DeleteShortVerb#1{%
  \expandafter\ifx\csname cc\string#1\endcsname\relax
    \@shortvrbinfo\@empty{#1 not}{\@empty\verb(*)}%
  \else
    \@shortvrbinfo{Deleted }{#1 as}{\@empty\verb(*)}%
    \rem@special{#1}%
    \global\catcode`#1\csname cc\string#1\endcsname
    \global \expandafter\let \csname cc\string#1\endcsname \relax
    \ifnum\catcode`#1=\active
      \begingroup
        \catcode`\~\active   \lccode`\~`#1%
        \lowercase{%
          \global\expandafter\let\expandafter~%
          \csname ac\string#1\endcsname}%
      \endgroup \fi \fi}
\def\@shortvrbinfo#1#2#3{%
  \PackageInfo{shortvrb}{%
     #1\expandafter\@gobble\string#2 a short reference
                                          for \expandafter\string#3}}
\def\add@special#1{%
  \rem@special{#1}%
  \expandafter\gdef\expandafter\dospecials\expandafter
    {\dospecials \do #1}%
  \expandafter\gdef\expandafter\@sanitize\expandafter
    {\@sanitize \@makeother #1}}
\def\rem@special#1{%
  \def\do##1{%
    \ifnum`#1=`##1 \else \noexpand\do\noexpand##1\fi}%
  \xdef\dospecials{\dospecials}%
  \begingroup
    \def\@makeother##1{%
      \ifnum`#1=`##1 \else \noexpand\@makeother\noexpand##1\fi}%
    \xdef\@sanitize{\@sanitize}%
  \endgroup}
\endinput
%%
%% End of file `shortvrb.sty'.
\end{teX}

We will spent the rest of the book in trying to understand and write code like this. My ultimate aim is  to be able to produce \tex\ code like any other program. 

\section{Example}

In this example we wish to redefine some of the active codes to act as text only:

\begin{teX}
\newenvironment{plaintext}{%
        \catcode`\$12
        \def\&{&}%
        \catcode`\&12
        \def\_{_}%
        \catcode`\_12
        \def\^{^}%
        \catcode`\^12
        \catcode`\#12
        \catcode`\%12
        \let\~~%
        \catcode`\~12
}{}
\end{teX}

\newenvironment{plaintext}{%
        \catcode`\$12
        \def\&{&}%
        \catcode`\&12
        \def\_{_}%
        \catcode`\_12
        \def\^{^}%
        \catcode`\^12
        \catcode`\#12
        \catcode`\%12
        \let\~~%
        \catcode`\~12
}{}
Use it like

\begin{plaintext}
Here is some test text % ^ & _ $ # &.

How about some math \(x\_y\^z\). You're still out of luck with braces
though.
\end{plaintext}

\begingroup
\catcode`\{=11 
\catcode`\}=11
\catcode`\[=1
\catcode`\]=2

{This is a test}

\endgroup


\section{Checking to see the meaning of a control sequence
}
Finding out just what a control sequence has been defined to be with |\let| can be done using

%\meaning: the sequence

\begin{teXXX}
\let\x=3 \meaning\x
\end{teXXX}
\graybox{
gives 'the character 3'.}




    \parindent1em

\chapter{Boxes and glue in TeX}

\setlength{\columnsep}{2em}
{\it Once you understand \tex\rq{}s concept of glue, you may well decide that
it was misnamed; real glue doesn't stretch or shrink in such ways, nor does it
contribute much space between boxes that it welds together. Another word like
\emph{spring} would be much closer to the essential idea, since springs have a natural
width, and since different springs compress and expand at different rates
under tension. But whenever the author has suggested changing \tex's terminology,
numerous people have said that they like the word \emph{glue} in spite of its
inappropriateness; so the original name has stuck. }
\smallskip

{\hfill  ---  Donald E. Knuth}

\medskip   


\parindent1em




\newthought{Traditional typesetting} was a task that depended on assembling the types and inserting them one by one on holding frames. In a way it was an assembly of boxes.
The \tex typesetting system uses a similar model of boxes to typeset content but in addition it also uses the concept of glue to stretch or shrink the text so that it will look better typographically. Boxes contain
typeset objects, such as text, mathematical displays, and pictures, and glue
is flexible space that can stretch and/or shrink by amounts that are under
user control.

\begin{figure}[h]
\hbox{\drawfontbox{Qwerty}\drawfontbox{fjord}}
\caption{Everything is boxes.}
\end{figure}

\begin{center}
\printfontparams
\end{center}

\section*{Boxes}

Boxes in \tex have  a rectangular shape but have
three associated measurements called \emph{height}, \emph{width}, and \emph{depth}.
Figure \ref{fig:boxes} shows a 
picture of a typical box, showing its so-called \emph{reference point} and \emph{baseline}

The reason that they have three dimensions is that a character of text has normally three dimensions as shown in figure \ref{fig:boxes}. As characters need to be lined on a baseline, the depth provides a datum point on which they can be aligned and the depth provides a measure of the portion of the character that is below the baseline.


Boxes and glue are the main tools of \tex. The box can hold text and other items. Glue is simply spacing. It can be horizontal or veritcal spacing, and it can be made as rigid or as flexible as desired.



\textbf{One important feature of \tex is that it has no knowledge of the shape of the characters it typesets, just the dimensions of each character box.}


When \tex is typesetting, it is normally in horizontal mode, such as while
it is working on this paragraph. Otherwise, \tex can be in vertical mode, or
in math mode, or three others described in Chapter 13 of The \texbook.
Two low-level TEX commands for boxes are \cmd{hbox}, for a horizontal box,
and \cmd{vbox} for a box in vertical mode. In the latter, \tex is normally still
collecting material for display from right to left: it is not building up a
column of text, as in classical Chinese writing.

In both kinds of boxes, the result is an unbreakable object that acts
much like a single character. \tex reads input as a string of characters,
then breaks that string up in words, each of which forms a box. 
\emph{Word boxes}\index{word boxes}
are then collected into lines, lines into paragraphs, and paragraphs into a
page galley. The space between the words can be normal \emph{interword space},
or \emph{sentence-ending} space, which is somewhat larger in English-language
typesetting, and the space is normally glue, rather than of fixed size.

\tex has a sophisticated mathematical algorithm for figuring out the
best way to stretch or shrink interbox glue to optimize the appearance of
lines and paragraphs. Every so often, \tex checks to see whether it has
enough material saved on the growing page galley to fill a complete output
page, and it asynchronously (and effectively, unpredictably) calls the
output routine whose job it is to figure out where the page break should
happen, ship out a completed page to the |DVI|  file, and replace the galley by
whatever is left over.

In traditional \tex you  can force a line break with the carriage-return command \docAuxCommand{cr},
and a page break with the command \docAuxCommand{eject}, but \tex is an expert system,
and normally handles line and page breaking on its own. \latex provides its own commands such as \cmd{\clearpage} and \cmd{\newpage} and so do all other \tex based formats and systems.

\latex does not modify any of \tex's algorithms but simply it is a set of implemenation
macros.

\section{Units of measurement in TEX}

\tex allows you to specify sizes of typographical objects in any of nine different
units:


\begin{table}[htbp]
\begin{center}
\begin{tabular}{llp{5cm}}
\toprule
bp &big point &1 inch is exactly 72 bp; the PostScript pagedescription language uses these units, but just calls them points\\
cc &cicero: &1 cc is exactly 12 didot points, and is thus the European  analogue of the pica\\
cm &centimeter: &1 in is exactly 2.54 cm\\
dd &didot point: &1 dd is (1238/1157) pt, and is a typographical unit common in some parts of Europe\\
in &inch: &an archaic unit, roughly the width of a man's thumb; it has been discarded by most countries, but still used in the USA and its sattelites.\\
mm &millimeter: &1 in is exactly 25.4 mm\\
pc &pica: &1 pc is exactly 12 pt\\
pt &printer's point: &1 in is exactly 72.27 pt\\ 
sp &scaled point: &1 pt is exactly $2^{16}$ = 65536 sp.\\
\bottomrule
\end{tabular}
\end{center}
\end{table}



The units can be separated from their numeric value with optional space, so
\texttt{3pc} and \verb*+3 pc+ are equivalent. The little half box in the latter is a convenient
way to indicate explicit spaces in typewriter text. It can be printed by typing \cmd{\char32} and a suitable font or using |\textvisiblespace|.

Internally, \tex\ stores dimensions as integral numbers of scaled points:
1 sp is tiny ---  smaller than the wavelength of visible light.\footnote{The visible light has wavelengths from 380--450 nm for violet up to 620--750 nm for red (sp = 280 nm} It is sometimes
useful to create objects that small so that they differ from empty objects,
but are nevertheless invisible. It also ensures that TeX will look the same irrespective on which computer you actually compiled your document.

\tex deals only with 32-bit integer words, and does not take advantage
of extra precision available on historical machines with larger words. The
lower 16 bits of a dimension can be viewed as a fractional number of points,
and the uppermost bit is needed for a sign (0 for plus, 1 for minus). That
leaves 15 bits to hold an integral number of points, but TEX only expects 14
to be used, so that addition of two dimensions does not overflow. Thus, the
largest dimension in TEX is exactly 214 + (1 26) points, or about 5.758  
meters or 18.89 feet. 

\tex has several kinds of special storage locations, called registers, numbered
from 0 to 255. For example,\cs{dimen0} can hold a fixed dimension,
which can be specified in any of the nine units of measurement that are
recognized by \tex.

Here is how you can assign a dimension to a register, and then have \tex
display it back for you:

\begin{dispListing}
\dimen1 = 25.4mm (*@\protect\footnote{You shouldn't assign dimenensions to primitive registers, but rather use one of the allocation schemes provided by \latex to do so.}  @*)
\the\dimen1
\end{dispListing}


Notice that \tex’s output is always in points, showing that it converts different
input units to a common system of measurement.

You can convert a dimension to the much-smaller units of scaled points
by assigning it to another kind of \tex register designed to hold signed integers,
the\cs{count0} through\cs{count255} registers:

\begin{texexample}{}{}
\bgroup
  \dimen4 = 1pt
  \count4 = \dimen4
  \the\count4
\egroup
\end{texexample}

{\noindent This time we get the size as \texttt{sp} as 65536 }


You might have noticed that the conversion from inches to points was not
quite what we claimed in the summary of \tex units. Here is how to see the
differences:

\verb+\dimen1 = 1in+


\section{Skip registers}
\index{registers!skip}
\begin{docCommand}{skip}{}
\tex glue is specified as a fixed dimension, and optionally, with a plus and/
or minus dimension. Along with \cs{dimen} registers, TEX has glue registers,
called \cs{skip0} through \cs{skip255}. Here is how you can save glue settings in
 registers, and ask \tex to display the contents of one of them:
\end{docCommand}

\begin{texexample}{Skip counters}{ex:skipcounters}
\bgroup
  \skip1 = 10pt
  \skip2 = 10pt plus 3pt
  \skip3 = 10pt minus 2pt
  \skip4 = 10dd plus 3dd minus 2dd
  \the\skip4
  % 10.70007pt plus 3.21002pt minus 2.14001pt.
\egroup  
\end{texexample}


The four sample glue settings store, respectively, \textit{fixed glue}, \textit{stretchable
glue}, \textit{shrinkable glue}, and \textit{flexible glue} that can both stretch and shrink,
but only up to a specified amount. Interword and intersentence spaces are
generally defined with glue like this, so that if more stretch or shrink of
\index{glue}\index{glue!flexibe}\index{glue!stretchable}\index{glue!shrinkable}

\begin{teX}
\dimen2 = 72.27pt
\count1 = \dimen1
\count2 = \dimen2
\showthe \count1
> 4736286.
\showthe \count2
> 4736287.
\end{teX}

The two values differ by the tiny value 1 \textit{sp}, so we can in practice ignore
that difference. If we use higher-precision arithmetic, we find the exact
decimal equivalents of the fractions as

\begin{teX}
4 736 286=65 536 = 72.269 989 013 671 875;
4 736 287=65 536 = 72.270 004 272 460 937 5;
4 736 286.72=65 536 = 72.27
\end{teX}


Actually \tex uses that last relation as the definition of the conversion of
inches to scaled points, so that our assignment of 1 in to \verb+\dimen1+ has to
be rounded to the nearest integral number of scaled points. That is why
in the round-trip conversion from decimal to binary and back to decimal,
1 in became 72.26999 pt. \tex guarantees that its output decimal numbers
are always converted on input back to the original binary numbers from
whence they came. For more on the story of \tex’s I/O conversions, see [3].


Both \tex and \latex define a number of predefined dimensions and these are discussed in the relevant Chapters discussing the \latex kernel. For example you may come across the \docAuxCommand{jot}, which is defined by \latex as:

\begin{teX}
\newdimen\jot
\jot=3pt
\end{teX}

\begin{texexample}{jot}{ex:jot}
\bgroup
\parindent\jot
This is some sample text with a one |\jot| left indentation. Which is really too small to see in a paragraph.
\egroup
\end{texexample}

Defining |\parindent=jot| we can see that the indentation almost disappeared. This is obvious since |\jot| is used normally for maths.

Glue is the binder that lets \tex\ do its job. This chapter discusses some
preset forms of glue and their uses. Along with glue parameters there are a
number of special commands for inserting glue. The most interesting have
different degrees of infinity, namely |\hfil|, |\hfill|, |\hfilneg|, |\hss|,
and their vertical counterparts. 

Because of the special spacing requirements of mathematics, \tex\ defines
skips and spacing that are valid in mathematics mode only. Examples are the
are the special preset |mu| glues of |\thickmuskip|,
|\medmuskip|, |\thinmuskip|.
The |\newskip| and |\newmuskip| commands allocate skip (or really glue)
or muskip registers respectively for special uses. For example the specific
glues surrounding section heads are held in glue registers. 

It should be noted that the various |\...muskip| do not cause horizontal
spacing in math mode by themselves. They are used with |\mskip| to actually
cause the insertion of the glue.

These are all various forms of horizontal (or math) mode commands that insert
infinite quantities of glue. |\hfil|, |\hfill|, |\hfilneg|, and |\hss| insert
|plus 1fil|, |plus 1fill|, |minus 1fil|, and |plus 1fil minus 1fil|. 
The first two are {\it stretch} glues, the third is {\it shrink} glue and the
last is both. It
should be noted that when \tex\ tries to stretch or shrink glue values, they
vary according to their value. If there exists both |fil| and finite value
glue in a box or line, then all the stretch if it is |plus| glue 
or shrink if it is |minus| glue will be in
the section with the |fil| glue. If there is |fill| glue and either |fil| or
finite glue, then all the stretch or shrink will be in the |fill| sections.
Similarly with |filll| glue, which is not supplied in as readily usable form.
The difference in behaviour between |\hfil| and |\hss| is that |\hss| glue
will allow the contents of a box to spill outside without resulting in an
overfull box while |\hfil| will only fill or push contents to the edge of the
box. The major use for these glues are to center or to force stuff to either
edge of a box.  For instance this

\begin{texexample}{}{}
\def\aline{\vrule \hfil one \hfil two \hfil three 
         \hfill four \hfil five \hfil six 
         \hfill seven \hfil eight \hfil nine\vrule}

\aline

This is a preset |mu| glue. |\medmuskip = 4mu plus 2mu minus 4mu| for use
with |\mskip|. This is \hbox{\strut\vrule$\mskip\medmuskip$\vrule}.
\end{texexample}

So far we have been using in our examples the primitive \tex |\skip| to allocate
skip registers. This is dangerous, as they might have been defined elsewhere and our
definitions will overwrite them. Plain \tex and \latexe provide an allocation scheme
where this is done automatically. 

\begin{docCommand}{newskip}{}
The command \cs{newskip}\meta{skip name} assigns a new skip or glue register to
the name |\<skip name>|. 

Glue values may be assigned to it by |\<skip name> [=] <glue>|.
This assigns a new muskip or muglue register to the name |\<muskip name>|. Glue
values may be assigned to it by |\<muskip name> [=] <muglue>|.

This is a preset |mu| glue. |\thickmuskip = 5mu plus 5mu| for use
with |\mskip|. This is \hbox{\strut\vrule$\mskip\thickmuskip$\vrule}.

This is a preset |mu| glue. |\thinmuskip = 3mu| for use
with |\mskip|. This is \hbox{\strut\vrule$\mskip\thinmuskip$\vrule}
\end{docCommand}


\newskip\hides 
\hides= -1000pt% plus 1fill

atest atest

\hskip\hides atest

\hides= -1000pt  plus 1fill

atest atest

\hskip\hides atest

|\vfil \vfill|

|\vfilneg|

|\vss|

These are the vertical analogues to the infinite horizontal glues and act in
much the same manner. See the section on |\hfil ...|.
 



\normalsize



\section{Glue}

\tex joins the boxes it creates with some special mortar as Knuth writes, called glue. To understand how glue works we will
borrow a figure from the \tex Book.

\begin{figure}
  \centering
  \includegraphics[width=0.9\linewidth]{./images/glue.png}
  \caption{Glue in \TeX}
  \label{fig:glue}
\end{figure}


\section{How to specify glue}

The usual way to specify \textit{glue} to \tex is
$<dimen>< plus~dimen><minus~dimen>$

where the plus and minus are optional and assumed to be zero if not
present; plus\index{glue!plus} introduces the amount of stretchability\index{glue!stretchability}, minus introduces the amount of shrinkability \index{glue!shrinkability}. 

For example, Appendix B of the TexBook defines \cs{medskip} to be an abbreviation for
|\vskip6pt plus2pt minus2p|. The normal-space component of glue must always be
given as an explicit dimen, even when it is zero. The ability of \TeX to stretch and shrink this glue has given it its beautiful looks. Strangely enough, although the algorithm is public it has not been used widely in other software.



\subsection{hfil and hfill}

{\obeylines
{This text will be flush left.\hfil}
{\hfil This text will be flush right.}
{\hfil This text will be centered.\hfil}
{Some text flush left\hfil and some flush right.}
{Alpha\hfil centered between Alpha and Omega\hfil Omega}
{Five\hfil words\hfil equally\hfil spaced\hfil out.}
}

Consider the following definitions:

\begin{verbatim}
\def\centerlinea#1{\hfil#1\hfill}
\def\centerlineb#1{\hfill#1\hfill}
\def\centerlinec#1{\hss#1\hss}
We define quickly a \cs{lineX}\footnote{Strange but my \LaTeX\ distribution has not got on. (This definition is from \texttt{plain.sty}}

\def\lineX{\hbox to\hsize}
\def\lineX{\hbox to\hsize}
\def\centerlinea#1{\hfil#1\hfil}
\def\centerlineb#1{\hfill#1\hfill}
\def\centerlinec#1{\hss#1\hss}

\lineX{\centerlinea{\test}}
\lineX{\centerlineb{\test}}
\lineX{\centerlinec{\test}}
\centerline{\test}
\begin{center}\test\end{center}

\end{verbatim}


\section{Specifying glue amounts}

\tex glue is specified as a fixed dimension, and optionally, with a plus and
or minus dimension. Along with \cs{dimen} registers, TEX has glue registers,
called \cs{skip0} through \cs{skip255}. Here is how you can save glue settings in
\tex registers, and ask \tex to display the contents of one of them:

\begin{teX}
\skip1 = 10pt
\skip2 = 10pt plus 3pt
\skip3 = 10pt minus 2pt
\skip4 = 10dd plus 3dd minus 2dd
\the \skip4
\end{teX}


\texttt{> 10.70007pt plus 3.21002pt minus 2.14001pt}

The four sample glue settings store, respectively, {\em fixed glue}, {\em  stretchable
glue}, {\em shrinkable glue}, and {\em flexible glue}  that can both stretch and shrink,
but only up to a specified amount. Interword and intersentence spaces are
generally defined with glue like this, so that if more stretch or shrink of  a
re underfull (too little text to fill the line), or overfull (too much text in the
line).



\section{Overfull lines}

Although overfull lines are reported in the \tex log file, they can be hard
to find in the typeset document if they only stick out a little. To make
them highly visible while you are fine tuning your final document, assign
the variable \cs{overfullrule} a nonzero dimension, such as 2 cm. \tex then
displays a solid black box, called a \emph{rule}, of that width in the right margin
on each line that is overfull. Using the \docpkg{microtype} package one can adjust the parameters to minimize this.

To make the rules disappear, simply remove it,
or comment out, the assignment, or reset its value to 0 pt. 

Just as you can assign dimension registers to count registers to convert
from points to scaled points, you can assign skip registers to dimension and
count registers to discard the flexible parts:


\begin{teX}
\skip1 = 10pt plus 3pt minus 2pt
\the\skip1
 \dimen1 = \skip1
\the \dimen1
\count1 = \skip1
\the \count1
\end{teX}




\section{More on glue in boxes}

Besides normal glue with fixed amounts of stretch and shrink, \tex also has
two kinds of glue that are \emph{infinitely} stretchable and shrinkable: \cs{hfil} and
\cs{hfill} in horizontal mode, and \cs{vfil} and \cs{vfill} in vertical mode. Notice that there two versions
of the commands, the one ends with one ell and the second one with two. The
two-ell forms are more flexible than the one-ell forms.

The boxes and glue model is powerful, and \tex's author, Donald Knuth,
has written that he views it as the key idea that he discovered when he
first sat down in 1977--1978 to design a computer program for typesetting.
For example, to set something flush left, put infinitely-stretchable glue on
its right. To set it flush right, put the glue on the left. For centered material,
put the glue on both sides. Here are four examples, with vertical
bars marking the ends of the horizontal box (boxes have no visible frames,
although it is possible to write \tex commands to give them such outlines,
and we use that feature shortly):





\section{Horizontal and vertical boxes}


\begin{docCommand}{hbox}{\marg{material}}
\end{docCommand}

Like their dimensions \TeX's boxes are not what one thinks when thinking of boxes. TeX's boxes come in basically two flavours, horizontal boxes and vertical boxes. An \cs{hbox} is created by the command \refCom{hbox}\marg{material}. It has the following properties:

\begin{enumerate}
\item The material is placed from left to right and it becomes a \textit{horizontal list}.\index{horizontal list}
\item The box \textbf{cannot be broken across lines}; it is an indivisible unit.
\end{enumerate}

An |hbox| can contain, characters, horizontal glue, horizontal leaders or other boxes. While in many cases these other boxes can be other |\hbox|es, |\vbox| can be used.


The \refCom{hbox} command has another form |\hbox to <dimen>|\marg{material}. This
creates a box whose width is the given (dimen). Thus |\hbox to lcm{<material>}|
will create a 1 inch wide box \hbox to 1cm{text}. However, we have to supply exactly 1 cm worth of
material to fill up the box; otherwise we end up with an error message. It is best
to consider this form of the command as a promise; we promise '\tex that we will
supply just enough material to fill up the box. 

We can place other hboxes in an hbox. By adding glue we can then move them left or right

\begin{texexample}{hbox and glue}{ex:hbox}
\bgroup
\Huge
\hbox to \textwidth{\hfill \hbox{\EOofficerI}\hbox{\EOofficerII}\hbox{\EOofficerIII} \hfill}

\hbox to \textwidth{\hfill \hbox{\EOofficerI}\hbox{\EOofficerII}\hbox{\EOofficerIII} \hfil}

\hbox to \textwidth{\hfill \hbox{\EOofficerI}\hfill \hbox{\EOofficerII}\hfill \hbox{\EOofficerIII} \hfill}
\egroup
\end{texexample}

The last command that affects the shape of an |\hbox| is 'spread(dimen)', which
spreads the box beyond its natural width. An |\hbox spread12pt|{(material)}
makes the box 12 points wider than its natural size. If the material in the box has
no flexibility, it cannot spread to fill up the additional space, resulting in an underfull
box. This is why 'spread' is normally used with flexible glues.

\begin{texexample}{hbox and glue}{ex:hbox}
\bgroup
\LARGE
\hbox to 5cm{\EOofficerI\EOofficerII\hfill\EOofficerIII}

\hbox spread5cm{\hfill\EOofficerI\hfill\EOofficerII\hfil\EOofficerIII}

\hbox spread9cm{\EOofficerI\hfill\EOofficerII\hfil\hfil\EOofficerIII}

\hbox spread7cm{\EOofficerI\hfill\EOofficerII\hfil\hfil\EOofficerIII} 


\makeatletter
\hb@xt@ 5cm {\EOofficerI\EOofficerII\hfill\EOofficerIII}
\makeatother
\egroup
\end{texexample}



Boxes can be moved up or down using |\raise| or |\lower|. Each of these primitives is followed by a dimension indicating how far the box can be lowered or raised.

Other material that can go in an hbox, is \textbf{vertical rules}. 

\subsection{The null macro}

The |\null| macro is defined both in Plain as well as LaTeX and generates an empty box. Its definition is:

\begin{teXXX}
\def\null{\hbox{}}
\end{teXXX}


\fbox{\hbox{This is a test}}

{
\fbox{\hsize=5cm
A test of a box at the end of a 2.0 inch line\par}

\fbox{\hsize=5.0cm in A test of a box at the end of a \hbox to 2cm{2.0 cm} line\par}

}

What happens when we have more than two boxes on a line? TeX will stuck them one under another. If they are enclosed within another hbox they will be inlined.



\begin{texexample}{}{}
\hbox to 1cm {A} \hbox to 1cm {B}

If we however, put them together in another |\hbox|, we get:

\hbox{\hbox to 1cm {A} \hbox to 1cm{B}}
\end{texexample}




An |\hbox| does not imply horizontal mode, so an attempt to start a paragraph with a box, for
instance
|\hbox to 0cm{\hss$\bullet$\hskip1em}Text ...|

will make the text following the box wind up one line below the box. It is necessary to switch
to horizontal mode explicitly, using for instance |\noindent| or |\leavevmode|. The latter is defined
using |\unhbox|, which is a horizontal command.


\begin{texexample}{}{}
\hbox to 0cm{\hss$\bullet$\hskip1em} Text ...


\leavevmode\hbox to 0cm{\hss$\bullet$\hskip1em} Text ...

\end{texexample}




\section{Kerning}


Using the command \cs{kern}, we can move boxes either left or right. Kerning is extensively used to build internal commands and we discuss it in more detail under the chapter for fonts.

\begin{docCommand}{kern}{\meta{dimen}}
A |\kern| is similar to glue [75], with two differences: (1) |\kern| is rigid; (2)
|\kern| specifies a point where a line, or a page, should not be broken. Since a box is
indivisible anyway, |\kern| is used in a box to indicate rigid spacing. It is interesting
to note that the same command, |\kern|, indicates horizontal spacing when used in
an |\hbox| and indicates vertical spacing when used in a |\vbox|.
\end{docCommand}
Consider two horizontal boxes, holding the letters A and V:
As you can observe, the letters AB are a bit afar, from what would be a visually pleasant arrangement, we can kern them as follows:
\medskip

\begin{teXXX}
\hbox{\Huge AV A\kern-5ptV}
\end{teXXX}
\medskip

Note that hbox, does not produce a frame. I~have used a frame |\fbox|, which will cover a bit later as well as scaled the image by 2, in order to see the effects more clearly.


\drawfontbox{\upshape\Huge FJord F\kern-5pt Jorp}





\noindent\begin{tabular}{ll}
|\hbox{\kern4pt A\kern8pt B\kern8pt C\kern4pt}| & \fbox{\hbox{\kern4pt A\kern8pt B\kern8pt C\kern4pt}} \\
~ &\\
\midrule
|\hbox{\kern4pt\raise1pt\hbox{A}|  & \fbox{\hbox{\kern4pt\raise1pt\hbox{A} \kern8pt BC\kern8pt\lower6pt\hbox{D} \kern4pt} \kern8pt BC\kern8pt\lower6pt\hbox{D}\kern4pt} \\
|\kern8pt BC|                      &\\
|\kern8pt\lower6pt\hbox{D}|        &\\
|\kern4pt}|                        &\\ 
|\kern8pt BC|                      &\\ 
|\kern8pt\lower6pt\hbox{D}|        &\\
|\kern4pt}| &\\
\midrule
\end{tabular}


\vbox{
\noindent\rule{\linewidth}{0.4pt}
\begin{minipage}{4.5cm}
 \begin{teXX}
\fbox{\hbox{\kern4pt A\kern8pt 
      B\kern8pt C\kern4pt}}
\end{teXX}
\end{minipage}
\hfill\hfill
\begin{minipage}{3cm}
\hfill\hfill\fbox{\hbox{\kern4pt A\kern8pt 
      B\kern8pt C\kern4pt}}
\end{minipage}

\medskip
\noindent\rule{\linewidth}{0.4pt}
}

Notice that an |\hbox| is constructed by setting its components side by side so that their \textit{baselines} are aligned. When \cs{raise}, \cs{lower} are used the baselines are no longer aligned. In such a case the baseline of the box is defined as the baseline shared by the components before any vertical movements. In the example above the box now has a depth, as a result of lowering |D|.


\vbox{
\noindent\rule{\linewidth}{0.4pt}
\begin{minipage}{4.5cm}
\begin{teXXX}
\hbox{\kern4pt\raise1pt\hbox{A} 
  \kern8pt BC\kern8pt
  \lower6pt\hbox{D} 
  \kern4pt} 
\end{teXXX}
\end{minipage}
\hfill
\begin{minipage}{3cm}
\fbox{\hbox{\kern4pt\raise1pt\hbox{A} 
\kern8pt BC\kern8pt\lower6pt\hbox{D} \kern4pt}}
\end{minipage}

\medskip
\noindent\rule{\linewidth}{0.4pt}
}



\noindent\textbf{Vertical boxes.}\quad A vertical box is build in a similar manner to that of a horizontal list, except it is composed of material in the \textit{vertical list}.
When horizontal boxes are added in the list, they are stuck on top of each other as shown in the example below. 
\medskip

\bgroup
\parindent0pt
\fbox{\vbox{\hsize=3cm\fbox{\hbox{ABCDEFGH}} \fbox{\hbox{AB}}}}
\egroup


\begin{docCommand}{vbox}{ to \meta{dimen}\marg{\meta{material}}}
Typesets a box in vertical mode.
\end{docCommand}

It is important to remember the two main differences between hboxes and vboxes. An hbox will expand to hold its material. If it need be it will overfill the line and produce an overful warning. A vbox will expand to hold its material. It is perfectly normal for a vbox to hold paragraphs, as shown. This is not possible with an hbox. However, the common pattern is for an |\hbox| to contain a |vbox| .

\begin{texexample}{hbox/vbox example}{ex:vbox}
\noindent\fbox{\vbox{\lorem\par\lorem\par}}

\hbox to \linewidth{\vbox{\lorem\par\lorem\par}}
\end{texexample}


\begin{docCommand}{hsize}{\meta{dimen}}
 Controls the width of text in a |vbox|.
\end{docCommand}

\noindent\textbf{Controlling the size of a vbox.}\quad What controls the size, is the containing environment. This in TeX, is specified using |\hsize|. In LaTeX this is controlled by an enclosing environment, maybe a minipage (which is build this way) or one of the page width parameters.


\begingroup
\parindent0pt
\fboxsep5pt
\hsize=3.9cm\footnotesize
\hfil\fbox{\vbox{\RaggedRight\lorem\par}} 
\hfil\fbox{\vbox{\RaggedRight\lorem\par}}
\hfil\fbox{\vbox{\RaggedRight\lorem\par}}\hfill
\endgroup
\captionof{figure}{Output to demonstrate the use of vboxes.}



The code to typest the boxes shown above follows:
\medskip
\emphasis{hsize}
\begin{teXXX}
\bgroup
\parindent0pt
\hsize=3.3cm\footnotesize
\hfil\fbox{\vbox{\lorem\par}} 
\hfil\fbox{\vbox{\lorem\par}}
\hfil\fbox{\vbox{\lorem\par}}
\hfill
\egroup
\end{teXXX}


Note, the use of \docAuxCommand{hsize}. We define the font size as |\footnotesize|. We have done this in order not to have overfull boxes--Latin words don't have a full set of hyphenation patterns in \latex. The macro |\lorem|, we have defined internally for this document. We place the code in a group in order not to affect the rest of the document.



\clearpage

\noindent\textbf{Vertical centering}\quad can be achieved by applying vertical infinite glue \cs{vfill}. In the example that follows, first we place two letters in individual |\hboxes| and we enclose them in a vbox. We apply |\vfill| both on top and at bottom.

\emphasis{\vfill}

\vbox{
\noindent\rule{\linewidth}{0.4pt}
\begin{minipage}{5cm}
\begin{teX}
\fbox{\vbox to 0.9cm{\vfil\hbox{M}\nointerlineskip\hbox{i}\vfil}} 
\end{teX}
\end{minipage}
\hfill
\begin{minipage}{3cm}
\hfill\fbox{\vbox to 0.9cm{\vfil\hbox{M}\nointerlineskip\hbox{i}\vfil}}\hfill\hfill 
\end{minipage}

\medskip
\noindent\rule{\linewidth}{0.4pt}
}



A |\vbox| can be combined with text and may appear anywhere within a paragraph. The baseline of the box will be aligned with the baseline of the current line.


\vbox{%
\noindent\rule{\linewidth}{0.4pt}}

\begin{teX}
A vbox can be placed within a paragraph \fbox{\vbox to 0.6cm{\vfil\hbox{M}\nointerlineskip\hbox{i}\vfil}} as shown here.


\hfill

A vbox can be placed within a paragraph \fbox{\vbox to 0.6cm{\vfil\hbox{M}
  \nointerlineskip\hbox{i}\vfil}} as shown here.

\end{teX}

\medskip
\noindent\rule{\linewidth}{0.4pt}






\noindent\textbf{Top alignment.}\quad\cs{vtop} is similar to a |\vbox|. The depth of this box is zero, since both A and B are capital letters. The width of this box is |\hsize|, since it contains text. 


\begin{codeexample}[]
\vtop{\hbox{A} \hbox{B}}
\end{codeexample}






Centering a picture in a box, both vertically and horizontally can be achieved using the methods we described so far.


\emphasis{hfill,hbox}
\begin{texexample}{}{}
     \fbox{%
          \vtop{\medskip
                    \hfill
                      \hbox{\includegraphics[width=1.5cm]{./images/amato.jpg}}%
                    \hfill 
                   \medskip%
                }%
      }%
\end{texexample}

\begin{texexample}{}{}
    \fbox{%
          \vtop{\medskip
                    \hfill
                      \hbox{\includegraphics[width=1.5cm]{./images/amato.jpg}}%
                      \hbox{\includegraphics[width=1.5cm]{./images/amato.jpg}}%
                      \hbox{\includegraphics[width=1.5cm]{./images/amato.jpg}}%    
                    \hfill 
                   \medskip%
                }%
      }%
\end{texexample}

Study the example a bit more carefully, as we have said earlier on that \cs{hbox}'es are stacked vertically, the reason why in the above example they are next to each other is that they are in an
\cs{fbox} which in turn is an \cs{hbox}  that can draw  frame around the box and is defined in the
\latex2e kernel.

So if we had only three images in hboxes we will get:

\begin{texexample}{Three Images Lined}{}
%\leavevmode
%\parindent30pt
\hbox{\includegraphics[width=1.5cm]{./images/amato.jpg}}%
\hbox{\includegraphics[width=1.5cm]{./images/amato.jpg}}%
\hbox{\includegraphics[width=1.5cm]{./images/amato.jpg}}%
\end{texexample}

An hbox does not start a paragraph. If we started a paragraph the behaviour will be different.

\begin{texexample}{Three Images Lined}{}
.\hbox{\includegraphics[width=1.5cm]{./images/amato.jpg}}%
\hbox{\includegraphics[width=1.5cm]{./images/amato.jpg}}%
\hbox{\includegraphics[width=1.5cm]{./images/amato.jpg}}%
\end{texexample}

If you notice carefully, we have started the paragraph by inserting a `.' before the first |\hbox|, an alternative way is to 
use |\leavevmode|. The effect of this command is to leave vertical mode, and to enter horizontal mode. Thus, if the mode is vmode (typically, outside any paragraph), a new paragraph is started. This paragraph may be flushed left, flushed right. 

\begin{texexample}{Three Images Lined}{}
\leavevmode
\hbox{\includegraphics[width=1.5cm]{./images/amato.jpg}}%
\hbox{\includegraphics[width=1.5cm]{./images/amato.jpg}}%
\hbox{\includegraphics[width=1.5cm]{./images/amato.jpg}}%

\meaning\leavevmode
\end{texexample}

The macro |\leavevmode| as its name implies forces \tex to leave vertical mode and enter horizontal mode. In this case the photos are just treated by \tex similarly to any character and tehy are typeset next to each other. 

\begin{docCommand}{kern}{}
If we wanted to add a bit of space between the horizontal images, we could use \cs{kern}
Kern again. This is from the book TeX for The Impatient page 157. You can use kern in math mode, but you cannot use the \texttt{mu} units. If you want to use \texttt{mu} units use \cs{mkern} instead.
\end{docCommand}

\emphasize{kern}
\begin{texexample}{}{}
\leavevmode
\hbox{\includegraphics[width=1.5cm]{./images/amato.jpg}}\kern10pt
\hbox{\includegraphics[width=1.5cm]{./images/amato.jpg}}\kern10pt
\hbox{\includegraphics[width=1.5cm]{./images/amato.jpg}}%
\end{texexample}

One needs to be careful as to where you issue |\leavevmode|. If it is in the middle of a paragraph it will have no effect.
\emphasize{This,is,some,text}
\begin{texexample}{Example with leavevmode}{}
This is some text
\leavevmode
\hbox{\includegraphics[width=1.5cm]{botticelli-34.jpg}}\kern10pt
\hbox{\includegraphics[width=1.5cm]{botticelli-34.jpg}}\kern10pt
\hbox{\includegraphics[width=1.5cm]{images/botticelli-34.jpg}}%
\end{texexample}

A very common way in \latex2e is to issue a |\par| command before |\leavevmode| to avoid this problem. Another way is to use
one of the |\ifvmode| or |\ifhmode| and act accordingly. We now fix our example and get what we want. 

\emphasize{par}
\begin{texexample}{}{}
This is some text
\par\leavevmode
\hbox{\includegraphics[width=1.5cm]{botticelli-34.jpg}}\kern10pt
\hbox{\includegraphics[width=1.5cm]{botticelli-34.jpg}}\kern10pt
\hbox{\includegraphics[width=1.5cm]{images/botticelli-34.jpg}}%
\end{texexample}

\begin{texexample}{}{}
   \HHUGE
   \fboxsep=0pt
   \fbox{%
          \vtop{\medskip
                    \hfill
                       \hbox{ H\kern10pt i\kern10pt j}%    
                       \hbox{ A\kern10pt C\kern10pt j}%
                    \hfill 
                   \medskip%
                }%
   }%
\end{texexample}

This example shows how letters are typeset and you can see that they are aligned at the baseline. They are no different than the eimage example that we have shown earlier, except we don't need the boxes.

\medskip

\vbox{
\noindent\rule{\linewidth}{0.4pt}
\begin{minipage}{4.9cm}
\begin{teX}
\centerline{$\Downarrow$}\kern 3pt%
\centerline{$\Longrightarrow$\kern 6pt% horizontal kern
  \textit{A note about kern}\kern 6pt
    $\Longleftarrow$}
\kern 3pt
\centerline{$\Uparrow$}  
\end{teX}
\end{minipage}
\hspace{0.3cm}
\begin{minipage}{4.5cm}
\centerline{$\Downarrow$}\kern 3pt%
\centerline{$\Longrightarrow$\kern 6pt% horizontal kern
  \textit{A note about kern}\kern 6pt
    $\Longleftarrow$}
\kern 3pt
\centerline{$\Uparrow$}
\end{minipage}

\medskip
\noindent\rule{\linewidth}{0.4pt}
}
\medskip

To make a point again, |\vbox| lines boxes at their bottom while, |\vtop| lines them at their top.

\medskip

\vbox{
\noindent\rule{\linewidth}{0.4pt}
\begin{minipage}{4.9cm}
\begin{teX}
 \hbox{\hsize=2cm \raggedright
\vbox to 0.5in{\hrule This box is .5in deep. \vfil\hrule}
\qquad
\vbox to 0.75in{\hrule This box is .75in deep. \vfil\hrule}
\qquad
\end{teX}
\end{minipage}
\hspace{0.3cm}
\begin{minipage}{4.5cm}
\hbox{\hsize=2cm \raggedright
\vbox to 0.5in{\hrule This box is .5in deep. \vfil\hrule}
\qquad
\vbox to 0.75in{\hrule This box is .75in deep. \vfil\hrule}
\qquad}
\end{minipage}

\medskip
\noindent\rule{\linewidth}{0.4pt}
}

\medskip


Trying the same with vtop

\medskip

\vbox{
\noindent\rule{\linewidth}{0.4pt}
\begin{minipage}{4.9cm}
\begin{teX}
 \hbox{\hsize=2cm \raggedright
\vbox to 0.5in{\hrule This box is .5in deep. \vfil\hrule}
\qquad
\vbox to 0.75in{\hrule This box is .75in deep. \vfil\hrule}
\qquad
\end{teX}
\end{minipage}
\hspace{0.3cm}
\begin{minipage}{4.5cm}
\hbox{\hsize=2cm \raggedright
\vtop to 0.5in{\hrule \smallskip This box is .5in deep. \vfil\hrule}
\qquad
\vtop to 0.75in{\hrule \smallskip This box is .75in deep. \vfil\hrule}
\qquad}

\hbox{\hsize=2cm \raggedright
\vbox to 0.5in{\hrule \smallskip This box is .5in deep. \vfil\hrule}
\qquad
\vbox to 0.75in{\hrule \smallskip This box is .75in deep. \vfil\hrule}
\qquad}
\end{minipage}

\medskip
\noindent\rule{\linewidth}{0.4pt}
}

\medskip

There are some other special macros defined by Plain TeX that we will only touch briefly here. One of them is \cs{underbar}{\index{Plain!\textbackslash underbar}.
The macro puts its argument into an hbox and underlines it.

\medskip

\vbox{
\noindent\rule{\linewidth}{0.4pt}
\begin{minipage}{4.9cm}
\begin{teX}
 \underbar{1,000,788.22}
\end{teX}
\end{minipage}
\hspace{0.4cm}
\begin{minipage}{4.0cm}
\medskip
\hfill\hfill{}\hspace*{1em}a1,000,700.22 \hfill

\smallskip

\hfill\[\underbar 1,000,788.22 \]\hfill
\end{minipage}

\medskip
\noindent\rule{\linewidth}{0.4pt}
}

\medskip


The \cs{everyvbox} command inserts a series of tokens at the beginning of every |\vbox|.


\medskip

\vbox{
\noindent\rule{\linewidth}{0.4pt}
\begin{minipage}{4.9cm}
\begin{teX}
 \everyvbox{$\bullet$}...
\end{teX}
\end{minipage}
\hspace{0.4cm}
\begin{minipage}{4.0cm}
\begingroup% Without this group, there are tons of problems!
   \everyvbox{$\bullet$}
   \global\setbox1=\vbox{This is a paragraph without an initial indent. It is   \the\hsize\ long lines.}
   \global\setbox2=\vtop{\copy1}
\endgroup
 \hbox{\box1} 

 \hbox{\box2}
\end{minipage}

\medskip
\noindent\rule{\linewidth}{0.4pt}
}

\medskip
Knuth in the TexBook Chapter 24, has some short description of the every commands. The `everyhbox` inserts a token list just before as its name implies a horizontal box.

Here is a short example. We define a `oneLineBox`, which is simply an hbox with some text and we add spread to spread the line. Using |\everybox| we add the letter \textbf{a} in each horizontal box. 


\tex considers the box overfull if the excess width of the box is larger than \cs{hfuzz} or \cs{hbadness} is less than 100. If I change  the badness to hbadness, I get 1000.

\medskip

\vbox{
\noindent\rule{\linewidth}{0.4pt}
\begin{minipage}{10.0cm}
\begin{teX}
 \begingroup
     \everyhbox{a}
     \def\oneLineBox#1#2%
     {%
          \hfuzz=0pt
          \overfullrule=0.25pt
          \setbox0=\hbox spread#2{#1}%
          \setbox1=\hbox{\the\badness}% 
          \setbox2=\hbox to 4.5cm{\box0\hfil\box1}%
          \box2
     }
     \oneLineBox{Badness of line }{-1em}
     \oneLineBox{Badness of line }{-0.54em}
     \oneLineBox{Badness of line }{-0.4em}
     \oneLineBox{Badness of line }{0em}
     \oneLineBox{Badness of line }{1em}
     \oneLineBox{Badness of line }{2em}
     \oneLineBox{Badness of line }{3em}
 \endgroup
\end{teX}
\end{minipage}


\begin{minipage}{10.0cm}
\begingroup
     \everyhbox{a}
     \def\oneLineBox#1#2%
     {%
          \hfuzz=0pt
          \overfullrule=0.25pt
          \setbox0=\hbox spread#2{#1}%
          \setbox1=\hbox{\the\badness}% 
          \setbox2=\hbox to 4.5cm{\box0\hfil\box1}%
          \box2
     }
     \oneLineBox{Badness of line }{-1em}
     \oneLineBox{Badness of line }{-0.54em}
     \oneLineBox{Badness of line }{-0.4em}
     \oneLineBox{Badness of line }{0em}
     \oneLineBox{Badness of line }{1em}
     \oneLineBox{Badness of line }{2em}
     \oneLineBox{Badness of line }{3em}
 \endgroup
\end{minipage}

\medskip
\noindent\rule{\linewidth}{0.4pt}
}

\medskip










\parindent1em




\section{More features of horizontal boxes}

Characters in the Latin alphabet have different shapes, and in most typefaces,
different widths. The letters \texttt{d f h k l t} have ascenders, making them
higher than the vowels \texttt{a e o u}, while the letters \texttt{f g j p q y} have descenders,
giving them added depth below the vowels. Similarly, an \texttt{m} is wider than
an \texttt{i}. 

\drawfontbox{(fjord)}

When \tex makes a normal horizontal box, the box width is the sum
of the widths of the characters, and the fixed parts of any glue, contained
in it. Shrink and stretch components of glue are discarded for the width
calculation. The box also has both a height above the baseline, the invisible
line on which the characters rest, and a depth below the baseline. The
depth is zero if there are no objects with descenders. The height and depth
are chosen from the largest vertical extents of the contained objects.

If you look carefully at typeset material, you will observe that, in most
typefaces, parentheses, brackets, and braces have both descenders and ascenders,
and the typeface designer usually makes their extents the maximum
among all of the characters in the design. This sample text shows
document: ( h g ) [ k j ] { l p }.

You can force TEX to choose a larger height and depth than normal when
you write a command for a horizontal box by ensuring that it has suitable
contents, such as an invisible vertical rule of zero width. The command

\verb+\hbox to 50pt {\vrule height 20pt depth 10pt width 0pt \it stuff}+

produces a box whose (invisible) outline looks like this: 

\hbox to 50pt {\vrule height 20pt depth 10pt width 0pt \it Great}

\drawfontbox{\kern 5pt\vrule height20pt depth 10pt width 1pt fjord}

The
three extents of the vertical rule can appear in any order, and any convenient
units.

In order to see the otherwise-invisible box edges in that example, we
used the \latex  built-in command \cs{fbox} to create a frame, and we eliminated
the default margin inside the frame by setting \cs{fboxsep = 0pt}. Plain \tex
does not have the \cs{fbox} command, but The TEXbook shows how to make
something like it on pp. 223 and 321.

One particular zero-width vertical rule is convenient for ensuring that
separate boxes all get the same height and depth. It has the height and
depth of parentheses in the normal prose font, and is given the macro name \refCom{strut}.
Its definition in the plain.tex file of macro definitions is roughly
equivalent to this:

\begin{docCommand}{strut}{}
\end{docCommand}

\begin{teX}
  \def \strut {\vrule height 8.5pt depth 3.5pt width 0pt}
\end{teX}

We insert a |\vrule| at the figure on the left below with a height of 20pt and a depth of 10pt. You can observe the difference on the right box, without the |\vrule|. The \textit{strut} is the blue line, which we gave a width of one point to make it visible. Real life struts, would have a width of 0pt and will not be visible. 

\drawfontbox{\kern5pt{\color{blue}\vrule height20pt depth 10pt width 1pt} fjord}
\drawfontbox{fjord}



\section{Horizontal alignment of boxes in TEX}
\fboxsep0.4pt

When horizontal boxes are set together, they are treated as separate words,
and therefore spaced accordingly. The input
\verb+ \fbox{one} \fbox{two} \fbox{three} \fbox{four}  +
produces  \fbox{one} \fbox{two} \fbox{three} \fbox{four}. As the example shows, we can put spaces
between them, or run them together so that they fit tightly.


\section{Vertical boxes in TEX}


\begin{minipage}{2.0in}
\begin{verbatim}
\noindent
\fbox{%
  \it
  \hbox to 80pt{%
     \parindent = 0pt
     \vbox to 30pt {%
         left text
         \vfil
         more left text%
     }%
  }%
}%
\end{verbatim}
\end{minipage}


%\noindent
\fbox{%
  \it
  \hbox to 80pt{%
     \parindent = 0pt
     \vbox to 30pt {%
         left text
         \vfil
         more left text%
     }%
  }%
}%

Firstly we use a noindent to ensure that the box is not indented. If you comment the\cs{fbox} out, you can see that the right amount of space has been left in the paragraph above.

\mbox{}
 
\noindent
\fbox{%
\it
\hbox to 80pt{%
\parindent = 0pt
\hsize = 80pt
\vbox to 30pt {\hfill right text
\vfil
\hfill more right text}
}%
}%



\noindent
\fbox{%
\it
\hbox to 80pt{%
\parindent = 0pt
\hsize = 80pt
\vbox to 30pt {\hfil center text
\vfil
 more center text \hfil}
}%
}%

We can aslo center the text for both lines, by modifying the code slightly.
\begin{teX}
\noindent
\fbox{%
\it \hbox to 80pt{
   \parindent = 0pt
   \hsize = 80pt
   \vbox to 30pt {
   center text \hfill
    \vfil
    \hfil more center text}
   }%
}%
\end{teX}


\noindent
\fbox{%
\it
\hbox to 80pt{%
\parindent = 0pt
\hsize = 80pt
\vbox to 30pt {\hfil center text
\vfil
\hfil more center text}
}%
}%



\chapter{Boxes with \protect\LaTeXe}

The \tex primitive commands have been abstracted by \latexe into more user friendly commands that are easier to use. One other reason for using these \LaTeX\ commands is that they are ``color safe''. Later on we will see other possibilities given by the \pkgname{color} or \pkgname{xcolor} package for drawing colored boxes, but we want to recall that the code for |\makebox| and the like has already a protection mechanism for colors, which the primitive commands do not have. \latexe also provides boxes that are self-aware of the width of their contents. For example |\fbox| will frame its contents in an |\hbox|. This simple task is very convoluted to achieve using basic \tex commands. 

\begin{docCommand}{framebox} {\marg{dim}}
One useful box command provided by \latex2e is \cmd{\framebox}. This command builds a box with any material you want to provide it with. The contents of this box are unbreakable, and as far as \tex is concerned it is treated the same way as it would treat a letter. 
\end{docCommand}

\begin{docCommand}{fboxsep}{\marg{dim}}
\end{docCommand}
\begin{docCommand}{fboxrule}{\marg{dim}}
Two associated lengths control the width of the rule and the space around the contents. We can change their default value by using |\setlength{\fboxsep}{0pt}| or just simply |\fboxsep=0pt| or even |\fboxsep0pt|. 
\end{docCommand}


Another interesting property is this: \emph{the contents of a box need not lie inside it}. You may have
noticed that, given the contents as an argument, the
|\framebox| command sets the dimensions of the box
to those of the contents (in reality, to the ``sub-boxes"
that compose the contents). But you can define the
dimensions explicitly as well. For example,

\begin{texexample}{framebox example}{ex:framebox}
|\framebox[13em]{Some text}|

\framebox[13em]{Some text}

\fcolorbox{theblue}{cyan}{Some text}

\end{texexample}

The box as is shown in the example will not break and it occupies more space than its contents. A second optional command allows us to typeset the contents, left, center or right.

\begin{codeexample}[vbox]
\fboxrule1pt

\framebox[13em][l]{Some text}\par

\framebox[13em][r]{Some text}\par

\framebox[13em][c]{Some text}\par

\framebox[1em][l]{Some text}\par

\framebox[1em][c]{Some text}\par

\framebox[1em][r]{Some text}\par
\end{codeexample}

As you can observe \latexe has abstracted the |\hfill| and similar commands and allows boxes to be constructed with ease. We have started the discussion with |\framebox|, but most practical uses of boxes is when they remain invisible.

\begin{docCommand}{makebox} { \oarg{width}\oarg{position}\marg{contents} } 
 is \latex's box workhorse.
 \end{docCommand}

The |source2e| manual states. If the width is missing, then position is also missing and |obj|  is put in an \cs{hbox} of its natural width. This is true as far as the looks are concerned, but not the behaviour, as you can see
from the following example is not an unqualified \cmd{\hbox} it is an hbox preceded by leavevmode.\footnote{\url{http://tex.stackexchange.com/questions/105585/latex2e-makebox-hbox}} This is of course good practice and brings consistency to the LaTeX kernel. I would recommend that you follow such practices in your own code. 

\begin{texexample}{}{}
\newbox\temp
\savebox\temp{test}
LaTeX

\makebox{test} \mbox{test}

TeX

\hbox{test} \hbox{test}

\indent\hbox{test} \hbox{test}

LaTeX with \cs{leavemode}

\makeatletter
\leavevmode\hbox to \wd\temp{test} \indent\hbox to \wd\temp{test}
\makeatother
\end{texexample}



\latex's analog of a\cs{hbox} is called \cs{mbox}. They are 
much the same thing, but \cs{mbox} is defined to be more widely usable. We have already used \latex's framed companion to \cs{mbox}, \cs{fbox}.

A horizontal box of specified width is provided in \latex with the command
\doccmd{makebox[width][position]\{contents\}}. Bracketed command arguments
in \latex are always optional. 

Here, the width is a \tex dimension,
and defaults to the natural width of the contents if not given. The position
is one of the letters \textbf{l} (flush left) or \textbf{r} (flush right); if it is omitted, the text
is centered in the box. If the specified width is smaller than needed, the
contents protrude from the box, and may overlap surrounding material. If
the specified width is zero, then we have equivalents of the TEX \cs{rlap} and
\cs{llap} commands.


Here are several examples of these three LATEX box commands:

{\obeylines
\mbox{stuff}

\fbox{stuff} 

|\makebox{stuff}|

|\makebox[40pt][l]{stuff}|

|\makebox[40pt][r]{stuff}|

|\makebox[0pt]{stuff}|

|\makebox[0pt][l]{stuff}|

|\makebox[0pt][r]{stuff}|
}



\subsection{Positioning boxes}

To help in positioning boxes within other objects, \latex provides the command
\docAuxCmd{raisebox} to raise and lower boxes:

\begin{teX}
\raisebox{raiselength}[height][depth]{contents}
\end{teX}

A negative first argument lowers the box, where the \cmd{\lowerbox} will lower the box. Here are some examples:

\begin{texexample}{Raising and lowering boxes}{ex:raise}
A \raisebox{10pt}{\fbox{upper}} A
upper
A \raisebox{10pt}{\
fbox{lower}} A
lower
A \fbox{\raisebox{10pt}[25pt]{\fbox{upper}}} A
upper
A \fbox{\raisebox{10pt}[
25pt]{\fbox{lower}}} A
lower
A \fbox{\raisebox{10pt}[25pt][15pt]{\fbox{upper}}} A
upper
A \fbox{\raisebox{10pt}[
25pt][15pt]{\fbox{lower}}} A
lower
\end{texexample}

\section{Paragraph Boxes}

\begin{docCommand}{parbox}{\oarg{position}\oarg{height}\oarg{innerpos}\marg{width}\marg{contents} }
  For longer strings of text, \latex provides the paragraph box \cs{parbox} 
\end{docCommand}


The optional position
is a letter \textbf{b} for alignment of the bottomline with the current baseline,
or \textbf{t} for alignment of the top line with the surrounding baseline. Without

The box can be used as if it were a letter or a word, so we can put it in
the middle of a sentence. The input

This is text \parbox{30pt}{\it and this is boxed text} and
this is more text.

This is text \fbox{\parbox{30pt}{\it and this is boxed text}}
and this is more text.
produces


Flush-right typesetting generally looks bad in narrow columns, so we
can insert a \cs{raggedright} command inside the last argument of the paragraph
box to get output like this:

\begin{texexample}{}{}

\parbox[b][120pt][t]{130pt}{\lorem}%
\hspace{1cm}%
\parbox[b][150pt][t]{130pt}{Only some short line of text here.}%



\parbox[b][120pt][t]{130pt}{\lorem}\hspace{1cm}\parbox[b][120pt][c]{130pt}{Only some short line of text here.}

\end{texexample}


\section{The minipage environment}

Another kind of paragraph box can be obtained in a more general, and
more powerful, way with the \docAuxEnv{minipage} environment:

\emphasis{minipage}
\begin{phdverbatim}
\begin{minipage}[position]{width}
   contents
\end{minipage}   
\end{phdverbatim}


The positioning works just like that for \verb+\parbox+, with alignment letters \textbf{b}
and \textbf{t}, and if they are omitted, a default of vertical centering.
In particular, verbatim text produced with the verb command is illegal
in macro arguments, so it cannot be used with \cs{fbox}, \cs{framebox}, \cs{makebox},
\cs{mbox}, or\cs{ parbox}, but it can be used inside a minipage. The input


\begin{texexample}{}{}
\begin{minipage}{170pt}
This is inline verbatim \verb=\verb|\%{}|=, and this
is a verbatim display:

\begin{verbatim}
#include <stdio.h>
#include <stdlib.h>
int main(void)
{
  printf("Hello, world\n");
  exit (EXIT_SUCCESS);
}
\end{verbatim}
\end{minipage}

\end{texexample}


A minipage can go everywhere and can hold virtually any content.


\section{Scaling and resizing boxes}

\begin{docCommand}{resizebox}{\marg{width}}{\marg{general material}}
Resizes the contents of a box
\end{docCommand}

The command \cs{resizebox}\marg{width}\marg{height}\marg{object} can be used with tabular to specify the height and width of a table. The following example shows how to resize a table to 8cm width while maintaining the original width/height ratio.

\begin{teX}
\resizebox{8cm}{!} {
  \begin{tabular}...
  \end{tabular}
}
\end{teX}

Alternatively you can use \cs{scalebox}{ratio}{object} in the same way but with ratios rather than fixed sizes:

\begin{teX}
\scalebox{0.7}{
  \begin{tabular}...
  \end{tabular}
}
\end{teX}

Both |\resizebox| and |\scalebox| require the \pkg{graphicx}\footfullcite{graphicx} package.
To tweak the space between columns (LaTeX will by default chose very tight columns), one can alter the column separation: |\setlength{\tabcolsep}{5pt}|. The default value is |6pt|.

The scalebox is great if you want to magnify a letter so that you can observe the design closer.

\bigskip
\noindent\begin{tabular}{|c|c|c|c|c|c|}\hline
Kp-Fonts & Kp-\textit{light} & CM & Palatino & Utopia & Times\\\hline\hline
\scalebox{2}{ag713} &
\scalebox{2}{\fontfamily{jkpl}\selectfont 7} &
\scalebox{2}{\fontfamily{lmr}\selectfont 713}  &
\scalebox{2}{\fontfamily{ppl}\selectfont 713}  &
\scalebox{2}{\fontfamily{put}\selectfont 7} &
\scalebox{2}{\fontfamily{ptm}\selectfont \oldstylenums{7}} \\\hline
\end{tabular}


\begin{teX}
\hspace{-6mm}\begin{tabular}{|c|c|c|c|c|c|}\hline
Kp-Fonts & Kp-\textit{light} & CM & Palatino & Utopia & Times\\
\hline\hline
  \scalebox{10}{a} &
  \scalebox{10}{\fontfamily{jkpl}\selectfont a} &
  \scalebox{10}{\fontfamily{lmr}\selectfont a}  &
  \scalebox{10}{\fontfamily{ppl}\selectfont 7}  &
  \scalebox{9.2}{\rule{0pt}{1.25ex}\fontfamily{put}\selectfont a} &
  \scalebox{10}{\fontfamily{ptm}\selectfont a}\\\hline
\end{tabular}
\end{teX}
\bigskip



\section{Glues with Negative and zero dimensions}

A box with a natural size of zero with the right glue amount can become very useful. For example the glue
|0pt plus1fil minus1fil| can stretch to infinity and also shring to minus infinity. Of course in the case of
\tex infinity is \docAuxCommand{maxdimen}. A \tex primitive is defined with this glue \refCom{hss}.

\begin{docCommand}{hss}{}
\end{docCommand}

There is also a corresponding \refCom{vss}.

\begin{docCommand}{vss}{}
\end{docCommand}


These macros place text on a full line either centred or left or right adjusted.

\begin{texexample}{}{}
\makeatletter
368 \def\@@line{\hb@xt@\hsize}
369 \def\leftline#1{\@@line{#1\hss}}
370 \def\rightline#1{\@@line{\hss#1}}
371 \def\centerline#1{\@@line{\hss#1\hss}}
\rlap
\llap
These macros place text to the left or right of the current reference point without
taking up space.
372 \def\rlap#1{\hb@xt@\z@{#1\hss}}
373 \def\llap#1{\hb@xt@\z@{\hss#1}}

$a\mathrel{\rlap{\;/}{=}}b $

{\Huge
\leavevmode
\rlap{Y}L
\rlap{C}\kern2.6pt\lower3.5pt\hbox{,}
}
\makeatother
\end{texexample}

\begin{docCommand}{rlap}{\marg{material}}

\end{docCommand}

Of course neither |llap| or |rlap| start a paragraph, so we need to use a |leavevmode| or one of the other ways to start a paragraph.

\begin{docCommand}{llap}{\marg{material}}
\end{docCommand}


\begin{docCommand}{smash}{\marg{material}}
The |\smash| command typesets the material with a height and depth of zero.
\end{docCommand}

\begin{docCommand}{phantom}{\meta{material}}
\end{docCommand}

\begin{docCommand}{vphantom}{\meta{material}}
\end{docCommand}

\begin{texexample}{Defining smash}{}
\bgroup
\def\smash{%
   \relax % \relax, in case this comes first in \halign
   \ifmmode
   \expandafter\mathpalette\expandafter\mathsm@sh
   \else
    \expandafter\makesm@sh
   \fi}
   
\def\makesm@sh#1{%
   \setbox\z@\hbox{\color@begingroup#1\color@endgroup}\finsm@sh}

\def\mathsm@sh#1#2{%
   \setbox\z@\hbox{$\m@th#1{#2}$}\finsm@sh}

\def\finsm@sh{\ht\z@\z@ \dp\z@\z@ \box\z@}
\egroup

\vbox{\smash {\hbox{A} } \hbox{B}} Test

\end{texexample}

\cxset{geometry units = pt,
       fontbox font=\Huge\upshape}

  


Consider the letters `Q' and `P', shown below. The capital letter `Q' has a depth of 1.72mm, we might wish to smash
it in a two line title block to reduce the line spacing between two consecutive lines. This can be accomplished with the
\refCom{smash} command.

\centerline{\drawfontbox{Q} \drawfontbox{P}}

Smashing it produces the following results.

\centerline{\drawfontbox{\vbox{\smash {\hbox{Q} } \hbox{P}}}  \drawfontbox{\vbox{\hbox{Q}  \hbox{P}}}}  

The command is more useful in math environments and is used extensively both by authors and package developers.

\begin{teXXX}
\def\rightarrowfill{$\m@th\smash-\mkern-7mu%
454 \cleaders\hbox{$\mkern-2mu\smash-\mkern-2mu$}\hfill
455 \mkern-7mu\mathord\rightarrow$}

456 \def\leftarrowfill{$\m@th\mathord\leftarrow\mkern-7mu%
457 \cleaders\hbox{$\mkern-2mu\smash-\mkern-2mu$}\hfill
458 \mkern-7mu\smash-$}
\end{teXXX}

Two further macros can be useful to authors of mathematical documents, \docAuxCommand*{phantom} and \docAuxCommand*{vphantom}. 

When typesetting roots, sometimes there are issues with heights. The following example
from \citetitle{mathmode}\footcite{mathmode} illustrates the point.

\begin{equation}
 \sqrt{a}\,%
 \sqrt{T}\,%
 \sqrt{2\alpha k_{B_1}T^i}\label{eq:root1}
\end{equation}

This can be corrected using \refCom{vphantom}. 

\begin{texexample}{Correcting height issues}{ex:sqrtheights}
\begin{equation}\label{eq:root2}
 \sqrt{a\vphantom{k_{B_1}T^i}}\,%
 \sqrt{T\vphantom{k_{B_1}T^i}}\,%
 \sqrt{2\alpha k_{B_1}T^i}
\end{equation}

\begin{equation}
x = \sqrt[3]{6+\sqrt[3]{6+\sqrt[3]{6+\sqrt[3]{6+\cdots}}}}
\end{equation}
\end{texexample}

Using \pkgname{amsmath} \docAuxCommand{smash} can be used for even better results when
using inline or displayed roots. It must be noted that \cs{smash} in \latexe is defined
without such an optional argument.



\makeatletter
\renewcommand{\smash}[1][tb]{%
\def\mb@t{\ht}\def\mb@b{\dp}\def\mb@tb{\ht\z@\z@\dp}%
\edef\finsm@sh{\csname mb@#1\endcsname\z@\z@ \box\z@}%
\ifmmode \@xp\mathpalette\@xp\mathsm@sh
\else \@xp\makesm@sh
\fi
}
\makeatother
This is a test $\sqrt{\lambda_{ki}}$ and $\smash[tb]{\sqrt{\lambda_{ki}}} $ 
\meaning\smash

\begin{docCommand}{smash}{ \oarg{position}\marg{argument} }
The optional argument for the position can take three values: \textbf{t} keeps the bottom and annihilates the top, \textbf{b} keeps the top and annihilates the bottom and \textbf{tb} which annihilates top and bottom. The latter is the default.
\end{docCommand}

\begin{texexample}{Use of Amsmath smash}{ex:amssmash}
xxx
\fbox{\rule{0.5cm}{2cm}}
\fbox{\rule[-1cm]{0.5cm}{2cm}}
\fbox{\smash{\rule{0.5cm}{2cm}}}
\fbox{\smash{\rule[-1cm]{0.5cm}{2cm}}}
\fbox{\raisebox{0pt}[0pt][0pt]{\rule[-1cm]{0.5cm}{2cm}}}
\fbox{\raisebox{-1cm}[0pt][0pt]{\rule{0.5cm}{2cm}}}
\end{texexample}


\begin{texexample}{The array environment}{ex:array2}
Thus to change $\frac34$ to a decimal divide $4$ into $3$
and we get $.75$ as a result, thus:
\[
\begin{array}{r@{}r@{}}
4 \; & \vline \; 3.00 \\\cline{2-2}
     &            .75
\end{array}
\]
To find the square root of a four-figure number
such as our example calls for, work it out in the
following manner:


\[
\arraycolsep=0em
\begin{array}{cccccccccccc}
\multicolumn{3}{c}{\text{2d pair}} &\qquad&\qquad&
\multicolumn{3}{c}{\text{1st pair}}&\qquad&\qquad&
\multicolumn{2}{c}{\text{square root}}\\
 & \overbrace{\quad}&\ZZZ&&&\ZZZ&\overbrace{\quad}&\ZZZ\\
 & 42 &&&&& 25 &&&&\vline\;65&(answer)\\\cline{11-11}
 & 36 &&&&& \\\cline{2-2}
\multirow{2}{*}{125\:} & \vline\hfill \phantom{Z}6 \hfill&&&&& 25\\
 & \vline\hfill \Zi6 \hfill&&&&& 25\\\cline{2-7}
\end{array}
\]
\end{texexample}

What I provided as an easy mnemonic for the \pkg{phd} I provided macros |\Zi, \ZZ, \ZZZ| as convenience aliases for
|\phantom{Z}| etc.

With graphic programs becoming available, most of the drawing of small complicated boxes, has been overtaken by using
\tikzname and especially its option to overlay a node at a particular point of the page without any impact on the spacing.








    %\cxset{custom = fashion,
%          fashion image=./images/venus.jpg}

\chapter{Rules and Leaders}
\pagestyle{headings}

\epigraph{He had forty-two boxes, all carefully packed,
With his name painted clearly on each:
But, since he omitted to mention the fact,
They were all left behind on the beach.}{---Lewis Carroll, The Hunting of the Snark}

\section{Rules}

Rules, both horizontal and vertical, are traditionally used in typesetting. In
\tex, a rule does not necessarily have to be long and thin; it has three dimensions,
like a box, and can have any rectangular shape. There are two types of rules, |\hrule| and |\vrule|.

\begin{docCommand}{hrule}{ height\meta{dimen} width \meta{dimen} depth\meta{dimen} }
Draws a rule in vertical mode.
\end{docCommand}

\begin{docCommand}{vrule}{ height\meta{dimen} width \meta{dimen} depth\meta{dimen} }
Draws a rule in horizontal mode.
\end{docCommand}

The shape of the rule does not depend on whether it is \textsc{h} or \textsc{v}, and the difference
between the two types is in the context in which they can be used, not in their
shapes. An |\hrule| is considered vertical material and can be part of a vertical list.

A |\vrule| is the opposite and can only appear in horizontal lists. The reason for
this convention is that a horizontal rule is a good separator between items stacked
vertically, whereas a vertical rule is a natural separator for items laid horizontally,
from left to right.

As a result, a |\vrule| should be used inside a paragraph, such as this \vrule, or in
an |\hbox|. An |\hrule| should be used between paragraphs or in a |\vbox|.

Any unspecified dimensions of a rule are determined [221] by these defaults:

\begin{enumerate}
\item The height of an |\hrule| is 0.4pt, and the depth is 0pt.
\item The width of a |\vrule| is 0.4pt.
\item Other dimensions are determined by extending the rule to the size of the smallest
box containing it. An example of this rule is the |\vrule| above. Its depth is set
equal to the depth of the line it happens to be on.
\end{enumerate}



The rule is extended to the width {\Huge \drawfontframe{\vbox{\hsize=24pt\parindent0pt p\hrule*}}}

\paragraph{Struts} The word \emph{strut} has already been mentioned. It refers to a \refCom{vrule} with width zero. It refers to a |\vrule| with
width zero. A standard strut is part of the plain format and is defined, on [353], as
|\vrule height8.5pt depth3. 5pt width0pt| (the actual definition is slightly more
complicated and takes into account the current mode). Such a rule does not show
up in print and is used to open up boxes. Inexperienced users find it hard to believe
that such a rule can be useful, but a glance at [478] shows that it is one of the most
frequently mentioned terms in the \texbook.

A horizontal strut can also be defined. It is an |\hrule| with height and depth
of zero. Surprisingly, such a thing is rarely used (but see discussion of |\hphantom|
in section 3.24)

\begin{texexample}{Drawing a Ruler}{ex:ruler}
\bgroup

\def\1{\vrule height 0pt depth 2pt}

\def\2{\vrule height 0pt depth 4pt}

\def\3{\vrule height 0pt depth 6pt}

\def\4{\vrule height 0pt depth 8pt}

\def\ruler#1#2#3{%
    \leftline{$\vcenter{%
    \hrule\hbox{\4#1}}\,\,\rm#2\,{#3}$}}%
  
\def\\#1{\hbox to .125in{\hfil#1}}
  
\def\8{\\\1\\\2\\\1\\\3\\\1\\\2\\\1\\\4}%
  
\ruler{\8\8\8\8}4{in}
\egroup
\end{texexample}

Lamport in \latex developed a macro |\rule| to enable users to draw lines without remembering all the rules for horizontal or vertical modes and the like.\footnote{In the latest releases this has been changed to a robust macro, using \textbackslash DeclareRobustCommand.}

\begin{docCommand}{rule}{\oarg{raised}\marg{width}\marg{height} }
Typesets a rule with a  \meta{width} and\meta{height}, raised by \meta{raised}.
\end{docCommand}

\begin{teX}
\def\rule{\@ifnextchar[\@rule{\@rule[\z@]}}%
\def\@rule[#1]#2#3{%
\leavevmode
\hbox{%
  \setlength\@tempdima{#1}%
  \setlength\@tempdimb{#2}%
  \setlength\@tempdimc{#3}%
  \advance\@tempdimc\@tempdima
  \vrule\@width\@tempdimb\@height\@tempdimc\@depth-\@tempdima}
}
\end{teX}

The important macro is |@rule| which sets the lengths and widths to the parameters required by the user. The raising of the rule is achieved by adjusting the depth to the given amount of length to raise the rule.

This is a Lamport rule |\rule[6.5pt]{4pt}{7pt}| typeset as:\rule[6.5pt]{4pt}{7pt} Many \latexe packages 
provide rules for common cases, such as \pkg{booktabs} providing rules that can be used in tables. 

Another useful \latexe macro is |\underline| that can be used to underline text. The \latex version is a modification of the \textsc{plain} version to enable it to be used in math mode. The \textsc{plain} version can still be used in \latexe by using |\@@underline|. 

\section{Applications}

One example of \refCom{vrule} is to provide the color background of a box. This method is used for
example by the \pkg{xcolor} to provide generic drivers. First a |vrule| with the require box dimensions
is typeset in a zero width box using \refCom{rlap} and then the text is overwritten to provide the typeset box, with a background color. One can extend such macros to draw numerous lines at different colors to also 
achieve  a gradient effect.


\begin{texexample}{}{}
\makeatletter
\bgroup
\renewcommand*\color@block[3]%
{{%
\color{blue}%
    \rlap{%
      \ifcolors@
        \vrule\@width#1\@height#2\@depth#3
      \fi
    }%
}} 
\hbox{\color@block{80pt}{30pt}{3.5pt}%
      \sffamily\bfseries\Huge\color{white}FFji}
\egroup 
\makeatother 
\end{texexample}

Of course the example is trivial. In a more detailed macro, it would be preferable to measure the dimensions
of the text and size the background accordingly. 

\section{Leaders}

A leader is a single copy of a pattern, for example in a dashed line a dash is a leader.
Dot leaders are a row of dots that visually connect the chapter titles and section headings to their corresponding page numbers. 

Leaders don't have to be composed of dots, with \tex leaders can be used fill a space with copies of a pattern,
\eg, to put repeated dots between a title and a page number in a table
of contents. 

The Plain Format provides six standard leader definitions. All these definitions are equivalent to an |\hfill| type of horizontal glue.

\medskip

\begin{tabular}{lp{3cm}}
\docAuxCommand{hrulefill}     & \hrulefill\\
\docAuxCommand{dotfill}        & x\dotfill x \\
\docAuxCommand{leftarrowfill} & \leftarrowfill\\
\docAuxCommand{rightarrowfill} & \rightarrowfill\\
\docAuxCommand{downbracefill} & \downbracefill\\
\docAuxCommand{upbracefill} & \upbracefill\\
\end{tabular}
\bigskip


A leader is a single copy of the pattern. The specification of
leaders contains three pieces of information:

\begin{enumerate}
\item  what a single leader is
\item  how much space needs to be filled
\item  how the copies of the pattern should be arranged within the space
\end{enumerate}

In \tex leaders are actually \emph{visual glue}. Wherever glue can go a row of leaders can go.

\begin{texexample}{Leaders}{ex:leaders}
\meaning\dotfill  \par
\meaning\hrulefill\par
\meaning\downbracefill\par
\end{texexample}

\begin{docCommand}{leaders}{}
\tex applies an imaginary window and only those leader boxes are printed which fully fit into the window. This ensures that the leader dots of different lines line up vertically.
\end{docCommand}


\begin{docCommand}{cleaders}{}
\end{docCommand}

\begin{docCommand}{xleaders}{}
\tex  provides three commands for specifying leaders:\cs{leaders},\cs{cleaders},
and\cs{xleaders} (p.~174). The argument of each command specifies the
leader. The command must be followed by glue; the size of the glue specifies
how much space is to be filled. The choice of command determines how
the leaders are arranged within the space.
\end{docCommand}

Rule leaders \textit{fill} the specified amount of space with a rule extending in the direction of the skip
specified. \index{rules and leaders>rule leaders}

The most common application for leaders is to fill the space with either a rule or with dots, such as shown in Example~\ref{leaders} below.

\emphasis{leaders,hbox,hfill}
\begin{texexample}{Leader example}{leaders}
\hbox{Exa\leaders\hrule\hskip20pt e}
\hbox to \linewidth{Section 1.2 \leaders\hbox{..}\hfill\space 15}
Section 1.3 \leaders\hbox{..}\hfill\space 15

\parfillskip=0pt plus1fil

\lipsum*[1]\leaders\hbox{..}\hfill\space 15
\end{texexample}

Leaders must be in a box, such as an \cs{hbox}. If they are not in a box an error is issued by \tex.

\begin{texexample}{}{hboxleaders}
\hbox to \textwidth{g\leaders\hbox{+}\hfill 112}
\end{texexample}

because a horizontal rule has a default height of |.4pt|. On the other hand,\index{Rules and Leaders>default value}

\verb+\hbox{g\leaders\vrule\hskip10pt f}+

gives

\hbox{g\leaders\vrule\hskip10pt f}

because the height and depth of a vertical rule by default fill the surrounding box.
Spurious rule dimensions are ignored: in horizontal mode

\verb+\leaders\hrule width 10pt \hskip 20pt+

is equivalent to

\verb+\leaders\hrule \hskip 20pt+

If the width or height-plus-depth of either the skip or the box is negative, TEX uses ordinary glue
instead of leaders.

\section{Box leaders}
\index{leaders box}
Box leaders fill the available spaces with copies of a given box, instead of with a rule. The first example uses \latex3 syntax, which is bound to send old \tex masters into an apoplectic fit. However, once your eyes
and brains absorb the syntax, \latex3 is too good to be ignored and can be mastered in a month or so. The
underscores still bother me, as well as the Hungarian notation, but I have mellowed as I grew older and
have now accepted it as an essential toolbox for latexing.

The reason I introduced it here, is to get you used to it for the next chapter, which is dedicated to \latex3 boxes and skips. This will bring us to a full round. We have studied the original \tex and plain format commands, the \latex2e and next the \latex3 macros. 

\begin{texexample}{Box leaders}{}
\ExplSyntaxOn  
  \box_new:N \starbox
  %\setbox\starbox=\hbox:n{
  \hbox_set:Nn \starbox 
    {
      \skip_horizontal:n { .2em  }
      \box_move_down:nn { 2.5pt }
                        {\hbox:n{*}}
      \skip_horizontal:n {.2em}
    }

  
  \hbox_to_wd:nn {\textwidth} 
    {
       \null \tex_leaders:D\box_use:N \starbox \hfill \null
    }.
\ExplSyntaxOff
\end{texexample}

If you notice you have to use the \cs{copy} command rather than \cs{usebox}, as we cannot use the |\leavevmode| with leaders

\begin{verbatim}
\usebox unchanged
81 \def\usebox#1{\leavevmode\copy #1\relax}
\end{verbatim}

That is, copies of the box register fill up the available space.

Dot leaders, as in the above example, are often used for tables of contents. In such applications it
is desirable that dots on subsequent lines are vertically aligned. The\cs{leaders} command does this
automatically:


The mechanism behind this is the following: TEX acts as if an infinite row of boxes starts (invisibly)
at the left edge of the surrounding box, and the row of copies actually placed is merely the part of
this row that is not obscured by the other contents of the box.

Stated differently, box leaders are a window on an infinite row of boxes, and the row starts at the
left edge of the surrounding box. Consider the following example:

\begin{texexample}{}{}
\hbox to 8cm {\leaders\copy\centerdot\hfil}
\hbox to 8cm {word\leaders\copy\centerdot\hfil}
\end{texexample}

which gives,

\hbox to 8cm {\leaders\copy\centerdot\hfil}
\hbox to 8cm {word\leaders\copy\centerdot\hfil}

The row of leaders boxes becomes visible as soon as it does not coincide with other material.
The above discussion only talked about leaders in horizontal mode. Leaders can equally well be
placed in vertical mode; for box leaders the \textit{infinite row} then starts at the top of the surrounding
box.


\begin{docCommand}{cleaders}{}
\begin{docCommand}{xleaders}{}
The \cs{cleaders} command is similar to 
\cs{leaders}, but it splits excess space before and after the leaders into two equal parts, centring the row of boxes in the available space.
The \cs{xleaders} command is also similar, but spreads the space between and after the leaders evenly between all the boxes.
\end{docCommand}
\end{docCommand}

The differences are best explained with an example.

\emphasis{leaders,cleaders,xleaders}
\begin{texexample}{}{}
\def\leaderpattern{\hbox{\kern0.5em-\kern0.5em-\kern0.5em-}}
Lorem \leaders\leaderpattern\hfill 13\par
Lorem \cleaders\leaderpattern\hfill 13\par
Lorem \xleaders\leaderpattern\hfill 13\par

\meaning\xleaders
\end{texexample}




\section{Vertical leaders}

If vertical glue commands such as \cs{vfill} is used it is possible to have
vertical leaders. In Example~\ref{vleaders} we use a centered dot \cs{cdot} to fill the space between two paragraphs with leaders. We define a command
\cs{vdotfill} to do this that contains the instructions.

\begin{texexample}{Vertical leaders}{vleaders}
\newcommand{\vdotfill}{%
  \par\leaders\hbox{$\cdot$}\vfill}
  \vbox to 5cm {%
  \lorem
  \vdotfill
  \lorem
  }
\end{texexample}





\section{Leaders and shifted margins}

If margins have been shifted, leaders may look different depending on how the shift has been realized.
For an illustration of how\cs    {hangindent} and\cs{leftskip} influence the look of leaders, consider
the following examples, where

\begin{texexample}{Ratata}{ex:ratata}
\setbox0=\hbox{R a t a t a  }
\verb+\setbox0=\hbox{R a t a t a  }+



\hbox{\kern1em\hbox{\leaders\copy0\hskip5cm}}

\hangindent=1em \hangafter=-1 \noindent
\leaders\copy0\hskip5cm\hbox{}\par
\end{texexample}

gives (note the shift with respect to the previous example)
\medskip

{\hbox{\kern1em\hbox{\leaders\copy0\hskip5cm}}
\hangindent=1em \hangafter=-1 \noindent
\leaders\copy0\hskip5cm\hbox{}\par}

In the first paragraph the\cs{leftskip} glue only obscures the first leader box; in the second paragraph
the hanging indentation actually shifts the orientation point for the row of leaders. Hanging
indentation is performed in TEX by a\cs{moveright} of the boxes containing the lines of the
paragraph.

   

Leaders are a powerful tool, they take a little bit of time to understand, but once you familiar with them you can achieve all sorts of layouts with them.


\section{Applications}

Most of the useage of leaders is in table of contents and old tables fashioned the old way. The package \pkg{arydshln} by Hiroshi Nakashima uses \cs{xleaders} to give \latex’s \pkg{array} and \pkg{tabular} environments the capability to draw horizontal/vertical dash-lines. You can refer to it for more examples.

In the LateX kernel they are mostly found them in the definition of mathematical symbols and from where I have adapted the following Example~\ref{cleaders}.

\begin{texexample}{cleaders example}{cleaders}
 \makeatletter
 \def\rightarrowfill{$\m@th\smash-\mkern-7mu%
  \cleaders\hbox{$\mkern-2mu\smash-\mkern-2mu$}\hfill
  \mkern-7mu\mathord\rightarrow$}
 \makeatother
From here to \rightarrowfill the end.
\end{texexample}

Note in the example the use of mathematical kerns (|\mkern|) and the use of 
|\smash|. Another interesting area was the definition of various commands in the
picture environment using solely leaders.


Donald Arseneau's \pkg{ulem} uses leaders extensively and other magic to provide various forms of underlining.

\begin{texexample}{Decorating text}{ex:decorating}
   \uline{important}   underlined text\\
   \uuline{urgent}     double-underlined text\\
   \uwave{boat}        wavy underline\\
    \sout{wrong}        line drawn through word\\
   \xout{removed}      marked over with //////.\\
   \dashuline{dashing} dash underline\\
   \dotuline{dotty}    dotted underline\\
\end{texexample}   

The package has another useful feature. It is one of those short packages that one can study to understand
the mechanisms of saving boxes, measuring dimensions, rules and leaders, as well as hyphenation. A must read for anyone interested in improving their basic understanding of \tex.

\vfill














 }


\def\visualizations{
  \part{VISUALIZATIONS AND PLOTTING}
   \cxset{chapter name = Chapter}

\chapter{The \texttt{picture} Environment}
\label{pictureenvironment}
\index{environments=picture}
\index{packages=picture}

When TeX was developed, the notion of graphic output was very limited, although Knuth presented a method
using boxes to draw primitive commands at any point on the page. This of course is achieved using zero width or height |\hbox| or |\vbox| commands. LaTeX uses a similar approach with the picture environment. 
The |picture| environment comes straight out of the box and can be used to draw simple figures. For more sophisticated graphics |TikZ| is a better choice. It can be used in package documentation and simple tasks. The learning curve for using it is minimal.

Using the picture environment is much easier to code for drawing shapes or rulers around sectioning commands.
This type of heading is very popular in many modern books. Figure~\ref{fig:picture-sections}

\begin{figure}[htbp]
\includegraphics[width=\textwidth]{./images/picture-sections.jpg}
\caption{A section with some fancy lines around the text. From \textit{Probabilities and Statistics for Engineers and Scientists}, by Walpole \textit{et.al}, 2011. }
\label{fig:picture-sections}
\end{figure}

Of course this is also achievable without the picture environment, simpy using TeX commands or with tikZ. From graphics point of view, the environment is also useful for short mathematical diagrams.

\section{The Basic Commands}

\begin{docEnvironment}{picture}{}{}
\end{docEnvironment}
The |picture| environment is created using one of two commands.

\emphasis{picture}
\begin{teXXX}
 \begin{picture}(x, y). . . \end{picture}
\end{teXXX}

\noindent or

\begin{teXX}
  \begin{picture}(x, y)(x0,y0). . . \end{picture}
\end{teXX}

\begin{docCommand}{unitlength}{\marg{dim}}
Most people prefer the first type which they combine, with a |setlength| command that sets the \cs{unitlength}.
\end{docCommand}

The optional argument gives the coordinates of the point at the lower-left corner of the picture (thereby determining the origin). For example, if \cs{unitlength} has been set to 1mm, the command

\begin{texexample}{}{}
  \setlength\unitlength{1mm}
  \begin{picture}(40,40)(0,0)
    \put(10,30){\vector(0,-1){30}}
    \put(10,30){\vector(1,0){30}}
    \put(25,30.5){$a$} 
  \end{picture}
\end{texexample}

produces a picture of width 100 millimeters and height 200 millimeters, whose lower-left corner is the point (10,20) and whose upper-right corner is therefore the point (110,220). When you first draw a picture, you will omit the optional argument, leaving the origin at the lower-left corner. If you then want to modify your picture by shifting everything, you just add the appropriate optional argument.

\section{Text and Formulae}

%\begin{macro}{\linethickness}
%\begin{macro}{\thicklines}
%\begin{macro}{\thinlines}
Text and formulas can be written into a picture
environment with the \cs{put} command in the usual way. The line thickness can be
set by using \cs{linethickness}\marg{dim}. The command \cs{thinlines} is half the thickness of the \cs{linethickness} dimension and \cs{thicklines} is the current line width. The \cs{linethickness} does not change width of slanted lines
or circles as it is drawn using a font and would render badly.
%\end{macro}
%\end{macro}
%\end{macro}

\emphasis{thicklines}
\begin{texexample}{Text and Formulae}{}
\setlength{\unitlength}{0.8cm}
\begin{picture}(6,5)
 \thicklines
 \put(1,0.5){\line(2,1){3}}
 \put(4,2){\line(-2,1){2}}
 \put(2,3){\line(-2,-5){1}}
 \put(0.7,0.3){$A$}
 \put(4.05,1.9){$B$}
 \put(1.7,2.95){$C$}
 \put(3.1,2.5){$a$}
 \put(1.3,1.7){$b$}
 \put(2.5,1.05){$c$}
 \put(0.3,4){$F=
 \sqrt{s(s-a)(s-b)(s-c)}$}
 \put(3.5,0.4){$\displaystyle
 s:=\frac{a+b+c}{2}$}
\end{picture}
\end{texexample}



\setlength{\unitlength}{5cm}
\begin{picture}(1,1)
\put(0,0){\line(0,1){1}}
\put(0,0){\line(1,0){1}}
\put(0,0){\color{blue}\line(1,1){1}}
\put(0,0){\color{orange}\line(1,2){0.5}}
\end{picture}


\section{multiput and linethickness}
The \cmd{\multiput} is used to place multiple objects onto the picture. It has the general format shown below:

\setlength{\unitlength}{2mm}
\begin{picture}(30,20)
  \color{green}
   \linethickness{0.075mm}
   \multiput(0,0)(1,0){25}%
   {\line(0,1){20}}
   \multiput(0,0)(0,1){21}%
   {\line(1,0){25}}
   \linethickness{0.15mm}
   \multiput(0,0)(5,0){6}%
   {\line(0,1){20}}
   \multiput(0,0)(0,5){5}%
   {\line(1,0){25}}
   \linethickness{0.3mm}
   \multiput(5,0)(10,0){2}%
    {\line(0,1){20}}
   \multiput(0,5)(0,10){2}%
   {\line(1,0){25}}
\end{picture}



\begin{docCommand}{multiput} {(x,y) (Dx, Dy) \marg{n} \marg{object} }
The command |\multiput| allows to repeat
a \cmd{\put} a number of times.
\end{docCommand}

\begin{figure}
\setlength{\unitlength}{0.8cm}
\begin{picture}(6,5)
 \thicklines
 \put(1,0.5){\line(2,1){3}}
 \put(4,2){\line(-2,1){2}}
 \put(2,3){\line(-2,-5){1}}
 \put(0.7,0.3){$A$}
 \put(4.05,1.9){$B$}
 \put(1.7,2.95){$C$}
 \put(3.1,2.5){$a$}
 \put(1.3,1.7){$b$}
 \put(2.5,1.05){$c$}
 \put(0.3,4){$F=
 \sqrt{s(s-a)(s-b)(s-c)}$}
 \put(3.5,0.4){$\displaystyle
 s:=\frac{a+b+c}{2}$}
\end{picture}
\caption{Figures can have captions, if you enclose in a figure environment}
\end{figure}

\begin{figure}
\scalebox{0.7}{
\setlength{\unitlength}{0.5mm}
\begin{picture}(120,168)
\newsavebox{\foldera}
\savebox{\foldera}
(40,32)[bl]{% definition
\multiput(0,0)(0,28){2}
{\line(1,0){40}}
\multiput(0,0)(40,0){2}
{\line(0,1){28}}
\put(1,28){\oval(2,2)[tl]}
\put(1,29){\line(1,0){5}}
\put(9,29){\oval(6,6)[tl]}
\put(9,32){\line(1,0){8}}
\put(17,29){\oval(6,6)[tr]}
\put(20,29){\line(1,0){19}}
\put(39,28){\oval(2,2)[tr]}
}
\newsavebox{\folderb}
\savebox{\folderb}
(40,32)[l]{% definition
\put(0,14){\line(1,0){8}}
\put(8,0){\usebox{\foldera}}
\put(0.2,1.4)
{$\beta=v/c=\tanh\chi$}
}
\put(34,26){\line(0,1){102}}
\put(14,128){\usebox{\foldera}}
\multiput(34,86)(0,-37){3}
{\usebox{\folderb}}
\end{picture}}
\caption{Pictures can be scaled using \protect\textbackslash scalebox.}
\end{figure}

\section{Some examples}
Any vertex-symmetric graph is regular, but edge-symmetric graphs
need not be regular. For example,
\begin{verbatim}
$$\unitlength=10pt
\def\putdisk(#1,#2){\put(#1,#2){\disk{.4}}}
$\vcenter{
\hbox{\beginpicture(2,1.5)(0,0)
\putdisk(0,0)
\putdisk(2,0)
\putdisk(1,.5)
\putdisk(1,1.5)
\put(0,0){\line(2,1){1}}
\put(2,0){\line(-2,1){1}}
\put(1,.5){\line(0,1){1}}
\endpicture}}
\quad&\hbox{is edge-symmetric, not vertex-symmetric;}\cr
\noalign{\smallskip}
\vcenter{
\hbox{\beginpicture(2,2)(0,0)
\putdisk(1,0)
\putdisk(1,2)
\putdisk(0,.5)
\putdisk(0,1.5)
\putdisk(2,.5)
\putdisk(2,1.5)
\put(0,.5){\line(2,1){2}}
\put(2,.5){\line(-2,1){2}}
\put(0,.5){\line(2,3){1}}
\put(2,.5){\line(-2,3){1}}
\put(0,.5){\line(1,0){2}}
\put(0,1.5){\line(1,0){2}}
\put(1,0){\line(-2,3){1}}
\put(1,0){\line(2,3){1}}
\put(1,0){\line(0,1){2}}
\endpicture}}
\quad&\hbox{is vertex-symmetric, not edge-symmetric.}\qquad
 (\vcenter{\hbox{\beginpicture(1,2)(0,0)
\putdisk(.5,0)\putdisk(.5,2)\put(.5,0){\line(0,1){2}}\endpicture}}
\hbox{ is a maximal clique})\cr}$$
\end{verbatim}



\section{picture package}

The \pkg{picture} package by Heiko Oberdiek redefines the default \pkg{picture} macros and adds code that detects
whether such an argument is given as number or as length. In the latter case, the
length is used directly without multiplying with \cs{unitlength}. Th following
example i from the documentation of the package.

 \setlength{\unitlength}{1pt}
 \begin{picture}(\widthof{Hello World}, 10mm)
   \put(0, 0){\makebox(0,0)[lb]{Hello World}}%
   \put(0, \heightof{Hello World} + \fboxsep){%
   \line(1, 0){\widthof{Hello World}}%
 }%
 \put(\widthof{Hello World}, 10mm){%
   \line(0, -1){10mm}%
 }%
 \put(0,0){\line(966,259){8}}
 \end{picture}

The package |calc| is used for calculations or etex. The picture package requires that the package |calc| is loaded before
the |picture| package and is loaded correctly by |phd|.

The package also supports the packages \pkg{pspicture} and \pkg{pict2e}, but they must be loaded before package picture.

\section{pict2e}

The package pict2e by Hubert G\"a\ss lein, Rolf Niepraschk and Joseph Tkadlec extends the existing LATEX picture environment, using the familiar
technique (cf. the graphics and color packages) of driver files. In the user-level part of
this documentation there is a fair number of examples of use, showing where things are
improved by comparison with the Standard LaTeX picture environment.

The package is loaded automatically by |phd|.







   \def\storyi{The best graphics package ever developed is the TikZ package. 
Its parent package is PGF which is short of a miracle that has been programmed
using \tex, a more than thirty years old program. This has taken over almost all other
packages and is very popular with newcomers to \latex. It is frustrating at first, but once 
you over the basic ideas and concepts it opens infinite possibilities for typesetting
great articles and books.}



\cxset{chapter format=stewart,
       texti=\storyi,
       textii=\storyi}

\newcommand\seepgfmanual[1]{%
    \textit{see} the PGFmanual page #1}%
    
%\cxset{chapter format = traditional}    
\chapter{TikZ}

\section{The \protect\texttt{TikZ} package}
\pkg{TikZ}, a high-level interface to \pkg{PGF}, is a language-based tool for specifying graphics.
It uses familiar graphics-related concepts, such as point, line, and circle and
has a concise and natural syntax. It meshes well with pdfLATEX in the sense that
no additional processing steps are needed. Another positive aspect of \pkg{TikZ} is
its ability to blend \tex fonts, symbols, and mathematics within the generated
graphics.


All the TikZ commands can be used inline using \docAuxCommand{tikz} or within the \docAuxCommand{tikzpicture} environment. When we want to use captions and labels, we enclose it in the figure environment or use \docAuxCommand{captionof}, but it can be called anywhere in the text or math of a Tex document:

\begin{teXXX}
\begin{figure}
\centering
%\tikzset{external/force remake}
\begin{tikzpicture}
... TikZ commands ...
\end{tikzpicture}
\caption{A diagram drawn with TikZ.}
\label{Fig:_diagram1}
\end{figure}
\end{teXXX}

We can also use them in math:

\begin{teXXX}
\begin{align*}
\int dx\; f(x) =
\alpha
%\tikzset{external/force remake}
\begin{tikzpicture}
... TikZ commands ...
\end{tikzpicture}
\end{align*}
\end{teXXX}



\section{Draw simple lines}

\begin{texexample}{Draw a Line}{ex:line}
\begin{tikzpicture}
\node[draw] (S1) at (0,0) {Paris};
\node[draw] (S2) at (3,0) {Stratsbourg};
\draw (S1) -- (S2);
\end{tikzpicture}
\end{texexample}


The syntax of the command is:

|\node|\oarg{options} (\meta{name}) at (\meta{position}) |{|\meta{contents}|}|

If we look
 carefully, we see that the two writings give
Slightly different results:
- In the first case, node is an operation executed on a path. We
Can consider each node as a decoration of the point at which it
is associated. The line drawn by the draw command joins two points, the
Nodes are objects added later and centered on points. The option
Draw of the node trace operation the node outline.
- In the second case, \ node is a TikZ command which allows to define
A node, to name it and to draw it. One can then consider the
Nodes as pre-existing objects that will then be linked with the \docAuxCommand{node}.


\begin{texexample}{Draw a Line}{ex:line}
\begin{tikzpicture}
\node[draw] (S1) at (0,0) {Paris};
\node[draw] (S2) at (0,3) {Stratsbourg};
\draw[->] (S1) -- (S2);
\end{tikzpicture}
\end{texexample}

The basic building block of all pictures in \tikzname is the path. A path is a series of straight lines and curves
that are connected (that is not the whole picture, but let us ignore the complications for the moment). You
start a path by specifying the coordinates of the start position as a point in round brackets, as in (0,0).
This is followed by a series of \enquote{path extension operations.}


\begin{texexample}{Draw a Line}{ex:line}
\begin{tikzpicture}
\draw[->] (0,0) -- (1.5,0) -- (0, 1.2);
\end{tikzpicture}
\end{texexample}


\subsection*{Adding Text} 

So far we have seen how to draw lines and arcs. However, an important component is still missing the addition of text. When
\tikzname is constructing a path and it encounters the keyword |node| typically followed by some options  it reads a \textit{node specification}. Options can typically follow and then it terminates by curly brackets. 
 

\begin{texexample}{Draw a Line}{ex:line}
\begin{tikzpicture}
\draw[->] (0,0) -- (1.5,0) node {First Node} -- (0, 1.2) node[shape = circle] {Second Node};
\end{tikzpicture}
\end{texexample}


The \docAuxCommand*{node} can be used to abbreviate the operation. A longer example can demonstrate this better. How can we draw the following figure?

\begin{tikzpicture}
\node[circle,fill=black,inner sep=0.8pt,draw] (a) at (0,0) {};
\node[circle,fill=black,inner sep=0.8pt,draw] (b) at (1.5,0) {};
\node[circle,fill=black,inner sep=1.5pt,draw] (c) at (.75,2) {};
\node[circle,fill=black,inner sep=0.8pt,draw] (d) at (0.75,.75) {};
\node[circle,fill=black,inner sep=0.8pt,draw] (e) at (2,1) {};


\node () at (-0.3,0) {\tiny$1$};
\node () at (0.75,0.45) {\tiny$2$};
\node () at (0.75,2.3) {\tiny$4$};
\node () at (2,1.3) {\tiny$-1$};
\node () at (1.8,0) {\tiny$-1$};

\draw (a)--(b)--(e)--(c) --(a)--(d)--(b)--(c);
\draw (c)--(d);

\node at (3,1) {\Large{$\sim$}};

\begin{scope}[shift={(+4,0)}]
\node[circle,fill=black,inner sep=0.8pt,draw] (a) at (0,0) {};
\node[circle,fill=black,inner sep=0.8pt,draw] (b) at (1.5,0) {};
\node[circle,fill=black,inner sep=0.8pt,draw] (c) at (.75,2) {};
\node[circle,fill=black,inner sep=0.8pt,draw] (d) at (0.75,.75) {};
\node[circle,fill=black,inner sep=0.8pt,draw] (e) at (2,1) {};


\node () at (-0.3,0) {\tiny$2$};
\node () at (0.75,0.45) {\tiny$3$};
\node () at (0.75,2.3) {\tiny$0$};
\node () at (2,1.3) {\tiny$0$};
\node () at (1.8,0) {\tiny$0$};

\draw (a)--(b)--(e)--(c) --(a)--(d)--(b)--(c);
\draw (c)--(d);

\end{scope}
\end{tikzpicture}

\begin{texexample}{A larger example}{ex:larger}
\begin{tikzpicture}
\node[circle,fill=black,inner sep=0.8pt,draw] (a) at (0,0) {};
\node[circle,fill=black,inner sep=0.8pt,draw] (b) at (1.5,0) {};
\node[circle,fill=black,inner sep=1.5pt,draw] (c) at (.75,2) {};
\node[circle,fill=black,inner sep=0.8pt,draw] (d) at (0.75,.75) {};
\node[circle,fill=black,inner sep=0.8pt,draw] (e) at (2,1) {};


\node () at (-0.3,0) {\tiny$1$};
\node () at (0.75,0.45) {\tiny$2$};
\node () at (0.75,2.3) {\tiny$4$};
\node () at (2,1.3) {\tiny$-1$};
\node () at (1.8,0) {\tiny$-1$};

\draw (a)--(b)--(e)--(c) --(a)--(d)--(b)--(c);
\draw (c)--(d);

\node at (3,1) {\Large{$\sim$}};

\begin{scope}[shift={(+4,0)}]
\node[circle,fill=black,inner sep=0.8pt,draw] (a) at (0,0) {};
\node[circle,fill=black,inner sep=0.8pt,draw] (b) at (1.5,0) {};
\node[circle,fill=black,inner sep=0.8pt,draw] (c) at (.75,2) {};
\node[circle,fill=black,inner sep=0.8pt,draw] (d) at (0.75,.75) {};
\node[circle,fill=black,inner sep=0.8pt,draw] (e) at (2,1) {};


\node () at (-0.3,0) {\tiny$2$};
\node () at (0.75,0.45) {\tiny$3$};
\node () at (0.75,2.3) {\tiny$0$};
\node () at (2,1.3) {\tiny$0$};
\node () at (1.8,0) {\tiny$0$};

\draw (a)--(b)--(e)--(c) --(a)--(d)--(b)--(c);
\draw (c)--(d);

\end{scope}
\end{tikzpicture}
\captionof{figure}{The larger vertex fires once to move from the left configuration to the right configuration.}
\end{texexample}

Behind the scenes pgf uses the basic system command \docAuxCommand{pgfnode} to create the nodes. The syntax of the command is given on \seepgfmanual{1026} as:

\begin{docCommand}{pgfnode}{\marg{shape}\marg{anchor}\marg{label text}\marg{name}\marg{path usage command}}
This command creates a new node. The \marg{shape} of the node must have been declared previously using
\lstinline{pgfdeclareshape}.

The shape is shifted such that the \marg{anchor} is at the origin. In order to place the shape somewhere else,
use the coordinate transformation prior to calling this command.
The hnamei is a name for later reference. If no name is given, nothing will be “saved” for the node, it
will just be drawn.

The \marg{path usage command} is executed for the background and the foreground path (if the shape defines
them).
\end{docCommand}


A good workflow, is to first define the nodes, next label them and then draw any connecting lines.

\begin{texexample}{Named nodes}{ex:named} 
\begin{tikzpicture}
\node[circle,fill=black,inner sep=0.8pt,draw] (a) at (0,0) {};
\node[circle,fill=black,inner sep=0.8pt,draw] (b) at (1.5,0) {};
\node[circle,fill=black,inner sep=1.5pt,draw] (c) at (.75,2) {};
\node[circle,fill=black,inner sep=0.8pt,draw] (d) at (0.75,.75) {};
\node[circle,fill=black,inner sep=0.8pt,draw] (e) at (2,1) {};
\end{tikzpicture}
\end{texexample}

\begin{texexample}{Named nodes}{ex:named} 
\begin{tikzpicture}
\node[circle,fill=black,inner sep=0.8pt,draw] (a) at (0,0) {};
\node[circle,fill=black,inner sep=0.8pt,draw] (b) at (1.5,0) {};
\node[circle,fill=black,inner sep=1.5pt,draw] (c) at (.75,2) {};
\node[circle,fill=black,inner sep=0.8pt,draw] (d) at (0.75,.75) {};
\node[circle,fill=black,inner sep=0.8pt,draw] (e) at (2,1) {};
% absolute labelling
\node () at (-0.3,0) {\tiny$1$};
\node () at (0.75,0.45) {\tiny$2$};
\node () at (0.75,2.3) {\tiny$4$};
\node () at (2,1.3) {\tiny$-1$};
\node () at (1.8,0) {\tiny$-1$};
\end{tikzpicture}
\end{texexample}

\begin{texexample}{Named nodes}{ex:named} 
\begin{tikzpicture}
\pgfdeclarelayer{background}
\pgfdeclarelayer{foreground}
\pgfsetlayers{background,main,foreground}
\node[circle,fill=black,inner sep=0.8pt,draw] (a) at (0,0) {};
\node[circle,fill=black,inner sep=0.8pt,draw] (b) at (1.5,0) {};
\node[circle,fill=black,inner sep=1.5pt,draw] (c) at (.75,2) {};
\node[circle,fill=black,inner sep=0.8pt,draw] (d) at (0.75,.75) {};
\node[circle,fill=black,inner sep=0.8pt,draw] (e) at (2,1) {};
% absolute labelling
\node () at (-0.3,0) {\tiny$1$};
\node () at (0.75,0.45) {\tiny$2$};
\node () at (0.75,2.3) {\tiny$4$};
\node () at (2,1.3) {\tiny$-1$};
\node () at (1.8,0) {\tiny$-1$};
% draw connecting lines
\draw (a)--(b)--(e)--(c) --(a)--(d)--(b)--(c);
\draw (c)--(d);
%\begin{pgfonlayer}{background}
\begin{scope}[on background layer={color=blue!10}]
\node [fill=blue!10,fit=(a) (b) (c)
(d) (e)] {};
\end{scope}
%\end{pgfonlayer}
\end{tikzpicture}
\end{texexample}

Just to recap, using \docAuxCommand*{node} and the \textbf{at} we can position accurately any node. We could have used the much longer command |path node|, but in our case above this is unecessary (\seepgfmanual{49}), for more explanations if you are still unsure.

Nodes can be named or unnamed. There are two ways to name them, with the key \docValue{name} or within brackets. The second method is to be preferred. Names for nodes can be pretty arbitrary, but they should not contain commas, periods, parentheses, colons, and some other special characters. However, they can contain underscores and hyphens

\subsection{Layers and Scope}

We can add a backround layer, using the library \textit{backgrounds}, which provides key values for adding backgrounds. \pgfname\ provides a layering mechanism for composing graphics from
multiple layers. (This mechanism is not to be confused with the
conceptual ``software layers'' the \pgfname\ system is composed of.)
Layers are often used in graphic programs. The idea is that you can
draw on the different layers in any order. So you might start drawing
something on the ``background'' layer, then something on the
``foreground'' layer, then something on the ``middle'' layer, and then
something on the background layer once more, and so on. At the end, no
matter in which ordering you drew on the different layers, the layers
are ``stacked on top of each other'' in a fixed ordering to produce
the final picture. Thus, anything drawn on the middle layer would come
on top of everything of the background layer.

Normally, you do not need to use different layers since you will have
little trouble ``ordering'' your graphic commands in such a way that
layers are superfluous. However, in certain situations you only
``know'' what you should draw behind something else after the
``something else'' has been drawn.

For example, suppose you wish to draw a yellow background behind your
picture. The background should be as large as the bounding box of the
picture, plus a little border. If you know the size of the bounding box
of the picture at its beginning, this is easy to accomplish. However,
in general this is not the case and you need to create a
``background'' layer in addition to the standard ``main'' layer. Then,
at the end of the picture, when the bounding box has been established,
you can add a rectangle of the appropriate size to the picture.

\subsection{Declaring Layers}

In \pgfname\ layers are referenced using names. The standard layer,
which is a bit special in certain ways, is called |main|. If nothing
else is specified, all graphic commands are added to the |main|
layer. You can declare a new layer using the following command:

\begin{docCommand}{pgfdeclarelayer}{\marg{name}}
  This command declares a layer named \meta{name} for later
  use. Mainly, this will set up some internal bookkeeping.
\end{docCommand}

The next step toward using a layer is to tell \pgfname\ which layers
will be part of the actual picture and which will be their
ordering. Thus, it is possible to have more layers declared than are
actually used.

\begin{docCommand}{pgfsetlayers}{\marg{layer list}}
  This command tells \pgfname\ which layers will be used in
  pictures. They are stacked on top of each other in the order
  given. The layer |main| should always be part of the list. Here is
  an example:
\begin{codeexample}[code only]
\pgfdeclarelayer{background}
\pgfdeclarelayer{foreground}  
\pgfsetlayers{background,main,foreground}
\end{codeexample}

  This command should be given either outside of any picture or ``directly inside'' of a picture.
  Here, the ``directly inside'' means that there should be no further level of \TeX\ grouping between |\pgfsetlayers| and the matching |\end{pgfpicture}| (no closing braces, no |\end{...}|). It will also work if |\pgfsetlayers| is provided before |\end{tikzpicture}| (with similar restrictions).
\end{docCommand}


\subsection{Using Layers}

Once the layers of your picture have been declared, you can start to
``fill'' them. As said before, all graphics commands are normally
added to the |main| layer. Using the |{pgfonlayer}| environment, you
can tell \pgfname\ that certain commands should, instead, be added to
the given layer.

\begin{docEnvironment}{pgfonlayer}{\marg{layer name}}
\end{docEnvironment}

The whole \meta{environment contents} is added to the layer with the
name \meta{layer name}. This environment can be used anywhere inside
a picture. Thus, even if it is used inside a |{pgfscope}| or a \TeX\
group, the contents will still be added to the ``whole'' picture.
Using this environment multiple times inside the same picture will
cause the \meta{environment contents} to accumulate.

  \emph{Note:} You can \emph{not} add anything to the |main| layer
  using this environment. The only way to add anything to the main
  layer is to give graphic commands outside all |{pgfonlayer}|
  environments. 



\begin{codeexample}[]
\pgfdeclarelayer{background layer}
\pgfdeclarelayer{foreground layer}
\pgfsetlayers{background layer,main,foreground layer}
\begin{tikzpicture}
  % On main layer:
  \fill[blue] (0,0) circle (1cm);
  
  \begin{pgfonlayer}{background layer}
    \fill[yellow] (-1,-1) rectangle (1,1);
  \end{pgfonlayer}
  
  \begin{pgfonlayer}{foreground layer}
    \node[white] {foreground};
  \end{pgfonlayer}
  
  \begin{pgfonlayer}{background layer}
    \fill[black] (-.8,-.8) rectangle (.8,.8);
  \end{pgfonlayer}

  % On main layer again:
  \fill[blue!50] (-.5,-1) rectangle (.5,1);
\end{tikzpicture}
\end{codeexample}



\long\gdef\mytriangle{
\node[circle,fill=black,inner sep=0.8pt,draw] (a) at (0,0) {};
\node[circle,fill=black,inner sep=0.8pt,draw] (b) at (1.5,0) {};
\node[circle,fill=black,inner sep=1.5pt,draw] (c) at (.75,2) {};
\node[circle,fill=black,inner sep=0.8pt,draw] (d) at (0.75,.75) {};
\node[circle,fill=black,inner sep=0.8pt,draw] (e) at (2,1) {};
% absolute labelling
\node () at (-0.3,0) {\tiny$1$};
\node () at (0.75,0.45) {\tiny$2$};
\node () at (0.75,2.3) {\tiny$4$};
\node () at (2,1.3) {\tiny$-1$};
\node () at (1.8,0) {\tiny$-1$};
% draw connecting lines
\draw (a)--(b)--(e)--(c) --(a)--(d)--(b)--(c);
\draw (c)--(d);
}

\begin{texexample}{Adding backgrouns}{ex:backgrounds}
\begin{tikzpicture}
\pgfdeclarelayer{background}
\pgfdeclarelayer{foreground}
\pgfsetlayers{background,main,foreground}
\mytriangle
%\begin{pgfonlayer}{background}
\begin{scope}[on background layer={color=blue!10}]
\mytriangle
\node [fill=blue!10,fit=(a) (b) (c)
(d) (e)] {};
\end{scope}
%\end{pgfonlayer}
\end{tikzpicture}
\end{texexample}


\begin{texexample}{Adding backgrouns}{ex:backgrounds}
\begin{tikzpicture}
\pgfdeclarelayer{background}
\pgfdeclarelayer{foreground}
\pgfsetlayers{background,main,foreground}
\mytriangle
%\begin{pgfonlayer}{background}
\begin{scope}[on background layer={color=blue!10}]
\node [fill=blue!10,fit=(a) (b) (c)
(d) (e)] {};
\end{scope}

\begin{scope}[shift={(+4,0)}]
\mytriangle
\begin{pgfonlayer}{background}
\node [pattern=checkerboard light gray,fit=(a) (b) (c)
(d) (e)] {};
\end{pgfonlayer}
\end{scope}
\end{tikzpicture}
\end{texexample}

This brings us to the end of our discussion. Time for a coffee and a break.                

\section{Adding styles}

In our previous example, we cut and pasted many of the repetitive keys. \pgfname offers a way to set a new key to the values of other keys using the handler |.style|. This is a very powerful way of redefining new keys, but also simplifying the code. Styles in \tikzname can be considered similar to macros in standard LaTeX. When I made a drawing, we can still tweak the styles and look how the drawing changes, until it's perfect. You should never have to tweak each node.

\begin{texexample}{Using styles}{ex:usingstyles}
\tikzset{BN/.style = {circle,fill=black,inner sep=0.8pt,draw},
         tiny/.style = {font=\tiny}, 
}
\begin{tikzpicture}
\node[BN] (a) at (0,0) {};
\node[BN] (b) at (1,0) {};
\node[BN] (c) at (1,1) {};
\node[BN] (d) at (0,1) {};
\node[BN] (e) at (-1,0) {};

\node () at (-1.3,0) [tiny]{$v_1$};
\node () at (-.3,1)  [tiny]{$v_2$};
\node () at (1.3,0)  [tiny]{$w_1$};
\node () at (1.3,1)  [tiny]{$w_2$};

\node[tiny] () at (0.5,-0.2) {$a$};
\node[tiny] () at (0.5,1.2) {$b$};
\node[tiny] () at (0.2,0.5) {$c$};
\node[tiny] () at (-0.5,-.2) {$d$};

\draw (e) -- (a) -- (b) -- (c) -- (d) -- (a);
\draw (e) -- (d);

\end{tikzpicture}
\end{texexample}



\section{Arcs and options for lines}

\begin{texexample}{Draw a Line}{ex:line}
\begin{tikzpicture}
\draw[->] (0,0) -- (1.5,0) node[draw, ellipse] {First Node} -| (0, 1.2) node[draw,ellipse,rotate=45] {Second Node};
\end{tikzpicture}
\end{texexample}

\begin{texexample}{Drawing arcs}{ex:matharcs}
We define 
\begin{gather*}
    \bar{d}_{k,l}:=\hspace{6pt}
    \begin{tikzpicture}[baseline=(current bounding box.center)]
    \draw[->] (3,2) arc (-180:180:5mm);
	  \fill (3.95,2.2) circle [radius=2pt];
    \draw (3.95,1.8) circle [radius=2pt];
    \node at (4.2,1.8) {$l$};
    \node at (4.2,2.2) {$k$};
    \end{tikzpicture}
    \hspace{0.5cm}
    \text{and}
    \hspace{0.5cm}
    d_{k,l}:=\hspace{6pt}
    \begin{tikzpicture}[baseline=(current bounding box.center)]
    \draw[<-] (3,2) arc (-180:180:5mm);
    \fill (3.95,2.2) circle [radius=2pt];
    \draw (3.95,1.8) circle [radius=2pt];
    \node at (4.2,1.8) {$l$};
    \node at (4.2,2.2) {$k$};
    \end{tikzpicture}
    \hspace{0.5cm}
    \text{for}
    \hspace{2mm} k,l\in\mathbb{Z}_{\geq 0}.
\end{gather*}
\end{texexample}


Here is a figure that you should try and reproduce.
\newcommand{\G}{\Gamma}

\begin{tikzpicture}
\draw (-3.5,-1)--(-2.5,0); \draw (-2.5,-1)--(-3.5,0); \draw (-1.5,-1)--(-1.5,0);\draw[fill=black] (-3,-0.5) circle (0.1cm); \draw (-3.5,0)--(-3.5,1); \draw (-2.5,0)--(-1.5,1); \draw (-1.5,0)--(-2.5,1);\draw[fill=black] (-2,0.5) circle (0.1cm); \draw[->] (-3.5,1)--(-2.5,2); \draw[->] (-2.5,1)--(-3.5,2); \draw[->] (-1.5,1)--(-1.5,2); \draw[fill=black] (-3,1.5) circle (0.1cm); \draw (-3.6,0)--(-3.4,0);\draw (-2.6,0)--(-2.4,0);\draw (-1.6,0)--(-1.4,0); \draw (-3.6,1)--(-3.4,1);\draw (-2.6,1)--(-2.4,1);\draw (-1.6,1)--(-1.4,1); \node at (-3.5,-1.2) {$x_1$};\node at (-2.5,-1.2) {$x_2$};\node at (-1.5,-1.2) {$x_3$}; \node at (-3.5,2.2) {$y_1$};\node at (-2.5,2.2) {$y_2$};\node at (-1.5,2.2) {$y_3$}; \node at (-3.8,0) {$t_1$};\node at (-2.2,0) {$t_2$};\node at (-1.2,0) {$t_3$}; \node at (-3.8,1) {$t_4$};\node at (-2.8,1) {$t_5$};\node at (-1.2,1) {$t_6$}; \node at (-2.5,-1.65) {$\Gamma$};
\draw[->] (0,0)--(1,1); \draw[->] (1,0)--(0,1); \draw[fill=black] (0.5,0.5) circle (0.1cm); \draw[->] (2,0)--(3,1); \draw[->] (3,0)--(2,1); \draw[fill=black] (2.5,0.5) circle (0.1cm); \draw[->] (4,0)--(5,1); \draw[->] (5,0)--(4,1); \draw[fill=black] (4.5,0.5) circle (0.1cm); \draw[->] (6,0)--(6,1); \draw[->] (7,0)--(7,1); \draw[->] (8,0)--(8,1);
\node at (0,-.2) {$x_1$};\node at (1,-.2) {$x_2$}; \node at (2,-.2) {$t_2$};\node at (3,-.2) {$t_3$}; \node at (4,-.2) {$t_4$};\node at (5,-.2) {$t_5$}; \node at (6,-.2) {$x_3$}; \node at (7,-.2) {$t_1$}; \node at (8,-.2) {$t_6$};
\node at (0,1.2) {$t_1$};\node at (1,1.2) {$t_2$}; \node at (2,1.2) {$t_5$};\node at (3,1.2) {$t_6$}; \node at (4,1.2) {$y_1$};\node at (5,1.2) {$y_2$}; \node at (6,1.2) {$t_3$}; \node at (7,1.2) {$t_4$}; \node at (8,1.2) {$y_3$};
\node at (0.5,-0.65) {$\G_1$}; \node at (2.5,-0.65) {$\G_2$}; \node at (4.5,-0.65) {$\G_3$}; \node at (6,-0.65) {$\G_4$};\node at (7,-0.65) {$\G_5$};\node at (8,-0.65) {$\G_6$}; 
\end{tikzpicture}

This brings us to the end.




The |node| can take numerous options who are then used to set the typesetting of the text that follows:


\begin{texexample}{Draw a Line}{ex:line}
\begin{tikzpicture}
\draw[->] (0,0) -- (1.5,0) node[draw, ellipse] {First Node} -| (0, 1.2) node[draw,ellipse,rotate=45, text width=3cm, fill=creamy, text justified] {\lorem};
\end{tikzpicture}
\end{texexample}


\begin{texexample}{Draw a Line}{ex:line}
\begin{tikzpicture}[funny ellipse/.style = {draw,ellipse,rotate=45, text width=3cm, fill=creamy, text justified} ]
\draw[->] (0,0) -- (1.5,0) node[draw, ellipse] {First Node} -| (0, 1.2) node[funny ellipse] {\lorem};
\end{tikzpicture}
\end{texexample}

This can also be written by using \docAuxCommand{tikzset} for setting out all the keys. This can written just before the environment or within the scope of the environment. See \href{https://tex.stackexchange.com/questions/52372/should-tikzset-or-tikzstyle-be-used-to-define-tikz-styles}{TX.SX discussion}, for the option to set |\tikzstyle| which should not be used, even if it is quicker to write.


\begin{texexample}{Draw a Line}{ex:line}
\tikzset{funny ellipse/.style = {draw,ellipse,rotate=45, text width=3cm, fill=creamy, text justified} }
\begin{tikzpicture}
\draw[->] (0,0) -- (1.5,0) node[draw, ellipse] {First Node} -| (0, 1.2) node[funny ellipse] {\lorem};
\end{tikzpicture}
\end{texexample}

A |node| can possibly be rendered with a choice from a list of over 720 keys.

ed. 



Using the |TikZ| package you can draw figures and intermingle them with text. To draw a simple diamond as shown in \fref{fig:diamond} we use
the following commands. The package comes with a very comprehensive manual of over 500 pages long. One can state that there is nothing that you cannot draw with PGF/TikZ, if you have the patience and perseverance. TikZ's language has a syntax of its own with very little connection to what we have used so far. You will need to set aside adequate time to study this, especially if your work has a lot of specially drawn figures that you need. The result like anything else in \tex make the effort worthwhile.

\begin{texexample}{Draw a Diamond}{fig:diamond}
\begin{tikzpicture}
 \draw (1,0) -- (0,1) -- (-1,0) -- (0,-1) -- cycle;
\end{tikzpicture}
\end{texexample}


\begin{texexample}{Text long path}{ex:decorations}
\begin{tikzpicture}
\draw [help lines] grid (3,2);
\draw [red, dashed]
[postaction={decoration={text along path, text={a big juicy apple},
text align=fit to path}, decorate}]
(0,0) .. controls (0,2) and (3,2) .. (3,0);
\node (A) at (1.5,0) {!};
\end{tikzpicture}
\end{texexample}


\begin{texexample}{Text long path}{ex:decorations}

Hello \begin{pgfpicture}
\pgfpathrectangle{\pgfpointorigin}{\pgfpoint{2ex}{1ex}}
\pgfusepath{stroke}
\end{pgfpicture} World!

\end{texexample}


\emphasis{-,draw,begin,end,tikzpicture}
\begin{teXXX}
\begin{tikzpicture}
\draw (1,0) -- (0,1) -- (-1,0) -- (0,-1) -- cycle;
\end{tikzpicture}
\end{teXXX}



\makeatletter
The value of $x$ is \pgfsys@markposition{here}important.

Lots of text.
\hbox{\pgfsys@markposition{myorigin}%
\begin{pgfpicture}
% Switch of size protocol
\pgfpathmoveto{\pgfpointorigin}
\pgfusepath{use as bounding box}
\pgfsys@getposition{here}{\hereposition}
\pgfsys@getposition{myorigin}{\thispictureposition}
\pgftransformshift{\pgfpointscale{-1}{\thispictureposition}}
\pgftransformshift{\hereposition}
\pgfpathcircle{\pgfpointorigin}{1cm}
\pgfusepath{draw}
\end{pgfpicture}}

\makeatother


You cannot write directly into a picture environment. The command \docAuxCommand{pgftext} can be used. 

\begin{texexample}{Using text directly}{ex:pgftext}
\tikz{\draw[help lines] (0,0) grid (3,2);
\pgftext[base,x=1cm,y=0.5cm] {lovely}}
\end{texexample}





Sometimes it is quite useful when debugging to add a backround grid. 


\begin{centering}
\begin{tikzpicture}
\draw[step=0.25cm,color=creamy] (-1,-1) grid (1,1);
\draw [color=bgsexy](1,0) -- (0,1) -- (-1,0) -- (0,-1) -- cycle;
\end{tikzpicture}
\captionof{figure}{You can add a background grid using \texttt{step=0.25cm, color=green} as an option}
\end{centering}


\emphasis{step,color,green,grid,begin,end}
\begin{teXXX}
\begin{tikzpicture}
  \draw[step=0.25cm,color=green] (-1,-1) grid (1,1);
  \draw (1,0) -- (0,1) -- (-1,0) -- (0,-1) -- cycle;
\end{tikzpicture}
\end{teXXX}

The grid is specified by providing two diagonally opposing points: (-1,-1)
and (1, 1). The two options supplied give a step size for the grid lines and a
specification for the color of the grid lines, using the \docpkg{xcolor} package

\subsection{Specifying points and paths}

\begin{texexample}{Specifying points and paths}{ex:points}
\centering
\begin{tikzpicture}[scale=1.8]
% Define the points of a regular pentagon
\path (0,0) coordinate (origin);
\path (0:1cm) coordinate (P0);
\path (1*72:1cm) coordinate (P1);
\path (2*72:1cm) coordinate (P2);
\path (3*72:1cm) coordinate (P3);
\path (4*72:1cm) coordinate (P4);
% Draw the edges of the pentagon
\draw[color=bgsexy] (P0) -- (P1) -- (P2) -- (P3) -- (P4) -- cycle;
% Add "spokes"
\draw[color=bgsexy] (origin) -- (P0) (origin) -- (P1) (origin) -- (P2)
(origin) -- (P3) (origin) -- (P4);
\end{tikzpicture}
\captionof{figure}{Drawing a complicated polygon, using paths and the \texttt{draw} command}
\end{texexample}


Two key ideas used in \tikzname\ are points and paths. Both of these ideas were used
in the diamond examples. Much more is possible, however. For example, points
can be specified in any of the following ways:
\begin{enumerate}
\item  Cartesian coordinates
\item  Polar coordinates
\item  Named points
\item  Relative points
\end{enumerate}



\subsection{coordinates}
The cartesian coordinates can be defined and named using the following syntax.

%\emphasis{begin,end,coordinate,at,draw}
%\begin{teXXX}
%\begin{tikzpicture}
%  \coordinate (A) at (0,0);
%  \coordinate (B) at (1.25,0.25);
%  \draw[blue] (A) -- (B);
%\end{tikzpicture}
%\end{teXXX}

\noindent This produces:
\begin{tikzpicture}
\coordinate (A) at (0,0);
\coordinate (B) at (1.25,0.25);
\draw[blue] (A) -- (B);
\end{tikzpicture}


We can add labels to the points by using the |label| option. A label is distinct from the text of a |node|.

\begin{tikzpicture}
\coordinate [label=left:\textcolor{orange}{$A$}] (A) at (0,0);
\coordinate [label=right:\textcolor{orange}{$B$}]  (B) at (1.15,0.25);
\draw[blue] (A) -- (B);
\end{tikzpicture}

\emphasis{label,left,label:,right}
\begin{teXXX}
\begin{tikzpicture}
  \coordinate [label=left:\textcolor{orange}{$A$}] (A) at (0,0);
  \coordinate [label=west:\textcolor{orange}{$B$}] (B) at (1.25,0.25);
  \draw[blue] (A) -- (B);
\end{tikzpicture}
\end{teXXX}




If you tempted to write \texttt{label=top:} it will not work, as the command accepts the following keywords.

\begin{tikzpicture}
  \coordinate [label=left:\textcolor{orange}{east}]  (A) at (0,0);
  \coordinate [label=right:\textcolor{orange}{west}] (B) at (0,0);
  \draw[blue] (A)--(B);
\end{tikzpicture}


\section{Graphic Parameters: Line Width, Line Cap, and Line Join}

The width of lines can be specified using the key:

\begin{docKey}[tikz]{line width}{=\marg{dimension}} {no default, initially 0.4pt}
Specifies the line width \seepgfmanual{166}
\end{docKey}



\bgroup
\def\mkl#1{\tikz \draw[#1] (0,0)--(1.0, 1.5ex);}
\scriptsize\arial
\begin{tabular}{|l|l|l|l|l|l|l|l|}
\hline
\mkl{line width=2pt}& \mkl{ultra thin} &\mkl{very thin} & \mkl{thin} & \mkl{semithick} & \mkl{thick} &\mkl{very thick} &\mkl{ultra thick} \\
\hline
line width=2pt &ultra thin & very thin & thin &semithick & thick & very thick & ultra thick \\
\hline
\end{tabular}
\egroup

\begin{docKey}[tikz]{line cap}{=\marg{dimension}} {no default, initially 0.4pt}
Specifies how lines “end.” Permissible types are round, rect, and butt \seepgfmanual{167}. 
\end{docKey}

\bgroup
\def\mkl#1{\begin{tikzpicture} \draw[line width=10pt, line cap=#1] (0,0)--(1.0, 1.5ex);\draw[white,line width=2pt]
(0,0 )--(1.0,1.5ex);\end{tikzpicture}}
\scriptsize\arial
\begin{tabular}{|l|l|l|}
\hline
\mkl{rect}& \mkl{butt} &\mkl{round}  \\
\hline
rect &butt & round \\
\hline
\end{tabular}
\egroup




\begin{docKey}[tikz]{line join}{=\marg{type}}{no default, initially miter}
Specifies how lines “join.” Permissible type are round, bevel, and miter. They have the following
effects:
\end{docKey}

\begin{texexample}{Joining Lines}{es:joinlines}
\begin{tikzpicture}[line width=10pt]
\draw[line join=round] (0,0) -- ++(.5,1) -- ++(.5,-1);
\draw[line join=bevel] (1.25,0) -- ++(.5,1) -- ++(.5,-1);
\draw[line join=miter] (2.5,0) -- ++(.5,1) -- ++(.5,-1);
\end{tikzpicture}
\end{texexample}


\begin{docKey}[tikz]{dash pattern}{=\marg{dash pattern}}{no default}
Sets the dashing pattern. The syntax is the same as in \metafontlogo. For example following pattern on
2pt off 3pt on 4pt off 4pt means \enquote{draw 2pt, then leave out 3pt, then draw 4pt once more, then
leave out 4pt again, repeat}.
\end{docKey}

\bgroup
\def\ml#1{\tikz \draw[ #1] (0pt,0pt) -- (50pt,0pt);}
\def\alist{solid, dotted, densely dotted, loosely dotted,% 
           dashed,densely dashed, loosely dashed, %
           dash dot, densely dash dot, loosely dash dot, %
           dash dot dot, densely dash dot dot, loosely dash dot dot.}

For patterns there are numerous settings {\arial \alist }


\scriptsize
\begin{tabular}{lll}
\hline
\ml{solid} &  & \\
solid      &  & \\
\hline
\ml{dotted} &\ml{densely dotted} & \ml{loosely dotted}\\
\textit{dotted} & densely dotted  &loosely dotted \\
\hline
\ml{dashed} & \ml{densely dashed} & \ml{loosely dashed}  \\
\textit{dashed}      & densely dashed & loosely dashed            \\
\hline

\ml{dash dot} & \ml{densely dash dot} & \ml{loosely dash dot} \\
\textit{dash dot} & densely dash dot & loosely dash dot \\
\hline

\ml{dash dot dot} & \ml{densely dash dot dot} & \ml{loosely dash dot dot} \\
\textit{dash dot dot} & densely dash dot dot & loosely dash dot dot \\
\hline
\end{tabular}
\egroup


\subsection{Pattern Library}

The library patterns can be used to draw predetermined patterns. This will be a longer than usual section as it explains how to create new patterns. Most of the content is straight from the \pgfname manual. Before we start with the creation f a new pattern let us examine how a pattern is used.

\begin{texexample}{Using Library Patterns}{ex:libpatterns}
\begin{tikzpicture}
\pattern [path fading=west,pattern=checkerboard light gray]
      (0,0) rectangle (5cm,2em);
\end{tikzpicture}
\end{texexample}


\label{section-library-patterns}


The package defines patterns for filling areas. \docAuxCommand*{usetikzlibrary}\marg{patterns}.




\subsection{Form-Only Patterns}

\begin{tabular}{ll}
  \emph{Pattern name} & \emph{Example (pattern in black, blue, and red
    on faded checkerboard)} \\ 
  \patternindex{horizontal lines} 
  \patternindex{vertical lines} 
  \patternindex{north east lines} 
  \patternindex{north west lines} 
  \patternindex{grid} 
  \patternindex{crosshatch} 
  \patternindex{dots} 
  \patternindex{crosshatch dots} 
  \patternindex{fivepointed stars} 
  \patternindex{sixpointed stars} 
  \patternindex{bricks}
  \patternindex{checkerboard}
\end{tabular}
  
\subsection{Inherently Colored Patterns}


\begin{tabular}{ll}
  \emph{Pattern name} & \emph{Example} \\
  \patternindexinherentlycolored{checkerboard light gray} 
  \patternindexinherentlycolored{horizontal lines light gray} 
  \patternindexinherentlycolored{horizontal lines gray} 
  \patternindexinherentlycolored{horizontal lines dark gray} 
  \patternindexinherentlycolored{horizontal lines light blue} 
  \patternindexinherentlycolored{horizontal lines dark blue} 
  \patternindexinherentlycolored{crosshatch dots gray} 
  \patternindexinherentlycolored{crosshatch dots light steel blue} 
\end{tabular}
  


% Copyright 2006 by Till Tantau
%
% This file may be distributed and/or modified
%
% 1. under the LaTeX Project Public License and/or
% 2. under the GNU Free Documentation License.
%
% See the file doc/generic/pgf/licenses/LICENSE for more details.


\section{Creating Patterns}

\label{section-patterns}

\subsection{Overview}

There are many ways of filling a path. First, you can fill it using a
solid color and this is also the fastest method. Second, you can also
fill it using a shading, which means that the color changes smoothly
between two (or more) different colors. Third, you can fill it using a
tiling pattern and it is explained in the following how this is done.

A tiling pattern can be imagined as a rectangular tile (hence the
name) on which a small picture is painted. There is not a single tile,
but (conceptually) an infinite number of tiles, all showing the same
picture, and these tiles are arranged horizontally and vertically to
fill the plane. When you use a tiling pattern to fill a path, what
happens is that the path clips out a ``window'' through which we see
part of this infinite plane.

Patterns come in two versions: \emph{inherently colored patterns} and
\emph{form-only patterns}. (These are often called ``color patterns''
and ``uncolored patterns,'' but these names are misleading since
uncolored patterns do have a color and the color changes. As I said,
the name is misleading\dots) An inherently colored pattern is just a
colored tile like, say, a red star with a black outline. A form-only
pattern can be imagined as a tile that is a kind of rubber stamp. When
this pattern is used, the stamp is used to print copies of the stamp
picture onto the plane, but we can use a different stamp color each
time we use a form-only pattern.

\pgfname\ provides a special support for patterns. You can declare a
pattern and then use it very much like a fill color. \pgfname\
directly maps patterns to the pattern facilities of the underlying
graphic languages (PostScript, \textsc{pdf}, and \textsc{svg}). This
means that filling a path using a pattern will be nearly as fast as if
you used a uniform color.

There are a number of pitfalls and restrictions when using
patterns. First, once a pattern has been declared, you cannot change
it anymore. In particular, it is not possible to enlarge it or change
the line width. Such flexibility would require that the repeating of
the pattern were not done by the graphic language, but on the
\pgfname\ level. This would make patterns orders of magnitude slower
to produce and to render. However, \pgfname{} does provide a
more-or-less successful emulation of ``mutable'' patterns, although
internally, a new (fixed) instance of a pattern is declared when
the parameters of a pattern change.

Second, the phase of patterns is not well-defined, that is, it is not
clear where the origin of the ``first'' tile is. To be more precise,
PostScript and \textsc{pdf} on the one hand and \textsc{svg} on the
other hand define the origin differently. PostScript and \textsc{pdf}
define a fixed origin that is independent of where the path lies. This
has the highly desirable effect that if you use the same pattern to
fill multiple paths, the outcome is the same as if you had filled a 
single path consisting of the union of all these paths. By
comparison, \textsc{svg} uses the upper-left (?) corner of the path to
be filled as the origin. However, the \textsc{svg} specification is a
bit vague on this question.


\subsection{Declaring a Pattern}

Before a pattern can be used, it must be declared. The following
command is used for this:

\begin{docCommand}{pgfdeclarepatternformonly}{%
	\oarg{variables}%
	\marg{name}%
	\marg{bottom left}%
	\marg{top right}%
	\marg{tile size}%
	\marg{code}}

	This command declares a new form-only pattern. The \meta{name} is a
  name for later reference. The two parameters \meta{lower left} and
  \meta{upper right} must describe a bounding box that is large enough
  to encompass the complete tile.
\end{docCommand}

  The size of a tile is given by \meta{tile size}, that is, a tile is
  a rectangle whose lower left   corner is the origin and whose upper
  right corner is given by \meta{tile size}. This might make you
  wonder why the second and third parameters are needed. First, the
  bounding box might be smaller than the tile size if the tile is
  larger than the picture on the tile. Second, the bounding box might
  be bigger, in which case the picture will ``bleed'' over the tile.

  The \meta{code} should be \pgfname\ code than can be protocolled. It
  should not contain any color code.


\begin{codeexample}[]
\pgfdeclarepatternformonly{stars}
{\pgfpointorigin}{\pgfpoint{1cm}{1cm}}
{\pgfpoint{1cm}{1cm}}
{
  \pgftransformshift{\pgfpoint{.5cm}{.5cm}}
  \pgfpathmoveto{\pgfpointpolar{0}{4mm}}
  \pgfpathlineto{\pgfpointpolar{144}{4mm}}
  \pgfpathlineto{\pgfpointpolar{288}{4mm}}
  \pgfpathlineto{\pgfpointpolar{72}{4mm}}
  \pgfpathlineto{\pgfpointpolar{216}{4mm}}
  \pgfpathclose%
  \pgfusepath{fill}
}
\begin{tikzpicture}
  \filldraw[pattern=stars] (0,0)   rectangle (1.5,2);
  \filldraw[pattern=stars,pattern color=red]
                           (1.5,0) rectangle (3,2);
\end{tikzpicture}
\end{codeexample}

	The optional argument \meta{variables} consists of a comma
	separated	list of macros,	registers or keys, representing the
	parameters of the pattern that may vary. If a variable is a key,
	then the full path name must be used (specifically, it must start
	with |/|).
	As an example, a list might look like the following:
	|\mymacro,\mydimen,/pgf/my key|. Note that macros and keys should
	be ``simple''. They should only store values in themselves.
	
	The effect of \meta{variables}, is the following:
  Normally, when this argument is empty, once a pattern has been
  declared, it becomes ``frozen''. This means that it is not possible
  to enlarge the pattern or change the line width later on.
  By specifying \meta{variables}, no pattern is actually created.
  Instead, the arguments are stored away
  (so the macros,	registers or keys do not have to be defined in advance).

  When the fill pattern is set, \pgfname{} checks if the pattern has
  already been created with the \meta{variables} set to their current
  values (\pgfname{} is usually ``smart enough'' to distinguish between
  macros, registers and keys). If so, this already-declared-pattern
  is used as the fill pattern.
  If not, a new instance of the pattern (which will have a
  unique internal name) is declared using the current values of
  \meta{variables}. These values are then saved and the fill pattern
  set accordingly.
	
	The following shows an example of a pattern which varies
	according to the values of the macro |\size|, the key |/tikz/radius|,
	and the \TeX{} dimension |\thickness|.

\begin{texexample}{New Pattern Example}{ex:newpattern}
\pgfdeclarepatternformonly[/tikz/radius,\thickness,\size]{rings}
{\pgfpoint{-0.5*\size}{-0.5*\size}}
{\pgfpoint{0.5*\size}{0.5*\size}}
{\pgfpoint{\size}{\size}}
{
  \pgfsetlinewidth{\thickness}
  \pgfpathcircle\pgfpointorigin{\pgfkeysvalueof{/tikz/radius}}
  \pgfusepath{stroke}
}
\newdimen\thickness
\tikzset{
  radius/.initial=4pt,
  size/.store in=\size, size=20pt,
  thickness/.code={\thickness=#1},
  thickness=0.75pt
}
\begin{tikzpicture}[rings/.style={pattern=rings}]
  \filldraw [rings, radius=2pt, size=6pt]      (0,0)   rectangle +(1.5,2);
  \filldraw [rings, radius=2pt, size=8pt]      (2,0)   rectangle +(1.5,2);
  \filldraw [rings, radius=6pt, thickness=2pt] (0,2.5) rectangle +(1.5,2);
  \filldraw [rings, radius=8pt, thickness=4pt] (2,2.5) rectangle +(1.5,2);
\end{tikzpicture}
\end{texexample}



\begin{docCommand}{pgfdeclarepatterninherentlycolored}{\oarg{variables}
    \marg{name}
    \marg{lower left}
    \marg{upper right}
    \marg{tile size}
    \marg{code}}
  This command works like |\pgfdeclarepatternuncolored|, only the
  pattern will have an inherent color. To set the color, you should
  use \pgfname's color commands, not the |\color| command, since this
  fill is not protocolled.
\end{docCommand}

\begin{texexample}{Inherently Colored}{ex:ingerentlycolored}
\pgfdeclarepatterninherentlycolored{green stars}
{\pgfpointorigin}{\pgfpoint{1cm}{1cm}}
{\pgfpoint{1cm}{1cm}}
{
  \pgfsetfillcolor{green!50!black}
  \pgftransformshift{\pgfpoint{.5cm}{.5cm}}
  \pgfpathmoveto{\pgfpointpolar{0}{4mm}}
  \pgfpathlineto{\pgfpointpolar{144}{4mm}}
  \pgfpathlineto{\pgfpointpolar{288}{4mm}}
  \pgfpathlineto{\pgfpointpolar{72}{4mm}}
  \pgfpathlineto{\pgfpointpolar{216}{4mm}}
  \pgfpathclose%
  \pgfusepath{stroke,fill}
}
\begin{tikzpicture}
  \filldraw[pattern=green stars] (0,0) rectangle (3,2);
\end{tikzpicture}
\end{texexample}



\subsection{Setting a Pattern}

Once a pattern has been declared, it can be used.

\begin{docCommand}{pgfsetfillpattern}{\marg{name}\marg{color}}
  This command specifies that paths that are filled should be filled
  with the ``color'' by the pattern \meta{name}. For an inherently
  colored pattern, the \meta{color} parameter is ignored. For
  form-only patterns, the \meta{color} parameter specifies the color
  to be used for the pattern.
\end{docCommand}
  
\begin{codeexample}[]
\begin{tikzpicture}
  \pgfsetfillpattern{stars}{red}
  \filldraw (0,0) rectangle (1.5,2);

  \pgfsetfillpattern{green stars}{red}
  \filldraw (1.5,0) rectangle (3,2);
\end{tikzpicture}
\end{codeexample}



\endinput
%To summarize, what we have been doing so far is to learn a set of primitive TikZ commands for drawing paths, drawing shapes and labeling them. All TikZ command work by passing options to them. For example to change the above line to an arrow, we just pass the option |->| to the |draw| command.
%

%\begin{tikzpicture}
%  \coordinate [label=left:\textcolor{orange}{$A$}] (A) at (0,0);
%  \coordinate [label=right:\textcolor{orange}{$B$}] (B) at (1.25,0.25);
%  \draw[->,o-stealth] (A)--(B);
%\end{tikzpicture}
%\caption{Effect of the option \protect\texttt{draw[->]}.}

%\emphasis{begin,end,->,draw}
%\begin{teXXX}
%\begin{tikzpicture}
%  ...
%  ...
%  \draw[->,blue] (A)--(B);
%\end{tikzpicture}
%\end{teXXX}
%
%\section*{Relative coordinates}
%\index{TikZ!coordinates, relative}
%A coordinate can be made "relative" by prefixing it with |++|. relative coordinates are useful in many applications.
%\medskip
%
%\noindent The code is simple, except before the coordinate you add the |++| signs. This tells the PGF engine to add the x,y dimensions of the new coordinate to that of its predecessor's. In many instances this is more intuitive and easier to determine.



%\begin{tikzpicture}
%\draw[step=0.5cm,color=gray] (-1,-1) grid (3.5,3);
%\draw[->,red,thick] (0,0) -- ++(1,0) -- ++(0,1) -- ++(-1,0) -- cycle;
%\draw[->,red,thick] (2,0) -- ++(1,0) -- ++(0,1) -- ++(-1,0) -- cycle;
%\draw[arrows=o-stealth,blue] (1.5,1.5) -- ++(1,0) -- ++(0,1) -- ++(-1,0) -- cycle;
%\end{tikzpicture}
%\caption{Example of use of the \protect\texttt{++} to specify relative coordinates.}
%\label{fig:relative}

%\begin{teXXX}
%\begin{tikzpicture}
%  \draw[step=0.5cm,color=gray] (-1,-1) grid (3.5,3);
%  \draw[red,very thick] (0,0) -- ++(1,0) -- ++(0,1) -- ++(-1,0) -- cycle;
%  \draw[red,very thick] (2,0) -- ++(1,0) -- ++(0,1) -- ++(-1,0) -- cycle;
%  \draw[->,red,very thick] (1.5,1.5) -- ++(1,0) -- ++(0,1) -- ++(-1,0) -- cycle;
%\end{tikzpicture}
%\end{teXXX}
%
%Instead of |++| you can also use a single |+|. This also specifies a relative coordinate, but it does not "update"
%the current point for subsequent usages of relative coordinates. Thus, you can use this notation to specify
%numerous points, all relative to the same "initial" point:
%

%\begin{tikzpicture}
%\draw[step=0.5cm,color=gray] (-1,-1) grid (3.5,3);
%\draw[purple, fill=white] (0,0) -- +(1,0) -- +(1,1) -- +(0,1) -- cycle;
%\draw[purple, fill=white] (2,0) -- +(1,0) -- +(1,1) -- +(0,1) -- cycle;
%\draw[purple, fill=white] (1.5,1.5) -- +(1,0) -- +(1,1) -- +(0,1) -- cycle;
%\path (0,0) node [shape=circle,draw]{(0,0)};
%\end{tikzpicture}
%\caption{Example of use of the \protect\texttt{+} to specify relative coordinates.}
%\label{fig:relative1}

%\begin{teXXX}
%  \draw (0,0) -- +(1,0) -- +(1,1) -- +(0,1) -- cycle;
%  \draw (2,0) -- +(1,0) -- +(1,1) -- +(0,1) -- cycle;
%  \draw (1.5,1.5) -- +(1,0) -- +(1,1) -- +(0,1) -- cycle;
%\end{teXXX}
%
%
%Personally, I don't favour this method of specifying co-ordinates, but it can be useful, if you are automating the production of figures through an external script\sidenote{For drawing Bezier curves, the \texttt{+} behaves differently.  You can refer to the PGF Manual for more details.}.
%
%
%\section*{Arrows}
%\index{TikZ>arrows}
%The function |->| creates a tooltip arrow. You can use different arrow tips and there is a long section for them in the PGF manual. You can even define your own.

\bgroup
%\centering
%\begin{tikzpicture}
%  \draw[->] (0,0) -- (2,0);
%  \draw[arrows=o-stealth,blue] (0,-0.3) -- (2,-0.3);
%  \draw[->,o-stealth,orange] (0,-0.6) -- (2,-0.6);
%  \draw[arrows=|-stealth,purple] (0,-0.9) -- (2,-0.9);
%\end{tikzpicture}
%\captionof{figure}{Special arrow endings}
%\label{fig:specials}
\egroup
%
%\emphasis{o,stealth,begin,end,draw}
%\begin{teXXX}
%\begin{tikzpicture}
% \draw[->] (0,0) -- (2,0);
% \draw[arrows=o-stealth,blue] (0,-0.3) -- (2,-0.3);
% \draw[->,o-stealth,orange] (0,-0.6) -- (2,-0.6);
% \draw[arrows=X-stealth,purple] (0,-0.9) -- (2,-0.9);
%\end{tikzpicture}
%\end{teXXX}

%

\begin{verbatim}
\begin{tikzpicture}
% Define the points of a regular pentagon
\path (0,0) coordinate (origin);
\path (0:1cm) coordinate (P0);
\path (1*72:1cm) coordinate (P1);
\path (2*72:1cm) coordinate (P2);
\path (3*72:1cm) coordinate (P3);
\path (4*72:1cm) coordinate (P4);
% Draw the edges of the pentagon
\draw (P0) -- (P1) -- (P2) -- (P3) -- (P4) -- cycle;
% Add "spokes"
\draw (origin) -- (P0) (origin) -- (P1) (origin) -- (P2)
(origin) -- (P3) (origin) -- (P4);
\end{tikzpicture}
\end{verbatim}





\section{Nodes}

A node is a small part of a picture. When a node is created, you provide a position where the node
should be drawn and a shape. A node of shape circle will be drawn as a |circle|, a node of shape |rectangle|
as a rectangle, and so on. A node may also contain same text, which is why they can used nodes to show text.

Finally, a node can get a name for later reference.



\emphasis{node,shape,draw}
\begin{teXXX}
\begin{tikzpicture}
\path ( 0,2) node [shape=circle,draw] {.}
( 0,1) node [shape=circle,draw] {..}
( 0,0) node [shape=circle,draw] {...}
( 1,1) node [shape=rectangle,draw] {....}
(-2,1) node [shape=rectangle,draw] {rectangle (-2,1)};
\end{tikzpicture}
\end{teXXX}
\medskip

\begin{tikzpicture}
\path ( 0,2) node [shape=circle,draw] {1}
( 0,1) node [shape=circle,draw] {\textbf{10}}
( 0,0) node [shape=circle,draw] {\textbf{100}}
( 1,1) node [shape=circle,draw] {\textbf{1000}}
(-2,1) node [shape=circle,draw] {\textbf{10000}};
\end{tikzpicture}

In the above code, this text is empty (because of the
|empty {}|). So, why do we see anything at all at all the nodes? The answer is the draw option for the node operation: It
causes the |shape| around the text" to be drawn. If you have an empty |{}|, PGF still sees the empty space as a character and justs draws around it. The reason is than TikZ automatically adds some space around the text. The amount is set
using the option |inner sep|. So, to increase the size of the nodes. Modifying the example slightly we get.



\begin{tikzpicture}
\path ( 0,2) node [shape=circle,draw] {.}
( 0,1) node [shape=circle,draw] {..}
( 0,0) node [shape=circle,draw] {...}
( 1,1) node [shape=circle,draw] {....}
(-1,1) node [shape=circle,draw] {.....};
\end{tikzpicture}

As you can observe the size of the circle has been adjusted to fit the text that is enclosing it. 
Another way to simply add a node is using the |at| syntax:

\begin{texexample}{The node command}{}
\begin{tikzpicture}
\node at (0,0) [circle, draw] {\textbf{100}};
\node at (1,1) [diamond,draw] {\textbf{100}};
\end{tikzpicture}
\end{texexample}

The \cmd{\node} is an abbreviation of the |\path| node. This is a much shorter syntax than |\path| where one would need to add a lot of redundant move-tos  \seepgfmanual{215}.

If you have many nodes another way of achieving the example outlined above is to use the |\draw| command in comination with node and at.

\begin{texexample}{The node command}{}
\begin{tikzpicture}
\tikz \draw[fill=yellow!80!black]
(0,0) node {first node}
-- (1,1) node[draw, behind path] {second node}
-- (0,2) node[fill=red!20,draw,double,rounded corners] {third node};

\node at (0,0) [circle, draw] {\textbf{100}};
\node at (1,1) [diamond,draw]{\textbf{100}};
\end{tikzpicture}
\end{texexample}

\subsection*{Drawing shapes}

PGF abd \tikzname\ come with a number of predefined shapes:
\begin{itemize}
\item rectangle
\item circle, and
\item coordinate
\end{itemize}


\begin{tikzpicture}
\draw (0,0) circle (1cm);
\draw (0.5,0) circle (0.5cm);
\draw (0,0.5) circle (0.5cm);
\draw (-0.5,0) circle (0.5cm);
\draw (0,-0.5) circle (0.5cm);
\end{tikzpicture}



A circle is specified by providing its center point and the desired radius. The
command:

\medskip

\begin{tikzpicture}
  \draw[step=0.25cm,color=green] (-1,-1) grid (1,1);
  \draw (0,0) circle (1cm);
\end{tikzpicture}
\medskip

\begin{teXXX}
\begin{tikzpicture}
  \draw (x,y) circle (dia);
\end{tikzpicture}
\end{teXXX}



You  can use one |\draw| command to draw multiple circles as shown in \fref{fig:circles}


\begin{tikzpicture} 
 \draw (0,0) 
  circle (1cm)
  circle (0.6cm)
  circle (0.2cm)
 ;
\end{tikzpicture}

\emphasis{circle,begin,end}
\begin{teXXX}
\begin{tikzpicture} 
 \draw (0,0) 
  circle (1cm)
  circle (0.6cm)
  circle (0.2cm)
 ;
\end{tikzpicture}
\end{teXXX}





\begin{center}
\begin{tikzpicture}
\draw (0,0) circle (1cm)
circle (0.6cm)
circle (0.2cm);
\end{tikzpicture}
\captionof{figure}{You can use one draw command to draw multiple circles}
\label{fig:circles}
\end{center}
\captionof{figure}{Drawing multiple circles, using mutiple \texttt{circle} commands}


\subsection{Drawing ellipses}

Ellipses can be drawn in a similar fashion to circles. As an ellipse needs two center points to be specified the command used has the following general form:

\begin{verbatim}
\draw (a,b) ellipse (r1 dim and r2 dim);
\end{verbatim}

We can draw two ellipses as shown in the figure, using the code:
\begin{teX}
\begin{tikzpicture}[scale=0.6]
\draw[color=red] (0,0) ellipse (2cm and 1cm);
\draw[color=red] (0,0) ellipse (1cm and 2cm);
\end{tikzpicture}
\end{teX}

\begin{centering}
\begin{tikzpicture}[scale=0.6]
\draw[color=red] (0,0) ellipse (2cm and 1cm);
\draw[color=red] (0,0) ellipse (1cm and 2cm);
\end{tikzpicture}
\caption[Drawing ellipses]{Use the draw command in combination with ellipse to draw ellipses}
\end{centering}


\begin{teX}
\begin{tikzpicture}
\draw (0,0) ellipse (2cm and 1cm)
ellipse (0.5cm and 1 cm)
ellipse (0.5cm and 0.25cm);
\end{tikzpicture}
\caption{Drawing multiple circles, using mutiple \texttt{draw} commands}
\end{teX}

\section{Drawing more complicated shapes}
we can place a parabola in a rectangle as shown in \fref{fig:parabola}, by using the |rectangle| and the |parabola| options.

\bgroup
\centering

\begin{tikzpicture}
\draw[color=blue] (0,0) rectangle (1,1.5)
(0,0) parabola[color=orange] (1,1.5);
\draw[xshift=1.5cm] (0,0) rectangle (1,1.5)
(0,0) parabola[bend at end] (1,1.5);
\draw[xshift=3cm] (0,0) rectangle (1,1.5)
(0,0) parabola bend (.75,1.75) (1,1.5);
\end{tikzpicture}
\captionof{figure}{Parabolas drawn using the parabola and rectangle options.}
\label{fig:parabola}
\egroup




\emphasis{parabola,rectangle}
\begin{teX}
\begin{tikzpicture}
\draw[color=blue] (0,0) rectangle (1,1.5)
(0,0) parabola[color=orange] (1,1.5);
\draw[xshift=1.5cm] (0,0) rectangle (1,1.5)
(0,0) parabola[bend at end] (1,1.5);
\draw[xshift=3cm] (0,0) rectangle (1,1.5)
(0,0) parabola bend (.75,1.75) (1,1.5);
\end{tikzpicture}
\caption{Parabolas drawn using the parabola command}
\label{fig:parabola}
\end{teX}

\subsection*{The shape library}

\begin{tikzpicture}
\draw [help lines] (0,0) grid (2,2);
\draw [blue, dashed] (1,1) circle(1cm);
\draw [red, dashed] (1,1) circle(.5cm);
\node [star, star point height=.5cm, minimum size=2cm, draw]
at (1,1) {S};
\end{tikzpicture}

\section{Iterations}
One convenient construct provided with TikZ is a |foreach| command sequence

\begin{texexample}{Tikz loops}{tz:ex}
\centering
\begin{tikzpicture}[scale=2, color=bgsexy]
\foreach \i in {1,...,4}
{
  \path (\i,0) coordinate (X\i);
  \fill (X\i) circle (1pt);
}
  \foreach \j in {1,...,3}
{
  \path (\j,1) coordinate (Y\j);
  \fill (Y\j) circle (1pt);
}
\foreach \i in {1,...,4}
{
  \foreach \j in {1,...,3}
  {
     \draw[color=bgsexy] (X\i) -- (Y\j);
  }
}
\end{tikzpicture}
\captionof{figure}{Drawing a bi-partite garph using foreach loops}
\end{texexample}



\section{The pgfplots package}



\subsection{Loading data from files}

Scientific work, especially that associated with research tends to generate
a lot of data. The data would normally come from external applications and stored in files. With |TikZ| one can import the data
by using the word |file|:

\emphasis{addplot,file,x}
\begin{teXXX}
 \addplot file {./raw/wavefunctions/wavefunc\x.dat};
\end{teXXX}

In the example we use a file with a path. The data is saved in
files with the same name but a different ending. We use a |foreach| function to add the ending i.e, the file names are |wavefunc1|, |wavefunc2| and |wavefunc3|. By using external data files and the foreach command it can substantially reduce the amount of text in the macros. This improves debugging and readability.

\begin{texexample}[colback=white]{Loading files}{ex:lfiles}
\centering
\begin{tikzpicture}[scale=0.8]
    \begin{axis}[smooth,
    xlabel=$n$,
    ylabel=$\Theta{j}{n}$]
    \foreach \x in {0,...,2}
    {
        \addplot file {./raw/wavefunctions/wavefunc\x.dat};
    }
    \legend{$j=0$,$j=1$,$j=2$};
    \end{axis}
\end{tikzpicture}
\captionof{figure}{Example plot with data imported from external files, using \texttt{file}}
\end{texexample}


\begin{teXXX}
\begin{tikzpicture}[scale=0.6]
  \begin{axis}[
    xlabel=$n$,
    ylabel=$\Theta{j}{n}$]
    \foreach \x in {0,...,2}
    {
      \addplot file {./raw/wavefunctions/wavefunc\x.dat};
    }
    \legend{$j=0$,$j=1$,$j=2$};
  \end{axis}
\end{tikzpicture}
\end{teXXX}



\section*{Plotting functions}
Functions can be defined for plotting using a variety of methods. They are powerful but generally difficult to remember.



\section{Saving Data to a file}

You can save your data to a file in many ways. One easy way is to use
the \docpkg{filecontents} package. This package extends the LaTeX environment
with the same name, but allows you to overwrite the file {\protect\ctan{filecontents}}.

\begin{teXXX}
\documentclass[justified]{tufte-book}
\usepackage{pgfplots,lipsum,booktabs}
\usepackage{pgfplotstable}
\pgfplotsset{compat=newest}
\usepackage{filecontents*}
\begin{filecontents}{my1.dat}
    Label       value       num
    Integrity     33         4
    Standalone    14         3
    Interface      6         2
    Overall       18         1
\end{filecontents*}
\begin{document}
    your code here ...
\end{document}
\end{teXXX}

It is good practice to keep, such data at the top of your file, although with
the |filecontents| package, they can be inserted anywhere. Sometimes it maybe
easier to have a number of minimal files with the type of charts you using regularly and just update the data on top. In general if the data is entered
by hand rather than generated automatically by software this is a good way
to keep your work tidy.

\newenvironment{Chart}[1][black!70!green]{%
%%  defaults
    \gdef\level##1{Level ##1}
    \def\setchartwidth##1{%
      \def\chartwidth{##1}}%
    \setchartwidth{3.9cm}%
    \def\chartcolor{#1}
    \newcommand\addTitle[2][test]{
    
    
%% For the chart title we set it in a minipage for
%% better control
    \def\charttitle{\minipage{4cm}%
       \footnotesize %
       \centering\textbf{##2}\\##1%
       \endminipage}}%
   \def\xlabel{Completion (\%)}%
%% renders the chart 
    \def\renderChart{%
%%
    \footnotesize%
%%
%%
    \IfFileExists{#1.dat}{Test}{}
   \begin{tikzpicture}
   \begin{axis}[
    xbar, width=\chartwidth,title=\charttitle,
    y=0.5cm, enlarge y limits={true, abs value=0.75},
    xmin=0, xmax=100,enlarge x limits={upper, value=0.25},
    xlabel=\xlabel,
    %ylabel=Label,
    xmajorgrids=true,
    ytick=data,
    yticklabels from table={\dataTable}{Label},
    nodes near coords, nodes near coords align=horizontal
     ]
    \addplot[draw=none, fill=\chartcolor] table [x=value, y=num]
    {\dataTable};
    \end{axis}%
    \end{tikzpicture}}}
{}

\begin{comment}
\begin{figure*}
\centering

\hskip-2cm\begin{Chart}
 \addTitle[Mechanical Systems]{Shangri-la}
 \def\dataTable{SH-mechanical.dat}
 \renderChart
\end{Chart}\hspace{0.3cm}
\begin{Chart}
 \addTitle[FM-200 System]{All areas}
 \def\dataTable{my1.dat}
 \renderChart
\end{Chart}
\begin{Chart}
 \addTitle[Electrical Works]{Merweb}
 \def\dataTable{my6.dat}
 \renderChart
\end{Chart}
\caption{Mechanical Systems Shangrila. Commissioning status}
\end{figure*}


\begin{filecontents*}{my1.dat}
Label     value       num
Integrity         33            4
Standalone      14            3
Interface        6            2
Overall           18            1
\end{filecontents*}

\begin{filecontents*}{SH-mechanical.dat}
Label     value       num
{Fan coil units}       43             8
{Air Handling Units}       13             7 
{CW Pumps}       13             6
{ECU}       11             5
{Pressurization Fans}        15             4
{Smoke Extract Fan}       5             3
{Jet fan}       5             2
{Overall}       12              1
\end{filecontents*}

\begin{filecontents*}{my6.dat}
Label    value         num   other
{Level 7}  50           11   13
L6         90           10   12
L5       80             9    16
L4       90             8    18
L3       70             7    90
L2       80             6    21
L1       70             5    22
\end{filecontents*}

\begin{filecontents*}{carparkventilation.dat}
Label    value         num   other
L5         50           11   13
L4         90           10   12
L3         80           9    16
GR         90           8    18
B1         70           7    90
B2         80           6    21
B3         70           5    22
\end{filecontents*}
%% CO SYSTEM
%% DATA
\begin{filecontents*}{carparkco.dat}
Label    value         num   other
L5         78           7   13
L4         90           6   12
L3         80           5    16
GR         90           4    18
B1         70           3    90
B2         80           2    21
B3         70           1    22
B5         50          {}    {}
\end{filecontents*}

\begin{filecontents*}{carparkco2.dat}
value,   num,   other,
78,       7,   13,
90,       6,   12,
80,       5,    16,
90,       4,    18,
70,       3,    90,
80,       2,    21,
70,       1,    22,
\end{filecontents*}
\end{comment}






















  \chapter{Presenting Data in Figures}

There can be no doubt that the hallmark of scientific reports and publications is the graphical presentation of the results. Graphs show relationships underlying observations in a way no other device can provide\footnote{\textit{Doing science: design, analysis, and communication of scientific research}
 By Ivan Valiela}. 
Charting is both an art and a science. Modern typography on charts and infographics look at Tufte as inspiration.
Tufte advocates to minimize the ink to data ratio and although this is not always possible it is good advice.
In this section we would look at charting in general which is probably of interest to most of the readers
in this book. 

\section{Tufte like charts}

During the last stages of a Project, it maybe easier to visualize the
main areas where effort needs to be exerted by using simple charts. One
such chart is shown in Figure~\ref{fig:tufte-overall}. When this chart
was prepared efforts were made to complete the physical installation
as well as plan and commission the plant. The use of colour in this
chart highlights the commissioning, so one can easily see the expectations. Although the percentages are written on top of the bars,
one need not read them to visualize how difficult is to achieve
100\% completion in a Project. On the other hand commissiong can go
fairly fast and can jump by a large percentage, just by
commissioning a couple of additional ELV systems that have approximately
a 10\% weigh factor.

One can easily fit approximately, six to seven months data on
a portrait chart, changing it around to landscape one can fit
more than a year. Personally I am not very happy with such long
projections as they are more like guesses rather than proper estimates.

One other chart that can be used to visualize progress and is more
commonly found in construction is the infamous S-curve. Now, if
the actual planning is detailed enough and granular enough to be
able to pin-point \textit{continuous} progress then it is
appropriate. using it if you can at least obtain weekly progress
estimates.


\begin{figure}[b]
 \begin{tikzpicture}
  \footnotesize
  \centering
  \begin{axis}[
        ybar, axis on top,
        title={Cumulative Progress of Works},
        height=5cm, width=13.2cm,
        bar width=0.43cm,
        ymajorgrids, tick align=inside,
        major grid style={draw=white},
        enlarge y limits={value=.1,upper},
        ymin=0, ymax=100,
        axis x line*=bottom,
        axis y line*=right,
        y axis line style={opacity=0},
        ytick={0,25,50,75,100},
        tickwidth=0pt,
        legend style={
            at={(0.5,-0.2)},
            anchor=north,
            legend columns=-1,
            % adds space between the legends
            /tikz/every even column/.append style={column sep=0.7cm}
        },
        ylabel={Percentage (\%)},
        symbolic x coords={
           Sep-11,Oct-11,Nov-11,Dec-11,
           Jan-12,Feb-12,
           Mar-12,
          Apr-12},
       xtick=data,
       nodes near coords={
        \pgfmathprintnumber[precision=2]{\pgfplotspointmeta}
       }
    ]
    \addplot [draw=none, fill=gray] coordinates {
      (Oct-11, 98)
      (Nov-11,99)
      (Dec-11,99.5)
      (Jan-12,99.7)
      (Feb-12,99.8)
       };
   \addplot [draw=none,fill=gray!75!white] coordinates {
      (Oct-11, 96)
      (Nov-11,97)
      (Dec-11,98)
      (Jan-12,98.5)
      (Feb-12,99)
        };
   \addplot [draw=none, fill=gray!50!white] coordinates {
      (Oct-11, 50)
      (Nov-11, 60)
      (Dec-11, 70)
      (Jan-12, 80)
      (Feb-12, 90)
            };
    \addplot [draw=none, fill=orange!90!white] coordinates {
      (Oct-11, 25)
      (Nov-11, 35)
      (Dec-11, 45)
      (Jan-12, 55)
      (Feb-12, 65)
          };
    \legend{First Fix,Second Fix,Final Fix,Commissioning}
  \end{axis}
  \end{tikzpicture}

\caption{\protect\raggedright Cumulative progress for all MEP works. Notice the slower rate of production during the last three months.}
\label{fig:tufte-overall}
\end{figure}

\section{Graph Design}
A good graph is uncluttered, clear and focused.

\subsection{Axis Lines}

Most problems with graphs arise from misuse of axes: too heavy, too long, wrong intersection,
ambiquous breaks or too confusing increments and incorrect proportions. An axis is a ruler that established
regular intervals for measuring the information provided. Axes may emphasize, diminish, distort, simplify
or clutter the information.

\clearpage
\begin{multicols}{2}
\subsection{Axis Length}

Graphs should utilize their space around them, as the graph itself is mostly white space. In publications the journal might want to minimize the cost of printing. An axis should not extend beyond the labeled unit od minor tick closest to the last data point.
\columnbreak
\begin{tikzpicture}
\begin{axis}
\addplot coordinates {
(0,0)
(0.5,1)
(1,2)
};
\addplot coordinates {
(0,0)
(0.9,1.3)
(1.2,2.5)
};
\end{axis}
\end{tikzpicture}
\end{multicols}














}

 
 \def\languages{%
    \part{SCRIPTS AND LANGUAGES}
    \newtcolorbox{scriptexample}[2][shavian]{colback=graphicbackground,
boxrule=0pt,toprule=0pt,colframe=white}


\chapter{Those Other Languages}
\minitoc
\parindent1em

\pagestyle{myheadings}

Probably there are more users of \latexe whose mother tongue is not English than those who speak the language. \tex out of the box does not offer facilities for using non-latin based scripts easily; presents numerous problems. The biggest problem---which has been solved to a large extent---was the entering of text without having to mark all the special
characters such as umlauts (\"o) with commands. The second issue and which has been addressed by packages such as Babel, is redefining the strings such as "Chapter" to another language. In software this is called internationalization and a governing standard is |i18n|. None of the current packages take such an approach and none of them as yet offer a satisfactory solution for |LuaLaTeX|. 

Another issue with writing systems and scripts is that of appropriate fonts. Most writing systems that have ever existed are now extinct. Only minute vestiges of one of the most ancient - Egyptian hieroglyphs - live on, unrecognized, in the Latin alphabet in which English, among hundreds of other languages, is conveyed today. The latin \textit{m}, for example, ultimately derives from the Egyptian's cononantal n-sign, depicting waves.

Many of the scripts have other peculiarities, some languages such as Hanunó'o is written vertically from bottom to top. Others from top to bottom and many others from right to left. 

\section{TeX's support for different languages}

TeX's primitives such as \cmd{\language}=\meta{number} can be used to store hyphenation patterns and exceptions for up to 256 different languages. This primitive is then used by TeX to apply an appropriate set of hyphenation rules for each paragraph or part of a paragraph in a document\footnote{\url{http://www.tug.org/utilities/plain/cseq.html language-rp}}. When TeX begins a ne paragraph it sets the \emph{current language} to \cmd{\language}. Just before it adds each new character to the paragraph in unrestricted horizontal mode, it compares the current language to \cmd{\language}. If they are different, TeX : a) changes the current language to \cmd{\language}; b) inserts a whatsit\index{whatsit>language} containing the new language and the values of |\lefthyphenmin| and |\righthyphenmin|; and c) inserts the character. The |\setlanguage| command should be used to change languages in restricted horizontal mode (i.e., inside an |\hbox|). If \meta{number} is less than 0 or greater than 255, 0 is used [455]. Plain TeX has a |\newlanguage| command which may be used to allocate numbers for languages [347]. Changes made to |\language| are local to the group containing the change 

\section{LaTeX}

As far as hyphenation patterns are concerned \latexe follows very closely to the methods employed by \tex and Plain Tex. In the source2e the File |lthyphen.dtx| describes the approach to loading the default file |hyphen.ltx| . If a file hyphen.cfg is found \latexe will load the appropriate hyphenaion patterns. Traditionally language management was achieved via Johan 
Braams package Babel which we describe in the next section.


\section{The Babel Package} 

Babel \citet{babel} was the first package to systematically offer foreign language
support for \tex. It has been updated for use with |XeTeX| and |LuaTeX| and provides an environment
in which documents can be typeset in a language
other than US English, or in more than one language
or script. However, no attempt has been done to
take full advantage of the features provided by the
latter, which would require a completely new core
(as for example polyglossia or as part of \latex3).

The package has a number of predefined language files with the extension |ldf|. 


\Describe\selectlanguage{\marg{language}}{}
When a user wants to switch from one language to another he can
do so using the macro |\selectlanguage|. This macro takes the
language, defined previously by a language definition file, as
its argument. It calls several macros that should be defined in
the language definition files to activate the special definitions
for the language chosen. For ``historical reasons'', a macro name is
converted to a language name without the leading |\|; in other words,
the two following declarations are equivalent:
\begin{verbatim}
\selectlanguage{german}
\selectlanguage{\german}
\end{verbatim}

\Describe\foreignlanguage{\marg{language}\marg{text}}
The command |\foreignlanguage| takes two arguments; the second
argument is a phrase to be typeset according to the rules of the
language named in its first argument. This command (1) only
switches the extra definitions and the hyphenation rules for the
language, \emph{not} the names and dates, (2) does not send
information about the language to auxiliary files (i.e., the
surrounding language is still in force), and (3) it works even if
the language has not been set as package option (but in such a
case it only sets the hyphenation patterns and a warning is shown).

\Describe{otherlanguage*}%
{\marg{language}{otherlanguage*}}

Same as |\foreignlanguage| but as environment. Spaces after the
environment are \textit{not} ignored.



\section{The Polyglossia package}

The \pkgname{polyglossia} package has a lot of potential and has solved many issues
but its integration with large parts of the traditional |pdfLaTeX| world
is still under development and will probably take a while before one could
declare it easy to use and bug free. For example anything with the |bidi| package has issues with loading orders for a number of packages and least of which is with
the Ams packages. So if you are going to mix a number of languages in a \XeTeX\ document
you need to take extra care.

 Polyglossia is a package for facilitating multilingual typesetting with
 \XeLaTeX\ and (at an early stage) \LuaLaTeX.  Basically, it
 can be used as a replacement of \pkg{babel} for performing the following
 tasks automatically:
 
 \begin{enumerate}
 \item Loading the appropriate hyphenation patterns.
 \item Setting the script and language tags of the current font (if possible and
       available), via the package \pkg{fontspec}.
 \item Switching to a font assigned by the user to a particular script or language.
 \item Adjusting some typographical conventions according to the current language
       (such as afterindent, frenchindent, spaces before or after punctuation marks,
       etc.).
 \item Redefining all document strings (like chapter, “figure”, “bibliography”).
 \item Adapting the formatting of dates (for non-Gregorian calendars via external
       packages bundled with polyglossia: currently the Hebrew, Islamic and Farsi
       calendars are supported).
 \item For languages that have their own numbering system, modifying the formatting
       of numbers appropriately (this also includes redefining the alphabetic sequence
       for non-Latin alphabets).\footnote{ %
         For the Arabic script this is now done by the bundled package \pkg{arabicnumbers}.}
 \item Ensuring proper directionality if the document contains languages
       that are written from right to left (via the package \pkg{bidi},
       available separately).
 \end{enumerate}
 
 Several features of \pkg{babel} that do not make sense in the \XeTeX\ world (like font
 encodings, shorthands, etc.) are not supported.
 Generally speaking, \pkg{polyglossia} aims to remain as compatible as possible
 with the fundamental features of \pkg{babel} while being cleaner, light-weight,
 and modern. The package \pkg{antomega} has been very beneficial in our attempt to
 reach this objective.


\section{Loading language definition files}

The recommended way of \pkg{polyglossia} to load language definition files
is given in the manual as:
 
\Describe{\setdefaultlanguage}{\oarg{options}\marg{lang}}
 (or equivalently \cmd\setmainlanguage).
 Secondary languages can be loaded with

\Describe{\setotherlanguage}{\oarg{options}\marg{lang}}
 These commands have the advantage of being explicit and of allowing you to set
 language-specific options.\footnote{ %
 More on language-specific options below.}
 It is also possible to load a series of secondary languages at once using

\Describe\setotherlanguages{\marg{lang1,lang2,lang3,\ldots}}

 Language-specific options can be set or changed at any time by means of
\Describe\setkeys{\marg{lang}\marg{opt1=value1,opt2=value2,\ldots}}

\subsection{Bidirectional languages}





\begin{comment}
\begin{Arabic}
ّ هو إذ الغاية؛ شريف الفوائد، جم المذهب، عزيز فنّ التاريخ فنّ أنّ اعلم
والملوك سيرهم، في والأنبياء أخلاقهم، في الأمم من الماضين أحوال على يوقفنا
ّ أحوال في يرومه لمن ذلك في الإقتداء فائدة تتم حتّى وسياستهم؛ دولهم في
والدنيا. الدين
\end{Arabic}
\end{comment}

The Greek language is represented both in modern Greek as well as its ancient variants.

\begin{verbatim}
\begin{greek}
\textbf{Η ελληνική γλώσσα} είναι μία από τις ινδοευρωπαϊκές γλώσσες, για την
οποία έχουμε γραπτά κείμενα από τον 15ο αιώνα π.Χ. μέχρι σήμερα. Αποτελεί το
μοναδικό μέλος ενός κλάδου της ινδοευρωπαϊκής οικογένειας γλωσσών. Ανήκει
επίσης στον βαλκανικό γλωσσικό δεσμό.\\	
(\today) 
\end{greek}
\end{verbatim}

\topline

\textbf{Η ελληνική γλώσσα} είναι μία από τις ινδοευρωπαϊκές γλώσσες, για την
οποία έχουμε γραπτά κείμενα από τον 15ο αιώνα π.Χ. μέχρι σήμερα. Αποτελεί το
μοναδικό μέλος ενός κλάδου της ινδοευρωπαϊκής οικογένειας γλωσσών. Ανήκει
επίσης στον βαλκανικό γλωσσικό δεσμό.\\	
(\today) 

\bottomline

\begin{verbatim}
\begin{russian}
\textbf{Русский язык} — один из восточнославянских языков, один из 
крупнейших языков мира, в том числе самый распространённый из славянских
языков и самый распространённый язык Европы, как географически, так и по
числу носителей языка как родного (хотя значительная, и географически бо́
льшая, часть русского языкового ареала находится в Азии).	\\
(\today)
\end{russian}
\end{verbatim}



\textbf{Русский язык} — один из восточнославянских языков, один из крупнейших языков мира, в том числе самый распространённый из славянских языков и самый распространённый язык Европы, как географически, так и по числу носителей языка как родного (хотя значительная, и географически бо́льшая, часть русского языкового ареала находится в Азии).	\\
(\today)


\section{The Translator package}

The \pkgname{translator} package was developed by \person{Till Tantau} \citep{translator}. It provides a flexible
mechanism for translating individual words into different languages.
For example, it can be used to translate a word like ``figure'' into,
say, the German word ``Abbildung''. Such a translation mechanism is
useful when the author of some package would like to localize the
package such that texts are correctly translated into the language
preferred by the user. The translator package is \emph{not} intended
to be used to automatically translate more than a few words. 

You may wonder whether the translator package is really necessary
since there is the (very nice) |babel| package available for
\LaTeX. This package already provides translations for words like
``figure''. Unfortunately, the architecture of the babel package was
designed in such a way that there is no way of adding translations of
new words to the (very short) list of translations directly build into
babel.

The translator package was specifically designed to allow an easy
extension of the vocabulary. It is both possible to add new words that
should be translated and translations of these words.

\subsection{Using the Translator Package}

  The \pkg{Translator} needs to be used with Babel and I am not too sure yet 
  if it is ready  to be used with Polyglossia.

Once the package has loaded a language or a set of languages the optional argument to the
\cmd{\translate} can be used to translate a string. 

\begin{texexample}{Translating strings}{ex:translator}
  \translate[to=german]{rightpagename}
  \translate[to=dutch]{rightpagename}
\end{texexample}

Before you can provide the translations you need to provide your own dictionaries, where you require them. These need to be installed at a place where \tex can find them.

\CMDI{\ProvidesDictionary}

The dictionary has to be saved in a specific format that relates to the \cmd{\ProvidesDictionary} command. The second argument of the command must be appended to the file name; for the example the file is saved as\footnote{This  example is from the translator package bundle and is under the folder \texttt{base}}:

|translator-basic-dictionary-German|

The concepts take a bit of time to sink in, but once you have everything set up, it is quite easy and straight forward to incorporate it, into your package. 

\begin{teXXX}
\ProvidesDictionary{translator-basic-dictionary}{German}

\providetranslation{Abstract}{Zusammenfassung}
\providetranslation{Addresses}{Adressen}
\providetranslation{addresses}{Adressen}
\providetranslation{Address}{Adresse}
\providetranslation{address}{Adresse}
\providetranslation{and}{und}
\providetranslation{Appendix}{Anhang}
\providetranslation{Authors}{Autoren}
\providetranslation{authors}{Autoren}
\providetranslation{Author}{Autor}
\providetranslation{author}{Autor}
\end{teXXX} 

This is in contrast to Babel and Polyglossia that define
commands for each string to be translated such as,

\begin{teXXX}
\def\captionsdutch{%
    \def\prefacename{Voorwoord}%
    \def\refname{Referenties}%
    \def\abstractname{Samenvatting}%
    \def\bibname{Bibliografie}%
    \def\chaptername{Hoofdstuk}%
    \def\appendixname{Bijlage}%
    \def\contentsname{Inhoudsopgave}%
    \def\listfigurename{Lijst van figuren}%
    \def\listtablename{Lijst van tabellen}%
    \def\indexname{Index}%
    \def\figurename{Figuur}%
    \def\tablename{Tabel}%
    \def\partname{Deel}%
    \def\enclname{Bijlage(n)}%
    \def\ccname{cc}%
    \def\headtoname{Aan}%
    \def\pagename{Pagina}%
    \def\seename{zie}%
    \def\alsoname{zie ook}%
    \def\proofname{Bewijs}%
    \def\glossaryname{Verklarende woordenlijst}%
    \def\today{\number\day~\ifcase\month%
      \or januari\or februari\or maart\or april\or mei\or juni\or
      juli\or augustus\or september\or oktober\or november\or
      december\fi
      \space \number\year}}
\end{teXXX}

\begin{macro}{\usedictionary}\marg{kind}
  This command tells the |translator| package, that at the beginning of
  the document it should load \textit{all} dictionaries of kind \meta{kind} for
  the languages used in the document. Note that the dictionaries are
  not loaded immediately, but only at the beginning of the document.

  If no dictionary of the given \emph{kind} exists for one of the
  language, nothing bad happens.

  Invocations of this command accumulate, that is, you can call it
  multiple times for different dictionaries.
\end{macro}

\Describe{\uselanguage}{\marg{list of languages}}
  This command tells the |translator| package that it should load the
  dictionaries for all languages in the \meta{list of languages}. The
  dictionaries are loaded at the beginning of the document.

\section{Fonts for All the World Scripts}

Many commercial as well as open source fonts exist that can be used to typeset text the world's scripts and languages. The aim of this section of the documentation is to present an overview of the most common scripts represented in the Unicode~7.0 standard. All the examples require the use of the \XeTeX\ engine. In addition you need to have a copy of the font on your own system. If you do not have them, the font loading mechanism of \XeTeX\ will take some time to search all the directories and slows compilation tremendously. 




\section{Pan-Unicode Fonts}

Thousands of fonts exist on the market, but fewer than a dozen fonts—sometimes described as "pan-Unicode" fonts—attempt to support the majority of Unicode's character repertoire. Instead, Unicode-based fonts typically focus on supporting only basic ASCII and particular scripts or sets of characters or symbols. Several reasons justify this approach: applications and documents rarely need to render characters from more than one or two writing systems; fonts tend to demand resources in computing environments; and operating systems and applications show increasing intelligence in regard to obtaining glyph information from separate font files as needed, i.e. font substitution. Furthermore, designing a consistent set of rendering instructions for tens of thousands of glyphs constitutes a monumental task; such a venture passes the point of diminishing returns for most typefaces.

The \texttt{NotSerif} font from Google\footnote{\protect\url{http://www.google.com/get/noto/}} has good support for many languages.

Another freeware pan-Unicode font is Titus\footnote{\protect\url{http://titus.fkidg1.uni-frankfurt.de/unicode/tituut.asp?Inp1=A&Inp2=B&Inp3=C&Inp4=d%40e.com&Inp6=0&Inp5=1}}
This is an extended version of this font is TITUS Cyberbit Unicode, includes 36,161 characters in v4.0.

\newfontfamily\titus[Scale=1.05]{TITUSCBZ.ttf}
\newfontfamily\noto{NotoSerif-Regular.ttf}

\begin{scriptexample}[]{Titus}
\titus

\lorem
\end{scriptexample}
\bigskip

\begin{scriptexample}[]{Noto}
\noto

\lorem
\end{scriptexample}


\section{The \texttt{ucharclasses} package}

For multilingual texts font switching can become cumbersome. The use of a pan-Unicode font as the default can help. However, if the languages are distinct enough to use different Unicode blocks, which are not covered by the \pkgname{polyglossia} package Mike Kamermans' package \pkgname{ucharclasses} can be used.

\begin{verbatim}
% and the font switching magic
\usepackage[CJK, Latin, Thai, Sinhala, Malayalam, DominoTiles, MahjongTiles]{ucharclasses}
\usepackage{fontspec}

% default transition uses the widest coverage font I know of
\setDefaultTransitions{\fontspec{Code2000.ttf}}{}

% overrides on the default rules for specific informal groups
\setTransitionsForLatin{\fontspec{Palatino Linotype}}{}
\setTransitionsForCJK{\fontspec{code2000.ttf}}{}%HAN NOM A
\setTransitionsForJapanese{\fontspec{code2000.ttf}}{}%Ume Mincho

% overrides on the default rules for specific unicode blocks
\setTransitionTo{CJKUnifiedIdeographsExtensionB}{\fontspec{SimSun-ExtB}}
\setTransitionTo{Thai}{\fontspec{IrisUPC}}
\setTransitionTo{Sinhala}{\fontspec{Iskoola Pota}}
\setTransitionTo{Malayalam}{\fontspec{Arial Unicode MS}}

\end{verbatim}

\bgroup
\begin{verbatim}
domino tiles, 🁇 🀼 🁐 🁋 🁚 🁝, and mahjong tiles: 🀑 🀑 🀑 🀒 🀒 🀒 🀕 🀕 🀕 🀗 🀗 🀗 🀅 🀅 (using FreeFont)
\end{verbatim}

domino tiles, 🁇 🀼 🁐 🁋 🁚 🁝, and mahjong tiles: 🀑 🀑 🀑 🀒 🀒 🀒 🀕 🀕 🀕 🀗 🀗 🀗 🀅 🀅 (using FreeFont)
\egroup

\section{PhD Settings}

\def\test{}
\cxset{language/.code=\test}
\cxset{language=greek}
\cxset{languages/.code=\test}
\cxset{languages={english,greek,spanish,chinese}}
\cxset{greek font/.code=\test}
\cxset{greek font=code2000.ttf}

\begin{key}{/chapter/language=\meta{language name}}  
The key language sets the main language for the document. This language will be used for the sectioning commands and common string translations.

If the language is English Polyglossia or Babel are not loaded automatically. If the language is other than English we load either Babel or Polyglossia depending on the engine used.
\end{key}


\begin{key}{/chapter/languages=\meta{language1, language2, language3}}  
The key |languages|, determines all the other scripts available for typesetting. For each language default font commands are create automatically. The aim is to be able to run a fully multilingual system with the minimum of upfront settings. These we leave to customize in the style template files.
\end{key}

\begin{key}{/chapter/greek font=\meta{options}\meta{font file}}  
The package comes with numerous language and appropriate default fonts
for each operating system. 
\end{key}

\section{Ancient and Historic Scripts}

Unicode encodes a number of ancient scripts, which have not been in normal use for a millennium or more, as well as historic scripts, whose usage ended in recent centuries. Although these scripts are no longer used to write living languages, documents and inscriptions using these languages exist, both for extinct languages and for precursors of modern languages. The primary user communities for these scripts are scholars, interested in studying the scripts and the languages written in them. A few, such as Coptic, also have contemporary liturgical or other special purposes. Some of the historic scripts are related to each other as well as to modern alphabets. The following are provides as of Unicode version~6.2.

\begin{center}
\begin{tabular}{lll}
Ogham.     &Ancient Anatolian Alphabets. &Avestan.\\
Old Italic. &Old South Arabian. &Ugaritic\\
Runic &Phoenician. &Old Persian\\
Gothic &Imperial Aramaic &Sumero-Akkadian\\
Old Turkic. &Mandaic &Egyptian Hieroglyphs.\\
Linear B &Inscriptional Parthian &Meroitic.\\
Cypriot Syllabary &Inscriptional Pahlavi&\\
\end{tabular}
\end{center}

The following scripts are also encoded but following the Unicode
convention are described in other sections

\begin{center}
\begin{tabular}{llllll}
Coptic &Glagolithic &Phags-pa. &Kaithi &Kharoshi &Brahmi.\\
\end{tabular}
\end{center}


^^A\subsection{Ogham}

\newfontfamily\ogham{code2000.ttf}

Ogham was added to the Unicode Standard in September 1999 with the release of version 3.0.
The spelling of the names given is a standardisation dating to 1997, used in Unicode Standard and in Irish Standard 434:1999.
The Unicode block for ogham is \texttt{U+1680–U+169F}.

\begin{scriptexample}[]{Ogham}
\bgroup
\ogham
0	1	2	3	4	5	6	7	8	9	A	B	C	D	E	F\\
U+168x	   	ᚁ	ᚂ	ᚃ	ᚄ	ᚅ	ᚆ	ᚇ	ᚈ	ᚉ	ᚊ	ᚋ	ᚌ	ᚍ	ᚎ	ᚏ\\
U+169x	ᚐ	ᚑ	ᚒ	ᚓ	ᚔ	ᚕ	ᚖ	ᚗ	ᚘ	ᚙ	ᚚ	᚛	᚜	\\

\titus

0	1	2	3	4	5	6	7	8	9	A	B	C	D	E	F\\
U+168x	   	ᚁ	ᚂ	ᚃ	ᚄ	ᚅ	ᚆ	ᚇ	ᚈ	ᚉ	ᚊ	ᚋ	ᚌ	ᚍ	ᚎ	ᚏ\\
U+169x	ᚐ	ᚑ	ᚒ	ᚓ	ᚔ	ᚕ	ᚖ	ᚗ	ᚘ	ᚙ	ᚚ	᚛	᚜
\egroup		
\end{scriptexample}
^^A\section{Ancient Anatolian Alphabets}

The Anatolian scripts described in this section all date from the first millenium BCE, and were used to write various ancient Indo-European languages of western and southwestern Anatolia (now Turkey). All are related to the Greek script and are probably adaptations of it. 

\newfontfamily\lycian{Aegean.ttf}
\let\lydian\lycian
\let\carian\lydian

\begin{description}
\item [Lycian] The Lycian alphabet was used to write the Lycian language. It was an extension of the Greek alphabet, with half a dozen additional letters for sounds not found in Greek. It was largely similar to the Lydian and the Phrygian alphabets.
 
\bgroup
\lydian
\obeylines
0	1	2	3	4	5	6	7	8	9	A	B	C	D	E	F
U+1028x	𐊀	𐊁	𐊂	𐊃	𐊄	𐊅	𐊆	𐊇	𐊈	𐊉	𐊊	𐊋	𐊌	𐊍	𐊎	𐊏
U+1029x	𐊐	𐊑	𐊒	𐊓	𐊔	𐊕	𐊖	𐊗	𐊘	𐊙	𐊚	𐊛	𐊜

Typeset with the \idxfont{Aegean.ttf} and the command \cmd{\lydian}
\egroup

\item[Lydian] Lydian script was used to write the Lydian language. That the language preceded the script is indicated by names in Lydian, which must have existed before they were written. Like other scripts of Anatolia in the Iron Age, the Lydian alphabet is a modification of the East Greek alphabet, but it has unique features. The same Greek letters may not represent the same sounds in both languages or in any other Anatolian language (in some cases it may). Moreover, the Lydian script is alphabetic.
Early Lydian texts are written both from left to right and from right to left. Later texts are exclusively written from right to left. One text is boustrophedon. Spaces separate words except that one text uses dots. Lydian uniquely features a quotation mark in the shape of a right triangle.
The first codification was made by Roberto Gusmani in 1964 in a combined lexicon (vocabulary), grammar, and text collection.


\bgroup
\lycian
\obeylines
	0	1	2	3	4	5	6	7	8	9	A	B	C	D	E	F
U+1092x	𐤠	𐤡	𐤢	𐤣	𐤤	𐤥	𐤦	𐤧	𐤨	𐤩	𐤪	𐤫	𐤬	𐤭	𐤮	𐤯
U+1093x	𐤰	𐤱	𐤲	𐤳	𐤴	𐤵	𐤶	𐤷	𐤸	𐤹						𐤿
Typeset with the \idxfont{Aegean.ttf} and the command \cmd{\lycian}

Examples of words

𐤬𐤭𐤠  - Ora - "Month"

𐤬𐤳𐤦𐤭𐤲𐤬𐤩  - Laqrisa - "Wall"

𐤬𐤭𐤦𐤡  - "House, Home"

\egroup

\item [Carian] The Carian alphabets are a number of regional scripts used to write the Carian language of western Anatolia. They consisted of some 30 alphabetic letters, with several geographic variants in Caria and a homogeneous variant attested from the Nile delta, where Carian mercenaries fought for the Egyptian pharaohs. They were written left-to-right in Caria (apart from the Carian–Lydian city of Tralleis) and right-to-left in Egypt. Carian was deciphered primarily through Egyptian–Carian bilingual tomb inscriptions, starting with John Ray in 1981; previously only a few sound values and the alphabetic nature of the script had been demonstrated. The readings of Ray and subsequent scholars were largely confirmed with a Carian–Greek bilingual inscription discovered in Kaunos in 1996, which for the first time verified personal names, but the identification of many letters remains provisional and debated, and a few are wholly unknown.

\begin{scriptexample}[]{Carian}
\bgroup
\carian
\obeylines
 	0	1	2	3	4	5	6	7	8	9	A	B	C	D	E	F
U+102Ax	𐊠	𐊡	𐊢	𐊣	𐊤	𐊥	𐊦	𐊧	𐊨	𐊩	𐊪	𐊫	𐊬	𐊭	𐊮	𐊯
U+102Bx	𐊰	𐊱	𐊲	𐊳	𐊴	𐊵	𐊶	𐊷	𐊸	𐊹	𐊺	𐊻	𐊼	𐊽	𐊾	𐊿
U+102Cx	𐋀	𐋁	𐋂	𐋃	𐋄	𐋅	𐋆	𐋇	𐋈	𐋉	𐋊	𐋋	𐋌	𐋍	𐋎	𐋏
U+102Dx	𐋐
\egroup
\end{scriptexample}

\newfontfamily\oldpunctuation{code2000.ttf}

Word dividers are infrequent, \emph{scriptio continua}\footnote{a style of writing without word dividers, that is, without spaces or other marks between words or sentences} is common. Words dividers which are attested are U+00B7 (\char"00B7) \textsc{MIDLE DOT} (or U+2E31 word separator middle dot), U+205A TWO DOT PUNCTUATION, and U+205D TRICOLON ({\oldpunctuation\char"205D}). In modern editions U+0020 SPACE may be found.

\end{description}
^^A

\section{Avestan script}
\label{s:avestan}
The Avestan alphabet is a writing system developed during Iran's Sassanid era (AD 226–651) to render the Avestan language.
As a side effect of its development, the script was also used for Pazend, a method of writing Middle Persian that was used primarily for the Zend commentaries on the texts of the Avesta. In the texts of Zoroastrian tradition, the alphabet is referred to as \emph{din dabireh} or \emph{din dabiri}, Middle Persian for "the religion's script".

The Avestan alphabet was replaced by the Arabic alphabet after Persia converted to Islam during the 7th century CE. 


Notable Features

The alphabet is written from right to left, in the same way as Syriac, Arabic and Hebrew.
See more at: \url{http://www.iranchamber.com/scripts/avestan_alphabet.php#sthash.ZRu7AkEb.dpuf}

\newfontfamily\avestan{NotoSansAvestan-Regular.ttf}



\begin{scriptexample}[]{Avestan}
\ifxetex\TeXXeTstate=1
\beginR\fi
\avestan\raggedleft
𐬄	
𐬅	
𐬆	
𐬇	
𐬈	
𐬉	
𐬊	
𐬋	
𐬌	
𐬍	
𐬎	
𐬏	
𐬐	
	
𐬒	
𐬓	
𐬔	
	
𐬖	
𐬗	
𐬘	
𐬙	
𐬚	
𐬛	
𐬜	
𐬝	
𐬞	
𐬟	
𐬠	
𐬡	
𐬢	
𐬣	
𐬤	
𐬥	
𐬦	
𐬧	
𐬨	
𐬩	
𐬪	
𐬫	
𐬬	
𐬭	
𐬮	
𐬯	
𐬰	
𐬱	
𐬲	
𐬳	
𐬴	
𐬵	
\ifxetex\endR
\TeXXeTstate=0\fi
\end{scriptexample}

The recent Google font \url{NotoSansAvestan-Regular_0.ttf} can be used to typeset the Avestan script, but really not suitable for any serious study of the language.
^^A\subsection{Old Turkic}

\newfontfamily\oldturkic{Segoe UI Symbol}
\begin{scriptexample}[]{Old Turkish}
\oldturkic
\obeylines
Orkhon	Yenisei
variants	Transliteration / transcription
Old Turkic letter  𐰀	𐰁 𐰂	a, ä
Old Turkic letter  𐰃	𐰄 𐰅	y, i (e)
Old Turkic letter  𐰆		o, u
Old Turkic letter  𐰇	𐰈	ö, ü

	0	1	2	3	4	5	6	7	8	9	A	B	C	D	E	F
U+10C0x	𐰀	𐰁	𐰂	𐰃	𐰄	𐰅	𐰆	𐰇	𐰈	𐰉	𐰊	𐰋	𐰌	𐰍	𐰎	𐰏
U+10C1x	𐰐	𐰑	𐰒	𐰓	𐰔	𐰕	𐰖	𐰗	𐰘	𐰙	𐰚	𐰛	𐰜	𐰝	𐰞	𐰟
U+10C2x	𐰠	𐰡	𐰢	𐰣	𐰤	𐰥	𐰦	𐰧	𐰨	𐰩	𐰪	𐰫	𐰬	𐰭	𐰮	𐰯
U+10C3x	𐰰	𐰱	𐰲	𐰳	𐰴	𐰵	𐰶	𐰷	𐰸	𐰹	𐰺	𐰻	𐰼	𐰽	𐰾	𐰿
U+10C4x	𐱀	𐱁	𐱂	𐱃	𐱄	𐱅	𐱆	𐱇	𐱈	

\hfill  Typeset with \texttt{Segoe UI Symbol} \cmd{\oldturkic} 
\end{scriptexample}

Irk Bitig or Irq Bitig (Old Turkic: {\bfseries\Large\oldturkic 𐰃𐰺𐰴 𐰋𐰃𐱅𐰃𐰏}), known as the Book of Omens or Book of Divination in English, is a 9th-century manuscript book on divination that was discovered in the "Library Cave" of the Mogao Caves in Dunhuang, China, by Aurel Stein in 1907, and is now in the collection of the British Library in London, England. The book is written in Old Turkic using the Old Turkic script (also known as "Orkhon" or "Turkic runes"); it is the only known complete manuscript text written in the Old Turkic script.[1] It is also an important source for early Turkic mythology.

The Old Turkic text does not have any sentence punctuation, but uses two black lines in a red circle as a word separation mark in order to indicate word boundaries as shown in Figure~{\ref{omen}}

\begin{figure}[htb]
\includegraphics[width=0.7\textwidth]{./images/omen.jpg}
\caption{Omen 11 (4-4-3 dice) of the Irk Bitig (folio 13a): "There comes a messenger on a yellow horse (and) an envoy on a dark brown horse, bringing good tidings, it says. Know thus: (The omen) is extremely good."}
\label{omen}
\end{figure}
^^A\section{Phoenician}
\label{s:phoenician}
\arial

The Phoenician alphabet and its successors were widely used over a broad area surrounding the Mediterranean Sea.

\let\phoenician\lycian

\begin{scriptexample}[]{Phoenician}

\unicodetable{phoenician}{"10900,"10910}

\end{scriptexample}

The Phoenician alphabet, called by convention the Proto-Canaanite alphabet for inscriptions older than around 1200 BCE, is the oldest verified consonantal alphabet, or abjad.[1] It was used for the writing of Phoenician, a Northern Semitic language, used by the civilization of Phoenicia. It is classified as an abjad because it records only consonantal sounds (matres lectionis were used for some vowels in certain late varieties).

Phoenician became one of the most widely used writing systems, spread by Phoenician merchants across the Mediterranean world, where it evolved and was assimilated by many other cultures. The Aramaic alphabet, a modified form of Phoenician, was the ancestor of modern Arabic script, while Hebrew script is a stylistic variant of the Aramaic script. The Greek alphabet (and by extension its descendants such as the Latin, the Cyrillic, and the Coptic) was a direct successor of Phoenician, though certain letter values were changed to represent vowels.

\begin{figure}[ht]
\includegraphics[width=\textwidth]{./images/phoenician.jpg}
\captionof{figure}{
Phoenician votive inscription from Idalion (Cyprus), 390 BC. BM 125315 from The Early Alphabet by John F. Healy.}
\end{figure}

As the letters were originally incised with a stylus, most of the shapes are angular and straight, although more cursive versions are increasingly attested in later times, culminating in the Neo-Punic alphabet of Roman-era North Africa. Phoenician was usually written from right to left, although there are some texts written in boustrophedon.


\printunicodeblock{./languages/phoenician.txt}{\phoenician}


\newpage
\section{Palmyrene}
\idxlanguage{Palmyrene}
\arial

Palmyrene is the very widely attested Aramaic dialect and script
of Palmyra in the Syrian desert. The texts date from the midfirst century to the destruction of Palmyra by the Romans in AD 272. Palmyra in the Roman period was a major trading centre.
\medskip

\begin{figure}[ht]
\centering

\includegraphics[width=0.9\textwidth]{./images/palmyrene.jpg}
\captionof{figure}{\protect\arial Limestone bust with Palmyrene inscription. Palmyra late 2nd century AD. BM WA 102612}

\end{figure}

\medskip
The longest of the Palmyrene texts, is the bilingual  taxation tariff written for the year 137 AD in Palmyrene Aramaic and Greek.\footnote{For more details see:MILIK J.T., Dédicaces faites par des dieux (Palmyre, Hatra, 
Tyr) et de thiases sémitiques à l'époque romaine, Paris 1972; ROSENTHAL R., Die 
Sprache der palmyrenischen Inschriften, Leipzig 1936; STARK J.K., Personal Names in 
Palmyrene Inscriptions, Oxford 1971; DRIJVERS H.J.W., The Religion of Palmyra, 
Leiden 1976; TEIXIDOR J., 'Palmyre et son commerce d'Auguste à Caracalla', in 
Semitica 34, (1984) 1-127.  } Trade connections 
took the Palmyrene script to other places, some not far away, such as Dura Europos on the Euphrates, butothers at a great distance. A particular inscription is from South Shields, Roman Arbeia, in the north-east of England, carved on behalf of a Palmyrene mechant for his deceased wife and probably dating to the early third century AD. 

The Palmyrene script existed in two main varieties, a monumental and a cursive one, though the latter is little known and the evidence  mostly from Palmyra itself. The Syriac script of Edessa in southern Turkey, is often regarded as derived or closely related to the Palmyrene---similarities are found in the letters: ', b, g, d, w, h, y, k, l, m, n, `, r and t---though a strong case can also be made for connecting Syriac with a northern Mesopotamian script-family represented principally in texts from Hatra, a city more or less contemporary with Palmyra in Upper Mesopotamia. 


\begin{figure}[ht]
\includegraphics[width=\textwidth]{./images/regina-epigraph.jpg}
\caption{It was customary for Palmyrenes to offer bilingual texts (Greek or Latin) on funerary monuments. The final line of Regina's epitaph is Barates' personal lament in Palmyrene: Regina, freedwoman of Barate, alas. (See \href{http://www2.cnr.edu/home/araia/regina.html}{regina}.)}
\end{figure}

A good article on the classification of Aramaic languages can be found in \textit{The Aramaic language and Its Classification} by Efrem Yildiz.\footnote{\url{http://www.jaas.org/edocs/v14n1/e8.pdf}}








^^A\newfontfamily\aegyptus{AegyptusR.ttf}

\chapter{Aegyptian Hieroglyphics}

\index{fonts>Aegyptus}\index{Aegyptus (font)}
\index{fonts>Hieroglyphics}\index{languages>hieroglyphics}

\newfontfamily\hiero{NotoSansEgyptianHieroglyphs-Regular.ttf}

Hieroglyphic writing appeared in Egypt at the end of the fourth millennium bce. The writing
system is pictographic: the glyphs represent tangible objects, most of which modern
scholars have been able to identify. A great many of the pictographs are easily recognizable
even by nonspecialists. Egyptian hieroglyphs represent people and animals, parts of the
bodies of people and animals, clothing, tools, vessels, and so on.

Hieroglyphs were used to write Egyptian for more than 3,000 years, retaining characteristic
features such as use of color and detail in the more elaborated expositions. Throughout the
Old Kingdom, the Middle Kingdom, and the New Kingdom, between 700 and 1,000 hieroglyphs
were in regular use. During the Greco-Roman period, the number of variants, as
distinguished by some modern scholars, grew to somewhere between 6,000 and 8,000.

Hieroglyphs were carved in stone, painted on frescoes, and could also be written with a reed
stylus, though this cursive writing eventually became standardized in what is called \emph{hieratic}
writing. Unicode does not encode the hieratic forms separately, but ust considers them as cursive forms of the hieroglyphs encoded block.

The Demotic script and then later the Coptic script replaced the earlier hieroglyphic and
hieratic forms for much practical writing of Egyptian, but hieroglyphs and hieratic continued
in use until the fourth century ce. An inscription dated August 24, 394 ce has been
found on the Gateway of Hadrian in the temple complex at Philae; this is thought to be
among the latest examples of Ancient Egyptian writing in hieroglyphs

\begin{figure}[htb]
\includegraphics[width=\textwidth]{./images/bookofthedead.jpg}
\end{figure}

In hieroglyphic texts, these drawings are not only simply arranged in sequential order, but also grouped on top of and next to each other. This rather complicates matters trying to register and reproduce hieroglyphic texts using a computer.

\section{Computer Typesetting}

Typesetting hieroglyphics with computers presents a number of problems. First is the method of inputting the characters and second the various methods required to stack hieroglyphics, the direction of writing which can be one of four different directions.

When the first computers were introduced in Egyptology in the late 1970s and the beginning of the 1980s, the graphical capacity of the machines was still in its infancy. Early attempts to register the hieroglyphic pictorial writing on computer therefore chose an encoding system to do this, using alphanumeric codes to represent or replace the graphics. To prevent many people from reinventing the wheel, during the first "Table Ronde Informatique et Egyptologie" in 1984 a committee was charged with the task to develop a uniform system for the encoding of hieroglyphic texts on computer. The resulting Manual for the Encoding of Hieroglyphic Texts for Computer-input (Jan Buurman, Nicolas Grimal, Jochen Hallof, Michael Hainsworth and Dirk van der Plas, Informatique et Egyptologie 2, Paris 1988), simply called Manuel de Codage, presents an easy to use and intuitive way of encoding hieroglyphic writing as well as the abbreviated hieroglyphic transcription (transliteration). The system proposed by the Manuel de Codage has since been adopted by international Egyptology as the official common standard for registering hieroglyphic texts on computer. Mark-Jan Nederhof proposed an enhanced encoding scheme to remove many of the limitations in the Manuel de Codage.

\pkgname{HieroTeX} is a \latexe package developed by to typeset hieroglyphic texts and still works well. The advantages of using \tex is of course its excellent typesetting capabilities and the usage of macros. Although inputting the texts as MdC codes is not that difficult, repeating the same codes over and over can be avoided with easily constructed simple substitution macros. 

\subsection{fonts}

One of the best fonts I came across is \idxfont{Aegyptus} from \url{http://users.teilar.gr/~g1951d/}\footnote{The site also has fonts for Aegean Numbers, Ancient Greek Musical Notation, Ancient Greek Numbers, Ancient Roman Symbols, Arkalochori Axe, Carian, Cypriot Syllabary, Dispilio tablet, Linear A, Linear B Ideograms, Linear B Syllabary, Lycian, Lydian, Old Italic, Old Persian, Phaistos Disc, Phoenician, Phrygian, Sidetic, Troy vessels’ signs and Ugaritic. Cretan Hieroglyphs and Cypro-Minoan script(s) are offered in separate files.}. The font provides all the unicode characters and also offers an additional number of glyphs that are not in the Unicode standard. The font uses the Unicode Private Use Areas to encode the glyphs. 

Another font is the Noto Egyptian Hieroglyphics from Google. This is a lightweight font with the symbols in their proper unicode slots. Mark-Jan Nederhof's \idxfont{NewGardiner} font is another one with support only for the Gardiner set. The codepoint mappings are incorrect, as the font has been  
encoded to EGPZ. The font is similar to the Aegyptus font, however it is just transposed and not recommended unless it is transposed. 

The editor software JSesh\footnote{\protect\url{http://jsesh.qenherkhopeshef.org/}} also provides a free font |JSeshFont.ttf|. This offers a correctly mapped unicode and is another good alternative. The symbols are drawn somewhat simpler and is just a matter of taste what you want to use.

My recommendation is for short demonstration purposes, the Noto font is to be preferred while for more serious work the Aegyptus font will be more useful. Using Lua the font can be transposed automatically to allow the use of commands that refer to unicode numbers. Another advantage of the Aegyptus font is that the glyphs are named with their Gardiner numbers, so it is somewhat easier to programmatically access them by name.\footnote{Unicode does not name the glyphs, but simply calls the Egyptian Hieroglyph $n$. } 

\medskip

\ifxetex
\bgroup
\centering 
\font\myfont = "Aegyptus"
\scalebox{7}{\myfont\XeTeXglyph 201}
\scalebox{7}{\myfont\XeTeXglyph 203}
\scalebox{7}{\myfont\XeTeXglyph 163}
\scalebox{7}{\myfont\XeTeXglyph 164}
\scalebox{7}{\myfont\XeTeXglyph 165}
\scalebox{7}{\myfont\XeTeXglyph 168}
\captionof{table}{Example of Egyptian Hieroglyphics typeset with the \textit{Aegyptus} font.} 
\egroup
\fi

\ifluatex
\bgroup
\centering 
\aegyptus
\scalebox{7}{\char"F300C}
\scalebox{7}{\char"F3001}
\scalebox{7}{\char"F3010}
\scalebox{7}{\char"F308B}
\scalebox{7}{\char"F3097}
\scalebox{7}{\char"F3091}
\captionof{table}{Example of Egyptian Hieroglyphics typeset with the \textit{Aegyptus} font.} 
\egroup

\fi


\subsection{Unicode Block}

Egyptian hieroglyphs is a Unicode block containing the Gardiner's sign list of Egyptian hieroglyphics.
The code points, in the range |0x13000| to |0x1342E|, are available starting from
\href{http://unicode.org/charts/PDF/U13000.pdf}{Unicode 5.2}

\begin{scriptexample}[]{Hieroglyphic}
\bgroup
\unicodetable{hiero}{"13000,"13010,"13020,"13030,"13040,"13050,"13060,"13070,%
"13080,%
"13090,"130A0,"130B0,"130C0,"130D0,"130E0,"130F0,%
"13100,"13110,"13120,"13130,"13140,"13150,"13060,"13070,"13080,"13090}
\egroup
\end{scriptexample}

\subsection{Gardiner's classification}

The standard reference on Egyptian hieroglyphics is Gartiner's Sign List, which lists common Egyptian hieroglyphs. These are grouped in categories from A-Aa. Each category represents a theme for example category A, is "man and his occupations". Based on this list ``Queen with flower" is denoted as \texttt{B7}. 

\subsubsection{Character Names} 

Egyptian hieroglyphic characters have traditionally been designated in
several ways:

\begin{enumerate}
\item  By complex description of the pictographs: \texttt{GOD WITH HEAD OF IBIS}, and so forth.
\item By standardized sign number: C3, E34, G16, G17, G24.
\item For a minority of characters, by transliterated sound.
\end{enumerate}

The characters in the Unicode Standard make use of the standard Egyptological catalog
numbers for the signs. Thus, the name for {\hiero\char"130F9} |U+13049| egyptian hieroglyph e034 refers
uniquely and unambiguously to the Gardiner list sign E34, described as a “{\aegean DESERT HARE}” ({\hiero \char"130FA}) and used for the sound “wn”. The Unicode catalog values are padded to three places with
zeros, so where the Gardiner classification is shown as \texttt{E34}, the unicode value is \texttt{E034}. 

Names for hieroglyphic characters identified explicitly in Gardiner 1953 or other sources as
variants for other hieroglyphic characters are given names by appending “A”, “B”, ... to the sign number. In the sources these are often identified using asterisks. Thus Gardiner’s G7,
G7*, and G7** correspond to U+13146 egyptian sign g007 {\hiero \char"13147}, U+13147 egyptian sign g007a, and U+13148 egyptian sign g007b, respectively.

\def\texthiero#1{{\color{black!95}\hiero #1}}

\begin{longtable}{>{\Large}lll>{\ttfamily}l}
{\hiero \char"13000}&A1-A70 & Man and his occupations &U+13000-1304F\\
{\hiero \char"13050}&B1-B9  &Woman and her occupations &U+13050-13059\\
{\hiero \char"1305A} &C1-C24 &Anthropomorphic Deities &U+1305A-13075\\
{\hiero \char"13076} &D1-D67 &Parts of the Human Body &U+13076-130D1\\
{\hiero \char"130D2} &E1-E38 &Mammals &U+13076-130D1\\
{\hiero \char"130FE}  &F1-F53	&Parts of Mammals &U+130FE-1313E\\
{\hiero\char"1313F}	&G1-G54	&Birds &U+1313F-1317E\\
{\hiero \char"1317F}	&H1-H8	&Parts of Birds &U+1317F-13187\\
\texthiero{\char"13188}	&I1-I15	&Amphibious Animals, Reptiles, etc. &U+13188-1319A\\
\texthiero{\char"1319B}	&K1-K8	&Fishes and Parts of Fishes &U+1319B-131A2\\
\texthiero{\char"131A3}	&L1-L8	&Invertebrata and Lesser Animals &U+131A3-131AC\\
\texthiero{\char"131AD}	&M1-M44	&Trees and Plants &U+13AD-131EE\\
\texthiero{\char"131EF}	&N1-N42	&Sky, Earth, Water &U+131EF-1321F\\
\texthiero{\char"13250}	&O1-O51	&Buildings and Parts of Buildings &U+13250-1329A\\
\texthiero{\char"1329B}	&P1-P11	&Ships and Parts of Ships &U+1329B-132A7\\
\texthiero{\char"132A8}	&Q1-Q7	& Domestic and Funerary Furniture &U+132A8-132AE\\
\texthiero{\char"132AF}	&R1-R29	&Temple Furniture and Sacret Emblems &U+132AF-132D0\\
\texthiero{\char"132D1}	&S1-S46	&Crowns, Dress, Staves, etc. &U+132D1-13306\\
\texthiero{\char"13307}	&T1-T36	&Warfare, Hunting, Butchery &U+13307-13332\\
\texthiero{\char"13333}	&U1-42	&Agriculture, Crafts and Professions &U+13333-13361\\
\texthiero{\char"13362}	&V1-V40a	&Rope, Fibre, Baskets, Bags, etc. &U+13362-133AE\\
\texthiero{\char"133AF}	&W1-W25	&Vessels of Stone and Earthenware &U+133AF-133CE\\
\texthiero{\char"133CF}	&X1-X8a	&Loaves and Cakes &U+133CF-133DA\\
\texthiero{\char"133DB}	&Y1-Y8	&Writing, Games, Music &U+133DB-133E3\\
\texthiero{\char"133E4}	&Z1-Z16H	&Strokes, Geometrical Figures, etc. &U+133E4-1340C\\
\texthiero{\char"1340D}	&Aa1-Aa32	&Unclassified &U+1340D-1342E\\
\end{longtable}

I particularly like the crocodile sign \def\crocodile{\color{teal}{\Huge\texthiero{\char"13188}}} {\crocodile}, as it is applicable to describe people in my field of work. 

\begin{scriptexample}[]{Woman and her occupations}
\unicodetable{hiero}{"13050}
\end{scriptexample}

\section{Positioning}

One of the core assumptions of any hieroglyphic encoding or mark-up scheme following the MdC is that signs and groups of signs maybe positioned next to each other or above each other. The former is indicated by the operator * and the latter by :. One may also use -, which functions as * for horizontal texts and as : for vertical text. 

In some dialects of the MdC relative positioning has been extended by the use of the |&| operator. This is used to form a kind of ligature, such as |D&t| can be defined to represent the \textit{Cobra at rest} sign I10 with sign X1 underneath, as follows:

\begin{center}
{\hiero\HUGE
       \mbox{\rlap{\char"133CF}\char"13193\hfill\hfill}\\
       {\large|insert[bs](I10,X1)|}

\mbox{\rlap{\scalebox{0.5}{\char"133E3}}\char"13193\hfill\hfill}\\
 	
}
\end{center}

This is only a partial solution and to automate it via kerning tables, will require hundreds of entries in the kerning tables. It will also need constant modifications as researchers discover new combinations. A better approach and which is easily applied to \tex based systems would be to adopt Nederhof's method of creating a new command |insert[bs](I10,X1)|. 

In \tex one could simply define a command \cmd{\insert} with one optional argument to handle the positioning. The positioning uses the letters [b,t,s,e] to position the glyph. the letters s and e stand for start and end, whereas b,t for bottom and top respectively. When there are only two symbols involved, this is not such a difficult operation, but when three or more symbols are to be grouped and kerned together, inserting with some form of scaling is necessary.

\subsection{Enclosures}

Enclosures. The two principal names of the king, the \emph{nomen} and \emph{prenomen}, were normally
written inside a \emph{cartouche}: a pictographic representation of a coil of rope.

In the Unicode representation of hieroglyphic text, the beginning and end of the cartouche
are represented by separate paired characters, somewhat like parentheses. The Unicode manual states that `rendering of a full cartouche surrounding a name requires specialized layout software', which is of course an easy task for \tex.

\begin{macro}{\cartouche}
The commands \cmd{\cartouche} and \cmd{\cartouche}, from Peter Wilson's \pkgname{hierglyph} package have been used for many years to demonstrate the use of hieroglyphics with \latexe. 
\end{macro}

There are a several characters for these start and end cartouche characters, reflecting various styles for the enclosures.

\cartouche{{\hiero \char"13147}$sin^{2} x + cos^{2} x = 1$}
\Cartouche{{\hiero \char"13147}$sin^{2} x + cos^{2} x = 1$}

Unicode:{\hiero 𓇓𓏏𓊵𓏙𓊩𓁹𓏃𓋀𓅂𓊹𓉻𓎟𓍋𓈋𓃀𓊖𓏤𓄋𓈐𓎟𓇾𓈅𓏤𓂦𓈉 }

\textpmhg{\HQ} 

\cartouche{\pmglyph{K:l-i-o-p-a-d:r-a}}
%\translitpmhg{\HK\Hl\Hi\Ho\Hp\Ha\Hd\Hr\Ha}

\printunicodeblock{./languages/hieroglyphics.txt}{\hiero}
\printunicodeblock{./languages/hieroglyphics-13100.txt}{\hiero}
\printunicodeblock{./languages/hieroglyphics-13200.txt}{\hiero}
\printunicodeblock{./languages/hieroglyphics-13300.txt}{\hiero}
\printunicodeblock{./languages/hieroglyphics-13400.txt}{\hiero}
\section{Numerals}

Egyptian numbers are encoded following the same principles used for the
encoding of Aegean and Cuneiform numbers. Gardiner does not supply a full set of
numerals with catalog numbers in his Egyptian Grammar, but does describe the system of
numerals in detail, so that it is possible to deduce the required set of numeric characters.

Two conventions of representing Egyptian numerals are supported in the Unicode Standard.
The first relates to the way in which hieratic numerals are represented. Individual
signs for each of the 1s, the 10s, the 100s, the 1000s, and the 10,000s are encoded, because in
hieratic these are written as units, often quite distinct from the hieroglyphic shapes into
which they are transliterated. The other convention is based on the practice of the \emph{Manual
de Codage}, and is comprised of five basic text elements used to build up Egyptian numerals.
There is some overlap between these two systems.

%% Needs some work to get it into LuaLaTeX
%% omitted for the time being
%\ifxetex
%\begin{texexample}{TeXeXglyph}{ex:xetexglyph}
%\raggedright
%\font\myfont = "Aegyptus"
%\setcounter{glyphcount}{136}
%
%\whiledo
%{\value{glyphcount}<\XeTeXcountglyphs\myfont}
%{\arabic{glyphcount}:~
%{\myfont\XeTeXglyph\arabic{glyphcount}}\quad
%\stepcounter{glyphcount}}
%\end{texexample}
%\fi

\section{Input Methods}

If you writing a document with a lot of hieroglyphics inputting of hieroglyphics can be problematic. Most researchers in the field will use special keyboards or editors. They also use MS/Word or OpenOffice. They can both be coerced to produce reasonable documents, but with \tex obviously better results can be achieved. One such editor is \href{http://jsesh.qenherkhopeshef.org/}{jsesh}. 


\begin{luacode*}
    local h = {}
          h = dofile("hiero.lua")
    local options = {style="block",
                     echo=true,
                     direction="RL",
                     size = "\\Huge",
                     color = "green",
                     headings = "captionof{figure}"  -- section/tablecaption/figurecaption
                     }
   -- prints full symbol list
   h.printgardiner(t,options)

   tex.print("\\par")
   local options = {style="block",
                     echo=true,
                     heading="\\par",
                     direction="RL",
                     color = "teal",
                     scale = 8}

   h.printhierochar("hiero","1317D",options)
   h.printhierochar("hiero","13000",{direction="RL",
                                        color = "teal",
                                        scale = 8})
   h.printhierochar("hiero","13003",{direction="LR",
                                        color = "teal",
                                        scale = 1})
   h.parseMdC([[M23-X1-R4-X8-Q2-D4-W17-R14-G4-R8-O29-
               V30-U23-N26-D58-O49-Z1-F13-N31-V30-N16-
               N21-Z1-D45-N25!]])

   tex.print("\\par")
   h.printgardinercat("B")

\end{luacode*}

\newcommand\hierochar[2][direction = "LR",
                         color     = "teal",
                         scale     = 1]{% 
               \luaexec{
                h = h or {}
                h = require("hiero.lua")  
                h.parseMdC(#2,{#1})}}
               
\newcommand\printhierochar[3][direction = "LR",
                              color     = "teal",
                              scale     = 4]{% 
               \luaexec{
                h = h or {}
                h = require("hiero.lua")  
                h.printhierochar(#2,#3,{#1})}}

This file just tests the various commands available for manipulating hieroglyphics. We tried to 
generalize the commands, so they can be re-used for other type of hieroglyphics.

{
\hierochar{"A1-A2-A3!"}

\centering 

\def\options{direction = "LR",
             color     = "teal",
             scale     = 7}

\def\fontname{"hiero"}

\def\hierochar#1{\printhierochar[\options]{\fontname}{#1}}
}


\begin{scriptexample}[]{Some Example}
Sometimes kerning might be required, especially if the
glyphs are scaled.This is easily achieved with a \cmd{\kern}
command and a suitable skip dimension.

\medskip

\bgroup
\fboxsep=0pt\fboxsep.4pt
\def\options{direction = "RL",
             color     = "black!95",
             scale     = 5}
\centering

\color{teal}
\fbox{\hierochar{"13051"}}
\kern-4mm
\hierochar{"13003"}
\def\options{direction = "LR",
             color     = "black!95",
             scale     = 5}
\fbox{\hierochar{"13003"}}\color{red}
\kern-4mm
\hierochar{"13051"}
\color{black!95}
\egroup
\begin{verbatim}
\centering
\hierochar{"13051"}
\kern-4mm
\hierochar{"13003"}
\def\options{direction = "RL",
             color     = "black!95",
             scale     = 5}
\hierochar{"13003"}
\kern-4mm
\hierochar{"13051"}
\end{verbatim}
\end{scriptexample}

A bit of a diversion is appropriate at this point. Our attempt after the historical overview, is to provide some routines for the capturing and display of hieroglyphic texts using LuaTeX. This involves getting low level information from the system regarding fonts. 

\begin{figure}[ht]
\begin{minipage}{0.45\textwidth}
\centering
\includegraphics[width=0.6\textwidth]{./images/fontforge.jpg}
\end{minipage}
\begin{minipage}[t]{0.45\textwidth}
\caption{Viewing font information with fontforge.}
\end{minipage}
\end{figure}

For each glyph, we are interested to get its unicode number, the position in the font table, its name and most importantly the font metrics. The font metrics are a set of parameters that are used to measure the bounding box, any ascenders or descenders and similar information. Using fontforge, these parameters can easily be viewed. However, we are not interested to make any modifications manually; what we are interested is to programmatically obtain this information using Lua. Lua's philosophy and a mantra repeated often by the developers, is that it provides the tools and not the solutions. What this means to the LuaTeX programmer, is that we need to reach very low level  to get this information, which is a road with many bumps. Luckily the tools have been provided by the LuaTeX developers. This comes with a lot of benefits as we can also do our own on the fly mapping, such as creating an index table holding all the Gardiner numbers. 

The |fontloader.open| function loads a font, but it's not usable by itself; the result should be turned into a table with
\textbf{fontloader.to\_table}, as follows:

\begin{verbatim}
  local f = fontloader.open
     ("c:/windows/fonts/NotSansEgyptianHieroglyphics-
       Regulat.ttf")
  fonttable = fontloader.to_table(f)
  fontloader.close(f)
\end{verbatim}

We will use the Google No Tofu Egyptian Hieroglyphic font to experiment with our hieroglyphics. I have used a full path to load the font, which resides on my windows machine in the fonts folder. Once we load all the information in the |fonttable| we use |fontloader.close| to discard the userdata from which the table is extracted. 

What makes OpenType fonts special is that they describe every aspect that you might be able to think of when you think of putting letters together to form words. In addition to the obvious "this is what letters look like" information, OpenType fonts also specify things like the name of each letter that is available in the font, how much of the Unicode standard the font implements, which horizontal and vertical metrics apply to which letters, exactly how the letters are arranged inside the font so that they can quickly be read out, what kind of font classifications apply (is it a fantasy font? is it bold face? is it fixed width? etc), what kind of memory allocation a printer needs to perform in order to be able to even load the font, etc. etc. etc. All these are stored in tables upon tables, similat to a collection of Russian dolls.

To view the values in the fonttable, we will first iterate over the \textbf{fonttable} and extract all the first level keys.

\begin{texexample}{Iterating through a font table}{}
\begin{luacode*}
local z={}
tf=fontloader.to_table(fontloader.open("c:/windows/fonts/NotoSansEgyptianHieroglyphs-Regular.ttf"))

-- we sort the keys to create a table
-- important keys to us are tf.glyphs

for k,v in pairs (tf) do
   --tex.print(k.."\\par")
   table.insert(z, k)
end

table.sort(z)
tex.print("\\begin{multicols}{3}\\raggedright")
for k,v in pairs (z) do
   z[k] = string.gsub(z[k],"%_","\\textunderscore ")
   local s = tf[v]
   tex.print("\\textbullet\\hskip3pt\\hangindent2em " .. z[k].." [\\textit{"..type(s).."}] ","\\par")
end
tex.print("\\end{multicols}")
\end{luacode*}
\end{texexample}

We iterate through the \textbf{fonttable} using the Lua  "pair" iterator and we simply print all the keys and the type of the values in a human readable form as shown in the example. Note the use of |\textunderscore| that replaces all underscores in the fields with its text equivalent to sanitize the output. This is a quick and dirty way to avoid the use of catcodes. Many of the keys, bear intuitive names and are not difficult to discern: \textit{version}, \textit{copyright} and the like. Getting the type of Lua variables is important in order to use them for error trapping. When you attempt for example to print a nil value an error will occur.

Now that we have peeked under the font we will iterate and capture the information of interest, which we will put into another table with two keys \textbf{info}  and \textbf{metrics}. In the metrics file we will get the bounding box related metrics of each and every glyph in the font and save it, into our own table. 

\begin{texexample}{More Metrics}{}
  \begin{luacode*}
   tex.print("units per em = ", tf.units_per_em,"\\par")
   for i,j in ipairs (tf.glyphs[6].boundingbox) do
      tex.print("bounding box["..i.."]".." = ", j,"\\par")
   end 
   local w = (tf.glyphs[6].boundingbox[3]-tf.glyphs[6].boundingbox[1])/tf.units_per_em
   local h = tf.glyphs[6].boundingbox[4]/tf.units_per_em
   tex.print("glyph width = ", w,"em\\par")
   tex.print("glyph height = ", h,"em\\par")

-- presents a nicely typeset table 

local rep, write = string.rep, tex.print
function ExploreTable (tab, offset)
    offset = offset or ""
    for k, v in pairs (tab) do
        local newoffset = offset .. "\\mbox{.}"
        if type(v) == "table" then
           -- if k == "boundingbox" then write("BB") end
           write(offset..k .. " = \\{\\par ")
           ExploreTable(v, newoffset)
           write(offset..newoffset .. "\\}\\par")
         else
           write(offset..k .. " = "..tostring(v),"\\par")
         end
      end
end

write("\\par{\\ttfamily ")
ExploreTable(tf.glyphs[38],"\\mbox{.}")
write("}")
  \end{luacode*}
\end{texexample}

The OpenType fonts standard, provides for so much information that we will ignore most of the items and focus on only a few tables and fields. A small utility after Paul Isambert's article is necessary to enable us to view tables easily within this book,


\begin{texexample}{ExploreTable utility}{}
\begin{luacode*}
-- presents a nicely typeset table 

local rep, write = string.rep, tex.print
function ExploreTable (tab, offset)
    offset = offset or ""
    for k, v in pairs (tab) do
        local newoffset = offset .. "\\mbox{.}"
        if type(v) == "table" then
           -- if k == "boundingbox" then write("BB") end
           write(offset..k .. " = \\{\\par ")
           ExploreTable(v, newoffset)
           write(offset..newoffset .. "\\}\\par")
         else
           write(offset..k .. " = "..tostring(v),"\\par")
         end
      end
end

write("\\par{\\ttfamily ")
ExploreTable(tf.glyphs[38],"\\mbox{.}")
write("}")
  \end{luacode*}
\end{texexample}

A good utility also is |TTX| that will convert an OTF font to XML and back. This requires that you have python installed.\footnote{See some good guidelines as to how to install it at \url{http://www.glyphrstudio.com/ttx/}.} The utility uses python to do the conversion. The archive can be downloaded from \url{http://sourceforge.net/projects/fonttools/files/latest/download}. This is a three prong attack. You need to have python install, the numpy library and then the TTX package. The |TTX| program was written by the font designer Just van Rossum, brother of the creator of the Python language, Guido van Rossum. The tool converts TrueType into human-readable |XML| format. The most attractive feature of this tool is that it also perform the opposite operation that is create a TruType font from an |XML| file. The |XML| format makes the hierarchy of the format clearer. Since SVG fonts are also described in |XML| it becomes an easier task to convert an |SVG| font to a TrueType font. To convert |bar.ttf| into |bar.ttx| you simply write:

\begin{verbatim}
ttx bar.ttf
\end{verbatim}

Similarly for the opposite conversion, from |.ttx| to |.ttf|

\begin{verbatim}
ttx bar.ttx
\end{verbatim}

The generated ttx file is approximately ten times larger than the original |.ttf| file. The files generated are huge affairs and difficult to manage.The command line option |-l| prints a list of the tables in the font. |TTX| is indispensable in the ``humanization'' of TrueType fonts. The details of the tables and what each field represents are eloquently described in that indispensable book by Yannis Haralambous \textit{Fonts \& Encodings.} Although the book is now somewhat dated, it is still the best source of information on many esoteric topics related to fonts. 






^^A\input{./languages/meroitic}

\subsection{Old Italic}

\newfontfamily\olditalic{seguisym.ttf}

Old Italic refers to any of several now extinct alphabet systems used on the Italian Peninsula in ancient times for various Indo-European languages (predominantly Italic) and non-Indo-European (e.g. Etruscan) languages. The alphabets derive from the Euboean Greek Cumaean alphabet, used at Ischia and Cumae in the Bay of Naples in the eighth century BC.

Various Indo-European languages belonging to the Italic branch (Faliscan and members of the Sabellian group, including Oscan, Umbrian, and South Picene, and other Indo-European branches such as Celtic, Venetic and Messapic) originally used the alphabet. Faliscan, Oscan, Umbrian, North Picene, and South Picene all derive from an Etruscan form of the alphabet.

The Germanic runic alphabet was derived from one of these alphabets by the 2nd century.
Old Italic is a Unicode block containing a unified repertoire of the three stylistic variants of pre-Roman Italic scripts.

\begin{scriptexample}[]{}
\unicodetable{olditalic}{"10300,"10310,"10320}
\end{scriptexample}

\subsection{Old South Arabian}

\newfontfamily\oldsoutharabian{NotoSansOldSouthArabian-Regular.ttf}

The ancient Yemeni alphabet (Old South Arabian ms3nd; modern Arabic: {\arabicfont المُسنَد‎}  musnad) branched from the Proto-Sinaitic alphabet in about the 9th century BC. It was used for writing the Old South Arabian languages of the Sabaic, Qatabanic, Hadramautic, Minaic (or Madhabic), Himyaritic, and proto-Ge'ez (or proto-Ethiosemitic) in Dʿmt. The earliest inscriptions in the alphabet date to the 9th century BC in Akkele Guzay, Eritrea[3] and in the 10th century BC in Yemen. There are no vowels, instead using the \emph{mater lectionis} to mark them.

Its mature form was reached around 500 BC, and its use continued until the 6th century AD, including Old North Arabian inscriptions in variants of the alphabet, when it was displaced by the Arabic alphabet.[4] In Ethiopia and Eritrea it evolved later into the Ge'ez alphabet,[1][2] which, with added symbols throughout the centuries, has been used to write Amharic, Tigrinya and Tigre, as well as other languages (including various Semitic, Cushitic, and Nilo-Saharan languages).

It is usually written from right to left but can also be written from left to right. When written from left to right the characters are flipped horizontally (see the photo).
The spacing or separation between words is done with a vertical bar mark (\textbar).
Letters in words are not connected together.

Old South Arabian script does not implement any diacritical marks (dots, etc.), differing in this respect from the modern Arabic alphabet.

\begin{scriptexample}[]{South Arabian}
\unicodetable{oldsoutharabian}{"10A60,"10A70}
\end{scriptexample}

Support in \latexe is provided via Peter Wilson's package \pkgname{sarabian}. The package provides all the |metafont| sources as well as transliteration commands and other utilities \seedocs{SARAB}.

\def\SAtdu{\oldsoutharabian\char"10A77}

A comparison between  the unicode and the rendering (scaled 5) \pkgname{sarabian} is shown below.

\centerline{\scalebox{3}{\SAtdu} \scalebox{3}{\textsarab{\SAtd}}}

There is no real advantage in using unicode fonts, if all you interested is to write some South Arabian text for inscriptions. 

\begin{symtable}[SARAB]{\SARAB\ South Arabian Letters}
\index{South Arabian alphabet}
\index{alphabets>South Arabian}
\label{sarabian}
\begin{tabular}{*4{ll@{\qquad}}ll}
\K[\textsarab{\SAa}]\SAa   & \K[\textsarab{\SAz}]\SAz   & \K[\textsarab{\SAm}]\SAm   & \K[\textsarab{\SAsd}]\SAsd & \K[\textsarab{\SAdb}]\SAdb \\
\K[\textsarab{\SAb}]\SAb   & \K[\textsarab{\SAhd}]\SAhd & \K[\textsarab{\SAn}]\SAn   & \K[\textsarab{\SAq}]\SAq   & \K[\textsarab{\SAtb}]\SAtb \\
\K[\textsarab{\SAg}]\SAg   & \K[\textsarab{\SAtd}]\SAtd & \K[\textsarab{\SAs}]\SAs   & \K[\textsarab{\SAr}]\SAr   & \K[\textsarab{\SAga}]\SAga \\
\K[\textsarab{\SAd}]\SAd   & \K[\textsarab{\SAy}]\SAy   & \K[\textsarab{\SAf}]\SAf   & \K[\textsarab{\SAsv}]\SAsv & \K[\textsarab{\SAzd}]\SAzd \\
\K[\textsarab{\SAh}]\SAh   & \K[\textsarab{\SAk}]\SAk   & \K[\textsarab{\SAlq}]\SAlq & \K[\textsarab{\SAt}]\SAt   & \K[\textsarab{\SAsa}]\SAsa \\
\K[\textsarab{\SAw}]\SAw   & \K[\textsarab{\SAl}]\SAl   & \K[\textsarab{\SAo}]\SAo   & \K[\textsarab{\SAhu}]\SAhu & \K[\textsarab{\SAdd}]\SAdd \\
\end{tabular}

\bigskip
\begin{tablenote}
  \usefontcmdmessage{\textsarab}{\sarabfamily}.  Single-character
  shortcuts are also supported: Both
  ``\verb+\textsarab{\SAb\SAk\SAn}+'' and ``\verb+\textsarab{bkn}+''
  produce ``\textsarab{bkn}'', for example.  \seedocs{\SARAB}.
\end{tablenote}
\end{symtable}


\section{South East Asian Scripts}

This section documents the facilities offered to typeset Southeast Asian Scripts. These scripts are used in most of Southeast Asia, Indonesia and the Philippines.

\begin{table}[htb]
\centering
\begin{tabular}{lll}
Thai. & Tai Tham &Balinese.\\
Lao.  &Tai Viet  &Javanese.\\
Myanmar &Kayah Li &Rejang\\
Khmer. &Cham &Batak\\
Tai Le &Philippine Scripts &Sundanese.\\
New Tai Lue & Buginese\\
\end{tabular}
\end{table}

\subsection{Thai}

\newfontfamily\thai[Scale=1.0,Script=Thai]{IrisUPC}

\def\thaitext#1{{\thai#1}}

\begin{scriptexample}[]{Thai}
\centerline{\LARGE\thaitext{◌ะ; ◌ัวะ; เ◌ะ; เ◌อะ; เ◌าะ; เ◌ียะ; เ◌ือะ; แ◌ะ; โ◌ะ}}


\hfill Typeset with \idxfont{IrisUPC} and the command \cmd{\thai}
\end{scriptexample}
\subsection{Balinese}

The Balinese script, natively known as Aksara Bali and Hanacaraka, is an abugida used in the island of Bali, Indonesia, commonly for writing the Austronesian Balinese language, Old Javanese, and the liturgical language Sanskrit. With some modifications, the script is also used to write the Sasak language, used in the neighboring island of Lombok.[1] The script is a descendant of the Brahmi script, and so has many similarities with the modern scripts of South and Southeast Asia. The Balinese script, along with the Javanese script, is considered the most elaborate and ornate among Brahmic scripts of Southeast Asia.[2]

Though everyday use of the script has largely been supplanted by the Latin alphabet, the Balinese script has significant prevalence in many of the island's traditional ceremonies and is strongly associated with the Hindu religion. The script is mainly used today for copying lontar or palm leaf manuscripts containing religious texts.[2][3]

\newfontfamily\balinese{AksaraBali.ttf}
\newfontfamily\indicative{code2000.ttf}

{\indicative ◌ }

\newcounter{under}
\setcounter{under}{"1B00}

\def\cb#1 {
\hspace*{2.5pt}
 \large
 $\text{◌#1}_{\pgfmathparse{Hex(\theunder)}\pgfmathresult}$
\stepcounter{under}
\vskip5pt\par
}
\begin{scriptexample}[]{Balinese}


\balinese
	 
᭐	᭑	᭒	᭓	᭔	᭕	᭖	᭗	᭘	᭙	᭚	᭛	᭜	᭝	᭞	᭟\\\
 
\def\columnseprulecolor{\color{thegray}}
\columnseprule.4pt
\begin{multicols}{8}

\texttt{U+1B0x}	

\cb{ᬀ }  \cb{ ᬁ } 	\cb{ ᬂ } 	\cb ᬃ	\cb ᬄ 	\cb ᬅ	\cb ᬆ	\cb ᬇ	\cb ᬈ	\cb ᬉ	\cb ᬊ	\cb ᬋ	\cb ᬌ	\cb ᬍ	\cb ᬎ	\cb ᬏ

\columnbreak

\texttt{U+1B1x}	 

\cb ᬐ	 \cb ᬑ 	\cb ᬒ 	\cb ᬓ	\cb ᬔ	\cb ᬕ	\cb ᬖ \cb ᬗ 	\cb ᬘ 	\cb ᬙ 	\cb ᬚ	\cb ᬛ 	\cb ᬜ 	\cb ᬝ 	\cb ᬞ	\cb ᬟ 

\columnbreak

U+1B2x	 

\cb ᬠ◌ 	\cb ᬡ	\cb ᬢ	\cb ᬣ	\cb ᬤ	\cb ᬥ	\cb ᬦ	\cb ᬧ	\cb ᬨ	\cb ᬩ	\cb ᬪ	\cb ᬫ	\cb ᬬ	\cb ᬭ	\cb ᬮ	\cb ᬯ

\columnbreak
U+1B3x 

\cb ᬰ	\cb ᬱ	\cb ᬲ	\cb ᬳ	\cb ᬴	\cb ᬵ	\cb ᬶ	\cb ᬷ	\cb ᬸ	\cb ᬹ	\cb ᬺ	\cb ᬻ	\cb ᬼ	\cb ᬽ	\cb ᬾ	\cb ᬿ


\columnbreak
U+1B4x	 

\cb ᭀ	 \cb ᭁ	\cb ᭂ	\cb ᭃ	\cb ᭄	\cb ᭅ	\cb ᭆ	\cb ᭇ	\cb ᭈ	\cb ᭉ	\cb ᭊ	\cb ᭋ

\columnbreak				
U+1B5x	 

\cb ᭐	\cb ᭑	\cb ᭒	\cb ᭓	\cb ᭔	\cb ᭕	\cb ᭖	\cb ᭗	\cb ᭘	\cb ᭙	\cb ᭚	\cb ᭛	\cb ᭜	\cb ᭝	\cb ᭞	\cb ᭟\\

\columnbreak

U+1B6x 

\cb ᭠	\cb ᭡	\cb ᭢	\cb ᭣	\cb ᭤	\cb ᭥	\cb ᭦	\cb ᭧	\cb ᭨◌ 	\cb ᭩◌ 	\cb ᭪◌ 	\cb ᭫	\cb ᭬	\cb ᭭	\cb ᭮	\cb ᭯

\columnbreak
U+1B7x	 

\cb ᭰	 \cb ᭱  \cb ᭲  \cb ᭳	 \cb ᭴	\cb ᭵	\cb ᭶	\cb ᭷	\cb ᭸	\cb ᭹	\cb ᭺	\cb ᭻	\cb ᭼


\end{multicols}

\end{scriptexample}
\defaulttext

One of the most comprehensive fonts is Aksara Bali\footnote{\url{http://www.alanwood.net/downloads/index.html}}. This is obtainable at Alan Wood's website.
\parindent1em
\section{Lao Alphabet}

\def\laotext#1{{\lao#1}}

The Lao alphabet, Akson Lao (Lao: \laotext{ອັກສອນລາວ} [ʔáksɔ̌ːn láːw]), is the main script used to write the Lao language and other minority languages in Laos. It is ultimately of Indic origin, the alphabet includes 27 consonants (\laotext{ພະຍັນຊະນະ} [pʰāɲánsānā]), 7 consonantal ligatures (\laotext{ພະຍັນຊະນະປະສົມ} [pʰāɲánsānā pá sǒm]), 33 vowels (\laotext{ສະຫລະ} [sálā]) (some based on combinations of symbols), and 4 tone marks (\laotext{ວັນນະຍຸດ} [ván nā ɲūt]). 



According to Article 89 of Amended Constitution of 2003 of the Lao People's Democratic Republic, the Lao alphabet is the official script to the official language, but is also used to transcribe minority languages in the country, but some minority language speakers continue to use their traditional writing systems while the Hmong have adopted the Roman Alphabet.[1] An older version of the script was also used by the ethnic Lao of Thailand's Isan region, who make up a third of Thailand's population, before Isan was incorporated into Siam, until its use was banned and supplemented with the very similar Thai alphabet in 1871, although the region remained distant culturally and politically until further government campaigns and integration into the Thai state (Thaification) were imposed in the 20th century.[2] The letters of the Lao Alphabet are very similar to the Thai alphabet, which has the same roots. They differ in the fact, that in Thai there are still more letters to write one sound and the more circular style of writing in Lao.

Lao, like most indic scripts, is traditionally written from left to right. Traditionally considered an \emph{abugida} script, where certain 'implied' vowels are unwritten, recent spelling reforms make this definition somewhat problematic, as all vowel sounds today are marked with diacritics when written according the Lao PDR's propagated and promoted spelling standard. However most Lao outside of Laos, and many inside Laos, continue to write according to former spelling standards, which continues the use of the implied vowel maintaining the Lao script's status as an \emph{abugida}. Vowels can be written above, below, in front of, or behind consonants, with some vowel combinations written before, over and after. Spaces for separating words and punctuations were traditionally not used, but a space is used and functions in place of a comma or period. The letters have no \emph{majuscule} or \emph{minuscule} (upper and lower case) differentiations

The Unicode block for the Lao script is U+0E80–U+0EFF, added in Unicode version 1.0. The first 10 characters of the row U+0EDx are the Lao numerals 0 through 9. Throughout the chart grey (unassigned) code points are shown, because the assigned Lao characters intentionally match the relative positions of the corresponding Thai characters. This has created the anomaly that the Lao letter \laotext{ສ} is not in alphabetical order, since it occupies the same codepoint as the Thai letter \laotext{ส}.

\begin{scriptexample}[]{}
\unicodetable{lao}{"0E80,"0E90,"0EA0,"0EB0,"0EC0,"0ED0,"0EE0,"0EF0}
\end{scriptexample}

\subsubsection{Numerals}
\bgroup
\lao
\begin{tabular}{rllllllllllll}
Hindu-Arabic numerals	&0	&1	&2	&3	&4	&5	&6	&7	&8	&9	&10 &	20\\
Lao numerals	&໐	&໑	&໒	&໓	&໔	&໕	&໖	&໗	&໘	&໙	&໑໐	&໒໐\\
Lao names	&ສູນ	&ນຶ່ງ	&ສອງ	&ສາມ	&ສີ່	&ຫ້າ 	&ຫົກ	&ເຈັດ	&ແປດ	&ເກົ້າ	&ສິບ	&ຊາວ\\
\end{tabular}
\egroup




\newfontfamily\javanese{Noto Sans Javanese}

%\newfontfamily\javanese{TuladhaJejeg_gr.ttf}

\section{Javanese}
\label{s:javanese}
\index{scripts>Javanese}


The Javanese (Ngoko Javanese: {\javanese ꦮꦺꦴꦁꦗꦮ},[3] Madya Javanese: {\javanese\   ꦠꦶꦪꦁꦗꦮꦶ},[4] Krama Javanese: ꦥꦿꦶꦪꦤ꧀ꦠꦸꦤ꧀ꦗꦮꦶ,[4] Ngoko Gêdrìk: wòng Jåwå, Madya Gêdrìk: tiyang Jawi, Krama Gêdrìk: priyantun Jawi, Indonesian: suku Jawa)[5] are an ethnic group native to the Indonesian island of Java. With approximately 100 million people (as of 2011), they form the largest ethnic group in Indonesia. They are predominantly located in the central to eastern parts of the island. There are also significant numbers of people of Javanese descent in most provinces of Indonesia, Malaysia, Singapore, Suriname, Saudi Arabia and the Netherlands.

The Javanese ethnic group has many sub-groups, such as the Mataram, Cirebonese, Osing, Tenggerese, Samin, Naganese, Banyumasan, etc.[6]

A majority of the Javanese people identify themselves as Muslims, with a minority identifying as Christians and Hindus. However, Javanese civilization has been influenced by more than a millennium of interactions between the native animism Kejawen and the Indian Hindu—Buddhist culture, and this influence is still visible in Javanese history, culture, traditions, and art forms. With a sizeable global population, the Javanese are considered significant as they are the fourth largest ethnic group among Muslims, in the world, after the Arabs,[7] Bengalis[8] and Punjabis.[9]


\paragraph{Javanese} is one of the Austronesian languages, but it is not particularly close to other languages and is difficult to classify. Its closest relatives are the neighbouring languages such as Sundanese, Madurese and Balinese. Most speakers of Javanese also speak Indonesian, the standardized form of Malay spoken in Indonesia, for official and commercial purposes as well as a means to communicate with non-Javanese-speaking Indonesians.

There are speakers of Javanese in Malaysia (concentrated in the states of Selangor and Johor) and Singapore. Some people of Javanese descent in Suriname (the Dutch colony of Suriname until 1975) speak a creole descendant of the language.

\begin{figure}[htbp]
\includegraphics[width=\textwidth]{javanese-people}
\end{figure}

The language is spoken in Yogyakarta, Central and East Java, as well as on the north coast of West Java. It is also spoken elsewhere by the Javanese people in other provinces of Indonesia, which are numerous due to the government-sanctioned transmigration program in the late 20th century, including Lampung, Jambi, and North Sumatra provinces. In Suriname, creolized Javanese is spoken among descendants of plantation migrants brought by the Dutch during the 19th century. In Madura, Bali, Lombok, and the Sunda region of West Java, it is also used as a literary language. It was the court language in Palembang, South Sumatra, until the palace was sacked by the Dutch in the late 18th century.

Javanese is written with the Latin script, Javanese script, and Arabic script.[5] In the present day, the Latin script dominates writings, although the Javanese script is still taught as part of the compulsory Javanese language subject in elementary up to high school levels in Yogyakarta, Central and East Java.

Javanese is the tenth largest language by native speakers and the largest language without official status. It is spoken or understood by approximately 100 million people. At least 45\% of the total population of Indonesia are of Javanese descent or live in an area where Javanese is the dominant language. All seven Indonesian presidents since 1945 have been of Javanese descent.[6] It is therefore not surprising that Javanese has had a deep influence on the development of Indonesian, the national language of Indonesia.

There are three main dialects of the modern language: Central Javanese, Eastern Javanese, and Western Javanese. These three dialects form a dialect continuum from northern Banten in the extreme west of Java to Banyuwangi Regency in the eastern corner of the island. All Javanese dialects are more or less mutually intelligible.


\paragraph{The Javanese script} (Hanacaraka/Carakan) is a script for writing the Javanese language, the native language of one of the peoples of the Island of Java. It is a descendent of the ancient Brahmi script of India, and so has many similarities with modern scripts of South Asia and Southeast Asia. The Javanese script is also used for writing Sanskrit, Old Javanese, and transcriptions of Kawi, as well as the Sundanese language, and the Sasak language.

\begin{figure}[htbp]
\hspace*{-1.5cm}\includegraphics[width=1.2\textwidth]{java-palm-leave-manuscript}
\end{figure}





\begin{scriptexample}[]{Javanese}
\bgroup
\javanese

꧋ꦱꦧꦼꦤ꧀ꦮꦺꦴꦁꦏꦭꦲꦶꦂꦲꦏꦺꦏꦤ꧀ꦛꦶꦩꦂꦢꦶꦏꦭꦤ꧀ꦢꦂꦧꦺꦩꦂꦠꦧꦠ꧀ꦭꦤ꧀ꦲꦏ꧀ꦲꦏ꧀ꦏꦁꦥꦝ꧉

꧋ ꦲꦮꦶꦠ꧀ꦲꦶꦏꦁꦄꦱ꧀ꦩꦄꦭ꧀ꦭꦃ꧈ ꦏꦁꦩꦲꦩꦸꦫꦃꦠꦸꦂ ꦩꦲꦲꦱꦶꦃ꧉ 	 
 ۝꧋ ꦄꦭꦶꦥꦃ꧀ ꦭ ꦩ꧀ ꦫ ꧌ ꦏꦁ — — ꦥꦿꦶꦏ꧀ꦱ ꦏꦉꦪꦥ꧀ꦥꦩꦸꦁꦄꦭ꧀ꦭꦃꦥꦶꦪꦺꦩ꧀ꦧꦏ꧀ ꧌꧉ ꦩꦁꦪꦏꦴꦪꦤꦴ ꦲꦶꦏꦸꦄꦪꦺꦪꦠꦴꦏꦶꦠꦧ꧀ꦑꦸꦂꦄꦤ꧀ꦏꦁꦥꦿꦪꦠꦭ꧉ 	 
᭐	᭑	᭒	᭓	᭔	᭕	᭖	᭗	᭘	᭙	᭚	᭛	᭜	᭝	᭞	᭟
 
\egroup
\end{scriptexample}


The Javanese script was added to Unicode Standard in version 5.2 on the code points \texttt{A980 - A9DF}. There are 91 code points for Javanese script: 53 letters, 19 punctuation marks, 10 numbers, and 9 vowels:
\medskip

\unicodetable{javanese}{"A980,"A990,"A9A0, "A9B0, "A9C0,"A9D0}

\medskip



As of the writing of this document (2017), there are several widely published fonts able to support Javanese, ANSI-based Hanacaraka/Pallawa by Teguh Budi Sayoga,[21] Adjisaka by Sudarto HS/Ki Demang Sokowanten,[22] JG Aksara Jawa by Jason Glavy,[23] Carakan Anyar by Pavkar Dukunov,[24] and Tuladha Jejeg by R.S. Wihananto,[25] which is based on Graphite (SIL) smart font technology. Other fonts with limited publishing includes Surakarta made by Matthew Arciniega in 1992 for Mac's screen font,[26] and Tjarakan developed by AGFA Monotype around 2000.[27] There is also a symbol-based font called Aturra developed by Aditya Bayu in 2012–2013.[28]

Due to the script's complexity, many Javanese fonts have different input method compared to other Indic scripts and may exhibit several flaws. \docFont{JG Aksara Jawa}, in particular, may cause conflicts with other writing system, as the font use code points from other writing systems to complement Javanese's extensive repertoire. This is to be expected, as the font was made before Javanese implementation in Unicode.[29]

Arguably, the most "complete" font, in terms of technicality and glyph count, is \docFont{TuladhaJejeg}. It comes with keyboard facilities, displaying complex syllable structure, and support extensive glyph repertoire including non-standard forms which may not be found in regular Javanese texts, by utilizing Graphite (SIL) smart font technology. |Tuladha Jejeg| uses variable stroke widths on its glyphs with serifs on some glyphs\footnote{\protect\url{https://sites.google.com/site/jawaunicode/main-page}}.

However, as not many writing systems require such complex feature, use is limited to programs with Graphite technology, such as Firefox browser, Thunderbird email client, and several OpenType word processor and of course XeLaTeX. The font was chosen for displaying Javanese script in the Javanese Wikipedia.[16]

\paragraph{jawaTeX} Jawa\TeX{} project is initial effort to make Javanese characters typesetting program using \TeX{}/\LaTeX{}. This project is aimed to make Javanese widely used. The main project is developing transliteration models to transliterate Latin document into Javanese document. Perl and \TeX{}/\LaTeX{} are use in this project, the program are develop to run in text mode (console) both Linux and Windows but not limit on it. Web based program also developed, and automatic embedded Javanese characters in HTML See \href{http://jawatex.org/jawa/jawatex}{jawatex}.


\section{Khmer}
\newfontfamily\normaltext{Arial Unicode MS}
\normaltext

\def\khmerdefaultfont#1{\newfontfamily\khmer[Scale=MatchUppercase]{#1}}
\def\khmertext#1{{\khmer#1}}

\cxset{khmer font/.code=\khmerdefaultfont{#1}}

\cxset{khmer font/.default=Khmer}

\cxset{language=khmer, 
       khmer font = Khmer UI}

\begin{key}{/chapter/khmer font=\meta{font name} (Khmer  UI)} Loads the font
command \cmd{\khmer}. When the command is used it typesets text in
khmer unicode. There is no need to load the language, unless it is the main document language. For windows the default font is \texttt{DaunPenh} this font is in general too small to read; a better font to use is Khmer UI.
\end{key}

\begin{key}{/tikz/turtle/right=\meta{angle} (default 90)}
  Turns the turtle right by the given angle. 
\end{key}


The Khmer script (Khmer: {\Large\khmertext{អក្សរខ្មែរ}}; IPA: [ʔaʔsɑː kʰmaːe]) [2] is an \textit{abugida} (alphasyllabary) script used to write the Khmer language (the official language of Cambodia). It is also used to write Pali among the Buddhist liturgy of Cambodia and Thailand.

It was adapted from the Pallava script, a variant of Grantha alphabet descended from the Brahmi script of India, which was used in southern India and South East Asia during the 5th and 6th Centuries AD.[3] The oldest dated inscription in Khmer was found at Angkor Borei District in Takéo Province south of Phnom Penh and dates from 611.[4] The modern Khmer script differs somewhat from precedent forms seen on the inscriptions of the ruins of Angkor.

Not all Khmer consonants can appear in syllable-final position. The most common syllable-final consonants include {\khmer កងញតនបមល}. The pronunciation of the consonant in final position may differ from it's normal pronunciation.


\begin{tabular}{llp{9cm}}
\khmertext{ំ}	&nĭkkôhĕt (\khmertext{និគ្គហិត})	&niggahita; nasalizes the inherent vowels and some of the dependent vowels, see anusvara, sometimes used to represent [aɲ] in Sanskrit loanwords\\
\khmertext{ះ}	&reăhmŭkh (\khmertext{រះមុខ})	&"shining face"; adds final aspiration to dependent or inherent vowels, usually omitted, corresponds to the visarga diacritic, it maybe included as dependent vowel symbol\\
\khmertext{ៈ}	&yŭkôleăkpĭntŭ (\khmertext{យុគលពិន្ទុ})	&yugalabindu ("pair of dots"); adds final glottalness to dependent or inherent vowels, usually omitted\\
\khmertext{៉}	 &musĕkâtônd (\khmertext{មូសិកទន្ត})	&mūsikadanta ("mouse teeth"); used to convert some o-series consonants (\khmertext{ង ញ ម យ រ វ}) to the a-series\\
\khmertext{៊}	&treisâpt (\khmertext{ត្រីសព្ទ})	trīsabda; used to convert some a-series consonants (\khmertext{ស ហ ប អ}) to the o-series\\
\end{tabular}




ុ	kbiĕh kraôm (ក្បៀសក្រោម)	also known as bŏkcheung (បុកជើង); used in place of the diacritics treisâpt and musĕkâtônd when they would be impeded by superscript vowels
់	bântăk (បន្តក់)	used to shorten some vowels; the diacritic is placed on the last consonant of the syllable
៌	rôbat (របាទ)
répheăk (រេផៈ)	rapāda, repha; behave similarly to the tôndâkhéat, corresponds to the Devanagari diacritic repha, however it lost its original function which was to represent a vocalic r
 ៍	tôndâkhéat (ទណ្ឌឃាដ)	daṇḍaghāta; used to render some letters as unpronounced
៎	kakâbat (កាកបាទ)	kākapāda ("crow's foot"); more a punctuation mark than a diacritic; used in writing to indicate the rising intonation of an exclamation or interjection; often placed on particles such as /na/, /nɑː/, /nɛː/, /vəːj/, and the feminine response /cah/
៏	âsda (អស្តា)	denotes stressed intonation in some single-consonant words[5]
័	sanhyoŭk sannha (សំយោគសញ្ញា)	represents a short inherent vowel in Sanskrit and Pali words; usually omitted
៑	vĭréam (វិរាម)	a mostly obsolete diacritic, corresponds to the virāma
្	cheung (ជើង)	a.w. coeng; a sign developed for Unicode to input subscript consonants, appearance of this sign varies among fonts
\section{Sundanese}
\newfontfamily\sundanese{SundaneseUnicode-1.0.5.ttf}
^^A\newfontfamily\sundanese{Arial Unicode MS}
\def\ublock#1{\texttt{{\arial #1}}}

The Sundanese script (Aksara Sunda, {\sundanese ᮃᮊ᮪ᮞᮛ ᮞᮥᮔ᮪ᮓ}) is a writing system which is used by the Sundanese people. It is built based on Old Sundanese script (Aksara Sunda Kuno) which was used by the ancient Sundanese between the 14th and 18th centuries.

\begin{scriptexample}[]{Sundanese}
\unicodetable{sundanese}{"1B80,"1B90,"1BA0,"1BB0}

\sundanese
\obeylines
\bgroup
᮱ {\arial= 1}	᮲ {\arial= 2}	᮳{\arial = 3}
᮴ {\arial= 4}	᮵ {\arial = 5} 	᮶ {\arial= 6}
᮷ {\arial= 7}	᮸ {\arial= 8}	᮹ {\arial= 9}
᮰ {\arial= 0}

\egroup
\end{scriptexample}

\begin{scriptexample}[]{Sundanese}
\bgroup
\sundanese
\centering

◌ᮃᮄᮅᮆᮇᮈᮉᮊᮋᮌᮍᮎᮏᮐᮕᮔᮓᮑᮖᮗᮚᮛᮜᮝᮞᮟᮠᮠ


\egroup
\end{scriptexample}

\bgroup
\def\1{\sundanese ᮱}
\TextOrMath\1\1

$\1$
\egroup

In text In texts, numbers are written surrounded with dual pipe sign \textbar \ldots \textbar. Example: {\textbar \sundanese ᮲᮰᮱᮰\textbar} = 2010













^^A\subsection{Oriya alphabet}
\newfontfamily\oriya[Scale=1.1,Script=Oriya]{code2000.ttf}

\def\oriyatext#1{{\oriya#1}}
The Oriya script or Utkala Lipi (Oriya: \oriyatext{ଉତ୍କଳ ଲିପି}) or Utkalakshara (Oriya: \oriyatext{ଉତ୍କଳାକ୍ଷର}) is used to write the Oriya language, and can be used for several other Indian languages, for example, Sanskrit.

\centerline{\Huge\oriyatext{ଉତ୍କଳ ଲିପି}}

\bgroup
\oriya
୦୧୨୩୪୫୬୭୮୯
ଅ ଆ ଇ ଈ ଉ ଊ ଋ ୠ ଌ ୡ ଏ ଐ ଓ ଔ କ ଖ ଗ ଘ ଙ ଚ ଛ ଜ ଝ ଞ ଟ ଠ ଡ ଢ ଣ ତ ଥ ଦ ଧ ନ ପ ଫ ବ ଵ ଭ ମ ଯ ର ଳ ୱ ଶ ଷ ସ ହ ୟ ଲ
\egroup

\begin{quotation}
Oṛiyā is encumbered with the drawback of an excessively awkward and cumbrous written character. ... At first glance, an Oṛiyā book seems to be all curves, and it takes a second look to notice that there is something inside each.(G. A. Grierson, Linguistic Survey of India, 1903)
\end{quotation}

Comparison of Oṛiyā script with its neighbours[edit]
At a first look the great number of signs with round shapes suggests a closer relation to the southern neighbour Telugu than to the other neighbours Bengali in the north and Devanāgarī in the west. The reason for the round shapes in Oriya and Telugu (and also in Kannaḍa and Malayāḷam) is the former method of writing using a stylus to scratch the signs into a palm leaf. These tools do not allow for horizontal strokes because that would damage the leaf.

Oriya letters are mostly round shaped whereas in Devanāgarī and Bengali have horizontal lines. So in most cases the reader of Oṛiyā will find the distinctive parts of a letter only below the hoop. Considering this the  closer relation to Devanāgarī and Bengali exists than to any southern script, though both northern and southern scripts have the same origin, Brāhmī.

Oriya (\oriyatext{ଓଡ଼ିଆ} oṛiā), officially spelled Odia,[3][4] is an Indian language belonging to the Indo-Aryan branch of the Indo-European language family. It is the predominant language of the Indian states of Odisha, where native speakers comprise 80\% of the population,[5] and it is spoken in parts of West Bengal, Jharkhand, Chhattisgarh and Andhra Pradesh. Oriya is one of the many official languages in India; it is the official language of Odisha and the second official language of Jharkhand. [6][7][8] Oriya is the sixth Indian language to be designated a Classical Language in India, on the basis of having a long literary history and not having borrowed extensively from other languages.

^^A
^^A\subsection{Mongolian Script}

\newfontfamily\mongolian[Language=Mongolian, Scale=1.3]{code2000.ttf}

The classical Mongolian script (in Mongolian script: {\mongolian  ᠮᠣᠩᠭᠣᠯ ᠪᠢᠴᠢᠭ᠌} Mongγol bičig; in Mongolian Cyrillic: Монгол бичиг Mongol bichig), also known as Uyghurjin Mongol bichig, was the first writing system created specifically for the Mongolian language, and was the most successful until the introduction of Cyrillic in 1946. Derived from Uighur, Mongolian is a true alphabet, with separate letters for consonants and vowels. The Mongolian script has been adapted to write languages such as Oirat and Manchu. Alphabets based on this classical vertical script are used in Inner Mongolia and other parts of China to this day to write Mongolian, Sibe and, experimentally, Evenki.
\medskip

\bgroup\par
\noindent
\colorbox{graphicbackground}{\color{black}^^A
\begin{minipage}{\textwidth}^^A
\parindent1pt
\vskip10pt
\leftskip10pt \rightskip\leftskip
\mongolian
\large
ᠬᠦᠮᠦᠨ ᠪᠦᠷ ᠲᠥᠷᠥᠵᠦ ᠮᠡᠨᠳᠡᠯᠡᠬᠦ ᠡᠷᠬᠡ ᠴᠢᠯᠥᠭᠡ ᠲᠡᠢ᠂ ᠠᠳᠠᠯᠢᠬᠠᠨ ᠨᠡᠷ᠎ᠡ ᠲᠥᠷᠥ ᠲᠡᠢ᠂ ᠢᠵᠢᠯ ᠡᠷᠬᠡ ᠲᠡᠢ ᠪᠠᠢᠠᠭ᠃ ᠣᠶᠤᠨ ᠤᠬᠠᠭᠠᠨ᠂ ᠨᠠᠨᠳᠢᠨ ᠴᠢᠨᠠᠷ ᠵᠠᠶᠠᠭᠠᠰᠠᠨ ᠬᠦᠮᠦᠨ ᠬᠡᠭᠴᠢ ᠥᠭᠡᠷ᠎ᠡ ᠬᠣᠭᠣᠷᠣᠨᠳᠣ᠎ᠨ ᠠᠬᠠᠨ ᠳᠡᠭᠦᠦ ᠢᠨ ᠦᠵᠢᠯ ᠰᠠᠨᠠᠭᠠ ᠥᠠᠷ ᠬᠠᠷᠢᠴᠠᠬᠥ ᠤᠴᠢᠷ ᠲᠠᠢ᠃
\par
\vspace*{10pt}
\end{minipage}
}
\medskip
^^A
^^A\subsection{Tibetan}

^^A\newfontfamily\tibetan{TibMachUni.ttf}

^^A\newfontfamily\tibetan{Qomolangma-Chuyig.ttf}

^^A should pick it up automatically \tibetan

Fonts described in this section can be obtained from The Tibetan \& Himalayan Library
\footnote{\url{http://www.thlib.org/tools/scripts/wiki/tibetan%20machine%20uni.html}  }

I have tried a few \texttt{Tibetan Machine Uni (TMU)} seems to be used by a number of scholars. 

A tip when you are trying to locate fonts is to find a related article in Wikipedia, such as Tibetan alphabet and inspect the element using your browser to see what fonts are being used.


|style="font-family:'Jomolhari','Tibetan Machine Uni','DDC Uchen', 'Kailash';| 


If you cannot see the script and rather than boxes or question marks then you can search and download one of the fonts in |font-family|.

\def\tibetandefaultfont#1{\newfontfamily\tibetan[Language=Tibetan]{#1}}


\cxset{language=tibetan} 
\cxset{tibetan font/.code=\tibetandefaultfont{#1}}


^^A\cxset{tibetan font = TibMachUni.ttf}




\begin{key}{/chapter/language = tibetan} The key |language=tibetan| sets the default language as Tibetan, using the main font given by the key |tibetan font=TibMachUni.ttf|.
\end{key}

\begin{key}{/chapter/tibetan font = TibMachUni.ttf} The key |tibetan font=font-name| sets the default font for the Tibetan language. It will also create the switch \cmd{\tibetan} for typesetting text in Tibetan.
\end{key}

\begin{texexample}{Tibetan language setttings}{ex:tibetan}
\cxset{language=tibetan, tibetan font = TibMachUni.ttf}
\tibetan

\tibetan Tibetan: དབུ་ཅན
\end{texexample}


The Tibetan alphabet is an \emph{abugida} of Indic origin used to write the Tibetan language as well as Dzongkha, the Sikkimese language, Ladakhi, and sometimes Balti. 

The printed form of the alphabet is called \textit{uchen} script (Tibetan: དབུ་ཅན་, Wylie: dbu-can; "with a head") while the hand-written cursive form used in everyday writing is called umê script (Tibetan: དབུ་མེད་, Wylie: dbu-med; "headless").
\uccoff
The alphabet is very closely linked to a broad ethnic Tibetan identity. Besides Tibet, it has also been used for Tibetan languages in Bhutan, India, Nepal, and Pakistan.[1] The Tibetan alphabet is ancestral to the Limbu alphabet, the Lepcha alphabet,[2] and the multilingual 'Phags-pa script.[2]
\uccon

The Tibetan alphabet is romanized in a variety of ways.[3] This article employs the Wylie transliteration system.

The Tibetan alphabet has thirty basic letters, sometimes known as "radicals", for consonants.[2]

ཀ ka /ká/	ཁ kha /kʰá/	ག ga /kà, kʰà/	ང nga /ŋà/
ཅ ca /tʃá/	ཆ cha /tʃʰá/	ཇ ja /tʃà/	ཉ nya /ɲà/
ཏ ta /tá/	ཐ tha /tʰá/	ད da /tà, tʰà/	ན na /nà/
པ pa /pá/	ཕ pha /pʰá/	བ ba /pà, pʰà/	མ ma /mà/
ཙ tsa /tsá/	ཚ tsha /tsʰá/	ཛ dza /tsà/	ཝ wa /wà/ (not originally part of the alphabet)[5]
ཞ zha /ʃà/[6]	ཟ za /sà/	འ 'a /hà/[7]
ཡ ya /jà/	ར ra /rà/	ལ la /là/
ཤ sha /ʃá/[6]	ས sa /sá/	ཧ ha /há/[8]
ཨ a /á/

\subsubsection{Unicode Block Tibetan}


\bgroup\large
\begin{tabular}{llllllllllllllll l}
\toprule
	           &|0|	&|1|	&|2|	&|3|	&|4|	&|5|	&|6|	&|7|	&|8|	&|9|	&|A|	&|B|	&|C|	&|D|	&|E|	&|F|\\
\midrule
\texttt{U+0F0x}	&ༀ	&༁	&༂	&༃	&༄	&༅	&༆	&༇	&༈	&༉	&༊	&་	&༌  &	།	&༎	&༏\\
\midrule
\texttt{U+0F1x} &༐	&༑	&༒	&༓	&༔	&༕	&༖	&༗	&༘&	༙	&༚	&༛	&༜	&༝	&༞	&༟\\
\midrule
\texttt{U+0F2x} &༠	&༡	&༢	&༣	&༤	&༥	&༦	&༧	&༨	&༩	&༪	&༫	&༬	&༭	&༮	&༯\\
\midrule
\texttt{U+0F3x}	&༰ &༱	 &༲ &༳	&༴ &༵	&༶ & ༷	&༸&	༹	&༺&	༻	&༼&	༽	&༾	&༿\\
\midrule
\texttt{U+0F4x} &ཀ	&ཁ	&ག	&གྷ	&ང	&ཅ	&ཆ	&ཇ	&	&ཉ	&ཊ	&ཋ	&ཌ	&ཌྷ	&ཎ	&ཏ\\
\midrule
\texttt{U+0F5x}	 &ཐ	&ད	&དྷ	&ན	&པ	&ཕ	&བ	&བྷ	&མ	&ཙ	&ཚ	&ཛ	&ཛྷ	&ཝ	&ཞ	&ཟ\\
\midrule
\texttt{U+0F6x} &འ	&ཡ	&ར	&ལ	&ཤ	&ཥ	&ས	&ཧ	&ཨ	&ཀྵ	&ཪ	&ཫ	&ཬ	&&&\\
^^A\texttt{U+0F7x}&&	ཱ &	& &ི	ཱི&	ུ&	ཱུ&	ྲྀ&	ཷ&	ླྀ&	ཹ&	ེ&	ཻ&	ོ&	ཽ&	&ཾ	&ཿ\\
\midrule
\texttt{U+0F8x}&    ྀ   & 	ཱྀ&	ྂ&	&ྃ &	྄	&྅&	྆	&྇	ྈ&	ྉ&	ྊ&	ྋ&	ྌ&	ྍ&	ྎ&	ྏ\\
\midrule
\texttt{U+0F9x} &	ྐ&	ྑ   & 	ྒ &	ྒྷ &	ྔ &	ྕ &	ྖ &	ྗ &		ྙ &	ྚ &	ྛ &	ྜ &	ྜྷ &	ྞ &	ྟ\\
\texttt{U+0FAx} &	ྠ &	ྡ &	ྡྷ &	ྣ &	ྤ &	ྥ &		&ྦ	&ྦྷ	ྨ&	ྩ&	ྪ&	ྫ&	ྫྷ&	ྭ&	ྮ&	ྯ\\
\midrule
\texttt{U+0FBx} 
&	  ྰ 
&	
& ྱ  	 
&ྲ	
&ླ	
&ྴ
&	ྵ
&	ྶ
&	ྷ
&ྸ
&
&
&
&	
&྾	
&྿\\
\midrule
\texttt{U+0FCx}	 &࿀&	࿁&	࿂&	࿃&	࿄&	࿅&	&࿇	&࿈	&࿉	&࿊	&࿋	&࿌	&&	࿎	&࿏\\
\midrule
\texttt{U+0FDx}	&࿐	&࿑	&࿒	&࿓	&࿔	&࿕	&࿖	&࿗	&࿘	&࿙	&࿚	&&&&&\\
\midrule
\texttt{U+0FEx} &&&&&&&&&&&&&&&&\\
\midrule
\texttt{U+0FFx}  &&&&&&&&&&&&&&&&\\
\bottomrule
\end{tabular}
\egroup




\subsubsection{Fonts for Tibetan}

Fonts for Tibetan need to be downloaded one set of fonts are the \texttt{Qomolangma}. They come in different flavours, but they appear
to offer advantages as compared to the Tibetan Machine Uni.
\medskip


\newfontfamily\betsu{Qomolangma-Betsu.ttf}
\newfontfamily\drutsa{Qomolangma-Drutsa.ttf}
\newfontfamily\chuyig{Qomolangma-Chuyig.ttf}
\newfontfamily\tsumachu{Qomolangma-Tsumachu.ttf}
\newfontfamily\uchensutung{Qomolangma-UchenSutung.ttf}
\newfontfamily\uchensuring{Qomolangma-UchenSuring.ttf}
\newfontfamily\uchensarchen{Qomolangma-UchenSarchen.ttf}
\newfontfamily\uchensarchung{Qomolangma-UchenSarchung.ttf}
\newfontfamily\tsuring{Qomolangma-Tsuring.ttf}
\newfontfamily\TMU{TibMachUni.ttf}
\newfontfamily\himalaya{Microsoft Himalaya}
\uccoff

{
\centering

\renewcommand{\arraystretch}{1.5}

\begin{tabular}{lr}
\toprule
|Qomolangma-Betsu.ttf| & {\betsu  དབུ་མེད }\\
\midrule
|Qomolangma-Chuyig.ttf| &{\chuyig  དབུ་མེད}\\
\midrule
|Qomolangma-Drutsa.ttf| &{\drutsa  དབུ་མེད}\\
\midrule
|Qomolangma-Tsumachu.ttf|&{\tsumachu  དབུ་མེད}\\
\midrule
|Qomolangma-Tsuring.ttf| &{\tsuring  དབུ་མེད}\\
\midrule
|Qomolangma-UchenSarchen.ttf| &{\uchensarchen དབུ་མེད}\\
\midrule
|Qomolangma-UchenSarchung.ttf|&{\uchensarchung དབུ་མེད }\\
\midrule
|Qomolangma-UchenSuring.ttf|&{\uchensuring དབུ་མེད}\\
\midrule
|Qomolangma-UchenSutung.ttf|&{\uchensutung དབུ་མེད }\\
\midrule
|TibMachUni.ttf| &{\TMU དབུ་མེད }\\
\midrule
|Microsoft Himalaya| &{\himalaya དབུ་མེད ཽ}\\
\bottomrule
\end{tabular}

}
\bigskip

\bgroup
\LARGE\tsuring
\noindent༆ །ཨ་ཡིག་དཀར་མཛེས་ལས་འཁྲུངས་ཤེས་བློ  འི་\par
གཏེར༑ །ཕས་རྒོལ་ཝ་སྐྱེས་ཟིལ་གནོན་གདོང་ལྔ་བཞིན།།\par
ཆགས་ཐོགས་ཀུན་བྲལ་མཚུངས་མེད་འཇམ་དབྱངསམཐུས།།\par
མཧཱ་མཁས་པའི་གཙོ་བོ་ཉིད་འགྱུར་ཅིག། །མངྒལཾ༎\par
\egroup

\subsubsection{Tibetan numbers}
\cxset{language=tibetan, tibetan font = TibMachUni.ttf}

{
\obeylines
\small
TIBETAN DIGIT ZERO	༠
TIBETAN DIGIT ONE	༡	
TIBETAN DIGIT TWO	༢	
TIBETAN DIGIT THREE	༣	
TIBETAN DIGIT FOUR	༤	
TIBETAN DIGIT FIVE	༥	
TIBETAN DIGIT SIX	༦	
TIBETAN DIGIT SEVEN	༧	
TIBETAN DIGIT EIGHT	༨	
TIBETAN DIGIT NINE	༩	
TIBETAN DIGIT HALF ONE	\tibetan༪	
TIBETAN DIGIT HALF TWO	༫	
TIBETAN DIGIT HALF THREE	༬
TIBETAN DIGIT HALF FOUR ༭	
TIBETAN DIGIT HALF FIVE ༯	
TIBETAN DIGIT HALF SIX	 ༯	
TIBETAN DIGIT HALF SEVEN	༰	
TIBETAN DIGIT HALF EIGHT	༱	
TIBETAN DIGIT HALF NINE	༲	
TIBETAN DIGIT HALF ZERO	༳	
}


Tibetan numbers

The usage is not certain. By some interpretations, this has the value of 9.5. Used only in some traditional contexts, these appear as the last digit of a multidigit number, eg. ༤༬ represents 42.5. These are very rarely used, however, and other uses have been postulated.

\defaulttext

^^A
^^A
^^A

^^A\section{Tamil}
\newfontfamily\tamil[Scale=1.1,Script=Tamil]{code2000.ttf}

\def\tamiltext#1{{\tamil#1}}

The Tamil script (\tamiltext{தமிழ் அரிச்சுவடி} tamiḻ ariccuvaṭi) is an abugida script that is used by the Tamil people in India, Sri Lanka, Malaysia and elsewhere, to write the Tamil language, as well as to write the liturgical language Sanskrit, using consonants and diacritics not represented in the Tamil alphabet.[1] Certain minority languages such as Saurashtra, Badaga, Irula, and Paniya are also written in the Tamil script

The Tamil script has 12 vowels (\tamiltext{உயிரெழுத்து} uyireḻuttu "soul-letters"), 18 consonants (\tamiltext{மெய்யெழுத்து} meyyeḻuttu "body-letters") and one character, the āytam \tamiltext{ஃ (ஆய்தம்)}, which is classified in Tamil grammar as being neither a consonant nor a vowel (\tamiltext{அலியெழுத்து} aliyeḻuttu "the hermaphrodite letter"), though often considered as part of the vowel set (\tamiltext{உயிரெழுத்துக்கள்} uyireḻuttukkaḷ "vowel class"). The script, however, is syllabic and not alphabetic.[3] The complete script, therefore, consists of the thirty-one letters in their independent form, and an additional 216 combinant letters representing a total 247 combinations (\tamiltext{உயிர்மெய்யெழுத்து} uyirmeyyeḻuttu) of a consonant and a vowel, a mute consonant, or a vowel alone. These combinant letters are formed by adding a vowel marker to the consonant. Some vowels require the basic shape of the consonant to be altered in a way that is specific to that vowel. Others are written by adding a vowel-specific suffix to the consonant, yet others a prefix, and finally some vowels require adding both a prefix and a suffix to the consonant. In every case the vowel marker is different from the standalone character for the vowel.
The Tamil script is written from left to right.

Tamil is a Unicode block containing characters for the Tamil, Badaga, and Saurashtra languages of Tamil Nadu India, Sri Lanka, Singapore, and Malaysia. In its original incarnation, the code points U+0B02..U+0BCD were a direct copy of the Tamil characters A2-ED from the 1988 ISCII standard. The Devanagari, Bengali, Gurmukhi, Gujarati, Oriya, Telugu, Kannada, and Malayalam blocks were similarly all based on their ISCII encodings.

\begin{scriptexample}[]{Tamil}
\unicodetable{tamil}{"0B80,"0B90,"0BA0,"0BB0,"0BC0,"0BE0,"0BF0}

\hfill  Typeset with \cmd{\tamil} and \texttt{code2000.ttf}
\end{scriptexample}

\subsection{Tamil Numbers and Numerals}

Originally, Tamils did not use zero, nor did they use positional digits (having separate 
symbols for the numbers 10, 100 and 1000). Symbols for the numbers are similar to 
other Tamil letters, with some minor changes. 

For example, the number 3782 is not written as \tamiltext{௩௭௮௨} as in modern usage. Instead it 
is written as \tamiltext{௩ ௲ ௭ ௱ ௮ ௰ ௨}. This would be read as they are written as 
Three Thousands, Seven Hundreds, Eight Tens, Two; or in Tamil as 
\tamiltext{௩௲௭௱௮௰௨ž}.\footnote{https://cloud.github.com/downloads/raaman/Tamil-Numeral/tamilnumbers.html}

\subsection{Dates}

Once the script is loaded the day, month and year can be loaded using the command  \cmd{\tamildate}, which returns the |\today| formatted as per custom Tamil. 

\begin{center}
\bgroup
\tamil
\begin{tabular}{lll}
day	 &month	&year	\\

௳	&௴	      &௵	\\

u	&mee	      &wa	\\
\egroup
\end{center}











^^A\chapter{Armenian}

\label{s:armenian}\index{Armenian}\index{scripts>Armenian}

As we present the scripts in alphabetic order, the first script we will typeset is in Armenian. There are many fonts available for the language. We use two in the example, the first one is \textit{FreeSans} and the second is \textit{Sylphaen} which is found on Windows Operating systems. The language is not supported by the \pkg{Babel} and partially supported by the \pkgname{Polyglossia}. \tcbdocmarginnote{china revision}

\def\ucfirst#1#2;{\MakeUppercase#1#2}


\def\armeniantest#1#2{
  {\parindent0pt
  \topline \vskip3pt
  \noindent\mbox{
     \ucfirst#1;\hfill\hbox{[\texttt{U+0530-U+058F}]}
  }}
 \nobreak

\begin{minipage}{0.45\textwidth}
\bgroup
%\setotherlanguage{#1}
\begin{#1}
#2
[\today]
\end{#1}
\egroup
\end{minipage}\hspace*{1em}
\begin{minipage}{0.45\textwidth}
\bgroup
  \parindent0pt
  \ttfamily\raggedright
  \string\documentclass\{article\}\par
  \string\usepackage[no-math]\{fontspec\}\\
  \string\newfontfamily\textbackslash#1font[Script=\ucfirst #1;,\\   ~~~~~~~Scale=MatchLowercase]
\{FreeSans\}\par
  \string\begin\{document\}\\
  \string\setotherlanguage\{#1\}\\
  \string\begin\{#1\}\\
  \egroup
\begin{#1}
\hskip10pt\vbox{#2}
\end{#1}
\bgroup
  \ttfamily[\detokenize{\today}]\\
  \string\end\{#1\}\\
  \string\end\{document\}
\egroup
\end{minipage}


\textit{FreeSans}: \url{ http://www.gnu.org/software/freefont/}
}

\armeniantest{armenian}{Բոլոր մարդիկ ծնվում են ազատ ու հավասար իրենց
արժանապատվությամբ ու իրավունքներով։       
Նրանք ունեն բանականություն ու խիղճ և միմյանց
պետք է եղբայրաբար վերաբերվեն։}

The Armenian script was invented around 407 AD, by Mesrop Maštoc, a cleric who wanted to 
translate Greek and Syriac scriptures and liturgical texts into Armenian. The system he devised 
is a pure alphabet, closely modelled on the traditional order of Greek phonetic values, with 
additional graphemes to represent Armenian sounds not found in Greek. The orthography is, 
phonetically, a near perfect representation of the Armenian language, and has remained almost 
entirely unchanged since its invention. In recent times, the letterforms in many Armenian 
typefaces have consciously modelled Latin types in their treatment of serifs, stroke weight and 
stress, and other details. This is the approach that Geraldine adopted for the Sylfaen Armenian, 
in order to harmonise the different scripts within the font. 

This kind of harmonisation has to be 
very carefully handled, as there is, of course, a point at which one can corrupt the normative 
letterforms and produce something which will be unacceptable to native readers. Once again, 
we sought expert review of the design, this time from Manvel Shmavonyan, an Armenian type designer, and his Russian colleague Vladimir Yefimov at 
ParaType in Moscow.

\bgroup
\medskip
\fontspec[Script=Armenian,Scale=1.7]{Sylfaen}
\centering

Աա Բբ Գգ Դդ Եե Զզ Էէ Ըը Թթ Ժժ Իի \\
Լլ Խխ Ծծ Կկ Հհ Ձձ Ղղ Ճճ Մմ Յյ Նն \\
Շշ Ոո Չչ Պպ Ջջ Ռռ Սս Վվ Տտ Րր Ցց \\
Ււ Փփ Քք Օօ Ֆֆ / և ﬓ ﬔ ﬕ ﬖ ﬗ\\
\egroup
\captionof{table}{Armenian, showing the basic alphabet (typeset using the \textit{Sylfaen} font.}
\medskip

\bgroup
\def\m#1 #2 #3\\{\makebox[2em]{#1}\makebox[2em]{{\fontspec{code2000.ttf}#2}}\makebox[2em]{\hfill#3 \\ }}
\fontspec[Script=Armenian,Scale=1.1]{Sylfaen}

\begin{multicols}{4}
\m Ա	A	1\\
\m Բ	B	2\\
\m Գ	G	3\\
\m Դ	D	4\\
\m Ե	E	5\\
\m Զ	Z	6\\
\m Է	ē	7\\
\m Ը	ə	8\\
\m Թ	tʿ	9\\
\m Ժ	ž	10\\
\m Ի	I	20\\
\m Լ	L	30\\
\m Խ	X	40\\
\m Ծ	C	50\\
\m Կ	K	60\\
\m Հ	H	70\\
\m Ձ	J	80\\
\m Ղ	ł	90\\
\m Ճ	č	100\\
\m Մ	M	200\\
\m Յ	Y	300\\
\m Ն	N	400\\
\m Շ	š	500\\
\m Ո	O	600\\
\m Չ	čʿ	700\\
\m Պ	P	800\\
\m Ջ	ǰ	900\\
\m Ռ	ṙ	1000\\ 
\m Ս	S	2000\\
\m Վ	V	3000\\
\m Տ	T	4000\\
\m Ր	R	5000\\
\m Ց	cʿ	6000\\
\m Ւ	W	7000\\
\m Փ	pʿ	8000\\
\m Ք	kʿ	9000\\

\end{multicols}
\captionof{table}{Armenian Numerals \textit{(from Wikipedia).}
The first column is the classical Armenian numeral, the second the transliteration and the third the arabic numeral it represents.}

\medskip

Numbers in the Armenian numeral system are obtained by simple addition. Armenian numerals are written left-to-right (as in the Armenian language). Although the order of the numerals is irrelevant since only addition is performed, the convention is to write them in decreasing order of value.

\begin{align*}
\text{ՌՋՀԵ} &= 1975 = 1000 + 900 + 70 + 5\\
\text{ՍՄԻԲ} &= 2222 = 2000 + 200 + 20 + 2\\
\text{ՍԴ}   &= 2004 = 2000 + 4\\
\text{ՃԻ}   &= 120 = 100 + 20\\
\text{Ծ}    &= 50
\end{align*}

To write numbers greater than 9999, it is necessary to have numerals with values greater than 9000. This is done by drawing a line over them, indicating their value is to be multiplied by 10000:

\begin{align*}
\overline{\text{Ա}} &= 10000\\
\overline{\text{Ջ}} &= 9000000\\
\overline{\text{ՌՃԽԳ}}\text{ՌՄԾԵ} &= 11431255
\end{align*}
\egroup

^^A

\section{Bopomofo}
\label{s:bopomofo}
Bopomofo is the colloquial name of the \textit{zhuyin fuhao} or \textit{zhuyin} system of phonetic notation for the transcription of spoken Chinese, particularly the Mandarin dialect. Consisting of 37 characters and four tone marks, it transcribes all possible sounds in Mandarin. 

Bopomofo was introduced in China by the Republican Government, in the 1910s and used alongside the Wade-Giles system, which used a modified Latin alphabet. The Wade system was replaced by \textit{Hanyu Pinyin} in 1958 by the Government of the People's Republic of China,[1] at the International Organization for Standardization (ISO) in 1982 (ISO 7098:1982). Bopomofo remains widely used as an educational tool and electronic input method in Taiwan. On Windows the font Microsoft JhengHei can be used. 

Windows fonts that can be used \texttt{Microsoft JhengHei} and \texttt{SimSun}.

U+3100–U+312F
\newfontfamily\bopomofo{Microsoft JhengHei}

\begin{scriptexample}[]{Bopomofo}
{\centering\bopomofo 

伯帛勃脖舶博渤霸壩灞

}

\hfill \texttt{Typeset with \cmd{\bopomofo} and Microsoft JhengHei font }
\end{scriptexample}

\begin{scriptexample}[]{Bopomofo}

{\centering\bopomofo

伯帛勃脖舶博渤霸壩灞

}
\hfill \texttt{Typeset with \cmd{\bopomofo} and JhengHei font }
\end{scriptexample}


The Bopomofo Extended block, running from \unicodenumber{U+31A0-U31BF}, contains less universally recognized Bopomofo characters used to write various non-Mandarin Chinese languages. A few additional tone marks are unified with characters in the Spacing Modifier Letters block. 










^^A\newfontfamily\georgian[Script=Georgian,Scale=1.2]{code2000.ttf}

\newfontfamily\georgianarial[Script=Georgian,Scale=1.2]{Arial Unicode MS}
\section{Georgian}
\label{sec:georgian}
The Georgian scripts are the three writing systems used to write the Georgian language: Asomtavruli, Nuskhuri and Mkhedruli. Their letters are equivalent, sharing the same names and alphabetical order and all three are unicameral (make no distinction between upper and lower case). Although each continues to be used, Mkhedruli (see below) is taken as the standard for Georgian and its related Kartvelian languages\footnote{Unicode Standard, V. 6.3. U10A0, p. 3}. 

\bgroup
\topline



\begin{scriptexample}[]{}
\georgian 

\centering
 
ყველა ადამიანი იბადება თავისუფალი და თანასწორი თავისი ღირსებითა და უფლებებით. მათ მინიჭებული აქვთ გონება და სინდისი და ერთმანეთის მიმართ უნდა იქცეოდნენ ძმობის სულისკვეთებით.
\medskip

\georgianarial
ყველა ადამიანი იბადება თავისუფალი და თანასწორი თავისი ღირსებითა და უფლებებით. მათ მინიჭებული აქვთ გონება და სინდისი და ერთმანეთის მიმართ უნდა იქცეოდნენ ძმობის სულისკვეთებით.
\bottomline
\captionof{table}{Article 1 of the Universal Declaration of Human Rights in Georgian, typeset in \texttt{code2000} (top) and \texttt{Arial Unicode MS } (bottom).}

\end{scriptexample}

The scripts originally had 38 letters. Georgian is currently written in a 33-letter alphabet, as five of the letters are obsolete in that language. The Mingrelian alphabet uses 36: the 33 of Georgian, one letter obsolete for that language, and two additional letters specific to Mingrelian and Svan. That same obsolete letter, plus a letter borrowed from Greek, are used in the 35-letter Laz alphabet. The fourth Kartvelian language, Svan, is not commonly written, but when it is it uses the letters of the Mingrelian alphabet, with an additional obsolete Georgian letter and sometimes supplemented by diacritics for its many vowels.

^^A
^^A\section{Malayalam}
\label{sec:malayam}
\newfontfamily\malayam[Scale=1.1]{Lohit-Malayalam.ttf}

\def\malamtext#1{{\malayam#1}}

The Malayalam script (Malayalam: \malamtext{മലയാളലിപി}, Malayāḷalipi, IPA: [mɐləjaːɭɐ lɪβɪ], also known as Kairali script (Malayalam: \malamtext{കൈരളീലിപി}), is a Brahmic script used commonly to write the Malayalam language—which is the principal language of the Indian state of Kerala, spoken by 35 million people in the world.[3] Like many other Indic scripts, it is an alphasyllabary (\textit{abugida}), a writing system that is partially “alphabetic” and partially syllable-based. The modern Malayalam alphabet has 15 vowel letters, 41 consonant letters, and a few other symbols. The Malayalam script is a Vattezhuttu script, which had been extended with Grantha script symbols to represent Indo-Aryan loanwords.[4] The script is also used to write several minority languages such as Paniya, Betta Kurumba, and Ravula.[5] The Malayalam language itself was historically written in several different scripts.

\begin{scriptexample}[]{Malayalam}
\centerline{\Huge\malamtext{കൈരളീലിപി}}
\end{scriptexample}
^^A\subsection{Greek}
\index{languages>Greek}\index{Herodotus}\index{alphabets>Greek}
\newfontfamily\greek[Script=Greek,Scale=1.02]{NotoSerif-Regular.ttf}
\def\greektext#1{\greek{#1}}

`The Phoenicians who came with Kadmos,' wrote Herodotus in the fifth century BC of the legendary Phoenician prince of Tyre and brother of Europa, `\ldots introduced into Greece, after their settlement in the country, a number of accomplishments of which the most important was writing, an art which probably was unknown to the Greeks until then'. 

The Greek alphabet is the script that has been used to write the Greek language since the 8th century BC.[2] It was derived from the earlier Phoenician alphabet, and was in turn the ancestor of numerous other European and Middle Eastern scripts, including Cyrillic and Latin.[3] Apart from its use in writing the Greek language, both in its ancient and its modern forms, the Greek alphabet today also serves as a source of technical symbols and labels in many domains of mathematics, science and other fields.

In its classical and modern forms, the alphabet has 24 letters, ordered from alpha to omega. Like Latin and Cyrillic, Greek originally had only a single form of each letter; it developed the letter case distinction between upper-case and lower-case forms in parallel with Latin during the modern era.

\bgroup
\greek\obeyspaces

Α	ἄλφα	aleph	alpha	[alpʰa]	[ˈalfa]	Listeni/ˈælfə/
Β	βῆτα	beth	beta	[bɛːta]	[ˈvita]	/ˈbiːtə/, US /ˈbeɪtə/
Γ	γάμμα	gimel	gamma	[ɡamma]	[ˈɣama]	/ˈɡæmə/
Δ	δέλτα	daleth	delta	[delta]	[ˈðelta]	/ˈdɛltə/
Η	ἦτα	  heth	   eta	 [hɛːta], [ɛːta]	[ˈita]	/ˈiːtə/, US /ˈeɪtə/
Θ	θῆτα	teth	theta	[tʰɛːta]	[ˈθita]	/ˈθiːtə/, US Listeni/ˈθeɪtə/
Ι	ἰῶτα	yodh	iota	[iɔːta]	[ˈʝota]	Listeni/aɪˈoʊtə/
Κ	κάππα	kaph	kappa	[kappa]	[ˈkapa]	Listeni/ˈkæpə/
Λ	λάμβδα	lamedh	lambda	[lambda]	[ˈlamða]	Listeni/ˈlæmdə/
Μ	μῦ	mem	mu	[myː]	[mi]	Listeni/ˈmjuː/; occasionally US /ˈmuː/
Ν	νῦ	nun	nu	[nyː]	[ni]	/ˈnjuː/ (US /ˈnuː/)
Ρ	ῥῶ	reš	rho	[rɔː]	[ro]	Listeni/ˈroʊ/
Τ	ταῦ	taw	tau	[tau]	[taf]	/ˈtaʊ/ or /ˈtɔː/

\topline
\begin{quote}
Ἡροδότου Ἁλικαρνησσέος ἱστορίης ἀπόδεξις ἥδε, ὡς μήτε τὰ γενόμενα ἐξ ἀνθρώπων τῷ χρόνῳ ἐξίτηλα γένηται, μήτε ἔργα μεγάλα τε καὶ θωμαστά, τὰ μὲν Ἕλλησι, τὰ δὲ βαρβάροισι ἀποδεχθέντα, ἀκλεᾶ γένηται, τὰ τε ἄλλα καὶ δι' ἣν αἰτίην ἐπολέμησαν ἀλλήλοισι.[2]

Herodotus of Halicarnassus, his Researches are set down to preserve the memory of the past by putting on record the astonishing achievements of both the Greeks and the Barbarians; and more particularly, to show how they came into conflict.[3]
\end{quote}
\bottomline

\symbol{"1F00}
\symbol{"1F01}
\egroup
^^A
^^A\subsection{Kannada alphabet}

\newfontfamily\kannada[Scale=1.0,Script=Kannada]{Lohit-Kannada.ttf}

\def\kannadatext#1{{\kannada#1}}

The Kannada alphabet (\kannadatext{ಕನ್ನಡ ಲಿಪಿ}) is an abugida of the Brahmic family,[2] used primarily to write the Kannada language, one of the Dravidian languages of southern India. Several minor languages, such as Tulu, Konkani, Kodava, and Beary, also use alphabets based on the Kannada script.[3] The Kannada and Telugu scripts share high mutual intellegibility with each other, and are often considered to be regional variants of single script. Similarly, Goykanadi, a variant of Old Kannada, has been historically used to write Konkani in the state of Goa.[4]

\begin{scriptexample}[]{Kannada}
\centerline{\LARGE\kannadatext{ಙ	ಙ್ಕ	ಙ್ಖ	ಙ್ಗ	ಙ್ಘ	ಙ್ಙ	ಙ್ಚ	ಙ್ಛ	ಙ್ಜ	ಙ್ಝ	ಙ್ಞ	ಙ್ಟ	ಙ್ಠ	ಙ್ಡ	ಙ್ಢ}}
\end{scriptexample}

\medskip

The Kannada script (aksharamale or varnamale) is a phonemic abugida of forty-nine letters, and is written from left to right. The character set is almost identical to that of other Brahmic scripts. Consonantal letters imply an inherent vowel. Letters representing consonants are combined to form digraphs (ottaksharas) when there is no intervening vowel. Otherwise, each letter corresponds to a syllable.
The letters are classified into three categories: swara (vowels), vyanjana (consonants), and yogavaahaka (part vowel, part consonant).
The Kannada words for a letter of the script are akshara, akkara, and varna. Each letter has its own form (ākāra) and sound (shabda), providing the visible and audible representations, respectively. Kannada is written from left to right.[7]
^^A\section{Myanmar}
\label{s:myanmar}
\index{Myanmar}\index{Burmese}\index{Mon}\index{Unicode>Myanmar}\index{Fonts>Padauk}

%\newfontfamily\myanmar{Padauk}

The Burmese script (Burmese:{\myanmar မြန်မာအက္ခရာ}; MLCTS: mranma akkha.ra; pronounced: [mjəmà ʔɛʔkʰəjà]) is an abugida in the Brahmic family, used for writing Burmese. It is an adaptation of the Old Mon script[2] or the Pyu script. In recent decades, other alphabets using the Mon script, including Shan and Mon itself, have been restructured according to the standard of the now-dominant Burmese alphabet. Besides the Burmese language, the Burmese alphabet is also used for the liturgical languages of Pali and Sanskrit.

The characters are rounded in appearance because the traditional palm leaves used for writing on with a stylus would have been ripped by straight lines.[3] It is written from left to right and requires no spaces between words, although modern writing usually contains spaces after each clause to enhance readability.

The earliest evidence of the Burmese alphabet is dated to 1035, while a casting made in the 18th century of an old stone inscription points to 984.[1] Burmese calligraphy originally followed a square format but the cursive format took hold from the 17th century when popular writing led to the wider use of palm leaves and folded paper known as parabaiks.[3] The alphabet has undergone considerable modification to suit the evolving phonology of the Burmese language.

Mon/Burmese script was added to the Unicode Standard in September, 1999 with the release of version 3.0. It was extended in October, 2009 with the release of version 5.2 and again in June, 2014 with the release of version 7.0.

\begin{docKey}[phd]{myanmar font}{=\meta{font name}}{default none initial Padauk}
Loads the font and creates associated environments and commands.
\end{docKey}

\begin{scriptexample}[]{Myanmar}
\unicodetable{myanmar}{"1000,"1010,"1020,"1030,"1040,"1050,"1060,"1070,"1080,"1090}
\end{scriptexample}







^^A
^^A\subsection{Osmanian Alphabet}

\bgroup
\newfontfamily\osmanian{code2001.ttf}
\osmanian
𐒚𐒁𐒖𐒄 𐒚𐒐 𐒚 𐒎𐒚𐒍𐒚𐒐 𐒑𐒚𐒒𐒠𐒚𐒐 𐒎𐒚𐒑𐒁𐒗 𐒚𐒁𐒖𐒄 𐒚𐒌𐒖𐒄 𐒚𐒁𐒖𐒄𐒖 𐒚
𐒌𐒜
\egroup
^^A\newfontfamily\hanunoo{NotoSansHanunoo-Regular.ttf}

\section{Hanunó’o}

Hanunó’o is one of the indigenous scripts of the Philippines and is used by the Mangyan peoples of southern Mindoro to write the Hanunó'o language.[1] 

It is an \emphasis{abugida} descended from the Brahmic scripts, closely related to Baybayin, and is famous for being written vertical but written upward, rather than downward as nearly all other scripts (however, it's read horizontally left to right). It is usually written on bamboo by incising characters with a knife.[2][3] Most known Hanunó'o inscriptions are relatively recent because of the perishable nature of bamboo. It is therefore difficult to trace the history of the script



\begin{scriptexample}[width=2cm]{Hanunoo}
\hanunoo

{\Large
\obeylines
ᜠ 
ᜫ
ᜨᜲ
ᜫᜲ
ᜰ
ᜮ
ᜥ
ᜦ᜴}

Typeset with \texttt{NotoSansHanunoo-Regular.ttf} and the command \cmd{\hanunoo}
\end{scriptexample}

Vertically positionning the text is not currently supported by \pkgname{fontspec} and the manual says \textsc{Todo!}. You are your own here, or you can just put the characters in a box and give it a try.

\begin{minipage}[t]{2cm}
\begin{tcolorbox}[width=2cm,colback=graphicbackground,
boxrule=0pt,toprule=0pt,colframe=white]
\Large\hanunoo
ᜩ\\
ᜤ\\
ᜮ\\
ᜥᜳ\\
ᜨ᜴ \\
ᜨ᜴\\
ᜫᜳ\\
ᜥ\\
\end{tcolorbox}
\end{minipage}
\begin{minipage}[t]{2cm}
\begin{tcolorbox}[width=2cm,colback=graphicbackground,
boxrule=0pt,toprule=0pt,colframe=white]
\LARGE\hanunoo
ᜩ\\
ᜤ\\
ᜮ\\
ᜥᜳ\\
ᜨ᜴ \\
ᜨ᜴\\
ᜫᜳ\\
ᜥ\\
\end{tcolorbox}
\end{minipage}
\begin{minipage}[t]{\textwidth-6cm}

The script is written from bottom to top. Typesetting this type of script automatically is not without its problems. One way is to use the build-in features of the font if they are available, but currently this gives problems---at least with the fonts that I have tried. Entering the text is also problematic as you will more than likely see little boxes rather than the actual glyph with most text editors common to \latexe. If you only need a couple of characters or a short sentence, an easy solution is to use |\rotatebox|. Another solution is to use a macro that can add the letters onto a stack, then place them in a box with a limited width. We can use |\@tfor| for this.  
\end{minipage}
^^A
^^A\newfontfamily\glagolitic{MPH 2B Damase}

\section{Glagolitic}

\epigraph{The average Ph.D. thesis is nothing but a transference of bones from one graveyard to another.}{%
J. Frank Dobie (1888-1964)}


\label{s:glagolitic}
\fboxrule0pt\fboxsep0pt

\noindent
The Glagolitic alphabet /{\glagolitic ˌɡlæɡɵˈlɪtɨk/}, also known as Glagolitsa, is the oldest known Slavic alphabet, from the 9th century.

It was created in the 9th century by Saint Cyril, a Byzantine monk from Thessaloniki. He and his brother, Saint Methodius, were sent by the Byzantine Emperor Michael III in 863 to Great Moravia to spread Christianity among the West Slavs in the area. The brothers decided to translate liturgical books into the Old Slavic language that was understandable to the general population, but as the words of that language could not be easily written by using either the Greek or Latin alphabets, Cyril decided to invent a new script, Glagolitic, which he based on the local dialect of the Slavic tribes from the Byzantine Salonika region.
After the deaths of Cyril and Methodius, the Glagolitic alphabet ceased to be used in Moravia, but their students continued to propagate it in the west and south. 

After a long career, Glagolitic writing stopped being used, except for
religious purposes in certain dioceses of Bosnia and Dalmatia (Croatia).
The Cyrillic alphabet was adopted by all Orthodox Slays and served to note
their literary language. Most of the Slays who rallied to Rome rejected it,
however, which created the paradoxical situation in ex-Yugoslavia, where
two peoples who speak the same language write in different scripts, the
Serbs in Cyrillic and the Croats with Roman characters. Finally, as is
known, the ex-Soviet Union did much to put into writing the languages
spoken by the peoples within its borders, for the most part noting them in
adaptations of the Cyrillic alphabet, while Russian became the language of
culture throughout the Soviet Union.\cite{henri1994}

Slavic printing in Glagolitic characters originated in Venice, where a
\textit{Sluzebnik} (or \textit{Leitourgikon}) was published in 1483, followed by missals and
breviaries, all printed by Andrea Torresani, the future father-in-law and
associate of Aldus Manutius. After 1494 some attempts were made to create
printshops in Croatia itself, first in Senj in 1508, then, after 1530, in
Rijeka (Fiume). The work of these firms was almost totally liturgical (religious,
at any rate), and it had strong competition from manuscript works
that were better adapted to the diversity of local liturgical customs. Religion
also dictated the output of a printshop founded to provide Protestant propaganda
that was set up in Tubingen between 1560 and 1564 by Baron
Hans von Ungnad and that printed the great Lutheran texts in Glagolitic
characters.\footfullcite{henri1994}

Figure~\ref{fig:zograf} illustrates an example of the language.\footnote{\url{https://en.wikipedia.org/wiki/Glagolitic_script\#/media/File:ZographensisColour.jpg}}

\begin{figure}[htbp]
\centering

\includegraphics[width=0.45\linewidth]{glagolitic}
\caption[The first page of the Gospel of Mark from the 10th–11th century Codex Zographensis, found in the Zograf Monastery in 1843.]{The first page of the Gospel of Mark from the 10th–11th century Codex Zographensis, found in the Zograf Monastery in 1843.}
\label{fig:zograf}
\end{figure}

\section{Unicode Support}
The Glagolitic alphabet was added to the Unicode Standard in March 2005 with the release of version 4.1.
The Unicode block for Glagolitic is U+2C00–U+2C5F.



\begin{scriptexample}[]{glacolitic}

\unicodetable{glagolitic}{%
"2C00,"2C10,"2C20,"2C30,"2C40,"2C50}

\texttt{typeset with Damase version 2.0 MPH 2B Damase}
\end{scriptexample}
\bgroup
\glagolitic

The name was not coined until many centuries after its creation, and comes from the Old Church Slavonic glagolъ "utterance" (also the origin of the Slavic name for the letter G). The verb glagoliti means "to speak". It has been conjectured that the name glagolitsa developed in Croatia around the 14th century and was derived from the word glagolity, applied to adherents of the liturgy in Slavonic.[1]

In Old Church Slavonic the name is {\glagolitic ⰍⰫⰓⰊⰎⰎⰑⰂⰋⰜⰀ}, Кѷрїлловица.
The name Glagolitic in Bulgarian, Russian, Macedonian глаголица (glagolica), Belarusian is глаголіца (hłaholica), Croatian glagoljica, Serbian глагољица / glagoljica, Czech hlaholice, Polish głagolica, Slovene glagolica, Slovak hlaholika, and Ukrainian глаголиця (hlaholyća).



\egroup

\section{Additional Modern Scripts}

\begin{center}
\begin{tabular}{lp{5cm}l}
Ethiopic. &Vai. &Deseret.\\
Mongolian. &Bamum. &Shavian.\\
Osmanya.   &Cherokee. &Lisu.\\
Tifinagh.  &Canadian Aboriginal Syllabics. &Miao.\\
N’Ko.&&\\
\end{tabular}
\end{center}

Ethiopic, Mongolian, and Tifinagh are scripts with long histories. Although their roots can
be traced back to the original Semitic and North African writing systems, they would not
be classified as Middle Eastern scripts today

The Cherokee script is a syllabary developed between 1815 and 1821, to write the Cherokee
language, still spoken by small communities in Oklahoma and North Carolina. Canadian
Aboriginal Syllabics were invented in the 1830s for Algonquian languages in Canada. The
system has been extended many times, and is now actively used by other communities, including speakers of Inuktitut and Athapascan languages.

Deseret is a phonemic alphabet devised in the 1850s to write English. It saw limited use for
a few decades by members of The Church of Jesus Christ of Latter-day Saints. Shavian is
another phonemic alphabet, invented in the 1950s to write English. It was used to publish
one book in 1962, but remains of some current interest




\subsection{Ethiopic}
Ge'ez (ግዕዝ Gəʿəz), (also known as Ethiopic) is a script used as an abugida (syllable alphabet) for several languages of Ethiopia and Eritrea. It originated as an abjad (consonant-only alphabet) and was first used to write Ge'ez, now the liturgical language of the Ethiopian Orthodox Tewahedo Church and the Eritrean Orthodox Tewahedo Church. In Amharic and Tigrinya, the script is often called fidäl (ፊደል), meaning "script" or "alphabet".

The Ge'ez script has been adapted to write other, mostly Semitic, languages, particularly Amharic in Ethiopia, and Tigrinya in both Eritrea and Ethiopia. It is also used for Sebatbeit, Me'en, and most other languages of Ethiopia. In Eritrea it is used for Tigre, and it has traditionally been used for Blin, a Cushitic language. Tigre, spoken in western and northern Eritrea, is considered to resemble Ge'ez more than do the other derivative languages.[citation needed] Some other languages in the Horn of Africa, such as Oromo, used to be written using Ge'ez, but have migrated to Latin-based orthographies.
For the representation of sounds, this article uses a system that is common (though not universal) among linguists who work on Ethiopian Semitic languages. This differs somewhat from the conventions of the International Phonetic Alphabet. See the articles on the individual languages for information on the pronunciation.

There are a number of fonts available and we have selected the Google \idxfont{NotoSansEthiopic}
\newfontfamily\ethiopic{NotoSansEthiopic-Bold.ttf}

\begin{scriptexample}[]{Ethiopic}
\unicodetable{ethiopic}{"1200,"1210,"1220,"1230,"1240,"1250,"1260,"1270,"1280,"1290,^^A
"12A0,"12B0,"12C0,"12E0,"12F0,"1300,"1310,"1330,"1340,"1350,"1360,"1370}
\end{scriptexample}
\section{Vai}
\label{s:vai}

The Vai syllabary is a syllabic writing system devised for the Vai language by Momolu Duwalu Bukele of Jondu, in what is now Grand Cape Mount County, Liberia.[1] [2] Bukele is regarded within the Vai community, as well as by most scholars, as the syllabary's inventor and chief promoter when it was first documented in the 1830s. It is one of the two most successful indigenous scripts in West Africa.

\newfontfamily\vai{code2000.ttf}
\begin{scriptexample}[]{Vai}
\unicodetable{vai}{"A500,"A510,"A520,"A530,"A540,"A550,"A560,"A570,^^A
"A580,"A590,"A5A0,"A5B0,^^A
"A5C0,"A5D0,"A5E0,"A5F0,"A610,"A620,"A630}
\end{scriptexample}

In the 1920s ten decimal digits were devised for Vai; these were “Vai-style” glyph variants of
European digits (see Figure 11). They were not popular with Vai people  even for historical purposes. All
the modern literature uses European digits.


\begin{scriptexample}[]{Vai}
\bgroup
\vai
\obeylines\Large
0	1	2	3	4	5	6	7	8	9
꘠	꘡	꘢	꘣	꘤	꘥	꘦	꘧	꘨	꘩
\vai
\egroup
\end{scriptexample}



\printunicodeblock{./languages/vai.txt}{\vai}
\section{Deseret script}
\newfontfamily\deseret{code2001.ttf}

The Deseret alphabet (dɛz.əˈrɛt.) (Deseret: {\deseret 𐐔𐐯𐑅𐐨𐑉𐐯𐐻 or 𐐔𐐯𐑆𐐲𐑉𐐯𐐻}) is a phonemic English spelling reform developed in the mid-19th century by the board of regents of the University of Deseret (later the University of Utah) under the direction of Brigham Young, second president of The Church of Jesus Christ of Latter-day Saints.

In public statements, Young claimed the alphabet was intended to replace the traditional Latin alphabet with an alternative, more phonetically accurate alphabet for the English language. This would offer immigrants an opportunity to learn to read and write English, he said, the orthography of which is often less phonetically consistent than those of many other languages. Similar experiments were not uncommon during the period, the most well-known of which is the Shavian alphabet.

Young also prescribed the learning of Deseret to the school system, stating "It will be the means of introducing uniformity in our orthography, and the years that are now required to learn to read and spell can be devoted to other studies".[2]


Deseret script {\deseret 𐐔𐐯𐑅𐐨𐑉𐐯𐐻}  [U+10400-U+1044F]
\medskip

\bgroup
\par
\noindent
\colorbox{graphicbackground}{\color{black}^^A
\begin{minipage}{\textwidth}^^A
\parindent1pt
\vskip10pt
\leftskip10pt \rightskip\leftskip
\deseret
\large

𐐂 𐑌𐐲𐑉𐑅𐐨𐑉𐐮 𐐮𐑆 𐐪 𐐹𐐨𐑅 𐐱𐑂 𐑊𐐰𐑌𐐼 𐐱𐑌 𐐸𐐶𐐮𐐽 𐑁𐑉𐐭𐐻𐐻𐑉𐐨𐑆 𐐪𐑉 𐑅𐐻𐐪𐑉𐐻𐐯𐐼,


\par
\vspace*{10pt}
\end{minipage}
}

Text: Deseret alphabet http://www.omniglot.com/writing/deseret.htm
\medskip
\egroup

\PrintUnicodeBlock{./languages/deseret.txt}{\deseret}

\chapter{Bamum}
\label{s:bamum}
\epigraph{"No known alphabet was ever invented by a European."}{Jeffreys' translation from the Royal script.}

\label{s:bamum}
\index{scripts>Bamum}
\newfontfamily\bamum{NotoSansBamum-Regular.ttf}

The Bamum scripts are an evolutionary series of six scripts created for the Bamum language by King Njoya of Cameroon at the turn of the 20th century. They are notable for evolving from a pictographic system to a partially alphabetic syllabic script in the space of 14 years, from 1896 to 1910. Bamum type was cast in 1918, but the script fell into disuse around 1931.

\begin{figure}[htbp]
\parindent=0pt

\centering

\includegraphics[width=\textwidth]{bamum}

\caption{King Njoya of Bamum receiving an oil painting of Kaiser Wilhelm II. The gift was in return for his support in the German campaign against the Nso'.}
\end{figure}

The Bamum, sometimes called Bamoum, Bamun, Bamoun, or Mum, are a Bantoid ethnic group of Cameroon with around 215,000 members.



\begin{scriptexample}[]{Bamum}
\unicodetable{bamum}{"A6A0,"A6B0,"A6C0,"A6D0,"A6E0,"A6F0}
\end{scriptexample}
\section{Shavian}
\label{s:shavian}
\def\shaviansetup#1{}
\newfontfamily\shavian{code2001.ttf}
^^A\newfontfamily\shavian{NotoSansShavian-Regular.ttf}
\cxset{shavian font/.code=\shaviansetup{#1}}
\cxset{shavian font=shavian}




\begin{scriptexample}[]{shavian}
\shavian

𐑳 𐑡𐑻𐑯𐑰 𐑑 𐑞 𐑕𐑧𐑯𐑑𐑻 𐑝 𐑞 𐑻𐑔
𐑚𐑲 - ·𐑡𐑵𐑤𐑟 ·𐑝𐑻𐑯

𐑗𐑩𐑐𐑑𐑻 1 - 𐑥𐑲 𐑳𐑙𐑒𐑳𐑤 𐑥𐑱𐑒𐑕 𐑳 𐑜𐑮𐑱𐑑 𐑛𐑦𐑕𐑒𐑳𐑝𐑻𐑰

     𐑤𐑫𐑒𐑦𐑙 𐑚𐑩𐑒 𐑑 𐑷𐑤 𐑞𐑩𐑑 𐑣𐑩𐑟 𐑳𐑒𐑻𐑛 𐑑 𐑥𐑰 𐑕𐑦𐑯𐑕 𐑞𐑩𐑑 𐑦𐑝𐑧𐑯𐑑𐑓𐑳𐑤 𐑛𐑱, 𐑲 𐑩𐑥 𐑕𐑒𐑧𐑮𐑕𐑤𐑰 𐑱𐑚𐑳𐑤 𐑑 𐑚𐑦𐑤𐑰𐑝 𐑦𐑯 𐑞 𐑮𐑰𐑩𐑤𐑳𐑑𐑰 𐑝 𐑥𐑲 𐑩𐑛𐑝𐑧𐑯𐑗𐑻𐑟. 𐑞𐑱 𐑢𐑻 𐑑𐑮𐑵𐑤𐑰 𐑕𐑴 𐑢𐑳𐑯𐑛𐑻𐑓𐑳𐑤 𐑞𐑩𐑑 𐑰𐑝𐑦𐑯 𐑯𐑬 𐑲 𐑩𐑥 𐑚𐑦𐑢𐑦𐑤𐑛𐑻𐑛 𐑢𐑧𐑯 𐑲 𐑔𐑦𐑙𐑒 𐑝 𐑞𐑧𐑥.
     𐑥𐑲 𐑳𐑙𐑒𐑳𐑤 𐑢𐑪𐑟 𐑳 𐑡𐑻𐑥𐑳𐑯, 𐑣𐑩𐑝𐑦𐑙 𐑥𐑧𐑮𐑰𐑛 𐑥𐑲 𐑥𐑳𐑞𐑻𐑟 𐑕𐑦𐑕𐑑𐑻, 𐑩𐑯 𐑦𐑙𐑜𐑤𐑦𐑖𐑢𐑫𐑥𐑳𐑯. 𐑚𐑰𐑦𐑙 𐑝𐑧𐑮𐑰 𐑥𐑳𐑗 𐑳𐑑𐑩𐑗𐑑 𐑑 𐑣𐑦𐑟 𐑓𐑪𐑞𐑻𐑤𐑳𐑕 𐑯𐑧𐑓𐑘𐑵, 𐑣𐑰 𐑦𐑯𐑝𐑲𐑑𐑳𐑛 𐑥𐑰 𐑑 𐑕𐑑𐑳𐑛𐑰 𐑳𐑯𐑛𐑻 𐑣𐑦𐑥 𐑦𐑯 𐑣𐑦𐑟 𐑣𐑴𐑥 𐑦𐑯 𐑞 𐑓𐑪𐑞𐑻𐑤𐑩𐑯𐑛. 𐑞𐑦𐑕 𐑣𐑴𐑥 𐑢𐑪𐑟 𐑦𐑯 𐑳 𐑤𐑪𐑮𐑡 𐑑𐑬𐑯, 𐑯 𐑥𐑲 𐑳𐑙𐑒𐑳𐑤 𐑳 𐑐𐑮𐑳𐑓𐑧𐑕𐑻 𐑝 𐑓𐑳𐑤𐑪𐑕𐑳𐑓𐑰, 𐑒𐑧𐑥𐑳𐑕𐑑𐑮𐑰, 𐑡𐑰𐑪𐑤𐑳𐑡𐑰, 𐑥𐑦𐑯𐑻𐑪𐑤𐑳𐑡𐑰, 𐑯 𐑥𐑧𐑯𐑰 𐑳𐑞𐑻 𐑳𐑤𐑴𐑡𐑰𐑕.

\arial

\hfill Excerpt from Jules Vern,  \textit{Journey to the Center of the Earth from \href{http://shavian.weebly.com/}{shavian}}
\end{scriptexample}

The example is typeset using \texttt{code2001.ttf}. There are numerous fonts that provide Shavian glyphs. \texttt{ESL Gothic Unicode} font by Ethan Lamoreaux\footnote{\url{http://www.fontspace.com/ethan-lamoreaux/esl-gothic-unicode}}. The Noto fonts also have a Shavian font. 

You can activate typesetting in Shavian using the key:

\begin{key}{/chapter/shavian font = \meta{font name}} The key will setup the
default font for the Shavian script and define the commands \cmd{\shavian} and \cmd{\textshavian}. 
\end{key}

\PrintUnicodeBlock{./languages/shavian.txt}{\shavian}





\subsection{Osmanya}

\newfontfamily\osmanya{NotoSansOsmanya-Regular.ttf}

\begin{scriptexample}[]{Osmanya}
\unicodetable{osmanya}{"10480,"10490,"104A0}
\end{scriptexample}

The Osmanya alphabet (Somali: Cismaanya; Osmanya: {\osmanya 𐒋𐒘𐒈𐒑𐒛𐒒𐒕𐒀}), also known as Far Soomaali ("Somali writing"), is a writing script created to transcribe the Somali language. It was invented between 1920 and 1922 by Osman Yusuf Kenadid of the Majeerteen Darod clan, the nephew of Sultan Yusuf Ali Kenadid of the Sultanate of Hobyo.

While Osmanya gained reasonably wide acceptance in Somalia and quickly produced a considerable body of literature, it proved difficult to spread among the population mainly due to stiff competition from the long-established Arabic script as well as the emerging Somali alphabet developed by the Somali linguist, Shire Jama Ahmed, which was based on the Latin script.

As nationalist sentiments grew and since the Somali language had long lost its ancient script,[1] the adoption of a universally recognized writing script for the Somali language became an important point of discussion. After independence, little progress was made on the issue, as opinion was divided over whether the Arabic or Latin scripts should be used instead.

In October 1972, due to its simplicity, the fact that it lent itself well to writing Somali since it could cope with all of the sounds in the language, and the already widespread existence of machines and typewriters designed for its use,[2][3] the government of Somali president Mohamed Siad Barre unilaterally elected to use only the Latin script for writing Somali instead of the Arabic or Osmanya scripts.[4] Barre's administration subsequently launched a massive literacy campaign designed to ensure its sole adoption. This led to a sharp decline in use of Osmanya.
\section{Cherokee}
\index{scripts>Cherokee}
\index{scripts>Cherokee>fonts}
\label{sec:cherokee}
Windows comes with |Plantagenet Cherokee| font. The |code2000| also has good support for the alphabet. The \texttt{SIL font Charis SIL} also has good support and can be downloaded at \href{http://scripts.sil.org/cms/scripts/page.php?item_id=CharisSIL_download}{scripts.sel.org}, the latest version gave me problems when used with Windows. 

  
\def\textcherokee#1{{\cherokee   #1}}


\begin{docKey}[phd]{cherokee font}{ = \meta{font name}} {default none, initial=code2000}
 Loads the font
command \cmd{\cherokee}. When the command is used it typesets text in
cherokee unicode. There is no need to load the language, unless it is the main document language. For windows the default font is  |Plantagenet Cherokee|. Another font is FreeSerif, which we are using here.
\end{docKey}

\begin{scriptexample}[]{Cherokee}
{\cherokee
\begin{tabular}{lp{8.5cm}}
Translation	  &John (ᏣᏂ) 3:16\\
American Bible Society 1860	&ᎾᏍᎩᏰᏃ ᏂᎦᎥᎩ ᎤᏁᎳᏅᎯ ᎤᎨᏳᏒᎩ ᎡᎶᎯ, ᏕᏅᏲᏒᎩ ᎤᏤᎵᎦ ᎤᏪᏥ ᎤᏩᏒᎯᏳ ᎤᏕᏁᎸᎯ, ᎩᎶ ᎾᏍᎩ ᏱᎪᎯᏳᎲᏍᎦ ᎤᏲᎱᎯᏍᏗᏱ ᏂᎨᏒᎾ, ᎬᏂᏛᏉᏍᎩᏂ ᎤᏩᏛᏗ.\\

(Transliteration)	& nasgiyeno nigavgi unelanvhi ugeyusvgi elohi, denvyosvgi utseliga uwetsi uwasvhiyu udenelvhi, gilo nasgi yigohiyuhvsga uyohuhisdiyi nigesvna, gvnidvquosgini uwadvdi.\\
\end{tabular}}
\end{scriptexample}

\begin{texexample}{Using text...}{cherokee}
\bgroup
\cherokee \large\textbf{ᎾᏍᎩᏰᏃ}
\textcherokee{ᎾᏍᎩᏰᏃ}
\egroup
\end{texexample}

If you have trouble getting them to work\footnote{\url{http://tex.stackexchange.com/questions/132087/displaying-cherokee-text}}

\url{http://www.cherokee.org/AboutTheNation/Language/CherokeeFont.aspx}




\section{Tifnagh}

\newfontfamily\tifinagh{code2000.ttf}

Tifinagh (Berber pronunciation: [tifinaɣ]; also written Tifinaɣ in the Berber Latin alphabet, {\tifinagh  ⵜⵉⴼⵉⵏⴰⵖ} in Neo-Tifinagh, and تيفيناغ in the Berber Arabic alphabet) is a series of abjad and alphabetic scripts used by Berber peoples to write Berber languages.[1]
A modern derivate of the traditional script, known as Neo-Tifinagh, was introduced in the 20th century. A slightly modified version of the traditional script, called Tifinagh Ircam, is used in a number of Moroccan elementary schools in teaching the Berber language to children as well as a number of publications.[2][3]

The word tifinagh is thought to be a Berberized feminine plural cognate of Punic, through the Berber feminine prefix ti- and Latin Punicus; thus tifinagh could possibly mean "the Phoenician (letters)"[4][5] or "the Punic letters".

\bgroup

\noindent\tifinagh
\colorbox{thecodebackground}{\color{black}^^A
\begin{minipage}{\textwidth}
\parindent1pt
\vskip10pt
\leftskip10pt \rightskip\leftskip
Tifnagh     ⵜⵉⴼⵉⵏⴰⵖ [U+2D30-U+2D7F]

ⴰⴳⵍⴷⵓⵏ ⴰⵎⵥⵥⴰ

ⵙ ⵡⴰⵡⴰⵍ ⴳⵔⵉ ⵉⴷⵙ, ⵙⵙⵏⵖ ⵢⴰⵜ ⵜⵖⴰⵡⵙⴰ ⵜⵉⵙⵙ ⵙⵏⴰⵜ  ⵉⵅⴰⵜⵔⵏ: ⵉⵜⵔⵉ ⵙⴳ ⴷⴷ ⵉⴷⴷⴰ ⵓⵔ ⵉⵎⵇⵇⵓⵔ, ⵉⵍⵍⴰ ⵖⴰⵙ ⴰⵏⵛⵜ ⵏ ⵢⴰⵜ ⵜⴰⴷⴷⴰⵔⵜ !

ⴰⵢⴰ ⵓⴽⵣⵖ ⵜ. ⵙⵙⵏⵖ ⵉⵙ ⴱⵕⵕⴰ ⵏ ⵉⵜⵔⴰⵏ ⵣⵓⵏⴷ ⴰⴽⴰⵍ, ⵊⵓⴱⵉⵜⵔ, ⵎⴰⵔⵙ, ⴱⵉⵏⵓⵙ – ⵉⵜⵔⴰⵏ ⵎⵉ ⵏⴽⴼⴰ ⵉⵙⵎⴰⵡⵏ – ⵍⵍⴰⵏ ⴷⵉⵖ ⵉⵜⵔⴰⵏ ⵢⴰⴹⵏ ⵎⵥⵥⵉⵢⵏⵉⵏ, ⵡⵉⵏⵏⴰ ⵓⵔ ⵏⵣⵎⵉⵔ ⴰⴷ ⵏⵥⵔ ⵙ ⵓⵜⵉⵍⵉⵙⴽⵓⴱ. ⴰⴷⴷⴰⵢ ⵢⵓⴼⴰ ⵓⴰⵙⵜⵕⵓⵏⵓⵎ ⵢⴰⵏ ⴷⵉⴳⵙⵏ, ⴷⴰ ⵢⴰⵙ ⵉⵜⵜⴳⴰ ⵙ ⵢⵉⵙⵎ ⵢⴰⵏ ⵡⵓⵜⵜⵓⵏ. ⴷⴰ ⵢⴰⵙ ⵉⵇⵇⴰⵔ ⵙ ⵓⵎⴷⵢⴰⵜ : « ⴰⵙⵜⵔⵓⵉⴷ 3251 ».

ⵓⴽⵣⵖ ⵉⵙ ⴷⴷ ⵉⴷⴷⴰ ⵓⴳⵍⴷⵓⵏ ⵎⵥⵥⵉⵢⵏ ⵙⴳ ⵉⵜⵔⵉ ⵎⵉ ⵇⵇⴰⵔⵏ ⴰⵙⵜⵔⵓⵉⴷ ⴱ612. ⴰⵙⵜⵔⵓⵉⴷ ⴰ, ⵓⵔ ⵉⵜⵓⵥⵔⴰ ⴰⵔ 1909 ⵙ ⵓⵜⵉⵍⵉⵙⴽⵓⴱ. ⵉⵥⵔⴰ ⵜ ⵢⴰⵏ ⵓⴰⵙⵜⵕⵓⵏⵓⵎ ⴰⵜⵓⵔⴽⵉⵢ. ⵉⵙⵙⴽⵏ ⵜⵓⴼⴰⵢⵜ ⵏⵏⵙ ⴳ ⵢⴰⵏ ⵓⴳⵔⴰⵡ ⴰⴳⵔⴰⵖⵍⴰⵏ ⵏ ⵍⴰⵙⵜⵕⵓⵏⵓⵎⵢ. ⵎⴰⵛⴰ, ⴰⴽⴷ ⵢⵉⵡⵏ ⵓⵔ ⵜ ⵢⵓⵎⵏ ⴰⵛⴽⵓ ⵉⵍⵍⴰ ⵉⵍⵙⴰ ⵢⴰⵜ ⵎⵍⵙⵉⵡⵜ ⵓⵔ ⵉⴳⵉⵏ ⴰⵎⵎ ⵜⵉⵏ ⵎⴷⴷⵏ. ⵎⴷⴷⵏ ⵉⵎⵇⵔⴰⵏⴻⵏ, ⴰⵎⴽⴰ ⴰⴽⴽ ⴰⵢ ⴳⴰⵏ.

ⵎⴰⵛⴰ ⵙ ⵓⵎⴷⴰⵣ ⵏ ⵜⵓⵙⵙⵏⴰ ⵏ ⴰⵙⵜⵔⵓⵉⴷ ⴱ612, ⵉⴽⴽⵔ ⵢⴰⵏ ⵓⴷⵉⴽⵜⴰⵜⵓⵔ ⴰⵜⵓⵔⴽⵢ, ⵉⴳⴳ ⴰⵙⵏ ⵛⵛⵉⵍ ⵉ ⵎⴷⴷⵏ ⴰⴷ ⵍⵙⵙⴰⵏ ⵎⵍⵙⵉⵡⵜ ⵏ ⵓⵔⵓⴱⵉⵢⵏ, ⵡⴰⵏⵏⴰ ⵢⴰⴳⵉⵏ ⵉⵏⵖ ⵜ. ⴰⵙⵜⵔⵓⵏⵓⵎ ⵏⵏⴰⵖ, ⵢⵓⵍⵙ ⴷⵉⵖ ⵉ ⵜⵎⵙⴽⴰⵏⵜ ⵏⵏⵙ ⴰⵙⴳⴳⴰⵙ ⵏ 1920, ⵜⵉⴽⴽⵍⵜ ⵏⵏⴰⵖ ⵉⵍⵍⴰ ⵉⵍⵙⴰ ⵢⴰⵜ ⵎⵍⵙⵉⵡⵜ ⵢⵖⵓⴷⴰⵏ ⵛⵉⴳⴰⵏ. ⵜⵉⴽⴽⵍⵜ ⵏⵏⴰⵖ, ⵎⴷⴷⵏ ⴰⴽⴽ ⵓⵎⴻⵏ ⴰⵡⴰⵍ ⵏⵏⵙ.
\par
\vspace*{10pt}
\end{minipage}
}

\subsection{Unified Canadian Aboriginal Syllabics}

Unified Canadian Aboriginal Syllabics is a Unicode block containing characters for writing Inuktitut, Carrier, several dialects of Cree, and Canadian Athabascan languages. Additions for some Cree dialects, Ojibwe, and Dene can be found at the Unified Canadian Aboriginal Syllabics Extended block.
\medskip

\newfontfamily\aboriginal{code2000.ttf}
\bgroup
\par
\noindent
\colorbox{graphicbackground}{\color{black}^^A
\begin{minipage}{\textwidth}^^A
\parindent1pt
\vskip10pt
\leftskip10pt \rightskip\leftskip

\aboriginal
ᒥᓯᐌ ᐃᓂᓂᐤ ᑎᐯᓂᒥᑎᓱᐎᓂᐠ ᐁᔑ ᓂᑕᐎᑭᐟ ᓀᐢᑕ ᐯᔭᑾᐣ ᑭᒋ ᐃᔑ
\bfseries ᑲᓇᐗᐸᒥᑯᐎᓯᐟ ᑭᐢᑌᓂᒥᑎᓱᐎᓂᐠ ᓀᐢᑕ ᒥᓂᑯᐎᓯᐎᓇ᙮
Unicode Block: Unified Canadian Aboriginal Syllabics, UCAS Extended
Text: UDHR: Cree, Swampy ᐯᔭᐠ ᐱᐢᑭᑕᓯᓇᐃᑲᐣ ᐁᐢᐱᑕᐢᑲᒥᑲᐠ ᐊᐢᑭᐠ ᑭᒋ ᐃᑗᐎᐣ ᐃᓂᓂᐎ ᒥᓂᑯᐎᓯᐎᓇ ᐅᒋ
\par
\vspace*{10pt}
\end{minipage}
}
\medskip
\egroup
\subsection{Miao}

The Pollard script, also known as Pollard Miao (Chinese: 柏格理苗文 Bó Gélǐ Miao-wen) or Miao, is an abugida loosely based on the Latin alphabet and invented by Methodist missionary Sam Pollard. Pollard invented the script for use with A-Hmao, one of several Miao languages. The script underwent a series of revisions until 1936, when a translation of the New Testament was published using it. The introduction of Christian materials in the script that Pollard invented caused a great impact among the Miao. Part of the reason was that they had a legend about how their ancestors had possessed a script but lost it. According to the legend, the script would be brought back some day. When the script was introduced, many Miao came from far away to see and learn it.[1][2]

Pollard credited the basic idea of the script to the Cree syllabics designed by James Evans in 1838–1841, “While working out the problem, we remembered the case of the syllabics used by a Methodist missionary among the Indians of North America, and resolved to do as he had done” (1919:174). He also gave credit to a Chinese pastor, “Stephen Lee assisted me very ably in this matter, and at last we arrived at a system” (1919:174). In listing the phrases he used to describe devising the script, there is clear indication of intellectual work, not revelation: “we looked about”, “resolved to attempt”, “adapting the system”, “solved our problem” (Pollard 1919:174,175).

Changing politics in China led to the use of several competing scripts, most of which were romanizations. The Pollard script remains popular among Hmong in China, although Hmong outside China tend to use one of the alternative scripts. A revision of the script was completed in 1988, which remains in use.

As with most other abugidas, the Pollard letters represent consonants, whereas vowels are indicated by diacritics. Uniquely, however, the position of this diacritic is varied to represent tone. For example, in Western Hmong, placing the vowel diacritic above the consonant letter indicates that the syllable has a high tone, whereas placing it at the bottom right indicates a low tone.

A still experimental font, that supports Graphite technology is \idxfont{Mia Unicode}\footnote{\url{http://phjamr.github.io/miao.html\#intro}}. The font is licenced under the SIL terms and we are using it in the |phd| package as the default font for the Miao script.

\newfontfamily\miao{MiaoUnicode-Regular.ttf}

\begin{scriptexample}[]{Miao}
\unicodetable{miao}{"16F00,"16F10,"16F20,"16F30,"16F40,"16F70,"16F80,"16F90}
\end{scriptexample}

{\miao 𖼴	𖼵	𖼶	𖼷	𖼸	𖼹	𖼺	}

Features for Miao
There are three features currently available for the Miao script:
\bgroup
\miao
Chuxiong ‘wart’ variant
Stylistic alternates for 𖼳 and 𖼴
Aspiration marker always on right
The ‘wart’ (a translated technical term!) is the small circle in characters like 𖼁, 𖼅, and 𖼾. In the Chuxiong orthography, it is rendered not as a circle but as a dot on the right of the letter, as shown in point 5 here (pdf).

Miao Unicode has a feature called “chux” for handling this. In LibreOffice you can use this style by typing “Miao Unicode:chux=1” into the font field.
\section{N'ko}

\newfontfamily\nko{NotoSansNKo-Regular.ttf}

N'Ko {\nko(ߒߞߏ)} is both a script devised by Solomana Kante in 1949 as a writing system for the Manding languages of West Africa, and the name of the literary language itself written in the script. The term N'Ko means ``I say'' in all Manding languages.

The script has a few similarities to the Arabic script, notably its direction (right-to-left) and the connected letters. It obligatorily marks both tone and vowels.


\begin{scriptexample}[]{N'ko}
\unicodetable{nko}{"07C0,"07D0,"07E0,"07F0}
\end{scriptexample}

The N'Ko alphabet is written from right to left, with letters being connected to one another.

The script is principally used in Guinea and Côte d'Ivoire (respectively by Maninka and Dioula-speakers), with an active user community in Mali (by Bambara-speakers). Publications include a translation of the Qur'an, a variety of textbooks on subjects such as physics and geography, poetic and philosophical works, descriptions of traditional medicine, a dictionary, and several local newspapers. It has been classed as the most successful of the West African scripts.[3] The literary language used is intended as a koine blending elements of the principal Manding languages (which are mutually intelligible), but has a particularly strong Maninka flavour.

The Latin script with several extended characters (phonetic additions) is used for all Manding languages to one degree or another for historical reasons and because of its adoption for "official" transcriptions of the languages by various governments. In some cases, such as with Bambara in Mali, promotion of literacy using this orthography has led to a fair degree of literacy in it. Arabic transcription is commonly used for Mandinka in The Gambia and Senegal.


\subsection{Mongolian}
\newfontfamily\mongolian{NotoSansMongolian-Regular.ttf}

The classical Mongolian script (in Mongolian script:{\mongolian ᠮᠣᠩᠭᠣᠯ ᠪᠢᠴᠢᠭ᠌} Mongγol bičig; in Mongolian Cyrillic: Монгол бичиг Mongol bichig), also known as Uyghurjin Mongol bichig, was the first writing system created specifically for the Mongolian language, and was the most successful until the introduction of Cyrillic in 1946. Derived from Uighur, Mongolian is a true alphabet, with separate letters for consonants and vowels. The Mongolian script has been adapted to write languages such as Oirat and Manchu. Alphabets based on this classical vertical script are used in Inner Mongolia and other parts of China to this day to write Mongolian, Sibe and, experimentally, Evenki.

\begin{scriptexample}[]{Mongolian}
\unicodetable{mongolian}{"1820,"1830,"1840,"1850,"1860,"1870,"1880,"1890,"18A0}
\end{scriptexample}



\section{Middle Eastern Scripts}

The scripts in this section have a common origin in the ancient Phoenician alphabet. They include:

\begin{center}
\begin{tabular}{ll}
Hebrew & Samaritan\\
Arabic & Thaana\\
Syriac &\\
\end{tabular}
\end{center}

The Hebrew script is used in Israel and for languages of the Diaspora. The Arabic script is
used to write many languages throughout the Middle East, North Africa, and certain parts
of Asia. The Syriac script is used to write a number of Middle Eastern languages. These
three also function as major liturgical scripts, used worldwide by various religious groups.

The Samaritan script is used in small communities in Israel and the Palestinian Territories
to write the Samaritan Hebrew and Samaritan Aramaic languages. The Thaana script is
used to write Dhivehi, the language of the Republic of Maldives, an island nation in the
middle of the Indian Ocean. 

Text in these scripts is written from right to left. Arabic and Syriac are cursive scripts even when typeset, unlike Hebrew, Samaritan  and Thaana, where letters are unconnected. Most letters in Arabic and Syriac assume different forms depending on their position in a word. Shaping rules are not required for Hebrew because only five letters have position-dependent forms, and these forms are separately encoded.

Historically, Middle Eastern  scripts did not write short vowels. In modern scripts they are represented  by marks positioned above or below a consonantal letter. Vowels and other
marks of pronunciation (“vocalization”) are encoded as combining characters, so support
for vocalized text necessitates use of composed character sequences. Yiddish, Syriac, and
Thaana are normally written with vocalization; Hebrew, Samaritan, and Arabic are usually written unvocalized. 

\section{Hebrew}
\newfontfamily\hebrew{Miriam}
\fontspec{Arial Unicode MS}
To properly typeset Hebrew texts you first need to choose an appropriate font and also set the directionality of the text. This
is done using the etex commands:

\CMDI{\beginL} and \CMDI{\beginR} 

For \XeTeX\ you also need to add near the top of your document |\TeXXeTstate=1|. The package \pkgname{bidi} can be used to set all parameters. Be warned that it redefines almost all of \latexe's commands, so for short mixed texts, I wouldn't recommend its usage. 



The Hebrew alphabet (Hebrew: אָלֶף־בֵּית עִבְרִי[a], alefbet ʿIvri ), known variously by scholars as the Jewish script, square script, block script, is used in the writing of the Hebrew language, as well as other Jewish languages, most notably Yiddish, Ladino, and Judeo-Arabic. There have been two script forms in use; the original old Hebrew script is known as the paleo-Hebrew script (which has been largely preserved, in an altered form, in the Samaritan script), while the present "square" form of the Hebrew alphabet is a stylized form of the Assyrian script. Various "styles" (in current terms, "fonts") of representation of the letters exist. There is also a cursive Hebrew script, which has also varied over time and place. On Windows you can use the \texttt{Miriam} font or \texttt{Arial Unicode MS} or \texttt{Miriam Fixed}.
\medskip

\topline

\bgroup\TeXXeTstate=1
\raggedleft\hebrew{}\beginR

הכתב הכנעני הקדום הלך והתפשט וסימניו היו מוכרים כל כך, עד כי המשתמשים בו התחילו "להתעצל" בהשלמת הציורים, והניחו כי הקורא יבין גם מתוך שרטוטים סכמתיים באיזו אות מדובר. כך, למשל, הפך הראש למשולש עם צוואר; כף היד מלאת האצבעות הפכה לשרטוט דל, ומהדג נותר רק הזנב. כשהעברים אמצו את הכתב הכנעני הם התקשו לזהות חלק מהציורים המקוריים והניחו למשל כי הסימן המתאר את המילה "זהה" הוא כלי נשק; שזנב הדג המשולש הוא דלת, ושדווקא הנחש הוא דג. כך נולדו שמותיהם העבריים של האותיות זי"ן, דל"ת ונו"ן (נון הוא דג, כמו אמנון, שפמנון וכו'). הציורים שהפכו לסימנים התגלגלו לכתבים נוספים, ואפילו ליוונית וללטינית. גם בכתב העברי המודרני ניתן לזהות המשך התפתחותי ברור מן הכתב הכנעני הקדום, והשתמרות שמות האותיות מקלה מאוד על פענוח המקור.


בתקופת בית שני, אומץ האלפבית הארמי לשימוש השפה העברית במקום האלפבית העברי העתיק, כאשר בזה האחרון נעשה שימוש מועט כגון כתיבת השמות הקדושים והטבעת מטבעות. עם הזמן, נעלם גם שימוש זה של הכתב העתיק. האלפבית העברי של ימינו הוא אפוא פיתוח של האלפבית הארמי ולא של הכתב העברי העתיק.	
{}

 לֹ֥א תִשָּׂ֛א

\endR


\egroup
\bottomline
\medskip

To make all paragraphs  RL use the \cmd{\everypar}\footnote{See discussions at \url{http://tex.stackexchange.com/questions/141867/minimal-bidi-for-typesetting-rl-text} and \url{http://www.tug.org/pipermail/xetex/2004-August/000697.html}}. 

\begin{verbatim}
\newbox\mybox \everypar{\setbox\mybox\lastbox\beginR\box\mybox}
\everypar={% at the start of each paragraph, do....
    \setbox0=\lastbox % save the paragraph indent, if any
    \beginR % set R-L direction
    \box0 % then re-insert the indent
	}
\end{verbatim}

The Hebrew alphabet has 22 letters, of which five have different forms when used at the end of a word. Hebrew is written from right to left. Originally, the alphabet was an abjad consisting only of consonants. Like other \textit{abjads}, such as the Arabic alphabet, means were later devised to indicate vowels by separate vowel points, known in Hebrew as niqqud. In rabbinic Hebrew, the letters א ה ו י are also used as matres lectionis to represent vowels. When used to write Yiddish, the writing system is a true alphabet (except for borrowed Hebrew words). In modern usage of the alphabet, as in the case of Yiddish (except that ע replaces ה) and to some extent modern Israeli Hebrew, vowels may be indicated. Today, the trend is toward full spelling with these letters acting as true vowels.

\section{Samaritan}
\newfontfamily\samaritan{NotoSansSamaritan-Regular.ttf}

The Samaritan alphabet is used by the Samaritans for religious writings, including the Samaritan Pentateuch, writings in Samaritan Hebrew, and for commentaries and translations in Samaritan Aramaic and occasionally Arabic.

The Samaritans are, consider themselves to be the descendants of the Northern Tribes of Israel that were not sent into Assyrian captivity, and have continuously resided in the land of Israel.

The Torah Scroll of the Samaritans uses an alphabet that is very different from the one used on Jewish Torah Scrolls. According to the Samaritans themselves and Hebrew scholars, this alphabet is the original "Old Hebrew" alphabet.

Even as far back as 1691, this connection between the Samaritan and the "Old" Hebrew alphabets was made by Henry Dodwell; "[the Samaritans] still preserve [the Pentateuch] in the Old Hebrew characters."

Samaritan is a direct descendant of the Paleo-Hebrew alphabet, which was a variety of the Phoenician alphabet in which large parts of the Hebrew Bible were originally penned. All these scripts are believed to be descendants of the Proto-Sinaitic script. That script was used by the ancient Israelites, both Jews and Samaritans. The better-known "square script" Hebrew alphabet traditionally used by Jews is a stylized version of the Aramaic alphabet which they adopted from the Persian Empire (which in turn adopted it from the Arameans). 

After the fall of the Persian Empire, Judaism used both scripts before settling on the Aramaic form. For a limited time thereafter, the use of paleo-Hebrew (proto-Samaritan) among Jews was retained only to write the Tetragrammaton, but soon that custom was also abandoned.



ShofarRegular StamAshkenazCLM.ttf

\begin{scriptexample}[]{Samaritan}
\bgroup
\TeXXeTstate=1
\unicodetable{samaritan}{"0800,"0810,"0820,"0830}
\egroup
\TeXXeTstate=0
\end{scriptexample}

I battled to get an appropriate font for the Samaritan script and had to use the \idxfont{Noto Sans Samaritan} from Google


^^A\printunicodeblock{./languages/samaritan.txt}{\samaritan}


\url{http://www.ancient-hebrew.org/ahh/ahh.htm#_Toc314842274}



\section{Arabic}

\newfontfamily\arabian{Scheherazade-R.ttf}

The Arabic script is a writing system used for writing several languages of Asia and Africa, such as Arabic, Sorani and Luri Dialects of Kurdish language, Persian, Pashto and Urdu.[1] Even until the 16th century, it was used to write some texts in Spanish.[2] After the Latin script, Chinese characters, and Devanagari, it is the fourth-most widely used writing system in the world.[3]
The Arabic script is written from right to left in a cursive style. In most cases the letters transcribe consonants, or consonants and a few vowels, so most Arabic alphabets are abjads.

The script was first used to write texts in Arabic, most notably the Qurʼān, the holy book of Islam. With the spread of Islam, it came to be used to write languages of many language families, leading to the addition of new letters and other symbols, with some versions, such as Kurdish, Uyghur, and old Bosnian being abugidas or true alphabets. It is also the basis for a rich tradition of Arabic calligraphy.

\begin{verbatim}
\begin{Arabic}
ّ هو إذ الغاية؛ شريف الفوائد، جم المذهب، عزيز فنّ التاريخ فنّ أنّ اعلم
والملوك سيرهم، في والأنبياء أخلاقهم، في الأمم من الماضين أحوال على يوقفنا
ّ أحوال في يرومه لمن ذلك في الإقتداء فائدة تتم حتّى وسياستهم؛ دولهم في
والدنيا. الدين
\end{Arabic}
\end{verbatim}




As of Unicode 7.0, the Arabic script is contained in the following blocks:
Arabic (0600—06FF, 255 characters)
Arabic Supplement (0750—077F, 48 characters)
Arabic Extended-A (08A0—08FF, 39 characters)
Arabic Presentation Forms-A (FB50—FDFF, 608 characters)
Arabic Presentation Forms-B (FE70—FEFF, 140 characters)
Rumi Numeral Symbols (10E60—10E7F, 31 characters)
Arabic Mathematical Alphabetic Symbols (1EE00—1EEFF, 143 characters)[1][2]

The basic Arabic range encodes the standard letters and diacritics, but does not encode contextual forms (U+0621–U+0652 being directly based on ISO 8859-6); and also includes the most common diacritics and Arabic-Indic digits. The Arabic Supplement range encodes letter variants mostly used for writing African (non-Arabic) languages. The Arabic Extended-A range encodes additional Qur'anic annotations and letter variants used for various non-Arabic languages. The Arabic Presentation Forms-A range encodes contextual forms and ligatures of letter variants needed for Persian, Urdu, Sindhi and Central Asian languages. The Arabic Presentation Forms-B range encodes spacing forms of Arabic diacritics, and more contextual letter forms. The presentation forms are present only for compatibility with older standards, and are not currently needed for coding text.[3] 

The Arabic Mathematical Alphabetical Symbols block encodes characters used in Arabic mathematical expressions.

\begin{multicols}{3}
\printunicodeblock{./languages/arabic.txt}{\arabian}
\end{multicols}








\section{Thaana}

\newfontfamily\thaana{MV Boli}
Thaana, Taana or Tāna ({\thaana  ތާނަ}‎ in Tāna script) is the modern writing system of the Maldivian language spoken in the Maldives. Thaana has characteristics of both an abugida (diacritic, vowel-killer strokes) and a true alphabet (all vowels are written), with consonants derived from indigenous and Arabic numerals, and vowels derived from the vowel diacritics of the Arabic abjad. Its orthography is largely phonemic.

The Thaana script first appeared in a Maldivian document towards the beginning of the 18th century in a crude initial form known as Gabulhi Thaana which was written scripta continua. This early script slowly developed, its characters slanting 45 degrees, becoming more graceful and spaces were added between words. 

As time went by it gradually replaced the older Dhives Akuru alphabet. The oldest written sample of the Thaana script is found in the island of Kanditheemu in Northern Miladhunmadulu Atoll. It is inscribed on the door posts of the main Hukuru Miskiy (Friday mosque) of the island and dates back to 1008 AH (AD 1599) and 1020 AH (AD 1611) when the roof of the building were built and the renewed during the reigns of Ibrahim Kalaafaan (Sultan Ibrahim III) and Hussain Faamuladeyri Kilege (Sultan Hussain II) respectively.

\begin{scriptexample}[]{Thaana}
\unicodetable{thaana}{"0780,"0790,"07A0,"07B0}

\hfill Typeset with MV Boli and the command \cmd{\thaana}.
\end{scriptexample}


^^A\printunicodeblock{./languages/thaana.txt}{\thaana}

\subsection{Syriac}

\newfontfamily\syriac{Estrangelo Edessa}

Syriac /ˈsɪriæk/ ({\syriac{ܠܫܢܐ ܣܘܪܝܝܐ}} Leššānā Suryāyā) is a dialect of Middle Aramaic that was once spoken across much of the Fertile Crescent and Eastern Arabia.[1][2][5] Having first appeared as a script in the 1st century AD after being spoken as an unwritten language for five centuries,[6] Classical Syriac became a major literary language throughout the Middle East from the 4th to the 8th centuries,[7] the classical language of Edessa, preserved in a large body of Syriac literature.
It became the vehicle of Syriac Christianity and culture, spreading throughout Asia as far as the Indian Malabar Coast and Eastern China,[8] and was the medium of communication and cultural dissemination for Arabs and, to a lesser extent, Persians. Primarily a Christian medium of expression, Syriac had a fundamental cultural and literary influence on the development of Arabic,[9] which largely replaced it towards the 14th century.[3] Syriac remains the liturgical language of Syriac Christianity.
Syriac is a Middle Aramaic language, and, as such, it is a language of the Northwestern branch of the Semitic family. It is written in the Syriac alphabet, a derivation of the Aramaic alphabet.

\begin{scriptexample}[]{Syriac}
\unicodetable{syriac}{"0700,"0710,"0720,"0730,"0740}
\end{scriptexample}

The Syriac Abbreviation (a type of overline) can be represented with a special control character called the Syriac Abbreviation Mark (U+070F {\syriac \char"070F ܘ}).


\cxset{steward,
  numbering=arabic,
  custom=stewart,
  offsety=0cm,
  image={asia.jpg},
  texti={An introduction to the use of font related commands. The chapter also gives a historical background to font selection using \tex and \latex. },
  textii={In this chapter we discuss keys that are available through the \texttt{phd} package and give a background as to how fonts are used
in \latex.
 },
 pagestyle = empty
}

\arial


\chapter{South Asian Scripts}

The scripts of South Asia share so many characteristics that a side by side comparison of a few often reveal structural similarities even in the 
modern letterforms.
\medskip

\begin{center}
\begin{tabular}{lll}
Devanagari. &Gujarati &Telugu\\
Bengali   &Oriya &Kannada\\
Gurmukhi &Tamil  &Malayalam\\
Sinhala &Kaithi  &Meetei Mayek\\
Tibetan &Saurashtra &Ol Chiki.\\
Lepcha  &Sharada &Sora Sompeng\\
Phags-pa &Takri &Kharoshthi\\
Limbu &Chakma & Brahmi\\
Syloti Nagri & &\\
\end{tabular}
\end{center}

The sections that follow describe the scripts briefly and the |phd| settings
to activate the relevant commands and load appropriate fonts. 

\section{Devanagari}
\parindent1em

Devanagari is part of the Brahmic family of scripts of India, Nepal, Tibet, and South-East Asia.[2] It is a descendant of the Gupta script, along with Siddham and Sharada.[2] Eastern variants of Gupta called nāgarī are first attested from the 7th century CE; from c. 1200 CE these gradually replaced Siddham, which survived as a vehicle for Tantric Buddhism in East Asia, and Sharada, which remained in parallel use in Kashmir. An early version of Devanagari is visible in the Kutila inscription of Bareilly dated to Vikram Samvat 1049 (i.e. 992 CE), which demonstrates the emergence of the horizontal bar to group letters belonging to a word.[3]

Sanskrit nāgarī is the feminine of nāgara "relating or belonging to a town or city". It is feminine from its original phrasing with lipi ("script") as nāgarī lipi "script relating to a city", that is, probably from its having originated in some city.[4]

The use of the name devanāgarī is relatively recent, and the older term nāgarī is still common.[2] The rapid spread of the term devanāgarī may be related to the almost exclusive use of this script to publish Sanskrit texts in print since the 1870s.[2]

On Windows use \texttt{Arial Unicode MS}. 
\medskip

\newfontfamily\devanagari[Script=Devanagari,Scale=1.5]{Arial Unicode MS}

\begin{scriptexample}[]{Devanagari}
{\begin{center}\parindent0pt\devanagari

ंःअआइईउऊऋऌऍऎएऐऑऒओऔऔँ \par 

ी	ु	ू	ृ	ॄ	ॅ	ॆ	े	ै	ॉ	ॊ	ो	ौ	्	\par

\bigskip		
\begin{tabular}{lll lll lll l}
०	&१	&२	&३	&४	&५	&६	&७	&८	&९\\
0	&1	&2	&3	&4	&5	&6	&7	&8	&9\\
\end{tabular}
\end{center}	
}
\end{scriptexample}


On Linux \texttt{Lohit} is a font family designed to cover Indic scripts and released by Red Hat. The Lohit fonts currently cover 11 languages: Assamese, Bengali, Gujarati, Hindi, Kannada, Malayalam, Marathi, Oriya, Punjabi, Tamil, Telugu.[1] The fonts were supplied by Modular Infotech and licensed under the GPL. In September 2011, they were retroactively relicensed under the OFL.[2] The Lohit fonts are used as web fonts by some Wikimedia Foundation sites, like Wikipedia, since March 2012.The font currently support 21 Indian languages. 

\newfontfamily\devanagarilohit[Script=Devanagari,Scale=1.5]{Lohit-Devanagari.ttf}

\begin{scriptexample}[]{Devanagari}
\begin{center}\parindent0pt\devanagarilohit

ंःअआइईउऊऋऌऍऎएऐऑऒओऔऔँ \par 

ी	ु	ू	ृ	ॄ	ॅ	ॆ	े	ै	ॉ	ॊ	ो	ौ	्	\par

\bigskip		
\begin{tabular}{lll lll lll l}
०	&१	&२	&३	&४	&५	&६	&७	&८	&९\\
0	&1	&2	&3	&4	&5	&6	&7	&8	&9\\
\end{tabular}
\end{center}
\end{scriptexample}

\subsubsection{Punctuation} 
The end of a sentence or half-verse may be marked with a dot known as a pūrna virām or a vertical line danda: \textbar. The end of a full verse may be marked with two vertical lines: \textbar\textbar. A comma, or alpa virām, is used to denote a natural pause in speech. With expansion of English speakers in India, the full stop is also sometimes used.

\subsection{LaTeX support}

\latex2e support can be found in the \pkgname{sanskrit}. The package contains the font files and pre-processor for printing Sanskrit
text in both devanāgarī and transliterated Roman with diacritics. Another package that can be used with \XeTeX\ is support \pkgname{devnag}.  This was originally developed by Frans Velthuis for the University of Groningen, The Netherlands, and it was the first system to provide
support for the script for \tex. The package was  extended by Anshuman Pandey. The package provides both fonts as well as tranliteration macros.


\subsection{Gujarati}


Gujarati has its own writing system, distinct but related to several other Indian languages' writing systems, such as the one used to write Hindi. Strictly speaking, the Gujarati writing system is what is called an \emph{abugida} (and not an \textit{alphabet}), because the consonant characters all contain an inherent vowel, and other vowels are written as accents added on to the consonant characters. There are also symbols for stand-alone vowels.

The Gujarati script ({\gujarati{ગુજરાતી લિપિ }} Gujǎrātī Lipi), which like all Nāgarī writing systems is strictly speaking an abugida rather than an alphabet, is used to write the Gujarati and Kutchi languages. It is a variant of Devanāgarī script differentiated by the loss of the characteristic horizontal line running above the letters and by a small number of modifications in the remaining characters.
With a few additional characters, added for this purpose, the Gujarati script is also often used to write Sanskrit and Hindi.
Gujarati numerical digits are also different from their Devanagari counterparts.
\medskip

\bgroup
\newfontfamily\gujaratilohit[Script=Gujarati,Scale=1.5]{Lohit-Gujarati.ttf}
\gujarati

\centering

English/Hindi/Gujarati Alphabets

\begin{tabular}{lllllllllllllllllllll}
A &B &bh &C &ch &chh &D &dh &E &F &G &gh &H &I &J &K &kh &L &M &N &O\\

अ &ब &भ &क &च &छ &ड/द &ध/ढ़ &इ &फ &ग &घ &ह &ई &ज &क &ख &ल &म &न/ण &ऑ\\

અ &બ &ભ &ક &ચ &છ &ડ/દ &ધ /ઢ &ઇ &ફ &ગ &ઘ &હ &ઈ &જ &ક &ખ &લ &મ &ન/ણ &ઓ\\

\end{tabular}
\egroup

\medskip

Gujarati has its own set of numeric signs (placed alongside their Hindu-Arabic [or Indo-Arabic] counterparts in the tables below), they are employed in much the same way as English;  that is to say, they are put together in the same manner in order to express larger numbers. It is quite possible to simply substitute the Gujarati numerals for the Hindu-Arabic ones.

The Gujarati words for 1-10 are as follows:
\medskip

\bgroup
\begin{center}
\gujarati
\begin{tabular}{ccl}
Arabic & Gujarati &Name\\
Numeral &Numeral  &\\
0	&૦	&mīṇḍuṃ or shunya\\
1	&૧	&ekaṛo or ek\\
2	&૨	&bagaṛo or bay\\
3	&૩	&tragaṛo or tran\\
4	&૪	&chogaṛo or chaar\\
5	&૫	&pāchaṛo or paanch\\
6	&૬	&chagaṛo or chah\\
7	&૭	&sātaṛo or sāt\\
8	&૮	&āṭhaṛo or āanth\\
9	&૯	&navaṛo or nav\\
10 &૧૦ &દસ das\\

\end{tabular}
\end{center}
\egroup

\subsection{Bengali}

There are two Windows fonts that can be used with Windows \textit{Shonar Bangla} and \textit{Vrinda}. For open source fonts one can use, \textit{code2000}.
\bigskip

\bgroup
\newfontfamily\bengali[Script=Bengali,Scale=4]{Shonar Bangla}


\bengali
\centering

  অ  আ ই  ঈ  উ  ঊ  ঋ  এ  ঐ\par

\fontspec[Script=Bengali,Scale=3.2]{Vrinda}

\centering

  অ  আ ই  ঈ  উ  ঊ  ঋ  এ  ঐ\par


\fontspec[Script=Bengali,Scale=3.2]{code2000.ttf}

\centering

  অ  আ ই  ঈ  উ  ঊ  ঋ  এ  ঐ\par

\captionof{table}{The consonant{\protect\bengal{} ক (kô)} along with the diacritic form of the vowels {\protect\bengal{} অ, আ, ই, ঈ, উ, ঊ, ঋ, এ, ঐ, ও and ঔ} \textit{from Wikipedia}.}
\egroup

\subsection{Saurashtra}

\newfontfamily\saurashtra{code2000.ttf}

Saurashtra or Sourashtra or {\saurashtra ꢱꣃꢬꢵꢰ꣄ꢜ꣄ꢬꢵ} or Palkar or Patkar (Sanskrit: सौराष्ट्र, Tamil: சௌராட்டிரம்) is an Indo-Aryan language[3] spoken by the Saurashtrian community native to Gujarat, who migrated and settled in Southern India. Madurai in Tamil Nadu has the highest number of people belonging to this community and also remains as their cultural center.

The language is largely only in spoken form even though the language has its own script. The lack of schools teaching Saurashtra script and the language is often cited as a reason for the very few number of people who actually know to read and write in Saurashtra script. Latin, Devanagari or Tamil script is used as alternative for Saurashtra Script by many Saurashtrians.

Census of India places the language under Gujarati. Official figures show the number of speakers as 185,420 (2001 census).[4]



\begin{scriptexample}[]{Saurashtra}
\bgroup
\saurashtra

ꢮꢶꢯ꣄ꢮ ꢱꣃꢬꢵꢰ꣄ꢜ꣄ꢬꢪ꣄ ꢦꢡ꣄ꢬꢶꢒꢾ ꢱꢵꢡ꣄ꢡꢒꢸ ꢂꢮꢬꢾ
ꢮꣁꢭꢱ꣄ꢢꢵꢥꢪꢸꢒ꣄(ꣀꢵꢮꢾꢔꢹ ꢂꢮ꣄ꢬꢶꢫꣁ


\arial

Text: Vishwa Sourashtram \url{http://www.sourashtra.info/ghEr.htm}
\egroup
\end{scriptexample}

\subsection{Ol Chiki script}

The Ol Chiki script, also known as Ol Cemetʼ (Santali: ol 'writing', cemet' 'learning'), Ol Ciki, Ol, and sometimes as the Santali alphabet, was created in 1925 by Raghunath Murmu for the Santali language.

Previously, Santali had been written with the Latin alphabet. But because Santali is not an Indo-Aryan language (like most other languages in the south of India), Indic scripts did not have letters for all of Santali's phonemes, especially its stop consonants and vowels, which made writing the language accurately in an unmodified Indic script difficult. The detailed analysis was given by Dr. Byomkes Chakrabarti in his 'Comparative Study of Santali and Bengali'. Missionaries (first of all Paul Olaf Bodding, a Norwegian) brought the Latin script, which is better at representing Santali stops, phonemes and nasal sounds with the use of diacritical marks and accents. Unlike most Indic scripts, which are derived from Brahmi, Ol Chiki is not an abugida, with vowels given equal representation with consonants. Additionally, it was designed specifically for the language, but one letter could not be assigned to each phoneme because the sixth vowel in Ol Chiki is still problematic.
Ol Chiki has 30 letters, the forms of which are intended to evoke natural shapes. Linguist Norman Zide said "The shapes of the letters are not arbitrary, but reflect the names for the letters, which are words, usually the names of objects or actions representing conventionalized form in the pictorial shape of the characters."[1] It is written from left to right.

\newfontfamily\olchiki{code2000.ttf}

\begin{scriptexample}[]{olchiki}
\bgroup
\olchiki
\obeylines

U+1C5x 	᱐	᱑	᱒	᱓	᱔	᱕	᱖	᱗	᱘	᱙	ᱚ	ᱛ	ᱜ	ᱝ	ᱞ	ᱟ
U+1C6x	   ᱠ	ᱡ	ᱢ	ᱣ	ᱤ	ᱥ	ᱦ	ᱧ	ᱨ	ᱩ	ᱪ	ᱫ	ᱬ	ᱭ	ᱮ	ᱯ
U+1C7x  	ᱰ	ᱱ	ᱲ	ᱳ	ᱴ	ᱵ	ᱶ	ᱷ	ᱸ	ᱹ	ᱺ	ᱻ	ᱼ	ᱽ	᱾	᱿
\egroup
\end{scriptexample}

\subsection{Lepcha}
\newfontfamily\lepcha{Mingzat-R.ttf}

The Lepcha script, or Róng script is an abugida used by the Lepcha people to write the Lepcha language. Unusually for an abugida, syllable-final consonants are written as diacritics.

The Mingzat font is still under development by SIL so I am not too sure if the rendering is correct\footnote{\url{http://scripts.sil.org/cms/scripts/page.php?site_id=nrsi&id=Mingzat}}.

\begin{scriptexample}[]{Lepcha}
\bgroup
\lepcha
\obeylines
 	    0	1	2	3	4	5	6	7	8	9	A	B	C	D	E	F
U+1C0x	 ᰀ	ᰁ	ᰂ	ᰃ	ᰄ	ᰅ	ᰆ	ᰇ	ᰈ	ᰉ	ᰊ	ᰋ	ᰌ	ᰍ	ᰎ	ᰏ
U+1C1x	 ᰐ	ᰑ	ᰒ	ᰓ	ᰔ	ᰕ	ᰖ	ᰗ	ᰘ	ᰙ	ᰚ	ᰛ	ᰜ	ᰝ	ᰞ	ᰟ
U+1C2x	 ᰠ	ᰡ	ᰢ	ᰣ	ᰤ	ᰥ	ᰦ	ᰧ	ᰨ	ᰩ	ᰪ	ᰫ	ᰬ	ᰭ	ᰮ	ᰯ
U+1C3x	 ᰰ	ᰱ	ᰲ	ᰳ	ᰴ	ᰵ	ᰶ	᰷	x	x	x	᰻	᰼	᰽	᰾	᰿
U+1C4x	 ᱀	᱁	᱂	᱃	᱄	᱅	᱆	᱇	᱈	᱉	x	x	x	ᱍ	ᱎ	ᱏ

\egroup
\end{scriptexample}

\subsection{Sharada}

The Śāradā, or Sharada, script (शारदा) is an abugida writing system of the Brahmic family of scripts, developed around the 8th century. It was used for writing Sanskrit and Kashmiri. The Gurmukhī script was developed from Śāradā. Originally more widespread, its use became later restricted to Kashmir, and it is now rarely used except by the Kashmiri Pandit community for ceremonial purposes. Śāradā is another name for Saraswati, the goddess of learning.
Śāradā script was added to the Unicode Standard in January, 2012 with the release of version 6.1.

The Unicode block for Śāradā script, called Sharada, is U+11180–U+111DF: Unable to locate font in unicode.


\subsection{Sora Sompeng}

Sorang Sompeng script is used to write in Sora, a Munda language with 300,000 speakers in India. The script was created by Mangei Gomango in 1936 and is used in religious contexts.[1] He was familiar with Oriya, Telugu and English, so the parent systems of the script are Brahmi and Latin.[2]
The Sora language is also written in the Latin alphabet and the Telugu script.

Sorang Sompeng script was added to the Unicode Standard in January, 2012 with the release of version 6.1. Nirmala UI.ttf (Windows 8.1)



\unicodetable{arial}{"110D0,"110E0,"110F0}
 	
This did not work with Windows 7, and the experiment failed. 

\subsection{Phags-pa}

The 'Phags-pa script,[1], (Mongolian: дөрвөлжин үсэг "Square script") was an alphabet designed by the Tibetan monk and vice-king Drogön Chögyal Phagpa for the Mongol Yuan emperor Kublai Khan as a unified script for the literary languages of the Yuan. Widespread use was limited to about a hundred years during the Yuan Dynasty, and it fell out of use with the advent of the Ming dynasty. The documentation of its use provides clues about the changes in the varieties of Chinese, the Tibetic languages, Mongolian and other neighboring languages during the Yuan era.

\newfontfamily\phagspa{code2000.ttf}

\begin{scriptexample}[]{Phags-pa}
\bgroup
\obeylines
\phagspa

 	0	1	2	3	4	5	6	7	8	9	A	B	C	D	E	F
U+A84x	ꡀ	ꡁ	ꡂ	ꡃ	ꡄ	ꡅ	ꡆ	ꡇ	ꡈ	ꡉ	ꡊ	ꡋ	ꡌ	ꡍ	ꡎ	ꡏ
U+A85x	ꡐ	ꡑ	ꡒ	ꡓ	ꡔ	ꡕ	ꡖ	ꡗ	ꡘ	ꡙ	ꡚ	ꡛ	ꡜ	ꡝ	ꡞ	ꡟ
U+A86x	ꡠ	ꡡ	ꡢ	ꡣ	ꡤ	ꡥ	ꡦ	ꡧ	ꡨ	ꡩ	ꡪ	ꡫ	ꡬ	ꡭ	ꡮ	ꡯ
U+A87x	ꡰ	ꡱ	ꡲ	ꡳ	꡴	꡵	꡶	


ꡏꡟ ꡋꡞ ꡏꡟ ꡋꡞ ꡏ ꡜꡖ ꡏꡟ ꡋꡞ ꡓꡞ ꡏꡟ
ꡈꡋ ꡋꡋ ꡓꡘ ꡈ ꡭ ꡏ ꡏ ꡝ ꡭꡟꡘ ꡓꡋ ꡮꡟꡊ
\egroup
\bgroup
\raggedright

\setcounter{glyphcount}{"A840}

\topline
\phagspa
\newcount\n
\n="A840

\def\htable{^^A
  \def\fm##1{\makebox[2em]##1}^^A
  U+A840\fm 0\fm1\fm2\fm3\fm4\fm5\fm 6\fm 7\fm 8\fm	9\fm A\fm B\fm C\fm D\fm E\fm F}

\htable\par
U+A840^^A 
\loop^^A
  \makebox[2em]{\char\n }^^A   
   \advance\n by1 ^^A
   \ifnum\n<"A850^^A
\repeat
\par U+A850^^A
\loop^^A
  \makebox[2em]{\char\n }^^A   
   \advance\n by1 ^^A
  \ifnum\n<"A860^^A
\repeat
\par U+A860^^A
\loop^^A
  \makebox[2em]{\char\n }^^A   
   \advance\n by1 ^^A
  \ifnum\n<"A870^^A
\repeat
\par U+A870^^A
\loop^^A
  \makebox[2em]{\char\n }^^A   
   \advance\n by1 ^^A
  \ifnum\n<"A878^^A
\repeat

\bottomline

\arial
\hfill Typeset with \texttt{code2000.ttf} and \cmd{\phagspa}

Text: \href{http://babelstone.blogspot.com/2006/12/phags-pa-fonts-1-babelstone-phags-pa.html}{babelstone}
\egroup
\end{scriptexample}

Phags-pa is a historical script related to Tibetan that was created as the national script of
the Mongol empire. Even though Phags-pa was used mostly in Eastern and Central Asia for
writing text in the Mongolian and Chinese languages, it is discussed in this chapter because
of its close historical connection to the Tibetan script. The script has very limited modern use. It bears similarity to Tibetan and has no case distinctions. It is written vertically in columns running for left to right, like Mongolian. Units are often composed of several syllables and sometimes are separated by whitespace.


\subsection{Syloti Nagri}
\index{languages>Sylheti Nagari}
Sylheti Nagari or Syloti Nagri (Silôṭi Nagôri) is the original script used for writing the Sylheti language. It is an almost extinct script, this is because the Sylheti Language itself was reduced to only dialect status after Bangladesh gained independence and because it did not make sense for a dialect to have its own script, its use was heavily discouraged. The government of the newly formed Bangladesh did so to promote a greater "Bengali" identity. This led to the informal adoption of the Eastern Nagari script also used for Bengali and Assamese. It is also known as Jalalabadi Nagri, Mosolmani Nagri, Ful Nagri etc.

\newfontfamily\syloti{NotoSansSylotiNagri-Regular.ttf}
\newfontfamily\damase{damase_v.2.ttf}
\bgroup
\damase
\obeylines
	0	1	2	3	4	5	6	7	8	9	A	B	C	D	E	F
U+A80x	ꠀ	ꠁ	ꠂ	ꠃ	ꠄ	ꠅ	꠆	ꠇ	ꠈ	ꠉ	ꠊ	ꠋ	ꠌ	ꠍ	ꠎ	ꠏ
U+A81x	ꠐ	ꠑ	ꠒ	ꠓ	ꠔ	ꠕ	ꠖ	ꠗ	ꠘ	ꠙ	ꠚ	ꠛ	ꠜ	ꠝ	ꠞ	ꠟ
U+A82x	ꠠ	ꠡ	ꠢ	ꠣ	ꠤ	ꠥ	ꠦ	ꠧ	꠨	꠩	꠪	꠫
\egroup

\subsection{Chakma}

\newfontfamily\chakma{RibengUni.ttf}

\bgroup
\chakma
𑄇𑄳𑄇 Kkā = 𑄇 Kā + 𑄳 VIRAMA + 𑄇 Kā
𑄇𑄳𑄑 Ktā = 𑄇 Kā + 𑄳 VIRAMA + 𑄑 Tā
𑄇𑄳𑄖 Ktā = 𑄇 Kā + 𑄳 VIRAMA + 𑄖 Tā
𑄇𑄳𑄟 Kmā = 𑄇 Kā + 𑄳 VIRAMA + 𑄟 Mā
𑄇𑄳𑄌 Kcā = 𑄇 Kā + 𑄳 VIRAMA + 𑄌 Cā
𑄋𑄳𑄇 ńkā = 𑄋 ńā + 𑄳 VIRAMA + 𑄇 Kā
𑄋𑄳𑄉 ńkā = 𑄋 ńā + 𑄳 VIRAMA + 𑄉 Gā
𑄌𑄳𑄌 ccā = 𑄌 cā + 𑄳 VIRAMA + 𑄌 Cā

\egroup

\subsection{Limbu}

The Limbu script is used to write the Limbu language. The Limbu script is an abugida derived from the Tibetan script. Limbu is a Tibeto-Burman language spoken mainly in Nepal,[3] significant communities in Bhutan, Sikkim, Darjeeling district, India by the Limbu community. Virtually all Limbus are bilingual in Nepali.

\newfontfamily\limbu{code2000.ttf}
\bgroup
\obeylines
\limbu
0	1	2	3	4	5	6	7	8	9	A	B	C	D	E	F
U+190x	ᤀ	ᤁ	ᤂ	ᤃ	ᤄ	ᤅ	ᤆ	ᤇ	ᤈ	ᤉ	ᤊ	ᤋ	ᤌ	ᤍ	ᤎ	ᤏ
U+191x	ᤐ	ᤑ	ᤒ	ᤓ	ᤔ	ᤕ	ᤖ	ᤗ	ᤘ	ᤙ	ᤚ	ᤛ	ᤜ	ᤝ	ᤞ	
U+192x	ᤠ	ᤡ	ᤢ	ᤣ	ᤤ	ᤥ	ᤦ	ᤧ	ᤨ	ᤩ	ᤪ	ᤫ				
U+193x	ᤰ	ᤱ	ᤲ	ᤳ	ᤴ	ᤵ	ᤶ	ᤷ	ᤸ	᤹	᤺	᤻				
U+194x	᥀				᥄	᥅	᥆	᥇	᥈	᥉	᥊	᥋	᥌	᥍	᥎	᥏
\egroup

\subsection{Brahmi}



Brāhmī is the modern name given to one of the oldest writing systems used in the Indian subcontinent and in Central Asia during the final centuries BCE and the early centuries CE. Like its contemporary, Kharoṣṭhī, which was used in what is now Afghanistan and Western Pakistan, Brahmi (native to north and central India) was an \emph{abugida}.

The best-known Brahmi inscriptions are the rock-cut edicts of Ashoka in north-central India, dated to 250–232 BCE. The script was deciphered in 1837 by James Prinsep, an archaeologist, philologist, and official of the East India Company.[1] The origin of the script is still much debated, with current Western academic opinion generally agreeing (with some exceptions) that Brahmi was derived from or at least influenced by one or more contemporary Semitic scripts, but a current of opinion in India favors the idea that it is connected to the much older and as-yet undeciphered Indus script

\subsection{Unicode [U+11000-U+1107F]}


\newfontfamily\brahmi{code2000.ttf}

\begin{scriptexample}[]{Brahmi}
\bgroup
\raggedleft
\brahmi

         
   

\arial
\hfill Text: Asokan Edict typeset with \texttt{NotoSansBrahmi-Regular.ttf} 
\egroup
\end{scriptexample}


\begin{description}
\item[Abkhazia] (Abkhaz: Аҧсны́ Apsny [apʰsˈnɨ]; Georgian: აფხაზეთი Apkhazeti; Russian: Абхазия Abkhaziya) is a disputed territory and partially recognised state controlled by a separatist government on the eastern coast of the Black Sea and the south-western flank of the Caucasus.

\item[Achinese] Acehnese language (Achinese) is a Malayo-Polynesian language spoken by Acehnese people natively in Aceh, Sumatra, Indonesia. This language is also spoken in some parts in Malaysia by Acehnese descendents there, such as in Yan, Kedah.

Formerly, Acehnese language was written in Arabic script called Jawoë or Jawi in Malay language. The script is less common nowadays.[citation needed] Now, Acehnese language is written in Latin script since colonization by the Dutch; with the addition of supplementary letters. The additional letters are é, è, ë, ö and ô.[8] The sound ɨ is represented by 'eu' and the sound ʌ is represented by 'ö' respectively. The letter 'ë' is used to represent the schwa sound which forms the second part in the diphthongs.

\item[Adyghe] Adyghe (/ˈædɨɡeɪ/ or /ˌɑːdɨˈɡeɪ/;[3] Adyghe: Адыгэбзэ adyghabze), also known as West Circassian (КӀахыбзэ), is one of the two official languages of the Republic of Adygea in the Russian Federation, the other being Russian. It is spoken by various tribes of the Adyghe people: Abzekh,[4] Adamey, Bzhedug;[5] Hatuqwai, Temirgoy, Mamkhegh; Natekuay, Shapsug;[6] Zhaney, Yegerikuay, each with its own dialect. The language is referred to by its speakers as Adygebze or Adəgăbză, and alternatively spelled in English as Adygean, Adygeyan or Adygei. The literary language is based on the Temirgoy dialect.
There are apparently around 128,000 speakers of the language on the native territory in Russia, almost all of them native speakers. In the whole world, some 300,000 speak the language. The largest Adyghe-speaking community is in Turkey, spoken by the post Russian–Circassian War (circa 1763–1864) diaspora; in addition to that, the Adyghe language is spoken by the Cherkesogai in Krasnodar Krai.

Ублапӏэм ыдэжь Гущыӏэр щыӏагъ. Ар Тхьэм ыдэжь щыӏагъ, а Гущыӏэри Тхьэу арыгъэ. Ублапӏэм щегъэжьагъэу а Гущыӏэр Тхьэм ыдэжь щыӏагъ. Тхьэм а Гущыӏэм зэкӏэри къыригъэгъэхъугъ. Тхьэм къыгъэхъугъэ пстэуми ащыщэу а Гущыӏэм къыримыгъгъэхъугъэ зи щыӏэп. Мыкӏодыжьын щыӏэныгъэ а Гущыӏэм хэлъыгъ, а щыӏэныгъэри цӏыфхэм нэфынэ афэхъугъ. Нэфынэр шӏункӏыгъэм щэнэфы, шӏункӏыгъэри нэфынэм текӏуагъэп.

Translation: In the beginning was the Word, and the Word was with God, and the Word was God. The same was in the beginning with God. All things were made by him, and without him was not any thing made that was made. In him was life, and the life was the light of men. And the light shineth in darkness, and the darkness comprehended it not.

\item[Albanian]Albanian (shqip [ʃcip] or gjuha shqipe [ˈɟuha ˈʃcipɛ], meaning Albanian language) is an Indo-European language spoken by approximately 7.6 million people,[3] primarily in Albania, Kosovo, the Republic of Macedonia and Greece, but also in other areas of Southeastern Europe in which there is an Albanian population, including Montenegro and Serbia (Presevo Valley). Centuries-old communities speaking Albanian-based dialects can be found scattered in Greece, southern Italy,[4] Sicily, and Ukraine.[5] As a result of a modern diaspora, there are also Albanian speakers elsewhere in those countries and in other parts of the world, including Scandinavia, Switzerland, Germany, Austria and Hungary, United Kingdom, Turkey, Australia, New Zealand, Netherlands, Singapore, Brazil, Canada, and the United States.

Letter:	A	B	C	Ç	D	Dh	E	Ë	F	G	Gj	H	I	J	K	L	Ll	M	N	Nj	O	P	Q	R	Rr	S	Sh	T	Th	U	V	X	Xh	Y	Z	Zh\\
IPA value:	a	b	t͡s	t͡ʃ	d	ð	e	ə	f	ɡ	ɟ	h	i	j	k	l	ɫ	m	n	ɲ	o	p	c	ɾ	r	s	ʃ	t	θ	u	v	d͡z	d͡ʒ	y	z	ʒ\\

\end{description}

\begin{multicols}{5}
\raggedright
Abkhazian\\
Abron\\
Achinese\\
Acoli\\
Adyghe\\
Afar\\
Afrikaans\\
Aghem\\
Akan\\
Akoose\\
Albanian\\
Albay\\
Bikol\\
Amo\\
Asturian\\
Asu\\
Atikamekw
Atsam
Avaric
Aymara
Azerbaijani (Cyrillic script)\\
Azerbaijani (Latin script)\\
Bafia\\
Bafut\\
Balinese\\
Balkan Gagauz Turkish
Bambara (Latin script)
Banjar
Baoulé
Basaa
Bashkir
Basque
Batak
Batak Toba
Belarusian
Bemba
Bena
Betawi
Bikol
Bini
Bislama
Bomu
Bosnian (Cyrillic script)
Bosnian (Latin script)
Breton
Bube
Buginese
Buhid
Bulgarian
Bulu
Buriat
Bushi
Catalan
Cebaara Senoufo
Cebuano
Central Atlas Tamazight (Latin script)
Central-Eastern Niger Fulfulde
Central Huasteca Nahuatl
Central Mazahua
Chamorro
Chechen
Chiga
Chipewyan
Church Slavic
Chuukese
Chuvash
Colognian
Congo Swahili
Cornish
Corsican
Croatian
Czech
Dan
Danish
Dargwa
Dogrib
Duala
Dutch
Dyula
Eastern Huasteca Nahuatl
East Futuna
Efik
Embu
English
Erzya
Esperanto
Estonian
Ewe
Ewondo
Fang
Faroese
Fijian
Filipino
Finnish
Fon
French
Friulian
Fulah
Ga
Gagauz
Galician
Ganda
German
Ghomala
Gilbertese
Gorontalo
Greek
Gronings
Guajajára
Guarani
Guianese Creole French
Gusii
Gwichʼin
Haitian
Hanunoo
Hausa (Latin script)
Hawaiian
Hiligaynon
Hiri Motu
Hungarian
Ibibio
Icelandic
Igbo
Iloko
Inari Sami
Indonesian
Ingush
Interlingua
Inuinnaqtun
Inuktitut (Latin script)
Inupiaq
Irish
Italian
Javanese
Jenaama Bozo
Jju
Jola-Fonyi
Kabardian
Kabuverdianu
Kabyle
Kaingang
Kako
Kalaallisut
Kalanga
Kalenjin
Kalo Finnish Romani
Kamba
Karachay-Balkar
Kara-Kalpak
Karelian
Kashubian

Kazakh (Cyrillic script)

Kerinci
Khasi
Kʼicheʼ
Kikuyu
Kimbundu
Kinyarwanda
Kita Maninkakan
Kom
Komering
Komi
Komi-Permyak
Kongo
Koro
Koro Wachi
Kosraean
Koyraboro Senni
Koyra Chiini
Kpelle
Krio
Kuanyama
Kumyk
Kurdish (Latin script)

Kwasio

Kyrgyz (Cyrillic script)

Kyrgyz (Latin script)

Lak\\
Lakota\\
Lampung Api\\
Langi\\
Lango\\
Latin\\
Latvian\\
Lezghian\\
Limburgish\\
Lingala\\
Lithuanian\\
Lombard
Lomwe
Lower Sorbian
Low German
Lozi
Luba-Katanga
Luba-Lulua
Lule Sami
Luo
Luxembourgish
Luyia
Maasina Fulfulde
Macedonian
Machame
Madurese
Mafa
Maguindanaon
Makasar
Makhu
Makhuwa-Meetto
Makonde
Malagasy
Malay (Latin script)
Maltese
Mandar
Mandingo (Latin script)
Manx
Manyika
Maori
Mapuche
Mari
Marshallese
Masaaba
Masai
Mbunga
Medumba
Mende
Meru
Meta’
Minangkabau
Mohawk
Moksha
Mongo
Mongolian (Cyrillic script)
Montagnais
Morisyen
Mossi
Mundang
Nama
Nauru
Navajo
Naxi
Ndau
Ndonga
Neapolitan
Negeri Sembilan Malay
Ngaju
Ngiemboon
Ngomba
Nigerian Fulfulde
Nigerian Pidgin
Niuean
Northern Sami
Northern Sotho
North Ndebele
North Slavey
Norwegian Bokmål
Norwegian Nynorsk
Nuer
Nyamwezi
Nyanja
Nyankole
Occitan
Oromo
Ossetic
Palauan
Pampanga
Pangasinan
Papiamento
Pohnpeian
Pökoot
Polish
Portuguese
Punu
Quechua
Rajasthani
Rejang
Réunion Creole French
Riang
Rinconada Bikol
Romanian
Romansh
Rombo
Ronga
Rundi
Russian
Rusyn
Rwa
Safaliba
Saho
Sakha
Samburu
Samoan
Sangir
Sango
Sangu
Santali
Sasak
Scots
Scottish Gaelic
Sena
Serbian (Cyrillic script)
Serbian (Latin script)
Serer
Seselwa Creole French
Shambala
Shona
Sicilian
Sidamo
Sinte Romani
Skolt Sami
Slave
Slovak
Slovenian
Soga
Somali
Soninke
Southern Altai
Southern Sami
Southern Sotho
South Ndebele
Spanish
Sranan Tongo
Sukuma
Sundanese
Susu
Swahili
Swati
Swedish
Swiss German
Tachelhit (Latin script)
Tae’
Tagbanwa
Tahitian
Taita
Tajik (Cyrillic script)
Tamashek
Taroko
Tasawaq
Tatar
Tausug
Tavringer Romani
Teso
Tetum
Timne
Tiv
Tokelau
Tok Pisin
Tolaki
Tomo Kan Dogon
Tongan
Tooro
Tornedalen Finnish
Tsonga
Tswana
Tumbuka
Turkish
Turkmen (Latin script)
Tuvalu
Tuvinian
Tyap
Uab Meto
Udmurt
Ukrainian
Ulithian
Umbundu
Unknown Language
Uyghur (Cyrillic script)
Uzbek (Cyrillic script)
Uzbek (Latin script)
Vai (Latin script)
Venda
Vietnamese
Virgin Islands Creole English
Vunjo
Wallisian
Walloon
Walser
Waray
Welsh
Western Frisian
Western Huasteca Nahuatl
Western Mari
Wolof
Xaasongaxango
Xavánte
Xhosa
Yangben
Yao
Yapese
Yemba
Yoruba
Yucatec Maya
Zarma
Zaza
Zeelandic
Zhuang
Zulu
\end{multicols}





\end{document}



    \cxset{steward,
  numbering=arabic,
  custom=stewart,
  offsety=0cm,
  image={fellah-woman.jpg},
  texti={An introduction to the use of font related commands. The chapter also gives a historical background to font selection using \tex and \latex. },
  textii={In this chapter we discuss keys that are available through the \texttt{phd} package and give a background as to how fonts are used
in \latex.
 },
 pagestyle = fancy
}

\pgfpagesuselayout{2 on 1}[a3paper,landscape,border shrink=0mm]

\chapter{Ancient and Historic Scripts}

Writing was perhaps the most important human invention. \tex authors and developers either due to need or fascination developed macros and fonts for many archaic writing systems. Many of these packages are now outdated, as the Unicode standard and the newer engines opened up a fascinating world. My own fascination with writing systems prompted me to add support for such scripts in the \pkgname{phd} package. The development to an extend was frustrating as the overloading of numerous fonts caused compilation to be very slow. Finding the right font was also problematic in many cases, as we opted to identify Open Source fonts. The \tex engine of preference is \luatex. To avoid loading too many fonts, unless they are required, we provide the keys:

\def\loadscripts{}
\cxset{scripts/.store in = \loadscripts}

\begin{key}{/phd/scripts = \meta{all, lineara, linearb, phaestos,\ldots}} The scripts key takes a list of options to enable or disable the loading of fonts and the usage of the key is explained later on. You set it with our only command |\cxset|\meta{key value list}
\end{key}

Supplementary keys, exist for each individual script enabling the setting of specific fonts to a particular script. However, if all the recommended fonts have been installed is quicker to use the |scripts| key.

\def\olmecfontstore{}

\cxset{olmec font/.store in=\olmecfontstore}

\cxset{olmec font=epiolmec}

\begin{key}{/phd/olmec font = \meta{font name}}
\end{key}

The key |script| can be used on its own. It will then load the default fonts built-in, in the |phd| package.

\cxset{script/.store in = \scripttempt}

\begin{key}{/phd/script = \meta{script name}}
\end{key}

\begin{figure}[b]
\centering
\includegraphics[width=0.6\textwidth]{./images/rongo.jpg}
\caption{Rongo rongo writing. Tablet B Aruku kurenga, verso. One of four texts which provided the Jaussen list, the first attempt at decipherment. Made of Pacific rosewood, mid-nineteenth century, Easter Island.
(Collection of the SS.CC., Rome)}
\end{figure}

The first attempt  at communication via writing was through ideographic or mnemonic symbols. Undoubtedly symbolic writing must have existed much earlier than the surviving artifacts, carved in woord or scribled on muddy walls. The earliest surviving symbolic writing are the Jiahu symbols. They were carved on tortoise shells in Jiahu, ca.~6600~BC. Jiahu was a neolithic Peligang culture site found in Henan, China. In Europe the Tărtăria tablets are three tablets, discovered in 1961 by archaeologist Nicolae Vlassa at a Neolithic site in the village of Tărtăria (about 30 km (19 mi) from Alba Iulia), in Romania.[1] The tablets, dated to around 5300 BC,[2] bear incised symbols - the Vinča symbols - and have been the subject of considerable controversy among archaeologists, some of whom claim that the symbols represent the earliest known form of writing in the world. The Indus script appeared ca. 3500 BC and the Nsibidi script of Nigeria, ca. before 500 AD. 

No type of writing system is superior or inferior to another, as the type is often dependent on the language they represent. For example, the syllabary works perfectly fine in Japanese because it can reproduce all Japanese words, but it wouldn't work with English because the English language has a lot of consonant clusters that a syllabary will have trouble to spell out. The pretense that the alphabet is more "efficient" is also flawed. Yes, the number of letters is smaller, but when you read a sentence in English, do you really spell individual letters to form a word? The answer is no. You scan the entire word as if it is a logogram.

And finally, writing system is not a marker of civilization. There are many major urban cultures in the world did not employ writing such as the Andean cultures (Moche, Chimu, Inca, etc), but that didn't prevent them from building impressive states and empires whose complexity rivals those in the Old World

Unicode encodes a number of ancient scripts, which have not been in normal use for a millennium or more, as well as historic scripts, whose usage ended in recent centuries. Although these scripts are no longer used to write living languages, documents and inscriptions using these languages exist, both for extinct languages and for precursors of modern languages. The primary user communities for these scripts are scholars, interested in studying the scripts and the languages written in them. A few, such as Coptic, also have contemporary liturgical or other special purposes. Some of the historic scripts are related to each other as well as to modern alphabets. The following are provides as of Unicode version~7.2.
\index{Ancient and Historic Scripts>Ogham}
\index{Ancient and Historic Scripts>Old Italic}
\index{Ancient and Historic Scripts>Runic}
\index{Ancient and Historic Scripts>Gothic}
\index{Ancient and Historic Scripts>Akkadian}
\index{Ancient and Historic Scripts>Old Turkic}
\index{Ancient and Historic Scripts>Hieroglyphs}
\index{Ancient and Historic Scripts>Linear B}
\index{Ancient and Historic Scripts>Linear A}
\index{Ancient and Historic Scripts>Phoenician}
\index{Ancient and Historic Scripts>Old South Arabian}
\index{Ancient and Historic Scripts>Mandaic}
\index{Ancient and Historic Scripts>Avestan}
\index{Ancient Anatolian Alphabets}
\index{Old South Arabian}
\index{Phoenician}
\index{Imperial Aramaic}
\begin{center}
\begin{tabular}{lll}
Ogham (see \S\ref{s:ogham})           
&Ancient Anatolian Alphabets (see \S\ref{s:anatolian})
&Avestan (see \S\ref{s:avestan})\\
Old Italic (see \S\ref{s:olditalic})       
&Old South Arabian (see \S\ref{s:oldsoutharabian})          
&Ugaritic (see \S\ref{s:ugaritic})\\
Runic (see \S\ref{s:runic})            
&Phoenician (see \S\ref{s:phoenician})                  
&Old Persian (see \S\ref{s:oldpersian})\\
Gothic            
&Imperial Aramaic (see \S\ref{s:imperialaramaic})            
&Sumero-Akkadian. (see \S\ref{s:sumero})\\
Old Turkic (see \S\ref{s:oldturkic})     
&Mandaic (see \S\ref{s:mandaic}) 
&Egyptian Hieroglyphs.\\
Linear B (see \S\ref{s:linearb})          
&Inscriptional Parthian (see \S\ref{s:parthian})      
&Meroitic (see \S\ref{s:meroitic})\\
Cypriot (see \S\ref{s:cypriot})
&Inscriptional Pahlavi  (see \S\ref{s:inscriptionalpahlavi})       
&Linear A (see \S\ref{s:linearb})\\
\end{tabular}
\end{center}

The following scripts are also encoded but following the Unicode
convention are described in other sections

\begin{center}
\begin{tabular}{llllll}
Coptic &Glagolithic &Phags-pa. &Kaithi &Kharoshi &Brahmi.\\
\end{tabular}
\end{center}

Some scripts such as Epi-Olmec are not described in the Unicode standard, but we provide support for them.

\section{Linear A}
\label{s:lineara}
\newfontfamily\lineara{Aegean.ttf}

\section{Aegean and Cypriote Syllabaries}

The Greeks had evidently already occupied the mainland and islands of the
Ægean, including Crete, by the middle of the third millennium
BC. Around 2000 BC, following their consolidation of power on
Crete, new wealth from trade with cosmopolitan Canaan
allowed the creation of a complex palace economy, with major
centres at Knossos, Phaistos and other Cretan sites – Europe’s
first high civilization, the Minoan. Trade with Canaan had evidently
also brought Greeks into contact with Byblos’ pictorial
syllabic writing, whose underlying principle the Minoans borrowed.
Now, Cretans could also write their Minoan Greek language
using a small corpus of syllabo-logographic signs
representing \textit{in-di-vi-du-al} syllables. The signs themselves and
their phonetic values – nearly all V (e) or CV (te) – were wholly
indigenous: what the rebus signs, all originating from the
Cretan world, depicted, one pronounced in Minoan Greek, not
in a Semitic language. (Minoan Greek appears to have been an
archaic sister tongue of the mainland’s Mycenæan Greek.\footnote{A History of Writing. })

Three separate but related forms of syllabo-logographic
writing emerged in the Ægean between c. 2000 and 1200 BC: the
Minoan Greeks’ ‘hieroglyphic’ script and Linear A, and the
later Mycenæan Greeks’ Linear B. Minoan Greeks apparently
also took their writing at an early date to Cyprus, where it experienced
two stages: Cypro-Minoan (evidently derived from
Linear A is one of two currently undeciphered writing systems used in ancient Greece. Cretan hieroglyphic is the other. Linear A was the primary script used in palace and religious writings of the Minoan civilization. It was discovered by archaeologist Arthur Evans. It is the origin of the Linear B script, which was later used by the Mycenaean civilization.

Linear A and its daughter Linear C, the ‘Cypriote Syllabic
Script’. All Ægean and Cypriote scripts are clearly syllabologographic,
as the objective identity of each rebus sign would
have been immediately recognizable to each learner and user. It
seems that determinatives were never employed in any of the
Ægean or Cypriote scripts; however, logograms additionally
depicted most spelt-out items on accounting tablets. All Ægean
and Cypriote scripts, but for these separate logograms, were
completely phonetic.
\medskip

\includegraphics[width=0.8\textwidth]{./images/cretan-hieroglyphs.png}

\medskip
Crete’s `hieroglyphic’ script is the patriarch of this robust
family, its inspiration perhaps derived from Byblos via Cyprus
around 2000 BC. As its name implies, this script used
pictorial signs to reproduce the syllabic inventory of the Minoan
Greek language, here used in rebus fashion as at Byblos. This
writing occurs on seal stones (and their clay impressions), baked
clay, and metal and stone objects, most of these discovered at
Knossos and dating from 2000– 1400 BC (the script was concurrent
with Linear A). There exist about 140 different signs in all –
that is, 70 to 80 syllabic signs and their alloglyphs (different signs
with the same sound value), as well as logograms: human figures,
parts of the body, flora, fauna, boats and geometrical shapes.
Writing direction was open: from left to right, from right to left,
with every other line reversed, even spiral. That this script also
included logograms and numerals suggests that it was initially
used for book-keeping, among other things, until its replacement
in this function with its simplification, Linear A. Thereafter, like
Anatolian hieroglyphs, the Cretan hieroglyphic script appears to
have assumed a ceremonial role in Minoan Greek society,
reserved for sacred inscriptions, dedications and royal proclamations
on round clay disks.

In the 1950s, Linear B was largely udeciphered and found to encode an early form of Greek. Although the two systems share many symbols, this did not lead to a subsequent decipherment of Linear A. Using the values associated with Linear B in Linear A mainly produces unintelligible words. If it uses the same or similar syllabic values as Linear B, then its underlying language appears unrelated to any known language. This has been dubbed the Minoan language.\footnote{\url{http://www.people.ku.edu/~jyounger/LinearA/LinAIdeograms/}}

\begin{scriptexample}[]{Linear A}
\unicodetable{lineara}{  
\number"10600,"10610,"10620,"10630,"10640,"10650,"10660,"10670,
"10680,"10690,"106A0,"106B0,"106C0,"106D0,"106E0,"106F0,"10710,"10720,"10730,"10740,"10750,"10760,"10770}
\end{scriptexample}

Many of the characters form group and specialists name them such as vases in transliterations.

\begin{scriptexample}[]{Vases}
\begin{center}
\scalebox{3}{{\lineara \char"106A6}}
\scalebox{3}{{\lineara \char"106A5}}
\scalebox{3}{{\lineara \char"106A7}}
\scalebox{3}{{\lineara \char"106A9}}
\end{center}
\end{scriptexample}

Linear A contains more than 90 signs (open vowels and consonants+vowels) in regular use and a host of
logograms, many of which are ligatured with syllabograms and/or fractions; about 80\% of these
logograms do not appear in Linear B. While many of Linear A’s signs are also found in Linear B, some
signs are unique to A (e.g., A *301 and following), while some signs found in Linear B are not yet found
in Linear A (e.g., B 12, 14-15, 18-19, 25, 32-33, 36, 42-43, 52, 62-64, 68, 71-72, 75, 83-84, 89-91).

The Unicode Linear A encoding is broadly based on the GORILA ([{\arial ɡɔɹɪˈlɑː}]) catalogue
(Godart and Olivier 1976–1985), which is the basic set of characters used in decipherment efforts.However, “ligatures” which consist of simple horizontal juxtapositions are not uniquely encoded here, as
these may be composed of their constituent parts. On the other hand, “ligatures” which consist of stacked
or touching elements have been encoded. 

\def\codex#1{\emph{Codex #1}\index{codex>#1}}
%\newfontfamily{\gothicfamily}{Noto Sans Gothic}
\newfontfamily{\gothicfamily}{code2001.ttf}
\section{Gothic}

\label{s:gothic}

\subsection{Introduction}

East Germanic Goths rose to prominence during the Great
Migrations of the fourth and fifth centuries AD 31 Their Gothic
languages are primarily known to us today through a few surviving
fragments of Bible translations. It was the Visigothic bishop
Wulfila († AD 383), according to three ecclesiastical historians
writing a century later, who created ‘Gothic letters’ in order to
translate the Bible into the Visigothic language. The fourth century
Greek alphabet was Wulfila’s only apparent source.

Though the bishop’s original Visigothic hand has not survived,
closely related derivative scripts preserved in two later Gothic
manuscripts no older than the sixth century have been preserved
(illus. 116).

‘Wulfila’s script’, as it perhaps should properly be designated,
is an alphabetic script written from left to right without word
separation. Spaces indicate sentences or passages, as does a
colon or a centred dot (as with the Iberian scripts). Nasal suspension
– that is, marking where an /m/ or /n/ should be – is
sometimes indicated by a macron (a topping stroke) above the
preceding letter. Ligatures are even rarer than macrons. There
are frequent contractions: for example, ius is often used to spell
‘Jesus’. Apart from rare profane relics – witness the sixth-century
Latin-Gothic Deed of Naples – Wulfila’s script, measured
by those few inscriptions that have survived, appears to have
conveyed exclusively ecclesiastical texts.

\begin{figure}[htb]
\includegraphics[width=.45\textwidth]{gothic}
\caption{Codex Carolinus}
\end{figure}

The Gothic script that Wulfila devised from the Greek
alphabet did not engender daughter scripts. After the sixth century
AD, it was replaced almost everywhere by related descendants
of Greek and Latin alphabets. Gothic’s last sentinel, the
ninth-century \codex{Vindobonensis} 795, was perhaps by then
only an antiquarian curiosity. The \emph{Codex Carolinus} preserves papal correspondence
with Frankish rulers, including letters exchanged by popes from Gregory III (731-741) to Hadrian I (772-795). the Codex was written in 791 on the orders of Charlemagne in order to rescue papyrus copies threatened with decay. It contains 99 letters, almost exclusively papal, and survives today in Vienna, \"Osterreichische Nationalbibliotek 449, in a copy probably made at Colone during the pontificate of Archbishop Willibert (870-889). The preface of the \codex{Carolinus} appears to refer to a second part that may have contained letters to byzantine rulers, now lost. Parallel copies of the Codex have not turned up. \citep{jasper2001papal}. 

\subsection{Unicode}

The Gothic alphabet was added to the Unicode Standard in March, 2001 with the release of version 3.1.

The Unicode block for Gothic is U+10330–U+1034F in the Supplementary Multilingual Plane. As older software that uses UCS-2 (the predecessor of UTF-16) assumes that all Unicode codepoints can be expressed as 16 bit numbers (U+FFFF or lower, the Basic Multilingual Plane), problems may be encountered using the Gothic alphabet Unicode range and others outside of the Basic Multilingual Plane.

\begin{scriptexample}[]{Gothic}
\unicodetable{gothicfamily}{"10330,"10340}
\end{scriptexample}
{\gothicfamily
𐍀	𐍁	𐍂	𐍃	𐍄	𐍅	𐍆	𐍇	𐍈	𐍉	𐍊}
%http://www.gotica.de/carolinus.html

%\begin{thebibliography}
%\bibitem[Fitzmyer(1995)]{fitzmyer}
%J.~A. Fitzmyer.
%\newblock \emph{The Aramaic inscriptions of Sefīre}, volume~19 of
%  \emph{Biblica et orientalia Sacra Scriptura antiquitatibus orientalibus
%  illustrata}.
%\newblock Pontificial Biblical Institute, Rome, 1995.
%\newblock URL
%  \url{http://web.archive.org/web/20051104215025/http://www.nelc.ucla.edu/Faculty/Schniedewind_files/NWSemitic/Aramaic_ABD.pdf}.
%\end{thebibliography}  











\newfontfamily\linearb{Aegean.ttf}
\section{Linear B}
\label{s:linearb}
\index{scripts>Linear B}
The Linear B script is a syllabic writing system that was used on the island of Crete and
parts of the nearby mainland to write the oldest recorded variety of the Greek language.

Linear B clay tablets predate Homeric Greek by some 700 years; the latest tablets date from
the mid- to late thirteenth century \bce. Major archaeological sites include Knossos, first
uncovered about 1900 by Sir Arthur Evans, and a major site near Pylos. The majority of
currently known inscriptions are inventories of commodities and accounting records.

The first tablets bearing the scripts were discovered by Sir Arthur Evans (1851-1941) while he was excavating the Minoan palace at Knossos in Crete. 


\medskip

\begin{figure}[ht]
\centering
\begin{minipage}{5cm}
\includegraphics[width=5cm]{./images/iklaina.jpg}
\end{minipage}\hspace{2em}
\begin{minipage}{7cm}
\captionof{figure}{Recently discovered fragment with Linear B, inscription. Found in an olive grove in what's now the village of Iklaina, the tablet was created by a Greek-speaking Mycenaean scribe between 1450 and 1350 B.C. (See \protect\href{http://news.nationalgeographic.com/news/2011/03/110330-oldest-writing-europe-tablet-greece-science-mycenae-greek/}{National Geographic}).}
\end{minipage}

\end{figure}


Early attempts to decipher the script failed until Michael Ventris, an architect and amateur
decipherer, came to the realization that the language might be Greek and not, as previously
thought, a completely unknown language. Ventris worked together with John Chadwick,
and decipherment proceeded quickly. The two published a joint paper in 1953. See \fullcite{ventrisa}.




Linear B was added to the Unicode Standard in April, 2003 with the release of version 4.0.

The Linear B Syllabary block is \unicodenumber{U+10000–U+1007F}. The Linear B Ideograms block is {\smallcps U+10080–U+100FF}. The Unicode block for the related Aegean Numbers is U+10100–U+1013F.

\begin{scriptexample}[]{Linear B}
\unicodetable{linearb}{"10000,"10010,"10020,"10030,"10040,"10050,"10060,"10070}

\captionof{table}{Linear B Typeset with command \protect\string\linearb\ and the \texttt{Aegean} font.}
\end{scriptexample}

\begin{scriptexample}[]{Linear B}
\unicodetable{linearb}{"10080,"10090,"100A0,"100B0,"100C0,"100D0,"100E0,"100F0}
\captionof{table}{Linear B Ideograms. Typeset with command \protect\string\linearb\ and the \texttt{Aegean} font.}
\end{scriptexample}


\begin{scriptexample}[]{Aegean Numbers}
\unicodetable{linearb}{"10100,"10110,"10110,"10120,"10130}

\captionof{table}{Aegean Numbers}
\end{scriptexample}





\section{Phaestos Disc}


One of the puzzles of Minoan Crete is the Phaestos disc. The Phaistos Disc was discovered in the Minoan palace-site of Phaistos, near Hagia Triada, on the south coast of Crete;[1] specifically the disc was found in the basement of room 8 in building 101 of a group of buildings to the northeast of the main palace. This grouping of four rooms also served as a formal entry into the palace complex. Italian archaeologist Luigi Pernier recovered the intact \enquote{dish}, about 15 cm (5.9 in) in diameter and uniformly slightly more than 1 centimetre (0.39 inches) in thickness, on 3 July 1908 during his excavation of the first Minoan palace.

It was found in the main cell of an underground \enquote{temple depository}. These basement cells, only accessible from above, were neatly covered with a layer of fine plaster. Their content was poor in precious artifacts, but rich in black earth and ashes, mixed with burnt bovine bones. In the northern part of the main cell, in the same black layer, a few inches south-east of the disc and about 20 inches (51 centimetres) above the floor, Linear A tablet PH 1 was also found. The site apparently collapsed as a result of an earthquake, possibly linked with the eruption of the Santorini volcano that affected large parts of the Mediterranean region during the mid second millennium B.C.

\begin{figure}[htp]
\centering

\includegraphics[width=0.67\textwidth]{./phaistosdiscs.jpg}
\caption{Phaistos discs.}
\end{figure}

The Phaistos Disc is generally accepted as authentic by archaeologists.[2] The assumption of authenticity is based on the excavation records by Luigi Pernier. This assumption is supported by the later discovery of the Arkalochori Axe with similar but not identical glyphs.[3]


The possibility that the disc is a 1908 forgery or hoax has been raised by two scholars.[4][5][6] In his 2008 review, Robinson does not endorse the forgery arguments, but argues that \enquote{a thermoluminescence test for the Phaistos Disc is imperative. It will either confirm that new finds are worth hunting for, or it will stop scholars from wasting their effort.}[4]

A gold signet ring from Knossos (the Mavro Spilio ring), found in 1926, contains a Linear A inscription developed in a field defined by a spiral—similar to the Phaistos Disc.\footnote{See University of Cologne website \url{http://arachne.uni-koeln.de/arachne/index.php?view[layout]=objekt_item\&search[constraints][objekt][searchSeriennummer]=159123}} A sealing found in 1955 shows the only known parallel to sign 21 (the \enquote{comb}) of the Phaistos disc.[9] This is considered as evidence that the Phaistos Disc is a genuine Minoan artifact.[10]

\begin{figure}[htbp]
\centering

\includegraphics[width=4.5cm]{crete-spiral-ring}\includegraphics[width=4.5cm]{crete-spiral-ring-01}\includegraphics[width=4.5cm]{crete-spiral-ring-02}

\caption{A gold signet ring from Knossos (the Mavro Spilio ring), found in 1926, contains a Linear A inscription developed in a field defined by a spiral—similar to the Phaistos Disc}
\end{figure}

The disc is made of fine clay.  Both side of the disc carry an inscription arranged in a spiral around the centre. The characters were impressed with a punch or stamp before the clay was fired. There are
241 or 242 characters (one is damaged), which
comprise 45 signs of variable frequency. For
comparison, there are thousands of characters in a few pages of printed English text, comprising the 26 signs we call letters. Lines partition
the disc’s characters into 31 short sections on
side A and 30 on side B, most of which contain
three, four or five characters. It is tempting to
speculate that these sections represent words
in the language of the disc.

That the characters were printed, not carved,
is beyond dispute. But no one knows why the disc’s maker bothered to produce a punch or stamp for each sign, rather than inscribing each character afresh. Egyptian hieroglyphs or Mesopotamian cuneiform of the second
millennium bce are inscribed on stone or clay;
simlarly the Minoan scripts Linear A and B found
at Phaistos, Knossos and other Cretan sites. If
the punch or stamp was to \enquote{print} many copies of documents, one would expect further sam-
ples to have turned up in a century of intensive Mediterranean excavatio

There is patchy and inconclusive evidence for and against the disc’s Cretan origin. The
signs look nothing like those of Linear A, Linear B or any other Minoan script, except coincidentally. This has led some, including Evans and Chadwick, to propose that the disc — and presumably its language, too — was an import.

One sign bears a remarkable resemblance to the architecture of rock tombs found in Anatolia in modern Turkey. One or two others
resemble signs found on a few contemporaneous objects from different sites in Crete. Most
scholars today, including Duhoux, think it a plausible working hypothesis that the disc was made in Crete. Gareth Owens and his Team claim to have read the disc and you can hear how it sounded at a TED Talk\footnote{\url{https://www.youtube.com/watch?v=6Chcplx3tZ8}}.



\subsection{Signs}

There are 242 tokens on the disc, comprising 45 distinct signs. Many of these 45 signs represent easily identifiable every-day things. In addition to these, there is a small diagonal line that occurs underneath the final sign in a group a total of 18 times. The disc shows traces of corrections made by the scribe in several places. The 45 symbols were numbered by Arthur Evans from 01 to 45, and this numbering has become the conventional reference used by most researchers. Some symbols have been compared with Linear A characters by Nahm,[17] Timm,[3] and others. Other scholars (J. Best, S. Davis) have pointed to similar resemblances with the Anatolian hieroglyphs, or with Egyptian hieroglyphs (A. Cuny). In the table below, the character "names" as given by Louis Godart (1995) are given in upper case; where other description or elaboration applies, they are given in lower case.




\PrintUnicodeBlock{./languages/phaistos.txt}{\linearb}




The ideograms are symbols, not pictures of the objects in question, e.g. one tablet records a tripod with missing legs, but the ideogram used is of a tripod with three legs. In modern transcriptions of Linear B tablets, it is typically convenient to represent an ideogram by its Latin or English name or by an abbreviation of the Latin name. Ventris and Chadwick generally used English; Bennett, Latin. Neither the English nor the Latin can be relied upon as an accurate name of the object; in fact, the identification of some of the more obscure objects is a matter of exegesis.

\begingroup

\linearb

Vessels
\let\l\unicodenumber

\begin{tabular}{l>{\smallcps}l>{\smallcps}l>{\smallcps}l>{\smallcps}l}
𐃟	&U+100DF	&200	&\l{sartāgo}	&\l{Boiling Pan}\\
𐃠	&U+100E0	&201	&\l{tripūs}	&\l{Tripod Cauldron}\\
𐃡	&U+100E1	&202	&\l{pōculum}	&\l{Goblet}\\
𐃢	&U+100E2	&203	&\l{urceus}	&\l{Wine Jar?}\\
𐃣	&U+100E3	&204  &\l{Tahirnea}	&\l{Ewer}\\
𐃤	&U+100E4	&205  &\l{Tnhirnula}	&\l{Jug}\\
𐃥	&U+100E5	&206	&\l{hydria}	&Hydria\\
𐃦	&U+100E6	&207	&\l{TRIPOD}  &AMPHORA\\
𐃧	&\l{U+100E7}	&\l{208}	&\l{PAT patera}	&\l{BOWL}\\
𐃨	&U+100E8	&209	&AMPH amphora	&AMPHORA\\
𐃩	&U+100E9	&210	&STIRRIP &JAR\\
𐃪	&U+100EA	&211	&WATER &BOWL?\\
𐃫	&U+100EB	&212	&SIT situla	&WATER JAR?\\
𐃬	&U+100EC	&213	&LANX lanx	&COOKING BOWL\\
\end{tabular}




\subsection{Online Resources}

Corpora and GORILA \url{http://www.people.ku.edu/~jyounger/LinearA/\#3}



\endgroup











\section{Cypriot Syllabary}
\label{s:cypriot}
The Cypriot or Cypriote syllabary is a syllabic script used in Iron Age Cyprus, from ca. the 11th to the 4th centuries BCE, when it was replaced by the Greek alphabet. A pioneer of that change was king Evagoras of Salamis. It is descended from the Cypro-Minoan syllabary, in turn a variant or derivative of Linear A. Most texts using the script are in the Arcadocypriot dialect of Greek, but some bilingual (Greek and Eteocypriot) inscriptions were found in Amathus.

\begin{figure}[htb]
\centering
\begin{minipage}{7cm}
\includegraphics[width=7cm]{./images/idalion-tablet.jpg}
\end{minipage}\hspace{1.5em}
\begin{minipage}{6cm}
\captionof{figure}{The bronze Idalion Tablet, from Idalium, (Greek: Ιδάλιον), is from the 5th century BCE Cyprus. The tablet is inscribed on both sides.
The script of the tablet is in the Cypro-Minoan syllabary, and the inscription is in Greek. The tablet records a contract between "the king and the city":[1] the topic of the tablet rewards a family of physicians, of the city, for providing free health services to individuals fighting an invading force of Persians.}
\end{minipage}
\end{figure}


The characters are \textit{syllabic}. There is one character for each  vowel, \textit{a, e, i, o, u,} and perhaps one for \textit{o}. There is no distinction between long and short vowels. The other characters represent what are called \textit{open syllables}\footnote{ If a syllable ends with a consonant, it is called a closed syllable. If a syllable ends with a vowel, it is called an open syllable. }, i.e., beginning with a consonant and ending with a vowel. 

No distinction is made between smooth, middle and rough mutes. The same character stands for τά τ\’ασs, δα in Εδαλιον ανδ δα ιν Αθανα  κε, κη, γε, γη, χε, χη. This fact constitutes the greatest difficulty in reading Cypriote.  

The Cypriot syllabary was added to the Unicode Standard in April, 2003 with the release of version 4.0.
The Unicode block for Cypriot is \unicodenumber{U+10800–U+1083F}. The Unicode block for the related Aegean Numbers is \unicodenumber{U+10100–U+1013F}.

\newfontfamily\cypriote{Aegean.ttf}

\begin{scriptexample}[]{Cypriot Syllabary}
\unicodetable{cypriote}{"10800,"10810,"10820,"10830}

\cypriote \symbol{"10803}
\end{scriptexample}


\printunicodeblock{./languages/cyprus.txt}{\cypriote}

\section{Old Persian}
\label{s:oldpersian}


Old Persian, like Hittite an Indo-European language, was written in cuneiforms as of the first millenium BC, mostly between 550 and 350. King Darius’ monumental inscription at
Bisothum – in Old Persian, Elamite and Neo-Babylonian – furnished
the ‘key’ to cuneiform’s decipherment and the reconstruction
of these languages.28 Darius’ Old Persian scribes
effected the most drastic simplification of the borrowed Near
Eastern script (illus. 35). They reduced the cuneiform inventory
to only 41 signs of both syllabic (ka) and phonemic (/k/) values.
Thus, Old Persian cuneiform is ‘half syllabic, half letter writing’.
29 It appears to be on the fence between the Babylonians’
cuneiforms and the Levantines’ consonantal writing, a hybrid
solution using only four logograms and 36 syllabo-phonemic
signs written in wedges. Of particular significance is the fact
that Old Persian also conveys the individual long and short
vowels /a/ (pronounced AH), /i/ (EE) and /u/ (OO) that the
Ugaritic system had conveyed a thousand years earlier.

Old Persian cuneiform is a semi-alphabetic cuneiform script that was the primary script for the Old Persian language. Texts written in this cuneiform were found in Persepolis, Susa, Hamadan, Armenia, and along the Suez Canal.[1] They were mostly inscriptions from the time period of Darius the Great and his son Xerxes. Later kings down to Artaxerxes III used corrupted forms of the language classified as “pre-Middle Persian”.

\begin{scriptexample}[]{Old Persian}
\unicodetable{oldpersian}{"103A0,"103B0,"103C0,"103D0}
\end{scriptexample}

Scholars today mostly agree that the Old Persian script was invented by about 525 BC to provide monument inscriptions for the Achaemenid king Darius I, to be used at Behistun. While a few Old Persian texts seem to be inscribed during the reigns of Cyrus the Great (CMa, CMb, and CMc, all found at Pasargadae), the first Achaemenid emperor, or Arsames and Ariaramnes (AsH and AmH, both found at Hamadan), grandfather and great-grandfather of Darius I, all five, specially the later two, are generally agreed to have been later inscriptions.
Around the time period in which Old Persian was used, nearby languages included Elamite and Akkadian. One of the main differences between the writing systems of these languages is that Old Persian is a semi-alphabet while Elamite and Akkadian were syllabic. In addition, while Old Persian is written in a consistent semi-alphabetic system, Elamite and Akkadian used borrowings from other languages, creating mixed systems.
\medskip

{\leftskip-1.25cm
\includegraphics[width=\textwidth+2.5cm]{./images/naghshe.jpg}
\captionof{figure}{Panoramic view of the Naqsh-e Rustam. This site contains the tombs of four Achaemenid kings, including those of Darius I and Xerxes. (\textit{Wikimedia})}
}
\section{Inscriptional Pahlavi}
\label{s:inscriptionalpahlavi}
\newfontfamily\inscriptionalpahlavi{Noto Sans Inscriptional Pahlavi}

Pahlavi or Pahlevi denotes a particular and exclusively written form of various Middle Iranian languages. The essential characteristics of Pahlavi are[1]
the use of a specific Aramaic-derived script, the Pahlavi script;
the high incidence of Aramaic words used as heterograms (called hozwārishn, "archaisms").

Pahlavi compositions have been found for the dialects/ethnolects of Parthia, Parsa, Sogdiana, Scythia, and Khotan.[2] Independent of the variant for which the Pahlavi system was used, the written form of that language only qualifies as Pahlavi when it has the characteristics noted above.


Pahlavi is then an admixture of
written Imperial Aramaic, from which Pahlavi derives its script, logograms, and some of its vocabulary.

spoken Middle Iranian, from which Pahlavi derives its terminations, symbol rules, and most of its vocabulary.
Pahlavi may thus be defined as a system of writing applied to (but not unique for) a specific language group, but with critical features alien to that language group. It has the characteristics of a distinct language, but is not one. It is an exclusively written system, but much Pahlavi literature remains essentially an oral literature committed to writing and so retains many of the characteristics of oral composition.

\begin{scriptexample}[]{Pahlavi}
\unicodetable{inscriptionalpahlavi}{"10B60,"10B70}
\end{scriptexample}

\section{Imperial Aramaic}
\label{s:imperialaramaic}

\subsection{History}

Aramaic is the best-attested and longest-attested
member of the NW Semitic subfamily of languages
(which also includes inter alia \nameref{s:hebrew}, \nameref{s:phoenician},
\nameref{s:ugaritic}, Moabite, Ammonite, and Edomite). The
relatively small proportion of the biblical text
preserved in an Aramaic original (Dan 2:4–7:28; Ezra
4:8–68 and 7:12–26; Jeremiah 10:11; Gen 31:47 [two
words] as well as isolated words and phrases in
Christian Scriptures) belies the importance of this
language for biblical studies and for religious studies
in general, for Aramaic was the primary international
language of literature and communication throughout
the Near East from ca. 600 B.C.E. to ca. 700 C.E. and
was the major spoken language of Palestine, Syria,
and Mesopotamia in the formative periods of
Christianity and rabbinic Judaism. 



Aramaic survived over a period of 3,000 years, during which time its grammar, vocabulary and usage experienced great changes. Aramaic scholars found it useful to divide the several Aramaic dialects into periods, groups and subgroups based both on the chronology as well as the geography.

\begin{enumerate}
\item Old Aramaic
\item Imperial Aramaic
\item  Middle Aramaic
\item Late Aramaic
\item Modern Aramaic
\end{enumerate}


\subsection{Old Aramaic (to ca. 612 BCE)}
This period
witnessed the rise of the Arameans as a major force
in ANE history, the adoption of their language as an
international language of diplomacy in the latter days
of the Neo-Assyrian Empire, and the dispersal of
Aramaic-speaking peoples from Egypt to Lower
Mesopotamia as a result of the Assyrian policies of
deportation. The scattered and generally brief
remains of inscriptions on imperishable materials
preserved from these times are enough to
demonstrate that an international standard dialect had
not yet been developed. The extant texts may be
grouped into several dialects:

\subsection{Middle Aramaic (to ca. 250 C.E.)}
In the Hellenistic and Roman periods, Greek replaced
Aramaic as the administrative language of the Near
East, while in the various Aramaic-speaking regions
the dialects began to develop independently of one
another. Written Aramaic, however, as is the case
with most written languages, by providing a
somewhat artificial, cross-dialectal uniformity,
continued to serve as a vehicle of communication
within and among the various groups. For this
purpose, the literary standard developed in the
previous period, Standard Literary Aramaic, was
used, but lexical and grammatical differences based
on the language(s) and dialect(s) of the local
population are always evident. It is helpful to divide
the texts surviving from this period into two major
categories: epigraphic and canonical.

\subsection{Late Aramaic (to ca. 1200 C.E.)}
The bulk of
our evidence for Aramaic comes from the vast
literature and occasional inscriptions of this period.
During the early centuries of this period Aramaic
dialects were still widely spoken. During the second
half of this period, however, Arabic had already
displaced Aramaic as the spoken language of much
of the population. Consequently, many of our texts
were composed and/or transmitted by persons whose
Aramaic dialect was only a learned language.
Although the dialects of this period were previously
divided into two branches (Eastern and Western), it
now seems best to think rather of three: Palestinian,
Syrian, and Babylonian.

The Aramaic alphabet is adapted from the \nameref{s:phoenician} alphabet and became distinctive from it by the 8th century BCE.  The letters all represent consonants, some of which are \emph{matres lectionis}, which also indicate long vowels.

\subsection{Modern Aramaic (to the present day)}

These dialects can be divided into the same three
geographic groups.

\begin{description}

\item[a. Western]
Here Aramaic is still spoken only in
the town of Ma’lula (ca. 30 miles NNE of Damascus)
and surrounding villages. The vocabulary is heavily
Arabized.

\item[b. Syrian]
Western Syrian (Turoyo) is the language
of Jacobite Christians in the region of Tur-Abdin in
SE Turkey. This dialect is the descendant of
something very like classical Syriac. Eastern Syrian
is spoken in the Kurdistani regions of Iraq, Iran,
Turkey, and Azerbaijan by Christians and, formerly,
by Jews. Substantial communities of the former are
now found in North America. The Jewish speakers
have mostly settled in Israel. These dialects are
widely spoken by their respective communities and
have been studied extensively during the past
century. It has become clear that they are not the
descendants of any known literary Aramaic dialect.

\item[c. Babylonian] 

\nameref{s:mandaic} is still used, at least until
recently, by some Mandaeans in southernmost Iraq
and adjacent areas in Iran.

In addition, in recent years classical \nameref{s:syriac} has
undergone somewhat of a revival as a learned vehicle
of communication for Syriac Christians, both in the
Middle East and among immigrant communities in
Europe and North America.
\end{description}

\begin{figure}[htbp]
\centering
\includegraphics[width=0.6\textwidth]{./images/elephantine-papyrus.jpg}

\caption{The Elephantine papyri are ancient Jewish papyri dating to the 5th century BC, requesting the rebuilding of a Jewish temple. It also name three persons mentioned in Nehemiah: Darius II, Sanballat the Horonite and Johanan the high priest.}

\end{figure}


\subsection{Alphabet and typesetting}

The Aramaic alphabet is historically significant, since virtually all modern Middle Eastern writing systems can be traced back to it, as well as numerous non-Chinese writing systems of Central and East Asia. This is primarily due to the widespread usage of the Aramaic language as both a \emph{lingua franca} and the official language of the Neo-Assyrian Empire, and its successor, the Achaemenid Empire. Among the scripts in modern use, the Hebrew alphabet bears the closest relation to the Imperial Aramaic script of the 5th century BC, with an identical letter inventory and, for the most part, nearly identical letter shapes.

Writing systems that indicate consonants but do not indicate most vowels (like the Aramaic one) or indicate them with added diacritical signs, have been called abjads by Peter T. Daniels to distinguish them from later alphabets, such as Greek, that represent vowels more systematically. This is to avoid the notion that a writing system that represents sounds must be either a syllabary or an alphabet, which implies that a system like Aramaic must be either a syllabary (as argued by Gelb) or an incomplete or deficient alphabet (as most other writers have said); rather, it is a different type.

The Imperial Aramaic alphabet was added to the Unicode Standard in October 2009 with the release of version 5.2.
The Unicode block for Imperial Aramaic is \unicodenumber{U+10840–U+1085F}.

\begin{scriptexample}[]{Aramaic}
\unicodetable{imperialaramaic}{"10840,"10850}
\end{scriptexample}




\PrintUnicodeBlock{./languages/imperial-aramaic.txt}{\imperialaramaic}
\subsection{Ogham}

\newfontfamily\ogham{code2000.ttf}

Ogham was added to the Unicode Standard in September 1999 with the release of version 3.0.
The spelling of the names given is a standardisation dating to 1997, used in Unicode Standard and in Irish Standard 434:1999.
The Unicode block for ogham is \texttt{U+1680–U+169F}.

\begin{scriptexample}[]{Ogham}
\bgroup
\ogham
0	1	2	3	4	5	6	7	8	9	A	B	C	D	E	F\\
U+168x	   	ᚁ	ᚂ	ᚃ	ᚄ	ᚅ	ᚆ	ᚇ	ᚈ	ᚉ	ᚊ	ᚋ	ᚌ	ᚍ	ᚎ	ᚏ\\
U+169x	ᚐ	ᚑ	ᚒ	ᚓ	ᚔ	ᚕ	ᚖ	ᚗ	ᚘ	ᚙ	ᚚ	᚛	᚜	\\

\titus

0	1	2	3	4	5	6	7	8	9	A	B	C	D	E	F\\
U+168x	   	ᚁ	ᚂ	ᚃ	ᚄ	ᚅ	ᚆ	ᚇ	ᚈ	ᚉ	ᚊ	ᚋ	ᚌ	ᚍ	ᚎ	ᚏ\\
U+169x	ᚐ	ᚑ	ᚒ	ᚓ	ᚔ	ᚕ	ᚖ	ᚗ	ᚘ	ᚙ	ᚚ	᚛	᚜
\egroup		
\end{scriptexample}
\section{Ancient Anatolian Alphabets}

The Anatolian scripts described in this section all date from the first millenium BCE, and were used to write various ancient Indo-European languages of western and southwestern Anatolia (now Turkey). All are related to the Greek script and are probably adaptations of it. 

\newfontfamily\lycian{Aegean.ttf}
\let\lydian\lycian
\let\carian\lydian

\begin{description}
\item [Lycian] The Lycian alphabet was used to write the Lycian language. It was an extension of the Greek alphabet, with half a dozen additional letters for sounds not found in Greek. It was largely similar to the Lydian and the Phrygian alphabets.
 
\bgroup
\lydian
\obeylines
0	1	2	3	4	5	6	7	8	9	A	B	C	D	E	F
U+1028x	𐊀	𐊁	𐊂	𐊃	𐊄	𐊅	𐊆	𐊇	𐊈	𐊉	𐊊	𐊋	𐊌	𐊍	𐊎	𐊏
U+1029x	𐊐	𐊑	𐊒	𐊓	𐊔	𐊕	𐊖	𐊗	𐊘	𐊙	𐊚	𐊛	𐊜

Typeset with the \idxfont{Aegean.ttf} and the command \cmd{\lydian}
\egroup

\item[Lydian] Lydian script was used to write the Lydian language. That the language preceded the script is indicated by names in Lydian, which must have existed before they were written. Like other scripts of Anatolia in the Iron Age, the Lydian alphabet is a modification of the East Greek alphabet, but it has unique features. The same Greek letters may not represent the same sounds in both languages or in any other Anatolian language (in some cases it may). Moreover, the Lydian script is alphabetic.
Early Lydian texts are written both from left to right and from right to left. Later texts are exclusively written from right to left. One text is boustrophedon. Spaces separate words except that one text uses dots. Lydian uniquely features a quotation mark in the shape of a right triangle.
The first codification was made by Roberto Gusmani in 1964 in a combined lexicon (vocabulary), grammar, and text collection.


\bgroup
\lycian
\obeylines
	0	1	2	3	4	5	6	7	8	9	A	B	C	D	E	F
U+1092x	𐤠	𐤡	𐤢	𐤣	𐤤	𐤥	𐤦	𐤧	𐤨	𐤩	𐤪	𐤫	𐤬	𐤭	𐤮	𐤯
U+1093x	𐤰	𐤱	𐤲	𐤳	𐤴	𐤵	𐤶	𐤷	𐤸	𐤹						𐤿
Typeset with the \idxfont{Aegean.ttf} and the command \cmd{\lycian}

Examples of words

𐤬𐤭𐤠  - Ora - "Month"

𐤬𐤳𐤦𐤭𐤲𐤬𐤩  - Laqrisa - "Wall"

𐤬𐤭𐤦𐤡  - "House, Home"

\egroup

\item [Carian] The Carian alphabets are a number of regional scripts used to write the Carian language of western Anatolia. They consisted of some 30 alphabetic letters, with several geographic variants in Caria and a homogeneous variant attested from the Nile delta, where Carian mercenaries fought for the Egyptian pharaohs. They were written left-to-right in Caria (apart from the Carian–Lydian city of Tralleis) and right-to-left in Egypt. Carian was deciphered primarily through Egyptian–Carian bilingual tomb inscriptions, starting with John Ray in 1981; previously only a few sound values and the alphabetic nature of the script had been demonstrated. The readings of Ray and subsequent scholars were largely confirmed with a Carian–Greek bilingual inscription discovered in Kaunos in 1996, which for the first time verified personal names, but the identification of many letters remains provisional and debated, and a few are wholly unknown.

\begin{scriptexample}[]{Carian}
\bgroup
\carian
\obeylines
 	0	1	2	3	4	5	6	7	8	9	A	B	C	D	E	F
U+102Ax	𐊠	𐊡	𐊢	𐊣	𐊤	𐊥	𐊦	𐊧	𐊨	𐊩	𐊪	𐊫	𐊬	𐊭	𐊮	𐊯
U+102Bx	𐊰	𐊱	𐊲	𐊳	𐊴	𐊵	𐊶	𐊷	𐊸	𐊹	𐊺	𐊻	𐊼	𐊽	𐊾	𐊿
U+102Cx	𐋀	𐋁	𐋂	𐋃	𐋄	𐋅	𐋆	𐋇	𐋈	𐋉	𐋊	𐋋	𐋌	𐋍	𐋎	𐋏
U+102Dx	𐋐
\egroup
\end{scriptexample}

\newfontfamily\oldpunctuation{code2000.ttf}

Word dividers are infrequent, \emph{scriptio continua}\footnote{a style of writing without word dividers, that is, without spaces or other marks between words or sentences} is common. Words dividers which are attested are U+00B7 (\char"00B7) \textsc{MIDLE DOT} (or U+2E31 word separator middle dot), U+205A TWO DOT PUNCTUATION, and U+205D TRICOLON ({\oldpunctuation\char"205D}). In modern editions U+0020 SPACE may be found.

\end{description}

\section{Phoenician}
\label{s:phoenician}
\arial

The Phoenician alphabet and its successors were widely used over a broad area surrounding the Mediterranean Sea.

\let\phoenician\lycian

\begin{scriptexample}[]{Phoenician}

\unicodetable{phoenician}{"10900,"10910}

\end{scriptexample}

The Phoenician alphabet, called by convention the Proto-Canaanite alphabet for inscriptions older than around 1200 BCE, is the oldest verified consonantal alphabet, or abjad.[1] It was used for the writing of Phoenician, a Northern Semitic language, used by the civilization of Phoenicia. It is classified as an abjad because it records only consonantal sounds (matres lectionis were used for some vowels in certain late varieties).

Phoenician became one of the most widely used writing systems, spread by Phoenician merchants across the Mediterranean world, where it evolved and was assimilated by many other cultures. The Aramaic alphabet, a modified form of Phoenician, was the ancestor of modern Arabic script, while Hebrew script is a stylistic variant of the Aramaic script. The Greek alphabet (and by extension its descendants such as the Latin, the Cyrillic, and the Coptic) was a direct successor of Phoenician, though certain letter values were changed to represent vowels.

\begin{figure}[ht]
\includegraphics[width=\textwidth]{./images/phoenician.jpg}
\captionof{figure}{
Phoenician votive inscription from Idalion (Cyprus), 390 BC. BM 125315 from The Early Alphabet by John F. Healy.}
\end{figure}

As the letters were originally incised with a stylus, most of the shapes are angular and straight, although more cursive versions are increasingly attested in later times, culminating in the Neo-Punic alphabet of Roman-era North Africa. Phoenician was usually written from right to left, although there are some texts written in boustrophedon.


\printunicodeblock{./languages/phoenician.txt}{\phoenician}


\newpage
\section{Palmyrene}
\idxlanguage{Palmyrene}
\arial

Palmyrene is the very widely attested Aramaic dialect and script
of Palmyra in the Syrian desert. The texts date from the midfirst century to the destruction of Palmyra by the Romans in AD 272. Palmyra in the Roman period was a major trading centre.
\medskip

\begin{figure}[ht]
\centering

\includegraphics[width=0.9\textwidth]{./images/palmyrene.jpg}
\captionof{figure}{\protect\arial Limestone bust with Palmyrene inscription. Palmyra late 2nd century AD. BM WA 102612}

\end{figure}

\medskip
The longest of the Palmyrene texts, is the bilingual  taxation tariff written for the year 137 AD in Palmyrene Aramaic and Greek.\footnote{For more details see:MILIK J.T., Dédicaces faites par des dieux (Palmyre, Hatra, 
Tyr) et de thiases sémitiques à l'époque romaine, Paris 1972; ROSENTHAL R., Die 
Sprache der palmyrenischen Inschriften, Leipzig 1936; STARK J.K., Personal Names in 
Palmyrene Inscriptions, Oxford 1971; DRIJVERS H.J.W., The Religion of Palmyra, 
Leiden 1976; TEIXIDOR J., 'Palmyre et son commerce d'Auguste à Caracalla', in 
Semitica 34, (1984) 1-127.  } Trade connections 
took the Palmyrene script to other places, some not far away, such as Dura Europos on the Euphrates, butothers at a great distance. A particular inscription is from South Shields, Roman Arbeia, in the north-east of England, carved on behalf of a Palmyrene mechant for his deceased wife and probably dating to the early third century AD. 

The Palmyrene script existed in two main varieties, a monumental and a cursive one, though the latter is little known and the evidence  mostly from Palmyra itself. The Syriac script of Edessa in southern Turkey, is often regarded as derived or closely related to the Palmyrene---similarities are found in the letters: ', b, g, d, w, h, y, k, l, m, n, `, r and t---though a strong case can also be made for connecting Syriac with a northern Mesopotamian script-family represented principally in texts from Hatra, a city more or less contemporary with Palmyra in Upper Mesopotamia. 


\begin{figure}[ht]
\includegraphics[width=\textwidth]{./images/regina-epigraph.jpg}
\caption{It was customary for Palmyrenes to offer bilingual texts (Greek or Latin) on funerary monuments. The final line of Regina's epitaph is Barates' personal lament in Palmyrene: Regina, freedwoman of Barate, alas. (See \href{http://www2.cnr.edu/home/araia/regina.html}{regina}.)}
\end{figure}

A good article on the classification of Aramaic languages can be found in \textit{The Aramaic language and Its Classification} by Efrem Yildiz.\footnote{\url{http://www.jaas.org/edocs/v14n1/e8.pdf}}








\cxset{quotation font-size=\normalsize,
       quote font-size=\normalsize}


\section{Mandaic}
\label{s:mandaic}
\newfontfamily\mandaic{NotoSansMandaic-Regular.ttf}


The Mandaic script is used to write a dialect of Eastern Aramaic, which, in its classical
form, is currently used as the liturgical language of the Mandaean religion. A living language descended
from Classical Mandaic is spoken by a small number of people living in and around Ahvaz, Khūzestān,
in southwestern Iran; speakers are also found in emigrant communities in Sweden, Australia, and the
United States. There is a considerable amount of Iranian influence on the lexicon of Classical Mandaic,
and Arabic and Persian influence on the grammar and lexicon of the contemporary dialect. The script
itself is likely derived from the Parthian chancery script.

Mandaic is a right-to-left script. It is a true alphabet, using letters regularly for vowels
rather than as the \emph{matres lectionis} from which they derived. The three diacritical marks are used in
teaching materials to differentiate vowel quality. At present, at least, the rule is that they may be omitted
from ordinary text. In this regard they are very like the Arabic fatha, kasra, and damma or the Hebrew
vowel points.

The only so far I could find that can display the script is the Google \idxfont{NotoSansMandaic.ttf}.

\begin{scriptexample}[]{Mandaic}
\bgroup
\unicodetable{mandaic}{"0840,"0850}
\egroup
\end{scriptexample}

In 1943, Lady Ethel Drower published extracts from several magic “recipe books” that served the writers of amulets in Baghdad in the early 20th century, in particular from two manuscripts in her possession, DC 45 and DC 46.

\begin{figure}[hb]
\centering

\includegraphics[height=4cm]{./magic-letters.jpg}
\includegraphics[height=4cm]{./45-453.jpg}
\includegraphics[height=4cm]{./36-448.jpg}

\captionof{figure}{Mandaic Incantation vessels. The left image is from \protect\href{http://thesacredalphabet.blogspot.ae/}{thesacredalphabet}, whereas the last two are from \protect\href{http://www.archaeological-center.com/en/auctions/45-453}{archaeological-center} }
\end{figure}

 While Drower, following her native informants, entitled the work ‘A Mandæan Book of Black Magic’, the manuscripts themselves contain a wide range of formulae for amulets and talismans for various purposes, as Drower herself was well aware. Alongside spells for healing, protection and success, we find others for enflaming love or stirring up enmity.

 The manuscripts themselves appear to have been copied in the late 19th or early 20th centuries; in particular, DC 46, a substantial codex of 264 sides, is written on an extremely modern “clean” paper. DC 45 is written on a rougher paper and appears to be somewhat earlier. It is also more fragmentary, and contains several leaves that were copied by a different hand and inserted into the main part of the manuscript at a later date, though it is clear from their contents that they were intended to replace pages that had been worn or damaged, as they begin and end exactly as required by the preceding and following pages. As it survives today, DC 45 is also considerably shorter than DC 46; however, it also contains several spells that are not found in DC 46.\footnote{\protect\href{http://www.academia.edu/8294938/Arabic_Magic_Texts_in_Mandaic_Script_A_Forgotten_Chapter_in_Near-Eastern_Magic}{Magic Texts}}

Lady Drower inform us that among the Mandaens:

\begin{quote}
Writing in itself is a magic art, and the alphabet is sacred.
Each letter is supposed to invoke a spirit of light and is a thing of power. It is a practice to write the letters separately and to sleep each night with a letter beneath the pillow. If the sleeper sees in a dream something which will enlighten him, the letters upon which he slept that night is taken to a silversmith and a replica in gold or silver is made and worn around the neck as amulet See Mandaic Incantation Texts by Edwin M Yamauchi.
\end{quote}












\newcounter{glyphcount}
^^A\newfontfamily\aegyptus{AegyptusR.ttf}

\chapter{Aegyptian Hieroglyphics}

\index{fonts>Aegyptus}\index{Aegyptus (font)}
\index{fonts>Hieroglyphics}\index{languages>hieroglyphics}

\newfontfamily\hiero{NotoSansEgyptianHieroglyphs-Regular.ttf}

Hieroglyphic writing appeared in Egypt at the end of the fourth millennium bce. The writing
system is pictographic: the glyphs represent tangible objects, most of which modern
scholars have been able to identify. A great many of the pictographs are easily recognizable
even by nonspecialists. Egyptian hieroglyphs represent people and animals, parts of the
bodies of people and animals, clothing, tools, vessels, and so on.

Hieroglyphs were used to write Egyptian for more than 3,000 years, retaining characteristic
features such as use of color and detail in the more elaborated expositions. Throughout the
Old Kingdom, the Middle Kingdom, and the New Kingdom, between 700 and 1,000 hieroglyphs
were in regular use. During the Greco-Roman period, the number of variants, as
distinguished by some modern scholars, grew to somewhere between 6,000 and 8,000.

Hieroglyphs were carved in stone, painted on frescoes, and could also be written with a reed
stylus, though this cursive writing eventually became standardized in what is called \emph{hieratic}
writing. Unicode does not encode the hieratic forms separately, but ust considers them as cursive forms of the hieroglyphs encoded block.

The Demotic script and then later the Coptic script replaced the earlier hieroglyphic and
hieratic forms for much practical writing of Egyptian, but hieroglyphs and hieratic continued
in use until the fourth century ce. An inscription dated August 24, 394 ce has been
found on the Gateway of Hadrian in the temple complex at Philae; this is thought to be
among the latest examples of Ancient Egyptian writing in hieroglyphs

\begin{figure}[htb]
\includegraphics[width=\textwidth]{./images/bookofthedead.jpg}
\end{figure}

In hieroglyphic texts, these drawings are not only simply arranged in sequential order, but also grouped on top of and next to each other. This rather complicates matters trying to register and reproduce hieroglyphic texts using a computer.

\section{Computer Typesetting}

Typesetting hieroglyphics with computers presents a number of problems. First is the method of inputting the characters and second the various methods required to stack hieroglyphics, the direction of writing which can be one of four different directions.

When the first computers were introduced in Egyptology in the late 1970s and the beginning of the 1980s, the graphical capacity of the machines was still in its infancy. Early attempts to register the hieroglyphic pictorial writing on computer therefore chose an encoding system to do this, using alphanumeric codes to represent or replace the graphics. To prevent many people from reinventing the wheel, during the first "Table Ronde Informatique et Egyptologie" in 1984 a committee was charged with the task to develop a uniform system for the encoding of hieroglyphic texts on computer. The resulting Manual for the Encoding of Hieroglyphic Texts for Computer-input (Jan Buurman, Nicolas Grimal, Jochen Hallof, Michael Hainsworth and Dirk van der Plas, Informatique et Egyptologie 2, Paris 1988), simply called Manuel de Codage, presents an easy to use and intuitive way of encoding hieroglyphic writing as well as the abbreviated hieroglyphic transcription (transliteration). The system proposed by the Manuel de Codage has since been adopted by international Egyptology as the official common standard for registering hieroglyphic texts on computer. Mark-Jan Nederhof proposed an enhanced encoding scheme to remove many of the limitations in the Manuel de Codage.

\pkgname{HieroTeX} is a \latexe package developed by to typeset hieroglyphic texts and still works well. The advantages of using \tex is of course its excellent typesetting capabilities and the usage of macros. Although inputting the texts as MdC codes is not that difficult, repeating the same codes over and over can be avoided with easily constructed simple substitution macros. 

\subsection{fonts}

One of the best fonts I came across is \idxfont{Aegyptus} from \url{http://users.teilar.gr/~g1951d/}\footnote{The site also has fonts for Aegean Numbers, Ancient Greek Musical Notation, Ancient Greek Numbers, Ancient Roman Symbols, Arkalochori Axe, Carian, Cypriot Syllabary, Dispilio tablet, Linear A, Linear B Ideograms, Linear B Syllabary, Lycian, Lydian, Old Italic, Old Persian, Phaistos Disc, Phoenician, Phrygian, Sidetic, Troy vessels’ signs and Ugaritic. Cretan Hieroglyphs and Cypro-Minoan script(s) are offered in separate files.}. The font provides all the unicode characters and also offers an additional number of glyphs that are not in the Unicode standard. The font uses the Unicode Private Use Areas to encode the glyphs. 

Another font is the Noto Egyptian Hieroglyphics from Google. This is a lightweight font with the symbols in their proper unicode slots. Mark-Jan Nederhof's \idxfont{NewGardiner} font is another one with support only for the Gardiner set. The codepoint mappings are incorrect, as the font has been  
encoded to EGPZ. The font is similar to the Aegyptus font, however it is just transposed and not recommended unless it is transposed. 

The editor software JSesh\footnote{\protect\url{http://jsesh.qenherkhopeshef.org/}} also provides a free font |JSeshFont.ttf|. This offers a correctly mapped unicode and is another good alternative. The symbols are drawn somewhat simpler and is just a matter of taste what you want to use.

My recommendation is for short demonstration purposes, the Noto font is to be preferred while for more serious work the Aegyptus font will be more useful. Using Lua the font can be transposed automatically to allow the use of commands that refer to unicode numbers. Another advantage of the Aegyptus font is that the glyphs are named with their Gardiner numbers, so it is somewhat easier to programmatically access them by name.\footnote{Unicode does not name the glyphs, but simply calls the Egyptian Hieroglyph $n$. } 

\medskip

\ifxetex
\bgroup
\centering 
\font\myfont = "Aegyptus"
\scalebox{7}{\myfont\XeTeXglyph 201}
\scalebox{7}{\myfont\XeTeXglyph 203}
\scalebox{7}{\myfont\XeTeXglyph 163}
\scalebox{7}{\myfont\XeTeXglyph 164}
\scalebox{7}{\myfont\XeTeXglyph 165}
\scalebox{7}{\myfont\XeTeXglyph 168}
\captionof{table}{Example of Egyptian Hieroglyphics typeset with the \textit{Aegyptus} font.} 
\egroup
\fi

\ifluatex
\bgroup
\centering 
\aegyptus
\scalebox{7}{\char"F300C}
\scalebox{7}{\char"F3001}
\scalebox{7}{\char"F3010}
\scalebox{7}{\char"F308B}
\scalebox{7}{\char"F3097}
\scalebox{7}{\char"F3091}
\captionof{table}{Example of Egyptian Hieroglyphics typeset with the \textit{Aegyptus} font.} 
\egroup

\fi


\subsection{Unicode Block}

Egyptian hieroglyphs is a Unicode block containing the Gardiner's sign list of Egyptian hieroglyphics.
The code points, in the range |0x13000| to |0x1342E|, are available starting from
\href{http://unicode.org/charts/PDF/U13000.pdf}{Unicode 5.2}

\begin{scriptexample}[]{Hieroglyphic}
\bgroup
\unicodetable{hiero}{"13000,"13010,"13020,"13030,"13040,"13050,"13060,"13070,%
"13080,%
"13090,"130A0,"130B0,"130C0,"130D0,"130E0,"130F0,%
"13100,"13110,"13120,"13130,"13140,"13150,"13060,"13070,"13080,"13090}
\egroup
\end{scriptexample}

\subsection{Gardiner's classification}

The standard reference on Egyptian hieroglyphics is Gartiner's Sign List, which lists common Egyptian hieroglyphs. These are grouped in categories from A-Aa. Each category represents a theme for example category A, is "man and his occupations". Based on this list ``Queen with flower" is denoted as \texttt{B7}. 

\subsubsection{Character Names} 

Egyptian hieroglyphic characters have traditionally been designated in
several ways:

\begin{enumerate}
\item  By complex description of the pictographs: \texttt{GOD WITH HEAD OF IBIS}, and so forth.
\item By standardized sign number: C3, E34, G16, G17, G24.
\item For a minority of characters, by transliterated sound.
\end{enumerate}

The characters in the Unicode Standard make use of the standard Egyptological catalog
numbers for the signs. Thus, the name for {\hiero\char"130F9} |U+13049| egyptian hieroglyph e034 refers
uniquely and unambiguously to the Gardiner list sign E34, described as a “{\aegean DESERT HARE}” ({\hiero \char"130FA}) and used for the sound “wn”. The Unicode catalog values are padded to three places with
zeros, so where the Gardiner classification is shown as \texttt{E34}, the unicode value is \texttt{E034}. 

Names for hieroglyphic characters identified explicitly in Gardiner 1953 or other sources as
variants for other hieroglyphic characters are given names by appending “A”, “B”, ... to the sign number. In the sources these are often identified using asterisks. Thus Gardiner’s G7,
G7*, and G7** correspond to U+13146 egyptian sign g007 {\hiero \char"13147}, U+13147 egyptian sign g007a, and U+13148 egyptian sign g007b, respectively.

\def\texthiero#1{{\color{black!95}\hiero #1}}

\begin{longtable}{>{\Large}lll>{\ttfamily}l}
{\hiero \char"13000}&A1-A70 & Man and his occupations &U+13000-1304F\\
{\hiero \char"13050}&B1-B9  &Woman and her occupations &U+13050-13059\\
{\hiero \char"1305A} &C1-C24 &Anthropomorphic Deities &U+1305A-13075\\
{\hiero \char"13076} &D1-D67 &Parts of the Human Body &U+13076-130D1\\
{\hiero \char"130D2} &E1-E38 &Mammals &U+13076-130D1\\
{\hiero \char"130FE}  &F1-F53	&Parts of Mammals &U+130FE-1313E\\
{\hiero\char"1313F}	&G1-G54	&Birds &U+1313F-1317E\\
{\hiero \char"1317F}	&H1-H8	&Parts of Birds &U+1317F-13187\\
\texthiero{\char"13188}	&I1-I15	&Amphibious Animals, Reptiles, etc. &U+13188-1319A\\
\texthiero{\char"1319B}	&K1-K8	&Fishes and Parts of Fishes &U+1319B-131A2\\
\texthiero{\char"131A3}	&L1-L8	&Invertebrata and Lesser Animals &U+131A3-131AC\\
\texthiero{\char"131AD}	&M1-M44	&Trees and Plants &U+13AD-131EE\\
\texthiero{\char"131EF}	&N1-N42	&Sky, Earth, Water &U+131EF-1321F\\
\texthiero{\char"13250}	&O1-O51	&Buildings and Parts of Buildings &U+13250-1329A\\
\texthiero{\char"1329B}	&P1-P11	&Ships and Parts of Ships &U+1329B-132A7\\
\texthiero{\char"132A8}	&Q1-Q7	& Domestic and Funerary Furniture &U+132A8-132AE\\
\texthiero{\char"132AF}	&R1-R29	&Temple Furniture and Sacret Emblems &U+132AF-132D0\\
\texthiero{\char"132D1}	&S1-S46	&Crowns, Dress, Staves, etc. &U+132D1-13306\\
\texthiero{\char"13307}	&T1-T36	&Warfare, Hunting, Butchery &U+13307-13332\\
\texthiero{\char"13333}	&U1-42	&Agriculture, Crafts and Professions &U+13333-13361\\
\texthiero{\char"13362}	&V1-V40a	&Rope, Fibre, Baskets, Bags, etc. &U+13362-133AE\\
\texthiero{\char"133AF}	&W1-W25	&Vessels of Stone and Earthenware &U+133AF-133CE\\
\texthiero{\char"133CF}	&X1-X8a	&Loaves and Cakes &U+133CF-133DA\\
\texthiero{\char"133DB}	&Y1-Y8	&Writing, Games, Music &U+133DB-133E3\\
\texthiero{\char"133E4}	&Z1-Z16H	&Strokes, Geometrical Figures, etc. &U+133E4-1340C\\
\texthiero{\char"1340D}	&Aa1-Aa32	&Unclassified &U+1340D-1342E\\
\end{longtable}

I particularly like the crocodile sign \def\crocodile{\color{teal}{\Huge\texthiero{\char"13188}}} {\crocodile}, as it is applicable to describe people in my field of work. 

\begin{scriptexample}[]{Woman and her occupations}
\unicodetable{hiero}{"13050}
\end{scriptexample}

\section{Positioning}

One of the core assumptions of any hieroglyphic encoding or mark-up scheme following the MdC is that signs and groups of signs maybe positioned next to each other or above each other. The former is indicated by the operator * and the latter by :. One may also use -, which functions as * for horizontal texts and as : for vertical text. 

In some dialects of the MdC relative positioning has been extended by the use of the |&| operator. This is used to form a kind of ligature, such as |D&t| can be defined to represent the \textit{Cobra at rest} sign I10 with sign X1 underneath, as follows:

\begin{center}
{\hiero\HUGE
       \mbox{\rlap{\char"133CF}\char"13193\hfill\hfill}\\
       {\large|insert[bs](I10,X1)|}

\mbox{\rlap{\scalebox{0.5}{\char"133E3}}\char"13193\hfill\hfill}\\
 	
}
\end{center}

This is only a partial solution and to automate it via kerning tables, will require hundreds of entries in the kerning tables. It will also need constant modifications as researchers discover new combinations. A better approach and which is easily applied to \tex based systems would be to adopt Nederhof's method of creating a new command |insert[bs](I10,X1)|. 

In \tex one could simply define a command \cmd{\insert} with one optional argument to handle the positioning. The positioning uses the letters [b,t,s,e] to position the glyph. the letters s and e stand for start and end, whereas b,t for bottom and top respectively. When there are only two symbols involved, this is not such a difficult operation, but when three or more symbols are to be grouped and kerned together, inserting with some form of scaling is necessary.

\subsection{Enclosures}

Enclosures. The two principal names of the king, the \emph{nomen} and \emph{prenomen}, were normally
written inside a \emph{cartouche}: a pictographic representation of a coil of rope.

In the Unicode representation of hieroglyphic text, the beginning and end of the cartouche
are represented by separate paired characters, somewhat like parentheses. The Unicode manual states that `rendering of a full cartouche surrounding a name requires specialized layout software', which is of course an easy task for \tex.

\begin{macro}{\cartouche}
The commands \cmd{\cartouche} and \cmd{\cartouche}, from Peter Wilson's \pkgname{hierglyph} package have been used for many years to demonstrate the use of hieroglyphics with \latexe. 
\end{macro}

There are a several characters for these start and end cartouche characters, reflecting various styles for the enclosures.

\cartouche{{\hiero \char"13147}$sin^{2} x + cos^{2} x = 1$}
\Cartouche{{\hiero \char"13147}$sin^{2} x + cos^{2} x = 1$}

Unicode:{\hiero 𓇓𓏏𓊵𓏙𓊩𓁹𓏃𓋀𓅂𓊹𓉻𓎟𓍋𓈋𓃀𓊖𓏤𓄋𓈐𓎟𓇾𓈅𓏤𓂦𓈉 }

\textpmhg{\HQ} 

\cartouche{\pmglyph{K:l-i-o-p-a-d:r-a}}
%\translitpmhg{\HK\Hl\Hi\Ho\Hp\Ha\Hd\Hr\Ha}

\printunicodeblock{./languages/hieroglyphics.txt}{\hiero}
\printunicodeblock{./languages/hieroglyphics-13100.txt}{\hiero}
\printunicodeblock{./languages/hieroglyphics-13200.txt}{\hiero}
\printunicodeblock{./languages/hieroglyphics-13300.txt}{\hiero}
\printunicodeblock{./languages/hieroglyphics-13400.txt}{\hiero}
\section{Numerals}

Egyptian numbers are encoded following the same principles used for the
encoding of Aegean and Cuneiform numbers. Gardiner does not supply a full set of
numerals with catalog numbers in his Egyptian Grammar, but does describe the system of
numerals in detail, so that it is possible to deduce the required set of numeric characters.

Two conventions of representing Egyptian numerals are supported in the Unicode Standard.
The first relates to the way in which hieratic numerals are represented. Individual
signs for each of the 1s, the 10s, the 100s, the 1000s, and the 10,000s are encoded, because in
hieratic these are written as units, often quite distinct from the hieroglyphic shapes into
which they are transliterated. The other convention is based on the practice of the \emph{Manual
de Codage}, and is comprised of five basic text elements used to build up Egyptian numerals.
There is some overlap between these two systems.

%% Needs some work to get it into LuaLaTeX
%% omitted for the time being
%\ifxetex
%\begin{texexample}{TeXeXglyph}{ex:xetexglyph}
%\raggedright
%\font\myfont = "Aegyptus"
%\setcounter{glyphcount}{136}
%
%\whiledo
%{\value{glyphcount}<\XeTeXcountglyphs\myfont}
%{\arabic{glyphcount}:~
%{\myfont\XeTeXglyph\arabic{glyphcount}}\quad
%\stepcounter{glyphcount}}
%\end{texexample}
%\fi

\section{Input Methods}

If you writing a document with a lot of hieroglyphics inputting of hieroglyphics can be problematic. Most researchers in the field will use special keyboards or editors. They also use MS/Word or OpenOffice. They can both be coerced to produce reasonable documents, but with \tex obviously better results can be achieved. One such editor is \href{http://jsesh.qenherkhopeshef.org/}{jsesh}. 


\begin{luacode*}
    local h = {}
          h = dofile("hiero.lua")
    local options = {style="block",
                     echo=true,
                     direction="RL",
                     size = "\\Huge",
                     color = "green",
                     headings = "captionof{figure}"  -- section/tablecaption/figurecaption
                     }
   -- prints full symbol list
   h.printgardiner(t,options)

   tex.print("\\par")
   local options = {style="block",
                     echo=true,
                     heading="\\par",
                     direction="RL",
                     color = "teal",
                     scale = 8}

   h.printhierochar("hiero","1317D",options)
   h.printhierochar("hiero","13000",{direction="RL",
                                        color = "teal",
                                        scale = 8})
   h.printhierochar("hiero","13003",{direction="LR",
                                        color = "teal",
                                        scale = 1})
   h.parseMdC([[M23-X1-R4-X8-Q2-D4-W17-R14-G4-R8-O29-
               V30-U23-N26-D58-O49-Z1-F13-N31-V30-N16-
               N21-Z1-D45-N25!]])

   tex.print("\\par")
   h.printgardinercat("B")

\end{luacode*}

\newcommand\hierochar[2][direction = "LR",
                         color     = "teal",
                         scale     = 1]{% 
               \luaexec{
                h = h or {}
                h = require("hiero.lua")  
                h.parseMdC(#2,{#1})}}
               
\newcommand\printhierochar[3][direction = "LR",
                              color     = "teal",
                              scale     = 4]{% 
               \luaexec{
                h = h or {}
                h = require("hiero.lua")  
                h.printhierochar(#2,#3,{#1})}}

This file just tests the various commands available for manipulating hieroglyphics. We tried to 
generalize the commands, so they can be re-used for other type of hieroglyphics.

{
\hierochar{"A1-A2-A3!"}

\centering 

\def\options{direction = "LR",
             color     = "teal",
             scale     = 7}

\def\fontname{"hiero"}

\def\hierochar#1{\printhierochar[\options]{\fontname}{#1}}
}


\begin{scriptexample}[]{Some Example}
Sometimes kerning might be required, especially if the
glyphs are scaled.This is easily achieved with a \cmd{\kern}
command and a suitable skip dimension.

\medskip

\bgroup
\fboxsep=0pt\fboxsep.4pt
\def\options{direction = "RL",
             color     = "black!95",
             scale     = 5}
\centering

\color{teal}
\fbox{\hierochar{"13051"}}
\kern-4mm
\hierochar{"13003"}
\def\options{direction = "LR",
             color     = "black!95",
             scale     = 5}
\fbox{\hierochar{"13003"}}\color{red}
\kern-4mm
\hierochar{"13051"}
\color{black!95}
\egroup
\begin{verbatim}
\centering
\hierochar{"13051"}
\kern-4mm
\hierochar{"13003"}
\def\options{direction = "RL",
             color     = "black!95",
             scale     = 5}
\hierochar{"13003"}
\kern-4mm
\hierochar{"13051"}
\end{verbatim}
\end{scriptexample}

A bit of a diversion is appropriate at this point. Our attempt after the historical overview, is to provide some routines for the capturing and display of hieroglyphic texts using LuaTeX. This involves getting low level information from the system regarding fonts. 

\begin{figure}[ht]
\begin{minipage}{0.45\textwidth}
\centering
\includegraphics[width=0.6\textwidth]{./images/fontforge.jpg}
\end{minipage}
\begin{minipage}[t]{0.45\textwidth}
\caption{Viewing font information with fontforge.}
\end{minipage}
\end{figure}

For each glyph, we are interested to get its unicode number, the position in the font table, its name and most importantly the font metrics. The font metrics are a set of parameters that are used to measure the bounding box, any ascenders or descenders and similar information. Using fontforge, these parameters can easily be viewed. However, we are not interested to make any modifications manually; what we are interested is to programmatically obtain this information using Lua. Lua's philosophy and a mantra repeated often by the developers, is that it provides the tools and not the solutions. What this means to the LuaTeX programmer, is that we need to reach very low level  to get this information, which is a road with many bumps. Luckily the tools have been provided by the LuaTeX developers. This comes with a lot of benefits as we can also do our own on the fly mapping, such as creating an index table holding all the Gardiner numbers. 

The |fontloader.open| function loads a font, but it's not usable by itself; the result should be turned into a table with
\textbf{fontloader.to\_table}, as follows:

\begin{verbatim}
  local f = fontloader.open
     ("c:/windows/fonts/NotSansEgyptianHieroglyphics-
       Regulat.ttf")
  fonttable = fontloader.to_table(f)
  fontloader.close(f)
\end{verbatim}

We will use the Google No Tofu Egyptian Hieroglyphic font to experiment with our hieroglyphics. I have used a full path to load the font, which resides on my windows machine in the fonts folder. Once we load all the information in the |fonttable| we use |fontloader.close| to discard the userdata from which the table is extracted. 

What makes OpenType fonts special is that they describe every aspect that you might be able to think of when you think of putting letters together to form words. In addition to the obvious "this is what letters look like" information, OpenType fonts also specify things like the name of each letter that is available in the font, how much of the Unicode standard the font implements, which horizontal and vertical metrics apply to which letters, exactly how the letters are arranged inside the font so that they can quickly be read out, what kind of font classifications apply (is it a fantasy font? is it bold face? is it fixed width? etc), what kind of memory allocation a printer needs to perform in order to be able to even load the font, etc. etc. etc. All these are stored in tables upon tables, similat to a collection of Russian dolls.

To view the values in the fonttable, we will first iterate over the \textbf{fonttable} and extract all the first level keys.

\begin{texexample}{Iterating through a font table}{}
\begin{luacode*}
local z={}
tf=fontloader.to_table(fontloader.open("c:/windows/fonts/NotoSansEgyptianHieroglyphs-Regular.ttf"))

-- we sort the keys to create a table
-- important keys to us are tf.glyphs

for k,v in pairs (tf) do
   --tex.print(k.."\\par")
   table.insert(z, k)
end

table.sort(z)
tex.print("\\begin{multicols}{3}\\raggedright")
for k,v in pairs (z) do
   z[k] = string.gsub(z[k],"%_","\\textunderscore ")
   local s = tf[v]
   tex.print("\\textbullet\\hskip3pt\\hangindent2em " .. z[k].." [\\textit{"..type(s).."}] ","\\par")
end
tex.print("\\end{multicols}")
\end{luacode*}
\end{texexample}

We iterate through the \textbf{fonttable} using the Lua  "pair" iterator and we simply print all the keys and the type of the values in a human readable form as shown in the example. Note the use of |\textunderscore| that replaces all underscores in the fields with its text equivalent to sanitize the output. This is a quick and dirty way to avoid the use of catcodes. Many of the keys, bear intuitive names and are not difficult to discern: \textit{version}, \textit{copyright} and the like. Getting the type of Lua variables is important in order to use them for error trapping. When you attempt for example to print a nil value an error will occur.

Now that we have peeked under the font we will iterate and capture the information of interest, which we will put into another table with two keys \textbf{info}  and \textbf{metrics}. In the metrics file we will get the bounding box related metrics of each and every glyph in the font and save it, into our own table. 

\begin{texexample}{More Metrics}{}
  \begin{luacode*}
   tex.print("units per em = ", tf.units_per_em,"\\par")
   for i,j in ipairs (tf.glyphs[6].boundingbox) do
      tex.print("bounding box["..i.."]".." = ", j,"\\par")
   end 
   local w = (tf.glyphs[6].boundingbox[3]-tf.glyphs[6].boundingbox[1])/tf.units_per_em
   local h = tf.glyphs[6].boundingbox[4]/tf.units_per_em
   tex.print("glyph width = ", w,"em\\par")
   tex.print("glyph height = ", h,"em\\par")

-- presents a nicely typeset table 

local rep, write = string.rep, tex.print
function ExploreTable (tab, offset)
    offset = offset or ""
    for k, v in pairs (tab) do
        local newoffset = offset .. "\\mbox{.}"
        if type(v) == "table" then
           -- if k == "boundingbox" then write("BB") end
           write(offset..k .. " = \\{\\par ")
           ExploreTable(v, newoffset)
           write(offset..newoffset .. "\\}\\par")
         else
           write(offset..k .. " = "..tostring(v),"\\par")
         end
      end
end

write("\\par{\\ttfamily ")
ExploreTable(tf.glyphs[38],"\\mbox{.}")
write("}")
  \end{luacode*}
\end{texexample}

The OpenType fonts standard, provides for so much information that we will ignore most of the items and focus on only a few tables and fields. A small utility after Paul Isambert's article is necessary to enable us to view tables easily within this book,


\begin{texexample}{ExploreTable utility}{}
\begin{luacode*}
-- presents a nicely typeset table 

local rep, write = string.rep, tex.print
function ExploreTable (tab, offset)
    offset = offset or ""
    for k, v in pairs (tab) do
        local newoffset = offset .. "\\mbox{.}"
        if type(v) == "table" then
           -- if k == "boundingbox" then write("BB") end
           write(offset..k .. " = \\{\\par ")
           ExploreTable(v, newoffset)
           write(offset..newoffset .. "\\}\\par")
         else
           write(offset..k .. " = "..tostring(v),"\\par")
         end
      end
end

write("\\par{\\ttfamily ")
ExploreTable(tf.glyphs[38],"\\mbox{.}")
write("}")
  \end{luacode*}
\end{texexample}

A good utility also is |TTX| that will convert an OTF font to XML and back. This requires that you have python installed.\footnote{See some good guidelines as to how to install it at \url{http://www.glyphrstudio.com/ttx/}.} The utility uses python to do the conversion. The archive can be downloaded from \url{http://sourceforge.net/projects/fonttools/files/latest/download}. This is a three prong attack. You need to have python install, the numpy library and then the TTX package. The |TTX| program was written by the font designer Just van Rossum, brother of the creator of the Python language, Guido van Rossum. The tool converts TrueType into human-readable |XML| format. The most attractive feature of this tool is that it also perform the opposite operation that is create a TruType font from an |XML| file. The |XML| format makes the hierarchy of the format clearer. Since SVG fonts are also described in |XML| it becomes an easier task to convert an |SVG| font to a TrueType font. To convert |bar.ttf| into |bar.ttx| you simply write:

\begin{verbatim}
ttx bar.ttf
\end{verbatim}

Similarly for the opposite conversion, from |.ttx| to |.ttf|

\begin{verbatim}
ttx bar.ttx
\end{verbatim}

The generated ttx file is approximately ten times larger than the original |.ttf| file. The files generated are huge affairs and difficult to manage.The command line option |-l| prints a list of the tables in the font. |TTX| is indispensable in the ``humanization'' of TrueType fonts. The details of the tables and what each field represents are eloquently described in that indispensable book by Yannis Haralambous \textit{Fonts \& Encodings.} Although the book is now somewhat dated, it is still the best source of information on many esoteric topics related to fonts. 






\input{./languages/meroitic}
\chapter{Ugaritic}
\label{s:ugaritic}
\index{Ugaritic fonts>Noto Sans Ugaritic}
\index{Ugaritic}
\index{Akkadian}
\index{Unicode>Ugaritic}
\parindent1em
\newfontfamily\ugaritic{NotoSansUgaritic-Regular.ttf}

\section{Background}
Sometime between 1190-1185 bce, the houses of Ugarit were abandoned by their inhabitants, then pillaged and burned. If they were destroyed by the Sea Peoples we will never know for sure, although this is very likely. This catastrophe ended a history of almost 6000 years. Ugarit was never rebuild and the ruins were buried for centuries before they were discovered in 1929. 

\begin{figure}[htbp]
\centering
\includegraphics[width=\textwidth]{ugarit-excavations}
%http://www.persee.fr/docAsPDF/syria_0039-7946_1936_num_17_2_3887.pdf
\end{figure}

Merchants figure prominently in Ugarit’s archives. The citizens engaged in trade, and many foreign merchants were based in the state, for example from Cyprus exchanging copper ingots in the shape of ox hides. The presence of Minoan and Mycenaean pottery suggests Aegean contacts with the city. It was also the central storage place for grain supplies moving from the wheat plains of northern Syria to the Hittite court.

common defence system (§ 11.5.4.3). The abundance of Cypriot
pottery,173 the Cypro-Minoan texts found in Ugarit ( L i v e r a n i 1979a,
1322-3) as well as letters18 and administrative texts,19 are also witness
to relationhips between the two communities at both the cultural
and the commercial levels. 

Some Cypriots (ally, altyy, DLU, 33)
receive from the Ugaritian administration food and clothing,20 others
belong to the guild of craftsmen.21 On the other hand, from its structure
the administrative text KTU 4.102 = RS 11.857 seems to be
a list of prisoners of war, or of persons detained for some reason,
who come from Cyprus ( V i t a 1995a, 108). An unpublished letter
found in Ras Shamra in 1994, which reports the dispatch of an
emissary of the king of Cyprus to Ugarit to deal with the freeing of
Cypriots detained on Ugaritic soil,22 could support this hypothesis

The \idxlanguage{Ugaritic} language  is written in alphabetic cuneiform. This was an innovative blending of an alphabetic script (like \hyperref[s:hebrew]{Hebrew}) and cuneiform (like Akkadian). The development of alphabetic cuneiform seems to reflect a decline in the use of Akkadian as a \textit{lingua franca} and a transition to alphabetic scripts in the eastern Mediterranean. Ugaritic, as both a cuneiform and alphabetic script, bridges the cuneiform and alphabetic cultures of the ancient Near East.


\begin{figure}[hb]
\centering
\includegraphics[width=\textwidth]{ugaritic-first-tablet}
\caption{A list of offerings with the first tablet number (KTU 1.39 = RS 1.001; Photo: UGARIT - FORSCHUNG Archive)}
\end{figure}

The Ugaritic script is a cuneiform (wedge-shaped) abjad used from around either the fifteenth century BCE[1] or 1300 BCE[2] for Ugaritic, an extinct Northwest Semitic language, and discovered in Ugarit (modern Ras Shamra), Syria, in 1928. It has 30 letters. Other languages (particularly Hurrian) were occasionally written in the Ugaritic script in the area around Ugarit, although not elsewhere.


\section{Material Culture}

Excavations at Ugarit have yielded an abundance of objects of everyday life that we can deduce the every day life of its inhabitants in a higher level of detail than many other civilizations. Objects recovered include mirrors, combs, cooking and drinking utensils, pottery, gems. An interesting item is the clepsydra shown in Figure~\ref{fig:clepsidra} used as a shower head. The religion and cults is also well represented. This is not easy to use as an individual and it was probably used with the help of a servant.

The Ugarites were actively interacting in trading. 

\begin{figure}[htbp]
\includegraphics[width=\textwidth]{clepsidra}
\caption{“Clepsydra” or shower vase RS 81.509
1981, City Center, House E, room 1201. Latakia Museum
H 19.5 cm, Diameter (max.) 18 cm. Fine plain buff pottery with burnished surface. Jug with a large,
ovoid body. The opening is narrow, contracting to a small hole 1 cm in diameter. The bottom is
pierced with 22 small holes to form a strainer. The narrowness of the opening does not permit filling
by any means other than plunging the vase entirely into a large container full of water. It holds about
1 liter. The function of this sort of vase is obvious. The container remained full if the opening was
sealed with one’s thumb to prohibit the entrance of air; the liquid could not flow out through the
bottom. When the thumb was removed (allowing air to enter the jug), the water could flow out
through the bottom, creating a type of shower head.
This object matches the definition of a clepsydra mentioned by ancient authors (Hieron): in its
primary sense, the term clepsydra is not restricted to a measure of time. What we have here is an instrument
used for washing, like a shower in a bathing installation (or shower stall). This was an object
of everyday life, but only in a relatively refined context. This vase was found with other personal
funerary objects fallen from the upper floor of a house of medium status in the city center. Other examples
(e.g., RS 30.325) show that this was not an uncommon item in homes at Ugarit.\\
– Bib.: M. Yon, P. Lombard, and M. Renisio, in RSO III, 1987, p. 106, fig. 87; P. Lombard, ibid., pp. 351–57.}
\label{fig:clepsidra}
\end{figure}

Clay tablets written in Ugaritic provide the earliest evidence of both the North Semitic and South Semitic orders of the alphabet, which gave rise to the alphabetic orders of Arabic (starting with the earliest order of its abjad), the reduced Hebrew, and more distantly the Greek and Latin alphabets on the one hand, and of the Ge'ez alphabet on the other. Arabic and Old South Arabian are the only other Semitic alphabets which have letters for all or almost all of the 29 commonly reconstructed proto-Semitic consonant phonemes. 

According to Dietrich and Loretz in Handbook of Ugaritic Studies (ed. Watson and Wyatt, 1999): "The language they [the 30 signs] represented could be described as an idiom which in terms of content seemed to be comparable to Canaanite texts, but from a phonological perspective, however, was more like Arabic."
The script was written from left to right. Although cuneiform and pressed into clay, its symbols were unrelated to those of the Akkadian cuneiform.

\begin{scriptexample}[]{Ugaritic}
\unicodetable{ugaritic}{"10380,"10390}
\end{scriptexample}

{\let\aegean\arial
\printunicodeblock{./languages/ugaritic.txt}{\ugaritic}
}

\bgroup

\let\a\arial
\Large
\begin{longtable}[l]{%
>{\arial\large}r|
>{\ugaritic}c| 
>{\arial\large}c 
>{\arial\large}c 
>{\arial\large}c >{\arial\large}c
}

&\a Sign	&\a Trans.	&\a IPA	&\a Hebrew	&\a Arabic \\
\hline
\inc &𐎀	&ʾa	& ʔa	&א	&أ \\
\inc &𐎁	&b	& b	    &ב	&ب \\
\inc &𐎂	&g	&ɡ	&ג	&ج\\
\inc &𐎃	&ḫ	&x	&	&خ\\
\inc &𐎄	&d	&d	&ד	&د\\
\inc &𐎅	&h	&h	&ה	&ه\\
\inc &𐎆	&w	&w	&ו	&و\\
\inc &𐎇	&z	&z	&ז	&ز\\
\inc &𐎈	&ḥ	&ħ	&ח	&ح\\
\inc &𐎉	&ṭ	&t̴	&ט	&ط\\
\inc &𐎊	&y	&j	&י	&ي\\
\inc &𐎋	&k	&k	&כ	&ك\\
\inc &𐎌	&š	&ʃ	&ש	&ش\\
\inc &𐎍	&l	&l	&ל	&ل\\
\inc &𐎎	&m	&m	&מ	&م\\
\inc &𐎏	&ḏ	&ð	&	&ذ\\
\inc &𐎐	&n	&n	&נ	&ن\\
\inc &𐎑	&ẓ	&θ̴	&	&ظ\\
\inc &𐎒	&s	&s	&ס	&س\\
\inc &𐎓	&ʿ 	&ʕ	&ע	&ع\\
\inc &𐎔	&p	&p	&פ	&ف\\
\inc &𐎕	&ṣ	&s̴	&צ	&ص\\
\inc &𐎖	&q	&q	&ק	&ق\\
\inc &𐎗	&r	&r	&ר	&ر\\
\inc &𐎘	&ṯ	&θ	&	&ث\\
\inc &𐎙	&ġ	&ɣ	&	&غ\\
\inc &𐎚	&t	&t	&ת	&ت\\
\inc &𐎛	&ʾi	&ʔi	&	&ئ\\
\inc &𐎜	&ʾu	&ʔu	&	&ؤ\\
\end{longtable}
\egroup


\textit{\LARGE$$\stackrel{\mbox{ho}}{.}$$}

% Tranliteration macros 
% 
\bgroup\ugaritic
\def\a{\char"10380}
\def\b{\char"10381}
\def\g{\char"10382}
\LARGE \a \b \g 
\egroup

\section{Online Collections}

http://digital.library.stonybrook.edu/











\section{Sumero Akkadian Cuneiform}
\label{s:sumero}
\newfontfamily\sumero{NotoSansSumeroAkkadianCuneiform-Regular.ttf}
In Unicode, the Sumero-Akkadian Cuneiform script is covered in two blocks:
U+12000–U+1237F Cuneiform
U+12400–U+1247F Cuneiform Numbers and Punctuation
These blocks, in version 6.0, are in the Supplementary Multilingual Plane (SMP).

The sample glyphs in the chart file published by the Unicode Consortium[2] show the characters in their Classical Sumerian form (Early Dynastic period, mid 3rd millennium BCE). The characters as written during the 2nd and 1st millennia BCE, the era during which the vast majority of cuneiform texts were written, are considered font variants of the same characters.

The character set as published in version 5.2 has been criticized, mostly because of its treatment of a number of common characters as ligatures, omitting them from the encoding standard.

\begin{scriptexample}[]{Sumero Akkadian}
\unicodetable{sumero}{"12000,"12010,"12020,"12030,"12040,"12050,"12060,"12070,
"12080,"12090,"12400,"12410,"12420,"12430}
\end{scriptexample}

\begin{table}[b]
\begin{scriptexample}[]{textbox}
From Plato's dialogue Phaedrus 14, 274c-275b:

Socrates: [274c] I heard, then, that   in Egypt, was one of the ancient gods of that country, the one whose sacred bird is called the ibis, and the name of the god himself was Theuth. He it was who [274d] invented numbers and arithmetic and geometry and astronomy, also draughts and dice, and, most important of all, letters. 

Now the king of all Egypt at that time was the god Thamus, who lived in the great city of the upper region, which the Greeks call the Egyptian Thebes, and they call the god himself Ammon. To him came Theuth to show his inventions, saying that they ought to be imparted to the other Egyptians. But Thamus asked what use there was in each, and as Theuth enumerated their uses, expressed praise or blame, according as he approved [274e] or disapproved.  

"The story goes that Thamus said many things to Theuth in praise or blame of the various arts, which it would take too long to repeat; but when they came to the letters, [274e] “This invention, O king,” said Theuth, “will make the Egyptians wiser and will improve their memories; for it is an elixir of memory and wisdom that I have discovered.” But Thamus replied, “Most ingenious Theuth, one man has the ability to beget arts, but the ability to judge of their usefulness or harmfulness to their users belongs to another; [275a] and now you, who are the father of letters, have been led by your affection to ascribe to them a power the opposite of that which they really possess.  

"For this invention will produce forgetfulness in the minds of those who learn to use it, because they will not practice their memory. Their trust in writing, produced by external characters which are no part of themselves, will discourage the use of their own memory within them. You have invented an elixir not of memory, but of reminding; and you offer your pupils the appearance of wisdom, not true wisdom, for they will read many things without instruction and will therefore seem [275b] to know many things, when they are for the most part ignorant and hard to get along with, since they are not wise, but only appear wise." 
\end{scriptexample}
\end{table}


\printunicodeblock{./languages/cuneiform.txt}{\sumero}





\section{Inscriptional Parthian}
\label{s:parthian}
\index{Ancient and Historic Scripts>Inscriptional Parthian}
\index{Inscriptional Parthian fonts>Noto Sans Inscriptional Parthian}

The Parthian script developed from the Aramaic alphabet around the 2nd century BCE and was used during the Parthian and Sassanid periods of the Persian Empire. The latest known inscription dates from 292 CE. 

\newfontfamily\parthian{NotoSansInscriptionalParthian-Regular.ttf}
Inscriptional Parthian is a Unicode block containing characters of the official script of the Sassanid Empire.

\newenvironment{parthiannumbers}{^^A
\def\1{\parthian\char"10B58}
\def\2{\parthian\char"10B59}
\def\3{\text{\parthian\char"10B5A}}
\def\4{\text{\parthian\char"10B5B}}^^A 
\TextOrMath\4 \4
\TextOrMath\3 \3
}{}
\index{Parthian numbers}
\begin{scriptexample}[]{}

\unicodetable{parthian}{"10B40,"10B50}



\end{scriptexample}

Inscriptional Parthian has its own numbers, which have right-to-left
directionality. The numbers are built up out of 1, 2, 3, 4, 10, 20, 100, and 1000 which is not such a great scheme. The inscriptions are not
normalized uniformly. The units are sometimes written with strokes of the same height, or with a final
stroke that is longer, either descending or ascending to show the end of the number; compare 5 in 15 ({\parthian \char"10B59 \char"10B5B}
or 2 + 3) and in 45 (òõ or 1 + 4); compare 6 in 16 (öö or 3 + 3) and in 36 (òôö or 1 + 2 + 3). The
encoding here allows the specialist to choose his or her preferred representation. The following is an list
of numbers attested in Inscriptional Parthian. The third column is displayed in visual order.

The |phd| package offers rudimentary support for Parthian numbers in the form of an environment |parthiannumbers|, which can be used as follows:

\begin{texexample}{Inscriptional Parthian numbers}{parth}
\begin{parthiannumbers}
\1 $= 1$
\2 $= 2$
\begin{align*}
\3 &= 3\\
\4 &= 4\\
\3\4 &=7
\end{align*}
\end{parthiannumbers}
\end{texexample}



\printunicodeblock{./languages/inscriptional-parthian.txt}{\parthian}




\footnote{\url{http://www.unicode.org/L2/L2007/07207-n3286-parthian-pahlavi.pdf}} 


\section{Old Italic}

\epigraph{A society grows great when old men plant
trees in whose shade they know they will never sit.}{Greek proverb}
\label{s:olditalic}
\index{scripts>Old Italic}
\newfontfamily\olditalic{Noto Sans Old Italic}


Old Italic refers to any of several now extinct alphabet systems used on the Italian Peninsula in ancient times for various Indo-European languages (predominantly Italic) and non-Indo-European (e.g. Etruscan) languages. The alphabets derive from the Euboean Greek Cumaean alphabet, used at Ischia and Cumae in the Bay of Naples in the eighth century BC.

Various Indo-European languages belonging to the Italic branch (Faliscan and members of the Sabellian group, including Oscan, Umbrian, and South Picene, and other Indo-European branches such as Celtic, Venetic and Messapic) originally used the alphabet. Faliscan, Oscan, Umbrian, North Picene, and South Picene all derive from an Etruscan form of the alphabet.

\section{Etruscan}

Many peoples took the system that the Greeks had elaborated and
adapted it to their own language. This was particularly true in Lemnos and
in Etruria, where signs inspired by Greek letters were put to the service of
languages that probably were closely related to Greek—signs that we can
read without fully comprehending them. The Etruscans seem to have used
writing largely for religious purposes. According to Cicero (De divinatione)
they bequeathed their sacred texts to the Romans, who held the Etruscan
religion to be the religion of the Book par excellence.\cite{henri1994}

\begin{figure}[htbp]
\centering
\includegraphics[width=0.7\textwidth]{marsiliana}
\caption{The Marsiliana Tablet}
\end{figure}

The Germanic runic alphabet was derived from one of these alphabets by the 2nd century.


Old Italic is a Unicode block containing a unified repertoire of the three stylistic variants of pre-Roman Italic scripts.

\begin{scriptexample}[]{Testing}
\unicodetable{olditalic}{"10300,"10310,"10320}

{\leavevmode
\hfill\hfill\hfill\footnotesize Typeset with \texttt{Noto Sans Old Italic~}
}
\end{scriptexample}
\section{Old South Arabian}
\label{s:oldsoutharabian}

\index{Ancient and Historic Scripts>Old South Arabian}
\index{Old South Arabian fonts>Noto Sans Old South Arabian}
\index{alphabets>Yemeni}

\newfontfamily\oldsoutharabian{NotoSansOldSouthArabian-Regular.ttf}

The ancient Yemeni alphabet (Old South Arabian ms3nd; modern Arabic: {\arabicfont المُسنَد‎}  musnad) branched from the Proto-Sinaitic alphabet in about the 9th century BC. It was used for writing the Old South Arabian languages of the Sabaic, Qatabanic, Hadramautic, Minaic (or Madhabic), Himyaritic, and proto-Ge'ez (or proto-Ethiosemitic) in Dʿmt. The earliest inscriptions in the alphabet date to the 9th century BC in Akkele Guzay, Eritrea[3] and in the 10th century BC in Yemen. There are no vowels, instead using the \emph{mater lectionis} to mark them.

Its mature form was reached around 500 BC, and its use continued until the 6th century AD, including Old North Arabian inscriptions in variants of the alphabet, when it was displaced by the Arabic alphabet.[4] In Ethiopia and Eritrea it evolved later into the Ge'ez alphabet,[1][2] which, with added symbols throughout the centuries, has been used to write Amharic, Tigrinya and Tigre, as well as other languages (including various Semitic, Cushitic, and Nilo-Saharan languages).

It is usually written from right to left but can also be written from left to right. When written from left to right the characters are flipped horizontally (see the photo).
The spacing or separation between words is done with a vertical bar mark (\textbar).
Letters in words are not connected together.

Old South Arabian script does not implement any diacritical marks (dots, etc.), differing in this respect from the modern Arabic alphabet.

\begin{scriptexample}[]{South Arabian}
\unicodetable{oldsoutharabian}{"10A60,"10A70}
\end{scriptexample}

Support in \latexe is provided via Peter Wilson's package \pkgname{sarabian}\citep{sarabian}. The package provides all the |metafont| sources as well as transliteration commands and other utilities \seedocs{\SARAB}. The package is based on fonts developed originally by Alan Stanier of Essex University.

The package provides the commands \docAuxCmd{sarabfamily} that selects the South Arabian font family. The family name is \texttt{sarab}. Another command \docAuxCmd{textsarab}\meta{text} typesets \meta{text} in the South Arabian font. The package provides two ways of accessing
glyphs: (a) by \texttt{ASCII} character commands, and (b) via commands. These are illustrated in
Table~\ref{sarabian1} which is a modified version of that provided in the Comprehensive Symbols.



\def\SAtdu{\oldsoutharabian\char"10A77}

A comparison between  the unicode and the rendering (scaled 5) \pkgname{sarabian} is shown below.

\centerline{\scalebox{3}{\SAtdu} \scalebox{3}{\textsarab{\SAtd}}}

There is no real advantage in using unicode fonts, if all you interested is to write some South Arabian text for inscriptions. 

\begin{symtable}[SARAB]{\SARAB\ South Arabian Letters}
\index{South Arabian alphabet}
\index{alphabets>South Arabian}
\label{sarabian1}
\begin{tabular}{*4{ll@{\qquad}}ll}
\K[\textsarab{\SAa}]\SAa   & \K[\textsarab{\SAz}]\SAz   & \K[\textsarab{\SAm}]\SAm   & \K[\textsarab{\SAsd}]\SAsd & \K[\textsarab{\SAdb}]\SAdb \\
\K[\textsarab{\SAb}]\SAb   & \K[\textsarab{\SAhd}]\SAhd & \K[\textsarab{\SAn}]\SAn   & \K[\textsarab{\SAq}]\SAq   & \K[\textsarab{\SAtb}]\SAtb \\
\K[\textsarab{\SAg}]\SAg   & \K[\textsarab{\SAtd}]\SAtd & \K[\textsarab{\SAs}]\SAs   & \K[\textsarab{\SAr}]\SAr   & \K[\textsarab{\SAga}]\SAga \\
\K[\textsarab{\SAd}]\SAd   & \K[\textsarab{\SAy}]\SAy   & \K[\textsarab{\SAf}]\SAf   & \K[\textsarab{\SAsv}]\SAsv & \K[\textsarab{\SAzd}]\SAzd \\
\K[\textsarab{\SAh}]\SAh   & \K[\textsarab{\SAk}]\SAk   & \K[\textsarab{\SAlq}]\SAlq & \K[\textsarab{\SAt}]\SAt   & \K[\textsarab{\SAsa}]\SAsa \\
\K[\textsarab{\SAw}]\SAw   & \K[\textsarab{\SAl}]\SAl   & \K[\textsarab{\SAo}]\SAo   & \K[\textsarab{\SAhu}]\SAhu & \K[\textsarab{\SAdd}]\SAdd \\
\end{tabular}

\bigskip
\begin{tablenote}
  \usefontcmdmessage{\textsarab}{\sarabfamily}.  Single-character
  shortcuts are also supported: Both
  ``\verb+\textsarab{\SAb\SAk\SAn}+'' and ``\verb+\textsarab{bkn}+''
  produce ``\textsarab{bkn}'', for example.  \seedocs{\SARAB}.
\end{tablenote}
\end{symtable}



\section{Avestan script}
\label{s:avestan}
The Avestan alphabet is a writing system developed during Iran's Sassanid era (AD 226–651) to render the Avestan language.
As a side effect of its development, the script was also used for Pazend, a method of writing Middle Persian that was used primarily for the Zend commentaries on the texts of the Avesta. In the texts of Zoroastrian tradition, the alphabet is referred to as \emph{din dabireh} or \emph{din dabiri}, Middle Persian for "the religion's script".

The Avestan alphabet was replaced by the Arabic alphabet after Persia converted to Islam during the 7th century CE. 


Notable Features

The alphabet is written from right to left, in the same way as Syriac, Arabic and Hebrew.
See more at: \url{http://www.iranchamber.com/scripts/avestan_alphabet.php#sthash.ZRu7AkEb.dpuf}

\newfontfamily\avestan{NotoSansAvestan-Regular.ttf}



\begin{scriptexample}[]{Avestan}
\ifxetex\TeXXeTstate=1
\beginR\fi
\avestan\raggedleft
𐬄	
𐬅	
𐬆	
𐬇	
𐬈	
𐬉	
𐬊	
𐬋	
𐬌	
𐬍	
𐬎	
𐬏	
𐬐	
	
𐬒	
𐬓	
𐬔	
	
𐬖	
𐬗	
𐬘	
𐬙	
𐬚	
𐬛	
𐬜	
𐬝	
𐬞	
𐬟	
𐬠	
𐬡	
𐬢	
𐬣	
𐬤	
𐬥	
𐬦	
𐬧	
𐬨	
𐬩	
𐬪	
𐬫	
𐬬	
𐬭	
𐬮	
𐬯	
𐬰	
𐬱	
𐬲	
𐬳	
𐬴	
𐬵	
\ifxetex\endR
\TeXXeTstate=0\fi
\end{scriptexample}

The recent Google font \url{NotoSansAvestan-Regular_0.ttf} can be used to typeset the Avestan script, but really not suitable for any serious study of the language.
\subsection{Old Turkic}

\newfontfamily\oldturkic{Segoe UI Symbol}
\begin{scriptexample}[]{Old Turkish}
\oldturkic
\obeylines
Orkhon	Yenisei
variants	Transliteration / transcription
Old Turkic letter  𐰀	𐰁 𐰂	a, ä
Old Turkic letter  𐰃	𐰄 𐰅	y, i (e)
Old Turkic letter  𐰆		o, u
Old Turkic letter  𐰇	𐰈	ö, ü

	0	1	2	3	4	5	6	7	8	9	A	B	C	D	E	F
U+10C0x	𐰀	𐰁	𐰂	𐰃	𐰄	𐰅	𐰆	𐰇	𐰈	𐰉	𐰊	𐰋	𐰌	𐰍	𐰎	𐰏
U+10C1x	𐰐	𐰑	𐰒	𐰓	𐰔	𐰕	𐰖	𐰗	𐰘	𐰙	𐰚	𐰛	𐰜	𐰝	𐰞	𐰟
U+10C2x	𐰠	𐰡	𐰢	𐰣	𐰤	𐰥	𐰦	𐰧	𐰨	𐰩	𐰪	𐰫	𐰬	𐰭	𐰮	𐰯
U+10C3x	𐰰	𐰱	𐰲	𐰳	𐰴	𐰵	𐰶	𐰷	𐰸	𐰹	𐰺	𐰻	𐰼	𐰽	𐰾	𐰿
U+10C4x	𐱀	𐱁	𐱂	𐱃	𐱄	𐱅	𐱆	𐱇	𐱈	

\hfill  Typeset with \texttt{Segoe UI Symbol} \cmd{\oldturkic} 
\end{scriptexample}

Irk Bitig or Irq Bitig (Old Turkic: {\bfseries\Large\oldturkic 𐰃𐰺𐰴 𐰋𐰃𐱅𐰃𐰏}), known as the Book of Omens or Book of Divination in English, is a 9th-century manuscript book on divination that was discovered in the "Library Cave" of the Mogao Caves in Dunhuang, China, by Aurel Stein in 1907, and is now in the collection of the British Library in London, England. The book is written in Old Turkic using the Old Turkic script (also known as "Orkhon" or "Turkic runes"); it is the only known complete manuscript text written in the Old Turkic script.[1] It is also an important source for early Turkic mythology.

The Old Turkic text does not have any sentence punctuation, but uses two black lines in a red circle as a word separation mark in order to indicate word boundaries as shown in Figure~{\ref{omen}}

\begin{figure}[htb]
\includegraphics[width=0.7\textwidth]{./images/omen.jpg}
\caption{Omen 11 (4-4-3 dice) of the Irk Bitig (folio 13a): "There comes a messenger on a yellow horse (and) an envoy on a dark brown horse, bringing good tidings, it says. Know thus: (The omen) is extremely good."}
\label{omen}
\end{figure}
\section{Runic}
\label{s:runic}
\newfontfamily\runic{NotoSansRunic-Regular.ttf}

Runes (Proto-Norse:{\runic ᚱᚢᚾᛟ }(runo), Old Norse: rún) are the letters in a set of related alphabets known as runic alphabets, which were used to write various Germanic languages before the adoption of the Latin alphabet and for specialised purposes thereafter. The Scandinavian variants are also known as futhark or fuþark (derived from their first six letters of the alphabet: F, U, Þ, A, R, and K); the Anglo-Saxon variant is futhorc or fuþorc (due to sound changes undergone in Old English by the names of those six letters)

\begin{scriptexample}[]{Runic}
 \unicodetable{runic}{"16A0,"16B0,"16C0,"16D0,"16E0,"16F0}
\end{scriptexample}


\printunicodeblock{./languages/runic.txt}{\runic}


\ifscriptolmec
  \section{Epi-Olmec}
\label{s:olmec}
Epi-Olmec is an ancient Mesoamerican logosyllabic script which has been deciphered by Terrence Kaufman and John Justeson. A complete description of the script has been described by \cite{kaufman}. The most famous inscription is on the Tuxtla Statuette. The Tuxtla Statuette is a small 6.3 inch (16 cm) rounded greenstone figurine, carved to resemble a squat, bullet-shaped human with a duck-like bill and wings. Most researchers believe the statuette represents a shaman wearing a bird mask and bird cloak.[1] It is incised with 75 glyphs of the Epi-Olmec or Isthmian script, one of the few extant examples of this very early Mesoamerican writing system. The Tuxtla Statuette is particularly notable in that its glyphs include the Mesoamerican Long Count calendar date of March 162 CE, which in 1902 was the oldest Long Count date discovered. A product of the final century of the Epi-Olmec culture, the statuette is from the same region and period as La Mojarra Stela 1 and may refer to the same events or persons.[3] Similarities between the Tuxtla Statuette and Cerro de las Mesas Monument 5, a boulder carved to represent a semi-nude figure with a duckbill-like buccal mask, have also been noted.[4]

\begin{figure}[ht]
\centering
\includegraphics[height=0.35\textheight]{./images/tuxtla-statuette.png}\hspace{1em} 
\includegraphics[height=0.35\textheight]{./images/tuxtla-statuette-01.jpg}
\caption{Frontal view of the Tuxtla Statuette. Note the Mesoamerican Long Count calendar date of March 162 CE (8.6.2.4.17) down the front of the statuette. The left figure is from wikipedia and the right from the original \protect\href{http://www.readcube.com/articles/10.1525/aa.1907.9.4.02a00030}{Holmes} paper.}
\end{figure}

\subsection{The epiolmec package}

The script has not been as yet encoded as by the Unicode consortium. Syropoulos \citep{syropoulos} created a font for the script and also wrote an article for TUGboat. Interestingly the paper describes the procedure used to develop the font. The package \pkgname{epiolmec} which is available both in \TeX live and Mik\tex, provides commands to access the glyphs. It is also possibly easier to typeset the script using traditional \latexe techniques, as they provide transcription commands rather than using a unicode font with the glyphs allocated in the private area directly.

\begin{verbatim}
\documentclass{article}
\usepackage{epiolmec,multicol}
\begin{document}
  \begin{center}
      \begin{minipage}{80pt}
      \begin{multicols}{3}
         \EOku\\ \EOji\\  
         \EOtze\\ \EOstep \\
       \end{multicols}    
     \end{minipage}       
  \end{center}
\end{document}
\end{verbatim}

Since the Epi-Olmec script is a logosyllagraphy we
need some practical way to access the symbols of the
script. Originally Syropoulos used the Ω translation
process that mapped words and “syllables” to the
corresponding glyphs of the font. In this way one obtains
a natural way for typing in Epi-Olmec texts. In addition,
in order to avoid the problem mentioned above,
he used a wrapper that typesets the text vertically.
For short texts \cmd{\shortstacks} is adequate, while
for longer texts, he used a |multicols| environment
inside a relatively narrow minipage. 

\begin{scriptexample}{Epi-Olmec}
\bgroup
\HUGE
\centering
\EOpi   \EOofficerI \EOofficerII \EOofficerIII

\captionof{figure}{The output of \string\EOpi, \string\EOofficerI, \string\EOofficerII\ and \string\EOofficerIII\ commands. }
\egroup
\end{scriptexample}

\subsection{Numbering System}\index{Epi-Olmec>vigesimal system}

The Epi-Olmec people used the same numbering system  
 as the Maya. Their numbering system was a vigesimal system and
 the digits were written in a top-down fashion. Thus, we need a macro
 that will typeset numbers in this fashion when it is used with \LaTeX\
 (actually $\epsilon$-\LaTeX). In addition, we need a macro that will
 just output the vigesimal digits. Such a macro could be used with
 $\Lambda$ with the |LTL| text and paragraph directions. To recapitulate,
 we need to define two macros that will basically typeset vigesimal numbers
 in either horizontal or vertical mode.

 For the various calculations that are performed, we need at least three
 counter variables. The fourth is needed for the macro that typesets the
 vigesimal numbers vertically and its usage is explained below. 

\begin{scriptexample}{EpicOtmec}
\def\textb#1{\text{\makebox[6em]{\hss#1~~   \protect\string#1\hfill}}}
\begin{multicols}{3}
\bgroup
\parindent0pt
$\textb{\EOzero}=0$\\
$\textb{\EOi} = 1$\\
$\textb{\EOii} = 2$\\
$\textb{\EOiii} =3$\\
$\textb{\EOiv}  =4$\\
$\textb{\EOv}   =5$\\
$\textb{\EOvi}  =6$\\
$\textb{\EOvii} =7$\\
$\textb{\EOviii} =8$\\
$\textb{\EOix} =9$\\ 
$\textb{\EOx} =10$\\
$\textb{\EOxi} =11$\\
$\textb\EOxii =12$\\
$\textb{\EOxiii} =13$\\
$\textb{\EOxiv} =14$\\
$\textb{\EOxv} =15$\\
$\textb{\EOxvi} =16$\\
$\textb\EOxvii =17$\\
$\textb{\EOxviii} =18$\\
$\textb{\EOxix} =19$\\
$\textb{\EOxx} =20$\\
\egroup
\end{multicols}
\end{scriptexample}


%% TODO add to index all symbols

\begin{multicols}{4}
\bgroup
\def\K#1{\makebox[3em]{{\color{blue}\hss#1\hfill}} \string#1}
\parindent0pt
\K\EOSpan\\ 
\K\EOJI \\
\K\EOvarji\\ 
\K\EOvarki \\
\K\EOpi \\
\K\EOpe \\
\K\EOpuu \\
\K\EOpa \\
\K\EOvarpa\\ 
\K\EOpu \\
\K\EOpo \\
\K\EOti \\
\K\EOte \\
\K\EOtuu \\
\K\EOta \\
\K\EOtu \\
\K\EOto \\
\K\EOtzi \\
\K\EOtze \\
\K\EOtzuu \\
\K\EOtza \\
\K\EOvartza\\ 
\K\EOtzu \\
\K\EOki \\
\K\EOke \\
\K\EOkuu \\
\K\EOvarkuu\\ 
\K\EOku\\ 
\K\EOko \\
\K\EOSi \\
\K\EOvarSi\\ 
\K\EOSuu \\
\K\EOSa \\
\K\EOSu \\
\K\EOSo \\
\K\EOsi \\
\K\EOvarsi\\ 
\K\EOsuu \\
\K\EOsa \\
\K\EOsu \\
\K\EOji \\
\K\EOje \\
\K\EOja \\
\K\EOvarja\\ 
\K\EOju \\
\K\EOjo \\
\K\EOmi \\
\K\EOme \\
\K\EOmuu \\
\K\EOma \\
\K\EOni \\
\K\EOvarni\\
\K\EOne \\
\K\EOnuu \\
\K\EOna \\
\K\EOnu \\
\K\EOwi \\
\K\EOwe \\
\K\EOwuu \\
\K\EOvarwuu\\
\K \EOwa\\
\K\EOwo \\
\K\EOye \\
\K\EOyuu \\
\K\EOya \\
\K\EOkak \\
\K\EOpak \\
\K\EOpuuk\\
\K\EOyaj \\
\K\EOScorpius\\
\K\EODealWith\\
\K\EOYear \\
\K\EOBeardMask \\
\K\EOBlood \\
\K\EOBundle \\
\K\EOChop \\
\K\EOCloth \\
\K\EOSaw \\
\K\EOGuise \\
\K\EOofficerI\\
\K\EOofficerII \\
\K\EOofficerIII \\
\K\EOofficerIV \\
\K\EOKing \\
\K\EOloinCloth \\
\K\EOlongLipII \\
\K\EOLose \\
\K\EOmexNew \\
\K\EOMiddle \\
\K\EOPlant \\
\K\EOPlay \\
\K\EOPrince \\
\K\EOSky \\
\K\EOskyPillar \\
\K\EOSprinkle \\
\K\EOstarWarrior\\
\K\EOTitleII \\
\K\EOtuki \\
\K\EOtzetze\\
\K\EOChronI \\
\K\EOPatron \\
\K\EOandThen\\
\K\EOAppear \\
\K\EODeer \\
\K\EOeat \\
\K\EOPatronII \\
\K\EOPierce \\
\K\EOkij \\
\K\EOstar  \\
\K\EOsnake \\
\K\EOtime \\
\K\EOtukpa  \\
\K\EOflint \\
\K\EOafter \\
\K\EOvarBeardMask \\
\K\EOBedeck \\
\K\EObrace \\
\K\EOflower  \\
\K\EOGod \\
\K\EOMountain \\
\K\EOgovernor \\
\K\EOHallow \\
\K\EOjaguar \\
\K\EOSini \\
\K\EOknottedCloth \\
\K\EOknottedClothStraps \\
\K\EOLord \\
\K\EOmacaw \\
\K\EOmonster \\
\K\EOmacawI \\
\K\EOskyAnimal\\
\K\EOnow \\
\K\EOTitleIV \\
\K\EOpenis \\
\K\EOpriest  \\
\K\EOstep\\
\K\EOsing \\
\K\EOskin \\
\K\EOStarWarrior \\
\K\EOsun \\
\K\EOthrone\\
\K\EOTime \\
\K\EOHallow \\
\K\EOTitle \\
\K\EOturtle \\
\K\EOundef \\
\K\EOGoUp \\
\K\EOLetBlood \\
\K\EORain \\
\K\EOset \\
\K\EOvarYear\\
\K\EOFold \\
\K\EOsacrifice \\
\K\EObuilding \\
\egroup
\end{multicols} 

\subsection{Technical}

The font is defined with the local encoding \texttt{LEO}. 

\begin{verbatim}
\DeclareFontEncoding{LEO}{}{}
\DeclareFontSubstitution{LEO}{cmr}{m}{n}
\DeclareFontFamily{LEO}{cmr}{\hyphenchar\font=-1}
\end{verbatim}

Note the |\hyphenchar\font=-1| that disables hyphenation in the |\DeclareFontFamily|  declaration. You cannot behead the \EOofficerII\ in order to hyphenate the text!



\fi

    \cxset{steward,
  numbering=arabic,
  custom=stewart,
  offsety=0cm,
  image={asia.jpg},
  texti={An introduction to the use of font related commands. The chapter also gives a historical background to font selection using \tex and \latex. },
  textii={In this chapter we discuss keys that are available through the \texttt{phd} package and give a background as to how fonts are used
in \latex.
 },
 pagestyle = empty
}

\arial


\chapter{South Asian Scripts}

The scripts of South Asia share so many characteristics that a side by side comparison of a few often reveal structural similarities even in the 
modern letterforms.
\medskip

\begin{center}
\begin{tabular}{lll}
Devanagari. &Gujarati &Telugu\\
Bengali   &Oriya &Kannada\\
Gurmukhi &Tamil  &Malayalam\\
Sinhala &Kaithi  &Meetei Mayek\\
Tibetan &Saurashtra &Ol Chiki.\\
Lepcha  &Sharada &Sora Sompeng\\
Phags-pa &Takri &Kharoshthi\\
Limbu &Chakma & Brahmi\\
Syloti Nagri & &\\
\end{tabular}
\end{center}

The sections that follow describe the scripts briefly and the |phd| settings
to activate the relevant commands and load appropriate fonts. 

\section{Devanagari}
\parindent1em

Devanagari is part of the Brahmic family of scripts of India, Nepal, Tibet, and South-East Asia.[2] It is a descendant of the Gupta script, along with Siddham and Sharada.[2] Eastern variants of Gupta called nāgarī are first attested from the 7th century CE; from c. 1200 CE these gradually replaced Siddham, which survived as a vehicle for Tantric Buddhism in East Asia, and Sharada, which remained in parallel use in Kashmir. An early version of Devanagari is visible in the Kutila inscription of Bareilly dated to Vikram Samvat 1049 (i.e. 992 CE), which demonstrates the emergence of the horizontal bar to group letters belonging to a word.[3]

Sanskrit nāgarī is the feminine of nāgara "relating or belonging to a town or city". It is feminine from its original phrasing with lipi ("script") as nāgarī lipi "script relating to a city", that is, probably from its having originated in some city.[4]

The use of the name devanāgarī is relatively recent, and the older term nāgarī is still common.[2] The rapid spread of the term devanāgarī may be related to the almost exclusive use of this script to publish Sanskrit texts in print since the 1870s.[2]

On Windows use \texttt{Arial Unicode MS}. 
\medskip

\newfontfamily\devanagari[Script=Devanagari,Scale=1.5]{Arial Unicode MS}

\begin{scriptexample}[]{Devanagari}
{\begin{center}\parindent0pt\devanagari

ंःअआइईउऊऋऌऍऎएऐऑऒओऔऔँ \par 

ी	ु	ू	ृ	ॄ	ॅ	ॆ	े	ै	ॉ	ॊ	ो	ौ	्	\par

\bigskip		
\begin{tabular}{lll lll lll l}
०	&१	&२	&३	&४	&५	&६	&७	&८	&९\\
0	&1	&2	&3	&4	&5	&6	&7	&8	&9\\
\end{tabular}
\end{center}	
}
\end{scriptexample}


On Linux \texttt{Lohit} is a font family designed to cover Indic scripts and released by Red Hat. The Lohit fonts currently cover 11 languages: Assamese, Bengali, Gujarati, Hindi, Kannada, Malayalam, Marathi, Oriya, Punjabi, Tamil, Telugu.[1] The fonts were supplied by Modular Infotech and licensed under the GPL. In September 2011, they were retroactively relicensed under the OFL.[2] The Lohit fonts are used as web fonts by some Wikimedia Foundation sites, like Wikipedia, since March 2012.The font currently support 21 Indian languages. 

\newfontfamily\devanagarilohit[Script=Devanagari,Scale=1.5]{Lohit-Devanagari.ttf}

\begin{scriptexample}[]{Devanagari}
\begin{center}\parindent0pt\devanagarilohit

ंःअआइईउऊऋऌऍऎएऐऑऒओऔऔँ \par 

ी	ु	ू	ृ	ॄ	ॅ	ॆ	े	ै	ॉ	ॊ	ो	ौ	्	\par

\bigskip		
\begin{tabular}{lll lll lll l}
०	&१	&२	&३	&४	&५	&६	&७	&८	&९\\
0	&1	&2	&3	&4	&5	&6	&7	&8	&9\\
\end{tabular}
\end{center}
\end{scriptexample}

\subsubsection{Punctuation} 
The end of a sentence or half-verse may be marked with a dot known as a pūrna virām or a vertical line danda: \textbar. The end of a full verse may be marked with two vertical lines: \textbar\textbar. A comma, or alpa virām, is used to denote a natural pause in speech. With expansion of English speakers in India, the full stop is also sometimes used.

\subsection{LaTeX support}

\latex2e support can be found in the \pkgname{sanskrit}. The package contains the font files and pre-processor for printing Sanskrit
text in both devanāgarī and transliterated Roman with diacritics. Another package that can be used with \XeTeX\ is support \pkgname{devnag}.  This was originally developed by Frans Velthuis for the University of Groningen, The Netherlands, and it was the first system to provide
support for the script for \tex. The package was  extended by Anshuman Pandey. The package provides both fonts as well as tranliteration macros.


\subsection{Gujarati}


Gujarati has its own writing system, distinct but related to several other Indian languages' writing systems, such as the one used to write Hindi. Strictly speaking, the Gujarati writing system is what is called an \emph{abugida} (and not an \textit{alphabet}), because the consonant characters all contain an inherent vowel, and other vowels are written as accents added on to the consonant characters. There are also symbols for stand-alone vowels.

The Gujarati script ({\gujarati{ગુજરાતી લિપિ }} Gujǎrātī Lipi), which like all Nāgarī writing systems is strictly speaking an abugida rather than an alphabet, is used to write the Gujarati and Kutchi languages. It is a variant of Devanāgarī script differentiated by the loss of the characteristic horizontal line running above the letters and by a small number of modifications in the remaining characters.
With a few additional characters, added for this purpose, the Gujarati script is also often used to write Sanskrit and Hindi.
Gujarati numerical digits are also different from their Devanagari counterparts.
\medskip

\bgroup
\newfontfamily\gujaratilohit[Script=Gujarati,Scale=1.5]{Lohit-Gujarati.ttf}
\gujarati

\centering

English/Hindi/Gujarati Alphabets

\begin{tabular}{lllllllllllllllllllll}
A &B &bh &C &ch &chh &D &dh &E &F &G &gh &H &I &J &K &kh &L &M &N &O\\

अ &ब &भ &क &च &छ &ड/द &ध/ढ़ &इ &फ &ग &घ &ह &ई &ज &क &ख &ल &म &न/ण &ऑ\\

અ &બ &ભ &ક &ચ &છ &ડ/દ &ધ /ઢ &ઇ &ફ &ગ &ઘ &હ &ઈ &જ &ક &ખ &લ &મ &ન/ણ &ઓ\\

\end{tabular}
\egroup

\medskip

Gujarati has its own set of numeric signs (placed alongside their Hindu-Arabic [or Indo-Arabic] counterparts in the tables below), they are employed in much the same way as English;  that is to say, they are put together in the same manner in order to express larger numbers. It is quite possible to simply substitute the Gujarati numerals for the Hindu-Arabic ones.

The Gujarati words for 1-10 are as follows:
\medskip

\bgroup
\begin{center}
\gujarati
\begin{tabular}{ccl}
Arabic & Gujarati &Name\\
Numeral &Numeral  &\\
0	&૦	&mīṇḍuṃ or shunya\\
1	&૧	&ekaṛo or ek\\
2	&૨	&bagaṛo or bay\\
3	&૩	&tragaṛo or tran\\
4	&૪	&chogaṛo or chaar\\
5	&૫	&pāchaṛo or paanch\\
6	&૬	&chagaṛo or chah\\
7	&૭	&sātaṛo or sāt\\
8	&૮	&āṭhaṛo or āanth\\
9	&૯	&navaṛo or nav\\
10 &૧૦ &દસ das\\

\end{tabular}
\end{center}
\egroup

\subsection{Bengali}

There are two Windows fonts that can be used with Windows \textit{Shonar Bangla} and \textit{Vrinda}. For open source fonts one can use, \textit{code2000}.
\bigskip

\bgroup
\newfontfamily\bengali[Script=Bengali,Scale=4]{Shonar Bangla}


\bengali
\centering

  অ  আ ই  ঈ  উ  ঊ  ঋ  এ  ঐ\par

\fontspec[Script=Bengali,Scale=3.2]{Vrinda}

\centering

  অ  আ ই  ঈ  উ  ঊ  ঋ  এ  ঐ\par


\fontspec[Script=Bengali,Scale=3.2]{code2000.ttf}

\centering

  অ  আ ই  ঈ  উ  ঊ  ঋ  এ  ঐ\par

\captionof{table}{The consonant{\protect\bengal{} ক (kô)} along with the diacritic form of the vowels {\protect\bengal{} অ, আ, ই, ঈ, উ, ঊ, ঋ, এ, ঐ, ও and ঔ} \textit{from Wikipedia}.}
\egroup

\subsection{Saurashtra}

\newfontfamily\saurashtra{code2000.ttf}

Saurashtra or Sourashtra or {\saurashtra ꢱꣃꢬꢵꢰ꣄ꢜ꣄ꢬꢵ} or Palkar or Patkar (Sanskrit: सौराष्ट्र, Tamil: சௌராட்டிரம்) is an Indo-Aryan language[3] spoken by the Saurashtrian community native to Gujarat, who migrated and settled in Southern India. Madurai in Tamil Nadu has the highest number of people belonging to this community and also remains as their cultural center.

The language is largely only in spoken form even though the language has its own script. The lack of schools teaching Saurashtra script and the language is often cited as a reason for the very few number of people who actually know to read and write in Saurashtra script. Latin, Devanagari or Tamil script is used as alternative for Saurashtra Script by many Saurashtrians.

Census of India places the language under Gujarati. Official figures show the number of speakers as 185,420 (2001 census).[4]



\begin{scriptexample}[]{Saurashtra}
\bgroup
\saurashtra

ꢮꢶꢯ꣄ꢮ ꢱꣃꢬꢵꢰ꣄ꢜ꣄ꢬꢪ꣄ ꢦꢡ꣄ꢬꢶꢒꢾ ꢱꢵꢡ꣄ꢡꢒꢸ ꢂꢮꢬꢾ
ꢮꣁꢭꢱ꣄ꢢꢵꢥꢪꢸꢒ꣄(ꣀꢵꢮꢾꢔꢹ ꢂꢮ꣄ꢬꢶꢫꣁ


\arial

Text: Vishwa Sourashtram \url{http://www.sourashtra.info/ghEr.htm}
\egroup
\end{scriptexample}

\subsection{Ol Chiki script}

The Ol Chiki script, also known as Ol Cemetʼ (Santali: ol 'writing', cemet' 'learning'), Ol Ciki, Ol, and sometimes as the Santali alphabet, was created in 1925 by Raghunath Murmu for the Santali language.

Previously, Santali had been written with the Latin alphabet. But because Santali is not an Indo-Aryan language (like most other languages in the south of India), Indic scripts did not have letters for all of Santali's phonemes, especially its stop consonants and vowels, which made writing the language accurately in an unmodified Indic script difficult. The detailed analysis was given by Dr. Byomkes Chakrabarti in his 'Comparative Study of Santali and Bengali'. Missionaries (first of all Paul Olaf Bodding, a Norwegian) brought the Latin script, which is better at representing Santali stops, phonemes and nasal sounds with the use of diacritical marks and accents. Unlike most Indic scripts, which are derived from Brahmi, Ol Chiki is not an abugida, with vowels given equal representation with consonants. Additionally, it was designed specifically for the language, but one letter could not be assigned to each phoneme because the sixth vowel in Ol Chiki is still problematic.
Ol Chiki has 30 letters, the forms of which are intended to evoke natural shapes. Linguist Norman Zide said "The shapes of the letters are not arbitrary, but reflect the names for the letters, which are words, usually the names of objects or actions representing conventionalized form in the pictorial shape of the characters."[1] It is written from left to right.

\newfontfamily\olchiki{code2000.ttf}

\begin{scriptexample}[]{olchiki}
\bgroup
\olchiki
\obeylines

U+1C5x 	᱐	᱑	᱒	᱓	᱔	᱕	᱖	᱗	᱘	᱙	ᱚ	ᱛ	ᱜ	ᱝ	ᱞ	ᱟ
U+1C6x	   ᱠ	ᱡ	ᱢ	ᱣ	ᱤ	ᱥ	ᱦ	ᱧ	ᱨ	ᱩ	ᱪ	ᱫ	ᱬ	ᱭ	ᱮ	ᱯ
U+1C7x  	ᱰ	ᱱ	ᱲ	ᱳ	ᱴ	ᱵ	ᱶ	ᱷ	ᱸ	ᱹ	ᱺ	ᱻ	ᱼ	ᱽ	᱾	᱿
\egroup
\end{scriptexample}

\subsection{Lepcha}
\newfontfamily\lepcha{Mingzat-R.ttf}

The Lepcha script, or Róng script is an abugida used by the Lepcha people to write the Lepcha language. Unusually for an abugida, syllable-final consonants are written as diacritics.

The Mingzat font is still under development by SIL so I am not too sure if the rendering is correct\footnote{\url{http://scripts.sil.org/cms/scripts/page.php?site_id=nrsi&id=Mingzat}}.

\begin{scriptexample}[]{Lepcha}
\bgroup
\lepcha
\obeylines
 	    0	1	2	3	4	5	6	7	8	9	A	B	C	D	E	F
U+1C0x	 ᰀ	ᰁ	ᰂ	ᰃ	ᰄ	ᰅ	ᰆ	ᰇ	ᰈ	ᰉ	ᰊ	ᰋ	ᰌ	ᰍ	ᰎ	ᰏ
U+1C1x	 ᰐ	ᰑ	ᰒ	ᰓ	ᰔ	ᰕ	ᰖ	ᰗ	ᰘ	ᰙ	ᰚ	ᰛ	ᰜ	ᰝ	ᰞ	ᰟ
U+1C2x	 ᰠ	ᰡ	ᰢ	ᰣ	ᰤ	ᰥ	ᰦ	ᰧ	ᰨ	ᰩ	ᰪ	ᰫ	ᰬ	ᰭ	ᰮ	ᰯ
U+1C3x	 ᰰ	ᰱ	ᰲ	ᰳ	ᰴ	ᰵ	ᰶ	᰷	x	x	x	᰻	᰼	᰽	᰾	᰿
U+1C4x	 ᱀	᱁	᱂	᱃	᱄	᱅	᱆	᱇	᱈	᱉	x	x	x	ᱍ	ᱎ	ᱏ

\egroup
\end{scriptexample}

\subsection{Sharada}

The Śāradā, or Sharada, script (शारदा) is an abugida writing system of the Brahmic family of scripts, developed around the 8th century. It was used for writing Sanskrit and Kashmiri. The Gurmukhī script was developed from Śāradā. Originally more widespread, its use became later restricted to Kashmir, and it is now rarely used except by the Kashmiri Pandit community for ceremonial purposes. Śāradā is another name for Saraswati, the goddess of learning.
Śāradā script was added to the Unicode Standard in January, 2012 with the release of version 6.1.

The Unicode block for Śāradā script, called Sharada, is U+11180–U+111DF: Unable to locate font in unicode.


\subsection{Sora Sompeng}

Sorang Sompeng script is used to write in Sora, a Munda language with 300,000 speakers in India. The script was created by Mangei Gomango in 1936 and is used in religious contexts.[1] He was familiar with Oriya, Telugu and English, so the parent systems of the script are Brahmi and Latin.[2]
The Sora language is also written in the Latin alphabet and the Telugu script.

Sorang Sompeng script was added to the Unicode Standard in January, 2012 with the release of version 6.1. Nirmala UI.ttf (Windows 8.1)



\unicodetable{arial}{"110D0,"110E0,"110F0}
 	
This did not work with Windows 7, and the experiment failed. 

\subsection{Phags-pa}

The 'Phags-pa script,[1], (Mongolian: дөрвөлжин үсэг "Square script") was an alphabet designed by the Tibetan monk and vice-king Drogön Chögyal Phagpa for the Mongol Yuan emperor Kublai Khan as a unified script for the literary languages of the Yuan. Widespread use was limited to about a hundred years during the Yuan Dynasty, and it fell out of use with the advent of the Ming dynasty. The documentation of its use provides clues about the changes in the varieties of Chinese, the Tibetic languages, Mongolian and other neighboring languages during the Yuan era.

\newfontfamily\phagspa{code2000.ttf}

\begin{scriptexample}[]{Phags-pa}
\bgroup
\obeylines
\phagspa

 	0	1	2	3	4	5	6	7	8	9	A	B	C	D	E	F
U+A84x	ꡀ	ꡁ	ꡂ	ꡃ	ꡄ	ꡅ	ꡆ	ꡇ	ꡈ	ꡉ	ꡊ	ꡋ	ꡌ	ꡍ	ꡎ	ꡏ
U+A85x	ꡐ	ꡑ	ꡒ	ꡓ	ꡔ	ꡕ	ꡖ	ꡗ	ꡘ	ꡙ	ꡚ	ꡛ	ꡜ	ꡝ	ꡞ	ꡟ
U+A86x	ꡠ	ꡡ	ꡢ	ꡣ	ꡤ	ꡥ	ꡦ	ꡧ	ꡨ	ꡩ	ꡪ	ꡫ	ꡬ	ꡭ	ꡮ	ꡯ
U+A87x	ꡰ	ꡱ	ꡲ	ꡳ	꡴	꡵	꡶	


ꡏꡟ ꡋꡞ ꡏꡟ ꡋꡞ ꡏ ꡜꡖ ꡏꡟ ꡋꡞ ꡓꡞ ꡏꡟ
ꡈꡋ ꡋꡋ ꡓꡘ ꡈ ꡭ ꡏ ꡏ ꡝ ꡭꡟꡘ ꡓꡋ ꡮꡟꡊ
\egroup
\bgroup
\raggedright

\setcounter{glyphcount}{"A840}

\topline
\phagspa
\newcount\n
\n="A840

\def\htable{^^A
  \def\fm##1{\makebox[2em]##1}^^A
  U+A840\fm 0\fm1\fm2\fm3\fm4\fm5\fm 6\fm 7\fm 8\fm	9\fm A\fm B\fm C\fm D\fm E\fm F}

\htable\par
U+A840^^A 
\loop^^A
  \makebox[2em]{\char\n }^^A   
   \advance\n by1 ^^A
   \ifnum\n<"A850^^A
\repeat
\par U+A850^^A
\loop^^A
  \makebox[2em]{\char\n }^^A   
   \advance\n by1 ^^A
  \ifnum\n<"A860^^A
\repeat
\par U+A860^^A
\loop^^A
  \makebox[2em]{\char\n }^^A   
   \advance\n by1 ^^A
  \ifnum\n<"A870^^A
\repeat
\par U+A870^^A
\loop^^A
  \makebox[2em]{\char\n }^^A   
   \advance\n by1 ^^A
  \ifnum\n<"A878^^A
\repeat

\bottomline

\arial
\hfill Typeset with \texttt{code2000.ttf} and \cmd{\phagspa}

Text: \href{http://babelstone.blogspot.com/2006/12/phags-pa-fonts-1-babelstone-phags-pa.html}{babelstone}
\egroup
\end{scriptexample}

Phags-pa is a historical script related to Tibetan that was created as the national script of
the Mongol empire. Even though Phags-pa was used mostly in Eastern and Central Asia for
writing text in the Mongolian and Chinese languages, it is discussed in this chapter because
of its close historical connection to the Tibetan script. The script has very limited modern use. It bears similarity to Tibetan and has no case distinctions. It is written vertically in columns running for left to right, like Mongolian. Units are often composed of several syllables and sometimes are separated by whitespace.


\subsection{Syloti Nagri}
\index{languages>Sylheti Nagari}
Sylheti Nagari or Syloti Nagri (Silôṭi Nagôri) is the original script used for writing the Sylheti language. It is an almost extinct script, this is because the Sylheti Language itself was reduced to only dialect status after Bangladesh gained independence and because it did not make sense for a dialect to have its own script, its use was heavily discouraged. The government of the newly formed Bangladesh did so to promote a greater "Bengali" identity. This led to the informal adoption of the Eastern Nagari script also used for Bengali and Assamese. It is also known as Jalalabadi Nagri, Mosolmani Nagri, Ful Nagri etc.

\newfontfamily\syloti{NotoSansSylotiNagri-Regular.ttf}
\newfontfamily\damase{damase_v.2.ttf}
\bgroup
\damase
\obeylines
	0	1	2	3	4	5	6	7	8	9	A	B	C	D	E	F
U+A80x	ꠀ	ꠁ	ꠂ	ꠃ	ꠄ	ꠅ	꠆	ꠇ	ꠈ	ꠉ	ꠊ	ꠋ	ꠌ	ꠍ	ꠎ	ꠏ
U+A81x	ꠐ	ꠑ	ꠒ	ꠓ	ꠔ	ꠕ	ꠖ	ꠗ	ꠘ	ꠙ	ꠚ	ꠛ	ꠜ	ꠝ	ꠞ	ꠟ
U+A82x	ꠠ	ꠡ	ꠢ	ꠣ	ꠤ	ꠥ	ꠦ	ꠧ	꠨	꠩	꠪	꠫
\egroup

\subsection{Chakma}

\newfontfamily\chakma{RibengUni.ttf}

\bgroup
\chakma
𑄇𑄳𑄇 Kkā = 𑄇 Kā + 𑄳 VIRAMA + 𑄇 Kā
𑄇𑄳𑄑 Ktā = 𑄇 Kā + 𑄳 VIRAMA + 𑄑 Tā
𑄇𑄳𑄖 Ktā = 𑄇 Kā + 𑄳 VIRAMA + 𑄖 Tā
𑄇𑄳𑄟 Kmā = 𑄇 Kā + 𑄳 VIRAMA + 𑄟 Mā
𑄇𑄳𑄌 Kcā = 𑄇 Kā + 𑄳 VIRAMA + 𑄌 Cā
𑄋𑄳𑄇 ńkā = 𑄋 ńā + 𑄳 VIRAMA + 𑄇 Kā
𑄋𑄳𑄉 ńkā = 𑄋 ńā + 𑄳 VIRAMA + 𑄉 Gā
𑄌𑄳𑄌 ccā = 𑄌 cā + 𑄳 VIRAMA + 𑄌 Cā

\egroup

\subsection{Limbu}

The Limbu script is used to write the Limbu language. The Limbu script is an abugida derived from the Tibetan script. Limbu is a Tibeto-Burman language spoken mainly in Nepal,[3] significant communities in Bhutan, Sikkim, Darjeeling district, India by the Limbu community. Virtually all Limbus are bilingual in Nepali.

\newfontfamily\limbu{code2000.ttf}
\bgroup
\obeylines
\limbu
0	1	2	3	4	5	6	7	8	9	A	B	C	D	E	F
U+190x	ᤀ	ᤁ	ᤂ	ᤃ	ᤄ	ᤅ	ᤆ	ᤇ	ᤈ	ᤉ	ᤊ	ᤋ	ᤌ	ᤍ	ᤎ	ᤏ
U+191x	ᤐ	ᤑ	ᤒ	ᤓ	ᤔ	ᤕ	ᤖ	ᤗ	ᤘ	ᤙ	ᤚ	ᤛ	ᤜ	ᤝ	ᤞ	
U+192x	ᤠ	ᤡ	ᤢ	ᤣ	ᤤ	ᤥ	ᤦ	ᤧ	ᤨ	ᤩ	ᤪ	ᤫ				
U+193x	ᤰ	ᤱ	ᤲ	ᤳ	ᤴ	ᤵ	ᤶ	ᤷ	ᤸ	᤹	᤺	᤻				
U+194x	᥀				᥄	᥅	᥆	᥇	᥈	᥉	᥊	᥋	᥌	᥍	᥎	᥏
\egroup

\subsection{Brahmi}



Brāhmī is the modern name given to one of the oldest writing systems used in the Indian subcontinent and in Central Asia during the final centuries BCE and the early centuries CE. Like its contemporary, Kharoṣṭhī, which was used in what is now Afghanistan and Western Pakistan, Brahmi (native to north and central India) was an \emph{abugida}.

The best-known Brahmi inscriptions are the rock-cut edicts of Ashoka in north-central India, dated to 250–232 BCE. The script was deciphered in 1837 by James Prinsep, an archaeologist, philologist, and official of the East India Company.[1] The origin of the script is still much debated, with current Western academic opinion generally agreeing (with some exceptions) that Brahmi was derived from or at least influenced by one or more contemporary Semitic scripts, but a current of opinion in India favors the idea that it is connected to the much older and as-yet undeciphered Indus script

\subsection{Unicode [U+11000-U+1107F]}


\newfontfamily\brahmi{code2000.ttf}

\begin{scriptexample}[]{Brahmi}
\bgroup
\raggedleft
\brahmi

         
   

\arial
\hfill Text: Asokan Edict typeset with \texttt{NotoSansBrahmi-Regular.ttf} 
\egroup
\end{scriptexample}
    \chapter{Middle Eastern Scripts}

The scripts in this section have a common origin in the ancient Phoenician alphabet. They include:

\begin{center}
\begin{tabular}{ll}
Hebrew & Samaritan\\
Arabic & Thaana\\
Syriac &\\
\end{tabular}
\end{center}

The Hebrew script is used in Israel and for languages of the Diaspora. The Arabic script is
used to write many languages throughout the Middle East, North Africa, and certain parts
of Asia. The Syriac script is used to write a number of Middle Eastern languages. These
three also function as major liturgical scripts, used worldwide by various religious groups.

The Samaritan script is used in small communities in Israel and the Palestinian Territories
to write the Samaritan Hebrew and Samaritan Aramaic languages. The Thaana script is
used to write Dhivehi, the language of the Republic of Maldives, an island nation in the
middle of the Indian Ocean. 

Text in these scripts is written from right to left. Arabic and Syriac are cursive scripts even when typeset, unlike Hebrew, Samaritan  and Thaana, where letters are unconnected. Most letters in Arabic and Syriac assume different forms depending on their position in a word. Shaping rules are not required for Hebrew because only five letters have position-dependent forms, and these forms are separately encoded.

Historically, Middle Eastern  scripts did not write short vowels. In modern scripts they are represented  by marks positioned above or below a consonantal letter. Vowels and other
marks of pronunciation (“vocalization”) are encoded as combining characters, so support
for vocalized text necessitates use of composed character sequences. Yiddish, Syriac, and
Thaana are normally written with vocalization; Hebrew, Samaritan, and Arabic are usually written unvocalized. 

\section{Hebrew}
\newfontfamily\hebrew{Miriam}
\fontspec{Arial Unicode MS}
To properly typeset Hebrew texts you first need to choose an appropriate font and also set the directionality of the text. This
is done using the etex commands:

\CMDI{\beginL} and \CMDI{\beginR} 

For \XeTeX\ you also need to add near the top of your document |\TeXXeTstate=1|. The package \pkgname{bidi} can be used to set all parameters. Be warned that it redefines almost all of \latexe's commands, so for short mixed texts, I wouldn't recommend its usage. 



The Hebrew alphabet (Hebrew: אָלֶף־בֵּית עִבְרִי[a], alefbet ʿIvri ), known variously by scholars as the Jewish script, square script, block script, is used in the writing of the Hebrew language, as well as other Jewish languages, most notably Yiddish, Ladino, and Judeo-Arabic. There have been two script forms in use; the original old Hebrew script is known as the paleo-Hebrew script (which has been largely preserved, in an altered form, in the Samaritan script), while the present "square" form of the Hebrew alphabet is a stylized form of the Assyrian script. Various "styles" (in current terms, "fonts") of representation of the letters exist. There is also a cursive Hebrew script, which has also varied over time and place. On Windows you can use the \texttt{Miriam} font or \texttt{Arial Unicode MS} or \texttt{Miriam Fixed}.
\medskip

\topline

\bgroup\TeXXeTstate=1
\raggedleft\hebrew{}\beginR

הכתב הכנעני הקדום הלך והתפשט וסימניו היו מוכרים כל כך, עד כי המשתמשים בו התחילו "להתעצל" בהשלמת הציורים, והניחו כי הקורא יבין גם מתוך שרטוטים סכמתיים באיזו אות מדובר. כך, למשל, הפך הראש למשולש עם צוואר; כף היד מלאת האצבעות הפכה לשרטוט דל, ומהדג נותר רק הזנב. כשהעברים אמצו את הכתב הכנעני הם התקשו לזהות חלק מהציורים המקוריים והניחו למשל כי הסימן המתאר את המילה "זהה" הוא כלי נשק; שזנב הדג המשולש הוא דלת, ושדווקא הנחש הוא דג. כך נולדו שמותיהם העבריים של האותיות זי"ן, דל"ת ונו"ן (נון הוא דג, כמו אמנון, שפמנון וכו'). הציורים שהפכו לסימנים התגלגלו לכתבים נוספים, ואפילו ליוונית וללטינית. גם בכתב העברי המודרני ניתן לזהות המשך התפתחותי ברור מן הכתב הכנעני הקדום, והשתמרות שמות האותיות מקלה מאוד על פענוח המקור.


בתקופת בית שני, אומץ האלפבית הארמי לשימוש השפה העברית במקום האלפבית העברי העתיק, כאשר בזה האחרון נעשה שימוש מועט כגון כתיבת השמות הקדושים והטבעת מטבעות. עם הזמן, נעלם גם שימוש זה של הכתב העתיק. האלפבית העברי של ימינו הוא אפוא פיתוח של האלפבית הארמי ולא של הכתב העברי העתיק.	
{}

 לֹ֥א תִשָּׂ֛א

\endR


\egroup
\bottomline
\medskip

To make all paragraphs  RL use the \cmd{\everypar}\footnote{See discussions at \url{http://tex.stackexchange.com/questions/141867/minimal-bidi-for-typesetting-rl-text} and \url{http://www.tug.org/pipermail/xetex/2004-August/000697.html}}. 

\begin{verbatim}
\newbox\mybox \everypar{\setbox\mybox\lastbox\beginR\box\mybox}
\everypar={% at the start of each paragraph, do....
    \setbox0=\lastbox % save the paragraph indent, if any
    \beginR % set R-L direction
    \box0 % then re-insert the indent
	}
\end{verbatim}

The Hebrew alphabet has 22 letters, of which five have different forms when used at the end of a word. Hebrew is written from right to left. Originally, the alphabet was an abjad consisting only of consonants. Like other \textit{abjads}, such as the Arabic alphabet, means were later devised to indicate vowels by separate vowel points, known in Hebrew as niqqud. In rabbinic Hebrew, the letters א ה ו י are also used as matres lectionis to represent vowels. When used to write Yiddish, the writing system is a true alphabet (except for borrowed Hebrew words). In modern usage of the alphabet, as in the case of Yiddish (except that ע replaces ה) and to some extent modern Israeli Hebrew, vowels may be indicated. Today, the trend is toward full spelling with these letters acting as true vowels.


\subsection{Syriac}

\newfontfamily\syriac{Estrangelo Edessa}

Syriac /ˈsɪriæk/ ({\syriac{ܠܫܢܐ ܣܘܪܝܝܐ}} Leššānā Suryāyā) is a dialect of Middle Aramaic that was once spoken across much of the Fertile Crescent and Eastern Arabia.[1][2][5] Having first appeared as a script in the 1st century AD after being spoken as an unwritten language for five centuries,[6] Classical Syriac became a major literary language throughout the Middle East from the 4th to the 8th centuries,[7] the classical language of Edessa, preserved in a large body of Syriac literature.
It became the vehicle of Syriac Christianity and culture, spreading throughout Asia as far as the Indian Malabar Coast and Eastern China,[8] and was the medium of communication and cultural dissemination for Arabs and, to a lesser extent, Persians. Primarily a Christian medium of expression, Syriac had a fundamental cultural and literary influence on the development of Arabic,[9] which largely replaced it towards the 14th century.[3] Syriac remains the liturgical language of Syriac Christianity.
Syriac is a Middle Aramaic language, and, as such, it is a language of the Northwestern branch of the Semitic family. It is written in the Syriac alphabet, a derivation of the Aramaic alphabet.

\begin{scriptexample}[]{Syriac}
\unicodetable{syriac}{"0700,"0710,"0720,"0730,"0740}
\end{scriptexample}

The Syriac Abbreviation (a type of overline) can be represented with a special control character called the Syriac Abbreviation Mark (U+070F {\syriac \char"070F ܘ}).

\section{Samaritan}
\newfontfamily\samaritan{NotoSansSamaritan-Regular.ttf}

The Samaritan alphabet is used by the Samaritans for religious writings, including the Samaritan Pentateuch, writings in Samaritan Hebrew, and for commentaries and translations in Samaritan Aramaic and occasionally Arabic.

The Samaritans are, consider themselves to be the descendants of the Northern Tribes of Israel that were not sent into Assyrian captivity, and have continuously resided in the land of Israel.

The Torah Scroll of the Samaritans uses an alphabet that is very different from the one used on Jewish Torah Scrolls. According to the Samaritans themselves and Hebrew scholars, this alphabet is the original "Old Hebrew" alphabet.

Even as far back as 1691, this connection between the Samaritan and the "Old" Hebrew alphabets was made by Henry Dodwell; "[the Samaritans] still preserve [the Pentateuch] in the Old Hebrew characters."

Samaritan is a direct descendant of the Paleo-Hebrew alphabet, which was a variety of the Phoenician alphabet in which large parts of the Hebrew Bible were originally penned. All these scripts are believed to be descendants of the Proto-Sinaitic script. That script was used by the ancient Israelites, both Jews and Samaritans. The better-known "square script" Hebrew alphabet traditionally used by Jews is a stylized version of the Aramaic alphabet which they adopted from the Persian Empire (which in turn adopted it from the Arameans). 

After the fall of the Persian Empire, Judaism used both scripts before settling on the Aramaic form. For a limited time thereafter, the use of paleo-Hebrew (proto-Samaritan) among Jews was retained only to write the Tetragrammaton, but soon that custom was also abandoned.



ShofarRegular StamAshkenazCLM.ttf

\begin{scriptexample}[]{Samaritan}
\bgroup
\TeXXeTstate=1
\unicodetable{samaritan}{"0800,"0810,"0820,"0830}
\egroup
\TeXXeTstate=0
\end{scriptexample}

I battled to get an appropriate font for the Samaritan script and had to use the \idxfont{Noto Sans Samaritan} from Google


^^A\printunicodeblock{./languages/samaritan.txt}{\samaritan}


\url{http://www.ancient-hebrew.org/ahh/ahh.htm#_Toc314842274}




\section{Arabic}

\newfontfamily\arabian{Scheherazade-R.ttf}

The Arabic script is a writing system used for writing several languages of Asia and Africa, such as Arabic, Sorani and Luri Dialects of Kurdish language, Persian, Pashto and Urdu.[1] Even until the 16th century, it was used to write some texts in Spanish.[2] After the Latin script, Chinese characters, and Devanagari, it is the fourth-most widely used writing system in the world.[3]
The Arabic script is written from right to left in a cursive style. In most cases the letters transcribe consonants, or consonants and a few vowels, so most Arabic alphabets are abjads.

The script was first used to write texts in Arabic, most notably the Qurʼān, the holy book of Islam. With the spread of Islam, it came to be used to write languages of many language families, leading to the addition of new letters and other symbols, with some versions, such as Kurdish, Uyghur, and old Bosnian being abugidas or true alphabets. It is also the basis for a rich tradition of Arabic calligraphy.

\begin{verbatim}
\begin{Arabic}
ّ هو إذ الغاية؛ شريف الفوائد، جم المذهب، عزيز فنّ التاريخ فنّ أنّ اعلم
والملوك سيرهم، في والأنبياء أخلاقهم، في الأمم من الماضين أحوال على يوقفنا
ّ أحوال في يرومه لمن ذلك في الإقتداء فائدة تتم حتّى وسياستهم؛ دولهم في
والدنيا. الدين
\end{Arabic}
\end{verbatim}




As of Unicode 7.0, the Arabic script is contained in the following blocks:
Arabic (0600—06FF, 255 characters)
Arabic Supplement (0750—077F, 48 characters)
Arabic Extended-A (08A0—08FF, 39 characters)
Arabic Presentation Forms-A (FB50—FDFF, 608 characters)
Arabic Presentation Forms-B (FE70—FEFF, 140 characters)
Rumi Numeral Symbols (10E60—10E7F, 31 characters)
Arabic Mathematical Alphabetic Symbols (1EE00—1EEFF, 143 characters)[1][2]

The basic Arabic range encodes the standard letters and diacritics, but does not encode contextual forms (U+0621–U+0652 being directly based on ISO 8859-6); and also includes the most common diacritics and Arabic-Indic digits. The Arabic Supplement range encodes letter variants mostly used for writing African (non-Arabic) languages. The Arabic Extended-A range encodes additional Qur'anic annotations and letter variants used for various non-Arabic languages. The Arabic Presentation Forms-A range encodes contextual forms and ligatures of letter variants needed for Persian, Urdu, Sindhi and Central Asian languages. The Arabic Presentation Forms-B range encodes spacing forms of Arabic diacritics, and more contextual letter forms. The presentation forms are present only for compatibility with older standards, and are not currently needed for coding text.[3] 

The Arabic Mathematical Alphabetical Symbols block encodes characters used in Arabic mathematical expressions.

\begin{multicols}{3}
\printunicodeblock{./languages/arabic.txt}{\arabian}
\end{multicols}









\section{Thaana}

\newfontfamily\thaana{MV Boli}
Thaana, Taana or Tāna ({\thaana  ތާނަ}‎ in Tāna script) is the modern writing system of the Maldivian language spoken in the Maldives. Thaana has characteristics of both an abugida (diacritic, vowel-killer strokes) and a true alphabet (all vowels are written), with consonants derived from indigenous and Arabic numerals, and vowels derived from the vowel diacritics of the Arabic abjad. Its orthography is largely phonemic.

The Thaana script first appeared in a Maldivian document towards the beginning of the 18th century in a crude initial form known as Gabulhi Thaana which was written scripta continua. This early script slowly developed, its characters slanting 45 degrees, becoming more graceful and spaces were added between words. 

As time went by it gradually replaced the older Dhives Akuru alphabet. The oldest written sample of the Thaana script is found in the island of Kanditheemu in Northern Miladhunmadulu Atoll. It is inscribed on the door posts of the main Hukuru Miskiy (Friday mosque) of the island and dates back to 1008 AH (AD 1599) and 1020 AH (AD 1611) when the roof of the building were built and the renewed during the reigns of Ibrahim Kalaafaan (Sultan Ibrahim III) and Hussain Faamuladeyri Kilege (Sultan Hussain II) respectively.

\begin{scriptexample}[]{Thaana}
\unicodetable{thaana}{"0780,"0790,"07A0,"07B0}

\hfill Typeset with MV Boli and the command \cmd{\thaana}.
\end{scriptexample}


^^A\printunicodeblock{./languages/thaana.txt}{\thaana}



\endinput











    \chapter{South East Asian Scripts}
\label{ch:southeastasia}
\section{Introduction}

This section documents the facilities offered to typeset Southeast Asian Scripts. These scripts are used in most of Southeast Asia, Indonesia and the Philippines.

\pagestyle{headings}

\begin{table}[htb]
\centering
\begin{tabular}{lll}
  \hyperref[s:thai]{Thai} 
& Tai Tham 
& \hyperref[s:balinese]{Balinese}\\
\hyperref[s:lao]{Lao}  
&Tai Viet  
& \hyperref[s:javanese]{Javanese}\\
Myanmar 
&Kayah Li 
&Rejang\\
 \hyperref[s:khmer]{Khmer} 
&Cham 
&Batak\\
Tai Le 
&Philippine Scripts 
& \hyperref[s:sundanese]{Sundanese}\\
  \hyperref[s:newtailue]{New Tail Lue}
& Buginese\\
\end{tabular}
\end{table}

\subsection{Balinese}

The Balinese script, natively known as Aksara Bali and Hanacaraka, is an abugida used in the island of Bali, Indonesia, commonly for writing the Austronesian Balinese language, Old Javanese, and the liturgical language Sanskrit. With some modifications, the script is also used to write the Sasak language, used in the neighboring island of Lombok.[1] The script is a descendant of the Brahmi script, and so has many similarities with the modern scripts of South and Southeast Asia. The Balinese script, along with the Javanese script, is considered the most elaborate and ornate among Brahmic scripts of Southeast Asia.[2]

Though everyday use of the script has largely been supplanted by the Latin alphabet, the Balinese script has significant prevalence in many of the island's traditional ceremonies and is strongly associated with the Hindu religion. The script is mainly used today for copying lontar or palm leaf manuscripts containing religious texts.[2][3]

\newfontfamily\balinese{AksaraBali.ttf}
\newfontfamily\indicative{code2000.ttf}

{\indicative ◌ }

\newcounter{under}
\setcounter{under}{"1B00}

\def\cb#1 {
\hspace*{2.5pt}
 \large
 $\text{◌#1}_{\pgfmathparse{Hex(\theunder)}\pgfmathresult}$
\stepcounter{under}
\vskip5pt\par
}
\begin{scriptexample}[]{Balinese}


\balinese
	 
᭐	᭑	᭒	᭓	᭔	᭕	᭖	᭗	᭘	᭙	᭚	᭛	᭜	᭝	᭞	᭟\\\
 
\def\columnseprulecolor{\color{thegray}}
\columnseprule.4pt
\begin{multicols}{8}

\texttt{U+1B0x}	

\cb{ᬀ }  \cb{ ᬁ } 	\cb{ ᬂ } 	\cb ᬃ	\cb ᬄ 	\cb ᬅ	\cb ᬆ	\cb ᬇ	\cb ᬈ	\cb ᬉ	\cb ᬊ	\cb ᬋ	\cb ᬌ	\cb ᬍ	\cb ᬎ	\cb ᬏ

\columnbreak

\texttt{U+1B1x}	 

\cb ᬐ	 \cb ᬑ 	\cb ᬒ 	\cb ᬓ	\cb ᬔ	\cb ᬕ	\cb ᬖ \cb ᬗ 	\cb ᬘ 	\cb ᬙ 	\cb ᬚ	\cb ᬛ 	\cb ᬜ 	\cb ᬝ 	\cb ᬞ	\cb ᬟ 

\columnbreak

U+1B2x	 

\cb ᬠ◌ 	\cb ᬡ	\cb ᬢ	\cb ᬣ	\cb ᬤ	\cb ᬥ	\cb ᬦ	\cb ᬧ	\cb ᬨ	\cb ᬩ	\cb ᬪ	\cb ᬫ	\cb ᬬ	\cb ᬭ	\cb ᬮ	\cb ᬯ

\columnbreak
U+1B3x 

\cb ᬰ	\cb ᬱ	\cb ᬲ	\cb ᬳ	\cb ᬴	\cb ᬵ	\cb ᬶ	\cb ᬷ	\cb ᬸ	\cb ᬹ	\cb ᬺ	\cb ᬻ	\cb ᬼ	\cb ᬽ	\cb ᬾ	\cb ᬿ


\columnbreak
U+1B4x	 

\cb ᭀ	 \cb ᭁ	\cb ᭂ	\cb ᭃ	\cb ᭄	\cb ᭅ	\cb ᭆ	\cb ᭇ	\cb ᭈ	\cb ᭉ	\cb ᭊ	\cb ᭋ

\columnbreak				
U+1B5x	 

\cb ᭐	\cb ᭑	\cb ᭒	\cb ᭓	\cb ᭔	\cb ᭕	\cb ᭖	\cb ᭗	\cb ᭘	\cb ᭙	\cb ᭚	\cb ᭛	\cb ᭜	\cb ᭝	\cb ᭞	\cb ᭟\\

\columnbreak

U+1B6x 

\cb ᭠	\cb ᭡	\cb ᭢	\cb ᭣	\cb ᭤	\cb ᭥	\cb ᭦	\cb ᭧	\cb ᭨◌ 	\cb ᭩◌ 	\cb ᭪◌ 	\cb ᭫	\cb ᭬	\cb ᭭	\cb ᭮	\cb ᭯

\columnbreak
U+1B7x	 

\cb ᭰	 \cb ᭱  \cb ᭲  \cb ᭳	 \cb ᭴	\cb ᭵	\cb ᭶	\cb ᭷	\cb ᭸	\cb ᭹	\cb ᭺	\cb ᭻	\cb ᭼


\end{multicols}

\end{scriptexample}
\defaulttext

One of the most comprehensive fonts is Aksara Bali\footnote{\url{http://www.alanwood.net/downloads/index.html}}. This is obtainable at Alan Wood's website.
\clearpage

%\newfontfamily\javanese{Noto Sans Javanese}

%\newfontfamily\javanese{TuladhaJejeg_gr.ttf}

\section{Javanese}
\label{s:javanese}
\index{scripts>Javanese}


The Javanese (Ngoko Javanese: {\javanese ꦮꦺꦴꦁꦗꦮ},[3] Madya Javanese: {\javanese\   ꦠꦶꦪꦁꦗꦮꦶ},[4] Krama Javanese: ꦥꦿꦶꦪꦤ꧀ꦠꦸꦤ꧀ꦗꦮꦶ,[4] Ngoko Gêdrìk: wòng Jåwå, Madya Gêdrìk: tiyang Jawi, Krama Gêdrìk: priyantun Jawi, Indonesian: suku Jawa)[5] are an ethnic group native to the Indonesian island of Java. With approximately 100 million people (as of 2011), they form the largest ethnic group in Indonesia. They are predominantly located in the central to eastern parts of the island. There are also significant numbers of people of Javanese descent in most provinces of Indonesia, Malaysia, Singapore, Suriname, Saudi Arabia and the Netherlands.

The Javanese ethnic group has many sub-groups, such as the Mataram, Cirebonese, Osing, Tenggerese, Samin, Naganese, Banyumasan, etc.[6]

A majority of the Javanese people identify themselves as Muslims, with a minority identifying as Christians and Hindus. However, Javanese civilization has been influenced by more than a millennium of interactions between the native animism Kejawen and the Indian Hindu—Buddhist culture, and this influence is still visible in Javanese history, culture, traditions, and art forms. With a sizeable global population, the Javanese are considered significant as they are the fourth largest ethnic group among Muslims, in the world, after the Arabs,[7] Bengalis[8] and Punjabis.[9]


\paragraph{Javanese} is one of the Austronesian languages, but it is not particularly close to other languages and is difficult to classify. Its closest relatives are the neighbouring languages such as Sundanese, Madurese and Balinese. Most speakers of Javanese also speak Indonesian, the standardized form of Malay spoken in Indonesia, for official and commercial purposes as well as a means to communicate with non-Javanese-speaking Indonesians.

There are speakers of Javanese in Malaysia (concentrated in the states of Selangor and Johor) and Singapore. Some people of Javanese descent in Suriname (the Dutch colony of Suriname until 1975) speak a creole descendant of the language.

\begin{figure}[htbp]
\includegraphics[width=\textwidth]{javanese-people}
\end{figure}

The language is spoken in Yogyakarta, Central and East Java, as well as on the north coast of West Java. It is also spoken elsewhere by the Javanese people in other provinces of Indonesia, which are numerous due to the government-sanctioned transmigration program in the late 20th century, including Lampung, Jambi, and North Sumatra provinces. In Suriname, creolized Javanese is spoken among descendants of plantation migrants brought by the Dutch during the 19th century. In Madura, Bali, Lombok, and the Sunda region of West Java, it is also used as a literary language. It was the court language in Palembang, South Sumatra, until the palace was sacked by the Dutch in the late 18th century.

Javanese is written with the Latin script, Javanese script, and Arabic script.[5] In the present day, the Latin script dominates writings, although the Javanese script is still taught as part of the compulsory Javanese language subject in elementary up to high school levels in Yogyakarta, Central and East Java.

Javanese is the tenth largest language by native speakers and the largest language without official status. It is spoken or understood by approximately 100 million people. At least 45\% of the total population of Indonesia are of Javanese descent or live in an area where Javanese is the dominant language. All seven Indonesian presidents since 1945 have been of Javanese descent.[6] It is therefore not surprising that Javanese has had a deep influence on the development of Indonesian, the national language of Indonesia.

There are three main dialects of the modern language: Central Javanese, Eastern Javanese, and Western Javanese. These three dialects form a dialect continuum from northern Banten in the extreme west of Java to Banyuwangi Regency in the eastern corner of the island. All Javanese dialects are more or less mutually intelligible.


\paragraph{The Javanese script} (Hanacaraka/Carakan) is a script for writing the Javanese language, the native language of one of the peoples of the Island of Java. It is a descendent of the ancient Brahmi script of India, and so has many similarities with modern scripts of South Asia and Southeast Asia. The Javanese script is also used for writing Sanskrit, Old Javanese, and transcriptions of Kawi, as well as the Sundanese language, and the Sasak language.

\begin{figure}[htbp]
\hspace*{-1.5cm}\includegraphics[width=1.2\textwidth]{java-palm-leave-manuscript}
\end{figure}





\begin{scriptexample}[]{Javanese}
\bgroup
\javanese

꧋ꦱꦧꦼꦤ꧀ꦮꦺꦴꦁꦏꦭꦲꦶꦂꦲꦏꦺꦏꦤ꧀ꦛꦶꦩꦂꦢꦶꦏꦭꦤ꧀ꦢꦂꦧꦺꦩꦂꦠꦧꦠ꧀ꦭꦤ꧀ꦲꦏ꧀ꦲꦏ꧀ꦏꦁꦥꦝ꧉

꧋ ꦲꦮꦶꦠ꧀ꦲꦶꦏꦁꦄꦱ꧀ꦩꦄꦭ꧀ꦭꦃ꧈ ꦏꦁꦩꦲꦩꦸꦫꦃꦠꦸꦂ ꦩꦲꦲꦱꦶꦃ꧉ 	 
 ۝꧋ ꦄꦭꦶꦥꦃ꧀ ꦭ ꦩ꧀ ꦫ ꧌ ꦏꦁ — — ꦥꦿꦶꦏ꧀ꦱ ꦏꦉꦪꦥ꧀ꦥꦩꦸꦁꦄꦭ꧀ꦭꦃꦥꦶꦪꦺꦩ꧀ꦧꦏ꧀ ꧌꧉ ꦩꦁꦪꦏꦴꦪꦤꦴ ꦲꦶꦏꦸꦄꦪꦺꦪꦠꦴꦏꦶꦠꦧ꧀ꦑꦸꦂꦄꦤ꧀ꦏꦁꦥꦿꦪꦠꦭ꧉ 	 
᭐	᭑	᭒	᭓	᭔	᭕	᭖	᭗	᭘	᭙	᭚	᭛	᭜	᭝	᭞	᭟
 
\egroup
\end{scriptexample}


The Javanese script was added to Unicode Standard in version 5.2 on the code points \texttt{A980 - A9DF}. There are 91 code points for Javanese script: 53 letters, 19 punctuation marks, 10 numbers, and 9 vowels:
\medskip

\unicodetable{javanese}{"A980,"A990,"A9A0, "A9B0, "A9C0,"A9D0}

\medskip



As of the writing of this document (2017), there are several widely published fonts able to support Javanese, ANSI-based Hanacaraka/Pallawa by Teguh Budi Sayoga,[21] Adjisaka by Sudarto HS/Ki Demang Sokowanten,[22] JG Aksara Jawa by Jason Glavy,[23] Carakan Anyar by Pavkar Dukunov,[24] and Tuladha Jejeg by R.S. Wihananto,[25] which is based on Graphite (SIL) smart font technology. Other fonts with limited publishing includes Surakarta made by Matthew Arciniega in 1992 for Mac's screen font,[26] and Tjarakan developed by AGFA Monotype around 2000.[27] There is also a symbol-based font called Aturra developed by Aditya Bayu in 2012–2013.[28]

Due to the script's complexity, many Javanese fonts have different input method compared to other Indic scripts and may exhibit several flaws. \docFont{JG Aksara Jawa}, in particular, may cause conflicts with other writing system, as the font use code points from other writing systems to complement Javanese's extensive repertoire. This is to be expected, as the font was made before Javanese implementation in Unicode.[29]

Arguably, the most "complete" font, in terms of technicality and glyph count, is \docFont{TuladhaJejeg}. It comes with keyboard facilities, displaying complex syllable structure, and support extensive glyph repertoire including non-standard forms which may not be found in regular Javanese texts, by utilizing Graphite (SIL) smart font technology. |Tuladha Jejeg| uses variable stroke widths on its glyphs with serifs on some glyphs\footnote{\protect\url{https://sites.google.com/site/jawaunicode/main-page}}.

However, as not many writing systems require such complex feature, use is limited to programs with Graphite technology, such as Firefox browser, Thunderbird email client, and several OpenType word processor and of course XeLaTeX. The font was chosen for displaying Javanese script in the Javanese Wikipedia.[16]

\paragraph{jawaTeX} Jawa\TeX{} project is initial effort to make Javanese characters typesetting program using \TeX{}/\LaTeX{}. This project is aimed to make Javanese widely used. The main project is developing transliteration models to transliterate Latin document into Javanese document. Perl and \TeX{}/\LaTeX{} are use in this project, the program are develop to run in text mode (console) both Linux and Windows but not limit on it. Web based program also developed, and automatic embedded Javanese characters in HTML See \href{http://jawatex.org/jawa/jawatex}{jawatex}.




\section{Khmer}
\newfontfamily\normaltext{Arial Unicode MS}
\normaltext

\def\khmerdefaultfont#1{\newfontfamily\khmer[Scale=MatchUppercase]{#1}}
\def\khmertext#1{{\khmer#1}}

\cxset{khmer font/.code=\khmerdefaultfont{#1}}

\cxset{khmer font/.default=Khmer}

\cxset{language=khmer, 
       khmer font = Khmer UI}

\begin{key}{/chapter/khmer font=\meta{font name} (Khmer  UI)} Loads the font
command \cmd{\khmer}. When the command is used it typesets text in
khmer unicode. There is no need to load the language, unless it is the main document language. For windows the default font is \texttt{DaunPenh} this font is in general too small to read; a better font to use is Khmer UI.
\end{key}

\begin{key}{/tikz/turtle/right=\meta{angle} (default 90)}
  Turns the turtle right by the given angle. 
\end{key}


The Khmer script (Khmer: {\Large\khmertext{អក្សរខ្មែរ}}; IPA: [ʔaʔsɑː kʰmaːe]) [2] is an \textit{abugida} (alphasyllabary) script used to write the Khmer language (the official language of Cambodia). It is also used to write Pali among the Buddhist liturgy of Cambodia and Thailand.

It was adapted from the Pallava script, a variant of Grantha alphabet descended from the Brahmi script of India, which was used in southern India and South East Asia during the 5th and 6th Centuries AD.[3] The oldest dated inscription in Khmer was found at Angkor Borei District in Takéo Province south of Phnom Penh and dates from 611.[4] The modern Khmer script differs somewhat from precedent forms seen on the inscriptions of the ruins of Angkor.

Not all Khmer consonants can appear in syllable-final position. The most common syllable-final consonants include {\khmer កងញតនបមល}. The pronunciation of the consonant in final position may differ from it's normal pronunciation.


\begin{tabular}{llp{9cm}}
\khmertext{ំ}	&nĭkkôhĕt (\khmertext{និគ្គហិត})	&niggahita; nasalizes the inherent vowels and some of the dependent vowels, see anusvara, sometimes used to represent [aɲ] in Sanskrit loanwords\\
\khmertext{ះ}	&reăhmŭkh (\khmertext{រះមុខ})	&"shining face"; adds final aspiration to dependent or inherent vowels, usually omitted, corresponds to the visarga diacritic, it maybe included as dependent vowel symbol\\
\khmertext{ៈ}	&yŭkôleăkpĭntŭ (\khmertext{យុគលពិន្ទុ})	&yugalabindu ("pair of dots"); adds final glottalness to dependent or inherent vowels, usually omitted\\
\khmertext{៉}	 &musĕkâtônd (\khmertext{មូសិកទន្ត})	&mūsikadanta ("mouse teeth"); used to convert some o-series consonants (\khmertext{ង ញ ម យ រ វ}) to the a-series\\
\khmertext{៊}	&treisâpt (\khmertext{ត្រីសព្ទ})	trīsabda; used to convert some a-series consonants (\khmertext{ស ហ ប អ}) to the o-series\\
\end{tabular}




ុ	kbiĕh kraôm (ក្បៀសក្រោម)	also known as bŏkcheung (បុកជើង); used in place of the diacritics treisâpt and musĕkâtônd when they would be impeded by superscript vowels
់	bântăk (បន្តក់)	used to shorten some vowels; the diacritic is placed on the last consonant of the syllable
៌	rôbat (របាទ)
répheăk (រេផៈ)	rapāda, repha; behave similarly to the tôndâkhéat, corresponds to the Devanagari diacritic repha, however it lost its original function which was to represent a vocalic r
 ៍	tôndâkhéat (ទណ្ឌឃាដ)	daṇḍaghāta; used to render some letters as unpronounced
៎	kakâbat (កាកបាទ)	kākapāda ("crow's foot"); more a punctuation mark than a diacritic; used in writing to indicate the rising intonation of an exclamation or interjection; often placed on particles such as /na/, /nɑː/, /nɛː/, /vəːj/, and the feminine response /cah/
៏	âsda (អស្តា)	denotes stressed intonation in some single-consonant words[5]
័	sanhyoŭk sannha (សំយោគសញ្ញា)	represents a short inherent vowel in Sanskrit and Pali words; usually omitted
៑	vĭréam (វិរាម)	a mostly obsolete diacritic, corresponds to the virāma
្	cheung (ជើង)	a.w. coeng; a sign developed for Unicode to input subscript consonants, appearance of this sign varies among fonts
\section{Sundanese}
\newfontfamily\sundanese{SundaneseUnicode-1.0.5.ttf}
^^A\newfontfamily\sundanese{Arial Unicode MS}
\def\ublock#1{\texttt{{\arial #1}}}

The Sundanese script (Aksara Sunda, {\sundanese ᮃᮊ᮪ᮞᮛ ᮞᮥᮔ᮪ᮓ}) is a writing system which is used by the Sundanese people. It is built based on Old Sundanese script (Aksara Sunda Kuno) which was used by the ancient Sundanese between the 14th and 18th centuries.

\begin{scriptexample}[]{Sundanese}
\unicodetable{sundanese}{"1B80,"1B90,"1BA0,"1BB0}

\sundanese
\obeylines
\bgroup
᮱ {\arial= 1}	᮲ {\arial= 2}	᮳{\arial = 3}
᮴ {\arial= 4}	᮵ {\arial = 5} 	᮶ {\arial= 6}
᮷ {\arial= 7}	᮸ {\arial= 8}	᮹ {\arial= 9}
᮰ {\arial= 0}

\egroup
\end{scriptexample}

\begin{scriptexample}[]{Sundanese}
\bgroup
\sundanese
\centering

◌ᮃᮄᮅᮆᮇᮈᮉᮊᮋᮌᮍᮎᮏᮐᮕᮔᮓᮑᮖᮗᮚᮛᮜᮝᮞᮟᮠᮠ


\egroup
\end{scriptexample}

\bgroup
\def\1{\sundanese ᮱}
\TextOrMath\1\1

$\1$
\egroup

In text In texts, numbers are written surrounded with dual pipe sign \textbar \ldots \textbar. Example: {\textbar \sundanese ᮲᮰᮱᮰\textbar} = 2010












%\newfontfamily\hanunoo{NotoSansHanunoo-Regular.ttf}

\section{Hanunó’o}

Hanunó’o is one of the indigenous scripts of the Philippines and is used by the Mangyan peoples of southern Mindoro to write the Hanunó'o language.[1] 

It is an \emphasis{abugida} descended from the Brahmic scripts, closely related to Baybayin, and is famous for being written vertical but written upward, rather than downward as nearly all other scripts (however, it's read horizontally left to right). It is usually written on bamboo by incising characters with a knife.[2][3] Most known Hanunó'o inscriptions are relatively recent because of the perishable nature of bamboo. It is therefore difficult to trace the history of the script



\begin{scriptexample}[width=2cm]{Hanunoo}
\hanunoo

{\Large
\obeylines
ᜠ 
ᜫ
ᜨᜲ
ᜫᜲ
ᜰ
ᜮ
ᜥ
ᜦ᜴}

Typeset with \texttt{NotoSansHanunoo-Regular.ttf} and the command \cmd{\hanunoo}
\end{scriptexample}

Vertically positionning the text is not currently supported by \pkgname{fontspec} and the manual says \textsc{Todo!}. You are your own here, or you can just put the characters in a box and give it a try.

\begin{minipage}[t]{2cm}
\begin{tcolorbox}[width=2cm,colback=graphicbackground,
boxrule=0pt,toprule=0pt,colframe=white]
\Large\hanunoo
ᜩ\\
ᜤ\\
ᜮ\\
ᜥᜳ\\
ᜨ᜴ \\
ᜨ᜴\\
ᜫᜳ\\
ᜥ\\
\end{tcolorbox}
\end{minipage}
\begin{minipage}[t]{2cm}
\begin{tcolorbox}[width=2cm,colback=graphicbackground,
boxrule=0pt,toprule=0pt,colframe=white]
\LARGE\hanunoo
ᜩ\\
ᜤ\\
ᜮ\\
ᜥᜳ\\
ᜨ᜴ \\
ᜨ᜴\\
ᜫᜳ\\
ᜥ\\
\end{tcolorbox}
\end{minipage}
\begin{minipage}[t]{\textwidth-6cm}

The script is written from bottom to top. Typesetting this type of script automatically is not without its problems. One way is to use the build-in features of the font if they are available, but currently this gives problems---at least with the fonts that I have tried. Entering the text is also problematic as you will more than likely see little boxes rather than the actual glyph with most text editors common to \latexe. If you only need a couple of characters or a short sentence, an easy solution is to use |\rotatebox|. Another solution is to use a macro that can add the letters onto a stack, then place them in a box with a limited width. We can use |\@tfor| for this.  
\end{minipage}
\section{New Tai Lue Script}
\label{s:newtailue}
\newfontfamily\tailue{Noto Sans New Tai Lue}


New Tai Lue script, also known as Simplified Tai Lue, is an alphabet used to write the Tai Lü language. Developed in China in the 1950s, New Tai Lue is based on the traditional Tai Le alphabet developed ca. 1200 AD. The government of China promoted the alphabet for use as a replacement for the older script; teaching the script was not mandatory, however, and as a result many are illiterate in New Thai Lue. 

\begin{figure}[htbp]
\centering

\includegraphics[width=\linewidth-2\parindent]{tailue}

\caption{Tai Le costumes. (pininterest)}
\end{figure}

In addition, communities in Burma, Laos, Thailand and Vietnam still use the Tai Le alphabet. There are probably less than one million native speakers of the language who can be found in China, Burma, Laos, Thailand and Vietnam.

\begin{figure}[htbp]
\centering

\includegraphics[width=\linewidth-2\parindent]{tai-lu}

\caption{Tai Le costumes. (pininterest)}
\end{figure}

\begin{scriptexample}[]{Tai Lue}
{\centering\tailue \LARGE

ᦒ	ᦓ	ᦔ	ᦕ	ᦖ	ᦗ	ᦘ	ᦙ	ᦚ	ᦛ	ᦜ	ᦝ	ᦞ	

}
\end{scriptexample}

The New Tai Lue script was added to the Unicode Standard in March, 2005 with the release of version 4.1.

The Unicode block for New Tai Lue is |U+1980|–|U+19DF|:

\begin{scriptexample}[]{New Tai Lue}
\unicodetable{tailue}{"1980,"1990,"19A0,"19B0,"19C0,"19D0}

\texttt{typeset using NotoSansNewTaiLue-Regular.ttf.}
\end{scriptexample}
\section{Myanmar}
\label{s:myanmar}
\index{Myanmar}\index{Burmese}\index{Mon}\index{Unicode>Myanmar}\index{Fonts>Padauk}

%\newfontfamily\myanmar{Padauk}

The Burmese script (Burmese:{\myanmar မြန်မာအက္ခရာ}; MLCTS: mranma akkha.ra; pronounced: [mjəmà ʔɛʔkʰəjà]) is an abugida in the Brahmic family, used for writing Burmese. It is an adaptation of the Old Mon script[2] or the Pyu script. In recent decades, other alphabets using the Mon script, including Shan and Mon itself, have been restructured according to the standard of the now-dominant Burmese alphabet. Besides the Burmese language, the Burmese alphabet is also used for the liturgical languages of Pali and Sanskrit.

The characters are rounded in appearance because the traditional palm leaves used for writing on with a stylus would have been ripped by straight lines.[3] It is written from left to right and requires no spaces between words, although modern writing usually contains spaces after each clause to enhance readability.

The earliest evidence of the Burmese alphabet is dated to 1035, while a casting made in the 18th century of an old stone inscription points to 984.[1] Burmese calligraphy originally followed a square format but the cursive format took hold from the 17th century when popular writing led to the wider use of palm leaves and folded paper known as parabaiks.[3] The alphabet has undergone considerable modification to suit the evolving phonology of the Burmese language.

Mon/Burmese script was added to the Unicode Standard in September, 1999 with the release of version 3.0. It was extended in October, 2009 with the release of version 5.2 and again in June, 2014 with the release of version 7.0.

\begin{docKey}[phd]{myanmar font}{=\meta{font name}}{default none initial Padauk}
Loads the font and creates associated environments and commands.
\end{docKey}

\begin{scriptexample}[]{Myanmar}
\unicodetable{myanmar}{"1000,"1010,"1020,"1030,"1040,"1050,"1060,"1070,"1080,"1090}
\end{scriptexample}








%\subsection{Oriya alphabet}
\newfontfamily\oriya[Scale=1.1,Script=Oriya]{code2000.ttf}

\def\oriyatext#1{{\oriya#1}}
The Oriya script or Utkala Lipi (Oriya: \oriyatext{ଉତ୍କଳ ଲିପି}) or Utkalakshara (Oriya: \oriyatext{ଉତ୍କଳାକ୍ଷର}) is used to write the Oriya language, and can be used for several other Indian languages, for example, Sanskrit.

\centerline{\Huge\oriyatext{ଉତ୍କଳ ଲିପି}}

\bgroup
\oriya
୦୧୨୩୪୫୬୭୮୯
ଅ ଆ ଇ ଈ ଉ ଊ ଋ ୠ ଌ ୡ ଏ ଐ ଓ ଔ କ ଖ ଗ ଘ ଙ ଚ ଛ ଜ ଝ ଞ ଟ ଠ ଡ ଢ ଣ ତ ଥ ଦ ଧ ନ ପ ଫ ବ ଵ ଭ ମ ଯ ର ଳ ୱ ଶ ଷ ସ ହ ୟ ଲ
\egroup

\begin{quotation}
Oṛiyā is encumbered with the drawback of an excessively awkward and cumbrous written character. ... At first glance, an Oṛiyā book seems to be all curves, and it takes a second look to notice that there is something inside each.(G. A. Grierson, Linguistic Survey of India, 1903)
\end{quotation}

Comparison of Oṛiyā script with its neighbours[edit]
At a first look the great number of signs with round shapes suggests a closer relation to the southern neighbour Telugu than to the other neighbours Bengali in the north and Devanāgarī in the west. The reason for the round shapes in Oriya and Telugu (and also in Kannaḍa and Malayāḷam) is the former method of writing using a stylus to scratch the signs into a palm leaf. These tools do not allow for horizontal strokes because that would damage the leaf.

Oriya letters are mostly round shaped whereas in Devanāgarī and Bengali have horizontal lines. So in most cases the reader of Oṛiyā will find the distinctive parts of a letter only below the hoop. Considering this the  closer relation to Devanāgarī and Bengali exists than to any southern script, though both northern and southern scripts have the same origin, Brāhmī.

Oriya (\oriyatext{ଓଡ଼ିଆ} oṛiā), officially spelled Odia,[3][4] is an Indian language belonging to the Indo-Aryan branch of the Indo-European language family. It is the predominant language of the Indian states of Odisha, where native speakers comprise 80\% of the population,[5] and it is spoken in parts of West Bengal, Jharkhand, Chhattisgarh and Andhra Pradesh. Oriya is one of the many official languages in India; it is the official language of Odisha and the second official language of Jharkhand. [6][7][8] Oriya is the sixth Indian language to be designated a Classical Language in India, on the basis of having a long literary history and not having borrowed extensively from other languages.


%\subsection{Mongolian Script}

\newfontfamily\mongolian[Language=Mongolian, Scale=1.3]{code2000.ttf}

The classical Mongolian script (in Mongolian script: {\mongolian  ᠮᠣᠩᠭᠣᠯ ᠪᠢᠴᠢᠭ᠌} Mongγol bičig; in Mongolian Cyrillic: Монгол бичиг Mongol bichig), also known as Uyghurjin Mongol bichig, was the first writing system created specifically for the Mongolian language, and was the most successful until the introduction of Cyrillic in 1946. Derived from Uighur, Mongolian is a true alphabet, with separate letters for consonants and vowels. The Mongolian script has been adapted to write languages such as Oirat and Manchu. Alphabets based on this classical vertical script are used in Inner Mongolia and other parts of China to this day to write Mongolian, Sibe and, experimentally, Evenki.
\medskip

\bgroup\par
\noindent
\colorbox{graphicbackground}{\color{black}^^A
\begin{minipage}{\textwidth}^^A
\parindent1pt
\vskip10pt
\leftskip10pt \rightskip\leftskip
\mongolian
\large
ᠬᠦᠮᠦᠨ ᠪᠦᠷ ᠲᠥᠷᠥᠵᠦ ᠮᠡᠨᠳᠡᠯᠡᠬᠦ ᠡᠷᠬᠡ ᠴᠢᠯᠥᠭᠡ ᠲᠡᠢ᠂ ᠠᠳᠠᠯᠢᠬᠠᠨ ᠨᠡᠷ᠎ᠡ ᠲᠥᠷᠥ ᠲᠡᠢ᠂ ᠢᠵᠢᠯ ᠡᠷᠬᠡ ᠲᠡᠢ ᠪᠠᠢᠠᠭ᠃ ᠣᠶᠤᠨ ᠤᠬᠠᠭᠠᠨ᠂ ᠨᠠᠨᠳᠢᠨ ᠴᠢᠨᠠᠷ ᠵᠠᠶᠠᠭᠠᠰᠠᠨ ᠬᠦᠮᠦᠨ ᠬᠡᠭᠴᠢ ᠥᠭᠡᠷ᠎ᠡ ᠬᠣᠭᠣᠷᠣᠨᠳᠣ᠎ᠨ ᠠᠬᠠᠨ ᠳᠡᠭᠦᠦ ᠢᠨ ᠦᠵᠢᠯ ᠰᠠᠨᠠᠭᠠ ᠥᠠᠷ ᠬᠠᠷᠢᠴᠠᠬᠥ ᠤᠴᠢᠷ ᠲᠠᠢ᠃
\par
\vspace*{10pt}
\end{minipage}
}
\medskip
%\subsection{Tibetan}

^^A\newfontfamily\tibetan{TibMachUni.ttf}

^^A\newfontfamily\tibetan{Qomolangma-Chuyig.ttf}

^^A should pick it up automatically \tibetan

Fonts described in this section can be obtained from The Tibetan \& Himalayan Library
\footnote{\url{http://www.thlib.org/tools/scripts/wiki/tibetan%20machine%20uni.html}  }

I have tried a few \texttt{Tibetan Machine Uni (TMU)} seems to be used by a number of scholars. 

A tip when you are trying to locate fonts is to find a related article in Wikipedia, such as Tibetan alphabet and inspect the element using your browser to see what fonts are being used.


|style="font-family:'Jomolhari','Tibetan Machine Uni','DDC Uchen', 'Kailash';| 


If you cannot see the script and rather than boxes or question marks then you can search and download one of the fonts in |font-family|.

\def\tibetandefaultfont#1{\newfontfamily\tibetan[Language=Tibetan]{#1}}


\cxset{language=tibetan} 
\cxset{tibetan font/.code=\tibetandefaultfont{#1}}


^^A\cxset{tibetan font = TibMachUni.ttf}




\begin{key}{/chapter/language = tibetan} The key |language=tibetan| sets the default language as Tibetan, using the main font given by the key |tibetan font=TibMachUni.ttf|.
\end{key}

\begin{key}{/chapter/tibetan font = TibMachUni.ttf} The key |tibetan font=font-name| sets the default font for the Tibetan language. It will also create the switch \cmd{\tibetan} for typesetting text in Tibetan.
\end{key}

\begin{texexample}{Tibetan language setttings}{ex:tibetan}
\cxset{language=tibetan, tibetan font = TibMachUni.ttf}
\tibetan

\tibetan Tibetan: དབུ་ཅན
\end{texexample}


The Tibetan alphabet is an \emph{abugida} of Indic origin used to write the Tibetan language as well as Dzongkha, the Sikkimese language, Ladakhi, and sometimes Balti. 

The printed form of the alphabet is called \textit{uchen} script (Tibetan: དབུ་ཅན་, Wylie: dbu-can; "with a head") while the hand-written cursive form used in everyday writing is called umê script (Tibetan: དབུ་མེད་, Wylie: dbu-med; "headless").
\uccoff
The alphabet is very closely linked to a broad ethnic Tibetan identity. Besides Tibet, it has also been used for Tibetan languages in Bhutan, India, Nepal, and Pakistan.[1] The Tibetan alphabet is ancestral to the Limbu alphabet, the Lepcha alphabet,[2] and the multilingual 'Phags-pa script.[2]
\uccon

The Tibetan alphabet is romanized in a variety of ways.[3] This article employs the Wylie transliteration system.

The Tibetan alphabet has thirty basic letters, sometimes known as "radicals", for consonants.[2]

ཀ ka /ká/	ཁ kha /kʰá/	ག ga /kà, kʰà/	ང nga /ŋà/
ཅ ca /tʃá/	ཆ cha /tʃʰá/	ཇ ja /tʃà/	ཉ nya /ɲà/
ཏ ta /tá/	ཐ tha /tʰá/	ད da /tà, tʰà/	ན na /nà/
པ pa /pá/	ཕ pha /pʰá/	བ ba /pà, pʰà/	མ ma /mà/
ཙ tsa /tsá/	ཚ tsha /tsʰá/	ཛ dza /tsà/	ཝ wa /wà/ (not originally part of the alphabet)[5]
ཞ zha /ʃà/[6]	ཟ za /sà/	འ 'a /hà/[7]
ཡ ya /jà/	ར ra /rà/	ལ la /là/
ཤ sha /ʃá/[6]	ས sa /sá/	ཧ ha /há/[8]
ཨ a /á/

\subsubsection{Unicode Block Tibetan}


\bgroup\large
\begin{tabular}{llllllllllllllll l}
\toprule
	           &|0|	&|1|	&|2|	&|3|	&|4|	&|5|	&|6|	&|7|	&|8|	&|9|	&|A|	&|B|	&|C|	&|D|	&|E|	&|F|\\
\midrule
\texttt{U+0F0x}	&ༀ	&༁	&༂	&༃	&༄	&༅	&༆	&༇	&༈	&༉	&༊	&་	&༌  &	།	&༎	&༏\\
\midrule
\texttt{U+0F1x} &༐	&༑	&༒	&༓	&༔	&༕	&༖	&༗	&༘&	༙	&༚	&༛	&༜	&༝	&༞	&༟\\
\midrule
\texttt{U+0F2x} &༠	&༡	&༢	&༣	&༤	&༥	&༦	&༧	&༨	&༩	&༪	&༫	&༬	&༭	&༮	&༯\\
\midrule
\texttt{U+0F3x}	&༰ &༱	 &༲ &༳	&༴ &༵	&༶ & ༷	&༸&	༹	&༺&	༻	&༼&	༽	&༾	&༿\\
\midrule
\texttt{U+0F4x} &ཀ	&ཁ	&ག	&གྷ	&ང	&ཅ	&ཆ	&ཇ	&	&ཉ	&ཊ	&ཋ	&ཌ	&ཌྷ	&ཎ	&ཏ\\
\midrule
\texttt{U+0F5x}	 &ཐ	&ད	&དྷ	&ན	&པ	&ཕ	&བ	&བྷ	&མ	&ཙ	&ཚ	&ཛ	&ཛྷ	&ཝ	&ཞ	&ཟ\\
\midrule
\texttt{U+0F6x} &འ	&ཡ	&ར	&ལ	&ཤ	&ཥ	&ས	&ཧ	&ཨ	&ཀྵ	&ཪ	&ཫ	&ཬ	&&&\\
^^A\texttt{U+0F7x}&&	ཱ &	& &ི	ཱི&	ུ&	ཱུ&	ྲྀ&	ཷ&	ླྀ&	ཹ&	ེ&	ཻ&	ོ&	ཽ&	&ཾ	&ཿ\\
\midrule
\texttt{U+0F8x}&    ྀ   & 	ཱྀ&	ྂ&	&ྃ &	྄	&྅&	྆	&྇	ྈ&	ྉ&	ྊ&	ྋ&	ྌ&	ྍ&	ྎ&	ྏ\\
\midrule
\texttt{U+0F9x} &	ྐ&	ྑ   & 	ྒ &	ྒྷ &	ྔ &	ྕ &	ྖ &	ྗ &		ྙ &	ྚ &	ྛ &	ྜ &	ྜྷ &	ྞ &	ྟ\\
\texttt{U+0FAx} &	ྠ &	ྡ &	ྡྷ &	ྣ &	ྤ &	ྥ &		&ྦ	&ྦྷ	ྨ&	ྩ&	ྪ&	ྫ&	ྫྷ&	ྭ&	ྮ&	ྯ\\
\midrule
\texttt{U+0FBx} 
&	  ྰ 
&	
& ྱ  	 
&ྲ	
&ླ	
&ྴ
&	ྵ
&	ྶ
&	ྷ
&ྸ
&
&
&
&	
&྾	
&྿\\
\midrule
\texttt{U+0FCx}	 &࿀&	࿁&	࿂&	࿃&	࿄&	࿅&	&࿇	&࿈	&࿉	&࿊	&࿋	&࿌	&&	࿎	&࿏\\
\midrule
\texttt{U+0FDx}	&࿐	&࿑	&࿒	&࿓	&࿔	&࿕	&࿖	&࿗	&࿘	&࿙	&࿚	&&&&&\\
\midrule
\texttt{U+0FEx} &&&&&&&&&&&&&&&&\\
\midrule
\texttt{U+0FFx}  &&&&&&&&&&&&&&&&\\
\bottomrule
\end{tabular}
\egroup




\subsubsection{Fonts for Tibetan}

Fonts for Tibetan need to be downloaded one set of fonts are the \texttt{Qomolangma}. They come in different flavours, but they appear
to offer advantages as compared to the Tibetan Machine Uni.
\medskip


\newfontfamily\betsu{Qomolangma-Betsu.ttf}
\newfontfamily\drutsa{Qomolangma-Drutsa.ttf}
\newfontfamily\chuyig{Qomolangma-Chuyig.ttf}
\newfontfamily\tsumachu{Qomolangma-Tsumachu.ttf}
\newfontfamily\uchensutung{Qomolangma-UchenSutung.ttf}
\newfontfamily\uchensuring{Qomolangma-UchenSuring.ttf}
\newfontfamily\uchensarchen{Qomolangma-UchenSarchen.ttf}
\newfontfamily\uchensarchung{Qomolangma-UchenSarchung.ttf}
\newfontfamily\tsuring{Qomolangma-Tsuring.ttf}
\newfontfamily\TMU{TibMachUni.ttf}
\newfontfamily\himalaya{Microsoft Himalaya}
\uccoff

{
\centering

\renewcommand{\arraystretch}{1.5}

\begin{tabular}{lr}
\toprule
|Qomolangma-Betsu.ttf| & {\betsu  དབུ་མེད }\\
\midrule
|Qomolangma-Chuyig.ttf| &{\chuyig  དབུ་མེད}\\
\midrule
|Qomolangma-Drutsa.ttf| &{\drutsa  དབུ་མེད}\\
\midrule
|Qomolangma-Tsumachu.ttf|&{\tsumachu  དབུ་མེད}\\
\midrule
|Qomolangma-Tsuring.ttf| &{\tsuring  དབུ་མེད}\\
\midrule
|Qomolangma-UchenSarchen.ttf| &{\uchensarchen དབུ་མེད}\\
\midrule
|Qomolangma-UchenSarchung.ttf|&{\uchensarchung དབུ་མེད }\\
\midrule
|Qomolangma-UchenSuring.ttf|&{\uchensuring དབུ་མེད}\\
\midrule
|Qomolangma-UchenSutung.ttf|&{\uchensutung དབུ་མེད }\\
\midrule
|TibMachUni.ttf| &{\TMU དབུ་མེད }\\
\midrule
|Microsoft Himalaya| &{\himalaya དབུ་མེད ཽ}\\
\bottomrule
\end{tabular}

}
\bigskip

\bgroup
\LARGE\tsuring
\noindent༆ །ཨ་ཡིག་དཀར་མཛེས་ལས་འཁྲུངས་ཤེས་བློ  འི་\par
གཏེར༑ །ཕས་རྒོལ་ཝ་སྐྱེས་ཟིལ་གནོན་གདོང་ལྔ་བཞིན།།\par
ཆགས་ཐོགས་ཀུན་བྲལ་མཚུངས་མེད་འཇམ་དབྱངསམཐུས།།\par
མཧཱ་མཁས་པའི་གཙོ་བོ་ཉིད་འགྱུར་ཅིག། །མངྒལཾ༎\par
\egroup

\subsubsection{Tibetan numbers}
\cxset{language=tibetan, tibetan font = TibMachUni.ttf}

{
\obeylines
\small
TIBETAN DIGIT ZERO	༠
TIBETAN DIGIT ONE	༡	
TIBETAN DIGIT TWO	༢	
TIBETAN DIGIT THREE	༣	
TIBETAN DIGIT FOUR	༤	
TIBETAN DIGIT FIVE	༥	
TIBETAN DIGIT SIX	༦	
TIBETAN DIGIT SEVEN	༧	
TIBETAN DIGIT EIGHT	༨	
TIBETAN DIGIT NINE	༩	
TIBETAN DIGIT HALF ONE	\tibetan༪	
TIBETAN DIGIT HALF TWO	༫	
TIBETAN DIGIT HALF THREE	༬
TIBETAN DIGIT HALF FOUR ༭	
TIBETAN DIGIT HALF FIVE ༯	
TIBETAN DIGIT HALF SIX	 ༯	
TIBETAN DIGIT HALF SEVEN	༰	
TIBETAN DIGIT HALF EIGHT	༱	
TIBETAN DIGIT HALF NINE	༲	
TIBETAN DIGIT HALF ZERO	༳	
}


Tibetan numbers

The usage is not certain. By some interpretations, this has the value of 9.5. Used only in some traditional contexts, these appear as the last digit of a multidigit number, eg. ༤༬ represents 42.5. These are very rarely used, however, and other uses have been postulated.

\defaulttext





\section{Tamil}
\newfontfamily\tamil[Scale=1.1,Script=Tamil]{code2000.ttf}

\def\tamiltext#1{{\tamil#1}}

The Tamil script (\tamiltext{தமிழ் அரிச்சுவடி} tamiḻ ariccuvaṭi) is an abugida script that is used by the Tamil people in India, Sri Lanka, Malaysia and elsewhere, to write the Tamil language, as well as to write the liturgical language Sanskrit, using consonants and diacritics not represented in the Tamil alphabet.[1] Certain minority languages such as Saurashtra, Badaga, Irula, and Paniya are also written in the Tamil script

The Tamil script has 12 vowels (\tamiltext{உயிரெழுத்து} uyireḻuttu "soul-letters"), 18 consonants (\tamiltext{மெய்யெழுத்து} meyyeḻuttu "body-letters") and one character, the āytam \tamiltext{ஃ (ஆய்தம்)}, which is classified in Tamil grammar as being neither a consonant nor a vowel (\tamiltext{அலியெழுத்து} aliyeḻuttu "the hermaphrodite letter"), though often considered as part of the vowel set (\tamiltext{உயிரெழுத்துக்கள்} uyireḻuttukkaḷ "vowel class"). The script, however, is syllabic and not alphabetic.[3] The complete script, therefore, consists of the thirty-one letters in their independent form, and an additional 216 combinant letters representing a total 247 combinations (\tamiltext{உயிர்மெய்யெழுத்து} uyirmeyyeḻuttu) of a consonant and a vowel, a mute consonant, or a vowel alone. These combinant letters are formed by adding a vowel marker to the consonant. Some vowels require the basic shape of the consonant to be altered in a way that is specific to that vowel. Others are written by adding a vowel-specific suffix to the consonant, yet others a prefix, and finally some vowels require adding both a prefix and a suffix to the consonant. In every case the vowel marker is different from the standalone character for the vowel.
The Tamil script is written from left to right.

Tamil is a Unicode block containing characters for the Tamil, Badaga, and Saurashtra languages of Tamil Nadu India, Sri Lanka, Singapore, and Malaysia. In its original incarnation, the code points U+0B02..U+0BCD were a direct copy of the Tamil characters A2-ED from the 1988 ISCII standard. The Devanagari, Bengali, Gurmukhi, Gujarati, Oriya, Telugu, Kannada, and Malayalam blocks were similarly all based on their ISCII encodings.

\begin{scriptexample}[]{Tamil}
\unicodetable{tamil}{"0B80,"0B90,"0BA0,"0BB0,"0BC0,"0BE0,"0BF0}

\hfill  Typeset with \cmd{\tamil} and \texttt{code2000.ttf}
\end{scriptexample}

\subsection{Tamil Numbers and Numerals}

Originally, Tamils did not use zero, nor did they use positional digits (having separate 
symbols for the numbers 10, 100 and 1000). Symbols for the numbers are similar to 
other Tamil letters, with some minor changes. 

For example, the number 3782 is not written as \tamiltext{௩௭௮௨} as in modern usage. Instead it 
is written as \tamiltext{௩ ௲ ௭ ௱ ௮ ௰ ௨}. This would be read as they are written as 
Three Thousands, Seven Hundreds, Eight Tens, Two; or in Tamil as 
\tamiltext{௩௲௭௱௮௰௨ž}.\footnote{https://cloud.github.com/downloads/raaman/Tamil-Numeral/tamilnumbers.html}

\subsection{Dates}

Once the script is loaded the day, month and year can be loaded using the command  \cmd{\tamildate}, which returns the |\today| formatted as per custom Tamil. 

\begin{center}
\bgroup
\tamil
\begin{tabular}{lll}
day	 &month	&year	\\

௳	&௴	      &௵	\\

u	&mee	      &wa	\\
\egroup
\end{center}














\subsection{Kannada alphabet}

\newfontfamily\kannada[Scale=1.0,Script=Kannada]{Lohit-Kannada.ttf}

\def\kannadatext#1{{\kannada#1}}

The Kannada alphabet (\kannadatext{ಕನ್ನಡ ಲಿಪಿ}) is an abugida of the Brahmic family,[2] used primarily to write the Kannada language, one of the Dravidian languages of southern India. Several minor languages, such as Tulu, Konkani, Kodava, and Beary, also use alphabets based on the Kannada script.[3] The Kannada and Telugu scripts share high mutual intellegibility with each other, and are often considered to be regional variants of single script. Similarly, Goykanadi, a variant of Old Kannada, has been historically used to write Konkani in the state of Goa.[4]

\begin{scriptexample}[]{Kannada}
\centerline{\LARGE\kannadatext{ಙ	ಙ್ಕ	ಙ್ಖ	ಙ್ಗ	ಙ್ಘ	ಙ್ಙ	ಙ್ಚ	ಙ್ಛ	ಙ್ಜ	ಙ್ಝ	ಙ್ಞ	ಙ್ಟ	ಙ್ಠ	ಙ್ಡ	ಙ್ಢ}}
\end{scriptexample}

\medskip

The Kannada script (aksharamale or varnamale) is a phonemic abugida of forty-nine letters, and is written from left to right. The character set is almost identical to that of other Brahmic scripts. Consonantal letters imply an inherent vowel. Letters representing consonants are combined to form digraphs (ottaksharas) when there is no intervening vowel. Otherwise, each letter corresponds to a syllable.
The letters are classified into three categories: swara (vowels), vyanjana (consonants), and yogavaahaka (part vowel, part consonant).
The Kannada words for a letter of the script are akshara, akkara, and varna. Each letter has its own form (ākāra) and sound (shabda), providing the visible and audible representations, respectively. Kannada is written from left to right.[7]


\subsection{Osmanian Alphabet}

\bgroup
\newfontfamily\osmanian{code2001.ttf}
\osmanian
𐒚𐒁𐒖𐒄 𐒚𐒐 𐒚 𐒎𐒚𐒍𐒚𐒐 𐒑𐒚𐒒𐒠𐒚𐒐 𐒎𐒚𐒑𐒁𐒗 𐒚𐒁𐒖𐒄 𐒚𐒌𐒖𐒄 𐒚𐒁𐒖𐒄𐒖 𐒚
𐒌𐒜
\egroup



\cxset{steward,
  offsety=0cm,
  image={ethiopianbride.jpg},
  texti={An introduction to the use of font related commands. The chapter also gives a historical background to font selection using \tex and \latex. },
  textii={In this chapter we discuss keys that are available through the \texttt{phd} package and give a background as to how fonts are used
in \latex.
 },
 pagestyle = empty,
}




\cxset{steward,
  offsety=0cm,
  image={fellah-woman.jpg},
  texti={An introduction to the use of font related commands. The chapter also gives a historical background to font selection using \tex and \latex. },
  textii={In this chapter we discuss keys that are available through the \texttt{phd} package and give a background as to how fonts are used
in \latex.
 },
 pagestyle = empty
}

    \index{Katakana}\index{Hiragana}
\index{Bopomofo}\index{Hangul}\index{Yi}
\index{East Asian Scripts>Katakana}
\index{East Asian Scripts>Hiragana}
\index{East Asian Scripts>Hangul}
\index{East Asian Scripts>Bopomofo}
\index{East Asian Scripts>Yi}
\index{scripts>cjk}
\pagestyle{headings}
\index{Yi fonts>Microsoft Yi Baiti}
\chapter{East Asian Scripts}
\epigraph{

For writing is the foundation of the classics and the arts, the beginning of
royal government. It is the means by which people of the past reach posterity,
by which people of the future know the past. 

{\cjk 蓋文字者,經藝之本,王政之始。前人所以垂後,後人所以識古。}
}{ Xu Shen  in the ``Postface'' of the \emph{Shuowen}}

\bigskip

\noindent This chapter presents the most common scripts currently in use in East Asia. This includes Chinese, Japanese and Korean. It also discusses several scripts for minority languages spoken in southern China. The scripts discussed are as follows:


\begin{center}
\begin{tabular}{lll}
\nameref{s:han} &Hiragana &Hangul\\
\nameref{s:bopomofo} &Katakana &\nameref{s:yi}\\
\end{tabular}
\end{center}
\bigskip

\parindent1em

Settings for |cjk| languages and scripts follow:

\begin{docKey}[phd]{cjk font}{\meta{font name}}{default none, initial code2000.ttf}
This key when set produces all necessary command to set the font for cjk typesetting.
\end{docKey}

\parindent1em
\section{Han CJK Unified Ideographs}
\label{s:han}
\index{CJK}
The Chinese, Japanese and Korean (CJK) scripts share a common background. In the process called Han unification the common (shared) characters were identified, and named "CJK Unified Ideographs". Unicode defines a total of 74,617 CJK Unified Ideographs.[1]\footnote{\protect\url{http://shahon.org/wp-content/uploads/2010/02/Galambos-2006-Orthography-of-early-Chinese-writing.pdf}}

The terms ideographs or ideograms may be misleading, since the Chinese script is not strictly a picture writing system.
Historically, Vietnam used Chinese ideographs too, so sometimes the abbreviation "CJKV" is used. This system was replaced by the Latin-based Vietnamese alphabet in the 1920s.


\unicodetable{cjk}{"4E00,"4E10,"4E20,"4E30,"4E40,"4}




\section{Bopomofo}
\label{s:bopomofo}
Bopomofo is the colloquial name of the \textit{zhuyin fuhao} or \textit{zhuyin} system of phonetic notation for the transcription of spoken Chinese, particularly the Mandarin dialect. Consisting of 37 characters and four tone marks, it transcribes all possible sounds in Mandarin. 

Bopomofo was introduced in China by the Republican Government, in the 1910s and used alongside the Wade-Giles system, which used a modified Latin alphabet. The Wade system was replaced by \textit{Hanyu Pinyin} in 1958 by the Government of the People's Republic of China,[1] at the International Organization for Standardization (ISO) in 1982 (ISO 7098:1982). Bopomofo remains widely used as an educational tool and electronic input method in Taiwan. On Windows the font Microsoft JhengHei can be used. 

Windows fonts that can be used \texttt{Microsoft JhengHei} and \texttt{SimSun}.

U+3100–U+312F
\newfontfamily\bopomofo{Microsoft JhengHei}

\begin{scriptexample}[]{Bopomofo}
{\centering\bopomofo 

伯帛勃脖舶博渤霸壩灞

}

\hfill \texttt{Typeset with \cmd{\bopomofo} and Microsoft JhengHei font }
\end{scriptexample}

\begin{scriptexample}[]{Bopomofo}

{\centering\bopomofo

伯帛勃脖舶博渤霸壩灞

}
\hfill \texttt{Typeset with \cmd{\bopomofo} and JhengHei font }
\end{scriptexample}


The Bopomofo Extended block, running from \unicodenumber{U+31A0-U31BF}, contains less universally recognized Bopomofo characters used to write various non-Mandarin Chinese languages. A few additional tone marks are unified with characters in the Spacing Modifier Letters block. 












\section{Yi}
\label{s:yi}

The Yi script (Yi: {\yi ꆈꌠꁱꂷ} nuosu bburma [nɔ̄sū bū̠mā]; Chinese: {\cjk 彝文}; pinyin: Yí wén) is an umbrella term for two scripts used to write the Yi language; Classical Yi, an ideogram script, the later Yi Syllabary. The script is also historically known in Chinese as Cuan Wen (Chinese: {\cjk 爨文}; pinyin: Cuàn wén) or Wei Shu (simplified Chinese: {\cjk韪书}; traditional Chinese: {\cjk 違書}; pinyin: Wéi shū) and various other names ({\cjk夷字、倮語、倮倮文、毕摩文}), among them "tadpole writing" ({\cjk蝌蚪文}).[1]

This is to be distinguished from romanized Yi ({\yi 彝文罗马拼音} Yiwen Luoma pinyin) which was a system (or systems) invented by missionaries and intermittently used afterwards by some government institutions.[2][3] There was also a Yi abugida or alphasyllabary devised by Sam Pollard, the Pollard script for the Miao language, which he adapted into "Nasu" as well.[4][5] Present day traditional Yi writing can be sub-divided into five main varieties (Huáng Jiànmíng 1993); Nuosu (the prestige form of the Yi language centred on the Liangshan area), Nasu (including the Wusa), Nisu (Southern Yi), Sani (撒尼) and Azhe (阿哲).[6][7]

The Unicode block for Modern Yi is Yi syllables (U+A000 to U+A48C), and comprises 1,164 syllables (syllables with a diacritic mark are encoded separately, and are not decomposable into syllable plus combining diacritical mark) and one syllable iteration mark (U+A015, incorrectly named YI SYLLABLE WU). In addition, a set of 55 radicals for use in dictionary classification are encoded at U+A490 to U+A4C6 (Yi Radicals).[11] Yi syllables and Yi radicals were added as new blocks to Unicode Standard Version 3.0.[12]

Classical Yi - which is an ideographic script like the Chinese characters - has not yet been encoded in Unicode, but a proposal to encode 88,613 Classical Yi characters was made in 2007.[13]

\bgroup
\yi \char"A000: Yi Syllable It\\

\yi \char"A001: Yi Syllable Ix\\

\yi \char"A002: Yi Syllable I\\
\egroup

\begin{scriptexample}[]{Yi}
\unicodetable{yi}{"A000,"A010,"A020,"A030,"A040,"A050,"A060,"A070,"A080,"A090,"A0A0,"A0B0,"A0C0}
\end{scriptexample}



    \chapter{Additional Modern Scripts}

\begin{center}
\begin{tabular}{lp{5cm}l}
Ethiopic. &Vai. &Deseret.\\
Mongolian. &Bamum. &Shavian.\\
Osmanya.   &Cherokee. &Lisu.\\
Tifinagh.  &Canadian Aboriginal Syllabics. &Miao.\\
N’Ko.&&\\
\end{tabular}
\end{center}

Ethiopic, Mongolian, and Tifinagh are scripts with long histories. Although their roots can
be traced back to the original Semitic and North African writing systems, they would not
be classified as Middle Eastern scripts today

The Cherokee script is a syllabary developed between 1815 and 1821, to write the Cherokee
language, still spoken by small communities in Oklahoma and North Carolina. Canadian
Aboriginal Syllabics were invented in the 1830s for Algonquian languages in Canada. The
system has been extended many times, and is now actively used by other communities, including speakers of Inuktitut and Athapascan languages.

Deseret is a phonemic alphabet devised in the 1850s to write English. It saw limited use for
a few decades by members of The Church of Jesus Christ of Latter-day Saints. Shavian is
another phonemic alphabet, invented in the 1950s to write English. It was used to publish
one book in 1962, but remains of some current interest




\subsection{Ethiopic}
Ge'ez (ግዕዝ Gəʿəz), (also known as Ethiopic) is a script used as an abugida (syllable alphabet) for several languages of Ethiopia and Eritrea. It originated as an abjad (consonant-only alphabet) and was first used to write Ge'ez, now the liturgical language of the Ethiopian Orthodox Tewahedo Church and the Eritrean Orthodox Tewahedo Church. In Amharic and Tigrinya, the script is often called fidäl (ፊደል), meaning "script" or "alphabet".

The Ge'ez script has been adapted to write other, mostly Semitic, languages, particularly Amharic in Ethiopia, and Tigrinya in both Eritrea and Ethiopia. It is also used for Sebatbeit, Me'en, and most other languages of Ethiopia. In Eritrea it is used for Tigre, and it has traditionally been used for Blin, a Cushitic language. Tigre, spoken in western and northern Eritrea, is considered to resemble Ge'ez more than do the other derivative languages.[citation needed] Some other languages in the Horn of Africa, such as Oromo, used to be written using Ge'ez, but have migrated to Latin-based orthographies.
For the representation of sounds, this article uses a system that is common (though not universal) among linguists who work on Ethiopian Semitic languages. This differs somewhat from the conventions of the International Phonetic Alphabet. See the articles on the individual languages for information on the pronunciation.

There are a number of fonts available and we have selected the Google \idxfont{NotoSansEthiopic}
\newfontfamily\ethiopic{NotoSansEthiopic-Bold.ttf}

\begin{scriptexample}[]{Ethiopic}
\unicodetable{ethiopic}{"1200,"1210,"1220,"1230,"1240,"1250,"1260,"1270,"1280,"1290,^^A
"12A0,"12B0,"12C0,"12E0,"12F0,"1300,"1310,"1330,"1340,"1350,"1360,"1370}
\end{scriptexample}
\section{Vai}
\label{s:vai}

The Vai syllabary is a syllabic writing system devised for the Vai language by Momolu Duwalu Bukele of Jondu, in what is now Grand Cape Mount County, Liberia.[1] [2] Bukele is regarded within the Vai community, as well as by most scholars, as the syllabary's inventor and chief promoter when it was first documented in the 1830s. It is one of the two most successful indigenous scripts in West Africa.

\newfontfamily\vai{code2000.ttf}
\begin{scriptexample}[]{Vai}
\unicodetable{vai}{"A500,"A510,"A520,"A530,"A540,"A550,"A560,"A570,^^A
"A580,"A590,"A5A0,"A5B0,^^A
"A5C0,"A5D0,"A5E0,"A5F0,"A610,"A620,"A630}
\end{scriptexample}

In the 1920s ten decimal digits were devised for Vai; these were “Vai-style” glyph variants of
European digits (see Figure 11). They were not popular with Vai people  even for historical purposes. All
the modern literature uses European digits.


\begin{scriptexample}[]{Vai}
\bgroup
\vai
\obeylines\Large
0	1	2	3	4	5	6	7	8	9
꘠	꘡	꘢	꘣	꘤	꘥	꘦	꘧	꘨	꘩
\vai
\egroup
\end{scriptexample}



\printunicodeblock{./languages/vai.txt}{\vai}
\section{Deseret script}
\newfontfamily\deseret{code2001.ttf}

The Deseret alphabet (dɛz.əˈrɛt.) (Deseret: {\deseret 𐐔𐐯𐑅𐐨𐑉𐐯𐐻 or 𐐔𐐯𐑆𐐲𐑉𐐯𐐻}) is a phonemic English spelling reform developed in the mid-19th century by the board of regents of the University of Deseret (later the University of Utah) under the direction of Brigham Young, second president of The Church of Jesus Christ of Latter-day Saints.

In public statements, Young claimed the alphabet was intended to replace the traditional Latin alphabet with an alternative, more phonetically accurate alphabet for the English language. This would offer immigrants an opportunity to learn to read and write English, he said, the orthography of which is often less phonetically consistent than those of many other languages. Similar experiments were not uncommon during the period, the most well-known of which is the Shavian alphabet.

Young also prescribed the learning of Deseret to the school system, stating "It will be the means of introducing uniformity in our orthography, and the years that are now required to learn to read and spell can be devoted to other studies".[2]


Deseret script {\deseret 𐐔𐐯𐑅𐐨𐑉𐐯𐐻}  [U+10400-U+1044F]
\medskip

\bgroup
\par
\noindent
\colorbox{graphicbackground}{\color{black}^^A
\begin{minipage}{\textwidth}^^A
\parindent1pt
\vskip10pt
\leftskip10pt \rightskip\leftskip
\deseret
\large

𐐂 𐑌𐐲𐑉𐑅𐐨𐑉𐐮 𐐮𐑆 𐐪 𐐹𐐨𐑅 𐐱𐑂 𐑊𐐰𐑌𐐼 𐐱𐑌 𐐸𐐶𐐮𐐽 𐑁𐑉𐐭𐐻𐐻𐑉𐐨𐑆 𐐪𐑉 𐑅𐐻𐐪𐑉𐐻𐐯𐐼,


\par
\vspace*{10pt}
\end{minipage}
}

Text: Deseret alphabet http://www.omniglot.com/writing/deseret.htm
\medskip
\egroup

\PrintUnicodeBlock{./languages/deseret.txt}{\deseret}

\chapter{Bamum}
\label{s:bamum}
\epigraph{"No known alphabet was ever invented by a European."}{Jeffreys' translation from the Royal script.}

\label{s:bamum}
\index{scripts>Bamum}
\newfontfamily\bamum{NotoSansBamum-Regular.ttf}

The Bamum scripts are an evolutionary series of six scripts created for the Bamum language by King Njoya of Cameroon at the turn of the 20th century. They are notable for evolving from a pictographic system to a partially alphabetic syllabic script in the space of 14 years, from 1896 to 1910. Bamum type was cast in 1918, but the script fell into disuse around 1931.

\begin{figure}[htbp]
\parindent=0pt

\centering

\includegraphics[width=\textwidth]{bamum}

\caption{King Njoya of Bamum receiving an oil painting of Kaiser Wilhelm II. The gift was in return for his support in the German campaign against the Nso'.}
\end{figure}

The Bamum, sometimes called Bamoum, Bamun, Bamoun, or Mum, are a Bantoid ethnic group of Cameroon with around 215,000 members.



\begin{scriptexample}[]{Bamum}
\unicodetable{bamum}{"A6A0,"A6B0,"A6C0,"A6D0,"A6E0,"A6F0}
\end{scriptexample}
\section{Shavian}
\label{s:shavian}
\def\shaviansetup#1{}
\newfontfamily\shavian{code2001.ttf}
^^A\newfontfamily\shavian{NotoSansShavian-Regular.ttf}
\cxset{shavian font/.code=\shaviansetup{#1}}
\cxset{shavian font=shavian}




\begin{scriptexample}[]{shavian}
\shavian

𐑳 𐑡𐑻𐑯𐑰 𐑑 𐑞 𐑕𐑧𐑯𐑑𐑻 𐑝 𐑞 𐑻𐑔
𐑚𐑲 - ·𐑡𐑵𐑤𐑟 ·𐑝𐑻𐑯

𐑗𐑩𐑐𐑑𐑻 1 - 𐑥𐑲 𐑳𐑙𐑒𐑳𐑤 𐑥𐑱𐑒𐑕 𐑳 𐑜𐑮𐑱𐑑 𐑛𐑦𐑕𐑒𐑳𐑝𐑻𐑰

     𐑤𐑫𐑒𐑦𐑙 𐑚𐑩𐑒 𐑑 𐑷𐑤 𐑞𐑩𐑑 𐑣𐑩𐑟 𐑳𐑒𐑻𐑛 𐑑 𐑥𐑰 𐑕𐑦𐑯𐑕 𐑞𐑩𐑑 𐑦𐑝𐑧𐑯𐑑𐑓𐑳𐑤 𐑛𐑱, 𐑲 𐑩𐑥 𐑕𐑒𐑧𐑮𐑕𐑤𐑰 𐑱𐑚𐑳𐑤 𐑑 𐑚𐑦𐑤𐑰𐑝 𐑦𐑯 𐑞 𐑮𐑰𐑩𐑤𐑳𐑑𐑰 𐑝 𐑥𐑲 𐑩𐑛𐑝𐑧𐑯𐑗𐑻𐑟. 𐑞𐑱 𐑢𐑻 𐑑𐑮𐑵𐑤𐑰 𐑕𐑴 𐑢𐑳𐑯𐑛𐑻𐑓𐑳𐑤 𐑞𐑩𐑑 𐑰𐑝𐑦𐑯 𐑯𐑬 𐑲 𐑩𐑥 𐑚𐑦𐑢𐑦𐑤𐑛𐑻𐑛 𐑢𐑧𐑯 𐑲 𐑔𐑦𐑙𐑒 𐑝 𐑞𐑧𐑥.
     𐑥𐑲 𐑳𐑙𐑒𐑳𐑤 𐑢𐑪𐑟 𐑳 𐑡𐑻𐑥𐑳𐑯, 𐑣𐑩𐑝𐑦𐑙 𐑥𐑧𐑮𐑰𐑛 𐑥𐑲 𐑥𐑳𐑞𐑻𐑟 𐑕𐑦𐑕𐑑𐑻, 𐑩𐑯 𐑦𐑙𐑜𐑤𐑦𐑖𐑢𐑫𐑥𐑳𐑯. 𐑚𐑰𐑦𐑙 𐑝𐑧𐑮𐑰 𐑥𐑳𐑗 𐑳𐑑𐑩𐑗𐑑 𐑑 𐑣𐑦𐑟 𐑓𐑪𐑞𐑻𐑤𐑳𐑕 𐑯𐑧𐑓𐑘𐑵, 𐑣𐑰 𐑦𐑯𐑝𐑲𐑑𐑳𐑛 𐑥𐑰 𐑑 𐑕𐑑𐑳𐑛𐑰 𐑳𐑯𐑛𐑻 𐑣𐑦𐑥 𐑦𐑯 𐑣𐑦𐑟 𐑣𐑴𐑥 𐑦𐑯 𐑞 𐑓𐑪𐑞𐑻𐑤𐑩𐑯𐑛. 𐑞𐑦𐑕 𐑣𐑴𐑥 𐑢𐑪𐑟 𐑦𐑯 𐑳 𐑤𐑪𐑮𐑡 𐑑𐑬𐑯, 𐑯 𐑥𐑲 𐑳𐑙𐑒𐑳𐑤 𐑳 𐑐𐑮𐑳𐑓𐑧𐑕𐑻 𐑝 𐑓𐑳𐑤𐑪𐑕𐑳𐑓𐑰, 𐑒𐑧𐑥𐑳𐑕𐑑𐑮𐑰, 𐑡𐑰𐑪𐑤𐑳𐑡𐑰, 𐑥𐑦𐑯𐑻𐑪𐑤𐑳𐑡𐑰, 𐑯 𐑥𐑧𐑯𐑰 𐑳𐑞𐑻 𐑳𐑤𐑴𐑡𐑰𐑕.

\arial

\hfill Excerpt from Jules Vern,  \textit{Journey to the Center of the Earth from \href{http://shavian.weebly.com/}{shavian}}
\end{scriptexample}

The example is typeset using \texttt{code2001.ttf}. There are numerous fonts that provide Shavian glyphs. \texttt{ESL Gothic Unicode} font by Ethan Lamoreaux\footnote{\url{http://www.fontspace.com/ethan-lamoreaux/esl-gothic-unicode}}. The Noto fonts also have a Shavian font. 

You can activate typesetting in Shavian using the key:

\begin{key}{/chapter/shavian font = \meta{font name}} The key will setup the
default font for the Shavian script and define the commands \cmd{\shavian} and \cmd{\textshavian}. 
\end{key}

\PrintUnicodeBlock{./languages/shavian.txt}{\shavian}





\subsection{Osmanya}

\newfontfamily\osmanya{NotoSansOsmanya-Regular.ttf}

\begin{scriptexample}[]{Osmanya}
\unicodetable{osmanya}{"10480,"10490,"104A0}
\end{scriptexample}

The Osmanya alphabet (Somali: Cismaanya; Osmanya: {\osmanya 𐒋𐒘𐒈𐒑𐒛𐒒𐒕𐒀}), also known as Far Soomaali ("Somali writing"), is a writing script created to transcribe the Somali language. It was invented between 1920 and 1922 by Osman Yusuf Kenadid of the Majeerteen Darod clan, the nephew of Sultan Yusuf Ali Kenadid of the Sultanate of Hobyo.

While Osmanya gained reasonably wide acceptance in Somalia and quickly produced a considerable body of literature, it proved difficult to spread among the population mainly due to stiff competition from the long-established Arabic script as well as the emerging Somali alphabet developed by the Somali linguist, Shire Jama Ahmed, which was based on the Latin script.

As nationalist sentiments grew and since the Somali language had long lost its ancient script,[1] the adoption of a universally recognized writing script for the Somali language became an important point of discussion. After independence, little progress was made on the issue, as opinion was divided over whether the Arabic or Latin scripts should be used instead.

In October 1972, due to its simplicity, the fact that it lent itself well to writing Somali since it could cope with all of the sounds in the language, and the already widespread existence of machines and typewriters designed for its use,[2][3] the government of Somali president Mohamed Siad Barre unilaterally elected to use only the Latin script for writing Somali instead of the Arabic or Osmanya scripts.[4] Barre's administration subsequently launched a massive literacy campaign designed to ensure its sole adoption. This led to a sharp decline in use of Osmanya.
\section{Cherokee}
\index{scripts>Cherokee}
\index{scripts>Cherokee>fonts}
\label{sec:cherokee}
Windows comes with |Plantagenet Cherokee| font. The |code2000| also has good support for the alphabet. The \texttt{SIL font Charis SIL} also has good support and can be downloaded at \href{http://scripts.sil.org/cms/scripts/page.php?item_id=CharisSIL_download}{scripts.sel.org}, the latest version gave me problems when used with Windows. 

  
\def\textcherokee#1{{\cherokee   #1}}


\begin{docKey}[phd]{cherokee font}{ = \meta{font name}} {default none, initial=code2000}
 Loads the font
command \cmd{\cherokee}. When the command is used it typesets text in
cherokee unicode. There is no need to load the language, unless it is the main document language. For windows the default font is  |Plantagenet Cherokee|. Another font is FreeSerif, which we are using here.
\end{docKey}

\begin{scriptexample}[]{Cherokee}
{\cherokee
\begin{tabular}{lp{8.5cm}}
Translation	  &John (ᏣᏂ) 3:16\\
American Bible Society 1860	&ᎾᏍᎩᏰᏃ ᏂᎦᎥᎩ ᎤᏁᎳᏅᎯ ᎤᎨᏳᏒᎩ ᎡᎶᎯ, ᏕᏅᏲᏒᎩ ᎤᏤᎵᎦ ᎤᏪᏥ ᎤᏩᏒᎯᏳ ᎤᏕᏁᎸᎯ, ᎩᎶ ᎾᏍᎩ ᏱᎪᎯᏳᎲᏍᎦ ᎤᏲᎱᎯᏍᏗᏱ ᏂᎨᏒᎾ, ᎬᏂᏛᏉᏍᎩᏂ ᎤᏩᏛᏗ.\\

(Transliteration)	& nasgiyeno nigavgi unelanvhi ugeyusvgi elohi, denvyosvgi utseliga uwetsi uwasvhiyu udenelvhi, gilo nasgi yigohiyuhvsga uyohuhisdiyi nigesvna, gvnidvquosgini uwadvdi.\\
\end{tabular}}
\end{scriptexample}

\begin{texexample}{Using text...}{cherokee}
\bgroup
\cherokee \large\textbf{ᎾᏍᎩᏰᏃ}
\textcherokee{ᎾᏍᎩᏰᏃ}
\egroup
\end{texexample}

If you have trouble getting them to work\footnote{\url{http://tex.stackexchange.com/questions/132087/displaying-cherokee-text}}

\url{http://www.cherokee.org/AboutTheNation/Language/CherokeeFont.aspx}




\section{Tifnagh}

\newfontfamily\tifinagh{code2000.ttf}

Tifinagh (Berber pronunciation: [tifinaɣ]; also written Tifinaɣ in the Berber Latin alphabet, {\tifinagh  ⵜⵉⴼⵉⵏⴰⵖ} in Neo-Tifinagh, and تيفيناغ in the Berber Arabic alphabet) is a series of abjad and alphabetic scripts used by Berber peoples to write Berber languages.[1]
A modern derivate of the traditional script, known as Neo-Tifinagh, was introduced in the 20th century. A slightly modified version of the traditional script, called Tifinagh Ircam, is used in a number of Moroccan elementary schools in teaching the Berber language to children as well as a number of publications.[2][3]

The word tifinagh is thought to be a Berberized feminine plural cognate of Punic, through the Berber feminine prefix ti- and Latin Punicus; thus tifinagh could possibly mean "the Phoenician (letters)"[4][5] or "the Punic letters".

\bgroup

\noindent\tifinagh
\colorbox{thecodebackground}{\color{black}^^A
\begin{minipage}{\textwidth}
\parindent1pt
\vskip10pt
\leftskip10pt \rightskip\leftskip
Tifnagh     ⵜⵉⴼⵉⵏⴰⵖ [U+2D30-U+2D7F]

ⴰⴳⵍⴷⵓⵏ ⴰⵎⵥⵥⴰ

ⵙ ⵡⴰⵡⴰⵍ ⴳⵔⵉ ⵉⴷⵙ, ⵙⵙⵏⵖ ⵢⴰⵜ ⵜⵖⴰⵡⵙⴰ ⵜⵉⵙⵙ ⵙⵏⴰⵜ  ⵉⵅⴰⵜⵔⵏ: ⵉⵜⵔⵉ ⵙⴳ ⴷⴷ ⵉⴷⴷⴰ ⵓⵔ ⵉⵎⵇⵇⵓⵔ, ⵉⵍⵍⴰ ⵖⴰⵙ ⴰⵏⵛⵜ ⵏ ⵢⴰⵜ ⵜⴰⴷⴷⴰⵔⵜ !

ⴰⵢⴰ ⵓⴽⵣⵖ ⵜ. ⵙⵙⵏⵖ ⵉⵙ ⴱⵕⵕⴰ ⵏ ⵉⵜⵔⴰⵏ ⵣⵓⵏⴷ ⴰⴽⴰⵍ, ⵊⵓⴱⵉⵜⵔ, ⵎⴰⵔⵙ, ⴱⵉⵏⵓⵙ – ⵉⵜⵔⴰⵏ ⵎⵉ ⵏⴽⴼⴰ ⵉⵙⵎⴰⵡⵏ – ⵍⵍⴰⵏ ⴷⵉⵖ ⵉⵜⵔⴰⵏ ⵢⴰⴹⵏ ⵎⵥⵥⵉⵢⵏⵉⵏ, ⵡⵉⵏⵏⴰ ⵓⵔ ⵏⵣⵎⵉⵔ ⴰⴷ ⵏⵥⵔ ⵙ ⵓⵜⵉⵍⵉⵙⴽⵓⴱ. ⴰⴷⴷⴰⵢ ⵢⵓⴼⴰ ⵓⴰⵙⵜⵕⵓⵏⵓⵎ ⵢⴰⵏ ⴷⵉⴳⵙⵏ, ⴷⴰ ⵢⴰⵙ ⵉⵜⵜⴳⴰ ⵙ ⵢⵉⵙⵎ ⵢⴰⵏ ⵡⵓⵜⵜⵓⵏ. ⴷⴰ ⵢⴰⵙ ⵉⵇⵇⴰⵔ ⵙ ⵓⵎⴷⵢⴰⵜ : « ⴰⵙⵜⵔⵓⵉⴷ 3251 ».

ⵓⴽⵣⵖ ⵉⵙ ⴷⴷ ⵉⴷⴷⴰ ⵓⴳⵍⴷⵓⵏ ⵎⵥⵥⵉⵢⵏ ⵙⴳ ⵉⵜⵔⵉ ⵎⵉ ⵇⵇⴰⵔⵏ ⴰⵙⵜⵔⵓⵉⴷ ⴱ612. ⴰⵙⵜⵔⵓⵉⴷ ⴰ, ⵓⵔ ⵉⵜⵓⵥⵔⴰ ⴰⵔ 1909 ⵙ ⵓⵜⵉⵍⵉⵙⴽⵓⴱ. ⵉⵥⵔⴰ ⵜ ⵢⴰⵏ ⵓⴰⵙⵜⵕⵓⵏⵓⵎ ⴰⵜⵓⵔⴽⵉⵢ. ⵉⵙⵙⴽⵏ ⵜⵓⴼⴰⵢⵜ ⵏⵏⵙ ⴳ ⵢⴰⵏ ⵓⴳⵔⴰⵡ ⴰⴳⵔⴰⵖⵍⴰⵏ ⵏ ⵍⴰⵙⵜⵕⵓⵏⵓⵎⵢ. ⵎⴰⵛⴰ, ⴰⴽⴷ ⵢⵉⵡⵏ ⵓⵔ ⵜ ⵢⵓⵎⵏ ⴰⵛⴽⵓ ⵉⵍⵍⴰ ⵉⵍⵙⴰ ⵢⴰⵜ ⵎⵍⵙⵉⵡⵜ ⵓⵔ ⵉⴳⵉⵏ ⴰⵎⵎ ⵜⵉⵏ ⵎⴷⴷⵏ. ⵎⴷⴷⵏ ⵉⵎⵇⵔⴰⵏⴻⵏ, ⴰⵎⴽⴰ ⴰⴽⴽ ⴰⵢ ⴳⴰⵏ.

ⵎⴰⵛⴰ ⵙ ⵓⵎⴷⴰⵣ ⵏ ⵜⵓⵙⵙⵏⴰ ⵏ ⴰⵙⵜⵔⵓⵉⴷ ⴱ612, ⵉⴽⴽⵔ ⵢⴰⵏ ⵓⴷⵉⴽⵜⴰⵜⵓⵔ ⴰⵜⵓⵔⴽⵢ, ⵉⴳⴳ ⴰⵙⵏ ⵛⵛⵉⵍ ⵉ ⵎⴷⴷⵏ ⴰⴷ ⵍⵙⵙⴰⵏ ⵎⵍⵙⵉⵡⵜ ⵏ ⵓⵔⵓⴱⵉⵢⵏ, ⵡⴰⵏⵏⴰ ⵢⴰⴳⵉⵏ ⵉⵏⵖ ⵜ. ⴰⵙⵜⵔⵓⵏⵓⵎ ⵏⵏⴰⵖ, ⵢⵓⵍⵙ ⴷⵉⵖ ⵉ ⵜⵎⵙⴽⴰⵏⵜ ⵏⵏⵙ ⴰⵙⴳⴳⴰⵙ ⵏ 1920, ⵜⵉⴽⴽⵍⵜ ⵏⵏⴰⵖ ⵉⵍⵍⴰ ⵉⵍⵙⴰ ⵢⴰⵜ ⵎⵍⵙⵉⵡⵜ ⵢⵖⵓⴷⴰⵏ ⵛⵉⴳⴰⵏ. ⵜⵉⴽⴽⵍⵜ ⵏⵏⴰⵖ, ⵎⴷⴷⵏ ⴰⴽⴽ ⵓⵎⴻⵏ ⴰⵡⴰⵍ ⵏⵏⵙ.
\par
\vspace*{10pt}
\end{minipage}
}

\subsection{Unified Canadian Aboriginal Syllabics}

Unified Canadian Aboriginal Syllabics is a Unicode block containing characters for writing Inuktitut, Carrier, several dialects of Cree, and Canadian Athabascan languages. Additions for some Cree dialects, Ojibwe, and Dene can be found at the Unified Canadian Aboriginal Syllabics Extended block.
\medskip

\newfontfamily\aboriginal{code2000.ttf}
\bgroup
\par
\noindent
\colorbox{graphicbackground}{\color{black}^^A
\begin{minipage}{\textwidth}^^A
\parindent1pt
\vskip10pt
\leftskip10pt \rightskip\leftskip

\aboriginal
ᒥᓯᐌ ᐃᓂᓂᐤ ᑎᐯᓂᒥᑎᓱᐎᓂᐠ ᐁᔑ ᓂᑕᐎᑭᐟ ᓀᐢᑕ ᐯᔭᑾᐣ ᑭᒋ ᐃᔑ
\bfseries ᑲᓇᐗᐸᒥᑯᐎᓯᐟ ᑭᐢᑌᓂᒥᑎᓱᐎᓂᐠ ᓀᐢᑕ ᒥᓂᑯᐎᓯᐎᓇ᙮
Unicode Block: Unified Canadian Aboriginal Syllabics, UCAS Extended
Text: UDHR: Cree, Swampy ᐯᔭᐠ ᐱᐢᑭᑕᓯᓇᐃᑲᐣ ᐁᐢᐱᑕᐢᑲᒥᑲᐠ ᐊᐢᑭᐠ ᑭᒋ ᐃᑗᐎᐣ ᐃᓂᓂᐎ ᒥᓂᑯᐎᓯᐎᓇ ᐅᒋ
\par
\vspace*{10pt}
\end{minipage}
}
\medskip
\egroup
\subsection{Miao}

The Pollard script, also known as Pollard Miao (Chinese: 柏格理苗文 Bó Gélǐ Miao-wen) or Miao, is an abugida loosely based on the Latin alphabet and invented by Methodist missionary Sam Pollard. Pollard invented the script for use with A-Hmao, one of several Miao languages. The script underwent a series of revisions until 1936, when a translation of the New Testament was published using it. The introduction of Christian materials in the script that Pollard invented caused a great impact among the Miao. Part of the reason was that they had a legend about how their ancestors had possessed a script but lost it. According to the legend, the script would be brought back some day. When the script was introduced, many Miao came from far away to see and learn it.[1][2]

Pollard credited the basic idea of the script to the Cree syllabics designed by James Evans in 1838–1841, “While working out the problem, we remembered the case of the syllabics used by a Methodist missionary among the Indians of North America, and resolved to do as he had done” (1919:174). He also gave credit to a Chinese pastor, “Stephen Lee assisted me very ably in this matter, and at last we arrived at a system” (1919:174). In listing the phrases he used to describe devising the script, there is clear indication of intellectual work, not revelation: “we looked about”, “resolved to attempt”, “adapting the system”, “solved our problem” (Pollard 1919:174,175).

Changing politics in China led to the use of several competing scripts, most of which were romanizations. The Pollard script remains popular among Hmong in China, although Hmong outside China tend to use one of the alternative scripts. A revision of the script was completed in 1988, which remains in use.

As with most other abugidas, the Pollard letters represent consonants, whereas vowels are indicated by diacritics. Uniquely, however, the position of this diacritic is varied to represent tone. For example, in Western Hmong, placing the vowel diacritic above the consonant letter indicates that the syllable has a high tone, whereas placing it at the bottom right indicates a low tone.

A still experimental font, that supports Graphite technology is \idxfont{Mia Unicode}\footnote{\url{http://phjamr.github.io/miao.html\#intro}}. The font is licenced under the SIL terms and we are using it in the |phd| package as the default font for the Miao script.

\newfontfamily\miao{MiaoUnicode-Regular.ttf}

\begin{scriptexample}[]{Miao}
\unicodetable{miao}{"16F00,"16F10,"16F20,"16F30,"16F40,"16F70,"16F80,"16F90}
\end{scriptexample}

{\miao 𖼴	𖼵	𖼶	𖼷	𖼸	𖼹	𖼺	}

Features for Miao
There are three features currently available for the Miao script:
\bgroup
\miao
Chuxiong ‘wart’ variant
Stylistic alternates for 𖼳 and 𖼴
Aspiration marker always on right
The ‘wart’ (a translated technical term!) is the small circle in characters like 𖼁, 𖼅, and 𖼾. In the Chuxiong orthography, it is rendered not as a circle but as a dot on the right of the letter, as shown in point 5 here (pdf).

Miao Unicode has a feature called “chux” for handling this. In LibreOffice you can use this style by typing “Miao Unicode:chux=1” into the font field.
\section{N'ko}

\newfontfamily\nko{NotoSansNKo-Regular.ttf}

N'Ko {\nko(ߒߞߏ)} is both a script devised by Solomana Kante in 1949 as a writing system for the Manding languages of West Africa, and the name of the literary language itself written in the script. The term N'Ko means ``I say'' in all Manding languages.

The script has a few similarities to the Arabic script, notably its direction (right-to-left) and the connected letters. It obligatorily marks both tone and vowels.


\begin{scriptexample}[]{N'ko}
\unicodetable{nko}{"07C0,"07D0,"07E0,"07F0}
\end{scriptexample}

The N'Ko alphabet is written from right to left, with letters being connected to one another.

The script is principally used in Guinea and Côte d'Ivoire (respectively by Maninka and Dioula-speakers), with an active user community in Mali (by Bambara-speakers). Publications include a translation of the Qur'an, a variety of textbooks on subjects such as physics and geography, poetic and philosophical works, descriptions of traditional medicine, a dictionary, and several local newspapers. It has been classed as the most successful of the West African scripts.[3] The literary language used is intended as a koine blending elements of the principal Manding languages (which are mutually intelligible), but has a particularly strong Maninka flavour.

The Latin script with several extended characters (phonetic additions) is used for all Manding languages to one degree or another for historical reasons and because of its adoption for "official" transcriptions of the languages by various governments. In some cases, such as with Bambara in Mali, promotion of literacy using this orthography has led to a fair degree of literacy in it. Arabic transcription is commonly used for Mandinka in The Gambia and Senegal.


\subsection{Mongolian}
\newfontfamily\mongolian{NotoSansMongolian-Regular.ttf}

The classical Mongolian script (in Mongolian script:{\mongolian ᠮᠣᠩᠭᠣᠯ ᠪᠢᠴᠢᠭ᠌} Mongγol bičig; in Mongolian Cyrillic: Монгол бичиг Mongol bichig), also known as Uyghurjin Mongol bichig, was the first writing system created specifically for the Mongolian language, and was the most successful until the introduction of Cyrillic in 1946. Derived from Uighur, Mongolian is a true alphabet, with separate letters for consonants and vowels. The Mongolian script has been adapted to write languages such as Oirat and Manchu. Alphabets based on this classical vertical script are used in Inner Mongolia and other parts of China to this day to write Mongolian, Sibe and, experimentally, Evenki.

\begin{scriptexample}[]{Mongolian}
\unicodetable{mongolian}{"1820,"1830,"1840,"1850,"1860,"1870,"1880,"1890,"18A0}
\end{scriptexample}


    \subsection{Greek}
\index{languages>Greek}\index{Herodotus}\index{alphabets>Greek}
\newfontfamily\greek[Script=Greek,Scale=1.02]{NotoSerif-Regular.ttf}
\def\greektext#1{\greek{#1}}

`The Phoenicians who came with Kadmos,' wrote Herodotus in the fifth century BC of the legendary Phoenician prince of Tyre and brother of Europa, `\ldots introduced into Greece, after their settlement in the country, a number of accomplishments of which the most important was writing, an art which probably was unknown to the Greeks until then'. 

The Greek alphabet is the script that has been used to write the Greek language since the 8th century BC.[2] It was derived from the earlier Phoenician alphabet, and was in turn the ancestor of numerous other European and Middle Eastern scripts, including Cyrillic and Latin.[3] Apart from its use in writing the Greek language, both in its ancient and its modern forms, the Greek alphabet today also serves as a source of technical symbols and labels in many domains of mathematics, science and other fields.

In its classical and modern forms, the alphabet has 24 letters, ordered from alpha to omega. Like Latin and Cyrillic, Greek originally had only a single form of each letter; it developed the letter case distinction between upper-case and lower-case forms in parallel with Latin during the modern era.

\bgroup
\greek\obeyspaces

Α	ἄλφα	aleph	alpha	[alpʰa]	[ˈalfa]	Listeni/ˈælfə/
Β	βῆτα	beth	beta	[bɛːta]	[ˈvita]	/ˈbiːtə/, US /ˈbeɪtə/
Γ	γάμμα	gimel	gamma	[ɡamma]	[ˈɣama]	/ˈɡæmə/
Δ	δέλτα	daleth	delta	[delta]	[ˈðelta]	/ˈdɛltə/
Η	ἦτα	  heth	   eta	 [hɛːta], [ɛːta]	[ˈita]	/ˈiːtə/, US /ˈeɪtə/
Θ	θῆτα	teth	theta	[tʰɛːta]	[ˈθita]	/ˈθiːtə/, US Listeni/ˈθeɪtə/
Ι	ἰῶτα	yodh	iota	[iɔːta]	[ˈʝota]	Listeni/aɪˈoʊtə/
Κ	κάππα	kaph	kappa	[kappa]	[ˈkapa]	Listeni/ˈkæpə/
Λ	λάμβδα	lamedh	lambda	[lambda]	[ˈlamða]	Listeni/ˈlæmdə/
Μ	μῦ	mem	mu	[myː]	[mi]	Listeni/ˈmjuː/; occasionally US /ˈmuː/
Ν	νῦ	nun	nu	[nyː]	[ni]	/ˈnjuː/ (US /ˈnuː/)
Ρ	ῥῶ	reš	rho	[rɔː]	[ro]	Listeni/ˈroʊ/
Τ	ταῦ	taw	tau	[tau]	[taf]	/ˈtaʊ/ or /ˈtɔː/

\topline
\begin{quote}
Ἡροδότου Ἁλικαρνησσέος ἱστορίης ἀπόδεξις ἥδε, ὡς μήτε τὰ γενόμενα ἐξ ἀνθρώπων τῷ χρόνῳ ἐξίτηλα γένηται, μήτε ἔργα μεγάλα τε καὶ θωμαστά, τὰ μὲν Ἕλλησι, τὰ δὲ βαρβάροισι ἀποδεχθέντα, ἀκλεᾶ γένηται, τὰ τε ἄλλα καὶ δι' ἣν αἰτίην ἐπολέμησαν ἀλλήλοισι.[2]

Herodotus of Halicarnassus, his Researches are set down to preserve the memory of the past by putting on record the astonishing achievements of both the Greeks and the Barbarians; and more particularly, to show how they came into conflict.[3]
\end{quote}
\bottomline

\symbol{"1F00}
\symbol{"1F01}
\egroup
 }

\makeatletter
\mainmatter
\newcommand{\normalencoding}{\fontencoding{OT1}}
\newcommand\ttverb[1]{\texttt{\string#1}}
\newfontfamily{\tiresias}{Tiresias PCfont}
\cxset{style87/.style={
 chapter opening=any,
 name=Chapter,
 % positioning and float - inline is 0
 %  float right is 2
 number display=block,
 number float=right,
 number shape=starburst,
 numbering=Words,
 number spaceout=none,
 number font-size=huge,
 number font-weight=bold,
 number font-family=rmfamily,
 number font-shape=normal,
 number before=,
 number display=inline,
 number float=none,
% 
 number border-top-width=0pt,
 number border-right-width=0pt,
 number border-bottom-width=0pt,
 number border-left-width=0pt,
 number border-width=0pt,
%  
 number padding-left=0em,
 number padding-right=0.5em,
 number padding-top=0em,
 number padding-bottom=0pt,
  %number margin-top=, to do
 %number margin-left=0pt,  to create
 %
 number after=\par,
 number dot=,
 number position=rightname,
 number color=sweet,
 number background-color=white,
 %chapter name
 chapter display=block,
 chapter float=left,
 chapter shape=ellipse,
 chapter color=black,
 chapter background-color=sweet,
 chapter font-size= Huge,
 chapter font-weight=bfseries,
 chapter font-family=itshape,
 chapter before=,
 chapter spaceout=none,
 chapter after=,
 chapter margin-left=0cm,
 chapter margin-top=0pt,
 %
 chapter border-width=2pt,
 chapter border-top-width=1pt,
 chapter border-right-width=1pt,
 chapter border-bottom-width=1pt,
 chapter border-left-width=4pt,
% 
 chapter padding-left=20pt,
 chapter padding-right=20pt,
 chapter padding-top=20pt,
 chapter padding-bottom=10pt,
  %chapter title
 title font-family=rmfamily,
 title font-color=black!80,
 title font-weight=bfseries,
 title font-size=huge,
 chapter title align=none,
 title margin-left=1cm,
 title margin bottom=1.3cm,
 title margin top=30pt,
 % title borders
 title border-width=0pt,
 title padding=0pt,
 title border-color=black!80,
% title border-top-color=spot!50,
% title border-top-width=20pt,
 title border-left-color=black!80,
 title border-left-width=2pt,
 title border-color=black!80,
 title padding-top=10pt,
 title padding-bottom=10pt,
 title padding-left=10pt,
 title padding-right=0pt,
% title border-right-color=spot!50,
% title border-right-width=20pt,
% title border-bottom-color=spot!50,
% title border-bottom-width=20pt,
 %
 chapter title align=left,
 chapter title text-align=left,
 chapter title width=0.8\textwidth,
 title before=,
 title after=,
 title display=block,
 title beforeskip=12pt,
 title afterskip=12pt,
 author block=false,
 section font-family=rmfamily,
 section font-size=LARGE,
 section font-weight=bfseries,
 section indent=0pt,
  section font-weight=mdseries,
 section align=left,
 subsubsection font-family=tiresias,
 subsubsection font-shape=upshape,
 subsubsection font-weight=mdseries,
 subsubsection align=flushleft,
 epigraph width=\dimexpr(\textwidth-2cm)\relax,
 epigraph align=center,
 epigraph text align=center,
 epigraph rule width=0pt,
 header style=plain}}
 
\cxset{style87}
\renewsection\renewsubsection\renewsubsubsection

\makeatletter
\cxset{enumerate numberingi/.is choice,
  enumerate numberingi/.code={\renewcommand\theenumi {\csname#1\endcsname{enumi}}},
  enumerate numberingii/.code={\renewcommand\theenumii {\csname#1\endcsname{enumii}}},
  enumerate numberingiii/.code={\renewcommand\theenumiii {\csname#1\endcsname{enumiii}}},
  enumerate numberingiv/.code={\renewcommand\theenumiv {\csname#1\endcsname{enumiv}}},
  enumerate labeli punctuation/.store in=\enumeratepunctuationi@cx,
  enumerate labeli/.is choice,
  enumerate labeli/brackets/.code={\renewcommand\labelenumi{(\theenumi\enumeratepunctuationi@cx)}},
  enumerate labeli/square brackets/.code={\renewcommand\labelenumi{[\theenumi\enumeratepunctuationi@cx]}},
  enumerate labeli/right bracket/.code={\renewcommand\labelenumi{\theenumi\enumeratepunctuationi@cx)}},
  enumerate label left/.store in=\enumeratelabelleft@cx,
  enumerate label right/.code=\renewcommand\labelenumi{\enumeratelabelleft@cx\theenumi\enumeratepunctuationi@cx#1},
  enumerate leftmargini/.code={\setlength\leftmargini{#1}},
  enumerate leftmarginii/.code={\setlength\leftmarginii{#1}},
  enumerate leftmarginiii/.code={\setlength\leftmarginiii{#1}},
  enumerate leftmarginiv/.code={\setlength\leftmarginiv{#1}},
  listi topsep/.store in=\listitopsep@cx,
  listi partopsep/.store in=\listipartopsep@cx,
  listi itemsep/.store in=\listiitemsep@cx,
  listi parsep/.store in=\listiparsep@cx,
  listii topsep/.store in=\listiitopsep@cx,
  listii partopsep/.store in=\listiipartopsep@cx,
  listii itemsep/.store in=\listiiitemsep@cx,
  listii parsep/.store in=\listiiparsep@cx,
  listiii topsep/.store in=\listiiitopsep@cx,
  listiii partopsep/.store in=\listiiipartopsep@cx,
  listiii itemsep/.store in=\listiiiitemsep@cx,
  listiii parsep/.store in=\listiiiparsep@cx,
}
\cxset{compact1/.style={%
  enumerate numberingi=arabic,
  enumerate numberingii=alph,
  enumerate numberingiii=alph,
  enumerate numberingiv=roman,
  enumerate labeli punctuation=.,
  enumerate label left=,
  enumerate label right=,
  enumerate leftmargini=2.2em,
  enumerate leftmarginii=2.1em,
  enumerate leftmarginiii=1.5em,
  enumerate leftmarginiv=2em,
  listi topsep=8\p@ \@plus2\p@ \@minus\p@,
  listi itemsep=0\p@ \@plus2\p@ \@minus\p@,
  listi parsep=0\p@ \@plus2\p@ \@minus\p@,
  listii topsep=0\p@ \@plus2\p@ \@minus\p@,
  listii itemsep=0\p@ \@plus2\p@ \@minus\p@,
  listii parsep=0\p@ \@plus2\p@ \@minus\p@,
  listiii topsep=0\p@ \@plus2\p@ \@minus\p@,
  listiii itemsep=0\p@ \@plus2\p@ \@minus\p@,
  listiii parsep=0\p@ \@plus2\p@ \@minus\p@,
}}
\cxset{compact2/.style={%
  enumerate numberingi=alph,
  enumerate numberingii=roman,
  enumerate numberingiii=alph,
  enumerate numberingiv=roman,
  enumerate labeli punctuation=,
  enumerate label left=(,
  enumerate label right=),
  enumerate leftmargini=2.2em,
  enumerate leftmarginii=2.1em,
  enumerate leftmarginiii=1.5em,
  enumerate leftmarginiv=2em,
  listi topsep   = 8\p@ \@plus2\p@ \@minus\p@,
  listi itemsep = 0\p@ \@plus2\p@ \@minus\p@,
  listi parsep   = 0\p@ \@plus2\p@ \@minus\p@,
  listii topsep  = 0\p@ \@plus2\p@ \@minus\p@,
  listii itemsep= 0\p@ \@plus2\p@ \@minus\p@,
  listii parsep  = 0\p@ \@plus2\p@ \@minus\p@,
  listiii topsep = 0\p@ \@plus2\p@ \@minus\p@,
  listiii itemsep= 0\p@ \@plus2\p@ \@minus\p@,
  listiii parsep  = 0\p@ \@plus2\p@ \@minus\p@,
}}

\ExplSyntaxOn
\def\setenumerate#1{
\cxset{#1}
\def\@listi{%
           \leftmargin\leftmargini
            \parsep\listiparsep@cx
            \topsep\listitopsep@cx\relax
            \itemsep\listiitemsep@cx}
            
\def\@listii{\leftmargin\leftmarginii
            \parsep\listiiparsep@cx
            \topsep\listiitopsep@cx\relax
            \itemsep\listiiitemsep@cx}
            
\def\@listiii{\leftmargin\leftmarginiii
            \parsep\listiiiparsep@cx
            \topsep\listiiitopsep@cx\relax
            \itemsep\listiiiitemsep@cx}
}


\setenumerate{compact1}
\ExplSyntaxOff
\makeatother
\cxset{style87/.style={
 chapter opening=any,
 name=Chapter,
 % positioning and float - inline is 0
 %  float right is 2
 number display=block,
 number float=right,
 number shape=starburst,
 numbering=Words,
 number spaceout=none,
 number font-size=huge,
 number font-weight=bold,
 number font-family=rmfamily,
 number font-shape=normal,
 number before=,
 number display=inline,
 number float=none,
% 
 number border-top-width=0pt,
 number border-right-width=0pt,
 number border-bottom-width=0pt,
 number border-left-width=0pt,
 number border-width=0pt,
%  
 number padding-left=0em,
 number padding-right=0.5em,
 number padding-top=0em,
 number padding-bottom=0pt,
  %number margin-top=, to do
 %number margin-left=0pt,  to create
 %
 number after=\par,
 number dot=,
 number position=rightname,
 number color=sweet,
 number background-color=white,
 %chapter name
 chapter display=block,
 chapter float=left,
 chapter shape=ellipse,
 chapter color=black,
 chapter background-color=sweet,
 chapter font-size= Huge,
 chapter font-weight=bfseries,
 chapter font-family=itshape,
 chapter before=,
 chapter spaceout=none,
 chapter after=,
 chapter margin-left=0cm,
 chapter margin-top=0pt,
 %
 chapter border-width=2pt,
 chapter border-top-width=1pt,
 chapter border-right-width=1pt,
 chapter border-bottom-width=1pt,
 chapter border-left-width=4pt,
% 
 chapter padding-left=20pt,
 chapter padding-right=20pt,
 chapter padding-top=20pt,
 chapter padding-bottom=10pt,
  %chapter title
 title font-family=rmfamily,
 title font-color=black!80,
 title font-weight=bfseries,
 title font-size=huge,
 chapter title align=none,
 title margin-left=1cm,
 title margin bottom=1.3cm,
 title margin top=30pt,
 % title borders
 title border-width=0pt,
 title padding=0pt,
 title border-color=black!80,
% title border-top-color=spot!50,
% title border-top-width=20pt,
 title border-left-color=black!80,
 title border-left-width=2pt,
 title border-color=black!80,
 title padding-top=10pt,
 title padding-bottom=10pt,
 title padding-left=10pt,
 title padding-right=0pt,
% title border-right-color=spot!50,
% title border-right-width=20pt,
% title border-bottom-color=spot!50,
% title border-bottom-width=20pt,
 %
 chapter title align=left,
 chapter title text-align=left,
 chapter title width=0.8\textwidth,
 title before=,
 title after=,
 title display=block,
 title beforeskip=12pt,
 title afterskip=12pt,
 author block=false,
 section font-family=rmfamily,
 section font-size=LARGE,
 section font-weight=bfseries,
 section indent=0pt,
  section font-weight=mdseries,
 section align=left,
 subsubsection font-family=tiresias,
 subsubsection font-shape=upshape,
 subsubsection font-weight=mdseries,
 subsubsection align=flushleft,
 epigraph width=\dimexpr(\textwidth-2cm)\relax,
 epigraph align=center,
 epigraph text align=center,
 epigraph rule width=0pt,
 header style=plain}}
 
\cxset{style87}
\renewsection\renewsubsection\renewsubsubsection

\makeatletter
\cxset{enumerate numberingi/.is choice,
  enumerate numberingi/.code={\renewcommand\theenumi {\csname#1\endcsname{enumi}}},
  enumerate numberingii/.code={\renewcommand\theenumii {\csname#1\endcsname{enumii}}},
  enumerate numberingiii/.code={\renewcommand\theenumiii {\csname#1\endcsname{enumiii}}},
  enumerate numberingiv/.code={\renewcommand\theenumiv {\csname#1\endcsname{enumiv}}},
  enumerate labeli punctuation/.store in=\enumeratepunctuationi@cx,
  enumerate labeli/.is choice,
  enumerate labeli/brackets/.code={\renewcommand\labelenumi{(\theenumi\enumeratepunctuationi@cx)}},
  enumerate labeli/square brackets/.code={\renewcommand\labelenumi{[\theenumi\enumeratepunctuationi@cx]}},
  enumerate labeli/right bracket/.code={\renewcommand\labelenumi{\theenumi\enumeratepunctuationi@cx)}},
  enumerate label left/.store in=\enumeratelabelleft@cx,
  enumerate label right/.code=\renewcommand\labelenumi{\enumeratelabelleft@cx\theenumi\enumeratepunctuationi@cx#1},
  enumerate leftmargini/.code={\setlength\leftmargini{#1}},
  enumerate leftmarginii/.code={\setlength\leftmarginii{#1}},
  enumerate leftmarginiii/.code={\setlength\leftmarginiii{#1}},
  enumerate leftmarginiv/.code={\setlength\leftmarginiv{#1}},
  listi topsep/.store in=\listitopsep@cx,
  listi partopsep/.store in=\listipartopsep@cx,
  listi itemsep/.store in=\listiitemsep@cx,
  listi parsep/.store in=\listiparsep@cx,
  listii topsep/.store in=\listiitopsep@cx,
  listii partopsep/.store in=\listiipartopsep@cx,
  listii itemsep/.store in=\listiiitemsep@cx,
  listii parsep/.store in=\listiiparsep@cx,
  listiii topsep/.store in=\listiiitopsep@cx,
  listiii partopsep/.store in=\listiiipartopsep@cx,
  listiii itemsep/.store in=\listiiiitemsep@cx,
  listiii parsep/.store in=\listiiiparsep@cx,
}
\cxset{compact1/.style={%
  enumerate numberingi=arabic,
  enumerate numberingii=alph,
  enumerate numberingiii=alph,
  enumerate numberingiv=roman,
  enumerate labeli punctuation=.,
  enumerate label left=,
  enumerate label right=,
  enumerate leftmargini=2.2em,
  enumerate leftmarginii=2.1em,
  enumerate leftmarginiii=1.5em,
  enumerate leftmarginiv=2em,
  listi topsep=8\p@ \@plus2\p@ \@minus\p@,
  listi itemsep=0\p@ \@plus2\p@ \@minus\p@,
  listi parsep=0\p@ \@plus2\p@ \@minus\p@,
  listii topsep=0\p@ \@plus2\p@ \@minus\p@,
  listii itemsep=0\p@ \@plus2\p@ \@minus\p@,
  listii parsep=0\p@ \@plus2\p@ \@minus\p@,
  listiii topsep=0\p@ \@plus2\p@ \@minus\p@,
  listiii itemsep=0\p@ \@plus2\p@ \@minus\p@,
  listiii parsep=0\p@ \@plus2\p@ \@minus\p@,
}}
\cxset{compact2/.style={%
  enumerate numberingi=alph,
  enumerate numberingii=roman,
  enumerate numberingiii=alph,
  enumerate numberingiv=roman,
  enumerate labeli punctuation=,
  enumerate label left=(,
  enumerate label right=),
  enumerate leftmargini=2.2em,
  enumerate leftmarginii=2.1em,
  enumerate leftmarginiii=1.5em,
  enumerate leftmarginiv=2em,
  listi topsep   = 8\p@ \@plus2\p@ \@minus\p@,
  listi itemsep = 0\p@ \@plus2\p@ \@minus\p@,
  listi parsep   = 0\p@ \@plus2\p@ \@minus\p@,
  listii topsep  = 0\p@ \@plus2\p@ \@minus\p@,
  listii itemsep= 0\p@ \@plus2\p@ \@minus\p@,
  listii parsep  = 0\p@ \@plus2\p@ \@minus\p@,
  listiii topsep = 0\p@ \@plus2\p@ \@minus\p@,
  listiii itemsep= 0\p@ \@plus2\p@ \@minus\p@,
  listiii parsep  = 0\p@ \@plus2\p@ \@minus\p@,
}}

\ExplSyntaxOn
\def\setenumerate#1{
\cxset{#1}
\def\@listi{%
           \leftmargin\leftmargini
            \parsep\listiparsep@cx
            \topsep\listitopsep@cx\relax
            \itemsep\listiitemsep@cx}
            
\def\@listii{\leftmargin\leftmarginii
            \parsep\listiiparsep@cx
            \topsep\listiitopsep@cx\relax
            \itemsep\listiiitemsep@cx}
            
\def\@listiii{\leftmargin\leftmarginiii
            \parsep\listiiiparsep@cx
            \topsep\listiiitopsep@cx\relax
            \itemsep\listiiiitemsep@cx}
}


\setenumerate{compact1}
\ExplSyntaxOff
\makeatother
\MakePercentComment


\chapter[Template 87]{More on Boxes: Using packages to automate boxing calculations and the drawing of borders.}
\thispagestyle{plain}
\pagestyle{headings}
\large


The boxing of contents is such an important concept in \tex and also in typography that it is worth examining some of the available packages that can be used.

The most elaborate package is Martin Scharrer’s package \pkgname{adjustbox}. The package uses numerous keys
to draw borders, adjust spacing and margings but also for the clipping of images. At the background the package uses and extends the \pkgname{graphicx} key value system.

This package allows to adjust general \latexe material in several w+ays using a key=value interface.
It got inspired by the interface of \cmd{\includegraphics} from the \pkg{graphicx} package.
 This package also loads the \pkg{trimclip} package which code was once included in this package.


\adjustbox{frame}{This is some text}

 \subsection{Trimming and Clipping}
 
 Trimming and clipping is achieved by loading the \pkgname{trimclip}. This package forms part of the suit of packages developed by Martin Scharrer and or related to adjusting boxes and their sizes. The package allows for
 verbatim material as well. 
 
 \let\Macro\cmd
 
 The following keys allow content to be trimmed (i.e.\ the official size is made smaller, so the remaining material
 laps over the official boundaries) or clipped (overlapping material is not displayed).
 These keys come in different variants, where the lower-case keys represent the behavior of
 the corresponding \Macro\includegraphics keys. The corresponding macros (\Macro\trimbox, \Macro\clipbox, etc.)
 and environments (\env{trimbox}, \env{clipbox}, etc.) are included in the
 accompanying \pkg{trimclip} package and are explained in its manual.

 
 This key represents the original \option{trim} key of \Macro\includegraphics but accepts its value in different forms.
 Unlike most other keys it always acts on the original content independent in which order it is used with other keys.
 The key trims the given amounts from the lower left (ll) and the upper right (ur) corner of the box. This means that
 the amount \meta{llx} is trimmed from the left side, \meta{lly} from the bottom and \meta{urx} and \meta{ury} from the
 right and top of the box, respectively.
 If only one value is given it will be used for all four sites.
 If only two values are given the first one will be used for the left and right side (llx, urx) and the second for the
 bottom and top side (lly, ury).
 
\begin{texexample}{Example}{clipping}
\adjustbox{Clip=1, min width=8cm, center,}{This is some test}
\medskip

The untrimmed version is shown below
\medskip

\adjustbox{Clip=.1, min width=8cm, center,}{This is some test}%
\end{texexample}
 
 
 


\cxset{section align=left}
\cxset{section font-weight=bold}
\cxset{section font-family=sffamily}
\cxset{subsection beforeskip=10pt} 
\cxset{subsection afterskip=10pt,
       subsection font-weight=bfseries,
       subsection font-family=tiresias,
       subsection font-size=Large,
       subsection font-shape=upshape,
       subsection align=flushleft,%don't say left
       subsection color=spot!50,
       subsection indent=0pt,
       %subsection afterindent=false,
       subsection numbering prefix=\thesection.,
       subsubsection indent=0pt,
       section font-family=tiresias,
       subsection font-family=tiresias,
       subsubsection font-family=tiresias,
       subsubsection indent=0pt,
       subsubsection font-size=large,
       chapter spaceout=soul,
       }
%  \larger
% \phddoc
%  \textdocs
%%
%  \fontsandsymbols
%%
%%  
% \mathdocs 
% \latexiiidocs
%%   
% \bibandindex 
%% 
% \programming

 \luadocs 
% 
% \end{document}
%%
% \graphicsdocs
% \docboxing
% \visualizations
%%
% \languages 
%% % programming source2e
% \kernel 
% \bibliography{phd} 
%% %
 \PrintIndex

%\debugchapter
 \end{document}
 %
% \chapter{THE OUTPUT ROUTINE (OTR)}

\epigraph{Sherlock Holmes in "The sign of four": "'My mind," he said, "rebels at stagnation. Give me problems, give me work, give me the most abstruse cryptogram or the most intricate analysis, and I am in my own proper atmosphere.'" }{}
\normalsize
The output routine is one of the more mysterious pieces
of \tex.
and as  David Salomon noted\footnote{TUGboat/tb-11-1/tb27salomon.pdf}, advanced users hardly need to be convinced that an unerstanding of OTRs is important, since they must be used whenever, special output is desired.
 The chapter of the \texbook discussing output
routines claims that designing output routines makes one:

\begin{quotation}
achieve the level of a `\tex Grandmaster'.
As is so often the case, mastery of the concept of an
output routine in plain TEX will only barely prepare you
for the complexities awaiting you with LATEX’s variant of
an output routine.
\end{quotation}


The subject is considered complex for the following reasons:

\begin{enumerate}
\item OTRS are asynchronous with the
rest of TEX (this is explained later) and involve difficult concepts such as splitting boxes and insertions.
\item Certain features, which could be useful in OTRs are not supported by \tex. Specifically there are no commands to identify marks, rules and |whatsits| in a box and to break up a line of text into individual characters.
\end{enumerate}

\tex\ 's page breaking algorithm is simpler than the line breaking one. The reason for this is that global optimization
of page breakpoints, the way is done in the paragraph algorithm is prohibitively in terms of memory (especially in the 1980s).

Theoretically, page breaking is a more complicated \footnote{\href{test}{http://www.cs.utk.edu/~eijkhout/594-LaTeX/handouts/breaking/page-tutorial.pdf}}than line breaking. First we will briefly discuss the algoithms that \tex\ actually
uses.


\section{Page breaking algorithm}

The problem of page breaking has two components. One is that of stretching or shrinking
available glue (mostly around display math or section headings) to find typographically
desirable breakpoints. The other is that of placing ‘floating’ material, such as tables and
figures. These are typically placed at the top or the bottom of a page, on or after the first
page where they are referenced. These ‘inserts’, as they are called in TEX, considerably
complicate the page breaking algorithms, as well as the theory.

\subsection{Typographical constraints}

There are various typographical guidelines for what a page should look like, and TEX has
mechanisms that can encourage, if not always enforce, this behaviour.

\begin{enumerate}
\item The first line of every page should be at the same distance from the top. This changes
if the page starts with a section heading which is a larger type size.

\item The last line should also be at the same distance, this time from the bottom. This
is easy to satisfy if all pages only contain text, but it becomes harder if there are
figures, headings, and display math on the page. In that case, a ‘ragged bottom’ can
be specified.

\item  A page may absolutely not be broken between a section heading and the subsequent
paragraph or subsection heading.

\item It is desirable that

\begin{enumerate}
\item the top of the page does not have the last line of a paragraph started on the
preceding page

\item the bottom of the page does not have the first line of a paragraph that continues
on the next page.
\end{enumerate}

\end{enumerate}



For ordinary purposes you will probably find that \tex's automatic
method of page breaking is satisfactory. And when it occasionally gives unpleasant
results, you can force the machine to break at your favorite place by
typing |\eject|. But be careful: |eject| will cause \tex to stretch the page
out, if necessary, so that the top and bottom baselines agree with those on other
pages.  If you want to eject a short page, filling it with blank space at the bottom,
type | \vfill\eject|  instead.

\section{The current page and the recent contributions list}

The main vertical list of TEX is divided in two parts: the \emph{current page} and the list of \emph{recent
contributions}. Any material that is added to the main vertical list is appended to the recent
contributions; the act of moving the recent contributions to the current page is known as
\emph{exercising the page builder}.

Every time something is moved to the current page, TEX computes the cost of breaking the
page at that point. If it decides that it is past the optimal point, the current page up to the
best break so far is put in |box255| and the remainder of the current page is moved back
on top of the recent contributions. If the page is broken at a penalty, that value is recorded
in |outputpenalty|, and a penalty of size 10 000 is placed on top of the recent contributions;
otherwise, |outputpenalty| is set to 10 000.

If the current page is empty, discardable items that are moved from the recent contributions
are discarded. This is the mechanism that lets glue disappear after a page break and at the
top of the first page. When the first non-discardable item is moved to the current page, the
|topskip| glue is inserted; 



\section{When is the page builder activated?}


The page builder comes into play in the following circumstances.

\begin{enumerate}
\item  Around paragraphs: after the \cs{everypar} tokens have been inserted, and after the
paragraph has been added to the vertical list. See the end of this chapter for an
example.

\item  Around display formulas: after the \cs{everydisplay} tokens have been inserted, and after
the display has been added to the list.

\item  After \cs{par} commands, boxes, insertions, and explicit penalties in vertical mode.

\item  After an output routine has ended.
\end{enumerate}



In these places the page builder moves the recent contributions to the current page. Note that
\tex\  need not be in vertical mode when the page builder is exercised. In horizontal mode,
activating the page builder serves to move preceding vertical glue (for example, \cs{parskip},
\cs{abovedisplayskip}) to the page.

The \cs{end} command – which is only allowed in external vertical mode – terminates a TEX job,
but only if the main vertical list is empty and \cs{deadcycles} = 0. If this is not the case the
combination


|\hbox{}\vfill\penalty+ $-2^{30}$|

is appended, which forces the output routine to act.

\section{The depth of the current page}
The depth of the page is important since normally in good typesetting successive pages should have the same (or almost the same vertical size. (flushbottom). The height of a page is controlled and set exactly by \tex equal to |\vsize|. Consider a large |vbox| with lines of text, glue and penalties. The depth of this box, is the depth of the last component [80]. If the last component is a glue or penalty, the depth is zero. If it is a box, then its depth becomes the depth of the entire |\vbox|, except that it is limited to the value of parameter |\boxmaxdepth|.

If
|\boxmaxdepth=1pt| and the depth of the bottom box
is 1.94444pt, then the depth of the entire |\vbox|
will be 1pt and its height will be incremented
by .94444pt. This is equivalent to lowering the
reference point (or, equivalently, the baseline) of
the |\vbox| by .94444pt. In the plain format,
|\boxmaxdepth=\maxdimen| [348], so it has no effect
on the depths of boxes. However, |\boxmaxdepth|
can always be changed by the user \footnote{This \texttt{\textbackslash boxmaxdepth} setting is to ensure that deep footnotes do not overwrite the
footer (on account of the negative skip added later): it should use \texttt{\textbackslash @maxdepth}
otherwise the change is pointless when there are footnotes.
But see also its use when combining 
floats.  \latex uses a value of 5.5pt whereas plain a value of 4pt [348].}



If the last line on a page, contains letters that happen to not have any depth, the page depth will be zero. Try for example this:

\begin{teXXX}
....
\showthe\pagedepth
\bye
\end{teXXX}

You can also try it with a \latex minimal and will produce the same output.


\section{The height of a box of text}

Following the literature we denote the value of |\baselineskip| (which is normally 12pt) by $b$. 
A
large |\vbox| with text consists mainly of lines of
text, each an |\hbox|, separated by globs of glue,
normally in the (varying) amounts necessary to
separate baselines by exactly $b$, but sometimes just
the amount |\lineskip|. We assume a simple case
where no large characters or equations are used. In
such a case, all lines of text are separated by $b$. The
height of the box is thus:
\begin{gather}
b(n - 1) + \text{the height of the first line}
\end{gather}
where $n$ is the number of text lines. Remember that the first line is a special case and adjustments can be made using the value of |\topskip|.

\section{The height of \texttt{\textbackslash box255}}

In the case of |\box255|,
enough glue is placed above the first line of text
to reach to |\topskip| from the first baseline. We
denote the value of |\topskip| by $h$ (10pt in plain).
So if the baseline of the first line is now h below the
top of the page, the height H of |\box255| should
be b(n - 1) + h (Fig. 3). However, the height of
|\box255| is always set, by the page builder, to
|\vsize|. The difference between the two heights is
usually supplied by the flexible glues on the page,
the most common of which is |\parskip|

\begin{figure}[htp]
\includegraphics{./graphics/heightofpagebox}
\end{figure}


\subsection{Dead cycles.} An execution of the OTR without shipping any material is called a \texttt{dead cycle}. Dead cycles, have their uses and we will explain this a bit later on. However, long iterations that just return \textit{dead cycles} is an indication of an error somewhere. \tex counts the number of dead cycles in a counter named |\deadcycles| and stops the run if |\deadcycles >= \maxdeadcycles|.  In the \textit{plain} format |\maxdeadcycles| is set as 25 and in \latex as \the\deadcycles. |\maxdeadcycles = 100| is \the\maxdeadcycles. Each time |\shipout| is invoked, it resets |\deadcycles| to zero.

\begin{teXXX}
If the file is not included, reset \deadcycles, so that a long list of non-included
files does not generate an `Output loop' error.
115 \deadcycles\z@
116 \@nameuse{cp@#1}%
117 \fi
118 \let\@auxout\@mainaux}
\end{teXXX}


\subsection{\tex's Page Number.} The page number can come from any source. Salomon provides an example where the \textsc{OTR} typesets a page number from a |\count| variable. This is typeset centered below the printed area.

\begin{teXXX}
\newcount\pageNum
\output={
\shipout\vbox{
\box255\smallskip
\centerline{\tenrm\the\pageNum}}
\global\advance\pageNum by1}
\end{teXXX}

Notice that the output macro, just passes the contents of the box to |\shipout|. This is not actually a very good method, but is shown here to illustrate a point.

Note the |\tenrm| in the preceding example. It
is necessary because of the asynchronous nature of
the \otr. When the \otr is invoked, \tex can be
anywhere on the next page. Specifically, it could
be inside a group where a different font is used.
Without the |\tenrm|, that font (the current font)
would be used in the otr.
In the plain format, the |\count0| variable
serves as the page number, and the following two
macros are especially useful.




\subsection{The \texttt{\textbackslash vsplit} operation.} 

Supposed you have inserted the material required to go on a page on a big |\vbox|, but the material is a bit extra that what is required to fill a page exactly. You would need an operation to split the box in two. The |vsplit| operation does that. It is important to the understanding of OTR operations to have an intimate knowledge of |\vsplit|. Its syntax is: 

|\vsplit|\meta{box number} to \meta{dim}

The result of the operation is a box. Most often it appears in an assignment such as: |\setbox1=\vsplit0 to2.6in|. This sets |\box1| to a
height of 2.6in, moves material from the top of
|\box0| to |\box1|, and keeps the remainder in |\box0|.

\begin{macro}{\loremlines}
It is important to remember that most of \tex's commands work with \latex as well. In Example~\ref{ex:loremlines}, we define a box to hold |lipsum| text in a two column layout. We want to define a macro that can split the box in as many lines as we require. 
\end{macro}

\begin{texexample}{Splitting a vbox}{ex:loremlines}
\newbox\one
\newbox\two
\long\gdef\loremlines#1#2{%
   \setbox\one=\vbox {#2}
   \setbox\two=\vsplit\one to #1\baselineskip
   \unvbox\two
   \gdef\boxone{#2}
}
\begin{multicols}{2}
\small
\loremlines{16}{\lipsum[1-2]}
\end{multicols}
\boxone
\end{texexample}


\tex assumes that the new |\box1| may have to
be shipped out as part of the page. It therefore
places a glue similar to $h$ at the top of |\box1|.
This glue is called |\splittopskip| and has a plain
format value of 10pt [348].

One important thing to note is that a box can only be split \textit{between} lines of text. 
If we split a box to another size, |\box1| will come out underfull.

Here is an \otr which splits the page, ships
out the top part and returns the rest to the MVL
(actually, to the recent contributions):

\begin{teXXX}
\output={\setbox0=\vsplit255 to1in
\shipout\box0 \unvbox255}
\end{teXXX}






\section{Communicating with the OTR: Marks}

\begin{multicols}{2}
The user can pass information to the output routine through \textit{marks}. Marks have the syntax

\begin{teX}
\mark{mark text}
\end{teX}

which is put in a mark item on the current vertical list. The mark text is subject to expansion
as in \cs{edef}.
If the mark is given in horizontal mode it migrates to the surrounding vertical lists like an
insertion item (see page Text By Topic 77); however, if this is not the external vertical list, the output routine
will not find the mark.

Marks are the main mechanism through which the output routine can obtain information
about the contents of the currently broken-off page, in particular its top and bottom. TEX sets
three variables:

{\obeylines
\cs{botmark} the last mark occurring on the current page;
\cs{firstmark} the first mark occurring on the current page;
\cs{topmark} the last mark of the previous page, that is, the value of \cs{botmark} on the previous
page.
}



If no marks have occurred yet, all three are empty; if no marks occured on the current pagr, all three variables are equal to the \cs{botmark} of the previous page. 

Marks can be used to get a section heading into the headline or footline of the page.

\begin{verbatim}
\def\section#1{ ... \mark{#1} ... }
\def\rightheadline{\hbox to \hsize
    {\headlinefont \botmark\hfil\pagenumber}}
\def\leftheadline{\hbox to \hsize
   {\headlinefont \pagenumber\hfil\firstmark}}
\end{verbatim}

This places the title of the first section that starts on a left page in the left
headline, and the title of the last section that starts on the right page in
the right headline. Placing the headlines on the page is the job of the output
routine; see below.

It is important that no page breaks can occur in between the mark and the
box that places the title:

\emphasis{mark,nobreak}
\begin{teXXX}
\def\section#1{ ...
   \penalty\beforesectionpenalty
   \mark{#1}
   \hbox{ ... #1 ...}
   \nobreak
   \vskip\aftersectionskip
   \noindent}
\end{teXXX}
\end{multicols}



\section{Insertions}
Insertions are considered one of  the most  com- 
plex  topics in \tex. Many users master  topics  such 
as tokens,  file  I/O, macros,  and  even  OTRS  before 
they dare  tackle  insertions.  The  reason  is  that 
insertions  are  complex,  and  The \texbook, while 
covering all the relevant material, is somewhat cryp- 
tic regarding  insertions, and  lacks  simple examples. 
The  main  discussion  of  insertions takes  place  on 
[115-1251.  where \tex' s  registers  are also discussed. 
Examples  of  insertions are  shown, mostly  without 
explanations,  on  [363-364,  423-424].  A lot of what is described here is based on an article in TUGboat by David Salomon\footnote{ 
http://www.tug.org/TUGboat/Articles/tb11-4/tb30salomon.pdf}

Many users understand the idea of floats. Certain material to be typeset needs to be held in a buffer and inserted at different points on a page, for example a a figure that does not fit on a page it has to be inserted at the top of the next page. An \textit{insertion} is just a piece of a document that is generated at a certain point but appears at another point. Common examples are figures, footnotes and endnotes. Quoting Knuth:

\begin{quote}
  This  algorithm  is  admittedly  complicated, 
but  no  simpler  mechanism  seems  to  do  nearly 
as  much.
\end{quote}

\section{OTR Example}

\begin{figure}%
 \centering
  \includegraphics[width=0.37\linewidth]{./graphics/framedpage}
  \caption{A boxed page}
  \label{fig:framedpage}
\end{figure}

Here is an OTR for a \textit{framed} page. It surrounds the
page with double rules on all sides, and centers the
page number below the double box. Note that the
page shipped out is wider and taller than \cs{box255}.
The value of \cs{hsize} in this case is, therefore, not
the width of the final page shipped out, but the
width of the text lines in \cs{box255}.

Macro \cs{frameit} typesets text and surrounds it
with 4 rules (see [Ex. 21.3]). Parameter \#2 is the
space between the rules and the text. \#1 is a box
containing the text.

\emphasis{output,shipout}
\begin{teXXX}
\def\frameit#1#2{%
 \vbox{\hrule
  \hbox{%
    \vrule \kern#2pt
      \vbox{\kern#2pt #1
         \kern#2pt}%
      \kern#2pt\vrule}
\hrule}}

\output={
   \shipout\vbox{
   \boxit{\frameit{\box255}9}
      \medskip
      \centerline{Test Framed Page}}
  \advancepageno}
\end{teXXX}


Plain TeX has an output routine that takes care of  simple things like page numbering and insertions
using \cs{footnote} and \cs{topinsert}. 

\section{\LaTeX\  output routines}

So far we have examined the \tex OTR in detail. I hope it has given you enough understanding, not only to write your own output routine, but also to now be ready to study the \latex output routine, which is much more complicated. We have so far seen that  when \tex 
is typesetting pages of continuous text, it will gather material until it can find a least-cost page break intended to
make the gathered material fit the \cs{pagegoal size}. The
gathered material will then be placed into |\box255| and
the output routine stored in the token register \cs{output}
will be processed in a group of its own. 

Usually it will
arrange the gathered material in some way, add headers,
footlines and page numbers, and ship the gathered results out in typeset form with the \cs{shipout} command.
At the time of the \cs{shipout} command all \cs{open} and
\cs{write} commands stored in the box shipped out are expanded and written out. This is what makes it possible to have page labels corresponding to the actual page
numbers at the time of shipout: the corresponding info
is written to the |.aux| file at that time.
The output routine may decide to place material
back on the main vertical list instead of shipping it out.

\LaTeX\ output routine is described in \texttt{ltoutput.dtx}. You should also have a look at \texttt{ltfloat.dtx}. The algorithm is revisited i \latex3 and Frank Mittelbach, published a paper
\footnote{\protect\url{http://www.latex-project.org/papers/xo-pfloat.pdf}} in which he explains some of the problems facing the team, when dealing with the output routine.


Information on the output routine is rather scarce. Best source is a series of  articles in the TUGBoat by David Salomon.

\href{http://www.tug.org/TUGboat/Articles/tb11-1/tb27salomon.pdf}{Output Routines: Examples and Techniques. Part I: Introduction and Examples.}

\href{http://www.tug.org/TUGboat/Articles/tb11-2/tb28salomon.pdf}{Output Routines: Examples and Techniques. Part II: OTR Techniques}

\href{http://www.tug.org/TUGboat/Articles/tb11-4/tb30salomon.pdf}{Output Routines: Examples and Techniques. 
Part III: Insertions}

\href{http://www.tug.org/TUGboat/Articles/tb15-1/tb42salomon-output.pdf}{Output routines: Examples and techniques Part IV: Horizontal techniques}


David Kastrup's article \href{http://www.tug.org/TUGboat/Articles/tb24-3/kastrup.pdf}{Output Routine Requirements for Advanced Typesetting Tasks} (Proceedings of EuroTEX 2003) otlined some of the difficult areas and specifications for generic routines

The standard blocks are well described above and most tasks could be accomplished 
by rather working from
standard building blocks like \textit{insertion lists}, \textit{here points},
default mechanisms for \textit{margin notes} and so on.


\section*{Calling the output routine}

The output routine is called either by TeX's normal page-breaking
mechanism, or by a macro putting a penalty < or = -10000 in the output
list. In the latter case, the penalty indicates why the output
routine was called, using the following code.
penalty reason

\begin{tabular}{ll}
\toprule
penalty &reason\\
\midrule
-10000  &\ pagebreak\\
~       &\ newpage\\
-10001  &clearpage (\ penalty -10000 \ vbox{}| \ penalty -10001)|\\
-10002  &float insertion, called from horizontal mode\\
-10003 &float insertion, called from vertical mode.\\
-10004 &float insertion.\\
\bottomrule
\end{tabular}
\medskip

Note: A |float| or |marginpar| puts the following sequence in the output
list: 

\begin{enumerate}
\item a penalty of -10004,

\item a null |\vbox|

\item a penalty of -10002 or -10003.
\end{enumerate}

This solves two special problems:

\begin{enumerate}
\item If the float comes right after a |\newpage| or |\clearpage|,
then the first penalty is ignored, but the second one
invokes the output routine.

\item If there is a split footnote on the page, the second 'page'
puts out the rest of the footnote
\end{enumerate}

\latex first defines some helper routines and increase the \cs{maxdeadcycles}. The helper macros are for
manipulating lisst.

\begin{teX}
 \maxdeadcycles = 100
 \let\@elt\relax
 \def\@next#1#2#3#4{\ifx#2\@empty #4\else
   \expandafter\@xnext #2\@@#1#2#3\fi}
   \@next \CS \LIST {NONEMPTY}{EMPTY} == %% NOTE: ASSUME
\@elt = \relax
 BEGIN assume that \LIST == \@elt \B1 ... \@elt \Bn
 if n = 0
 then EMPTY
 else 
   \CS :=L \B1
   \LIST :=G \@elt \B2 ... \@elt \Bn
   NONEMPTY
 fi
END
\end{teX}


\begin{teX}
11 \def\@xnext \@elt #1#2\@@#3#4{\def#3{#1}\gdef#4{#2}}

12 \def\@testfalse{\global\let\if@test\iffalse}
13 \def\@testtrue {\global\let\if@test\iftrue}
14 \@testfalse}
   }

15 \def\@bitor#1#2{\@testfalse {\let\@elt\@xbitor
16   \@tempcnta #1\relax #2}}

17 \def\@xbitor #1{\@tempcntb \count#1
18    \ifnum \@tempcnta =\z@
19    \else
20      \divide\@tempcntb\@tempcnta
21    \ifodd\@tempcntb \@testtrue\fi
22   \fi}
\end{teX}

\begin{multicols}{2}
\subsection{Float boxes and lists.} 
A \textit{float list} consisting of the 
floats in boxes |\boxa ... \boxN| has
the form:

|\@elt \boxa ... \@elt \boxN|
where |\boxI| is defined by

|\newinsert\boxI|

Normally, |\@elt| is |\let| to |\relax|. A test can be performed on the
entire 
oat list by locally |\def|'ing |\@elt| appropriately and
executing the list.
This is a lot more efficient than looping through the list.
\LaTeX\ defines float boxes as |bx@A| to |bx@R| to make them available for 
inserts. These will be used later to define the lists that hold these boxes. 


\latex now defines the float boxes. Each one is defined as an insert.
\begin{teXXX}
\newinsert\bx@A
...
\newinsert\bx@I
\newinsert\bx@J
\newinsert\bx@K
\newinsert\bx@L
\newinsert\bx@M
\newinsert\bx@N
\newinsert\bx@O
\newinsert\bx@P
\newinsert\bx@Q
\newinsert\bx@R
\end{teXXX}


\end{multicols}
Once these boxes are defined they are inserted in the |@freelist|. At this point all the other lists are defined.

\emphasis{@freelist,@toplist,@botlist,@midlist,@currlist}
\begin{teXXX}
41 \gdef\@freelist{\@elt\bx@A\@elt\bx@B\@elt\bx@C\@elt\bx@D\@elt\bx@E
42                 \@elt\bx@F\@elt\bx@G\@elt\bx@H\@elt\bx@I\@elt\bx@J
43                 \@elt\bx@K\@elt\bx@L\@elt\bx@M\@elt\bx@N
44                 \@elt\bx@O\@elt\bx@P\@elt\bx@Q\@elt\bx@R}
\end{teXXX}

All the lists are defined initially to be empty.
\begin{teXXX}
45 \gdef\@toplist{}
46 \gdef\@botlist{}
47 \gdef\@midlist{}
48 \gdef\@currlist{}
49 \gdef\@deferlist{}
50 \gdef\@dbltoplist{}
51 \gdef\@dbldeferlist{}
\end{teXXX}


The lists are similar to those defined in \texttt{plain}.

\begin{description}
\item[\string\@freelist] : List of empty boxes for placing new 
floats.
\item[\string\@toplist] : List of 
floats to go at top of current column.
\item[\string\@midlist] : List of 
floats in middle of current column.
\item[\string\@botlist] : List of 
floats to go at bottom of current column.
\item[\string\@deferlist] : List of 
floats to go after current column.
\item[\string\@dbltoplist] : List of double-col. 
floats to go at top of current
page.
\item[\string\@dbldeferlist] : List of double-column 
floats to go on subsequent
pages.

\end{description}

\begin{multicols}{2}
Check was prudent when defining the newinsert boxes in order to reserve space and memory. The package \docpkg{morefloats} can be used to add more floats to this list. This should have definitely been included here in a revision.

\subsection{Defining Layout parameters} All the page layout parameters are defined next. 

\begin{teXXX}
52 \newdimen\topmargin
53 \newdimen\oddsidemargin
54 \newdimen\evensidemargin
55 \let\@themargin=\oddsidemargin
56 \newdimen\headheight
57 \newdimen\headsep
58 \newdimen\footskip
59 \newdimen\textheight
60 \newdimen\textwidth
61 \newdimen\columnwidth
62 \newdimen\columnsep
63 \newdimen\columnseprule
64 \newdimen\marginparwidth
65 \newdimen\marginparsep
66 \newdimen\marginparpush
\end{teXXX}

Remember  that TeX knows littel about a page. The problem is that TEX has no idea how
wide and tall the paper is. All it knows is the
left and top offsets, and the dimensions of the
printed area (|\hsize| and |\vsize|). All these dimensions need to be calculated and adjustments made within the \otr.

A document normally  starts by specifying:

\begin{teXXX}
\newdimen\paperheight
\newdimen\paperwidth
\paperheight=..in \paperwidth=..in
\end{teXXX}


\end{multicols}


\subsection*{The AtBeginDvi}
A box register is used  to put stuff that must appear before anything else
in the |.dvi| file.

The stuff in the box should not add any typeset material to the page when it
is unboxed.

\emphasis{AtBeginDvi,@begindvibox}

\begin{teXXX}
67 \newbox\@begindvibox
68 \def \AtBeginDvi #1{%
69 \global \setbox \@begindvibox
70 \vbox{\unvbox \@begindvibox #1}%
71 }
\end{teXXX}

\begin{teXXX}
72 \newdimen\@maxdepth
73 \@maxdepth = \maxdepth
\end{teXXX}


Some new registers for paperheight and paperwidth are defined:

\begin{teXXX}
74 \newdimen\paperheight
75 \newdimen\paperwidth
76 \newif \if@insert
These should definitely be global:
77 \newif \if@fcolmade
78 \newif \if@specialpage \@specialpagefalse
These should be global but are not always set globally in other les.
79 \newif \if@firstcolumn \@firstcolumntrue
80 \newif \if@twocolumn \@twocolumnfalse
Not sure about these: two questions. Should things which must apply to a whole
doument be local or global (they probably should be `preamble only' commands)?
Are these three such things?
81 \newif \if@twoside \@twosidefalse
82 \newif \if@reversemargin \@reversemarginfalse
83 \newif \if@mparswitch \@mparswitchfalse
This counter has been imported from `multicol'.
84 \newcount \col@number
85 \col@number \@ne
\end{teXXX}

and a lot of other internal registers

\begin{teX}
86 \newcount\@topnum
87 \newdimen\@toproom
88 \newcount\@dbltopnum
89 \newdimen\@dbltoproom
90 \newcount\@botnum
91 \newdimen\@botroom
92 \newcount\@colnum
93 \newdimen\@textmin
94 \newdimen\@fpmin
95 \newdimen\@colht
96 \newdimen\@colroom
97 \newdimen\@pageht
98 \newdimen\@pagedp
99 \newdimen\@mparbottom \@mparbottom\z@
100 \newcount\@currtype
101 \newbox\@outputbox
102 \newbox\@leftcolumn
103 \newbox\@holdpg
104 \def\@thehead{\@oddhead} % initialization
105 \def\@thefoot{\@oddfoot}
\end{teX}


\subsection{\texttt{\textbackslash clearpage}}

The clearpage macro is a bit complicated, as it needs to avoid a complete empty page after a |\twocolumn[..]|. This prevents the text from the argument
vanishing into a  float box, never to be seen again. We hope that it does not
produce wrong formatting in other cases.

\begin{teX}
106 \def\clearpage{%
107   \ifvmode
108   \ifnum \@dbltopnum =\m@ne
109     \ifdim \pagetotal <\topskip
110       \hbox{}%
111     \fi
112   \fi
113  \fi
114 \newpage
115 \write\m@ne{}%
116 \vbox{}%
117 \penalty -\@Mi
118 }
\end{teXXX}

\subsection{The \texttt{\textbackslash clearpagedoublepage} macro} 

This checks for odd and even pages by using the
page counter |c@page|.  It also provides switches of twoside printing. 
\TODO{Why not from auxiliary?}

\begin{teXXX}
119 \def\cleardoublepage{\clearpage\if@twoside \ifodd\c@page\else
120 \hbox{}\newpage\if@twocolumn\hbox{}\newpage\fi\fi\fi}
\end{teXXX}

Note the |\newpage| is defined a bit further on. This is a fairly simple definition, since most of the code that follows only gets a bit complicated with the twocolumn option. It sets the dimensions and the booleans to those appropriate for the |onecolumn| option. An important note we back to \tex's |\hsize|. Both the linewidth as well as the columnwidth are set to this.

\begin{teXXX}
123 \def\onecolumn{%
124   \clearpage
125   \global\columnwidth\textwidth
126   \global\hsize\columnwidth
127   \global\linewidth\columnwidth
128   \global\@twocolumnfalse
129   \col@number \@ne
130   \@floatplacement
     }
\end{teXXX}

\subsection{\string newpage.} 

The |\newpage| macro is programmed defensively. The two checks at the beginning ensure that an item label or run-in section title
immediately before a |\newpage| get printed on the correct page, the one before
the page break.
All three tests are largely to make error processing more robust; that is why
they all reset the 
flags explicitly, even when it would appear that this would be
done by a |\leavevmode|.

\begin{teXXX}
131 \def \newpage {%
132  \if@noskipsec
133    \ifx \@nodocument\relax
134      \leavevmode
135      \global \@noskipsecfalse
136    \fi
137 \fi
138 \if@inlabel
139   \leavevmode
140   \global \@inlabelfalse
141 \fi
142 \if@nobreak \@nobreakfalse \everypar{}\fi
143 \par
144 \vfil
145 \penalty -\@M}
\end{teXXX}

An empty cols is defined. There is a note here, that an invisible rule might have been a better idea.

\begin{teXXX}
146 \def \@emptycol {\vbox{}\penalty -\@M}
\end{teXXX}

\subsection{The \string twocolumn macro.} This is the longest definition so far. We will leave it for a while and then come back. There are several bug fixes to the two-column stuff here. Firstly, like the onecolumn the page parameters are set to the correct parameters.


\begin{teXXX}
147 \def \twocolumn {%
148 \clearpage
149 \global\columnwidth\textwidth
150 \global\advance\columnwidth-\columnsep
151 \global\divide\columnwidth\tw@
152 \global\hsize\columnwidth
153 \global\linewidth\columnwidth
154 \global\@twocolumntrue
155 \global\@firstcolumntrue
156 \col@number \tw@
\end{teXXX}



\section*{The output macro}

The setting of the \cs{output} is quite short but it belies its complexity.
After having checked verious parameters it redirects to |@specialoutput|. This is the heart of the routines. Notice that \latex just fills in the token list of \tex's |output| routine, it does not attempt to redefine it or save it. 
Should some hooks be defined here, life might have been made easier, however, what one can do is to first save the \latex output routine and then redefine the output as one may wish. Return to it can happen after it. If you take this approach, you should be careful of packages that redefine output, such as |multicol| and |longtable|. An approach such as this is taken by |revtex|.

\emphasis{ifnum,fi,else,ifdimen,@specialoutput}
\begin{teX}
204 \output {%
205 \let \par \@@par
206 \ifnum \outputpenalty<-\@M
207    \@specialoutput
208 \else
209    \@makecol
210    \@opcol
211    \@startcolumn
212    \@whilesw \if@fcolmade \fi
213      {%
218      \@opcol\@startcolumn}%
219 \fi
220 \ifnum \outputpenalty>-\@Miv
221 \ifdim \@colroom<1.5\baselineskip
222 \ifdim \@colroom<\textheight
223 \@latex@warning@no@line {Text page \thepage\space
224 contains only floats}%
225 \@emptycol
226 % \if@twocolumn
227 % \if@firstcolumn
228 % \else
229 % \@emptycol
230 % \fi
231 % \fi
232 \else
  233 \global \vsize \@colroom
234 \fi
235 \else
236   \global \vsize \@colroom
237 \fi
238 \else
239   \global \vsize \maxdimen
240 \fi
241 }
\end{teX}



\begin{teXXX}
244 \gdef\@specialoutput{%
245   \ifnum \outputpenalty>-\@Mii
246     \@doclearpage
247   \else
248     \ifnum \outputpenalty<-\@Miii
249         \ifnum \outputpenalty<-\@MM \deadcycles \z@ \fi
250                 \global \setbox\@holdpg \vbox {\unvbox\@cclv}%
251         \else
252         \global \setbox\@holdpg \vbox{%
253                 \unvbox\@holdpg
254                 \unvbox\@cclv
We must now remove the box added by the 
oat mechanism and the \topskip
glue therefore added above it by TEX.
255                \setbox\@tempboxa \lastbox
256                \unskip
257 }%
These two are needed as separate dimensions only by \@addmarginpar; for other
purposes we put the whole size into \@pageht (see below).
258                \@pagedp \dp\@holdpg
259                \@pageht \ht\@holdpg
260                \unvbox \@holdpg

261                \@next\@currbox\@currlist{%
262                \ifnum \count\@currbox>\z@
Putting the whole size into \@pageht (see above).
263                  \advance \@pageht \@pagedp
264                  \ifvoid\footins \else
265                    \advance \@pageht \ht\footins
266                    \advance \@pageht \skip\footins
267                    \advance \@pageht \dp\footins
268                \fi
\end{teXXX}



\subsection{The \string @doclearpage macro.} This is an emergency action. It dumps everything: footnotes first and then floats. 


\section*{The Kludgeins}

The kludgeins are simply inserts that fool \tex in enlarging a page by a small amount, normally used to allow one or two lines of text to go in the same page.

The two kludgeins mentioned in the kernel are are \cs{enlargethisspace} and its star version.\footnote{The Oxford English Dictionary (2nd ed., 1989) kludge entry cites one source for this word's earliest recorded usage, definition, and etymology: Jackson W. Granholm's 1962 "How to Design a Kludge" article, which appeared in the American computer magazine Datamation
kludge  Also kluge. [J. W. Granholm's jocular invention: see first quot.; cf. also bodge v., fudge v.]

'An ill-assorted collection of poorly-matching parts, forming a distressing whole' (Granholm); esp. in Computing, a machine, system, or program that has been improvised or 'bodged' together; a hastily improvised and poorly thought-out solution to a fault or 'bug'.

The word 'kludge' is...derived from the same root as the German Kluge..., originally meaning 'smart' or 'witty'.... 'Kludge' eventually came to mean 'not so smart' or 'pretty ridiculous'.}



\begin{teXX}
\gdef \enlargethispage{%
1198 \@ifstar
1199 {%
1203   \@enlargepage{\hbox{\kern\p@}}}%
1204 {%
1208   \@enlargepage\@empty}%
1209 }
\end{teXX}

Adds |<dim>| to the height of the current column only. On the printed page the
bottom of this column is extended downwards by exactly |<dim>| without having
any effect on the placement of the footer; this may result in an overprinting.
\cs{enlargethispage}.

Similar to |\enlargethispage| but it tries to squeeze the column to be printed
in as small a space as possible, ie it uses any shrinkability in the column. If the
column was not explicitly broken (e.g. with |\pagebreak|) this may result in an
overfull box message but except for this it will come out as expected (if you know
what to expect).
The star form of this command is dedicated to Leslie Lamport, the other we
need for ourselves (FMi, CAR).
These commands may well have unwanted if used soon before a\ldots

 




\section{Using packages to ease the pain}

OTR routines are notoriously difficult to debug and define. Some of the available packages at CTAN
can make the programming job easier.

The |everypage| package by Sergio Callegari provides hooks into the \latex\ internal commands to
to do actions on every page or on the current page. Specifically, actions  are performed \emph{before} the page is shipped, so they can be
used to put watermarks \emph{in the background} of a page, or to
set the page layout. 

The package provides two hooks:

\emphasis{AddEverypageHook,AddThisPageHook}
\begin{teXXX}
  \AddEverypageHook{Test}
  \AddThisPageHook
\end{teXXX}

The package reminds in some sense
\docpkg{bobhook} by Karsten Tinnefeld, but it differs in the way in
 which the hooks are implemented, as detailed in the following.
 In some sense it may also be related to the package
 \docpkg{everyshi} by Martin Schroeder, but again the implementation
 is different.

 
 This program adds two \LaTeX\ hooks that get run when document
 pages are finalized and output to the |.dvi| or |.pdf|
 file. Specifically, one hook gets executed on every page, while the
 other is executed for the current page. Hook actions are are performed
 \emph{before} the page is output on the medium, and this is
 important to be able to play with the page layout or to put things
 \emph{behind} the page contents (e.g., watermarks such as an image,
 framing, the ``DRAFT'' word, and the like).
 
 The package reminds in some sense \Lpack{bobhook} by Karsten
 Tinnefeld, but it differs in the way in which the hooks are
 implemented:
 


 \begin{enumerate}
 \item there is no formatting inherent in the hooks. If one wants to
   put some watermark on a page, it is his own duty to put in the
   hook the code to place the watermark in the right position. Also
   note that the hooks code should \emph{eat up no space} in the
   page.  Again, if the hooks are meant to place some material on the
   page, it is the duty of the hook programmer to put code in the
   hooks to pretend that the material has zero width and zero height.
   The implementation is \emph{lighter} than the \Lpack{bobhook} one,
   and possibly more flexible, since one is not limited by any
   pre-coded formatting for the hooks. On the other hand it is
   possibly more difficult to use. Nonetheless, it is easy to think
   of other packages relying on \Lpack{everypage} to deliver more
   user-friendly and \emph{task specific} interfaces. Already there
   are a couple of them: the package \Lpack{flippdf} produces
   mirrored pages in a PDF document and \Lpack{draftwatermark}
   watermarks document pages.
 \item similarly to \Lpack{bobhook} and \Lpack{watermark}, the
   package relies on the manipolatoin of the internal \LaTeX\ macro
   |\@begindvi| to do the job. However, the redefinition of
   |\@begindvi| is here postponed as much as possible, striving to
   avoid interference with other packages using |\AtBeginDvi| or
   anyway manipulating |\@begindvi|. Specifically \Lpack{everypage}
   makes no special assumption on the initial code that |\@begindvi|
   might contain.
 \end{enumerate}



Also in some sense \Lpack{everypage} can be related to package
 \Lpack{everyshi} by Martin Schroeder, but it differs radically from
 it in the implementation. In fact,\Lpack{everypage} operates by
 manipulation of the |\@begindvi| macro, rather than at the
 lower level |shipout| macro.


\section{How to place a background image}

One can use TikZ to place a background image on a page

First we define some utility macros:


\begin{teXXX}
  \def\bg@contents{Draft}
  \def\bg@color{red!45}
  \def\bg@angle{60}
  \def\bg@opacity{.5}
  \def\bg@scale{15}
  \def\bg@position{current page.center}
  \def\bg@anchor{}
  \def\bg@hshift{0}
  \def\bg@vshift{0}
\end{teXXX}

A new command is then developed to describe the background material

\begin{teX}
\newcommand\bg@material{%
   \begin{tikzpicture}[remember picture,overlay]
   \node [rotate=\bg@angle,scale=\bg@scale,opacity=\bg@opacity,%
   xshift=\bg@hshift,yshift=\bg@vshift,color=\bg@color]
   at (\bg@position) [\bg@anchor] {\bg@contents};
  \end{tikzpicture}}%
\end{teX}

Once the background material has been defined we can place it on the page by simply calling

\begin{teXXX}
   \newcommand\BgThispage{\AddThispageHook{\bg@material}}
\end{teXXX}

The background package has capitalized on two good packages the TikZ and the everypage. Similarly you can use your own ingenuity to design whatever you want




\section{hooking at shipout}


This package provides the hooks \cs{EveryShipout} and 
  \cs{AtNextShipout} whose arguments are executed after the output 
  routine has constructed \cs{box255}, and before \cs{shipout} is 
  called.

  An example application for this package would be a package for
  adding text to the bottom of each page.
  Such a package does exist: \docpkg{prelim2e}\cite{package!prelim2e}.

The solution  uses is based on code developed in  \textsf{quire.tex} by
 Marcel R.~van der Goot.  It is based upon \cs{afterassignment} and \cs{aftergroup}.



 







































% 
%</driver>
% \fi
%  
%  \CheckSum{0}
%  \CharacterTable
%  {Upper-case    \A\B\C\D\E\F\G\H\I\J\K\L\M\N\O\P\Q\R\S\T\U\V\W\X\Y\Z
%   Lower-case    \a\b\c\d\e\f\g\h\i\j\k\l\m\n\o\p\q\r\s\t\u\v\w\x\y\z
%   Digits        \0\1\2\3\4\5\6\7\8\9
%   Exclamation   \!     Double quote  \"     Hash (number) \#
%   Dollar        \$     Percent       \%     Ampersand     \&
%   Acute accent  \'     Left paren    \(     Right paren   \)
%   Asterisk      \*     Plus          \+     Comma         \,
%   Minus         \-     Point         \.     Solidus       \/
%   Colon         \:     Semicolon     \;     Less than     \<
%   Equals        \=     Greater than  \>     Question mark \?
%   Commercial at \@     Left bracket  \[     Backslash     \\
%   Right bracket \]     Circumflex    \^     Underscore    \_
%   Grave accent  \`     Left brace    \{     Vertical bar  \|
%   Right brace   \}     Tilde         \~}
%
%
%
% \changes{1.0}{2013/01/26}{Converted to DTX file}
%
% \DoNotIndex{\newcommand,\newenvironment}
% \GetFileInfo{phd.dtx}
% 
%  \def\fileversion{v1.0}          
%  \def\filedate{2012/03/06}
% \title{The \textsf{phd} package.
% \thanks{This
%        file (\texttt{phd.dtx}) has version number \fileversion, last revised
%        \filedate.}
% }
% \author{Dr. Yiannis Lazarides \\ \url{yannislaz@gmail.com}}
% \date{\filedate}
%
%
% 
% ^^A\maketitle
% 
% ^^A\frontmatter
%  ^^A\coverpage{./images/hine02.jpg}{Book Design }{Camel Press}
%  ^^A\newpage
% ^^A\secondpage
% ^^A\pagestyle{empty}
%
%
% 
% \newif\ifmulticols
% \IfFileExists{multicol.sty}{\multicolstrue}{}
%
%
% {\parskip 0pt                ^^A We have to reset \parskip
%  
% 
%
% \gdef\dotsep{10000}          ^^A (bug in \LaTeX?)
% \makeatletter
%  \tableofcontents
%   ^^A\listoftables
%  ^^A\listoffigures
% }

% ^^A\pagenumbering{empty}
% \mainmatter
% \pagestyle{headings}
% \raggedbottom
%  
%  \def\partname{Part}
%  \let\sidenote\footnote
%  \let\oldinput\input
%  
%  ^^A\def\input#1{\MakePercentComment \oldinput{#1}\MakePercentIgnore}
%  \makeatletter \@debugfalse\makeatother
%  
% 
% ^^A\input{./sections/pgfmanual-en-pgfkeys}
% ^^A% !TeX root = tcolorbox.tex
% include file of tcolorbox.tex (manual of the LaTeX package tcolorbox)
\cxset{section align=left, chapter border-left-width=1pt, 
         chapter border-left-color=sweet,
         chapter border-top-style=solid,
         chapter border-bottom-style=solid,
         chapter border-left-style=solid,
         chapter border-right-style=solid,
         chapter border-style=solid,
         subsubsection font-shape=upshape}

\chapter{How to Design and Adjust Chapter Heading Parameters}

\section{Quick Reference}\label{sec:quickref}

Each element of a heading such as \textit{chapter} or \textit{title} is drawn as a box, with a large number of parameters that can adjust spacing, borders, fonts and other typographic parameters.

\bigskip\bigskip\bigskip\bigskip
\let\oldrefkey\refKey
\let\refKey\texttt
\makeatletter
\long\def\demobox#1#2{%
\par\bigskip\bigskip\bigskip
\begin{tcolorbox}[enhanced,left=0pt, top=0pt, bottom=0pt,width=\textwidth,
  enlarge top initially by=1cm,enlarge bottom finally by=1cm,left skip=1cm,right skip=1cm,
  colframe=white,colback=white,
  colbacktitle=red!30!white,colupper=black!7!white,
  code={\appto\kvtcb@shadow{%
    \path[fill=white,draw=yellow!50!black,dashed,line width=0.4pt]
      ([xshift=-1cm,yshift=-1cm]frame.south west) rectangle
      ([xshift=1cm,yshift=1cm]frame.north east);
     \path[fill=blue!20!white, 
              opacity=0.3, draw=yellow!50!black,solid,line width=1pt]
      ([xshift=-2cm,yshift=-2cm]frame.south west) rectangle
      ([xshift=2cm,yshift=2cm]frame.north east);  
    }},
  finish={
  \draw[thick,<->] ([yshift=-1.3cm]frame.north west)-- node[below]{\texttt{#1 width}}
    ([yshift=-1.3cm]frame.north east);
  \draw[thick,<->] ([xshift=-15mm]frame.north east)-- node[above]{\refKey{#1 height}}
    ([xshift=-15mm]frame.south east);
  \draw[thick,<->] (frame.north)-- node[right]{\refKey{#1 padding-top}} +(0,1);
  \draw[thick,<->] ([yshift=1cm]frame.north)-- node[right]{\refKey{#1 margin-top}} +(0,1);
  \draw[thick,<->] (frame.south)-- node[right, align=left]{\refKey{#1 padding-bottom}}+(0,-1);
  %left padding
  \draw[thick,<->] (frame.west)-- node[below right,align=center]{\refKey{#1 padding-left }}+(-1,0);
  %left margin
  \draw[thick,<->] ([xshift=-1cm,yshift=-0.9cm]frame.west)-- node[below right,xshift=-1,align=left]{\refKey{#1 margin-left }\\\refKey{#1 grow to left by}}+(-1,0);
  %right padding
  \draw[thick,<->] (frame.east)-- node[below left,align=center]{\refKey{#1 padding-right}}+(1,0);
 %right margin
  \draw[thick,<->] ([xshift=1cm,yshift=-0.9cm]frame.east)-- node[below left,xshift=1, align=right]{\refKey{#1 margin-right}\\\refKey{#1 grow to right by}}+(1,0);
 \draw[thick,<->] ([yshift=-2cm]frame.south)-- node[right, align=left]{\refKey{#1 margin-bottom},\\ \refKey{#1 after skip}}+(0,1);
  }
    ]
#2%
%\hrule width0pt height4.5cm depth0pt\relax% \vspace*{4.5cm}% \lipsum[1]
\end{tcolorbox}\par
\bigskip\bigskip\bigskip}
\makeatother

\demobox{chapter}{\scalebox{1.17}{\HHHUGE Chapter}}

The number box is again drawn in a box similar to a chapter with all properties generalized.

\demobox{number}{\scalebox{1.15}{\HHHUGE Thirteen}}



All parameters shown in the diagram can be set using the command \cs{cxset}. The property names follow conventions similar to those of |css|, rather than typical conventions of \tikzname that are more widely known to the programming community. The prefix to these properties (in the example \textit{chapter}) can be thought of
as similar to a |class| or |id| name in |css|.  

\begin{docCommand}{cxset}{\marg{options}}
  Sets options for every following \refEnv{tcolorbox} inside the current \TeX\ group.
  By default, this does not apply to nested boxes, see \Vref{subsec:everybox}.\par
  For example, the colors of the boxes may be defined for the whole document by this:
\begin{dispListing}
\tcbset{colback=red!5!white,colframe=red!75!black}
\end{dispListing}
\end{docCommand}

\begin{docKey}[]{chapter padding-top}{=\meta{dimension}}{no default, initial value 0pt}
All padding keys take one argument, which is a dimension. The length is also stored in a register
\cmd{\chapterpaddingtop}. In this chapter it was set at \the\chapterpaddingtop.
\end{docKey}

\begin{docKey}[]{chapter padding-right}{=\meta{dimension}}{no default, initial value 0pt}
All padding keys take one argument, which is a dimension. The length is also stored in a register
\cmd{\chapterpaddingright}.  In this chapter it was set at \the\chapterpaddingright.
\end{docKey}

\begin{docKey}[]{chapter padding-bottom}{=\meta{dimension}}{no default, initial value 0pt}
All padding keys take one argument, which is a dimension. The length is also stored in a register
\cmd{\chapterpaddingbottom}.  In this chapter it was set at \the\chapterpaddingbottom.
\end{docKey}

\begin{docKey}[]{chapter padding-left}{=\meta{dimension}}{no default, initial value 0pt}
All padding keys take one argument, which is a dimension. The length is also stored in a register
\cmd{\chapterpaddingleft}.  In this chapter it was set at \the\chapterpaddingleft.
\end{docKey}

%% margin

\begin{docKey}[]{chapter margin-top}{=\meta{dimension}}{no default, initial value 0pt}
All padding keys take one argument, which is a dimension. The length is also stored in a register
\cmd{\chaptermargintop}. In this chapter it was set at .
\end{docKey}

\begin{docKey}[]{chapter margin-right}{=\meta{dimension}}{no default, initial value 0pt}
All padding keys take one argument, which is a dimension. The length is also stored in a register
\cmd{\chapterpaddingright}.  In this chapter it was set at \the\chapterpaddingright.
\end{docKey}

\begin{docKey}[]{chapter margin-bottom}{=\meta{dimension}}{no default, initial value 0pt}
All padding keys take one argument, which is a dimension. The length is also stored in a register
\cmd{\chapterpaddingbottom}.  In this chapter it was set at \the\chapterpaddingbottom.
\end{docKey}

\begin{docKey}[]{chapter margin-left}{=\meta{dimension}}{no default, initial value 0pt}
All padding keys take one argument, which is a dimension. The length is also stored in a register
\cmd{\chaptermarginleft}.  In this chapter it was set at \the\chaptermarginleft.
\end{docKey}

\subsection{Borders}

Border have three properties \emph{width, color} and \emph{style}. They can set individually for
each side of the box or using the shorter key .

\begin{docKey}[]{chapter border-top-width}{=\meta{dimension}}{no default, initial value 0pt}
All border keys take one argument, which is a dimension.
\end{docKey}

\begin{docKey}[]{chapter border-right-width}{=\meta{dimension}}{no default, initial value 0pt}
All border keys take one argument, which is a dimension.
\end{docKey}

\begin{docKey}[]{chapter border-bottom-width}{=\meta{dimension}}{no default, initial value 0pt}
All border keys take one argument, which is a dimension.
\end{docKey}

\begin{docKey}[]{chapter border-left-width}{=\meta{dimension}}{no default, initial value 0pt}
All border keys take one argument, which is a dimension.
\end{docKey}

\subsubsection{Border Colors}

The colors follow the same pattern for |border-width| and again they can be set individually or using
a shorter key to set all of them in one color. 

\begin{docKey}[]{chapter border-top-color}{=\meta{color name}}{no default, initial value black}
All border keys take one argument, which is a dimension.
\end{docKey}

\begin{docKey}[]{chapter border-right-color}{=\meta{color name}}{no default, initial value black}
All border keys take one argument, which is a dimension.
\end{docKey}

\begin{docKey}[]{chapter border-bottom-color}{=\meta{color name}}{no default, initial value black}
All border keys take one argument, which is a dimension.
\end{docKey}

\begin{docKey}[]{chapter border-left-color}{=\meta{color name}}{no default, initial value black}
This key is stored in \cmd{\chapterborderrightcolor} and the value in this chapter is \texttt{\chapterborderrightcolor}.
\end{docKey}

\subsubsection{Border Styles}

Standard |css|  offers four styles \emph{dotted, solid, double, dashed}. We offer almost an unlimited set of styles.

\begin{docKey}[]{chapter border-top-style}{=\meta{style name}}{no default, initial value \texttt{none}}
The |border-style| properties take a value, which can be |solid, double, dotted, dashed, asterisk|.
\end{docKey}

\begin{docKey}[]{chapter border-right-style}{=\meta{style name}}{no default, initial value \texttt{none}}
The |border-style| properties take a value, which can be |solid, double, dotted, dashed, asterisk|.
\end{docKey}

\begin{docKey}[]{chapter border-bottom-style}{=\meta{style name}}{no default, initial value \texttt{none}}
The |border-style| properties take a value, which can be |solid, double, dotted, dashed, asterisk|.
\end{docKey}

\begin{docKey}[]{chapter border-left-style}{=\meta{style name}}{no default, initial value \texttt{none}}
The |border-style| properties take a value, which can be |solid, double, dotted, dashed, asterisk|.
\end{docKey}

\begin{docKey}[phd]{chapter border-style}{=\meta{style name}}{no default, initial value \texttt{none}}
This key sets all chapter-border-\meta{top,right,bottom,left}-style to a single value.
\end{docKey}
%\bigskip
%\bigskip
%
%\begin{tcolorbox}[enhanced,title={tcolorbox},before skip=5mm,after skip=5mm,
%  colframe=red!50!black!30!white,colback=red!10!white!40!white,
%  colbacktitle=red!30!white,coltext=black!20!white,
%  toptitle=1mm,bottomtitle=1mm,
%  overlay={\begin{tcbclipinterior}%
%    \path[fill=red!10!white!40!yellow!20!white,draw=yellow!50!black,dotted]n
%      ([xshift=1mm,yshift=1mm]interior.south west)
%      rectangle ([xshift=-1mm,yshift=-1mm]interior.north east);
%    \path[fill=red!10!white!40!white,draw=yellow!50!black,dotted] (
%      [xshift=5mm,yshift=3mm]interior.south west)
%      rectangle ([xshift=-5mm,yshift=-3mm]interior.north east);
%    \path[fill=red!10!white!40!yellow!20!white,draw=yellow!50!black,dotted]
%      ([xshift=5mm,yshift=-1mm]segmentation.south west)
%      rectangle ([xshift=-5mm,yshift=1mm]segmentation.north east);
%    \path[fill=red!10!white!40!white,draw=yellow!50!black,dotted]
%      ([xshift=5mm,yshift=1mm]segmentation.south west)
%      rectangle ([xshift=-5mm,yshift=-1mm]segmentation.north east);
%    \path[dashed,draw=red!50!black!30!white] (segmentation.west) -- (segmentation.east);
%    \end{tcbclipinterior}%
%    \begin{tcbcliptitle}
%    \path[fill=red!30!white!70!yellow,draw=yellow!50!black,dotted]
%      ([xshift=1mm,yshift=1mm]title.south west)
%      rectangle ([xshift=-1mm,yshift=-1mm]title.north east);
%    \path[fill=red!30!white,draw=yellow!50!black,dotted]
%      ([xshift=5mm,yshift=2mm]title.south west)
%      rectangle ([xshift=-5mm,yshift=-2mm]title.north east);
%    \end{tcbcliptitle}},
%  finish={
%  \coordinate (A) at ([yshift=-0.25mm]frame.north);
%  \draw[thick,<-] (A) -- +(-1,0.3) node[left]{\refKey{/tcb/toprule}};
%  \coordinate (A) at ([yshift=-0.75mm]A);
%  \draw[thick,<-] (A) -- +(1,0) node[right]{\refKey{/tcb/boxsep}};
%  \coordinate (A) at ([yshift=-1mm]A);
%  \draw[thick,<-] (A) -- +(-1,0) node[left]{\refKey{/tcb/toptitle}};
%  %
%  \coordinate (A) at ([yshift=1.00mm]interior.north);
%  \draw[thick,<-] (A) -- +(1,0) node[right]{\refKey{/tcb/boxsep}};
%  \coordinate (A) at ([yshift=1mm]A);
%  \draw[thick,<-] (A) -- +(-1,0) node[left]{\refKey{/tcb/bottomtitle}};
%  \coordinate (A) at ([yshift=0.25mm]interior.north);
%  \draw[thick,<-] (A) -- +(-1,-0.4) node[left]{\refKey{/tcb/titlerule}};
%  \coordinate (A) at ([yshift=-0.5mm]interior.north);
%  \draw[thick,<-] (A) -- +(1,-0.2) node[right]{\refKey{/tcb/boxsep}};
%  \coordinate (A) at ([yshift=-1.5mm]A);
%  \draw[thick,<-] (A) -- +(-1,-0.6) node[left]{\refKey{/tcb/top}};
%  %
%  \coordinate (A) at ([yshift=2.0mm]segmentation);
%  \draw[thick,<-] (A) -- +(-1,0) node[left]{\refKey{/tcb/middle}};
%  \coordinate (A) at ([yshift=0.5mm]segmentation);
%  \draw[thick,<-] (A) -- +(1,0.2) node[right]{\refKey{/tcb/boxsep}};
%  \coordinate (A) at ([yshift=-0.5mm]segmentation);
%  \draw[thick,<-] (A) -- +(1,-0.2) node[right]{\refKey{/tcb/boxsep}};
%  \coordinate (A) at ([yshift=-2.0mm]segmentation);
%  \draw[thick,<-] (A) -- +(-1,0) node[left]{\refKey{/tcb/middle}};
%  %
%  \coordinate (A) at ([yshift=0.25mm]frame.south);
%  \draw[thick,<-] (A) -- +(-1,-0.3) node[left]{\refKey{/tcb/bottomrule}};
%  \coordinate (A) at ([yshift=0.75mm]A);
%  \draw[thick,<-] (A) -- +(1,0) node[right]{\refKey{/tcb/boxsep}};
%  \coordinate (A) at ([yshift=1.5mm]A);
%  \draw[thick,<-] (A) -- +(-1,0) node[left]{\refKey{/tcb/bottom}};
%  %
%  \coordinate (A) at ([xshift=0.25mm]frame.west);
%  \draw[thick,<-] (A) -- +(-0.3,-1) node[below]{\refKey{/tcb/leftrule}};
%  \coordinate (A) at ([xshift=0.75mm]A);
%  \draw[thick,<-] (A) -- +(0,1) node[above]{\refKey{/tcb/boxsep}};
%  \coordinate (A) at ([xshift=2.5mm]A);
%  \draw[thick,<-] (A) -- +(0.7,0.5) node[above right]{\refKey{/tcb/left}};
%  %
%  \coordinate (A) at ([xshift=-0.25mm]frame.east);
%  \draw[thick,<-] (A) -- +(0.3,-1) node[below]{\refKey{/tcb/rightrule}};
%  \coordinate (A) at ([xshift=-0.75mm]A);
%  \draw[thick,<-] (A) -- +(0,1) node[above]{\refKey{/tcb/boxsep}};
%  \coordinate (A) at ([xshift=-2.5mm]A);
%  \draw[thick,<-] (A) -- +(-0.7,0.5) node[above left]{\refKey{/tcb/right}};
%  }
%    ]
%  \lipsum[1]
%  \tcblower
%  \lipsum[2]
%\end{tcolorbox}
%
%\let\refKey\oldrefkey
%
%\end{document}
% ^^A\makeatletter
\cxset{defaults/.style ={% 
    chapter title margin-top-width    =  0cm,
    chapter title margin-right-width  =  1cm,
    chapter title margin-bottom-width = 10pt,
    chapter title margin-left-width   = 0pt,
    chapter align                     = left,
    chapter title align               = left, %checked
    chapter name                      = CHAPTER,
    chapter format                    = block,
    chapter font-size                 = Huge,
    chapter font-weight               = bold,
    chapter font-family               = sffamily,
    chapter font-shape                = upshape,
    chapter background-color          = white,
  % chapter label    
    chapter color               = black,
    chapter number prefix             = ,
    chapter number suffix             = ,
    chapter numbering                 = arabic,
    chapter indent                    = 0pt,
    chapter beforeskip                = -3cm,
    chapter afterskip                 = 30pt,
    chapter afterindent               = off,
    chapter number after              = ,
    chapter arc                       = 0mm,
    chapter label background-color    = white,
    chapter label color               = black,
   % chapter afterindent               = on,
    chapter grow left                 = 0mm,
    chapter grow right                = 0mm,
    chapter rounded corners           = northeast,
    chapter shadow                    = fuzzy halo,
    chapter border-left-width         = 0pt,
    chapter border-right-width        = 0pt,
    chapter border-top-width          = 0pt,
    chapter border-bottom-width       = 0pt,
    chapter padding-left-width        = 0pt,
    chapter padding-right-width       = 10pt,
    chapter padding-top-width         = 10pt,
    chapter padding-bottom-width      = 10pt,
    %  
    chapter number color              = black,
    chapter number background-color   = white,
    chapter number font-size        = huge,
    chapter number font-weight      = bfseries,
    chapter number font-family      = sffamily,
    chapter number font-shape       = upshape,
    chapter number align            = Centering,
    %
    chapter title font-size        = Huge,
     chapter title font-weight      = bold,
     chapter title font-family      = sffamily,
     chapter title font-shape       = upshape,
     chapter title color            = black,
     chapter title background-color = white,
     }%
   }  
\makeatother     
%\makeatletter
%\cxset{toc image=\@empty,
%       chapter toc=true,
%       title beforeskip=1pt}
%
%\@specialfalse
%
%
%\renewcommand\stewart[2][]{%
%\fancypagestyle{fancy}{%
%\lhead{}\rhead{}
%\chead{}
%\cfoot{}
%\lfoot{}
%\rfoot{\thepage}
%\def\footrule#1{{\color{blue}%
%  \hrule width\paperwidth}\vskip3pt
%}
%
%\renewcommand{\headrulewidth}{0pt}
%\renewcommand{\footrulewidth}{0.4pt}}
%
%\clearpage
%
%\begin{tikzpicture}[remember picture,overlay]
%% Main shading block
%\node [xshift=5cm,yshift=-\paperheight] at (current page.north west)
%[text width=0.98\textwidth,text height=\paperheight, fill=thecream!30,rounded corners,above right]
%{};
%\node [xshift=6.5cm,yshift=-1.5cm-\soffsety] at (current page.north west)
%[text width=0.9\textwidth,below right]{\sffamily \bfseries \huge #2};
%
%\node [xshift=3cm,yshift=-1.5cm] at (current page.north west)
%[text width=3cm,align=center,minimum height=2.5cm, fill=blue,below right]
%{\[\text{\HHUGE\bfseries\sffamily\color{white}\thechapter}\]
%\par\vspace*{3pt}
%};
%
%\node [xshift=-0.2cm,yshift=-21.5cm] at (current page.north west)
%[text width=3cm,above right]%
%{\includegraphics[width=1.0\paperwidth]{\image@cx}};
%% second box left
%\node [xshift=3cm,yshift=-19.5cm] at (current page.north west)
%[text width=9cm,minimum height=2.5cm,inner sep=0.5em, fill=blue,below right]
%{\color{white}
%  \bfseries\sffamily \texti@cx
%};
%% Last block
%\node [xshift=6.5cm,yshift=-26cm] at (current page.north west)
%[text width=12cm,above right]
%{\textii@cx
%};
%\end{tikzpicture}
%\par
%\clearpage
%}





\cxset{steward,
  chapter numbering=arabic,
  chapter format = stewart,
  offsety=0cm,
  image= {./images/hine02.jpg},
  texti={When Lamport designed the original \LaTeX\ sectioning commands he did not provide a fully comprehensive interface for modifying their design. With current tools available improvements are much easier to program and this chapter provides the details.},
  textii={\precis{In this chapter we discuss a method that allows the production of fancy chapter headings and formatting, based on a set of key values. Central  to this process is the separation of content from presentation.
We also discuss the basic formatting tools that are available and how one can modify them to mould new book designs.}
 }
}


\chapter{Designing Chapter Headings}
\addtocimage{-12pt}{-20pt}{./images/tocblock-man-01.jpg}

\section*{Introduction}

A \textls*{crowded} first page is as unsightly as a crowded title page, wrote De Vinne in \emph{Modern Methods of Book Composition} in 1904.  Not much has changed since. A new chapter must make a good impression and must give an immediate signal that a different topic is going to be discussed. Traditionally chapter openings in LaTeX are an unimpressive and dry event. Our aim is to brighten it up a bit, while keeping true separation of content from presentation, but avoiding the pit traps of over ornamenting the design. A book is to be read and we should provide minimal ornamentation. \index[phdkeys]{chapter> ornamentation}

% \usepackage{array,tabularx}
%\newcolumntype{Y}{>{\raggedleft\arraybackslash}X}% see tabularx
%\tcbset{enhanced,fonttitle=\bfseries\large,fontupper=\normalsize\sffamily,
%colback=yellow!10!white,colframe=red!50!black,colbacktitle=thecodebackground,
%coltitle=black,center title,
%tabularx={X||Y|Y|Y|Y||Y},% this sets ’before upper’ and ’after upper’
%before upper app={Group & One & Two & Three & Four & Sum\\\hline\hline} }
%
%\begin{tcolorbox}[title=My table]
%Red & 1000.00 & 2000.00 & 3000.00 & 4000.00 & 10000.00\\\hline
%Green & 2000.00 & 3000.00 & 4000.00 & 5000.00 & 14000.00\\\hline
%Blue & 3000.00 & 4000.00 & 5000.00 & 6000.00 & 18000.00\\\hline\hline
%Sum & 6000.00 & 9000.00 & 12000.00 & 15000.00 & 42000.00
%\end{tcolorbox}

\begin{figure}[htbp]
\centering
\parindent=0pt
\fbox{\includegraphics[width=\textwidth]{metropolitan-spread}}
\par
\caption{A chapter opening from the Metropolitan Museum of Art publicaion, \textit{Assyrian Reliefs and Ivories} by Vaughn. E. Crawford et. al., 1980. The spread is simple and the chapters are not numbered. This is a common characteristic of many more recently published books.}
\end{figure}


What is to us now a common occurence with instant book-printing was not always so. The cost of illustrated books was a prime factor and as Tschichold wrote:
\begin{quotation}
In the area of book design, in the last few years a revolution has taken place, until recently recognized by only a few. but which now begins to influence a much wider range of action.
It means placing much greater emphasis on the appearance of the book and a wholly contemporary use of typographic and photographic means. Before the invention of printing, literature of that time was spread around by the mouth of the author himself or by professional bards. The books of the Middle Ages - like the "Mannessische Liederhandschrift" - had
\end{quotation}

The type of book you are writing and its contents will determine an appropriate design for chapter headings and the type of design and numbering if any for subsections. Here we are merely providing a mechanism to produce them. These methods can produce a mastepiece or an ugly piece of work. Some simple suggestions follow (from my observations of styles in books I like). In general you need to think what type of book you are developing. For example a novel, should be sectioned very carefully. Many books avoid marking of sections other than chapters totally, perhaps marking them just with a soft ornament such as three centered asterisks.

\section{Numbering of Sections}


In general books do not number sections beyond subsection. You can avoid them all together, if you are not going to reference the sections extensively. 

In works of fiction, authors sometimes number their chapters eccentrically, often as a metafictional statement. For example:
Seiobo There Below by László Krasznahorkai has chapters numbered according to the Fibonacci sequence.

The Curious Incident of the Dog in the Night-Time by Mark Haddon only has chapters which are prime numbers.

At Swim-Two-Birds by Flann O'Brien has the first page titled Chapter 1, but has no further chapter divisions.

God, A Users' Guide by Seán Moncrieff is chaptered backwards (i.e., the first chapter is chapter 20 and the last is chapter 1). The novel The Running Man by Stephen King also uses a similar chapter numbering scheme.
Every novel in the series A Series of Unfortunate Events by Lemony Snicket has thirteen chapters, except the final instalment (The End), which has a fourteenth chapter formatted as its own novel.

Mammoth by John Varley has the chapters ordered chronologically from the point of view of a non-time-traveler, but, as most of the characters travel through time, this leads to the chapters defying the conventional order.


\begin{pgfpicture}
\pgfpathmoveto{\pgfpointorigin}
\pgfpathlineto{\pgfpoint{1cm}{1cm}}
\pgfpathlineto{\pgfpoint{1cm}{0cm}}
\pgfusepath{fill}
\end{pgfpicture}




\begin{figure}[tbp]
\centering
\parindent=0pt
\fbox{\includegraphics[width=\textwidth]{fantasy-architecture}}
\par
\caption{A chapter opening from the Metropolitan Museum of Art publicaion, \textit{Assyrian Reliefs and Ivories} by Vaughn. E. Crawford et. al., 1980. The spread is simple and the chapters are not numbered. This is a common characteristic of many more recent books.}
\end{figure}


\begin{figure}[tbp]
\centering
\parindent=0pt
\fbox{\includegraphics[width=\textwidth]{fantasy-architecture-02}}
\par
\caption{A chapter opening from the Metropolitan Museum of Art publicaion, \textit{Assyrian Reliefs and Ivories} by Vaughn. E. Crawford et. al., 1980. The spread is simple and the chapters are not numbered. This is a common characteristic of many more recent books.}
\end{figure}


\section*{Use of Color}

The modern books that Tschilchod was discussing have long been overwhelmed by the appearance of larger, coffee book type of books. Our brains our now conditioned by branding and graphic design is everywhere. 

Once you have decided that the book is going to be a bit more colorfull, the choice of color will follow. The decision what to color will be an important one, which brings us to color theory. The history of color is perhaps as colorfull as the rest. Attempts to formalize and recognize order date back to Aristotle (384-322 bce) but began in earnest with Leonardo da Vinci (1452-1519) and have progressed ever since. Leonardo noted that certain colors intensify each other, discovering \textit{contrary} and \textit{complementary} colors. The first color wheel was invented by Britain's Sir Isaac Newton (1642-1727), who split white light into red, orange, yellow, green, blue, indigo and violet beams, then joined the two ends of the spectrum to form a circle showing the natural progression of colors. When Newton created the color wheel, he noticed that mixing two colors from opposite positions produced a neutral or \textit{anonymous} color.


\begin{figure}[htbp]
\parindent=0pt
\centering
\fbox{\includegraphics[width=\textwidth]{line-designs} }
\caption{Spread from \textit{Beautiful Geometry}, Eli Maor and Eugen Jost, Princeton Univeristy Press, 2014. A subtle coloring of the chapter heading, de-emphasizing the chapter number and coloring the chapter title. There is no chapter label. A dropcap with the same color starts the first paragraph. This style is easy to achive with the phd system.}
\end{figure}


\begin{figure}[htbp]
\parindent=0pt
\centering
\fbox{\includegraphics[width=\textwidth]{color-book01.jpg} }
\bigskip

\fbox{\includegraphics[width=\textwidth]{color-book02.jpg} }
\end{figure}

One would expect a book written for the sole purpose of describing color theory and its application to the Graphic Arts, is expected to be colorful. Note the de-emphasizing of the label and number. 

\begin{figure}[htbp]
\parindent=0pt
\centering
\fbox{\includegraphics[width=\textwidth]{color-book-03.jpg} }
The chapter heading label and number are almost invisible. The heading text, is typeset in large bold letters, shouting what is coming next. Not your typical scintific book\ldots
\bigskip

\fbox{\includegraphics[width=\textwidth]{color-book-04.jpg} }
\end{figure}

Advertizing people understand that they need to present the message of an advertizement loud and clear so as to catch the busy eye. A heading's message is the title description. Neither the label not the chapter if any are necessary to convey the message. The chapter heading is analogous to the stop at the end of a sentence. The brain gets a signal to absorb what was written before it and get ready for the next. The heading signals the end of a topic. One must not dwell on it.


\section{Contemporary Chapter Headings}

In the book \textit{China} the designer used both a chapter heading on a spread of two images, as well as repeated the chapter number on the text pages \ref{fig:threepage}. The images distill the message of the chapter, although the chapter subtitle is almost unreadable, dominated by the surrounding text. From a technical perspective, the chapter command must paint the two images, set the right type of heading for each page and then without increasing the counter, change the counter to one that displays the chapter number in words and then continue with typesetting the text. A careful choice of images is necessary for such chapters, as well as cropping the images to match the aspect ratio of the book pages. One also needs to be carefull for \latexe not to place any floats in between the page spreads. 

\begin{figure}[htbp]
\parindent=0pt
\centering
\fbox{\includegraphics[width=\textwidth]{beijing.jpg} }\par
\vfill

\fbox{\includegraphics[width=\textwidth]{beijing-01.jpg} }\par
%\fbox{\includegraphics[width=\textwidth]{pearl-river.jpg} }
\caption{A full page chapter spread.}
\label{fig:threepage}
\end{figure}

\begin{figure}[htbp]
\parindent=0pt
\centering
\fbox{\includegraphics[width=\textwidth]{beijing.jpg} }\par
\vfill

\fbox{\includegraphics[width=\textwidth]{beijing-01.jpg} }\par
%\fbox{\includegraphics[width=\textwidth]{pearl-river.jpg} }
\caption{A full page chapter spread.}
\label{fig:threepage}
\end{figure}


\clearpage



In Figure~\ref{fig:photospread} the bands are black, but position low on the page. The size of the pages are 9.69 \texttimes 11.42. The books sections are not numbered. Text i sbroken through inserts of bigger text. Many of the examples here are from
commercial nude photography books, as they tend to break with tradition. In the 1970s and 1980s, fashion photographers began to present a
new, confrontational image of the female body. The pioneer in this
respect was the German Helmut Newton (1920–2004). Newton’s
photographs of nudes were overtly sexual, with an undertone of
menace, and although his models tended to be depicted as part
of the social elite they were often placed, apparently caught out
in reportage style, in sordid environments engaged in fantasy and
fetish. His work made him highly influential in fashion photography,
though some of it was thought too highly sexual for American
magazines and appeared only in those published in Europe.


\begin{figure}[htbp]
\parindent=0pt
\includegraphics[width=\textwidth]{baetens-01.jpg} \par
\vfill\vfill\vfill\vfill
\includegraphics[width=\textwidth]{baetens-02.jpg}\par
\caption{Chapter spread and first pages after the chapter title which is on the right page of the chapter spread. From \textit{New Photography, Art and the Craft}, Pascal Baetens, DK Publications. }
\label{fig:photospread}
\end{figure}

In the 1980s, Newton undressed the dynamic and independent
female in a series called Big Nudes. In this series the women are
indeed naked and very tall, wearing nothing but makeup and high
heels. The Big Nudes were exhibited in the form of life-size prints
that were intended to provoke the viewer by showing self-confident
women who knew what they wanted and were very aware of their
beauty and sexuality



\chapter{Package Usage}

To use the package include it just like any other package:

\begin{teXXX}
\documentclass{book}
\usepackage{phd}
\cxset{style13}
\begin{document}
\chapter{Introduction}
\end{document}
\end{teXXX}

The command \docAuxCommand{cxset} sets the default style for the example to the style defined as \meta{style13}. The package currently offers  100 templates and numerous keys to manipulate them further. Styles are similar to \enquote{themes} used in web programming; they are a collection of keys that resemble in many ways \texttt{css}. Styles can have any names and I am sure as package usage increases and evolve,they will get better names. 

\section{Background}

Before describing in detail how to specify a new layout for headings, we offer an overview of how the task can be accomplished and the design philosophy behind the approach. 

Irrespective of the technique and tools used, the creation of new layouts can always be divided into the following three tasks: constructing a document from “layout bricks”, which we can term as “blocks” or “elements”; establishing the layout semantics of each block; and finally, creating a layout engine supporting any document constructed from such blocks.

\begin{description}
\item [Canned Layouts] At one end of the spectrum, the most accessible approach consists of picking, a canned layout, such as LaTeX itself and perhaps only provide rudimentary macros to manipulate it.
\item [Constraints] Constraints offer a middle ground between canned layouts and handwritten layout engines. Constraints are arguably the most widespread and successful layout programming technique. For, instance, the foundations of \tex are laid upon constraint. CSS, the ubiquitous web template language, also relies on constraints, although in a more restricted and indirect manner.
\end{description}

\subsection{Blocks and Elements}

We define an \emph{element} as a document block, that cannot be subdivided further. For example the chapter title element, is composed of the text of the chapter title. 

A \emph{block} on the other hand is can contain other blocks and or numerous elements. We can consider the chapter headings as \emph{blocks}, composed of three blocks the chapter, number and title. Each block is then composed of elements. Each element has properties and traits. One of these mandary properties is the name. 

Blocks are either \emph{configured} (all constraints are mandatory), or flexible (there are optional/alternative constraints). By bundling optional constraints, flexible blocks make their specification customizable by non-technical users. 

\subsection{Language semantics}

One of the aims of the syntax of the templates was to offer familiar terminology and to remove the use
of \tex macros as far as possible from templates. 
\medskip

{\parindent0pt

 \textit{section}| font-family=serif,|\\
 \textit{section}| font-size=LARGE,|\\
 \textit{section}| font-weight=bold,|\\
}

The restriction I imposed is problematic when dealing with fractions of linewidths and textwidths. So
at present we allow for example |title text-width=0.5\texwidth| or |title text-width=10cm| or any other valid units. Ideas for improvements can only come from user feedback in the future.

Some experimental ideas incorporated are:

\begin{verbatim}
title text-width = 0.5 text-width,
title text-width = 1.2 text-width,
\end{verbatim}

A better parser will need to be programmed for dimensions, which are all currently handled as etex |dimexpr|. 

The syntax must allows both for microtypography as well as macro-typographical features. The former would deal with mostly fonts, spacing and text justification, where the latter deals with layouts, borders shapes and the positioning of elements on the page and also reletively to other elements or blocks.

An advantage of this approach is that it also opens the possibility of parsing the text with a language other than \tex and translating the document to another format, such as |HTML| or |XML| either fully or partially. Next we will describe both the syntax as well as the usage of the settings.

\section{Chapter opening page}

The standard \latexe classes offer only two options to either open a chapter on an odd page or at any page. This package offers five alternatives:

\begin{docKey}[phd]{chapter opening}{=\meta{any, left, right, anywhere, ifafter}}{default none, initial=any}
For documents that are primarily to be read on the web, use |any| for normal books, use \textit{right}. Some templates that we provide use |any| and the examples use |anywhere| to enable us to display the heading at any position on the page.
\end{docKey}

\begin{decription}
\item [any] Opens a chapter at any page, either \textit{verso} or \textit{recto}.
\item [left] Opens a chapter on an even page
\item [right] Opens a chapter on a right page.
\item [anywhere] Opens a chapter at the point where the \cs{chapter} is typed.
\item [none] Alias for \marg{anywhere}.
\item [ifafter] Opens a chapter at the next page if the page has material that does not exceed a certain portion of \cs{textheight}.
\end{description}

\colorlet{theoption}{bgsexy}

To change a setting you just modify the value of the key \oarg{\option{chapter opening}} to one of the values described earlier. 

\begin{dispListing}
\cxset{chapter opening = anywhere}
\end{dispListing}
 
We use this key to print the many examples typesetting chapter heads that follow (see the example~\ref{ex:anywhere}).  


\begin{texexample}{title=Inline Chapter Example}{ex:anywhere}
\cxset{examplestyle/.style = {chapter format = block,
       chapter opening = anywhere,
       chapter name = CHAPTER, 
       %label
       chapter label font-family      = sffamily,
       chapter label color            = primary,
       chapter label background-color = white,
       % number
       chapter number font-family = sffamily,
       chapter number font-size = HUGE,
       chapter number color     = primary,
       chapter label align = centering,
       chapter number background-color = white,
       %title
       chapter title font-family = rmfamily,
       chapter title align = centering,
       chapter title background-color = bgsexy!15,
       chapter title before background-color=white}}
\cxset{examplestyle}       
\lorem
\chapter{Typography Example}
\lorem
\chapter{Another Chapter Heading}
\lorem
\end{texexample}


%\cxset{toc chapter = true}
\addtocounter{chapter}{-1}

Examples for other types of chapter openings follow in the rest of the documentation.

\subsection{Blank pages before chapters}

In the standard LaTeX book class when the \texttt{openany} option is not given or in the report class when the openright is given, chapters start at odd-numbered pages. This can cause a blank page to be printed. Some book designers prefer this page to be completely empty, without any headers or footers. This cannot be done with \lstinline{\thispagestyle} as this command will have to be issued on the \textit{previous} page. However by a suitable redefinition of the
\lstinline{\clearpage} this can be done automatically.
\medskip

\begin{teXXX}
\makeatletter
\def\cleardoublepage{\clearpage\if@twoside\ifodd\c@page\else
  \hbox{}
  \vspace*{\fill}
  \begin{center}
    This page left intentionally blank.
  \end{center}
  \vspace{\fill}
  \thispagestyle{empty}
  \newpage
  \if@twocolumn\hbox{}\newpage\fi\fi\fi}
\makeatother
\end{teXXX}


This is achieved easily by setting the following options:
\bigskip

\begin{tcolorbox}
\lstinline{chapter blank page=empty}\par
\lstinline{chapter blank page text=Some text.}\par
\lstinline{chapter blank page=plain}\par
\end{tcolorbox}
\medskip



The last one refers to a \lstinline!\thispagestyle{plain}!.
\cxset{chapter opening = right, chapter format = block}
\chapter{Test}

\cxset{defaults, chapter opening= anywhere}



\section*{Keys for chapter head formatting}

A chapter heading can be considered of being constructed of several parts, the \textit{chapter number}, the chapter name typically \textit{chapter} and the \textit{title}. Predefined keys handle all the elements of formatting. Additional keys are defined to handle other elements such as inclusion of images or producing complicated examples with graphics constructed with \texttt{TikZ} and other similar packages.


\bigskip\bigskip\bigskip\bigskip
\let\oldrefkey\refKey
\let\refKey\texttt
\makeatletter
\long\def\demobox#1#2{%
\par\bigskip\bigskip\bigskip
\begin{tcolorbox}[enhanced,left=0pt, top=0pt, bottom=0pt,width=\textwidth,
  enlarge top initially by=1cm,enlarge bottom finally by=1cm,left skip=1cm,right skip=1cm,
  colframe=white,colback=white,
  colbacktitle=red!30!white,colupper=black!7!white,
  code={\appto\kvtcb@shadow{%
    \path[fill=white,draw=yellow!50!black,dashed,line width=0.4pt]
      ([xshift=-1cm,yshift=-1cm]frame.south west) rectangle
      ([xshift=1cm,yshift=1cm]frame.north east);
     \path[fill=blue!20!white, 
              opacity=0.3, draw=yellow!50!black,solid,line width=1pt]
      ([xshift=-2cm,yshift=-2cm]frame.south west) rectangle
      ([xshift=2cm,yshift=2cm]frame.north east);  
    }},
  finish={
  \draw[thick,<->] ([yshift=-1.3cm]frame.north west)-- node[below]{\texttt{#1 width}}
    ([yshift=-1.3cm]frame.north east);
  \draw[thick,<->] ([xshift=-15mm]frame.north east)-- node[above]{\refKey{#1 height}}
    ([xshift=-15mm]frame.south east);
  \draw[thick,<->] (frame.north)-- node[right]{\refKey{#1 padding-top}} +(0,1);
  \draw[thick,<->] ([yshift=1cm]frame.north)-- node[right]{\refKey{#1 margin-top}} +(0,1);
  \draw[thick,<->] (frame.south)-- node[right, align=left]{\refKey{#1 padding-bottom}}+(0,-1);
  %left padding
  \draw[thick,<->] (frame.west)-- node[below right,align=center]{\refKey{#1 padding-left }}+(-1,0);
  %left margin
  \draw[thick,<->] ([xshift=-1cm,yshift=-0.9cm]frame.west)-- node[below right,xshift=-1,align=left]{\refKey{#1 margin-left }\\\refKey{#1 grow to left by}}+(-1,0);
  %right padding
  \draw[thick,<->] (frame.east)-- node[below left,align=center]{\refKey{#1 padding-right}}+(1,0);
 %right margin
  \draw[thick,<->] ([xshift=1cm,yshift=-0.9cm]frame.east)-- node[below left,xshift=1, align=right]{\refKey{#1 margin-right}\\\refKey{#1 grow to right by}}+(1,0);
 \draw[thick,<->] ([yshift=-2cm]frame.south)-- node[right, align=left]{\refKey{#1 margin-bottom},\\ \refKey{#1 after skip}}+(0,1);
  }
    ]
#2%
%\hrule width0pt height4.5cm depth0pt\relax% \vspace*{4.5cm}% \lipsum[1]
\end{tcolorbox}\par
\bigskip\bigskip\bigskip}
\makeatother

\demobox{chapter}{\scalebox{1.17}{\HHHUGE Chapter}}

The number box is again drawn in a box similar to a chapter with all properties generalized.

\demobox{number}{\scalebox{1.15}{\HHHUGE Thirteen}}



All parameters shown in the diagram can be set using the command \cs{cxset}. The property names follow conventions similar to those of |css|, rather than typical conventions of \tikzname that are more widely known to the programming community. The prefix to these properties (in the example \textit{chapter}) can be thought of
as similar to a |class| or |id| name in |css|.  

\begin{docCommand}{cxset}{\marg{options}}
  Sets options for every following \refEnv{tcolorbox} inside the current \TeX\ group.
  By default, this does not apply to nested boxes, see \Vref{subsec:everybox}.\par
  For example, the colors of the boxes may be defined for the whole document by this:
\begin{dispListing}
\cxset{chapter numbering = Roman,
       chapter number color = blue}
\end{dispListing}
\end{docCommand}

\begin{docKey}[]{chapter padding-top}{=\meta{dimension}}{no default, initial value 0pt}
All padding keys take one argument, which is a dimension. The length is also stored in a register
\cmd{\chapterpaddingtop}. In this chapter it was set at %\the\chapterpaddingtop.
\begin{dispListing}
\cxset{colback=red!5!white,colframe=red!75!black, chapter padding-top=2pt}
\end{dispListing}
\end{docKey}



\begin{docKey}[]{chapter padding-right}{=\meta{dimension}}{no default, initial value 0pt}
All padding keys take one argument, which is a dimension. The length is also stored in a register
\cmd{\chapterpaddingright}.  In this chapter it was set at %\the\chapterpaddingright.
\end{docKey}

\begin{docKey}[]{chapter padding-bottom}{=\meta{dimension}}{no default, initial value 0pt}
All padding keys take one argument, which is a dimension. The length is also stored in a register
\cmd{\chapterpaddingbottom}.  In this chapter it was set at %\the\chapterpaddingbottom.
\end{docKey}

\begin{docKey}[]{chapter padding-left}{=\meta{dimension}}{no default, initial value 0pt}
All padding keys take one argument, which is a dimension. The length is also stored in a register
\cmd{\chapterpaddingleft}.  In this chapter it was set at %\the\chapterpaddingleft.
\end{docKey}

%% margin

\begin{docKey}[]{chapter margin-top}{=\meta{dimension}}{no default, initial value 0pt}
All padding keys take one argument, which is a dimension. The length is also stored in a register
\cmd{\chaptermargintop}. In this chapter it was set at .
\end{docKey}

\begin{docKey}[]{chapter margin-right}{=\meta{dimension}}{no default, initial value 0pt}
All padding keys take one argument, which is a dimension. The length is also stored in a register
\cmd{\chapterpaddingright}.  In this chapter it was set at %\the\chapterpaddingright.
\end{docKey}

\begin{docKey}[]{chapter margin-bottom}{=\meta{dimension}}{no default, initial value 0pt}
All padding keys take one argument, which is a dimension. The length is also stored in a register
\cmd{\chapterpaddingbottom}.  In this chapter it was set at %\the\chapterpaddingbottom.
\end{docKey}

\begin{docKey}[]{chapter margin-left}{=\meta{dimension}}{no default, initial value 0pt}
All padding keys take one argument, which is a dimension. The length is also stored in a register
\cmd{\chaptermarginleft}.  In this chapter it was set at %\the\chaptermarginleft.
\end{docKey}

\subsection{Borders}

Border have three properties \emph{width, color} and \emph{style}. They can set individually for
each side of the box or using the shorter key .

\begin{docKey}[]{chapter border-top-width}{ = \meta{dimension}}{no default, initial value 0pt}
All border keys take one argument, which is a dimension.
\end{docKey}

\begin{docKey}[]{chapter border-right-width}{=\meta{dimension}}{no default, initial value 0pt}
All border keys take one argument, which is a dimension.
\end{docKey}

\begin{docKey}[]{chapter border-bottom-width}{ = \meta{dimension}}{no default, initial value 0pt}
All border keys take one argument, which is a dimension.
\end{docKey}

\begin{docKey}[]{chapter border-left-width}{ = \meta{dimension}}{no default, initial value 0pt}
All border keys take one argument, which is a dimension.
\end{docKey}

\subsubsection{Border Colors}

The colors follow the same pattern for |border-width| and again they can be set individually or using
a shorter key to set all of them in one color. 

\begin{docKey}[]{chapter border-top-color}{=\meta{color name}}{no default, initial value black}
All border keys take one argument, which is a dimension.
\end{docKey}

\begin{docKey}[]{chapter border-right-color}{=\meta{color name}}{no default, initial value black}
All border keys take one argument, which is a dimension.
\end{docKey}

\begin{docKey}[]{chapter border-bottom-color}{=\meta{color name}}{no default, initial value black}
All border keys take one argument, which is a dimension.
\end{docKey}

\begin{docKey}[]{chapter border-left-color}{=\meta{color name}}{no default, initial value black}
This key is stored in \cmd{\chapterborderrightcolor} and the value in this chapter is 
%\ExplSyntaxOn \l_phd_chapter_border_right_color_tl.
\ExplSyntaxOff
\end{docKey}



\subsubsection{Border Styles}

Standard |css|  offers four styles \emph{dotted, solid, double, dashed}. We offer almost an unlimited set of styles.

\begin{docKey}[phd]{chapter border-top-style}{=\meta{style name}}{no default, initial value \texttt{none}}
The |border-style| properties take a value, which can be |solid, double, dotted, dashed, asterisk|.
\end{docKey}

\begin{docKey}[phd]{chapter border-right-style}{=\meta{style name}}{no default, initial value \texttt{none}}
The |border-style| properties take a value, which can be |solid, double, dotted, dashed, asterisk|.
\end{docKey}

\begin{docKey}[]{chapter border-bottom-style}{=\meta{style name}}{no default, initial value \texttt{none}}
The |border-style| properties take a value, which can be |solid, double, dotted, dashed, asterisk|.
\end{docKey}

\begin{docKey}[]{chapter border-left-style}{=\meta{style name}}{no default, initial value \texttt{none}}
The |border-style| properties take a value, which can be |solid, double, dotted, dashed, asterisk|.
\end{docKey}

\begin{docKey}[phd]{chapter border-style}{=\meta{style name}}{no default, initial value \texttt{none}}
This key sets all chapter-border-\meta{top,right,bottom,left}-style to a single value.
\end{docKey}

\subsubsection{Fonts and colors}

All font parameters can be set using individual keys. The naming scheme in general follows |css| conventions.

\begin{docKey}[phd]{chapter color}{=\meta{color name}}{no default, initial value \texttt{black}}
This key sets the color for the \textit{chapter element}. The color name is stored in \cmd{\chaptercolor@cx}.
The value in this chapter is% \makeatletter\texttt{\chaptercolor@cx}\makeatother.
\end{docKey}

\begin{docKey}[phd]{chapter font-size}{=\meta{Huge, Large}}{no default, initial value \texttt{Huge}}
This sets the size for rendering the \textit{chapter element}. Use one of the following predefined values.
Note that you can either use a command i.e, |chapter font-size=|\cmd{\huge} 
or the command name i.e., |chapter font-size=huge|. The latter is the recommended method.
\end{docKey}

\begin{marglist}
\item [tiny] renders as {\tiny tiny}.
\item[footnotesize] renders as {\footnotesize footnotesize}
\item [small] Opens a chapter on an even page
\item [large] Opens a chapter on a right page.
\item [LARGE] Opens a chapter at the point where the \cs{chapter} is typed.
\item [huge] Alias for \marg{anywhere}.
\item [Huge] Opens a chapter at the next page if the page has material that does not exceed a certain portion of
 \cs{textheight}.
 \item[HUGE] renders as {\HUGE HUGE}.
 \item[HHUGE] renders as {\HHUGE HUGE}.
\end{marglist}

\begin{texexample}{Sizing settings}{}
\cxset{
          chapter format = block,
          chapter label font-size= HUGE,
          chapter name = Chapter,
          chapter format=block,
          chapter number font-size= HUGE,
          chapter title font-size=LARGE,
         % 
         % chapter padding-top=0pt,
         % chapter padding-bottom=0pt,
         % title margin-top=3pt,
         %
          }
\chapter{Setting font-sizes}          
\lorem

\end{texexample}


\begin{docKey}{chapter font-family}{ = \meta{sffamily, rmfamily etc.}}{no default, initial value \texttt{sffamily}}
The |font-family| key accepts \latexe conventional family names or |css| names such as |serif| and |non-serif|. The
value is stored in \docAuxCommand{chapter_font_family}, in this chapter it is set as {\ExplSyntaxOn\meaning\chapter_font_family\ExplSyntaxOff}
\end{docKey}


\begin{marglist}
\item [sffamily] The \emph{chapter element} is rendered in the document default \cmd{\sffamily}.
\item [rmfamily] The \emph{chapter element} is rendered in the document default \cmd{\rmfamily}.
\end{marglist}

%% Font weights
\begin{docKey}[]{chapter font-weight}{=\meta{mdseries,bfseries,etc.}}{no default, initial value \texttt{bfseries}}
The |font-weight| key accepts \latexe conventional family names or |css| names such as |bold| and |bfseries|. The
value is stored in \cmd{\chapterfontweight@cx}, in this chapter it is set as 
{\ExplSyntaxOn\expandafter\string\l_phd_chapter_label_fontweight_tl\ExplSyntaxOff}

\begin{texexample}{Setting chapter element font-weights}{fontweight}
\cxset{chapter label font-weight=normal}
\chapter{Font-weight is normal}
\cxset{chapter label font-weight= bfseries}
\chapter{Font-weight is bfseries}
\lorem
\end{texexample}
\end{docKey}


\begin{marglist}
\item [normal] The \emph{chapter element} is rendered in the document default \cmd{\sffamily}.
\item [bold] The \emph{chapter element} is rendered in the document default \cmd{\rmfamily}.
\item[bfseries] Renders as bold.
\item[mdseries] renders as medium series.
\item[light] This is an alias for normal.
\item[\upshape\ttfamily\string\bfseries] The command version of the setting.
\item[\upshape\ttfamily\string\mdseries] The command version of the setting.
\end{marglist}



\begin{docKey}[]{chapter font-shape}{=\meta{itshape,upshape,etc.}}{no default, initial value \texttt{upshape}}
The |font-weight| key accepts \latexe conventional family names or |css| names such as |bold| and |bfseries|. The
value is stored in |chapter_font_weight|, in this chapter it is set as %\ExplSyntaxOn \texttt{\chapter_font_shape}\ExplSyntaxOff.
\end{docKey}

In |css| the |font-shape| is named as |font-style| so we alias it as well. 

%\begin{marglist}
%\item[normal] normal font-style, defaults to |upshape|.
%\item[upshape] normal font-style, defaults to |upshape|. 
%\item[italic] italic shape, renders as {\itshape italic}. For some fonts it might not be available.
%\item[itshape] italic shape, alias of |italic|.
%\item[oblique] oblique font, in \latexe is equivalent to \cmd{\slshape} and renders as {\slshape slshape}, which might be slightly different than {\itshape italic}.
%\end{marglist}


\begin{texexample}{Setting up Fonts}{chapterfonts}
\cxset{   chapter format = block,
          chapter opening=anywhere,
          chapter label font-weight=normal,
          chapter label font-shape=upshape,
          %chapter border-width=0pt,
          %chapter border-style=none,
          %chapter padding-top=0pt,
          chapter label font-size=large,
          chapter number font-size=large,
          chapter number color=black,
          %title font-size=large,
          }
\chapter[fonts]{Test Font Weights}
\lorem
\cxset{chapter label font-shape=itshape}
\chapter{Test Italic Shape}
\lorem
\cxset{chapter label font-shape=normal}
\chapter{Test normal font-shape}
\lorem
\end{texexample}



The specification of font families is somewhat problematic. In the web the |css| allows |font-family|  to hold several font names as a ``fallback” system. If the browser does not support the first font, it tries the next font.

There are two types of font family names:

\begin{description}
\item[family-name] The name of a font-family, like “times”, “courier”, “arial”, etc.
\item[generic-family] The name of a generic family, like “serif”, “sans-serif”, “cursive”, “fantasy”, “monospace”.
\end{description}

Generally in the \tex community leaving the choice of font  open to what is available on a user’s computer is frowned upon. Knuth’s original aim to render consistently documents, irrespective of a user’s computer installation has served the community well, and it is possible three decades later to produce documents identical in all respects to the original. 

If this is still a valid requirement for documents is debatable. Current document processing requirements are focusing more in the preservation of content and document structure rather than form. Typeset documents in soft copy are now widely preserved in |pdf| or |postcript|  formats. One can archive the |.tex| file as well as the |pdf| file.  Back to the provision of keys, a key defined in a 
similar fashion to those of |css| could help, but there is also the issue of slow compilation. If a font cannot be
found, with the current code, it can slow down compilation tremendously. I am leaving the choice where it belongs to the user and the package writer. It makes no harm if a more flexible definition is provided. The user can then decide to only provide one or many fonts. 

This avoids complicated and almost unintelligible commands such as:

\begin{dispListing}
\setkomafont{subsection}{\usefont{T1}{fvm}{m}{n}}
\setkomafont{section}{\usefont{T1}{fvs}{b}{n}\Large}
\end{dispListing}

Here are some examples. 

\begin{texexample}{Serif and non-serif}{ex:fontfamily}
\cxset{chapter label font-family=serif, 
       chapter opening=anywhere}
\chapter{Serif font}
\lorem
\end{texexample}


\section{Floating and Alignment} 

This particular key bothered me, as the term \emph{float} has a different meaning in \latexe. However, to
be consistent with |css| terminology I have yielded to the temptation and included it.

\begin{docKey}[]{chapter float}{=\meta{left,center,right,none}}{no default, initial value \texttt{none}}
Key that controls the horizontal alignment of the \emph{chapter element}. I order for the
element to float, its |display| property must be set to |inline|.
\end{docKey}

%\begin{texexample}{Floating}{chapter:float}
%\cxset{chapter opening=anywhere, chapter float=center}
%\chapter{Centered Chapter}
%\lorem
%\cxset{chapter float=left}
%\chapter{Left Aligned}
%\lorem
%\cxset{chapter float=right}
%\chapter{Right Aligned}
%\lorem
%\end{texexample}


\subsection{The display property}

Both the |css| box model as well as the \TeX layout engine provide numerous complex algorithms in managing the floating of elements. This is normally controlled using two properties |display| and |float|.


\makeatletter

\begin{docKey}[phd]{chapter position}{ = \meta{absolute, relative}}{no default, initial value black}
This positioning directive instructs the engine to position the element at an exact position.
\end{docKey}



\tcbox[nobeforeafter]{$box_1$}\tcbox[nobeforeafter]{$box_2$}\tcbox[nobeforeafter]{$box_3$}\dotfill\tcbox[nobeforeafter]{$box_n$}
\tcbox[before skip=0.2cm, after skip=0pt, width=1cm, enlarge left by=10cm,width=5cm,enhanced,show bounding box]{title before element}
\tcbox[before skip=0pt, width=1cm, enlarge left by=10cm,width=5cm,enhanced,show bounding box]{
\tcbox{tb}\tcbox{title}\tcbox[nobeforeafter, width=1cm,]{tb}}
\tcbox[before skip=0pt, after skip=12pt, width=1cm, enlarge left by=10cm,width=5cm,enhanced,show bounding box]{\emph{title after} element \fbox{some}}
\makeatother

\begin{docKey}[phd]{chapter float}{=\meta{left,center,right,none}}{no default, initial value \texttt{none}}
Key that controls the horizontal alignment of the \emph{chapter element}. I order for the
element to float, its |display| property must be set to |inline|.
\end{docKey}
In document preparation systems or web page development the layout is user generated, i.e., the user is expected to type the html and the |css| will then specify as to how the page will be rendered by the browser. In our case for documents we can specify how we want the headings to look. The layout manager for each element, creates other associated elements, as shown for the title here. This way most layouts can be accomplished with the declarative visual language of the \pkgname{phd} package. 

\subsubsection{In-line elements}

When an element is specified as |inline| the rendering algorithm places the boxes after each other. This is widely used in |chapter elements| to render the number inline with the chapter name.
\medskip
\bgroup

\noindent
\tcbox[nobeforeafter,width=3cm, height=1cm]{Chapter}\tcbox[nobeforeafter]{twelve}
 
When the property is set as |block| the elements are stacked below each other.
\medskip

\tcbox{chapter  display=block   CHAPTER}
\tcbox{number display=block    TWELVE}

The elements can be considered to be enclosed in a \emph{ghost} element. If the property is set to float we
\begin{figure}[htbp]
\makeatletter
\parindent0pt\fboxsep0pt
\fbox{\vbox to 0pt{\hbox to \dimexpr(\textwidth)\relax{{\hss\tcbox[capture=minipage,width=5cm, height=2cm, top=0pt]{\raggedright number display=block\\ number float=right }}%
}%
}%
}\par
\vspace*{2cm}
\makeatother
\end{figure}
signalling to the layout engine that the element must be placed to the right of the page, as shown in the figure. 


\begin{figure}[htbp]
\makeatletter
\parindent0pt\fboxsep0pt
\fbox{\vbox to 0pt{\hbox to \dimexpr(\textwidth+2cm)\relax{{\hss\tcbox[capture=minipage,width=5cm, height=2cm, top=0pt]{\raggedright number display=block\\ \emph{element} float=right }
\tcbox[capture=minipage,width=5cm, height=2cm, top=0pt]{\raggedright \emph{element} display=block\\ \emph{element} float=right }
}%
}%
}%
}\par
\vspace*{2cm}
\makeatother
\end{figure}

\subsection{Absolute positioning}

Absolute positioning mode, will place an element at an exact position on the page. They are more difficult to
achieve and inflexible. 

\begin{docKey}{position}{=\meta{absolute},\meta{relative}}{no default, initial none}{}

\end{docKey}



This positioning directive instructs the engine to position the element at an exact position.


\begin{docKey}[]{chapter float}{=\meta{left,center,right,none}}{no default, initial value \texttt{none}}
Key that controls the horizontal alignment of the \emph{chapter element}. In order for the
element to float, its |display| property must be set to |inline|.
\end{docKey}
\egroup



\section{Number Element Keys}


\subsection*{Keys for numbering}

Chapter numbering follows that of the standard \LaTeX\ classes and is extended to cover some additional cases such as fully spelled out numbers. This of course is only good for languages that use the arabic numeralsn. For other languages numerals in different formats can be added with simple keys and without the need of \pkgname{polyglossia} or \pkgname{babel}. 

Note that the package uses Heiko Oberdiek's package \pkgname{alphalph} to allow for alphabetic numbering that extends beyond the normal 26 letters of the alphabet. Examples for numbering can be seen in \ref{ex:romannumbering}


\begin{docKey}[phd]{number numbering}{= \oarg{alph,Alph,roman,Roman,none,WORDS,words,none}}{default arabic}
Style of numbering.
\end{docKey}

\begin{marglist}
\item [arabic] Despite that the Arabs call what the West calls Arabic numbers Indian numbers, we provide the value arabic to have normal numbers printed.
\item [alph] Lowercase alphabetic numbering.
\item [Alph] Uppercase alphabetic numbering.
\item [roman] Lowercase roman numbering.
\item [Roman] Uppercase roman numbering.
\item [words] The number is in lowercase words.
\item [WORDS] The number is in uppercase literal numerals.
\item [Words] Prints the number in words and capitalizes the first letter, for example the number 21 will be printed as `Twenty One'\footnote{Currently limited to the first hundred numbers}.
\index{chapter design>numbering>words}
\item [ordinals] Prints the number as ordinal.
\item [Ordinals] Prints the number as Ordinal.
\item [ORDINALS] Prinst the number as ORDINALS.
\item [none] This is equivalent to using the star version of the command. It does not print any number and does not increment the chapter counter.\footnote{I am ambivalent about this, perhaps it will be better to increment it, as it can give a more general approach.}

\end{marglist}
\begin{texexample}{Literal Numbering}{ex:literal}
\cxset{chapter numbering=WORDS} 
\chapter{Literal numbering}
\lorem
\cxset{chapter numbering=words,chapter name=chapter}
\chapter{Literal numbering} 
\lorem
\end{texexample}




\cxset{chapter opening=anywhere, chapter numbering=Roman, chapter number font-shape=upshape}
\index{chapter design>numbering>roman}

\begin{texexample}{Setting up keys for numbering}{ex:romannumberingx}
\bgroup
\cxset{chapter format = traditional, 
       chapter name = CHAPTER, 
       chapter numbering = Roman,
       chapter label color = bgsexy}
\chapter{Roman numbering}
\lorem
\egroup
\end{texexample}





To emulate some old books we also offer an ordinal numbering scheme.

\begin{texexample}{Literal Numbering}{ex:ordinals}
\cxset{chapter numbering=ORDINALS} 
\chapter{Ordinals numbering}
\lorem
\cxset{chapter numbering=words,chapter name=chapter}
\chapter{Literal numbering} 
\lorem
\end{texexample}

\cxset{chapter numbering=arabic}

\subsection{Fonts and colors}
\begin{docKey}[phd]{number color}{=\meta{color name}}{no default, initial value \texttt{black}}
This key sets the color for the \textit{number element}. The color name is stored in %\cmd{\numbercolor@cx}.
The value in this chapter is %\makeatletter\texttt{\numbercolor@cx}\makeatother.
\end{docKey}

\begin{docKey}[phd]{number font-size}{=\meta{Huge, Large}}{no default, initial value \texttt{Huge}}
This sets the size for rendering the \textit{number element}. Use one of the predefined values, as described
in the section for the \emph{chapter} element.
Note that you can either use a command i.e, |number font-size=|\cmd{\huge} 
or the command name i.e., |number font-size=huge|. The latter is the recommended method.
\end{docKey}

Letter spacing can be achieved using the soul package in a combination with the key |spaceout|.
The following examples illustrate the usage.

\index[phdkeys]{{\ttfamily phd/chapter design test}}

%\begin{texexample}{Letter Spacing}{ex:letterspacing}
%\cxset{numbering=Roman,
%        % number letter-spacing=soul,
%        % chapter spaceout=soul,
%         %title spaceout=soul,
%         title font-size=Large,
%         title font-family=rmfamily,
%         title font-shape=scshape}
%\chapter{Letter Spacing}
%
%\lorem
%\end{texexample}

\begin{docKey}[phd]{chapter number letter-spacing}{=\meta{none, true, etc.}}{no default, initial value \texttt{none}}.
\end{docKey}

\begin{marglist}
\item[none] Default value no tracking is used and the letters are spaced as per the basic font information.
\item[inherit] Inherits the letter-spacing settings from the \emph{chapter} element.
\item[true] Letter spacing is employed, using the |soul| package.
\item[false] Alias for |none|.
\item[soul] The \pkgname{soul} package is used for letter-spacing.
\item[microtype] The \pkgname{microtype} package is used for letter-spacing. When the microtype package is used more fine tuning of parameters is available.
\end{marglist}

The example that follows, explains how the features offered by the \pkgname{microtype} package can be used to
set different tracking options.

\begin{texexample}{Microtypography}{micro}
\bgroup

\SetTracking
 [ no ligatures = {f},
 spacing = {600*,-100*, },
 outer spacing = {450,250,150},
 outer kerning = {*,*} ]
 { encoding = * }
 { 100 }

{\huge \textls{Chapter Twenty}}

\SetTracking
 [ no ligatures = {f},
 spacing = {600*,-100*, },
 outer spacing = {450,250,150},
 outer kerning = {*,*} ]
 { encoding = * }
 { 200 }
 
{\huge \textls{Chapter Twenty}}

\egroup
\end{texexample}


\hbox{\drawfontbox{\huge \upshape\textls(Chapter Twenty}}

\hbox{\drawfontbox{\huge \upshape\textls{Chapter Twenty}}}


\section{Styling the chapter title}

Similarly to the number and chapter styling keys exist for styling the chapter title. We summarize the available standard keys below:

\index{chapter design!labels!letter spacing}
\begin{texexample}{Styling the Title}{ex:title} 
\cxset{chapter numbering=arabic, chapter title font-shape=itshape}
\chapter{Chapter title}
\lorem
\end{texexample}


\begin{docKey}[phd]{chapter title font-family}{=\marg{family}}{no default, initial inherit document font}
Selects a predefined font family
\end{docKey}

\begin{texexample}{Title element font styling}{}
\cxset{chapter title font-family=sffamily}
\chapter{Title font family settings}
\lorem
\cxset{chapter title font-shape=itshape}
\chapter{Title font-style settings}
\lorem
\end{texexample}


\begin{docKey}[phd]{chapter title font-weight}{ = \marg{\cs{bfseries},\cs{normalseries}}} {}
Font weight.
\end{docKey}

\begin{docKey}[phd]{chapter title font-size}{= \marg{large, Large, huge, Huge, HUGE, HHuge}}{}
Font sizing commands or their names. Both \docAuxCommand{\HUGE} and HUGE are allowed to be used as values for the key.
\end{docKey}

\begin{docKey}[phd]{chapter title color} { = \marg{color}} {}
The color of the chapter title letters. This takes any predefined color name. 
\end{docKey}


\begin{docKey}[phd]{chapter title spaceout}{ = \marg{soul,none}} {no default, initial = none}
 This key will space out the title. 
\end{docKey}

\begin{texexample}{Title element spacing}{}
\cxset{chapter name=none,
       chapter numbering=none,
       chapter title font-size=Large,
       chapter title color=black,
       chapter title width=0.6\textwidth,
       %title spaceout=soul,
         }
\chapter{The Prehistoric Period in South-East Asia: 2300 BC--AD 400}        
\lorem 
    
\end{texexample}
\cxset{defaults}


\subsection*{Adding content before and after the title element}

Like all the other elements, the title element can be decorated with additional content,
before and after the text. There are two different forms. 

\begin{docKey}[phd]{title before}{=\marg{code}}{default none}
Contents before the title (vertical material)
\end{docKey}

\begin{docKey}[phd]{title after}{=\marg{code}}{default none}
Contents after the title (vertical material)
\end{docKey}

\begin{docKey}[phd]{title content before}{=\marg{code}}{default none}
Contents before the title (horizontal material)
\end{docKey}

\begin{docKey}[phd]{title content after}{=\marg{code}}{default none}
Contents after the title (horizontal material)
\end{docKey}

The difference between the two type of settings, consider the following situation. Assume you have a title that has a rule at the top and bottom and the text is surrounded by two ornaments. The surrounding ornaments will be inserted using the |title before content|, and the rules using the |title before| form. The |title before| is a full fledged element on its own. 

%{
%\hrule
%\centering
%*** Introduction ***
%\par
%\hrule
%}
%
%{
%\MakePercentComment
%\startlineat{200}
%\lstinputlisting{./styles/style13.tex}
%\MakePercentIgnore
%}



 
\begin{docKey}{/phd/ chapter title before skip}{= \marg{soul,none}}{}
Before title string skip.
\end{docKey}

\begin{docKey}{/phd/ chapter title after skip}{ = \marg{soul,none} }{}
After title string skip.
\end{docKey}

\lorem 
%
%\begin{texexample}{letter spacing the chapter title block}{ex:title3}
%
%\cxset{chapter spaceout=none,
%         numbering=arabic}
%         
%\chapter{Chapter Title Styling}
%\end{texexample}
%
%\end{document}



\cxset{chapter opening=right}
\section{Table of Contents}\index{table of contents!key settings}

Traditionally a chapter will be added to the Table of Contents if the \cs{chapter} command is issued. The starred version will not produce a number and will not add a contents line. Since we have adopted an approach where we use a key value interface we can dispense with the starred version of the command, by setting the \option{chapter toc} option to false. For example if we want to define a command for a ``Foreward'' or ``Epiloque'' without wishing them to be added to the table of contents we can use the following setting.\index{Foreward>definitions}\index{Epilogue>definitions}



\begin{texexample}{changing the chapter label name}{}
\cxset{chapter name=Chapteris, chapter numbering=arabic,}
\chapter{Foreward}
\lorem
\end{texexample}

Note that the key \option{numbering=none} still has to be set.


Please note that when \textbf{numbering=none} the chapter number is not available anymore and yo may have to reset it if required again. Although this might be seen as rather cumbersome than simply using \cs{chapter*} the advantage is consistency in the user interface and the use of appropriate semantic definitions for all sectioning commands thus achieving a bit more separation of context from style.


%\cxset{chapter toc=true}

\section{Defining styles}

Named styles can be defined using the standard \textsc{PGF} conventions. To define a style for the forward above we can use:

\begin{texexample}{}{}
\cxset{foreward/.style={chapter numbering=none,
          chapter name=none,
          chapter title font-size= Large,
          chapter title font-family= sffamily,
          chapter numbering=none}}
\cxset{foreward}
\chapter{Foreward.}
\lorem
\end{texexample}



\cxset{chapter numbering=arabic}
\section{Creating semantic names for commands and environments}

To keep our search for semantic commands and true separation of contents it is prudent to define some macros for typesetting the  `foreward' section.

\bgroup
\begin{texexample}{defining a \textit{Foreward} macro.}{}
\begin{lstlisting}
\cxset{foreward/.style={chapter toc=false,
          name=none,
          title font-size = Large,
          title font-family = sffamily,
          numbering=none}}
\newcommand\forewardname{foreward}
\expandafter\newenvironment\expandafter{\forewardname}{%
\cxset{foreward}\chapter{Foreward}}%
{}
\begin{foreward}
\lorem
\end{foreward}
\end{lstlisting}
\end{texexample}
\egroup

Notice the use of a new command \cmd{\forewardname} to allow for internationlization using Babel or other methods. One is tempted to let the English name, but a better approach perhaps is to define both.

\makeatletter



% ^^A\chapter{Futurelet}
\precis{A discussion on one of the most esoteric commands of \protect\tex, with examples as to how to write macros with optional arguments.}
\addtocimage{-12pt}{-20pt}{../images/tocblock-futurelet.jpg}
\epigraph{Life can only be understood backwards; but it must be lived forwards.}{
---S Kierkegaard}

The \cmd{\futurelet} primitive deserves its own chapter, as most people have difficulty in understanding the command. The instruction allows the user to \textit{look ahead}. By look ahead we mean that \tex will look at a future token\footnote{remember that a token is either a single character or a macro command} without absorbing it, i.e, without removing that token from the token list. This operation allows the programmer to perform a test to check what token is 'coming'. You can read a couple of articles about it for example \citep{Eijkhout2001}, but generally they are difficult to follow. The information about the command is also very sparse in the TeXBook.  Another TUGboat article is \citep{bechto88}, which gives pretty much the same example as we describe below. 

The token looked at through
|\futurelet| will be removed later, typically as part
of an argument of a later macro call as we will see
shortly. It is not removed by the action of the
|\futurelet| primitive.

Let us be more precise now; the |\futurelet|
instruction has the following format:


\begin{teX}
\futurelet (tokenl) (token2) (token3)
\end{teX}


\begin{enumerate}
\item  \tex will execute a \cmd{\let}\meta{tokenl}=\meta{token3}.
We therefore have generated a copy of (token3)
stored under the name of (tokenl).\label{lettoken}


\item  removes (tokenl) from the main token list.

\item \tex expands (token2). This token is for all
practical purposes a macro with the following
properties:

(a) The macro will use (tokenl), which is a
copy of (token3), to find out what (token3)
is, in other words what token is to be
expected later.
(b) It will cause another macro to be expanded
which will ultimately absorb (token3).

This other macro ordinarily depends on
what $<token_l>$ is.

\end{enumerate}

The description above, is a bit of a mouthful and it is better to describe it with an example. In Example~\ref{futurelet} we will try and find if the next token is the opening square bracket `['. We then according to the definition in \ref{lettoken} this should be stored in \cs{tokenone}. We verify this by peeking at its meaning.

\begin{texexample}{futurelet}{futurelet}
\def\tokentwo#1{}
\futurelet\tokenone\tokentwo[
\meaning\tokenone
\end{texexample}

The second token \cs{tokentwo} we have defined it, so that it justs absorbs its next argument and does nothing for the time being. As you can see its meaning is \texttt{the character [}. Now what happens if there was a space between the \cs{tokentwo} and the `['?

\begin{texexample}{futurelet second}{futurelet2}
\def\tokentwo#1{}
\futurelet\tokenone\tokentwo     [
\meaning\tokenone
\end{texexample}

As you can see so far the spaces have been absorbed, but let us now change the definition of \cs{tokentwo}.

\begin{texexample}{futurelet second}{futurelet2}
\def\tokentwo#1{}
\futurelet\tokenone\tokentwo     
\meaning\tokenone
\end{texexample}



\begin{texexample}{futurelet}{futurelet}
\def\tokentwo#1{%
   \ifx\tokenone[ true [\else false\fi
}
\futurelet\tokenone\tokentwo[
\meaning\tokenone
\end{texexample}

We try again with spaces,

\emphasis{tokentwo,[}
\begin{texexample}{futurelet}{futurelet}
\def\tokentwo#1{%
   \ifx\tokenone[ true [\else false\fi
}
\futurelet\tokenone\tokentwo     [
\meaning\tokenone
\end{texexample}

As you can see from the examples we cannot capture the spaces. This might present a problem, if we enclose everything in other macros as \tex might leave extra spaces in the stream. Better to absorb them. We will see how later, using LaTeX. 


\section{Applications}

There are many applications of |\futurelet|.
will here present only one example, although
we will present it in quite some detail so the user
will know how to apply |\futurelet| in different
circumstances.

\subsection{Using \textbackslash futurelet in Macros with Optional
Arguments}

A typical application of |\futurelet| is the handling
of macros with optional arguments\cite{Becht1988} as they are used,
for instance, in \latex. By "optional argument" we
mean an argument which in most cases is omitted,
and is provided only occasionally in macro calls.\footnote{See also the discussion at \url{http://tex.stackexchange.com/questions/4557/how-to-use-futurelet-to-define-optional-parameters}}

\textbf{Defining the Problem}

Let us give a specific example: we would like to
define a macro \cmd{xx}, which can be called in two
different ways:

\begin{enumerate}
\item With optional argument as in |\xx [opt]{arg}|
where opt is the optional argument enclosed
in square brackets and \meta{arg} is the mandatory argument
argument.

\item Without optional argument as in |\xx{arg}|
where \meta{arg} is again the regular argument.

\end{enumerate}


Before we discuss how this can be done in \tex,
observe that we do not really have to use an
optional argument. We could simply define two
different macros \cmd{xxwithoptions} for the case where an
optional argument is given, and \cmd{xxnooptions} for the
case where no optional argument is given:


\begin{texexample}{two macros}{ex:twomacros}
\def\xxWithOpt [#1]#2{...}
\def\xxNoOpt #1{...}
\def\xxWithOpt (#1)#2{\fbox{#2}}
\xxWithOpt (box){Testing}
\end{texexample}

How we can use |\futurelet| to find out
whether an optional argument was given or not?

We will define a macro |\xx| whose only function is
to check whether there is an opening square bracket
(optional argument is present) or not (no optional
argument). The |\xx| macro will, after this has been
determined, cause the |\xxWithOpt| macro to be invoked
when there is an optional argument, and the
|\xxNoOpt| macro to be called if there is no opening
bracket. In other words the macros |\xxWithOpt|
and |\xxNoOpt| do the "real work while the only
purpose of the |\xx| macro is to decide which of the
two macros should be invoked.


Here is the completely worked out example.


\begin{teX}
\def \xxWithOpt [#1] #2{...}
\def\xxNoOpt #2{...}

\def\xx {%
\futurelet\xxLookedAtToken
    \xxDecide
}

% (3) The \xxDecide macro, based on
% the lookahead of \xx, calls
% either \xxWithOpt or \xxNoOpt .
\def\xxDecide {%
 \ifx\xxLookedAtToken [%
\let\next = \xxWithOpt
\else
 \let\next = \xxNoOpt
 \fi
\next
}
\end{teX}

\section{Other Applications in the LaTeX kernel}

\begin{teX}
\def\elidebefore[#1]#2{[$\ldots$] #2}
\def\elideafter#1{#1$\ldots$}

\def\elide {%
\futurelet\ifoptions
    \choosemacro
}

\elide{Lorem Ipsum}

\elide[b]{Lorem ipsum}
\end{teX}

\begin{comment}
% The \choosemacro, based on
% the lookahead of \elide, calls
% either \elidebefore or \elideafter 
\end{comment}

\begin{teX}
\def\choosemacro{%
 \ifx\ifoptions [%
     \let\choice = \elidebefore 
 \else
    \let\choice = \elideafter
 \fi
\choice
}
\end{teX}



\begin{teX}
\elide{Lorem Ipsum}

\elide[b]{Lorem ipsum}

\end{teX}

\begin{teX}
\def \xxWithOpt [#1] #2{...}
\def\xxNoOpt #2{...}

\def\xx {%
\futurelet\xxLookedAtToken
    \xxDecide
}

% (3) The \xxDecide macro, based on
% the lookahead of \xx, calls
% either \xxWithOpt or \xxNoOpt .
\def\xxDecide {%
 \ifx\xxLookedAtToken [%
\let\next = \xxWithOpt
\else
 \let\next = \xxNoOpt
 \fi
\next
}
\end{teX}



To build a command with any optional parameter, as you find in many of LaTeX's commands, you will need two things:

\begin{itemize}
\item a macro with delimited parameters

\item a way to grab the first non-space token that follows the command
\end{itemize}


The first part is fairly easy using delimited argument macros, for example we can say

\begin{verbatim}
\def\test(#1)#2#3{#1, #2, #3}
\end{verbatim}

We can then call this macro as:

\begin{verbatim}
\test(a){b}{c}
\end{verbatim}


resulting in a,b,c

To define the |()| as an optional parameter, we effectively need to define the macro as a conditional a sort of a "yes-no" switch. If \tex finds the "(" bracket the "yes-code" will be called and if it finds only the normal arguments the "no-code" will be executed.

For this we can use the |\@ifnextchar| macro from the LaTeX kernel.
You can say |@ifnextchar{char}{yes-code}{no-code}| to test for |(|. The result then will depend on the token that follows. If this token is the same as the first argument, then the "yes-code" is executed, otherwise the "no-code" is executed. The first argument should be a single token (for instance a character). Spaces are ignored. 

As for example we can redefine the LaTeX code for `rule` to accept an optional parameter in round brackets, rather than the traditional square brackets.

\begin{texexample}{Using ifnextchar}{}
\makeatletter
\def\Rule{\@ifnextchar(\@Rule%
        {\@Rule(\z@)}}
\def\@Rule(#1)#2#3{%
 \leavevmode
 \hbox{%
 \setlength\@tempdima{#1}%
 \setlength\@tempdimb{#2}%
 \setlength\@tempdimc{#3}%
 \advance\@tempdimc\@tempdima
 \vrule\@width\@tempdimb\@height\@tempdimc\@depth-\@tempdima}}
\makeatother

A test \Rule(6.5pt){100pt}{1pt}

Another test \Rule{100pt}{3pt}

Not that difficult but you will need to , but why on earth do you need this?
\end{texexample}




\begin{comment}
\def\elidebefore[#1]#2{[$\ldots$] #2}
\def\elideafter#1{#1$\ldots$}

\def\elide {%
\futurelet\ifoptions
    \choosemacro
}

% The \choosemacro, based on
% the lookahead of \elide, calls
% either \elidebefore or \elideafter 


\def\choosemacro{%
 \ifx\ifoptions [%
     \let\choice = \elidebefore 
 \else
    \let\choice = \elideafter
 \fi
\choice
}

Testing \elide[b]{Lorem ipsum}

\elide{Lorem Ipsum}

\elide[b]{Lorem ipsum}

\end{comment}


\section{Using LaTeX \protect\textbackslash @ifnextchar}

\latex defines the |\@ifnextchar| kernel command that is used effectively to
determine the token that follows the command. It is used in the definitions
of macros with optional arguments amongst other things.

\begin{teXXX}
\@ifnextchar]{true}{false}] 
\@ifnextchar[{true}{false}[
\end{teXXX}
The result would both be true,

\begin{texexample}{Example ifnextchar}{ifnextchar}
\makeatletter
\@ifnextchar]{true ]}{false} ] %notice ]
\@ifnextchar[{true [}{false} [ %notice [
\makeatother
\end{texexample}






















% ^^A\chapter{Iteration}
\precis{A discussion as to how to program simple for loops, in
TeX and LaTeX.}

\section{\TeX's simple \protect\texttt{loop}}

\newthought{Knuth in the TeXBook} provided a simple loop macro that can be used for iteration. It must be pointed out that there are no real looping structures in \tex other than pure recursion (including tail recursion). All looping mechanisms are build on top of these.

\begin{docCommand}{loop}{\meta{body}\docAuxCommand*{repeat}}
The |\loop...\repeat| construction is defined in Plain TeX and works like this:
You say `|\loop| $\alpha$ |\if|\dots $\beta$  |\repeat|', where $\alpha$ and $\beta$ are any sequences of
commands, and where |\if...| is any conditional test (without a matching |\fi|). 
Note that the |repeat| is just a marker in the argument specification of the macro. It can in essence be anything.
\end{docCommand}

\tex
will first do $\alpha$; then if the condition is true, \tex will do $\beta$ and repeat the whole process
again starting with $\alpha$. If the condition ever turns out to be false, the loop will stop.


The \cmd{\loop} macro that does all these wonderful things is actually quite simple.
It puts the code that's supposed to be repeated into a control sequence called
\cmd{\body}, and then another control sequence iterates until the condition is false:

\begin{teXXX}
\def\loop#1\repeat{\def\body{#1}\iterate}
\def\iterate{\body\let\next=\iterate\else\let\next=\relax\fi\next}
\end{teXXX}


\begin{texexample}{loop...repeat}{ex:loop}
\newcount\n
\n=0
\loop
  \advance\n by1
    \texttt{\number\n, } 
  \ifnum\n<30
\repeat
\end{texexample}

Just observe that the |\loop| arguments are delimited by |\repeat|. We could as well named it |\endloop| (a repeat at the end of a loop somehow sounds wrong!)


\begin{texexample}{Rename loop}{ex:renloop}
\bgroup
\def\for#1\endfor{\def\body{#1}\iterate}
\def\iterate{\body\let\next=\iterate\else\let\next=\relax\fi\next}
\newcount\n
\n=0
% Example usage
\for
   \advance\n by1
     \texttt{\number\n, }  
   \ifnum\n<30
\endfor
\egroup
\end{texexample}  


\subsection{Breaking out of a loop}

\index{Iteration!break}\index{\protect\textbackslash break}
Although the loop macros are fairly simple, breaking out of them or using conditionals needs some work.

\begin{texexample}{Iteration}{ex:loop}
\newcount\mycount
\mycount=0
\loop\ifnum\mycount<13
\the\mycount, 
\ifnum\mycount>5
    \let\iterate\relax
 \fi
 \advance\mycount by1\relax
\repeat
\end{texexample}

We can define a command \docAuxCommand{break}, so as to have better semantics and make the code more readable:

\emphasize{break,loop,repeat,}

\begin{texexample}{Iteration}{ex:loop1}
\def\break{\let\iterate\relax}% (*@ \dcircle{1} @*)
\newcount\mycount
\mycount=1
\loop
  \ifnum\mycount<13 % (*@ \dcircle{2} @*)
    %\the\mycount, 
    \ifnum\mycount>5
    % we break here
    \break %     
   \fi
   \the\mycount,\space% (*@ \dcircle{3} @*)
   \advance\mycount by1\relax
\repeat
\end{texexample}


Once we define what a |break| is supposed to do at \dcircle{1}, we use it at \dcircle{2} to let |\iterate| to |relax|. Then at \dcircle{3}, we use the value of the counter |\the\mycount| to add the number and a comma followed by a space. 


\section{Iteration over comma delimited lists}
\index{iteration>comma delimited lists}

The comma delimited list is one of the most common programming datastructure. A list is simply defined using a macro:

\begin{verbatim}
\def\mylist{John,Mary,Mathew,George,Maria}
\end{verbatim}

Although, lists can be defined as shown in the |\mylist| macro, this is not very useful. In most cases the \textit{elements} of the list would be added programmatically. Such list are used for example by \latex to keep track of input files.

Unlike many other programming languages, lists can be delimited by any character or even macros. Many package authors use a semicolon. This a perfectly legal in \tex.

\begin{teX}
\mylist{John;Mary;Mathew;George;Maria}
\end{teX}
as well as this:

\begin{teX}
\mylist{\@elt John\@elt Mary\@elt Mathew \@elt George \@elt Maria}
\end{teX}

\begin{docCommand}{@elt}{}
Using a macro to delimit the list elements, has the advantage that when we invoke the |mylist| list macro the |@elt| macro can map a function over the elements. We will see that a bit later in more detail but for the time being we will demonstrate this with an example:
\end{docCommand}

\begin{texexample}{Elt Lists}{}
\def\mylist{\@elt John,\@elt Mary,\@elt Mathew, \@elt George, \@elt Maria,}
\def\@elt#1,{\textit{#1} }
\mylist
\end{texexample}

Note that the macro |\@elt| when it is defined is delimited with a comma. 

\newthought{Adding Elements}

\begin{docCommand}{g@addto@macro}{}
There are many ways to add an element to the list, but perhaps the easiest is to use the \latex \cmd{\g@addto@macro}. 
\end{docCommand}

\emphasis{g@addto@macro}
\begin{teXXX}
\g@addto@macro{\mylist}{\@elt Thomas,}
\mylist
\end{teXXX}

One disadvantage of this approach is that the last item on the list will have a comma. A better approach would be to check if
the list is empty and to insert an elemen

\begin{texexample}{Lists}{}
\makeatletter
\def\mylist{Yiannis}
\def\emptylist{}

\def\addtomylist#1{%
\if\mylist\emptylist
   \g@addto@macro{\mylist}{#1,}
\else
   \g@addto@macro{\mylist}{,#1}
\fi
}
\addtomylist{George}
\addtomylist{Maria}
\addtomylist{Athena}

\mylist
\makeatother
\end{texexample}
 



\subsection{How to Use LaTeX’s kernel looping constructs}

\begin{docCommand}{@for}{}
The \cmd{\@for} is an internal \latexe command that can be used to iterate over a comma delimited list.
\end{docCommand}

\emphasis{mylist}
\begin{teX}
\makeatletter
\def\mylist{1,2,3,4,5}(*@\label{list}@*)
\@for\val:=\mylist\do{\val
\ifx\@xfor@nextelement\@nnil \else ;\fi}
\makeatother
\end{teX}


\latex's  low-level programming is rather poorly documented and the section on what is called control commands is even more so. The current \latex team are trying to provide some proper looping structures in \latex3. 

If you want to loop over comma-lists, \latex provides the \cmd{\@for} macro. This works by repeatedly assigning list items to a temporary variable:

To use it we need to define a list:

\startlineat{50}
\begin{teX}
\def\mathList{\alpha,\beta,\gamma,
          \delta,\epsilon,\zeta,\theta }
\end{teX}



Using the \refCom{@for} loop we can iterate over the list as follows:

\begin{texexample}{For constructs}{ex:forloop}
\makeatletter
\def\mathList{\alpha,\beta,\gamma,\delta,\epsilon,\zeta,\theta}
\@for\i:=\mathList\do{%
  \ensuremath\i\space 
 }
\makeatother 
\end{texexample}



Running the example we simply get the list but now without the comma

\begin{teX}
\makeatletter
\def\mathList{\alpha, \beta, \gamma, \delta, \epsilon, \zeta, \theta }
\@for\i:=\mathList\do{%
  \ensuremath \i  \space 
 }
\makeatother
\end{teX}




\begin{teX}
\makeatletter
\def\atestiii{}
\def\alist{a,b,c,d,v,e,f,g,h}
Test 1 \@removeelement{v}{a,b,c,d,v,e,f,g,h}{\atestiii} 
returns \atestiii
\alist
\gdef\blist{1,2,3,4,5,v,6,7}%
Test 2 \@removeelement{v}{\expand\blist}{\atestii} prints \atestii
\meaning\atestii
\meaning\@removeelement

\def\remove#1#2{
 \@removeelement #2{#1}\atestiii \atestiii
}

removes an element \atestiii ~~and \alist

\remove c\alist 


the variable holding the list \atestiii
\end{teX}


The iteration does not have the proper meaning that you would normally expect in other programming languages, it is defined as follows:


\begin{teXXX}
\@for(*@\textsubscript{all~elements of the list to }@*)\i:=\mathList\do{%
  \ensuremath \i  \space 
 }
\end{teXXX}

The other interesting thing to note as well as watch out is `:=', which is just a delimiter. I have used |\i| for simplicity, but \cmd{\i}, is a reserved word, meaning a \textit{dotless} \i, which is found in some language like Turkish. If you going to use it in your writings you will need to save it and restore it afterwards. A lot of macro writers also use |\ii| or |\@i| or other similar variables. It is simply a temporary variable that at the end of the iteration gets the value |\@nil| and since |@nil| is undefined it essentially destroys it. 

We need to remind ourselves again about |##|
20.5. The |##| feature is indispensable when the replacement text of a definition
contains other definitions. For example, consider


\begin{teX}
\def\a#1{\def\b##1{##1#1}}
after which `\a!' will expand to `\def\b#1{#1!}'. We will see later that ## is also
important for alignments; see, for example, the definition of \matrix in Appendix B.
\end{teX}

\begin{teX}
\long\def\@for#1:=#2\do#3{%
\expandafter\def\expandafter\@fortmp\expandafter{#2}%
\ifx\@fortmp\@empty \else
\expandafter\@forloop#2 ;\@nil;\@nil\@@#1{#3}\fi}

\long\def\@iforloop#1;#2\@@#3#4{\def#3{#1}\ifx #3\@nnil
\expandafter\@fornoop \else
#4\relax\expandafter\@iforloop\fi#2\@@#3{#4}}


\@for\i:=\mathList\do{%
  \ensuremath \i --  
 }


\meaning\loop 
\def\a#1{\textcolor{blue}{\uppercase{#1}}}
\def\b{test}
\expandafter\a\b

\a\b
\end{teX}


\section{Recursion}
\epigraph{“I love you,” said Bekka.

“I know,” I said.

“I know you know,” she said. “But I didn’t know that you knew I knew you knew. And now I do.” }
{Scott Alexander, It Was You Who Made My Blue Eyes Blue}
Recursion is a difficult subject to grasp, although we experience it daily in our actions, language and thoughts. 
The main characteristic of recursion, is that it can take its own output as the next input, a loop that can be extended indefinitely to create sequences of structures of unbounded length or complexity. In language we understand that a sentence can in principle be extended indefinitely, even though in practice it cannot be---although the novelist Henry James had a damn good try in the \emph{The Figure in the Carpet}. Of course what we are interested here is to study how we can write recursive macros in \tex rather than the more interesting aspects of recursion as it applies to thoughts and language. 

When we write:

\begin{verbatim}
\def\mymacro{\mymacro}
\end{verbatim}

The macro will expand itself indefinitely. As it does not save anything in memory it does not exceed the capacity of any data structure, it will just cause your computer to hang.

If we slightly modify the above macro to |\def\mymacro{a \mymacro}| during expansion the macro will typeset and then call itself again, typeset a and call itself again forever. After a while, it overflows the computer’s memory. The reason for this is that we never finish a paragraph. The letters accumulate in main memory as part of the same paragraph.


\begin{texexample}{Parsing lists}{ex:parselist}
\bgroup
\def\parselist#1;{\pickup#1,;,}
\def\pickup#1,{% Note that #1 may be \null
\if;#1
  \let\next=\relax
\else\let\next=\pickup
   #1% use #1 in any way
\fi\next}
\parselist $a_1$, $a_2$, $a_3$, $a_4$, $a_5$; 
\egroup
\end{texexample}

The macro \cmd{\pickup} expects its argument to be delimited by `,’, so it ends up
getting the first component of the original argument. It uses it in any desired way and
then expands itself recursively. The process ends when the current argument becomes the `;’. The compound argument may have any number of components (even zero).\ref{test}

\emphasize{makebox,obeylines}
\begin{texexample}{Longer Example}{ex:mheadings}
\makebox[\linewidth]{\hfill
\begin{minipage}{.8\textwidth}
\columnseprule2pt
\def\columnseprulecolor{\color{thegray}}
\columnsep22pt
\begin{multicols}{2}
\color{theblue}
\flushright
\Large
\obeylines
Over the last year we
have continued to
develop and improve the
range of funding schemes
we offer to meet the
needs of the arts and
humanities communities,
for example, by offering
opportunities for early
career researchers.
\columnbreak
\color{thegray}

\small
\flushleft

\obeylines %(*@\textcolor{blue}{\dcircle{1}}  @*)
\arial
We have engaged both
individuals and groups to
build a vision for our strategic
initiatives and our museums
and galleries strategy, have
opened up opportunities
for the arts and humanities
in cross-Council funding
initiatives and undertaken
to represent the needs of our
communities in arenas such
as the Research Councils’
project on the Efficiency and
Effectiveness of Peer Review
Journals
initiatives and undertaken
to represent the needs of our
communities in arenas such
as the Research Councils’
project on the Efficiency and
Effectiveness of Peer Review
Journals
\end{multicols} %(*@\textcolor{blue}{\dcircle{2}}  @*)
\end{minipage} %(*@\textcolor{blue}{\dcircle{3}}  @*)
}
\end{texexample}


Obviously this does not make for a good user interface. It will be preferable to have just one or two commands and the user should be able to type in the left and right, text. All setting will be preferable to be done via keys, which map to macros. 

%%endinput iteration.tex










% ^^A\newtcolorbox{scriptexample}[2][shavian]{colback=graphicbackground,
boxrule=0pt,toprule=0pt,colframe=white}


\chapter{Those Other Languages}
\minitoc
\parindent1em

\pagestyle{myheadings}

Probably there are more users of \latexe whose mother tongue is not English than those who speak the language. \tex out of the box does not offer facilities for using non-latin based scripts easily; presents numerous problems. The biggest problem---which has been solved to a large extent---was the entering of text without having to mark all the special
characters such as umlauts (\"o) with commands. The second issue and which has been addressed by packages such as Babel, is redefining the strings such as "Chapter" to another language. In software this is called internationalization and a governing standard is |i18n|. None of the current packages take such an approach and none of them as yet offer a satisfactory solution for |LuaLaTeX|. 

Another issue with writing systems and scripts is that of appropriate fonts. Most writing systems that have ever existed are now extinct. Only minute vestiges of one of the most ancient - Egyptian hieroglyphs - live on, unrecognized, in the Latin alphabet in which English, among hundreds of other languages, is conveyed today. The latin \textit{m}, for example, ultimately derives from the Egyptian's cononantal n-sign, depicting waves.

Many of the scripts have other peculiarities, some languages such as Hanunó'o is written vertically from bottom to top. Others from top to bottom and many others from right to left. 

\section{TeX's support for different languages}

TeX's primitives such as \cmd{\language}=\meta{number} can be used to store hyphenation patterns and exceptions for up to 256 different languages. This primitive is then used by TeX to apply an appropriate set of hyphenation rules for each paragraph or part of a paragraph in a document\footnote{\url{http://www.tug.org/utilities/plain/cseq.html language-rp}}. When TeX begins a ne paragraph it sets the \emph{current language} to \cmd{\language}. Just before it adds each new character to the paragraph in unrestricted horizontal mode, it compares the current language to \cmd{\language}. If they are different, TeX : a) changes the current language to \cmd{\language}; b) inserts a whatsit\index{whatsit>language} containing the new language and the values of |\lefthyphenmin| and |\righthyphenmin|; and c) inserts the character. The |\setlanguage| command should be used to change languages in restricted horizontal mode (i.e., inside an |\hbox|). If \meta{number} is less than 0 or greater than 255, 0 is used [455]. Plain TeX has a |\newlanguage| command which may be used to allocate numbers for languages [347]. Changes made to |\language| are local to the group containing the change 

\section{LaTeX}

As far as hyphenation patterns are concerned \latexe follows very closely to the methods employed by \tex and Plain Tex. In the source2e the File |lthyphen.dtx| describes the approach to loading the default file |hyphen.ltx| . If a file hyphen.cfg is found \latexe will load the appropriate hyphenaion patterns. Traditionally language management was achieved via Johan 
Braams package Babel which we describe in the next section.


\section{The Babel Package} 

Babel \citet{babel} was the first package to systematically offer foreign language
support for \tex. It has been updated for use with |XeTeX| and |LuaTeX| and provides an environment
in which documents can be typeset in a language
other than US English, or in more than one language
or script. However, no attempt has been done to
take full advantage of the features provided by the
latter, which would require a completely new core
(as for example polyglossia or as part of \latex3).

The package has a number of predefined language files with the extension |ldf|. 


\Describe\selectlanguage{\marg{language}}{}
When a user wants to switch from one language to another he can
do so using the macro |\selectlanguage|. This macro takes the
language, defined previously by a language definition file, as
its argument. It calls several macros that should be defined in
the language definition files to activate the special definitions
for the language chosen. For ``historical reasons'', a macro name is
converted to a language name without the leading |\|; in other words,
the two following declarations are equivalent:
\begin{verbatim}
\selectlanguage{german}
\selectlanguage{\german}
\end{verbatim}

\Describe\foreignlanguage{\marg{language}\marg{text}}
The command |\foreignlanguage| takes two arguments; the second
argument is a phrase to be typeset according to the rules of the
language named in its first argument. This command (1) only
switches the extra definitions and the hyphenation rules for the
language, \emph{not} the names and dates, (2) does not send
information about the language to auxiliary files (i.e., the
surrounding language is still in force), and (3) it works even if
the language has not been set as package option (but in such a
case it only sets the hyphenation patterns and a warning is shown).

\Describe{otherlanguage*}%
{\marg{language}{otherlanguage*}}

Same as |\foreignlanguage| but as environment. Spaces after the
environment are \textit{not} ignored.



\section{The Polyglossia package}

The \pkgname{polyglossia} package has a lot of potential and has solved many issues
but its integration with large parts of the traditional |pdfLaTeX| world
is still under development and will probably take a while before one could
declare it easy to use and bug free. For example anything with the |bidi| package has issues with loading orders for a number of packages and least of which is with
the Ams packages. So if you are going to mix a number of languages in a \XeTeX\ document
you need to take extra care.

 Polyglossia is a package for facilitating multilingual typesetting with
 \XeLaTeX\ and (at an early stage) \LuaLaTeX.  Basically, it
 can be used as a replacement of \pkg{babel} for performing the following
 tasks automatically:
 
 \begin{enumerate}
 \item Loading the appropriate hyphenation patterns.
 \item Setting the script and language tags of the current font (if possible and
       available), via the package \pkg{fontspec}.
 \item Switching to a font assigned by the user to a particular script or language.
 \item Adjusting some typographical conventions according to the current language
       (such as afterindent, frenchindent, spaces before or after punctuation marks,
       etc.).
 \item Redefining all document strings (like chapter, “figure”, “bibliography”).
 \item Adapting the formatting of dates (for non-Gregorian calendars via external
       packages bundled with polyglossia: currently the Hebrew, Islamic and Farsi
       calendars are supported).
 \item For languages that have their own numbering system, modifying the formatting
       of numbers appropriately (this also includes redefining the alphabetic sequence
       for non-Latin alphabets).\footnote{ %
         For the Arabic script this is now done by the bundled package \pkg{arabicnumbers}.}
 \item Ensuring proper directionality if the document contains languages
       that are written from right to left (via the package \pkg{bidi},
       available separately).
 \end{enumerate}
 
 Several features of \pkg{babel} that do not make sense in the \XeTeX\ world (like font
 encodings, shorthands, etc.) are not supported.
 Generally speaking, \pkg{polyglossia} aims to remain as compatible as possible
 with the fundamental features of \pkg{babel} while being cleaner, light-weight,
 and modern. The package \pkg{antomega} has been very beneficial in our attempt to
 reach this objective.


\section{Loading language definition files}

The recommended way of \pkg{polyglossia} to load language definition files
is given in the manual as:
 
\Describe{\setdefaultlanguage}{\oarg{options}\marg{lang}}
 (or equivalently \cmd\setmainlanguage).
 Secondary languages can be loaded with

\Describe{\setotherlanguage}{\oarg{options}\marg{lang}}
 These commands have the advantage of being explicit and of allowing you to set
 language-specific options.\footnote{ %
 More on language-specific options below.}
 It is also possible to load a series of secondary languages at once using

\Describe\setotherlanguages{\marg{lang1,lang2,lang3,\ldots}}

 Language-specific options can be set or changed at any time by means of
\Describe\setkeys{\marg{lang}\marg{opt1=value1,opt2=value2,\ldots}}

\subsection{Bidirectional languages}





\begin{comment}
\begin{Arabic}
ّ هو إذ الغاية؛ شريف الفوائد، جم المذهب، عزيز فنّ التاريخ فنّ أنّ اعلم
والملوك سيرهم، في والأنبياء أخلاقهم، في الأمم من الماضين أحوال على يوقفنا
ّ أحوال في يرومه لمن ذلك في الإقتداء فائدة تتم حتّى وسياستهم؛ دولهم في
والدنيا. الدين
\end{Arabic}
\end{comment}

The Greek language is represented both in modern Greek as well as its ancient variants.

\begin{verbatim}
\begin{greek}
\textbf{Η ελληνική γλώσσα} είναι μία από τις ινδοευρωπαϊκές γλώσσες, για την
οποία έχουμε γραπτά κείμενα από τον 15ο αιώνα π.Χ. μέχρι σήμερα. Αποτελεί το
μοναδικό μέλος ενός κλάδου της ινδοευρωπαϊκής οικογένειας γλωσσών. Ανήκει
επίσης στον βαλκανικό γλωσσικό δεσμό.\\	
(\today) 
\end{greek}
\end{verbatim}

\topline

\textbf{Η ελληνική γλώσσα} είναι μία από τις ινδοευρωπαϊκές γλώσσες, για την
οποία έχουμε γραπτά κείμενα από τον 15ο αιώνα π.Χ. μέχρι σήμερα. Αποτελεί το
μοναδικό μέλος ενός κλάδου της ινδοευρωπαϊκής οικογένειας γλωσσών. Ανήκει
επίσης στον βαλκανικό γλωσσικό δεσμό.\\	
(\today) 

\bottomline

\begin{verbatim}
\begin{russian}
\textbf{Русский язык} — один из восточнославянских языков, один из 
крупнейших языков мира, в том числе самый распространённый из славянских
языков и самый распространённый язык Европы, как географически, так и по
числу носителей языка как родного (хотя значительная, и географически бо́
льшая, часть русского языкового ареала находится в Азии).	\\
(\today)
\end{russian}
\end{verbatim}



\textbf{Русский язык} — один из восточнославянских языков, один из крупнейших языков мира, в том числе самый распространённый из славянских языков и самый распространённый язык Европы, как географически, так и по числу носителей языка как родного (хотя значительная, и географически бо́льшая, часть русского языкового ареала находится в Азии).	\\
(\today)


\section{The Translator package}

The \pkgname{translator} package was developed by \person{Till Tantau} \citep{translator}. It provides a flexible
mechanism for translating individual words into different languages.
For example, it can be used to translate a word like ``figure'' into,
say, the German word ``Abbildung''. Such a translation mechanism is
useful when the author of some package would like to localize the
package such that texts are correctly translated into the language
preferred by the user. The translator package is \emph{not} intended
to be used to automatically translate more than a few words. 

You may wonder whether the translator package is really necessary
since there is the (very nice) |babel| package available for
\LaTeX. This package already provides translations for words like
``figure''. Unfortunately, the architecture of the babel package was
designed in such a way that there is no way of adding translations of
new words to the (very short) list of translations directly build into
babel.

The translator package was specifically designed to allow an easy
extension of the vocabulary. It is both possible to add new words that
should be translated and translations of these words.

\subsection{Using the Translator Package}

  The \pkg{Translator} needs to be used with Babel and I am not too sure yet 
  if it is ready  to be used with Polyglossia.

Once the package has loaded a language or a set of languages the optional argument to the
\cmd{\translate} can be used to translate a string. 

\begin{texexample}{Translating strings}{ex:translator}
  \translate[to=german]{rightpagename}
  \translate[to=dutch]{rightpagename}
\end{texexample}

Before you can provide the translations you need to provide your own dictionaries, where you require them. These need to be installed at a place where \tex can find them.

\CMDI{\ProvidesDictionary}

The dictionary has to be saved in a specific format that relates to the \cmd{\ProvidesDictionary} command. The second argument of the command must be appended to the file name; for the example the file is saved as\footnote{This  example is from the translator package bundle and is under the folder \texttt{base}}:

|translator-basic-dictionary-German|

The concepts take a bit of time to sink in, but once you have everything set up, it is quite easy and straight forward to incorporate it, into your package. 

\begin{teXXX}
\ProvidesDictionary{translator-basic-dictionary}{German}

\providetranslation{Abstract}{Zusammenfassung}
\providetranslation{Addresses}{Adressen}
\providetranslation{addresses}{Adressen}
\providetranslation{Address}{Adresse}
\providetranslation{address}{Adresse}
\providetranslation{and}{und}
\providetranslation{Appendix}{Anhang}
\providetranslation{Authors}{Autoren}
\providetranslation{authors}{Autoren}
\providetranslation{Author}{Autor}
\providetranslation{author}{Autor}
\end{teXXX} 

This is in contrast to Babel and Polyglossia that define
commands for each string to be translated such as,

\begin{teXXX}
\def\captionsdutch{%
    \def\prefacename{Voorwoord}%
    \def\refname{Referenties}%
    \def\abstractname{Samenvatting}%
    \def\bibname{Bibliografie}%
    \def\chaptername{Hoofdstuk}%
    \def\appendixname{Bijlage}%
    \def\contentsname{Inhoudsopgave}%
    \def\listfigurename{Lijst van figuren}%
    \def\listtablename{Lijst van tabellen}%
    \def\indexname{Index}%
    \def\figurename{Figuur}%
    \def\tablename{Tabel}%
    \def\partname{Deel}%
    \def\enclname{Bijlage(n)}%
    \def\ccname{cc}%
    \def\headtoname{Aan}%
    \def\pagename{Pagina}%
    \def\seename{zie}%
    \def\alsoname{zie ook}%
    \def\proofname{Bewijs}%
    \def\glossaryname{Verklarende woordenlijst}%
    \def\today{\number\day~\ifcase\month%
      \or januari\or februari\or maart\or april\or mei\or juni\or
      juli\or augustus\or september\or oktober\or november\or
      december\fi
      \space \number\year}}
\end{teXXX}

\begin{macro}{\usedictionary}\marg{kind}
  This command tells the |translator| package, that at the beginning of
  the document it should load \textit{all} dictionaries of kind \meta{kind} for
  the languages used in the document. Note that the dictionaries are
  not loaded immediately, but only at the beginning of the document.

  If no dictionary of the given \emph{kind} exists for one of the
  language, nothing bad happens.

  Invocations of this command accumulate, that is, you can call it
  multiple times for different dictionaries.
\end{macro}

\Describe{\uselanguage}{\marg{list of languages}}
  This command tells the |translator| package that it should load the
  dictionaries for all languages in the \meta{list of languages}. The
  dictionaries are loaded at the beginning of the document.

\section{Fonts for All the World Scripts}

Many commercial as well as open source fonts exist that can be used to typeset text the world's scripts and languages. The aim of this section of the documentation is to present an overview of the most common scripts represented in the Unicode~7.0 standard. All the examples require the use of the \XeTeX\ engine. In addition you need to have a copy of the font on your own system. If you do not have them, the font loading mechanism of \XeTeX\ will take some time to search all the directories and slows compilation tremendously. 




\section{Pan-Unicode Fonts}

Thousands of fonts exist on the market, but fewer than a dozen fonts—sometimes described as "pan-Unicode" fonts—attempt to support the majority of Unicode's character repertoire. Instead, Unicode-based fonts typically focus on supporting only basic ASCII and particular scripts or sets of characters or symbols. Several reasons justify this approach: applications and documents rarely need to render characters from more than one or two writing systems; fonts tend to demand resources in computing environments; and operating systems and applications show increasing intelligence in regard to obtaining glyph information from separate font files as needed, i.e. font substitution. Furthermore, designing a consistent set of rendering instructions for tens of thousands of glyphs constitutes a monumental task; such a venture passes the point of diminishing returns for most typefaces.

The \texttt{NotSerif} font from Google\footnote{\protect\url{http://www.google.com/get/noto/}} has good support for many languages.

Another freeware pan-Unicode font is Titus\footnote{\protect\url{http://titus.fkidg1.uni-frankfurt.de/unicode/tituut.asp?Inp1=A&Inp2=B&Inp3=C&Inp4=d%40e.com&Inp6=0&Inp5=1}}
This is an extended version of this font is TITUS Cyberbit Unicode, includes 36,161 characters in v4.0.

\newfontfamily\titus[Scale=1.05]{TITUSCBZ.ttf}
\newfontfamily\noto{NotoSerif-Regular.ttf}

\begin{scriptexample}[]{Titus}
\titus

\lorem
\end{scriptexample}
\bigskip

\begin{scriptexample}[]{Noto}
\noto

\lorem
\end{scriptexample}


\section{The \texttt{ucharclasses} package}

For multilingual texts font switching can become cumbersome. The use of a pan-Unicode font as the default can help. However, if the languages are distinct enough to use different Unicode blocks, which are not covered by the \pkgname{polyglossia} package Mike Kamermans' package \pkgname{ucharclasses} can be used.

\begin{verbatim}
% and the font switching magic
\usepackage[CJK, Latin, Thai, Sinhala, Malayalam, DominoTiles, MahjongTiles]{ucharclasses}
\usepackage{fontspec}

% default transition uses the widest coverage font I know of
\setDefaultTransitions{\fontspec{Code2000.ttf}}{}

% overrides on the default rules for specific informal groups
\setTransitionsForLatin{\fontspec{Palatino Linotype}}{}
\setTransitionsForCJK{\fontspec{code2000.ttf}}{}%HAN NOM A
\setTransitionsForJapanese{\fontspec{code2000.ttf}}{}%Ume Mincho

% overrides on the default rules for specific unicode blocks
\setTransitionTo{CJKUnifiedIdeographsExtensionB}{\fontspec{SimSun-ExtB}}
\setTransitionTo{Thai}{\fontspec{IrisUPC}}
\setTransitionTo{Sinhala}{\fontspec{Iskoola Pota}}
\setTransitionTo{Malayalam}{\fontspec{Arial Unicode MS}}

\end{verbatim}

\bgroup
\begin{verbatim}
domino tiles, 🁇 🀼 🁐 🁋 🁚 🁝, and mahjong tiles: 🀑 🀑 🀑 🀒 🀒 🀒 🀕 🀕 🀕 🀗 🀗 🀗 🀅 🀅 (using FreeFont)
\end{verbatim}

domino tiles, 🁇 🀼 🁐 🁋 🁚 🁝, and mahjong tiles: 🀑 🀑 🀑 🀒 🀒 🀒 🀕 🀕 🀕 🀗 🀗 🀗 🀅 🀅 (using FreeFont)
\egroup

\section{PhD Settings}

\def\test{}
\cxset{language/.code=\test}
\cxset{language=greek}
\cxset{languages/.code=\test}
\cxset{languages={english,greek,spanish,chinese}}
\cxset{greek font/.code=\test}
\cxset{greek font=code2000.ttf}

\begin{key}{/chapter/language=\meta{language name}}  
The key language sets the main language for the document. This language will be used for the sectioning commands and common string translations.

If the language is English Polyglossia or Babel are not loaded automatically. If the language is other than English we load either Babel or Polyglossia depending on the engine used.
\end{key}


\begin{key}{/chapter/languages=\meta{language1, language2, language3}}  
The key |languages|, determines all the other scripts available for typesetting. For each language default font commands are create automatically. The aim is to be able to run a fully multilingual system with the minimum of upfront settings. These we leave to customize in the style template files.
\end{key}

\begin{key}{/chapter/greek font=\meta{options}\meta{font file}}  
The package comes with numerous language and appropriate default fonts
for each operating system. 
\end{key}

\section{Ancient and Historic Scripts}

Unicode encodes a number of ancient scripts, which have not been in normal use for a millennium or more, as well as historic scripts, whose usage ended in recent centuries. Although these scripts are no longer used to write living languages, documents and inscriptions using these languages exist, both for extinct languages and for precursors of modern languages. The primary user communities for these scripts are scholars, interested in studying the scripts and the languages written in them. A few, such as Coptic, also have contemporary liturgical or other special purposes. Some of the historic scripts are related to each other as well as to modern alphabets. The following are provides as of Unicode version~6.2.

\begin{center}
\begin{tabular}{lll}
Ogham.     &Ancient Anatolian Alphabets. &Avestan.\\
Old Italic. &Old South Arabian. &Ugaritic\\
Runic &Phoenician. &Old Persian\\
Gothic &Imperial Aramaic &Sumero-Akkadian\\
Old Turkic. &Mandaic &Egyptian Hieroglyphs.\\
Linear B &Inscriptional Parthian &Meroitic.\\
Cypriot Syllabary &Inscriptional Pahlavi&\\
\end{tabular}
\end{center}

The following scripts are also encoded but following the Unicode
convention are described in other sections

\begin{center}
\begin{tabular}{llllll}
Coptic &Glagolithic &Phags-pa. &Kaithi &Kharoshi &Brahmi.\\
\end{tabular}
\end{center}


^^A\subsection{Ogham}

\newfontfamily\ogham{code2000.ttf}

Ogham was added to the Unicode Standard in September 1999 with the release of version 3.0.
The spelling of the names given is a standardisation dating to 1997, used in Unicode Standard and in Irish Standard 434:1999.
The Unicode block for ogham is \texttt{U+1680–U+169F}.

\begin{scriptexample}[]{Ogham}
\bgroup
\ogham
0	1	2	3	4	5	6	7	8	9	A	B	C	D	E	F\\
U+168x	   	ᚁ	ᚂ	ᚃ	ᚄ	ᚅ	ᚆ	ᚇ	ᚈ	ᚉ	ᚊ	ᚋ	ᚌ	ᚍ	ᚎ	ᚏ\\
U+169x	ᚐ	ᚑ	ᚒ	ᚓ	ᚔ	ᚕ	ᚖ	ᚗ	ᚘ	ᚙ	ᚚ	᚛	᚜	\\

\titus

0	1	2	3	4	5	6	7	8	9	A	B	C	D	E	F\\
U+168x	   	ᚁ	ᚂ	ᚃ	ᚄ	ᚅ	ᚆ	ᚇ	ᚈ	ᚉ	ᚊ	ᚋ	ᚌ	ᚍ	ᚎ	ᚏ\\
U+169x	ᚐ	ᚑ	ᚒ	ᚓ	ᚔ	ᚕ	ᚖ	ᚗ	ᚘ	ᚙ	ᚚ	᚛	᚜
\egroup		
\end{scriptexample}
^^A\section{Ancient Anatolian Alphabets}

The Anatolian scripts described in this section all date from the first millenium BCE, and were used to write various ancient Indo-European languages of western and southwestern Anatolia (now Turkey). All are related to the Greek script and are probably adaptations of it. 

\newfontfamily\lycian{Aegean.ttf}
\let\lydian\lycian
\let\carian\lydian

\begin{description}
\item [Lycian] The Lycian alphabet was used to write the Lycian language. It was an extension of the Greek alphabet, with half a dozen additional letters for sounds not found in Greek. It was largely similar to the Lydian and the Phrygian alphabets.
 
\bgroup
\lydian
\obeylines
0	1	2	3	4	5	6	7	8	9	A	B	C	D	E	F
U+1028x	𐊀	𐊁	𐊂	𐊃	𐊄	𐊅	𐊆	𐊇	𐊈	𐊉	𐊊	𐊋	𐊌	𐊍	𐊎	𐊏
U+1029x	𐊐	𐊑	𐊒	𐊓	𐊔	𐊕	𐊖	𐊗	𐊘	𐊙	𐊚	𐊛	𐊜

Typeset with the \idxfont{Aegean.ttf} and the command \cmd{\lydian}
\egroup

\item[Lydian] Lydian script was used to write the Lydian language. That the language preceded the script is indicated by names in Lydian, which must have existed before they were written. Like other scripts of Anatolia in the Iron Age, the Lydian alphabet is a modification of the East Greek alphabet, but it has unique features. The same Greek letters may not represent the same sounds in both languages or in any other Anatolian language (in some cases it may). Moreover, the Lydian script is alphabetic.
Early Lydian texts are written both from left to right and from right to left. Later texts are exclusively written from right to left. One text is boustrophedon. Spaces separate words except that one text uses dots. Lydian uniquely features a quotation mark in the shape of a right triangle.
The first codification was made by Roberto Gusmani in 1964 in a combined lexicon (vocabulary), grammar, and text collection.


\bgroup
\lycian
\obeylines
	0	1	2	3	4	5	6	7	8	9	A	B	C	D	E	F
U+1092x	𐤠	𐤡	𐤢	𐤣	𐤤	𐤥	𐤦	𐤧	𐤨	𐤩	𐤪	𐤫	𐤬	𐤭	𐤮	𐤯
U+1093x	𐤰	𐤱	𐤲	𐤳	𐤴	𐤵	𐤶	𐤷	𐤸	𐤹						𐤿
Typeset with the \idxfont{Aegean.ttf} and the command \cmd{\lycian}

Examples of words

𐤬𐤭𐤠  - Ora - "Month"

𐤬𐤳𐤦𐤭𐤲𐤬𐤩  - Laqrisa - "Wall"

𐤬𐤭𐤦𐤡  - "House, Home"

\egroup

\item [Carian] The Carian alphabets are a number of regional scripts used to write the Carian language of western Anatolia. They consisted of some 30 alphabetic letters, with several geographic variants in Caria and a homogeneous variant attested from the Nile delta, where Carian mercenaries fought for the Egyptian pharaohs. They were written left-to-right in Caria (apart from the Carian–Lydian city of Tralleis) and right-to-left in Egypt. Carian was deciphered primarily through Egyptian–Carian bilingual tomb inscriptions, starting with John Ray in 1981; previously only a few sound values and the alphabetic nature of the script had been demonstrated. The readings of Ray and subsequent scholars were largely confirmed with a Carian–Greek bilingual inscription discovered in Kaunos in 1996, which for the first time verified personal names, but the identification of many letters remains provisional and debated, and a few are wholly unknown.

\begin{scriptexample}[]{Carian}
\bgroup
\carian
\obeylines
 	0	1	2	3	4	5	6	7	8	9	A	B	C	D	E	F
U+102Ax	𐊠	𐊡	𐊢	𐊣	𐊤	𐊥	𐊦	𐊧	𐊨	𐊩	𐊪	𐊫	𐊬	𐊭	𐊮	𐊯
U+102Bx	𐊰	𐊱	𐊲	𐊳	𐊴	𐊵	𐊶	𐊷	𐊸	𐊹	𐊺	𐊻	𐊼	𐊽	𐊾	𐊿
U+102Cx	𐋀	𐋁	𐋂	𐋃	𐋄	𐋅	𐋆	𐋇	𐋈	𐋉	𐋊	𐋋	𐋌	𐋍	𐋎	𐋏
U+102Dx	𐋐
\egroup
\end{scriptexample}

\newfontfamily\oldpunctuation{code2000.ttf}

Word dividers are infrequent, \emph{scriptio continua}\footnote{a style of writing without word dividers, that is, without spaces or other marks between words or sentences} is common. Words dividers which are attested are U+00B7 (\char"00B7) \textsc{MIDLE DOT} (or U+2E31 word separator middle dot), U+205A TWO DOT PUNCTUATION, and U+205D TRICOLON ({\oldpunctuation\char"205D}). In modern editions U+0020 SPACE may be found.

\end{description}
^^A

\section{Avestan script}
\label{s:avestan}
The Avestan alphabet is a writing system developed during Iran's Sassanid era (AD 226–651) to render the Avestan language.
As a side effect of its development, the script was also used for Pazend, a method of writing Middle Persian that was used primarily for the Zend commentaries on the texts of the Avesta. In the texts of Zoroastrian tradition, the alphabet is referred to as \emph{din dabireh} or \emph{din dabiri}, Middle Persian for "the religion's script".

The Avestan alphabet was replaced by the Arabic alphabet after Persia converted to Islam during the 7th century CE. 


Notable Features

The alphabet is written from right to left, in the same way as Syriac, Arabic and Hebrew.
See more at: \url{http://www.iranchamber.com/scripts/avestan_alphabet.php#sthash.ZRu7AkEb.dpuf}

\newfontfamily\avestan{NotoSansAvestan-Regular.ttf}



\begin{scriptexample}[]{Avestan}
\ifxetex\TeXXeTstate=1
\beginR\fi
\avestan\raggedleft
𐬄	
𐬅	
𐬆	
𐬇	
𐬈	
𐬉	
𐬊	
𐬋	
𐬌	
𐬍	
𐬎	
𐬏	
𐬐	
	
𐬒	
𐬓	
𐬔	
	
𐬖	
𐬗	
𐬘	
𐬙	
𐬚	
𐬛	
𐬜	
𐬝	
𐬞	
𐬟	
𐬠	
𐬡	
𐬢	
𐬣	
𐬤	
𐬥	
𐬦	
𐬧	
𐬨	
𐬩	
𐬪	
𐬫	
𐬬	
𐬭	
𐬮	
𐬯	
𐬰	
𐬱	
𐬲	
𐬳	
𐬴	
𐬵	
\ifxetex\endR
\TeXXeTstate=0\fi
\end{scriptexample}

The recent Google font \url{NotoSansAvestan-Regular_0.ttf} can be used to typeset the Avestan script, but really not suitable for any serious study of the language.
^^A\subsection{Old Turkic}

\newfontfamily\oldturkic{Segoe UI Symbol}
\begin{scriptexample}[]{Old Turkish}
\oldturkic
\obeylines
Orkhon	Yenisei
variants	Transliteration / transcription
Old Turkic letter  𐰀	𐰁 𐰂	a, ä
Old Turkic letter  𐰃	𐰄 𐰅	y, i (e)
Old Turkic letter  𐰆		o, u
Old Turkic letter  𐰇	𐰈	ö, ü

	0	1	2	3	4	5	6	7	8	9	A	B	C	D	E	F
U+10C0x	𐰀	𐰁	𐰂	𐰃	𐰄	𐰅	𐰆	𐰇	𐰈	𐰉	𐰊	𐰋	𐰌	𐰍	𐰎	𐰏
U+10C1x	𐰐	𐰑	𐰒	𐰓	𐰔	𐰕	𐰖	𐰗	𐰘	𐰙	𐰚	𐰛	𐰜	𐰝	𐰞	𐰟
U+10C2x	𐰠	𐰡	𐰢	𐰣	𐰤	𐰥	𐰦	𐰧	𐰨	𐰩	𐰪	𐰫	𐰬	𐰭	𐰮	𐰯
U+10C3x	𐰰	𐰱	𐰲	𐰳	𐰴	𐰵	𐰶	𐰷	𐰸	𐰹	𐰺	𐰻	𐰼	𐰽	𐰾	𐰿
U+10C4x	𐱀	𐱁	𐱂	𐱃	𐱄	𐱅	𐱆	𐱇	𐱈	

\hfill  Typeset with \texttt{Segoe UI Symbol} \cmd{\oldturkic} 
\end{scriptexample}

Irk Bitig or Irq Bitig (Old Turkic: {\bfseries\Large\oldturkic 𐰃𐰺𐰴 𐰋𐰃𐱅𐰃𐰏}), known as the Book of Omens or Book of Divination in English, is a 9th-century manuscript book on divination that was discovered in the "Library Cave" of the Mogao Caves in Dunhuang, China, by Aurel Stein in 1907, and is now in the collection of the British Library in London, England. The book is written in Old Turkic using the Old Turkic script (also known as "Orkhon" or "Turkic runes"); it is the only known complete manuscript text written in the Old Turkic script.[1] It is also an important source for early Turkic mythology.

The Old Turkic text does not have any sentence punctuation, but uses two black lines in a red circle as a word separation mark in order to indicate word boundaries as shown in Figure~{\ref{omen}}

\begin{figure}[htb]
\includegraphics[width=0.7\textwidth]{./images/omen.jpg}
\caption{Omen 11 (4-4-3 dice) of the Irk Bitig (folio 13a): "There comes a messenger on a yellow horse (and) an envoy on a dark brown horse, bringing good tidings, it says. Know thus: (The omen) is extremely good."}
\label{omen}
\end{figure}
^^A\section{Phoenician}
\label{s:phoenician}
\arial

The Phoenician alphabet and its successors were widely used over a broad area surrounding the Mediterranean Sea.

\let\phoenician\lycian

\begin{scriptexample}[]{Phoenician}

\unicodetable{phoenician}{"10900,"10910}

\end{scriptexample}

The Phoenician alphabet, called by convention the Proto-Canaanite alphabet for inscriptions older than around 1200 BCE, is the oldest verified consonantal alphabet, or abjad.[1] It was used for the writing of Phoenician, a Northern Semitic language, used by the civilization of Phoenicia. It is classified as an abjad because it records only consonantal sounds (matres lectionis were used for some vowels in certain late varieties).

Phoenician became one of the most widely used writing systems, spread by Phoenician merchants across the Mediterranean world, where it evolved and was assimilated by many other cultures. The Aramaic alphabet, a modified form of Phoenician, was the ancestor of modern Arabic script, while Hebrew script is a stylistic variant of the Aramaic script. The Greek alphabet (and by extension its descendants such as the Latin, the Cyrillic, and the Coptic) was a direct successor of Phoenician, though certain letter values were changed to represent vowels.

\begin{figure}[ht]
\includegraphics[width=\textwidth]{./images/phoenician.jpg}
\captionof{figure}{
Phoenician votive inscription from Idalion (Cyprus), 390 BC. BM 125315 from The Early Alphabet by John F. Healy.}
\end{figure}

As the letters were originally incised with a stylus, most of the shapes are angular and straight, although more cursive versions are increasingly attested in later times, culminating in the Neo-Punic alphabet of Roman-era North Africa. Phoenician was usually written from right to left, although there are some texts written in boustrophedon.


\printunicodeblock{./languages/phoenician.txt}{\phoenician}


\newpage
\section{Palmyrene}
\idxlanguage{Palmyrene}
\arial

Palmyrene is the very widely attested Aramaic dialect and script
of Palmyra in the Syrian desert. The texts date from the midfirst century to the destruction of Palmyra by the Romans in AD 272. Palmyra in the Roman period was a major trading centre.
\medskip

\begin{figure}[ht]
\centering

\includegraphics[width=0.9\textwidth]{./images/palmyrene.jpg}
\captionof{figure}{\protect\arial Limestone bust with Palmyrene inscription. Palmyra late 2nd century AD. BM WA 102612}

\end{figure}

\medskip
The longest of the Palmyrene texts, is the bilingual  taxation tariff written for the year 137 AD in Palmyrene Aramaic and Greek.\footnote{For more details see:MILIK J.T., Dédicaces faites par des dieux (Palmyre, Hatra, 
Tyr) et de thiases sémitiques à l'époque romaine, Paris 1972; ROSENTHAL R., Die 
Sprache der palmyrenischen Inschriften, Leipzig 1936; STARK J.K., Personal Names in 
Palmyrene Inscriptions, Oxford 1971; DRIJVERS H.J.W., The Religion of Palmyra, 
Leiden 1976; TEIXIDOR J., 'Palmyre et son commerce d'Auguste à Caracalla', in 
Semitica 34, (1984) 1-127.  } Trade connections 
took the Palmyrene script to other places, some not far away, such as Dura Europos on the Euphrates, butothers at a great distance. A particular inscription is from South Shields, Roman Arbeia, in the north-east of England, carved on behalf of a Palmyrene mechant for his deceased wife and probably dating to the early third century AD. 

The Palmyrene script existed in two main varieties, a monumental and a cursive one, though the latter is little known and the evidence  mostly from Palmyra itself. The Syriac script of Edessa in southern Turkey, is often regarded as derived or closely related to the Palmyrene---similarities are found in the letters: ', b, g, d, w, h, y, k, l, m, n, `, r and t---though a strong case can also be made for connecting Syriac with a northern Mesopotamian script-family represented principally in texts from Hatra, a city more or less contemporary with Palmyra in Upper Mesopotamia. 


\begin{figure}[ht]
\includegraphics[width=\textwidth]{./images/regina-epigraph.jpg}
\caption{It was customary for Palmyrenes to offer bilingual texts (Greek or Latin) on funerary monuments. The final line of Regina's epitaph is Barates' personal lament in Palmyrene: Regina, freedwoman of Barate, alas. (See \href{http://www2.cnr.edu/home/araia/regina.html}{regina}.)}
\end{figure}

A good article on the classification of Aramaic languages can be found in \textit{The Aramaic language and Its Classification} by Efrem Yildiz.\footnote{\url{http://www.jaas.org/edocs/v14n1/e8.pdf}}








^^A\newfontfamily\aegyptus{AegyptusR.ttf}

\chapter{Aegyptian Hieroglyphics}

\index{fonts>Aegyptus}\index{Aegyptus (font)}
\index{fonts>Hieroglyphics}\index{languages>hieroglyphics}

\newfontfamily\hiero{NotoSansEgyptianHieroglyphs-Regular.ttf}

Hieroglyphic writing appeared in Egypt at the end of the fourth millennium bce. The writing
system is pictographic: the glyphs represent tangible objects, most of which modern
scholars have been able to identify. A great many of the pictographs are easily recognizable
even by nonspecialists. Egyptian hieroglyphs represent people and animals, parts of the
bodies of people and animals, clothing, tools, vessels, and so on.

Hieroglyphs were used to write Egyptian for more than 3,000 years, retaining characteristic
features such as use of color and detail in the more elaborated expositions. Throughout the
Old Kingdom, the Middle Kingdom, and the New Kingdom, between 700 and 1,000 hieroglyphs
were in regular use. During the Greco-Roman period, the number of variants, as
distinguished by some modern scholars, grew to somewhere between 6,000 and 8,000.

Hieroglyphs were carved in stone, painted on frescoes, and could also be written with a reed
stylus, though this cursive writing eventually became standardized in what is called \emph{hieratic}
writing. Unicode does not encode the hieratic forms separately, but ust considers them as cursive forms of the hieroglyphs encoded block.

The Demotic script and then later the Coptic script replaced the earlier hieroglyphic and
hieratic forms for much practical writing of Egyptian, but hieroglyphs and hieratic continued
in use until the fourth century ce. An inscription dated August 24, 394 ce has been
found on the Gateway of Hadrian in the temple complex at Philae; this is thought to be
among the latest examples of Ancient Egyptian writing in hieroglyphs

\begin{figure}[htb]
\includegraphics[width=\textwidth]{./images/bookofthedead.jpg}
\end{figure}

In hieroglyphic texts, these drawings are not only simply arranged in sequential order, but also grouped on top of and next to each other. This rather complicates matters trying to register and reproduce hieroglyphic texts using a computer.

\section{Computer Typesetting}

Typesetting hieroglyphics with computers presents a number of problems. First is the method of inputting the characters and second the various methods required to stack hieroglyphics, the direction of writing which can be one of four different directions.

When the first computers were introduced in Egyptology in the late 1970s and the beginning of the 1980s, the graphical capacity of the machines was still in its infancy. Early attempts to register the hieroglyphic pictorial writing on computer therefore chose an encoding system to do this, using alphanumeric codes to represent or replace the graphics. To prevent many people from reinventing the wheel, during the first "Table Ronde Informatique et Egyptologie" in 1984 a committee was charged with the task to develop a uniform system for the encoding of hieroglyphic texts on computer. The resulting Manual for the Encoding of Hieroglyphic Texts for Computer-input (Jan Buurman, Nicolas Grimal, Jochen Hallof, Michael Hainsworth and Dirk van der Plas, Informatique et Egyptologie 2, Paris 1988), simply called Manuel de Codage, presents an easy to use and intuitive way of encoding hieroglyphic writing as well as the abbreviated hieroglyphic transcription (transliteration). The system proposed by the Manuel de Codage has since been adopted by international Egyptology as the official common standard for registering hieroglyphic texts on computer. Mark-Jan Nederhof proposed an enhanced encoding scheme to remove many of the limitations in the Manuel de Codage.

\pkgname{HieroTeX} is a \latexe package developed by to typeset hieroglyphic texts and still works well. The advantages of using \tex is of course its excellent typesetting capabilities and the usage of macros. Although inputting the texts as MdC codes is not that difficult, repeating the same codes over and over can be avoided with easily constructed simple substitution macros. 

\subsection{fonts}

One of the best fonts I came across is \idxfont{Aegyptus} from \url{http://users.teilar.gr/~g1951d/}\footnote{The site also has fonts for Aegean Numbers, Ancient Greek Musical Notation, Ancient Greek Numbers, Ancient Roman Symbols, Arkalochori Axe, Carian, Cypriot Syllabary, Dispilio tablet, Linear A, Linear B Ideograms, Linear B Syllabary, Lycian, Lydian, Old Italic, Old Persian, Phaistos Disc, Phoenician, Phrygian, Sidetic, Troy vessels’ signs and Ugaritic. Cretan Hieroglyphs and Cypro-Minoan script(s) are offered in separate files.}. The font provides all the unicode characters and also offers an additional number of glyphs that are not in the Unicode standard. The font uses the Unicode Private Use Areas to encode the glyphs. 

Another font is the Noto Egyptian Hieroglyphics from Google. This is a lightweight font with the symbols in their proper unicode slots. Mark-Jan Nederhof's \idxfont{NewGardiner} font is another one with support only for the Gardiner set. The codepoint mappings are incorrect, as the font has been  
encoded to EGPZ. The font is similar to the Aegyptus font, however it is just transposed and not recommended unless it is transposed. 

The editor software JSesh\footnote{\protect\url{http://jsesh.qenherkhopeshef.org/}} also provides a free font |JSeshFont.ttf|. This offers a correctly mapped unicode and is another good alternative. The symbols are drawn somewhat simpler and is just a matter of taste what you want to use.

My recommendation is for short demonstration purposes, the Noto font is to be preferred while for more serious work the Aegyptus font will be more useful. Using Lua the font can be transposed automatically to allow the use of commands that refer to unicode numbers. Another advantage of the Aegyptus font is that the glyphs are named with their Gardiner numbers, so it is somewhat easier to programmatically access them by name.\footnote{Unicode does not name the glyphs, but simply calls the Egyptian Hieroglyph $n$. } 

\medskip

\ifxetex
\bgroup
\centering 
\font\myfont = "Aegyptus"
\scalebox{7}{\myfont\XeTeXglyph 201}
\scalebox{7}{\myfont\XeTeXglyph 203}
\scalebox{7}{\myfont\XeTeXglyph 163}
\scalebox{7}{\myfont\XeTeXglyph 164}
\scalebox{7}{\myfont\XeTeXglyph 165}
\scalebox{7}{\myfont\XeTeXglyph 168}
\captionof{table}{Example of Egyptian Hieroglyphics typeset with the \textit{Aegyptus} font.} 
\egroup
\fi

\ifluatex
\bgroup
\centering 
\aegyptus
\scalebox{7}{\char"F300C}
\scalebox{7}{\char"F3001}
\scalebox{7}{\char"F3010}
\scalebox{7}{\char"F308B}
\scalebox{7}{\char"F3097}
\scalebox{7}{\char"F3091}
\captionof{table}{Example of Egyptian Hieroglyphics typeset with the \textit{Aegyptus} font.} 
\egroup

\fi


\subsection{Unicode Block}

Egyptian hieroglyphs is a Unicode block containing the Gardiner's sign list of Egyptian hieroglyphics.
The code points, in the range |0x13000| to |0x1342E|, are available starting from
\href{http://unicode.org/charts/PDF/U13000.pdf}{Unicode 5.2}

\begin{scriptexample}[]{Hieroglyphic}
\bgroup
\unicodetable{hiero}{"13000,"13010,"13020,"13030,"13040,"13050,"13060,"13070,%
"13080,%
"13090,"130A0,"130B0,"130C0,"130D0,"130E0,"130F0,%
"13100,"13110,"13120,"13130,"13140,"13150,"13060,"13070,"13080,"13090}
\egroup
\end{scriptexample}

\subsection{Gardiner's classification}

The standard reference on Egyptian hieroglyphics is Gartiner's Sign List, which lists common Egyptian hieroglyphs. These are grouped in categories from A-Aa. Each category represents a theme for example category A, is "man and his occupations". Based on this list ``Queen with flower" is denoted as \texttt{B7}. 

\subsubsection{Character Names} 

Egyptian hieroglyphic characters have traditionally been designated in
several ways:

\begin{enumerate}
\item  By complex description of the pictographs: \texttt{GOD WITH HEAD OF IBIS}, and so forth.
\item By standardized sign number: C3, E34, G16, G17, G24.
\item For a minority of characters, by transliterated sound.
\end{enumerate}

The characters in the Unicode Standard make use of the standard Egyptological catalog
numbers for the signs. Thus, the name for {\hiero\char"130F9} |U+13049| egyptian hieroglyph e034 refers
uniquely and unambiguously to the Gardiner list sign E34, described as a “{\aegean DESERT HARE}” ({\hiero \char"130FA}) and used for the sound “wn”. The Unicode catalog values are padded to three places with
zeros, so where the Gardiner classification is shown as \texttt{E34}, the unicode value is \texttt{E034}. 

Names for hieroglyphic characters identified explicitly in Gardiner 1953 or other sources as
variants for other hieroglyphic characters are given names by appending “A”, “B”, ... to the sign number. In the sources these are often identified using asterisks. Thus Gardiner’s G7,
G7*, and G7** correspond to U+13146 egyptian sign g007 {\hiero \char"13147}, U+13147 egyptian sign g007a, and U+13148 egyptian sign g007b, respectively.

\def\texthiero#1{{\color{black!95}\hiero #1}}

\begin{longtable}{>{\Large}lll>{\ttfamily}l}
{\hiero \char"13000}&A1-A70 & Man and his occupations &U+13000-1304F\\
{\hiero \char"13050}&B1-B9  &Woman and her occupations &U+13050-13059\\
{\hiero \char"1305A} &C1-C24 &Anthropomorphic Deities &U+1305A-13075\\
{\hiero \char"13076} &D1-D67 &Parts of the Human Body &U+13076-130D1\\
{\hiero \char"130D2} &E1-E38 &Mammals &U+13076-130D1\\
{\hiero \char"130FE}  &F1-F53	&Parts of Mammals &U+130FE-1313E\\
{\hiero\char"1313F}	&G1-G54	&Birds &U+1313F-1317E\\
{\hiero \char"1317F}	&H1-H8	&Parts of Birds &U+1317F-13187\\
\texthiero{\char"13188}	&I1-I15	&Amphibious Animals, Reptiles, etc. &U+13188-1319A\\
\texthiero{\char"1319B}	&K1-K8	&Fishes and Parts of Fishes &U+1319B-131A2\\
\texthiero{\char"131A3}	&L1-L8	&Invertebrata and Lesser Animals &U+131A3-131AC\\
\texthiero{\char"131AD}	&M1-M44	&Trees and Plants &U+13AD-131EE\\
\texthiero{\char"131EF}	&N1-N42	&Sky, Earth, Water &U+131EF-1321F\\
\texthiero{\char"13250}	&O1-O51	&Buildings and Parts of Buildings &U+13250-1329A\\
\texthiero{\char"1329B}	&P1-P11	&Ships and Parts of Ships &U+1329B-132A7\\
\texthiero{\char"132A8}	&Q1-Q7	& Domestic and Funerary Furniture &U+132A8-132AE\\
\texthiero{\char"132AF}	&R1-R29	&Temple Furniture and Sacret Emblems &U+132AF-132D0\\
\texthiero{\char"132D1}	&S1-S46	&Crowns, Dress, Staves, etc. &U+132D1-13306\\
\texthiero{\char"13307}	&T1-T36	&Warfare, Hunting, Butchery &U+13307-13332\\
\texthiero{\char"13333}	&U1-42	&Agriculture, Crafts and Professions &U+13333-13361\\
\texthiero{\char"13362}	&V1-V40a	&Rope, Fibre, Baskets, Bags, etc. &U+13362-133AE\\
\texthiero{\char"133AF}	&W1-W25	&Vessels of Stone and Earthenware &U+133AF-133CE\\
\texthiero{\char"133CF}	&X1-X8a	&Loaves and Cakes &U+133CF-133DA\\
\texthiero{\char"133DB}	&Y1-Y8	&Writing, Games, Music &U+133DB-133E3\\
\texthiero{\char"133E4}	&Z1-Z16H	&Strokes, Geometrical Figures, etc. &U+133E4-1340C\\
\texthiero{\char"1340D}	&Aa1-Aa32	&Unclassified &U+1340D-1342E\\
\end{longtable}

I particularly like the crocodile sign \def\crocodile{\color{teal}{\Huge\texthiero{\char"13188}}} {\crocodile}, as it is applicable to describe people in my field of work. 

\begin{scriptexample}[]{Woman and her occupations}
\unicodetable{hiero}{"13050}
\end{scriptexample}

\section{Positioning}

One of the core assumptions of any hieroglyphic encoding or mark-up scheme following the MdC is that signs and groups of signs maybe positioned next to each other or above each other. The former is indicated by the operator * and the latter by :. One may also use -, which functions as * for horizontal texts and as : for vertical text. 

In some dialects of the MdC relative positioning has been extended by the use of the |&| operator. This is used to form a kind of ligature, such as |D&t| can be defined to represent the \textit{Cobra at rest} sign I10 with sign X1 underneath, as follows:

\begin{center}
{\hiero\HUGE
       \mbox{\rlap{\char"133CF}\char"13193\hfill\hfill}\\
       {\large|insert[bs](I10,X1)|}

\mbox{\rlap{\scalebox{0.5}{\char"133E3}}\char"13193\hfill\hfill}\\
 	
}
\end{center}

This is only a partial solution and to automate it via kerning tables, will require hundreds of entries in the kerning tables. It will also need constant modifications as researchers discover new combinations. A better approach and which is easily applied to \tex based systems would be to adopt Nederhof's method of creating a new command |insert[bs](I10,X1)|. 

In \tex one could simply define a command \cmd{\insert} with one optional argument to handle the positioning. The positioning uses the letters [b,t,s,e] to position the glyph. the letters s and e stand for start and end, whereas b,t for bottom and top respectively. When there are only two symbols involved, this is not such a difficult operation, but when three or more symbols are to be grouped and kerned together, inserting with some form of scaling is necessary.

\subsection{Enclosures}

Enclosures. The two principal names of the king, the \emph{nomen} and \emph{prenomen}, were normally
written inside a \emph{cartouche}: a pictographic representation of a coil of rope.

In the Unicode representation of hieroglyphic text, the beginning and end of the cartouche
are represented by separate paired characters, somewhat like parentheses. The Unicode manual states that `rendering of a full cartouche surrounding a name requires specialized layout software', which is of course an easy task for \tex.

\begin{macro}{\cartouche}
The commands \cmd{\cartouche} and \cmd{\cartouche}, from Peter Wilson's \pkgname{hierglyph} package have been used for many years to demonstrate the use of hieroglyphics with \latexe. 
\end{macro}

There are a several characters for these start and end cartouche characters, reflecting various styles for the enclosures.

\cartouche{{\hiero \char"13147}$sin^{2} x + cos^{2} x = 1$}
\Cartouche{{\hiero \char"13147}$sin^{2} x + cos^{2} x = 1$}

Unicode:{\hiero 𓇓𓏏𓊵𓏙𓊩𓁹𓏃𓋀𓅂𓊹𓉻𓎟𓍋𓈋𓃀𓊖𓏤𓄋𓈐𓎟𓇾𓈅𓏤𓂦𓈉 }

\textpmhg{\HQ} 

\cartouche{\pmglyph{K:l-i-o-p-a-d:r-a}}
%\translitpmhg{\HK\Hl\Hi\Ho\Hp\Ha\Hd\Hr\Ha}

\printunicodeblock{./languages/hieroglyphics.txt}{\hiero}
\printunicodeblock{./languages/hieroglyphics-13100.txt}{\hiero}
\printunicodeblock{./languages/hieroglyphics-13200.txt}{\hiero}
\printunicodeblock{./languages/hieroglyphics-13300.txt}{\hiero}
\printunicodeblock{./languages/hieroglyphics-13400.txt}{\hiero}
\section{Numerals}

Egyptian numbers are encoded following the same principles used for the
encoding of Aegean and Cuneiform numbers. Gardiner does not supply a full set of
numerals with catalog numbers in his Egyptian Grammar, but does describe the system of
numerals in detail, so that it is possible to deduce the required set of numeric characters.

Two conventions of representing Egyptian numerals are supported in the Unicode Standard.
The first relates to the way in which hieratic numerals are represented. Individual
signs for each of the 1s, the 10s, the 100s, the 1000s, and the 10,000s are encoded, because in
hieratic these are written as units, often quite distinct from the hieroglyphic shapes into
which they are transliterated. The other convention is based on the practice of the \emph{Manual
de Codage}, and is comprised of five basic text elements used to build up Egyptian numerals.
There is some overlap between these two systems.

%% Needs some work to get it into LuaLaTeX
%% omitted for the time being
%\ifxetex
%\begin{texexample}{TeXeXglyph}{ex:xetexglyph}
%\raggedright
%\font\myfont = "Aegyptus"
%\setcounter{glyphcount}{136}
%
%\whiledo
%{\value{glyphcount}<\XeTeXcountglyphs\myfont}
%{\arabic{glyphcount}:~
%{\myfont\XeTeXglyph\arabic{glyphcount}}\quad
%\stepcounter{glyphcount}}
%\end{texexample}
%\fi

\section{Input Methods}

If you writing a document with a lot of hieroglyphics inputting of hieroglyphics can be problematic. Most researchers in the field will use special keyboards or editors. They also use MS/Word or OpenOffice. They can both be coerced to produce reasonable documents, but with \tex obviously better results can be achieved. One such editor is \href{http://jsesh.qenherkhopeshef.org/}{jsesh}. 


\begin{luacode*}
    local h = {}
          h = dofile("hiero.lua")
    local options = {style="block",
                     echo=true,
                     direction="RL",
                     size = "\\Huge",
                     color = "green",
                     headings = "captionof{figure}"  -- section/tablecaption/figurecaption
                     }
   -- prints full symbol list
   h.printgardiner(t,options)

   tex.print("\\par")
   local options = {style="block",
                     echo=true,
                     heading="\\par",
                     direction="RL",
                     color = "teal",
                     scale = 8}

   h.printhierochar("hiero","1317D",options)
   h.printhierochar("hiero","13000",{direction="RL",
                                        color = "teal",
                                        scale = 8})
   h.printhierochar("hiero","13003",{direction="LR",
                                        color = "teal",
                                        scale = 1})
   h.parseMdC([[M23-X1-R4-X8-Q2-D4-W17-R14-G4-R8-O29-
               V30-U23-N26-D58-O49-Z1-F13-N31-V30-N16-
               N21-Z1-D45-N25!]])

   tex.print("\\par")
   h.printgardinercat("B")

\end{luacode*}

\newcommand\hierochar[2][direction = "LR",
                         color     = "teal",
                         scale     = 1]{% 
               \luaexec{
                h = h or {}
                h = require("hiero.lua")  
                h.parseMdC(#2,{#1})}}
               
\newcommand\printhierochar[3][direction = "LR",
                              color     = "teal",
                              scale     = 4]{% 
               \luaexec{
                h = h or {}
                h = require("hiero.lua")  
                h.printhierochar(#2,#3,{#1})}}

This file just tests the various commands available for manipulating hieroglyphics. We tried to 
generalize the commands, so they can be re-used for other type of hieroglyphics.

{
\hierochar{"A1-A2-A3!"}

\centering 

\def\options{direction = "LR",
             color     = "teal",
             scale     = 7}

\def\fontname{"hiero"}

\def\hierochar#1{\printhierochar[\options]{\fontname}{#1}}
}


\begin{scriptexample}[]{Some Example}
Sometimes kerning might be required, especially if the
glyphs are scaled.This is easily achieved with a \cmd{\kern}
command and a suitable skip dimension.

\medskip

\bgroup
\fboxsep=0pt\fboxsep.4pt
\def\options{direction = "RL",
             color     = "black!95",
             scale     = 5}
\centering

\color{teal}
\fbox{\hierochar{"13051"}}
\kern-4mm
\hierochar{"13003"}
\def\options{direction = "LR",
             color     = "black!95",
             scale     = 5}
\fbox{\hierochar{"13003"}}\color{red}
\kern-4mm
\hierochar{"13051"}
\color{black!95}
\egroup
\begin{verbatim}
\centering
\hierochar{"13051"}
\kern-4mm
\hierochar{"13003"}
\def\options{direction = "RL",
             color     = "black!95",
             scale     = 5}
\hierochar{"13003"}
\kern-4mm
\hierochar{"13051"}
\end{verbatim}
\end{scriptexample}

A bit of a diversion is appropriate at this point. Our attempt after the historical overview, is to provide some routines for the capturing and display of hieroglyphic texts using LuaTeX. This involves getting low level information from the system regarding fonts. 

\begin{figure}[ht]
\begin{minipage}{0.45\textwidth}
\centering
\includegraphics[width=0.6\textwidth]{./images/fontforge.jpg}
\end{minipage}
\begin{minipage}[t]{0.45\textwidth}
\caption{Viewing font information with fontforge.}
\end{minipage}
\end{figure}

For each glyph, we are interested to get its unicode number, the position in the font table, its name and most importantly the font metrics. The font metrics are a set of parameters that are used to measure the bounding box, any ascenders or descenders and similar information. Using fontforge, these parameters can easily be viewed. However, we are not interested to make any modifications manually; what we are interested is to programmatically obtain this information using Lua. Lua's philosophy and a mantra repeated often by the developers, is that it provides the tools and not the solutions. What this means to the LuaTeX programmer, is that we need to reach very low level  to get this information, which is a road with many bumps. Luckily the tools have been provided by the LuaTeX developers. This comes with a lot of benefits as we can also do our own on the fly mapping, such as creating an index table holding all the Gardiner numbers. 

The |fontloader.open| function loads a font, but it's not usable by itself; the result should be turned into a table with
\textbf{fontloader.to\_table}, as follows:

\begin{verbatim}
  local f = fontloader.open
     ("c:/windows/fonts/NotSansEgyptianHieroglyphics-
       Regulat.ttf")
  fonttable = fontloader.to_table(f)
  fontloader.close(f)
\end{verbatim}

We will use the Google No Tofu Egyptian Hieroglyphic font to experiment with our hieroglyphics. I have used a full path to load the font, which resides on my windows machine in the fonts folder. Once we load all the information in the |fonttable| we use |fontloader.close| to discard the userdata from which the table is extracted. 

What makes OpenType fonts special is that they describe every aspect that you might be able to think of when you think of putting letters together to form words. In addition to the obvious "this is what letters look like" information, OpenType fonts also specify things like the name of each letter that is available in the font, how much of the Unicode standard the font implements, which horizontal and vertical metrics apply to which letters, exactly how the letters are arranged inside the font so that they can quickly be read out, what kind of font classifications apply (is it a fantasy font? is it bold face? is it fixed width? etc), what kind of memory allocation a printer needs to perform in order to be able to even load the font, etc. etc. etc. All these are stored in tables upon tables, similat to a collection of Russian dolls.

To view the values in the fonttable, we will first iterate over the \textbf{fonttable} and extract all the first level keys.

\begin{texexample}{Iterating through a font table}{}
\begin{luacode*}
local z={}
tf=fontloader.to_table(fontloader.open("c:/windows/fonts/NotoSansEgyptianHieroglyphs-Regular.ttf"))

-- we sort the keys to create a table
-- important keys to us are tf.glyphs

for k,v in pairs (tf) do
   --tex.print(k.."\\par")
   table.insert(z, k)
end

table.sort(z)
tex.print("\\begin{multicols}{3}\\raggedright")
for k,v in pairs (z) do
   z[k] = string.gsub(z[k],"%_","\\textunderscore ")
   local s = tf[v]
   tex.print("\\textbullet\\hskip3pt\\hangindent2em " .. z[k].." [\\textit{"..type(s).."}] ","\\par")
end
tex.print("\\end{multicols}")
\end{luacode*}
\end{texexample}

We iterate through the \textbf{fonttable} using the Lua  "pair" iterator and we simply print all the keys and the type of the values in a human readable form as shown in the example. Note the use of |\textunderscore| that replaces all underscores in the fields with its text equivalent to sanitize the output. This is a quick and dirty way to avoid the use of catcodes. Many of the keys, bear intuitive names and are not difficult to discern: \textit{version}, \textit{copyright} and the like. Getting the type of Lua variables is important in order to use them for error trapping. When you attempt for example to print a nil value an error will occur.

Now that we have peeked under the font we will iterate and capture the information of interest, which we will put into another table with two keys \textbf{info}  and \textbf{metrics}. In the metrics file we will get the bounding box related metrics of each and every glyph in the font and save it, into our own table. 

\begin{texexample}{More Metrics}{}
  \begin{luacode*}
   tex.print("units per em = ", tf.units_per_em,"\\par")
   for i,j in ipairs (tf.glyphs[6].boundingbox) do
      tex.print("bounding box["..i.."]".." = ", j,"\\par")
   end 
   local w = (tf.glyphs[6].boundingbox[3]-tf.glyphs[6].boundingbox[1])/tf.units_per_em
   local h = tf.glyphs[6].boundingbox[4]/tf.units_per_em
   tex.print("glyph width = ", w,"em\\par")
   tex.print("glyph height = ", h,"em\\par")

-- presents a nicely typeset table 

local rep, write = string.rep, tex.print
function ExploreTable (tab, offset)
    offset = offset or ""
    for k, v in pairs (tab) do
        local newoffset = offset .. "\\mbox{.}"
        if type(v) == "table" then
           -- if k == "boundingbox" then write("BB") end
           write(offset..k .. " = \\{\\par ")
           ExploreTable(v, newoffset)
           write(offset..newoffset .. "\\}\\par")
         else
           write(offset..k .. " = "..tostring(v),"\\par")
         end
      end
end

write("\\par{\\ttfamily ")
ExploreTable(tf.glyphs[38],"\\mbox{.}")
write("}")
  \end{luacode*}
\end{texexample}

The OpenType fonts standard, provides for so much information that we will ignore most of the items and focus on only a few tables and fields. A small utility after Paul Isambert's article is necessary to enable us to view tables easily within this book,


\begin{texexample}{ExploreTable utility}{}
\begin{luacode*}
-- presents a nicely typeset table 

local rep, write = string.rep, tex.print
function ExploreTable (tab, offset)
    offset = offset or ""
    for k, v in pairs (tab) do
        local newoffset = offset .. "\\mbox{.}"
        if type(v) == "table" then
           -- if k == "boundingbox" then write("BB") end
           write(offset..k .. " = \\{\\par ")
           ExploreTable(v, newoffset)
           write(offset..newoffset .. "\\}\\par")
         else
           write(offset..k .. " = "..tostring(v),"\\par")
         end
      end
end

write("\\par{\\ttfamily ")
ExploreTable(tf.glyphs[38],"\\mbox{.}")
write("}")
  \end{luacode*}
\end{texexample}

A good utility also is |TTX| that will convert an OTF font to XML and back. This requires that you have python installed.\footnote{See some good guidelines as to how to install it at \url{http://www.glyphrstudio.com/ttx/}.} The utility uses python to do the conversion. The archive can be downloaded from \url{http://sourceforge.net/projects/fonttools/files/latest/download}. This is a three prong attack. You need to have python install, the numpy library and then the TTX package. The |TTX| program was written by the font designer Just van Rossum, brother of the creator of the Python language, Guido van Rossum. The tool converts TrueType into human-readable |XML| format. The most attractive feature of this tool is that it also perform the opposite operation that is create a TruType font from an |XML| file. The |XML| format makes the hierarchy of the format clearer. Since SVG fonts are also described in |XML| it becomes an easier task to convert an |SVG| font to a TrueType font. To convert |bar.ttf| into |bar.ttx| you simply write:

\begin{verbatim}
ttx bar.ttf
\end{verbatim}

Similarly for the opposite conversion, from |.ttx| to |.ttf|

\begin{verbatim}
ttx bar.ttx
\end{verbatim}

The generated ttx file is approximately ten times larger than the original |.ttf| file. The files generated are huge affairs and difficult to manage.The command line option |-l| prints a list of the tables in the font. |TTX| is indispensable in the ``humanization'' of TrueType fonts. The details of the tables and what each field represents are eloquently described in that indispensable book by Yannis Haralambous \textit{Fonts \& Encodings.} Although the book is now somewhat dated, it is still the best source of information on many esoteric topics related to fonts. 






^^A\input{./languages/meroitic}

\subsection{Old Italic}

\newfontfamily\olditalic{seguisym.ttf}

Old Italic refers to any of several now extinct alphabet systems used on the Italian Peninsula in ancient times for various Indo-European languages (predominantly Italic) and non-Indo-European (e.g. Etruscan) languages. The alphabets derive from the Euboean Greek Cumaean alphabet, used at Ischia and Cumae in the Bay of Naples in the eighth century BC.

Various Indo-European languages belonging to the Italic branch (Faliscan and members of the Sabellian group, including Oscan, Umbrian, and South Picene, and other Indo-European branches such as Celtic, Venetic and Messapic) originally used the alphabet. Faliscan, Oscan, Umbrian, North Picene, and South Picene all derive from an Etruscan form of the alphabet.

The Germanic runic alphabet was derived from one of these alphabets by the 2nd century.
Old Italic is a Unicode block containing a unified repertoire of the three stylistic variants of pre-Roman Italic scripts.

\begin{scriptexample}[]{}
\unicodetable{olditalic}{"10300,"10310,"10320}
\end{scriptexample}

\subsection{Old South Arabian}

\newfontfamily\oldsoutharabian{NotoSansOldSouthArabian-Regular.ttf}

The ancient Yemeni alphabet (Old South Arabian ms3nd; modern Arabic: {\arabicfont المُسنَد‎}  musnad) branched from the Proto-Sinaitic alphabet in about the 9th century BC. It was used for writing the Old South Arabian languages of the Sabaic, Qatabanic, Hadramautic, Minaic (or Madhabic), Himyaritic, and proto-Ge'ez (or proto-Ethiosemitic) in Dʿmt. The earliest inscriptions in the alphabet date to the 9th century BC in Akkele Guzay, Eritrea[3] and in the 10th century BC in Yemen. There are no vowels, instead using the \emph{mater lectionis} to mark them.

Its mature form was reached around 500 BC, and its use continued until the 6th century AD, including Old North Arabian inscriptions in variants of the alphabet, when it was displaced by the Arabic alphabet.[4] In Ethiopia and Eritrea it evolved later into the Ge'ez alphabet,[1][2] which, with added symbols throughout the centuries, has been used to write Amharic, Tigrinya and Tigre, as well as other languages (including various Semitic, Cushitic, and Nilo-Saharan languages).

It is usually written from right to left but can also be written from left to right. When written from left to right the characters are flipped horizontally (see the photo).
The spacing or separation between words is done with a vertical bar mark (\textbar).
Letters in words are not connected together.

Old South Arabian script does not implement any diacritical marks (dots, etc.), differing in this respect from the modern Arabic alphabet.

\begin{scriptexample}[]{South Arabian}
\unicodetable{oldsoutharabian}{"10A60,"10A70}
\end{scriptexample}

Support in \latexe is provided via Peter Wilson's package \pkgname{sarabian}. The package provides all the |metafont| sources as well as transliteration commands and other utilities \seedocs{SARAB}.

\def\SAtdu{\oldsoutharabian\char"10A77}

A comparison between  the unicode and the rendering (scaled 5) \pkgname{sarabian} is shown below.

\centerline{\scalebox{3}{\SAtdu} \scalebox{3}{\textsarab{\SAtd}}}

There is no real advantage in using unicode fonts, if all you interested is to write some South Arabian text for inscriptions. 

\begin{symtable}[SARAB]{\SARAB\ South Arabian Letters}
\index{South Arabian alphabet}
\index{alphabets>South Arabian}
\label{sarabian}
\begin{tabular}{*4{ll@{\qquad}}ll}
\K[\textsarab{\SAa}]\SAa   & \K[\textsarab{\SAz}]\SAz   & \K[\textsarab{\SAm}]\SAm   & \K[\textsarab{\SAsd}]\SAsd & \K[\textsarab{\SAdb}]\SAdb \\
\K[\textsarab{\SAb}]\SAb   & \K[\textsarab{\SAhd}]\SAhd & \K[\textsarab{\SAn}]\SAn   & \K[\textsarab{\SAq}]\SAq   & \K[\textsarab{\SAtb}]\SAtb \\
\K[\textsarab{\SAg}]\SAg   & \K[\textsarab{\SAtd}]\SAtd & \K[\textsarab{\SAs}]\SAs   & \K[\textsarab{\SAr}]\SAr   & \K[\textsarab{\SAga}]\SAga \\
\K[\textsarab{\SAd}]\SAd   & \K[\textsarab{\SAy}]\SAy   & \K[\textsarab{\SAf}]\SAf   & \K[\textsarab{\SAsv}]\SAsv & \K[\textsarab{\SAzd}]\SAzd \\
\K[\textsarab{\SAh}]\SAh   & \K[\textsarab{\SAk}]\SAk   & \K[\textsarab{\SAlq}]\SAlq & \K[\textsarab{\SAt}]\SAt   & \K[\textsarab{\SAsa}]\SAsa \\
\K[\textsarab{\SAw}]\SAw   & \K[\textsarab{\SAl}]\SAl   & \K[\textsarab{\SAo}]\SAo   & \K[\textsarab{\SAhu}]\SAhu & \K[\textsarab{\SAdd}]\SAdd \\
\end{tabular}

\bigskip
\begin{tablenote}
  \usefontcmdmessage{\textsarab}{\sarabfamily}.  Single-character
  shortcuts are also supported: Both
  ``\verb+\textsarab{\SAb\SAk\SAn}+'' and ``\verb+\textsarab{bkn}+''
  produce ``\textsarab{bkn}'', for example.  \seedocs{\SARAB}.
\end{tablenote}
\end{symtable}


\section{South East Asian Scripts}

This section documents the facilities offered to typeset Southeast Asian Scripts. These scripts are used in most of Southeast Asia, Indonesia and the Philippines.

\begin{table}[htb]
\centering
\begin{tabular}{lll}
Thai. & Tai Tham &Balinese.\\
Lao.  &Tai Viet  &Javanese.\\
Myanmar &Kayah Li &Rejang\\
Khmer. &Cham &Batak\\
Tai Le &Philippine Scripts &Sundanese.\\
New Tai Lue & Buginese\\
\end{tabular}
\end{table}

\subsection{Thai}

\newfontfamily\thai[Scale=1.0,Script=Thai]{IrisUPC}

\def\thaitext#1{{\thai#1}}

\begin{scriptexample}[]{Thai}
\centerline{\LARGE\thaitext{◌ะ; ◌ัวะ; เ◌ะ; เ◌อะ; เ◌าะ; เ◌ียะ; เ◌ือะ; แ◌ะ; โ◌ะ}}


\hfill Typeset with \idxfont{IrisUPC} and the command \cmd{\thai}
\end{scriptexample}
\subsection{Balinese}

The Balinese script, natively known as Aksara Bali and Hanacaraka, is an abugida used in the island of Bali, Indonesia, commonly for writing the Austronesian Balinese language, Old Javanese, and the liturgical language Sanskrit. With some modifications, the script is also used to write the Sasak language, used in the neighboring island of Lombok.[1] The script is a descendant of the Brahmi script, and so has many similarities with the modern scripts of South and Southeast Asia. The Balinese script, along with the Javanese script, is considered the most elaborate and ornate among Brahmic scripts of Southeast Asia.[2]

Though everyday use of the script has largely been supplanted by the Latin alphabet, the Balinese script has significant prevalence in many of the island's traditional ceremonies and is strongly associated with the Hindu religion. The script is mainly used today for copying lontar or palm leaf manuscripts containing religious texts.[2][3]

\newfontfamily\balinese{AksaraBali.ttf}
\newfontfamily\indicative{code2000.ttf}

{\indicative ◌ }

\newcounter{under}
\setcounter{under}{"1B00}

\def\cb#1 {
\hspace*{2.5pt}
 \large
 $\text{◌#1}_{\pgfmathparse{Hex(\theunder)}\pgfmathresult}$
\stepcounter{under}
\vskip5pt\par
}
\begin{scriptexample}[]{Balinese}


\balinese
	 
᭐	᭑	᭒	᭓	᭔	᭕	᭖	᭗	᭘	᭙	᭚	᭛	᭜	᭝	᭞	᭟\\\
 
\def\columnseprulecolor{\color{thegray}}
\columnseprule.4pt
\begin{multicols}{8}

\texttt{U+1B0x}	

\cb{ᬀ }  \cb{ ᬁ } 	\cb{ ᬂ } 	\cb ᬃ	\cb ᬄ 	\cb ᬅ	\cb ᬆ	\cb ᬇ	\cb ᬈ	\cb ᬉ	\cb ᬊ	\cb ᬋ	\cb ᬌ	\cb ᬍ	\cb ᬎ	\cb ᬏ

\columnbreak

\texttt{U+1B1x}	 

\cb ᬐ	 \cb ᬑ 	\cb ᬒ 	\cb ᬓ	\cb ᬔ	\cb ᬕ	\cb ᬖ \cb ᬗ 	\cb ᬘ 	\cb ᬙ 	\cb ᬚ	\cb ᬛ 	\cb ᬜ 	\cb ᬝ 	\cb ᬞ	\cb ᬟ 

\columnbreak

U+1B2x	 

\cb ᬠ◌ 	\cb ᬡ	\cb ᬢ	\cb ᬣ	\cb ᬤ	\cb ᬥ	\cb ᬦ	\cb ᬧ	\cb ᬨ	\cb ᬩ	\cb ᬪ	\cb ᬫ	\cb ᬬ	\cb ᬭ	\cb ᬮ	\cb ᬯ

\columnbreak
U+1B3x 

\cb ᬰ	\cb ᬱ	\cb ᬲ	\cb ᬳ	\cb ᬴	\cb ᬵ	\cb ᬶ	\cb ᬷ	\cb ᬸ	\cb ᬹ	\cb ᬺ	\cb ᬻ	\cb ᬼ	\cb ᬽ	\cb ᬾ	\cb ᬿ


\columnbreak
U+1B4x	 

\cb ᭀ	 \cb ᭁ	\cb ᭂ	\cb ᭃ	\cb ᭄	\cb ᭅ	\cb ᭆ	\cb ᭇ	\cb ᭈ	\cb ᭉ	\cb ᭊ	\cb ᭋ

\columnbreak				
U+1B5x	 

\cb ᭐	\cb ᭑	\cb ᭒	\cb ᭓	\cb ᭔	\cb ᭕	\cb ᭖	\cb ᭗	\cb ᭘	\cb ᭙	\cb ᭚	\cb ᭛	\cb ᭜	\cb ᭝	\cb ᭞	\cb ᭟\\

\columnbreak

U+1B6x 

\cb ᭠	\cb ᭡	\cb ᭢	\cb ᭣	\cb ᭤	\cb ᭥	\cb ᭦	\cb ᭧	\cb ᭨◌ 	\cb ᭩◌ 	\cb ᭪◌ 	\cb ᭫	\cb ᭬	\cb ᭭	\cb ᭮	\cb ᭯

\columnbreak
U+1B7x	 

\cb ᭰	 \cb ᭱  \cb ᭲  \cb ᭳	 \cb ᭴	\cb ᭵	\cb ᭶	\cb ᭷	\cb ᭸	\cb ᭹	\cb ᭺	\cb ᭻	\cb ᭼


\end{multicols}

\end{scriptexample}
\defaulttext

One of the most comprehensive fonts is Aksara Bali\footnote{\url{http://www.alanwood.net/downloads/index.html}}. This is obtainable at Alan Wood's website.
\parindent1em
\section{Lao Alphabet}

\def\laotext#1{{\lao#1}}

The Lao alphabet, Akson Lao (Lao: \laotext{ອັກສອນລາວ} [ʔáksɔ̌ːn láːw]), is the main script used to write the Lao language and other minority languages in Laos. It is ultimately of Indic origin, the alphabet includes 27 consonants (\laotext{ພະຍັນຊະນະ} [pʰāɲánsānā]), 7 consonantal ligatures (\laotext{ພະຍັນຊະນະປະສົມ} [pʰāɲánsānā pá sǒm]), 33 vowels (\laotext{ສະຫລະ} [sálā]) (some based on combinations of symbols), and 4 tone marks (\laotext{ວັນນະຍຸດ} [ván nā ɲūt]). 



According to Article 89 of Amended Constitution of 2003 of the Lao People's Democratic Republic, the Lao alphabet is the official script to the official language, but is also used to transcribe minority languages in the country, but some minority language speakers continue to use their traditional writing systems while the Hmong have adopted the Roman Alphabet.[1] An older version of the script was also used by the ethnic Lao of Thailand's Isan region, who make up a third of Thailand's population, before Isan was incorporated into Siam, until its use was banned and supplemented with the very similar Thai alphabet in 1871, although the region remained distant culturally and politically until further government campaigns and integration into the Thai state (Thaification) were imposed in the 20th century.[2] The letters of the Lao Alphabet are very similar to the Thai alphabet, which has the same roots. They differ in the fact, that in Thai there are still more letters to write one sound and the more circular style of writing in Lao.

Lao, like most indic scripts, is traditionally written from left to right. Traditionally considered an \emph{abugida} script, where certain 'implied' vowels are unwritten, recent spelling reforms make this definition somewhat problematic, as all vowel sounds today are marked with diacritics when written according the Lao PDR's propagated and promoted spelling standard. However most Lao outside of Laos, and many inside Laos, continue to write according to former spelling standards, which continues the use of the implied vowel maintaining the Lao script's status as an \emph{abugida}. Vowels can be written above, below, in front of, or behind consonants, with some vowel combinations written before, over and after. Spaces for separating words and punctuations were traditionally not used, but a space is used and functions in place of a comma or period. The letters have no \emph{majuscule} or \emph{minuscule} (upper and lower case) differentiations

The Unicode block for the Lao script is U+0E80–U+0EFF, added in Unicode version 1.0. The first 10 characters of the row U+0EDx are the Lao numerals 0 through 9. Throughout the chart grey (unassigned) code points are shown, because the assigned Lao characters intentionally match the relative positions of the corresponding Thai characters. This has created the anomaly that the Lao letter \laotext{ສ} is not in alphabetical order, since it occupies the same codepoint as the Thai letter \laotext{ส}.

\begin{scriptexample}[]{}
\unicodetable{lao}{"0E80,"0E90,"0EA0,"0EB0,"0EC0,"0ED0,"0EE0,"0EF0}
\end{scriptexample}

\subsubsection{Numerals}
\bgroup
\lao
\begin{tabular}{rllllllllllll}
Hindu-Arabic numerals	&0	&1	&2	&3	&4	&5	&6	&7	&8	&9	&10 &	20\\
Lao numerals	&໐	&໑	&໒	&໓	&໔	&໕	&໖	&໗	&໘	&໙	&໑໐	&໒໐\\
Lao names	&ສູນ	&ນຶ່ງ	&ສອງ	&ສາມ	&ສີ່	&ຫ້າ 	&ຫົກ	&ເຈັດ	&ແປດ	&ເກົ້າ	&ສິບ	&ຊາວ\\
\end{tabular}
\egroup




\newfontfamily\javanese{Noto Sans Javanese}

%\newfontfamily\javanese{TuladhaJejeg_gr.ttf}

\section{Javanese}
\label{s:javanese}
\index{scripts>Javanese}


The Javanese (Ngoko Javanese: {\javanese ꦮꦺꦴꦁꦗꦮ},[3] Madya Javanese: {\javanese\   ꦠꦶꦪꦁꦗꦮꦶ},[4] Krama Javanese: ꦥꦿꦶꦪꦤ꧀ꦠꦸꦤ꧀ꦗꦮꦶ,[4] Ngoko Gêdrìk: wòng Jåwå, Madya Gêdrìk: tiyang Jawi, Krama Gêdrìk: priyantun Jawi, Indonesian: suku Jawa)[5] are an ethnic group native to the Indonesian island of Java. With approximately 100 million people (as of 2011), they form the largest ethnic group in Indonesia. They are predominantly located in the central to eastern parts of the island. There are also significant numbers of people of Javanese descent in most provinces of Indonesia, Malaysia, Singapore, Suriname, Saudi Arabia and the Netherlands.

The Javanese ethnic group has many sub-groups, such as the Mataram, Cirebonese, Osing, Tenggerese, Samin, Naganese, Banyumasan, etc.[6]

A majority of the Javanese people identify themselves as Muslims, with a minority identifying as Christians and Hindus. However, Javanese civilization has been influenced by more than a millennium of interactions between the native animism Kejawen and the Indian Hindu—Buddhist culture, and this influence is still visible in Javanese history, culture, traditions, and art forms. With a sizeable global population, the Javanese are considered significant as they are the fourth largest ethnic group among Muslims, in the world, after the Arabs,[7] Bengalis[8] and Punjabis.[9]


\paragraph{Javanese} is one of the Austronesian languages, but it is not particularly close to other languages and is difficult to classify. Its closest relatives are the neighbouring languages such as Sundanese, Madurese and Balinese. Most speakers of Javanese also speak Indonesian, the standardized form of Malay spoken in Indonesia, for official and commercial purposes as well as a means to communicate with non-Javanese-speaking Indonesians.

There are speakers of Javanese in Malaysia (concentrated in the states of Selangor and Johor) and Singapore. Some people of Javanese descent in Suriname (the Dutch colony of Suriname until 1975) speak a creole descendant of the language.

\begin{figure}[htbp]
\includegraphics[width=\textwidth]{javanese-people}
\end{figure}

The language is spoken in Yogyakarta, Central and East Java, as well as on the north coast of West Java. It is also spoken elsewhere by the Javanese people in other provinces of Indonesia, which are numerous due to the government-sanctioned transmigration program in the late 20th century, including Lampung, Jambi, and North Sumatra provinces. In Suriname, creolized Javanese is spoken among descendants of plantation migrants brought by the Dutch during the 19th century. In Madura, Bali, Lombok, and the Sunda region of West Java, it is also used as a literary language. It was the court language in Palembang, South Sumatra, until the palace was sacked by the Dutch in the late 18th century.

Javanese is written with the Latin script, Javanese script, and Arabic script.[5] In the present day, the Latin script dominates writings, although the Javanese script is still taught as part of the compulsory Javanese language subject in elementary up to high school levels in Yogyakarta, Central and East Java.

Javanese is the tenth largest language by native speakers and the largest language without official status. It is spoken or understood by approximately 100 million people. At least 45\% of the total population of Indonesia are of Javanese descent or live in an area where Javanese is the dominant language. All seven Indonesian presidents since 1945 have been of Javanese descent.[6] It is therefore not surprising that Javanese has had a deep influence on the development of Indonesian, the national language of Indonesia.

There are three main dialects of the modern language: Central Javanese, Eastern Javanese, and Western Javanese. These three dialects form a dialect continuum from northern Banten in the extreme west of Java to Banyuwangi Regency in the eastern corner of the island. All Javanese dialects are more or less mutually intelligible.


\paragraph{The Javanese script} (Hanacaraka/Carakan) is a script for writing the Javanese language, the native language of one of the peoples of the Island of Java. It is a descendent of the ancient Brahmi script of India, and so has many similarities with modern scripts of South Asia and Southeast Asia. The Javanese script is also used for writing Sanskrit, Old Javanese, and transcriptions of Kawi, as well as the Sundanese language, and the Sasak language.

\begin{figure}[htbp]
\hspace*{-1.5cm}\includegraphics[width=1.2\textwidth]{java-palm-leave-manuscript}
\end{figure}





\begin{scriptexample}[]{Javanese}
\bgroup
\javanese

꧋ꦱꦧꦼꦤ꧀ꦮꦺꦴꦁꦏꦭꦲꦶꦂꦲꦏꦺꦏꦤ꧀ꦛꦶꦩꦂꦢꦶꦏꦭꦤ꧀ꦢꦂꦧꦺꦩꦂꦠꦧꦠ꧀ꦭꦤ꧀ꦲꦏ꧀ꦲꦏ꧀ꦏꦁꦥꦝ꧉

꧋ ꦲꦮꦶꦠ꧀ꦲꦶꦏꦁꦄꦱ꧀ꦩꦄꦭ꧀ꦭꦃ꧈ ꦏꦁꦩꦲꦩꦸꦫꦃꦠꦸꦂ ꦩꦲꦲꦱꦶꦃ꧉ 	 
 ۝꧋ ꦄꦭꦶꦥꦃ꧀ ꦭ ꦩ꧀ ꦫ ꧌ ꦏꦁ — — ꦥꦿꦶꦏ꧀ꦱ ꦏꦉꦪꦥ꧀ꦥꦩꦸꦁꦄꦭ꧀ꦭꦃꦥꦶꦪꦺꦩ꧀ꦧꦏ꧀ ꧌꧉ ꦩꦁꦪꦏꦴꦪꦤꦴ ꦲꦶꦏꦸꦄꦪꦺꦪꦠꦴꦏꦶꦠꦧ꧀ꦑꦸꦂꦄꦤ꧀ꦏꦁꦥꦿꦪꦠꦭ꧉ 	 
᭐	᭑	᭒	᭓	᭔	᭕	᭖	᭗	᭘	᭙	᭚	᭛	᭜	᭝	᭞	᭟
 
\egroup
\end{scriptexample}


The Javanese script was added to Unicode Standard in version 5.2 on the code points \texttt{A980 - A9DF}. There are 91 code points for Javanese script: 53 letters, 19 punctuation marks, 10 numbers, and 9 vowels:
\medskip

\unicodetable{javanese}{"A980,"A990,"A9A0, "A9B0, "A9C0,"A9D0}

\medskip



As of the writing of this document (2017), there are several widely published fonts able to support Javanese, ANSI-based Hanacaraka/Pallawa by Teguh Budi Sayoga,[21] Adjisaka by Sudarto HS/Ki Demang Sokowanten,[22] JG Aksara Jawa by Jason Glavy,[23] Carakan Anyar by Pavkar Dukunov,[24] and Tuladha Jejeg by R.S. Wihananto,[25] which is based on Graphite (SIL) smart font technology. Other fonts with limited publishing includes Surakarta made by Matthew Arciniega in 1992 for Mac's screen font,[26] and Tjarakan developed by AGFA Monotype around 2000.[27] There is also a symbol-based font called Aturra developed by Aditya Bayu in 2012–2013.[28]

Due to the script's complexity, many Javanese fonts have different input method compared to other Indic scripts and may exhibit several flaws. \docFont{JG Aksara Jawa}, in particular, may cause conflicts with other writing system, as the font use code points from other writing systems to complement Javanese's extensive repertoire. This is to be expected, as the font was made before Javanese implementation in Unicode.[29]

Arguably, the most "complete" font, in terms of technicality and glyph count, is \docFont{TuladhaJejeg}. It comes with keyboard facilities, displaying complex syllable structure, and support extensive glyph repertoire including non-standard forms which may not be found in regular Javanese texts, by utilizing Graphite (SIL) smart font technology. |Tuladha Jejeg| uses variable stroke widths on its glyphs with serifs on some glyphs\footnote{\protect\url{https://sites.google.com/site/jawaunicode/main-page}}.

However, as not many writing systems require such complex feature, use is limited to programs with Graphite technology, such as Firefox browser, Thunderbird email client, and several OpenType word processor and of course XeLaTeX. The font was chosen for displaying Javanese script in the Javanese Wikipedia.[16]

\paragraph{jawaTeX} Jawa\TeX{} project is initial effort to make Javanese characters typesetting program using \TeX{}/\LaTeX{}. This project is aimed to make Javanese widely used. The main project is developing transliteration models to transliterate Latin document into Javanese document. Perl and \TeX{}/\LaTeX{} are use in this project, the program are develop to run in text mode (console) both Linux and Windows but not limit on it. Web based program also developed, and automatic embedded Javanese characters in HTML See \href{http://jawatex.org/jawa/jawatex}{jawatex}.


\section{Khmer}
\newfontfamily\normaltext{Arial Unicode MS}
\normaltext

\def\khmerdefaultfont#1{\newfontfamily\khmer[Scale=MatchUppercase]{#1}}
\def\khmertext#1{{\khmer#1}}

\cxset{khmer font/.code=\khmerdefaultfont{#1}}

\cxset{khmer font/.default=Khmer}

\cxset{language=khmer, 
       khmer font = Khmer UI}

\begin{key}{/chapter/khmer font=\meta{font name} (Khmer  UI)} Loads the font
command \cmd{\khmer}. When the command is used it typesets text in
khmer unicode. There is no need to load the language, unless it is the main document language. For windows the default font is \texttt{DaunPenh} this font is in general too small to read; a better font to use is Khmer UI.
\end{key}

\begin{key}{/tikz/turtle/right=\meta{angle} (default 90)}
  Turns the turtle right by the given angle. 
\end{key}


The Khmer script (Khmer: {\Large\khmertext{អក្សរខ្មែរ}}; IPA: [ʔaʔsɑː kʰmaːe]) [2] is an \textit{abugida} (alphasyllabary) script used to write the Khmer language (the official language of Cambodia). It is also used to write Pali among the Buddhist liturgy of Cambodia and Thailand.

It was adapted from the Pallava script, a variant of Grantha alphabet descended from the Brahmi script of India, which was used in southern India and South East Asia during the 5th and 6th Centuries AD.[3] The oldest dated inscription in Khmer was found at Angkor Borei District in Takéo Province south of Phnom Penh and dates from 611.[4] The modern Khmer script differs somewhat from precedent forms seen on the inscriptions of the ruins of Angkor.

Not all Khmer consonants can appear in syllable-final position. The most common syllable-final consonants include {\khmer កងញតនបមល}. The pronunciation of the consonant in final position may differ from it's normal pronunciation.


\begin{tabular}{llp{9cm}}
\khmertext{ំ}	&nĭkkôhĕt (\khmertext{និគ្គហិត})	&niggahita; nasalizes the inherent vowels and some of the dependent vowels, see anusvara, sometimes used to represent [aɲ] in Sanskrit loanwords\\
\khmertext{ះ}	&reăhmŭkh (\khmertext{រះមុខ})	&"shining face"; adds final aspiration to dependent or inherent vowels, usually omitted, corresponds to the visarga diacritic, it maybe included as dependent vowel symbol\\
\khmertext{ៈ}	&yŭkôleăkpĭntŭ (\khmertext{យុគលពិន្ទុ})	&yugalabindu ("pair of dots"); adds final glottalness to dependent or inherent vowels, usually omitted\\
\khmertext{៉}	 &musĕkâtônd (\khmertext{មូសិកទន្ត})	&mūsikadanta ("mouse teeth"); used to convert some o-series consonants (\khmertext{ង ញ ម យ រ វ}) to the a-series\\
\khmertext{៊}	&treisâpt (\khmertext{ត្រីសព្ទ})	trīsabda; used to convert some a-series consonants (\khmertext{ស ហ ប អ}) to the o-series\\
\end{tabular}




ុ	kbiĕh kraôm (ក្បៀសក្រោម)	also known as bŏkcheung (បុកជើង); used in place of the diacritics treisâpt and musĕkâtônd when they would be impeded by superscript vowels
់	bântăk (បន្តក់)	used to shorten some vowels; the diacritic is placed on the last consonant of the syllable
៌	rôbat (របាទ)
répheăk (រេផៈ)	rapāda, repha; behave similarly to the tôndâkhéat, corresponds to the Devanagari diacritic repha, however it lost its original function which was to represent a vocalic r
 ៍	tôndâkhéat (ទណ្ឌឃាដ)	daṇḍaghāta; used to render some letters as unpronounced
៎	kakâbat (កាកបាទ)	kākapāda ("crow's foot"); more a punctuation mark than a diacritic; used in writing to indicate the rising intonation of an exclamation or interjection; often placed on particles such as /na/, /nɑː/, /nɛː/, /vəːj/, and the feminine response /cah/
៏	âsda (អស្តា)	denotes stressed intonation in some single-consonant words[5]
័	sanhyoŭk sannha (សំយោគសញ្ញា)	represents a short inherent vowel in Sanskrit and Pali words; usually omitted
៑	vĭréam (វិរាម)	a mostly obsolete diacritic, corresponds to the virāma
្	cheung (ជើង)	a.w. coeng; a sign developed for Unicode to input subscript consonants, appearance of this sign varies among fonts
\section{Sundanese}
\newfontfamily\sundanese{SundaneseUnicode-1.0.5.ttf}
^^A\newfontfamily\sundanese{Arial Unicode MS}
\def\ublock#1{\texttt{{\arial #1}}}

The Sundanese script (Aksara Sunda, {\sundanese ᮃᮊ᮪ᮞᮛ ᮞᮥᮔ᮪ᮓ}) is a writing system which is used by the Sundanese people. It is built based on Old Sundanese script (Aksara Sunda Kuno) which was used by the ancient Sundanese between the 14th and 18th centuries.

\begin{scriptexample}[]{Sundanese}
\unicodetable{sundanese}{"1B80,"1B90,"1BA0,"1BB0}

\sundanese
\obeylines
\bgroup
᮱ {\arial= 1}	᮲ {\arial= 2}	᮳{\arial = 3}
᮴ {\arial= 4}	᮵ {\arial = 5} 	᮶ {\arial= 6}
᮷ {\arial= 7}	᮸ {\arial= 8}	᮹ {\arial= 9}
᮰ {\arial= 0}

\egroup
\end{scriptexample}

\begin{scriptexample}[]{Sundanese}
\bgroup
\sundanese
\centering

◌ᮃᮄᮅᮆᮇᮈᮉᮊᮋᮌᮍᮎᮏᮐᮕᮔᮓᮑᮖᮗᮚᮛᮜᮝᮞᮟᮠᮠ


\egroup
\end{scriptexample}

\bgroup
\def\1{\sundanese ᮱}
\TextOrMath\1\1

$\1$
\egroup

In text In texts, numbers are written surrounded with dual pipe sign \textbar \ldots \textbar. Example: {\textbar \sundanese ᮲᮰᮱᮰\textbar} = 2010













^^A\subsection{Oriya alphabet}
\newfontfamily\oriya[Scale=1.1,Script=Oriya]{code2000.ttf}

\def\oriyatext#1{{\oriya#1}}
The Oriya script or Utkala Lipi (Oriya: \oriyatext{ଉତ୍କଳ ଲିପି}) or Utkalakshara (Oriya: \oriyatext{ଉତ୍କଳାକ୍ଷର}) is used to write the Oriya language, and can be used for several other Indian languages, for example, Sanskrit.

\centerline{\Huge\oriyatext{ଉତ୍କଳ ଲିପି}}

\bgroup
\oriya
୦୧୨୩୪୫୬୭୮୯
ଅ ଆ ଇ ଈ ଉ ଊ ଋ ୠ ଌ ୡ ଏ ଐ ଓ ଔ କ ଖ ଗ ଘ ଙ ଚ ଛ ଜ ଝ ଞ ଟ ଠ ଡ ଢ ଣ ତ ଥ ଦ ଧ ନ ପ ଫ ବ ଵ ଭ ମ ଯ ର ଳ ୱ ଶ ଷ ସ ହ ୟ ଲ
\egroup

\begin{quotation}
Oṛiyā is encumbered with the drawback of an excessively awkward and cumbrous written character. ... At first glance, an Oṛiyā book seems to be all curves, and it takes a second look to notice that there is something inside each.(G. A. Grierson, Linguistic Survey of India, 1903)
\end{quotation}

Comparison of Oṛiyā script with its neighbours[edit]
At a first look the great number of signs with round shapes suggests a closer relation to the southern neighbour Telugu than to the other neighbours Bengali in the north and Devanāgarī in the west. The reason for the round shapes in Oriya and Telugu (and also in Kannaḍa and Malayāḷam) is the former method of writing using a stylus to scratch the signs into a palm leaf. These tools do not allow for horizontal strokes because that would damage the leaf.

Oriya letters are mostly round shaped whereas in Devanāgarī and Bengali have horizontal lines. So in most cases the reader of Oṛiyā will find the distinctive parts of a letter only below the hoop. Considering this the  closer relation to Devanāgarī and Bengali exists than to any southern script, though both northern and southern scripts have the same origin, Brāhmī.

Oriya (\oriyatext{ଓଡ଼ିଆ} oṛiā), officially spelled Odia,[3][4] is an Indian language belonging to the Indo-Aryan branch of the Indo-European language family. It is the predominant language of the Indian states of Odisha, where native speakers comprise 80\% of the population,[5] and it is spoken in parts of West Bengal, Jharkhand, Chhattisgarh and Andhra Pradesh. Oriya is one of the many official languages in India; it is the official language of Odisha and the second official language of Jharkhand. [6][7][8] Oriya is the sixth Indian language to be designated a Classical Language in India, on the basis of having a long literary history and not having borrowed extensively from other languages.

^^A
^^A\subsection{Mongolian Script}

\newfontfamily\mongolian[Language=Mongolian, Scale=1.3]{code2000.ttf}

The classical Mongolian script (in Mongolian script: {\mongolian  ᠮᠣᠩᠭᠣᠯ ᠪᠢᠴᠢᠭ᠌} Mongγol bičig; in Mongolian Cyrillic: Монгол бичиг Mongol bichig), also known as Uyghurjin Mongol bichig, was the first writing system created specifically for the Mongolian language, and was the most successful until the introduction of Cyrillic in 1946. Derived from Uighur, Mongolian is a true alphabet, with separate letters for consonants and vowels. The Mongolian script has been adapted to write languages such as Oirat and Manchu. Alphabets based on this classical vertical script are used in Inner Mongolia and other parts of China to this day to write Mongolian, Sibe and, experimentally, Evenki.
\medskip

\bgroup\par
\noindent
\colorbox{graphicbackground}{\color{black}^^A
\begin{minipage}{\textwidth}^^A
\parindent1pt
\vskip10pt
\leftskip10pt \rightskip\leftskip
\mongolian
\large
ᠬᠦᠮᠦᠨ ᠪᠦᠷ ᠲᠥᠷᠥᠵᠦ ᠮᠡᠨᠳᠡᠯᠡᠬᠦ ᠡᠷᠬᠡ ᠴᠢᠯᠥᠭᠡ ᠲᠡᠢ᠂ ᠠᠳᠠᠯᠢᠬᠠᠨ ᠨᠡᠷ᠎ᠡ ᠲᠥᠷᠥ ᠲᠡᠢ᠂ ᠢᠵᠢᠯ ᠡᠷᠬᠡ ᠲᠡᠢ ᠪᠠᠢᠠᠭ᠃ ᠣᠶᠤᠨ ᠤᠬᠠᠭᠠᠨ᠂ ᠨᠠᠨᠳᠢᠨ ᠴᠢᠨᠠᠷ ᠵᠠᠶᠠᠭᠠᠰᠠᠨ ᠬᠦᠮᠦᠨ ᠬᠡᠭᠴᠢ ᠥᠭᠡᠷ᠎ᠡ ᠬᠣᠭᠣᠷᠣᠨᠳᠣ᠎ᠨ ᠠᠬᠠᠨ ᠳᠡᠭᠦᠦ ᠢᠨ ᠦᠵᠢᠯ ᠰᠠᠨᠠᠭᠠ ᠥᠠᠷ ᠬᠠᠷᠢᠴᠠᠬᠥ ᠤᠴᠢᠷ ᠲᠠᠢ᠃
\par
\vspace*{10pt}
\end{minipage}
}
\medskip
^^A
^^A\subsection{Tibetan}

^^A\newfontfamily\tibetan{TibMachUni.ttf}

^^A\newfontfamily\tibetan{Qomolangma-Chuyig.ttf}

^^A should pick it up automatically \tibetan

Fonts described in this section can be obtained from The Tibetan \& Himalayan Library
\footnote{\url{http://www.thlib.org/tools/scripts/wiki/tibetan%20machine%20uni.html}  }

I have tried a few \texttt{Tibetan Machine Uni (TMU)} seems to be used by a number of scholars. 

A tip when you are trying to locate fonts is to find a related article in Wikipedia, such as Tibetan alphabet and inspect the element using your browser to see what fonts are being used.


|style="font-family:'Jomolhari','Tibetan Machine Uni','DDC Uchen', 'Kailash';| 


If you cannot see the script and rather than boxes or question marks then you can search and download one of the fonts in |font-family|.

\def\tibetandefaultfont#1{\newfontfamily\tibetan[Language=Tibetan]{#1}}


\cxset{language=tibetan} 
\cxset{tibetan font/.code=\tibetandefaultfont{#1}}


^^A\cxset{tibetan font = TibMachUni.ttf}




\begin{key}{/chapter/language = tibetan} The key |language=tibetan| sets the default language as Tibetan, using the main font given by the key |tibetan font=TibMachUni.ttf|.
\end{key}

\begin{key}{/chapter/tibetan font = TibMachUni.ttf} The key |tibetan font=font-name| sets the default font for the Tibetan language. It will also create the switch \cmd{\tibetan} for typesetting text in Tibetan.
\end{key}

\begin{texexample}{Tibetan language setttings}{ex:tibetan}
\cxset{language=tibetan, tibetan font = TibMachUni.ttf}
\tibetan

\tibetan Tibetan: དབུ་ཅན
\end{texexample}


The Tibetan alphabet is an \emph{abugida} of Indic origin used to write the Tibetan language as well as Dzongkha, the Sikkimese language, Ladakhi, and sometimes Balti. 

The printed form of the alphabet is called \textit{uchen} script (Tibetan: དབུ་ཅན་, Wylie: dbu-can; "with a head") while the hand-written cursive form used in everyday writing is called umê script (Tibetan: དབུ་མེད་, Wylie: dbu-med; "headless").
\uccoff
The alphabet is very closely linked to a broad ethnic Tibetan identity. Besides Tibet, it has also been used for Tibetan languages in Bhutan, India, Nepal, and Pakistan.[1] The Tibetan alphabet is ancestral to the Limbu alphabet, the Lepcha alphabet,[2] and the multilingual 'Phags-pa script.[2]
\uccon

The Tibetan alphabet is romanized in a variety of ways.[3] This article employs the Wylie transliteration system.

The Tibetan alphabet has thirty basic letters, sometimes known as "radicals", for consonants.[2]

ཀ ka /ká/	ཁ kha /kʰá/	ག ga /kà, kʰà/	ང nga /ŋà/
ཅ ca /tʃá/	ཆ cha /tʃʰá/	ཇ ja /tʃà/	ཉ nya /ɲà/
ཏ ta /tá/	ཐ tha /tʰá/	ད da /tà, tʰà/	ན na /nà/
པ pa /pá/	ཕ pha /pʰá/	བ ba /pà, pʰà/	མ ma /mà/
ཙ tsa /tsá/	ཚ tsha /tsʰá/	ཛ dza /tsà/	ཝ wa /wà/ (not originally part of the alphabet)[5]
ཞ zha /ʃà/[6]	ཟ za /sà/	འ 'a /hà/[7]
ཡ ya /jà/	ར ra /rà/	ལ la /là/
ཤ sha /ʃá/[6]	ས sa /sá/	ཧ ha /há/[8]
ཨ a /á/

\subsubsection{Unicode Block Tibetan}


\bgroup\large
\begin{tabular}{llllllllllllllll l}
\toprule
	           &|0|	&|1|	&|2|	&|3|	&|4|	&|5|	&|6|	&|7|	&|8|	&|9|	&|A|	&|B|	&|C|	&|D|	&|E|	&|F|\\
\midrule
\texttt{U+0F0x}	&ༀ	&༁	&༂	&༃	&༄	&༅	&༆	&༇	&༈	&༉	&༊	&་	&༌  &	།	&༎	&༏\\
\midrule
\texttt{U+0F1x} &༐	&༑	&༒	&༓	&༔	&༕	&༖	&༗	&༘&	༙	&༚	&༛	&༜	&༝	&༞	&༟\\
\midrule
\texttt{U+0F2x} &༠	&༡	&༢	&༣	&༤	&༥	&༦	&༧	&༨	&༩	&༪	&༫	&༬	&༭	&༮	&༯\\
\midrule
\texttt{U+0F3x}	&༰ &༱	 &༲ &༳	&༴ &༵	&༶ & ༷	&༸&	༹	&༺&	༻	&༼&	༽	&༾	&༿\\
\midrule
\texttt{U+0F4x} &ཀ	&ཁ	&ག	&གྷ	&ང	&ཅ	&ཆ	&ཇ	&	&ཉ	&ཊ	&ཋ	&ཌ	&ཌྷ	&ཎ	&ཏ\\
\midrule
\texttt{U+0F5x}	 &ཐ	&ད	&དྷ	&ན	&པ	&ཕ	&བ	&བྷ	&མ	&ཙ	&ཚ	&ཛ	&ཛྷ	&ཝ	&ཞ	&ཟ\\
\midrule
\texttt{U+0F6x} &འ	&ཡ	&ར	&ལ	&ཤ	&ཥ	&ས	&ཧ	&ཨ	&ཀྵ	&ཪ	&ཫ	&ཬ	&&&\\
^^A\texttt{U+0F7x}&&	ཱ &	& &ི	ཱི&	ུ&	ཱུ&	ྲྀ&	ཷ&	ླྀ&	ཹ&	ེ&	ཻ&	ོ&	ཽ&	&ཾ	&ཿ\\
\midrule
\texttt{U+0F8x}&    ྀ   & 	ཱྀ&	ྂ&	&ྃ &	྄	&྅&	྆	&྇	ྈ&	ྉ&	ྊ&	ྋ&	ྌ&	ྍ&	ྎ&	ྏ\\
\midrule
\texttt{U+0F9x} &	ྐ&	ྑ   & 	ྒ &	ྒྷ &	ྔ &	ྕ &	ྖ &	ྗ &		ྙ &	ྚ &	ྛ &	ྜ &	ྜྷ &	ྞ &	ྟ\\
\texttt{U+0FAx} &	ྠ &	ྡ &	ྡྷ &	ྣ &	ྤ &	ྥ &		&ྦ	&ྦྷ	ྨ&	ྩ&	ྪ&	ྫ&	ྫྷ&	ྭ&	ྮ&	ྯ\\
\midrule
\texttt{U+0FBx} 
&	  ྰ 
&	
& ྱ  	 
&ྲ	
&ླ	
&ྴ
&	ྵ
&	ྶ
&	ྷ
&ྸ
&
&
&
&	
&྾	
&྿\\
\midrule
\texttt{U+0FCx}	 &࿀&	࿁&	࿂&	࿃&	࿄&	࿅&	&࿇	&࿈	&࿉	&࿊	&࿋	&࿌	&&	࿎	&࿏\\
\midrule
\texttt{U+0FDx}	&࿐	&࿑	&࿒	&࿓	&࿔	&࿕	&࿖	&࿗	&࿘	&࿙	&࿚	&&&&&\\
\midrule
\texttt{U+0FEx} &&&&&&&&&&&&&&&&\\
\midrule
\texttt{U+0FFx}  &&&&&&&&&&&&&&&&\\
\bottomrule
\end{tabular}
\egroup




\subsubsection{Fonts for Tibetan}

Fonts for Tibetan need to be downloaded one set of fonts are the \texttt{Qomolangma}. They come in different flavours, but they appear
to offer advantages as compared to the Tibetan Machine Uni.
\medskip


\newfontfamily\betsu{Qomolangma-Betsu.ttf}
\newfontfamily\drutsa{Qomolangma-Drutsa.ttf}
\newfontfamily\chuyig{Qomolangma-Chuyig.ttf}
\newfontfamily\tsumachu{Qomolangma-Tsumachu.ttf}
\newfontfamily\uchensutung{Qomolangma-UchenSutung.ttf}
\newfontfamily\uchensuring{Qomolangma-UchenSuring.ttf}
\newfontfamily\uchensarchen{Qomolangma-UchenSarchen.ttf}
\newfontfamily\uchensarchung{Qomolangma-UchenSarchung.ttf}
\newfontfamily\tsuring{Qomolangma-Tsuring.ttf}
\newfontfamily\TMU{TibMachUni.ttf}
\newfontfamily\himalaya{Microsoft Himalaya}
\uccoff

{
\centering

\renewcommand{\arraystretch}{1.5}

\begin{tabular}{lr}
\toprule
|Qomolangma-Betsu.ttf| & {\betsu  དབུ་མེད }\\
\midrule
|Qomolangma-Chuyig.ttf| &{\chuyig  དབུ་མེད}\\
\midrule
|Qomolangma-Drutsa.ttf| &{\drutsa  དབུ་མེད}\\
\midrule
|Qomolangma-Tsumachu.ttf|&{\tsumachu  དབུ་མེད}\\
\midrule
|Qomolangma-Tsuring.ttf| &{\tsuring  དབུ་མེད}\\
\midrule
|Qomolangma-UchenSarchen.ttf| &{\uchensarchen དབུ་མེད}\\
\midrule
|Qomolangma-UchenSarchung.ttf|&{\uchensarchung དབུ་མེད }\\
\midrule
|Qomolangma-UchenSuring.ttf|&{\uchensuring དབུ་མེད}\\
\midrule
|Qomolangma-UchenSutung.ttf|&{\uchensutung དབུ་མེད }\\
\midrule
|TibMachUni.ttf| &{\TMU དབུ་མེད }\\
\midrule
|Microsoft Himalaya| &{\himalaya དབུ་མེད ཽ}\\
\bottomrule
\end{tabular}

}
\bigskip

\bgroup
\LARGE\tsuring
\noindent༆ །ཨ་ཡིག་དཀར་མཛེས་ལས་འཁྲུངས་ཤེས་བློ  འི་\par
གཏེར༑ །ཕས་རྒོལ་ཝ་སྐྱེས་ཟིལ་གནོན་གདོང་ལྔ་བཞིན།།\par
ཆགས་ཐོགས་ཀུན་བྲལ་མཚུངས་མེད་འཇམ་དབྱངསམཐུས།།\par
མཧཱ་མཁས་པའི་གཙོ་བོ་ཉིད་འགྱུར་ཅིག། །མངྒལཾ༎\par
\egroup

\subsubsection{Tibetan numbers}
\cxset{language=tibetan, tibetan font = TibMachUni.ttf}

{
\obeylines
\small
TIBETAN DIGIT ZERO	༠
TIBETAN DIGIT ONE	༡	
TIBETAN DIGIT TWO	༢	
TIBETAN DIGIT THREE	༣	
TIBETAN DIGIT FOUR	༤	
TIBETAN DIGIT FIVE	༥	
TIBETAN DIGIT SIX	༦	
TIBETAN DIGIT SEVEN	༧	
TIBETAN DIGIT EIGHT	༨	
TIBETAN DIGIT NINE	༩	
TIBETAN DIGIT HALF ONE	\tibetan༪	
TIBETAN DIGIT HALF TWO	༫	
TIBETAN DIGIT HALF THREE	༬
TIBETAN DIGIT HALF FOUR ༭	
TIBETAN DIGIT HALF FIVE ༯	
TIBETAN DIGIT HALF SIX	 ༯	
TIBETAN DIGIT HALF SEVEN	༰	
TIBETAN DIGIT HALF EIGHT	༱	
TIBETAN DIGIT HALF NINE	༲	
TIBETAN DIGIT HALF ZERO	༳	
}


Tibetan numbers

The usage is not certain. By some interpretations, this has the value of 9.5. Used only in some traditional contexts, these appear as the last digit of a multidigit number, eg. ༤༬ represents 42.5. These are very rarely used, however, and other uses have been postulated.

\defaulttext

^^A
^^A
^^A

^^A\section{Tamil}
\newfontfamily\tamil[Scale=1.1,Script=Tamil]{code2000.ttf}

\def\tamiltext#1{{\tamil#1}}

The Tamil script (\tamiltext{தமிழ் அரிச்சுவடி} tamiḻ ariccuvaṭi) is an abugida script that is used by the Tamil people in India, Sri Lanka, Malaysia and elsewhere, to write the Tamil language, as well as to write the liturgical language Sanskrit, using consonants and diacritics not represented in the Tamil alphabet.[1] Certain minority languages such as Saurashtra, Badaga, Irula, and Paniya are also written in the Tamil script

The Tamil script has 12 vowels (\tamiltext{உயிரெழுத்து} uyireḻuttu "soul-letters"), 18 consonants (\tamiltext{மெய்யெழுத்து} meyyeḻuttu "body-letters") and one character, the āytam \tamiltext{ஃ (ஆய்தம்)}, which is classified in Tamil grammar as being neither a consonant nor a vowel (\tamiltext{அலியெழுத்து} aliyeḻuttu "the hermaphrodite letter"), though often considered as part of the vowel set (\tamiltext{உயிரெழுத்துக்கள்} uyireḻuttukkaḷ "vowel class"). The script, however, is syllabic and not alphabetic.[3] The complete script, therefore, consists of the thirty-one letters in their independent form, and an additional 216 combinant letters representing a total 247 combinations (\tamiltext{உயிர்மெய்யெழுத்து} uyirmeyyeḻuttu) of a consonant and a vowel, a mute consonant, or a vowel alone. These combinant letters are formed by adding a vowel marker to the consonant. Some vowels require the basic shape of the consonant to be altered in a way that is specific to that vowel. Others are written by adding a vowel-specific suffix to the consonant, yet others a prefix, and finally some vowels require adding both a prefix and a suffix to the consonant. In every case the vowel marker is different from the standalone character for the vowel.
The Tamil script is written from left to right.

Tamil is a Unicode block containing characters for the Tamil, Badaga, and Saurashtra languages of Tamil Nadu India, Sri Lanka, Singapore, and Malaysia. In its original incarnation, the code points U+0B02..U+0BCD were a direct copy of the Tamil characters A2-ED from the 1988 ISCII standard. The Devanagari, Bengali, Gurmukhi, Gujarati, Oriya, Telugu, Kannada, and Malayalam blocks were similarly all based on their ISCII encodings.

\begin{scriptexample}[]{Tamil}
\unicodetable{tamil}{"0B80,"0B90,"0BA0,"0BB0,"0BC0,"0BE0,"0BF0}

\hfill  Typeset with \cmd{\tamil} and \texttt{code2000.ttf}
\end{scriptexample}

\subsection{Tamil Numbers and Numerals}

Originally, Tamils did not use zero, nor did they use positional digits (having separate 
symbols for the numbers 10, 100 and 1000). Symbols for the numbers are similar to 
other Tamil letters, with some minor changes. 

For example, the number 3782 is not written as \tamiltext{௩௭௮௨} as in modern usage. Instead it 
is written as \tamiltext{௩ ௲ ௭ ௱ ௮ ௰ ௨}. This would be read as they are written as 
Three Thousands, Seven Hundreds, Eight Tens, Two; or in Tamil as 
\tamiltext{௩௲௭௱௮௰௨ž}.\footnote{https://cloud.github.com/downloads/raaman/Tamil-Numeral/tamilnumbers.html}

\subsection{Dates}

Once the script is loaded the day, month and year can be loaded using the command  \cmd{\tamildate}, which returns the |\today| formatted as per custom Tamil. 

\begin{center}
\bgroup
\tamil
\begin{tabular}{lll}
day	 &month	&year	\\

௳	&௴	      &௵	\\

u	&mee	      &wa	\\
\egroup
\end{center}











^^A\chapter{Armenian}

\label{s:armenian}\index{Armenian}\index{scripts>Armenian}

As we present the scripts in alphabetic order, the first script we will typeset is in Armenian. There are many fonts available for the language. We use two in the example, the first one is \textit{FreeSans} and the second is \textit{Sylphaen} which is found on Windows Operating systems. The language is not supported by the \pkg{Babel} and partially supported by the \pkgname{Polyglossia}. \tcbdocmarginnote{china revision}

\def\ucfirst#1#2;{\MakeUppercase#1#2}


\def\armeniantest#1#2{
  {\parindent0pt
  \topline \vskip3pt
  \noindent\mbox{
     \ucfirst#1;\hfill\hbox{[\texttt{U+0530-U+058F}]}
  }}
 \nobreak

\begin{minipage}{0.45\textwidth}
\bgroup
%\setotherlanguage{#1}
\begin{#1}
#2
[\today]
\end{#1}
\egroup
\end{minipage}\hspace*{1em}
\begin{minipage}{0.45\textwidth}
\bgroup
  \parindent0pt
  \ttfamily\raggedright
  \string\documentclass\{article\}\par
  \string\usepackage[no-math]\{fontspec\}\\
  \string\newfontfamily\textbackslash#1font[Script=\ucfirst #1;,\\   ~~~~~~~Scale=MatchLowercase]
\{FreeSans\}\par
  \string\begin\{document\}\\
  \string\setotherlanguage\{#1\}\\
  \string\begin\{#1\}\\
  \egroup
\begin{#1}
\hskip10pt\vbox{#2}
\end{#1}
\bgroup
  \ttfamily[\detokenize{\today}]\\
  \string\end\{#1\}\\
  \string\end\{document\}
\egroup
\end{minipage}


\textit{FreeSans}: \url{ http://www.gnu.org/software/freefont/}
}

\armeniantest{armenian}{Բոլոր մարդիկ ծնվում են ազատ ու հավասար իրենց
արժանապատվությամբ ու իրավունքներով։       
Նրանք ունեն բանականություն ու խիղճ և միմյանց
պետք է եղբայրաբար վերաբերվեն։}

The Armenian script was invented around 407 AD, by Mesrop Maštoc, a cleric who wanted to 
translate Greek and Syriac scriptures and liturgical texts into Armenian. The system he devised 
is a pure alphabet, closely modelled on the traditional order of Greek phonetic values, with 
additional graphemes to represent Armenian sounds not found in Greek. The orthography is, 
phonetically, a near perfect representation of the Armenian language, and has remained almost 
entirely unchanged since its invention. In recent times, the letterforms in many Armenian 
typefaces have consciously modelled Latin types in their treatment of serifs, stroke weight and 
stress, and other details. This is the approach that Geraldine adopted for the Sylfaen Armenian, 
in order to harmonise the different scripts within the font. 

This kind of harmonisation has to be 
very carefully handled, as there is, of course, a point at which one can corrupt the normative 
letterforms and produce something which will be unacceptable to native readers. Once again, 
we sought expert review of the design, this time from Manvel Shmavonyan, an Armenian type designer, and his Russian colleague Vladimir Yefimov at 
ParaType in Moscow.

\bgroup
\medskip
\fontspec[Script=Armenian,Scale=1.7]{Sylfaen}
\centering

Աա Բբ Գգ Դդ Եե Զզ Էէ Ըը Թթ Ժժ Իի \\
Լլ Խխ Ծծ Կկ Հհ Ձձ Ղղ Ճճ Մմ Յյ Նն \\
Շշ Ոո Չչ Պպ Ջջ Ռռ Սս Վվ Տտ Րր Ցց \\
Ււ Փփ Քք Օօ Ֆֆ / և ﬓ ﬔ ﬕ ﬖ ﬗ\\
\egroup
\captionof{table}{Armenian, showing the basic alphabet (typeset using the \textit{Sylfaen} font.}
\medskip

\bgroup
\def\m#1 #2 #3\\{\makebox[2em]{#1}\makebox[2em]{{\fontspec{code2000.ttf}#2}}\makebox[2em]{\hfill#3 \\ }}
\fontspec[Script=Armenian,Scale=1.1]{Sylfaen}

\begin{multicols}{4}
\m Ա	A	1\\
\m Բ	B	2\\
\m Գ	G	3\\
\m Դ	D	4\\
\m Ե	E	5\\
\m Զ	Z	6\\
\m Է	ē	7\\
\m Ը	ə	8\\
\m Թ	tʿ	9\\
\m Ժ	ž	10\\
\m Ի	I	20\\
\m Լ	L	30\\
\m Խ	X	40\\
\m Ծ	C	50\\
\m Կ	K	60\\
\m Հ	H	70\\
\m Ձ	J	80\\
\m Ղ	ł	90\\
\m Ճ	č	100\\
\m Մ	M	200\\
\m Յ	Y	300\\
\m Ն	N	400\\
\m Շ	š	500\\
\m Ո	O	600\\
\m Չ	čʿ	700\\
\m Պ	P	800\\
\m Ջ	ǰ	900\\
\m Ռ	ṙ	1000\\ 
\m Ս	S	2000\\
\m Վ	V	3000\\
\m Տ	T	4000\\
\m Ր	R	5000\\
\m Ց	cʿ	6000\\
\m Ւ	W	7000\\
\m Փ	pʿ	8000\\
\m Ք	kʿ	9000\\

\end{multicols}
\captionof{table}{Armenian Numerals \textit{(from Wikipedia).}
The first column is the classical Armenian numeral, the second the transliteration and the third the arabic numeral it represents.}

\medskip

Numbers in the Armenian numeral system are obtained by simple addition. Armenian numerals are written left-to-right (as in the Armenian language). Although the order of the numerals is irrelevant since only addition is performed, the convention is to write them in decreasing order of value.

\begin{align*}
\text{ՌՋՀԵ} &= 1975 = 1000 + 900 + 70 + 5\\
\text{ՍՄԻԲ} &= 2222 = 2000 + 200 + 20 + 2\\
\text{ՍԴ}   &= 2004 = 2000 + 4\\
\text{ՃԻ}   &= 120 = 100 + 20\\
\text{Ծ}    &= 50
\end{align*}

To write numbers greater than 9999, it is necessary to have numerals with values greater than 9000. This is done by drawing a line over them, indicating their value is to be multiplied by 10000:

\begin{align*}
\overline{\text{Ա}} &= 10000\\
\overline{\text{Ջ}} &= 9000000\\
\overline{\text{ՌՃԽԳ}}\text{ՌՄԾԵ} &= 11431255
\end{align*}
\egroup

^^A

\section{Bopomofo}
\label{s:bopomofo}
Bopomofo is the colloquial name of the \textit{zhuyin fuhao} or \textit{zhuyin} system of phonetic notation for the transcription of spoken Chinese, particularly the Mandarin dialect. Consisting of 37 characters and four tone marks, it transcribes all possible sounds in Mandarin. 

Bopomofo was introduced in China by the Republican Government, in the 1910s and used alongside the Wade-Giles system, which used a modified Latin alphabet. The Wade system was replaced by \textit{Hanyu Pinyin} in 1958 by the Government of the People's Republic of China,[1] at the International Organization for Standardization (ISO) in 1982 (ISO 7098:1982). Bopomofo remains widely used as an educational tool and electronic input method in Taiwan. On Windows the font Microsoft JhengHei can be used. 

Windows fonts that can be used \texttt{Microsoft JhengHei} and \texttt{SimSun}.

U+3100–U+312F
\newfontfamily\bopomofo{Microsoft JhengHei}

\begin{scriptexample}[]{Bopomofo}
{\centering\bopomofo 

伯帛勃脖舶博渤霸壩灞

}

\hfill \texttt{Typeset with \cmd{\bopomofo} and Microsoft JhengHei font }
\end{scriptexample}

\begin{scriptexample}[]{Bopomofo}

{\centering\bopomofo

伯帛勃脖舶博渤霸壩灞

}
\hfill \texttt{Typeset with \cmd{\bopomofo} and JhengHei font }
\end{scriptexample}


The Bopomofo Extended block, running from \unicodenumber{U+31A0-U31BF}, contains less universally recognized Bopomofo characters used to write various non-Mandarin Chinese languages. A few additional tone marks are unified with characters in the Spacing Modifier Letters block. 










^^A\newfontfamily\georgian[Script=Georgian,Scale=1.2]{code2000.ttf}

\newfontfamily\georgianarial[Script=Georgian,Scale=1.2]{Arial Unicode MS}
\section{Georgian}
\label{sec:georgian}
The Georgian scripts are the three writing systems used to write the Georgian language: Asomtavruli, Nuskhuri and Mkhedruli. Their letters are equivalent, sharing the same names and alphabetical order and all three are unicameral (make no distinction between upper and lower case). Although each continues to be used, Mkhedruli (see below) is taken as the standard for Georgian and its related Kartvelian languages\footnote{Unicode Standard, V. 6.3. U10A0, p. 3}. 

\bgroup
\topline



\begin{scriptexample}[]{}
\georgian 

\centering
 
ყველა ადამიანი იბადება თავისუფალი და თანასწორი თავისი ღირსებითა და უფლებებით. მათ მინიჭებული აქვთ გონება და სინდისი და ერთმანეთის მიმართ უნდა იქცეოდნენ ძმობის სულისკვეთებით.
\medskip

\georgianarial
ყველა ადამიანი იბადება თავისუფალი და თანასწორი თავისი ღირსებითა და უფლებებით. მათ მინიჭებული აქვთ გონება და სინდისი და ერთმანეთის მიმართ უნდა იქცეოდნენ ძმობის სულისკვეთებით.
\bottomline
\captionof{table}{Article 1 of the Universal Declaration of Human Rights in Georgian, typeset in \texttt{code2000} (top) and \texttt{Arial Unicode MS } (bottom).}

\end{scriptexample}

The scripts originally had 38 letters. Georgian is currently written in a 33-letter alphabet, as five of the letters are obsolete in that language. The Mingrelian alphabet uses 36: the 33 of Georgian, one letter obsolete for that language, and two additional letters specific to Mingrelian and Svan. That same obsolete letter, plus a letter borrowed from Greek, are used in the 35-letter Laz alphabet. The fourth Kartvelian language, Svan, is not commonly written, but when it is it uses the letters of the Mingrelian alphabet, with an additional obsolete Georgian letter and sometimes supplemented by diacritics for its many vowels.

^^A
^^A\section{Malayalam}
\label{sec:malayam}
\newfontfamily\malayam[Scale=1.1]{Lohit-Malayalam.ttf}

\def\malamtext#1{{\malayam#1}}

The Malayalam script (Malayalam: \malamtext{മലയാളലിപി}, Malayāḷalipi, IPA: [mɐləjaːɭɐ lɪβɪ], also known as Kairali script (Malayalam: \malamtext{കൈരളീലിപി}), is a Brahmic script used commonly to write the Malayalam language—which is the principal language of the Indian state of Kerala, spoken by 35 million people in the world.[3] Like many other Indic scripts, it is an alphasyllabary (\textit{abugida}), a writing system that is partially “alphabetic” and partially syllable-based. The modern Malayalam alphabet has 15 vowel letters, 41 consonant letters, and a few other symbols. The Malayalam script is a Vattezhuttu script, which had been extended with Grantha script symbols to represent Indo-Aryan loanwords.[4] The script is also used to write several minority languages such as Paniya, Betta Kurumba, and Ravula.[5] The Malayalam language itself was historically written in several different scripts.

\begin{scriptexample}[]{Malayalam}
\centerline{\Huge\malamtext{കൈരളീലിപി}}
\end{scriptexample}
^^A\subsection{Greek}
\index{languages>Greek}\index{Herodotus}\index{alphabets>Greek}
\newfontfamily\greek[Script=Greek,Scale=1.02]{NotoSerif-Regular.ttf}
\def\greektext#1{\greek{#1}}

`The Phoenicians who came with Kadmos,' wrote Herodotus in the fifth century BC of the legendary Phoenician prince of Tyre and brother of Europa, `\ldots introduced into Greece, after their settlement in the country, a number of accomplishments of which the most important was writing, an art which probably was unknown to the Greeks until then'. 

The Greek alphabet is the script that has been used to write the Greek language since the 8th century BC.[2] It was derived from the earlier Phoenician alphabet, and was in turn the ancestor of numerous other European and Middle Eastern scripts, including Cyrillic and Latin.[3] Apart from its use in writing the Greek language, both in its ancient and its modern forms, the Greek alphabet today also serves as a source of technical symbols and labels in many domains of mathematics, science and other fields.

In its classical and modern forms, the alphabet has 24 letters, ordered from alpha to omega. Like Latin and Cyrillic, Greek originally had only a single form of each letter; it developed the letter case distinction between upper-case and lower-case forms in parallel with Latin during the modern era.

\bgroup
\greek\obeyspaces

Α	ἄλφα	aleph	alpha	[alpʰa]	[ˈalfa]	Listeni/ˈælfə/
Β	βῆτα	beth	beta	[bɛːta]	[ˈvita]	/ˈbiːtə/, US /ˈbeɪtə/
Γ	γάμμα	gimel	gamma	[ɡamma]	[ˈɣama]	/ˈɡæmə/
Δ	δέλτα	daleth	delta	[delta]	[ˈðelta]	/ˈdɛltə/
Η	ἦτα	  heth	   eta	 [hɛːta], [ɛːta]	[ˈita]	/ˈiːtə/, US /ˈeɪtə/
Θ	θῆτα	teth	theta	[tʰɛːta]	[ˈθita]	/ˈθiːtə/, US Listeni/ˈθeɪtə/
Ι	ἰῶτα	yodh	iota	[iɔːta]	[ˈʝota]	Listeni/aɪˈoʊtə/
Κ	κάππα	kaph	kappa	[kappa]	[ˈkapa]	Listeni/ˈkæpə/
Λ	λάμβδα	lamedh	lambda	[lambda]	[ˈlamða]	Listeni/ˈlæmdə/
Μ	μῦ	mem	mu	[myː]	[mi]	Listeni/ˈmjuː/; occasionally US /ˈmuː/
Ν	νῦ	nun	nu	[nyː]	[ni]	/ˈnjuː/ (US /ˈnuː/)
Ρ	ῥῶ	reš	rho	[rɔː]	[ro]	Listeni/ˈroʊ/
Τ	ταῦ	taw	tau	[tau]	[taf]	/ˈtaʊ/ or /ˈtɔː/

\topline
\begin{quote}
Ἡροδότου Ἁλικαρνησσέος ἱστορίης ἀπόδεξις ἥδε, ὡς μήτε τὰ γενόμενα ἐξ ἀνθρώπων τῷ χρόνῳ ἐξίτηλα γένηται, μήτε ἔργα μεγάλα τε καὶ θωμαστά, τὰ μὲν Ἕλλησι, τὰ δὲ βαρβάροισι ἀποδεχθέντα, ἀκλεᾶ γένηται, τὰ τε ἄλλα καὶ δι' ἣν αἰτίην ἐπολέμησαν ἀλλήλοισι.[2]

Herodotus of Halicarnassus, his Researches are set down to preserve the memory of the past by putting on record the astonishing achievements of both the Greeks and the Barbarians; and more particularly, to show how they came into conflict.[3]
\end{quote}
\bottomline

\symbol{"1F00}
\symbol{"1F01}
\egroup
^^A
^^A\subsection{Kannada alphabet}

\newfontfamily\kannada[Scale=1.0,Script=Kannada]{Lohit-Kannada.ttf}

\def\kannadatext#1{{\kannada#1}}

The Kannada alphabet (\kannadatext{ಕನ್ನಡ ಲಿಪಿ}) is an abugida of the Brahmic family,[2] used primarily to write the Kannada language, one of the Dravidian languages of southern India. Several minor languages, such as Tulu, Konkani, Kodava, and Beary, also use alphabets based on the Kannada script.[3] The Kannada and Telugu scripts share high mutual intellegibility with each other, and are often considered to be regional variants of single script. Similarly, Goykanadi, a variant of Old Kannada, has been historically used to write Konkani in the state of Goa.[4]

\begin{scriptexample}[]{Kannada}
\centerline{\LARGE\kannadatext{ಙ	ಙ್ಕ	ಙ್ಖ	ಙ್ಗ	ಙ್ಘ	ಙ್ಙ	ಙ್ಚ	ಙ್ಛ	ಙ್ಜ	ಙ್ಝ	ಙ್ಞ	ಙ್ಟ	ಙ್ಠ	ಙ್ಡ	ಙ್ಢ}}
\end{scriptexample}

\medskip

The Kannada script (aksharamale or varnamale) is a phonemic abugida of forty-nine letters, and is written from left to right. The character set is almost identical to that of other Brahmic scripts. Consonantal letters imply an inherent vowel. Letters representing consonants are combined to form digraphs (ottaksharas) when there is no intervening vowel. Otherwise, each letter corresponds to a syllable.
The letters are classified into three categories: swara (vowels), vyanjana (consonants), and yogavaahaka (part vowel, part consonant).
The Kannada words for a letter of the script are akshara, akkara, and varna. Each letter has its own form (ākāra) and sound (shabda), providing the visible and audible representations, respectively. Kannada is written from left to right.[7]
^^A\section{Myanmar}
\label{s:myanmar}
\index{Myanmar}\index{Burmese}\index{Mon}\index{Unicode>Myanmar}\index{Fonts>Padauk}

%\newfontfamily\myanmar{Padauk}

The Burmese script (Burmese:{\myanmar မြန်မာအက္ခရာ}; MLCTS: mranma akkha.ra; pronounced: [mjəmà ʔɛʔkʰəjà]) is an abugida in the Brahmic family, used for writing Burmese. It is an adaptation of the Old Mon script[2] or the Pyu script. In recent decades, other alphabets using the Mon script, including Shan and Mon itself, have been restructured according to the standard of the now-dominant Burmese alphabet. Besides the Burmese language, the Burmese alphabet is also used for the liturgical languages of Pali and Sanskrit.

The characters are rounded in appearance because the traditional palm leaves used for writing on with a stylus would have been ripped by straight lines.[3] It is written from left to right and requires no spaces between words, although modern writing usually contains spaces after each clause to enhance readability.

The earliest evidence of the Burmese alphabet is dated to 1035, while a casting made in the 18th century of an old stone inscription points to 984.[1] Burmese calligraphy originally followed a square format but the cursive format took hold from the 17th century when popular writing led to the wider use of palm leaves and folded paper known as parabaiks.[3] The alphabet has undergone considerable modification to suit the evolving phonology of the Burmese language.

Mon/Burmese script was added to the Unicode Standard in September, 1999 with the release of version 3.0. It was extended in October, 2009 with the release of version 5.2 and again in June, 2014 with the release of version 7.0.

\begin{docKey}[phd]{myanmar font}{=\meta{font name}}{default none initial Padauk}
Loads the font and creates associated environments and commands.
\end{docKey}

\begin{scriptexample}[]{Myanmar}
\unicodetable{myanmar}{"1000,"1010,"1020,"1030,"1040,"1050,"1060,"1070,"1080,"1090}
\end{scriptexample}







^^A
^^A\subsection{Osmanian Alphabet}

\bgroup
\newfontfamily\osmanian{code2001.ttf}
\osmanian
𐒚𐒁𐒖𐒄 𐒚𐒐 𐒚 𐒎𐒚𐒍𐒚𐒐 𐒑𐒚𐒒𐒠𐒚𐒐 𐒎𐒚𐒑𐒁𐒗 𐒚𐒁𐒖𐒄 𐒚𐒌𐒖𐒄 𐒚𐒁𐒖𐒄𐒖 𐒚
𐒌𐒜
\egroup
^^A\newfontfamily\hanunoo{NotoSansHanunoo-Regular.ttf}

\section{Hanunó’o}

Hanunó’o is one of the indigenous scripts of the Philippines and is used by the Mangyan peoples of southern Mindoro to write the Hanunó'o language.[1] 

It is an \emphasis{abugida} descended from the Brahmic scripts, closely related to Baybayin, and is famous for being written vertical but written upward, rather than downward as nearly all other scripts (however, it's read horizontally left to right). It is usually written on bamboo by incising characters with a knife.[2][3] Most known Hanunó'o inscriptions are relatively recent because of the perishable nature of bamboo. It is therefore difficult to trace the history of the script



\begin{scriptexample}[width=2cm]{Hanunoo}
\hanunoo

{\Large
\obeylines
ᜠ 
ᜫ
ᜨᜲ
ᜫᜲ
ᜰ
ᜮ
ᜥ
ᜦ᜴}

Typeset with \texttt{NotoSansHanunoo-Regular.ttf} and the command \cmd{\hanunoo}
\end{scriptexample}

Vertically positionning the text is not currently supported by \pkgname{fontspec} and the manual says \textsc{Todo!}. You are your own here, or you can just put the characters in a box and give it a try.

\begin{minipage}[t]{2cm}
\begin{tcolorbox}[width=2cm,colback=graphicbackground,
boxrule=0pt,toprule=0pt,colframe=white]
\Large\hanunoo
ᜩ\\
ᜤ\\
ᜮ\\
ᜥᜳ\\
ᜨ᜴ \\
ᜨ᜴\\
ᜫᜳ\\
ᜥ\\
\end{tcolorbox}
\end{minipage}
\begin{minipage}[t]{2cm}
\begin{tcolorbox}[width=2cm,colback=graphicbackground,
boxrule=0pt,toprule=0pt,colframe=white]
\LARGE\hanunoo
ᜩ\\
ᜤ\\
ᜮ\\
ᜥᜳ\\
ᜨ᜴ \\
ᜨ᜴\\
ᜫᜳ\\
ᜥ\\
\end{tcolorbox}
\end{minipage}
\begin{minipage}[t]{\textwidth-6cm}

The script is written from bottom to top. Typesetting this type of script automatically is not without its problems. One way is to use the build-in features of the font if they are available, but currently this gives problems---at least with the fonts that I have tried. Entering the text is also problematic as you will more than likely see little boxes rather than the actual glyph with most text editors common to \latexe. If you only need a couple of characters or a short sentence, an easy solution is to use |\rotatebox|. Another solution is to use a macro that can add the letters onto a stack, then place them in a box with a limited width. We can use |\@tfor| for this.  
\end{minipage}
^^A
^^A\newfontfamily\glagolitic{MPH 2B Damase}

\section{Glagolitic}

\epigraph{The average Ph.D. thesis is nothing but a transference of bones from one graveyard to another.}{%
J. Frank Dobie (1888-1964)}


\label{s:glagolitic}
\fboxrule0pt\fboxsep0pt

\noindent
The Glagolitic alphabet /{\glagolitic ˌɡlæɡɵˈlɪtɨk/}, also known as Glagolitsa, is the oldest known Slavic alphabet, from the 9th century.

It was created in the 9th century by Saint Cyril, a Byzantine monk from Thessaloniki. He and his brother, Saint Methodius, were sent by the Byzantine Emperor Michael III in 863 to Great Moravia to spread Christianity among the West Slavs in the area. The brothers decided to translate liturgical books into the Old Slavic language that was understandable to the general population, but as the words of that language could not be easily written by using either the Greek or Latin alphabets, Cyril decided to invent a new script, Glagolitic, which he based on the local dialect of the Slavic tribes from the Byzantine Salonika region.
After the deaths of Cyril and Methodius, the Glagolitic alphabet ceased to be used in Moravia, but their students continued to propagate it in the west and south. 

After a long career, Glagolitic writing stopped being used, except for
religious purposes in certain dioceses of Bosnia and Dalmatia (Croatia).
The Cyrillic alphabet was adopted by all Orthodox Slays and served to note
their literary language. Most of the Slays who rallied to Rome rejected it,
however, which created the paradoxical situation in ex-Yugoslavia, where
two peoples who speak the same language write in different scripts, the
Serbs in Cyrillic and the Croats with Roman characters. Finally, as is
known, the ex-Soviet Union did much to put into writing the languages
spoken by the peoples within its borders, for the most part noting them in
adaptations of the Cyrillic alphabet, while Russian became the language of
culture throughout the Soviet Union.\cite{henri1994}

Slavic printing in Glagolitic characters originated in Venice, where a
\textit{Sluzebnik} (or \textit{Leitourgikon}) was published in 1483, followed by missals and
breviaries, all printed by Andrea Torresani, the future father-in-law and
associate of Aldus Manutius. After 1494 some attempts were made to create
printshops in Croatia itself, first in Senj in 1508, then, after 1530, in
Rijeka (Fiume). The work of these firms was almost totally liturgical (religious,
at any rate), and it had strong competition from manuscript works
that were better adapted to the diversity of local liturgical customs. Religion
also dictated the output of a printshop founded to provide Protestant propaganda
that was set up in Tubingen between 1560 and 1564 by Baron
Hans von Ungnad and that printed the great Lutheran texts in Glagolitic
characters.\footfullcite{henri1994}

Figure~\ref{fig:zograf} illustrates an example of the language.\footnote{\url{https://en.wikipedia.org/wiki/Glagolitic_script\#/media/File:ZographensisColour.jpg}}

\begin{figure}[htbp]
\centering

\includegraphics[width=0.45\linewidth]{glagolitic}
\caption[The first page of the Gospel of Mark from the 10th–11th century Codex Zographensis, found in the Zograf Monastery in 1843.]{The first page of the Gospel of Mark from the 10th–11th century Codex Zographensis, found in the Zograf Monastery in 1843.}
\label{fig:zograf}
\end{figure}

\section{Unicode Support}
The Glagolitic alphabet was added to the Unicode Standard in March 2005 with the release of version 4.1.
The Unicode block for Glagolitic is U+2C00–U+2C5F.



\begin{scriptexample}[]{glacolitic}

\unicodetable{glagolitic}{%
"2C00,"2C10,"2C20,"2C30,"2C40,"2C50}

\texttt{typeset with Damase version 2.0 MPH 2B Damase}
\end{scriptexample}
\bgroup
\glagolitic

The name was not coined until many centuries after its creation, and comes from the Old Church Slavonic glagolъ "utterance" (also the origin of the Slavic name for the letter G). The verb glagoliti means "to speak". It has been conjectured that the name glagolitsa developed in Croatia around the 14th century and was derived from the word glagolity, applied to adherents of the liturgy in Slavonic.[1]

In Old Church Slavonic the name is {\glagolitic ⰍⰫⰓⰊⰎⰎⰑⰂⰋⰜⰀ}, Кѷрїлловица.
The name Glagolitic in Bulgarian, Russian, Macedonian глаголица (glagolica), Belarusian is глаголіца (hłaholica), Croatian glagoljica, Serbian глагољица / glagoljica, Czech hlaholice, Polish głagolica, Slovene glagolica, Slovak hlaholika, and Ukrainian глаголиця (hlaholyća).



\egroup

\section{Additional Modern Scripts}

\begin{center}
\begin{tabular}{lp{5cm}l}
Ethiopic. &Vai. &Deseret.\\
Mongolian. &Bamum. &Shavian.\\
Osmanya.   &Cherokee. &Lisu.\\
Tifinagh.  &Canadian Aboriginal Syllabics. &Miao.\\
N’Ko.&&\\
\end{tabular}
\end{center}

Ethiopic, Mongolian, and Tifinagh are scripts with long histories. Although their roots can
be traced back to the original Semitic and North African writing systems, they would not
be classified as Middle Eastern scripts today

The Cherokee script is a syllabary developed between 1815 and 1821, to write the Cherokee
language, still spoken by small communities in Oklahoma and North Carolina. Canadian
Aboriginal Syllabics were invented in the 1830s for Algonquian languages in Canada. The
system has been extended many times, and is now actively used by other communities, including speakers of Inuktitut and Athapascan languages.

Deseret is a phonemic alphabet devised in the 1850s to write English. It saw limited use for
a few decades by members of The Church of Jesus Christ of Latter-day Saints. Shavian is
another phonemic alphabet, invented in the 1950s to write English. It was used to publish
one book in 1962, but remains of some current interest




\subsection{Ethiopic}
Ge'ez (ግዕዝ Gəʿəz), (also known as Ethiopic) is a script used as an abugida (syllable alphabet) for several languages of Ethiopia and Eritrea. It originated as an abjad (consonant-only alphabet) and was first used to write Ge'ez, now the liturgical language of the Ethiopian Orthodox Tewahedo Church and the Eritrean Orthodox Tewahedo Church. In Amharic and Tigrinya, the script is often called fidäl (ፊደል), meaning "script" or "alphabet".

The Ge'ez script has been adapted to write other, mostly Semitic, languages, particularly Amharic in Ethiopia, and Tigrinya in both Eritrea and Ethiopia. It is also used for Sebatbeit, Me'en, and most other languages of Ethiopia. In Eritrea it is used for Tigre, and it has traditionally been used for Blin, a Cushitic language. Tigre, spoken in western and northern Eritrea, is considered to resemble Ge'ez more than do the other derivative languages.[citation needed] Some other languages in the Horn of Africa, such as Oromo, used to be written using Ge'ez, but have migrated to Latin-based orthographies.
For the representation of sounds, this article uses a system that is common (though not universal) among linguists who work on Ethiopian Semitic languages. This differs somewhat from the conventions of the International Phonetic Alphabet. See the articles on the individual languages for information on the pronunciation.

There are a number of fonts available and we have selected the Google \idxfont{NotoSansEthiopic}
\newfontfamily\ethiopic{NotoSansEthiopic-Bold.ttf}

\begin{scriptexample}[]{Ethiopic}
\unicodetable{ethiopic}{"1200,"1210,"1220,"1230,"1240,"1250,"1260,"1270,"1280,"1290,^^A
"12A0,"12B0,"12C0,"12E0,"12F0,"1300,"1310,"1330,"1340,"1350,"1360,"1370}
\end{scriptexample}
\section{Vai}
\label{s:vai}

The Vai syllabary is a syllabic writing system devised for the Vai language by Momolu Duwalu Bukele of Jondu, in what is now Grand Cape Mount County, Liberia.[1] [2] Bukele is regarded within the Vai community, as well as by most scholars, as the syllabary's inventor and chief promoter when it was first documented in the 1830s. It is one of the two most successful indigenous scripts in West Africa.

\newfontfamily\vai{code2000.ttf}
\begin{scriptexample}[]{Vai}
\unicodetable{vai}{"A500,"A510,"A520,"A530,"A540,"A550,"A560,"A570,^^A
"A580,"A590,"A5A0,"A5B0,^^A
"A5C0,"A5D0,"A5E0,"A5F0,"A610,"A620,"A630}
\end{scriptexample}

In the 1920s ten decimal digits were devised for Vai; these were “Vai-style” glyph variants of
European digits (see Figure 11). They were not popular with Vai people  even for historical purposes. All
the modern literature uses European digits.


\begin{scriptexample}[]{Vai}
\bgroup
\vai
\obeylines\Large
0	1	2	3	4	5	6	7	8	9
꘠	꘡	꘢	꘣	꘤	꘥	꘦	꘧	꘨	꘩
\vai
\egroup
\end{scriptexample}



\printunicodeblock{./languages/vai.txt}{\vai}
\section{Deseret script}
\newfontfamily\deseret{code2001.ttf}

The Deseret alphabet (dɛz.əˈrɛt.) (Deseret: {\deseret 𐐔𐐯𐑅𐐨𐑉𐐯𐐻 or 𐐔𐐯𐑆𐐲𐑉𐐯𐐻}) is a phonemic English spelling reform developed in the mid-19th century by the board of regents of the University of Deseret (later the University of Utah) under the direction of Brigham Young, second president of The Church of Jesus Christ of Latter-day Saints.

In public statements, Young claimed the alphabet was intended to replace the traditional Latin alphabet with an alternative, more phonetically accurate alphabet for the English language. This would offer immigrants an opportunity to learn to read and write English, he said, the orthography of which is often less phonetically consistent than those of many other languages. Similar experiments were not uncommon during the period, the most well-known of which is the Shavian alphabet.

Young also prescribed the learning of Deseret to the school system, stating "It will be the means of introducing uniformity in our orthography, and the years that are now required to learn to read and spell can be devoted to other studies".[2]


Deseret script {\deseret 𐐔𐐯𐑅𐐨𐑉𐐯𐐻}  [U+10400-U+1044F]
\medskip

\bgroup
\par
\noindent
\colorbox{graphicbackground}{\color{black}^^A
\begin{minipage}{\textwidth}^^A
\parindent1pt
\vskip10pt
\leftskip10pt \rightskip\leftskip
\deseret
\large

𐐂 𐑌𐐲𐑉𐑅𐐨𐑉𐐮 𐐮𐑆 𐐪 𐐹𐐨𐑅 𐐱𐑂 𐑊𐐰𐑌𐐼 𐐱𐑌 𐐸𐐶𐐮𐐽 𐑁𐑉𐐭𐐻𐐻𐑉𐐨𐑆 𐐪𐑉 𐑅𐐻𐐪𐑉𐐻𐐯𐐼,


\par
\vspace*{10pt}
\end{minipage}
}

Text: Deseret alphabet http://www.omniglot.com/writing/deseret.htm
\medskip
\egroup

\PrintUnicodeBlock{./languages/deseret.txt}{\deseret}

\chapter{Bamum}
\label{s:bamum}
\epigraph{"No known alphabet was ever invented by a European."}{Jeffreys' translation from the Royal script.}

\label{s:bamum}
\index{scripts>Bamum}
\newfontfamily\bamum{NotoSansBamum-Regular.ttf}

The Bamum scripts are an evolutionary series of six scripts created for the Bamum language by King Njoya of Cameroon at the turn of the 20th century. They are notable for evolving from a pictographic system to a partially alphabetic syllabic script in the space of 14 years, from 1896 to 1910. Bamum type was cast in 1918, but the script fell into disuse around 1931.

\begin{figure}[htbp]
\parindent=0pt

\centering

\includegraphics[width=\textwidth]{bamum}

\caption{King Njoya of Bamum receiving an oil painting of Kaiser Wilhelm II. The gift was in return for his support in the German campaign against the Nso'.}
\end{figure}

The Bamum, sometimes called Bamoum, Bamun, Bamoun, or Mum, are a Bantoid ethnic group of Cameroon with around 215,000 members.



\begin{scriptexample}[]{Bamum}
\unicodetable{bamum}{"A6A0,"A6B0,"A6C0,"A6D0,"A6E0,"A6F0}
\end{scriptexample}
\section{Shavian}
\label{s:shavian}
\def\shaviansetup#1{}
\newfontfamily\shavian{code2001.ttf}
^^A\newfontfamily\shavian{NotoSansShavian-Regular.ttf}
\cxset{shavian font/.code=\shaviansetup{#1}}
\cxset{shavian font=shavian}




\begin{scriptexample}[]{shavian}
\shavian

𐑳 𐑡𐑻𐑯𐑰 𐑑 𐑞 𐑕𐑧𐑯𐑑𐑻 𐑝 𐑞 𐑻𐑔
𐑚𐑲 - ·𐑡𐑵𐑤𐑟 ·𐑝𐑻𐑯

𐑗𐑩𐑐𐑑𐑻 1 - 𐑥𐑲 𐑳𐑙𐑒𐑳𐑤 𐑥𐑱𐑒𐑕 𐑳 𐑜𐑮𐑱𐑑 𐑛𐑦𐑕𐑒𐑳𐑝𐑻𐑰

     𐑤𐑫𐑒𐑦𐑙 𐑚𐑩𐑒 𐑑 𐑷𐑤 𐑞𐑩𐑑 𐑣𐑩𐑟 𐑳𐑒𐑻𐑛 𐑑 𐑥𐑰 𐑕𐑦𐑯𐑕 𐑞𐑩𐑑 𐑦𐑝𐑧𐑯𐑑𐑓𐑳𐑤 𐑛𐑱, 𐑲 𐑩𐑥 𐑕𐑒𐑧𐑮𐑕𐑤𐑰 𐑱𐑚𐑳𐑤 𐑑 𐑚𐑦𐑤𐑰𐑝 𐑦𐑯 𐑞 𐑮𐑰𐑩𐑤𐑳𐑑𐑰 𐑝 𐑥𐑲 𐑩𐑛𐑝𐑧𐑯𐑗𐑻𐑟. 𐑞𐑱 𐑢𐑻 𐑑𐑮𐑵𐑤𐑰 𐑕𐑴 𐑢𐑳𐑯𐑛𐑻𐑓𐑳𐑤 𐑞𐑩𐑑 𐑰𐑝𐑦𐑯 𐑯𐑬 𐑲 𐑩𐑥 𐑚𐑦𐑢𐑦𐑤𐑛𐑻𐑛 𐑢𐑧𐑯 𐑲 𐑔𐑦𐑙𐑒 𐑝 𐑞𐑧𐑥.
     𐑥𐑲 𐑳𐑙𐑒𐑳𐑤 𐑢𐑪𐑟 𐑳 𐑡𐑻𐑥𐑳𐑯, 𐑣𐑩𐑝𐑦𐑙 𐑥𐑧𐑮𐑰𐑛 𐑥𐑲 𐑥𐑳𐑞𐑻𐑟 𐑕𐑦𐑕𐑑𐑻, 𐑩𐑯 𐑦𐑙𐑜𐑤𐑦𐑖𐑢𐑫𐑥𐑳𐑯. 𐑚𐑰𐑦𐑙 𐑝𐑧𐑮𐑰 𐑥𐑳𐑗 𐑳𐑑𐑩𐑗𐑑 𐑑 𐑣𐑦𐑟 𐑓𐑪𐑞𐑻𐑤𐑳𐑕 𐑯𐑧𐑓𐑘𐑵, 𐑣𐑰 𐑦𐑯𐑝𐑲𐑑𐑳𐑛 𐑥𐑰 𐑑 𐑕𐑑𐑳𐑛𐑰 𐑳𐑯𐑛𐑻 𐑣𐑦𐑥 𐑦𐑯 𐑣𐑦𐑟 𐑣𐑴𐑥 𐑦𐑯 𐑞 𐑓𐑪𐑞𐑻𐑤𐑩𐑯𐑛. 𐑞𐑦𐑕 𐑣𐑴𐑥 𐑢𐑪𐑟 𐑦𐑯 𐑳 𐑤𐑪𐑮𐑡 𐑑𐑬𐑯, 𐑯 𐑥𐑲 𐑳𐑙𐑒𐑳𐑤 𐑳 𐑐𐑮𐑳𐑓𐑧𐑕𐑻 𐑝 𐑓𐑳𐑤𐑪𐑕𐑳𐑓𐑰, 𐑒𐑧𐑥𐑳𐑕𐑑𐑮𐑰, 𐑡𐑰𐑪𐑤𐑳𐑡𐑰, 𐑥𐑦𐑯𐑻𐑪𐑤𐑳𐑡𐑰, 𐑯 𐑥𐑧𐑯𐑰 𐑳𐑞𐑻 𐑳𐑤𐑴𐑡𐑰𐑕.

\arial

\hfill Excerpt from Jules Vern,  \textit{Journey to the Center of the Earth from \href{http://shavian.weebly.com/}{shavian}}
\end{scriptexample}

The example is typeset using \texttt{code2001.ttf}. There are numerous fonts that provide Shavian glyphs. \texttt{ESL Gothic Unicode} font by Ethan Lamoreaux\footnote{\url{http://www.fontspace.com/ethan-lamoreaux/esl-gothic-unicode}}. The Noto fonts also have a Shavian font. 

You can activate typesetting in Shavian using the key:

\begin{key}{/chapter/shavian font = \meta{font name}} The key will setup the
default font for the Shavian script and define the commands \cmd{\shavian} and \cmd{\textshavian}. 
\end{key}

\PrintUnicodeBlock{./languages/shavian.txt}{\shavian}





\subsection{Osmanya}

\newfontfamily\osmanya{NotoSansOsmanya-Regular.ttf}

\begin{scriptexample}[]{Osmanya}
\unicodetable{osmanya}{"10480,"10490,"104A0}
\end{scriptexample}

The Osmanya alphabet (Somali: Cismaanya; Osmanya: {\osmanya 𐒋𐒘𐒈𐒑𐒛𐒒𐒕𐒀}), also known as Far Soomaali ("Somali writing"), is a writing script created to transcribe the Somali language. It was invented between 1920 and 1922 by Osman Yusuf Kenadid of the Majeerteen Darod clan, the nephew of Sultan Yusuf Ali Kenadid of the Sultanate of Hobyo.

While Osmanya gained reasonably wide acceptance in Somalia and quickly produced a considerable body of literature, it proved difficult to spread among the population mainly due to stiff competition from the long-established Arabic script as well as the emerging Somali alphabet developed by the Somali linguist, Shire Jama Ahmed, which was based on the Latin script.

As nationalist sentiments grew and since the Somali language had long lost its ancient script,[1] the adoption of a universally recognized writing script for the Somali language became an important point of discussion. After independence, little progress was made on the issue, as opinion was divided over whether the Arabic or Latin scripts should be used instead.

In October 1972, due to its simplicity, the fact that it lent itself well to writing Somali since it could cope with all of the sounds in the language, and the already widespread existence of machines and typewriters designed for its use,[2][3] the government of Somali president Mohamed Siad Barre unilaterally elected to use only the Latin script for writing Somali instead of the Arabic or Osmanya scripts.[4] Barre's administration subsequently launched a massive literacy campaign designed to ensure its sole adoption. This led to a sharp decline in use of Osmanya.
\section{Cherokee}
\index{scripts>Cherokee}
\index{scripts>Cherokee>fonts}
\label{sec:cherokee}
Windows comes with |Plantagenet Cherokee| font. The |code2000| also has good support for the alphabet. The \texttt{SIL font Charis SIL} also has good support and can be downloaded at \href{http://scripts.sil.org/cms/scripts/page.php?item_id=CharisSIL_download}{scripts.sel.org}, the latest version gave me problems when used with Windows. 

  
\def\textcherokee#1{{\cherokee   #1}}


\begin{docKey}[phd]{cherokee font}{ = \meta{font name}} {default none, initial=code2000}
 Loads the font
command \cmd{\cherokee}. When the command is used it typesets text in
cherokee unicode. There is no need to load the language, unless it is the main document language. For windows the default font is  |Plantagenet Cherokee|. Another font is FreeSerif, which we are using here.
\end{docKey}

\begin{scriptexample}[]{Cherokee}
{\cherokee
\begin{tabular}{lp{8.5cm}}
Translation	  &John (ᏣᏂ) 3:16\\
American Bible Society 1860	&ᎾᏍᎩᏰᏃ ᏂᎦᎥᎩ ᎤᏁᎳᏅᎯ ᎤᎨᏳᏒᎩ ᎡᎶᎯ, ᏕᏅᏲᏒᎩ ᎤᏤᎵᎦ ᎤᏪᏥ ᎤᏩᏒᎯᏳ ᎤᏕᏁᎸᎯ, ᎩᎶ ᎾᏍᎩ ᏱᎪᎯᏳᎲᏍᎦ ᎤᏲᎱᎯᏍᏗᏱ ᏂᎨᏒᎾ, ᎬᏂᏛᏉᏍᎩᏂ ᎤᏩᏛᏗ.\\

(Transliteration)	& nasgiyeno nigavgi unelanvhi ugeyusvgi elohi, denvyosvgi utseliga uwetsi uwasvhiyu udenelvhi, gilo nasgi yigohiyuhvsga uyohuhisdiyi nigesvna, gvnidvquosgini uwadvdi.\\
\end{tabular}}
\end{scriptexample}

\begin{texexample}{Using text...}{cherokee}
\bgroup
\cherokee \large\textbf{ᎾᏍᎩᏰᏃ}
\textcherokee{ᎾᏍᎩᏰᏃ}
\egroup
\end{texexample}

If you have trouble getting them to work\footnote{\url{http://tex.stackexchange.com/questions/132087/displaying-cherokee-text}}

\url{http://www.cherokee.org/AboutTheNation/Language/CherokeeFont.aspx}




\section{Tifnagh}

\newfontfamily\tifinagh{code2000.ttf}

Tifinagh (Berber pronunciation: [tifinaɣ]; also written Tifinaɣ in the Berber Latin alphabet, {\tifinagh  ⵜⵉⴼⵉⵏⴰⵖ} in Neo-Tifinagh, and تيفيناغ in the Berber Arabic alphabet) is a series of abjad and alphabetic scripts used by Berber peoples to write Berber languages.[1]
A modern derivate of the traditional script, known as Neo-Tifinagh, was introduced in the 20th century. A slightly modified version of the traditional script, called Tifinagh Ircam, is used in a number of Moroccan elementary schools in teaching the Berber language to children as well as a number of publications.[2][3]

The word tifinagh is thought to be a Berberized feminine plural cognate of Punic, through the Berber feminine prefix ti- and Latin Punicus; thus tifinagh could possibly mean "the Phoenician (letters)"[4][5] or "the Punic letters".

\bgroup

\noindent\tifinagh
\colorbox{thecodebackground}{\color{black}^^A
\begin{minipage}{\textwidth}
\parindent1pt
\vskip10pt
\leftskip10pt \rightskip\leftskip
Tifnagh     ⵜⵉⴼⵉⵏⴰⵖ [U+2D30-U+2D7F]

ⴰⴳⵍⴷⵓⵏ ⴰⵎⵥⵥⴰ

ⵙ ⵡⴰⵡⴰⵍ ⴳⵔⵉ ⵉⴷⵙ, ⵙⵙⵏⵖ ⵢⴰⵜ ⵜⵖⴰⵡⵙⴰ ⵜⵉⵙⵙ ⵙⵏⴰⵜ  ⵉⵅⴰⵜⵔⵏ: ⵉⵜⵔⵉ ⵙⴳ ⴷⴷ ⵉⴷⴷⴰ ⵓⵔ ⵉⵎⵇⵇⵓⵔ, ⵉⵍⵍⴰ ⵖⴰⵙ ⴰⵏⵛⵜ ⵏ ⵢⴰⵜ ⵜⴰⴷⴷⴰⵔⵜ !

ⴰⵢⴰ ⵓⴽⵣⵖ ⵜ. ⵙⵙⵏⵖ ⵉⵙ ⴱⵕⵕⴰ ⵏ ⵉⵜⵔⴰⵏ ⵣⵓⵏⴷ ⴰⴽⴰⵍ, ⵊⵓⴱⵉⵜⵔ, ⵎⴰⵔⵙ, ⴱⵉⵏⵓⵙ – ⵉⵜⵔⴰⵏ ⵎⵉ ⵏⴽⴼⴰ ⵉⵙⵎⴰⵡⵏ – ⵍⵍⴰⵏ ⴷⵉⵖ ⵉⵜⵔⴰⵏ ⵢⴰⴹⵏ ⵎⵥⵥⵉⵢⵏⵉⵏ, ⵡⵉⵏⵏⴰ ⵓⵔ ⵏⵣⵎⵉⵔ ⴰⴷ ⵏⵥⵔ ⵙ ⵓⵜⵉⵍⵉⵙⴽⵓⴱ. ⴰⴷⴷⴰⵢ ⵢⵓⴼⴰ ⵓⴰⵙⵜⵕⵓⵏⵓⵎ ⵢⴰⵏ ⴷⵉⴳⵙⵏ, ⴷⴰ ⵢⴰⵙ ⵉⵜⵜⴳⴰ ⵙ ⵢⵉⵙⵎ ⵢⴰⵏ ⵡⵓⵜⵜⵓⵏ. ⴷⴰ ⵢⴰⵙ ⵉⵇⵇⴰⵔ ⵙ ⵓⵎⴷⵢⴰⵜ : « ⴰⵙⵜⵔⵓⵉⴷ 3251 ».

ⵓⴽⵣⵖ ⵉⵙ ⴷⴷ ⵉⴷⴷⴰ ⵓⴳⵍⴷⵓⵏ ⵎⵥⵥⵉⵢⵏ ⵙⴳ ⵉⵜⵔⵉ ⵎⵉ ⵇⵇⴰⵔⵏ ⴰⵙⵜⵔⵓⵉⴷ ⴱ612. ⴰⵙⵜⵔⵓⵉⴷ ⴰ, ⵓⵔ ⵉⵜⵓⵥⵔⴰ ⴰⵔ 1909 ⵙ ⵓⵜⵉⵍⵉⵙⴽⵓⴱ. ⵉⵥⵔⴰ ⵜ ⵢⴰⵏ ⵓⴰⵙⵜⵕⵓⵏⵓⵎ ⴰⵜⵓⵔⴽⵉⵢ. ⵉⵙⵙⴽⵏ ⵜⵓⴼⴰⵢⵜ ⵏⵏⵙ ⴳ ⵢⴰⵏ ⵓⴳⵔⴰⵡ ⴰⴳⵔⴰⵖⵍⴰⵏ ⵏ ⵍⴰⵙⵜⵕⵓⵏⵓⵎⵢ. ⵎⴰⵛⴰ, ⴰⴽⴷ ⵢⵉⵡⵏ ⵓⵔ ⵜ ⵢⵓⵎⵏ ⴰⵛⴽⵓ ⵉⵍⵍⴰ ⵉⵍⵙⴰ ⵢⴰⵜ ⵎⵍⵙⵉⵡⵜ ⵓⵔ ⵉⴳⵉⵏ ⴰⵎⵎ ⵜⵉⵏ ⵎⴷⴷⵏ. ⵎⴷⴷⵏ ⵉⵎⵇⵔⴰⵏⴻⵏ, ⴰⵎⴽⴰ ⴰⴽⴽ ⴰⵢ ⴳⴰⵏ.

ⵎⴰⵛⴰ ⵙ ⵓⵎⴷⴰⵣ ⵏ ⵜⵓⵙⵙⵏⴰ ⵏ ⴰⵙⵜⵔⵓⵉⴷ ⴱ612, ⵉⴽⴽⵔ ⵢⴰⵏ ⵓⴷⵉⴽⵜⴰⵜⵓⵔ ⴰⵜⵓⵔⴽⵢ, ⵉⴳⴳ ⴰⵙⵏ ⵛⵛⵉⵍ ⵉ ⵎⴷⴷⵏ ⴰⴷ ⵍⵙⵙⴰⵏ ⵎⵍⵙⵉⵡⵜ ⵏ ⵓⵔⵓⴱⵉⵢⵏ, ⵡⴰⵏⵏⴰ ⵢⴰⴳⵉⵏ ⵉⵏⵖ ⵜ. ⴰⵙⵜⵔⵓⵏⵓⵎ ⵏⵏⴰⵖ, ⵢⵓⵍⵙ ⴷⵉⵖ ⵉ ⵜⵎⵙⴽⴰⵏⵜ ⵏⵏⵙ ⴰⵙⴳⴳⴰⵙ ⵏ 1920, ⵜⵉⴽⴽⵍⵜ ⵏⵏⴰⵖ ⵉⵍⵍⴰ ⵉⵍⵙⴰ ⵢⴰⵜ ⵎⵍⵙⵉⵡⵜ ⵢⵖⵓⴷⴰⵏ ⵛⵉⴳⴰⵏ. ⵜⵉⴽⴽⵍⵜ ⵏⵏⴰⵖ, ⵎⴷⴷⵏ ⴰⴽⴽ ⵓⵎⴻⵏ ⴰⵡⴰⵍ ⵏⵏⵙ.
\par
\vspace*{10pt}
\end{minipage}
}

\subsection{Unified Canadian Aboriginal Syllabics}

Unified Canadian Aboriginal Syllabics is a Unicode block containing characters for writing Inuktitut, Carrier, several dialects of Cree, and Canadian Athabascan languages. Additions for some Cree dialects, Ojibwe, and Dene can be found at the Unified Canadian Aboriginal Syllabics Extended block.
\medskip

\newfontfamily\aboriginal{code2000.ttf}
\bgroup
\par
\noindent
\colorbox{graphicbackground}{\color{black}^^A
\begin{minipage}{\textwidth}^^A
\parindent1pt
\vskip10pt
\leftskip10pt \rightskip\leftskip

\aboriginal
ᒥᓯᐌ ᐃᓂᓂᐤ ᑎᐯᓂᒥᑎᓱᐎᓂᐠ ᐁᔑ ᓂᑕᐎᑭᐟ ᓀᐢᑕ ᐯᔭᑾᐣ ᑭᒋ ᐃᔑ
\bfseries ᑲᓇᐗᐸᒥᑯᐎᓯᐟ ᑭᐢᑌᓂᒥᑎᓱᐎᓂᐠ ᓀᐢᑕ ᒥᓂᑯᐎᓯᐎᓇ᙮
Unicode Block: Unified Canadian Aboriginal Syllabics, UCAS Extended
Text: UDHR: Cree, Swampy ᐯᔭᐠ ᐱᐢᑭᑕᓯᓇᐃᑲᐣ ᐁᐢᐱᑕᐢᑲᒥᑲᐠ ᐊᐢᑭᐠ ᑭᒋ ᐃᑗᐎᐣ ᐃᓂᓂᐎ ᒥᓂᑯᐎᓯᐎᓇ ᐅᒋ
\par
\vspace*{10pt}
\end{minipage}
}
\medskip
\egroup
\subsection{Miao}

The Pollard script, also known as Pollard Miao (Chinese: 柏格理苗文 Bó Gélǐ Miao-wen) or Miao, is an abugida loosely based on the Latin alphabet and invented by Methodist missionary Sam Pollard. Pollard invented the script for use with A-Hmao, one of several Miao languages. The script underwent a series of revisions until 1936, when a translation of the New Testament was published using it. The introduction of Christian materials in the script that Pollard invented caused a great impact among the Miao. Part of the reason was that they had a legend about how their ancestors had possessed a script but lost it. According to the legend, the script would be brought back some day. When the script was introduced, many Miao came from far away to see and learn it.[1][2]

Pollard credited the basic idea of the script to the Cree syllabics designed by James Evans in 1838–1841, “While working out the problem, we remembered the case of the syllabics used by a Methodist missionary among the Indians of North America, and resolved to do as he had done” (1919:174). He also gave credit to a Chinese pastor, “Stephen Lee assisted me very ably in this matter, and at last we arrived at a system” (1919:174). In listing the phrases he used to describe devising the script, there is clear indication of intellectual work, not revelation: “we looked about”, “resolved to attempt”, “adapting the system”, “solved our problem” (Pollard 1919:174,175).

Changing politics in China led to the use of several competing scripts, most of which were romanizations. The Pollard script remains popular among Hmong in China, although Hmong outside China tend to use one of the alternative scripts. A revision of the script was completed in 1988, which remains in use.

As with most other abugidas, the Pollard letters represent consonants, whereas vowels are indicated by diacritics. Uniquely, however, the position of this diacritic is varied to represent tone. For example, in Western Hmong, placing the vowel diacritic above the consonant letter indicates that the syllable has a high tone, whereas placing it at the bottom right indicates a low tone.

A still experimental font, that supports Graphite technology is \idxfont{Mia Unicode}\footnote{\url{http://phjamr.github.io/miao.html\#intro}}. The font is licenced under the SIL terms and we are using it in the |phd| package as the default font for the Miao script.

\newfontfamily\miao{MiaoUnicode-Regular.ttf}

\begin{scriptexample}[]{Miao}
\unicodetable{miao}{"16F00,"16F10,"16F20,"16F30,"16F40,"16F70,"16F80,"16F90}
\end{scriptexample}

{\miao 𖼴	𖼵	𖼶	𖼷	𖼸	𖼹	𖼺	}

Features for Miao
There are three features currently available for the Miao script:
\bgroup
\miao
Chuxiong ‘wart’ variant
Stylistic alternates for 𖼳 and 𖼴
Aspiration marker always on right
The ‘wart’ (a translated technical term!) is the small circle in characters like 𖼁, 𖼅, and 𖼾. In the Chuxiong orthography, it is rendered not as a circle but as a dot on the right of the letter, as shown in point 5 here (pdf).

Miao Unicode has a feature called “chux” for handling this. In LibreOffice you can use this style by typing “Miao Unicode:chux=1” into the font field.
\section{N'ko}

\newfontfamily\nko{NotoSansNKo-Regular.ttf}

N'Ko {\nko(ߒߞߏ)} is both a script devised by Solomana Kante in 1949 as a writing system for the Manding languages of West Africa, and the name of the literary language itself written in the script. The term N'Ko means ``I say'' in all Manding languages.

The script has a few similarities to the Arabic script, notably its direction (right-to-left) and the connected letters. It obligatorily marks both tone and vowels.


\begin{scriptexample}[]{N'ko}
\unicodetable{nko}{"07C0,"07D0,"07E0,"07F0}
\end{scriptexample}

The N'Ko alphabet is written from right to left, with letters being connected to one another.

The script is principally used in Guinea and Côte d'Ivoire (respectively by Maninka and Dioula-speakers), with an active user community in Mali (by Bambara-speakers). Publications include a translation of the Qur'an, a variety of textbooks on subjects such as physics and geography, poetic and philosophical works, descriptions of traditional medicine, a dictionary, and several local newspapers. It has been classed as the most successful of the West African scripts.[3] The literary language used is intended as a koine blending elements of the principal Manding languages (which are mutually intelligible), but has a particularly strong Maninka flavour.

The Latin script with several extended characters (phonetic additions) is used for all Manding languages to one degree or another for historical reasons and because of its adoption for "official" transcriptions of the languages by various governments. In some cases, such as with Bambara in Mali, promotion of literacy using this orthography has led to a fair degree of literacy in it. Arabic transcription is commonly used for Mandinka in The Gambia and Senegal.


\subsection{Mongolian}
\newfontfamily\mongolian{NotoSansMongolian-Regular.ttf}

The classical Mongolian script (in Mongolian script:{\mongolian ᠮᠣᠩᠭᠣᠯ ᠪᠢᠴᠢᠭ᠌} Mongγol bičig; in Mongolian Cyrillic: Монгол бичиг Mongol bichig), also known as Uyghurjin Mongol bichig, was the first writing system created specifically for the Mongolian language, and was the most successful until the introduction of Cyrillic in 1946. Derived from Uighur, Mongolian is a true alphabet, with separate letters for consonants and vowels. The Mongolian script has been adapted to write languages such as Oirat and Manchu. Alphabets based on this classical vertical script are used in Inner Mongolia and other parts of China to this day to write Mongolian, Sibe and, experimentally, Evenki.

\begin{scriptexample}[]{Mongolian}
\unicodetable{mongolian}{"1820,"1830,"1840,"1850,"1860,"1870,"1880,"1890,"18A0}
\end{scriptexample}



\section{Middle Eastern Scripts}

The scripts in this section have a common origin in the ancient Phoenician alphabet. They include:

\begin{center}
\begin{tabular}{ll}
Hebrew & Samaritan\\
Arabic & Thaana\\
Syriac &\\
\end{tabular}
\end{center}

The Hebrew script is used in Israel and for languages of the Diaspora. The Arabic script is
used to write many languages throughout the Middle East, North Africa, and certain parts
of Asia. The Syriac script is used to write a number of Middle Eastern languages. These
three also function as major liturgical scripts, used worldwide by various religious groups.

The Samaritan script is used in small communities in Israel and the Palestinian Territories
to write the Samaritan Hebrew and Samaritan Aramaic languages. The Thaana script is
used to write Dhivehi, the language of the Republic of Maldives, an island nation in the
middle of the Indian Ocean. 

Text in these scripts is written from right to left. Arabic and Syriac are cursive scripts even when typeset, unlike Hebrew, Samaritan  and Thaana, where letters are unconnected. Most letters in Arabic and Syriac assume different forms depending on their position in a word. Shaping rules are not required for Hebrew because only five letters have position-dependent forms, and these forms are separately encoded.

Historically, Middle Eastern  scripts did not write short vowels. In modern scripts they are represented  by marks positioned above or below a consonantal letter. Vowels and other
marks of pronunciation (“vocalization”) are encoded as combining characters, so support
for vocalized text necessitates use of composed character sequences. Yiddish, Syriac, and
Thaana are normally written with vocalization; Hebrew, Samaritan, and Arabic are usually written unvocalized. 

\section{Hebrew}
\newfontfamily\hebrew{Miriam}
\fontspec{Arial Unicode MS}
To properly typeset Hebrew texts you first need to choose an appropriate font and also set the directionality of the text. This
is done using the etex commands:

\CMDI{\beginL} and \CMDI{\beginR} 

For \XeTeX\ you also need to add near the top of your document |\TeXXeTstate=1|. The package \pkgname{bidi} can be used to set all parameters. Be warned that it redefines almost all of \latexe's commands, so for short mixed texts, I wouldn't recommend its usage. 



The Hebrew alphabet (Hebrew: אָלֶף־בֵּית עִבְרִי[a], alefbet ʿIvri ), known variously by scholars as the Jewish script, square script, block script, is used in the writing of the Hebrew language, as well as other Jewish languages, most notably Yiddish, Ladino, and Judeo-Arabic. There have been two script forms in use; the original old Hebrew script is known as the paleo-Hebrew script (which has been largely preserved, in an altered form, in the Samaritan script), while the present "square" form of the Hebrew alphabet is a stylized form of the Assyrian script. Various "styles" (in current terms, "fonts") of representation of the letters exist. There is also a cursive Hebrew script, which has also varied over time and place. On Windows you can use the \texttt{Miriam} font or \texttt{Arial Unicode MS} or \texttt{Miriam Fixed}.
\medskip

\topline

\bgroup\TeXXeTstate=1
\raggedleft\hebrew{}\beginR

הכתב הכנעני הקדום הלך והתפשט וסימניו היו מוכרים כל כך, עד כי המשתמשים בו התחילו "להתעצל" בהשלמת הציורים, והניחו כי הקורא יבין גם מתוך שרטוטים סכמתיים באיזו אות מדובר. כך, למשל, הפך הראש למשולש עם צוואר; כף היד מלאת האצבעות הפכה לשרטוט דל, ומהדג נותר רק הזנב. כשהעברים אמצו את הכתב הכנעני הם התקשו לזהות חלק מהציורים המקוריים והניחו למשל כי הסימן המתאר את המילה "זהה" הוא כלי נשק; שזנב הדג המשולש הוא דלת, ושדווקא הנחש הוא דג. כך נולדו שמותיהם העבריים של האותיות זי"ן, דל"ת ונו"ן (נון הוא דג, כמו אמנון, שפמנון וכו'). הציורים שהפכו לסימנים התגלגלו לכתבים נוספים, ואפילו ליוונית וללטינית. גם בכתב העברי המודרני ניתן לזהות המשך התפתחותי ברור מן הכתב הכנעני הקדום, והשתמרות שמות האותיות מקלה מאוד על פענוח המקור.


בתקופת בית שני, אומץ האלפבית הארמי לשימוש השפה העברית במקום האלפבית העברי העתיק, כאשר בזה האחרון נעשה שימוש מועט כגון כתיבת השמות הקדושים והטבעת מטבעות. עם הזמן, נעלם גם שימוש זה של הכתב העתיק. האלפבית העברי של ימינו הוא אפוא פיתוח של האלפבית הארמי ולא של הכתב העברי העתיק.	
{}

 לֹ֥א תִשָּׂ֛א

\endR


\egroup
\bottomline
\medskip

To make all paragraphs  RL use the \cmd{\everypar}\footnote{See discussions at \url{http://tex.stackexchange.com/questions/141867/minimal-bidi-for-typesetting-rl-text} and \url{http://www.tug.org/pipermail/xetex/2004-August/000697.html}}. 

\begin{verbatim}
\newbox\mybox \everypar{\setbox\mybox\lastbox\beginR\box\mybox}
\everypar={% at the start of each paragraph, do....
    \setbox0=\lastbox % save the paragraph indent, if any
    \beginR % set R-L direction
    \box0 % then re-insert the indent
	}
\end{verbatim}

The Hebrew alphabet has 22 letters, of which five have different forms when used at the end of a word. Hebrew is written from right to left. Originally, the alphabet was an abjad consisting only of consonants. Like other \textit{abjads}, such as the Arabic alphabet, means were later devised to indicate vowels by separate vowel points, known in Hebrew as niqqud. In rabbinic Hebrew, the letters א ה ו י are also used as matres lectionis to represent vowels. When used to write Yiddish, the writing system is a true alphabet (except for borrowed Hebrew words). In modern usage of the alphabet, as in the case of Yiddish (except that ע replaces ה) and to some extent modern Israeli Hebrew, vowels may be indicated. Today, the trend is toward full spelling with these letters acting as true vowels.

\section{Samaritan}
\newfontfamily\samaritan{NotoSansSamaritan-Regular.ttf}

The Samaritan alphabet is used by the Samaritans for religious writings, including the Samaritan Pentateuch, writings in Samaritan Hebrew, and for commentaries and translations in Samaritan Aramaic and occasionally Arabic.

The Samaritans are, consider themselves to be the descendants of the Northern Tribes of Israel that were not sent into Assyrian captivity, and have continuously resided in the land of Israel.

The Torah Scroll of the Samaritans uses an alphabet that is very different from the one used on Jewish Torah Scrolls. According to the Samaritans themselves and Hebrew scholars, this alphabet is the original "Old Hebrew" alphabet.

Even as far back as 1691, this connection between the Samaritan and the "Old" Hebrew alphabets was made by Henry Dodwell; "[the Samaritans] still preserve [the Pentateuch] in the Old Hebrew characters."

Samaritan is a direct descendant of the Paleo-Hebrew alphabet, which was a variety of the Phoenician alphabet in which large parts of the Hebrew Bible were originally penned. All these scripts are believed to be descendants of the Proto-Sinaitic script. That script was used by the ancient Israelites, both Jews and Samaritans. The better-known "square script" Hebrew alphabet traditionally used by Jews is a stylized version of the Aramaic alphabet which they adopted from the Persian Empire (which in turn adopted it from the Arameans). 

After the fall of the Persian Empire, Judaism used both scripts before settling on the Aramaic form. For a limited time thereafter, the use of paleo-Hebrew (proto-Samaritan) among Jews was retained only to write the Tetragrammaton, but soon that custom was also abandoned.



ShofarRegular StamAshkenazCLM.ttf

\begin{scriptexample}[]{Samaritan}
\bgroup
\TeXXeTstate=1
\unicodetable{samaritan}{"0800,"0810,"0820,"0830}
\egroup
\TeXXeTstate=0
\end{scriptexample}

I battled to get an appropriate font for the Samaritan script and had to use the \idxfont{Noto Sans Samaritan} from Google


^^A\printunicodeblock{./languages/samaritan.txt}{\samaritan}


\url{http://www.ancient-hebrew.org/ahh/ahh.htm#_Toc314842274}



\section{Arabic}

\newfontfamily\arabian{Scheherazade-R.ttf}

The Arabic script is a writing system used for writing several languages of Asia and Africa, such as Arabic, Sorani and Luri Dialects of Kurdish language, Persian, Pashto and Urdu.[1] Even until the 16th century, it was used to write some texts in Spanish.[2] After the Latin script, Chinese characters, and Devanagari, it is the fourth-most widely used writing system in the world.[3]
The Arabic script is written from right to left in a cursive style. In most cases the letters transcribe consonants, or consonants and a few vowels, so most Arabic alphabets are abjads.

The script was first used to write texts in Arabic, most notably the Qurʼān, the holy book of Islam. With the spread of Islam, it came to be used to write languages of many language families, leading to the addition of new letters and other symbols, with some versions, such as Kurdish, Uyghur, and old Bosnian being abugidas or true alphabets. It is also the basis for a rich tradition of Arabic calligraphy.

\begin{verbatim}
\begin{Arabic}
ّ هو إذ الغاية؛ شريف الفوائد، جم المذهب، عزيز فنّ التاريخ فنّ أنّ اعلم
والملوك سيرهم، في والأنبياء أخلاقهم، في الأمم من الماضين أحوال على يوقفنا
ّ أحوال في يرومه لمن ذلك في الإقتداء فائدة تتم حتّى وسياستهم؛ دولهم في
والدنيا. الدين
\end{Arabic}
\end{verbatim}




As of Unicode 7.0, the Arabic script is contained in the following blocks:
Arabic (0600—06FF, 255 characters)
Arabic Supplement (0750—077F, 48 characters)
Arabic Extended-A (08A0—08FF, 39 characters)
Arabic Presentation Forms-A (FB50—FDFF, 608 characters)
Arabic Presentation Forms-B (FE70—FEFF, 140 characters)
Rumi Numeral Symbols (10E60—10E7F, 31 characters)
Arabic Mathematical Alphabetic Symbols (1EE00—1EEFF, 143 characters)[1][2]

The basic Arabic range encodes the standard letters and diacritics, but does not encode contextual forms (U+0621–U+0652 being directly based on ISO 8859-6); and also includes the most common diacritics and Arabic-Indic digits. The Arabic Supplement range encodes letter variants mostly used for writing African (non-Arabic) languages. The Arabic Extended-A range encodes additional Qur'anic annotations and letter variants used for various non-Arabic languages. The Arabic Presentation Forms-A range encodes contextual forms and ligatures of letter variants needed for Persian, Urdu, Sindhi and Central Asian languages. The Arabic Presentation Forms-B range encodes spacing forms of Arabic diacritics, and more contextual letter forms. The presentation forms are present only for compatibility with older standards, and are not currently needed for coding text.[3] 

The Arabic Mathematical Alphabetical Symbols block encodes characters used in Arabic mathematical expressions.

\begin{multicols}{3}
\printunicodeblock{./languages/arabic.txt}{\arabian}
\end{multicols}








\section{Thaana}

\newfontfamily\thaana{MV Boli}
Thaana, Taana or Tāna ({\thaana  ތާނަ}‎ in Tāna script) is the modern writing system of the Maldivian language spoken in the Maldives. Thaana has characteristics of both an abugida (diacritic, vowel-killer strokes) and a true alphabet (all vowels are written), with consonants derived from indigenous and Arabic numerals, and vowels derived from the vowel diacritics of the Arabic abjad. Its orthography is largely phonemic.

The Thaana script first appeared in a Maldivian document towards the beginning of the 18th century in a crude initial form known as Gabulhi Thaana which was written scripta continua. This early script slowly developed, its characters slanting 45 degrees, becoming more graceful and spaces were added between words. 

As time went by it gradually replaced the older Dhives Akuru alphabet. The oldest written sample of the Thaana script is found in the island of Kanditheemu in Northern Miladhunmadulu Atoll. It is inscribed on the door posts of the main Hukuru Miskiy (Friday mosque) of the island and dates back to 1008 AH (AD 1599) and 1020 AH (AD 1611) when the roof of the building were built and the renewed during the reigns of Ibrahim Kalaafaan (Sultan Ibrahim III) and Hussain Faamuladeyri Kilege (Sultan Hussain II) respectively.

\begin{scriptexample}[]{Thaana}
\unicodetable{thaana}{"0780,"0790,"07A0,"07B0}

\hfill Typeset with MV Boli and the command \cmd{\thaana}.
\end{scriptexample}


^^A\printunicodeblock{./languages/thaana.txt}{\thaana}

\subsection{Syriac}

\newfontfamily\syriac{Estrangelo Edessa}

Syriac /ˈsɪriæk/ ({\syriac{ܠܫܢܐ ܣܘܪܝܝܐ}} Leššānā Suryāyā) is a dialect of Middle Aramaic that was once spoken across much of the Fertile Crescent and Eastern Arabia.[1][2][5] Having first appeared as a script in the 1st century AD after being spoken as an unwritten language for five centuries,[6] Classical Syriac became a major literary language throughout the Middle East from the 4th to the 8th centuries,[7] the classical language of Edessa, preserved in a large body of Syriac literature.
It became the vehicle of Syriac Christianity and culture, spreading throughout Asia as far as the Indian Malabar Coast and Eastern China,[8] and was the medium of communication and cultural dissemination for Arabs and, to a lesser extent, Persians. Primarily a Christian medium of expression, Syriac had a fundamental cultural and literary influence on the development of Arabic,[9] which largely replaced it towards the 14th century.[3] Syriac remains the liturgical language of Syriac Christianity.
Syriac is a Middle Aramaic language, and, as such, it is a language of the Northwestern branch of the Semitic family. It is written in the Syriac alphabet, a derivation of the Aramaic alphabet.

\begin{scriptexample}[]{Syriac}
\unicodetable{syriac}{"0700,"0710,"0720,"0730,"0740}
\end{scriptexample}

The Syriac Abbreviation (a type of overline) can be represented with a special control character called the Syriac Abbreviation Mark (U+070F {\syriac \char"070F ܘ}).


\cxset{steward,
  numbering=arabic,
  custom=stewart,
  offsety=0cm,
  image={asia.jpg},
  texti={An introduction to the use of font related commands. The chapter also gives a historical background to font selection using \tex and \latex. },
  textii={In this chapter we discuss keys that are available through the \texttt{phd} package and give a background as to how fonts are used
in \latex.
 },
 pagestyle = empty
}

\arial


\chapter{South Asian Scripts}

The scripts of South Asia share so many characteristics that a side by side comparison of a few often reveal structural similarities even in the 
modern letterforms.
\medskip

\begin{center}
\begin{tabular}{lll}
Devanagari. &Gujarati &Telugu\\
Bengali   &Oriya &Kannada\\
Gurmukhi &Tamil  &Malayalam\\
Sinhala &Kaithi  &Meetei Mayek\\
Tibetan &Saurashtra &Ol Chiki.\\
Lepcha  &Sharada &Sora Sompeng\\
Phags-pa &Takri &Kharoshthi\\
Limbu &Chakma & Brahmi\\
Syloti Nagri & &\\
\end{tabular}
\end{center}

The sections that follow describe the scripts briefly and the |phd| settings
to activate the relevant commands and load appropriate fonts. 

\section{Devanagari}
\parindent1em

Devanagari is part of the Brahmic family of scripts of India, Nepal, Tibet, and South-East Asia.[2] It is a descendant of the Gupta script, along with Siddham and Sharada.[2] Eastern variants of Gupta called nāgarī are first attested from the 7th century CE; from c. 1200 CE these gradually replaced Siddham, which survived as a vehicle for Tantric Buddhism in East Asia, and Sharada, which remained in parallel use in Kashmir. An early version of Devanagari is visible in the Kutila inscription of Bareilly dated to Vikram Samvat 1049 (i.e. 992 CE), which demonstrates the emergence of the horizontal bar to group letters belonging to a word.[3]

Sanskrit nāgarī is the feminine of nāgara "relating or belonging to a town or city". It is feminine from its original phrasing with lipi ("script") as nāgarī lipi "script relating to a city", that is, probably from its having originated in some city.[4]

The use of the name devanāgarī is relatively recent, and the older term nāgarī is still common.[2] The rapid spread of the term devanāgarī may be related to the almost exclusive use of this script to publish Sanskrit texts in print since the 1870s.[2]

On Windows use \texttt{Arial Unicode MS}. 
\medskip

\newfontfamily\devanagari[Script=Devanagari,Scale=1.5]{Arial Unicode MS}

\begin{scriptexample}[]{Devanagari}
{\begin{center}\parindent0pt\devanagari

ंःअआइईउऊऋऌऍऎएऐऑऒओऔऔँ \par 

ी	ु	ू	ृ	ॄ	ॅ	ॆ	े	ै	ॉ	ॊ	ो	ौ	्	\par

\bigskip		
\begin{tabular}{lll lll lll l}
०	&१	&२	&३	&४	&५	&६	&७	&८	&९\\
0	&1	&2	&3	&4	&5	&6	&7	&8	&9\\
\end{tabular}
\end{center}	
}
\end{scriptexample}


On Linux \texttt{Lohit} is a font family designed to cover Indic scripts and released by Red Hat. The Lohit fonts currently cover 11 languages: Assamese, Bengali, Gujarati, Hindi, Kannada, Malayalam, Marathi, Oriya, Punjabi, Tamil, Telugu.[1] The fonts were supplied by Modular Infotech and licensed under the GPL. In September 2011, they were retroactively relicensed under the OFL.[2] The Lohit fonts are used as web fonts by some Wikimedia Foundation sites, like Wikipedia, since March 2012.The font currently support 21 Indian languages. 

\newfontfamily\devanagarilohit[Script=Devanagari,Scale=1.5]{Lohit-Devanagari.ttf}

\begin{scriptexample}[]{Devanagari}
\begin{center}\parindent0pt\devanagarilohit

ंःअआइईउऊऋऌऍऎएऐऑऒओऔऔँ \par 

ी	ु	ू	ृ	ॄ	ॅ	ॆ	े	ै	ॉ	ॊ	ो	ौ	्	\par

\bigskip		
\begin{tabular}{lll lll lll l}
०	&१	&२	&३	&४	&५	&६	&७	&८	&९\\
0	&1	&2	&3	&4	&5	&6	&7	&8	&9\\
\end{tabular}
\end{center}
\end{scriptexample}

\subsubsection{Punctuation} 
The end of a sentence or half-verse may be marked with a dot known as a pūrna virām or a vertical line danda: \textbar. The end of a full verse may be marked with two vertical lines: \textbar\textbar. A comma, or alpa virām, is used to denote a natural pause in speech. With expansion of English speakers in India, the full stop is also sometimes used.

\subsection{LaTeX support}

\latex2e support can be found in the \pkgname{sanskrit}. The package contains the font files and pre-processor for printing Sanskrit
text in both devanāgarī and transliterated Roman with diacritics. Another package that can be used with \XeTeX\ is support \pkgname{devnag}.  This was originally developed by Frans Velthuis for the University of Groningen, The Netherlands, and it was the first system to provide
support for the script for \tex. The package was  extended by Anshuman Pandey. The package provides both fonts as well as tranliteration macros.


\subsection{Gujarati}


Gujarati has its own writing system, distinct but related to several other Indian languages' writing systems, such as the one used to write Hindi. Strictly speaking, the Gujarati writing system is what is called an \emph{abugida} (and not an \textit{alphabet}), because the consonant characters all contain an inherent vowel, and other vowels are written as accents added on to the consonant characters. There are also symbols for stand-alone vowels.

The Gujarati script ({\gujarati{ગુજરાતી લિપિ }} Gujǎrātī Lipi), which like all Nāgarī writing systems is strictly speaking an abugida rather than an alphabet, is used to write the Gujarati and Kutchi languages. It is a variant of Devanāgarī script differentiated by the loss of the characteristic horizontal line running above the letters and by a small number of modifications in the remaining characters.
With a few additional characters, added for this purpose, the Gujarati script is also often used to write Sanskrit and Hindi.
Gujarati numerical digits are also different from their Devanagari counterparts.
\medskip

\bgroup
\newfontfamily\gujaratilohit[Script=Gujarati,Scale=1.5]{Lohit-Gujarati.ttf}
\gujarati

\centering

English/Hindi/Gujarati Alphabets

\begin{tabular}{lllllllllllllllllllll}
A &B &bh &C &ch &chh &D &dh &E &F &G &gh &H &I &J &K &kh &L &M &N &O\\

अ &ब &भ &क &च &छ &ड/द &ध/ढ़ &इ &फ &ग &घ &ह &ई &ज &क &ख &ल &म &न/ण &ऑ\\

અ &બ &ભ &ક &ચ &છ &ડ/દ &ધ /ઢ &ઇ &ફ &ગ &ઘ &હ &ઈ &જ &ક &ખ &લ &મ &ન/ણ &ઓ\\

\end{tabular}
\egroup

\medskip

Gujarati has its own set of numeric signs (placed alongside their Hindu-Arabic [or Indo-Arabic] counterparts in the tables below), they are employed in much the same way as English;  that is to say, they are put together in the same manner in order to express larger numbers. It is quite possible to simply substitute the Gujarati numerals for the Hindu-Arabic ones.

The Gujarati words for 1-10 are as follows:
\medskip

\bgroup
\begin{center}
\gujarati
\begin{tabular}{ccl}
Arabic & Gujarati &Name\\
Numeral &Numeral  &\\
0	&૦	&mīṇḍuṃ or shunya\\
1	&૧	&ekaṛo or ek\\
2	&૨	&bagaṛo or bay\\
3	&૩	&tragaṛo or tran\\
4	&૪	&chogaṛo or chaar\\
5	&૫	&pāchaṛo or paanch\\
6	&૬	&chagaṛo or chah\\
7	&૭	&sātaṛo or sāt\\
8	&૮	&āṭhaṛo or āanth\\
9	&૯	&navaṛo or nav\\
10 &૧૦ &દસ das\\

\end{tabular}
\end{center}
\egroup

\subsection{Bengali}

There are two Windows fonts that can be used with Windows \textit{Shonar Bangla} and \textit{Vrinda}. For open source fonts one can use, \textit{code2000}.
\bigskip

\bgroup
\newfontfamily\bengali[Script=Bengali,Scale=4]{Shonar Bangla}


\bengali
\centering

  অ  আ ই  ঈ  উ  ঊ  ঋ  এ  ঐ\par

\fontspec[Script=Bengali,Scale=3.2]{Vrinda}

\centering

  অ  আ ই  ঈ  উ  ঊ  ঋ  এ  ঐ\par


\fontspec[Script=Bengali,Scale=3.2]{code2000.ttf}

\centering

  অ  আ ই  ঈ  উ  ঊ  ঋ  এ  ঐ\par

\captionof{table}{The consonant{\protect\bengal{} ক (kô)} along with the diacritic form of the vowels {\protect\bengal{} অ, আ, ই, ঈ, উ, ঊ, ঋ, এ, ঐ, ও and ঔ} \textit{from Wikipedia}.}
\egroup

\subsection{Saurashtra}

\newfontfamily\saurashtra{code2000.ttf}

Saurashtra or Sourashtra or {\saurashtra ꢱꣃꢬꢵꢰ꣄ꢜ꣄ꢬꢵ} or Palkar or Patkar (Sanskrit: सौराष्ट्र, Tamil: சௌராட்டிரம்) is an Indo-Aryan language[3] spoken by the Saurashtrian community native to Gujarat, who migrated and settled in Southern India. Madurai in Tamil Nadu has the highest number of people belonging to this community and also remains as their cultural center.

The language is largely only in spoken form even though the language has its own script. The lack of schools teaching Saurashtra script and the language is often cited as a reason for the very few number of people who actually know to read and write in Saurashtra script. Latin, Devanagari or Tamil script is used as alternative for Saurashtra Script by many Saurashtrians.

Census of India places the language under Gujarati. Official figures show the number of speakers as 185,420 (2001 census).[4]



\begin{scriptexample}[]{Saurashtra}
\bgroup
\saurashtra

ꢮꢶꢯ꣄ꢮ ꢱꣃꢬꢵꢰ꣄ꢜ꣄ꢬꢪ꣄ ꢦꢡ꣄ꢬꢶꢒꢾ ꢱꢵꢡ꣄ꢡꢒꢸ ꢂꢮꢬꢾ
ꢮꣁꢭꢱ꣄ꢢꢵꢥꢪꢸꢒ꣄(ꣀꢵꢮꢾꢔꢹ ꢂꢮ꣄ꢬꢶꢫꣁ


\arial

Text: Vishwa Sourashtram \url{http://www.sourashtra.info/ghEr.htm}
\egroup
\end{scriptexample}

\subsection{Ol Chiki script}

The Ol Chiki script, also known as Ol Cemetʼ (Santali: ol 'writing', cemet' 'learning'), Ol Ciki, Ol, and sometimes as the Santali alphabet, was created in 1925 by Raghunath Murmu for the Santali language.

Previously, Santali had been written with the Latin alphabet. But because Santali is not an Indo-Aryan language (like most other languages in the south of India), Indic scripts did not have letters for all of Santali's phonemes, especially its stop consonants and vowels, which made writing the language accurately in an unmodified Indic script difficult. The detailed analysis was given by Dr. Byomkes Chakrabarti in his 'Comparative Study of Santali and Bengali'. Missionaries (first of all Paul Olaf Bodding, a Norwegian) brought the Latin script, which is better at representing Santali stops, phonemes and nasal sounds with the use of diacritical marks and accents. Unlike most Indic scripts, which are derived from Brahmi, Ol Chiki is not an abugida, with vowels given equal representation with consonants. Additionally, it was designed specifically for the language, but one letter could not be assigned to each phoneme because the sixth vowel in Ol Chiki is still problematic.
Ol Chiki has 30 letters, the forms of which are intended to evoke natural shapes. Linguist Norman Zide said "The shapes of the letters are not arbitrary, but reflect the names for the letters, which are words, usually the names of objects or actions representing conventionalized form in the pictorial shape of the characters."[1] It is written from left to right.

\newfontfamily\olchiki{code2000.ttf}

\begin{scriptexample}[]{olchiki}
\bgroup
\olchiki
\obeylines

U+1C5x 	᱐	᱑	᱒	᱓	᱔	᱕	᱖	᱗	᱘	᱙	ᱚ	ᱛ	ᱜ	ᱝ	ᱞ	ᱟ
U+1C6x	   ᱠ	ᱡ	ᱢ	ᱣ	ᱤ	ᱥ	ᱦ	ᱧ	ᱨ	ᱩ	ᱪ	ᱫ	ᱬ	ᱭ	ᱮ	ᱯ
U+1C7x  	ᱰ	ᱱ	ᱲ	ᱳ	ᱴ	ᱵ	ᱶ	ᱷ	ᱸ	ᱹ	ᱺ	ᱻ	ᱼ	ᱽ	᱾	᱿
\egroup
\end{scriptexample}

\subsection{Lepcha}
\newfontfamily\lepcha{Mingzat-R.ttf}

The Lepcha script, or Róng script is an abugida used by the Lepcha people to write the Lepcha language. Unusually for an abugida, syllable-final consonants are written as diacritics.

The Mingzat font is still under development by SIL so I am not too sure if the rendering is correct\footnote{\url{http://scripts.sil.org/cms/scripts/page.php?site_id=nrsi&id=Mingzat}}.

\begin{scriptexample}[]{Lepcha}
\bgroup
\lepcha
\obeylines
 	    0	1	2	3	4	5	6	7	8	9	A	B	C	D	E	F
U+1C0x	 ᰀ	ᰁ	ᰂ	ᰃ	ᰄ	ᰅ	ᰆ	ᰇ	ᰈ	ᰉ	ᰊ	ᰋ	ᰌ	ᰍ	ᰎ	ᰏ
U+1C1x	 ᰐ	ᰑ	ᰒ	ᰓ	ᰔ	ᰕ	ᰖ	ᰗ	ᰘ	ᰙ	ᰚ	ᰛ	ᰜ	ᰝ	ᰞ	ᰟ
U+1C2x	 ᰠ	ᰡ	ᰢ	ᰣ	ᰤ	ᰥ	ᰦ	ᰧ	ᰨ	ᰩ	ᰪ	ᰫ	ᰬ	ᰭ	ᰮ	ᰯ
U+1C3x	 ᰰ	ᰱ	ᰲ	ᰳ	ᰴ	ᰵ	ᰶ	᰷	x	x	x	᰻	᰼	᰽	᰾	᰿
U+1C4x	 ᱀	᱁	᱂	᱃	᱄	᱅	᱆	᱇	᱈	᱉	x	x	x	ᱍ	ᱎ	ᱏ

\egroup
\end{scriptexample}

\subsection{Sharada}

The Śāradā, or Sharada, script (शारदा) is an abugida writing system of the Brahmic family of scripts, developed around the 8th century. It was used for writing Sanskrit and Kashmiri. The Gurmukhī script was developed from Śāradā. Originally more widespread, its use became later restricted to Kashmir, and it is now rarely used except by the Kashmiri Pandit community for ceremonial purposes. Śāradā is another name for Saraswati, the goddess of learning.
Śāradā script was added to the Unicode Standard in January, 2012 with the release of version 6.1.

The Unicode block for Śāradā script, called Sharada, is U+11180–U+111DF: Unable to locate font in unicode.


\subsection{Sora Sompeng}

Sorang Sompeng script is used to write in Sora, a Munda language with 300,000 speakers in India. The script was created by Mangei Gomango in 1936 and is used in religious contexts.[1] He was familiar with Oriya, Telugu and English, so the parent systems of the script are Brahmi and Latin.[2]
The Sora language is also written in the Latin alphabet and the Telugu script.

Sorang Sompeng script was added to the Unicode Standard in January, 2012 with the release of version 6.1. Nirmala UI.ttf (Windows 8.1)



\unicodetable{arial}{"110D0,"110E0,"110F0}
 	
This did not work with Windows 7, and the experiment failed. 

\subsection{Phags-pa}

The 'Phags-pa script,[1], (Mongolian: дөрвөлжин үсэг "Square script") was an alphabet designed by the Tibetan monk and vice-king Drogön Chögyal Phagpa for the Mongol Yuan emperor Kublai Khan as a unified script for the literary languages of the Yuan. Widespread use was limited to about a hundred years during the Yuan Dynasty, and it fell out of use with the advent of the Ming dynasty. The documentation of its use provides clues about the changes in the varieties of Chinese, the Tibetic languages, Mongolian and other neighboring languages during the Yuan era.

\newfontfamily\phagspa{code2000.ttf}

\begin{scriptexample}[]{Phags-pa}
\bgroup
\obeylines
\phagspa

 	0	1	2	3	4	5	6	7	8	9	A	B	C	D	E	F
U+A84x	ꡀ	ꡁ	ꡂ	ꡃ	ꡄ	ꡅ	ꡆ	ꡇ	ꡈ	ꡉ	ꡊ	ꡋ	ꡌ	ꡍ	ꡎ	ꡏ
U+A85x	ꡐ	ꡑ	ꡒ	ꡓ	ꡔ	ꡕ	ꡖ	ꡗ	ꡘ	ꡙ	ꡚ	ꡛ	ꡜ	ꡝ	ꡞ	ꡟ
U+A86x	ꡠ	ꡡ	ꡢ	ꡣ	ꡤ	ꡥ	ꡦ	ꡧ	ꡨ	ꡩ	ꡪ	ꡫ	ꡬ	ꡭ	ꡮ	ꡯ
U+A87x	ꡰ	ꡱ	ꡲ	ꡳ	꡴	꡵	꡶	


ꡏꡟ ꡋꡞ ꡏꡟ ꡋꡞ ꡏ ꡜꡖ ꡏꡟ ꡋꡞ ꡓꡞ ꡏꡟ
ꡈꡋ ꡋꡋ ꡓꡘ ꡈ ꡭ ꡏ ꡏ ꡝ ꡭꡟꡘ ꡓꡋ ꡮꡟꡊ
\egroup
\bgroup
\raggedright

\setcounter{glyphcount}{"A840}

\topline
\phagspa
\newcount\n
\n="A840

\def\htable{^^A
  \def\fm##1{\makebox[2em]##1}^^A
  U+A840\fm 0\fm1\fm2\fm3\fm4\fm5\fm 6\fm 7\fm 8\fm	9\fm A\fm B\fm C\fm D\fm E\fm F}

\htable\par
U+A840^^A 
\loop^^A
  \makebox[2em]{\char\n }^^A   
   \advance\n by1 ^^A
   \ifnum\n<"A850^^A
\repeat
\par U+A850^^A
\loop^^A
  \makebox[2em]{\char\n }^^A   
   \advance\n by1 ^^A
  \ifnum\n<"A860^^A
\repeat
\par U+A860^^A
\loop^^A
  \makebox[2em]{\char\n }^^A   
   \advance\n by1 ^^A
  \ifnum\n<"A870^^A
\repeat
\par U+A870^^A
\loop^^A
  \makebox[2em]{\char\n }^^A   
   \advance\n by1 ^^A
  \ifnum\n<"A878^^A
\repeat

\bottomline

\arial
\hfill Typeset with \texttt{code2000.ttf} and \cmd{\phagspa}

Text: \href{http://babelstone.blogspot.com/2006/12/phags-pa-fonts-1-babelstone-phags-pa.html}{babelstone}
\egroup
\end{scriptexample}

Phags-pa is a historical script related to Tibetan that was created as the national script of
the Mongol empire. Even though Phags-pa was used mostly in Eastern and Central Asia for
writing text in the Mongolian and Chinese languages, it is discussed in this chapter because
of its close historical connection to the Tibetan script. The script has very limited modern use. It bears similarity to Tibetan and has no case distinctions. It is written vertically in columns running for left to right, like Mongolian. Units are often composed of several syllables and sometimes are separated by whitespace.


\subsection{Syloti Nagri}
\index{languages>Sylheti Nagari}
Sylheti Nagari or Syloti Nagri (Silôṭi Nagôri) is the original script used for writing the Sylheti language. It is an almost extinct script, this is because the Sylheti Language itself was reduced to only dialect status after Bangladesh gained independence and because it did not make sense for a dialect to have its own script, its use was heavily discouraged. The government of the newly formed Bangladesh did so to promote a greater "Bengali" identity. This led to the informal adoption of the Eastern Nagari script also used for Bengali and Assamese. It is also known as Jalalabadi Nagri, Mosolmani Nagri, Ful Nagri etc.

\newfontfamily\syloti{NotoSansSylotiNagri-Regular.ttf}
\newfontfamily\damase{damase_v.2.ttf}
\bgroup
\damase
\obeylines
	0	1	2	3	4	5	6	7	8	9	A	B	C	D	E	F
U+A80x	ꠀ	ꠁ	ꠂ	ꠃ	ꠄ	ꠅ	꠆	ꠇ	ꠈ	ꠉ	ꠊ	ꠋ	ꠌ	ꠍ	ꠎ	ꠏ
U+A81x	ꠐ	ꠑ	ꠒ	ꠓ	ꠔ	ꠕ	ꠖ	ꠗ	ꠘ	ꠙ	ꠚ	ꠛ	ꠜ	ꠝ	ꠞ	ꠟ
U+A82x	ꠠ	ꠡ	ꠢ	ꠣ	ꠤ	ꠥ	ꠦ	ꠧ	꠨	꠩	꠪	꠫
\egroup

\subsection{Chakma}

\newfontfamily\chakma{RibengUni.ttf}

\bgroup
\chakma
𑄇𑄳𑄇 Kkā = 𑄇 Kā + 𑄳 VIRAMA + 𑄇 Kā
𑄇𑄳𑄑 Ktā = 𑄇 Kā + 𑄳 VIRAMA + 𑄑 Tā
𑄇𑄳𑄖 Ktā = 𑄇 Kā + 𑄳 VIRAMA + 𑄖 Tā
𑄇𑄳𑄟 Kmā = 𑄇 Kā + 𑄳 VIRAMA + 𑄟 Mā
𑄇𑄳𑄌 Kcā = 𑄇 Kā + 𑄳 VIRAMA + 𑄌 Cā
𑄋𑄳𑄇 ńkā = 𑄋 ńā + 𑄳 VIRAMA + 𑄇 Kā
𑄋𑄳𑄉 ńkā = 𑄋 ńā + 𑄳 VIRAMA + 𑄉 Gā
𑄌𑄳𑄌 ccā = 𑄌 cā + 𑄳 VIRAMA + 𑄌 Cā

\egroup

\subsection{Limbu}

The Limbu script is used to write the Limbu language. The Limbu script is an abugida derived from the Tibetan script. Limbu is a Tibeto-Burman language spoken mainly in Nepal,[3] significant communities in Bhutan, Sikkim, Darjeeling district, India by the Limbu community. Virtually all Limbus are bilingual in Nepali.

\newfontfamily\limbu{code2000.ttf}
\bgroup
\obeylines
\limbu
0	1	2	3	4	5	6	7	8	9	A	B	C	D	E	F
U+190x	ᤀ	ᤁ	ᤂ	ᤃ	ᤄ	ᤅ	ᤆ	ᤇ	ᤈ	ᤉ	ᤊ	ᤋ	ᤌ	ᤍ	ᤎ	ᤏ
U+191x	ᤐ	ᤑ	ᤒ	ᤓ	ᤔ	ᤕ	ᤖ	ᤗ	ᤘ	ᤙ	ᤚ	ᤛ	ᤜ	ᤝ	ᤞ	
U+192x	ᤠ	ᤡ	ᤢ	ᤣ	ᤤ	ᤥ	ᤦ	ᤧ	ᤨ	ᤩ	ᤪ	ᤫ				
U+193x	ᤰ	ᤱ	ᤲ	ᤳ	ᤴ	ᤵ	ᤶ	ᤷ	ᤸ	᤹	᤺	᤻				
U+194x	᥀				᥄	᥅	᥆	᥇	᥈	᥉	᥊	᥋	᥌	᥍	᥎	᥏
\egroup

\subsection{Brahmi}



Brāhmī is the modern name given to one of the oldest writing systems used in the Indian subcontinent and in Central Asia during the final centuries BCE and the early centuries CE. Like its contemporary, Kharoṣṭhī, which was used in what is now Afghanistan and Western Pakistan, Brahmi (native to north and central India) was an \emph{abugida}.

The best-known Brahmi inscriptions are the rock-cut edicts of Ashoka in north-central India, dated to 250–232 BCE. The script was deciphered in 1837 by James Prinsep, an archaeologist, philologist, and official of the East India Company.[1] The origin of the script is still much debated, with current Western academic opinion generally agreeing (with some exceptions) that Brahmi was derived from or at least influenced by one or more contemporary Semitic scripts, but a current of opinion in India favors the idea that it is connected to the much older and as-yet undeciphered Indus script

\subsection{Unicode [U+11000-U+1107F]}


\newfontfamily\brahmi{code2000.ttf}

\begin{scriptexample}[]{Brahmi}
\bgroup
\raggedleft
\brahmi

         
   

\arial
\hfill Text: Asokan Edict typeset with \texttt{NotoSansBrahmi-Regular.ttf} 
\egroup
\end{scriptexample}


\begin{description}
\item[Abkhazia] (Abkhaz: Аҧсны́ Apsny [apʰsˈnɨ]; Georgian: აფხაზეთი Apkhazeti; Russian: Абхазия Abkhaziya) is a disputed territory and partially recognised state controlled by a separatist government on the eastern coast of the Black Sea and the south-western flank of the Caucasus.

\item[Achinese] Acehnese language (Achinese) is a Malayo-Polynesian language spoken by Acehnese people natively in Aceh, Sumatra, Indonesia. This language is also spoken in some parts in Malaysia by Acehnese descendents there, such as in Yan, Kedah.

Formerly, Acehnese language was written in Arabic script called Jawoë or Jawi in Malay language. The script is less common nowadays.[citation needed] Now, Acehnese language is written in Latin script since colonization by the Dutch; with the addition of supplementary letters. The additional letters are é, è, ë, ö and ô.[8] The sound ɨ is represented by 'eu' and the sound ʌ is represented by 'ö' respectively. The letter 'ë' is used to represent the schwa sound which forms the second part in the diphthongs.

\item[Adyghe] Adyghe (/ˈædɨɡeɪ/ or /ˌɑːdɨˈɡeɪ/;[3] Adyghe: Адыгэбзэ adyghabze), also known as West Circassian (КӀахыбзэ), is one of the two official languages of the Republic of Adygea in the Russian Federation, the other being Russian. It is spoken by various tribes of the Adyghe people: Abzekh,[4] Adamey, Bzhedug;[5] Hatuqwai, Temirgoy, Mamkhegh; Natekuay, Shapsug;[6] Zhaney, Yegerikuay, each with its own dialect. The language is referred to by its speakers as Adygebze or Adəgăbză, and alternatively spelled in English as Adygean, Adygeyan or Adygei. The literary language is based on the Temirgoy dialect.
There are apparently around 128,000 speakers of the language on the native territory in Russia, almost all of them native speakers. In the whole world, some 300,000 speak the language. The largest Adyghe-speaking community is in Turkey, spoken by the post Russian–Circassian War (circa 1763–1864) diaspora; in addition to that, the Adyghe language is spoken by the Cherkesogai in Krasnodar Krai.

Ублапӏэм ыдэжь Гущыӏэр щыӏагъ. Ар Тхьэм ыдэжь щыӏагъ, а Гущыӏэри Тхьэу арыгъэ. Ублапӏэм щегъэжьагъэу а Гущыӏэр Тхьэм ыдэжь щыӏагъ. Тхьэм а Гущыӏэм зэкӏэри къыригъэгъэхъугъ. Тхьэм къыгъэхъугъэ пстэуми ащыщэу а Гущыӏэм къыримыгъгъэхъугъэ зи щыӏэп. Мыкӏодыжьын щыӏэныгъэ а Гущыӏэм хэлъыгъ, а щыӏэныгъэри цӏыфхэм нэфынэ афэхъугъ. Нэфынэр шӏункӏыгъэм щэнэфы, шӏункӏыгъэри нэфынэм текӏуагъэп.

Translation: In the beginning was the Word, and the Word was with God, and the Word was God. The same was in the beginning with God. All things were made by him, and without him was not any thing made that was made. In him was life, and the life was the light of men. And the light shineth in darkness, and the darkness comprehended it not.

\item[Albanian]Albanian (shqip [ʃcip] or gjuha shqipe [ˈɟuha ˈʃcipɛ], meaning Albanian language) is an Indo-European language spoken by approximately 7.6 million people,[3] primarily in Albania, Kosovo, the Republic of Macedonia and Greece, but also in other areas of Southeastern Europe in which there is an Albanian population, including Montenegro and Serbia (Presevo Valley). Centuries-old communities speaking Albanian-based dialects can be found scattered in Greece, southern Italy,[4] Sicily, and Ukraine.[5] As a result of a modern diaspora, there are also Albanian speakers elsewhere in those countries and in other parts of the world, including Scandinavia, Switzerland, Germany, Austria and Hungary, United Kingdom, Turkey, Australia, New Zealand, Netherlands, Singapore, Brazil, Canada, and the United States.

Letter:	A	B	C	Ç	D	Dh	E	Ë	F	G	Gj	H	I	J	K	L	Ll	M	N	Nj	O	P	Q	R	Rr	S	Sh	T	Th	U	V	X	Xh	Y	Z	Zh\\
IPA value:	a	b	t͡s	t͡ʃ	d	ð	e	ə	f	ɡ	ɟ	h	i	j	k	l	ɫ	m	n	ɲ	o	p	c	ɾ	r	s	ʃ	t	θ	u	v	d͡z	d͡ʒ	y	z	ʒ\\

\end{description}

\begin{multicols}{5}
\raggedright
Abkhazian\\
Abron\\
Achinese\\
Acoli\\
Adyghe\\
Afar\\
Afrikaans\\
Aghem\\
Akan\\
Akoose\\
Albanian\\
Albay\\
Bikol\\
Amo\\
Asturian\\
Asu\\
Atikamekw
Atsam
Avaric
Aymara
Azerbaijani (Cyrillic script)\\
Azerbaijani (Latin script)\\
Bafia\\
Bafut\\
Balinese\\
Balkan Gagauz Turkish
Bambara (Latin script)
Banjar
Baoulé
Basaa
Bashkir
Basque
Batak
Batak Toba
Belarusian
Bemba
Bena
Betawi
Bikol
Bini
Bislama
Bomu
Bosnian (Cyrillic script)
Bosnian (Latin script)
Breton
Bube
Buginese
Buhid
Bulgarian
Bulu
Buriat
Bushi
Catalan
Cebaara Senoufo
Cebuano
Central Atlas Tamazight (Latin script)
Central-Eastern Niger Fulfulde
Central Huasteca Nahuatl
Central Mazahua
Chamorro
Chechen
Chiga
Chipewyan
Church Slavic
Chuukese
Chuvash
Colognian
Congo Swahili
Cornish
Corsican
Croatian
Czech
Dan
Danish
Dargwa
Dogrib
Duala
Dutch
Dyula
Eastern Huasteca Nahuatl
East Futuna
Efik
Embu
English
Erzya
Esperanto
Estonian
Ewe
Ewondo
Fang
Faroese
Fijian
Filipino
Finnish
Fon
French
Friulian
Fulah
Ga
Gagauz
Galician
Ganda
German
Ghomala
Gilbertese
Gorontalo
Greek
Gronings
Guajajára
Guarani
Guianese Creole French
Gusii
Gwichʼin
Haitian
Hanunoo
Hausa (Latin script)
Hawaiian
Hiligaynon
Hiri Motu
Hungarian
Ibibio
Icelandic
Igbo
Iloko
Inari Sami
Indonesian
Ingush
Interlingua
Inuinnaqtun
Inuktitut (Latin script)
Inupiaq
Irish
Italian
Javanese
Jenaama Bozo
Jju
Jola-Fonyi
Kabardian
Kabuverdianu
Kabyle
Kaingang
Kako
Kalaallisut
Kalanga
Kalenjin
Kalo Finnish Romani
Kamba
Karachay-Balkar
Kara-Kalpak
Karelian
Kashubian

Kazakh (Cyrillic script)

Kerinci
Khasi
Kʼicheʼ
Kikuyu
Kimbundu
Kinyarwanda
Kita Maninkakan
Kom
Komering
Komi
Komi-Permyak
Kongo
Koro
Koro Wachi
Kosraean
Koyraboro Senni
Koyra Chiini
Kpelle
Krio
Kuanyama
Kumyk
Kurdish (Latin script)

Kwasio

Kyrgyz (Cyrillic script)

Kyrgyz (Latin script)

Lak\\
Lakota\\
Lampung Api\\
Langi\\
Lango\\
Latin\\
Latvian\\
Lezghian\\
Limburgish\\
Lingala\\
Lithuanian\\
Lombard
Lomwe
Lower Sorbian
Low German
Lozi
Luba-Katanga
Luba-Lulua
Lule Sami
Luo
Luxembourgish
Luyia
Maasina Fulfulde
Macedonian
Machame
Madurese
Mafa
Maguindanaon
Makasar
Makhu
Makhuwa-Meetto
Makonde
Malagasy
Malay (Latin script)
Maltese
Mandar
Mandingo (Latin script)
Manx
Manyika
Maori
Mapuche
Mari
Marshallese
Masaaba
Masai
Mbunga
Medumba
Mende
Meru
Meta’
Minangkabau
Mohawk
Moksha
Mongo
Mongolian (Cyrillic script)
Montagnais
Morisyen
Mossi
Mundang
Nama
Nauru
Navajo
Naxi
Ndau
Ndonga
Neapolitan
Negeri Sembilan Malay
Ngaju
Ngiemboon
Ngomba
Nigerian Fulfulde
Nigerian Pidgin
Niuean
Northern Sami
Northern Sotho
North Ndebele
North Slavey
Norwegian Bokmål
Norwegian Nynorsk
Nuer
Nyamwezi
Nyanja
Nyankole
Occitan
Oromo
Ossetic
Palauan
Pampanga
Pangasinan
Papiamento
Pohnpeian
Pökoot
Polish
Portuguese
Punu
Quechua
Rajasthani
Rejang
Réunion Creole French
Riang
Rinconada Bikol
Romanian
Romansh
Rombo
Ronga
Rundi
Russian
Rusyn
Rwa
Safaliba
Saho
Sakha
Samburu
Samoan
Sangir
Sango
Sangu
Santali
Sasak
Scots
Scottish Gaelic
Sena
Serbian (Cyrillic script)
Serbian (Latin script)
Serer
Seselwa Creole French
Shambala
Shona
Sicilian
Sidamo
Sinte Romani
Skolt Sami
Slave
Slovak
Slovenian
Soga
Somali
Soninke
Southern Altai
Southern Sami
Southern Sotho
South Ndebele
Spanish
Sranan Tongo
Sukuma
Sundanese
Susu
Swahili
Swati
Swedish
Swiss German
Tachelhit (Latin script)
Tae’
Tagbanwa
Tahitian
Taita
Tajik (Cyrillic script)
Tamashek
Taroko
Tasawaq
Tatar
Tausug
Tavringer Romani
Teso
Tetum
Timne
Tiv
Tokelau
Tok Pisin
Tolaki
Tomo Kan Dogon
Tongan
Tooro
Tornedalen Finnish
Tsonga
Tswana
Tumbuka
Turkish
Turkmen (Latin script)
Tuvalu
Tuvinian
Tyap
Uab Meto
Udmurt
Ukrainian
Ulithian
Umbundu
Unknown Language
Uyghur (Cyrillic script)
Uzbek (Cyrillic script)
Uzbek (Latin script)
Vai (Latin script)
Venda
Vietnamese
Virgin Islands Creole English
Vunjo
Wallisian
Walloon
Walser
Waray
Welsh
Western Frisian
Western Huasteca Nahuatl
Western Mari
Wolof
Xaasongaxango
Xavánte
Xhosa
Yangben
Yao
Yapese
Yemba
Yoruba
Yucatec Maya
Zarma
Zaza
Zeelandic
Zhuang
Zulu
\end{multicols}





\end{document}




% ^^A
\let\frogking\lorem

\chapter{Floats}


Most publications contain a lot of figures and tables. There are instances where
a table can be broken across pages, but this is unacceptable for figures. For this reason
figures and short tables need special treatment. The rather naıve method of treating these
objects is to start a new page every time a floating object is too large to fit on the present
page. A more sophisticated method to tackle this problem is to ‘‘float’’ any object that
does not fit on the current page to a later page while filling the current page with text.

This is why these objects are called floating objects. LATEX provides two environments
that are treated as floating objects: the figure and the table environments. Both environments
are written the same way; they differ only in the text that is prepended in the
caption. Moreover, there are two environments that can be used in double column documents
to generate floats that may occupy both columns: the \cs{figure*} and the \cs{table*}
environments. Here is how we can begin a table or a figure:

\begin{teXXX}
 \begin{table}[placement specifier ]
 \begin{figure}[placement specifier]
\end{teXXX}

An optional placement specifier is used to tell LATEX where the float is allowed to be
moved to. The placement specifier consists of a sequence of float placing permissions:

\begin{table}[htbp]
\begin{tabular}{ll}
\toprule
Placement   & position\\
\midrule
h                 & here\\
t                  & top\\
b                 & bottom\\
p                 & on a special page containing only floats\\
\bottomrule
\end{tabular}
\end{table}




Apart from the float placing permissions above there exists a fifth one, namely (!), which
forces LATEX to actually ignore most of the internal parameters related to float placement.
LATEX also provides the command \cs{suppressfloats},which prevents LATEX from putting


\section{The float package}

The \docpkg{float} package provides a friendly interface to define new float objects. Moreover, the package
defines certain ‘‘float styles’’ that can be used to define new floating objects.  It
was designed by Anselm Lingnau. New float objects can be defined with the command

\begin{verbatim}
\newfloat{type}{placement}{ext }[within ]
\end{verbatim}



Here type is the `��type'�� of the new class of floats (e.g., program, diagram, etc.),
placement gives the default placement specifier, and ext is the filename extension
for the file that will keep the captions in cases wherewewant to have a list of programs,
list of diagrams, or other lists. The optional argument within is used to number float
objects within some sectioning unit (e.g.,chapter, section). Here is a complete example:

\begin{teXXX}
\floatstyle{plain}
\newfloat{Photo}{htbp}{fot}[section]
\end{teXXX}



\makeatletter
%\newcommand\fs@framed{\def\@fs@cfont{\bfseries}\let\@fs@capt\floatc@ruled
%\def\@fs@pre{\hrule height.8pt depth0pt \kern2pt}%
% \def\@fs@post{\kern3pt\hrule\relax}%
% \def\@fs@mid{\kern2pt\hrule\kern2pt}%
% \let\@fs@iftopcapt\iftrue}
%\makeatother


\floatstyle{plain}
\newfloat{Photo}{htbp}{fot}%[section]

\begin{Photo}
 \centering
 \includegraphics[width=0.65\linewidth]{china-05}
\caption[a short caption]{If the caption is very long it is formatted as a paragraph, which is flushleft. If it is short it will be centered. }
\end{Photo}


\begin{Photo}
 \centering
 \includegraphics[width=0.65\linewidth]{china-06}
\caption{. . . caption . . . }
\end{Photo}

\begin{Photo}
 \centering
 \includegraphics[width=0.65\linewidth]{yaleartschool}
\caption{. . . caption . . . }
\end{Photo}



Note that after each such definition, a new 
environment will be available. Naturally,
its name depends on the ��type�� (e.g., the example code above will create the program
environment). The ��float style�� can be specified with the \cs{floatstyle} command. The
command takes only one argument, which is the name of a ‘‘float style’’:

\begin{teXXX}
\begin{Example}
     First verbatim line.
     Second verbatim line.
     Third verbatim line.
\end{Example}
\end{teXXX}



\floatstyle{ruled}
\newfloat{Example}{htbp}{loe}[chapter]

 \begin{Example}
 \begin{verbatim}
   \begin{Photo}
      \centering
      \includegraphics[width=0.65\linewidth]{./graphics/level3}
      \caption{. . . caption . . . }
   \end{Photo}
\end{verbatim}
\caption{Example using verbatim code}
 \end{Example}

\begin{Photo}
 \centering
 \includegraphics[width=0.85\linewidth]{old-timer-structural-worker}
\caption{. . . caption . . . }
\end{Photo}

\newlength{\egwidth}\setlength{\egwidth}{0.48\textwidth}

\newenvironment{ega}%
{\begin{list}{}{\setlength{\leftmargin}{0.02\textwidth}%
\setlength{\rightmargin}{\leftmargin}}\item[]\footnotesize}%
{\end{list}}

\newenvironment{egbox}%
{\begin{minipage}[t]{\egwidth}}%
{\end{minipage}}

\newcommand{\egstart}{\begin{ega}\begin{egbox}}
\newcommand{\egmid}{\end{egbox}\hfill\begin{egbox}}
\newcommand{\egend}{\end{egbox}\end{ega}}

% one or two other commands

\newcommand{\fn}[1]{\hbox{\tt #1}}
\newcommand{\llo}[1]{(see line #1)}
\newcommand{\lls}[1]{(see lines #1)}


\egstart
\begin{verbatim}
Here is some advice to remember:
\begin{quotation}
Environments for making
...other things as well.

Many problems
...environments.
\end{quotation}
\end{verbatim}
\egmid%
Here is some advice to remember:
\begin{quotation}
Environments for making quotations
can be used for other things as well.

Many problems can be solved by
novel applications of existing
environments.
\end{quotation}
\egend

The \cs{tabbing} environment overcomes this problem. Within it you set
tabstops and tab to them much like you do on a typewriter.  Tabstops are
set with the |\=| command, and the |\>| command moves to the
next stop.  The
|\\| command is used to separate each line.  A line that ends |\kill|
produces no output, and can be used to set tabstops:


\begin{teX}
\begin{tabbing}
 Income \=Expenditure \= \kill
 Income \>Expenditure \>Result\\
 20s 0d  \>19s 11d \>Happiness\\
 20s 0d  \>20s 1d  \>Misery \\
\end{tabbing}
\end{teX}

\smallskip

\begin{tabbing}
Income \=Expenditure \=    \kill
Income \>Expenditure \>Result \\
20s 0d \>19s 11d \>Happiness   \\
20s 0d \>20s 1d  \>Misery    \\
\end{tabbing}


Unlike a typewriter's tab key, the |\>| command always moves to the next
tabstop in sequence, even if this means moving to the left.  This can cause
text to be overwritten if the gap between two tabstops is too small.



\section{Environment}

\begin{teX}
\def\beginstory{
  \vskip 0.5in                 % Skip down 1/2 before story
   \begingroup                  % Start of formatting properties
   \leftskip 1in\rightskip 1in  % Wider margins for narrower text
   \itshape                     % Italic font
   \noindent{.\dotfill{}.\par}  % Make dotted line
	% Text after close of \beginstory will be story formatted
}

\def\endstory{
  \par\noindent{.\dotfill{}.\par}  % Make dotted line
  \endgroup                        % End of formatting properties
  \vskip 0.5in                    % Skip final 1/2 inch
}

\beginstory 

  Just a short story 

\endstory

\end{teX}



Here's a story about the formative era of personal computing. I
originally wrote it in 1999, but the point it makes is still valid.
Hope you like it.



\subsection*{The \protect\latex way}
LaTeX implements macros |\begin| and |\end|. These are a generic pair whose argument determines the environment that's being begun or ended.

LaTeX makes it much easier to code environments. Here's a generic environment definition:

|\newenvironment{environment_name}{stuff to do before text}{stuff to do after text}|

That's it -- the \cs{newenvironment} macro takes three arguments:
The name of the environment being created
The stuff to do before the text being assigned that environment
The stuff to do after the text being assigned that environment

The resemblance to \tex paired macros is obvious, but \latex  environments make it generic across all environments, and place the beginning and ending code in one place. Not only that, but because the environment has one name instead of two different names, it's very easy for a front end like LyX to assign environments to highlighted stretches of text.

The |\newenvironment|  macro works only when the environment name is undefined. If there's already an environment with that name, use|\renewenvironment|  instead. If you don't know, there are ways to test.

The following is a LaTeX version of the |\beginstory| \ldots | \endstory| example, with the two macros folded into the definition of one environment called story.


Lamport, cleverly defined macros that automatically, create the necessary \tex \cs{begingroup} and \cs{endgroup} commands. You can find the code in the |source2e| file and which is shown below:\footnote{You can find the full code in \texttt{File y, for ltmiscen.dtx}}

\begin{teX}
\def\begin#1{%
  \@ifundefined{#1}%
  {\def\reserved@a{\@latex@error{Environment #1 undefined}\@eha}}%
  {\def\reserved@a{\def\@currenvir{#1}%
  \edef\@currenvline{\on@line}%
  \csname #1\endcsname}}%
  \@ignorefalse
  \begingroup\@endpefalse\reserved@a}


\def\end#1{%
  \csname end#1\endcsname\@checkend{#1}%
  \expandafter\endgroup\if@endpe\@doendpe\fi
  \if@ignore\@ignorefalse\ignorespaces\fi}
\end{teX}

In \latex environments are defined as 
|\begin{foo}| and |\end{foo}| which are are used to delimit environment |foo|.
|\begin{foo}| starts a group and calls |\foo| if it is defined, otherwise it does
nothing.

|\end{foo}| checks to see that it matches the corresponding |\begin| and if so,
it calls |\endfoo| and does an |\endgroup|. Otherwise, |\end{foo}| does nothing.
If |\end{foo}| needs to ignore blanks after it, then |\endfoo| should globally set
the |@ignore| switch true with |\@ignoretrue| (this will automatically be global).

NOTE: |\@@end| is defined to be the |\end| command of TEX82.
|\enddocument| is the user's command for ending the manuscript file.
|\stop| is a panic button to end \tex in the middle.


\section*{Checking the environment}

This is interesting in that we can use \cs{@currenvir} to check if a command is within a particular environment. The following code will be used to typeout the environment.

\begin{teX}
\begin{enumerate}
   \item Check environment with |@currenvir|
   \makeatletter
   \item The current environment is \@currenvir
   \makeatother
\end{enumerate} 
\end{teX}

\begin{enumerate}
\item Check environment with |@currenvir|
\makeatletter
 \item The current environment is \@currenvir
\makeatother
\end{enumerate}

The \cs{@checkend} \index{Latex kernel!@checkend} uses the \cs{@currenvir}\index{Latex kernel!@currenvir} to see if there is a matching
begin environment and if it cannot find it produces an error.

\begin{teX}
\def\@checkend#1{%
   \def\reserved@a{#1}
   \ifx\reserved@a\@currenvir 
   \else
     \@badend{#1}
   \fi
}
\end{teX}

It is a pity that there is no real guide for explaining the \latex macros, other than just reading through them. Lamport and later his associates managed to produce code that offers the user a friendly API. Besides the scenes of this API, it also offers the package writers hundreds of useful commands.






% ^^A \chapter{Creating Book Designs}

\section{First Steps}

In this chapter we will develop a full book template from scratch. Before we delve into it further, I would like to emphasize that the |phd| system is a bit different from classes. A |phd|  style includes all the information necessary for the typesetting of a document. I have called this a style template. It is slightly different from a class system where generic commands might be included that can develop a totally different look. An identical design with perhaps different colors and fonts and other minor changes, is termed a \textit{theme}. 

Unlike book designers who would first focus on fonts, we will first give our attention to the structural elements of the book. I will be using as an example the \textit{Linear Algebra}. 

\begin{figure}[htbp]
\includegraphics[width=\textwidth]{linear-chapter}
\caption{The opening chapter can leave a blank page. }
\label{fig:linear1}
\end{figure}

Figure~\ref{fig:linear1} shows the chapter head design. This is an interesting and challenging design that we will not
easily make with the |phd| standard chapter head routines. The chapter starts with a full line and a structural element that is called \emph{Introductory Example}. The heading of this also goes to the Table of Contents. So the chapter opening page starts with a rule and end with a rule. The ending rule in Figure~\ref{fig:linear2} can be seen in the next figure. 

\subsection{Chapter Opening}

One of the first things you will need to take care of, is to design if the style template should cater for opening at right or if it is to open at any place. Another decision you will need to make, is what to do with blank pages. Personally I dislike them and suggest, if you are going to have them to either introduce epigraphs or full page images.

\begin{figure}[htbp]
\includegraphics[width=\textwidth]{./images/intentionally-blank.jpg}
\caption{The opening chapter can leave a blank page. }
\label{fig:linear2}
\end{figure}

\subsection{The User Commands}

It is always best to start thinking about the user commands, as we go along in order to provide a user friendly
interface, without the introduction of too many keys. We also need to name our template. We will name it \emph{andrea} in honour of the Designer of the book, who was Andrea Nix. Andrea designed many of the Pearson books that were mostly textbooks and has a unique distinctive design style that can make a mathematics book fun to read.

\begin{verbatim}
\cxset{chapter template = andrea,
          chapter opening = right}
\end{verbatim}

\def\fullwidthrule{%
\bgroup
\color{thesectioncolor}
\noindent\makebox[\linewidth]{\rule{\paperwidth}{2.5pt}}%
\egroup
}

\fullwidthrule
\fullwidthrule

\makeatletter
\newcommand\print@finalparams@cx{%
  \parindent0pt
  \par\noindent
   Final page layout dimensions and booleans
  \string\paperwidth\space\space\the\paperwidth\\%
  \string\paperheight\space\space\the\paperheight\\%
  \string\textwidth\space\space\the\textwidth\\%
  \string\textheight\space\space\the\textheight\\%
  \string\oddsidemargin\space\space\the\oddsidemargin\\%
  \string\evensidemargin\space\space\the\evensidemargin\\%
  \string\topmargin\space\space\the\topmargin\\%
  \string\headheight\space\space\the\headheight\\%
  \string\headsep\space\space\the\headsep\\%
  \string\footskip\space\space\the\footskip\\%
  \string\marginparwidth\space\space\the\marginparwidth\\%
  \string\marginparsep\space\space\the\marginparsep\\%
  \string\columnsep\space\space\the\columnsep\\%
  \string\columnseprule\space\space\the\columnseprule\\%
  \string\skip\string\footins\space\space\the\footins\\%
  \string\hoffset\space\space\the\hoffset\\%
  \string\voffset\space\space\the\voffset\\%
  \string\mag\space\space\the\mag\\%
  \if@twocolumn\string\@twocolumntrue\space\fi%
  
  \if@twoside\string\@twosidetrue\space\fi%
  
  \if@mparswitch\string\@mparswitchtrue\space\fi%
  
  \if@reversemargin\string\@reversemargintrue\space\fi%
  
}%

\print@finalparams@cx
\par
\makeatother

As we will not be sure our calculations are right or wrong (the rules can disappear at the edge of the page) I have
taken 5pt out from the left or right parameters to see that we have done the calculations properly.


We also need to check on oddside pages as well. Remember the switch \cmd{\@mparswitchfalse} will set the margin pars to be on the same size. This layout only has them on the right pages. We need to set it to false.



Another decision we need to make is if we going to draw the layout using TeX commands or one of the graphic units. Using TikZ, can be much easier, but we need to ensure we know where we are on the page. Alternatively we can use the remember picture, overlay hack to accomplish it. We will first give it a try with rules and boxes.

Now we have the dimensions of the left margin and right margin width right we can continue with the layout.

The next item we will draw is the corner frame.
\bigskip

\bgroup
\parindent0pt
\parskip0pt
\offinterlineskip

\newbox\sectiontitlebox
\setbox\sectiontitlebox=\hbox{\hspace*{-2.5cm}\Large\bfseries\arial \mbox{\color{orange200}45.1} \mbox{\color{smithsonian}Sections}}\copy\sectiontitlebox\par
\vskip-6.5pt
\leavevmode%
\color{cyan500}%
\llap{\vrule height0.9pt width2.5cm}%
\rlap{\vrule height0.9pt width\textwidth depth0pt \relax}
\egroup
%\makebox[0pt]{\rule{3cm}{1pt}}%

\section{Sections}

The sections follow a very similar style to that of the chapter heading with rules and similar colours. 

\begin{figure}[htbp]
\includegraphics[width=\textwidth]{./images/linear-section.jpg}
\caption{The opening chapter can leave a blank page. }
\end{figure}

The book does not use subsection. As a matter of fact most books don’t consider that numbering of subsections offers an advantage to the reader. 

\begin{figure}[htbp]
\includegraphics[width=\textwidth]{./images/linear-theorem.jpg}
\caption{The opening chapter can leave a blank page. }
\end{figure}


\section{Examples and Solutions}

The examples are straight forward typesetting and numbering. The specification should be that they be numbered consequently with the example and solution in capital letters to be distinguished by the type size. The colour is to be identical. The example heading is inlined with about a quadd of space between it and the text that follow. The solution is on its own line and it is followed normally by a list which is numbered alphabetically. In other cases it is in-lined see the page at the left. We can perhaps handle this with a starred command, one for stand alone heading and another for an inlined. I will come back with some suggestions for this before, we delve into codin.

\begin{figure}[htbp]
\includegraphics[width=\textwidth]{./images/linear-example.jpg}
\caption{The book will have a lot of examples and their solutions. }
\end{figure}

\section{Exercises}

These are modelled after sections and are also numbered. They are numbered in a different counter from that of sections and are reset at every chapter. 

\begin{figure}[htbp]
\includegraphics[width=\textwidth]{./images/linear-exercises.jpg}
\caption{The book will have a lot of examples and their solutions. }
\end{figure}

\begin{figure}[htbp]
\includegraphics[width=\textwidth]{./images/linear-supplementary.jpg}
\caption{The book will have a lot of examples and their solutions. }
\end{figure}

\section{Figures and diagrams}

\begin{figure}[htbp]
\includegraphics[width=\textwidth]{./images/linear-figures.jpg}
\caption{The book will have a lot of examples and their solutions. }
\end{figure}

The user commands should also be minimized and would follow normal LaTeX conventions, with the exception we will redefine an environment \cmd{\begin}\meta{marginfigure}\ldots. The margin figures are both numbered as well as unumbered, so we will use normal LaTeX conventions to both define them as well as for author commands. 



\section{Geometry}

Although we spend a good part of the Chapter on page design, reviewing historical typographical paper sizes, modern book production of text books is not bound with tradition but economics. High speed printing technology uses rolls and pages can be printed up to 64 pages at a time. We will follow the books dimensions which are 7.75x10.25in. The text area occupies approximately 0.67 of the textwidth and is particularly well balanced. Many mathematical text books come out too dense and are difficult to be used by students. 







%


%  ^^A \makeatletter
\newenvironment{adjustmargins}[2]{%
 \begin{list}{}{%
 \topsep\z@%
 \listparindent\parindent%
 \parsep\parskip%
 \checkoddpage
 \ifoddpage % odd numbered page
 \@ifmtarg{#1}{\setlength{\leftmargin}{\z@}}%
 {\setlength{\leftmargin}{#1}}%
 \@ifmtarg{#2}{\setlength{\rightmargin}{\z@}}%
 {\setlength{\rightmargin}{#2}}%
 \else % even numbered page
 \@ifmtarg{#2}{\setlength{\leftmargin}{\z@}}%
 {\setlength{\leftmargin}{#2}}%
 \@ifmtarg{#1}{\setlength{\rightmargin}{\z@}}%
 {\setlength{\rightmargin}{#1}}%
\fi
}
\item[]}{\end{list}}

\makeatother


\chapter{Pages}

\parindent1em

The page is the main element in a book and its geometry and layout has been studied extensively by typographers. In this chapter we outline the typographical tradition, methods to specify layouts using \latex and associated issues, such as adjusting margins within a page.

Bringhurst notes that ``much typography is based, for the sake of economy on standard industrial sizes, from $35\times45$ inch press sheets to $3 1/2$ x 2 inch conventional business cards. Some formats as the booklets that accompany mobile telephone kits, are condemned to especially rigid restrictions of size.  

There may already be some restrictions on the page size you choose depending on your method of production and distribution. If you aim to output pages on a desktop printer then a standard size like A4 ($297\times210$)mm or US letter ($11\times 8 1/2$ inches) is advisable. If you have the opportunity and necessity of selecting the dimensions of the page you have a great opportunity to enhance the page layout of your book.

\section{Selecting  paper sizes}

Besides the limitations of the method of printing, another consideration is the size of book you writing and the
audience you are addressing it. If you are only producing a 60 page book, paper with smaller dimensions might be more appropriate than a blockbuster novel. 

History, natural science, geometry and mathematics are all relevant to typography.


\begin{figure}[ht]
\centering
\includegraphics[width=0.5\textwidth]{./images/preparing-paper.jpg}
\caption{Getting paper prepared for printing \protect\cite{moxon}.}
\end{figure}

Originally, paper sizes were determined by the moulds the paper was
made in and the use the result was put to. While many hundreds of variations have occurred throughout the centuries, in the main there have seldom been more than six categories of sizes in use since the fourteenth century. These have often come down to us bearing the names of the figures featured in the paper's watermarks, such as \emph{foolscap},
\emph{elephant}, \emph{pot}, and \emph{crown}. To enable the creation of smaller sizes from
existing larger sizes, the sheets have since the Middle Ages been proportioned
with their sides in the ratio of \(1:\sqrt{2}\). For example, quarto (4to,
formerly 4to) and octavo (8vo, formerly 8vo) sizes are obtained by cutting
or folding standard sizes four and eight times respectively.
Former British paper dimensions still used the old sizes before decimalized
versions replaced them; US dimensions still retain most of these
(untrimmed) paper sizes, in inches. Both are still encountered in specialist and bibliographic work, and in reproducing earlier or foreign formats:


\section{Canonical Layouts}

Typographers derive proportions that naturally occur in nature, and pages that embody
them recur in manuscripts and books from Rennaissance Europe, Tang and Song dynasty
China, early Egypt and ancient Rome.  
These numbers are $\pi=$3.14159\ldots , which is the circumference of a circle whose diameter
is one; $\sqrt{2}=$1.41421\ldots , which is the diagonal of a unit square; 
$e=2.71828$  \ldots ,which is the base of the natural logarithms; and $\phi=1.61803$ \ldots ,a number which is discussed later on. Certain of these proportions appear in he structure of the human body; other appear in musical scales. Indeed, one of the simplest of all systems of 
page proportions is based on the familiar intervals of the diatonic scale. Pages that
embody these basic musical proportions have been in common use in Europe for more than a thousand year.

 \begin{figure}
 \makebox[\textwidth]{\makebox[1.1\textwidth][r]{%
 \unitlength=0.0015\textwidth
 \let\ul\unitlength
	\begin{picture}(184,320)(0,-20)
	\put(0,0){\framebox(184,297){}}
	\put(27,70){\makebox(0,0)[bl]{\color{thegray}\rule{113\ul}{184\ul}}}
	\put(92,-20){\makebox(0,0)[t]{Golden number canonical layout}}
	\color{red}
	\put(92,162){\circle{184}}
	\linethickness{.2pt}
	\multiput(0,297)(1.98918918919,-3.21081081082){93}{\line(184,-297){1}}
	\end{picture}%
 \hfill
	\unitlength0.0015\textwidth
	\let\ul\unitlength
		\begin{picture}(210,330)(0,-20)
		\put(0,0){\framebox(210,297){}}
		\put(25,51){\makebox(0,0)[bl]{\color{thegray}\rule{149\ul}{210\ul}}}
		\put(105,-35){\makebox(0,0)[b]{ISO canonical layout}}
		\color{red}
		\put(105,156){\circle{210}}
		\multiput(0,297)(2.27027027027,-3.21081081082){93}{\line(210,-297){1}}
		\end{picture}%
 \hfill
   \unitlength0.001591\textwidth
   \let\ul=\unitlength
   \begin{picture}(220,330)(0,-20)
   		\put(0,0){\framebox(220,280){}}
   		\put(20.742,33.6){\makebox(0,0)[bl]{\color{thegray}\rule{172.86\ul}{220\ul}}}
   		\put(110,-35){\makebox(0,0)[b]{Letter paper canonical layout}}
   		\color{red}
   		\put(110,143.6){\circle{220}}
		\multiput(0,280)(2.37837837838,-3.02702702703){93}{\line(220,-280){1}}
   \end{picture}
 }}%
 \caption{A right page with the relevant diagonal, the text block and the canonical circle.
 In this figure the important information is the page proportions, not the scale; matter of
 fact the letter paper is 17.6 mm shorter than the A4 paper, but the drawings to the same
 height emphasize the relative proportions of the various page parts. \cite{canonicallayout}}
 \label{fig:canoniclayout}
 \end{figure}
 
The package \pkgname{xlayouts} and also Beccari’s \pkgname{canonical} layouts provide both graphical as well as settings for determining page layouts that approach canonical layouts. In reality modern book design has diverged from these principles. 

\begin{figure}[hb]
\cxset{spread xsteps=9,
          spread scale=0.20,
          spread width=0.5\textwidth}
\centering
\drawcanons
\end{figure}


\begin{figure}
  \includegraphics[width=0.5\linewidth]{./graphics/A-sizes.png}
  \caption{When a sheet whose proportions are $1$:$\surd{2}$ is folded in half, the result is a sheet half as large but with \emph{the same proportions}. Standard paper sizes on this principle have been in use in Germany since the early 1920s. The basis of this system is the A0 sheet, which has an are of 1 m$^2$. Yes because it is \textit{reciprocal with nothing but itself}, the ISO page in isolation is the least musical of all the major page shapes. It needs a textblock of another shape or contrast.}
   \label{fig:marginfig1}
\end{figure}

The advantages of basing a paper size upon an aspect ratio of $\surd{2}$ were already noted in 1786 by the German scientist Georg Christoph Lichtenberg, in a letter to Johann Beckmann[2]. The formats that became |A2|, |A3|, |B3|, |B4| and |B5| were developed in France, and published in 1798 during the French Revolution, but were subsequently forgotten. \cite{letimbre2136}

Early in the twentieth century, Dr Walter Porstmann turned Lichtenberg's idea into a proper system of different paper sizes. Porstmann's system was introduced as a DIN standard (DIN 476) in Germany in 1922, replacing a vast variety of other paper formats. Even today the paper sizes are called "DIN Ax" in everyday use in Germany.

The main advantage of this system is its scaling: if a sheet with an aspect ratio of $\surd{2}$ is divided into two equal halves parallel to its shortest sides, then the halves will again have an aspect ratio of $\surd{2}$. Folded brochures of any size can be made by using sheets of the next larger size, e.g. |A4| sheets are folded to make |A5| brochures. The system allows scaling without compromising the aspect ratio from one size to another – as provided by office photocopiers, e.g. enlarging |A4| to |A3| or reducing |A3| to |A4|. Similarly, two sheets of |A4| can be scaled down and fit exactly 1 sheet without any cutoff or margins.

%\cxset{try grid=false}
%\thispagestyle{grid}


The weight of each sheet is also easy to calculate given the basis weight in grams per square metre (g/m² or `'gsm"). Since an |A0| sheet has an area of 1m² , its weight in grams is the same as its basis weight in g/m². A standard |A4| sheet made from 80 g/m² paper weighs 5g, as it is one 16th (four halvings) of an A0 page. Thus the weight, and the associated postage rate, can be easily calculated by counting the number of sheets used.

Unlike the |A4| standard paper, the origin of the dimensions of letter size paper are lost in tradition. The American Forest and Paper Association argues that the dimension originates from the days of manual paper making, and that the 11-inch length of the page is about a quarter of ``the average maximum stretch of an experienced vatman's arms".[1] However, this does not explain the width or aspect ratio. What is known is that Ronald Reagan made this the paper size for U.S. federal forms; previously, the smaller "official" size (8 in × 10½ in or 203.2 mm × 266.7 mm) was used.[1] Letter or US Letter is the most common paper size for office use in the United States and Canada. It is 8$\frac{1}{2}$ by 11 inches (exactly 215.9 mm × 279.4 mm).

\section{The Typearea}

According to \cite{bringhurst2005}, in typography margins must do three things. They must lock the
textblock to the page and lock the facing pages to each other through the force of their proportions. Second, they must frame the textblock in a manner that suits its design. Third, they must protect the textblock, leaving it easy for the reader to see and convenient to handle. 

In most well designed books fifty per cent of the character and integrity of a printed page lies in its letterforms. Much of the other fifty per cent resides in its margins.


\subsection{Readability}

Another aspect that determines the text area, is the readability of the text. Here you need to take into account the readers of your book. For children and older persons a larger type and shorter lines are preferred.

\begin{macro}{\alphabetlength}
The macro |\alphabetlength| prints the length of the alphabet. The length of the alphabet in this text is \alphabetlength. If this is a good choice is debatable, but after all this is just a long document, with many chapters and my aim was to produce a reference and a test document. The macro is defined in the |xlayouts|  package, which is loaded automatically by the |phd| package or class. 
\end{macro}

Traditionally  a line that is approximately 1.4 times the alphabet length is considered good practice. The length of one line of text in this document is \the\textwidth giving a ratio of \alphabetsperline.

\DescribeMacro{\printreadability} prints a small table with some readability figures. If LuaTeX is used, this table is slightly longer and prints some other statistics as well. 

\begin{figure}[htbp]
\drawtriallayout
\bigskip

\printreadability
\captionof{figure}{Page layout diagram and readability statistics (using the \pkgname{xlayouts} package).}
\end{figure}

The macros described above are loaded by the |xlayouts| package, which forms part of the |phd| budle. There are macros for drawing trial layouts 


\section{Examples}
Folowing the nomenclature introduced b Bringhurst in analyzing the examples on the following pages, 
these symbols are used:

%% Align at the = sign 
\begin{table}[htbp]
\begin{tabular}{l l @{ = } p{6cm}}
\textit{Proportions:}      &P  &  page proportion $h/w$\\
~                      &T &  textblock proportion: $d/m$\\
\textit{Page size:}         &w &  width of page (trim-size)\\
~                      &h  & height of page (trim-size)\\
\textit{Textblock:}           &m & measure (width of primary textblock)\\
~                      &d  & depth of primary textblock (excluding running heads, folios etc)\\                      
~                      &$\lambda$ & line height (type size plus added lead)\\
~                      &$n$ & secondary measure (width of secondary column)\\
~                      &$c$  & column width, where there are even multiple columns\\
\textit{Margins}  &$s$  & spine margin (back margin)\\
~                      &$t$   & top margin (header margin)\\
                        &$e$  & fore-edge (front margin)\\
                        &$f$   & foot margin\\
                        &$g$  & internal gutter (on a multiple-column page)\\
\end{tabular}
\caption{Symbols used to demonstrate various ratios in books}
\end{table}
\medskip

\begin{figure}
  \includegraphics[width=\linewidth]{./graphics/page.png}
  \caption{Page nomenclature}
   \label{fig:marginfig1}
\end{figure}

More variables are necessary to specify all the variables handled by a \latex\
page. For the time being the examples refer to dimensions from historical works
in typography and should sufffice.

\subsection{Hypneroto}

\begin{figure}[htbp]
\centering
  \includegraphics[width=\linewidth]{./graphics/hypneroto.jpg}
\caption{The work is lauded for the originality of its
design. Several sequential double page
illustrations add a visual dimension to the
progression of the narrative, and act like an
early form of the strip cartoon. There is an
obsession with movement throughout which is driven
on by the illustrations, resulting in the
impression of bodies moving from one page to the
next. Other typographical innovations include
playing with the traditional layout of the text;
in the opening shown here, for example, the pages
are shaped in the form of goblets. The dimensions
of the text are: $P=1.5[2:3]$; T=1.7 (tall pentagon);
Margins: s=t=w/9; e=2 s. The text is a fantasy
novel, Francesco Colonna's Hypnerotomachia
Poliphili, set in a roman font cut by Francesco
Griffo. (Aldus Manutius, Venice, 1499). Original
size: $20.5\times31$\thinspace cm.}
\label{fig:hypneroto}
\end{figure}


%  \label{fig:layout}



The book was printed by Aldus Manutius in Venice in December 1499. The book is anonymous, but an acrostic formed by the first, elaborately decorated letter in each chapter in the original Italian reads \textsc{\small POLIAM FRATER FRANCISCVS COLVMNA PERAMAVIT}, \enquote{Brother Francesco Colonna has dearly loved Polia.} However, the book has also been attributed to Leon Battista Alberti by several scholars, and earlier, to Lorenzo de Medici. The latest contribution in this respect was the attribution to Aldus Manutius, and arguably, a Francesco Colonna, a wealthy Roman Governor. The author of the illustrations is even less certain, but contemporary opinion gives the work to Benedetto Bordone.

\section{Contemporary book layouts}

All these sound mystical with religious undertones, but we need to remember that early printers made their livelihoods from printing mostly religious books.

From the mystics to the modern, let us study Figure~\ref{fig:nudes}

\begin{figure}[htbp]
\centering

\fbox{\includegraphics[width=\linewidth]{nudes.jpg}}

\caption{In this layout, the placement of various size images on the right pages, makes the margins disappear to the eye. As the whole book, is made of similar pages\ldots }
\end{figure}

Modern designers are more cryptic. One book that I found more useful is Ambrose/Harris \textit{Layout}. The book brings together examples of layout, both contemporary  and historic, from aroudnd the world. It contains examples from leading graphic designers to provide a sample of rich and diverse possibilities for the creative use of layout.

As it will become apparent from what follows, although at first look it appears that all design principles have disappeared into post modern designs, all design is undertaken with reference to a certain set of principles, either by consciously
choosing to follow or by deliberately ignoring or subverting them. The collective body of principles represents different approaches to design and layout construction.

The principles in this section have been used
through the ages to produce designs that are
pleasing to the eye and that organise information
clearly and efficiently, two of the challenges facing
every graphic designer. These principles affect
decisions made at the heart of the design process,
as they provide the basis of how space is divided.

\section{A design must capture the spirit of the times.}

The word \emph{zeitgeist} originates from the German zeit (time) and geist (spirit),
and so literally means spirit of the age. In graphic design, each decade can
be defined by several predominant zeitgeists that somehow seem to capture
their essence. Today, in graphic design, we can see a zeitgeist for the use of
sophisticated computer graphics giving a very close approximation to reality
in addition to another, which is a backlash to this, in the form of rough-and-ready
hand-drawn designs.

\section{Objects on a Page}

How an object is placed on a page has a dramatic
impact on how it is received and interpreted by
the viewer, and the message that it delivers. We
have looked at how grids can be used to guide
element placement on a page, but maintaining a
sense of order is not the only consideration when
laying out a design.

Object placement helps form the narrative of
a design and is constructed from an understanding
of how we read a page. The narrative of a design
can be created and altered by a wide range of
placement and intervention strategies, such as
how white space is used, the balance and relative
weight given to other objects, the juxtaposition or
contrast of objects and so on.

This chapter will outline some of the
fundamental approaches to object placement.

\section{White Space}

White space is not necessarily white, as it refers to any space in the design
without text or graphic elements. Designers naturally insert white space into
their designs to help the composition and make the information the design
contains easier to access, such as leaving margins at the sides of the page that
create space around text blocks. Swiss typographer Jan Tschichold called white
space ‘the lungs of a good design’. Without white space, with every part of the
design area filled, there is a danger that a design would look cramped and
difficult to understand.

White space can instil different perceptions in a viewer depending on how it is
used and the content it is associated with. White space may give the impression
of luxury and extravagance for a full-page photograph. However, it may also give
the impression that there are gaps in a layout that is rather full, or worse, that
there is insufficient content to fill a page. Newspapers try to establish a rational
balance between giving space to page elements to meet the conflicting demands
of the need for typographical sensitivity and readability, while filling a page with
news so that the reader feels they are getting value for money. Habitually readers
expect a newspaper to be ‘full’, which means it is harder to achieve typographic
balance. In contrast, where filling space is of less concern, such as the example
below, white space becomes a more overt part of the design.



\section{Grids}

\subsection{The Baseline Grid}

The baseline grid is the (invisble) graphic foundation upon which a design is constructed and provides a visual guide for positioning and ligning page elements with an accuracy that is difficult to achieve by eye alone. Knuth's TeX focuses almost primarily on getting this one right.

\section{Pace}

It came to me as a big surprise that a books layout must have \textit{pace}. This essentially is the alternation of pages, between say images and text.

\begin{figure}[htbp]
\parindent=0pt
\includegraphics[width=\textwidth]{pace}

\end{figure}

Thumbnails are smaller versions of the spreads of a publication presented on a
page that allow a designer to gauge its pace and balance at the macro level
without focusing on details. Thumbnails allow a designer to look at the
narrative of the publication and tune it as a whole, rather than on a spread-byspread
basis.

Pictured are thumbnails for Miss X, a book for underwear retailer Agent
Provocateur art directed by Mike Figgis and published by Anova, with design by
Gavin Ambrose. The absence of folios and minimal text mean the image flow
takes prominence.

The images can let us set a method for defining such spreads. 


\chapter{Temporarily changing the text width}

\index{pagewidth>change temporarily}


Margins in a page can be changed temporarily by adjusting, the lengths of \cmd{\leftskip} and \cmd{\rightskip}. The |memoir| class provides an environment |adjustwidth| see page 422 (based on This code is based on the \pkgname{chngpage} package.) for doing so and the \class{tufte-book} class provides an environment \textit{fullwidth}. The following code is an adaptation of that found in the \class{memoir} class.


\begin{teXX}
\begin{adjustmargins}{left}{right} 
\end{teXX}


adds the given lengths to the left and
right hand margins. A positive value will shorten the text and a negative value
will widen it. The starred version of the environment will cause the margin changes to switch between odd and even pages. 



\eject
\newgeometry{left=10mm,right=10mm,bottom=1.5cm,top=1cm}

\section*{The \texttt{adjustmargins} environment}
\lorem

\vfill\vfill
\begin{multicols}{2}
\lorem
\end{multicols}

\begin{adjustmargins}{0cm}{0in}
{\leftskip1em\rule{13cm}{.4pt}\par}

\centering



\parbox{\textwidth}{{\leftskip1em\rightskip1em There are no engineers in the hottest parts of hell, because the existence of a 'hottest part' implies a temperature difference, and any marginally competent engineer would immediately use this to run a heat engine and make some other part of hell comfortably cool.  This is obviously impossible.\par}
}
\par
\medskip
\par
\noindent\includegraphics[width=0.9\textwidth]{./graphics/lilian.jpg}\par


\end{adjustmargins}

\clearpage

\restoregeometry


\lipsum[1]


\begin{adjustmargins}{-0.4\textwidth}{0.1\textwidth}
\fboxsep2pt%
\fbox{\includegraphics[width=1.2\textwidth]{./graphics/leoncroll.jpg}}
\end{adjustmargins}

\lipsum[2]

\begin{teX}
\begingroup
\makeatletter
 \catcode`\Q=3
 \long\gdef\@ifmtarg#1{\@xifmtarg#1QQ\@secondoftwo\@firstoftwo\@nil}
 \long\gdef\@xifmtarg#1#2Q#3#4#5\@nil{#4}
 \long\gdef\@ifnotmtarg#1{\@xifmtarg#1QQ\@firstofone\@gobble\@nil}
 \endgroup


\newenvironment{adjustmargins}[2]{%
  \begin{list}{}{%
    \topsep\z@%
    \listparindent\parindent%
    \parsep\parskip%
   \@ifmtarg{#1}{\setlength{\leftmargin}{\z@}}%
   {\setlength{\leftmargin}{#1}}%
   \@ifmtarg{#2}{\setlength{\rightmargin}{\z@}}%
   {\setlength{\rightmargin}{#2}}%
}
\item[]}{\end{list}}
\makeatother
\end{teX}

 
\section{Setting Dimensions in \latex}

To set dimensions for page layout in \latex is not straightforward. You need to adjust several \latex
native dimensions to place a text area where you want. If you want to center the text area in the paper
you use, for example, you have to specify native dimensions as follows:

\begin{verbatim}
\usepackage{calc}
\setlength\textwidth{7in}
\setlength\textheight{10in}
\setlength\oddsidemargin{(\paperwidth-\textwidth)/2 - 1in}
\setlength\topmargin{(\paperheight-\textheight
-\headheight-\headsep-\footskip)/2 - 1in}.
\end{verbatim}

Without package |calc|, the above example would need more tedious settings. To adjust all parameters from scratch one should have a good understanding, of \latexe's definitions of all parameters. The companion package |xlayouts| can be used to display these parameters on an actual printed page. All settings are parameterized and I find the use of colours assists in viewing the rulers better.


\subsection{The Geometry package}

The package \pkg{geometry} \cite{geometry} provides
an easy way to set page layout parameters. In this case, what you have to do is just load the package and set
the page geometry using keys.

\begin{teX}
\usepackage[text={7in,10in},centering]{geometry}.
\end{teX}

Besides centering problem, setting margins from each edge of the paper is also troublesome. But geometry
also make it easy. If you want to set each margin to 1.5in, you can type

\begin{comment}
\label{sec:geometry}

\def\OpenB{{\ttfamily\char`\{}}
 \def\Comma{{\ttfamily\char`,}}
 \def\CloseB{{\ttfamily\char`\}}}
 \def\Gm{\textsf{geometry}}
\newcommand\gpart[1]{\textsf{\textsl{\color[rgb]{.0,.45,.7}#1}}}%

\newcommand\glen[1]{\textsf{#1}}

\bgroup
\makeatletter
 \begin{figure}
  \small
  \unitlength=.65pt
  \begin{picture}(450,250)(0,-10)
  \put(20,0){\framebox(170,230){}}
  \put(20,235){\makebox(170,230)[br]{\gpart{paper}}}
  \begingroup\thicklines
  \put(40,30){\framebox(120,170){}}
  \put(40,30){\makebox(120,165)[tr]{\gpart{total body}~}}
  \put(45,30){\makebox(0,170)[l]{|height|}}
  \put(40,35){\makebox(120,0)[bc]{|width|}}
  \put(50,-20){\makebox(120,0)[bc]{|paperwidth|}}
  \put(10,45){\makebox(0,170)[r]{|paperheight|}}
  \put(90,200){\makebox(0,30)[lc]{|top|}}
  \put(90,0){\makebox(0,30)[lc]{|bottom|}}
  \put(10,70){\makebox(0,0)[r]{|left|}}
  \put(10,55){\makebox(0,0)[r]{(|inner|)}}
  \put(200,70){\makebox(0,0)[l]{|right|}}
  \put(200,55){\makebox(0,0)[l]{(|outer|)}}
  \put(80,230){\vector(0,-1){30}}\put(80,30){\vector(0,-1){30}}
  \put(80,200){\vector(0,1){30}}\put(80,0){\vector(0,1){30}}
  \put(20,70){\vector(1,0){20}}\put(40,70){\vector(-1,0){20}}
  \put(160,70){\vector(1,0){30}}\put(190,70){\vector(-1,0){30}}
  \multiput(160,30)(5,0){24}{\line(1,0){2}}
  \multiput(160,200)(5,0){24}{\line(1,0){2}}
  \begingroup\thicklines
  \put(280,30){\framebox(120,170){}}\endgroup
  \put(283,133){\makebox(0,12)[l]{|textheight|}}
  \put(295,130){\vector(0,-1){100}}\put(295,150){\vector(0,1){50}}
  \multiput(280,220)(5,0){24}{\line(1,0){3}}
  \put(280,208){\makebox(120,20)[bc]{\gpart{head}}}
  \multiput(280,207)(5,0){24}{\line(1,0){3}}
  \put(420,225){\makebox(0,0)[l]{|headheight|}}
  \put(418,225){\line(-2,-1){20}}
  \put(420,213){\makebox(0,0)[l]{|headsep|}}
  \put(418,213){\line(-2,-1){20}}
  \put(420,12){\makebox(0,0)[l]{|footskip|}}
  \put(418,12){\line(-2,1){20}}
  \put(280,40){\makebox(120,140)[c]{\gpart{body}}}
  \put(305,45){\vector(-1,0){25}}\put(375,45){\vector(1,0){25}}
  \put(80,230){\vector(0,-1){30}}\put(80,30){\vector(0,-1){30}}
  \put(280,48){\makebox(120,0)[c]{|textwidth|}}
  \put(280,15){\makebox(120,10)[c]{\gpart{foot}}}
  \multiput(280,14)(5,0){24}{\line(1,0){2}}
  \put(410,30){\dashbox{3}(30,170){}}
  \put(415,30){\makebox(30,170)[l]{\gpart{marginal note}}}
  \put(425,45){\vector(-1,0){15}}\put(425,45){\vector(1,0){15}}
  \put(450,70){\makebox(0,0)[l]{|marginparsep|}}
  \put(448,70){\line(-3,-1){43}}
  \put(450,45){\makebox(0,0)[l]{|marginparwidth|}}
  \end{picture}
\caption{Dimension names used in the geometry package. width $=$ textwidth and height $=$ textheight by default. left, right, top and bottom are margins. If margins on verso pages are swapped by twoside option, margins specified by left and right options are used for the inside and outside margins respectively. inner and outer are aliases of left and right
respectively.}
\label{fig:geometrylayout}
\end{figure}
\makeatother
\egroup
\end{comment}

 The \pkg{geometry} package provides a flexible and easy interface to page dimensions.
 You can change the page layout with intuitive parameters. For instance,
 if you want to set a margin to 2cm from each edge of the paper,
 you can type just |\usepackage[margin=2cm]{geometry}|. 
 The page layout can be changed in the middle of the document
 with \cs{newgeometry} command.  The \ref{fig:geometrylayout}, reproduced from the package documentation, illustrates the variety of parameters that can be set using the package.


\section{Footnotes}
The history of footnotes is as long and complicated as the history of scholarship and commentary. Hebrew scholars more than two thousand years ago used systems of glossing and annotation to work on religious texts. 

Scribes in the Christian tradition in the medieval period made use of annotations in their manuscript copying practices: surrounding the original text with glosses in small letters. After the advent of printing, similar kinds of marginal annotation appeared in printed texts of the late fifteenth century. 

Humanist scholars producing printed editions of classical learning in the sixteenth century also made use of the resources of typography to display both the surviving classical text and their commentary on the same page. References to classical sources - and to modern printed editions - became more systematic, as did the expectation that such references would be consistent with scholarly practices. Scholars increasingly marked their professionalism by using complex citational conventions, which by the seventeenth century were so well established as to be the subject of parody and satire. Scriblerian satire of the early eighteenth century, whose purpose was to mock the pedantry and folly of the works of the learned, frequently included extensive parodies of footnotes and the scholarly contests they encoded. Nonetheless, during the eighteenth century, to appear authoritative and learned an author had to adopt the scholarly machinery of the reference citation.

The footnote was born out of a desire to rationalise the relation between text and citation. 

Robert Connors argues that marginal notations fell out of favour for two practical reasons: they left too much blank paper at the side of the text; and they were difficult for typographers to set. The same notes placed at the bottom of the page were more efficient, both in paper and time[1]. 

Anthony Grafton's The Footnote: A Curious History suggests the modern footnote, inaugurated by Pierre Bayle's Dictionaire Critique et Historique in 1697, signalled an epistemic revolution in historical scholarship, indicating the end of credulous scholasticism and the emergence of analytical historical methodologies. Both scholars note the considerable impact of historians such as David Hume and Edward Gibbon on the stylistic development of the discursive and citational footnote as a location for the display of gentlemanly ease as much as scholarly acumen. In the nineteenth century, German scholars such as Leopold von Ranke and Alexander von Humboldt established a systematic basis for the footnote citation, creating a methodical methodological approach that all competing scholars had to obey. In this way, the idea of the footnote was established, yet no there was no general agreement on the form these footnotes should adopt. A systematic approach to the form of the footnote was needed.

In this section we will discuss how lines and paragraphs are turned into pages and how elements of pages such as footnotes, headers etc are inserted. As with the other chapters we will mix TeX basic commands with the more convenient \LaTeXe\ commnads. We will also look at some of the packages and classes that are availble to assist us with page layouts. 


Besides illustrations that are inserted at the top of a page, plain TEX will also
insert footnotes at the bottom of a page. The ootnote macro is provided
for use within paragraphs;  for example, the footnote in the present sentence was typed
in the following way:


There are two parameters to a footnote[ first comes the reference mark, which will
appear both in the paragraph** and in the footnote itself, and then comes the text of
the footnote.45 The latter text may be several paragraphs long, and it may contain
\footnote{Sidenote: ``Where God meant footnotes to go.'' ---Tufte}

\marginpar{

The history of footnotes is as long and complicated as the history of scholarship and commentary. Hebrew scholars more than two thousand years ago used systems of glossing and annotation to work on religious texts. Scribes in the Christian tradition in the medieval period made use of annotations in their manuscript copying practices: surrounding the original text with glosses in small letters. After the advent of printing, similar kinds of marginal annotation appeared in printed texts of the late fifteenth century. Humanist scholars producing printed editions of classical learning in the sixteenth century also made use of the resources of typography to display both the surviving classical text and their commentary on the same page. References to classical sources - and to modern printed editions - became more systematic, as did the expectation that such references would be consistent with scholarly practices. Scholars increasingly marked their professionalism by using complex citational conventions, which by the seventeenth century were so well established as to be the subject of parody and satire. Scriblerian satire of the early eighteenth century, whose purpose was to mock the pedantry and folly of the works of the learned, frequently included extensive parodies of footnotes and the scholarly contests they encoded. Nonetheless, during the eighteenth century, to appear authoritative and learned an author had to adopt the scholarly machinery of the reference citation.

The footnote was born out of a desire to rationalise the relation between text and citation. Robert Connors argues that marginal notations fell out of favour for two practical reasons: they left too much blank paper at the side of the text; and they were difficult for typographers to set. The same notes placed at the bottom of the page were more efficient, both in paper and time[1]. Anthony Grafton's The Footnote: A Curious History suggests the modern footnote, inaugurated by Pierre Bayle's Dictionaire Critique et Historique in 1697, signalled an epistemic revolution in historical scholarship, indicating the end of credulous scholasticism and the emergence of analytical historical methodologies. Both scholars note the considerable impact of historians such as David Hume and Edward Gibbon on the stylistic development of the discursive and citational footnote as a location for the display of gentlemanly ease as much as scholarly acumen. In the nineteenth century, German scholars such as Leopold von Ranke and Alexander von Humboldt established a systematic basis for the footnote citation, creating a methodical methodological approach that all competing scholars had to obey. In this way, the idea of the footnote was established, yet no there was no general agreement on the form these footnotes should adopt. A systematic approach to the form of the footnote was needed.}

Further reading:

Connors, Robert J., 'The Rhetoric of Citation Systems, Part I: The Development of Annotation Structures from the Renaissance to 1900', Rhetoric Review, 17 (1998), 6-48.

Connors, Robert J., 'The Rhetoric of Citation Systems, Part II: Competing Epistemic Values in Citation', Rhetoric Review, 17 (1999), 219-245.

Grafton, Anthony, The Footnote: A Curious History (London: Faber and Faber, 1997)

Grafton, Anthony, 'The Footnote from De Thou to Ranke', History and Theory, 33 (1994), 53-76

Zerby, Chuck, The Devil's Details: A History of Footnotes (Lancaster: Gazelle, 2002)

[1] Robert J. Connors, The Rhetoric of Citation Systems, Part I: The Development of Annotation Structures from the Renaissance to 1900, Rhetoric Review, 17 (1998), 6-48 (p. 30).
  messes layout
%  ^^A 
\let\frogking\lorem

\chapter{Floats}


Most publications contain a lot of figures and tables. There are instances where
a table can be broken across pages, but this is unacceptable for figures. For this reason
figures and short tables need special treatment. The rather naıve method of treating these
objects is to start a new page every time a floating object is too large to fit on the present
page. A more sophisticated method to tackle this problem is to ‘‘float’’ any object that
does not fit on the current page to a later page while filling the current page with text.

This is why these objects are called floating objects. LATEX provides two environments
that are treated as floating objects: the figure and the table environments. Both environments
are written the same way; they differ only in the text that is prepended in the
caption. Moreover, there are two environments that can be used in double column documents
to generate floats that may occupy both columns: the \cs{figure*} and the \cs{table*}
environments. Here is how we can begin a table or a figure:

\begin{teXXX}
 \begin{table}[placement specifier ]
 \begin{figure}[placement specifier]
\end{teXXX}

An optional placement specifier is used to tell LATEX where the float is allowed to be
moved to. The placement specifier consists of a sequence of float placing permissions:

\begin{table}[htbp]
\begin{tabular}{ll}
\toprule
Placement   & position\\
\midrule
h                 & here\\
t                  & top\\
b                 & bottom\\
p                 & on a special page containing only floats\\
\bottomrule
\end{tabular}
\end{table}




Apart from the float placing permissions above there exists a fifth one, namely (!), which
forces LATEX to actually ignore most of the internal parameters related to float placement.
LATEX also provides the command \cs{suppressfloats},which prevents LATEX from putting


\section{The float package}

The \docpkg{float} package provides a friendly interface to define new float objects. Moreover, the package
defines certain ‘‘float styles’’ that can be used to define new floating objects.  It
was designed by Anselm Lingnau. New float objects can be defined with the command

\begin{verbatim}
\newfloat{type}{placement}{ext }[within ]
\end{verbatim}



Here type is the `��type'�� of the new class of floats (e.g., program, diagram, etc.),
placement gives the default placement specifier, and ext is the filename extension
for the file that will keep the captions in cases wherewewant to have a list of programs,
list of diagrams, or other lists. The optional argument within is used to number float
objects within some sectioning unit (e.g.,chapter, section). Here is a complete example:

\begin{teXXX}
\floatstyle{plain}
\newfloat{Photo}{htbp}{fot}[section]
\end{teXXX}



\makeatletter
%\newcommand\fs@framed{\def\@fs@cfont{\bfseries}\let\@fs@capt\floatc@ruled
%\def\@fs@pre{\hrule height.8pt depth0pt \kern2pt}%
% \def\@fs@post{\kern3pt\hrule\relax}%
% \def\@fs@mid{\kern2pt\hrule\kern2pt}%
% \let\@fs@iftopcapt\iftrue}
%\makeatother


\floatstyle{plain}
\newfloat{Photo}{htbp}{fot}%[section]

\begin{Photo}
 \centering
 \includegraphics[width=0.65\linewidth]{china-05}
\caption[a short caption]{If the caption is very long it is formatted as a paragraph, which is flushleft. If it is short it will be centered. }
\end{Photo}


\begin{Photo}
 \centering
 \includegraphics[width=0.65\linewidth]{china-06}
\caption{. . . caption . . . }
\end{Photo}

\begin{Photo}
 \centering
 \includegraphics[width=0.65\linewidth]{yaleartschool}
\caption{. . . caption . . . }
\end{Photo}



Note that after each such definition, a new 
environment will be available. Naturally,
its name depends on the ��type�� (e.g., the example code above will create the program
environment). The ��float style�� can be specified with the \cs{floatstyle} command. The
command takes only one argument, which is the name of a ‘‘float style’’:

\begin{teXXX}
\begin{Example}
     First verbatim line.
     Second verbatim line.
     Third verbatim line.
\end{Example}
\end{teXXX}



\floatstyle{ruled}
\newfloat{Example}{htbp}{loe}[chapter]

 \begin{Example}
 \begin{verbatim}
   \begin{Photo}
      \centering
      \includegraphics[width=0.65\linewidth]{./graphics/level3}
      \caption{. . . caption . . . }
   \end{Photo}
\end{verbatim}
\caption{Example using verbatim code}
 \end{Example}

\begin{Photo}
 \centering
 \includegraphics[width=0.85\linewidth]{old-timer-structural-worker}
\caption{. . . caption . . . }
\end{Photo}

\newlength{\egwidth}\setlength{\egwidth}{0.48\textwidth}

\newenvironment{ega}%
{\begin{list}{}{\setlength{\leftmargin}{0.02\textwidth}%
\setlength{\rightmargin}{\leftmargin}}\item[]\footnotesize}%
{\end{list}}

\newenvironment{egbox}%
{\begin{minipage}[t]{\egwidth}}%
{\end{minipage}}

\newcommand{\egstart}{\begin{ega}\begin{egbox}}
\newcommand{\egmid}{\end{egbox}\hfill\begin{egbox}}
\newcommand{\egend}{\end{egbox}\end{ega}}

% one or two other commands

\newcommand{\fn}[1]{\hbox{\tt #1}}
\newcommand{\llo}[1]{(see line #1)}
\newcommand{\lls}[1]{(see lines #1)}


\egstart
\begin{verbatim}
Here is some advice to remember:
\begin{quotation}
Environments for making
...other things as well.

Many problems
...environments.
\end{quotation}
\end{verbatim}
\egmid%
Here is some advice to remember:
\begin{quotation}
Environments for making quotations
can be used for other things as well.

Many problems can be solved by
novel applications of existing
environments.
\end{quotation}
\egend

The \cs{tabbing} environment overcomes this problem. Within it you set
tabstops and tab to them much like you do on a typewriter.  Tabstops are
set with the |\=| command, and the |\>| command moves to the
next stop.  The
|\\| command is used to separate each line.  A line that ends |\kill|
produces no output, and can be used to set tabstops:


\begin{teX}
\begin{tabbing}
 Income \=Expenditure \= \kill
 Income \>Expenditure \>Result\\
 20s 0d  \>19s 11d \>Happiness\\
 20s 0d  \>20s 1d  \>Misery \\
\end{tabbing}
\end{teX}

\smallskip

\begin{tabbing}
Income \=Expenditure \=    \kill
Income \>Expenditure \>Result \\
20s 0d \>19s 11d \>Happiness   \\
20s 0d \>20s 1d  \>Misery    \\
\end{tabbing}


Unlike a typewriter's tab key, the |\>| command always moves to the next
tabstop in sequence, even if this means moving to the left.  This can cause
text to be overwritten if the gap between two tabstops is too small.



\section{Environment}

\begin{teX}
\def\beginstory{
  \vskip 0.5in                 % Skip down 1/2 before story
   \begingroup                  % Start of formatting properties
   \leftskip 1in\rightskip 1in  % Wider margins for narrower text
   \itshape                     % Italic font
   \noindent{.\dotfill{}.\par}  % Make dotted line
	% Text after close of \beginstory will be story formatted
}

\def\endstory{
  \par\noindent{.\dotfill{}.\par}  % Make dotted line
  \endgroup                        % End of formatting properties
  \vskip 0.5in                    % Skip final 1/2 inch
}

\beginstory 

  Just a short story 

\endstory

\end{teX}



Here's a story about the formative era of personal computing. I
originally wrote it in 1999, but the point it makes is still valid.
Hope you like it.



\subsection*{The \protect\latex way}
LaTeX implements macros |\begin| and |\end|. These are a generic pair whose argument determines the environment that's being begun or ended.

LaTeX makes it much easier to code environments. Here's a generic environment definition:

|\newenvironment{environment_name}{stuff to do before text}{stuff to do after text}|

That's it -- the \cs{newenvironment} macro takes three arguments:
The name of the environment being created
The stuff to do before the text being assigned that environment
The stuff to do after the text being assigned that environment

The resemblance to \tex paired macros is obvious, but \latex  environments make it generic across all environments, and place the beginning and ending code in one place. Not only that, but because the environment has one name instead of two different names, it's very easy for a front end like LyX to assign environments to highlighted stretches of text.

The |\newenvironment|  macro works only when the environment name is undefined. If there's already an environment with that name, use|\renewenvironment|  instead. If you don't know, there are ways to test.

The following is a LaTeX version of the |\beginstory| \ldots | \endstory| example, with the two macros folded into the definition of one environment called story.


Lamport, cleverly defined macros that automatically, create the necessary \tex \cs{begingroup} and \cs{endgroup} commands. You can find the code in the |source2e| file and which is shown below:\footnote{You can find the full code in \texttt{File y, for ltmiscen.dtx}}

\begin{teX}
\def\begin#1{%
  \@ifundefined{#1}%
  {\def\reserved@a{\@latex@error{Environment #1 undefined}\@eha}}%
  {\def\reserved@a{\def\@currenvir{#1}%
  \edef\@currenvline{\on@line}%
  \csname #1\endcsname}}%
  \@ignorefalse
  \begingroup\@endpefalse\reserved@a}


\def\end#1{%
  \csname end#1\endcsname\@checkend{#1}%
  \expandafter\endgroup\if@endpe\@doendpe\fi
  \if@ignore\@ignorefalse\ignorespaces\fi}
\end{teX}

In \latex environments are defined as 
|\begin{foo}| and |\end{foo}| which are are used to delimit environment |foo|.
|\begin{foo}| starts a group and calls |\foo| if it is defined, otherwise it does
nothing.

|\end{foo}| checks to see that it matches the corresponding |\begin| and if so,
it calls |\endfoo| and does an |\endgroup|. Otherwise, |\end{foo}| does nothing.
If |\end{foo}| needs to ignore blanks after it, then |\endfoo| should globally set
the |@ignore| switch true with |\@ignoretrue| (this will automatically be global).

NOTE: |\@@end| is defined to be the |\end| command of TEX82.
|\enddocument| is the user's command for ending the manuscript file.
|\stop| is a panic button to end \tex in the middle.


\section*{Checking the environment}

This is interesting in that we can use \cs{@currenvir} to check if a command is within a particular environment. The following code will be used to typeout the environment.

\begin{teX}
\begin{enumerate}
   \item Check environment with |@currenvir|
   \makeatletter
   \item The current environment is \@currenvir
   \makeatother
\end{enumerate} 
\end{teX}

\begin{enumerate}
\item Check environment with |@currenvir|
\makeatletter
 \item The current environment is \@currenvir
\makeatother
\end{enumerate}

The \cs{@checkend} \index{Latex kernel!@checkend} uses the \cs{@currenvir}\index{Latex kernel!@currenvir} to see if there is a matching
begin environment and if it cannot find it produces an error.

\begin{teX}
\def\@checkend#1{%
   \def\reserved@a{#1}
   \ifx\reserved@a\@currenvir 
   \else
     \@badend{#1}
   \fi
}
\end{teX}

It is a pity that there is no real guide for explaining the \latex macros, other than just reading through them. Lamport and later his associates managed to produce code that offers the user a friendly API. Besides the scenes of this API, it also offers the package writers hundreds of useful commands.






%   ^^A%% DESIGNING HEADERS AND FOOTERS  **************************


\chapter{Running Titles and Paging}

Early printed books had no running title or paging figures. The first attempt to satisfy this need of the reader was to repeat the number of the chapter at the head of each page.\footnote{De Vinne, pg 142.}  As books and styles evolved, if the words of the running title or chapter began appearing together with the page number. Practical considerations regarding the wearing of plates, school-books and all works that were printed frequently had running titles in capitals of light-faced antique. 

\begin{figure}[htp]
\includegraphics[width=\textwidth]{./images/beauty-and-art-spread.jpg}
\end{figure}

Almost every type of design has been adopted by typographers and book designers; sometimes the text is centered and in other cases it is set flush up to the inner or outer margin of the facing pages. The book chapter and the section of the book is sometimes specified in the running title, the chapter name on the left and the section on the right. When the running title consists of the name of the book, it was sometimes divided so that one half only of this name would appear on one page and the other half on the facing page. De Vrinde was highly critical of such practices and remarked `Nor is this a commendable fashion, for a line of many words can seldom be evenly divided; if it is not so divided, one heading will be longer than the other.’  Some modern books that follow in a similar fashion would place the chapter label and number at the left and the chapter title on the right. 

I am unsure if repeating the name of the book in its running title has any benefits to the reader, especially if the name of the book is well known to the reader. This title is most useful when it explains or to some extend defines the matter on the page, and this explanation should refer not to the first but to the last paragraph on that page.  Many authors prefer to not have sections in chapters and in such cases running the book name in the header rather than having left and right headers that just repeat the chapter name is preferable. An example of this is Tufte’s \textit{Beautiful Evidence}.  Tufte’s books do not have any footer material.  Many specialist scientific books have multi-authors, sometimes the running head includes the authors name (See figure from ). This particular illustration also shows the use of rules. Traditionally the rules were applied to protect the top of the block from mechanical wear during printing. 

\begin{figure}[hb]
\includegraphics[width=\textwidth]{./images/headers/header-humidification-odd.jpg}
\includegraphics[width=\textwidth]{./images/headers/header-humidification-even.jpg}
\end{figure}

As a rule,  paging with arabic figures begins with the text of the book. The matter before the text (as the title, preface, introduction, etc., which are printed last of all) is paged with roman lower-case numerals. Appendices, indices and all additions to the text take arabi figures for paging, but publisher’s advertisements at the end of the book should receive their special paging in a figure of a different face. Maps, portraits, and illustrations made on separate pages never receive printed paging, although they may be reckoned as pages in the table of contents or the index. 
\begin{figure}[htb]
\includegraphics[width=\textwidth]{./images/headers/architect.jpg}
\caption{The headers here, have a background shading.}
\end{figure}

\begin{figure}[htb]
\includegraphics[width=\textwidth]{logic.jpg}
\caption{The headers shown here include small dotted rules, running to the outer page end. This type of header can be build by adding properties and inheriting the properties of other headers.}
\end{figure}

\begin{figure}[htb]
\hskip-.1\textwidth\includegraphics[width=1.2\textwidth]{./images/headers/tulip-01.jpg}

\vspace*{1cm}

\hskip-.1\textwidth\hbox to 0pt{\includegraphics[width=1.2\textwidth]{./images/headers/tulip-02.jpg}}
\caption{The headers here, have a background shading.}
\end{figure}

\begin{figure}[htp]
\includegraphics[width=1\textwidth]{./images/headers/small-flash-01.jpg}

\vspace*{1cm}
\includegraphics[width=1\textwidth]{./images/headers/small-flash-02.jpg}

\caption{The headers here, have a background shading.}
\end{figure}


\begin{figure}[htp]
\includegraphics[width=1\textwidth]{./images/headers/power-and-politics-01}

\vspace*{1cm}
\includegraphics[width=1\textwidth]{./images/headers/power-and-politics-02}

\caption{The headers here, have a background shading.}
\end{figure}

\begin{figure}[htp]
\includegraphics[width=1\textwidth]{./images/headers/economic-warfare-01}

\vspace*{1cm}
\includegraphics[width=1\textwidth]{./images/headers/economic-warfare-02}

\caption{The headers here, have a background shading.}
\end{figure}


\section{The Requirements}

The brief discussion above and the examples from various publications can help in definind the final requirements of what we are about to program. The header or the footer for that matter as they are very similar needs to communicate with the page that is currently being processed to obtain the page number and any other marks that need to go into the running head.

\begin{tabular}{>{\raggedright}p{5cm}l}
Access to the page number & \\
Build up string from sections, chapters, titles or subtitles &\\
Distingusih between left and right numbers &\\
Add user data  &\\
Provide an intuitive user interface&\\
\end{tabular}

A more modern approach would be to offer a small templating language to deal with the headers and footers. This is for example, now common in web applications where variables are sent by the server to the web page being build and transformed in templates.

Another approach is to use a graphical language, such as metapost.

Since we have to deal with odd and even pages and a header and or a footer, the minimum variables needed to hold this information is four. 

A graphicablock can also happily contain the necessary information.

The algorithm is described below:

\begin{enumerate}
\item Set the variable headerleft and headerright to indicate one page or two page printing.
\item Define text block templates as macros to set the typesetting to a named style. Each header style
         will have its own name. Standardize parameters to enable easy redefinition of commands. As a 
         final fully flexible approach the key header = custom will provide full capabilities for any user
         defined design.
\item Distinguish how headers and footers will be typeset on title pages, chapter openings, bibliographies, 
         automatically generated pages, such as float pages etc.
\item Hook into LaTeX’s output routine to obtain information about the top and bottom inserts and other marks.         
\item Inherit properties, such as language and directionality.
\item Provide less intrusive ways to define different styles by the user.
\item define block commands to mark start of different headings for example |\mainmatter|. This will define
         the start of the main text of the publication and issue a command to process the pages that follow.
\end{enumerate}

A special type of header is something that will be repeated on every page, say a watermark of some sorts. These are dealt as backgrounds.
 
\section{Traditional LaTeX page style commands}
  
One of the first tasks of any \LaTeXe\ class is to redefine the headers and footers. The format of the running headers or footers in \LaTeX\ terminology is called the \textit{page style}. Each different format is given names like \textit{empty} or \textit{plain} to make it easier to select and remember. 

\begin{figure}[hbt]
\includegraphics[width=\textwidth]{./images/headers/Running-heads-lace.png}
\caption{This last example shows what kind of atmosphere you can create with running heads. Here a bit of lace texture has been softened and graduated, creating a kind of gentle, suggestive frame around these pages. I’ve also used line drawings, logos, and other graphic elements to dress up running heads like these. From the \protect\href{bookk  }{bookdesigner.com}}
\end{figure}


The LaTeX kernel\footnote{In File J file{ltpage.dtx}, page 311.} defines two commands for selecting the running heads:

\begin{lstlisting}
\pagestyle{<style>} : sets the page style of the current and succeeding pages to style
\thispagestyle{<style>} : sets the page style of the current page only to style.
\end{lstlisting}

\section{Traditional LaTeX page style definition}

To define a page style \textit{style}, you must define the \lstinline{\ps@style} to set the page parameters.

\subsection{How a page style makes running heads and feet}
The \lstinline{\ps@}. . . command defines the macros \lstinline{\@oddhead}, \lstinline{\@oddfoot}, \lstinline{\@evenhead},
and \lstinline{\@evenfoot} to define the running heads and feet. (See output routine.) As some headings contain information such as the chapter name or section number these
headings are based on the sectioning commands, which define them. The page style defines the commands




\verb!\chaptermark,\sectionmark!, etc., where

\verb+\chaptermark{<text>}+ is called by \verb+\chapter+ to set a mark. The  ...mark commands and the ...head
macros are defined with the help of the following macros.
%(All the \ ...mark commands should be initialized to no-ops.)



\subsection{marking conventions}

LaTeX produces two kinds of marks a `left' and a `right' mark using the following commands.

markboth

markright



\section{The low level page style interface}

The basic mechanics of defining page styles is provided in the \LaTeXe\ kernel and it  involves defining or redefining four macros:

\begin{marglist}
\item [\cs{oddhead}] For two-sided printing, it generates the header for the odd-numbered
pages; otherwise, it generates the header for all pages.

\item [\cs{oddfoot}] For two-sided printing, it generates the footer for the odd-numbered pages; otherwise, it generates the footer for all pages.

\item [\cs{evenhead}] For two-sided printing, it generates the header of the even-numbered
pages; it is ignored in one-sided printing.

\item [\cs{evenfoot}] For two-sided printing, it generates the footer of the even-numbered
pages; it is ignored in one-sided printing.

\end{marglist}
A named page style, involves the redefinition of these commands stored in a macro \cs{ps@<style>}.
The \cs{pagestyle}\marg{plain} is defined as:



%\begin{tcolorbox}
%\begin{lstlisting}
%\newcommand\ps@plain{%
%  \renewcommand\@oddhead{}%
%  \let\@evenhead\@oddhead
%  \renewcommand\@evenfoot{%
%  {\hfil\normalfont\textrm{\thepage}\hfil}}%
%  \let\@oddfoot\@evenfoot
%}
%\end{lstlisting}
%\end{tcolorbox}

Since the \textit{plain} style treats both the odd and even pages the same way, the \cs{@evenfoot} and \cs{@evenhead} are let to the \cs{@oddhead} and \cs{@oddfoot} commands. The style only prints a page number at the center of the footer.


\subsection{A longer example}

\index{watermark}\index{water mark!sample page style}
\thispagestyle{samplepage}
Consider the case, where we need to print on a page the words \textsc{sample page}, as you might have noticed in some places of this document and at the margin of this page. Sometimes this type of mark is called a \textit{watermark.}

We will call this type of page style \textit{samplepage} and we will activate it on a particular page by typing \cs{thispagestyle}\marg{samplepage}.




%\begin{tcolorbox}
%\begin{lstlisting}
%%% Some special styles
%\IfFileExists{rotating.sty}{\RequirePackage{rotating}}{}
%
%\def\even@samplepage{%
% \begin{picture}(0,0)
%   \put(\Xeven,\Yeven){\turnbox{90}{\Huge \textcolor{\watermark@textcolor}{\watermark@text}}}
%\end{picture}
%}
%
%\def\odd@samplepage{%
% \begin{picture}(0,0)
%   \put(\Xodd,\Yodd){\turnbox{90}{\Huge \textcolor{\watermark@textcolor}{\watermark@text}}}
% \end{picture}
%}
%
%\def\watermarktext#1{\gdef\watermark@text{\fontfamily{phv}\selectfont#1}}
%\def\watermarktextcolor#1{\gdef\watermark@textcolor{#1}}
%\watermarktext{SAMPLE PAGE}
%\watermarktextcolor{purple}
%
%\def\ps@samplepage{\let\@mkboth\@gobbletwo
% \let\@oddhead\odd@samplepage\def\@oddfoot{\reset@font\hfil\thepage}
% \let\@evenhead\even@samplepage\def\@evenfoot{\reset@font\thepage\hfil}}
%
%\def\Xodd{500}
%\def\Xeven{-70}\def\Yeven{-810}
%\def\Yeven{-\expandafter\strip@pt\textheight}
%\let\Yodd\Yeven
%\end{lstlisting}
%\end{tcolorbox}

If you study the code in the example, you will notice that we are using \LaTeXe's \env{picture} environment to
place the text exactly where we need it. This is one way of absolutely positioning text on a page, another way is to use |pgf|’s absolute positioning methods.




\subsection{The key value interface}

The key value interface provides a number of mechanisms to tap into the page styles, enabling consistency in the user interface.

\medskip

\keyval{header style}{\marg{text}}{Triggers a page style for one page only.} The following values can be used.

\begin{marglist}
\item [empty] Standard class empty headers.
\item [plain] Standard class plain headers.
\item [headings] Standard class headings.
\item [fancy] If you use the fancyhdr package any fancy header style.
\item [sample page] Prints sample at the edge of the paper.
\item [preprint] Prints preprint at the edge of the paper.
\item [watermark] Prints a watermark at predefined places.
\end{marglist}

\keyval{watermark}{\marg{true|false}}{Prints a watermark on all pages, defaults to false.}
\keyval{watermark text}{\marg{text}}{The watermark text.}
\keyval{watermark text left}{\marg{text}}{The watermark text on left pages.}
\keyval{watermark text right}{\marg{text}}{The watermark text on right pages.}
\keyval{watermark angle}{\marg{number}}{The rotation angle of the water mark}




%\cxset{ watermark text/.store in=\watermark@text,
%           watermark text color/.store in=\watermark@textcolor,
%           watermark font-size/.store in=\watermarkfontsize@cx,
%           watermark odd x/.store in=\watermarkoddx@cx,
%           watermark even x/.store in=\watermarkevenx@cx,
%           watermark even y/.store in=\watermarkeveny@cx}
%
%\cxset{watermark text= PRE-PRINT,
%          watermark text color=theblue,
%          watermark font-size=\huge,
%          watermark odd x=470,
%          watermark even y=700,
%          watermark even x=60}
%
%\def\Xodd{\watermarkoddx@cx}
%\def\Xeven{-\watermarkevenx@cx}
%\def\Yeven{-\watermarkeveny@cx}
%%\def\Yeven{-\expandafter\strip@pt\textheight}
%\let\Yodd\Yeven
%
%\def\even@samplepage{%
% \begin{picture}(0,0)
%   \put(\Xeven,\Yeven){\turnbox{60}{\watermarkfontsize@cx \textcolor{\watermark@textcolor}{\watermark@text}}}
%\end{picture}
%}
%
%\def\odd@samplepage{%
% \begin{picture}(0,0)
%   \put(\Xodd,\Yodd){\turnbox{90}{\watermarkfontsize@cx\textcolor{\watermark@textcolor}{\watermark@text}}}
% \end{picture}
%}






\subsection{Using the headings as hooks}

Since the headings are added to the page during processing of the output routine, they are sometimes used
to insert material on the page at places other than the head, through the use of a zero width box. For example we
can use this approach to add a watermark on a page. Other approaches to position material at absolute positions
on a page, is to hook at \emph{shipout}. Some packages such as TikZ can also be used through the |remember picture, overlay |  key settings. 

The |phd| package has a predefined style, named samplepage that can be used to typeset some text at the outer margin of a page. The text is configurable and you can set it for example to typeset “PRE-PRINT” rather than the “SAMPLE PAGE” string. 

\begin{tcolorbox}
\begin{lstlisting}
\cxset{
     watermark text= PRE-PRINT,
     watermark text color=theblue,
     watermark font-size=\huge
}
\end{lstlisting}
\end{tcolorbox}

\makeatletter
\cxset{watermark text/.code =\watermarktext }
\makeatother

\watermarktext{PRE-PRINT}
   
\pagestyle{samplepage}


\section{Adding marks}

Most books will have headers that include marks such as the chapter name and number and or other combinations together with section numbers.

The standard book class include two styles one called \textit{headings} and another called \textit{myheadings} that style such headers.




\subsection{Key value interface}
\makeatletter
\cxset{
   chaptermark name color/.store in=\chaptermarknamecolor@cx,
   sectionmark name color/.store in=\sectionmarkcolor@cx,
   sectionmark title font/.store in=\sectionmarktitlefont@cx,
   section title color/.store in=\sectiontitlecolor@cx,
}

\makeatother

\cxset{chaptermark name color=thered,
          sectionmark name color=thered}





\begin{tcolorbox}
\begin{lstlisting}
%% STYLE 57 QUANTUM FRONTIER
\cxset{headings style57/.style={
          headings titlestyle,
% Chaptermarks
          chaptermark name={\bfseries EVOLUTION OF THE INSECTS},
% Leftmarks
          leftmark before=\thepage\quad, %even pages
          leftmark after=\hfill\hfill,
% Right marks influenced by chapter name?
          rightmark before=\hfill\hfill, %odd pages
          rightmark after=\thepage,
% Section marks
          sectionmark name custom=\chaptertitle@cx,
          sectionmark after title=\quad,
%  rules we remove or inherit
          header top rule=false,
          header bottom rule=false,
          header offset even=0pt,
          header offset odd=0pt,
          }}
\end{lstlisting}
\end{tcolorbox}


%\if@twoside
%  \def\ps@headings{%
%      \let\@oddfoot\@empty
%      \def\@oddfoot{\rule{\textwidth}{0.4pt}}
%      \let\@evenfoot\@empty
%      \def\@evenhead{\parbox{\textwidth}{%
%                                   \leavevmode
%                                   \if@headertoprule\rule{\textwidth}{0.4pt}%
%                                       \vskip2pt plus1pt minus1pt\fi
%%typesetter
%                                     \hskip\headeroffseteven@cx\hbox to \textwidth{%
%                                           \leftmarkbefore@cx
%                                           \leftmark
%                                           \leftmarkafter@cx
%                                     }%
%                                     \if@headerbottomrule\vskip-7pt plus1pt minus1pt
%                                    \rule{\textwidth}{0.4pt}\fi%
%          }% end parbox
%       }%
%%% Defines the odd head
%      \def\@oddhead{
%         \parbox{\textwidth}{%
%                                   \leavevmode
%                                   \if@headertoprule\rule{\textwidth}{0.4pt}%
%                                       \vskip2pt plus1pt minus1pt\fi
%%typesetter
%                                     \hskip\headeroffsetodd@cx\hbox to \textwidth{%
%                                           \rightmarkbefore@cx
%                                           \rightmark
%                                           \rightmarkafter@cx
%                                     }%
%                                     \if@headerbottomrule\vskip-7pt plus1pt minus1pt
%                                    \rule{\textwidth}{0.4pt}\fi%
%          }% end parbox
%      }%
%      \let\@mkboth\markboth
% % chaptermark called by chapter and also by table of contents etc. This is essentially a
%%  leftmark
%\def\chaptermark##1{%
%     \gdef\chaptertitle@cx{##1}%
%      \markboth {%
%       \ifnum \c@secnumdepth >\m@ne
%          \if@mainmatter%
%              \color{\chaptermarknamecolor@cx}%
%              \MakeUppercase{\chaptermarkname@cx\ }%
%              \chaptermarknumber%
%              \chaptermarkafternumber@cx%
%          \fi
%        \fi
%        \color{\chaptermarktitlecolor@cx}%
%       % \hfill%
%        \MakeUppercase{\chaptermarktitlebefore@cx{##1}}}{}%
%}%end chaptermark
%% section
%  \def\sectionmark##1{%
%      \markright {%
%        \ifnum \c@secnumdepth >\z@
%           {\bfseries\textcolor{\sectionmarkcolor@cx}{\sectionmarkname@cx\sectionmarknumber@cx\sectionmarkafternumber@cx}%
%        } %
%  \fi
%         \color{\sectionmarktitlecolor@cx}\MakeUppercase{\normalfont\sffamily \sectionmarkbeforetitle@cx{##1}\sectionmarkaftertitle@cx}}}}%
%\else
%  \def\ps@headings{%
%    \let\@oddfoot\@empty
%    \def\@oddhead{{\slshape\rightmark}\hfil\thepage}%
%    \let\@mkboth\markboth
%    \def\chaptermark##1{%
%      \markright {%
%        \ifnum \c@secnumdepth >\m@ne
%          \if@mainmatter
%            \@chapapp\ \thechapter... \ %
%          \fi
%        \fi
%        ##1}}}
%\fi
%\def\ps@myheadings{%
%    \let\@oddfoot\@empty\let\@evenfoot\@empty
%    \def\@evenhead{\thepage\hfil\slshape\leftmark}%
%    \def\@oddhead{{\slshape\rightmark}\hfil\thepage}%
%    \let\@mkboth\@gobbletwo
%    \let\chaptermark\@gobble
%    \let\sectionmark\@gobble
% }

Note that the \cs{markboth} command takes two arguments the left mark and the right mark. It works reasonably well.



%\cxset{headings boxedpagenumber}
%\cxset{headings style58}
%\pagestyle{headings}


%   ^^A
\cxset{lineskip/.code=\setlength\lineskip{#1},
       lineskip/.default=1pt,
          normallineskip/.code=\setlength\normallineskip{#1},
          parindent/.code=\setlength\parindent{#1},
          parskip/.code=\setlength\parskip{#1},
          text-indent/.code=\setlength\parindent{#1},
          baselinestretch/.code=\renewcommand\baselinestretch{#1},
          single spacing/.code=\singlespacing,
          single spacing/.default=\singlespacing,
          double spacing/.code=\doublespacing}

\cxset{lineskip=1pt,
          normallineskip=1pt,
          parindent=1em,
          parskip=1pt,
          text-indent=1em,
          baselinestretch={},
          single spacing}

\makeatletter\@specialtrue\makeatother
\cxset{steward,
  numbering=arabic,
  custom=stewart,
  offsety=0cm,
  image={./images/hine05.jpg},
  texti={When Lamport designed the original \LaTeX\ sectioning commands, limitations of computer power forced him to restrict the abstraction of complicated chapter layouts. With current tools available improvements are much easier to program.},
  textii={In this chapter we discuss a method that allows the production of fancy chapter headings and formatting, based on a set of key values. Central  to this process is the separation of content from presentation.
We also discuss the basic formatting tools that are available and how one can modify them to mould new book designs.
 }
}
\cxset{chapter opening=left}

\chapter{General Settings}

\section{Introduction}

Here we define and set general paragraph settings. The parameters which control \TeX's behaviour when typesetting paragraphs can receive a bit of a tweak here. We also describe a set of options to handle parameters that can influence grid typesetting. This is especially important for two or more column typesetting. The commands act only on the text within a grouped environment. They do not affect captions or footnotes. Use anything over \emph{single spacing} with care, as books are meant to be single spaced.  



\section{Controlling inter-line spacing}
\index{line spacing}
Interline spacing traditionally has been controlled using the \pkgname{setspace} or by setting appropriate primitive \tex commands \cite{setspace}. The \pkgname{phd} loads the |setspace| package and then provides parameterized commands for setting styles. 

\begin{key}{/chapter/single spacing} 
	The Lineskip parameter emulates \TeX's \cmd{\parindent} command.
\end{key}
\begin{key}{/chapter/one half spacing} 
	The Lineskip parameter emulates \TeX's \cmd{\parindent} command.
\end{key}
\begin{key}{/chapter/double spacing} 
	Sets the document line-spacing to double.
\end{key}

If you want to use larger inter-line spacing in a document, you can change its value by putting the

\CMDI{\linespread}\meta{factor} Use |\linespread{1.3}| for "one and a half" line spacing, and |\linespread{1.6}| for "double" line spacing. Normally the lines are not spread, so the default line spread factor is~1.

The setspace package allows more fine-grained control over line spacing. To set "one and a half" line spacing document-wide, but not where it is usually unnecessary (e.g. footnotes, captions):

\begin{teXXX}
\usepackage{setspace}
%\singlespacing
\onehalfspacing
%\doublespacing
%\setstretch{1.1}
\end{teXXX}

The |phd| package provides the settings

\begin{key}{/chapter/single spacing}
We use the \pkgname{setspace} to effect the desired line spread effect.
\end{key}


These command offer little value over the normal \TeX\ macros other than keeping the interface, uniform. One can also extend the interface to cover CSS style commands:

\begin{verbatim}
\cxset{text-indent=50pt}

\cxset{double spacing}
\lipsum*[1]

\cxset{single spacing}
\lipsum*[1]
\end{verbatim}



\subsection{Parameters controlling paragraphs}\index{Paragraphs!controlling parameters}
The parameters \cs{lineskip} and \cs{normallineskip} influence \TeX\ when two lines come two close.
\medskip



\begin{key}{/chapter/lineskip=1pt} 
	The Lineskip parameter emulates \TeX's \cmd{\lineskip} command.
\end{key}

\begin{key}{/chapter/normallineskip=\marg{dim}} 
	The normallineskip parameter emulates \TeX's \cmd{\normallineskip} command.
\end{key}

\begin{key}{/chapter/lineskiplimit=\marg{dim}} 
	The Lineskip parameter emulates \TeX's \cmd{\lineskiplimit} command.
\end{key}

\begin{key}{/chapter/parindent=\marg{dim}} 
	The Lineskip parameter emulates \TeX's \cmd{\parindent} command.
\end{key}

\keyval{parindent}{\marg{dim}}{Paragraph indentation.}
\keyval{text-indent}{\marg{dim}}{Alias for \cs{parindent}.}
\keyval{parskip}{\marg{dim}}{Spacing between paragraphs.}


Another advantage, the package offers a few pre-configured styles, just setting a style to latex will revert everything back to latex.

\section{Technical discussion}

Most classes, including the standard \LaTeXe\ classes as well as packages attempting to achieve a grid typesetting try define a text height that is a multiple of \cs{baselineskip}. This way they give little opportunity to TeX to adjust the vertical glue to achieve a flush bottom.

\section{Dropcaps and Lettrines}\index{Lettrine!basic typesetting}

Dropcaps or lettrines are those letters that start paragraphs with a fancy larger letter. The class uses a parameterized version of the lettrine package of Daniel Flipo. Lettrine letters are easily typed and produced, but they are notoriously difficult to get right and no-one seems to agree on settings. These settings depend on the font the sizing of the text and the personal taste of the book interior designer. As I don't profess to be one, I have done what I think Knuth have done (just studied existing sources) allowed programming hooks and provided defaults as close as possible to the originals.




% 

%  \OnlyDescription
%
%  \StopEventually{}
%<*package>
% \CodelineNumbered
% \pagestyle{headings}
% \cxset{style87}
%
% \part{IMPLEMENTATION}
% 
% \chapter{Implementation Strategy}
%
% The implementation is divided into parts. Perhaps cutting,
% these parts into smaller packages might have been a better
% choice, but as the aim of the package is to minimize
% the loading of packages and let |phd| to handle
% this, it made more sense to me, anyway to keep everything
% together.
% 	
%
% \begin{description}
%
%  \item[The Package Manager] This section is responsible 
%       for pre-loading  packages, resolving conflicts and 
%       providing a interfacing commands.
%
%  \item[The Sectioning Layouts Manager] This section manages 
%       the design of complex layouts for sectioning commands.
%
%  \item [The Image Page Manager] This section manages the design of 
%       pages that consist primarily of images and complex
%		page layouts.
%
%  \item[Common Macros] We provide a number of predefined commands
%		for macros that us and other people found useful.
%
%  \item[MWE] The package generates a large number
%		of Minimum Working Examples that we use for testing. 
%		Most of them can also used as examples for training 
%		or self-study.
%
% \end{description}
%
% \section{The Package Manager}
%
% The basic requirement for the Package Manager is to load
% an adequate number of packages to enable the typesetting
% of a diverse number of large documents without requiring
% additional packages to be loaded by typical groups of
% authors. This has its advantages, but of course it does 
% slow things down. A long term objective is to select
% packages depending as an option on the type of document
% being prepared.
%
% \section{Preliminaries}
%
%    Standard file identification. We first announce the package 
%	 and require that it be used with \LaTeX2e. 
%
%    \begin{macrocode}
\NeedsTeXFormat{LaTeX2e}[1994/12/01]%
\ProvidesPackage{phd}[2015/1/13 v1.0 less preamble (YL)]%
\let\ltxtoday\today
%    \end{macrocode}
%
% Load the package \pkgname{fixltx2e} to update \LaTeX2e for various fixes. The package fixes a number things in the LaTeX2e kernel. Due to LaTeX's stability policy, these corrections have not been incorporated into the LaTeX2e kernel, but this package does things most people would agree are bugfixes. So to load this package is always recommended for newly created documents. The corrections have no commonalities, but the package's description has a nice summary:
%
%ensure one-column floats don't get ahead of two-column floats;
%correct page headers in twocolumn documents;
%stop spaces disappearing in moving arguments;
%allowing |\fnsymbol| to use text symbols;
%allow the first word after a float to hyphenate;
% cs{emph} can produce caps/small caps text;
%bugs in \cs{setlength} and \cs{flushbottom.}
% 
%    \begin{macrocode}
\RequirePackage{fixltx2e}[2006/03/24]
\RequirePackage{adjustbox}
\RequirePackage{fancybox}
% mock chapters where necessary
\@ifundefined{c@chapter}{%  
      \newcounter{chapter}
      \def\thechapter{\@arabic\c@chapter}
}{}
%    \end{macrocode}
% We load the \pkg{pgf} package early so we can use it for key management.
% We create a family for keys, unimaginatively named phd. 
% This might  change in the future.
% 
% \begin{macro}{\pkgfamilyname}
% \begin{macro}{\cxset}
% \begin{macro}{\cxsetvalue}
% The macro \cmd{\cxset} is the workhorse of the package. It is used to define or to set options
% for styling documents and also offers other utilities.
%
%    \begin{macrocode}
\RequirePackage{pgf}      
\usepgfmodule{parser}%for svg     
\usepgflibrary{svg.path}%for futurelet and parser demo       
\def\pkgfamilyname{phd}
\pgfkeys{/\pkgfamilyname/.is family}   
\newcommand\cxset{\pgfqkeys{/\pkgfamilyname}} 
\def\cxkeydef#1#2{%
 \pgfkeyssetvalue{/\pkgfamilyname/#1}{#2}%
}
\def\cxvalueof#1{%
 \expandafter\pgfkeysvalueof{#1}%
}
%\RequirePackage{silence} gives errors with varwidth
 \hfuzz=999pt % reduce overfull hbox errors
 \hbadness=10000 % reduce underfull hbox errors
% 
%    \end{macrocode}
% \end{macro}
% \end{macro}
% \end{macro}
%
%    \begin{macrocode}
\def\cx@optionlist{}
\def\cxuselibrary#1{\cxset{library/.cd,#1}}
%
% The library is added by inputting the file and setting the path accordingly.
\def\cx@add@library#1#2{%
  \cxset{library/#1/.code={\@ifundefined{cxlibrary@#1@loaded}{\input #2}{}}}%
  \DeclareOption{#1}{\edef\cx@optionlist{\cx@optionlist,#1}}%
}
%    \end{macrocode}
%
% Here is our attempt to play nice with the three
% main TeX engines.
%
%    \begin{macrocode}
\RequirePackage{phdsort}%% to check
\RequirePackage{ifluatex}
\RequirePackage{ifxetex}
\def\ifengine#1#2#3{
  \ifxetex
    #1%
  \else
    \ifluatex
      #2%
    \else
      #3%
    \fi
\fi
}
%    \end{macrocode}
%
% We use \pkgname{luacode} and luatextra only if we are using LuaTeX. Many of the
% packages we load ourselves later in any case. We need to check this.
%    \begin{macrocode}
\ifluatex
  \RequirePackage{luacode}
  %\RequirePackage{luatextra}
\fi
%    \end{macrocode}
%
% \section{Front matter and backmatter}
%
% These are both provided by the classes but we intent to parameterize them
% so we redefine them.
%
% \begin{macro}{\frontmatter}
% \begin{macro}{\mainmatter}
%    \begin{macrocode}

%\cxset{mainmatter numbering/.is choice,
%          mainmatter numbering/arabic/.code=\def\setpagenumbering{\pagenumbering{arabic}},
%	    mainmatter numbering/roman/.code=\def\setpagenumbering{\pagenumbering{roman}},
%	    mainmatter numbering/Roman/.code=\def\setpagenumbering{\pagenumbering{Roman}},
%	}
%%
%\cxset{mainmatter numbering=arabic}     
%     
%\newif\if@mainmatter \@mainmattertrue
%\def\frontmatter{
%          %\cleardoublepage
%            \@mainmatterfalse
%            %\setpagenumbering%
%}
%
%
%\def\mainmatter{%
%       %\cleardoublepage
%       \@mainmattertrue
%       \setpagenumbering}
%       
\def\backmatter{\if@openright\cleardoublepage\else\clearpage\fi
      \@mainmatterfalse}
%    \end{macrocode}      
% \end{macro}
% \end{macro}
%
% \section{Font Manager}
%
% \subsection{Sizing}
%
%    \begin{macrocode}
\setlength\lineskip{1\p@}
\setlength\normallineskip{1\p@}
\renewcommand\baselinestretch{}
%    \end{macrocode}
%
%    \begin{macrocode}
\newskip\AJW@baseskip
\newskip\AJW@theskip
\def\AJW@setskips#1{\AJW@theskip #1\relax%
  \abovedisplayskip      0.50\AJW@theskip \@plus 0.25\AJW@theskip \@minus 1\p@%
  \belowdisplayskip      \abovedisplayskip
  \abovedisplayshortskip 0.25\AJW@theskip \@plus 0.25\AJW@theskip
  \belowdisplayshortskip 0.50\AJW@theskip \@plus 0.25\AJW@theskip \@minus 1\p@%
}
\def\AJW@setlists#1{\AJW@theskip #1\relax%
 \def\@listi{\leftmargin\leftmargini
  \topsep  0.5\AJW@theskip \@plus 2\p@ \@minus 1\p@%
  \parsep  \z@
  \itemsep \z@}}
%
%
% common Sizes
\newcommand{\@viiiptv}{8.5}%                  8.5pt
\DeclareMathSizes{\@viiiptv}{\@viiiptv}{\@vipt}{\@vpt}
\newcommand{\@ixptv}{9.5}%                    9.5pt
\DeclareMathSizes{\@ixptv}{\@ixptv}{\@viipt}{\@vpt}
\newcommand{\@xptv}{10.5}%                    10.5pt (normalsize)
\DeclareMathSizes{\@xptv}{\@xptv}{\@viipt}{\@vpt}
\newcommand{\@xvipt}{16}%                     16pt size
\DeclareMathSizes{\@xvipt}{\@xvipt}{\@xiipt}{\@xpt}
\newcommand{\@xviiipt}{18}%                   18pt size
\DeclareMathSizes{\@xviiipt}{\@xviiipt}{\@xiipt}{\@xpt}

\renewcommand\tiny{\@setfontsize\tiny\@vpt{6}\AJW@setskips\AJW@setlists}
\renewcommand\footnotesize{\AJW@baseskip 10.5pt%
  \@setfontsize\footnotesize\@viiipt{10.5}\AJW@setskips\AJW@setlists}
\renewcommand\large{\@setfontsize\large\@xipt{14}\AJW@setskips\AJW@setlists}
\newcommand\Aheadsize{\@setfontsize\Aheadsize\@xipt{13}\AJW@setskips\AJW@setlists}
\renewcommand\Large{\@setfontsize\Large\@xiipt{17}\AJW@setskips\AJW@setlists}
\renewcommand\LARGE{\@setfontsize\LARGE\@xvipt{19}\AJW@setskips\AJW@setlists}
\renewcommand\huge{\@setfontsize\huge\@xviiipt{24}\AJW@setskips\AJW@setlists}
% Fix me
\def\huge{\@setfontsize\Huge{24}{26}}
\def\HUGE{\@setfontsize\Huge{38}{47}}
\def\HHUGE{\@setfontsize\HHUGE{58}{67}}
\def\HHHUGE{\@setfontsize\HHHUGE{94}{105}}
 \newcommand\smallverbatimsize{\AJW@baseskip 11.5pt%
    \@setfontsize\smallverbatimsize{10.5}{11.5}\AJW@setskips\AJW@setlists}
  \renewcommand\scriptsize{%
    \@setfontsize\scriptsize\@viipt{8}\AJW@setskips\AJW@setlists}
  \newcommand\figcaptionsize{\AJW@baseskip 10.5pt%
    \@setfontsize\figcaptionsize\@viiipt{10.5}\AJW@setskips\AJW@setlists}
  \let\smallertablesize\figcaptionsize
  \renewcommand\small{\AJW@baseskip 10pt%
    \@setfontsize\small\@ixpt{10}\AJW@setskips\AJW@setlists}
  \let\indexsize\small
  \newcommand\enotesize{\AJW@baseskip 11.5pt%
    \@setfontsize\enotesize\@ixpt{11.5}\AJW@setskips\AJW@setlists}
  \newcommand\smallish{\AJW@baseskip 11pt%
    \@setfontsize\smallish\@ixpt{11}\AJW@setskips\AJW@setlists}
  \let\bibliosize\smallish
  \newcommand\tablesize{\AJW@baseskip 11pt%
    \@setfontsize\tablesize\@ixptv{11}\AJW@setskips\AJW@setlists}
  \newcommand\exercisesize{\AJW@baseskip 12pt%
    \@setfontsize\exercisesize\@ixptv{12}\AJW@setskips\AJW@setlists}
  \newcommand\normalsmall{\AJW@baseskip 12pt%
    \@setfontsize\normalsmall\@xpt{12}\AJW@setskips\AJW@setlists}
  \newcommand\verbatimsize{\AJW@baseskip 13pt%
    \@setfontsize\verbatimsize\@xpt{13}\AJW@setskips\AJW@setlists}
  \newcommand\xheadsize{\AJW@baseskip 12pt%
    \@setfontsize\xheadsize\@xptv{12}\AJW@setskips\AJW@setlists}
  \newcommand\largerstill{\AJW@baseskip 14pt%
    \@setfontsize\largerstill\@xiipt{14}\AJW@setskips\AJW@setlists}
    
  \renewcommand\normalsize{\AJW@baseskip 13.5pt%
    \@setfontsize\normalsize\@xptv{13.5}\AJW@setskips{\AJW@baseskip}
    \let\@listi\@listI}
  \let\listsize\normalsize
%
%    \end{macrocode}
%
% \subsection{Loading fontspec} We load the \pkgname{fontspec} package \citep{fontspec} both
% for XeTeX or LuaTeX.
% The font manager needs to first distinguish  between the various
% engines used, and secondly needs to use the right packages. This is a bit of
% a kludge at the moment.
%
%  The handler |.set font-face| creates a command for activating a font face
%  and also activates it? DOES NOT WORK
%
%    \begin{macrocode}
 \pgfkeys{/handlers/.set font-face/.code=\pgfkeysalso{\pgfkeyscurrentpath/.code=
           \def\tempa{##1}%
           \def\tempb{normal}%
           \def#1{##1}%
           \ifx\tempa\tempb%
              \def#1{\normalfont}%
           \fi%
   }}%   
%
\cxset{document font/.code 2 args=\setmainfont[#2]{#1}}
%
\def\defaultfontsxetexluatex{% 
  \RequirePackage{expl3}
  \RequirePackage{xcoffins}
  \RequirePackage{xtemplate}
  \RequirePackage{l3sort}
%  
  \RequirePackage[no-math]{fontspec}
   \setmainfont[
      BoldFont={timesbd.ttf},
      ItalicFont={timesi.ttf},
      BoldItalicFont={timesbi.ttf},
      SmallCapsFont={times.ttf},
                      ]{times.ttf} % on windows
 \setsansfont{Tex Gyre Heros}% work okay with palatino
 \setmonofont[Scale=.95]{consola.ttf}
% 
 \newfontfamily\verbatimfamily[Scale=0.95]{consola.ttf}
 \DeclareTextFontCommand{\texttt}{\verbatimfamily} %CHANGED
 \newfontfamily\arial{Arial}
 \let\pan\arial
 \let\unicodenumberfam\arial
 \newfontfamily\smallcps[Scale=0.8]{Arial}
 \def\phd@font@times{Times New Roman}
  \cxset{document font={\phd@font@times}{Scale=1}}
  \newfontfamily\arabicfont[Script=Arabic]{Amiri}
 \newfontfamily\arabicfonttt[Script=Arabic,Scale=.75]{DejaVu   Sans Mono}
}  
%    \end{macrocode}
%
%  The management of default fonts for scripts and languages is attempted here first. 
%  This is a difficult topic, as the user might not have the fonts installed in their system.
%  If polyglossia is used it checks that language<languagename>  is defined, hence we
%  need to define |\arabicfont|  for the arabic language etc. 
%  
%    \begin{macrocode}  
%
\ifengine{\defaultfontsxetexluatex}{\defaultfontsxetexluatex}{}
%    \end{macrocode}
%
%  We need to cater for LTR without changing basic commands of XeTeX or eTeX.
%
%    \begin{macrocode}
\ifluatex
   \RequirePackage{luaotfload}
%   \let\TeXXeTstate\@gobble
   \def\setRTL{\luatextextdir TRT}%\pardir TRT\textdir TRT}
   \def\endLTR{\luatextextdir TLT}%{\pardir TLT\textdir TLT}
   \let\beginR\setRTL
   \let\endR\endLTR
\fi
%    \end{macrocode}
%
%  \subsection{Creating a Small Verbatim Environment}
%  This is a modified version from Cambridge classes
%    \begin{macrocode}
\begingroup \catcode `|=0 \catcode `[= 1
\catcode`]=2 \catcode `\{=12 \catcode `\}=12
\catcode`\\=12 |gdef|@xsmallverbatim#1\end{smallverbatim}[#1|end[smallverbatim]]
|gdef|@sxsmallverbatim#1\end{smallverbatim*}[#1|end[smallverbatim*]]
|endgroup
\def\@smallverbatim{\trivlist \item\relax
  \if@minipage\else\vskip\parskip\fi
  \leftskip\@totalleftmargin\rightskip\z@skip
  \parindent\z@\parfillskip\@flushglue\parskip\z@skip
  \@@par
  \@tempswafalse
  \def\par{%
    \if@tempswa
      \leavevmode \null \@@par\penalty\interlinepenalty
    \else
      \@tempswatrue
      \ifhmode\@@par\penalty\interlinepenalty\fi
    \fi}%
  \let\do\@makeother \dospecials
  \obeylines \smallverbatim@font \@noligs
  \hyphenchar\font\m@ne
  \everypar \expandafter{\the\everypar \unpenalty}%
}
\def\smallverbatim{\@smallverbatim \frenchspacing\@vobeyspaces \@xsmallverbatim}
\def\endsmallverbatim{\if@newlist \leavevmode\fi\endtrivlist}
\def\smallverbatim@font{\normalfont\smallverbatimsize\ttfamily}
%    \end{macrocode}
%
% The package fonttable provides a number of interesting testing tests
% for fonts
%
%    \begin{macrocode}					
\RequirePackage{fonttable}	
%    \end{macrocode}
%
% \section{Language Manager} 
%
% We use the package \pkgname{polyglossia} for language management for the
% newer engines and \pkgname{babel} for pdfLaTeX.
% This is full of holes which need to be closed for cases where the
% bidi package is loaded.
%
%    \begin{macrocode} 
\ifengine{\RequirePackage{polyglossia}
  \setdefaultlanguage{english}}{%
  \RequirePackage{polyglossia}
  \setdefaultlanguage{english}}{\RequirePackage[dutch, german, main=english]{babel}}
  \RequirePackage[dutch,german,english]{xlayouts}
%    \end{macrocode}
%
% \chapter{Package Management} 
%  
% In order to keep track of all the packages and keys we require a
% number of macros will be defined first.
% 
% Each of the packages used by this document is loaded conditionally.
% However, it might be nice to know if we have a complete set.  So we
% define |\ifcomplete| which starts true, but gets set to false if any
% package is missing. Some code is necessary in order to manage 
% the complexity.
% I am indebted to the source of |symbols.tex| for some of the macros.
% There are a number of symbols (e.g., \cmd{\Square}) that are defined by      
% multiple packages.  In order to typeset all the variants in this       
% document, we have to give glyph a unique name.  
% To do that, we define :
%
% 
% \cs{savesymbol{XXX}}, which renames a symbol from \cs{XXX} to \cmd{\origXXX}, and    
% \cmd{\restoresymbols{yyy}{XXX}}, which renames \cmd{\origXXX} back to \cmd{XXX} and     
% defines a new command, |\yyyXXX|, which corresponds to the most recently 
% loaded version of |\XXX|.                                                
%                                                                        
% This implementation of \savesymbol and \restoresymbol was copied from  
% the |savesym| package, which started with symbol.tex's old definitions   
% of those macros and improved upon them.  However, \renamerobustsymbol  
% and |\ifnotsavedsym| are from  the list of |symbols| documentation.                                
%                                                                        
% 
%
% \begin{macro}{\savesymbol}
%    \begin{macrocode}
\NewDocumentCommand{\savesymbol}{ m }{%
  \expandafter\let\csname orig#1\expandafter\endcsname\csname#1\endcsname
  \expandafter\let\csname #1\endcsname\relax
}

%    \end{macrocode}
% \end{macro}
%    
%
% \begin{macro}{\restoresymbols}
% 	Restore a previously saved symbol, and rename the current one.
%    \begin{macrocode}
\newcommand*{\restoresymbol}[2]{%
  \expandafter\global\expandafter\let\csname#1#2\expandafter\endcsname%
    \csname#2\endcsname
  \expandafter\global\expandafter\let\csname#2\expandafter\endcsname%
    \csname orig#2\endcsname
}
%    \end{macrocode}  
% \end{macro}
 
% Rename a robust command.
%    \begin{macrocode}
\newcommand*{\renamerobustsymbol}[2]{%
  \expandafter\let\expandafter\origrealcommand
    \csname #2\space\endcsname
    \expandafter\global\expandafter\let\csname#1#2\endcsname=\origrealcommand
}
%    \end{macrocode}
% Test if a symbol is not saved.
%    \begin{macrocode}
\def\ifnotsavedsym@helper#1#2!{\expandafter\ifx\csname orig#2\endcsname\relax}
\newcommand*{\ifnotsavedsym}[1]{%
  \expandafter\ifnotsavedsym@helper\string#1!%
}
%    \end{macrocode}
% \begin{macro}{\ifcomplete}
%    \begin{macrocode}
\let\oldcontentsline\contentsline
\newif\ifcomplete
%    \end{macrocode}
% \end{macro}    
%    
% For debugging purposes we define a switch that enables us to toggle
% on and off the loading of packages.
% 
%    \begin{macrocode}
\newif\ifloadpackages
\loadpackagestrue
%    \end{macrocode}
%    
% |\IfStyFileExists*| is just like |\IfFileExists|, except that it appends
% ".sty" to its first argument.  |\IfStyFileExists| is the same as
% |\IfStyFileExists*|, but it additionally adds its first argument to a list
% (|\missingpkgs|) and marks the document as incomplete (with
% |\completefalse|) if the |.sty| file doesn't exist.
% 
% \begin{macro}{\missingpkgs}
% \begin{macro}{\foundpkgs}
%   \begin{macrocode}
\newcommand{\missingpkgs}{}
\newcommand{\foundpkgs}{}
\newcommand{\if@sty@file@exists@star}[3]{%
  \ifloadpackages
    \IfFileExists{#1.sty}{#2}{#3}%
  \else
    #3%
  \fi
}
\newcommand{\if@sty@file@exists}[3]{%
  \ifloadpackages
    \IfFileExists{#1.sty}%
                 {#2\@cons\foundpkgs{{#1}}}%
                 {#3\completefalse\@cons\missingpkgs{{#1}}}%
  \else
    #3\completefalse\@cons\missingpkgs{{#1}}%
  \fi
}
\newcommand{\IfStyFileExists}{%
  \@ifstar{\if@sty@file@exists@star}{\if@sty@file@exists}%
}
%    \end{macrocode}
% \end{macro}
% \end{macro}
%
% \subsection{Utility macros for displaying symbols and fonts}
%
% In the sections that follow, we use a number of utilities for
% displaying fonts and utilities in tables and figures, we collect
% them here and make them available to the user for document
% use. Many are modifications from other packages.
%
% \begin{macro}{\symbols}
% \begin{macro}{\endsymbols}
%    \begin{macrocode}
% From stmarysrd symbols package
% A very convenient command to typeset symbols.
% Much preferable than tables. Slight modifications to
% make it a bit more clear
% CHECK END SYMBOLS
\newcommand\symbols{\flushleft}
\def\endsymbols{\endflushleft}

\def\dosymbol#1{%
   \leavevmode\hbox to .33\textwidth{%
    \hbox to 1.2em%
    {\hss$#1$\hfil}%
   \footnotesize\texttt{\string#1}\hss}%
   \penalty10}
%    \end{macrocode}
% \end{macro}
% \end{macro}
%
% \symbols
% \dosymbol{\Square} \dosymbol{\square} \dosymbol{\Diamond} \dosymbol{\diamond} \(\dosymbol{\symbol{60}}\)
% \endsymbols
%   
% \section{Best practices macros} 
% 
% We load a few packages for fixes and errors and |nag| if outdated packages are used.
% Modify to suit your requirements.  
% Package management is a bit complex to avoid errors
% with options.
%
%To find out if a package has already been loaded, use
%|\@ifpackageloaded|\meta{package}\meta{true}\meta{false}.
%|\@ifpackagelater| To find out if a package has already been loaded with a version more recent
%|\@ifclasslater| than version, use |\@ifpackagelater|\meta{hpackagei}\meta{version}\meta{true}\meta{false}.
%|\@ifpackagewith| To find out if a package has already been loaded with at least the options
%options, use |\@ifpackagewith|\meta{package}\meta{options}\meta{true}\meta{false}.
% 
%There exists one package that can't be tested with the above commands: the
%fontenc package pretends that it was never loaded to allow for repeated reloading
%with different options (see ltoutenc.dtx for details).
%
% 
% We include the following two packages to provide the standard 
% fixes for \LaTeX2e\ and the |nag| package to provide some guidance
% as to good
% practices. We set the |nag| keys to |orthodox| and |l2tabu.|
% \url{http://tex.stackexchange.com/questions/19264/techniques-and-packages-to-keep-up-with-good-practices?rq=1}
% and \href{http://stackoverflow.com/questions/193298/best-practices-in-latex}{best practices in LaTeX.}
%
%
% \begin{environment}{etex}
%    \begin{macrocode}
\ifxetex
   \else
     \ifluatex
        \RequirePackage{etex}
     \else
        \RequirePackage{etex}
  \fi
\fi
%    \end{macrocode}
% \end{environment}
%

%    \begin{macrocode}
\cxset{nag keys/.store in =\nagkeys@cx,
       onlyamsmath keys/.store in=\onlyamsmathkeys@cx,
       xcolor keys/.store in=\xcolorkeys@cx}
%        
%
%%
%% This is file `settings.tex',
%% generated with the docstrip utility.
%%
%% The original source files were:
%%
%% minimals.dtx  (with options: `settings')
%% ----------------------------------------------------------------
%% phd --- A package to beautify documents.
%% E-mail: yannislaz@gmail.com
%% Released under the LaTeX Project Public License v1.3c or later
%% See http://www.latex-project.org/lppl.txt
%% ----------------------------------------------------------------
\endinput
%%
%% End of file `settings.tex'.
 % experimental
%    \end{macrocode}
%\cxset{nag keys = {l2tabu,%
%                   orthodox,%
%                   %
%                  }}
%
% \begin{macro}{xcolor}
% 	For |xcolor| we try and load as many pre-defined colornames as
% 	possible.
%    \begin{macrocode}
\cxset{xcolor keys={fixpdftex,usenames,dvipsnames,
                  svgnames,x11names,table}}                     
% Set amsmath keys
%
%    \end{macrocode}
% \end{macro}
%
%    \begin{macrocode}
% \PassOptionsToPackage{\nagkeys@cx}{nag}
% \RequirePackage{nag}   
%    \end{macrocode}
%
%
%
% \begin{macro}{onlyasmath}
% The package |onlyasmath| also provides errors for deprecated math
% commands like using |$$|\ldots|$$| which can result in unwanted spaces
% being introduced in the typsetting of the document. The recommended 
% way is to use |\[|\ldots|\]|. The package was developed by Harold Harders
% and although targetted for class writers one might as well use it directly.
%
%% |\PassOptionsToPackage{\onlyamsmathkeys@cx}{onlyamsmath}|
% |\RequirePackage{onlyamsmath} |
% \section{microtype}
%
% \section{Typography}
%
% The package \pkgname{microtype} is loaded with no options
% as it provides facilities for loading individual features
% at run time. (This enables the use of phd keys).
%  The package had some issues with LuaLaTeX and  XeTeX
%  but now it works as advertized. As it is a great package we include it here.
%
% \subsection{Microtypography}
% With LuaTeX microtype must come after fontspec.
%    \begin{macrocode}
% 

\ifengine%
	  {\RequirePackage[tracking=true]{microtype}}%
	  {\RequirePackage[tracking=true]{microtype}}%
	  {\RequirePackage[tracking=true]{microtype}}%
%
%    \end{macrocode}
% \end{macro}
%
% \subsection{ragged2e}
%
% We load \pkgname{ragged2e} package for typography
%
% \begin{macro}{ragged2e}
%
% This package by Martin Schr\"oeder provides new commands and environments for
% setting ragged text which are easy to configure to allow hyphenation. The
% way Martin explains it, the main purpose of the package is to restore the
% plain TEX definitions which have been changed by LaTex2e. On the way it
% defines a number of useful environments. The package also loads the
% |footmisc| package if loaded with the option |footnotes|. Hm.. It also
% loads the package |everysel|. More fun. Passing of options, should be in 
% a settings file? \index{justification>ragged}
% \index{justification>ragged2e (package)}
% 
%
%
%    \begin{macrocode}
\newif\ifRAGGEDTWOE
\newif\ifEVERYSEL
\newif\ifFOOTMISC
\PassOptionsToPackage{ragged2e}{footnotes,raggedrightboxes}
\RequirePackage{ragged2e}
%    \end{macrocode}
% \end{macro}
%
% \subsection{Highlighting}
% We load the \pkgname{soul} for spacing out and for
% highlighting words. We do not pass any options. These
% are left to the user.
%
% \begin{macro}{\sethlcolor}\marg{color name} sets
% the color for soul's \hl{highlight} command \cmd{\hl}.
%    \begin{macrocode}
\newif\ifSOUL
\IfStyFileExists{soul}
{\SOULtrue\RequirePackage{soul}
    \sethlcolor{thehighlight}}
{}
%    \end{macrocode}
% \end{macro} 
% \subsection{Dropcaps} 
%  
% We use the |lettrine| package of Daniel Flipo for drop caps. We do not pass any
% defaults and leave it to the configuration file. The lettrine configuration
% file is |lettrine.cfg|. We define a command \cs{dropcap} for some settings
% that we think are generally acceptable. 
%
%    \begin{macrocode}
\RequirePackage{lettrine}
\ifx\dropcap\undefined
  \def\dropcap#1#2{%
    \lettrine[lines=3, lraise=0.1, nindent=0em, slope=.1em]{#1}{#2}
  }%
\fi
%    \end{macrocode}
%
% \subsection{Units and formatting of numbers}
% 
% We load the defacto standard for formatting units in SI units. Note
% it loads LaTeX3 packages.
%
% \begin{macro}{siunitx}
% \begin{macro}{numprint} The package \pkg{numprint} has some
% useful macros for formatting large numbers. We use it for some
% of the examples.
%    \begin{macrocode}
\RequirePackage{siunitx}
  \sisetup{fixed-exponent =0,
           scientific-notation = false}
%\RequirePackage{numprint} 
%    \end{macrocode}
% \end{macro}
% \end{macro}
%
%
% \subsection{Acronyms}
%
% For acronyms and abbreviations we load the \pkgname{acronym} package
% by Tobias Oetiker \citeyearpar{acronym}. This package makes sure, that all acronyms used 
% in the text are spelled out in full at least once. In one of its
% options it loads the |relsize| package. My recommendation is to
% load the package with the options |smaller, printonly| and 
% |withpage|. Please note that the |withpage| option only works, if the
% |printonlyused| option is present.
% 
%    \begin{macrocode}
\cxset{acronym keys/.store in = \acronymkeys@cx}
\cxset{acronym keys={smaller,printonlyused,withpage}}
\PassOptionsToPackage{\acronymkeys@cx}{acronym}
\RequirePackage{acronym}
%    \end{macrocode}
%
% We also define some common abbreviations
%    \begin{macrocode}
     \RequirePackage{mdframed}

%    \end{macrocode}
% 
% 
% \section{Graphics}
% 
%  The package \pkg{graphicx} developed by David Carlisle and Sebastian Rahtz is 
%  part of the 
%  Standard LaTeX `Graphics Bundle'.  It provides extensions to the original
% \pkg{graphics}, which it loads. The \pkg{graphics} in its turn loads
% the package \pkg{trig} which helps with trigonometrical
% calculations.
%
% We load the package |graphicx| with no options. We let |graphicx|, to 
% handle any draft options via the class itself. 
% \href{http://tex.stackexchange.com/questions/3131/graphicspath-for-miktex}{graphicspath for MikTeX} check
% adds figures etc to paths. 
% 
% \begin{docCommand} {DeclareGraphicExtensions} { \meta{extensions} } 
%   We define some common paths and extensions. 
%   to enable the user to just call he image without specifying the extension or
%   folders.
% \end{docCommand}
%
% \begin{macro}{\graphicspath}
%    \begin{macrocode}
\RequirePackage{graphicx}[1999/02/16]
\DeclareGraphicsExtensions{.jpg, .JPG, .jpeg, .png, .eps}
\graphicspath{{graphics/}{graphics//}{../images/}{images//}{./images/}{./graphics/}%
   {../graphics/}{./pic/}{../pic}}
%    \end{macrocode}
% \end{macro}
% 
%
% Various `keys' or named arguments are supported.
% \begin{description}
% \item[bb] Set the bounding box. The argument should be four
% dimensions, separated by spaces. 
% \item[bbllx,bblly,bburx,bbury] Set the bounding box. Mainly for
% compatibility with older packages. |bbllx=a,bblly=b,bburx=c,bbury=d|
% is equivalent to |bb = a b c d|.
% \item[natwidth,natheight] Again an alternative to |bb|. 
% |natheight=h,natwidth=w| is equivalent to |bb = 0 0 h w|.
% \item[viewport] Modify the bounding box specified in the file.
% The four values specify a bounding box \emph{relative} to the
% |llx|,|lly| coordinate of the original box.
% \item[trim] Modify the bounding box specified in the file.
% The four values specify the amounts to remove from
% the left, bottom, right and top of the original box.
% \item[hiresbb] Boolean valued key. Defaults to |true|. 
% Causes \TeX\ to look for |%%HiResBoundingBox| comments rather than
% the standard |%%BoundingBox|. May be set to |false| to override
% a default setting of true specified by the |hiresbb| package option.
% \item[angle] Rotation angle.
% \item[origin] Rotation origin (see |\rotatebox|, below).
% \item[width] Required width, a dimension (default units |bp|). The
% graphic will be scaled to make the width the specified dimension.
% \item[height] Required height. a dimension (default units |bp|).
% \item[totalheight] Required totalheight (ie height $+$ depth). a
% dimension (default units |bp|). Most useful after a rotation (when the
% height might be zero).
% \item[keepaspectratio] Boolean valued key (like |clip|). If it is set
%  to true, modify the meaning of the |width| and |height| (and
% |totalheight|) keys such that if both are specified then rather than
% distort the figure the figure is scaled such that neither dimension
% \emph{exceeds} the stated dimensions.
% \item[scale] Scale factor.
% \item[clip] Either `true' or `false' (or no value, which is equivalent
% to `true'). Clip the graphic to the bounding box (or viewport if one
% is specified).
% \item[draft] a boolean valued key, like `clip'. locally switches to
% draft mode, ie.\ do not include the graphic, but leave the
% correct space, and print the filename.
% \item[type] Specify the file type. (Normally determined from the file
% extension.) 
% \item[ext] Specify the file extension.
%        \emph{Only} for use with |type|.
% \item[read] Specify the `read file' which is used for determining the
% size of the graphic. \emph{Only} for use with |type|.
% \item[command] Specify the file command.
%         \emph{Only} for use with |type|.
% \end{description}
%
% The arguments are interpreted left to right. |clip|, |draft|, |bb|,,
% and |bbllx| etc.\ have the same effect wherever they appear. but the
% scaling and rotation keys interact.
%
% \begin{macro}{wrapfig} The package \pkg{wrapfig} is loaded next. 
% 
%    \begin{macrocode}
\RequirePackage{wrapfig}
%    \end{macrocode} 
% \end{macro}
%
% \begin{macro}{rotating}
% The package \pkgname{rotating} performs
% most sorts of rotation one might like, including rotation of complete floating
% figures and tables. The package was developed by Robin Fairbairns
% Sebastian Rahtz and Leonor Barroca. We use the option |quiet| as the 
% package is rather verbose.
%
%    \begin{macrocode}
\RequirePackage[quiet]{rotating}
%    \end{macrocode} 
% \end{macro}
% 
% \section{Color Management}
%
% Most classes load the |xcolor| package. Including
% it here, should either be able to check if it was 
% loaded by the class or to pass the options before
% the class itself. This package is a common source
% of errors, as classes load it with mostly different options.
% Because of this is also a good example to test our code
% in a number of minimal working examples.
%
%    \begin{macrocode}
\@ifpackageloaded{xcolor}{}%
 {\PassOptionsToPackage{\xcolorkeys@cx}{xcolor}
  \RequirePackage{xcolor}}
%    \end{macrocode}
%
%	We adopt the convention that colour names used in code should be
%	prefixed by a |the|. For simplicity we also adopt the convention
%    that all colours defined in colour schemes should be in lowercase
%	(less keystrokes and matches the styles of |pgf| keys). 
%
% \subsection{Color management}
%    \begin{macrocode}
\providecommand\href[2]{\texttt{#1}}
\definecolor{lstbgcolor}{rgb}{0.9,0.9,0.9}
\colorlet{examplefill}{yellow!80!black}
% codepalettes
%\definecolor{codebackground}{rgb}{0.972,0.929,0.753}
%\definecolor{codebackground}{rgb}{0.972,0.929,0.753}
%\definecolor{codebackground}{HTML}{B7C1C1}
%\definecolor{thekeywordstyle}{HTML}{435969}
%\definecolor{thecommentstyle}{HTML}{F87F01}
%1890
\definecolor{codebackground}{HTML}{F2F2EA}
\definecolor{thekeywordstyle}{HTML}{392726}
\definecolor{thecommentstyle}{HTML}{DF8743}

\definecolor{graphicbackground}{rgb}{0.972,0.929,0.753}
\colorlet{graphicbackground}{codebackground}




\definecolor{glyphbox}{rgb}{0.86,0.86,0.8}
%\definecolor{codebackground}{rgb}{0.8,0.8,1}
\definecolor{theblue} {rgb}{0.02,0.04,0.48}
\definecolor{thered}  {rgb}{0.65,0.04,0.07}
\definecolor{thedoccommandcolor}{rgb}{0.65,0.04,0.07}% doc command colors
\colorlet{Headings}{black} %font examples
\colorlet{Subheadings}{black} %font examples
\colorlet{thefontname}{black}%font examples
\colorlet{thehighlight}{yellow}%soul  highlight
\colorlet{thecancel}{thered}%for cancel commands
\definecolor{thegreen}{rgb}{0.06,0.44,0.08}
\definecolor{thelightgreen}{rgb}{0.06,0.44,0.06}
\definecolor{thegrey} {gray}{0.5}
\definecolor{thegray} {gray}{0.5}
\definecolor{thedarkgray} {gray}{0.95}
\definecolor{lightgray}{gray}{0.6}
\definecolor{shadedcolor}{gray}{0.6}
\definecolor{thelightgray}{gray}{0.6}
\definecolor{theshade}{gray}{0.94}
\definecolor{theframe}{gray}{0.75}
\definecolor{thecream}{rgb}{1,0.95,0.4}
\definecolor{spot}{rgb}{0,0.2,0.6}%some shades of blue
\definecolor{sweet}{rgb}{0,.68,.93}%shades of blue
%\colorlet{codebackground}{spot!5!white}
\definecolor{boxframe}{gray}{0.8}
\definecolor{boxfill}{rgb}{0.95,0.95,0.99}
\definecolor{theoption}{gray}{0.6}
\definecolor{themacro}{rgb}{0.784,0.06,0.176}
\definecolor{ExampleFrame}{rgb}{0.628,0.705,0.942}
\definecolor{ExampleBack}{rgb}{0.963,0.971,0.994}
\definecolor{Hyperlink}{rgb}{0.281,0.275,0.485}
\colorlet{thehyperlink}{theblue}
\colorlet{preciscolor}{sweet}
\colorlet{toccolor}{sweet}
%\newcommand*{\defaultcolor}{\color{black}}
%\newcommand*{\spotcolor}{\color{spot}}
%    \end{macrocode}
% 
%    \begin{macrocode}
\newcommand{\done}{\cellcolor{teal}done}  
\newcommand{\partialdone}{\cellcolor{yellow}done}
\newcommand{\hcyan}[1]{{\color{teal} #1}}
%    \end{macrocode}
%
% \section{Rules}
% We need to define a number of rules to use in typesetting styles.
% \begin{macro}{\thinrule}
% \begin{macro}{\mediumrule}
% \begin{macro}{\thickrule}
%	We will use later on different styles of rules to decorate chapter headings.
%	We define a few here to simplify code later on.
%
%    \begin{macrocode}
\DeclareRobustCommand\thickrule{%
    \leavevmode \leaders \hrule height 2pt \hfill \kern \z@}
\DeclareRobustCommand\thinrule{\vrule width\textwidth height0.4pt depth0pt\relax}%
%
\DeclareRobustCommand\mediumrule{\rule{\textwidth}{0.8pt}}
%    Adjusted to get toc parameters in
\DeclareRobustCommand\Rule{{\color{\tocchapternumberfill@cx}\rule[-4.1pt]{13cm}{0.4pt}}}
\DeclareRobustCommand\bottomline{\medskip
   \noindent\rule{\linewidth}{0.4pt}\medskip}
\DeclareRobustCommand\topline{\par\medskip
   \noindent\rule{\linewidth}{0.4pt}\medskip} 
%    \end{macrocode}
% \end{macro}
% \end{macro}
% \end{macro}
%
%  Some chapter and sectioning heads include rules, we define them here for convenience.
%
%    \begin{macrocode}
\cxset{chapter rule color/.store in={\chapter@rule@color}}%
\cxset{chapter rule color=spot!50}
\DeclareRobustCommand\tikzrule{%
  \tikz [color=\chapter@rule@color, very thick, inner sep=0pt, outer sep=0pt]%
        \draw(0,0)--(\the\linewidth,0);
}%
%  The trim left is required to align the rule exactly
%    \begin{macrocode}           
%#1  options #2 width  #3 height           
\newcommand\drawrule[3][]{%
    \offinterlineskip
          \tikz [ name=s,trim left,
                   anchor=base,
                   draw=black, 
                 % double distance=.2pt,
                  line width=#3,
                  %very thick,
                  inner sep=0pt, 
                  outer sep=0pt,#1]   \draw(0,0)--(#2,0);
}
 \def\drawdoublerule#1#2{%
    \drawrule{#1}{#2}%
    \vskip2.5pt
    \drawrule{#1}{#2}%
 }
%    \end{macrocode}
%
% \section{Filler Text}
%
%
% \begin{macro}{lipsum}
%
% 	In publishing and graphic design, lorem ipsum is placeholder text (filler text) 
% 	commonly used to demonstrate the graphics elements of a document or visual 
% 	presentation, such as font, typography, and layout, by removing the distraction 
% 	of meaningful content. The lorem ipsum text is typically a section of a Latin text 
% 	by Cicero with words altered, added and removed that make it nonsensical in meaning 
% 	and not proper Latin. Other packages exist such as |kantlipsum| and |blindtext|, 
% 	however, both result in somewhat legible texts, which defeats the purpose of 
% 	providing texts that the reader is not going to read. the extensions |lipsumx| 
% 	aim at providing a gap between the three packages. It provides extensions
% 	for full document testing.
%
%    \begin{macrocode}
\newif\ifLIPSUM
\RequirePackage{lipsum}
\RequirePackage{kantlipsum}
\RequirePackage{blindtext}
%    \end{macrocode}
% \end{macro}
%
% 
% \begin{macro}{\lorem} 
% 
%	We declare a short macro \cs{lorem} to be used for testing, as well as 
%	testing captions and the like.
% 
%    \begin{macrocode}
\DeclareDocumentCommand\lorem{ o }{Fusce adipiscing justo nec ante. Nullam in enim.
 Pellentesque felis orci, sagittis ac, malesuada et, facilisis in,
 ligula. Nunc non magna sit amet mi aliquam dictum. In mi. Curabitur
 sollicitudin justo sed quam et quadd. \par}
%    \end{macrocode}
% \end{macro}
%
% \begin{macro}{\fox} A classic one liner containing all the letters
%	 of the alphabet, used as a testing code.
%    \begin{macrocode}
\newcommand{\fox}{``The quick brown fox jumps over the lazy dog''} 
%    \end{macrocode}
% \end{macro}
%
% \begin{macro}{\frogking}
%    \begin{macrocode}
\newcommand\frogking{%
\leavevmode
\hskip1em In olden times when wishing
still helped one, there lived a
king whose daughters were all
beautiful, but the youngest was so
beautiful that the sun itself,
which has seen so much, was
astonished whenever it shone in
her face. Close by the king's
castle lay a great dark forest,
and under an old lime-tree in the
forest was a well, and when
the day was very warm, the
king's child went out into the 
forest and sat down by the side
of the cool fountain, and when she was bored she
took a golden ball, and threw it up on a high and caught it, and this
ball was her favorite plaything. \par}%
%    \end{macrocode}
%\end{macro}
%
%    \begin{macrocode}
\newcommand\onepar{In olden times when wishing
still helped one, there lived a
king whose daughters were all
beautiful, but the youngest was so
beautiful that the sun itself,
which has seen so much, was
astonished whenever it shone in
her face. Close by the king's
castle lay a great dark forest,
and under an old lime-tree in the
forest was a well, and when
the day was very warm, the
king's child went out into the 
forest and sat down by the side
of the cool fountain, and when she was bored she
took a golden ball, and threw it up on a high and caught it, and this
ball was her favorite plaything.}%

\newcommand\alicei{%	
  The King and Queen of Hearts were seated on their throne
  when they arrived, with a great crowd assembled about them
  ---all sorts of little birds and beasts, as well as the
  whole pack of cards: the Knave was standing before them,
  in chains, with a soldier on each side to guard him; and
  near the King was the White Rabbit, with a trumpet in one
  hand, and a scroll of parchment in the other.  In the very
  middle of the court was a table, with a large dish of
  tarts upon it: they looked so good, that it made Alice
  quite hungry to look at them---``I wish they'd get the
  trial done,'' she thought, ``and hand round the
  refreshments!''.  But there seemed to be no chance of this,
  so she began looking at everything about her to pass away
  the time.}%

\newcommand\aliceii{%
  Alice had never been in a court of justice before, but she
  had read about them in books, and she was quite pleased to
  find that she knew the name of nearly everything there.
  ``That's the judge,'' she said to herself, ``because of his
  great wig.''.
  
  The judge, by the way, was the King, and as he wore his
  crown over the wig, (look at the frontispiece if you want
  to see how he did it,) he did not look at all comfortable,
  and it was certainly not becoming.
}

 \newcommand\aliceiii{``And that's the jury-box,'' thought Alice, ``and those
  twelve creatures,'' (she was obliged to say ``creatures,''
  you see, because some of them were animals, and some were
  birds) ``I suppose they are the jurors.''.  She said this
  last word two or three times over to herself being rather
  proud of it: for she thought, and rightly too, that very
  few little girls of her age knew the meaning of it at all.
  However, ``jurymen'' would have done just as well.}

 \newcommand\aliceiv{The twelve jurors were all writing very busily on slates.
  ``What are they doing?'' Alice whispered to the Gryphon.
  ``They can't have anything to put down yet, before the
  trial's begun.''.}
  
\newcommand\alicev{``They're putting down their names,'' the Gryphon
  whispered in reply, ``for fear they should forget them
  before the end of the trial.''.}
  
\newcommand\alicevi{``Stupid things!'' Alice began in a loud indignant voice,
  but she stopped herself hastily, for the White Rabbit
  cried out, ``Silence in the court!''; and the King put on
  his spectacles and looked anxiously round, to make out who
  was talking.\par}

% ALPHABETS FOR TESTING FONTS
\def\ALPHABET {A B C D E F G H I J K L M N O P Q R S T U V W X Y Z}
\def\alphabet {a b c d e f g h i j k l m n o p q r s t u v w x y z}
\newcommand{\punctuation}{! ? . / , : }
%    \end{macrocode}
%
% \section{Tables}
% 
% \subsection{booktabs and helper macros}
% \begin{macro}{\inc}
% \begin{macro}{\resetinc}
%
% 	It is unlikely that a publication, would not have a table
% 	somewhere, to make life easier we load Simon Fear's |booktabs| \citep{booktabs}. The manual is a must
% read if you want to typeset typographically attractive tables.\footnote{Notice I haven't said
% typographically correct, there is no such thing.} We don't need to set any keys for the
% package.
%	The counter |inc| is used to increment serial numbers in tables and
%	\cs{resetinc} resets this counter to zero.
%
%    \begin{macrocode}
\RequirePackage{booktabs}
\newcounter{step}
\newcommand\resetinc{\setcounter{step}{0}}
\newcommand\inc{\stepcounter{step}\thestep}
%    \end{macrocode}
% \end{macro}
% \end{macro}
% 
%
%
% \begin{macro}{tabularx}
% This package by David Carlisle's enables the typesetting of fixed width 
% tables and can stretch
% specific columns. The package loads the |array| package, but we save it from some
% trouble by pre-loading it first, so we can capture its loading. The package has two keys
% |infoshow| and |debugshow| which we don't bother at this stage to load.
% 
%    \begin{macrocode}
\RequirePackage{tabularx}
%    \end{macrocode}
% \end{macro}
%
% \begin{macro}{array}
% \begin{macro}{delarray}
% The addition to array.sty added in delarray.sty is a system of implicit |\left|
%|\right| pairs. If you want an array surrounded by parentheses, you can enter:
%|\begin{array}({cc}) . .|
% 
%    \begin{macrocode}
% \RequirePackage{delarray} gives problems
\RequirePackage{array}
%    \end{macrocode}
% \end{macro}
% \end{macro}
%
% \begin{macro}{dcolumn}
%  The |dcolumn| package also by David Carlisle is loaded next. This package 
%  defines a system for defining columns of entries in an |array|
%  or tabular which are to be aligned on a `decimal point'. It also loads the |array|
%  package, which we have already loaded.
% 
%    \begin{macrocode}
\RequirePackage{dcolumn}
\RequirePackage{rccol}
%    \end{macrocode}
% \end{macro}
%
% \begin{macro}{longtable}
% \makeatletter
%  The |longtable| package, needs no introduction. It has some
%  peculiar settings and sometimes a couple of runs before it settles
%  down. The package has four keys |errorshow|, |pausing|, |set| and |final| looks 
%   as if they deprecated, at this stage we make onlty a mental note of it.
%  The package cannot be used within |multicolumn| environments and will
%  emit an error. 
% 
% 
%    \begin{macrocode}
	 \RequirePackage{longtable}
%    \end{macrocode}
% The following environment was copied verbatim from the Comprehensive
% Symbols. I have been trying to understand it ever since.
%
%    \begin{macrocode}
\let\origLT@array=\LT@array
\let\origLT@start=\LT@start

\newenvironment{longsymtable}[2][true]{%
  \expandafter\global\expandafter\let
  \expandafter\ifshowsymtable\csname if#1\endcsname
  \ifshowsymtable
    \mbox{}%
    \Needspace*{1\baselineskip}%
    \mbox{}%
    \begin{center}%
    \phantomsection
    \refstepcounter{table}%
    \let\refstepcounter=\@gobble
    \let\LT@array=\origLT@array
    \let\LT@start=\origLT@start
%
    \addcontentsline{toc}{subsection}{%
     \protect\numberline{\tablename~\thetable:}{#2}}%
    \@makecaption{\fnum@table}{#2}%
    \gdef\lt@indexed{}%
    \let\next=\relax
  \else
    % The following was taken verbatim from verbatim.sty.
    \let\do\@makeother\dospecials\catcode`\^^M\active
    \let\verbatim@startline\relax
    \let\verbatim@addtoline\@gobble
    \let\verbatim@processline\relax
    \let\verbatim@finish\relax
    \let\next=\verbatim@
  \fi
  \next
}{%
  \ifshowsymtable
    \end{center}
    \let\@elt=\index\lt@indexed  % Close our index ranges.
    \gdef\lt@indexed{}%
    \vskip 8ex minus 2ex
  \fi
}


% Define \index-like commands for use with longsymtable that
% automatically apply to the entire table, not just the start of it.

\newcommand{\ltindex}[1]{%
  \index{#1|(}%
  \@cons{\lt@indexed}{{#1|)}}%
}
\newcommand{\ltidxboth}[2]{\mbox{}\ltindex{#1 #2}\ltindex{#2>#1}}

\let\LT@array=\origLT@array
\let\LT@start=\origLT@start
%    \end{macrocode}
% \end{macro}   
%
% \begin{macro}{multirow}
% 
% 
% The \pkg{multirow} by Piet van Oostrum and its two companion packages
% \pkgname{bigdelim} and \pkgname{bigstrut} can be used to define multirow cells. They are difficult
% to get right and in most instances one can redesign the tables better without
% resorting to multi-rows. It has a strange interaction with the \pkgname{colortbl}
% and a hack around its usage which we will load next.
% 
%    \begin{macrocode}
% If we have type1cm.sty, use it.
% 
\IfStyFileExists*{type1cm}
  {\usepackage{type1cm}}
  {}
\RequirePackage{colortbl}
% If we have multirow.sty, use it.
\newif\ifhavemultirow
\IfStyFileExists*{multirow}
  {\havemultirowtrue\RequirePackage{multirow}}
  {}
%    \end{macrocode}
% \end{macro}
% 
% \begin{macro}{threeparttable}
% The package \pkgname{threeparttable} by Donald Arsenau facilitates tables with titles (captions) and notes. The
% package comes with a number of options |para|, |flushleft|, online and normal. We also load Lars Madsen's 
% \pkgname{threeparttablex} that extends the package to work with \pkgname{longtable}.
%
%   \label{threeparttable}
%    \begin{macrocode}
\RequirePackage{threeparttable}
\RequirePackage{threeparttablex}
%    \end{macrocode} 
% \end{macro}
%
%  {\begin{center}
%
%  \begin{threeparttable}[b]
%    \caption{...}
%  \begin{tabular}{ll} 
%   \toprule
%    one cell 42\tnote{1}&\\
%    another cell \tnote[2]&\\
%  \bottomrule
%  \end{tabular}
%  \begin{tablenotes}
%    \item [1] the first note ...
%    \item [2] the second note
%  \end{tablenotes}
%  \end{threeparttable}
%  \end{center}}
%
% \subsection{arydshln}
%
%This package by Hiroshi Nakashima  gives \latex’s \pkg{array} and \pkg{tabular} environments the capability to draw horizontal/vertical dash-lines.
%\begin{macro}{arydshln}
% I do not have a normal use for it, personally but I have included it here for correct ordering i3n case there is a need for it. According to the package documentation, it has to be loaded after \pkg{array}, \pkg{longtable}, \pkg{colortab}, \pkg{colortbl}. Also according to the hyperref documentation it has to be loaded after hyperref as well. 
%\end{macro}
%
%^^A \begin{center}
%^^A\begin{tabular}{|l::c:r|}\hline
%^^A A&B&C\\\hdashline
%^^A %AAA&BBB&CCC\\\cdashline{1-2}
%^^A %\multicolumn{2}{|l:}{AB}&C\\\hdashline\hdashline
%^^A %\end{tabular}
%^^A % \end{center}
%  
% \section{Landscape Pages}
%
% A common request from authors is to rotate text, tables and or
% figures and to typeset the content using a landscape page.
%
% \begin{macro}{pdflscape}
% \begin{macro}{lscape}
% The package \pkgname{pdflscape} by Heiko Oberdiek  adds PDF support 
% to the environment |landscape| 
% of  package |lscape| by setting the PDF page attribute /Rotate. 
% It has to be loaded after \pkgname{lscape} so we let it load it itself.
%
%    \begin{macrocode}
\RequirePackage{pdflscape}
%    \end{macrocode}
% \end{macro}
% \end{macro}
%  
% \chapter{Maths Implementation}
%
% 	Although we cognizant that there are documents that do not use math
% 	and perhaps others that our selection of packages is inadequate, we
% 	offer a bundle of what we think will cover most of the cases. One
% 	issue with maths is that we are limited with TeX's built-in math
% 	alphabet limitations. We aim to satisfy the most common requirements.
% 
% \begin{macro}{empheq}
% \begin{macro}{mathtools} 
%	We start with \pkgname{mathtools}, as it loads the
% 	|amsmath| package and can also pass options to it. The package was developed 
% 	by Lars Madsen and is maintained by Will Robertson and Joseph Wright. It
% 	appears to be very popular with a lot of scholars in the sciences and
% 	mathematical fields and hence I decided to  include it here.
% 
%  We provide egreg's hack to extend the math alphabets for XeTeX
%  These hacks only work for XeTeX and LuaTeX.
%    \begin{macrocode}
\ifxetex
  \def\new@mathgroup{\alloc@8\mathgroup\mathchardef\@cclvi}
  \patchcmd{\document@select@group}{\sixt@@n}{\@cclvi}{}{}
  \patchcmd{\select@group}{\sixt@@n}{\@cclvi}{}{}
\fi
\ifluatex
  \def\new@mathgroup{\alloc@8\mathgroup\mathchardef\@cclvi}
  \patchcmd{\document@select@group}{\sixt@@n}{\@cclvi}{}{}
  \patchcmd{\select@group}{\sixt@@n}{\@cclvi}{}{}
\fi
%    \end{macrocode}
%
%
%    \begin{macrocode}
\newif\ifAMS
%\newcommand\AMS{\AmS\index{AMS=\AmS}}
\AMStrue
%    \end{macrocode}
%\IfStyFileExists{amssymb}
%  {\AMStrue
%   \savesymbol{angle} \savesymbol{rightleftharpoons}
%   \savesymbol{leftharpoondown} \savesymbol{rightharpoonup}
%   \savesymbol{iint} \savesymbol{iiint}
%   \savesymbol{iiiint} \savesymbol{idotsint}
%   \let\orig@ifstar=\@ifstar
%   \let\overleftrightarrow \undefined%CHECK
%   \let\underleftarrow\undefined
%    \let\underrightarrow\undefined 
%    \let\underleftrightarrow\undefined 
%   \RequirePackage{amsmath}
%   \RequirePackage{amssymb}
%   \let\@ifstar=\orig@ifstar
%   \restoresymbol{AMS}{angle} \restoresymbol{AMS}{rightleftharpoons}
%   \restoresymbol{AMS}{lefthapoondown} \restoresymbol{AMS}{rightharpoonup}
%   \restoresymbol{AMS}{iint} \restoresymbol{AMS}{iiint}
%   \restoresymbol{AMS}{iiiint} \restoresymbol{AMS}{idotsint}
%  }
%  {
%    % The following was modified from amsmath.sty.
%    \newcommand{\AmSfont}{%
%      \usefont{OMS}{cmsy}{m}{n}}
%    \providecommand{\AmS}{{\protect\AmSfont
%      A\kern-.1667em\lower.5ex\hbox{M}\kern-.125emS}}
%  }
%    
%
% 
% Macros to try and find available fonts for XeTeX sample docs. This is
% copied verbatim from XeTeX documentation bundle.
%
% Usage:
%
% |\testFontIsAvailable{font-name}|
%   sets |\ifFontIsAvailable| according to whether or not it could be found
%
% |\FindAnInstalledFont{font-name/alternative/another/yet-another}{\cs}|
%   searches for an available font from among the names given,
%   and |\def|'s the control sequence |\cs| to the first one found
%   or to <No suitable font found> if none (which will subsequently
%   cause an error when used in a |\font| command). A word of warning this
%   can cause the system to compile the document very slowly.
%
%    \begin{macrocode}
%
\newif\ifFontIsAvailable
\def\testFontAvailability#1{%
  \count255=\interactionmode
  \batchmode
  \let\preload=\nullfont
  \font\preload="#1" at 10pt
  \ifx\preload\nullfont \FontIsAvailablefalse
  \else \FontIsAvailabletrue \fi
  \interactionmode=\count255
}

\def\FindAnInstalledFont#1#2{
  \expandafter\getFirstFontName#1/\end
  \let\next\gobbleTwo
  \ifx\trialFontName\empty
    \def#2{<No suitable font found>}%
  \else
    \testFontAvailability{\trialFontName}
    \ifFontIsAvailable
      \edef#2{\trialFontName}%
    \else
      \let\next\FindAnInstalledFont
    \fi
  \fi
  \expandafter\next\expandafter{\remainingNames}{#2}
}
\def\getFirstFontName#1/#2\end{\def\trialFontName{#1}\def\remainingNames{#2}}
\def\gobbleTwo#1#2{}
%
%    \end{macrocode}
% \begin{macro}{\ligatures}
%    \begin{macrocode}
\newcommand\ligatures[2][Old Standard-Regular]{%
  \bgroup
  \fontspec[Ligatures = Common]{#1}%
  \textit{#2}%
  \egroup
}
\renewcommand\U[1]{{\texttt{U+#1}}(\char"#1)\xspace}
%    \end{macrocode}
% \end{macro}

% 
%
%
% 
% \subsection{ymath}
%
% We load Yiannis Haralambous \pkgname{ymath}\ctan{ymath} package for its extensible wide accents. 
% not loaded? 
%    \begin{macrocode}
\newif\ifYH
\newcommand\YH{yhmath}
\IfStyFileExists{yhmath}
  {\YHtrue
   \let\origRequirePackage=\RequirePackage    % We don't want amsmath loaded.
   \def\RequirePackage##1{}
   \RequirePackage{yhmath}
   \let\RequirePackage=\origRequirePackage
  }
  {}
%    \end{macrocode}

%

%
% \section{The accents package}
%
% If we have the \pkgname{accents}\ctan{accents} package \citep{accents}, use it (for an example in the section
% on constructing new symbols). Please do note that you need to use the
% right command name if we have restored it. Do note that the package redeclares
%\index{accents (package commands)>\ttfamily\string\underaccent}%
%\index{accents (package commands)>\ttfamily\string\ring}%
%\index{accents (package commands)>\ttfamily\string\undertilde}%
%\index{accents (package commands)>\ttfamily\string\dddot}%
%\index{accents (package commands)>\ttfamily\string\ddddot}%
%    \begin{macrocode}
\newif\ifACCENTS
\IfStyFileExists{accents}
  {\ACCENTStrue
   \savesymbol{undertilde}
   \savesymbol{dddot}
   \savesymbol{ddddot}
   \RequirePackage{accents}
   \restoresymbol{ACCENTS}{undertilde}
   \restoresymbol{ACCENTS}{dddot}
   \restoresymbol{ACCENTS}{ddddot}
  }
  {}   
%    \end{macrocode}
%
% \subsection{mathrsfs}
%
%  The package \pkgname{mathrsfs} provides calligraphic style fonts.
%  ^^A\mathscr{A B C D E F G}
% \begin{macro}{\mathscr}
%    \begin{macrocode}
\IfStyFileExists{mathrsfs}
  {\newcommand{\mathscr}[1]{\mbox{\usefont{U}{rsfs}{m}{n}##1}}}
  {}
%    \end{macrocode}
% \end{macro}
%
% \section{txfonts}
% 
% pxfonts relies on txfonts (I think), so either package can be loaded.
% Note that txfonts/pxfonts redefine every LaTeX and AMS character,
% which is not what we want.  As a result, we have to rely on some
% serious trickery to prevent our old characters from getting redefined.
% If we are running with XeTeX this has to be before AMS and other packages
% and on top of fontspec. 
%    \begin{macrocode}
\def\TX{txfonts}
%    \end{macrocode}
% 
%
% \subsection{mathabx}
%
% Here's a real problem child: mathabx, which also redefines virtually
% every symbol provided by LaTeX2e and AMS.  We have to resort to our
% most devious trickery to get mathabx to load properly.
%
%    \begin{macrocode}
%
%\newif\ifABX
%\def\ABX{\pkgname{mathabx}}
%\let\origDeclareMathSymbol=\DeclareMathSymbol
%\let\origDeclareMathDelimiter=\DeclareMathDelimiter
%\let\origDeclareMathRadical=\DeclareMathRadical
%\let\origDeclareMathAccent=\DeclareMathAccent
%
%  % Redefine \DeclareMathSymbol to stick "ABX" in front of each symbol name.
%  \renewcommand{\DeclareMathSymbol}[4]{%
%    \let\mathabx@undefine=\@gobble  % Undefining symbols causes all sorts of problems for us.
%    \edef\newname{\expandafter\@gobble\string#1}
%    \ifx\newname\@empty
%    \else
%      \edef\newname{ABX\newname}
%      \expandafter\origDeclareMathSymbol\expandafter{%
%        \csname\newname\endcsname}{#2}{#3}{#4}%
%    \fi
%  }
%  % Do the same for \DeclareMathDelimiter.
%  \def\DeclareMathDelimiter#1{%
%    \edef\newname{\expandafter\@gobble\string#1}
%    \def\eatfour##1##2##3##4{}%
%    \def\eatfive##1##2##3##4##5{}%
%    \ifx\newname\@empty
%      \if\relax\noexpand#1%
%        \def\next{\eatfive}
%      \else
%        \def\next{\eatfour}
%      \fi
%    \else
%      \edef\newname{ABX\newname}
%      \def\next{%
%        \expandafter\origDeclareMathDelimiter\expandafter{%
%          \csname\newname\endcsname}}
%    \fi
%    \next
%  }
%  % Do the same for \DeclareMathAccent.
%  \renewcommand{\DeclareMathAccent}[4]{%
%    \edef\newname{\expandafter\@gobble\string#1}
%    \ifx\newname\@empty
%    \else
%      \edef\newname{ABX\newname}
%      \expandafter\origDeclareMathAccent\expandafter{%
%        \csname\newname\endcsname}{#2}{#3}{#4}%
%    \fi
%  }
%  % Redefine \DeclareMathRadical to do nothing.
%  \renewcommand{\DeclareMathRadical}[5]{}
%
%\let\proofmode=1
%\RequirePackage{mathabx}
%\IfStyFileExists{mathabx}
%  {\ABXtrue
%   \savesymbol{not} \savesymbol{widering}\savesymbol{Moon}
%   \savesymbol{notowner} \savesymbol{iint} \savesymbol{iiint}
%   \savesymbol{oint} \savesymbol{oiint} \savesymbol{bigboxperp}
%   \savesymbol{bigoperp} \savesymbol{boxedcirc} \savesymbol{boxeddash}
%   \savesymbol{boxeedast} \savesymbol{boxperp} \savesymbol{boy}
%   \savesymbol{Cap} \savesymbol{centerdot} \savesymbol{circledast}
%   \savesymbol{circledcirc} \savesymbol{circleddash} \savesymbol{Cup}
%   \savesymbol{curvearrowtopleft} \savesymbol{curvearrowtopleftright}
%   \savesymbol{curvearrowtopright} \savesymbol{doteqdot}
%   \savesymbol{geqslant} \savesymbol{gets} \savesymbol{girl}
%   \savesymbol{Join} \savesymbol{land} \savesymbol{leqslant}
%   \savesymbol{looparrowupleft} \savesymbol{looparrowupright}
%   \savesymbol{lor} \savesymbol{lsemantic}
%   \savesymbol{mayaleftdelimiter} \savesymbol{mayarightdelimiter}
%   \savesymbol{ndivides} \savesymbol{nequiv} \savesymbol{ngeqslant}
%   \savesymbol{ni} \savesymbol{nleqslant} \savesymbol{notni}
%   \savesymbol{notowns} \savesymbol{notsign} \savesymbol{operp}
%   \savesymbol{rsemantic} \savesymbol{sqCap} \savesymbol{sqCup}
%   \savesymbol{to} \savesymbol{ulsh} \savesymbol{ursh}
%   \savesymbol{overbrace} \savesymbol{underbrace}
%   \savesymbol{overgroup} \savesymbol{undergroup}
%   \savesymbol{dddot} \savesymbol{ddddot}
%
%   \RequirePackage{mathabx}
%
%   \restoresymbol{ABX}{not} \restoresymbol{ABX}{widering}
%   \restoresymbol{ABX}{Moon} \restoresymbol{ABX}{notowner}
%   \restoresymbol{ABX}{iint} \restoresymbol{ABX}{iiint}
%   \restoresymbol{ABX}{oint} \restoresymbol{ABX}{oiint}
%   \restoresymbol{ABX}{bigboxperp} \restoresymbol{ABX}{bigoperp}
%   \restoresymbol{ABX}{boxedcirc} \restoresymbol{ABX}{boxeddash}
%   \restoresymbol{ABX}{boxeedast} \restoresymbol{ABX}{boxperp}
%   \restoresymbol{ABX}{boy} \restoresymbol{ABX}{Cap}
%   \restoresymbol{ABX}{centerdot} \restoresymbol{ABX}{circledast}
%   \restoresymbol{ABX}{circledcirc} \restoresymbol{ABX}{circleddash}
%   \restoresymbol{ABX}{Cup} \restoresymbol{ABX}{curvearrowtopleft}
%   \restoresymbol{ABX}{curvearrowtopleftright}
%   \restoresymbol{ABX}{curvearrowtopright}
%   \restoresymbol{ABX}{doteqdot} \restoresymbol{ABX}{geqslant}
%   \restoresymbol{ABX}{gets} \restoresymbol{ABX}{girl}
%   \restoresymbol{ABX}{Join} \restoresymbol{ABX}{land}
%   \restoresymbol{ABX}{leqslant} \restoresymbol{ABX}{looparrowupleft}
%   \restoresymbol{ABX}{looparrowupright} \restoresymbol{ABX}{lor}
%   \restoresymbol{ABX}{lsemantic}
%   \restoresymbol{ABX}{mayaleftdelimiter}
%   \restoresymbol{ABX}{mayarightdelimiter}
%   \restoresymbol{ABX}{ndivides} \restoresymbol{ABX}{nequiv}
%   \restoresymbol{ABX}{ngeqslant} \restoresymbol{ABX}{ni}
%   \restoresymbol{ABX}{nleqslant} \restoresymbol{ABX}{notni}
%   \restoresymbol{ABX}{notowns} \restoresymbol{ABX}{notsign}
%   \restoresymbol{ABX}{operp} \restoresymbol{ABX}{rsemantic}
%   \restoresymbol{ABX}{sqCap} \restoresymbol{ABX}{sqCup}
%   \restoresymbol{ABX}{to} \restoresymbol{ABX}{ulsh}
%   \restoresymbol{ABX}{ursh} \restoresymbol{ABX}{overbrace}
%   \restoresymbol{ABX}{underbrace} \restoresymbol{ABX}{overgroup}
%   \restoresymbol{ABX}{undergroup}
%   \restoresymbol{ABX}{dddot} \restoresymbol{ABX}{ddddot}
%  }
%  {}
%\let\DeclareMathAccent=\origDeclareMathAccent
%\let\DeclareMathRadical=\origDeclareMathRadical
%\let\DeclareMathDelimiter=\origDeclareMathDelimiter
%\let\DeclareMathSymbol=\origDeclareMathSymbol
%\ifABX
%  % Define only those accents that are not defined elsewhere.
%  \DeclareMathAccent{\widecheck}     {0}{mathx}{"71}
%  \DeclareMathAccent{\widebar}       {0}{mathx}{"73}
%  \DeclareMathAccent{\widearrow}     {0}{mathx}{"74}
%  % Redefine all let-bound symbols.
%  \let\ABXcenterdot=\ABXsqbullet
%  \let\ABXcircledast=\ABXoasterisk
%  \let\ABXcircledcirc=\ABXocirc
%  % Ensure that \ABXwidering invokes \ABXwideparen, not \wideparen.
%  \def\ABXwidering#1{\ring{\ABXwideparen{#1}}}
%  % Redefine commands that are used by other commands.
%  \DeclareMathSymbol{\ABXnotsign}    {3}{matha}{"7F}
%  \DeclareMathSymbol{\ABXvarnotsign} {3}{mathb}{"7F}
%  \DeclareMathSymbol{\ABXnotowner}   {3}{matha}{"53}
%  
%    \def\ABXoverbrace{\overbrace@{\bracefill\ABXbraceld\ABXbracemd\ABXbracerd\ABXbracexd}}
%    \def\ABXunderbrace{\underbrace@{\bracefill\ABXbracelu\ABXbracemu\ABXbraceru\ABXbracexu}}
%    \def\ABXovergroup{\overbrace@{\bracefill\ABXbraceld{}\ABXbracerd\ABXbracexd}}
%    \def\ABXundergroup{\underbrace@{\bracefill\ABXbracelu{}\ABXbraceru\ABXbracexu}}
%  
%  % Define a command to select the mathb font.
%  \newcommand{\mathbfont}{\usefont{U}{mathb}{m}{n}}
%\fi    % ABX test
%%
%    \end{macrocode}
%
%   
%
% \section{mathtools}
%
% 
%    \begin{macrocode}
%
\newif\ifMTOOLS
\newcommand\MTOOLS{\pkgname{mathtools}}
% \RequirePackage{mathtools}
 \RequirePackage{suffix}
\IfStyFileExists{mathtools}
  {\MTOOLStrue
   \savesymbol{xleftrightarrow} \savesymbol{xLeftarrow}
   \savesymbol{xRightarrow} \savesymbol{xLeftrightarrow}
   \savesymbol{xrightharpoondown} \savesymbol{xrightharpoonup}
   \savesymbol{xleftharpoondown} \savesymbol{xleftharpoonup}
   \savesymbol{xleftrightharpoons} \savesymbol{xrightleftharpoons}
   \savesymbol{xhookleftarrow} \savesymbol{xhookrightarrow}
   \savesymbol{xmapsto} \savesymbol{underbracket}
   \savesymbol{overbracket} \savesymbol{lparen} \savesymbol{rparen}
   \savesymbol{dblcolon} \savesymbol{coloneqq} \savesymbol{Coloneqq}
   \savesymbol{coloneq} \savesymbol{Coloneq} \savesymbol{eqqcolon}
   \savesymbol{Eqqcolon} \savesymbol{eqcolon} \savesymbol{Eqcolon}
   \savesymbol{colonapprox} \savesymbol{Colonapprox}
   \savesymbol{colonsim} \savesymbol{Colonsim} \savesymbol{overbrace}
   \savesymbol{underbrace}

   % The mathtools package delays the definitions of some of its symbols
   % to the \begin{document}.  We redefine \AtBeginDocument to force
   % mathtools to define everything immediately.
   \let\origAtBeginDocument=\AtBeginDocument
   \def\AtBeginDocument##1{##1}
   \usepackage[donotfixamsmathbugs]{mathtools}
   \let\AtBeginDocument=\origAtBeginDocument

   \restoresymbol{MTOOLS}{xleftrightarrow}
   \restoresymbol{MTOOLS}{xLeftarrow}
   \restoresymbol{MTOOLS}{xRightarrow}
   \restoresymbol{MTOOLS}{xLeftrightarrow}
   \restoresymbol{MTOOLS}{xrightharpoondown}
   \restoresymbol{MTOOLS}{xrightharpoonup}
   \restoresymbol{MTOOLS}{xleftharpoondown}
   \restoresymbol{MTOOLS}{xleftharpoonup}
   \restoresymbol{MTOOLS}{xleftrightharpoons}
   \restoresymbol{MTOOLS}{xrightleftharpoons}
   \restoresymbol{MTOOLS}{xhookleftarrow}
   \restoresymbol{MTOOLS}{xhookrightarrow}
   \restoresymbol{MTOOLS}{xmapsto}
   \restoresymbol{MTOOLS}{underbracket}
   \restoresymbol{MTOOLS}{overbracket} \restoresymbol{MTOOLS}{lparen}
   \restoresymbol{MTOOLS}{rparen} \restoresymbol{MTOOLS}{dblcolon}
   \restoresymbol{MTOOLS}{coloneqq} \restoresymbol{MTOOLS}{Coloneqq}
   \restoresymbol{MTOOLS}{coloneq} \restoresymbol{MTOOLS}{Coloneq}
   \restoresymbol{MTOOLS}{eqqcolon} \restoresymbol{MTOOLS}{Eqqcolon}
   \restoresymbol{MTOOLS}{eqcolon} \restoresymbol{MTOOLS}{Eqcolon}
   \restoresymbol{MTOOLS}{colonapprox}
   \restoresymbol{MTOOLS}{Colonapprox}
   \restoresymbol{MTOOLS}{colonsim} \restoresymbol{MTOOLS}{Colonsim}
   \restoresymbol{MTOOLS}{overbrace} \restoresymbol{MTOOLS}{underbrace}

   % Some of the above are defined in terms of \dblcolon.  At the time
   % of this writing it doesn't seem like any other package uses the
   % name \dblcolon so it should be safe to retain its mathtools
   % definition.
   \let\dblcolon=\MTOOLSdblcolon
  }
  {}
%    \end{macrocode}
 \PassOptionsToPackage{leqno}{mathtools}
% \section{empheq}
%
%  This is not on ctan and I removed it
%    \begin{macrocode}
%^^A\RequirePackage[allowspaces]{empheq} %defines harpoon macros
%    \end{macrocode}
%
% \section{Fractions}
% We load two packages for fractions, but our preference is to use the
% \pkgname{xfrac}. We load \pkgname{nicefrac} in case anyone disagrees.
% 
% \subsection{The nicefrac and xfrac package}
% The package \pkgname{xfrac} produces better fractions. 
% The \pkgname{nicefrac} is an older package. I am told some people still use it.
%    \begin{macrocode}
\RequirePackage{nicefrac}
\RequirePackage{xfrac}
%    \end{macrocode}
% \end{macro}
% \end{macro}
%
% Many of the tools shown in this manual can be turned on and off by 
% setting a switch to
% either true or false. In all cases it is done with the command 
% |\mathtoolsset|. A typical
%  use could be something like
% \begin{equation}
%  a=c+d
% \end{equation}
%
% This provides a useful way to hook into the package options
% using our setting interface.
%
%    \begin{macrocode}
\cxset{tag left bracket/.store in = \leftbracket@cx,
         tag right bracket/.store in = \rightbracket@cx,
         tag font-weight/.store in = \tagfontweight@cx,
         mathtool center colon/.store in=\centeredcolon@cx}
%
\cxset{tag left bracket =[,
         tag right bracket =],
         tag font-weight=\textbf,
         mathtool center colon=false} 

\newtagform{brackets}[\tagfontweight@cx]{\leftbracket@cx}%
           {\rightbracket@cx}
\mathtoolsset{centercolon=true,mathic}%italic correction in math
\numberwithin{equation}{section}
%    \end{macrocode}
%
% HAS ERRORS FIX ME
% 
% \begin{macro}{amssymb}
% \begin{macro}{amsthm}
% \begin{macro}{amsopn}
%    \begin{macrocode}
\RequirePackage{amssymb}[2002/01/22]
\RequirePackage{amsthm}[2002/01/22]
\RequirePackage{amsopn}
\RequirePackage{amscd}
% add more tabs for bmatrix
\setcounter{MaxMatrixCols}{20}
%    \end{macrocode}
%
% \subsection{dsfont}
%
% The \pkgname{dsfont} which is available in MikTeX as \pkgname{dstroke} can be useful
% for typesetting the mathematical symbols for the natural numbers
% \person{Olaf}{Kummer} \citep{dsfont}. 
% It breaks XeTeX and LuaTeX so we only load it for
% LaTeX.
% 
%    \begin{macrocode} 
\ifengine{}{}{% 
 \IfStyFileExists{dsfont}%
   {\newcommand{\mathds}[1]{\mbox{\usefont{U}{dsrom}{m}{n}##1}}
    \newcommand{\mathdsss}[1]{\mbox{\usefont{U}{dsss}{m}{n}##1}}}
   {}}
%    \end{macrocode}
% \end{macro}
% \end{macro}
% \end{macro}
%
% The package |stmaryrd| can be used for additional symbols. 
%    \begin{macrocode}
%		\RequirePackage{stmaryrd}
%    \end{macrocode}
% 
% The \pkgname{amscd} is probably not useful at all as people are
% moving to graphical programs such as TikZ for their commutative
% diagrams.
%  
% \begin{macro}{empheq} 
%    \begin{macrocode}
%        \RequirePackage{empheq}
%    \end{macrocode}
% \end{macro}
%
%This package\footnote{The package is part of the \texttt{mh}-bundle 
%of Morten H\o gholm (\href{http://www.ctan.org/tex-archive/macros/latex/contrib/mh/}{CTAN://macros/latex/contrib/mh/}).} 
%supports different frames for math environments of the 
% AmSmath
%package. It doesn't support  all the environments from %standard \LaTeX{} which 
% are not modified by \AmS{}math.
%
%With the optional argument of the empheq
%the preferred box type
%can be specified. A simple one is |fbox|.
%
%  ^^A\begin{empheq}[box=\fbox]{align}
%	^^Af(x)=\int_1^{\infty}\frac{1}{x^2}\,\mathrm{d}x=1
%  ^^A\end{empheq}

% \subsection{xpfeil}
%
%  The package \pkgname{extpfeil} loads \pkgname{stmaryd} with limited options
%  we temporarily make |\RequirePckage| a no-op to prevent
%   LaTeX from complaining.
% 
% Manually define every symbol in \pkgname{cmll} so we don't have to use any more
% math alphabets.

%    \begin{macrocode} 
% feyn provides yet another math font for which we have no room.
% Fortunately, it's relatively easy to define all of its symbols in
% terms of a text font.
\newif\ifFEYN
\newcommand\FEYN{\pkgname{feyn}}
\IfStyFileExists{feyn}
  {\FEYNtrue
   \let\origProvidesPackage=\ProvidesPackage
   \def\ProvidesPackage##1[##2]{\origProvidesPackage{##1}[##2]\endinput}
   \savesymbol{filename}
   \usepackage{feyn}
   \restoresymbol{FEYN}{filename}
   \let\ProvidesPackage=\origProvidesPackage
   \DeclareFontFamily{OMS}{textfeyn}{\skewchar\font'000}
   \DeclareFontShape{OMS}{textfeyn}{m}{n}{%
     <-10.5>feyntext10%
     <10.5-11.5>feyntext11%
     <11.5->feyntext12%
   }{}
   \DeclareRobustCommand{\feyn}[1]{{\usefont{OMS}{textfeyn}{m}{n}##1}}
   \DeclareRobustCommand{\wfermion}{\feyn{\char"64}}
   \DeclareRobustCommand{\hfermion}{\feyn{\char"6B}}
   \DeclareRobustCommand{\shfermion}{\feyn{\char"6C}}
   \DeclareRobustCommand{\whfermion}{\feyn{\char"6D}}
   \DeclareRobustCommand{\gvcropped}{\feyn{\char"07}}
   \DeclareRobustCommand{\bigbosonloop}{\feyn{\char"7B}}
   \DeclareRobustCommand{\smallbosonloop}{\feyn{\char"7C}}
   \DeclareRobustCommand{\bigbosonloopA}{\feyn{\char"5B}}
   \DeclareRobustCommand{\smallbosonloopA}{\feyn{\char"5C}}
   \DeclareRobustCommand{\bigbosonloopV}{\feyn{\char"1B}}
   \DeclareRobustCommand{\smallbosonloopV}{\feyn{\char"1C}}
  }
  {}
\newif\ifULSY
\newcommand\ULSY{\pkgname{ulsy}}
\IfStyFileExists{ulsy}
  {\ULSYtrue\usepackage{ulsy}}
  {}

%\newif\ifIGO
%\newcommand\IGO{\pkgname{igo}}
%\RequirePackage{igo}
%\IfStyFileExists{igo}
%  {\savesymbol{black}
%   \savesymbol{white}
%   \savesymbol{repeat}
%   % Don't let igo redefine all of the font-size commands.
%   \savesymbol{scriptsize}\newcommand{\scriptsize}{}
%   \savesymbol{tiny}\newcommand{\tiny}{}
%   \savesymbol{large}\newcommand{\large}{}
%   \savesymbol{Large}\newcommand{\Large}{}
%   \savesymbol{LARGE}\newcommand{\LARGE}{}
%   \savesymbol{huge}\newcommand{\huge}{}
%   \savesymbol{Huge}\newcommand{\Huge}{}
%   \IGOtrue\usepackage{igo}
%   \restoresymbol{IGO}{black}
%   \restoresymbol{IGO}{white}
%   \restoresymbol{IGO}{repeat}
%   \restoresymbol{IGO}{tiny}
%   \restoresymbol{IGO}{large}
%   \restoresymbol{IGO}{Large}
%   \restoresymbol{IGO}{LARGE}
%   \restoresymbol{IGO}{huge}
%   \restoresymbol{IGO}{Huge}
%   % Define a version of \whitestone and \blackstone that avoid
%   % bracketed arguments.
%   \DeclareRobustCommand{\igowhitestone}[1]{\whitestone[##1]}
%   \DeclareRobustCommand{\igoblackstone}[1]{\blackstone[##1]}
%  }
%  {}
%

\newif\ifCEQ
\newcommand\CEQ{\pkgname{colonequals}}
\IfStyFileExists{colonequals}
  {\savesymbol{colonapprox}
   \savesymbol{colonsim}
   \CEQtrue
   \usepackage{colonequals}
   \restoresymbol{CEQ}{colonapprox}
   \restoresymbol{CEQ}{colonsim}
  }
  {}
%    \end{macrocode}
%
% \section{Linear Logic Symbols cmll}
% 
% The \pkgname{cmll} font defines a handful of symbols useful in linear logic that were not defined in other fonts and packages. The package needs to be loaded
% after txtfonts. We rename some of 
%
%   \CMLLbigparr 
%   \CMLLbigwith
%
%    \begin{macrocode}
\newif\ifCMLL
\newcommand\CMLL{\pkgname{cmll}}
\IfStyFileExists{cmll}
  {\CMLLtrue
   \newcommand*{\textCMLL}[1]{{\usefont{U}{cmllr}{m}{n}##1}}
   \DeclareRobustCommand{\CMLLparr}{\textCMLL{\char0}}
   \DeclareRobustCommand{\CMLLshpos}{\textCMLL{\char1}}
   \DeclareRobustCommand{\CMLLshneg}{\textCMLL{\char2}}
   \DeclareRobustCommand{\CMLLshift}{\textCMLL{\char3}}
   \DeclareRobustCommand{\CMLLcoh}{\textCMLL{\char4}}
   \DeclareRobustCommand{\CMLLscoh}{\textCMLL{\char5}}
   \DeclareRobustCommand{\CMLLincoh}{\textCMLL{\char6}}
   \DeclareRobustCommand{\CMLLsincoh}{\textCMLL{\char7}}
   \DeclareRobustCommand{\CMLLbigwith}{\raisebox{2ex}{\textCMLL{\char8}}}
   \DeclareRobustCommand{\CMLLbigparr}{\raisebox{2ex}{\textCMLL{\char10}}}
  }
  {}
%% Stmaryd package
 \newif\ifST
 \newcommand\ST{\pkgname{stmaryrd}}
 \IfStyFileExists{stmaryrd}
  {\STtrue
   \savesymbol{lightning}
   \savesymbol{bigtriangleup} \savesymbol{bigtriangledown}
  % \RequirePackage{stmaryrd}
   \restoresymbol{ST}{lightning}
   \restoresymbol{ST}{bigtriangleup} \restoresymbol{ST}{bigtriangledown}
  }
  {} 
\newif\ifXPFEIL
\newcommand\XPFEIL{\pkgname{extpfeil}}
\IfStyFileExists{extpfeil}
  {\XPFEILtrue
   % extpfeil tries to do a \RequirePackage of stmaryrd with
   % conflicting options from what we used to load stmaryd.  We
   % therefore temporarily make \RequirePackage a no-op to prevent LaTeX
   % from complaining.
   \let\origRequirePackage=\RequirePackage
   \renewcommand*{\RequirePackage}[2][]{}
   \savesymbol{xlongequal}
   \savesymbol{xmapsto}
   \RequirePackage{extpfeil}
   \restoresymbol{XPFEIL}{xlongequal}
   \restoresymbol{XPFEIL}{xmapsto}
   \let\RequirePackage=\origRequirePackage
  }
  {}
%    \end{macrocode}
%   
% \section{euscript} 
%  
%  For calligraphic math fonts we load the package \pkgname{euscript}. 
%
%  The expected normal use of the Euler Script alphabet is as a substitute
%  for the Computer Modern calligraphic alphabet found in |cmsy|. Therefore we
%  change the meaning of \cmd{\mathcal}. The package uses the Euler script alphabet found in |cmy|
%  and changes the meaning of \cmd{\mathcal} \seedocs{euscript}
%
% |\[ \mathcal{A} = \EuScript{A} \neq \CMcal{A} \] |
% 
%    \begin{macrocode}
\iffalse
\newif\ifEU
\IfStyFileExists{euscript}
  {\EUtrue\RequirePackage[mathcal]{euscript}
   \renewcommand{\mathcal}[1]{\mbox{\usefont{U}{eus}{m}{n}##1}}
  }
  {\let\CMcal\mathcal}
\fi
%    \end{macrocode}
%
% \section{Blackboard fonts}
% \subsection{The bm and bbm fonts}
% \begin{macro}{bm}
% \begin{macro}{bbm}
% These two packages provide bold math fonts. If we have the bm package, use it (to show how to typeset bold math).
% ^^A\mathbbmtt{\ALPHABET} 
% ^^AThe characters can be also be used for subscripts and superscripts.
% ^^A$M_{\mathbbm{i}}$. The package is the work of \person{Torsten}{Hilbrich}
%
%    \begin{macrocode}

\newif\ifBM
\IfStyFileExists{bm}
  {\BMtrue
   \RequirePackage{bm}
  }
  {}  
\IfStyFileExists{bbm}
  {\newcommand{\mathbbm}[1]{\mbox{\usefont{U}{bbm}{m}{n}##1}}
   \newcommand{\mathbbmss}[1]{\mbox{\usefont{U}{bbmss}{m}{n}##1}}
   \newcommand{\mathbbmtt}[1]{\mbox{\usefont{U}{bbmtt}{m}{n}##1}}}
  {}
%    \end{macrocode}
%
%    \begin{macrocode}
\IfStyFileExists{bbold}
  {\newcommand{\BBmathbb}[1]{\mbox{\usefont{U}{bbold}{m}{n}##1}}
   % We have to manually define all of the symbols we care about.
   \newcommand{\BBsym}[1]{\ensuremath{\BBmathbb{\char##1}}}
   \newcommand{\Langle}{\BBsym{`<}}
   \newcommand{\Lbrack}{\BBsym{`[}}
   \newcommand{\Lparen}{\BBsym{`(}}
   \newcommand{\bbalpha}{\BBsym{"0B}}
   \newcommand{\bbbeta}{\BBsym{"0C}}
   \newcommand{\bbgamma}{\BBsym{"0D}}
   \newcommand{\Rparen}{\BBsym{`)}}
   \newcommand{\Rbrack}{\BBsym{`]}}
   \newcommand{\Rangle}{\BBsym{"3E}}
  }
  {}
%  The font calligra provides a calligraphic font. Calligra can be found on the CTAN
%  in the directory tex-archive/fonts/calligra. This package provides means to
%  use this font in \latexe \citep{calligra}.\idxfont{Calligra}
%  idxfont{Calligra (font)>\string\textcalligra}
%
\IfStyFileExists{calligra}
  {\savesymbol{filename}
   \RequirePackage{calligra}
   \restoresymbol{CAL}{filename}
  }
  {}
%    \end{macrocode}
%
%
% \section{Chancery}
%    \begin{macrocode}
\IfStyFileExists{chancery}
  {\newcommand{\mathpzc}[1]{\mbox{\usefont{OT1}{pzc}{m}{it}##1}}}
  {}
%    \end{macrocode}
%
% \section{mbboard}
%
%  
%    \begin{macrocode}
\IfStyFileExists{mbboard}
  {\newcommand{\MBBmathbb}[1]{\mbox{\usefont{OT1}{mbb}{m}{n}##1}}}
  {}
\ifx\MBBmathbb\undefined
\else
  % Define only the symbols we actually use.
  \newcommand{\bbnabla}{\MBBmathbb{\char"9A}}
  \newcommand{\bbdollar}{\MBBmathbb{\char"24}}
  \newcommand{\bbeuro}{\MBBmathbb{\char"FB}}
  \newcommand{\bbpe}{\MBBmathbb{\char"D4}}
  \newcommand{\bbqof}{\MBBmathbb{\char"D7}}
  \newcommand{\bbyod}{\MBBmathbb{\char"C9}}
  \newcommand{\bbfinalnun}{\MBBmathbb{\char"CF}}

  % The following was copied from mbboard.sty.
  \DeclareFontFamily{OT1}{mbb}{\hyphenchar\font45}
  \DeclareFontShape{OT1}{mbb}{m}{n}{
        <5> <6> <7> <8> <9> <10> gen * mbb
        <10.95> mbb10 <12> <14.4> mbb12 <17.28> <20.74> <24.88> mbb17
        }{}
\fi

% \mathfrak is defined by a number of packages, to check for it by name.
\ifx\mathfrak\undefined
\else
  \renewcommand{\mathfrak}[1]{\mbox{\fontencoding{U}\fontfamily{euf}\selectfont#1}}
\fi
%    \end{macrocode}
% \end{macro}
% \end{macro}
%
%
% \begin{macro}{upgreek} 
%
% This package by Walter Schmidt provides fonts
% and commands for an upright Greek alphabet. It makes the upright
% Greek characters from the `Euler'  or `Adobe Symbol' typefaces available as 
% math symbols. It defaults to the Euler option. The package offers three
% options |Euler|, |Symbol| and |Symbolsmallscale|. This is in a bundle
% called |was|, so there are problems downloading it automatically via MikTeX.
% CHECK
% 
%    \begin{macrocode}
\newif\ifUPGR
    \RequirePackage[Symbol]{upgreek}
%    \end{macrocode}
% \end{macro}
%
% \[
%  \begin{array}{lll}
%   \upalpha  &\upbeta    &\upgamma\\ 
%   \updelta  &\upepsilon &\upzeta\\
%   \upeta    &\uptheta   &\upiota \\
%   \upkappa  &\uplambda   &\upmu\\
%   \upnu     &\upxi      &\uppi\\
%   \uprho    &\upsigma  &\uptau\\
%   \upupsilon &\upphi   &\upchi\\
%   \uppsi     &\upomega  &\upvarepsilon\\
%   \upvartheta &\upvarpi  &\\
%  \end{array}
% \]
%
% 
% 
%  \section{Special Symbols}
%
%  The next section of the package, deals exclusively for packages that
%  handle symbols. The best guide to such symbols is 
%  \textit{The Comprehensive LaTeX Symbols Guide}. One needs to distinguish
%  a number of different types of symbols required for a manual and it is
%  a difficult exercise to make a selection. Another issue with symbols
%  is that there is a bit of overlap between the various fonts and commands
%  as to be expected.
%
%  \subsection{ASCII}
%
%  The \pkgname{ascii} will typeset a document in typewriter
%  font. We only need some of its commands to print
%  the ASCI table from 1-32. Can you imagine conflicting with 
%  siunix!!!
%    \begin{macrocode}
\let\oldSI\SI
\let\SI\undefined
\newif\ifASCII
\newcommand\ASCII{\pkgname{ascii}}
\IfStyFileExists{ascii}
	  {\ASCIItrue
	   \savesymbol{HT}
	   \RequirePackage{ascii}
	   \restoresymbol{ascii}{HT}
	   \let\SI\undefined
	  }
	  {}
\let\SI\oldSI	  
%    \end{macrocode}
%
%  \subsection{The china2e package}
%
%  The \pkgname{china2e} is a fairly old package, but can provide some
%  useful commands. It also provides helpful Chinese lunar symbols, although
%  now with specialized Chinese packages, these is fairly redundant for any
%  major use.
%
%  Of interest is some useful maths commands. \cmd{\Natural} \cmd{\NATURAL}
%  {\huge\color{thered}\Fire} \seedocs{china}.
% 
%    \begin{macrocode}
\newif\ifCHINA
\newcommand\CHINA{%
  \Chinasym
  \index{china2e=\textsf{china2e} (package)}%
  \index{packages>china2e=\textsf{china2e}}}
%  
\IfStyFileExists{china2e}
  {\CHINAtrue
   \savesymbol{Info}
   \savesymbol{Earth}
   \savesymbol{Telephone}
   \savesymbol{Fire}
   \savesymbol{vdots}
   \let\origDeclareSymbolFont=\DeclareSymbolFont
   \let\origDeclareMathSymbol=\DeclareMathSymbol
   \renewcommand{\DeclareSymbolFont}[5]{}
   \renewcommand{\DeclareMathSymbol}[4]{%
     \DeclareRobustCommand{##1}{{\uchr##4}}}
   \usepackage{china2e}
   \let\DeclareSymbolFont=\origDeclareSymbolFont
   \let\DeclareMathSymbol=\origDeclareMathSymbol
   \restoresymbol{china}{Info}
   \restoresymbol{china}{Earth}
   \restoresymbol{china}{Telephone}
   \restoresymbol{china}{Fire}
   \restoresymbol{CHINA}{vdots}
  }
  {}
%    \end{macrocode}
%
%  \subsection{harpoon}
%  This package is quite old 1994 by Tobias Kuipers.
%  \overleftharp{This is some text}, 
%  \overrightharp {Other text}. I do not know if any ever uses it.
%
%    \begin{macrocode}
\newif\ifHARP
\newcommand\HARP{\pkgname{harpoon}}
\IfStyFileExists{harpoon}
  {\HARPtrue\usepackage{harpoon}}
  {}

%    \end{macrocode}
%
% \section{texcomp and mathcomp}
%
% We use the \pkg{texcomp} package for special symbols, such as |\checkmark|
%  \( \mho \Diamond \leadsto \rhd \diamond \Diamond \). The sort of the standard package
% latexsym is not loaded as it duplicates functionality of the if one makes use of the packages |amsfonts| or |amssymb|.
%
% \begin{macro}{textcomp} 
% \begin{macro}{mathcomp} The package |textcomp| is part of the \LaTeXe
% distribution. The description of the package
% on ctan can give the erroneous idea that it is obsolete; on the contrary 
% is part of the original distribution.textcomp is not obsolete, it's just not distributed as extra package any more since it's distributed with the basic LaTeX distribution. The \pkg{mathcomp} package defines macros for using some of these text... symbols with math and the abbreviation tc...
%
%  The symbols are used by calling them by their name. E.g. \ifxetex\else\textleaf\fi:
%  \verb|\textleaf|.
%  
%  In mathematics the package \verb|mathcomp| works. Then the prefix
%  \verb|text| is replaced by \verb|tc|, for \emph{t}ext\emph{c}omp:
%  |tcohm|  
% 
% The |mathcomp| package takes one option to describe the
% font to be used. We use |rmdfault| as the option to default to
% the \cs{rmdefault} font.
% 
%    \begin{macrocode}
%  Redefine the LaTeX commands that are replaced by textcomp.
%  This was swiped right out of ltoutenc.dtx, but with "\text..."
%  changed to "\ltext...". This also conflicts with fontspec
%  better to handle the errors 
\DeclareTextCommandDefault{\ltextcopyright}{\textcircled{c}}
\DeclareTextCommandDefault{\ltextregistered}{\textcircled{\scshape r}}
\DeclareTextCommandDefault{\ltexttrademark}{\textsuperscript{TM}}
\DeclareTextCommandDefault{\ltextordfeminine}{\textsuperscript{a}}
\DeclareTextCommandDefault{\ltextordmasculine}{\textsuperscript{o}}
%
\DeclareTextSymbol{\textcentoldstyle}{TS1}{'213}
\DeclareTextSymbolDefault{\textcentoldstyle}{TS1}
\DeclareTextSymbol{\textdollaroldstyle}{TS1}{'212}
\DeclareTextSymbolDefault{\textdollaroldstyle}{TS1}
\DeclareTextSymbol{\textguarani}{TS1}{'220}
\DeclareTextSymbolDefault{\textguarani}{TS1}
% Not many fonts support these code-points yet.
% So leave these undefined at present.  from fontspec

\def\UTFDeclarations{%
  \DeclareUTFcharacter[\UTFencname]{x3008}{\textlangle}
  \DeclareUTFcharacter[\UTFencname]{x3009}{\textrangle}
  \DeclareUTFcharacter[\UTFencname]{x301A}{\textlbrackdbl}
  \DeclareUTFcharacter[\UTFencname]{x301B}{\textrbrackdbl}

% old-style numbers

  \DeclareUTFcharacter[\UTFencname]{xFF10}{\textzerooldstyle}
  \DeclareUTFcharacter[\UTFencname]{xFF11}{\textoneoldstyle}
  \DeclareUTFcharacter[\UTFencname]{xFF12}{\texttwooldstyle}
  \DeclareUTFcharacter[\UTFencname]{xFF13}{\textthreeoldstyle}
  \DeclareUTFcharacter[\UTFencname]{xFF14}{\textfouroldstyle}
  \DeclareUTFcharacter[\UTFencname]{xFF15}{\textfiveoldstyle}
  \DeclareUTFcharacter[\UTFencname]{xFF16}{\textsixoldstyle}
  \DeclareUTFcharacter[\UTFencname]{xFF17}{\textsevenoldstyle}
  \DeclareUTFcharacter[\UTFencname]{xFF18}{\texteightoldstyle}
  \DeclareUTFcharacter[\UTFencname]{xFF19}{\textnineoldstyle}

% For circled letters and small numbers
%

  \DeclareEncodedCompositeCharacter{\UTFencname}{\textcircled}{20DD}{25EF}
  \DeclareUTFcomposite[\UTFencname]{x2460}{\textcircled}{1}
  \DeclareUTFcomposite[\UTFencname]{x2461}{\textcircled}{2}
  \DeclareUTFcomposite[\UTFencname]{x2462}{\textcircled}{3}
  \DeclareUTFcomposite[\UTFencname]{x2463}{\textcircled}{4}
  \DeclareUTFcomposite[\UTFencname]{x2464}{\textcircled}{5}
  \DeclareUTFcomposite[\UTFencname]{x2465}{\textcircled}{6}
  \DeclareUTFcomposite[\UTFencname]{x2466}{\textcircled}{7}
  \DeclareUTFcomposite[\UTFencname]{x2467}{\textcircled}{8}
  \DeclareUTFcomposite[\UTFencname]{x2468}{\textcircled}{9}
  \DeclareUTFcomposite[\UTFencname]{x2469}{\textcircled}{10}
  \DeclareUTFcomposite[\UTFencname]{x246A}{\textcircled}{11}
  \DeclareUTFcomposite[\UTFencname]{x246B}{\textcircled}{12}
  \DeclareUTFcomposite[\UTFencname]{x246C}{\textcircled}{13}
  \DeclareUTFcomposite[\UTFencname]{x246D}{\textcircled}{14}
  \DeclareUTFcomposite[\UTFencname]{x246E}{\textcircled}{15}
  \DeclareUTFcomposite[\UTFencname]{x246F}{\textcircled}{16}
  \DeclareUTFcomposite[\UTFencname]{x2470}{\textcircled}{17}
  \DeclareUTFcomposite[\UTFencname]{x2471}{\textcircled}{18}
  \DeclareUTFcomposite[\UTFencname]{x2472}{\textcircled}{19}
  \DeclareUTFcomposite[\UTFencname]{x2473}{\textcircled}{20}
  \DeclareUTFcomposite[\UTFencname]{x24B6}{\textcircled}{A}
  \DeclareUTFcomposite[\UTFencname]{x24B7}{\textcircled}{B}
  \DeclareUTFcomposite[\UTFencname]{x24B8}{\textcircled}{C}
  \DeclareUTFcomposite[\UTFencname]{x24B9}{\textcircled}{D}
  \DeclareUTFcomposite[\UTFencname]{x24BA}{\textcircled}{E}
  \DeclareUTFcomposite[\UTFencname]{x24BB}{\textcircled}{F}
  \DeclareUTFcomposite[\UTFencname]{x24BC}{\textcircled}{G}
  \DeclareUTFcomposite[\UTFencname]{x24BD}{\textcircled}{H}
  \DeclareUTFcomposite[\UTFencname]{x24BE}{\textcircled}{I}
  \DeclareUTFcomposite[\UTFencname]{x24BF}{\textcircled}{J}
  \DeclareUTFcomposite[\UTFencname]{x24C0}{\textcircled}{K}
  \DeclareUTFcomposite[\UTFencname]{x24C1}{\textcircled}{L}
  \DeclareUTFcomposite[\UTFencname]{x24C2}{\textcircled}{M}
  \DeclareUTFcomposite[\UTFencname]{x24C3}{\textcircled}{N}
  \DeclareUTFcomposite[\UTFencname]{x24C4}{\textcircled}{O}
  \DeclareUTFcomposite[\UTFencname]{x24C5}{\textcircled}{P}
  \DeclareUTFcomposite[\UTFencname]{x24C6}{\textcircled}{Q}
  \DeclareUTFcomposite[\UTFencname]{x24C7}{\textcircled}{R}
  \DeclareUTFcomposite[\UTFencname]{x24C8}{\textcircled}{S}
  \DeclareUTFcomposite[\UTFencname]{x24C9}{\textcircled}{T}
  \DeclareUTFcomposite[\UTFencname]{x24CA}{\textcircled}{U}
  \DeclareUTFcomposite[\UTFencname]{x24CB}{\textcircled}{V}
  \DeclareUTFcomposite[\UTFencname]{x24CC}{\textcircled}{W}
  \DeclareUTFcomposite[\UTFencname]{x24CD}{\textcircled}{X}
  \DeclareUTFcomposite[\UTFencname]{x24CE}{\textcircled}{Y}
  \DeclareUTFcomposite[\UTFencname]{x24CF}{\textcircled}{Z}
  \DeclareUTFcomposite[\UTFencname]{x24D0}{\textcircled}{a}
  \DeclareUTFcomposite[\UTFencname]{x24D1}{\textcircled}{b}
  \DeclareUTFcomposite[\UTFencname]{x24D2}{\textcircled}{c}
  \DeclareUTFcomposite[\UTFencname]{x24D3}{\textcircled}{d}
  \DeclareUTFcomposite[\UTFencname]{x24D4}{\textcircled}{e}
  \DeclareUTFcomposite[\UTFencname]{x24D5}{\textcircled}{f}
  \DeclareUTFcomposite[\UTFencname]{x24D6}{\textcircled}{g}
  \DeclareUTFcomposite[\UTFencname]{x24D7}{\textcircled}{h}
  \DeclareUTFcomposite[\UTFencname]{x24D8}{\textcircled}{i}
  \DeclareUTFcomposite[\UTFencname]{x24D9}{\textcircled}{j}
  \DeclareUTFcomposite[\UTFencname]{x24DA}{\textcircled}{k}
  \DeclareUTFcomposite[\UTFencname]{x24DB}{\textcircled}{l}
  \DeclareUTFcomposite[\UTFencname]{x24DC}{\textcircled}{m}
  \DeclareUTFcomposite[\UTFencname]{x24DD}{\textcircled}{n}
  \DeclareUTFcomposite[\UTFencname]{x24DE}{\textcircled}{o}
  \DeclareUTFcomposite[\UTFencname]{x24DF}{\textcircled}{p}
  \DeclareUTFcomposite[\UTFencname]{x24E0}{\textcircled}{q}
  \DeclareUTFcomposite[\UTFencname]{x24E1}{\textcircled}{r}
  \DeclareUTFcomposite[\UTFencname]{x24E2}{\textcircled}{s}
  \DeclareUTFcomposite[\UTFencname]{x24E3}{\textcircled}{t}
  \DeclareUTFcomposite[\UTFencname]{x24E4}{\textcircled}{u}
  \DeclareUTFcomposite[\UTFencname]{x24E5}{\textcircled}{v}
  \DeclareUTFcomposite[\UTFencname]{x24E6}{\textcircled}{w}
  \DeclareUTFcomposite[\UTFencname]{x24E7}{\textcircled}{x}
  \DeclareUTFcomposite[\UTFencname]{x24E8}{\textcircled}{y}
  \DeclareUTFcomposite[\UTFencname]{x24E9}{\textcircled}{z}
  \DeclareUTFcomposite[\UTFencname]{x24EA}{\textcircled}{0}
  \DeclareUTFcharacter[\UTFencname]{x25EF}{\textbigcircle}
  \DeclareUTFcomposite[\UTFencname]{x3251}{\textcircled}{21}
  \DeclareUTFcomposite[\UTFencname]{x3252}{\textcircled}{22}
  \DeclareUTFcomposite[\UTFencname]{x3253}{\textcircled}{23}
  \DeclareUTFcomposite[\UTFencname]{x3254}{\textcircled}{24}
  \DeclareUTFcomposite[\UTFencname]{x3255}{\textcircled}{25}
  \DeclareUTFcomposite[\UTFencname]{x3256}{\textcircled}{26}
  \DeclareUTFcomposite[\UTFencname]{x3257}{\textcircled}{27}
  \DeclareUTFcomposite[\UTFencname]{x3258}{\textcircled}{28}
  \DeclareUTFcomposite[\UTFencname]{x3259}{\textcircled}{29}
  \DeclareUTFcomposite[\UTFencname]{x325A}{\textcircled}{30}
  \DeclareUTFcomposite[\UTFencname]{x325B}{\textcircled}{31}
  \DeclareUTFcomposite[\UTFencname]{x325C}{\textcircled}{32}
  \DeclareUTFcomposite[\UTFencname]{x325D}{\textcircled}{33}
  \DeclareUTFcomposite[\UTFencname]{x325E}{\textcircled}{34}
  \DeclareUTFcomposite[\UTFencname]{x325F}{\textcircled}{35}
  \DeclareUTFcomposite[\UTFencname]{x32B1}{\textcircled}{36}
  \DeclareUTFcomposite[\UTFencname]{x32B2}{\textcircled}{37}
  \DeclareUTFcomposite[\UTFencname]{x32B3}{\textcircled}{38}
  \DeclareUTFcomposite[\UTFencname]{x32B4}{\textcircled}{39}
  \DeclareUTFcomposite[\UTFencname]{x32B5}{\textcircled}{40}
  \DeclareUTFcomposite[\UTFencname]{x32B6}{\textcircled}{41}
  \DeclareUTFcomposite[\UTFencname]{x32B7}{\textcircled}{42}
  \DeclareUTFcomposite[\UTFencname]{x32B8}{\textcircled}{43}
  \DeclareUTFcomposite[\UTFencname]{x32B9}{\textcircled}{44}
  \DeclareUTFcomposite[\UTFencname]{x32BA}{\textcircled}{45}
  \DeclareUTFcomposite[\UTFencname]{x32BB}{\textcircled}{46}
  \DeclareUTFcomposite[\UTFencname]{x32BC}{\textcircled}{47}
  \DeclareUTFcomposite[\UTFencname]{x32BD}{\textcircled}{48}
  \DeclareUTFcomposite[\UTFencname]{x32BE}{\textcircled}{49}
  \DeclareUTFcomposite[\UTFencname]{x32BF}{\textcircled}{50}
}
\ifengine{\UTFDeclarations}{\UTFDeclarations}{}
%
%
\ifxetex\else\ifluatex\else
  \RequirePackage{textcomp}
  \PassOptionsToPackage{mathcomp}{rmdefault}
  \RequirePackage{mathcomp}
  \fi
\fi
%    \end{macrocode}
% \end{macro}
% \end{macro}
%
% \begin{macro}{exscale}
%
% 
% \index{Packages>exscale}
%
%This package implements scaling of the math extension font |cmex|. If this package is used the site needs 
% scaled versions of the font |cmex10| in the sizes 10.95pt, 12pt, 14.4pt, 17.28pt, 20.74pt, and 24.88pt (corresponding 
% to standard magsteps using |\magstephalf|, and |\magstep1| through |\magstep5|). 
% Additionally |cmex| variants for the sizes |7pt| to |9pt| are necessary. These fonts are part of the AMS font pack­age.
%
%    \begin{macrocode}
\ifxetex
    \else
     \ifluatex
     \else
       \RequirePackage{exscale}
       \RequirePackage{relsize}
     \fi
\fi
%    \end{macrocode}
% \end{macro}
%
% An example to scale math using the \pkg{exscale} package. Perhaps for
% using slides etc.
% ^^A\begin{minipage}[c]{1.0\textwidth}%
%^^A \centering\large\[
%^^A\int_{-1}^{+1}\frac{f(x)}{\sqrt{1-x^{2}}}\,\mathrm{d}x\approx\frac{\pi}{n}%
%^^A\sum_{i=1}^{n}f\left(\cos\left(\frac{2i-1}{2n}\right)\right)\]
%^^A\end{minipage}%
%
% \section{textcomp}
%
% \begin{macro}{\tabitem} The \pkgname{textcomp} 
%  provides a nice helper macro for typesetting symbols in normal, bold
%  and italics. I must think of a more semantic name than |tabitem|.
%
%    \begin{macrocode}
\newcommand{\tabitem}[2]{%
  \texttt{\symbol{`\\}#1} & \@nameuse{#1} 
   & \bfseries\@nameuse{#1}& \itshape\@nameuse{#1}
   \ifthenelse{\equal{#2}{}}
    {}
    {& \texttt{\symbol{`\\}#2} & \@nameuse{#2} 
     & \bfseries\@nameuse{#2}
     & \itshape\@nameuse{#2} \\}
}
%    \end{macrocode}
% \end{macro}
%

%\setlength{\LTleft}{0pt}%
%\setlength{\LTright}{0pt}
%\noindent
%\begin{longtable}{%
%    @{}ll@{}l@{}l@{\extracolsep{\fill}}l!{\extracolsep{0pt}}l@{}l@{}l@{}}
%  \multicolumn{4}{c}{\textbf{Symbol}} & 
%  \multicolumn{4}{c}{\textbf{Symbol}} \\ 
%  \midrule
%\endhead
%
%%  \tabitem{textcapitalcompwordmark}{textascendercompwordmark}
%  \tabitem{textquotestraightbase}{textquotestraightdblbase}
%  \tabitem{texttwelveudash}{textthreequartersemdash}
%  \tabitem{textleftarrow}{textrightarrow}
%  \tabitem{textblank}{textdollar}
%  \tabitem{textquotesingle}{textasteriskcentered}
%  \tabitem{textdblhyphen}{textfractionsolidus}
%  \tabitem{textzerooldstyle}{textoneoldstyle}
%  \tabitem{texttwooldstyle}{textthreeoldstyle}
%  \tabitem{textfouroldstyle}{textfiveoldstyle}
%  \tabitem{textsixoldstyle}{textsevenoldstyle}
%  \tabitem{texteightoldstyle}{textnineoldstyle}
%  \tabitem{textlangle}{textminus}
%  \tabitem{textrangle}{textmho}
%  \tabitem{textbigcircle}{textohm}
%  \tabitem{textlbrackdbl}{textrbrackdbl}
%  \tabitem{textuparrow}{textdownarrow}
%  \tabitem{textasciigrave}{textborn}
%  \tabitem{textdivorced}{textdied}
%  \tabitem{textleaf}{textmarried}
%  \tabitem{textmusicalnote}{texttildelow}
%  \tabitem{textdblhyphenchar}{textasciibreve}
%  \tabitem{textasciicaron}{textgravedbl}
%  \tabitem{textacutedbl}{textdagger}
%  \tabitem{textdaggerdbl}{textbardbl}
%  \tabitem{textperthousand}{textbullet}
%  \tabitem{textcelsius}{textdollaroldstyle}
%  \tabitem{textcentoldstyle}{textflorin}
%  \tabitem{textcolonmonetary}{textwon}
%  \tabitem{textnaira}{textguarani}
%  \tabitem{textpeso}{textlira}
%  \tabitem{textrecipe}{textinterrobang}
%  \tabitem{textinterrobangdown}{textdong}
%  \tabitem{texttrademark}{textpertenthousand}
%  \tabitem{textpilcrow}{textbaht}
%  \tabitem{textnumero}{textdiscount}
%  \tabitem{textestimated}{textopenbullet}
%  \tabitem{textservicemark}{textlquill}
%  \tabitem{textrquill}{textcent}
%  \tabitem{textsterling}{textcurrency}
%  \tabitem{textyen}{textbrokenbar}
%  \tabitem{textsection}{textasciidieresis}
%  \tabitem{textcopyright}{textordfeminine}
%  \tabitem{textcopyleft}{textlnot}
%  \tabitem{textcircledP}{textregistered}
%  \tabitem{textasciimacron}{textdegree}
%  \tabitem{textpm}{texttwosuperior}
%  \tabitem{textthreesuperior}{textasciiacute}
%  \tabitem{textmu}{textparagraph}
%  \tabitem{textperiodcentered}{textreferencemark}
%  \tabitem{textonesuperior}{textordmasculine}
%  \tabitem{textsurd}{textonequarter}
%  \tabitem{textonehalf}{textthreequarters}
%  \tabitem{texteuro}{texttimes}
%  \tabitem{textdiv}{}
%
%
%\end{longtable}
%
% \begin{macro}{wasysym} \url{http://tex.stackexchange.com/questions/80053/wasysym-symbols-render-to-something-different}
%    \begin{macrocode}
\newif\ifWASY
\newcommand\WASY{\pkgname{wasysym}}
\IfStyFileExists{wasysym}
  {\WASYtrue
   \savesymbol{lightning}
   \savesymbol{Box}
   \savesymbol{Diamond}
   \savesymbol{clock}
   \RequirePackage{wasysym}
   \restoresymbol{WASY}{lightning}
   \restoresymbol{WASY}{Box}
   \restoresymbol{WASY}{Diamond}
   \restoresymbol{WASY}{clock}
  }
  {}
%    \end{macrocode}
% \end{macro}
%
% \begin{macro}{pifont}
% Using symbol fonts is supported by means of the 
% \Lpack{pifont} package, providing commands for using the Zapf Dingbats font,
% as well as an interface to other families.\footnote{%
% This section was adopted, with minor changes, 
% from \cite{Mittelbach2004}}.
% 
%    \begin{macrocode}
\newif\ifPI
\newcommand\PI{\pkgname{pifont}}
\IfStyFileExists{pifont}
  {\PItrue\RequirePackage{pifont}}
  {}  
%    \end{macrocode}
% \end{macro}
% 
% 
%\begin{table}[bt!]
% \bgroup
% \let\oldding\ding
% \def\ding#1{{\color{teal}\oldding{#1}}}
% 
%  \caption{The characters in the postscript font Zapf Dingbats} 
%  \label{tab:dingbats}
%  \centering
%  
%{\footnotesize
%\begin{tabular}{|rr|rr|rr|rr|rr|rr|rr|rr|}
%\hline
%32 &  \ding{32} & 33 &  \ding{33} & 34 &  \ding{34} & 35 &  \ding{35} & 36 &  \ding{36} & 37 &  \ding{37} & 38 &  \ding{38} & 39 &  \ding{39}  \\ \hline
%40 &  \ding{40} & 41 &  \ding{41} & 42 &  \ding{42} & 43 &  \ding{43} & 44 &  \ding{44} & 45 &  \ding{45} & 46 &  \ding{46} & 47 &  \ding{47}  \\ \hline
%48 &  \ding{48} & 49 &  \ding{49} & 50 &  \ding{50} & 51 &  \ding{51} & 52 &  \ding{52} & 53 &  \ding{53} & 54 &  \ding{54} & 55 &  \ding{55}  \\ \hline
%56 &  \ding{56} & 57 &  \ding{57} & 58 &  \ding{58} & 59 &  \ding{59} & 60 &  \ding{60} & 61 &  \ding{61} & 62 &  \ding{62} & 63 &  \ding{63}  \\ \hline
%64 &  \ding{64} & 65 &  \ding{65} & 66 &  \ding{66} & 67 &  \ding{67} & 68 &  \ding{68} & 69 &  \ding{69} & 70 &  \ding{70} & 71 &  \ding{71}  \\ \hline
%72 &  \ding{72} & 73 &  \ding{73} & 74 &  \ding{74} & 75 &  \ding{75} & 76 &  \ding{76} & 77 &  \ding{77} & 78 &  \ding{78} & 79 &  \ding{79}  \\ \hline
%80 &  \ding{80} & 81 &  \ding{81} & 82 &  \ding{82} & 83 &  \ding{83} & 84 &  \ding{84} & 85 &  \ding{85} & 86 &  \ding{86} & 87 &  \ding{87}  \\ \hline
%88 &  \ding{88} & 89 &  \ding{89} & 90 &  \ding{90} & 91 &  \ding{91} & 92 &  \ding{92} & 93 &  \ding{93} & 94 &  \ding{94} & 95 &  \ding{95}  \\ \hline
%96 &  \ding{96} & 97 &  \ding{97} & 98 &  \ding{98} & 99 &  \ding{99} & 100 &  \ding{100} & 101 &  \ding{101} & 102 &  \ding{102} & 103 &  \ding{103}  \\ \hline
%104 &  \ding{104} & 105 &  \ding{105} & 106 &  \ding{106} & 107 &  \ding{107} & 108 &  \ding{108} & 109 &  \ding{109} & 110 &  \ding{110} & 111 &  \ding{111}  \\ \hline
%112 &  \ding{112} & 113 &  \ding{113} & 114 &  \ding{114} & 115 &  \ding{115} & 116 &  \ding{116} & 117 &  \ding{117} & 118 &  \ding{118} & 119 &  \ding{119}  \\ \hline
%120 &  \ding{120} & 121 &  \ding{121} & 122 &  \ding{122} & 123 &  \ding{123} & 124 &  \ding{124} & 125 &  \ding{125} & 126 &  \ding{126} &     &              \\ \hline
%    &             & 161 &  \ding{161} & 162 &  \ding{162} & 163 &  \ding{163} & 164 &  \ding{164} & 165 &  \ding{165} & 166 &  \ding{166} & 167 &  \ding{167}  \\ \hline
%168 &  \ding{168} & 169 &  \ding{169} & 170 &  \ding{170} & 171 &  \ding{171} & 172 &  \ding{172} & 173 &  \ding{173} & 174 &  \ding{174} & 175 &  \ding{175}  \\ \hline
%176 &  \ding{176} & 177 &  \ding{177} & 178 &  \ding{178} & 179 &  \ding{179} & 180 &  \ding{180} & 181 &  \ding{181} & 182 &  \ding{182} & 183 &  \ding{183}  \\ \hline
%184 &  \ding{184} & 185 &  \ding{185} & 186 &  \ding{186} & 187 &  \ding{187} & 188 &  \ding{188} & 189 &  \ding{189} & 190 &  \ding{190} & 191 &  \ding{191}  \\ \hline
%192 &  \ding{192} & 193 &  \ding{193} & 194 &  \ding{194} & 195 &  \ding{195} & 196 &  \ding{196} & 197 &  \ding{197} & 198 &  \ding{198} & 199 &  \ding{199}  \\ \hline
%200 &  \ding{200} & 201 &  \ding{201} & 202 &  \ding{202} & 203 &  \ding{203} & 204 &  \ding{204} & 205 &  \ding{205} & 206 &  \ding{206} & 207 &  \ding{207}  \\ \hline
%208 &  \ding{208} & 209 &  \ding{209} & 210 &  \ding{210} & 211 &  \ding{211} & 212 &  \ding{212} & 213 &  \ding{213} & 214 &  \ding{214} & 215 &  \ding{215}  \\ \hline
%216 &  \ding{216} & 217 &  \ding{217} & 218 &  \ding{218} & 219 &  \ding{219} & 220 &  \ding{220} & 221 &  \ding{221} & 222 &  \ding{222} & 223 &  \ding{223}  \\ \hline
%224 &  \ding{224} & 225 &  \ding{225} & 226 &  \ding{226} & 227 &  \ding{227} & 228 &  \ding{228} & 229 &  \ding{229} & 230 &  \ding{230} & 231 &  \ding{231}  \\ \hline
%232 &  \ding{232} & 233 &  \ding{233} & 234 &  \ding{234} & 235 &  \ding{235} & 236 &  \ding{236} & 237 &  \ding{237} & 238 &  \ding{238} & 239 &  \ding{239}  \\ \hline
%    &             & 241 &  \ding{241} & 242 &  \ding{242} & 243 &  \ding{243} & 244 &  \ding{244} & 245 &  \ding{245} & 246 &  \ding{246} & 247 &  \ding{247}  \\ \hline
%248 &  \ding{248} & 249 &  \ding{249} & 250 &  \ding{250} & 251 &  \ding{251} & 252 &  \ding{252} & 253 &  \ding{253} & 254 &  \ding{254} &     &              \\ \hline
%\end{tabular}
% \let\ding\oldding
%\egroup
%\par}
% \label{tbl:dingbats}
% \end{table}
%
%
%    
% \begin{macro}{marvosym}
%
% The package \ctan{marvosym} underwent a major rewrite for the 2000/05/01 version, adding
% a large number of new symbols.  If it looks like we have only the
% older version, pretend we don't have it at all. The tables illustrating the available symbols have been extracted from \citep{marvosym}.
%
%    \begin{macrocode}  
\newif\ifMARV
\newcommand\MARV{\pkgname{marvosym}}
\IfStyFileExists*{marvosym}
  {\savesymbol{CheckedBox}
   \RequirePackage{marvosym}[2000/05/01]% Major rewrite at this version.
   \global\MARVtrue
   \@ifundefined{Denarius} % \Denarius is a newer symbol.
     {\global\MARVfalse}
     {}
   \@ifundefined{MVRightarrow}% \Mvrightarrow is an even newer symbol.
     {\global\MARVfalse}
     {}
  }
  {}
%    \end{macrocode}
% \end{macro}
% 
%


%
%
% \begin{macro}{bbding} The package provides an easy-to-use interface to the \texttt{bbding} symbol
%   set developed by \emph{Karel Horak}.  The naming conventions is made
%   close to \emph{Zapf-Dingbat} as it can be found in \texttt{Wordperfect
%   6.0}, however, sometimes shortening the names.
%
%    \begin{macrocode}  
\newif\ifDING
\newcommand\DING{\pkgname{bbding}}
\IfStyFileExists{bbding}
  {\DINGtrue
   \savesymbol{Cross} \savesymbol{Square}
   \RequirePackage{bbding}
   \restoresymbol{ding}{Cross} \restoresymbol{ding}{Square}
  }
  {}     

\newcount\c@lumnsleft
\newcount\t@talcolumns
\newdimen\c@lumnwidth
\newenvironment{commandsInColumns}[1]{%
  \t@talcolumns=#1\advance\t@talcolumns-1\c@lumnsleft=\t@talcolumns%
  \c@lumnwidth=-2em\multiply\c@lumnwidth by \t@talcolumns%
  \advance\c@lumnwidth by\hsize \divide\c@lumnwidth by #1%
  \vskip\z@     % Ensures vertical mode
  \catcode`\^^M=12%
  \hbox\bgroup%
  \st@rtenv%
}
{\ifnum\c@lumnsleft=\t@talcolumns \egroup
 \else \egroup \fi}
%
{\catcode`\^^M=12%
 \gdef\st@rtenv{\@ifnextchar^^M{\dr@pnext\doNextComm@nd}{\doNextComm@nd}}%
 \gdef\setComm@nd#1#2^^M{%
   \hbox to \c@lumnwidth%
     {\hbox to .5cm{#1\hss}\hbox{\expandafter\setn@me\string#1.}\hss}%
   \advance\c@lumnsleft-1%
   \ifnum\c@lumnsleft>0%
     \hskip2em%
   \else%
     \egroup%
     \hbox\bgroup%
     \c@lumnsleft\t@talcolumns%
   \fi%
   \doNextComm@nd%
  }}
\def\dr@pnext#1#2{#1}
\def\doNextComm@nd{\@ifnextchar\end{}{\setComm@nd}}%
\def\setn@me#1#2.{\CSname{#2}}
%
%
\newcommand{\CSname}[1]{\texttt{\protect\bslash#1}}
%    \end{macrocode}
% \end{macro}
%
%    \begin{macrocode}
\newif\ifEUSYM\EUSYMfalse
\newcommand\EUSYM{\pkgname{eurosym}}
\IfStyFileExists{eurosym}
  {\EUSYMtrue
   \savesymbol{EUR}
   \usepackage{eurosym}
   \restoresymbol{MARV}{EUR}
  }
  {}
%
\newif\ifESV\ESVfalse
\newcommand\ESV{\pkgname{esvect}}
%\IfStyFileExists{esvect}
%  {\ESVtrue
%   \usepackage{esvect}
%   \DeclareMathSymbol{\fldra}{\mathrel}{esvector}{'021}
%   \DeclareMathSymbol{\fldrb}{\mathrel}{esvector}{'022}
%   \DeclareMathSymbol{\fldrc}{\mathrel}{esvector}{'023}
%   \DeclareMathSymbol{\fldrd}{\mathrel}{esvector}{'024}
%   \DeclareMathSymbol{\fldre}{\mathrel}{esvector}{'025}
%   \DeclareMathSymbol{\fldrf}{\mathrel}{esvector}{'026}
%   \DeclareMathSymbol{\fldrg}{\mathrel}{esvector}{'027}
%   \DeclareMathSymbol{\fldrh}{\mathrel}{esvector}{'030}
%  }
%  {}
%    \end{macrocode}
%
%
% \begin{figure}[tbp] \small 
% \begin{commandsInColumns}{2}
%   \ScissorRight
%   \ScissorRightBrokenBottom
%   \ScissorRightBrokenTop
%   \ScissorHollowRight
%   \ScissorLeft
%   \ScissorLeftBrokenBottom
%   \ScissorLeftBrokenTop
%   \ScissorHollowLeft
% \end{commandsInColumns}
% \caption{Scissors from the \texttt{bbding} package.}
% \end{figure}
%
% \begin{figure} \small 
% \begin{commandsInColumns}{3}
%   \HandRight
%   \HandRightUp
%   \HandCuffRight
%   \HandCuffRightUp
%   \HandLeft
%   \HandLeftUp
%   \HandCuffLeft
%   \HandCuffLeftUp
%   \HandPencilLeft
% \end{commandsInColumns}
% \caption{Hands}
% \end{figure}
%
% \begin{figure} \small 
% \begin{commandsInColumns}{3}
%   \PencilRight
%   \PencilRightUp
%   \PencilRightDown
%   \PencilLeft
%   \PencilLeftUp
%   \PencilLeftDown
%   \NibRight
%   \NibSolidRight
%   \NibLeft
%   \NibSolidLeft
% \end{commandsInColumns}
% \caption{Writing tools}
% \end{figure}
%
% \begin{figure} \small 
% \begin{commandsInColumns}{3}
%   \XSolid
%   \XSolidBold
%   \XSolidBrush
%   \Plus
%   \PlusOutline
%   \PlusCenterOpen
%   \PlusThinCenterOpen
%   \Cross
%   \CrossOpenShadow
%   \CrossOutline
%   \CrossBoldOutline
%   \CrossClowerTips
%   \CrossMaltese
% \end{commandsInColumns}
% \caption{Crosses, plusses and the like}
% \end{figure}
%
% \begin{figure} \small 
% \begin{commandsInColumns}{3}
%   \DavidStar
%   \DavidStarSolid
%   \JackStar
%   \JackStarBold
%   \FourStar
%   \FourStarOpen
%   \FiveStar
%   \FiveStarLines
%   \FiveStarOpen
%   \FiveStarOpenCircled
%   \FiveStarCenterOpen
%   \FiveStarOpenDotted
%   \FiveStarOutline
%   \FiveStarOutlineHeavy
%   \FiveStarConvex
%   \FiveStarShadow
%   \SixStar
%   \EightStar
%   \EightStarBold
%   \EightStarTaper
%   \EightStarConvex
%   \TwelweStar
%   \SixteenStarLight
%   \Asterisk
%   \AsteriskBold
%   \AsteriskCenterOpen
%   \AsteriskThin
%   \AsteriskThinCenterOpen
%   \AsteriskRoundedEnds
%   \FourAsterisk
%   \EightAsterisk
% \end{commandsInColumns}
% \caption{All kind of stars}
% \end{figure}
%
% \begin{figure} \small 
% \begin{commandsInColumns}{2}
%   \FiveFlowerOpen
%   \FiveFlowerPetal
%   \SixFlowerOpenCenter
%   \SixFlowerRemovedOpenPetal
%   \SixFlowerAlternate
%   \SixFlowerAltPetal
%   \SixFlowerPetalDotted
%   \SixFlowerPetalRemoved
%   \EightFlowerPetalRemoved
%   \EightFlowerPetal
%   \FourClowerOpen
%   \FourClowerSolid
%   \Sparkle
%   \SparkleBold
%   \SnowflakeChevron
%   \SnowflakeChevronBold
%   \Snowflake
% \end{commandsInColumns}
% \caption{Flowers, snowflakes and the like}
% \end{figure}
%
% \begin{figure} \small 
% \begin{commandsInColumns}{2}
%   \CircleSolid
%   \CircleShadow
%   \HalfCircleRight
%   \HalfCircleLeft
%   \Ellipse
%   \EllipseSolid
%   \EllipseShadow
%   \Square
%   \SquareSolid
%   \SquareShadowBottomRight
%   \SquareShadowTopRight
%   \SquareShadowTopLeft
%   \SquareCastShadowBottomRight
%   \SquareCastShadowTopRight
%   \SquareCastShadowTopLeft
%   \TriangleUp
%   \TriangleDown
%   \DiamondSolid
%   \OrnamentDiamondSolid
%   \RectangleThin
%   \Rectangle
%   \RectangleBold
% \end{commandsInColumns}
% \caption{Geometrical Shapes}
% \end{figure}
%
% \begin{figure} \small 
% \begin{commandsInColumns}{3}
%   \Phone
%   \PhoneHandset
%   \Tape
%   \Plane
%   \Envelope
%   \Peace
%   \Checkmark
%   \CheckmarkBold
%   \SunshineOpenCircled
%   \ArrowBoldRightStrobe
%   \ArrowBoldUpRight
%   \ArrowBoldDownRight
%   \ArrowBoldRightShort
%   \ArrowBoldRightCircled
% \end{commandsInColumns}
% \caption{Miscellaneous}
% \end{figure}
% 
% \subsection{Chemistry}

%    \begin{macrocode}
  \RequirePackage{mhchem}
  \RequirePackage{chemfig}
%    \end{macrocode}

% \subsection{The \texttt{manfnt} package.}
% 
% The \TeX{} and metafont manuals use some special symbols not found in
% the normal CM-fonts. The \pkgname{manfnt} provides additional symbols.
% Most of these symbols will be of little use for
% the average author, but some, like the ``Dangerous Bend'' sign may be
% approriate for some textbooks. As the author states, these symbols tend
% to detract the user; I have included them for the sake of the dangerbend
% symbol. The package is currently maintained by Axel Kielhorn.
%
%    \begin{macrocode}
\newif\ifMAN
\newcommand\MAN{\pkgname{manfnt}}
\IfStyFileExists{manfnt}
  {\MANtrue\RequirePackage{manfnt}}
  {}   
%    \end{macrocode}
%
% \begin{figure} \small 
%   \begin{commandsInColumns}{3}
%     \dbend
%     \lhdbend
%     \reversedvideodbend
%   \end{commandsInColumns}
% \caption{Double bend warning signs from the manfnt package.}
% \end{figure}
%
%
% I am not too sure if I should leave the package here for the long
% term or remove it, perhaps make a "bundle" for LaTeX authors. This package I normally use for the fire symbol for hot issues for my Team.
%
% \subsection{ifsym}
% \begin{macro}{ifsym}
%    \begin{macrocode}
\newif\ifIFS
\newcommand\IFS{\pkgname{ifsym}}
\IfStyFileExists{ifsym}
  {\IFStrue
   \savesymbol{Letter} \savesymbol{Square} \savesymbol{Cross} \savesymbol{Sun}
   \savesymbol{TriangleUp} \savesymbol{TriangleDown} \savesymbol{Circle}
   \savesymbol{Lightning}
   \RequirePackage[alpine,clock,electronic,geometry,misc,weather]{ifsym}[2000/04/18]
   \restoresymbol{ifs}{Letter} \restoresymbol{ifs}{Square}
   \restoresymbol{ifs}{Cross} \restoresymbol{ifs}{Sun}
   \restoresymbol{ifs}{TriangleUp} \restoresymbol{ifs}{TriangleDown}
   \restoresymbol{ifs}{Circle} \restoresymbol{ifs}{Lightning}
   \DeclareRobustCommand{\allCubes}{%
     \Cube{1}~%
     \Cube{2}~%
     \Cube{3}~%
     \Cube{4}~%
     \Cube{5}~%
     \Cube{6}%
   }
  }
  {}  
  
%    \end{macrocode}
% \end{macro}
% The |ifsym| package can produce some fancy symbols such as \Cube{1},\Cube{6} etc. a cross \Cross
% a \TriangleUp      {\color{red}\TriangleDown}. The documentation is in postscript?  \PulseHigh \showclock{0}{45} \ifsLightning \lhdbend
%
% \subsection{Weather Symbols}
% \begin{figure}[h] \small 
% \begin{commandsInColumns}{3}
% \Sun
% \HalfSun
% \NoSun
% \Fog
% \ThinFog
% \Rain
% \WeakRain
% \Hail
% \Sleet
% \Snow
% \Lightning
% \Cloud
% \RainCloud
% \WeakRainCloud
% \SunCloud
% \SnowCloud
% \FilledCloud
% \FilledRainCloud
% \FilledWeakRainCloud
% \FilledSunCloud
% \FilledSnowCloud
%\end{commandsInColumns}
% \allCubes
% \caption{ifsym Weather symbols}
% \end{figure}
%
% \begin{figure}[h] \small 
% \begin{commandsInColumns}{3}
% \Telephone
% \SectioningDiamond
% \FilledSectioningDiamond
% \PaperPortrait
% \PaperLandscape
% \Irritant
% \Fire
% \Radiation
% \StrokeOne
% \StrokeTwo
% \StrokeThree
% \StrokeFour
% \StrokeFive
%\end{commandsInColumns}
% {\Huge\color{yellow!60}\Radiation}
% \caption{ifsym misc symbols}
% \end{figure}
%
% \begin{figure}[h] \small 
% \begin{commandsInColumns}{3}
%\Taschenuhr
%\VarTaschenuhr
%\StopWatchStart
%\StopWatchEnd
%\Interval
%\Wecker
%\VarClock
%\end{commandsInColumns}
% 
% \caption{ifsym clock option symbols symbols}
% \end{figure}
%
% \subsection{The \texttt{undertilde} package}      
%    \begin{macrocode}    
\newif\ifUTILD
\newcommand\UTILD{\pkgname{undertilde}}
\IfStyFileExists{undertilde}
  {\UTILDtrue\RequirePackage{undertilde}}
  {}
%    \end{macrocode}
%
%

%    

% 

% 
% \section{Saving files on the fly filecontents}
% We use the \pkg{filecontents} package, to open and write files on disk on the fly.
% See the sample manual as to how to use.
%    \begin{macrocode}latex
\RequirePackage{phdfilecontents}
%    \end{macrocode}   
%    
% \section{Utilities for programming}
% The below packages offer some good utilities that you may find useful, if you are
% going to program and develop additional macros.
% |\strictpagecheck| can be used effectively for a number of situations, where you need to 
% know if you are on an odd or even page.
%    \begin{macrocode}
\RequirePackage{changepage}    
\RequirePackage{keyval}
\usepackage{xkvview}
\RequirePackage{ifmtarg}
\RequirePackage{fp}
%    \end{macrocode}
%
% \subsection{ifthenx}
%
%    \begin{macrocode}
\RequirePackage{ifthenx}
\RequirePackage{xspace}
\RequirePackage{xstring}
% \RequirePackage{cool, coolstr} conflicts to be resolved.
\RequirePackage{multido}
\RequirePackage{etoolbox}
\RequirePackage{parselines}
% for testing in tutorials
%
\def\TRUE{ \meta{true code} }
\def\FALSE{ \meta{false code} }
\def\PASS{\par{\bfseries\textcolor{green!50!blue}{PASS}}\ ~}
\def\FAIL{\par{\bfseries\textcolor{red!70!black}{FAIL}}\ ~}
% upquote needs to be loaded before listings? Must test
% does not seem to matter actually...
%\RequirePackage{upquote}
% \RequirePackage{remreset}
\RequirePackage{calc}
%    \end{macrocode}
%
% \section{Picture packages}
%
% \LaTeXe provides the |picture| environment, which is by now mostly
% outdated. However it is still useful for placing text or other
% objects at absolute positions on a page. We load its successors,
% the package \pkg{pict2e} and the \pkg{picture} for maximum
% flexibility.
%
% We also load a new kid on the block xpicture
%    \begin{macrocode}
% Used in chapter for picture environment. |pict2e| must be used before.
\RequirePackage{pict2e}
\RequirePackage{picture}
%\RequirePackage{xpicture}
\RequirePackage{tikz}
\usetikzlibrary{%
  arrows,%
  calc,%
  fit,%
  patterns,%
  plotmarks,%
  shapes.geometric,%
  shapes.misc,%
  shapes.symbols,%
  shapes.arrows,%
  shapes.callouts,%
  shapes.multipart,%
  shapes.gates.logic.US,%
  shapes.gates.logic.IEC,%
  er,%
  automata,%
  backgrounds,%
  chains,%
  topaths,%
  trees,%
  petri,%
  mindmap,%
  matrix,%
  calendar,%
  folding,%
  fadings,%
  through,%
  positioning,%
  scopes,%
  decorations.fractals,%
  decorations.shapes,%
  decorations.text,%
  decorations.pathmorphing,%
  decorations.pathreplacing,%
  decorations.footprints,%
  decorations.markings,%
  shadows}

\usetikzlibrary{tikzmark}
\RequirePackage{pgfplots}
\pgfplotsset{compat=1.11}
\RequirePackage{pgfplotstable}
%    \end{macrocode}
%
%  The \pkgname{drawstack} can be used to draw stacks and other similar structures. Add it to
%  the list for computer science packags.
%
%    \begin{macrocode}
\RequirePackage{drawstack}
%    \end{macrocode}
%    
% \section{Code Typesetting}
%
% 	A lot of users use LaTeX for computer related code we include all 
%    the necessary code to use the |listings| package. We also provide 
%    some predefined environments for styling.
%
%  \section{Using Hyperref}
%
% 	The \pkgname{hyperref} by Sebastian Rahtz and Heiko Oberdiek \cite{hyperref} is indespensible
%    for producing electronic versions of documents. As it redefines many commands care
%    needs to be taken with certain packages.
%
%    \begin{macrocode}
% in document
\newcommand*{\BeforeHyperrefHook}{%
  \RequirePackage{float}%
  \RequirePackage{newfloat}}
 % \RequirePackage{verse}} TO FIX
\newcommand*{\AfterHyperrefHook}{%
  \RequirePackage{algorithm2e}%
  \RequirePackage{fancyhdr}

  \RequirePackage{datetime} %%scrtime
  \RequirePackage{scrtime}
  \RequirePackage{datenumber}
  \RequirePackage{natbib}
%  \bibliographystyle{cambridgeauthordate}
   \bibliographystyle{abbrvnat}
  \usepackage{bibentry} % needs checking
  %\bibpunct{(}{)}{;}{a}{,}{,}
  %%%%%%%%%%%%%%%%%%%%%%%%%%%%%%%%%%%%%%%%%%%%%%%%%%%%%%%%%%%%%%%
%
% 4. to bring natbib.sty into line with cambridge style
%    and make 'References' the default
%
\@ifpackageloaded{natbib}{%
    \providecommand\refname{References} % internationalize?
    \providecommand\bibname{Bibliography}

\ifAJW@multisty
  \def\NAT@sectionbib{on}% natbib will use \section* headings
\fi

\setlength\bibhang{1em}
\renewenvironment{thebibliography}[1]{%
% \bibsection\parindent \z@\bibpreamble\bibfont\list
  \bibsection\parindent \z@\bibpreamble\bibliosize\list
   {\@biblabel{\arabic{NAT@ctr}}}{\@bibsetup{##1}%changed
    \setcounter{NAT@ctr}{0}}%
    \ifNAT@openbib
      \renewcommand\newblock{\par}
    \else
      \renewcommand\newblock{\hskip .11em \@plus.33em \@minus.07em}%
    \fi
    \sloppy\clubpenalty4000\widowpenalty4000
    \sfcode`\.=1000\relax
    \let\citeN\cite \let\shortcite\cite
    \let\citeasnoun\cite
 }{\def\@noitemerr{%
  \PackageWarning{natbib}
     {Empty `thebibliography' environment}}%
  \endlist\vskip-\lastskip}


\ifAJW@multisty
  \renewcommand\bibsection{\section{\refname %FIX ME FOR PARAMETERS
    %\ifx\@mkboth\@gobbletwo\else\markright{\refname}\fi
    }}%
\else
  \renewcommand\bibsection{\chapter{\refname %FIX ME STAR CHAPTER
    %\@mkboth{\refname}{\refname}
    }}%
\fi}{}
}
%
%    \end{macrocode}
%
%  \section{The hyperref package}
%
%  The \pkgname{hyperref} is an excellent piece of software, but the redefining of a lot
%  of kernel commands needs special treatment, so we will provide hooks for packages
%  to be loaded before and after the hyperref package.
%  
%  We call it with no options, as we will set them a bit later.
%
% \begin{macro}{sethyperref}   
%    \begin{macrocode}
\def\sethyperref{%
  \BeforeHyperrefHook
  \RequirePackage{hyperref}
%% hyperdoc has a problem with tcolorboc documentation
%% macros.
%%\usepackage{hypdoc}
\hypersetup{
  bookmarks,
  raiselinks,
  pageanchor,
  hyperindex=true,
  colorlinks,
  allcolors=theblue, 
  linktocpage,
  hyperfootnotes=true,
  breaklinks=true,
  anchorcolor= theblue,
  filecolor=blue,
  hypertexnames=true, %useguessable names for links
  urlcolor= theblue,
  linkcolor= theblue,
  pdftitle={My Title},
  pdfauthor={Yiannis Lazarides},
  pdfsubject={The phd LaTeX package},
  pdfkeywords={LaTeX package management, document design},
  plainpages=true%do page number anchors as plain Arabic
 }
\AfterHyperrefHook
}
%    \end{macrocode}
% \end{macro}

% \section{Date and Time}
%
%    \begin{macrocode}
\ifluatex 
\newcommand\printtime[5][0]{%
   \luadirect{
      local m =require("i18n.datetime")
      m:printDayTime(#2, #3, #4, #5, #1)
    }%
 }%

\newcommand\datetimetodecimal[4]{%
   \luadirect{
      local m =require("i18n.datetime")
      m:dayTimeToDecimal(#1, #2, #3, #4)
    }%
 }%
   \newcommand\datetimetofractional[2][0]{%
   \luadirect{
      local m =require("i18n.datetime")
      m:dayTimeToFractional(#2,#1)
    }}
    
\fi
\ExplSyntaxOn
 \DeclareDocumentCommand\printtimeinterval{ m m g g }
 {
  #1\textsuperscript{d}%
  #2\textsuperscript{h}%
  \IfNoValueTF {#3} {} {#3\textsuperscript{m}}
  \IfNoValueTF {#4} {} {#4\textsuperscript{s}}
 }
 \let\PrintTimeInterval\printtimeinterval
 \ExplSyntaxOff
%\usepackage{dateiliste}
%    \end{macrocode}
%
% \section{tcolorbox}
% We load \pkgname{tcolorbox} with options theorems, skins, documentation etc
% for internal and external use.
%
% We also provide an interface, between the \pkgname{tcolorbox} documentation
% keys and our own.
% 
% The indexing keys are still to be sorted out with other sections of the
% documentation, but they seem to be working for the moment.
% 
%    \begin{macrocode}
\let\oldcs\cs
\RequirePackage[theorems, skins, documentation,
                breakable,listings]{tcolorbox}
                \tcbset{index format=pgf,
                        index actual={=},
                        index level = {>},
                        index quote = {!},
                        index german settings,
                        }
\let\cs\oldcs                
%    \end{macrocode}                
%
%\cxset {doc command color/.code = \tcbset{color command = #1}}
%\cxset {doc command color= thedoccommandcolor}

%    \begin{macrocode}
\lstdefinelanguage{extras}{morekeywords={%
      poemtitle, poemtoc, versewidth, 
      vin, poemlines,poemtitlefont, 
      ProvidesClass,IfFileExists,
      RequirePackage,ifthenelse,chapter,
      includegraphics, newarray,readarray,of
}}
\lstloadlanguages{[LaTeX]TeX, [primitive]TeX, extras}
%    \end{macrocode}
%
% Note the |gobble=1| option. We use this to make the colorboxes
% with code not to show the `\%` sign in this documentation.
% Ideally you should fork the code below and adapt it to 
% your own needs.
%
% Also note that this is the default that is to be used in
% \pkg{tcolorbox} commands.
% 
%    \begin{macrocode}

   
\newtcolorbox{scriptexample}[2][shavian]{colback=codebackground,
boxrule=0pt,toprule=0pt,colframe=white}

\newtcolorbox{commands}[2][shavian]{colback=codebackground,
boxrule=0pt,toprule=0pt,colframe=white}

\lstset{language={[LaTeX]TeX},
       escapeinside={{(*@}{@*)}}, 
       numbers=left, 
       gobble=0,
       stepnumber=1,numbersep=5pt, 
       numberstyle={\footnotesize\color{gray}},
       firstnumber=last,
       breaklines=false,
       framesep=5pt,
       basicstyle=\small\ttfamily,
       showstringspaces=false,
       stringstyle={\color{orange}\footnotesize},
       commentstyle=\color{black},
       rulecolor=\color{theshade},
       breakatwhitespace=true,
       showspaces=false, 
       xleftmargin=10pt,
       xrightmargin=10pt,
       aboveskip=3pt plus1pt minus1pt, 
       belowskip=7pt plus1pt minus1pt,  
       backgroundcolor=\color{theshade},
}
%    \end{macrocode}
%	
%	
% 	The environment |\begin{TeX}..\end{TeX}| provides a listings environment
% 	for typesetting, either TeX or LaTeX code.
% 	
%    \begin{macrocode}
\lstnewenvironment{teX}[1][]
  {\lstset{language=[LaTeX]TeX}\lstset{%
      breaklines=true,
      framesep=5pt,
      basicstyle=\verbatimfamily,
      showstringspaces=false,
      keywordstyle=\verbatimfamily,
      stringstyle={\color{gray!90}\footnotesize},
	  commentstyle={\color{gray!90}\footnotesize},
	  rulecolor=\color{theshade},
      breakatwhitespace=true,
	  xleftmargin=15pt,
	  xrightmargin=5pt,
	  aboveskip=\medskipamount,
	  belowskip=\medskipamount,
        backgroundcolor=\color{white}, #1
}}
{}


\lstnewenvironment{teXX}[1][]
  {\lstset{language=[LaTeX]TeX}\lstset{%
      breaklines=true,
      framesep=5pt,
      basicstyle=\small\ttfamily,
      showstringspaces=false,
      keywordstyle=\ttfamily\color{blue},
      stringstyle=\color{maroon},
	  commentstyle=\color{black},
	  rulecolor=\color{gray!10},
      breakatwhitespace=true,
	  xleftmargin=0pt,
	  xrightmargin=5pt,
	  aboveskip=\medskipamount,
	  belowskip=\medskipamount,
      backgroundcolor=\color{gray!10}, #1
}}
{}
%    \end{macrocode}

% \begin{macro}{\continuelinenumber} 
% \begin{macro}{\startnumberat} 
%  The macro \cs{continueLineNumber}, provides a command
%  to start the next block of code with the code numbers continuing.
%  This requires the |listings| which is already included.
%  
%    \begin{macrocode}
% Always I forget this so I created some aliases
\newcommand\continuelinenumber{\lstset{firstnumber=last}}
\newcommand\startlineat[1]{\lstset{firstnumber=#1}}
\let\numberlineat\startlineat
\let\startnumberat\numberlineat
%    \end{macrocode}
% \end{macro}
% \end{macro}
%
%    \begin{macrocode}
\newcommand\emphasis[2][black!80]{\lstset{emph={write, writeln,#2},escapeinside={(*@}{@*)},
   emphstyle={\verbatimfont\bfseries\textcolor{#1}}}}%changed to textbf
      
   
\lstnewenvironment{teXXX}[1][]
  {\lstset{language=[LaTeX]TeX}%
    \lstset{%
      emph={cs, use,new,seq,map,inline,eq,gincr,incr,IfNoValueF,if,If,exist,protect,nopar,gset,%
      set,undefine,define,add,gadd,remove,div,%
      round,truncate,max,min,mod,gzero,int,%
      zero,newcount,protected,msg,error,DeclareDocumentCommand},
      emphstyle=\verbatimfont\bfseries\color{black!80},
      firstnumber=last,
      stepnumber=1,
      escapeinside={{(*@}{@*)}},
      breaklines=false,
      framesep=5pt,
      basicstyle= \verbatimfont,
      showstringspaces=false,
      keywordstyle=\color{thegreen},
      stringstyle=\color{black!50},
      commentstyle=\color{black!50},
	  rulecolor=\color{gray!10},
      breakatwhitespace=true,
      showspaces=false,  % shows spacing symbol
	   %xleftmargin=0pt,
	   %xrightmargin=5pt,
	xleftmargin=15pt,
	xrightmargin=5pt,
	 %  aboveskip=0pt, % compact the code looks ugly in type
	  % belowskip=0pt,  % user responsible to insert any skips
	 aboveskip=\medskipamount,
	 belowskip=\medskipamount,
       backgroundcolor=,
       #1
}}
{}

\lstnewenvironment{phdverbatim}[1][]
  {\lstset{language=[LaTeX]TeX}%
    \lstset{%
      emph={cs, use,new,seq,map,inline,eq,gincr,incr,IfNoValueF,if,If,exist,protect,nopar,gset,%
      set,undefine,define,add,gadd,remove,div,%
      round,truncate,max,min,mod,gzero,int,%
      zero,newcount,protected,msg,error,DeclareDocumentCommand},
      emphstyle=\verbatimfont\bfseries\color{black!80},
      numbers=none,
     % stepnumber=1,
      escapeinside={{(*@}{@*)}},
      breaklines=false,
      framesep=5pt,
      basicstyle= \verbatimfont,
      showstringspaces=false,
      keywordstyle=\color{thegreen},
      stringstyle=\color{black!50},
      commentstyle=\color{black!50},
	  rulecolor=\color{gray!10},
      breakatwhitespace=true,
      showspaces=false,  % shows spacing symbol
	  xleftmargin=15pt,
	  xrightmargin=5pt,
	 %  aboveskip=0pt, % compact the code looks ugly in type
	  % belowskip=0pt,  % user responsible to insert any skips
	  aboveskip=\medskipamount,
	  belowskip=\medskipamount,
      backgroundcolor=,
       #1
}}
{}
%    \end{macrocode}
% 
%
%
%    \begin{macrocode}
\lstnewenvironment{lualisting}[1][]
{\lstset{language=[LaTeX]TeX,
  basicstyle           = \ttfamily,
  showstringspaces     = false,
  upquote              = true,
  keywordstyle         =\color{blue},
  commentstyle         =\color{black!50},
  stringstyle          =\color{black!80},
  backgroundcolor      =\color{white},
  xleftmargin          =15pt,
  xrightmargin         =5pt,
  aboveskip            =\medskipamount,
  belowskip	            =\medskipamount,
  #1}}
{}

%    \end{macrocode}
%    
%	

% 
% \subsection{algorithms}
% 
% This package must always be loaded after |hyperref|
%
%    \begin{macrocode} 
\newif\ifALGORITHM
\@ifpackageloaded{hyperref}{%
    %%\RequirePackage{algorithms}
 }
 {\typeout{Algorithm loaded}}
  \RequirePackage{algorithm2e} 
%    \end{macrocode}
%     
% \section{Common packages for structuring documents}
% The structuring commands, should ideally be loaded by the class. In case the class
% does not loaded them. We use the \pkg{multicol}, for multiple columns.
%    \begin{macrocode}
\RequirePackage{multicol}
%\RequirePackage[toc]{multitoc}
%    \end{macrocode} 
%    
% 
%

%
% \section{cancel}  
%  
% The \pkgname{ulem}  redefines emphasis so we rather
% use the cancel package.
% \cmd{\uline}\uline{important} underlined text like important
% \uuline{urgent} double-underlined text like urgent
% \uwave{boat} wavy underline like 
% boat
% \sout{wrong} line struck through word like wrong
% \xout{removed} marked over like removed
% \dashuline{dashing} dashed underline like dashing
% \dotuline{dotty} dotted underline like 
% dotty
% 
% Similar functionality is also offered by the \pkgname{soul}
%
%The following commands are defined for general use:\\[5pt]
%  \indent \begin{tabular}{l@{\quad}l}\hline\noalign{\vskip2pt}
%   |\uline{important}|  & underlined text like \uline{important}\\[1pt]
%   |\uuline{urgent}|    & double-underlined text like  \uuline{urgent}\\[1pt]
%   |\uwave{boat}|       & wavy underline like {\let\ULleaders\cleaders\uwave{boat}}\\[1pt]
%   |\sout{wrong}|       & line struck through word like \sout{wrong}\\[1pt]
%   |\xout{removed}|     & marked over like \xout{removed} \\[1pt]
%   |\dashuline{dashing}|& dashed underline like \dashuline{dashing}\\[1pt]
%   |\dotuline{dotty}|   & dotted underline like \dotuline{dotty}\\[3pt]\hline
%  \end{tabular}\\[6pt]
%   Other similar commands can be defined with relative ease by utilizing the
%   \cs{markoverwith} command provided by ulem.

%    \begin{macrocode}
\newif\ifULEM
\IfStyFileExists{ulem}
{\ULEMtrue\RequirePackage[normalem]{ulem}}
{}
%    \end{macrocode}
% 
%
%This is a nice package for canceling anything in mathmode with a slash, 
%backslash or a \verb+X+. To get
%a horizontal line we can define an additional macro called 
%with an optional argument
%for the line color (requires package \pkg{color}):
%
%$ 
%\slashed{D} \slashed{p} \slashed{k} \slashed{r} \slashed{A}
% \slashed{f}
%\slashed{U} \slashed{\partial}
% $
%    \begin{macrocode}
% If we have slashed.sty, use it.
\newif\ifhaveslashed
\IfStyFileExists*{slashed}
  {\haveslashedtrue\RequirePackage{slashed}}
  {}

\newif\ifhavecancel
\IfStyFileExists*{cancel}
  {\havecanceltrue\RequirePackage{cancel}}
  {}
%    \end{macrocode}
%
%
%It is no problem to redefine the cancel macros to get also colored lines. 
%    \begin{macrocode}
\newcommand\hcancel[2][red]{\setbox0=\hbox{#2}%
	\rlap{\raisebox{.45\ht0}{\textcolor{#1}{\rule{\wd0}{1pt}}}}#2}
%    \end{macrocode}
%A horizontal line for
%single characters is also decribed in section~\vref{sec:Accents}.
%
%\medskip
%\noindent
%\verb+\cancel+: $f(x)=\dfrac{\left(x^2+1\right)\cancel{(x-1)}}{\cancel{(x-1)}(x+1)}$\\[0.5cm]
%\verb+\bcancel+: $\bcancel{3}\qquad\bcancel{1234567}$\\[0.5cm]
%\verb+\xcancel+: $\xcancel{3}\qquad\xcancel{1234567}$\\[0.5cm]
%\verb+\hcancel+: $\hcancel{3}\qquad\hcancel[red]{1234567}$
%
%\bigskip
%\begin{verbatim}
% $f(x)=\dfrac{\left(x^2+1\right)\cancel{(x-1)}}{\cancel{(x-1)}(x+1)}$\\[0.5cm]
% $\bcancel{3}\qquad\bcancel{1234567}$\\[0.5cm]
% $\xcancel{3}\qquad\xcancel{1234567}$\\[0.5cm]
% $\hcancel{3}\qquad\hcancel[red]{1234567}$
%\end{verbatim}
% \section{staves}
%
% This is a peculiar package providing some old Icelandic runes.
% \runictext{\alphabet}
% \staveXXXV \staveVI \runictext{abcdef}
%    \begin{macrocode}
\newif\ifSTAVE
\newcommand\STAVE{\pkgname{staves}}
\IfStyFileExists{staves}
  {\STAVEtrue\usepackage{staves}}
  {}
%    \end{macrocode}

% No point wasting a math alphabet on shuffle.
%    \begin{macrocode}
\newif\ifSHUF
\newcommand\SHUF{\pkgname{shuffle}}
\IfStyFileExists{shuffle}
  {\let\origDeclareSymbolFont=\DeclareSymbolFont
   \let\origDeclareMathSymbol=\DeclareMathSymbol
   \renewcommand{\DeclareSymbolFont}[5]{}
   \renewcommand{\DeclareMathSymbol}[4]{%
     \DeclareRobustCommand{##1}{{\usefont{U}{shuffle}{m}{n}\char##4\relax}}
   }
   \SHUFtrue
   \RequirePackage{shuffle}
   \let\DeclareSymbolFont=\origDeclareSymbolFont
   \let\DeclareMathSymbol=\origDeclareMathSymbol
  }
  {}
%    \end{macrocode}



%    \begin{macrocode}
\RequirePackage{framed}
\RequirePackage{varioref}
\RequirePackage{setspace}
%    \end{macrocode}
%    \begin{macrocode}

\providecommand*\switch[2]{{\fontfamily{#1}\selectfont #2}}
%    \end{macrocode}  

%
% \section{Producing Math Symbols}
% 
% The centernot package  provides \cs{centernot} 
% that prints the symbol \cs{not} on the
% following argument. Unlike \cs{not} the symbol is horizontally centered. The \pkgname{amssymb} and \pkgname{mathbax} provide built-in symbols. The package
%can be used for building other symbols. 
% (\seedocs{centernot}).
%    \begin{macrocode}
\newif\ifhavecenternot
\IfStyFileExists*{centernot}%
  {\havecenternottrue\RequirePackage{centernot}}
  {}
%    \end{macrocode}
%

% and spacing commands 
% which can be handy,
%
% \section{spacing}
% \begin{macro}{\hspace} This is a \textit{hairspace}, here defined 
% as 1pt.
% \begin{macro}{\hquad} This is a half squad space
%    \begin{macrocode}
\newcommand{\hairsp}{\hspace{1pt}}% hair space
\newcommand{\hquad}{\hskip0.5em\relax}% half quad space
% Sometimes, we need a little more horizontal spacing, too (used for symbols).
\newcommand{\qqquad}{\qquad\quad}
\newcommand{\TODO}{\textcolor{red}{\bf TODO!}\xspace}
\newcommand{\ie}{\textit{i.\hairsp{}e}\xspace} %removed\@
\newcommand{\eg}{\textit{e.\hairsp{}g.}\xspace}
\newcommand{\BC}[1]{\textsc{#1 BC}} %European Union Style Guide FIX
\newcommand{\AD}[1]{\textsc{AD #1}} %European Union Style Guide FIX
%    \end{macrocode}
% \end{macro}
% \end{macro}
%
% \subsection{Standard phantom widths}
%
%    \begin{macrocode}
\newcommand\Zi{\phantom{0}} %Z conflicts with symbols 
\newcommand\ZZ{\phantom{00}}
\newcommand\ZZZ{\phantom{000}}
\newcommand\ZZZZ{\phantom{0000}}
\providecommand\newthought[1]{%
   \addvspace{1.0\baselineskip plus 0.5ex minus 0.2ex}%
   \noindent\textsc{#1}%
}
%    \end{macrocode}
%
%  \let\equation\gather             %% See tabu and hyperref docs
%  \let\endequation\endgather
%
%
% \section{Logos and other common elements}
%
% Here we define some of the most commonly used logos. Different
% authors preferences vary. Some like to type \cmd{\TeX}, others
% myself included prefer all lowercase typing, e.g., \cmd{\tex}
% and others uppercasing the commands. We provide as many variants
% as possible. There are two or three packages providing logos. In
% the end we provide our own.
%
%    \begin{macrocode}
\newcommand{\seedocs}[1]{%
  See the #1 documentation for more information%
}
%    \end{macrocode}
% 
% \subsection{hologo}
% If you intend to have any fancy logos in bookmarks then the
% \pkgname{hologo} can be used.
% The package starts a collection of logos with 
% support for bookmarks strings. \seedocs{hologo}.
%    \begin{macrocode}
\RequirePackage{hologo}
%    \end{macrocode}
%
% \subsection{metalogo}
% 
% The package \pkgname{metalogo} exposes the spacing parameters for 
% the various TEX logos to the end
% user (and suitably redefines the logos in a generalised way). It is intended to help
% XeLaTeX users, who use various typefaces, to easily optimise the logos for each
% typeface. Still, the package remains useful if any typeface is used, not necessarily
% loaded through XeTEX. It is known that, in Plain TEX’s definition of \TeX, the
% lower right serif on the ‘E’ protrudes through the ‘X’ in cmr and cmr; this
% package can be used to fix this sort of unacceptable grotesque.
%
%    \begin{macrocode}
\RequirePackage{metalogo}
\newcommand\TEX      {\TeX\xspace}
\let\tex\TEX
\newcommand\LUA      {Lua\xspace}
\let\lua\LUA
\newcommand\PDFTEX   {pdf\TeX\xspace}
\let\pdftex\PDFTEX
\newcommand\LUATEX   {Lua\TeX\xspace}
\let\luatex\LUATEX
\newcommand\XETEX    {\XeTeX\xspace}
\let\xetex\XETEX
\newcommand\LATEX    {\LaTeX\xspace}
\let\latex\LATEX
\newcommand\pdfLaTeX {pdf\latex}
\newcommand\LUALATEX {Lua\LaTeX\xspace}
\let\lualatex\LUALATEX
\newcommand\CONTEXT  {Con\TeX t\xspace}
\let\context\CONTEXT
\newcommand\OpenType {\texttt{Open\kern-.25ex Type}\xspace}
\let\opentype\OpenType
\def\latexe{\LaTeX\xspace}
\def\bibtex{\texttt{bibTeX\xspace}}
\newcommand{\fontdefdtx}{fontdef.dtx\xspace}
\newcommand{\postscript}{PostScript\index{PostScript}\xspace}
\newcommand{\TC}{\pkgname{textcomp}}
%\newcommand\TX{\pkgname{txfonts}}
\newcommand\PX{\pkgname{pxfonts}}
\newcommand{\TeXbook}{%
  The \TeX{}book\index{TeXbook, The=\TeX{}book, The}~\cite{Knuth:ct-a}\xspace}
\newcommand{\ctt}{%
  \texttt{comp.text.tex}%
  \index{comp.text.tex=\texttt{comp.text.tex} (newsgroup)}\xspace}
\newcommand{\fntenc}[1][]{%
  \def\firstarg{#1}%
  font encoding%
  \ifx\firstarg\empty%
    \index{font encodings}%
  \else
    \index{font encodings>\firstarg}%
  \fi
}
\DeclareRobustCommand{\xelatexInternal}{%
  \mbox{X\lower0.5ex\hbox{\kern-0.15em\reflectbox{E}}\kern-0.1em\LaTeX}}
  \newcommand{\xelatex}{\xelatexInternal\index{XeLaTeX=\xelatexInternal}\xspace}
  
\DeclareRobustCommand\otr{OTR\xspace}
\let\alltex\LaTeX
\let\doccmd\cmd
%
\def\texbook{\TeX book\xspace}
\def\alltex{(All\kern-.075em)\kern-.075em\TeX\xspace}
\def\ams{American Mathematical Society\xspace}
\def\AmS{$\mathcal{A}$\kern-.1667em\lower.5ex\hbox
    {$\mathcal{M}$}\kern-.125em$\mathcal{S}$\xspace}
\def\amsmath{\AmS{}math\xspace}
\def\amslatex{\AmS-\LaTeX\xspace}
\def\amstex{\AmS-\TeX\xspace}
%
\def\docpkg#1{\texttt{#1}}
%    \end{macrocode}
%
% The package \pkgname{scalefnt} should not be used, with XeLaTeX or LuaTeX.
% It might have some uses with older schemes.
%    \begin{macrocode}
\ifengine{}{}{\RequirePackage{scalefnt}	}
%    \end{macrocode}
%
%
% \section{Miscellaneous Packages}
%
%
% We include here everything that does not fit into the other categories.
% 
%    \begin{macrocode} 
\RequirePackage{chngcntr}
\RequirePackage{multienum}
\RequirePackage{fourier-orns}
%    \end{macrocode}
%
% \subsection{eso-pic}
% Since we loading pgf, many of the things that eso-pic does can be taken over by |pgf|. I am not too sure
% if we should leave this in the long-term.
%
%    \begin{macrocode}
\RequirePackage{eso-pic}
%\RequirePackage{layouts}
%    \end{macrocode}
%
% The package \pkgname{aplhalph} provides alphabetical numbering. 
%
%   \begin{macrocode}
\RequirePackage{alphalph}
\RequirePackage{fmtcount}
% 
\RequirePackage{varwidth}
\RequirePackage{comment}
\RequirePackage{textcase}
\RequirePackage[autostyle=false]{csquotes}
\RequirePackage{alltt}[1997/06/16]
\RequirePackage{caption} % check
%\RequirePackage{currfile} affects FileInput problematic

%\RequirePackage{filemod}
\RequirePackage{afterpage}
\RequirePackage{environ}
\RequirePackage{mwe}
%    \end{macrocode}
%
% \begin{macro}{pdfpages}
% If you need to insert an existing, possibly multi-page, |PDF| file into your 
% LaTeX document, whether or not the included |PDF| was compiled with LaTeX or 
% another tool, consider using the \pkg{pdfpages} package. We load it with
% the option final.
% 
%    \begin{macrocode}
\RequirePackage[final]{pdfpages}
%    \end{macrocode}
% \end{macro}
%
% Include the pages you want using:
%
%    |\includepdf[pages=3-8]{sample.pdf}|
%
%    \begin{macrocode}
\newif\ifCCLIC
\newcommand\CCLIC{\pkgname{cclicenses}}
\IfStyFileExists{cclicenses}
  {\CCLICtrue
   \RequirePackage{cclicenses}
   % cclicenses doesn't get along with textcomp's remapping of
   % \textcircled to the TS1 font encoding.  Mapping it back doesn't
   % _seem_ to cause any problems.
   \DeclareTextAccentDefault{\textcircled}{OMS}
  }
  {}
%    \end{macrocode}
%  
% \subsection{Ornaments}      
% 
% The \pkgname{fourier} defines a lot of math symbols, but we care about only a few of
% them.  Hence, we load only the fourier-orns package and manually
% define everything else as text-mode symbols.
% 
%    \begin{macrocode} 
\ifxetex\else
\newif\ifFOUR
\newcommand\FOUR{\pkgname{fourier}}
\IfStyFileExists{fourier}
  {\FOURtrue
   \RequirePackage{fourier-orns}
   % Define single-glyph symbols.
   \DeclareFontEncoding{FMS}{}{}
   \DeclareFontSubstitution{FMS}{futm}{m}{n}
   \DeclareFontEncoding{FML}{}{}
   \DeclareFontSubstitution{FML}{futmi}{m}{it}
   \newcommand{\fourierdef}[6]{%
     \DeclareRobustCommand{##1}{{\usefont{##2}{##3}{##4}{##5}\char##6}}}
   \fourierdef{\parallelslant}{FMS}{futm}{m}{n}{134}
   \fourierdef{\nparallelslant}{FMS}{futm}{m}{n}{143}
   \fourierdef{\FOURrho}{FML}{futmi}{m}{it}{26}
   \fourierdef{\FOURvarrho}{FML}{futmi}{m}{it}{37}
   \fourierdef{\varvarrho}{FML}{futmi}{m}{it}{129}
   \fourierdef{\FOURpi}{FML}{futmi}{m}{it}{25}
   \fourierdef{\FOURvarpi}{FML}{futmi}{m}{it}{36}
   \fourierdef{\varvarpi}{FML}{futmi}{m}{it}{131}
   \fourierdef{\FOURpartial}{FML}{futmi}{m}{it}{64}
   \fourierdef{\varpartialdiff}{FML}{futmi}{m}{it}{130}
   \fourierdef{\FOURtexteuro}{TS1}{futx}{m}{n}{191}
   % Fake a math accent with text-mode commands.
   \DeclareRobustCommand{\FOURfakewidetopaccent}[5]{%
     \setbox0=\hbox{\ensuremath{##1}}%
     \setbox1=\hbox{\ensuremath{abc}}%
     \leavevmode
     \ifdim\wd0<\wd1
       \kern1pt
       \rlap{\raisebox{##2}{\makebox[\wd0]{\usefont{FMX}{futm}{m}{n}\char##3}}}%
       \kern-0.1em
       \box0
     \else
       \rlap{\raisebox{##4}{\makebox[\wd0]{\usefont{FMX}{futm}{m}{n}\char##5}}}%
       \box0
     \fi
   }

   % Manually define Fourier's extensible accents.  Note that we don't
   % bother trying to use Fourier's \mathring to construct the
   % \FOURwidering symbol.
   \DeclareFontEncoding{FMX}{}{}
   \DeclareFontSubstitution{FMX}{futm}{m}{n}
   \DeclareRobustCommand{\FOURwidearc}[1]{%
     \FOURfakewidetopaccent{##1}{0ex}{216}{0.5ex}{217}}
   \DeclareRobustCommand{\FOURwideOarc}[1]{%
     \FOURfakewidetopaccent{##1}{0ex}{228}{0.5ex}{229}}
   \DeclareRobustCommand{\FOURwideparen}[1]{%
     \FOURfakewidetopaccent{##1}{0ex}{148}{0.5ex}{150}}
   \DeclareRobustCommand{\FOURwidering}[1]{\overset{\smash{\vbox to .2ex{%
     \hbox{$\mathring{}$}}}}{\FOURwideparen{##1}}}

   % Manually define Fourier's variable-sized delimiters.
   \newcommand{\fouriercdef}[6]{%
     \DeclareRobustCommand{##1}{%
       \textvcenter{\usefont{##2}{##3}{##4}{##5}\char##6}}}
   \fouriercdef{\FOURtllbracket}{FMX}{futm}{m}{n}{133}
   \fouriercdef{\FOURdllbracket}{FMX}{futm}{m}{n}{139}
   \fouriercdef{\FOURtrrbracket}{FMX}{futm}{m}{n}{134}
   \fouriercdef{\FOURdrrbracket}{FMX}{futm}{m}{n}{140}
   \newcommand*{\FOURverticals}[1]{%
     \vbox{%
       \baselineskip=-\maxdimen
       \lineskiplimit=\maxdimen
       \lineskip=0pt%
       \usefont{FMX}{futm}{m}{n}%
       \ialign{####\cr##1}%
     }%
   }
   \DeclareRobustCommand{\FOURtVERT}{%
     \raisebox{0.5ex}{\textvcenter{\FOURverticals{\char147\cr\char147\cr}}}}
   \DeclareRobustCommand{\FOURdVERT}{%
     \raisebox{0.5ex}{\textvcenter{\FOURverticals{\char147\cr\char147\cr\char147\cr\char147\cr}}}}
  }
  {}
\fi
%    \end{macrocode} 
%
% \begin{macro}{dirtree}
% The \pkgname{dirtree} provides commands to draw directory-like charts.|forest| is a much better
% alternative.
%    \begin{macrocode}
\RequirePackage{dirtree}
%    \end{macrocode}
% \end{macro}
%    
% \section{Archaic Symbols}     
%
% These packages are included here, only because I have an interest in
% them in some documents I have. I understand that for the average user
% they might not be of interest. We conditionally load them based on
% a conditional and also to develop the concept of `bundles' which  I
% explain a bit later on.
%
% Uncial font
% 
% \subsection{Linear A}
%    \begin{macrocode}
\RequirePackage{uncial}
\newif\ifarchaic
  \archaictrue
\ifarchaic
%    \end{macrocode}


%    \begin{macrocode}  
\newif\ifLINA
\newcommand\LINA{\pkgname{lineara}}
\IfStyFileExists{lineara}
  {\LINAtrue\RequirePackage{lineara}}
  {}

\newif\ifLINB
\newcommand\LINB{\pkgname{linearb}}
\IfStyFileExists{linearb}
  {\LINBtrue\RequirePackage{linearb}}
  {}

\newif\ifCYPR
\newcommand\CYPR{\pkgname{cypriot}}
\IfStyFileExists{cypriot}
  {\CYPRtrue\RequirePackage{cypriot}}
  {}
%    \end{macrocode}
%
%
%    \begin{macrocode}
\newif\ifSARAB
\newcommand\SARAB{\pkgname{sarabian}}
\IfStyFileExists{sarabian}
  {\SARABtrue\RequirePackage{sarabian}}
  {}
%    \end{macrocode}
%
% \subsection{Cuneiform}
%
% Cuneiform .
%    \begin{macrocode}
\newif\ifPRSN
\newcommand\PRSN{\pkgname{oldprsn}}
\IfStyFileExists{oldprsn}
  {\PRSNtrue\RequirePackage{oldprsn}}
  {}

\RequirePackage{hieroglf}
\newif\ifUGAR
\newcommand\UGAR{\pkgname{ugarite}}
\RequirePackage{ugarite}
\IfStyFileExists{ugarite}
  {\UGARtrue\RequirePackage{ugarite}}
  {}
%end archaic   
%    \end{macrocode}
%
% \section{Epi-Olmec}
%
% We load the \pkgname{epiolmec} for typesetting the Epi-Olmec script. This is described
% in the scripts chapters.
%
%    \begin{macrocode}
\newif\ifOLMEC
\newif\ifscriptolmec \scriptolmectrue
\cxset{olmec/.is if=scriptolmec}
\cxset{olmec=true}
% 
\ifscriptolmec
\RequirePackage{epiolmec}
\IfStyFileExists{epiolmec}
  {\OLMECtrue\RequirePackage{epiolmec}}
  {}
\fi
%    \end{macrocode}
%
% \section{Ancient Greek}
%
% \subsection{Philokalia}
%
% We load the \pkgname{philokalia} for typesetting ancient greek using the \idxfont{philokalia} font.
% The package loads the \pkgname{xlextra}, which we do not want. It is loaded by fontspec
% as required.
% If we are using luatex this will issue a warning and abort. Better to fake it for both.
% Also modifies lettrine package !aha this took long!
%    \begin{macrocode}

\newif\ifPHILOKALIA
\def\loadphilokalia{%
  \@namedef{ver@xltxtra.sty}{}% a fake for a "xlextra" package
  \RequirePackage{philokalia}
  \IfStyFileExists{philokalia}
    {\PHILOKALIAtrue\RequirePackage{philokalia}}
    {}
}%
%\ifengine{\loadphilokalia}{\loadphilokalia}{}
%\ifPHILOKALIA
%  \newfontfamily\plk{Philokalia-Regular}
%  \newfontfamily\PHtitl[Script=Greek,RawFeature=+titl;grek]{Philokalia-Regular}
%\fi
%    \end{macrocode}
%
%
%
% \section{Titles, authors, abstracts and the like}
%
% 	We want to have the option to make titles both as normally used in the |book| class
%	but also as used in articles i.e., not to emit a new page after it is invoked.
%	The definition is straight from the article class.
% \begin{macro}{\@maketitle}
%    This macro takes care of formatting the title information when we
%    have no separate title page.
%
%    We always start a new page, leave some white space and center the
%    information. The title is set in a |\LARGE| font, the author
%    names and the date in a |\large| font. CHECK THIS IF HERE
%    \begin{macrocode}
\def\@maketitle{%
  %\newpage
  \null
  \vskip 2em%
  \begin{center}%
  \let \footnote \thanks
    {\LARGE \@title \par}%
    \vskip 1.5em%
    {\large
      \lineskip .5em%
      \begin{tabular}[t]{c}%
        \@author
      \end{tabular}\par}%
    \vskip 1em%
    {\large \@date}%
  \end{center}%
  \par
  \vskip 1.5em}
\fi
%    \end{macrocode}
% \end{macro}
%
% \begin{macro}{\maketitle}
%    The macro to generate titles is easily altered in order that it
%    can be used more than once (an article with many titles)\footnote{Definition is straight 	out of the |doc| package and I only added minor tweaks to only start a new page 
%	on demand.}.  In the
%    original, diverse macros were concealed after use with
%    |\relax|. We must cancel anything that may have been put
%    into |\@thanks|, etc., otherwise {\em all\/} titles will
%    carry forward any earlier such setting!
%                 \cs{@makefnmark} and \cs{@makefntext}.
%    \begin{macrocode}
\def\nonewpage{}
\def\maketitle{\par
      \begingroup \def \thefootnote {\fnsymbol {footnote}}%
      \setcounter {footnote}\z@
      \def\@makefnmark{\hbox to\z@{$\m@th^{\@thefnmark}$\hss}}%
      \long\def\@makefntext##1{\parindent 1em\noindent
            \hbox to1.8em{\hss$\m@th^{\@thefnmark}$}##1}%
      \if@twocolumn \twocolumn [\@maketitle ]%
      \else \nonewpage \global \@topnum \z@ \@maketitle \fi
%    \end{macrocode}
%    For special formatting requirements (such as in TUGboat), we use
%    pagestyle |titlepage| for this; this is later defined to be
%    |plain|, unless already defined, as, for example, by
%    |ltugboat.sty|.
%    \begin{macrocode}
       \thispagestyle{titlepage}\@thanks \endgroup
%    \end{macrocode}
%    If the driver file documents many files, we don't want parts of a
%    title of one to propagate to the next, so we have to cancel
%    these, however before we save in another macro for later
%    usage in headers, if required. :
%    \begin{macrocode}
      \setcounter {footnote}\z@
      \gdef\@date{\today}\gdef\@thanks{}%
      \let\doctitle@cx\@title
      \let\docauthor@cx\@author
%
      \gdef\@author{}\gdef\@title{}%
}
%    \end{macrocode}
% \end{macro}
%
%	As you can see from below, it can now work anywhere. 
% \maketitle
% 
%  Test |\@author| and test |\doctitle@cx| |\docauthor@cx|,
% 
%
% \begin{macro}{\ps@titlepage}
%	 When a number of articles are concatenated into a
%    journal, for example, it is not usual for the title pages of such
%    documents to be formatted differently.  Therefore, a class
%    such as \textsf{ltugboat} can define this macro in advance.
%    However, if no such definition exists, we use pagestyle
%    \texttt{plain} for title pages.Again the definition is 
%	from the \pkg{doc} package.
%    \begin{macrocode}
\@ifundefined{ps@titlepage}
    {\let\ps@titlepage=\ps@plain}{}
%    \end{macrocode}
% \end{macro}
%
% \section{Defining Abstracts, summaries, precis, keywords etc}
%
% \subsection{Abstract}
%
% \begin{environment}{abstract}
%
%	This is an interesting environment provided in the standard
%	classes only for articles. However too many publications 
%	require such abstracts in other sections as well so we redefine
%	it here to make it more extensive.
%	
%    When we are producing a separate titlepage we also put the
%    abstract on a page of its own. It will be centred vertically on
%    the page.
%
%    Note that this environment is not defined for books.
%         to avoid page break after abstract heading.
%    \begin{macrocode}
\def\abstractname{Abstract}
\@ifundefined{abstract}{%
  \newenvironment{abstract}{%
      \titlepage
      \null\vfil
      \@beginparpenalty\@lowpenalty
      \begin{center}%
        \bfseries \abstractname
        \@endparpenalty\@M
      \end{center}}%
     {\par\vfil\null}
%    \end{macrocode}
%
%    When we are not making a separate titlepage --the default for the
%    article document class-- we have to check if we are in twocolumn
%    mode. In that case the abstract is as a |\section*|, otherwise
%    the quotation environment is used to typeset the abstract.
%    \begin{macrocode}
}{}
%    \end{macrocode}
% \end{environment}
%
% \begin{environment}{chapterabstract} This is an identical environment to that
%	provided for abstract and can be used anywhere in the document. 
%    \begin{macrocode}
\def\chapterabstractname{Summary}

\newenvironment{chapterabstract}{%
   \center
     {\bfseries \chapterabstractname\vspace{-.5em}\vspace{\z@}}
   \endcenter\quotation
}{\endquotation}
%    \end{macrocode}
% \end{environment}
%
% \begin{chapterabstract}
%   \lorem
% \end{chapterabstract}
%
% \begin{macro}{chapter abstractname}  We define a key for the summary or
% 	or abstract at the top of a chapter. In most cases it is just called a summary.
%     One can use the \cs{chapterabstractname} to change it to another language.
%    \begin{macrocode} 
\cxset{chapter abstractname/.store in =\chapterabstractname}
\cxset{chapter abstractname= SUMMARY}
%    \end{macrocode}
% \end{macro}
% \begin{macro}{\precis} Precis is a command to be used for summaries. The same summary 
% can also be used for the toc. 
%    \begin{macrocode}
\newcommand\precis[1]{%
     \precis@cx{#1}%
     \precistoc@cx{#1}%
     \addvspace{20pt} % check this should be settable
}
%
\def\precis@cx#1{%
\bgroup
\small
\centering
\parbox{.8\textwidth}{#1}
\par\medskip\egroup}

\def\precistoc@cx#1{%
  \addtocontents{toc}{%
    \protect\bgroup
    \protect\parindent0pt
    \protect\color{preciscolor}#1\par
    \protect\medskip%
 \egroup}
}
%    \end{macrocode}
% \end{macro}
% \begin{macro}{\addtocimage@cx}
%    \begin{macrocode}
\long\def\addtocimage@cx#1#2#3{%
\tikz[remember picture,overlay] {%
\node[anchor=east,xshift=#1,yshift=#2] at (0,0) {\includegraphics[width=.15\linewidth]{#3}};}%
}
\def\addtocimage#1#2#3{%
 \addtocontents{toc}{\protect\addtocimage@cx{#1}{#2}{#3}}%
}
%    \end{macrocode}
% \end{macro}

%
% \begin{chapterabstract}
%   \lorem
% \end{chapterabstract}
%
% \subsection{Limitations of the Lamport's approach and some alternatives}
%
%	When Lamport \textit{et al.} incorporated quotations as a means to
%	defining the \texttt{abstract} commands there was a need to conserve 
%	memory and computing type. Utilizing these commands as I have shown 
%	is a bit of a kludge and perhaps not semantically correct. Some
%	books have quotations and quotes that do not exactly fit to such styling
%	so changing the layout of a quotation can potentially break oher
%	parts. We discuss these issues further in the chapter 
%	\textit{The Special Environments Quotation and Quote} on
%	 \pageref{quotations} and we illustrate it 
%    with \fref{frightquotation}.
%
% \section{Quotations}
% 
%    \begin{macrocode}
\cxset{
  quotation above/.store in=\quotationabove@cx,
  quotation left margin/.store in=\quotationleftmargin@cx,
  quotation right margin/.store in=\quotationrightmargin@cx,
  quotation parsep/.store in=\quotationparsep@cx,
  quotation font-size/.store in=\quotationfontsize@cx,
  quotation parindent/.store in=\quotationparindent@cx,
  quotation font-name/.store in=\quotationfontname@cx,
 }
%    \end{macrocode}
%
% \begin{macro}{\setquotation} Macro to create the quotation
%	environment. We need to think of a better way here. Saved
%   old environment.
%    \begin{macrocode}
\let\latexquotation\quotation
\let\endlatexquotation\endquotation
\def\setquotation#1{%
\cxset{#1}
\renewenvironment{quotation}
               {\par\addvspace{\quotationabove@cx}
                \list{}{\listparindent\quotationparindent@cx%
                        \leftmargin=\quotationleftmargin@cx%
                        \itemindent    \listparindent
                        \rightmargin \quotationrightmargin@cx
                        \parsep=\quotationparsep@cx%
                        \quotationfontname@cx\quotationfontsize@cx}%
                \item\relax\hskip-\listparindent}
               {\endlist}
}
%    \end{macrocode}
% \end{macro}
%CHANGE FOR SETFONT
%    \begin{macrocode}
\setquotation{%
  quotation above=20pt, 
  quotation left margin=50pt,
  quotation right margin=0pt,
  quotation parsep=0pt,
  quotation font-size=\normalsize,
  quotation parindent=12pt,
  quotation font-name=, 
}
%    \end{macrocode}
%
% \begin{quotation}
% \lipsum[1]
% \end{quotation}
%
% \begin{macro}{\setquote}
%    \begin{macrocode}
\cxset{
  quote above/.store in=\quoteabove@cx,
  quote left margin/.store in=\quoteleftmargin@cx,
  quote right margin/.store in=\quoterightmargin@cx,
  quote parsep/.store in=\quoteparsep@cx,
  quote font-size/.store in=\quotefontsize@cx,
  quote parindent/.store in=\quoteparindent@cx,
  quote font-name/.store in=\quotefontname@cx,
 }
\let\latexquote\quote
\let\endlatexquote\endquote
\def\setquote#1{%
  \cxset{#1}
  \renewenvironment{quote}
               {\par\addvspace{\quoteabove@cx}
                \list{}{\listparindent\quoteparindent@cx%
                        \leftmargin=\quoteleftmargin@cx%
                        \itemindent  \listparindent
                        \rightmargin\leftmargin
                        \parsep=\quoteparsep@cx%
                        \quotefontsize@cx\quotefontname@cx}%
                \item\relax\hskip-\listparindent}
               {\endlist}
  }

% Some default values
\setquotation{%
  quotation above=36pt,
  quotation left margin=50pt,
  quotation parsep=0pt,
  quotation font-size=\small,
  quotation parindent=12pt,
}
\setquote{%
  quote above=0pt,
  quote left margin=20pt,
  quote parsep=0pt,
  quote font-size=\small,
  quote parindent=12pt,
  quote font-name=,
}
%    \end{macrocode}
% \end{macro}
%
% \section{Miscellaneous macros}  
%
% \subsection{Margin notes and margin emphasis}
% 
% \parindent1em 
% 
%   \marge{Margin boxes} Marginal notes are commonly found in
% 	many publications, Tufte goes to the extreme and requires
%	all footnotes and citations to be as `sidenotes'. We provide
%	a number of commands, if nothing else to illustrate 
%	techniques for defining them.
%
%
% \begin{macro}{\MarginBox}
%    \begin{macrocode}
\newcommand{\marginbox@cx}[2][]{%
  \mbox{}\marginpar{\centering\scriptsize \color{teal}#2}%
  \ifthenelse{\not\equal{#1}{}}{\phantomsection\label{#1}}{}%
}
\newcommand{\marge}[2][]{%
  \bigskip\par\marginbox@cx[#1]{#2}%
}
%    \end{macrocode}
% \end{macro}

% 
% \subsection{Paragraph setting commands}
%
%    \begin{macrocode}
%
\providecommand*{\linenottooshort}[1][4em]{%
  \@tempdima=\hsize
 \advance\@tempdima-#1
 \leftskip0pt
 \rightskip\leftskip
\parfillskip\@tempdima\@minus\@tempdima
}
\providecommand*{\lastlineparrule}{%
  \hrule height 0.5ex depth \@tempdimb\relax}

\providecommand*{\lastlinerulefill}{%
  \let\\\@centercr
  \@tempdimb=-0.5ex \advance\@tempdimb 0.4pt
  \unskip\nobreak\space
  \leaders\lastlineparrule\hskip\@flushglue
  \vadjust{}{\parfillskip\z@\@@par}}
%    \end{macrocode}
%
%    \begin{macrocode}
\newcommand{\hangleft}[1]{\makebox[0pt][r]{#1}}

\DeclareRobustCommand\ctan[1]{%
  \textcolor{green}{%
      \href{http://www.ctan.org/pkg/#1} {#1}%
  \footnote{\protect\url{http://www.ctan.org/pkg/#1}}}
  \index{Packages>#1}%
}
%
%
%    \end{macrocode}
%
% \begin{macro}{\keyval}
%	The macro \cs{keyval} typesets, key value lists and their options.
%	\medskip
%
%    \keyval{test}{\marg{option1|option2|option2|option4}}{ As per this example.}
%    \keyval{test}{\marg{option1|option2|option2|option4}}{ As per this example.}
%
%	We first measure the width of the option and not use it (want to make it a bit
%	flexible at a later stage. We also ensure that the catcode of \verb+|+ is set properly
%	in case anyone is using short verbatim commands, as we do in this document.
%
%    \begin{macrocode}
\newlength\temp@cx
\def\keyval{%
  \bgroup
  \catcode`|=11
  \@keyval}
%
\def\@keyval#1#2#3{%
  \settowidth\temp@cx{#1}%
  \parindent-30pt
  \hangindent30pt
  \par\leavevmode%
{\color{teal}\bfseries #1}\thinspace=\thinspace#2% 
\hspace*{.5em}#3%
\par\addvspace{1.5pt}%
\egroup
}
%
%    \end{macrocode}
% \end{macro}
%
% \section{Documentation Macros}

% This section defines commands for printing documentation
% such as this one. It draws inspiration and plagiarizes pgf,
% doc,symbols and many other packages for which I am grateful.
% First some macros for indexing commands.
% 
%    \begin{macrocode}
% Define a table environment that's similar to symtable except that it
% floats and it doesn't write an entry into the Table of Contents.  This
% is used for tables that contain something other than symbol lists.
\def\oarg#1{%
  \colOpt{{\ttfamily[}\meta{#1}{\ttfamily]}}}%
%  
\def\DescribeMacro{\leavevmode\@bsphack
   \begingroup\MakePrivateLetters\Describe@Macro}
\def\Describe@Macro#1{\endgroup
              {\raggedleft\PrintDescribeMacro{#1}}%
              \SpecialUsageIndex{#1}\@esphack\ignorespaces}


\def\DescribeEnv{\leavevmode\@bsphack\begingroup\MakePrivateLetters
  \Describe@Env}
\def\Describe@Env#1{\endgroup
              {\raggedleft\PrintDescribeEnv{#1}}{}%
              \SpecialEnvIndex{#1}\@esphack\ignorespaces}
\setlength\marginparpush{0pt}  



\newlength{\atemp}
 \def\PrintDescribeMacro#1{%
  \settowidth\atemp{\string #1} 
  \strut\MacroFont\color{thered}\normalsize\string#1}

\def\Describe#1{%
   \settowidth\atemp{\string #1}% 
  \par\penalty-500\vskip3ex\noindent
  \DescribeMacro{#1}\args}
\def\DescribeOther{\vskip-4ex\Describe}

\def\args#1{%
  \def\bbl@tempa{#1}%
  \ifx\bbl@tempa\@empty\else#1\vskip1ex\fi\ignorespaces}


\newenvironment{nonsymtable}[1]{%
  \begin{table}[htbp]
  \centering
  \caption{#1}\medskip
}{%
  \end{table}
}
%    \end{macrocode}
%
% \section{The package needspace}
%
% The \pkgname{needspace} is currently mainatained by Wills Robertson and was originally developed by 
% Peter Wilson \citeyearpar{needspace}.
% It provides the commands \CMDI{\needspace}\marg{length} and \CMDI{\Needspace}\marg{length}, that
% will reserve an addition amount of space on the page as specified by the parameter \emph{length}. 
% 
%    \begin{macrocode}
\IfStyFileExists*{needspace}
  {\RequirePackage{needspace}}
  {\newcommand{\Needspace}[2]{\par \penalty-100\begingroup
     \setlength{\dimen@}{##2}%
     \dimen@ii\pagegoal \advance\dimen@ii-\pagetotal
     \ifdim \dimen@>\dimen@ii
       \break
     \fi\endgroup}
  }




% Index "X Y" and "Y, X".  The "begin" and "end" variants are for page ranges.

\newcommand{\cmdI}[2][]{%
  \def\first@arg{#1}%
  \ifx\first@arg\@empty
    \texttt{\verbatimfont\string#2}\indexcommand[#2]{#2}%
  \else
    \texttt{\verbatimfont\string#2}\indexcommand[#1]{#2}%
  \fi
}


\newcommand{\cmdX}[1]{\cmdI[$\string#1$]{#1}}
\newcommand{\cmdW}[1]{\cmdI[$\string\blackacc{\string#1}$]{#1}}
\newcommand{\cmdIp}[1]{\texttt{\string#1}\indexpunct[$#1$]{#1}}
%    \end{macrocode}

% \begin{macro}{\CMDI}\oarg[symbol command]\marg{command}
% This macro \#1 symbol to be typeset next to
% \#2 in the index |\gothic (symbol)|
%    \begin{macrocode}
\DeclareRobustCommand\CMDI[1]{%
\bgroup%
\smallskip 
\noindent\texttt{\verbatimfont\string#1}%
\indexcommand{#1}%
\egroup%
}

\DeclareRobustCommand\luacmd[1]{%
  \bgroup
    \smallskip
    \noindent\color{black}\textbf{\string#1}%
    \indexcommand{#1}
 \egroup%
}

\DeclareRobustCommand\luafunction[1]{%
  \bgroup
    \smallskip
    \noindent\color{black}\textbf{\verbatimfont#1}%
    \indexcommand{#1}
 \egroup%
}

%    \end{macrocode}
% \end{macro}
%
%    \begin{macrocode}
\newcommand{\utfviii}{\mbox{UTF-8}\index{UTF-8}\xspace}
\newcommand{\idxboth}[2]{\mbox{}\index{#1 #2}\index{#2>#1}}
\newcommand{\idxbothbegin}[2]{\mbox{}\index{#1 #2|(}\index{#2>#1|(}}
\newcommand{\idxbothend}[2]{\mbox{}\index{#1 #2|)}\index{#2>#1|)}}
% Index TeXbook symbols and the CTAN repository.
\newcommand{\idxTBsyms}{%
  \index{symbols>TeXbook=\TeX{}book}%
  \index{TeXbook, The=\TeX{}book, The>symbols from}%
}

% Index logical styles.
\newcommand{\pkgname}[1]{%
  \href{http://ctan.org/pkg/#1}{#1}%
  \index{#1=\textsf{#1} (package)}%
  \index{packages>#1=\textsf{#1}}}
\let\pkg\pkgname

\newcommand*{\Lpack}[1]{\textsf {#1}}  
%\let\package\Lpack

\newcommand{\optname}[2]{%
  \textsf{#2}%
  \index{#2=\textsf{#2} (\textsf{#1} package option)}%
  \index{package options>#2=\textsf{#2} (\textsf{#1})}}
%    \end{macrocode}
%
% \begin{macro}{\docfilename}\marg{filename}
% This macro and all similar macros starting from doc
% typeset their argument and also add the argument to the 
% index.
%
%    \begin{macrocode}
\newcommand{\docfilename}[1]{%
  \texttt{#1}
  \index{#1=\texttt{#1} (file)}}
%    \end{macrocode}
% \end{macro}
%    \begin{macrocode}
\newcommand{\docfileextension}[1]{%
  \texttt{#1}%
  \index{#1=\texttt{#1} (file extension)}}
   \index{#1=\texttt{#1}}
\newcommand{\PSfont}[1]{%
  #1%
  \index{#1 (font)}%
  \index{fonts>#1}%
}
%    \end{macrocode}
% 
%    \begin{macrocode}
\DeclareRobustCommand{\person}[2]{#1\index{#2, #1} #2}
\newcommand{\idxCTAN}{%
  \index{Comprehensive TeX Archive Network=Comprehensive \string\TeX{} Archive Network}}
% Typeset a string in various encodings.
\newcommand{\encone}[1]{{\fontencoding{T1}\selectfont#1}}
\newcommand{\encfour}[1]{{\fontencoding{T4}\selectfont#1}}
\newcommand{\encfive}[1]{{\fontencoding{T5}\selectfont#1}}
\newcommand{\encgreek}[1]{{\fontencoding{LGR}\selectfont#1}}

% Various punctuation marks confuse makeindex when used directly.
\let\magicrbrack=]
\let\magicequal=\=
\DeclareRobustCommand{\magicequalname}{\texttt{\string\=}}
\DeclareRobustCommand{\magicvertname}{\texttt{|}}
\DeclareRobustCommand{\magicVertname}{\texttt{\string\|}}

% Vertically center a text-mode symbol.
\newsavebox{\tvcbox}
\newcommand*{\textvcenter}[1]{%
  \savebox{\tvcbox}{#1}%
  \raisebox{0.5\dp\tvcbox}{\raisebox{-0.5\ht\tvcbox}{\usebox{\tvcbox}}}%
}
% Many tables have notes beneath them.  Define an environment in which to
% display such a note, with an optional, superscripted math symbol
% preceding it.
\newenvironment{tablenote}[1][]{
  \makebox[1em]{\ensuremath{^{#1}}}%
  \begin{minipage}[t]{0.75\textwidth}%
  \setlength{\parskip}{2ex}
}{%
  \end{minipage}%
}

% Define various messages we reuse repeatedly.
\newcommand{\twosymbolmessage}{%
  \begin{tablenote}
    Where two symbols are present, the left one is the ``faked'' symbol
    that \latexe provides by default, and the right one is the ``true''
    symbol that \TC\ makes available.
  \end{tablenote}
}

\newcommand{\notpredefinedmessage}{%
  \begin{tablenote}[*]
    Not predefined in \latexe.  Use one of the packages
    \pkgname{latexsym}, \pkgname{amsfonts}, \pkgname{amssymb},
    \pkgname{txfonts}, \pkgname{pxfonts}, or \pkgname{wasysym}.
  \end{tablenote}
}

\newcommand{\notpredefinedmessageABX}{%
  \begin{tablenote}[*]
    Not predefined in \latexe.  Use one of the packages
    \pkgname{latexsym}, \pkgname{amsfonts}, \pkgname{amssymb},
    \pkgname{mathabx}, \pkgname{txfonts}, \pkgname{pxfonts}, or
    \pkgname{wasysym}.
  \end{tablenote}
}

\newcommand{\usetextmathmessage}[1][]{%
  \begin{tablenote}[#1]
    It's generally preferable to use the corresponding symbol from
    \vref{math-text} because the symbols in that table work
    properly in both text mode and math mode.
  \end{tablenote}
}



\newcommand{\usefontcmdmessage}[2]{%
  These symbols must appear either within the argument to \cmd{#1} or
  following the \cmd{#2} font-selection command within a scope%
}
% Define an environment in which to write a single table of symbols.  The
% environment looks a lot like a table, but it doesn't float, and it gets
% an entry in the table of contents as opposed to the list of tables.
%
% The first argument is a conditional.  The table will appear only if
% the value of the conditional is true.  The second argument is the
% table's caption.

\def\fnum@table{\tablename~\thetable}

\newenvironment{symtable}[2][true]{%
  \expandafter\global\expandafter\let%
    \expandafter\ifshowsymtable\csname if#1\endcsname
  \ifshowsymtable
    \noindent%
    \begin{minipage}[t]{\linewidth}    % Prevent page breaks
    \begin{center}
    \refstepcounter{table}%
    \phantomsection
    \addcontentsline{toc}{subsection}{%
      \protect\numberline{\tablename~\thetable:}{#2}}%
    \@makecaption{\fnum@table}{#2}\medskip
    \let\next=\relax
  \else
    % The following was taken verbatim from verbatim.sty.
    \let\do\@makeother\dospecials\catcode`\^^M\active
    \let\verbatim@startline\relax
    \let\verbatim@addtoline\@gobble
    \let\verbatim@processline\relax
    \let\verbatim@finish\relax
    \let\next=\verbatim@
  \fi
  \next
}{%
  \ifshowsymtable
    \end{center}
    \end{minipage}
    \vskip 8ex minus 2ex
  \fi
}
%    \end{macrocode}
% \section{Scripts and Languages }

% |\g_phd_scripts_clist| holds a list of all the scripts that have been loaded.
% Managing the user interface is problematic, we will have users that require
% only one script and users that might want all of them.
% There is also the issue between the blurring of alphabets, langages and scripts
% Since we will always specify a pan-unicode font, which we will make available
% with the |phd| package. We map all scripts to this font first.
%
%  Declare two global lists to hold all the scripts available.
% The |\script_prop| holds info for each script loaded
%
%    \begin{macrocode}
\ExplSyntaxOn
\clist_new:N \g_phd_scripts_clist
\clist_new:N \g_phd_noto_clist
\prop_new:N \script_prop
%\clist_gset:Nn \g_phd_noto_clist {
NotoKufiArabic-Bold.ttf, 
NotoKufiArabic-Regular.ttf, 
NotoNaskhArabic-Bold.ttf, 
NotoNaskhArabic-Regular.ttf, 
NotoNastaliqUrduDraft.ttf, 
NotoSans-Bold.ttf, 
NotoSans-BoldItalic.ttf, 
NotoSans-Italic.ttf, 
NotoSans-Regular.ttf, 
NotoSansArmenian-Bold.ttf, 
NotoSansArmenian-Regular.ttf, 
NotoSansAvestan-Regular.ttf, 
NotoSansBalinese-Regular.ttf, 
NotoSansBamum-Regular.ttf, 
NotoSansBatak-Regular.ttf, 
NotoSansBengali-Bold.ttf, 
NotoSansBengali-Regular.ttf, 
NotoSansBrahmi-Regular.ttf, 
NotoSansBuginese-Regular.ttf, 
NotoSansBuhid-Regular.ttf, 
NotoSansCanadianAboriginal-Regular.ttf, 
NotoSansCarian-Regular.ttf, 
NotoSansCham-Bold.ttf, 
NotoSansCham-Regular.ttf, 
NotoSansCherokee-Regular.ttf, 
NotoSansCJKjp-Black.ttf, 
NotoSansCJKjp-Bold.ttf, 
NotoSansCJKjp-DemiLight.ttf, 
NotoSansCJKjp-Light.ttf, 
NotoSansCJKjp-Medium.ttf, 
NotoSansCJKjp-Regular.ttf, 
NotoSansCJKjp-Thin.ttf, 
NotoSansCJKkr-Black.ttf, 
NotoSansCJKkr-Bold.ttf, 
NotoSansCJKkr-DemiLight.ttf, 
NotoSansCJKkr-Light.ttf, 
NotoSansCJKkr-Medium.ttf, 
NotoSansCJKkr-Regular.ttf, 
NotoSansCJKkr-Thin.ttf, 
NotoSansCJKsc-Black.ttf, 
NotoSansCJKsc-Bold.ttf, 
NotoSansCJKsc-DemiLight.ttf, 
NotoSansCJKsc-Light.ttf, 
NotoSansCJKsc-Medium.ttf, 
NotoSansCJKsc-Regular.ttf, 
NotoSansCJKsc-Thin.ttf, 
NotoSansCJKtc-Black.ttf, 
NotoSansCJKtc-Bold.ttf, 
NotoSansCJKtc-DemiLight.ttf, 
NotoSansCJKtc-Light.ttf, 
NotoSansCJKtc-Medium.ttf, 
NotoSansCJKtc-Regular.ttf, 
NotoSansCJKtc-Thin.ttf, 
NotoSansCoptic-Regular.ttf, 
NotoSansCuneiform-Regular.ttf, 
NotoSansCypriot-Regular.ttf, 
NotoSansDeseret-Regular.ttf, 
NotoSansDevanagari-Bold.ttf, 
NotoSansDevanagari-Regular.ttf, 
NotoSansEgyptianHieroglyphs-Regular.ttf, 
NotoSansEthiopic-Bold.ttf, 
NotoSansEthiopic-Regular.ttf, 
NotoSansGeorgian-Bold.ttf, 
NotoSansGeorgian-Regular.ttf, 
NotoSansGlagolitic-Regular.ttf, 
NotoSansGothic-Regular.ttf, 
NotoSansGujarati-Bold.ttf, 
NotoSansGujarati-Regular.ttf, 
NotoSansGurmukhi-Bold.ttf, 
NotoSansGurmukhi-Regular.ttf, 
NotoSansHanunoo-Regular.ttf, 
NotoSansHebrew-Bold.ttf, 
NotoSansHebrew-Regular.ttf, 
NotoSansImperialAramaic-Regular.ttf, 
NotoSansInscriptionalPahlavi-Regular.ttf, 
NotoSansInscriptionalParthian-Regular.ttf, 
NotoSansJavanese-Regular.ttf, 
NotoSansKaithi-Regular.ttf, 
NotoSansKannada-Bold.ttf, 
NotoSansKannada-Regular.ttf, 
NotoSansKayahLi-Regular.ttf, 
NotoSansKharoshthi-Regular.ttf, 
NotoSansKhmer-Bold.ttf, 
NotoSansKhmer-Regular.ttf, 
NotoSansLao-Bold.ttf, 
NotoSansLao-Regular.ttf, 
NotoSansLepcha-Regular.ttf, 
NotoSansLimbu-Regular.ttf, 
NotoSansLinearB-Regular.ttf, 
NotoSansLisu-Regular.ttf, 
NotoSansLycian-Regular.ttf, 
NotoSansLydian-Regular.ttf, 
NotoSansMalayalam-Bold.ttf, 
NotoSansMalayalam-Regular.ttf, 
NotoSansMandaic-Regular.ttf, 
NotoSansMeeteiMayek-Regular.ttf, 
NotoSansMongolian-Regular.ttf, 
NotoSansMyanmar-Bold.ttf, 
NotoSansMyanmar-Regular.ttf, 
NotoSansNewTaiLue-Regular.ttf, 
NotoSansNKo-Regular.ttf, 
NotoSansOgham-Regular.ttf, 
NotoSansOlChiki-Regular.ttf, 
NotoSansOldItalic-Regular.ttf, 
NotoSansOldPersian-Regular.ttf, 
NotoSansOldSouthArabian-Regular.ttf, 
NotoSansOldTurkic-Regular.ttf, 
NotoSansOriya-Bold.ttf, 
NotoSansOriya-Regular.ttf, 
NotoSansOsmanya-Regular.ttf, 
NotoSansPhagsPa-Regular.ttf, 
NotoSansPhoenician-Regular.ttf, 
NotoSansRejang-Regular.ttf, 
NotoSansRunic-Regular.ttf, 
NotoSansSamaritan-Regular.ttf, 
NotoSansSaurashtra-Regular.ttf, 
NotoSansShavian-Regular.ttf, 
NotoSansSinhala-Bold.ttf, 
NotoSansSinhala-Regular.ttf, 
NotoSansSundanese-Regular.ttf, 
NotoSansSylotiNagri-Regular.ttf, 
NotoSansSymbols-Regular.ttf, 
NotoSansSyriacEastern-Regular.ttf, 
NotoSansSyriacEstrangela-Regular.ttf, 
NotoSansSyriacWestern-Regular.ttf, 
NotoSansTagalog-Regular.ttf, 
NotoSansTagbanwa-Regular.ttf, 
NotoSansTaiLe-Regular.ttf, 
NotoSansTaiTham-Regular.ttf, 
NotoSansTaiViet-Regular.ttf, 
NotoSansTamil-Bold.ttf, 
NotoSansTamil-Regular.ttf, 
NotoSansTelugu-Bold.ttf, 
NotoSansTelugu-Regular.ttf, 
NotoSansThaana-Bold.ttf, 
NotoSansThaana-Regular.ttf, 
NotoSansThai-Bold.ttf, 
NotoSansThai-Regular.ttf, 
NotoSansTifinagh-Regular.ttf, 
NotoSansUgaritic-Regular.ttf, 
NotoSansVai-Regular.ttf, 
NotoSansYi-Regular.ttf, 
NotoSerif-Bold.ttf, 
NotoSerif-BoldItalic.ttf, 
NotoSerif-Italic.ttf, 
NotoSerif-Regular.ttf, 
NotoSerifArmenian-Bold.ttf, 
NotoSerifArmenian-Regular.ttf, 
NotoSerifGeorgian-Bold.ttf, 
NotoSerifGeorgian-Regular.ttf, 
NotoSerifKhmer-Bold.ttf, 
NotoSerifKhmer-Regular.ttf, 
NotoSerifLao-Bold.ttf, 
NotoSerifLao-Regular.ttf, 
NotoSerifThai-Bold.ttf, 
NotoSerifThai-Regular.ttf, 
}
%    \end{macrocode}
%
% The noto list holds all the available noto fonts as of June 2015. It typesets a list in a 
% two column environment.
%
% \begin{macro}{\notofontlist}
%    \begin{macrocode}
\cs_set:Npn \notofontlist 
  {
    \begin{multicols}{2}
      \clist_map_inline:Nn \g_phd_noto_clist
        {
          ##1\par 
		  }
    \end{multicols}  
  }
%    \end{macrocode}	
% \end{macro}
% 
%    \begin{macrocode}	
\prop_put:Nnn \script_prop {name}{Armenian}
\prop_put:Nnn \script_prop {fonts}{NonoArmenian-Regular.ttf, Others}
\prop_get:NnN \script_prop {fonts}\l_tempa_tl
\prop_put:Nnn \script_prop {group}{Europe}
\prop_get:NnN \script_prop {group} \l_tempa_tl
%    \end{macrocode}
%
% \begin{docCommand}{SetPanUnicodeFont}{\marg{font name}}
%  Sets the pan-unicode font. This font is to be used as a default for all the scripts
%  The user can override it with another font.
% \end{docCommand}
%
%    \begin{macrocode}
\NewDocumentCommand\SetPanUnicodeFont { m }
  {
     \gdef\panunicodefontface{#1}
     \newfontfamily\panunicode[Scale=MatchUppercase]{#1}
  }
\SetPanUnicodeFont{code2000.ttf}    
%    \end{macrocode}

%    \begin{macrocode}
\cs_gset:Npn \makepanfontfamily#1{
%  \newfontfamily\cs:w #1fontfamily\cs_end: { #2 }
  \cs_gset_eq:cN {#1fontfamily}\panunicode
  \cs_gset_eq:cc {#1} {#1fontfamily}
}

\cs_gset:Npn \add_a_script:n #1
 {
   \clist_gput_left:Nn \g_phd_scripts_clist {#1 }
   \createscriptenvironment {#1}
   \createtextscript {#1}
 }   
 
 % add a script
\NewDocumentCommand\addascript { m } 
  {
    \add_a_script:n {#1}
  }
  
% Mock an environment 
\gdef\createscriptenvironment #1{
   \exp_after:wN\gdef\csname #1script\endcsname{\group_begin:
      \csname #1fontfamily\endcsname}
   \exp_after:wN\gdef\cs:w end#1script\cs_end:{\group_end: }
}  
\ExplSyntaxOff
%    \end{macrocode}
%  
% \begin{docCommand}{createtextscript}{ \marg{script name}}
% This creates a command of the form |\text|\meta{script name} i.e., for tibetan
% it will produce |\texttibetan|
% \end{docCommand}
%    \begin{macrocode}
\ExplSyntaxOn
\cs_gset:Npn \createtextscript #1{
   \long\exp_after:wN\gdef\csname text#1\endcsname ##1
   {
      \group_begin: 
      \cs:w #1fontfamily\cs_end:
        ##1
     \group_end:
   }
}  
%
%
\cs_gset:Npn \makefontfamily#1#2 {
\if_meaning:w\panunicodefontface#2
  \else:
  \exp_after:wN
  \newfontfamily\cs:w #1fontfamily\cs_end: { #2 }
  \cs_gset_eq:cc {#1} {#1fontfamily}
\fi:  
}
\ExplSyntaxOff
\NewDocumentCommand\AddScript { m } {
    \cxset{script/.code=\addascript{##1}}
    \cxset{#1 font/.code=\makefontfamily{#1}{##1}}
    \cxset{script=#1}
    \cxset{#1 font=\panunicodefontface}
}
\cxset{add script/.code = \AddScript{#1}}

\ExplSyntaxOn
\clist_gset:Nn \g_phd_scripts_clist {
      armenian,
      %hebrew,
     % arabic,
      syriac,
      thaana,
      devanagari,
      bamum,
      bengali,
      brahmi,
      coptic,
      gurmukhi,
      gujarati,
      oriya,
      tamil,
      telugu,
      kannada,
      malayalam,
      thai,
      lao,
      lisu,
      myanmar,
      georgian,
      ethiopic,
      cherokee,
      ogham,
      runic,
      buhid,
      bopomofo,
      tibetan, 
      cypriot, 
      telugu, 
      phoenician, 
      cham,
      vai,
      rejang,
      glagolitic,
      saurashtra,
         sinhala,
      sylhetinagari,
      tifinagh,
      kayahli,
     mongolian,
     oldturkic,
     cjk,
}

\clist_map_inline:Nn\g_phd_scripts_clist 
  {
    \AddScript{#1}
    \makepanfontfamily {#1}
  }
\ExplSyntaxOff
%    \end{macrocode}
%
% A small utility macro to typeset unicode tables
% examples can be see in the chapters for scripts.
%puts the unicode label (removes last char and adds x)
%
% \begin{macro}{\putunicode@label}\marg{unformatted string} 
% This macro receives a number in hexadecimal, removes the last
% 0 and replaces it with an x. It then prepends a U+ to fomat it
% as a Unicode number e.g. U+0100x
%    \begin{macrocode}
\newcounter{glyph@count}%counts glyphs
\def\textU#1{{\unicodenumberfam #1}}
\def\putunicode@label#1#2;{%
\def\reformat@unicode@string##1{%
   \textU{U+}%
  \let\z\empty%
  \expandafter\@tfor\expandafter\i\expandafter:\expandafter=#2;\do{%
  \if\i;%
    \textU{x}%
  \else%
    \textU{\z}%
  \fi%
  \edef\z{\i}%
 }%
}%
  \makebox[5em]{\reformat@unicode@string{#2}\hfill}%
}
%    \end{macrocode}
% \end{macro}
% \begin{macro}{\putchar@cx}
%    \begin{macrocode}
\def\putchar@cx#1{%
\stepcounter{glyph@count}
\let\oldactive@prefix\active@prefix
\let\active@prefix\relax
   \iffontchar\font\n
     \char\the\n$_{\pgfmathparse{Hex(\the\r@cx)}\text{\pgfmathresult}}$%
      %
   \else
    {\arial\graybox}
   \fi
\let\active@prefix\oldactive@prefix
 }

\def\urow@cx#1{%
    \n=#1% 
    \r@cx=0%
    \expandafter\putunicode@label#1;%
    \loop%
        \ifnum\n<\numexpr#1+16\relax%
        \makebox[2.1em]{\expandafter\putchar@cx{#1}}%
        \advance\r@cx by1%  
        \ifnum\r@cx>16\r@cx=1\relax\else\fi
        \advance\n by1%
    \repeat
    \par
}

\def\typeseturows@cx#1{%
\@for\next:=#1\do{%
  \urow@cx\next\vskip3pt}%
}

\newcount\r@cx%
\newcount\n%
\newcommand\unicodetable[2]{%
\bgroup
  \par
  \leavevmode%
   \r@cx=0%
   {\hbox to 5em{\ignorespaces}}%
   \loop%
    \ifnum\r@cx<16\ignorespaces 
    \makebox[2.1em]{\pgfmathparse{Hex(\the\r@cx)}\pgfmathresult}%
    \advance\r@cx by\@ne%  
   \repeat
   \vskip3pt\par
   \@nameuse{#1}%
   \typeseturows@cx{#2}%
\egroup
}
%%%%%%%%%%%%%%%%%%%%%% REVISIT THIS
\DeclareRobustCommand\unicodenumber[1]{{\smallcps #1\xspace}}

\def\putdescription#1:{%
  \parindent0pt 
  \begin{minipage}[t]{4cm}
  \bgroup\aegean
  \hangindent20pt
  #1\par
  \egroup
  \end{minipage} 
}


\long\def\parsefields #1:#2\@@{%
    \ifx\par#1
    \else 
        {\small\aegean U+#1}%
         %%\iffontchar\font"#1 %
          \makebox[2.1em]{\color{blue}\symbol{"#1}}% 
          \expandafter\putdescription#2\vskip3pt
        %%\else
          %%{\aegean \makebox[2.1em]{} Unallocated\par}%
        %%\fi
    \fi  
  }%
% Check if it can be saved
\newread\tempstream

% begin{macro}{\printunicodeblock}\marg{}\oarg{} The macro
% prints a unicode table from a file of definitions. This is
% printed in a two column environment by default. 
% #1 filename and path
% #2 font command
\DeclareDocumentCommand{\printunicodeblock}{O{2} m m }{%
  \bgroup
  \leavevmode\parindent0pt\par
  \begin{multicols}{#1}%
  #3
  \openin\@inputcheck=#2
  \loop\unless\ifeof\@inputcheck
    \read\@inputcheck to\fileline %
    \expandafter\parsefields \fileline:\@@ 
  \repeat
  \end{multicols}%
  \immediate\closein\@inputcheck
  \egroup
}
\let\PrintUnicodeBlock\printunicodeblock
%    \end{macrocode}
% \end{macro}

%\subsection{Indexing macros}
%

%    \begin{macrocode}


 
 \ExplSyntaxOn
 \DeclareDocumentCommand\indexmany {o m }
 {
   \clist_gset:Nn \indexmany: {#2} 
   \IfValueTF {#1}
    { 
      \clist_map_inline:Nn\indexmany: 
        {
          \index{#1>##1}\index{##1}
        }
    }
    { 
     \clist_map_inline:Nn\indexmany: 
      {
        \index{##1}
      } 
    }
 }
 \ExplSyntaxOff

%    \end{macrocode}
% We define a related macro for indexing accents.  In a previous version
% of this file, \indexaccent additionally included "see also accents" in
% the index.  This became distracting so I made \indexaccent a synonym
% for \indexcommand for the time being.  Because punctuation marks can
% be problematic for makeindex, we define an \indexpunct macro that
% sorts its argument under the comparatively innocuous "\_".
%
%    \begin{macrocode}
\begingroup
 \catcode`\|=0
 \catcode`\\=12
 |gdef|sanitize#1#2!!!{%
   |ifx#1\%
     #2%
   |else
     #1#2%
   |fi
}
|endgroup
%    \end{macrocode}
%
% \begin{macro}{\indexcommand}\oarg{}\marg{command} 
%
% Index a symbol, which may or may not begin with a backslash.  (Is
% there a better way to do this?)  Also, if symbol is given as an
% optional argument is given, typeset that symbol in the index, as well
%
%    \begin{macrocode}
  \newcommand{\indexcommand}[2][]{%
    \edef\sanitized{\expandafter\sanitize\string#2!!!}%
    \def\first@arg{#1}%
    \ifx\first@arg\@empty
      \expandafter\index\expandafter{\sanitized=\string\verb+\string#2+}%
    \else
      \expandafter\index\expandafter{\sanitized=\string\verb+\string#2+ (#1)}%
    \fi
  }
%    \end{macrocode}
% 
%    \begin{macrocode}
  \let\indexaccent=\indexcommand
  \def\CLSLpipe{|}%
%    \end{macrocode}
%   \end{macro}

% \begin{macro}{\indexpunct}
%    \begin{macrocode}
%  
  \newcommand{\indexpunct}[2][]{%
    \def\first@arg{#1}%
    \def\second@arg{#2}%
    \ifx\first@arg\@empty
      \ifx\second@arg\CLSLpipe
        \index{_=\magicvertname}%
      \else
        \index{_=\string\verb+\string#2+}%
      \fi
    \else
      \ifx\second@arg\CLSLpipe
        \index{_=\magicvertname{} (#1)}%
      \else
        \index{_=\string\verb+\string#2+ (#1)}%
      \fi
    \fi
  }
%    \end{macrocode}
%    \begin{macrocode}
\DeclareRobustCommand{\idxfont}[1]{\index{#1 (font)}\texttt{#1}\xspace}%
\DeclareRobustCommand{\idxlanguage}[1]{\index{#1 (script)}\index{scripts>#1}\texttt{#1}\xspace}%
%    \end{macrocode}
% \end{macro}
%    \begin{macrocode}
% Define a counter to keep track of how many symbols are listed.
% Output this counter to the log file at the end of each run.
% Define \prevtotalsymbols to be the total number of symbols from
% the previous run.
%
% \subsection{mathdots}

\newif\ifMDOTS
\newcommand\MDOTS{\pkgname{mathdots}}
\IfStyFileExists{mathdots}
  {\MDOTStrue
   \savesymbol{ddots}
   \savesymbol{vdots}
   \savesymbol{iddots}
   \savesymbol{dddot}
   \savesymbol{ddddot}
   \usepackage{mathdots}
   \restoresymbol{MDOTS}{ddots}
   \restoresymbol{MDOTS}{vdots}
   \restoresymbol{MDOTS}{iddots}
   \restoresymbol{MDOTS}{dddot}
   \restoresymbol{MDOTS}{ddddot}
  }
  {}

%\usepackage{longdiv}
\newcommand\FC{\pkgname{fc}}
\newcommand\VIET{\pkgname{vietnam}}
%\newcommand\ABX{\pkgname{mathabx}}
%    \end{macrocode}
%
% \begin{macro}{graybox}
% \begin{macro}{\incsyms}
%    \begin{macrocode}
\newcounter{totalsymbols}
\newcommand{\incsyms}{\addtocounter{totalsymbols}{1}}

\newcommand*{\graybox}{\textcolor{thegray!60}{\rule[-\adp]{\awd}{\aht}}}
 
% Define \blackacc to display an accented box, given an accent command.
% Define \blackacchack to display an accented "a" and then black out
% the "a".
\newlength\awd
\newlength\aht
\newlength\adp
\settowidth{\awd}{\normalfont m}
\settoheight{\aht}{\normalfont I}
\settodepth{\adp}{\normalfont m}
\advance\adp by 0.06pt    % In Computer Modern, "a" extends slightly below its bounding box.
\advance\aht by \adp
\gdef\blackacchack#1{#1a\llap{\graybox}}
\gdef\blackacc#1{#1{\graybox}}
\gdef\blackacctwo#1{#1{\graybox}{\graybox}}
%    \end{macrocode}
% \end{macro}
% \end{macro}
%
% Symbol+verbatim for various types of symbols
%    \begin{macrocode}
\def\E#1{%
  \begingroup
    \lccode`|=`\\
    \def\EStruename{ES#1T}
    \lowercase{\incsyms\index{#1=\string\verb+\string|#1+ (\string|\EStruename)}}
  \endgroup
  \csname ES#1T\endcsname & \csname ES#1D\endcsname &
  \ttfamily\expandafter\string\csname#1\endcsname
}

\def\K@opt@arg[#1]#2{\incsyms\indexcommand[#1]{#2}#1 &\ttfamily\string#2}
  \def\K@no@opt@arg#1{\incsyms\indexcommand[#1]{#1}#1 &\ttfamily\string#1}

\def\K{\@ifnextchar[{\K@opt@arg}{\K@no@opt@arg}}

\def\Kp#1{\incsyms\indexpunct[$#1$]{#1}#1 &\ttfamily\string#1}

\def\KED[#1][#2][#3]#4{\incsyms\indexcommand[#1]{#2}#3 &\ttfamily\string#4}
\def\Kfeyn#1{\incsyms\indexcommand[\string\feyn{#1}]{\feyn{#1}}\feyn{#1} &\ttfamily\string\feyn\string{\string#1\string}}

\def\Kp#1{\incsyms\indexpunct[$#1$]{#1}#1 &\ttfamily\string#1}

\def\Kpig#1{\incsyms\index{pigpenfont #1=\string\verb+{\string\pigpenfont\space#1}+\space(\string\CLSLpig{#1})}\CLSLpig{#1} &\ttfamily\string{\string\pigpenfont\space\string#1\string}}
\def\Ks#1{\incsyms\indexcommand[\string\encone{\string#1}]{#1}{\encone{#1}} &\ttfamily\string#1$^*$}
%    \end{macrocode}

% \begin{macro}{Kt}
%
% This macro is also from the comprehensive and takes
% the symbol command as its only argument. It provides
% |T1| encoding and also adds the command to the index.
%    \begin{macrocode}   
\newcommand\Kt[1]{%
        \incsyms\indexcommand[\string\encone{\string#1}]{#1}{%
        \encone{#1}} &\ttfamily\string#1}%
%    \end{macrocode}
% \end{macro}
%    \begin{macrocode}
\def\Kv#1{\incsyms\indexcommand[\string\encfive{\string#1}]{#1}{\encfive{#1}} &\ttfamily\string#1}

\def\Kgr@opt@arg[#1]#2{\incsyms\indexcommand[\string\encgreek{\string#1}]{#2}{\encgreek{#1}} &\ttfamily\string#2}
  \def\Kgr@no@opt@arg#1{\incsyms\indexcommand[\string\encgreek{\string#1}]{#1}{\encgreek{#1}} &\ttfamily\string#1}
  \def\Kgr{\@ifnextchar[{\Kgr@opt@arg}{\Kgr@no@opt@arg}}

\def\KN[#1][#2]#3{\incsyms\indexcommand[\string#1]{#3} #1 & #2 & \ttfamily\string#3}
\def\KNbig[#1][#2]#3{\incsyms\indexcommand[\string#2]{#3} #1 & #2 & \ttfamily\string#3}
\def\Knoidx#1{\incsyms#1 &\ttfamily\string#1}
%% N
 \def\N@opt@arg[#1]#2{\incsyms\indexcommand[$\string#1$]{#2}$#1$ & $\Big#1$ &\ttfamily\string#2}
  \def\N@no@opt@arg#1{\incsyms\indexcommand[$\string#1$]{#1}$#1$ & $\Big#1$ &\ttfamily\string#1}
  \def\N{\@ifnextchar[{\N@opt@arg}{\N@no@opt@arg}}
  \def\Nn[#1]#2{%
    \incsyms\indexcommand[$\string\nathdouble\string#1$]{#2}%
    $\nathdouble#1$ & $\nathdouble{\Big#1}$ & \ttfamily\string#2}
  \def\Nnt#1[#2]#3{%
    \incsyms\indexcommand{\triple}%
    $\nathtriple#2$ & $\nathtriple{\Big#2}$ &
    \ttfamily\expandafter\string\csname#1triple\endcsname\string#3}
  \def\Np@opt@args[#1]{\@ifnextchar[{\Np@two@opt@args[#1]}{\Np@one@opt@arg[#1]}}
  \def\Np@two@opt@args[#1][#2]#3{\incsyms\index{_=\string#2{} ($\string#1$)}$#1$ & $\Big#1$ &\ttfamily\string#3}
  \def\Np@one@opt@arg[#1]#2{\incsyms\indexpunct[$\string#1$]{#2}$#1$ & $\Big#1$ &\ttfamily\string#2}
  \def\Np@no@opt@args#1{\incsyms\indexpunct[$\string#1$]{#1}$#1$ & $\Big#1$ &\ttfamily\string#1}
  \def\Np{\@ifnextchar[{\Np@opt@args}{\Np@no@opt@args}}
  \def\Nbig[#1]#2{\incsyms\indexcommand[$\string\Big\string#1$]{#2}$#1$ & $\Big#1$ &\ttfamily\string#2}
%% Q commands
 \def\Q@opt@arg[#1]#2{\incsyms\indexaccent[\string\blackacchack{\string#1}]{#2}#1{A}#1{a} &
           \ttfamily\string#2\string{A\string}\string#2\string{a\string}}
  \def\Q@no@opt@arg#1{\incsyms\indexaccent[\string\blackacchack{\string#1}]{#1}#1{A}#1{a} &
           \ttfamily\string#1\string{A\string}\string#1\string{a\string}}
  \def\Q{\@ifnextchar[{\Q@opt@arg}{\Q@no@opt@arg}}

\def\Qc#1{\incsyms\indexaccent[\string\blackacc{\string#1}]{#1}#1{A}#1{a} &
         \ttfamily\string#1\string{A\string}\string#1\string{a\string}}
\def\Qe[#1][#2]#3{%
  \incsyms\incsyms\index{_=\string#2{} (\string\blackacchack{\string#1})}%
  #3{A}#3{a} &
  \ttfamily\string#3\string{A\string}\string#3\string{a\string}}
\def\Qt#1{\incsyms\indexaccent[\string\encone{\string\blackacc{\string#1}}]{#1}{\encone{#1{A}#1{a}}} &
          \ttfamily\string#1\string{A\string}\string#1\string{a\string}}

\def\Qpc#1#2{\incsyms\indexcommand{#2}{\raisebox{1pt}{\tiny[#1]}} &
             \ttfamily\string#2\string{A\string}\string#2\string{a\string}}
\def\Qpfc[#1]#2{\incsyms\indexaccent[\string\encfour{\string\blackacchack{\string#1}}]{#2}\encfour{#1{A}#1{a}} &
           \ttfamily\string#2\string{A\string}\string#2\string{a\string}}
%% TODO
\newif\ifFC\FCfalse
\ifFC
  \def\Qiv#1#2{\incsyms\indexaccent[\string\encfour{\string\blackacchack{\string#1}}]{#1}\encfour{#1{A}#1{a}} &
               \ttfamily\string#1\string{A\string}\string#1\string{a\string}$^#2$}
  \def\QivBAR#1{\incsyms\index{_=\string\magicVertname{}
                (\string\encfour{\string\blackacchack{\string\FCbar}})}
                \encfour{\FCbar{A}\FCbar{a}} &
                \ttfamily\string\|\string{A\string}\string\|\string{a\string}$^#1$}
\else
  \def\Qiv#1#2{\Qpc{T4}{#1}$^#2$}
  \def\QivBAR#1{\Qpc{T4}{\|}$^#1$}
\fi
\newif\ifVIET\VIETfalse
\ifVIET
  \def\Qv#1#2{\incsyms\indexaccent[\string\encfive{\string\blackacchack{\string#1}}]{#1}{\encfive{#1{A}#1{a}}} &
              \ttfamily\string#1\string{A\string}\string#1\string{a\string}$^#2$}
\else
  \def\Qv#1#2{\Qpc{T5}{#1}$^#2$}\def\Qv#1#2{Err}%TODO
\fi
%% R Commands
  % We use \cmd{displaystyle} so that variable-sized symbols will be big.
  \def\R@opt@arg[#1]#2{\incsyms\indexcommand[$\string#1$]{#2}$#1$ & $\displaystyle#1$ &\ttfamily\string#2}
  \def\R@no@opt@arg#1{\incsyms\indexcommand[$\string#1$]{#1}$#1$ & $\displaystyle#1$ &\ttfamily\string#1}
  \def\R{\@ifnextchar[{\R@opt@arg}{\R@no@opt@arg}}
%% T commands
\def\Tp#1{\incsyms\indexcommand{\ding}\ding{#1} &\ttfamily\string\ding\string{#1\string}}
\def\Tm#1{\incsyms\indexcommand{\maya}$\mayadigit{#1}$ &\ttfamily\string\maya\string{#1\string}}
\def\Tmoon#1{\incsyms\indexcommand{\MoonPha}\MoonPha{#1} &\ttfamily\string\MoonPha\string{#1\string}}

%% This command typesets its argument and also puts

\newcommand{\V}[2][]{%
   \incsyms#1 & 
   \indexcommand[#2]{#2}% necessary to put symbol \text
   #2%  
   &\ttfamily\string#2}

% new attempt (needs work)
\newcommand{\docV}[2][]{%
  % \incsyms#1 & 
  % \indexcommand[#2]{#2}% necessary to put symbol \text
   &\csname#2\endcsname%  
   &\tcbset{color command=blue}
      \docAuxCommand {#2}}
   
\newcommand{\Vp}[2][]{\incsyms#1 & \indexpunct[$#2$]{#2}#2 &\ttfamily\string#2}

%W
\def\W@opt@arg[#1]#2#3{%
    \incsyms\indexaccent[$\string\blackacc{\string#1}$]{#2}%
    $#1{#3}$ &\ttfamily\string#2\string{#3\string}}
  \def\W@no@opt@arg#1#2{%
    \incsyms\indexaccent[$\string\blackacc{\string#1}$]{#1}%
    $#1{#2}$ &\ttfamily\string#1\string{#2\string}}
  \def\W{\@ifnextchar[{\W@opt@arg}{\W@no@opt@arg}}

\def\Wf#1#2{\incsyms\indexcommand{#1}$#1{#2}$ &\ttfamily\string#1\string{#2\string}}
\def\Ww#1#2#3{\incsyms\indexcommand{#2}$#1{#3}$ &\ttfamily\string#2\string{#3\string}}
\def\Wul#1#2#3{%
  \incsyms\indexaccent[$\string\blackacctwo{\string#1}$]{#1}%
  $#1{#2}{#3}$ &\ttfamily\string#1\string{#2\string}\string{#3\string}}

\def\X@opt@arg[#1]#2{\incsyms\indexcommand[$\string#1$]{#2}$#1$ &\ttfamily\string#2}
  \def\X@no@opt@arg#1{\incsyms\indexcommand[$\string#1$]{#1}$#1$ &\ttfamily\string#1}
  
\def\X{\@ifnextchar[{\X@opt@arg}{\X@no@opt@arg}}

\def\Y#1{\incsyms\indexcommand[$\string\big\string#1$]{#1}$\big#1$ & $\Bigg#1$ &\ttfamily\string#1}
\def\Z#1{\incsyms\indexcommand[$\string#1$]{#1}\ttfamily\string#1}
%    \end{macrocode}
%
%
% \begin{macro}{\docfile}
% NEED TO CHECK IF THIS IS NECESSARY
%    \begin{macrocode}
\def\docfile#1{\protect\texttt{\textbackslash #1}\index{#1}}
%    \end{macrocode}
% \end{macro}
%
% \begin{macro}{\bibsample}
%  Typesets a sample of bib
%    \begin{macrocode}
\newenvironment{bibsample}
  {\trivlist\samepage
   \setlength{\itemsep}{0pt}}
  {\endtrivlist}
%% doccommands
\newcommand*{\marglistfont}{\itshape}
\newcommand*{\margoptionfont}{\ttfamily}
\newcommand*{\margnotefont}{}

\newcommand*{\optionlistfont}{\bfseries}

\newcommand*{\ltxsyntaxfont}{\ttfamily}

\newcommand*{\ltxsyntaxlabelfont}{\bfseries}

\newcommand*{\changelogfont}{\normalfont}

\newcommand*{\changeloglabelfont}{\bfseries}

%% needed for listings????
\newcommand*{\verbatimfont}{\ttfamily}%


\let\displayverbfont\ttfamily

\renewcommand*{\verbatim@font}{\verbatimfamily}

\def\cmd#1{\cs{\expandafter\cmd@to@cs\string#1}}%

\def\cmd@to@cs#1#2{\char\number`#2\relax}

\newrobustcmd*{\env}[1]{\mbox{\verbatimfont\bfseries\textcolor{thegreen}{#1}}}

\newrobustcmd*{\len}[1]{\mbox{\verbatimfont\textbackslash#1}}

\newrobustcmd*{\cnt}[1]{\mbox{\verbatimfont#1}}

\newlength{\marglistsep}

\newlength{\marglistwidth}
\setlength{\marglistwidth}{(\oddsidemargin+1in)*85/100}%
\deflength{\marglistsep}{10pt}
%% This needs thorough checking as to restore previous definitions
%% of parsep we want parsep to be a bit higher than standard enumerated lists.


\global\newlength\oldparsep
\newenvironment*{marglist}
  {\setlength\oldparsep{\parsep}\list{}{%
     \parsep 3.5\p@ \@plus0\p@ \@minus\p@
     \setlength{\labelwidth}{\marglistwidth}%
     \setlength{\labelsep}{\marglistsep}%
     \setlength{\leftmargin}{0pt}%
     \renewcommand*{\makelabel}[1]{\hss\marglistfont##1}}}
  {\endlist\setlength\parsep{\oldparsep}}

% tt 
\newenvironment*{margoptionslist}
  {\setlength\oldparsep{\parsep}\list{}{%
     \parsep 3.5\p@ \@plus0\p@ \@minus\p@
     \setlength{\labelwidth}{\marglistwidth}%
     \setlength{\labelsep}{\marglistsep}%
     \setlength{\leftmargin}{0pt}%
     \renewcommand*{\makelabel}[1]{\hss\margoptionfont\detokenize{##1}}}}
  {\endlist\setlength\parsep{\oldparsep}}
  
  

\newenvironment*{keymarglist}
  {\marglist
   \setlength{\itemsep}{0pt}%
   \raggedright}
  {\endmarglist}
% color definitions
\def\colDef#1{\textcolor{themacro}{#1}}
% color for options
\def\colOpt#1{\textcolor{theblue}{#1}}
\newcommand{\option}[1]{\colOpt{#1}}
%    \end{macrocode}
% \end{macro}
%
% \section{Documentation Symbols for PGF type docs}
%
% Copyright 2006 by Till Tantau
%
% These type of documentation macros. 
%
%    \begin{macrocode}
\newenvironment{pgfmanualentry}{\list{}{\leftmargin=2em\itemindent-\leftmargin
 \def\makelabel##1{\hss##1}}}{\endlist}%%
%    \end{macrocode}
%
% \begin{pgfmanualentry}
%  \item pgfversion
%  \item test
% \end{pgfmanualentry}
%    \begin{macrocode}
\newcommand\pgfmanualentryheadline[1]{\itemsep=0pt\parskip=0pt\item\strut{#1}\par\topsep=0pt}
%    \end{macrocode}
%
%
% \begin{macro}{\pgfmanualbody} Just a helper macro to insert a 
% \cs{parskip}.
%    \begin{macrocode}
\newcommand\pgfmanualbody{\parskip3pt}
%    \end{macrocode}
% \end{macro}
%
% 
%    \begin{macrocode}
\newenvironment{pgflayout}[1]{
  \begin{pgfmanualentry}
    \pgfmanualentryheadline{\texttt{\string\pgfpagesuselayout\char`\{\declare{#1}\char`\}}\oarg{options}}
    \index{#1@\protect\texttt{#1} layout}%
    \index{Page layouts!#1@\protect\texttt{#1}}%
    \pgfmanualbody
}
{
  \end{pgfmanualentry}
}
%
%    \end{macrocode}
%
% \begin{environment}{command}
%  command environment
% The command strips the backslash and  handles the at for 
% indexing.
% \end{environment}
%    \begin{macrocode}
\newenvironment{command}[1]{
  \begin{pgfmanualentry}
    \extractcommand#1\@@
    \pgfmanualbody
}
{
  \end{pgfmanualentry}
}
%% MW: START MATH MACROS
\def\mvar#1{{\rmfamily\textit{#1}}}
\def\extractmathfunctionname#1{\extractmathfunctionname@#1(,)\tmpa\tmpb}
\def\extractmathfunctionname@#1(#2)#3\tmpb{\def\mathname{#1}}

\def\extractmathoperatorname{\begingroup\def\mvar##1{}\def\ {}\extractmathoperatorname@}
\def\extractmathoperatorname@#1{\xdef\mathname{#1}\endgroup}
\def\vskipspecial#1{\vskip#1\vskip0em}

\newenvironment{math-function}[1]{
	\begin{pgfmanualentry}
		\extractmathfunctionname{#1}
		\pgfmanualentryheadline{\texttt{#1}}%
		\index{\mathname @\protect\texttt{\mathname} math function}%
		\index{Math functions!\mathname @\protect\texttt{\mathname}}
		\pgfmanualbody
}
{
	\end{pgfmanualentry}\vskipspecial{-3em}
}
\newenvironment{math-operator}[1]{	
	\begin{pgfmanualentry}
		\extractmathoperatorname{#1}
		\pgfmanualentryheadline{\texttt{#1}}%
		\index{\mathname @\protect\texttt{\mathname} math operator}%
		\index{Math operators!\mathname @\protect\texttt{\mathname}}
    	\pgfmanualbody
}
{%
	\end{pgfmanualentry}\vskipspecial{-3em}
}
\newenvironment{math-constant}[1]{
	\begin{pgfmanualentry}
		\pgfmanualentryheadline{\texttt{#1}}%
		\index{#1@\protect\texttt{#1} math constant}%
		\index{Math constants!#1@\protect\texttt{#1}}
		\pgfmanualbody
}
{
	\end{pgfmanualentry}\vskipspecial{-3em}
}
\def\calcname{\textsc{calc}}
%% MW: END MATH MACROS
\def\extractcommand#1#2\@@{%
  \pgfmanualentryheadline{\declare{\texttt{\bfseries\string#1}}#2}%
  \removeats{#1}%
  %%\index{\strippedat @\protect\myprintocmmand{\strippedat}}
}

\@ifundefined{environment}{
\newenvironment{environment}[1]{
  \begin{pgfmanualentry}
    \extractenvironement#1\@@
    \pgfmanualbody
}
{
  \end{pgfmanualentry}
}}{}%
 %
\renewenvironment{environment}[1]{
  \begin{pgfmanualentry}
    \extractenvironement#1\@@
    \pgfmanualbody
}
{
  \end{pgfmanualentry}
}

\def\extractenvironement#1#2\@@{%
  \pgfmanualentryheadline{{\ttfamily\char`\\begin\char`\{\declare{#1}\char`\}}#2}%
  \pgfmanualentryheadline{{\ttfamily\ \ }\meta{environment contents}}%
  \pgfmanualentryheadline{{\ttfamily\char`\\end\char`\{\declare{#1}\char`\}}}%
  \index{#1@\protect\texttt{#1} environment}%
  \index{Environments!#1@\protect\texttt{#1}}}


\newenvironment{plainenvironment}[1]{
  \begin{pgfmanualentry}
    \extractplainenvironement#1\@@
    \pgfmanualbody
}
{
  \end{pgfmanualentry}
}

\def\extractplainenvironement#1#2\@@{%
  \pgfmanualentryheadline{{\ttfamily\declare{\char`\\#1}}#2}%
  \pgfmanualentryheadline{{\ttfamily\ \ }\meta{environment contents}}%
  \pgfmanualentryheadline{{\ttfamily\declare{\char`\\end#1}}}%
  \index{#1@\protect\texttt{#1} environment}%
  \index{Environments!#1@\protect\texttt{#1}}}


\newenvironment{contextenvironment}[1]{
  \begin{pgfmanualentry}
    \extractcontextenvironement#1\@@
    \pgfmanualbody
}
{
  \end{pgfmanualentry}
}

\def\extractcontextenvironement#1#2\@@{%
  \pgfmanualentryheadline{{\ttfamily\declare{\char`\\start#1}}#2}%
  \pgfmanualentryheadline{{\ttfamily\ \ }\meta{environment contents}}%
  \pgfmanualentryheadline{{\ttfamily\declare{\char`\\stop#1}}}%
  \index{#1@\protect\texttt{#1} environment}%
  \index{Environments!#1@\protect\texttt{#1}}}


\newenvironment{shape}[1]{
  \begin{pgfmanualentry}
  	\pgfmanualentryheadline{Shape {\ttfamily\declare{#1}}}%
    \index{#1@\protect\texttt{#1} shape}%
    \index{Shapes!#1@\protect\texttt{#1}}
    \pgfmanualbody
}
{
  \end{pgfmanualentry}
}


\newenvironment{handler}[1]{
  \begin{pgfmanualentry}
    \extracthandler#1\@nil%
    \pgfmanualbody
}
{
  \end{pgfmanualentry}
}

%% Changed must watch out!!
\def\gobble#1{}
\def\extracthandler#1#2\@nil{%
  \pgfmanualentryheadline{Key handler \meta{key}{\bfseries\ttfamily/\declare{#1}}#2}%
  \index{\gobble#1@\protect\texttt{#1} handler}%
  \index{Key handlers>#1=\protect\texttt{#1}}
}
\newenvironment{stylekey}[1]{
  \begin{pgfmanualentry}
    \def\extrakeytext{style, }
    \extractkey#1\@nil%
    \pgfmanualbody
}
{
  \end{pgfmanualentry}
}
%    \end{macrocode}
%
%    \begin{macrocode}
\newenvironment{key}[1]{
  \begin{pgfmanualentry}
    \def\extrakeytext{}
    %\def\altpath{\emph{\color{gray}or}}%
    \extractkey#1\@nil%
    \pgfmanualbody
}
{
  \end{pgfmanualentry}
}
%    \end{macrocode}

%    \begin{macrocode}
\def\extractkey#1\@nil{%
  \pgfutil@in@={#1}%
  \ifpgfutil@in@%
    \extractkeyequal#1\@nil
  \else%
    \pgfutil@in@{(initial}{#1}%
    \ifpgfutil@in@%
      \extractequalinitial#1\@nil%
    \else
      \pgfmanualentryheadline{{\ttfamily\declare{#1}}\hfill(\extrakeytext no value)}%
      \def\mykey{#1}%
      \def\mypath{}%
      \def\myname{}%
      \firsttimetrue%
      \decompose#1/\nil%ERROR?
    \fi
  \fi%
}

\def\extractkeyequal#1=#2\@nil{%
  \pgfutil@in@{(default}{#2}%
  \ifpgfutil@in@%
    \extractdefault{#1}#2\@nil%
  \else%
    \pgfutil@in@{(initial}{#2}%
    \ifpgfutil@in@%
      \extractinitial{#1}#2\@nil%
    \else
      \pgfmanualentryheadline{{\ttfamily\declare{#1}=}#2\hfill(\extrakeytext no default)}%
    \fi%
  \fi%
  \def\mykey{#1}%
  \def\mypath{}%
  \def\myname{}%
  \firsttimetrue%
  \decompose#1/\nil%
}

\def\extractdefault#1#2(default #3)\@nil{%
  \pgfmanualentryheadline{{\ttfamily\declare{#1}\opt{=}}\opt{#2}\hfill (\extrakeytext default {\ttfamily#3})}%
}

\def\extractinitial#1#2(initially #3)\@nil{%
  \pgfmanualentryheadline{{\ttfamily\declare{#1}=}#2\hfill (\extrakeytext no default, initially {\ttfamily#3})}%
}

\def\extractequalinitial#1 (initially #2)\@nil{%
  \pgfmanualentryheadline{{\ttfamily\declare{#1}}\hfill (\extrakeytext initially {\ttfamily#2})}%
  \def\mykey{#1}%
  \def\mypath{}%
  \def\myname{}%
  \firsttimetrue%
  \decompose#1/\nil%
}

\def\keyalias#1{\vspace{-3pt}\item{\small alias {\ttfamily/#1/\myname}}\vspace{-2pt}\par}

\newif\iffirsttime


\def\decompose/#1/#2\nil{%
  \def\test{#2}%
  \ifx\test\empty%
    % aha.
    \index{#1=\protect\texttt{#1} key}%@=
    \index{\mypath#1=\protect\texttt{#1}}%@
    \def\myname{#1}%
  \else%
    \iffirsttime
      \def\mypath{#1@\protect\texttt{/#1/}!}%
      \firsttimefalse
    \else
      \expandafter\def\expandafter\mypath\expandafter{\mypath#1@\protect\texttt{#1/}!}%
    \fi
    \def\firsttime{}
    \decompose/#2\nil%
  \fi%
}


\newenvironment{predefinednode}[1]{
  \begin{pgfmanualentry}
    \pgfmanualentryheadline{Predefined node {\ttfamily\declare{#1}}}%
    \index{#1=\protect\texttt{#1} node}%=
    \index{Predefined node!#1=\protect\texttt{#1}}=
    \pgfmanualbody
}
{
  \end{pgfmanualentry}
}

\newenvironment{coordinatesystem}[1]{
  \begin{pgfmanualentry}
    \pgfmanualentryheadline{Coordinate system {\ttfamily\declare{#1}}}%
    \index{#1@\protect\texttt{#1} coordinate system}%
    \index{Coordinate systems!#1@\protect\texttt{#1}}
    \pgfmanualbody
}
{
  \end{pgfmanualentry}
}


\newenvironment{decoration}[1]{
  \begin{pgfmanualentry}
    \pgfmanualentryheadline{Decoration {\ttfamily\declare{#1}}}%
    \index{#1@\protect\texttt{#1} decoration}%
    \index{Decorations!#1@\protect\texttt{#1}}
    \pgfmanualbody
}
{
  \end{pgfmanualentry}
}


\def\pgfmanualbar{\char`\|}

\newenvironment{pathoperation}[3][]{
  \begin{pgfmanualentry}
    \pgfmanualentryheadline{\textcolor{gray}{{\ttfamily\char`\\path}\
        \ \dots}
      \declare{\texttt{#2}}#3\ \textcolor{gray}{\dots\texttt{;}}}%
    \def\pgfmanualtest{#1}%
    \ifx\pgfmanualtest\@empty%
      \index{#2=\protect\texttt{#2} path operation}%=
      \index{Path operations!#2=\protect\texttt{#2}}%=
    \fi%
    \pgfmanualbody
}
{
  \end{pgfmanualentry}
}


\def\extractcommand#1#2\@@{%
  \pgfmanualentryheadline{\declare{\texttt{\string#1}}#2}%
  \removeats{#1}%
  \index{\strippedat @\protect\myprintocmmand{\strippedat}}}

\def\doublebs{\texttt{\char`\\\char`\\}}


\newenvironment{package}[1]{
  \begin{pgfmanualentry}
    \pgfmanualentryheadline{{\ttfamily\char`\\usepackage\char`\{\declare{#1}\char`\}\space\space \char`\%\space\space  \LaTeX}}
    \index{#1@\protect\texttt{#1} package}%
    \index{Packages and files!#1@\protect\texttt{#1}}%
    \pgfmanualentryheadline{{\ttfamily\char`\\input \declare{#1}.tex\space\space\space \char`\%\space\space  plain \TeX}}
    \pgfmanualentryheadline{{\ttfamily\char`\\usemodule[\declare{#1}]\space\space \char`\%\space\space  Con\TeX t}}
    \pgfmanualbody
}
{
  \end{pgfmanualentry}
}


\newenvironment{pgfmodule}[1]{
  \begin{pgfmanualentry}
    \pgfmanualentryheadline{{\ttfamily\char`\\usepgfmodule\char`\{\declare{#1}\char`\}\space\space\space
        \char`\%\space\space  \LaTeX\space and plain \TeX\space and pure pgf}}
    \index{#1@\protect\texttt{#1} module}%
    \index{Modules!#1@\protect\texttt{#1}}%
    \pgfmanualentryheadline{{\ttfamily\char`\\usepgfmodule[\declare{#1}]\space\space \char`\%\space\space  Con\TeX t\space and pure pgf}}
    \pgfmanualbody
}
{
  \end{pgfmanualentry}
}

\newenvironment{pgflibrary}[1]{
  \begin{pgfmanualentry}
    \pgfmanualentryheadline{{\ttfamily\char`\\usepgflibrary\char`\{\declare{#1}\char`\}\space\space\space
        \char`\%\space\space  \LaTeX\space and plain \TeX\space and pure pgf}}
    \index{#1@\protect\texttt{#1} library}%
    \index{Libraries!#1@\protect\texttt{#1}}%
    \pgfmanualentryheadline{{\ttfamily\char`\\usepgflibrary[\declare{#1}]\space\space \char`\%\space\space  Con\TeX t\space and pure pgf}}
    \pgfmanualentryheadline{{\ttfamily\char`\\usetikzlibrary\char`\{\declare{#1}\char`\}\space\space
        \char`\%\space\space  \LaTeX\space and plain \TeX\space when using \tikzname}}
    \pgfmanualentryheadline{{\ttfamily\char`\\usetikzlibrary[\declare{#1}]\space
        \char`\%\space\space  Con\TeX t\space when using \tikzname}}
    \pgfmanualbody
}
{
  \end{pgfmanualentry}
}

\newenvironment{tikzlibrary}[1]{
  \begin{pgfmanualentry}
    \pgfmanualentryheadline{{\ttfamily\char`\\usetikzlibrary\char`\{\declare{#1}\char`\}\space\space \char`\%\space\space  \LaTeX\space and plain \TeX}}
    \index{#1@\protect\texttt{#1} library}%
    \index{Libraries!#1@\protect\texttt{#1}}%
    \pgfmanualentryheadline{{\ttfamily\char`\\usetikzlibrary[\declare{#1}]\space \char`\%\space\space Con\TeX t}}
    \pgfmanualbody
}
{
  \end{pgfmanualentry}
}



\newenvironment{filedescription}[1]{
  \begin{pgfmanualentry}
    \pgfmanualentryheadline{File {\ttfamily\declare{#1}}}%
    \index{#1@\protect\texttt{#1} file}%
    \index{Packages and files!#1@\protect\texttt{#1}}%
    \pgfmanualbody
}
{
  \end{pgfmanualentry}
}


\newenvironment{packageoption}[1]{
  \begin{pgfmanualentry}
    \pgfmanualentryheadline{{\ttfamily\char`\\usepackage[\declare{#1}]\char`\{pgf\char`\}}}
    \index{#1@\protect\texttt{#1} package option}%
    \index{Package options for \textsc{pgf}!#1@\protect\texttt{#1}}%
    \pgfmanualbody
}
{
  \end{pgfmanualentry}
}

\newcommand\opt[1]{{\color{black!50!green}#1}}
\newcommand\ooarg[1]{{\ttfamily[}\meta{#1}{\ttfamily]}}

\def\opt{\afterassignment\pgfmanualopt\let\next=}
\def\pgfmanualopt{\ifx\next\bgroup\bgroup\color{black!50!green}\else{\color{black!50!green}\next}\fi}



\def\beamer{\textsc{beamer}}
\def\pdf{\textsc{pdf}}
\def\pgfname{\textsc{pgf}\xspace}
\def\tikzname{Ti\emph{k}Z\xspace}
\def\pstricks{\textsc{pstricks}}
\def\prosper{\textsc{prosper}}
\def\seminar{\textsc{seminar}}
\def\texpower{\textsc{texpower}}
\def\foils{\textsc{foils}}

{
  \makeatletter
  \global\let\myempty=\@empty
  \global\let\mygobble=\@gobble
  \catcode`\@=12
  \gdef\getridofats#1@#2\relax{%
    \def\getridtest{#2}%
    \ifx\getridtest\myempty%
      \expandafter\def\expandafter\strippedat\expandafter{\strippedat#1}
    \else%
      \expandafter\def\expandafter\strippedat\expandafter{\strippedat#1\protect\printanat}
      \getridofats#2\relax%
    \fi%
  }

  \gdef\removeats#1{%
    \let\strippedat\myempty%
    \edef\strippedtext{\stripcommand#1}%
    \expandafter\getridofats\strippedtext @\relax%
  }
  
  \gdef\stripcommand#1{\expandafter\mygobble\string#1}
}

\def\printanat{\char`\@}

\def\declare{\afterassignment\pgfmanualdeclare\let\next=}
\def\pgfmanualdeclare{\ifx\next\bgroup\bgroup\color{red!75!black}\else{\color{red!75!black}\next}\fi}


\let\textoken=\command
\let\endtextoken=\endcommand

\def\myprintocmmand#1{\texttt{\char`\\#1}}

\def\example{\par\smallskip\noindent\textit{Example: }}
\def\themeauthor{\par\smallskip\noindent\textit{Theme author: }}


\def\indexoption#1{%
  \index{#1@\protect\texttt{#1} option}%
  \index{Graphic options and styles!#1@\protect\texttt{#1}}%
}

\def\itemcalendaroption#1{\item \declare{\texttt{#1}}%
  \index{#1@\protect\texttt{#1} date test}%
  \index{Date tests!#1@\protect\texttt{#1}}%
}



\def\class#1{\list{}{\leftmargin=2em\itemindent-\leftmargin\def\makelabel##1{\hss##1}}%
\extractclass#1@\par\topsep=0pt}

\def\endclass{\endlist}

\def\extractclass#1#2@{%
\item{{{\ttfamily\char`\\documentclass}#2{\ttfamily\char`\{\declare{#1}\char`\}}}}%
  \index{#1@\protect\texttt{#1} class}%
  \index{Classes!#1@\protect\texttt{#1}}}



\def\index@prologue{\section*{Index}\addcontentsline{toc}{section}{Index}
  This index only contains automatically generated entries. A good
  index should also contain carefully selected keywords. This index is
  not a good index.
  \bigskip
}
\@ifundefined{c@IndexColumns}{\newcount\c@IndexColumns}{}
\c@IndexColumns=2
  \def\theindex{\@restonecoltrue
    \columnseprule \z@  \columnsep 29\p@
    \twocolumn[\index@prologue]%
       \parindent -30pt
       \columnsep 15pt
       \parskip 0pt plus 1pt
       \leftskip 30pt
       \rightskip 0pt plus 2cm
       \small
       \def\@idxitem{\par}%
    \let\item\@idxitem \ignorespaces}
  \def\endtheindex{\onecolumn}
\def\noindexing{\let\index=\@gobble}



\newcommand\symarrow[1]{
  \index{#1@\protect\texttt{#1} arrow tip}%
  \index{Arrow tips!#1@\protect\texttt{#1}}
  \texttt{#1}& yields thick  
  \begin{tikzpicture}[arrows={#1-#1},thick,baseline]
    \useasboundingbox (0pt,-0.5ex) rectangle (1cm,2ex);
    \draw (0pt,.5ex) -- (1cm,.5ex);
  \end{tikzpicture} and thin
  \begin{tikzpicture}[arrows={#1-#1},thin,baseline]
    \useasboundingbox (0pt,-0.5ex) rectangle (1cm,2ex);
    \draw (0pt,.5ex) -- (1cm,.5ex);
  \end{tikzpicture}
}

\newcommand\sarrow[2]{
  \index{#1@\protect\texttt{#1} arrow tip}%
  \index{Arrow tips!#1@\protect\texttt{#1}}
  \index{#2@\protect\texttt{#2} arrow tip}%
  \index{Arrow tips!#2@\protect\texttt{#2}}
  \texttt{#1-#2}& yields thick  
  \begin{tikzpicture}[arrows={#1-#2},thick,baseline]
    \useasboundingbox (0pt,-0.5ex) rectangle (1cm,2ex);
    \draw (0pt,.5ex) -- (1cm,.5ex);
  \end{tikzpicture} and thin
  \begin{tikzpicture}[arrows={#1-#2},thin,baseline]
    \useasboundingbox (0pt,-0.5ex) rectangle (1cm,2ex);
    \draw (0pt,.5ex) -- (1cm,.5ex);
  \end{tikzpicture}
}

\newcommand\carrow[1]{
  \index{#1@\protect\texttt{#1} arrow tip}%
  \index{Arrow tips!#1@\protect\texttt{#1}}
  \texttt{#1}& yields for line width 1ex
  \begin{tikzpicture}[arrows={#1-#1},line width=1ex,baseline]
    \useasboundingbox (0pt,-0.5ex) rectangle (1.5cm,2ex);
    \draw (0pt,.5ex) -- (1.5cm,.5ex);
  \end{tikzpicture}
}
\def\myvbar{\char`\|}
\newcommand\plotmarkentry[1]{%
  \index{#1@\protect\texttt{#1} plot mark}%
  \index{Plot marks!#1@\protect\texttt{#1}}
  \texttt{\char`\\pgfuseplotmark\char`\{\declare{#1}\char`\}} &
  \tikz\draw[color=black!25] plot[mark=#1,mark options={fill=examplefill,draw=black}] coordinates{(0,0) (.5,0.2) (1,0) (1.5,0.2)};\\
}
\newcommand\plotmarkentrytikz[1]{%
  \index{#1@\protect\texttt{#1} plot mark}%
  \index{Plot marks!#1@\protect\texttt{#1}}
  \texttt{mark=\declare{#1}} & \tikz\draw[color=black!25]
  plot[mark=#1,mark options={fill=examplefill,draw=black}] 
    coordinates {(0,0) (.5,0.2) (1,0) (1.5,0.2)};\\
}



\ifx\scantokens\@undefined
  \PackageError{phd}{You need to use extended latex
    (elatex) or (pdfelatex) to process this document}{}
\fi

\begingroup
\catcode`|=0
\catcode`[= 1
\catcode`]=2
\catcode`\{=12
\catcode `\}=12
\catcode`\\=12 |gdef|find@example#1\end{codeexample}[|endofcodeexample[#1]]
|endgroup

\begingroup
\catcode`\^=7
\catcode`\^^M=13
\catcode`\ =13%
\gdef\returntospace{\catcode`\ =13\def {\space}\catcode`\^^M=13\def^^M{}}%
\endgroup

\begingroup
\catcode`\%=13
\catcode`\^^M=13
\gdef\commenthandler{\catcode`\%=13\def%{\@gobble@till@return}}
\gdef\@gobble@till@return#1^^M{}
\gdef\@gobble@till@return@ignore#1^^M{\ignorespaces}
\gdef\typesetcomment{\catcode`\%=13\def%{\@typeset@till@return}}
\gdef\@typeset@till@return#1^^M{{\def%{\char`\%}\textsl{\char`\%#1}}\par}
\endgroup

\define@key{codeexample}{width}{\setlength\codeexamplewidth{#1}}
\define@key{codeexample}{graphic}{\colorlet{codebackground}{#1}}
\define@key{codeexample}{code}{\colorlet{codebackground}{#1}}
\define@key{codeexample}{execute code}{\csname code@execute#1\endcsname}
\define@key{codeexample}{code only}[]{\code@executefalse}
\define@key{codeexample}{pre}{\def\code@pre{#1}}
\define@key{codeexample}{post}{\def\code@post{#1}}
\define@key{codeexample}{vbox}[]{\def\code@pre{\vbox\bgroup\setlength{\hsize}{\linewidth-6pt}}\def\code@post{\egroup}}
\define@key{codeexample}{ignorespaces}[]{\let\@gobble@till@return=\@gobble@till@return@ignore}
\define@key{codeexample}{leave comments}[]{\def\code@catcode@hook{\catcode`\%=12}\let\commenthandler=\relax\let\typesetcomment=\relax}
\def\code@pre{}
\def\code@post{}
\def\code@catcode@hook{}

\newdimen\codeexamplewidth
\newif\ifcode@execute
\newbox\codeexamplebox
\def\codeexample[#1]{%
  \begingroup%
  \code@executetrue
  \setlength\codeexamplewidth{4cm+7pt}
  \setkeys{codeexample}{#1}%
  \parindent0pt
  \begingroup%
  \par%
  \medskip%
  \let\do\@makeother%
  \dospecials%
  \obeylines%
  \@vobeyspaces%
  \catcode`\%=13%
  \catcode`\^^M=13%
  \code@catcode@hook%
  \relax%
  \find@example}
\def\endofcodeexample#1{%
  \endgroup%
  \ifcode@execute%
    \setbox\codeexamplebox=\hbox{%
      {%
        {%
          \returntospace%
          \commenthandler%
          \xdef\code@temp{#1}% removes returns and comments
        }%
        \colorbox{codebackground}{\color{black}\ignorespaces%
          \code@pre\expandafter\scantokens\expandafter{\code@temp\ignorespaces}\code@post\ignorespaces}%
      }%
    }%
    \ifdim\wd\codeexamplebox>\codeexamplewidth%
      \def\code@start{\par}%
      \def\code@flushstart{}\def\code@flushend{}%
      \def\code@mid{\parskip2pt\par\noindent}%
      \def\code@width{\linewidth-6pt}%
      \def\code@end{}%
    \else%
      \def\code@start{%
        \linewidth=\textwidth%
        \parshape \@ne 0pt \linewidth
        \leavevmode%
        \hbox\bgroup}%
      \def\code@flushstart{\hfill}%
      \def\code@flushend{\hbox{}}%
      \def\code@mid{\hskip6pt}%
      \def\code@width{\linewidth-12pt-\codeexamplewidth}%
      \def\code@end{\egroup}%
    \fi%
    \code@start%
    \noindent%
    \begin{minipage}[t]{\codeexamplewidth}\raggedright
      \hrule width0pt%
      \footnotesize\vskip-1em%
      \code@flushstart\box\codeexamplebox\code@flushend%
      \vskip-1ex
      \leavevmode%
    \end{minipage}%
  \else%
    \def\code@mid{\par}
    \def\code@width{\linewidth-6pt}
    \def\code@end{}
  \fi%
  \code@mid%  
  \colorbox{codebackground}{%
    \begin{minipage}[t]{\code@width}%
      {%
        \let\do\@makeother
        \dospecials
        \frenchspacing\@vobeyspaces
        \normalfont\ttfamily\footnotesize
        \typesetcomment%
        \@tempswafalse
        \def\par{%
          \if@tempswa
          \leavevmode \null \@@par\penalty\interlinepenalty
          \else
          \@tempswatrue
          \ifhmode\@@par\penalty\interlinepenalty\fi
          \fi}%
        \obeylines
        \everypar \expandafter{\the\everypar \unpenalty}%
        #1}
    \end{minipage}}%
  \code@end%
  \par%
  \medskip
  \end{codeexample}
}

\def\endcodeexample{\endgroup}
%    \end{macrocode}
%
% \begin{macro}{codeexample}
% From pgfplots manual
% \end{macro}
%    \begin{macrocode}
\long\def\codeexamplenl{\noexpand\par}%
\pgfqkeys{/codeexample}{%
	every codeexample/.style={
		width=3.9cm,
		/pgfplots/every axis/.append style={legend style={fill=codebackground}}
	},
	narrow/.style={width=6.9cm},
	%tabsize=4,
	%pre={\begin{minipage}{\linewidth}\begingroup},
	%post={\endgroup\end{minipage}},
	%vbox,
	%newline=\codeexamplenl,
}
%%% Local Variables: 
%%% mode: latex
%%% TeX-master: "beameruserguide"
%%% End: 
%    \end{macrocode}
%
%
%
% \section{Phonetic Symbols}
% \subsection{Tipa}
%
% Users that make extensive use of the Tipa symbols would
% probably have no use for this package, however now and then
% these symbols can be useful when definining words and their
% pronunciation. 
%\href{http://tex.stackexchange.com/questions/36542/using-tex-for-writing-papers-on-linguistics}{using Tex for linguistics}
%
% I am indebted to egreg at \url{http://tex.stackexchange.com/questions/64830/using-tipa-with-fontspec} for the hack to get tipa to work with fontspec.
% The \pkgname{Tipa} was developed by Rei Fukui at the Graduate School  of Humanities and Sociology,
% The University of Tokyo \cite{tipa}.

%    \begin{macrocode}
\newif\ifTIPA 
\newcommand\TIPA{\pkgname{tipa}}
\newcommand\WIPA{\pkgname{wipa}}
\ifxetex
\else
  \ifluatex
  \else
    \TIPAtrue
    \RequirePackage[tone,extra,safe]{tipa}
  \fi
\fi
%    \end{macrocode}
% 
% This is also quite useful for Wikipedia transcriptions. 
% For example `phonetics' is pronounced as  |\textipa{\sffamily f@"nEtIks}| and typed as
% |\textipa{\sffamily f@"nEtIks}|
%
% |texdoc tipaman| for the full manual if this is part of your field
% of research.
% 
% \section{Referencing}
%
% Most authors that use \LaTeXe\ develop shorthands for common tasks such as, typing
% |See figure~\ref{fig:myplot}|. The advantage of a macro is that one can be consistent
% with capitalization or abbreviations.
%
% At first I thought of providing two macros for example \cs{sref} and \cs{Sref}, however
% the problem with such an approach is internationalization. If we allow the user to
% load her language then we need to pick-up the name from the \LaTeX2e\ definitions. There
% is also the additional issue that for paragraphs and sections, sometimes people prefer
% using an abbreviation. So we stay with lowercase commands and rather set the names using
% keys in the style settings file.
% 
%    \begin{macrocode}
\cxset{ref sectionname/.store in =\refsectionname@cx,
       ref chaptername/.store in =\refchaptername@cx,
       ref appendixname/.store in = \refappendixname@cx,
       ref equationname/.store in = \refequationname@cx,
       ref figurename/.store in = \reffigurename@cx,
       ref tablename/.store in = \reftablename@cx,
       ref paragraphname/.store in =\refparagraphname@cx,
       ref examplename/.store in=\refexamplename@cx,
}
\cxset{ref sectionname = \S\thinspace,
       ref chaptername = Chapter,
       ref appendixname = \appendixname,
       ref equationname = Equation,
       ref figurename = \figurename,
       ref tablename  = \tablename,
       ref paragraphname = \P,
       ref examplename=Example,
}
\newcommand{\fref}[1]{\reffigurename@cx~\ref{#1}}
\newcommand{\tref}[1]{\tablename~\ref{#1}}
\newcommand{\eref}[1]{equation~\ref{#1}}
\@ifundefined{cref}{\newcommand{\cref}[1]{chapter~\ref{#1}}}{}
\newcommand{\sref}[1]{\refsectionname@cx\ref{#1}}
\newcommand{\aref}[1]{\refappendixname@cx~\ref{#1}}
\newcommand{\pref}[1]{\refparagraphname@cx\ref{#1}}
\newcommand\seeref[1]{\textit{see} \textbf{\ref{#1}}}
%    \end{macrocode}
%
%
%\subsection{meta}\label{meta}
%
% This has been lifted from |doc|
% \begin{macro}{\meta}
%    The |\meta| macro is a bit tricky. We want to allow line
%    breaks at blanks in the argument but we don't want a break
%    in between. In the past this was done by defining |\meta| in a way that a
%    \verb*+ + is active when the argument is scanned. Words are then
%    scanned into |\hbox|es. The active \verb*+ + will end the
%    preceding |\hbox| add an ordinary space and open a new
%    |\hbox|. In this way breaks are only possible at spaces.  The
%    disadvantage of this method was that |\meta| was neither robust
%    nor could it be |\protect|ed. The new implementation  fixes this
%    problem by defining |\meta| in a radically different way: we
%    prevent hypenation by defining a |\language| which has no
%    patterns associated with it and use this to typeset the words
%    within the angle brackets. see \sref{meta}
% 
%    \begin{macrocode}
\ifx\l@nohyphenation\undefined
  \newlanguage\l@nohyphenation
\fi
%    \end{macrocode}
%    
%    \begin{macrocode}
\DeclareRobustCommand\meta[1]{%
%    \end{macrocode}
%    Since the old implementation of |\meta| could be used in math we
%    better ensure that this is possible with the new one as
%    well. So we use |\ensuremath| around |\langle| and
%    |\rangle|. However this is not enough: if |\meta@font@select|
%    below expands to |\itshape| it will fail if used in math
%    mode. For this reason we hide the whole thing inside an
%    |\nfss@text| box in that case.
%    \begin{macrocode}
     \ensuremath\langle
     \ifmmode \expandafter \nfss@text \fi
     {%
      \meta@font@select
%    \end{macrocode}
%    Need to keep track of what we changed just in case the user
%    changes font inside the argument so we store the font explicitly.
%    \begin{macrocode}
      \edef\meta@hyphen@restore
        {\hyphenchar\the\font\the\hyphenchar\font}%
      \hyphenchar\font\m@ne
      \language\l@nohyphenation
      #1\/%
      \meta@hyphen@restore
     }\ensuremath\rangle
}
%    \end{macrocode}
% \end{macro}
%
%
% \begin{macro}{\meta@font@select} 
%  	We default the definition to upshape.
%    \begin{macrocode}
\def\meta@font@select{\upshape}
%    \end{macrocode}
% \end{macro}
%
% \begin{macro}{\macro} 
%	The \cs{macro} environment is straight out of the
%	\pkg{doc} also. We redefine it here to allow usage in documents that have not 
%	preloaded the package.
%
%    \begin{macrocode}
\def\macro{\begingroup
   \catcode`\\12
   \MakePrivateLetters \m@cro@ \iftrue}
%    \end{macrocode}
% \end{macro}
%
% \begin{macro}{\environment}
%    The ``environment'' envrironment will be implemented just like the
%    ``macro'' environment flagging any differences in the code by
%    passing |\iffalse| or |\iftrue| to the |\m@cro@| environment
%    doing the actual work.
%    \begin{macrocode}
\def\environment{\begingroup
   \catcode`\\12
   \MakePrivateLetters \m@cro@ \iffalse}
%    \end{macrocode}
% \end{macro}
%
%    After scanning the argument we close the group to get the normal
%    |\catcode|$\,$s back. Then we assign a special value to
%    |\topsep| and start a \textsf{trivlist} environment. (Modified for normal indexing by YL)
%
%    \begin{macrocode}
\long\def\m@cro@#1#2{\index{\string#2}\endgroup \topsep\MacroTopsep \trivlist
%    \end{macrocode}
% We also save the name being described in |\saved@macroname| for
% 
%    \begin{macrocode}
   \edef\saved@macroname{\string#2}%
%    \end{macrocode}
%    Now there follows a variation of |\makelabel| which is used
%    should the environment not be nested, or should it lie between
%    two successive |\begin{macro}| instructions or explanatory
%    text.  One can recognize this with the switch |\if@inlabel|
%    which will be |true| in the case of successive |\item|
%    commands.
%
%    \begin{macrocode}
  \def\makelabel##1{\llap{##1}}%\llap
%    \end{macrocode}
%
%    If |@inlabel| is |true| and if $\verb=\macro@cnt= > 0$
%    then the above definition needs to be changed, because in this
%    case \LaTeX{} would otherwise put the labels all on the same line
%    and this would lead to them being overprinted on top of each
%    other.  Because of this |\makelabel| needs to be redefined
%    in this case.
%    \begin{macrocode}
  \if@inlabel
%    \end{macrocode}
%    If |\macro@cnt| has the value $1$, then we redefine
%    |\makelabel| so that the label will be positioned in the
%    second line of the margin.  As a result of this, two macro names
%    appear correctly, one under the other.  It's important whilst
%    doing this that the generated label box is not allowed to have
%    more depth than a normal line since otherwise the distance
%    between the first two text lines of \TeX{} will be incorrectly
%    calculated. The definition should then look like:
%\begin{verbatim}
%     \def\makelabel##1{\llap{\vtop to \baselineskip
%          {\hbox{\strut}\hbox{##1}\vss}}}
%\end{verbatim}
%    Completely analogous to this is the case where labels need to be
%    placed one under the other.  The lines above are only an example
%    typeset with the \textsf{verbatim} environment. To produce the real
%    definition we save the value of |\macro@cnt| in
%    |\count@| and empty the temp macro |\@tempa| for later
%    use.
%    \begin{macrocode}
    \let\@tempa\@empty \count@\macro@cnt
%    \end{macrocode}
%    In the following loop we append for every already typeset label
%    an |\hbox{\strut}| to the definition of |\@tempa|.
%    \begin{macrocode}
    \loop \ifnum\count@>\z@
      \edef\@tempa{\@tempa\hbox{\strut}}\advance\count@\m@ne \repeat
%    \end{macrocode}
%    Now be put the definition of |\makelabel| together.
%
%    \begin{macrocode}
    \edef\makelabel##1{\llap{\vtop to\baselineskip
                               {\@tempa\hbox{##1}\vss}}}%
%    \end{macrocode}
%    Next we increment the value of the nesting depth counter.  This
%    value inside the \textsf{macro} environment is always at least one
%    after this point, but its toplevel definition is zero. Provided
%    this environment has been used correctly, $|\macro@cnt|=0$
%    should not occur when |@inlabel|=\textsf{true}.  It is
%    however possible if this environment is used within other list
%    environments (but this would have little point).
%    \begin{macrocode}
    \advance \macro@cnt \@ne
%    \end{macrocode}
%    If |@inlabel| is false we reset |\macro@cnt| assuming
%    that there is enough room to print the macro name without
%    shifting.
%    \begin{macrocode}
  \else  \macro@cnt\@ne  \fi
%    \end{macrocode}
%    Now the label will be produced using |\item|. The following
%    line is only a hack saving the day until a better solution is
%    implemented.  We have to face two problems: the argument might be
%    a |\par| which is forbidden in the argument of other macros
%    if they are not defined as |\long|, or it is something like
%    |\iffalse| or |\else|, i.e.\ something which will be
%    misinterpreted when \TeX{} is skipping conditional text. In both
%    cases |\item| will bomb, so we protect the argument by using
%    |\string|.
%    \begin{macrocode}
  \edef\@tempa{\noexpand\item[%
%    \end{macrocode}
%    Depending on whether we are inside a ``macro'' or ``environment''
%    environment we use |\PrintMacroName| or |\PrintEnvName| to
%    display the name.
%    \begin{macrocode}
     #1%
       \noexpand\PrintMacroName
     \else
       \noexpand\PrintEnvName
     \fi
     {\string#2}]}%
  \@tempa
%    \end{macrocode}
%    At this point we also produce an index entry.  Because it is not
%    known which index sorting program will be used, we do not use the
%    command |\index|, but rather a command
%    |\SpecialMainIndex| after advancing the counter for indexing
%    by line number.  This may be redefined by the user in
%    order to generate an index entry which will be understood by the
%    index program in use (note the definition of
%    |\SpecialMainIndex| for our installation).
%
%    We advance the current codeline number and after producing an
%    index entry revert to the original value
%    \begin{macrocode}
  \global\advance\c@CodelineNo\@ne
%    \end{macrocode}
%    Again the macro to call depends on the environment we are
%    actually in.
%    \begin{macrocode}
   #1%
      \nobreak
      \DoNotIndex{#2}%
   \else
      \SpecialMainEnvIndex{#2}\nobreak
   \fi
  \global\advance\c@CodelineNo\m@ne
%    \end{macrocode}
%    The |\nobreak| is needed to prevent a page break after the
%    |\write| produced by the |\SpecialMainIndex| macro.  We
%    exclude the new macro in the cross-referencing feature, to
%    prevent spurious non-main entry references.  Regarding possibly
%    problematic arguments, the implementation takes
%    care of |\par| and the conditionals are uncritical.
%
%    Because the space symbol should be ignored between the
%    |\begin{macro}{...}| and the following text we must take
%    care of this with |\ignorespaces|.
%    \begin{macrocode}
  \ignorespaces}
%    \end{macrocode}
% 
%	We now ready to define the code for the end of the environments.	
%	
% \begin{macro}{\endmacro}
% \begin{macro}{\endenvironment}
%     Older releases of this environment omit the |\endgroup| token,
%     when being nested. This was done to avoid unnessary stack usage.
%     However it does not work if \textsf{macro} and
%     \textsf{environment} environments are mixed, therefore we now
%     use a simpler approach.
%
%    \begin{macrocode}
\let\endmacro \endtrivlist
\let\endenvironment\endmacro
%    \end{macrocode}
%  \end{macro}
%  \end{macro}
%
% \begin{macro}{\MacroTopsep}
%    Here is the default value for the |\MacroTopsep| parameter
%    used above.
%    \begin{macrocode}
\newskip\MacroTopsep     \MacroTopsep = 7pt plus 2pt minus 2pt
%    \end{macrocode}
% \end{macro}
%
%
% \subsection{Formatting the margin}
%
% The following three macros should be user definable.
% Therefore we define those macros only if they have not already
% been defined.
%
% \begin{macro}{\PrintMacroName}
% \begin{macro}{\PrintEnvName}
% \begin{macro}{\PrintDescribeMacro}
% \begin{macro}{\PrintDescribeEnv}
%    The formatting of the macro name in the left margin is done by
%    these macros. We first set a |\strut| to get the height and
%    depth of the normal lines. Then we change to the
%    |\MacroFont| using |\string| to |\catcode| the
%    argument to other (assuming that it is a macro name). Finally we
%    print a space.  The font change remains local since this macro
%    will be called inside an |\hbox|. NEED TO FIX
%    \begin{macrocode}
\@ifundefined{PrintMacroName}
   {\def\PrintMacroName#1{\strut \MarginMacroFonts \string #1\ }}{\def\PrintMacroName#1{\strut \MarginMacroFonts \string #1\ }}
%    \end{macrocode}
%    We use the same formatting conventions when describing a macro.
%    \begin{macrocode}
\@ifundefined{PrintDescribeMacro}
   {\def\PrintDescribeMacro#1{\strut \MacroFonts \string #1\ }}{\def\PrintDescribeMacro#1{\strut \MarginMacroFonts \string #1\ }}
%    \end{macrocode}
%    To format the name of a new environment there is no need to use
%    |\string|.
%    \begin{macrocode}
\@ifundefined{PrintDescribeEnv}
   {\def\PrintDescribeEnv#1{\strut \MacroFonts #1\ }}{\def\PrintDescribeEnv#1{\strut \MarginMacroFonts #1\ }}
\@ifundefined{PrintEnvName}
   {\def\PrintEnvName#1{\strut \MarginMacroFonts #1\ }}{\def\PrintEnvName#1{\strut \MarginMacroFonts #1\ }}
%    \end{macrocode}
% \end{macro}
% \end{macro}
% \end{macro}
% \end{macro}
%
% \begin{macro}{\MarginMacroFont} As we dont care for older versions of LaTeX we simplify the
%	code provide by \pkg{doc}. We also add a hook for color.
%	as we do not expect that the package will be used in places with no colour support.
%	The command used in the original \pkg{doc} is the same as the one used to
% 	typeset the code for |macrocode|, however we wish to have the option to color
%	the margin macros separately.
%	
%    \begin{macrocode}
\ifxetex
  \def\MarginMacroFonts{\color{spot!60}\ttfamily}
\else
  \ifluatex
    \def\MarginMacroFonts{\color{spot!60}\ttfamily}
  \else
    \def\MarginMacroFonts{%
                  \fontencoding\encodingdefault
                   \fontfamily\ttdefault
                   \fontseries\mddefault
                   \fontshape\updefault
                   \color{red}\small}%
  \fi
\fi
%    \end{macrocode}
% \end{macro}
%
% 
% \section{Code demo environments}
%
%	To demonstrate LaTeX code it is sometimes desirable to have the code
%	be executed. This was pioneered in a number of packages. One of
%	the better packages to do so is \pkg{tcolorbox}. We use it to define
%	a special environment.
%

% \begin{environment}{texexample} The environment |texexample| will list the code
%	using the listings package, so we can have a nice box and shows the
%	output at the bottom section.
%	First we define a new counter which resets at every chapter. If |c@chapter|
%	is not defined we reset it based on sections.
%
% \begin{enumerate}
%	\item [\#1] Title of the example
%	\item [\#2] label for referencing
% \end{enumerate}
% 
%    \begin{macrocode}
  \ifx\c@chapter\@undefined
    \newcounter{texexp}[section]
    \@addtoreset{c@texexp}{c@section}
  \else
    \newcounter{texexp}[chapter]
    \@addtoreset{c@texexp}{c@chapter}
  \fi 
%    \end{macrocode}
%
%	
%    \begin{macrocode}
%\tcbset{listing utf8=latin1}% optional; ’latin1’ is the default.
 
\def\thetexexp{\@arabic\c@section.\arabic{texexp}}
\tcbset{texexp/.style={% 
    fonttitle=\small\ttfamily, 
    fontupper=\small, 
    fontlower=\small,
    coltitle=black,
    colback = codebackground,% background
    colframe=codebackground, 
      %colupper=spot!,
   },
   listing options = {%
     keywordstyle=\color{thekeywordstyle},
     belowskip=0pt, 
     escapeinside={(*@}{@*)},%
     breaklines=true,%
     backgroundcolor=\color{codebackground},%
     firstnumber=last,%
     stepnumber=1,%
     upquote=true,%
     alsoletter={_,:},%
     commentstyle=\color{thecommentstyle},%
     emph={cs,new,seq,map,inline,eq,gincr,incr,IfNoValueF,if,%
            If,exist,protect,nopar,gset,%
            set,undefine,define,add,gadd,remove,div,%
            round,truncate,max,min,mod,gzero,int,exp,put,left,args,%
            zero,newcount,protected,msg,error,%
            eval,to,arabic,alph,Alph,roman,Roman,dim%
            DeclareDocumentCommand,%
            NewDocumentCommand,%
            RenewDocumentCommand,includegraphics,
            function,local,return
         },%
           %
          % For LaTeX3 we need to add these, note % is important
          % dn’t miss, at the end...
          moretexcs    = {DeclareDocumentCommand,IfBooleanTF,tex_def:D,%
          cs_new:Nn,cs_new:Npn,cs_new:cn,cs_set_nopar:Npn,token_to_meaning:N,%
          %primitives
          cs:w,cs_end:,tex_underline,group_begin:, group_end:,%
          %coffins
          NewCoffin,JoinCoffins,SetHorizontalCoffin,TypesetCoffin,%
          %properties
          prop_new:N,prop_new:c,prop_put:Nnn,%
          %boolean
          bool_new:N,bool_set_true:N,bool_set_false:N,%
          bool_if:NTF,%
          hbox_to_wd:nn,%
          IfNoValueTF,%
          %token lists
          tl_new:N,tl_set:Nn,tl_concat:NNN,%
          token_to_meaning:N,%
          seq_pop_left:NN,%
          %
          %int
          int_if_exist:cT,int_use:c,int_new:c,int_new:N,int_eval:n,%
          int_add,int_use,int_to_roman,%
          %boxes
          box_new:c,hbox_set:cn,box_use:c,vbox_set:cn,box_move_down:nn,%
          %string
          str_if_eq_x:nnTF,%
          tl_tail:n,%
          DeclareObjectType,%
          DeclareTemplateInterface,%
          DeclareTemplateCode,%
          DeclareInstance,UseInstance,AssignTemplateKeys%
          keys_set,keys_define,%      
          },%
     emphstyle=\verbatimfont\bfseries\color{black!80},%
          %
   },%close listings options
      % added for better control
      arc=0pt,  
      outer arc=0pt,
      example1/.code 2 args={\refstepcounter{texexp}\label{#2}}%Reference
     \pgfkeysalso{texexp, enhanced, breakable, title={Example \thetexexp\ #1}%
 },
}
%
\newenvironment{texexp}[1]{\tcblisting{texexp,#1}}{\endtcblisting}
\newenvironment{example1}[3][]{\tcblisting{example1={#2}{#3},#1}}%
    {\endtcblisting}
%
\newenvironment{texexample}[3][]{\noindent\tcblisting{example1={#2}{#3},#1}}%
    {\endtcblisting}
%    
% Need to fix
\let\luaexample\texexample        
\let\endluaexample\endtexexample    
%    \end{macrocode}
% \end{environment}      
%    \begin{macrocode}
%\tcbset{luacode/.style={%
%      fonttitle=\small\ttfamily, 
%      fontupper=\small, 
%      fontlower=\small,
%      coltitle=black,
%      colback = codebackground,% background
%      colframe=codebackground, 
%      %colupper=spot!,
%      },
%      listing options = {
%          language={[5.2]Lua},
%          belowskip=0pt, 
%          escapeinside={(*@}{@*)},%
%          breaklines=true,%
%          backgroundcolor=\color{codebackground},%
%          firstnumber=last,%
%          stepnumber=1,%
%          upquote=true,%
%          alsoletter={_,:},%
%          commentstyle=\bfseries\color{black!90},%
%          stringstyle = \color{black!90},
%          emphstyle=\verbatimfont\bfseries\color{black!80},%
%          keywordstyle= \bfseries\color{black!80},%
%          },
%      % added for better control
%      arc=0pt,  
%      outer arc=0pt,
%      luaexp1/.code 2 args={\refstepcounter{texexp}\label{#2}}%Reference
%     \pgfkeysalso{luacode, enhanced, breakable, title={Example \thetexexp\ #1}},
%}
%\newenvironment{luaexp1}[1]{\tcblisting{luacode,#1}}{\endtcblisting}
%
%\newenvironment{luaexample}[3][]{\noindent\tcblisting{luaexp1={#2}{#3},#1}}%
%    {\endtcblisting}
%%
%    \end{macrocode} 
%
% The following demonstrates the usage.
%
% 	\begin{texexample}[]{atest}{This is a comment?}
%	  \def\demomacro{Hello World!}
%	\end{texexample}
%
% 	\begin{example}{A Test}{test}{This is a comment?}
%	  \def\demomacro{Hello World!}
%	\end{example}
%
%
%
% \section{Floats settings} 
%                   
% We use Donald Arseneau's improved float parameters. I am not too sure when this was first referenced
% once I find it, will provide a citation and or a link.
% 
% For some of the rationale behind |topfraction| values see \ref{topfraction}.
%    \begin{macrocode}
\renewcommand{\topfraction}{.85}
\renewcommand{\bottomfraction}{.7} % .3 in kernel.
\renewcommand{\textfraction}{.15}
\renewcommand{\floatpagefraction}{.7}
\renewcommand{\dbltopfraction}{.66}
\renewcommand{\dblfloatpagefraction}{.66}
\setcounter{topnumber}{9}
\setcounter{bottomnumber}{9}
\setcounter{totalnumber}{20}
\setcounter{dbltopnumber}{9}
%    \end{macrocode}
%	
%\section{Hyphenation}
%
% The below have been mostly lifted from TUGBoat class. I would appreciate
% any additions. See also \url{http://ctan.um.ac.ir/info/digests/tugboat/hyphenex/tb0hyf.pdf}
%
% Hyphenation exceptions for US English,
% based on hyphenation exception log articles in TUGboat.
%
% Copyright 2008 TeX Users Group.
% You may freely use, modify and/or distribute this file.
%
% Please contact the TUGboat editorial staff <tugboat@tug.org>
% for corrections and omissions.
% We also assume that the class has provided some decent hyphenation rules
% I acknowledge that this might not be totally satisfactory and one needs
% to think of a better way in a future version.
%
% \index{hyphenation}
%    \begin{macrocode}
\ifx\tubomithyphenations\@thisisundefined
\hyphenation{%
  acad-e-my
  acad-e-mies
  ac-cu-sa-tive
  acro-nym
  acro-nyms
  acryl-amide
  acryl-amides
  acryl-alde-hyde
  acu-punc-ture
  acu-punc-tur-ist
  add-a-ble
  add-i-ble
  adren-a-line
  aero-space
  af-ter-thought
  af-ter-thoughts
  agron-o-mist
  agron-o-mists
  al-ge-bra-i-cal-ly
  am-phet-a-mine
  am-phet-a-mines
  anach-ro-nism
  anach-ro-nis-tic
  an-a-lyse
  an-a-lysed
  analy-ses
  analy-sis
  an-eu-rysm
  an-eu-rysms
  an-eu-rys-mal
  an-iso-trop-ic
  an-iso-trop-i-cal-ly
  an-isot-ro-pism
  an-isot-ropy
  anom-aly
  anom-alies
  anti-deriv-a-tive
  anti-deriv-a-tives
  anti-holo-mor-phic
  an-tin-o-my
  an-tin-o-mies
  anti-nu-clear
  anti-nu-cle-on
  anti-rev-o-lu-tion-ary
  apoth-e-o-ses
  apoth-e-o-sis
  ap-pen-di-ces
  ap-pen-dix
  ap-pen-dixes
  ar-chi-me-dean
  ar-chi-pel-ago
  ar-chi-pel-a-gos
  ar-chive
  ar-chives
  ar-chiv-ing
  ar-chiv-ist
  ar-chiv-ists
  ar-che-typ-al
  ar-che-type
  ar-che-types
  ar-che-typ-i-cal
  arc-tan-gent
  arc-tan-gents
  as-sign-a-ble
  as-sign-or
  as-sign-ors
  as-sist-ant
  as-sist-ance
  as-sist-ant-ship
  as-sist-ant-ships
  asymp-to-matic
  as-ymp-tot-ic
  asyn-chro-nous
  ath-er-o-scle-ro-sis
  at-mos-phere
  at-mos-pheres
  at-tri-bute
  at-trib-uted
  at-trib-ut-able
  au-to-ma-tion
  au-tom-a-ton
  au-tom-a-ta
  auto-num-ber-ing
  au-ton-o-mous
  auto-re-gres-sion
  auto-re-gres-sive
  auto-round-ing
  av-oir-du-pois
  band-lead-er
  band-lead-ers
  bank-rupt
  bank-rupts
  bank-rupt-cy
  bank-rupt-cies
  bar-onies
  base-line-skip
  ba-thym-e-try
  bathy-scaphe
  bean-ies
  be-drag-gle
  be-drag-gled
  bed-rock
  be-dwarf
  be-dwarfs
  be-hav-iour
  be-hav-iours
  bevies
  bib-lio-graph-i-cal
  bib-li-og-ra-phy-style
  bib-units
  bi-dif-fer-en-tial
  big-gest
  bill-able
  bio-math-e-mat-ics
  bio-med-i-cal
  bio-med-i-cine
  bio-rhythms
  bio-weap-ons
  bio-weap-on-ry
  bit-map
  bit-maps
  bland-er
  bland-est
  blind-er
  blind-est
  blondes
  blue-print
  blue-prints
  bo-lom-e-ter
  bo-lom-e-ters
  book-sell-er
  book-sell-ers
  bool-ean
  bool-eans
  bor-no-log-i-cal
  bot-u-lism
  brusquer
  buf-fer
  buf-fers
  bun-gee
  bun-gees
  busier
  busi-est
  bussing
  butted
  buzz-word
  buzz-words
  ca-coph-o-ny
  ca-coph-o-nies
  call-er
  call-ers
  cam-era-men
  cart-wheel
  cart-wheels
  ca-tarrhs
  cat-a-stroph-ic
  cat-a-stroph-i-cally
  cat-e-noid
  cat-e-noids
  cau-li-flow-er
  chap-ar-ral
  char-treuse
  chemo-ther-apy
  chemo-ther-a-pies
  chloro-meth-ane
  chloro-meth-anes
  cho-les-teric
  cig-a-rette
  cig-a-rettes
  cinque-foil
  co-asso-cia-tive
  coch-leas
  coch-lear
  co-designer
  co-designers
  co-gnac
  co-gnacs
  co-ker-nel
  co-ker-nels
  col-lin-ea-tion
  col-umns
  com-par-and
  com-par-ands
  com-pen-dium
  com-po-nent-wise
  comp-trol-ler
  comp-trol-lers
  com-put-able
  com-put-abil-ity
  con-form-able
  con-form-ist
  con-form-ists
  con-form-ity
  con-ge-ries
  con-gress
  con-gresses
  con-struc-ti-ble
  con-struc-ti-bil-ity
  con-trib-ute
  con-trib-utes
  con-trib-uted
  copy-right-able
  co-re-la-tion
  co-re-la-tions
  co-re-li-gion-ist
  co-re-li-gion-ists
  co-re-op-sis
  co-re-spon-dent
  co-re-spon-dents
  co-se-cant
  co-semi-sim-ple
  co-tan-gent
  cour-ses
  co-work-er
  co-work-ers
  crank-case
  crank-shaft
  croc-o-dile
  croc-o-diles
  cross-hatch
  cross-hatched
  cross-hatch-ing
  cross-over
  cryp-to-gram
  cryp-to-grams
  cuff-link
  cuff-links
  cu-nei-form
  cus-tom-iz-a-ble
  cus-tom-ize
  cus-tom-izes
  cus-tom-ized
  cy-ber-virus
  cy-ber-viruses
  cy-ber-wea-pon
  cy-ber-wea-pons
  dachs-hund
  dam-sel-fly
  dam-sel-flies
  dactyl-o-gram
  dactyl-o-graph
  data-base
  data-bases
  data-path
  data-paths
  date-stamp
  date-stamps
  de-allo-cate
  de-allo-cates
  de-allo-cated
  de-allo-ca-tion
  de-allo-ca-tions
  de-clar-able
  de-fin-i-tive
  de-lec-ta-ble
  demi-semi-qua-ver
  demi-semi-qua-vers
  de-moc-ra-tism
  demos
  der-i-va-tion
  der-i-va-tions
  der-i-va-tion-al
  de-riv-a-tive
  de-riv-a-tives
  dia-lec-tic
  dia-lec-tics
  dia-lec-ti-cian
  dia-lec-ti-cians
  di-chloro-meth-ane
  dif-fract
  dif-fracts
  dif-frac-tion
  dif-frac-tions
  direr
  dire-ness
  dis-par-and
  dis-par-ands
  dis-traught-ly
  dis-trib-ut-able
  dis-trib-ute
  dis-trib-utes
  dis-trib-uted
  dis-trib-u-tive
  dou-ble-space
  dou-ble-spaced
  dou-ble-spac-ing
  doll-ish
  drift-age
  driv-ers
  drom-e-dary
  drom-e-daries
  du-op-o-list
  du-op-o-lists
  du-op-oly
  dys-lexia
  dys-lec-tic
  east-end-ers
  eco-sys-tem
  eco-sys-tems
  eco-nom-ics
  econ-o-mies
  econ-o-mist
  econ-o-mists
  ei-gen-class
  ei-gen-classes
  ei-gen-val-ue
  ei-gen-val-ues
  electro-mechan-i-cal
  electro-mechano-acoustic
  elit-ist
  elit-ists
  en-dos-copies
  en-dos-copy
  en-tre-pre-neur
  en-tre-pre-neurs
  en-tre-pre-neur-ial
  ep-i-neph-rine
  eps-to-pdf
  equi-vari-ant
  equi-vari-ance
  er-go-nom-ic
  er-go-nom-ics
  er-go-nom-i-cally
  es-sence
  es-sences
  eth-ane
  eth-yl-am-ine
  eth-yl-ate
  eth-yl-ated
  eth-yl-ene
  ethy-nyl
  ethy-nyl-a-tion
  eu-sta-chian
  ever-si-ble
  evert
  everts
  evert-ed
  evert-ing
  ex-quis-ite
  ex-tra-or-di-nary
  fall-ing
  fermi-ons
  figu-rine
  figu-rines
  fi-nite-ly
  fla-gel-lum
  fla-gel-la
  flam-ma-bles
  fledg-ling
  flow-chart
  flow-charts
  fluoro-car-bon
  fluor-os-copies
  fluor-os-copy
  for-mi-da-ble
  for-mi-da-bly
  for-syth-ia
  forth-right
  free-loader
  free-loaders
  friend-lier
  friend-li-est
  fri-vol-ity
  fri-vol-i-ties
  friv-o-lous
  front-end
  front-ends
  ga-lac-tic
  gal-axy
  gal-ax-ies
  gas-om-e-ter
  ge-o-des-ic
  ge-o-det-ic
  geo-met-ric
  geo-met-rics
  ge-o-strophic
  geo-ther-mal
  ge-ot-ro-pism
  gno-mon
  gno-mons
  grand-uncle
  grand-uncles
  griev-ance
  griev-ances
  griev-ous
  griev-ous-ly
  group-like
  hair-style
  hair-styles
  hair-styl-ist
  hair-styl-ists
  half-life
  half-lives
  half-space
  half-spaces
  half-way
  har-bin-ger
  har-bin-gers
  har-le-quin
  har-le-quins
  hatch-eries
  he-lio-pause
  he-lio-trope
  hemi-demi-semi-qua-ver
  hemi-demi-semi-qua-vers
  he-mo-glo-bin
  he-mo-phil-ia
  he-mo-phil-iac
  he-mo-phil-iacs
  hemo-rhe-ol-ogy
  he-pat-ic
  he-pat-ica
  her-maph-ro-dite
  her-maph-ro-dit-ic
  he-roes
  hexa-dec-i-mal
  holo-deck
  holo-decks
  ho-lo-no-my
  ho-meo-mor-phic
  ho-meo-mor-phism
  ho-meo-stat-ic
  ho-meo-stat-ics
  ho-meo-sta-sis
  ho-mo-thetic
  horse-rad-ish
  hot-bed
  hot-beds
  hounds-teeth
  hounds-tooth
  hy-dro-ther-mal
  hy-per-elas-tic-ity
  hy-po-elas-tic-ity
  hy-po-thal-a-mus
  ideals
  ideo-graphs
  idio-syn-crasy
  idio-syn-cra-sies
  idio-syn-cratic
  idio-syn-crat-i-cal-ly
  ig-nit-er
  ig-nit-ers
  ig-ni-tor
  ignore-spaces
  il-li-quid
  il-li-quid-ity
  im-ped-ance
  im-ped-ances
  in-du-bi-ta-ble
  in-fin-ite-ly
  in-fin-i-tes-i-mal
  in-fra-struc-ture
  in-fra-struc-tures
  input-enc
  in-stall-er
  in-stall-ers
  in-teg-rity
  in-ter-dis-ci-pli-nary
  in-ter-ga-lac-tic
  in-ter-view-ee
  in-ter-view-ees
  in-utile
  in-util-i-ty
  ir-re-duc-ible
  ir-re-duc-ibly
  ir-rev-o-ca-ble
  iso-geo-met-ric
  iso-geo-met-rics
  iso-ther-mal
  isot-ropy
  iso-trop-ic
  itin-er-ary
  itin-er-ar-ies
  je-re-mi-ads
  key-note
  key-notes
  key-stroke
  key-strokes
  kiln-ing
  lac-i-est
  lam-en-ta-ble
  land-scap-er
  land-scap-ers
  lar-ce-n
  lar-ce-ny
  lar-ce-nies
  lar-ce-nist
  leaf-hop-per
  leaf-hop-pers
  let-ter-spaces
  let-ter-spaced
  let-ter-spac-ing
  leu-ko-cyte
  leu-ko-cytes
  life-span
  life-spans
  life-style
  life-styles
  light-weight
  lim-ou-sines
  line-backer
  line-spacing
  li-on-ess
  li-quid-ity
  lith-o-graphed
  lith-o-graphs
  lo-bot-omy
  lo-bot-om-ize
  loges
  long-est
  look-ahead
  lo-quac-ity
  love-struck
  macro-eco-nomic
  macro-eco-nomics
  macro-econ-omy
  make-in-dex
  mal-a-prop-ism
  mal-a-prop-isms
  man-slaugh-ter
  man-u-script
  man-u-scripts
  mar-gin-al
  math-e-ma-ti-cian
  math-e-ma-ti-cians
  mattes
  med-ic-aid
  medi-ocre
  medi-oc-ri-ties
  mega-fau-na
  mega-fau-nal
  mega-lith
  mega-liths
  meta-bol-ic
  me-tab-o-lism
  me-tab-o-lisms
  me-tab-o-lite
  me-tab-o-lites
  meta-form
  meta-forms
  meta-lan-guage
  meta-lan-guages
  meta-phor-ic
  meta-sta-bil-ity
  meta-stable
  meta-table
  meta-tables
  meth-am-phet-a-mine
  meth-ane
  meth-od
  meth-yl-am-mo-nium
  meth-yl-ate
  meth-yl-ated
  meth-yl-a-tion
  meth-yl-ene
  me-trop-o-lis
  me-trop-o-lises
  met-ro-pol-i-tan
  met-ro-pol-i-tans
  micro-eco-nomic
  micro-eco-nomics
  micro-econ-omy
  micro-en-ter-prise
  micro-en-ter-prises
  mi-cro-fiche
  mi-cro-fiches
  micro-organ-ism
  micro-organ-isms
  mi-cro-struc-ture
  mill-age
  mil-li-liter
  mimeo-graphed
  mimeo-graphs
  mim-ic-ries
  mine-sweeper
  mine-sweepers
  min-is
  mini-sym-po-sium
  mini-sym-po-sia
  mi-nut-er
  mi-nut-est
  mis-chie-vous-ly
  mi-sers
  mi-sog-a-my
  mne-mon-ic
  mne-mon-ics
  mod-el-ling
  mol-e-cule
  mol-e-cules
  mon-archs
  money-len-der
  money-len-ders
  mono-chrome
  mono-en-er-getic
  mon-oid
  mon-oph-thong
  mon-oph-thongs
  mono-pole
  mono-poles
  mo-nop-oly
  mono-space
  mono-spaced
  mono-spacing
  mono-spline
  mono-splines
  mono-strofic
  mo-not-o-nies
  mo-not-o-nous
  mo-ron-ism
  mos-qui-to
  mos-qui-tos
  mos-qui-toes
  mud-room
  mud-rooms
  mul-ti-fac-eted
  mul-ti-plic-able
  mul-ti-plic-ably
  multi-user
  name-space
  name-spaces
  neo-fields
  neo-nazi
  neo-nazis
  neph-ews
  neph-rite
  neph-ritic
  new-est
  news-let-ter
  news-let-ters
  nil-po-tent
  nitro-meth-ane
  node-list
  node-lists
  no-name
  non-ar-ith-met-ic
  non-emer-gency
  non-equi-vari-ance
  none-the-less
  non-euclid-ean
  non-iso-mor-phic
  non-pseudo-com-pact
  non-smooth
  non-uni-form
  non-uni-form-ly
  non-zero
  nor-ep-i-neph-rine
  not-with-stand-ing
  nu-cleo-tide
  nu-cleo-tides
  nut-crack-er
  nut-crack-ers
  oer-steds
  off-line
  off-load
  off-loads
  off-loaded
  oli-gop-o-list
  oli-gop-o-lists
  oli-gop-oly
  oli-gop-ol-ies
  om-ni-pres-ent
  om-ni-pres-ence
  ono-mat-o-poe-ia
  ono-mat-o-po-et-ic
  op-er-and
  op-er-ands
  orang-utan
  orang-utans
  or-tho-don-tist
  or-tho-don-tists
  or-tho-ker-a-tol-ogy
  ortho-nitro-toluene
  over-view
  over-views
  ox-id-ic
  pad-ding
  page-rank
  pain-less-ly
  pal-ette
  pal-ettes
  pa-rab-ola
  par-a-bol-ic
  pa-rab-o-loid
  par-a-digm
  par-a-digms
  para-chute
  para-chutes
  para-di-methyl-benzene
  para-fluoro-toluene
  para-graph-er
  para-le-gal
  par-al-lel-ism
  para-mag-net-ism
  para-medic
  para-methyl-anisole
  pa-ram-e-tri-za-tion
  pa-ram-e-trize
  para-mil-i-tary
  para-mount
  path-o-gen-ic
  peev-ish
  peev-ish-ness
  pen-ta-gon
  pen-ta-gons
  pe-tro-le-um
  phe-nol-phthalein
  phe-nom-e-non
  phenyl-ala-nine
  phi-lat-e-list
  phi-lat-e-lists
  pho-neme
  pho-nemes
  pho-ne-mic
  phos-phor-ic
  pho-to-graphs
  pho-to-off-set
  phtha-lam-ic
  phthal-ate
  phthi-sis
  pic-a-dor
  pic-a-dors
  pipe-line
  pipe-lines
  pipe-lin-ing
  pi-ra-nhas
  placa-ble
  plant-hop-per
  plant-hop-pers
  pla-teau
  pla-teaus
  pleas-ance
  plug-in
  plug-ins
  pol-ter-geist
  poly-an-dr
  poly-an-dry
  poly-an-drous
  poly-dac-tyl
  poly-dac-tyl-lic
  poly-ene
  poly-eth-yl-ene
  po-lyg-a-mist
  po-lyg-a-mists
  polyg-on-i-za-tion
  po-lyg-y-n
  po-lyg-y-ny
  po-lyg-y-nous
  pol-yp
  pol-yps
  po-lyph-o-n
  po-lyph-o-ny
  po-lyph-o-nous
  poly-phon-ic
  poly-styrene
  pome-gran-ate
  poro-elas-tic
  por-ous
  por-ta-ble
  post-am-ble
  post-am-bles
  post-hu-mous
  post-script
  post-scripts
  pos-tur-al
  pre-am-ble
  pre-am-bles
  pre-loaded
  pre-par-ing
  pre-print
  pre-prints
  pre-proces-sor
  pre-proces-sors
  pres-ent-ly
  pre-split-ting
  pre-wrap
  pre-wrapped
  priest-esses
  pret-ty-prin-ter
  pret-ty-prin-ting
  pro-ce-dur-al
  process
  pro-cur-ance
  prog-e-nies
  prog-e-ny
  pro-gram-mable
  pro-kary-ote
  pro-kary-otes
  pro-kary-ot-ic
  prom-i-nent
  pro-mis-cu-ous
  prom-is-sory
  prom-ise
  prom-ises
  pro-pel-ler
  pro-pel-lers
  pro-pel-ling
  pro-hib-i-tive
  pro-hib-i-tive-ly
  pro-sciut-to
  pro-style
  pro-styles
  pro-test-er
  pro-test-ers
  pro-tes-tor
  pro-tes-tors
  pro-to-lan-guage
  pro-to-typ-al
  prov-ince
  prov-inces
  pro-vin-cial
  pro-virus
  pro-viruses
  prow-ess
  pseu-do-dif-fer-en-tial
  pseu-do-fi-nite
  pseu-do-fi-nite-ly
  pseu-do-forces
  pseu-dog-ra-pher
  pseu-do-group
  pseu-do-groups
  pseu-do-nym
  pseu-do-nyms
  pseu-do-word
  pseu-do-words
  psy-che-del-ic
  psychs
  pu-bes-cence
  pur-ges
  quad-ding
  qua-drat-ic
  qua-drat-ics
  quad-ra-ture
  quad-ri-pleg-ic
  quaint-er
  quaint-est
  qua-si-equiv-a-lence
  qua-si-equiv-a-lences
  qua-si-equiv-a-lent
  qua-si-hy-po-nor-mal
  qua-si-rad-i-cal
  qua-si-resid-ual
  qua-si-smooth
  qua-si-sta-tion-ary
  qua-si-topos
  qua-si-tri-an-gu-lar
  qua-si-triv-ial
  quin-tes-sence
  quin-tes-sences
  quin-tes-sen-tial
  rab-bit-ry
  ra-di-og-ra-phy
  raff-ish
  raff-ish-ly
  ram-shackle
  rav-en-ous
  re-allo-cate
  re-allo-cates
  re-allo-cated
  re-arrange
  re-arranges
  re-arranged
  re-arrange-ment
  re-arrange-ments
  rec-i-proc-i-ties
  rec-i-proc-i-ty
  rec-tan-gle
  rec-tan-gles
  rec-tan-gu-lar
  re-di-rect
  re-di-rect-ion
  re-duc-ible
  re-echo
  re-edu-cate
  ref-u-gee
  ref-u-gees
  re-phrase
  re-phrases
  re-phrased
  re-po-si-tion
  re-po-si-tions
  re-print
  re-prints
  re-print-ed
  re-stor-able
  retro-fit
  retro-fit-ted
  re-us-able
  re-use
  re-wire
  re-wrap
  re-wrapped
  re-write
  rhi-noc-er-os
  right-eous
  right-eous-ness
  ring-leader
  ring-leaders
  ro-bot
  ro-bots
  ro-botic
  ro-bot-ics
  round-table
  round-tables
  sales-clerk
  sales-clerks
  sales-woman
  sales-women
  sal-mo-nel-la
  sal-ta-tion
  sar-sa-par-il-la
  sat-el-lite
  sat-el-lites
  sauer-kraut
  scat-o-log-i-cal
  sched-ul-ing
  schiz-o-phrenic
  schnau-zer
  school-child
  school-child-ren
  school-teacher
  school-teach-ers
  scru-ti-ny
  scyth-ing
  sell-er
  sell-ers
  sec-re-tar-iat
  sec-re-tar-iats
  sem-a-phore
  sem-a-phores
  se-mes-ter
  semi-def-i-nite
  semi-di-rect
  semi-ho-mo-thet-ic
  semi-ring
  semi-rings
  semi-sim-ple
  semi-skilled
  sem-itic
  ser-geant
  ser-geants
  sero-epi-de-mi-o-log-i-cal
  ser-vo-me-chan-i-cal
  ser-vo-mech-a-nism
  ser-vo-mech-a-nisms
  ses-qui-pe-da-lian
  set-up
  set-ups
  se-vere-ly
  shap-able
  shape-able
  shoe-string
  shoe-strings
  show-hy-phens
  side-step
  side-steps
  side-swipe
  single-space
  single-spaced
  single-spacing
  sky-scraper
  sky-scrapers
  sln-uni-code
  smoke-stack
  smoke-stacks
  snor-kel-ing
  so-le-noid
  so-le-noids
  solute
  solutes
  sov-er-eign
  sov-er-eigns
  spa-ces
  spe-cious
  spell-er
  spell-ers
  spell-ing
  spe-lunk-er
  spend-thrift
  spher-oid
  spher-oids
  spher-oid-al
  sphin-ges
  spic-i-ly
  spin-or
  spin-ors
  spokes-man
  spokes-per-son
  spokes-per-sons
  spokes-woman
  spokes-women
  sports-cast
  sports-cast-er
  spor-tive-ly
  sports-wear
  sports-writer
  sports-writers
  spright-lier
  squea-mish
  stand-alone
  star-tling
  star-tling-ly
  sta-tis-tics
  stealth-ily
  steeple-chase
  stereo-graph-ic
  sto-chas-tic
  strange-ness
  strap-hanger
  strat-a-gem
  strat-a-gems
  stretch-i-er
  strip-tease
  strong-est
  strong-hold
  stu-pid-er
  stu-pid-est
  sub-dif-fer-en-tial
  sub-ex-pres-sion
  sub-ex-pres-sions
  sub-scrib-er
  sub-scrib-ers
  sub-tables
  sum-ma-ble
  super-deri-va-tion
  super-deri-va-tions
  super-ego
  super-egos
  su-prem-a-cist
  su-prem-a-cists
  sur-gery
  sur-ge-ries
  sur-ges
  sur-veil-lance
  swim-ming-ly
  symp-to-matic
  syn-chro-mesh
  syn-chro-nous
  syn-chro-tron
  taff-rail
  take-over
  take-overs
  talk-a-tive
  ta-pes-try
  ta-pes-tries
  tar-pau-lin
  tar-pau-lins
  te-leg-ra-pher
  te-leg-ra-phers
  tele-ki-net-ic
  tele-ki-net-ics
  tele-ro-bot-ics
  tell-er
  tell-ers
  tem-po-rar-ily
  ten-ure
  test-bed
  tetra-butyl-ammo-nium
  text-height
  text-length
  text-width
  thal-a-mus
  ther-mo-elas-tic
  time-stamp
  time-stamps
  tool-kit
  tool-kits
  topo-graph-i-cal
  topo-iso-mer-ase
  topo-iso-mer-ases
  toques
  trai-tor-ous
  trans-ceiver
  trans-ceivers
  trans-par-en-cy
  trans-par-en-cies
  trans-gress
  trans-ver-sal
  trans-ver-sals
  trans-ves-tite
  trans-ves-tites
  tra-vers-a-ble
  tra-ver-sal
  tra-ver-sals
  tri-ethyl-amine
  treach-eries
  tribes-man
  trou-ba-dour
  tur-key
  tur-keys
  turn-around
  turn-arounds
  typ-al
  un-at-tached
  un-err-ing-ly
  un-friend-ly
  un-friend-li-er
  vaguer
  vaude-ville
  vic-ars
  vil-lain-ess
  vis-ual
  vis-ual-ly
  vi-vip-a-rous
  voice-print
  vspace
  wad-ding
  wall-flower
  wall-flow-ers
  warm-er
  warm-est
  waste-water
  wave-guide
  wave-guides
  wave-let
  wave-lets
  weap-ons
  weap-on-ry
  web-like
  week-night
  week-nights
  wheel-chair
  wheel-chairs
  which-ever
  white-sided
  white-space
  white-spaces
  wide-spread
  wing-span
  wing-spans
  wing-spread
  witch-craft
  word-spac-ing
  work-around
  work-arounds
  work-horse
  work-horses
  wrap-around
  wrap-arounds
  wretch-ed
  wretch-ed-ly
  yes-ter-year
  al-ge-brai-sche
  Al-le-ghe-ny
  Apol-lo-dorus
  Ar-kan-sas
  ATP-ase
  ATP-ases
  Aus-tral-asian
  auto-ma-ti-sier-ter
  Beb-chuk
  Be-die-nung
  Bembo
  bi-blio-gra-phi-sche
  Bos-ton
  Brown-ian
  Bruns-wick
  Bu-da-pest
  Burck-hardt
  Car-ib-bean
  Charles-ton
  Char-lottes-ville
  Ches-ter
  Chiang
  Chich-es-ter
  Cohen
  Co-lum-bia
  Czecho-slo-va-kia
  Del-a-ware
  Dijk-stra
  Dor-ches-ter
  Dorf-leit-ner
  Drechs-ler
  Duane
  dy-na-mi-sche
  Eijk-hout
  Engle
  Engel
  Eng-lish
  Euler-ian
  Evan-ston
  Feb-ru-ary
  Fest-schrift
  Flor-i-da
  Flor-i-d-ian
  For-schungs-in-sti-tut
  Free-BSD
  funk-tsional
  Gauss-ian
  Ge-sell-schaft
  Ghost-script
  Ghost-View
  Gott-lieb
  Grass-mann-ian
  Greifs-wald
  Grothen-dieck
  Grund-leh-ren
  Ha-da-mard
  Hai-fa
  Hamil-ton-ian
  Hel-sinki
  Her-mit-ian
  Hibbs
  Hoek-water
  Hok-kai-do
  Huber
  Image-Magick
  Jac-kow-ski
  Jan-u-ary
  Ja-pa-nese
  Java-Script
  Jung-ian
  Kad-om-tsev
  Kan-sas
  Karls-ruhe
  Keynes-ian
  Kor-te-weg
  Krishna
  Krish-na-ism
  Krish-nan
  Kron-ecker
  Lan-cas-ter
  Le-gendre
  Leices-ter
  Lip-schitz
  Lip-schitz-ian
  Loj-ban
  Lou-i-si-ana
  Lucas
  MacBeth
  Mac-OS
  Ma-gel-lan
  Ma-la-ya-lam
  Man-ches-ter
  Mar-kov-ian
  Markt-ober-dorf
  Mass-a-chu-setts
  Max-well
  Meth-od-ist
  Meth-od-ism
  Mi-cro-soft
  Min-kow-ski
  Min-ne-ap-o-lis
  Min-ne-sota
  Mont-real
  Mos-cow
  Nach-rich-ten
  Nash-ville
  Net-BSD
  Net-scape
  Nij-me-gen
  Noe-ther-ian
  Noord-wijker-hout
  Noto-wi-digdo
  No-vem-ber
  Obst-feld
  Open-BSD
  Open-Office
  Oreo-pou-los
  Pala-tino
  Pa-ler-mo
  Pe-trov-ski
  Pfaff-ian
  Phil-a-del-phia
  phi-lo-so-phi-sche
  Poin-care
  Po-ten-tial-glei-chung
  Pres-by-terian
  Pres-by-terians
  Py-thag-o-ras
  Py-thag-o-re-an
  Ra-dha-krish-nan
  raths-kel-ler
  Ravi-kumar
  Reich-lin
  Rie-mann-ian
  Ryd-berg
  Schim-mel-pfen-nig
  schot-ti-sche
  Schro-din-ger
  Schwa-ba-cher
  Schwarz-schild
  Schweid-nitz
  Schwert
  Sep-tem-ber
  Shore-ditch
  Skoup
  Stokes-sche
  Stutt-gart
  Sus-que-han-na
  Tau-ber-ian
  tech-ni-sche
  Ten-nes-see
  Thiruv-ananda-puram
  Tol-ches-ter
  To-ma-szew-ski
  Toyo-ta
  ty-po-graphique
  Ukrain-ian
  ver-all-ge-mei-nerte
  Ver-ei-ni-gung
  Ver-tei-lun-gen
  Vid-ias-sov
  Vieth
  viiith
  viith
  Wahr-schein-lich-keits-theo-rie
  Wein-stein
  Wer-ner
  Wer-ther-ian
  Will-iam
  Will-iams
  Win-ches-ter
  Wirt-schaft
  wis-sen-schaft-lich
  Wolff-ian
  xviiith
  xviith
  xxiiird
  xxiind
  Ying-yong Shu-xue Ji-suan
  Zea-land
  Zeit-schrift
}
\fi
%    \end{macrocode}
%
% We done with a very long and exhausting, preamble but hopefully
% will save countless hours for other people. If you use it in your
% publication send me a copy of it.  What follow is the special keys
% for formatting sectioning commands.
% 	
% \chapter{Section Formatting}
%
% \section{Introduction}
%
%  The code that follows deals exclusively with sectioning commands.
% The macros \cs{HUGE} and \cs{HHUGE} provide larger sizes than those
% provided by \LaTeXe that are used in the production of titles and
% chapter heads.
%
% \begin{macro}{\words@cx} Utility macro for translating a 
% number from numbers to words.
% 
%    \begin{macrocode}
\def\words@cx#1{%
  \ifcase#1 zero\or one\or two\or three\or four\or five\or six\or seven
\or eight\or nine\or ten\or eleven\or twelve\or thirteen\or
fourteen
\or fifteen\or sixteen\or seventeen\or eighteen\or nineteen \or
twenty
\or twenty one\or twenty two\or twenty three\or twenty four\or
twenty five
\or twenty six\or twenty seven \or twenty eight \or twenty
nine\or thirty
\or thirty one\or thirty two\or thirty three\or thirty four\or
thirty five
\or thirty six\or thirty seven\or thirty eight\or thirty nine\or
forty\or forty one
\or forty two \or forty three\or forty four\or forty five \or
forty six \or forty seven
\or forty eight \or forty nine\or fifty\or fifty on\or fifty
two\or fifty three
\or fifty four\or fifty five\or fifty six\or fifty seven\or
fifty eight\or fifty nine
  \or sixty \or sixty one \or sixty two
  \or sixty three \or sixty four \or sixty five
    \else
    #1
    %\@ctrerr
    \fi
}

\def\Words@cx#1{%
\ifcase#1 Zero\or One\or Two\or Three\or Four\or Five\or Six\or
Seven\or Eight\or Nine\or Ten\or
Eleven\or Twelve\or Thirteen\or Fourteen\or Fifteen\or
Sixteen\or Seventeen\or Eighteen\or Nineteen \or Twenty\or
Twenty One\or Twenty Two\or Twenty Three\or Twenty Four\or
Twenty Five\or Twenty Six\or Twenty Seven \or Twenty Eight \or
Twenty Nine\or Thirty\or Thirty One\or Thirty Two\or Thirty
Three\or Thirty Four\or Thirty Five\or Thirty Six\or Thirty
Seven\or Thirty Eight\or Thirty Nine\or Forty\or Forty One\or
Forty Two \or Forty Three\or Forty Four\or Forty Five \or Forty
Six \or Forty Seven\or Forty Eight \or Forty Nine\or Fifty\or
Fifty One\or Fifty Two\or Fifty Three\or Fifty four\or Fifty
Five\or Fifty Six\or Fifty Seven\or Fifty Eight\or Fifty Nine\or
Sixty \or Sixty One \or Sixty Two
\or Sixty Three \or Sixty Four \or Sixty Five \or SixtySix \or SixtySeven
\or Sixty Eight \or SixtyNine \or Seventy \or Seventy One \or Seventy Two
\else
#1
%\@ctrerr
\fi}

\def\WORDS@cx#1{%
\ifcase#1 ZERO\or ONE\or TWO\or THREE\or FOUR\or FIVE\or SIX\or
SEVEN\or EIGHT\or NINE\or TEN\or
ELEVEN\or TWELVE\or THIRTEEN\or FOURTEEN\or FIFTEEN\or
SIXTEEN\or SEVENTEEN\or EIGHTEEN\or NINETEEN \or TWENTY\or
TWENTY ONE\or TWENTY TWO\or TWENTY THREE\or TWENTY FOUR\or
TWENTY FIVE\or TWENTY SIX\or TWENTY SEVEN \or TWENTY EIGHT \or
TWENTY NINE\or THIRTY\or THIRTY ONE\or THIRTY TWO\or THIRTY
THREE\or THIRTY FOUR\or THIRTY FIVE\or THIRTY SIX\or THIRTY
SEVEN\or THIRTY EIGHT\or THIRTY NINE\or FORTY\or FORTY ONE\or
FORTY TWO \or FORTY THREE\or FORTY FOUR\or FORTY FIVE\or FORTY
SIX\or FORTY SEVEN\or FORTY EIGHT\or FORTY NINE\or FIFTY\or
FIFTY ONE\or FIFTY TWO\or FIFTY THREE\or FIFTY FOUR\or FIFTY
FIVE\or FIFTY SIX\or FIFTY SEVEN\or FIFTY EIGHT\or FIFTY NINE\or
SIXTY\or SIXTY ONE\or SIXTY TWO\or SIXTY THREE \or SIXTY FOUR\or
SIXTY FIVE\or SIXTY SIX\or SIXTY SEVEN\or SIXTY EIGHT\or SIXTY
NINE\or SEVENTY\or SEVENTY ONE\or SEVENTY TWO\or SEVENTY
THREE\or SEVENTY FOUR\or SEVENTY FIVE\or SEVENTY SIX\or SEVENTY
SEVEN\or SEVENTY EIGHT\or SEVENTY NINE\or EIGHTY\or EIGHTY
ONE\or EIGHTY TWO\or EIGHTY THREE\or EIGHTY FOUR\or EIGHTY
FIVE\or EIGHTY SIX\or EIGHTY SEVEN\or EIGHTY EIGHT\or EIGHTY
NINE\or NINETY \or NINETY ONE \or NINETY TWO \or NINETY THREE
\or NINETY FOUR \or NINETY FIVE
\else
#1
%\@ctrerr
\fi}
   
\def\ORDINALS@cx#1{%
\ifcase#1 ZEROETH\or FIRST\or SECOND\or THIRD\or FOURTH\or
FIFTH\or SIXTH\or SEVENTH\or EIGHTTH\or NINTH\or TENTH\or
ELEVENTH\or TWELFTH\or THIRTEENTH\or FOURTEENTH\or FIFTEENTH\or
SIXTEENTH\or SEVENTEEN\or EIGHTEEN\or NINETEEN \or TWENTY\or
TWENTY ONE\or TWENTY TWO\or TWENTY THREE\or TWENTY FOUR\or
TWENTY FIVE\or TWENTY SIX\or TWENTY SEVEN \or TWENTY EIGHT \or
TWENTY NINE\or THIRTY\or THIRTY ONE\or THIRTY TWO\or THIRTY
THREE\or THIRTY FOUR\or THIRTY FIVE\or THIRTY SIX\or THIRTY
SEVEN\or THIRTY EIGHT\or THIRTY NINE\or FORTY\or FORTY ONE\or
FORTY TWO \or FORTY THREE\or FORTY FOUR\or FORTY FIVE\or FORTY
SIX\or FORTY SEVEN\or FORTY EIGHT\or FORTY NINE\or FIFTY\or
FIFTY ONE\or FIFTY TWO\or FIFTY THREE\or FIFTY FOUR\or FIFTY
FIVE\or FIFTY SIX\or FIFTY SEVEN\or FIFTY EIGHT\or FIFTY NINE\or
SIXTY\or SIXTY ONE\or SIXTY TWO\or SIXTY THREE \or SIXTY FOUR\or
SIXTY FIVE \or SIXTY SIX \or SIXTY SEVEN \or SIXTY EIGHT \or SIXTY NINE
\or SEVENTY \or SEVENTY ONE \or SEVENTY TWO \or SEVENTY THREE
\or SEVENTY FOUR \or SEVENTY FIVE \or SEVENTY SIX \or SEVENTY SEVEN
\or SEVENTY EIGHT \or SEVENTY NINE \or EIGHTY
\else
#1
%\@ctrerr
\fi}

\def\ordinals@cx#1{%
  \ifcase#1 Zeroeth\or First\or Second\or Third\or Fourth\or Fifth\or Sixth
  \or Seventh\or Eighth\or Ninth\or Tenth\or
 Eleventh\or Twelfth\or Thirteenth\or Fourteenth\or Fifteenth
\or SIXTEENTH\or SEVENTEEN\or EIGHTEEN\or NINETEEN \or TWENTY\or
TWENTY ONE\or TWENTY TWO\or TWENTY THREE\or TWENTY FOUR\or
TWENTY FIVE\or TWENTY SIX\or TWENTY SEVEN \or TWENTY EIGHT \or
TWENTY NINE\or THIRTY\or THIRTY ONE\or THIRTY TWO\or THIRTY
THREE\or THIRTY FOUR\or THIRTY FIVE\or THIRTY SIX\or THIRTY
SEVEN\or THIRTY EIGHT\or THIRTY NINE\or FORTY\or FORTY ONE\or
FORTY TWO \or FORTY THREE\or FORTY FOUR\or FORTY FIVE\or FORTY
SIX\or FORTY SEVEN\or FORTY EIGHT\or FORTY NINE\or FIFTY\or
FIFTY ONE\or FIFTY TWO\or FIFTY THREE\or FIFTY FOUR\or FIFTY
FIVE\or FIFTY SIX\or FIFTY SEVEN\or FIFTY EIGHT\or FIFTY NINE\or
SIXTY\or SIXTY ONE\or SIXTY TWO\or SIXTY THREE \or SIXTY FOUR\or
SIXTY FIVE\or SIXTY SIX \or SIXTY SEVEN \or \else
#1
%\@ctrerr
\fi
}

%    \end{macrocode}
% \end{macro}
%
%
% \subsection{General Utility Environments}
%
%
%    \begin{macrocode}
\newenvironment{absolutequote}
               {\list{}{\leftmargin2cm\rightmargin\leftmargin}%
                \item\relax\footnotesize}
               {\endlist}

\newenvironment{summary}
               {\list{}{\listparindent0pt %
                        \itemindent\listparindent
                        \leftmargin0pt
                        \rightmargin\leftmargin
                        \parsep\z@ \@plus\p@}%
                \item\relax\itshape}
               {\endlist}
%
\def\solution{%
   \everypar{}
   \parindent0pt
  \leavevmode\par
  \makebox{\llap{\bfseries\textit{Solution }:}\thinspace}%
  \parindent2em
  }
%    \end{macrocode}
%
% \subsection{Setting up the key system}
%
% We are going to use a few conditionals and we start by defining 
% them here:
%
%    \begin{macrocode}
\newif\if@left
\newif\if@right
\newif\if@center
\@leftfalse
\@rightfalse
\@centerfalse
% newifs for number position
\newif\if@lefttitle
\newif\if@righttitle
\newif\if@leftname
\newif\if@rightname
\newif\if@chapterspaceout\@chapterspaceoutfalse
\newif\if@soulspaceout\@soulspaceoutfalse
\newif\if@numberspaceout\@numberspaceoutfalse
\newif\if@titlespaceout\@chapterspaceoutfalse
\newif\if@sectionspaceout\@sectionspaceoutfalse
\newif\if@openanywhere\@openanywherefalse
%    \end{macrocode}
%
% The standard LaTeX2e settings does not allow for open left chapters.
% However, quite a few designs do have this incorporated.
%    \begin{macrocode}
\newif\if@openleft\@openleftfalse
\newif\if@openany\@openanyfalse
%    \end{macrocode}
%
% Some publications allow chapters to be written by different authors
% we provide a conditional for this.
% 
%    \begin{macrocode}
\newif\if@special\@specialfalse
\newif\if@chaptertitlespecial
\@chaptertitlespecialfalse

\newif\if@authorblock
%    \end{macrocode}
%
% We are going to allow the user to use a key to add a toc, also
% wea re allowing to incorporate images in such table of contents.
% We creating two conditionals to hold this information.
%
%    \begin{macrocode}
\newif\if@toc  \@toctrue
\newif\if@tocimage \@tocimagefalse
%    \end{macrocode}
%
% \subsection{Defining Document Keys}
%
% As we aim to make the package generic to be used with any base class
% we define some conditionals and keys.
%
%    \begin{macrocode}
\newif\if@book
\newif\if@report
\newif\if@article
\cxset{document type/.is choice,
  document type/book/.code = {\@booktrue},
  document type/article/.code = {\@reporttrue},
  document type/report/.code = {\@articletrue}, 
}
%    \end{macrocode}
%
% \begin{macro}{\setfontparam@cx} 
% \begin{macro}{\setfont@cx} 
% This macro enables font setting keys to either
% be entered by an author as  a command e.g., |\Huge| or as a macro name |Huge|. It uses
% the \pkg{etoolbox} |\ifdef| macro.
%
%    \begin{macrocode}
\global\def\setfontparam@cx#1;{%
  \ifdefmacro{#1}{#1}{\csname#1\endcsname}%
}
\def\setfont@cx#1#2#3#4{%
  \expandafter\setfontparam@cx#1;%
  \expandafter\setfontparam@cx#2;%
  \expandafter\setfontparam@cx#3;%
  \expandafter\setfontparam@cx#4;%
}
\let\bold\bfseries
\let\normal\mdseries
\let\serif\rmfamily
%    \end{macrocode}
% \end{macro}
% \end{macro}
%
% \subsection{Defining Chapter Head Keys}
%
% Parametric definitions for chapters 
%    \begin{macrocode}
\@ifundefined{@openright}{%
  }{}
\def\afterindenton@cx{\def\afterindent@cx{\@afterindenttrue}}
\def\afterindentoff@cx{\def\afterindent@cx{\@afterindentfalse}}
\edef\zeroboxalign@cx{c}
%    \end{macrocode}
%
%    \begin{macrocode}
%
\global\newlength\chaptermarginleft
    \setlength\chaptermarginleft{30pt}%
    
\global\newlength\titlemarginbottom
\global\setlength\titlemarginbottom{0pt}
\global\newlength\titlemargintop
\global\setlength\titlemargintop{0pt}
\gdef\titlemargintop@cx{10pt}
\gdef\titlemarginbottom@cx{10pt}

\gdef\chaptermarginleft@cx{0pt}
   
%
\def\chaptertitleblockalign{}
\newcounter{chapterdisplay} \setcounter{chapterdisplay}{0}
\newcounter{numberdisplay} \setcounter{numberdisplay}{0}
%
% numbers padding
\global\newlength\numberpaddingright
\global\setlength\numberpaddingright{0pt}
\global\newlength\numberpaddingleft
     \global\setlength\numberpaddingleft{0pt}

\global\newlength\numberpaddingtop
\global\setlength\numberpaddingtop{10pt}
\global\newlength\numberpaddingbottom
\global\setlength\numberpaddingbottom{10pt}
% number border width
\global\newlength\numberborderleftwidth
   \setlength\numberborderleftwidth{0pt}
\global\newlength\numberborderrightwidth
   \setlength\numberborderrightwidth{2pt}
\global\newlength\numberbordertopwidth
   \setlength\numberbordertopwidth{2pt}
\global\newlength\numberborderbottomwidth
   \setlength\numberborderbottomwidth{2pt}
\gdef\numberbgcolor{spot!20}   
%    \end{macrocode}
% All chapter titles can be fully framed with borders. We create length
% registers for these and appropriate keys.
% (See style 87 for usage examples)
%    \begin{macrocode}
\newlength\titletextwidth
   \setlength{\titletextwidth}{\textwidth}
\newlength\titlepadding\setlength{\titlepadding}{5pt}%
\newlength\titlepaddingtop
   \setlength{\titlepaddingtop}{5pt}%
\newlength\titlepaddingbottom
    \setlength{\titlepaddingbottom}{5pt}%
\newlength\titlepaddingleft
    \setlength{\titlepaddingleft}{5pt}%    
\newlength\titlepaddingright
    \setlength{\titlepaddingright}{5pt}%     
%
\newlength\titleborderwidth\setlength{\titleborderwidth}{0pt}%
\newlength\titlebordertopwidth\setlength{\titlebordertopwidth}{0pt}
\newlength\titleborderbottomwidth\setlength{\titleborderbottomwidth}{0pt}
\newlength\titleborderleftwidth\setlength{\titleborderleftwidth}{0pt}
\newlength\titleborderrightwidth\setlength{\titleborderrightwidth}{0pt}
\def\gluestart{\hss}\def\glueend{\hss}
%
\newif\if@hascomma
\@hascommatrue
\def\hascomma#1{%
    \@hascommafalse
    \@tfor\next:=#1\do{%
    \edef\@tempa{\next}%
    \edef\tempb{,}%
    {\if\@tempa\tempb
         \global\@hascommatrue
    \fi}%
}}%


\def\checkforcomma@cx#1{%
\hascomma{#1}%
\if@hascomma%
    \expandafter\@firstoftwo%
  \else%
    \expandafter\@secondoftwo% 
\fi
}
%
%    \end{macrocode}
%
% It is not envisioned that the chapter name key be set directly by the user.
% This should be set by the document language tag (like Babel). 
% |name is legacy and will be removed.|  
% However, the user might decide that this is an easier approach.
%    \begin{macrocode}
\cxset{
  name/.store in=\chaptername,
  chapter name/.code=\pgfkeysalso{name=#1},
  chaptername/.code = \gdef\chaptername@cx{#1},
  }
  
  
\cxset{  
  %legacy to remove
  color/.store in = \color@cx,
  color/.default = black,
  color/.initial = black,
  % 
  chapter color/.store in=\chaptercolor@cx,
  chapter background-color/.store in=\chapterbgcolor,
%  
  number background-color/.store in=\numberbgcolor,
}


\cxset{    
  chapter opening/.is choice,
  chapter opening/right/.code={\@openrighttrue},
  chapter opening/left/.code={\@openlefttrue},
  chapter opening/any/.code={\@openanytrue},
  chapter opening/none/.code={\@openanywheretrue\@openrightfalse%
                                                  \@openleftfalse\@openanyfalse},
  chapter opening/anywhere/.code={\@openanywheretrue\@openrightfalse
     \@openleftfalse\@openanyfalse},
  chapter opening/ifafter/.code={},
}


% This handler first checks if the font-family has been supplied as a
% list. In this case it will call check font and pick the first available
% font.
% 
\pgfkeys{/handlers/.font-family in/.code=\pgfkeysalso{\pgfkeyscurrentpath/.code=
      \checkforcomma@cx{##1}{}{%
          \edef\tempaa{sans-serif}%
          \edef\tempbb{##1}%
          \ifx\tempaa\tempbb%
              \def#1{sffamily}% 
        \else%
           \def#1{##1}%Check for initial???
       \fi%
       \edef\tempaa{monospace}%
        \ifx\tempaa\tempbb%
          \def#1{ttfamily}%
       \fi
       \edef\tempaa{serif}
       \ifx\tempaa\tempbb%
           \def#1{rmfamily}%
       \fi%    
       \edef\tempaa{inherit}%needs work???
       \ifx\tempaa\tempbb%
           \def#1{rmfamily}
       \fi    
     }%   
}}
\pgfkeys{/handlers/.font-style in/.code=\pgfkeysalso{\pgfkeyscurrentpath/.code=
           \edef\tempa{##1}%
           \edef\tempb{normal}%
           \edef\tempc{italic}%
           \edef\tempd{oblique}%
           \def#1{##1}%
           \ifx\tempa\tempb%
              \def#1{upshape}%
           \fi%
           \ifx\tempa\tempc%
              \def#1{itshape}%
            \fi%
           \ifx\tempa\tempd%
              \def#1{slshape}%
           \fi  
   }}
\pgfkeys{/handlers/.font-weight in/.code=\pgfkeysalso{\pgfkeyscurrentpath/.code=
           \def\tempa{##1}%
           \def\tempb{normal}%
           \def\tempc{bold}%
           \def\tempd{bfseries}%
           \def#1{##1}%
           \ifx\tempa\tempb%
              \def#1{mdseries}%
           \fi%
           \ifx\tempa\tempc%
              \def#1{bfseries}%
            \fi%
           \ifx\tempa\tempd%
              \def#1{bfseries}%
           \fi   
           \def\tempb{\bfseries}%
           \ifx\tempa\tempb
              \def#1{bfseries}%       
           \fi  
           \def\tempb{\mdseries}%
           \ifx\tempa\tempb
              \def#1{mdseries}%       
           \fi  
   }} 
%    \end{macrocode}
%
%  The font-size handler is defined next.
%  This can be set both as a command or a name.
%  Code can be optimized with a list, but needs revisiting.
%    \begin{macrocode}   
   \pgfkeys{/handlers/.font-size in/.code=\pgfkeysalso{\pgfkeyscurrentpath/.code=
           \def\tempa{##1}%
           \def\tempb{normal}%
           \def#1{##1}%
           \ifx\tempa\tempb%
              \def#1{\normalsize}%
           \fi%
            \def\tempb{tiny}%
           \ifx\tempa\tempb%
              \def#1{tiny}%
            \fi%
            \ifx\tempa\tempb%
              \def#1{footnotesize}%
            \fi% 
           \def\tempb{footnotesize}%
           \ifx\tempa\tempb%
              \def#1{small}%
            \fi%
           \def\tempb{large}% 
           \ifx\tempa\tempb%
              \def#1{large}%
           \fi   
            \def\tempb{Large}% 
           \def\tempb{\bfseries}%
           \ifx\tempa\tempb%
              \def#1{Large}%       
           \fi%  
           \def\tempb{huge}%
           \ifx\tempa\tempb%
              \def#1{huge}%       
           \fi%  
           \def\tempb{Huge}%
           \ifx\tempa\tempb%
              \def#1{Huge}%       
           \fi% 
           \def\tempb{HUGE}%
           \ifx\tempa\tempb%
              \def#1{HUGE}%       
           \fi 
           \def\tempb{HHUGE}%
            \ifx\tempa\tempb%
              \def#1{HHUGE}%       
           \fi% 
           \def\tempb{HHHUGE}%
            \ifx\tempa\tempb%
              \def#1{HHHUGE}%       
           \fi%  
   }}%   
%    \end{macrocode}


% \subsubsection{Numbers}
%
%    \begin{macro}{\itf@number}
%    \cs{itf@number} tries to get a number from its argument, and stores the result
%    into the TeX count register given as first argument. Same syntax as \LaTeX \cs{@defaultunits}.
%
%    A number is an integer that can be assigned to a count register, may be followed
%    by one or more \cs{relax}.
%
%    \begin{macrocode}
%%   G E T   A   N U M B E R (if possible)
%% USAGE: \itf@number\count<text>\relax\@nnil (like \@defaultunits)
\def\itf@number#1#2\relax\@nnil{% \relax is to mimick the syntax of \@defaultunits
   \afterassignment\itf@number@#1\number0#2\relax\itf@number@
}
\def\itf@number@#1\relax#2\itf@number@{%
   \ifblank{#1}
      {\itf@number@@{}#2\itf@number@@\@nnil
      \csname ltx@\ifx\@let@token\itf@number@@
         first\else second\fi oftwo\endcsname}
      \ltx@secondoftwo
}% \itf@number@
\def\itf@number@@#1{\futurelet\@let@token\itf@number@@@}
\def\itf@number@@@{%
   \csname \ifx\@let@token\relax itf@number@@%
   \else remove@to@nnil%
   \fi \endcsname
}% \itf@number@@@
%    \end{macrocode}
%    \end{macro}
%
% \subsubsection{Units and math units}
%
%    \begin{macro}{\itf@setlength}
%
%    To be able to write: \cs{pgfkeys}\M*{key=\cs{widthof\M*{some text}}} (package \pkgname{calc}),
%    
%
%    \begin{macrocode}
\protected\def\itf@setlength#1#2{#1\glueexpr#2\relax}%
%    \end{macrocode}
%    \end{macro}
%
%    \begin{macro}{\itf@units@scale}
%
%    For multiplication of units: \cs{glueexpr}...\cs{relax}*\cs{itf@units@scale}\meta{scaling factor}!
%
%    The scaling factor can be a fraction or a real number. Real numbers are approximated by a fraction
%    of 65536.
%
%    \begin{macrocode}
%% glue * 1.5 = glue * 98302 / 65536
%% glue * 3/2 = glue * 3/2
\def\itf@units@scale#1!{\itf@units@scale@normalize#1/\@nnil/\@nil}
\def\itf@units@scale@normalize#1/#2/#3\@nil{%
   \ifx#2\@nnil \number\dimexpr#1pt\relax/65536
   \else#1/#2
   \fi
}% \itf@units@scale@normalize
\def\itf@units@scaleNoCalc#1!{\itf@units@scale@normalizeNoCalc#1/\@nnil/\@nil}
\let\itf@units@scale@normalizeNoCalc \itf@units@scale@normalize
%    \end{macrocode}
%    \end{macro}
%
%    \begin{macro}{\itf@setlength@calc}
%
%^^A    The version for the \xpackage{calc} package. \xpackage{calc} removes stretch
%    and shrink components of glues when a scaling operation is done using \cs{real}
%    or \cs{ratio}. This limitation is lifted for \texttt{key=value} assignments.
%
%    \begin{macro}{\itf@units@scale@normalize@calc}
%
%    The version of \cs{itf@units@scale@normalize} when using the \pkgname{calc} package.
%
%    \cs{ratio} is used.
%
%    \begin{macrocode}
\protected\def\itf@setlength@calc#1#2{\begingroup
   \let\calc@multiply@by@real \itf@calc@multiply@by@real
   \let\calc@Adimen \itf@calc@Askip  \let\calc@Bdimen \itf@calc@Bskip
   \setlength{#1}{#2}%
   \expandafter\endgroup\expandafter#1\the#1\relax
}% \itf@setlength@calc
\def\itf@units@scale@normalize@calc#1/#2/#3\@nil{%
   \ifx#2\@nnil 1*\ratio{\dimexpr#1pt\relax}\p@
   \else        1*\ratio{\dimexpr#1pt\relax}/{\dimexpr#2pt\relax}%
   \fi
}% 
%    \end{macrocode}
%    \end{macro}
%    \end{macro}
%
%
%   The code below is an adaptation from the \pkgname{interfaces}
%   Some of it I don’t understand very well and I need to revisit it.
%
%       
%    \begin{macro}{\pgfkeysmeaning}
%    \begin{macrocode}
\providecommand*\pgfkeysmeaning[1]{\pgfkeysifdefined{#1}
   {\expandafter\meaning\csname pgfk@#1\endcsname}
   {\meaning\@undefined}%
}% \pgfkeysmeaning
%    \end{macrocode}
%    \end{macro}
%
%    \begin{macro}{\pgfkeysvalueof@unexpanded}
%    \begin{macrocode}
\def\pgfkeysvalueof@unexpanded#1{%
   \unexpanded\expandafter\expandafter\expandafter{%
                        \csname pgfk@#1\endcsname}%
}% \pgfkeysvalueof@unexpanded
%    \end{macrocode}
%    \end{macro}
%
%
%    \begin{macro}{\pgfkeysEsetvalue}
%    \begin{macro}{\pgfkeysEaddvalue}
%    \begin{macrocode}
\ifdefined\pgfkeys@ifexecutehandler \long \fi% pgf version 2.1
\def\pgfkeysEsetvalue#1#2{%
   \expandafter\edef\csname pgfk@#1\endcsname{#2}%
}% \pgfkeysEsetvalue
\ifdefined\pgfkeys@ifexecutehandler \long \fi % pgf v2.1
\def\pgfkeysEaddvalue#1#2#3{% \csepreappto{pgfk@#1}{#2}{#3}
   \edef\pgfkeys@global@temp{%
      #2%
      \ifcsname pgfk@#1\endcsname
         \unexpanded\expandafter\expandafter\expandafter{%
            \csname pgfk@#1\endcsname}%
      \fi
      #3}%
   \pgfkeyslet{#1}\pgfkeys@global@temp
}% \pgfkeysEaddvalue
%    \end{macrocode}
%    \end{macro}
%    \end{macro}
%    
%
%  We define a key to store dimensions. We test for a number and for an
%  empty value. If the value is empty we leave as is. We use the Roman
%  numeral trick to test if it is a number. What a lovely way to remember
%  your history! THIS IS WRONG
%
%    \begin{macrocode}
\pgfkeysdef{/handlers/.store as glue}{\pgfkeysdef{\pgfkeyscurrentpath}{%
\def\temp{##1}%
\ifx\temp\@empty%
  \gdef#1{1pt}%
\else
   \gdef#1{1pt}%
   \if\relax\detokenize\expandafter{\romannumeral-0##1}\relax
        \gdef#1{\dimexpr(##1)\relax}%
  \fi           
\fi  
}}%
% /.store as glue
%
%    \end{macrocode}
% 
% \subsection{Chapter font options}
%  The font options use handlers to get the values. This alows for more flexibiliy.
%    \begin{macrocode}
\cxset{%    
  chapter font-family/.font-family in=\chapterfontfamily@cx,
  chapter font-weight/.font-weight in = \chapterfontweight@cx,
  chapter font-size/.font-size in=\chapterfontsize@cx,
  chapter font-shape/.font-style in=\chapterfontshape@cx,
  chapter font-style/.font-style in=\chapterfontshape@cx,}
%    \end{macrocode}
%
% \subsection{Chapter display and float properties}
%
% This generates keys for float and display. The attribute display determines if the
% element is on a line of its own or not. The float determines glue to be
% set to float the element left or right.
%
%    \begin{macrocode}
\newcounter{lastelementfloat}
     \setcounter{lastelementfloat}{-1}
\newcounter{chapterfloat} 
      \setcounter{chapterfloat}{1}  
\newcounter{numberfloat} 
      \setcounter{numberfloat}{1}        
\newcounter{currentelementfloat}
      \setcounter{currentelementfloat}{-1}
%
\global\newlength\chapterborderrightwidth
    \setlength\chapterborderrightwidth{2pt} 
\global\newlength\chapterborderleftwidth
    \setlength\chapterborderleftwidth{2pt}     
\global\newlength\chapterborderbottomwidth
    \setlength\chapterborderbottomwidth{2pt}  
\global\newlength\chapterbordertopwidth
    \setlength\chapterbordertopwidth{2pt}             
%
\global\newlength\chapterpaddingleft
    \setlength\chapterpaddingleft{10pt}
\global\newlength\chapterpaddingright
    \setlength\chapterpaddingright{10pt}  
\global\newlength\chapterpaddingtop
    \setlength\chapterpaddingtop{10pt}        
\global\newlength\chapterpaddingbottom
    \setlength\chapterpaddingbottom{10pt}       
\ExplSyntaxOn
\int_zero_new:c {chapterdisplaycounter}
\int_zero_new:c {chapterfloatcounter}
\cs_gset:Npn \phdsetcounter #1 #2 
 {
   \int_gset:cn {#1} {#2}
 }  
\ExplSyntaxOff
%
\cxset{%  
  chapter display/.is choice,
  chapter display/inline/.code=\global\setcounter{chapterdisplay}{0}
                               \phdsetcounter{chapterdisplaycounter}{0},
  chapter display/block/.code=\global\setcounter{chapterdisplay}{2}
                               \phdsetcounter{chapterdisplaycounter}{2}, 
%    
  chapter float/.is choice,
  chapter float/none/.code= \global\setcounter{chapterfloat}{0}%
                            \phdsetcounter{chapterfloatcounter}{0},
  chapter float/left/.code= \global\setcounter{chapterfloat}{0}
                            \phdsetcounter{chapterfloatcounter}{0},
% center = 1                             
  chapter float/center/.code= \global\setcounter{chapterfloat}{1}%
                             \phdsetcounter{chapterfloatcounter}{1}, 
% right = 2                             
  chapter float/right/.code= \global\setcounter{chapterfloat}{2}%
                             \phdsetcounter{chapterfloatcounter}{2},%   
 }
\cxset{chapter display=block,
       chapter float=left,
       } 
%    \end{macrocode}
%
%
% \subsection{Chapter content before and after}
%
%    \begin{macrocode}   
\cxset{chapter before content/.store in=\chapterbeforecontent@cx,
       chapter before/.store in=\chapterbefore@cx,
 }
%    \end{macrocode}
% 
%  
%
% \subsection{Chapter margins and padding}
%
% The dual code is interim we will avoid all these in the future
%    \begin{macrocode}
\ExplSyntaxOn
%
\dim_new:c {chapter_margin_top}
\dim_new:c {chapter_margin_right}
\dim_new:c {chapter_margin_bottom}
\dim_new:c {chapter_margin_left}
\dim_new:c {chapter_margin}
%
\dim_new:c {chapter_border_top_width}
\dim_new:c {chapter_border_right_width}
\dim_new:c {chapter_border_bottom_width}
\dim_new:c {chapter_border_left_width}
\dim_new:c {chapter_border}
%
\dim_new:c {chapter_padding_top}
\dim_new:c {chapter_padding_right}
\dim_new:c {chapter_padding_bottom}
\dim_new:c {chapter_padding_left}
\dim_new:c {chapter_padding}
%
\cxset{   
  chapter~margin-top/.code=\gdef\chaptermargintop@cx{\topskip0pt\vskip#1\relax}
    \dim_gset:cn { chapter_margin_top } { #1 },
  % left margin 
  chapter~margin-left/.code=\setlength\chaptermarginleft{#1}%
                            \global\chaptermarginleft\chaptermarginleft\relax
                            \def\gluestart{\hskip#1}%
                            \def\glueend{\hss}
                            \dim_gset:cn {chapter_margin_left}{#1},
  chapter~margin-right/.code = \dim_gset:cn {chapter_margin_right} { #1 },
  chapter~margin-bottom/.code = \dim_gset:cn {chapter_margin_bottom} { #1 },                           
  }

\ExplSyntaxOff 
%    \end{macrocode}
%
% \subsection{Chapter borders}
%
% Next we set keys for all border width
% \subsubsection{Chapter border widths}
%    \begin{macrocode}      
\ExplSyntaxOn
\cxset{  
   chapter~border-top-width/.code= \setlength\chapterbordertopwidth{#1}%
     \global\chapterbordertopwidth\chapterbordertopwidth\relax
     \dim_gset:cn {chapter_border_top_width} {#1},                                            
   chapter~border-right-width/.code= \setlength\chapterborderrightwidth{#1}%
     \global\chapterborderrightwidth\chapterborderrightwidth\relax
     \dim_gset:cn {chapter_border_right_width} {#1},                                                                                      
   chapter~border-bottom-width/.code= \setlength\chapterborderbottomwidth{#1}%
     \global\chapterborderbottomwidth\chapterborderbottomwidth\relax
     \dim_gset:cn {chapter_border_bottom_width} {#1},                                                                                                                              
   chapter~border-left-width/.code= \setlength\chapterborderleftwidth{#1}%
     \global\chapterborderleftwidth\chapterborderleftwidth\relax
     \dim_gset:cn {chapter_border_left_width} {#1},
}                        
\ExplSyntaxOff                                            
%    \end{macrocode}
  %  
  %   Next we write keys for developing the short form of the border keys. These keys reset
  %  all borders to one value
  %    \begin{macrocode}
\ExplSyntaxOn  
\cxset{  
  chapter~border-width/.code = \pgfkeysalso{chapter~border-top-width=#1,
                                            chapter~border-right-width=#1,
                                            chapter~border-bottom-width=#1,
                                            chapter~border-left-width=#1,
                                            },
}            
\ExplSyntaxOff                                     
%    \end{macrocode}  
% 
% We now deal with padding the same way including the generic version                                             
%    \begin{macrocode}  
\ExplSyntaxOn                                             
\cxset{
  chapter~padding-left/.code= \setlength\chapterpaddingleft{#1}%
    \global\chapterpaddingleft\chapterpaddingleft\relax
    \dim_gset:cn {chapter_padding_left}{#1},                                                                                                                           
%    
  chapter~padding-right/.code= \setlength\chapterpaddingright{#1}%
    \global\chapterpaddingright\chapterpaddingright\relax
    \dim_gset:cn {chapter_padding_right}{#1},                                                                                                             
%    
  chapter~padding-top/.code= \setlength\chapterpaddingtop{#1}%
    \global\chapterpaddingtop\chapterpaddingtop\relax
    \dim_gset:cn {chapter_padding_top}{#1},                                                                                                                                                       
% 
  chapter~padding-bottom/.code= \setlength\chapterpaddingbottom{#1}%
    \global\chapterpaddingbottom\chapterpaddingbottom\relax
    \dim_gset:cn {chapter_padding_bottom}{#1},                                                                                                                                                                                                 
}
\ExplSyntaxOff
%    \end{macrocode}    
%
% The default if no value is entered is 0pt. Also
% 
%    \begin{macrocode}
\cxset{
  chapter padding/.code={\def\@tempa{none}%
    \def\@tempb{#1}%
    \ifx\@tempa\@tempb%
      \global\setlength\chapterpaddingleft{0pt}%
      \global\setlength\chapterpaddingright{0pt}%
      \global\setlength\chapterpaddingtop{0pt}%
      \global\setlength\chapterpaddingbottom{0pt}%
    \else
      \setlength\chapterpaddingleft{#1}%
      \global\chapterpaddingleft\chapterpaddingleft\relax                                                                                                                           
      \setlength\chapterpaddingright{#1}%
      \setlength\chapterpaddingtop{#1}%
      \setlength\chapterpaddingbottom{#1}%
    \fi}
}    
%    \end{macrocode}  
%
% \subsubsection{Chapter letter spacing} 
% NEEDS REVISITING TO ALLOW FOR  SOUL OR MICROTYPE  LEAVE ALSO LETTER SPACING 
% ALSO TO TAKE OUT SPACEOUT
%    \begin{macrocode}   
\ExplSyntaxOn


\ExplSyntaxOff                                   
\cxset{
  chapter after/.store in=\chapterafter@cx,
  chapter spaceout/.is choice,
  chapter spaceout/soul/.code=\@chapterspaceouttrue\@soulspaceouttrue,
  chapter spaceout/microtype/.code=\@chapterspaceouttrue\@soulspaceouttrue,
  chapter spaceout/none/.code=\@chapterspaceoutfalse\@soulspaceoutfalse,
  %  
%  chapter letter-spacing/.is choice,
%  chapter letter-spacing/soul/.style=\pgfkeysalso{chapter spaceout=soul},
%  chapter letter-spacing/microtype/.style=\pgfkeysalso{chapter spaceout=microtype},
%  chapter letter-spacing/true/.code=\@chapterspaceouttrue,
%  chapter letter-spacing/none/.code=\@chapterspaceoutfalse,
%  chapter letter-spacing/false/.code=\@chapterspaceoutfalse,
 }  
%    \end{macrocode}  
%
%  Next we define styles. This must be distinguished from shapes and only
%  apply to rectangular boxed content, using \cmd{\phd@fbox}
%
%    \begin{macrocode}
\cxset{
  chapter border-top-style/.store in=\chapterbordertopstyle, 
  chapter border-right-style/.store in=\chapterborderrightstyle, 
  chapter border-bottom-style/.store in=\chapterborderbottomstyle,
  chapter border-left-style/.store in=\chapterborderleftstyle, 
  chapter border-style/.code=\pgfkeysalso{chapter border-top-style=#1,%
    chapter border-right-style=#1,%
    chapter border-bottom-style=#1,%
    chapter border-left-style=#1%,
  },
}
\cxset{
  chapter border-top-style=none,
  chapter border-right-style=none,
  chapter border-bottom-style=none,
  chapter border-left-style=none,}  
%    \end{macrocode}

%
%  \begin{macro}{\chaptershape}
%    \begin{macrocode}
\gdef\chaptershape{rounded rectangle}
\cxset{
chapter shape/.is choice,
  chapter shape/rectangle/.code=\gdef\chaptershape{rectangle},
  chapter shape/ellipse/.code=\gdef\chaptershape{ellipse},
  chapter shape/circle/.code=\gdef\chaptershape{circle},
  chapter shape/rounded rectangle/.code=\gdef\chaptershape{rounded rectangle},
  chapter shape/diamond/.code=\gdef\chaptershape{diamond},
  chapter shape/starburst/.code=\gdef\chaptershape{starburst},
  chapter shape/none/.code=\gdef\chaptershape{},
  chapter shape/star/.code=\gdef\chaptershape{star},
}
%    \end{macrocode}
%  \end{macro}
%
% \subsection{Chapter title parameters}
%   We define a key to set the chapter title text width. The calculated value of this box will
%   be less than that specified by the user, if padding and borders are defined. This is to avoid
%   overful boxes, when the user specifies full width such as textwidth.
%
%    \begin{macrocode}  
\ExplSyntaxOn
\dim_new:c {chapter_title_text_width}
\dim_new:c {chapter_title_width}
\cxset{ 
  chapter~title~width/.code= \gdef\chaptertitlewidth@cx{#1}
    \dim_gset:cn {chapter_title_width}{#1},
  title~text-width/.code=\gdef\chaptertitletextwidth@cx {#1}                                          
    \dim_gset:cn {chapter_title_text_width} {#1},
}    
\ExplSyntaxOff                                            
%    \end{macrocode}
%
%  Next we deal with the title alignment. The title is typeset in a minipage
%  We allow for the total to be positioned. The key text-align specifies the alignment
%  of the inner text block. 
%
%  TeX does not distinguish the type of boxes found in CSS. As a matter of fact TeX’s model
%  is much more complicated and also allows the different types to be nested indefinetly.
%  Rendering depends on the typesetting mode. 
%  The display block, should just add |\vskip|s and terminate horizontal mode. This might
%  avoid to have to type some keys.
%
% \begin{docKey}{title display}{ = \oarg{none,block,inline,inline-block}}{default block}
% The title display key determines how the title is aligned with its neighbours.
% It defaults to block, which it means is typeset on its own line.
% \end{docKey}
%    \begin{macrocode}
\ExplSyntaxOn
\int_gzero_new:c {chapter_title_display}
\tl_new:c {chapter_title_text_align}

%

\cxset{
  title~display/.is~choice,
  title~display/none/.code = \gdef\titledisplay@cx{0}
    \int_gset:cn {chapter_title_display}{0},
  title~display/block/.code = \gdef\titledisplay@cx{1}
    \int_gset:cn {chapter_title_display}{1},
  title~display/in-line block/.code=\gdef\titledisplay@cx{2}
    \int_gset:cn {chapter_title_display}{2},
  title~display/inline/.code=\gdef\titledisplay@cx{3}
    \int_gset:cn {chapter_title_display}{3},
  chapter~title~display/.style = \pgfkeysalso{title~display=#1}  
}
\cxset{title~display=block}  
\ExplSyntaxOff
%    \end{macrocode}
%
%    \begin{macrocode}  
\ExplSyntaxOn

\cxset{
  chapter~title~text-align/.is~choice,
  chapter~title~text-align/center/.code=\gdef\chaptertitletextalign@cx{\Centering}
    \tl_gset:cn {chapter_title_text_align}{\Centering},                                                                                                 
%    
  chapter~title~text-align/centering/.code=\gdef\chaptertitletextalign@cx{\centering}
    \tl_gset:cn {chapter_title_text_align}{\centering},                                                                                                 
%    
  chapter~title~text-align/none/.code=\gdef\chaptertitletextalign@cx{}
     \tl_gset:cn {chapter_title_text_align}{},                                                                                                 
%     
  chapter~title~text-align/justified/.code=\gdef\chaptertitletextalign@cx{}
    \tl_gset:cn {chapter_title_text_align}{},
%  
  chapter~title~text-align/left/.code=\gdef\chaptertitletextalign@cx{\RaggedRight}
    \tl_gset:cn {chapter_title_text_align}{\RaggedRight},
%    
  chapter~title~text-align/raggedleft/.code=\gdef\chaptertitletextalign@cx{\RaggedLeft}
   \tl_gset:cn {chapter_title_text_align}{\RaggedLeft},
%   
  chapter~title~text-align/right/.code=\gdef\chaptertitletextalign@cx{\RaggedLeft}
   \tl_gset:cn {chapter_title_text_align}{\RaggedLeft},
%   
  chapter~title~text-align/raggedright/.code=\gdef\chaptertitletextalign@cx{\RaggedRight}
   \tl_gset:cn {chapter_title_text_align}{\RaggedRight},
}
\ExplSyntaxOff
%
\cxset{chapter title text-align=left}
\ExplSyntaxOn
\tl_new:c {chapter_title_align}
\cxset{    
  % aligning the block title 
  chapter~title~align/.is~choice,
  chapter~title~align/centering/.code=\def\gluestart{\hss}\def\glueend{\hss}
   \tl_gset:cn {chapter_title_align}{centering},
  % alias
  chapter~title~align/center/.code=\def\gluestart{\hss}\def\glueend{\hss}, 
%   
  chapter~title~align/raggedright/.code=\def\gluestart{\hss}\def\glueend{},
%  
  chapter~title~align/raggedleft/.code=\def\gluestart{}\def\glueend{\hss}
    \tl_gset:cn {chapter_title_align}{raggedleft},
%                                                          
  chapter~title~align/right/.code=\def\gluestart{\hss}\def\glueend{}                                                      
    \tl_gset:cn {chapter_title_align}{right},  
%  
  chapter~title~align/left/.code=\def\gluestart{}\def\glueend{\hss}
    \tl_gset:cn {chapter_title_align}{left},
%                                                 
  chapter~title~align/none/.code=\def\gluestart{}\def\glueend{}
     \tl_gset:cn {chapter_title_align}{none},
}

%  
\ExplSyntaxOff
\cxset{chapter title align=centering}
%
%

\cxset{
  title font-family/.store in=\titlefontfamily@cx,
  title font-weight/.font-weight in=\titlefontweight@cx,
  title font-size/.store in=\titlefontsize@cx,
  title font-color/.store in=\titlefontcolor@cx,
  title font-shape/.store in=\titlefontshape@cx}

\cxset{title font-shape=upshape}
%    \end{macrocode}
%
%  Letter-spacing is handled in a similar fashion defining keys both for 
%  the common \latex community terminology (spaceout) and also
% using |letter-spacing|.
%

%    \begin{macrocode}
\cxset{  
  title spaceout/.is choice,
  title spaceout/soul/.code=\@titlespaceouttrue,
  title spaceout/none/.code=\@titlespaceoutfalse,
  title spaceout/true/.code=\@titlespaceouttrue,
  title spaceout/false/.code=\@titlespaceoutfalse,
  title letter-spacing/true/.code=\@titlespaceouttrue,
  title letter-spacing/false/.code=\@titlespaceoutfalse,  
}
%    \end{macrocode}
%
%    \begin{macrocode}
\cxset{  
  title font/.style={title font-family=#1},
  title before/.store in=\titlebefore@cx,
  title after/.store in=\titleafter@cx,
  title beforeskip/.store in=\titlebeforeskip@cx,
  }
%    \end{macrocode}
%
%   The margin top property should primarily be used for non-inline headings. If the value is
%   zero we ensure we stay in horizontal mode. We also ensure that only specified skips
%   are inserted so as to be able to specify exact amounts for grids.
%
%   \begin{macrocode}
\pgfkeys{/handlers/.dimstore in/.code=\pgfkeysalso{\pgfkeyscurrentpath/.code=\def#1{\the\dimexpr##1\relax}}}%
%
\newlength\titlemarginleft \setlength\titlemarginleft{0pt}

\cxset{
  title margin top/.code=%
      \gdef\titlemargintop@cx{%
           \global\setlength\titlemargintop{#1}%
           \ifdim#1=0pt\relax%
            \else%
                \par\ifvmode\nointerlineskip%
                    \addvspace{\glueexpr#1-\parskip}% 
                \fi%    
           \fi%
        },
  title margin-top/.style={title margin top=#1},       
      %        
  title margin bottom/.code=%
    \gdef\titlemarginbottom@cx{%
    \global\setlength\titlemarginbottom{\the\dimexpr#1\relax}%
    \par\nointerlineskip
    \vspace*{#1}% 
    }%
    \titlemarginbottom@cx,
  title margin-bottom/.style =\pgfkeysalso{title margin bottom=#1},      
  title margin-left/.code=\global\setlength{\titlemarginleft}{#1}
                                      \gdef\titlemarginleft@cx{\hspace*{#1}},%,
  title margin right/.code=\def\titlemarginright@cx{\rightskip#1},
  title afterskip/.store in=\titleafterskip@cx,
  position/.is choice,
  position/left/.code={\@lefttrue},
  position/right/.code={\@righttrue},
  position/center/.code={\@centertrue},
  % title padding
  title padding-top/.code=\global\setlength{\titlepaddingtop}{#1},
  title padding-bottom/.code=\global\setlength{\titlepaddingbottom}{#1},
  title padding-left/.code=\global\setlength{\titlepaddingleft}{#1},
  title padding-right/.code=\global\setlength{\titlepaddingright}{#1},
  title padding/.code=\global\setlength{\titlepaddingtop}{#1}%
                                  \global\setlength{\titlepaddingbottom}{#1}%
                                  \global\setlength{\titlepaddingright}{#1}%
                                  \global\setlength{\titlepaddingleft}{#1},
  % borders left
  title border-top-color/.store in=\titlebordertopcolor@cx,
  title border-top-width/.code=\global\setlength\titlebordertopwidth{#1},
  % borders left
  title border-left-color/.store in=\titleborderleftcolor@cx,
  title border-left-width/.code=\global\setlength\titleborderleftwidth{#1},
  % borders right
  title border-right-color/.store in=\titleborderrightcolor@cx,
  title border-right-width/.code=\global\setlength\titleborderrightwidth{#1},
  % borders bottom
  title border-bottom-color/.store in=\titleborderbottomcolor@cx,
  title border-bottom-width/.code=\global\setlength\titleborderbottomwidth{#1},
  % we set the full short-hand keys
  title border-color/.code=\def\titlebordercolor@cx{#1}%
                                       \def\titleborderleftcolor@cx{#1}%
                                       \def\titleborderrightcolor@cx{#1}%
                                       \def\titlebordertopcolor@cx{#1}%
                                       \def\titleborderbottomcolor@cx{#1},
  title border-width/.code=\global\setlength\titleborderwidth{#1}%
                                         \global\setlength\titlebordertopwidth{#1}%
                                         \global\setlength\titleborderrightwidth{#1}%
                                         \global\setlength\titleborderbottomwidth{#1}%
                                         \global\setlength\titleborderleftwidth{#1},
%    \end{macrocode}
%
% The numbering keys deal with the typesetting of the chapter number
% in the chapter head. We use two packages for expressing numbers into
% words. The padzeroes is to produce EWD style notes. 
%
%    \begin{macrocode}
  chapter numbering/.is choice,
  chapter numbering/none/.code={\gdef\thechapter{}},
  chapter numbering/roman/.code={\gdef\thechapter{\@roman\c@chapter}},
  chapter numbering/Roman/.code={\gdef\thechapter{\@Roman\c@chapter}},
  chapter numbering/arabic/.code={\gdef\thechapter{\@arabic\c@chapter}},
  chapter numbering/alpha/.code={\gdef\thechapter{\alphalph\c@chapter}\relax},
  chapter numbering/Alpha/.code={\gdef\thechapter{\AlphAlph\c@chapter}},
  chapter  numbering/words/.code={\gdef\thechapter{\expandafter\words@cx{\expandafter\@arabic\c@chapter}}},
  chapter numbering/WORDS/.code= {\gdef\thechapter{\expandafter\WORDS@cx{\expandafter\@arabic\c@chapter}}},
  chapter numbering/ORDINALS/.code=\gdef\thechapter{%
  \expandafter\ordinals@cx{\@arabic\c@chapter}},%{\gdef\thechapter{\NUMBERstring{chapter}}},
  chapter numbering/Words/.code={\gdef\thechapter{\expandafter\Words@cx{\expandafter\@arabic\c@chapter}}},
chapter numbering/padzeroes/.code={\gdef\thechapter{\mbox{EWD -\padzeroes[4]\decimal{chapter}}}},%
  numbering/.is choice,
  numbering/roman/.code={\gdef\thechapter{\@roman\c@chapter}},
  numbering/Roman/.code={\gdef\thechapter{\@Roman\c@chapter}},
  numbering/arabic/.code={\gdef\thechapter{\@arabic\c@chapter}},
  numbering/alpha/.code={\gdef\thechapter{\alphalph\c@chapter}},
  numbering/Alpha/.code={\gdef\thechapter{\AlphAlph\c@chapter}},
  %numbering/words/.code={\gdef\thechapter{\MakeTextLowercase{\expandafter\words@cx{\expandafter\@arabic\c@chapter}}}},
  numbering/WORDS/.code={\gdef\thechapter{\expandafter\WORDS@cx{\expandafter\@arabic\c@chapter}}},
%% These proved a bit trouble some and ended up calling the fmtcount package routines
  %numbering/WORDS/.code={\gdef\thechapter{\NUMBERstring{chapter}}},
  numbering/Words/.code={\gdef\thechapter{\expandafter\Words@cx{\expandafter\@arabic\c@chapter}}},
  numbering/padzeroes/.code={\gdef\thechapter{\mbox{EWD -\padzeroes[4]\decimal{chapter}}
  }},
  numbering/none/.code={\gdef\thechapter{}}, % do not leave empty
  chapter numbering custom/.code=\gdef\thechapter{#1},
   number spaceout/.is choice,
  number spaceout/soul/.code=\@numberspaceouttrue,
  number spaceout/none/.code=\@numberspaceoutfalse,
  number spaceout/inherit/.code=\let\@numberspaceout\@chapterspaceout,
  number spaceout/microtype/.code=\@numberspaceouttrue,
  number letter-spacing/.code=\pgfkeysalso{number spaceout=soul},
  number dot/.store in=\numberpunctuation@cx,
  number position/.is choice,
  number position/leftname/.code={\@leftnametrue\@rightnamefalse},
  number position/rightname/.code={\@rightnametrue\@leftnamefalse},
  number position/absolute/.code={},
  number position/righttitle/.code=\@righttitletrue,
  number position/lefttitle/.code=\@lefttitletrue,
  number after/.store in=\numberafter@cx,
  number after content/.store in=\numberaftercontent@cx,
  number before/.store in=\numberbefore@cx,
  number background-color/.code=\gdef\numberbgcolor{#1},
  number color/.store in=\numbercolor@cx,
  number font-size/.store in=\numberfontsize@cx,
  number font-family/.font-family in=\numberfontfamily@cx,
  %% new style handlers
  number font-weight/.font-weight in=\numberfontweight@cx,
  number font-shape/.font-style in =\numberfontshape@cx,
  number font-style/.font-style in=\numberfontshape@cx,
  number font-name/.store in=\numberfontname@cx,%new
  %% 
  number margin top/.store in=\numbermargintop@cx,
  number margin left/.store in=\numbermarginleft@cx,
  number margin right/.store in=\numbermarginright@cx,
  % number borders
  number border top/.store in=\numberborderleft@cx,
  number border bottom/.store in=\numberborderbottom@cx,
  number border-top-width/.code=\global\setlength\numberbordertopwidth{#1},
  number border-bottom-width/.code=\global\setlength\numberborderbottomwidth{#1},
  number border-left-width/.code=\global\setlength\numberborderleftwidth{#1},
  number border-right-width/.code=\global\setlength\numberborderrightwidth{#1},
  number border-width/.code=\pgfkeysalso{number border-top-width=#1,
                                                         number border-right-width=#1,
                                                         number border-bottom-width=#1,
                                                         number border-left-width=#1},
  number display/.is choice,
  number display/inline/.code=\global\setcounter{numberdisplay}{0},
  number display/block/.code=\global\setcounter{numberdisplay}{2},   
  number float/.is choice,
  number float/left/.code=\global\setcounter{numberfloat}{0},
  number float/none/.code=\global\setcounter{numberfloat}{0},                                                    
  number float/center/.code=\global\setcounter{numberfloat}{1},
  number float/right/.code=\global\setcounter{numberfloat}{2},     
}
%    \end{macrocode}
%  \subsection{Shapes}
% I wasn’t too sure how to incorporate this in a nice way, so I defined a new property key,
% shape
% A shape can also have a style, if you want to really get fancy.
%  \begin{macro}{\numbershape}
%    \begin{macrocode}
\gdef\numbershape{rounded rectangle}
\cxset{
  number shape/.is choice,
  number shape/rectangle/.code=\gdef\numbershape{rectangle},
  number shape/ellipse/.code=\gdef\numbershape{ellipse},
  number shape/circle/.code=\gdef\numbershape{circle},
  number shape/rounded rectangle/.code=\gdef\numbershape{rounded rectangle},
  number shape/diamond/.code=\gdef\numbershape{diamond},
  number shape/starburst/.code=\gdef\numbershape{starburst},
  number shape/none/.code=\gdef\numbershape{},
  number shape/star/.code=\gdef\numbershape{star},
}
%    \end{macrocode}
%  \end{macro}
%
% We define border styles first individually per side and then globally with
% a short-hand key.
% CSS has a dotted solid dashed double
% 
%    \begin{macrocode}
\newcounter{numberborderstyle} \global\setcounter{numberborderstyle}{0}
\cxset{                                                         
  number border-style/.is choice,
  number border-style/solid/.code=\def\numberborderstyle@cx{1},
  number border-style/double/.code=\def\numberborderstyle@cx{2},
  number border-style/dashed/.code=\def\numberborderstyle@cx{3},                                                         
  number border-style/none/.code=\def\numberborderstyle@cx{-1},%
}  
% number padding
\cxset{  
  number padding-top/.code=\global\setlength\numberpaddingtop{#1},
  number padding-right/.code=\global\setlength\numberpaddingright{#1},
  number padding-bottom/.code=\global\setlength\numberpaddingbottom{#1},
  number padding-left/.code=\global\setlength\numberpaddingleft{#1},
  number padding/.code=\pgfkeysalso{number padding-top=#1,
                                                           number padding-right=#1,
                                                           number padding-bottom=#1,
                                                           number padding-left=#1},%
}   
%    \end{macrocode}                                                        
%
% \subsection{Author blocks}
% 
% Author blocks are only set if the boolean |\@authorblock| is set to true.
%
%    \begin{macrocode}
\cxset{
  author block/.is choice,
  author block/true/.code={\@authorblocktrue},
  author block/false/.code={\@authorblockfalse},
  author names/.store in=\authorblock@cx,  
  author block format/.store in=\authorblockformat@cx,
  author block afterskip/.store in=\authorblockafterskip@cx,
  chapter toc/.is choice,
  chapter toc/true/.code=\@toctrue,
  chapter toc/false/.code=\@tocfalse,
  chapter toc/none/.code=\@tocfalse,
}
%    \end{macrocode}
    
%    \begin{macrocode}    
\def\debugtitle{%                                         
\cxset{title border-top-color=sweet,
          title border-top-width=10pt,
          title border-left-color=sweet,
          title border-left-width=10pt,
          title border-right-color=sweet,
          title border-right-width=20pt,
          title border-bottom-color=sweet,
          title border-bottom-width=20pt,
          title border-width=0.2pt,
          title border-color=red,
          title padding-top=50pt,
          title padding-bottom=50pt,
          title padding-left=0pt,
          title padding-right=0pt,
          title padding=0pt,
          }}                
\cxset{chapter margin-top=0pt,
          chapter margin-left=20pt,
          chapter title align=left,
          chapter background-color=white,
          chapter border-left-width=0pt,
          chapter border-right-width=0pt,
          chapter border-bottom-width=0pt,
          chapter border-top-width=0pt,
          chapter font-shape=upshape,
}
\cxset{number background-color=white,
          number padding-left=0pt,
          number padding-right=0pt,
          number padding-top=0pt,
          number padding-bottom=0pt,
          number border-top-width=0pt,
          number border-bottom-width=0pt,
          number border-left-width=0pt,
          number border-right-width=0pt,
          number border-style=solid,
          number font-shape=upshape}
\cxset{author block=false,
          author block afterskip=,
          title margin top=0pt,
          title margin bottom=0pt,
          title margin-left=0pt,
          title margin right=0pt,
          title border-top-color=sweet,
          title border-top-width=10pt,
          title border-left-color=sweet,
          title border-left-width=10pt,
          title border-right-color=sweet,
          title border-right-width=20pt,
          title border-bottom-color=sweet,
          title border-bottom-width=20pt,
          title border-width=0pt,
          title border-color=red,
          title padding-top=50pt,
          title padding-bottom=50pt,
          title padding-left=50pt,
          title padding-right=50pt,
          title padding=0pt,
          chapter title text-align=center,
          title display=block,
          }
\cxset{author names=}
\cxset{author block format=}
\cxset{chapter title width=0.7\textwidth}
\cxset{chapter title align=centering}
%    \end{macrocode}
%
% \begin{macro}{\debugchapter}
%    \begin{macrocode}
\def\debugchapter{%
\cxset{chapter margin-top=0pt,
          chapter margin-left=0pt,
          chapter background-color=white,
%          
          chapter border-left-width=0.2pt,
          chapter border-right-width=0.2pt,
          chapter border-bottom-width=0.2pt,
          chapter border-top-width=0.2pt,
%          
          chapter padding-top=1pt,
          chapter padding-bottom=0pt,
          chapter padding-left=0pt,
          chapter padding-right=0pt,
%         
          number border-left-width=0.2pt,
          number border-right-width=0.2pt,
          number border-bottom-width=0.2pt,
          number border-top-width=0.2pt,
%          
          number padding-top=1pt,
          number padding-bottom=0pt,
          number padding-left=0pt,
          number padding-right=0pt,
}}
\debugchapter
%    \end{macrocode}
% \end{macro}
%
% \section{Stacked Heads}
%
% We now define keys for stacked heads, in a similar fashion to the chapter section. All keys
% are prefixed with `section' and font related commands are similar to 
% those found in CSS. As users that are familiar with pgf conventions might make mistakes by 
% writing |\cxset{section font size}| rather than |\cxset{section font-size}|  we create aliases to cater
% for both. 
%
% \begin{macro}{section font-size}
%    \begin{macrocode}
\cxset{
  section font-size/.store in=\sectionfontsize@cx,
  section font size/.store in=\sectionfontsize@cx,
  section font-weight/.store in=\sectionfontweight@cx,
  section font-family/.store in=\sectionfontfamily@cx,
  section font family/.store in = \sectionfontfamly@cx,
  section font-shape/.store in=\sectionfontshape@cx,
  section color/.code=\gdef\sectioncolor@cx{#1}\renewsection,
  section color/.initial=black,
  section color/.default=black,
 } 
%    \end{macrocode}
% \end{macro}
%
% Next we define keys for the section numbering system. We cater for 
% |roman|, |Roman|, and within brackets |(roman)|, |arabic| or |numeric|. 
% Since unlike the standard
% classes we are aiming at a more generic template we need to care
% for the document type. If we have a chapter we will allow prefixing 
% of numbers. We use |sectionnumberingprefix@cx| as a key. 
% 
% \begin{macro}{section numbering}
%    \begin{macrocode}
\def\sectionnumberingsuffix@cx{}% 
\def\sectionnumberingprefix@cx{}%
\cxset{%
  section numbering suffix/.store in=\sectionnumberingsuffix@cx, 
  %alias
  section number after/.store in =\sectionnumberafter@cx,
  section numbering prefix/.store in=\sectionnumberingprefix@cx,
  section numbering/.is choice,
  section numbering/roman/.code={%
       \gdef\thesection{\sectionnumberingprefix@cx\@roman\c@section}%
          \renewsection},
 section numbering/Roman/.code={%
       \gdef\thesection{\sectionnumberingprefix@cx\@Roman\c@section}%
          \renewsection},
  section numbering/(roman)/.code={%
       \gdef\thesection{\sectionnumberingprefix@cx(\@roman\c@section)}%
       \renewsection},
  section numbering/(Roman)/.code={%
       \gdef\thesection{\sectionnumberingprefix@cx(\@Roman\c@section)}%
          \renewsection},
  section numbering/arabic/.code={%
       \gdef\thesection{\sectionnumberingprefix@cx\@arabic\c@section\sectionnumberingsuffix@cx}%
          \renewsection},
  section numbering/numeric/.code={%
       \gdef\thesection{\sectionnumberingprefix@cx\@arabic\c@section\sectionnumberingsuffix@cx}%
          \renewsection},
  section numbering/none/.code={\gdef\thesection{\hspace*{-1em}}\renewsection},
  section numbering/alpha/.code={\gdef\thesection{\alphalph\c@section}},
  section numbering/Alpha/.code={\gdef\thesection{\AlphAlph\c@section}},
  section numbering/words/.code={\gdef\thesection{\sectionnumberingprefix@cx%
                                                   \words@cx{\@arabic\c@section}}},
  section numbering/Words/.code={%
 \gdef\thesection{\sectionnumberingprefix@cx\words@cx{\@arabic\c@section}}},
   section numbering/WORDS/.code={\gdef\thesection{\sectionnumberingprefix@cx \words@cx{\@arabic\c@section}}},
%    \end{macrocode}
% \end{macro}
%
% The |section numbering custom| is a catch-all key to define a special
% definition for |thesection|. Just pass on the tokens you require.t
%    \begin{macrocode}
     section numbering custom/.code=\gdef\thesection{#1}\renewsection,
%    \end{macrocode}
%
%  We next define choice keys for the alignment of sections. These can be one
% of |left|, |right|, |center| or |centering|.
%
%    \begin{macrocode}
  section align/.is choice,
  section align/right/.code = \gdef\sectionalign@cx{flushright},
  section align/center/.code = \gdef\sectionalign@cx{\centering},
  section align/centering/.code = \gdef\sectionalign@cx{\centering},
  section align/Centering/.code = \gdef\sectionalign@cx{\Centering},
  section align/left/.code = \gdef\sectionalign@cx{flushleft},
   section align/flushleft/.code = \gdef\sectionalign@cx{flushleft},
  section align/right/.code=\gdef\sectionalign@cx{flushright},
  section align/flushright/.code=\gdef\sectionalign@cx{flushright},
  section align/RaggedRight/.code=\gdef\sectionalign@cx{RaggedRight},
  section align/raggedright/.code=\gdef\sectionalign@cx{RaggedRight},
  %
  section afterindent/.is choice,
  section afterindent/on/.code = \afterindenton@cx,
  section afterindent/off/.code = \afterindentoff@cx,
  section afterindent/true/.code = \afterindenton@cx,
  section afterindent/false/.code = \afterindentoff@cx,
  %
  section beforeskip/.store in=\sectionbeforeskip@cx,
  section afterskip/.store in=\sectionafterskip@cx,
  section indent/.store in=\sectionindent@cx,
  section spaceout/.is choice,
  section spaceout/soul/.code={\@sectionspaceouttrue},
  section spaceout/none/.code={\@sectionspaceoutfalse},
  section number after/.store in=\sectionnumberafter@cx,
%    \end{macrocode}
%
% \begin{macro}{subsection options}
% 
% From now on almost everything is a repetition of whatever was previously
% defined for higher order sectioning commands.
%
%    \begin{macrocode}
% subsections
  subsection font-size/.store in=\subsectionfontsize@cx,
  subsection font-weight/.store in=\subsectionfontweight@cx,
  subsection font-family/.store in=\subsectionfontfamily@cx,
  subsection font-shape/.store in=\subsectionfontshape@cx,
%
  subsection color/.store in=\subsectioncolor@cx,
  subsection numbering prefix/.store in = \subsectionnumberingprefix@cx,
%    \end{macrocode}
%  Numbering choices
%    \begin{macrocode}
  subsection numbering/.is choice,
  subsection numbering/arabic/.code={%
    \gdef\thesubsection{\subsectionnumberingprefix@cx\@arabic\c@subsection}},
  subsection numbering/custom/.store in=\thesubsection@cx,
  subsection numbering/none/.code={\gdef\thesubsection{\hspace*{-1em}}\renewsubsection},
  subsection align/.store in=\subsectionalign@cx,
  subsection beforeskip/.store in=\subsectionbeforeskip@cx,
  subsection afterskip/.store in=\subsectionafterskip@cx,
  subsection indent/.store in=\subsectionindent@cx,
  subsection numbering custom/.code=\gdef\thesubsection{%
          \subsectionnumberingprefix@cx#1}\renewsubsection,
  subsection number after/.store in=\subsectionnumberafter@cx,       
%    \end{macrocode}
% \end{macro}
%
% \subsubsection{subsubsections}
% 
% we are now five levels down at the headings
% Part, Chapter, section, subsection, subsubsection and lots of cut and paste and
% modifying commands.
% 
%    \begin{macrocode}
  % subsubsection keys
  %
  subsubsection font-size/.code = \gdef\subsubsectionfontsize@cx{#1},
  subsubsection font-weight/.store in=\subsubsectionfontweight@cx,
  subsubsection font-family/.store in=\subsubsectionfontfamily@cx,
  subsubsection font-shape/.store in=\subsubsectionfontshape@cx,
  % no hyphen vesion
  subsubsection font size/.store in=\subsubsectionfontsize@cx,
  subsubsection font weight/.store in=\subsubsectionfontweight@cx,
  subsubsection font family/.store in=\subsubsectionfontfamily@cx,
  subsubsection font shape/.store in=\subsubsectionfontshape@cx,
  %e
  subsubsection color/.store in=\subsubsectioncolor@cx,
  subsubsection numbering prefix/.store in = \subsubsectionnumberingprefix@cx,
  subsubsection numbering/.is choice,
  subsubsection numbering/arabic/.code={%
    \gdef\thesubsubsection{\subsubsectionnumberingprefix@cx\@arabic\c@subsubsection}%
         \renewsubsubsection},
  subsubsection numbering/numeric/.code={\gdef\thesubsubsection{\thesubsection.\@arabic\c@subsubsection}\renewsubsubsection},
  subsubsection numbering custom/.code= \gdef\thesubsubsection{%
       \subsubsectionnumberingprefix@cx#1}\renewsubsubsection,
  subsubsection numbering/none/.code={\gdef\thesubsubsection{}\renewsubsubsection},
  % needs handler
  subsubsection align/.store in=\subsubsectionalign@cx,
  subsubsection beforeskip/.store in=\subsubsectionbeforeskip@cx,
  subsubsection afterskip/.store in=\subsubsectionafterskip@cx,
  subsubsection indent/.store in=\subsubsectionindent@cx,
  subsubsection number after/.store in=\subsubsectionnumberafter@cx,
    %
% paragraph
%
  paragraph font-size/.store in = \paragraphfontsize@cx,
  paragraph font size/.store in = \paragraphfontsize@cx,
  paragraph font-weight/.store in=\paragraphfontweight@cx,
  paragraph font-family/.store in=\paragraphfontfamily@cx,
  paragraph font-shape/.store in=\paragraphfontshape@cx,
  paragraph color/.store in=\paragraphcolor@cx,
  paragraph numbering/.is choice,
  paragraph numbering/numeric/.code={\gdef\theparagraph{\thesubsubsection.\@arabic\c@paragraph}},
  paragraph numbering/custom/.store in=\theparagraph@cx,
  paragraph numbering/none/.code={\gdef\theparagraph{}},
  paragraph align/.store in=\paragraphalign@cx,
  paragraph beforeskip/.store in=\paragraphbeforeskip@cx,
  paragraph afterskip/.store in=\paragraphafterskip@cx,
  paragraph indent/.store in=\paragraphindent@cx,
  paragraph number after/.store in=\paragraphnumberafter@cx,
%% subparagraphs
%
  subparagraph font-size/.store in=\subparagraphfontsize@cx,
  subparagraph font-weight/.store in=\subparagraphfontweight@cx,
  subparagraph font-family/.store in=\subparagraphfontfamily@cx,
  subparagraph font-shape/.store in=\subparagraphfontshape@cx,
  subparagraph color/.store in=\subparagraphcolor@cx,
  subparagraph numbering/.is choice,
  subparagraph numbering/numeric/.code={\gdef\thesubparagraph{\theparagraph.\@arabic\c@subparagraph}},
  subparagraph numbering/arabic/.code={\gdef\thesubparagraph{\theparagraph.\@arabic\c@subparagraph}},
  subparagraph numbering/custom/.store in=\thesubparagraph@cx,
  subparagraph numbering/none/.code={\gdef\thesubparagraph{}},
  subparagraph align/.store in=\subparagraphalign@cx,
  subparagraph beforeskip/.store in=\subparagraphbeforeskip@cx,
  subparagraph afterskip/.store in=\subparagraphafterskip@cx,
  subparagraph indent/.store in=\subparagraphindent@cx,
  subparagraph number after/.store in=\subparagraphnumberafter@cx,
  subparagraph number after/.default=,
  subparagraph number after/.initial=,
%
%% headers and footers
  header style/.store in=\headerstyle@cx,
% general draft rules
  rule /.is choice,
  rule on/.code={\gdef\rulewidth@cx{0.4pt}},
  rule off/.code={\gdef\rulewidth@cx{0pt}},
% headers and footers
  lhead/.code ={\lhead{#1}},
  rhead/.code={\rhead{#1}},
  chead/.code={\chead{#1}},
  lfoot/.code ={\lhead{#1}},
  cfoot/.code={\chead{#1}},
  rfoot/.code={\rhead{#1}},
  headrulewidth/.code={\renewcommand\headrulewidth{#1}},
  footrulewidth/.code={\renewcommand\footrulewidth{#1}},
}
%    \end{macrocode}
%

% \subsection{Renewsection commands}
%
% These have to be called explicitly after key definitions, it is just 
% the way LaTeX works. One could add them in settings or explore a
% bit more deeply.
%
% We also define \cs{@startsection} as somehow there are problems
% with after indent false.
%
%  |#1|  name i.e, section
%  |#2| level number 2 section
%  |#3| indent
%  |#4| beforeskip
%  |#5| afterskip
%  |#6|  styling command
%
%    \begin{macrocode}
\def\@startsection#1#2#3#4#5#6{%
    \if@noskipsec \leavevmode \fi
    \par
    \@tempskipa #4\relax 
    \afterindent@cx%\@afterindentfalse
    \ifdim \@tempskipa <\z@
        \@tempskipa -\@tempskipa\afterindent@cx %\@afterindentfalse
     \fi
\if@nobreak
\everypar{}%
\else
   \addpenalty\@secpenalty\addvspace\@tempskipa
\fi
\@ifstar
 {\@ssect{#3}{#4}{#5}{#6}}%defined in the kernel
{\@dblarg{\@sect{#1}{#2}{#3}{#4}{#5}{#6}}}}%defined in the kernel
%    \end{macrocode}
%
% When LaTeX is typesetting the section number it calls |\@seccntformat|. This is common for
% all the subsectioning commands. We modify it based on code from \pkgname{sectsty} in order
% to generalize it.
%
% We first check if \meta{section}|@cntformat| is defined and then we redirect
% to specific section level command.
%
% \begin{macro}{\@seccntformat}
%    \begin{macrocode}
 \def\@seccntformat#1{\@ifundefined{#1@cntformat}%
{\csname the#1\endcsname\sectionnumberafter@cx}% default
{\csname #1@cntformat\endcsname}% individual control
}
\long\def\testsections{%
\section{Sections}
\lorem\par
\subsection{Subsections}
\lorem\par
\subsubsection{Subsubsections}
\lorem\par
}
\def\sectionnumberafter@cx{\quad}%default value only space
\def\subsectionnumberafter@cx{\quad}%default value only space
\def\subsubsectionnumberafter@cx{\quad}%default value only space

\def\section@cntformat{\thesection\sectionnumberafter@cx}
\def\subsection@cntformat{\thesubsection\subsectionnumberafter@cx}
\def\subsubsection@cntformat{\thesubsubsection\subsectionnumberafter@cx\quad}
%    \end{macrocode}
% \end{macro}
%    \begin{macrocode}
\def\renewsection{%
\renewcommand\section{%
\@startsection{section}%
{1}%level check this conflicts with source2e
{\sectionindent@cx}%indent#2
{\sectionbeforeskip@cx}%before skip#3
{\sectionafterskip@cx}% after skip#4
{% 
 \setfont@cx{\sectionfontweight@cx}%
    {\sectionfontfamily@cx}{\sectionfontsize@cx}{\sectionfontshape@cx}%
   \expandafter\setfontparam@cx\sectionalign@cx;%
   \color{\sectioncolor@cx}%5
}}%
}%
%    \end{macrocode}
%
% Next we set the keys to a default style to avoid errors, if the user does not set them.
%
%    \begin{macrocode}
\cxset{
  section font-size= LARGE,
  section font size= LARGE,
  section font-weight=mdseries,
  section font family = sffamily,
  section font-shape= upshape,
  section color =spot!50,
  section numbering prefix=,
  section numbering=arabic,
  section indent=0pt,
  section beforeskip=0pt,
  section afterskip=10pt,
  section afterindent=off,
  section align=centering,
  section numbering suffix=,
  section number after=\quad,
}  
\renewsection
%    \end{macrocode}

%     \begin{macrocode}
\def\renewsubsection{%
\renewcommand\subsection{%
 \@startsection{subsection}%
{2}%level
{\subsectionindent@cx}%indent
{\subsectionbeforeskip@cx}%
{\subsectionafterskip@cx}%
{\setfont@cx{\subsectionfontweight@cx}%
    {\subsectionfontfamily@cx}{\subsectionfontsize@cx}{\subsectionfontshape@cx}%
   \expandafter\setfontparam@cx\subsectionalign@cx;%
  \color{\subsectioncolor@cx}%
}%
}%
}
%    \end{macrocode}
%
%  The |subsubsection|  keys need to be activated with a renew command.
%   \begin{macrocode}
\def\renewsubsubsection{%
\renewcommand\subsubsection{%
 \@startsection{subsubsection}%
{3}%level
{\subsubsectionindent@cx}%indent
{\subsubsectionbeforeskip@cx}%
{\subsubsectionafterskip@cx}%
{\setfont@cx{\subsubsectionfontweight@cx}%
    {\subsubsectionfontfamily@cx}{\subsubsectionfontsize@cx}{\subsubsectionfontshape@cx}%
   \expandafter\setfontparam@cx\subsubsectionalign@cx;%
  \color{\subsubsectioncolor@cx}%
}%
}
 %\def\@seccntformat##1{\csname the##1\endcsname\subsubsectionnumberafter@cx\hskip.5em}%
}

\cxset{
  subsubsection font-family=tiresias, 
  subsubsection font-size= large,
  subsubsection font size= large,
  subsubsection font-weight=bfseries,
  subsubsection font-family= tiresias,
  subsubsection font family = tiresias,
  subsubsection font-shape= upshape,
  subsubsection color =spot!50,
  subsubsection numbering prefix= ,
  subsubsection numbering = arabic,
  subsubsection indent=0pt,
  subsubsection beforeskip=0pt,
  subsubsection afterskip=10pt,
  subsubsection align=flushleft,
  subsubsection number after=,
}
\renewsubsubsection
%    \end{macrocode}
%
%
% \section{Runin Heads}
%
%  We now deal with paragraphs and subparagraphs, normally termed `runin’ heads, as they produce
%  headings that are inlined with the text that follows. We add hooks, so that later the key mechanism
%  can be used to pick-up values. 
% 
% \begin{macro}{\renewparagraph}
%    \begin{macrocode}
\def\renewparagraph{%
  \renewcommand\paragraph{%
     \@startsection{paragraph}%
     {4}%level
     {\paragraphindent@cx}%indent
     {\paragraphbeforeskip@cx}%
     {\paragraphafterskip@cx}%
     {\setfont@cx{\paragraphfontweight@cx}%
     {\paragraphfontfamily@cx}{\paragraphfontsize@cx}{\paragraphfontshape@cx}%
     \expandafter\setfontparam@cx\paragraphalign@cx;%
         \color{\paragraphcolor@cx}%
     }%
 }
%\def\@seccntformat##1{\csname the##1\endcsname\paragraphnumberafter@cx\hskip.5em}%
}

\def\renewsubparagraph{%
\renewcommand\subparagraph{%
 \@startsection{subparagraph}%
{5}%level
{\subparagraphindent@cx}%indent
{\subparagraphbeforeskip@cx}%
{\subparagraphafterskip@cx}%
{\setfont@cx{\subparagraphfontweight@cx}%
    {\subparagraphfontfamily@cx}{\subparagraphfontsize@cx}{\subparagraphfontshape@cx}%
   \expandafter\setfontparam@cx\subparagraphalign@cx;%
   \color{\subparagraphcolor@cx}%
 }%
}%This command formats the section number including the space following it.
}
%    \end{macrocode}
% \end{macro}
%
% \subsection{Setting up the special chapter head mechanism}
%
%    We divide chapter heads in two broad categories, the
%	standard chapter heads that utilize macros similar to
%	the standard classes and the \textit{special} chapter
%	heads that have their own typesetter commands.
%	For example we provide a special type of design for
%	this book called \textit{stewart}. The \cs{stewart}
%	is a template author defined command.
%	Any special design requires, two items. A macro defining
%	the design and setting the custom key to point to this macro.
%
%	 
% \begin{macro}{custom}
% \begin{macro}{\customdesign@cx} 
%	This key holds the name of a macro that is to be
%	trigerred for a custom designed template. 
% 
%    \begin{macrocode}
\cxset{custom/.code=\global\@specialtrue
                \gdef\customdesign@cx{%
                      \csname#1\endcsname},
          fill/.store in=\fill@cx}
%    \end{macrocode}
% \end{macro}
% \end{macro} 
%
% 
%
% 	This macro  typesets the chapter label i.e., |CHAPTER|. We
%	set the font parameters as defined by the key value system.
%	The label is defined first as |CHAPTER| by the standard
%	class and later on as |Appendix|. 
%	If we need small caps or spaceout we use the |\so| command
%	from the |soul| package.
%    \begin{macrocode}
 \newcommand\inshape[2][fill=sweet,white]{%
% \rightline{\fbox{#2R}}
%\leftline{\fbox{#2}}
  %
        \begin{tikzpicture} 
         \filldraw[gray]  (0,0) circle [radius=1.5pt];%
         \node at (0,0) [%rounded rectangle,
                      trim left, 
                      name=s,
                      %anchor=midway,
                       behind path,
                       circle,
                       drop shadow={opacity=0.5,fill=sweet}, %box shadow in css
                        black,
                       % double=sweet,
                        %text height=1.5ex,
                        %text depth=1ex,
                        %anchor=s.base,
                        draw,
                        outer ysep=0pt, %no outer so that lines can align nicely
                        inner ysep=0pt,
                        inner xsep=0pt,
                        line width=1pt,%#1
                         ]{#2};
             \end{tikzpicture}%
\ignorespaces}%
%             
\newif\if@debug \@debugfalse
\def\tikzi{%
    \tikz[remember picture,overlay] 
    \draw[<->] (0,0)--(0,1.5)--++(-.2,0) node[left,fill=blue!15,text=black]%
       {{\ttfamily\footnotesize\string\chaptermarginleft}};%\space%
}%
%
%
\global\newsavebox\chapternamebox
\global\newsavebox\numbernamebox
\global\newsavebox\bothboxes
\global\newsavebox\tempboxa@cx
\global\newsavebox\tempboxb@cx
\global\newsavebox\tempboxc@cx
%
\def\setchapterfont{%
 \expandafter\setfontparam@cx\chapterfontfamily@cx;%
         \expandafter\setfontparam@cx\chapterfontsize@cx;%
         \expandafter\setfontparam@cx\chapterfontweight@cx;%
         \expandafter\setfontparam@cx\chapterfontshape@cx;%
}
%    \end{macrocode}
%
% \begin{macro}{\drawmaybe}
%   Borders for elements are specified either individually per side, or in a short form specifying
%   the same width for all sides. When the short form is used it will set the width of all sides,
%   so we do not really need to check for the short form.
%   
%    \begin{macrocode}
\gdef\tempcmd@cx{}% 
\newcounter{draw}\setcounter{draw}{1}
\gdef\drawmaybe#1{%
    \setcounter{draw}{0}
    \expandafter\ifdim\csname#1bordertopwidth\endcsname>0pt \addtocounter{draw}{1}\fi
    \expandafter\ifdim\csname#1borderrightwidth\endcsname>0pt \addtocounter{draw}{1}\fi
    \expandafter\ifdim\csname#1borderbottomwidth\endcsname>0pt \addtocounter{draw}{1}\fi
    \expandafter\ifdim\csname#1borderleftwidth\endcsname>0pt\addtocounter{draw}{1}\fi
     \expandafter%
     \ifnum\thedraw>0
           \gdef\tempcmd@cx{draw}%
      \else
           \gdef\tempcmd@cx{}%
      \fi
}
       
%   
%
%    \end{macrocode}
% \end{macro}
%
%  Define some fancy border styles to abstract border styles
%  \#1  rulewidth
%    \begin{macrocode}
\newcommand\phd@rule[2][chapter]{%
          \color{\rulebottomcolor}%
           \edef\tempa{\csname#1borderbottomstyle\endcsname}
           \edef\tempb{double}
           \edef\tempc{solid}
           \edef\tempd{dashed}
           \edef\tempe{dotted}
           \edef\tempf{double dotted}
           \edef\tempg{asterisk}
            \edef\temph{double asterisk}
           % solid
           \ifx\tempa\tempc 
              \hrule\@height\fboxrulebottom
           \fi  
           % double
           \ifx\tempa\tempb% 
              \hrule\@height\fboxrulebottom%thickness of rule
                  \@width\dimexpr(\totalboxwidth+8pt)\relax%
              \vskip1pt%
              \hrule\@height\fboxrulebottom\relax
           \fi    
          % dotted
           \ifx\tempa\tempe\nointerlineskip% 
                \hbox to \totalboxwidth{\xleaders\hbox{.}\hfill\kern\z@}%a
           \fi  
           % double dotted
           \ifx\tempa\tempf\nointerlineskip% 
                \hbox to \totalboxwidth{\cleaders\hbox{.}\hfill\kern\z@}%
                \vskip1pt\nointerlineskip%
                 \hbox to \totalboxwidth{\cleaders\hbox{.}\hfill\kern\z@}%
           \fi  
           % asterisk
            \ifx\tempa\tempg\nointerlineskip% 
                \hbox to \totalboxwidth{\cleaders\hbox{*}\hfill\kern\z@}%
           \fi  
            % double asterisk
            \ifx\tempa\temph\nointerlineskip% 
                \hbox to \totalboxwidth{\cleaders\hbox{\textasteriskcentered}\hfill\kern\z@}%
                \vskip1pt\nointerlineskip%
                \hbox to \totalboxwidth{\cleaders\hbox{\textasteriskcentered}\hfill\kern\z@}%
           \fi
}           
%    \end{macrocode}
%
%
% \begin{macro}{\fboxrule}
% \begin{macro}{\fboxsep}
% \begin{macro}{\fboxseptop}
% \begin{macro}{\fboxsepright}
% \begin{macro}{\fboxsepbottom}
% \begin{macro}{\fboxsepleft}
% user level parameters,
%    \begin{macrocode}
\newdimen\fboxrule
\newdimen\fboxsep
\fboxrule.4pt
\fboxsep1pt
\newdimen\fboxseptop
\newdimen\fboxsepright
\newdimen\fboxsepbottom
\newdimen\fboxsepleft
\fboxseptop\fboxsep
\fboxsepright\fboxsep
\fboxsepbottom\fboxsep
\fboxsepleft\fboxsep
\newdimen\fboxruletop
   \fboxruletop\fboxrule
\newdimen\fboxruleright
   \fboxruleright\fboxrule
\newdimen\fboxrulebottom
   \fboxrulebottom\fboxrule 
\newdimen\fboxruleleft
   \fboxruleleft\fboxrule
%    \end{macrocode}
% \end{macro}
% \end{macro}
% \end{macro}
% \end{macro}
% \end{macro}
% \end{macro}
%
% \begin{macro}{\fbox}
%   Abbreviated framed box command.
%  The definition is from the kernel modified for variable padding and border
%  widths. 
% 
%    \begin{macrocode}
\def\ruletopcolor{black}
\def\rulebottomcolor{black}
\def\ruleleftcolor{black}
\def\rulerightcolor{black}
%
\long\def\phd@fbox#1{%
  \leavevmode
  \setbox\@tempboxa\hbox{%
    \color@begingroup%
      \kern\fboxsepleft{#1}\kern\fboxsepright%
    \color@endgroup}%
 \edef\totalboxwidth{\expandafter\the\wd\@tempboxa}   
  \phd@frameb@x\relax}
%    \end{macrocode}
% \end{macro}
%
% \begin{macro}{\framebox}
% Framed version of |\makebox|.
%    \begin{macrocode}
\def\phd@framebox{%
  \@ifnextchar(%)
    \phd@framepicbox{\@ifnextchar[\phd@framebox\phd@fbox}}
%    \end{macrocode}
% \end{macro}
%
% \begin{macro}{\@framebox}
% Deal with optional arguments.
%    \begin{macrocode}
\def\phd@framebox[#1]{%
  \@ifnextchar[%]
    {\phd@iframebox[#1]}%
    {\phd@iframebox[#1][c]}}
%    \end{macrocode}
% \end{macro}
%
% \begin{macro}{\@iframebox}
%    The handling the optional arguments.
%    In order to set the whole box, including the frame to the
%    specified dimension, we first determine that dimension
%    from the natural size of the text, |#3|.
%    calculated width.
%    \begin{macrocode}
\long\def\phd@iframebox[#1][#2]#3{%
  \leavevmode
  \@begin@tempboxa\hbox{#3}%
    \setlength\@tempdima{#1}%
    \setbox\@tempboxa\hb@xt@\@tempdima
         {\kern\fboxsep\csname bm@#2\endcsname\kern\fboxsep}%
    \@frameb@x{\kern-\fboxrule}%
  \@end@tempboxa}
%    \end{macrocode}
% \end{macro}
%
% \begin{macro}{\@frameb@x}
% Common part of |\framebox| and |\fbox|. |#1| is a negative kern
% in the |\framebox| case so that the vertical rules do not add to the
% width of the box.
%
%    \begin{macrocode}
\def\phd@frameb@x#1{%
  \@tempdima\fboxruletop
  \advance\@tempdima\fboxseptop
  \advance\@tempdima\dp\@tempboxa
  \hbox{%
    \lower\@tempdima\hbox{%
      \vbox{%
        \bgroup%
           \ifdim\fboxruletop>0pt 
              \phd@rule{\totalboxwidth}%
           \fi   
           %\color{\ruletopcolor}\hrule\@height\fboxruletop\egroup%toprule
            \egroup   
            \hbox{%
              \bgroup%
              \ifdim\fboxruleleft>0pt%
                  \color{\ruleleftcolor}%
                  \edef\tempa{\chapterborderleftstyle}%
                  \edef\tempc{solid}%
                  \edef\tempb{double}%
                  \ifx\tempa\tempb%
                    \vrule\@width\fboxruleleft%
                    \kern1pt\vrule\@width\fboxruleleft%
                  \fi
                  \ifx\tempa\tempc%
                    \vrule\@width\fboxruleleft%
                  \fi
             \fi   
              \egroup%\fboxrule%leftrule
              #1%
              \vbox{%
                 \vskip\fboxseptop%toppadding
                  \box\@tempboxa
                  \vskip\fboxsepbottom}%bottompadding
                  #1%
                  \bgroup%
                  \edef\tempa{\chapterborderrightstyle}%
                  \edef\tempb{double}
                  \edef\tempc{solid}
                  \color{\rulerightcolor}%
                  \ifx\tempa\tempb%
                      \vrule\@width\fboxruleright%
                       \kern1pt%
                       \vrule\@width\fboxruleright%
                  \fi%
                  \ifx\tempa\tempc%
                       \vrule\@width\fboxruleright%
                  \fi%
                  \egroup 
            }%%
           \bgroup
              \phd@rule{\totalboxwidth}%
           \egroup
           }%
           }%
        }%
}
%    \end{macrocode}
% \end{macro}
%
%   This is the main rendering routine for a generic block element. The element can either be
%    rendered in-line or as a block.
%
%    \#1  class of the element or id
%    \#2  the contents of the element e.g chapter or number.
%
%    Any element to be used here has to have a series of keys associated with it. They keep a naming
%    convention as for example,
%
%     |chapter border-left-width| \\
%     |chapter font-size|
%       
%    The prefix |chapter|  or |number|  or |title|  then enable to use the generic commands.
%    TeX is not an object orientated language, and future improvements are possible with LuaTeX.
%     
% \begin{macro}{\saveelementbox}
%    \begin{macrocode}
\def\chapterbordertopcolor{}%                                                                                                                   
\cxset{chapter border-top-color/.store in=\chapterbordertopcolor,
       chapter border-right-color/.store in=\chapterborderrightcolor,
       chapter border-bottom-color/.store in=\chapterborderbottomcolor,
       chapter border-left-color/.store in=\chapterborderleftcolor,}
 \cxset{%
          chapter border-top-color=sweet,
          chapter border-right-color=sweet,
          chapter border-bottom-color=sweet,
          chapter border-left-color=sweet,
           }%
%                            
\cxset{chapter border-color/.code=\pgfkeysalso{chapter border-top-color={#1},%
                                                      chapter border-right-color={#1},
                                                      chapter border-bottom-color={#1},
                                                      chapter border-left-color={#1}}}%
%                                                           
 % set some defaults                                                                
\cxset{%
          chapter border-top-color=sweet,
          chapter border-right-color=sweet,
          chapter border-bottom-color=sweet,
          chapter border-left-color=white,
         %chapter border-color=red%needs to be fixed
          }%
\def\saveelementbox#1#2#3{%
%    \end{macrocode}
% \end{macro}
%  
%  Before we save the box, we set all its properties so we can measure it
%  correctly. As this is a generalized routine all properties use the prefix \#2
%  i.e., \meta{chapter}paddingtop etc.
%
%    \begin{macrocode}
%          
   \expandafter\fboxseptop\csname#2paddingtop\endcsname
   \expandafter\fboxsepright\csname#2paddingright\endcsname
   \expandafter\fboxsepbottom\csname#2paddingbottom\endcsname
   \expandafter\fboxsepleft\csname#2paddingleft\endcsname
%
  \expandafter\fboxruletop\csname#2bordertopwidth\endcsname\relax
  \expandafter\fboxruleright\csname#2borderrightwidth\endcsname\relax
  \expandafter\fboxrulebottom\csname#2borderbottomwidth\endcsname\relax
  \expandafter\fboxruleleft\csname#2borderleftwidth\endcsname\relax
%
%
   \let\ruletopcolor\chapterbordertopcolor
   \let\rulerightcolor\chapterborderrightcolor       
   \let\rulebottomcolor\chapterborderbottomcolor
  \let\ruleleftcolor\chapterborderleftcolor
%
%
\cxset{number border-left-width=0.1pt,
          number padding-top=0pt,
          number border-bottom-width=0pt,
          chapter border-right-color=white,
          chapter border-top-color=white}%
%          
  \expandafter\savebox\csname#2namebox\endcsname{%
       \phd@fbox{#3}
      % \shadowbox{#3}%
       %\Ovalbox{#3}%
       %\doublebox{#3}
%      \tcbox[size=normal,
%                colframe=white, colback =white, borderline={2pt}{5pt}{black},
%               frame style={top color=white, bottom color=black, left color=black, right color=black},
%       %borderline west={2pt}{-2pt}{red},
%        %
%      % arc=5pt,outer arc=5pt, %!hyberbola
%      %outer arc=180pt,rounded corners=all,
%      tikz={rotate=0}]{#3}%
      %
%        \tcbox[colframe=thelightgray,arc=3pt,%!hyberbola arcs 200
%      outer arc=3.5pt,rounded corners=all,
%      tikz={rotate=30}]{#3}%
 %         \phd@fbox{#3}%
  }%
}
%
%
%  
  \newcommand\printchaptername[2][chapter]{%
%    \end{macrocode}
% 
%    \begin{macrocode}
   \saveelementbox{}{#1}{#2}%   
%   if there is a margin on top set it      
%   #0 is inline   #2 block 
%   This decides if the element and subsequent elements are to be floated left or right. If the first element
%    is to be floated right, then all subsequent elements are floated right.
%    If we are on the first element, we set glue at the beginning to float all subsequent elements to the
%    right, if centered we do the same. 
%          <0  first element rendering
%           0    float left no glue
%           1    center inline 
%           2    right  - glue only at first element
%           3   float left and break
%           4   center and break
%           5   center no break 
%    
%     
%\global\setcounter{chapterfloat}{2}
%\global\setcounter{numberfloat}{2}
%  The following is only executed  for the first element, giving a signal as to how the next elements are to be floated
%  The first element is a negative number and hence will only be activated once.
%   
\setcounter{currentelementfloat}{\csname c@#1float\endcsname}%
      \ifcase \@arabic\c@currentelementfloat                      
                 \expandafter\renderleftblock{#1}\or         %0
                 \expandafter\rendercenterblock{#1}\or         %1
                 \expandafter\renderrightblock{#1}  \or         %2
                 \expandafter\renderinline{#1} \or          %3
      \else
                \rendercenterblock{#1}%
      \fi
%  We now can deal with any material that has to be rendered outside the |element| block, possibly material
%  such as horizontal or vertical rules.
   \ifnum\@arabic\c@numberdisplay=0
      %\hrule 
      %\csname#1after@cx\endcsname% 
     \else
     \@@par       
  \fi    
     }
    
%    \end{macrocode}
%
%  We now ready to render the text. If a border width has been defined we need to use
%  the |draw| property of the node to show it. If not we do not draw it. However, we might
%  still need to fill it, if a background color has been specified.  
%
% \begin{macro}{\rendercenterblock}   
%  For block elements, i.e., elements that are allowed to float, we use a full line to float them. 
%    \begin{macrocode}
\def\rendercenterblock#1{%
       \appendtoheading{heading}{%
       \centerline{%
         \expandafter\unhcopy\csname#1namebox\endcsname
        }%
    }%
  }
%    \end{macrocode}  
% 
%    \begin{macrocode}
 \def\renderleftblock#1{%
   \appendtoheading{heading}{%
     % \leftline%
      }%
  }
  %
\def\renderrightblock#1{%
   \appendtoheading{heading}{%
     \rightline{%
         \expandafter\unhcopy\csname#1namebox\endcsname
      }% 
     }%   
}

\def\renderinline#1{%
  \appendtoheading{headingtoks}{%
     \leftline{%
         \expandafter\unhcopy\csname#1namebox\endcsname
      }%    
      }%  
    }

\def\renderboxcontents#1{%
        \drawmaybe{#1}%  
        \edef\tmp{\tempcmd@cx}
       \inshape[\expandafter\csname#1color@cx\endcsname,
                            fill=\expandafter\csname#1bgcolor\endcsname, 
                            ellipse, 
                            \expandafter\csname#1shape\endcsname, 
                            behind path,
                            line width=1pt,  %!fixme
                            \tmp,]{\expandafter\copy%
                                 \expandafter\csname#1namebox\endcsname}%
  }
%    \end{macrocode}
% \end{macro}
%
% \subsection{Author blocks}
% 
% \begin{macro}{printauthorblock} 
%	An author author block is  printed for some chapter 
%	designs such as those in multi-author books, hence we provide a macro to typeset it. 
%	
%    \begin{macrocode}
\def\printauthorblock{%
  \if@debug
    \tikz[remember picture,overlay] 
       \draw[<->] (0,0)--(0,0.5)--++(-.2,0)% 
              node[left,fill=blue!15,text=black]%
               {{\ttfamily\footnotesize author block=true}};%
  \fi       
  \authorblockformat@cx\authorblock@cx}%
%    \end{macrocode}
% \end{macro}  
%

%	We also provide a macro to typeset the number with appropriate
%	hooks for key value parameters.
%
%  \begin{docCommand}{setnumberfont}{}
%  sets the font for the number part of a chapter heading
% \end{docCommand}
%    \begin{macrocode}
\def\setnumberfont{%
    \expandafter\setfontparam@cx\numberfontsize@cx;%
    \expandafter\setfontparam@cx\numberfontfamily@cx;%
    \expandafter\setfontparam@cx\numberfontweight@cx;%
    \expandafter\setfontparam@cx\numberfontshape@cx;%
  }%
%    \end{macrocode}
% 
%    \begin{macrocode}
\newsavebox\numberbox
%    \end{macrocode}
% 
% \begin{macro}{\printnumber@cx}
%    \begin{macrocode}
% \begin{macro}{\printnumber@cx}
\def\printnumber@cx{%
     \bgroup%
       % \kern\numberpaddingleft\relax
       \@debugtrue
         \numberbefore@cx%
         \if@debug%
         \begin{tikzpicture}[remember picture,overlay]%
         \draw[<->] (0,0)--(-.2,1.0)--++(-.2,0) node[right, text=black]%
          {\footnotesize\string\numberbefore@cx};%
         \end{tikzpicture}%
         \fi%
         \if@chapterspaceout
           \if@soulspaceout
               \SetTracking
							 [ no ligatures = {f},
							 spacing = {600*,-100*, },
							 outer spacing = {450,250,150},
							 outer kerning = {*,*} ]
							 { encoding = * }
							 { 160 }
                \def\x{\thechapter}%
                \def\soxx{\textls\x\strutbox}%
                 %
          \fi       
          \else%
             \def\soxx{\thechapter}%
        \fi%
% We measure the width of the number box before we
% insert any padding and borders          
          \savebox\numberbox{%
                   \color{\numbercolor@cx}%
                   \setnumberfont%
                    \soxx}%
%                    
            \colorbox{\numberbgcolor}{\hbox to \dimexpr(\wd\numberbox%
                               +\numberborderleftwidth
                               +\numberborderrightwidth
                               +\numberpaddingright
                               +\numberpaddingleft){%
                 \vbox{%
                %toprule
                \hrule width\dimexpr(
                                 \wd\numberbox
                                +\numberpaddingright +\numberpaddingleft
                                +\numberborderleftwidth
                               +\numberborderrightwidth
                                 )
                  height\numberbordertopwidth\relax%
              % left rule takes care of padding  
               \vrule height\dimexpr(
                                   \ht\numberbox 
                                  + \numberpaddingtop)% 
                          width\numberborderleftwidth
                          depth\expandafter\dimexpr(
                                   +\numberpaddingbottom-\dp\numberbox)% 
              \kern\numberpaddingleft%
             % \copy\numberbox%\@chapapp
             \drawmaybe{number}%
              \inshape[\temp]{\copy\numberbox}%
              %right rule
              \kern\numberpaddingright%
              \def\sideborderrule{%
                    \vrule height\dimexpr(
                                   \ht\numberbox 
                                  +\numberpaddingtop)% 
                          width\numberborderrightwidth
                          depth\expandafter\dimexpr(-\dp\numberbox
                                   +\numberpaddingbottom)\relax}
               \sideborderrule\hskip1pt\sideborderrule 
               %                      
%               \def\bottomrule{% 
%                  \bgroup
                   \@tempdima\dimexpr(\wd\numberbox
                                    +\numberpaddingright +\numberpaddingleft
                                    +\numberborderleftwidth
                                    +\numberborderrightwidth)
                   \expandafter\@tempdimb\numberborderbottomwidth\relax                 
                  % \drawrule{\@tempdima}{\@tempdimb}%
                   %\egroup
               %}%
               %
               \ifnum\numberborderstyle@cx>0 %
                      \drawrule{\@tempdima}{\@tempdimb}%
                       \ifnum\numberborderstyle@cx>1 %
                           \drawrule{\@tempdima}{\@tempdimb}%
                           % \drawdoublerule{\@tempdima}{\@tempdimb}%
                       \fi
               \fi
             %
            %
          }}}%
          \numberafter@cx%
          %\aftergroup\offinterlineskip
   \egroup         
}% 
%    \end{macrocode}
% \end{macro}
%
% \begin{macro}{\printchaptertitle}
% \begin{macro}{\setchaptertitlefont}
%    \begin{macrocode}
\newif\if@runinhead \@runinheadfalse
\def\afteralignhook@cx{\par}
\def\chaptertitletextalign@cx{\Centering}
\newsavebox\titlebox
\newif\if@titleborderleft \@titleborderlefttrue
\newif\if@titleborderright \@titleborderrighttrue
%
\def\setchaptertitlefont{%
     \expandafter\setfontparam@cx\titlefontweight@cx;%
      \expandafter\setfontparam@cx\titlefontfamily@cx;%
      \expandafter\setfontparam@cx\titlefontshape@cx;%
      \expandafter\setfontparam@cx\titlefontsize@cx;%
      \color{\titlefontcolor@cx}%
}%
%    \end{macrocode}
%
% Provide inerface to expl3 dimensions
% and other variables 
%
%    \begin{macrocode}
\ExplSyntaxOn
  \dim_new:N \chaptertitleboxwidth
  \dim_set_eq:Nc \chaptertitleboxwidth {chapter_title_width}
\ExplSyntaxOff
%
%
\long\def\printchaptertitle#1{%
\parindent0pt
     %\vskip\titlebeforeskip@cx  %
     \if@lefttitle%
       \beforenumber@cx%
       \counterdisplay\c@chapter\afternumber@cx%
     \fi%
% If the title is letter spaced we define a macro to expand it.
      \if@titlespaceout%
         \long\def\SSS{{\so{#1}}}%    
      \else
         \long\def\SSS{#1}%
      \fi%
       \if@runinhead% for runin heads
            \def\afternumber@cx{\space\textbar\space}%
            \thechapter\afternumber@cx\space\SSS\par%
       \fi
%    \end{macrocode}      
%
% We save the contents in order to measure the height. Change to lrbox???
%
%    \begin{macrocode}
      \savebox{\titlebox}{%   
      \begin{minipage}[t]{\chaptertitleboxwidth}%
         \language-1% no hyphenation works with lua different with others
         \setchaptertitlefont
         \chaptertitletextalign@cx%
         \SSS%
         \par
       \end{minipage} 
         }%
         % attempt to emulate display modes
          \ifnum\titledisplay@cx=0\fi %do nothing
          \ifnum\titledisplay@cx=1
             \appendtoheading{heading}{\vskip0pt%
                       \titlemargintop@cx%
                       \leavevmode\noindent}%
             %\g@addto@macro{\titleafter@cx}{\par}
          \fi
          \ifnum\titledisplay@cx=2\fi
          \titlebefore@cx%
          %\titlemarginleft@cx%
 %         \chaptertitleblockalign@cx%
%    
%    \end{macrocode}
%
%   Having measured the title block, we now typeset it. Before we typeset it
%   we will provide borders all around if required and also allow for padding. 
%   We will not repeat the browser wars here, so we will provide the borders
%   outside the block and the padding inside.
%   The top border is typeset first
%    \begin{macrocode}
%
\appendtoheading{heading}{\mbox{%
     \fboxrule0pt\fbox{\hbox to \titlemarginleft{}}%
          \hbox to \chaptertitlewidth@cx{\vbox{%
               \color{\titlebordertopcolor@cx}%              
               \vskip0pt
                \nointerlineskip 
                 \vrule width
                        \expandafter\dimexpr(\wd\titlebox+\titleborderleftwidth 
                                                      +\titleborderrightwidth+\titlepaddingleft+\titlepaddingright) 
                         height\titlebordertopwidth
                 %
                 \vskip0pt\nointerlineskip
%   
%                 
                \color{\titleborderleftcolor@cx}%
                   \vrule width\titleborderleftwidth 
                             height\expandafter\dimexpr(\ht\titlebox
                             +\titlepaddingtop)\relax
                             depth\dimexpr(\titlepaddingbottom+\dp\titlebox)
                % we kern to emulate left padding      
                \kern\titlepaddingleft
%    \end{macrocode}
%  we typeset the text which is now in the minipage and we go on
%  and also add the right padding and border if required.
%  We copy instead of |\usebox| as we do not wanr to go into paragraph mode
%  Unsure about this item will need to revisit. Gives issues with some of the templates. 
%    \begin{macrocode}   
       \begin{minipage}[t]{\chaptertitlewidth@cx}%
           \setchaptertitlefont 
           \chaptertitletextalign@cx%
           \language-1
           \SSS\par
       \end{minipage}% \copy\titlebox% 
%    \end{macrocode}
%   finally we typeset the right border. 
%    \begin{macrocode}             
             \ifdim\titleborderrightwidth>0pt
                    \color{\titleborderrightcolor@cx}% 
                    \kern\titlepaddingright
                     \vrule width\titleborderrightwidth
                              height\expandafter\dimexpr(
                              \ht\titlebox+\titlepaddingtop)
                              depth\dimexpr(\titlepaddingbottom+\dp\titlebox)
              \fi
%    \end{macrocode}
%
%  next we skip to the bottom and draw a bottom rule
%  if it is specified. We are still in paragraph mode
%    \begin{macrocode}
%      
             \color{\titleborderbottomcolor@cx}%     
             \vskip0pt
             \nointerlineskip 
             \vrule width
                      \expandafter\dimexpr(
                       \wd\titlebox
                      +\titleborderleftwidth
                      +\titleborderrightwidth
                      +\titlepaddingleft + \titlepaddingright
                ) 
             height\titleborderbottomwidth\relax
             % hook for aligning normally par, unless there is more material  
%
%             
%       
              %
              %
%Hook before skips
      \if@righttitle%
         \afternumber@cx%
         \counterdisplay\c@chapter\afternumber@cx%
      \fi%
      %
      %\titleafterskip@cx 
%    \end{macrocode}
%  
% Finally we add the bottom margin to the title block and we are done.
%    \begin{macrocode}       
      \if@debug%
          %\par\leavevmode%
          \vbox to 0pt{%
              \offinterlineskip
               \vrule width.4pt depth0pt height% 
                \expandafter\the\titlemarginbottom\relax% 
                \kern2pt\raisebox{3pt}{\hbox to 0pt{\tiny\the\titlemarginbottom}}
            }%
     \fi%
%
 %     
}}}%\hbox
}%\afteralignhook@cx
%    \end{macrocode}
% We finally add the titleafter@cx hook. This takes care of any material added at the end of the
% text block. Also here we inject the end of paragraph marker and return to vertical mode. 
% 
%    \begin{macrocode}
     \titlemarginbottom@cx%   
     \titleafter@cx%introduces space if par???  
}%  end macro here
%    \end{macrocode}
% \end{macro}
% \end{macro}
%
%
% \begin{macro}{\@makechapterhead}
%	The macro calls the main typesetting activities of the chaper head
%	We begin our typesetting by checking, if the macro has a special
%	design which we then call.
%  
%    \begin{macrocode}
%    \newif\if@mainmatter \@mainmatterfalse CHECK THIS NOT NECESSARY
%
\newtoks\chapterprelimtoks
\newtoks\chaptertoks
\newtoks\numbertoks
\newtoks\titletoks
\newtoks\headingtoks \headingtoks={}
%
%    \end{macrocode}
%
%
%  \begin{macro}{\appendtoheading}
%
%  The command \cmd{\appendtoheading} is just a helper macro to add tokens to a
%  token register, defined as \meta{element name}|toks|. 
%
%    \begin{macrocode}
\long\def\appendtoheading #1#2{%
  \expandafter\expandafter\expandafter\csname#1toks\endcsname\expandafter{%
     \the\csname#1toks\endcsname #2}%
}
%    \end{macrocode}
% \end{macro}
%
% 
%    \begin{macrocode}
\renewcommand\@makechapterhead[2][]{%
  \if@special
       \customdesign@cx{#2}%
  \else
%    \end{macrocode}
%
% 	We now ready to typeset the chapter heading, we run everything 
% 	within a group and
% 	activate the chapter only if the |thesecumdepth>-1| and if only we are 
% 	within mainmatter. 
%
%    \begin{macrocode}
      \bgroup%
      \parindent0pt 
      %\offinterlineskip
      %
      \normalfont%
      \ifnum \c@secnumdepth>\m@ne%
          \if@mainmatter%
%    \end{macrocode}
%
% We first check if we need to print anything before the chapter head, as for
% example an image or a graphic, we then check if the number is to the left of
% the right of the image and typeset it. We follow it by printing any chapter
% 
%    \begin{macrocode}
         \lineskip0pt% 
          \topskip0pt%check this out
           %\chaptermargintop@cx%
            \leavevmode% 
           \if@runinhead%
           \else%
               \if@leftname%
                    \printnumber@cx%
               \fi%
%    \end{macrocode}
%
%  The typesetting of the chapter and number combination and as a matter of fact
%  any generalized string of element blocks depends if the value of the property |display|
%  is inline, inline block or block. A block can float freely, whereas the others restrict the
%  typesetting to a linear mode.
%  
%  We start with checking if the chapter name is to be displayed or not. If it is empty
%  we save execution speed, by skipping it.
%  
%  Both token registers have a common part and a unique part
% TO FIX THIS IS INCORRECT ALL FOR DIFFERENT SPACEOUT
%    \begin{macrocode}
          \appendtoheading{number}{\numberbefore@cx}%
          \appendtoheading{number}{\color{\numbercolor@cx}}%
           \appendtoheading{number}{\setnumberfont}%     
           \if@chapterspaceout
              \if@soulspaceout
                \expandafter\appendtoheading{number}{\so\thechapter}%
              \fi
           \else
              \expandafter\appendtoheading{number}{\thechapter}%   
           \fi   
           \appendtoheading{number}{\numberafter@cx}%
%    \end{macrocode}
%
%    If there is no chapter name to print we just typeset the number.
%
%    \begin{macrocode}  
           \xdef\xtemp{\chaptername}%
           \xdef\ytemp{}%         
           \ifx\xtemp\ytemp%
               \printchaptername[number]{\the\numbertoks}%               
           \else
%    \end{macrocode}   
%  
%  We collect all the commands in a token register. We start with the color and font
%  settings which we add in a group.
%             
%    \begin{macrocode}
             \appendtoheading{chapterprelim}{\leavevmode \chapterbefore@cx\par}%
             \appendtoheading{chapter}{%
               \leavevmode\noindent
               \color\chaptercolor@cx%
                \setchapterfont
              }%
%
             \@ifundefined{chapterbeforecontent@cx}{%
                  \def\chapterbeforecontent@cx{}}{}%
%                  
             \if@chapterspaceout
                \if@soulspaceout       
                  \expandafter
                    \appendtoheading{chapter}{\chapterbeforecontent@cx\so\chaptername}%
                \else
                   \expandafter
                    \appendtoheading{chapter}{\chapterbeforecontent@cx\chaptername}% 
                \fi                                
             \else
              \appendtoheading{chapter}{\chapterbeforecontent@cx\chaptername}%
            \fi   
%
%    \end{macrocode}
%
%  Next we need to add the tokens for decorating the number. We expect all headings to ne
%  numbered if the word `chapter’ is prefixed to the heading.  
%
%    \begin{macrocode} 
           \@ifundefined{numberaftercontent@cx}{\def\numberaftercontent@cx{}}{}%  
            \ifnum\thenumberdisplay=0 %                    
              \appendtoheading{chapter}{\kern0.5em}%
              \appendtoheading{chapter}{\the\numbertoks\numberaftercontent@cx}%
%    \end{macrocode}
%
%   We are done with inline headings and we can typeset them.
%    \begin{macrocode}
                 %   
              \the\chapterprelimtoks                 
              \printchaptername[chapter]{\the\chaptertoks}%
                 %
%    \end{macrocode}
%
%  If both the chapter name as well as the number are displayed as blocks
%  we typeset them in two operations.
%    \begin{macrocode}                  
             \else
                \the\chapterprelimtoks%
                \expandafter\printchaptername[chapter]{\the\chaptertoks}%
                \expandafter\printchaptername[chapter]{\the\numbertoks}%
              \fi  
            \fi
           \fi%
         \fi% mainmatter
      \fi%secnum
%    \end{macrocode}
%
% 	We typeset the title block and if there is an author block we print it.  
%	and print it.
%    \begin{macrocode}
\if@chaptertitlespecial%
       \csname ethics\endcsname{#2}%
\else%
       \printchaptertitle{#2}%
\fi%   
\the\headingtoks
\titlemarginbottom@cx
      \if@authorblock
           \printauthorblock
            \authorblockafterskip@cx
    \fi%
    \par\nobreak%
   %
    \egroup%
 \fi% We close the main conditional @special
  
}%
%    \end{macrocode}
% \end{macro}
%
%
% \begin{macro}{\chapter} The \cs{chapter} is modified to
% 	add hooks for openings and headers. 
% 	The |book| standard class states that a chapter should
% 	always start on a new page. In reality many book styles
%	allow the chapter heading to be continuous i.e., more
%	like a section. \label{code:chapterafterindent} 
%    \begin{macrocode}
\global\newif\if@chapterafterindent@cx \@chapterafterindent@cxfalse
%    \end{macrocode}
%
%  We set keys for |\afterindent| to enable it via the key value interface
%  
%    \begin{macrocode}
\cxset{chapter afterindent/.is choice,
           chapter afterindent/true/.code=\gdef\chapterafterindent@cx{%
                                                            \global\@chapterafterindent@cxtrue},
           chapter afterindent/false/.code=\gdef\chapterafterindent@cx{%
            \global\@chapterafterindent@cxfalse},
}           
% We set this to false by default
\cxset{chapter afterindent=false}
% call it after a heading
\def\chapterafterheading@cx{%
     \@nobreaktrue
     \everypar{%
     \if@nobreak
         \@nobreakfalse
         \clubpenalty \@M
          \if@chapterafterindent@cx \else
            {\setbox\z@\lastbox}%
          \fi
       \else
       \clubpenalty \@clubpenalty
       \everypar{}%
     \fi}}
%    \end{macrocode} 
% 
% 
%    \begin{macrocode}
\renewcommand\chapter{%
    \if@openright\cleardoublepage\fi
    \if@openleft\cleartoevenpage\fi
    \if@openany\clearpage\fi
%    \end{macrocode}
%  
%	Floats are prevented from floating at the top of chapter
%	opening pages as they look out of place.
% 	|\headerstyle@cx| defaults to empty.
%    Then we suppress the indentation of the first paragraph by
%    setting the switch |\@afterindent| to |false|. We use |\secdef|
%    to specify the macros to use for actually setting the chapter
%    title.
%
%    \begin{macrocode}
    \thispagestyle{empty}%
    \global\@topnum\z@%
%    \end{macrocode}
%
%  We provide a hook to handle indentation after a chapter. This would also necessitate to 
% change the afterheading macro and make it specific to a chapter head.
%    \begin{macrocode}  
      \@chapterafterindent@cxfalse%
   % \@afterindentfalse
%    \end{macrocode}
% \end{macro}
%
% 	Everything is now ready to call |secdef|, which is defined in the
% 	kernel. This command takes two arguments and calls the auxiliary
% 	macros for starred and unstarred commands. 
%
%    \begin{macrocode}
      \secdef\@chapter\@schapter}%[optional]{title} follows
%    \end{macrocode}
%
% We first define the unstarred version of the command.
% This is modified to include
% our hooks.
%
%
% \begin{macro}{\@chapter} 
%    This macro is called when we have a numbered chapter. When
%    |secnumdepth| is larger than $-1$ and, in the book
%    class, |\@mainmatter| is true, we display the chapter
%    number. We also inform the user that a new chapter is about to be
%    typeset by writing a message to the terminal. We hook
%	 here to add a number of typesetting key value macros.
%	
%    \#1 what to write in toc if has optional argument
%    \#2 title
% 
%    \begin{macrocode}
\newif\if@tocspecial\@tocfalse
\def\formattoctitle{}
% 
\def\@chapter[#1]#2{%
 %\refstepcounter{chapter}%
  \ifnum \c@secnumdepth >\m@ne%
    \if@mainmatter
      \if@toc% added extra if
        \refstepcounter{chapter}%
        \typeout{\@chapapp\space\thechapter.}%
%    \end{macrocode} 
%
%	We provide a hook for special design of toc layouts.
%	If set we call the specially defined custom layout otherwise
%	we default to the standard class layout.
%	We have a subtle change below to be able to catch
%	the title and number separately in a \cs{numberline} command.
%	Although the std classes have |{\protect\numberline{\thechapter}#1}|
%	we prefer |{\protect\numberline{\thechapter}{#1}}|. 
%   This way we can have a separate case for typesetting titles and numbers.
%   TODO CHECK WHY BOOKMARKS HAS AN ISSUE WITH THIS
%    \begin{macrocode}
        \def\tocchapternumber@cx{\@arabic\c@chapter}%
        \phantomsection
         \addcontentsline{toc}{chapter}{%
         \protect\chapternumberline{\tocchapternumber@cx}{#1}{\tocimage@cx}}%
  	\fi%
            %\fi%
    \else
      \addcontentsline{toc}{chapter}{#1}%for part???
    \fi%
   \else%
      \addcontentsline{toc}{chapter}{#1}%????? for part
   \fi%
%    \end{macrocode}
%
%	After having written the entry to the table of contents we
%	store the alternative title of this chapter with |\chaptermark|
%	and add some white space to the lists of figures and tables.
%    In one column mode we call |\@afterheading| which takes care 
%	of supressing the indentation after a chapter heading. 
%
%    \begin{macrocode}
  \chaptermark{#1}%
  \addtocontents{lof}{\protect\addvspace{10\p@}}%
  \addtocontents{lot}{\protect\addvspace{10\p@}}%
     \if@twocolumn
         \@topnewpage[\@makechapterhead{#2}]%
      \else%
         \@makechapterhead{#2}%
% This was afterindent, we redefined it to ensure we can make it more flexible.
%                 
                %\@afterheading
                \chapterafterheading@cx%
      \fi%
   }
%    \end{macrocode}
% \end{macro}
%
%    \begin{macrocode}
\cxset{toc image/.store in = \tocimage@cx}
\cxset{toc image ={}}
%
\gdef\setdefaults{%
\cxset{%
  chapter toc=true,
  toc image={},
  chapter name=CHAPTER,
  title font-family=\rmfamily,
  title font-weight=\bfseries,
  title font-size=\Huge,
  title font-color=purple,
  title margin bottom=20pt,
  numbering=arabic,
  number dot=,
  number before=,
  number after=,
  %chapter name
 chapter display=block,
 chapter float=left,
 chapter shape=ellipse,
 chapter color=black,
 chapter background-color=sweet,
 chapter font-size= Huge,
 chapter font-weight=bfseries,
 chapter font-family=itshape,
 chapter before=,
 chapter spaceout=none,
 chapter after=,
 chapter margin-left=0cm,
 chapter margin-top=0pt,
 %
 chapter border-width=2pt,
 chapter border-top-width=1pt,
 chapter border-right-width=1pt,
 chapter border-bottom-width=1pt,
 chapter border-left-width=4pt,
% 
 chapter padding-left=20pt,
 chapter padding-right=20pt,
 chapter padding-top=20pt,
 chapter padding-bottom=10pt,
% 
 number display=block,
 number float=right,
 number shape=starburst,
 numbering=Words,
 number spaceout=soul,
 number font-size=huge,
 number font-weight=bold,
 number font-family=rmfamily,
 number font-shape=itshape,
 number before=,
 number display=inline,
 number float=none,
% 
 number border-top-width=1pt,
 number border-right-width=0pt,
 number border-bottom-width=0pt,
 number border-left-width=0pt,
 number border-width=0pt,
%  
 number padding-left=0em,
 number padding-right=0.5em,
 number padding-top=0em,
 number padding-bottom=0pt,
  %number margin-top=, to do
 %number margin-left=0pt,  to create
 %
 number after=\par,
 number dot=,
 number position=rightname,
 number color=sweet,
 number background-color=white,
 title before=,
 title after=,
 title afterskip={\vskip24pt},
 title beforeskip=,
 title font=rmfamily,
 chapter title width=\columnwidth,
 header style=plain,
 section font-weight=bfseries,
 section font-family=sffamily,
 section font-size=Large,
 section font-shape=upshape,
 section align= left,
 section numbering suffix=,
 title font-shape=upshape,
 chapter title align=left,
 chapter title text-align=left,
 chapter title width=0.8\textwidth,
 title before=,
 title after=,
 title display=block,
 title beforeskip=1pt,
 title afterskip=1pt,
 author block=false,
 section font-family=rmfamily,
 section font-size=LARGE,
 section font-weight=bfseries,
 section indent=0pt,
  section font-weight=mdseries,
 section align=left,
% epigraph width=\dimexpr(\textwidth-2cm)\relax,
% epigraph align=center,
 epigraph text align=center,
 epigraph rule width=0pt,
 header style=plain
 % blank page text=,
}
}
\cxset{rule color/.store in={\rulecolor@cx},
          block color/.store in={\blockcolor@cx}}
\cxset{rule color=blue, block color=teal}
\cxset{style87/.style={
 chapter opening=any,
 chapter numbering=arabic,
 name=Chapter,
 % positioning and float - inline is 0
 %  float right is 2
 number display=block,
 number float=right,
 number shape=starburst,
 %numbering=Words,
 number spaceout=soul,
 number font-size=huge,
 number font-weight=bold,
 number font-family=rmfamily,
 number font-shape=normal,
 number before=,
 number display=inline,
 number float=none,
% 
 number border-top-width=0pt,
 number border-right-width=0pt,
 number border-bottom-width=0pt,
 number border-left-width=0pt,
 number border-width=0pt,
%  
 number padding-left=0em,
 number padding-right=0.5em,
 number padding-top=0em,
 number padding-bottom=0pt,
  %number margin-top=, to do
 %number margin-left=0pt,  to create
 %
 number after=\par,
 number dot=,
 number position=rightname,
 number color=sweet,
 number background-color=white,
 %chapter name
 chapter display=block,
 chapter float=left,
 chapter shape=ellipse,
 chapter color=black,
 chapter background-color=sweet,
 chapter font-size= Huge,
 chapter font-weight=bfseries,
 chapter font-family=itshape,
 chapter before=,
 chapter spaceout=none,
 chapter after=,
 chapter margin-left=0cm,
 chapter margin-top=0pt,
 %
 chapter border-width=2pt,
 chapter border-top-width=1pt,
 chapter border-right-width=1pt,
 chapter border-bottom-width=1pt,
 chapter border-left-width=4pt,
% 
 chapter padding-left=20pt,
 chapter padding-right=20pt,
 chapter padding-top=20pt,
 chapter padding-bottom=10pt,
  %chapter title
 title font-family=rmfamily,
 title font-color=black!80,
 title font-weight=bfseries,
 title font-size=huge,
 chapter title align=none,
 title margin-left=1cm,
 title margin bottom=1.3cm,
 title margin top=30pt,
 % title borders
 title border-width=0pt,
 title padding=0pt,
 title border-color=black!80,
% title border-top-color=spot!50,
% title border-top-width=20pt,
 title border-left-color=black!80,
 title border-left-width=2pt,
 title border-color=black!80,
 title padding-top=10pt,
 title padding-bottom=10pt,
 title padding-left=10pt,
 title padding-right=0pt,
% title border-right-color=spot!50,
% title border-right-width=20pt,
% title border-bottom-color=spot!50,
% title border-bottom-width=20pt,
 %
 chapter title align=left,
 chapter title text-align=left,
 chapter title width=0.8\textwidth,
 title before=,
 title after=,
 title display=block,
 title beforeskip=12pt,
 title afterskip=12pt,
 author block=false,
 section font-family=sffamily,
 section font-size=LARGE,
 section font-weight=bfseries,
 section indent=0pt,
  section font-weight=mdseries,
 section align=left,
 section font-family=tiresias,
       subsection font-family=tiresias,
       subsubsection font-family=tiresias,
       subsubsection indent=0pt,
       subsubsection font-size=large,
 %epigraph width=\dimexpr(\textwidth-2cm)\relax,
 %epigraph align=center,
 %epigraph text align=center,
 %epigraph rule width=0pt,
 header style=plain,
 chapter toc=true,
 }}
 
\cxset{section align=left}
\cxset{section font-weight=bold}
\cxset{section font-family=sffamily}
\cxset{subsection beforeskip=10pt} 
\cxset{subsection afterskip=10pt,
       subsection font-weight=\bfseries,
       subsection font-family=\sffamily,
       subsection font-size=\Large,
       subsection font-shape=\upshape,
       subsection align=flushleft,%don't say left
       subsection color=spot!50,
       subsection indent=0pt,
       %subsection afterindent=false,
       subsection numbering prefix=\thesection.,
       subsubsection indent=0pt,
       section font-family=tiresias,
       subsection font-family=tiresias,
       subsubsection font-family=tiresias,
       subsubsection indent=0pt,
       subsubsection font-size=large,
       subsection afterskip=10pt,
       subsection font-weight=\bfseries,
       subsection font-family=tiresias,
       subsection font-size=\Large,
       subsection font-shape=\upshape,
       subsection align=flushleft,%don't say left
       subsection color=spot!50,
       subsection indent=0pt,
       %subsection afterindent=false,
       subsection numbering prefix=\thesection.,
       }
\cxset{section align=left}
\cxset{section font-weight=bold}
\cxset{section font-family=sffamily}
\cxset{subsection beforeskip=10pt} 
\cxset{subsection afterskip=10pt,
       subsection font-weight=\bfseries,
       subsection font-family=tiresias,
       subsection font-size=\Large,
       subsection font-shape=\upshape,
       subsection align=flushleft,%don't say left
       subsection color=spot!50,
       subsection indent=0pt,
       %subsection afterindent=false,
       subsection numbering prefix=\thesection.,
       }

\cxset{style87}
\renewsection
\renewsubsection
%
%\section{List Management}
%
\cxset{enumerate numberingi/.is choice,
  enumerate numberingi/.code={\renewcommand\theenumi {\csname#1\endcsname{enumi}}},
  enumerate numberingii/.code={\renewcommand\theenumii {\csname#1\endcsname{enumii}}},
  enumerate numberingiii/.code={\renewcommand\theenumiii {\csname#1\endcsname{enumiii}}},
  enumerate numberingiv/.code={\renewcommand\theenumiv {\csname#1\endcsname{enumiv}}},
  enumerate labeli punctuation/.store in=\enumeratepunctuationi@cx,
  enumerate labeli/.is choice,
  enumerate labeli/brackets/.code={\renewcommand\labelenumi{(\theenumi\enumeratepunctuationi@cx)}},
  enumerate labeli/square brackets/.code={\renewcommand\labelenumi{[\theenumi\enumeratepunctuationi@cx]}},
  enumerate labeli/right bracket/.code={\renewcommand\labelenumi{\theenumi\enumeratepunctuationi@cx)}},
  enumerate label left/.store in=\enumeratelabelleft@cx,
  enumerate label right/.code=\renewcommand\labelenumi{\enumeratelabelleft@cx\theenumi\enumeratepunctuationi@cx#1},
  enumerate leftmargini/.code={\setlength\leftmargini{#1}},
  enumerate leftmarginii/.code={\setlength\leftmarginii{#1}},
  enumerate leftmarginiii/.code={\setlength\leftmarginiii{#1}},
  enumerate leftmarginiv/.code={\setlength\leftmarginiv{#1}},
  listi topsep/.store in=\listitopsep@cx,
  listi partopsep/.store in=\listipartopsep@cx,
  listi itemsep/.store in=\listiitemsep@cx,
  listi parsep/.store in=\listiparsep@cx,
  listii topsep/.store in=\listiitopsep@cx,
  listii partopsep/.store in=\listiipartopsep@cx,
  listii itemsep/.store in=\listiiitemsep@cx,
  listii parsep/.store in=\listiiparsep@cx,
  listiii topsep/.store in=\listiiitopsep@cx,
  listiii partopsep/.store in=\listiiipartopsep@cx,
  listiii itemsep/.store in=\listiiiitemsep@cx,
  listiii parsep/.store in=\listiiiparsep@cx,
}
\cxset{compact1/.style={%
  enumerate numberingi=arabic,
  enumerate numberingii=alph,
  enumerate numberingiii=alph,
  enumerate numberingiv=roman,
  enumerate labeli punctuation=.,
  enumerate label left=,
  enumerate label right=,
  enumerate leftmargini=2.2em,
  enumerate leftmarginii=2.1em,
  enumerate leftmarginiii=1.5em,
  enumerate leftmarginiv=2em,
  listi topsep=8\p@ \@plus2\p@ \@minus\p@,
  listi itemsep=0\p@ \@plus2\p@ \@minus\p@,
  listi parsep=0\p@ \@plus2\p@ \@minus\p@,
  listii topsep=0\p@ \@plus2\p@ \@minus\p@,
  listii itemsep=0\p@ \@plus2\p@ \@minus\p@,
  listii parsep=0\p@ \@plus2\p@ \@minus\p@,
  listiii topsep=0\p@ \@plus2\p@ \@minus\p@,
  listiii itemsep=0\p@ \@plus2\p@ \@minus\p@,
  listiii parsep=0\p@ \@plus2\p@ \@minus\p@,
}}
\cxset{compact2/.style={%
  enumerate numberingi=alph,
  enumerate numberingii=roman,
  enumerate numberingiii=alph,
  enumerate numberingiv=roman,
  enumerate labeli punctuation=,
  enumerate label left=(,
  enumerate label right=),
  enumerate leftmargini=2.2em,
  enumerate leftmarginii=2.1em,
  enumerate leftmarginiii=1.5em,
  enumerate leftmarginiv=2em,
  listi topsep   = 8\p@ \@plus2\p@ \@minus\p@,
  listi itemsep = 0\p@ \@plus2\p@ \@minus\p@,
  listi parsep   = 0\p@ \@plus2\p@ \@minus\p@,
  listii topsep  = 0\p@ \@plus2\p@ \@minus\p@,
  listii itemsep= 0\p@ \@plus2\p@ \@minus\p@,
  listii parsep  = 0\p@ \@plus2\p@ \@minus\p@,
  listiii topsep = 0\p@ \@plus2\p@ \@minus\p@,
  listiii itemsep= 0\p@ \@plus2\p@ \@minus\p@,
  listiii parsep  = 0\p@ \@plus2\p@ \@minus\p@,
}}

\ExplSyntaxOn
\def\setenumerate#1{
\cxset{#1}
\def\@listi{%
           \leftmargin\leftmargini
            \parsep\listiparsep@cx
            \topsep\listitopsep@cx\relax
            \itemsep\listiitemsep@cx}
            
\def\@listii{\leftmargin\leftmarginii
            \parsep\listiiparsep@cx
            \topsep\listiitopsep@cx\relax
            \itemsep\listiiitemsep@cx}
            
\def\@listiii{\leftmargin\leftmarginiii
            \parsep\listiiiparsep@cx
            \topsep\listiiitopsep@cx\relax
            \itemsep\listiiiitemsep@cx}
}


\setenumerate{compact1}
\ExplSyntaxOff

%\setdefaults
%    \end{macrocode}
%
% \section{Stewart Special Design}
%
%	We provide a number of predefined \textit{special designs} 
%    to illustrate the
%	technique. This special opening chapter page has been used in this document.
%	In order to typeset it we need a number of additional fields. We need an
%	image name and the text for two text blocks. We create fields for them.
%
%	\keyval{image}{\marg{filename}}{The filename for the 
%    image. You can use any
%	name that is acceptable to the command \texttt{includegraphics}.}
%    The special designs, require that you define a new set
%    of keys, if required and to create a command to typeset
%	these.
%
%    \begin{macrocode}
\cxset{steward/.style={
  custom = stewart,
  offsety/.store in=\soffsety,
  image/.store in=\image@cx,
  texti/.store in=\texti@cx,
  textii/.store in=\textii@cx,
}}
%
\newcommand\stewart[2][]{%
   \fancypagestyle{fancy}{%
   \lhead{}\rhead{}%
   \chead{}%
   \cfoot{}%
   \lfoot{}%
   \rfoot{\thepage}%
   \def\footrule#1{{\color{blue}%
   \hrule width\paperwidth}\vskip3pt
}

\renewcommand{\headrulewidth}{0pt}
\renewcommand{\footrulewidth}{0.4pt}}

\clearpage

\begin{tikzpicture}[remember picture,overlay]
\node [xshift=5cm,yshift=-\paperheight] at (current page.north west)
[text width=0.98\textwidth,text height=\paperheight, fill=thecream!30,rounded corners,above right]
{};
\node [xshift=6.5cm,yshift=-1.5cm-\soffsety] at (current page.north west)
[text width=0.9\textwidth,below right]{\sffamily \bfseries \huge #2};

\node [xshift=3cm,yshift=-1.5cm] at (current page.north west)
[text width=3cm,align=center,minimum height=2.5cm, fill=blue,below right]
{\[\text{\HHUGE\bfseries\sffamily\color{white}\thechapter}\]
\par\vspace*{3pt}
};

\node [xshift=-0.2cm,yshift=-21.5cm] at (current page.north west)
[text width=3cm,above right]%
{\includegraphics[width=1.0\paperwidth,height=\textheight,keepaspectratio]{\image@cx}};
\node [xshift=3cm,yshift=-19.5cm] at (current page.north west)
[text width=9cm,minimum height=2.5cm,inner sep=0.5em, fill=blue,below right]
{\color{white}
  \bfseries\sffamily\texti@cx
};
\node [xshift=6.5cm,yshift=-26cm] at (current page.north west)
[text width=12cm,above right]
{\textii@cx
};
\end{tikzpicture}
\par
\clearpage
}
%    \end{macrocode}
%
% \subsection{tikzspecials}
%    \begin{macrocode}
\cxset{band height/.store in=\bandheight@cx}
\cxset{band height=5cm}

\newcommand{\tikzspecials}[2][]{%
\@specialtrue
\clearpage

\begin{tikzpicture}[remember picture,overlay]
    \node[yshift=-\bandheight@cx] at (current page.north west)
      {\begin{tikzpicture}[remember picture, overlay]
        \draw[fill=\fill@cx, draw=none] (0,0) rectangle (\paperwidth,\bandheight@cx);
        \node[anchor=east,xshift=.9\paperwidth,rectangle,
              rounded corners=10pt,inner sep=11pt,
              fill=\fill@cx]{%
        \titlefontcolor@cx
        \titlefontsize@cx\bfseries
        \titlefontfamily@cx
        \thechapter\
        \textsc{#2}};
      \draw [fill=\fill@cx] (0,10cm) -- (5cm,10cm);
       \end{tikzpicture}
      };
\end{tikzpicture}
\mbox{}
\vspace*{\bandheight@cx}\par
}
%    \end{macrocode}
%
% \subsection{The genetics special design}
%
%    \begin{macrocode}
\cxset{image/.store in=\image@cx,
       image caption/.store in=\caption@cx,
       textiii/.store in=\textiii@cx}
%    \end{macrocode}
%
% \begin{macro}{\genetics}
%	This macro is a special template that requires settings via
%	a |\cxset| command. 
%    \begin{macrocode}
\newcommand\genetics[2][]{%
  %    \end{macrocode}
% \end{macro}
%
%	We set everything in a minipage to ensure that no breaks will
%	occur. If the user added too much text it will just overflow and it
%	will have to be adjusted.
%    \begin{macrocode}
\begin{minipage}[b][\textheight][t]{\textwidth}%
\hbox{}%
%    \end{macrocode}
%	We first draw the rules.
%    \begin{macrocode}
      \vbox to 0pt {%
      \color{teal}%
      \hbox{\rule{\textwidth}{0.4pt}}%
      \hbox{\rule{0.4pt}{\textheight}\rule{4cm}{0.4pt}}%
    }%
\vspace*{10pt}%
%    \end{macrocode}
%	The next two parboxes, place the subtitle and the image.
% 	they are aligned at the bottom and a rule can be used to adjust the 
%	subtitle.
%
%    \begin{macrocode} 
\begin{minipage}[b]{\linewidth}%
\fboxsep0pt%
\fboxrule0pt%
\fbox{\begin{minipage}[b]{0.25\linewidth}%
\lineskip0pt\topskip0pt%
\leftskip0.5cm%
\leavevmode%
\bfseries\color{teal}\Large\sffamily%
\caption@cx%
\vspace*{2cm}%

\includegraphics[width=\dimexpr\linewidth-0.5cm\relax,totalheight=3.8cm]{./images/chapterconcept-01.jpg}\llap{\raise20pt\hbox to \linewidth{\HHUGE \hskip1cm\color{lightgray!40}\thechapter}\hfill}%
\end{minipage}%
}%
%
\fbox{\begin{minipage}[b]{0.75\linewidth}%
\lineskip0pt%
\leavevmode
\includegraphics[width=1\linewidth]{\image@cx}%

\includegraphics[width=\linewidth]{./images/chapterconcept-02.jpg}.%
\end{minipage}}%
\end{minipage}%
%    \end{macrocode}
%    \begin{macrocode}
\par
\vspace{1.5cm  plus25pt minus25pt}
\parbox[t]{0.3\linewidth}{%
  \titlefontsize@cx
  \titlefontweight@cx
  \titlefontfamily@cx
  \leftskip0.5em 
  \color{teal}#2%
}%
\begin{minipage}[t]{0.6\linewidth}%
\vspace{-2\baselineskip}
\textiii@cx
\end{minipage}

\end{minipage}
}


%    \end{macrocode}


% \subsection{Epigraphs}
%
% This section deals with epigraphs.\index{epigraph}\index{epigraph!rule}
% The memoir class defines the epigraph command.
%    \begin{macrocode}
\@ifundefined{epigraph}{%
   \RequirePackage{epigraph}
   %% Set up the epigraph to be a bit wider
  \setlength{\epigraphwidth}{8cm} 
  \setlength{\epigraphrule}{0pt}
  \newcommand{\theepigraph}[2]{\epigraphhead[30]{\epigraph{#1}{\textit{#2}}}}
}{\setlength{\epigraphwidth}{8cm} 
\setlength{\epigraphrule}{0pt}
\newcommand{\theepigraph}[2]{\epigraphhead[30]{\epigraph{#1}{\textit{#2}}}}%
}
%    \end{macrocode}
%
%    \begin{macrocode}

\cxset{
  epigraph width/.code={\setlength\epigraphwidth{#1}},
  epigraph font-size/.code={\renewcommand{\epigraphsize}{#1}},
  epigraph beforeskip/.code={\setlength\beforeepigraphskip{#1}},
  epigraph afterskip/.code={\setlength\afterepigraphskip{#1}},
  epigraph align/.is choice,
  epigraph align/center/.code={\renewcommand{\epigraphflush}{center}},
  epigraph align/left/.code={\renewcommand{\epigraphflush}{flushleft}},
  epigraph align/right/.code={\renewcommand{\epigraphflush}{flushright}},
  epigraph source align/.is choice,
  epigraph source align/left/.code={\renewcommand{\sourceflush}{flushleft}},
  epigraph source align/right/.code={\renewcommand{\sourceflush}{flushright}},
  epigraph source align/center/.code={\renewcommand{\sourceflush}{center}},
  epigraph text align/.is choice,
  epigraph text align/left/.code={\renewcommand{\textflush}{flushleft}},
  epigraph text align/right/.code={\renewcommand{\textflush}{flushright}},
  epigraph text align/center/.code={\renewcommand{\textflush}{center}},
  epigraph rule width/.code={\setlength\epigraphrule{#1}},
  epigraph rule color/.store in = \epigraphrulecolor@cx,
  epigraph rule/.code={
 \renewcommand{\@epirule}{\color{\epigraphrulecolor@cx}\rule[.5ex]{\epigraphwidth}{\epigraphrule}}
},
}


\cxset{}

\cxset{epigraph width=0.5\linewidth,
    epigraph font-size=\small,
    epigraph rule width=0.4pt,
    epigraph align=right,
    epigraph source align=right,
    epigraph text align=right,
    epigraph rule color=black,
    epigraph rule}
\newif\if@headertoprule
\newif\if@headerbottomrule
\cxset{
   chaptermark name color/.store in=\chaptermarknamecolor@cx,
   chaptermark title color/.store in=\chaptermarktitlecolor@cx,
   chaptermark title before/.store in=\chaptermarktitlebefore@cx,
   chaptermark after number/.store in=\chaptermarkafternumber@cx,
   chaptermark name/.store in=\chaptermarkname@cx,
   chaptermark numbering/.is choice,
   leftmark before/.store in=\leftmarkbefore@cx,
   leftmark after/.store in=\leftmarkafter@cx,
   rightmark before/.store in=\rightmarkbefore@cx,
   rightmark after/.store in=\rightmarkafter@cx,
   chaptermark numbering/none/.code=\def\chaptermarknumber{},
   sectionmark name/.is choice,
   sectionmark name/none/.code=\def\sectionmarkname@cx{},
   sectionmark name custom/.code=\def\sectionmarkname@cx{#1},
   sectionmark number/.is choice,
   sectionmark number/none/.code=\def\sectionmarknumber@cx{},
   sectionmark after number/.store in=\sectionmarkafternumber@cx,
   sectionmark name color/.store in=\sectionmarkcolor@cx,
   sectionmark title font/.store in=\sectionmarktitlefont@cx,
   sectionmark title color/.store in=\sectionmarktitlecolor@cx,
   sectionmark before title/.store in=\sectionmarkbeforetitle@cx,
   sectionmark after title/.store in=\sectionmarkaftertitle@cx,
   header offset even/.store in=\headeroffseteven@cx,
   header offset odd/.store in=\headeroffsetodd@cx,
  %
   header top rule/.is if=@headertoprule,
   header bottom rule/.is if=@headerbottomrule,
}
%    \end{macrocode}
%
%    \begin{macrocode}
\cxset{header offset even=0pt,
       header offset odd=0pt,
       rightmark before=,
       rightmark after=,
       chaptermark title before=,}
%    \end{macrocode}
%
% Set the pagestyles to a default
%
%    \begin{macrocode}
\cxset{pagestyle/.code=\pagestyle{#1}}
\cxset{pagestyle=plain}
%
%
\cxset{headings ruled-01/.style={pagestyle=headings,
          header style=headings,
          chaptermark name color=theblue,
          chaptermark after number={\thinspace:\space },
          chaptermark name=,
          chaptermark title color=black!80,
          leftmark before=\thepage\hfill, 
          leftmark after=,
          sectionmark name color=theblue,
          sectionmark title color=black!80,
          header offset even=0pt,
          header offset odd=0pt,
          header top rule=false,
          header bottom rule=true}}
%
\cxset{headings ruled-02/.style={pagestyle=headings,
          header style=headings,
          chaptermark name color=theblue,
          chaptermark after number=,
          chaptermark name=,%\@chapapp,
          chaptermark numbering=none,
          chaptermark title color=black!80,
          sectionmark name=none,
          sectionmark number=none,
          leftmark before=,
          leftmark after=\qquad\quad\thepage,
          rightmark before=\thepage,
          rightmark after=\hfill\hfill,
          sectionmark name color=theblue,
          sectionmark title color=black!80,
          sectionmark after title=,
          sectionmark after number=\qquad,
          header top rule=false,
          header bottom rule=true,
          header offset even=1.5cm,
          header offset odd=-1.5cm,
          header bottom rule=false}}
%
\cxset{%
          header style=headings,
          chaptermark name=,
          chaptermark name color=black,
          chaptermark after number={\thinspace:\space },
          chaptermark title color=black!80,
          sectionmark name color=black,
          sectionmark title color=black!80,
          sectionmark after title=,
          sectionmark after number,
          header top rule=false,
          header bottom rule=true} 
          
\newif\ifAJW@multisty \AJW@multistyfalse
\newcommand\copyrightline[1]{%
  \def\@copyrightline{#1}}

\edef\@copyrightline{\relax}

\newcommand\c@pyrightline[1]{%
  \gdef\@c@pyrightline{#1}}

\gdef\@c@pyrightline{%
  \vbox to 5.5\p@{\noindent
  \parbox[t]{\textwidth}{\normalfont\footnotesize\baselineskip 9\p@
  \@copyrightline
  }%
  \vss}%
}

\def\ps@plain{\leftskip\z@\let\@mkboth\@gobbletwo\vfuzz=5\p@
    \def\@oddhead{}%
    \def\@evenhead{}%
  \def\@oddfoot{\verbatimsize
    \vbox{\vspace{15pt}%
      \global\hoffset=0pc%
      \noindent\hbox to\textwidth{\mbox{}\hfill{\rm\thepage}}
      \makebox[\z@][l]{\@c@pyrightline}%
%     \noindent\hspace*{-9pc}\rule{37pc}{0.25pt}%
    }%
  }%
  \def\@evenfoot{\verbatimsize
    \vbox{\vspace{15pt}%
    \global\hoffset=6pc%
    \noindent\hbox to\textwidth{{\rmfamily\thepage}\hfill\mbox{}}
    \makebox[\z@][l]{\@c@pyrightline}%
%   \noindent\rule{37pc}{0.25pt}%
    }%
  }%
  \def\sectionmark##1{}%
  \def\subsectionmark##1{}%
 }
 
 
% For style 22 need 
\def\ps@verticalrule{\leftskip\z@\let\@mkboth\@gobbletwo\vfuzz=5\p@
    \def\@oddhead{}%
    \def\@evenhead{}%
  \def\@oddfoot{\verbatimsize
    \vbox{\vspace{15pt}%
      \global\hoffset=0pc%
      \noindent\hbox to\textwidth{\hbox to 0pt{\rule{1pt}{\textheight}\color{blue}\thepage}}
      \makebox[\z@][l]{\@c@pyrightline}%
%     \noindent\hspace*{-9pc}\rule{37pc}{0.25pt}%
    }%
  }%
  \def\@evenfoot{\verbatimsize
    \vbox{\vspace{15pt}%
    \global\hoffset=6pc%
    \noindent\hbox to\textwidth{{\color{blue}\rm\thepage}\hfill\makebox[0pt][l]{\rule{1pt}{30pt}}}
    \makebox[\z@][l]{\@c@pyrightline}%
%   \noindent\rule{37pc}{0.25pt}%
    }%
  }%
  \def\sectionmark##1{}%
  \def\subsectionmark##1{}%
 }
\def\ps@headings{%
 \let\@mkboth=\markboth
 \def\@evenfoot{}
  \def\@oddfoot {}
  \def\@evenhead{\verbatimsize
    \vbox{\global\hoffset=6pc\noindent
    \makebox[\z@][l]{\rm \thepage}%
      \it \strut\hfill\leftmark\hbox{}%\par\vbox to 13pt{}%
%    \noindent\rule{37pc}{0.25pt}%
    }%
  }%
  \def\@oddhead{\verbatimsize
    \vbox{\global\hoffset=0pc\noindent
    \mbox{}\it \strut\rightmark\hfill\hbox{}\makebox[\z@][r]{\rm
      \thepage}%\par\vbox to 13pt{}%
%    \noindent\hspace*{-9pc}\rule{37pc}{0.25pt}%
    }%
  }%
  \def\chaptermark##1{\markboth{##1}{##1}}
  \def\sectionmark##1{\markright{\ifnum \c@secnumdepth >\z@
    \thesection\enskip\fi ##1}}%
  \ifAJW@multisty
    \def\chaptermark##1{}
    \def\sectionmark##1{}
  \else
    \def\chaptermark##1{\markboth{##1}{##1}}
    \def\sectionmark##1{\markright{\ifnum \c@secnumdepth >\z@
      \thesection\hspace{0.5em}\fi ##1}}%
  \fi
}
%
% centered headings
\def\ps@centerheadings{%
 \let\@mkboth=\markboth
 \def\@evenfoot{}
  \def\@oddfoot {}
  \def\@evenhead{\verbatimsize
    \vbox{\global\hoffset=6pc\noindent
    \makebox[\z@][l]{\rm \thepage}%
      \it \strut\hfill\leftmark\hfill\hbox{}%\par\vbox to 13pt{}%
%    \noindent\rule{37pc}{0.25pt}%
    }%
  }%
  \def\@oddhead{\verbatimsize
    \vbox{\global\hoffset=0pc\noindent\hfill
    \mbox{}\it \strut\rightmark\hfill\hbox{}\makebox[\z@][r]{\rm
      \thepage}%\par\vbox to 13pt{}%
%    \noindent\hspace*{-9pc}\rule{37pc}{0.25pt}%
    }%
  }%
  \def\chaptermark##1{\markboth{##1}{##1}}
  \def\sectionmark##1{\markright{\ifnum \c@secnumdepth >\z@
    \thesection\enskip\fi ##1}}%
  \ifAJW@multisty
    \def\chaptermark##1{}
    \def\sectionmark##1{}
  \else
    \def\chaptermark##1{\markboth{##1}{##1}}
    \def\sectionmark##1{\markright{\ifnum \c@secnumdepth >\z@
      \thesection\hspace{0.5em}\fi ##1}}%
  \fi
}



\def\ps@chapterstyle{%
    \let\@oddfoot\@empty\let\@evenfoot\@empty
    \def\@evenhead{\thepage\hfil\slshape\leftmark}%
    \def\@oddhead{{\slshape\rightmark}\hfil\thepage}%
    \let\@mkboth\@gobbletwo
    \let\chaptermark\@gobble
    \let\sectionmark\@gobble}
%    \end{macrocode}
%
%
% Definition of 'myheadings' page style
%
%    \begin{macrocode}
\def\ps@myheadings{%
  \let\@mkboth=\@gobbletwo
  \def\@evenfoot{}
  \def\@oddfoot {}
  \def\@evenhead{\verbatimsize
    \vbox{\global\hoffset=6pc\noindent
    \makebox[\z@][l]{\rm \thepage}%
      \it \strut\hfill\leftmark\hbox{}%\par\vbox to 13pt{}%
%    \noindent\rule{37pc}{0.25pt}%
    }%
  }%
  \def\@oddhead{\verbatimsize
    \vbox{\global\hoffset=0pc\noindent
    \mbox{}\it \strut\rightmark\hfill\hbox{}\makebox[\z@][r]{\rm
      \thepage}%\par\vbox to 13pt{}%
%    \noindent\hspace*{-9pc}\rule{37pc}{0.25pt}%
    }%
  }%
%
  \def\chaptermark##1{}
  \def\sectionmark##1{}
  \def\subsectionmark##1{}}
%
% \section{Watermarks}

% Some special styles
%    \begin{macrocode}
\IfFileExists{changepage.sty}{\RequirePackage{changepage}}{}
\IfFileExists{rotating.sty}{\RequirePackage{rotating}}{}
%    \end{macrocode}
%
% \begin{macro}{\even@samplepage}
% \begin{macro}{\odd@samplepage}
%    \begin{macrocode}
\def\even@samplepage{%
 \begin{picture}(0,0)
   \put(\Xeven,\Yeven){\turnbox{90}{\Huge \textcolor{\watermark@textcolor}{\watermark@text}}}
\end{picture}
}
%% Define a macro to print SAMPLE PAGE IN THE MARGIN
\def\odd@samplepage{%
 \begin{picture}(0,0)
   \put(\Xodd,\Yodd){\turnbox{90}{\Huge \textcolor{\watermark@textcolor}{\watermark@text}}}
 \end{picture}
}
%    \end{macrocode}
% \end{macro}
% \end{macro}

% \begin{macro}{watermarktext}
%  Define the watermark words
%    \begin{macrocode}
\gdef\watermarktext#1{\gdef\watermark@text{\fontfamily{phv}\selectfont#1}}
\def\watermarktextcolor#1{\gdef\watermark@textcolor{#1}}
\watermarktext{SAMPLE PAGE}
\watermarktextcolor{black!50}
%    \end{macrocode}
% \end{macro}
%    \begin{macrocode}
\def\ps@samplepage{\let\@mkboth\@gobbletwo
 \let\@oddhead\odd@samplepage\def\@oddfoot{\reset@font\hfil\thepage}
 \let\@evenhead\even@samplepage\def\@evenfoot{\reset@font\thepage\hfil}}
%
% We define two macros to position the watermark on the page
\def\Xodd{480}
\def\Xeven{-15}\def\Yeven{-810}
\def\Yeven{-\expandafter\strip@pt\textheight}
\let\Yodd\Yeven


\cxset{blank page text/.store in=\blankpagetext@cx{#1}}
\cxset{blank page text={}}

\def\cleardoublepage{\clearpage\if@twoside\ifodd\c@page\else
  \hbox{}
  \vspace*{\fill}
  \begin{center}
     \blankpagetext@cx      
  \end{center}
  \vspace{\fill}
  \thispagestyle{empty}
  \newpage
  \if@twocolumn\hbox{}\newpage\fi\fi\fi}
%    \end{macrocode}
%
% \section{Front Matter Commands}
% 
%	We define author commands for coverpages and
%	second pages. If they have been defined by the
%	author we do nothing, otherwise we provide some
%	defaults as examples.
% \begin{macro}{\coverpage}
%    \begin{macrocode}
\@ifundefined{coverpage}{%
  \newcommand\coverpage[3]{%
  \vspace*{2cm}
  \vbox{%
      \vspace*{-8\baselineskip} %-1
      \hskip-3.8cm\includegraphics[width=\paperwidth]{hine02.jpg}\par %hine-02
      \vspace*{1\baselineskip} %3
      \hbox to \hsize{%
         \Huge \hfill\hfill{\MakeUppercase{\bfseries  
         \textsf{WATER HAMMER}}}}%
      \vspace*{0.3cm}
      \hbox to \hsize{\Huge \hfill\hfill{\MakeUppercase{\bfseries   \textsf{ARRESTORS}}}}
      \vspace*{2\baselineskip}
      \hbox to \hsize{\huge \hfill\hfill\textsf{\hbox{#2}}}
      \vspace*{1.35cm}
      \hbox to \hsize{\huge \hfill\hfill\textsf{\hbox{#3}}}
}
}}{}
%    \end{macrocode}
% \end{macro}
% 
% 	\begin{macro}{\secondpage}
%	This macro typesets what a copyright page. It is not a general
%	command, but rather a command that you will need to redefine. It is included
%	here as an example and to typeset the second page of this publication.
%	\begin{macrocode}
\newcommand\secondpage{\clearpage\null\vfill\vfill
  \begin{minipage}[b]{0.9\textwidth}
  \includegraphics[width=3cm]{./images/zha.jpg}\par
  \raggedright
  \textit{Cover image: }
    The cover image shows Jo Bodeon, a back-roper in the mule room at 
    Chace Cotton Mill. Burlington, Vermont. This and other similar images 
    in this book were taken by Lewis W. Hine, in the period between 
    1908-1912. These images as well as social campaigns by many including 
    Hine, helped to formulate America's anti-child labour laws.
  \end{minipage}\par
  \vspace*{\baselineskip}
  \begin{minipage}[b]{0.9\textwidth}
  \RaggedRight
  \setlength{\parskip}{0.5\baselineskip}
  Copyright \copyright 2012  Dr Yiannis Lazarides\par
  Permission is granted to copy, distribute and\slash or modify this document 
  under the terms of the GNU Free Documentation License, version 1.2, with no 
  invariant sections, no front-cover texts, and no back-cover texts.\par
  A copy of the license is included in the appendix.\par
  This document is distributed in the hope that it will be useful, but without 
  any warranty; without even the implied warranty of merchantability or 
  fitness for a particular purpose.
  \end{minipage}
  \vspace*{2\baselineskip}
  \clearpage
}
%    \end{macrocode}
% \end{macro}
%
% \chapter{Table of Contents}
%
%	Most of the macros here re-write the LaTeX macros in a way that 
%	we can add appropriate hooks for styling. In writing this section
%	we had inspiration and used liberally code from Peter Wilson's 
%	\pkg{tocloft}., including the code for the image.
%
% \newcommand{\maxx}{120}       ^^A picture width
% \newcommand{\maxxm}{118}      ^^A \maxx - 2\
% \newcommand{\maxy}{55}        ^^A picture height
% \newcommand{\maxym}{53}       ^^A \maxy - 2
% \newcommand{\findent}{20}     ^^A indent
% \newcommand{\findentp}{22}    ^^A \findent + 2
% \newcommand{\fnumwidth}{10}   ^^A numwidth
% \newcommand{\ftocrmarg}{30}   ^^A \@tocrmarg
% \newcommand{\fpnumwidth}{20}  ^^A \@pnumwidth
% \newcommand{\fipn}{30}        ^^A \findent + \fnumwidth
% \newcommand{\frmarg}{90}      ^^A \maxx - \ftocrmarg
% \newcommand{\frnum}{100}      ^^A \maxx - \fpnumwidth
% \newcommand{\fyi}{10}         ^^A 1st y height
% \newcommand{\fyim}{8}         ^^A \fyi - 2
% \newcommand{\fyii}{20}        ^^A 2nd y height
% \newcommand{\fyiii}{25}       ^^A 3rd y height
% \newcommand{\fyiv}{30}        ^^A 4th y height
% \newcommand{\fyv}{40}         ^^A 5th y height
% \newcommand{\fyvp}{42}        ^^A \fyv + 2
% \newcommand{\flin}{4}         ^^A length of leader lines
% \newcommand{\frmargm}{89}     ^^A \frmarg (90) - a little bit
% 
% \providecommand{\bs}{\textbackslash}
% \begin{figure}
% \centering
% \setlength{\unitlength}{1mm}
% \begin{picture}(\maxx,\maxy)
%     ^^A side lines and linewidth
%   \put(0,0){\line(0,1){\maxy}}
%   \put(\maxx,0){\line(0,1){\maxy}}
%   \put(0,\maxy){\vector(1,0){\maxx}}
%   \put(2,\maxym){\makebox(0,0)[tl]{\texttt{\bs linewidth}}}
%     ^^A \@pnumwidth
%   \put(\maxx,\fyi){\vector(-1,0){\fpnumwidth}}
%   \put(\maxxm,\fyim){\makebox(0,0)[tr]{\texttt{\bs @pnumwidth}}}
%   \put(\frnum,\fyi){\line(0,1){\flin}}
%     ^^A \@tocrmarg
%   \put(\maxx,\fyv){\vector(-1,0){\ftocrmarg}}
%   \put(\maxxm,\fyvp){\makebox(0,0)[br]{\texttt{\bs @tocrmarg}}}
%   \put(\frmarg,\fyv){\line(0,-1){\flin}}
%     ^^A indent
%   \put(0,\fyv){\vector(1,0){\findent}}
%   \put(2,\fyvp){\makebox(0,0)[bl]{\textit{indent}}}
%   \put(\findent,\fyv){\line(0,-1){\flin}}
%     ^^A numwidth
%   \put(\findent,\fyv){\vector(1,0){\fnumwidth}}
%   \put(\findentp,\fyvp){\makebox(0,0)[bl]{\textit{numwidth}}}
%   \put(\fipn,\fyv){\line(0,-1){\flin}}
%     ^^A last title line
%   \put(\maxx,\fyii){\makebox(0,0)[br]{487}}
%   \put(\fipn,\fyii){title end}
%     ^^A second title line
%   \put(\fipn,\fyiii){continue\ldots}
%   \put(\frmarg,\fyiii){\makebox(0,0)[br]{\ldots title}}
%     ^^A first title line
%   \put(\findent,\fyiv){\textbf{3.5}}
%   \put(\fipn,\fyiv){Heading\ldots}
%   \put(\frmarg,\fyiv){\makebox(0,0)[br]{\ldots title}}
%     ^^A dotted leader
%   \multiput(\frmargm,\fyii)(-\flin,0){12}{.}
%   \multiput(\frmarg,\fyi)(-\flin,0){2}{\line(0,1){\flin}}
%   \put(\frmarg,\fyi){\vector(-1,0){\flin}}
%   \put(\frmarg,\fyi){\vector(1,0){0}}
%   \put(\frmarg,\fyim){\makebox(0,0)[tr]{\texttt{\bs @dotsep}}}
% 
% \end{picture}
% \setlength{\unitlength}{1pt}
% \caption{Layout of a ToC (LoF, LoT) entry} \label{fig:ltoc}
% \end{figure}
%
% \begin{command}{\quit@cx}
% \begin{macro}{\if@haschapter@cx}
% We will be using either chapter or section type headings for the ToC, etc.,
% so we need to know which of these the document class supports.
%    \begin{macrocode}
\newcommand{\quit@cx}{}
\newif\if@haschapter@cx\@haschapter@cxtrue
%    \end{macrocode}
% \end{macro}
% \end{command}
% \begin{macro}{\if@koma@cx}
% The \pkg{koma} classes have different defaults than the standard classes,
% so we need to know if a \pkg{koma} class has been loaded.
%    \begin{macrocode}
\newif\if@koma@cx  \@koma@cxfalse
\@ifclassloaded{scrartcl}{\@koma@cxtrue}{}
\@ifclassloaded{scrreprt}{\@koma@cxtrue}{}
\@ifclassloaded{scrbook}{\@koma@cxtrue}{}
%    \end{macrocode}
% \end{macro}
%
% \begin{macro}{\if@memoir@cx}
%    \begin{macrocode}
\newif\if@memoir@cx  \@memoir@cxfalse
\@ifclassloaded{memoir}{\@memoir@cxtrue}{}
%    \end{macrocode}
% \end{macro}
%
% Issue a warning if there are no recognised sectional divisions 
% and then skip the rest of the package code.
%    \begin{macrocode}
\@ifundefined{chapter}{%
  \@haschapter@cxfalse
  \@ifundefined{section}{%
    \PackageWarning{phd}%
      {I don't recognize any sectional divisions so I'll do very little and many things can break}
    \renewcommand{\quit@cx}{\endinput}
    }{\PackageInfo{phd}{The document has section divisions}}
  }{\@haschapter@cxtrue
    \PackageInfo{phd}{The document has chapter divisions}}
%    \end{macrocode}
% bailing out or continue.
%    \begin{macrocode}
\quit@cx
%    \end{macrocode}
%
% \begin{macro}{\settocpagestyle}
% \begin{macro}{\tocpagestyle@cx}
%	We define a user macro and to be used in keys
%   a pagestyle for the first page of the ToC.
%   The default is the |plain| pagestyle. CHECK THIS.
%    \begin{macrocode}
\newcommand{\settocpagestyle}[1]{%
  \def\tocpagestyle@cx{\thispagestyle{#1}}}
  \thispagestyle{plain}
%    \end{macrocode}
% \end{macro}
% \end{macro}
%
% \begin{macro}{\tocparskip@cx}
% The |\parskip| local to the ToC, etc., is set to the length |\tocparskip@cx|.
%
%    \begin{macrocode}
\newlength{\tocparskip@cx}
\setlength{\tocparskip@cx}{0pt}
%    \end{macrocode}
% \end{macro}
%
% 
% \begin{macro}{\tableofcontents}
%
%  This is a parameterised version of the default |\tableofcontents| command.
%  Each class has its own definition, but we have to cater for all classes
%  in one definition, hence some of the checks. The definition is
%  modified after all packages have been loaded. The normal LaTeX way is to use
% the chapter to set it in the book class and the section in others. Here we opted to
% leave it up to the user.
%	Consider more checks here
%
% \begin{macro}{tocstart@cx}
% \begin{macro}{tocfinish@cx}
%    \begin{macrocode}
\def\tocstart@cx{}
\def\tocfinish@cx{}
 \renewcommand{\tableofcontents}{%
    \tocstart@cx
%    \end{macrocode}
% \end{macro}
% \end{macro}
%
%     Ensure that any previous paragraph has been finished. 
%	  within a group set
%     the local paragraphing style and typeset the title. \label{code:tableofcontents}
%    \begin{macrocode}
    \par
    \begingroup
      \parindent\z@ \parskip\tocparskip@cx
      \maketoctitle@cx
%    \end{macrocode}
%
% Finally, read the \docfile{.toc} file and finish up.
%    \begin{macrocode}
      \@starttoc{toc}%
    \endgroup
    \tocfinish@cx
}%
%    \end{macrocode}
%
% \end{macro}
%
%    \begin{macrocode}
\newif\if@lowercase
\@lowercasetrue
\cxset{toc name/.code = \def\contentsname{#1},
          toc name before/.store in = \contentsnamebefore@cx,
          toc name after/.store in = \contentsnameafter@cx,
          toc name font-size/.store in = \contentsnamefontsize@cx,
          toc name font-weight/.store in = \contentsnamefontweight@cx,
          toc name font-family/.store in = \contentsnamefontfamily@cx,
          toc name font-shape/.store in = \contentsnamefontshape@cx,
          toc name color/.store in = \tocnamecolor@cx,
          toc name afterskip/.store in=\tocnameafterskip@cx,
          toc name align/.is choice,
          toc name align/center/.code=\def\startalign@cx{\bgroup\centering}\def\endalign@cx{\par\egroup},
          toc name align/right/.code=\def\startalign@cx{\flushright}\def\endalign@cx{\endflushright},
          toc name align/left/.code=\def\startalign@cx{\@empty}\def\endalign@cx{\@empty},
          toc name align/none/.code=\def\startalign@cx{\@empty}\def\endalign@cx{\@empty},
          toc name indent/.store in=\tocnameindent@cx,
          toc name case/.is choice,
          toc name case/lower/.code=\def\tocnamecase@cx{\@lowercasetrue
                             \if@lowercase\expandafter\MakeTextLowercase\fi},
          toc name case/upper/.code=\def\tocnamecase@cx{\@lowercasefalse
                             \if@lowercase\else\expandafter\MakeTextUppercase \fi},
          toc name case/none/.code=\def\tocnamecase@cx{\@empty},
     %    \end{macrocode}      
%
% The contents page is enabled to have its own pagestyle. We default this later on
% to plain.
% This needs also a bit of a thought, if we require to enable it further down the line.
%
%    \begin{macrocode}
      toc pagestyle/.code=\gdef\contentspagestyle@cx{\thispagestyle{#1}},
    }
%
\cxset{toc name= CONTENTS,
       toc name before = ,
       toc name after =, 
       toc name color = sweet,
       toc name font-weight=bold,
       toc name font-family=sffamily,
       toc name font-shape=upshape,
       toc name font-size=LARGE,
       toc name afterskip=10pt, %set as 40pt in LaTeX
       toc name after=\par,
       toc name align=right,
       toc name indent=\hspace*{4cm},
       toc name case=upper,
       toc pagestyle=plain,
  }%
%    \end{macrocode}
%
% \begin{macro}{\maketoctitle@cx}
%	\cs{maketitle@cx} is the typeset the heading that goes on top of the |ToC| page.
%	We cater for a few hooks, so the code is rather longish. At this point we can also 
%    divert to any custom design. TODO DEFAULT TO PARAMS FROM SECTIONING
%
%    \begin{macrocode}
\newcommand{\maketoctitle@cx}{%
  \addpenalty\@secpenalty
  \if@haschapter@cx
    \vspace*{10pt}
    \pdfbookmark[0]{\contentsname}{toc}%EXPERIMENTAL
  \else
    \vspace{10pt}
  \fi
  \markboth{\contentsname}{\contentsname}%
  \contentspagestyle@cx
  {\interlinepenalty\@M
  {\contentsnamebefore@cx
     \setfont@cx{\contentsnamefontweight@cx}%
    {\contentsnamefontfamily@cx}{\contentsnamefontsize@cx}%
    {\contentsnamefontshape@cx}%
     \color{\tocnamecolor@cx}%
    \startalign@cx%
         \tocnameindent@cx \tocnamecase@cx%
         \contentsname
     \endalign@cx}%
     \contentsnameafter@cx%
    \par\nobreak
  \vskip\tocnameafterskip@cx\relax
  \@afterheading}%
 }%
 \let\sampletoctitle\maketoctitle@cx
%    \end{macrocode}
% \end{macro}
%
% \begin{macro}{\setpnumwidth@cx}
% \begin{macro}{\setocmarg@cx}
%  Users commands for setting |\@pnumwidth| and |\@tocrmarg|.
%    \begin{macrocode}
\newcommand{\setpnumwidth@cx}[1]{\renewcommand{\@pnumwidth}{#1}}
\newcommand{\settocmarg@cx}[1]{\renewcommand{\@tocrmarg}{#1}}
\setpnumwidth@cx{25pt}
\settocmarg@cx{20pt}
%    \end{macrocode}
% \end{macro}
% \end{macro}
%
% \section{Styling the dot leaders}
%  	Here we will allow the user to either have dotfills and
%    be	able to specify the type and spacing of the dots.
%	We also provide a key to disable dotfills.
%
% \begin{macro}{\dot@cx}
% \begin{macro}{\dotfill@cx}
%   In the default |ToC|, a dotted line can be used to provide a leader between
%   a title and the page number. As Peter Wilson wrote and I found at my
%   distress the definition of the leader is buried
%   in the \cs{@dottedtocline} command. The 
%	\cs{dotfill@cx}\marg{sep}
%   command provides a parameterised version of the leader code, where
%   \marg{sep} is the seperation between the dots in mu units.
%   The symbol used for the `dots' in the leader is given by the 
%   value  of |\dot@cx|. 
% 
%    \begin{macrocode}
\newcommand{\dot@cx}{.}
\newcommand{\dotfill@cx}[1]{%
  \leaders\hbox{$\m@th\mkern #1 mu\hbox{\dot@cx}\mkern #1 mu$}\hfill}
%    \end{macrocode}
% \end{macro}
% \end{macro}
%
%    \begin{macrocode}
\def\nodotfill@cx{}
\cxset{toc dotfill/.is choice,
       toc dotfill/none/.code = \nodotfill@cx,
       toc dotfill symbol/.code= \renewcommand{\dot@cx}{#1},
       toc dotfill sep/.store in=\dotfillsep@cx,
}
\cxset{toc dotfill symbol=.,
       toc dotfill sep=4.5}
%    \end{macrocode}
%
% \begin{macro}{\parfillskip@CX}
% The |\l@kind| commands modify (locally) the value of |\parfillskip|.
% |\parfillskip@CX| is a copy of the default \texbook\ 
% |\parfillskip| definition.
%    \begin{macrocode}
\newcommand{\parfillskip@CX}{\parfillskip=0pt plus1fil}
%    \end{macrocode}
% \end{macro}
%
% \begin{macro}{\numberline}
% The purpose of the |\numberline{|\meta{secnum}|}| command is to typeset
% \meta{secnum} left justified in a box of width |\@tempdima|. I redefine
% it to add three additional parameters, namely |\tocnumberbefore@cx|, 
% |\@cftasnum| and |\@cftasnumb| 
% (see \docfile{ltsect.dtx} for the original
% definition).
%    \begin{macrocode}
\renewcommand{\numberline}[1]{% 
   \hb@xt@\@tempdima{\tocnumberbefore@cx #1\@cftasnum\hfil}\@cftasnumb}
%    \end{macrocode}
% \end{macro}
%
% \begin{macro}{\tocnumberbefore@cx}
% \begin{macro}{\@cftasnum}
% \begin{macro}{\@cftasnumb}
%
% Originally these were not defined but were |\let| to appropriate commands
% in the |\l@...| commands, but they
% have to be defined in case something unexpected 
% calls |\numberline|,
% for example through use of the \Lpack{float} package.\footnote{This bug WAS NOTED IN TOCLOF
% was discovered by Andrew Thurber when using the \Lpack{tocloft} and
% \Lpack{algorithm} packages together.}
%
%    \begin{macrocode}
\newcommand{\tocnumberbefore@cx}{[}
\newcommand{\@cftasnum}{}
\newcommand{\@cftasnumb}{}
%    \end{macrocode}
% \end{macro}
% \end{macro}
% \end{macro}
%
%
% \section{Styling Part in the Toc}

% \begin{macro}{\l@part}
% \begin{macro}{\if@dopart}
%  |\l@part{|\meta{title}|}{|\meta{page}|}| typesets the ToC entry for
% a |part| heading. It is a parameterised copy of the default |\l@part|
% (see \docfile{classes.dtx} for the original definition and the code
%  below for |\l@part| for an explanation of most of this
%  code). 
%
% By default, Parts
% (and Chapters) do not have dotted leaders. This package provides
% for all entries to have the ability to have dotted leaders, as some styles treat the part in a similar manner.
%
% In article class, Part level is 0 not -1 and hence the conditional below.
%	
%	We start by defining a number of keys and macros to store parameters.
%	An entry to the ToC consists always of a section number, the title and 
%	a page number. For each part there are different styling keys.
%
%	\begin{macro}{\tocpartindent@cx}	 
%    \begin{macrocode}
\cxset{toc part indent/.store in = \tocpartindent@cx,
       toc part numwidth/.store in = \tocpartnumwidth@cx,
       toc part font-size/.store in =\tocpartfontsize@cx,
       toc part before number/.store in = \tocpartbeforenumber@cx,
       toc part after number/.store in=\partafterpnum@cx,
       toc part beforeskip/.store in = \tocpartbeforeskip@cx,
       toc part page font-size/.store in=\tocpartpagefontsize@cx
}



\cxset{toc part beforeskip = 2.25em \@plus\p@,
       toc part indent=-50pt,
       toc part numwidth=0em,
       toc part after number=:,
       toc part font-size={\color{teal}\large\bfseries\sffamily},
       toc part before number={\kern1.5pt},
       toc part page font-size=\bfseries
       }
  %    \end{macrocode}
% \end{macro}
%
% We need first to define conditionals to switch from
% printing the part or not.
%
%    \begin{macrocode}
\newif\if@dopart@cx
\newif\if@haspart@cx
  \@ifundefined{part}{\@haspart@cxfalse}{\@haspart@cxtrue}
\if@haspart@cx
%    \end{macrocode}
%
% We now renew the command, in order to allow for hooks. 
% This might be cloberred by hyperref if too many changes
% are carried out. It takes two parameters (one for the caption and another for the title if different).
% 
%    \begin{macrocode}
\renewcommand*{\l@part}[2]{%
  \@dopart@cxfalse
  \ifnum \c@tocdepth >-2\relax
    \if@haschapter@cx
      \@dopart@cxtrue
    \fi
    \ifnum \c@tocdepth >\m@ne
      \if@haschapter@cx\else
        \@dopart@cxtrue
      \fi
    \fi
  \fi
%    \end{macrocode}
%
% The code needs to distinguish between a chapter or a section
% level and adds the appropriate penalties.
% 
%    \begin{macrocode}
%
  \if@dopart@cx
    \if@haschapter@cx
      \addpenalty{-\@highpenalty}%
    \else
      \addpenalty\@secpenalty
    \fi
%    \end{macrocode}
%
%	We add vertical spacing before the section if
%	required.
%    \begin{macrocode}
    \addvspace{\tocpartbeforeskip@cx}%
    \begingroup
      {\leftskip \tocpartindent@cx\relax
       \rightskip \@tocrmarg % need to check this for conflics\@tocrmarg
       \parfillskip -\rightskip
       \parindent \tocpartindent@cx\relax
       \@afterindenttrue
       \interlinepenalty\@M
       \leavevmode    
       \@tempdima \tocpartnumwidth@cx\relax
%       \let\tocnumberbefore@cx \cftpartpresnum
%       \let\@cftasnum \cftpartaftersnum
%       \let\@cftasnumb \cftpartaftersnumb
       \advance\leftskip \@tempdima \null\nobreak\hskip -\leftskip
       %
       {\tocpartfontsize@cx 
       \tocpartbeforenumber@cx #1}%
       \partfillnum@cx{#2}}%
      \nobreak
      \if@haschapter@cx
        \global\@nobreaktrue
        \everypar{\global\@nobreakfalse\everypar{}}%
 	   \fi
    \endgroup
  \fi}
\fi
%    \end{macrocode}
% \end{macro}
% \end{macro}
%
% \begin{macro}{\beforepartskip@cx}
% \begin{macro}{\partnumwidth@cx}
% \begin{macro}{\cftpartfont}
% \begin{macro}{\partpresnum@cx}
% \begin{macro}{\partaftersnum@cx}
% \begin{macro}{\partaftersnumb@cx}
% \begin{macro}{\partleader@cx}
% \begin{macro}{\partdotsep@cx}
% \begin{macro}{\partpagefont@cx}
% \begin{macro}{\partafterpnum@cx}
% \begin{macro}{\partindent@cx}
% \begin{macro}{\partfillnum@cx}
%  These are the user commands to control the typesetting of Part entries.
%  They are initialised to give the standard appearance.
%    \begin{macrocode}
\if@haspart@cx
%  \newlength{\beforepartskip@cx}
%    \setlength{\beforepartskip@cx}{2.25em \@plus\p@}
%  \newlength{\partnumwidth@cx}
%    \setlength{\partnumwidth@cx}{0em}
%  \newcommand{\cftpartfont}{\large\bfseries}
  \newcommand{\partpresnum@cx}{}
  \newcommand{\partaftersnum@cx}{..}
  % defined in parameters \newcommand{\partaftersnumb@cx}{}
%
% 
  \newcommand{\partleader@cx}{\large\bfseries\dotfill@cx{\partdotsep@cx}}
%
  \def\cftnodots{10000}
  \newcommand{\partdotsep@cx}{\cftnodots}
  \newcommand{\cftpartpagefont}{\large\bfseries}
  %\newcommand{\partafterpnum@cx}{}
  \newlength{\partindent@cx}
  \setlength{\partindent@cx}{0em}
  \newcommand{\partfillnum@cx}[1]{%
    {\partleader@cx}%
    {\hb@xt@\@pnumwidth{\hss {%
       \tocpartpagefontsize@cx #1}}}\partafterpnum@cx\par}%
%    \end{macrocode}
% \Lpack{koma} classes use some different settings.
%   \begin{macrocode}
  \if@koma@cx
    %\setlength{\partnumwidth@cx}{2em}
    %\renewcommand{\cftpartfont}{\sectfont\large}
    %\renewcommand{\cftpartpagefont}{\sectfont\large}
  \fi
\fi

%    \end{macrocode}
% \end{macro}
% \end{macro}
% \end{macro}
% \end{macro}
% \end{macro}
% \end{macro}
% \end{macro}
% \end{macro}
% \end{macro}
% \end{macro}
% \end{macro}
% \end{macro}
%


% \section{Handling of chapters in ToC.}
%
% \begin{macro}{\beforetocchapterskip@cx}
% \begin{macro}{\cftchapindent}
% \begin{macro}{\cftchapnumwidth}
% \begin{macro}{\cftchapfont}
% \begin{macro}{\cftchappresnum}
% \begin{macro}{\cftchapaftersnum}
% \begin{macro}{\cftchapaftersnumb}
% \begin{macro}{\cftchapleader}
% \begin{macro}{\cftchapdotsep}
% \begin{macro}{\cftchappagefont}
% \begin{macro}{\cftchapafterpnum}
% \begin{macro}{\cftchapfillnum}
%  These are the user commands to control the typesetting of Chapter entries.
%  They are initialised to give the standard appearance.
%    \begin{macrocode}
\@debugfalse
\if@debug
      \fboxsep1pt
      \fboxrule-1pt
\else
      \fboxsep0pt
      \fboxrule0pt
\fi
\if@haschapter@cx
   \newlength{\beforetocchapterskip@cx}
   \setlength{\beforetocchapterskip@cx}{1.0em \@plus\p@}
  \newlength{\cftchapindent}
  \setlength{\cftchapindent}{0em}
  \newlength{\cftchapnumwidth}\setlength{\cftchapnumwidth}{1.5em}
  \newcommand{\cftchapfont}{\bfseries}
  \newcommand{\cftchappresnum}{}
  \newcommand{\cftchapaftersnum}{}
  \newcommand{\cftchapaftersnumb}{}
  \newcommand{\cftchapleader}{\bfseries\dotfill@cx{\cftchapdotsep}}
%    \end{macrocode}
%
%	The following code determines the spacing of the dots.
%    \begin{macrocode}
  \newcommand{\cftchapdotsep}{\chapterdotsep@cx} 
  \newcommand{\cftchappagefont}{\sffamily\bfseries\color{teal}}
  \newcommand{\cftchapafterpnum}{}
%
 %    \end{macrocode}
%    \begin{macrocode}
%
% \pkgname{koma} classes have different chapter settings.
%    \begin{macrocode}
%  \if@cftkoma
%    \renewcommand{\cftchapfont}{\sectfont}
%  \fi
\fi

%    \end{macrocode}
% \end{macro}
% \end{macro}
% \end{macro}
% \end{macro}
% \end{macro}
% \end{macro}
% \end{macro}
% \end{macro}
% \end{macro}
% \end{macro}
% \end{macro}
% \end{macro}
%
% \subsection{l@chapter}
%
% \begin{macro}{\l@chapter}
%  \cs{l@chapter}\marg{title}\marg{page} typesets the ToC entry for
% a |chapter| heading. It is a parameterised copy of the default |\l@chapter|
%  (see \docfile{classes.dtx} for the original definition). This only applies
%  to chaptered documents.
% \begin{macro}{\chapternumberline}
%  \#1  number
% \#2  title
% \#3 images <-- consider removing or add as hook
%    \begin{macrocode}
\newcommand{\chapternumberline}[3]{% 
\if@debug
   \fboxsep1pt
   \fboxrule-1pt
\else
   \fboxsep0pt
   \fboxrule0pt
\fi
 \hrule width\textwidth height1pt depth0pt\relax 
  \hbox to 0pt{%
    \hspace*{0cm}%
    \vbox to 0pt{\vspace*{.9cm}%
       \parbox[t]{2cm}{%
%         \ifx#3\empty\else
%            \includegraphics[width=1.5cm]{#3}%
%         \fi
         }}}%
 % Typesets the Chapter  number   
 % and chapter name     
  \hspace*{0cm}%
   \fbox{\parbox[t]{2.5cm}{%
      \fbox{\chaptername
      \kern0.5em #1}%
  }}%
  \@@par  
  %   Typeset the chapter title width
\fbox{\parbox[t]{\tocchaptertitlewidth@cx}{%
      \bgroup
         \raggedright%
         \language-1
         \MakeTextUppercase{#2}%
        \par
        \egroup
       }}%
  }%
% 
% TODO ADD FONT DIRECTIVES FOR PAGE 
 \newcommand{\typesettocchapterpage@cx}[1]{%
 \if@debug
   \fboxsep1pt
   \fboxrule-1pt
\else
   \fboxsep0pt
   \fboxrule0pt
\fi
      \hfill\fbox{\parbox[b]{3em}{%
       \hfill%
       #1}}%
      \par\addvspace{0.5\baselineskip}
     %  \hrule \@width\textwidth \@height1pt \@depth0pt\relax 
      }% 
%    \end{macrocode}  
% \end{macro}
%    \begin{macrocode}
\cxset{toc chapter beforeskip/.store in=\tocchapterbeforeskip@cx,
          toc chapter afterskip/.store in=\tocafterchapterskip@cx,
          toc chapter indent/.store in = \tocchapterindent@cx,
          toc chapter dotsep/.store in = \chapterdotsep@cx,
          toc chapter no dots/.code=\def\chapterdotsep@cx{10000},
          toc chapter numberwidth/.store in = \tocchapternumberwidth@cx,
          toc chapter font/.store in=\tocchapterfont@cx,
          toc chapter title width/.store in=\tocchaptertitlewidth@cx}
%
%

\cxset{toc chapter beforeskip =10pt,
          toc chapter indent= 0pt,
          toc chapter dotsep=4.5,
          toc chapter no dots,
          toc  chapter numberwidth=0pt,
          toc chapter font= \bfseries\sffamily\large\color{sweet},
          toc chapter title width=0.8\textwidth}
%
% The l@chapter takes two parameters and is used to typeset
% the c
\if@haschapter@cx
  \renewcommand*{\l@chapter}[2]{%
     \ifnum \c@tocdepth >\m@ne
       \addpenalty{-\@highpenalty}%
       \vskip \tocchapterbeforeskip@cx\relax 
        {%\leftskip\tocchapterindent@cx\relax
        %\rightskip \@tocrmarg
        %\parfillskip -\rightskip
         \parindent \tocchapterindent@cx\relax%
         \@afterindenttrue
        %\interlinepenalty\@M
        \leavevmode
        %\@tempdima \tocchapternumberwidth@cx\relax
        %\let\tocnumberbefore@cx \cftchappresnum
        %\let\@cftasnum \cftchapaftersnum
        %\let\@cftasnumb \cftchapaftersnumb
        %\advance\leftskip \@tempdima \null\nobreak\hskip -\leftskip
        {\tocchapterfont@cx#1}\nobreak
         \typesettocchapterpage@cx{#2}}%
    \fi}%
\fi
%    \end{macrocode}
% \end{macro}
% \begin{macro}{\sampletocchapter}
% We define a macro for mocking sample toc chapters for the documentation
% 
%    \begin{macrocode}
\let\sampletocchapter\l@chapter
%    \end{macrocode}
% \end{macro}
%
% \section{ToC section styling}
%
%  |\l@section{|\meta{title}|}{|\meta{page}|}| typesets the ToC entry for
% a |section| heading. It is a parameterised copy of the default |\l@section|
% (see \docfile{classes.dtx} for the original definition). 
% 	We start by defining all our parameters and variables.
%
%    \begin{macrocode}
\renewcommand{\numberline}[1]{%
    \hb@xt@\@tempdima{#1\hfil}} %#1
 % 
%  
\cxset{toc section beforeskip/.store in=\tocsectionbeforeskip@cx,
          toc section beforeskip/.default={0pt plus.2pt},
          toc section indent/.store in=\tocsectionindent@cx,
%      fonts for title &num
       toc section font-size/.store in=\tocsectionfontsize@cx, 
       toc section font-family/.store in=\tocsectionfontfamily@cx, 
       toc section font-shape/.store in=\tocsectionfontshape@cx, 
       toc section font-weight/.store in=\tocsectionfontweight@cx, 
       toc section color/.store in=\tocsectioncolor@cx,
       toc section numwidth/.store in=\tocsectionnumwidth@cx,
%	  fonts etc for page number
       toc section page font-size/.store in=\tocsectionpagefontsize@cx,
       toc section page font-family/.store in=\tocsectionpagefontfamily@cx,
       toc section page font-shape/.store in=\tocsectionpagefontshape@cx,
       toc section page font-weight/.store in=\tocsectionpagefonteight@cx,
       toc section page color/.store in=\tocsectionpagecolor@cx,
%      leaders
	   toc section dotsep/.store in = \tocsecdotsep@cx,
%      before and after page number
       toc section page before/.store in=\tocsectionpagebefore@cx,
       toc section page after/.store in=\tocsectionpageafter@cx,
}
%
\cxset{%
       toc section beforeskip=\z@ \@plus.2\p@,
       toc section beforeskip,
       toc section indent=0em,
       toc section font-family= sffamily,
       toc section font-weight = mdseries,
       toc section font-shape = upshape,
       toc section color= sweet,
       toc section font-size=,
       toc section numwidth = 4.2em,
       toc section page font-size=,
       toc section page font-shape= upshape,  
       toc section page font-weight=, 
       toc section page font-family= sffamily,
       toc section page color = sweet, 
       toc section page before =,% \{,
       toc section page after =,% \},
       toc section dotsep = 2.7,
}
%    \end{macrocode}
%
%  
% \begin{macro}{\tocsectionpagefont@cx}
%    For convenience we define font setting commands for
%    the page number. We use \cs{setfont@cx}, which we have
%	defined earlier.
%  
%    \begin{macrocode}
\newcommand\tocsectionpagefont@cx{%
	\setfont@cx{\tocsectionpagefonteight@cx}%
       {\tocsectionpagefontfamily@cx}{\tocsectionpagefontsize@cx}%
       {\tocsectionpagefontshape@cx}\color{\tocsectionpagecolor@cx}
}%
%    \end{macrocode}
% \end{macro}
%
% \begin{macro}{\l@section} 
% 	This macro is called when the \cs{tableofcontents}
%	is read from the |.toc| file and it typesets
%	the title and the page number. Note this is also
% 
%    \#1 section title
%    \#2 page number
%      
%    \begin{macrocode}
\renewcommand*{\l@section}[2]{%
  \ifnum \c@tocdepth >\z@
    \if@haschapter@cx
      \vskip \tocsectionbeforeskip@cx
    \else
      \addpenalty \@secpenalty
      \addvspace{\tocsectionbeforeskip@cx}%
    \fi
    {\leftskip \tocsectionindent@cx\relax
     \rightskip \@tocrmarg
     \parfillskip -\rightskip
     \parindent \tocsectionindent@cx\relax\@afterindenttrue
     \interlinepenalty\@M
     \leavevmode
     \@tempdima \tocsectionnumwidth@cx\relax
     \let\tocnumberbefore@cx \cftsecpresnum
     \let\@cftasnum \cftsecaftersnum
     \let\@cftasnumb \cftsecaftersnumb
     \advance\leftskip \@tempdima \null\nobreak\hskip -\leftskip
%    \end{macrocode}
%
%	We are now ready to print out the toc section title,
%	we set the font information then typeset the title.
%	The dot leaders are typeset by calling the 
%	macro \cs{sectionfillnum}
%
%    \begin{macrocode}
     {%
      \setfont@cx{\tocsectionfontweight@cx}%
        {\tocsectionfontfamily@cx}{\tocsectionfontsize@cx}%
        {\tocsectionfontshape@cx}%
        \color{\tocsectioncolor@cx}%
      #1}\nobreak
%    \end{macrocode}
%	We are now ready to typeset the leaders and the page number. 
%    We then pass \#2 to the \cs{tocsectionfillnum} which 
%	does the typesetting.
%    \begin{macrocode}
      \tocsectionfillnum@cx{#2}}%
  \fi}
%    \end{macrocode}
% \end{macro}
%
%    These are the user commands to control the typesetting 
%	 of Section entries.
%    They are initialised to give the standard appearance.
%	 These are hooks to \cs{numberline}.
%    \begin{macrocode}
\newcommand{\cftsecpresnum}{}
\newcommand{\cftsecaftersnum}{}
\newcommand{\cftsecaftersnumb}{}
\newcommand{\tocsectionleader@cx}  {\normalfont\dotfill@cx{\tocsecdotsep@cx}}
%\newcommand{\cftsecdotsep}{\cftdotsep}
%    \end{macrocode}
%    We can now define the command \cmd{\tocsectionfillnum@cx}. 
%    will print the 
%	leaders if any and the page number \#1. 
%    \begin{macrocode}
\newcommand{\tocsectionfillnum@cx}[1]{%
  {\tocsectionleader@cx}\nobreak
  \hb@xt@\@pnumwidth{\hfil\tocsectionpagefont@cx
   \tocsectionpagebefore@cx #1}%
   \tocsectionpageafter@cx\par}%
%    \end{macrocode}
%
%
% \section{Toc subsection styling}
%
% \begin{macro}{\l@subsection}
%  |\l@subsection{|\meta{title}|}{|\meta{page}|}| typesets the ToC entry for
% a |section| heading. It is a parameterised copy of the default |\l@section|
% (see \docfile{classes.dtx} for the original definition). 
% 	We start by defining all our parameters and variables.
%
%    \begin{macrocode}
\newif\if@lowercasesubsection
\cxset{toc subsection beforeskip/.store in=\tocsubsectionbeforeskip@cx,
       toc subsection indent/.store in=\tocsubsectionindent@cx,
%      fonts for title &num
       toc subsection font-size/.store in=\tocsubsectionfontsize@cx, 
       toc subsection font-family/.store in=\tocsubsectionfontfamily@cx, 
       toc subsection font-shape/.store in=\tocsubsectionfontshape@cx, 
       toc subsection font-weight/.store in=\tocsubsectionfontweight@cx, 
       toc subsection color/.store in=\tocsubsectioncolor@cx,
toc subsection case/.is choice,
       toc subsection case/lower/.code=\def\tocsubsectioncase@cx{\@lowercasesubsectiontrue
                             \if@lowercasesubsection\expandafter\MakeTextLowercase\fi},
       toc subsection case/upper/.code=\def\tocsubsectioncase@cx{\@lowercasesubsectionfalse
                    \if@lowercasesubsection\else\expandafter\MakeTextUppercase \fi},
       toc subsection case/none/.code=\def\tocsubsectioncase@cx{\@empty},
       toc subsection numwidth/.store in=\tocsubsectionnumwidth@cx,
%	  fonts etc for page number
       toc subsection page font-size/.store in=\tocsubsectionpagefontsize@cx,
       toc subsection page font-family/.store in=\tocsubsectionpagefontfamily@cx,
       toc subsection page font-shape/.store in=\tocsubsectionpagefontshape@cx,
       toc subsection page font-weight/.store in=\tocsubsectionpagefonteight@cx,
       toc subsection page color/.store in=\tocsubsectionpagecolor@cx,
%      leaders
	   toc subsection dotsep/.store in = \tocsubsecdotsep@cx,
%      before and after page number
       toc subsection page before/.store in=\tocsubsectionpagebefore@cx,
       toc subsection page after/.store in=\tocsubsectionpageafter@cx,
}
%
\cxset{toc subsection beforeskip=\z@ \@plus.2\p@,
       toc subsection indent=0em,
       toc subsection font-family= sffamily,
       toc subsection font-weight = mdseries,
       toc subsection font-shape = upshape,
       toc subsection color= sweet,
       toc subsection font-size=,
       toc subsection case = none,
       toc subsection numwidth = 4.2em,
       toc subsection page font-size=,
       toc subsection page font-shape= upshape,  
       toc subsection page font-weight=, 
       toc subsection page font-family= sffamily,
       toc subsection page color = sweet, 
       toc subsection page before =,% \{,
       toc subsection page after =,% \},
       toc subsection dotsep = 2.7,
}%
%    \end{macrocode}
%
%    For convenience we define font setting commands for
%    the page number. We use \cs{setfont@cx}, which we have
%	defined earlier.
%    
%    \begin{macrocode}
\newcommand\tocsubsectionpagefont@cx{%
	\setfont@cx{\tocsubsectionpagefonteight@cx}%
       {\tocsubsectionpagefontfamily@cx}{\tocsubsectionpagefontsize@cx}%
       {\tocsubsectionpagefontshape@cx}\color{\tocsubsectionpagecolor@cx}
}%
%        
%
\renewcommand*{\l@subsection}[2]{%
  \ifnum \c@tocdepth >\z@
    \if@haschapter@cx
      \vskip \tocsubsectionbeforeskip@cx
    \else
      \addpenalty \@secpenalty
      \addvspace{\tocsubsectionbeforeskip@cx}%
    \fi
    {\leftskip \tocsubsectionindent@cx\relax
     \rightskip \@tocrmarg
     \parfillskip -\rightskip
     \parindent \tocsubsectionindent@cx\relax\@afterindenttrue
     \interlinepenalty\@M
     \leavevmode
     \@tempdima \tocsubsectionnumwidth@cx\relax
     \let\tocnumberbefore@cx \cftsecpresnum
     \let\@cftasnum \cftsecaftersnum
     \let\@cftasnumb \cftsecaftersnumb
     \advance\leftskip \@tempdima \null\nobreak\hskip -\leftskip
%    \end{macrocode}
%
%	We are now ready to print out the toc subsection title,
%	we set the font information then typeset the title.
%	The dot leaders are typeset by calling the macro \cs{sectionfillnum}
%
%    \begin{macrocode}
     {%
      \setfont@cx{\tocsubsectionfontweight@cx}%
        {\tocsubsectionfontfamily@cx}{\tocsubsectionfontsize@cx}%
        {\tocsubsectionfontshape@cx}%
        \color{\tocsubsectioncolor@cx}%
      \tocsubsectioncase@cx#1}\nobreak
%    \end{macrocode}
%	We are now ready to typeset the leaders and the page number. 
%    We then pass \#2 to the \cs{tocsectionfillnum} which 
%	does the typesetting.
%    \begin{macrocode}
      \tocsubsectionfillnum{#2}}%
  \fi}
%    \end{macrocode}
% \end{macro}
%
%    These are the user commands to control the typesetting of Section entries.
%    They are initialised to give the standard appearance.
%    \begin{macrocode}
% hooks to |\numberline|
\renewcommand{\cftsecpresnum}{.}
\renewcommand{\cftsecaftersnum}{..}
\renewcommand{\cftsecaftersnumb}{...}
\newcommand{\tocsubsectionleader}{\normalfont\dotfill@cx{\tocsubsecdotsep@cx}}
%\newcommand{\cftsecdotsep}{\cftdotsep}
%    \end{macrocode}
%    We can now define the command \cmd{\tocsectionfillnum}. It will print the 
%	leaders if any and the page number \#1. TODO IS par necessary??
%    \begin{macrocode}
\newcommand{\tocsubsectionfillnum}[1]{%
  {\tocsubsectionleader}\nobreak
  \hb@xt@\@pnumwidth{\hfil\tocsubsectionpagefont@cx
   \tocsubsectionpagebefore@cx #1}%
   \tocsubsectionpageafter@cx\par}%
%    \end{macrocode}
%
% \section{Toc subsubsection styling}
%
% \begin{macro}{\l@subsubsection}
%  |\l@subsubsection{|\meta{title}|}{|\meta{page}|}| typesets the ToC entry for
% a |subsubsection| heading. It is a parameterised copy of the default |\l@subsubsection|
%	We start by defining all our parameters and variables.
%
%    \begin{macrocode}
\newif\if@lowercasesubsubsection
\cxset{toc subsubsection beforeskip/.store in=\tocsubsubsectionbeforeskip@cx,
       toc subsubsection indent/.store in=\tocsubsubsectionindent@cx,
%      fonts for title &num
       toc subsubsection font-size/.store in=\tocsubsubsectionfontsize@cx, 
       toc subsubsection font-family/.store in=\tocsubsubsectionfontfamily@cx, 
       toc subsubsection font-shape/.store in=\tocsubsubsectionfontshape@cx, 
       toc subsubsection font-weight/.store in=\tocsubsubsectionfontweight@cx, 
       toc subsubsection color/.store in=\tocsubsubsectioncolor@cx,
toc subsubsection case/.is choice,
       toc subsubsection case/lower/.code=\def\tocsubsubsectioncase@cx{\@lowercasesubsubsectiontrue
                             \if@lowercasesubsubsection\expandafter\MakeTextLowercase\fi},
       toc subsubsection case/upper/.code=\def\tocsubsubsectioncase@cx{\@lowercasesubsubsectionfalse
                    \if@lowercasesubsubsection\else\expandafter\MakeTextUppercase \fi},
       toc subsubsection case/none/.code=\def\tocsubsubsectioncase@cx{\@empty},
       toc subsubsection numwidth/.store in=\tocsubsubsectionnumwidth@cx,
%	  fonts etc for page number
       toc subsubsection page font-size/.store in=\tocsubsubsectionpagefontsize@cx,
       toc subsubsection page font-family/.store in=\tocsubsubsectionpagefontfamily@cx,
       toc subsubsection page font-shape/.store in=\tocsubsubsectionpagefontshape@cx,
       toc subsubsection page font-weight/.store in=\tocsubsubsectionpagefonteight@cx,
       toc subsubsection page color/.store in=\tocsubsubsectionpagecolor@cx,
%      leaders
	   toc subsubsection dotsep/.store in = \tocsubsubsecdotsep@cx,
%      before and after page number
       toc subsubsection page before/.store in=\tocsubsubsectionpagebefore@cx,
       toc subsubsection page after/.store in=\tocsubsubsectionpageafter@cx,
}
%
\cxset{toc subsubsection beforeskip=\z@ \@plus.2\p@,
       toc subsubsection indent=0em,
       toc subsubsection font-family= sffamily,
       toc subsubsection font-weight = mdseries,
       toc subsubsection font-shape = upshape,
       toc subsubsection color= sweet,
       toc subsubsection font-size=,
       toc subsubsection case = none,
       toc subsubsection numwidth = 4.2em,
       toc subsubsection page font-size=,
       toc subsubsection page font-shape= upshape,  
       toc subsubsection page font-weight=, 
       toc subsubsection page font-family= sffamily,
       toc subsubsection page color = teal, 
       toc subsubsection page before =,% \{,
       toc subsubsection page after =,% \},
       toc subsubsection dotsep = 2.7,
}
%    \end{macrocode}
%
%    For convenience we define font setting commands for
%    the page number. We use \cs{setfont@cx}, which we have
%	defined earlier. Note this might be clobbered if 
%  hyperref is to provide a page link.
%    
%    \begin{macrocode}
\newcommand\tocsubsubsectionpagefont@cx{%
	\setfont@cx{\tocsubsubsectionpagefonteight@cx}%
       {\tocsubsubsectionpagefontfamily@cx}{\tocsubsubsectionpagefontsize@cx}%
       {\tocsubsubsectionpagefontshape@cx}\color{\tocsubsubsectionpagecolor@cx}
}%
%        
%
\renewcommand*{\l@subsubsection}[2]{%
  \ifnum \c@tocdepth >\z@
    \if@haschapter@cx
      \vskip \tocsubsubsectionbeforeskip@cx
    \else
      \addpenalty \@secpenalty
      \addvspace{\tocsubsubsectionbeforeskip@cx}%
    \fi
    {\leftskip \tocsubsubsectionindent@cx\relax
     \rightskip \@tocrmarg
     \parfillskip -\rightskip
     \parindent \tocsubsubsectionindent@cx\relax\@afterindenttrue
     \interlinepenalty\@M
     \leavevmode
     \@tempdima \tocsubsubsectionnumwidth@cx\relax
     \let\tocnumberbefore@cx \cftsecpresnum
     \let\@cftasnum \cftsecaftersnum
     \let\@cftasnumb \cftsecaftersnumb
     \advance\leftskip \@tempdima \null\nobreak\hskip -\leftskip
%    \end{macrocode}
%
%	We are now ready to print out the toc subsection title,
%	we set the font information then typeset the title.
%	The dot leaders are typeset by calling the macro \cs{sectionfillnum}
%
%    \begin{macrocode}
     {%
      \setfont@cx{\tocsubsubsectionfontweight@cx}%
        {\tocsubsubsectionfontfamily@cx}{\tocsubsubsectionfontsize@cx}%
        {\tocsubsubsectionfontshape@cx}%
        \color{\tocsubsubsectioncolor@cx}%
      \tocsubsubsectioncase@cx#1}\nobreak
%    \end{macrocode}
%
%	We are now ready to typeset the leaders and the page number. 
%    We then pass \#2 to the \cs{tocsectionfillnum} which 
%	does the typesetting.
%    \begin{macrocode}
      \tocsubsubsectionfillnum{#2}}%
  \fi}
%    \end{macrocode}
% \end{macro}
%
%    These are the user commands to control the typesetting of Section entries.
%    They are initialised to give the standard appearance.
%    \begin{macrocode}
% hooks to |\numberline|
\renewcommand{\cftsecpresnum}{.}
\renewcommand{\cftsecaftersnum}{..}
\renewcommand{\cftsecaftersnumb}{...}
\newcommand{\tocsubsubsectionleader}{\normalfont\dotfill@cx{\tocsubsubsecdotsep@cx}}
%\newcommand{\cftsecdotsep}{\cftdotsep}
%    \end{macrocode}
%    We can now define the command \cmd{\tocsectionfillnum}. It will print the 
%	leaders if any and the page number \#1. TODO IS par necessary??
%    \begin{macrocode}
\newcommand{\tocsubsubsectionfillnum}[1]{%
  {\tocsubsubsectionleader}\nobreak
  \hb@xt@\@pnumwidth{\hfil\tocsubsubsectionpagefont@cx
   \tocsubsubsectionpagebefore@cx #1}%
   \tocsubsubsectionpageafter@cx\par}%
%    \end{macrocode}
%
% \section{Toc paragraph styling}
% \begin{macro}{\l@paragraph}
%  |\l@subsubsection{|\meta{title}|}{|\meta{page}|}| typesets the ToC entry for
% a |subsubsection| heading. It is a parameterised copy of the default |\l@subsubsection|
%	We start by defining all our parameters and variables.
%
%    \begin{macrocode}
\newif\if@lowercaseparagraph
\cxset{toc paragraph beforeskip/.store in=\tocparagraphbeforeskip@cx,
       toc paragraph indent/.store in=\tocparagraphindent@cx,
%      fonts for title &num
       toc paragraph font-size/.store in=\tocparagraphfontsize@cx, 
       toc paragraph font-family/.store in=\tocparagraphfontfamily@cx, 
       toc paragraph font-shape/.store in=\tocparagraphfontshape@cx, 
       toc paragraph font-weight/.store in=\tocparagraphfontweight@cx, 
       toc paragraph color/.store in=\tocparagraphcolor@cx,
toc paragraph case/.is choice,
       toc paragraph case/lower/.code=\def\tocparagraphcase@cx{\@lowercaseparagraphtrue
                             \if@lowercaseparagraph\expandafter\MakeTextLowercase\fi},
       toc paragraph case/upper/.code=\def\tocparagraphcase@cx{\@lowercaseparagraphfalse
                    \if@lowercaseparagraph\else\expandafter\MakeTextUppercase \fi},
       toc paragraph case/none/.code=\def\tocparagraphcase@cx{\@empty},
       toc paragraph numwidth/.store in=\tocparagraphnumwidth@cx,
%	  fonts etc for page number
       toc paragraph page font-size/.store in=\tocparagraphpagefontsize@cx,
       toc paragraph page font-family/.store in=\tocparagraphpagefontfamily@cx,
       toc paragraph page font-shape/.store in=\tocparagraphpagefontshape@cx,
       toc paragraph page font-weight/.store in=\tocparagraphpagefonteight@cx,
       toc paragraph page color/.store in=\tocparagraphpagecolor@cx,
%      leaders
	   toc paragraph dotsep/.store in = \tocparagraphdotsep@cx,
%      before and after page number
       toc paragraph page before/.store in=\tocparagraphpagebefore@cx,
       toc paragraph page after/.store in=\tocparagraphpageafter@cx,
}

\cxset{toc paragraph beforeskip=\z@ \@plus.2\p@,
       toc paragraph indent=0em,
       toc paragraph font-family= sffamily,
       toc paragraph font-weight = mdseries,
       toc paragraph font-shape = upshape,
       toc paragraph color= teal,
       toc paragraph font-size=,
       toc paragraph case = none,
       toc paragraph numwidth = 4.2em,
       toc paragraph page font-size=,
       toc paragraph page font-shape= upshape,  
       toc paragraph page font-weight=, 
       toc paragraph page font-family= sffamily,
       toc paragraph page color = teal, 
       toc paragraph page before =,% \{,
       toc paragraph page after =,% \},
       toc paragraph dotsep = 2.7,
}
%    \end{macrocode}
%
%	\subsection{Default paragraph style}
%   \subsubsection{With a subsubsection}
%
%
%	\paragraph{Test paragraph} This is a test.
%
%    For convenience we define font setting commands for
%    the page number. We use \cs{setfont@cx}, which we have
%	defined earlier.
%    
%    \begin{macrocode}
\newcommand\tocparagraphpagefont@cx{%
	\setfont@cx{\tocparagraphpagefonteight@cx}%
       {\tocparagraphpagefontfamily@cx}{\tocparagraphpagefontsize@cx}%
       {\tocparagraphpagefontshape@cx}\color{\tocparagraphpagecolor@cx}
}%
%        
%
\renewcommand*{\l@paragraph}[2]{%
  \ifnum \c@tocdepth >\z@
    \if@haschapter@cx
      \vskip \tocparagraphbeforeskip@cx
    \else
      \addpenalty \@secpenalty
      \addvspace{\tocparagraphbeforeskip@cx}%
    \fi
    {\leftskip \tocparagraphindent@cx\relax
     \rightskip \@tocrmarg
     \parfillskip -\rightskip
     \parindent \tocparagraphindent@cx\relax\@afterindenttrue
     \interlinepenalty\@M
     \leavevmode
     \@tempdima \tocparagraphnumwidth@cx\relax
     \let\tocnumberbefore@cx \cftsecpresnum
     \let\@cftasnum \cftsecaftersnum
     \let\@cftasnumb \cftsecaftersnumb
     \advance\leftskip \@tempdima \null\nobreak\hskip -\leftskip
%    \end{macrocode}
%
%	We are now ready to print out the toc subsection title,
%	we set the font information then typeset the title.
%	The dot leaders are typeset by calling the macro \cs{sectionfillnum}
%
%    \begin{macrocode}
     {%
      \setfont@cx{\tocparagraphfontweight@cx}%
        {\tocparagraphfontfamily@cx}{\tocparagraphfontsize@cx}%
        {\tocparagraphfontshape@cx}%
        \color{\tocparagraphcolor@cx}%
      \tocparagraphcase@cx#1}\nobreak
%    \end{macrocode}
%	We are now ready to typeset the leaders and the page number. 
%    We then pass \#2 to the \cs{tocsectionfillnum} which 
%	does the typesetting.
%    \begin{macrocode}
      \tocparagraphfillnum{#2}}%
  \fi}
%    \end{macrocode}
% \end{macro}
%
%    These are the user commands to control the typesetting of Section entries.
%    They are initialised to give the standard appearance.
%    \begin{macrocode}
% hooks to |\numberline|
\renewcommand{\cftsecpresnum}{.}
\renewcommand{\cftsecaftersnum}{..}
\renewcommand{\cftsecaftersnumb}{...}
\newcommand{\tocparagraphleader}{\normalfont\dotfill@cx{\tocparagraphdotsep@cx}}
%\newcommand{\cftsecdotsep}{\cftdotsep}
%    \end{macrocode}
%    We can now define the command \cmd{\tocsectionfillnum}. It will print the 
%	leaders if any and the page number \#1. TODO IS par necessary??
%    \begin{macrocode}
\newcommand{\tocparagraphfillnum}[1]{%
  {\tocparagraphleader}\nobreak
  \hb@xt@\@pnumwidth{\hfil\tocparagraphpagefont@cx
   \tocparagraphpagebefore@cx #1}%
   \tocparagraphpageafter@cx\par}%
%    \end{macrocode}
%
% This brings us, dear reader to a long and arduous 
% path. Surely there must be an easier way. we have 
% added parameters in all sectioning commands, down to 
% paragraph level and we can even lower if you want
% for the legal guys and for construction specs.
%
% \chapter{Handling Footnotes and Endnotes}
%
% \precis{Handling of Footnotes and Endnotes.}
%
% Keeping up with the spirit of the package, we now
% have a go at footnotes and endnotes. This is a difficult
% topic, with many packages and a diverse way of handlingg
% things.
% TO DO STORE IN PREHOOKS
% AND POST HOOKS
%
%    \begin{macrocode}
\cxset{endnotes package/.code ={\gdef\endnotes@cs{#1}%
                   \RequirePackage{\endnotes@cs}%
                }%
}%
\cxset{endotes package/.default=pagenote}
\cxset{endnotes package=pagenote}%
%
%    \end{macrocode}
%
%   This also has a number of other packages loaded
%
%
% \chapter{Handling Images}
%
% \precis{Commands for laying out complex pages composed primarily of images.}
%
% \section{creating Image Page Styles}
%
% We now develop a method to produce variable environments
% that can include images in a page. We start using designs
% that incorporate two columns, as shown on 
% Page~\pageref{krollcode}.
%
% \begin{macro}{\miniwidthi}
% \begin{macro}{\miniwidthii}
% \begin{macro}{\sepmainhorizontal}
%    \begin{macrocode}
\global\newlength{\miniwidthi}
\global\newlength{\miniwidthii}
\global\newlength{\sepmainhorizontal}
\def\tinyskip{\vskip2pt}
%    \end{macrocode}
% \end{macro}
% \end{macro}
% \end{macro}
%    \begin{macrocode}
\newenvironment{leftcolumn}{}{}
\newenvironment{rightcolumn}{}{}

\newlength\offsetfromright
\setlength\offsetfromright{0em}
%    \end{macrocode}
%
% \begin{macro}{\onelinecaption} The oneline caption
% is the description that goes underneath images that
% unlike figures, they are not described in the text.
%    \begin{macrocode}
\newcommand\onelinecaption[2][]{%
    \setlength\offsetfromright{0em}%
    \bgroup%
        \vskip0pt plus1pt minus1pt %
        \reset@font
        \sffamily
        \bfseries%
        \footnotesize%
        \hfill\hfill#2\hbox to \offsetfromright{}%
     \egroup%
}
%    \end{macrocode}
% \end{macro}
% 
% \begin{macro}{\onelineheader} This macro takes one parameter
%   and styles the main header.
%    \begin{macrocode}
\long\def\onelineheader#1{%
 \vspace{1.5\baselineskip}%
 {\sffamily{\bgroup\LARGE\bf \mbox{#1}\egroup}%
 \vspace{0.5\baselineskip}}%
}
%    \end{macrocode}
% \end{macro}
%    \begin{macrocode}
\newcommand\byline[2][]{\small{\bfseries#1}#2}
 \newcommand\MainHeader[1]{{\leavevmode\par\centering \textrm{\fontsize{50pt}{65pt}\selectfont #1}\par\vspace{1cm}}}
 \newcommand\MainHeadera[1]{{\leavevmode\par\centering \textrm{\fontsize{30pt}{42pt}\selectfont #1}\par\vspace{1cm}}}
%    \end{macrocode}
%
%    \begin{macrocode}
\def\aheader#1{\footnotesize \textbf{SELF-PORTRAIT}#1}
 \renewenvironment{leftcolumn}[1]{%
        \begin{minipage}[b]{\miniwidthi} #1}{\end{minipage} \hspace{\sepmainhorizontal}}%
    \renewenvironment{rightcolumn}[1]{%
        \begin{minipage}[b]{\miniwidthii} #1}{\end{minipage}}%
\def\starttemplate#1{%
  %% we now calculate some of the parameters
%% required
    \setlength\miniwidthi{0.3\textwidth}%
    \setlength\miniwidthii{0.67\textwidth}%
    \setlength\sepmainhorizontal{0.03\textwidth}%
   %
   %
%% Create environments for convenience
   %% Create right column environment
}
 \def\stoptemplate{}
%
%% Defining kroll style
  %% We need to find a way to define the templates
%% We will assume that images have been saved in a database
%% image@file
%% image@caption
%% this is a must to avoid long typing and keep the environments
%% short
\fboxsep=0pt
\fboxrule=1pt
\define@key{img}{width}[1cm]{\def\img@width{#1}}
\define@key{img}{height}{\def\img@height{#1}}
\define@key{img}{offsetx}{\def\img@offsetx{#1}}
\define@key{img}{offsety}{\def\img@offsety{#1}}
\define@key{img}{border}{\def\img@border{#1}}
\define@key{img}{padding}{\def\img@padding{#1}}
\define@key{img}{style}{\def\img@style{#1}}
\define@key{img}{bottommargin}{\def\img@bottommargin{#1}}
\define@key{img}{keepaspectratio}{\def\img@keepaspectratio{keepaspectratio}}
\define@key{imgpg}{pagestyle}{\def\imgpg@pagestyle{#1}}
%% Set defaults for all keys
\setkeys{img}{offsetx=1sp, offsety=0pt,width=3cm, keepaspectratio=keepaspectratio,
                      border=0pt, padding=0pt,bottommargin=0pt}
%% Create the command graphic
\newlength\tempal
%%
%% We create a new command to place images 
\newcommand\putimage[2][0pt]{%
%% Set the keys
\setkeys{img}{#1}%
\setlength\fboxrule\img@border%
\setlength\fboxsep\img@padding%
\ifdim\img@offsety=0pt% 
\else%
\vspace*{\img@offsety}%
\fi%
\hskip\img@offsetx%
\setlength{\tempal}{\img@width}
\fboxsep=1pt
\def\setcaption{\captionof{figure}{This is the caption for the figure\lorem}}%
\begin{minipage}{\textwidth}%
\fbox{\includegraphics[width=\textwidth]{#2}}%
\end{minipage}
}%\vspace*{\img@bottommargin}}%
%    \end{macrocode}
%
% 
% \clearpage
% 
% ^^A\newgeometry{top=0.5cm, bottom=1cm, left=1cm, right=1cm,
%   ^^A            marginparsep=0cm, marginpar=0pt}
% 
% \clearpage 
% \newpage
%
% \hrule
% \mbox{}
%    
% \label{krollcode}
% \renewenvironment{leftcolumn}{%
%   \minipage[b]{.3\textwidth}%
%  }{\endminipage}\hspace*{0cm}%
% 
% \starttemplate{kroll}%
% \vspace*{-.8cm}
% \hspace*{-1cm}\begin{leftcolumn}%
%   \MainHeader{Leon\\[15pt] Kroll}
%   \putimage[width=0.5\linewidth]{krollportrait.jpg}\par
%   \aheader{shows Kroll at 59. Says he. ``Painting is 
%             fascinating'' even when motif my own mug.}
% \end{leftcolumn}%
% \begin{minipage}[b]{0.8\textwidth}%
%   \includegraphics[width=\linewidth]{nudeback.jpg}
%       \onelinecaption{{\resizebox{\linewidth}{5.5pt}{\bfseries \hfill NUDE \hfill BACK \hfill SHOWS \hfill  A \hfill DANCER \hfill WHOSE \hfill BACK \hfill SAYS \hfill KROLL, \hfill HAS \hfill BEAUTIFUL \hfill PLANES }}\par}
%       \onelineheader{THE DEAN OF U.S. NUDE-PAINTERS}
%      \begin{multicols}{2}
%      \small
%      \lettrine{A}{t the} age of 63 when businessmen are thinking of retiring Leon Kroll according to Life Magazine was having the busiest time of his life, just doing what comes naturally.  \lorem\lorem
%      \end{multicols}
%   \end{minipage}
%\stoptemplate
% 
%
% \newpage
%
%\starttemplate{kroll}
%    \begin{minipage}[b]{0.3\textwidth}
%       \MainHeadera{Sandro Botticelli}
%       \includegraphics[width=1.0\linewidth]{botticelli-34.jpg}\par
%       \byline[BOTTICELLI ]{ painted hundreds of portraits. He is famous for his `Young Woman' series. Even in his larger compositions, he took extreme care of the details of women's faces.}
%   \end{minipage}\hspace*{0.2cm}
%   \begin{minipage}[b]{0.67\textwidth}
%       \putimage[width=\linewidth]{youngwoman.png}\par
%       \tinyskip
%       \onelinecaption{YOUNG WOMAN}\par
%       \onelineheader{SADRO BOTTICELLI'S PORTRAITS}
%      \begin{multicols}{2}
%      \small
%      \lettrine{A}{t the} age of 63 when businessmen are thinking of retiring leon Kroll according to Life Magazine was having the busiest time of his life, just doing what comes naturally.  \lorem\lorem
%      \end{multicols}
%    \end{minipage}
%\stoptemplate
%
%
% ^^A\newgeometry{top=1cm,left=1cm,right=1cm,bottom=1cm}
% \newtheorem{process}{Algorithm}
% \begin{process}
% Test exam
% \end{process}
%\clearpage
%
% \chapter{Posters}
% \makeatletter
%\def\HUGE{\@setfontsize\HUGE{65}{90}}
% \makeatother
%\raggedbottom
%\begin{minipage}{0.8\textwidth}
%\sffamily
%\centering
%\HUGE{\bf SYMPOSIUM}\\
%
%\LARGE{\textbf{\so{CAROLINA DYNAMICS GROUP}}}\\
%
%\Large{\textbf{\so{CLEMSON UNIVERSITY}}}\\
%%\large{\textbf{APRIL 5, 2012}}
%
%\bigskip
%
%\begin{minipage}[b]{0.6\textwidth}
%\normalsize
%\includegraphics[width=\linewidth]{lorenzattractor01.jpg}
%
%\textbf{PRESENTED BY THE CAROLINA DYNAMICAL \\ SYSTEMS GROUP}. Some more text here to fill the space. You need to get the reader to stop and read a bit more. Some more text here to fill the space. \par 
%\rule{0pt}{32pt}
%\end{minipage}\hspace{5pt}
%\begin{minipage}[b]{0.37\textwidth}
%\textbf{\large INVITED SPEAKERS}\par
%
%{\leavevmode \raggedright
%Dr Liang Foo\\
%Dr Berry Ling\\
%Dr Zezsko Petrovick \\
%Dr A Berchowitz\\
%\par{}
%}
%
%\medskip
%
%\large{\textbf{VENUE}}
%
%The symposium will take place at Clemson University.
%
%\medskip
%
%{\large\raggedright
%{\textbf{DYNAMICAL SYSTEMS}}
%\par
%}
%\normalsize
%\smallskip
%
%\lorem\lorem\lorem
%
%\medskip
%\textbf{\large ORGANIZERS}\par
%Martin Schmoll, Clemson University\\
%Predrag Punosevac, Augusta State \\
%University
%\medskip
%
%\textbf{\large CONTACT }\par
%
%Predrag Punosevac, Augusta State\\
%University
%\textcolor{blue}{email@mail.com}
%
%\rule{0pt}{78pt}
%\end{minipage}\par
% ^^A\vspace*{-0pt}
%
%\hbox to \textwidth{\HUGE{\bfseries\rmfamily APRIL 5 $\cdot$  2012}}
%\end{minipage}
%
%
% ^^A\newgeometry{top=2cm, bottom=3cm, left=3.5cm, right=3.5cm}

% \appendix
% \cxset{
%  chapter name = Appendix,
%  section numbering prefix = \thechapter.}
%  
% \chapter{MWE and Testing Macros}
%
% As far as LaTeX is concerned, there is nothing special in styling an appendix. It is either a chapter or a section with a different name. This name in order to allow internationalization is called \lstinline{\appendixname}.
%\bigskip
%
%\begin{tcolorbox}[width=\linewidth]
%\begin{lstlisting}
%\newcommand\appendix{\par
%  \setcounter{chapter}{0}%
%  \setcounter{section}{0}%
%  \gdef\@chapapp{\appendixname}(*@\footnote{The actual literal used for   \textbackslash{appendixname} is defined later on, so that you can customize the language}\label{appendixname}@*)
%  \gdef\thechapter{\@Alph\c@chapter}
%}
%\end{lstlisting}
%\end{tcolorbox}
%\medskip
%
%The code above is only a simplified version of the command. One might need to add more formatting information such as resetting equation numbers, tables and figures and any special floating environments that have their own numbering.
%
%\begin{tcolorbox}[width=\linewidth]
%\begin{lstlisting}
%\renewcommand\appendix{\par
%                \stepcounter{chapter}
%                \setcounter{chapter}{0}
%                \stepcounter{section}
%                \setcounter{section}{0}
%                \setcounter{equation}{0}
%                \setcounter{figure}{0}
%                \setcounter{table}{0}
%                \setcounter{footnote}{0}
%  \def\@chapapp{\appendixname}%
%  \renewcommand\thechapter{\@Alph\c@chapter}}
%\end{lstlisting}
%\end{tcolorbox}
%
%  
%

%\iffalse
%</package>
%\fi
%
% \section{Experimental Lua Code}
% \subsection{hiero lua module}
% 
% \iffalse
%<*hhiero1>
% \fi
%
%    \begin{macrocode}
-- returns a tex string
-- with a font command to print 
-- hieroglyphics

 local hiero 				= hiero or {}
 local inlinemath 		    = "$"

 local function getglyph(cmd, codepoint)
 local texstring = "\\Large\\"..cmd.." \\char".."\""..codepoint
  return texstring
end

function getglyphRL(cmd, codepoint)
  local texstring = "\\scalebox{-1}[1]{\\"..cmd.." \\char".."\""..codepoint.."}"
  return texstring
end

function printglyphRL(cmd, codepoint)
  tex.print(getglyphRL(cmd,codepoint))
end

local printhierochar = function (cmd, unicode, options)
	-- prints single hieroglyph 
   -- formatting commands are set via options
   -- @cmd = font family command

   local tx = "\\"..cmd.." \\char\""..unicode.."" 
   local direction, size, color
   direction = "LR"

   if options.size then size = options.size 
   else
     size = ""
   end
   if options.color then color = "\\color{"..options.color.."} " 
   else
     color = ""
   end

   local scale
   if options.scale then scale = options.scale 
   else
      scale = 1
   end

   local texcmds = size..color

   if options.direction then direction = options.direction end
   if direction == "RL" then
      tx = "{\\scalebox{"..-1*scale.."}["..scale.."]{"..texcmds.." "..tx.."}}"
   else
		tx = "{\\scalebox{"..1*scale.."}["..scale.."]{"..texcmds.." "..tx.."}}"
   end
   return tex.print(tx)
end



function stackrelf(a,b,c)
  return ""..inlinemath.."\\stackrel{\\mbox{"..a.."}}{\\mbox{"..b.."}}"..inlinemath..""
end

--[[
We define a local table to hold data for glyphs 
@t : table
--]]

local t = t or {}

--- returns a number of transformations and data
-- to the table holding codepoints for Gardiner lists
---

local function f(codepoint, gardiner, mnemonic,
                 description, determinant)

    	local theglyph=""  -- typeset glyph
    	local glyphslot
	 	if gardiner==nil or gardiner=="" then gardiner="empty" end
    	if mnemonic==nil or mnemonic=="" then mnemonic="\\phantom{z}" end
    	if description==nil then description="no description" end
 
-- stackengine 
   local hook = "\\color{blue}\\huge\\hiero"
   local stackrel = stackrelf(hook.."\\char\""..codepoint.." ","\\footnotesize 0x"..codepoint)
   stackrel = stackrelf(stackrel,"\\footnotesize"..tonumber(codepoint, 16))
   stackrel = stackrelf(stackrel,"\\footnotesize \\arial ".. gardiner)
   stackrel = stackrelf(stackrel,"\\footnotesize \\arial ".. mnemonic)
  
   theglyph = "\\scalebox{-1}[1]{\\hiero\\char\""..codepoint.."\\hskip0pt }"
   glyphslot = codepoint


return {  fullblock     = "\\scalebox{1}[1]{"..stackrel.."}", 
          unicode       = codepoint, --
          gardiner      = gardiner,
          mnemonic      = mnemonic,
          description   = description,
          theglyph      = theglyph,
          glyphslot     = codepoint}
end

-- The Gardiner classification table
-- and mnemonics as per MdC1988 
--   

t = {
   ["A1"    ]     =  f(13000,"A1","",
                         "kneeling man",
                          "Det.I (Masculine) (paeu)"),
   ["A2"    ] 		=  f(13001,"A2"),
   ["A3"    ] 		=  f(13002,"A3"),
   ["A4"    ] 		=  f(13003,"A4"),
   ["A5"    ]		  =  f(13004,"A5"), 
   ["A5a"   ] 		=  f(13005,"A5a"),
   ["A6"    ]		  =  f(13006,"A6"),
   ["A6a"   ]		  =  f(13007,"A6a"),
   ["A6b"   ]		  =  f(13008,"A6b"),
   ["A7"    ] 		=  f(13009,"A7"),
   ["A8"    ] 		=  f("1300A","A8"),
   ["A9"    ] 		=  f("1300B","A9"),
   ["A10"   ] 		=  f("1300C","A10"),
   ["A11"   ] 		=  f("1300D","A11"),
   ["A12"   ] 		=  f("1300E","A12","mSa"),
   ["A13"   ] 		=  f("1300F","A13"),
   ["A14"   ] 		=  f("13010","A14"),
   ["A14a"  ] 		=  f("13011","A14a"),
   ["A15"   ]		  =  f(13012,"A15","xr"),
   ["A16"   ] 		=  f(13013,"A16"),
   ["A17"   ] 		=  f(13014,"A17","Xrd"),
   ["A17a"  ] 		=  f(13015,"A17a"),
   ["A18"   ] 		=  f(13016,"A18"),
   ["A19"   ] 		=  f(13017,"A19"),
   ["A20"   ] 		=  f(13018,"A20"),
   ["A21"   ] 		=  f("13019","A21","sr"),
   ["A22"   ] 		=  f("1301A","A22"),
   ["A23"   ] 		=  f("1301B","A23"),
   ["A24"   ] 		=  f("1301C","A24"),
   ["A25"   ] 		=  f("1301D","A25"),
   ["A26"   ] 		=  f("1301E","A26"),
   ["A27"   ] 		=  f("1301F","A27"),
   ["A28"   ] 		=  f("13020","A28"),
   ["A29"   ] 		=  f("13021","A29"),
   ["A30"   ] 		=  f("13022","A30"),
   ["A31"   ] 		=  f("13023","A31"),
   ["A32"   ] 		=  f("13024","A32"),
   ["A32a"  ] 		=  f("13025","A32a"),
   ["A33"   ] 		=  f(13026,"A33","mniw"),
   ["A34"   ] 		=  f("13027","A34"),
   ["A35"   ] 		=  f("13028","A35"),
   ["A36"   ] 		=  f("13029","A36"),
   ["A37"   ] 		=  f("1302A","A37"),
   ["A38"   ] 		=  f("1302B","A38","qiz"),
   ["A39"   ] 		=  f("1302C","A39"),
   ["A40"   ] 		=  f("1302D","A40"),
   ["A40a"  ] 		=  f("1302E","A40a"),
   ["A41"   ] 		=  f("1302F","A41"),
   ["A42"   ] 		=  f("13030","A42"),
   ["A42a"  ] 		=  f("13031","A42a"),
   ["A43"   ] 		=  f("13032","A43"),
   ["A43a"  ] 		=  f("13033","A43a"),
   ["A44"   ] 		=  f("13034","A44"),
   ["A45"   ] 		=  f("13035","A45"),
   ["A45a"  ] 		=  f("13036","A45a"),
   ["A46"   ] 		=  f("13037","A46"),
   ["A47"   ] 		=  f(13038,"A47","iry"),
   ["A48"   ] 		=  f("13039","A48"),
   ["A49"   ] 		=  f("1303A","A49"),
   ["A50"   ] 		=  f("1303B","A50","Sps"),
   ["A51"   ] 		=  f("1303C","Spsi"),
   ["A52"   ] 		=  f("1303D","A52"),
   ["A53"   ] 		=  f("1303E","A53"),
   ["A54"   ] 		=  f("1303F","A54"),
   ["A55"   ] 		=  f("13040","A55"),
   ["A56"   ] 		=  f("13041","A56"),
   ["A57"   ] 		=  f("13042","A57"),
   ["A58"   ] 		=  f("13043","A58"),
   ["A59"   ]     =  f("13044","A59"),
   ["A60"   ] 		=  f("13045","A60"),
   ["A61"   ] 		=  f("13046","A61"),
   ["A62"   ]     =  f("13047","A62"),
   ["A63"   ] 		=  f("13048","A63"),
   ["A64"   ] 		=  f("13049","A64"),
   ["A65"   ] 		=  f("1304A","A65"),
   ["A66"   ] 		=  f("1304B","A66"),
   ["A67"   ] 		=  f("1304C","A67"),
   ["A68"   ] 		=  f("1304D","A68"),
   ["A69"   ] 		=  f("1304E","A69"),
   ["A70"   ] 		=  f("1304F","A70"),

-- Woman and her occupations
-- Unicode points 

	["B1"   ] 		= f("13050","B1"),
	["B2"   ] 		= f("13051","B2"),
	["B3"   ] 		= f(13052, "B3", "msi"),
	["B4"   ] 		= f("13053","B4"),
	["B5"   ] 		= f("13054","B5"),
	["B5a"  ]		= f("13055","B5a"),
	["B6"   ] 		= f("13056","B6"),
	["B7"   ] 		= f("13057","B7"),
	["B8"   ] 		= f("13058","B8"),
	["B9"   ] 		= f("13059","B9"),

-- C series

	["C1"   ] 		= f("1305A","C1"),
	["C2"   ] 		= f("1305B","C2"),
	["C2a"  ] 		= f("1305C","C2a"),
	["C2b"  ] 		= f("1305D","C2b"),
	["C2c"  ] 		= f("1305E","C2c"),
	["C3"   ] 		= f("1305F","DHwty"),
	["C4"   ] 		= f(13060,"Xnmw"),
	["C5"   ] 		= f(13061,"Xnmw"),
	["C6"   ] 		= f(13062,"inpw"),
	["C7"   ] 		= f(13063,"stX"),
	["C8"   ] 		= f(13064,"mnw"),
    ["C9"   ]  		= f(13065,"C9"),  
	["C10"  ] 		= f(13066,"mAat"),
	["C10a" ] 		= f(13067,"C10a"),
	["C11"  ] 		= f(13068,"HH"),
	["C12"  ] 		= f(13069,"C12"),
	["C13"  ] 		= f("1306A","C13"),
	["C14"  ] 		= f("1306B","C14"),
	["C15"  ] 		= f("1306C","C15"),
	["C16"  ] 		= f("1306D","C16"),
	["C17"  ] 		= f("1306E","C17"),
	["C18"  ] 		= f("1306F","C18"),
	["C19"  ] 		= f(13070,"C19"),
	["C20"  ] 		= f(13071,"C20"),
	["C21"  ] 		= f(13072,"C21"),
	["C22"  ] 		= f(13073,"C22"),
	["C23"  ] 		= f(13074,"C23"),
	["C24"  ] 		= f(13075,"C24"),

-- D series Parts of the Human Body
-- needs checking missed some

	["D1"   ]  		= f(13076,"tp"),
	["D2"   ]  		= f(13077,"Hr"),
	["D3"   ] 		= f(13078,"Sny"),
	["D4"   ]  		= f(13079,"ir"),
	["D5"   ]  		= f("1307A","D5"),
	["D6"   ]  		= f("1307B","D6"),
	["D7"   ]  		= f("1307C","D7"),
	["D8"   ] 		= f("1307D","D8"),
	["D8a"  ]		= f("1307E","D8a"),
	["D9"   ]  		= f("1307F","rmi"),
	["D10"  ] 		= f(13080,"wDAt"),
	["D11"  ] 		= f(13081,"D11"),
	["D12"  ] 		= f(13082,"D12"),
	["D13"  ] 		= f(13083,"D13"),
	["D14"  ] 		= f(13084,"D14"),
	["D15"  ] 		= f(13085,"D15"),
	["D16"  ] 		= f(13086,"D16"),
	["D17"  ] 		= f(13087,"D17"),
	["D18"  ] 		= f(13088,"D18"),
	["D19"  ] 		= f(13089,"fnD"),
	["D20"  ] 		= f("1308A","D20"),
	["D21"  ] 		= f("1308B","r"),
	["D22"  ] 		= f("1308C","D22"),
	["D23"  ] 		= f("1308D","D23"),
	["D24"  ] 		= f("1308E","spt"),
	["D25"  ] 		= f("1308F","spty"),
	["D26"  ] 		= f("13090","D26"),
	["D27"  ] 		= f(13091,"mnD"),

	["D27a" ] 		= f(13092,"kA"),
	["D28"  ] 		= f(13093,"D29"),
	["D29"  ] 		= f(13094,"D30"),
	["D30"  ] 		= f(13095,"D31"),
	["D31"  ] 		= f(13096,"D32"),
  
	["D31a" ] 		= f(13097,"D33"),
	["D32"  ] 		= f(13098,"aHA"),
	["D33"  ] 		= f(13099,"D34"),
	["D34"  ] 		= f("1309A","aHA"),
	["D34a" ] 		= f("1309B","a"),
	["D35"  ] 		= f("1309C","D35"),
	["D36"  ] 		= f("1309D","D36"),
	["D37"  ] 		= f("1309E","D37"),
	["D38"  ] 		= f("1309F","D38"),
	["D39"  ] 		= f("130A0","D39"),

	["D40" ] 		= f("130A1","Dsr"),
	["D41" ] 		= f("130A2","d"),
	["D42" ] 		= f("130A3","Dba"),
	["D43" ] 		= f("130A4","D43"),
	["D44" ] 		= f("130A5","D44"),
	["D45" ] 		= f("130A6","D45"),
	["D46" ] 		= f("130A7","D46"),
	["D46a"] 		= f("130A8","D46a"),
	["D47" ] 		= f("130A9","D47"),
	["D48" ] 		= f("130AA","D48"),
	["D48a"] 		= f("130AB","D48a"),
	["D49" ] 		= f("130AC","D49"),
	["D50" ] 		= f("130AD","D50"),
	["D50a"] 		= f("130AE","D50a"),
	["D50b"] 		= f("130AF","D50b"),
	["D50c"] 		= f("130B0","D50c"),
	["D50d"] 		= f("130B1","D50d"),
	["D50e"] 		= f("130B2","D50e"),
	["D50f"] 		= f("130B3","D50f"),
	["D50g"] 		= f("130B4","D50g"),
	["D50h"] 		= f("130B5","D50h"),
	["D50i"] 		= f("130B6","D50i"),
	["D51" ] 		= f("130B7","D51"),
	["D52" ] 		= f("130B8","mt"),
	["D52a"] 		= f("130B9","D52a"),
	["D53" ] 		= f("130BA","D53"),
	["D54" ] 		= f("130BB","D54"),
	["D54a"] 		= f("130BC","D54a"),
	["D55" ] 		= f("130BD","D55"),
	["D56" ] 		= f("130BE","rd"),
	["D57" ] 		= f("130BF","D57"),
	["D58" ] 		= f("130C0","b"),
	["D59" ] 		= f("130C1","ab"),
	["D60" ] 		= f("130C2","wab"),
	["D61" ] 		= f("130C3","sAH"),
	["D62" ] 		= f("130C4","D62"),
	["D63" ] 		= f("130C5","D63"),
	["D64" ] 		= f("130C6","D64"),
	["D65" ] 		= f("130C7","D65"),
	["D66" ] 		= f("130C8","D66"),
	["D67" ] 		= f("130C9","D67"),
	["D67a"] 		= f("130CA","D67a"),
	["D67b"] 		= f("130CB","D67b"),
	["D67c"] 		= f("130CC","D67c"),
	["D67d"] 		= f("130CD","D67d"),
	["D67e"] 		= f("130CE","D67e"),
	["D67f"] 		= f("130CF","D67f"),
	["D67g"] 		= f("130D0","D67g"),
	["D67h"] 		= f("130D1","D67h"),

-- E 

    ["E1"  ] 		= f("130D2","E1"),
    ["E2"  ] 		= f("130D3","E2"),
    ["E3"  ] 		= f("130D4","E3"),
    ["E4"  ] 		= f("130D5","E4"),
    ["E5"  ] 		= f("130D6","E5"),
    ["E6"  ]        = f("130D7","E6","zzmt"),
    ["E7"  ]        = f("130D8","E7"),
    ["E8"  ] 		= f("130D9","E8"),
    ["E8a" ] 	    = f("130DA","E8a"),
    ["E9"  ] 		= f("130DB","E9"),
    ["E9a" ] 		= f("130DC","E9a"),
    ["E10" ] 		= f("130DD","E10"),
	["E11" ] 		= f("130DE","E11"),
	["E12" ] 		= f("130DF","E12"),
	["E13" ] 		= f("130E0","E13"),
	["E14" ] 		= f("130E1","E14"),
	["E15" ] 		= f("130E2","E15"),
	["E16" ] 		= f("130E3","E16"),
	["E16a"] 		= f("130E4","E16a"),
	["E17" ] 		= f("130E5","E17","zAb"),
	["E17a"] 		= f("130E6","E17a",""),
	["E18" ] 		= f("130E7","E18",""),
	["E19" ] 		= f("130E8","E19",""),
	["E20" ] 		= f("130E9","E20"),
	["E20a"] 		= f("130EA","E20a"),
	["E21" ] 		= f("130EB","E21",""),
	["E22" ] 		= f("130EC","E22","mAi"),
	["E23" ] 		= f("130ED","E23","l"),
	["E24" ] 		= f("130EE","E24","Aby"),
	["E25" ] 		= f("130EF","E25",""),
	["E26" ] 		= f("130F0","E26",""),
	["E27" ] 		= f("130F1","E27",""),
	["E28" ] 		= f("130F2","E28",""),
	["E28a"] 		= f("130F3","E28a",""),
	["E29" ] 		= f("130F4","E29",""),
	["E30" ] 		= f("130F5","E30",""),
	["E31" ] 		= f("130F6","E31",""),
	["E32" ] 		= f("130F7","E32",""),
	["E33" ] 		= f("130F8","E33",""),
	["E34" ] 		= f("130F9","E34","wn"),
	["E34a"] 		= f("130FA","E34a",""),
	["E35" ] 		= f("0000","E35","unused"), -- see http://www.rostau.org.uk/aegyptian-l/archives/week617.txt
	["E36" ] 		= f("130FB","E36",""),
	["E37" ] 		= f("130FC","E37",""),
	["E38" ] 		= f("130FD","E38",""),

-- F series

	["F1"  ] 		= f("130FF","F1",""),
	["F2"  ] 		= f("13100","F2",""),
	["F3"  ] 		= f("13101","F3",""),
	["F4"  ] 		= f(13102,"F4","HAt"),
	["F5"  ] 		= f(13103,"F5","SsA"),
	["F6"  ] 		= f(13104,"F6",""),
	["F7"  ] 		= f(13105,"F7",""),
	["F8"  ] 		= f(13106,"F8",""),
	["F9"  ] 		= f(13107,"F9",""),
	["F10" ] 		= f(13108,"F10",""),
	["F11" ] 		=  f(13109,"F11",""),
	["F12" ] 		=  f("1310A","F12","wsr"),
	["F13" ] 		=  f("1310B","F13","wp"),
	["F13a"] 		=  f("1310C","F13a",""),
	["F14" ] 		=  f("1310D","F14",""),
	["F15" ] 		=  f("1310E","F15",""),
	["F16" ] 		=  f("1310F","F16","db"),
	["F17" ] 		=  f("13110","F17",""),
	["F18" ] 		= f(13111,"F18","Hw"),
	["F19" ] 		= f(13112,"F19",""),
	["F20" ] 		= f(13113,"F20","ns"),
	["F21" ] 		= f(13114,"F21","sDm"),
	["F21a"] 		= f(13115,"F21a",""),
	["F22" ] 		= f(13116,"F22","pH"),
	["F23" ]       = f(13117,"F23","xpS"),
	["F24" ]       = f(13118,"F24","xpS"),
	["F25" ]       = f(13119,"F25","wHm"),
	["F26" ]       = f("1311A","F26","Xn"),
	["F27" ]       = f("1311B","F27",""),
	["F28" ]       = f("1311C","F28",""),
	["F29" ]       = f("1311D","F29","sti"),
	["F30" ]       = f("1311E","F30","Sd"),
	["F31" ]       = f("1311F","F31","ms"),
    ["F31a"]       = f("13120","F31a",""),
    ["F32" ]       = f(13121,"F32","X"),
	["F33" ]       = f(13122,"F33","sd"),
	["F34" ]       = f(13123,"F34","ib"),
	["F35" ]       =  f(13124,"F35","nfr"),
	["F36" ]       = f(13125,"F36","zmA"),
	["F37" ]       = f(13126,"F37",""),
	["F37a"]       = f(13127,"F37a",""),
	["F38" ]       = f(13128,"F38",""),
	["F38a"]       = f(13129,"F38",""),
	["F39" ]       = f("1312A","F39","imAx"),
	["F40" ]       = f("1312B","F40","Aw"),
	["F41" ]       = f("1312C","F41","Aw"),
	["F42" ]       = f("1312D","F42","spr"),
	["F43" ]       = f("1312E","F43","Aw"),
	["F44" ]       = f("1312F","F44","isw"),
	["F45" ]       = f("13130","F45",""),
	["F45a"]       = f("13131","F45a",""),
	["F46" ]       = f(13132,"F46","pXr"),
	["F46a"]       = f(13133,"F46a",""),
	["F47" ]       = f(13134,"F47","qAb"),
	["F47a"]       = f(13135,"F47a",""),
	["F48" ]       = f(13136,"F48",""),
	["F49" ]       = f(13137,"F49",""),
	["F50" ]       = f(13138,"F50",""),
	["F51" ]       = f(13139,"F51",""),
	["F51a"]       = f("1313A","F51a",""),
	["F51b"]       = f("1313B","F51b",""),
	["F51c"]       = f("1313C","F51c",""),
	["F52" ]       = f("1313D","F52",""),
	["F53" ]       = f("1313E","F53",""),
	["G1"  ]       = f("1313F","G1","A"),
	["G2"  ]       = f("13140","G2","AA"),
	["G3"  ]       = f("13141","G3",""),
	["G4"  ]       = f(13142,"G4","tyw"),
	["G5"  ]       = f("13143","G5",""),
	["G6"  ]       = f("13144","G6",""),
	["G6a" ]       = f("13145","G6a",""),
	["G7"  ]       = f("13146","G7",""),
	["G7a" ]       = f("13147","G7a",""),
	["G7b" ]       = f("13148","G7b",""),
	["G8"  ]       = f("13149","G8",""),
	["G9"  ]       = f("1314A","G9",""),
	["G10" ]       = f("1314B","G10",""),
	["G11" ]       = f("1314C","G11",""),
	["G11a"]       = f("1314D","G11a",""),
	["G12" ]       = f("1314E","G12",""),
	["G13" ]       = f("1314F","G13",""),
	["G14" ]       = f("13150","G14","mwt"),
	["G15" ]       = f("13151","G15",""),
	["G16" ]       = f(13152,"G16","nbty"),
	["G17" ]       = f(13153,"G17","m"),
	["G18" ]       = f(13154,"G18","mm"),
	["G19" ]       = f("13155","G19","mwt"),
	["G20" ]       = f("13156","G20","mwt"),
	["G20a"]       = f("13157","G20a","mwt"),
	["G21" ]       = f(13158,"G21","nH"),
	["G22" ]       = f(13159,"G22","Db"),
	["G23" ]       = f("1315A","G23","rxyt"),
	["G24" ]       = f("1315B","G24",""),
	["G25" ]       = f("1315C","G25","Ax"),
	["G26" ]       = f("1315D","G26",""),
	["G26a"]       = f("1315E","G26a",""),
	["G27" ]       = f("1315F","G27","dSr"),	
	["G28" ]       = f("13160","G28","gm"),
	["G29" ]       = f("13161","G29","bA"),
	["G30" ]       = f(13162,"G30","baHi"),
	["G31" ]       = f(13163,"G31",""),
	["G32" ]       = f(13164,"G32","baHi"),
	["G33" ]       = f(13165,"G33",""),
	["G34" ]       = f(13166,"G34","baHi"),
	["G35" ]       = f(13167,"G35","aq"),
	["G36" ]       = f(13168,"G36","wr"),
	["G36a"]       = f(13169,"G36a",""),
	["G37" ]       = f("1316A","G37",""),
	["G37a"]       = f("1316B","G37a",""),
	["G38" ]       = f("1316C","G38","gb"),
	["G39" ]       = f("1316D","G39","zA"),
	["G40" ]       = f("1316E","G40","pA"),
	["G41" ]       = f("1316F","G41","xn"),
	["G42" ]       = f("13170","G42","wSA"),
	["G43" ]       = f(13171,"G43","w"),
	["G43a"]       = f(13172,"G43a","w"),
	["G44" ]       = f(13173,"G44","ww"),
	["G45" ]       = f(13174,"G45",""),
	["G45a"]       = f(13175,"G45a",""),
	["G46" ]       = f(13176,"G46","mAw"),
	["G47" ]       = f(13177,"G47","TA"),
	["G48" ]       = f(13178,"G48",""),
	["G49" ]       = f(13179,"G49",""),
	["G50" ]       = f("1317A","G50",""),
	["G51" ]       = f("1317B","G51",""),
	["G52" ]       = f("1317C","G52",""),
	["G53" ]       = f("1317D","G53",""),
	["G54" ]       = f("1317E","G54","snD"),
	["H1"  ]       = f("1317F","H1",""),
	["H2"  ]       = f(13180,"H2","pq"),
	["H3"  ]       = f(13181,"H3","pAq"),
	["H4"  ]       = f(13182,"H4","nr"),
	["H5"  ]       = f(13183,"H5",""),
	["H6"  ]       = f(13184,"H6","Sw"),


	["I1"  ]       = f(13188,"I1","aSA"),
	["I2"  ]       = f(13189,"I2","Styw"),
	["I3"  ]       = f("1318A","I3","mzH"),
	["I4"  ]       = f("1318B","I4","sbk"),
	["I5"  ]       = f("1318C","I5","sAq"),
	["I5a" ]       = f("1318D","I5a",""),

	["I6"  ]       = f("1318E","I6","km"),
	["I7"  ]       = f("1318F","I7",""),
	["I8"  ]       = f(13190,"I8","Hfn"),
	["I9"  ]       = f(13191,"I9","f"),
	["I9a" ]       = f(13192,"I9a",""),

	["I10" ]       = f(13193,"I10","D"),
	["I10a"]       = f(13194,"I10a",""),

	["I11" ]       = f(13195,"I11","DD"),
	["I11a"]       = f(13196,"I11a",""),

	["I12" ]       = f(13197,"I12",""),
	["I13" ]       = f(13198,"I13",""),
	["I14" ]       = f(13199,"I14",""),
	["I15" ]       = f("1319A","I15",""),




	["K1"  ]       = f("1319B","K1","in"),
	["K2"  ]       = f("1319C","K2","in"),
	["K3"  ]       = f("1319D","K3","ad"),
	["K4"  ]       = f("1319E","K4","XA"),
	["K5"  ]       = f("1319F","K5","bz"),
	["K6"  ]       = f("131A0","K6","nSmt"),
	["K7"  ]       = f("131A1","K7",""),
	["K8"  ]       = f("131A2","K8",""),

-- Insects

	["L1"  ]       = f("131A3","xpr"),
	["L2"  ]       = f("131A4","L2","bit"),
	["L2a" ]       = f("131A5","L2a","bit"),
	["L3"  ]       = f("131A6","L3","bit"),
	["L4"  ]       = f("131A7","L4",""),
	["L5"  ]       = f("131A8","L5",""),
	["L6"  ]       = f("131A9","L6",""),
	["L6a" ]       = f("131AA","L6a",""),
	["L7"  ]       = f("131AB","L7","srqt"),
	["L8"  ]       = f("131AC","L8",""),

-- 
	["M1"  ]       = f("131AD","M1","iAm"),
	["M1a" ]       = f("131AE","M1a",""),
	["M1b" ]       = f("131AF","M1b",""),
	["M2"  ]       = f("131B0","M2","Hn"),
	["M3"  ]       = f("131B1","M3","xt"),
	["M3a" ]       = f("131B2","M3a",""),
	["M4"  ]       = f("131B3","M4","rnp"),
	["M5"  ]       = f("131B4","M5",""),
	["M6"  ]       = f("131B5","M6","tr"),
	["M7"  ]       = f("131B6","M7",""),
	["M8"  ]       = f("131B7","M8","SA"),
	["M9"  ]       = f("131B8","M9","zSn"),
	["M10" ]       = f("131B9","M10",""),
	["M10a"]       = f("131BA","M10a",""),


	["M11" ]       = f("131BB","M11","wdn"),
	["M12" ]       = f("131BC","M12","xA"),
	["M12a"]       = f("131BD","M12a",""),
	["M12b"]       = f("131BE","M12b",""),
	["M12c"]       = f("131BF","M12c",""),

	["M12d"]       = f("131C0","M12d",""),
	["M12e"]       = f("131C1","M12e",""),
	["M12f"]       = f("131C2","M12f",""),
	["M12g"]       = f("131C3","M12g",""),
	["M12h"]       = f("131C4","M12h",""),


	["M13" ]       = f("131C5","M13","wAD"),
	["M14" ]       = f("131C6","M14",""),


	["M15" ] = f("131C7","M15",""),
	["M15a"] = f("131C8","M15a",""),
	["M16" ] = f("131C9","HA"),
	["M16a"] = f("131CA","M16a",""),


	["M17" ] = f("131CB","M17","i"),
	["M17a"] = f("131CC","M17a",""),

	["M18" ] = f("131CD","M18","ii"),
	["M19" ] = f("131CE","M19",""),

	["M20" ] = f("131CF","M20","sxt"),
	["M21" ] = f("131D0","M21","sm"),
	["M22" ] = f("131D1","M22",""),
	["M22a"] = f("131D2","M22a",""),

	["M23" ] = f("131D3","M23","sw"),
	["M24" ]  = f("131D4","M24","rsw"),
	["M24a"] = f("131D5","M24a","M24a",""),

	["M25" ] = f("131D6","M25",""),

	["M26" ] = f("131D7","M26","Sma"),
	["M27" ] = f("131D8","M27",""),
	["M28" ] = f("131D9","M28",""),
	["M28a"] = f("131DA","M28a",""),

	["M29" ] = f("131DB","M29","nDm"),
	["M30" ] = f("131DC","M30","bnr"),
	["M31" ] = f("131DD","M31",""),
	["M31a"] = f("131DE","M31a",""),

	["M32" ] = f("131EF","M32",""),
	["M33" ] = f("131E0","M33",""),
	["M33a"] = f("131E1","M33a",""),
    ["M33b"] = f("131E2","M33b",""),
	["M34" ] = f("131E3","M34","bdt"),
	["M35" ] = f("131E4","M35",""),

	["M36" ] = f("131E5","M36","Dr"),
	["M37" ] = f("131E6","M37",""),
	["M38" ] = f("131E7","M38",""),
	["M39" ] = f("131E8","M39",""),


	["M40" ] = f("131E9","M40","iz"),
	["M40a"] = f("131EA","M40a",""),
	["M41" ] = f("131EB","M41",""),
	["M42" ] = f("131EC","M42",""),
	["M43" ] = f("131ED","M43",""),
	["M44" ] = f("131EE","M44",""),
--
	["N1"  ] = f("131EF","N1","pt"),
	["N2"  ] = f("131F0","N2",""),
	["N3"  ] = f("131F1","N3",""),
	["N4"  ] = f("131F2","N4","iAdt,idt"),

	["N5"  ] = f("131F3","ra,zw"),
	["N6"  ] = f("131F4","N6",""),
	["N7"  ] = f("131F5","N7",""),

	["N8"  ] = f("131F6","N8","Hnmmt"),
	["N9"  ] = f("131F7","N9","pzD"),
	["N10" ] = f("131F8","N10",""),



	["N11" ] = f("131F9","N11","Abd,iaH"),
	["N12" ] = f("131FA","N12","iaH"),
	["N13" ] = f("131FB","N13",""),

	["N14" ] = f("131FC","N14","sbA.dwA"),

	["N15" ] = f("131FD","N15","dwAt"),
	["N16" ] = f("131FE","N16","tA"),
	["N17" ] = f("131FF","N17",""),

	["N18" ] = f("13200","N18","iw"),
	["N18a"] = f("13201","N18a",""),
	["N18b"] = f("13202","N18b",""),

	["N19" ] = f(13203,"N19",""),

	["N20" ] = f(13204,"wDb"),
	["N21" ] = f(13205,"N21",""),
	["N22" ] = f(13206,"N22",""),
	["N23" ] = f(13207,"N23",""),

	["N24" ] = f(13208,"spAt"),
	["N25" ] = f(13209,"xAst"),
	["N25a"] = f("1320A","N25a",""),

	["N26" ] = f("1320B","N26","Dw"),
	["N27" ] = f("1320C","N27","Axt"),
	["N28" ] = f("1320D","N28","xa"),
	["N29" ] = f("1320E","N29","q"),
	["N30" ] = f("1320F","N30","iAt"),

	["N31" ] = f("13210","N31",""),
	["N32" ] = f("13211","N32",""),
	["N33" ] = f("13212","N33",""),
	["N33a"] = f("13213","N33a",""),
	["N34" ] = f(13214,"N34",""),
	["N34a"] = f(13215,"N34a",""),


	["N35" ] = f(13216,"N35","n"),
	["N35a"] = f(13217,"N35a","mw"),
	["N36" ] = f(13218,"N36",""),

	["N37" ] = f("13219","N37","S"),
	["N38" ] = f("1321A","N38",""),
	["N39" ] = f("1321B","N39",""),
	["N40" ] = f("1321C","N40","Sm",""),
	["N41" ] = f("1321D","N41","id"),
	["N42" ] = f("1321E","N42",""),


	["O1"  ] = f("13250","O1","pr"),
	["O1a" ] = f("13251","O1a","pr"),
	["O2"  ] = f("13252","O2",""),
	["O3"  ] = f("13253","O3",""),
	["O4"  ] = f(13254,"O4","h"),
	["O5"  ] = f(13255,"O5","h"),
	["O5a" ] = f(13256,"O5a","h"),

	["O6"  ] = f(13257,"O6","Hwt"),
	["O6a" ] = f(13258,"O6a",""),
	["O6b" ] = f(13259,"O6b",""),
	["O6c" ] = f("1325A","O6c",""),
	["O6d" ] = f("1325B","O6d",""),
	["O6e" ] = f("1325C","O6e",""),
	["O6f" ] = f("1325D","O6f",""),

	["O7"  ] = f("1325E","O7","h"),
	["O8"  ] = f("1325F","O8","h"),
	["O9"  ] = f(13260,"O9",""),
	["O10" ] = f(13261,"O10",""),
	["O10a"] = f(13262,"O10a",""),
	["O10b"] = f(13262,"O10b",""),
	["O10c"] = f(13262,"O10c",""),

	["O11" ] = f(13265,"aH"),
	["O12" ] = f(13266,"O12",""),
	["O13" ] = f(13267,"O13",""),
	["O14" ] = f(13268,"O14",""),


	["O15" ] = f(13269,"wsxt"),
	["O16" ] = f("1326A","O16",""),
	["O17" ] = f("1326B","O17",""),

	["O18" ] = f("1326C","kAr"),
	["O19" ] = f("1326D",""),
	["O19a"] = f("1326E",""),

	["O20" ] = f("1326F","O20",""),
	["O20a"] = f("13270","O20a",""),
	["O21" ] = f("13271","O21",""),
	["O22" ] = f("13272","O22","zH"),
	["O23" ] = f("13273","O23",""),
	["O24" ] = f("13274","O24",""),
	["O24a"] = f("13275","O24a",""),

	["O25" ] = f(13276,"O25","txn"),
	["O25a"] = f("13277","O25a",""),

	["O26" ] = f(13278,"O26",""),
	["O27" ] = f(13279,"O27","txn"),
	["O28" ] = f("1327A","O28","iwn"),
	["O29" ] = f("1327B","O29","aA"),
	["O29a"] = f("1327C","O29a"),

	["O30" ] = f("1327D","O30","zxnt"),
	["O30a"] = f("1327E","O30a",""),

	["O31" ] = f("1327F","O31"),
	["O32" ] = f("13280","O32"),
	["O33" ] = f("13281","O33"),
	["O33a"] = f("13282","O33a"),

	["O34" ] = f(13283,"O34","z"),
	["O35" ] = f(13284,"O35","zb"),
	["O36" ] = f(13285,"O36","inb"),
	["O36a"] = f(13286,"O36a",""),
	["O36b"] = f(13287,"O36b",""),
	["O36c"] = f(13288,"O36c","inb"),
	["O36d"] = f(13289,"O36d",""),
	["O37" ] = f("1328A","O37",""),
	["O38" ] = f("1328B","O38",""),
	["O39" ] = f("1328C","O39",""),
	["O40" ] = f("1328D","O40",""),
	["O41" ] = f("1328E","O41",""),
	["O42" ] = f("1329F","O42","Szp"),
	["O43" ] = f(13290,"O43",""),
	["O44" ] = f(13291,"O44","Szp"),
	["O45" ] = f(13292,"O45","ipt"),
	["O46" ] = f(13293,"O46",""),
	["O47" ] = f(13294,"O47","nxn"),
	["O48" ] = f(13295,"O42","Szp"),
	["O49" ] = f(13296,"O49","niwt"),
	["O50" ] = f(13297,"O50","zp"),
	["O50a"] = f(13298,"O50a","Snwt"),
	["O50b"] = f(13299,"O50b",""),
	["O51" ] = f("1329A","O51",""),

	["P1"  ] = f("1329B","P1",""),
	["P1a" ] = f("1329C","P1a",""),
	["P2"  ] = f("1329D","P2",""),
	["P3"  ] = f("1329E","P3",""),
	["P3a" ] = f("1329F","P3a",""),
	["P4"  ] = f("132A0","P4","wHa"),
	["P5"  ] = f("132A1","P5","nfw"),
	["P6"  ] = f("132A2","P6","TAw"),
	["P7"  ] = f("132A3","P7","aHa"),
	["P8"  ] = f("132A4","P8","xrw"),
	["P9"  ] = f("132A5","P9",""),
	["P10" ] = f("132A6","P10",""),
	["P11" ] = f("132A7","P11",""),


	["Q1"  ] = f("132A8","Q1","st"),
	["Q2"  ] = f("132A9","Q2","wz"),
	["Q3"  ] = f("132AA","Q3","p"),
	["Q4"  ] = f("132AB","Q4",""),
	["Q5"  ] = f("132AC","Q5",""),
	["Q6"  ] = f("132AD","Q6","qrsw"),
	["Q7"  ] = f("132AE","Q7",""),


	["R1"  ] = f("132AF","R1","xAwt"),
	["R2"  ] = f("132B0","R2",""),
	["R2a" ] = f("132B1","R2a",""),
	["R3"  ] = f("132B2","R3",""),
	["R3a" ] = f("132B3","R3a",""),
	["R3b" ] = f("132B4","R3b",""),
	["R4"  ] = f("132B5","R4","Htp,kAp"),
	["R5"  ] = f("132B6","R5","kAp"),
	["R6"  ] = f("132B7","R6",""),
	["R7"  ] = f("132B8","R7","snTr"),
	["R8"  ] = f("132B9","R8","nTr"),
	["R9"  ] = f("132BA","R9","bd"),
	["R10" ] = f("132BB","R10",""),
	["R10a"] = f("132BC","R10a",""),
	["R11" ] = f("132BD","R11","dd,Dd"),
	["R12" ] = f("132BE","R12",""),
	["R13" ] = f("132BF","R13",""),
	["R14" ] = f("132C0","R14","imnt"),
	["R15" ] = f("132C1","R15","iAb"),
	["R16" ] = f("132C2","R16","wx"),
	["R16a"] = f("132C3","R16a",""),
	["R17" ] = f("132C4","R17",""),
	["R18" ] = f("132C5","R18",""),
	["R19" ] = f("132C6","R19",""),
	["R20" ] = f("132C7","R20",""),
	["R21" ] = f("132C8","R21",""),
	["R22" ] = f("132C9","R22","xm"),
	["R23" ] = f("132CA","R23",""),
	["R24" ] = f("132CB","R24",""),
	["R25" ] = f("132CC","R25",""),
	["R26" ] = f("132CD","R26",""),
	["R27" ] = f("132CE","R27",""),
	["R28" ] = f("132CF","R28",""),
	["R29" ] = f("132D0","R29",""),


	["S1"  ] = f("132D1","S1","HDt"),
	["S2"  ] = f("132D2","S2",""),
	["S2a" ] = f("132D3","S2a","HDt"),
	["S3"  ] = f("132D4","S3","dSrt,N"),
	["S4"  ] = f("132D5","S4","HDt"),
	["S5"  ] = f("132D6","S5","HDt"),
	["S6"  ] = f("132D7","S6","sxmty"),
	["S6a" ] = f("132D8","S6a",""),


	["S7"  ] = f("132D9","S7","xprS"),
	["S8"  ] = f("132DA","S8","Atf"),
	["S9"  ] = f("132DB","S9","Swty"),
	["S10" ] = f("132DC","S10","mDH"),
	["S11" ] = f("132DD","S11","wsx"),
	["S12" ] = f("132DE","S12","nbw"),
	["S13" ] = f("132DF","S13","sxmty"),
	["S14" ] = f("132E0","S14",""),
	["S14a"] = f("132E1","S14a",""),
	["S14b"] = f("132E2","S14b",""),
	["S15" ] = f("132E3","S15","tHn"),
	["S16" ] = f("132E4","S16",""),
	["S17" ] = f("132E5","S17",""),
	["S17a"] = f("132E6","S17a",""),


	["S18" ] = f("132E7","S18","mnit"),
	["S19" ] = f("132E8","S19","sDAw"),
	["S20" ] = f("132E9","S20","xtm"),
	["S21" ] = f("132EA","S21",""),

	["S22" ] = f("132EB","S22","sT"),
	["S23" ] = f("132EC","S23","dmD"),
	["S24" ] = f("132ED","S24","Tz"),
	["S25" ] = f("132EE","S25","Tz"),

	["S26" ] = f("132EF","S26","Sndyt"),
	["S26a"] = f("132F0","S26a",""),
	["S26b"] = f("132F1","S26b",""),

	["S27" ] = f("132F2","S27","mnxt"),
	["S28" ] = f("132F3","S28","Tz"),
	["S29" ] = f("132F4","S29","s"),

	["S30" ] = f("132F5","S30","sf"),
	["S31" ] = f("132F6","S31",""),


	["S32" ] = f("132F7","S32",""),
	["S33" ] = f("132F8","S33",""),
	["S34" ] = f("132F9","S34",""),


	["S35" ] = f("132FA","S35","Swt"),
	["S35a"] = f("132FB","S35a",""),
	["S36" ] = f("132FC","S36",""),
	["S37" ] = f("132FD","S37","xw"),
	["S38" ] = f("132FE","S38","HqA"),
	["S39" ] = f("132FF","S39","awt"),
	["S40" ] = f(13300,"S40","wAs"),
	["S41" ] = f(13301,"S41","Dam"),
	["S42" ] = f(13302,"S42","abA"),
	["S43" ] = f(13303,"S43","xrp"),
	["S44" ] = f(13304,"S44","sxm"),
	["S45" ] = f(13305,"S45","nxxw"),
	["S46" ] = f("13306","S46",""),

	["T1"  ] = f("13307","T1",""),
	["T2"  ] = f("13308","T2",""),
	["T3"  ] = f(13309,"T3","HD"),
	["T3a" ] = f("1330A","T3a",""),
	["T4"  ] = f("1330B","T4",""),
	["T5"  ] = f("1330C","T5",""),

	["T6"  ] = f("1330D","T6","HDD"),
	["T7"  ] = f("1330E","T7",""),
	["T7a" ] = f("1330F","T7a",""),
	["T8"  ] = f("13310","T8",""),
	["T8a" ] = f("13311","T8a",""),
	["T9"  ] = f(13312,"T9","pd"),

	["T10" ] = f(13314,"T10","pD"),
	["T11" ] = f(13315,"T11","zin,zwn,sXr"),
    ["T11a"] = f("13316","T11a",""),
	["T12" ] = f(13317,"T12","Ai,Ar,rwd,ewD"),
	["T13" ] = f(13318,"T13","rs"),

	["T14" ] = f(13319,"qmA"),
    ["T15" ] = f("1331A","T15",""),
    ["T16" ] = f("1331B","T16",""),
    ["T16a"] = f("1331C","T16a",""),
	["T17" ] = f("1331D","T17","wrrt"),
	["T18" ] = f("1331E","T18","Sms"),
	["T19" ] = f("1331F","T19","qs"),
    ["T20" ] = f("13320","T20",""),
    ["T21" ] = f("13321","T21",""),
	["T22" ] = f(13322,"T22","sn"),
    ["T23" ] = f("13323","T23",""),
	["T24" ] = f(13324,"T24","iH"),
	["T25" ] = f(13325,"T25","DbA"),
    ["T26" ] = f("13326","T26","T2",""),
    ["T27" ] = f("13327","T27",""),
	["T28" ] = f(13328,"T28","Xr"),
	["T29" ] = f(13329,"T29","nmt"),
    ["T30" ] = f("1333A","T30","T2",""),
	["T31" ] = f("1333B","T31","sSm"),
    ["T32" ] = f("1333C","T32",""),
    ["T32a"] = f("1333D","T32a",""),
    ["T33" ] = f("1333E","T33",""),
    ["T33a"] = f("1333F","T33a",""),
	["T34" ] = f("13330","nm"),
    ["T35" ] = f("13331","T35",""),
    ["T36" ] = f("13332","T36",""),

	["U1"  ] = f("13333","U1","mA"),
	["U2"  ] = f("13334","U2",""),
	["U3"  ] = f("13335","U3",""),
	["U4"  ] = f("13336","U4",""),
	["U5"  ] = f("13337","U5",""),


	["U6"  ] = f(13338,"mr"),
	["U6a" ] = f("13339","U6a",""),
	["U6b" ] = f("1333A","U6b",""),
	["U7"  ] = f("1333B","U7",""),
	["U8"  ] = f("1333C","U8",""),
	["U9"  ] = f("1333D","U9",""),

	["U10" ] = f("1333E","it"),

	["U11" ] = f("1333F","U11","HqAt"),
	["U12" ] = f("13340","U12",""),
	["U13" ] = f(13341,"U13","hb,Sna"),
	["U14" ] = f(13342,"U14",""),

	["U15" ] = f(13343,"U15","tm"),
	["U16" ] = f(13344,"U16","biA"),
	["U17" ] = f(13345,"U17","grg"),
	["U18" ] = f(13346,"U18",""),
	["U19" ] = f(13347,"U19",""),
	["U20" ] = f(13348,"U20",""),
	["U21" ] = f(13349,"U21","stp"),

	["U22" ] = f("1334A","U22","mnx"),

	["U23" ] = f("1334B","U23","Ab"),
	["U23a"] = f("1334C","U23a",""),
	["U24" ] = f("1334D","U24","Hmt"),
	["U25" ] = f("1334E","U25",""),

	["U26" ] = f("1334F","U26","wbA"),
	["U27" ] = f("13350","U27",""),
	["U28" ] = f("13351","U28","DA"),
	["U29" ] = f("13352","U29",""),
	["U29a"] = f("13353","U29a",""),
	["U30" ] = f("13354","U30",""),
	["U31" ] = f("13355","U31","rtH"),
	["U32" ] = f(13356,"U32","zmn"),
	["U32a"] = f("13357","U32a",""),
	["U33" ] = f("13358","U33","ti"),
	["U34" ] = f(13359,"U34","xsf"),
	["U35" ] = f("1335A","U35",""),
	["U36" ] = f("1335B","U36","Hm"),
	["U37" ] = f("1335C","U37",""),
	["U38" ] = f("1335D","U38","mxAt"),
	["U39" ] = f("1335E","U39",""),
	["U40" ] = f("1335F","U40",""),
	["U41" ] = f("13360","U41",""),
	["U42" ] 	= f("13361","U42",""),
---
-- Gardiner symbols V mapped to Unicode
-- and MdC
---   
	["V1"  ] 	= f(13362,"V1","100"),
	["V1a" ] 	= f(13363,"V1a",""),
    ["V1b" ] 	= f(13364,"V1b",""),
    ["V1c" ]	= f(13365,"V1c",""),
  	["V1d" ]	= f(13366,"V1d",""),
  	["V1e" ] 	= f(13367,"V1e",""),
  	["V1f" ] 	= f(13368,"V1f",""),
  	["V1g" ] 	= f(13369,"V1g",""),
  	["V1h" ] 	= f("1336A","V1h",""),
  	["V1i" ] 	= f("1336B","V1i",""),
	["V2"  ] 	= f("1336C","V2","sTA"),
	["V2a" ] 	= f("1336D","V2a",""),
	["V3"  ] 	= f("1336E","V3","sTAw"),
	["V4"  ] 	= f("1336F","V4","wA"),
	["V5"  ] 	= f("13370","V5","snT"),
	["V6"  ] 	= f(13371,"V6","Ss"),
	["V7"  ] 	= f(13372,"V7","Sn"),
	["V7a" ] 	= f(13373,"V7a",""),
	["V7b" ] 	= f(13374,"V7b",""),
	["V8"  ] 	= f(13375,"V8",""),
	["V9"  ] 	= f(13376,"V9",""),
	["V10" ] 	= f(13377,"V10",""),
	["V11" ] 	= f(13378,"V11",""),
	["V11a"] 	= f(13379,"V11a",""),
	["V11b"] 	= f("1337A","V11b",""),
	["V11c"] 	= f("1337B","V11c",""),
	["V12" ] 	= f("1337C","V12","arq"),	
	["V12a"] 	= f("1337D","V12a",""),
	["V12b"] 	= f("1337E","V12b",""),
	["V13" ] 	= f("1337F","V13","T"),
	["V14" ] 	= f("13380","V14","T"),
	["V15" ] 	= f("13381","V15","iTi"),
	["V16" ] 	= f("13382","V16",""),
	["V17" ] 	= f("13383","V17",""),
	["V18" ] 	= f("13384","V18",""),
	["V19" ] 	= f("13385","V19","mDt,XAr,TmA"),
	["V20" ] 	= f(13386,"V20","10,mD"),
	["V20a"] 	= f(13387,"V20a",""),
	["V20b"]	= f(13388,"V20b",""),
	["V20c"] 	= f(13389,"V20c",""),
	["V20d"] 	= f("1338A","V20d",""),
	["V20e"] 	= f("1338B","V20e",""),
	["V20f"] 	= f("1338C","V20f",""),
	["V20g"] 	= f("1338D","V20g",""),
	["V20h"] 	= f("1338E","V20h",""),
	["V20i"] 	= f("1338F","V20i",""),
	["V20j"] 	= f("13390","V20j",""),
	["V20k"] 	= f("13391","V20k",""),
	["V20l"] 	= f("13392","V20l",""),
	["V21" ] 	= f("13393","V21",""),
	["V22" ]    = f(13394,"V22","mH"),
	["V23" ]    = f("13395","V23",""),
	["V23a"]    = f("13396","V23a",""),
	["V24" ]    = f(13397,"V24","wD"),
	["V25" ]    = f("13398","V25",""),
	["V26" ]    = f(13399,"V26","aD"),
	["V27" ]    = f("1339A","V27",""),
	["V28" ]    = f("1339B","V28","H"),
	["V28a"]    = f("1339C","V28a",""),
	["V29" ]    = f("1339D","V29","wAH,sk"),
	["V29a"]    = f("1339E","V29a","H"),
	["V30" ]    = f("1339F","V30","nb"),
	["V30a"]    = f("133A0","V30a","b"),
	["V31" ]    = f("133A1","V31","k"),
	["V31a"]    = f("133A2","V31a",""),
	["V32" ]    = f("133A3","V32","msn"),
	["V33" ]    = f("133A4","V33","sSr"),
	["V33a"]    = f("133A5","V33a",""),
	["V34" ]    = f("133A6","V34",""),
	["V35" ]    = f("133A7","V35",""),
	["V36" ]    = f("133A8","V36",""),
	["V37" ]    = f("133A9","V37","idr"),
	["V37a"]    = f("133AA","V37a",""),
	["V38" ]    = f("133AB","V38",""),
	["V39" ]    = f("133AC","V39",""),
	["V40" ]    = f("133AD","V40",""),
	["V40a"]    = f("133AE","V40a",""),
---
--  Series W Gardiner set
---

	["W1"  ] 	= f("133AF","W1",""),
	["W2"  ] 	= f("133B0","W1","bAs"),
	["W3"  ] 	= f("133B1","W3","Hb"),
	["W3a" ] 	= f("133B2","W3a","Hb"),
	["W4"  ] 	= f("133B3","W4",""),
	["W5"  ] 	= f("133B4","W5","Hb"),
	["W6"  ] 	= f("133B5","W6",""),
	["W7"  ] 	= f("133B6","W7",""),
	["W8"  ] 	= f("133B7","W8",""),
	["W9"  ] 	= f("133B8","W9","Xnm"),
	["W9a" ] 	= f("133B9","W9a",""),

	["W10" ] 	= f("133BA","W10","iab"),
	["W10a"] 	= f("133BB","W10a",""),

	["W11" ] 	= f("133BC","W11","g,nzt"),
	["W12" ] 	= f("133BD","W12",""),
	["W13" ] 	= f("133BE","W13",""),

	["W14" ] 	= f("133BF","W14","Hz"),
	["W14a"] 	= f("133C0","W14a",""),
	["W15" ] 	= f("133C1","W15",""),
	["W16" ] 	= f("133C2","W16",""),
	["W17" ] 	= f("133C3","W17","xnt"),
	["W17a"] 	= f("133C4","W17a",""),
	["W18" ] 	= f("133C5","W18",""),
	["W18a"] 	= f("133C6","W18a",""),

	["W19" ] 	= f("133C7","W19","mi"),
	["W20" ] 	= f("133C8","W20",""),
	["W21" ] 	= f("133C9","W21",""),
	["W22" ] 	= f("133CA","W22","Hnqt"),
	["W23" ] 	= f("133CB","W23",""),
	["W24" ] 	= f("133CC","W24","nw"),
	["W24a"] 	= f("133CD","W24a","nw"),
	["W25" ] 	= f("133CE","W25","ini"),


	["X1"  ]    = f("133CF","X1","t"),
 	["X2"  ]    = f("133D0","X2",""),
 	["X3"  ]    = f("133D1","X3",""),
 	["X4"  ]    = f("133D2","X4",""),
 	["X4a" ]    = f("133D3","X4a",""),
 	["X4b" ]    = f("133D4","X4b",""),
 	["X5"  ]    = f("133D5","X5",""),
 	["X6"  ]    = f("133D6","X6",""),
 	["X6a" ]    = f("133D7","X6a",""),
 	["X7"  ]    = f("133D8","X7",""),
	["X8"  ]    = f("133D9","rdi,di"),
 	["X8a" ]    = f("133DA","X8a",""),


	["Y1"  ]    = f("133DB","Y1","mDAt"),
	["Y1a" ]    = f("133DC","Y1a",""),
 	["Y2"  ]    = f("133DD","Y2",""),

	["Y3"  ]    = f("133DE","mnhd,zS"),
 	["Y4"  ]    = f("133DF","Y4",""),
	["Y5"  ]    = f("133E0","Y5","mn"),
	["Y6"  ]    = f("133E1","Y6","ibA"),
 	["Y7"  ]    = f("133E2","Y7",""),
	["Y8"  ]    = f("133E3","Y8","zSSt"),

 	["Z1"  ]    = f("133E4","Z1",""),
 	["Z2"  ]    = f("133E5","Z2",""),
 	["Z2a" ]    = f("133E6","Z2a",""),
 	["Z2b" ]    = f("133E7","Z2b",""),
	["Z2c" ]    = f("133E8","Z2c",""),
	["Z2d" ]    = f("133E9","Z2d",""),


 	["Z3"  ]    = f("133EA","Z3",""),
 	["Z3a" ]    = f("133EB","Z3a",""),
	["Z3b" ]    = f("133EC","Z3b",""),
	["Z4"  ]    = f("133ED","Z4","y"),
	["Z5"  ]    = f("133EE","Z4a","y"),
 	["Z5"  ]    = f("133EF","Z5",""),
 	["Z5a" ]    = f("133F0","Z5a",""),
 	["Z6"  ]    = f("133F1","Z6",""),
	["Z7"  ]    = f("133F2","Z7","W"),
 	["Z8"  ]    = f("133F3","Z8",""),
 	["Z9"  ]    = f("133F4","Z9",""),
 	["Z10" ]    = f("133F5","Z10",""),


	["Z11" ]    = f("133F6","Z11","imi"),
 	["Z12" ]    = f("133F7","Z12",""),
 	["Z13" ]    = f("133F8","Z13",""),
 	["Z14" ]    = f("133F9","Z14",""),
 	["Z15" ]    = f("133FA","Z15",""),
 	["Z15a"]    = f("133FB","Z15a",""),
 	["Z15b"]    = f("133FC","Z15b",""),
 	["Z15c"]    = f("133FD","Z15c",""),
 	["Z15d"]    = f("133FE","Z15d",""),
 	["Z15e"]    = f("133FF","Z15e",""),
 	["Z15f"]    = f("13400","Z15f",""),
 	["Z15g"]    = f("13401","Z15g",""),
 	["Z15h"]    = f("13402","Z15h",""),
	["Z15i"]    = f("13403","Z15i",""),
 	["Z16" ]    = f("13404","Z16",""),
 	["Z16a"]    = f("13405","Z16a",""),
 	["Z16b"]    = f("13406","Z16b",""),
 	["Z16c"]    = f("13407","Z16c",""),
 	["Z16d"]    = f("13408","Z16d",""),
 	["Z16e"]    = f("13409","Z16e",""),
 	["Z16f"]    = f("1340A","Z16f",""),
 	["Z16g"] 	= f("1340B","Z16g",""),
 	["Z16h"] 	= f("1340C","Z16h",""),
	["Aa1" ] 	= f("1340D","Aa1","x"),
   	["Aa2" ]    = f("1340E","Aa2","x"),
	["Aa3" ]    = f("1340F","Aa3","x"),
	["Aa4" ]    = f("13410","Aa4","x"),
	["Aa5" ]    = f("13411","Aa5","Hp"),
	["Aa6" ]    = f("13412","Aa6","x"),
	["Aa7" ]    = f("13413","Aa7","x"),
	["Aa7a"]    = f("13414","Aa7a","x"),
	["Aa7b"]    = f("13415","Aa7b",""),
	["Aa8" ]    = f("13416","Aa8","qn"),
	["Aa9" ]    = f("13417","Aa9","x"),
	["Aa10"]    = f("13418","Aa10","x"),
	["Aa11"]    = f("13419","A11","mAa"),
	["Aa12"]    = f("1341A","Aa12",""),
	["Aa13"]    = f("1341B","Aa13","im,gs,M"),
	["Aa14"]    = f("1341C","Aa14",""),
	["Aa15"]    = f("1341D","Aa15",""),
	["Aa16"]    = f("1341E","Aa16",""),
	["Aa17"]    = f("1341F","Aa17","sA"),
	["Aa18"]    = f("13420","Aa18",""),
	["Aa19"]    = f("13421","Aa19",""),
	["Aa20"]    = f("13422","Aa20","apr"),
	["Aa21"]    = f("13423","Aa21","wDa"),
	["Aa22"]    = f("13424","Aa22",""),
	["Aa23"]    = f("13425","Aa23",""),
	["Aa24"]    = f("13426","Aa24",""),
	["Aa25"]    = f("13427","Aa25",""),
	["Aa26"]    = f("13428","Aa26",""),
	["Aa27"]    = f("13429","Aa27","nD"),
	["Aa28"]    = f("1342A","Aa28","qd"),
	["Aa29"]    = f("1342B","Aa29",""),
      ["Aa30"]    = f("1342C","Aa30","Xkr"),
	["Aa31"]    = f("1342D","Aa31",""),
	["Aa32"]    = f("1342E","Aa32","")
}

-- Gardiner categories
-- we store in a table for convenience

local cat = {}

cat["A"]   =   {[[A1-A2-A3-A4-A5-A5a-A6-A6a-A6b-A7-A8-
                 A9-A10-A11-A12-A13-A14-A14a-A15-A16-
                 A17-A17a-A18-A19-A20-A21-A22-A23-A24-
                 A25-A26-A27-A28-A29-A30-A31-A32-A32a-
                 A33-A34-A35-A36-A37-A38-A39-A40-A40a-
                 A41-A42-A42a-A43-A43a-A44-A45-A45a-A46-
                 A47-A49-A50-A51-A52-A53-A54-A55-A56-
                 A57-A58-A59-A60-A61-A62-A63-A64-A65-
                 A66-A67-A68-A69-A70!]],
                 heading = "Man and his occupations"}

cat["B"]   =   {[[B1-B2-B3-B4-B5-B5a-B6-B7-B8!]], 
                 heading = "Woman and her occupations"}

cat["C"]   =   {[[C1-C2-C2a-C2b-C2c-C3-C4-C5-C6-C7-C8-C9-
                 C10-C10a 
                 C11-C12-C13-C14-C15-C16 C17-C18-C19-C20-
                 C21-C22  C23 - C24!]], 
                 heading = "Anthropomorphic Deities"}

cat["D"]   =   {[[D1-D2-D3-D4 D5 D6 D7-D8-D9-D10-D11-D12-
                 D13-D14-D15-D16-D17-D18-D19-D20-D21-D22-
                 D23-D24-D25-D26-D27-D28-D29-D30-D31-D31a-
                 D32-D33-D34-D35-D36-D37-D38-D39-D40-D41-
                 D42-D43-D44-D45-D46-D47-D48-D49-D50-D50a-
                 D50b-D50c-D50d-D50e-D50f-D50g-D50h-D50i-
                 D51-D52-D53-D54-D55-D56-D57-D58-D59-D60-
                 D61-D62-D63-D64-D65-D66-D67-D67a-D67b-D67c-
                 D67d-D67e-D67f-D67g-D67h!]],
                 heading = "parts of the human body parts"}


cat["E"]   =   {[[E1-E2-E3-E4-E5-E6-E7-E8-E8a-E9-E9a-E10-
                 E11-E12-E13-E14-E15-E16-E16a-E17-E17a-
                 E18-E19-E20-E20a-E22-E23-E24-E25-E26-
                 E27-E28-E28a-E29-E30-E31-E32-E33-E34-
                 E34a-E35-E36-E37-E38!]],
                 heading = "Mammals"}

cat["F"]   =   {[[F1-F2-F3-F4-F5-F6-F7-F8-F9-F10-F11-F12-
                  F13-F14-F15-F16-F17-F18-F19-F20-F21-F21a-
                  F22-F23-F24-F25-F25-F26-F27-F28-F29-F30-
                  F31-F31a-F32-F33-F34-F35-F36-F37-F37a-F38-
                  F38a-F39-F40-F41-F42-F43-F44-F45-F45a-F46-
                  F46a-F47-F47a-F48-F49-F50-F51-F51a-F51b-
                  F51c-F52-F53!]],
                  heading = "Parts of Mammals"}

cat["G"]  =   {[[G1-G2-G3-G4-G5-G6-G6a-G7-G7a-G8-G9-G10-
                 G11-G11a-G12-G13-G14-G15-G16-G17-G18-
                 G19-G20-G20a-G21-G22-G23-G24-G25-G26-
                 G26a-G27-G28-G29-G30-G31-G32-G33-G34-
                 G35-G36-G36a-G37-G37a-G38-G39-G40-G42-
                 G43-G43a-G44-G45-G45a-G46-G47-G48-G49-
                 G50-G51-G52-G53-G54!]],
                 heading = "Birds"}

cat["H"] 	 =   {"H1-H2-H3-H4-H5-H6-H6!",
                   heading = "Parts of Birds"}

cat["I"] 	 =   {[[I1-I2-I3-I4-I5-I5a-I6-I7-I8-I9-
                   I9a-I10-I10a-I11-I12-I13-I14-I15]],
                   heading = "Amphibious Animals, Reptiles etc."}

cat["K"] 	 =   {"K1-K2-K3-K4-K5-K6-K7-K8",
                   heading = "Fish and parts of fish"}

cat["L"] 	 =   {"L1-L2-L2a-L3-L4-L5-L6-L6a-L7-L8",
                   heading = "Invertrbrates and lesser animals"}

cat["M"] 	 =   {[[M1-M1a-M1b-M2-M3-M3a-M4-M5-M6-M7-M8-
                 M9-M10-M10a-M11-M12-M12a-M12b-M12c-
                 M12d-M12e-M12f-M12g-M12h-M13-M14-M15-
                 M15a-M16-M16a-M17-M18-M19-M20-M21-M22-
                 M22a-M23-M24-M25-M26-M27-M28-M29-M30-
                 M31-M32-M33-M34-M35-M36-M37-M38-M39-
                 M40-M41-M42-M43-M44!]],
                 heading = "Trees and plants"}

cat["N"] =     {[[N1-N2-N3-N4-N5-N6-N7-N8-N9-N10-N11-N12-
                 N13-N14-N15-N16-N17-N18-N18a-N18b-N19-
				    N20-N21-N22-N23-N24-N25-N26-N27-N28-N29-
                 N30-N31-N32-N33-N34-N34a-N35-N36-N37-N38-
                 N39-N40-N41-N42!]], 
                 heading = "Sky, earth, water"}

cat["O"] = 	  {[[O1-O1a-O2-O3-O4-O5-O6-O6a-O6b-O6c-O6d-O6e-
                  O6f-O7-O8-O9-O10-O10a-O10b-O10c-O11-O12-
                  O13-O14-O15-O16-O17-O18-O19-O19a-O20-O20a-
                  O21-O22-O23-O24-O24a-O25-O25a-O26-O27-O28-
                  O29-O29a-O30-O30a-O31-O32-O33-O34-O35-O36-
                  O36a-O36b-O36c-O36d-O37-O38-O39-O40-O41-
                  O42-O43-O44-O45-O46-O47-O48-O49-O50-O50a-
                  O50b-O51-O51!]],
                  heading = "Buildings, parts of buildings"}

cat["P"]   =   {"P1-P1a-P2-P3-P3a-P4-P5-P6-P7-P8-P9-P10-P11",
                heading = "Ships and parts of ships"}

cat["Q"]   =   {"Q1-Q2-Q3-Q4-Q5-Q6-Q7!",
                heading = "Domestic and funerary furniture"}

cat["R"]	 =   {[[R1-R2-R2a-R3-R3a-R3b-R4-R5-R6-R7-R8-R9-
                  R10-R11-R12-R13-R14-R15-R16-R16a-R17-
                  R18-R19-R20-R21-R22-R23-R24-R25-R26-R27-
                  R28-R29]],
                  heading = "Temple furniture and sacred emblems"}

cat["S"]   =   {[[S1-S2-S2a-S3-S4-S5-S6-S6a-S7-S8-S9-
                 S10-S11-S12-S13-S14-S14a-S14b-S15-
                 S16-S17-S17a-S18-S19-S20-S21-S22-
                 S23-S24-S25-S26-S26a-S26b-S27-S28-
                 S29-S30-S31-S32-S33-S34-S35-S35a-S36-
                 S37-S38-S39-S40-S41-S42-S43-S44-S45-S46]],
                heading = "Crowns, dress, staves."}

cat["T"]   =   {[[T1-T2-T3-T3a-T4-T5-T6-T7-T8-T8a-T9-
                 T10-T11-T11a-T12-T13-T14-T15-T16-T16a-
                 T17-T18-T19-T20-T21-T22-T23-T24-T25-
                 T26-T27-T28-T29-T30-T31-T32-T32a-T33-
                 T33a-T34!]],
heading    =     "Warfare, hunting, butchery"}

cat["U"]   =    {[[U1-U2-U3-U4-U5-U6-U6a-U7-U8-U9-U10-
                  U11-U12-U13-U14-U15-U16-U17-U18-U19-
                  U20-U21-U22-U23-U24-U25-U26-U27-U28-
                  U29-U29a-U30-U31-U32-U32a-U33-U34-
                  U35-U36-U37-U38-U39-U40-U41-U42!]], 
                  heading = "Agriculture, crafts and Professions"}


cat["V"]   =    {[[V1-V1a-V1b-V1c-V1d-V1e-V1f-V1g-V1h-V1i-
                  V2-V2a-V3-V4-V5-V6-V7-V7a-V7b-V8-V9-
                  V10-V11-V11a-V11b-V11c-V12-V12a-V12b-
                  V13-V14-V15-V16-V17-V18-V19-V20-V20a-
                  V20b-V20c-V20d-V20e-V20f-V20g-V20h-V20i-
                  V20k-V20l-V21-V22-V23-V24-V25-V26-V27-
                  V28-V28a-V29-V29a-V30-V31-V32-V33-V34-
                  V35-V36-V37-V38-V39-V40-V40a!]], 
                  heading = "Rope, fiber, baskets, bags"}

cat["X"]   =    {[[X1-X2-X3-X4-X4a-X4b-X5-X6-X6a-X7-X8-X8a!]],
                   heading = "Loaves and cakes"}


cat["W"]   =     {[[W1-W2-W3-W4-W5-W6-W7-W8-W9-W9a-
                    W10-W10a-W11-W12-W13-W14-W14a-
                    W15-W16-W17-W17a-W18-W18a-W19-
                    W20-W21-W22-W23-W24-W24a-W25!]],
                    heading = "Vessels of stone and earthenware"}

cat["Y"]   =     {"Y1-Y1a-Y2-Y3-Y4-Y5-Y6-Y7!",
                   heading = "Writing, games, music"}

cat["Z"]   =     {[[Z1-Z2-Z2a-Z2b-Z3-Z4-Z5-Z5a-Z6-Z7-Z8-Z9-
                    Z10-Z11-Z12-Z13-Z14-Z15-Z15a-Z15b-Z15c-
                    Z15d-Z15e-Z15f-Z15g-Z15h-Z15i-Z16-Z16a-
                    Z16b-Z16c-Z16d-Z16f-Z16g-Z16h!]],
                    heading = [[Strokes, signs derived from 
                           hieratic, geometrical figures]],
                    label = "cat:z"}
 

cat["Aa"]  =     {[[Aa1-Aa2-Aa3-Aa4-Aa5-Aa6-Aa7-Aa7a-Aa8-Aa9-
						Aa10-Aa11-Aa12-Aa13-Aa14-Aa15-Aa16-Aa17-
						Aa18-Aa19-Aa20-Aa21-Aa22-Aa23-Aa24-Aa25-
						Aa26-Aa27-Aa28-Aa29-Aa30-Aa31-Aa32!]],
                   heading = "Unclassified"}

function cleanstring (s)
  s = string.gsub(s,"-", " ")
  s = string.gsub(s,"%t", " ") -- remove tabs?
  return string.gsub(s, "%s+", "-")
end

-- enable hyphenation of the input string
-- 
local function printstring (s)
  --s = string.gsub(s,"-", " ")
			  return string.gsub(s, "-", "\\hskip1sp-\\hskip1sp")
end

local function flushbuffer(a)
   if a~=nil then tex.print(a) end
   return ""
end

-- add the glyphs together
-- 
local function addtobuffer(str,bf)
    return bf..str
end

local function addtobottom(str)

end

local function placenext(str)

end

local function exclamationmark(str)
   
end



--[[
This is the main parser. 
@str = string formatted as per MoC 
@opt = 
The parsing occurs in two modes. Under normal mode
it parses strings and collects them in a buffer. If an
end of symbol is found, it flushes the buffer.

In the second mode it stacks the symbols.
--]]

local numberglyphs=0

local function parseMdC (str, opt) 

  local options = opt or {style=nil,
                    switch=nil,
                    echo=true,
                    strings=true,
                    heading="Man"}
  --if options.echo == true then 
        -- tex.print("{\\ttfamily "..printstring(str).." --}") end

  if options.heading == nil then options.heading="" end
  if options.style=="block" then  
     tex.print("\\par "..options.heading.."\\par ") 
  end

  local buff = ""
  local bufftop = "" 
  local buffbottom
  str = cleanstring(str)
  local counttop = 1
  local countbottom = 1
  local stackon = false
  local staron = false

-- parsing loop
  local i = 0
  while i < #str do
    local c = string.sub(str,i,i) 
    i = i + 1	
    if c=="-" then 
       if type(t[buff].theglyph~=nil) then
         tex.print(t[buff].fullblock)
       end
       buff =""
     elseif c == "!" then
       tex.print(t[buff].fullblock,"\\par")
     elseif c=="*" then -- just signs next to each other
       staron = true
       -- if star is completed in means previous symbol
       -- is completed stack next to each other
     elseif c==":" then
         stackon = true
         c="-" --ignore for the time being flush buffer
		   --buff = flushbuffer(t[buff].fullblock)
     else
       if type(t[buff]~=nil) then
            buff = addtobuffer(c,buff)
       else
            buff = buff..c
       end
  end
    -- end of line
  end

  numberglyphs = numberglyphs+i

end

-- tex.print(" ","Number of Glyphs",numberglyphs)





local printgardiner = function (t,options) 
   local ts = {} -- table to hold the cat keys
   local tmp = {}
   local hcmd = ""
   local str = ''
   for k,v in pairs (cat) do
      table.insert(ts,k)
   end
   table.sort(ts)

   for k,v in pairs (ts) do
      tmp = cat[ts[k]]
      if (options.headings and tmp.heading) then
          hcmd = "\\"..options.headings --TODO improve on interface 
          tex.print(hcmd.." {"..tmp.heading.."}")
      end 
      if tmp.label then tex.print("\\label{"..tmp.label.."}") end
      tex.print("\\par\\noindent",parseMdC(tmp[1],options))
   end
end

-- prints a single list from gardiner's series
local printgardinercat = function (series)
      local tmp = cat[series]
      return tex.print(parseMdC(tmp[1], options))
end
%    \end{macrocode} 
% We r
%    \begin{macrocode}
return {printhierochar = printhierochar,
        printgardiner  = printgardiner,
        printgardinercat   = printgardinercat,
        parseMdC = parseMdC}


%    \end{macrocode}
%
%
% \iffalse  
%</hhiero1>
% \fi
%
% \section{Test macros}
% We provide a series of MWE for testing purposes.
% \subsection{Letter MWE}
% \iffalse
%<*MWE-01>
% \fi
% \subsection{MWE-01, using \texttt{multicol} and \texttt{multitoc}}
%    \begin{macrocode}
% This is an example using the default |lstlisting|
% A environment.
\documentclass{article}
\usepackage{phd}
\begin{document}
  \begin{lstlisting}
    \def\test{This is a test.}  
  \end{lstlisting}
  \begin{teX}
    \def\test{This is a test.} 
  \end{teX}
  \begin{teXX}
    \def\test{This is a test.} 
  \end{teXX}
  
  \startnumberat{25}
  \begin{teX}
    \def\test{This is a test.} 
  \end{teX}
  
  \startnumberat{35}
  \begin{teX}
    % coloring comments is done in orange
    \def\test{This is a test.} 
  \end{teX}
\end{document}
%    \end{macrocode}
% \iffalse  
%</MWE-01>
% \fi
% \iffalse
%<*MWE-02>
% \fi
% \subsection{Listings MWE-02}
%    \begin{macrocode}
%% Example using |multitoc|
\documentclass{article}
\usepackage{phd}
\renewcommand{\multicolumntoc}{2}
\title{Typesetting the Table of Contents in Multiple Columns}
\author{Dr Y. Lazarides}
\begin{document}
\maketitle
\tableofcontents
\section{First section}
\begin{multicols}{3}
\lipsum[1-2]
\end{multicols}
\subsection{First subsection}
\subsection{Second subsection}
\subsection{Third subsection}
\subsection{Last subsection}
\section{Second section}
\subsection{First subsection}
\subsection{Second subsection}
\subsection{Third subsection}
\subsection{Last subsection}
\section{Last section}
\subsection{First subsection}
\subsection{Second subsection}
\subsection{Third subsection}
\subsection{Last subsection}
\end{document}
%    \end{macrocode}
% \iffalse  
%</MWE-02>
% \fi
% 
% \subsection{Listings MWE-03}
% 
% \iffalse
%<*MWE-03>
% \fi
%    \begin{macrocode}
%% example for using encoded commands such as guillemets. (If you need
%% shorthands you need to load babel.
%%
\documentclass{article}
\usepackage{phd}
\newcommand{\encone}[1]{{\fontencoding{T1}\selectfont#1}}
\begin{document}


\noindent\begin{tabular}{@{}*8l@{}}
\toprule
\Kt\guillemotleft  & \Kt\guilsinglleft & \Kt\quotedblbase & \Kt\textquotedbl \\
\Kt\guillemotright & \Kt\guilsinglright & \Kt\quotesinglbase \\
\bottomrule
\end{tabular}
\medskip

\lipsum[1]
\end{document}
%    \end{macrocode}
% \iffalse  
%</MWE-03>
% \fi
% 
% \iffalse
%<*test-tufte>
% \fi
%    \begin{macrocode}
\documentclass[justified]{tufte-book}
\usepackage{phd}
\begin{document}
\lipsum[1]
\sidenote{\RaggedRight \protect\lipsum[5]}
\centering
\begin{minipage}{5cm}
\RaggedRight
\lipsum[1]
\end{minipage}
\end{document}
%    \end{macrocode}
% \iffalse  
%</test-tufte>
% \fi
%
% \iffalse
%<*test-memoir>
% \fi
%    \begin{macrocode}
% clashes with options
\documentclass{memoir}
\usepackage{phd}
\begin{document}
\lipsum[1]
\end{document}
%    \end{macrocode}
% \iffalse  
%</test-memoir>
% \fi
%
%\iffalse
%<*test-scrartcl>
% \fi
%    \begin{macrocode}
% clashes with options
\documentclass{scrartcl}
\usepackage{phd}
\begin{document}
\lipsum[1]

\[ A = \upalpha r^2/4\]

% check for complaints
$$ A = a + b $$
\end{document}
%    \end{macrocode}
% \iffalse  
%</test-scrartcl>
% \fi
% 
%\iffalse
%<*test-hyphenation>
% \fi
%    \begin{macrocode}
\documentclass{scrartcl}
\usepackage{phd}
\begin{document}
%% Tests if hyphenation routines have been
%% found.
\hsize2cm
\noindent Florida appendix asynchronous
\end{document}
%    \end{macrocode}
% \iffalse  
%</test-hyphenation>
% \fi
%
%\iffalse
%<*test-algorithms>
% \fi
%    \begin{macrocode}
% clashes with options
\documentclass{article}
\usepackage{phd}
\begin{document}
\begin{algorithm}[H]
\SetAlgoLined
\KwData{this text}
\KwResult{how to write algorithm with \LaTeX2e }
initialization\;
\While{not at end of this document}{
read current\;
\eIf{understand}{
go to next section\;
current section becomes this one\;
}{
go back to the beginning of current section\;
}
}
\caption{How to write algorithms}
\end{algorithm}
\IncMargin{1em}
\begin{algorithm}
\SetKwData{Left}{left}\SetKwData{This}{this}\SetKwData{Up}{up}
\SetKwFunction{Union}{Union}\SetKwFunction{FindCompress}{FindCompress}
\SetKwInOut{Input}{input}\SetKwInOut{Output}{output}
\Input{A bitmap $Im$ of size $w\times l$}
\Output{A partition of the bitmap}
\BlankLine
\emph{special treatment of the first line}\;
\For{$i\leftarrow 2$ \KwTo $l$}{
\emph{special treatment of the first element of line $i$}\;
\For{$j\leftarrow 2$ \KwTo $w$}{\label{forins}
\Left$\leftarrow$ \FindCompress{$Im[i,j-1]$}\;
\Up$\leftarrow$ \FindCompress{$Im[i-1,]$}\;
\This$\leftarrow$ \FindCompress{$Im[i,j]$}\;
\If(\tcp*[h]{O(\Left,\This)==1}){\Left compatible with \This}{\label{lt}
\lIf{\Left $<$ \This}{\Union{\Left,\This}}\;
\lElse{\Union{\This,\Left}\;}
}
\If(\tcp*[f]{O(\Up,\This)==1}){\Up compatible with \This}{\label{ut}
\lIf{\Up $<$ \This}{\Union{\Up,\This}}\;
\tcp{\This is put under \Up to keep tree as flat as possible}\label{cmt}
\lElse{\Union{\This,\Up}}\tcp*[r]{\This linked to \Up}\label{lelse}
}
}
\lForEach{element $e$ of the line $i$}{\FindCompress{p}}
}
\caption{disjoint decomposition}\label{algo_disjdecomp}
\end{algorithm}\DecMargin{1em}
\end{document}
%    \end{macrocode}
% \iffalse  
%</test-algorithms>
% \fi
%
%<*test-spacing>
% Uses the package setspace to set the spacing of the
% document.
%    \begin{macrocode}
\documentclass{article}
\usepackage{setspace}
\usepackage{lipsum}
\onehalfspacing
\begin{document}
\title{merry setspace christmas test}
\author{who cares?}
\date{2011-12-19}
\maketitle

\section{dummy}
\lipsum[1]
\subsection{first offspring}
\lipsum[2]
\begin{tabular}{llr}
  a & silly & story about latex \\
  silly & a & story
\end{tabular}
\end{document}
%    \end{macrocode}
%</test-spacing>
%
% \iffalse
%<*settings>
% \fi
%% Some settings
%    \begin{macrocode}
\cxset{nag keys = {l2tabu,%
                   orthodox}}
\cxset{onlyamsmath keys = {warning}}
%    \end{macrocode}
% \iffalse
%</settings>
% \fi
%
% ^^A\PrintIndex
%
% \Finale

% \newpage
%
% \section{List of Packages and Usage Statistics}
%
% Table~\ref{tbl:listofpaks} provides a list of the packages loaded as default by |phd|.
% The column describing usage statistics
% is from \url{http://arxmliv.kwarc.info/package_usage.php}. It is by no means an
% indication of overall popularity, but I have used these statistics as an
% guide in selecting what packages to include here in order to at least cover
% the scientific side well.
%
% 
\setcounter{step}{0}
\begingroup
\centering
\begin{longtable}{llp{3.5cm}p{3.5cm}}
\toprule
Ser.  &Usage &Remarks\\
\midrule
\inc &fixltx2e & patches to LaTeX2e&\\
\inc &nag      & nag provides routines to warn
                 user against using outdated
                 packages and commands.           &\\
\inc &nag      & microtype&\\
\inc &onlyamsmath &This package inhibits 
					the usage of 
                plain \tex and 
                on demand of standard
					\latex math environments. 
					This is useful for class writers 
					who want to force
					their clients to use the environments 
					provided by the amsmath package. &\\
\midrule
\inc &graphicx  &  & \\
\inc &wrapfig   &  & \\
\inc &rotating  &  & \\
\inc &subfig    &  & \\
\inc &xcolor    &  & If loaded by class we skip \\
\midrule
\inc &booktabs  &  & \\
\inc &tabularx  &  &\\
\inc &dcolumn   &  &\\
\inc &longtable &  &\\
\inc &colortabl &  &\\
\inc &multirow  &  &\\
\inc &landscape & &\\
\inc &threeparttable & &\\
\midrule
\inc &array     & &\\
\inc &amsfonts  & &\\
\inc &amsmath   & & (66226)\\
\inc &amssymb   & & (74838)\\
\inc &amsthm    & & (15606)\\
\inc &mathtools & &\\
\inc &stmaryd   & &\\
\inc &xpfeil    & &\\
\inc &extpfeil  & &\\
\inc &euscript  &For calligraphic fonts &\\
\inc &bm        &                       &\\
\inc &bbm       &                       &(2200)\\
\inc &upgreek   &                       & \\
\midrule
\multicolumn{4}{c}{Symbols}\\
\midrule
\inc &latexsym  & &\\
\inc &wasymsym  & &\\
\inc &textcomp  & &\\
\inc &pifont    & &\\
\inc &marvosym  & &\\
\inc &manfnt    & &\\
\inc &bbding    & &\\
\inc &ifsym     & &\\
\inc &eurosym   & &\\
\midrule
\inc &epigraph  & &\\
\inc &siunitx   & &\\
\inc &filecontents & &\\
\midrule
\inc & changepage         & &\\
\inc & keyval             & &\\
\inc & ifmtarg            & &\\
\inc & fp                & &\\
\inc & ifthen             & &\\
\inc & xstring            & &\\
\inc & etoolbox           & &\\
\inc & algorithms         & &\\
\inc & algorithmicx       & &\\
\inc & algorithm2e        & &\\
\midrule
\inc & multicol           & &\\
\inc & multitoc           & &\\
\inc & ragged2e           & &\\
\inc & soul               & &\\
\inc & xspace             & &\\
\inc & ulem               & &\\
\inc & alltt              & &(259)\\
\bottomrule
\inc & idxlayout          & &\\
\bottomrule
\inc & tcolorbox          & &\\
\inc & listings           & &\\
\midrule
\multicolumn{4}{c}{Miscellaneous} \\
\midrule
\inc & fourier/fourier-orns & ornaments/math  &\\
\inc & cclicenses         & &\\
\inc & dirtree            &directory trees &\\
\midrule
\multicolumn{4}{c}{Archaic} \\
\midrule
\inc  &linearA & &\\
\inc  &linearB & &\\
\inc  &cypriot & &\\
\inc  &sarabian & &\\
\bottomrule
\end{longtable}
^^A\captionof{table}{List of packages loaded by the phd package.}
\endgroup

% \label{tbl:listofpaks}
%
% \bibliography{phd}
% ^^A \PrintIndex
\endinput




%  \parindent=1.5em


\newacro{SPQR}{Senatus Populusque Romanus}
\newacro{URL}{uniform resource locator}
\newacro{OUP}{Oxford University Press}

\chapter{Acronyms and Abbreviations}


In this section we will discuss the use and typesetting of symbols, abbreviations and acronyms. The |phd| package loads a number of packages and also offers a number of commands in managing symbols, abbreviations and acronyms. The main package we use to manage acronyms is \pkgname{acronym} \cite{acronym}. We also use some build-in commands for abbreviations and to assist in enforcing in-house style guides.  

\section{General Principles}

Abbreviations and symbols represent, through a variety of means, a
shortened form of a word or words. Abbreviations fall into three categories:
only the first of these is technically an abbreviation, though the
term loosely covers them all, and guidelines for their use overlap.

\begin{itemize}
\item \textit{Abbreviations} are formed by omitting the end of a word or words (VCR, lbw, Lieut.).
\item \textit{Contractions} are formed by omitting the middle of a word or words (I've,
mustn't, ne'er-do-well
\item \textit{Acronyms} are formed from the initial letters of words (SALT, Nazi, radar), the results being pronounced as words themselves.
\end{itemize}


\section{Acronyms}

An acronym is distinguished from other abbreviated forms by being a series of letters or 
syllables pronounced as a complete word: \textsc{NATO}
and UEFA are acronyms, but MI6 and BBC are not. Acronyms take no
points, whether all in caps (NAAFI, SALT, WASP), in initial capitals with
upper and lower case (Aga, Fiat, Sogat), or entirely in lower case (derv,
laser, scuba). Since they perform as words they can begin sentences, with
lower-case forms being capitalized normally, such as \textit{Laser treatment}. 

Any all-capital proper-name acronym is, in some house styles, fashioned
with a single initial capital if it exceeds four letters (Basic, Unesco, Unicef).

The \textit{Oxford Guide} suggests that editors should avoid this rule, useful though it is, where the result runs
against the common practice of a discipline (CARPE, SSHRCC, WYSIWYG), or where similar terms would be treated dissimilarly based on length alone.


Acronyms are not new language inventions, they were used well back in antiquity.  For example, the official name for the Roman Empire, and the Republic before it, was abbreviated as \ac{SPQR}. Inscriptions dating from antiquity, both on stone and on coins, use a lot of abbreviations and acronyms to save room and work. For example, Roman first names, of which there was only a small set, were almost always abbreviated. Common terms were abbreviated too, such as writing just "F" for "filius", meaning "son of", a very common part of memorial inscriptions mentioning people. Grammatical markers were abbreviated or left out entirely if they could be inferred from the rest of the text.\ac{SPQR}

So called \textit{Nomina Sacra} were used in many Greek biblical manuscripts. The common words "God" (Θεός), "Jesus" (Ιησούς), "Christ" (Χριστός), and some others, would be abbreviated by their first and last letters, marked with an overline. This was just one of many kinds of conventional scribal abbreviation, used to reduce the time-consuming workload of the scribe and save on valuable writing materials. The same convention is still commonly used in the inscriptions on religious icons and the stamps used to mark the eucharistic bread in eastern churches.

The early Christians in Rome, most of whom were Greek rather than Latin speakers, used the image of a fish as a symbol for Jesus in part because of an acronym—fish in Greek is ΙΧΘΥΣ (ichthys), which was said to stand for Ἰησοῦς Χριστός Θεοῦ Υἱός Σωτήρ (Iesous CHristos THeou hUios Soter: Jesus Christ, God's Son, Savior). Evidence of this interpretation dates from the 2nd and 3rd centuries and is preserved in the catacombs of Rome. And for centuries, the Church has used the inscription INRI over the crucifix, which stands for the Latin \textit{Iesus Nazarenus Rex Iudaeorum} (``Jesus the Nazarene, King of the Jews'').

The Hebrew language has a long history of formation of acronyms pronounced as words, stretching back many centuries. The Hebrew Bible ("Old Testament") is known as "Tanakh", an acronym composed from the Hebrew initial letters of its three major sections: Torah (five books of Moses), Nevi'im (prophets), and K'tuvim (writings). Many rabbinical figures from the Middle Ages onward are referred to in rabbinical literature by their pronounced acronyms, such as Rambam (aka Maimonides, from the initial letters of his full Hebrew name (Rabbi Moshe ben Maimon) and Rashi (Rabbi Shlomo Yitzkhaki).


The main package we load to assist with acronyms and abbreviations is |acronym|, developed by Tobias Oetiker \citeyearpar{acronym}. The package offers a number of useful commands to help with managing acronyms and to produce lists of acronyms and abbreviations. The package works by offering commands that you use to define an acronym as well as an environment serving the same purpose.

\begin{docCommand}{ac}{\meta{short version of the acronym}}
    To enter an acronym inside the text, use the |\ac{NATO}|
\end{docCommand}
    
    \begin{quote}
     |\ac{|\meta{acronym}|}|
    \end{quote}
    command. The first time you use an acronym, the full name of the
    acronym along with the acronym in brackets will be printed. If you
    specify the |footnote| option while loading the package, the full
    name of the acronym is printed as a footnote.
    The next time you access the acronym only the acronym will
    be printed.

\section{Symbols}

Symbols or signs, are a shorthand notation signifying a word or concept, and are frequent features of scientific and technical writing. The distinction between abbreviation and symbol may be blurred when, lie an abbreviation, a symbol is derived directly from a word or words (\textit{Ag} from \textit{argentum}, \textit{Pa} from \textit{pascal}, \textit{U} from \textit{uranium}), and in setting they are often treated similarly. Unlike abbreviations, however, symbols never take points, even if a single letter, or used alone or in conjuction with figures or words: \textit{F}  for \textit{false}, \textit{fluorine}, \textit{phenylalanine}.

Abstract, purely typographical symbols follow similar rules, being either close up (\ding{38}\ding{33}\ding{43})or spaced (\ding{38} \ding{33} \ding{43}). In \latex you can insert a non-breaking space if you want or a |hairsp|.

|\ding{38}~\ding{33}~\ding{43}|

Personally for the example I would prefer not to split them and the non-breaking space is a better option in this instance.

Symbols' uses can differ between disciplines. For example, in philological
works an asterisk (\textasteriskcentered) prefixed to a word signifies a reconstructed
form; in grammatical works it signifies an incorrect or nonstandard
form. A dagger (\textdagger) may signify an obsolete word, or 'deceased' when
placed before a person's name (this convention should be used only in
relation to Christians). In German a double dagger (\textdaggerdbl) follows the name
and signifies `killed in battle', \emph{gefallen} or  \gtrsymKilled.

A full set of these genealogical symbols can be found in the \pkgname{genealogytree} package  developed by  \person{Thomas F. Sturm} \citeyearpar{genealogytree} and are shown below,

\begin{scriptexample}[]{}{}
\textsl{\gtrSymbolsFullLegend[english]}
\end{scriptexample}

The package is loaded automatically by the |phd| package. Besides these symbols numerous other symbols
are loaded and described in the Chapter for Symbols.
\section{Abbreviations}

\subsection{Time Designations}

Most style guides recommend that you spell out the names of the months in the text but abbreviate them in the list  of works cited, except for May, June and July \cite{MLA}. The same manual suggests that words denoting units of time are also spelled out in the text (\textit{second}, \textit{minute}, \textit{week}, \textit{month}, \textit{year}, \textit{century}, some time designations are used only in the abbreviated form (\textit{a.m., p.m., AD, BC, BC, BCE, CE}). The |phd| package provides some assistance by loading the \pkg{datetime} package; more information on using it and of date and time formatting as well as calculations in Handling Dates and Time can be found in Pages~\pageref{ch:dates}--\pageref{datesend}.
\medskip

\begin{longtable}{lp{8cm}}
AD & after the birth of Christ (from the Latin \textit{anno Domini} `in the year of the Lord'; used before numerals ["\AD 14"] and after references for centuries ["twelfth century \AD"]\\
a.m. & before noon (from the Latin \textit{ante meridiem})\\
Apr. &April\\
Aug. &August\\
BC   &before Christ (used after numerals [``18 BC''] and referenced to centuries [``sixth century BC'']\\
BCE &before the common era (used after numerals and references to centuries)\\
CE  &common era (used after numerals and references to centuries)\\
cent. &century\\
Dec. &December\\
Feb  &February\\
Fri. &Friday\\
hr. &hour\\
Jan. &January\\
Mar. &March\\
min. &minute\\
mo. &month\\
Mon. &Monday\\
Nov. &November\\
Oct. &October\\
p.m. &after noon (from the Latin \textit{post meridiem})\\
Sat. &Saturday\\
sec. &second\\
Sept.&September\\
Sun. &Sunday\\
Thurs. &Thursday\\
Tues. &Tuesday\\
Wed. &Wednesday\\
wk. &week\\
yr. &year\\
\end{longtable}

Tables such as the one above, if not provided by the Publisher, can be very helpful, if you develop them on your own and refer back to them for consistency.

\subsection{Geographic Names}

\subsection{Common Scholarly Abbreviations and Reference Words}

\begin{figure}[tp]
\centering
\fbox{\includegraphics[width=1.0\textwidth]{./images/abbreviations.pdf}}
\caption{A typical Abbreviations page. This has been extracted from \protect\cite{bacchae}.}
\end{figure}


\section{The indefinite article with abbreviations}

The choice between \emph{a} and \emph{an} before an abbreviation depends on pronunciation,
not spelling. Use a before abbreviations beginning with a
consonant sound, including an aspirated h and a vowel pronounced with the sound of w or y.

\begin{scriptexample}
a BA degree a KLM flight a BBC announcer
a Herts, address a hilac demonstration a YMCA bed
a SEATO delegate a U-boat captain a UNICEF card
\end{scriptexample}

Use \emph{an} before abbreviations beginning with a vowel sound, including
unaspirated \emph{h}:

an AB degree an MCC ruling an FA cup match
an H-bomb an IOU an MP
an MA an RAC badge an SOS signal

This distinction assumes the reader will pronounce the sounds of the
letters, rather than the words they stand for (a Football Association cup
match, a hydrogen bomb). MS for manuscript is normally pronounced as
the full word, manuscript, and so takes a; MS for multiple sclerosis is
often pronounced em-ess, and so takes an. Likewise 'R.' for rabbi is
pronounced as rabbi ('a R. Shimon wrote'), but 'R' for a restricted classification
is normally pronounced as arr ('an R film').

The difference between sounding and spelling letters is equally important
when choosing the article for abbreviations that are acronyms and
for those that are not: a NASA launch but an NAMB award. The same holds
for names of symbols, which can vary: in America a hash symbol (\#) is a
'number sign' or, more formally, an \textit{octothorp}; in linguistic use an
asterisk may be called a 'star' and in mathematics an exclamation
mark called a 'factorial', 'shriek', or 'bang', so the correct forms are a *
and a ! rather than an * and an !. 

As abbreviated terms enter the
language there can be a period of confusion as to how they are pronounced:
in computing, for example, \ac{URL} is
pronounced by some as an abbreviation (you-are-ell) and others as an
acronym (earl), with the result that some write it as a URL and others as
an URL. Until a single pronunciation becomes generally accepted, the
best practice is simply to ensure consistency within a given work.

\section{Latin abbreviations}
\normalfont

Do not confuse 'e.g.' (\emph{exempli gratia}), meaning 'for example', with 'i.e.'
{id est), meaning 'that is'. Compare hand tools, e.g. hammer and screwdriver
with hand tools, \ie those able to be held in the user's hands. Print both lower-case roman, with two points and no spaces, and preceded by a
comma. In OUP style 'e.g.' and 'i.e.' are not followed by commas, to avoid
double punctuation; commas are often used in US practice.

Although many people employ 'e.g.' and 'i.e.' quite naturally in speech
as well as writing, prefer 'for example' and 'that is' in running text.
(Since 'e.g.' and 'i.e.' are prone to overuse in text, this convention helps
to limit their profusion.) Conversely, adopt 'e.g.' and 'i.e.' within parentheses
or notes, since abbreviations are preferred there. A sentence in text cannot begin with 'e.g.' or 'i.e.'; however, a note can, in which case
they—exceptionally—remain lower case. The \textit{Oxford Guide} gives an example of exception to the rule

The package offers two commands:

\begin{verbatim}
\newcommand{\ie}{\textit{i.\hairsp{}e.}\xspace}
\newcommand{\eg}{\textit{e.\hairsp{}g.}\xspace}
\end{verbatim}

The commands handle the spacing and if they are to be in italics or not. Renew the commands to set the style you want.

\section{Units}
\label{units}

Most users of \latex will have a need for specifying units in  mathematical or text contexts. We load the \pkgname{siunitx} package. The package was developed by Joseph Wright\cite{siunitx}. The correct application of units of measurement is very important in technical applications. For this reason, carefully-crafted definitions of a coherent units system have been
laid down by the \textit{Conférence Générale des Poids} et Mesures (CGPM): this has resulted in
the \textit{Système International d’Unités} (SI). At the same time, typographic conventions for
correctly displaying both numbers and units exist to ensure that no loss of meaning
occurs in printed matter.

|siunitx| aims to provide a unified method for \latex users to typeset numbers and
units correctly and easily. The design philosophy of |siunitx| is to follow the agreed rules
by default, but to allow variation through option settings. In this way, users can use
|siunitx| to follow the requirements of publishers, co-authors, universities, etc. without needing to alter the input at all.



\begin{ddanger}
Angles can be typeset using the \cs{ang} command.  The
 \meta{angle} can be given either as a decimal number or as a
 semi-colon separated list of degrees, minutes and seconds, which
 is called \enquote{arc format} in this document. The numbers which
 make up an angle are processed using the same system as other numbers.
\end{ddanger}

%  \parindent1em

\chapter{Boxes and glue in TeX}

\setlength{\columnsep}{2em}
{\it Once you understand \tex\rq{}s concept of glue, you may well decide that
it was misnamed; real glue doesn't stretch or shrink in such ways, nor does it
contribute much space between boxes that it welds together. Another word like
\emph{spring} would be much closer to the essential idea, since springs have a natural
width, and since different springs compress and expand at different rates
under tension. But whenever the author has suggested changing \tex's terminology,
numerous people have said that they like the word \emph{glue} in spite of its
inappropriateness; so the original name has stuck. }
\smallskip

{\hfill  ---  Donald E. Knuth}

\medskip   


\parindent1em




\newthought{Traditional typesetting} was a task that depended on assembling the types and inserting them one by one on holding frames. In a way it was an assembly of boxes.
The \tex typesetting system uses a similar model of boxes to typeset content but in addition it also uses the concept of glue to stretch or shrink the text so that it will look better typographically. Boxes contain
typeset objects, such as text, mathematical displays, and pictures, and glue
is flexible space that can stretch and/or shrink by amounts that are under
user control.

\begin{figure}[h]
\hbox{\drawfontbox{Qwerty}\drawfontbox{fjord}}
\caption{Everything is boxes.}
\end{figure}

\begin{center}
\printfontparams
\end{center}

\section*{Boxes}

Boxes in \tex have  a rectangular shape but have
three associated measurements called \emph{height}, \emph{width}, and \emph{depth}.
Figure \ref{fig:boxes} shows a 
picture of a typical box, showing its so-called \emph{reference point} and \emph{baseline}

The reason that they have three dimensions is that a character of text has normally three dimensions as shown in figure \ref{fig:boxes}. As characters need to be lined on a baseline, the depth provides a datum point on which they can be aligned and the depth provides a measure of the portion of the character that is below the baseline.


Boxes and glue are the main tools of \tex. The box can hold text and other items. Glue is simply spacing. It can be horizontal or veritcal spacing, and it can be made as rigid or as flexible as desired.



\textbf{One important feature of \tex is that it has no knowledge of the shape of the characters it typesets, just the dimensions of each character box.}


When \tex is typesetting, it is normally in horizontal mode, such as while
it is working on this paragraph. Otherwise, \tex can be in vertical mode, or
in math mode, or three others described in Chapter 13 of The \texbook.
Two low-level TEX commands for boxes are \cmd{hbox}, for a horizontal box,
and \cmd{vbox} for a box in vertical mode. In the latter, \tex is normally still
collecting material for display from right to left: it is not building up a
column of text, as in classical Chinese writing.

In both kinds of boxes, the result is an unbreakable object that acts
much like a single character. \tex reads input as a string of characters,
then breaks that string up in words, each of which forms a box. 
\emph{Word boxes}\index{word boxes}
are then collected into lines, lines into paragraphs, and paragraphs into a
page galley. The space between the words can be normal \emph{interword space},
or \emph{sentence-ending} space, which is somewhat larger in English-language
typesetting, and the space is normally glue, rather than of fixed size.

\tex has a sophisticated mathematical algorithm for figuring out the
best way to stretch or shrink interbox glue to optimize the appearance of
lines and paragraphs. Every so often, \tex checks to see whether it has
enough material saved on the growing page galley to fill a complete output
page, and it asynchronously (and effectively, unpredictably) calls the
output routine whose job it is to figure out where the page break should
happen, ship out a completed page to the |DVI|  file, and replace the galley by
whatever is left over.

In traditional \tex you  can force a line break with the carriage-return command \docAuxCommand{cr},
and a page break with the command \docAuxCommand{eject}, but \tex is an expert system,
and normally handles line and page breaking on its own. \latex provides its own commands such as \cmd{\clearpage} and \cmd{\newpage} and so do all other \tex based formats and systems.

\latex does not modify any of \tex's algorithms but simply it is a set of implemenation
macros.

\section{Units of measurement in TEX}

\tex allows you to specify sizes of typographical objects in any of nine different
units:


\begin{table}[htbp]
\begin{center}
\begin{tabular}{llp{5cm}}
\toprule
bp &big point &1 inch is exactly 72 bp; the PostScript pagedescription language uses these units, but just calls them points\\
cc &cicero: &1 cc is exactly 12 didot points, and is thus the European  analogue of the pica\\
cm &centimeter: &1 in is exactly 2.54 cm\\
dd &didot point: &1 dd is (1238/1157) pt, and is a typographical unit common in some parts of Europe\\
in &inch: &an archaic unit, roughly the width of a man's thumb; it has been discarded by most countries, but still used in the USA and its sattelites.\\
mm &millimeter: &1 in is exactly 25.4 mm\\
pc &pica: &1 pc is exactly 12 pt\\
pt &printer's point: &1 in is exactly 72.27 pt\\ 
sp &scaled point: &1 pt is exactly $2^{16}$ = 65536 sp.\\
\bottomrule
\end{tabular}
\end{center}
\end{table}



The units can be separated from their numeric value with optional space, so
\texttt{3pc} and \verb*+3 pc+ are equivalent. The little half box in the latter is a convenient
way to indicate explicit spaces in typewriter text. It can be printed by typing \cmd{\char32} and a suitable font or using |\textvisiblespace|.

Internally, \tex\ stores dimensions as integral numbers of scaled points:
1 sp is tiny ---  smaller than the wavelength of visible light.\footnote{The visible light has wavelengths from 380--450 nm for violet up to 620--750 nm for red (sp = 280 nm} It is sometimes
useful to create objects that small so that they differ from empty objects,
but are nevertheless invisible. It also ensures that TeX will look the same irrespective on which computer you actually compiled your document.

\tex deals only with 32-bit integer words, and does not take advantage
of extra precision available on historical machines with larger words. The
lower 16 bits of a dimension can be viewed as a fractional number of points,
and the uppermost bit is needed for a sign (0 for plus, 1 for minus). That
leaves 15 bits to hold an integral number of points, but TEX only expects 14
to be used, so that addition of two dimensions does not overflow. Thus, the
largest dimension in TEX is exactly 214 + (1 26) points, or about 5.758  
meters or 18.89 feet. 

\tex has several kinds of special storage locations, called registers, numbered
from 0 to 255. For example,\cs{dimen0} can hold a fixed dimension,
which can be specified in any of the nine units of measurement that are
recognized by \tex.

Here is how you can assign a dimension to a register, and then have \tex
display it back for you:

\begin{dispListing}
\dimen1 = 25.4mm (*@\protect\footnote{You shouldn't assign dimenensions to primitive registers, but rather use one of the allocation schemes provided by \latex to do so.}  @*)
\the\dimen1
\end{dispListing}


Notice that \tex’s output is always in points, showing that it converts different
input units to a common system of measurement.

You can convert a dimension to the much-smaller units of scaled points
by assigning it to another kind of \tex register designed to hold signed integers,
the\cs{count0} through\cs{count255} registers:

\begin{texexample}{}{}
\bgroup
  \dimen4 = 1pt
  \count4 = \dimen4
  \the\count4
\egroup
\end{texexample}

{\noindent This time we get the size as \texttt{sp} as 65536 }


You might have noticed that the conversion from inches to points was not
quite what we claimed in the summary of \tex units. Here is how to see the
differences:

\verb+\dimen1 = 1in+


\section{Skip registers}
\index{registers!skip}
\begin{docCommand}{skip}{}
\tex glue is specified as a fixed dimension, and optionally, with a plus and/
or minus dimension. Along with \cs{dimen} registers, TEX has glue registers,
called \cs{skip0} through \cs{skip255}. Here is how you can save glue settings in
 registers, and ask \tex to display the contents of one of them:
\end{docCommand}

\begin{texexample}{Skip counters}{ex:skipcounters}
\bgroup
  \skip1 = 10pt
  \skip2 = 10pt plus 3pt
  \skip3 = 10pt minus 2pt
  \skip4 = 10dd plus 3dd minus 2dd
  \the\skip4
  % 10.70007pt plus 3.21002pt minus 2.14001pt.
\egroup  
\end{texexample}


The four sample glue settings store, respectively, \textit{fixed glue}, \textit{stretchable
glue}, \textit{shrinkable glue}, and \textit{flexible glue} that can both stretch and shrink,
but only up to a specified amount. Interword and intersentence spaces are
generally defined with glue like this, so that if more stretch or shrink of
\index{glue}\index{glue!flexibe}\index{glue!stretchable}\index{glue!shrinkable}

\begin{teX}
\dimen2 = 72.27pt
\count1 = \dimen1
\count2 = \dimen2
\showthe \count1
> 4736286.
\showthe \count2
> 4736287.
\end{teX}

The two values differ by the tiny value 1 \textit{sp}, so we can in practice ignore
that difference. If we use higher-precision arithmetic, we find the exact
decimal equivalents of the fractions as

\begin{teX}
4 736 286=65 536 = 72.269 989 013 671 875;
4 736 287=65 536 = 72.270 004 272 460 937 5;
4 736 286.72=65 536 = 72.27
\end{teX}


Actually \tex uses that last relation as the definition of the conversion of
inches to scaled points, so that our assignment of 1 in to \verb+\dimen1+ has to
be rounded to the nearest integral number of scaled points. That is why
in the round-trip conversion from decimal to binary and back to decimal,
1 in became 72.26999 pt. \tex guarantees that its output decimal numbers
are always converted on input back to the original binary numbers from
whence they came. For more on the story of \tex’s I/O conversions, see [3].


Both \tex and \latex define a number of predefined dimensions and these are discussed in the relevant Chapters discussing the \latex kernel. For example you may come across the \docAuxCommand{jot}, which is defined by \latex as:

\begin{teX}
\newdimen\jot
\jot=3pt
\end{teX}

\begin{texexample}{jot}{ex:jot}
\bgroup
\parindent\jot
This is some sample text with a one |\jot| left indentation. Which is really too small to see in a paragraph.
\egroup
\end{texexample}

Defining |\parindent=jot| we can see that the indentation almost disappeared. This is obvious since |\jot| is used normally for maths.

Glue is the binder that lets \tex\ do its job. This chapter discusses some
preset forms of glue and their uses. Along with glue parameters there are a
number of special commands for inserting glue. The most interesting have
different degrees of infinity, namely |\hfil|, |\hfill|, |\hfilneg|, |\hss|,
and their vertical counterparts. 

Because of the special spacing requirements of mathematics, \tex\ defines
skips and spacing that are valid in mathematics mode only. Examples are the
are the special preset |mu| glues of |\thickmuskip|,
|\medmuskip|, |\thinmuskip|.
The |\newskip| and |\newmuskip| commands allocate skip (or really glue)
or muskip registers respectively for special uses. For example the specific
glues surrounding section heads are held in glue registers. 

It should be noted that the various |\...muskip| do not cause horizontal
spacing in math mode by themselves. They are used with |\mskip| to actually
cause the insertion of the glue.

These are all various forms of horizontal (or math) mode commands that insert
infinite quantities of glue. |\hfil|, |\hfill|, |\hfilneg|, and |\hss| insert
|plus 1fil|, |plus 1fill|, |minus 1fil|, and |plus 1fil minus 1fil|. 
The first two are {\it stretch} glues, the third is {\it shrink} glue and the
last is both. It
should be noted that when \tex\ tries to stretch or shrink glue values, they
vary according to their value. If there exists both |fil| and finite value
glue in a box or line, then all the stretch if it is |plus| glue 
or shrink if it is |minus| glue will be in
the section with the |fil| glue. If there is |fill| glue and either |fil| or
finite glue, then all the stretch or shrink will be in the |fill| sections.
Similarly with |filll| glue, which is not supplied in as readily usable form.
The difference in behaviour between |\hfil| and |\hss| is that |\hss| glue
will allow the contents of a box to spill outside without resulting in an
overfull box while |\hfil| will only fill or push contents to the edge of the
box. The major use for these glues are to center or to force stuff to either
edge of a box.  For instance this

\begin{texexample}{}{}
\def\aline{\vrule \hfil one \hfil two \hfil three 
         \hfill four \hfil five \hfil six 
         \hfill seven \hfil eight \hfil nine\vrule}

\aline

This is a preset |mu| glue. |\medmuskip = 4mu plus 2mu minus 4mu| for use
with |\mskip|. This is \hbox{\strut\vrule$\mskip\medmuskip$\vrule}.
\end{texexample}

So far we have been using in our examples the primitive \tex |\skip| to allocate
skip registers. This is dangerous, as they might have been defined elsewhere and our
definitions will overwrite them. Plain \tex and \latexe provide an allocation scheme
where this is done automatically. 

\begin{docCommand}{newskip}{}
The command \cs{newskip}\meta{skip name} assigns a new skip or glue register to
the name |\<skip name>|. 

Glue values may be assigned to it by |\<skip name> [=] <glue>|.
This assigns a new muskip or muglue register to the name |\<muskip name>|. Glue
values may be assigned to it by |\<muskip name> [=] <muglue>|.

This is a preset |mu| glue. |\thickmuskip = 5mu plus 5mu| for use
with |\mskip|. This is \hbox{\strut\vrule$\mskip\thickmuskip$\vrule}.

This is a preset |mu| glue. |\thinmuskip = 3mu| for use
with |\mskip|. This is \hbox{\strut\vrule$\mskip\thinmuskip$\vrule}
\end{docCommand}


\newskip\hides 
\hides= -1000pt% plus 1fill

atest atest

\hskip\hides atest

\hides= -1000pt  plus 1fill

atest atest

\hskip\hides atest

|\vfil \vfill|

|\vfilneg|

|\vss|

These are the vertical analogues to the infinite horizontal glues and act in
much the same manner. See the section on |\hfil ...|.
 



\normalsize



\section{Glue}

\tex joins the boxes it creates with some special mortar as Knuth writes, called glue. To understand how glue works we will
borrow a figure from the \tex Book.

\begin{figure}
  \centering
  \includegraphics[width=0.9\linewidth]{./images/glue.png}
  \caption{Glue in \TeX}
  \label{fig:glue}
\end{figure}


\section{How to specify glue}

The usual way to specify \textit{glue} to \tex is
$<dimen>< plus~dimen><minus~dimen>$

where the plus and minus are optional and assumed to be zero if not
present; plus\index{glue!plus} introduces the amount of stretchability\index{glue!stretchability}, minus introduces the amount of shrinkability \index{glue!shrinkability}. 

For example, Appendix B of the TexBook defines \cs{medskip} to be an abbreviation for
|\vskip6pt plus2pt minus2p|. The normal-space component of glue must always be
given as an explicit dimen, even when it is zero. The ability of \TeX to stretch and shrink this glue has given it its beautiful looks. Strangely enough, although the algorithm is public it has not been used widely in other software.



\subsection{hfil and hfill}

{\obeylines
{This text will be flush left.\hfil}
{\hfil This text will be flush right.}
{\hfil This text will be centered.\hfil}
{Some text flush left\hfil and some flush right.}
{Alpha\hfil centered between Alpha and Omega\hfil Omega}
{Five\hfil words\hfil equally\hfil spaced\hfil out.}
}

Consider the following definitions:

\begin{verbatim}
\def\centerlinea#1{\hfil#1\hfill}
\def\centerlineb#1{\hfill#1\hfill}
\def\centerlinec#1{\hss#1\hss}
We define quickly a \cs{lineX}\footnote{Strange but my \LaTeX\ distribution has not got on. (This definition is from \texttt{plain.sty}}

\def\lineX{\hbox to\hsize}
\def\lineX{\hbox to\hsize}
\def\centerlinea#1{\hfil#1\hfil}
\def\centerlineb#1{\hfill#1\hfill}
\def\centerlinec#1{\hss#1\hss}

\lineX{\centerlinea{\test}}
\lineX{\centerlineb{\test}}
\lineX{\centerlinec{\test}}
\centerline{\test}
\begin{center}\test\end{center}

\end{verbatim}


\section{Specifying glue amounts}

\tex glue is specified as a fixed dimension, and optionally, with a plus and
or minus dimension. Along with \cs{dimen} registers, TEX has glue registers,
called \cs{skip0} through \cs{skip255}. Here is how you can save glue settings in
\tex registers, and ask \tex to display the contents of one of them:

\begin{teX}
\skip1 = 10pt
\skip2 = 10pt plus 3pt
\skip3 = 10pt minus 2pt
\skip4 = 10dd plus 3dd minus 2dd
\the \skip4
\end{teX}


\texttt{> 10.70007pt plus 3.21002pt minus 2.14001pt}

The four sample glue settings store, respectively, {\em fixed glue}, {\em  stretchable
glue}, {\em shrinkable glue}, and {\em flexible glue}  that can both stretch and shrink,
but only up to a specified amount. Interword and intersentence spaces are
generally defined with glue like this, so that if more stretch or shrink of  a
re underfull (too little text to fill the line), or overfull (too much text in the
line).



\section{Overfull lines}

Although overfull lines are reported in the \tex log file, they can be hard
to find in the typeset document if they only stick out a little. To make
them highly visible while you are fine tuning your final document, assign
the variable \cs{overfullrule} a nonzero dimension, such as 2 cm. \tex then
displays a solid black box, called a \emph{rule}, of that width in the right margin
on each line that is overfull. Using the \docpkg{microtype} package one can adjust the parameters to minimize this.

To make the rules disappear, simply remove it,
or comment out, the assignment, or reset its value to 0 pt. 

Just as you can assign dimension registers to count registers to convert
from points to scaled points, you can assign skip registers to dimension and
count registers to discard the flexible parts:


\begin{teX}
\skip1 = 10pt plus 3pt minus 2pt
\the\skip1
 \dimen1 = \skip1
\the \dimen1
\count1 = \skip1
\the \count1
\end{teX}




\section{More on glue in boxes}

Besides normal glue with fixed amounts of stretch and shrink, \tex also has
two kinds of glue that are \emph{infinitely} stretchable and shrinkable: \cs{hfil} and
\cs{hfill} in horizontal mode, and \cs{vfil} and \cs{vfill} in vertical mode. Notice that there two versions
of the commands, the one ends with one ell and the second one with two. The
two-ell forms are more flexible than the one-ell forms.

The boxes and glue model is powerful, and \tex's author, Donald Knuth,
has written that he views it as the key idea that he discovered when he
first sat down in 1977--1978 to design a computer program for typesetting.
For example, to set something flush left, put infinitely-stretchable glue on
its right. To set it flush right, put the glue on the left. For centered material,
put the glue on both sides. Here are four examples, with vertical
bars marking the ends of the horizontal box (boxes have no visible frames,
although it is possible to write \tex commands to give them such outlines,
and we use that feature shortly):





\section{Horizontal and vertical boxes}


\begin{docCommand}{hbox}{\marg{material}}
\end{docCommand}

Like their dimensions \TeX's boxes are not what one thinks when thinking of boxes. TeX's boxes come in basically two flavours, horizontal boxes and vertical boxes. An \cs{hbox} is created by the command \refCom{hbox}\marg{material}. It has the following properties:

\begin{enumerate}
\item The material is placed from left to right and it becomes a \textit{horizontal list}.\index{horizontal list}
\item The box \textbf{cannot be broken across lines}; it is an indivisible unit.
\end{enumerate}

An |hbox| can contain, characters, horizontal glue, horizontal leaders or other boxes. While in many cases these other boxes can be other |\hbox|es, |\vbox| can be used.


The \refCom{hbox} command has another form |\hbox to <dimen>|\marg{material}. This
creates a box whose width is the given (dimen). Thus |\hbox to lcm{<material>}|
will create a 1 inch wide box \hbox to 1cm{text}. However, we have to supply exactly 1 cm worth of
material to fill up the box; otherwise we end up with an error message. It is best
to consider this form of the command as a promise; we promise '\tex that we will
supply just enough material to fill up the box. 

We can place other hboxes in an hbox. By adding glue we can then move them left or right

\begin{texexample}{hbox and glue}{ex:hbox}
\bgroup
\Huge
\hbox to \textwidth{\hfill \hbox{\EOofficerI}\hbox{\EOofficerII}\hbox{\EOofficerIII} \hfill}

\hbox to \textwidth{\hfill \hbox{\EOofficerI}\hbox{\EOofficerII}\hbox{\EOofficerIII} \hfil}

\hbox to \textwidth{\hfill \hbox{\EOofficerI}\hfill \hbox{\EOofficerII}\hfill \hbox{\EOofficerIII} \hfill}
\egroup
\end{texexample}

The last command that affects the shape of an |\hbox| is 'spread(dimen)', which
spreads the box beyond its natural width. An |\hbox spread12pt|{(material)}
makes the box 12 points wider than its natural size. If the material in the box has
no flexibility, it cannot spread to fill up the additional space, resulting in an underfull
box. This is why 'spread' is normally used with flexible glues.

\begin{texexample}{hbox and glue}{ex:hbox}
\bgroup
\LARGE
\hbox to 5cm{\EOofficerI\EOofficerII\hfill\EOofficerIII}

\hbox spread5cm{\hfill\EOofficerI\hfill\EOofficerII\hfil\EOofficerIII}

\hbox spread9cm{\EOofficerI\hfill\EOofficerII\hfil\hfil\EOofficerIII}

\hbox spread7cm{\EOofficerI\hfill\EOofficerII\hfil\hfil\EOofficerIII} 


\makeatletter
\hb@xt@ 5cm {\EOofficerI\EOofficerII\hfill\EOofficerIII}
\makeatother
\egroup
\end{texexample}



Boxes can be moved up or down using |\raise| or |\lower|. Each of these primitives is followed by a dimension indicating how far the box can be lowered or raised.

Other material that can go in an hbox, is \textbf{vertical rules}. 

\subsection{The null macro}

The |\null| macro is defined both in Plain as well as LaTeX and generates an empty box. Its definition is:

\begin{teXXX}
\def\null{\hbox{}}
\end{teXXX}


\fbox{\hbox{This is a test}}

{
\fbox{\hsize=5cm
A test of a box at the end of a 2.0 inch line\par}

\fbox{\hsize=5.0cm in A test of a box at the end of a \hbox to 2cm{2.0 cm} line\par}

}

What happens when we have more than two boxes on a line? TeX will stuck them one under another. If they are enclosed within another hbox they will be inlined.



\begin{texexample}{}{}
\hbox to 1cm {A} \hbox to 1cm {B}

If we however, put them together in another |\hbox|, we get:

\hbox{\hbox to 1cm {A} \hbox to 1cm{B}}
\end{texexample}




An |\hbox| does not imply horizontal mode, so an attempt to start a paragraph with a box, for
instance
|\hbox to 0cm{\hss$\bullet$\hskip1em}Text ...|

will make the text following the box wind up one line below the box. It is necessary to switch
to horizontal mode explicitly, using for instance |\noindent| or |\leavevmode|. The latter is defined
using |\unhbox|, which is a horizontal command.


\begin{texexample}{}{}
\hbox to 0cm{\hss$\bullet$\hskip1em} Text ...


\leavevmode\hbox to 0cm{\hss$\bullet$\hskip1em} Text ...

\end{texexample}




\section{Kerning}


Using the command \cs{kern}, we can move boxes either left or right. Kerning is extensively used to build internal commands and we discuss it in more detail under the chapter for fonts.

\begin{docCommand}{kern}{\meta{dimen}}
A |\kern| is similar to glue [75], with two differences: (1) |\kern| is rigid; (2)
|\kern| specifies a point where a line, or a page, should not be broken. Since a box is
indivisible anyway, |\kern| is used in a box to indicate rigid spacing. It is interesting
to note that the same command, |\kern|, indicates horizontal spacing when used in
an |\hbox| and indicates vertical spacing when used in a |\vbox|.
\end{docCommand}
Consider two horizontal boxes, holding the letters A and V:
As you can observe, the letters AB are a bit afar, from what would be a visually pleasant arrangement, we can kern them as follows:
\medskip

\begin{teXXX}
\hbox{\Huge AV A\kern-5ptV}
\end{teXXX}
\medskip

Note that hbox, does not produce a frame. I~have used a frame |\fbox|, which will cover a bit later as well as scaled the image by 2, in order to see the effects more clearly.


\drawfontbox{\upshape\Huge FJord F\kern-5pt Jorp}





\noindent\begin{tabular}{ll}
|\hbox{\kern4pt A\kern8pt B\kern8pt C\kern4pt}| & \fbox{\hbox{\kern4pt A\kern8pt B\kern8pt C\kern4pt}} \\
~ &\\
\midrule
|\hbox{\kern4pt\raise1pt\hbox{A}|  & \fbox{\hbox{\kern4pt\raise1pt\hbox{A} \kern8pt BC\kern8pt\lower6pt\hbox{D} \kern4pt} \kern8pt BC\kern8pt\lower6pt\hbox{D}\kern4pt} \\
|\kern8pt BC|                      &\\
|\kern8pt\lower6pt\hbox{D}|        &\\
|\kern4pt}|                        &\\ 
|\kern8pt BC|                      &\\ 
|\kern8pt\lower6pt\hbox{D}|        &\\
|\kern4pt}| &\\
\midrule
\end{tabular}


\vbox{
\noindent\rule{\linewidth}{0.4pt}
\begin{minipage}{4.5cm}
 \begin{teXX}
\fbox{\hbox{\kern4pt A\kern8pt 
      B\kern8pt C\kern4pt}}
\end{teXX}
\end{minipage}
\hfill\hfill
\begin{minipage}{3cm}
\hfill\hfill\fbox{\hbox{\kern4pt A\kern8pt 
      B\kern8pt C\kern4pt}}
\end{minipage}

\medskip
\noindent\rule{\linewidth}{0.4pt}
}

Notice that an |\hbox| is constructed by setting its components side by side so that their \textit{baselines} are aligned. When \cs{raise}, \cs{lower} are used the baselines are no longer aligned. In such a case the baseline of the box is defined as the baseline shared by the components before any vertical movements. In the example above the box now has a depth, as a result of lowering |D|.


\vbox{
\noindent\rule{\linewidth}{0.4pt}
\begin{minipage}{4.5cm}
\begin{teXXX}
\hbox{\kern4pt\raise1pt\hbox{A} 
  \kern8pt BC\kern8pt
  \lower6pt\hbox{D} 
  \kern4pt} 
\end{teXXX}
\end{minipage}
\hfill
\begin{minipage}{3cm}
\fbox{\hbox{\kern4pt\raise1pt\hbox{A} 
\kern8pt BC\kern8pt\lower6pt\hbox{D} \kern4pt}}
\end{minipage}

\medskip
\noindent\rule{\linewidth}{0.4pt}
}



\noindent\textbf{Vertical boxes.}\quad A vertical box is build in a similar manner to that of a horizontal list, except it is composed of material in the \textit{vertical list}.
When horizontal boxes are added in the list, they are stuck on top of each other as shown in the example below. 
\medskip

\bgroup
\parindent0pt
\fbox{\vbox{\hsize=3cm\fbox{\hbox{ABCDEFGH}} \fbox{\hbox{AB}}}}
\egroup


\begin{docCommand}{vbox}{ to \meta{dimen}\marg{\meta{material}}}
Typesets a box in vertical mode.
\end{docCommand}

It is important to remember the two main differences between hboxes and vboxes. An hbox will expand to hold its material. If it need be it will overfill the line and produce an overful warning. A vbox will expand to hold its material. It is perfectly normal for a vbox to hold paragraphs, as shown. This is not possible with an hbox. However, the common pattern is for an |\hbox| to contain a |vbox| .

\begin{texexample}{hbox/vbox example}{ex:vbox}
\noindent\fbox{\vbox{\lorem\par\lorem\par}}

\hbox to \linewidth{\vbox{\lorem\par\lorem\par}}
\end{texexample}


\begin{docCommand}{hsize}{\meta{dimen}}
 Controls the width of text in a |vbox|.
\end{docCommand}

\noindent\textbf{Controlling the size of a vbox.}\quad What controls the size, is the containing environment. This in TeX, is specified using |\hsize|. In LaTeX this is controlled by an enclosing environment, maybe a minipage (which is build this way) or one of the page width parameters.


\begingroup
\parindent0pt
\fboxsep5pt
\hsize=3.9cm\footnotesize
\hfil\fbox{\vbox{\RaggedRight\lorem\par}} 
\hfil\fbox{\vbox{\RaggedRight\lorem\par}}
\hfil\fbox{\vbox{\RaggedRight\lorem\par}}\hfill
\endgroup
\captionof{figure}{Output to demonstrate the use of vboxes.}



The code to typest the boxes shown above follows:
\medskip
\emphasis{hsize}
\begin{teXXX}
\bgroup
\parindent0pt
\hsize=3.3cm\footnotesize
\hfil\fbox{\vbox{\lorem\par}} 
\hfil\fbox{\vbox{\lorem\par}}
\hfil\fbox{\vbox{\lorem\par}}
\hfill
\egroup
\end{teXXX}


Note, the use of \docAuxCommand{hsize}. We define the font size as |\footnotesize|. We have done this in order not to have overfull boxes--Latin words don't have a full set of hyphenation patterns in \latex. The macro |\lorem|, we have defined internally for this document. We place the code in a group in order not to affect the rest of the document.



\clearpage

\noindent\textbf{Vertical centering}\quad can be achieved by applying vertical infinite glue \cs{vfill}. In the example that follows, first we place two letters in individual |\hboxes| and we enclose them in a vbox. We apply |\vfill| both on top and at bottom.

\emphasis{\vfill}

\vbox{
\noindent\rule{\linewidth}{0.4pt}
\begin{minipage}{5cm}
\begin{teX}
\fbox{\vbox to 0.9cm{\vfil\hbox{M}\nointerlineskip\hbox{i}\vfil}} 
\end{teX}
\end{minipage}
\hfill
\begin{minipage}{3cm}
\hfill\fbox{\vbox to 0.9cm{\vfil\hbox{M}\nointerlineskip\hbox{i}\vfil}}\hfill\hfill 
\end{minipage}

\medskip
\noindent\rule{\linewidth}{0.4pt}
}



A |\vbox| can be combined with text and may appear anywhere within a paragraph. The baseline of the box will be aligned with the baseline of the current line.


\vbox{%
\noindent\rule{\linewidth}{0.4pt}}

\begin{teX}
A vbox can be placed within a paragraph \fbox{\vbox to 0.6cm{\vfil\hbox{M}\nointerlineskip\hbox{i}\vfil}} as shown here.


\hfill

A vbox can be placed within a paragraph \fbox{\vbox to 0.6cm{\vfil\hbox{M}
  \nointerlineskip\hbox{i}\vfil}} as shown here.

\end{teX}

\medskip
\noindent\rule{\linewidth}{0.4pt}






\noindent\textbf{Top alignment.}\quad\cs{vtop} is similar to a |\vbox|. The depth of this box is zero, since both A and B are capital letters. The width of this box is |\hsize|, since it contains text. 


\begin{codeexample}[]
\vtop{\hbox{A} \hbox{B}}
\end{codeexample}






Centering a picture in a box, both vertically and horizontally can be achieved using the methods we described so far.


\emphasis{hfill,hbox}
\begin{texexample}{}{}
     \fbox{%
          \vtop{\medskip
                    \hfill
                      \hbox{\includegraphics[width=1.5cm]{./images/amato.jpg}}%
                    \hfill 
                   \medskip%
                }%
      }%
\end{texexample}

\begin{texexample}{}{}
    \fbox{%
          \vtop{\medskip
                    \hfill
                      \hbox{\includegraphics[width=1.5cm]{./images/amato.jpg}}%
                      \hbox{\includegraphics[width=1.5cm]{./images/amato.jpg}}%
                      \hbox{\includegraphics[width=1.5cm]{./images/amato.jpg}}%    
                    \hfill 
                   \medskip%
                }%
      }%
\end{texexample}

Study the example a bit more carefully, as we have said earlier on that \cs{hbox}'es are stacked vertically, the reason why in the above example they are next to each other is that they are in an
\cs{fbox} which in turn is an \cs{hbox}  that can draw  frame around the box and is defined in the
\latex2e kernel.

So if we had only three images in hboxes we will get:

\begin{texexample}{Three Images Lined}{}
%\leavevmode
%\parindent30pt
\hbox{\includegraphics[width=1.5cm]{./images/amato.jpg}}%
\hbox{\includegraphics[width=1.5cm]{./images/amato.jpg}}%
\hbox{\includegraphics[width=1.5cm]{./images/amato.jpg}}%
\end{texexample}

An hbox does not start a paragraph. If we started a paragraph the behaviour will be different.

\begin{texexample}{Three Images Lined}{}
.\hbox{\includegraphics[width=1.5cm]{./images/amato.jpg}}%
\hbox{\includegraphics[width=1.5cm]{./images/amato.jpg}}%
\hbox{\includegraphics[width=1.5cm]{./images/amato.jpg}}%
\end{texexample}

If you notice carefully, we have started the paragraph by inserting a `.' before the first |\hbox|, an alternative way is to 
use |\leavevmode|. The effect of this command is to leave vertical mode, and to enter horizontal mode. Thus, if the mode is vmode (typically, outside any paragraph), a new paragraph is started. This paragraph may be flushed left, flushed right. 

\begin{texexample}{Three Images Lined}{}
\leavevmode
\hbox{\includegraphics[width=1.5cm]{./images/amato.jpg}}%
\hbox{\includegraphics[width=1.5cm]{./images/amato.jpg}}%
\hbox{\includegraphics[width=1.5cm]{./images/amato.jpg}}%

\meaning\leavevmode
\end{texexample}

The macro |\leavevmode| as its name implies forces \tex to leave vertical mode and enter horizontal mode. In this case the photos are just treated by \tex similarly to any character and tehy are typeset next to each other. 

\begin{docCommand}{kern}{}
If we wanted to add a bit of space between the horizontal images, we could use \cs{kern}
Kern again. This is from the book TeX for The Impatient page 157. You can use kern in math mode, but you cannot use the \texttt{mu} units. If you want to use \texttt{mu} units use \cs{mkern} instead.
\end{docCommand}

\emphasize{kern}
\begin{texexample}{}{}
\leavevmode
\hbox{\includegraphics[width=1.5cm]{./images/amato.jpg}}\kern10pt
\hbox{\includegraphics[width=1.5cm]{./images/amato.jpg}}\kern10pt
\hbox{\includegraphics[width=1.5cm]{./images/amato.jpg}}%
\end{texexample}

One needs to be careful as to where you issue |\leavevmode|. If it is in the middle of a paragraph it will have no effect.
\emphasize{This,is,some,text}
\begin{texexample}{Example with leavevmode}{}
This is some text
\leavevmode
\hbox{\includegraphics[width=1.5cm]{botticelli-34.jpg}}\kern10pt
\hbox{\includegraphics[width=1.5cm]{botticelli-34.jpg}}\kern10pt
\hbox{\includegraphics[width=1.5cm]{images/botticelli-34.jpg}}%
\end{texexample}

A very common way in \latex2e is to issue a |\par| command before |\leavevmode| to avoid this problem. Another way is to use
one of the |\ifvmode| or |\ifhmode| and act accordingly. We now fix our example and get what we want. 

\emphasize{par}
\begin{texexample}{}{}
This is some text
\par\leavevmode
\hbox{\includegraphics[width=1.5cm]{botticelli-34.jpg}}\kern10pt
\hbox{\includegraphics[width=1.5cm]{botticelli-34.jpg}}\kern10pt
\hbox{\includegraphics[width=1.5cm]{images/botticelli-34.jpg}}%
\end{texexample}

\begin{texexample}{}{}
   \HHUGE
   \fboxsep=0pt
   \fbox{%
          \vtop{\medskip
                    \hfill
                       \hbox{ H\kern10pt i\kern10pt j}%    
                       \hbox{ A\kern10pt C\kern10pt j}%
                    \hfill 
                   \medskip%
                }%
   }%
\end{texexample}

This example shows how letters are typeset and you can see that they are aligned at the baseline. They are no different than the eimage example that we have shown earlier, except we don't need the boxes.

\medskip

\vbox{
\noindent\rule{\linewidth}{0.4pt}
\begin{minipage}{4.9cm}
\begin{teX}
\centerline{$\Downarrow$}\kern 3pt%
\centerline{$\Longrightarrow$\kern 6pt% horizontal kern
  \textit{A note about kern}\kern 6pt
    $\Longleftarrow$}
\kern 3pt
\centerline{$\Uparrow$}  
\end{teX}
\end{minipage}
\hspace{0.3cm}
\begin{minipage}{4.5cm}
\centerline{$\Downarrow$}\kern 3pt%
\centerline{$\Longrightarrow$\kern 6pt% horizontal kern
  \textit{A note about kern}\kern 6pt
    $\Longleftarrow$}
\kern 3pt
\centerline{$\Uparrow$}
\end{minipage}

\medskip
\noindent\rule{\linewidth}{0.4pt}
}
\medskip

To make a point again, |\vbox| lines boxes at their bottom while, |\vtop| lines them at their top.

\medskip

\vbox{
\noindent\rule{\linewidth}{0.4pt}
\begin{minipage}{4.9cm}
\begin{teX}
 \hbox{\hsize=2cm \raggedright
\vbox to 0.5in{\hrule This box is .5in deep. \vfil\hrule}
\qquad
\vbox to 0.75in{\hrule This box is .75in deep. \vfil\hrule}
\qquad
\end{teX}
\end{minipage}
\hspace{0.3cm}
\begin{minipage}{4.5cm}
\hbox{\hsize=2cm \raggedright
\vbox to 0.5in{\hrule This box is .5in deep. \vfil\hrule}
\qquad
\vbox to 0.75in{\hrule This box is .75in deep. \vfil\hrule}
\qquad}
\end{minipage}

\medskip
\noindent\rule{\linewidth}{0.4pt}
}

\medskip


Trying the same with vtop

\medskip

\vbox{
\noindent\rule{\linewidth}{0.4pt}
\begin{minipage}{4.9cm}
\begin{teX}
 \hbox{\hsize=2cm \raggedright
\vbox to 0.5in{\hrule This box is .5in deep. \vfil\hrule}
\qquad
\vbox to 0.75in{\hrule This box is .75in deep. \vfil\hrule}
\qquad
\end{teX}
\end{minipage}
\hspace{0.3cm}
\begin{minipage}{4.5cm}
\hbox{\hsize=2cm \raggedright
\vtop to 0.5in{\hrule \smallskip This box is .5in deep. \vfil\hrule}
\qquad
\vtop to 0.75in{\hrule \smallskip This box is .75in deep. \vfil\hrule}
\qquad}

\hbox{\hsize=2cm \raggedright
\vbox to 0.5in{\hrule \smallskip This box is .5in deep. \vfil\hrule}
\qquad
\vbox to 0.75in{\hrule \smallskip This box is .75in deep. \vfil\hrule}
\qquad}
\end{minipage}

\medskip
\noindent\rule{\linewidth}{0.4pt}
}

\medskip

There are some other special macros defined by Plain TeX that we will only touch briefly here. One of them is \cs{underbar}{\index{Plain!\textbackslash underbar}.
The macro puts its argument into an hbox and underlines it.

\medskip

\vbox{
\noindent\rule{\linewidth}{0.4pt}
\begin{minipage}{4.9cm}
\begin{teX}
 \underbar{1,000,788.22}
\end{teX}
\end{minipage}
\hspace{0.4cm}
\begin{minipage}{4.0cm}
\medskip
\hfill\hfill{}\hspace*{1em}a1,000,700.22 \hfill

\smallskip

\hfill\[\underbar 1,000,788.22 \]\hfill
\end{minipage}

\medskip
\noindent\rule{\linewidth}{0.4pt}
}

\medskip


The \cs{everyvbox} command inserts a series of tokens at the beginning of every |\vbox|.


\medskip

\vbox{
\noindent\rule{\linewidth}{0.4pt}
\begin{minipage}{4.9cm}
\begin{teX}
 \everyvbox{$\bullet$}...
\end{teX}
\end{minipage}
\hspace{0.4cm}
\begin{minipage}{4.0cm}
\begingroup% Without this group, there are tons of problems!
   \everyvbox{$\bullet$}
   \global\setbox1=\vbox{This is a paragraph without an initial indent. It is   \the\hsize\ long lines.}
   \global\setbox2=\vtop{\copy1}
\endgroup
 \hbox{\box1} 

 \hbox{\box2}
\end{minipage}

\medskip
\noindent\rule{\linewidth}{0.4pt}
}

\medskip
Knuth in the TexBook Chapter 24, has some short description of the every commands. The `everyhbox` inserts a token list just before as its name implies a horizontal box.

Here is a short example. We define a `oneLineBox`, which is simply an hbox with some text and we add spread to spread the line. Using |\everybox| we add the letter \textbf{a} in each horizontal box. 


\tex considers the box overfull if the excess width of the box is larger than \cs{hfuzz} or \cs{hbadness} is less than 100. If I change  the badness to hbadness, I get 1000.

\medskip

\vbox{
\noindent\rule{\linewidth}{0.4pt}
\begin{minipage}{10.0cm}
\begin{teX}
 \begingroup
     \everyhbox{a}
     \def\oneLineBox#1#2%
     {%
          \hfuzz=0pt
          \overfullrule=0.25pt
          \setbox0=\hbox spread#2{#1}%
          \setbox1=\hbox{\the\badness}% 
          \setbox2=\hbox to 4.5cm{\box0\hfil\box1}%
          \box2
     }
     \oneLineBox{Badness of line }{-1em}
     \oneLineBox{Badness of line }{-0.54em}
     \oneLineBox{Badness of line }{-0.4em}
     \oneLineBox{Badness of line }{0em}
     \oneLineBox{Badness of line }{1em}
     \oneLineBox{Badness of line }{2em}
     \oneLineBox{Badness of line }{3em}
 \endgroup
\end{teX}
\end{minipage}


\begin{minipage}{10.0cm}
\begingroup
     \everyhbox{a}
     \def\oneLineBox#1#2%
     {%
          \hfuzz=0pt
          \overfullrule=0.25pt
          \setbox0=\hbox spread#2{#1}%
          \setbox1=\hbox{\the\badness}% 
          \setbox2=\hbox to 4.5cm{\box0\hfil\box1}%
          \box2
     }
     \oneLineBox{Badness of line }{-1em}
     \oneLineBox{Badness of line }{-0.54em}
     \oneLineBox{Badness of line }{-0.4em}
     \oneLineBox{Badness of line }{0em}
     \oneLineBox{Badness of line }{1em}
     \oneLineBox{Badness of line }{2em}
     \oneLineBox{Badness of line }{3em}
 \endgroup
\end{minipage}

\medskip
\noindent\rule{\linewidth}{0.4pt}
}

\medskip










\parindent1em




\section{More features of horizontal boxes}

Characters in the Latin alphabet have different shapes, and in most typefaces,
different widths. The letters \texttt{d f h k l t} have ascenders, making them
higher than the vowels \texttt{a e o u}, while the letters \texttt{f g j p q y} have descenders,
giving them added depth below the vowels. Similarly, an \texttt{m} is wider than
an \texttt{i}. 

\drawfontbox{(fjord)}

When \tex makes a normal horizontal box, the box width is the sum
of the widths of the characters, and the fixed parts of any glue, contained
in it. Shrink and stretch components of glue are discarded for the width
calculation. The box also has both a height above the baseline, the invisible
line on which the characters rest, and a depth below the baseline. The
depth is zero if there are no objects with descenders. The height and depth
are chosen from the largest vertical extents of the contained objects.

If you look carefully at typeset material, you will observe that, in most
typefaces, parentheses, brackets, and braces have both descenders and ascenders,
and the typeface designer usually makes their extents the maximum
among all of the characters in the design. This sample text shows
document: ( h g ) [ k j ] { l p }.

You can force TEX to choose a larger height and depth than normal when
you write a command for a horizontal box by ensuring that it has suitable
contents, such as an invisible vertical rule of zero width. The command

\verb+\hbox to 50pt {\vrule height 20pt depth 10pt width 0pt \it stuff}+

produces a box whose (invisible) outline looks like this: 

\hbox to 50pt {\vrule height 20pt depth 10pt width 0pt \it Great}

\drawfontbox{\kern 5pt\vrule height20pt depth 10pt width 1pt fjord}

The
three extents of the vertical rule can appear in any order, and any convenient
units.

In order to see the otherwise-invisible box edges in that example, we
used the \latex  built-in command \cs{fbox} to create a frame, and we eliminated
the default margin inside the frame by setting \cs{fboxsep = 0pt}. Plain \tex
does not have the \cs{fbox} command, but The TEXbook shows how to make
something like it on pp. 223 and 321.

One particular zero-width vertical rule is convenient for ensuring that
separate boxes all get the same height and depth. It has the height and
depth of parentheses in the normal prose font, and is given the macro name \refCom{strut}.
Its definition in the plain.tex file of macro definitions is roughly
equivalent to this:

\begin{docCommand}{strut}{}
\end{docCommand}

\begin{teX}
  \def \strut {\vrule height 8.5pt depth 3.5pt width 0pt}
\end{teX}

We insert a |\vrule| at the figure on the left below with a height of 20pt and a depth of 10pt. You can observe the difference on the right box, without the |\vrule|. The \textit{strut} is the blue line, which we gave a width of one point to make it visible. Real life struts, would have a width of 0pt and will not be visible. 

\drawfontbox{\kern5pt{\color{blue}\vrule height20pt depth 10pt width 1pt} fjord}
\drawfontbox{fjord}



\section{Horizontal alignment of boxes in TEX}
\fboxsep0.4pt

When horizontal boxes are set together, they are treated as separate words,
and therefore spaced accordingly. The input
\verb+ \fbox{one} \fbox{two} \fbox{three} \fbox{four}  +
produces  \fbox{one} \fbox{two} \fbox{three} \fbox{four}. As the example shows, we can put spaces
between them, or run them together so that they fit tightly.


\section{Vertical boxes in TEX}


\begin{minipage}{2.0in}
\begin{verbatim}
\noindent
\fbox{%
  \it
  \hbox to 80pt{%
     \parindent = 0pt
     \vbox to 30pt {%
         left text
         \vfil
         more left text%
     }%
  }%
}%
\end{verbatim}
\end{minipage}


%\noindent
\fbox{%
  \it
  \hbox to 80pt{%
     \parindent = 0pt
     \vbox to 30pt {%
         left text
         \vfil
         more left text%
     }%
  }%
}%

Firstly we use a noindent to ensure that the box is not indented. If you comment the\cs{fbox} out, you can see that the right amount of space has been left in the paragraph above.

\mbox{}
 
\noindent
\fbox{%
\it
\hbox to 80pt{%
\parindent = 0pt
\hsize = 80pt
\vbox to 30pt {\hfill right text
\vfil
\hfill more right text}
}%
}%



\noindent
\fbox{%
\it
\hbox to 80pt{%
\parindent = 0pt
\hsize = 80pt
\vbox to 30pt {\hfil center text
\vfil
 more center text \hfil}
}%
}%

We can aslo center the text for both lines, by modifying the code slightly.
\begin{teX}
\noindent
\fbox{%
\it \hbox to 80pt{
   \parindent = 0pt
   \hsize = 80pt
   \vbox to 30pt {
   center text \hfill
    \vfil
    \hfil more center text}
   }%
}%
\end{teX}


\noindent
\fbox{%
\it
\hbox to 80pt{%
\parindent = 0pt
\hsize = 80pt
\vbox to 30pt {\hfil center text
\vfil
\hfil more center text}
}%
}%



\chapter{Boxes with \protect\LaTeXe}

The \tex primitive commands have been abstracted by \latexe into more user friendly commands that are easier to use. One other reason for using these \LaTeX\ commands is that they are ``color safe''. Later on we will see other possibilities given by the \pkgname{color} or \pkgname{xcolor} package for drawing colored boxes, but we want to recall that the code for |\makebox| and the like has already a protection mechanism for colors, which the primitive commands do not have. \latexe also provides boxes that are self-aware of the width of their contents. For example |\fbox| will frame its contents in an |\hbox|. This simple task is very convoluted to achieve using basic \tex commands. 

\begin{docCommand}{framebox} {\marg{dim}}
One useful box command provided by \latex2e is \cmd{\framebox}. This command builds a box with any material you want to provide it with. The contents of this box are unbreakable, and as far as \tex is concerned it is treated the same way as it would treat a letter. 
\end{docCommand}

\begin{docCommand}{fboxsep}{\marg{dim}}
\end{docCommand}
\begin{docCommand}{fboxrule}{\marg{dim}}
Two associated lengths control the width of the rule and the space around the contents. We can change their default value by using |\setlength{\fboxsep}{0pt}| or just simply |\fboxsep=0pt| or even |\fboxsep0pt|. 
\end{docCommand}


Another interesting property is this: \emph{the contents of a box need not lie inside it}. You may have
noticed that, given the contents as an argument, the
|\framebox| command sets the dimensions of the box
to those of the contents (in reality, to the ``sub-boxes"
that compose the contents). But you can define the
dimensions explicitly as well. For example,

\begin{texexample}{framebox example}{ex:framebox}
|\framebox[13em]{Some text}|

\framebox[13em]{Some text}

\fcolorbox{theblue}{cyan}{Some text}

\end{texexample}

The box as is shown in the example will not break and it occupies more space than its contents. A second optional command allows us to typeset the contents, left, center or right.

\begin{codeexample}[vbox]
\fboxrule1pt

\framebox[13em][l]{Some text}\par

\framebox[13em][r]{Some text}\par

\framebox[13em][c]{Some text}\par

\framebox[1em][l]{Some text}\par

\framebox[1em][c]{Some text}\par

\framebox[1em][r]{Some text}\par
\end{codeexample}

As you can observe \latexe has abstracted the |\hfill| and similar commands and allows boxes to be constructed with ease. We have started the discussion with |\framebox|, but most practical uses of boxes is when they remain invisible.

\begin{docCommand}{makebox} { \oarg{width}\oarg{position}\marg{contents} } 
 is \latex's box workhorse.
 \end{docCommand}

The |source2e| manual states. If the width is missing, then position is also missing and |obj|  is put in an \cs{hbox} of its natural width. This is true as far as the looks are concerned, but not the behaviour, as you can see
from the following example is not an unqualified \cmd{\hbox} it is an hbox preceded by leavevmode.\footnote{\url{http://tex.stackexchange.com/questions/105585/latex2e-makebox-hbox}} This is of course good practice and brings consistency to the LaTeX kernel. I would recommend that you follow such practices in your own code. 

\begin{texexample}{}{}
\newbox\temp
\savebox\temp{test}
LaTeX

\makebox{test} \mbox{test}

TeX

\hbox{test} \hbox{test}

\indent\hbox{test} \hbox{test}

LaTeX with \cs{leavemode}

\makeatletter
\leavevmode\hbox to \wd\temp{test} \indent\hbox to \wd\temp{test}
\makeatother
\end{texexample}



\latex's analog of a\cs{hbox} is called \cs{mbox}. They are 
much the same thing, but \cs{mbox} is defined to be more widely usable. We have already used \latex's framed companion to \cs{mbox}, \cs{fbox}.

A horizontal box of specified width is provided in \latex with the command
\doccmd{makebox[width][position]\{contents\}}. Bracketed command arguments
in \latex are always optional. 

Here, the width is a \tex dimension,
and defaults to the natural width of the contents if not given. The position
is one of the letters \textbf{l} (flush left) or \textbf{r} (flush right); if it is omitted, the text
is centered in the box. If the specified width is smaller than needed, the
contents protrude from the box, and may overlap surrounding material. If
the specified width is zero, then we have equivalents of the TEX \cs{rlap} and
\cs{llap} commands.


Here are several examples of these three LATEX box commands:

{\obeylines
\mbox{stuff}

\fbox{stuff} 

|\makebox{stuff}|

|\makebox[40pt][l]{stuff}|

|\makebox[40pt][r]{stuff}|

|\makebox[0pt]{stuff}|

|\makebox[0pt][l]{stuff}|

|\makebox[0pt][r]{stuff}|
}



\subsection{Positioning boxes}

To help in positioning boxes within other objects, \latex provides the command
\docAuxCmd{raisebox} to raise and lower boxes:

\begin{teX}
\raisebox{raiselength}[height][depth]{contents}
\end{teX}

A negative first argument lowers the box, where the \cmd{\lowerbox} will lower the box. Here are some examples:

\begin{texexample}{Raising and lowering boxes}{ex:raise}
A \raisebox{10pt}{\fbox{upper}} A
upper
A \raisebox{10pt}{\
fbox{lower}} A
lower
A \fbox{\raisebox{10pt}[25pt]{\fbox{upper}}} A
upper
A \fbox{\raisebox{10pt}[
25pt]{\fbox{lower}}} A
lower
A \fbox{\raisebox{10pt}[25pt][15pt]{\fbox{upper}}} A
upper
A \fbox{\raisebox{10pt}[
25pt][15pt]{\fbox{lower}}} A
lower
\end{texexample}

\section{Paragraph Boxes}

\begin{docCommand}{parbox}{\oarg{position}\oarg{height}\oarg{innerpos}\marg{width}\marg{contents} }
  For longer strings of text, \latex provides the paragraph box \cs{parbox} 
\end{docCommand}


The optional position
is a letter \textbf{b} for alignment of the bottomline with the current baseline,
or \textbf{t} for alignment of the top line with the surrounding baseline. Without

The box can be used as if it were a letter or a word, so we can put it in
the middle of a sentence. The input

This is text \parbox{30pt}{\it and this is boxed text} and
this is more text.

This is text \fbox{\parbox{30pt}{\it and this is boxed text}}
and this is more text.
produces


Flush-right typesetting generally looks bad in narrow columns, so we
can insert a \cs{raggedright} command inside the last argument of the paragraph
box to get output like this:

\begin{texexample}{}{}

\parbox[b][120pt][t]{130pt}{\lorem}%
\hspace{1cm}%
\parbox[b][150pt][t]{130pt}{Only some short line of text here.}%



\parbox[b][120pt][t]{130pt}{\lorem}\hspace{1cm}\parbox[b][120pt][c]{130pt}{Only some short line of text here.}

\end{texexample}


\section{The minipage environment}

Another kind of paragraph box can be obtained in a more general, and
more powerful, way with the \docAuxEnv{minipage} environment:

\emphasis{minipage}
\begin{phdverbatim}
\begin{minipage}[position]{width}
   contents
\end{minipage}   
\end{phdverbatim}


The positioning works just like that for \verb+\parbox+, with alignment letters \textbf{b}
and \textbf{t}, and if they are omitted, a default of vertical centering.
In particular, verbatim text produced with the verb command is illegal
in macro arguments, so it cannot be used with \cs{fbox}, \cs{framebox}, \cs{makebox},
\cs{mbox}, or\cs{ parbox}, but it can be used inside a minipage. The input


\begin{texexample}{}{}
\begin{minipage}{170pt}
This is inline verbatim \verb=\verb|\%{}|=, and this
is a verbatim display:

\begin{verbatim}
#include <stdio.h>
#include <stdlib.h>
int main(void)
{
  printf("Hello, world\n");
  exit (EXIT_SUCCESS);
}
\end{verbatim}
\end{minipage}

\end{texexample}


A minipage can go everywhere and can hold virtually any content.


\section{Scaling and resizing boxes}

\begin{docCommand}{resizebox}{\marg{width}}{\marg{general material}}
Resizes the contents of a box
\end{docCommand}

The command \cs{resizebox}\marg{width}\marg{height}\marg{object} can be used with tabular to specify the height and width of a table. The following example shows how to resize a table to 8cm width while maintaining the original width/height ratio.

\begin{teX}
\resizebox{8cm}{!} {
  \begin{tabular}...
  \end{tabular}
}
\end{teX}

Alternatively you can use \cs{scalebox}{ratio}{object} in the same way but with ratios rather than fixed sizes:

\begin{teX}
\scalebox{0.7}{
  \begin{tabular}...
  \end{tabular}
}
\end{teX}

Both |\resizebox| and |\scalebox| require the \pkg{graphicx}\footfullcite{graphicx} package.
To tweak the space between columns (LaTeX will by default chose very tight columns), one can alter the column separation: |\setlength{\tabcolsep}{5pt}|. The default value is |6pt|.

The scalebox is great if you want to magnify a letter so that you can observe the design closer.

\bigskip
\noindent\begin{tabular}{|c|c|c|c|c|c|}\hline
Kp-Fonts & Kp-\textit{light} & CM & Palatino & Utopia & Times\\\hline\hline
\scalebox{2}{ag713} &
\scalebox{2}{\fontfamily{jkpl}\selectfont 7} &
\scalebox{2}{\fontfamily{lmr}\selectfont 713}  &
\scalebox{2}{\fontfamily{ppl}\selectfont 713}  &
\scalebox{2}{\fontfamily{put}\selectfont 7} &
\scalebox{2}{\fontfamily{ptm}\selectfont \oldstylenums{7}} \\\hline
\end{tabular}


\begin{teX}
\hspace{-6mm}\begin{tabular}{|c|c|c|c|c|c|}\hline
Kp-Fonts & Kp-\textit{light} & CM & Palatino & Utopia & Times\\
\hline\hline
  \scalebox{10}{a} &
  \scalebox{10}{\fontfamily{jkpl}\selectfont a} &
  \scalebox{10}{\fontfamily{lmr}\selectfont a}  &
  \scalebox{10}{\fontfamily{ppl}\selectfont 7}  &
  \scalebox{9.2}{\rule{0pt}{1.25ex}\fontfamily{put}\selectfont a} &
  \scalebox{10}{\fontfamily{ptm}\selectfont a}\\\hline
\end{tabular}
\end{teX}
\bigskip



\section{Glues with Negative and zero dimensions}

A box with a natural size of zero with the right glue amount can become very useful. For example the glue
|0pt plus1fil minus1fil| can stretch to infinity and also shring to minus infinity. Of course in the case of
\tex infinity is \docAuxCommand{maxdimen}. A \tex primitive is defined with this glue \refCom{hss}.

\begin{docCommand}{hss}{}
\end{docCommand}

There is also a corresponding \refCom{vss}.

\begin{docCommand}{vss}{}
\end{docCommand}


These macros place text on a full line either centred or left or right adjusted.

\begin{texexample}{}{}
\makeatletter
368 \def\@@line{\hb@xt@\hsize}
369 \def\leftline#1{\@@line{#1\hss}}
370 \def\rightline#1{\@@line{\hss#1}}
371 \def\centerline#1{\@@line{\hss#1\hss}}
\rlap
\llap
These macros place text to the left or right of the current reference point without
taking up space.
372 \def\rlap#1{\hb@xt@\z@{#1\hss}}
373 \def\llap#1{\hb@xt@\z@{\hss#1}}

$a\mathrel{\rlap{\;/}{=}}b $

{\Huge
\leavevmode
\rlap{Y}L
\rlap{C}\kern2.6pt\lower3.5pt\hbox{,}
}
\makeatother
\end{texexample}

\begin{docCommand}{rlap}{\marg{material}}

\end{docCommand}

Of course neither |llap| or |rlap| start a paragraph, so we need to use a |leavevmode| or one of the other ways to start a paragraph.

\begin{docCommand}{llap}{\marg{material}}
\end{docCommand}


\begin{docCommand}{smash}{\marg{material}}
The |\smash| command typesets the material with a height and depth of zero.
\end{docCommand}

\begin{docCommand}{phantom}{\meta{material}}
\end{docCommand}

\begin{docCommand}{vphantom}{\meta{material}}
\end{docCommand}

\begin{texexample}{Defining smash}{}
\bgroup
\def\smash{%
   \relax % \relax, in case this comes first in \halign
   \ifmmode
   \expandafter\mathpalette\expandafter\mathsm@sh
   \else
    \expandafter\makesm@sh
   \fi}
   
\def\makesm@sh#1{%
   \setbox\z@\hbox{\color@begingroup#1\color@endgroup}\finsm@sh}

\def\mathsm@sh#1#2{%
   \setbox\z@\hbox{$\m@th#1{#2}$}\finsm@sh}

\def\finsm@sh{\ht\z@\z@ \dp\z@\z@ \box\z@}
\egroup

\vbox{\smash {\hbox{A} } \hbox{B}} Test

\end{texexample}

\cxset{geometry units = pt,
       fontbox font=\Huge\upshape}

  


Consider the letters `Q' and `P', shown below. The capital letter `Q' has a depth of 1.72mm, we might wish to smash
it in a two line title block to reduce the line spacing between two consecutive lines. This can be accomplished with the
\refCom{smash} command.

\centerline{\drawfontbox{Q} \drawfontbox{P}}

Smashing it produces the following results.

\centerline{\drawfontbox{\vbox{\smash {\hbox{Q} } \hbox{P}}}  \drawfontbox{\vbox{\hbox{Q}  \hbox{P}}}}  

The command is more useful in math environments and is used extensively both by authors and package developers.

\begin{teXXX}
\def\rightarrowfill{$\m@th\smash-\mkern-7mu%
454 \cleaders\hbox{$\mkern-2mu\smash-\mkern-2mu$}\hfill
455 \mkern-7mu\mathord\rightarrow$}

456 \def\leftarrowfill{$\m@th\mathord\leftarrow\mkern-7mu%
457 \cleaders\hbox{$\mkern-2mu\smash-\mkern-2mu$}\hfill
458 \mkern-7mu\smash-$}
\end{teXXX}

Two further macros can be useful to authors of mathematical documents, \docAuxCommand*{phantom} and \docAuxCommand*{vphantom}. 

When typesetting roots, sometimes there are issues with heights. The following example
from \citetitle{mathmode}\footcite{mathmode} illustrates the point.

\begin{equation}
 \sqrt{a}\,%
 \sqrt{T}\,%
 \sqrt{2\alpha k_{B_1}T^i}\label{eq:root1}
\end{equation}

This can be corrected using \refCom{vphantom}. 

\begin{texexample}{Correcting height issues}{ex:sqrtheights}
\begin{equation}\label{eq:root2}
 \sqrt{a\vphantom{k_{B_1}T^i}}\,%
 \sqrt{T\vphantom{k_{B_1}T^i}}\,%
 \sqrt{2\alpha k_{B_1}T^i}
\end{equation}

\begin{equation}
x = \sqrt[3]{6+\sqrt[3]{6+\sqrt[3]{6+\sqrt[3]{6+\cdots}}}}
\end{equation}
\end{texexample}

Using \pkgname{amsmath} \docAuxCommand{smash} can be used for even better results when
using inline or displayed roots. It must be noted that \cs{smash} in \latexe is defined
without such an optional argument.



\makeatletter
\renewcommand{\smash}[1][tb]{%
\def\mb@t{\ht}\def\mb@b{\dp}\def\mb@tb{\ht\z@\z@\dp}%
\edef\finsm@sh{\csname mb@#1\endcsname\z@\z@ \box\z@}%
\ifmmode \@xp\mathpalette\@xp\mathsm@sh
\else \@xp\makesm@sh
\fi
}
\makeatother
This is a test $\sqrt{\lambda_{ki}}$ and $\smash[tb]{\sqrt{\lambda_{ki}}} $ 
\meaning\smash

\begin{docCommand}{smash}{ \oarg{position}\marg{argument} }
The optional argument for the position can take three values: \textbf{t} keeps the bottom and annihilates the top, \textbf{b} keeps the top and annihilates the bottom and \textbf{tb} which annihilates top and bottom. The latter is the default.
\end{docCommand}

\begin{texexample}{Use of Amsmath smash}{ex:amssmash}
xxx
\fbox{\rule{0.5cm}{2cm}}
\fbox{\rule[-1cm]{0.5cm}{2cm}}
\fbox{\smash{\rule{0.5cm}{2cm}}}
\fbox{\smash{\rule[-1cm]{0.5cm}{2cm}}}
\fbox{\raisebox{0pt}[0pt][0pt]{\rule[-1cm]{0.5cm}{2cm}}}
\fbox{\raisebox{-1cm}[0pt][0pt]{\rule{0.5cm}{2cm}}}
\end{texexample}


\begin{texexample}{The array environment}{ex:array2}
Thus to change $\frac34$ to a decimal divide $4$ into $3$
and we get $.75$ as a result, thus:
\[
\begin{array}{r@{}r@{}}
4 \; & \vline \; 3.00 \\\cline{2-2}
     &            .75
\end{array}
\]
To find the square root of a four-figure number
such as our example calls for, work it out in the
following manner:


\[
\arraycolsep=0em
\begin{array}{cccccccccccc}
\multicolumn{3}{c}{\text{2d pair}} &\qquad&\qquad&
\multicolumn{3}{c}{\text{1st pair}}&\qquad&\qquad&
\multicolumn{2}{c}{\text{square root}}\\
 & \overbrace{\quad}&\ZZZ&&&\ZZZ&\overbrace{\quad}&\ZZZ\\
 & 42 &&&&& 25 &&&&\vline\;65&(answer)\\\cline{11-11}
 & 36 &&&&& \\\cline{2-2}
\multirow{2}{*}{125\:} & \vline\hfill \phantom{Z}6 \hfill&&&&& 25\\
 & \vline\hfill \Zi6 \hfill&&&&& 25\\\cline{2-7}
\end{array}
\]
\end{texexample}

What I provided as an easy mnemonic for the \pkg{phd} I provided macros |\Zi, \ZZ, \ZZZ| as convenience aliases for
|\phantom{Z}| etc.

With graphic programs becoming available, most of the drawing of small complicated boxes, has been overtaken by using
\tikzname and especially its option to overlay a node at a particular point of the page without any impact on the spacing.










% \end{document}

%   \begin{epigraphpage}
 \epigraph{Begin at the beginning,'' the King said, gravely, ``Then
 go till you come to the end; then stop.''}{Lewis Carroll, {\it Alice
 in Wonderland}}

 \epigraph{You can never get a cup of tea large enough or a book long enough to
 suit me''}{C. S. Lewis}
 \end{epigraphpage}

\parindent1em
%\cxset{style13}
%\cxset{title margin bottom=10pt,
%          title beforeskip=1pt}

\chapter{Introduction}
\addtocimage{-12pt}{-20pt}{../images/tocblock-fish}


\epigraph{``Begin at the beginning,'' the king said
"and then go on till you come to the end, then stop."}{
---Lewis Carroll, Alice in Wonderland}

 \parskip3pt plus 5pt 
\noindent This package and its documentation attempts to eliminate some common 
problems encountered when using \LaTeX2e. The first one is the loading of 
recommended packages for a large and perhaps complicated document and 
the second is the re-designing of styles for a document.

 \LaTeX2e, does not provide a standard library, but comes equipped with
 a package mechanism that allows code extensions to be loaded as required.
 This has created a strong vibrant community, hundreds of packages and a 
 headache to both new and seasoned users. What packages are available, when
 to use them and in which order is a common theme for many questions on
 lists and |TX.SE|.

 It is quite common during the writing of a thesis or book
 for the author to keep on adding macros and packages
 at the preamble of the document. In most cases this can
 be satisfactory but in many others it leads to
 incompatibilities and errors. This package aims at
 minimizing one's preamble, by prefetching a number of
 commonly used packages. It also aims at loading them
 in the right order and providing patches for conflicts.
 
 I am hoping that using this package, will lead to less
 frustrations with the intricacies of \LaTeX2e\ packages.

The package code is complicated, but its usage is simple. You first load the package and then
you use one of the available templates:

 \begin{commands}[]{}
 \begin{verbatim}
 \usepackage{phd}
 \usetemplate{style13}
 \end{verbatim}
 \end{commands}

This is what you need to typeset a good looking book or thesis. The rest of this book is a footnote and you can skip them if you want. 

It will be better for the longer projects to just fork the
 package and adapt it to your needs. In this respect, I have
 uploaded the package to |github|.\footnote{\url{https://github.com/yannisl/phd}}

 My goal in selecting the packages and adding a number of 
 commands for the authors was to be able to typeset a 
 document for most common use cases, without the need of
 additional packages. The packages I selected are biased
 towards academic publications, although they can find use
 in almost any fields. The package provides a mechanism via
 PGF keys to provide a settings file. 
 
 Most of the documentation can be found in the implementation part.

Browse any books in a library or bookshop and the striking thing is that their design is very individualistic. They might have similarities but their main features vary. In many respects they resemble people's faces where minor differences have striking effects.

This package arose out of a question at stackexchange. How to redefine chapter heads. Having seen the popularity of the |pgf| package \cite{pkg-pgf} I realized that \latex users prefer this method of styling rather the traditional \latex method.

The user interface can be extended to basically all major packages. The principle is to keep to a minimum changes that can affect the LaTeX core commands. If there are any additions a key setting is provided to be able to revert back to normal LaTeX.

The workflow can be simplified. In addition I want to believe that the interface can provide a useful addition to the open source community and that other people will contribute style libraries, which will be simpler to write. It is also possible
to device an easy and uncomplicated web interface to handle
such a great number of variables.


Most people when they get started with \LaTeX\ will either use one of the standard classes such as the \docFile{book.cls} or one of the generic classes notably koma-script or memoir. Most students will be forced to use on of the many thesis classes available.

\section{The key value concept}

The key-value concept that originated with \LaTeX\ has been extended many times, the last and most serious implementation of it by Tantau in the PGF package. What essentially Tantau developed is a scripting language to script TeX code. The \tikzname and pgfplots packages are two major packaged that use keys effectively. Their popularity is growing and what this package does is to offer a user interface that has been modelled to be similar to that of \texttt{css} (cascade style sheets). 
\smallskip

\begin{scriptexample}{}{}
\textit{chapter number} font-size = Large,\\
\textit{chapter number}     color = theblue
\end{scriptexample}
\smallskip

The main idea behind the package, is that you are configuring a document style by means of \emph{settings} rather than writing macros. In the example above the \emph{number, chapter} can be thought of as class or id names in css style sheets and the |font-size, color| as property settings that apply to the particular element. 


\subsection{Settings}

Settings are activated either by using the command |\cxset|  or by loading a full style sheet. In most cases you will probably import a style sheet and then modify some of the properties using |cxset|.  For example this heading has a dot after the subsection number. This was accomplished by setting,

We can de-activate it for the next and subsequent subsection headings with the setting:

\lorem

\begin{scriptexample}{}{}
\begin{verbatim}
\cxset{subsection number after=\quad}
\end{verbatim}
\end{scriptexample}




\subsection{Cascading}

Most values once set for a higher section will be seen in a cascade by all subsectioning commands in a similar fashion similar to CSS. These include properties such as color, font families and alignment. Best though to specify all of them for maximum flexibility to your users.

\section{On typography}

This package hopefully will assist in improving the typography of books set with \latexe. Any typographical comments on the various styles are just my own ramblingss and not necessarily absolute truths. Like fashion and art typography has opinions rather than absolute truths. In many styles the design is slightly adapted to blend a bit better with this manual. Also I did not select fonts as per the samples but this is left on you the user to decide.



\section{Packages and Fonts}

This manual has been typeset with numerous fonts in order to enable the typsetting of almost all the scripts provided by the Unicode standard. In order to process it from the |.dtx| file, these fonts must be available in your system, otherwise \XeLaTeX\ will have a problem finding the fonts and it will take an awful long time to process. This is especially true for the scripts section, where virtually all the Unicode defined scripts are discussed. You will need a fast computer and a fast hard disk to process the document within a reasonable time. When using \pkg{fontspec} always define your fonts with the \cmd{\newfontfamily} this will speed up processing by an order of magnitude. Compiling from the command prompt will speed up compilation. Average speed 2-3 pages per second.

Many of \tex's parameters are stretched to the limit with a complicated document such as this manual. You will require a full distribution otherwise expect some errors. Important packages is \pkg{morefloats} and \pkg{morewrites}. The package will also expect that you have |e-tex| installed. Ubuntu users are normally one year behind in updates, so you might wish to update manually. It will take upwards of 5 minutes to compile fully on an old laptop and a couple of minutes on a state of the art computer.

The |dtx| should be processed best with its own make file provided for Windows only |phd-lua.bat|. The make file will process the documentation using \lualatex. You can also process the document with \xelatex but is prone to produce errors. Using \latexe the sections on scripts etc will not be printed and a much shorter version of the manual is provided. 

\section{Scripts and Languages}

The package and the documentation offer a full repertoire of font selection keys for different scripts and languages. It hasn't been possible, however hard I tried to compile this section of the documentation with \xelatex, as it kept giving errors of too many files open. This was also not possible even with the \pkg{morewrites} package loaded. With \lualatex the document compiled with no major problems other than the font rendering being of a lower quality to that of XeLaTeX on windows, other than disabling incompatible packages and a number of commands that were redefined. 

Some good news for multi-script typesetting is the |Noto| fonts from Google. These fonts named Noto from "No Tofu" meaning you do not see any little square blocks for undefined glyphs, are fast to load. Disantvantage you need to switch between font commands fairly often.

\section{This book}

When developing the templates, I started using \emph{lorem ipsum} text as samples. Half-way through this
became a jumble mass of uninteresting pages interspersed with code. Headings and the contents of the book
determine both the structure and the selection of fonts, so I went back and wrote narratives  to accompany
the headings. Many of the narratives are semi-autobiographical in nature; others are clustered around books I read and my own interests. Some I stumbled on them accidentally and are mostly there to demonstrate some code.

Besides the templates and the code there is another narrative which is based on notes I kept on \tex and its friends over the years and are offered as a more advanced introduction to coding \latexe and \tex. The whole manual was typeset in a |ltxdoc| class, slightly modified to turn into a book class.

The implementation code is also available and it was mostly for my own benefit. The whole manual with the exception of the |\cxset| introduction, is just a test document. The notes and the “dissection” of the standard \latexe and the standard classes are there to explain the background to the many coding decisions that I took while I was developing the package.

PhD students are notorious for going in all directions and exploring many adjacent fields before they sit down and write their theses. Some become life-time students. To all these new men and women of the Renaissance that slave away to inch knowledge one thesis at a time, I dedicate this book and the name of the package.

\subsection{The TeX hacking sections}

To start programming \tex you need to have a knowldge of \tex basic commands and approach. \latex2015 is a format build on top of \tex to provide a more structured approach. To program \latexe packages you need to understand \latexe concepts, code organization and conventions. To program in \latex3, you need to learn a whole new language and you still need to understand \tex, \latexe and the expl3 language and conventions. To program using LuaTeX, other than the Lua language you need to understand \tex very well.
None of these can be found in one place.  I have gathered a lot of material and put it together. This is not a language you can master easily or quickly, but can teach you a lot about typesetting, computer science and many other interesting topics.


 \section{Version control with Git and Github}
 
 If you are involved with code or a publication that will have frequent changes, you should consider
 some type of version control system. My own recommendation is to use |git| and an online repository such
 as |github|. The latter is currently very fashionable and makes sharing code easier. Note that the |github|
 offers both public as well as private repositories. The general recommendation is that for unpublished work
 such as a thesis or code under development, it is preferable to go for a private repository. 
 
 \lorem\lorem

 \section{Ordering of Packages}
 
One package that normally leads to errors is the 
\pkg{hyperref}. The package which is an outstanding example of software engineering and supported single handledy by Heiko Oberdiek\footcite{hyperref} redefines a a lot of internal commands of the kernel. As a lot of other packages do the same it has to be loaded at the end of the preable with the exception of some packages! 
 
 This manual is typeset according to the conventions of the
 \LaTeX \textsc{docstrip} utility which enables the automatic
 extraction of the \LaTeX{} macro source files~\cite{GOOSSENS94}.

 
 \href{http://tex.stackexchange.com/questions/96350/problem-with-algorithmic-and-hyperref}{problem with algorithmic and hyperref}

 \begin{verbatim}
\usepackage{float}  % load float package first!

\usepackage{hyperref} % let hyperref patch the float package stuff
.
 \usepackage{algorithm} % let algorithm use the patched version of the float package
 \end{verbatim}
 

\section{Known problems}

Perhaps the biggest issue with the package is the speed of
compilation with \XeLaTeX\ or \LuaTeX. This is to be expected, as both engines spend a lot of resources in font management. On demand loading of packages is something I have in the back of my mind. This should be done via document styles i.e., if a book is for the humanities, perhaps only a rudimentary amount of maths packages should be loaded.

\section{Future Directions}

\latexe and \tex usage appears to be increasing. This is mostly by programs that export results with \latexe code rather than authors writing books.  The method adopted here is easier to automate all sorts of reports and automated texts. I would like too develop a web interface for processing such templates and at the same time export into html instead of just producing pdfs. I have already a prototype.   

\section{Tooling}

Some of the scripts on a Windows machine need MSYS\footnote{\url{http://mingw.org/wiki/MSYS}}









%  \newtcolorbox{scriptexample}[2][shavian]{colback=graphicbackground,
boxrule=0pt,toprule=0pt,colframe=white}


\chapter{Those Other Languages}
\minitoc
\parindent1em

\pagestyle{myheadings}

Probably there are more users of \latexe whose mother tongue is not English than those who speak the language. \tex out of the box does not offer facilities for using non-latin based scripts easily; presents numerous problems. The biggest problem---which has been solved to a large extent---was the entering of text without having to mark all the special
characters such as umlauts (\"o) with commands. The second issue and which has been addressed by packages such as Babel, is redefining the strings such as "Chapter" to another language. In software this is called internationalization and a governing standard is |i18n|. None of the current packages take such an approach and none of them as yet offer a satisfactory solution for |LuaLaTeX|. 

Another issue with writing systems and scripts is that of appropriate fonts. Most writing systems that have ever existed are now extinct. Only minute vestiges of one of the most ancient - Egyptian hieroglyphs - live on, unrecognized, in the Latin alphabet in which English, among hundreds of other languages, is conveyed today. The latin \textit{m}, for example, ultimately derives from the Egyptian's cononantal n-sign, depicting waves.

Many of the scripts have other peculiarities, some languages such as Hanunó'o is written vertically from bottom to top. Others from top to bottom and many others from right to left. 

\section{TeX's support for different languages}

TeX's primitives such as \cmd{\language}=\meta{number} can be used to store hyphenation patterns and exceptions for up to 256 different languages. This primitive is then used by TeX to apply an appropriate set of hyphenation rules for each paragraph or part of a paragraph in a document\footnote{\url{http://www.tug.org/utilities/plain/cseq.html language-rp}}. When TeX begins a ne paragraph it sets the \emph{current language} to \cmd{\language}. Just before it adds each new character to the paragraph in unrestricted horizontal mode, it compares the current language to \cmd{\language}. If they are different, TeX : a) changes the current language to \cmd{\language}; b) inserts a whatsit\index{whatsit>language} containing the new language and the values of |\lefthyphenmin| and |\righthyphenmin|; and c) inserts the character. The |\setlanguage| command should be used to change languages in restricted horizontal mode (i.e., inside an |\hbox|). If \meta{number} is less than 0 or greater than 255, 0 is used [455]. Plain TeX has a |\newlanguage| command which may be used to allocate numbers for languages [347]. Changes made to |\language| are local to the group containing the change 

\section{LaTeX}

As far as hyphenation patterns are concerned \latexe follows very closely to the methods employed by \tex and Plain Tex. In the source2e the File |lthyphen.dtx| describes the approach to loading the default file |hyphen.ltx| . If a file hyphen.cfg is found \latexe will load the appropriate hyphenaion patterns. Traditionally language management was achieved via Johan 
Braams package Babel which we describe in the next section.


\section{The Babel Package} 

Babel \citet{babel} was the first package to systematically offer foreign language
support for \tex. It has been updated for use with |XeTeX| and |LuaTeX| and provides an environment
in which documents can be typeset in a language
other than US English, or in more than one language
or script. However, no attempt has been done to
take full advantage of the features provided by the
latter, which would require a completely new core
(as for example polyglossia or as part of \latex3).

The package has a number of predefined language files with the extension |ldf|. 


\Describe\selectlanguage{\marg{language}}{}
When a user wants to switch from one language to another he can
do so using the macro |\selectlanguage|. This macro takes the
language, defined previously by a language definition file, as
its argument. It calls several macros that should be defined in
the language definition files to activate the special definitions
for the language chosen. For ``historical reasons'', a macro name is
converted to a language name without the leading |\|; in other words,
the two following declarations are equivalent:
\begin{verbatim}
\selectlanguage{german}
\selectlanguage{\german}
\end{verbatim}

\Describe\foreignlanguage{\marg{language}\marg{text}}
The command |\foreignlanguage| takes two arguments; the second
argument is a phrase to be typeset according to the rules of the
language named in its first argument. This command (1) only
switches the extra definitions and the hyphenation rules for the
language, \emph{not} the names and dates, (2) does not send
information about the language to auxiliary files (i.e., the
surrounding language is still in force), and (3) it works even if
the language has not been set as package option (but in such a
case it only sets the hyphenation patterns and a warning is shown).

\Describe{otherlanguage*}%
{\marg{language}{otherlanguage*}}

Same as |\foreignlanguage| but as environment. Spaces after the
environment are \textit{not} ignored.



\section{The Polyglossia package}

The \pkgname{polyglossia} package has a lot of potential and has solved many issues
but its integration with large parts of the traditional |pdfLaTeX| world
is still under development and will probably take a while before one could
declare it easy to use and bug free. For example anything with the |bidi| package has issues with loading orders for a number of packages and least of which is with
the Ams packages. So if you are going to mix a number of languages in a \XeTeX\ document
you need to take extra care.

 Polyglossia is a package for facilitating multilingual typesetting with
 \XeLaTeX\ and (at an early stage) \LuaLaTeX.  Basically, it
 can be used as a replacement of \pkg{babel} for performing the following
 tasks automatically:
 
 \begin{enumerate}
 \item Loading the appropriate hyphenation patterns.
 \item Setting the script and language tags of the current font (if possible and
       available), via the package \pkg{fontspec}.
 \item Switching to a font assigned by the user to a particular script or language.
 \item Adjusting some typographical conventions according to the current language
       (such as afterindent, frenchindent, spaces before or after punctuation marks,
       etc.).
 \item Redefining all document strings (like chapter, “figure”, “bibliography”).
 \item Adapting the formatting of dates (for non-Gregorian calendars via external
       packages bundled with polyglossia: currently the Hebrew, Islamic and Farsi
       calendars are supported).
 \item For languages that have their own numbering system, modifying the formatting
       of numbers appropriately (this also includes redefining the alphabetic sequence
       for non-Latin alphabets).\footnote{ %
         For the Arabic script this is now done by the bundled package \pkg{arabicnumbers}.}
 \item Ensuring proper directionality if the document contains languages
       that are written from right to left (via the package \pkg{bidi},
       available separately).
 \end{enumerate}
 
 Several features of \pkg{babel} that do not make sense in the \XeTeX\ world (like font
 encodings, shorthands, etc.) are not supported.
 Generally speaking, \pkg{polyglossia} aims to remain as compatible as possible
 with the fundamental features of \pkg{babel} while being cleaner, light-weight,
 and modern. The package \pkg{antomega} has been very beneficial in our attempt to
 reach this objective.


\section{Loading language definition files}

The recommended way of \pkg{polyglossia} to load language definition files
is given in the manual as:
 
\Describe{\setdefaultlanguage}{\oarg{options}\marg{lang}}
 (or equivalently \cmd\setmainlanguage).
 Secondary languages can be loaded with

\Describe{\setotherlanguage}{\oarg{options}\marg{lang}}
 These commands have the advantage of being explicit and of allowing you to set
 language-specific options.\footnote{ %
 More on language-specific options below.}
 It is also possible to load a series of secondary languages at once using

\Describe\setotherlanguages{\marg{lang1,lang2,lang3,\ldots}}

 Language-specific options can be set or changed at any time by means of
\Describe\setkeys{\marg{lang}\marg{opt1=value1,opt2=value2,\ldots}}

\subsection{Bidirectional languages}





\begin{comment}
\begin{Arabic}
ّ هو إذ الغاية؛ شريف الفوائد، جم المذهب، عزيز فنّ التاريخ فنّ أنّ اعلم
والملوك سيرهم، في والأنبياء أخلاقهم، في الأمم من الماضين أحوال على يوقفنا
ّ أحوال في يرومه لمن ذلك في الإقتداء فائدة تتم حتّى وسياستهم؛ دولهم في
والدنيا. الدين
\end{Arabic}
\end{comment}

The Greek language is represented both in modern Greek as well as its ancient variants.

\begin{verbatim}
\begin{greek}
\textbf{Η ελληνική γλώσσα} είναι μία από τις ινδοευρωπαϊκές γλώσσες, για την
οποία έχουμε γραπτά κείμενα από τον 15ο αιώνα π.Χ. μέχρι σήμερα. Αποτελεί το
μοναδικό μέλος ενός κλάδου της ινδοευρωπαϊκής οικογένειας γλωσσών. Ανήκει
επίσης στον βαλκανικό γλωσσικό δεσμό.\\	
(\today) 
\end{greek}
\end{verbatim}

\topline

\textbf{Η ελληνική γλώσσα} είναι μία από τις ινδοευρωπαϊκές γλώσσες, για την
οποία έχουμε γραπτά κείμενα από τον 15ο αιώνα π.Χ. μέχρι σήμερα. Αποτελεί το
μοναδικό μέλος ενός κλάδου της ινδοευρωπαϊκής οικογένειας γλωσσών. Ανήκει
επίσης στον βαλκανικό γλωσσικό δεσμό.\\	
(\today) 

\bottomline

\begin{verbatim}
\begin{russian}
\textbf{Русский язык} — один из восточнославянских языков, один из 
крупнейших языков мира, в том числе самый распространённый из славянских
языков и самый распространённый язык Европы, как географически, так и по
числу носителей языка как родного (хотя значительная, и географически бо́
льшая, часть русского языкового ареала находится в Азии).	\\
(\today)
\end{russian}
\end{verbatim}



\textbf{Русский язык} — один из восточнославянских языков, один из крупнейших языков мира, в том числе самый распространённый из славянских языков и самый распространённый язык Европы, как географически, так и по числу носителей языка как родного (хотя значительная, и географически бо́льшая, часть русского языкового ареала находится в Азии).	\\
(\today)


\section{The Translator package}

The \pkgname{translator} package was developed by \person{Till Tantau} \citep{translator}. It provides a flexible
mechanism for translating individual words into different languages.
For example, it can be used to translate a word like ``figure'' into,
say, the German word ``Abbildung''. Such a translation mechanism is
useful when the author of some package would like to localize the
package such that texts are correctly translated into the language
preferred by the user. The translator package is \emph{not} intended
to be used to automatically translate more than a few words. 

You may wonder whether the translator package is really necessary
since there is the (very nice) |babel| package available for
\LaTeX. This package already provides translations for words like
``figure''. Unfortunately, the architecture of the babel package was
designed in such a way that there is no way of adding translations of
new words to the (very short) list of translations directly build into
babel.

The translator package was specifically designed to allow an easy
extension of the vocabulary. It is both possible to add new words that
should be translated and translations of these words.

\subsection{Using the Translator Package}

  The \pkg{Translator} needs to be used with Babel and I am not too sure yet 
  if it is ready  to be used with Polyglossia.

Once the package has loaded a language or a set of languages the optional argument to the
\cmd{\translate} can be used to translate a string. 

\begin{texexample}{Translating strings}{ex:translator}
  \translate[to=german]{rightpagename}
  \translate[to=dutch]{rightpagename}
\end{texexample}

Before you can provide the translations you need to provide your own dictionaries, where you require them. These need to be installed at a place where \tex can find them.

\CMDI{\ProvidesDictionary}

The dictionary has to be saved in a specific format that relates to the \cmd{\ProvidesDictionary} command. The second argument of the command must be appended to the file name; for the example the file is saved as\footnote{This  example is from the translator package bundle and is under the folder \texttt{base}}:

|translator-basic-dictionary-German|

The concepts take a bit of time to sink in, but once you have everything set up, it is quite easy and straight forward to incorporate it, into your package. 

\begin{teXXX}
\ProvidesDictionary{translator-basic-dictionary}{German}

\providetranslation{Abstract}{Zusammenfassung}
\providetranslation{Addresses}{Adressen}
\providetranslation{addresses}{Adressen}
\providetranslation{Address}{Adresse}
\providetranslation{address}{Adresse}
\providetranslation{and}{und}
\providetranslation{Appendix}{Anhang}
\providetranslation{Authors}{Autoren}
\providetranslation{authors}{Autoren}
\providetranslation{Author}{Autor}
\providetranslation{author}{Autor}
\end{teXXX} 

This is in contrast to Babel and Polyglossia that define
commands for each string to be translated such as,

\begin{teXXX}
\def\captionsdutch{%
    \def\prefacename{Voorwoord}%
    \def\refname{Referenties}%
    \def\abstractname{Samenvatting}%
    \def\bibname{Bibliografie}%
    \def\chaptername{Hoofdstuk}%
    \def\appendixname{Bijlage}%
    \def\contentsname{Inhoudsopgave}%
    \def\listfigurename{Lijst van figuren}%
    \def\listtablename{Lijst van tabellen}%
    \def\indexname{Index}%
    \def\figurename{Figuur}%
    \def\tablename{Tabel}%
    \def\partname{Deel}%
    \def\enclname{Bijlage(n)}%
    \def\ccname{cc}%
    \def\headtoname{Aan}%
    \def\pagename{Pagina}%
    \def\seename{zie}%
    \def\alsoname{zie ook}%
    \def\proofname{Bewijs}%
    \def\glossaryname{Verklarende woordenlijst}%
    \def\today{\number\day~\ifcase\month%
      \or januari\or februari\or maart\or april\or mei\or juni\or
      juli\or augustus\or september\or oktober\or november\or
      december\fi
      \space \number\year}}
\end{teXXX}

\begin{macro}{\usedictionary}\marg{kind}
  This command tells the |translator| package, that at the beginning of
  the document it should load \textit{all} dictionaries of kind \meta{kind} for
  the languages used in the document. Note that the dictionaries are
  not loaded immediately, but only at the beginning of the document.

  If no dictionary of the given \emph{kind} exists for one of the
  language, nothing bad happens.

  Invocations of this command accumulate, that is, you can call it
  multiple times for different dictionaries.
\end{macro}

\Describe{\uselanguage}{\marg{list of languages}}
  This command tells the |translator| package that it should load the
  dictionaries for all languages in the \meta{list of languages}. The
  dictionaries are loaded at the beginning of the document.

\section{Fonts for All the World Scripts}

Many commercial as well as open source fonts exist that can be used to typeset text the world's scripts and languages. The aim of this section of the documentation is to present an overview of the most common scripts represented in the Unicode~7.0 standard. All the examples require the use of the \XeTeX\ engine. In addition you need to have a copy of the font on your own system. If you do not have them, the font loading mechanism of \XeTeX\ will take some time to search all the directories and slows compilation tremendously. 




\section{Pan-Unicode Fonts}

Thousands of fonts exist on the market, but fewer than a dozen fonts—sometimes described as "pan-Unicode" fonts—attempt to support the majority of Unicode's character repertoire. Instead, Unicode-based fonts typically focus on supporting only basic ASCII and particular scripts or sets of characters or symbols. Several reasons justify this approach: applications and documents rarely need to render characters from more than one or two writing systems; fonts tend to demand resources in computing environments; and operating systems and applications show increasing intelligence in regard to obtaining glyph information from separate font files as needed, i.e. font substitution. Furthermore, designing a consistent set of rendering instructions for tens of thousands of glyphs constitutes a monumental task; such a venture passes the point of diminishing returns for most typefaces.

The \texttt{NotSerif} font from Google\footnote{\protect\url{http://www.google.com/get/noto/}} has good support for many languages.

Another freeware pan-Unicode font is Titus\footnote{\protect\url{http://titus.fkidg1.uni-frankfurt.de/unicode/tituut.asp?Inp1=A&Inp2=B&Inp3=C&Inp4=d%40e.com&Inp6=0&Inp5=1}}
This is an extended version of this font is TITUS Cyberbit Unicode, includes 36,161 characters in v4.0.

\newfontfamily\titus[Scale=1.05]{TITUSCBZ.ttf}
\newfontfamily\noto{NotoSerif-Regular.ttf}

\begin{scriptexample}[]{Titus}
\titus

\lorem
\end{scriptexample}
\bigskip

\begin{scriptexample}[]{Noto}
\noto

\lorem
\end{scriptexample}


\section{The \texttt{ucharclasses} package}

For multilingual texts font switching can become cumbersome. The use of a pan-Unicode font as the default can help. However, if the languages are distinct enough to use different Unicode blocks, which are not covered by the \pkgname{polyglossia} package Mike Kamermans' package \pkgname{ucharclasses} can be used.

\begin{verbatim}
% and the font switching magic
\usepackage[CJK, Latin, Thai, Sinhala, Malayalam, DominoTiles, MahjongTiles]{ucharclasses}
\usepackage{fontspec}

% default transition uses the widest coverage font I know of
\setDefaultTransitions{\fontspec{Code2000.ttf}}{}

% overrides on the default rules for specific informal groups
\setTransitionsForLatin{\fontspec{Palatino Linotype}}{}
\setTransitionsForCJK{\fontspec{code2000.ttf}}{}%HAN NOM A
\setTransitionsForJapanese{\fontspec{code2000.ttf}}{}%Ume Mincho

% overrides on the default rules for specific unicode blocks
\setTransitionTo{CJKUnifiedIdeographsExtensionB}{\fontspec{SimSun-ExtB}}
\setTransitionTo{Thai}{\fontspec{IrisUPC}}
\setTransitionTo{Sinhala}{\fontspec{Iskoola Pota}}
\setTransitionTo{Malayalam}{\fontspec{Arial Unicode MS}}

\end{verbatim}

\bgroup
\begin{verbatim}
domino tiles, 🁇 🀼 🁐 🁋 🁚 🁝, and mahjong tiles: 🀑 🀑 🀑 🀒 🀒 🀒 🀕 🀕 🀕 🀗 🀗 🀗 🀅 🀅 (using FreeFont)
\end{verbatim}

domino tiles, 🁇 🀼 🁐 🁋 🁚 🁝, and mahjong tiles: 🀑 🀑 🀑 🀒 🀒 🀒 🀕 🀕 🀕 🀗 🀗 🀗 🀅 🀅 (using FreeFont)
\egroup

\section{PhD Settings}

\def\test{}
\cxset{language/.code=\test}
\cxset{language=greek}
\cxset{languages/.code=\test}
\cxset{languages={english,greek,spanish,chinese}}
\cxset{greek font/.code=\test}
\cxset{greek font=code2000.ttf}

\begin{key}{/chapter/language=\meta{language name}}  
The key language sets the main language for the document. This language will be used for the sectioning commands and common string translations.

If the language is English Polyglossia or Babel are not loaded automatically. If the language is other than English we load either Babel or Polyglossia depending on the engine used.
\end{key}


\begin{key}{/chapter/languages=\meta{language1, language2, language3}}  
The key |languages|, determines all the other scripts available for typesetting. For each language default font commands are create automatically. The aim is to be able to run a fully multilingual system with the minimum of upfront settings. These we leave to customize in the style template files.
\end{key}

\begin{key}{/chapter/greek font=\meta{options}\meta{font file}}  
The package comes with numerous language and appropriate default fonts
for each operating system. 
\end{key}

\section{Ancient and Historic Scripts}

Unicode encodes a number of ancient scripts, which have not been in normal use for a millennium or more, as well as historic scripts, whose usage ended in recent centuries. Although these scripts are no longer used to write living languages, documents and inscriptions using these languages exist, both for extinct languages and for precursors of modern languages. The primary user communities for these scripts are scholars, interested in studying the scripts and the languages written in them. A few, such as Coptic, also have contemporary liturgical or other special purposes. Some of the historic scripts are related to each other as well as to modern alphabets. The following are provides as of Unicode version~6.2.

\begin{center}
\begin{tabular}{lll}
Ogham.     &Ancient Anatolian Alphabets. &Avestan.\\
Old Italic. &Old South Arabian. &Ugaritic\\
Runic &Phoenician. &Old Persian\\
Gothic &Imperial Aramaic &Sumero-Akkadian\\
Old Turkic. &Mandaic &Egyptian Hieroglyphs.\\
Linear B &Inscriptional Parthian &Meroitic.\\
Cypriot Syllabary &Inscriptional Pahlavi&\\
\end{tabular}
\end{center}

The following scripts are also encoded but following the Unicode
convention are described in other sections

\begin{center}
\begin{tabular}{llllll}
Coptic &Glagolithic &Phags-pa. &Kaithi &Kharoshi &Brahmi.\\
\end{tabular}
\end{center}


^^A\subsection{Ogham}

\newfontfamily\ogham{code2000.ttf}

Ogham was added to the Unicode Standard in September 1999 with the release of version 3.0.
The spelling of the names given is a standardisation dating to 1997, used in Unicode Standard and in Irish Standard 434:1999.
The Unicode block for ogham is \texttt{U+1680–U+169F}.

\begin{scriptexample}[]{Ogham}
\bgroup
\ogham
0	1	2	3	4	5	6	7	8	9	A	B	C	D	E	F\\
U+168x	   	ᚁ	ᚂ	ᚃ	ᚄ	ᚅ	ᚆ	ᚇ	ᚈ	ᚉ	ᚊ	ᚋ	ᚌ	ᚍ	ᚎ	ᚏ\\
U+169x	ᚐ	ᚑ	ᚒ	ᚓ	ᚔ	ᚕ	ᚖ	ᚗ	ᚘ	ᚙ	ᚚ	᚛	᚜	\\

\titus

0	1	2	3	4	5	6	7	8	9	A	B	C	D	E	F\\
U+168x	   	ᚁ	ᚂ	ᚃ	ᚄ	ᚅ	ᚆ	ᚇ	ᚈ	ᚉ	ᚊ	ᚋ	ᚌ	ᚍ	ᚎ	ᚏ\\
U+169x	ᚐ	ᚑ	ᚒ	ᚓ	ᚔ	ᚕ	ᚖ	ᚗ	ᚘ	ᚙ	ᚚ	᚛	᚜
\egroup		
\end{scriptexample}
^^A\section{Ancient Anatolian Alphabets}

The Anatolian scripts described in this section all date from the first millenium BCE, and were used to write various ancient Indo-European languages of western and southwestern Anatolia (now Turkey). All are related to the Greek script and are probably adaptations of it. 

\newfontfamily\lycian{Aegean.ttf}
\let\lydian\lycian
\let\carian\lydian

\begin{description}
\item [Lycian] The Lycian alphabet was used to write the Lycian language. It was an extension of the Greek alphabet, with half a dozen additional letters for sounds not found in Greek. It was largely similar to the Lydian and the Phrygian alphabets.
 
\bgroup
\lydian
\obeylines
0	1	2	3	4	5	6	7	8	9	A	B	C	D	E	F
U+1028x	𐊀	𐊁	𐊂	𐊃	𐊄	𐊅	𐊆	𐊇	𐊈	𐊉	𐊊	𐊋	𐊌	𐊍	𐊎	𐊏
U+1029x	𐊐	𐊑	𐊒	𐊓	𐊔	𐊕	𐊖	𐊗	𐊘	𐊙	𐊚	𐊛	𐊜

Typeset with the \idxfont{Aegean.ttf} and the command \cmd{\lydian}
\egroup

\item[Lydian] Lydian script was used to write the Lydian language. That the language preceded the script is indicated by names in Lydian, which must have existed before they were written. Like other scripts of Anatolia in the Iron Age, the Lydian alphabet is a modification of the East Greek alphabet, but it has unique features. The same Greek letters may not represent the same sounds in both languages or in any other Anatolian language (in some cases it may). Moreover, the Lydian script is alphabetic.
Early Lydian texts are written both from left to right and from right to left. Later texts are exclusively written from right to left. One text is boustrophedon. Spaces separate words except that one text uses dots. Lydian uniquely features a quotation mark in the shape of a right triangle.
The first codification was made by Roberto Gusmani in 1964 in a combined lexicon (vocabulary), grammar, and text collection.


\bgroup
\lycian
\obeylines
	0	1	2	3	4	5	6	7	8	9	A	B	C	D	E	F
U+1092x	𐤠	𐤡	𐤢	𐤣	𐤤	𐤥	𐤦	𐤧	𐤨	𐤩	𐤪	𐤫	𐤬	𐤭	𐤮	𐤯
U+1093x	𐤰	𐤱	𐤲	𐤳	𐤴	𐤵	𐤶	𐤷	𐤸	𐤹						𐤿
Typeset with the \idxfont{Aegean.ttf} and the command \cmd{\lycian}

Examples of words

𐤬𐤭𐤠  - Ora - "Month"

𐤬𐤳𐤦𐤭𐤲𐤬𐤩  - Laqrisa - "Wall"

𐤬𐤭𐤦𐤡  - "House, Home"

\egroup

\item [Carian] The Carian alphabets are a number of regional scripts used to write the Carian language of western Anatolia. They consisted of some 30 alphabetic letters, with several geographic variants in Caria and a homogeneous variant attested from the Nile delta, where Carian mercenaries fought for the Egyptian pharaohs. They were written left-to-right in Caria (apart from the Carian–Lydian city of Tralleis) and right-to-left in Egypt. Carian was deciphered primarily through Egyptian–Carian bilingual tomb inscriptions, starting with John Ray in 1981; previously only a few sound values and the alphabetic nature of the script had been demonstrated. The readings of Ray and subsequent scholars were largely confirmed with a Carian–Greek bilingual inscription discovered in Kaunos in 1996, which for the first time verified personal names, but the identification of many letters remains provisional and debated, and a few are wholly unknown.

\begin{scriptexample}[]{Carian}
\bgroup
\carian
\obeylines
 	0	1	2	3	4	5	6	7	8	9	A	B	C	D	E	F
U+102Ax	𐊠	𐊡	𐊢	𐊣	𐊤	𐊥	𐊦	𐊧	𐊨	𐊩	𐊪	𐊫	𐊬	𐊭	𐊮	𐊯
U+102Bx	𐊰	𐊱	𐊲	𐊳	𐊴	𐊵	𐊶	𐊷	𐊸	𐊹	𐊺	𐊻	𐊼	𐊽	𐊾	𐊿
U+102Cx	𐋀	𐋁	𐋂	𐋃	𐋄	𐋅	𐋆	𐋇	𐋈	𐋉	𐋊	𐋋	𐋌	𐋍	𐋎	𐋏
U+102Dx	𐋐
\egroup
\end{scriptexample}

\newfontfamily\oldpunctuation{code2000.ttf}

Word dividers are infrequent, \emph{scriptio continua}\footnote{a style of writing without word dividers, that is, without spaces or other marks between words or sentences} is common. Words dividers which are attested are U+00B7 (\char"00B7) \textsc{MIDLE DOT} (or U+2E31 word separator middle dot), U+205A TWO DOT PUNCTUATION, and U+205D TRICOLON ({\oldpunctuation\char"205D}). In modern editions U+0020 SPACE may be found.

\end{description}
^^A

\section{Avestan script}
\label{s:avestan}
The Avestan alphabet is a writing system developed during Iran's Sassanid era (AD 226–651) to render the Avestan language.
As a side effect of its development, the script was also used for Pazend, a method of writing Middle Persian that was used primarily for the Zend commentaries on the texts of the Avesta. In the texts of Zoroastrian tradition, the alphabet is referred to as \emph{din dabireh} or \emph{din dabiri}, Middle Persian for "the religion's script".

The Avestan alphabet was replaced by the Arabic alphabet after Persia converted to Islam during the 7th century CE. 


Notable Features

The alphabet is written from right to left, in the same way as Syriac, Arabic and Hebrew.
See more at: \url{http://www.iranchamber.com/scripts/avestan_alphabet.php#sthash.ZRu7AkEb.dpuf}

\newfontfamily\avestan{NotoSansAvestan-Regular.ttf}



\begin{scriptexample}[]{Avestan}
\ifxetex\TeXXeTstate=1
\beginR\fi
\avestan\raggedleft
𐬄	
𐬅	
𐬆	
𐬇	
𐬈	
𐬉	
𐬊	
𐬋	
𐬌	
𐬍	
𐬎	
𐬏	
𐬐	
	
𐬒	
𐬓	
𐬔	
	
𐬖	
𐬗	
𐬘	
𐬙	
𐬚	
𐬛	
𐬜	
𐬝	
𐬞	
𐬟	
𐬠	
𐬡	
𐬢	
𐬣	
𐬤	
𐬥	
𐬦	
𐬧	
𐬨	
𐬩	
𐬪	
𐬫	
𐬬	
𐬭	
𐬮	
𐬯	
𐬰	
𐬱	
𐬲	
𐬳	
𐬴	
𐬵	
\ifxetex\endR
\TeXXeTstate=0\fi
\end{scriptexample}

The recent Google font \url{NotoSansAvestan-Regular_0.ttf} can be used to typeset the Avestan script, but really not suitable for any serious study of the language.
^^A\subsection{Old Turkic}

\newfontfamily\oldturkic{Segoe UI Symbol}
\begin{scriptexample}[]{Old Turkish}
\oldturkic
\obeylines
Orkhon	Yenisei
variants	Transliteration / transcription
Old Turkic letter  𐰀	𐰁 𐰂	a, ä
Old Turkic letter  𐰃	𐰄 𐰅	y, i (e)
Old Turkic letter  𐰆		o, u
Old Turkic letter  𐰇	𐰈	ö, ü

	0	1	2	3	4	5	6	7	8	9	A	B	C	D	E	F
U+10C0x	𐰀	𐰁	𐰂	𐰃	𐰄	𐰅	𐰆	𐰇	𐰈	𐰉	𐰊	𐰋	𐰌	𐰍	𐰎	𐰏
U+10C1x	𐰐	𐰑	𐰒	𐰓	𐰔	𐰕	𐰖	𐰗	𐰘	𐰙	𐰚	𐰛	𐰜	𐰝	𐰞	𐰟
U+10C2x	𐰠	𐰡	𐰢	𐰣	𐰤	𐰥	𐰦	𐰧	𐰨	𐰩	𐰪	𐰫	𐰬	𐰭	𐰮	𐰯
U+10C3x	𐰰	𐰱	𐰲	𐰳	𐰴	𐰵	𐰶	𐰷	𐰸	𐰹	𐰺	𐰻	𐰼	𐰽	𐰾	𐰿
U+10C4x	𐱀	𐱁	𐱂	𐱃	𐱄	𐱅	𐱆	𐱇	𐱈	

\hfill  Typeset with \texttt{Segoe UI Symbol} \cmd{\oldturkic} 
\end{scriptexample}

Irk Bitig or Irq Bitig (Old Turkic: {\bfseries\Large\oldturkic 𐰃𐰺𐰴 𐰋𐰃𐱅𐰃𐰏}), known as the Book of Omens or Book of Divination in English, is a 9th-century manuscript book on divination that was discovered in the "Library Cave" of the Mogao Caves in Dunhuang, China, by Aurel Stein in 1907, and is now in the collection of the British Library in London, England. The book is written in Old Turkic using the Old Turkic script (also known as "Orkhon" or "Turkic runes"); it is the only known complete manuscript text written in the Old Turkic script.[1] It is also an important source for early Turkic mythology.

The Old Turkic text does not have any sentence punctuation, but uses two black lines in a red circle as a word separation mark in order to indicate word boundaries as shown in Figure~{\ref{omen}}

\begin{figure}[htb]
\includegraphics[width=0.7\textwidth]{./images/omen.jpg}
\caption{Omen 11 (4-4-3 dice) of the Irk Bitig (folio 13a): "There comes a messenger on a yellow horse (and) an envoy on a dark brown horse, bringing good tidings, it says. Know thus: (The omen) is extremely good."}
\label{omen}
\end{figure}
^^A\section{Phoenician}
\label{s:phoenician}
\arial

The Phoenician alphabet and its successors were widely used over a broad area surrounding the Mediterranean Sea.

\let\phoenician\lycian

\begin{scriptexample}[]{Phoenician}

\unicodetable{phoenician}{"10900,"10910}

\end{scriptexample}

The Phoenician alphabet, called by convention the Proto-Canaanite alphabet for inscriptions older than around 1200 BCE, is the oldest verified consonantal alphabet, or abjad.[1] It was used for the writing of Phoenician, a Northern Semitic language, used by the civilization of Phoenicia. It is classified as an abjad because it records only consonantal sounds (matres lectionis were used for some vowels in certain late varieties).

Phoenician became one of the most widely used writing systems, spread by Phoenician merchants across the Mediterranean world, where it evolved and was assimilated by many other cultures. The Aramaic alphabet, a modified form of Phoenician, was the ancestor of modern Arabic script, while Hebrew script is a stylistic variant of the Aramaic script. The Greek alphabet (and by extension its descendants such as the Latin, the Cyrillic, and the Coptic) was a direct successor of Phoenician, though certain letter values were changed to represent vowels.

\begin{figure}[ht]
\includegraphics[width=\textwidth]{./images/phoenician.jpg}
\captionof{figure}{
Phoenician votive inscription from Idalion (Cyprus), 390 BC. BM 125315 from The Early Alphabet by John F. Healy.}
\end{figure}

As the letters were originally incised with a stylus, most of the shapes are angular and straight, although more cursive versions are increasingly attested in later times, culminating in the Neo-Punic alphabet of Roman-era North Africa. Phoenician was usually written from right to left, although there are some texts written in boustrophedon.


\printunicodeblock{./languages/phoenician.txt}{\phoenician}


\newpage
\section{Palmyrene}
\idxlanguage{Palmyrene}
\arial

Palmyrene is the very widely attested Aramaic dialect and script
of Palmyra in the Syrian desert. The texts date from the midfirst century to the destruction of Palmyra by the Romans in AD 272. Palmyra in the Roman period was a major trading centre.
\medskip

\begin{figure}[ht]
\centering

\includegraphics[width=0.9\textwidth]{./images/palmyrene.jpg}
\captionof{figure}{\protect\arial Limestone bust with Palmyrene inscription. Palmyra late 2nd century AD. BM WA 102612}

\end{figure}

\medskip
The longest of the Palmyrene texts, is the bilingual  taxation tariff written for the year 137 AD in Palmyrene Aramaic and Greek.\footnote{For more details see:MILIK J.T., Dédicaces faites par des dieux (Palmyre, Hatra, 
Tyr) et de thiases sémitiques à l'époque romaine, Paris 1972; ROSENTHAL R., Die 
Sprache der palmyrenischen Inschriften, Leipzig 1936; STARK J.K., Personal Names in 
Palmyrene Inscriptions, Oxford 1971; DRIJVERS H.J.W., The Religion of Palmyra, 
Leiden 1976; TEIXIDOR J., 'Palmyre et son commerce d'Auguste à Caracalla', in 
Semitica 34, (1984) 1-127.  } Trade connections 
took the Palmyrene script to other places, some not far away, such as Dura Europos on the Euphrates, butothers at a great distance. A particular inscription is from South Shields, Roman Arbeia, in the north-east of England, carved on behalf of a Palmyrene mechant for his deceased wife and probably dating to the early third century AD. 

The Palmyrene script existed in two main varieties, a monumental and a cursive one, though the latter is little known and the evidence  mostly from Palmyra itself. The Syriac script of Edessa in southern Turkey, is often regarded as derived or closely related to the Palmyrene---similarities are found in the letters: ', b, g, d, w, h, y, k, l, m, n, `, r and t---though a strong case can also be made for connecting Syriac with a northern Mesopotamian script-family represented principally in texts from Hatra, a city more or less contemporary with Palmyra in Upper Mesopotamia. 


\begin{figure}[ht]
\includegraphics[width=\textwidth]{./images/regina-epigraph.jpg}
\caption{It was customary for Palmyrenes to offer bilingual texts (Greek or Latin) on funerary monuments. The final line of Regina's epitaph is Barates' personal lament in Palmyrene: Regina, freedwoman of Barate, alas. (See \href{http://www2.cnr.edu/home/araia/regina.html}{regina}.)}
\end{figure}

A good article on the classification of Aramaic languages can be found in \textit{The Aramaic language and Its Classification} by Efrem Yildiz.\footnote{\url{http://www.jaas.org/edocs/v14n1/e8.pdf}}








^^A\newfontfamily\aegyptus{AegyptusR.ttf}

\chapter{Aegyptian Hieroglyphics}

\index{fonts>Aegyptus}\index{Aegyptus (font)}
\index{fonts>Hieroglyphics}\index{languages>hieroglyphics}

\newfontfamily\hiero{NotoSansEgyptianHieroglyphs-Regular.ttf}

Hieroglyphic writing appeared in Egypt at the end of the fourth millennium bce. The writing
system is pictographic: the glyphs represent tangible objects, most of which modern
scholars have been able to identify. A great many of the pictographs are easily recognizable
even by nonspecialists. Egyptian hieroglyphs represent people and animals, parts of the
bodies of people and animals, clothing, tools, vessels, and so on.

Hieroglyphs were used to write Egyptian for more than 3,000 years, retaining characteristic
features such as use of color and detail in the more elaborated expositions. Throughout the
Old Kingdom, the Middle Kingdom, and the New Kingdom, between 700 and 1,000 hieroglyphs
were in regular use. During the Greco-Roman period, the number of variants, as
distinguished by some modern scholars, grew to somewhere between 6,000 and 8,000.

Hieroglyphs were carved in stone, painted on frescoes, and could also be written with a reed
stylus, though this cursive writing eventually became standardized in what is called \emph{hieratic}
writing. Unicode does not encode the hieratic forms separately, but ust considers them as cursive forms of the hieroglyphs encoded block.

The Demotic script and then later the Coptic script replaced the earlier hieroglyphic and
hieratic forms for much practical writing of Egyptian, but hieroglyphs and hieratic continued
in use until the fourth century ce. An inscription dated August 24, 394 ce has been
found on the Gateway of Hadrian in the temple complex at Philae; this is thought to be
among the latest examples of Ancient Egyptian writing in hieroglyphs

\begin{figure}[htb]
\includegraphics[width=\textwidth]{./images/bookofthedead.jpg}
\end{figure}

In hieroglyphic texts, these drawings are not only simply arranged in sequential order, but also grouped on top of and next to each other. This rather complicates matters trying to register and reproduce hieroglyphic texts using a computer.

\section{Computer Typesetting}

Typesetting hieroglyphics with computers presents a number of problems. First is the method of inputting the characters and second the various methods required to stack hieroglyphics, the direction of writing which can be one of four different directions.

When the first computers were introduced in Egyptology in the late 1970s and the beginning of the 1980s, the graphical capacity of the machines was still in its infancy. Early attempts to register the hieroglyphic pictorial writing on computer therefore chose an encoding system to do this, using alphanumeric codes to represent or replace the graphics. To prevent many people from reinventing the wheel, during the first "Table Ronde Informatique et Egyptologie" in 1984 a committee was charged with the task to develop a uniform system for the encoding of hieroglyphic texts on computer. The resulting Manual for the Encoding of Hieroglyphic Texts for Computer-input (Jan Buurman, Nicolas Grimal, Jochen Hallof, Michael Hainsworth and Dirk van der Plas, Informatique et Egyptologie 2, Paris 1988), simply called Manuel de Codage, presents an easy to use and intuitive way of encoding hieroglyphic writing as well as the abbreviated hieroglyphic transcription (transliteration). The system proposed by the Manuel de Codage has since been adopted by international Egyptology as the official common standard for registering hieroglyphic texts on computer. Mark-Jan Nederhof proposed an enhanced encoding scheme to remove many of the limitations in the Manuel de Codage.

\pkgname{HieroTeX} is a \latexe package developed by to typeset hieroglyphic texts and still works well. The advantages of using \tex is of course its excellent typesetting capabilities and the usage of macros. Although inputting the texts as MdC codes is not that difficult, repeating the same codes over and over can be avoided with easily constructed simple substitution macros. 

\subsection{fonts}

One of the best fonts I came across is \idxfont{Aegyptus} from \url{http://users.teilar.gr/~g1951d/}\footnote{The site also has fonts for Aegean Numbers, Ancient Greek Musical Notation, Ancient Greek Numbers, Ancient Roman Symbols, Arkalochori Axe, Carian, Cypriot Syllabary, Dispilio tablet, Linear A, Linear B Ideograms, Linear B Syllabary, Lycian, Lydian, Old Italic, Old Persian, Phaistos Disc, Phoenician, Phrygian, Sidetic, Troy vessels’ signs and Ugaritic. Cretan Hieroglyphs and Cypro-Minoan script(s) are offered in separate files.}. The font provides all the unicode characters and also offers an additional number of glyphs that are not in the Unicode standard. The font uses the Unicode Private Use Areas to encode the glyphs. 

Another font is the Noto Egyptian Hieroglyphics from Google. This is a lightweight font with the symbols in their proper unicode slots. Mark-Jan Nederhof's \idxfont{NewGardiner} font is another one with support only for the Gardiner set. The codepoint mappings are incorrect, as the font has been  
encoded to EGPZ. The font is similar to the Aegyptus font, however it is just transposed and not recommended unless it is transposed. 

The editor software JSesh\footnote{\protect\url{http://jsesh.qenherkhopeshef.org/}} also provides a free font |JSeshFont.ttf|. This offers a correctly mapped unicode and is another good alternative. The symbols are drawn somewhat simpler and is just a matter of taste what you want to use.

My recommendation is for short demonstration purposes, the Noto font is to be preferred while for more serious work the Aegyptus font will be more useful. Using Lua the font can be transposed automatically to allow the use of commands that refer to unicode numbers. Another advantage of the Aegyptus font is that the glyphs are named with their Gardiner numbers, so it is somewhat easier to programmatically access them by name.\footnote{Unicode does not name the glyphs, but simply calls the Egyptian Hieroglyph $n$. } 

\medskip

\ifxetex
\bgroup
\centering 
\font\myfont = "Aegyptus"
\scalebox{7}{\myfont\XeTeXglyph 201}
\scalebox{7}{\myfont\XeTeXglyph 203}
\scalebox{7}{\myfont\XeTeXglyph 163}
\scalebox{7}{\myfont\XeTeXglyph 164}
\scalebox{7}{\myfont\XeTeXglyph 165}
\scalebox{7}{\myfont\XeTeXglyph 168}
\captionof{table}{Example of Egyptian Hieroglyphics typeset with the \textit{Aegyptus} font.} 
\egroup
\fi

\ifluatex
\bgroup
\centering 
\aegyptus
\scalebox{7}{\char"F300C}
\scalebox{7}{\char"F3001}
\scalebox{7}{\char"F3010}
\scalebox{7}{\char"F308B}
\scalebox{7}{\char"F3097}
\scalebox{7}{\char"F3091}
\captionof{table}{Example of Egyptian Hieroglyphics typeset with the \textit{Aegyptus} font.} 
\egroup

\fi


\subsection{Unicode Block}

Egyptian hieroglyphs is a Unicode block containing the Gardiner's sign list of Egyptian hieroglyphics.
The code points, in the range |0x13000| to |0x1342E|, are available starting from
\href{http://unicode.org/charts/PDF/U13000.pdf}{Unicode 5.2}

\begin{scriptexample}[]{Hieroglyphic}
\bgroup
\unicodetable{hiero}{"13000,"13010,"13020,"13030,"13040,"13050,"13060,"13070,%
"13080,%
"13090,"130A0,"130B0,"130C0,"130D0,"130E0,"130F0,%
"13100,"13110,"13120,"13130,"13140,"13150,"13060,"13070,"13080,"13090}
\egroup
\end{scriptexample}

\subsection{Gardiner's classification}

The standard reference on Egyptian hieroglyphics is Gartiner's Sign List, which lists common Egyptian hieroglyphs. These are grouped in categories from A-Aa. Each category represents a theme for example category A, is "man and his occupations". Based on this list ``Queen with flower" is denoted as \texttt{B7}. 

\subsubsection{Character Names} 

Egyptian hieroglyphic characters have traditionally been designated in
several ways:

\begin{enumerate}
\item  By complex description of the pictographs: \texttt{GOD WITH HEAD OF IBIS}, and so forth.
\item By standardized sign number: C3, E34, G16, G17, G24.
\item For a minority of characters, by transliterated sound.
\end{enumerate}

The characters in the Unicode Standard make use of the standard Egyptological catalog
numbers for the signs. Thus, the name for {\hiero\char"130F9} |U+13049| egyptian hieroglyph e034 refers
uniquely and unambiguously to the Gardiner list sign E34, described as a “{\aegean DESERT HARE}” ({\hiero \char"130FA}) and used for the sound “wn”. The Unicode catalog values are padded to three places with
zeros, so where the Gardiner classification is shown as \texttt{E34}, the unicode value is \texttt{E034}. 

Names for hieroglyphic characters identified explicitly in Gardiner 1953 or other sources as
variants for other hieroglyphic characters are given names by appending “A”, “B”, ... to the sign number. In the sources these are often identified using asterisks. Thus Gardiner’s G7,
G7*, and G7** correspond to U+13146 egyptian sign g007 {\hiero \char"13147}, U+13147 egyptian sign g007a, and U+13148 egyptian sign g007b, respectively.

\def\texthiero#1{{\color{black!95}\hiero #1}}

\begin{longtable}{>{\Large}lll>{\ttfamily}l}
{\hiero \char"13000}&A1-A70 & Man and his occupations &U+13000-1304F\\
{\hiero \char"13050}&B1-B9  &Woman and her occupations &U+13050-13059\\
{\hiero \char"1305A} &C1-C24 &Anthropomorphic Deities &U+1305A-13075\\
{\hiero \char"13076} &D1-D67 &Parts of the Human Body &U+13076-130D1\\
{\hiero \char"130D2} &E1-E38 &Mammals &U+13076-130D1\\
{\hiero \char"130FE}  &F1-F53	&Parts of Mammals &U+130FE-1313E\\
{\hiero\char"1313F}	&G1-G54	&Birds &U+1313F-1317E\\
{\hiero \char"1317F}	&H1-H8	&Parts of Birds &U+1317F-13187\\
\texthiero{\char"13188}	&I1-I15	&Amphibious Animals, Reptiles, etc. &U+13188-1319A\\
\texthiero{\char"1319B}	&K1-K8	&Fishes and Parts of Fishes &U+1319B-131A2\\
\texthiero{\char"131A3}	&L1-L8	&Invertebrata and Lesser Animals &U+131A3-131AC\\
\texthiero{\char"131AD}	&M1-M44	&Trees and Plants &U+13AD-131EE\\
\texthiero{\char"131EF}	&N1-N42	&Sky, Earth, Water &U+131EF-1321F\\
\texthiero{\char"13250}	&O1-O51	&Buildings and Parts of Buildings &U+13250-1329A\\
\texthiero{\char"1329B}	&P1-P11	&Ships and Parts of Ships &U+1329B-132A7\\
\texthiero{\char"132A8}	&Q1-Q7	& Domestic and Funerary Furniture &U+132A8-132AE\\
\texthiero{\char"132AF}	&R1-R29	&Temple Furniture and Sacret Emblems &U+132AF-132D0\\
\texthiero{\char"132D1}	&S1-S46	&Crowns, Dress, Staves, etc. &U+132D1-13306\\
\texthiero{\char"13307}	&T1-T36	&Warfare, Hunting, Butchery &U+13307-13332\\
\texthiero{\char"13333}	&U1-42	&Agriculture, Crafts and Professions &U+13333-13361\\
\texthiero{\char"13362}	&V1-V40a	&Rope, Fibre, Baskets, Bags, etc. &U+13362-133AE\\
\texthiero{\char"133AF}	&W1-W25	&Vessels of Stone and Earthenware &U+133AF-133CE\\
\texthiero{\char"133CF}	&X1-X8a	&Loaves and Cakes &U+133CF-133DA\\
\texthiero{\char"133DB}	&Y1-Y8	&Writing, Games, Music &U+133DB-133E3\\
\texthiero{\char"133E4}	&Z1-Z16H	&Strokes, Geometrical Figures, etc. &U+133E4-1340C\\
\texthiero{\char"1340D}	&Aa1-Aa32	&Unclassified &U+1340D-1342E\\
\end{longtable}

I particularly like the crocodile sign \def\crocodile{\color{teal}{\Huge\texthiero{\char"13188}}} {\crocodile}, as it is applicable to describe people in my field of work. 

\begin{scriptexample}[]{Woman and her occupations}
\unicodetable{hiero}{"13050}
\end{scriptexample}

\section{Positioning}

One of the core assumptions of any hieroglyphic encoding or mark-up scheme following the MdC is that signs and groups of signs maybe positioned next to each other or above each other. The former is indicated by the operator * and the latter by :. One may also use -, which functions as * for horizontal texts and as : for vertical text. 

In some dialects of the MdC relative positioning has been extended by the use of the |&| operator. This is used to form a kind of ligature, such as |D&t| can be defined to represent the \textit{Cobra at rest} sign I10 with sign X1 underneath, as follows:

\begin{center}
{\hiero\HUGE
       \mbox{\rlap{\char"133CF}\char"13193\hfill\hfill}\\
       {\large|insert[bs](I10,X1)|}

\mbox{\rlap{\scalebox{0.5}{\char"133E3}}\char"13193\hfill\hfill}\\
 	
}
\end{center}

This is only a partial solution and to automate it via kerning tables, will require hundreds of entries in the kerning tables. It will also need constant modifications as researchers discover new combinations. A better approach and which is easily applied to \tex based systems would be to adopt Nederhof's method of creating a new command |insert[bs](I10,X1)|. 

In \tex one could simply define a command \cmd{\insert} with one optional argument to handle the positioning. The positioning uses the letters [b,t,s,e] to position the glyph. the letters s and e stand for start and end, whereas b,t for bottom and top respectively. When there are only two symbols involved, this is not such a difficult operation, but when three or more symbols are to be grouped and kerned together, inserting with some form of scaling is necessary.

\subsection{Enclosures}

Enclosures. The two principal names of the king, the \emph{nomen} and \emph{prenomen}, were normally
written inside a \emph{cartouche}: a pictographic representation of a coil of rope.

In the Unicode representation of hieroglyphic text, the beginning and end of the cartouche
are represented by separate paired characters, somewhat like parentheses. The Unicode manual states that `rendering of a full cartouche surrounding a name requires specialized layout software', which is of course an easy task for \tex.

\begin{macro}{\cartouche}
The commands \cmd{\cartouche} and \cmd{\cartouche}, from Peter Wilson's \pkgname{hierglyph} package have been used for many years to demonstrate the use of hieroglyphics with \latexe. 
\end{macro}

There are a several characters for these start and end cartouche characters, reflecting various styles for the enclosures.

\cartouche{{\hiero \char"13147}$sin^{2} x + cos^{2} x = 1$}
\Cartouche{{\hiero \char"13147}$sin^{2} x + cos^{2} x = 1$}

Unicode:{\hiero 𓇓𓏏𓊵𓏙𓊩𓁹𓏃𓋀𓅂𓊹𓉻𓎟𓍋𓈋𓃀𓊖𓏤𓄋𓈐𓎟𓇾𓈅𓏤𓂦𓈉 }

\textpmhg{\HQ} 

\cartouche{\pmglyph{K:l-i-o-p-a-d:r-a}}
%\translitpmhg{\HK\Hl\Hi\Ho\Hp\Ha\Hd\Hr\Ha}

\printunicodeblock{./languages/hieroglyphics.txt}{\hiero}
\printunicodeblock{./languages/hieroglyphics-13100.txt}{\hiero}
\printunicodeblock{./languages/hieroglyphics-13200.txt}{\hiero}
\printunicodeblock{./languages/hieroglyphics-13300.txt}{\hiero}
\printunicodeblock{./languages/hieroglyphics-13400.txt}{\hiero}
\section{Numerals}

Egyptian numbers are encoded following the same principles used for the
encoding of Aegean and Cuneiform numbers. Gardiner does not supply a full set of
numerals with catalog numbers in his Egyptian Grammar, but does describe the system of
numerals in detail, so that it is possible to deduce the required set of numeric characters.

Two conventions of representing Egyptian numerals are supported in the Unicode Standard.
The first relates to the way in which hieratic numerals are represented. Individual
signs for each of the 1s, the 10s, the 100s, the 1000s, and the 10,000s are encoded, because in
hieratic these are written as units, often quite distinct from the hieroglyphic shapes into
which they are transliterated. The other convention is based on the practice of the \emph{Manual
de Codage}, and is comprised of five basic text elements used to build up Egyptian numerals.
There is some overlap between these two systems.

%% Needs some work to get it into LuaLaTeX
%% omitted for the time being
%\ifxetex
%\begin{texexample}{TeXeXglyph}{ex:xetexglyph}
%\raggedright
%\font\myfont = "Aegyptus"
%\setcounter{glyphcount}{136}
%
%\whiledo
%{\value{glyphcount}<\XeTeXcountglyphs\myfont}
%{\arabic{glyphcount}:~
%{\myfont\XeTeXglyph\arabic{glyphcount}}\quad
%\stepcounter{glyphcount}}
%\end{texexample}
%\fi

\section{Input Methods}

If you writing a document with a lot of hieroglyphics inputting of hieroglyphics can be problematic. Most researchers in the field will use special keyboards or editors. They also use MS/Word or OpenOffice. They can both be coerced to produce reasonable documents, but with \tex obviously better results can be achieved. One such editor is \href{http://jsesh.qenherkhopeshef.org/}{jsesh}. 


\begin{luacode*}
    local h = {}
          h = dofile("hiero.lua")
    local options = {style="block",
                     echo=true,
                     direction="RL",
                     size = "\\Huge",
                     color = "green",
                     headings = "captionof{figure}"  -- section/tablecaption/figurecaption
                     }
   -- prints full symbol list
   h.printgardiner(t,options)

   tex.print("\\par")
   local options = {style="block",
                     echo=true,
                     heading="\\par",
                     direction="RL",
                     color = "teal",
                     scale = 8}

   h.printhierochar("hiero","1317D",options)
   h.printhierochar("hiero","13000",{direction="RL",
                                        color = "teal",
                                        scale = 8})
   h.printhierochar("hiero","13003",{direction="LR",
                                        color = "teal",
                                        scale = 1})
   h.parseMdC([[M23-X1-R4-X8-Q2-D4-W17-R14-G4-R8-O29-
               V30-U23-N26-D58-O49-Z1-F13-N31-V30-N16-
               N21-Z1-D45-N25!]])

   tex.print("\\par")
   h.printgardinercat("B")

\end{luacode*}

\newcommand\hierochar[2][direction = "LR",
                         color     = "teal",
                         scale     = 1]{% 
               \luaexec{
                h = h or {}
                h = require("hiero.lua")  
                h.parseMdC(#2,{#1})}}
               
\newcommand\printhierochar[3][direction = "LR",
                              color     = "teal",
                              scale     = 4]{% 
               \luaexec{
                h = h or {}
                h = require("hiero.lua")  
                h.printhierochar(#2,#3,{#1})}}

This file just tests the various commands available for manipulating hieroglyphics. We tried to 
generalize the commands, so they can be re-used for other type of hieroglyphics.

{
\hierochar{"A1-A2-A3!"}

\centering 

\def\options{direction = "LR",
             color     = "teal",
             scale     = 7}

\def\fontname{"hiero"}

\def\hierochar#1{\printhierochar[\options]{\fontname}{#1}}
}


\begin{scriptexample}[]{Some Example}
Sometimes kerning might be required, especially if the
glyphs are scaled.This is easily achieved with a \cmd{\kern}
command and a suitable skip dimension.

\medskip

\bgroup
\fboxsep=0pt\fboxsep.4pt
\def\options{direction = "RL",
             color     = "black!95",
             scale     = 5}
\centering

\color{teal}
\fbox{\hierochar{"13051"}}
\kern-4mm
\hierochar{"13003"}
\def\options{direction = "LR",
             color     = "black!95",
             scale     = 5}
\fbox{\hierochar{"13003"}}\color{red}
\kern-4mm
\hierochar{"13051"}
\color{black!95}
\egroup
\begin{verbatim}
\centering
\hierochar{"13051"}
\kern-4mm
\hierochar{"13003"}
\def\options{direction = "RL",
             color     = "black!95",
             scale     = 5}
\hierochar{"13003"}
\kern-4mm
\hierochar{"13051"}
\end{verbatim}
\end{scriptexample}

A bit of a diversion is appropriate at this point. Our attempt after the historical overview, is to provide some routines for the capturing and display of hieroglyphic texts using LuaTeX. This involves getting low level information from the system regarding fonts. 

\begin{figure}[ht]
\begin{minipage}{0.45\textwidth}
\centering
\includegraphics[width=0.6\textwidth]{./images/fontforge.jpg}
\end{minipage}
\begin{minipage}[t]{0.45\textwidth}
\caption{Viewing font information with fontforge.}
\end{minipage}
\end{figure}

For each glyph, we are interested to get its unicode number, the position in the font table, its name and most importantly the font metrics. The font metrics are a set of parameters that are used to measure the bounding box, any ascenders or descenders and similar information. Using fontforge, these parameters can easily be viewed. However, we are not interested to make any modifications manually; what we are interested is to programmatically obtain this information using Lua. Lua's philosophy and a mantra repeated often by the developers, is that it provides the tools and not the solutions. What this means to the LuaTeX programmer, is that we need to reach very low level  to get this information, which is a road with many bumps. Luckily the tools have been provided by the LuaTeX developers. This comes with a lot of benefits as we can also do our own on the fly mapping, such as creating an index table holding all the Gardiner numbers. 

The |fontloader.open| function loads a font, but it's not usable by itself; the result should be turned into a table with
\textbf{fontloader.to\_table}, as follows:

\begin{verbatim}
  local f = fontloader.open
     ("c:/windows/fonts/NotSansEgyptianHieroglyphics-
       Regulat.ttf")
  fonttable = fontloader.to_table(f)
  fontloader.close(f)
\end{verbatim}

We will use the Google No Tofu Egyptian Hieroglyphic font to experiment with our hieroglyphics. I have used a full path to load the font, which resides on my windows machine in the fonts folder. Once we load all the information in the |fonttable| we use |fontloader.close| to discard the userdata from which the table is extracted. 

What makes OpenType fonts special is that they describe every aspect that you might be able to think of when you think of putting letters together to form words. In addition to the obvious "this is what letters look like" information, OpenType fonts also specify things like the name of each letter that is available in the font, how much of the Unicode standard the font implements, which horizontal and vertical metrics apply to which letters, exactly how the letters are arranged inside the font so that they can quickly be read out, what kind of font classifications apply (is it a fantasy font? is it bold face? is it fixed width? etc), what kind of memory allocation a printer needs to perform in order to be able to even load the font, etc. etc. etc. All these are stored in tables upon tables, similat to a collection of Russian dolls.

To view the values in the fonttable, we will first iterate over the \textbf{fonttable} and extract all the first level keys.

\begin{texexample}{Iterating through a font table}{}
\begin{luacode*}
local z={}
tf=fontloader.to_table(fontloader.open("c:/windows/fonts/NotoSansEgyptianHieroglyphs-Regular.ttf"))

-- we sort the keys to create a table
-- important keys to us are tf.glyphs

for k,v in pairs (tf) do
   --tex.print(k.."\\par")
   table.insert(z, k)
end

table.sort(z)
tex.print("\\begin{multicols}{3}\\raggedright")
for k,v in pairs (z) do
   z[k] = string.gsub(z[k],"%_","\\textunderscore ")
   local s = tf[v]
   tex.print("\\textbullet\\hskip3pt\\hangindent2em " .. z[k].." [\\textit{"..type(s).."}] ","\\par")
end
tex.print("\\end{multicols}")
\end{luacode*}
\end{texexample}

We iterate through the \textbf{fonttable} using the Lua  "pair" iterator and we simply print all the keys and the type of the values in a human readable form as shown in the example. Note the use of |\textunderscore| that replaces all underscores in the fields with its text equivalent to sanitize the output. This is a quick and dirty way to avoid the use of catcodes. Many of the keys, bear intuitive names and are not difficult to discern: \textit{version}, \textit{copyright} and the like. Getting the type of Lua variables is important in order to use them for error trapping. When you attempt for example to print a nil value an error will occur.

Now that we have peeked under the font we will iterate and capture the information of interest, which we will put into another table with two keys \textbf{info}  and \textbf{metrics}. In the metrics file we will get the bounding box related metrics of each and every glyph in the font and save it, into our own table. 

\begin{texexample}{More Metrics}{}
  \begin{luacode*}
   tex.print("units per em = ", tf.units_per_em,"\\par")
   for i,j in ipairs (tf.glyphs[6].boundingbox) do
      tex.print("bounding box["..i.."]".." = ", j,"\\par")
   end 
   local w = (tf.glyphs[6].boundingbox[3]-tf.glyphs[6].boundingbox[1])/tf.units_per_em
   local h = tf.glyphs[6].boundingbox[4]/tf.units_per_em
   tex.print("glyph width = ", w,"em\\par")
   tex.print("glyph height = ", h,"em\\par")

-- presents a nicely typeset table 

local rep, write = string.rep, tex.print
function ExploreTable (tab, offset)
    offset = offset or ""
    for k, v in pairs (tab) do
        local newoffset = offset .. "\\mbox{.}"
        if type(v) == "table" then
           -- if k == "boundingbox" then write("BB") end
           write(offset..k .. " = \\{\\par ")
           ExploreTable(v, newoffset)
           write(offset..newoffset .. "\\}\\par")
         else
           write(offset..k .. " = "..tostring(v),"\\par")
         end
      end
end

write("\\par{\\ttfamily ")
ExploreTable(tf.glyphs[38],"\\mbox{.}")
write("}")
  \end{luacode*}
\end{texexample}

The OpenType fonts standard, provides for so much information that we will ignore most of the items and focus on only a few tables and fields. A small utility after Paul Isambert's article is necessary to enable us to view tables easily within this book,


\begin{texexample}{ExploreTable utility}{}
\begin{luacode*}
-- presents a nicely typeset table 

local rep, write = string.rep, tex.print
function ExploreTable (tab, offset)
    offset = offset or ""
    for k, v in pairs (tab) do
        local newoffset = offset .. "\\mbox{.}"
        if type(v) == "table" then
           -- if k == "boundingbox" then write("BB") end
           write(offset..k .. " = \\{\\par ")
           ExploreTable(v, newoffset)
           write(offset..newoffset .. "\\}\\par")
         else
           write(offset..k .. " = "..tostring(v),"\\par")
         end
      end
end

write("\\par{\\ttfamily ")
ExploreTable(tf.glyphs[38],"\\mbox{.}")
write("}")
  \end{luacode*}
\end{texexample}

A good utility also is |TTX| that will convert an OTF font to XML and back. This requires that you have python installed.\footnote{See some good guidelines as to how to install it at \url{http://www.glyphrstudio.com/ttx/}.} The utility uses python to do the conversion. The archive can be downloaded from \url{http://sourceforge.net/projects/fonttools/files/latest/download}. This is a three prong attack. You need to have python install, the numpy library and then the TTX package. The |TTX| program was written by the font designer Just van Rossum, brother of the creator of the Python language, Guido van Rossum. The tool converts TrueType into human-readable |XML| format. The most attractive feature of this tool is that it also perform the opposite operation that is create a TruType font from an |XML| file. The |XML| format makes the hierarchy of the format clearer. Since SVG fonts are also described in |XML| it becomes an easier task to convert an |SVG| font to a TrueType font. To convert |bar.ttf| into |bar.ttx| you simply write:

\begin{verbatim}
ttx bar.ttf
\end{verbatim}

Similarly for the opposite conversion, from |.ttx| to |.ttf|

\begin{verbatim}
ttx bar.ttx
\end{verbatim}

The generated ttx file is approximately ten times larger than the original |.ttf| file. The files generated are huge affairs and difficult to manage.The command line option |-l| prints a list of the tables in the font. |TTX| is indispensable in the ``humanization'' of TrueType fonts. The details of the tables and what each field represents are eloquently described in that indispensable book by Yannis Haralambous \textit{Fonts \& Encodings.} Although the book is now somewhat dated, it is still the best source of information on many esoteric topics related to fonts. 






^^A\input{./languages/meroitic}

\subsection{Old Italic}

\newfontfamily\olditalic{seguisym.ttf}

Old Italic refers to any of several now extinct alphabet systems used on the Italian Peninsula in ancient times for various Indo-European languages (predominantly Italic) and non-Indo-European (e.g. Etruscan) languages. The alphabets derive from the Euboean Greek Cumaean alphabet, used at Ischia and Cumae in the Bay of Naples in the eighth century BC.

Various Indo-European languages belonging to the Italic branch (Faliscan and members of the Sabellian group, including Oscan, Umbrian, and South Picene, and other Indo-European branches such as Celtic, Venetic and Messapic) originally used the alphabet. Faliscan, Oscan, Umbrian, North Picene, and South Picene all derive from an Etruscan form of the alphabet.

The Germanic runic alphabet was derived from one of these alphabets by the 2nd century.
Old Italic is a Unicode block containing a unified repertoire of the three stylistic variants of pre-Roman Italic scripts.

\begin{scriptexample}[]{}
\unicodetable{olditalic}{"10300,"10310,"10320}
\end{scriptexample}

\subsection{Old South Arabian}

\newfontfamily\oldsoutharabian{NotoSansOldSouthArabian-Regular.ttf}

The ancient Yemeni alphabet (Old South Arabian ms3nd; modern Arabic: {\arabicfont المُسنَد‎}  musnad) branched from the Proto-Sinaitic alphabet in about the 9th century BC. It was used for writing the Old South Arabian languages of the Sabaic, Qatabanic, Hadramautic, Minaic (or Madhabic), Himyaritic, and proto-Ge'ez (or proto-Ethiosemitic) in Dʿmt. The earliest inscriptions in the alphabet date to the 9th century BC in Akkele Guzay, Eritrea[3] and in the 10th century BC in Yemen. There are no vowels, instead using the \emph{mater lectionis} to mark them.

Its mature form was reached around 500 BC, and its use continued until the 6th century AD, including Old North Arabian inscriptions in variants of the alphabet, when it was displaced by the Arabic alphabet.[4] In Ethiopia and Eritrea it evolved later into the Ge'ez alphabet,[1][2] which, with added symbols throughout the centuries, has been used to write Amharic, Tigrinya and Tigre, as well as other languages (including various Semitic, Cushitic, and Nilo-Saharan languages).

It is usually written from right to left but can also be written from left to right. When written from left to right the characters are flipped horizontally (see the photo).
The spacing or separation between words is done with a vertical bar mark (\textbar).
Letters in words are not connected together.

Old South Arabian script does not implement any diacritical marks (dots, etc.), differing in this respect from the modern Arabic alphabet.

\begin{scriptexample}[]{South Arabian}
\unicodetable{oldsoutharabian}{"10A60,"10A70}
\end{scriptexample}

Support in \latexe is provided via Peter Wilson's package \pkgname{sarabian}. The package provides all the |metafont| sources as well as transliteration commands and other utilities \seedocs{SARAB}.

\def\SAtdu{\oldsoutharabian\char"10A77}

A comparison between  the unicode and the rendering (scaled 5) \pkgname{sarabian} is shown below.

\centerline{\scalebox{3}{\SAtdu} \scalebox{3}{\textsarab{\SAtd}}}

There is no real advantage in using unicode fonts, if all you interested is to write some South Arabian text for inscriptions. 

\begin{symtable}[SARAB]{\SARAB\ South Arabian Letters}
\index{South Arabian alphabet}
\index{alphabets>South Arabian}
\label{sarabian}
\begin{tabular}{*4{ll@{\qquad}}ll}
\K[\textsarab{\SAa}]\SAa   & \K[\textsarab{\SAz}]\SAz   & \K[\textsarab{\SAm}]\SAm   & \K[\textsarab{\SAsd}]\SAsd & \K[\textsarab{\SAdb}]\SAdb \\
\K[\textsarab{\SAb}]\SAb   & \K[\textsarab{\SAhd}]\SAhd & \K[\textsarab{\SAn}]\SAn   & \K[\textsarab{\SAq}]\SAq   & \K[\textsarab{\SAtb}]\SAtb \\
\K[\textsarab{\SAg}]\SAg   & \K[\textsarab{\SAtd}]\SAtd & \K[\textsarab{\SAs}]\SAs   & \K[\textsarab{\SAr}]\SAr   & \K[\textsarab{\SAga}]\SAga \\
\K[\textsarab{\SAd}]\SAd   & \K[\textsarab{\SAy}]\SAy   & \K[\textsarab{\SAf}]\SAf   & \K[\textsarab{\SAsv}]\SAsv & \K[\textsarab{\SAzd}]\SAzd \\
\K[\textsarab{\SAh}]\SAh   & \K[\textsarab{\SAk}]\SAk   & \K[\textsarab{\SAlq}]\SAlq & \K[\textsarab{\SAt}]\SAt   & \K[\textsarab{\SAsa}]\SAsa \\
\K[\textsarab{\SAw}]\SAw   & \K[\textsarab{\SAl}]\SAl   & \K[\textsarab{\SAo}]\SAo   & \K[\textsarab{\SAhu}]\SAhu & \K[\textsarab{\SAdd}]\SAdd \\
\end{tabular}

\bigskip
\begin{tablenote}
  \usefontcmdmessage{\textsarab}{\sarabfamily}.  Single-character
  shortcuts are also supported: Both
  ``\verb+\textsarab{\SAb\SAk\SAn}+'' and ``\verb+\textsarab{bkn}+''
  produce ``\textsarab{bkn}'', for example.  \seedocs{\SARAB}.
\end{tablenote}
\end{symtable}


\section{South East Asian Scripts}

This section documents the facilities offered to typeset Southeast Asian Scripts. These scripts are used in most of Southeast Asia, Indonesia and the Philippines.

\begin{table}[htb]
\centering
\begin{tabular}{lll}
Thai. & Tai Tham &Balinese.\\
Lao.  &Tai Viet  &Javanese.\\
Myanmar &Kayah Li &Rejang\\
Khmer. &Cham &Batak\\
Tai Le &Philippine Scripts &Sundanese.\\
New Tai Lue & Buginese\\
\end{tabular}
\end{table}

\subsection{Thai}

\newfontfamily\thai[Scale=1.0,Script=Thai]{IrisUPC}

\def\thaitext#1{{\thai#1}}

\begin{scriptexample}[]{Thai}
\centerline{\LARGE\thaitext{◌ะ; ◌ัวะ; เ◌ะ; เ◌อะ; เ◌าะ; เ◌ียะ; เ◌ือะ; แ◌ะ; โ◌ะ}}


\hfill Typeset with \idxfont{IrisUPC} and the command \cmd{\thai}
\end{scriptexample}
\subsection{Balinese}

The Balinese script, natively known as Aksara Bali and Hanacaraka, is an abugida used in the island of Bali, Indonesia, commonly for writing the Austronesian Balinese language, Old Javanese, and the liturgical language Sanskrit. With some modifications, the script is also used to write the Sasak language, used in the neighboring island of Lombok.[1] The script is a descendant of the Brahmi script, and so has many similarities with the modern scripts of South and Southeast Asia. The Balinese script, along with the Javanese script, is considered the most elaborate and ornate among Brahmic scripts of Southeast Asia.[2]

Though everyday use of the script has largely been supplanted by the Latin alphabet, the Balinese script has significant prevalence in many of the island's traditional ceremonies and is strongly associated with the Hindu religion. The script is mainly used today for copying lontar or palm leaf manuscripts containing religious texts.[2][3]

\newfontfamily\balinese{AksaraBali.ttf}
\newfontfamily\indicative{code2000.ttf}

{\indicative ◌ }

\newcounter{under}
\setcounter{under}{"1B00}

\def\cb#1 {
\hspace*{2.5pt}
 \large
 $\text{◌#1}_{\pgfmathparse{Hex(\theunder)}\pgfmathresult}$
\stepcounter{under}
\vskip5pt\par
}
\begin{scriptexample}[]{Balinese}


\balinese
	 
᭐	᭑	᭒	᭓	᭔	᭕	᭖	᭗	᭘	᭙	᭚	᭛	᭜	᭝	᭞	᭟\\\
 
\def\columnseprulecolor{\color{thegray}}
\columnseprule.4pt
\begin{multicols}{8}

\texttt{U+1B0x}	

\cb{ᬀ }  \cb{ ᬁ } 	\cb{ ᬂ } 	\cb ᬃ	\cb ᬄ 	\cb ᬅ	\cb ᬆ	\cb ᬇ	\cb ᬈ	\cb ᬉ	\cb ᬊ	\cb ᬋ	\cb ᬌ	\cb ᬍ	\cb ᬎ	\cb ᬏ

\columnbreak

\texttt{U+1B1x}	 

\cb ᬐ	 \cb ᬑ 	\cb ᬒ 	\cb ᬓ	\cb ᬔ	\cb ᬕ	\cb ᬖ \cb ᬗ 	\cb ᬘ 	\cb ᬙ 	\cb ᬚ	\cb ᬛ 	\cb ᬜ 	\cb ᬝ 	\cb ᬞ	\cb ᬟ 

\columnbreak

U+1B2x	 

\cb ᬠ◌ 	\cb ᬡ	\cb ᬢ	\cb ᬣ	\cb ᬤ	\cb ᬥ	\cb ᬦ	\cb ᬧ	\cb ᬨ	\cb ᬩ	\cb ᬪ	\cb ᬫ	\cb ᬬ	\cb ᬭ	\cb ᬮ	\cb ᬯ

\columnbreak
U+1B3x 

\cb ᬰ	\cb ᬱ	\cb ᬲ	\cb ᬳ	\cb ᬴	\cb ᬵ	\cb ᬶ	\cb ᬷ	\cb ᬸ	\cb ᬹ	\cb ᬺ	\cb ᬻ	\cb ᬼ	\cb ᬽ	\cb ᬾ	\cb ᬿ


\columnbreak
U+1B4x	 

\cb ᭀ	 \cb ᭁ	\cb ᭂ	\cb ᭃ	\cb ᭄	\cb ᭅ	\cb ᭆ	\cb ᭇ	\cb ᭈ	\cb ᭉ	\cb ᭊ	\cb ᭋ

\columnbreak				
U+1B5x	 

\cb ᭐	\cb ᭑	\cb ᭒	\cb ᭓	\cb ᭔	\cb ᭕	\cb ᭖	\cb ᭗	\cb ᭘	\cb ᭙	\cb ᭚	\cb ᭛	\cb ᭜	\cb ᭝	\cb ᭞	\cb ᭟\\

\columnbreak

U+1B6x 

\cb ᭠	\cb ᭡	\cb ᭢	\cb ᭣	\cb ᭤	\cb ᭥	\cb ᭦	\cb ᭧	\cb ᭨◌ 	\cb ᭩◌ 	\cb ᭪◌ 	\cb ᭫	\cb ᭬	\cb ᭭	\cb ᭮	\cb ᭯

\columnbreak
U+1B7x	 

\cb ᭰	 \cb ᭱  \cb ᭲  \cb ᭳	 \cb ᭴	\cb ᭵	\cb ᭶	\cb ᭷	\cb ᭸	\cb ᭹	\cb ᭺	\cb ᭻	\cb ᭼


\end{multicols}

\end{scriptexample}
\defaulttext

One of the most comprehensive fonts is Aksara Bali\footnote{\url{http://www.alanwood.net/downloads/index.html}}. This is obtainable at Alan Wood's website.
\parindent1em
\section{Lao Alphabet}

\def\laotext#1{{\lao#1}}

The Lao alphabet, Akson Lao (Lao: \laotext{ອັກສອນລາວ} [ʔáksɔ̌ːn láːw]), is the main script used to write the Lao language and other minority languages in Laos. It is ultimately of Indic origin, the alphabet includes 27 consonants (\laotext{ພະຍັນຊະນະ} [pʰāɲánsānā]), 7 consonantal ligatures (\laotext{ພະຍັນຊະນະປະສົມ} [pʰāɲánsānā pá sǒm]), 33 vowels (\laotext{ສະຫລະ} [sálā]) (some based on combinations of symbols), and 4 tone marks (\laotext{ວັນນະຍຸດ} [ván nā ɲūt]). 



According to Article 89 of Amended Constitution of 2003 of the Lao People's Democratic Republic, the Lao alphabet is the official script to the official language, but is also used to transcribe minority languages in the country, but some minority language speakers continue to use their traditional writing systems while the Hmong have adopted the Roman Alphabet.[1] An older version of the script was also used by the ethnic Lao of Thailand's Isan region, who make up a third of Thailand's population, before Isan was incorporated into Siam, until its use was banned and supplemented with the very similar Thai alphabet in 1871, although the region remained distant culturally and politically until further government campaigns and integration into the Thai state (Thaification) were imposed in the 20th century.[2] The letters of the Lao Alphabet are very similar to the Thai alphabet, which has the same roots. They differ in the fact, that in Thai there are still more letters to write one sound and the more circular style of writing in Lao.

Lao, like most indic scripts, is traditionally written from left to right. Traditionally considered an \emph{abugida} script, where certain 'implied' vowels are unwritten, recent spelling reforms make this definition somewhat problematic, as all vowel sounds today are marked with diacritics when written according the Lao PDR's propagated and promoted spelling standard. However most Lao outside of Laos, and many inside Laos, continue to write according to former spelling standards, which continues the use of the implied vowel maintaining the Lao script's status as an \emph{abugida}. Vowels can be written above, below, in front of, or behind consonants, with some vowel combinations written before, over and after. Spaces for separating words and punctuations were traditionally not used, but a space is used and functions in place of a comma or period. The letters have no \emph{majuscule} or \emph{minuscule} (upper and lower case) differentiations

The Unicode block for the Lao script is U+0E80–U+0EFF, added in Unicode version 1.0. The first 10 characters of the row U+0EDx are the Lao numerals 0 through 9. Throughout the chart grey (unassigned) code points are shown, because the assigned Lao characters intentionally match the relative positions of the corresponding Thai characters. This has created the anomaly that the Lao letter \laotext{ສ} is not in alphabetical order, since it occupies the same codepoint as the Thai letter \laotext{ส}.

\begin{scriptexample}[]{}
\unicodetable{lao}{"0E80,"0E90,"0EA0,"0EB0,"0EC0,"0ED0,"0EE0,"0EF0}
\end{scriptexample}

\subsubsection{Numerals}
\bgroup
\lao
\begin{tabular}{rllllllllllll}
Hindu-Arabic numerals	&0	&1	&2	&3	&4	&5	&6	&7	&8	&9	&10 &	20\\
Lao numerals	&໐	&໑	&໒	&໓	&໔	&໕	&໖	&໗	&໘	&໙	&໑໐	&໒໐\\
Lao names	&ສູນ	&ນຶ່ງ	&ສອງ	&ສາມ	&ສີ່	&ຫ້າ 	&ຫົກ	&ເຈັດ	&ແປດ	&ເກົ້າ	&ສິບ	&ຊາວ\\
\end{tabular}
\egroup




\newfontfamily\javanese{Noto Sans Javanese}

%\newfontfamily\javanese{TuladhaJejeg_gr.ttf}

\section{Javanese}
\label{s:javanese}
\index{scripts>Javanese}


The Javanese (Ngoko Javanese: {\javanese ꦮꦺꦴꦁꦗꦮ},[3] Madya Javanese: {\javanese\   ꦠꦶꦪꦁꦗꦮꦶ},[4] Krama Javanese: ꦥꦿꦶꦪꦤ꧀ꦠꦸꦤ꧀ꦗꦮꦶ,[4] Ngoko Gêdrìk: wòng Jåwå, Madya Gêdrìk: tiyang Jawi, Krama Gêdrìk: priyantun Jawi, Indonesian: suku Jawa)[5] are an ethnic group native to the Indonesian island of Java. With approximately 100 million people (as of 2011), they form the largest ethnic group in Indonesia. They are predominantly located in the central to eastern parts of the island. There are also significant numbers of people of Javanese descent in most provinces of Indonesia, Malaysia, Singapore, Suriname, Saudi Arabia and the Netherlands.

The Javanese ethnic group has many sub-groups, such as the Mataram, Cirebonese, Osing, Tenggerese, Samin, Naganese, Banyumasan, etc.[6]

A majority of the Javanese people identify themselves as Muslims, with a minority identifying as Christians and Hindus. However, Javanese civilization has been influenced by more than a millennium of interactions between the native animism Kejawen and the Indian Hindu—Buddhist culture, and this influence is still visible in Javanese history, culture, traditions, and art forms. With a sizeable global population, the Javanese are considered significant as they are the fourth largest ethnic group among Muslims, in the world, after the Arabs,[7] Bengalis[8] and Punjabis.[9]


\paragraph{Javanese} is one of the Austronesian languages, but it is not particularly close to other languages and is difficult to classify. Its closest relatives are the neighbouring languages such as Sundanese, Madurese and Balinese. Most speakers of Javanese also speak Indonesian, the standardized form of Malay spoken in Indonesia, for official and commercial purposes as well as a means to communicate with non-Javanese-speaking Indonesians.

There are speakers of Javanese in Malaysia (concentrated in the states of Selangor and Johor) and Singapore. Some people of Javanese descent in Suriname (the Dutch colony of Suriname until 1975) speak a creole descendant of the language.

\begin{figure}[htbp]
\includegraphics[width=\textwidth]{javanese-people}
\end{figure}

The language is spoken in Yogyakarta, Central and East Java, as well as on the north coast of West Java. It is also spoken elsewhere by the Javanese people in other provinces of Indonesia, which are numerous due to the government-sanctioned transmigration program in the late 20th century, including Lampung, Jambi, and North Sumatra provinces. In Suriname, creolized Javanese is spoken among descendants of plantation migrants brought by the Dutch during the 19th century. In Madura, Bali, Lombok, and the Sunda region of West Java, it is also used as a literary language. It was the court language in Palembang, South Sumatra, until the palace was sacked by the Dutch in the late 18th century.

Javanese is written with the Latin script, Javanese script, and Arabic script.[5] In the present day, the Latin script dominates writings, although the Javanese script is still taught as part of the compulsory Javanese language subject in elementary up to high school levels in Yogyakarta, Central and East Java.

Javanese is the tenth largest language by native speakers and the largest language without official status. It is spoken or understood by approximately 100 million people. At least 45\% of the total population of Indonesia are of Javanese descent or live in an area where Javanese is the dominant language. All seven Indonesian presidents since 1945 have been of Javanese descent.[6] It is therefore not surprising that Javanese has had a deep influence on the development of Indonesian, the national language of Indonesia.

There are three main dialects of the modern language: Central Javanese, Eastern Javanese, and Western Javanese. These three dialects form a dialect continuum from northern Banten in the extreme west of Java to Banyuwangi Regency in the eastern corner of the island. All Javanese dialects are more or less mutually intelligible.


\paragraph{The Javanese script} (Hanacaraka/Carakan) is a script for writing the Javanese language, the native language of one of the peoples of the Island of Java. It is a descendent of the ancient Brahmi script of India, and so has many similarities with modern scripts of South Asia and Southeast Asia. The Javanese script is also used for writing Sanskrit, Old Javanese, and transcriptions of Kawi, as well as the Sundanese language, and the Sasak language.

\begin{figure}[htbp]
\hspace*{-1.5cm}\includegraphics[width=1.2\textwidth]{java-palm-leave-manuscript}
\end{figure}





\begin{scriptexample}[]{Javanese}
\bgroup
\javanese

꧋ꦱꦧꦼꦤ꧀ꦮꦺꦴꦁꦏꦭꦲꦶꦂꦲꦏꦺꦏꦤ꧀ꦛꦶꦩꦂꦢꦶꦏꦭꦤ꧀ꦢꦂꦧꦺꦩꦂꦠꦧꦠ꧀ꦭꦤ꧀ꦲꦏ꧀ꦲꦏ꧀ꦏꦁꦥꦝ꧉

꧋ ꦲꦮꦶꦠ꧀ꦲꦶꦏꦁꦄꦱ꧀ꦩꦄꦭ꧀ꦭꦃ꧈ ꦏꦁꦩꦲꦩꦸꦫꦃꦠꦸꦂ ꦩꦲꦲꦱꦶꦃ꧉ 	 
 ۝꧋ ꦄꦭꦶꦥꦃ꧀ ꦭ ꦩ꧀ ꦫ ꧌ ꦏꦁ — — ꦥꦿꦶꦏ꧀ꦱ ꦏꦉꦪꦥ꧀ꦥꦩꦸꦁꦄꦭ꧀ꦭꦃꦥꦶꦪꦺꦩ꧀ꦧꦏ꧀ ꧌꧉ ꦩꦁꦪꦏꦴꦪꦤꦴ ꦲꦶꦏꦸꦄꦪꦺꦪꦠꦴꦏꦶꦠꦧ꧀ꦑꦸꦂꦄꦤ꧀ꦏꦁꦥꦿꦪꦠꦭ꧉ 	 
᭐	᭑	᭒	᭓	᭔	᭕	᭖	᭗	᭘	᭙	᭚	᭛	᭜	᭝	᭞	᭟
 
\egroup
\end{scriptexample}


The Javanese script was added to Unicode Standard in version 5.2 on the code points \texttt{A980 - A9DF}. There are 91 code points for Javanese script: 53 letters, 19 punctuation marks, 10 numbers, and 9 vowels:
\medskip

\unicodetable{javanese}{"A980,"A990,"A9A0, "A9B0, "A9C0,"A9D0}

\medskip



As of the writing of this document (2017), there are several widely published fonts able to support Javanese, ANSI-based Hanacaraka/Pallawa by Teguh Budi Sayoga,[21] Adjisaka by Sudarto HS/Ki Demang Sokowanten,[22] JG Aksara Jawa by Jason Glavy,[23] Carakan Anyar by Pavkar Dukunov,[24] and Tuladha Jejeg by R.S. Wihananto,[25] which is based on Graphite (SIL) smart font technology. Other fonts with limited publishing includes Surakarta made by Matthew Arciniega in 1992 for Mac's screen font,[26] and Tjarakan developed by AGFA Monotype around 2000.[27] There is also a symbol-based font called Aturra developed by Aditya Bayu in 2012–2013.[28]

Due to the script's complexity, many Javanese fonts have different input method compared to other Indic scripts and may exhibit several flaws. \docFont{JG Aksara Jawa}, in particular, may cause conflicts with other writing system, as the font use code points from other writing systems to complement Javanese's extensive repertoire. This is to be expected, as the font was made before Javanese implementation in Unicode.[29]

Arguably, the most "complete" font, in terms of technicality and glyph count, is \docFont{TuladhaJejeg}. It comes with keyboard facilities, displaying complex syllable structure, and support extensive glyph repertoire including non-standard forms which may not be found in regular Javanese texts, by utilizing Graphite (SIL) smart font technology. |Tuladha Jejeg| uses variable stroke widths on its glyphs with serifs on some glyphs\footnote{\protect\url{https://sites.google.com/site/jawaunicode/main-page}}.

However, as not many writing systems require such complex feature, use is limited to programs with Graphite technology, such as Firefox browser, Thunderbird email client, and several OpenType word processor and of course XeLaTeX. The font was chosen for displaying Javanese script in the Javanese Wikipedia.[16]

\paragraph{jawaTeX} Jawa\TeX{} project is initial effort to make Javanese characters typesetting program using \TeX{}/\LaTeX{}. This project is aimed to make Javanese widely used. The main project is developing transliteration models to transliterate Latin document into Javanese document. Perl and \TeX{}/\LaTeX{} are use in this project, the program are develop to run in text mode (console) both Linux and Windows but not limit on it. Web based program also developed, and automatic embedded Javanese characters in HTML See \href{http://jawatex.org/jawa/jawatex}{jawatex}.


\section{Khmer}
\newfontfamily\normaltext{Arial Unicode MS}
\normaltext

\def\khmerdefaultfont#1{\newfontfamily\khmer[Scale=MatchUppercase]{#1}}
\def\khmertext#1{{\khmer#1}}

\cxset{khmer font/.code=\khmerdefaultfont{#1}}

\cxset{khmer font/.default=Khmer}

\cxset{language=khmer, 
       khmer font = Khmer UI}

\begin{key}{/chapter/khmer font=\meta{font name} (Khmer  UI)} Loads the font
command \cmd{\khmer}. When the command is used it typesets text in
khmer unicode. There is no need to load the language, unless it is the main document language. For windows the default font is \texttt{DaunPenh} this font is in general too small to read; a better font to use is Khmer UI.
\end{key}

\begin{key}{/tikz/turtle/right=\meta{angle} (default 90)}
  Turns the turtle right by the given angle. 
\end{key}


The Khmer script (Khmer: {\Large\khmertext{អក្សរខ្មែរ}}; IPA: [ʔaʔsɑː kʰmaːe]) [2] is an \textit{abugida} (alphasyllabary) script used to write the Khmer language (the official language of Cambodia). It is also used to write Pali among the Buddhist liturgy of Cambodia and Thailand.

It was adapted from the Pallava script, a variant of Grantha alphabet descended from the Brahmi script of India, which was used in southern India and South East Asia during the 5th and 6th Centuries AD.[3] The oldest dated inscription in Khmer was found at Angkor Borei District in Takéo Province south of Phnom Penh and dates from 611.[4] The modern Khmer script differs somewhat from precedent forms seen on the inscriptions of the ruins of Angkor.

Not all Khmer consonants can appear in syllable-final position. The most common syllable-final consonants include {\khmer កងញតនបមល}. The pronunciation of the consonant in final position may differ from it's normal pronunciation.


\begin{tabular}{llp{9cm}}
\khmertext{ំ}	&nĭkkôhĕt (\khmertext{និគ្គហិត})	&niggahita; nasalizes the inherent vowels and some of the dependent vowels, see anusvara, sometimes used to represent [aɲ] in Sanskrit loanwords\\
\khmertext{ះ}	&reăhmŭkh (\khmertext{រះមុខ})	&"shining face"; adds final aspiration to dependent or inherent vowels, usually omitted, corresponds to the visarga diacritic, it maybe included as dependent vowel symbol\\
\khmertext{ៈ}	&yŭkôleăkpĭntŭ (\khmertext{យុគលពិន្ទុ})	&yugalabindu ("pair of dots"); adds final glottalness to dependent or inherent vowels, usually omitted\\
\khmertext{៉}	 &musĕkâtônd (\khmertext{មូសិកទន្ត})	&mūsikadanta ("mouse teeth"); used to convert some o-series consonants (\khmertext{ង ញ ម យ រ វ}) to the a-series\\
\khmertext{៊}	&treisâpt (\khmertext{ត្រីសព្ទ})	trīsabda; used to convert some a-series consonants (\khmertext{ស ហ ប អ}) to the o-series\\
\end{tabular}




ុ	kbiĕh kraôm (ក្បៀសក្រោម)	also known as bŏkcheung (បុកជើង); used in place of the diacritics treisâpt and musĕkâtônd when they would be impeded by superscript vowels
់	bântăk (បន្តក់)	used to shorten some vowels; the diacritic is placed on the last consonant of the syllable
៌	rôbat (របាទ)
répheăk (រេផៈ)	rapāda, repha; behave similarly to the tôndâkhéat, corresponds to the Devanagari diacritic repha, however it lost its original function which was to represent a vocalic r
 ៍	tôndâkhéat (ទណ្ឌឃាដ)	daṇḍaghāta; used to render some letters as unpronounced
៎	kakâbat (កាកបាទ)	kākapāda ("crow's foot"); more a punctuation mark than a diacritic; used in writing to indicate the rising intonation of an exclamation or interjection; often placed on particles such as /na/, /nɑː/, /nɛː/, /vəːj/, and the feminine response /cah/
៏	âsda (អស្តា)	denotes stressed intonation in some single-consonant words[5]
័	sanhyoŭk sannha (សំយោគសញ្ញា)	represents a short inherent vowel in Sanskrit and Pali words; usually omitted
៑	vĭréam (វិរាម)	a mostly obsolete diacritic, corresponds to the virāma
្	cheung (ជើង)	a.w. coeng; a sign developed for Unicode to input subscript consonants, appearance of this sign varies among fonts
\section{Sundanese}
\newfontfamily\sundanese{SundaneseUnicode-1.0.5.ttf}
^^A\newfontfamily\sundanese{Arial Unicode MS}
\def\ublock#1{\texttt{{\arial #1}}}

The Sundanese script (Aksara Sunda, {\sundanese ᮃᮊ᮪ᮞᮛ ᮞᮥᮔ᮪ᮓ}) is a writing system which is used by the Sundanese people. It is built based on Old Sundanese script (Aksara Sunda Kuno) which was used by the ancient Sundanese between the 14th and 18th centuries.

\begin{scriptexample}[]{Sundanese}
\unicodetable{sundanese}{"1B80,"1B90,"1BA0,"1BB0}

\sundanese
\obeylines
\bgroup
᮱ {\arial= 1}	᮲ {\arial= 2}	᮳{\arial = 3}
᮴ {\arial= 4}	᮵ {\arial = 5} 	᮶ {\arial= 6}
᮷ {\arial= 7}	᮸ {\arial= 8}	᮹ {\arial= 9}
᮰ {\arial= 0}

\egroup
\end{scriptexample}

\begin{scriptexample}[]{Sundanese}
\bgroup
\sundanese
\centering

◌ᮃᮄᮅᮆᮇᮈᮉᮊᮋᮌᮍᮎᮏᮐᮕᮔᮓᮑᮖᮗᮚᮛᮜᮝᮞᮟᮠᮠ


\egroup
\end{scriptexample}

\bgroup
\def\1{\sundanese ᮱}
\TextOrMath\1\1

$\1$
\egroup

In text In texts, numbers are written surrounded with dual pipe sign \textbar \ldots \textbar. Example: {\textbar \sundanese ᮲᮰᮱᮰\textbar} = 2010













^^A\subsection{Oriya alphabet}
\newfontfamily\oriya[Scale=1.1,Script=Oriya]{code2000.ttf}

\def\oriyatext#1{{\oriya#1}}
The Oriya script or Utkala Lipi (Oriya: \oriyatext{ଉତ୍କଳ ଲିପି}) or Utkalakshara (Oriya: \oriyatext{ଉତ୍କଳାକ୍ଷର}) is used to write the Oriya language, and can be used for several other Indian languages, for example, Sanskrit.

\centerline{\Huge\oriyatext{ଉତ୍କଳ ଲିପି}}

\bgroup
\oriya
୦୧୨୩୪୫୬୭୮୯
ଅ ଆ ଇ ଈ ଉ ଊ ଋ ୠ ଌ ୡ ଏ ଐ ଓ ଔ କ ଖ ଗ ଘ ଙ ଚ ଛ ଜ ଝ ଞ ଟ ଠ ଡ ଢ ଣ ତ ଥ ଦ ଧ ନ ପ ଫ ବ ଵ ଭ ମ ଯ ର ଳ ୱ ଶ ଷ ସ ହ ୟ ଲ
\egroup

\begin{quotation}
Oṛiyā is encumbered with the drawback of an excessively awkward and cumbrous written character. ... At first glance, an Oṛiyā book seems to be all curves, and it takes a second look to notice that there is something inside each.(G. A. Grierson, Linguistic Survey of India, 1903)
\end{quotation}

Comparison of Oṛiyā script with its neighbours[edit]
At a first look the great number of signs with round shapes suggests a closer relation to the southern neighbour Telugu than to the other neighbours Bengali in the north and Devanāgarī in the west. The reason for the round shapes in Oriya and Telugu (and also in Kannaḍa and Malayāḷam) is the former method of writing using a stylus to scratch the signs into a palm leaf. These tools do not allow for horizontal strokes because that would damage the leaf.

Oriya letters are mostly round shaped whereas in Devanāgarī and Bengali have horizontal lines. So in most cases the reader of Oṛiyā will find the distinctive parts of a letter only below the hoop. Considering this the  closer relation to Devanāgarī and Bengali exists than to any southern script, though both northern and southern scripts have the same origin, Brāhmī.

Oriya (\oriyatext{ଓଡ଼ିଆ} oṛiā), officially spelled Odia,[3][4] is an Indian language belonging to the Indo-Aryan branch of the Indo-European language family. It is the predominant language of the Indian states of Odisha, where native speakers comprise 80\% of the population,[5] and it is spoken in parts of West Bengal, Jharkhand, Chhattisgarh and Andhra Pradesh. Oriya is one of the many official languages in India; it is the official language of Odisha and the second official language of Jharkhand. [6][7][8] Oriya is the sixth Indian language to be designated a Classical Language in India, on the basis of having a long literary history and not having borrowed extensively from other languages.

^^A
^^A\subsection{Mongolian Script}

\newfontfamily\mongolian[Language=Mongolian, Scale=1.3]{code2000.ttf}

The classical Mongolian script (in Mongolian script: {\mongolian  ᠮᠣᠩᠭᠣᠯ ᠪᠢᠴᠢᠭ᠌} Mongγol bičig; in Mongolian Cyrillic: Монгол бичиг Mongol bichig), also known as Uyghurjin Mongol bichig, was the first writing system created specifically for the Mongolian language, and was the most successful until the introduction of Cyrillic in 1946. Derived from Uighur, Mongolian is a true alphabet, with separate letters for consonants and vowels. The Mongolian script has been adapted to write languages such as Oirat and Manchu. Alphabets based on this classical vertical script are used in Inner Mongolia and other parts of China to this day to write Mongolian, Sibe and, experimentally, Evenki.
\medskip

\bgroup\par
\noindent
\colorbox{graphicbackground}{\color{black}^^A
\begin{minipage}{\textwidth}^^A
\parindent1pt
\vskip10pt
\leftskip10pt \rightskip\leftskip
\mongolian
\large
ᠬᠦᠮᠦᠨ ᠪᠦᠷ ᠲᠥᠷᠥᠵᠦ ᠮᠡᠨᠳᠡᠯᠡᠬᠦ ᠡᠷᠬᠡ ᠴᠢᠯᠥᠭᠡ ᠲᠡᠢ᠂ ᠠᠳᠠᠯᠢᠬᠠᠨ ᠨᠡᠷ᠎ᠡ ᠲᠥᠷᠥ ᠲᠡᠢ᠂ ᠢᠵᠢᠯ ᠡᠷᠬᠡ ᠲᠡᠢ ᠪᠠᠢᠠᠭ᠃ ᠣᠶᠤᠨ ᠤᠬᠠᠭᠠᠨ᠂ ᠨᠠᠨᠳᠢᠨ ᠴᠢᠨᠠᠷ ᠵᠠᠶᠠᠭᠠᠰᠠᠨ ᠬᠦᠮᠦᠨ ᠬᠡᠭᠴᠢ ᠥᠭᠡᠷ᠎ᠡ ᠬᠣᠭᠣᠷᠣᠨᠳᠣ᠎ᠨ ᠠᠬᠠᠨ ᠳᠡᠭᠦᠦ ᠢᠨ ᠦᠵᠢᠯ ᠰᠠᠨᠠᠭᠠ ᠥᠠᠷ ᠬᠠᠷᠢᠴᠠᠬᠥ ᠤᠴᠢᠷ ᠲᠠᠢ᠃
\par
\vspace*{10pt}
\end{minipage}
}
\medskip
^^A
^^A\subsection{Tibetan}

^^A\newfontfamily\tibetan{TibMachUni.ttf}

^^A\newfontfamily\tibetan{Qomolangma-Chuyig.ttf}

^^A should pick it up automatically \tibetan

Fonts described in this section can be obtained from The Tibetan \& Himalayan Library
\footnote{\url{http://www.thlib.org/tools/scripts/wiki/tibetan%20machine%20uni.html}  }

I have tried a few \texttt{Tibetan Machine Uni (TMU)} seems to be used by a number of scholars. 

A tip when you are trying to locate fonts is to find a related article in Wikipedia, such as Tibetan alphabet and inspect the element using your browser to see what fonts are being used.


|style="font-family:'Jomolhari','Tibetan Machine Uni','DDC Uchen', 'Kailash';| 


If you cannot see the script and rather than boxes or question marks then you can search and download one of the fonts in |font-family|.

\def\tibetandefaultfont#1{\newfontfamily\tibetan[Language=Tibetan]{#1}}


\cxset{language=tibetan} 
\cxset{tibetan font/.code=\tibetandefaultfont{#1}}


^^A\cxset{tibetan font = TibMachUni.ttf}




\begin{key}{/chapter/language = tibetan} The key |language=tibetan| sets the default language as Tibetan, using the main font given by the key |tibetan font=TibMachUni.ttf|.
\end{key}

\begin{key}{/chapter/tibetan font = TibMachUni.ttf} The key |tibetan font=font-name| sets the default font for the Tibetan language. It will also create the switch \cmd{\tibetan} for typesetting text in Tibetan.
\end{key}

\begin{texexample}{Tibetan language setttings}{ex:tibetan}
\cxset{language=tibetan, tibetan font = TibMachUni.ttf}
\tibetan

\tibetan Tibetan: དབུ་ཅན
\end{texexample}


The Tibetan alphabet is an \emph{abugida} of Indic origin used to write the Tibetan language as well as Dzongkha, the Sikkimese language, Ladakhi, and sometimes Balti. 

The printed form of the alphabet is called \textit{uchen} script (Tibetan: དབུ་ཅན་, Wylie: dbu-can; "with a head") while the hand-written cursive form used in everyday writing is called umê script (Tibetan: དབུ་མེད་, Wylie: dbu-med; "headless").
\uccoff
The alphabet is very closely linked to a broad ethnic Tibetan identity. Besides Tibet, it has also been used for Tibetan languages in Bhutan, India, Nepal, and Pakistan.[1] The Tibetan alphabet is ancestral to the Limbu alphabet, the Lepcha alphabet,[2] and the multilingual 'Phags-pa script.[2]
\uccon

The Tibetan alphabet is romanized in a variety of ways.[3] This article employs the Wylie transliteration system.

The Tibetan alphabet has thirty basic letters, sometimes known as "radicals", for consonants.[2]

ཀ ka /ká/	ཁ kha /kʰá/	ག ga /kà, kʰà/	ང nga /ŋà/
ཅ ca /tʃá/	ཆ cha /tʃʰá/	ཇ ja /tʃà/	ཉ nya /ɲà/
ཏ ta /tá/	ཐ tha /tʰá/	ད da /tà, tʰà/	ན na /nà/
པ pa /pá/	ཕ pha /pʰá/	བ ba /pà, pʰà/	མ ma /mà/
ཙ tsa /tsá/	ཚ tsha /tsʰá/	ཛ dza /tsà/	ཝ wa /wà/ (not originally part of the alphabet)[5]
ཞ zha /ʃà/[6]	ཟ za /sà/	འ 'a /hà/[7]
ཡ ya /jà/	ར ra /rà/	ལ la /là/
ཤ sha /ʃá/[6]	ས sa /sá/	ཧ ha /há/[8]
ཨ a /á/

\subsubsection{Unicode Block Tibetan}


\bgroup\large
\begin{tabular}{llllllllllllllll l}
\toprule
	           &|0|	&|1|	&|2|	&|3|	&|4|	&|5|	&|6|	&|7|	&|8|	&|9|	&|A|	&|B|	&|C|	&|D|	&|E|	&|F|\\
\midrule
\texttt{U+0F0x}	&ༀ	&༁	&༂	&༃	&༄	&༅	&༆	&༇	&༈	&༉	&༊	&་	&༌  &	།	&༎	&༏\\
\midrule
\texttt{U+0F1x} &༐	&༑	&༒	&༓	&༔	&༕	&༖	&༗	&༘&	༙	&༚	&༛	&༜	&༝	&༞	&༟\\
\midrule
\texttt{U+0F2x} &༠	&༡	&༢	&༣	&༤	&༥	&༦	&༧	&༨	&༩	&༪	&༫	&༬	&༭	&༮	&༯\\
\midrule
\texttt{U+0F3x}	&༰ &༱	 &༲ &༳	&༴ &༵	&༶ & ༷	&༸&	༹	&༺&	༻	&༼&	༽	&༾	&༿\\
\midrule
\texttt{U+0F4x} &ཀ	&ཁ	&ག	&གྷ	&ང	&ཅ	&ཆ	&ཇ	&	&ཉ	&ཊ	&ཋ	&ཌ	&ཌྷ	&ཎ	&ཏ\\
\midrule
\texttt{U+0F5x}	 &ཐ	&ད	&དྷ	&ན	&པ	&ཕ	&བ	&བྷ	&མ	&ཙ	&ཚ	&ཛ	&ཛྷ	&ཝ	&ཞ	&ཟ\\
\midrule
\texttt{U+0F6x} &འ	&ཡ	&ར	&ལ	&ཤ	&ཥ	&ས	&ཧ	&ཨ	&ཀྵ	&ཪ	&ཫ	&ཬ	&&&\\
^^A\texttt{U+0F7x}&&	ཱ &	& &ི	ཱི&	ུ&	ཱུ&	ྲྀ&	ཷ&	ླྀ&	ཹ&	ེ&	ཻ&	ོ&	ཽ&	&ཾ	&ཿ\\
\midrule
\texttt{U+0F8x}&    ྀ   & 	ཱྀ&	ྂ&	&ྃ &	྄	&྅&	྆	&྇	ྈ&	ྉ&	ྊ&	ྋ&	ྌ&	ྍ&	ྎ&	ྏ\\
\midrule
\texttt{U+0F9x} &	ྐ&	ྑ   & 	ྒ &	ྒྷ &	ྔ &	ྕ &	ྖ &	ྗ &		ྙ &	ྚ &	ྛ &	ྜ &	ྜྷ &	ྞ &	ྟ\\
\texttt{U+0FAx} &	ྠ &	ྡ &	ྡྷ &	ྣ &	ྤ &	ྥ &		&ྦ	&ྦྷ	ྨ&	ྩ&	ྪ&	ྫ&	ྫྷ&	ྭ&	ྮ&	ྯ\\
\midrule
\texttt{U+0FBx} 
&	  ྰ 
&	
& ྱ  	 
&ྲ	
&ླ	
&ྴ
&	ྵ
&	ྶ
&	ྷ
&ྸ
&
&
&
&	
&྾	
&྿\\
\midrule
\texttt{U+0FCx}	 &࿀&	࿁&	࿂&	࿃&	࿄&	࿅&	&࿇	&࿈	&࿉	&࿊	&࿋	&࿌	&&	࿎	&࿏\\
\midrule
\texttt{U+0FDx}	&࿐	&࿑	&࿒	&࿓	&࿔	&࿕	&࿖	&࿗	&࿘	&࿙	&࿚	&&&&&\\
\midrule
\texttt{U+0FEx} &&&&&&&&&&&&&&&&\\
\midrule
\texttt{U+0FFx}  &&&&&&&&&&&&&&&&\\
\bottomrule
\end{tabular}
\egroup




\subsubsection{Fonts for Tibetan}

Fonts for Tibetan need to be downloaded one set of fonts are the \texttt{Qomolangma}. They come in different flavours, but they appear
to offer advantages as compared to the Tibetan Machine Uni.
\medskip


\newfontfamily\betsu{Qomolangma-Betsu.ttf}
\newfontfamily\drutsa{Qomolangma-Drutsa.ttf}
\newfontfamily\chuyig{Qomolangma-Chuyig.ttf}
\newfontfamily\tsumachu{Qomolangma-Tsumachu.ttf}
\newfontfamily\uchensutung{Qomolangma-UchenSutung.ttf}
\newfontfamily\uchensuring{Qomolangma-UchenSuring.ttf}
\newfontfamily\uchensarchen{Qomolangma-UchenSarchen.ttf}
\newfontfamily\uchensarchung{Qomolangma-UchenSarchung.ttf}
\newfontfamily\tsuring{Qomolangma-Tsuring.ttf}
\newfontfamily\TMU{TibMachUni.ttf}
\newfontfamily\himalaya{Microsoft Himalaya}
\uccoff

{
\centering

\renewcommand{\arraystretch}{1.5}

\begin{tabular}{lr}
\toprule
|Qomolangma-Betsu.ttf| & {\betsu  དབུ་མེད }\\
\midrule
|Qomolangma-Chuyig.ttf| &{\chuyig  དབུ་མེད}\\
\midrule
|Qomolangma-Drutsa.ttf| &{\drutsa  དབུ་མེད}\\
\midrule
|Qomolangma-Tsumachu.ttf|&{\tsumachu  དབུ་མེད}\\
\midrule
|Qomolangma-Tsuring.ttf| &{\tsuring  དབུ་མེད}\\
\midrule
|Qomolangma-UchenSarchen.ttf| &{\uchensarchen དབུ་མེད}\\
\midrule
|Qomolangma-UchenSarchung.ttf|&{\uchensarchung དབུ་མེད }\\
\midrule
|Qomolangma-UchenSuring.ttf|&{\uchensuring དབུ་མེད}\\
\midrule
|Qomolangma-UchenSutung.ttf|&{\uchensutung དབུ་མེད }\\
\midrule
|TibMachUni.ttf| &{\TMU དབུ་མེད }\\
\midrule
|Microsoft Himalaya| &{\himalaya དབུ་མེད ཽ}\\
\bottomrule
\end{tabular}

}
\bigskip

\bgroup
\LARGE\tsuring
\noindent༆ །ཨ་ཡིག་དཀར་མཛེས་ལས་འཁྲུངས་ཤེས་བློ  འི་\par
གཏེར༑ །ཕས་རྒོལ་ཝ་སྐྱེས་ཟིལ་གནོན་གདོང་ལྔ་བཞིན།།\par
ཆགས་ཐོགས་ཀུན་བྲལ་མཚུངས་མེད་འཇམ་དབྱངསམཐུས།།\par
མཧཱ་མཁས་པའི་གཙོ་བོ་ཉིད་འགྱུར་ཅིག། །མངྒལཾ༎\par
\egroup

\subsubsection{Tibetan numbers}
\cxset{language=tibetan, tibetan font = TibMachUni.ttf}

{
\obeylines
\small
TIBETAN DIGIT ZERO	༠
TIBETAN DIGIT ONE	༡	
TIBETAN DIGIT TWO	༢	
TIBETAN DIGIT THREE	༣	
TIBETAN DIGIT FOUR	༤	
TIBETAN DIGIT FIVE	༥	
TIBETAN DIGIT SIX	༦	
TIBETAN DIGIT SEVEN	༧	
TIBETAN DIGIT EIGHT	༨	
TIBETAN DIGIT NINE	༩	
TIBETAN DIGIT HALF ONE	\tibetan༪	
TIBETAN DIGIT HALF TWO	༫	
TIBETAN DIGIT HALF THREE	༬
TIBETAN DIGIT HALF FOUR ༭	
TIBETAN DIGIT HALF FIVE ༯	
TIBETAN DIGIT HALF SIX	 ༯	
TIBETAN DIGIT HALF SEVEN	༰	
TIBETAN DIGIT HALF EIGHT	༱	
TIBETAN DIGIT HALF NINE	༲	
TIBETAN DIGIT HALF ZERO	༳	
}


Tibetan numbers

The usage is not certain. By some interpretations, this has the value of 9.5. Used only in some traditional contexts, these appear as the last digit of a multidigit number, eg. ༤༬ represents 42.5. These are very rarely used, however, and other uses have been postulated.

\defaulttext

^^A
^^A
^^A

^^A\section{Tamil}
\newfontfamily\tamil[Scale=1.1,Script=Tamil]{code2000.ttf}

\def\tamiltext#1{{\tamil#1}}

The Tamil script (\tamiltext{தமிழ் அரிச்சுவடி} tamiḻ ariccuvaṭi) is an abugida script that is used by the Tamil people in India, Sri Lanka, Malaysia and elsewhere, to write the Tamil language, as well as to write the liturgical language Sanskrit, using consonants and diacritics not represented in the Tamil alphabet.[1] Certain minority languages such as Saurashtra, Badaga, Irula, and Paniya are also written in the Tamil script

The Tamil script has 12 vowels (\tamiltext{உயிரெழுத்து} uyireḻuttu "soul-letters"), 18 consonants (\tamiltext{மெய்யெழுத்து} meyyeḻuttu "body-letters") and one character, the āytam \tamiltext{ஃ (ஆய்தம்)}, which is classified in Tamil grammar as being neither a consonant nor a vowel (\tamiltext{அலியெழுத்து} aliyeḻuttu "the hermaphrodite letter"), though often considered as part of the vowel set (\tamiltext{உயிரெழுத்துக்கள்} uyireḻuttukkaḷ "vowel class"). The script, however, is syllabic and not alphabetic.[3] The complete script, therefore, consists of the thirty-one letters in their independent form, and an additional 216 combinant letters representing a total 247 combinations (\tamiltext{உயிர்மெய்யெழுத்து} uyirmeyyeḻuttu) of a consonant and a vowel, a mute consonant, or a vowel alone. These combinant letters are formed by adding a vowel marker to the consonant. Some vowels require the basic shape of the consonant to be altered in a way that is specific to that vowel. Others are written by adding a vowel-specific suffix to the consonant, yet others a prefix, and finally some vowels require adding both a prefix and a suffix to the consonant. In every case the vowel marker is different from the standalone character for the vowel.
The Tamil script is written from left to right.

Tamil is a Unicode block containing characters for the Tamil, Badaga, and Saurashtra languages of Tamil Nadu India, Sri Lanka, Singapore, and Malaysia. In its original incarnation, the code points U+0B02..U+0BCD were a direct copy of the Tamil characters A2-ED from the 1988 ISCII standard. The Devanagari, Bengali, Gurmukhi, Gujarati, Oriya, Telugu, Kannada, and Malayalam blocks were similarly all based on their ISCII encodings.

\begin{scriptexample}[]{Tamil}
\unicodetable{tamil}{"0B80,"0B90,"0BA0,"0BB0,"0BC0,"0BE0,"0BF0}

\hfill  Typeset with \cmd{\tamil} and \texttt{code2000.ttf}
\end{scriptexample}

\subsection{Tamil Numbers and Numerals}

Originally, Tamils did not use zero, nor did they use positional digits (having separate 
symbols for the numbers 10, 100 and 1000). Symbols for the numbers are similar to 
other Tamil letters, with some minor changes. 

For example, the number 3782 is not written as \tamiltext{௩௭௮௨} as in modern usage. Instead it 
is written as \tamiltext{௩ ௲ ௭ ௱ ௮ ௰ ௨}. This would be read as they are written as 
Three Thousands, Seven Hundreds, Eight Tens, Two; or in Tamil as 
\tamiltext{௩௲௭௱௮௰௨ž}.\footnote{https://cloud.github.com/downloads/raaman/Tamil-Numeral/tamilnumbers.html}

\subsection{Dates}

Once the script is loaded the day, month and year can be loaded using the command  \cmd{\tamildate}, which returns the |\today| formatted as per custom Tamil. 

\begin{center}
\bgroup
\tamil
\begin{tabular}{lll}
day	 &month	&year	\\

௳	&௴	      &௵	\\

u	&mee	      &wa	\\
\egroup
\end{center}











^^A\chapter{Armenian}

\label{s:armenian}\index{Armenian}\index{scripts>Armenian}

As we present the scripts in alphabetic order, the first script we will typeset is in Armenian. There are many fonts available for the language. We use two in the example, the first one is \textit{FreeSans} and the second is \textit{Sylphaen} which is found on Windows Operating systems. The language is not supported by the \pkg{Babel} and partially supported by the \pkgname{Polyglossia}. \tcbdocmarginnote{china revision}

\def\ucfirst#1#2;{\MakeUppercase#1#2}


\def\armeniantest#1#2{
  {\parindent0pt
  \topline \vskip3pt
  \noindent\mbox{
     \ucfirst#1;\hfill\hbox{[\texttt{U+0530-U+058F}]}
  }}
 \nobreak

\begin{minipage}{0.45\textwidth}
\bgroup
%\setotherlanguage{#1}
\begin{#1}
#2
[\today]
\end{#1}
\egroup
\end{minipage}\hspace*{1em}
\begin{minipage}{0.45\textwidth}
\bgroup
  \parindent0pt
  \ttfamily\raggedright
  \string\documentclass\{article\}\par
  \string\usepackage[no-math]\{fontspec\}\\
  \string\newfontfamily\textbackslash#1font[Script=\ucfirst #1;,\\   ~~~~~~~Scale=MatchLowercase]
\{FreeSans\}\par
  \string\begin\{document\}\\
  \string\setotherlanguage\{#1\}\\
  \string\begin\{#1\}\\
  \egroup
\begin{#1}
\hskip10pt\vbox{#2}
\end{#1}
\bgroup
  \ttfamily[\detokenize{\today}]\\
  \string\end\{#1\}\\
  \string\end\{document\}
\egroup
\end{minipage}


\textit{FreeSans}: \url{ http://www.gnu.org/software/freefont/}
}

\armeniantest{armenian}{Բոլոր մարդիկ ծնվում են ազատ ու հավասար իրենց
արժանապատվությամբ ու իրավունքներով։       
Նրանք ունեն բանականություն ու խիղճ և միմյանց
պետք է եղբայրաբար վերաբերվեն։}

The Armenian script was invented around 407 AD, by Mesrop Maštoc, a cleric who wanted to 
translate Greek and Syriac scriptures and liturgical texts into Armenian. The system he devised 
is a pure alphabet, closely modelled on the traditional order of Greek phonetic values, with 
additional graphemes to represent Armenian sounds not found in Greek. The orthography is, 
phonetically, a near perfect representation of the Armenian language, and has remained almost 
entirely unchanged since its invention. In recent times, the letterforms in many Armenian 
typefaces have consciously modelled Latin types in their treatment of serifs, stroke weight and 
stress, and other details. This is the approach that Geraldine adopted for the Sylfaen Armenian, 
in order to harmonise the different scripts within the font. 

This kind of harmonisation has to be 
very carefully handled, as there is, of course, a point at which one can corrupt the normative 
letterforms and produce something which will be unacceptable to native readers. Once again, 
we sought expert review of the design, this time from Manvel Shmavonyan, an Armenian type designer, and his Russian colleague Vladimir Yefimov at 
ParaType in Moscow.

\bgroup
\medskip
\fontspec[Script=Armenian,Scale=1.7]{Sylfaen}
\centering

Աա Բբ Գգ Դդ Եե Զզ Էէ Ըը Թթ Ժժ Իի \\
Լլ Խխ Ծծ Կկ Հհ Ձձ Ղղ Ճճ Մմ Յյ Նն \\
Շշ Ոո Չչ Պպ Ջջ Ռռ Սս Վվ Տտ Րր Ցց \\
Ււ Փփ Քք Օօ Ֆֆ / և ﬓ ﬔ ﬕ ﬖ ﬗ\\
\egroup
\captionof{table}{Armenian, showing the basic alphabet (typeset using the \textit{Sylfaen} font.}
\medskip

\bgroup
\def\m#1 #2 #3\\{\makebox[2em]{#1}\makebox[2em]{{\fontspec{code2000.ttf}#2}}\makebox[2em]{\hfill#3 \\ }}
\fontspec[Script=Armenian,Scale=1.1]{Sylfaen}

\begin{multicols}{4}
\m Ա	A	1\\
\m Բ	B	2\\
\m Գ	G	3\\
\m Դ	D	4\\
\m Ե	E	5\\
\m Զ	Z	6\\
\m Է	ē	7\\
\m Ը	ə	8\\
\m Թ	tʿ	9\\
\m Ժ	ž	10\\
\m Ի	I	20\\
\m Լ	L	30\\
\m Խ	X	40\\
\m Ծ	C	50\\
\m Կ	K	60\\
\m Հ	H	70\\
\m Ձ	J	80\\
\m Ղ	ł	90\\
\m Ճ	č	100\\
\m Մ	M	200\\
\m Յ	Y	300\\
\m Ն	N	400\\
\m Շ	š	500\\
\m Ո	O	600\\
\m Չ	čʿ	700\\
\m Պ	P	800\\
\m Ջ	ǰ	900\\
\m Ռ	ṙ	1000\\ 
\m Ս	S	2000\\
\m Վ	V	3000\\
\m Տ	T	4000\\
\m Ր	R	5000\\
\m Ց	cʿ	6000\\
\m Ւ	W	7000\\
\m Փ	pʿ	8000\\
\m Ք	kʿ	9000\\

\end{multicols}
\captionof{table}{Armenian Numerals \textit{(from Wikipedia).}
The first column is the classical Armenian numeral, the second the transliteration and the third the arabic numeral it represents.}

\medskip

Numbers in the Armenian numeral system are obtained by simple addition. Armenian numerals are written left-to-right (as in the Armenian language). Although the order of the numerals is irrelevant since only addition is performed, the convention is to write them in decreasing order of value.

\begin{align*}
\text{ՌՋՀԵ} &= 1975 = 1000 + 900 + 70 + 5\\
\text{ՍՄԻԲ} &= 2222 = 2000 + 200 + 20 + 2\\
\text{ՍԴ}   &= 2004 = 2000 + 4\\
\text{ՃԻ}   &= 120 = 100 + 20\\
\text{Ծ}    &= 50
\end{align*}

To write numbers greater than 9999, it is necessary to have numerals with values greater than 9000. This is done by drawing a line over them, indicating their value is to be multiplied by 10000:

\begin{align*}
\overline{\text{Ա}} &= 10000\\
\overline{\text{Ջ}} &= 9000000\\
\overline{\text{ՌՃԽԳ}}\text{ՌՄԾԵ} &= 11431255
\end{align*}
\egroup

^^A

\section{Bopomofo}
\label{s:bopomofo}
Bopomofo is the colloquial name of the \textit{zhuyin fuhao} or \textit{zhuyin} system of phonetic notation for the transcription of spoken Chinese, particularly the Mandarin dialect. Consisting of 37 characters and four tone marks, it transcribes all possible sounds in Mandarin. 

Bopomofo was introduced in China by the Republican Government, in the 1910s and used alongside the Wade-Giles system, which used a modified Latin alphabet. The Wade system was replaced by \textit{Hanyu Pinyin} in 1958 by the Government of the People's Republic of China,[1] at the International Organization for Standardization (ISO) in 1982 (ISO 7098:1982). Bopomofo remains widely used as an educational tool and electronic input method in Taiwan. On Windows the font Microsoft JhengHei can be used. 

Windows fonts that can be used \texttt{Microsoft JhengHei} and \texttt{SimSun}.

U+3100–U+312F
\newfontfamily\bopomofo{Microsoft JhengHei}

\begin{scriptexample}[]{Bopomofo}
{\centering\bopomofo 

伯帛勃脖舶博渤霸壩灞

}

\hfill \texttt{Typeset with \cmd{\bopomofo} and Microsoft JhengHei font }
\end{scriptexample}

\begin{scriptexample}[]{Bopomofo}

{\centering\bopomofo

伯帛勃脖舶博渤霸壩灞

}
\hfill \texttt{Typeset with \cmd{\bopomofo} and JhengHei font }
\end{scriptexample}


The Bopomofo Extended block, running from \unicodenumber{U+31A0-U31BF}, contains less universally recognized Bopomofo characters used to write various non-Mandarin Chinese languages. A few additional tone marks are unified with characters in the Spacing Modifier Letters block. 










^^A\newfontfamily\georgian[Script=Georgian,Scale=1.2]{code2000.ttf}

\newfontfamily\georgianarial[Script=Georgian,Scale=1.2]{Arial Unicode MS}
\section{Georgian}
\label{sec:georgian}
The Georgian scripts are the three writing systems used to write the Georgian language: Asomtavruli, Nuskhuri and Mkhedruli. Their letters are equivalent, sharing the same names and alphabetical order and all three are unicameral (make no distinction between upper and lower case). Although each continues to be used, Mkhedruli (see below) is taken as the standard for Georgian and its related Kartvelian languages\footnote{Unicode Standard, V. 6.3. U10A0, p. 3}. 

\bgroup
\topline



\begin{scriptexample}[]{}
\georgian 

\centering
 
ყველა ადამიანი იბადება თავისუფალი და თანასწორი თავისი ღირსებითა და უფლებებით. მათ მინიჭებული აქვთ გონება და სინდისი და ერთმანეთის მიმართ უნდა იქცეოდნენ ძმობის სულისკვეთებით.
\medskip

\georgianarial
ყველა ადამიანი იბადება თავისუფალი და თანასწორი თავისი ღირსებითა და უფლებებით. მათ მინიჭებული აქვთ გონება და სინდისი და ერთმანეთის მიმართ უნდა იქცეოდნენ ძმობის სულისკვეთებით.
\bottomline
\captionof{table}{Article 1 of the Universal Declaration of Human Rights in Georgian, typeset in \texttt{code2000} (top) and \texttt{Arial Unicode MS } (bottom).}

\end{scriptexample}

The scripts originally had 38 letters. Georgian is currently written in a 33-letter alphabet, as five of the letters are obsolete in that language. The Mingrelian alphabet uses 36: the 33 of Georgian, one letter obsolete for that language, and two additional letters specific to Mingrelian and Svan. That same obsolete letter, plus a letter borrowed from Greek, are used in the 35-letter Laz alphabet. The fourth Kartvelian language, Svan, is not commonly written, but when it is it uses the letters of the Mingrelian alphabet, with an additional obsolete Georgian letter and sometimes supplemented by diacritics for its many vowels.

^^A
^^A\section{Malayalam}
\label{sec:malayam}
\newfontfamily\malayam[Scale=1.1]{Lohit-Malayalam.ttf}

\def\malamtext#1{{\malayam#1}}

The Malayalam script (Malayalam: \malamtext{മലയാളലിപി}, Malayāḷalipi, IPA: [mɐləjaːɭɐ lɪβɪ], also known as Kairali script (Malayalam: \malamtext{കൈരളീലിപി}), is a Brahmic script used commonly to write the Malayalam language—which is the principal language of the Indian state of Kerala, spoken by 35 million people in the world.[3] Like many other Indic scripts, it is an alphasyllabary (\textit{abugida}), a writing system that is partially “alphabetic” and partially syllable-based. The modern Malayalam alphabet has 15 vowel letters, 41 consonant letters, and a few other symbols. The Malayalam script is a Vattezhuttu script, which had been extended with Grantha script symbols to represent Indo-Aryan loanwords.[4] The script is also used to write several minority languages such as Paniya, Betta Kurumba, and Ravula.[5] The Malayalam language itself was historically written in several different scripts.

\begin{scriptexample}[]{Malayalam}
\centerline{\Huge\malamtext{കൈരളീലിപി}}
\end{scriptexample}
^^A\subsection{Greek}
\index{languages>Greek}\index{Herodotus}\index{alphabets>Greek}
\newfontfamily\greek[Script=Greek,Scale=1.02]{NotoSerif-Regular.ttf}
\def\greektext#1{\greek{#1}}

`The Phoenicians who came with Kadmos,' wrote Herodotus in the fifth century BC of the legendary Phoenician prince of Tyre and brother of Europa, `\ldots introduced into Greece, after their settlement in the country, a number of accomplishments of which the most important was writing, an art which probably was unknown to the Greeks until then'. 

The Greek alphabet is the script that has been used to write the Greek language since the 8th century BC.[2] It was derived from the earlier Phoenician alphabet, and was in turn the ancestor of numerous other European and Middle Eastern scripts, including Cyrillic and Latin.[3] Apart from its use in writing the Greek language, both in its ancient and its modern forms, the Greek alphabet today also serves as a source of technical symbols and labels in many domains of mathematics, science and other fields.

In its classical and modern forms, the alphabet has 24 letters, ordered from alpha to omega. Like Latin and Cyrillic, Greek originally had only a single form of each letter; it developed the letter case distinction between upper-case and lower-case forms in parallel with Latin during the modern era.

\bgroup
\greek\obeyspaces

Α	ἄλφα	aleph	alpha	[alpʰa]	[ˈalfa]	Listeni/ˈælfə/
Β	βῆτα	beth	beta	[bɛːta]	[ˈvita]	/ˈbiːtə/, US /ˈbeɪtə/
Γ	γάμμα	gimel	gamma	[ɡamma]	[ˈɣama]	/ˈɡæmə/
Δ	δέλτα	daleth	delta	[delta]	[ˈðelta]	/ˈdɛltə/
Η	ἦτα	  heth	   eta	 [hɛːta], [ɛːta]	[ˈita]	/ˈiːtə/, US /ˈeɪtə/
Θ	θῆτα	teth	theta	[tʰɛːta]	[ˈθita]	/ˈθiːtə/, US Listeni/ˈθeɪtə/
Ι	ἰῶτα	yodh	iota	[iɔːta]	[ˈʝota]	Listeni/aɪˈoʊtə/
Κ	κάππα	kaph	kappa	[kappa]	[ˈkapa]	Listeni/ˈkæpə/
Λ	λάμβδα	lamedh	lambda	[lambda]	[ˈlamða]	Listeni/ˈlæmdə/
Μ	μῦ	mem	mu	[myː]	[mi]	Listeni/ˈmjuː/; occasionally US /ˈmuː/
Ν	νῦ	nun	nu	[nyː]	[ni]	/ˈnjuː/ (US /ˈnuː/)
Ρ	ῥῶ	reš	rho	[rɔː]	[ro]	Listeni/ˈroʊ/
Τ	ταῦ	taw	tau	[tau]	[taf]	/ˈtaʊ/ or /ˈtɔː/

\topline
\begin{quote}
Ἡροδότου Ἁλικαρνησσέος ἱστορίης ἀπόδεξις ἥδε, ὡς μήτε τὰ γενόμενα ἐξ ἀνθρώπων τῷ χρόνῳ ἐξίτηλα γένηται, μήτε ἔργα μεγάλα τε καὶ θωμαστά, τὰ μὲν Ἕλλησι, τὰ δὲ βαρβάροισι ἀποδεχθέντα, ἀκλεᾶ γένηται, τὰ τε ἄλλα καὶ δι' ἣν αἰτίην ἐπολέμησαν ἀλλήλοισι.[2]

Herodotus of Halicarnassus, his Researches are set down to preserve the memory of the past by putting on record the astonishing achievements of both the Greeks and the Barbarians; and more particularly, to show how they came into conflict.[3]
\end{quote}
\bottomline

\symbol{"1F00}
\symbol{"1F01}
\egroup
^^A
^^A\subsection{Kannada alphabet}

\newfontfamily\kannada[Scale=1.0,Script=Kannada]{Lohit-Kannada.ttf}

\def\kannadatext#1{{\kannada#1}}

The Kannada alphabet (\kannadatext{ಕನ್ನಡ ಲಿಪಿ}) is an abugida of the Brahmic family,[2] used primarily to write the Kannada language, one of the Dravidian languages of southern India. Several minor languages, such as Tulu, Konkani, Kodava, and Beary, also use alphabets based on the Kannada script.[3] The Kannada and Telugu scripts share high mutual intellegibility with each other, and are often considered to be regional variants of single script. Similarly, Goykanadi, a variant of Old Kannada, has been historically used to write Konkani in the state of Goa.[4]

\begin{scriptexample}[]{Kannada}
\centerline{\LARGE\kannadatext{ಙ	ಙ್ಕ	ಙ್ಖ	ಙ್ಗ	ಙ್ಘ	ಙ್ಙ	ಙ್ಚ	ಙ್ಛ	ಙ್ಜ	ಙ್ಝ	ಙ್ಞ	ಙ್ಟ	ಙ್ಠ	ಙ್ಡ	ಙ್ಢ}}
\end{scriptexample}

\medskip

The Kannada script (aksharamale or varnamale) is a phonemic abugida of forty-nine letters, and is written from left to right. The character set is almost identical to that of other Brahmic scripts. Consonantal letters imply an inherent vowel. Letters representing consonants are combined to form digraphs (ottaksharas) when there is no intervening vowel. Otherwise, each letter corresponds to a syllable.
The letters are classified into three categories: swara (vowels), vyanjana (consonants), and yogavaahaka (part vowel, part consonant).
The Kannada words for a letter of the script are akshara, akkara, and varna. Each letter has its own form (ākāra) and sound (shabda), providing the visible and audible representations, respectively. Kannada is written from left to right.[7]
^^A\section{Myanmar}
\label{s:myanmar}
\index{Myanmar}\index{Burmese}\index{Mon}\index{Unicode>Myanmar}\index{Fonts>Padauk}

%\newfontfamily\myanmar{Padauk}

The Burmese script (Burmese:{\myanmar မြန်မာအက္ခရာ}; MLCTS: mranma akkha.ra; pronounced: [mjəmà ʔɛʔkʰəjà]) is an abugida in the Brahmic family, used for writing Burmese. It is an adaptation of the Old Mon script[2] or the Pyu script. In recent decades, other alphabets using the Mon script, including Shan and Mon itself, have been restructured according to the standard of the now-dominant Burmese alphabet. Besides the Burmese language, the Burmese alphabet is also used for the liturgical languages of Pali and Sanskrit.

The characters are rounded in appearance because the traditional palm leaves used for writing on with a stylus would have been ripped by straight lines.[3] It is written from left to right and requires no spaces between words, although modern writing usually contains spaces after each clause to enhance readability.

The earliest evidence of the Burmese alphabet is dated to 1035, while a casting made in the 18th century of an old stone inscription points to 984.[1] Burmese calligraphy originally followed a square format but the cursive format took hold from the 17th century when popular writing led to the wider use of palm leaves and folded paper known as parabaiks.[3] The alphabet has undergone considerable modification to suit the evolving phonology of the Burmese language.

Mon/Burmese script was added to the Unicode Standard in September, 1999 with the release of version 3.0. It was extended in October, 2009 with the release of version 5.2 and again in June, 2014 with the release of version 7.0.

\begin{docKey}[phd]{myanmar font}{=\meta{font name}}{default none initial Padauk}
Loads the font and creates associated environments and commands.
\end{docKey}

\begin{scriptexample}[]{Myanmar}
\unicodetable{myanmar}{"1000,"1010,"1020,"1030,"1040,"1050,"1060,"1070,"1080,"1090}
\end{scriptexample}







^^A
^^A\subsection{Osmanian Alphabet}

\bgroup
\newfontfamily\osmanian{code2001.ttf}
\osmanian
𐒚𐒁𐒖𐒄 𐒚𐒐 𐒚 𐒎𐒚𐒍𐒚𐒐 𐒑𐒚𐒒𐒠𐒚𐒐 𐒎𐒚𐒑𐒁𐒗 𐒚𐒁𐒖𐒄 𐒚𐒌𐒖𐒄 𐒚𐒁𐒖𐒄𐒖 𐒚
𐒌𐒜
\egroup
^^A\newfontfamily\hanunoo{NotoSansHanunoo-Regular.ttf}

\section{Hanunó’o}

Hanunó’o is one of the indigenous scripts of the Philippines and is used by the Mangyan peoples of southern Mindoro to write the Hanunó'o language.[1] 

It is an \emphasis{abugida} descended from the Brahmic scripts, closely related to Baybayin, and is famous for being written vertical but written upward, rather than downward as nearly all other scripts (however, it's read horizontally left to right). It is usually written on bamboo by incising characters with a knife.[2][3] Most known Hanunó'o inscriptions are relatively recent because of the perishable nature of bamboo. It is therefore difficult to trace the history of the script



\begin{scriptexample}[width=2cm]{Hanunoo}
\hanunoo

{\Large
\obeylines
ᜠ 
ᜫ
ᜨᜲ
ᜫᜲ
ᜰ
ᜮ
ᜥ
ᜦ᜴}

Typeset with \texttt{NotoSansHanunoo-Regular.ttf} and the command \cmd{\hanunoo}
\end{scriptexample}

Vertically positionning the text is not currently supported by \pkgname{fontspec} and the manual says \textsc{Todo!}. You are your own here, or you can just put the characters in a box and give it a try.

\begin{minipage}[t]{2cm}
\begin{tcolorbox}[width=2cm,colback=graphicbackground,
boxrule=0pt,toprule=0pt,colframe=white]
\Large\hanunoo
ᜩ\\
ᜤ\\
ᜮ\\
ᜥᜳ\\
ᜨ᜴ \\
ᜨ᜴\\
ᜫᜳ\\
ᜥ\\
\end{tcolorbox}
\end{minipage}
\begin{minipage}[t]{2cm}
\begin{tcolorbox}[width=2cm,colback=graphicbackground,
boxrule=0pt,toprule=0pt,colframe=white]
\LARGE\hanunoo
ᜩ\\
ᜤ\\
ᜮ\\
ᜥᜳ\\
ᜨ᜴ \\
ᜨ᜴\\
ᜫᜳ\\
ᜥ\\
\end{tcolorbox}
\end{minipage}
\begin{minipage}[t]{\textwidth-6cm}

The script is written from bottom to top. Typesetting this type of script automatically is not without its problems. One way is to use the build-in features of the font if they are available, but currently this gives problems---at least with the fonts that I have tried. Entering the text is also problematic as you will more than likely see little boxes rather than the actual glyph with most text editors common to \latexe. If you only need a couple of characters or a short sentence, an easy solution is to use |\rotatebox|. Another solution is to use a macro that can add the letters onto a stack, then place them in a box with a limited width. We can use |\@tfor| for this.  
\end{minipage}
^^A
^^A\newfontfamily\glagolitic{MPH 2B Damase}

\section{Glagolitic}

\epigraph{The average Ph.D. thesis is nothing but a transference of bones from one graveyard to another.}{%
J. Frank Dobie (1888-1964)}


\label{s:glagolitic}
\fboxrule0pt\fboxsep0pt

\noindent
The Glagolitic alphabet /{\glagolitic ˌɡlæɡɵˈlɪtɨk/}, also known as Glagolitsa, is the oldest known Slavic alphabet, from the 9th century.

It was created in the 9th century by Saint Cyril, a Byzantine monk from Thessaloniki. He and his brother, Saint Methodius, were sent by the Byzantine Emperor Michael III in 863 to Great Moravia to spread Christianity among the West Slavs in the area. The brothers decided to translate liturgical books into the Old Slavic language that was understandable to the general population, but as the words of that language could not be easily written by using either the Greek or Latin alphabets, Cyril decided to invent a new script, Glagolitic, which he based on the local dialect of the Slavic tribes from the Byzantine Salonika region.
After the deaths of Cyril and Methodius, the Glagolitic alphabet ceased to be used in Moravia, but their students continued to propagate it in the west and south. 

After a long career, Glagolitic writing stopped being used, except for
religious purposes in certain dioceses of Bosnia and Dalmatia (Croatia).
The Cyrillic alphabet was adopted by all Orthodox Slays and served to note
their literary language. Most of the Slays who rallied to Rome rejected it,
however, which created the paradoxical situation in ex-Yugoslavia, where
two peoples who speak the same language write in different scripts, the
Serbs in Cyrillic and the Croats with Roman characters. Finally, as is
known, the ex-Soviet Union did much to put into writing the languages
spoken by the peoples within its borders, for the most part noting them in
adaptations of the Cyrillic alphabet, while Russian became the language of
culture throughout the Soviet Union.\cite{henri1994}

Slavic printing in Glagolitic characters originated in Venice, where a
\textit{Sluzebnik} (or \textit{Leitourgikon}) was published in 1483, followed by missals and
breviaries, all printed by Andrea Torresani, the future father-in-law and
associate of Aldus Manutius. After 1494 some attempts were made to create
printshops in Croatia itself, first in Senj in 1508, then, after 1530, in
Rijeka (Fiume). The work of these firms was almost totally liturgical (religious,
at any rate), and it had strong competition from manuscript works
that were better adapted to the diversity of local liturgical customs. Religion
also dictated the output of a printshop founded to provide Protestant propaganda
that was set up in Tubingen between 1560 and 1564 by Baron
Hans von Ungnad and that printed the great Lutheran texts in Glagolitic
characters.\footfullcite{henri1994}

Figure~\ref{fig:zograf} illustrates an example of the language.\footnote{\url{https://en.wikipedia.org/wiki/Glagolitic_script\#/media/File:ZographensisColour.jpg}}

\begin{figure}[htbp]
\centering

\includegraphics[width=0.45\linewidth]{glagolitic}
\caption[The first page of the Gospel of Mark from the 10th–11th century Codex Zographensis, found in the Zograf Monastery in 1843.]{The first page of the Gospel of Mark from the 10th–11th century Codex Zographensis, found in the Zograf Monastery in 1843.}
\label{fig:zograf}
\end{figure}

\section{Unicode Support}
The Glagolitic alphabet was added to the Unicode Standard in March 2005 with the release of version 4.1.
The Unicode block for Glagolitic is U+2C00–U+2C5F.



\begin{scriptexample}[]{glacolitic}

\unicodetable{glagolitic}{%
"2C00,"2C10,"2C20,"2C30,"2C40,"2C50}

\texttt{typeset with Damase version 2.0 MPH 2B Damase}
\end{scriptexample}
\bgroup
\glagolitic

The name was not coined until many centuries after its creation, and comes from the Old Church Slavonic glagolъ "utterance" (also the origin of the Slavic name for the letter G). The verb glagoliti means "to speak". It has been conjectured that the name glagolitsa developed in Croatia around the 14th century and was derived from the word glagolity, applied to adherents of the liturgy in Slavonic.[1]

In Old Church Slavonic the name is {\glagolitic ⰍⰫⰓⰊⰎⰎⰑⰂⰋⰜⰀ}, Кѷрїлловица.
The name Glagolitic in Bulgarian, Russian, Macedonian глаголица (glagolica), Belarusian is глаголіца (hłaholica), Croatian glagoljica, Serbian глагољица / glagoljica, Czech hlaholice, Polish głagolica, Slovene glagolica, Slovak hlaholika, and Ukrainian глаголиця (hlaholyća).



\egroup

\section{Additional Modern Scripts}

\begin{center}
\begin{tabular}{lp{5cm}l}
Ethiopic. &Vai. &Deseret.\\
Mongolian. &Bamum. &Shavian.\\
Osmanya.   &Cherokee. &Lisu.\\
Tifinagh.  &Canadian Aboriginal Syllabics. &Miao.\\
N’Ko.&&\\
\end{tabular}
\end{center}

Ethiopic, Mongolian, and Tifinagh are scripts with long histories. Although their roots can
be traced back to the original Semitic and North African writing systems, they would not
be classified as Middle Eastern scripts today

The Cherokee script is a syllabary developed between 1815 and 1821, to write the Cherokee
language, still spoken by small communities in Oklahoma and North Carolina. Canadian
Aboriginal Syllabics were invented in the 1830s for Algonquian languages in Canada. The
system has been extended many times, and is now actively used by other communities, including speakers of Inuktitut and Athapascan languages.

Deseret is a phonemic alphabet devised in the 1850s to write English. It saw limited use for
a few decades by members of The Church of Jesus Christ of Latter-day Saints. Shavian is
another phonemic alphabet, invented in the 1950s to write English. It was used to publish
one book in 1962, but remains of some current interest




\subsection{Ethiopic}
Ge'ez (ግዕዝ Gəʿəz), (also known as Ethiopic) is a script used as an abugida (syllable alphabet) for several languages of Ethiopia and Eritrea. It originated as an abjad (consonant-only alphabet) and was first used to write Ge'ez, now the liturgical language of the Ethiopian Orthodox Tewahedo Church and the Eritrean Orthodox Tewahedo Church. In Amharic and Tigrinya, the script is often called fidäl (ፊደል), meaning "script" or "alphabet".

The Ge'ez script has been adapted to write other, mostly Semitic, languages, particularly Amharic in Ethiopia, and Tigrinya in both Eritrea and Ethiopia. It is also used for Sebatbeit, Me'en, and most other languages of Ethiopia. In Eritrea it is used for Tigre, and it has traditionally been used for Blin, a Cushitic language. Tigre, spoken in western and northern Eritrea, is considered to resemble Ge'ez more than do the other derivative languages.[citation needed] Some other languages in the Horn of Africa, such as Oromo, used to be written using Ge'ez, but have migrated to Latin-based orthographies.
For the representation of sounds, this article uses a system that is common (though not universal) among linguists who work on Ethiopian Semitic languages. This differs somewhat from the conventions of the International Phonetic Alphabet. See the articles on the individual languages for information on the pronunciation.

There are a number of fonts available and we have selected the Google \idxfont{NotoSansEthiopic}
\newfontfamily\ethiopic{NotoSansEthiopic-Bold.ttf}

\begin{scriptexample}[]{Ethiopic}
\unicodetable{ethiopic}{"1200,"1210,"1220,"1230,"1240,"1250,"1260,"1270,"1280,"1290,^^A
"12A0,"12B0,"12C0,"12E0,"12F0,"1300,"1310,"1330,"1340,"1350,"1360,"1370}
\end{scriptexample}
\section{Vai}
\label{s:vai}

The Vai syllabary is a syllabic writing system devised for the Vai language by Momolu Duwalu Bukele of Jondu, in what is now Grand Cape Mount County, Liberia.[1] [2] Bukele is regarded within the Vai community, as well as by most scholars, as the syllabary's inventor and chief promoter when it was first documented in the 1830s. It is one of the two most successful indigenous scripts in West Africa.

\newfontfamily\vai{code2000.ttf}
\begin{scriptexample}[]{Vai}
\unicodetable{vai}{"A500,"A510,"A520,"A530,"A540,"A550,"A560,"A570,^^A
"A580,"A590,"A5A0,"A5B0,^^A
"A5C0,"A5D0,"A5E0,"A5F0,"A610,"A620,"A630}
\end{scriptexample}

In the 1920s ten decimal digits were devised for Vai; these were “Vai-style” glyph variants of
European digits (see Figure 11). They were not popular with Vai people  even for historical purposes. All
the modern literature uses European digits.


\begin{scriptexample}[]{Vai}
\bgroup
\vai
\obeylines\Large
0	1	2	3	4	5	6	7	8	9
꘠	꘡	꘢	꘣	꘤	꘥	꘦	꘧	꘨	꘩
\vai
\egroup
\end{scriptexample}



\printunicodeblock{./languages/vai.txt}{\vai}
\section{Deseret script}
\newfontfamily\deseret{code2001.ttf}

The Deseret alphabet (dɛz.əˈrɛt.) (Deseret: {\deseret 𐐔𐐯𐑅𐐨𐑉𐐯𐐻 or 𐐔𐐯𐑆𐐲𐑉𐐯𐐻}) is a phonemic English spelling reform developed in the mid-19th century by the board of regents of the University of Deseret (later the University of Utah) under the direction of Brigham Young, second president of The Church of Jesus Christ of Latter-day Saints.

In public statements, Young claimed the alphabet was intended to replace the traditional Latin alphabet with an alternative, more phonetically accurate alphabet for the English language. This would offer immigrants an opportunity to learn to read and write English, he said, the orthography of which is often less phonetically consistent than those of many other languages. Similar experiments were not uncommon during the period, the most well-known of which is the Shavian alphabet.

Young also prescribed the learning of Deseret to the school system, stating "It will be the means of introducing uniformity in our orthography, and the years that are now required to learn to read and spell can be devoted to other studies".[2]


Deseret script {\deseret 𐐔𐐯𐑅𐐨𐑉𐐯𐐻}  [U+10400-U+1044F]
\medskip

\bgroup
\par
\noindent
\colorbox{graphicbackground}{\color{black}^^A
\begin{minipage}{\textwidth}^^A
\parindent1pt
\vskip10pt
\leftskip10pt \rightskip\leftskip
\deseret
\large

𐐂 𐑌𐐲𐑉𐑅𐐨𐑉𐐮 𐐮𐑆 𐐪 𐐹𐐨𐑅 𐐱𐑂 𐑊𐐰𐑌𐐼 𐐱𐑌 𐐸𐐶𐐮𐐽 𐑁𐑉𐐭𐐻𐐻𐑉𐐨𐑆 𐐪𐑉 𐑅𐐻𐐪𐑉𐐻𐐯𐐼,


\par
\vspace*{10pt}
\end{minipage}
}

Text: Deseret alphabet http://www.omniglot.com/writing/deseret.htm
\medskip
\egroup

\PrintUnicodeBlock{./languages/deseret.txt}{\deseret}

\chapter{Bamum}
\label{s:bamum}
\epigraph{"No known alphabet was ever invented by a European."}{Jeffreys' translation from the Royal script.}

\label{s:bamum}
\index{scripts>Bamum}
\newfontfamily\bamum{NotoSansBamum-Regular.ttf}

The Bamum scripts are an evolutionary series of six scripts created for the Bamum language by King Njoya of Cameroon at the turn of the 20th century. They are notable for evolving from a pictographic system to a partially alphabetic syllabic script in the space of 14 years, from 1896 to 1910. Bamum type was cast in 1918, but the script fell into disuse around 1931.

\begin{figure}[htbp]
\parindent=0pt

\centering

\includegraphics[width=\textwidth]{bamum}

\caption{King Njoya of Bamum receiving an oil painting of Kaiser Wilhelm II. The gift was in return for his support in the German campaign against the Nso'.}
\end{figure}

The Bamum, sometimes called Bamoum, Bamun, Bamoun, or Mum, are a Bantoid ethnic group of Cameroon with around 215,000 members.



\begin{scriptexample}[]{Bamum}
\unicodetable{bamum}{"A6A0,"A6B0,"A6C0,"A6D0,"A6E0,"A6F0}
\end{scriptexample}
\section{Shavian}
\label{s:shavian}
\def\shaviansetup#1{}
\newfontfamily\shavian{code2001.ttf}
^^A\newfontfamily\shavian{NotoSansShavian-Regular.ttf}
\cxset{shavian font/.code=\shaviansetup{#1}}
\cxset{shavian font=shavian}




\begin{scriptexample}[]{shavian}
\shavian

𐑳 𐑡𐑻𐑯𐑰 𐑑 𐑞 𐑕𐑧𐑯𐑑𐑻 𐑝 𐑞 𐑻𐑔
𐑚𐑲 - ·𐑡𐑵𐑤𐑟 ·𐑝𐑻𐑯

𐑗𐑩𐑐𐑑𐑻 1 - 𐑥𐑲 𐑳𐑙𐑒𐑳𐑤 𐑥𐑱𐑒𐑕 𐑳 𐑜𐑮𐑱𐑑 𐑛𐑦𐑕𐑒𐑳𐑝𐑻𐑰

     𐑤𐑫𐑒𐑦𐑙 𐑚𐑩𐑒 𐑑 𐑷𐑤 𐑞𐑩𐑑 𐑣𐑩𐑟 𐑳𐑒𐑻𐑛 𐑑 𐑥𐑰 𐑕𐑦𐑯𐑕 𐑞𐑩𐑑 𐑦𐑝𐑧𐑯𐑑𐑓𐑳𐑤 𐑛𐑱, 𐑲 𐑩𐑥 𐑕𐑒𐑧𐑮𐑕𐑤𐑰 𐑱𐑚𐑳𐑤 𐑑 𐑚𐑦𐑤𐑰𐑝 𐑦𐑯 𐑞 𐑮𐑰𐑩𐑤𐑳𐑑𐑰 𐑝 𐑥𐑲 𐑩𐑛𐑝𐑧𐑯𐑗𐑻𐑟. 𐑞𐑱 𐑢𐑻 𐑑𐑮𐑵𐑤𐑰 𐑕𐑴 𐑢𐑳𐑯𐑛𐑻𐑓𐑳𐑤 𐑞𐑩𐑑 𐑰𐑝𐑦𐑯 𐑯𐑬 𐑲 𐑩𐑥 𐑚𐑦𐑢𐑦𐑤𐑛𐑻𐑛 𐑢𐑧𐑯 𐑲 𐑔𐑦𐑙𐑒 𐑝 𐑞𐑧𐑥.
     𐑥𐑲 𐑳𐑙𐑒𐑳𐑤 𐑢𐑪𐑟 𐑳 𐑡𐑻𐑥𐑳𐑯, 𐑣𐑩𐑝𐑦𐑙 𐑥𐑧𐑮𐑰𐑛 𐑥𐑲 𐑥𐑳𐑞𐑻𐑟 𐑕𐑦𐑕𐑑𐑻, 𐑩𐑯 𐑦𐑙𐑜𐑤𐑦𐑖𐑢𐑫𐑥𐑳𐑯. 𐑚𐑰𐑦𐑙 𐑝𐑧𐑮𐑰 𐑥𐑳𐑗 𐑳𐑑𐑩𐑗𐑑 𐑑 𐑣𐑦𐑟 𐑓𐑪𐑞𐑻𐑤𐑳𐑕 𐑯𐑧𐑓𐑘𐑵, 𐑣𐑰 𐑦𐑯𐑝𐑲𐑑𐑳𐑛 𐑥𐑰 𐑑 𐑕𐑑𐑳𐑛𐑰 𐑳𐑯𐑛𐑻 𐑣𐑦𐑥 𐑦𐑯 𐑣𐑦𐑟 𐑣𐑴𐑥 𐑦𐑯 𐑞 𐑓𐑪𐑞𐑻𐑤𐑩𐑯𐑛. 𐑞𐑦𐑕 𐑣𐑴𐑥 𐑢𐑪𐑟 𐑦𐑯 𐑳 𐑤𐑪𐑮𐑡 𐑑𐑬𐑯, 𐑯 𐑥𐑲 𐑳𐑙𐑒𐑳𐑤 𐑳 𐑐𐑮𐑳𐑓𐑧𐑕𐑻 𐑝 𐑓𐑳𐑤𐑪𐑕𐑳𐑓𐑰, 𐑒𐑧𐑥𐑳𐑕𐑑𐑮𐑰, 𐑡𐑰𐑪𐑤𐑳𐑡𐑰, 𐑥𐑦𐑯𐑻𐑪𐑤𐑳𐑡𐑰, 𐑯 𐑥𐑧𐑯𐑰 𐑳𐑞𐑻 𐑳𐑤𐑴𐑡𐑰𐑕.

\arial

\hfill Excerpt from Jules Vern,  \textit{Journey to the Center of the Earth from \href{http://shavian.weebly.com/}{shavian}}
\end{scriptexample}

The example is typeset using \texttt{code2001.ttf}. There are numerous fonts that provide Shavian glyphs. \texttt{ESL Gothic Unicode} font by Ethan Lamoreaux\footnote{\url{http://www.fontspace.com/ethan-lamoreaux/esl-gothic-unicode}}. The Noto fonts also have a Shavian font. 

You can activate typesetting in Shavian using the key:

\begin{key}{/chapter/shavian font = \meta{font name}} The key will setup the
default font for the Shavian script and define the commands \cmd{\shavian} and \cmd{\textshavian}. 
\end{key}

\PrintUnicodeBlock{./languages/shavian.txt}{\shavian}





\subsection{Osmanya}

\newfontfamily\osmanya{NotoSansOsmanya-Regular.ttf}

\begin{scriptexample}[]{Osmanya}
\unicodetable{osmanya}{"10480,"10490,"104A0}
\end{scriptexample}

The Osmanya alphabet (Somali: Cismaanya; Osmanya: {\osmanya 𐒋𐒘𐒈𐒑𐒛𐒒𐒕𐒀}), also known as Far Soomaali ("Somali writing"), is a writing script created to transcribe the Somali language. It was invented between 1920 and 1922 by Osman Yusuf Kenadid of the Majeerteen Darod clan, the nephew of Sultan Yusuf Ali Kenadid of the Sultanate of Hobyo.

While Osmanya gained reasonably wide acceptance in Somalia and quickly produced a considerable body of literature, it proved difficult to spread among the population mainly due to stiff competition from the long-established Arabic script as well as the emerging Somali alphabet developed by the Somali linguist, Shire Jama Ahmed, which was based on the Latin script.

As nationalist sentiments grew and since the Somali language had long lost its ancient script,[1] the adoption of a universally recognized writing script for the Somali language became an important point of discussion. After independence, little progress was made on the issue, as opinion was divided over whether the Arabic or Latin scripts should be used instead.

In October 1972, due to its simplicity, the fact that it lent itself well to writing Somali since it could cope with all of the sounds in the language, and the already widespread existence of machines and typewriters designed for its use,[2][3] the government of Somali president Mohamed Siad Barre unilaterally elected to use only the Latin script for writing Somali instead of the Arabic or Osmanya scripts.[4] Barre's administration subsequently launched a massive literacy campaign designed to ensure its sole adoption. This led to a sharp decline in use of Osmanya.
\section{Cherokee}
\index{scripts>Cherokee}
\index{scripts>Cherokee>fonts}
\label{sec:cherokee}
Windows comes with |Plantagenet Cherokee| font. The |code2000| also has good support for the alphabet. The \texttt{SIL font Charis SIL} also has good support and can be downloaded at \href{http://scripts.sil.org/cms/scripts/page.php?item_id=CharisSIL_download}{scripts.sel.org}, the latest version gave me problems when used with Windows. 

  
\def\textcherokee#1{{\cherokee   #1}}


\begin{docKey}[phd]{cherokee font}{ = \meta{font name}} {default none, initial=code2000}
 Loads the font
command \cmd{\cherokee}. When the command is used it typesets text in
cherokee unicode. There is no need to load the language, unless it is the main document language. For windows the default font is  |Plantagenet Cherokee|. Another font is FreeSerif, which we are using here.
\end{docKey}

\begin{scriptexample}[]{Cherokee}
{\cherokee
\begin{tabular}{lp{8.5cm}}
Translation	  &John (ᏣᏂ) 3:16\\
American Bible Society 1860	&ᎾᏍᎩᏰᏃ ᏂᎦᎥᎩ ᎤᏁᎳᏅᎯ ᎤᎨᏳᏒᎩ ᎡᎶᎯ, ᏕᏅᏲᏒᎩ ᎤᏤᎵᎦ ᎤᏪᏥ ᎤᏩᏒᎯᏳ ᎤᏕᏁᎸᎯ, ᎩᎶ ᎾᏍᎩ ᏱᎪᎯᏳᎲᏍᎦ ᎤᏲᎱᎯᏍᏗᏱ ᏂᎨᏒᎾ, ᎬᏂᏛᏉᏍᎩᏂ ᎤᏩᏛᏗ.\\

(Transliteration)	& nasgiyeno nigavgi unelanvhi ugeyusvgi elohi, denvyosvgi utseliga uwetsi uwasvhiyu udenelvhi, gilo nasgi yigohiyuhvsga uyohuhisdiyi nigesvna, gvnidvquosgini uwadvdi.\\
\end{tabular}}
\end{scriptexample}

\begin{texexample}{Using text...}{cherokee}
\bgroup
\cherokee \large\textbf{ᎾᏍᎩᏰᏃ}
\textcherokee{ᎾᏍᎩᏰᏃ}
\egroup
\end{texexample}

If you have trouble getting them to work\footnote{\url{http://tex.stackexchange.com/questions/132087/displaying-cherokee-text}}

\url{http://www.cherokee.org/AboutTheNation/Language/CherokeeFont.aspx}




\section{Tifnagh}

\newfontfamily\tifinagh{code2000.ttf}

Tifinagh (Berber pronunciation: [tifinaɣ]; also written Tifinaɣ in the Berber Latin alphabet, {\tifinagh  ⵜⵉⴼⵉⵏⴰⵖ} in Neo-Tifinagh, and تيفيناغ in the Berber Arabic alphabet) is a series of abjad and alphabetic scripts used by Berber peoples to write Berber languages.[1]
A modern derivate of the traditional script, known as Neo-Tifinagh, was introduced in the 20th century. A slightly modified version of the traditional script, called Tifinagh Ircam, is used in a number of Moroccan elementary schools in teaching the Berber language to children as well as a number of publications.[2][3]

The word tifinagh is thought to be a Berberized feminine plural cognate of Punic, through the Berber feminine prefix ti- and Latin Punicus; thus tifinagh could possibly mean "the Phoenician (letters)"[4][5] or "the Punic letters".

\bgroup

\noindent\tifinagh
\colorbox{thecodebackground}{\color{black}^^A
\begin{minipage}{\textwidth}
\parindent1pt
\vskip10pt
\leftskip10pt \rightskip\leftskip
Tifnagh     ⵜⵉⴼⵉⵏⴰⵖ [U+2D30-U+2D7F]

ⴰⴳⵍⴷⵓⵏ ⴰⵎⵥⵥⴰ

ⵙ ⵡⴰⵡⴰⵍ ⴳⵔⵉ ⵉⴷⵙ, ⵙⵙⵏⵖ ⵢⴰⵜ ⵜⵖⴰⵡⵙⴰ ⵜⵉⵙⵙ ⵙⵏⴰⵜ  ⵉⵅⴰⵜⵔⵏ: ⵉⵜⵔⵉ ⵙⴳ ⴷⴷ ⵉⴷⴷⴰ ⵓⵔ ⵉⵎⵇⵇⵓⵔ, ⵉⵍⵍⴰ ⵖⴰⵙ ⴰⵏⵛⵜ ⵏ ⵢⴰⵜ ⵜⴰⴷⴷⴰⵔⵜ !

ⴰⵢⴰ ⵓⴽⵣⵖ ⵜ. ⵙⵙⵏⵖ ⵉⵙ ⴱⵕⵕⴰ ⵏ ⵉⵜⵔⴰⵏ ⵣⵓⵏⴷ ⴰⴽⴰⵍ, ⵊⵓⴱⵉⵜⵔ, ⵎⴰⵔⵙ, ⴱⵉⵏⵓⵙ – ⵉⵜⵔⴰⵏ ⵎⵉ ⵏⴽⴼⴰ ⵉⵙⵎⴰⵡⵏ – ⵍⵍⴰⵏ ⴷⵉⵖ ⵉⵜⵔⴰⵏ ⵢⴰⴹⵏ ⵎⵥⵥⵉⵢⵏⵉⵏ, ⵡⵉⵏⵏⴰ ⵓⵔ ⵏⵣⵎⵉⵔ ⴰⴷ ⵏⵥⵔ ⵙ ⵓⵜⵉⵍⵉⵙⴽⵓⴱ. ⴰⴷⴷⴰⵢ ⵢⵓⴼⴰ ⵓⴰⵙⵜⵕⵓⵏⵓⵎ ⵢⴰⵏ ⴷⵉⴳⵙⵏ, ⴷⴰ ⵢⴰⵙ ⵉⵜⵜⴳⴰ ⵙ ⵢⵉⵙⵎ ⵢⴰⵏ ⵡⵓⵜⵜⵓⵏ. ⴷⴰ ⵢⴰⵙ ⵉⵇⵇⴰⵔ ⵙ ⵓⵎⴷⵢⴰⵜ : « ⴰⵙⵜⵔⵓⵉⴷ 3251 ».

ⵓⴽⵣⵖ ⵉⵙ ⴷⴷ ⵉⴷⴷⴰ ⵓⴳⵍⴷⵓⵏ ⵎⵥⵥⵉⵢⵏ ⵙⴳ ⵉⵜⵔⵉ ⵎⵉ ⵇⵇⴰⵔⵏ ⴰⵙⵜⵔⵓⵉⴷ ⴱ612. ⴰⵙⵜⵔⵓⵉⴷ ⴰ, ⵓⵔ ⵉⵜⵓⵥⵔⴰ ⴰⵔ 1909 ⵙ ⵓⵜⵉⵍⵉⵙⴽⵓⴱ. ⵉⵥⵔⴰ ⵜ ⵢⴰⵏ ⵓⴰⵙⵜⵕⵓⵏⵓⵎ ⴰⵜⵓⵔⴽⵉⵢ. ⵉⵙⵙⴽⵏ ⵜⵓⴼⴰⵢⵜ ⵏⵏⵙ ⴳ ⵢⴰⵏ ⵓⴳⵔⴰⵡ ⴰⴳⵔⴰⵖⵍⴰⵏ ⵏ ⵍⴰⵙⵜⵕⵓⵏⵓⵎⵢ. ⵎⴰⵛⴰ, ⴰⴽⴷ ⵢⵉⵡⵏ ⵓⵔ ⵜ ⵢⵓⵎⵏ ⴰⵛⴽⵓ ⵉⵍⵍⴰ ⵉⵍⵙⴰ ⵢⴰⵜ ⵎⵍⵙⵉⵡⵜ ⵓⵔ ⵉⴳⵉⵏ ⴰⵎⵎ ⵜⵉⵏ ⵎⴷⴷⵏ. ⵎⴷⴷⵏ ⵉⵎⵇⵔⴰⵏⴻⵏ, ⴰⵎⴽⴰ ⴰⴽⴽ ⴰⵢ ⴳⴰⵏ.

ⵎⴰⵛⴰ ⵙ ⵓⵎⴷⴰⵣ ⵏ ⵜⵓⵙⵙⵏⴰ ⵏ ⴰⵙⵜⵔⵓⵉⴷ ⴱ612, ⵉⴽⴽⵔ ⵢⴰⵏ ⵓⴷⵉⴽⵜⴰⵜⵓⵔ ⴰⵜⵓⵔⴽⵢ, ⵉⴳⴳ ⴰⵙⵏ ⵛⵛⵉⵍ ⵉ ⵎⴷⴷⵏ ⴰⴷ ⵍⵙⵙⴰⵏ ⵎⵍⵙⵉⵡⵜ ⵏ ⵓⵔⵓⴱⵉⵢⵏ, ⵡⴰⵏⵏⴰ ⵢⴰⴳⵉⵏ ⵉⵏⵖ ⵜ. ⴰⵙⵜⵔⵓⵏⵓⵎ ⵏⵏⴰⵖ, ⵢⵓⵍⵙ ⴷⵉⵖ ⵉ ⵜⵎⵙⴽⴰⵏⵜ ⵏⵏⵙ ⴰⵙⴳⴳⴰⵙ ⵏ 1920, ⵜⵉⴽⴽⵍⵜ ⵏⵏⴰⵖ ⵉⵍⵍⴰ ⵉⵍⵙⴰ ⵢⴰⵜ ⵎⵍⵙⵉⵡⵜ ⵢⵖⵓⴷⴰⵏ ⵛⵉⴳⴰⵏ. ⵜⵉⴽⴽⵍⵜ ⵏⵏⴰⵖ, ⵎⴷⴷⵏ ⴰⴽⴽ ⵓⵎⴻⵏ ⴰⵡⴰⵍ ⵏⵏⵙ.
\par
\vspace*{10pt}
\end{minipage}
}

\subsection{Unified Canadian Aboriginal Syllabics}

Unified Canadian Aboriginal Syllabics is a Unicode block containing characters for writing Inuktitut, Carrier, several dialects of Cree, and Canadian Athabascan languages. Additions for some Cree dialects, Ojibwe, and Dene can be found at the Unified Canadian Aboriginal Syllabics Extended block.
\medskip

\newfontfamily\aboriginal{code2000.ttf}
\bgroup
\par
\noindent
\colorbox{graphicbackground}{\color{black}^^A
\begin{minipage}{\textwidth}^^A
\parindent1pt
\vskip10pt
\leftskip10pt \rightskip\leftskip

\aboriginal
ᒥᓯᐌ ᐃᓂᓂᐤ ᑎᐯᓂᒥᑎᓱᐎᓂᐠ ᐁᔑ ᓂᑕᐎᑭᐟ ᓀᐢᑕ ᐯᔭᑾᐣ ᑭᒋ ᐃᔑ
\bfseries ᑲᓇᐗᐸᒥᑯᐎᓯᐟ ᑭᐢᑌᓂᒥᑎᓱᐎᓂᐠ ᓀᐢᑕ ᒥᓂᑯᐎᓯᐎᓇ᙮
Unicode Block: Unified Canadian Aboriginal Syllabics, UCAS Extended
Text: UDHR: Cree, Swampy ᐯᔭᐠ ᐱᐢᑭᑕᓯᓇᐃᑲᐣ ᐁᐢᐱᑕᐢᑲᒥᑲᐠ ᐊᐢᑭᐠ ᑭᒋ ᐃᑗᐎᐣ ᐃᓂᓂᐎ ᒥᓂᑯᐎᓯᐎᓇ ᐅᒋ
\par
\vspace*{10pt}
\end{minipage}
}
\medskip
\egroup
\subsection{Miao}

The Pollard script, also known as Pollard Miao (Chinese: 柏格理苗文 Bó Gélǐ Miao-wen) or Miao, is an abugida loosely based on the Latin alphabet and invented by Methodist missionary Sam Pollard. Pollard invented the script for use with A-Hmao, one of several Miao languages. The script underwent a series of revisions until 1936, when a translation of the New Testament was published using it. The introduction of Christian materials in the script that Pollard invented caused a great impact among the Miao. Part of the reason was that they had a legend about how their ancestors had possessed a script but lost it. According to the legend, the script would be brought back some day. When the script was introduced, many Miao came from far away to see and learn it.[1][2]

Pollard credited the basic idea of the script to the Cree syllabics designed by James Evans in 1838–1841, “While working out the problem, we remembered the case of the syllabics used by a Methodist missionary among the Indians of North America, and resolved to do as he had done” (1919:174). He also gave credit to a Chinese pastor, “Stephen Lee assisted me very ably in this matter, and at last we arrived at a system” (1919:174). In listing the phrases he used to describe devising the script, there is clear indication of intellectual work, not revelation: “we looked about”, “resolved to attempt”, “adapting the system”, “solved our problem” (Pollard 1919:174,175).

Changing politics in China led to the use of several competing scripts, most of which were romanizations. The Pollard script remains popular among Hmong in China, although Hmong outside China tend to use one of the alternative scripts. A revision of the script was completed in 1988, which remains in use.

As with most other abugidas, the Pollard letters represent consonants, whereas vowels are indicated by diacritics. Uniquely, however, the position of this diacritic is varied to represent tone. For example, in Western Hmong, placing the vowel diacritic above the consonant letter indicates that the syllable has a high tone, whereas placing it at the bottom right indicates a low tone.

A still experimental font, that supports Graphite technology is \idxfont{Mia Unicode}\footnote{\url{http://phjamr.github.io/miao.html\#intro}}. The font is licenced under the SIL terms and we are using it in the |phd| package as the default font for the Miao script.

\newfontfamily\miao{MiaoUnicode-Regular.ttf}

\begin{scriptexample}[]{Miao}
\unicodetable{miao}{"16F00,"16F10,"16F20,"16F30,"16F40,"16F70,"16F80,"16F90}
\end{scriptexample}

{\miao 𖼴	𖼵	𖼶	𖼷	𖼸	𖼹	𖼺	}

Features for Miao
There are three features currently available for the Miao script:
\bgroup
\miao
Chuxiong ‘wart’ variant
Stylistic alternates for 𖼳 and 𖼴
Aspiration marker always on right
The ‘wart’ (a translated technical term!) is the small circle in characters like 𖼁, 𖼅, and 𖼾. In the Chuxiong orthography, it is rendered not as a circle but as a dot on the right of the letter, as shown in point 5 here (pdf).

Miao Unicode has a feature called “chux” for handling this. In LibreOffice you can use this style by typing “Miao Unicode:chux=1” into the font field.
\section{N'ko}

\newfontfamily\nko{NotoSansNKo-Regular.ttf}

N'Ko {\nko(ߒߞߏ)} is both a script devised by Solomana Kante in 1949 as a writing system for the Manding languages of West Africa, and the name of the literary language itself written in the script. The term N'Ko means ``I say'' in all Manding languages.

The script has a few similarities to the Arabic script, notably its direction (right-to-left) and the connected letters. It obligatorily marks both tone and vowels.


\begin{scriptexample}[]{N'ko}
\unicodetable{nko}{"07C0,"07D0,"07E0,"07F0}
\end{scriptexample}

The N'Ko alphabet is written from right to left, with letters being connected to one another.

The script is principally used in Guinea and Côte d'Ivoire (respectively by Maninka and Dioula-speakers), with an active user community in Mali (by Bambara-speakers). Publications include a translation of the Qur'an, a variety of textbooks on subjects such as physics and geography, poetic and philosophical works, descriptions of traditional medicine, a dictionary, and several local newspapers. It has been classed as the most successful of the West African scripts.[3] The literary language used is intended as a koine blending elements of the principal Manding languages (which are mutually intelligible), but has a particularly strong Maninka flavour.

The Latin script with several extended characters (phonetic additions) is used for all Manding languages to one degree or another for historical reasons and because of its adoption for "official" transcriptions of the languages by various governments. In some cases, such as with Bambara in Mali, promotion of literacy using this orthography has led to a fair degree of literacy in it. Arabic transcription is commonly used for Mandinka in The Gambia and Senegal.


\subsection{Mongolian}
\newfontfamily\mongolian{NotoSansMongolian-Regular.ttf}

The classical Mongolian script (in Mongolian script:{\mongolian ᠮᠣᠩᠭᠣᠯ ᠪᠢᠴᠢᠭ᠌} Mongγol bičig; in Mongolian Cyrillic: Монгол бичиг Mongol bichig), also known as Uyghurjin Mongol bichig, was the first writing system created specifically for the Mongolian language, and was the most successful until the introduction of Cyrillic in 1946. Derived from Uighur, Mongolian is a true alphabet, with separate letters for consonants and vowels. The Mongolian script has been adapted to write languages such as Oirat and Manchu. Alphabets based on this classical vertical script are used in Inner Mongolia and other parts of China to this day to write Mongolian, Sibe and, experimentally, Evenki.

\begin{scriptexample}[]{Mongolian}
\unicodetable{mongolian}{"1820,"1830,"1840,"1850,"1860,"1870,"1880,"1890,"18A0}
\end{scriptexample}



\section{Middle Eastern Scripts}

The scripts in this section have a common origin in the ancient Phoenician alphabet. They include:

\begin{center}
\begin{tabular}{ll}
Hebrew & Samaritan\\
Arabic & Thaana\\
Syriac &\\
\end{tabular}
\end{center}

The Hebrew script is used in Israel and for languages of the Diaspora. The Arabic script is
used to write many languages throughout the Middle East, North Africa, and certain parts
of Asia. The Syriac script is used to write a number of Middle Eastern languages. These
three also function as major liturgical scripts, used worldwide by various religious groups.

The Samaritan script is used in small communities in Israel and the Palestinian Territories
to write the Samaritan Hebrew and Samaritan Aramaic languages. The Thaana script is
used to write Dhivehi, the language of the Republic of Maldives, an island nation in the
middle of the Indian Ocean. 

Text in these scripts is written from right to left. Arabic and Syriac are cursive scripts even when typeset, unlike Hebrew, Samaritan  and Thaana, where letters are unconnected. Most letters in Arabic and Syriac assume different forms depending on their position in a word. Shaping rules are not required for Hebrew because only five letters have position-dependent forms, and these forms are separately encoded.

Historically, Middle Eastern  scripts did not write short vowels. In modern scripts they are represented  by marks positioned above or below a consonantal letter. Vowels and other
marks of pronunciation (“vocalization”) are encoded as combining characters, so support
for vocalized text necessitates use of composed character sequences. Yiddish, Syriac, and
Thaana are normally written with vocalization; Hebrew, Samaritan, and Arabic are usually written unvocalized. 

\section{Hebrew}
\newfontfamily\hebrew{Miriam}
\fontspec{Arial Unicode MS}
To properly typeset Hebrew texts you first need to choose an appropriate font and also set the directionality of the text. This
is done using the etex commands:

\CMDI{\beginL} and \CMDI{\beginR} 

For \XeTeX\ you also need to add near the top of your document |\TeXXeTstate=1|. The package \pkgname{bidi} can be used to set all parameters. Be warned that it redefines almost all of \latexe's commands, so for short mixed texts, I wouldn't recommend its usage. 



The Hebrew alphabet (Hebrew: אָלֶף־בֵּית עִבְרִי[a], alefbet ʿIvri ), known variously by scholars as the Jewish script, square script, block script, is used in the writing of the Hebrew language, as well as other Jewish languages, most notably Yiddish, Ladino, and Judeo-Arabic. There have been two script forms in use; the original old Hebrew script is known as the paleo-Hebrew script (which has been largely preserved, in an altered form, in the Samaritan script), while the present "square" form of the Hebrew alphabet is a stylized form of the Assyrian script. Various "styles" (in current terms, "fonts") of representation of the letters exist. There is also a cursive Hebrew script, which has also varied over time and place. On Windows you can use the \texttt{Miriam} font or \texttt{Arial Unicode MS} or \texttt{Miriam Fixed}.
\medskip

\topline

\bgroup\TeXXeTstate=1
\raggedleft\hebrew{}\beginR

הכתב הכנעני הקדום הלך והתפשט וסימניו היו מוכרים כל כך, עד כי המשתמשים בו התחילו "להתעצל" בהשלמת הציורים, והניחו כי הקורא יבין גם מתוך שרטוטים סכמתיים באיזו אות מדובר. כך, למשל, הפך הראש למשולש עם צוואר; כף היד מלאת האצבעות הפכה לשרטוט דל, ומהדג נותר רק הזנב. כשהעברים אמצו את הכתב הכנעני הם התקשו לזהות חלק מהציורים המקוריים והניחו למשל כי הסימן המתאר את המילה "זהה" הוא כלי נשק; שזנב הדג המשולש הוא דלת, ושדווקא הנחש הוא דג. כך נולדו שמותיהם העבריים של האותיות זי"ן, דל"ת ונו"ן (נון הוא דג, כמו אמנון, שפמנון וכו'). הציורים שהפכו לסימנים התגלגלו לכתבים נוספים, ואפילו ליוונית וללטינית. גם בכתב העברי המודרני ניתן לזהות המשך התפתחותי ברור מן הכתב הכנעני הקדום, והשתמרות שמות האותיות מקלה מאוד על פענוח המקור.


בתקופת בית שני, אומץ האלפבית הארמי לשימוש השפה העברית במקום האלפבית העברי העתיק, כאשר בזה האחרון נעשה שימוש מועט כגון כתיבת השמות הקדושים והטבעת מטבעות. עם הזמן, נעלם גם שימוש זה של הכתב העתיק. האלפבית העברי של ימינו הוא אפוא פיתוח של האלפבית הארמי ולא של הכתב העברי העתיק.	
{}

 לֹ֥א תִשָּׂ֛א

\endR


\egroup
\bottomline
\medskip

To make all paragraphs  RL use the \cmd{\everypar}\footnote{See discussions at \url{http://tex.stackexchange.com/questions/141867/minimal-bidi-for-typesetting-rl-text} and \url{http://www.tug.org/pipermail/xetex/2004-August/000697.html}}. 

\begin{verbatim}
\newbox\mybox \everypar{\setbox\mybox\lastbox\beginR\box\mybox}
\everypar={% at the start of each paragraph, do....
    \setbox0=\lastbox % save the paragraph indent, if any
    \beginR % set R-L direction
    \box0 % then re-insert the indent
	}
\end{verbatim}

The Hebrew alphabet has 22 letters, of which five have different forms when used at the end of a word. Hebrew is written from right to left. Originally, the alphabet was an abjad consisting only of consonants. Like other \textit{abjads}, such as the Arabic alphabet, means were later devised to indicate vowels by separate vowel points, known in Hebrew as niqqud. In rabbinic Hebrew, the letters א ה ו י are also used as matres lectionis to represent vowels. When used to write Yiddish, the writing system is a true alphabet (except for borrowed Hebrew words). In modern usage of the alphabet, as in the case of Yiddish (except that ע replaces ה) and to some extent modern Israeli Hebrew, vowels may be indicated. Today, the trend is toward full spelling with these letters acting as true vowels.

\section{Samaritan}
\newfontfamily\samaritan{NotoSansSamaritan-Regular.ttf}

The Samaritan alphabet is used by the Samaritans for religious writings, including the Samaritan Pentateuch, writings in Samaritan Hebrew, and for commentaries and translations in Samaritan Aramaic and occasionally Arabic.

The Samaritans are, consider themselves to be the descendants of the Northern Tribes of Israel that were not sent into Assyrian captivity, and have continuously resided in the land of Israel.

The Torah Scroll of the Samaritans uses an alphabet that is very different from the one used on Jewish Torah Scrolls. According to the Samaritans themselves and Hebrew scholars, this alphabet is the original "Old Hebrew" alphabet.

Even as far back as 1691, this connection between the Samaritan and the "Old" Hebrew alphabets was made by Henry Dodwell; "[the Samaritans] still preserve [the Pentateuch] in the Old Hebrew characters."

Samaritan is a direct descendant of the Paleo-Hebrew alphabet, which was a variety of the Phoenician alphabet in which large parts of the Hebrew Bible were originally penned. All these scripts are believed to be descendants of the Proto-Sinaitic script. That script was used by the ancient Israelites, both Jews and Samaritans. The better-known "square script" Hebrew alphabet traditionally used by Jews is a stylized version of the Aramaic alphabet which they adopted from the Persian Empire (which in turn adopted it from the Arameans). 

After the fall of the Persian Empire, Judaism used both scripts before settling on the Aramaic form. For a limited time thereafter, the use of paleo-Hebrew (proto-Samaritan) among Jews was retained only to write the Tetragrammaton, but soon that custom was also abandoned.



ShofarRegular StamAshkenazCLM.ttf

\begin{scriptexample}[]{Samaritan}
\bgroup
\TeXXeTstate=1
\unicodetable{samaritan}{"0800,"0810,"0820,"0830}
\egroup
\TeXXeTstate=0
\end{scriptexample}

I battled to get an appropriate font for the Samaritan script and had to use the \idxfont{Noto Sans Samaritan} from Google


^^A\printunicodeblock{./languages/samaritan.txt}{\samaritan}


\url{http://www.ancient-hebrew.org/ahh/ahh.htm#_Toc314842274}



\section{Arabic}

\newfontfamily\arabian{Scheherazade-R.ttf}

The Arabic script is a writing system used for writing several languages of Asia and Africa, such as Arabic, Sorani and Luri Dialects of Kurdish language, Persian, Pashto and Urdu.[1] Even until the 16th century, it was used to write some texts in Spanish.[2] After the Latin script, Chinese characters, and Devanagari, it is the fourth-most widely used writing system in the world.[3]
The Arabic script is written from right to left in a cursive style. In most cases the letters transcribe consonants, or consonants and a few vowels, so most Arabic alphabets are abjads.

The script was first used to write texts in Arabic, most notably the Qurʼān, the holy book of Islam. With the spread of Islam, it came to be used to write languages of many language families, leading to the addition of new letters and other symbols, with some versions, such as Kurdish, Uyghur, and old Bosnian being abugidas or true alphabets. It is also the basis for a rich tradition of Arabic calligraphy.

\begin{verbatim}
\begin{Arabic}
ّ هو إذ الغاية؛ شريف الفوائد، جم المذهب، عزيز فنّ التاريخ فنّ أنّ اعلم
والملوك سيرهم، في والأنبياء أخلاقهم، في الأمم من الماضين أحوال على يوقفنا
ّ أحوال في يرومه لمن ذلك في الإقتداء فائدة تتم حتّى وسياستهم؛ دولهم في
والدنيا. الدين
\end{Arabic}
\end{verbatim}




As of Unicode 7.0, the Arabic script is contained in the following blocks:
Arabic (0600—06FF, 255 characters)
Arabic Supplement (0750—077F, 48 characters)
Arabic Extended-A (08A0—08FF, 39 characters)
Arabic Presentation Forms-A (FB50—FDFF, 608 characters)
Arabic Presentation Forms-B (FE70—FEFF, 140 characters)
Rumi Numeral Symbols (10E60—10E7F, 31 characters)
Arabic Mathematical Alphabetic Symbols (1EE00—1EEFF, 143 characters)[1][2]

The basic Arabic range encodes the standard letters and diacritics, but does not encode contextual forms (U+0621–U+0652 being directly based on ISO 8859-6); and also includes the most common diacritics and Arabic-Indic digits. The Arabic Supplement range encodes letter variants mostly used for writing African (non-Arabic) languages. The Arabic Extended-A range encodes additional Qur'anic annotations and letter variants used for various non-Arabic languages. The Arabic Presentation Forms-A range encodes contextual forms and ligatures of letter variants needed for Persian, Urdu, Sindhi and Central Asian languages. The Arabic Presentation Forms-B range encodes spacing forms of Arabic diacritics, and more contextual letter forms. The presentation forms are present only for compatibility with older standards, and are not currently needed for coding text.[3] 

The Arabic Mathematical Alphabetical Symbols block encodes characters used in Arabic mathematical expressions.

\begin{multicols}{3}
\printunicodeblock{./languages/arabic.txt}{\arabian}
\end{multicols}








\section{Thaana}

\newfontfamily\thaana{MV Boli}
Thaana, Taana or Tāna ({\thaana  ތާނަ}‎ in Tāna script) is the modern writing system of the Maldivian language spoken in the Maldives. Thaana has characteristics of both an abugida (diacritic, vowel-killer strokes) and a true alphabet (all vowels are written), with consonants derived from indigenous and Arabic numerals, and vowels derived from the vowel diacritics of the Arabic abjad. Its orthography is largely phonemic.

The Thaana script first appeared in a Maldivian document towards the beginning of the 18th century in a crude initial form known as Gabulhi Thaana which was written scripta continua. This early script slowly developed, its characters slanting 45 degrees, becoming more graceful and spaces were added between words. 

As time went by it gradually replaced the older Dhives Akuru alphabet. The oldest written sample of the Thaana script is found in the island of Kanditheemu in Northern Miladhunmadulu Atoll. It is inscribed on the door posts of the main Hukuru Miskiy (Friday mosque) of the island and dates back to 1008 AH (AD 1599) and 1020 AH (AD 1611) when the roof of the building were built and the renewed during the reigns of Ibrahim Kalaafaan (Sultan Ibrahim III) and Hussain Faamuladeyri Kilege (Sultan Hussain II) respectively.

\begin{scriptexample}[]{Thaana}
\unicodetable{thaana}{"0780,"0790,"07A0,"07B0}

\hfill Typeset with MV Boli and the command \cmd{\thaana}.
\end{scriptexample}


^^A\printunicodeblock{./languages/thaana.txt}{\thaana}

\subsection{Syriac}

\newfontfamily\syriac{Estrangelo Edessa}

Syriac /ˈsɪriæk/ ({\syriac{ܠܫܢܐ ܣܘܪܝܝܐ}} Leššānā Suryāyā) is a dialect of Middle Aramaic that was once spoken across much of the Fertile Crescent and Eastern Arabia.[1][2][5] Having first appeared as a script in the 1st century AD after being spoken as an unwritten language for five centuries,[6] Classical Syriac became a major literary language throughout the Middle East from the 4th to the 8th centuries,[7] the classical language of Edessa, preserved in a large body of Syriac literature.
It became the vehicle of Syriac Christianity and culture, spreading throughout Asia as far as the Indian Malabar Coast and Eastern China,[8] and was the medium of communication and cultural dissemination for Arabs and, to a lesser extent, Persians. Primarily a Christian medium of expression, Syriac had a fundamental cultural and literary influence on the development of Arabic,[9] which largely replaced it towards the 14th century.[3] Syriac remains the liturgical language of Syriac Christianity.
Syriac is a Middle Aramaic language, and, as such, it is a language of the Northwestern branch of the Semitic family. It is written in the Syriac alphabet, a derivation of the Aramaic alphabet.

\begin{scriptexample}[]{Syriac}
\unicodetable{syriac}{"0700,"0710,"0720,"0730,"0740}
\end{scriptexample}

The Syriac Abbreviation (a type of overline) can be represented with a special control character called the Syriac Abbreviation Mark (U+070F {\syriac \char"070F ܘ}).


\cxset{steward,
  numbering=arabic,
  custom=stewart,
  offsety=0cm,
  image={asia.jpg},
  texti={An introduction to the use of font related commands. The chapter also gives a historical background to font selection using \tex and \latex. },
  textii={In this chapter we discuss keys that are available through the \texttt{phd} package and give a background as to how fonts are used
in \latex.
 },
 pagestyle = empty
}

\arial


\chapter{South Asian Scripts}

The scripts of South Asia share so many characteristics that a side by side comparison of a few often reveal structural similarities even in the 
modern letterforms.
\medskip

\begin{center}
\begin{tabular}{lll}
Devanagari. &Gujarati &Telugu\\
Bengali   &Oriya &Kannada\\
Gurmukhi &Tamil  &Malayalam\\
Sinhala &Kaithi  &Meetei Mayek\\
Tibetan &Saurashtra &Ol Chiki.\\
Lepcha  &Sharada &Sora Sompeng\\
Phags-pa &Takri &Kharoshthi\\
Limbu &Chakma & Brahmi\\
Syloti Nagri & &\\
\end{tabular}
\end{center}

The sections that follow describe the scripts briefly and the |phd| settings
to activate the relevant commands and load appropriate fonts. 

\section{Devanagari}
\parindent1em

Devanagari is part of the Brahmic family of scripts of India, Nepal, Tibet, and South-East Asia.[2] It is a descendant of the Gupta script, along with Siddham and Sharada.[2] Eastern variants of Gupta called nāgarī are first attested from the 7th century CE; from c. 1200 CE these gradually replaced Siddham, which survived as a vehicle for Tantric Buddhism in East Asia, and Sharada, which remained in parallel use in Kashmir. An early version of Devanagari is visible in the Kutila inscription of Bareilly dated to Vikram Samvat 1049 (i.e. 992 CE), which demonstrates the emergence of the horizontal bar to group letters belonging to a word.[3]

Sanskrit nāgarī is the feminine of nāgara "relating or belonging to a town or city". It is feminine from its original phrasing with lipi ("script") as nāgarī lipi "script relating to a city", that is, probably from its having originated in some city.[4]

The use of the name devanāgarī is relatively recent, and the older term nāgarī is still common.[2] The rapid spread of the term devanāgarī may be related to the almost exclusive use of this script to publish Sanskrit texts in print since the 1870s.[2]

On Windows use \texttt{Arial Unicode MS}. 
\medskip

\newfontfamily\devanagari[Script=Devanagari,Scale=1.5]{Arial Unicode MS}

\begin{scriptexample}[]{Devanagari}
{\begin{center}\parindent0pt\devanagari

ंःअआइईउऊऋऌऍऎएऐऑऒओऔऔँ \par 

ी	ु	ू	ृ	ॄ	ॅ	ॆ	े	ै	ॉ	ॊ	ो	ौ	्	\par

\bigskip		
\begin{tabular}{lll lll lll l}
०	&१	&२	&३	&४	&५	&६	&७	&८	&९\\
0	&1	&2	&3	&4	&5	&6	&7	&8	&9\\
\end{tabular}
\end{center}	
}
\end{scriptexample}


On Linux \texttt{Lohit} is a font family designed to cover Indic scripts and released by Red Hat. The Lohit fonts currently cover 11 languages: Assamese, Bengali, Gujarati, Hindi, Kannada, Malayalam, Marathi, Oriya, Punjabi, Tamil, Telugu.[1] The fonts were supplied by Modular Infotech and licensed under the GPL. In September 2011, they were retroactively relicensed under the OFL.[2] The Lohit fonts are used as web fonts by some Wikimedia Foundation sites, like Wikipedia, since March 2012.The font currently support 21 Indian languages. 

\newfontfamily\devanagarilohit[Script=Devanagari,Scale=1.5]{Lohit-Devanagari.ttf}

\begin{scriptexample}[]{Devanagari}
\begin{center}\parindent0pt\devanagarilohit

ंःअआइईउऊऋऌऍऎएऐऑऒओऔऔँ \par 

ी	ु	ू	ृ	ॄ	ॅ	ॆ	े	ै	ॉ	ॊ	ो	ौ	्	\par

\bigskip		
\begin{tabular}{lll lll lll l}
०	&१	&२	&३	&४	&५	&६	&७	&८	&९\\
0	&1	&2	&3	&4	&5	&6	&7	&8	&9\\
\end{tabular}
\end{center}
\end{scriptexample}

\subsubsection{Punctuation} 
The end of a sentence or half-verse may be marked with a dot known as a pūrna virām or a vertical line danda: \textbar. The end of a full verse may be marked with two vertical lines: \textbar\textbar. A comma, or alpa virām, is used to denote a natural pause in speech. With expansion of English speakers in India, the full stop is also sometimes used.

\subsection{LaTeX support}

\latex2e support can be found in the \pkgname{sanskrit}. The package contains the font files and pre-processor for printing Sanskrit
text in both devanāgarī and transliterated Roman with diacritics. Another package that can be used with \XeTeX\ is support \pkgname{devnag}.  This was originally developed by Frans Velthuis for the University of Groningen, The Netherlands, and it was the first system to provide
support for the script for \tex. The package was  extended by Anshuman Pandey. The package provides both fonts as well as tranliteration macros.


\subsection{Gujarati}


Gujarati has its own writing system, distinct but related to several other Indian languages' writing systems, such as the one used to write Hindi. Strictly speaking, the Gujarati writing system is what is called an \emph{abugida} (and not an \textit{alphabet}), because the consonant characters all contain an inherent vowel, and other vowels are written as accents added on to the consonant characters. There are also symbols for stand-alone vowels.

The Gujarati script ({\gujarati{ગુજરાતી લિપિ }} Gujǎrātī Lipi), which like all Nāgarī writing systems is strictly speaking an abugida rather than an alphabet, is used to write the Gujarati and Kutchi languages. It is a variant of Devanāgarī script differentiated by the loss of the characteristic horizontal line running above the letters and by a small number of modifications in the remaining characters.
With a few additional characters, added for this purpose, the Gujarati script is also often used to write Sanskrit and Hindi.
Gujarati numerical digits are also different from their Devanagari counterparts.
\medskip

\bgroup
\newfontfamily\gujaratilohit[Script=Gujarati,Scale=1.5]{Lohit-Gujarati.ttf}
\gujarati

\centering

English/Hindi/Gujarati Alphabets

\begin{tabular}{lllllllllllllllllllll}
A &B &bh &C &ch &chh &D &dh &E &F &G &gh &H &I &J &K &kh &L &M &N &O\\

अ &ब &भ &क &च &छ &ड/द &ध/ढ़ &इ &फ &ग &घ &ह &ई &ज &क &ख &ल &म &न/ण &ऑ\\

અ &બ &ભ &ક &ચ &છ &ડ/દ &ધ /ઢ &ઇ &ફ &ગ &ઘ &હ &ઈ &જ &ક &ખ &લ &મ &ન/ણ &ઓ\\

\end{tabular}
\egroup

\medskip

Gujarati has its own set of numeric signs (placed alongside their Hindu-Arabic [or Indo-Arabic] counterparts in the tables below), they are employed in much the same way as English;  that is to say, they are put together in the same manner in order to express larger numbers. It is quite possible to simply substitute the Gujarati numerals for the Hindu-Arabic ones.

The Gujarati words for 1-10 are as follows:
\medskip

\bgroup
\begin{center}
\gujarati
\begin{tabular}{ccl}
Arabic & Gujarati &Name\\
Numeral &Numeral  &\\
0	&૦	&mīṇḍuṃ or shunya\\
1	&૧	&ekaṛo or ek\\
2	&૨	&bagaṛo or bay\\
3	&૩	&tragaṛo or tran\\
4	&૪	&chogaṛo or chaar\\
5	&૫	&pāchaṛo or paanch\\
6	&૬	&chagaṛo or chah\\
7	&૭	&sātaṛo or sāt\\
8	&૮	&āṭhaṛo or āanth\\
9	&૯	&navaṛo or nav\\
10 &૧૦ &દસ das\\

\end{tabular}
\end{center}
\egroup

\subsection{Bengali}

There are two Windows fonts that can be used with Windows \textit{Shonar Bangla} and \textit{Vrinda}. For open source fonts one can use, \textit{code2000}.
\bigskip

\bgroup
\newfontfamily\bengali[Script=Bengali,Scale=4]{Shonar Bangla}


\bengali
\centering

  অ  আ ই  ঈ  উ  ঊ  ঋ  এ  ঐ\par

\fontspec[Script=Bengali,Scale=3.2]{Vrinda}

\centering

  অ  আ ই  ঈ  উ  ঊ  ঋ  এ  ঐ\par


\fontspec[Script=Bengali,Scale=3.2]{code2000.ttf}

\centering

  অ  আ ই  ঈ  উ  ঊ  ঋ  এ  ঐ\par

\captionof{table}{The consonant{\protect\bengal{} ক (kô)} along with the diacritic form of the vowels {\protect\bengal{} অ, আ, ই, ঈ, উ, ঊ, ঋ, এ, ঐ, ও and ঔ} \textit{from Wikipedia}.}
\egroup

\subsection{Saurashtra}

\newfontfamily\saurashtra{code2000.ttf}

Saurashtra or Sourashtra or {\saurashtra ꢱꣃꢬꢵꢰ꣄ꢜ꣄ꢬꢵ} or Palkar or Patkar (Sanskrit: सौराष्ट्र, Tamil: சௌராட்டிரம்) is an Indo-Aryan language[3] spoken by the Saurashtrian community native to Gujarat, who migrated and settled in Southern India. Madurai in Tamil Nadu has the highest number of people belonging to this community and also remains as their cultural center.

The language is largely only in spoken form even though the language has its own script. The lack of schools teaching Saurashtra script and the language is often cited as a reason for the very few number of people who actually know to read and write in Saurashtra script. Latin, Devanagari or Tamil script is used as alternative for Saurashtra Script by many Saurashtrians.

Census of India places the language under Gujarati. Official figures show the number of speakers as 185,420 (2001 census).[4]



\begin{scriptexample}[]{Saurashtra}
\bgroup
\saurashtra

ꢮꢶꢯ꣄ꢮ ꢱꣃꢬꢵꢰ꣄ꢜ꣄ꢬꢪ꣄ ꢦꢡ꣄ꢬꢶꢒꢾ ꢱꢵꢡ꣄ꢡꢒꢸ ꢂꢮꢬꢾ
ꢮꣁꢭꢱ꣄ꢢꢵꢥꢪꢸꢒ꣄(ꣀꢵꢮꢾꢔꢹ ꢂꢮ꣄ꢬꢶꢫꣁ


\arial

Text: Vishwa Sourashtram \url{http://www.sourashtra.info/ghEr.htm}
\egroup
\end{scriptexample}

\subsection{Ol Chiki script}

The Ol Chiki script, also known as Ol Cemetʼ (Santali: ol 'writing', cemet' 'learning'), Ol Ciki, Ol, and sometimes as the Santali alphabet, was created in 1925 by Raghunath Murmu for the Santali language.

Previously, Santali had been written with the Latin alphabet. But because Santali is not an Indo-Aryan language (like most other languages in the south of India), Indic scripts did not have letters for all of Santali's phonemes, especially its stop consonants and vowels, which made writing the language accurately in an unmodified Indic script difficult. The detailed analysis was given by Dr. Byomkes Chakrabarti in his 'Comparative Study of Santali and Bengali'. Missionaries (first of all Paul Olaf Bodding, a Norwegian) brought the Latin script, which is better at representing Santali stops, phonemes and nasal sounds with the use of diacritical marks and accents. Unlike most Indic scripts, which are derived from Brahmi, Ol Chiki is not an abugida, with vowels given equal representation with consonants. Additionally, it was designed specifically for the language, but one letter could not be assigned to each phoneme because the sixth vowel in Ol Chiki is still problematic.
Ol Chiki has 30 letters, the forms of which are intended to evoke natural shapes. Linguist Norman Zide said "The shapes of the letters are not arbitrary, but reflect the names for the letters, which are words, usually the names of objects or actions representing conventionalized form in the pictorial shape of the characters."[1] It is written from left to right.

\newfontfamily\olchiki{code2000.ttf}

\begin{scriptexample}[]{olchiki}
\bgroup
\olchiki
\obeylines

U+1C5x 	᱐	᱑	᱒	᱓	᱔	᱕	᱖	᱗	᱘	᱙	ᱚ	ᱛ	ᱜ	ᱝ	ᱞ	ᱟ
U+1C6x	   ᱠ	ᱡ	ᱢ	ᱣ	ᱤ	ᱥ	ᱦ	ᱧ	ᱨ	ᱩ	ᱪ	ᱫ	ᱬ	ᱭ	ᱮ	ᱯ
U+1C7x  	ᱰ	ᱱ	ᱲ	ᱳ	ᱴ	ᱵ	ᱶ	ᱷ	ᱸ	ᱹ	ᱺ	ᱻ	ᱼ	ᱽ	᱾	᱿
\egroup
\end{scriptexample}

\subsection{Lepcha}
\newfontfamily\lepcha{Mingzat-R.ttf}

The Lepcha script, or Róng script is an abugida used by the Lepcha people to write the Lepcha language. Unusually for an abugida, syllable-final consonants are written as diacritics.

The Mingzat font is still under development by SIL so I am not too sure if the rendering is correct\footnote{\url{http://scripts.sil.org/cms/scripts/page.php?site_id=nrsi&id=Mingzat}}.

\begin{scriptexample}[]{Lepcha}
\bgroup
\lepcha
\obeylines
 	    0	1	2	3	4	5	6	7	8	9	A	B	C	D	E	F
U+1C0x	 ᰀ	ᰁ	ᰂ	ᰃ	ᰄ	ᰅ	ᰆ	ᰇ	ᰈ	ᰉ	ᰊ	ᰋ	ᰌ	ᰍ	ᰎ	ᰏ
U+1C1x	 ᰐ	ᰑ	ᰒ	ᰓ	ᰔ	ᰕ	ᰖ	ᰗ	ᰘ	ᰙ	ᰚ	ᰛ	ᰜ	ᰝ	ᰞ	ᰟ
U+1C2x	 ᰠ	ᰡ	ᰢ	ᰣ	ᰤ	ᰥ	ᰦ	ᰧ	ᰨ	ᰩ	ᰪ	ᰫ	ᰬ	ᰭ	ᰮ	ᰯ
U+1C3x	 ᰰ	ᰱ	ᰲ	ᰳ	ᰴ	ᰵ	ᰶ	᰷	x	x	x	᰻	᰼	᰽	᰾	᰿
U+1C4x	 ᱀	᱁	᱂	᱃	᱄	᱅	᱆	᱇	᱈	᱉	x	x	x	ᱍ	ᱎ	ᱏ

\egroup
\end{scriptexample}

\subsection{Sharada}

The Śāradā, or Sharada, script (शारदा) is an abugida writing system of the Brahmic family of scripts, developed around the 8th century. It was used for writing Sanskrit and Kashmiri. The Gurmukhī script was developed from Śāradā. Originally more widespread, its use became later restricted to Kashmir, and it is now rarely used except by the Kashmiri Pandit community for ceremonial purposes. Śāradā is another name for Saraswati, the goddess of learning.
Śāradā script was added to the Unicode Standard in January, 2012 with the release of version 6.1.

The Unicode block for Śāradā script, called Sharada, is U+11180–U+111DF: Unable to locate font in unicode.


\subsection{Sora Sompeng}

Sorang Sompeng script is used to write in Sora, a Munda language with 300,000 speakers in India. The script was created by Mangei Gomango in 1936 and is used in religious contexts.[1] He was familiar with Oriya, Telugu and English, so the parent systems of the script are Brahmi and Latin.[2]
The Sora language is also written in the Latin alphabet and the Telugu script.

Sorang Sompeng script was added to the Unicode Standard in January, 2012 with the release of version 6.1. Nirmala UI.ttf (Windows 8.1)



\unicodetable{arial}{"110D0,"110E0,"110F0}
 	
This did not work with Windows 7, and the experiment failed. 

\subsection{Phags-pa}

The 'Phags-pa script,[1], (Mongolian: дөрвөлжин үсэг "Square script") was an alphabet designed by the Tibetan monk and vice-king Drogön Chögyal Phagpa for the Mongol Yuan emperor Kublai Khan as a unified script for the literary languages of the Yuan. Widespread use was limited to about a hundred years during the Yuan Dynasty, and it fell out of use with the advent of the Ming dynasty. The documentation of its use provides clues about the changes in the varieties of Chinese, the Tibetic languages, Mongolian and other neighboring languages during the Yuan era.

\newfontfamily\phagspa{code2000.ttf}

\begin{scriptexample}[]{Phags-pa}
\bgroup
\obeylines
\phagspa

 	0	1	2	3	4	5	6	7	8	9	A	B	C	D	E	F
U+A84x	ꡀ	ꡁ	ꡂ	ꡃ	ꡄ	ꡅ	ꡆ	ꡇ	ꡈ	ꡉ	ꡊ	ꡋ	ꡌ	ꡍ	ꡎ	ꡏ
U+A85x	ꡐ	ꡑ	ꡒ	ꡓ	ꡔ	ꡕ	ꡖ	ꡗ	ꡘ	ꡙ	ꡚ	ꡛ	ꡜ	ꡝ	ꡞ	ꡟ
U+A86x	ꡠ	ꡡ	ꡢ	ꡣ	ꡤ	ꡥ	ꡦ	ꡧ	ꡨ	ꡩ	ꡪ	ꡫ	ꡬ	ꡭ	ꡮ	ꡯ
U+A87x	ꡰ	ꡱ	ꡲ	ꡳ	꡴	꡵	꡶	


ꡏꡟ ꡋꡞ ꡏꡟ ꡋꡞ ꡏ ꡜꡖ ꡏꡟ ꡋꡞ ꡓꡞ ꡏꡟ
ꡈꡋ ꡋꡋ ꡓꡘ ꡈ ꡭ ꡏ ꡏ ꡝ ꡭꡟꡘ ꡓꡋ ꡮꡟꡊ
\egroup
\bgroup
\raggedright

\setcounter{glyphcount}{"A840}

\topline
\phagspa
\newcount\n
\n="A840

\def\htable{^^A
  \def\fm##1{\makebox[2em]##1}^^A
  U+A840\fm 0\fm1\fm2\fm3\fm4\fm5\fm 6\fm 7\fm 8\fm	9\fm A\fm B\fm C\fm D\fm E\fm F}

\htable\par
U+A840^^A 
\loop^^A
  \makebox[2em]{\char\n }^^A   
   \advance\n by1 ^^A
   \ifnum\n<"A850^^A
\repeat
\par U+A850^^A
\loop^^A
  \makebox[2em]{\char\n }^^A   
   \advance\n by1 ^^A
  \ifnum\n<"A860^^A
\repeat
\par U+A860^^A
\loop^^A
  \makebox[2em]{\char\n }^^A   
   \advance\n by1 ^^A
  \ifnum\n<"A870^^A
\repeat
\par U+A870^^A
\loop^^A
  \makebox[2em]{\char\n }^^A   
   \advance\n by1 ^^A
  \ifnum\n<"A878^^A
\repeat

\bottomline

\arial
\hfill Typeset with \texttt{code2000.ttf} and \cmd{\phagspa}

Text: \href{http://babelstone.blogspot.com/2006/12/phags-pa-fonts-1-babelstone-phags-pa.html}{babelstone}
\egroup
\end{scriptexample}

Phags-pa is a historical script related to Tibetan that was created as the national script of
the Mongol empire. Even though Phags-pa was used mostly in Eastern and Central Asia for
writing text in the Mongolian and Chinese languages, it is discussed in this chapter because
of its close historical connection to the Tibetan script. The script has very limited modern use. It bears similarity to Tibetan and has no case distinctions. It is written vertically in columns running for left to right, like Mongolian. Units are often composed of several syllables and sometimes are separated by whitespace.


\subsection{Syloti Nagri}
\index{languages>Sylheti Nagari}
Sylheti Nagari or Syloti Nagri (Silôṭi Nagôri) is the original script used for writing the Sylheti language. It is an almost extinct script, this is because the Sylheti Language itself was reduced to only dialect status after Bangladesh gained independence and because it did not make sense for a dialect to have its own script, its use was heavily discouraged. The government of the newly formed Bangladesh did so to promote a greater "Bengali" identity. This led to the informal adoption of the Eastern Nagari script also used for Bengali and Assamese. It is also known as Jalalabadi Nagri, Mosolmani Nagri, Ful Nagri etc.

\newfontfamily\syloti{NotoSansSylotiNagri-Regular.ttf}
\newfontfamily\damase{damase_v.2.ttf}
\bgroup
\damase
\obeylines
	0	1	2	3	4	5	6	7	8	9	A	B	C	D	E	F
U+A80x	ꠀ	ꠁ	ꠂ	ꠃ	ꠄ	ꠅ	꠆	ꠇ	ꠈ	ꠉ	ꠊ	ꠋ	ꠌ	ꠍ	ꠎ	ꠏ
U+A81x	ꠐ	ꠑ	ꠒ	ꠓ	ꠔ	ꠕ	ꠖ	ꠗ	ꠘ	ꠙ	ꠚ	ꠛ	ꠜ	ꠝ	ꠞ	ꠟ
U+A82x	ꠠ	ꠡ	ꠢ	ꠣ	ꠤ	ꠥ	ꠦ	ꠧ	꠨	꠩	꠪	꠫
\egroup

\subsection{Chakma}

\newfontfamily\chakma{RibengUni.ttf}

\bgroup
\chakma
𑄇𑄳𑄇 Kkā = 𑄇 Kā + 𑄳 VIRAMA + 𑄇 Kā
𑄇𑄳𑄑 Ktā = 𑄇 Kā + 𑄳 VIRAMA + 𑄑 Tā
𑄇𑄳𑄖 Ktā = 𑄇 Kā + 𑄳 VIRAMA + 𑄖 Tā
𑄇𑄳𑄟 Kmā = 𑄇 Kā + 𑄳 VIRAMA + 𑄟 Mā
𑄇𑄳𑄌 Kcā = 𑄇 Kā + 𑄳 VIRAMA + 𑄌 Cā
𑄋𑄳𑄇 ńkā = 𑄋 ńā + 𑄳 VIRAMA + 𑄇 Kā
𑄋𑄳𑄉 ńkā = 𑄋 ńā + 𑄳 VIRAMA + 𑄉 Gā
𑄌𑄳𑄌 ccā = 𑄌 cā + 𑄳 VIRAMA + 𑄌 Cā

\egroup

\subsection{Limbu}

The Limbu script is used to write the Limbu language. The Limbu script is an abugida derived from the Tibetan script. Limbu is a Tibeto-Burman language spoken mainly in Nepal,[3] significant communities in Bhutan, Sikkim, Darjeeling district, India by the Limbu community. Virtually all Limbus are bilingual in Nepali.

\newfontfamily\limbu{code2000.ttf}
\bgroup
\obeylines
\limbu
0	1	2	3	4	5	6	7	8	9	A	B	C	D	E	F
U+190x	ᤀ	ᤁ	ᤂ	ᤃ	ᤄ	ᤅ	ᤆ	ᤇ	ᤈ	ᤉ	ᤊ	ᤋ	ᤌ	ᤍ	ᤎ	ᤏ
U+191x	ᤐ	ᤑ	ᤒ	ᤓ	ᤔ	ᤕ	ᤖ	ᤗ	ᤘ	ᤙ	ᤚ	ᤛ	ᤜ	ᤝ	ᤞ	
U+192x	ᤠ	ᤡ	ᤢ	ᤣ	ᤤ	ᤥ	ᤦ	ᤧ	ᤨ	ᤩ	ᤪ	ᤫ				
U+193x	ᤰ	ᤱ	ᤲ	ᤳ	ᤴ	ᤵ	ᤶ	ᤷ	ᤸ	᤹	᤺	᤻				
U+194x	᥀				᥄	᥅	᥆	᥇	᥈	᥉	᥊	᥋	᥌	᥍	᥎	᥏
\egroup

\subsection{Brahmi}



Brāhmī is the modern name given to one of the oldest writing systems used in the Indian subcontinent and in Central Asia during the final centuries BCE and the early centuries CE. Like its contemporary, Kharoṣṭhī, which was used in what is now Afghanistan and Western Pakistan, Brahmi (native to north and central India) was an \emph{abugida}.

The best-known Brahmi inscriptions are the rock-cut edicts of Ashoka in north-central India, dated to 250–232 BCE. The script was deciphered in 1837 by James Prinsep, an archaeologist, philologist, and official of the East India Company.[1] The origin of the script is still much debated, with current Western academic opinion generally agreeing (with some exceptions) that Brahmi was derived from or at least influenced by one or more contemporary Semitic scripts, but a current of opinion in India favors the idea that it is connected to the much older and as-yet undeciphered Indus script

\subsection{Unicode [U+11000-U+1107F]}


\newfontfamily\brahmi{code2000.ttf}

\begin{scriptexample}[]{Brahmi}
\bgroup
\raggedleft
\brahmi

         
   

\arial
\hfill Text: Asokan Edict typeset with \texttt{NotoSansBrahmi-Regular.ttf} 
\egroup
\end{scriptexample}


\begin{description}
\item[Abkhazia] (Abkhaz: Аҧсны́ Apsny [apʰsˈnɨ]; Georgian: აფხაზეთი Apkhazeti; Russian: Абхазия Abkhaziya) is a disputed territory and partially recognised state controlled by a separatist government on the eastern coast of the Black Sea and the south-western flank of the Caucasus.

\item[Achinese] Acehnese language (Achinese) is a Malayo-Polynesian language spoken by Acehnese people natively in Aceh, Sumatra, Indonesia. This language is also spoken in some parts in Malaysia by Acehnese descendents there, such as in Yan, Kedah.

Formerly, Acehnese language was written in Arabic script called Jawoë or Jawi in Malay language. The script is less common nowadays.[citation needed] Now, Acehnese language is written in Latin script since colonization by the Dutch; with the addition of supplementary letters. The additional letters are é, è, ë, ö and ô.[8] The sound ɨ is represented by 'eu' and the sound ʌ is represented by 'ö' respectively. The letter 'ë' is used to represent the schwa sound which forms the second part in the diphthongs.

\item[Adyghe] Adyghe (/ˈædɨɡeɪ/ or /ˌɑːdɨˈɡeɪ/;[3] Adyghe: Адыгэбзэ adyghabze), also known as West Circassian (КӀахыбзэ), is one of the two official languages of the Republic of Adygea in the Russian Federation, the other being Russian. It is spoken by various tribes of the Adyghe people: Abzekh,[4] Adamey, Bzhedug;[5] Hatuqwai, Temirgoy, Mamkhegh; Natekuay, Shapsug;[6] Zhaney, Yegerikuay, each with its own dialect. The language is referred to by its speakers as Adygebze or Adəgăbză, and alternatively spelled in English as Adygean, Adygeyan or Adygei. The literary language is based on the Temirgoy dialect.
There are apparently around 128,000 speakers of the language on the native territory in Russia, almost all of them native speakers. In the whole world, some 300,000 speak the language. The largest Adyghe-speaking community is in Turkey, spoken by the post Russian–Circassian War (circa 1763–1864) diaspora; in addition to that, the Adyghe language is spoken by the Cherkesogai in Krasnodar Krai.

Ублапӏэм ыдэжь Гущыӏэр щыӏагъ. Ар Тхьэм ыдэжь щыӏагъ, а Гущыӏэри Тхьэу арыгъэ. Ублапӏэм щегъэжьагъэу а Гущыӏэр Тхьэм ыдэжь щыӏагъ. Тхьэм а Гущыӏэм зэкӏэри къыригъэгъэхъугъ. Тхьэм къыгъэхъугъэ пстэуми ащыщэу а Гущыӏэм къыримыгъгъэхъугъэ зи щыӏэп. Мыкӏодыжьын щыӏэныгъэ а Гущыӏэм хэлъыгъ, а щыӏэныгъэри цӏыфхэм нэфынэ афэхъугъ. Нэфынэр шӏункӏыгъэм щэнэфы, шӏункӏыгъэри нэфынэм текӏуагъэп.

Translation: In the beginning was the Word, and the Word was with God, and the Word was God. The same was in the beginning with God. All things were made by him, and without him was not any thing made that was made. In him was life, and the life was the light of men. And the light shineth in darkness, and the darkness comprehended it not.

\item[Albanian]Albanian (shqip [ʃcip] or gjuha shqipe [ˈɟuha ˈʃcipɛ], meaning Albanian language) is an Indo-European language spoken by approximately 7.6 million people,[3] primarily in Albania, Kosovo, the Republic of Macedonia and Greece, but also in other areas of Southeastern Europe in which there is an Albanian population, including Montenegro and Serbia (Presevo Valley). Centuries-old communities speaking Albanian-based dialects can be found scattered in Greece, southern Italy,[4] Sicily, and Ukraine.[5] As a result of a modern diaspora, there are also Albanian speakers elsewhere in those countries and in other parts of the world, including Scandinavia, Switzerland, Germany, Austria and Hungary, United Kingdom, Turkey, Australia, New Zealand, Netherlands, Singapore, Brazil, Canada, and the United States.

Letter:	A	B	C	Ç	D	Dh	E	Ë	F	G	Gj	H	I	J	K	L	Ll	M	N	Nj	O	P	Q	R	Rr	S	Sh	T	Th	U	V	X	Xh	Y	Z	Zh\\
IPA value:	a	b	t͡s	t͡ʃ	d	ð	e	ə	f	ɡ	ɟ	h	i	j	k	l	ɫ	m	n	ɲ	o	p	c	ɾ	r	s	ʃ	t	θ	u	v	d͡z	d͡ʒ	y	z	ʒ\\

\end{description}

\begin{multicols}{5}
\raggedright
Abkhazian\\
Abron\\
Achinese\\
Acoli\\
Adyghe\\
Afar\\
Afrikaans\\
Aghem\\
Akan\\
Akoose\\
Albanian\\
Albay\\
Bikol\\
Amo\\
Asturian\\
Asu\\
Atikamekw
Atsam
Avaric
Aymara
Azerbaijani (Cyrillic script)\\
Azerbaijani (Latin script)\\
Bafia\\
Bafut\\
Balinese\\
Balkan Gagauz Turkish
Bambara (Latin script)
Banjar
Baoulé
Basaa
Bashkir
Basque
Batak
Batak Toba
Belarusian
Bemba
Bena
Betawi
Bikol
Bini
Bislama
Bomu
Bosnian (Cyrillic script)
Bosnian (Latin script)
Breton
Bube
Buginese
Buhid
Bulgarian
Bulu
Buriat
Bushi
Catalan
Cebaara Senoufo
Cebuano
Central Atlas Tamazight (Latin script)
Central-Eastern Niger Fulfulde
Central Huasteca Nahuatl
Central Mazahua
Chamorro
Chechen
Chiga
Chipewyan
Church Slavic
Chuukese
Chuvash
Colognian
Congo Swahili
Cornish
Corsican
Croatian
Czech
Dan
Danish
Dargwa
Dogrib
Duala
Dutch
Dyula
Eastern Huasteca Nahuatl
East Futuna
Efik
Embu
English
Erzya
Esperanto
Estonian
Ewe
Ewondo
Fang
Faroese
Fijian
Filipino
Finnish
Fon
French
Friulian
Fulah
Ga
Gagauz
Galician
Ganda
German
Ghomala
Gilbertese
Gorontalo
Greek
Gronings
Guajajára
Guarani
Guianese Creole French
Gusii
Gwichʼin
Haitian
Hanunoo
Hausa (Latin script)
Hawaiian
Hiligaynon
Hiri Motu
Hungarian
Ibibio
Icelandic
Igbo
Iloko
Inari Sami
Indonesian
Ingush
Interlingua
Inuinnaqtun
Inuktitut (Latin script)
Inupiaq
Irish
Italian
Javanese
Jenaama Bozo
Jju
Jola-Fonyi
Kabardian
Kabuverdianu
Kabyle
Kaingang
Kako
Kalaallisut
Kalanga
Kalenjin
Kalo Finnish Romani
Kamba
Karachay-Balkar
Kara-Kalpak
Karelian
Kashubian

Kazakh (Cyrillic script)

Kerinci
Khasi
Kʼicheʼ
Kikuyu
Kimbundu
Kinyarwanda
Kita Maninkakan
Kom
Komering
Komi
Komi-Permyak
Kongo
Koro
Koro Wachi
Kosraean
Koyraboro Senni
Koyra Chiini
Kpelle
Krio
Kuanyama
Kumyk
Kurdish (Latin script)

Kwasio

Kyrgyz (Cyrillic script)

Kyrgyz (Latin script)

Lak\\
Lakota\\
Lampung Api\\
Langi\\
Lango\\
Latin\\
Latvian\\
Lezghian\\
Limburgish\\
Lingala\\
Lithuanian\\
Lombard
Lomwe
Lower Sorbian
Low German
Lozi
Luba-Katanga
Luba-Lulua
Lule Sami
Luo
Luxembourgish
Luyia
Maasina Fulfulde
Macedonian
Machame
Madurese
Mafa
Maguindanaon
Makasar
Makhu
Makhuwa-Meetto
Makonde
Malagasy
Malay (Latin script)
Maltese
Mandar
Mandingo (Latin script)
Manx
Manyika
Maori
Mapuche
Mari
Marshallese
Masaaba
Masai
Mbunga
Medumba
Mende
Meru
Meta’
Minangkabau
Mohawk
Moksha
Mongo
Mongolian (Cyrillic script)
Montagnais
Morisyen
Mossi
Mundang
Nama
Nauru
Navajo
Naxi
Ndau
Ndonga
Neapolitan
Negeri Sembilan Malay
Ngaju
Ngiemboon
Ngomba
Nigerian Fulfulde
Nigerian Pidgin
Niuean
Northern Sami
Northern Sotho
North Ndebele
North Slavey
Norwegian Bokmål
Norwegian Nynorsk
Nuer
Nyamwezi
Nyanja
Nyankole
Occitan
Oromo
Ossetic
Palauan
Pampanga
Pangasinan
Papiamento
Pohnpeian
Pökoot
Polish
Portuguese
Punu
Quechua
Rajasthani
Rejang
Réunion Creole French
Riang
Rinconada Bikol
Romanian
Romansh
Rombo
Ronga
Rundi
Russian
Rusyn
Rwa
Safaliba
Saho
Sakha
Samburu
Samoan
Sangir
Sango
Sangu
Santali
Sasak
Scots
Scottish Gaelic
Sena
Serbian (Cyrillic script)
Serbian (Latin script)
Serer
Seselwa Creole French
Shambala
Shona
Sicilian
Sidamo
Sinte Romani
Skolt Sami
Slave
Slovak
Slovenian
Soga
Somali
Soninke
Southern Altai
Southern Sami
Southern Sotho
South Ndebele
Spanish
Sranan Tongo
Sukuma
Sundanese
Susu
Swahili
Swati
Swedish
Swiss German
Tachelhit (Latin script)
Tae’
Tagbanwa
Tahitian
Taita
Tajik (Cyrillic script)
Tamashek
Taroko
Tasawaq
Tatar
Tausug
Tavringer Romani
Teso
Tetum
Timne
Tiv
Tokelau
Tok Pisin
Tolaki
Tomo Kan Dogon
Tongan
Tooro
Tornedalen Finnish
Tsonga
Tswana
Tumbuka
Turkish
Turkmen (Latin script)
Tuvalu
Tuvinian
Tyap
Uab Meto
Udmurt
Ukrainian
Ulithian
Umbundu
Unknown Language
Uyghur (Cyrillic script)
Uzbek (Cyrillic script)
Uzbek (Latin script)
Vai (Latin script)
Venda
Vietnamese
Virgin Islands Creole English
Vunjo
Wallisian
Walloon
Walser
Waray
Welsh
Western Frisian
Western Huasteca Nahuatl
Western Mari
Wolof
Xaasongaxango
Xavánte
Xhosa
Yangben
Yao
Yapese
Yemba
Yoruba
Yucatec Maya
Zarma
Zaza
Zeelandic
Zhuang
Zulu
\end{multicols}





\end{document}



%  \chapter{Middle Eastern Scripts}

The scripts in this section have a common origin in the ancient Phoenician alphabet. They include:

\begin{center}
\begin{tabular}{ll}
Hebrew & Samaritan\\
Arabic & Thaana\\
Syriac &\\
\end{tabular}
\end{center}

The Hebrew script is used in Israel and for languages of the Diaspora. The Arabic script is
used to write many languages throughout the Middle East, North Africa, and certain parts
of Asia. The Syriac script is used to write a number of Middle Eastern languages. These
three also function as major liturgical scripts, used worldwide by various religious groups.

The Samaritan script is used in small communities in Israel and the Palestinian Territories
to write the Samaritan Hebrew and Samaritan Aramaic languages. The Thaana script is
used to write Dhivehi, the language of the Republic of Maldives, an island nation in the
middle of the Indian Ocean. 

Text in these scripts is written from right to left. Arabic and Syriac are cursive scripts even when typeset, unlike Hebrew, Samaritan  and Thaana, where letters are unconnected. Most letters in Arabic and Syriac assume different forms depending on their position in a word. Shaping rules are not required for Hebrew because only five letters have position-dependent forms, and these forms are separately encoded.

Historically, Middle Eastern  scripts did not write short vowels. In modern scripts they are represented  by marks positioned above or below a consonantal letter. Vowels and other
marks of pronunciation (“vocalization”) are encoded as combining characters, so support
for vocalized text necessitates use of composed character sequences. Yiddish, Syriac, and
Thaana are normally written with vocalization; Hebrew, Samaritan, and Arabic are usually written unvocalized. 

\section{Hebrew}
\newfontfamily\hebrew{Miriam}
\fontspec{Arial Unicode MS}
To properly typeset Hebrew texts you first need to choose an appropriate font and also set the directionality of the text. This
is done using the etex commands:

\CMDI{\beginL} and \CMDI{\beginR} 

For \XeTeX\ you also need to add near the top of your document |\TeXXeTstate=1|. The package \pkgname{bidi} can be used to set all parameters. Be warned that it redefines almost all of \latexe's commands, so for short mixed texts, I wouldn't recommend its usage. 



The Hebrew alphabet (Hebrew: אָלֶף־בֵּית עִבְרִי[a], alefbet ʿIvri ), known variously by scholars as the Jewish script, square script, block script, is used in the writing of the Hebrew language, as well as other Jewish languages, most notably Yiddish, Ladino, and Judeo-Arabic. There have been two script forms in use; the original old Hebrew script is known as the paleo-Hebrew script (which has been largely preserved, in an altered form, in the Samaritan script), while the present "square" form of the Hebrew alphabet is a stylized form of the Assyrian script. Various "styles" (in current terms, "fonts") of representation of the letters exist. There is also a cursive Hebrew script, which has also varied over time and place. On Windows you can use the \texttt{Miriam} font or \texttt{Arial Unicode MS} or \texttt{Miriam Fixed}.
\medskip

\topline

\bgroup\TeXXeTstate=1
\raggedleft\hebrew{}\beginR

הכתב הכנעני הקדום הלך והתפשט וסימניו היו מוכרים כל כך, עד כי המשתמשים בו התחילו "להתעצל" בהשלמת הציורים, והניחו כי הקורא יבין גם מתוך שרטוטים סכמתיים באיזו אות מדובר. כך, למשל, הפך הראש למשולש עם צוואר; כף היד מלאת האצבעות הפכה לשרטוט דל, ומהדג נותר רק הזנב. כשהעברים אמצו את הכתב הכנעני הם התקשו לזהות חלק מהציורים המקוריים והניחו למשל כי הסימן המתאר את המילה "זהה" הוא כלי נשק; שזנב הדג המשולש הוא דלת, ושדווקא הנחש הוא דג. כך נולדו שמותיהם העבריים של האותיות זי"ן, דל"ת ונו"ן (נון הוא דג, כמו אמנון, שפמנון וכו'). הציורים שהפכו לסימנים התגלגלו לכתבים נוספים, ואפילו ליוונית וללטינית. גם בכתב העברי המודרני ניתן לזהות המשך התפתחותי ברור מן הכתב הכנעני הקדום, והשתמרות שמות האותיות מקלה מאוד על פענוח המקור.


בתקופת בית שני, אומץ האלפבית הארמי לשימוש השפה העברית במקום האלפבית העברי העתיק, כאשר בזה האחרון נעשה שימוש מועט כגון כתיבת השמות הקדושים והטבעת מטבעות. עם הזמן, נעלם גם שימוש זה של הכתב העתיק. האלפבית העברי של ימינו הוא אפוא פיתוח של האלפבית הארמי ולא של הכתב העברי העתיק.	
{}

 לֹ֥א תִשָּׂ֛א

\endR


\egroup
\bottomline
\medskip

To make all paragraphs  RL use the \cmd{\everypar}\footnote{See discussions at \url{http://tex.stackexchange.com/questions/141867/minimal-bidi-for-typesetting-rl-text} and \url{http://www.tug.org/pipermail/xetex/2004-August/000697.html}}. 

\begin{verbatim}
\newbox\mybox \everypar{\setbox\mybox\lastbox\beginR\box\mybox}
\everypar={% at the start of each paragraph, do....
    \setbox0=\lastbox % save the paragraph indent, if any
    \beginR % set R-L direction
    \box0 % then re-insert the indent
	}
\end{verbatim}

The Hebrew alphabet has 22 letters, of which five have different forms when used at the end of a word. Hebrew is written from right to left. Originally, the alphabet was an abjad consisting only of consonants. Like other \textit{abjads}, such as the Arabic alphabet, means were later devised to indicate vowels by separate vowel points, known in Hebrew as niqqud. In rabbinic Hebrew, the letters א ה ו י are also used as matres lectionis to represent vowels. When used to write Yiddish, the writing system is a true alphabet (except for borrowed Hebrew words). In modern usage of the alphabet, as in the case of Yiddish (except that ע replaces ה) and to some extent modern Israeli Hebrew, vowels may be indicated. Today, the trend is toward full spelling with these letters acting as true vowels.


\subsection{Syriac}

\newfontfamily\syriac{Estrangelo Edessa}

Syriac /ˈsɪriæk/ ({\syriac{ܠܫܢܐ ܣܘܪܝܝܐ}} Leššānā Suryāyā) is a dialect of Middle Aramaic that was once spoken across much of the Fertile Crescent and Eastern Arabia.[1][2][5] Having first appeared as a script in the 1st century AD after being spoken as an unwritten language for five centuries,[6] Classical Syriac became a major literary language throughout the Middle East from the 4th to the 8th centuries,[7] the classical language of Edessa, preserved in a large body of Syriac literature.
It became the vehicle of Syriac Christianity and culture, spreading throughout Asia as far as the Indian Malabar Coast and Eastern China,[8] and was the medium of communication and cultural dissemination for Arabs and, to a lesser extent, Persians. Primarily a Christian medium of expression, Syriac had a fundamental cultural and literary influence on the development of Arabic,[9] which largely replaced it towards the 14th century.[3] Syriac remains the liturgical language of Syriac Christianity.
Syriac is a Middle Aramaic language, and, as such, it is a language of the Northwestern branch of the Semitic family. It is written in the Syriac alphabet, a derivation of the Aramaic alphabet.

\begin{scriptexample}[]{Syriac}
\unicodetable{syriac}{"0700,"0710,"0720,"0730,"0740}
\end{scriptexample}

The Syriac Abbreviation (a type of overline) can be represented with a special control character called the Syriac Abbreviation Mark (U+070F {\syriac \char"070F ܘ}).

\section{Samaritan}
\newfontfamily\samaritan{NotoSansSamaritan-Regular.ttf}

The Samaritan alphabet is used by the Samaritans for religious writings, including the Samaritan Pentateuch, writings in Samaritan Hebrew, and for commentaries and translations in Samaritan Aramaic and occasionally Arabic.

The Samaritans are, consider themselves to be the descendants of the Northern Tribes of Israel that were not sent into Assyrian captivity, and have continuously resided in the land of Israel.

The Torah Scroll of the Samaritans uses an alphabet that is very different from the one used on Jewish Torah Scrolls. According to the Samaritans themselves and Hebrew scholars, this alphabet is the original "Old Hebrew" alphabet.

Even as far back as 1691, this connection between the Samaritan and the "Old" Hebrew alphabets was made by Henry Dodwell; "[the Samaritans] still preserve [the Pentateuch] in the Old Hebrew characters."

Samaritan is a direct descendant of the Paleo-Hebrew alphabet, which was a variety of the Phoenician alphabet in which large parts of the Hebrew Bible were originally penned. All these scripts are believed to be descendants of the Proto-Sinaitic script. That script was used by the ancient Israelites, both Jews and Samaritans. The better-known "square script" Hebrew alphabet traditionally used by Jews is a stylized version of the Aramaic alphabet which they adopted from the Persian Empire (which in turn adopted it from the Arameans). 

After the fall of the Persian Empire, Judaism used both scripts before settling on the Aramaic form. For a limited time thereafter, the use of paleo-Hebrew (proto-Samaritan) among Jews was retained only to write the Tetragrammaton, but soon that custom was also abandoned.



ShofarRegular StamAshkenazCLM.ttf

\begin{scriptexample}[]{Samaritan}
\bgroup
\TeXXeTstate=1
\unicodetable{samaritan}{"0800,"0810,"0820,"0830}
\egroup
\TeXXeTstate=0
\end{scriptexample}

I battled to get an appropriate font for the Samaritan script and had to use the \idxfont{Noto Sans Samaritan} from Google


^^A\printunicodeblock{./languages/samaritan.txt}{\samaritan}


\url{http://www.ancient-hebrew.org/ahh/ahh.htm#_Toc314842274}




\section{Arabic}

\newfontfamily\arabian{Scheherazade-R.ttf}

The Arabic script is a writing system used for writing several languages of Asia and Africa, such as Arabic, Sorani and Luri Dialects of Kurdish language, Persian, Pashto and Urdu.[1] Even until the 16th century, it was used to write some texts in Spanish.[2] After the Latin script, Chinese characters, and Devanagari, it is the fourth-most widely used writing system in the world.[3]
The Arabic script is written from right to left in a cursive style. In most cases the letters transcribe consonants, or consonants and a few vowels, so most Arabic alphabets are abjads.

The script was first used to write texts in Arabic, most notably the Qurʼān, the holy book of Islam. With the spread of Islam, it came to be used to write languages of many language families, leading to the addition of new letters and other symbols, with some versions, such as Kurdish, Uyghur, and old Bosnian being abugidas or true alphabets. It is also the basis for a rich tradition of Arabic calligraphy.

\begin{verbatim}
\begin{Arabic}
ّ هو إذ الغاية؛ شريف الفوائد، جم المذهب، عزيز فنّ التاريخ فنّ أنّ اعلم
والملوك سيرهم، في والأنبياء أخلاقهم، في الأمم من الماضين أحوال على يوقفنا
ّ أحوال في يرومه لمن ذلك في الإقتداء فائدة تتم حتّى وسياستهم؛ دولهم في
والدنيا. الدين
\end{Arabic}
\end{verbatim}




As of Unicode 7.0, the Arabic script is contained in the following blocks:
Arabic (0600—06FF, 255 characters)
Arabic Supplement (0750—077F, 48 characters)
Arabic Extended-A (08A0—08FF, 39 characters)
Arabic Presentation Forms-A (FB50—FDFF, 608 characters)
Arabic Presentation Forms-B (FE70—FEFF, 140 characters)
Rumi Numeral Symbols (10E60—10E7F, 31 characters)
Arabic Mathematical Alphabetic Symbols (1EE00—1EEFF, 143 characters)[1][2]

The basic Arabic range encodes the standard letters and diacritics, but does not encode contextual forms (U+0621–U+0652 being directly based on ISO 8859-6); and also includes the most common diacritics and Arabic-Indic digits. The Arabic Supplement range encodes letter variants mostly used for writing African (non-Arabic) languages. The Arabic Extended-A range encodes additional Qur'anic annotations and letter variants used for various non-Arabic languages. The Arabic Presentation Forms-A range encodes contextual forms and ligatures of letter variants needed for Persian, Urdu, Sindhi and Central Asian languages. The Arabic Presentation Forms-B range encodes spacing forms of Arabic diacritics, and more contextual letter forms. The presentation forms are present only for compatibility with older standards, and are not currently needed for coding text.[3] 

The Arabic Mathematical Alphabetical Symbols block encodes characters used in Arabic mathematical expressions.

\begin{multicols}{3}
\printunicodeblock{./languages/arabic.txt}{\arabian}
\end{multicols}









\section{Thaana}

\newfontfamily\thaana{MV Boli}
Thaana, Taana or Tāna ({\thaana  ތާނަ}‎ in Tāna script) is the modern writing system of the Maldivian language spoken in the Maldives. Thaana has characteristics of both an abugida (diacritic, vowel-killer strokes) and a true alphabet (all vowels are written), with consonants derived from indigenous and Arabic numerals, and vowels derived from the vowel diacritics of the Arabic abjad. Its orthography is largely phonemic.

The Thaana script first appeared in a Maldivian document towards the beginning of the 18th century in a crude initial form known as Gabulhi Thaana which was written scripta continua. This early script slowly developed, its characters slanting 45 degrees, becoming more graceful and spaces were added between words. 

As time went by it gradually replaced the older Dhives Akuru alphabet. The oldest written sample of the Thaana script is found in the island of Kanditheemu in Northern Miladhunmadulu Atoll. It is inscribed on the door posts of the main Hukuru Miskiy (Friday mosque) of the island and dates back to 1008 AH (AD 1599) and 1020 AH (AD 1611) when the roof of the building were built and the renewed during the reigns of Ibrahim Kalaafaan (Sultan Ibrahim III) and Hussain Faamuladeyri Kilege (Sultan Hussain II) respectively.

\begin{scriptexample}[]{Thaana}
\unicodetable{thaana}{"0780,"0790,"07A0,"07B0}

\hfill Typeset with MV Boli and the command \cmd{\thaana}.
\end{scriptexample}


^^A\printunicodeblock{./languages/thaana.txt}{\thaana}



\endinput











%  \chapter{Additional Modern Scripts}

\begin{center}
\begin{tabular}{lp{5cm}l}
Ethiopic. &Vai. &Deseret.\\
Mongolian. &Bamum. &Shavian.\\
Osmanya.   &Cherokee. &Lisu.\\
Tifinagh.  &Canadian Aboriginal Syllabics. &Miao.\\
N’Ko.&&\\
\end{tabular}
\end{center}

Ethiopic, Mongolian, and Tifinagh are scripts with long histories. Although their roots can
be traced back to the original Semitic and North African writing systems, they would not
be classified as Middle Eastern scripts today

The Cherokee script is a syllabary developed between 1815 and 1821, to write the Cherokee
language, still spoken by small communities in Oklahoma and North Carolina. Canadian
Aboriginal Syllabics were invented in the 1830s for Algonquian languages in Canada. The
system has been extended many times, and is now actively used by other communities, including speakers of Inuktitut and Athapascan languages.

Deseret is a phonemic alphabet devised in the 1850s to write English. It saw limited use for
a few decades by members of The Church of Jesus Christ of Latter-day Saints. Shavian is
another phonemic alphabet, invented in the 1950s to write English. It was used to publish
one book in 1962, but remains of some current interest




\subsection{Ethiopic}
Ge'ez (ግዕዝ Gəʿəz), (also known as Ethiopic) is a script used as an abugida (syllable alphabet) for several languages of Ethiopia and Eritrea. It originated as an abjad (consonant-only alphabet) and was first used to write Ge'ez, now the liturgical language of the Ethiopian Orthodox Tewahedo Church and the Eritrean Orthodox Tewahedo Church. In Amharic and Tigrinya, the script is often called fidäl (ፊደል), meaning "script" or "alphabet".

The Ge'ez script has been adapted to write other, mostly Semitic, languages, particularly Amharic in Ethiopia, and Tigrinya in both Eritrea and Ethiopia. It is also used for Sebatbeit, Me'en, and most other languages of Ethiopia. In Eritrea it is used for Tigre, and it has traditionally been used for Blin, a Cushitic language. Tigre, spoken in western and northern Eritrea, is considered to resemble Ge'ez more than do the other derivative languages.[citation needed] Some other languages in the Horn of Africa, such as Oromo, used to be written using Ge'ez, but have migrated to Latin-based orthographies.
For the representation of sounds, this article uses a system that is common (though not universal) among linguists who work on Ethiopian Semitic languages. This differs somewhat from the conventions of the International Phonetic Alphabet. See the articles on the individual languages for information on the pronunciation.

There are a number of fonts available and we have selected the Google \idxfont{NotoSansEthiopic}
\newfontfamily\ethiopic{NotoSansEthiopic-Bold.ttf}

\begin{scriptexample}[]{Ethiopic}
\unicodetable{ethiopic}{"1200,"1210,"1220,"1230,"1240,"1250,"1260,"1270,"1280,"1290,^^A
"12A0,"12B0,"12C0,"12E0,"12F0,"1300,"1310,"1330,"1340,"1350,"1360,"1370}
\end{scriptexample}
\section{Vai}
\label{s:vai}

The Vai syllabary is a syllabic writing system devised for the Vai language by Momolu Duwalu Bukele of Jondu, in what is now Grand Cape Mount County, Liberia.[1] [2] Bukele is regarded within the Vai community, as well as by most scholars, as the syllabary's inventor and chief promoter when it was first documented in the 1830s. It is one of the two most successful indigenous scripts in West Africa.

\newfontfamily\vai{code2000.ttf}
\begin{scriptexample}[]{Vai}
\unicodetable{vai}{"A500,"A510,"A520,"A530,"A540,"A550,"A560,"A570,^^A
"A580,"A590,"A5A0,"A5B0,^^A
"A5C0,"A5D0,"A5E0,"A5F0,"A610,"A620,"A630}
\end{scriptexample}

In the 1920s ten decimal digits were devised for Vai; these were “Vai-style” glyph variants of
European digits (see Figure 11). They were not popular with Vai people  even for historical purposes. All
the modern literature uses European digits.


\begin{scriptexample}[]{Vai}
\bgroup
\vai
\obeylines\Large
0	1	2	3	4	5	6	7	8	9
꘠	꘡	꘢	꘣	꘤	꘥	꘦	꘧	꘨	꘩
\vai
\egroup
\end{scriptexample}



\printunicodeblock{./languages/vai.txt}{\vai}
\section{Deseret script}
\newfontfamily\deseret{code2001.ttf}

The Deseret alphabet (dɛz.əˈrɛt.) (Deseret: {\deseret 𐐔𐐯𐑅𐐨𐑉𐐯𐐻 or 𐐔𐐯𐑆𐐲𐑉𐐯𐐻}) is a phonemic English spelling reform developed in the mid-19th century by the board of regents of the University of Deseret (later the University of Utah) under the direction of Brigham Young, second president of The Church of Jesus Christ of Latter-day Saints.

In public statements, Young claimed the alphabet was intended to replace the traditional Latin alphabet with an alternative, more phonetically accurate alphabet for the English language. This would offer immigrants an opportunity to learn to read and write English, he said, the orthography of which is often less phonetically consistent than those of many other languages. Similar experiments were not uncommon during the period, the most well-known of which is the Shavian alphabet.

Young also prescribed the learning of Deseret to the school system, stating "It will be the means of introducing uniformity in our orthography, and the years that are now required to learn to read and spell can be devoted to other studies".[2]


Deseret script {\deseret 𐐔𐐯𐑅𐐨𐑉𐐯𐐻}  [U+10400-U+1044F]
\medskip

\bgroup
\par
\noindent
\colorbox{graphicbackground}{\color{black}^^A
\begin{minipage}{\textwidth}^^A
\parindent1pt
\vskip10pt
\leftskip10pt \rightskip\leftskip
\deseret
\large

𐐂 𐑌𐐲𐑉𐑅𐐨𐑉𐐮 𐐮𐑆 𐐪 𐐹𐐨𐑅 𐐱𐑂 𐑊𐐰𐑌𐐼 𐐱𐑌 𐐸𐐶𐐮𐐽 𐑁𐑉𐐭𐐻𐐻𐑉𐐨𐑆 𐐪𐑉 𐑅𐐻𐐪𐑉𐐻𐐯𐐼,


\par
\vspace*{10pt}
\end{minipage}
}

Text: Deseret alphabet http://www.omniglot.com/writing/deseret.htm
\medskip
\egroup

\PrintUnicodeBlock{./languages/deseret.txt}{\deseret}

\chapter{Bamum}
\label{s:bamum}
\epigraph{"No known alphabet was ever invented by a European."}{Jeffreys' translation from the Royal script.}

\label{s:bamum}
\index{scripts>Bamum}
\newfontfamily\bamum{NotoSansBamum-Regular.ttf}

The Bamum scripts are an evolutionary series of six scripts created for the Bamum language by King Njoya of Cameroon at the turn of the 20th century. They are notable for evolving from a pictographic system to a partially alphabetic syllabic script in the space of 14 years, from 1896 to 1910. Bamum type was cast in 1918, but the script fell into disuse around 1931.

\begin{figure}[htbp]
\parindent=0pt

\centering

\includegraphics[width=\textwidth]{bamum}

\caption{King Njoya of Bamum receiving an oil painting of Kaiser Wilhelm II. The gift was in return for his support in the German campaign against the Nso'.}
\end{figure}

The Bamum, sometimes called Bamoum, Bamun, Bamoun, or Mum, are a Bantoid ethnic group of Cameroon with around 215,000 members.



\begin{scriptexample}[]{Bamum}
\unicodetable{bamum}{"A6A0,"A6B0,"A6C0,"A6D0,"A6E0,"A6F0}
\end{scriptexample}
\section{Shavian}
\label{s:shavian}
\def\shaviansetup#1{}
\newfontfamily\shavian{code2001.ttf}
^^A\newfontfamily\shavian{NotoSansShavian-Regular.ttf}
\cxset{shavian font/.code=\shaviansetup{#1}}
\cxset{shavian font=shavian}




\begin{scriptexample}[]{shavian}
\shavian

𐑳 𐑡𐑻𐑯𐑰 𐑑 𐑞 𐑕𐑧𐑯𐑑𐑻 𐑝 𐑞 𐑻𐑔
𐑚𐑲 - ·𐑡𐑵𐑤𐑟 ·𐑝𐑻𐑯

𐑗𐑩𐑐𐑑𐑻 1 - 𐑥𐑲 𐑳𐑙𐑒𐑳𐑤 𐑥𐑱𐑒𐑕 𐑳 𐑜𐑮𐑱𐑑 𐑛𐑦𐑕𐑒𐑳𐑝𐑻𐑰

     𐑤𐑫𐑒𐑦𐑙 𐑚𐑩𐑒 𐑑 𐑷𐑤 𐑞𐑩𐑑 𐑣𐑩𐑟 𐑳𐑒𐑻𐑛 𐑑 𐑥𐑰 𐑕𐑦𐑯𐑕 𐑞𐑩𐑑 𐑦𐑝𐑧𐑯𐑑𐑓𐑳𐑤 𐑛𐑱, 𐑲 𐑩𐑥 𐑕𐑒𐑧𐑮𐑕𐑤𐑰 𐑱𐑚𐑳𐑤 𐑑 𐑚𐑦𐑤𐑰𐑝 𐑦𐑯 𐑞 𐑮𐑰𐑩𐑤𐑳𐑑𐑰 𐑝 𐑥𐑲 𐑩𐑛𐑝𐑧𐑯𐑗𐑻𐑟. 𐑞𐑱 𐑢𐑻 𐑑𐑮𐑵𐑤𐑰 𐑕𐑴 𐑢𐑳𐑯𐑛𐑻𐑓𐑳𐑤 𐑞𐑩𐑑 𐑰𐑝𐑦𐑯 𐑯𐑬 𐑲 𐑩𐑥 𐑚𐑦𐑢𐑦𐑤𐑛𐑻𐑛 𐑢𐑧𐑯 𐑲 𐑔𐑦𐑙𐑒 𐑝 𐑞𐑧𐑥.
     𐑥𐑲 𐑳𐑙𐑒𐑳𐑤 𐑢𐑪𐑟 𐑳 𐑡𐑻𐑥𐑳𐑯, 𐑣𐑩𐑝𐑦𐑙 𐑥𐑧𐑮𐑰𐑛 𐑥𐑲 𐑥𐑳𐑞𐑻𐑟 𐑕𐑦𐑕𐑑𐑻, 𐑩𐑯 𐑦𐑙𐑜𐑤𐑦𐑖𐑢𐑫𐑥𐑳𐑯. 𐑚𐑰𐑦𐑙 𐑝𐑧𐑮𐑰 𐑥𐑳𐑗 𐑳𐑑𐑩𐑗𐑑 𐑑 𐑣𐑦𐑟 𐑓𐑪𐑞𐑻𐑤𐑳𐑕 𐑯𐑧𐑓𐑘𐑵, 𐑣𐑰 𐑦𐑯𐑝𐑲𐑑𐑳𐑛 𐑥𐑰 𐑑 𐑕𐑑𐑳𐑛𐑰 𐑳𐑯𐑛𐑻 𐑣𐑦𐑥 𐑦𐑯 𐑣𐑦𐑟 𐑣𐑴𐑥 𐑦𐑯 𐑞 𐑓𐑪𐑞𐑻𐑤𐑩𐑯𐑛. 𐑞𐑦𐑕 𐑣𐑴𐑥 𐑢𐑪𐑟 𐑦𐑯 𐑳 𐑤𐑪𐑮𐑡 𐑑𐑬𐑯, 𐑯 𐑥𐑲 𐑳𐑙𐑒𐑳𐑤 𐑳 𐑐𐑮𐑳𐑓𐑧𐑕𐑻 𐑝 𐑓𐑳𐑤𐑪𐑕𐑳𐑓𐑰, 𐑒𐑧𐑥𐑳𐑕𐑑𐑮𐑰, 𐑡𐑰𐑪𐑤𐑳𐑡𐑰, 𐑥𐑦𐑯𐑻𐑪𐑤𐑳𐑡𐑰, 𐑯 𐑥𐑧𐑯𐑰 𐑳𐑞𐑻 𐑳𐑤𐑴𐑡𐑰𐑕.

\arial

\hfill Excerpt from Jules Vern,  \textit{Journey to the Center of the Earth from \href{http://shavian.weebly.com/}{shavian}}
\end{scriptexample}

The example is typeset using \texttt{code2001.ttf}. There are numerous fonts that provide Shavian glyphs. \texttt{ESL Gothic Unicode} font by Ethan Lamoreaux\footnote{\url{http://www.fontspace.com/ethan-lamoreaux/esl-gothic-unicode}}. The Noto fonts also have a Shavian font. 

You can activate typesetting in Shavian using the key:

\begin{key}{/chapter/shavian font = \meta{font name}} The key will setup the
default font for the Shavian script and define the commands \cmd{\shavian} and \cmd{\textshavian}. 
\end{key}

\PrintUnicodeBlock{./languages/shavian.txt}{\shavian}





\subsection{Osmanya}

\newfontfamily\osmanya{NotoSansOsmanya-Regular.ttf}

\begin{scriptexample}[]{Osmanya}
\unicodetable{osmanya}{"10480,"10490,"104A0}
\end{scriptexample}

The Osmanya alphabet (Somali: Cismaanya; Osmanya: {\osmanya 𐒋𐒘𐒈𐒑𐒛𐒒𐒕𐒀}), also known as Far Soomaali ("Somali writing"), is a writing script created to transcribe the Somali language. It was invented between 1920 and 1922 by Osman Yusuf Kenadid of the Majeerteen Darod clan, the nephew of Sultan Yusuf Ali Kenadid of the Sultanate of Hobyo.

While Osmanya gained reasonably wide acceptance in Somalia and quickly produced a considerable body of literature, it proved difficult to spread among the population mainly due to stiff competition from the long-established Arabic script as well as the emerging Somali alphabet developed by the Somali linguist, Shire Jama Ahmed, which was based on the Latin script.

As nationalist sentiments grew and since the Somali language had long lost its ancient script,[1] the adoption of a universally recognized writing script for the Somali language became an important point of discussion. After independence, little progress was made on the issue, as opinion was divided over whether the Arabic or Latin scripts should be used instead.

In October 1972, due to its simplicity, the fact that it lent itself well to writing Somali since it could cope with all of the sounds in the language, and the already widespread existence of machines and typewriters designed for its use,[2][3] the government of Somali president Mohamed Siad Barre unilaterally elected to use only the Latin script for writing Somali instead of the Arabic or Osmanya scripts.[4] Barre's administration subsequently launched a massive literacy campaign designed to ensure its sole adoption. This led to a sharp decline in use of Osmanya.
\section{Cherokee}
\index{scripts>Cherokee}
\index{scripts>Cherokee>fonts}
\label{sec:cherokee}
Windows comes with |Plantagenet Cherokee| font. The |code2000| also has good support for the alphabet. The \texttt{SIL font Charis SIL} also has good support and can be downloaded at \href{http://scripts.sil.org/cms/scripts/page.php?item_id=CharisSIL_download}{scripts.sel.org}, the latest version gave me problems when used with Windows. 

  
\def\textcherokee#1{{\cherokee   #1}}


\begin{docKey}[phd]{cherokee font}{ = \meta{font name}} {default none, initial=code2000}
 Loads the font
command \cmd{\cherokee}. When the command is used it typesets text in
cherokee unicode. There is no need to load the language, unless it is the main document language. For windows the default font is  |Plantagenet Cherokee|. Another font is FreeSerif, which we are using here.
\end{docKey}

\begin{scriptexample}[]{Cherokee}
{\cherokee
\begin{tabular}{lp{8.5cm}}
Translation	  &John (ᏣᏂ) 3:16\\
American Bible Society 1860	&ᎾᏍᎩᏰᏃ ᏂᎦᎥᎩ ᎤᏁᎳᏅᎯ ᎤᎨᏳᏒᎩ ᎡᎶᎯ, ᏕᏅᏲᏒᎩ ᎤᏤᎵᎦ ᎤᏪᏥ ᎤᏩᏒᎯᏳ ᎤᏕᏁᎸᎯ, ᎩᎶ ᎾᏍᎩ ᏱᎪᎯᏳᎲᏍᎦ ᎤᏲᎱᎯᏍᏗᏱ ᏂᎨᏒᎾ, ᎬᏂᏛᏉᏍᎩᏂ ᎤᏩᏛᏗ.\\

(Transliteration)	& nasgiyeno nigavgi unelanvhi ugeyusvgi elohi, denvyosvgi utseliga uwetsi uwasvhiyu udenelvhi, gilo nasgi yigohiyuhvsga uyohuhisdiyi nigesvna, gvnidvquosgini uwadvdi.\\
\end{tabular}}
\end{scriptexample}

\begin{texexample}{Using text...}{cherokee}
\bgroup
\cherokee \large\textbf{ᎾᏍᎩᏰᏃ}
\textcherokee{ᎾᏍᎩᏰᏃ}
\egroup
\end{texexample}

If you have trouble getting them to work\footnote{\url{http://tex.stackexchange.com/questions/132087/displaying-cherokee-text}}

\url{http://www.cherokee.org/AboutTheNation/Language/CherokeeFont.aspx}




\section{Tifnagh}

\newfontfamily\tifinagh{code2000.ttf}

Tifinagh (Berber pronunciation: [tifinaɣ]; also written Tifinaɣ in the Berber Latin alphabet, {\tifinagh  ⵜⵉⴼⵉⵏⴰⵖ} in Neo-Tifinagh, and تيفيناغ in the Berber Arabic alphabet) is a series of abjad and alphabetic scripts used by Berber peoples to write Berber languages.[1]
A modern derivate of the traditional script, known as Neo-Tifinagh, was introduced in the 20th century. A slightly modified version of the traditional script, called Tifinagh Ircam, is used in a number of Moroccan elementary schools in teaching the Berber language to children as well as a number of publications.[2][3]

The word tifinagh is thought to be a Berberized feminine plural cognate of Punic, through the Berber feminine prefix ti- and Latin Punicus; thus tifinagh could possibly mean "the Phoenician (letters)"[4][5] or "the Punic letters".

\bgroup

\noindent\tifinagh
\colorbox{thecodebackground}{\color{black}^^A
\begin{minipage}{\textwidth}
\parindent1pt
\vskip10pt
\leftskip10pt \rightskip\leftskip
Tifnagh     ⵜⵉⴼⵉⵏⴰⵖ [U+2D30-U+2D7F]

ⴰⴳⵍⴷⵓⵏ ⴰⵎⵥⵥⴰ

ⵙ ⵡⴰⵡⴰⵍ ⴳⵔⵉ ⵉⴷⵙ, ⵙⵙⵏⵖ ⵢⴰⵜ ⵜⵖⴰⵡⵙⴰ ⵜⵉⵙⵙ ⵙⵏⴰⵜ  ⵉⵅⴰⵜⵔⵏ: ⵉⵜⵔⵉ ⵙⴳ ⴷⴷ ⵉⴷⴷⴰ ⵓⵔ ⵉⵎⵇⵇⵓⵔ, ⵉⵍⵍⴰ ⵖⴰⵙ ⴰⵏⵛⵜ ⵏ ⵢⴰⵜ ⵜⴰⴷⴷⴰⵔⵜ !

ⴰⵢⴰ ⵓⴽⵣⵖ ⵜ. ⵙⵙⵏⵖ ⵉⵙ ⴱⵕⵕⴰ ⵏ ⵉⵜⵔⴰⵏ ⵣⵓⵏⴷ ⴰⴽⴰⵍ, ⵊⵓⴱⵉⵜⵔ, ⵎⴰⵔⵙ, ⴱⵉⵏⵓⵙ – ⵉⵜⵔⴰⵏ ⵎⵉ ⵏⴽⴼⴰ ⵉⵙⵎⴰⵡⵏ – ⵍⵍⴰⵏ ⴷⵉⵖ ⵉⵜⵔⴰⵏ ⵢⴰⴹⵏ ⵎⵥⵥⵉⵢⵏⵉⵏ, ⵡⵉⵏⵏⴰ ⵓⵔ ⵏⵣⵎⵉⵔ ⴰⴷ ⵏⵥⵔ ⵙ ⵓⵜⵉⵍⵉⵙⴽⵓⴱ. ⴰⴷⴷⴰⵢ ⵢⵓⴼⴰ ⵓⴰⵙⵜⵕⵓⵏⵓⵎ ⵢⴰⵏ ⴷⵉⴳⵙⵏ, ⴷⴰ ⵢⴰⵙ ⵉⵜⵜⴳⴰ ⵙ ⵢⵉⵙⵎ ⵢⴰⵏ ⵡⵓⵜⵜⵓⵏ. ⴷⴰ ⵢⴰⵙ ⵉⵇⵇⴰⵔ ⵙ ⵓⵎⴷⵢⴰⵜ : « ⴰⵙⵜⵔⵓⵉⴷ 3251 ».

ⵓⴽⵣⵖ ⵉⵙ ⴷⴷ ⵉⴷⴷⴰ ⵓⴳⵍⴷⵓⵏ ⵎⵥⵥⵉⵢⵏ ⵙⴳ ⵉⵜⵔⵉ ⵎⵉ ⵇⵇⴰⵔⵏ ⴰⵙⵜⵔⵓⵉⴷ ⴱ612. ⴰⵙⵜⵔⵓⵉⴷ ⴰ, ⵓⵔ ⵉⵜⵓⵥⵔⴰ ⴰⵔ 1909 ⵙ ⵓⵜⵉⵍⵉⵙⴽⵓⴱ. ⵉⵥⵔⴰ ⵜ ⵢⴰⵏ ⵓⴰⵙⵜⵕⵓⵏⵓⵎ ⴰⵜⵓⵔⴽⵉⵢ. ⵉⵙⵙⴽⵏ ⵜⵓⴼⴰⵢⵜ ⵏⵏⵙ ⴳ ⵢⴰⵏ ⵓⴳⵔⴰⵡ ⴰⴳⵔⴰⵖⵍⴰⵏ ⵏ ⵍⴰⵙⵜⵕⵓⵏⵓⵎⵢ. ⵎⴰⵛⴰ, ⴰⴽⴷ ⵢⵉⵡⵏ ⵓⵔ ⵜ ⵢⵓⵎⵏ ⴰⵛⴽⵓ ⵉⵍⵍⴰ ⵉⵍⵙⴰ ⵢⴰⵜ ⵎⵍⵙⵉⵡⵜ ⵓⵔ ⵉⴳⵉⵏ ⴰⵎⵎ ⵜⵉⵏ ⵎⴷⴷⵏ. ⵎⴷⴷⵏ ⵉⵎⵇⵔⴰⵏⴻⵏ, ⴰⵎⴽⴰ ⴰⴽⴽ ⴰⵢ ⴳⴰⵏ.

ⵎⴰⵛⴰ ⵙ ⵓⵎⴷⴰⵣ ⵏ ⵜⵓⵙⵙⵏⴰ ⵏ ⴰⵙⵜⵔⵓⵉⴷ ⴱ612, ⵉⴽⴽⵔ ⵢⴰⵏ ⵓⴷⵉⴽⵜⴰⵜⵓⵔ ⴰⵜⵓⵔⴽⵢ, ⵉⴳⴳ ⴰⵙⵏ ⵛⵛⵉⵍ ⵉ ⵎⴷⴷⵏ ⴰⴷ ⵍⵙⵙⴰⵏ ⵎⵍⵙⵉⵡⵜ ⵏ ⵓⵔⵓⴱⵉⵢⵏ, ⵡⴰⵏⵏⴰ ⵢⴰⴳⵉⵏ ⵉⵏⵖ ⵜ. ⴰⵙⵜⵔⵓⵏⵓⵎ ⵏⵏⴰⵖ, ⵢⵓⵍⵙ ⴷⵉⵖ ⵉ ⵜⵎⵙⴽⴰⵏⵜ ⵏⵏⵙ ⴰⵙⴳⴳⴰⵙ ⵏ 1920, ⵜⵉⴽⴽⵍⵜ ⵏⵏⴰⵖ ⵉⵍⵍⴰ ⵉⵍⵙⴰ ⵢⴰⵜ ⵎⵍⵙⵉⵡⵜ ⵢⵖⵓⴷⴰⵏ ⵛⵉⴳⴰⵏ. ⵜⵉⴽⴽⵍⵜ ⵏⵏⴰⵖ, ⵎⴷⴷⵏ ⴰⴽⴽ ⵓⵎⴻⵏ ⴰⵡⴰⵍ ⵏⵏⵙ.
\par
\vspace*{10pt}
\end{minipage}
}

\subsection{Unified Canadian Aboriginal Syllabics}

Unified Canadian Aboriginal Syllabics is a Unicode block containing characters for writing Inuktitut, Carrier, several dialects of Cree, and Canadian Athabascan languages. Additions for some Cree dialects, Ojibwe, and Dene can be found at the Unified Canadian Aboriginal Syllabics Extended block.
\medskip

\newfontfamily\aboriginal{code2000.ttf}
\bgroup
\par
\noindent
\colorbox{graphicbackground}{\color{black}^^A
\begin{minipage}{\textwidth}^^A
\parindent1pt
\vskip10pt
\leftskip10pt \rightskip\leftskip

\aboriginal
ᒥᓯᐌ ᐃᓂᓂᐤ ᑎᐯᓂᒥᑎᓱᐎᓂᐠ ᐁᔑ ᓂᑕᐎᑭᐟ ᓀᐢᑕ ᐯᔭᑾᐣ ᑭᒋ ᐃᔑ
\bfseries ᑲᓇᐗᐸᒥᑯᐎᓯᐟ ᑭᐢᑌᓂᒥᑎᓱᐎᓂᐠ ᓀᐢᑕ ᒥᓂᑯᐎᓯᐎᓇ᙮
Unicode Block: Unified Canadian Aboriginal Syllabics, UCAS Extended
Text: UDHR: Cree, Swampy ᐯᔭᐠ ᐱᐢᑭᑕᓯᓇᐃᑲᐣ ᐁᐢᐱᑕᐢᑲᒥᑲᐠ ᐊᐢᑭᐠ ᑭᒋ ᐃᑗᐎᐣ ᐃᓂᓂᐎ ᒥᓂᑯᐎᓯᐎᓇ ᐅᒋ
\par
\vspace*{10pt}
\end{minipage}
}
\medskip
\egroup
\subsection{Miao}

The Pollard script, also known as Pollard Miao (Chinese: 柏格理苗文 Bó Gélǐ Miao-wen) or Miao, is an abugida loosely based on the Latin alphabet and invented by Methodist missionary Sam Pollard. Pollard invented the script for use with A-Hmao, one of several Miao languages. The script underwent a series of revisions until 1936, when a translation of the New Testament was published using it. The introduction of Christian materials in the script that Pollard invented caused a great impact among the Miao. Part of the reason was that they had a legend about how their ancestors had possessed a script but lost it. According to the legend, the script would be brought back some day. When the script was introduced, many Miao came from far away to see and learn it.[1][2]

Pollard credited the basic idea of the script to the Cree syllabics designed by James Evans in 1838–1841, “While working out the problem, we remembered the case of the syllabics used by a Methodist missionary among the Indians of North America, and resolved to do as he had done” (1919:174). He also gave credit to a Chinese pastor, “Stephen Lee assisted me very ably in this matter, and at last we arrived at a system” (1919:174). In listing the phrases he used to describe devising the script, there is clear indication of intellectual work, not revelation: “we looked about”, “resolved to attempt”, “adapting the system”, “solved our problem” (Pollard 1919:174,175).

Changing politics in China led to the use of several competing scripts, most of which were romanizations. The Pollard script remains popular among Hmong in China, although Hmong outside China tend to use one of the alternative scripts. A revision of the script was completed in 1988, which remains in use.

As with most other abugidas, the Pollard letters represent consonants, whereas vowels are indicated by diacritics. Uniquely, however, the position of this diacritic is varied to represent tone. For example, in Western Hmong, placing the vowel diacritic above the consonant letter indicates that the syllable has a high tone, whereas placing it at the bottom right indicates a low tone.

A still experimental font, that supports Graphite technology is \idxfont{Mia Unicode}\footnote{\url{http://phjamr.github.io/miao.html\#intro}}. The font is licenced under the SIL terms and we are using it in the |phd| package as the default font for the Miao script.

\newfontfamily\miao{MiaoUnicode-Regular.ttf}

\begin{scriptexample}[]{Miao}
\unicodetable{miao}{"16F00,"16F10,"16F20,"16F30,"16F40,"16F70,"16F80,"16F90}
\end{scriptexample}

{\miao 𖼴	𖼵	𖼶	𖼷	𖼸	𖼹	𖼺	}

Features for Miao
There are three features currently available for the Miao script:
\bgroup
\miao
Chuxiong ‘wart’ variant
Stylistic alternates for 𖼳 and 𖼴
Aspiration marker always on right
The ‘wart’ (a translated technical term!) is the small circle in characters like 𖼁, 𖼅, and 𖼾. In the Chuxiong orthography, it is rendered not as a circle but as a dot on the right of the letter, as shown in point 5 here (pdf).

Miao Unicode has a feature called “chux” for handling this. In LibreOffice you can use this style by typing “Miao Unicode:chux=1” into the font field.
\section{N'ko}

\newfontfamily\nko{NotoSansNKo-Regular.ttf}

N'Ko {\nko(ߒߞߏ)} is both a script devised by Solomana Kante in 1949 as a writing system for the Manding languages of West Africa, and the name of the literary language itself written in the script. The term N'Ko means ``I say'' in all Manding languages.

The script has a few similarities to the Arabic script, notably its direction (right-to-left) and the connected letters. It obligatorily marks both tone and vowels.


\begin{scriptexample}[]{N'ko}
\unicodetable{nko}{"07C0,"07D0,"07E0,"07F0}
\end{scriptexample}

The N'Ko alphabet is written from right to left, with letters being connected to one another.

The script is principally used in Guinea and Côte d'Ivoire (respectively by Maninka and Dioula-speakers), with an active user community in Mali (by Bambara-speakers). Publications include a translation of the Qur'an, a variety of textbooks on subjects such as physics and geography, poetic and philosophical works, descriptions of traditional medicine, a dictionary, and several local newspapers. It has been classed as the most successful of the West African scripts.[3] The literary language used is intended as a koine blending elements of the principal Manding languages (which are mutually intelligible), but has a particularly strong Maninka flavour.

The Latin script with several extended characters (phonetic additions) is used for all Manding languages to one degree or another for historical reasons and because of its adoption for "official" transcriptions of the languages by various governments. In some cases, such as with Bambara in Mali, promotion of literacy using this orthography has led to a fair degree of literacy in it. Arabic transcription is commonly used for Mandinka in The Gambia and Senegal.


\subsection{Mongolian}
\newfontfamily\mongolian{NotoSansMongolian-Regular.ttf}

The classical Mongolian script (in Mongolian script:{\mongolian ᠮᠣᠩᠭᠣᠯ ᠪᠢᠴᠢᠭ᠌} Mongγol bičig; in Mongolian Cyrillic: Монгол бичиг Mongol bichig), also known as Uyghurjin Mongol bichig, was the first writing system created specifically for the Mongolian language, and was the most successful until the introduction of Cyrillic in 1946. Derived from Uighur, Mongolian is a true alphabet, with separate letters for consonants and vowels. The Mongolian script has been adapted to write languages such as Oirat and Manchu. Alphabets based on this classical vertical script are used in Inner Mongolia and other parts of China to this day to write Mongolian, Sibe and, experimentally, Evenki.

\begin{scriptexample}[]{Mongolian}
\unicodetable{mongolian}{"1820,"1830,"1840,"1850,"1860,"1870,"1880,"1890,"18A0}
\end{scriptexample}


%  \chapter{South East Asian Scripts}
\label{ch:southeastasia}
\section{Introduction}

This section documents the facilities offered to typeset Southeast Asian Scripts. These scripts are used in most of Southeast Asia, Indonesia and the Philippines.

\pagestyle{headings}

\begin{table}[htb]
\centering
\begin{tabular}{lll}
  \hyperref[s:thai]{Thai} 
& Tai Tham 
& \hyperref[s:balinese]{Balinese}\\
\hyperref[s:lao]{Lao}  
&Tai Viet  
& \hyperref[s:javanese]{Javanese}\\
Myanmar 
&Kayah Li 
&Rejang\\
 \hyperref[s:khmer]{Khmer} 
&Cham 
&Batak\\
Tai Le 
&Philippine Scripts 
& \hyperref[s:sundanese]{Sundanese}\\
  \hyperref[s:newtailue]{New Tail Lue}
& Buginese\\
\end{tabular}
\end{table}

\subsection{Balinese}

The Balinese script, natively known as Aksara Bali and Hanacaraka, is an abugida used in the island of Bali, Indonesia, commonly for writing the Austronesian Balinese language, Old Javanese, and the liturgical language Sanskrit. With some modifications, the script is also used to write the Sasak language, used in the neighboring island of Lombok.[1] The script is a descendant of the Brahmi script, and so has many similarities with the modern scripts of South and Southeast Asia. The Balinese script, along with the Javanese script, is considered the most elaborate and ornate among Brahmic scripts of Southeast Asia.[2]

Though everyday use of the script has largely been supplanted by the Latin alphabet, the Balinese script has significant prevalence in many of the island's traditional ceremonies and is strongly associated with the Hindu religion. The script is mainly used today for copying lontar or palm leaf manuscripts containing religious texts.[2][3]

\newfontfamily\balinese{AksaraBali.ttf}
\newfontfamily\indicative{code2000.ttf}

{\indicative ◌ }

\newcounter{under}
\setcounter{under}{"1B00}

\def\cb#1 {
\hspace*{2.5pt}
 \large
 $\text{◌#1}_{\pgfmathparse{Hex(\theunder)}\pgfmathresult}$
\stepcounter{under}
\vskip5pt\par
}
\begin{scriptexample}[]{Balinese}


\balinese
	 
᭐	᭑	᭒	᭓	᭔	᭕	᭖	᭗	᭘	᭙	᭚	᭛	᭜	᭝	᭞	᭟\\\
 
\def\columnseprulecolor{\color{thegray}}
\columnseprule.4pt
\begin{multicols}{8}

\texttt{U+1B0x}	

\cb{ᬀ }  \cb{ ᬁ } 	\cb{ ᬂ } 	\cb ᬃ	\cb ᬄ 	\cb ᬅ	\cb ᬆ	\cb ᬇ	\cb ᬈ	\cb ᬉ	\cb ᬊ	\cb ᬋ	\cb ᬌ	\cb ᬍ	\cb ᬎ	\cb ᬏ

\columnbreak

\texttt{U+1B1x}	 

\cb ᬐ	 \cb ᬑ 	\cb ᬒ 	\cb ᬓ	\cb ᬔ	\cb ᬕ	\cb ᬖ \cb ᬗ 	\cb ᬘ 	\cb ᬙ 	\cb ᬚ	\cb ᬛ 	\cb ᬜ 	\cb ᬝ 	\cb ᬞ	\cb ᬟ 

\columnbreak

U+1B2x	 

\cb ᬠ◌ 	\cb ᬡ	\cb ᬢ	\cb ᬣ	\cb ᬤ	\cb ᬥ	\cb ᬦ	\cb ᬧ	\cb ᬨ	\cb ᬩ	\cb ᬪ	\cb ᬫ	\cb ᬬ	\cb ᬭ	\cb ᬮ	\cb ᬯ

\columnbreak
U+1B3x 

\cb ᬰ	\cb ᬱ	\cb ᬲ	\cb ᬳ	\cb ᬴	\cb ᬵ	\cb ᬶ	\cb ᬷ	\cb ᬸ	\cb ᬹ	\cb ᬺ	\cb ᬻ	\cb ᬼ	\cb ᬽ	\cb ᬾ	\cb ᬿ


\columnbreak
U+1B4x	 

\cb ᭀ	 \cb ᭁ	\cb ᭂ	\cb ᭃ	\cb ᭄	\cb ᭅ	\cb ᭆ	\cb ᭇ	\cb ᭈ	\cb ᭉ	\cb ᭊ	\cb ᭋ

\columnbreak				
U+1B5x	 

\cb ᭐	\cb ᭑	\cb ᭒	\cb ᭓	\cb ᭔	\cb ᭕	\cb ᭖	\cb ᭗	\cb ᭘	\cb ᭙	\cb ᭚	\cb ᭛	\cb ᭜	\cb ᭝	\cb ᭞	\cb ᭟\\

\columnbreak

U+1B6x 

\cb ᭠	\cb ᭡	\cb ᭢	\cb ᭣	\cb ᭤	\cb ᭥	\cb ᭦	\cb ᭧	\cb ᭨◌ 	\cb ᭩◌ 	\cb ᭪◌ 	\cb ᭫	\cb ᭬	\cb ᭭	\cb ᭮	\cb ᭯

\columnbreak
U+1B7x	 

\cb ᭰	 \cb ᭱  \cb ᭲  \cb ᭳	 \cb ᭴	\cb ᭵	\cb ᭶	\cb ᭷	\cb ᭸	\cb ᭹	\cb ᭺	\cb ᭻	\cb ᭼


\end{multicols}

\end{scriptexample}
\defaulttext

One of the most comprehensive fonts is Aksara Bali\footnote{\url{http://www.alanwood.net/downloads/index.html}}. This is obtainable at Alan Wood's website.
\clearpage

%\newfontfamily\javanese{Noto Sans Javanese}

%\newfontfamily\javanese{TuladhaJejeg_gr.ttf}

\section{Javanese}
\label{s:javanese}
\index{scripts>Javanese}


The Javanese (Ngoko Javanese: {\javanese ꦮꦺꦴꦁꦗꦮ},[3] Madya Javanese: {\javanese\   ꦠꦶꦪꦁꦗꦮꦶ},[4] Krama Javanese: ꦥꦿꦶꦪꦤ꧀ꦠꦸꦤ꧀ꦗꦮꦶ,[4] Ngoko Gêdrìk: wòng Jåwå, Madya Gêdrìk: tiyang Jawi, Krama Gêdrìk: priyantun Jawi, Indonesian: suku Jawa)[5] are an ethnic group native to the Indonesian island of Java. With approximately 100 million people (as of 2011), they form the largest ethnic group in Indonesia. They are predominantly located in the central to eastern parts of the island. There are also significant numbers of people of Javanese descent in most provinces of Indonesia, Malaysia, Singapore, Suriname, Saudi Arabia and the Netherlands.

The Javanese ethnic group has many sub-groups, such as the Mataram, Cirebonese, Osing, Tenggerese, Samin, Naganese, Banyumasan, etc.[6]

A majority of the Javanese people identify themselves as Muslims, with a minority identifying as Christians and Hindus. However, Javanese civilization has been influenced by more than a millennium of interactions between the native animism Kejawen and the Indian Hindu—Buddhist culture, and this influence is still visible in Javanese history, culture, traditions, and art forms. With a sizeable global population, the Javanese are considered significant as they are the fourth largest ethnic group among Muslims, in the world, after the Arabs,[7] Bengalis[8] and Punjabis.[9]


\paragraph{Javanese} is one of the Austronesian languages, but it is not particularly close to other languages and is difficult to classify. Its closest relatives are the neighbouring languages such as Sundanese, Madurese and Balinese. Most speakers of Javanese also speak Indonesian, the standardized form of Malay spoken in Indonesia, for official and commercial purposes as well as a means to communicate with non-Javanese-speaking Indonesians.

There are speakers of Javanese in Malaysia (concentrated in the states of Selangor and Johor) and Singapore. Some people of Javanese descent in Suriname (the Dutch colony of Suriname until 1975) speak a creole descendant of the language.

\begin{figure}[htbp]
\includegraphics[width=\textwidth]{javanese-people}
\end{figure}

The language is spoken in Yogyakarta, Central and East Java, as well as on the north coast of West Java. It is also spoken elsewhere by the Javanese people in other provinces of Indonesia, which are numerous due to the government-sanctioned transmigration program in the late 20th century, including Lampung, Jambi, and North Sumatra provinces. In Suriname, creolized Javanese is spoken among descendants of plantation migrants brought by the Dutch during the 19th century. In Madura, Bali, Lombok, and the Sunda region of West Java, it is also used as a literary language. It was the court language in Palembang, South Sumatra, until the palace was sacked by the Dutch in the late 18th century.

Javanese is written with the Latin script, Javanese script, and Arabic script.[5] In the present day, the Latin script dominates writings, although the Javanese script is still taught as part of the compulsory Javanese language subject in elementary up to high school levels in Yogyakarta, Central and East Java.

Javanese is the tenth largest language by native speakers and the largest language without official status. It is spoken or understood by approximately 100 million people. At least 45\% of the total population of Indonesia are of Javanese descent or live in an area where Javanese is the dominant language. All seven Indonesian presidents since 1945 have been of Javanese descent.[6] It is therefore not surprising that Javanese has had a deep influence on the development of Indonesian, the national language of Indonesia.

There are three main dialects of the modern language: Central Javanese, Eastern Javanese, and Western Javanese. These three dialects form a dialect continuum from northern Banten in the extreme west of Java to Banyuwangi Regency in the eastern corner of the island. All Javanese dialects are more or less mutually intelligible.


\paragraph{The Javanese script} (Hanacaraka/Carakan) is a script for writing the Javanese language, the native language of one of the peoples of the Island of Java. It is a descendent of the ancient Brahmi script of India, and so has many similarities with modern scripts of South Asia and Southeast Asia. The Javanese script is also used for writing Sanskrit, Old Javanese, and transcriptions of Kawi, as well as the Sundanese language, and the Sasak language.

\begin{figure}[htbp]
\hspace*{-1.5cm}\includegraphics[width=1.2\textwidth]{java-palm-leave-manuscript}
\end{figure}





\begin{scriptexample}[]{Javanese}
\bgroup
\javanese

꧋ꦱꦧꦼꦤ꧀ꦮꦺꦴꦁꦏꦭꦲꦶꦂꦲꦏꦺꦏꦤ꧀ꦛꦶꦩꦂꦢꦶꦏꦭꦤ꧀ꦢꦂꦧꦺꦩꦂꦠꦧꦠ꧀ꦭꦤ꧀ꦲꦏ꧀ꦲꦏ꧀ꦏꦁꦥꦝ꧉

꧋ ꦲꦮꦶꦠ꧀ꦲꦶꦏꦁꦄꦱ꧀ꦩꦄꦭ꧀ꦭꦃ꧈ ꦏꦁꦩꦲꦩꦸꦫꦃꦠꦸꦂ ꦩꦲꦲꦱꦶꦃ꧉ 	 
 ۝꧋ ꦄꦭꦶꦥꦃ꧀ ꦭ ꦩ꧀ ꦫ ꧌ ꦏꦁ — — ꦥꦿꦶꦏ꧀ꦱ ꦏꦉꦪꦥ꧀ꦥꦩꦸꦁꦄꦭ꧀ꦭꦃꦥꦶꦪꦺꦩ꧀ꦧꦏ꧀ ꧌꧉ ꦩꦁꦪꦏꦴꦪꦤꦴ ꦲꦶꦏꦸꦄꦪꦺꦪꦠꦴꦏꦶꦠꦧ꧀ꦑꦸꦂꦄꦤ꧀ꦏꦁꦥꦿꦪꦠꦭ꧉ 	 
᭐	᭑	᭒	᭓	᭔	᭕	᭖	᭗	᭘	᭙	᭚	᭛	᭜	᭝	᭞	᭟
 
\egroup
\end{scriptexample}


The Javanese script was added to Unicode Standard in version 5.2 on the code points \texttt{A980 - A9DF}. There are 91 code points for Javanese script: 53 letters, 19 punctuation marks, 10 numbers, and 9 vowels:
\medskip

\unicodetable{javanese}{"A980,"A990,"A9A0, "A9B0, "A9C0,"A9D0}

\medskip



As of the writing of this document (2017), there are several widely published fonts able to support Javanese, ANSI-based Hanacaraka/Pallawa by Teguh Budi Sayoga,[21] Adjisaka by Sudarto HS/Ki Demang Sokowanten,[22] JG Aksara Jawa by Jason Glavy,[23] Carakan Anyar by Pavkar Dukunov,[24] and Tuladha Jejeg by R.S. Wihananto,[25] which is based on Graphite (SIL) smart font technology. Other fonts with limited publishing includes Surakarta made by Matthew Arciniega in 1992 for Mac's screen font,[26] and Tjarakan developed by AGFA Monotype around 2000.[27] There is also a symbol-based font called Aturra developed by Aditya Bayu in 2012–2013.[28]

Due to the script's complexity, many Javanese fonts have different input method compared to other Indic scripts and may exhibit several flaws. \docFont{JG Aksara Jawa}, in particular, may cause conflicts with other writing system, as the font use code points from other writing systems to complement Javanese's extensive repertoire. This is to be expected, as the font was made before Javanese implementation in Unicode.[29]

Arguably, the most "complete" font, in terms of technicality and glyph count, is \docFont{TuladhaJejeg}. It comes with keyboard facilities, displaying complex syllable structure, and support extensive glyph repertoire including non-standard forms which may not be found in regular Javanese texts, by utilizing Graphite (SIL) smart font technology. |Tuladha Jejeg| uses variable stroke widths on its glyphs with serifs on some glyphs\footnote{\protect\url{https://sites.google.com/site/jawaunicode/main-page}}.

However, as not many writing systems require such complex feature, use is limited to programs with Graphite technology, such as Firefox browser, Thunderbird email client, and several OpenType word processor and of course XeLaTeX. The font was chosen for displaying Javanese script in the Javanese Wikipedia.[16]

\paragraph{jawaTeX} Jawa\TeX{} project is initial effort to make Javanese characters typesetting program using \TeX{}/\LaTeX{}. This project is aimed to make Javanese widely used. The main project is developing transliteration models to transliterate Latin document into Javanese document. Perl and \TeX{}/\LaTeX{} are use in this project, the program are develop to run in text mode (console) both Linux and Windows but not limit on it. Web based program also developed, and automatic embedded Javanese characters in HTML See \href{http://jawatex.org/jawa/jawatex}{jawatex}.




\section{Khmer}
\newfontfamily\normaltext{Arial Unicode MS}
\normaltext

\def\khmerdefaultfont#1{\newfontfamily\khmer[Scale=MatchUppercase]{#1}}
\def\khmertext#1{{\khmer#1}}

\cxset{khmer font/.code=\khmerdefaultfont{#1}}

\cxset{khmer font/.default=Khmer}

\cxset{language=khmer, 
       khmer font = Khmer UI}

\begin{key}{/chapter/khmer font=\meta{font name} (Khmer  UI)} Loads the font
command \cmd{\khmer}. When the command is used it typesets text in
khmer unicode. There is no need to load the language, unless it is the main document language. For windows the default font is \texttt{DaunPenh} this font is in general too small to read; a better font to use is Khmer UI.
\end{key}

\begin{key}{/tikz/turtle/right=\meta{angle} (default 90)}
  Turns the turtle right by the given angle. 
\end{key}


The Khmer script (Khmer: {\Large\khmertext{អក្សរខ្មែរ}}; IPA: [ʔaʔsɑː kʰmaːe]) [2] is an \textit{abugida} (alphasyllabary) script used to write the Khmer language (the official language of Cambodia). It is also used to write Pali among the Buddhist liturgy of Cambodia and Thailand.

It was adapted from the Pallava script, a variant of Grantha alphabet descended from the Brahmi script of India, which was used in southern India and South East Asia during the 5th and 6th Centuries AD.[3] The oldest dated inscription in Khmer was found at Angkor Borei District in Takéo Province south of Phnom Penh and dates from 611.[4] The modern Khmer script differs somewhat from precedent forms seen on the inscriptions of the ruins of Angkor.

Not all Khmer consonants can appear in syllable-final position. The most common syllable-final consonants include {\khmer កងញតនបមល}. The pronunciation of the consonant in final position may differ from it's normal pronunciation.


\begin{tabular}{llp{9cm}}
\khmertext{ំ}	&nĭkkôhĕt (\khmertext{និគ្គហិត})	&niggahita; nasalizes the inherent vowels and some of the dependent vowels, see anusvara, sometimes used to represent [aɲ] in Sanskrit loanwords\\
\khmertext{ះ}	&reăhmŭkh (\khmertext{រះមុខ})	&"shining face"; adds final aspiration to dependent or inherent vowels, usually omitted, corresponds to the visarga diacritic, it maybe included as dependent vowel symbol\\
\khmertext{ៈ}	&yŭkôleăkpĭntŭ (\khmertext{យុគលពិន្ទុ})	&yugalabindu ("pair of dots"); adds final glottalness to dependent or inherent vowels, usually omitted\\
\khmertext{៉}	 &musĕkâtônd (\khmertext{មូសិកទន្ត})	&mūsikadanta ("mouse teeth"); used to convert some o-series consonants (\khmertext{ង ញ ម យ រ វ}) to the a-series\\
\khmertext{៊}	&treisâpt (\khmertext{ត្រីសព្ទ})	trīsabda; used to convert some a-series consonants (\khmertext{ស ហ ប អ}) to the o-series\\
\end{tabular}




ុ	kbiĕh kraôm (ក្បៀសក្រោម)	also known as bŏkcheung (បុកជើង); used in place of the diacritics treisâpt and musĕkâtônd when they would be impeded by superscript vowels
់	bântăk (បន្តក់)	used to shorten some vowels; the diacritic is placed on the last consonant of the syllable
៌	rôbat (របាទ)
répheăk (រេផៈ)	rapāda, repha; behave similarly to the tôndâkhéat, corresponds to the Devanagari diacritic repha, however it lost its original function which was to represent a vocalic r
 ៍	tôndâkhéat (ទណ្ឌឃាដ)	daṇḍaghāta; used to render some letters as unpronounced
៎	kakâbat (កាកបាទ)	kākapāda ("crow's foot"); more a punctuation mark than a diacritic; used in writing to indicate the rising intonation of an exclamation or interjection; often placed on particles such as /na/, /nɑː/, /nɛː/, /vəːj/, and the feminine response /cah/
៏	âsda (អស្តា)	denotes stressed intonation in some single-consonant words[5]
័	sanhyoŭk sannha (សំយោគសញ្ញា)	represents a short inherent vowel in Sanskrit and Pali words; usually omitted
៑	vĭréam (វិរាម)	a mostly obsolete diacritic, corresponds to the virāma
្	cheung (ជើង)	a.w. coeng; a sign developed for Unicode to input subscript consonants, appearance of this sign varies among fonts
\section{Sundanese}
\newfontfamily\sundanese{SundaneseUnicode-1.0.5.ttf}
^^A\newfontfamily\sundanese{Arial Unicode MS}
\def\ublock#1{\texttt{{\arial #1}}}

The Sundanese script (Aksara Sunda, {\sundanese ᮃᮊ᮪ᮞᮛ ᮞᮥᮔ᮪ᮓ}) is a writing system which is used by the Sundanese people. It is built based on Old Sundanese script (Aksara Sunda Kuno) which was used by the ancient Sundanese between the 14th and 18th centuries.

\begin{scriptexample}[]{Sundanese}
\unicodetable{sundanese}{"1B80,"1B90,"1BA0,"1BB0}

\sundanese
\obeylines
\bgroup
᮱ {\arial= 1}	᮲ {\arial= 2}	᮳{\arial = 3}
᮴ {\arial= 4}	᮵ {\arial = 5} 	᮶ {\arial= 6}
᮷ {\arial= 7}	᮸ {\arial= 8}	᮹ {\arial= 9}
᮰ {\arial= 0}

\egroup
\end{scriptexample}

\begin{scriptexample}[]{Sundanese}
\bgroup
\sundanese
\centering

◌ᮃᮄᮅᮆᮇᮈᮉᮊᮋᮌᮍᮎᮏᮐᮕᮔᮓᮑᮖᮗᮚᮛᮜᮝᮞᮟᮠᮠ


\egroup
\end{scriptexample}

\bgroup
\def\1{\sundanese ᮱}
\TextOrMath\1\1

$\1$
\egroup

In text In texts, numbers are written surrounded with dual pipe sign \textbar \ldots \textbar. Example: {\textbar \sundanese ᮲᮰᮱᮰\textbar} = 2010












%\newfontfamily\hanunoo{NotoSansHanunoo-Regular.ttf}

\section{Hanunó’o}

Hanunó’o is one of the indigenous scripts of the Philippines and is used by the Mangyan peoples of southern Mindoro to write the Hanunó'o language.[1] 

It is an \emphasis{abugida} descended from the Brahmic scripts, closely related to Baybayin, and is famous for being written vertical but written upward, rather than downward as nearly all other scripts (however, it's read horizontally left to right). It is usually written on bamboo by incising characters with a knife.[2][3] Most known Hanunó'o inscriptions are relatively recent because of the perishable nature of bamboo. It is therefore difficult to trace the history of the script



\begin{scriptexample}[width=2cm]{Hanunoo}
\hanunoo

{\Large
\obeylines
ᜠ 
ᜫ
ᜨᜲ
ᜫᜲ
ᜰ
ᜮ
ᜥ
ᜦ᜴}

Typeset with \texttt{NotoSansHanunoo-Regular.ttf} and the command \cmd{\hanunoo}
\end{scriptexample}

Vertically positionning the text is not currently supported by \pkgname{fontspec} and the manual says \textsc{Todo!}. You are your own here, or you can just put the characters in a box and give it a try.

\begin{minipage}[t]{2cm}
\begin{tcolorbox}[width=2cm,colback=graphicbackground,
boxrule=0pt,toprule=0pt,colframe=white]
\Large\hanunoo
ᜩ\\
ᜤ\\
ᜮ\\
ᜥᜳ\\
ᜨ᜴ \\
ᜨ᜴\\
ᜫᜳ\\
ᜥ\\
\end{tcolorbox}
\end{minipage}
\begin{minipage}[t]{2cm}
\begin{tcolorbox}[width=2cm,colback=graphicbackground,
boxrule=0pt,toprule=0pt,colframe=white]
\LARGE\hanunoo
ᜩ\\
ᜤ\\
ᜮ\\
ᜥᜳ\\
ᜨ᜴ \\
ᜨ᜴\\
ᜫᜳ\\
ᜥ\\
\end{tcolorbox}
\end{minipage}
\begin{minipage}[t]{\textwidth-6cm}

The script is written from bottom to top. Typesetting this type of script automatically is not without its problems. One way is to use the build-in features of the font if they are available, but currently this gives problems---at least with the fonts that I have tried. Entering the text is also problematic as you will more than likely see little boxes rather than the actual glyph with most text editors common to \latexe. If you only need a couple of characters or a short sentence, an easy solution is to use |\rotatebox|. Another solution is to use a macro that can add the letters onto a stack, then place them in a box with a limited width. We can use |\@tfor| for this.  
\end{minipage}
\section{New Tai Lue Script}
\label{s:newtailue}
\newfontfamily\tailue{Noto Sans New Tai Lue}


New Tai Lue script, also known as Simplified Tai Lue, is an alphabet used to write the Tai Lü language. Developed in China in the 1950s, New Tai Lue is based on the traditional Tai Le alphabet developed ca. 1200 AD. The government of China promoted the alphabet for use as a replacement for the older script; teaching the script was not mandatory, however, and as a result many are illiterate in New Thai Lue. 

\begin{figure}[htbp]
\centering

\includegraphics[width=\linewidth-2\parindent]{tailue}

\caption{Tai Le costumes. (pininterest)}
\end{figure}

In addition, communities in Burma, Laos, Thailand and Vietnam still use the Tai Le alphabet. There are probably less than one million native speakers of the language who can be found in China, Burma, Laos, Thailand and Vietnam.

\begin{figure}[htbp]
\centering

\includegraphics[width=\linewidth-2\parindent]{tai-lu}

\caption{Tai Le costumes. (pininterest)}
\end{figure}

\begin{scriptexample}[]{Tai Lue}
{\centering\tailue \LARGE

ᦒ	ᦓ	ᦔ	ᦕ	ᦖ	ᦗ	ᦘ	ᦙ	ᦚ	ᦛ	ᦜ	ᦝ	ᦞ	

}
\end{scriptexample}

The New Tai Lue script was added to the Unicode Standard in March, 2005 with the release of version 4.1.

The Unicode block for New Tai Lue is |U+1980|–|U+19DF|:

\begin{scriptexample}[]{New Tai Lue}
\unicodetable{tailue}{"1980,"1990,"19A0,"19B0,"19C0,"19D0}

\texttt{typeset using NotoSansNewTaiLue-Regular.ttf.}
\end{scriptexample}
\section{Myanmar}
\label{s:myanmar}
\index{Myanmar}\index{Burmese}\index{Mon}\index{Unicode>Myanmar}\index{Fonts>Padauk}

%\newfontfamily\myanmar{Padauk}

The Burmese script (Burmese:{\myanmar မြန်မာအက္ခရာ}; MLCTS: mranma akkha.ra; pronounced: [mjəmà ʔɛʔkʰəjà]) is an abugida in the Brahmic family, used for writing Burmese. It is an adaptation of the Old Mon script[2] or the Pyu script. In recent decades, other alphabets using the Mon script, including Shan and Mon itself, have been restructured according to the standard of the now-dominant Burmese alphabet. Besides the Burmese language, the Burmese alphabet is also used for the liturgical languages of Pali and Sanskrit.

The characters are rounded in appearance because the traditional palm leaves used for writing on with a stylus would have been ripped by straight lines.[3] It is written from left to right and requires no spaces between words, although modern writing usually contains spaces after each clause to enhance readability.

The earliest evidence of the Burmese alphabet is dated to 1035, while a casting made in the 18th century of an old stone inscription points to 984.[1] Burmese calligraphy originally followed a square format but the cursive format took hold from the 17th century when popular writing led to the wider use of palm leaves and folded paper known as parabaiks.[3] The alphabet has undergone considerable modification to suit the evolving phonology of the Burmese language.

Mon/Burmese script was added to the Unicode Standard in September, 1999 with the release of version 3.0. It was extended in October, 2009 with the release of version 5.2 and again in June, 2014 with the release of version 7.0.

\begin{docKey}[phd]{myanmar font}{=\meta{font name}}{default none initial Padauk}
Loads the font and creates associated environments and commands.
\end{docKey}

\begin{scriptexample}[]{Myanmar}
\unicodetable{myanmar}{"1000,"1010,"1020,"1030,"1040,"1050,"1060,"1070,"1080,"1090}
\end{scriptexample}








%\subsection{Oriya alphabet}
\newfontfamily\oriya[Scale=1.1,Script=Oriya]{code2000.ttf}

\def\oriyatext#1{{\oriya#1}}
The Oriya script or Utkala Lipi (Oriya: \oriyatext{ଉତ୍କଳ ଲିପି}) or Utkalakshara (Oriya: \oriyatext{ଉତ୍କଳାକ୍ଷର}) is used to write the Oriya language, and can be used for several other Indian languages, for example, Sanskrit.

\centerline{\Huge\oriyatext{ଉତ୍କଳ ଲିପି}}

\bgroup
\oriya
୦୧୨୩୪୫୬୭୮୯
ଅ ଆ ଇ ଈ ଉ ଊ ଋ ୠ ଌ ୡ ଏ ଐ ଓ ଔ କ ଖ ଗ ଘ ଙ ଚ ଛ ଜ ଝ ଞ ଟ ଠ ଡ ଢ ଣ ତ ଥ ଦ ଧ ନ ପ ଫ ବ ଵ ଭ ମ ଯ ର ଳ ୱ ଶ ଷ ସ ହ ୟ ଲ
\egroup

\begin{quotation}
Oṛiyā is encumbered with the drawback of an excessively awkward and cumbrous written character. ... At first glance, an Oṛiyā book seems to be all curves, and it takes a second look to notice that there is something inside each.(G. A. Grierson, Linguistic Survey of India, 1903)
\end{quotation}

Comparison of Oṛiyā script with its neighbours[edit]
At a first look the great number of signs with round shapes suggests a closer relation to the southern neighbour Telugu than to the other neighbours Bengali in the north and Devanāgarī in the west. The reason for the round shapes in Oriya and Telugu (and also in Kannaḍa and Malayāḷam) is the former method of writing using a stylus to scratch the signs into a palm leaf. These tools do not allow for horizontal strokes because that would damage the leaf.

Oriya letters are mostly round shaped whereas in Devanāgarī and Bengali have horizontal lines. So in most cases the reader of Oṛiyā will find the distinctive parts of a letter only below the hoop. Considering this the  closer relation to Devanāgarī and Bengali exists than to any southern script, though both northern and southern scripts have the same origin, Brāhmī.

Oriya (\oriyatext{ଓଡ଼ିଆ} oṛiā), officially spelled Odia,[3][4] is an Indian language belonging to the Indo-Aryan branch of the Indo-European language family. It is the predominant language of the Indian states of Odisha, where native speakers comprise 80\% of the population,[5] and it is spoken in parts of West Bengal, Jharkhand, Chhattisgarh and Andhra Pradesh. Oriya is one of the many official languages in India; it is the official language of Odisha and the second official language of Jharkhand. [6][7][8] Oriya is the sixth Indian language to be designated a Classical Language in India, on the basis of having a long literary history and not having borrowed extensively from other languages.


%\subsection{Mongolian Script}

\newfontfamily\mongolian[Language=Mongolian, Scale=1.3]{code2000.ttf}

The classical Mongolian script (in Mongolian script: {\mongolian  ᠮᠣᠩᠭᠣᠯ ᠪᠢᠴᠢᠭ᠌} Mongγol bičig; in Mongolian Cyrillic: Монгол бичиг Mongol bichig), also known as Uyghurjin Mongol bichig, was the first writing system created specifically for the Mongolian language, and was the most successful until the introduction of Cyrillic in 1946. Derived from Uighur, Mongolian is a true alphabet, with separate letters for consonants and vowels. The Mongolian script has been adapted to write languages such as Oirat and Manchu. Alphabets based on this classical vertical script are used in Inner Mongolia and other parts of China to this day to write Mongolian, Sibe and, experimentally, Evenki.
\medskip

\bgroup\par
\noindent
\colorbox{graphicbackground}{\color{black}^^A
\begin{minipage}{\textwidth}^^A
\parindent1pt
\vskip10pt
\leftskip10pt \rightskip\leftskip
\mongolian
\large
ᠬᠦᠮᠦᠨ ᠪᠦᠷ ᠲᠥᠷᠥᠵᠦ ᠮᠡᠨᠳᠡᠯᠡᠬᠦ ᠡᠷᠬᠡ ᠴᠢᠯᠥᠭᠡ ᠲᠡᠢ᠂ ᠠᠳᠠᠯᠢᠬᠠᠨ ᠨᠡᠷ᠎ᠡ ᠲᠥᠷᠥ ᠲᠡᠢ᠂ ᠢᠵᠢᠯ ᠡᠷᠬᠡ ᠲᠡᠢ ᠪᠠᠢᠠᠭ᠃ ᠣᠶᠤᠨ ᠤᠬᠠᠭᠠᠨ᠂ ᠨᠠᠨᠳᠢᠨ ᠴᠢᠨᠠᠷ ᠵᠠᠶᠠᠭᠠᠰᠠᠨ ᠬᠦᠮᠦᠨ ᠬᠡᠭᠴᠢ ᠥᠭᠡᠷ᠎ᠡ ᠬᠣᠭᠣᠷᠣᠨᠳᠣ᠎ᠨ ᠠᠬᠠᠨ ᠳᠡᠭᠦᠦ ᠢᠨ ᠦᠵᠢᠯ ᠰᠠᠨᠠᠭᠠ ᠥᠠᠷ ᠬᠠᠷᠢᠴᠠᠬᠥ ᠤᠴᠢᠷ ᠲᠠᠢ᠃
\par
\vspace*{10pt}
\end{minipage}
}
\medskip
%\subsection{Tibetan}

^^A\newfontfamily\tibetan{TibMachUni.ttf}

^^A\newfontfamily\tibetan{Qomolangma-Chuyig.ttf}

^^A should pick it up automatically \tibetan

Fonts described in this section can be obtained from The Tibetan \& Himalayan Library
\footnote{\url{http://www.thlib.org/tools/scripts/wiki/tibetan%20machine%20uni.html}  }

I have tried a few \texttt{Tibetan Machine Uni (TMU)} seems to be used by a number of scholars. 

A tip when you are trying to locate fonts is to find a related article in Wikipedia, such as Tibetan alphabet and inspect the element using your browser to see what fonts are being used.


|style="font-family:'Jomolhari','Tibetan Machine Uni','DDC Uchen', 'Kailash';| 


If you cannot see the script and rather than boxes or question marks then you can search and download one of the fonts in |font-family|.

\def\tibetandefaultfont#1{\newfontfamily\tibetan[Language=Tibetan]{#1}}


\cxset{language=tibetan} 
\cxset{tibetan font/.code=\tibetandefaultfont{#1}}


^^A\cxset{tibetan font = TibMachUni.ttf}




\begin{key}{/chapter/language = tibetan} The key |language=tibetan| sets the default language as Tibetan, using the main font given by the key |tibetan font=TibMachUni.ttf|.
\end{key}

\begin{key}{/chapter/tibetan font = TibMachUni.ttf} The key |tibetan font=font-name| sets the default font for the Tibetan language. It will also create the switch \cmd{\tibetan} for typesetting text in Tibetan.
\end{key}

\begin{texexample}{Tibetan language setttings}{ex:tibetan}
\cxset{language=tibetan, tibetan font = TibMachUni.ttf}
\tibetan

\tibetan Tibetan: དབུ་ཅན
\end{texexample}


The Tibetan alphabet is an \emph{abugida} of Indic origin used to write the Tibetan language as well as Dzongkha, the Sikkimese language, Ladakhi, and sometimes Balti. 

The printed form of the alphabet is called \textit{uchen} script (Tibetan: དབུ་ཅན་, Wylie: dbu-can; "with a head") while the hand-written cursive form used in everyday writing is called umê script (Tibetan: དབུ་མེད་, Wylie: dbu-med; "headless").
\uccoff
The alphabet is very closely linked to a broad ethnic Tibetan identity. Besides Tibet, it has also been used for Tibetan languages in Bhutan, India, Nepal, and Pakistan.[1] The Tibetan alphabet is ancestral to the Limbu alphabet, the Lepcha alphabet,[2] and the multilingual 'Phags-pa script.[2]
\uccon

The Tibetan alphabet is romanized in a variety of ways.[3] This article employs the Wylie transliteration system.

The Tibetan alphabet has thirty basic letters, sometimes known as "radicals", for consonants.[2]

ཀ ka /ká/	ཁ kha /kʰá/	ག ga /kà, kʰà/	ང nga /ŋà/
ཅ ca /tʃá/	ཆ cha /tʃʰá/	ཇ ja /tʃà/	ཉ nya /ɲà/
ཏ ta /tá/	ཐ tha /tʰá/	ད da /tà, tʰà/	ན na /nà/
པ pa /pá/	ཕ pha /pʰá/	བ ba /pà, pʰà/	མ ma /mà/
ཙ tsa /tsá/	ཚ tsha /tsʰá/	ཛ dza /tsà/	ཝ wa /wà/ (not originally part of the alphabet)[5]
ཞ zha /ʃà/[6]	ཟ za /sà/	འ 'a /hà/[7]
ཡ ya /jà/	ར ra /rà/	ལ la /là/
ཤ sha /ʃá/[6]	ས sa /sá/	ཧ ha /há/[8]
ཨ a /á/

\subsubsection{Unicode Block Tibetan}


\bgroup\large
\begin{tabular}{llllllllllllllll l}
\toprule
	           &|0|	&|1|	&|2|	&|3|	&|4|	&|5|	&|6|	&|7|	&|8|	&|9|	&|A|	&|B|	&|C|	&|D|	&|E|	&|F|\\
\midrule
\texttt{U+0F0x}	&ༀ	&༁	&༂	&༃	&༄	&༅	&༆	&༇	&༈	&༉	&༊	&་	&༌  &	།	&༎	&༏\\
\midrule
\texttt{U+0F1x} &༐	&༑	&༒	&༓	&༔	&༕	&༖	&༗	&༘&	༙	&༚	&༛	&༜	&༝	&༞	&༟\\
\midrule
\texttt{U+0F2x} &༠	&༡	&༢	&༣	&༤	&༥	&༦	&༧	&༨	&༩	&༪	&༫	&༬	&༭	&༮	&༯\\
\midrule
\texttt{U+0F3x}	&༰ &༱	 &༲ &༳	&༴ &༵	&༶ & ༷	&༸&	༹	&༺&	༻	&༼&	༽	&༾	&༿\\
\midrule
\texttt{U+0F4x} &ཀ	&ཁ	&ག	&གྷ	&ང	&ཅ	&ཆ	&ཇ	&	&ཉ	&ཊ	&ཋ	&ཌ	&ཌྷ	&ཎ	&ཏ\\
\midrule
\texttt{U+0F5x}	 &ཐ	&ད	&དྷ	&ན	&པ	&ཕ	&བ	&བྷ	&མ	&ཙ	&ཚ	&ཛ	&ཛྷ	&ཝ	&ཞ	&ཟ\\
\midrule
\texttt{U+0F6x} &འ	&ཡ	&ར	&ལ	&ཤ	&ཥ	&ས	&ཧ	&ཨ	&ཀྵ	&ཪ	&ཫ	&ཬ	&&&\\
^^A\texttt{U+0F7x}&&	ཱ &	& &ི	ཱི&	ུ&	ཱུ&	ྲྀ&	ཷ&	ླྀ&	ཹ&	ེ&	ཻ&	ོ&	ཽ&	&ཾ	&ཿ\\
\midrule
\texttt{U+0F8x}&    ྀ   & 	ཱྀ&	ྂ&	&ྃ &	྄	&྅&	྆	&྇	ྈ&	ྉ&	ྊ&	ྋ&	ྌ&	ྍ&	ྎ&	ྏ\\
\midrule
\texttt{U+0F9x} &	ྐ&	ྑ   & 	ྒ &	ྒྷ &	ྔ &	ྕ &	ྖ &	ྗ &		ྙ &	ྚ &	ྛ &	ྜ &	ྜྷ &	ྞ &	ྟ\\
\texttt{U+0FAx} &	ྠ &	ྡ &	ྡྷ &	ྣ &	ྤ &	ྥ &		&ྦ	&ྦྷ	ྨ&	ྩ&	ྪ&	ྫ&	ྫྷ&	ྭ&	ྮ&	ྯ\\
\midrule
\texttt{U+0FBx} 
&	  ྰ 
&	
& ྱ  	 
&ྲ	
&ླ	
&ྴ
&	ྵ
&	ྶ
&	ྷ
&ྸ
&
&
&
&	
&྾	
&྿\\
\midrule
\texttt{U+0FCx}	 &࿀&	࿁&	࿂&	࿃&	࿄&	࿅&	&࿇	&࿈	&࿉	&࿊	&࿋	&࿌	&&	࿎	&࿏\\
\midrule
\texttt{U+0FDx}	&࿐	&࿑	&࿒	&࿓	&࿔	&࿕	&࿖	&࿗	&࿘	&࿙	&࿚	&&&&&\\
\midrule
\texttt{U+0FEx} &&&&&&&&&&&&&&&&\\
\midrule
\texttt{U+0FFx}  &&&&&&&&&&&&&&&&\\
\bottomrule
\end{tabular}
\egroup




\subsubsection{Fonts for Tibetan}

Fonts for Tibetan need to be downloaded one set of fonts are the \texttt{Qomolangma}. They come in different flavours, but they appear
to offer advantages as compared to the Tibetan Machine Uni.
\medskip


\newfontfamily\betsu{Qomolangma-Betsu.ttf}
\newfontfamily\drutsa{Qomolangma-Drutsa.ttf}
\newfontfamily\chuyig{Qomolangma-Chuyig.ttf}
\newfontfamily\tsumachu{Qomolangma-Tsumachu.ttf}
\newfontfamily\uchensutung{Qomolangma-UchenSutung.ttf}
\newfontfamily\uchensuring{Qomolangma-UchenSuring.ttf}
\newfontfamily\uchensarchen{Qomolangma-UchenSarchen.ttf}
\newfontfamily\uchensarchung{Qomolangma-UchenSarchung.ttf}
\newfontfamily\tsuring{Qomolangma-Tsuring.ttf}
\newfontfamily\TMU{TibMachUni.ttf}
\newfontfamily\himalaya{Microsoft Himalaya}
\uccoff

{
\centering

\renewcommand{\arraystretch}{1.5}

\begin{tabular}{lr}
\toprule
|Qomolangma-Betsu.ttf| & {\betsu  དབུ་མེད }\\
\midrule
|Qomolangma-Chuyig.ttf| &{\chuyig  དབུ་མེད}\\
\midrule
|Qomolangma-Drutsa.ttf| &{\drutsa  དབུ་མེད}\\
\midrule
|Qomolangma-Tsumachu.ttf|&{\tsumachu  དབུ་མེད}\\
\midrule
|Qomolangma-Tsuring.ttf| &{\tsuring  དབུ་མེད}\\
\midrule
|Qomolangma-UchenSarchen.ttf| &{\uchensarchen དབུ་མེད}\\
\midrule
|Qomolangma-UchenSarchung.ttf|&{\uchensarchung དབུ་མེད }\\
\midrule
|Qomolangma-UchenSuring.ttf|&{\uchensuring དབུ་མེད}\\
\midrule
|Qomolangma-UchenSutung.ttf|&{\uchensutung དབུ་མེད }\\
\midrule
|TibMachUni.ttf| &{\TMU དབུ་མེད }\\
\midrule
|Microsoft Himalaya| &{\himalaya དབུ་མེད ཽ}\\
\bottomrule
\end{tabular}

}
\bigskip

\bgroup
\LARGE\tsuring
\noindent༆ །ཨ་ཡིག་དཀར་མཛེས་ལས་འཁྲུངས་ཤེས་བློ  འི་\par
གཏེར༑ །ཕས་རྒོལ་ཝ་སྐྱེས་ཟིལ་གནོན་གདོང་ལྔ་བཞིན།།\par
ཆགས་ཐོགས་ཀུན་བྲལ་མཚུངས་མེད་འཇམ་དབྱངསམཐུས།།\par
མཧཱ་མཁས་པའི་གཙོ་བོ་ཉིད་འགྱུར་ཅིག། །མངྒལཾ༎\par
\egroup

\subsubsection{Tibetan numbers}
\cxset{language=tibetan, tibetan font = TibMachUni.ttf}

{
\obeylines
\small
TIBETAN DIGIT ZERO	༠
TIBETAN DIGIT ONE	༡	
TIBETAN DIGIT TWO	༢	
TIBETAN DIGIT THREE	༣	
TIBETAN DIGIT FOUR	༤	
TIBETAN DIGIT FIVE	༥	
TIBETAN DIGIT SIX	༦	
TIBETAN DIGIT SEVEN	༧	
TIBETAN DIGIT EIGHT	༨	
TIBETAN DIGIT NINE	༩	
TIBETAN DIGIT HALF ONE	\tibetan༪	
TIBETAN DIGIT HALF TWO	༫	
TIBETAN DIGIT HALF THREE	༬
TIBETAN DIGIT HALF FOUR ༭	
TIBETAN DIGIT HALF FIVE ༯	
TIBETAN DIGIT HALF SIX	 ༯	
TIBETAN DIGIT HALF SEVEN	༰	
TIBETAN DIGIT HALF EIGHT	༱	
TIBETAN DIGIT HALF NINE	༲	
TIBETAN DIGIT HALF ZERO	༳	
}


Tibetan numbers

The usage is not certain. By some interpretations, this has the value of 9.5. Used only in some traditional contexts, these appear as the last digit of a multidigit number, eg. ༤༬ represents 42.5. These are very rarely used, however, and other uses have been postulated.

\defaulttext





\section{Tamil}
\newfontfamily\tamil[Scale=1.1,Script=Tamil]{code2000.ttf}

\def\tamiltext#1{{\tamil#1}}

The Tamil script (\tamiltext{தமிழ் அரிச்சுவடி} tamiḻ ariccuvaṭi) is an abugida script that is used by the Tamil people in India, Sri Lanka, Malaysia and elsewhere, to write the Tamil language, as well as to write the liturgical language Sanskrit, using consonants and diacritics not represented in the Tamil alphabet.[1] Certain minority languages such as Saurashtra, Badaga, Irula, and Paniya are also written in the Tamil script

The Tamil script has 12 vowels (\tamiltext{உயிரெழுத்து} uyireḻuttu "soul-letters"), 18 consonants (\tamiltext{மெய்யெழுத்து} meyyeḻuttu "body-letters") and one character, the āytam \tamiltext{ஃ (ஆய்தம்)}, which is classified in Tamil grammar as being neither a consonant nor a vowel (\tamiltext{அலியெழுத்து} aliyeḻuttu "the hermaphrodite letter"), though often considered as part of the vowel set (\tamiltext{உயிரெழுத்துக்கள்} uyireḻuttukkaḷ "vowel class"). The script, however, is syllabic and not alphabetic.[3] The complete script, therefore, consists of the thirty-one letters in their independent form, and an additional 216 combinant letters representing a total 247 combinations (\tamiltext{உயிர்மெய்யெழுத்து} uyirmeyyeḻuttu) of a consonant and a vowel, a mute consonant, or a vowel alone. These combinant letters are formed by adding a vowel marker to the consonant. Some vowels require the basic shape of the consonant to be altered in a way that is specific to that vowel. Others are written by adding a vowel-specific suffix to the consonant, yet others a prefix, and finally some vowels require adding both a prefix and a suffix to the consonant. In every case the vowel marker is different from the standalone character for the vowel.
The Tamil script is written from left to right.

Tamil is a Unicode block containing characters for the Tamil, Badaga, and Saurashtra languages of Tamil Nadu India, Sri Lanka, Singapore, and Malaysia. In its original incarnation, the code points U+0B02..U+0BCD were a direct copy of the Tamil characters A2-ED from the 1988 ISCII standard. The Devanagari, Bengali, Gurmukhi, Gujarati, Oriya, Telugu, Kannada, and Malayalam blocks were similarly all based on their ISCII encodings.

\begin{scriptexample}[]{Tamil}
\unicodetable{tamil}{"0B80,"0B90,"0BA0,"0BB0,"0BC0,"0BE0,"0BF0}

\hfill  Typeset with \cmd{\tamil} and \texttt{code2000.ttf}
\end{scriptexample}

\subsection{Tamil Numbers and Numerals}

Originally, Tamils did not use zero, nor did they use positional digits (having separate 
symbols for the numbers 10, 100 and 1000). Symbols for the numbers are similar to 
other Tamil letters, with some minor changes. 

For example, the number 3782 is not written as \tamiltext{௩௭௮௨} as in modern usage. Instead it 
is written as \tamiltext{௩ ௲ ௭ ௱ ௮ ௰ ௨}. This would be read as they are written as 
Three Thousands, Seven Hundreds, Eight Tens, Two; or in Tamil as 
\tamiltext{௩௲௭௱௮௰௨ž}.\footnote{https://cloud.github.com/downloads/raaman/Tamil-Numeral/tamilnumbers.html}

\subsection{Dates}

Once the script is loaded the day, month and year can be loaded using the command  \cmd{\tamildate}, which returns the |\today| formatted as per custom Tamil. 

\begin{center}
\bgroup
\tamil
\begin{tabular}{lll}
day	 &month	&year	\\

௳	&௴	      &௵	\\

u	&mee	      &wa	\\
\egroup
\end{center}














\subsection{Kannada alphabet}

\newfontfamily\kannada[Scale=1.0,Script=Kannada]{Lohit-Kannada.ttf}

\def\kannadatext#1{{\kannada#1}}

The Kannada alphabet (\kannadatext{ಕನ್ನಡ ಲಿಪಿ}) is an abugida of the Brahmic family,[2] used primarily to write the Kannada language, one of the Dravidian languages of southern India. Several minor languages, such as Tulu, Konkani, Kodava, and Beary, also use alphabets based on the Kannada script.[3] The Kannada and Telugu scripts share high mutual intellegibility with each other, and are often considered to be regional variants of single script. Similarly, Goykanadi, a variant of Old Kannada, has been historically used to write Konkani in the state of Goa.[4]

\begin{scriptexample}[]{Kannada}
\centerline{\LARGE\kannadatext{ಙ	ಙ್ಕ	ಙ್ಖ	ಙ್ಗ	ಙ್ಘ	ಙ್ಙ	ಙ್ಚ	ಙ್ಛ	ಙ್ಜ	ಙ್ಝ	ಙ್ಞ	ಙ್ಟ	ಙ್ಠ	ಙ್ಡ	ಙ್ಢ}}
\end{scriptexample}

\medskip

The Kannada script (aksharamale or varnamale) is a phonemic abugida of forty-nine letters, and is written from left to right. The character set is almost identical to that of other Brahmic scripts. Consonantal letters imply an inherent vowel. Letters representing consonants are combined to form digraphs (ottaksharas) when there is no intervening vowel. Otherwise, each letter corresponds to a syllable.
The letters are classified into three categories: swara (vowels), vyanjana (consonants), and yogavaahaka (part vowel, part consonant).
The Kannada words for a letter of the script are akshara, akkara, and varna. Each letter has its own form (ākāra) and sound (shabda), providing the visible and audible representations, respectively. Kannada is written from left to right.[7]


\subsection{Osmanian Alphabet}

\bgroup
\newfontfamily\osmanian{code2001.ttf}
\osmanian
𐒚𐒁𐒖𐒄 𐒚𐒐 𐒚 𐒎𐒚𐒍𐒚𐒐 𐒑𐒚𐒒𐒠𐒚𐒐 𐒎𐒚𐒑𐒁𐒗 𐒚𐒁𐒖𐒄 𐒚𐒌𐒖𐒄 𐒚𐒁𐒖𐒄𐒖 𐒚
𐒌𐒜
\egroup



\cxset{steward,
  offsety=0cm,
  image={ethiopianbride.jpg},
  texti={An introduction to the use of font related commands. The chapter also gives a historical background to font selection using \tex and \latex. },
  textii={In this chapter we discuss keys that are available through the \texttt{phd} package and give a background as to how fonts are used
in \latex.
 },
 pagestyle = empty,
}




\cxset{steward,
  offsety=0cm,
  image={fellah-woman.jpg},
  texti={An introduction to the use of font related commands. The chapter also gives a historical background to font selection using \tex and \latex. },
  textii={In this chapter we discuss keys that are available through the \texttt{phd} package and give a background as to how fonts are used
in \latex.
 },
 pagestyle = empty
}

%  \cxset{steward,
  numbering=arabic,
  custom=stewart,
  offsety=0cm,
  image={asia.jpg},
  texti={An introduction to the use of font related commands. The chapter also gives a historical background to font selection using \tex and \latex. },
  textii={In this chapter we discuss keys that are available through the \texttt{phd} package and give a background as to how fonts are used
in \latex.
 },
 pagestyle = empty
}

\arial


\chapter{South Asian Scripts}

The scripts of South Asia share so many characteristics that a side by side comparison of a few often reveal structural similarities even in the 
modern letterforms.
\medskip

\begin{center}
\begin{tabular}{lll}
Devanagari. &Gujarati &Telugu\\
Bengali   &Oriya &Kannada\\
Gurmukhi &Tamil  &Malayalam\\
Sinhala &Kaithi  &Meetei Mayek\\
Tibetan &Saurashtra &Ol Chiki.\\
Lepcha  &Sharada &Sora Sompeng\\
Phags-pa &Takri &Kharoshthi\\
Limbu &Chakma & Brahmi\\
Syloti Nagri & &\\
\end{tabular}
\end{center}

The sections that follow describe the scripts briefly and the |phd| settings
to activate the relevant commands and load appropriate fonts. 

\section{Devanagari}
\parindent1em

Devanagari is part of the Brahmic family of scripts of India, Nepal, Tibet, and South-East Asia.[2] It is a descendant of the Gupta script, along with Siddham and Sharada.[2] Eastern variants of Gupta called nāgarī are first attested from the 7th century CE; from c. 1200 CE these gradually replaced Siddham, which survived as a vehicle for Tantric Buddhism in East Asia, and Sharada, which remained in parallel use in Kashmir. An early version of Devanagari is visible in the Kutila inscription of Bareilly dated to Vikram Samvat 1049 (i.e. 992 CE), which demonstrates the emergence of the horizontal bar to group letters belonging to a word.[3]

Sanskrit nāgarī is the feminine of nāgara "relating or belonging to a town or city". It is feminine from its original phrasing with lipi ("script") as nāgarī lipi "script relating to a city", that is, probably from its having originated in some city.[4]

The use of the name devanāgarī is relatively recent, and the older term nāgarī is still common.[2] The rapid spread of the term devanāgarī may be related to the almost exclusive use of this script to publish Sanskrit texts in print since the 1870s.[2]

On Windows use \texttt{Arial Unicode MS}. 
\medskip

\newfontfamily\devanagari[Script=Devanagari,Scale=1.5]{Arial Unicode MS}

\begin{scriptexample}[]{Devanagari}
{\begin{center}\parindent0pt\devanagari

ंःअआइईउऊऋऌऍऎएऐऑऒओऔऔँ \par 

ी	ु	ू	ृ	ॄ	ॅ	ॆ	े	ै	ॉ	ॊ	ो	ौ	्	\par

\bigskip		
\begin{tabular}{lll lll lll l}
०	&१	&२	&३	&४	&५	&६	&७	&८	&९\\
0	&1	&2	&3	&4	&5	&6	&7	&8	&9\\
\end{tabular}
\end{center}	
}
\end{scriptexample}


On Linux \texttt{Lohit} is a font family designed to cover Indic scripts and released by Red Hat. The Lohit fonts currently cover 11 languages: Assamese, Bengali, Gujarati, Hindi, Kannada, Malayalam, Marathi, Oriya, Punjabi, Tamil, Telugu.[1] The fonts were supplied by Modular Infotech and licensed under the GPL. In September 2011, they were retroactively relicensed under the OFL.[2] The Lohit fonts are used as web fonts by some Wikimedia Foundation sites, like Wikipedia, since March 2012.The font currently support 21 Indian languages. 

\newfontfamily\devanagarilohit[Script=Devanagari,Scale=1.5]{Lohit-Devanagari.ttf}

\begin{scriptexample}[]{Devanagari}
\begin{center}\parindent0pt\devanagarilohit

ंःअआइईउऊऋऌऍऎएऐऑऒओऔऔँ \par 

ी	ु	ू	ृ	ॄ	ॅ	ॆ	े	ै	ॉ	ॊ	ो	ौ	्	\par

\bigskip		
\begin{tabular}{lll lll lll l}
०	&१	&२	&३	&४	&५	&६	&७	&८	&९\\
0	&1	&2	&3	&4	&5	&6	&7	&8	&9\\
\end{tabular}
\end{center}
\end{scriptexample}

\subsubsection{Punctuation} 
The end of a sentence or half-verse may be marked with a dot known as a pūrna virām or a vertical line danda: \textbar. The end of a full verse may be marked with two vertical lines: \textbar\textbar. A comma, or alpa virām, is used to denote a natural pause in speech. With expansion of English speakers in India, the full stop is also sometimes used.

\subsection{LaTeX support}

\latex2e support can be found in the \pkgname{sanskrit}. The package contains the font files and pre-processor for printing Sanskrit
text in both devanāgarī and transliterated Roman with diacritics. Another package that can be used with \XeTeX\ is support \pkgname{devnag}.  This was originally developed by Frans Velthuis for the University of Groningen, The Netherlands, and it was the first system to provide
support for the script for \tex. The package was  extended by Anshuman Pandey. The package provides both fonts as well as tranliteration macros.


\subsection{Gujarati}


Gujarati has its own writing system, distinct but related to several other Indian languages' writing systems, such as the one used to write Hindi. Strictly speaking, the Gujarati writing system is what is called an \emph{abugida} (and not an \textit{alphabet}), because the consonant characters all contain an inherent vowel, and other vowels are written as accents added on to the consonant characters. There are also symbols for stand-alone vowels.

The Gujarati script ({\gujarati{ગુજરાતી લિપિ }} Gujǎrātī Lipi), which like all Nāgarī writing systems is strictly speaking an abugida rather than an alphabet, is used to write the Gujarati and Kutchi languages. It is a variant of Devanāgarī script differentiated by the loss of the characteristic horizontal line running above the letters and by a small number of modifications in the remaining characters.
With a few additional characters, added for this purpose, the Gujarati script is also often used to write Sanskrit and Hindi.
Gujarati numerical digits are also different from their Devanagari counterparts.
\medskip

\bgroup
\newfontfamily\gujaratilohit[Script=Gujarati,Scale=1.5]{Lohit-Gujarati.ttf}
\gujarati

\centering

English/Hindi/Gujarati Alphabets

\begin{tabular}{lllllllllllllllllllll}
A &B &bh &C &ch &chh &D &dh &E &F &G &gh &H &I &J &K &kh &L &M &N &O\\

अ &ब &भ &क &च &छ &ड/द &ध/ढ़ &इ &फ &ग &घ &ह &ई &ज &क &ख &ल &म &न/ण &ऑ\\

અ &બ &ભ &ક &ચ &છ &ડ/દ &ધ /ઢ &ઇ &ફ &ગ &ઘ &હ &ઈ &જ &ક &ખ &લ &મ &ન/ણ &ઓ\\

\end{tabular}
\egroup

\medskip

Gujarati has its own set of numeric signs (placed alongside their Hindu-Arabic [or Indo-Arabic] counterparts in the tables below), they are employed in much the same way as English;  that is to say, they are put together in the same manner in order to express larger numbers. It is quite possible to simply substitute the Gujarati numerals for the Hindu-Arabic ones.

The Gujarati words for 1-10 are as follows:
\medskip

\bgroup
\begin{center}
\gujarati
\begin{tabular}{ccl}
Arabic & Gujarati &Name\\
Numeral &Numeral  &\\
0	&૦	&mīṇḍuṃ or shunya\\
1	&૧	&ekaṛo or ek\\
2	&૨	&bagaṛo or bay\\
3	&૩	&tragaṛo or tran\\
4	&૪	&chogaṛo or chaar\\
5	&૫	&pāchaṛo or paanch\\
6	&૬	&chagaṛo or chah\\
7	&૭	&sātaṛo or sāt\\
8	&૮	&āṭhaṛo or āanth\\
9	&૯	&navaṛo or nav\\
10 &૧૦ &દસ das\\

\end{tabular}
\end{center}
\egroup

\subsection{Bengali}

There are two Windows fonts that can be used with Windows \textit{Shonar Bangla} and \textit{Vrinda}. For open source fonts one can use, \textit{code2000}.
\bigskip

\bgroup
\newfontfamily\bengali[Script=Bengali,Scale=4]{Shonar Bangla}


\bengali
\centering

  অ  আ ই  ঈ  উ  ঊ  ঋ  এ  ঐ\par

\fontspec[Script=Bengali,Scale=3.2]{Vrinda}

\centering

  অ  আ ই  ঈ  উ  ঊ  ঋ  এ  ঐ\par


\fontspec[Script=Bengali,Scale=3.2]{code2000.ttf}

\centering

  অ  আ ই  ঈ  উ  ঊ  ঋ  এ  ঐ\par

\captionof{table}{The consonant{\protect\bengal{} ক (kô)} along with the diacritic form of the vowels {\protect\bengal{} অ, আ, ই, ঈ, উ, ঊ, ঋ, এ, ঐ, ও and ঔ} \textit{from Wikipedia}.}
\egroup

\subsection{Saurashtra}

\newfontfamily\saurashtra{code2000.ttf}

Saurashtra or Sourashtra or {\saurashtra ꢱꣃꢬꢵꢰ꣄ꢜ꣄ꢬꢵ} or Palkar or Patkar (Sanskrit: सौराष्ट्र, Tamil: சௌராட்டிரம்) is an Indo-Aryan language[3] spoken by the Saurashtrian community native to Gujarat, who migrated and settled in Southern India. Madurai in Tamil Nadu has the highest number of people belonging to this community and also remains as their cultural center.

The language is largely only in spoken form even though the language has its own script. The lack of schools teaching Saurashtra script and the language is often cited as a reason for the very few number of people who actually know to read and write in Saurashtra script. Latin, Devanagari or Tamil script is used as alternative for Saurashtra Script by many Saurashtrians.

Census of India places the language under Gujarati. Official figures show the number of speakers as 185,420 (2001 census).[4]



\begin{scriptexample}[]{Saurashtra}
\bgroup
\saurashtra

ꢮꢶꢯ꣄ꢮ ꢱꣃꢬꢵꢰ꣄ꢜ꣄ꢬꢪ꣄ ꢦꢡ꣄ꢬꢶꢒꢾ ꢱꢵꢡ꣄ꢡꢒꢸ ꢂꢮꢬꢾ
ꢮꣁꢭꢱ꣄ꢢꢵꢥꢪꢸꢒ꣄(ꣀꢵꢮꢾꢔꢹ ꢂꢮ꣄ꢬꢶꢫꣁ


\arial

Text: Vishwa Sourashtram \url{http://www.sourashtra.info/ghEr.htm}
\egroup
\end{scriptexample}

\subsection{Ol Chiki script}

The Ol Chiki script, also known as Ol Cemetʼ (Santali: ol 'writing', cemet' 'learning'), Ol Ciki, Ol, and sometimes as the Santali alphabet, was created in 1925 by Raghunath Murmu for the Santali language.

Previously, Santali had been written with the Latin alphabet. But because Santali is not an Indo-Aryan language (like most other languages in the south of India), Indic scripts did not have letters for all of Santali's phonemes, especially its stop consonants and vowels, which made writing the language accurately in an unmodified Indic script difficult. The detailed analysis was given by Dr. Byomkes Chakrabarti in his 'Comparative Study of Santali and Bengali'. Missionaries (first of all Paul Olaf Bodding, a Norwegian) brought the Latin script, which is better at representing Santali stops, phonemes and nasal sounds with the use of diacritical marks and accents. Unlike most Indic scripts, which are derived from Brahmi, Ol Chiki is not an abugida, with vowels given equal representation with consonants. Additionally, it was designed specifically for the language, but one letter could not be assigned to each phoneme because the sixth vowel in Ol Chiki is still problematic.
Ol Chiki has 30 letters, the forms of which are intended to evoke natural shapes. Linguist Norman Zide said "The shapes of the letters are not arbitrary, but reflect the names for the letters, which are words, usually the names of objects or actions representing conventionalized form in the pictorial shape of the characters."[1] It is written from left to right.

\newfontfamily\olchiki{code2000.ttf}

\begin{scriptexample}[]{olchiki}
\bgroup
\olchiki
\obeylines

U+1C5x 	᱐	᱑	᱒	᱓	᱔	᱕	᱖	᱗	᱘	᱙	ᱚ	ᱛ	ᱜ	ᱝ	ᱞ	ᱟ
U+1C6x	   ᱠ	ᱡ	ᱢ	ᱣ	ᱤ	ᱥ	ᱦ	ᱧ	ᱨ	ᱩ	ᱪ	ᱫ	ᱬ	ᱭ	ᱮ	ᱯ
U+1C7x  	ᱰ	ᱱ	ᱲ	ᱳ	ᱴ	ᱵ	ᱶ	ᱷ	ᱸ	ᱹ	ᱺ	ᱻ	ᱼ	ᱽ	᱾	᱿
\egroup
\end{scriptexample}

\subsection{Lepcha}
\newfontfamily\lepcha{Mingzat-R.ttf}

The Lepcha script, or Róng script is an abugida used by the Lepcha people to write the Lepcha language. Unusually for an abugida, syllable-final consonants are written as diacritics.

The Mingzat font is still under development by SIL so I am not too sure if the rendering is correct\footnote{\url{http://scripts.sil.org/cms/scripts/page.php?site_id=nrsi&id=Mingzat}}.

\begin{scriptexample}[]{Lepcha}
\bgroup
\lepcha
\obeylines
 	    0	1	2	3	4	5	6	7	8	9	A	B	C	D	E	F
U+1C0x	 ᰀ	ᰁ	ᰂ	ᰃ	ᰄ	ᰅ	ᰆ	ᰇ	ᰈ	ᰉ	ᰊ	ᰋ	ᰌ	ᰍ	ᰎ	ᰏ
U+1C1x	 ᰐ	ᰑ	ᰒ	ᰓ	ᰔ	ᰕ	ᰖ	ᰗ	ᰘ	ᰙ	ᰚ	ᰛ	ᰜ	ᰝ	ᰞ	ᰟ
U+1C2x	 ᰠ	ᰡ	ᰢ	ᰣ	ᰤ	ᰥ	ᰦ	ᰧ	ᰨ	ᰩ	ᰪ	ᰫ	ᰬ	ᰭ	ᰮ	ᰯ
U+1C3x	 ᰰ	ᰱ	ᰲ	ᰳ	ᰴ	ᰵ	ᰶ	᰷	x	x	x	᰻	᰼	᰽	᰾	᰿
U+1C4x	 ᱀	᱁	᱂	᱃	᱄	᱅	᱆	᱇	᱈	᱉	x	x	x	ᱍ	ᱎ	ᱏ

\egroup
\end{scriptexample}

\subsection{Sharada}

The Śāradā, or Sharada, script (शारदा) is an abugida writing system of the Brahmic family of scripts, developed around the 8th century. It was used for writing Sanskrit and Kashmiri. The Gurmukhī script was developed from Śāradā. Originally more widespread, its use became later restricted to Kashmir, and it is now rarely used except by the Kashmiri Pandit community for ceremonial purposes. Śāradā is another name for Saraswati, the goddess of learning.
Śāradā script was added to the Unicode Standard in January, 2012 with the release of version 6.1.

The Unicode block for Śāradā script, called Sharada, is U+11180–U+111DF: Unable to locate font in unicode.


\subsection{Sora Sompeng}

Sorang Sompeng script is used to write in Sora, a Munda language with 300,000 speakers in India. The script was created by Mangei Gomango in 1936 and is used in religious contexts.[1] He was familiar with Oriya, Telugu and English, so the parent systems of the script are Brahmi and Latin.[2]
The Sora language is also written in the Latin alphabet and the Telugu script.

Sorang Sompeng script was added to the Unicode Standard in January, 2012 with the release of version 6.1. Nirmala UI.ttf (Windows 8.1)



\unicodetable{arial}{"110D0,"110E0,"110F0}
 	
This did not work with Windows 7, and the experiment failed. 

\subsection{Phags-pa}

The 'Phags-pa script,[1], (Mongolian: дөрвөлжин үсэг "Square script") was an alphabet designed by the Tibetan monk and vice-king Drogön Chögyal Phagpa for the Mongol Yuan emperor Kublai Khan as a unified script for the literary languages of the Yuan. Widespread use was limited to about a hundred years during the Yuan Dynasty, and it fell out of use with the advent of the Ming dynasty. The documentation of its use provides clues about the changes in the varieties of Chinese, the Tibetic languages, Mongolian and other neighboring languages during the Yuan era.

\newfontfamily\phagspa{code2000.ttf}

\begin{scriptexample}[]{Phags-pa}
\bgroup
\obeylines
\phagspa

 	0	1	2	3	4	5	6	7	8	9	A	B	C	D	E	F
U+A84x	ꡀ	ꡁ	ꡂ	ꡃ	ꡄ	ꡅ	ꡆ	ꡇ	ꡈ	ꡉ	ꡊ	ꡋ	ꡌ	ꡍ	ꡎ	ꡏ
U+A85x	ꡐ	ꡑ	ꡒ	ꡓ	ꡔ	ꡕ	ꡖ	ꡗ	ꡘ	ꡙ	ꡚ	ꡛ	ꡜ	ꡝ	ꡞ	ꡟ
U+A86x	ꡠ	ꡡ	ꡢ	ꡣ	ꡤ	ꡥ	ꡦ	ꡧ	ꡨ	ꡩ	ꡪ	ꡫ	ꡬ	ꡭ	ꡮ	ꡯ
U+A87x	ꡰ	ꡱ	ꡲ	ꡳ	꡴	꡵	꡶	


ꡏꡟ ꡋꡞ ꡏꡟ ꡋꡞ ꡏ ꡜꡖ ꡏꡟ ꡋꡞ ꡓꡞ ꡏꡟ
ꡈꡋ ꡋꡋ ꡓꡘ ꡈ ꡭ ꡏ ꡏ ꡝ ꡭꡟꡘ ꡓꡋ ꡮꡟꡊ
\egroup
\bgroup
\raggedright

\setcounter{glyphcount}{"A840}

\topline
\phagspa
\newcount\n
\n="A840

\def\htable{^^A
  \def\fm##1{\makebox[2em]##1}^^A
  U+A840\fm 0\fm1\fm2\fm3\fm4\fm5\fm 6\fm 7\fm 8\fm	9\fm A\fm B\fm C\fm D\fm E\fm F}

\htable\par
U+A840^^A 
\loop^^A
  \makebox[2em]{\char\n }^^A   
   \advance\n by1 ^^A
   \ifnum\n<"A850^^A
\repeat
\par U+A850^^A
\loop^^A
  \makebox[2em]{\char\n }^^A   
   \advance\n by1 ^^A
  \ifnum\n<"A860^^A
\repeat
\par U+A860^^A
\loop^^A
  \makebox[2em]{\char\n }^^A   
   \advance\n by1 ^^A
  \ifnum\n<"A870^^A
\repeat
\par U+A870^^A
\loop^^A
  \makebox[2em]{\char\n }^^A   
   \advance\n by1 ^^A
  \ifnum\n<"A878^^A
\repeat

\bottomline

\arial
\hfill Typeset with \texttt{code2000.ttf} and \cmd{\phagspa}

Text: \href{http://babelstone.blogspot.com/2006/12/phags-pa-fonts-1-babelstone-phags-pa.html}{babelstone}
\egroup
\end{scriptexample}

Phags-pa is a historical script related to Tibetan that was created as the national script of
the Mongol empire. Even though Phags-pa was used mostly in Eastern and Central Asia for
writing text in the Mongolian and Chinese languages, it is discussed in this chapter because
of its close historical connection to the Tibetan script. The script has very limited modern use. It bears similarity to Tibetan and has no case distinctions. It is written vertically in columns running for left to right, like Mongolian. Units are often composed of several syllables and sometimes are separated by whitespace.


\subsection{Syloti Nagri}
\index{languages>Sylheti Nagari}
Sylheti Nagari or Syloti Nagri (Silôṭi Nagôri) is the original script used for writing the Sylheti language. It is an almost extinct script, this is because the Sylheti Language itself was reduced to only dialect status after Bangladesh gained independence and because it did not make sense for a dialect to have its own script, its use was heavily discouraged. The government of the newly formed Bangladesh did so to promote a greater "Bengali" identity. This led to the informal adoption of the Eastern Nagari script also used for Bengali and Assamese. It is also known as Jalalabadi Nagri, Mosolmani Nagri, Ful Nagri etc.

\newfontfamily\syloti{NotoSansSylotiNagri-Regular.ttf}
\newfontfamily\damase{damase_v.2.ttf}
\bgroup
\damase
\obeylines
	0	1	2	3	4	5	6	7	8	9	A	B	C	D	E	F
U+A80x	ꠀ	ꠁ	ꠂ	ꠃ	ꠄ	ꠅ	꠆	ꠇ	ꠈ	ꠉ	ꠊ	ꠋ	ꠌ	ꠍ	ꠎ	ꠏ
U+A81x	ꠐ	ꠑ	ꠒ	ꠓ	ꠔ	ꠕ	ꠖ	ꠗ	ꠘ	ꠙ	ꠚ	ꠛ	ꠜ	ꠝ	ꠞ	ꠟ
U+A82x	ꠠ	ꠡ	ꠢ	ꠣ	ꠤ	ꠥ	ꠦ	ꠧ	꠨	꠩	꠪	꠫
\egroup

\subsection{Chakma}

\newfontfamily\chakma{RibengUni.ttf}

\bgroup
\chakma
𑄇𑄳𑄇 Kkā = 𑄇 Kā + 𑄳 VIRAMA + 𑄇 Kā
𑄇𑄳𑄑 Ktā = 𑄇 Kā + 𑄳 VIRAMA + 𑄑 Tā
𑄇𑄳𑄖 Ktā = 𑄇 Kā + 𑄳 VIRAMA + 𑄖 Tā
𑄇𑄳𑄟 Kmā = 𑄇 Kā + 𑄳 VIRAMA + 𑄟 Mā
𑄇𑄳𑄌 Kcā = 𑄇 Kā + 𑄳 VIRAMA + 𑄌 Cā
𑄋𑄳𑄇 ńkā = 𑄋 ńā + 𑄳 VIRAMA + 𑄇 Kā
𑄋𑄳𑄉 ńkā = 𑄋 ńā + 𑄳 VIRAMA + 𑄉 Gā
𑄌𑄳𑄌 ccā = 𑄌 cā + 𑄳 VIRAMA + 𑄌 Cā

\egroup

\subsection{Limbu}

The Limbu script is used to write the Limbu language. The Limbu script is an abugida derived from the Tibetan script. Limbu is a Tibeto-Burman language spoken mainly in Nepal,[3] significant communities in Bhutan, Sikkim, Darjeeling district, India by the Limbu community. Virtually all Limbus are bilingual in Nepali.

\newfontfamily\limbu{code2000.ttf}
\bgroup
\obeylines
\limbu
0	1	2	3	4	5	6	7	8	9	A	B	C	D	E	F
U+190x	ᤀ	ᤁ	ᤂ	ᤃ	ᤄ	ᤅ	ᤆ	ᤇ	ᤈ	ᤉ	ᤊ	ᤋ	ᤌ	ᤍ	ᤎ	ᤏ
U+191x	ᤐ	ᤑ	ᤒ	ᤓ	ᤔ	ᤕ	ᤖ	ᤗ	ᤘ	ᤙ	ᤚ	ᤛ	ᤜ	ᤝ	ᤞ	
U+192x	ᤠ	ᤡ	ᤢ	ᤣ	ᤤ	ᤥ	ᤦ	ᤧ	ᤨ	ᤩ	ᤪ	ᤫ				
U+193x	ᤰ	ᤱ	ᤲ	ᤳ	ᤴ	ᤵ	ᤶ	ᤷ	ᤸ	᤹	᤺	᤻				
U+194x	᥀				᥄	᥅	᥆	᥇	᥈	᥉	᥊	᥋	᥌	᥍	᥎	᥏
\egroup

\subsection{Brahmi}



Brāhmī is the modern name given to one of the oldest writing systems used in the Indian subcontinent and in Central Asia during the final centuries BCE and the early centuries CE. Like its contemporary, Kharoṣṭhī, which was used in what is now Afghanistan and Western Pakistan, Brahmi (native to north and central India) was an \emph{abugida}.

The best-known Brahmi inscriptions are the rock-cut edicts of Ashoka in north-central India, dated to 250–232 BCE. The script was deciphered in 1837 by James Prinsep, an archaeologist, philologist, and official of the East India Company.[1] The origin of the script is still much debated, with current Western academic opinion generally agreeing (with some exceptions) that Brahmi was derived from or at least influenced by one or more contemporary Semitic scripts, but a current of opinion in India favors the idea that it is connected to the much older and as-yet undeciphered Indus script

\subsection{Unicode [U+11000-U+1107F]}


\newfontfamily\brahmi{code2000.ttf}

\begin{scriptexample}[]{Brahmi}
\bgroup
\raggedleft
\brahmi

         
   

\arial
\hfill Text: Asokan Edict typeset with \texttt{NotoSansBrahmi-Regular.ttf} 
\egroup
\end{scriptexample}

%  %% Check on why fonttable gives problems in Index
\parindent1em
\chapter{Symbols}

\section{Introduction}
\label{ch:comprehensivesymbols}

The \pkgname{phd} package, preloads a number of packages, to provide as
many symbols as possible irrespective of the \tex engine used. Many of these
symbols can easily be replaced by the use of \pkgname{fontspec} and if
the package is set to unicode math with the use of and suitable fonts Open Type Fonts. What follows has been largely copied from \emph{The Comprehensive List of \latexe Symbols}, which has been the authoritative publication, using almost all available symbols. 
The publication lists many symbols which I have dropped due to having exceeded the number of math alphabets allowed by \tex. 


With the newer engines \xetex \luatex you can now use any symbol you can imagine, but there is still room and advantages for using commands. It is at least for me faster than looking up a symbol's unicode character or trying it out with a screen keyboard (unless of course you are using a foreign language keyboard). 

The sections that follow describe the commands and packages that are
available, by simply including the \pkgname{phd} package. Most of the conflicts have been resolved and I am hoping that in the next version we will add some more symbols. 
Currently these are over 1500 as commands and in excess of 60,000 unicode glyphs, provided you have access to the fonts.\footnote{This document has been compiled using \luatex.}
  

\subsection{Reserved Symbols}
\tex has a number of symbols that need to be escaped, as they have 
special meanings during processing see Table~\vref{special-escapable} and also Chapter~\vref{ch:characters}.


\begin{symtable}{\latexe{} Escapable ``Special'' Characters}
\index{special characters=``special'' characters}
\index{escapable characters}
\index{underline}
\label{special-escapable}
\begin{tabular}{*6{ll@{\qqquad}}ll}
\K\$   & \K\%   & \K\_$\,^*$  & \Kp\}  & \K\&   & \K\#   & \Kp\{   \\
\end{tabular}
\end{symtable}

The \latexe kernel command \refCom{@sanitize} changes the catcode of these characters so they can be included in commands such as |\index|. In text just escape them with a (\textbackslash).


\begin{longsymtable}{Predefined \latexe{} Text-mode Commands}
\index{inequalities}
\index{tilde}
\index{underline}
\index{copyright}
\idxboth{dot}{symbols}
\index{dots (ellipses)} \index{ellipses (dots)}
\idxboth{legal}{symbols}
\label{text-predef}
\begin{longtable}{lll@{\qqquad}lll}
\indexTextcomp\textasciicircum$^*$    					& \indexTextcomp\textless                             \\
\indexTextcomp\textasciitilde$^*$     						& \indexTextcomp[\ltextordfeminine]\textordfeminine   \\
\indexTextcomp\textasteriskcentered   					& \indexTextcomp[\ltextordmasculine]\textordmasculine \\
\indexTextcomp{\textbackslash}          				    & \indexTextcomp\textparagraph$^\dag$                 \\
texbar                                              & \indexTextcomp\textperiodcentered                   \\
\indexTextcomp{textbraceleft}           $^\dag$   & \indexTextcomp\textquestiondown                     \\
\indexTextcomp\textbraceright$^\dag$  & \indexTextcomp\textquotedblleft                     \\
\indexTextcomp\textbullet             & \indexTextcomp\textquotedblright                    \\
\indexTextcomp[\ltextcopyright]\textcopyright$^\dag$
                          & \indexTextcomp\textquoteleft                        \\
\indexTextcomp\textdagger$^\dag$      & \indexTextcomp\textquoteright                       \\
\indexTextcomp\textdaggerdbl$^\dag$   & \indexTextcomp[\ltextregistered]\textregistered     \\
\indexTextcomp\textdollar$^\dag$      & \indexTextcomp\textsection$^\dag$                   \\
\indexTextcomp\textellipsis$^\dag$    & \indexTextcomp\textsterling$^\dag$                  \\
\indexTextcomp\textemdash             & \indexTextcomp[\ltexttrademark]\texttrademark       \\
\indexTextcomp\textendash             & \indexTextcomp\textunderscore$^\dag$                \\
\indexTextcomp\textexclamdown         & \indexTextcomp\textvisiblespace                     \\
\indexTextcomp\textgreater                                                      \\
\end{longtable}

\bigskip
\twosymbolmessage

\bigskip
\begin{tablenote}[*]
  \docAuxCommand{^} and
%  \cmdI[\string\~{}]{\~{}}\verb|{}| can be used instead of
  \docAuxCommand{textasciicircum} and \docAuxCommand{textasciitilde}.  See the
  discussion of ``\texttt{\textasciitilde}'' \vpageref[below]{page:tildes}.
\end{tablenote}

\bigskip
\usetextmathmessage[\dag]
\end{longsymtable}



\begin{symtable}{\latexe{} Commands Defined to Work in Both Math and Text Mode}
\index{dots (ellipses)} \index{ellipses (dots)}
\index{copyright}
\idxboth{legal}{symbols}
\label{math-text}
\begin{tabular}{*3{lll@{\qqquad}}lll}
\indexTextcomp\$ & \indexTextcomp\_              & \indexTextcomp\ddag    & \Vp\{ \\
\indexTextcomp\P & \indexTextcomp[\ltextcopyright]\copyright
                         & \indexTextcomp\dots    & \Vp\} \\
 & \indexTextcomp\dag            & \indexTextcomp\pounds          \\%V\S removed
\end{tabular}

\bigskip
\twosymbolmessage
\end{symtable}

\begin{symtable}{AMS Commands Defined to Work in Both Math and Text Mode}
\index{check marks}
\label{ams-math-text}
\begin{tabular}{*2{ll@{\qquad}}ll}
\X\checkmark & \X\circledR & \X\maltese
\end{tabular}
\end{symtable}


\begin{symtable}{Non-ASCII Letters (Excluding Accented Letters)}
\index{letters>non-ASCII} %\K\l to fix
\index{ASCII}
\label{non-ascii}
\begin{tabular}{*4{ll@{\qqquad}}ll}
\K\aa      & \Ks\DH     & \Ks\L      & \K\o       & \K\ss                   \\
\K\AA      & \Ks\dh     & &          & \K\O       & \K\SS                   \\
\K\AE      & \Ks\DJ     & \Ks\NG     & \K\OE      & \Ks\TH                  \\
\K\ae      & \Ks\dj     & \Ks\ng     & \K\oe      & \Ks\th                  \\
\end{tabular}

\bigskip
\begin{tablenote}[*]
  Not available in the OT1 \fntenc[OT1].  Use the \pkgname{fontenc}
  package to select an alternate \fntenc[T1], such as T1.
\end{tablenote}
\end{symtable}

\section{Punctuation marks}

\begin{longsymtable}{Punctuation Marks Not Found in OT1}
\index{punctuation}
\label{punc-no-OT1}
\begin{longtable}{*8l}
\Kt\guillemotleft  & \Kt\guilsinglleft & \Kt\quotedblbase & \Kt\textquotedbl \\
\Kt\guillemotright & \Kt\guilsinglright & \Kt\quotesinglbase \\
\end{longtable}
\end{longsymtable}


\begin{longsymtable}[PI]{\PI\ Decorative Punctuation Marks}
\index{punctuation}
\label{pi-punctuation}
\begin{longtable}{*5{ll}}
\indexDing{123} & \indexDing{125} & \indexDing{161} & \indexDing{163} \\
\indexDing{124} & \indexDing{126} & \indexDing{162} \\
\end{longtable}
\medskip
\begin{tablenote}
  To get these symbols, use the \pkgname{fontenc} package to select an
  alternate \fntenc[T1], such as~T1.
\end{tablenote}

\end{longsymtable}

\section{Accents}
\begin{symtable}{Text-mode Accents}
\index{accents}
\index{accents>acute=acute (\blackacchack\')}   
\index{accents>arc=arc (\blackacchack\newtie)}
\index{accents>breve=breve (\blackacchack\u)}   
\index{accents>caron=caron (\blackacchack\v)} 
\index{accents>cedilla=cedilla (\blackacc\c)} 
\index{accents>circumflex=circumflex (\blackacchack\^)}  
\index{accents>diaeresis=di\ae{}resis (\blackacchack\")}  
\index{accents>dot=dot (\blackacchack\. or \blackacc\d)} 
\index{accents>double acute=double acute (\blackacchack\H)}  
\index{accents>grave=grave (\blackacchack\`)}  
\index{accents>ogonek=ogonek (\encone{\blackacc\k})} 
\index{accents>ring=ring (\blackacchack\r)} 
\label{text-accents}
\begin{tabular}{*3{ll@{\qqquad}}ll}
\Q\"                                & \Q\`         & \Q\d         & \Q\r        \\
\Q\'                                & \QivBAR\ddag & \Qiv\G\ddag  & \Q\t        \\
\Q\.                                & \Q\~         & \Qv\h\S      &        \\ %Q/u removed
\Qe[\magicequal][\magicequalname]\= & \Q\b         & \Q\H         & \Qiv\U\ddag \\
\Q\^                                & \Q\c         & \Qt\k$^\dag$ & \Q\v        \\
\end{tabular}
\par\medskip
\begin{tabular}{ll@{\qqquad}ll}
\Q\newtie$^*$ & \Qc\textcircled
\end{tabular}

\bigskip
\begin{tablenote}[*]
  Requires the \TC\ package.
\end{tablenote}

\medskip
\begin{tablenote}[\dag]
  Not available in the OT1 \fntenc[OT1].  Use the \pkgname{fontenc}
  package to select an alternate \fntenc[T1], such as T1.
\end{tablenote}

\medskip
\begin{tablenote}[\ddag]
  Requires the T4 \fntenc[T4], provided by the \FC\ package.
\end{tablenote}

\medskip
\begin{tablenote}[\S]
  Requires the T5 \fntenc[T5], provided by the \VIET\ package.
\end{tablenote}

\bigskip
\begin{tablenote}
  \index{dotless i=dotless $i~(\imath)$>text mode} \index{dotless
  j=dotless $j~(\jmath)$>text mode} Also note the existence of
  \docAuxCommand{i} and \docAuxCommand{j}, which produce dotless versions of ``i'' and
  ``j'' (viz., ``\i'' and ``\j'').  These are useful when the accent
  is supposed to replace the dot in encodings that need to
  composite\index{composited accents} (i.e.,~combine) letters and
  accents.  For example, ``\verb|na\"{\i}ve|'' always produces a
  correct ``na\"{\i}ve'', while ``\verb|na\"{i}ve|'' yields the rather
  odd-looking na\"{i}ve
  \makeatletter
  ``na\add@accent{127}{i}ve''\index{i=\add@accent{127}{i}}
  \makeatother
  when using the OT1 \fntenc[OT1] and older versions of \latex.  Font
  encodings other than OT1 and newer versions of \latex properly
  typeset ``\verb|na\"{i}ve|'' as ``na\"{\i}ve''.
\end{tablenote}
\end{symtable}

\section{Diacritics and Accents}

Again the most convenient way to get diagritics is to use the
\pkgname{textcomp}. The \TC\ package defines all of the above as ordinary characters,
  and not as accents. Of course with Unicode and True Type fonts, the worlds accents and
  diagritics, make these tables pale in comparison. 

\begin{longsymtable}{\TC\ Diacritics}
\index{accents}
\index{accents>acute=acute (\blackacchack\')}   
\index{accents>breve=breve (\blackacchack\u)}  
\index{accents>caron=caron (\blackacchack\v)}  
\index{accents>diaeresis=di\ae{}resis (\blackacchack\")} 
\index{accents>double acute=double acute (\blackacchack\H)}
\index{accents>grave=grave (\blackacchack\`)}  
\index{diacritics}
  
\label{tc-accent-chars}
\begin{longtable}{*3{ll}}
\K\textacutedbl      & \K\textasciicaron    & \K\textasciimacron \\
\K\textasciiacute    & \K\textasciidieresis & \K\textgravedbl    \\
\K\textasciibreve    & \K\textasciigrave                         \\
\end{longtable}
\end{longsymtable}


\begin{longsymtable}{\TC\ Currency Symbols}
\idxboth{currency}{symbols}
\idxboth{monetary}{symbols}
\index{euro signs}
\label{tc-currency}
\begin{longtable}{*4{ll}}
\K\textbaht          & \K\textdollar$^*$     & \K\textguarani  & \K\textwon \\
\K\textcent          & \K\textdollaroldstyle & \K\textlira     & \K\textyen \\
\K\textcentoldstyle  & \K\textdong           & \K\textnaira    \\
\K\textcolonmonetary & \K\texteuro           & \K\textpeso     \\
\K\textcurrency      & \K\textflorin         & \K\textsterling$^*$ \\
\end{longtable}
\end{longsymtable}

\begin{symtable}[MARV]{\MARV\ Currency Symbols}
\idxboth{currency}{symbols}
\idxboth{monetary}{symbols}
\index{euro signs}
\label{marv-currency}
\begin{tabular}{*4{ll}ll}
\K\Denarius   & \K\EUR    & \K\EURdig   & \K\EURtm      & \K\Pfund      \\
\K\Ecommerce  & \K\EURcr  & \K\EURhv    & \K\EyesDollar & \K\Shilling   \\
{\arial \char"20AC}                      &                &                   &                      &                   \\
\end{tabular}

\bigskip

\begin{tablenote}
  The different euro signs are meant to be visually compatible with
  different fonts---\PSfont{Courier} (\texttt{\string\EURcr}),
  \PSfont{Helvetica} (\texttt{\string\EURhv}), \PSfont{Times Roman}
  (\texttt{\string\EURtm}), and the \MARV\ digits listed in
  \ref{marv-digits} (\texttt{\string\EURdig}).
%  
%
%\ifMDES
%  The \MDES\ package redefines \cmdI[\MDEStexteuro]{\texteuro} to be
%  visually compatible with one of three additional fonts:
%  \PSfont{Utopia}~({\usefont{TS1}{mdput}{m}{n}\char"BF}),
%  \PSfont{Charter}~({\usefont{TS1}{mdbch}{m}{n}\char"BF}), or
%  \PSfont{Garamond}~({\usefont{TS1}{mdugm}{m}{n}\char"BF}).
%\fi
%
\end{tablenote}
\end{symtable}


\begin{symtable}[WASY]{\WASY\ Currency Symbols}
\idxboth{currency}{symbols}
\idxboth{monetary}{symbols}
\label{wasy-currency}
\begin{tabular}{ll@{\qquad}ll}
\K\cent & \K\currency \\
\end{tabular}
\end{symtable}

There is another package providing Euro related signs the \pkgname{eurosym}. The package provides the commands, \docAuxCommand{geneuro}, \docAuxCommand{geneuronarrow}, \docAuxCommand{geneurowide} and \cmd{\officialeuro}. You can read more at \url{http://www.theiling.de/eurosym.html}. \texttt{eurosym}  provides a new symbol to be used for the European currency, the Euro. The specifications were taken from a picture in the c't magazine 11/98 p.211 and from Encyclopaedia Britannica, Book of the Year 2002 (thanks to Dr. Werner Gans).

\texttt{eurosym}'s Euro symbol is implemented in \texttt{MetaFont}, and thus fits smoothly into a \texttt{LaTeX} installation. It is now part of major Linux distributions, including Debian, Suse, Mandrake and probably others.

Apart from the official form, the eurosym package provides some generalisations that fit non-roman font faces better.

\ifEUSYM
\begin{symtable}[EUSYM]{\EUSYM\ Euro Signs}
\idxboth{currency}{symbols}
\idxboth{monetary}{symbols}
\index{euro signs}
\label{eurosym-euros}
\begin{tabular}{*4{ll}}
\K\geneuro & \K\geneuronarrow & \K\geneurowide & \K\officialeuro \\
\end{tabular}

\bigskip

\begin{tablenote}
  \cmd{\euro} is automatically mapped to one of the above---by
  default, \docAuxCommand{officialeuro}---based on a \EUSYM\ package option.
  \seedocs{\EUSYM}.  The \verb|\geneuro|\dots{} characters are
  generated from the current body font's ``C'' character and therefore
  may not appear exactly as shown.
\end{tablenote}

\begin{tablenote}
To use the symbol with fontspec see the package documentation.
\end{tablenote}
\end{symtable}
\fi



\begin{symtable}[CHINA]{\CHINA\ Currency Symbols}
\idxboth{currency}{symbols}
\idxboth{monetary}{symbols}
\index{euro signs}
\label{china-euro}
\begin{tabular}{ll@{\qquad}ll}
  \K\Euro & \K\Pound \\
\end{tabular}
\end{symtable}



\begin{symtable}{\TC\ Legal Symbols}
\index{copyright}
\idxboth{legal}{symbols}
\label{tc-legal}
\begin{tabular}{*2{lll@{\qquad}}lll}
\indexTextcomp\textcircledP & \indexTextcomp[\ltextcopyright]\textcopyright   
&\indexTextcomp\textservicemark \\

\indexTextcomp\textcopyleft 
& \indexTextcomp[\ltextregistered]\textregistered 
& \indexTextcomp[\ltexttrademark]\texttrademark \\
\end{tabular}

\bigskip
\twosymbolmessage
\medskip
\begin{tablenote}
  \hspace*{15pt}%
  See \url{http://www.tex.ac.uk/cgi-bin/texfaq2html?label=tradesyms}
  for solutions to common problems that occur when using these symbols
  (e.g.,~getting a~``\textcircled{r}'' when you expected to get
  a~``\textregistered'').
\end{tablenote}
\end{symtable}


\begin{symtable}[CCLIC]{\CCLIC\ Creative Commons License Icons}
\index{Creative Commons licenses}
\index{copyright}
\idxboth{legal}{symbols}
\label{creativecommons}
\begin{tabular}{*4{ll@{\qqquad}}ll}
\K\cc & \K\ccby & \K\ccnc$^*$ & \K\ccnd & \K\ccsa$^*$ \\
\end{tabular}

\bigskip
\begin{tablenote}[*]
  These symbols utilize the \pkgname{rotating} package and therefore
  display improperly in some DVI\index{DVI} viewers.
\end{tablenote}
\end{symtable}


\begin{symtable}{\TC\ Old-style Numerals}
\idxboth{old-style}{digits}
\index{numerals>old style}
\label{old-style-nums}
\begin{tabular}{*3{ll}}
\K\textzerooldstyle  & \K\textfouroldstyle  & \K\texteightoldstyle \\
\K\textoneoldstyle   & \K\textfiveoldstyle  & \K\textnineoldstyle  \\
\K\texttwooldstyle   & \K\textsixoldstyle   \\
\K\textthreeoldstyle & \K\textsevenoldstyle \\
\end{tabular}

\bigskip
\begin{tablenote}
  Rather than use the bulky \cmd{\textoneoldstyle},
  \cmd{\texttwooldstyle}, etc.\ commands shown above, consider using
  \docAuxCommand{oldstylenums}\verb|{|$\ldots$\verb|}| to typeset an old-style eg. abcde{\oldstylenums 123456789}fgh. These type of
symbols and commands become redundant with the correct font, as the old style numbers are a feature of the font. Not all fonts provide old style numbers.
\end{tablenote}
\end{symtable}

\section{Miscellaneous Symbols}

\begin{longsymtable}{Miscellaneous \TC\ Symbols}
\idxboth{musical}{symbols}
\index{tilde}
\label{tc-misc}
\begin{longtable}{lll@{\qquad}lll}
\indexTextcomp\textasteriskcentered & \indexTextcomp[\ltextordfeminine]\textordfeminine   \\
\indexTextcomp\textbardbl           & \indexTextcomp[\ltextordmasculine]\textordmasculine \\
\indexTextcomp\textbigcircle        & \indexTextcomp\textparagraph$^*$                    \\
\indexTextcomp\textblank            & \indexTextcomp\textperiodcentered                   \\
\indexTextcomp\textbrokenbar        & \indexTextcomp\textpertenthousand                   \\
\indexTextcomp\textbullet           & \indexTextcomp\textperthousand                      \\
\indexTextcomp\textdagger$^*$       & \indexTextcomp\textpilcrow                          \\
\indexTextcomp\textdaggerdbl$^*$    & \indexTextcomp\textquotesingle                      \\
\indexTextcomp\textdblhyphen        & \indexTextcomp\textquotestraightbase                \\
\indexTextcomp\textdblhyphenchar    & \indexTextcomp\textquotestraightdblbase             \\
\indexTextcomp\textdiscount         & \indexTextcomp\textrecipe                           \\
\indexTextcomp\textestimated        & \indexTextcomp\textreferencemark                    \\
\indexTextcomp\textinterrobang      & \indexTextcomp\textsection$^*$                      \\
\indexTextcomp\textinterrobangdown  & \indexTextcomp\textthreequartersemdash              \\
\indexTextcomp\textmusicalnote      & \indexTextcomp\texttildelow                         \\
\indexTextcomp\textnumero           & \indexTextcomp\texttwelveudash                      \\
\indexTextcomp\textopenbullet                                                 \\
\end{longtable}

\bigskip
\twosymbolmessage

\bigskip
\usetextmathmessage[*]

\end{longsymtable}
%
%\begin{symtable}[WASY]{Miscellaneous \WASY\ Text-mode Symbols}
%\label{wasy-text}
%\begin{tabular}{ll}
%\K\permil \\
%\end{tabular}
%\end{symtable}
%\idxbothend{body-text}{symbols}



\section{Mathematical symbols}
\label{math-symbols}
\idxbothbegin{mathematical}{symbols}


Most, but not all, of the symbols in this section are math-mode only.
That is, they yield a ``\texttt{Missing~\$ inserted}''\index{Missing
\$ inserted=``\texttt{Missing~\$ inserted}''} error message if not
used within \verb|$|$\ldots$\verb|$|, \verb|\[|$\ldots$\verb|\]|, or
another math-mode environment.  Operators marked as ``variable-sized''
are taller in displayed formulas, shorter in in-text formulas, and
possibly shorter still when used in various levels of superscripts or
subscripts.

% The following definition is used both in the discussion of disjoint
% union and in the "Joining and overlapping existing symbols" section.

\newcommand{\dotcup}{\ensuremath{\mathaccent\cdot\cup}}


Alphanumeric symbols (e.g., $\mathscr{L}$, and
|\varmathbb{Z}|) are usually produced using one of the math
alphabets in \ref{alphabets} rather than with an explicit symbol
command.  Look there first if you need a symbol for a transform,
number set, or some other alphanumeric.

Although there have been many requests on \ctt for a
contradiction\idxboth{contradiction}{symbols} symbol, the ensuing
discussion invariably reveals innumerable ways to represent
contradiction in a proof, including ``|\blitza|''~(\cmd{\blitza}),
``$\Rightarrow\Leftarrow$''~(\docAuxCommand{Rightarrow}\docAuxCommand{Leftarrow}),
``$\bot$''~(\docAuxCommand{bot}),
``$\nleftrightarrow$''~(\docAuxCommand{nleftrightarrow}), and
%``\textreferencemark''~(\docAuxCommand{textreferencemark}).  Because of the
%lack of notational consensus, it is probably better to spell out
%``Contradiction!''\ than to use a symbol for this purpose.  Similarly,
%discussions on \ctt have revealed that there are a variety of ways to
%indicate the mathematical notion of ``is
%defined\idxboth{definition}{symbols} as''.  Common candidates include
%``$\triangleq$''~(\docAuxCommand{triangleq}), ``$\equiv$''~(\docAuxCommand{equiv}),
%``$\coloneqq$''~(\emph{various}\footnote{In \TX, \PX, and \MTOOLS\ the
%symbol is called \docAuxCommand{coloneqq}.  In |\ABX\| and MNS\footnote{Do not use it uses too many aplhabets} it's called
%\cmdI[$\string\ABXcoloneq$]{\coloneq}.  In \CEQ\ it's called
%colonequals}.}), and ``$\stackrel{\text{\tiny
%def}}{=}$''~(\cmd{\stackrel}\verb|{|\cmd{\text}\verb|{\tiny|
%\verb|def}}{=}|).  See also the example of \cmd{\equalsfill}
%\vpageref[below]{equalsfill-ex}.  Depending upon the context,
%disjoint\index{disjoint union} union may be represented as
%``$\coprod$''~(\docAuxCommand{coprod}), ``$\sqcup$''~(\docAuxCommand{sqcup}),
%``$\dotcup$''~(\docAuxCommand{dotcup}), ``$\oplus$''~(\docAuxCommand{oplus}), or any
%of a number of other symbols.\footnote{\person{Bob}{Tennent} listed
%these and other disjoint-union symbol possibilities in a November~2007
%post to \ctt.}  Finally, the average\index{average} value of a
%variable~$x$ is written by some people as
%``$\overline{x}$''~(\verb|\overline{x}|)\incsyms\indexaccent[$\string\blackacc{\string\overline}$]{\overline},
%by some people as ``$\langle x \rangle$''~(\docAuxCommand{langle} \texttt{x}
%\docAuxCommand{rangle}), and by some people as ``$\diameter x$'' or
%``$\varnothing x$''~(\docAuxCommand{diameter} \texttt{x} or \docAuxCommand{varnothing}
%\texttt{x}).  The moral of the story is that you should be careful
%always to explain your notation to avoid confusing your readers.



\bigskip

\begin{symtable}{Math-Mode Versions of Text Symbols}
\index{underline}
\label{math-text-vers}
\begin{tabular}{*3{ll}}
\X\mathdollar   & \X\mathparagraph & \X\mathsterling   \\
\X\mathellipsis & \X\mathsection   & \X\mathunderscore \\
\end{tabular}

\bigskip
\usetextmathmessage

\end{symtable}

\subsection{CMLL}
The \pkgname{cmll} defines a handful of symbols useful in linear logic and not found in other
%font packages \cite{cmll}. The package defines unary operators, binary operators, large operators, binary relations and letter-like symbols |\Bot| $\Bot$ and $\simbot$
%and |\simbot|.

\begin{symtable}[CMLL]{\CMLL\ Unary Operators}
\idxboth{unary}{operators}
\idxboth{linear logic}{symbols}
\label{cmll-unary}
\begin{tabular}{*2{ll@{\qquad}}ll}
\K[!]\oc$^*$         & \K[\CMLLshneg]\shneg & \K[?]\wn$^*$ \\
\K[\CMLLshift]\shift & \K[\CMLLshpos]\shpos &              \\
\end{tabular}

\bigskip

\begin{tablenote}[*]
  \docAuxCommand{oc} and \docAuxCommand{wn} differ from~``!''  and~``?'' in
  terms of their math-mode spacing: \verb|$A=!B$| produces ``$A=!B$'',
  for example, while \verb|$A=\oc B$| produces ``$A=\mathord{!}B$''.
\end{tablenote}
\end{symtable}


`Linear implication' is not included in the grammar of connectives, but is definable in CLL using linear negation and multiplicative disjunction, by $A⊸B:=A^{{\pan ⊥}}$.


\begin{symtable}{Binary Operators}
\idxboth{binary}{operators}
\index{division}
\idxboth{linear logic}{symbols}
\label{bin}
\begin{tabular}{*4{ll}}
\X\amalg           & \X\cup          & \X\oplus    & \X\times           \\
\X\ast             & \X\dagger       & \X\oslash   & \X\triangleleft    \\
\X\bigcirc         & \X\ddagger      & \X\otimes   & \X\triangleright   \\
\X\bigtriangledown & \X\diamond      & \X\pm       & \X\unlhd$^*$       \\
\X\bigtriangleup   & \X\div          & \X\rhd$^*$  & \X\unrhd$^*$       \\
\X\bullet          & \X\lhd$^*$      & \X\setminus & \X\uplus           \\
\X\cap             & \X\mp           & \X\sqcap    & \X\vee             \\
\X\cdot            & \X\odot         & \X\sqcup    & \X\wedge           \\
\X\circ            & \X\ominus       & \X\star     & \X\wr              \\
\end{tabular}

\bigskip
\notpredefinedmessage
\end{symtable}


\begin{symtable}{AMS Binary Operators}
\idxboth{binary}{operators}
\index{semidirect products}
\label{ams-bin}
\begin{tabular}{*3{ll}}
\X\barwedge        & \X\circledcirc     & \X\intercal$^*$    \\
\X\boxdot          & \X\circleddash     & \X\leftthreetimes  \\
\X\boxminus        & \X\Cup             & \X\ltimes          \\
\X\boxplus         & \X\curlyvee        & \X\rightthreetimes \\
\X\boxtimes        & \X\curlywedge      & \X\rtimes          \\
\X\Cap             & \X\divideontimes   & \X\smallsetminus   \\
\X\centerdot       & \X\dotplus         & \X\veebar          \\
\X\circledast      & \X\doublebarwedge  \\
\end{tabular}

\bigskip

\begin{tablenote}[*]
  \newcommand{\trpose}{{\mathpalette\raiseT{\intercal}}}
  \newcommand{\raiseT}[2]{\raisebox{0.25ex}{$#1#2$}}
%
  Some people use a superscripted \docAuxCommand{intercal} for matrix
  transpose\index{transpose}: ``\verb|A^\intercal|''~$\mapsto$
  ``$A^\intercal$''.  (See the May~2009 \ctt thread, ``raising math
  symbols'', for suggestions about altering the height of the
  superscript. and se.tex question \footnote{\url{http://tex.stackexchange.com/questions/30619/what-is-the-best-symbol -for-vector-matrix-transpose}})  \docAuxCommand{top} (\vref*{letter-like}), \verb|T|, and
  \verb|\mathsf{T}| are other popular choices: ``$A^\top$'',
  ``$A^T$'', ``$A^{\text{\textsf{T}}}$''.
\end{tablenote}

\end{symtable}



\subsection{St Mary Road Binary Operators}

%\begin{symtable}[ST]{\ST\ Binary Operators}
\idxboth{binary}{operators}
\idxboth{linear logic}{symbols}
\label{st-bin}
\captionof{table}{\ST\ Binary Operators}
\begin{longtable}{*3{ll}}
\X\baro                & \X\interleave          & \X\varoast             \\
\X\bbslash             & \X\leftslice           & \X\varobar             \\
\X\binampersand        & \X\merge               & \X\varobslash          \\
\X\bindnasrepma        & \X\minuso              & \X\varocircle          \\
\X\boxast              & \X\moo                 & \X\varodot             \\
\X\boxbar              & \X\nplus               & \X\varogreaterthan     \\
\X\boxbox              & \X\obar                & \X\varolessthan        \\
\X\boxbslash           & \X\oblong              & \X\varominus           \\
\X\boxcircle           & \X\obslash             & \X\varoplus            \\
\X\boxdot              & \X\ogreaterthan        & \X\varoslash           \\
\X\boxempty            & \X\olessthan           & \X\varotimes           \\
\X\boxslash            & \X\ovee                & \X\varovee             \\
\X\curlyveedownarrow   & \X\owedge              & \X\varowedge           \\
\X\curlyveeuparrow     & \X\rightslice          & \X\vartimes            \\
\X\curlywedgedownarrow & \X\sslash              & \X\Ydown               \\
\X\curlywedgeuparrow   & \X\talloblong          & \X\Yleft               \\
\X\fatbslash           & \X\varbigcirc          & \X\Yright              \\
\X\fatsemi             & \X\varcurlyvee         & \X\Yup                 \\
\X\fatslash            & \X\varcurlywedge       \\
\end{longtable}


%\end{symtable}


\begin{symtable}[WASY]{\WASY\ Binary Operators}
\idxboth{binary}{operators}
\label{wasy-bin}
\begin{tabular}{*4{ll}}
\X\lhd & \X\ocircle & \X\RHD   & \X\unrhd \\
\X\LHD & \X\rhd     & \X\unlhd            \\
\end{tabular}
\end{symtable}


\section{Unicode Binary Operators}
 \subsection{Binary operators}
 \index{binary operators}
 \begin{multicols}{2}
% \showsymbolbin+{000B}{}
 \showsymbolbin\pm{00B1}{}
 \showsymbolbin\cdotp{00B7}{}%?, \cmd\centerdot
 \showsymbolbin\times{00D7}{}
 \showsymbolbin\div{00F7}{}
 \showsymbolbin\dagger{2020}{}
 \showsymbolbin\ddagger{2021}{}
 \showsymbolbin\smblkcircle{2022}{}
 \showsymbolbin\fracslash{2044}{}
 \showsymbolbin\upand{214B}{}
% \showsymbolbin-{000D}{}
 \showsymbolbin\mp{2213}{}
 \showsymbolbin\dotplus{2214}{}
 \showsymbolbin\smallsetminus{2216}{}
 \showsymbolbin\ast{2217}{}
 \showsymbolbin\vysmwhtcircle{2218}{}
 \showsymbolbin\vysmblkcircle{2219}{}, {\small\cmd\bullet}
 \showsymbolbin\wedge{2227}{}, \cmd\land
 \showsymbolbin\vee{2228}{}, \cmd\lor
 \showsymbolbin\cap{2229}{}
 \showsymbolbin\cup{222A}{}
 \showsymbolbin\dotminus{2238}{}
 \showsymbolbin\invlazys{223E}{}
 \showsymbolbin\wr{2240}{}
 \showsymbolbin\cupleftarrow{228C}{}
 \showsymbolbin\cupdot{228D}{}
 \showsymbolbin\uplus{228E}{}
 \showsymbolbin\sqcap{2293}{}
 \showsymbolbin\sqcup{2294}{}
 \showsymbolbin\oplus{2295}{}
 \showsymbolbin\ominus{2296}{}
 \showsymbolbin\otimes{2297}{}
 \showsymbolbin\oslash{2298}{}
 \showsymbolbin\odot{2299}{}
 \showsymbolbin\circledcirc{229A}{}
 \showsymbolbin\circledast{229B}{}
 \showsymbolbin\circledequal{229C}{}
 \showsymbolbin\circleddash{229D}{}
 \showsymbolbin\boxplus{229E}{}
 \showsymbolbin\boxminus{229F}{}
 \showsymbolbin\boxtimes{22A0}{}
 \showsymbolbin\boxdot{22A1}{}
 \showsymbolbin\intercal{22BA}{}
 \showsymbolbin\veebar{22BB}{}
 \showsymbolbin\barwedge{22BC}{}
 \showsymbolbin\barvee{22BD}{}
 \showsymbolbin\diamond{22C4}{}, \cmd\smwhtdiamond
 \showsymbolbin\cdot{22C5}{*}
 \showsymbolbin\star{22C6}{}
 \showsymbolbin\divideontimes{22C7}{}
 \showsymbolbin\ltimes{22C9}{}
 \showsymbolbin\rtimes{22CA}{}
 \showsymbolbin\leftthreetimes{22CB}{}
 \showsymbolbin\rightthreetimes{22CC}{}
 \showsymbolbin\curlyvee{22CE}{}
 \showsymbolbin\curlywedge{22CF}{}
 \showsymbolbin\Cap{22D2}{}, \cmd\doublecap
 \showsymbolbin\Cup{22D3}{}, \cmd\doublecup
 \showsymbolbin\varbarwedge{2305}{*}
 \showsymbolbin\vardoublebarwedge{2306}{*}
 \showsymbolbin\obar{233D}{}
 \showsymbolbin\triangle{25B3}{}, \cmd\bigtriangleup
 \showsymbolbin\lhd{22B2}{}
 \showsymbolbin\rhd{22B3}{}
 \showsymbolbin\unlhd{22B4}{}
 \showsymbolbin\unrhd{22B5}{}
 \showsymbolbin\mdlgwhtcircle{25CB}{*}
 \showsymbolbin\boxbar{25EB}{*}
 \showsymbolbin\veedot{27C7}{*}
 \showsymbolbin\wedgedot{27D1}{*}
 \showsymbolbin\lozengeminus{27E0}{*}
 \showsymbolbin\concavediamond{27E1}{*}
 \showsymbolbin\concavediamondtickleft{27E2}{*}
 \showsymbolbin\concavediamondtickright{27E3}{*}
 \showsymbolbin\whitesquaretickleft{27E4}{*}
 \showsymbolbin\whitesquaretickright{27E5}{*}
 \showsymbolbin\typecolon{2982}{*}
 \showsymbolbin\circlehbar{29B5}{*}
 \showsymbolbin\circledvert{29B6}{}
 \showsymbolbin\circledparallel{29B7}{}
 \showsymbolbin\obslash{29B8}{}
 \showsymbolbin\operp{29B9}{*}
 \showsymbolbin\olessthan{29C0}{}
 \showsymbolbin\ogreaterthan{29C1}{}
 \showsymbolbin\boxdiag{29C4}{}
 \showsymbolbin\boxbslash{29C5}{}
 \showsymbolbin\boxast{29C6}{}
 \showsymbolbin\boxcircle{29C7}{}
 \showsymbolbin\boxbox{29C8}{*}
 \showsymbolbin\triangleserifs{29CD}{*}
 \showsymbolbin\hourglass{29D6}{*}
 \showsymbolbin\blackhourglass{29D7}{*}
 \showsymbolbin\shuffle{29E2}{*}
 \showsymbolbin\mdlgblklozenge{29EB}{*}
 \showsymbolbin\setminus{29F5}{*}
 \showsymbolbin\dsol{29F6}{*}
 \showsymbolbin\rsolbar{29F7}{*}
 \showsymbolbin\doubleplus{29FA}{*}
 \showsymbolbin\tripleplus{29FB}{*}
 \showsymbolbin\tplus{29FE}{*}
 \showsymbolbin\tminus{29FF}{*}
 \showsymbolbin\ringplus{2A22}{}
 \showsymbolbin\plushat{2A23}{}
 \showsymbolbin\simplus{2A24}{}
 \showsymbolbin\plusdot{2A25}{}
 \showsymbolbin\plussim{2A26}{}
 \showsymbolbin\plussubtwo{2A27}{}
 \showsymbolbin\plustrif{2A28}{*}
 \showsymbolbin\commaminus{2A29}{*}
 \showsymbolbin\minusdot{2A2A}{}
 \showsymbolbin\minusfdots{2A2B}{}
 \showsymbolbin\minusrdots{2A2C}{*}
 \showsymbolbin\opluslhrim{2A2D}{*}
 \showsymbolbin\oplusrhrim{2A2E}{*}
 \showsymbolbin\vectimes{2A2F}{*}
 \showsymbolbin\dottimes{2A30}{}
 \showsymbolbin\timesbar{2A31}{}
 \showsymbolbin\btimes{2A32}{}
 \showsymbolbin\smashtimes{2A33}{*}
 \showsymbolbin\otimeslhrim{2A34}{*}
 \showsymbolbin\otimesrhrim{2A35}{*}
 \showsymbolbin\otimeshat{2A36}{*}
 \showsymbolbin\Otimes{2A37}{*}
 \showsymbolbin\odiv{2A38}{*}
 \showsymbolbin\triangleplus{2A39}{*}
 \showsymbolbin\triangleminus{2A3A}{*}
 \showsymbolbin\triangletimes{2A3B}{*}
 \showsymbolbin\intprod{2A3C}{*}
 \showsymbolbin\intprodr{2A3D}{*}
 \showsymbolbin\fcmp{2A3E}{*}
 \showsymbolbin\amalg{2A3F}{}
 \showsymbolbin\capdot{2A40}{*}
 \showsymbolbin\uminus{2A41}{*}
 \showsymbolbin\barcup{2A42}{*}
 \showsymbolbin\barcap{2A43}{*}
 \showsymbolbin\capwedge{2A44}{*}
 \showsymbolbin\cupvee{2A45}{*}
 \showsymbolbin\cupovercap{2A46}{*}
 \showsymbolbin\capovercup{2A47}{*}
 \showsymbolbin\cupbarcap{2A48}{*}
 \showsymbolbin\capbarcup{2A49}{*}
 \showsymbolbin\twocups{2A4A}{*}
 \showsymbolbin\twocaps{2A4B}{*}
 \showsymbolbin\closedvarcup{2A4C}{*}
 \showsymbolbin\closedvarcap{2A4D}{*}
 \showsymbolbin\Sqcap{2A4E}{*}
 \showsymbolbin\Sqcup{2A4F}{*}
 \showsymbolbin\closedvarcupsmashprod{2A50}{*}
 \showsymbolbin\wedgeodot{2A51}{*}
 \showsymbolbin\veeodot{2A52}{*}
 \showsymbolbin\Wedge{2A53}{*}
 \showsymbolbin\Vee{2A54}{*}
 \showsymbolbin\wedgeonwedge{2A55}{*}
 \showsymbolbin\veeonvee{2A56}{*}
 \showsymbolbin\bigslopedvee{2A57}{*}
 \showsymbolbin\bigslopedwedge{2A58}{*}
 \showsymbolbin\wedgemidvert{2A5A}{*}
 \showsymbolbin\veemidvert{2A5B}{*}
 \showsymbolbin\midbarwedge{2A5C}{*}
 \showsymbolbin\midbarvee{2A5D}{*}
 \showsymbolbin\doublebarwedge{2A5E}{}
 \showsymbolbin\wedgebar{2A5F}{*}
 \showsymbolbin\wedgedoublebar{2A60}{*}
 \showsymbolbin\varveebar{2A61}{*}
 \showsymbolbin\doublebarvee{2A62}{*}
 \showsymbolbin\veedoublebar{2A63}{}
 \showsymbolbin\dsub{2A64}{*}
 \showsymbolbin\rsub{2A65}{*}
 \showsymbolbin\eqqplus{2A71}{}
 \showsymbolbin\pluseqq{2A72}{}
 \showsymbolbin\interleave{2AF4}{}
 \showsymbolbin\nhVvert{2AF5}{}
 \showsymbolbin\threedotcolon{2AF6}{}
 \showsymbolbin\trslash{2AFB}{}
 \showsymbolbin\sslash{2AFD}{}
 \showsymbolbin\talloblong{2AFE}{}
 \end{multicols}





\begin{symtable}{Variable-sized Math Operators}
\idxboth{variable-sized}{symbols}
\idxboth{linear logic}{symbols}
\index{integrals}
\label{op}
\renewcommand{\arraystretch}{1.75}  
\begin{tabular}{*3{l@{$\:$}ll@{\qquad}}l@{$\:$}ll}
\R\bigcap    & \R\bigotimes & \R\bigwedge  & \R\prod      \\
\R\bigcup    & \R\bigsqcup  & \R\coprod    & \R\sum       \\
\R\bigodot   & \R\biguplus  & \R\int       \\
\R\bigoplus  & \R\bigvee    & \R\oint      \\
\end{tabular}
\end{symtable}




\begin{symtable}[AMS]{\AmS Variable-sized Math Operators}
\idxboth{variable-sized}{symbols}
\index{integrals}
\label{ams-large}
\renewcommand{\arraystretch}{2.5}  
\begin{tabular}{l@{$\:$}ll@{\qquad}l@{$\:$}ll}
% removed optional \R[\AMSiint]
\R\iint     & \R\iiint       \\
\R\iiiint & \R\idotsint \\
\end{tabular}
\end{symtable}


\begin{symtable}[ST]{\ST\ Variable-sized Math Operators}
\idxboth{variable-sized}{symbols}
\label{st-large}
\renewcommand{\arraystretch}{1.75} 
\begin{tabular}{*2{l@{$\:$}ll@{\qquad}}l@{$\:$}ll}
\R\bigbox        & \R\biginterleave & \R\bigsqcap                            \\
\R\bigcurlyvee   & \R\bignplus      & \R[\STbigtriangledown]\bigtriangledown \\
\R\bigcurlywedge & \R\bigparallel   & \R[\STbigtriangleup]\bigtriangleup     \\
\end{tabular}
\end{symtable}


\begin{symtable}[WASY]{\WASY\ Variable-sized Math Operators}
\idxboth{variable-sized}{symbols}
\index{integrals}
\label{wasy-large}
\renewcommand{\arraystretch}{2.5}  
\begin{tabular}{*2{l@{$\:$}ll@{\qquad}}l@{$\:$}ll}
\R[\varint]\int$^\dag$ & \R\iint        & \R\iiint \\
\R\varint$^*$          & \R\varoint$^*$ & \R\oiint \\
\end{tabular}

\bigskip
\begin{tablenote}
  None of the preceding symbols are defined when \WASY\ is passed the
  \optname{wasysym}{nointegrals} option.
\end{tablenote}

\medskip
\begin{tablenote}[*]
  Not defined when \WASY\ is passed the \optname{wasysym}{integrals} option.
\end{tablenote}

\medskip
\begin{tablenote}[\dag]
  Defined only when \WASY\ is passed the \optname{wasysym}{integrals}
  option.  Otherwise, the default \latex \docAuxCommand{int} glyph (as shown
  in \ref{op}) is used.
\end{tablenote}
\end{symtable}

\begin{symtable}{Negated Binary Relations}
\index{binary relations>negated}
\index{relational symbols>negated binary}
\label{ams-nrel}
\begin{tabular}{*3{ll}}
\X\ncong     & \X\nshortparallel & \X\nVDash      \\
\X\nmid      & \X\nsim           & \X\precnapprox \\
\X\nparallel & \X\nsucc          & \X\precnsim    \\
\X\nprec     & \X\nsucceq        & \X\succnapprox \\
\X\npreceq   & \X\nvDash         & \X\succnsim    \\
\X\nshortmid & \X\nvdash                          \\
\end{tabular}
\end{symtable}


%\begin{symtable}[ST]{\ST\ Binary Relations}
%\index{binary relations}
%\index{relational symbols>binary}
%\label{st-rel}
%\begin{tabular}{*2{ll}}
%\X\inplus & \X\niplus \\
%\end{tabular}
%\end{symtable}


\begin{symtable}[WASY]{\WASY\ Binary Relations}
\index{binary relations}
\index{relational symbols>binary}
\label{wasy-rel}
\begin{tabular}{*3{ll}}
\X\invneg & \X\leadsto & \X\wasypropto \\
\X\Join   & \X\logof                   \\
\end{tabular}
\end{symtable}


\begin{symtable}[CMLL]{\CMLL\ Binary Relations}
\index{binary relations}
\index{relational symbols>binary}
\idxboth{linear logic}{symbols}
\label{cmll-rel}
\begin{tabular}{ll@{\hspace*{2em}}ll}
\K[\CMLLcoh]\coh     & \K[\CMLLscoh]\scoh     \\
\K[\CMLLincoh]\incoh & \K[\CMLLsincoh]\sincoh \\
\end{tabular}
\end{symtable}



\begin{symtable}{Subset and Superset Relations}
\index{binary relations}
\index{relational symbols>binary}
\index{subsets}
\index{supersets}
\index{symbols>subset and superset}
\label{subsets}
\begin{tabular}{*3{ll}}
\X\sqsubset$^*$ & \X\sqsupseteq & \X\supset   \\
\X\sqsubseteq   & \X\subset     & \X\supseteq \\
\X\sqsupset$^*$ & \X\subseteq                 \\
\end{tabular}

\bigskip
\notpredefinedmessageABX
\end{symtable}

\section{Inequalities}
\begin{symtable}{Inequalities}
\index{binary relations}\index{relational symbols>binary}
\index{inequalities}
\label{inequal-rel}
\begin{tabular}{*5{ll}}
\X\geq & \X\gg & \X\leq & \X\ll & \X\neq \\
\end{tabular}
\end{symtable}
\begin{symtable}{ Subset and Superset Relations}
\index{binary relations}
\index{relational symbols>binary}
\index{subsets}
\index{supersets}
\index{symbols>subset and superset}
\label{ams-subsets}
\begin{tabular}{*3{ll}}
\X\nsubseteq  & \X\subseteqq  & \X\supsetneqq    \\
\X\nsupseteq  & \X\subsetneq  & \X\varsubsetneq  \\
\X\nsupseteqq & \X\subsetneqq & \X\varsubsetneqq \\
\X\sqsubset   & \X\Supset     & \X\varsupsetneq  \\
\X\sqsupset   & \X\supseteqq  & \X\varsupsetneqq \\
\X\Subset     & \X\supsetneq                     \\
\end{tabular}
\end{symtable}


%\begin{symtable}[ST]{\ST\ Subset and Superset Relations}
%\index{binary relations}
%\index{relational symbols>binary}
%\index{subsets}
%\index{supersets}
%\index{symbols>subset and superset}
%\label{st-subsets}
%\begin{tabular}{*2{ll}}
%\X\subsetplus   & \X\supsetplus   \\
%\X\subsetpluseq & \X\supsetpluseq \\
%\end{tabular}
%\end{symtable}


\begin{symtable}[WASY]{\WASY\ Subset and Superset Relations}
\index{binary relations}
\index{relational symbols>binary}
\index{subsets}
\index{supersets}
\index{symbols>subset and superset}
\label{wasy-subset}
\begin{tabular}{*2{ll}}
\X\sqsubset & \X\sqsupset \\
\end{tabular}
\end{symtable}


\begin{symtable}{AMS Triangle Relations}
\index{triangle relations}\index{relational symbols>triangle}
\label{ams-triangle-rel}
\begin{tabular}{*3{ll}}
\X\blacktriangleleft  & \X\ntriangleright    & \X\trianglerighteq  \\
\X\blacktriangleright & \X\ntrianglerighteq  & \X\vartriangleleft  \\
\X\ntriangleleft      & \X\trianglelefteq    & \X\vartriangleright \\
\X\ntrianglelefteq    & \X\triangleq         &                     \\
\end{tabular}
\end{symtable}


%\begin{symtable}[ST]{\ST\ Triangle Relations}
%\index{triangle relations}\index{relational symbols>triangle}
%\label{st-triangle-rel}
%\begin{tabular}{*2{ll}}
%\X\trianglelefteqslant  & \X\trianglerighteqslant  \\
%\X\ntrianglelefteqslant & \X\ntrianglerighteqslant \\
%\end{tabular}
%\end{symtable}




\begin{symtable}{Arrows}
\index{arrows}
\label{arrow}
\begin{tabular}{*3{ll}}
\X\Downarrow          & \X\longleftarrow      & \X\nwarrow     \\
\X\downarrow          & \X\Longleftarrow      & \X\Rightarrow  \\
\X\hookleftarrow      & \X\longleftrightarrow & \X\rightarrow  \\
\X\hookrightarrow     & \X\Longleftrightarrow & \X\searrow     \\
\X\leadsto$^*$        & \X\longmapsto         & \X\swarrow     \\
\X\leftarrow          & \X\Longrightarrow     & \X\uparrow     \\
\X\Leftarrow          & \X\longrightarrow     & \X\Uparrow     \\
\X\Leftrightarrow     & \X\mapsto             & \X\updownarrow \\
\X\leftrightarrow     & \X\nearrow$^\dag$     & \X\Updownarrow \\
\end{tabular}

\bigskip
\notpredefinedmessage

\bigskip
\begin{tablenote}[\dag]
  See the note beneath \ref{extensible-accents} for information
  about how to put a diagonal arrow across a mathematical expression%
%\ifhavecancel
%  ~(as in ``$\cancelto{0}{\nabla \cdot \vec{B}}\quad$'')
%\fi
.
\end{tablenote}
\end{symtable}


\begin{symtable}{Harpoons}
\index{harpoons}
\label{harpoons}
\begin{tabular}{*3{ll}}
\X\leftharpoondown   & \X\rightharpoondown  & \X\rightleftharpoons \\
\X\leftharpoonup     & \X\rightharpoonup                           \\
\end{tabular}
\end{symtable}


\begin{symtable}{\TC\ Text-mode Arrows}
\index{arrows}
\label{tc-arrows}
\begin{tabular}{*2{ll}}
\K\textdownarrow & \K\textrightarrow \\
\K\textleftarrow & \K\textuparrow    \\
\end{tabular}
\end{symtable}


\begin{symtable}{AmS Arrows}
\index{arrows}
\label{ams-arrows}
\begin{tabular}{*3{ll}}
\X\circlearrowleft    & \X\leftleftarrows          & \X\rightleftarrows   \\
\X\circlearrowright  & \X\leftrightarrows       & \X\rightrightarrows  \\
\X\curvearrowleft   & \X\leftrightsquigarrow & \X\rightsquigarrow   \\
\X\curvearrowright & \X\Lleftarrow              & \X\Rsh               \\
\X\dashleftarrow     & \X\looparrowleft        & \X\twoheadleftarrow  \\
\X\dashrightarrow  & \X\looparrowright      & \X\twoheadrightarrow \\
\X\downdownarrows   & \X\Lsh                   & \X\upuparrows        \\
\X\leftarrowtail       & \X\rightarrowtail        &                      \\
\end{tabular}
\end{symtable}


\begin{symtable}{\AmS Negated Arrows}
\index{arrows>negated}
\label{ams-narrows}
\begin{tabular}{*3{ll}}
\X\nLeftarrow       & \X\nLeftrightarrow  & \X\nRightarrow     \\
\X\nleftarrow       & \X\nleftrightarrow   & \X\nrightarrow     \\
\end{tabular}
\end{symtable}


\begin{symtable}{\AmS Harpoons}
\index{harpoons}
\label{ams-harpoons}
\begin{tabular}{*3{ll}}
\X\downharpoonleft  & \X\leftrightharpoons   & \X\upharpoonleft  \\
\X\downharpoonright & \X\rightleftharpoons & \X\upharpoonright \\
\end{tabular}
\end{symtable}




\section{Log-like Symbols}
\begin{symtable}{Log-like Symbols}
\idxboth{log-like}{symbols}
\index{atomic math objects}
\index{limits}
\label{log}
\begin{tabular}{*8l}
\Z\arccos & \Z\cos  & \Z\csc & \Z\exp & \Z\ker    & \Z\limsup & \Z\min & \Z\sinh \\
\Z\arcsin & \Z\cosh & \Z\deg & \Z\gcd & \Z\lg     & \Z\ln     & \Z\Pr  & \Z\sup  \\
\Z\arctan & \Z\cot  & \Z\det & \Z\hom & \Z\lim    & \Z\log    & \Z\sec & \Z\tan  \\
\Z\arg    & \Z\coth & \Z\dim & \Z\inf & \Z\liminf & \Z\max    & \Z\sin & \Z\tanh
\end{tabular}

\bigskip
\begin{tablenote}
  Calling the above ``symbols'' may be a bit
  misleading.\footnotemark{} Each log-like symbol merely produces the
  eponymous textual equivalent, but with proper surrounding spacing.
  See \ref{math-spacing} for more information about log-like
  symbols.  As \cmd{\bmod} and \cmd{\pmod} are arguably not symbols we
  refer the reader to the Short Math Guide for
  \latex~\cite{Downes:smg} for samples.
\end{tablenote}
\end{symtable}
\footnotetext{Michael\index{Downes, Michael J.} J. Downes prefers the
more general term, ``atomic\index{atomic math objects} math objects''.}


\begin{symtable}{AMS Log-like Symbols}
\idxboth{log-like}{symbols}
\index{atomic math objects}
\index{limits}
\label{ams-log}
\renewcommand{\arraystretch}{1.5} 
\begin{tabular}{*2{ll@{\qquad}}ll}
\X\injlim     & \X\varinjlim  & \X\varlimsup  \\
\X\projlim    & \X\varliminf  & \X\varprojlim
\end{tabular}

\bigskip
\begin{tablenote}
  Load the \pkgname{amsmath} package to get these symbols.  See
  \ref{math-spacing} for some additional comments regarding
  log-like symbols.  As \cmd{\mod} and \cmd{\pod} are arguably not
  symbols we refer the reader to the Short Math Guide for
  \latex~\cite{Downes:smg} for samples.
\end{tablenote}
\end{symtable}



\section{Greek Letters}
   
  For usage see also Chapter \vref{ch:maths}. Greek letters are fundamental
  for most mathematical documents and the control sequences to use them are shown in 
  Table~\vref{greek}.
   
  The remaining Greek majuscules\index{majuscules} can be produced
  with ordinary Latin letters.  The symbol ``M'', for instance, is
  used for both an uppercase ``m'' and an uppercase ``$\mu$''.

  See \ref{bold-math} for examples of how to produce bold Greek
  letters.\index{Greek>bold}

  The symbols in this table are intended to be used in mathematical
  typesetting.  Greek body text can be typeset using the
  \pkgname{babel} package's \optname{babel}{greek} (or
  \optname{babel}{polutonikogreek}\idxboth{polytonic}{Greek})
  option---and, of course, a font that provides the glyphs for the
  Greek alphabet.

\begin{longsymtable}{Greek Letters}
\index{Greek}\index{alphabets>Greek}
\label{greek}
\begin{longtable}{*8l}
\X\alpha        &\X\theta       &\X o           &\X\tau         \\
\X\beta         &\X\vartheta    &\X\pi          &\X\upsilon     \\
\X\gamma        &\X\iota        &\X\varpi       &\X\phi         \\
\X\delta        &\X\kappa       &\X\rho         &\X\varphi      \\
\X\epsilon      &\X\lambda      &\X\varrho      &\X\chi         \\
\X\varepsilon   &\X\mu          &\X\sigma       &\X\psi         \\
\X\zeta         &\X\nu          &\X\varsigma    &\X\omega       \\
\X\eta          &\X\xi                                          \\
                                                                \\
\X\Gamma        &\X\Lambda      &\X\Sigma       &\X\Psi         \\
\X\Delta        &\X\Xi          &\X\Upsilon     &\X\Omega       \\
\X\Theta        &\X\Pi          &\X\Phi
\end{longtable}
\end{longsymtable}


\begin{symtable}[AMS]{\AmS\ Greek Letters}
\index{Greek}\index{alphabets>Greek}
\label{ams-greek}
\begin{tabular}{*4l}
\X\digamma      &\X\varkappa
\end{tabular}
\end{symtable}



\section{Hebrew letters}
\begin{symtable}{AMS Hebrew Letters}
\index{Hebrew}\index{alphabets>Hebrew}
\label{ams-hebrew}
\begin{tabular}{*6l}
\X\beth & \X\gimel & \X\daleth
\end{tabular}

\bigskip
\begin{tablenote}
\docAuxCommand{aleph}~($\aleph$) appears in \vref{ord}.
\end{tablenote}
\end{symtable}

\section{Letter-like Symbols}

\begin{symtable}{Letter-like Symbols}
\idxboth{letter-like}{symbols}
\index{tacks}
\idxboth{linear logic}{symbols}
\label{letter-like}
\begin{tabular}{*5{ll}}
\X\bot    & \X\forall & \X\imath & \X\ni      & \X\top \\
\X\ell    & \X\hbar   & \X\in    & \X\partial & \X\wp  \\
\X\exists & \X\Im     & \X\jmath & \X\Re               \\
\end{tabular}
\end{symtable}


\begin{symtable}{\AmS Letter-like Symbols}
\idxboth{letter-like}{symbols}
\label{ams-letter-like}
\begin{tabular}{*3{ll}}
\X\Bbbk       & \X\complement & \X\hbar    \\
\X\circledR   & \X\Finv       & \X\hslash  \\
\X\circledS   & \X\Game       & \X\nexists \\
\end{tabular}
\end{symtable}


\section{Variable-sized delimiters}

\begin{symtable}{Variable-sized Delimiters}
\index{delimiters}
\index{delimiters>variable-sized}
\label{dels}
\renewcommand{\arraystretch}{1.75} 
\begin{tabular}{lll@{\qquad}lll@{\hspace*{1.5cm}}lll@{\qquad}lll}
  \N\downarrow & \N\Downarrow &               & \N[\magicrbrack]{. } \\
  \N\langle         & \N\rangle         & \Np[\vert][\magicvertname]|
                                                                          & \Np[\Vert][\magicVertname]\| \\
  \N\lceil            & \N\rceil             & \N\uparrow      & \N\Uparrow          \\
  \N\lfloor          & \N\rfloor           & \N\updownarrow  & \N\Updownarrow      \\
  \N(                  & \N)                   & \Np\{           & \Np\}               \\
  \N/                  & \N\backslash                                         \\
\end{tabular}

\bigskip
\begin{tablenote}
  When used with \cmd{\left} and \cmd{\right}, these symbols expand to
  the height of the enclosed math expression.  Note that \docAuxCommand{vert}
  is a synonym for \verb+|+\index{_=\magicvertname{} ($\vert$)}, and
  \docAuxCommand{Vert} is a synonym for \verb+\|+\index{_=\magicVertname{}
  ($\Vert$)}.

  $\varepsilon$-\TeX{}\index{e-tex=$\varepsilon$-\TeX} provides a
  \cmd{\middle} analogue to \cmd{\left} and \cmd{\right}.
  \cmd{\middle} can be used, for example, to make an internal ``$\vert$''
  expand to the height of the surrounding \cmd{\left} and \cmd{\right}
  symbols.  (This capability is commonly needed when typesetting
  adjacent bras\index{bra} and kets\index{ket} in Dirac\index{Dirac
  notation} notation: ``$\langle\phi\vert\psi\rangle$'').  A similar
  effect can be achieved in conventional \latex using the
  \pkgname{braket} package.
\end{tablenote}
\end{symtable}



\begin{symtable}[ST]{\ST\ Variable-sized Delimiters}
\index{delimiters}
\index{delimiters>variable-sized}
\index{semantic valuation}
\label{st-var-del}
\begin{tabular}{lll@{\qquad}lll}
\N\llbracket & \N\rrbracket
\end{tabular}
\end{symtable}

\begin{symtable}{\TC\ Text-mode Delimiters}
\index{delimiters}
\index{delimiters>text-mode}
\label{tc-delimiters}
\begin{tabular}{*2{ll}}
\K\textlangle    & \K\textrangle    \\
\K\textlbrackdbl & \K\textrbrackdbl \\
\K\textlquill    & \K\textrquill    \\
\end{tabular}
\end{symtable}




%%problematic skip for the moment
\section{Math-mode Accents}
%
%
%\begin{symtable}{Math-mode Accents}
%\index{accents}
%\index{accents>acute=acute (\blackacchack\')}   
%\index{accents>breve=breve (\blackacchack\u)}   
%\index{accents>caron=caron (\blackacchack\v)}   
%\index{accents>circumflex=circumflex (\blackacchack\^)}   
%\index{accents>diaeresis=di\ae{}resis (\blackacchack\")} 
%\index{accents>dot=dot (\blackacchack\. or \blackacc\d)}  
%\index{accents>grave=grave (\blackacchack\`)}   
%  
%\index{accents>ring=ring (\blackacchack\r)}     
%\index{tilde}
%\label{math-accents}
%\begin{tabular}{*4{ll}}
%\W\acute{a}    & \W\check{a}    & \W\grave{a}    & \W\tilde{a} \\
%\W\bar{a}      & \W\ddot{a}     & \W\hat{a}      & \W\vec{a}   \\
%\W\breve{a}    & \W\dot{a}      & \W\mathring{a}               \\
%\end{tabular}
%
%
%\bigskip
%
%\begin{tablenote}
%  \index{dotless i=dotless $i~(\imath)$>math mode}
%  \index{dotless j=dotless $j~(\jmath)$>math mode}
%  Also note the existence of \docAuxCommand{imath} and \docAuxCommand{jmath}, which
%  produce dotless versions of ``\textit{i}'' and ``\textit{j}''.  (See
%  \vref{ord}.)  These are useful when the accent is supposed to
%  replace the dot.  For example, ``\verb|\hat{\imath}|'' produces a
%  correct ``$\,\hat{\imath}\,$'', while ``\verb|\hat{i}|'' would yield
%  the rather odd-looking ``\,$\hat{i}\,$''.
%\end{tablenote}
%\end{symtable}
%
%
\begin{symtable}{AMS Math-mode Accents}
\index{accents}
\label{ams-math-accents}
\begin{tabular}{ll@{\hspace*{2em}}ll}
\W\dddot{a}    & \W\ddddot{a} \\
\end{tabular}

\bigskip

\begin{tablenote}
  These accents are also provided by the ABX and \pkgname{accents}
  packages and are redefined by the MDOTS package if the
  \pkgname{amsmath} and \pkgname{amssymb} packages have previously
  been loaded.  All of the variations except for the original AMS
  ones tighten the space between the dots%


\end{tablenote}
\end{symtable}

%
%
\subsection{Extensible Accents}
%
\begin{longsymtable}{Extensible Accents}
\index{accents}
\idxboth{extensible}{accents}
\idxboth{extensible}{arrows}
\index{underline}
\index{tilde}
\index{tilde>extensible}
\index{extensible tildes}
\index{symbols>extensible}
\index{accents>circumflex=circumflex (\blackacchack\^)}  
\label{extensible-accents}
\renewcommand{\arraystretch}{1.5}
\begin{longtable}{*4l}
\W\widetilde{abc}$^*$         & \W\widehat{abc}$^*$    \\
\W\overleftarrow{abc}$^\dag$  & \W\overrightarrow{abc}$^\dag$ \\
\W\overline{abc}              & \W\underline{abc}      \\
\W\overbrace{abc}             & \W\underbrace{abc}     \\[5pt]
\W\sqrt{abc}$^\ddag$                                   \\
\end{longtable}

\bigskip

\begin{tablenote}
  \def\longdivsign{%
    \ensuremath{\overline{\vphantom{)}%
      \hbox{\smash{\raise3.5\fontdimen8\textfont3\hbox{$)$}}}%
      abc}}}

  \index{long division|(}
  \index{division|(}
  \index{polynomial division|(}

  As demonstrated in a 1997 TUGboat\index{TUGboat} article about
  typesetting long-division problems~\cite{Gibbons:longdiv}, an
  extensible long-division sign (``\,\longdivsign\,'') can be faked by
  putting a ``\verb|\big)|'' in a \texttt{tabular} environment with an
  \verb|\hline| or \verb|\cline| in the preceding row.  The article
  also presents a piece of code (uploaded to CTAN as
  \texttt{longdiv.tex}%
  \index{longdiv=\textsf{longdiv} (package)}%
  \index{packages>\textsf{longdiv}}) that automatically solves and
  typesets---by putting an \docAuxCommand{overline} atop ``\verb|\big)|'' and
  the desired text---long-division problems.  Of course now we have
  a not so good unicode character for it \texttt{U+27cc} {{\pan3\char"27CC 123456}},
  which you can use with a font that supports it. 
  See also the
  \pkgname{polynom} package, which automatically solves and typesets
  polynomial-division problems in a similar manner.

  \index{long division|)}
  \index{division|)}
  \index{polynomial division|)}
\end{tablenote}

\bigskip

\begin{tablenote}[*]
  These symbols are made more extensible by the MNS package and even
  more extensible by the \pkgname{yhmath} package.
\end{tablenote}

\bigskip

\begin{tablenote}[\dag]
  If you're looking for an extensible \emph{diagonal} line or arrow to
  be used for canceling or reducing mathematical
  subexpressions\index{arrows>diagonal, for reducing subexpressions}
\ifhavecancel
  %(e.g.,~``$\cancel{x + -x}$'' or ``$\cancelto{5}{3+2}\quad$'')
\fi
  then consider using the \pkgname{cancel} package.
\end{tablenote}

\bigskip

\begin{tablenote}[\ddag]
  With an optional argument, \verb|\sqrt| typesets nth roots.  For
  example, ``\verb|\sqrt[3]{abc}|'' produces~``$\!\sqrt[3]{abc}$\,''
  and ``\verb|\sqrt[n]{abc}|'' produces~``$\!\sqrt[n]{abc}$\,''.
\end{tablenote}
\end{longsymtable}


The \pkgname{ymath} package provides some very wide and extensible accents, as well as the |\widetriangle{XYZ}| triangular hat. The latter is used in France to show that the notation $ABC$ where $A,B,C$ are three points means a triangle $\widetriangle{ABC}$ and not an 
angle $\wideparen{ABC}$ \citep{ymath}. 
\index{triangular hat accent}\index{wide triangle accent} 


\medskip
\bgroup
%\begin{longsymtable}[YH]{yhmath Extensible Accents}
\idxboth{extensible}{accents}
\index{symbols>extensible}
\index{accents>arc=arc (\blackacchack\newtie)} 
\label{yhmath-extensible-accents}
\renewcommand{\arraystretch}{1.5}
\begin{longtable}{*4l}
\W\wideparen{ABC}    & \W\widetriangle{ABC} \\[5pt]
\W\widering{ABC}     & \W\wideparen {ABC}      \\
%\W\widebar{ABC}
\end{longtable}
\captionof{table}{yhmath Extensible Accents}
\egroup
\medskip

Yiannis Haralambous stated that he called the |widering| 
because it plays the r\^ole of a wide
 symbol (and since the ring can't be widened, a parenthesis is used).
 
Here are some more examples from the documentation (the first one coded as |\ring{A}|):
 
\begin{texexample}{The ymath package} {ex:ymath}
 $$
 \ring{A},
 \widering{AB},
 \widering{ABC},
 \widering{ABCD},
 \widering{ABCDE},
 \widering{ABCDEF},
 \widering{ABCDEFG},
 $$
\end{texexample} 
 
%In this paper we give a Clifford bundle motivated approach to the wave equation of a free spin $1/2$ fermion in the de Sitter manifold, a brane with topology $M=\mathrm{S0}(4,1)/\mathrm{S0}(3,1)$ living in the bulk spacetime $\mathbb{R}^{4,1}=(\mathring{M}=\mathbb{R}^{5},\bm{\mathring{g}})$ and equipped with a metric field $\bm{g:=-i}^{\ast}\bm{\mathring{g}}$ with $\bm{i}:M\rightarrow\mathring{M}$ being the inclusion map. To obtain the analog of Dirac equation in Minkowski spacetime in the structure $\mathring{M}$ we appropriately factorize the two Casimir invariants $C_{1}$ and $C_{2}$ of the Lie algebra of the de Sitter group \ldots.

 \begin{gather}
 \begin{pmatrix} a & b\\ c & d\end{pmatrix}
 \begin{pmatrix} a & b & c\\ d & e & f\\ g & h & i\end{pmatrix}
 \begin{pmatrix} a & b & c & d\\ e & f & g & h\\ i & j & k & l\\
 m & n & o & p\end{pmatrix}
 \\
 \begin{pmatrix} a & b & c & d & e\\ f & g & h & i & j\\
 k & l & m & n & o\\ p & q & r & s & t\\ u & v & w & x & y\end{pmatrix}
 \begin{pmatrix} a & b & c & d & e & f \\ g & h & i & j & k & l \\
 m & n & o & p & q & r \\ s & t & u & v & w & x \\ y & z & \alpha &
 \beta & \gamma & \delta\end{pmatrix}
 \end{gather}

%A Kakeya set is a subset of ${\mathbb R}^d$ that contains a unit line segment in every direction. Let $\mathring S^{d-1}$ denote the unit sphere in ${\mathbb R}^d$ with antipodal points identified. We encode a Kakeya set in ${\mathbb R}^d$ as a bounded map $f:\mathring S^{d-1}\to{\mathbb R}^d$, where $f(x)$ gives the centre of the unit line segment orientated in the $x$ direction. Denoting by $B(\mathring S^{d-1})$ the collection of all such maps equipped with the supremum norm, we show that (i) for a dense set of $f$ the corresponding Kakeya set has positive Lebesgue measure and (ii) the set of those $f$ for which the corresponding Kakeya set has maximal upper box-counting (Minkowski) dimension $d$ is a residual subset of $B(\widering S^{d-1})$. We also give a very simple proof that the lower box-counting dimension of any Kakeya set is at least $d/2$.


\begin{symtable}[MTOOLS]{\MTOOLS\ Extensible Accents}
\idxboth{extensible}{accents}
\index{symbols>extensible}
\label{mathtools-extensible-accents}
\renewcommand{\arraystretch}{1.5}
\begin{tabular}{ll@{\qquad}ll}
\W[\MTOOLSoverbrace]\overbrace{abc}         & \W[\MTOOLSunderbrace]\underbrace{abc}         \\
\W[\MTOOLSoverbracket]\overbracket{abc}$^*$ & \W[\MTOOLSunderbracket]\underbracket{abc}$^*$ \\
\end{tabular}

\bigskip

\begin{tablenote}[*]
  \verb|\overbracket| and \verb|\underbracket| accept optional
  arguments that specify the bracket height and thickness.
  \seedocs{\MTOOLS}.
\end{tablenote}
\end{symtable}




\subsection{Extensible Arrows}

\begin{symtable}{AMS Extensible Arrows}
\index{arrows}
\idxboth{extensible}{arrows}
\index{symbols>extensible}
\label{ams-extensible-arrows}
\begin{tabular}{ll@{\qquad}ll}
\W\xleftarrow{abc} & \W\xrightarrow{abc} \\
\end{tabular}
\end{symtable}



\section{Dots}

%\begin{symtable}{Dots}
%\idxboth{dot}{symbols}
%\index{dots (ellipses)} \index{ellipses (dots)}
%\label{dots}
%\begin{tabular}{*{3}{ll@{\hspace*{1.5cm}}}ll}
%\X\cdotp & \X\colon$^*$    & \X\ldotp & \X\vdots$^\dag$ \\
%\X\cdots & \X\ddots$^\dag$ & \X\ldots                   \\
%\end{tabular}
%
%\bigskip
%
%\begin{tablenote}[*]
%  While ``\texttt{:}'' is valid in math mode, \cmd{\colon} uses
%  different surrounding spacing.  See \ref{math-spacing} and the
%  Short Math Guide for \latex~\cite{Downes:smg} for more information on
%  math-mode spacing.
%\end{tablenote}
%
%\bigskip
%
%\begin{tablenote}[\dag]
% \ifMDOTS
%    \let\mdcmdX=\cmdX
%  \else
%    \let\mdcmdX=\cmd
%  \fi
% The \MDOTS\ package redefines \docAuxCommand{ddots} and \docAuxCommand{vdots} to
%  make them scale properly with font size.  (They normally scale
%  horizontally but not vertically.)  \mdcmdX{\fixedddots} and
%  \mdcmdX{\fixedvdots} provide the original, fixed-height
%  functionality of \latexe's \docAuxCommand{ddots} and \docAuxCommand{vdots} macros.
%\end{tablenote}
%\end{symtable}
%
%
%
%\begin{symtable}{\AmS Dots}
%\idxboth{dot}{symbols}
%\index{dots (ellipses)} \index{ellipses (dots)}
%\label{ams-dots}
%\begin{tabular}{*{2}{ll@{\hspace*{1.5cm}}}ll}
%\X\because$^*$   & \X[\cdots]\dotsi & \X\therefore$^*$ \\
%\X[\cdots]\dotsb & \X[\cdots]\dotsm &                  \\
%\X[\ldots]\dotsc & \X[\ldots]\dotso &                  \\
%\end{tabular}
%
%\bigskip
%
%\begin{tablenote}[*]
%  \docAuxCommand{because} and \docAuxCommand{therefore} are defined as binary
%  relations and therefore also appear in \vref{ams-rel}.
%\end{tablenote}
%
%\bigskip
%
%\begin{tablenote}
%  The \AmS \verb*|\dots| symbols are named
%  according to their intended usage: \cmdI[$\string\cdots$]{\dotsb}
%  between pairs of binary operators/relations,
%  \cmdI[$\string\ldots$]{\dotsc} between pairs of commas,
%  \cmdI[$\string\cdots$]{\dotsi} between pairs of integrals,
%  \cmdI[$\string\cdots$]{\dotsm} between pairs of multiplication
%  signs, and \cmdI[$\string\ldots$]{\dotso} between other symbol
%  pairs.
%\end{tablenote}
%\end{symtable}
%


%\begin{symtable}{WASY Dots}
%\idxboth{dot}{symbols}
%\label{wasy-dots}
%\begin{tabular}{ll}
%\K\wasytherefore
%\end{tabular}
%\end{symtable}



\begin{symtable}{Miscellaneous \latexe{} Math Symbols}
\idxboth{miscellaneous}{symbols}
\index{card suits}
\index{diamonds (suit)}
\index{hearts (suit)}
\index{clubs (suit)}
\index{spades (suit)}
\idxboth{musical}{symbols}
\index{dots (ellipses)}
\index{ellipses (dots)}
\index{null set}
\index{dotless i=dotless $i~(\imath)$>math mode}
\index{dotless j=dotless $j~(\jmath)$>math mode}
\index{angles}
\label{ord}
\AMSfalse
\ifAMS
  \def\AMSfn{$^\ddag$}
\else
  \def\AMSfn{}
\fi
\begin{tabular}{*4{ll}}
\X\aleph          & \X\Diamond$^*$    & \X\infty   & \X\prime     \\
\X\angle          & \X\diamondsuit    & \X\mho$^*$ & \X\sharp     \\
\X\backslash      & \X\emptyset\AMSfn & \X\nabla   & \X\spadesuit \\
\X\Box$^{*,\dag}$ & \X\flat           & \X\natural & \X\surd      \\
\X\clubsuit       & \X\heartsuit      & \X\neg     & \X\triangle  \\
\end{tabular}

\bigskip
\begin{tablenote}[*]
  Not predefined in \latexe.  Use one of the packages
  \pkgname{latexsym}, \pkgname{amsfonts}, \pkgname{amssymb},
  \pkgname{txfonts}, \pkgname{pxfonts}, or \pkgname{wasysym}.  Note,
  however, that \pkgname{amsfonts} and \pkgname{amssymb} define
  \docAuxCommand{Diamond} to produce the same glyph as
  the other packages produce a squarer \docAuxCommand{Diamond} as depicted above.
\end{tablenote}

\bigskip
\begin{tablenote}[\dag]
  To use \docAuxCommand{Box}---or any other symbol---as an end-of-proof
  (Q.E.D\@.)\index{Q.E.D.}\index{end of proof}\index{proof, end of}
  marker, consider using the \pkgname{ntheorem} package, which
  properly juxtaposes a symbol with the end of the proof text.
\end{tablenote}
\end{symtable}



\subsection{Miscellaneous Text-mode Math Symbols}

\subsection{Biological Symbols}
\begin{symtable}[MARV]{\MARV\ Biological Symbols}
\idxboth{biological}{symbols}
\index{male}
\index{female}
\label{marv-bio}
\begin{tabular}{*3{ll}ll}
\K\Female        & \K\FemaleMale    & \K\MALE          & \K\Neutral       \\
\K\FEMALE        & \K\Hermaphrodite & \K\Male          \\
\K\FemaleFemale  & \K\HERMAPHRODITE & \K\MaleMale      \\
\end{tabular}
\end{symtable}

\begin{symtable}[WASY]{\WASY\ Biological Symbols}
\index{male}
\index{female}
\label{wasy-bio}
\begin{tabular}{*2{ll}}
\K\female & \K\male \\
\end{tabular}
\end{symtable}

\begin{symtable}[MARV]{\MARV\ Safety-related Symbols}
\idxboth{safety-related}{symbols}
\label{marv-safety}
\begin{tabular}{*3{ll}ll}
\K\Biohazard     & \K\CEsign        & \K\Explosionsafe & \K\Radioactivity \\
\K\BSEfree       & \K\Estatically   & \indexlinearb\Laserbeam     & \K\Stopsign      \\
\end{tabular}
\end{symtable}

\idxbothend{scientific}{symbols}
\idxbothend{technological}{symbols}


\section{Dingbats}
\idxbothbegin{dingbat}{symbols}

Dingbats are symbols such as stars, arrows, and geometric shapes.
They are commonly used as bullets in itemized lists or, more
generally, as a means to draw attention to the text that follows.

The \PI\ dingbat package warrants special mention.  Among other
capabilities, \PI\ provides a \latex\ interface to the \PSfont{Zapf
Dingbats} font (one of the standard~35 \postscript\index{PostScript
fonts} fonts).  However, rather than name each of the dingbats
individually, \PI\ merely provides a single \cmd{\ding} command, which
outputs the character that lies at a given position in the font.  The
consequence is that the \PI\ symbols can't be listed by name in this
document's index, so be mindful of that fact when searching for a
particular symbol.

\bigskip


\begin{symtable}[DING]{\DING\ Arrows}
\label{bbding-arrows}
\begin{tabular}{*3{ll}}
\K\ArrowBoldDownRight    & \K\ArrowBoldRightShort  & \K\ArrowBoldUpRight \\
\K\ArrowBoldRightCircled & \K\ArrowBoldRightStrobe \\
\end{tabular}
\end{symtable}


\begin{symtable}[PI]{\PI\ Arrows}
\index{arrows}
\idxboth{fletched}{arrows}
\label{pi-arrows}
\begin{tabular}{*5{ll}}
\indexDing{212} & \indexDing{221} & \indexDing{230} & \indexDing{239} & \indexDing{249} \\
\indexDing{213} & \indexDing{222} & \indexDing{231} & \indexDing{241} & \indexDing{250} \\
\indexDing{214} & \indexDing{223} & \indexDing{232} & \indexDing{242} & \indexDing{251} \\
\indexDing{215} & \indexDing{224} & \indexDing{233} & \indexDing{243} & \indexDing{252} \\
\indexDing{216} & \indexDing{225} & \indexDing{234} & \indexDing{244} & \indexDing{253} \\
\indexDing{217} & \indexDing{226} & \indexDing{235} & \indexDing{245} & \indexDing{254} \\
\indexDing{218} & \indexDing{227} & \indexDing{236} & \indexDing{246} \\
\indexDing{219} & \indexDing{228} & \indexDing{237} & \indexDing{247} \\
\indexDing{220} & \indexDing{229} & \indexDing{238} & \indexDing{248} \\
\end{tabular}
\end{symtable}



\begin{symtable}[MARV]{\MARV\ Scissors}
\index{scissors}
\label{marv-scissors}
\begin{tabular}{*3{ll}}
\K\Cutleft       & \K\Cutright      & \indexlinearb\Leftscissors  \\
\K\Cutline       & \K\Kutline       & \K\Rightscissors \\
\end{tabular}
\end{symtable}


\begin{symtable}[DING]{\DING\ Scissors}
\index{scissors}
\label{scissors}
\begin{tabular}{*2{ll}}
\K\ScissorHollowLeft        & \K\ScissorLeftBrokenTop     \\
\K\ScissorHollowRight       & \K\ScissorRight             \\
\K\ScissorLeft              & \K\ScissorRightBrokenBottom \\
\K\ScissorLeftBrokenBottom  & \K\ScissorRightBrokenTop    \\
\end{tabular}
\end{symtable}


\begin{symtable}[PI]{\PI\ Scissors}
\index{scissors}
\label{pi-scissors}
\begin{tabular}{*4{ll}}
\indexDing{33} & \indexDing{34} & \indexDing{35} & \indexDing{36} \\
\end{tabular}
\end{symtable}

\begin{symtable}[DING]{\DING\ Pencils and Nibs}
\index{pencils}
\index{nibs}
\label{pencils-nibs}
\begin{tabular}{*3{ll}}
\K\NibLeft         & \K\PencilLeft      & \K\PencilRightDown \\
\K\NibRight        & \K\PencilLeftDown  & \K\PencilRightUp   \\
\K\NibSolidLeft    & \K\PencilLeftUp    \\
\K\NibSolidRight   & \K\PencilRight     \\
\end{tabular}
\end{symtable}


\begin{symtable}[PI]{\PI\ Pencils and Nibs}
\index{pencils}
\index{nibs}
\label{pi-pencils}
\begin{tabular}{*5{ll}}
\indexDing{46} & \indexDing{47} & \indexDing{48} & \indexDing{49} & \indexDing{50} \\
\end{tabular}
\end{symtable}

\begin{symtable}[DING]{\DING\ Fists}
\index{fists}
\label{hands}
\begin{tabular}{*3{ll}}
\K\HandCuffLeft    & \K\HandCuffRightUp & \K\HandPencilLeft  \\
\K\HandCuffLeftUp  & \K\HandLeft        & \K\HandRight       \\
\K\HandCuffRight   & \K\HandLeftUp      & \K\HandRightUp     \\
\end{tabular}
\end{symtable}


\begin{symtable}[PI]{\PI\ Fists}
\index{fists}
\label{pi-hands}
\begin{tabular}{*4{ll}}
\indexDing{42} & \indexDing{43} & \indexDing{44} & \indexDing{45} \\
\end{tabular}
\end{symtable}

\begin{symtable}[DING]{\DING\ Crosses and Plusses}
\index{crosses}
\index{plusses}
\index{crucifixes}
\label{crosses-plusses}
\begin{tabular}{*3{ll}}
\K[\dingCross]\Cross  & \K\CrossOpenShadow    & \K\PlusOutline        \\
\K\CrossBoldOutline   & \K\CrossOutline       & \K\PlusThinCenterOpen \\
\K\CrossClowerTips    & \K\Plus               \\
\K\CrossMaltese       & \K\PlusCenterOpen     \\
\end{tabular}
\end{symtable}


\begin{symtable}[PI]{\PI\ Crosses and Plusses}
\index{symbols>crosses}
\index{symbols>plusses}
\index{symbols>crucifixes}
\label{pi-crosses-plusses}
\begin{tabular}{*4{ll}}
\indexDing{57} & \indexDing{59} & \indexDing{61} & \indexDing{63} \\
\indexDing{58} & \indexDing{60} & \indexDing{62} & \indexDing{64} \\
\end{tabular}
\end{symtable}


\begin{symtable}[DING]{\DING\ Xs and Check Marks}
\index{symbols>check marks}
\index{symbols>Xs}
\label{ding-check-marks}
\begin{tabular}{*3{ll}}
\K\Checkmark     & \K\XSolid        & \K\XSolidBrush   \\
\K\CheckmarkBold & \K\XSolidBold    \\
\end{tabular}
\end{symtable}


\begin{symtable}[PI]{\PI\ Xs and Check Marks}
\index{check marks}
\index{Xs}
\label{pi-check-marks}
\begin{tabular}{*3{ll}}
\indexDing{51} & \indexDing{53} & \indexDing{55} \\
\indexDing{52} & \indexDing{54} & \indexDing{56} \\
\end{tabular}
\end{symtable}


\begin{symtable}[WASY]{\WASY\ Xs and Check Marks}
\index{check marks}
\index{Xs}
\label{wasy-check-marks}
\begin{tabular}{*6l}
\K\CheckedBox & \K\Square & \K\XBox \\
\end{tabular}
\end{symtable}


\begin{symtable}[PI]{\PI\ Circled Numbers}
\index{circled numbers}
\index{numbers>circled}
\label{circled-numbers}
\begin{tabular}{*4{ll}}
\indexDing{172} & \indexDing{182} & \indexDing{192} & \indexDing{202} \\
\indexDing{173} & \indexDing{183} & \indexDing{193} & \indexDing{203} \\
\indexDing{174} & \indexDing{184} & \indexDing{194} & \indexDing{204} \\
\indexDing{175} & \indexDing{185} & \indexDing{195} & \indexDing{205} \\
\indexDing{176} & \indexDing{186} & \indexDing{196} & \indexDing{206} \\
\indexDing{177} & \indexDing{187} & \indexDing{197} & \indexDing{207} \\
\indexDing{178} & \indexDing{188} & \indexDing{198} & \indexDing{208} \\
\indexDing{179} & \indexDing{189} & \indexDing{199} & \indexDing{209} \\
\indexDing{180} & \indexDing{190} & \indexDing{200} & \indexDing{210} \\
\indexDing{181} & \indexDing{191} & \indexDing{201} & \indexDing{211} \\
\end{tabular}

\bigskip

\begin{tablenote}
  \PI\ (part of the \pkgname{psnfss} package) provides a
  \cmd{dingautolist} environment which resembles \texttt{enumerate}
  but uses circled numbers as bullets.\footnotemark{}
  \seedocs{\pkgname{psnfss}}.
\end{tablenote}
\end{symtable}
\footnotetext{In fact, \cmd{\dingautolist} can use any set of
  consecutive \PSfont{Zapf Dingbats} symbols.}


\begin{symtable}[WASY]{\WASY\ Stars}
\index{stars}
\index{Jewish star}\index{Star of David}
\label{wasy-stars}
\begin{tabular}{*6l}
\K\davidsstar & \K\hexstar & \K\varhexstar
\end{tabular}
\end{symtable}


\begin{symtable}[DING]{\DING\ Stars, Flowers, and Similar Shapes}
\index{asterisks}
\index{clovers}
\index{flowers}
\index{ornaments}
\index{sparkles}
\index{snowflakes}
\index{stars}
\index{Jewish star}\index{Star of David}
\label{star-like}
\begin{tabular}{*3{ll}}
\K\Asterisk                & \K\FiveFlowerPetal      & \K\JackStar                  \\
\K\AsteriskBold            & \K\FiveStar             & \K\JackStarBold              \\
\K\AsteriskCenterOpen      & \K\FiveStarCenterOpen   & \K\SixFlowerAlternate        \\
\K\AsteriskRoundedEnds     & \K\FiveStarConvex       & \K\SixFlowerAltPetal         \\
\K\AsteriskThin            & \K\FiveStarLines        & \K\SixFlowerOpenCenter       \\
\K\AsteriskThinCenterOpen  & \K\FiveStarOpen         & \K\SixFlowerPetalDotted      \\
\K\DavidStar               & \K\FiveStarOpenCircled  & \K\SixFlowerPetalRemoved     \\
\K\DavidStarSolid          & \K\FiveStarOpenDotted   & \K\SixFlowerRemovedOpenPetal \\
\K\EightAsterisk           & \K\FiveStarOutline      & \K\SixStar                   \\
\K\EightFlowerPetal        & \K\FiveStarOutlineHeavy & \K\SixteenStarLight          \\
\K\EightFlowerPetalRemoved & \K\FiveStarShadow       & \K\Snowflake                 \\
\K\EightStar               & \K\FourAsterisk         & \K\SnowflakeChevron          \\
\K\EightStarBold           & \K\FourClowerOpen       & \K\SnowflakeChevronBold      \\
\K\EightStarConvex         & \K\FourClowerSolid      & \K\Sparkle                   \\
\K\EightStarTaper          & \K\FourStar             & \K\SparkleBold               \\
\K\FiveFlowerOpen          & \K\FourStarOpen         & \K\TwelweStar                \\
\end{tabular}
\end{symtable}

\begin{symtable}[WASY]{\WASY\ Geometric Shapes}
\index{polygons}
\index{geometric shapes}
\label{wasy-geometrical}
\begin{tabular}{*8l}
\K\hexagon & \K\octagon & \K\pentagon & \K\varhexagon
\end{tabular}
\end{symtable}

\begin{symtable}[DING]{\DING\ Geometric Shapes}
\index{circles}
\index{diamonds}
\index{ellipses (ovals)}
\index{geometric shapes}
\index{ovals}
\index{rectangles}
\index{squares}
\index{triangles}
\label{ding-geometrical}
\begin{tabular}{*3{ll}}
\K\CircleShadow    & \K\Rectangle                   & \K\SquareShadowTopLeft     \\
\K\CircleSolid     & \K\RectangleBold               & \K\SquareShadowTopRight    \\
\K\DiamondSolid    & \K\RectangleThin               & \K\SquareSolid             \\
\K\Ellipse         & \K[\dingSquare]\Square         & \K\TriangleDown            \\
\K\EllipseShadow   & \K\SquareCastShadowBottomRight & \K\TriangleUp              \\
\K\EllipseSolid    & \K\SquareCastShadowTopLeft     \\
\K\HalfCircleLeft  & \K\SquareCastShadowTopRight    \\
\K\HalfCircleRight & \K\SquareShadowBottomRight     \\
\end{tabular}
\end{symtable}


\begin{symtable}[PI]{\PI\ Geometric Shapes}
\index{circles}
\index{diamonds}
\index{geometric shapes}
\index{rectangles}
\index{squares}
\index{triangles}
\label{pi-geometrical}
\begin{tabular}{*5{ll}}
\indexDing{108} & \indexDing{111} & \indexDing{114} & \indexDing{117} & \indexDing{121} \\
\indexDing{109} & \indexDing{112} & \indexDing{115} & \indexDing{119} & \indexDing{122} \\
\indexDing{110} & \indexDing{113} & \indexDing{116} & \indexDing{120} \\
\end{tabular}
\end{symtable}\begin{symtable}[DING]{Miscellaneous \DING\ Dingbats}
\idxboth{miscellaneous}{symbols}
\index{envelopes}
\label{bbding-misc}
\begin{tabular}{*4{ll}}
\K\Envelope             & \K\Peace & \K\PhoneHandset & \K\SunshineOpenCircled \\
\K\OrnamentDiamondSolid & \K\Phone & \K\Plane        & \K\Tape                \\
\end{tabular}
\end{symtable}


\begin{symtable}[PI]{Miscellaneous \PI\ Dingbats}
\idxboth{miscellaneous}{symbols}
\index{card suits}
\index{diamonds (suit)}
\index{hearts (suit)}
\index{clubs (suit)}
\index{spades (suit)}
\index{fleurons}
\index{leaves}
\index{ornaments}
\label{pi-misc}
\begin{tabular}{*5{ll}}
\indexDing{37} & \indexDing{40}  & \indexDing{164} & \indexDing{167} & \indexDing{171} \\
\indexDing{38} & \indexDing{41}  & \indexDing{165} & \indexDing{168} & \indexDing{169} \\
\indexDing{39} & \indexDing{118} & \indexDing{166} & \indexDing{170} \\
\end{tabular}
\end{symtable}
\idxbothend{dingbat}{symbols}

\begin{symtable}{\TC\ Genealogical Symbols}
\idxboth{genealogical}{symbols}
\label{genealogical}
\begin{tabular}{*3{ll}}
\K\textborn     & \K\textdivorced & \K\textmarried  \\
\K\textdied     & \K\textleaf     \\
\end{tabular}
\end{symtable}


\begin{symtable}[WASY]{\WASY\ General Symbols}
\index{symbols>general}
\index{smiley faces}
\index{frowny faces}
\index{faces}
\idxboth{clock}{symbols}
\index{check marks}
\label{wasy-general}
\begin{tabular}{*4{ll}}
\K\ataribox    & \K[\WASYclock]\clock & \indexlinearb\LEFTarrow  & \K\smiley      \\
\K\bell        & \K\diameter          & \K\lightning  & \K\sun         \\
\K\blacksmiley & \K\DOWNarrow         & \K\phone      & \K\UParrow     \\
\K\Bowtie      & \K\frownie           & \K\pointer    & \K\wasylozenge \\
\K\brokenvert  & \K\invdiameter       & \K\recorder                    \\
\K\checked     & \K\kreuz             & \K\RIGHTarrow                  \\
\end{tabular}
\end{symtable}


\begin{symtable}[WASY]{\WASY\ Circles}
\index{circles}
\label{wasy-circles}
\begin{tabular}{*8l}
\K\CIRCLE         & \indexlinearb\LEFTcircle     & \K\RIGHTcircle    & \K\rightturn      \\
\K\Circle         & \indexlinearb\Leftcircle     & \K\Rightcircle    \\
\indexlinearb\LEFTCIRCLE     & \K\RIGHTCIRCLE    & \K\leftturn       \\
\end{tabular}
\end{symtable}


\begin{symtable}[WASY]{\WASY\ Musical Symbols}
\idxboth{musical}{symbols}
\label{wasy-music}
\begin{tabular}{*{10}l}
\K\eighthnote & \K\halfnote    & \K\twonotes &
\K\fullnote   & \K\quarternote \\
\end{tabular}

\bigskip
\begin{tablenote}
  See also \docAuxCommand{flat}, \docAuxCommand{sharp}, and \docAuxCommand{natural}
  (\vref*{ord}).
\end{tablenote}
\end{symtable}

\begin{symtable}[MARV]{\MARV\ Navigation Symbols}
\idxboth{navigation}{symbols}
\label{marv-navigation}
\begin{tabular}{*3{ll}ll}
\K\Forward        & \K\MoveDown  & \K\RewindToIndex  & \K\ToTop \\
\K\ForwardToEnd   & \K\MoveUp    & \K\RewindToStart  \\
\K\ForwardToIndex & \K\Rewind    & \K\ToBottom       \\
\end{tabular}
\end{symtable}


\begin{symtable}[MARV]{\MARV\ Laundry Symbols}
\idxboth{laundry}{symbols}
\label{marv-laundry}
\begin{tabular}{*3{ll}}
\K\AtForty            & \K\Handwash           & \K\ShortNinetyFive    \\
\K\AtNinetyFive       & \K\IroningI           & \K\ShortSixty         \\
\K\AtSixty            & \K\IroningII          & \K\ShortThirty        \\
\K\Bleech             & \K\IroningIII         & \K\SpecialForty       \\
\K\CleaningA          & \K\NoBleech           & \K\Tumbler            \\
\K\CleaningF          & \K\NoChemicalCleaning & \K\WashCotton         \\
\K\CleaningFF         & \K\NoIroning          & \K\WashSynthetics     \\
\K\CleaningP          & \K\NoTumbler          & \K\WashWool           \\
\K\CleaningPP         & \K\ShortFifty         \\
\K\Dontwash           & \K\ShortForty         \\
\end{tabular}
\end{symtable}


\begin{symtable}[MARV]{\MARV\ Information Symbols}
\idxboth{information}{symbols}
\index{check marks}
\index{Xs}
\idxboth{clock}{symbols}
\label{marv-info}
\begin{tabular}{*3{ll}ll}
\K\Bicycle      & \K\Football     & \K\Pointinghand \\
\K\Checkedbox   & \K\Gentsroom    & \K\Wheelchair   \\
\K\Clocklogo    & \K\Industry     & \K\Writinghand  \\
\K\Coffeecup    & \K\Info         \\
\K\Crossedbox   & \indexlinearb\Ladiesroom   \\
\end{tabular}
\end{symtable}


\begin{symtable}[MARV]{Other \MARV\ Symbols}
\idxboth{miscellaneous}{symbols}
\index{crosses}
\index{crucifixes}
\index{smiley faces}
\index{frowny faces}
\index{faces}
\index{man}
\index{woman}
\index{globe}
\index{world}
\label{marv-other}
\begin{tabular}{*4{ll}}
\K\Ankh        & \K\Cross        & \K\Heart       & \K\Smiley      \\
\K\Bat         & \K\FHBOlogo     & \K\MartinVogel & \K\Womanface   \\
\K\Bouquet     & \K\FHBOLOGO     & \K\Mundus      & \K\Yinyang     \\
\K\Celtcross   & \K\Frowny       & \K\MVAt                         \\
\K\CircledA    & \K\FullFHBO     & \K\MVRightarrow                 \\
\end{tabular}
\end{symtable}

\section{Alphabets}

\begin{symtable}[CYPR]{\CYPR\ Cypriot Letters}
\index{Cypriot}
\index{alphabets>Cypriot}
\label{cypriot}
\begin{tabular}{*5{ll@{\qquad}}ll}
\Kcyp[{\Ca}]\Ca   & \Kcyp[{\Cku}]\Cku & \Kcyp[{\Cmu}]\Cmu & \Kcyp[{\Cpo}]\Cpo & \Kcyp[{\Cso}]\Cso & \Kcyp[{\Cwi}]\Cwi \\
\Kcyp[{\Ce}]\Ce   & \Kcyp[{\Cla}]\Cla & \Kcyp[{\Cna}]\Cna & \Kcyp[{\Cpu}]\Cpu & \Kcyp[{\Csu}]\Csu & \Kcyp[{\Cwo}]\Cwo \\
\Kcyp[{\Cga}]\Cga & \Kcyp[{\Cle}]\Cle & \Kcyp[{\Cne}]\Cne & \Kcyp[{\Cra}]\Cra & \Kcyp[{\Cta}]\Cta & \Kcyp[{\Cxa}]\Cxa \\
\Kcyp[{\Ci}]\Ci   & \Kcyp[{\Cli}]\Cli & \Kcyp[{\Cni}]\Cni & \Kcyp[{\Cre}]\Cre & \Kcyp[{\Cte}]\Cte & \Kcyp[{\Cxe}]\Cxe \\
\Kcyp[{\Cja}]\Cja & \Kcyp[{\Clo}]\Clo & \Kcyp[{\Cno}]\Cno & \Kcyp[{\Cri}]\Cri & \Kcyp[{\Cti}]\Cti & \Kcyp[{\Cya}]\Cya \\
\Kcyp[{\Cjo}]\Cjo & \Kcyp[{\Clu}]\Clu & \Kcyp[{\Cnu}]\Cnu & \Kcyp[{\Cro}]\Cro & \Kcyp[{\Cto}]\Cto & \Kcyp[{\Cyo}]\Cyo \\
\Kcyp[{\Cka}]\Cka & \Kcyp[{\Cma}]\Cma & \Kcyp[{\Co}]\Co   & \Kcyp[{\Cru}]\Cru & \Kcyp[{\Ctu}]\Ctu & \Kcyp[{\Cza}]\Cza \\
\Kcyp[{\Cke}]\Cke & \Kcyp[{\Cme}]\Cme & \Kcyp[{\Cpa}]\Cpa & \Kcyp[{\Csa}]\Csa & \Kcyp[{\Cu}]\Cu   & \Kcyp[{\Czo}]\Czo \\
\Kcyp[{\Cki}]\Cki & \Kcyp[{\Cmi}]\Cmi & \Kcyp[{\Cpe}]\Cpe & \Kcyp[{\Cse}]\Cse & \Kcyp[{\Cwa}]\Cwa &                         \\
\Kcyp[{\Cko}]\Cko & \Kcyp[{\Cmo}]\Cmo & \Kcyp[{\Cpi}]\Cpi & \Kcyp[{\Csi}]\Csi & \Kcyp[{\Cwe}]\Cwe &                         \\
\end{tabular}

\bigskip
\begin{tablenote}
  \usefontcmdmessage{}{\cyprfamily}.  Single-character
  shortcuts are also supported: Both
  ``\verb+{\Cpa\Cki\Cna}+'' and ``\verb+{pcn}+''
  produce ``{pcn}'', for example.  \seedocs{\CYPR}.
\end{tablenote}
\end{symtable}


\begin{symtable}[PRSN]{\PRSN\ Cuneiform Letters}
\index{cuneiform}
\index{alphabets>Old Persian (cuneiform)}
\label{oldprsn}
\begin{tabular}{*4{ll@{\qquad}}ll}
\indexoldpersian[\textcopsn{\Oa}]\Oa     & \indexoldpersian[\textcopsn{\Oga}]\Oga   & \indexoldpersian[\textcopsn{\Ola}]\Ola   & \indexoldpersian[\textcopsn{\Oru}]\Oru   & \indexoldpersian[\textcopsn{\Ovi}]\Ovi   \\
\indexoldpersian[\textcopsn{\Oba}]\Oba   & \indexoldpersian[\textcopsn{\Ogu}]\Ogu   & \indexoldpersian[\textcopsn{\Oma}]\Oma   & \indexoldpersian[\textcopsn{\Osa}]\Osa   & \indexoldpersian[\textcopsn{\Oxa}]\Oxa   \\
\indexoldpersian[\textcopsn{\Oca}]\Oca   & \indexoldpersian[\textcopsn{\Oha}]\Oha   & \indexoldpersian[\textcopsn{\Omi}]\Omi   & \indexoldpersian[\textcopsn{\Osva}]\Osva & \indexoldpersian[\textcopsn{\Oya}]\Oya   \\
\indexoldpersian[\textcopsn{\Occa}]\Occa & \indexoldpersian[\textcopsn{\Oi}]\Oi     & \indexoldpersian[\textcopsn{\Omu}]\Omu   & \indexoldpersian[\textcopsn{\Ota}]\Ota   & \indexoldpersian[\textcopsn{\Oza}]\Oza   \\
\indexoldpersian[\textcopsn{\Oda}]\Oda   & \indexoldpersian[\textcopsn{\Oja}]\Oja   & \indexoldpersian[\textcopsn{\Ona}]\Ona   & \indexoldpersian[\textcopsn{\Otha}]\Otha &                            \\
\indexoldpersian[\textcopsn{\Odi}]\Odi   & \indexoldpersian[\textcopsn{\Oji}]\Oji   & \indexoldpersian[\textcopsn{\Onu}]\Onu   & \indexoldpersian[\textcopsn{\Otu}]\Otu   &                            \\
\indexoldpersian[\textcopsn{\Odu}]\Odu   & \indexoldpersian[\textcopsn{\Oka}]\Oka   & \indexoldpersian[\textcopsn{\Opa}]\Opa   & \indexoldpersian[\textcopsn{\Ou}]\Ou     &                            \\
\indexoldpersian[\textcopsn{\Ofa}]\Ofa   & \indexoldpersian[\textcopsn{\Oku}]\Oku   & \indexoldpersian[\textcopsn{\Ora}]\Ora   & \indexoldpersian[\textcopsn{\Ova}]\Ova   &                            \\
\end{tabular}

\bigskip
\begin{tablenote}
  \usefontcmdmessage{\textcopsn}{\copsnfamily}.  Single-character
  shortcuts are also supported: Both
  ``\verb+\textcopsn{\Opa\Oka\Ona}+'' and ``\verb+\textcopsn{pkn}+''
  produce ``\textcopsn{pkn}'', for example.  \seedocs{\PRSN}.
\end{tablenote}
\end{symtable}


\begin{symtable}[PRSN]{\PRSN\ Cuneiform Numerals}
\index{cuneiform}
\index{numerals>cuneiform}
\label{oldprsn-nums}
\begin{tabular}{*4{ll@{\qquad}}ll}
\indexoldpersian[\textcopsn{\Oone}]\Oone & \indexoldpersian[\textcopsn{\Otwo}]\Otwo & \indexoldpersian[\textcopsn{\Oten}]\Oten & \indexoldpersian[\textcopsn{\Otwenty}]\Otwenty & \indexoldpersian[\textcopsn{\Ohundred}]\Ohundred \\
\end{tabular}

\bigskip
\begin{tablenote}
  \usefontcmdmessage{\textcopsn}{\copsnfamily}.
\end{tablenote}
\end{symtable}


\begin{symtable}[PRSN]{\PRSN\ Cuneiform Words}
\index{cuneiform}
\label{oldprsn-objs}
\begin{tabular}{*3{ll@{\qquad}}ll}
\indexoldpersian[\textcopsn{\OAura}]\OAura         & \indexoldpersian[\textcopsn{\Ocountrya}]\Ocountrya & \indexoldpersian[\textcopsn{\Ogod}]\Ogod           &                                      \\
\indexoldpersian[\textcopsn{\OAurb}]\OAurb         & \indexoldpersian[\textcopsn{\Ocountryb}]\Ocountryb & \indexoldpersian[\textcopsn{\Oking}]\Oking         &                                      \\
\indexoldpersian[\textcopsn{\OAurc}]\OAurc         & \indexoldpersian[\textcopsn{\Oearth}]\Oearth       & \indexoldpersian[\textcopsn{\Owd}]\Owd             &                                      \\
\end{tabular}

\bigskip
\begin{tablenote}
  \usefontcmdmessage{\textcopsn}{\copsnfamily}.
\end{tablenote}
\end{symtable}

\subsection{Ugaritic}

\begin{symtable}[UGAR]{\UGAR\ Cuneiform Letters}
\index{cuneiform}
\index{alphabets>Ugarite (cuneiform)}
\label{ugarite}
\begin{tabular}{*4{ll@{\qquad}}ll}
\indexugar[\textcugar{\Arq}]\Arq & \indexugar[\textcugar{\Az}]\Az   & \indexugar[\textcugar{\Am}]\Am   & \indexugar[\textcugar{\Asd}]\Asd & \indexugar[\textcugar{\Au}]\Au   \\
\indexugar[\textcugar{\Ab}]\Ab   & \indexugar[\textcugar{\Ahd}]\Ahd & \indexugar[\textcugar{\Adb}]\Adb & \indexugar[\textcugar{\Aq}]\Aq   & \indexugar[\textcugar{\Asg}]\Asg \\
\indexugar[\textcugar{\Ag}]\Ag   & \indexugar[\textcugar{\Atd}]\Atd & \indexugar[\textcugar{\An}]\An   & \indexugar[\textcugar{\Ar}]\Ar   & \indexugar[\textcugar{\Awd}]\Awd \\
\indexugar[\textcugar{\Ahu}]\Ahu & \indexugar[\textcugar{\Ay}]\Ay   & \indexugar[\textcugar{\Azd}]\Azd & \indexugar[\textcugar{\Atb}]\Atb &                          \\
\indexugar[\textcugar{\Ad}]\Ad   & \indexugar[\textcugar{\Ak}]\Ak   & \indexugar[\textcugar{\As}]\As   & \indexugar[\textcugar{\Agd}]\Agd &                          \\
\indexugar[\textcugar{\Ah}]\Ah   & \indexugar[\textcugar{\Asa}]\Asa & \indexugar[\textcugar{\Alq}]\Alq & \indexugar[\textcugar{\At}]\At   &                          \\
\indexugar[\textcugar{\Aw}]\Aw   & \indexugar[\textcugar{\Al}]\Al   & \indexugar[\textcugar{\Ap}]\Ap   & \indexugar[\textcugar{\Ai}]\Ai   &                          \\
\end{tabular}

\bigskip
\begin{tablenote}
  \usefontcmdmessage{\textcugar}{\cugarfamily}.  Single-character
  shortcuts and various aliases are also supported:
  ``\verb+\textcopsn{\Ap\Aq\An}+'',
  ``\verb+\textcopsn{\Ape\Aqoph\Anun}+'', and
  ``\verb+\textcopsn{pqn}+'' all produce ``\textcopsn{pqn}'', for
  example.  \seedocs{\UGAR}.
\end{tablenote}
\end{symtable}


\begin{longsymtable}[SARAB]{\SARAB\ South Arabian Letters}
\index{South Arabian alphabet}
\index{alphabets>South Arabian}
\label{sarabian}
\begin{longtable}{*4{ll@{\qquad}}ll}
\indexsoutharabian[\textsarab{\SAa}]\SAa   & \indexsoutharabian[\textsarab{\SAz}]\SAz   & \indexsoutharabian[\textsarab{\SAm}]\SAm   & \indexsoutharabian[\textsarab{\SAsd}]\SAsd & \indexsoutharabian[\textsarab{\SAdb}]\SAdb \\
\indexsoutharabian[\textsarab{\SAb}]\SAb   & \indexsoutharabian[\textsarab{\SAhd}]\SAhd & \indexsoutharabian[\textsarab{\SAn}]\SAn   & \indexsoutharabian[\textsarab{\SAq}]\SAq   & \indexsoutharabian[\textsarab{\SAtb}]\SAtb \\
\indexsoutharabian[\textsarab{\SAg}]\SAg   & \indexsoutharabian[\textsarab{\SAtd}]\SAtd & \indexsoutharabian[\textsarab{\SAs}]\SAs   & \indexsoutharabian[\textsarab{\SAr}]\SAr   & \indexsoutharabian[\textsarab{\SAga}]\SAga \\
\indexsoutharabian[\textsarab{\SAd}]\SAd   & \indexsoutharabian[\textsarab{\SAy}]\SAy   & \indexsoutharabian[\textsarab{\SAf}]\SAf   & \indexsoutharabian[\textsarab{\SAsv}]\SAsv & \indexsoutharabian[\textsarab{\SAzd}]\SAzd \\
\indexsoutharabian[\textsarab{\SAh}]\SAh   & \indexsoutharabian[\textsarab{\SAk}]\SAk   & \indexsoutharabian[\textsarab{\SAlq}]\SAlq & \indexsoutharabian[\textsarab{\SAt}]\SAt   & \indexsoutharabian[\textsarab{\SAsa}]\SAsa \\
\indexsoutharabian[\textsarab{\SAw}]\SAw   & \indexsoutharabian[\textsarab{\SAl}]\SAl   & \indexsoutharabian[\textsarab{\SAo}]\SAo   & \indexsoutharabian[\textsarab{\SAhu}]\SAhu & \indexsoutharabian[\textsarab{\SAdd}]\SAdd \\
\end{longtable}

\bigskip
\begin{tablenote}
  \usefontcmdmessage{\textsarab}{\sarabfamily}.  Single-character
  shortcuts are also supported: Both
  ``\verb+\textsarab{\SAb\SAk\SAn}+'' and ``\verb+\textsarab{bkn}+''
  produce ``\textsarab{bkn}'', for example.  \seedocs{\SARAB}.
\end{tablenote}
\end{longsymtable}

\begin{longsymtable}[LINA]{\LINA\ Linear~A Script}
\index{Linear A}
\index{alphabets>Linear A}
\label{linearA}
\begin{longtable}{*3{ll@{\quad}}ll}
\multicolumn{8}{l}{\small\textit{(continued from previous page)}} \\[1ex]
\endhead
\endfirsthead
\\[3ex]
\multicolumn{8}{r}{\small\textit{(continued on next page)}}
\endfoot
\endlastfoot
\indexlinearb\LinearAI           & \indexlinearb\LinearAXCIX        & \indexlinearb\LinearACXCVII      & \indexlinearb\LinearACCXCV       \\
\indexlinearb\LinearAII          & \indexlinearb\LinearAC           & \indexlinearb\LinearACXCVIII     & \indexlinearb\LinearACCXCVI      \\
\indexlinearb\LinearAIII         & \indexlinearb\LinearACI          & \indexlinearb\LinearACXCIX       & \indexlinearb\LinearACCXCVII     \\
\indexlinearb\LinearAIV          & \indexlinearb\LinearACII         & \indexlinearb\LinearACC          & \indexlinearb\LinearACCXCVIII    \\
\indexlinearb\LinearAV           & \indexlinearb\LinearACIII        & \indexlinearb\LinearACCI         & \indexlinearb\LinearACCXCIX      \\
\indexlinearb\LinearAVI          & \indexlinearb\LinearACIV         & \indexlinearb\LinearACCII        & \indexlinearb\LinearACCC         \\
\indexlinearb\LinearAVII         & \indexlinearb\LinearACV          & \indexlinearb\LinearACCIII       & \indexlinearb\LinearACCCI        \\
\indexlinearb\LinearAVIII        & \indexlinearb\LinearACVI         & \indexlinearb\LinearACCIV        & \indexlinearb\LinearACCCII       \\
\indexlinearb\LinearAIX          & \indexlinearb\LinearACVII        & \indexlinearb\LinearACCV         & \indexlinearb\LinearACCCIII      \\
\indexlinearb\LinearAX           & \indexlinearb\LinearACVIII       & \indexlinearb\LinearACCVI        & \indexlinearb\LinearACCCIV       \\
\indexlinearb\LinearAXI          & \indexlinearb\LinearACIX         & \indexlinearb\LinearACCVII       & \indexlinearb\LinearACCCV        \\
\indexlinearb\LinearAXII         & \indexlinearb\LinearACX          & \indexlinearb\LinearACCVIII      & \indexlinearb\LinearACCCVI       \\
\indexlinearb\LinearAXIII        & \indexlinearb\LinearACXI         & \indexlinearb\LinearACCIX        & \indexlinearb\LinearACCCVII      \\
\indexlinearb\LinearAXIV         & \indexlinearb\LinearACXII        & \indexlinearb\LinearACCX         & \indexlinearb\LinearACCCVIII     \\
\indexlinearb\LinearAXV          & \indexlinearb\LinearACXIII       & \indexlinearb\LinearACCXI        & \indexlinearb\LinearACCCIX       \\
\indexlinearb\LinearAXVI         & \indexlinearb\LinearACXIV        & \indexlinearb\LinearACCXII       & \indexlinearb\LinearACCCX        \\
\indexlinearb\LinearAXVII        & \indexlinearb\LinearACXV         & \indexlinearb\LinearACCXIII      & \indexlinearb\LinearACCCXI       \\
\indexlinearb\LinearAXVIII       & \indexlinearb\LinearACXVI        & \indexlinearb\LinearACCXIV       & \indexlinearb\LinearACCCXII      \\
\indexlinearb\LinearAXIX         & \indexlinearb\LinearACXVII       & \indexlinearb\LinearACCXV        & \indexlinearb\LinearACCCXIII     \\
\indexlinearb\LinearAXX          & \indexlinearb\LinearACXVIII      & \indexlinearb\LinearACCXVI       & \indexlinearb\LinearACCCXIV      \\
\indexlinearb\LinearAXXI         & \indexlinearb\LinearACXIX        & \indexlinearb\LinearACCXVII      & \indexlinearb\LinearACCCXV       \\
\indexlinearb\LinearAXXII        & \indexlinearb\LinearACXX         & \indexlinearb\LinearACCXVIII     & \indexlinearb\LinearACCCXVI      \\
\indexlinearb\LinearAXXIII       & \indexlinearb\LinearACXXI        & \indexlinearb\LinearACCXIX       & \indexlinearb\LinearACCCXVII     \\
\indexlinearb\LinearAXXIV        & \indexlinearb\LinearACXXII       & \indexlinearb\LinearACCXX        & \indexlinearb\LinearACCCXVIII    \\
\indexlinearb\LinearAXXV         & \indexlinearb\LinearACXXIII      & \indexlinearb\LinearACCXXI       & \indexlinearb\LinearACCCXIX      \\
\indexlinearb\LinearAXXVI        & \indexlinearb\LinearACXXIV       & \indexlinearb\LinearACCXXII      & \indexlinearb\LinearACCCXX       \\
\indexlinearb\LinearAXXVII       & \indexlinearb\LinearACXXV        & \indexlinearb\LinearACCXXIII     & \indexlinearb\LinearACCCXXI      \\
\indexlinearb\LinearAXXVIII      & \indexlinearb\LinearACXXVI       & \indexlinearb\LinearACCXXIV      & \indexlinearb\LinearACCCXXII     \\
\indexlinearb\LinearAXXIX        & \indexlinearb\LinearACXXVII      & \indexlinearb\LinearACCXXV       & \indexlinearb\LinearACCCXXIII    \\
\indexlinearb\LinearAXXX         & \indexlinearb\LinearACXXVIII     & \indexlinearb\LinearACCXXVI      & \indexlinearb\LinearACCCXXIV     \\
\indexlinearb\LinearAXXXI        & \indexlinearb\LinearACXXIX       & \indexlinearb\LinearACCXXVII     & \indexlinearb\LinearACCCXXV      \\
\indexlinearb\LinearAXXXII       & \indexlinearb\LinearACXXX        & \indexlinearb\LinearACCXXVIII    & \indexlinearb\LinearACCCXXVI     \\
\indexlinearb\LinearAXXXIII      & \indexlinearb\LinearACXXXI       & \indexlinearb\LinearACCXXIX      & \indexlinearb\LinearACCCXXVII    \\
\indexlinearb\LinearAXXXIV       & \indexlinearb\LinearACXXXII      & \indexlinearb\LinearACCXXX       & \indexlinearb\LinearACCCXXVIII   \\
\indexlinearb\LinearAXXXV        & \indexlinearb\LinearACXXXIII     & \indexlinearb\LinearACCXXXI      & \indexlinearb\LinearACCCXXIX     \\
\indexlinearb\LinearAXXXVI       & \indexlinearb\LinearACXXXIV      & \indexlinearb\LinearACCXXXII     & \indexlinearb\LinearACCCXXX      \\
\indexlinearb\LinearAXXXVII      & \indexlinearb\LinearACXXXV       & \indexlinearb\LinearACCXXXIII    & \indexlinearb\LinearACCCXXXI     \\
\indexlinearb\LinearAXXXVIII     & \indexlinearb\LinearACXXXVI      & \indexlinearb\LinearACCXXXIV     & \indexlinearb\LinearACCCXXXII    \\
\indexlinearb\LinearAXXXIX       & \indexlinearb\LinearACXXXVII     & \indexlinearb\LinearACCXXXV      & \indexlinearb\LinearACCCXXXIII   \\
\indexlinearb\LinearAXL          & \indexlinearb\LinearACXXXVIII    & \indexlinearb\LinearACCXXXVI     & \indexlinearb\LinearACCCXXXIV    \\
\indexlinearb\LinearAXLI         & \indexlinearb\LinearACXXXIX      & \indexlinearb\LinearACCXXXVII    & \indexlinearb\LinearACCCXXXV     \\
\indexlinearb\LinearAXLII        & \indexlinearb\LinearACXL         & \indexlinearb\LinearACCXXXVIII   & \indexlinearb\LinearACCCXXXVI    \\
\indexlinearb\LinearAXLIII       & \indexlinearb\LinearACXLI        & \indexlinearb\LinearACCXXXIX     & \indexlinearb\LinearACCCXXXVII   \\
\indexlinearb\LinearAXLIV        & \indexlinearb\LinearACXLII       & \indexlinearb\LinearACCXL        & \indexlinearb\LinearACCCXXXVIII  \\
\indexlinearb\LinearAXLV         & \indexlinearb\LinearACXLIII      & \indexlinearb\LinearACCXLI       & \indexlinearb\LinearACCCXXXIX    \\
\indexlinearb\LinearAXLVI        & \indexlinearb\LinearACXLIV       & \indexlinearb\LinearACCXLII      & \indexlinearb\LinearACCCXL       \\
\indexlinearb\LinearAXLVII       & \indexlinearb\LinearACXLV        & \indexlinearb\LinearACCXLIII     & \indexlinearb\LinearACCCXLI      \\
\indexlinearb\LinearAXLVIII      & \indexlinearb\LinearACXLVI       & \indexlinearb\LinearACCXLIV      & \indexlinearb\LinearACCCXLII     \\
\indexlinearb\LinearAXLIX        & \indexlinearb\LinearACXLVII      & \indexlinearb\LinearACCXLV       & \indexlinearb\LinearACCCXLIII    \\
\indexlinearb\LinearAL           & \indexlinearb\LinearACXLVIII     & \indexlinearb\LinearACCXLVI      & \indexlinearb\LinearACCCXLIV     \\
\indexlinearb\LinearALI          & \indexlinearb\LinearACXLIX       & \indexlinearb\LinearACCXLVII     & \indexlinearb\LinearACCCXLV      \\
\indexlinearb\LinearALII         & \indexlinearb\LinearACL          & \indexlinearb\LinearACCXLVIII    & \indexlinearb\LinearACCCXLVI     \\
\indexlinearb\LinearALIII        & \indexlinearb\LinearACLI         & \indexlinearb\LinearACCXLIX      & \indexlinearb\LinearACCCXLVII    \\
\indexlinearb\LinearALIV         & \indexlinearb\LinearACLII        & \indexlinearb\LinearACCL         & \indexlinearb\LinearACCCXLVIII   \\
\indexlinearb\LinearALV          & \indexlinearb\LinearACLIII       & \indexlinearb\LinearACCLI        & \indexlinearb\LinearACCCXLIX     \\
\indexlinearb\LinearALVI         & \indexlinearb\LinearACLIV        & \indexlinearb\LinearACCLII       & \indexlinearb\LinearACCCL        \\
\indexlinearb\LinearALVII        & \indexlinearb\LinearACLV         & \indexlinearb\LinearACCLIII      & \indexlinearb\LinearACCCLI       \\
\indexlinearb\LinearALVIII       & \indexlinearb\LinearACLVI        & \indexlinearb\LinearACCLIV       & \indexlinearb\LinearACCCLII      \\
\indexlinearb\LinearALIX         & \indexlinearb\LinearACLVII       & \indexlinearb\LinearACCLV        & \indexlinearb\LinearACCCLIII     \\
\indexlinearb\LinearALX          & \indexlinearb\LinearACLVIII      & \indexlinearb\LinearACCLVI       & \indexlinearb\LinearACCCLIV      \\
\indexlinearb\LinearALXI         & \indexlinearb\LinearACLIX        & \indexlinearb\LinearACCLVII      & \indexlinearb\LinearACCCLV       \\
\indexlinearb\LinearALXII        & \indexlinearb\LinearACLX         & \indexlinearb\LinearACCLVIII     & \indexlinearb\LinearACCCLVI      \\
\indexlinearb\LinearALXIII       & \indexlinearb\LinearACLXI        & \indexlinearb\LinearACCLIX       & \indexlinearb\LinearACCCLVII     \\
\indexlinearb\LinearALXIV        & \indexlinearb\LinearACLXII       & \indexlinearb\LinearACCLX        & \indexlinearb\LinearACCCLVIII    \\
\indexlinearb\LinearALXV         & \indexlinearb\LinearACLXIII      & \indexlinearb\LinearACCLXI       & \indexlinearb\LinearACCCLIX      \\
\indexlinearb\LinearALXVI        & \indexlinearb\LinearACLXIV       & \indexlinearb\LinearACCLXII      & \indexlinearb\LinearACCCLX       \\
\indexlinearb\LinearALXVII       & \indexlinearb\LinearACLXV        & \indexlinearb\LinearACCLXIII     & \indexlinearb\LinearACCCLXI      \\
\indexlinearb\LinearALXVIII      & \indexlinearb\LinearACLXVI       & \indexlinearb\LinearACCLXIV      & \indexlinearb\LinearACCCLXII     \\
\indexlinearb\LinearALXIX        & \indexlinearb\LinearACLXVII      & \indexlinearb\LinearACCLXV       & \indexlinearb\LinearACCCLXIII    \\
\indexlinearb\LinearALXX         & \indexlinearb\LinearACLXVIII     & \indexlinearb\LinearACCLXVI      & \indexlinearb\LinearACCCLXIV     \\
\indexlinearb\LinearALXXI        & \indexlinearb\LinearACLXIX       & \indexlinearb\LinearACCLXVII     & \indexlinearb\LinearACCCLXV      \\
\indexlinearb\LinearALXXII       & \indexlinearb\LinearACLXX        & \indexlinearb\LinearACCLXVIII    & \indexlinearb\LinearACCCLXVI     \\
\indexlinearb\LinearALXXIII      & \indexlinearb\LinearACLXXI       & \indexlinearb\LinearACCLXIX      & \indexlinearb\LinearACCCLXVII    \\
\indexlinearb\LinearALXXIV       & \indexlinearb\LinearACLXXII      & \indexlinearb\LinearACCLXX       & \indexlinearb\LinearACCCLXVIII   \\
\indexlinearb\LinearALXXV        & \indexlinearb\LinearACLXXIII     & \indexlinearb\LinearACCLXXI      & \indexlinearb\LinearACCCLXIX     \\
\indexlinearb\LinearALXXVI       & \indexlinearb\LinearACLXXIV      & \indexlinearb\LinearACCLXXII     & \indexlinearb\LinearACCCLXX      \\
\indexlinearb\LinearALXXVII      & \indexlinearb\LinearACLXXV       & \indexlinearb\LinearACCLXXIII    & \indexlinearb\LinearACCCLXXI     \\
\indexlinearb\LinearALXXVIII     & \indexlinearb\LinearACLXXVI      & \indexlinearb\LinearACCLXXIV     & \indexlinearb\LinearACCCLXXII    \\
\indexlinearb\LinearALXXIX       & \indexlinearb\LinearACLXXVII     & \indexlinearb\LinearACCLXXV      & \indexlinearb\LinearACCCLXXIII   \\
\indexlinearb\LinearALXXX        & \indexlinearb\LinearACLXXVIII    & \indexlinearb\LinearACCLXXVI     & \indexlinearb\LinearACCCLXXIV    \\
\indexlinearb\LinearALXXXI       & \indexlinearb\LinearACLXXIX      & \indexlinearb\LinearACCLXXVII    & \indexlinearb\LinearACCCLXXV     \\
\indexlinearb\LinearALXXXII      & \indexlinearb\LinearACLXXX       & \indexlinearb\LinearACCLXXVIII   & \indexlinearb\LinearACCCLXXVI    \\
\indexlinearb\LinearALXXXIII     & \indexlinearb\LinearACLXXXI      & \indexlinearb\LinearACCLXXIX     & \indexlinearb\LinearACCCLXXVII   \\
\indexlinearb\LinearALXXXIV      & \indexlinearb\LinearACLXXXII     & \indexlinearb\LinearACCLXXX      & \indexlinearb\LinearACCCLXXVIII  \\
\indexlinearb\LinearALXXXV       & \indexlinearb\LinearACLXXXIII    & \indexlinearb\LinearACCLXXXI     & \indexlinearb\LinearACCCLXXIX    \\
\indexlinearb\LinearALXXXVI      & \indexlinearb\LinearACLXXXIV     & \indexlinearb\LinearACCLXXXII    & \indexlinearb\LinearACCCLXXX     \\
\indexlinearb\LinearALXXXVII     & \indexlinearb\LinearACLXXXV      & \indexlinearb\LinearACCLXXXIII   & \indexlinearb\LinearACCCLXXXI    \\
\indexlinearb\LinearALXXXVIII    & \indexlinearb\LinearACLXXXVI     & \indexlinearb\LinearACCLXXXIV    & \indexlinearb\LinearACCCLXXXII   \\
\indexlinearb\LinearALXXXIX      & \indexlinearb\LinearACLXXXVII    & \indexlinearb\LinearACCLXXXV     & \indexlinearb\LinearACCCLXXXIII  \\
\indexlinearb\LinearALXXXX       & \indexlinearb\LinearACLXXXVIII   & \indexlinearb\LinearACCLXXXVI    & \indexlinearb\LinearACCCLXXXIV   \\
\indexlinearb\LinearAXCI         & \indexlinearb\LinearACLXXXIX     & \indexlinearb\LinearACCLXXXVII   & \indexlinearb\LinearACCCLXXXV    \\
\indexlinearb\LinearAXCII        & \indexlinearb\LinearACLXXXX      & \indexlinearb\LinearACCLXXXVIII  & \indexlinearb\LinearACCCLXXXVI   \\
\indexlinearb\LinearAXCIII       & \indexlinearb\LinearACXCI        & \indexlinearb\LinearACCLXXXIX    & \indexlinearb\LinearACCCLXXXVII  \\
\indexlinearb\LinearAXCIV        & \indexlinearb\LinearACXCII       & \indexlinearb\LinearACCLXXXX     & \indexlinearb\LinearACCCLXXXVIII \\
\indexlinearb\LinearAXCV         & \indexlinearb\LinearACXCIII      & \indexlinearb\LinearACCXCI       & \indexlinearb\LinearACCCLXXXIX   \\
\indexlinearb\LinearAXCVI        & \indexlinearb\LinearACXCIV       & \indexlinearb\LinearACCXCII      &                       \\
\indexlinearb\LinearAXCVII       & \indexlinearb\LinearACXCV        & \indexlinearb\LinearACCXCIII     &                       \\
\indexlinearb\LinearAXCVIII      & \indexlinearb\LinearACXCVI       & \indexlinearb\LinearACCXCIV      &                       \\
\end{longtable}
\end{longsymtable}

\begin{longsymtable}[LINB]{\LINB\ Linear~B Basic and Optional Letters}
\index{Linear B}
\index{alphabets>Linear B}
\label{linearB}
\begin{longtable}{*5{ll@{\qquad}}ll}
\indexlinearb[\textlinb{\Ba}]\Ba         & \indexlinearb[\textlinb{\Bja}]\Bja       & \indexlinearb[\textlinb{\Bmu}]\Bmu       & \indexlinearb[\textlinb{\Bpte}]\Bpte     & \indexlinearb[\textlinb{\Broii}]\Broii   & \indexlinearb[\textlinb{\Bto}]\Bto       \\
\indexlinearb[\textlinb{\Baii}]\Baii     & \indexlinearb[\textlinb{\Bje}]\Bje       & \indexlinearb[\textlinb{\Bna}]\Bna       & \indexlinearb[\textlinb{\Bpu}]\Bpu       & \indexlinearb[\textlinb{\Bru}]\Bru       & \indexlinearb[\textlinb{\Btu}]\Btu       \\
\indexlinearb[\textlinb{\Baiii}]\Baiii   & \indexlinearb[\textlinb{\Bjo}]\Bjo       & \indexlinearb[\textlinb{\Bne}]\Bne       & \indexlinearb[\textlinb{\Bpuii}]\Bpuii   & \indexlinearb[\textlinb{\Bsa}]\Bsa       & \indexlinearb[\textlinb{\Btwo}]\Btwo     \\
\indexlinearb[\textlinb{\Bau}]\Bau       & \indexlinearb[\textlinb{\Bju}]\Bju       & \indexlinearb[\textlinb{\Bni}]\Bni       & \indexlinearb[\textlinb{\Bqa}]\Bqa       & \indexlinearb[\textlinb{\Bse}]\Bse       & \indexlinearb[\textlinb{\Bu}]\Bu         \\
\indexlinearb[\textlinb{\Bda}]\Bda       & \indexlinearb[\textlinb{\Bka}]\Bka       & \indexlinearb[\textlinb{\Bno}]\Bno       & \indexlinearb[\textlinb{\Bqe}]\Bqe       & \indexlinearb[\textlinb{\Bsi}]\Bsi       & \indexlinearb[\textlinb{\Bwa}]\Bwa       \\
\indexlinearb[\textlinb{\Bde}]\Bde       & \indexlinearb[\textlinb{\Bke}]\Bke       & \indexlinearb[\textlinb{\Bnu}]\Bnu       & \indexlinearb[\textlinb{\Bqi}]\Bqi       & \indexlinearb[\textlinb{\Bso}]\Bso       & \indexlinearb[\textlinb{\Bwe}]\Bwe       \\
\indexlinearb[\textlinb{\Bdi}]\Bdi       & \indexlinearb[\textlinb{\Bki}]\Bki       & \indexlinearb[\textlinb{\Bnwa}]\Bnwa     & \indexlinearb[\textlinb{\Bqo}]\Bqo       & \indexlinearb[\textlinb{\Bsu}]\Bsu       & \indexlinearb[\textlinb{\Bwi}]\Bwi       \\
\indexlinearb[\textlinb{\Bdo}]\Bdo       & \indexlinearb[\textlinb{\Bko}]\Bko       & \indexlinearb[\textlinb{\Bo}]\Bo         & \indexlinearb[\textlinb{\Bra}]\Bra       & \indexlinearb[\textlinb{\Bswa}]\Bswa     & \indexlinearb[\textlinb{\Bwo}]\Bwo       \\
\indexlinearb[\textlinb{\Bdu}]\Bdu       & \indexlinearb[\textlinb{\Bku}]\Bku       & \indexlinearb[\textlinb{\Bpa}]\Bpa       & \indexlinearb[\textlinb{\Braii}]\Braii   & \indexlinearb[\textlinb{\Bswi}]\Bswi     & \indexlinearb[\textlinb{\Bza}]\Bza       \\
\indexlinearb[\textlinb{\Bdwe}]\Bdwe     & \indexlinearb[\textlinb{\Bma}]\Bma       & \indexlinearb[\textlinb{\Bpaiii}]\Bpaiii & \indexlinearb[\textlinb{\Braiii}]\Braiii & \indexlinearb[\textlinb{\Bta}]\Bta       & \indexlinearb[\textlinb{\Bze}]\Bze       \\
\indexlinearb[\textlinb{\Bdwo}]\Bdwo     & \indexlinearb[\textlinb{\Bme}]\Bme       & \indexlinearb[\textlinb{\Bpe}]\Bpe       & \indexlinearb[\textlinb{\Bre}]\Bre       & \indexlinearb[\textlinb{\Btaii}]\Btaii   & \indexlinearb[\textlinb{\Bzo}]\Bzo       \\
\indexlinearb[\textlinb{\Be}]\Be         & \indexlinearb[\textlinb{\Bmi}]\Bmi       & \indexlinearb[\textlinb{\Bpi}]\Bpi       & \indexlinearb[\textlinb{\Bri}]\Bri       & \indexlinearb[\textlinb{\Bte}]\Bte       &                               \\
\indexlinearb[\textlinb{\Bi}]\Bi         & \indexlinearb[\textlinb{\Bmo}]\Bmo       & \indexlinearb[\textlinb{\Bpo}]\Bpo       & \indexlinearb[\textlinb{\Bro}]\Bro       & \indexlinearb[\textlinb{\Bti}]\Bti       &                               \\
\end{longtable}

\bigskip
\begin{tablenote}
  \usefontcmdmessage{\textlinb}{\linbfamily}.  Single-character
  shortcuts are also supported: Both
  ``\verb+\textlinb{\Bpa\Bki\Bna}+'' and ``\verb+\textlinb{pcn}+''
  produce ``\textlinb{pcn}'', for example.  \seedocs{\LINB}.
\end{tablenote}
\end{longsymtable}


\begin{symtable}[LINB]{\LINB\ Linear~B Numerals}
\index{Linear B}
\index{numerals>Linear B}
\index{tally markers}
\label{linearB-nums}
\begin{tabular}{*4{ll@{\qquad}}ll}
\indexlinearb[\textlinb{\BNi}]\BNi       & \indexlinearb[\textlinb{\BNvii}]\BNvii   & \indexlinearb[\textlinb{\BNxl}]\BNxl     & \indexlinearb[\textlinb{\BNc}]\BNc       & \indexlinearb[\textlinb{\BNdcc}]\BNdcc   \\
\indexlinearb[\textlinb{\BNii}]\BNii     & \indexlinearb[\textlinb{\BNviii}]\BNviii & \indexlinearb[\textlinb{\BNl}]\BNl       & \indexlinearb[\textlinb{\BNcc}]\BNcc     & \indexlinearb[\textlinb{\BNdccc}]\BNdccc \\
\indexlinearb[\textlinb{\BNiii}]\BNiii   & \indexlinearb[\textlinb{\BNix}]\BNix     & \indexlinearb[\textlinb{\BNlx}]\BNlx     & \indexlinearb[\textlinb{\BNccc}]\BNccc   & \indexlinearb[\textlinb{\BNcm}]\BNcm     \\
\indexlinearb[\textlinb{\BNiv}]\BNiv     & \indexlinearb[\textlinb{\BNx}]\BNx       & \indexlinearb[\textlinb{\BNlxx}]\BNlxx   & \indexlinearb[\textlinb{\BNcd}]\BNcd     & \indexlinearb[\textlinb{\BNm}]\BNm       \\
\indexlinearb[\textlinb{\BNv}]\BNv       & \indexlinearb[\textlinb{\BNxx}]\BNxx     & \indexlinearb[\textlinb{\BNlxxx}]\BNlxxx & \indexlinearb[\textlinb{\BNd}]\BNd       &                               \\
\indexlinearb[\textlinb{\BNvi}]\BNvi     & \indexlinearb[\textlinb{\BNxxx}]\BNxxx   & \indexlinearb[\textlinb{\BNxc}]\BNxc     & \indexlinearb[\textlinb{\BNdc}]\BNdc     &                               \\
\end{tabular}

\bigskip
\begin{tablenote}
  \usefontcmdmessage{\textlinb}{\linbfamily}.
\end{tablenote}
\end{symtable}


\begin{symtable}[LINB]{\LINB\ Linear~B Weights and Measures}
\index{Linear B}
\label{linearB-weights}
\begin{tabular}{*4{ll@{\qquad}}ll}
\indexlinearb[\textlinb{\BPtalent}]\BPtalent & \indexlinearb[\textlinb{\BPvolb}]\BPvolb     & \indexlinearb[\textlinb{\BPvolcf}]\BPvolcf   & \indexlinearb[\textlinb{\BPwtb}]\BPwtb       & \indexlinearb[\textlinb{\BPwtd}]\BPwtd       \\
\indexlinearb[\textlinb{\BPvola}]\BPvola     & \indexlinearb[\textlinb{\BPvolcd}]\BPvolcd   & \indexlinearb[\textlinb{\BPwta}]\BPwta       & \indexlinearb[\textlinb{\BPwtc}]\BPwtc       &                                   \\
\end{tabular}

\bigskip
\begin{tablenote}
  \usefontcmdmessage{\textlinb}{\linbfamily}.
\end{tablenote}
\end{symtable}


\begin{symtable}[LINB]{\LINB\ Linear~B Ideograms}
\index{Linear B}
\index{arrows}
\index{animals}
\label{linearB-objs}
\begin{tabular}{*3{ll@{\qquad}}ll}
\indexlinearb[\textlinb{\BPamphora}]\BPamphora       & \indexlinearb[\textlinb{\BPchassis}]\BPchassis       & \indexlinearb[\textlinb{\BPman}]\BPman               & \indexlinearb[\textlinb{\BPwheat}]\BPwheat           \\
\indexlinearb[\textlinb{\BParrow}]\BParrow           & \indexlinearb[\textlinb{\BPcloth}]\BPcloth           & \indexlinearb[\textlinb{\BPnanny}]\BPnanny           & \indexlinearb[\textlinb{\BPwheel}]\BPwheel           \\
\indexlinearb[\textlinb{\BPbarley}]\BPbarley         & \indexlinearb[\textlinb{\BPcow}]\BPcow               & \indexlinearb[\textlinb{\BPolive}]\BPolive           & \indexlinearb[\textlinb{\BPwine}]\BPwine             \\
\indexlinearb[\textlinb{\BPbilly}]\BPbilly           & \indexlinearb[\textlinb{\BPcup}]\BPcup               & \indexlinearb[\textlinb{\BPox}]\BPox                 & \indexlinearb[\textlinb{\BPwineiih}]\BPwineiih       \\
\indexlinearb[\textlinb{\BPboar}]\BPboar             & \indexlinearb[\textlinb{\BPewe}]\BPewe               & \indexlinearb[\textlinb{\BPpig}]\BPpig               & \indexlinearb[\textlinb{\BPwineiiih}]\BPwineiiih     \\
\indexlinearb[\textlinb{\BPbronze}]\BPbronze         & \indexlinearb[\textlinb{\BPfoal}]\BPfoal             & \indexlinearb[\textlinb{\BPram}]\BPram               & \indexlinearb[\textlinb{\BPwineivh}]\BPwineivh       \\
\indexlinearb[\textlinb{\BPbull}]\BPbull             & \indexlinearb[\textlinb{\BPgoat}]\BPgoat             & \indexlinearb[\textlinb{\BPsheep}]\BPsheep           & \indexlinearb[\textlinb{\BPwoman}]\BPwoman           \\
\indexlinearb[\textlinb{\BPcauldroni}]\BPcauldroni   & \indexlinearb[\textlinb{\BPgoblet}]\BPgoblet         & \indexlinearb[\textlinb{\BPsow}]\BPsow               & \indexlinearb[\textlinb{\BPwool}]\BPwool             \\
\indexlinearb[\textlinb{\BPcauldronii}]\BPcauldronii & \indexlinearb[\textlinb{\BPgold}]\BPgold             & \indexlinearb[\textlinb{\BPspear}]\BPspear           &                                           \\
\indexlinearb[\textlinb{\BPchariot}]\BPchariot       & \indexlinearb[\textlinb{\BPhorse}]\BPhorse           & \indexlinearb[\textlinb{\BPsword}]\BPsword           &                                           \\
\end{tabular}

\bigskip
\begin{tablenote}
  \usefontcmdmessage{\textlinb}{\linbfamily}.
\end{tablenote}
\end{symtable}


\begin{longsymtable}[LINB]{\LINB\ Unidentified Linear~B Symbols}
\index{Linear B}
\label{linearB-unknown}
\begin{longtable}{*4{ll@{\qquad}}ll}
\indexlinearb[\textlinb{\BUi}]\BUi       & \indexlinearb[\textlinb{\BUiv}]\BUiv     & \indexlinearb[\textlinb{\BUvii}]\BUvii   & \indexlinearb[\textlinb{\BUx}]\BUx       & \indexlinearb[\textlinb{\Btwe}]\Btwe     \\
\indexlinearb[\textlinb{\BUii}]\BUii     & \indexlinearb[\textlinb{\BUv}]\BUv       & \indexlinearb[\textlinb{\BUviii}]\BUviii & \indexlinearb[\textlinb{\BUxi}]\BUxi     &                               \\
\indexlinearb[\textlinb{\BUiii}]\BUiii   & \indexlinearb[\textlinb{\BUvi}]\BUvi     & \indexlinearb[\textlinb{\BUix}]\BUix     & \indexlinearb[\textlinb{\BUxii}]\BUxii   &                               \\
\end{longtable}

\bigskip
\begin{tablenote}
  \usefontcmdmessage{\textlinb}{\linbfamily}.
\end{tablenote}
\end{longsymtable}

\section{Magical Staves}

\begin{longsymtable}[STAVE]{\STAVE\ Magical Staves}
\index{symbols>staves}
\index{symbols>magical signs}
\index{magical signs}
\index{staves}
\index{Icelandic staves}
\label{staves}
\small
\begin{longtable}{*2{ll@{\qqquad}}ll}
\multicolumn{6}{l}{\small\textit{(continued from previous page)}} \\[3ex]
\endhead
\endfirsthead
\\[3ex]
\multicolumn{6}{r}{\small\textit{(continued on next page)}}
\endfoot
\endlastfoot
\Kstav\staveI     & \Kstav\staveXXIV    & \Kstav\staveXLVII  \\
\Kstav\staveII    & \Kstav\staveXXV     & \Kstav\staveXLVIII \\
\Kstav\staveIII   & \Kstav\staveXXVI    & \Kstav\staveXLIX   \\
\Kstav\staveIV    & \Kstav\staveXXVII   & \Kstav\staveL      \\
\Kstav\staveV     & \Kstav\staveXXVIII  & \Kstav\staveLI     \\
\Kstav\staveVI    & \Kstav\staveXXIX    & \Kstav\staveLII    \\
\Kstav\staveVII   & \Kstav\staveXXX     & \Kstav\staveLIII   \\
\Kstav\staveVIII  & \Kstav\staveXXXI    & \Kstav\staveLIV    \\
\Kstav\staveIX    & \Kstav\staveXXXII   & \Kstav\staveLV     \\
\Kstav\staveX     & \Kstav\staveXXXIII  & \Kstav\staveLVI    \\
\Kstav\staveXI    & \Kstav\staveXXXIV   & \Kstav\staveLVII   \\
\Kstav\staveXII   & \Kstav\staveXXXV    & \Kstav\staveLVIII  \\
\Kstav\staveXIII  & \Kstav\staveXXXVI   & \Kstav\staveLIX    \\
\Kstav\staveXIV   & \Kstav\staveXXXVII  & \Kstav\staveLX     \\
\Kstav\staveXV    & \Kstav\staveXXXVIII & \Kstav\staveLXI    \\
\Kstav\staveXVI   & \Kstav\staveXXXIX   & \Kstav\staveLXII   \\
\Kstav\staveXVII  & \Kstav\staveXL      & \Kstav\staveLXIII  \\
\Kstav\staveXVIII & \Kstav\staveXLI     & \Kstav\staveLXIV   \\
\Kstav\staveXIX   & \Kstav\staveXLII    & \Kstav\staveLXV    \\
\Kstav\staveXX    & \Kstav\staveXLIII   & \Kstav\staveLXVI   \\
\Kstav\staveXXI   & \Kstav\staveXLIV    & \Kstav\staveLXVII  \\
\Kstav\staveXXII  & \Kstav\staveXLV     & \Kstav\staveLXVIII \\
\Kstav\staveXXIII & \Kstav\staveXLVI    &                \\
\end{longtable}

\bigskip

\begin{tablenote}
  The meanings of these symbols are described on the Web site for the
  Museum of Icelandic Sorcery and Witchcraft\index{Museum of Icelandic
  Sorcery and Witchcraft} at
  \url{http://www.galdrasyning.is/index.php?option=com_content&task=category&sectionid=5&id=18&Itemid=60}
  (TinyURL: \url{http://tinyurl.com/25979m}).  For example,
  \docAuxCommand{staveL}~(``\staveL'') is intended to ward off
  ghosts\index{ghosts} and evil\index{evil spirits} spirits.
\end{tablenote}
\end{longsymtable}


\subsection{Resizing symbols}
\label{resizing-symbols}
\index{symbols>resize}

Mathematical symbols listed in this document as
``variable-sized\idxboth{variable-sized}{symbols}'' are designed to
stretch vertically.  Each
variable-sized\idxboth{variable-sized}{symbols} symbol comes in one or
more basic sizes plus a variation comprising both stretchable and
nonstretchable segments.  Table \vref{var-sized-syms} presents the
symbols %\docAuxCommand{}}
 and \docAuxCommand{uparrow} in their default size, in their
\cmd{\big}, \cmd{\Big}, \cmd{\bigg}, and \cmd{\Bigg} sizes, in an even
larger size achieved using \cmd{\left}\slash\cmd{\right}, and---for
contrast---in a large size achieved by changing the font size using
\latexe's \cmd{\fontsize} command.  Because the symbols shown belong
to the \PSfont{Computer Modern} family, the \pkgname{type1cm} package
needs to be loaded to support font sizes larger than 24.88\,pt.

\begin{nonsymtable}{Sample resized delimiters}
\idxboth{variable-sized}{symbols}
\label{var-sized-syms}
\newcommand{\maketall}[1]{\ensuremath{\left.\rule{0pt}{1.5cm}\right#1}}
\newcommand{\makebig}[1]{\fontsize{3cm}{3cm}\selectfont\ensuremath{#1}}

\begin{tabular}{@{}*8c@{}}
  \toprule
  Symbol &
  Default size &
  \cmd{\big} &
  \cmd{\Big} &
  \cmd{\bigg} &
  \cmd{\Bigg} &
  \cmd{\left}\,/\,\cmd{\right} &
  \cmd{\fontsize} \\
  \midrule

  \verb|\}| &
  $\}$ &
  $\big\}$ &
  $\Big\}$ &
  $\bigg\}$ &
  $\Bigg\}$ &
  \maketall\} &
  \makebig\} \\

  \verb|\uparrow| &
  $\uparrow$ &
  $\big\uparrow$ &
  $\Big\uparrow$ &
  $\bigg\uparrow$ &
  $\Bigg\uparrow$ &
  \maketall\uparrow &
  \makebig\uparrow \\
  \bottomrule
\end{tabular}
\end{nonsymtable}

All variable-sized delimiters are defined (by the corresponding
\texttt{.tfm} file) in terms of up to five segments, as illustrated by
\vref{extensible-brace}.  The top, middle, and bottom segments
are of a fixed size.  The top-middle and middle-bottom segments (which
are constrained to be the same character) are repeated as many times
as necessary to achieve the desired height.

\begin{figure}[htbp]
\centering
\renewcommand{\arraystretch}{2}
\newcommand{\cmexchar}{\usefont{OMX}{cmex}{m}{n}\selectfont\char}
\newlength{\braceheight}
\setlength{\braceheight}{6.5\baselineskip}
\begin{tabular}{@{}ccl@{}}
  \multirow{5}*{$\left.\rule{0pt}{\braceheight}\right\} \longrightarrow$}
  & \cmexchar'71 & top \\
  & \cmexchar'76 & top-middle (extensible) \\
  & \cmexchar'75 & middle \\
  & \cmexchar'76 & middle-bottom (extensible) \\
  & \cmexchar'73 & bottom \\
  \\
\end{tabular}
\index{symbols>extensible}
\caption{Implementation of variable-sized delimiters}
\label{extensible-brace}
\end{figure}

  
\subsubsection{Reflecting and rotating existing symbols}

 
  \index{symbols>reversed|(}
  \index{symbols>rotated|(}
  \index{symbols>upside-down|(}
  \index{symbols>inverted|(}
  \index{reversed symbols|(}
  \index{rotated symbols|(}
  \index{upside-down symbols|(}
  \index{inverted symbols|(}
  
  
  \begin{texexample}{Create an Irony mark}{}
  \DeclareRobustCommand{\irony}{\textsuperscript{\reflectbox{?}}}
  \end{texexample}
  \DeclareRobustCommand{\irony}{\textsuperscript{\reflectbox{?}}}
  A common request on \ctt is for a reversed or rotated version of an
  existing symbol.  As a last resort, these effects can be achieved
  with the \pkgname{graphicx} (or \pkgname{graphics}) package's
  \cmd{\reflectbox} and \cmd{\rotatebox} macros.
  \newcommand{\definitedescription}{\rotatebox[origin=c]{180}{$\iota$}}
  For example, \verb|\textsuperscript{\reflectbox{?}}| produces an
  irony\index{irony mark=irony mark (\irony)} mark~(``\,\irony\,'';
  cf.~\url{http://en.wikipedia.org/wiki/Irony_mark}), and
  \verb|\rotatebox[origin=c]{180}{$\iota$}| produces the
  definite-description\index{definite-description operator
  (\definitedescription)}\index{iota, upside-down}
  operator~(``\rotatebox[origin=c]{180}{$\iota$}'').  
  
  The disadvantage
  of the \pkgname{graphicx}/\pkgname{graphics} approach is that not
  every \tex backend handles graphical transformations.\footnote{As an
  example, Xdvi\index{Xdvi} ignores both \cmd{\reflectbox} and
  \cmd{\rotatebox}.}  Far better is to find a suitable font that
  contains the desired symbol in the correct orientation.  For
  instance, if the PHON package is available, then
  \verb|\textit{\riota}| will yield a
  backend-independent~``\textit{\cmd{\riota}}''.
  Similarly,\label{page:such-that} \TIPA's
  \docAuxCommand{textrevepsilon}~(``\textrevepsilon'') or \WIPA's
  \docAuxCommand{textrevepsilon}~(``\textrevepsilon'') may be used to express the
  mathematical notion of ``such\index{such that} that'' in a cleaner
  manner than with \cmd{\reflectbox} or
  \cmd{\rotatebox}.\footnote{More common symbols for representing
  ``such\index{such that} that'' include ``\texttt{\textbar}'',
  ``\texttt{:}'', and ``\texttt{s.t.}''.}
  \index{symbols>reversed|)}
  \index{symbols>rotated|)}
  \index{symbols>upside-down|)}
  \index{symbols>inverted|)}
  \index{reversed symbols|)}
  \index{rotated symbols|)}
  \index{inverted symbols|)}



\begin{texexample}{Enlarging Delimiters}{ex:type1cm}
\newcommand{\makeBIG}[1]{\fontsize{1cm}{1cm}\selectfont\ensuremath{#1}}
  \makeBIG\>

\end{texexample}

\subsection{Where can I find the symbol for~\dots?}

\label{combining-symbols}

An easy way to find a symbol is to use \url{http://detexify.com}. This is a website service that you can use to identify a symbol by drawing it. The menu always adds the necessary package to a symbol after presenting possible matches to what you have drawn. But you also can click on the ``symbols'' button and enter the command. Here, too, the necessary package is added. 

If you can't find some symbol you're looking for in this document, there
are a few possible explanations:

\begin{itemize}
  \item The symbol isn't intuitively named.  As a few examples, the
  \IFS\ command to draw dice\index{dice} is
  ``\docAuxCommand{Cube}''; a plus sign with a circle around it
  (``exclusive or''\index{exclusive or} to computer engineers) is
  ``\docAuxCommand{oplus}''; and lightning bolts in fonts designed by German
  speakers may have ``blitz'' in their names as in the
  ULSY package.  The moral of the story is to be creative with
  synonyms when searching the index.

  \item The symbol is defined by some package that I overlooked (or
  deemed unimportant).  

  \item The symbol isn't defined in any package whatsoever.
\end{itemize}


  Even in the last case, all is not lost.  Sometimes, a symbol exists
  in a font, but there is no \latex{} binding for it.  For example,
  the \postscript \PSfont{Symbol} font contains a
  ``\Pisymbol{psy}{191}''\index{arrows} symbol, which may be useful
  for representing a carriage\index{carriage return} return, but there
  is no package (as far as I know) for accessing that symbol.  To
  produce an unnamed symbol, you need to switch to the font explicitly
  with \latexe's low-level font commands~\cite{fntguide} and use
  \tex's primitive \cmd{\char} command~\cite{Knuth:ct-a} to request a
  specific character number in the font.\footnote{\pkgname{pifont}
  defines a convenient \cmd{\Pisymbol} command for accessing symbols
  in \postscript\index{PostScript fonts} fonts by number.  For example,
  ``\cmd{\Pisymbol}\texttt{\string{psy\string}\string{191\string}}''
  produces ``\Pisymbol{psy}{191}''.}
   
  In fact, \cmd{\char} is not strictly necesssary; the character can
  often be entered symbolically.
 

  For example, the symbol for an impulse train or Tate-Shafarevich
  group (``{|\string\fontencoding{OT2}\string\selectfont SH|}'') is actually an
  uppercase \textit{sha} in the Cyrillic\index{alphabets>Cyrillic}
  alphabet.  (Cyrillic is supported by the OT2 \fntenc[OT2], for
  instance).  While a \textit{sha} can be defined numerically as
  
  it may be more intuitive to use the OT2 \fntenc[OT2]'s ``SH''
  ligature:
  
 
The \pkgname{slashed} package \citep{slashed}, although originally designed for
producing Feynman\index{Feynman slashed character notation}
slashed-character\idxboth{slashed}{letters} notation, in fact
facilitates the production of \emph{arbitrary} overlapped symbols.
\ifhaveslashed
  \newcommand{\rqm}{{\declareslashed{}{\text{-}}{0.04}{0}{I}\slashed{I}}}
  The default behavior is to overwrite a given character with ``$/$''.
  For example, \cmd{\slashed}\verb|{D}| produces ``$\slashed{D}$''.
  However, the \cmd{\declareslashed} command provides the flexibility
  to specify the mathematical context of the composite character
  (operator, relation, punctuation, etc., as will be discussed in
  \ref{math-spacing}), the overlapping symbol, horizontal and
  vertical adjustments in symbol-relative units, and the character to
  be overlapped.  Consider, for example, the symbol for reduced
  quadrupole moment~(``$\rqm$'').  This can be declared as follows:

\begin{verbatim}
    \newcommand{\rqm}{{%
      \declareslashed{}{\text{-}}{0.04}{0}{I}\slashed{I}}}
\end{verbatim}

  \noindent
  \newcommand{\curlyarg}{\texttt{\char`\{}$\cdot$\texttt{\char`\}}}%

  \cmd{\declareslashed}\curlyarg\curlyarg\curlyarg\curlyarg\verb|{I}|
  affects the meaning of all subsequent \cmd{\slashed}\verb|{I}|
  commands in the same scope.  The preceding definition of \docAuxCommand{rqm}
  therefore uses an extra set of curly braces to limit that scope to a
  single \cmd{\slashed}\verb|{I}|.  In addition, \docAuxCommand{rqm} uses
  \pkgname{amstext}'s \cmd{\text} macro
  (described~\vpageref[below]{text-macro}) to make
  \cmd{\declareslashed} use a text-mode hyphen~(``-'') instead of a
  math-mode minus sign~(``$-$'') and to ensure that the hyphen scales
  properly in size in subscripts and superscripts.
\fi  

See \pkgname{slashed}'s documentation (located in
\docfilename{slashed.sty} itself) for a detailed usage description of the
\cmd{\slashed} and \cmd{\declareslashed} commands.

Somewhat simpler than \pkgname{slashed} is the \pkgname{centernot}
package.  \pkgname{centernot} provides a single command,
\cmd{\centernot}, which, like \cmd{\not}, puts a slash over the
subsequent mathematical symbol.  However, instead of putting the slash
at a fixed location, \cmd{\centernot} centers the slash over its
argument.  \cmd{\centernot} might be used, for example, to create a
``does\index{does not imply} not imply'' symbol%

\ifhavecenternot
%   \begin{center}
%    \renewcommand{\arraystretch}{1.25}%
%    \begin{tabular}{cl}
%      $\not\Longrightarrow$       & \verb|\not\Longrightarrow| \\
%      \multicolumn{2}{c}{vs.} \\
%      $\centernot\Longrightarrow$ & \verb|\centernot\Longrightarrow| \\
%    \end{tabular}
%  \end{center}
\else
  .
\fi   
\seedocs{\pkgname{centernot}}


\subsection{How to make new symbols work in superscripts and subscripts}

\index{subscripts>new symbols used in|(}
\index{superscripts>new symbols used in|(}


To make composite symbols work properly within subscripts and
superscripts, you may need to use \tex's \cmd{\mathchoice} primitive.
\cmd{\mathchoice} evaluates one of four expressions, based on whether
the current math style is display, text, script, or scriptscript.
(See \TeXbook for a more complete description.)  For example, the
following \latex code---posted to \ctt by
\person{Torsten}{Bronger}---composes a sub/superscriptable
``\cmd{\topbot}'' symbol out of \docAuxCommand{top} and \docAuxCommand{bot} (``$\top$''
and ``$\bot$''):



\indexcommand{\displaystyle}%
\indexcommand{\textstyle}%
\indexcommand{\scriptstyle}%
\indexcommand{\scriptscriptstyle}%
\label{code:topbot}%

\begin{verbatim}
   \def\topbotatom#1{\hbox{\hbox to 0pt{$#1\bot$\hss}$#1\top$}}
   \newcommand*{\topbot}{\mathrel{\mathchoice{\topbotatom\displaystyle}
                                    {\topbotatom\textstyle}
                                    {\topbotatom\scriptstyle}
                                    {\topbotatom\scriptscriptstyle}}}
\end{verbatim}
\index{superscripts>new symbols used in|)}
\index{subscripts>new symbols used in|)}

\begin{texexample}{mathchoice}{ex:mathchoice}
\bgroup
\def\topbotatom#1{\hbox{\hbox to 0pt{$#1\bot$\hss}$#1\top$}}
   \def\topbot{\mathrel{\mathchoice{\topbotatom\displaystyle}
                                    {\topbotatom\textstyle}
                                    {\topbotatom\scriptstyle}
                                    {\topbotatom\scriptscriptstyle}}}
\[ a_{\topbot} + b^{\topbot} \]
\egroup
\end{texexample}


\subsection{Modifying \latex-generated symbols}

\index{dots (ellipses)|(}
\index{ellipses (dots)|(}
\index{dot symbols|(}
\index{symbols>dot|(}

Oftentimes, symbols composed in the \latexe source code can be
modified with minimal effort to produce useful variations.  For
example, \fontdefdtx composes the \docAuxCommand{ddots} symbol (see
\vref{dots}) out of three periods, raised~7\,pt., 4\,pt., and
1\,pt., respectively:

\begin{verbatim}
   \def\ddots{\mathinner{\mkern1mu\raise7\p@
       \vbox{\kern7\p@\hbox{.}}\mkern2mu
       \raise4\p@\hbox{.}\mkern2mu\raise\p@\hbox{.}\mkern1mu}}
\end{verbatim}

\noindent
\cmd{\p@} is a \latexe{} shortcut for ``\texttt{pt}'' or
``\texttt{1.0pt}''.  The remaining commands are defined in \TeXbook.
To\label{revddots} draw a version of \docAuxCommand{ddots} with the dots going
along the opposite diagonal, we merely have to reorder the
\verb|\raise7\p@|, \verb|\raise4\p@|, and \verb|\raise\p@|:

\begin{texexample}{revddots}{ex:revddots}
\makeatletter
   \def\revddots{\mathinner{\mkern1mu\raise\p@
      \vbox{\kern7\p@\hbox{.}}\mkern2mu
       \raise4\p@\hbox{.}\mkern2mu\raise7\p@\hbox{.}\mkern1mu}}
\makeatother

\[\revddots \]
\end{texexample}


    \makeatletter
      \def\revddots{\mathinner{\mkern1mu\raise\p@
        \vbox{\kern7\p@\hbox{.}}\mkern2mu
        \raise4\p@\hbox{.}\mkern2mu\raise7\p@\hbox{.}\mkern1mu}}
    \makeatother
\indexcommand[$\string\revddots$]{\revddots}

\noindent
\docAuxCommand{revddots} is essentially identical to the \MDOTS\
package's
\ifMDOTS
  \docAuxCommand{iddots}
\else
  \cmd{\iddots}
\fi
command or the \YH\ package's
%\ifYH
%  \docAuxCommand{adots}
%\else
  \cmd{\adots}
%\fi
command.
\index{symbols>dot|)}
\index{dot symbols|)}
\index{ellipses (dots)|)}
\index{dots (ellipses)|)}




\section{ASCII and Latin~1 quick reference}
\label{ascii-quickref}

\index{ASCII|(}

\vref{ascii-table} amalgamates data from various other tables in this
document into a convenient reference for \latexe typesetting of \texttt{ascii}
characters, i.e., the characters available on a typical U.S. computer
keyboard.  The first two columns list the character's \texttt{ascii} code in
decimal and hexadecimal.  The third column shows what the character
looks like.  The fourth column lists the \latexe command to typeset
the character as a text character.  And the fourth column lists the
\latexe command to typeset the character within a
\verb|\texttt{|$\ldots$\verb|}| command (or, more generally, when
\verb|\ttfamily| is in effect).


\index{ASCII|)}

\begin{nonsymtable}{\latexe ASCII Table}
  \index{ASCII>table}
  \label{ascii-table}
  ^^A Define an equivalent of \vdots that's the height of a "9".
  \newlength{\digitheight}
  \settoheight{\digitheight}{9}
  \newcommand{\digitvdots}{\raisebox{-1.5pt}[\digitheight]{$\vdots$}}

 ^^A Replace all glyphs in a row with vertical dots.
  \makeatletter
  \newcommand{\skipped}{%
    \settowidth{\@tempdima}{99} \makebox[\@tempdima]{\digitvdots} &
    \settowidth{\@tempdima}{99} \makebox[\@tempdima]{\digitvdots} &
    \digitvdots &
    \digitvdots &
    \digitvdots \\
  }
  \makeatother

  ^^A Typesetting a symbol by prefixing it with a "\".
  \newcommand{\bscommand}[1]{#1 & \cmd{#1} & \cmd{#1}}

  \begin{tabular}[t]{@{}*2{>{\ttfamily}r}c*2{>{\ttfamily}l}l@{}} \\ \toprule
    \multicolumn{1}{@{}c}{Dec} &
    \multicolumn{1}{c}{Hex} &
    \multicolumn{1}{c}{Char} &
    \multicolumn{1}{c}{Body text} &
    \multicolumn{1}{c@{}}{\ttfamily\string\texttt} \\ \midrule

    33 & 21 & ! & ! & ! \\
    34 & 22 & {\fontencoding{T1}\selectfont\textquotedbl} &
      \string\textquotedbl & " \\      ^^A Not available in OT1
    35 & 23 & \bscommand{\#} \\
    36 & 24 & \bscommand{\$} \\
    37 & 25 & \bscommand{\%} \\
    38 & 26 & \bscommand{\&} \\
    39 & 27 & ' & ' & ' \\
    40 & 28 & ( & ( & ( \\
    41 & 29 & ) & ) & ) \\
    42 & 2A & * & * & * \\
    43 & 2B & + & + & + \\
    44 & 2C & , & , & , \\
    45 & 2D & - & - & - \\
    46 & 2E & . & . & . \\
    47 & 2F & / & / & / \\
    48 & 30 & 0 & 0 & 0 \\
    49 & 31 & 1 & 1 & 1 \\
    50 & 32 & 2 & 2 & 2 \\
    \skipped
    57 & 39 & 9 & 9 & 9 \\
    58 & 3A & : & : & : \\
    59 & 3B & ; & ; & ; \\
    60 & 3C & \textless & \docAuxCommand{textless} & < \\       ^^A Or $<$
    61 & 3D & = & = & = \\ \bottomrule
  \end{tabular}
  \hfil
  \begin{tabular}[t]{@{}*2{>{\ttfamily}r}c*2{>{\ttfamily}l}l@{}} \\ \toprule
    \multicolumn{1}{@{}c}{Dec} &
    \multicolumn{1}{c}{Hex} &
    \multicolumn{1}{c}{Char} &
    \multicolumn{1}{c}{Body text} &
    \multicolumn{1}{c@{}}{\ttfamily\string\texttt} \\ \midrule

    62 & 3E & \textgreater & \docAuxCommand{textgreater} & > \\   
    63 & 3F & ? & ? & ? \\
    64 & 40 & @ & @ & @ \\
    65 & 41 & A & A & A \\
    66 & 42 & B & B & B \\
    67 & 43 & C & C & C \\
    \skipped
    90 & 5A & Z & Z & Z \\
    91 & 5B & [ & [ & [ \\
    92 & 5C & \textbackslash & \docAuxCommand{textbackslash} &
      \verb|\char`\\| \\   ^^A \textbackslash works in non-OT1
    93 & 5D & ] & ] & ] \\
    94 & 5E & \^{} & \verb|\^{}| & \verb|\^{}| \\   ^^A Or \textasciicircum
    95 & 5F & \_ & \verb|\_| & \verb|\char`\_| \\   ^^A \_ works in non-OT1
    96 & 60 & ` & ` & ` \\
    97 & 61 & a & a & a \\
    98 & 62 & b & b & b \\
    99 & 63 & c & c & c \\
    \skipped
   122 & 7A & z & z & z \\
   123 & 7B & \{ & \verb|\{| & \verb|\char`\{| \\   
   124 & 7C & \textbar & \docAuxCommand{textbar} & \textbar \\    
   125 & 7D & \} & \verb|\}| & \verb|\char`\}| \\   
   126 & 7E & \~{} & \verb|\~{}| & \verb|\~{}| \\   
   \\
   \bottomrule
  \end{tabular}
\end{nonsymtable}

The following are some additional notes about the contents of
\ref{ascii-table}:

\begin{itemize}
  \item
  ``\indexcommand[\string\encone{\string\textquotedbl}]{\textquotedbl}{\encone{\textquotedbl}}''
  is not available in the OT1 \fntenc[OT1].

  \item \ref{ascii-table} shows a close quote for character~39 for
    consistency with the open quote shown for character~96.  A
    straight quote can be typeset using \docAuxCommand{textquotesingle}
    (cf.~\ref{tc-misc}).

  \item
  The\label{upside-down}\index{symbols>upside-down|(}\index{upside-down
  symbols|(} characters ``\texttt{<}'', ``\texttt{>}'', and
  ``\texttt{\textbar}'' do work as expected in math mode, although they
  produce, respectively, ``<'', ``>'', and ``\textbar'' in text mode when
  using the OT1 \fntenc[OT1].\footnote{Donald\index{Knuth, Donald E.}
  Knuth didn't think such symbols were important outside of
  mathematics so he omitted them from his text fonts.} The following
  are some alternatives for typesetting ``\textless'',
  ``\textgreater'', and ``\textbar'':

  \begin{itemize}
    \item Specify a document \fntenc{} other than OT1 (as
    described~\vpageref[above]{altenc}).

    \item Use the appropriate symbol commands from
    \vref{text-predef}, viz.~\docAuxCommand{textless},
    \docAuxCommand{textgreater}, and \docAuxCommand{textbar}.

    \item Enter the symbols in math mode instead of text mode,
    i.e.,~\verb+$<$+, \verb+$>$+, and \verb+$|$+.
  \end{itemize}

  \noindent
  Note that for typesetting metavariables many people prefer
  \docAuxCommand{textlangle} and \docAuxCommand{textrangle} to \docAuxCommand{textless} and
  \docAuxCommand{textgreater}; i.e., ``\meta{filename}'' instead of
  ``$<$\textit{filename}$>$''.\index{symbols>upside-down|)}\index{upside-down
  symbols|)}

  \item Although ``\texttt{/}'' does not require any special
  treatment, \latex additionally defines a \docAuxCommand{slash} command which
  outputs the same glyph but permits a line~break afterwards.  That
  is, ``\texttt{increase/decrease}'' is always typeset as a single
  entity while ``\verb|increase\slash{}decrease|'' may be typeset with
  ``increase/'' on one line and ``decrease'' on the next.

  \item \label{page:tildes} \index{tilde|(} \docAuxCommand{textasciicircum}
  can be used instead of 
 % \cmdI[\string\^{}]{\^{}}\verb|{}|, 
  and
  \docAuxCommand{textasciitilde} can be used instead of
%  \cmdI[\string\~{}]{\~{}}\verb|{}|.  Note that
  \docAuxCommand{textasciitilde} and 
  %\cmdI[\string\~{}]{\~{}}\verb|{}|
  produce raised, diacritic tildes.  ``Text''
  (i.e.,~vertically\index{tilde>vertically centered} centered)
  tildes can be generated with either the math-mode \docAuxCommand{sim}
  command (shown in \vref{rel}), which produces a somewhat wide
  ``$\sim$'', or the \TC\ package's \docAuxCommand{texttildelow} (shown in
  \vref{tc-misc}), which produces a vertically centered
  ``{\fontfamily{ptm}\selectfont\texttildelow}'' in most fonts but a
  baseline-oriented ``\texttildelow'' in \PSfont{Computer Modern},
  \TX, \PX, and various other fonts originating from the
  \tex\ world.  If your goal is to typeset tildes in URLs or Unix
  filenames, your best bet is to use the \pkgname{url} package,
  which has a number of nice features such as proper line-breaking
  of such names.\index{tilde|)}

  \item The various \cmd{\char} commands within \verb|\texttt| are
  necessary only in the OT1 \fntenc[OT1].  In other encodings
  (e.g.,~T1)\index{font encodings>T1}, commands such as 
%  \cmdIp{\{},
  %\cmdIp{\}}, \
  %docAuxCommand{_}, 
  and \docAuxCommand{textbackslash} all work properly.

  \item The code\index{code page 437} page~437 (IBM~PC\index{IBM PC})
  version of \texttt{ascii} characters~1 to~31 can be typeset
  using the \ASCII\ package.
\ifASCII
  See \vref{ibm-ascii}.
\fi

  \item To replace~``\verb|`|'' and~``\verb|'|'' with the more
  computer-like (and more visibly distinct) ``\texttt{\char18}''
  and~``\texttt{\char13}'' within a \texttt{verbatim} environment,
  use the \pkgname{upquote} package.  Outside of \texttt{verbatim},
  you can use \cmd{\char}\texttt{18} and \cmd{\char}\texttt{13} to
  get the modified quote characters.  (The former is actually a
  grave accent.)
\end{itemize}





\subsection{Unicode characters}
\label{unicode-chars}

\index{Unicode|(}

\href{http://www.unicode.org/}{Unicode} is a ``universal character
set''---a standard for encoding (i.e.,~assigning unique numbers to)
the symbols appearing in many of the world's languages.  While \texttt{ascii}
can represent 128 symbols and Latin~1 can represent 256 symbols,
Unicode can represent an astonishing 1,114,112 symbols.

Because \tex and \latex{} predate the Unicode standard and Unicode
fonts by almost a decade, support for Unicode has had to be added to
the base \tex{} and \latex{} systems.  Note first that \latex{}
distinguishes between \emph{input} encoding---the characters used in
the \texttt{.tex} file---and \emph{output} encoding---the characters
that appear in the generated \texttt{.dvi}, \texttt{.pdf}, etc.\ file.
For a discusiion on Unicode for Mathematics see \citep{beetona}.

\begin{texexample}{How to add symbols}{unicodesymbols}
\ifxetex
  \newfontfamily{\codetwothousand}{code2000.ttf}
  \codetwothousand\char"1F050 \char"2603\char"2617
  \symbol{9825}
  \newfontfamily{\codetwothousandone}{code2001.ttf}
  \newfontfamily{\symbola}{symbola.ttf}
  {\codetwothousand \symbol{9742} \symbol{9743}
    Katakana (片仮名, カタカナ)
   \codetwothousandone \symbol{57508}
   \symbola \symbol{9816}
   
  }
\else
   Compile the document with XeTeX to see the example
\fi
\end{texexample}

\subsubsection{Inputting Unicode characters}

To include Unicode characters in a \texttt{.tex} file, load the
\pkgname{ucs} package and load the \pkgname{inputenc} package with the
\optname{inputenc}{utf8x} (``\utfviii extended'')
option.\footnote{\utfviii is the 8-bit Unicode Transformation Format,
  a popular mechanism for representing Unicode symbol numbers as
  sequences of one to four bytes.}  These packages enable \latex{} to
translate \utfviii sequences to \latex{} commands, which are
subsequently processed as normal.  For example, the \utfviii text
``\texttt{Copyright~\textcopyright\ \the\year}''---``\texttt{\textcopyright}''
is not an \texttt{ascii} character and therefore cannot be input directly
without packages such as \pkgname{ucs}/\pkgname{inputenc}---is
converted internally by \pkgname{inputenc} to ``\texttt{Copyright}
\verb+\textcopyright{}+ \texttt{\the\year}'' and therefore typeset as
``Copyright~\textcopyright\ \the\year''.

The \pkgname{ucs}\slash\pkgname{inputenc} combination supports only a
tiny subset of Unicode's million-plus symbols.  Additional symbols can
be added manually using the \cmd{\DeclareUnicodeCharacter} command.
\cmd{\DeclareUnicodeCharacter} takes two arguments: a Unicode number
and a \latex{} command to execute when the corresponding Unicode
character is encountered in the input.  For example, the Unicode
character ``degree celsius''~(``\,\textcelsius\,'') appears at
character position U+2103.\footnote{The Unicode convention is to
  express character positions as ``U+\meta{hexadecimal number}''.}
However, ``\,\texttt{\textcelsius}\,'' is not one of the characters
that \pkgname{ucs} and \pkgname{inputenc} recognize.  

The following
document shows how to use \cmd{\DeclareUnicodeCharacter} to tell
\latex{} that the ``\,\texttt{\textcelsius}\,'' character should be
treated as a synonym for \docAuxCommand{textcelsius}:

\begin{verbatim}
   \documentclass{article}
   \usepackage{ucs}
   \usepackage[utf8x]{inputenc}
   \usepackage{textcomp}

   \DeclareUnicodeCharacter{"2103}{\textcelsius} % Enable direct input of U+2103.
\end{verbatim}
\noindent
\verb|   \begin{document}| \\
\verb|   |\texttt{It was a balmy 21\textcelsius.} \\
\verb|   \end{document}|

\medskip

\noindent
which produces

\begin{quotation}
  It was a balmy 21\textcelsius.
\end{quotation}

\seedocs{\pkgname{ucs}} and for descriptions of the various options that control \pkgname{ucs}'s behavior.


\subsection{Outputting Unicode characters}

Orthogonal to the ability to include Unicode characters in a
\latex\ input file is the ability to include a given Unicode character
in the corresponding output file.  By far the easiest approach is to
use \xelatex instead of pdf\LaTeX\index{pdfLaTeX=pdf\LaTeX} or
ordinary \latex.  \xelatex handles Unicode input and output natively
and can utilize system fonts directly without having to expose them
via \texttt{.tfm}, \texttt{.fd}, and other such files.  To output a
Unicode character, a \xelatex document can either include that
character directly as \utfviii text or use \tex's \cmd{\char}
primitive, which \xelatex extends to accept numbers larger than~255.

\DeclareRobustCommand{\trafficsign}{\includegraphics[height=10pt]{./images/traffic-sign-01.png}
}
Suppose we need to declare a traffic sign \trafficsign and for which we have some images ready.


\newfontfamily{\codetwothousand}{code2000.ttf}
\newfontfamily{\codetwothousandone}{code2001.ttf}
  \newfontfamily{\symbola}{symbola.ttf}
  
\DeclareRobustCommand{\versicle}{%
  \raisebox{-2.2bp}{\includegraphics{./images/versicle.jpg}}\kern-1pt}
\DeclareRobustCommand{\response}{%
  \raisebox{-1.2bp}{\includegraphics{./images/response.jpg}}\kern-1pt}
\newcommand{\versicleIDX}{\index{versicle=versicle (\versicle)}}
\newcommand{\responseIDX}{\index{response=response (\response)}}

Suppose we want to output the symbols for
versicle\versicleIDX~(``\versicle'') and
response\responseIDX~(``\response'') in a document.  The Unicode
charts list ``versicle\versicleIDX'' at position~U+2123 ({\codetwothousand\char"2123}) and
``response\responseIDX'' at position~U+211F ({\codetwothousand\char"211F}).  We therefore need to
install a font that contains those characters at their proper
positions.  One such font that is freely available from CTAN\idxCTAN{}
is Junicode Regular (\docfilename{Junicode-Regular.ttf}) from the
\pkgname{junicode} package.  

The \pkgname{fontspec} package makes it
easy for a \xelatex or \lualatex document to utilize a system font.  The following
example defines a \texttt{\string\textjuni} command that uses
\pkgname{fontspec} to typeset its argument in Junicode Regular:

\begin{verbatim}
   \documentclass{article}
   \usepackage{fontspec}

   \newcommand{\textjuni}[1]{{\fontspec{Junicode-Regular}#1}}

   \begin{document}
   We use ``\textjuni{\char"2123}'' for a versicle
   and ``\textjuni{\char"211F}'' for a response.
   \end{document}
\end{verbatim}

\noindent
which produces

\begin{quotation}
  We use ``\versicle'' for a versicle\versicleIDX\ and ``\response''
  for a response\responseIDX.
\end{quotation}

\noindent
(Typesetting the entire document in Junicode Regular would be even
easier.  \seedocs{\pkgname{fontspec}} regarding font selection.)  Note
how the preceding example uses \cmd{\char} to specify a Unicode
character by number.  The double quotes before the number indicate
that the number is represented in hexadecimal instead of decimal.

\index{Unicode|)}

\section{XeLaTeX and fontspec}

\index{maths>fontspec}
The best option so far for math fonts using XeLaTeX and \pkgname{fontspec} is to use the option |no-math|. When typesetting this document for example there were numerous problems with accents (I lost the |ring| accent, until I used this option. Unicode math fonts are not available in large numbers.



% Because the Math Alphabets table is a bit different from the symbol
% tables in this document we start it on its own page to emphasize it
% and to include enough room for some of the table notes.
\clearpage

\begin{symtable}{Math Alphabets}
\idxboth{math}{alphabets}
\label{alphabets}
\begin{tabular}{@{}*3l@{}}
\toprule
Font sample & Generating command & Required package           \\
\midrule
\Wf\mathrm{ABCabc123}    & \textit{none}                      \\
\Ww\textit\mathit{ABCabc123}    & \textit{none}               \\
\Wf\mathnormal{ABCabc123}& \textit{none}                      \\
|\Ww\CMcal\mathcal{ABC}|   & \textit{none}                      \\

\ifx\mathscr\undefined\else
\Wf\mathscr{ABC}         & \pkgname{mathrsfs} \\
\multicolumn{1}{r@{}}{\emph{or}}
        &\verb|\mathcal{ABC}|
                         & \pkgname{calrsfs} \\
\fi
%
%\ifEU
%\Wf\mathcal{ABC}         & \pkgname{euscript} with the
%                           \optname{euscript}{mathcal} option \\
%\multicolumn{1}{r@{}}{\emph{or}}
%        &\verb|\mathscr{ABC}|
%                         & \pkgname{euscript} with the
%                           \optname{euscript}{mathscr} option \\
%\fi

\ifx\mathpzc\undefined\else
\Wf\mathpzc{ABCdef123}   & \textit{none}; manually defined$^*$    \\
\fi

\ifx\mathbb\undefined\else
\Wf\mathbb{ABC}          & \pkgname{amsfonts},%
                           \ifx\MSYMmathbb\undefined\else$^\S$~\fi
                           \pkgname{amssymb}, \pkgname{txfonts}, or
                           \pkgname{pxfonts} \\
\fi

\ifx\varmathbb\undefined\else
\Wf\varmathbb{ABC}       & \pkgname{txfonts} or \pkgname{pxfonts} \\
\fi

%\ifx\BBmathbb\undefined\else
%\Ww\BBmathbb\mathbb{ABCdef123}
%                         & \pkgname{bbold} or \pkgname{mathbbol}$^\dag$  \\
%\fi
%
%\ifx\MBBmathbb\undefined\else
%\Ww\MBBmathbb\mathbb{ABCdef123}
%                         & \pkgname{mbboard}$^\dag$              \\
%\fi

%\ifx\mathbbm\undefined\else
%\Wf\mathbbm{ABCdef12}    & \pkgname{bbm}                         \\
%\Wf\mathbbmss{ABCdef12}  & \pkgname{bbm}                         \\
%\Wf\mathbbmtt{ABCdef12}  & \pkgname{bbm}                         \\
%\fi

\ifx\mathds\undefined\else
\Wf\mathds{ABC1}         & \pkgname{dsfont}                      \\
\Ww\mathdsss\mathds{ABC1}
                         & \pkgname{dsfont} with the
                           \optname{dsfont}{sans} option         \\
\fi

\ifx\symA\undefined\else
\symA\symB\symC & \docAuxCommand{symA}\docAuxCommand{symB}\docAuxCommand{symC}
                         & \pkgname{china2e}$^\ddag$             \\
\fi

\ifx\mathfrak\undefined\else
\Wf\mathfrak{ABCdef123}  & \pkgname{eufrak}                      \\
\fi

\ifx\textfrak\undefined\else
\Wf\textfrak{ABCdef123}  & \pkgname{yfonts}$^\P$                 \\
\Wf\textswab{ABCdef123}  & \pkgname{yfonts}$^\P$                 \\
\Wf\textgoth{ABCdef123}  & \pkgname{yfonts}$^\P$                 \\
\fi
\bottomrule
\end{tabular}
\end{symtable}
\unskip



\begin{center}
\ifx\mathpzc\undefined\else
\bigskip
\begin{tablenote}[*]
  Put ``\verb|\DeclareMathAlphabet{\mathpzc}{OT1}{pzc}{m}{it}|'' in your
  document's preamble to make \verb|\mathpzc| typeset its argument in
  \PSfont{Zapf Chancery}.
\ifx\textcalligra\undefined\else
  As a similar trick, you can typeset the \PSfont{Calligra} font's
  script ``{\Large\textcalligra{r}\,}'' (or other calligraphic symbols)
  in math mode by loading the \pkgname{calligra} package and putting
  ``\verb|\DeclareMathAlphabet{\mathcalligra}{T1}{calligra}{m}{n}|''
  in your document's preamble to make \verb|\mathcalligra| typeset its
  argument in the \PSfont{Calligra} font.  (You may also want to
  specify
  ``\verb|\DeclareFontShape{T1}{calligra}{m}{n}{<->s*[2.2]callig15}{}|''
  to set \PSfont{Calligra} at 2.2~times its design size for a better
  blend with typical body fonts.)
\fi
\end{tablenote}
\fi

\ifx\BBmathbb\undefined\else
\bigskip
\begin{tablenote}[\dag]
  The \pkgname{mathbbol} package defines some additional blackboard bold
  characters: parentheses, square brackets, angle brackets, and---if
  the \optname{mathbbol}{bbgreekl} option is passed to
  \pkgname{mathbbol}---Greek\index{Greek>blackboard bold} letters.  For
  instance,
%  ``$\BBmathbb{\char`<\char`[\char`(\char"0B\char"0C\char"0D\char`)\char`]\char`>}$''
%  is produced by
%  ``\cmd{\mathbb}\verb|{|\docAuxCommand{Langle}\linebreak[1]%
%  \docAuxCommand{Lbrack}\linebreak[1]\docAuxCommand{Lparen}\linebreak[1]%
%  \docAuxCommand{bbalpha}\linebreak[1]\docAuxCommand{bbbeta}\linebreak[1]%
%  \docAuxCommand{bbgamma}\linebreak[1]\docAuxCommand{Rparen}\linebreak[1]%
%  \docAuxCommand{Rbrack}\linebreak[1]\docAuxCommand{Rangle}\verb|}|''.

  \ifx\MBBmathbb\undefined
    \pkgname{mbboard} extends the blackboard bold symbol set
    significantly further.  It supports not only the
    Greek\index{Greek>blackboard bold}\index{alphabets>Greek}
    alphabet---including ``Greek-like'' symbols such as
    \cmd{\bbnabla}---but also \emph{all} punctuation marks, various
    currency\idxboth{currency}{symbols}\idxboth{monetary}{symbols}
    symbols such as \cmd{\bbdollar} and \cmd{\bbeuro},\index{euro
    signs>blackboard bold} and the
    Hebrew\index{Hebrew}\index{alphabets>Hebrew} alphabet.
  \else
    \pkgname{mbboard} extends the blackboard bold symbol set
    significantly further.  It supports not only the
    Greek\index{Greek>blackboard bold}\index{alphabets>Greek}
    alphabet---including ``Greek-like'' symbols such as
    \docAuxCommand{bbnabla}~(``\bbnabla'')---but also \emph{all} punctuation
    marks, various
    currency\idxboth{currency}{symbols}\idxboth{monetary}{symbols}
    symbols such as \docAuxCommand{bbdollar}~(``\bbdollar'') and
    \docAuxCommand{bbeuro}~(``\bbeuro''),\index{euro signs>blackboard bold}
    and the Hebrew\index{Hebrew}\index{alphabets>Hebrew}
    alphabet~(e.g.,~``\docAuxCommand{bbfinalnun}\linebreak[1]\docAuxCommand{bbyod}%
    \linebreak[1]\docAuxCommand{bbqof}\linebreak[1]\docAuxCommand{bbpe}''~$\rightarrow$
    ``\bbfinalnun\bbyod\bbqof\bbpe'').
  \fi    t
\end{tablenote}
\fi

\ifx\symA\undefined\else
\bigskip
\begin{tablenote}[\ddag]
  The \verb|\sym|\dots\ commands provided by the \pkgname{package} are
  actually text-mode commands.  They are included in \ref{alphabets}
  because they resemble the blackboard-bold symbols that appear in the
  rest of the table.  In addition to the 26 letters of the English
  alphabet, \CHINA\ provides three umlauted%
  \index{accents>diaeresis=di\ae{}resis (\blackacchack\")}  % 
  blackboard-bold letters:
  \docAuxCommand{symAE}~(``\symAE''), \docAuxCommand{symOE}~(``\symOE''), and
  \docAuxCommand{symUE}~(``\symUE'').  Note that \CHINA\ does provide
  math-mode commands for the most common number-set symbols.  These
  are presented in \vref{china-numsets}.
\end{tablenote}
\fi

\ifx\textfrak\undefined\else
\bigskip
\begin{tablenote}[\P]
  As their \verb|\text|\dots{} names imply, the fonts provided by the
  \pkgname{yfonts} package are actually text fonts.  They are
  included in \ref{alphabets} because they are frequently used
  in a mathematical context.
\end{tablenote}
\fi

\ifx\MSYMmathbb\undefined\else
\bigskip
\begin{tablenote}[\S]
  An older (i.e.,~prior to~1991) version of the \AMS's fonts rendered
  $\mathbb{C}$, $\mathbb{N}$, $\mathbb{R}$, $\mathbb{S}$,
  and~$\mathbb{Z}$ as $\MSYMmathbb{C}$, $\MSYMmathbb{N}$,
  $\MSYMmathbb{R}$, $\MSYMmathbb{S}$, and~$\MSYMmathbb{Z}$.  As some
  people prefer the older glyphs---much to the \AMS's surprise---and
  because those glyphs fail to build under modern versions of
  \metafont, \person{Berthold}{Horn} uploaded \postscript fonts for
  the older blackboard-bold glyphs to CTAN\idxCTAN{}, to the
  \texttt{fonts/msym10} directory.  As of this writing, however, there
  are no \latexE packages for utilizing the now-obsolete glyphs.
\end{tablenote}
\fi
\end{center}


\idxbothend{mathematical}{symbols}


\bgroup
\renewcommand\arraystretch{1.4}
\newcommand\leg[1]{{\tiny\tt\char92#1}}
\newcommand\sho[1]{{\large #1}}
\begin{tabular}{|*{10}{c}|} \hline
\leg{Pickup} &
\leg{Letter} & 
\leg{Mobilefone} &
\leg{Telefon} &
\leg{fax} &
\leg{FAX} &
\leg{Faxmachine} &
\leg{Email} &
\leg{Lightning} &
\leg{EmailCT} \\
\sho{\Pickup} &
\sho{\Letter} &
\sho{\Mobilefone} &
\sho{\Telefon} &
\sho{\fax} &
\sho{\FAX} &
\sho{\Faxmachine} &
\sho{\Email} &
\sho{\Lightning} &
\sho{\EmailCT} \\
\hline
\end{tabular}

\begin{tabular}{|*{8}{c}|} \hline
\leg{Beam} &
\leg{Bearing} &
\leg{LooseBearing} &
\leg{FixedBearing} &
\leg{LeftTorque} &
\leg{RightTorque} &
\leg{Lineload} &
\leg{MVArrowDown} \\
\sho{\Beam} &
\sho{\Bearing} &
\sho{\LooseBearing} &
\sho{\FixedBearing} &
\sho{\LeftTorque} &
\sho{\RightTorque} &
\sho{\Lineload} &
\sho{\MVArrowDown} \\
\hline
\leg{OktoSteel} &
\leg{HexaSteel} &
\leg{SquareSteel} & 
\leg{RectSteel} &
\leg{CircSteel} &
\leg{SquarePipe} &
\leg{RectPipe} &
\leg{CircPipe}
\\
\sho{\OktoSteel} &
\sho{\HexaSteel} &
\sho{\SquareSteel} &
\sho{\RectSteel} &
\sho{\CircSteel} &
\sho{\SquarePipe} &
\sho{\RectPipe} &
\sho{\CircPipe}
\\ \hline
\leg{LSteel} &
\leg{RoundedLSteel} &
\leg{TSteel} &
\leg{RoundedTSteel} &
\leg{TTsteel} &
\leg{RoundedTTSteel} &
\leg{FlatSteel} &
\leg{Valve}
\\
\sho{\LSteel} &
\sho{\RoundedLSteel} &
\sho{\TSteel} &
\sho{\RoundedTSteel} &
\sho{\TTSteel} &
\sho{\RoundedTTSteel} &
\sho{\FlatSteel} &
\sho{\Valve}
\\ \hline
\end{tabular}

\subsection{Information}

\begin{tabular}{|*{8}{c}|} \hline
\leg{Industry} &
\leg{Coffeecup} &
\leg{LeftScissors} &
\leg{CuttingLine} &
\leg{RightScissors} &
\leg{Football} &
\leg{Bicycle} & \\
\sho{\Industry} &
\sho{\Coffeecup} &
\sho{\LeftScissors} &
\sho{\CuttingLine} &
\sho{\RightScissors} &
\sho{\Football} &
\sho{\Bicycle} & \\
\hline
\leg{Info} &
\leg{ClockLogo} &
\leg{CutRight} &
\leg{CutLineine} &
\leg{CutLeft} &
\leg{Wheelchair} &
\leg{Gentsroom} &
\leg{Ladiesroom} \\
\sho{\Info} &
\sho{\ClockLogo} &
\sho{\CutRight} &
\sho{\CutLine} &
\sho{\CutLeft} &
\sho{\Wheelchair} &
\sho{\Gentsroom} &
\sho{\Ladiesroom} \\
\hline
\leg{Checkedbox} &
\leg{CrossedBox} &
\leg{HollowBox} &
\leg{PointingHand} &
\leg{WritingHand} &
\leg{MineSign} &
\leg{Recycling} &
\leg{PackingWaste} \\
\sho{\Checkedbox} &
\sho{\CrossedBox} &
\sho{\HollowBox} &
\sho{\PointingHand} &
\sho{\WritingHand} &
\sho{\MineSign} &
\sho{\Recycling} &
\sho{\PackingWaste} \\
\hline
\end{tabular}

\subsection{Laundry}

\begin{tabular}{|*{8}{c}|} \hline
\leg{WashCotton} &
\leg{WashSynthetics} &
\leg{WashWool} &
\leg{HandWash} &
\leg{NoWash} &
\leg{Tumbler} &
\leg{NoTumbler} &
\leg{NoChemicalCleaning} \\
\sho{\WashCotton} &
\sho{\WashSynthetics} &
\sho{\WashWool} &
\sho{\HandWash} &
\sho{\NoWash} &
\sho{\Tumbler} &
\sho{\NoTumbler} &
\sho{\NoChemicalCleaning} \\
\hline
\leg{Bleech} &
\leg{NoBleech} &
\leg{CleaningA} &
\leg{CleaningP} &
\leg{CleaningPP} &
\leg{CleaningF} &
\leg{CleaningFF} & \\
\sho{\Bleech} &
\sho{\NoBleech} &
\sho{\CleaningA} &
\sho{\CleaningP} &
\sho{\CleaningPP} &
\sho{\CleaningF} &
\sho{\CleaningFF} & \\
\hline
\leg{IroningI} &
\leg{IroningII} &
\leg{IroningIII} &
\leg{NoIroning} &
\leg{AtNinetyFive} &
\leg{ShortNinetyFive} &
\leg{AtSixty} &
\leg{ShortSixty} \\
\sho{\IroningI} &
\sho{\IroningII} &
\sho{\IroningIII} &
\sho{\NoIroning} &
\sho{\AtNinetyFive} &
\sho{\ShortNinetyFive} &
\sho{\AtSixty} &
\sho{\ShortSixty} \\
\hline
\leg{ShortFifty} &
\leg{AtForty} &
\leg{ShortForty} &
\leg{SpecialForty} &
\leg{ShortThirty} &&& \\
\sho{\ShortFifty} &
\sho{\AtForty} &
\sho{\ShortForty} &
\sho{\SpecialForty} &
\sho{\ShortThirty} &&& \\
\hline
\end{tabular}

\subsection{Currency}

\begin{tabular}{|*{11}{c}|} \hline
\leg{EUR} &
\leg{EURdig} &
\leg{EURhv} &
\leg{EURcr} &
\leg{EURtm} &
\leg{Ecommerce} &
\leg{Shilling} &
\leg{Denarius} &
\leg{Pfund} &
\leg{EyesDollar} &
\leg{Florin} \\
 &
\leg{EurDig} &
\leg{EurHv} &
\leg{EurCr} &
\leg{EurTm} &
\leg{EstimatedSign} &
 &
\leg{Deleatur} &
 &
 &
 \\
\sho{\EUR} &
\sho{\EurDig} &
\sho{\EurHv} &
\sho{\EurCr} &
\sho{\EurTm} &
\sho{\EstimatedSign} &
\sho{\Shilling} &
\sho{\Deleatur} &
\sho{\Pfund} &
\sho{\EyesDollar} &
\sho{\Florin} \\
\hline
\end{tabular}
\label{currencysymbols} 

\subsection{Safety}

\begin{tabular}{|*{8}{c}|} \hline
\leg{Stopsign} &
\leg{CESign} &
\leg{Estatically} &
\leg{Explosionsafe} &
\leg{Laserbeam} &
\leg{Biohazard} &
\leg{Radioactivity} &
\leg{BSEFree} \\
\sho{\Stopsign} &
\sho{\CESign} &
\sho{\Estatically} &
\sho{\Explosionsafe} &
\sho{\Laserbeam} &
\sho{\Biohazard} &
\sho{\Radioactivity} &
\sho{\BSEFree} \\
\hline
\end{tabular}

\subsection{Navigation}

\begin{tabular}{|*{10}{c}|} \hline
\leg{RewindToIndex} &
\leg{RewindToStart} &
\leg{Rewind} &
\leg{Forward} &
\leg{ForwardToEnd} &
\leg{ForwardToIndex} &
\leg{MoveUp} &
\leg{MoveDown} &
\leg{ToTop} &
\leg{ToBottom} \\
\sho{\RewindToIndex} &
\sho{\RewindToStart} &
\sho{\Rewind} &
\sho{\Forward} &
\sho{\ForwardToEnd} &
\sho{\ForwardToIndex} &
\sho{\MoveUp} &
\sho{\MoveDown} &
\sho{\ToTop} &
\sho{\ToBottom} \\
\hline
\end{tabular}

\subsection{Computers}

\begin{tabular}{|*{6}{c}|} \hline
\leg{ComputerMouse} &
\leg{SerialInterface} &
\leg{Keyboard} &
\leg{SerialPort} &
\leg{ParallelPort} &
\leg{Printer} \\
\sho{\ComputerMouse} &
\sho{\SerialInterface} &
\sho{\Keyboard} &
\sho{\SerialPort} &
\sho{\ParallelPort} &
\sho{\Printer} \\
\hline
\end{tabular}

\subsection{Numbers}

\begin{tabular}{|*{10}{c}|} \hline
\leg{MVZero} &
\leg{MVOne} &
\leg{MVTwo} &
\leg{MVThree} &
\leg{MVFour} &
\leg{MVFive} &
\leg{MVSix} &
\leg{MVSeven} &
\leg{MVEight} &
\leg{MVNine} \\
\sho{\MVZero} &
\sho{\MVOne} &
\sho{\MVTwo} &
\sho{\MVThree} &
\sho{\MVFour} &
\sho{\MVFive} &
\sho{\MVSix} &
\sho{\MVSeven} &
\sho{\MVEight} &
\sho{\MVNine} \\
\hline
\end{tabular}

\subsection{Maths}

\begin{tabular}{|*{8}{c}|} \hline
\leg{MVLeftBracket} &
\leg{MVRightBracket} &
\leg{MVComma} &
\leg{MVPeriod} &
\leg{MVMinus} &
\leg{MVPlus} &
\leg{MVDivision} &
\leg{MVMultiplication} \\
\sho{\MVLeftBracket} &
\sho{\MVRightBracket} &
\sho{\MVComma} &
\sho{\MVPeriod} &
\sho{\MVMinus} &
\sho{\MVPlus} &
\sho{\MVDivision} &
\sho{\MVMultiplication} \\
\hline
% \end{tabular}
% 
% \begin{tabular}{|*{10}{c}|} \hline
\leg{Conclusion} &
\leg{Equivalence} &
\leg{barOver} &
\leg{BarOver} &
\leg{arrowOver} &
\leg{ArrowOver} &
\leg{StrikingThrough} &
\leg{MultiplicationDot} \\
\sho{\Conclusion} &
\sho{\Equivalence} &
\sho{\barOver} &
\sho{\BarOver} &
\sho{\arrowOver} &
\sho{\ArrowOver} &
\sho{\StrikingThrough} &
\sho{\MultiplicationDot} \\
\hline
% \end{tabular}
 
% \begin{tabular}{|*{10}{c}|} \hline
\leg{LessOrEqual} &
\leg{LargerOrEqual} &
\leg{AngleSign} &
\leg{Corresponds} &
\leg{Congruent} &
\leg{NotCongruent} &
\leg{Divides} &
\leg{DividesNot} \\
\sho{\LessOrEqual} &
\sho{\LargerOrEqual} &
\sho{\AngleSign} &
\sho{\Corresponds} &
\sho{\Congruent} &
\sho{\NotCongruent} &
\sho{\Divides} &
\sho{\DividesNot} \\
\hline
\end{tabular}

 \subsection{Biology}
 
 \begin{tabular}{|*{10}{c}|} \hline
 \leg{Neutral} &
 \leg{Male} &
 \leg{Hermaphrodite} &
 \leg{Female} &
 \leg{MALE} &
 \leg{HERMAPHRODITE} &
 \leg{FEMALE} &
 \leg{MaleMale} &
 \leg{FemaleFemale} &
 \leg{FemaleMale} \\
 \sho{\Neutral} &
 \sho{\Male} &
 \sho{\Hermaphrodite} &
 \sho{\Female} &
 \sho{\MALE} &
 \sho{\HERMAPHRODITE} &
 \sho{\FEMALE} &
 \sho{\MaleMale} &
 \sho{\FemaleFemale} &
 \sho{\FemaleMale} \\
 \hline
 \end{tabular}

\subsection{Biology}

\begin{tabular}{|*{4}{c}|} \hline
\leg{Female} &
\leg{Male} &
\leg{Hermaphrodite} &
\leg{Neutral} \\
\sho{\Female} &
\sho{\Male} &
\sho{\Hermaphrodite} &
\sho{\Neutral} \\
\hline
\leg{FEMALE} &
\leg{MALE} &
\leg{HERMAPHRODITE} & \\
\sho{\FEMALE} &
\sho{\MALE} &
\sho{\HERMAPHRODITE} & \\
\hline
\leg{FemaleFemale} &
\leg{MaleMale} &
\leg{FemaleMale} & \\
\sho{\FemaleFemale} &
\sho{\MaleMale} &
\sho{\FemaleMale} & \\
\hline
\end{tabular}

\subsection{Astronomy}

\begin{tabular}{|*{11}{c}|} \hline
\leg{Sun} &
\leg{Moon} &
\leg{Mercury} &
\leg{Venus} &
\leg{Mars} &
\leg{Jupiter} &
\leg{Saturn} &
\leg{Uranus} &
\leg{Neptune} &
\leg{Pluto} &
\leg{Earth} \\
\sho{\Sun} &
\sho{\Moon} &
\sho{\Mercury} &
\sho{\Venus} &
\sho{\Mars} &
\sho{\Jupiter} &
\sho{\Saturn} &
\sho{\Uranus} &
\sho{\Neptune} &
\sho{\Pluto} &
\sho{\Earth} \\
\hline
\end{tabular}

\subsection{Astrology}



\begin{tabular}{|*{12}{c}|} \hline
\leg{Aries} &
\leg{Taurus} &
\leg{Gemini} &
\leg{Cancer} &
\leg{Leo} &
\leg{Virgo} &
\leg{Libra} &
\leg{Scorpio} &
\leg{Sagittarius} &
\leg{Capricorn} &
\leg{Aquarius} &
\leg{Pisces} \\
\sho{\Aries} &
\sho{\Taurus} &
\sho{\Gemini} &
\sho{\Cancer} &
\sho{\Leo} &
\sho{\Virgo} &
\sho{\Libra} &
\sho{\Scorpio} &
\sho{\Sagittarius} &
\sho{\Capricorn} &
\sho{\Aquarius} &
\sho{\Pisces} \\
\hline
\end{tabular}

\subsection{Others}

\begin{tabular}{|*{10}{c}|} \hline
\leg{YinYang} &
\leg{MVRightArrow} &
\leg{MVAt} &
\leg{BOLogo} &
\leg{BOLogoL} &
\leg{BALogoP} &
\leg{Mundus} &
\leg{Cross} &
\leg{CeltCross} &
\leg{Ankh} \\
\sho{\YinYang} &
\sho{\MVRightArrow} &
\sho{\MVAt} &
\sho{\BOLogo} &
\sho{\BOLogoL} &
\sho{\BOLogoP} &
\sho{\Mundus} &
\sho{\Cross} &
\sho{\CeltCross} &
\sho{\Ankh} \\
\hline
\leg{Heart} &
\leg{CircledA} &
\leg{Bouquet} &
\leg{Frowny} &
\leg{Smiley} &
\leg{PeaceDove} &
\leg{Bat} &
\leg{WomanFace} &
\leg{ManFace} & \\
\sho{\Heart} &
\sho{\CircledA} &
\sho{\Bouquet} &
\sho{\Frowny} &
\sho{\Smiley} &
\sho{\PeaceDove} &
\sho{\Bat} &
\sho{\WomanFace} &
\sho{\ManFace} & \\
\hline
\end{tabular}


\egroup

\thetotalsymbols











 































%  \part{Graphics}
%  \chapter{Drawing pictures and graphs}
\epigraph{Dear God\break If I have but one hour remaining to live, please allow me to spend this time
in a mathematics class so that it will seem to last forever.}{\textit{---A bored student's prayer}}


\begin{figure}%
  \centering
  \includegraphics[width=0.3\linewidth]{./graphics/pic37.png}
  \caption{During the early days of typography fonts were designed to emulate the looks of calligraphic texts.}
  \label{fig:marginfig1}
\end{figure}

\parindent=0em
\parskip=0.25\baselineskip plus .25pt minus .25pt\relax

\section{Inserting figures}

In order to insert figures, the \pkgname{graphicx} package has to included in the preamble (before the |\begin{document}|-command) of your LaTeX-document:

\begin{verbatim}
\usepackage{graphicx}
\end{verbatim}

Originally only EPS-figures could be inserted with the \pkgname{graphic}package. This has now been developed into the  \pkgname{graphicx}, which allows almost any common format to be inserted. 

The simplest way of including a graphic looks like this:


\CMDI{\includegraphics}\marg{filename}


If the image is not located in the same folder as the tex-file, you will have to specify the path relative to the tex-file.

\begin{verbatim}
\includegraphics{./images/filename}
\end{verbatim}


\subsection{Scaling and resizing images}

If you want the image to appear in a different size, you can specifiy the size as a parameter of the |\includegraphics|-command::

\begin{commands}[]{ex:graphics}
\cmd{\includegraphics}\oarg{width=3.9cm}\marg{filename}
\end{commands}

This will scale the image to the width of 3.9 centimeters. 

Use |\textwidth| command if you don't want to specify an absolute size but rather want the actual size to depend on the text width of the page. You can use any of the normal \tex units such as \texttt{em, pt, cm, in}:

\begin{commands}[]{ex:graphics}
\cmd{\includegraphics}\oarg{width=0.5\string\textwidth}\marg{filename}
\end{commands}

\noindent will scale the image to half of the text width. The images in the
figure below were produced by three |\includegraphics| commands. You can have as many as you like and the \tex engine will treat them the same way as text. If you a leave a space between the commands, they will be positioned vertically as they are treated as paragraphs.

\medskip

\begin{commands}[]{}
\begingroup

\centering
\includegraphics[width=0.3\textwidth]{./graphics/amato.jpg}
\includegraphics[width=0.3\textwidth]{./graphics/amato.jpg}
\includegraphics[width=0.3\textwidth]{./graphics/amato.jpg}

\endgroup

\begin{verbatim}
\begingroup

\centering
\includegraphics[width=0.3\textwidth]{./graphics/amato.jpg}
\includegraphics[width=0.3\textwidth]{./graphics/amato.jpg}
\includegraphics[width=0.3\textwidth]{./graphics/amato.jpg}

\endgroup
\end{verbatim}
\captionof{figure}{Images aligned horizontally.}
\end{commands}

The three photos were centered using the |\centering| command, within a group. The |\begingroup..\endgroup| is necessary to limit the effect of centering to
the group only, otherwise \tex would center everything from this point onwards.

\subsection{Controlling the aspect ratio}

You can control the picture aspect ratio by using the command:

\begin{commands}[]{}
\cmd{\includegraphics}\marg{keepaspectratio,width=3cm, height=3cm}\oarg{filename}
\end{commands}

If the key |keepaspectratio| is set to true then specifying 
both |width| and |height| (or |totalheight|) does not distort the figure but 
scales such that neither of the specified dimensions is exceeded.

\medskip
\begin{commands}[]{}
\begingroup

\centering
\includegraphics[width=0.3\textwidth, height=5cm]{./images/amato.jpg}
\includegraphics[keepaspectratio=true,width=4cm, height=5cm]{./images/amato.jpg}
\includegraphics[width=3cm]{./images/amato.jpg}

\endgroup

\begin{verbatim}
\begingroup

\centering
\includegraphics[width=0.3\textwidth, height=5cm]{./images/amato.jpg}
\includegraphics[keepaspectratio=true,width=4cm, height=5cm]{./images/amato.jpg}
\includegraphics[width=3cm]{./images/amato.jpg}

\endgroup

\end{verbatim}
\captionof{figure}{Controlling the aspect ratio.}
\end{commands}

This can be very useful if you have images shown side by side with different
aspect ratios. 


\subsection{Paths and file types}

For larger projects you will probably find it more convenient to have 
images in different folders. You can specify default paths using:


\CMDI{\graphicspath}\marg{dir-list}

This optional declaration may be used to specify a list of directories in which to
search for graphics files. The format is the same as for the \latexe primitive
|\input@path|. A list of directories, each in a \{\} group (even if there is only one
in the list). For example:


\graybox{\texttt{\textbackslash graphicspath\{\{eps/\}\{tiff/\}\}}}


The default image formats can be declared using:

\CMDI{\DeclareGraphicsExtensions}\marg{png, jpg}

This specifies the behaviour of the system when no file extension is specified in 
the argument to |\includegraphics|. \texttt{\{ext-list\}} should be a comma separated 
list of file extensions. (White space is ignored between the entries.) A file name
is produced by appending one extension from the list. If a file is found, the
system acts as if that extension had been specified. If not, the next extension
in \texttt{ext-list} is tried.



\subsection{The figure environment}

You use the figure-environment to let your image appear in a \emph{floating} environment, that is \latex will place it at the right position of a page and even on the next page:

\begin{teX}
\begin{figure}
  \includegraphics{filename.jpg}
  \caption{title of your figure}
  \label{labelname}
\end{figure}
\end{teX}

Here |\caption{...}| defines the title of the figure which will appear beneath the figure. |\label{..}| defines the label which can be used inside the document in order to insert references to the figure:

The figure

|\ref{labelname} on page \pageref{labelname} ..|

The|\label-command| inside the |\figure|-envirnonment hast to appear just after the|\caption|-command.
placing figures

If figures reside inside a |\figure|-environment, this will cause LaTeX to choose the actual location of the figure inside the document. There are different parameters for the placement strategy:

\begin{description}
\item[h (here)] Try to place the figure just where the command is located.

\item [t (top)] Try to place the figure at the top of the page.

\item[b (bottom)] Try to place the figure at the bottom of the page.

\item [p (float page)] Try to place the figure on a page which contains only floating elements.
\end{description}

The order of these parameters doesn't matter since placement is always tried in the order \textbf{h, t, b, p,} if these parameters are present:

If no parameter is present, the default order is  \texttt{[tbp]}.


The command for a figure-environment might for example look like this:

\begin{teX}
\begin{figure}[htbp]
...
\end{figure}
\end{teX}



\subsection{Table of figures}
\index{figures!Table of figures}
A table of figures is inserted (where you place the command) using the command


\begin{teX}
   \listoffigures
\end{teX}

The caption given in the \cmd{caption} command is also used in the list of figures. 
If you want to use different captions, you may add a parameter to the |\caption| command:
|\caption[caption for listoffigures]{caption inside the document}|


\subsection{Figures with a border}

Although drawing frames around tables should be discouraged, if you find the need
to draw them there are  two possible ways to achieve it: either only the figure itself is bordered or there is a border around the figure and its caption. You place a border around the figure using the \cmd{\fbox} command or the \cmd{\framebox}.

\emphasis{fbox,minipage}
\begin{teX}
\begin{figure}[htbp]
  \centering
  \fbox{
    \includegraphics{filename}
  }
  \caption{caption}
  \label{Labelname}
\end{figure}
\end{teX}

Placing a border around the figure and its title is a little more tricky: You need to place the figure and the title in a |\minipage| environment which is bordered again with the |\fbox| command:

\begin{figure}[htbp]
\begin{commands}[]{}
\centering
  \fbox{
    \begin{minipage}{.95\linewidth}
      \mbox{}
      \centering
      
      \includegraphics[width=.9\linewidth]{./images/asia.jpg}
      \caption{How to place a border around an image. }
      \label{labelname}
    \end{minipage}}
 
\begin{verbatim}
\begin{figure}[htbp]
  \centering
    \begin{minipage}{width=.8\linewidth}
     \centering
     
      \includegraphics[.9\linewidth]{filename}
      \caption{caption}
      \label{labelname}
    \end{minipage}
 \end{figure}
\end{verbatim}
\end{commands}
\end{figure}



Unfortunately the width of the border cannot be determined automatically. It has to be specified as a parameter of the |\minipage| environment. However, you may be bale to develop a macro to do this,  based on the ImageSize routines we developed in section.


\section{Complex Layouts}
\label{looting}
In reality most professionally typeset books will have their own style for image pages. In Figure~\ref{complex}
three images are set in a non-symmetrical layout. This type of setting is difficult to automate and manual intervention is possible.

This layout will require four minipages. Two for the top figure (one for the image and one for the caption) and two for the two bottom figures. The rightmost bottom figure will have to be put in a zero height box to let it overflow to the top. The figure has been reproduced from an Oriental Institute publication \emph{Catastrophe! The Looting and Destruction of Iraq’s Past} \cite{looting}. The book is interesting both for its contents as well as its simple but effective typography and appropriate for the topic. The volume has been pblished in conjuction with the exhibition titled as the name of the book, that described the loss of Iraq’s archaeological past to looters and to the war. 

\begin{figure}[p]
\centering
\includegraphics[height=0.8\textheight]{oriental}
\caption{More complex layouts. \emph{Copyright the Oriental Institute of the University of Chicago.}}
\label{complex}
\end{figure}

The style is reproduced in a \pkgname{phd} template (style 56) and both code and details can be found in the relevant pages.





\section{Side by side figures}

You might want to place to figures side by side but to use only one caption. This is achieved by placing both figures in its own |\minipage| which reside in the same |\figure|.

if only one |\caption| command is used, both figures will have a common title:

\medskip
\begin{verbatim}
\begin{figure}[htbp]
  \centering
  \begin{minipage}[b]{5 cm}
    \includegraphics{filename 1}  
  \end{minipage}
  \begin{minipage}[b]{5 cm}
    \includegraphics{filename 2}  
  \end{minipage}
  \caption{common caption}
  \label{Labelname}
\end{figure}
\end{verbatim}
\medskip

The first parameter of the |\minipage| environment determines how both graphics are aligned to each other. b (bottom) aligns the bottom borders of the figures, \textbf{t} (top) aligns the top borders and \textbf{c} aligns the centers.

If you want distinct titles for the two figures you will only have to supply a |\caption| command for both |\minipage|environments:

\begin{teX}
\begin{figure}[htbp]
  \centering
  \begin{minipage}[b]{5 cm}
    \includegraphics{filename 1} 
    \caption{caption 1}
    \label{labelname 1}
  \end{minipage}
  \begin{minipage}[b]{5 cm}
    \includegraphics{filename 2}  
    \caption{caption 2}
    \label{labelname 2}
  \end{minipage}
\end{figure}
\end{teX}


If you want to have subfigures with distinct caption, you use the |\subfig| package:


You can put as many figures as you like on a page, but a word of warning, you may need to make some manual adjustments before you get it right. The package provides support for the manipulation and reference of small or ‘sub’ floats within a single floating (e.g., figure or table) environment1 It is convenient to use this
package when your sub-floats are to be separately captioned, referenced, or when such
sub-captions are to be included on a List-of-Floats page.

The package is a replacement for the subfigure package, from which it was derived.
However, the new subfig package is not completely backward compatible.
Therefore, a new name was called for. The newer package is smaller and easier to use
than the older package, however, it now uses the following additional packages, 
caption (required), 
everysel (optional), 
keyval (required), 
ragged2e (optional).

It will work without the \pkgname{ragged2e} and \pkgname{everysel} packages if you do not use the following
justification options: ‘Center’, ‘RaggedRight’ and ‘RaggedLeft’. The other justification
options ‘center’, ‘raggedright’ and ‘raggedleft’ will work without the above two packages. If the ragged2e package is present, than the caption package will load it and it
will, in turn, load the everysel package. This happens whether or not you will be using
the justification options that require it. If it cannot find the ragged2e package, than the
caption package will print a message that ‘RaggedRight’, etc. will not be available.


\begin{figure}[htb]
\includegraphics[height=5cm]{dotty}
\includegraphics[height=5cm]{bette}
\includegraphics[height=5cm]{dotty}
\end{figure}

 A low bottle-shaped vase, of yellowish ware, with flaring rim and somewhat flattened body. Height, 5 inches; width 5 inches. \ref{fig:one}

A well-made bottle shaped vase, with low neck and globular body, somewhat conical above. Color dark brownish. 7½ inches in height. Shown in \ref{fig:two}


\begin{figure}
  \centering
  \includegraphics[width=0.7\linewidth]{./graphics/fig175.jpg}
   \centerline{From the tomb of a Pull\= arius.}
  \label{fig:marginfig1}
\end{figure}

The above figure is an effigy vase of the dark ware. The body is globular. A kneeling human figure forms the neck. The mouth of the vessel occurs at the back of the head—a rule in this class of vessels. Is is finely made and symmetrical. 9¾ inches high and 7 inches in diameter. being larger than the above two it is preferable to scale it to give the reader an indication. Based on the figure width, you may also need to adjust the distance between the figures to ensure that the whitespace is just about right. For screen reading this can be increased and for printed works you may wish to make it less.



\section{The wrapfig package}


\captionsetup[wrapfigure]{margin=10pt,font=small,labelfont=bf, name=Fig.} % [wrapfigure]{name=Fig.}


Donald Arseneau has created the \pkg{wrapfig} package to allow people to place figures or
tables at the side of a page and wrap text around them. The package provides the
environments wrapfigure and wraptable. Both environments have two required and
two optional arguments. You can see an example taht uses the package to wrap a picture into such a paragraph of text.

\begin{figure}[htbp]
   \includegraphics[width=\linewidth]{./graphics/cyprus.jpg} 
   \caption{\small Cyprian limestone group of Phoenician dancers, about 6½ in. high. There is a somewhat similar group, also from Cyprus, in the British Museum. The dress, a hooded cowl, appears to be of great antiquity.}
\end{figure}

\begin{wrapfigure}[20]{l}{3.8cm}
\centering\small
\includegraphics[width=\linewidth]{./graphics/egyptdance.jpg}  
\caption{\small The hieroglyphics describe the dance.}
\end{wrapfigure}
Amongst the earliest representations that are comprehensible, we have certain Egyptian paintings, and some of these exhibit postures that evidently had even then a settled meaning, and were a phrase in the sentences of the art. Not only were they settled at such an early period (B.C. 3000, fig. 1) but they appear to have been accepted and handed down to succeeding generations (fig. 2), and what is remarkable in some countries, even to our own times. The accompanying illustrations from Egypt and Greece exhibit what was evidently a traditional attitude. The hand-in-hand dance is another of these.

The earliest accompaniments to dancing appear to have been the clapping of hands, the pipes,[1] the guitar, the tambourine, the castanets, the cymbals, the tambour, and sometimes in the street, the drum.

The following account of Egyptian dancing is from Sir Gardiner Wilkinson's "Ancient Egypt"[2]:—
\begin{figure}
   \includegraphics[width=0.3\linewidth]{./graphics/lotus.jpg} 
   \caption{\small Cyprian limestone group of Phoenician dancers, about 6½ in. high. There is a somewhat similar group, also from Cyprus, in the British Museum. The dress, a hooded cowl, appears to be of great antiquity.}
\end{figure}
"The dance consisted mostly of a succession of figures, in which the performers endeavoured to exhibit a great variety of gesture. Men and women danced at the same time, or in separate parties, but the latter were generally preferred for their superior grace and elegance. Some danced to slow airs, adapted to the style of their movement; the attitudes they assumed frequently partook of a grace not unworthy of the Greeks; and some credit is due to the skill of the artist who represented the subject, which excites additional interest from its being in one of the oldest tombs of Thebes (B.C. 1450, Amenophis II.). Others preferred a lively step, regulated by an appropriate tune; and men sometimes danced with great spirit, bounding from the ground, more in the manner of Europeans than of Eastern people. On these occasions the music was not always composed of many instruments, and here we find only the cylindrical maces and a woman snapping her fingers in the time, in lieu of cymbals or castanets.

\begin{figure}
   \includegraphics[width=0.3\linewidth]{./graphics/patera.jpg} 
   \caption{\small Cyprian limestone group of Phoenician dancers, about 6½ in. high. There is a somewhat similar group, also from Cyprus, in the British Museum. The dress, a hooded cowl, appears to be of great antiquity.}
\end{figure}

"Graceful attitudes and gesticulations were the general style of their dance, but, as in all other countries, the taste of the performance varied according to the rank of the person by whom they were employed, or their own skill, and the dance at the house of a priest differed from that among the uncouth peasantry, etc.

"It was not customary for the upper orders of Egyptians to indulge in this amusement, either in public or private assemblies, and none appear to have practised it but the lower ranks of society, and those who gained their livelihood by attending festive meetings.

"Many of these postures resembled those of the modern ballet, and the pirouette delighted an Egyptian party 3,500 years ago.
\medskip

The wrapped figure is positioned using the \texttt{wrapfigure} environment, as shown below:

\begin{teX}
\begin{wrapfigure}[nlines]{placement}[overhang ]{width }
   \includegraphics[width=3.8cm]{./graphics/egyptdance} 
   \caption{\small The hieroglyphics describe the dance.}
\end{wrapfigure}
\end{teX}

The parameter |nlines|  is the number of narrow lines, and placement is one of r, l, i, o, R, L, I, or
O for right, le, inside, and outside, respectively. The uppercase placement specifiers
differ from their lowercase counterparts in that they force \latex to put the float \emph{here},
whereas the lowercase placement specifiers just give a hint to \latex to place them
\texttt{here}. The \meta{width} argument is the width of the figure or table that appears in the body
of the environment. Finally, \texttt{overhang} tells \latex how much the figure should hang out
into the margin of the page. Here is how one may create dangerous paragraphs bends!

The |wrapfig| package is compatible with the |caption| package. You can set the caption parameters using:---

\begin{teX}
\captionsetup[wrapfigure]{<options>}
\end{teX}

If you are probably wondering how |wrapfig| achieves this, you should read the package code. It basically uses \refCom{everypar}, and hence the limitations with |\par|. Here is an extract from the class.

\begin{teX}

% Subvert \everypar to float fig and do wrapping.  
% Also for non-float.
\def\WF@startfloating{%
 \WF@everypar\expandafter{\the\everypar}\let\everypar\WF@everypar
 \WF@@everypar{\ifvoid\WF@box\else\WF@floathand\fi \the\everypar
 \WF@wraphand
}}
\end{teX}

Moving now to a more scientific example that the previous ones, we will place two figures
one on top of each other and give them individual, sub-captions as shown in \ref{fig:honey}.
 
\captionsetup[figure]{margin=10pt,font=small,labelfont=bf,format=hang}%

\begin{figure}[htbp]
\centering
  \begin{subfigure}[b]{0.5\textwidth}
  \includegraphics[width=\linewidth]{./graphics/honey.png}
  \caption{Taylor instability in the surface of the honey in an inverted honey jar.}\label{fig:honey}
    \hspace{1cm}
  \end{subfigure}

  \begin{subfigure}[b]{0.9\textwidth}
     \centering
     \includegraphics[width=9cm]{./graphics/honeydrops.png}
     \caption{Taylor instability in the interface of the water condensing on the underside of a small water pipe.}
  \end{subfigure}  
  \caption{Two examples of Taylor instabilities that are commonly found.}%
    \label{fig:Athird}%
\end{figure}

The figures are from \textit{A Heat Transfer Textbook}, by J.H.Lienhard, which incidentally was typeset using
\tex . It is a McGrawHill publication. 

\begin{teX}
\begin{figure}[htbp]
    \captionsetup[figure]{margin=10pt}%
    \subfloat[Taylor instability...]
     {{\includegraphics[width=8cm]{./graphics/honey}}}
    \hspace{1cm}
     \subfloat[Taylor instability in the...]%
      {\includegraphics[width=9cm]{./graphics/honeydrops}}  
     \\[-10pt]
   \caption{Taylor instability in...}%
    \label{fig:Afirst}%
    \caption{Two examples of... }%
    \label{fig:honey}%
\end{figure}
\end{teX}


The text can have more than one paragraph. It is also possible to include figures
generated by |TikZ/pgf|, as shown in the next example, drawn from real code
in the book.


\begin{wrapfigure}[14]{l}{3.0cm}
\pgfplotsset{width=5.0cm,compat=1.3}
\begin{tikzpicture}
\begin{axis}[minor y tick num=4, 
minor x tick num=4, 
xmin=0,xmax=300,
ymin=0,ymax=60,
xlabel=\textsf{liquidus ($l/s$)},
ylabel=\textsf{capitis ($m$)}, 
ytick={0,15,30,45,60,75},
xtick={0,100,200,300}
]
\addplot[color=blue,mark=x, smooth] coordinates {
(0,44)
(50,43)
(100,42)
(150,40)
(200,33)
(220,29)
};

\end{axis}
\end{tikzpicture}
\caption{Pump headum and flowm}
\end{wrapfigure}


\providecommand\addcredit[1]{%
 \vspace*{-6.5pt}
 \scriptsize%
 \flushright%
 \textit{Credit: #1}%
}

\begin{figure}[htp]
\centering

\captionsetup{name=Photo., labelsep=period}%
   \begin{minipage}[t]{0.48\textwidth}
      \includegraphics[width=\textwidth]{./graphics/movingup.jpg}%
      \addcredit{U.S. DoD.}%
     \caption{The effects of the credit going past the edge of the figure. This can be corrected by adding a minipage to hold both commands. }
\end{minipage}\hfill\hfill
\begin{minipage}[t]{0.48\textwidth}
      \includegraphics[width=\textwidth]{./graphics/survivors.jpg}%
      \addcredit{U.S. DoD.}%
    {\footnotesize Marines awaiting resting before moving on to Japan. }
\end{minipage}

% \begin{minipage}[t]{0.48\textwidth}
%      \includegraphics[width=\textwidth]{./graphics/img009.jpg}%
%      \addcredit{U.S. DoD.}%
%     \caption{Engineer Construction Troops in Liberia, July 1942.}
%\end{minipage}\hfill\hfill
%\begin{minipage}[t]{0.48\textwidth}
%      \includegraphics[width=\textwidth]{./graphics/survivors.jpg}%
%      \addcredit{U.S. DoD.}%
%     \caption{The effects of the credit going past the edge of the figure. This can be corrected by adding a minipage to hold both commands. }
%\end{minipage}
% \begin{minipage}[t]{0.48\textwidth}
%      \includegraphics[width=\textwidth]{./graphics/img126.jpg}%
%      \addcredit{U.S. DoD.}%
%     \caption{Marine Reinforcements.
%A light machine gun squad of 3d Battalion, 1st Marines, arrives during the battle for ``Boulder City.'' }
%\end{minipage}\hfill\hfill
%\begin{minipage}[t]{0.48\textwidth}
%      \includegraphics[width=\textwidth]{./graphics/img124.jpg}%
%      \addcredit{U.S. DoD.}%
%     \caption{Brothers Under the Skin, inductees at Fort Sam Houston, Texas, 1953. }
%\end{minipage}
\end{figure}
\newpage


Armed with all these packages you can help the Gutenburg organization to transcribe
some of the old books that they have online. 

\clearpage








%  \chapter{Wrapped Illustrations}
\label{ch:wrapped}
\parindent2em
\let\onepar\lorem

Wrapped figures are not in vogue and most users of \latex avoid them.
If you are planning to have a more traditional book design wrapped figures might be more appropriate. Traditional typographers used
all sorts of styles to achieve wrapped figures which conserved paper. 
The best way to achieve it is to use Donald Arseneau's |wrafig| package \citep{wrapfig}.

\begin{wrapfigure}{l}{3.2cm}
    \includegraphics[width=3cm]{./images/amato.jpg}
    \caption{\footnotesize Wrapped figures}
\end{wrapfigure}

Get prepared to do a lot of manual adjustments, see your figures disappear on page refreshes and reruns. It is also recommended that you do your final adjustments once you are happy with the contents of your document and these final adjustments will not start jumping around. 
After a while though you get the hang of it and by minor adjustments you can really achieve great results. The manual uses \verb+everypar+ to insert commands for the shaping of the paragraphs that \emph{follow} the wrapped figure.

The package provides the environments \pkg{wrapfigure} and \pkg{wraptable} for typesetting a
narrow float at the edge of the text, and making the text wrap around it. The |wrapfigure|
and |wraptable| environments interact properly with the \verb+\caption+ command to produce
proper numbering, but they are not regular floats like \textit{figure} and \textit{table}, so be aware to do manual adjustments. If you do not take care 
they may also be printed out of sequence with the regular floats.

The |wrapfigure| environment  provides one of those monster locomotive type commands that stresses one's memory as it provides for four parameters.
 
The four param
for \verb+\begin{wrapfigure}+, two optional and two required, plus the text of the figure, with a caption perhaps.

\begin{macro}{wrapfigure}
\end{macro}

|\begin{wrapfigure}[12]{r}[34pt]{5cm}\meta{figure}\end{wrapfigure}|

  \begin{tikzpicture}[xshift=-15pt]
    \node (number) at (0mm, 0mm) {\oarg{number of narrow lines}};
    \node (placement) at (36mm, 0mm) {\marg{placement}};
    \node (overhang) at (60mm, 0mm) {\oarg{overhang}};
    \node (width) at (81mm, 0mm) {\marg{width}};
    \begin{scope}[->]
    \draw (number) -- (16mm, 17mm);
    \draw (placement) -- (24mm, 17mm);
    \draw (overhang) -- (35mm, 17mm);
    \draw (width) -- (47mm, 17mm);
    \end{scope}
  \end{tikzpicture}


First we will look at placing the figure without the use of optional commands.


\begin{verbatim}
\begin{wrapfigure}{r}{.4\textwidth}
    \includegraphics[width=.4\textwidth]{./path/file}
    \caption{\footnotesize Wrapped figures}
\end{wrapfigure}
\end{verbatim}

From the four parameters the first one indicates if the figure is to be typeset left or right.

\begin{verbatim}
\begin{wrapfigure}{l}{\imagewidth}
    \includegraphics[width=\imagewidth



]{./graphics/parasol-01}
    \caption{\footnotesize Wrapped figures}
\end{wrapfigure}
\end{verbatim}


\begin{wrapfigure}[18]{I}[0.1pt]{85pt}
    \captionsetup{name=Fig.}
    \vskip-10.5pt plus 2pt minus 2pt\relax
    \includegraphics[width=83pt]{./images/parasol-01.jpg}
    \caption{Wrapped figures, parameters set at \texttt\{l\}\{90pt\}.}
\end{wrapfigure}

Changing the parameters to suit we now have the illustration floating to the left. Allowing for the figure to be approximately two point  wider than the actual graphic, will leave a bit more margin. If the figure is end low in the page you need to be careful, that it does not disappear, as you will not get any warning.

The first parameter we are going to use an optional parameter is the one that determines the number of narrow lines. The format is \verb+[narrowlines]{l}{90pt}+. Think of this parameter as a fine tuning parameter and do not touch it until after your final draft is ready. If you see indented lines at the beginning of the page that follows the wrapped figure, reduce the number of lines, until you get satisfactory results.

The second optional parameter, comes after the \texttt{\{r\}[overhang]} parameter.

The second optional parameter (\#3) tells how much the figure should hang out into
the margin. The default overhang is given by the length \verb+\wrapoverhang+, which is 0pt
normally but can be changed using the command |\setlength|. For example, to have all wrapped figures you can 
use the space reserved for marginal notes,

\begin{verbatim}
\setlength{\wrapoverhang}{\marginparwidth}
\addtolength{\wrapoverhang}{\marginparsep}
\end{verbatim}

Again not recommended. The best approach is to specify the figures with \textbf{O} or \textbf{I}, let them float and if the results are
not very good then make manual adjustments. Get prepared to spend at least 5-10 minutes fiddling with the final result.

When you do specify the overhang explicitly for a particular figure, you can use a
special unit called \string\width meaning the width of the figure. For example, [0.5\string\width]
makes the center of the figure sit on the edge of the text, and [\string\width] puts the figure
entirely in the margin (and the adjacent text is indented by just \string\columnsep). This
\texttt{\string\width} is the actual width of the wrapfigure, which may be greater than the declared
width.

\begin{figure}[tb]
\includegraphics[width=\textwidth]{./graphics/chiefs.jpg}
\caption{Chiefs of Kelau or Kelaou.}
\label{fig:chiefs}
\end{figure}

\begin{figure}[p]
\centering

\includegraphics[width=0.8\textwidth,height=0.9\textheight, keepaspectratio]{./images/parasol-01.jpg}
\caption{Chiefs of Kelau or Kelaou.}
\label{fig:parasol-01}
\end{figure}

\section{Balancing the illustrations}

Illustrations come in various sizes, but in general they need to flow with the text. Place figures on top of the page and figures that would dominate the text on their own page. For example Figure~\ref{fig:chiefs} was allowed to float to the top of a page whereas Figure~\ref{fig:parasol-01} was placed on its own page, as I thought it will overwhelm the text if shown in a large size. However the same figure seems perfectly alright as a wrapped figure.

\begin{texexample}{}{}
\begin{wrapfigure}{I}{0pt}
    \includegraphics[width=75pt]{./images/parasol-01.jpg}
 \end{wrapfigure}
\lipsum[1-2]
\begin{wrapfigure}{l}{0pt}
    \includegraphics[width=75pt]{./images/parasol-01.jpg}
 \end{wrapfigure}
\lipsum[1-2]
\end{texexample}



\begin{texexample}{}{}
\begin{wrapfigure}{l}{0pt}
    \includegraphics[width=70pt]{./images/parasol-01.jpg}
    \includegraphics[width=70pt]{./images/parasol-01.jpg}
 \end{wrapfigure}

\lipsum[1-3]\lorem
\end{texexample}




\begin{texexample}{}{}
\begin{wrapfigure}[13]{L}{0pt}
    \includegraphics[width=100pt]{./graphics/conicalbasket.png}
\end{wrapfigure}

\onepar\onepar\onepar

\end{texexample}




%  \makeatletter
\newcommand\QEDit{\hspace{6pt}\textit{Q.~E.~D.}\quad}
\newcommand\QEFit{\hspace{6pt}\textit{Q.~E.~F.}\quad}
\newcommand\QEIit{\hspace{6pt}\textit{Q.~E.~I.}\quad}
\newcommand\QEDup{\hspace{6pt}Q.~E.~D.\quad}
\newcommand\QEFup{\hspace{6pt}Q.~E.~F.\quad}
\newcommand\QEIup{\hspace{6pt}Q.~E.~I.\quad}
\newcommand\QEOup{\hspace{6pt}Q.~E.~O.\quad}
\newcounter{wrapwidth}
\newcount \Zw
\newcount \Zh


\newcommand\pngright[4]{%
    \Zw=#2 \divide \Zw by 10
    \Zh=#3 \divide \Zh by 120  \advance\Zh by 1
    \setcounter{wrapwidth}{\Zw}
\begin{wrapfigure}[\Zh]{r}{\value{wrapwidth}pt}%
\begin{center}
\vspace{#4pt}%
\includegraphics*[width=\Zw pt]{images/#1}%
\end{center}
\end{wrapfigure}}

\newcommand\propnopage[1]{
\begin{center}{\large #1}\end{center}}

\parindent1em

\cxset{toc image=\@empty}
\chapter{PERICULA}

\noindent\textsc{Case Study: } We will now typeset a section, from Isaac Newton's \textit{Philosophi\ae\  Naturalis Principia Mathematica}. The typeset example is shown below.

\bottomline
\bgroup
\small

\cxset{section align=center,
         section numbering=none}

\section{{SECT}$\cdot$ VIII$\cdot$}

\begin{center}{\textit{De Motu per Fluida propagato.}}\end{center}

\makeatletter
%\propnopage{Prop.\ XLI\@. Theor.\ XXXI.}
\meaning\@
\makeatother

\textit{Pressio non propagatur per Fluidum secundum lineas rectas, nisi
ubi particul{\ae} Fluidi in directum jacent.}

Si jaceant particul{\ae} $a$, $b$, $c$, $d$, $e$ in linea recta, potest quidem
pressio directe

\begin{wrapfigure}[8]{O}[1pt]{0.3\textwidth}
  \vspace{-17pt}
  \includegraphics[width=0.27\textwidth]{images/362.png}
\end{wrapfigure}

\noindent propagari  ab $a$ ad $e$; at
particula $e$ urgebit particulas oblique positas
$f$ \& $g$ oblique, \& particul{\ae} ill{\ae} $f$ \& $g$
non sustinebunt pressionem illatam, nisi fulciantur
a particulis ulterioribus $h$ \& $k$;
quatenus autem fulciuntur, premunt particulas
fulcientes; \& h{\ae} non sustinebunt pressionem nisi fulciantur
ab ulterioribus $l$ \& $m$ easque premant, \& sic deinceps in infinitum.
Pressio igitur, quam primum propagatur ad particulas
qu{\ae} non in directum jacent, divaricare incipiet \& oblique propagabitur
in infinitum; \& postquam incipit oblique propagari, si
inciderit in particulas ulteriores, qu{\ae} non in directum jacent, iterum
divaricabit; idque toties, quoties in particulas non accurate
in directum jacentes inciderit. \QEDit

\topline

\vspace*{-\baselineskip}
\captionof{figure}{Example of a typeset page from Principi\ae.}
\egroup
\smallskip
Figure~\ref{fig:principia}, shows a scan of the actual page. We will not reproduce, the fonts and the page geometry exactly, but rather we will attempt to extract and reproduce the typographical rules employed in the printing of the \textit{Principi\ae}.

We begin by typesetting the section number and its heading. The use of roman numbers creates better harmony between the text and the heading

\begin{teX}
\sectpage{VIII$\middot$}
\begin{center}{\textit{De Motu per Fluida propagato.}}\end{center}
\end{teX}
The proposition and theorem line, has its own command
\begin{teX}
\makeatletter
\propnopage{Prop.\ XLI. Theor.\ XXXI.}
\makeatother
\end{teX}

\propnopage{\color{gray}Prop.\ XLI\@. Theor.\ XXXI.}
\vspace*{-37pt}
\propnopage{Prop. XLI. Theor. XXXI.}


Notice the small differences in the spacing with the commands as shown and with the black text, without them. The rest is based on normal \LaTeX\ commands.

\textit{Pressio non propagatur \ldots particul{\ae}\ldots}

\pngright{362.png}{709}{603}{-24}

Si jaceant particul{\ae} $a$, $b$, $c$, $d$,
$e$ in linea recta, potest quidem
pressio directe propagari ab $a$ ad $e$; at


Remember that it is important to start a new paragraph after the 
|pngright| command. The |wrapfig| package works by using |everypar| to insert the hanging indentation.

\begin{figure}[p]
\centering
\includegraphics[scale=1]{./images/page354.png}
\caption{Page 354 from Isaac Newton's \textit{Philosophi\ae\  Naturalis Principia Mathematica}. Image was obtained from Google's copy, available at Google Books.}
\label{fig:principia}
\end{figure}

%  \chapter{Subfigures}

So far we have been using the |caption| package to add captions to multiple figures, that are numbered individually, but how about if you want to have only one caption and number the subfigures alphabetically. If you want to have |subfigures| with distinct caption, you use the |subfig| package \citep{subfigure}. A newer package \ctan{subcaption} is also now available with the |caption| suite and we will discuss this also. The two packages are incompatible and the recommendation is to use the |subcaption| package. In the |phd| class we load the |caption| package which also loads the |subcaption| package. The latter is to be preferred as it integrates better both with captions as well as the |hyperref| package.

\begin{figure}[h]
\centering
\begin{minipage}[b]{.3\linewidth}
\includegraphics[width=4cm]{./graphics/pic37.png}\hspace{1em}
\subcaption{First fighting elephant}\label{fig:1a}
\end{minipage}\hspace{1em}
\begin{minipage}[b]{.3\linewidth}
\includegraphics[width=4cm]{./graphics/pic37.png}\hspace{1em}
\subcaption{Second fighting  elephant}\label{fig:1b}
\end{minipage}\hspace{1em}
\begin{minipage}[b]{.3\linewidth}
\includegraphics[width=4cm]{./graphics/pic37.png}\hspace{1em}
\subcaption{Third fighting elephant}\label{fig:1c}
\end{minipage}
\caption{Three fighting elephants example}
\end{figure}

You can put as many figures as you like on a page, but a word of warning, you may need to make some manual adjustments before you get it right. The package provides support for the manipulation and reference of small or \enquote{sub} floats within a single floating (e.g., figure or table) environment It is convenient to use this
package when your sub-floats are to be separately captioned, referenced, or when such
sub-captions are to be included on a List of Floats page.

The package is a replacement for the |subfigure| package, from which it was derived.
However, the new |subfig| package is not completely backward compatible.
Therefore, a new name was called for. The newer package is smaller and easier to use
than the older package, however, it now uses the following additional packages,  |caption| (required),  |everysel| (optional),
keyval (required),  |ragged2e| (optional). All these packages are included with the |phd| auto package manager.

It will work without the |ragged2e| and |everysel| packages if you do not use the following
justification options: \enquote{Center}, \enquote{RaggedRight} and \enquote{RaggedLeft}. The other justification
options \enquote{center}, \enquote{raggedright} and \enquote{raggedleft} will work without the above two packages. If the ragged2e package is present, than the caption package will load it and it
will, in turn, load the everysel package. This happens whether or not you will be using
the justification options that require it. If it cannot find the ragged2e package, than the
caption package will print a message that \enquote{RaggedRight}, etc. will not be available.

\section{Subcaption environments}

The |subcaption| package offers an environment for subfigures, which are essentially minipages. Within the environment, the normal caption command can be used rather than the \cmd{\subcaption}.

\begin{figure}%
    \centering
    \captionsetup[figure]{margin=3pt}%
    \begin{subfigure}[b]{.35\linewidth}
    \includegraphics[scale=0.65]{./graphics/fig155.jpg} 
    \label{fig:one}
    \caption{First Caption}
    \end{subfigure}\hspace{2em}
    \begin{subfigure}[b]{.35\linewidth}
    \includegraphics[scale=0.65]{./graphics/fig156.jpg} 
    \caption{First Caption}
    \end{subfigure}
    \caption{Two subfigures side by side.}
    \label{fig:two}
\end{figure}

The sub-figures can be referenced the same way as normal referencing.

\begin{quote}
 A low bottle-shaped vase, of yellowish ware, with flaring rim and somewhat flattened body. Height, 5 inches; width 5 inches. \ref{fig:one}


A well-made bottle shaped vase, with low neck and globular body, somewhat conical above. Color dark brownish. $7\frac{1}{2}$ inches in height. Shown in Figure~\ref{fig:two}.
\end{quote}

\begin{figure}[htp]
  \centering
  \includegraphics[width=0.5\linewidth]{./graphics/fig175.jpg}
  \vspace{3\baselineskip}

   \centerline{\textsc{From the tomb of a Pull\= arius.}}
  \label{fig:marginfig1}
  \caption{ effigy vase of the dark ware. The body is globular. A kneeling human figure forms the neck. The mouth of the vessel occurs at the back of the head—a rule in this class of vessels. Is is finely made and symmetrical. 9.75 inches high and 7 inches in diameter. being larger than the above two it is preferable to scale it to give the reader an indication.}
\end{figure}

The above figure is an Based on the figure width, you may also need to adjust the distance between the figures to ensure that the whitespace is just about right. For screen reading this can be increased and for printed works you may wish to make it less.

\begin{teXXX}
\begin{figure}[htb]
\begin{subfigure}[b]{.5\linewidth}
\centering\large A
\captionsetup{skip=3pt}
\caption{A subfigure}\label{fig:1a}
\end{subfigure}
\end{figure}
\end{teXXX}

\begin{comment}
\begin{figure}[htp]%
    \captionsetup[figure]{margin=3pt}%
    \subfloat[One subone.\label{fig:one}]
     {{\includegraphics[scale=0.65]{./graphics/fig155.jpg}}}
    \hspace{1cm}
    \subfloat[One subtwo.\label{fig:two} --- but this one has a
     very very long caption.  So long that it continues over into
     other lines so that we can test the list-of line settings.]%
      {\includegraphics[scale=0.65]{./graphics/fig156.jpg}}
     \\[-10pt]
    \caption{First figure --- but this one has a very very long caption.
     So long that it continues over into a second line so that we can
     test the margin setting and centering of the caption command in the
     full page mode.}%
    \label{fig:Afirst}%
    \caption{Typical pottery from Oklahoma (\emph{Smithsonian}).}%
    \label{fig:Athird}%
\end{figure}
\end{comment}

The figures have been placed using the code below:

\begin{verbatim}
\begin{figure}%
    \captionsetup[figure]{margin=3pt}%
    \subfloat[One subone.\label{fig:one}]
     {{\includegraphics[scale=0.65]{./graphics/fig155.jpg}}}
    \hspace{1cm}
    \subfloat[One subtwo.\label{fig:two} --- but this one has a
     very very long caption.  So long that it continues over into
     other lines so that we can test the list-of line settings.]%
      {\includegraphics[scale=0.65]{./graphics/fig156.jpg}}
     \\[-10pt]
 \caption{First figure but this one has a very very long caption.
 So long that it continues over into a second line so that we can
 test the margin setting and centering of the caption command in the
 full page mode.}%
 \label{fig:Afirst}%
 \caption{Typical pottery from Oklahoma (\emph{Smithsonian}).}%
 \label{fig:Athird}%
\end{figure}
\end{verbatim}

As you can observe, the |subcaption| package treats the two figures as one and places them side by side. Its trickery is to get them to line up, nicely and to provide all the necessary parameters for the captions. It is a feature-rich package and we will spent some time to explore it. The command |subfloat|, is used to denote the |subfigure|. The rest are self-explanatory. Note that the use of |\hspace{1cm}| to make these two figures come closer together. In the previous listings, |\hfill| was used to space them out as wide as possible. The command |captionsetup| is used to let the package know that we are captioning figures and not tables. (In this book all captions are placed in the side-margins, where God meant them to be! If you use the same code in another package they will be placed underneath the figures.




%  \begin{comment}
\documentclass[imperial, justified]{octavo}
\usepackage{caption}
\usepackage{natbib}
\usepackage{lstdoc}
\usepackage{lipsum}
\usepackage{graphicx}
\usepackage{overpic}
\usepackage{url}
\global\setlength\parindent{1em}
\newif\ifdebug
\debugfalse
\ifdebug  
  \setlength\fboxsep{1pt}
\else
  \setlength\fboxsep{0pt}
  \setlength\fboxrule{0pt}
\fi

%% temporary titles
% command to provide stretchy vertical space in proportion
\newcommand\nbvspace[1][1]{\vspace*{\stretch{#1}}}

% allow some slack to avoid under/overfull boxes
\newcommand\nbstretchyspace{\spaceskip0.5em plus 0.25em minus 0.25em}

% To improve spacing on titlepages
\newcommand{\nbtitlestretch}{\spaceskip0.6em}

% temporary length used for some tables
\newlength{\TmpLen}

\begin{document}
\clearpage
\pagestyle{empty}
\begin{center}
\bfseries

\nbvspace[1]
\Huge
{\nbtitlestretch\huge
 TYPESETTING  
WITH  \TeX\ AND SX.TX FRIENDS  \\
}

\nbvspace[2]
\normalsize
TO WHICH IS ADDED MANY USEFUL MACROS
AND CODE WRITTEN SO THAT HE WHO RUNS MAY HACK

\nbvspace[1]
\small BY\\
\nbvspace[1]
\Large THE STACKEXCHANGE COMMUNITY {\large\textsc{}}\\[0.5em]
%\footnotesize AUTHOR OF ``A WORKING ALGEBRA,'' ``WIRELESS TELEGRAPHY,\\
%ITS HISTORY, THEORY AND PRACTICE,'' ETC., ETC.

\nbvspace[4]

%\includegraphics[width=0.8in]{ejc.pdf}
\includegraphics[width=1.5in]{./images/fig176}
\par
\nbvspace[2]
\normalsize
%DOHA$\cdot$BERLIN$ \cdot$ WILD

\nbvspace[10]
\Large
PUBLISHED IN THE WILD
%
\end{center}


\long\def\secondpage{\clearpage\null\vfill\vfill
\pagestyle{empty}
\begin{minipage}[b]{0.9\textwidth}
\footnotesize\raggedright
\setlength{\parskip}{0.5\baselineskip}
Copyright \copyright 2010--\the\year\ Dr Yiannis Lazarides\par
Permission is granted to copy, distribute and\slash or modify this document under the terms of the GNU Free Documentation License, version 1.2, with no invariant sections, no front-cover texts, and no back-cover texts.\par
A copy of the license is included in the appendix.\par
This document is distributed in the hope that it will be useful, but without any warranty; without even the implied warranty of merchantability or fitness for a particular purpose.
\end{minipage}
\vspace*{2\baselineskip}}

\secondpage

\backmatter
\tableofcontents
\listoffigures

\chapter{PREFACE}
This small booklet aims to describe some of the common problems encountered with the 
placement of figures in books. It also tries to provide techniques for storing them within TeX.

\mainmatter
\end{comment}

\chapter{How to Typeset a lot of Figures}
\precis{In this chapter we develop a primitive database for storing graphics and then typesetting them.}
\addtocimage{-12pt}{-20pt}{../images/tocblock-men.jpg}


If you have a lot of figures, it is a lot of work to have to maintain them, as well as
to remember all the file names. The figures are from an old Catalogue of the Smithsonian Institution \citep{holmes1884}. 

\section{A long table for figures}

We are familiar with longtable for tables, this is an equivalent technique for lots of figures.
\smallskip

\def\figurename{\textbf{Plate}}

\DeclareRobustCommand\putgraphic[1]{%
\fboxsep0pt\fboxrule0pt
\fbox{%
\begin{minipage}[b]{2.0cm}%
 \centering
 \vspace{3.8pt}\fbox{%
 \includegraphics[width=0.98\linewidth,
                 height=2.3cm,
                 keepaspectratio]{./images-01/#1.jpg}}%
  \vspace{0.2cm} #1%
  \vspace{0.2cm}%
  \end{minipage}}\hfil
}

\long\def\putcaption#1{\captionof{figure}{#1}}

\makeatletter
{\centering

\gdef\alist{fig145,fig161,fig162,fig163,fig164,fig165,fig166,fig167,^^A,
fig168,fig169,fig170,fig171,fig172,fig173,^^A
fig174,fig175,fig176,fig177,fig180,fig181,fig182,fig183,fig185,fig186,fig187,fig188,fig189}
\@for \i:=\alist\do{^^A
\expandafter\putgraphic{\i}
}
\putcaption{Weaving and pottery artifacts from Arizona.}}

\medskip

The code leverages \tex's ability to create macro names with any character using the |\csname...\endcsname| construct. We first put the
images in a list. The images have been saved as |fig145| etc on the disk and hence what we simply do is just enclose them in a comma delimited list. They do not need to be numbered consequentially in the list.

\begin{verbatim}
\gdef\alist{fig145,fig161,fig162,..,fig187,fig188,fig189}
\end{verbatim}

We then loop over the |\alist| and get the output as shown in Example~\ref{ex:blist}. 

\begin{texexample}{Looping over the list}{ex:blist}
\def\blist{fig189,fig145,fig161,fig162}
\@for \i:=\blist\do{%
  \expandafter\putgraphic{\i}%
}
\end{texexample}


\section{More on figures and looping}

We can extend our macros and try and save some information for each image. To do this we
need to have a way to associate information with the figure number so we will create a number of commands
for each figure.

The \TeX\ way of defining commands on the fly that include non-letters is to use \verb+\csname+
\begin{verbatim}
\expandafter\def\csname fig170\endcsname#1{#1}
\@nameuse{fig170}{Pottery found in Apache%
    lands in Texas.}
\end{verbatim}

\@nameuse{fig170}{Pottery found in Apache %
    lands in Texas.}

This is not very useful, as it is. It is preferable to actually create a little command factory, that can create these
commands.

\begin{texexample}{}{factory command}
\bgroup
\gdef\commandfactory#1#2{
   \expandafter\def\csname #1\endcsname{#1}
   \expandafter\def\csname #1@caption\endcsname{#2}
}
\commandfactory{fig170}{Test}
\centering

\putgraphic{\csname fig170\endcsname}
\putcaption{\@nameuse{fig170@caption}}

\egroup
\end{texexample}

Since we are going to have to type a lot of information into a database to hold information for our images, we might as
well type it straight into our text.

Out of consideration for our users we may want to provide a short command for this.

\begin{verbatim}
\let\img\commandfactory
\end{verbatim}

\let\img\commandfactory

\img{fig171}{Testing again for something.}

We may also want to save the use of the curly brackets, that would visually distruct. We can redefine the Command factory to be a delimited macro. There is a lot of information on delimited macros. One of them is in such a place, hiding on \texttt{tex.sx}.

\def\commandfactory#1|#2|{
   \expandafter\def\csname #1\endcsname{#1}
   \expandafter\def\csname #1@caption\endcsname{#2}
}

\commandfactory fig172|This is figure 172|

\commandfactory fig173|This is figure 173|

\texttt{\@nameuse{fig172@caption}}

\texttt{\@nameuse{fig173@caption}}

Now that we have figured a way to define an efficient way to store information for our figures, we need to build some routines to sort them print them and other similar housekeeping routines.

\section{Sorting}

\global\setlength\parindent{1em}
I have still to find a better sorting routine other than the one available in the listings documentation. I did try my hands with LuaTeX but I am not very fond of jumping in and out of LaTeX. It can also create problems with updates and users that might not have LuaTeX installed.

We will store the record index in a macro that is essentially a comma delimited list. Don't be frighten about speed
I have used this method to store over 4000 figures and there was no problem either with the processing speed or with TeX'es memory.

We call this macro \verb+dbartifacts+, giving it a non-generic name. But first let us see, how we can add items in
and out of the macro. We start from an empty macro.

\begin{verbatim}
\def\dbartifacts{ }
\end{verbatim}
\let\dbartifacts\empty

We can use \LaTeX's \verb+\g@addto@macro+ to then add the items to the \verb+\dbartifacts+ macro.
\begin{verbatim}
\g@addto@macro{\dbartifacts}{fig172,}%
\g@addto@macro{\dbartifacts}{fig173,}%
\end{verbatim}

\g@addto@macro{\dbartifacts}{fig172,}%
\g@addto@macro{\dbartifacts}{fig173,}%


Testing it by just typing \verb+\texttt{\dbartifacts}+ we get: \texttt{\dbartifacts}. This of course is not very convenient and we would rather define a macro to save all the typing and have a more user friendly command.

\begin{verbatim}
\def\addtodb#1#2{%
  \g@addto@macro#1{#2,}%
}
\end{verbatim}
\def\addtodb#1#2{%
  \g@addto@macro#1{#2,}%
  \lst@BubbleSort\dbartifacts%
}

\clearpage

There are many other ways to manipulate the list, including using token registers, elt lists etc, but for such constructions as the ones described here, this is by far the simpler and the easiest.
We can now use this macro, when required:

\begin{verbatim}
\addtodb{\dbartifacts}{fig170}%
\addtodb{\dbartifacts}{fig171}%
\end{verbatim}

Testing again we get \texttt{\dbartifacts} an as you can see it works nicely. This method of trying out your code bit by bit, I call the water painting technique. So now that we have almost got all the routines we want, we can now look at sorting. This we achieve by adding \verb+ \lst@BubbleSort\dbartifacts+. Every time we add a record, the file will be sorted. Intuituitevely, this might not  be very efficient, especially if you are adding a lot of records at one time, but we can add more helper routines later for this.

\begin{verbatim}
\def\addtodb#1#2{%
  \g@addto@macro#1{#2,}%
  \lst@BubbleSort\dbartifacts%
}
\end{verbatim}

\def\figurename{\textbf{Figure}}
\begin{figure}
\vspace*{1cm}
\centering
\includegraphics[scale=0.6]{./images/fig172.jpg}
\caption{Textiles from Arizona. }
\end{figure}

\section{Adding some more user helper macros}

It is expected that the user will produce a file, either through some automatic means or by typing it to hold the data. Deletion and insertion is simply via editing this file through a text editor. However, for completeness, we will write a few macros to help with maintenace of the database. These include macros for delete and modify record etc.

Another set of macros that one can use is to typet the records in lists and or tabulat forms, if required. Early books on archaelogy for example listed all the items in the following format, interspersed with comments and figures.
\smallskip


\hangindent3em
2520. (39510). A double globe jar or canteen. White ground, with ornamentations in black, as seen in Fig. 649. Depression in the center is probably designed to receive a band or cord to carry it with.
\smallskip

Although one is tempted to produce a list for these, the next item from such a book points otherwise:
\smallskip

\hangindent3em
2677-2678. 2677, (39617), and 2678, (39618). With flared and notched rim.
\smallskip

Before extending the database for such forms of descriptions, we can develop the typesetting part. I am sure that Lamport would have used a list, possibly due memory and space limitations and just re-use the \verb+\item+ command, in our case it is better to rather define a small macro
to cater for such items. The indentation can easily be achieved using \verb+\hangindent3em+ or a similar amount of measure.

\begin{verbatim}
\long\def\catno#1\par{
\par%
\hangindent3em\noindent
#1
}
\end{verbatim}


\def\catno#1#2{%
    \@hangfrom{#1. }#2
}


\DescribeMacro{\@hangfrom}\marg{text}   
\LaTeX\ provides a macro named \verb+\@hangfrom{<text>}+, that puts \marg{text} in a box, and makes a hanging indentation of the following material up to the first \verb+\par+. This Should be used in vertical mode.\footnote{See source2e, \texttt{ltsect.dtx}, pg 287.}

\begin{verbatim}
121 \def\@hangfrom#1{\setbox\@tempboxa\hbox{{#1}}%
122 \hangindent \wd\@tempboxa\noindent\box\@tempboxa}
\end{verbatim}

\medskip

\catno{289}{(39914). Fig. 397. Red ware, with white lines on the lower globe and decorations in black on the upper, with orifice in each globe.}

\catno{1289}{(39914). Fig. 397. Red ware, with white lines on the lower globe and decorations in black on the upper, with orifice in each globe.}


\makeatother

\section{Epiloque}

We have managed to write a database, sort it, typeset its contents in a structured or freeform manner
and on the way we have documented the code using a form of \textit{literate programing.} On top which
other language expects you to code your own ifs and for? 
The amount of code we wrote was very minimal and competes well with modern computer languages. 

Hope you had fun. Go and make beautiful books. 

\begin{figure}[htp]
\centering
{\color{thegray}
\fbox{\includegraphics[width=1\linewidth]{./images//pottery-figures.pdf}}}
\caption{Many books in the humanities have figure pages, with many different styles and numbering schemes. This page extract is from \textit{The Cypro-Phoenician pottery of the Iron Age. }  \citep{schreiber1971}}
\end{figure}


\begin{figure}[htp]
\centering
{\color{thegray}
\fbox{\includegraphics[width=1\linewidth]{./images/sample-tof.pdf}}}
\caption{Many books in the humanities have figure pages, with many different styles and numbering schemes. This page extract is from \textit{The Cypro-Phoenician pottery of the Iron Age. }  \citep{schreiber1971} and shows a specific way of numbering subfigures, including references.}
\end{figure}
\clearpage



 \subsection{Acknowledgements}

 Octavo is a modification of \texttt{classes.dtx} written by Leslie Lamport (1992),
 Frank Mittelbach (1994-97) and Johannes Braams (1994-97). As can be seen
 from the code, my own input is restricted to a tweaking of some parameters
 and true credit is due to Lamport, Mittelbach and Braams for their
 monumental efforts.



\begin{comment}
 \begin{thebibliography}{16}

 \bibitem{knuth98} Knuth,~D. 1998. \emph{Digital Typography}. CSLI 
 Publications, Stanford.

 \bibitem{rosarivo61} Rosarivo,~R. 1961. \emph{Divina proportio typographica}. 
 Scherpe, Krefeld.

 \bibitem{taylor98} Taylor,~P. 1998. \emph{Book design for \TeX\ users, Part 1: 
 Theory.} TUGBoat, 19:65--74.

 \bibitem{taylor99} Taylor,~P. 1999. \emph{Book design for \TeX\ users, Part 2:
 Practice.} TUGBoat, 20:378--389.

 \bibitem{town} Town,~L. \emph{Bookbinding by hand.} Faber \& Faber, London.

 \bibitem{tschichold87} Tschichold,~ J. 1987. \emph{Ausgew\"{a}hlte Aufs\"{a}tze
 \"{u}ber Fragen der Gestalt des Buches und der Typographie}. Birkh\"{a}user
 Verlag, Basel.

 \bibitem{williamson66} Williamson,~H. 1966. \emph{Methods of book design.} Oxford 
 University Press, Oxford.

 \bibitem{wilson01} Wilson,~P. 2001. \emph{The Memoir class for configurable
 typesetting.} CTAN. \url{macros\\latex\\contrib\\memoir} 

 \end{thebibliography}
\end{comment}





\chapter[Overflowing Figures into Margins]{OVERFLOWING FIGURES INTO MARGINS}

Most users of \TeX\ are accustomed to let the system position images, either on top or bottom of the page and occasionally use the [h] positioning directive to place the image at the exact location it appears in the text. Traditional typography placed the image in many different positions. It also occasionally overflowed the image into the margins. The image below, copied from the \textit{American Antiquarian}, was placed in the original publication as such. Tufte advocates the use of such techniques in displaying not only information, but also other material such as tables. The Tufte class is discussed extensively in other sections. It has almost a religious following attached to it and I have personally used it for business reports.\citep{seraphini}

\begin{figure}[htbp]
\leftskip-.07\textwidth\includegraphics[width=1.14\textwidth]{./images/elephant-long.jpg}\par

\begin{multicols}{4}
\myanmar

လူတိုင်းသည် တူညီ လွတ်လပ်သော ဂုဏ်သိက္ခါဖြင့် လည်းကောင်း၊ တူညီလွတ်လပ်သော အခွင့်အရေးများဖြင့် လည်းကောင်း၊ မွေးဖွားလာသူများ ဖြစ်သည်။ ထိုသူတို့၌ ပိုင်းခြား ဝေဖန်တတ်သော ဉာဏ်နှင့် ကျင့်ဝတ် သိတတ်သော စိတ်တို့ရှိကြ၍ ထိုသူတို့သည် အချင်းချင်း မေတ္တာထား၍ ဆက်ဆံကျင့်သုံးသင့်၏။
\end{multicols}
\centerline{\protect\textsc{Codex Seraphinianus, Mystery Procession \protect\citep{seraphini}}.}
\end{figure}

Almost as a matter of rule, the caption for these images was in small caps. Using small caps brought the caption into the easy attention of the reader, but it did not distract from the other elements of the page.
The image is not necessarily positioned symmetrically in the page, you can offset it to suit your taste, but in general, unless the image has any particular features that would make it look better offset rather than centered, is best positioned symmetrically. This can be automated, by writing a macro that measures the dimensions of the image and introduces a \verb+\leftskip+ so that the image can be shifted accordingly. A macro to achieve this is now described.


The image can be included by simply using a \verb+\skip-1.2cm+ or \verb+\leftskip-1.2cm+ :

\begin{verbatim}
\begin{figure}[htbp]
\leftskip-1.2cm\includegraphics{image}\par
\centerline{\textsc{Copper Sheath}}
\end{figure}
\end{verbatim}

\long\def\imghangleft#1#2{%
     \figure
     \leftskip-#2\textwidth\includegraphics[width=#1\textwidth]{./images/elephant-long.jpg}\par
     \centerline{\textsc{Codex Serafinianus}}
    \endfigure
}

\imghangleft{1.14}{.07}






%  \section{SIDEWAYS PICTURES}
Figures can be rotated as shown in Figure~\ref{fig:sideways}  a landscape mode using the \texttt{rotating} package. A package for rotated objects in LATEX
Robin Fairbairns, Sebastian Rahtz, Leonor Barroca.


The code uses the \verb!sideways! environment. In this particular example, we use footnotes, in the caption and hence we add some code to achieve this.  Note that the package defaults take care of verso and recto page display so that you do not need to   worry about rotating the image clockwise or counterclockwise. The package rotates by default clockwise.

\begin{tcolorbox}
\begin{lstlisting}
\begin{sidewaysfigure}

\includegraphics[height=0.5\textheight, width=0.9\textwidth, keepaspectratio]{nudewithapple}
\captionof{figure}{The package sets the\protect\footnotemark[1] footnotes\protect\footnotemark[2] of a single-column document in two columns;
the package offers a range of parameters to determine\protect\footnotemark[3] the exact appearance\protect\footnotemark[4] of the two columns.}
\vspace{3\baselineskip}
\footnoterule\footnotesize
\begin{minipage}[t]{0.4\linewidth}
\textsuperscript{1} This is the first footnote. And here comes some nonsense text
                    to show that the linebreaks works \par
\textsuperscript{2} This is the second footnote.\par
\end{minipage}\hfill
\begin{minipage}[t]{0.4\linewidth}
\textsuperscript{3} This is the third footnote. \par
\textsuperscript{4} This is the fourth footnote.\par
\textsuperscript{5} This is the fourth footnote.\par
\textsuperscript{6} See \url{http://tex.stackexchange.com/questions/8174/how-to-achieve-a-multi-column-layout-for-footnotes}\par
\end{minipage}
\end{sidewaysfigure}
\end{lstlisting}
\end{tcolorbox}


\begin{sidewaysfigure}

\centering
\includegraphics[height=0.5\textheight, width=0.9\textwidth, keepaspectratio]{bathers-01}
\captionof{figure}{The package sets the\protect\footnotemark[1] footnotes\protect\footnotemark[2] of a single-column document in two columns;
the package offers a range of parameters to determine\protect\footnotemark[3] the exact appearance\protect\footnotemark[4] of the two columns.}
\vspace{3\baselineskip}
\footnoterule\footnotesize
\begin{minipage}[t]{0.49\linewidth}
\textsuperscript{1} This is the first footnote. And here comes some nonsense text
                    to show that the linebreaks works \par
\textsuperscript{2} This is the second footnote.\par
\end{minipage}\hfill
\begin{minipage}[t]{0.49\linewidth}
\textsuperscript{3} This is the third footnote. \par
\textsuperscript{4} This is the fourth footnote.\par
\textsuperscript{5} This is the fourth footnote.\par
\textsuperscript{6} See \url{http://tex.stackexchange.com/questions/8174/how-to-achieve-a-multi-column-layout-for-footnotes}\par
\end{minipage}
\end{sidewaysfigure}


\begin{sidewaysfigure}

\centering
\includegraphics[height=0.5\textheight, width=0.9\textwidth, keepaspectratio]{nudewithapple}
\captionof{figure}{The package sets the\protect\footnotemark[1] footnotes\protect\footnotemark[2] of a single-column document in two columns;
the package offers a range of parameters to determine\protect\footnotemark[3] the exact appearance\protect\footnotemark[4] of the two columns.}
\vspace{3\baselineskip}
\footnoterule\footnotesize
\begin{minipage}[t]{0.49\linewidth}
\textsuperscript{1} This is the first footnote. And here comes some nonsense text
                    to show that the linebreaks works \par
\textsuperscript{2} This is the second footnote.\par
\end{minipage}\hfill
\begin{minipage}[t]{0.49\linewidth}
\textsuperscript{3} This is the third footnote. \par
\textsuperscript{4} This is the fourth footnote.\par
\textsuperscript{5} This is the fourth footnote.\par
\textsuperscript{6} See \url{http://tex.stackexchange.com/questions/8174/how-to-achieve-a-multi-column-layout-for-footnotes}\par
\end{minipage}
\end{sidewaysfigure}




\clearpage

\begin{figure}

\centering
\includegraphics[height=\textheight, width=\textwidth, keepaspectratio]{julesbache}
\end{figure}

\begin{figure}

\centering
\includegraphics[height=\textheight, width=\textwidth, keepaspectratio]{goya01}
\end{figure}

\begin{figure}

\centering
\includegraphics[height=\textheight, width=\textwidth, keepaspectratio]{goya-sideways}
\end{figure}

\begin{figure}

\centering
\includegraphics[height=\textheight, width=\textwidth, keepaspectratio]{goya-sideways01}
\end{figure}

\pagebreak







%  
\parindent0pt

\begin{minipage}{1.05\textwidth}
\vspace{\baselineskip}
\parindent0pt
\fboxrule0pt
{
\centering
\fbox{\centering
\begin{minipage}[t]{0.89\textwidth}
\centering
\begin{minipage}[t]{0.41\textwidth}
\includegraphics[width=1\textwidth]{./images/threewomen01.png}\vspace*{-8pt}%
\captionof*{figure}{\noindent\footnotesize\textbf{WALDO PEIRCE}, a famous painting in his own right,
turned model for Bellows, posed for this impressive portrait in New York studio in 1920.}
\end{minipage}\hspace{0.5cm}
\begin{minipage}[t]{0.4\textwidth}
   \includegraphics[width=1\textwidth]{./images/threewomen02.png}\vspace*{-8pt}
    \captionof*{figure}{\noindent\footnotesize\textbf{MRS KATHERINE ROSEN,}
                 the daughter of Charles Rosen, he was an artist and neighbor of bellows, 
                 posed for this  meditative study in 1921.}
\end{minipage}
\end{minipage}
}}

\medskip

\fbox{\hskip-0.3cm\includegraphics[width=1.03\textwidth]{./images/twowomen-03.png}}\\[-27.5pt]
\setlength{\linewidth}{.95\textwidth}
\setlength{\columnsep}{8pt}
\begin{multicols}{2}
\noindent \footnotesize\textbf{TWO WOMEN,} portrays a professional model dressed and undressed. The range and richness of colors is unusual among Bellows' pictures. Bellows always had a horror of studio pictures and ``pretty nudes,'' rarely worked from professional models and never painted a still life.
\end{multicols}
\vfill

\captionof{figure}{Balancing three images on a page. Should the larger image be at the top or at the bottom?}
\end{minipage}

\newcommand\articleheading[1]{%
    \par
    \vspace*{2\baselineskip}
    \bgroup
    \LARGE\bf\textsf{\noindent #1}
    \egroup
   \vskip2\baselineskip
}
\clearpage

\begin{minipage}{\textwidth}
\includegraphics[width=\textwidth]{./images/yaleartschool.png}

\articleheading{TRADITION AND TECHNIQUE AT YALE'S SCHOOL OF  FINE ARTS}

\end{minipage}
\begin{multicols}{3}
        \lettrine{A}{t Yale}\lorem \lipsum[1-3]
        \par
\end{multicols}

\newgeometry{top=0pt, left=0pt, right=0pt, top=0pt, bottom=2cm}
\pagebreak

\begin{minipage}{\textwidth}
\includegraphics[width=\textwidth]{./images/sculpture-lesson.jpg}\par
\vspace{\baselineskip}

\centerline{\HUGE\bfseries SCULPTURE LESSON}
\vspace{0.5\baselineskip}

\centerline{\LARGE\bfseries Noted arist shows how adventurous amateurs can model with clay }

\end{minipage}

{
\leftskip1cm\rightskip1cm\columnsep-1.3cm\par\leavevmode

\begin{multicols}{3}
        \lettrine{A}{t Yale} \lorem \lorem \lorem \lorem
        
\end{multicols}
}

\newgeometry{top=1.5cm,left=2cm,right=2cm,bottom=2cm}

\pagebreak





\lipsum[1]
\includegraphics[height=0.8\textheight, width=\textwidth\relax]{./images/nino.png}

This is a short caption test and this one is a long caption test.

\includegraphics[width=\textheight, width=\textwidth]{./images/woman.png}
Donna Velata.

\clearpage
\raggedbottom


\noindent\includegraphics[width=\textwidth]{./images/odalisque.png}^^A
This is a short caption test and this one is a long caption test.
\vspace*{2\baselineskip}


\begin{minipage}[t]{0.3\textwidth}
\vbox to -6cm{\noindent\includegraphics[width=0.98\textwidth]{./images/ginerva.png}
This is a short caption test and this one is a long caption test.}
\end{minipage}%
\begin{minipage}[t]{.7\textwidth}%
\noindent\textbf{\Huge \hfill Kathleen Gilje\hskip0.1em\hfill}\\[2\baselineskip]
\end{minipage}


\leftskip0.41\textwidth

Lorem ipsum dolor sit amet, consectetur adipiscing elit. Etiam eu nunc dolor. Nam arcu nisi, hendrerit at facilisis et, aliquet sit amet massa. Aenean ullamcorper mi dolor. Sed ut urna vitae elit tristique varius tempus vitae orci. Maecenas tristique lectus vel enim posuere congue. Aliquam pellentesque nisl vel nunc iaculis dictum. Sed luctus, orci vehicula blandit rutrum, risus justo aliquet elit, id venenatis est libero nec sem. Sed varius molestie ante non fringilla.

Vestibulum ut mollis odio. Vivamus ut risus eu dolor laoreet viverra. Nullam elit erat, congue at placerat ut, posuere non diam. Suspendisse eget dui et mi varius bibendum at non orci. Morbi justo arcu, posuere non tempus at, vestibulum sit amet lorem. Class aptent taciti sociosqu ad litora torquent per conubia nostra, per inceptos himenaeos. Donec tempor dignissim tellus, vitae vestibulum tellus hendrerit tempus. Nullam varius justo sit amet risus semper non semper eros placerat. Integer eleifend ligula in est gravida ornare tincidunt velit tristique.


Donec vel erat a ipsum condimentum volutpat vel non odio. Vivamus non justo orci. Pellentesque ligula ipsum, vestibulum at molestie vel, mollis sed odio. Donec rhoncus, sem in auctor tincidunt, libero quam scelerisque urna, et volutpat purus magna ac nulla. Cras vel quam nec urna viverra ornare eu et nibh. Pellentesque tincidunt leo non odio varius vitae sollicitudin neque adipiscing. 

\section{Full Page Images}

\leftskip0pt\parindent1em

In euismod, enim a dictum pharetra, libero nibh tempor enim, vel fermentum justo justo eget sem. Integer convallis massa nec turpis volutpat tristique. Quisque fringilla volutpat sem porta elementum. Donec vel metus quis nisl venenatis vehicula ac quis est. Maecenas vulputate lacinia lacus quis porttitor. Aliquam consectetur consectetur metus eu bibendum. Lorem ipsum dolor sit amet, consectetur adipiscing elit. In sem mauris, mollis nec pulvinar posuere, facilisis quis turpis. Quisque vel laoreet mauris.

Quisque ultrices dignissim odio at malesuada. Duis euismod tellus nec ante porta vel ullamcorper orci semper. Vivamus in eros est. Etiam et pellentesque nisi. Sed faucibus dictum tortor vitae accumsan. Donec ante risus, ornare et iaculis eget, cursus at metus. Maecenas neque urna, rutrum sit amet lacinia non, accumsan nec tortor. Proin tempor dictum porta. Morbi luctus nulla et sapien elementum aliquam ut eget neque. Quisque lobortis eleifend lorem adipiscing semper. Quisque molestie magna lorem, non mollis est. Mauris urna arcu, pretium sed dignissim id, tempor accumsan massa



\noindent\includegraphics[width=\textwidth]{./images/napoleon.jpg}
This is a short caption test and this one is a long caption test.



\clearpage
\newenvironment{kathleen}[1][b]{\def\placement{#1}\parindent0pt
}{}

\cxset{kathleen align/.is choice,
       kathleen align/top/.code=\xdef\kathleenplacement@cx{t},
       kathleen align/bottom/.code=\xdef\kathleenplacement@cx{b},
       kathleen align/center/.code=\xdef\kathleenplacement@cx{c},
       kathleen imagei/.code=\def\imagei{\includegraphics[width=\textwidth]{#1}\par},
 kathleen imageii/.code=\def\imageii{\includegraphics[width=\textwidth]{#1}\par},
kathleen imageiii/.code=\def\imageiii{\includegraphics[width=\textwidth]{#1}\par},
kathleen imageiv/.code=\def\imageiv{\includegraphics[width=\textwidth]{#1}\par},
kathleen imagev/.code=\def\imagev{\includegraphics[width=\textwidth]{#1}\par},
kathleen captioni/.code=\def\captioni{\captionof{figure}{#1}},
kathleen captionii/.code=\def\captionii{\captionof{figure}{#1}},
kathleen captioniii/.code=\def\captioniii{\captionof{figure}{#1}},
kathleen scale/.store in=\kathleenscale@cx
}

\long\def\printkathleen{\begin{kathleen}[t]
\begin{minipage}{\kathleenscale@cx\textwidth}
\begin{minipage}[\kathleenplacement@cx]{0.3\textwidth}
\vbox{}
\imagei
\captioni
\imageii
\captionii
\imageiii
\captioniii
\end{minipage}\hspace{1cm}
\begin{minipage}[\kathleenplacement@cx]{0.46\textwidth}
\vbox{}
\imageiv
\captionof{figure}{This is a short caption test and this one is a long caption test.}\par
\imagev
\captionof{figure}{This is a short caption test and this one is a long caption test.}
\end{minipage}
\end{minipage}
\end{kathleen}}

\begin{figure}
\cxset{kathleen align = top,
       kathleen imagei = {./images/ladyagnew.png},
       kathleen imageii = {./images/etta.png},
       kathleen imageiii = {./images/etta.png},
       kathleen imageiv = {./images/ladyagnew.png},
       kathleen imagev  = {./images/etta.png},
       kathleen captioni = {Al contrario di quanto si pensi, Lorem Ipsum non \`e semplicemente una sequenza casuale di caratteri. Risale ad un classico della letteratura latina del 45 AC.}, 
       kathleen captionii = {Finibus Bonorum et Malorum di Cicerone. Questo testo un trattato su teorie di etica, molto popolare nel Rinascimento. La prima riga del Lorem Ipsum.},
       kathleen captioniii= This is a short caption.,
       kathleen scale = 1.1,
} 

\printkathleen

\caption{The Kathleen template page. It consists of five images and their caption text. Parameters can be set via a key value interface.}
\end{figure}
\clearpage

\cxset{kathleen align = top,
       kathleen imagei = {./images/ladyagnew.png},
       kathleen imageii = {./images/etta.png},
       kathleen imageiii = {./images/etta.png},
       kathleen imageiv = {./images/ladyagnew.png},
       kathleen imagev  = {./images/etta.png},
       kathleen captioni = {Al contrario di quanto si pensi, Lorem Ipsum non \`e semplicemente.}, 
       kathleen captionii = {Finibus Bonorum et Malorum di Cicerone. Questo testo  un trattato.},
       kathleen captioniii= This is a short caption.,
       kathleen scale = 0.7
} 

\cxset{kathleen align=bottom}




\begin{center}\printkathleen\par\label{kathleen}\end{center}

\newpage

\section{The Kathleen template} 

A lot of pages in image rich books have complicated settings for images.
These are difficult to manipulate and we provide here what we hope is
a better method. For example the Figure~\ref{kathleen} shows such a complex layout. This can be achieved by only filling in the template
values as shown below.

\begin{tcolorbox}
\begin{lstlisting}
\cxset{kathleen align = top,
       kathleen imagei = ladyagnew,
       kathleen imageii = etta,
       kathleen imageiii = etta,
       kathleen imageiv = ladyagnew,
       kathleen imagev  = etta,
       kathleen captioni = {Al contrario di quanto si pensi, Lorem Ipsum non \`e semplicemente una sequenza casuale di caratteri. Risale ad un classico della letteratura latina del 45 AC.}, 
       kathleen captionii = {Finibus Bonorum et Malorum di Cicerone. Questo testo un trattato su teorie di etica, molto popolare nel Rinascimento. La prima riga del Lorem Ipsum.},
       kathleen captioniii= This is a short caption.,} 
\cxset{kathleen align=bottom,
       kathleen scale=.5}

\printkathleen

\end{lstlisting}
\end{tcolorbox}


\newgeometry{top=0pt,left=1cm,right=1cm,marginparsep=0pt}

\clearpage


\parindent0pt
\pagestyle{empty}

\fboxsep0pt
\fboxrule0pt

\vspace*{-1cm}
\begin{minipage}{1.05\textwidth}
\hskip-0.9cm\includegraphics[width=1.03\textwidth]{./images/parasol-05.jpg}\\[-27.5pt]
\setlength{\linewidth}{0.95\textwidth}
\setlength{\columnsep}{10pt}
\begin{multicols}{2}
\noindent \footnotesize\textbf{DESIGNED FOR CONTRAST} with the wearer's ensemble, these plaid  tafetta and green rayon parasols, are best sellers at Maey's in New York. Set of matching parasol and shoes, or
gloves, scarves or bags, are also available to give simple dresses
a custom appearance.
\end{multicols}
\vspace{-0.25cm}
\rule{1.5cm}{0pt}\fbox{
\begin{minipage}[t]{0.87\textwidth}
\begin{minipage}[t]{0.41\textwidth}
\includegraphics[width=1.03\textwidth]{./images/parasol-06.jpg}\par%
\noindent \footnotesize\textbf{CHERRY ORNAMENTS} adorn handle and tip of this parasol, made by Jane Derby to go with the afternoon dress. Straight handles are very popular.
\end{minipage}\hspace{0.5cm}
\begin{minipage}[t]{0.4\textwidth}
   \includegraphics[width=1\textwidth]{./images/parasol-07.jpg}\par
\noindent \footnotesize\textbf{MATCHING SETS} of afternoon dress
and parasol, and four-piece polka dot weekend dress and parasol,
both designed by Briganne.
\end{minipage}
\end{minipage}
}

\vfill

\captionof{figure}{Balancing three images on a page. Should the larger image be at the top or at the bottom?}
\end{minipage}




\begin{minipage}{\textwidth}
\begin{minipage}[b][\textheight][b]{.47\linewidth}
\vspace*{2cm}

\includegraphics[width=\linewidth]{./images/parasol-03.jpg}\par
\vspace{2\baselineskip}

\centerline{\bfseries\Huge Parasols}
\vspace{2\baselineskip}

\begin{quote}
\lipsum[2]
\end{quote}

\vfill

\textbf{SHOES AND PARASOL SET} in pink are here combined with a dress, one of whose skirts is pink. Parasol is from New York's ``Uncle Sam'' parasol shop.
\end{minipage}\hspace*{1cm}
\begin{minipage}[b]{.53\linewidth}
\mbox{}
\includegraphics[width=\linewidth]{./images/parasol-01.jpg}\par
\end{minipage}
\end{minipage}

\newgeometry{top=1.5cm,bottom=3cm,left=3.5cm,right=3.5cm}

\clearpage
%  \cxset{toc image=botticelli-34}

\chapter{Image Pages}

\lipsum[1-5]

\clearpage

{
\parindent0pt
\pagestyle{empty}

\fboxsep0pt
\fboxrule0pt

\vspace*{-1cm}
\begin{minipage}{1.05\textwidth}
\hskip-0.9cm\includegraphics[width=1.03\textwidth]{twowomen-03}\\[-27.5pt]
\setlength{\linewidth}{0.95\textwidth}
\setlength{\columnsep}{10pt}
\begin{multicols}{2}
\noindent \footnotesize\textbf{TWO WOMEN,} portrays a professional model dressed and undressed. The range and richness of colors is unusual among Bellows' pictures. Bellows always had a horror of studio pictures and ``pretty nudes.'' He rarely worked from professional models and never painted a still life. This painting was published in Life Magazine.
\end{multicols}
\vspace{-0.25cm}
\rule{1.5cm}{0pt}\fbox{
\begin{minipage}[t]{0.87\textwidth}
\begin{minipage}[t]{0.41\textwidth}
\includegraphics[width=1.03\textwidth]{threewomen01}\par\vspace*{-8pt}%
\captionof*{figure}{\noindent\footnotesize\textbf{WALDO PEIRCE}, a famous painting in his own right,
turned model for Bellows, posed for this impressive portrait in New York studio in 1920.}
\end{minipage}\hspace{0.5cm}
\begin{minipage}[t]{0.4\textwidth}
   \includegraphics[width=1\textwidth]{threewomen02}\vspace*{-8pt}
    \captionof*{figure}{\noindent\footnotesize\textbf{Mrs Katherine Rosen,}
the daughter of Charles Rosen, he was an artist and neighbor of bellows, posed for this meditative study in 1921.}
\end{minipage}
\end{minipage}
}

\vfill

\captionof{figure}{Balancing three images on a page. Should the larger image be at the top or at the bottom?}
\end{minipage}
}


%   
%
\newgeometry{top=2cm, bottom=1cm, left=1cm, right=1cm,
               marginparsep=0cm, marginpar=0pt}
\makeatletter
\cxset{kroll scale/.store in = \scalekroll@cx,
       kroll left column width/.store in = \krollleftcolumnwidth@cx,
       kroll imagei/.store in = \krollimagei@cx,
       kroll imagei caption/.store in = \krollimageicaption@cx,
       kroll imageii/.store in = \krollimageii@cx,
       kroll imageii caption/.store in = \krollimageiicaption@cx,
       kroll left header/.store in = \krollleftheader@cx,
       kroll header/.store in = \krollheader@cx}

\cxset{kroll scale = 1,
       kroll left column width = .3\textwidth,
       kroll left header = Leon\\[15pt] Kroll,
       kroll imagei = krollportrait,
       kroll imagei caption = shows Kroll at 59. Says he. ``Painting is 
             fascinating'' even when motif my own mug.,
       kroll imageii = nudeback,
       kroll imageii caption = {NUDE  BACK  SHOWS   A  DANCER  WHOSE  BACK  SAYS  KROLL,  HAS  BEAUTIFUL  PLANES},
       kroll header = \scalebox{.97}{THE DEAN OF US NUDE-PAINTERS}
    }

\newenvironment{kroll}{%
\renewenvironment{leftcolumn}{%
   \minipage[b]{\krollleftcolumnwidth@cx}%
  }{\endminipage}\hspace*{0cm}%
 \renewenvironment{rightcolumn}{%
   \minipage[b]{.62\textwidth}%
  }{\endminipage}\hspace*{0cm}% 
\begin{minipage}{\scalekroll@cx\textwidth}%
 \noindent
  \begin{leftcolumn}%
   \MainHeader{\krollleftheader@cx}%
   \putimage[width=0.5\linewidth]{\krollimagei@cx}\par
   \aheader{\krollimageicaption@cx}%
\end{leftcolumn}\hfill%
\begin{rightcolumn}%
 \includegraphics[width=\linewidth]{\krollimageii@cx}%
 \onelinecaption{{\resizebox{\linewidth}{5.5pt}{\bfseries   \krollimageiicaption@cx}}\par}%
 \onelineheader{\krollheader@cx}%
 \begin{multicols}{2}}
{%
   \end{multicols}%
   \end{rightcolumn}%
   \end{minipage}} 
\makeatother



\begin{kroll}
 \lettrine{A}{t the} age of 63 when businessmen are thinking of retiring leon Kroll according to Life Magazine was having the busiest time of his life, just doing what comes naturally.  \lorem
\end{kroll}

\cxset{kroll scale = 1,
       kroll left column width = .3\textwidth,
       kroll left header = Cooling\\Water\\ Systems\vskip5pt
                          {\bfseries \Large \lorem},
       kroll imagei = industrial,
       kroll imagei caption = shows Kroll at 59. Says he. ``Painting is 
                                    fascinating'' even when motif my own mug.,
        kroll imageii = industrial,
       kroll imageii caption = {NUDE  BACK  SHOWS   A  DANCER  WHOSE  BACK  SAYS  KROLL,  HAS  BEAUTIFUL  PLANES},
       kroll header = \scalebox{1}{\hfill HVAC CHILLED WATER SYSTEMS \hfill}
    }


\begin{kroll}
 \lettrine{A}{t the} age of 63 when businessmen are thinking of retiring leon Kroll according to Life Magazine was having the busiest time of his life, just doing what comes naturally.  \lorem \the\pagetotal
\end{kroll}

\restoregeometry
%  % 
%
\newgeometry{top=1cm, bottom=1cm, left=1cm, right=1cm,
               marginparsep=0cm, marginpar=0pt}
\newpage

\makeatletter
\cxset{bache scale/.store in = \scalebache@cx,
    bache left column width/.store in = \bacheleftcolumnwidth@cx,
    bache imagei/.store in = \bacheimagei@cx,
    bache imagei caption/.store in = \bacheimageicaption@cx,
    bache imageii/.store in = \bacheimageii@cx,
    bache imageii caption/.store in = \bacheimageiicaption@cx,
    bache left header/.store in = \bacheleftheader@cx,
    bache header/.store in = \bacheheader@cx}%
\cxset{bache scale = 1,
    bache left column width = {\dimexpr\textwidth-.4\textwidth\relax},
    bache left header =,
    bache imagei = bache-01,
    bache imagei caption ={\begin{multicols}{2}\lorem\lorem\end{multicols}},
    bache imageii = nudeback,
    bache imageii caption = {JULES BACHE},
    bache header = \scalebox{.97}{THE DEAN OF US NUDE-PAINTERS}
    }%
\newenvironment{bache}{%
\parindent0pt
\renewenvironment{leftcolumn}{%
   \minipage[t]{\bacheleftcolumnwidth@cx}%
   \leavevmode   
  }{\endminipage}\hspace*{0cm}%
 \renewenvironment{rightcolumn}{%
   \minipage[t][\textheight-45pt][t]{.37\textwidth}%
   \mbox{}%
  }{\endminipage}\hspace*{0cm}% 
\begin{minipage}[t][\textheight][t]{\scalebache@cx\textwidth}%
\resizebox{\textwidth}{!}{\Large\bfseries\sffamily JULES BACHE GIVES HIS \$20,000,000 ART COLLECTION TO NEW YORK}\par%
\begin{leftcolumn}%
\mbox{}%ncessesary to line on top
\par\leavevmode\includegraphics[width=\linewidth]{\bacheimagei@cx}\par
\bacheimageicaption@cx%
\end{leftcolumn}\hfill%
\begin{rightcolumn}%
\mbox{}%
\intextsep0pt
\@afterindentfalse\parindent1em
\begin{wrapfigure}{l}{0pt}
 \includegraphics[width=.37\linewidth]{bache-02}
\caption*{\bfseries\sffamily \bacheimageiicaption@cx}
  \end{wrapfigure}\ignorespaces
 }
{\end{rightcolumn}%
\end{minipage}%
} %
%

\begin{bache}
This layout has a dominant left column image. It is important to
ensure that the image has an aspect ratio to suit. Unfortunately
it is very difficult to crop and scale an image via \tex so a bit
of experimentation is appropriate.

It is also important to ensure that you add an adequate amount
of text during editing, otherwise the layout will not look very good. The right
column has two images (it really looks better when it has two images rather than
one and the bottom image is really a filler, if you have more or less
text you may have to go back and crop the image to suit. Any extra space on the right column is used as glue. The template also has a manual mode, where one can adjust the lengths and writing
a bit more accurately. \label{bache}

\vfill
\includegraphics[width=\linewidth]{bache-03}
\end{bache}

\restoregeometry


\section{The bache template}
The bache template, named after the dominant photograph in the sample template
is an adaptation of a layout from a Life magazine. The basic layout is shown below and
a full page sample is shown on page~\pageref{bache}.

{\begin{center}

\cxset{bache scale=.7}
\fboxsep0pt
\resizebox{\scalebache@cx\textwidth}{!}{\begin{bache}
This layout has a dominant left column image. It is important to
ensure that the image has an aspect ratio to suit. Unfortunately
it is very difficult to crop and scale an image via \tex so a bit
of experimentation is appropriate.

It is also important to ensure that you add an adequate amount
of text during editing, otherwise the layout will not look very good. The right
column has two images (it really looks better when it has two images rather than
one and the bottom image is really a filler

\includegraphics[width=\linewidth]{bache-03}
\end{bache}}

\end{center}
}

\begin{lstlisting}
\cxset{bache scale = 1,
    bache left column width = {\dimexpr\textwidth-.4\textwidth\relax},
    bache left header =,
    bache imagei = bache-01,
    bache imagei caption ={\begin{multicols}{2}\lorem\lorem\end{multicols}},
    bache imageii = nudeback,
    bache imageii caption = {JULES BACHE},
    bache header = \scalebox{.97}{THE DEAN OF US NUDE-PAINTERS}
    }%
\end{lstlisting}

Keeping simplicity in mind, we only require the user to fill the above template and to type only
a short piece of code and the text. It is preferable to write the last piece of
text, rather than insert this type of writing in the template, as one may need to iterate a 
couple of times to get the right amount of text.
\begin{lstlisting}
\begin{bache}
    text body on right column.
\end{bache}
\end{lstlisting}

I believe that filling a few lines of information in a template and then a short environment, is
the simplest way possible. A more complicated way is to set the template on manual and
build it piece by piece with commands.





\restoregeometry
%  \newpage
\makeatletter
\@specialtrue
\makeatother

\tikzset{dim/.style={color=black!25,thick,>=stealth,}}%
\def\labelit{%
{\tikz[remember picture]\draw[dim,<-|,overlay] (0,0)--++(0,0.8)--++(1,0) node[right,fill=blue!15,text=black] {textii};}%
}%
\def\labelitt#1{%
\hbox to 0pt{{\tikz[remember picture]\draw[dim,<-|,overlay] (0,0)(0,0.3)--++(0.0,0.8)--++(0.5,0) node[right,fill=blue!15,text=black] {\footnotesize\texttt{#1}};}
}}%
\cxset{custom = genetics,
 title font-size=\Huge,
 title font-weight=\bfseries,
 title font-family=\bfseries,
 image = {./images/greco-02.jpg},
 image caption={\labelitt{image caption}EL GRECO},
 textiii={\labelit 
 \begin{itemize}
\large
\item How to set-up special chapter environments.
\item How to define special field variables.
\item How to set text styles.
  \end{itemize}
}}

\chapter{Grego}

\restoregeometry

\cxset{greco image/.store in=\grecoimage@cx,
       greco heading/.store in=\grecoheading@cx}
\cxset{greco image={./images/julio.jpg},
       greco heading=EL GRECO,
       chapter toc=true}

\newenvironment{greco}{%
\cxset{section align=center,
       section numbering= Roman}
\thispagestyle{plain}%
\checkoddpage
\ifoddpage
  \def\offsetx{0cm}%
\else
  \def\offsetx{-1.9cm}%
\fi%
%\labelitt{figure}

\hskip\offsetx\begin{minipage}[t]{\textwidth}%
\includegraphics[width=\textwidth+1.9cm]{\grecoimage@cx}%

\hbox to 1.15\textwidth{\hss{\tiny\textbf{FROM A PAINTING BY EL GRECO.}}}

\end{minipage}
\vspace*{2\baselineskip}

\section{\grecoheading@cx}
\offsetx
\begin{multicols}{3}
\parindent1em
}{\end{multicols}}



\cxset{greco heading= JULIO CLOVIO}
\begin{greco}

\noindent \lettrine{G}{iulio Clovio} was born in Croatia. He was a native of Griane, a village near the town of Modru.[4] It is not known where he had his early training, but he may have studied art with monks at Fiume of Novi Bazar when he was young. [5]

He moved to Italy at age 18 and entered the household Cardinal Marino Grimani where he was trained as a painter. Between 1516 and ca 1523 Clovio may have lived with Marino in the residence of the latter’s uncle Cardinal Domenico Grimani in Rome. [6] Clovio studied under Giulio Romano during this early period. [7]

While a protege of Cardinal Domenico Grimani Clovio engraved medals and seals for him, as well as the Grimani Commentary Ms., an important early illuminated book (now Sir John Soane's Museum, London).

By 1524 Clovio was at Buda, at the Hungarian court of King Louis II, for whom he painted the ``Judgment of Paris'' and ``Lucretia''. After Louis' death in the Battle of Mohács, Clovio travelled to Rome where he continued his career.[8]

After 1527 he visited several monasteries of the Canons Regular of St. Augustine. In 1534 Clovio returned to the household of Cardinal Marino Grimani.[8] A year later Clovio may have followed Marino when the latter was appointed as a papal legate to Perugia, where Clovio is thought to have worked on illustrations for the Soane Manuscript written by Marino Grimani around that time. Clovio likely returned to Rome by the end of 1538 when he is known to have met with the writer Francisco de Hollanda.

\end{greco}
\clearpage


\cxset{greco heading= EL GRECO}
\cxset{greco image={./images/greco-02.jpg}}
\begin{greco}
\noindent \lettrine{G}{iulio Clovio} was born in Croatia. He was a native of Griane, a village near the town of Modru.[4] It is not known where he had his early training, but he may have studied art with monks at Fiume of Novi Bazar when he was young. [5] 

He moved to Italy at age 18 and entered the household Cardinal Marino Grimani where he was trained as a painter. Between 1516 and ca 1523 Clovio may have lived with Marino in the residence of the latter’s uncle Cardinal Domenico Grimani in Rome. [6] Clovio studied under Giulio Romano during this early period. [7]

While a protege of Cardinal Domenico Grimani Clovio engraved medals and seals for him, as well as the Grimani Commentary Ms., an important early illuminated book (now Sir John Soane's Museum, London).
By 1524 Clovio was at Buda, at the Hungarian court of King Louis II, for whom he painted the ``Judgment of Paris'' and ``Lucretia''. After Louis' death in the Battle of Mohács, Clovio travelled to Rome where he continued his career.[8]

After 1527 he visited several monasteries of the Canons Regular of St.~Augustine. In 1534 Clovio returned to the household of Cardinal Marino Grimani.[8] A year later Clovio may have followed Marino when the latter was appointed as a papal legate to Perugia, where Clovio is thought to have worked on illustrations for the Soane Manuscript written by Marino Grimani around that time. Clovio likely returned to Rome by the end of 1538 when he is known to have met with the writer Francisco de Hollanda.
\end{greco}

\section{Developing special styles for image pages}

\makeatother





%  \chapter{Captions}

\parindent1em

\section{Setting the caption options}

Captions are very visual and both the text as well as its typography need careful consideration. Most readers will read the captions of figures, before reading the text. We will now in the sections that follow use the caption package to change all the parameters of the caption. This is achieved mainly through one macro, with key value styles.


%\DeclareRobustCommand\acaption{\protect\RaggedRight Lorem ipsum caption \protect\ldots.}
\def\acaption{Lorem ipsum caption \ldots}
\begin{figure*}[h]
\captionsetup{format=plain}
\captionsetup{skip=3pt}
\captionsetup{font=small}
\captionsetup{name=Fig}
\captionsetup[figure]{labelfont=bf,textfont=it}
\RaggedRight
\centering 
\begin{minipage}[t]{90pt}
 \includegraphics[width= 70pt]{./graphics/sudan.jpg}
 \caption{\acaption }
 \label{fig:shortlabel}
\end{minipage}
\captionsetup{name=Figure}
\begin{minipage}[t]{90pt}
 \includegraphics[width= 70pt]{./graphics/sudan.jpg}
 \caption{\acaption }
\end{minipage}
\captionsetup{name=Fig,labelsep=space}
\begin{minipage}[t]{90pt}
 \includegraphics[width= 70pt]{./graphics/sudan.jpg}
 \caption{\acaption }
\end{minipage}
\end{figure*}


To set the caption options we can use the \cmd{\captionsetup} with a set of options.
\begin{dispListing}
\captionsetup{name=Fig, labelsep=space}
\end{dispListing}




It is highly recommended to use the \texttt{caption} package to setup the captions of figures. This package developed by Axel Sommerfeldt offers customization of captions in floating environments such
figure and table and cooperates with many other packages. Most classes provide build-in options and commands for customizing captions. 

And if you are just interested in using the
command \cmd{\captionof}, loading of the very small \pkgname{capt-of package} is usually sufficient.

For wrapped figures the label name is preferable to be shorter, otherwise it leads to text that is either underfull or overfull. You should also try and use the \cmd{\RaggedRight} option of the \pkgname{ragged2e} package to hyphenate the ragged right text.

Figure~\ref{fig:shortlabel}, has its label shortened by using ``Fig'' rather than "Figure". I have done this as the space available is narrow. The setup is achieved using the \texttt{caption} package's \verb+\captiosetup+ command. We will use this command to specify, the fonts, numbering, labels, separators and other parameters of the captions.

\subsection{Adjusting the label}%%

The \emph{label} is the name of the figure. Sometimes it is abbreviated, sometimes it is not. Adjusting the label, is achieved by setting the key parameter |name| in \cmd{\captionsetup}. 

\begin{commands}[]{}
\cmd{\captionsetup}\marg{name=Figure}
\end{commands}



The figures were typeset by using a different setup style. The first one displays the  label fully, the second uses an abbreviation and the third has a new line, before the caption text is displayed.

\subsection{Fonts}

There are three font options which affects different parts of the caption: One affecting the
whole caption (font), one which only affects the caption label and separator (labelfont) and at least one which only affects the caption text (textfont). You set them up using the options shown in the table below:

\begin{table}[htp]
\centering
\smaller
\caption{Key values for fonts, using the caption package}
\begin{tabular}{ll}
\toprule
normalfont &Normal shape\\
up &Upright shape\\
it &Italic shape \\
sl &Slanted shape\\
sc & \textsc{Small Caps Shape}\\
md &Medium series\\
bf &Bold series\\
rm &Roman family\\
sf &Sans Serif family\\
tt &Typewriter family\\
\bottomrule
\end{tabular}
\end{table}

\emphasis{captionsetup,captionof}
\begin{teXXX}
\captionsetup{name=Figure.}
\captionof{figure}{\acaption}
\end{teXXX}


\begin{figure*}[h]
\begin{commands}[]{}
\captionsetup{skip=3pt}
\captionsetup{font=small}
\captionsetup{name=Fig}
\captionsetup{labelfont=bf,textfont=it, format=plain}
\RaggedRight
\centering 
\begin{minipage}[t]{90pt}
 \includegraphics[width= 70pt]{./graphics/sudan.jpg}
 \caption{\acaption }
\end{minipage}
\captionsetup{name=Figure}
\begin{minipage}[t]{90pt}
 \includegraphics[width= 70pt]{./graphics/sudan.jpg}
 \caption{\acaption }
\end{minipage}
\captionsetup{name=Fig,labelsep=space}
\begin{minipage}[t]{90pt}
 \includegraphics[width= 70pt]{./graphics/sudan.jpg}
 \caption{\acaption }
\end{minipage}
 \caption{Three boys example (changing the figure name).}
 \end{commands}
\end{figure*}


\section{Adjusting the Separator}


The separator can be adjusted in a similar manner. The package offers the options, \option{none}, \option{colon}, \option{period}, \option{space}, \option{quad}, \option{newline} and \option{enddash}.  The various options are illustrated
in \hbox{Figures~18-23}.


\section{Adjusting spacing before and after the figure}

Skips are the amount of vertical space between the caption and the figure. The caption package offers the option
\option{skip=amount}.\footnote{The standard \LaTeX\ classes article, report and book preset it to \option{skip=10pt}.} We will now make some recommendations as to how to adjust this spacing.

\medskip

\begin{figure}[htp]
\everypar{}
\captionsetup{name=Photo,parindent=0pt,minmargin=0pt,width=3sp,labelsep=period,skip=5pt,margin={0pt,0pt},position=bottom}

\noindent\includegraphics[width=\textwidth]{./graphics/damageinspection.jpg}

\noindent\caption{Damage Inspection.A squadron operations officer of the 332d Fighter Group points out a cannon hole to ground crew, Italy, 1945.}\par
\end{figure}

\medskip

The space between the image and the caption should be approximately half the point size of the text. The photo above had the following settings:


\begin{teX}

\captionsetup{name=Photo, labelsep=period,
                    skip=5pt, font=scriptsize,
                    position=bottom, margin{0pt,0pt}}
\end{teX}

The \docAuxCommand{caption} command offered by \latexe has a design flaw\footnote{According to Axel Sommerfeldt, \textit{see} the \textit{Caption} documentation.}: The command does not
know if it stands on the beginning of the figure or table, or at the end. Therefore it does
not know where to put the space separating the caption from the content of the figure
or table. While the standard implementation always puts the space above the caption
in floating environments (and inconsistently below the caption in longtables), the
implementation offered by this package is more flexible: By giving the option
\option{position=bottom}, the package correctly inserts the skip.  You can also try the \option{position=auto}.
\medskip

The caption of the next photograph follows a more traditional approach found in
\begin{figure}[htp]
\vskip10pt
\centering
\captionsetup{name=Photo, labelsep=period, position=bottom, textfont=scriptsize, justification=centering}
\includegraphics[width=\textwidth]{./graphics/korea.jpg}

\caption*{\textsc{25th Division Troops Unload Trucks and Equipment}\par
\textit{at Sasebo Railway Station, Japan, for transport to Korea, 1950.}}
\vskip10pt
\end{figure}
many books where, there is no label or number and the text is split into two lines. The first line is a photograph heading and the second line is printed in italics with some explanatory stuff about the photo.

To achieve this result we need to firstly use the \emph{starred} form of the caption command and override the formatting commands of the caption.

\begin{teX}
\begin{figure}[htp]
\vskip10pt
\centering
\captionsetup{...}
\includegraphics[width=\textwidth]{filename}
\caption*{\textsc{25th Division Troops Unload Trucks and Equipment}\par
\textit{at Sasebo Railway Station, Japan, for transport to Korea, 1950.}}
\vskip10pt
\end{figure}
\end{teX}

You will notice that the photograph is between the lines of the paragraph, so I have added some small skips to arrange proper spacing around it.


To my knowledge, you cannot customize the caption package to get the heading for the caption text. You can define your own command to do so:
\begin{phdverbatim}
\newcommand\captionx[2]{\par%
     \leavevmode 
     \caption*{\textsc{#1}\par%
     \textit{#1}}%
}
\end{phdverbatim}

\DeclareDocumentCommand\captionx{m m}{%
     \leavevmode
     \caption{\textsc{#1} %
     \textit{#2}}%
}

With photographs you need sometimes to add a "credit" to credit the photographer or even a copyright notice. This is necessary, especially if you have licensed images from an agency. For this I would prefer a simple solution where we
just define an \verb+addcredit+ macro. More customization might be possible, as well as a few setup macros. As an exercise have a look at some publications and see how they handle this type of photographs.

\begin{teX}
\newcommand\addcredit[1]{%
   \vspace*{-10.5pt}%
   \scriptsize
   \hfill\hfill
   \textit{Credit: #1}%
}
\end{teX}

\providecommand\addcredit[1]{%
 \scriptsize%
 \vspace*{-10.5pt}%
 \hfill\hfill\textit{Credit: #1}%
 \vspace{10pt}
}

The results of the code so farm can be seen in the photograph that follows. The credit has been added and
the text has been centered and styled as required.

The full code is now shown below:

\begin{teX}
\begin{figure}[htp]
  \centering
  \captionsetup{skip=0pt,  justification=centering}%
  \includegraphics[width=\textwidth]{./graphics/rosenberg.jpg}%
  \addcredit{U.S. DoD.}%
  \captionx{Assistant Secretary Rosenberg}{talks ...}
\end{figure}
\end{teX}

\begin{figure}[htp]
  \centering
  \captionsetup{name=Photo, labelsep=period, skip=0pt, position=top, textfont=scriptsize,    justification=centering}%
\includegraphics[width=\textwidth]{./graphics/rosenberg.jpg}%
\addcredit{U.S. DoD.}%
\captionx{Assistant Secretary Rosenberg}{talks with men of the 140th Medium Tank Battalion during a Far East tour.}
\vspace{10pt}
\end{figure}

It all looks perfect, but there is a snag. If the photo is narrower, there will be nothing to stop it floating past the edge of the photo. This can be corrected by enclosing the commands within a minipage.


\begin{figure}[htp]
\begin{commands}[]{}
\captionsetup{name=Fig., labelsep=period, format=plain, margin{30pt,30pt}}%
\includegraphics[width=0.97\textwidth]{./graphics/movingup.jpg}%
\addcredit{U.S. DoD.}%
\caption{The effects of the credit going past the edge of the figure. This can be corrected by adding a minipage to hold both the include graaphics, as well as the addcredit command. }

\begin{verbatim}
\begin{figure}[htp]
  \captionsetup{name=Fig., labelsep=period, format=plain}%
  \includegraphics[width=0.97\textwidth]{./graphics/movingup.jpg}%
  \addcredit{U.S. DoD.}%
  \caption{The effects of the credit going past the edge of the figure. This can be corrected by adding a minipage to hold both the include graaphics, as well as the addcredit command. }
\end{figure}
\end{verbatim}
\end{commands}
\end{figure}

\section{Presentation}

Presentation of a lot of figures can influence the appearance of a book tremendously. Like sectioning commands and text styling, magazines and books can be recognized from the styling of their pictures. In figures we imitated the appearance of photographs in Life Magazine. Life in the forties was in the front with the troops and had some great photographers.  It had a style still very hard to improve on.

\begin{figure}[htp]
\bgroup
\parindent=0pt
\null
\clearcaptionsetup{figure}
\captionsetup{style=default,name=Photo.,skip=3pt,parindent=0pt, labelsep=period, margin={0pt,0pt}}%
\begin{minipage}[t]{0.48\textwidth}%
      \includegraphics[width=\textwidth]{./graphics/movingup.jpg}%
      \addcredit{U.S. DoD.}\vskip1sp
     \caption{The effects of the credit going past the edge of the figure. This can be corrected by adding a minipage to hold both commands.}%
\end{minipage}\hfill\hfill
\begin{minipage}[t]{0.48\textwidth}
\includegraphics[width=\linewidth]{./graphics/survivors.jpg}%
%      \addcredit{U.S. DoD.}%
\caption{The effects of the credit going past the edge of the figure. This can be corrected by adding a minipage to hold both commands. }
    
\end{minipage}

 \begin{minipage}[t]{0.48\linewidth}
      \includegraphics[width=\linewidth]{./graphics/img009.jpg}%
      \addcredit{U.S. DoD.}%
     \caption{Engineer Construction Troops in Liberia, July 1942.}
\end{minipage}\hfill\hfill
\begin{minipage}[t]{0.48\textwidth}
      \includegraphics[width=\textwidth]{./graphics/survivors.jpg}%
      \addcredit{U.S. DoD.}%
     \caption{The effects of the credit going past the edge of the figure. This can be corrected by adding a minipage to hold both commands. }
\end{minipage}
\begin{minipage}[t]{0.48\textwidth}
      \includegraphics[width=\textwidth]{./graphics/img126.jpg}%
      \addcredit{U.S. DoD.}%
     \caption{Marine Reinforcements.
A light machine gun squad of 3d Battalion, 1st Marines, arrives during the battle for ``Boulder City.'' }
\end{minipage}\hfill\hfill
\begin{minipage}[t]{0.48\textwidth}
      \includegraphics[width=\textwidth]{./graphics/img124.jpg}%
      \addcredit{U.S. DoD.}%
     \caption{Brothers Under the Skin, inductees at Fort Sam Houston, Texas, 1953. }
\end{minipage}
\egroup
\end{figure}
\newpage


\endinput
%  \cxset{steward,
  chapter toc=true,
  toc image=false,
  numbering=arabic,
  custom = stewart,
  offsety=0cm,
  image={./images/hine06.jpg},
  texti={A picture is worth a thousand words, but if you don't add a good description of what it is in a caption, your readers will be left scratching their heads. Here we discuss captions in general as well as the formatting commands available in LaTeX, some common packages and athena.},
  textii={In this chapter we discuss methods that allow the formatting and positioning of captions, based on a set of key values. Central  to this process is the separation of content from presentation.
We also discuss the basic formatting tools that are available and how one can modify them to blend them with the rest of the design.
 }
}
\cxset{section numbering prefix=\thechapter.}
\chapter{Typesetting Captions}
\section{Introduction}

Publications that include figures and tables will normally dictate
the style of captions. Captions, besides normal typography 
requirements such as fonts, can vary in their numbering scheme, can
include a label such as figure or fig they can include a colon or stop
after the label and can be centered hanged or left justified. 
Numbering can also vary; the counters can be reset at every chapter or section or can be continuous. So
there are quite a few options to define in a template.

The formatting commands for the captions key value interface follow the same style of the rest of the package. We use the \pkg{caption} package to provide the interface to the key value settings. To format the captions you just include the appropriate keys in one of the style
files.


\section{Conventions}

All caption keys start with the word |caption|. The float type follows, so |caption figure font-size| refers to the caption of a \textit{figure environment}. If the word \textit{figure} is omitted the style is applicable to both tables and figures. 

As users will probably only have to set these keys once, my recommendation is to use the longer version that can give you finer control. Also your template will be easier to modify in the future.
\medskip

{
\keyval{caption format}{\marg{plain|hang}}{This affects all captions such as tables and figures and will produce either a hang caption or with plain will wrap arund the figure number like a normal paragraph.}

\keyval{caption figure format}{\marg{plain | hang}}{Affects ONLY figure captions such as tables and figures and will produce either a hang caption or with plain will wrap around the figure number like a normal paragraph.}

\keyval{caption figure numbering style}{\marg{auto|continuous|reset on sections|custom}}{}
\keyval{caption figure numbering}{\marg{arabic|alph|Alph|roman|Roman|custom}}{Sets the style of numbering.}
\keyval{caption separator}{\marg{colonsemicolon|none|custom}}{Sets the separator, such as \textbf{:} or a colon or none.}
\keyval{caption label name}{\marg{text}}{Sets the label name such as figure.}
\keyval{caption aboveskip}{\marg{dim}}{Sets the \cs{belowcaptionskip}.}
\keyval{caption belowskip}{\marg{dim}}{Sets the \cs{abovecaptionskip}. You use as simply \texttt{10pt} ot similar. In LaTeX this value is normally set as \texttt{0pt}. Note that below a float normally an additional skip is introduced.}

\keyval{caption font}{\marg{bf|tt|it}}{Sets the font commands. }
\keyval{caption figure name}{Figure}{Sets the figure name}
\keyval{caption defaults}{\marg{true|false}}{Sets all styling back to default styles.}
}

Although it looks a simple piece of text, as you notice there are about
a dozen of variables that one could set. Color can be determined both
from the caption labl colour as well as from hyperlinking if necessary.
More complicated styles can be build in a simila fashion to chapter
heads, by diverting to a custom command \cs{captionspecial}. This
will be provided at the next release of the package.


\cxset{caption format/.code=\captionsetup[figure]{format=#1}} 


\begin{texexample}{}{}
\bgroup
\cxset{caption format = hang}
\includegraphics[width=80pt]{../graphics/sudan.jpg}
\captionof{figure}{This is a very long command to see how all
these can wrap in a hang format, if the text is longer than
a paragraph.}
\egroup

\bgroup
\cxset{caption format = plain}
\captionof{figure}{This is a very long command to see how all
these can wrap in a hang format, if the text is longer than
a paragraph.}
\egroup
\end{texexample}

As you can see from the example, the changes can also be localized if
they are within a group.



\makeatletter
\def\captionlabelfont@cx{bf}
\cxset{caption font/.code = \captionsetup[figure]{font=#1}}
\cxset{caption font={bf}}
\makeatother



\begin{texexample}{}{}
\cxset{caption format = hang}
\cxset{caption font={bf}}
\captionof{figure}{This is a very long command to see how all
these can wrap in a hang format, if the text is longer than
a paragraph.}
\end{texexample}



\section{Technical discussion}

The formatting of the caption, happens in stages like the sectioning commands.  |\@makecaption|  command is responsible for the typesetting and is defined in the standard LaTeX classes. The \cs{caption} and command is defined in the LaTeX kernel in the 
|float.dtx| class. As always we will start our discussion from the user command and follow it through to the typesetting macros.

When the user command \cs{caption} is processed, LaTeX checks if it is outside a float and if it is issues an error message. It then swallows the argument. It then calls \cs{@caption} which does further processing.

\startlineat{5}
\begin{teXXX}
\def\caption{%
  \ifx\@captype\@undefined
   \@latex@error{\noexpand\caption outside float}\@ehd
   \expandafter\@gobble
 \else
   \refstepcounter\@captype
  \expandafter\@firstofone
 \fi
 {\@dblarg{\@caption\@captype}}%
}

\long\def\@caption#1[#2]#3{%
  \par
  \addcontentsline{\csname ext@#1\endcsname}{#1}%
  {\protect\numberline{\csname the#1\endcsname}{\ignorespaces  #2}}%
  \begingroup
        \@parboxrestore
  \if@minipage
     \@setminipage
  \fi
  \normalsize
 \@makecaption{\csname fnum@#1\endcsname}{\ignorespaces #3}
 \par
 \endgroup}
\end{teXXX}


The \cs{@makecaption} is the main typesetting macro and this is
where we need to hook if we want finer grain of control.

\makeatletter
\cxset{label punctuation/.code = \gdef\labelpunctuation@cx{#1}}
\cxset{label space/.code = \gdef\labelhspace@cx{\hskip#1}}
\cxset{caption above skip/.store in= \abovecaptionskip@cx}
\cxset{caption above skip=10pt}
\makeatother

\captionof{figure}{This is a very long command to see how all
these can wrap in a hang format, if the text is longer than
a paragraph.}

\begin{texexample}{}{}
\cxset{caption format = hang}
\cxset{label punctuation=?}
\cxset{label space =1.5em}
\captionof{figure}{This is a very long command to see how all
these can wrap in a hang format, if the text is longer than
a paragraph.}

\end{texexample}



\cxset{label punctuation=?}
\captionof{figure}{This is a very long command to see how all
these can wrap in a hang format, if the text is longer than
a paragraph.}

\cxset{label punctuation=:}
\cxset{label space =.5em}
\def\figurename{\textbf{Figure}}

\makeatletter
\setlength\abovecaptionskip{\abovecaptionskip@cx}
\setlength\belowcaptionskip{0\p@}

\long\def\@makecaption#1#2{%
  \vskip\abovecaptionskip
  \sbox\@tempboxa{#1\labelpunctuation@cx #2}
  \ifdim \wd\@tempboxa >\hsize
    #1\labelpunctuation@cx\labelhspace@cx#2\par
  \else
    \global \@minipagefalse
    \hb@xt@\hsize{\hfil\box\@tempboxa\hfil}%
  \fi
  \vskip\belowcaptionskip}
\makeatother

\begin{teXXX}
\newlength\abovecaptionskip
\newlength\belowcaptionskip
\setlength\abovecaptionskip{10\p@}
\setlength\belowcaptionskip{0\p@}

\long\def\@makecaption#1#2{%
  \vskip\abovecaptionskip
  \sbox\@tempboxa{#1:: #2}
  \ifdim \wd\@tempboxa >\hsize
    #1:: #2\par
  \else
    \global \@minipagefalse
    \hb@xt@\hsize{\hfil\box\@tempboxa\hfil}%
  \fi
  \vskip\belowcaptionskip}
\end{teXXX}



\section{List of Figures}

\begin{docCommand}{listoffigures}{}
The list of figures (lof) is included on a page by using the command \cs{listoffigures}.
\end{docCommand}

The command is not defined in the kernel but rather in the standard classes as shown below. By default it uses the |\chapter| to typeset its heading. Commands like |\tableofcontents| that should set the marks in some page
styles use a |\@mkboth| command, which is |\let| by the pagestyle command |(\ps@...)| to |\markboth| for setting the heading or to |\@gobbletwo| to do nothing.\footnote{See source ltpage.dtx Date: 2000/06/02 Version v1.0k, page311.}

\begin{teXXX}
\newcommand\listoffigures{%
    \if@twocolumn
      \@restonecoltrue\onecolumn
    \else
      \@restonecolfalse
    \fi
    \chapter*{\listfigurename}%
      \@mkboth{\MakeUppercase\listfigurename}%
              {\MakeUppercase\listfigurename}%
    \@starttoc{lof}%
    \if@restonecol\twocolumn\fi
    }
\end{teXXX}



In the |phd| package this is set as a property via a key-value interface and hence we can use a normal chapter. If it need be we can define a special chapter style only for this heading. This way we can control all aspects of the formatting of the head.

\begin{docCommand}{\listfigurename}{}
The \textit{List of Figures} for example in many Social Sciences books is typed as {List of Illustrations} and also adds credits.
\end{docCommand}




\begin{figure}[htp]
\includegraphics[width=\textwidth]{./images/listofillustrations.jpg}
\caption{List of Illustrations extract from \textit{Oxford History of Art, Portraiture}, Shearer West, Oxford University Press, 2004.}
\end{figure}
\begin{figure}[htp]
\includegraphics[width=0.67\textwidth]{./images/titian.jpg}
\centering
\caption{Figure from \textit{Oxford History of Art, Portraiture}, Shearer West, Oxford University Press, 2004. The figures are numbered consecutively and the text in the List of Illustrations have different formatting.}
\end{figure}



\section{Formatting the List of Figures Heading}

LaTeX formats the list of figures heading in a similar manner to that of the Table of Contents. The Title `List of Figures` is obtained from the \cs{listfigurename} and which is also accessible from Babel. It does not add an entry to the ToC.



It is good to know that \cs{captionsetup} has an effect on the current environment only.
So if you want to change settings for the current figure or table only, just place the
\cs{captionsetup} command inside the figure or table right before the \cs{caption}
command.


Many of the caption figures can be changed within \latexe itself. For example to get continuous numbering in the book class.

\begin{teXXX}
\makeatletter
\@removefromreset{table}{chapter}
\renewcommand{\thetable}{\arabic{table}}
\makeatother
\end{teXXX}

\begin{docCommand}{removefromreset}{}
The command \cs{removefromreset} can be found by loading the \pkg{remreset} package. Other combinations are also possible.
\end{docCommand}

\subsection{Caption numbering scheme}

The caption numbering scheme key value interface, provides five
options: 
\medskip

\keyval{caption numbering scheme}{\marg{default|continuous| chapter|section}}{The numbering style either continous or reset per spacing etc...}


\begin{comment}
% Date: Sat, 30 Jul 1994 17:58:55 PST
% From: Donald Arseneau <asnd@erich.triumf.ca>
%
%  |\@removefromreset{FOO}{BAR}| : Removes counter FOO from the list of
%                       counters |\cl@BAR| to be reset when counter BAR
%                       is stepped.  The opposite of |\@addtoreset|.
\end{comment}


\begin{teXXX}

\makeatletter
\setdefaults
\cxset{chapter opening=anywhere,
          chapter font-size=\normalfont,
          title font-size=\large}

\def\@removefromreset#1#2{\let\@tempb\@elt
   \expandafter\let\expandafter\@tempa\csname c@#1\endcsname
   
   \def\@elt##1{\expandafter\ifx\csname c@##1\endcsname\@tempa\else
         \noexpand\@elt{##1}\fi}%
   \expandafter\edef\csname cl@#2\endcsname{\csname cl@#2\endcsname}%
   \let\@elt\@tempb}

\@removefromreset{figure}{chapter}
\renewcommand{\thefigure}{\arabic{figure}}

\@specialfalse\@tocfalse
\gdef\continuousfigures@cx{\@removefromreset{figure}{chapter}
%\gdef{\thefigure}{\arabic{figure}}}

\cxset{caption numbering continuous/.code={\continuousfigures@cx}}


\chapter{This is the First Chapter}

\captionof{figure}{test}

\captionof{figure}{test}

\chapter{This is the Second Chapter}

\captionof{figure}{test}
\captionof{figure}{test}
\makeatother

\end{teXXX}


\begin{figure}[htp]
\includegraphics[width=0.98\textwidth]{./images/captionspecial.jpg}
\centering
\caption{Figure from \textit{Oxford History of Art, Portraiture}, Shearer West, Oxford University Press, 2004. The figures are numbered consecutively and the text in the List of Illustrations have different formatting.}
\end{figure}




%  %\cxset{custom = fashion,
%          fashion image=./images/venus.jpg}

\chapter{Rules and Leaders}
\pagestyle{headings}

\epigraph{He had forty-two boxes, all carefully packed,
With his name painted clearly on each:
But, since he omitted to mention the fact,
They were all left behind on the beach.}{---Lewis Carroll, The Hunting of the Snark}

\section{Rules}

Rules, both horizontal and vertical, are traditionally used in typesetting. In
\tex, a rule does not necessarily have to be long and thin; it has three dimensions,
like a box, and can have any rectangular shape. There are two types of rules, |\hrule| and |\vrule|.

\begin{docCommand}{hrule}{ height\meta{dimen} width \meta{dimen} depth\meta{dimen} }
Draws a rule in vertical mode.
\end{docCommand}

\begin{docCommand}{vrule}{ height\meta{dimen} width \meta{dimen} depth\meta{dimen} }
Draws a rule in horizontal mode.
\end{docCommand}

The shape of the rule does not depend on whether it is \textsc{h} or \textsc{v}, and the difference
between the two types is in the context in which they can be used, not in their
shapes. An |\hrule| is considered vertical material and can be part of a vertical list.

A |\vrule| is the opposite and can only appear in horizontal lists. The reason for
this convention is that a horizontal rule is a good separator between items stacked
vertically, whereas a vertical rule is a natural separator for items laid horizontally,
from left to right.

As a result, a |\vrule| should be used inside a paragraph, such as this \vrule, or in
an |\hbox|. An |\hrule| should be used between paragraphs or in a |\vbox|.

Any unspecified dimensions of a rule are determined [221] by these defaults:

\begin{enumerate}
\item The height of an |\hrule| is 0.4pt, and the depth is 0pt.
\item The width of a |\vrule| is 0.4pt.
\item Other dimensions are determined by extending the rule to the size of the smallest
box containing it. An example of this rule is the |\vrule| above. Its depth is set
equal to the depth of the line it happens to be on.
\end{enumerate}



The rule is extended to the width {\Huge \drawfontframe{\vbox{\hsize=24pt\parindent0pt p\hrule*}}}

\paragraph{Struts} The word \emph{strut} has already been mentioned. It refers to a \refCom{vrule} with width zero. It refers to a |\vrule| with
width zero. A standard strut is part of the plain format and is defined, on [353], as
|\vrule height8.5pt depth3. 5pt width0pt| (the actual definition is slightly more
complicated and takes into account the current mode). Such a rule does not show
up in print and is used to open up boxes. Inexperienced users find it hard to believe
that such a rule can be useful, but a glance at [478] shows that it is one of the most
frequently mentioned terms in the \texbook.

A horizontal strut can also be defined. It is an |\hrule| with height and depth
of zero. Surprisingly, such a thing is rarely used (but see discussion of |\hphantom|
in section 3.24)

\begin{texexample}{Drawing a Ruler}{ex:ruler}
\bgroup

\def\1{\vrule height 0pt depth 2pt}

\def\2{\vrule height 0pt depth 4pt}

\def\3{\vrule height 0pt depth 6pt}

\def\4{\vrule height 0pt depth 8pt}

\def\ruler#1#2#3{%
    \leftline{$\vcenter{%
    \hrule\hbox{\4#1}}\,\,\rm#2\,{#3}$}}%
  
\def\\#1{\hbox to .125in{\hfil#1}}
  
\def\8{\\\1\\\2\\\1\\\3\\\1\\\2\\\1\\\4}%
  
\ruler{\8\8\8\8}4{in}
\egroup
\end{texexample}

Lamport in \latex developed a macro |\rule| to enable users to draw lines without remembering all the rules for horizontal or vertical modes and the like.\footnote{In the latest releases this has been changed to a robust macro, using \textbackslash DeclareRobustCommand.}

\begin{docCommand}{rule}{\oarg{raised}\marg{width}\marg{height} }
Typesets a rule with a  \meta{width} and\meta{height}, raised by \meta{raised}.
\end{docCommand}

\begin{teX}
\def\rule{\@ifnextchar[\@rule{\@rule[\z@]}}%
\def\@rule[#1]#2#3{%
\leavevmode
\hbox{%
  \setlength\@tempdima{#1}%
  \setlength\@tempdimb{#2}%
  \setlength\@tempdimc{#3}%
  \advance\@tempdimc\@tempdima
  \vrule\@width\@tempdimb\@height\@tempdimc\@depth-\@tempdima}
}
\end{teX}

The important macro is |@rule| which sets the lengths and widths to the parameters required by the user. The raising of the rule is achieved by adjusting the depth to the given amount of length to raise the rule.

This is a Lamport rule |\rule[6.5pt]{4pt}{7pt}| typeset as:\rule[6.5pt]{4pt}{7pt} Many \latexe packages 
provide rules for common cases, such as \pkg{booktabs} providing rules that can be used in tables. 

Another useful \latexe macro is |\underline| that can be used to underline text. The \latex version is a modification of the \textsc{plain} version to enable it to be used in math mode. The \textsc{plain} version can still be used in \latexe by using |\@@underline|. 

\section{Applications}

One example of \refCom{vrule} is to provide the color background of a box. This method is used for
example by the \pkg{xcolor} to provide generic drivers. First a |vrule| with the require box dimensions
is typeset in a zero width box using \refCom{rlap} and then the text is overwritten to provide the typeset box, with a background color. One can extend such macros to draw numerous lines at different colors to also 
achieve  a gradient effect.


\begin{texexample}{}{}
\makeatletter
\bgroup
\renewcommand*\color@block[3]%
{{%
\color{blue}%
    \rlap{%
      \ifcolors@
        \vrule\@width#1\@height#2\@depth#3
      \fi
    }%
}} 
\hbox{\color@block{80pt}{30pt}{3.5pt}%
      \sffamily\bfseries\Huge\color{white}FFji}
\egroup 
\makeatother 
\end{texexample}

Of course the example is trivial. In a more detailed macro, it would be preferable to measure the dimensions
of the text and size the background accordingly. 

\section{Leaders}

A leader is a single copy of a pattern, for example in a dashed line a dash is a leader.
Dot leaders are a row of dots that visually connect the chapter titles and section headings to their corresponding page numbers. 

Leaders don't have to be composed of dots, with \tex leaders can be used fill a space with copies of a pattern,
\eg, to put repeated dots between a title and a page number in a table
of contents. 

The Plain Format provides six standard leader definitions. All these definitions are equivalent to an |\hfill| type of horizontal glue.

\medskip

\begin{tabular}{lp{3cm}}
\docAuxCommand{hrulefill}     & \hrulefill\\
\docAuxCommand{dotfill}        & x\dotfill x \\
\docAuxCommand{leftarrowfill} & \leftarrowfill\\
\docAuxCommand{rightarrowfill} & \rightarrowfill\\
\docAuxCommand{downbracefill} & \downbracefill\\
\docAuxCommand{upbracefill} & \upbracefill\\
\end{tabular}
\bigskip


A leader is a single copy of the pattern. The specification of
leaders contains three pieces of information:

\begin{enumerate}
\item  what a single leader is
\item  how much space needs to be filled
\item  how the copies of the pattern should be arranged within the space
\end{enumerate}

In \tex leaders are actually \emph{visual glue}. Wherever glue can go a row of leaders can go.

\begin{texexample}{Leaders}{ex:leaders}
\meaning\dotfill  \par
\meaning\hrulefill\par
\meaning\downbracefill\par
\end{texexample}

\begin{docCommand}{leaders}{}
\tex applies an imaginary window and only those leader boxes are printed which fully fit into the window. This ensures that the leader dots of different lines line up vertically.
\end{docCommand}


\begin{docCommand}{cleaders}{}
\end{docCommand}

\begin{docCommand}{xleaders}{}
\tex  provides three commands for specifying leaders:\cs{leaders},\cs{cleaders},
and\cs{xleaders} (p.~174). The argument of each command specifies the
leader. The command must be followed by glue; the size of the glue specifies
how much space is to be filled. The choice of command determines how
the leaders are arranged within the space.
\end{docCommand}

Rule leaders \textit{fill} the specified amount of space with a rule extending in the direction of the skip
specified. \index{rules and leaders>rule leaders}

The most common application for leaders is to fill the space with either a rule or with dots, such as shown in Example~\ref{leaders} below.

\emphasis{leaders,hbox,hfill}
\begin{texexample}{Leader example}{leaders}
\hbox{Exa\leaders\hrule\hskip20pt e}
\hbox to \linewidth{Section 1.2 \leaders\hbox{..}\hfill\space 15}
Section 1.3 \leaders\hbox{..}\hfill\space 15

\parfillskip=0pt plus1fil

\lipsum*[1]\leaders\hbox{..}\hfill\space 15
\end{texexample}

Leaders must be in a box, such as an \cs{hbox}. If they are not in a box an error is issued by \tex.

\begin{texexample}{}{hboxleaders}
\hbox to \textwidth{g\leaders\hbox{+}\hfill 112}
\end{texexample}

because a horizontal rule has a default height of |.4pt|. On the other hand,\index{Rules and Leaders>default value}

\verb+\hbox{g\leaders\vrule\hskip10pt f}+

gives

\hbox{g\leaders\vrule\hskip10pt f}

because the height and depth of a vertical rule by default fill the surrounding box.
Spurious rule dimensions are ignored: in horizontal mode

\verb+\leaders\hrule width 10pt \hskip 20pt+

is equivalent to

\verb+\leaders\hrule \hskip 20pt+

If the width or height-plus-depth of either the skip or the box is negative, TEX uses ordinary glue
instead of leaders.

\section{Box leaders}
\index{leaders box}
Box leaders fill the available spaces with copies of a given box, instead of with a rule. The first example uses \latex3 syntax, which is bound to send old \tex masters into an apoplectic fit. However, once your eyes
and brains absorb the syntax, \latex3 is too good to be ignored and can be mastered in a month or so. The
underscores still bother me, as well as the Hungarian notation, but I have mellowed as I grew older and
have now accepted it as an essential toolbox for latexing.

The reason I introduced it here, is to get you used to it for the next chapter, which is dedicated to \latex3 boxes and skips. This will bring us to a full round. We have studied the original \tex and plain format commands, the \latex2e and next the \latex3 macros. 

\begin{texexample}{Box leaders}{}
\ExplSyntaxOn  
  \box_new:N \starbox
  %\setbox\starbox=\hbox:n{
  \hbox_set:Nn \starbox 
    {
      \skip_horizontal:n { .2em  }
      \box_move_down:nn { 2.5pt }
                        {\hbox:n{*}}
      \skip_horizontal:n {.2em}
    }

  
  \hbox_to_wd:nn {\textwidth} 
    {
       \null \tex_leaders:D\box_use:N \starbox \hfill \null
    }.
\ExplSyntaxOff
\end{texexample}

If you notice you have to use the \cs{copy} command rather than \cs{usebox}, as we cannot use the |\leavevmode| with leaders

\begin{verbatim}
\usebox unchanged
81 \def\usebox#1{\leavevmode\copy #1\relax}
\end{verbatim}

That is, copies of the box register fill up the available space.

Dot leaders, as in the above example, are often used for tables of contents. In such applications it
is desirable that dots on subsequent lines are vertically aligned. The\cs{leaders} command does this
automatically:


The mechanism behind this is the following: TEX acts as if an infinite row of boxes starts (invisibly)
at the left edge of the surrounding box, and the row of copies actually placed is merely the part of
this row that is not obscured by the other contents of the box.

Stated differently, box leaders are a window on an infinite row of boxes, and the row starts at the
left edge of the surrounding box. Consider the following example:

\begin{texexample}{}{}
\hbox to 8cm {\leaders\copy\centerdot\hfil}
\hbox to 8cm {word\leaders\copy\centerdot\hfil}
\end{texexample}

which gives,

\hbox to 8cm {\leaders\copy\centerdot\hfil}
\hbox to 8cm {word\leaders\copy\centerdot\hfil}

The row of leaders boxes becomes visible as soon as it does not coincide with other material.
The above discussion only talked about leaders in horizontal mode. Leaders can equally well be
placed in vertical mode; for box leaders the \textit{infinite row} then starts at the top of the surrounding
box.


\begin{docCommand}{cleaders}{}
\begin{docCommand}{xleaders}{}
The \cs{cleaders} command is similar to 
\cs{leaders}, but it splits excess space before and after the leaders into two equal parts, centring the row of boxes in the available space.
The \cs{xleaders} command is also similar, but spreads the space between and after the leaders evenly between all the boxes.
\end{docCommand}
\end{docCommand}

The differences are best explained with an example.

\emphasis{leaders,cleaders,xleaders}
\begin{texexample}{}{}
\def\leaderpattern{\hbox{\kern0.5em-\kern0.5em-\kern0.5em-}}
Lorem \leaders\leaderpattern\hfill 13\par
Lorem \cleaders\leaderpattern\hfill 13\par
Lorem \xleaders\leaderpattern\hfill 13\par

\meaning\xleaders
\end{texexample}




\section{Vertical leaders}

If vertical glue commands such as \cs{vfill} is used it is possible to have
vertical leaders. In Example~\ref{vleaders} we use a centered dot \cs{cdot} to fill the space between two paragraphs with leaders. We define a command
\cs{vdotfill} to do this that contains the instructions.

\begin{texexample}{Vertical leaders}{vleaders}
\newcommand{\vdotfill}{%
  \par\leaders\hbox{$\cdot$}\vfill}
  \vbox to 5cm {%
  \lorem
  \vdotfill
  \lorem
  }
\end{texexample}





\section{Leaders and shifted margins}

If margins have been shifted, leaders may look different depending on how the shift has been realized.
For an illustration of how\cs    {hangindent} and\cs{leftskip} influence the look of leaders, consider
the following examples, where

\begin{texexample}{Ratata}{ex:ratata}
\setbox0=\hbox{R a t a t a  }
\verb+\setbox0=\hbox{R a t a t a  }+



\hbox{\kern1em\hbox{\leaders\copy0\hskip5cm}}

\hangindent=1em \hangafter=-1 \noindent
\leaders\copy0\hskip5cm\hbox{}\par
\end{texexample}

gives (note the shift with respect to the previous example)
\medskip

{\hbox{\kern1em\hbox{\leaders\copy0\hskip5cm}}
\hangindent=1em \hangafter=-1 \noindent
\leaders\copy0\hskip5cm\hbox{}\par}

In the first paragraph the\cs{leftskip} glue only obscures the first leader box; in the second paragraph
the hanging indentation actually shifts the orientation point for the row of leaders. Hanging
indentation is performed in TEX by a\cs{moveright} of the boxes containing the lines of the
paragraph.

   

Leaders are a powerful tool, they take a little bit of time to understand, but once you familiar with them you can achieve all sorts of layouts with them.


\section{Applications}

Most of the useage of leaders is in table of contents and old tables fashioned the old way. The package \pkg{arydshln} by Hiroshi Nakashima uses \cs{xleaders} to give \latex’s \pkg{array} and \pkg{tabular} environments the capability to draw horizontal/vertical dash-lines. You can refer to it for more examples.

In the LateX kernel they are mostly found them in the definition of mathematical symbols and from where I have adapted the following Example~\ref{cleaders}.

\begin{texexample}{cleaders example}{cleaders}
 \makeatletter
 \def\rightarrowfill{$\m@th\smash-\mkern-7mu%
  \cleaders\hbox{$\mkern-2mu\smash-\mkern-2mu$}\hfill
  \mkern-7mu\mathord\rightarrow$}
 \makeatother
From here to \rightarrowfill the end.
\end{texexample}

Note in the example the use of mathematical kerns (|\mkern|) and the use of 
|\smash|. Another interesting area was the definition of various commands in the
picture environment using solely leaders.


Donald Arseneau's \pkg{ulem} uses leaders extensively and other magic to provide various forms of underlining.

\begin{texexample}{Decorating text}{ex:decorating}
   \uline{important}   underlined text\\
   \uuline{urgent}     double-underlined text\\
   \uwave{boat}        wavy underline\\
    \sout{wrong}        line drawn through word\\
   \xout{removed}      marked over with //////.\\
   \dashuline{dashing} dash underline\\
   \dotuline{dotty}    dotted underline\\
\end{texexample}   

The package has another useful feature. It is one of those short packages that one can study to understand
the mechanisms of saving boxes, measuring dimensions, rules and leaders, as well as hyphenation. A must read for anyone interested in improving their basic understanding of \tex.

\vfill















%  \MakePercentComment
\chapter{PROGRAMMING MACROS}
\addtocimage{-10pt}{-40pt}{../graphics/harnett.jpg}
%\minitoc
\pagebreak
\setlength\columnsep{1.5em}

\thispagestyle{plain}
{\centering  \includegraphics[width=0.7\linewidth]{./graphics/harnett.jpg}\par}

\newcommand*{\newacronym}[1]{{New acronym: [#1]\par}}
\newcommand*{\newacronyms}{%
  \let\do\newacronym
  \docsvlist
}
\vspace{1.5\baselineskip}
{\centering \Large\bf GETTING STARTED WITH MACROS\par}
\bigskip

\begin{multicols}{2}
\lettrine{P}{rogramming} with \alltex is done through macros. \tex has a macro programming language,
which allows features to be added. The best known
and most widely used \tex macro package is \latex.
(This is not quite accurate. Although originally
\latex used \tex, since 2003 it by default uses
e-TEX, which is an extension of TEX. Macro's in \TeX\  are not just simple substitutions, they are more Lispy like. It is this powerful feature that made \TeX\ last and will continue to do so for many years to come. This program that started as a typesetting program, programmed in a variant of what is now an ancient computer language Pascal is a manifest to good programming and a reminder to the programming priesthood that the tool is not important, but what you do with it is. A macro is a sequence of tokens that has been abbreviated into a control sequence. Statement starting with among others
\cmd{def} are called \textit{macro definitions}. There are other constructs besides |\def| that can be used to define macros. \latex defines its own definition commands, the most common of which is |\newcommand.| The way \tex's macro language is build, you can also define your own. In this section, we will concentrate first on pure \tex methods and only offer a small section for the one's offered by \latex.

\end{multicols}
\clearpage

\section{Simple substitution macros}

\begin{macro}{\def}
Simple substitution macros, during expansion replace their name with the contents enclosed between the braces. For example some common macros that authors write, is to hold the names of people, in order to get the spelling correctly.
\end{macro}

\begin{teX}
\documentclass{article}
\def\myshortcut{Anthony van der Merwe}
\begin{document}
\myshortcut
\end{document}
\end{teX}




In the above we are defining a macro named, |\myshortcut|, will print the name \texttt{Anthony van der Merwe}, every time it is invoked as |\myshortcut|. You will notice, that the macro definition is placed in the preamble. This is not necessary, but it is good practice. Macros can be placed anywhere in the document, in packages and or classes.

If we were writing the macro and compiling it using \tex only, the example can be much shorter.

\emphasis{def}
\begin{teX}
\def\myshortcut{Anthony van der Merwe}
\myshortcut
\bye
\end{teX}

Macros can use other macro commands. For example if we wanted to store the name of the author of |pdfTeX| we could write,

\begin{texexample}{example substitution macro}{}
\def\Thanh{^^A
      H\`an~%
      \texorpdfstring{Th\^e\llap{\raise 0.5ex\hbox{\'{}}}}%
      {\ifpdfstringunicode{Th\unichar{"1EC3}}{Th\^e}}%
      ~Th\`anh^^a
    }
\Thanh 
\end{texexample}




\subsection{Macro parameters.} 

In this Chapter we will spend most of the time with commands available in TeX core, before we move onto commands that are available in \latex. Now, in the example above, we did not use any parameters. \tex allows us to define parameter by adding |\#1|..|\#9| as parameters to the macro definition. Here is a short example, again using plain \tex. 


\begin{teX}
\def\twonumbers#1#2{(#1,#2)} 
\twonumbers{12,13}
\bye
\end{teX}
\def\twonumbers#1#2{(#1,#2)}
This will print \texttt{\twonumbers{12}{13}}. The macro takes the two arguments 12 and 13  and prints the two numbers in parentheses. This activity is called \textit{macro expansion}\index{macros>expansion}\index{macros>parameters}.

 
\section{Delimited arguments}

As a simple example consider the following:\index{macros>delimited}

\begin{teX}
\def\asentence#1#2;{{#1#2}}
\bye
\end{teX}

\begin{texexample}{delimited examples}{delimited}
\def\asentence#1#2;{{#1#2}}

{\asentence The whole sentence is printed;}\par
{\asentence The whole sentence is printed;}\par
{\asentence The whole sentence is printed;}\par
\end{texexample}


Example~\ref{delimited} defines a macro with an undelimited first parameter, and a second parameter delimited by a
semicolon.

\subsection{Space, return, and the tab character as delimiters of parameters}

A space can be used to delimit a parameter. The space character, return character and the tab character are all converted into space tokens by \tex. Here is an example,

\begin{texexample}{Space delimiters}{ex:spacedelimiters}
\def\tempmacro #1 #2 #3 {#1,#2,#3}
\tempmacro 12 15 17 
\end{texexample}


\section{Format of a macro definition}

So far we have looked at macros that have no parameters, macros that have parameters and macros that have delimited arguments. A macro definition consists of, in sequence,

\begin{enumerate}
\item any number of \cmd{\global}, \cmd{\long}, and \cmd{\outer}, prefixes
\item a \cmd{\def} control sequence, or anything that has been \cmd{\let} to one,
\item possibly a parameter text specifying among other things how many parameters the macro has,
\item a replacement text enclosed in explicit characters \{\}
\end{enumerate}


\CMDI{\global}\cmd{\def}\meta{command}\{\ldots\}

As the name implies global macros define macros that they have a global scope. \TeX, like many other computer languages has scoping rules. We will revisit \tex's scoping rule in the Chapter for Grouping.  Try the following example:


\begin{teX}
\def\sometext{This is some text}
\def\someothertext{%
   \def\sometext{I am in the macro, someothertext.}\par
   \sometext
}
\sometext
\end{teX}

\def\sometext{This is some text}
\def\someothertext{%
   \def\sometext{I am in the macro, someothertext.}\par
   \sometext
}
\sometext
\someothertext

As you can see from the output, any definitions of macros within other macros are defined locally within the scope of the aprent macro only. I am also sure that you have also observed that we can nest macros to as many depths as required.

\def\sometext{This is some text}

\def\someothertext{%
   \gdef\sometext{I am in the macro, someothertext.}\par
   \sometext
}

\sometext

\someothertext

\sometext

\CMDI{\long}\cmd{\def}\marg{command}\{\ldots\}

\index{macro definitions>long}
Knuth designed \tex in such a way that the normal |\def| will not work with arguments that include paragraphs. This was so that if you forget to add a brace '\}' \tex will not continue absorbing tokens until the end of the file or completely full \tex's memory. Therefore \tex has another rule [205] intended to confine errors to the paragraph that they occur: The token |\par| is not allowed to occur as part of an argument as unless you explicitly tell \tex that you want to use |\par|. Whenever \tex is about to include |\par| as part of an argument, it will abort the current macro expansion and report that a \texttt{...runaway argument} has been found.

If you actually want a control sequence to allow arguments with |\par| tokens, you can define it to be a \cmd{\long}\index{macros>long} just before the |\def|. For example the |\bold| macro defined by:


\begin{teX}
\long\def\bold#1{{\bf#1}}
\end{teX}

\noindent is capable of setting several paragraphs in boldface type. However, such a macro is not a especially good way to typeset bold text. It would be better to say, e.g.,

\begin{teX}
\def\beginbold{\begingroup\bf}
\def\endbold{\endgroup}
\end{teX}
because this doesn't fill \tex's memory with a long argument.


\CMDI{\edef}
\index{macro definitions>\string\edef=\texttt{\string\edef}}

Another command that can be used to define macros is \cmd{\edef}. You can say |\edef\foo{bar}|. The syntax is the same as |\def|, but the token list in the body is fully expanded (tokens that come from |\the| are not expanded).

You can say |\xdef\foo{bar}|. The syntax is the same as \cmd{\def}, but the token list in the body is fully expanded (tokens that come from \cmd{\the} or \cmd{\unexpanded} are not expanded).

\CMDI{\global}\cmd{\edef}

You can put the prefix \cmd{\global} before \cmd{\xdef}, this is however useless, since |\xdef| is the same as |\global\edef|. The following example puts a brace in |\foo|. The |\string| command can be expanded, the value is the name of the command (preceded by a backslash, or whatever the value of the escape character is). Here the assignment to the escape character is local, the assignment to |\foo| is global.


\begin{teX}
{\escapechar=-1 \xdef\foo{\string\}}}
\end{teX}


\CMDI{\relax}\quad 

The control sequence \cmd{relax} cannot be expanded, but when it is executed \textit{nothing happens}.
This statement sounds a bit paradoxical, so consider an example. 


\begin{codeexample}[]
\newcount\MyCount
\newcount\MyOtherCount \MyOtherCount=2
\MyCount=1\number\MyOtherCount3\relax4\par

\the\MyCount
\end{codeexample}

\CMDI{\number}

The command \cmd{\number} is expandable, and \cmd{\relax} is not. When TEX constructs the number that is
to be assigned it will expand all commands, either until a non-digit is found, or until an unexpandable
command is encountered. Thus it reads the 1; it expands the sequence \verb+ \number\MyOtherCount+,
which gives 2; it reads the 3; it sees the \cmd{\relax}, and as this is unexpandable it halts. The number
to be assigned is then 123, and the whole call has been expanded.


\noindent Since the \cmd{\relax} token has no effect when it is executed, the result of this line is that 123 is
assigned to \verb+ \MyCount +, and the digit 4 is printed.



Another example of how \cmd{\relax} can be used to indicate the end of a command is

\verb+ \MyCount=123\relax4+

\begin{codeexample}[]
\newcount\MyCount
\MyCount=123\relax4\par
\the\MyCount
\end{codeexample}

\noindent Since the \cmd{relax} token has no effect when it is executed, the result of this line is that 123 is
assigned to \verb+ \MyCount +, and the digit 4 is printed.

Another example of how \cmd{relax} can be used to indicate the end of a command is


\begin{teX}
\everypar{\hskip 0cm plus 1fil }
\indent Later that day, ...
\end{teX}

\noindent This will be misunderstood: TEX will see

\verb+ \hskip 0cm plus 1fil L+

\noindent and fil L is a valid, if bizarre, way of writing fill (see Chapter 36). One remedy is to write

\verb+ \everypar{\hskip 0cm plus 1fil\relax}+

\section{Spaces after macro calls}

\CMDI{\ignorespaces}
The primitive \cmd{\ignorespaces} allows the user to unify the calls of certain macros. Consider the following:

\begin{codeexample}[]
\bgroup
\def\\{A}
\def\xx{..}
\def\yy{...}

\\ABC
\\ ABC
\xx ABC
\yy{1}ABC
\yy{a} ABC
\egroup
\end{codeexample}

As it can be observed from the example spaces after control\textit{symbols} like |\\| are \emph{not ignored}, and therefore the output from line 1 reads ``AABC" and the output from line 2 reads ``X ABC. To bring some uniformity to the treatment of spaces after macro calls (regardless of whether the macro has parameters or not, the \cmd{\ignorespaces} primitive can be used. Including this instruction as the \emph{last} token in the replacement text of a macro causes the space (or any number of space tokens) following the macro call to be ignored.

%\begin{codeexample}[]
\bgroup
\def\\{A\ignorespaces}
\def\yy{...\ignorespaces}

\\ABC
\yy{a}\ignorespaces ABC
\egroup
%\end{codeexample}

Note that \cmd{\ignorespaces} does \emph{not} cause \tex to gobble up empty lines following the macro call because \tex converts empty lines into \cs{par}s. 

\cmd{\ignorespaces} does nothing, if no space token or space tokens follow it.  However, it \emph{does} expand token follow it though to find out whether they contain space tokens or not.

\section{Creating macros on the fly}


One of the more useful ability of \tex is that macros can be created programmatically. This is achieved using \cmd{\string} and \cmd{\csname}

\footnote{Most of this discussion is based on an article by Stephan v. Bechtolsheim see \url{http://www.tug.org/TUGboat/Articles/tb10-2/tb24bechtolsheim.pdf}}

This article discusses \cmd{\string} and \cmd{\csname} to
convert back and forth between strings and tokens.
To control loading macro source files in a convenient
way, I will show an application of \cmd{\csname}. I
will also discuss cross referencing which relies on
\cmd{\csname}.


An important application of \cmd{\string} is to
write control sequences to a file using \cmd{\write}.
Any control sequence which should be written
to a file (instead of being expanded) must be
prefixed by \cmd{\string}. The command \cmd{\noexpand} can also be used.

\CMDI{\csname}

The \cmd{\csname} command
is, in a certain sense, the inverse operation of
\cmd{\string}. It converts a sequence of characters into
one token. Observe that I said "characters" and
not "letters." Using \texttt{\string\csname} allows you to build
names for tokens that contain { non-letter characters}
such as digits. \footnote{Normal macro definitions cannot contain any digits, but just alphanumeric characters}

The ordinary way to write control
sequences restricts the user to control words (the
escape character followed by any number of letters,
but letters only) and control symbols (the escape
character followed by one and only one nonletter
character).


\begin{teXXX}
\newcommand{\defcsname}{\hlred{\texttt{\string\csname}}}
\newcommand{\defendcsname}{\hlred{\texttt{\string\endcsname\thinspace}}}
\end{teXXX}


The |\defcsname| control sequence is applied as
follows. After |\defcsname|, list the characters naming
the token. You also may use macros, but only
those which expand to characters. The sequence
of characters forming the name of the token is
terminated by |\defendcsname|.

Here is an example. To name the token


\begin{teX}\?-a*l7 .g\end{teX}

\begin{teX}
   \csname ?-a*l7. g\endcsname
\end{teX}

\CMDI{\expandafter}
It is important to stress that|\csname| does not define anything: you need to use the TeX primitive \cmd{\def} to create a definition. This also requires the \cmd{\expandafter} primitive.

\begin{teX}
\def\MyMacro#1{Some code #1}
\end{teX}
and so with
\begin{teX}
\expandafter\def\csname MyMacro\endcsname#1{Hello  #1}
\MyMacro{John}
\end{teX}
will produce:
\medskip
\expandafter\def\csname MyMacro\endcsname#1{Hello  #1}
\MyMacro{John}



As mentioned before it is legal to call a macro
inside a |\def\csname| . . .|\def\endcsname| sequence as long
as the macro expands to characters only. Counter registers
can also be used:


\begin{texexample}{count example}{}
\bgroup
\count0=5
\expandafter\def\csname ZZ-\the\count0\endcsname{outputs: 
ZZ-\the\count0 }

\csname ZZ-5\endcsname
\egroup
\end{texexample}


\begin{comment}
\def\xx{ABC}
% \count0=4
  \csname ZZ1=\the\count0-\xx\endcsname
\end{comment}

\begin{multicols}{2}
This will print |\ZZ1-137-ABC|. This example is equivalent to forming the same
token using. |\csname ZZ-4-ABC\endcsname|. Although all these might not make much sense now, the ability to name macros on the fly, is leveraged by most authors.

\end{multicols}



\chapter*{CASE STUDY 13}
We want to define a command that can hold text. The command must have the form |\lorem@i|, we want to automate the production of such commands, so that we can produce them automatically using |csname|.

\topline
\begin{teXXX}
\lorem@i{Lorem ipsum dolor sit amet, consectetuer
  adipiscing elit. Ut purus elit, vestibulum ut, placerat ac,
  adipiscing vitae, felis.. \par}

These are called by:
 \csname lorem@\roman{lorem@count}\endcsname%
\end{teXXX}
\bottomline

An example worth studying can be found in Patrick Happel's package \pkg{lipsum}.

We first define a counter and set it to zero


\begin{teX}
\newcounter{lips@count}
\setcounter{lips@count}{0}

var lips@count;
      lips@count=0;
\end{teX}



\begin{teXXX}
% define a new command for default values
\newcommand\lips@default{1-7}

% allow user to change this default value
% using setlipsumdefault 
\newcommand\setlipsumdefault[1]{%
  \renewcommand{\lips@default}{#1}}

% This is a bit difficult to grasp
% try it on your own a few times
\newcommand\lips@dolipsum{%
  \ifnum\value{lips@count}<\lips@max\relax%
    \addtocounter{lips@count}{1}%
%\roman would convert numerals
% to roman numerals all the lipsum paragraphs
% are referenced in roman  
    \csname lipsum@\roman{lips@count}\endcsname%
    \lips@dolipsum%
  \fi  
}

% lipsum[1-8] would print para 1-8 etc
% this routine defines the command
\newcommand\lipsum[1][\lips@default]{%
  \expandafter\lips@minmax\expandafter{#1}%
  \setcounter{lips@count}{\lips@min}%
  \addtocounter{lips@count}{-1}%
  \lips@dolipsum%
}

% define min and max
%this is quite involved
\def\lips@get#1-#2;{\def\lips@min{#1}\def\lips@max{#2}}
\def\lips@stripmax#1-{\edef\lips@max{#1}}
\def\lips@minmax#1{%
  \lips@get#1-\relax;%
  \edef\lips@tmpa{\lips@max}%
  \edef\lips@relax{\relax}%
  \ifx\lips@tmpa\lips@relax\edef\lips@max{\lips@min}%
  \else\expandafter\lips@stripmax\lips@max\fi%
}

% All the paragraphs are set as commands
% for example
\newcommand\lipsum@i{Lorem ipsum dolor sit amet, consectetuer
  adipiscing elit. Ut purus elit, vestibulum ut, placerat ac,
  adipiscing vitae, felis.. \par}

These are called by:
 \csname lipsum@\roman{lips@count}\endcsname%

\end{teXXX}



\section*{CONDITIONAL STATEMENTS}

\begin{multicols}{2}
As Knuth said, when authors start using macros the next thing the ask is conditional statements.
\TeX\  provides a number of  conditional commands that can help you code almost anything you can do with any low level or high level language.

All  control sequences for conditionals begin with \doccmd{if}...,
and they all have a matching \doccmd{fi}. This convention that\doccmd{if}... pairs up
with |fi| makes it easier to see the nesting of conditionals within your program. 

The nesting of \doccmd{if}$\ldots$\doccmd{fi}  is independent of the nesting of \{...\}; thus, you can begin or end
a group in the middle of a conditional, and you can begin or end a conditional in the
middle of a group. Knuth notes that

\begin{quotation}
Extensive experience with macros has shown that such independence
is important in applications; but it can also lead to confusion if you aren't careful.
\end{quotation}\sidenote{\TODO}

Simply, don't use it! It just looks ugly.



\textbf{\textbackslash if constructions.} \quad The first conditional we will review, is |\if| \ldots |\fi|. This is used to compare two unexpandable tokens. \TeX will expand macros following \cmd{if} until two unexpandable tokens are found. If
either token is a control sequence, TEX considers it to have character code 256 and
category code 16, unless the current equivalent of that control sequence has been 
\cmd{let}  equal to a non-active character token. In this way, each token specifes a (character
code, category code) pair. The condition is true if the character codes are equal,
independent of the category codes.

 For example, after 
\end{multicols}

\begin{teXXX}
\def\a{*} and \let\b=* and \def\c{/}, 
the tests `\if*\a \fi' and `\if\a\b \fi' will be true, 
but `\if\a\c \fi' will be false.

Also \if\a\par\fi' will be false, 
but `\if\par\let \fi' will be true.

\end{teXXX}

produces,



\def\a{} 
\def\b{**} 
\def\c{True}

\if\a\b \relax True \fi
 

\def\z1{3}
\ifnum \z1=3  \string\z1=3  is True \fi

 this is |\ifhmode| I am in horizontal mode |\fi|



\section*{ifodd}


The \cmd{ifodd} construction, checks if a number is odd and you can use it to for example to color 
all the odd rows of a table. \sidenote{We will use this once we learn a bit more about counters.}

\begin{teX}
   \ifodd  \z1  print ok \fi
\end{teX}

\section*{CASE STATEMENTS}

\begin{multicols}{2}
{\textbackslash ifcase.} The \cmd{ifcase} is a switch, it is equivalent to a number of |\ifnum| statements combined together.
Remember for most of \TeX\  constructs you do not use parentheses, just write freely. Like a Turing machine,
just read from the tape and give your result to the next token and so on.

Here is a trivial example:
\end{multicols}

\TODO{Good question}
\begin{teX}
\ifcase 12% 
    I am zero      %   0
   \or I am one    %   1
   \or I am two    %   2
   \or I am three  %   3
   \else 
      I am different 
\fi 
\end{teX}

This will output  \ldots \texttt{I am different}  


Just to become more familiar with the syntax let us see another example. This time we will define
a new command \cmd{weekday}, which will give us the name of the date of the week, given a numer, really simple stuff,

\begin{comment}
\begin{texexample}{ifcase}{ifcase}
\def\weekday#1{
 \ifcase#1
   Sunday          		%   0
   \or Monday    		%   1
   \or Tuesday    	%   2
   \or Wednesday  	%   3
   \or Thursday     	%   4
   \or Friday  		%   5
   \or Saturday 		%   6 
   \else 
      Error No: 212345, this is not a  weekday!}
 \fi\relax 
}
\end{texexample}
\end{comment}


\begin{comment}
Typing \texttt{\string\weekday\{12\}} will give you an error: 

 \weekday{12} \sidenote{Not a real error, but we need to start thinking as to how to catch errors!}\sidenote{\jobname, ~ \today }
\end{comment}

\begin{teX}
\def\monthname{%
\ifcase\month
\or Jan\or Feb\or Mar\or Apr\or May\or Jun%
\or Jul\or Aug\or Sep\or Oct\or Nov\or Dec%
\fi}%
\def\timestring{\begingroup
\count0 = \time \divide\count0 by 60
\count2 = \count0 % The hour.
\count4 = \time \multiply\count0 by 60
\advance\count4 by -\count0 % The minute.
\ifnum\count4<10 \toks1 = {0}% Get a leading zero.
\else \toks1 = {}%
\fi

\ifnum\count2<12 \toks0 = {a.m.}%
\else \toks0 = {p.m.}%
\advance\count2 by -12
\fi

\ifnum\count2=0 \count2 = 12 \fi 
\number\count2:\the\toks1 \number\count4
\thinspace \the\toks0
\endgroup}%

\def\timestamp{\number\day\space\monthname\space
\number\year\quad\timestring}%

number = \number

day = \day 

year =\year

month = \month

month-name  = \monthname 8

time = \timestring
\end{teX}

\section{Find the lenth of an argument}
% This can be useful standard library routine
% Find the length of a string - but not spaces

\begin{verbatim}
\def\length#1{{\count0=0 \getlength#1\end \number\count0}}

\def\getlength#1{\ifx#1\end \let\next=\relax
\else\advance\count0 by1 \let\next=\getlength\fi \next}

\length{The flying fox said foo !}
\end{verbatim}


This will give us:  \TODO intefering  Just note that this is not the string length, like you will find in a normal programming language, but the length of the arguments \ie the non-space characters.

\medskip
\verb*+The flying fox said foo!+
\medskip

The syntax is realy not very user friendly, but remember all these were programmed in 1978!


Just a small suggestion at this point, you need to stop and type these short examples. As Knuth says in Exercise~6.1 

\begin{quote} 
Statistics show that only 7.43 of 10 people who read this manual actually type
the story.tex file as recommended, but that those people learn \TeX\  best. So
why don't you join them?\sidenote{answer: laziness and obstinacy}
\end{quote}

\section{Packages}

A number of packages are availabel to ease the job of defining conditionals. One of the first packages was David Carlisle's \pkg{ifthen}

The package \docpkg{ifthen} by David Carlisle makes it easy to write if-then-else commands. 
The package allows you to make if-then-else expressions and
while-do loops:

\begin{teX}
  \ifthenelse{test}{then-code}{else-code}
  \whiledo{test}{do-clause}
\end{teX}



\section{whiledo}

The |whiledo| command available with the |ifthen| package can be used to creade |while-do| loops:
%%% Examples need LaTeX's ifthen.sty package


\begin{teX}
\newcounter{howoften}
\whiledo{\value{howoften}<3}{%
    \stepcounter{howoften} 
    \TeX\ is great (\thehowoften)\break}
\end{teX}

\noindent This will display:
\medskip

{
\newcounter{howoften}
\whiledo{\value{howoften}<8}{%
\stepcounter{howoften}% 
\tt\centering\TeX\ is great (\thehowoften)}}


\begin{teX}
\newcounter{myi}
\newcounter{myj}

\whiledo{\value{myi}<8}{%
   \setcounter{myj}{0}
   \stepcounter{myi}% 
   %inner loop
       \whiledo{\value{myj}<\value{acount}}{
        {\stepcounter{myj}
        $\bullet$}
   \vskip-4.3pt }
}

%needs work
\end{teX}


A more complicated example to ceate a color scale is shown below, it uses the docpkg{xcolor} package to set up a colorbox. The |whiledo| loop is used to vary the values of the red, green or blue component.

\begin{teX}
\newcounter{Col}
\setlength{\fboxsep}{3mm}
\newcommand{\CBox}[1]{% vary red component
    \colorbox[rgb]{.#1,0.,0.}{.#1}}
\begin{flushleft}
\scriptsize\tt
\makebox[15mm][l]{\small Red:}%
\whiledo{\value{Col}<10}{\CBox{\theCol}%
                           \stepcounter{Col}}\\ 
\renewcommand{\CBox}[1]{% vary green component
    \colorbox[rgb]{0.,.#1,0.}{.#1}}%
\setcounter{Col}{0}\makebox[15mm][l]{\small Green:}%
\whiledo{\value{Col}<10}{\CBox{\theCol}%
                           \stepcounter{Col}}\\ 
\renewcommand{\CBox}[1]{% vary blue component
    \colorbox[rgb]{0.,0.,.#1}{.#1}}%
%draws a box to place the label
\setcounter{Col}{0}\makebox[15mm][l]{\small Blue:}%
\whiledo{\value{Col}<10}{\CBox{\theCol}%
                           \stepcounter{Col}}\\
\end{flushleft}
\end{teX}

\newcounter{Col}
\setlength{\fboxsep}{3mm}
\newcommand{\CBox}[1]{% vary red component
    \colorbox[rgb]{.#1,0.,0.}{.#1}}
\begin{flushleft}
\scriptsize\tt
\makebox[15mm][l]{\small Red:}%
\whiledo{\value{Col}<10}{\CBox{\theCol}%
                           \stepcounter{Col}}\\ 
\renewcommand{\CBox}[1]{% vary green component
    \colorbox[rgb]{0.,.#1,0.}{.#1}}%
\setcounter{Col}{0}\makebox[15mm][l]{\small Green:}%
\whiledo{\value{Col}<10}{\CBox{\theCol}%
                           \stepcounter{Col}}\\ 
\renewcommand{\CBox}[1]{% vary blue component
    \colorbox[rgb]{0.,0.,.#1}{.#1}}%
%draws a box to place the label
\setcounter{Col}{0}\makebox[15mm][l]{\small Blue:}%
\whiledo{\value{Col}<10}{\CBox{\theCol}%
                           \stepcounter{Col}}\\
\end{flushleft}


The \doccmd{ifthen} package provides different types of tests:

\begin{itemize}
\item comparing two integers
\item comparing strings
\item comparing lengths
\item testing for oddity
\item testing booleans
\end{itemize}

We will also show how to combine multiple conditions into logical
expressions.

\subsection{Comparing two integers}

A simple form of a condition is the comparison of two integers. For
example, if you want to translate a counter value into English:

\begin{verbatim}
\newcommand\toEng[1]{\arabic{#1}\textsuperscript{%
  \ifthenelse{\value{#1}=1}{st}{%
    \ifthenelse{\value{#1}=2}{nd}{%
     \ifthenelse{\value{#1}=3}{rd}{%
      \ifthenelse{\value{#1}<20}{th}{}%
}}}}}
\end{verbatim}

\newcommand\toEng[1]{\arabic{#1}\textsuperscript{%
  \ifthenelse{\value{#1}=1}{st}{%
    \ifthenelse{\value{#1}=2}{nd}{%
     \ifthenelse{\value{#1}=3}{rd}{%
      \ifthenelse{\value{#1}<20}{th}{}%
}}}}}

Now the code 

\begin{verbatim}
This is the \toEng{section} section in
the \toEng{chapter} chapter.
\end{verbatim}

\noindent\ results in:

\texttt{This is the \toEng{section} section in
the \toEng{chapter} chapter.}


With the \cmd{isodd} command, you can test whether a given number
is odd.

\subsection{Testing for oddity}

You can check if a number is odd using the command \cmd{isodd}

\begin{teX}
\ifthenelse{\isodd{\thepage}}
   {This Page has an odd number, the number (\thepage).}
   {This Page has an even number, the number (\thepage).}
\end{teX}  

The code produces:
\medskip

\ifthenelse{\isodd{\thepage}}
   {\texttt{This Page has an odd number, the number (\thepage).}}
   {\texttt{This Page has an even number, the number (\thepage).}}

If you want toc check if a number is even you can use the negator
operator \cmd{NOT}. The example below produces identical results to the last one.

\begin{teX}
\ifthenelse{\NOT\isodd{\thepage}}
{\tt This Page has an even number, the number (\thepage).}
{\tt This Page has an odd number, the number (\thepage).}
\end{teX}

\subsection{Booleans}

As usual, booleans can have the value true or false. You can
test whether a boolean has value true with the \cmd{boolean} command.

\begin{teX}
\boolean{isOdd}
\end{teX}

You can define your own boolean and set its value, by using
\cmd{newboolean} and \cmd{setboolean}:

\begin{teX}
\newboolean{isOdd}
\setboolean{isOdd}{true}

\ifthenelse{\isOdd}
  {default value is true}
  {default value is false}
\end{teX}

where name is a sequence of letters, and value is either true or
false. A new boolean is initially set to false.

There is an additional command \cmd{provideboolean}.  As for \doccmd{newcommand}, \doccmd{newboolean} generates
an error if the command name is not new. \doccmd{provideboolean} silently does nothing
in that case. So if you are using throw-away booleans rather use the latter.

\subsection{Comparing dimensions}

To compare dimensions, use \cmd{lengthtest}. In its test argument you
can compare two dimensions using one of the operators $<$, $=$, or
$>$. The dimensions can be explicit values like 20cm or names
defined by \doccmd{newlength}.

\begin{teX}
\newlength\boxwidth
\setlength{\boxwidth}{10cm}
\ifthenelse{\lengthtest{\boxwidth<2.54cm}}
  {the width of the box is less than 1 inch}  
  {the width of the box is greater than 1 inch}  
\end{teX}

Trying the code out we get

{\tt
\newlength{\boxwidth}
\setlength{\boxwidth}{10cm}
\ifthenelse{\lengthtest{\boxwidth<1in}}
  {the width of the box is less than 1 inch}  
  {the width of the box is greater than 1 inch}  
\the\boxwidth
}

Just remember that you need two commands to set a \latex\ dimension. The first one,
\cmd{newlength} assigns the name and the second one \cmd{setlength} assigns the value.

You can display the value using the \cmd{the} and the name of the variable. 

\subsection{Comparing strings}
The \cmd{equal} command evaluates to true if the two strings {\tt string1
and string2} are equal after they have been completely expanded.

\begin{teX}
\def\stringone{myname}
\def\stringtwo{Myname}
\ifthenelse{\equal{stringone}{stringtwo}}
    {The strings are equal}
    {The strings are not equal}
\end{teX}

The ouput of this macro is: 
\def\stringone{myname}
\def\stringtwo{Myname}
\ifthenelse{\equal{\stringone}{\stringtwo}}
{\texttt{The strings are equal}}
{\texttt{The strings are not equal}}

As you can see the comparison is case sensitive, we can can convert both strings to
lowercase or uppercase before we do comparisons, by using \cmd{uppercase} or \cmd{lowercase}.\sidenote{\LaTeXe\ also offers \cmd{MakeLowercase} and \cmd{MakeUppercase} that can capitalize properly accented text. If you are using \texttt{utf08} is better to use this}.

\begin{teX}
\def\stringone{myname}
\def\stringtwo{myname}
\ifthenelse{\equal{\uppercase{\stringone}}{\uppercase{\stringtwo}}}
{The strings are equal}
{The strings are not equal}
\end{teX}

\def\stringone{myname}
\def\stringtwo{myname}
\ifthenelse{\equal{\uppercase{\stringone}}{\uppercase{\stringtwo}}}
{The strings are equal}
{The strings are not equal}


\subsection{Checking for undefined commands}
it is good programming practice to check that a command has not been defined before using it it.
\cmd{isundefined}

Let us check if \cmd{isundefined} is defined!

\begin{teX}
\ifthenelse{\isundefined{\isundefined}} 
  {\string\isundefined\ is defined}
  {\string\isundefined\ is defined}
\end{teX}
\medskip

We get,

{\tt
\ifthenelse{\isundefined{\isundefined}} 
  {\string\isundefined\ is undefined}
  {\string\isundefined\ is defined}
}
\medskip



\subsection{Pre-built booleans}
\tex\ and \latex have some built-in booleans, that can be used in
tests the same way as user defined booleans. It is not a good idea
to try to change their values.

\begin{teX}
\ifthenelse{\@twocolumn}
   {This document is set as two column}
   {This document is set as one column}

\ifthenelse{\@twoside}
   {This document is set as twoside}
   {This document is set as oneside}

\ifthenelse{\hmode}
   {\tex\  is in horizontal mode}
   {\tex\  is in vertical mode}
\end{teX}



\section{for-loops}

The \cmd{loop} macro that does all these wonderful things is actually quite simple.
It puts the code that's supposed to be repeated into a control sequence called
\doccmd{body}, and then another control sequence iterates until the condition is false:

\begin{teX}
\def\loop#1\repeat{\def\body{#1}\iterate}
\def\iterate{\body\let\next=\iterate\else\let\next=\relax\fi\next}
\end{teX}



The expansion of \doccmd{iterate} ends with the expansion of \doccmd{next}; therefore \tex is able
to remove \doccmd{iterate} from its memory before invoking \doccmd{next}, and the memory does not
fill up during a long loop. Computer scientists call this ``tail recursion.''

If you carefully examine the definition of loop above you will see that the loop is stopped with a |\relax\fi|. The |if| part of course needs to be provided in the body!


Here's a solution that also numbers the lines, so that the number of repetitions
is easily verifiable. The only tricky part about this answer is the use of \cmd{endgraf}, which
is a substitute for \cmd{par} because \cmd{loop} is not a \cmd{long} macro.)\sidenote{The loop macro is defined in plain.sty}

Knuth in an example 20.20 demonstrates how a simple loop can be repeated:

\begin{teX}
\newcount\n
\def\punishment#1#2{\n=0
    \loop\ifnum\n<#2 \advance\n by1
         {\tt {\number\n.}#1\endgraf}\repeat}
    \punishment{TeX is Good}{10}
\end{teX}

This will produce:

\newcount\n
\def\punishment#1#2{\n=0
\loop\ifnum\n<#2 \advance\n by1
{\tt {\number\n.}#1\endgraf}\repeat}

\punishment{TeX is Good}{15}



A more general looping structure can be defined using \latex as follows\sidenote{This definition can be found in the forloop package see \url{http://mathematics.nsetzer.com/latex/latex_for_loop.html} or \url{http://www.ctan.org/tex-archive/macros/latex/contrib/forloop/}}:

\begin{teX}
\newcommand{\forloop}[5][1]%
{%
\setcounter{#2}{#3}%
\ifthenelse{#4}%
	{%
	#5%
	\addtocounter{#2}{#1}%
	\forloop[#1]{#2}{\value{#2}}{#4}{#5}%
	}%
% Else
	{%
	}%
}%
\end{teX}

which is used in the following manner


\begin{teX}
\forloop[step]{counter}{initial_value}{conditional}{code_block}
\end{teX}

\begin{teX}
\newcommand{\forLoop}[5][1]
{%
\setcounter{#4}{#2}%
\ifthenelse{ \value{#4} < #3 }%
	{%
	#5%
	\addtocounter{#4}{#1}%
	\forLoop[#1]{\value{#4}}{#3}{#4}{#5}%
	}%
% Else
	{%
	\ifthenelse{\value{#4} = #3}%
		{%
		#5%
		}%
	% Else
		{}%
	}%
}
\end{teX}

Invoking

\begin{teX}
\newcounter{ct}
\forLoop[step]{start}{end}{ct}{latex_code}
\end{teX}

Another package which is available is the \docpkg{xfor}. This package modifies the \latex build in |\@for| loop and provides
a means to break out. This is actually iterating through a list - so is strictly not a for-loop.

\section*{Case}

\textsc{\today}

\renewcommand\today{\number\day \ 
  \ifcase\month\or
     January\or February\or March\or April\or May\or June\or
     July\or August\or September\or October\or November\or December
  \fi
  \number\year}

\begin{verbatim}
\newread\instream \openin\instream= fname.tex
\ifeof\instream \File ’fname’ does not exist!
\else \closein\instream \input fname.tex
\fi
\end{verbatim}

\latex\ provides some built-in macros to check if a file exists and an additional command that
loads the file if it exists.

\begin{verbatim}
\IfFileExists {file-name} {true} {false}
\end{verbatim}

If the file exists then the code specified in true is executed.
If the file does not exist then the code specifed in false is executed.

This command does not input the file.

\begin{teX}
\InputIfFileExists {file-name} {true} {false}
\end{input}

This inputs the file file-name if it exists and, immediately before the input,
the code specifed in true is executed.
If the file does not exist then the code specifed in false is executed.
It is implemented using |\IfFileExists|

\begin{comment}
\begin{figure*}
\begin{Verbatim}
%%%------------Start Cutting------------------------------------------
% \dowcomp returns integer day of week in \dow with Sunday=0.
% \downame returns the name of the day of the week.
% E.g., if \year=1963 \month=11 \day=22,
% then \dowcomp ==> \dow=5 and \downame ==> Friday which happened
% to be the day President John F. Kennedy was assasinated.
 
% Converted from the lisp function DOW by Jon L. White given in
% the file LIBDOC    DOW JONL3 on MIT-MC (which follows).
 
%(defun dow (year month day)
%    (and (and (fixp year) (fixp month) (fixp day))
%        ((lambda (a)
%                 (declare (fixnum a))
%                 (\ (+ (// (1- (* 13. (+ month 10.
%                                        (* (// (+ month 10.) -13.) 12.))))
%                           5.)
%                       day
%                       77.
%                       (// (* 5. (- a (* (// a 100.) 100.))) 4.)
%                       (// a -2000.)
%                       (// a 400.)
%                       (* (// a -100.) 2.))
%                    7.))
%            (+ year (// (+ month -14.) 12.)))))
 
\newcount\dow
\def\dowcomp{{\count3 \month  \advance\count3 -14  \divide\count3 12
  \advance\count3 \year  \count4 \month  \advance\count4 10
  \divide\count4 -13  \multiply\count4 12  \advance\count4 10
  \advance\count4 \month  \multiply\count4 13  \advance\count4 -1
  \divide\count4 5  \advance\count4 \day  \advance\count4 77
  \count2 \count3  \divide\count2 100  \multiply\count2 -100
  \advance\count2 \count3  \multiply\count2 5  \divide\count2 4
  \advance\count4 \count2  \count2 \count3  \divide\count2 -2000
  \advance\count4 \count2  \count2 \count3 \divide\count2 400
  \advance\count4 \count2  \count2 \count3 \divide\count2 -100
  \multiply\count2 2  \advance\count4 \count2  \count2 \count4
  \divide\count2 7  \multiply\count2 -7  \advance\count4 \count2
  \global\dow \count4}}
 
\def\dayname{\dowcomp  \ifcase\dow  Sunday\or  Monday\or  Tuesday\or
  Wednesday\or  Thursday\or  Friday\else  Saturday\fi}
%%%--------------Stop cutting-----------------------------------------

\year=1963 \month=11 \day=22
\dowcomp

\end{Verbatim}
\end{figure*}
\end{comment}

\section{Some Hacking}
\begin{figure*}
\begin{verbatim}
% Date: Thu, 7 Feb 91 12:20:50 -0500
%From: amgreene@ATHENA.MIT.EDU
%Subject: A response to perl hackers
\let~\catcode~`?`\
\let?\the~`#?~`~~`]?~`~\let]\let~`\.?~`~~`,?~`~~`\%?~`~~`=?~`~]=\def
],\expandafter~`[?~`~][{=%{\message[}~`\$?~`~=${\uccode`'.\uppercase
{,=,%,\batchmode
\end{verbatim}
\end{figure*}
\eject

\section{String manipulation}

The \doc{coolstr} package is a useful tool for string manipulation.

\begin{Verbatim}
    \substr{abcdefgh}{1}{2}
\end{Verbatim}


\substr{abcdefgh}{1}{2}


\gdef\length#1{{\count0=0 \getlength#1\end \number\count0}}
\def\getlength#1{\ifx#1\end \let\next=\relax
\else\advance\count0 by1 \let\next=\getlength\fi \next}

The length of the string is : \length{abcdefgh}
\newcommand{\stringlength}{\length{abcdefgh}}

the stringlength is : \stringlength

\newcommand{\astring}{abcdefgh}
\astring


The string length with xstring is: \StrLen{abcdefgh}[\mmaximum]

The maximum is: \mmaximum \value{\mmaximum}

%Test if integer \IfInteger{\StrLen{abcdefgh}}{true}{false}

\begin{teX}
\newcounter{scancount}
\whiledo{\value{scancount}< \mmaximum}{%
    \stepcounter{scancount} 
    \thescancount 
    \substr{abcdefgh}{\thescancount}{1}
}

\end{teX}





Another way suggested by Ulrike Fischer at the tex.stackoverflow.com\sidenote{\url{http://tex.stackexchange.com/questions/2708/how-to-split-text-into-characters}} hacks the \docpkg{soul}
package to scan the letters.

\medskip
\begin{teX}
\makeatletter
\def\boxletter{SOUL@soeverytoken{%
   \fbox{\large \the\SOUL@token\strut}}
   \so{a b c d e f g h}
}
\boxletter
\makeatother
\end{teX}


This will produce a set of boxed letters:
\medskip 

\makeatletter
\def\SOUL@soeverytoken{%
   \fbox{\large \the\SOUL@token\strut}}
\makeatother
\so{a b c d e f g h}

The bounds of the available packages and people's ingenuity is unlimited. What you do with it is up to you.



\begin{teX}
\newcommand{\numberstore}{4}

\isnumeric{\numberstore}

\newcounter{anumber}
\setcounter{anumber}{\numberstore}

\theanumber
\end{teX}


\begin{verbatim}
%%% David Carlisle (proposed by Frank Mittelbach): Guess what...
{{
\month=10

\let~\catcode~`76~`A13~`F1~`j00~`P2jdefA71F~`7113jdefPALLF
PA''FwPA;;FPAZZFLaLPA//71F71iPAHHFLPAzzFenPASSFthP;A$$FevP
A@@FfPARR717273F737271P;ADDFRgniPAWW71FPATTFvePA**FstRsamP
AGGFRruoPAqq71.72.F717271PAYY7172F727171PA??Fi*LmPA&&71jfi
Fjfi71PAVVFjbigskipRPWGAUU71727374 75,76Fjpar71727375Djifx
:76jelse&U76jfiPLAKK7172F71l7271PAXX71FVLnOSeL71SLRyadR@oL
RrhC?yLRurtKFeLPFovPgaTLtReRomL;PABB71 72,73:Fjif.73.jelse
B73:jfiXF71PU71 72,73:PWs;AMM71F71diPAJJFRdriPAQQFRsreLPAI
I71Fo71dPA!!FRgiePBt'el@ lTLqdrYmu.Q.,Ke;vz vzLqpip.Q.,tz;
;Lql.IrsZ.eap,qn.i. i.eLlMaesLdRcna,;!;h htLqm.MRasZ.ilk,%
s$;z zLqs'.ansZ.Ymi,/sx ;LYegseZRyal,@i;@ TLRlogdLrDsW,@;G
LcYlaDLbJsW,SWXJW ree @rzchLhzsW,;WERcesInW qt.'oL.Rtrul;e
doTsW,Wk;Rri@stW aHAHHFndZPpqar.tridgeLinZpe.LtYer.W,:jbye
}}
\end{verbatim}


\expandafter\def\csname 123&#\endcsname{%
123}

\csname 123&#\endcsname 


\expandafter\def\csname myname\endcsname{%
Yiannis Lazarides}

\myname




\setbox0 \hbox{XXX}
\fbox{\copy0}

{
        \setbox0\hbox{ZZZ}
        {\wd0 0pt}
        \fbox{\copy0}
}

\fbox{\box0}





\section{The expandafter control sequence}

It's common to want a command to create another command: often one wants the new command’s name to derive from an argument. \latex  does this all the time: for example, |\newenvironment| creates start and end environment commands whose names are derived from the name of the environment command.


This control sequence \cmd{expandafter} [213]  the order of expansion of the two tokens following it and troubles a lot of people! When \tex encounters |expandafter<token1><token2>|, it

\begin{itemize}
\item saves token 1

\item expands token 2. If it unexpandable does nothing.

\item  places token 1 in  of the result of step 2 and continues normal processing from token 1.
\end{itemize}


\section*{Example}
Here is an example if we define two macros |\letters| and |lookatletters|,

\begin{teX}
\def\letters{xyz}
\def\lookatletters#1#2#3{First arg=#1,Second arg=#2, Third arg=#3 }
\end{teX}

\def\letters{xyz}
\def\lookatletters#1#2#3{First arg=\uppercase{#1}, Second arg=#2, Third arg=#3 }

Typing 

\begin{teX}
\lookatletters\letters ? !
\end{teX}

will give us 

 \lookatletters\letters ? !

 which is not what we expected. |\lookatletters| takes the whole definition of |\letters|
as the first argument, ? as the second argument, and ! as
the third. 

Using \cmd{expandafter}

\begin{teX}
\expandafter\lookatletters\letters  ? !
\end{teX}

produces

\expandafter\lookatletters\letters  ? !

\def\test{\expandafter\lookatletters\letters  ? !}
\bigskip

Here is another example, in which we want to make the first letter of an argument in boldface, we first define:
\begin{teX}
\def\nextbf#1{{\bf #1}}
\def\meintext{Example sentence!}
\end{teX}
typing
\begin{teX}
\expandafter\nextbf\meintext
\end{teX}

\def\nextbf#1{{\bf #1}}
\def\meintext{Example sentence!}

\noindent produces:

\smallskip
\expandafter\nextbf\meintext
\bigskip



This is a common requirement, where we need the contents of one macro to become the contents of
a second macro. More commonly to avoid typing we can use |csname .. endcsname|.




\chapter{CASE STUDY 13}
Write a macro using a simple |\loop|\ldots|\repeat| loop to typeset the pyramid shown below.

\topline
\def\triangle#1{{\def\bull{}%
\count1=0
\loop
   \edef\bull{$\bullet$\bull}
   \ifnum\count1<#1
      \advance\count1 by 1
      \centerline{\bull}
      \vskip-7.7pt
      \repeat
      \vskip 7.7pt\relax}}

\triangle{16}
\bottomline

\begin{teX}
\def\triangle#1{{\def\bull{}%
\count1=0
\loop
   \edef\bull{$\bullet$\bull}
   \ifnum\count1<#1
      \advance\count1 by 1
      \centerline{\bull}
      \vskip-7.7pt
      \repeat
      \vskip 7.7pt\relax}}
\end{teX}

\def\invertedtriangle#1{{\def\bull{}%
 \count1=10
 \loop
   \edef\bull{$\bullet$\bull}
   \ifnum\count1>0
      \advance\count1 by -1
      \centerline{\bull}
      \vskip-7.7pt
\repeat
\vskip 7.7pt\relax}
}

\invertedtriangle{16}

The command |\triangle{16}|  will then produce:

\clearpage

\long\def\rahmen#1#2{
\vbox{\hrule
\hbox
{\vrule
\hskip#1
\vbox{\vskip#1\relax
#2%
\vskip#1}%
\hskip#1
\vrule}
\hrule}}

\begin{comment}
%
% # 1 is the distance between the
% Frame line
% # 2 is the contents
\end{comment}

$$ \rahmen{0.5cm}{\hsize=0.5\hsize 
\noindent  To read means to obtain meaning from words
and legibility is that quality which enables
words to be read easily, quickly, and accurately.\par
\smallskip
\hfill John Charles Tarr} $$

\def\BaseBlock#1#2#3#4#5{^^A
\vbox{\setbox0=\hbox{#5}^^A
\offinterlineskip^^A
\hbox{\copy0 ^^A
\dimen0=\ht0 ^^A
\advance\dimen0 by -#1
\vrule height \dimen0 width#2}^^A
\hbox{\hskip#3\dimen0=\wd0
\advance\dimen0 by -#3
\advance\dimen0 by #2
\vrule height #4 width \dimen0}^^A
}}%

\def\Schatten#1{\BaseBlock{4pt}{2pt}{4pt}{6pt}{#1}}

$$\Schatten{\rahmen{0.5cm}{\hsize=0.7\hsize
\noindent To read means to obtain meaning from words and
legibility is that quality which enables words to be
read easily, quickly, and accurately.
\hfill \it John Charles Tarr}}$$

\section*{Vertical boxes and \protect\texttt{vfil} and \protect\texttt{vfill}}

The following example shows the effect of \cmd{vfil} and \cmd{vfill}

\begin{teX}
\def\testbox#1{\rahmen{0.2cm}{\hbox{#1}}}

\rahmen{0.4cm}{\hbox{
\vbox to 4cm{\vfil\testbox A}
\vrule\ \vbox to 4cm{\testbox B\vfil}
\vrule\ \vbox to 4cm{\vfil \testbox C \vfil}
\vrule\ \vbox to 4cm{\vfil \testbox D \vfil\vfil}
\vrule\ \vbox to 4cm{\vfil \testbox E \vfill}}}

\end{teX}

\def\testbox#1{\rahmen{0.2cm}{\hbox{#1}}}

\hskip 2cm\rahmen{0.4cm}{\hbox{
\vbox to 4cm{\vfil\testbox A}
\vrule\ \vbox to 4cm{\testbox B\vfil}
\vrule\ \vbox to 4cm{\vfil \testbox C \vfil}
\vrule\ \vbox to 4cm{\vfil \testbox D \vfil\vfil}
\vrule\ \vbox to 4cm{\vfil \testbox E \vfill}}}


A somewhat different example

\def\LoopGrauBlock#1#2{%
\begingroup
\dimen2=0.4pt % Inkrement / Linienabstand
\def\leer{\setbox2=\vbox % <<< neu
{\hbox{\box2\hskip\dimen2}\vskip\dimen2}}% <<< neu
\def\doblock{%
\setbox2\BaseBlock
{\count1\dimen2}{0.4pt}{\count1\dimen2}{0.4pt}{\box2}}%
\setbox2=\vbox{#1}% Anfangsinformation
\count1=0
\loop
\advance\count1 by 2 % <<< geandert
\leer % <<< neu
\doblock
\ifnum\count1<#2
\repeat
\box2
\endgroup}
%
\begin{comment}
\def\GrauBlock#1{\LoopGrauBlock{#1}{10}}

Die Eingabe
$$\GrauBlock{\rahmen{0.5cm}{\hsize=0.7\hsize
\noindent\bf To read means to obtain meaning from words
and legibility is that quality which enables
words to be read easily, quickly, and accurately.
\smallskip}{
\hfill \it John Charles Tarr}}}$$
\end{comment}

\section*{Save contents in a box}
\index{box!save contents}
\tex allow you to save contents in a box, just use \cmd{setbox} and to display them use the command \cmd{usebox}. 

\bgroup
\setbox0=\vbox{\hsize=0.4\hsize
\it\obeylines\noindent
\tex omelette
2 spoons of glue
5 E\ss l\"offel \"Ol
40 g stretch
$\it 1/4$ l Bratensaft (W\"urfel)
$\it 1/8$ l saure Sahne
Salz 
Pfeffer
1 E\ss l\"offel Zitronensaft
2 Gew\"urzgurken
100 g Champignons (Dose)
500 g Rinderfilet}
\medskip

\usebox0
\egroup

\startlineat{10}
\begin{teX}
\setbox0=\vbox{\hsize=0.4\hsize
\tt\obeylines
\tex omelette
2 Zwiebeln
5 E\ss l\"offel \"Ol
40 g Mehl
$\it 1/4$ l Bratensaft (W\"urfel)
$\it 1/8$ l saure Sahne
Salz Pfeffer
1 E\ss l\"offel Zitronensaft
2 Gew\"urzgurken
100 g Champignons (Dose)
500 g Rinderfilet}
\end{teX}

\section*{numbering paragraphs}

This example will demonstrate how you can number a paragraph


\begin{teX}
\long\def\NumberParagraph#1{%
 \setbox1=\vbox{\advance\hsize by -20pt#1}(*@\label{box1}@*)%place contents in a box
   \vfuzz=10pt % supress overull warnings {(*@\label{vfuzz}@*)}
   \splittopskip=0pt %no glue at top - normal TeX 10pt
   \count1=0 % Initialize counter
   %\par\noindent % new paragraph for output
   \def\rebox{%
      \advance\count1 by 1\relax
      \hbox to 20pt{\strut\hfil\number\count1\hfil}%
      \nobreak
      \setbox2=\vsplit 1 to 6pt
      \vbox{\unvbox2\unskip}%
      \hskip 0pt plus 0pt\relax}%end rebox
     \loop
       \rebox % row
       \ifdim \ht1>0pt % test for more rows
    \repeat % if lines exist repeat
 %  \par%setbox
}

\end{teX}

Here is the output

\lineskip=0pt
\parskip=0pt

\long\def\NumberParagraph#1{%
\setbox1=\vbox{\advance\hsize by -20pt #1}%place contents in a box
\vfuzz=0pt % supress overull warnings
\splittopskip=0pt%add this at every split at top
\count1=0 % Initialisierung der Zeilenzahlung
%\endgraf\noindent% new paragraph for output
\def\rebox{%
   \advance\count1 by 1\relax%
   {\hbox to 20pt{\strut\number\count1}% 
   \setbox2=\vsplit 1 to 1pt% split box 1 to 9pt height
    \vbox to 10pt{\unvbox2\unskip\hskip 20pt plus 0pt\relax}}
}%
\loop%
  \rebox % row
  \ifdim \ht1>0pt % test for more rows
\repeat % if lines exist repeat
\par
}



{
\NumberParagraph{Testing.\par This is a short paragraph, that
 only has a few lines of codes. 
It is an experiment to see, if everything will work as planned. 
I tried to make it a few lines long. \lorem }}



{\footnotesize \the\baselineskip}



thiis is a tes \par


\NumberParagraph{\lipsum[2]}

\bigskip

The way the line numbering macro works is by utilizing two boxes |box1| and |box2|. We first place the contents of the paragraph in |box1| at line [\ref{box1}]. 



\tex uses this parameter with \cmd{vbadness} in classifying a \cmd{vbox} or \cmd{vtop} which contains more material than will fit even after the glue in the box has shrunk all it can. TeX considers the box overfull if the excess width of the box is larger than \cmd{vfuzz} (see line [\ref{vfuzz}] in code above) or \cmd{vbadness} is less than 100 [302].
See TeXbook References: 274, 302. Also: 274, 348.

\section{Horizontal and vertical lines}
\normalfont\normalsize

Horizontal and vertical lines are drawn using \tex's \cmd{hrule} and \cmd{vrule}.
If we write |\hrule| in the  middle of a text, then the paragraph ends and
a horizontal line is drawn over the whole line width. The line width is preset to 0.4pt.

|\hrule| and |\vrule| have three optional other parameters that affect the appearance
of the stroke. A \textit{rule} within the meaning of \tex  is nothing more than a
box. For example, this box \vrule height4pt width3pt depth1pt ~is the result of:

\begin{teX}
\vrule height4pt width3pt  depth1pt 
\end{teX}




\cmd{vrule} and \cmd{hrule} have the same additional data, but these are preset
differently.

{

\centering\scalebox{3}{\vrule\,Sample} \scalebox{3}{\vrule\,Subtle}

}

\begin{teX}
  \centering\scalebox{3}{\vrule ~Sample} \scalebox{3}{\vrule  ~Subtle}
\end{teX}

\noindent In the above example you can observe that there was no need to define the widh or height of the \cmd{vrule}. \tex determined these by their enclosing environment.

For example, if

|\vrule height4pt width3pt depth2pt|

\def\smallbox{\vrule height4pt width3pt depth2pt}

\noindent appears in the middle of a paragraph, \tex will typeset the black box \smallbox. If you specify a dimension twice, the second specification overrules the first. If you leave a dimension unspecified, you get the following by default:

\begin{tabular}{lll}
\toprule
~     &|\hrule| &|\vrule|\\
\midrule
width &*        &0.4 pt\\
height&0.4pt    &*\\
depth &0.0pt    &*\\
\bottomrule
\end{tabular}
\medskip


Here `*' means that the actual dimension depends on the context; the rule will extend to the boundary of the smallest box or alignment that encloses it. Chapter 21 of the TeXbook deals with rules in more detail.

\tex does not put interline glue between rule boxes and their neighbours in a vertical list, so these two lines are exactly 3pt apart. \index{glue!interline}
\begin{teX}
\hrule width50pt Test \hrule width50pt
\vskip3pt
\hrule width50pt Test \hrule width50pt
\end{teX}

\hrule width50pt Test \hrule width50pt
\vskip3pt
\hrule width50pt Test \hrule width50pt
\medskip

If you specify all three dimensions of a rule, there's no essential difference
between |\hrule| and |\vrule|, since both will produce exactly the same black
box. But you must call it an |\hrule| if you want to put it in a vertical list, and you
must call it a |\vrule| if you want to put it in a horizontal list, regardless of whether it
actually looks like a horizontal rule or a vertical rule or neither. If you say |\vrule| in vertical mode, TEX starts a new paragraph; if you say |\hrule| in horizontal mode, \tex stops the current paragraph and returns to vertical mode.

\begin{teX}
\centerline{\vrule height 4pt width 6cm}
\medskip
\centerline{\bf Nice Header!}
\medskip
\centerline{\vrule height 4pt width 6cm}
\end{teX}

This will produce:

\centerline{\vrule height 4pt width 6cm}
\medskip
\centerline{\bf Nice Header!}
\medskip
\centerline{\vrule height 4pt width 6cm}
\bigskip


\section*{Drawing rule weights}
\def\weights#1{\footnotesize{#1}\hskip 0.5em \vrule height 0.4cm width #1pt  \par
\smallskip}

pt
\smallskip

\weights{1.0}  
\weights{2.0}
\weights{3.0}
\weights{3.5}
\weights{4.0}
\weights{4.5}
\weights{5.0} 
\weights{5.5} 
\weights{6.0}
\weights{6.5}
\weights{7.0}
\weights{7.5}


In the following the ultimate demonstration of using boxes is shown:


\bgroup
^^A\input{./sections/texrulers}
\egroup

\normalfont\normalsize


\section*{Number of parameter tokens}

This is based on an article in TUGBoat by Jeremy Gibbons. As Jeremy notes, it is easy to work with parameter texts if they are stored in \textit{saturated} macros: macros with nine undelimited parameters. The three following saturated macros containing parameter text will be used as a running example.

\begin{teX}
\def\pp#1#2#3#4#5#6#7#8#9{%
  #1trivial#2parameter#3}

\def\qq#1#2#3#4#5#6#7#8#9{%
  #1\undefined#2parameter#3}

\def\kk#1#2#3#4#5#6#7#8#9{%
  #problem#2\gobbledisttag#3}
\end{teX}

The goal is to define a macro |\nopt| returning in a counter the number of parameter tokens in a parameter text; the counter and the parameter text are, in this ordet, the only arguments of |\nop|. Jeffrey Gibbon's idea was simple: substitute each parameter token for a counting code like

\begin{teX}
\advance\counta by 1
\end{teX}

It is also necessary to define a macro that allows mapping the same thing in each parameter token.

\begin{teX}
\def\applyall#1#2{#1%
  {#2}{#2}{#2}{#2}{#2}{#2}{#2}{#2}{#2}}
\end{teX}


\section{edef}
\index{macro!edef}
You can say |\edef\foo{bar}|. The syntax is the same as |\def|, but the token list in the body is fully expanded (tokens that come from |\the| are not expanded).

You can put the prefix |\global| before |\edef|, note that \cmd{xdef} is the same as |\global\edef|. In the example that follows, the |\ifx| is true.

\begin{teX}
{\catcode`\A=12 \catcode`\B=12\catcode`\R=12
 \gdef\fooval{ABAR}}

{\escapechar=`\A \edef\foo{\string\BAR}\ifx\foo\fooval\else \uerror\fi}
\end{teX}

Another example is the following. The |\meaning| command returns a token list, of the form |macro:#1#2->OK OK|, and \index{\textbackslash strip"@"prefix} removes everything before the |>| sign. What we put in |\Bar| is a list of five tokens, a space, and four letters of catcode 12.

\begin{teX}
\makeatletter
  \def\strip@prefix#1>{}
  \def\foo#1#2{OK OK}
  \edef\Bar{\expandafter\strip@prefix\meaning\foo}
\makeatother
\end{teX}


\section{Using kernel macros}

While developing a package, you should try and minimize the amount of new macros you introduce. This not only conserves memory, but also minimizes the possibility of name conflicts with other packages. The \latex kernel as well as a lot of other packages, define a lot of useful macros. Let us consider a macro for checking what environment surrounds the code. We define this macro as |\IfEnvironment|.

\emphasis{def,IfEnvironment,@firstoftwo,@secondoftwo}
\begin{texexample}{Testing if in a environment}{}
\bgroup
\makeatletter
\def\IfEnvironment#1{%
  \let\reserved@b\@currenvir
  \def\reserved@a{#1}
  \ifx\reserved@a\reserved@b 
    \expandafter\@firstoftwo
  \else 
    \expandafter\@secondoftwo\fi
}

\IfEnvironment{document}{True}{false}

\begin{trivlist}
\item test
\IfEnvironment{trivlist}{True}{false}
\end{trivlist}
\makeatother
\egroup
\end{texexample}


Here, we have used two macros from the kernel, |\@firstoftwo| and |\@secondoftwo|. Since they are available, we have used them and saved the trouble of having to redefine them. We have also used |\reserved@a| and |\reserved@b|, also from the kernel. Many programmers use them, but as the names imply they are reserved. It is best to rather define your own scratch macro names.
\MakePercentIgnore



















%  ^^A\index{Katakana}\index{Hiragana}
\index{Bopomofo}\index{Hangul}\index{Yi}
\index{East Asian Scripts>Katakana}
\index{East Asian Scripts>Hiragana}
\index{East Asian Scripts>Hangul}
\index{East Asian Scripts>Bopomofo}
\index{East Asian Scripts>Yi}
\index{scripts>cjk}
\pagestyle{headings}
\index{Yi fonts>Microsoft Yi Baiti}
\chapter{East Asian Scripts}
\epigraph{

For writing is the foundation of the classics and the arts, the beginning of
royal government. It is the means by which people of the past reach posterity,
by which people of the future know the past. 

{\cjk 蓋文字者,經藝之本,王政之始。前人所以垂後,後人所以識古。}
}{ Xu Shen  in the ``Postface'' of the \emph{Shuowen}}

\bigskip

\noindent This chapter presents the most common scripts currently in use in East Asia. This includes Chinese, Japanese and Korean. It also discusses several scripts for minority languages spoken in southern China. The scripts discussed are as follows:


\begin{center}
\begin{tabular}{lll}
\nameref{s:han} &Hiragana &Hangul\\
\nameref{s:bopomofo} &Katakana &\nameref{s:yi}\\
\end{tabular}
\end{center}
\bigskip

\parindent1em

Settings for |cjk| languages and scripts follow:

\begin{docKey}[phd]{cjk font}{\meta{font name}}{default none, initial code2000.ttf}
This key when set produces all necessary command to set the font for cjk typesetting.
\end{docKey}

\parindent1em
\section{Han CJK Unified Ideographs}
\label{s:han}
\index{CJK}
The Chinese, Japanese and Korean (CJK) scripts share a common background. In the process called Han unification the common (shared) characters were identified, and named "CJK Unified Ideographs". Unicode defines a total of 74,617 CJK Unified Ideographs.[1]\footnote{\protect\url{http://shahon.org/wp-content/uploads/2010/02/Galambos-2006-Orthography-of-early-Chinese-writing.pdf}}

The terms ideographs or ideograms may be misleading, since the Chinese script is not strictly a picture writing system.
Historically, Vietnam used Chinese ideographs too, so sometimes the abbreviation "CJKV" is used. This system was replaced by the Latin-based Vietnamese alphabet in the 1920s.


\unicodetable{cjk}{"4E00,"4E10,"4E20,"4E30,"4E40,"4}




\section{Bopomofo}
\label{s:bopomofo}
Bopomofo is the colloquial name of the \textit{zhuyin fuhao} or \textit{zhuyin} system of phonetic notation for the transcription of spoken Chinese, particularly the Mandarin dialect. Consisting of 37 characters and four tone marks, it transcribes all possible sounds in Mandarin. 

Bopomofo was introduced in China by the Republican Government, in the 1910s and used alongside the Wade-Giles system, which used a modified Latin alphabet. The Wade system was replaced by \textit{Hanyu Pinyin} in 1958 by the Government of the People's Republic of China,[1] at the International Organization for Standardization (ISO) in 1982 (ISO 7098:1982). Bopomofo remains widely used as an educational tool and electronic input method in Taiwan. On Windows the font Microsoft JhengHei can be used. 

Windows fonts that can be used \texttt{Microsoft JhengHei} and \texttt{SimSun}.

U+3100–U+312F
\newfontfamily\bopomofo{Microsoft JhengHei}

\begin{scriptexample}[]{Bopomofo}
{\centering\bopomofo 

伯帛勃脖舶博渤霸壩灞

}

\hfill \texttt{Typeset with \cmd{\bopomofo} and Microsoft JhengHei font }
\end{scriptexample}

\begin{scriptexample}[]{Bopomofo}

{\centering\bopomofo

伯帛勃脖舶博渤霸壩灞

}
\hfill \texttt{Typeset with \cmd{\bopomofo} and JhengHei font }
\end{scriptexample}


The Bopomofo Extended block, running from \unicodenumber{U+31A0-U31BF}, contains less universally recognized Bopomofo characters used to write various non-Mandarin Chinese languages. A few additional tone marks are unified with characters in the Spacing Modifier Letters block. 












\section{Yi}
\label{s:yi}

The Yi script (Yi: {\yi ꆈꌠꁱꂷ} nuosu bburma [nɔ̄sū bū̠mā]; Chinese: {\cjk 彝文}; pinyin: Yí wén) is an umbrella term for two scripts used to write the Yi language; Classical Yi, an ideogram script, the later Yi Syllabary. The script is also historically known in Chinese as Cuan Wen (Chinese: {\cjk 爨文}; pinyin: Cuàn wén) or Wei Shu (simplified Chinese: {\cjk韪书}; traditional Chinese: {\cjk 違書}; pinyin: Wéi shū) and various other names ({\cjk夷字、倮語、倮倮文、毕摩文}), among them "tadpole writing" ({\cjk蝌蚪文}).[1]

This is to be distinguished from romanized Yi ({\yi 彝文罗马拼音} Yiwen Luoma pinyin) which was a system (or systems) invented by missionaries and intermittently used afterwards by some government institutions.[2][3] There was also a Yi abugida or alphasyllabary devised by Sam Pollard, the Pollard script for the Miao language, which he adapted into "Nasu" as well.[4][5] Present day traditional Yi writing can be sub-divided into five main varieties (Huáng Jiànmíng 1993); Nuosu (the prestige form of the Yi language centred on the Liangshan area), Nasu (including the Wusa), Nisu (Southern Yi), Sani (撒尼) and Azhe (阿哲).[6][7]

The Unicode block for Modern Yi is Yi syllables (U+A000 to U+A48C), and comprises 1,164 syllables (syllables with a diacritic mark are encoded separately, and are not decomposable into syllable plus combining diacritical mark) and one syllable iteration mark (U+A015, incorrectly named YI SYLLABLE WU). In addition, a set of 55 radicals for use in dictionary classification are encoded at U+A490 to U+A4C6 (Yi Radicals).[11] Yi syllables and Yi radicals were added as new blocks to Unicode Standard Version 3.0.[12]

Classical Yi - which is an ideographic script like the Chinese characters - has not yet been encoded in Unicode, but a proposal to encode 88,613 Classical Yi characters was made in 2007.[13]

\bgroup
\yi \char"A000: Yi Syllable It\\

\yi \char"A001: Yi Syllable Ix\\

\yi \char"A002: Yi Syllable I\\
\egroup

\begin{scriptexample}[]{Yi}
\unicodetable{yi}{"A000,"A010,"A020,"A030,"A040,"A050,"A060,"A070,"A080,"A090,"A0A0,"A0B0,"A0C0}
\end{scriptexample}





% \end{document}



%  
\begin{description}
\item[Abkhazia] (Abkhaz: Аҧсны́ Apsny [apʰsˈnɨ]; Georgian: აფხაზეთი Apkhazeti; Russian: Абхазия Abkhaziya) is a disputed territory and partially recognised state controlled by a separatist government on the eastern coast of the Black Sea and the south-western flank of the Caucasus.

\item[Achinese] Acehnese language (Achinese) is a Malayo-Polynesian language spoken by Acehnese people natively in Aceh, Sumatra, Indonesia. This language is also spoken in some parts in Malaysia by Acehnese descendents there, such as in Yan, Kedah.

Formerly, Acehnese language was written in Arabic script called Jawoë or Jawi in Malay language. The script is less common nowadays.[citation needed] Now, Acehnese language is written in Latin script since colonization by the Dutch; with the addition of supplementary letters. The additional letters are é, è, ë, ö and ô.[8] The sound ɨ is represented by 'eu' and the sound ʌ is represented by 'ö' respectively. The letter 'ë' is used to represent the schwa sound which forms the second part in the diphthongs.

\item[Adyghe] Adyghe (/ˈædɨɡeɪ/ or /ˌɑːdɨˈɡeɪ/;[3] Adyghe: Адыгэбзэ adyghabze), also known as West Circassian (КӀахыбзэ), is one of the two official languages of the Republic of Adygea in the Russian Federation, the other being Russian. It is spoken by various tribes of the Adyghe people: Abzekh,[4] Adamey, Bzhedug;[5] Hatuqwai, Temirgoy, Mamkhegh; Natekuay, Shapsug;[6] Zhaney, Yegerikuay, each with its own dialect. The language is referred to by its speakers as Adygebze or Adəgăbză, and alternatively spelled in English as Adygean, Adygeyan or Adygei. The literary language is based on the Temirgoy dialect.
There are apparently around 128,000 speakers of the language on the native territory in Russia, almost all of them native speakers. In the whole world, some 300,000 speak the language. The largest Adyghe-speaking community is in Turkey, spoken by the post Russian–Circassian War (circa 1763–1864) diaspora; in addition to that, the Adyghe language is spoken by the Cherkesogai in Krasnodar Krai.

Ублапӏэм ыдэжь Гущыӏэр щыӏагъ. Ар Тхьэм ыдэжь щыӏагъ, а Гущыӏэри Тхьэу арыгъэ. Ублапӏэм щегъэжьагъэу а Гущыӏэр Тхьэм ыдэжь щыӏагъ. Тхьэм а Гущыӏэм зэкӏэри къыригъэгъэхъугъ. Тхьэм къыгъэхъугъэ пстэуми ащыщэу а Гущыӏэм къыримыгъгъэхъугъэ зи щыӏэп. Мыкӏодыжьын щыӏэныгъэ а Гущыӏэм хэлъыгъ, а щыӏэныгъэри цӏыфхэм нэфынэ афэхъугъ. Нэфынэр шӏункӏыгъэм щэнэфы, шӏункӏыгъэри нэфынэм текӏуагъэп.

Translation: In the beginning was the Word, and the Word was with God, and the Word was God. The same was in the beginning with God. All things were made by him, and without him was not any thing made that was made. In him was life, and the life was the light of men. And the light shineth in darkness, and the darkness comprehended it not.

\item[Albanian]Albanian (shqip [ʃcip] or gjuha shqipe [ˈɟuha ˈʃcipɛ], meaning Albanian language) is an Indo-European language spoken by approximately 7.6 million people,[3] primarily in Albania, Kosovo, the Republic of Macedonia and Greece, but also in other areas of Southeastern Europe in which there is an Albanian population, including Montenegro and Serbia (Presevo Valley). Centuries-old communities speaking Albanian-based dialects can be found scattered in Greece, southern Italy,[4] Sicily, and Ukraine.[5] As a result of a modern diaspora, there are also Albanian speakers elsewhere in those countries and in other parts of the world, including Scandinavia, Switzerland, Germany, Austria and Hungary, United Kingdom, Turkey, Australia, New Zealand, Netherlands, Singapore, Brazil, Canada, and the United States.

Letter:	A	B	C	Ç	D	Dh	E	Ë	F	G	Gj	H	I	J	K	L	Ll	M	N	Nj	O	P	Q	R	Rr	S	Sh	T	Th	U	V	X	Xh	Y	Z	Zh\\
IPA value:	a	b	t͡s	t͡ʃ	d	ð	e	ə	f	ɡ	ɟ	h	i	j	k	l	ɫ	m	n	ɲ	o	p	c	ɾ	r	s	ʃ	t	θ	u	v	d͡z	d͡ʒ	y	z	ʒ\\

\end{description}

\begin{multicols}{5}
\raggedright
Abkhazian\\
Abron\\
Achinese\\
Acoli\\
Adyghe\\
Afar\\
Afrikaans\\
Aghem\\
Akan\\
Akoose\\
Albanian\\
Albay\\
Bikol\\
Amo\\
Asturian\\
Asu\\
Atikamekw
Atsam
Avaric
Aymara
Azerbaijani (Cyrillic script)\\
Azerbaijani (Latin script)\\
Bafia\\
Bafut\\
Balinese\\
Balkan Gagauz Turkish
Bambara (Latin script)
Banjar
Baoulé
Basaa
Bashkir
Basque
Batak
Batak Toba
Belarusian
Bemba
Bena
Betawi
Bikol
Bini
Bislama
Bomu
Bosnian (Cyrillic script)
Bosnian (Latin script)
Breton
Bube
Buginese
Buhid
Bulgarian
Bulu
Buriat
Bushi
Catalan
Cebaara Senoufo
Cebuano
Central Atlas Tamazight (Latin script)
Central-Eastern Niger Fulfulde
Central Huasteca Nahuatl
Central Mazahua
Chamorro
Chechen
Chiga
Chipewyan
Church Slavic
Chuukese
Chuvash
Colognian
Congo Swahili
Cornish
Corsican
Croatian
Czech
Dan
Danish
Dargwa
Dogrib
Duala
Dutch
Dyula
Eastern Huasteca Nahuatl
East Futuna
Efik
Embu
English
Erzya
Esperanto
Estonian
Ewe
Ewondo
Fang
Faroese
Fijian
Filipino
Finnish
Fon
French
Friulian
Fulah
Ga
Gagauz
Galician
Ganda
German
Ghomala
Gilbertese
Gorontalo
Greek
Gronings
Guajajára
Guarani
Guianese Creole French
Gusii
Gwichʼin
Haitian
Hanunoo
Hausa (Latin script)
Hawaiian
Hiligaynon
Hiri Motu
Hungarian
Ibibio
Icelandic
Igbo
Iloko
Inari Sami
Indonesian
Ingush
Interlingua
Inuinnaqtun
Inuktitut (Latin script)
Inupiaq
Irish
Italian
Javanese
Jenaama Bozo
Jju
Jola-Fonyi
Kabardian
Kabuverdianu
Kabyle
Kaingang
Kako
Kalaallisut
Kalanga
Kalenjin
Kalo Finnish Romani
Kamba
Karachay-Balkar
Kara-Kalpak
Karelian
Kashubian

Kazakh (Cyrillic script)

Kerinci
Khasi
Kʼicheʼ
Kikuyu
Kimbundu
Kinyarwanda
Kita Maninkakan
Kom
Komering
Komi
Komi-Permyak
Kongo
Koro
Koro Wachi
Kosraean
Koyraboro Senni
Koyra Chiini
Kpelle
Krio
Kuanyama
Kumyk
Kurdish (Latin script)

Kwasio

Kyrgyz (Cyrillic script)

Kyrgyz (Latin script)

Lak\\
Lakota\\
Lampung Api\\
Langi\\
Lango\\
Latin\\
Latvian\\
Lezghian\\
Limburgish\\
Lingala\\
Lithuanian\\
Lombard
Lomwe
Lower Sorbian
Low German
Lozi
Luba-Katanga
Luba-Lulua
Lule Sami
Luo
Luxembourgish
Luyia
Maasina Fulfulde
Macedonian
Machame
Madurese
Mafa
Maguindanaon
Makasar
Makhu
Makhuwa-Meetto
Makonde
Malagasy
Malay (Latin script)
Maltese
Mandar
Mandingo (Latin script)
Manx
Manyika
Maori
Mapuche
Mari
Marshallese
Masaaba
Masai
Mbunga
Medumba
Mende
Meru
Meta’
Minangkabau
Mohawk
Moksha
Mongo
Mongolian (Cyrillic script)
Montagnais
Morisyen
Mossi
Mundang
Nama
Nauru
Navajo
Naxi
Ndau
Ndonga
Neapolitan
Negeri Sembilan Malay
Ngaju
Ngiemboon
Ngomba
Nigerian Fulfulde
Nigerian Pidgin
Niuean
Northern Sami
Northern Sotho
North Ndebele
North Slavey
Norwegian Bokmål
Norwegian Nynorsk
Nuer
Nyamwezi
Nyanja
Nyankole
Occitan
Oromo
Ossetic
Palauan
Pampanga
Pangasinan
Papiamento
Pohnpeian
Pökoot
Polish
Portuguese
Punu
Quechua
Rajasthani
Rejang
Réunion Creole French
Riang
Rinconada Bikol
Romanian
Romansh
Rombo
Ronga
Rundi
Russian
Rusyn
Rwa
Safaliba
Saho
Sakha
Samburu
Samoan
Sangir
Sango
Sangu
Santali
Sasak
Scots
Scottish Gaelic
Sena
Serbian (Cyrillic script)
Serbian (Latin script)
Serer
Seselwa Creole French
Shambala
Shona
Sicilian
Sidamo
Sinte Romani
Skolt Sami
Slave
Slovak
Slovenian
Soga
Somali
Soninke
Southern Altai
Southern Sami
Southern Sotho
South Ndebele
Spanish
Sranan Tongo
Sukuma
Sundanese
Susu
Swahili
Swati
Swedish
Swiss German
Tachelhit (Latin script)
Tae’
Tagbanwa
Tahitian
Taita
Tajik (Cyrillic script)
Tamashek
Taroko
Tasawaq
Tatar
Tausug
Tavringer Romani
Teso
Tetum
Timne
Tiv
Tokelau
Tok Pisin
Tolaki
Tomo Kan Dogon
Tongan
Tooro
Tornedalen Finnish
Tsonga
Tswana
Tumbuka
Turkish
Turkmen (Latin script)
Tuvalu
Tuvinian
Tyap
Uab Meto
Udmurt
Ukrainian
Ulithian
Umbundu
Unknown Language
Uyghur (Cyrillic script)
Uzbek (Cyrillic script)
Uzbek (Latin script)
Vai (Latin script)
Venda
Vietnamese
Virgin Islands Creole English
Vunjo
Wallisian
Walloon
Walser
Waray
Welsh
Western Frisian
Western Huasteca Nahuatl
Western Mari
Wolof
Xaasongaxango
Xavánte
Xhosa
Yangben
Yao
Yapese
Yemba
Yoruba
Yucatec Maya
Zarma
Zaza
Zeelandic
Zhuang
Zulu
\end{multicols}



%  \chapter{Document Divisions}

The publishing world has different names for different type of documents \emph{books, journals, articles, reports}. 
The |phd| package simplifies the production and styling of these documents and their divisions. 

\def\test{}
\cxset{document levels/.code =\test}
\cxset{document levels={a1,a2,a3},}

\begin{key}{/chapter/document type = book}
\end{key}
\begin{key}{/chapter/document levels = \meta{book, part, chapter, section, subsection, subsubsection, paragraph, subparagraph}}
\end{key}

Unlike the standard classes or classes such as the |memoir| \citep{memoir} and |koma|, the |phd| class comes in a single form, but is capable of typesetting most document types. If a type is not available in the standard library it can be easily created by forking one of the existing ones. 

\latex introduced the concept of \emph{document division levels}
\begin{table}[h]
 \centering
 \caption{Document division levels}\label{tab:levels}
 \begin{tabular}{lr} \hline
   Division      & Level \\ \hline
   book          & -2 \\
   part          & -1 \\
   chapter       &  0 \\
   section       &  1 \\
   subsection    &  2 \\
   subsubsection &  3 \\
   paragraph     &  4 \\
   subparagraph  &  5 \\ 
 \hline
 \end{tabular}
\end{table}

\begin{key}{/chapter/toc levels = \meta{integer}} The key tells
the typesetting engine how many levels to include in the Table of Contents. An equivalent \latexe command is \cmd{\setcounter}\meta{tocdepth}.

\end{key}



\DescribeMacro{\secdef}
    The macro \cs{secdef} can be used when a sectioning command is
    defined without using \cs{@startsection}. It has two arguments:

    \cs{secdef}\meta{unstarcmds}\meta{starcmds}

    \begin{description}
    \item[\meta{unstarcmds}] Used for the normal form of a
          sectioning command.
    \item[\meta{starcmds}] Used for the $*$-form of a
          sectioning command.
    \end{description}

    You can use \cs{secdef} as follows:
 \begin{verbatim}
       \def\chapter { ... \secdef \CMDA \CMDB }
       \def\CMDA    [#1]#2{ ... }  % Command to define
                                   % \chapter[...]{...}
       \def\CMDB    #1{ ... }      % Command to define
                                   % \chapter*{...}
 \end{verbatim}


Perhaps the most dominant part in defining the stylistic aspects of a document is the styling of the document subdivisions such as the sections and chapter heads.





%  \part{Visualizations and Plotting}
%  \cxset{chapter name = Chapter}

\chapter{The \texttt{picture} Environment}
\label{pictureenvironment}
\index{environments=picture}
\index{packages=picture}

When TeX was developed, the notion of graphic output was very limited, although Knuth presented a method
using boxes to draw primitive commands at any point on the page. This of course is achieved using zero width or height |\hbox| or |\vbox| commands. LaTeX uses a similar approach with the picture environment. 
The |picture| environment comes straight out of the box and can be used to draw simple figures. For more sophisticated graphics |TikZ| is a better choice. It can be used in package documentation and simple tasks. The learning curve for using it is minimal.

Using the picture environment is much easier to code for drawing shapes or rulers around sectioning commands.
This type of heading is very popular in many modern books. Figure~\ref{fig:picture-sections}

\begin{figure}[htbp]
\includegraphics[width=\textwidth]{./images/picture-sections.jpg}
\caption{A section with some fancy lines around the text. From \textit{Probabilities and Statistics for Engineers and Scientists}, by Walpole \textit{et.al}, 2011. }
\label{fig:picture-sections}
\end{figure}

Of course this is also achievable without the picture environment, simpy using TeX commands or with tikZ. From graphics point of view, the environment is also useful for short mathematical diagrams.

\section{The Basic Commands}

\begin{docEnvironment}{picture}{}{}
\end{docEnvironment}
The |picture| environment is created using one of two commands.

\emphasis{picture}
\begin{teXXX}
 \begin{picture}(x, y). . . \end{picture}
\end{teXXX}

\noindent or

\begin{teXX}
  \begin{picture}(x, y)(x0,y0). . . \end{picture}
\end{teXX}

\begin{docCommand}{unitlength}{\marg{dim}}
Most people prefer the first type which they combine, with a |setlength| command that sets the \cs{unitlength}.
\end{docCommand}

The optional argument gives the coordinates of the point at the lower-left corner of the picture (thereby determining the origin). For example, if \cs{unitlength} has been set to 1mm, the command

\begin{texexample}{}{}
  \setlength\unitlength{1mm}
  \begin{picture}(40,40)(0,0)
    \put(10,30){\vector(0,-1){30}}
    \put(10,30){\vector(1,0){30}}
    \put(25,30.5){$a$} 
  \end{picture}
\end{texexample}

produces a picture of width 100 millimeters and height 200 millimeters, whose lower-left corner is the point (10,20) and whose upper-right corner is therefore the point (110,220). When you first draw a picture, you will omit the optional argument, leaving the origin at the lower-left corner. If you then want to modify your picture by shifting everything, you just add the appropriate optional argument.

\section{Text and Formulae}

%\begin{macro}{\linethickness}
%\begin{macro}{\thicklines}
%\begin{macro}{\thinlines}
Text and formulas can be written into a picture
environment with the \cs{put} command in the usual way. The line thickness can be
set by using \cs{linethickness}\marg{dim}. The command \cs{thinlines} is half the thickness of the \cs{linethickness} dimension and \cs{thicklines} is the current line width. The \cs{linethickness} does not change width of slanted lines
or circles as it is drawn using a font and would render badly.
%\end{macro}
%\end{macro}
%\end{macro}

\emphasis{thicklines}
\begin{texexample}{Text and Formulae}{}
\setlength{\unitlength}{0.8cm}
\begin{picture}(6,5)
 \thicklines
 \put(1,0.5){\line(2,1){3}}
 \put(4,2){\line(-2,1){2}}
 \put(2,3){\line(-2,-5){1}}
 \put(0.7,0.3){$A$}
 \put(4.05,1.9){$B$}
 \put(1.7,2.95){$C$}
 \put(3.1,2.5){$a$}
 \put(1.3,1.7){$b$}
 \put(2.5,1.05){$c$}
 \put(0.3,4){$F=
 \sqrt{s(s-a)(s-b)(s-c)}$}
 \put(3.5,0.4){$\displaystyle
 s:=\frac{a+b+c}{2}$}
\end{picture}
\end{texexample}



\setlength{\unitlength}{5cm}
\begin{picture}(1,1)
\put(0,0){\line(0,1){1}}
\put(0,0){\line(1,0){1}}
\put(0,0){\color{blue}\line(1,1){1}}
\put(0,0){\color{orange}\line(1,2){0.5}}
\end{picture}


\section{multiput and linethickness}
The \cmd{\multiput} is used to place multiple objects onto the picture. It has the general format shown below:

\setlength{\unitlength}{2mm}
\begin{picture}(30,20)
  \color{green}
   \linethickness{0.075mm}
   \multiput(0,0)(1,0){25}%
   {\line(0,1){20}}
   \multiput(0,0)(0,1){21}%
   {\line(1,0){25}}
   \linethickness{0.15mm}
   \multiput(0,0)(5,0){6}%
   {\line(0,1){20}}
   \multiput(0,0)(0,5){5}%
   {\line(1,0){25}}
   \linethickness{0.3mm}
   \multiput(5,0)(10,0){2}%
    {\line(0,1){20}}
   \multiput(0,5)(0,10){2}%
   {\line(1,0){25}}
\end{picture}



\begin{docCommand}{multiput} {(x,y) (Dx, Dy) \marg{n} \marg{object} }
The command |\multiput| allows to repeat
a \cmd{\put} a number of times.
\end{docCommand}

\begin{figure}
\setlength{\unitlength}{0.8cm}
\begin{picture}(6,5)
 \thicklines
 \put(1,0.5){\line(2,1){3}}
 \put(4,2){\line(-2,1){2}}
 \put(2,3){\line(-2,-5){1}}
 \put(0.7,0.3){$A$}
 \put(4.05,1.9){$B$}
 \put(1.7,2.95){$C$}
 \put(3.1,2.5){$a$}
 \put(1.3,1.7){$b$}
 \put(2.5,1.05){$c$}
 \put(0.3,4){$F=
 \sqrt{s(s-a)(s-b)(s-c)}$}
 \put(3.5,0.4){$\displaystyle
 s:=\frac{a+b+c}{2}$}
\end{picture}
\caption{Figures can have captions, if you enclose in a figure environment}
\end{figure}

\begin{figure}
\scalebox{0.7}{
\setlength{\unitlength}{0.5mm}
\begin{picture}(120,168)
\newsavebox{\foldera}
\savebox{\foldera}
(40,32)[bl]{% definition
\multiput(0,0)(0,28){2}
{\line(1,0){40}}
\multiput(0,0)(40,0){2}
{\line(0,1){28}}
\put(1,28){\oval(2,2)[tl]}
\put(1,29){\line(1,0){5}}
\put(9,29){\oval(6,6)[tl]}
\put(9,32){\line(1,0){8}}
\put(17,29){\oval(6,6)[tr]}
\put(20,29){\line(1,0){19}}
\put(39,28){\oval(2,2)[tr]}
}
\newsavebox{\folderb}
\savebox{\folderb}
(40,32)[l]{% definition
\put(0,14){\line(1,0){8}}
\put(8,0){\usebox{\foldera}}
\put(0.2,1.4)
{$\beta=v/c=\tanh\chi$}
}
\put(34,26){\line(0,1){102}}
\put(14,128){\usebox{\foldera}}
\multiput(34,86)(0,-37){3}
{\usebox{\folderb}}
\end{picture}}
\caption{Pictures can be scaled using \protect\textbackslash scalebox.}
\end{figure}

\section{Some examples}
Any vertex-symmetric graph is regular, but edge-symmetric graphs
need not be regular. For example,
\begin{verbatim}
$$\unitlength=10pt
\def\putdisk(#1,#2){\put(#1,#2){\disk{.4}}}
$\vcenter{
\hbox{\beginpicture(2,1.5)(0,0)
\putdisk(0,0)
\putdisk(2,0)
\putdisk(1,.5)
\putdisk(1,1.5)
\put(0,0){\line(2,1){1}}
\put(2,0){\line(-2,1){1}}
\put(1,.5){\line(0,1){1}}
\endpicture}}
\quad&\hbox{is edge-symmetric, not vertex-symmetric;}\cr
\noalign{\smallskip}
\vcenter{
\hbox{\beginpicture(2,2)(0,0)
\putdisk(1,0)
\putdisk(1,2)
\putdisk(0,.5)
\putdisk(0,1.5)
\putdisk(2,.5)
\putdisk(2,1.5)
\put(0,.5){\line(2,1){2}}
\put(2,.5){\line(-2,1){2}}
\put(0,.5){\line(2,3){1}}
\put(2,.5){\line(-2,3){1}}
\put(0,.5){\line(1,0){2}}
\put(0,1.5){\line(1,0){2}}
\put(1,0){\line(-2,3){1}}
\put(1,0){\line(2,3){1}}
\put(1,0){\line(0,1){2}}
\endpicture}}
\quad&\hbox{is vertex-symmetric, not edge-symmetric.}\qquad
 (\vcenter{\hbox{\beginpicture(1,2)(0,0)
\putdisk(.5,0)\putdisk(.5,2)\put(.5,0){\line(0,1){2}}\endpicture}}
\hbox{ is a maximal clique})\cr}$$
\end{verbatim}



\section{picture package}

The \pkg{picture} package by Heiko Oberdiek redefines the default \pkg{picture} macros and adds code that detects
whether such an argument is given as number or as length. In the latter case, the
length is used directly without multiplying with \cs{unitlength}. Th following
example i from the documentation of the package.

 \setlength{\unitlength}{1pt}
 \begin{picture}(\widthof{Hello World}, 10mm)
   \put(0, 0){\makebox(0,0)[lb]{Hello World}}%
   \put(0, \heightof{Hello World} + \fboxsep){%
   \line(1, 0){\widthof{Hello World}}%
 }%
 \put(\widthof{Hello World}, 10mm){%
   \line(0, -1){10mm}%
 }%
 \put(0,0){\line(966,259){8}}
 \end{picture}

The package |calc| is used for calculations or etex. The picture package requires that the package |calc| is loaded before
the |picture| package and is loaded correctly by |phd|.

The package also supports the packages \pkg{pspicture} and \pkg{pict2e}, but they must be loaded before package picture.

\section{pict2e}

The package pict2e by Hubert G\"a\ss lein, Rolf Niepraschk and Joseph Tkadlec extends the existing LATEX picture environment, using the familiar
technique (cf. the graphics and color packages) of driver files. In the user-level part of
this documentation there is a fair number of examples of use, showing where things are
improved by comparison with the Standard LaTeX picture environment.

The package is loaded automatically by |phd|.







%  \def\storyi{The best graphics package ever developed is the TikZ package. 
Its parent package is PGF which is short of a miracle that has been programmed
using \tex, a more than thirty years old program. This has taken over almost all other
packages and is very popular with newcomers to \latex. It is frustrating at first, but once 
you over the basic ideas and concepts it opens infinite possibilities for typesetting
great articles and books.}



\cxset{chapter format=stewart,
       texti=\storyi,
       textii=\storyi}

\newcommand\seepgfmanual[1]{%
    \textit{see} the PGFmanual page #1}%
    
%\cxset{chapter format = traditional}    
\chapter{TikZ}

\section{The \protect\texttt{TikZ} package}
\pkg{TikZ}, a high-level interface to \pkg{PGF}, is a language-based tool for specifying graphics.
It uses familiar graphics-related concepts, such as point, line, and circle and
has a concise and natural syntax. It meshes well with pdfLATEX in the sense that
no additional processing steps are needed. Another positive aspect of \pkg{TikZ} is
its ability to blend \tex fonts, symbols, and mathematics within the generated
graphics.


All the TikZ commands can be used inline using \docAuxCommand{tikz} or within the \docAuxCommand{tikzpicture} environment. When we want to use captions and labels, we enclose it in the figure environment or use \docAuxCommand{captionof}, but it can be called anywhere in the text or math of a Tex document:

\begin{teXXX}
\begin{figure}
\centering
%\tikzset{external/force remake}
\begin{tikzpicture}
... TikZ commands ...
\end{tikzpicture}
\caption{A diagram drawn with TikZ.}
\label{Fig:_diagram1}
\end{figure}
\end{teXXX}

We can also use them in math:

\begin{teXXX}
\begin{align*}
\int dx\; f(x) =
\alpha
%\tikzset{external/force remake}
\begin{tikzpicture}
... TikZ commands ...
\end{tikzpicture}
\end{align*}
\end{teXXX}



\section{Draw simple lines}

\begin{texexample}{Draw a Line}{ex:line}
\begin{tikzpicture}
\node[draw] (S1) at (0,0) {Paris};
\node[draw] (S2) at (3,0) {Stratsbourg};
\draw (S1) -- (S2);
\end{tikzpicture}
\end{texexample}


The syntax of the command is:

|\node|\oarg{options} (\meta{name}) at (\meta{position}) |{|\meta{contents}|}|

If we look
 carefully, we see that the two writings give
Slightly different results:
- In the first case, node is an operation executed on a path. We
Can consider each node as a decoration of the point at which it
is associated. The line drawn by the draw command joins two points, the
Nodes are objects added later and centered on points. The option
Draw of the node trace operation the node outline.
- In the second case, \ node is a TikZ command which allows to define
A node, to name it and to draw it. One can then consider the
Nodes as pre-existing objects that will then be linked with the \docAuxCommand{node}.


\begin{texexample}{Draw a Line}{ex:line}
\begin{tikzpicture}
\node[draw] (S1) at (0,0) {Paris};
\node[draw] (S2) at (0,3) {Stratsbourg};
\draw[->] (S1) -- (S2);
\end{tikzpicture}
\end{texexample}

The basic building block of all pictures in \tikzname is the path. A path is a series of straight lines and curves
that are connected (that is not the whole picture, but let us ignore the complications for the moment). You
start a path by specifying the coordinates of the start position as a point in round brackets, as in (0,0).
This is followed by a series of \enquote{path extension operations.}


\begin{texexample}{Draw a Line}{ex:line}
\begin{tikzpicture}
\draw[->] (0,0) -- (1.5,0) -- (0, 1.2);
\end{tikzpicture}
\end{texexample}


\subsection*{Adding Text} 

So far we have seen how to draw lines and arcs. However, an important component is still missing the addition of text. When
\tikzname is constructing a path and it encounters the keyword |node| typically followed by some options  it reads a \textit{node specification}. Options can typically follow and then it terminates by curly brackets. 
 

\begin{texexample}{Draw a Line}{ex:line}
\begin{tikzpicture}
\draw[->] (0,0) -- (1.5,0) node {First Node} -- (0, 1.2) node[shape = circle] {Second Node};
\end{tikzpicture}
\end{texexample}


The \docAuxCommand*{node} can be used to abbreviate the operation. A longer example can demonstrate this better. How can we draw the following figure?

\begin{tikzpicture}
\node[circle,fill=black,inner sep=0.8pt,draw] (a) at (0,0) {};
\node[circle,fill=black,inner sep=0.8pt,draw] (b) at (1.5,0) {};
\node[circle,fill=black,inner sep=1.5pt,draw] (c) at (.75,2) {};
\node[circle,fill=black,inner sep=0.8pt,draw] (d) at (0.75,.75) {};
\node[circle,fill=black,inner sep=0.8pt,draw] (e) at (2,1) {};


\node () at (-0.3,0) {\tiny$1$};
\node () at (0.75,0.45) {\tiny$2$};
\node () at (0.75,2.3) {\tiny$4$};
\node () at (2,1.3) {\tiny$-1$};
\node () at (1.8,0) {\tiny$-1$};

\draw (a)--(b)--(e)--(c) --(a)--(d)--(b)--(c);
\draw (c)--(d);

\node at (3,1) {\Large{$\sim$}};

\begin{scope}[shift={(+4,0)}]
\node[circle,fill=black,inner sep=0.8pt,draw] (a) at (0,0) {};
\node[circle,fill=black,inner sep=0.8pt,draw] (b) at (1.5,0) {};
\node[circle,fill=black,inner sep=0.8pt,draw] (c) at (.75,2) {};
\node[circle,fill=black,inner sep=0.8pt,draw] (d) at (0.75,.75) {};
\node[circle,fill=black,inner sep=0.8pt,draw] (e) at (2,1) {};


\node () at (-0.3,0) {\tiny$2$};
\node () at (0.75,0.45) {\tiny$3$};
\node () at (0.75,2.3) {\tiny$0$};
\node () at (2,1.3) {\tiny$0$};
\node () at (1.8,0) {\tiny$0$};

\draw (a)--(b)--(e)--(c) --(a)--(d)--(b)--(c);
\draw (c)--(d);

\end{scope}
\end{tikzpicture}

\begin{texexample}{A larger example}{ex:larger}
\begin{tikzpicture}
\node[circle,fill=black,inner sep=0.8pt,draw] (a) at (0,0) {};
\node[circle,fill=black,inner sep=0.8pt,draw] (b) at (1.5,0) {};
\node[circle,fill=black,inner sep=1.5pt,draw] (c) at (.75,2) {};
\node[circle,fill=black,inner sep=0.8pt,draw] (d) at (0.75,.75) {};
\node[circle,fill=black,inner sep=0.8pt,draw] (e) at (2,1) {};


\node () at (-0.3,0) {\tiny$1$};
\node () at (0.75,0.45) {\tiny$2$};
\node () at (0.75,2.3) {\tiny$4$};
\node () at (2,1.3) {\tiny$-1$};
\node () at (1.8,0) {\tiny$-1$};

\draw (a)--(b)--(e)--(c) --(a)--(d)--(b)--(c);
\draw (c)--(d);

\node at (3,1) {\Large{$\sim$}};

\begin{scope}[shift={(+4,0)}]
\node[circle,fill=black,inner sep=0.8pt,draw] (a) at (0,0) {};
\node[circle,fill=black,inner sep=0.8pt,draw] (b) at (1.5,0) {};
\node[circle,fill=black,inner sep=0.8pt,draw] (c) at (.75,2) {};
\node[circle,fill=black,inner sep=0.8pt,draw] (d) at (0.75,.75) {};
\node[circle,fill=black,inner sep=0.8pt,draw] (e) at (2,1) {};


\node () at (-0.3,0) {\tiny$2$};
\node () at (0.75,0.45) {\tiny$3$};
\node () at (0.75,2.3) {\tiny$0$};
\node () at (2,1.3) {\tiny$0$};
\node () at (1.8,0) {\tiny$0$};

\draw (a)--(b)--(e)--(c) --(a)--(d)--(b)--(c);
\draw (c)--(d);

\end{scope}
\end{tikzpicture}
\captionof{figure}{The larger vertex fires once to move from the left configuration to the right configuration.}
\end{texexample}

Behind the scenes pgf uses the basic system command \docAuxCommand{pgfnode} to create the nodes. The syntax of the command is given on \seepgfmanual{1026} as:

\begin{docCommand}{pgfnode}{\marg{shape}\marg{anchor}\marg{label text}\marg{name}\marg{path usage command}}
This command creates a new node. The \marg{shape} of the node must have been declared previously using
\lstinline{pgfdeclareshape}.

The shape is shifted such that the \marg{anchor} is at the origin. In order to place the shape somewhere else,
use the coordinate transformation prior to calling this command.
The hnamei is a name for later reference. If no name is given, nothing will be “saved” for the node, it
will just be drawn.

The \marg{path usage command} is executed for the background and the foreground path (if the shape defines
them).
\end{docCommand}


A good workflow, is to first define the nodes, next label them and then draw any connecting lines.

\begin{texexample}{Named nodes}{ex:named} 
\begin{tikzpicture}
\node[circle,fill=black,inner sep=0.8pt,draw] (a) at (0,0) {};
\node[circle,fill=black,inner sep=0.8pt,draw] (b) at (1.5,0) {};
\node[circle,fill=black,inner sep=1.5pt,draw] (c) at (.75,2) {};
\node[circle,fill=black,inner sep=0.8pt,draw] (d) at (0.75,.75) {};
\node[circle,fill=black,inner sep=0.8pt,draw] (e) at (2,1) {};
\end{tikzpicture}
\end{texexample}

\begin{texexample}{Named nodes}{ex:named} 
\begin{tikzpicture}
\node[circle,fill=black,inner sep=0.8pt,draw] (a) at (0,0) {};
\node[circle,fill=black,inner sep=0.8pt,draw] (b) at (1.5,0) {};
\node[circle,fill=black,inner sep=1.5pt,draw] (c) at (.75,2) {};
\node[circle,fill=black,inner sep=0.8pt,draw] (d) at (0.75,.75) {};
\node[circle,fill=black,inner sep=0.8pt,draw] (e) at (2,1) {};
% absolute labelling
\node () at (-0.3,0) {\tiny$1$};
\node () at (0.75,0.45) {\tiny$2$};
\node () at (0.75,2.3) {\tiny$4$};
\node () at (2,1.3) {\tiny$-1$};
\node () at (1.8,0) {\tiny$-1$};
\end{tikzpicture}
\end{texexample}

\begin{texexample}{Named nodes}{ex:named} 
\begin{tikzpicture}
\pgfdeclarelayer{background}
\pgfdeclarelayer{foreground}
\pgfsetlayers{background,main,foreground}
\node[circle,fill=black,inner sep=0.8pt,draw] (a) at (0,0) {};
\node[circle,fill=black,inner sep=0.8pt,draw] (b) at (1.5,0) {};
\node[circle,fill=black,inner sep=1.5pt,draw] (c) at (.75,2) {};
\node[circle,fill=black,inner sep=0.8pt,draw] (d) at (0.75,.75) {};
\node[circle,fill=black,inner sep=0.8pt,draw] (e) at (2,1) {};
% absolute labelling
\node () at (-0.3,0) {\tiny$1$};
\node () at (0.75,0.45) {\tiny$2$};
\node () at (0.75,2.3) {\tiny$4$};
\node () at (2,1.3) {\tiny$-1$};
\node () at (1.8,0) {\tiny$-1$};
% draw connecting lines
\draw (a)--(b)--(e)--(c) --(a)--(d)--(b)--(c);
\draw (c)--(d);
%\begin{pgfonlayer}{background}
\begin{scope}[on background layer={color=blue!10}]
\node [fill=blue!10,fit=(a) (b) (c)
(d) (e)] {};
\end{scope}
%\end{pgfonlayer}
\end{tikzpicture}
\end{texexample}

Just to recap, using \docAuxCommand*{node} and the \textbf{at} we can position accurately any node. We could have used the much longer command |path node|, but in our case above this is unecessary (\seepgfmanual{49}), for more explanations if you are still unsure.

Nodes can be named or unnamed. There are two ways to name them, with the key \docValue{name} or within brackets. The second method is to be preferred. Names for nodes can be pretty arbitrary, but they should not contain commas, periods, parentheses, colons, and some other special characters. However, they can contain underscores and hyphens

\subsection{Layers and Scope}

We can add a backround layer, using the library \textit{backgrounds}, which provides key values for adding backgrounds. \pgfname\ provides a layering mechanism for composing graphics from
multiple layers. (This mechanism is not to be confused with the
conceptual ``software layers'' the \pgfname\ system is composed of.)
Layers are often used in graphic programs. The idea is that you can
draw on the different layers in any order. So you might start drawing
something on the ``background'' layer, then something on the
``foreground'' layer, then something on the ``middle'' layer, and then
something on the background layer once more, and so on. At the end, no
matter in which ordering you drew on the different layers, the layers
are ``stacked on top of each other'' in a fixed ordering to produce
the final picture. Thus, anything drawn on the middle layer would come
on top of everything of the background layer.

Normally, you do not need to use different layers since you will have
little trouble ``ordering'' your graphic commands in such a way that
layers are superfluous. However, in certain situations you only
``know'' what you should draw behind something else after the
``something else'' has been drawn.

For example, suppose you wish to draw a yellow background behind your
picture. The background should be as large as the bounding box of the
picture, plus a little border. If you know the size of the bounding box
of the picture at its beginning, this is easy to accomplish. However,
in general this is not the case and you need to create a
``background'' layer in addition to the standard ``main'' layer. Then,
at the end of the picture, when the bounding box has been established,
you can add a rectangle of the appropriate size to the picture.

\subsection{Declaring Layers}

In \pgfname\ layers are referenced using names. The standard layer,
which is a bit special in certain ways, is called |main|. If nothing
else is specified, all graphic commands are added to the |main|
layer. You can declare a new layer using the following command:

\begin{docCommand}{pgfdeclarelayer}{\marg{name}}
  This command declares a layer named \meta{name} for later
  use. Mainly, this will set up some internal bookkeeping.
\end{docCommand}

The next step toward using a layer is to tell \pgfname\ which layers
will be part of the actual picture and which will be their
ordering. Thus, it is possible to have more layers declared than are
actually used.

\begin{docCommand}{pgfsetlayers}{\marg{layer list}}
  This command tells \pgfname\ which layers will be used in
  pictures. They are stacked on top of each other in the order
  given. The layer |main| should always be part of the list. Here is
  an example:
\begin{codeexample}[code only]
\pgfdeclarelayer{background}
\pgfdeclarelayer{foreground}  
\pgfsetlayers{background,main,foreground}
\end{codeexample}

  This command should be given either outside of any picture or ``directly inside'' of a picture.
  Here, the ``directly inside'' means that there should be no further level of \TeX\ grouping between |\pgfsetlayers| and the matching |\end{pgfpicture}| (no closing braces, no |\end{...}|). It will also work if |\pgfsetlayers| is provided before |\end{tikzpicture}| (with similar restrictions).
\end{docCommand}


\subsection{Using Layers}

Once the layers of your picture have been declared, you can start to
``fill'' them. As said before, all graphics commands are normally
added to the |main| layer. Using the |{pgfonlayer}| environment, you
can tell \pgfname\ that certain commands should, instead, be added to
the given layer.

\begin{docEnvironment}{pgfonlayer}{\marg{layer name}}
\end{docEnvironment}

The whole \meta{environment contents} is added to the layer with the
name \meta{layer name}. This environment can be used anywhere inside
a picture. Thus, even if it is used inside a |{pgfscope}| or a \TeX\
group, the contents will still be added to the ``whole'' picture.
Using this environment multiple times inside the same picture will
cause the \meta{environment contents} to accumulate.

  \emph{Note:} You can \emph{not} add anything to the |main| layer
  using this environment. The only way to add anything to the main
  layer is to give graphic commands outside all |{pgfonlayer}|
  environments. 



\begin{codeexample}[]
\pgfdeclarelayer{background layer}
\pgfdeclarelayer{foreground layer}
\pgfsetlayers{background layer,main,foreground layer}
\begin{tikzpicture}
  % On main layer:
  \fill[blue] (0,0) circle (1cm);
  
  \begin{pgfonlayer}{background layer}
    \fill[yellow] (-1,-1) rectangle (1,1);
  \end{pgfonlayer}
  
  \begin{pgfonlayer}{foreground layer}
    \node[white] {foreground};
  \end{pgfonlayer}
  
  \begin{pgfonlayer}{background layer}
    \fill[black] (-.8,-.8) rectangle (.8,.8);
  \end{pgfonlayer}

  % On main layer again:
  \fill[blue!50] (-.5,-1) rectangle (.5,1);
\end{tikzpicture}
\end{codeexample}



\long\gdef\mytriangle{
\node[circle,fill=black,inner sep=0.8pt,draw] (a) at (0,0) {};
\node[circle,fill=black,inner sep=0.8pt,draw] (b) at (1.5,0) {};
\node[circle,fill=black,inner sep=1.5pt,draw] (c) at (.75,2) {};
\node[circle,fill=black,inner sep=0.8pt,draw] (d) at (0.75,.75) {};
\node[circle,fill=black,inner sep=0.8pt,draw] (e) at (2,1) {};
% absolute labelling
\node () at (-0.3,0) {\tiny$1$};
\node () at (0.75,0.45) {\tiny$2$};
\node () at (0.75,2.3) {\tiny$4$};
\node () at (2,1.3) {\tiny$-1$};
\node () at (1.8,0) {\tiny$-1$};
% draw connecting lines
\draw (a)--(b)--(e)--(c) --(a)--(d)--(b)--(c);
\draw (c)--(d);
}

\begin{texexample}{Adding backgrouns}{ex:backgrounds}
\begin{tikzpicture}
\pgfdeclarelayer{background}
\pgfdeclarelayer{foreground}
\pgfsetlayers{background,main,foreground}
\mytriangle
%\begin{pgfonlayer}{background}
\begin{scope}[on background layer={color=blue!10}]
\mytriangle
\node [fill=blue!10,fit=(a) (b) (c)
(d) (e)] {};
\end{scope}
%\end{pgfonlayer}
\end{tikzpicture}
\end{texexample}


\begin{texexample}{Adding backgrouns}{ex:backgrounds}
\begin{tikzpicture}
\pgfdeclarelayer{background}
\pgfdeclarelayer{foreground}
\pgfsetlayers{background,main,foreground}
\mytriangle
%\begin{pgfonlayer}{background}
\begin{scope}[on background layer={color=blue!10}]
\node [fill=blue!10,fit=(a) (b) (c)
(d) (e)] {};
\end{scope}

\begin{scope}[shift={(+4,0)}]
\mytriangle
\begin{pgfonlayer}{background}
\node [pattern=checkerboard light gray,fit=(a) (b) (c)
(d) (e)] {};
\end{pgfonlayer}
\end{scope}
\end{tikzpicture}
\end{texexample}

This brings us to the end of our discussion. Time for a coffee and a break.                

\section{Adding styles}

In our previous example, we cut and pasted many of the repetitive keys. \pgfname offers a way to set a new key to the values of other keys using the handler |.style|. This is a very powerful way of redefining new keys, but also simplifying the code. Styles in \tikzname can be considered similar to macros in standard LaTeX. When I made a drawing, we can still tweak the styles and look how the drawing changes, until it's perfect. You should never have to tweak each node.

\begin{texexample}{Using styles}{ex:usingstyles}
\tikzset{BN/.style = {circle,fill=black,inner sep=0.8pt,draw},
         tiny/.style = {font=\tiny}, 
}
\begin{tikzpicture}
\node[BN] (a) at (0,0) {};
\node[BN] (b) at (1,0) {};
\node[BN] (c) at (1,1) {};
\node[BN] (d) at (0,1) {};
\node[BN] (e) at (-1,0) {};

\node () at (-1.3,0) [tiny]{$v_1$};
\node () at (-.3,1)  [tiny]{$v_2$};
\node () at (1.3,0)  [tiny]{$w_1$};
\node () at (1.3,1)  [tiny]{$w_2$};

\node[tiny] () at (0.5,-0.2) {$a$};
\node[tiny] () at (0.5,1.2) {$b$};
\node[tiny] () at (0.2,0.5) {$c$};
\node[tiny] () at (-0.5,-.2) {$d$};

\draw (e) -- (a) -- (b) -- (c) -- (d) -- (a);
\draw (e) -- (d);

\end{tikzpicture}
\end{texexample}



\section{Arcs and options for lines}

\begin{texexample}{Draw a Line}{ex:line}
\begin{tikzpicture}
\draw[->] (0,0) -- (1.5,0) node[draw, ellipse] {First Node} -| (0, 1.2) node[draw,ellipse,rotate=45] {Second Node};
\end{tikzpicture}
\end{texexample}

\begin{texexample}{Drawing arcs}{ex:matharcs}
We define 
\begin{gather*}
    \bar{d}_{k,l}:=\hspace{6pt}
    \begin{tikzpicture}[baseline=(current bounding box.center)]
    \draw[->] (3,2) arc (-180:180:5mm);
	  \fill (3.95,2.2) circle [radius=2pt];
    \draw (3.95,1.8) circle [radius=2pt];
    \node at (4.2,1.8) {$l$};
    \node at (4.2,2.2) {$k$};
    \end{tikzpicture}
    \hspace{0.5cm}
    \text{and}
    \hspace{0.5cm}
    d_{k,l}:=\hspace{6pt}
    \begin{tikzpicture}[baseline=(current bounding box.center)]
    \draw[<-] (3,2) arc (-180:180:5mm);
    \fill (3.95,2.2) circle [radius=2pt];
    \draw (3.95,1.8) circle [radius=2pt];
    \node at (4.2,1.8) {$l$};
    \node at (4.2,2.2) {$k$};
    \end{tikzpicture}
    \hspace{0.5cm}
    \text{for}
    \hspace{2mm} k,l\in\mathbb{Z}_{\geq 0}.
\end{gather*}
\end{texexample}


Here is a figure that you should try and reproduce.
\newcommand{\G}{\Gamma}

\begin{tikzpicture}
\draw (-3.5,-1)--(-2.5,0); \draw (-2.5,-1)--(-3.5,0); \draw (-1.5,-1)--(-1.5,0);\draw[fill=black] (-3,-0.5) circle (0.1cm); \draw (-3.5,0)--(-3.5,1); \draw (-2.5,0)--(-1.5,1); \draw (-1.5,0)--(-2.5,1);\draw[fill=black] (-2,0.5) circle (0.1cm); \draw[->] (-3.5,1)--(-2.5,2); \draw[->] (-2.5,1)--(-3.5,2); \draw[->] (-1.5,1)--(-1.5,2); \draw[fill=black] (-3,1.5) circle (0.1cm); \draw (-3.6,0)--(-3.4,0);\draw (-2.6,0)--(-2.4,0);\draw (-1.6,0)--(-1.4,0); \draw (-3.6,1)--(-3.4,1);\draw (-2.6,1)--(-2.4,1);\draw (-1.6,1)--(-1.4,1); \node at (-3.5,-1.2) {$x_1$};\node at (-2.5,-1.2) {$x_2$};\node at (-1.5,-1.2) {$x_3$}; \node at (-3.5,2.2) {$y_1$};\node at (-2.5,2.2) {$y_2$};\node at (-1.5,2.2) {$y_3$}; \node at (-3.8,0) {$t_1$};\node at (-2.2,0) {$t_2$};\node at (-1.2,0) {$t_3$}; \node at (-3.8,1) {$t_4$};\node at (-2.8,1) {$t_5$};\node at (-1.2,1) {$t_6$}; \node at (-2.5,-1.65) {$\Gamma$};
\draw[->] (0,0)--(1,1); \draw[->] (1,0)--(0,1); \draw[fill=black] (0.5,0.5) circle (0.1cm); \draw[->] (2,0)--(3,1); \draw[->] (3,0)--(2,1); \draw[fill=black] (2.5,0.5) circle (0.1cm); \draw[->] (4,0)--(5,1); \draw[->] (5,0)--(4,1); \draw[fill=black] (4.5,0.5) circle (0.1cm); \draw[->] (6,0)--(6,1); \draw[->] (7,0)--(7,1); \draw[->] (8,0)--(8,1);
\node at (0,-.2) {$x_1$};\node at (1,-.2) {$x_2$}; \node at (2,-.2) {$t_2$};\node at (3,-.2) {$t_3$}; \node at (4,-.2) {$t_4$};\node at (5,-.2) {$t_5$}; \node at (6,-.2) {$x_3$}; \node at (7,-.2) {$t_1$}; \node at (8,-.2) {$t_6$};
\node at (0,1.2) {$t_1$};\node at (1,1.2) {$t_2$}; \node at (2,1.2) {$t_5$};\node at (3,1.2) {$t_6$}; \node at (4,1.2) {$y_1$};\node at (5,1.2) {$y_2$}; \node at (6,1.2) {$t_3$}; \node at (7,1.2) {$t_4$}; \node at (8,1.2) {$y_3$};
\node at (0.5,-0.65) {$\G_1$}; \node at (2.5,-0.65) {$\G_2$}; \node at (4.5,-0.65) {$\G_3$}; \node at (6,-0.65) {$\G_4$};\node at (7,-0.65) {$\G_5$};\node at (8,-0.65) {$\G_6$}; 
\end{tikzpicture}

This brings us to the end.




The |node| can take numerous options who are then used to set the typesetting of the text that follows:


\begin{texexample}{Draw a Line}{ex:line}
\begin{tikzpicture}
\draw[->] (0,0) -- (1.5,0) node[draw, ellipse] {First Node} -| (0, 1.2) node[draw,ellipse,rotate=45, text width=3cm, fill=creamy, text justified] {\lorem};
\end{tikzpicture}
\end{texexample}


\begin{texexample}{Draw a Line}{ex:line}
\begin{tikzpicture}[funny ellipse/.style = {draw,ellipse,rotate=45, text width=3cm, fill=creamy, text justified} ]
\draw[->] (0,0) -- (1.5,0) node[draw, ellipse] {First Node} -| (0, 1.2) node[funny ellipse] {\lorem};
\end{tikzpicture}
\end{texexample}

This can also be written by using \docAuxCommand{tikzset} for setting out all the keys. This can written just before the environment or within the scope of the environment. See \href{https://tex.stackexchange.com/questions/52372/should-tikzset-or-tikzstyle-be-used-to-define-tikz-styles}{TX.SX discussion}, for the option to set |\tikzstyle| which should not be used, even if it is quicker to write.


\begin{texexample}{Draw a Line}{ex:line}
\tikzset{funny ellipse/.style = {draw,ellipse,rotate=45, text width=3cm, fill=creamy, text justified} }
\begin{tikzpicture}
\draw[->] (0,0) -- (1.5,0) node[draw, ellipse] {First Node} -| (0, 1.2) node[funny ellipse] {\lorem};
\end{tikzpicture}
\end{texexample}

A |node| can possibly be rendered with a choice from a list of over 720 keys.

ed. 



Using the |TikZ| package you can draw figures and intermingle them with text. To draw a simple diamond as shown in \fref{fig:diamond} we use
the following commands. The package comes with a very comprehensive manual of over 500 pages long. One can state that there is nothing that you cannot draw with PGF/TikZ, if you have the patience and perseverance. TikZ's language has a syntax of its own with very little connection to what we have used so far. You will need to set aside adequate time to study this, especially if your work has a lot of specially drawn figures that you need. The result like anything else in \tex make the effort worthwhile.

\begin{texexample}{Draw a Diamond}{fig:diamond}
\begin{tikzpicture}
 \draw (1,0) -- (0,1) -- (-1,0) -- (0,-1) -- cycle;
\end{tikzpicture}
\end{texexample}


\begin{texexample}{Text long path}{ex:decorations}
\begin{tikzpicture}
\draw [help lines] grid (3,2);
\draw [red, dashed]
[postaction={decoration={text along path, text={a big juicy apple},
text align=fit to path}, decorate}]
(0,0) .. controls (0,2) and (3,2) .. (3,0);
\node (A) at (1.5,0) {!};
\end{tikzpicture}
\end{texexample}


\begin{texexample}{Text long path}{ex:decorations}

Hello \begin{pgfpicture}
\pgfpathrectangle{\pgfpointorigin}{\pgfpoint{2ex}{1ex}}
\pgfusepath{stroke}
\end{pgfpicture} World!

\end{texexample}


\emphasis{-,draw,begin,end,tikzpicture}
\begin{teXXX}
\begin{tikzpicture}
\draw (1,0) -- (0,1) -- (-1,0) -- (0,-1) -- cycle;
\end{tikzpicture}
\end{teXXX}



\makeatletter
The value of $x$ is \pgfsys@markposition{here}important.

Lots of text.
\hbox{\pgfsys@markposition{myorigin}%
\begin{pgfpicture}
% Switch of size protocol
\pgfpathmoveto{\pgfpointorigin}
\pgfusepath{use as bounding box}
\pgfsys@getposition{here}{\hereposition}
\pgfsys@getposition{myorigin}{\thispictureposition}
\pgftransformshift{\pgfpointscale{-1}{\thispictureposition}}
\pgftransformshift{\hereposition}
\pgfpathcircle{\pgfpointorigin}{1cm}
\pgfusepath{draw}
\end{pgfpicture}}

\makeatother


You cannot write directly into a picture environment. The command \docAuxCommand{pgftext} can be used. 

\begin{texexample}{Using text directly}{ex:pgftext}
\tikz{\draw[help lines] (0,0) grid (3,2);
\pgftext[base,x=1cm,y=0.5cm] {lovely}}
\end{texexample}





Sometimes it is quite useful when debugging to add a backround grid. 


\begin{centering}
\begin{tikzpicture}
\draw[step=0.25cm,color=creamy] (-1,-1) grid (1,1);
\draw [color=bgsexy](1,0) -- (0,1) -- (-1,0) -- (0,-1) -- cycle;
\end{tikzpicture}
\captionof{figure}{You can add a background grid using \texttt{step=0.25cm, color=green} as an option}
\end{centering}


\emphasis{step,color,green,grid,begin,end}
\begin{teXXX}
\begin{tikzpicture}
  \draw[step=0.25cm,color=green] (-1,-1) grid (1,1);
  \draw (1,0) -- (0,1) -- (-1,0) -- (0,-1) -- cycle;
\end{tikzpicture}
\end{teXXX}

The grid is specified by providing two diagonally opposing points: (-1,-1)
and (1, 1). The two options supplied give a step size for the grid lines and a
specification for the color of the grid lines, using the \docpkg{xcolor} package

\subsection{Specifying points and paths}

\begin{texexample}{Specifying points and paths}{ex:points}
\centering
\begin{tikzpicture}[scale=1.8]
% Define the points of a regular pentagon
\path (0,0) coordinate (origin);
\path (0:1cm) coordinate (P0);
\path (1*72:1cm) coordinate (P1);
\path (2*72:1cm) coordinate (P2);
\path (3*72:1cm) coordinate (P3);
\path (4*72:1cm) coordinate (P4);
% Draw the edges of the pentagon
\draw[color=bgsexy] (P0) -- (P1) -- (P2) -- (P3) -- (P4) -- cycle;
% Add "spokes"
\draw[color=bgsexy] (origin) -- (P0) (origin) -- (P1) (origin) -- (P2)
(origin) -- (P3) (origin) -- (P4);
\end{tikzpicture}
\captionof{figure}{Drawing a complicated polygon, using paths and the \texttt{draw} command}
\end{texexample}


Two key ideas used in \tikzname\ are points and paths. Both of these ideas were used
in the diamond examples. Much more is possible, however. For example, points
can be specified in any of the following ways:
\begin{enumerate}
\item  Cartesian coordinates
\item  Polar coordinates
\item  Named points
\item  Relative points
\end{enumerate}



\subsection{coordinates}
The cartesian coordinates can be defined and named using the following syntax.

%\emphasis{begin,end,coordinate,at,draw}
%\begin{teXXX}
%\begin{tikzpicture}
%  \coordinate (A) at (0,0);
%  \coordinate (B) at (1.25,0.25);
%  \draw[blue] (A) -- (B);
%\end{tikzpicture}
%\end{teXXX}

\noindent This produces:
\begin{tikzpicture}
\coordinate (A) at (0,0);
\coordinate (B) at (1.25,0.25);
\draw[blue] (A) -- (B);
\end{tikzpicture}


We can add labels to the points by using the |label| option. A label is distinct from the text of a |node|.

\begin{tikzpicture}
\coordinate [label=left:\textcolor{orange}{$A$}] (A) at (0,0);
\coordinate [label=right:\textcolor{orange}{$B$}]  (B) at (1.15,0.25);
\draw[blue] (A) -- (B);
\end{tikzpicture}

\emphasis{label,left,label:,right}
\begin{teXXX}
\begin{tikzpicture}
  \coordinate [label=left:\textcolor{orange}{$A$}] (A) at (0,0);
  \coordinate [label=west:\textcolor{orange}{$B$}] (B) at (1.25,0.25);
  \draw[blue] (A) -- (B);
\end{tikzpicture}
\end{teXXX}




If you tempted to write \texttt{label=top:} it will not work, as the command accepts the following keywords.

\begin{tikzpicture}
  \coordinate [label=left:\textcolor{orange}{east}]  (A) at (0,0);
  \coordinate [label=right:\textcolor{orange}{west}] (B) at (0,0);
  \draw[blue] (A)--(B);
\end{tikzpicture}


\section{Graphic Parameters: Line Width, Line Cap, and Line Join}

The width of lines can be specified using the key:

\begin{docKey}[tikz]{line width}{=\marg{dimension}} {no default, initially 0.4pt}
Specifies the line width \seepgfmanual{166}
\end{docKey}



\bgroup
\def\mkl#1{\tikz \draw[#1] (0,0)--(1.0, 1.5ex);}
\scriptsize\arial
\begin{tabular}{|l|l|l|l|l|l|l|l|}
\hline
\mkl{line width=2pt}& \mkl{ultra thin} &\mkl{very thin} & \mkl{thin} & \mkl{semithick} & \mkl{thick} &\mkl{very thick} &\mkl{ultra thick} \\
\hline
line width=2pt &ultra thin & very thin & thin &semithick & thick & very thick & ultra thick \\
\hline
\end{tabular}
\egroup

\begin{docKey}[tikz]{line cap}{=\marg{dimension}} {no default, initially 0.4pt}
Specifies how lines “end.” Permissible types are round, rect, and butt \seepgfmanual{167}. 
\end{docKey}

\bgroup
\def\mkl#1{\begin{tikzpicture} \draw[line width=10pt, line cap=#1] (0,0)--(1.0, 1.5ex);\draw[white,line width=2pt]
(0,0 )--(1.0,1.5ex);\end{tikzpicture}}
\scriptsize\arial
\begin{tabular}{|l|l|l|}
\hline
\mkl{rect}& \mkl{butt} &\mkl{round}  \\
\hline
rect &butt & round \\
\hline
\end{tabular}
\egroup




\begin{docKey}[tikz]{line join}{=\marg{type}}{no default, initially miter}
Specifies how lines “join.” Permissible type are round, bevel, and miter. They have the following
effects:
\end{docKey}

\begin{texexample}{Joining Lines}{es:joinlines}
\begin{tikzpicture}[line width=10pt]
\draw[line join=round] (0,0) -- ++(.5,1) -- ++(.5,-1);
\draw[line join=bevel] (1.25,0) -- ++(.5,1) -- ++(.5,-1);
\draw[line join=miter] (2.5,0) -- ++(.5,1) -- ++(.5,-1);
\end{tikzpicture}
\end{texexample}


\begin{docKey}[tikz]{dash pattern}{=\marg{dash pattern}}{no default}
Sets the dashing pattern. The syntax is the same as in \metafontlogo. For example following pattern on
2pt off 3pt on 4pt off 4pt means \enquote{draw 2pt, then leave out 3pt, then draw 4pt once more, then
leave out 4pt again, repeat}.
\end{docKey}

\bgroup
\def\ml#1{\tikz \draw[ #1] (0pt,0pt) -- (50pt,0pt);}
\def\alist{solid, dotted, densely dotted, loosely dotted,% 
           dashed,densely dashed, loosely dashed, %
           dash dot, densely dash dot, loosely dash dot, %
           dash dot dot, densely dash dot dot, loosely dash dot dot.}

For patterns there are numerous settings {\arial \alist }


\scriptsize
\begin{tabular}{lll}
\hline
\ml{solid} &  & \\
solid      &  & \\
\hline
\ml{dotted} &\ml{densely dotted} & \ml{loosely dotted}\\
\textit{dotted} & densely dotted  &loosely dotted \\
\hline
\ml{dashed} & \ml{densely dashed} & \ml{loosely dashed}  \\
\textit{dashed}      & densely dashed & loosely dashed            \\
\hline

\ml{dash dot} & \ml{densely dash dot} & \ml{loosely dash dot} \\
\textit{dash dot} & densely dash dot & loosely dash dot \\
\hline

\ml{dash dot dot} & \ml{densely dash dot dot} & \ml{loosely dash dot dot} \\
\textit{dash dot dot} & densely dash dot dot & loosely dash dot dot \\
\hline
\end{tabular}
\egroup


\subsection{Pattern Library}

The library patterns can be used to draw predetermined patterns. This will be a longer than usual section as it explains how to create new patterns. Most of the content is straight from the \pgfname manual. Before we start with the creation f a new pattern let us examine how a pattern is used.

\begin{texexample}{Using Library Patterns}{ex:libpatterns}
\begin{tikzpicture}
\pattern [path fading=west,pattern=checkerboard light gray]
      (0,0) rectangle (5cm,2em);
\end{tikzpicture}
\end{texexample}


\label{section-library-patterns}


The package defines patterns for filling areas. \docAuxCommand*{usetikzlibrary}\marg{patterns}.




\subsection{Form-Only Patterns}

\begin{tabular}{ll}
  \emph{Pattern name} & \emph{Example (pattern in black, blue, and red
    on faded checkerboard)} \\ 
  \patternindex{horizontal lines} 
  \patternindex{vertical lines} 
  \patternindex{north east lines} 
  \patternindex{north west lines} 
  \patternindex{grid} 
  \patternindex{crosshatch} 
  \patternindex{dots} 
  \patternindex{crosshatch dots} 
  \patternindex{fivepointed stars} 
  \patternindex{sixpointed stars} 
  \patternindex{bricks}
  \patternindex{checkerboard}
\end{tabular}
  
\subsection{Inherently Colored Patterns}


\begin{tabular}{ll}
  \emph{Pattern name} & \emph{Example} \\
  \patternindexinherentlycolored{checkerboard light gray} 
  \patternindexinherentlycolored{horizontal lines light gray} 
  \patternindexinherentlycolored{horizontal lines gray} 
  \patternindexinherentlycolored{horizontal lines dark gray} 
  \patternindexinherentlycolored{horizontal lines light blue} 
  \patternindexinherentlycolored{horizontal lines dark blue} 
  \patternindexinherentlycolored{crosshatch dots gray} 
  \patternindexinherentlycolored{crosshatch dots light steel blue} 
\end{tabular}
  


% Copyright 2006 by Till Tantau
%
% This file may be distributed and/or modified
%
% 1. under the LaTeX Project Public License and/or
% 2. under the GNU Free Documentation License.
%
% See the file doc/generic/pgf/licenses/LICENSE for more details.


\section{Creating Patterns}

\label{section-patterns}

\subsection{Overview}

There are many ways of filling a path. First, you can fill it using a
solid color and this is also the fastest method. Second, you can also
fill it using a shading, which means that the color changes smoothly
between two (or more) different colors. Third, you can fill it using a
tiling pattern and it is explained in the following how this is done.

A tiling pattern can be imagined as a rectangular tile (hence the
name) on which a small picture is painted. There is not a single tile,
but (conceptually) an infinite number of tiles, all showing the same
picture, and these tiles are arranged horizontally and vertically to
fill the plane. When you use a tiling pattern to fill a path, what
happens is that the path clips out a ``window'' through which we see
part of this infinite plane.

Patterns come in two versions: \emph{inherently colored patterns} and
\emph{form-only patterns}. (These are often called ``color patterns''
and ``uncolored patterns,'' but these names are misleading since
uncolored patterns do have a color and the color changes. As I said,
the name is misleading\dots) An inherently colored pattern is just a
colored tile like, say, a red star with a black outline. A form-only
pattern can be imagined as a tile that is a kind of rubber stamp. When
this pattern is used, the stamp is used to print copies of the stamp
picture onto the plane, but we can use a different stamp color each
time we use a form-only pattern.

\pgfname\ provides a special support for patterns. You can declare a
pattern and then use it very much like a fill color. \pgfname\
directly maps patterns to the pattern facilities of the underlying
graphic languages (PostScript, \textsc{pdf}, and \textsc{svg}). This
means that filling a path using a pattern will be nearly as fast as if
you used a uniform color.

There are a number of pitfalls and restrictions when using
patterns. First, once a pattern has been declared, you cannot change
it anymore. In particular, it is not possible to enlarge it or change
the line width. Such flexibility would require that the repeating of
the pattern were not done by the graphic language, but on the
\pgfname\ level. This would make patterns orders of magnitude slower
to produce and to render. However, \pgfname{} does provide a
more-or-less successful emulation of ``mutable'' patterns, although
internally, a new (fixed) instance of a pattern is declared when
the parameters of a pattern change.

Second, the phase of patterns is not well-defined, that is, it is not
clear where the origin of the ``first'' tile is. To be more precise,
PostScript and \textsc{pdf} on the one hand and \textsc{svg} on the
other hand define the origin differently. PostScript and \textsc{pdf}
define a fixed origin that is independent of where the path lies. This
has the highly desirable effect that if you use the same pattern to
fill multiple paths, the outcome is the same as if you had filled a 
single path consisting of the union of all these paths. By
comparison, \textsc{svg} uses the upper-left (?) corner of the path to
be filled as the origin. However, the \textsc{svg} specification is a
bit vague on this question.


\subsection{Declaring a Pattern}

Before a pattern can be used, it must be declared. The following
command is used for this:

\begin{docCommand}{pgfdeclarepatternformonly}{%
	\oarg{variables}%
	\marg{name}%
	\marg{bottom left}%
	\marg{top right}%
	\marg{tile size}%
	\marg{code}}

	This command declares a new form-only pattern. The \meta{name} is a
  name for later reference. The two parameters \meta{lower left} and
  \meta{upper right} must describe a bounding box that is large enough
  to encompass the complete tile.
\end{docCommand}

  The size of a tile is given by \meta{tile size}, that is, a tile is
  a rectangle whose lower left   corner is the origin and whose upper
  right corner is given by \meta{tile size}. This might make you
  wonder why the second and third parameters are needed. First, the
  bounding box might be smaller than the tile size if the tile is
  larger than the picture on the tile. Second, the bounding box might
  be bigger, in which case the picture will ``bleed'' over the tile.

  The \meta{code} should be \pgfname\ code than can be protocolled. It
  should not contain any color code.


\begin{codeexample}[]
\pgfdeclarepatternformonly{stars}
{\pgfpointorigin}{\pgfpoint{1cm}{1cm}}
{\pgfpoint{1cm}{1cm}}
{
  \pgftransformshift{\pgfpoint{.5cm}{.5cm}}
  \pgfpathmoveto{\pgfpointpolar{0}{4mm}}
  \pgfpathlineto{\pgfpointpolar{144}{4mm}}
  \pgfpathlineto{\pgfpointpolar{288}{4mm}}
  \pgfpathlineto{\pgfpointpolar{72}{4mm}}
  \pgfpathlineto{\pgfpointpolar{216}{4mm}}
  \pgfpathclose%
  \pgfusepath{fill}
}
\begin{tikzpicture}
  \filldraw[pattern=stars] (0,0)   rectangle (1.5,2);
  \filldraw[pattern=stars,pattern color=red]
                           (1.5,0) rectangle (3,2);
\end{tikzpicture}
\end{codeexample}

	The optional argument \meta{variables} consists of a comma
	separated	list of macros,	registers or keys, representing the
	parameters of the pattern that may vary. If a variable is a key,
	then the full path name must be used (specifically, it must start
	with |/|).
	As an example, a list might look like the following:
	|\mymacro,\mydimen,/pgf/my key|. Note that macros and keys should
	be ``simple''. They should only store values in themselves.
	
	The effect of \meta{variables}, is the following:
  Normally, when this argument is empty, once a pattern has been
  declared, it becomes ``frozen''. This means that it is not possible
  to enlarge the pattern or change the line width later on.
  By specifying \meta{variables}, no pattern is actually created.
  Instead, the arguments are stored away
  (so the macros,	registers or keys do not have to be defined in advance).

  When the fill pattern is set, \pgfname{} checks if the pattern has
  already been created with the \meta{variables} set to their current
  values (\pgfname{} is usually ``smart enough'' to distinguish between
  macros, registers and keys). If so, this already-declared-pattern
  is used as the fill pattern.
  If not, a new instance of the pattern (which will have a
  unique internal name) is declared using the current values of
  \meta{variables}. These values are then saved and the fill pattern
  set accordingly.
	
	The following shows an example of a pattern which varies
	according to the values of the macro |\size|, the key |/tikz/radius|,
	and the \TeX{} dimension |\thickness|.

\begin{texexample}{New Pattern Example}{ex:newpattern}
\pgfdeclarepatternformonly[/tikz/radius,\thickness,\size]{rings}
{\pgfpoint{-0.5*\size}{-0.5*\size}}
{\pgfpoint{0.5*\size}{0.5*\size}}
{\pgfpoint{\size}{\size}}
{
  \pgfsetlinewidth{\thickness}
  \pgfpathcircle\pgfpointorigin{\pgfkeysvalueof{/tikz/radius}}
  \pgfusepath{stroke}
}
\newdimen\thickness
\tikzset{
  radius/.initial=4pt,
  size/.store in=\size, size=20pt,
  thickness/.code={\thickness=#1},
  thickness=0.75pt
}
\begin{tikzpicture}[rings/.style={pattern=rings}]
  \filldraw [rings, radius=2pt, size=6pt]      (0,0)   rectangle +(1.5,2);
  \filldraw [rings, radius=2pt, size=8pt]      (2,0)   rectangle +(1.5,2);
  \filldraw [rings, radius=6pt, thickness=2pt] (0,2.5) rectangle +(1.5,2);
  \filldraw [rings, radius=8pt, thickness=4pt] (2,2.5) rectangle +(1.5,2);
\end{tikzpicture}
\end{texexample}



\begin{docCommand}{pgfdeclarepatterninherentlycolored}{\oarg{variables}
    \marg{name}
    \marg{lower left}
    \marg{upper right}
    \marg{tile size}
    \marg{code}}
  This command works like |\pgfdeclarepatternuncolored|, only the
  pattern will have an inherent color. To set the color, you should
  use \pgfname's color commands, not the |\color| command, since this
  fill is not protocolled.
\end{docCommand}

\begin{texexample}{Inherently Colored}{ex:ingerentlycolored}
\pgfdeclarepatterninherentlycolored{green stars}
{\pgfpointorigin}{\pgfpoint{1cm}{1cm}}
{\pgfpoint{1cm}{1cm}}
{
  \pgfsetfillcolor{green!50!black}
  \pgftransformshift{\pgfpoint{.5cm}{.5cm}}
  \pgfpathmoveto{\pgfpointpolar{0}{4mm}}
  \pgfpathlineto{\pgfpointpolar{144}{4mm}}
  \pgfpathlineto{\pgfpointpolar{288}{4mm}}
  \pgfpathlineto{\pgfpointpolar{72}{4mm}}
  \pgfpathlineto{\pgfpointpolar{216}{4mm}}
  \pgfpathclose%
  \pgfusepath{stroke,fill}
}
\begin{tikzpicture}
  \filldraw[pattern=green stars] (0,0) rectangle (3,2);
\end{tikzpicture}
\end{texexample}



\subsection{Setting a Pattern}

Once a pattern has been declared, it can be used.

\begin{docCommand}{pgfsetfillpattern}{\marg{name}\marg{color}}
  This command specifies that paths that are filled should be filled
  with the ``color'' by the pattern \meta{name}. For an inherently
  colored pattern, the \meta{color} parameter is ignored. For
  form-only patterns, the \meta{color} parameter specifies the color
  to be used for the pattern.
\end{docCommand}
  
\begin{codeexample}[]
\begin{tikzpicture}
  \pgfsetfillpattern{stars}{red}
  \filldraw (0,0) rectangle (1.5,2);

  \pgfsetfillpattern{green stars}{red}
  \filldraw (1.5,0) rectangle (3,2);
\end{tikzpicture}
\end{codeexample}



\endinput
%To summarize, what we have been doing so far is to learn a set of primitive TikZ commands for drawing paths, drawing shapes and labeling them. All TikZ command work by passing options to them. For example to change the above line to an arrow, we just pass the option |->| to the |draw| command.
%

%\begin{tikzpicture}
%  \coordinate [label=left:\textcolor{orange}{$A$}] (A) at (0,0);
%  \coordinate [label=right:\textcolor{orange}{$B$}] (B) at (1.25,0.25);
%  \draw[->,o-stealth] (A)--(B);
%\end{tikzpicture}
%\caption{Effect of the option \protect\texttt{draw[->]}.}

%\emphasis{begin,end,->,draw}
%\begin{teXXX}
%\begin{tikzpicture}
%  ...
%  ...
%  \draw[->,blue] (A)--(B);
%\end{tikzpicture}
%\end{teXXX}
%
%\section*{Relative coordinates}
%\index{TikZ!coordinates, relative}
%A coordinate can be made "relative" by prefixing it with |++|. relative coordinates are useful in many applications.
%\medskip
%
%\noindent The code is simple, except before the coordinate you add the |++| signs. This tells the PGF engine to add the x,y dimensions of the new coordinate to that of its predecessor's. In many instances this is more intuitive and easier to determine.



%\begin{tikzpicture}
%\draw[step=0.5cm,color=gray] (-1,-1) grid (3.5,3);
%\draw[->,red,thick] (0,0) -- ++(1,0) -- ++(0,1) -- ++(-1,0) -- cycle;
%\draw[->,red,thick] (2,0) -- ++(1,0) -- ++(0,1) -- ++(-1,0) -- cycle;
%\draw[arrows=o-stealth,blue] (1.5,1.5) -- ++(1,0) -- ++(0,1) -- ++(-1,0) -- cycle;
%\end{tikzpicture}
%\caption{Example of use of the \protect\texttt{++} to specify relative coordinates.}
%\label{fig:relative}

%\begin{teXXX}
%\begin{tikzpicture}
%  \draw[step=0.5cm,color=gray] (-1,-1) grid (3.5,3);
%  \draw[red,very thick] (0,0) -- ++(1,0) -- ++(0,1) -- ++(-1,0) -- cycle;
%  \draw[red,very thick] (2,0) -- ++(1,0) -- ++(0,1) -- ++(-1,0) -- cycle;
%  \draw[->,red,very thick] (1.5,1.5) -- ++(1,0) -- ++(0,1) -- ++(-1,0) -- cycle;
%\end{tikzpicture}
%\end{teXXX}
%
%Instead of |++| you can also use a single |+|. This also specifies a relative coordinate, but it does not "update"
%the current point for subsequent usages of relative coordinates. Thus, you can use this notation to specify
%numerous points, all relative to the same "initial" point:
%

%\begin{tikzpicture}
%\draw[step=0.5cm,color=gray] (-1,-1) grid (3.5,3);
%\draw[purple, fill=white] (0,0) -- +(1,0) -- +(1,1) -- +(0,1) -- cycle;
%\draw[purple, fill=white] (2,0) -- +(1,0) -- +(1,1) -- +(0,1) -- cycle;
%\draw[purple, fill=white] (1.5,1.5) -- +(1,0) -- +(1,1) -- +(0,1) -- cycle;
%\path (0,0) node [shape=circle,draw]{(0,0)};
%\end{tikzpicture}
%\caption{Example of use of the \protect\texttt{+} to specify relative coordinates.}
%\label{fig:relative1}

%\begin{teXXX}
%  \draw (0,0) -- +(1,0) -- +(1,1) -- +(0,1) -- cycle;
%  \draw (2,0) -- +(1,0) -- +(1,1) -- +(0,1) -- cycle;
%  \draw (1.5,1.5) -- +(1,0) -- +(1,1) -- +(0,1) -- cycle;
%\end{teXXX}
%
%
%Personally, I don't favour this method of specifying co-ordinates, but it can be useful, if you are automating the production of figures through an external script\sidenote{For drawing Bezier curves, the \texttt{+} behaves differently.  You can refer to the PGF Manual for more details.}.
%
%
%\section*{Arrows}
%\index{TikZ>arrows}
%The function |->| creates a tooltip arrow. You can use different arrow tips and there is a long section for them in the PGF manual. You can even define your own.

\bgroup
%\centering
%\begin{tikzpicture}
%  \draw[->] (0,0) -- (2,0);
%  \draw[arrows=o-stealth,blue] (0,-0.3) -- (2,-0.3);
%  \draw[->,o-stealth,orange] (0,-0.6) -- (2,-0.6);
%  \draw[arrows=|-stealth,purple] (0,-0.9) -- (2,-0.9);
%\end{tikzpicture}
%\captionof{figure}{Special arrow endings}
%\label{fig:specials}
\egroup
%
%\emphasis{o,stealth,begin,end,draw}
%\begin{teXXX}
%\begin{tikzpicture}
% \draw[->] (0,0) -- (2,0);
% \draw[arrows=o-stealth,blue] (0,-0.3) -- (2,-0.3);
% \draw[->,o-stealth,orange] (0,-0.6) -- (2,-0.6);
% \draw[arrows=X-stealth,purple] (0,-0.9) -- (2,-0.9);
%\end{tikzpicture}
%\end{teXXX}

%

\begin{verbatim}
\begin{tikzpicture}
% Define the points of a regular pentagon
\path (0,0) coordinate (origin);
\path (0:1cm) coordinate (P0);
\path (1*72:1cm) coordinate (P1);
\path (2*72:1cm) coordinate (P2);
\path (3*72:1cm) coordinate (P3);
\path (4*72:1cm) coordinate (P4);
% Draw the edges of the pentagon
\draw (P0) -- (P1) -- (P2) -- (P3) -- (P4) -- cycle;
% Add "spokes"
\draw (origin) -- (P0) (origin) -- (P1) (origin) -- (P2)
(origin) -- (P3) (origin) -- (P4);
\end{tikzpicture}
\end{verbatim}





\section{Nodes}

A node is a small part of a picture. When a node is created, you provide a position where the node
should be drawn and a shape. A node of shape circle will be drawn as a |circle|, a node of shape |rectangle|
as a rectangle, and so on. A node may also contain same text, which is why they can used nodes to show text.

Finally, a node can get a name for later reference.



\emphasis{node,shape,draw}
\begin{teXXX}
\begin{tikzpicture}
\path ( 0,2) node [shape=circle,draw] {.}
( 0,1) node [shape=circle,draw] {..}
( 0,0) node [shape=circle,draw] {...}
( 1,1) node [shape=rectangle,draw] {....}
(-2,1) node [shape=rectangle,draw] {rectangle (-2,1)};
\end{tikzpicture}
\end{teXXX}
\medskip

\begin{tikzpicture}
\path ( 0,2) node [shape=circle,draw] {1}
( 0,1) node [shape=circle,draw] {\textbf{10}}
( 0,0) node [shape=circle,draw] {\textbf{100}}
( 1,1) node [shape=circle,draw] {\textbf{1000}}
(-2,1) node [shape=circle,draw] {\textbf{10000}};
\end{tikzpicture}

In the above code, this text is empty (because of the
|empty {}|). So, why do we see anything at all at all the nodes? The answer is the draw option for the node operation: It
causes the |shape| around the text" to be drawn. If you have an empty |{}|, PGF still sees the empty space as a character and justs draws around it. The reason is than TikZ automatically adds some space around the text. The amount is set
using the option |inner sep|. So, to increase the size of the nodes. Modifying the example slightly we get.



\begin{tikzpicture}
\path ( 0,2) node [shape=circle,draw] {.}
( 0,1) node [shape=circle,draw] {..}
( 0,0) node [shape=circle,draw] {...}
( 1,1) node [shape=circle,draw] {....}
(-1,1) node [shape=circle,draw] {.....};
\end{tikzpicture}

As you can observe the size of the circle has been adjusted to fit the text that is enclosing it. 
Another way to simply add a node is using the |at| syntax:

\begin{texexample}{The node command}{}
\begin{tikzpicture}
\node at (0,0) [circle, draw] {\textbf{100}};
\node at (1,1) [diamond,draw] {\textbf{100}};
\end{tikzpicture}
\end{texexample}

The \cmd{\node} is an abbreviation of the |\path| node. This is a much shorter syntax than |\path| where one would need to add a lot of redundant move-tos  \seepgfmanual{215}.

If you have many nodes another way of achieving the example outlined above is to use the |\draw| command in comination with node and at.

\begin{texexample}{The node command}{}
\begin{tikzpicture}
\tikz \draw[fill=yellow!80!black]
(0,0) node {first node}
-- (1,1) node[draw, behind path] {second node}
-- (0,2) node[fill=red!20,draw,double,rounded corners] {third node};

\node at (0,0) [circle, draw] {\textbf{100}};
\node at (1,1) [diamond,draw]{\textbf{100}};
\end{tikzpicture}
\end{texexample}

\subsection*{Drawing shapes}

PGF abd \tikzname\ come with a number of predefined shapes:
\begin{itemize}
\item rectangle
\item circle, and
\item coordinate
\end{itemize}


\begin{tikzpicture}
\draw (0,0) circle (1cm);
\draw (0.5,0) circle (0.5cm);
\draw (0,0.5) circle (0.5cm);
\draw (-0.5,0) circle (0.5cm);
\draw (0,-0.5) circle (0.5cm);
\end{tikzpicture}



A circle is specified by providing its center point and the desired radius. The
command:

\medskip

\begin{tikzpicture}
  \draw[step=0.25cm,color=green] (-1,-1) grid (1,1);
  \draw (0,0) circle (1cm);
\end{tikzpicture}
\medskip

\begin{teXXX}
\begin{tikzpicture}
  \draw (x,y) circle (dia);
\end{tikzpicture}
\end{teXXX}



You  can use one |\draw| command to draw multiple circles as shown in \fref{fig:circles}


\begin{tikzpicture} 
 \draw (0,0) 
  circle (1cm)
  circle (0.6cm)
  circle (0.2cm)
 ;
\end{tikzpicture}

\emphasis{circle,begin,end}
\begin{teXXX}
\begin{tikzpicture} 
 \draw (0,0) 
  circle (1cm)
  circle (0.6cm)
  circle (0.2cm)
 ;
\end{tikzpicture}
\end{teXXX}





\begin{center}
\begin{tikzpicture}
\draw (0,0) circle (1cm)
circle (0.6cm)
circle (0.2cm);
\end{tikzpicture}
\captionof{figure}{You can use one draw command to draw multiple circles}
\label{fig:circles}
\end{center}
\captionof{figure}{Drawing multiple circles, using mutiple \texttt{circle} commands}


\subsection{Drawing ellipses}

Ellipses can be drawn in a similar fashion to circles. As an ellipse needs two center points to be specified the command used has the following general form:

\begin{verbatim}
\draw (a,b) ellipse (r1 dim and r2 dim);
\end{verbatim}

We can draw two ellipses as shown in the figure, using the code:
\begin{teX}
\begin{tikzpicture}[scale=0.6]
\draw[color=red] (0,0) ellipse (2cm and 1cm);
\draw[color=red] (0,0) ellipse (1cm and 2cm);
\end{tikzpicture}
\end{teX}

\begin{centering}
\begin{tikzpicture}[scale=0.6]
\draw[color=red] (0,0) ellipse (2cm and 1cm);
\draw[color=red] (0,0) ellipse (1cm and 2cm);
\end{tikzpicture}
\caption[Drawing ellipses]{Use the draw command in combination with ellipse to draw ellipses}
\end{centering}


\begin{teX}
\begin{tikzpicture}
\draw (0,0) ellipse (2cm and 1cm)
ellipse (0.5cm and 1 cm)
ellipse (0.5cm and 0.25cm);
\end{tikzpicture}
\caption{Drawing multiple circles, using mutiple \texttt{draw} commands}
\end{teX}

\section{Drawing more complicated shapes}
we can place a parabola in a rectangle as shown in \fref{fig:parabola}, by using the |rectangle| and the |parabola| options.

\bgroup
\centering

\begin{tikzpicture}
\draw[color=blue] (0,0) rectangle (1,1.5)
(0,0) parabola[color=orange] (1,1.5);
\draw[xshift=1.5cm] (0,0) rectangle (1,1.5)
(0,0) parabola[bend at end] (1,1.5);
\draw[xshift=3cm] (0,0) rectangle (1,1.5)
(0,0) parabola bend (.75,1.75) (1,1.5);
\end{tikzpicture}
\captionof{figure}{Parabolas drawn using the parabola and rectangle options.}
\label{fig:parabola}
\egroup




\emphasis{parabola,rectangle}
\begin{teX}
\begin{tikzpicture}
\draw[color=blue] (0,0) rectangle (1,1.5)
(0,0) parabola[color=orange] (1,1.5);
\draw[xshift=1.5cm] (0,0) rectangle (1,1.5)
(0,0) parabola[bend at end] (1,1.5);
\draw[xshift=3cm] (0,0) rectangle (1,1.5)
(0,0) parabola bend (.75,1.75) (1,1.5);
\end{tikzpicture}
\caption{Parabolas drawn using the parabola command}
\label{fig:parabola}
\end{teX}

\subsection*{The shape library}

\begin{tikzpicture}
\draw [help lines] (0,0) grid (2,2);
\draw [blue, dashed] (1,1) circle(1cm);
\draw [red, dashed] (1,1) circle(.5cm);
\node [star, star point height=.5cm, minimum size=2cm, draw]
at (1,1) {S};
\end{tikzpicture}

\section{Iterations}
One convenient construct provided with TikZ is a |foreach| command sequence

\begin{texexample}{Tikz loops}{tz:ex}
\centering
\begin{tikzpicture}[scale=2, color=bgsexy]
\foreach \i in {1,...,4}
{
  \path (\i,0) coordinate (X\i);
  \fill (X\i) circle (1pt);
}
  \foreach \j in {1,...,3}
{
  \path (\j,1) coordinate (Y\j);
  \fill (Y\j) circle (1pt);
}
\foreach \i in {1,...,4}
{
  \foreach \j in {1,...,3}
  {
     \draw[color=bgsexy] (X\i) -- (Y\j);
  }
}
\end{tikzpicture}
\captionof{figure}{Drawing a bi-partite garph using foreach loops}
\end{texexample}



\section{The pgfplots package}



\subsection{Loading data from files}

Scientific work, especially that associated with research tends to generate
a lot of data. The data would normally come from external applications and stored in files. With |TikZ| one can import the data
by using the word |file|:

\emphasis{addplot,file,x}
\begin{teXXX}
 \addplot file {./raw/wavefunctions/wavefunc\x.dat};
\end{teXXX}

In the example we use a file with a path. The data is saved in
files with the same name but a different ending. We use a |foreach| function to add the ending i.e, the file names are |wavefunc1|, |wavefunc2| and |wavefunc3|. By using external data files and the foreach command it can substantially reduce the amount of text in the macros. This improves debugging and readability.

\begin{texexample}[colback=white]{Loading files}{ex:lfiles}
\centering
\begin{tikzpicture}[scale=0.8]
    \begin{axis}[smooth,
    xlabel=$n$,
    ylabel=$\Theta{j}{n}$]
    \foreach \x in {0,...,2}
    {
        \addplot file {./raw/wavefunctions/wavefunc\x.dat};
    }
    \legend{$j=0$,$j=1$,$j=2$};
    \end{axis}
\end{tikzpicture}
\captionof{figure}{Example plot with data imported from external files, using \texttt{file}}
\end{texexample}


\begin{teXXX}
\begin{tikzpicture}[scale=0.6]
  \begin{axis}[
    xlabel=$n$,
    ylabel=$\Theta{j}{n}$]
    \foreach \x in {0,...,2}
    {
      \addplot file {./raw/wavefunctions/wavefunc\x.dat};
    }
    \legend{$j=0$,$j=1$,$j=2$};
  \end{axis}
\end{tikzpicture}
\end{teXXX}



\section*{Plotting functions}
Functions can be defined for plotting using a variety of methods. They are powerful but generally difficult to remember.



\section{Saving Data to a file}

You can save your data to a file in many ways. One easy way is to use
the \docpkg{filecontents} package. This package extends the LaTeX environment
with the same name, but allows you to overwrite the file {\protect\ctan{filecontents}}.

\begin{teXXX}
\documentclass[justified]{tufte-book}
\usepackage{pgfplots,lipsum,booktabs}
\usepackage{pgfplotstable}
\pgfplotsset{compat=newest}
\usepackage{filecontents*}
\begin{filecontents}{my1.dat}
    Label       value       num
    Integrity     33         4
    Standalone    14         3
    Interface      6         2
    Overall       18         1
\end{filecontents*}
\begin{document}
    your code here ...
\end{document}
\end{teXXX}

It is good practice to keep, such data at the top of your file, although with
the |filecontents| package, they can be inserted anywhere. Sometimes it maybe
easier to have a number of minimal files with the type of charts you using regularly and just update the data on top. In general if the data is entered
by hand rather than generated automatically by software this is a good way
to keep your work tidy.

\newenvironment{Chart}[1][black!70!green]{%
%%  defaults
    \gdef\level##1{Level ##1}
    \def\setchartwidth##1{%
      \def\chartwidth{##1}}%
    \setchartwidth{3.9cm}%
    \def\chartcolor{#1}
    \newcommand\addTitle[2][test]{
    
    
%% For the chart title we set it in a minipage for
%% better control
    \def\charttitle{\minipage{4cm}%
       \footnotesize %
       \centering\textbf{##2}\\##1%
       \endminipage}}%
   \def\xlabel{Completion (\%)}%
%% renders the chart 
    \def\renderChart{%
%%
    \footnotesize%
%%
%%
    \IfFileExists{#1.dat}{Test}{}
   \begin{tikzpicture}
   \begin{axis}[
    xbar, width=\chartwidth,title=\charttitle,
    y=0.5cm, enlarge y limits={true, abs value=0.75},
    xmin=0, xmax=100,enlarge x limits={upper, value=0.25},
    xlabel=\xlabel,
    %ylabel=Label,
    xmajorgrids=true,
    ytick=data,
    yticklabels from table={\dataTable}{Label},
    nodes near coords, nodes near coords align=horizontal
     ]
    \addplot[draw=none, fill=\chartcolor] table [x=value, y=num]
    {\dataTable};
    \end{axis}%
    \end{tikzpicture}}}
{}

\begin{comment}
\begin{figure*}
\centering

\hskip-2cm\begin{Chart}
 \addTitle[Mechanical Systems]{Shangri-la}
 \def\dataTable{SH-mechanical.dat}
 \renderChart
\end{Chart}\hspace{0.3cm}
\begin{Chart}
 \addTitle[FM-200 System]{All areas}
 \def\dataTable{my1.dat}
 \renderChart
\end{Chart}
\begin{Chart}
 \addTitle[Electrical Works]{Merweb}
 \def\dataTable{my6.dat}
 \renderChart
\end{Chart}
\caption{Mechanical Systems Shangrila. Commissioning status}
\end{figure*}


\begin{filecontents*}{my1.dat}
Label     value       num
Integrity         33            4
Standalone      14            3
Interface        6            2
Overall           18            1
\end{filecontents*}

\begin{filecontents*}{SH-mechanical.dat}
Label     value       num
{Fan coil units}       43             8
{Air Handling Units}       13             7 
{CW Pumps}       13             6
{ECU}       11             5
{Pressurization Fans}        15             4
{Smoke Extract Fan}       5             3
{Jet fan}       5             2
{Overall}       12              1
\end{filecontents*}

\begin{filecontents*}{my6.dat}
Label    value         num   other
{Level 7}  50           11   13
L6         90           10   12
L5       80             9    16
L4       90             8    18
L3       70             7    90
L2       80             6    21
L1       70             5    22
\end{filecontents*}

\begin{filecontents*}{carparkventilation.dat}
Label    value         num   other
L5         50           11   13
L4         90           10   12
L3         80           9    16
GR         90           8    18
B1         70           7    90
B2         80           6    21
B3         70           5    22
\end{filecontents*}
%% CO SYSTEM
%% DATA
\begin{filecontents*}{carparkco.dat}
Label    value         num   other
L5         78           7   13
L4         90           6   12
L3         80           5    16
GR         90           4    18
B1         70           3    90
B2         80           2    21
B3         70           1    22
B5         50          {}    {}
\end{filecontents*}

\begin{filecontents*}{carparkco2.dat}
value,   num,   other,
78,       7,   13,
90,       6,   12,
80,       5,    16,
90,       4,    18,
70,       3,    90,
80,       2,    21,
70,       1,    22,
\end{filecontents*}
\end{comment}






















% 

%  \end{document}

% ^^A \chapter{Numbers}
\label{ch:numbers}
\newfontfamily\bonum{texgyreheros-regular.otf}
\section{General principles}

When describing an arithmetic value the terms \textit{number, numeral}, and \textit{figure} are largely interchangeable, though \textit{figure} signifies a numerical symbol, especially any of the ten arabic numbers, rather than a representation in words. \textit{Number} is the general term for both arabic as well as roman figures; standards such as |BS 2961| recommends the term \textit{numeral}, while many people reserve instead for roman figures. Do not use \textit{figure} when confusion between numbers and illustrations may result.

\section{Ranging (lining) and (non-lining) figures }

In typography, two different varieties or 'cuts' of type are used to set
figures. The \emph{old style}, also called \emph{non-ranging} or \emph{non-lining}, has descenders
and a few ascenders: \bgroup\bonum 
abcde 0123456789\egroup. The new style, also called \emph{ranging},
\emph{lining}, or \emph{modern}, has uniform ascenders and no descenders: 0123456789; these are used especially in scientific and technical work.

It is not recommended to mix old- and new-style figures in the same book without special
directions. There are, however, contexts in which mixing is a benefit.

For example, a different style should be used for superior figures indicating
editions or manuscript sigla, to avoid confusion with cues for
note references.

The |phd| package loads appropriate default fonts for the type of document being used. Normally it defaults to old style numerals for the text and for lining figures, in tables and sectioning commands. (See also Chapter~\ref{ch:fonts}  Page~\pageref{ch:fonts} for a more technical discussion on fonts.) 

When the |phd| package is used with |XeLaTeX| it loads |fontspec|, which provides settings for not only lining and old style figures, but also if the font has the appropriate glyph has an option |SlashedZero|. This is mostly used for publications describing computer related subjects and where there must be a distinction between the letter `O', the normal zero `0' and the slashed zero. 

\begin{comment}
\begin{texexample}{Lining and Old style Numerals}{oldstyle}
\bgroup
\def\temp{
\fontspec[Numbers=Lining]{TeX Gyre Bonum}
0123456789

\fontspec[Numbers=SlashedZero]{TeX Gyre Bonum}
0123456789

\fontspec[Numbers=OldStyle]{TeX Gyre Bonum}
abcde 0123456789
}
\ifxetex 
\temp
\else
  abcde \oldstylenums{0123456789}
\fi
\egroup
\end{texexample}
\end{comment}

\section{Figures and Words}

One common question that comes in publications dealing with guidelines for writing, is when to use words for figures. In non-technical contexts, OUP style recommends the words for numbers below
100. When a sentence contains one or more figures of 100 or above,
however, use arabic figures throughout for consistency within that
sentence: print for example 90 to 100 (not \textit{ninety} to 100), 30, 76, and 105
(not \textit{thirty}, \textit{seventy-six}, and 108). This convention holds only for the sentence
where this combination of numbers occurs: it does not influence
usage elsewhere in the text unless a similar situation exists.
However, clarity for the reader is always more important than blind adherence
to rule, and in some contexts a different approach is necessary. 

For example, it is sometimes clearer when two sets of figures are mixed to
use words for one and figures for the other, as in thirty 10-page pamphlets,
nine 6-room flats. This is especially useful when the two sets run throughout
a sustained expanse of text (as in comparing quantities):
\begin{quote}
The manuscript comprises thirty-five folios with 22 lines of writing, twenty
with 21 lines, and twenty-two with 20 lines.
\end{quote}

Anything more complicated, or involving more than two sets of quantities,
is probably presented more clearly in a table.

In technical contexts, it is preferable  spelling out numbers below
ten. Similar rules govern this convention: in a sentence containing
numbers above and below ten, style the numbers as figures rather
than words.

Use figures with all abbreviated forms of units, including units of time,
and with symbols:


|6'2" 250 BC 11a.m. 13 mm|

For units also see page~\pageref{units}, where we describe the use of the |siunitx| package. Spaces between numbers and units are important and also the package introduces, a non-breaking space, preventing printing the number in a different line during hyphenation.

\section{Numbers and Punctuation}

In nontechnical contexts, use commas in numbers of more than four
figures. Although optional, it is OUP style to use the comma in figures up
to 9,999, such as 1,863 or 6,523. Do not use a comma to separate groups of
three digits in technical and foreign-language work:
Continental languages,
and International Standards Organization (ISO) publications in
English, use a comma to denote a decimal sign, so that 2.3 becomes 2,3.

Use instead a thin space where necessary to separate numbers with five
or more digits either side of the decimal point:
14 785 652 1000000 3.141592 65 0.000 025.

Four-digit figures are not split with thin spaces—3.1416—except in
tabular matter, where four-digit figures are aligned with larger numbers (see the Chapter on Tables at page~\ref{ch:tables}).

Use the |siunitx| for consistency. The separator used between groups of digits is stored by the group-separator option.
This takes literal input and may be used in math mode: for a text-mode full space use the (\texttt{~}).

\begin{texexample}{Printing numbers}{}
\text{~}.
\num{12345} \\
\num[group-separator = {,}]{12345} \\
\num[group-separator = \text{~}]{12345}
\end{texexample}

For very large numbers use the \cmd{\numprint} from the \pkg{numprint} package \citep{numprint}. This package is also loaded automatically by |phd|. The options provided are not as extensive as those of the |siunitx| package, but it handles huge numbers robustly: \numprint{34567890768966645345}. 

\section{Ranges}

For a span of numbers generally, use an en rule, eliding to the fewest
number of figures possible: 30-1, 42-3, 132-6, 1841-5. But in each hundred
do not elide digits in the group 10 to 19, as these represent single
rather than compound numbers: 10-12,15-19,114-18, 214-15, 310-11.

For larger ranges give only the last two digits of the second number unless more are necessary.

\begin{longtable}{ll}
98-103    &923-1,003\\
103-05    &1,005-12\\
567-892   &1,669-1,722\\
\end{longtable}

The MLA manual \citep{MLA} recommends that for years beginning in AD 1000 or later, omit the first two digits of the second year if they are the same as the first two digits of the first year. Otherwise, write both years in full.

2005-09

1867-1901

For years that begin before AD 1 do not abbreviate any ranges.

741-560 BC

\BC{142}-\AD{158}


When describing a range in figures, repeat the quantity as necessary to
avoid ambiguity: 1~thousand to 2 thousand litres, 1 billion to 2 billion light years
away. The elision 1 to 2 thousand litres means the amount starts at only
1 litre, and 1 to 2 billion light years away means the distance begins only
1 light year away. Add non breaking spaces to avoid splitting |1~thousand| and similar words.


Use a comma to separate successive references to individual page
numbers: 6, 7, 8; use an en rule to connect the numbers if the subject
is continuous from one page to another: 6-8. OUP prefers references to
provide exact page extents; where this is impossible, print 51 f. if the
reference extends only to page 52, but 51 ff. if the reference is to more than one following page. For scientific work the \textit{folio} is not preferred and rather use |pages| or abbreviated forms \textit{pg.}. 


 For lists and ranges when a single unit is given, \pkg{siunitx} will
 automatically \enquote{compress} exponents when a fixed exponent is in use.

\begin{texexample}{Printing ranges with siunitx}{ex:ranges}
  \sisetup{
    fixed-exponent      = 3        ,
    list-units          = brackets ,
    range-units         = brackets ,
    scientific-notation = fixed
  }%
  \SIrange{1e3}{7e3}{\metre} \\
  \SIlist{1e3;2e3;3e3}{\kg}
\end{texexample}

If your document uses a lot of numbers and units, it will pay you to study the \pkg{siunitx}.

\section{Fractions}

This is an area where \tex excels and where possible always use math mode.

In statistical matter use one-piece (cased) fractions where available (\textonehalf). If these are not available, use split fractions (e.g. $2/3$).

In non-technical running text, set complex fractions in font-size numerals
with a solidus (\thinspace\textfractionsolidus\thinspace) between (19/100). Known also as 'shilling 
7 I Numbers 1 71
tions', they represent a quantity without needing extra interlinear space
to be displayed on more than one line. Decimal fractions are similarly
useful: 12.66 rather than $12\frac{2}{3}$, 99.9 rather than 99 and 9/10.

In mathematical texts the tradition is to use a \textit{virgule} and not a solidus.

\[ a/b + c/d + e/f = 1\]

Spell out simple fractions in textual matter, for example one-half, two-thirds,
one and three-quarters. Hyphenate compounded numerals in compound
fractions such as nine thirty-seconds, forty-seven sixty-fourths; the
numerator and denominator are hyphenated unless either already contains
a hyphen. Do not use a hyphen between a whole number and a
fraction: twenty-six and nine-tenths. Combinations such as half a mile, half a
dozen contain no hyphens, but write half-mile, half-dozen, etc. 

If at all possible, do not break spelt-out fractions at line endings.

\subsection{The xfrac package}

The |phd| package loads the package \pkg{xfrac} by default. The package was developed using \latex3 and offers an interface of declaring fonts via the \cmd{\DeclareInstance}. What the package attempts to do is to provide user commands for more granular choices for text fractions. For maths the recommendation is to leave everything to the \tex engine.

\DeclareInstance{xfrac}{cmr}{text}
{slash-symbol-font = ptm}

The package provides the command \cmd{\sfrac}, which produces text fractions such as \sfrac{7}{9} which you can compare with their maths siblings $\sfrac{7}{9}$. The package was developed by the \latex team, but was primarily the effort of Wills Robertson \citep{xfrac}. 

\section{Currency}

\subsection{The Euro}
\index{currency!euro}
The \textit{euro} can be typeset using the command \cmd{\texteuro} which produces the euro symbol (\texteuro). It can also be entered directly if you are using the XeLaTeX or LuaLaTeX engines. 

\subsection{British pounds}
\index{currency!pound}\index{currency!pennies}
Amounts in whole pounds should be printed with the £ symbol, numerals
and unit abbreviation close up: £2,542, £3m., £7.47m. Print 00
after the decimal point only if a sum appears in context with other
fractional amounts: They bought at £8.00 and sold at £9.50.

Amounts in pence are set with the numeral close up to the abbreviation,
which has no full point: 56p rather than £0.56. Mixed amounts
always extend to two places after the decimal point, and do not include
the pence abbreviation: £15.30, £15.79.

Amounts expressed in pre-decimal currency---£.s.d. (italic)---will
continue to be found in copy and must be retained. They will naturally occur in resetting books published before 1971, and in new books in
which the author refers to events and conditions, or quotations from
work dating, before 1971. For example:

\begin{quote}
In 1969 income tax stood at 8s. 3d. in the £.\\
The tenth edition cost 10s. 6d. in 1956.
\end{quote}

In new books, and in annotated editions of reset works, a decision must
be made as to whether to introduce decimal equivalents, for elucidation
or for ease of comparison (e.g. in statistics). Note the distinction in the
pound symbol between the earlier style of, for example, `£44. 3s. 10d.'
and the earlier style of `44l. 3s. 10d.'. In both these styles a normal space
of the line separates the elements. Note the spelling fourpence, ninepenny, etc. for amounts in pre-decimal pennies.  

\section{US currency}

\index{currency!US}
Sums of money in dollars and cents are treated like those in pounds and
pence: \$4,542, \$3m., \$7.47 m., 56c. In older books one can find ``cents'' abbreviated as ``\textcent'', but this is no more common. The \cmd{\textcent} can be used to typeset the \textcent symbol. There is no need to load any packages, as the |phd| package will load the appropriate package based on the \TeX engine used.

\section{Old currency Symbols}
\index{currency!denarius}
\index{currency!florin}
\index{\string \denarius}

The Denarius (\Denarius) and Florin (\Florin) glyphs can be found both in |UTF8| fonts as well as using packages with |pdfLaTeX|. The package \pkg{marvosym} \citep{marvosym} provides many such commands and they belong to specialized books, although now and then one can find such symbols being used for mathematics. You can view many of these symbols in the implementation part of this manual at Page~\pageref{currencysymbols}.

\section{Calendar}

Variations in time-reckoning systems between cultures and eras can
lead both author and editor to error. The following section offers some
guidance for those working with unfamiliar calendars; fuller explanation
may be found in Blackburn and Holford-Stevens, \textit{The Oxford Companion
to the Year}

\subsection{Old and New Style}

These terms are often applied to two different sets of facts. In 1582 Pope
Gregory XIII decreed that, in order to correct the Julian calendar, the
days 5-14 October ofthat year should be omitted and no future centennial
year (e.g. 1700, 1800, 1900) should be a leap year unless it was
divisible by four (e.g. 1600, 2000). This reformed 'Gregorian' calendar
was quickly adopted in Roman Catholic countries, more slowly elsewhere:
in Britain not till 1752 (when the days 3-13 September were
omitted), in Russia not till 1918 (when the days 16-28 February were
omitted). The discrepancy between the Julian and Gregorian calendars
was ten days until 28 February/10 March 1700, eleven days from 29
February/11 March 1700 to 28 February/11 March 1800, twelve days
from 29 February/12 March 1800 to 28 February/12 March 1900, and
thirteen days from 29 February/13 March 1900; it will become fourteen
days on 29 February/14 March 2100.

Until the middle of the eighteenth century, not all states reckoned the
new year from the same day: whereas France (which had previously
counted from Easter) adopted 1 January from 1563, and Scotland from
1600, England counted from 25 March in official usage as late as 1751; so
until 1749 did Florence and Pisa, but whereas Florence, like England,
counted from 25 March AD 1, Pisa counted from 25 March 1 BC, SO that the
Pisan year was always one higher than the Florentine. Thus the execution
of Charles~I was officially dated 30 January 1648 in England, but 30
January 1649 in Scotland. In Florence---which used the Gregorian calendar---
the same day was 9 February 1648, in France and Pisa 9 February
1649. Furthermore, although both Shakespeare and Cervantes died on 23
April 1616 according to their respective calendars, 23 April in Spain (and
other Roman Catholic countries) was only 13 April in England, 23 April in
England was 3 May in Spain, and in Pisa the year was 1617.

Confusion is caused in English-language writing by the adoption, in
England, Ireland, and the American colonies, of two reforms in quick
succession. The year 1751 began on 1 January in Scotland and on 25
March in England, but ended throughout Great Britain and its colonies
on 31 December, so that 1752 began on 1 January. So, whereas 1 January
1752 corresponded to 12 January in most Continental countries, from 14
September onwards there was no discrepancy.
As a result, many writers treat the two reforms as one, using Old Style
and New Style indiscriminately for the start of the new year and the
form of calendar, even with reference to countries in which the two
reforms were adopted at different times. This is unfortunate: Old Style
should be reserved for the Julian calendar and New Style for the Gregorian;
the 1 January reckoning should be called 'modern style' (or 'Circumcision
style'), that from 25 March 'Annunciation' or 'Lady Day' style (or
'Florentine' and 'Pisan' style with reference to those cities), and others
as appropriate, such as 'Easter style', 'Nativity style', 'Venetian style' =
1 March, 'Byzantine style' = 1 September

\subsection{Typesetting old and new style dates}

It is customary to give dates in Old or New Style according to the system
in force at the time in the country chiefly discussed; if the system may be
unfamiliar to the reader, a brief explanation should be added.

The OUP recommends that any dates
in the other style should be given in parentheses with an equals sign
preceding the date and the abbreviation of the style following it: 23
August 1637 NS(=13 August OS) in a history of England, or 13 August 1637
OS (= 23 August NS) in one of France. In either case, 13/23 August 1637 may
be used for short, but when citing documents take care to use this form
only when the original itself employs it. On the other hand, it is normal
to treat the year as beginning on 1 January: modern histories of England
date the execution of Charles I to 30 January 1649. When it is necessary to
keep both styles in mind, it is normal to write 30 January 1648/9, subject to
the same qualification when citing documents; otherwise the date
should be given as 30 January 1648 (= modern 1649). (Contemporary accounts
could manage this as well: George Washington's date of birth was
recorded in the family Bible as ye 11th day of February 1731/2.) 

Original
documents may exhibit the split fraction, for example 172\sfrac{1}{2}, this should
be used only when exact transcription is required. For dates in Pisan style
between 25 March and 31 December inclusive, write 15 August 1737/6. 











% ^^A 
\cxset{lineskip/.code=\setlength\lineskip{#1},
       lineskip/.default=1pt,
          normallineskip/.code=\setlength\normallineskip{#1},
          parindent/.code=\setlength\parindent{#1},
          parskip/.code=\setlength\parskip{#1},
          text-indent/.code=\setlength\parindent{#1},
          baselinestretch/.code=\renewcommand\baselinestretch{#1},
          single spacing/.code=\singlespacing,
          single spacing/.default=\singlespacing,
          double spacing/.code=\doublespacing}

\cxset{lineskip=1pt,
          normallineskip=1pt,
          parindent=1em,
          parskip=1pt,
          text-indent=1em,
          baselinestretch={},
          single spacing}

\makeatletter\@specialtrue\makeatother
\cxset{steward,
  numbering=arabic,
  custom=stewart,
  offsety=0cm,
  image={./images/hine05.jpg},
  texti={When Lamport designed the original \LaTeX\ sectioning commands, limitations of computer power forced him to restrict the abstraction of complicated chapter layouts. With current tools available improvements are much easier to program.},
  textii={In this chapter we discuss a method that allows the production of fancy chapter headings and formatting, based on a set of key values. Central  to this process is the separation of content from presentation.
We also discuss the basic formatting tools that are available and how one can modify them to mould new book designs.
 }
}
\cxset{chapter opening=left}

\chapter{General Settings}

\section{Introduction}

Here we define and set general paragraph settings. The parameters which control \TeX's behaviour when typesetting paragraphs can receive a bit of a tweak here. We also describe a set of options to handle parameters that can influence grid typesetting. This is especially important for two or more column typesetting. The commands act only on the text within a grouped environment. They do not affect captions or footnotes. Use anything over \emph{single spacing} with care, as books are meant to be single spaced.  



\section{Controlling inter-line spacing}
\index{line spacing}
Interline spacing traditionally has been controlled using the \pkgname{setspace} or by setting appropriate primitive \tex commands \cite{setspace}. The \pkgname{phd} loads the |setspace| package and then provides parameterized commands for setting styles. 

\begin{key}{/chapter/single spacing} 
	The Lineskip parameter emulates \TeX's \cmd{\parindent} command.
\end{key}
\begin{key}{/chapter/one half spacing} 
	The Lineskip parameter emulates \TeX's \cmd{\parindent} command.
\end{key}
\begin{key}{/chapter/double spacing} 
	Sets the document line-spacing to double.
\end{key}

If you want to use larger inter-line spacing in a document, you can change its value by putting the

\CMDI{\linespread}\meta{factor} Use |\linespread{1.3}| for "one and a half" line spacing, and |\linespread{1.6}| for "double" line spacing. Normally the lines are not spread, so the default line spread factor is~1.

The setspace package allows more fine-grained control over line spacing. To set "one and a half" line spacing document-wide, but not where it is usually unnecessary (e.g. footnotes, captions):

\begin{teXXX}
\usepackage{setspace}
%\singlespacing
\onehalfspacing
%\doublespacing
%\setstretch{1.1}
\end{teXXX}

The |phd| package provides the settings

\begin{key}{/chapter/single spacing}
We use the \pkgname{setspace} to effect the desired line spread effect.
\end{key}


These command offer little value over the normal \TeX\ macros other than keeping the interface, uniform. One can also extend the interface to cover CSS style commands:

\begin{verbatim}
\cxset{text-indent=50pt}

\cxset{double spacing}
\lipsum*[1]

\cxset{single spacing}
\lipsum*[1]
\end{verbatim}



\subsection{Parameters controlling paragraphs}\index{Paragraphs!controlling parameters}
The parameters \cs{lineskip} and \cs{normallineskip} influence \TeX\ when two lines come two close.
\medskip



\begin{key}{/chapter/lineskip=1pt} 
	The Lineskip parameter emulates \TeX's \cmd{\lineskip} command.
\end{key}

\begin{key}{/chapter/normallineskip=\marg{dim}} 
	The normallineskip parameter emulates \TeX's \cmd{\normallineskip} command.
\end{key}

\begin{key}{/chapter/lineskiplimit=\marg{dim}} 
	The Lineskip parameter emulates \TeX's \cmd{\lineskiplimit} command.
\end{key}

\begin{key}{/chapter/parindent=\marg{dim}} 
	The Lineskip parameter emulates \TeX's \cmd{\parindent} command.
\end{key}

\keyval{parindent}{\marg{dim}}{Paragraph indentation.}
\keyval{text-indent}{\marg{dim}}{Alias for \cs{parindent}.}
\keyval{parskip}{\marg{dim}}{Spacing between paragraphs.}


Another advantage, the package offers a few pre-configured styles, just setting a style to latex will revert everything back to latex.

\section{Technical discussion}

Most classes, including the standard \LaTeXe\ classes as well as packages attempting to achieve a grid typesetting try define a text height that is a multiple of \cs{baselineskip}. This way they give little opportunity to TeX to adjust the vertical glue to achieve a flush bottom.

\section{Dropcaps and Lettrines}\index{Lettrine!basic typesetting}

Dropcaps or lettrines are those letters that start paragraphs with a fancy larger letter. The class uses a parameterized version of the lettrine package of Daniel Flipo. Lettrine letters are easily typed and produced, but they are notoriously difficult to get right and no-one seems to agree on settings. These settings depend on the font the sizing of the text and the personal taste of the book interior designer. As I don't profess to be one, I have done what I think Knuth have done (just studied existing sources) allowed programming hooks and provided defaults as close as possible to the originals.




%  \part{The Structural Elements of Documents}
%
%\end{document}




%\end{document}
%  %\chapter{Typesetting Mathematics}
%\epigraph{Perhaps some day a typesetting language will become standardized to the 
%point where papers can be submitted to the American Mathematical Society 
%from computer to computer via telephone lines. Galley proofs will not be 
%necessary, but referees  and / or copy editors could send suggested changes to 
%the author, and he could insert these into the manuscript, again via telephone. }{Donald Knuth, 1979}
%
%%\renewcommand\figurename{\bf Fig.\thinspace }
%%\begin{marginfigure}%
%%\captionsetup[marginfigure]{margin=10pt,font=small,labelfont=bf}%
%%  \hskip -10pt\relax\includegraphics[width=1.5\linewidth]{./graphics/maths1}
%%  \captionof{figure}{Bookcases in the library of the University of Leiden: from a print by J.C. Woudanus, dated 1610. (Lent by the Syndics of the University Press.)}
%%  \label{fig:marginfig1}
%%\end{marginfigure}
%
%Most people discover, \alltex when they are faced with the production of a thesis or paper that includes lots of
%mathematical text. It is the rais\`on detrait for \tex. In this chapter we will discuss the typesetting of mathematics, common pitfalls and solutions. We will also review some typographical questions.
%
%You can start writing mathematical text, without any additional loading of packages, either in plain \tex or \latex. The reality is that most institutions have developed their own styles and common packages used by AMS have found widespread use. We will discussing these extensively, but first let us look at how to enter mathematical text. To enter mathematical text in plain \tex you just use the |\$|\ldots|\$| for inline math and |\$\$|\ldots|\$\$| for displayed math.
%% Example template kroll01


\chapter{maths}

\parindent1em

\starttemplate{kroll}
\thispagestyle{empty}
    \begin{leftcolumn}
       %\MainHeader{Leon\\[15pt] Kroll}
      {{\centering \huge  THE ART OF \\
       TYPESETTING\\
       MATHS\\}}
      \medskip
       {\justifying \small Perhaps some day a typesetting language will become standardized to the 
point where papers can be submitted to the American Mathematical Society 
from computer to computer via telephone lines. Galley proofs will not be 
necessary, but referees  and / or copy editors could send suggested changes to 
the author, and he could insert these into the manuscript, again via telephone.\par
\hfill \textit{Donald Knuth}}
\medskip
       \putimage[width=1.0\linewidth]{halmos}\par
       \aheader{shows Kroll at 59. Says he. ``Painting is fascinating'' even when motif my own mug.}
   \end{leftcolumn}
   \begin{rightcolumn}
       \putimage[width=\linewidth]{themathematician}
       \onelinecaption{{\resizebox{\linewidth}{5.5pt}{\bfseries The Mathematician (1918) an oil painting by the Mexican artist Diego Rivera. }}\par}

     \vspace*{1.5\baselineskip}

  \centerline{\onelineheader{TYPESETTING MATHEMATICS}}
      \begin{multicols}{2}
      \small
      \lettrine{M}{ost people} discover, \alltex when they are faced with the production of a thesis or paper that includes lots of
mathematical text. It is the rais\`on detrait for \tex. In this chapter we will discuss the typesetting of mathematics, common pitfalls and solutions. We will also review some typographical questions.
\parindent1em

You can start writing mathematical text, without any additional loading of packages, either in plain \tex or \latex. The reality is that most institutions have developed their own styles and common packages used by AMS have found widespread use. We will discussing these extensively, but first let us look at how to enter mathematical text. 

To enter mathematical text in plain \tex you just use the |$|\ldots|$| for inline math and |$$|\ldots|$$| for displayed math. We now illustrate, the usage with and example.
      \end{multicols}
   \end{rightcolumn}
\stoptemplate


\newgeometry{left=2.5cm,right=2.5cm}
\pagestyle{headings}
\definecolor{shadecolor}{rgb}{0.9,0.9,0.9}

\section{Inline and Display Math}
Mathematical typesetting involves either inline math formulae, which are part of a paragraph of text or display math which are a block of mathematical material. \tex will typeset inline math by enclosing the material between 
\$...\$. 
\bigskip

\begin{tcblisting}{colback=blue!5,boxrule=2pt,colframe=blue!75!black,title=\textbf{TeX style inline and display math},width=0.75\textwidth}
This is an inline equation $a^2+b^2=c^2$

And this equation is a display equation:
$$a^2+b^2=c^2$$
\end{tcblisting}
\bigskip

The above code, is valid in \latex and its variants as well. However in certain cases, especially for displayed math, this can introduce some blank lines. Lamport redefined
the \$\$. For inline math use |\(...\)| and for display math |\[...\] |. The above example can be written as:
\bigskip

\begin{tcblisting}{colback=blue!5,boxrule=2pt,colframe=blue!75!black,title=Basic Definitions,width=0.75\textwidth}
This is an inline equation \(a^2+b^2=c^2\)

And this equation is a display equation:
\[a^2+b^2=c^2\]

\begin{math}
\sum_{i=1}^{n}i=\frac{1}{2}n\cdot(n+1)
\end{math}
\end{tcblisting}
\bigskip


As you can observe the output remains the same.  Spaces within  maths environment are ignored, so use this to your advantage when writing maths. As mentioned earlier, we can use |\[...\]| for display math.

We can now try a rather longer example, before getting into more details:
\begin{shaded}
\begin{teXXX}
\[ \sum a_1+a_2+\dots a_n \]
\end{teXXX}
\[ \sum a_1+a_2+\dots a_n \]
\end{shaded}

\section{Displayed Equations}
Displayed equations, can either be displayed \emph{flushed left} or \emph{centered}, depending on the option loaded with the standard classes, for example in article use

\begin{teX}
\documentclass[imperial,11pt,openany, twoside,fleqn]{octavo}[2005/09/16]
\end{teX}

We will see, how to change the formatting a bit later on.

\section{Greek letters}

Mathematicians and Engineers, quickly run out of symbols to use with their equations and hence use Greek letters. All
the Greek letters are available in \tex{} 

\begin{table}[htbp]
\centering
\begin{tabular}{llllllll}
\toprule
$\alpha$  &\doccmd{alpha} &$\beta$ &\doccmd{beta} &$\gamma$ &\doccmd{gamma} &$\delta$ &\doccmd{delta}\\
$\epsilon$  &\doccmd{epsilon} &$\varepsilon$ &\doccmd{varepsilon} &$\zeta$ &\doccmd{zeta} &$\eta$ &\doccmd{eta}\\
\bottomrule
\end{tabular}
\end{table}



Sometimes accents are put above or below symbols. The control words used for accents
in mathematics are different from those used for normal text. The normal text control words
may not be used for mathematics and vice-versa.\sidenote{\texbook 135-136 }. See also
\href{mathmode.pdf}{http://www.tex.ac.uk/tex-archive/info/math/voss/mathmode/Mathmode.pdf}

\section{Fractions}

There are two ways of typesetting a fraction in \tex{}: it can be typeset as $1/4$ or in the form $\frac{1}{4}$. The first case is entered with no special control characters that is,  \verb+ $1/2$+. The second case is just entered with the control word \cmd{over}.  Hence\verb+ ${1} \over {4}$+ gives ${1} \over {4}$. \LaTeX\ provides a macro \cmd{frac}.

A more complex example,

\begin{teX}
\[
a+\gamma \over \delta^2
\]
\end{teX}

\[a+\gamma \over \delta^2\]

One complication with using maths in-line with text is that of using different type of fonts. There is a very useful and interesting discussion in the package \pkg{xfrac}. 

 One of the first exercises in \emph{The \TeX Book} is to design a
 macro for split level fractions. The solution presented is fairly
  simple, using a \emph{virgule} (a slash) for separating the two
  components. It looks okay because the text font and math font of
  Computer Modern look almost identical.\index{virgule}

  The proper symbol to use instead of the virgule is a \emph{solidus}\sidenote{The solidus (/) \index{solidus} is a punctuation mark used to indicate fractions including fractional currency. It may also be called a shilling mark, an in-line fraction bar, or a fraction slash. Its Unicode encoding is \texttt{U+2044}.

The solidus is similar to another punctuation mark, the slash, which is found on standard keyboards; the slash is closer to being vertical than the solidus. These are two distinct symbols that traditionally have entirely different uses. However, many people no longer distinguish between them, and when there is no alternative it is acceptable to use the slash in place of the solidus.
Both the ISO and Unicode designate the solidus as \texttt{FRACTION SLASH U+2044} and the slash as \texttt{SOLIDUS U+002F}. This contradicts long-established English typesetting terminology (See Bringhurst.}
  which does not exist in Computer Modern. It is however available in
  the European Computer Modern fonts, but I'll get back to that.

  The most common way to produce split level fractions within \LaTeX\
  is by means of the \docpkg{nicefrac} package. Part of the reason it
  has found widespread use is due to the strange design of the
  built-in text fractions of the EC fonts, which look like this:
  \textonehalf. The package is very simple to use but there are a few
  issues:

 \begin{itemize}
  \item It uses the virgule instead of the solidus.
  \item Font size of numerator and denominator is bigger than in the
    built-in symbol. Compare Palatino: \switch{ppl}{\nicefrac{1}{2}}
    vs. \switch{ppl}{\textonehalf }. (\sfrac{1}{2})

  \item It doesn't correct for fonts using text figures such as in the
    \docpkg{eco} package. Compare \switch{cmor}{\nicefrac{1}{2}} and
    \switch{cmor}{\nicefrac{8}{9}}.
  \item In math mode, it doesn't always pick up the correct math
    alphabet.
 \end{itemize}
 In short: \docpkg{nicefrac} doesn't attempt to be the answer to
 everything and so this is not a criticism of the package. It works
 quite well for Computer Modern which was pretty much what was widely
 available at the time it was developed. Users these days, however,
 have a choice of many fonts when they write their documents.

With |xfrac| we can switch fonts easily

 ``You take \sfrac[cmr2]{1}{2} cup of sugar, \ldots''




\section{Square roots}

The square root sign, which historically derived from a dot is now simply typeset a square root it is only necessary to use the construction \cmd{sqrt}. Hence

\begin{teX}
$\sqrt{x^2+y^2}$
\end{teX}

will give $\sqrt{x^2+y^2}$


Notice that \tex takes care of the placement of
symbols and the height and length of the radical. To make cube or other roots, the control
words \cmd{root} and \cmd{of} are used. You get $\root n \of {1+x^n}$ from the input 

\begin{teX}
   $\root n \of  {1+x^n}$ 
\end{teX}

A possible alternative is to use the control word \cmd{surd}; the input \verb+ $\surd 2$+ will
produce $\surd 2$.

When typesetting square roots care should be taken, to use struts appropriately to get the sizing right.

\section{Trigonometric and other Functions}
There are several types of functions that appear frequently in mathematical text. In
an equation like $\sin2x+\cos2x=1$ the trigonometric functions \cmd{sin} and \cmd{cos} are in
roman rather than italic type. This is the usual mathematical convention to indicate that
it is a function being described and not the product of three variables. The control words
\doccmd{sin} and \doccmd{cos} will use the right typeface automatically. Here is a table of these and some
other special functions:

\begin{teX}
\sin \cos \tan \cot \sec \csc \arcsin \arccos
\arctan \sinh \cosh \tanh \coth \lim \sup \inf
\limsup \liminf \log \ln \lg \exp \det \deg
\dim \hom \ker \max \min \arg \gcd \Pr
\end{teX}

\begin{tcblisting}{colback=blue!5,boxrule=2pt,colframe=blue!75!black,title=\textbf{TeX style inline and display math},width=0.75\textwidth}
\[ \cos(2\theta) = 2 \cos^{2}2 \theta-1\]
\end{tcblisting}

Functions that are missing from a basic instllation can be either defined or one can use one of the many packages that supplement the above.

\clearpage

\section{Using special symbols}

Special symbols such as dingbats loaded using the |pifont| package need to be encloded within an |\mbox| in order to be able to display the glyph properly.
\bigskip

\begin{tcblisting}{colback=blue!5,boxrule=2pt,colframe=blue!75!black,title=\textbf{DINGBATS},width=0.9\textwidth}
\label{e14}
 We start by showing that the function $f(x)=x^2$ is
continuous over the set $X_2$\label{p:X2} defined as the interval
$[0,1]$ where numerals $\frac{i}{\mbox{\ding{172}}}, 0 \le i \le
\mbox{\ding{172}},$ are used to express its points in units $\mu$.
First of all, note that the set $X_2$ is continuous in   $\mu$
because its points are equidistant with the distance
$d=\mbox{\ding{172}}^{-1}$. Since this function is strictly
increasing,  to show its continuity it is sufficient to check the
difference $f(x)-f(x^{-})$ at the point $x=1$. In this case,
$x^{-}=1-\mbox{\ding{172}}^{-1}$ and we have
\[
 f(1)-f(1-\mbox{\ding{172}}^{-1})=
 1-(1-\mbox{\ding{172}}^{-1})^2 =
 2\mbox{\ding{172}}^{-1}(-1)\mbox{\ding{172}}^{-2}.
\]
This number is infinitesimal, thus $f(x)=x^2$ is continuous over
the set $X_2$. \hfill $\Box$
\end{tcblisting}
\bigskip

\cs{Box}
Notice the use of the |\Box| command to draw a square for  end of proof symbol. This is from the amsmath package. The box is placed at the end of the line using |\hfill $\Box$|.



\section{Partial derivatives}

Partial derivatives can be typeset using the \latex{} command \cmd{partial}
\begin{teXX}
\[
 \sqrt{\frac{x^2}{k+1}}\qquad
  x^\frac{2}{k+1}\qquad
  \frac{\partial^2f}
  {\partial x^2}
\]
\end{teXX}




\begin{equation*}
\sqrt{\frac{x^2}{k+1}}\qquad
x^\frac{2}{k+1}\qquad
\frac{\partial^2f}
{\partial x^2}
\end{equation*}


\newthought{Binomial Coefficients}

To typeset binomial coefficients or similar structures, use the command
\cmd{binom} from \docpkg{amsmath}:\index{amsmath!binom}

Pascal's rule can be typeset as:

\begin{shaded}
\begin{teXX}
\[
\binom{n}{k} =\binom{n-1}{k}
+ \binom{n-1}{k-1}
\]
\end{teXX}
\[
\binom{n}{k} =\binom{n-1}{k}
+ \binom{n-1}{k-1}
\]
\end{shaded}

\clearpage




\section{Matrices}

\begin{equation*}
\mathbf{X} = \left(
\begin{array}{ccc}
x_1 & x_2 & \ldots \\
x_3 & x_4 & \ldots \\
\vdots & \vdots & \ddots
\end{array} \right)
\end{equation*}

\begin{equation*}
\begin{matrix}
1 & 2 \\
3 & 4
\end{matrix} \qquad
\begin{bmatrix}
p_{11} & p_{12} & \ldots
& p_{1n} \\
p_{21} & p_{22} & \ldots
& p_{2n} \\
\vdots & \vdots & \ddots
& \vdots \\
p_{m1} & p_{m2} & \ldots
& p_{mn}
\end{bmatrix}
\end{equation*}


\subsection{vmatrix}
\begin{gather*}
\begin{vmatrix}
aa' + bb' + cc' & ea' + fb' + gc' \\
ae' + bf' + cg' & ee' + ff' + gg'
\end{vmatrix} \\
%
{} = \begin{vmatrix}
a & b \\
e & f
\end{vmatrix}  \begin{vmatrix}
a' & b' \\
e' & f'
\end{vmatrix} + \begin{vmatrix}
a & c \\
e & g
\end{vmatrix}  \begin{vmatrix}
a' & c' \\
e' & g'
\end{vmatrix} + \begin{vmatrix}
b & c \\
f & g
\end{vmatrix}  \begin{vmatrix}
b' & c' \\
f' & g'
\end{vmatrix}.
\end{gather*}


\section{Displayed equations}
All of the mathematics covered so far has identical input whether it is to be typeset
in-line or displayed. At this point we’ll look at some situations that apply to displayed
equations only.





\section{Single equations that are too long}

In many cases equations need to be written over two or more lines. The \docpkg{amsmath} package, provides an environment that is suitable for this:

\emphasis{cos}
\begin{teXXX}
\begin{multline}
   a + b + c + d + e + f+ g + h + i  + k + l + m + n + o + p\\
              = j + k + l + m + n +\cos^{2}-1
\end{multline}
\end{teXXX}

\begin{multline}
a + b + c + d + e + f+ g + h + i  + k + l + m + n + o + p\\
= j + k + l + m + n +\cos^{2}-1
\end{multline}

\newpage
\section{array environment}
This is simply the same as the eqnarray environment only with the possibility of
variable rows and columns and the fact, that the whole formula has only one
equation number and that the array environment can only be part of another math
environment, like the equation environment or the displaymath environment. With
@{} before the first and after the last column the additional space |\arraycolsep| is
not used, which maybe important when using left aligned equations.

\begin{tcblisting}{colback=blue!5,boxrule=2pt,colframe=blue!75!black,title=\textbf{The egnarray Environment},width=1.05\textwidth}
\begin{eqnarray}
a & = & b + c \\
& = & d + e + f + g + h + i
+ j + k + l \nonumber \\
&& +\: m + n + o \\
& = & p + q + r + s
\end{eqnarray}
\end{tcblisting}



the equations
to be aligned are entered with each one terminated by \doccmd{cr}. In each equation there should be
one alignment symbol \& to indicate where the alignment should take place. This is usually
done at the equal signs, although it is not necessary to do so. For example

\verb*+ \qquad( )+ produce

\begin{tcblisting}{colback=blue!5,boxrule=2pt,colframe=blue!75!black,title=\textbf{The array Environment},width=1.05\textwidth}
Thus to change $\frac34$ to a decimal divide $4$ into $3$
and we get $.75$ as a result, thus:
\[
\begin{array}{r@{}r@{}}
4 \; & \vline \; 3.00 \\\cline{2-2}
     &            .75
\end{array}
\]

To find the square root of a four-figure number
such as our example calls for, work it out in the
following manner:
\[
\arraycolsep=0em
\begin{array}{cccccccccccc}
\multicolumn{3}{c}{\text{2d pair}} &\qquad&\qquad&
\multicolumn{3}{c}{\text{1st pair}}&\qquad&\qquad&
\multicolumn{2}{c}{\text{square root}}\\
 & \overbrace{\quad}&\ZZZ&&&\ZZZ&\overbrace{\quad}&\ZZZ\\
 & 42 &&&&& 25 &&&&\vline\;65&(answer)\\\cline{11-11}
 & 36 &&&&& \\\cline{2-2}
\multirow{2}{*}{125\:} & \vline\hfill \Z6 \hfill&&&&& 25\\
 & \vline\hfill \Z6 \hfill&&&&& 25\\\cline{2-7}
\end{array}
\]

\end{tcblisting}

\subsection{Array environment in game theory}
This example is from \footnote{From determinacy to Nash equilibrium,St\'ephane Le Roux, TU Darmstadt }
\begin{tcblisting}{colback=blue!5,boxrule=2pt,colframe=blue!75!black,title=\textbf{The array Environment},width=1.05\textwidth}
The game $\langle\{a,b,c\},\{1,2,3,4\}^3,\{0,1,2,3,4\},v,(<_d)_{d\in\{a,b,c\}}\rangle$ is represented below, where player $a$ chooses the row, $b$ the column, and $c$ the array. 
\[\begin{array}{c@{\hspace{1cm}}c@{\hspace{1cm}}c@{\hspace{1cm}}c}
\begin{array}{|c|c|c|c|}
\hline 1 & 1 & 1 & 1\\
\hline 1 & 1 & 1 & 1\\
\hline 1 & 1 & 1 & 1\\
\hline 4 & 1 & 1 & 1\\
\hline
\end{array}
&
\begin{array}{|c|c|c|c|}
\hline 1 & 2 & 1 & 1\\
\hline 2 & 2 & 2 & 2\\
\hline 1 & 2 & 1 & 1\\
\hline 4 & 2 & 1 & 1\\
\hline
\end{array}
&
\begin{array}{|c|c|c|c|}
\hline 1 & 1 & 3 & 1\\
\hline 1 & 1 & 3 & 1\\
\hline 3 & 3 & 3 & 3\\
\hline 4 & 1 & 3 & 1\\
\hline
\end{array}
&
\begin{array}{|c|c|c|c|}
\hline 2 & 4 & 4 & 4\\
\hline 4 & 3 & 4 & 4\\
\hline 4 & 4 & 4 & 4\\
\hline 0 & 0 & 0 & 0\\
\hline
\end{array}
\end{array}
\]
Let us show that the game $\langle\{a,b,c\},\{1,\dots,n\}^3,\{0,\dots,n\},v,(<_d)_{d\in\{a,b,c\}}\rangle$ witnesses the claim. First, the preferences are linear orders indeed. Second, let us show that there is no Nash equilibrium by case-splitting below. 
\end{tcblisting}

\section{The AMSmath Package}

The amsmath package offers four different align environments, align, alignat, falign, xalignat and xxalignat. In difference to the eqnarray environment from standard LATEX the ``three'' parts of one equation expr.-symbol-expr. are divided by only one
ampersand in two parts. In general the ampersand should be before the symbol
to get the right spacing, e.g., y \&= x. 

\subsection{The align environment}

The align environment is an improvement over \latex's eqnarray environment. It is very similar to a tabular
environment and is aligned at the |&|.

\begin{tcblisting}{colback=blue!5,boxrule=2pt,colframe=blue!75!black,title=\textbf{The Align Environment},width=1.05\textwidth}
\begin{align}
         y & =d\label{eq:IntoSection}\\
         y & =cx+d\\
 y_{12} & =bx^{2}+cx+d\\
     y(x) & =ax^{3}+bx^{2}+cx+d
 \end{align}

\begin{align*}
\therefore (13 - x_1) + (13 - x_2) + \dotsb + (13 - x_p) + r &= 52\,,\\
\therefore 13p - (x_1 + x_2 + \dotsb + x_p) + r              &= 52\,,\\
\therefore x_1 + x_2 + \dotsb + x_p                          &= 13p - 52 + r\\
                                                          &= 13 (p - 4) + r\,.
\end{align*}

whence we conclude that $\gamma$ is a primitive root modulo $p$. But
\begin{align*}
\gamma^{p-1}-1 &=
     g^{p-1} - 1 + \frac{p-1}{1!}g^{p-2}xp +
        \frac{(p-1)(p-2)}{2!}g^{p-3}x^2p^2 + \ldots \\
  &= p\left(kp + \frac{p-1}{1!}g^{p-2}x +
        \frac{(p-1)(p-2)}{2!}g^{p-3}x^2p + \ldots\right).
\end{align*}
\end{tcblisting}

\subsection{The aligned environment}
The aligned environment allows more than one horizontal alignment but has only one equation number.
\newcommand{\dotsb}{\ldots}			% use lower dots after +-
\newcommand{\dotsbsmall}{\ldot\!\ldot\!\ldot}
\newcommand{\ldot}{\mathbin{.}}			% dot with math spacing
\newcommand{\nobf}[1]{\no \textbf{#1}}		% \no with bold number
\begin{tcblisting}{colback=blue!5,boxrule=2pt,colframe=blue!75!black,title=\textbf{The \texttt{aligned} Environment},width=1.05\textwidth}
\begin{equation}
\begin{aligned}
 &\:C_1x^{r_1}\,[\varphi_{r_1 0} \,+ \varphi_{r_1 1}\log x \,+ \dotsb + \varphi_{r_1 \alpha_1}(\log x)^{\alpha_1}]\\
+&\:C_2x^{r_2}\,[\varphi_{r_2 0} \,+ \varphi_{r_2 1}\log x \,+ \dotsb + \varphi_{r_2 \alpha_2}(\log x)^{\alpha_2}]\\
+&\multispan{1}{\:\dotfill}\\
+&\:C_nx^{r_n}[\varphi_{r_n 0} + \varphi_{r_n 1}\log x + \dotsb + \varphi_{r_n \alpha_n}(\log x)^{\alpha_n}],
\end{aligned}
\end{equation}

\[
\tag{98}
\left\{\qquad
\begin{aligned}
T_1 &= T_2 = T_3 (=T)\\
p_1 &= p_2 = p_3\\
s_1-s_2 &= \frac{(u_1-u_2)+p_1(v_1-v_2)}{T}\\
s_2-s_3 &= \frac{(u_2-u_3)+p_2(v_2-v_3)}{T}.
\end{aligned}
\right.
\]
\end{tcblisting}

\subsection{Interrupting a display}

In many instances you will want to interrupt a display with some text, you can use |\intertext|.

\begin{tcblisting}{colback=blue!5,boxrule=2pt,colframe=blue!75!black,title=\texttt{intertext},width=1.05\textwidth}
\begin{align}
U &= M u = M(c_v T + b)\\
S &= M(c_v  \log T + \frac{R}{m}  \log v + a),\\
\intertext{and}
F &= M \left\{T(c_v - a - c_v \log T) - \frac{RT}{m} \log v + b \right\}.
\end{align}
\end{tcblisting}

The command intertext can onl come after a |\\| or |\\| command. Its function is to preserve the alignment after the text is typeset. This is a common reuirement in many mathematical structures and the command is available in all of amsmath aligning environments.






\subsection{The alignat environment}
The alignat environment means \emph{align at} and can be used to align a set of equations vertically at more than one place.



\begin{tcblisting}{colback=blue!5,boxrule=2pt,colframe=blue!75!black,title=\textbf{alignat},width=1.05\textwidth}
\renewcommand{\dotsb}{\ldots}			% use lower dots after +-
\renewcommand{\dotsbsmall}{\ldot\!\ldot\!\ldot}
\renewcommand{\ldot}{\mathbin{.}}			% dot with math spacing
\renewcommand{\nobf}[1]{\no \textbf{#1}}	

\begin{alignat*}{5}
  &p_{i+1}\dfrac{d^{m-i-1}y_1}{dx^{m-i-1}} &&+ \dotsbsmall +p_{m}y_1
  && = -\Big(\dfrac{d^{m}y_1}{dx^{m}} &&+p_1\dfrac{d^{m-1}y_1}{dx^{m-1}}
  &&+ \dotsbsmall +p_{i}\dfrac{d^{m-i}y_1}{dx^{m-i}}\Big), \\
\multispan{10}{\makebox[36em]{\dotfill},}\\
 &p_{i+1}\dfrac{d^{m-i-1}y_{m-i}}{dx^{m-i-1}} &&+ \dotsbsmall +p_{m}y_{m-i}\!
 &&= -\Big(\dfrac{d^{m}y_{m-i}}{dx^{m}} &&+ p_1\dfrac{d^{m-1}y_{m-i}}{dx^{m-1}}
 &&+ \dotsbsmall +p_{i}\dfrac{d^{m-i}y_{m-i}}{dx^{m-i}}\Big).
\end{alignat*}
\end{tcblisting}



When using one of the align environments, there should be no |\\| at the end of the
last line, otherwise you’ll get another equation number for this ``empty''  line


\clearpage
\subsection{Multline}
The |multline| environment is another attempt at displaying long equations. It will set the first line flush left and the last one flush right. It can be quite useful when one has very long equations. The line break is marked with |\\|. It is good typographical practice to have the first line shorter thn the last line and not the other way around.

\begin{tcblisting}{colback=blue!5,boxrule=2pt,colframe=blue!75!black,title=\textbf{Multline},width=1.05\textwidth}
Example unumbered
\begin{multline*}
x^{\rho}f(x, \rho) = x^{\rho} \Big [ u_{m}x^{m}\frac{\rho(\rho-1)\ldots (\rho-m+1)}{x^{m}} \\
                   + u_{m-1}x^{m-1}\frac{\rho(\rho-1)\ldots (\rho-m+2)}{x^{m-1}}+ \dotsb
                   + u_{2}x^{2}\frac{\rho(\rho-1)}{x^2}+u_{1}x\frac{\rho}{x}+u_0 \Big ].
\end{multline*}
Example  numbered
\begin{multline}
M \left[\delta u - \left(T_1\, \frac{dp_{12}}{dT_{12}} - p_1\right) \delta v\right] \\
= \delta T_{12} \left[M_{12}\, \frac{du_{12}}{dT_{12}} + M_{21}\, \frac{du_{21}}{dT_{12}}
  - \left(T_1\, \frac{dp_{12}}{dT_{12}} - p_1\right)
    \left(M_{12}\, \frac{dv_{12}}{dT_{12}}
        + M_{21}\, \frac{dv_{21}}{dT_{12}}\right)\right].
\end{multline}
\end{tcblisting}

\clearpage
\section{gathered}
The |gathered| environment is like the |aligned| or |alignat| environment. They use
only so much horizontal space as the widest line needs. In difference to the gather
environment it must be itself inside math mode.

\begin{tcblisting}{colback=blue!5,boxrule=2pt,colframe=blue!75!black,title=\textbf{The gathered environment},width=1.05\textwidth,before=\bigskip}
\[
  \left .
   \begin{gathered}
    \left [ \frac{\alpha}{p} \right ] +
    \left [ \frac{\alpha}{p^2} \right ] +
    \left [ \frac{\alpha}{p^3} \right ] +
    \ldots \\
    \left [ \frac{\beta}{p} \right ] +
    \left [ \frac{\beta}{p^2} \right ] +
    \left [ \frac{\beta}{p^3} \right ] +
    \ldots \\
      \vdots \\
    \left [ \frac{\lambda}{p} \right ] +
    \left [ \frac{\lambda}{p^2} \right ] +
    \left [ \frac{\lambda}{p^3} \right ] +
    \ldots
   \end{gathered}
  \right \} \tag{B}
\]
\end{tcblisting}

\section{The cases environment}

\newlength{\boxla}
\newlength{\boxlb}
\newlength{\boxlc}
\setlength{\boxla}{1.15in}
\setlength{\boxlb}{1.7in}
\setlength{\boxlc}{1.6in}
\newcommand{\boxa}[1]{\makebox[\boxla]{\small #1\dotfill}}
\newcommand{\boxb}[1]{\makebox[\boxlb]{\small #1\dotfill}}

\begin{tcblisting}{colback=blue!5,boxrule=2pt,colframe=blue!75!black,title=\textbf{Cases},width=1.05\textwidth}
\begin{align*}
\boxa{DOYEN} & \quad
\parbox{3.4in}{\small MM. \\
MILNE EDWARDS, Professeur. Zoologie, Anatomie, \\
\hspace*{1.5in} Physiologie compare.}
\\
\parbox[b]{\boxla}{\small PROFESSEURS\\HONORAIRES\dotfill} &
\begin{cases}
\text{\small DUMAS.}\\
\text{\small PASTEUR.}
\end{cases}
\\
\boxa{PROFESSEURS} &
\begin{cases}
\boxb{CHASLES}\text{\small Gomtrie suprieure.} \\
\boxb{P. DESAINS}\text{\small Physique.} \\
\boxb{PUISEUX}\text{\small Astronomie.} \\
\boxb{JAMIN}\text{\small Physique.} \\
\boxb{O. BONNET}\text{\small Astronomie.}
\end{cases}
\\
\boxa{AGROGES} &
\begin{cases}
\parbox{\boxlb}{%
\small BERTRAND\dotfill\\
J. VIEILLE\dotfill}\bigg\} \text{\small Sciences mathematiques.} \\
\boxb{PELIGOT}\text{\small Sciences physiques.}
\end{cases}\\
\boxa{SECRETAIRE} & \quad \text{\small PHILIPPON.}
\end{align*}
\end{tcblisting}

\clearpage
\begin{tcblisting}{colback=blue!5,boxrule=2pt,colframe=blue!75!black,title=\textbf{Cases},width=1.05\textwidth}
non plus orthogonale mais telle que
\[
{\sum_{i}}' a_{pi} a_{qi}
  = \begin{cases}
    0 & \text{ si } p \gtrless q \\
    1 & \text{ si } p = q
    \end{cases}
\]
alors on a aussi
\[
{\sum_{i}}' a_{pi} a_{iq}
  = \begin{cases}
    0 & \text{ si } p \gtrless q \\
    1 & \text{ si } p = q
    \end{cases}
\]
\end{tcblisting}

\begin{tcblisting}{colback=blue!5,boxrule=2pt,colframe=blue!75!black,title=\textbf{Cases},width=1.05\textwidth}
\section{Test}
\end{tcblisting}

\section{flalign}
\begin{flalign*}
&&
\chi\omega &= \omega - S \omega\, \nabla \centerdot \sigma\, dt,  && \\
&\text{whence}&
\chi'^{-1} \omega &= \omega + \nabla_1 S \omega \sigma_1\, dt, &&
\end{flalign*}

\section{Maths fonts}



$$\circlearrowleft$$

{\Large
$$
\dashleftarrow  \dashrightarrow
 \leftleftarrows \rightrightarrows
 \leftrightarrows  \rightleftarrows
 \Lleftarrow  \Rrightarrow
 \twoheadleftarrow  \twoheadrightarrow
 \leftarrowtail  \rightarrowtail
 \leftrightharpoons 
 \rightleftharpoons
 \Lsh  \Rsh
 \looparrowleft  \looparrowright
 \curvearrowleft  \curvearrowright
	 \circlearrowleft \circlearrowright
 \multimap  \upuparrows
 \downdownarrows  \upharpoonleft
 \upharpoonright  \downharpoonright
\rightsquigarrow  \leftrightsquigarrow
$$
}



The \href{http://www.ctan.org}{CTAN} website.


\section{Summation}
\begin{equation*}
P = \frac{\displaystyle{
\sum_{i=1}^n (x_i- x)
(y_i- y)}}
{\displaystyle{\left[
\sum_{i=1}^n(x_i-x)^2
\sum_{i=1}^n(y_i- y)^2
\right]^{1/2}}}
\end{equation*}

\section{Math accents}

Mathematical accents are a bit different that the ones used for normal text in order to cater, firstly for the exotic taste in diagratics taste by mathematicians and secondly to cater for the fact that mathematics is styled in italics.
This is a short summary of what is available. 
\bigskip

\begin{tabular}{llllll}
\toprule
$\hat{a}$    & \doccmd{hat\{a\}} & $\check{a}$ & \doccmd{check\{a\}} &$\tilde{a}$&\doccmd{tilde\{a\}}\\
$\grave{a}$ &\doccmd{grave\{a\}}    & $\dot{a}$ &\doccmd{dot\{a\}} &$\ddot{a}$ &\doccmd{ddot\{a\}}\\
 $\bar{a}$ &\doccmd{bar\{a\}} & $\vec{a}$ &\doccmd{vec\{a\}} & $\widehat{AAA}$ &\doccmd{widehat\{AAA\}}\\
$\acute{a}$ &\doccmd{acute\{a\}} &$\breve{a}$  &\doccmd{breve\{a\}} &$\widetilde{AAA}$ &\doccmd{widetilde\{AAA\}}\\
$\mathring{a}$ &\doccmd{mathring\{a\}} & & & &\\
\bottomrule
\end{tabular}

\section{Binary Relations}


\begin{tabular}{llllll}
\toprule
$<$ &$<$  &$>$ &$>$ &$=$ &$=$\\
$\le$  &\doccmd{leq} or \doccmd{le}  &$\geq$ &\doccmd{geq} or \doccmd{ge} &$\equiv$ &\doccmd{equiv}\\
$\ll$  &\doccmd{ll}   &$\gg$  &\doccmd{gg}   &$\doteq$  &\doccmd{doteq} \\
$\prec$ &\doccmd{prec} &$\succ$  &\doccmd{succ} &$\sim$ &\doccmd{sim}\\
$\preceq$ &\doccmd{preceq} &$\succeq$  &\doccmd{succeq} &$\simeq$ &\doccmd{simeq}\\
$\subset$ &\doccmd{subset}  &$\supset$ &\doccmd{supset} &$\approx$ &\doccmd{approx}\\
$\subseteq$ &\doccmd{subseteq} &$\supseteq$  &\doccmd{supseteq} &$\cong$  &\doccmd{cong} \\
$\sqsubset$  &\doccmd{sqsubset}  &$\sqsupset$  &\doccmd{sqsupset}  &$\Join$  &\doccmd{Join}\\
$\sqsubseteq$   &\doccmd{sqsubseteq}   &$\sqsupseteq$ &\doccmd{sqsupseteq}   &$\bowtie$ &\doccmd{bowtie} \\
$\in$ &\doccmd{in}  &$\ni$ &\doccmd{ni}, \doccmd{owns} &$\propto$ &\doccmd{propto}\\
$\vdash$ &\doccmd{vdash}  &$\dashv$ &\doccmd{dashv} &$\models$ &\doccmd{models}\\

\bottomrule
\end{tabular}


\section{Brackets, braces and parentheses}

In addition  to the previous commands \cmd{Bigg} and \cmd{Biggm} can be used to add a bit more horizontal space.

\[3\Big\downarrow aˆ2+bˆ{cˆ2}
\Big\Downarrow\]


$$3\Big\updownarrow
aˆ2+bˆ{cˆ2}
\Big\Updownarrow$$

Another way to typeset the big separators is to split them over a line as shown below

{\arraycolsep=2pt
 \begin{equation}
 \begin{array}{rcl}
 \frac{1}{2}\Delta(f_{ij}f^{ij}) & = & 2\Bigg({\displaystyle
 \sum_{i<j}}\chi_{ij}(\sigma_{i}-\sigma_{j})^{2}+f^{ij}%
 \nabla_{j}\nabla_{i}(\Delta f)+\\
 & & +\nabla_{k}f_{ij}\nabla^{k}f^{ij}+f^{ij}f^{k}[2
 \nabla_{i}R_{jk}-\nabla_{k}R_{ij}]\Bigg)
 \end{array}
 \end{equation}

This is achieved by typing

\begin{teX}
{\arraycolsep=2pt
 \begin{equation}
 \begin{array}{rcl}
 \frac{1}{2}\Delta(f_{ij}f^{ij}) & = & 2\Bigg({\displaystyle
 \sum_{i<j}}\chi_{ij}(\sigma_{i}-\sigma_{j})^{2}+f^{ij}%
 \nabla_{j}\nabla_{i}(\Delta f)+\\
 & & +\nabla_{k}f_{ij}\nabla^{k}f^{ij}+f^{ij}f^{k}[2
 \nabla_{i}R_{jk}-\nabla_{k}R_{ij}]\Bigg)
 \end{array}
 \end{equation}

\end{teX}



\section*{The \texttt{stmarysrd} package}

 If the \textsf{amssymb} package has been loaded then the following
 are also defined: \verb|\oast| and \verb|\ocircle|.

 The following large operators are defined:
 \begin{symbols}
 \dosymbol\bigbox
 \dosymbol\bigcurlyvee
 \dosymbol\bigcurlywedge
 \dosymbol\biginterleave
 \dosymbol\bignplus
 \dosymbol\bigparallel
 \dosymbol\bigsqcap
 \dosymbol\bigtriangledown
 \dosymbol\bigtriangleup
 \end{symbols}
 The following relations are defined:
 \begin{symbols}
 \dosymbol\inplus
 \dosymbol\niplus
 \dosymbol\ntrianglelefteqslant
 \dosymbol\ntrianglerighteqslant
 \dosymbol\subsetplus
 \dosymbol\subsetpluseq
 \dosymbol\supsetplus
 \dosymbol\supsetpluseq
 \dosymbol\trianglelefteqslant
 \dosymbol\trianglerighteqslant
 \end{symbols}
 The following arrows are defined:
 \begin{symbols}
 \dosymbol\Longmapsfrom
 \dosymbol\Longmapsto
 \dosymbol\Mapsfrom
 \dosymbol\Mapsto
 \dosymbol\leftarrowtriangle
 \dosymbol\leftrightarroweq
 \dosymbol\leftrightarrowtriangle
 \dosymbol\lightning
 \dosymbol\longmapsfrom
 \dosymbol\mapsfrom
 \dosymbol\nnearrow
 \dosymbol\nnwarrow
 \dosymbol\rightarrowtriangle
 \dosymbol\rrparenthesis
 \dosymbol\shortdownarrow
 \dosymbol\shortleftarrow
 \dosymbol\shortrightarrow
 \dosymbol\shortuparrow
 \dosymbol\ssearrow
 \dosymbol\sswarrow
 \end{symbols}
 The following delimiters are defined:
 \begin{symbols}
 \dosymbol\Lbag
 \dosymbol\Rbag
 \dosymbol\lbag
 \dosymbol\llbracket
 \dosymbol\llceil
 \dosymbol\llfloor
 \dosymbol\llparenthesis
 \dosymbol\rbag
 \dosymbol\rrbracket
 \dosymbol\rrceil
 \dosymbol\rrfloor
 \end{symbols}
% Note that \verb|\llbracket| and \verb|\rrbracket| are `growing'
% delimiters that can be used with \verb|\left| and \verb|\right|:
% \[
%    \left\llbracket {\cal P} \right\rrbracket \quad
%    \left\llbracket \bigbox {\cal P} \right\rrbracket \quad
%    \left\llbracket \bigbox_{i\inplus I}^{a \varoplus b} P_i
%        \right\rrbracket \quad
%    \left\llbracket \begin{array}{c}a\\b\\c\end{array}
%\right\rrbracket \quad
%    \left\llbracket \begin{array}{c}a\\b\\c\\d\\e\\f\end{array} \right\rrbracket
% \]
 The following special characters are used in building others:
 \begin{symbols}
 \dosymbol\Arrownot
 \dosymbol\Mapsfromchar
 \dosymbol\Mapstochar
 \dosymbol\arrownot
 \dosymbol\mapsfromchar
 \end{symbols}
 For example, if you type
 \verb|$\Arrownot\Rightarrow$|
 you get
 $\Arrownot\Rightarrow$,
 and if you type
 \verb|$\arrownot\rightarrowtriangle$|
 you get
 $\arrownot\rightarrowtriangle$.

%% using phantom for spaces
\def\z{\phantom{\text{install piping}\shortrightarrow}}
$\shortrightarrow\text{Install piping}\shortrightarrow\text{Close ceilings}$


$\phantom{\text{install piping}\shortrightarrow}\shortrightarrow\text{Final Fix grilles}$

$\z\shortrightarrow\text{Final Fix grilles}$

$\shortrightarrow\text{Clean filters}$

$\shortrightarrow\text{pre-commission}$

\[
\left\llbracket \begin{array}{c}a\\b\\ \text{complete drawing}\\ \text{install~ductwork}\\install~ grilles\\clean~ceiling\end{array} \right\rrbracket \Longmapsto
\]



\[
Test \alpha\Gamma^i_{\phantom{iiiiiiiiiii}jk}
\]


\section{Maths Typesetting}

\texttt{
Handbook of Typography for the\\
Mathematical Sciences\\
Steven G. Krantz\\
January 21, 2003}\par

\url{http://www.faqorama.net/tecno/[LaTeX]%20Handbook%20of%20Typography%20for%20the%20Mathematical%20Sciences%20-%20S.G.Krantz%20(2003).pdf}


Ellen Swanson’s book Mathematics into Type is a unique and important contribution to the literature of technical typesetting. It set a
standard for how mathematics should be translated from a handwritten
manuscript to a printed book or document. While Swanson’s book was
intended primarily as a resource for technical typesetters, it was also important to mathematical and other technical authors who wanted to take
an active role in ensuring that their work reached print in an attractive
and accurate form.
The landscape has now changed considerably. With the advent and
wide availability of TEX,
1
most mathematicians can take a more active
role in producing typeset versions of their work. Indeed, many mathematicians currently use TEX to write preliminary versions of their work
that are very similar (in many respects) to what will ultimately appear
in print.

While the output from \tex has a more typeset appearance than that
from most word processors, the TEX product is not automatically (without human intervention) \enquote{ready to go to press}. There are still \enquote{postprocessing} typesetting issues that must be addressed before a work
actually appears in print. The style and format of running heads, section headings and other titles, the formatting of theorems and other
enunciations, the text at the bottom of the page, page break issues, and
the fonts and spacing used in all of these go under the name of “page design”. These are often customized for a particular book or journal. The
index and table of contents must be designed and typeset. Graphics,
and sometimes new fonts, must be integrated. Additional questions of
style in the formatting of equations and superscripts and subscripts can
also arise. Most TEX users do not know how to handle the questions just
listed, which is why most publishers currently send TEX documents for
books or journal articles to a third-party TEX consultant. The purpose
of the present work is to serve as a touchstone for those who want to
learn to make typesetting decisions themselves.


\def\smsqr#1#2{\sqrt{{#1}^2 + {#2}^2} + \frac{1}{{#1}^2 + {#2}^2}}

\[ \smsqr{a}{c} \]

There are other aspects of consistency about which many authors
are blissfully unaware: spacing above and below a displayed equation,
spacing above and below a theorem,6
space after a proof, the mark at
the end of a proof (QED, or the Halmos "tombstone" |\qed|, for example).\sidenote{ "The symbol is definitely not my invention — it appeared in popular magazines (not mathematical ones) before I adopted it, but, once again, I seem to have introduced it into mathematics. It is the symbol that sometimes looks like \(\boxed{\thinspace}\), and is used to indicate an end, usually the end of a proof. It is most frequently called the 'tombstone', but at least one generous author referred to it as the 'halmos'.", Paul R. Halmos, I Want to Be a Mathematician: An Automathography, 1985, p. 403.}
Again, a good macro can be invaluable in addressing these issues; but
awareness of the problem is also a great asset.

\begin{quotation}
You make everyone's
life easier if you eschew the eccentric and stick to the most basic constructions. This advice is valid for the Plain \tex user, for the \latex
user, for the Microsoft Word user, and for every other user of electronic
tools.
\end{quotation}

\begin{marginfigure}
\includegraphics[scale=.8]{halmos}
\captionof{figure}{In \textit{How to Write Mathematics} P.R. Halmos writes: `This is a subjective essay, and its title is misleading; a more honest title might be `How I Write Mathematics'.}
\end{marginfigure}

\newthought{Choose your notation carefully}

Bad notation can make good exposition bad and bad exposition worse; ad hoc decisions about notation, made mid-sentence in the heat of composition, are almost certain to result in bad notation. Good notation has a kind of alphabetical harmony and avoids dissonance. Example: either + by$ or +a_2x_2$ is preferable to $bx_2$. Or: if you must use $\Sigma$ for an index set, make sure you don't run into $\sum_{\sigma \in \Sigma}a_\sigma$. Along the same lines: perhaps most readers wouldn't notice that you used $Izl$.

\newthought{One symbol, one letter}

A mathematical symbol is usually indicated by \emph{one} letter, not two or three. If for example we want to suggest that the \textit{factor of safety} is equal to three, we should write
\[F_{\mathrm{s}}=3\]
and not
\[F_{\mathrm{safetyfactor}}=3\]
or worse
\[F_{\mathrm{sf}}=3\]
typesetting the subscript in \textit{italic} font is also wrong
\[F_{s}=3\]
as it does not represent a mathematical symbol, but is just an abbreviation for safety factor.

Sometimes the use of the one symbol one letter rule cannot be applied, without the notation becoming complex
\medskip

{
\narrower\narrower
The static friction force \(F_{\mathrm{sf}}\) will exactly oppose forces applied to an object parallel to a surface contact up to the limit specified by the [[coefficient of static friction]] \(\mu_{\mathrm{sf}}\) multiplied by the normal force \(F_N\). In other words the magnitude of the static friction force satisfies the inequality:

\[ \le F_{\mathrm{sf}} \le \mu_{\mathrm{sf}} F_\mathrm{N}. \]

The kinetic friction force \(F_{\mathrm{kf}}\) is independent of both the forces applied and the movement of the object. Thus, the magnitude of the force equals:

\[F_{\mathrm{kf}} = \mu_{\mathrm{kf}} F_\mathrm{N}\]

where \(\mu_{\mathrm{kf}}\) is the coefficient of kinetic friction.
}




\newthought{Do not start a sentence with an equation}

\newthought{Display math}

In general mathematics typeset better when they are displayed. Use inline maths only for the simplest of equations and for explanations of symbols and the like. watch out for inconsistent spacing before and after displayed math.

\newthought{Correct badly sized math}

Some \tex constructions typeset rather badly, consider for example this:

\[
\sqrt{\frac{\beta}{\gamma}} = \sqrt{X} + \sqrt{y}
\]

\noindent or this,

\[
\surd{\frac{\beta}{\gamma}} = \surd{X} + \surd{y}
\]


You can remedy this by using a \cmd{mathstrut}.


\emphasis{sqrt,mathstrut}
\begin{teXX}
\sqrt{\mathstrut a}=\sqrt{\mathstrut X}+\sqrt{\mathstrut y}
\end{teXX}

\newthought{Multiplication}

One of the most common errors is to use the ``dot'' to indicate multiplication between scalars\sidenote{\url{http://www.tug.org/TUGboat/Articles/tb29-2/tb92guiggiani.pdf}}. For example the folowing formul\ae
\[a\cdot x^2+b\cdot x+c=0\]
should be written as
\[ax^2+bx+c=0\]

In fact, for the the sake of simplicity, the standard multiplication between letters, or between letters, or between a number and a letter, does not require any symbol. If, on the other hand, the multiplication is between two numbers, the $\times$ or $\cdot$ symbols are required to avoid ambiguity.
For example you should write

\[2\times 3=6 \text{ and not } 2\thickspace 3=6 \]


\newthought{Using the right font}

The Euler equation involves the five most important mathematical constants. First we typest it with no space corrections\sidenote{\texttt{\textbackslash eu\^\,\{\textbackslash iu\textbackslash pi\}}},
% The number `e'
\providecommand*{\eu}%
{\ensuremath{\mathrm{e}}}
% The imaginary unit
\providecommand*{\iu}%
{\ensuremath{\mathrm{j}}}
\[\scalebox{3}{$\eu^{\iu\pi}$}\]
a small correction to the space should be added

\[\scalebox{3}{$\eu^{\,\iu\pi}$}\]

\subsection{Differential operators}
A peculiar defnition is required to properly
write the differential symbol. It is in fact an operator that has a space only on its left. In Beccari (2007b) the following solution is proposed:

\bigskip


\clearpage
\section{tikz}
\begin{tcblisting}{colback=blue!5,boxrule=2pt,colframe=blue!75!black,title=Basic Definitions,width=0.75\textwidth}
because of the periodicity of the Jacobi theta functions involved in the construction of the vectors.
The height difference between starting point and endpoint of the path is thus $Lp$ as shown in figure \ref{fig:path}. Moreover, because of the periodicity of the theta functions it is sufficient to restrict the initial height to $\ell_1=0,1,\dots,L-1$ in this case.
  \centering
  \begin{tikzpicture}[>=stealth]
     \draw[scale=0.5,thick] (0,0)--(2,2)--(3,1)--(5,3)--(6,2)--(7,3);
           
     \draw[<->] (0,2) -- (0,-0.5) -- (4,-0.5);
     \foreach \x in {0.25,0.75,...,3.25}
       \draw[xshift=\x cm,yshift=-0.5cm] (0,-0.075)--(0,0.075);
     
    \foreach \y in {-0.5,0,...,1.5}
       \draw[yshift=\y cm] (-0.075,0)--(0.075,0);
 
    \draw (0.25,-0.5) node[below] {$1$};
    \draw (0.75,-0.5) node[below] {$2$};
    \draw (2,-0.5) node[below] {$\cdots$};
    \draw (3.25,-0.5) node[below] {$N$};
    \draw (4,-0.5) node[below] {$j$};
    \draw (0,0) node [left] {$\ell$};
    \draw (0,1.5) node [left] {$\ell+Lp$};
    \draw (0,2) node [right] {$\ell_j$};
     \clip[scale=0.5] (0,-1.5) rectangle (7.5,3.5);
     \draw[scale=0.5,dotted] (-1,-2) grid (8,5);
  \end{tikzpicture}
\end{tcblisting}
\clearpage


\begin{tcblisting}{colback=blue!5,boxrule=2pt,colframe=blue!75!black,title=Basic Definitions,width=0.75\textwidth}
\newcommand{\ud}{\ensuremath \mathop{}\!\mathrm{d}}
\(z=2\sin x\mathrm{d}x\) and \(z=2\sin x\ud x\)
\end{tcblisting}
\newcommand{\ud}{\mathop{}\!\mathrm{d}}
\bigskip

It uses an empty operator and eliminates the space
on its left with |\!|.

Note the difference between

\[z=2\sin x\mathrm{d}x  \]

\[z=2\sin x\ud x\]

where the diffrential is obtained respectively with
|\mathrm{d}| and |\ud|.

\newthought{God is in the details}

Sometimes you will be faced with small decisions for which the Journal style manual might not have an answer for you or the editor might have a different opinion to yours. One such question is if one needs to insert the thousand separator in coefficients.

\[
\operatorname{erf}^{-1}(z)=\tfrac{1}{2}\sqrt{\pi}\left (z+\frac{\pi}{12}z^3+\frac{7\pi^2}{480}z^5+\frac{127\pi^3}{40320}z^7+\frac{4369\pi^4}{5806080}z^9+\frac{34807\pi^5}{182476800}z^{11}+\cdots\right )
\]


\[
\operatorname{erf}^{-1}(z)=\tfrac{1}{2}\sqrt{\pi}\left (z+\frac{\pi}{12}z^3+\frac{7\pi^2}{480}z^5+\frac{127\pi^3}{40,320}z^7+\frac{4,369\pi^4}{5,806,080}z^9+\frac{34,807\pi^5}{182,476,800}z^{11}+\cdots\right )
\]

\[
\operatorname{erf}^{-1}(z)=\tfrac{1}{2}\sqrt{\pi}\left (z+\frac{\pi}{12}z^3+\frac{7\pi^2}{480}z^5+\frac{127\pi^3}{40{,}320}z^7+\frac{4{,}369\pi^4}{5{,}806{,}080}z^9+\frac{34{,}807\pi^5}{182{,}476,800}z^{11}+\cdots\right )
\]



\[
\operatorname{erf}^{-1}(z)=\tfrac{1}{2}\sqrt{\pi}\left (z+\frac{\pi}{12}z^3+\frac{7\pi^2}{480}z^5+\frac{127\pi^3}{40\thinspace 320}z^7+\frac{4\thinspace 369\pi^4}{5\thinspace 806\thinspace 080}z^9+\frac{34\thinspace 807\pi^5}{182\thinspace 476\thinspace 800}z^{11}+\cdots\right )
\]


It is interesting to note that Knuth believes that in equations this is unecessary.
He is quoted in Typesetting Mathematics.

\begin{quotation}
But where Don wrote 1000000 they substituted
1,000,000. Don objected that although this might be justifed in text, his use is perfectly OK in a formula. Well then, they replied, write \(10^6\).
Fine, said, Don, but what do I do 
when the number is 1234567? The IEEE standard here is to insert spaces, thus: 1 234 567.
Don doesn't like this in formulae, but agrees that it may be useful in a high precision context, such as numerical tables. 
\end{quotation}

The following are extracts from his paper \sidenote{\url{http://www-cs-faculty.stanford.edu/~uno/papers/jfsp.tex.gz}}

{
$$\vcenter{\halign{$#$\hfil\ &$#$\hfil\cr
\Sigma n^{11}&=39916800{n+6\choose 12}+
19958400{n+5\choose 10}+3160080{n+4\choose 8}
+168960{n+3\choose 6}\cr
\noalign{\smallskip}
&\qquad\null+2046{n+2\choose 4}+{n+1\choose 2}\,;\cr
\noalign{\smallskip}
\Sigma n^{13}&=6227020800{n+7\choose 14}+3632428800{n+6\choose 12}+
726485760{n+5\choose 10}\cr
\noalign{\smallskip}
&\qquad\null+57657600{n+4\choose 8}
+1561560{n+3\choose 6}+8190{n+2\choose 4}+{n+1\choose 2}\,.\cr}}$$
}

Also, note in the last equation the use of a period at the end. This is something that strong opinions and flaming wars in fora. I am not too sure if I agree on the last one, but the way that Knuth writes is very clear and his equations in a way are paragraphs. In this case the use of a period is recommended.


\newthought{Punctuation}

There are two schools of thought when it comes to punctuation, that is punctuation in display style formulae. Some authors (Beccari,2007 a), others that it is necessary and essential.

The authors of this article believe that formulae,
both in display and text style, are part of the argumentation
and so punctuation should be used to help
the reader. An example of good use of punctuation is:


Since
$$ a=b $$
and
$$b=c,$$
it is proven that
\[a =c. \]



\subsection{Numbering Equations}

One question that you may face is the numbering of display equations. Early books used numbering sparingly, whereas many authors go overboard and number all the equations.

According to Knuth et al:\footnote{\url{http://tex.loria.fr/typographie/mathwriting.pdf}}

Numbering all displayed formulas is usually a bad idea; number the important ones only.
Halmos\footnote{\url{http://www.math.uh.edu/~tomforde/Books/Halmos-How-To-Write.pdf}} offers pretty much the same good advice,

\begin{quotation}
What about "inequality (*)", or "equation (7)", or "formula (iii)"; should all displays be labelled or numbered? My answer is no. Reason: just as you shouldn't mention irrelevant assumptions or name irrelevant concepts, you also shouldn't attach irrelevant labels. Some small part of the reader's attention is attracted to the label, and some small part of his mind will wonder why the label is there. If there is a reason, then the wonder serves a healthy purpose by way of preparation, with no fuss, for a future reference to the same idea; if there is no reason, then the attention and the wonder were wasted.
\end{quotation}

See also discussion at \url{http://tex.stackexchange.com/questions/29267/which-equations-should-be numbered/49080\#49080}

\url{http://tex.stackexchange.com/questions/29267/which-equations-should-be-numbered/49080#49080}

Now if you wish to argue about this is fine.

\section{Mathmode}

TeX is in mathmode when it is reading mathematics. The |ifmmode| can be used to find out if TeX is in math mode. It denotes the start of an if-then-else control structure that tests whether \tex is currently in either math mode or display math mode. The |\else| part is optional. <TeX code 1> is processed if TeX is in one of the math modes, otherwise it is ignored. If the |\else| section is included and TeX is not in one of the math modes then <TeX code 2> is processed; otherwise it is ignored.

%\begin{teXXX}
%\def\A{\ifmmode \mathcal{A} \else $\mathcal{A}$ \fi}
%\end{teXXX}

\begin{tcblisting}{}
 \def\a{test}
\a
\end{tcblisting}
\medskip
\begin{tcolorbox}[colback=blue!5,boxrule=2pt,boxsep=3mm,colframe=blue!75!black,title=Basic Definitions,width=0.65\textwidth]
    |\def\A{\ifmmode \mathcal{A} \else \[\mathcal{A}\] \fi}|
\end{tcolorbox}
\medskip


defines a macro |\A| that can be used both in and out of math mode to typeset a calligraphy script A. 

This is a calligraphic \A\ or $\A$.


\[ \A \]


\clearpage
\section{Useful packages}
Besides the main packages that we have discussed so far and which should be in everyone's toolbox, there are a number of other packages that you may find useful. One such package is the |\multienum|, which although not really a packaged specializing in mathematical typesetting, it provides an environment to set multiple equations, as in an exercise or exam.

%\begin{teXXX}
%\documentclass{article}
%\usepackage{xstring,amsmath}
%\begin{document}
%\[\operatornamewithlimits{K}_{k=0}^\infty\frac{a_k}{b_k}\]
%\end{document}
%
%\end{teXXX}


\newthought{the multienum package}

The \docpkg{multienum} enables  the typestting of multiple equations on one line and numbering them, either with roman, arabic or alpha letters.

\emphasis{usepackage, multienum,begin,end,multienumerate}
\begin{teXXX}
\documentclass{article}
\usepackage{multienum}
\renewcommand{\regularlisti}{\setcounter{multienumi}{0}%
  \renewcommand{\labelenumi}%
  {\addtocounter{multienumi}{1}\alph{multienumi})}}
\begin{document}
\begin{multienumerate}[oddlist]
\mitemxxx{\(x^2 + y^2 = 1\)}{\(a + b = c\)}{\(r-x = y+z\)}
\mitemxxx{\(f - y = z\)}{\(a - b = 2d\)}{\(r+x = 2y-3z\)}
\end{multienumerate}
\end{document}
\end{teXXX}


\begin{multienumerate}[oddlist]
\mitemxxx{\(x^2 + y^2 = 1\)}{\(a + b = c\)}{\(r-x = y+z\)}
\end{multienumerate}
\begin{multienumerate}[evenlist]
\mitemxxx{\(f - y = z\)}{\(a - b = 2d\)}{\(r+x = 2y-3z\)}
\end{multienumerate}


\hrule

\bigskip

We can also enumerate the items using an even-only or odd only
counter.
\subsection*{Answers to Even-Numbered Exercises}
\begin{multienumerate}[evenlist]
\mitemxxxx{Not}{Linear}{Not}{Quadratic}
\mitemxxxo{Not}{Linear}{No; if $x=3$, then $y=-2$.}
\mitemxx{$(x_1,x_2)=(2+\frac{1}{3}t,t)$ or
$(s,3s-6)$}{$(x_1,x_2,x_3)=(2+\frac{5}{2}s-3t,s,t)$}
\mitemx{$(x_1,x_2,x_3,x_4)= (\frac{1}{4}+\frac{5}{4}s+\frac{3}{4}t-u,s,t,u)$
or $(s,t,u,\frac{1}{4}-s+\frac{5}{4}t+\frac{3}{4}u)$}
\mitemxxxx{$(2,-1,3)$}{None}{$(2,1,0,1)$}{$(0,0,0,0)$}
\end{multienumerate}
\bigskip

\hrule

\clearpage


\begin{casestudy}[The Riemann hypothesis.]{%
Typeset the text and the equations, shown below. Use a standard minimal to achieve it. Note the fraktur fonts. Text must all be as one paragraph.}

It is well known that the Riemann zeta function $\zeta(s)$ of a complex variable $s=\sigma+it$ is defined by
\[
\zeta(s)=\sum_{n=1}^{\infty}\frac{1}{n^{s}}
\]
for the real part $\mathfrak{R}(s)>1$ and its analytic continuation in the half plane $\sigma>0$ is
\begin{equation}\label{func:zeta}
\zeta(s)=\sum_{n=1}^{N}\frac{1}{n^{s}}-\frac{N^{1-s}}{1-s}-\frac{1}{2}N^{-s}
+s\int_{N}^{\infty}\frac{\frac{1}{2}-\{x\}}{x^{s+1}}dx
\end{equation}
for any integer $N\geq1$ and $\mathfrak{R}(s)>0$.
It extends to an analytic function in the whole complex plane except for having a simple pole at $s=1$. Trivially, $\zeta(-2n)=0$ for all positive integers. All other zeros of the Riemann zeta functions are called its nontrivial zeros.
\bottomline

\begin{teX}
It is well known that the Riemann zeta function $\zeta(s)$ of a complex variable $s=\sigma+it$ is defined by
\[
\zeta(s)=\sum_{n=1}^{\infty}\frac{1}{n^{s}}
\]
for the real part $\mathfrak{R}(s)>1$ and its analytic continuation in the half plane $\sigma>0$ is
\begin{equation}\label{func:zeta}
\zeta(s)=\sum_{n=1}^{N}\frac{1}{n^{s}}-\frac{N^{1-s}}{1-s}-\frac{1}{2}N^{-s}
+s\int_{N}^{\infty}\frac{\frac{1}{2}-\{x\}}{x^{s+1}}dx
\end{equation}
for any integer $N\geq1$ and $\mathfrak{R}(s)>0$.
It extends to an analytic function in the whole complex plane except for having a simple pole at $s=1$. Trivially, $\zeta(-2n)=0$ for all positive integers. All other zeros of the Riemann zeta functions are called its nontrivial zeros.
\end{teX}

Please note that the maths and the text, are typed as a single block. Do not leave any spaces in between. We have used |\mathfrak| for the fraktur font. We have also used $it$ for the imaginary part. This would depend on the style used in your field. 
\end{casestudy}

\clearpage
\section{Gather}
This is like a multi line environment with no special horizontal alignment. All rows
are centered and can have an own equation number:

\begin{tcblisting}{colback=blue!5,boxrule=2pt,colframe=blue!75!black,title=The gather environment,width=\textwidth,before=\bigskip,after=\bigskip}
\def\O{\mathcal{O}}
\begin{gather}
 \O,\O(E_4),\O(E_2),\O(H-E_3-E_5),\O(H-E_3),\O(H-E_5),\\ 
\O(2H-E_1-E_3-E_5-E_6),\O(2H-E_1-E_3-E_5),\O(2H-E_3-E_5-E_6).
\end{gather}

So lautet der Beweis des Satzes $2 \times 2 = 4$:
\begin{gather}
(\Omega^{\nu})^{\mu}{}'x = \Omega^{\nu \times \mu}{}'x \text{ Def.}\\
%\begin{split}
\Omega^{2 \times 2}{}'x = (\Omega^{2})^{2}{}'x = (\Omega^{2})^{1 + 1}{}'x = \Omega^{2}{}'\Omega^{2}{}'x = \Omega^{1 + 1}{}'\Omega^{1 + 1}{}'x\nonumber \\
= (\Omega'\Omega)'(\Omega'\Omega)'x = \Omega'\Omega'\Omega'\Omega'x = \Omega^{1 + 1 + 1 + 1}{}'x = \Omega^{4}{}'x.
%\end{split}
\end{gather}


\begin{gather*}
x = \Omega^{0}{}' x \text{ Def.\ and}\\
\Omega'\Omega^{\nu}{}'x = \Omega^{\nu+1}{}'x \text{ Def.}
\end{gather*}
\begin{equation}
x = \Omega^{0}{}' x \text{ Def.\ and}\\
\Omega'\Omega^{\nu}{}'x = \Omega^{\nu+1}{}'x \text{ Def.}
\end{equation}
\end{tcblisting}


The gather environment has an implicit |{c}| horizontal alignment with no
vertical column alignment. It is just like an one column array/table.
A nonumber-version |\begin{gather*}...\end{gather*}| exists. 

A common error is to forget the include `\$\$` in intertext text, if you want to include
maths as part of the textual description.

\emphasis{intertext}
\begin{tcblisting}{colback=blue!5,boxrule=2pt,colframe=blue!75!black,title=The gather environment,width=\textwidth,before=\bigskip,after=\bigskip}
\def\mat#1{\bm{\mathrm{#1}}}
\begin{align}
	 A &= \frac{1}{\sqrt{2}}
	\begin{pmatrix}
		1	&	1	\\
		i e^{-2 r_2}	&	-i e^{-2 r_2}
	\end{pmatrix}
	\,, \\
	B &= \frac{1}{\sqrt{2}}
	\begin{pmatrix}
		i e^{-2 r_1}	&	i e^{-2 r_1}	\\
		-1	&	1
	\end{pmatrix}
	\,,\\
\intertext{note that $z = i e^{r_1}$}
	A^{-1} &= \frac{1}{\sqrt{2}}
	\begin{pmatrix}
		1	&	-i e^{+2 r_2}	\\
		1	&	i e^{+2 r_2}
	\end{pmatrix}
	\,,
\end{align}
\end{tcblisting}



% 



%

%

%  
\@specialtrue
\cxset{steward,
  numbering=arabic,
  custom=stewart,
  offsety=0cm,
  image=hine03,
  texti={When Lamport designed the original \LaTeX\ sectioning commands, limitations of computer power forced him to restrict the abstraction of complicated chapter layouts. With current tools available improvements are much easier to program.},
%
  textii={In this chapter we discuss a method that allows the production of fancy sectionr headings and formatting, based on a set of key values. Central  to this process is the separation of content from presentation.
We also discuss the basic formatting tools that are available and how one can modify them to mould new book designs.
 }
}



\raggedbottom

\chapter{Lower Level Headings}
\@specialfalse

\section{Introduction}

Good book design dictates that sectioning styles follow that of the general book design and theme. An academic publication for example might have chapters and section numbered in arabic numerals, whereas a high school textbook might have sections marked in colored boxes.

Similarly to the chapter key value interface, the package offers a key value interface to adjust sectioning command parameters.



\cxset{section beforeskip={10pt},
      section indent=0pt}
\cxset{section afterskip={10pt}}
\renewsection

\section{Section styling}

In a similar fashion to the chapter commands the following keys are provided.

\subsection{Fonts and numerals}

Font and numeral keys are shown below.
\medskip

  \keyval{section font-size}{\marg{cmd}}{Font size command such as \cs{large.}}
  \keyval{section font-weight}{\marg{cmd}}{Font weight command such as \cs{bfseries.}}
  \keyval{section font-family}{\marg{cmd}}{Font family command such as \cs{sffamily.}}
  \keyval{section font-shape}{\marg{cmd}}{Font shape command such as \cs{itshape}}
  \keyval{section color}{\marg{color}}{Color of section.}
  \keyval{section numbering}{\marg{arabic|roman|Roman|alph|Alph|words|WORDS}}{Section number style.}
  \begin{marglist}
  \item [arabic] Typesers the section number in arabic numerals.
  \item [roman] Typesets the section number in lowercase roman numerals.
  \item [Roman] Typesets the section number in uppercase roman numerals.
  \item [alph] Typesets the section number in lowercase alphabetic numbering.
  \item [Alph] Typesets the section number in uppercase alphabetic numerals.
  \item [words] Typesets the numbers in words (lowercase).
  \item [WORDS] Typesets the number in words (uppercase).
  \end{marglist}

\subsection{Skip and indentation commands}

The keys for indentaion and above and below skips are shown below.
\medskip

\keyval{section beforeskip}{}{}
\keyval{section afterskip}{}{}
\keyval{section indent}{\marg{dim}}{Indentation from margin as per standard LaTeX class definitions.}
\keyval{section spaceout}{}{}
\begin{marglist}
 \item[soul]
 \item[none]
\end{marglist}

\subsection{align}

\keyval{section align}{\marg{cmd}}{One of the alignment commands centering, ragged right, raggedleft}

\subsection{Hooks}

Hooks for adding material are shown in the following sketch.
\medskip

\fbox{aboveskip}

\fbox{indent} \fbox{number}\fbox{hook}\fbox{title}

\fbox{belowskip}

%\lipsum

\section{Example usage}

\cxset{
 chapter toc=false,
 name=CHAPTER,
 numbering=arabic,
 number font-size=\huge,
 number font-family=\sffamily,
 number font-weight=\bfseries,
 number before=,
 number dot=,
 number after=\hspace{1em},
 number position=rightname,
 chapter opening=anywhere,
 chapter font-family=\sffamily,
 chapter font-weight=\bfseries,
 chapter font-size=\huge,
 chapter before={\vspace*{0.1\textheight}\hfill},
 chapter after={\hfill\hfill\vskip0pt\thinrule\par},
 chapter color={black!90},
 number color=\color{black!90},
 title beforeskip={\vspace*{30pt}},
 title afterskip={\vspace*{30pt}\par},
 title before={\hfill},
 title after={\hfill\hfill},
 title font-family=\sffamily,
 title font-color=\color{black!90},
 title font-weight=\bfseries,
 title font-size=\huge,
%%%%%%%%%% Sections
 section font-size=\LARGE,
 section font-weight=\normalfont,
 section font-family=\sffamily,
 section align=\centering,
 section numbering=arabic,
 section indent=0em,
 section align=\centering,
 section beforeskip=20pt,
 section afterskip=10pt,
 section spaceout=soul,
 section font-shape=\itshape,
}
\cxset{book/.style={
 section numbering=arabic,
 section font-size=\Large,
 section font-weight=\bfseries,
 section font-family=\rmfamily,
 section font-shape=\normalfont,
 section align=\raggedright,
 %section numbering custom=\color{gray}{Section} (\thechapter-\@arabic\c@section),
 subsection font-size=\large
 section indent=0em,
 section beforeskip=-3.5ex \@plus -1ex\@minus -0.2ex,
 section afterskip=2.3ex\@plus.2ex,
 subsection beforeskip=-3.5ex \@plus -1ex\@minus -0.2ex,
 subsection afterskip= 1.5ex \@plus .2ex,
}}


\begin{example}{Adjusting section parameters}{}
\cxset{ section font-size=\LARGE,
 section font-weight=\normalfont,
 section font-family=\sffamily,
 section align=\centering,
 section numbering=(roman),
 section indent=0em,
 section align=\centering,
 section beforeskip=20pt,
 section afterskip=10pt,}
\chapter{A First Look at the Sectioning Keys}
\section{First section}
\lorem
\end{example}

One notable thing to keep in mind is that the numbering of the chapter is independent of that for the section, so if you need to have strange combinations rather define a section numbering custom.\index{section formatting!vertical space}

\cxset{section numbering=arabic}
\subsection{Adjusting vertical spaces}

Perhaps the most important issues we need to consider is the adjusting of vertical spaces; example~\ref{ex:latex}, that follows illustrates settings from the Octavo class and compare them with those of standard the \LaTeXe\ book class. The Octavo class through settings that are based on baselineskip fractions and multiples endeavours to achieve a grid layout. The class also tones down the `loudness' of some of the headings compared to those of the book class.


\cxset{octavo/.style={
 section font-size=\large,
 section font-weight=\normalfont,
 section font-family=\rmfamily,
 section font-shape=\scshape,
 section indent=0em,
 section align=\centering,
 section beforeskip=-1.666\baselineskip\@minus -2\p@,
 section afterskip=0.835\baselineskip \@minus 2\p@,
 subsection numbering=none,
 subsection font-family=\rmfamily,
 subsection font-size=\normalfont,
 subsection font-shape=\scshape,
 subsection font-weight=\normalfont,
 subsection indent=1em,
 subsection align=\raggedright,
 subsection beforeskip=-0.666\baselineskip\@minus -2\p@,
 subsection afterskip=0.333\baselineskip \@minus 2\p@
 }}




\cxset{book/.style={
 section numbering=arabic,
 section font-size=\Large,
 section font-weight=\bfseries,
 section font-family=\rmfamily,
 section font-shape=\normalfont,
 section align=\raggedright,
 %section numbering custom=\color{gray}{Section} (\thechapter-\@arabic\c@section),
 subsection font-size=\large,
 section indent=0em,
 section beforeskip=-3.5ex \@plus -1ex\@minus -0.2ex,
 section afterskip=2.3ex\@plus.2ex,
 subsection font-size=\large,
 subsection font-weight=\bfseries,
 subsection numbering=arabic,
 subsection indent=0pt,
 subsection beforeskip=-3.5ex \@plus -1ex\@minus -0.2ex,
 subsection afterskip= 1.5ex \@plus .2ex,
}}

\cxset{octavo headings/.style={%
 section numbering=none,section font-size=\large,section font-weight=\normalfont,
 section font-family=\rmfamily, section font-shape=\scshape,
 section indent=0em, section align=\centering, section beforeskip=-1.666\baselineskip\@minus -2\p@,
 section afterskip=0.835\baselineskip \@minus 2\p@, subsection numbering=none,
 subsection font-family=\rmfamily, subsection font-size=\normalfont, subsection font-shape=\scshape,
 subsection font-weight=\normalfont, subsection indent=1em, subsection align=\raggedright,
 subsection beforeskip=-0.666\baselineskip\@minus -2\p@,
 subsection afterskip=0.333\baselineskip \@minus 2\p@,
 subsubsection numbering=none,
 subsubsection font-family=\rmfamily,
 subsubsection font-size=\normalfont,
 subsubsection font-shape=\itshape,
 subsubsection font-weight=\normalfont,
 subsubsection indent=1em,
 subsubsection align=\raggedright,
 subsubsection beforeskip=-0.666\baselineskip\@minus -2\p@,
 subsubsection afterskip=0.333\baselineskip \@minus 2\p@,
 paragraph numbering=none,
 paragraph font-family=\rmfamily,
 paragraph font-size=\normalfont,
 paragraph font-shape=\normalfont,
 paragraph font-weight=\normalfont,
 paragraph indent=-1em,
 paragraph align=\raggedright,
 paragraph beforeskip=\z@,
 paragraph afterskip=0\p@,
% subparagraph numbering=none,
% subparagraph font-family=\rmfamily,
% subparagraph font-size=\normalfont,
% subparagraph font-shape=\normalfont,
% subparagraph font-weight=\normalfont,
% subparagraph indent=0em,
% subparagraph align=\raggedright,
% subparagraph beforeskip=\z@,
% subparagraph afterskip=0\p@,
}}
\cxset{octavo headings}
\renewsection\renewsubsection\renewsubsubsection\renewparagraph

\begin{example}{Octavo class headings, settings}{}
\cxset{octavo headings/.style={%
 section numbering=none,section font-size=\large,section font-weight=\normalfont,
 section font-family=\rmfamily, section font-shape=\scshape,
 section indent=0em, section align=\centering, section beforeskip=-1.666\baselineskip\@minus -2\p@,
 section afterskip=0.835\baselineskip \@minus 2\p@, subsection numbering=none,
 subsection font-family=\rmfamily, subsection font-size=\normalfont, subsection font-shape=\scshape,
 subsection font-weight=\normalfont, subsection indent=1em, subsection align=\raggedright,
 subsection beforeskip=-0.666\baselineskip\@minus -2\p@,
 subsection afterskip=0.333\baselineskip \@minus 2\p@,
 subsubsection numbering=none,
 subsubsection font-family=\rmfamily,
 subsubsection font-size=\normalfont,
 subsubsection font-shape=\itshape,
 subsubsection font-weight=\normalfont,
 subsubsection indent=1em,
 subsubsection align=\raggedright,
 subsubsection beforeskip=-0.666\baselineskip\@minus -2\p@,
 subsubsection afterskip=0.333\baselineskip \@minus 2\p@,
 paragraph numbering=none,
 paragraph font-family=\rmfamily,
 paragraph font-size=\normalfont,
 paragraph font-shape=\normalfont,
 paragraph font-weight=\normalfont,
 paragraph indent=-1em,
 paragraph align=\raggedright,
 paragraph beforeskip=\z@,
 paragraph afterskip=0\p@,}}

\cxset{octavo headings}
\renewsection\renewsubsection\renewsubsubsection\renewparagraph
\section{Octavo Class Heading}
\lorem
\subsection{Octavo subsection}
This is some text short text\par
\subsubsection{Octavo sub-subsection}
\lorem
\paragraph{paragraph heading} This is some short text.
\end{example}

\begin{example}{}{}
\cxset{octavo}
\section{Octavo Class Heading}
\lorem
\subsection{Octavo subsection}
\lorem
\subsubsection{Octavo sub-subsection}
\lorem
\paragraph{paragraph heading} This is some short text.
\lorem
\paragraph{paragraph heading} This is some short text.
\lorem
\end{example}



\begin{example}{\LaTeXe\ book class headings settings}{ex:latex}
\cxset{book/.style={
 section numbering=arabic,
 section font-size=\Large,
 section font-weight=\bfseries,
 section font-family=\rmfamily,
 section font-shape=\normalfont,
 section align=\raggedright,
 %section numbering custom=\color{gray}{Section} (\thechapter-\@arabic\c@section),
 subsection font-size=\large,
 section indent=0em,
 section beforeskip=-3.5ex \@plus -1ex\@minus -0.2ex,
 section afterskip=2.3ex\@plus.2ex,
 subsection font-size=\large,
 subsection font-shape=\normalfont,
 subsection font-weight=\bfseries,
 subsection numbering=arabic,
 subsection indent=0pt,
 subsection beforeskip=-3.5ex \@plus -1ex\@minus -0.2ex,
 subsection afterskip= 1.5ex \@plus .2ex,
}}
\cxset{book}
\renewsubsection
\section{LaTeX Book  Class Heading}
\lorem
\subsection{A subsection}
\lorem
\end{example}

\section{Grid example}

One problem sometimes is that the sectioning commands create problems with grid layouts. Example~\ref{ex:grid} shows example settings.

\begin{example}{Section styles from the grid package}{ex:grid}
\cxset{grid/.style={
 section numbering=arabic,
 section font-size=\normalsize,
 section font-weight=\bfseries\mathversion{bold},
 section font-family=\rmfamily,
 section font-shape=\normalfont\bfseries\mathversion{bold},
 section beforeskip=-.999\baselineskip,
 section afterskip=0.001\baselineskip,
 section align=\raggedright,
 %section numbering custom=\color{gray}{Section} (\thechapter-\@arabic\c@section),
 subsection font-size=\normalsize,
 section indent=0em,
% section beforeskip=-3.5ex \@plus -1ex\@minus -0.2ex,
 %section afterskip=2.3ex\@plus.2ex,
 subsection font-shape=,
 subsection font-weight=\bfseries\mathversion{bold},
 subsection numbering=arabic,
 subsection indent=0pt,
 subsection beforeskip=\baselineskip,
 subsection afterskip= -.35\baselineskip,
% subsub section
 subsubsection font-shape=\itshape,
 subsubsection font-weight=\bfseries\mathversion{bold},
 subsubsection numbering=numeric,
 subsubsection indent=0pt,
 subsubsection beforeskip=\baselineskip,
 subsubsection afterskip= -.35\baselineskip,
}}
\cxset{grid}
\renewsubsection
\begin{multicols}{2}
\section{Grid  Class Heading}
\lorem
\subsection{Grid  subsection.}
\lorem
\subsubsection{A subsection grid.}
\lorem
\subsubsection{Another subsection grid.}
\lorem
\end{multicols}
\end{example}



The key \option{\bfseries section numbering custom}=\marg{code} is quite powerfull and can be used to define any type of section number style. Just remember that the numbering so far depends on two counters, the c@chapter and c@section. What the section numbering does, it redefines the macro \cs{thesection} to the new definition provided as argument for the key.

Although the temptation to define a lot of key combinations one would rather define new styles as a more user friendly approach.

\cxset{section numbering=arabic, section align=\raggedright, section font-shape=\upshape, section font-family=\rmfamily}
\section{Handling Other Section Levels}

Other sectioning commands such as \cs{subsubsection}, \cs{paragraph} and \cs{subparagraph} have equivalent keys. Examples can be found in the chapters that follow for specific styles.

\section{Technical discussion}

The standard LaTeX classes, book report and article have sections showing dot leaders, whereas in the article class the sections are shown without the dotted lines, as the l@section macro is redefined for articles.

\index{macros!\textbackslash @seccntformat}

\subsection{Indexing of Lower Section Headings}
\LaTeXe\ offers two pathways in redefining section commands, the first one is @startsection and the second is \cs{@seccntformat} \index{sectioning macros}. It also uses the macro \cs{secdef} to create the starred and unstarred versions of the sectioning commands.

\begin{tcolorbox}{}
\begin{lstlisting}
% \begin{macro}{\l@section}
%    In the article document class the entry in the table of contents
%    for sections looks much like the chapter entries for the report
%    and book document classes.
%
%    First we make sure that if a pagebreak should occur, it occurs
%    \emph{before} this entry. Also a little whitespace is added and a
%    group begun to keep changes local.
% \changes{v1.0h}{1993/12/18}{Replaced -\cs{@secpenalty} by
%    \cs{@secpenalty}.  ASAJ.}
% \changes{v1.2i}{1994/04/28}{Don't print a toc line when the tocdepth
%    counter is less than 1.}
% \changes{v1.4a}{1998/10/12}{we should use \cs{@tocrmarg}; see PR/2881.}
%    \begin{macrocode}
%<*article>
\newcommand*\l@section[2]{%
  \ifnum \c@tocdepth >\z@
    \addpenalty\@secpenalty
    \addvspace{1.0em \@plus\p@}%
%    \end{macrocode}
%
%    The macro |\numberline| requires that the width of the box that
%    holds the part number is stored in \LaTeX's scratch register
%    |\@tempdima|. Therefore we put it there. We begin a group, and
%    change some of the paragraph parameters (see also the remark at
%    \cs{l@part} regarding \cs{rightskip}).
%    \begin{macrocode}
    \setlength\@tempdima{1.5em}%
    \begingroup
      \parindent \z@ \rightskip \@pnumwidth
      \parfillskip -\@pnumwidth
%    \end{macrocode}
%    Then we leave vertical mode and switch to a bold font.
%    \begin{macrocode}
      \leavevmode \bfseries
%    \end{macrocode}
%    Because we do not use |\numberline| here, we have do some fine
%    tuning `by hand', before we can set the entry. We discourage but
%    not disallow a pagebreak immediately after a section entry.
%    \begin{macrocode}
      \advance\leftskip\@tempdima
      \hskip -\leftskip
      #1\nobreak\hfil \nobreak\hb@xt@\@pnumwidth{\hss #2}\par
    \endgroup
  \fi}
%</article>
\end{lstlisting}
\end{tcolorbox}

As you can see the dot leaders are not present in the above definition. Although we can get rid of dot leaders in other section by redefining them, it is not as easy to add them back.

As our aim is to be able to have all the classes used a common denominator we can define a command as follows (using book as a base)

\begin{tcolorbox}{}
\begin{lstlisting}
\def\articlesection{
\newcommand*\l@section[2]{%
  \ifnum \c@tocdepth >\z@
    \addpenalty\@secpenalty
    \addvspace{1.0em \@plus\p@}%
    \setlength\@tempdima{1.5em}%
    \begingroup
      \parindent \z@ \rightskip \@pnumwidth
      \parfillskip -\@pnumwidth
      \leavevmode \bfseries
      \advance\leftskip\@tempdima
      \hskip -\leftskip
      #1\nobreak\hfil \nobreak\hb@xt@\@pnumwidth{\hss #2}\par
    \endgroup
  \fi}
}
\end{lstlisting}
\end{tcolorbox}

%\articlesection

The \cs{@starredsection} macro is one of those locomotive type of commands. It takes 7 required arguments and 2 optional ones and hidden within it are two booleans. The full set looks like this:

\cs{@startsection} \marg{name} \marg{level} \marg{indent} \marg{beforeskip} \marg{afterskip} \marg{style}[*]
  [\marg{altheading}]\marg{heading}.

\begin{marglist}
\item[name] The name of the level command.
\item [level] A number denoting the depth of the section, chapter=1, section=2, etc. A section number will be printed only if \marg{level} is equal or smaller than the value of \textit{secnumdepth}
\item[indent] The indentation of the heading from the left margin.
\item[beforeskip]  The absolute value of this argument is the skip to leave above the heading. If it is negative, then the paragraph indent of the text following the heading is suppressed.
\item [afterskip] If positive, it is the skip to leave below the heading, else it is the skip to the right of a run-in heading.
\item [style] Sets the style of the heading.
\item[\textup{[*]}] When this is missing the heading is numbered and the corresponding counter is incremented.
\item[\textup{[\textit{altheading}]}] Gives an alternative heading to use in the table of contents and in the running heads. This should be present when the * form is used.
\item[heading] The heading of the new section.
\end{marglist}

\begin{example}{Example formatting run-in section}{}
\makeatletter
\bgroup
\renewcommand\section{%
    \@startsection{section}%
    {1}%
    {0em}%
    {-0.8em}%
    {-0.5em}%
    {\large\normalfont\scshape}}
\makeatother
\section[]{test}
\lorem
\egroup
\end{example}

Note we run the example in a group so that we will not influence the formatting of this document.

As mentioned earlier there is an additional way to introduce formatting for sections and this is using the command \cs{@seccntformat}, which is responsible for typesetting the counter part of a section title. The default definition of the command typesets the \cs{the} representation of the section counter.

\begin{example}{}{}
\bgroup
\renewcommand\section{%
    \@startsection{section}%
    {1}%
    {0em}%
    {-0.8em}%
    {-0.5em}%
    {\large\normalfont\scshape}}
\renewcommand\@seccntformat[1]{\fbox
{\csname the#1\endcsname}\hspace{0.5em}}
\makeatother
\section[]{test}\label{sec:ok}
\lorem

See section \ref{sec:ok}.
\egroup
\end{example}

The definition of \cs{@seccntformat} applies to all headings
defined with the \cs{@startsection} command (which is described in the next
section). Therefore, if you wish to use different definitions of \cs{@seccntformat}
for different headings, you must put the appropriate code into every heading
definition.

\begin{tcolorbox}
\begin{lstlisting}
\def\@seccntformat##1{\csname the##1\endcsname{}}
\end{lstlisting}
\end{tcolorbox}

\section{Custom headings}

It is also possible to define section headings without resorting to any of the above. To do this.

\begin{tcolorbox}
\begin{lstlisting}
\newcommand\part{\secdef\cmda\cmdb}
\end{lstlisting}
\end{tcolorbox}

the part and chapter and sometimes appendix are defined this way, but nothing stops us from doing the same for other sections. A generic section command can be defined as follows:

\begin{example}{}{}
\bgroup
\renewcommand\section[2] [?]{% % Complex form:
\refstepcounter{section}% % step counter/ set label
\addcontentsline{toc}{appendix}% % generate toe entry
{\protect\numberline{section-\thesection}#1}%
{\raggedright\large\bfseries section %\appendixname\ % typeset the title
\thesection\par \centering#2\par}% % and number
\sectionmark{#1}% % add to running header
\@afterheading % prepare indentation handling
%\addvspace{\baselineskip}
}
\section{Test}
\lorem
\egroup
\end{example}

Many other strategies can also be implemented that are perhaps easier to grasp.

\begin{example}{}{}
\bgroup
\def\strut{\vrule height12pt depth1pt width0pt}
\renewcommand\section[2] []{% % Complex form:
\refstepcounter{section}% % step counter/ set label
\addcontentsline{toc}{section}% % generate toc entry
{\protect\numberline{\thesection} }%
{\raggedright\large\bfseries\scshape %
\parbox[b]{\dimexpr(\linewidth-0.5\columnsep)}{\colorbox{brown!80}%
{{\vbox{\strut\raise2pt\hbox{#2}}}}}}\vskip0pt% % and number
\sectionmark{#1}% % add to running header
\@afterheading % prepare indentation handling
\vspace{\dimexpr\baselineskip+6pt}%must have a parameter
}
\chapter{Fossil Insects}
\begin{multicols*}{2}\raggedcolumns
\section[Insect Fossilization]{\raggedright \thinspace Insect Fossilization}
\lipsum[1]
\end{multicols*}
\egroup
\end{example}
% To answer http://tex.stackexchange.com/questions/52998/change-title-to-small-caps-but-not-in-toc

Of course some work is needed to center the text properly in the middle of the colour box. For all practical purposes it is lining up as per the sample.

In Chapter we discussed a forward, but this may not apply if there are no chapters or we need to treat these as sections, the example \ref{ex:forwardsection} shows such a method.

\begin{example}{Defining a Foreward Section}{ex:forwardsection}

\newcommand\prematter@sp[1]{% % Complex form:
%\refstepcounter{section}% % step counter/ set label
\addcontentsline{toc}{section}% % generate toe entry
{\protect\numberline{}\textsc{#1}}%
\sectionmark{#1}% % add to running header
{\LARGE\centering\normalfont\sffamily\colorbox{brown!80}{ \textsc{#1}}\par}%
\@afterheading % prepare indentation handling
\addvspace{\baselineskip}
\@afterindentfalse
}

\newenvironment{prematter}[1]{%
   \prematter@sp{#1}}
{}
\begin{multicols}{2}
\label{theok}
\begin{prematter}{Foreward}
\lipsum[1]
\end{prematter}\ref{theok}
\end{multicols}
\end{example}

\section{underlining}

I am aware that some people have no choice but have some sections underlined as dictated by archaic regulations in some establishments for thesis submission. If nobody is forcing you to underline it is best to avoid it. We use Donald Arsenau's ulem package to achieve underlining.



% 
%
%  \part{THE TYPESETTING ENGINES}
% 

%  \chapter{Presenting Data in Figures}

There can be no doubt that the hallmark of scientific reports and publications is the graphical presentation of the results. Graphs show relationships underlying observations in a way no other device can provide\footnote{\textit{Doing science: design, analysis, and communication of scientific research}
 By Ivan Valiela}. 
Charting is both an art and a science. Modern typography on charts and infographics look at Tufte as inspiration.
Tufte advocates to minimize the ink to data ratio and although this is not always possible it is good advice.
In this section we would look at charting in general which is probably of interest to most of the readers
in this book. 

\section{Tufte like charts}

During the last stages of a Project, it maybe easier to visualize the
main areas where effort needs to be exerted by using simple charts. One
such chart is shown in Figure~\ref{fig:tufte-overall}. When this chart
was prepared efforts were made to complete the physical installation
as well as plan and commission the plant. The use of colour in this
chart highlights the commissioning, so one can easily see the expectations. Although the percentages are written on top of the bars,
one need not read them to visualize how difficult is to achieve
100\% completion in a Project. On the other hand commissiong can go
fairly fast and can jump by a large percentage, just by
commissioning a couple of additional ELV systems that have approximately
a 10\% weigh factor.

One can easily fit approximately, six to seven months data on
a portrait chart, changing it around to landscape one can fit
more than a year. Personally I am not very happy with such long
projections as they are more like guesses rather than proper estimates.

One other chart that can be used to visualize progress and is more
commonly found in construction is the infamous S-curve. Now, if
the actual planning is detailed enough and granular enough to be
able to pin-point \textit{continuous} progress then it is
appropriate. using it if you can at least obtain weekly progress
estimates.


\begin{figure}[b]
 \begin{tikzpicture}
  \footnotesize
  \centering
  \begin{axis}[
        ybar, axis on top,
        title={Cumulative Progress of Works},
        height=5cm, width=13.2cm,
        bar width=0.43cm,
        ymajorgrids, tick align=inside,
        major grid style={draw=white},
        enlarge y limits={value=.1,upper},
        ymin=0, ymax=100,
        axis x line*=bottom,
        axis y line*=right,
        y axis line style={opacity=0},
        ytick={0,25,50,75,100},
        tickwidth=0pt,
        legend style={
            at={(0.5,-0.2)},
            anchor=north,
            legend columns=-1,
            % adds space between the legends
            /tikz/every even column/.append style={column sep=0.7cm}
        },
        ylabel={Percentage (\%)},
        symbolic x coords={
           Sep-11,Oct-11,Nov-11,Dec-11,
           Jan-12,Feb-12,
           Mar-12,
          Apr-12},
       xtick=data,
       nodes near coords={
        \pgfmathprintnumber[precision=2]{\pgfplotspointmeta}
       }
    ]
    \addplot [draw=none, fill=gray] coordinates {
      (Oct-11, 98)
      (Nov-11,99)
      (Dec-11,99.5)
      (Jan-12,99.7)
      (Feb-12,99.8)
       };
   \addplot [draw=none,fill=gray!75!white] coordinates {
      (Oct-11, 96)
      (Nov-11,97)
      (Dec-11,98)
      (Jan-12,98.5)
      (Feb-12,99)
        };
   \addplot [draw=none, fill=gray!50!white] coordinates {
      (Oct-11, 50)
      (Nov-11, 60)
      (Dec-11, 70)
      (Jan-12, 80)
      (Feb-12, 90)
            };
    \addplot [draw=none, fill=orange!90!white] coordinates {
      (Oct-11, 25)
      (Nov-11, 35)
      (Dec-11, 45)
      (Jan-12, 55)
      (Feb-12, 65)
          };
    \legend{First Fix,Second Fix,Final Fix,Commissioning}
  \end{axis}
  \end{tikzpicture}

\caption{\protect\raggedright Cumulative progress for all MEP works. Notice the slower rate of production during the last three months.}
\label{fig:tufte-overall}
\end{figure}

\section{Graph Design}
A good graph is uncluttered, clear and focused.

\subsection{Axis Lines}

Most problems with graphs arise from misuse of axes: too heavy, too long, wrong intersection,
ambiquous breaks or too confusing increments and incorrect proportions. An axis is a ruler that established
regular intervals for measuring the information provided. Axes may emphasize, diminish, distort, simplify
or clutter the information.

\clearpage
\begin{multicols}{2}
\subsection{Axis Length}

Graphs should utilize their space around them, as the graph itself is mostly white space. In publications the journal might want to minimize the cost of printing. An axis should not extend beyond the labeled unit od minor tick closest to the last data point.
\columnbreak
\begin{tikzpicture}
\begin{axis}
\addplot coordinates {
(0,0)
(0.5,1)
(1,2)
};
\addplot coordinates {
(0,0)
(0.9,1.3)
(1.2,2.5)
};
\end{axis}
\end{tikzpicture}
\end{multicols}














%
%  %  \part{Programming Topics}
%  
\chapter{GROUPING AND SCOPING RULES}
\index{Grouping}
\label{ch:grouping}

Like most computer languages \tex\ has a scoping mechanism that is able to confine most changes to a particular locality. This chapter explains what sort of actions can be local, and how groups are formed.
\medskip

\begin{docCommand}{bgroup}{}
Implicit beginning of group character.
\end{docCommand}

\begin{docCommand}{egroup}{}
 Implicit end of group character.
 \end{docCommand}

\begin{docCommand}{begingroup}{}
 Open a group that must be closed with |\endgroup|.
\end{docCommand}

\begin{docCommand}{endgroup}{} 
Close a group that was opened with |\begingroup|.
\end{docCommand}

\begin{docCommand}{aftergroup}{} 
Save the next token for insertion after the current group ends.
\end{docCommand}

\begin{docCommand}{global}{}
 Make assignments, macro definitions, and arithmetic global.
\end{docCommand} 

\begin{docCommand}{globaldefs}{}
 Parameter for overriding |\global| prefixes. IniTEX default: 0.
\end{docCommand}



The grouping mechanism can be thought of a bit like scope in other programming languages, with the
exception that in \tex the mechanism is much more Pascal-like. Most assignments made inside a group are local to that group
unless explicitly indicated otherwise, and outside the group old values are restored (pretty much like in Pascal). 

The most common way to group a portion of your program is to use braces. If we type the following  example:

\begin{texexample}{}{}
\def\i{42} 

{
  \def\i{43}
  \def\b{2}
}

The value of the \textbackslash i is now \i

\def\x{a}
\let\y\x
\bgroup
  \def\x{b}
  Within group \x\par
\egroup
  Outside group \x
\end{texexample}
We get   \texttt{The value of the \textbackslash i is now 42}. Due to the way \tex scoping rules work, the old program state
will be restored \textit{completely} after returning from the local group. Neither the change to |\i| nor the definition of |\b| will survive. This is also true for register changes or other assignments.



\section{Local and global assignments}

An assignment or macro definition is usually made global by prefixing it with \cs{global}, but nonzero
values of the integer parameter |globaldefs| override |doccmd{global}|
is positive every assignment is implicitly prefixed with \docAuxCommand{global}, and if |\globaldefs| is negative,
|\global| is ignored. Ordinarily this parameter is zero. It has very
limited use and even in the \latex\ kernel we can only find 3-4 uses when defining math fonts.\footnote{In file \texttt{ltfssbas.dtx}.}


Some assignment are always global: the \marg{global} assignments are:

\begin{description}
\item[font assignment] assignments involving \cs{fontdimen}, \cs{hyphenchar}, and \cs{skewchar}.

\item[hyphenation] assignment \cs{hyphenation} and \cs{patterns} commands.

\item[hbox size assignment] altering box dimensions with \cs{ht}, \cs{dp}, and \cs{wd} 

\item[interaction mode assignment] run modes for a \tex job.

\item[intimate assignment] assignments to a special integer or special dimen
\end{description}

\section{Braces}

The most common way to group is to use braces. They are used for two purposes:

\begin{enumerate}
\item to indicate the start and end of a group. For example |{\small here is some text}|.

\item to indicate that a string of tokens should be treated as one unit. For example in |\def\abc{...}| the braces are used
to delimit the argument.
\end{enumerate}

It is important to note that the characters `\{', `\}' are not hardwired in \tex. Any tokens with catcodes 1 and 2 can be used.
The plain format starts [343] by defining:

\begin{teX}
\catcode`\{ =1
\catcode `} = 2
\end{teX}

Tokens with catcodes 1 and 2 are called \emph{explicit braces}. An \emph{implicit} brace is a control sequence whose replacement text is an explicit brace. Thus the two |plain| control sequences 
|\bgroup| and |\egroup| are implicit braces. 

There is also a low-level \tex operator pair for creating groups. It works
just as the braces. A group is started with \cs{begingroup} and ended with
\cs{endgroup}. These operators may be freely mixed with braces but pairs
should be properly matched. So |{ \begingroup \endgroup }| is allowed
but |{ \begingroup } \endgroup| is not.

\begin{teX}
\let\bgroup={
let\egroup=}
\end{teX}

They can be used where unbalanced braces are needed.

Salomon gives an example to typeset a number of paragraphs with a negative indentation\footnote{This style can sometimes be found in old books.}:

\begin{teX}
\def\negIndent{\brgoup\parindent=-20pt}
\def\endIndent{\par\egroup}

\negIndent
  \small\lipsum[1]
\endIndent
\end{teX}

This will typeset:

\def\beginindent{\bgroup\parindent=-20pt}
\def\endindent{\par\egroup}

\beginindent
  \small\lipsum[1-3]
\endindent

\section{Forming Groups Using \textbackslash begingroup and \textbackslash endgroup} 

The other two primitives \docAuxCommand{begingroup} and \docAuxCommand{endgroup} can also be used to define a group. However a group that starts with a |\begingroup| must end with an |\endgroup|. This provides a mechanism for error checking, which \tex's parsing routines can easily catch.

Note that |\begingroup| and |\endgroup| can only be used to define a group, not to delimit a string. You can say:

\begin{teX}
\begingroup
  \it abc
\endgroup
\end{teX}

but the following will get \tex to complain about missing braces

\begin{teX}
\hbox\begingroup\it abc\endgroup
\end{teX}

It should be pointed out that |\begingroup| and |\endgroup| do not really
add any new grouping functionality that could not be provided by curly braces
or |\bgroup| and |\egroup|. On the other hand, these two instructions are very
useful in nested groups of complicated structures, where one wants to make sure
that a certain "begin group instruction" is matched by a certain "end group
instruction." For this pair of grouping instructions, and this pair only, use |\begingroup|
and |\endgroup|. In case a |\begingroup| is not matched by a |\endgroup|,
an error is generated by \tex.\footcite{bechto1993} 

The case when not to use |begingroup| is clear. However, if one should use it for cases where
|\bgroup| is possible, is a subject with different opinions.\footnote{See \url{https://tex.stackexchange.com/questions/1930/when-should-one-use-begingroup-instead-of-bgroup/1932\#1932}.} Unless you are using |mathmode| or have deeply nested structures, |bgroup| is fine to use. In all
other cases it is preferable to use |\begingroup|.

\section*{Examples}
From the TexBook Exercise 7.4

Suppose that the commands
\begin{texexample}{}{}
{\catcode`\<=1 \catcode`\>=2
 \bfseries test
>
 test
\end{texexample}

appear near the beginning of a group that begins with |{| these specifications instruct
TEX to treat |<| and |>| as group delimiters. According to \tex's rules of locality, the
characters |<| and |>| will revert to their previous categories when the group ends. But
should the group end with |}| or with |>| ?

It ends with either |>| or |}| or any character of category 2; then the effects of all
\cs{catcode} definitions within the group are wiped out, except those that were global.
\tex  doesn't have any built-in knowledge about how to pair up particular kinds of
grouping characters. New category codes take effect as soon as a |\catcode| assignment
has been digested. For example,

\begin{teX}
{\catcode`\>=2 >
\end{teX}

is a complete group. But without the space after |2|  it would not be complete, since TEX
would have read the |>|  and converted it to a token before knowing what category code
was being specified; \tex always reads the token following a constant before evaluating
that constant.

\topline

\textbf{Example}: \textsc{Adjusting the spacing of a font} An interesting example that illustrates some of the concepts that were discussed so far is to try and change the \textit{inter word spacing} of text using the \cs{fontdimen2} parameter. The interesting aspect of this example is that
we want to change the spacing, but since the font changes are global, we want to revert back to the original font at the end of the group. Although there are many other ways of achieving this we will use the \cs{aftergroup}.

\begin{teX}
\font \roman=cmr10
\font\specroman=cmr10
%% Next, the special registers
\newdimen\savedvalue
\savedvalue=\fontdimen2\roman
\newdimen\specialvalue
\specialvalue=13.0pt
%% Finally, definitions.
\def \rm{%
  \fontdimen2\roman=\savedvalue }
\def\specrm{%
  \aftergroup\restoredimen
  \fontdimen2\specroman=\specialvalue
  \specroman  }
\def\restoredimen{%
\fontdimen2\roman=\savedvalue }
\end{teX}
{
%% First, fonts.
\font \roman=cmr10
\font\specroman=cmr10
%% Next, the special registers
\newdimen\savedvalue
\savedvalue=\fontdimen2\roman
\newdimen\specialvalue
\specialvalue=13.0pt
%% Finally, definitions.
\def \rm{%
  \fontdimen2\roman=\savedvalue }
\def\specrm{%
  \aftergroup\restoredimen
  \fontdimen2\specroman=\specialvalue
  \specroman  }
\def\restoredimen{%
\fontdimen2\roman=\savedvalue }


{\bf Spaced Out Text} 
\medskip
{\specrm \lorem} dimension2 the interword   value \the\fontdimen2\font


{\bf  Back to Normal}
\medskip

\rm
\lorem

}

\section{\textbackslash aftergroup}

The \cs{aftergroup} control sequence saves a token for insertion after the current group. Several
tokens can be set aside by this command, and they are inserted in the left-to-right order in which
they were stated.

\begin{texexample}{}{}
\def\x#1;{#1}
\def\y{15}
{\globaldefs1
\bgroup
   \def\y{0}
   \aftergroup\x\aftergroup\y\aftergroup;
   \aftergroup}
\egroup
\y


\globaldefs0

\def\z{1}
{\def\z{0}
\z
}

\z

\end{texexample}

\begin{texexample}{}{}
{ \def\z{1}
  {\def\z{0}\globaldefs1
     \z
    {
	\z
    }
   \z
  }
 \z
}
\end{texexample}
\section{afterassignment}

An interesting primitive is \docAuxCommand{afterassignment}. The primitive saves the token immediately following it without
expansion. Nothing happens until after the next assignment; immediately after the next assignment the saved token is expanded.

\begin{texexample}{Aftergroup}{ex:aftergroup}
\def\yy{%
  \afterassignment\yyb
  \let\yyDiscard = 
}

\def\yyb{%
 ``%
 \bgroup
 \itshape
 \aftergroup\yyc
}
\def\yyc{%
  ''%
}

\yy{This is a test}  
\end{texexample}

The above example is not a very common or idiomatic way of writing macros. So what is |\afterassignment| good for? Its main use is to write macros with \enquote{arguments} similar to the way \tex assigns registers. Afterassignment allow you to define macros which avoid curly braces to enclose arguments.

The most common use of |\afterassignment| is in a macro whose parameter is glue or dimen. Consider the definition of a macro such as:
\begin{quote}
 |\def\myglue#1{\leftskip=#1 \rightskip=#1}|
\end{quote}

Such a macro can be called as |\myglue{3pt plus5pt minus3pt}|, but if we want to keep the same conventions as \tex we might prefer to have the ability to call it as |\myglue 3pt plus5pt minus3pt|. To achieve this we can do:

\begin{texexample}{Afterassignment}{ex:afterassignment}
\bgroup
\font\larger=cmr10 scaled\magstep1
\larger
\newskip\tempskip
\def\myglue{\afterassignment\myglueaux \tempskip}
\def\myglueaux{\leftskip=\tempskip \rightskip=\tempskip}
\myglue=30pt plus1pt minus1pt
\lorem\par
\egroup
\lorem
\end{texexample}



\section{Scoping Rules for boxes}

The scoping rules for boxes work similarly to those for other command sequences, since they are just macros defined by \latex or |plain|. In the example below, we define a box |\mybox| and we save a sentence both in global scope as well as local scope.

\begin{teX}
\documentclass{article}
\begin{document}
  \newsavebox{\mybox}
  \savebox{\mybox}{Outside scope}
  \usebox\mybox
  \begin{minipage}{5cm}
    \sbox{\mybox}{from first minipage}(*@ \label{global} @*)
    \usebox\mybox
  \end{minipage}
  \usebox{\mybox}
\end{document}
\end{teX}


This will typeset:
\medskip

\newsavebox{\myboxi}
\savebox{\myboxi}{\tt > Outside scope}

\noindent\usebox\myboxi

\noindent\begin{minipage}{5cm}
\sbox{\myboxi}{\tt > from first minipage}
\noindent\usebox\myboxi
\end{minipage}

\noindent\usebox{\myboxi}


\medskip 
Changing line [\ref{global}] to |\global\sbox| will make the definition of |\mybox| within the minipage environment global and would change the output to:
\medskip


To save memory space, box registers become empty by using them: \tex assumes
that after you have inserted a box by calling |\boxnn| in some mode, you do not need the contents of that register any more and empties it. In case you do need the contents of a box register more
than once, you can |\copy| it. Calling |\copynn| is equivalent to |\boxnn| in all respects except that the register is not cleared.


There are 256 box registers, numbered 0–255. Either a box register is empty (‘void’), or it contains
a horizontal or vertical box. This section discusses specifically box registers; the sizes of boxes,
and the way material is arranged inside them, is treated below.




\newbox\MyBox

\setbox\MyBox=\hbox{\hfil Test\hfill}

\unhbox\MyBox


\noindent\unhbox\MyBox

\noindent{\hfill Test \hfill}



\framebox{\parbox{\linewidth}{\color{theblue}
\textbf{\textcolor{purple}{\textsf{CAUTION}}}
\begin{enumerate}
\itemsep-5pt
\item \latex will not empty a box as it uses the \cs{copy} command in the definition of the \cs{newsavebox}.
\item It is better to use \LaTeX\ commands rather than \tex primitives, when defining boxes, as \latex tests for duplication of names - which is very important if a user uses a lot of different packages.
\item Give always preferences to local definitions rather than global. Globals always create maintenance problems in programming.
\end{enumerate}
}}


\section{Implicit Grouping}

There are  instances where grouping is \textit{implicit}. What this means is that \text starts and ends a group automatically and without any action by the user. There are two major cases where this happens:

\begin{enumerate}
\item The text inside a box such as |\hbox|, |\vbox|, |\vtop|, |\vcenter| etc. is automatically treated by \tex as a group.  For example |\hbox{\bf My Heading}|, will print  \hbox{\bf My Heading}  and it will not continue with the bold font once outside the group. All these commands have curly brackets and these curly brackets form implicit groups.
\item In five cases \tex forms implicit groups. In some of these cases not even curly braces are involved.
\end{enumerate}

\begin{enumerate}
\item The text inside math mode is treated as a group. This is true both for inline math as well as display math.
\item Matching |\left| and |\right| primitives treat the formula in between them as a group.
\item Fractions are treated as a group.
\item The execution of an ouput routine is implicitly enclosed in a group.
\item Columns in |\halign| based tables are local.
\end{enumerate} 

\subsection{\texttt{afterssignment and grouping}}

\begin{macro}{\afterassignment}
The primitive |\afterasignment| does not follow grouping in that it does not save the definition of a token when |\afterassignment| is executed. Consider the following example:
\end{macro}

Define the two macros |\xx| and |\yy|.

\begin{texexample}{afterassignment}{}
\def\xx{\string\xx\ executed\par }

\def\yy{\string\yy\ executed\par }

\afterassignment\xx
\end{texexample}

We start a group, where we have two definitions of |\xx| and |\yy|

\begin{texexample}{afterassignment}{}
\def\yy{42}
{
  \def\xx{\string\xx executed inside a group\par}

  \def\yy{\string\yy executed inside a group\par}

The second afterassignment is execute

  \afterassignment\yy

The group is ended

}
\end{texexample}

Note \cs{afterassignment} saves the token following \cs{afterassignment} without expanding it. Nothing happens until after the next assignment; immediately after the next assignment the saved token is expanded. This is a bit of a tricky part and you can go over it to make sure you understand it well.
\footnote{\url{http://tug.org/TUGboat/tb32-2/tb101grunewald.pdf}}
\footnote{\url{http://tex.stackexchange.com/questions/65462/plain-tex-theory-afterassignment}}


\begin{texexample}{Combining bgroup and begingroup}{}
\begingroup
\newbox\savedparbox

\def\saveparbox{\par\begingroup
  \def\par{\egroup\endgroup}
  \global\setbox\savedparbox\vbox\bgroup}

Ordinary paragraph.
\saveparbox
This paragraph will be saved in \string\box\string\savedparbox.
If you wish, you can unpack the box and do all kinds of processing on it.
In this demo, I won't do any processing.
Look in the log file to examine the box contents.

Another ordinary paragraph.
\endgroup
\end{texexample}



































%  \chapter{Expandafter}

One of the most often misunderstood \TeX\ commands is \cmd{\expandafter}
expandafter is an instruction that reverses the order of expansion. It is not a typesetting instruction, but an instruction that influences the expansion of macros. But what is \textit{expansion}? The term expansion means the replacement of the macro and its arguments, if there are any, by the \textit{replacement} text of the macro. If we have defined a macro

\begin{teX}
\def\test{ABC};
\end{teX}


\noindent then the replacement text of |\test| is |ABC| and the \textit{expansion} of |\test| is |ABC|.

As a control sequence |expandafter| can be followed by any number of tokens.

\begin{commands}[]{}
\cmd{\expandafter}\string\token$_e$\string\token$_1$\string\token$_2$\string\token$_n$ etc
\end{commands}

\noindent then the following rules describe the execution of |expandafter|:

\begin{enumerate}
\item  $<token_e$, the token immediately following |\expandafter|, is saved without expansion.
\item $<token_1>$, which is the token after the saved $token_e$, is analyzed. The following cases can be distinguished:
\begin{enumerate}
\item If is a macro: The macro will be expanded. In other words, the macro and its arguments, if any, will be replaced by the replacement text. After this \tex will \textbf{not} look at the first token of this new replacement text to expand it again or to execute it.
\end{enumerate}



\begin{teX}
\def\xx [#1]{[#1]}
\def\yy{[ABC]}

\expandafter\xx\yy
\end{teX}

This results in 
\def\xx [#1]{[#1]}
\def\yy{[ABC]}

\texttt{> \expandafter\xx\yy}


\item token1 is primitive: Normally a primitive token can not be expanded so the |\expandafter| has no effect; but there are exceptions, which we will discuss after the example.

\begin{texexample}{Expansion}{}
\expandafter AB
\end{texexample}

Character A is saved. Then \tex\ tries to expand it, but \textit{not} print B, because B cannot be expanded. Finally A is put back in front of the B ; in other words, the two characters are printed in the given order, and we may well have omitted the |\expandafter|. So what's the point here? |\expandafter| reverses the order of expansion, not of execution.

\noindent But there are exceptions to the above:
\begin{enumerate}
\item \textbf{temporarily suspend an opening curly brace} token 1 is is an opening curly brace which leads to the opening curly brace temporarily suspended. This is listed as a separate case because it has some interesting, applications;

\begin{teX}
\newtoks\ta
\newtoks\tb
\ta = {\a\b\c}
\tb=\expandafter{\the\ta}
\tb={\the\ta}
\tb
\end{teX}

\begin{texexample}{Expansion}{}
\begingroup

\def\a{A}
\def\b{B}
\def\c{C}
\newtoks\ta
\newtoks\tb
\ta = {\a\b\c}
\tb=\expandafter{\the\ta}
\tb={\the\ta}

\texttt{> \the\tb}

\texttt{> \the\ta}

\endgroup
\end{texexample}

\item \meta{$token_1$} is another expandafter. The best way to understand this is to write a \tex mnmal example and watch it in action

\begin{teX}
\tracingmacros=2  \tracingcommands=2
\def\a{A}
\def\b{B}
\def\c{C}

\expandafter\expandafter\expandafter\a\expandafter\b\c

\bye
\end{teX}

Checking the log file with |\tracingmacros=2 \tracingcommands=2| we get

\begin{verbatim}
{vertical mode: \def}
{blank space  }
{\def}
{blank space  }
{\def}
{blank space  }
{\par}
{\expandafter}
{\expandafter}
{\expandafter}

\c ->C
{\expandafter}

\b ->B

\a ->A
{the letter A}
{horizontal mode: the letter A}
{\par}

\meaning\futurenonspacelet
\end{verbatim}


\end{enumerate}


\section{Defining Macros on the fly}

This is a very common requirement. 

\begin{texexample}{csname}{}
\def\newtest#1#2{
  \expandafter\def\csname#1\endcsname{#2}%
}
\newtest{letters}{test for letters}
\letters
\end{texexample}





\end{enumerate}









 
%  \chapter{Futurelet}
\precis{A discussion on one of the most esoteric commands of \protect\tex, with examples as to how to write macros with optional arguments.}
\addtocimage{-12pt}{-20pt}{../images/tocblock-futurelet.jpg}
\epigraph{Life can only be understood backwards; but it must be lived forwards.}{
---S Kierkegaard}

The \cmd{\futurelet} primitive deserves its own chapter, as most people have difficulty in understanding the command. The instruction allows the user to \textit{look ahead}. By look ahead we mean that \tex will look at a future token\footnote{remember that a token is either a single character or a macro command} without absorbing it, i.e, without removing that token from the token list. This operation allows the programmer to perform a test to check what token is 'coming'. You can read a couple of articles about it for example \citep{Eijkhout2001}, but generally they are difficult to follow. The information about the command is also very sparse in the TeXBook.  Another TUGboat article is \citep{bechto88}, which gives pretty much the same example as we describe below. 

The token looked at through
|\futurelet| will be removed later, typically as part
of an argument of a later macro call as we will see
shortly. It is not removed by the action of the
|\futurelet| primitive.

Let us be more precise now; the |\futurelet|
instruction has the following format:


\begin{teX}
\futurelet (tokenl) (token2) (token3)
\end{teX}


\begin{enumerate}
\item  \tex will execute a \cmd{\let}\meta{tokenl}=\meta{token3}.
We therefore have generated a copy of (token3)
stored under the name of (tokenl).\label{lettoken}


\item  removes (tokenl) from the main token list.

\item \tex expands (token2). This token is for all
practical purposes a macro with the following
properties:

(a) The macro will use (tokenl), which is a
copy of (token3), to find out what (token3)
is, in other words what token is to be
expected later.
(b) It will cause another macro to be expanded
which will ultimately absorb (token3).

This other macro ordinarily depends on
what $<token_l>$ is.

\end{enumerate}

The description above, is a bit of a mouthful and it is better to describe it with an example. In Example~\ref{futurelet} we will try and find if the next token is the opening square bracket `['. We then according to the definition in \ref{lettoken} this should be stored in \cs{tokenone}. We verify this by peeking at its meaning.

\begin{texexample}{futurelet}{futurelet}
\def\tokentwo#1{}
\futurelet\tokenone\tokentwo[
\meaning\tokenone
\end{texexample}

The second token \cs{tokentwo} we have defined it, so that it justs absorbs its next argument and does nothing for the time being. As you can see its meaning is \texttt{the character [}. Now what happens if there was a space between the \cs{tokentwo} and the `['?

\begin{texexample}{futurelet second}{futurelet2}
\def\tokentwo#1{}
\futurelet\tokenone\tokentwo     [
\meaning\tokenone
\end{texexample}

As you can see so far the spaces have been absorbed, but let us now change the definition of \cs{tokentwo}.

\begin{texexample}{futurelet second}{futurelet2}
\def\tokentwo#1{}
\futurelet\tokenone\tokentwo     
\meaning\tokenone
\end{texexample}



\begin{texexample}{futurelet}{futurelet}
\def\tokentwo#1{%
   \ifx\tokenone[ true [\else false\fi
}
\futurelet\tokenone\tokentwo[
\meaning\tokenone
\end{texexample}

We try again with spaces,

\emphasis{tokentwo,[}
\begin{texexample}{futurelet}{futurelet}
\def\tokentwo#1{%
   \ifx\tokenone[ true [\else false\fi
}
\futurelet\tokenone\tokentwo     [
\meaning\tokenone
\end{texexample}

As you can see from the examples we cannot capture the spaces. This might present a problem, if we enclose everything in other macros as \tex might leave extra spaces in the stream. Better to absorb them. We will see how later, using LaTeX. 


\section{Applications}

There are many applications of |\futurelet|.
will here present only one example, although
we will present it in quite some detail so the user
will know how to apply |\futurelet| in different
circumstances.

\subsection{Using \textbackslash futurelet in Macros with Optional
Arguments}

A typical application of |\futurelet| is the handling
of macros with optional arguments\cite{Becht1988} as they are used,
for instance, in \latex. By "optional argument" we
mean an argument which in most cases is omitted,
and is provided only occasionally in macro calls.\footnote{See also the discussion at \url{http://tex.stackexchange.com/questions/4557/how-to-use-futurelet-to-define-optional-parameters}}

\textbf{Defining the Problem}

Let us give a specific example: we would like to
define a macro \cmd{xx}, which can be called in two
different ways:

\begin{enumerate}
\item With optional argument as in |\xx [opt]{arg}|
where opt is the optional argument enclosed
in square brackets and \meta{arg} is the mandatory argument
argument.

\item Without optional argument as in |\xx{arg}|
where \meta{arg} is again the regular argument.

\end{enumerate}


Before we discuss how this can be done in \tex,
observe that we do not really have to use an
optional argument. We could simply define two
different macros \cmd{xxwithoptions} for the case where an
optional argument is given, and \cmd{xxnooptions} for the
case where no optional argument is given:


\begin{texexample}{two macros}{ex:twomacros}
\def\xxWithOpt [#1]#2{...}
\def\xxNoOpt #1{...}
\def\xxWithOpt (#1)#2{\fbox{#2}}
\xxWithOpt (box){Testing}
\end{texexample}

How we can use |\futurelet| to find out
whether an optional argument was given or not?

We will define a macro |\xx| whose only function is
to check whether there is an opening square bracket
(optional argument is present) or not (no optional
argument). The |\xx| macro will, after this has been
determined, cause the |\xxWithOpt| macro to be invoked
when there is an optional argument, and the
|\xxNoOpt| macro to be called if there is no opening
bracket. In other words the macros |\xxWithOpt|
and |\xxNoOpt| do the "real work while the only
purpose of the |\xx| macro is to decide which of the
two macros should be invoked.


Here is the completely worked out example.


\begin{teX}
\def \xxWithOpt [#1] #2{...}
\def\xxNoOpt #2{...}

\def\xx {%
\futurelet\xxLookedAtToken
    \xxDecide
}

% (3) The \xxDecide macro, based on
% the lookahead of \xx, calls
% either \xxWithOpt or \xxNoOpt .
\def\xxDecide {%
 \ifx\xxLookedAtToken [%
\let\next = \xxWithOpt
\else
 \let\next = \xxNoOpt
 \fi
\next
}
\end{teX}

\section{Other Applications in the LaTeX kernel}

\begin{teX}
\def\elidebefore[#1]#2{[$\ldots$] #2}
\def\elideafter#1{#1$\ldots$}

\def\elide {%
\futurelet\ifoptions
    \choosemacro
}

\elide{Lorem Ipsum}

\elide[b]{Lorem ipsum}
\end{teX}

\begin{comment}
% The \choosemacro, based on
% the lookahead of \elide, calls
% either \elidebefore or \elideafter 
\end{comment}

\begin{teX}
\def\choosemacro{%
 \ifx\ifoptions [%
     \let\choice = \elidebefore 
 \else
    \let\choice = \elideafter
 \fi
\choice
}
\end{teX}



\begin{teX}
\elide{Lorem Ipsum}

\elide[b]{Lorem ipsum}

\end{teX}

\begin{teX}
\def \xxWithOpt [#1] #2{...}
\def\xxNoOpt #2{...}

\def\xx {%
\futurelet\xxLookedAtToken
    \xxDecide
}

% (3) The \xxDecide macro, based on
% the lookahead of \xx, calls
% either \xxWithOpt or \xxNoOpt .
\def\xxDecide {%
 \ifx\xxLookedAtToken [%
\let\next = \xxWithOpt
\else
 \let\next = \xxNoOpt
 \fi
\next
}
\end{teX}



To build a command with any optional parameter, as you find in many of LaTeX's commands, you will need two things:

\begin{itemize}
\item a macro with delimited parameters

\item a way to grab the first non-space token that follows the command
\end{itemize}


The first part is fairly easy using delimited argument macros, for example we can say

\begin{verbatim}
\def\test(#1)#2#3{#1, #2, #3}
\end{verbatim}

We can then call this macro as:

\begin{verbatim}
\test(a){b}{c}
\end{verbatim}


resulting in a,b,c

To define the |()| as an optional parameter, we effectively need to define the macro as a conditional a sort of a "yes-no" switch. If \tex finds the "(" bracket the "yes-code" will be called and if it finds only the normal arguments the "no-code" will be executed.

For this we can use the |\@ifnextchar| macro from the LaTeX kernel.
You can say |@ifnextchar{char}{yes-code}{no-code}| to test for |(|. The result then will depend on the token that follows. If this token is the same as the first argument, then the "yes-code" is executed, otherwise the "no-code" is executed. The first argument should be a single token (for instance a character). Spaces are ignored. 

As for example we can redefine the LaTeX code for `rule` to accept an optional parameter in round brackets, rather than the traditional square brackets.

\begin{texexample}{Using ifnextchar}{}
\makeatletter
\def\Rule{\@ifnextchar(\@Rule%
        {\@Rule(\z@)}}
\def\@Rule(#1)#2#3{%
 \leavevmode
 \hbox{%
 \setlength\@tempdima{#1}%
 \setlength\@tempdimb{#2}%
 \setlength\@tempdimc{#3}%
 \advance\@tempdimc\@tempdima
 \vrule\@width\@tempdimb\@height\@tempdimc\@depth-\@tempdima}}
\makeatother

A test \Rule(6.5pt){100pt}{1pt}

Another test \Rule{100pt}{3pt}

Not that difficult but you will need to , but why on earth do you need this?
\end{texexample}




\begin{comment}
\def\elidebefore[#1]#2{[$\ldots$] #2}
\def\elideafter#1{#1$\ldots$}

\def\elide {%
\futurelet\ifoptions
    \choosemacro
}

% The \choosemacro, based on
% the lookahead of \elide, calls
% either \elidebefore or \elideafter 


\def\choosemacro{%
 \ifx\ifoptions [%
     \let\choice = \elidebefore 
 \else
    \let\choice = \elideafter
 \fi
\choice
}

Testing \elide[b]{Lorem ipsum}

\elide{Lorem Ipsum}

\elide[b]{Lorem ipsum}

\end{comment}


\section{Using LaTeX \protect\textbackslash @ifnextchar}

\latex defines the |\@ifnextchar| kernel command that is used effectively to
determine the token that follows the command. It is used in the definitions
of macros with optional arguments amongst other things.

\begin{teXXX}
\@ifnextchar]{true}{false}] 
\@ifnextchar[{true}{false}[
\end{teXXX}
The result would both be true,

\begin{texexample}{Example ifnextchar}{ifnextchar}
\makeatletter
\@ifnextchar]{true ]}{false} ] %notice ]
\@ifnextchar[{true [}{false} [ %notice [
\makeatother
\end{texexample}






















%  \chapter[Data Structures]{Data Structures}

\parindent1em 
In computer science, a data structure is a particular way of organizing data in a computer so that it can be
used efficiently. \tex has only one type of data structure: the \emph{token list}, although one can consider a macro storing only contents as another primitive data structure. The original \tex had only 256 token list\footnote{See also etex for a more modern version.} registers that are
available to the user.\tex also offers some special token lists: the |\every|... variables, |\errhelp|,
and |\output|. Lamport in \latex developed additional data structures, using \tex’s primitive commands and  these are discussed in the chapters discussing the \latex kernel. In addition the \latex3 team is busy developing more familiar data structures and associated manipulation routines that one can truly say that the \tex programmer has now a full kit to program any complicated piece of code.


A token register stores a token list. A macro also stores a token list in its \meta{replacement text}, so you can perfectly get away without using token registers in most of your \tex programming. So where is the difference?

\begin{enumerate}
\item Token registers are faster.
\item Token registers will \emph{never} be expanded.
\end{enumerate} 


\begin{description}
\item [toks] Prefix for a token list register.
\item [toksdef]  Define a control sequence to be a synonym for a |\toks| register.
\item [newtoks] Macro that allocates a token list register in |plain.tex|.
\end{description}


Token lists are probably among the least obvious components of \tex: most \tex users will never
find occasion for their use, but format designers and other macro writers can find interesting
applications. Following are some examples of the sorts of things that can be done with token lists.

The number of primitive operations available for token lists is rather limited: assignment\index{token lists!assignment} and
unpacking\index{token lists>unpacking}. However, these are sufficient to implement other operations such as appending\index{token list>appending}.

\section{How to allocate to a token register}

\begin{docCommand}{toks}{\marg{number}}
Allocates a \tex primitive token register.
\end{docCommand}

\begin{texexample}{Basic usage}{ex:toksusage}
\bgroup
 \def\a{a test.}
 \toks0={\a}
 \edef\b{\the\toks0 }
 \def\c{\a}
 \edef\e{\c}
 \def\f{\the\toks0 }

\meaning\b

\meaning\e

\meaning\f
\egroup
\end{texexample}

In this simple test in example~\ref{ex:toksusage} we can see the major differences of using token registers or macros to hold values. An |\edef| holding the |\the\toks0 |, did not expand the values, where |\c| expanded the macro. This is an important difference and can enable one to build lists. Once you have lists, you have a full Turing complete language.

\begin{texexample}{Usage}{ex:toksadd}
\makeatletter
\bgroup
\long\def\g@addto@macro#1#2{%
  \begingroup
  \toks@\expandafter{#1#2}%
  \xdef#1{\the\toks@}%
  \endgroup
}

\def\a{My macro.\kern5pt}

\g@addto@macro\a{My other text }

\meaning\a

\a
\egroup
\makeatother
\end{texexample}

As lightly different macro can do a similar job:

\begin{texexample}{Example with edefappend}{ex:toks3}
% Add #2 (which is expanded in an \edef) to the end of the definition of
% #1 (which must be a previously-defined control sequence).  This is a
% way to construct simple lists. (xeplain)
% 
\makeatletter
\bgroup
\def\edefappend#1#2{%
  \toks@ = \expandafter{#1}%
  \edef#1{\the\toks@ #2}%
}%
%
\def\a{My macro.\kern5pt}
\edefappend\a{My other text. }

\meaning\a
\egroup
\makeatother
\end{texexample}

So far we have used the register |toks0| and \latex2e |toks@| which are the same register. This is bad practice, as it may already be in use. We can use \docAuxCommand{newtoks} to define a new register using a name.

\begin{docCommand}{newtoks}{\marg{name}}
Allocates a new token register.
\end{docCommand}


\begin{texexample}{Token List Allocation}{ex:toks}
\newtoks\alist 
\alist={token1,token2,token3}
\meaning\alist
\end{texexample}

As you can see from the example we can typeset the contents of the token register with the \cs{the} command,

\begin{texexample}{Token List with macros}{}
\begingroup
\def\c{one}
\def\d{two}
\newtoks\blist 
\blist={{\c} {\d}}
\the\blist. 
\endgroup
\end{texexample}


\begin{texexample}{Token List with macros}{}
\begingroup
\def\c{one}
\def\d{two}
\newtoks\blist 
\blist={\c \d}
\the\blist. 
\endgroup
\end{texexample}

New token lists can be created, by either using the \cmd{\newtoks} or the \cmd{\toks}\meta{number}. Where |\toks0| defines the zero register etc. It is always better to use \cmd{\newtoks} in order not to affect existing token registers used by the system or other packages.

Token register can store anything for example they can store \cmd{\vfil} or \cmd{\hfill}. \latexe uses them extensively, using one or two registers but mostly using |\@temptoken|. Do not be tempted to use this variable as it can affect the marks in your document. Good practice in your package is to define a scratch register |\mypackagetemptokena| etc.

\section{Operations}

\subsection{copying}

You can copy the contents of one token register to another by assignment. The copying is carried out \emph{without expansion}.

\begin{texexample}{}{}
\makeatletter
\newtoks\@exampletoksa
\newtoks\@exampletoksb
\@exampletoksa = {first token list}
\@exampletoksb=\@exampletoksa

% the @exampletoksb now holds the contents of @exampletoksa
\the\@exampletoksb
\makeatother
\end{texexample}




\section{Collecting Information with Token Registers}

One important use of token registers is to store information. Normally the package will provide some commands to add tokens, remove comments etc.

\begin{teXXX}
\def\addinfo #1{%
    \expandafter\expandafter\expandafter\collecttokens\expandafter{%
          \the\collecttokens #1}
}
\end{teXXX}


\begin{texexample}{Adding to a token list}{ex:adding}

\bgroup
\makeatletter
\def\addinfo#1{
  \expandafter\expandafter\expandafter
    \@exampletoksa\expandafter{%
     \the\@exampletoksa #1 }%
}
\addinfo{\hfill}
\addinfo{CHAPTER}
\addinfo{\kern0.5em}
\addinfo{50}
\the\@exampletoksa

\makeatother
\egroup

\end{texexample}


\section{Joining two token registers}

We have see so far how to allocate a token register to another, we can also join two token registers by expanding both in a macro or a token register:

\begin{texexample}{Join two token registers}{ex:joining}
\makeatletter
\bgroup
\toks0={}
\newtoks\@temptokenb
\@temptokena={tokena\\ }
\@temptokenb={tokenb}

\def\jointoks#1#2#3{%
  #1=\expandafter\expandafter\expandafter
    {\expandafter\the\expandafter#2\the#3}}

\jointoks{\toks0}{\@temptokena}{\@temptokenb}

\the\toks0 
\egroup
\makeatother
\end{texexample}




\begin{teXXX}
% 1. Vereinigung zweier token register
%
% Ergebnis " #1={ <Inhalt von #2> <Inhalt von #3>} "
%
\def\JoinToks#1=(#2+#3){#1=\expandafter\expandafter\expandafter
{\expandafter\the\expandafter#2\the#3}}
%===============================================================
%
% 2. Ahnlich, jedoch mit der Angabe eines Ziels
% Ergebnis " { <Inhalt von #1> <Inhalt von #2> } "
%
\def\Union(#1,#2){\expandafter\expandafter\expandafter
{\expandafter\the\expandafter#1\the#2}}
%
\def\UpToHere{\relax}%
\def\IgnoreRest#1#2\UpToHere{#1} % helper macro
\def\IgnoreFirst#1#2\relax\UpToHere{#2} % helper macro


%===============================================================
% 3. liefert das erste Element eines token register #1
%
\def\First#1{\expandafter\IgnoreRest\the#1{}\UpToHere}
%===============================================================
% 4. liefert das erste Element eines token register #1 mit
% umgebenden Klammern " { ... } "
%
\def\FirstOf#1{\expandafter\expandafter\expandafter
{\expandafter\IgnoreRest\the#1{}\UpToHere}}
%===============================================================
% 5. weist das erste Element eines token register #1
% auf das zweite token register #2 zu
%
\def\MoveFirst(#1to#2){#2=\FirstOf{#1}}
%===============================================================
% 6. gibt alle Elemente aus dem token register #1,
% außer dem ersten aus.
%
\def\Rest#1{\expandafter\IgnoreFirst\the#1\relax\UpToHere}
%===============================================================
% 7. wie in (6), jedoch mit umgebenden Klammern " { ... }"
%
\def\RestOf#1{\expandafter\expandafter\expandafter
{\expandafter\IgnoreFirst\the#1\relax\UpToHere}}
%===============================================================
% 8. weist alle Elemente des token register #1 außer dem
% ersten auf #2 zu
%
\end{teXXX}

\def\MoveRest(#1to#2){#2=\RestOf{#1}}

We can write some macros to manipulate the toks registers as shown in listing no 3. By capturing the firsttoken and the rest of tokens, you can actually write a macro to transverse the token list.

\begin{minipage}[t]{4cm}
\begin{teX}
\toks1={one}                         
\toks2={two}                         
\toks3={{one}{two}}            
\ToksOne={\number1}          
\ToksTwo=\toks2                   
\ToksThree=\toks3                
\end{teX}

\end{minipage}
\hspace{1.5cm}
\begin{minipage}[t]{4cm}
\begin{teX}
\toks4=\FirstOf{\toks1}
\toks5=\RestOf{\toks2}
\toks7=\FirstOf{\toks3}
\toks8=\RestOf\ToksOne
\toks9=\Union(\toks1,\toks2)
\MoveRest(\toks9 to\toks0)
\end{teX}
\end{minipage}
\bigskip  

The result is 

\bigskip

{\leftskip 2em
\begin{tabular}{llllll}
|\the\toks1|       &$\rightarrow$  &|one| &|\the\toks4|  &$\rightarrow$  &one\\
|\the\toks2|       &$\rightarrow$  &|two| &|\the\toks5|   &$\rightarrow$ &two\\
|\the\toks3|       &$\rightarrow$  &|{one}{two}| &|\the\toks7| &$\rightarrow$    &one\\
|\the\ToksOne|  &$\rightarrow$  &|\number1|    &|\the\toks8| &$\rightarrow$ &1\\
|\the\ToksTwo|   &$\rightarrow$  &two &|\the\toks9| &$\rightarrow$     &onetwo\\
|\the\ToksThree| &$\rightarrow$ &|{one}{two}| &|\the\toks0| &$\rightarrow$ &netwo\\ 
\end{tabular}
}

\section{Joining two registers}

Now, that we have described the basic commands of creating a token register (assignment) and unpacking it with the \cmd{the}, we can develop some macros to manipulate such lists. The first macro we will develop is a macro to \textit{join}
two lists and store the result in a third token list \cmd{\result}.

\begin{texexample}{Joining two registers}{}
\def\JoinToks#1#2+#3;{#1=\expandafter\expandafter\expandafter
    {\expandafter\the\expandafter#2\the#3}}

\toks1={{Alice }{John }{Mary }}
\toks2={{Marilou }{Maria }{Marianne }}
\toks3={{John}{Yannis}{Yiannis}}

\newtoks\result
\JoinToks\result\toks1+\toks2;
\the\result
\JoinToks\result\result+\toks3;
\end{texexample}

LaTeX2e has a temp scratch register \cmd{\@temptokena}.


Note that the equal sign in |\JoinToks\result=(\toks1+\toks2)| and the brackets are by design, ie, by the definition of \cmd{JoinToks}. The same with the plus sign (+). You could omit all of them and just use commas or just spaces. Also remember to separate the different elements of the list by using brackets curly brackets. If you omit them, \tex will only read the first letter!



\section{Union}
Similarly we can define a command \cmd{\Union} which is very similar to \cmd{\JoinToks} and is perhaps more intuitive.


\begin{texexample}{Union of Two Token Lists}{}
\toks1{test1,test2,test3}
\toks2{test1,test4,test5}
\def\Union(#1,#2){\expandafter\expandafter\expandafter
{\expandafter\the\expandafter#1\the#2}}
\toks9=\Union(\toks1,\toks2) 
\the\toks9
\end{texexample}





\subsection{Helper Macros}
The next macros are helper macros to assist in the rest of the definitions. The \cmd{\UpToHere} macro is a simple \cmd{\relax}, where the \cmd{\IgnoreRest} and \cmd{\IgnoreFirst} are defined as per their names.

\begin{teX}
% Helper macros 
%
\def\UpToHere{\relax}%
\def\IgnoreRest#1#2\UpToHere{#1} % helper macro
\def\IgnoreFirst#1#2\relax\UpToHere{#2} % helper macro
\end{teX}


\begin{teX}

% 3. Returns the first element of a token register #1
%
\def\First#1{\expandafter\IgnoreRest\the#1{}\UpToHere}

% 4. returns the first element of a token register # 1 with
% Surrounding brackets "{...}"
%
\def\FirstOf#1{\expandafter\expandafter\expandafter
{\expandafter\IgnoreRest\the#1{}\UpToHere}}

% 5. Move  the first element of a token register # 1
% To the second token register # 2 to
%
\def\MoveFirst(#1to#2){#2=\FirstOf{#1}}

% 6. Move all elements of the token register # 1,
% Except for the first out.
%
\def\Rest#1{\expandafter\IgnoreFirst\the#1\relax\UpToHere}

% 7. as in (6), but with surrounding brackets "{...}"
%
\def\RestOf#1{\expandafter\expandafter\expandafter
{\expandafter\IgnoreFirst\the#1\relax\UpToHere}}

% 8. weist alle Elemente des token register #1 auer dem
% ersten auf #2 zu

\end{teX}

If you have read up to here, you would be wondering as to how to access the \textit{length} of the token register, pop and push, slice etc. Common terminologies of lists and arrays\footnote{Remember that a list is a one dimensional array. It is quite possible to build all these commands using \TeX\ and as a matter of fact \LaTeX has all these constructs built-in.} We would demonstrate this by example.


\subsection{How to find the length of a token list}
\index{token lists!length}

These examples are from \TeX by Topic. They have been modified to print their
output, rather than |\message{}|, in order to print the result here.


We first define a toks register and a count register, named \cmd{auxlist} and \cmd{auxcount}.

\begin{teX}
\newtoks\auxlist 
\newcount\auxcount
\end{teX}



First of all there must be an operation to add auxiliary files:

\begin{teX}
\def\NewAuxFile#1{\AddToAuxList{#1}%
% plus other actions
}
\end{teX}

\def\NewAuxFile#1{\AddToAuxList{#1}%
% plus other actions
}

\noindent Next we define a macro that adds a token to the token list, but also delimits it using \docAuxCommand{elt}. The token |\@elt| is commonly used by \latex in list constructions. It has been borrowed from Lisp which and is a short for element. Knuth used throughout |plain| two backslashes (\textbackslash\textbackslash). As a matter of fact you can use any marker you want. It is preferable though to adhere to these two conventions as they are commonly used throughout packages and in the literature.

Map takes a function \textbf{f} and a list $xs$ and applies f
to every element of xs. For example,

$$
\mbox{Map}\,[1,2,3] = [f1,f2,f3]
$$



\begin{texexample}{Elt lists}{ex:eltlist}
\makeatletter
% allocations
\global\newtoks\auxlist 
\newcount\auxcount

% Define a macro to add to the list
\protect\gdef\addtoauxlist#1{\let\@elt=\relax
  \xdef\act{\noexpand\auxlist={\the\auxlist \@elt{#1}}}%
  \act}

% Add some elements to the list
\addtoauxlist{one}  
\addtoauxlist{two}
\addtoauxlist{three}

\the\auxlist

\the\auxlist
\makeatother
\end{texexample}



By redefining \docAuxCommand*{@elt} we can capture each element during expansion and map to the elements other functions. Consider that we just want to print the elements.


\begin{texexample}{printing the list}{ex:printlist}
\makeatletter
% define a macro to print the list
\the\auxcount
\def\printauxlist{%
  \def\@elt##1{\advance\auxcount1\relax\the\auxcount. \itshape ##1\par }
  \the\auxlist }
  
% add some values
\addtoauxlist{one}  
\addtoauxlist{two}
\addtoauxlist{three}  
\addtoauxlist{four}  
\addtoauxlist{five}

% print the results
\printauxlist
\makeatother  
\end{texexample}

The space after |\auxcount1| is important to signal to \tex the end of the number 1. It is better to actually write |\relax|. Try it without and it will just keep on printing only the dot only.



\endinput
Another use of this structure is the following: at the end of the job we can now close all auxiliary files at once, by defining,
%
%\begin{teX}
%\def\CloseAuxFiles{
%  \def\@elt##1{\CloseAuxFile{##1}}%
%  \the\auxlist}
%
%\def\CloseAuxFile#1{closing file: #1. %
%% plus other actions
%}
%\end{teX}
%\def\CloseAuxFiles{
%  \def\@elt##1{\CloseAuxFile{##1}}%
%  \the\auxlist}
%
%\def\CloseAuxFile#1{closing file: #1. %
%% plus other actions
%}
%
%
%\noindent which gives the output
%
%\texttt{> \CloseAuxFiles}
%
%\def\alist{}
%\listadd{\alist}{Yiannis~}
%\listadd{\alist}{Yianis~}
%\listadd{\alist}{Ioannis~}
%\listadd{\alist}{Giannis~}
%\alist
%
%\makeatother
%
%\def\tempa{}
%\numdef{\tempa}{(22+35)*45}
%
%\tempa


%
%\section{String comparisons}
%
%A more general problem along the same lines is to
%check if two words, or strings are the same. We can
%use |\ifx| for this as well. When |\ifx| compares two
%tokens that are macro names, the result is true if
%the macros have been defined in the same way, and
%if their first level replacement texts are the same.
%So, we define two macros whose replacement texts
%are the strings, and compare these.
%
%\numberLineAt{50}
%\begin{teX}
%\newif\ifsame
%\newcommand{\strcomp}[2]{%
% \samefalse
% \begingroup
%   \def\1{#1}\def\2{#2}%
%   \ifx\1\2\endgroup True \sametrue
%   \else False 
% \endgroup
%\fi}
%\strcomp{Yiannis}{Yiannis}\\
%\strcomp{Yiannis}{YIANNIS}\\
%\end{teX}
%
%\newif\ifsame
%\newcommand{\strcomp}[2]{%
%\samefalse
%\begingroup
%\def\1{#1}\def\2{#2}%
%\ifx\1\2\endgroup True \sametrue
%\else False \endgroup
%\fi}
%
%
%\printf{\strcomp{Yiannis}{Yiannis}}
%\printf{\strcomp{Yiannis}{YIANNIS}}
%
%
%
%\section{Lists}
%
%A list is simply a one dimensional array, normally delimited by commas:
%
%\begin{teX}
%  \def\somelist(1,2,3,4,5,6,7,8)
%\end{teX}
%
%In TeX you will probably better off defining it as:
%
%\begin{teX}
%  \def\somelist{1,2,3,4,5,6,7,8}
%\end{teX}
%
%The package \cmd{coolist} provides basic control sequences for manipulating such lists
%
% Lists are defined as a sequence of tokens separated by a comma.  The \texttt{coollist} package allows the user
% to access certain elements of the list while neglecting others---essentially turning lists into a sort of
% array. 
%
% List elements are accessed by specifying the position of the object within the list (the index of the item) and
% all lists start indexing at |1|.
%
% 
% \begin{tabular}{ll}
% |\listval{1,2,3,4}{2}|                & \listval{1,2,3,4}{2} (the null string)        \\
% |$\listval{\alpha,\beta,\gamma}{2}$|  & $\listval{\alpha,\beta,\gamma}{2}$            \\
% |\listval{a,b,c}{4}|                  & \listval{a,b,c}{4} (the null string)
% \end{tabular}
%
%The \pmac{coolist}{liststore} stores the length of the comma delimited list into the counter. It does so by creating variables with the same name as the argument.
%
%\begin{teX}
%\liststore{1,2,3,4}{temp}
%\tempi;\tempii;\tempiii;\tempiv 
%\end{teX}
%
%produces 
%
%\liststore{1,2,3,4}{temp}
%\tempi;\tempii;\tempiii;\tempiv 
%
%This can be used quite effectively to create a number of on the fly variables.
%
%The list can be numerical or alpha
%
%\begin{teX}
%\liststore{alpha,beta}{temp}
%\end{teX}
%
%
%will produce
%
%\liststore{alpha,beta}{temp}
%
%\texttt{\medskip\tempi; \tempii \medskip }
%
%\section*{length of list}
%
%The length of the list can be obtained by using the \pmac{listcool}{listlen} or \pmac{listcool}{listlenstore}
%
%\begin{teX}
%\listlen{1,2,3,4,5} 
%\listlen{} 
%\listlen{1,2} 
%\listlen{1} 
%\end{teX}
%
%\medskip
%
%|\listlen{1,2,3,4,5}| \texttt{\listlen{1,2,3,4,5}}\\
%
%
%\listlen{} 
%\listlen{1,2} 
%\listlen{1} 
%\medskip
%
%You can copy one list into another using \cmd{listcopy}
%
%\begin{teX}
%\liststore{1,2,3}{temp}
%\listcopy{temp}{copiedlist}
%\copiedlisti;\copiedlistii;\copiedlistiii 
%\end{teX}
%
%{\obeylines\tt
%\liststore{1,2,3}{temp}
%\listcopy{temp}{copiedlist}
%\copiedlisti;\copiedlistii;\copiedlistiii 
%}
%
%
%You can get the sum of a list by using \cmd{listsum}
%
%
%\listsum{1,2,3,4,5}{\thelistsum}
%\thelistsum 
%\listsum{1,2,3,a,b,a,a}{\thelistsum}
%\thelistsum 
%\liststore{1,2,3,5,j,k,j}{temp}
%
%
%\listsum[liststored=true]{temp}{\thelistsum}
%\thelistsum 11+2j+k
%\listsum{a,b,c,d}{\thelistsum}
%\thelistsum
%
%
%An ingenious way of providing a consistent user interface with the |coolllist| commands is in the package \docpkg{cool}, by the same author. The code below is from the package and is used to define the display of a Fibonacci number:
%
%\begin{teX}
%$$\Fibonacci{n,x}  or~ \Fibonacci{n}$$
%$$\Fibonacci{n,x+1}  or~ \Fibonacci{n}$$
%$$\Multinomial{n_1, n_2, \ldots, n_m}$$
%\end{teX}
%
%$$\Fibonacci{n,x}  or~ \Fibonacci{n}$$
%$$\Fibonacci{n,x+1}  or~ \Fibonacci{n}$$
%$$\Multinomial{n_1, n_2, \ldots, n_m}$$
%
%
%
%Note the $\ldots$ are not lost!
%
%\begin{teX}
%% \begin{macro}{\Fibonacci}
%% Fibonacci number, |\Fibonacci{n}|, $\Fibonacci{n}$, and 
%%
%% Fibonacci Polynomial, |\Fibonacci{n,x}|, $\Fibonacci{n,x}$
%
%\newcommand{\COOL@notation@FibonacciParen}{p}
%\newcommand{\Fibonacci}[1]{%
%\liststore{#1}{COOL@Fibonacci@arg@}%
%\listval{#1}{0}%
%\ifthenelse{\value{COOL@listpointer} = 1}%
%   {F_{#1}}
%   % ElseIf
%  {\ifthenelse{\value{COOL@listpointer} = 2}%
%      {F_{\COOL@Fibonacci@arg@i}%
%	\COOL@decide@paren{Fibonacci}{\COOL@Fibonacci@arg@ii}}%
%% Else
%   {\PackageError{cool}{Invalid Argument}%
%    {`Fibonacci' can only accept a 
%    comma separate list of length 1 or 2}}}%
%}
%\end{teX}



%  \parindent1em
\chapter{File Input and Output using Primitive TeX Commands}

\tex provides commands for writing and reading of streams
either from a file or a keyboard. Both the available commands as well as a limitation in the number of files that can be allocated, shows \tex's age. For more complicated programming one needs to escape to the shell and use a scripting language or LuaTeX (see \nameref{ch:luaio} and \nameref{ch:l3files} for \latex3 i/o handling.)

An example from the \latexe kernel, can illustrate better than
words the mechanism. The example is the definition of
the command \cs{bibliography}. Bibliographic information is written first to the |.aux| file, and then at the second run is read from a file the extension |.bbl|.

\begin{codeexample}[code only,vbox]
   \def\bibliography#1{%
   \if@filesw
     \immediate\write\@auxout{\string\bibdata{#1}}%
   \fi
   \@input@{\jobname.bbl}}
\end{codeexample}

\tex\ and \latex\  ability to read and write to external filesmakes it possible to produce
a Table of Contents or a List of Figures. 


Table , summarizes the available \tex\ commands. 



\begin{docCommand*}{input} {\marg{file name}}
The command \cs{input} inputs the specified file as \tex input and is perhaps the most widely  i/o command used by authors. 
\end{docCommand*}

\begin{texexample}{}{}
\meaning\input
\end{texexample}



\begin{docCommand}{endinput}{}
\cs{endinput} Terminate inputting the current file after the current line. This is recommended to be inserted at the
end of packages and other files in order to avoid spurious spaces after the file is included.
\end{docCommand}

\begin{docCommand*}{pausing}{}
\cs{pausing}  Specify that TEX should pause after each line that is read from a file.
\end{docCommand*}

\begin{docCommand*}{inputlineno}{}
\cs{inputlineno} Number of the current input line.
\end{docCommand*}

\begin{docCommand*}{message}{}
\cs{message} Write a message to the terminal. This is used widely in the kernel which is sprinklered with code such as:
\end{docCommand*}

\begin{teXXX}
\message{registers,}
\end{teXXX}

\begin{docCommand*}{write} {}
\cs{write} write  text to the terminal or to a file. 
\end{docCommand*}

\begin{docCommand*} {read} {}
\cs{read} Read a line from a stream into a control sequence.
\end{docCommand*}

\begin{docCommand*}{newwrite}{}
\cs{newread} \cs{newwrite} Macro for allocating a new input/output stream.
\end{docCommand*}

\begin{docCommand*}{openin}{\meta{4-bit number} = \meta{file name}}
\cs{openin} \cs{closein} Open/close an input stream. The command opens a file for input. The file may then 
be read line by line using |\read| or a test can be made to check if a file actually exixist or not. The number must be between 0 and 15. 
\end{docCommand*}

\begin{docCommand*}{openout}{}
\cs{openout} \cs{closeout} Open/close an output stream.
\end{docCommand*}

\begin{docCommand*}{ifeof}{}
\cs{ifeof} Test whether a file has been fully read, or does not exist.\\
\end{docCommand*}

\begin{docCommand*}{immediate}{}
\cs{immediate} Prefix to have output operations executed right away.\\
\end{docCommand*}

\begin{docCommand*}{escapechar}{}
\cs{escapechar} Number of the character that is used when control sequences are being converted into character tokens. IniTEX default: 92.\\
\end{docCommand*}

\begin{docCommand*}{newlinechar}{}
\cs{newlinechar} Number of the character that triggers a new line in \cs{write} and \cs{message}
statements.
\end{docCommand*}

\section{Opening and closing files}

It is easy to write and read text files from inside a \tex\  document. The \cmd{\input} is well known and is commonly used
to break a long document into smaller--and more logical parts. In addition the \cmd{\read} makes it  possible to
read a file record by record. New files can be created by \tex\ and data written on them record by record. There can be
a maximum of 16 input and 16 output files open at any given time. Each file is identified  internally by means of a file number. 


The \cmd{\newread} and \cmd{\newwrite} generate the next available file number. 



Output is done by write or \cmd{\immediate}\cmd{\write}. Input is done either by \refCom{read} to or 
\verb+ \input<filename>+. Each write creates a record on the file, whose maximum size is only limited by the operating system, so these can be quite large.

\subsection{Writing control sequences to files}

An important feature of file output is that expandable tokens are expanded during a \refCom{write}. If the
name of a control sequence, rather than its expansion, should be written on a file, either
\cmd{\noexpand} or \cmd{\string} should be used to inhibit the expansion. If a control sequence is unexpandable,
its name is written on the file. If it is undefined an error message is issued when \tex\ tries to expand it during
the \cmd{\write}. It is also possible to avoid expansion during a \\cs{write} by changing the catcode of `\textbackslash'.
This way, anything that starts with a backslash is no longer considered a control sequence.

To complicate matters more, the actual write is deferred until the current page is shipped out. The reason for that is that the user may want to write the page number on the file (this is common when a table-of-contents file or index file is generated) and this number is only known inside the Output Routine. If no page numbers are involved, the user can force the record to be written on the file immediately by using |\immediate\write|:

\begin{teXXX}
  \immediate\write
\end{teXXX}

\section{Interaction with the user}

File numbers are between 0 and 15. File numbers outside this range refer to the standard I/O devices. If you 
write

\begin{teXXX}
  \read-1 to \note
\end{teXXX}

will read from the keyboard without a prompt into \cmd{\note}. The quantity \cmd{\note} does not need to be predefined
or declared. Once input is read into it, \\cs{note}  can be expanded like a macro.
The |filenumber| 16 displays information to the screen.

\begin{teXXX}
  \write16{...}
\end{teXXX}


\section{Writing Arbitrary Strings on a File}

We start with a short review of \cmd{\edef}. In |\edef\abc{\xyz \kern1em}|, the control 
sequence |\xyz| is expanded immediately (when |\abc| is defined), but the 
\cmd{\kern} is only executed later, when |\abc| is expanded.


\section{Writing to standard latex files}

You can write to the aux file with

|\write\@auxout{hello}|

or

|\immediate\write\@auxout{hello2}|

or

|\protected@write\@auxout{}{hello3}|

Depending on requirements.

|\immediate\write| writes to the specified file at that point, expanding the supplied tokens (like |\edef|) so fragile commands will do the wrong thing.

\cs{write} does not write at that point it puts a write node into the current vertical or horizontal list and if that list is shipped out to make a page then the write happens. This is needed to get page numbers correct. (If the write is inside a box and that box is never used on the main page then nothing is written to the file.)


|\protected@write| is a LaTeX-defined macro that uses |\write| but arranges that |\protect| works as required in LaTeX to protect fragile commands. The extra argument unused above allows you to locally insert extra definitions to make more commands be safe or have special definition in the write, see for example the definition of |\index| or |\addtocontents|.


\subsection{Writing to the auxiliary file }

It is safe to write to the aux file, however you have to be aware that the file will be read back at least at the begin and end of the document, so you need to write lines that are safe in that context.

If you want to write to your own file then you just need to do

\begin{teXXX}
\newwrite\myfile
\immediate\openout\myfile=\jobname.foo
\end{teXXX}

in the preamble and then replace |\@auxout| by |\myfile| when writing.

Have a look at the way |\tableofcontents| or |\listoftables| or |\listoffigures| work in latex.ltx or documented in source2e. They basically all use

\begin{teXXX}
\def\@starttoc#1{%
  \begingroup
    \makeatletter
    \@input{\jobname.#1}%
    \if@filesw
      \expandafter\newwrite\csname tf@#1\endcsname
      \immediate\openout \csname tf@#1\endcsname \jobname.#1\relax
    \fi
    \@nobreakfalse
  \endgroup}
\end{teXXX}

\section{What to do when you run out of files}

The limitation on the number of files that can be open for writing is discussed further in Chapter~\ref{ch:l3files}.  In this package we have overcome it, by using the \pkgname{morewrites}. Do note that any packages that check for validity of newwrite (such as filecontents) will produce errors when used in conjuction with \pkgname{morewrites}. In this package we have patched the filecontents  package with an internal version.



































%  \let\sidenote\footnote
\let\citep\footcite
\chapter{How to Develop your Own Class or Package}

\cxset {epigraph width=0.67\textwidth}

\epigraph{First there was one user and I took a lot of time to satisfy myself. Then I had 10 users, and a whole new level of difficulties arose. Then I had a hundred users and another level of things happened. I had a thousand users, I had ten thousand each of those were special phases in the development, important. I couldn't have gone with ten thousand until I'd done
it with a thousand. But each time a new wave of
changes came along, the idea was to have \tex get
better, and not get more diverse as it needed to handle
new things.}{Donald Knuth}

\parindent1em

\section{Introduction}


To \emph{make} a book is an interesting and somewhat involved process\footcite{town}. The text is set in type and printed on pages, the pages are gathered and folded into signatures and these are gathered and folded into signatures and these are then bound and covered. Many of the aspects of this process that has passed down to us by previous generations is discussed extensively in other sections of this book.  Class authors have to distill this knowledge in a set of typographical rules to be described in a class file. The first thing such an author must do is to describe the \emph{rationale} of developing such a class. The \docClass{octavo}\citep{octavo} class was developed to enable printing books in dimensions that follow traditional styles. The \citep{memoir}  class to offer a flexible system on which other classes could be based and so does \citep{koma}. The |tufte-book| and |tufte-handout| classes to provide a style that resembles those found in Tufte books. Many Universities offer \emph{Thesis} classes to standardize the way these are produced. Many of these Universities, translated the styles previously typed and the results are a typographical disaster, only mitigated by the ability to display beautiful mathematics. As these are printed on standard \emph{photocopy paper} one cannot do much with the layout. 
\section{What is a class?}
A class is simply a file with the extension \docExtension{cls} containg a set of macros. 
A class can load another class.
\section{Identifying your class}

The first thing a class must do is to identify any other formats it needs and to announce
its name. This is accomplished using the two commands 
\refCom{NeedsTeXFormat} and \refCom{ProvidesClass}.

The following example, delares the version of \LaTeXe\ that it requires and then
gives the class name. It can be found in the preable of most well written classes. You should also put some remarks to identify you as the author, the version number and other similar details. These are discussed in more detail in the next Chapter, where you will see how to automate documentation for your class.

\begin{teX}
\NeedsTeXFormat{LaTeX2e}[1994/06/01]
\ProvidesClass{myclass-book}[2010/12/11 v3.5.0 myclass-book]
\end{teX}

The above syntax must be followed exactly so that this information can be
used by \texttt{LoadClass} or \texttt{documentclass} (for classes) or \docAuxCommand{RequirePackage}
 or\cmd{usepackage} (for packages) to test that the release is not too old.
The whole of this $<release-info>$ information is displayed by \docAuxCommand{listfiles} and
should therefore not be too long.

\begin{teX}
% Load the common style elements
\input{myclass-common.def}
\end{teX}


Another command that can be used is \docAuxCommand{ProvidesFile}. 
This is similar to the two previous commands except that here the fullname,
including the extension, must be given. It is used for declaring any files other
than main class and package files.

This is useful, if you decide to have your main definitions in a separate file.

\section{Class Options}

Before we see in detail how to add options to a class, we need to review a package called
\pkgname{xkeyval}. Unless you are in the business of re-discovering wheels, this is an absolute must
for developing, readable and maintenable code and your class is to provide many options. 
\begin{teX}
\usepackage[textcolor=red,font=times]{mypack}
\end{teX}

Class options are best set by using booleans\docAuxCommand{newboolean}.

We first set a new boolean that we |name@myclass@afourpaper.| This is used using the package
\texttt{ifthen}\sidenote{The ifthen package was developed by 
David Carlisle, can be downloaded at \url{ http://www.ifi.uio.no/it/latex-links/ifthen.pdf }} 
Then we can |DecalareOptionX| and we set the boolean to default to true. If the user then types

myclass[a4paper]

The a4paper options will be set. This is a much better and concise way of defining options.
\cmd{newboolean}


\begin{teX}
\newboolean{@myclass@afourpaper}
\DeclareOptionX[myclass]<common>{a4paper}
  {
   \setboolean{@myclass@afourpaper}
   {true}
  }
\end{teX}
\medskip

Note that the command provide by \texttt{ifthen} \docAuxCommand{setboolean} takes true or false, as \#2, and sets \#1 accordingly. In the above code we set the option as true. 


It is much easier and most programmers use the \texttt{ifthen} package to check
for option booleans

\begin{teX}
\ifthenelse{\boolean{@myclass@afourpaper}}
  {\geometry{
        a4paper,
        left=24.8mm,
        top=27.4mm,
        headsep=2\baselineskip,
        textwidth=107mm,
        marginparsep=8.2mm,
        marginparwidth=49.4mm,
        textheight=49\baselineskip,
        headheight=\baselineskip
    }
  }
 {}
\end{teX}

\section{Set-up the font sizes}

LaTeX does not provide definitions of all the font-sizes. Unless you are
extending an existing class, this is one of the first tasks you need to 
do in your new class.

Normally class authors will define all the commonly defined size commands,
such as  \cmd{small}, \cmd{normalsize} and other similar commands.

In the example shown below, we first start by defining the \cmd{normalsize} font
size. In this book the \cmd{\normalsize}  is defined as 14pt. We also define the vertical
spaces that we need to have abovedisplay and belowdisplayskip. These are all very difficult to
remember and once you have something you are happy with, just copy from class to class
or even define a samll definition file to keep them all together.


{\fontfamily{phv}\selectfont Helvetica looks like this}
and {\fontencoding{OT1}\fontfamily{ppl} Palatino looks like this}.


 The user has access to a number of commands which change the size of
 the fount, relative to the `main' size used for the bulk of the text.


 These \cmd{size} commands issue a \cmd{@setfontsize}\index{Latex kernel!@setfontsize} 
 command.

\begin{teX}
  \@setfontsize\size\font-size{baselineskip} where:
\end{teX}



  \begin{description}
    \item {font-size} The absolute size of the fount to use from
        now on.
    \item{baselineskip} The normal value of \cmd{baselineskip}
        for the size of the fount selected. (The actual value will be
       % |\baselinestretch| * \meta{baselineskip}.)
    \end{description}

A number of commands, defined in the \LaTeX  kernel, shorten the
following  definitions and are used throughout. These are:

    \begin{center}
    \begin{tabular}{ll@{\qquad}ll@{\qquad}ll}
    \verb=\@vpt= & 5 & \verb=\@vipt= & 6 & \verb=\@viipt= & 7 \\
    \verb=\@viiipt= & 8 & \verb=\@ixpt= & 9 & \verb=\@xpt= & 10 \\
    \verb=\@xipt= & 10.95 & \verb=\@xiipt= & 12 & \verb=\@xivpt= & 14.4\\
    \ldots
    \end{tabular}
    \end{center}


\subsection{Setting up the normalsize}
 The user command to obtain the `main' size is \cmd{normalsize}. \LaTeX\
 uses \cmd{@normalsize} \index{Latex kernel!@normalsize} when referring to the main size and maintains this
 value even if \docAuxCommand{normalsize} is redefined. The \docAuxCommand{normalsize} macro also
  sets values for \cmd{abovedisplayskip}, \cmd{abovedisplayshortskip} and 
\cmd{belowdisplayshortskip}.



\begin{teX}
%%
% Set the font sizes and baselines to match Tufte's books
% normalsize
%%
\renewcommand\normalsize{%
   \@setfontsize\normalsize\@xpt{14}%
   \abovedisplayskip 10\p@ \@plus2\p@ \@minus5\p@
   \abovedisplayshortskip \z@ \@plus3\p@
   \belowdisplayshortskip 6\p@ \@plus3\p@ \@minus3\p@
   \belowdisplayskip \abovedisplayskip
   \let\@listi\@listI}

\normalbaselineskip=14pt
\normalsize
\end{teX}



\begin{teX}
\renewcommand\small{%
   \@setfontsize\small\@ixpt{12}%
   \abovedisplayskip 8.5\p@ \@plus3\p@ \@minus4\p@
   \abovedisplayshortskip \z@ \@plus2\p@
   \belowdisplayshortskip 4\p@ \@plus2\p@ \@minus2\p@
   \def\@listi{\leftmargin\leftmargini
               \topsep 4\p@ \@plus2\p@ \@minus2\p@
               \parsep 2\p@ \@plus\p@ \@minus\p@
               \itemsep \parsep}%
   \belowdisplayskip \abovedisplayskip
}
\renewcommand\footnotesize{%
   \@setfontsize\footnotesize\@viiipt{10}%
   \abovedisplayskip 6\p@ \@plus2\p@ \@minus4\p@
   \abovedisplayshortskip \z@ \@plus\p@
   \belowdisplayshortskip 3\p@ \@plus\p@ \@minus2\p@
   \def\@listi{\leftmargin\leftmargini
               \topsep 3\p@ \@plus\p@ \@minus\p@
               \parsep 2\p@ \@plus\p@ \@minus\p@
               \itemsep \parsep}%
   \belowdisplayskip \abovedisplayskip
}
\renewcommand\scriptsize{\@setfontsize\scriptsize\@viipt\@viiipt}
\renewcommand\tiny{\@setfontsize\tiny\@vpt\@vipt}
\renewcommand\large{\@setfontsize\large\@xipt{15}}
\renewcommand\Large{\@setfontsize\Large\@xiipt{16}}
\renewcommand\LARGE{\@setfontsize\LARGE\@xivpt{18}}
\renewcommand\huge{\@setfontsize\huge\@xxpt{30}}
\renewcommand\Huge{\@setfontsize\Huge{24}{36}}

%% Define a HUGE for fun
\newcommand\HUGE{\@setfontsize\Huge{38}{47}}  
\end{teX}


\section{Adjusting paragraph parameters}

 The parameters which control \TeX 's behaviour when typesetting
 paragraphs receive a bit of a tweak here. Contrary to the usual
 behaviour of modifying the grid with glue when difficulties are
 encountered with vertical space, here we shall try to counteract
 these tendencies and enforce as much as possible uniformity of the 
 grid of lines.

A good value for paragraph indentation is \texttt{parindent 0.5pt}, for vertical spacing between
paragraphs that are indented use 0pt. At this point if you are using any marginals it is a good idea
to allow hyphenation with the \docpkg{ragged2e} package. Since marginals use very narrow paragraphs you may
get a very funny looking marginal text. Using the package, adjustments can be made to hyphenate
the marginal text.

\begin{teXXX}
%%
% \RaggedRight allows hyphenation

\RequirePackage{ragged2e}
\setlength{\RaggedRightRightskip}{\z@ plus 0.08\hsize}
\setlength{\RaggedRightParindent}{1pc}

% Paragraph indentation and separation for normal text
\newcommand{\@tufte@reset@par}{%
  \setlength{\RaggedRightParindent}{1.0pc}%
  \setlength{\parindent}{1pc}%
  \setlength{\parskip}{0pt}%
}
\@tufte@reset@par

% Paragraph indentation and separation for marginal text
\newcommand{\@tufte@margin@par}{%
  \setlength{\RaggedRightParindent}{0.5pc}%
  \setlength{\parindent}{0.5pc}%
  \setlength{\parskip}{0pt}%
}
\end{teXXX}


\section{Formatting Chapters and Sections}

The section on Chapters etc, has more on this, but we will touch on it briefly.
Most recent class developerss use the \pkg{titlesec} and \pkg{titletoc} package to handle the complexity 
of these commands. With the |phd| package this is unecessary. 

\begin{teXXX}
\titleformat{\subsection}%
  [hang]% shape
  {\normalfont\large}% format applied to label+text removed \itshape
  {\thesubsection}% label
  {1em}% horizontal separation between label and title body
  {}% before the title body
  []% after the title body
\end{teXXX}

These are normally followed by the ``titlespacing" commands to define the space around these sections.

\begin{teXXX}
%% We set the titlespacing using the package titlesec and titletoc
%
\titlespacing*{\chapter}{0pt}{20pt}{40pt}
\titlespacing*{\section}{0pt}{3.5ex plus 1ex minus .2ex}{2.3ex plus .2ex}
\titlespacing*{\subsection}{0pt}{3.25ex plus 1ex minus .2ex}{1.5ex plus.2ex}
\end{teXXX}

\section{Adjusting the Index}

For classes representing books, the index is treated like a chapter whereas for others it is normally
treated like a section. Whatever your document ends up like, indices are best done in a multi-column environment.
One possibility is shown below, using the package "multcol".

\begin{teXXX}
\RequirePackage{multicol}
\renewenvironment{theindex}
  {\begin{fullwidth}%
    \small%
    \ifthenelse{\equal{\@tufte@class}{book}}%
      {\chapter{\indexname}}%
      {\section*{\indexname}}%
    \parskip0pt%
    \parindent0pt%
    \let\item\@idxitem%
    \begin{multicols}{3}%
  }
  {\end{multicols}%
    
\renewcommand\@idxitem{\par\hangindent 2em}
\renewcommand\subitem{\par\hangindent 3em\hspace*{1em}}
\renewcommand\subsubitem{
    \par\hangindent 4em\hspace*{2em}
}
\renewcommand\indexspace{
    \par\addvspace{
       1.0\baselineskip plus 0.5ex minus 0.2ex}\relax
    }%
%we now  swallow the letter heading in the index
\newcommand{\lettergroup}[1]{}

\end{teXXX}

The code, renews the "theindex" environment, with minor tweaks and defines it as a three column
layout at "fullwidth".

\section{Provide some hooks}
It is useful at the end of the class to allow for localization of the class
by importing a local file. This is easily achieved by checking if the file exists
and then loading it.  If there is a |myclass-book-local.sty|  file, load it.

\begin{teX}
\IfFileExists{myclass-book-local.tex}
  {input{myclass-book-local}
   \MyClassInfoNL{Loading myclass-book-local.tex}}
  {}
\end{teX}

If you intent to publish your class, you may also want to consider adding a hook for a patch-file.


\section{The final act of kindness to your users}
Many common classes, such as the |memoir| use such a tactic to avoid breaking old code.\index{IfFileExists}

\begin{teX}
 \IfFileExists{mypatch.sty}{%
 \RequirePackage{mypatch}}{}
\end{teX}


\parindent1em
\chapter{How to Package Your Class}

In the previous chapter we have outlined the main sections that you probably need
to define in your class. In the examples we have used we just typed the examples
as |example.cls| or |package.sty|.

In this chapter we will go over the packaging of the class
and automating the generation of user documentation, using the |doc| and \pkg{DocStrip}\footcite{docstrip}
programs in files with an extension |.dtx|. The DocStrip program is an amazing piece of code that was originally
created by Frank Mittelbach to accompany the |doc| package. The idea behind it was to remove comment lines
in order to reduce the execution time of the program. Having created the DocStrip program to remove comment lines from  programs it became feasible to do more than just strip comments.
Wouldn't it be nice to have a way to include parts of the code only when some
condition is set true? Wouldn't it be as nice to have the possibility to split the
source of a \tex program into several smaller files and combine them later into
one `executable'? Both these wishes have been implemented in the DocStrip program.



You should also be
familiar with ``LaTeX2e'' for Class and Package Writers”, which is available
from CTAN (\url{http://www.ctan.org}) and comes with most LaTeX2e" distributions
in a file called clsguide.dvi.\footcite{pakin2004}  Finally, you should know how to
install packages that are shipped as a \texttt{.dtx} file plus a \texttt{.ins} file.

style (.sty) file is primarily a collection of macro and
environment definitions. One or more style files (e.g., a main style file that
\cs{input}  or \cs{RequirePackages} multiple helper files) is called a package.
Packages are loaded into a document with \cs{usepackage}\marg{main .sty fille}.
In the rest of this document, we use the notation \meta{package} to represent
the name of your package.


Motivation The important parts of a package are the code, the documentation
of the code, and the user documentation. Using the \docpkg{Doc}  and
DocStrip programs, it’s possible to combine all three of these into a single,
documented LATEX(.dtx) file. The primary advantage of a .dtx file is that
it enables you to use arbitrary LATEX constructs to comment your code.
Hence, macros, environments, code stanzas, variables, and so forth can be
explained using tables, figures, mathematics, and font changes. Code can
be organized into sections using LATEX’s sectioning commands. Doc even
facilitates generating a unified index that indexes both macro definitions (in
the LATEX code) and macro descriptions (in the user documentation). 

This emphasis on writing verbose, nicely typeset comments for code—essentially
treating a program as a book that describes a set of algorithms—is known
as literate programming \cite{literate} and has been in use since the early days of \tex\ .

Furthermore,
this tutorial shows how to write a single file that serves as both documentation
and driver file, which is a more typical usage of the \texttt{Doc} system than
using separate files.

\subsection{The .ins file}

The first step in preparing a package for distribution is to write an installer
(|.ins|) file. An installer file extracts the code from a |.dtx| file, uses \pkg{docstrip}
to strip off the comments and documentation, and outputs a |.sty| file. The
good news is that a |.ins| file is typically fairly short and doesn’t change
significantly from one package to another.

\paragraph{License} The |ins| files usually start with comments specifying the copyright and license
information:

\begin{minted}{latex}
%%
%% Copyright (C) year by your name %%
%% This file may be distributed and/or modified under the
%% conditions of the LaTeX Project Public License, either
%% version 1.2 of this license or (at your option) any later
%% version. The latest version of this license is in:
%%
%% http://www.latex-project.org/lppl.txt
%%
%% and version 1.2 or later is part of all distributions of
%% LaTeX version 1999/12/01 or later.
%%
\end{minted}

The LATEX Project Public License (LPPL) is the license under which most
packages—and LATEX itself—are distributed. Of course, you can release your
package under any license you want; the LPPL is merely the most common
license for LATEX packages. The LPPL specifies that a user can do whatever
he wants with your package—including sell it and give you nothing in return.
The only restrictions are that he must give you credit for your work, and
he must change the name of the package if he modifies anything to avoid
versioning confusion.
The next step is to load DocStrip:

\begin{teXXX}
%%\input docstrip.tex
%%\keepsilent
\end{teXXX}



By default, DocStrip gives a line-by-line account of its activity. These messages
aren’t terribly useful, so most people turn them off, by using the command \docAuxCommand{keepsilent}:

\begin{teXXX}
\keepsilent
\end{teXXX}


A system administrator can specify the base directory under which all
TEX-related files should be installed, e.g., \texttt{/usr/share/texmf}. (See
\cmd{\BaseDirectory} in the DocStrip manual.) The |ins| file specifies where
its files should be installed relative to that. The following is typical:

\begin{teXXX}
\usedir{tex/latex/packagename}
\preamble
htexti \endpreamble
\end{teXXX}



The next step is to specify a preamble, which is a block of commentary that
will be written to the top of every generated file:

\begin{minted}{latex}
\preamble
----------------------------------------------------------------
phddoc --- A class to typeset LaTeX code.
E-mail: yannislaz@gmail.com
Released under the LaTeX Project Public License v1.3c or later
See http://www.latex-project.org/lppl.txt
----------------------------------------------------------------
\endpreamble
\end{minted}


The preceding preamble would cause |package.sty|  to begin as follows:

\begin{minted}{latex}
%%
%% This is file `phddoc.cls',
%% generated with the docstrip utility.
%%
%% The original source files were:
%%
%% phddoc.dtx  (with options: `class')
%% ----------------------------------------------------------------
%% phddoc --- A class to typeset LaTeX code.
%% E-mail: yannislaz@gmail.com
%% Released under the LaTeX Project Public License v1.3c or later
%% See http://www.latex-project.org/lppl.txt
%% ----------------------------------------------------------------
\end{minted}

We now reach the most important part of a .ins file: the specification of
what files to generate from the |.dtx| file. The following tells DocStrip to
generate hpackagei.sty from hpackagei.dtx by extracting only those parts
marked as `package'  in the .dtx file. (Marking parts of a .dtx file is
described later on.)

\begin{teXXX}
\generate{\file{<package>.sty}{\from{<package>.dtx}{package}}}
\end{teXXX}

\cmd{\generate} can extract any number of files from a given .dtx file. It can
even extract a single file from multiple |.dtx| files. See the DocStrip manual
for details.

Personally I also generate README.md files in |markdown| format as well, so that
when they get uploaded to |github| they can be rendered nicely.

\begin{minted}{latex}
\generate{\file{\jobname.md}{\from{\jobname.dtx}{readmemd}}}
\end{minted}

The text has to be wriiten using `guards' with the tag |readmd|

\begin{minted}{latex}
%<*readmemd>
# The `phddoc` LaTeX2e class

The `phd` latex package and the class with the same name provide
convenient methods to create new styles for books, reports
and articles. It also loads the most commonly used packages 
and resolves conflicts.
%</readmemd>
\end{minted}

\subsection{Generating messages} 

The next part of a |.ins| file consists of commands to output a message to
the user, telling him what files need to be installed and reminding him how
to produce the user documentation. The following set of \cmd{Msg} commands is
typical:

\begin{minted}[
frame=lines,
framesep=2mm,
baselinestretch=1.2,
fontsize=\footnotesize,
linenos
]{latex}
\obeyspaces
\Msg{****************************************************}
\Msg{* *}
\Msg{* To finish the installation you have to move the *}
\Msg{* following file into a directory searched by TeX: *}
\Msg{* *}
\Msg{* packagei.sty *}
\Msg{* *}
\Msg{* To produce the documentation run the file *}
\Msg{* package.dtx through LaTeX. *}
\Msg{* *}
\Msg{* Happy TeXing! *}
\Msg{* *}
\Msg{****************************************************}
Note the use of \obeyspaces to inhibit \tex from collapsing multiple spaces
into one.
\endbatchfile
\end{minted}


Appendix A.1 lists a complete, skeleton .ins file. Appendix A.2 is similar
but contains slight modifications intended to produce a class (|.cls|) file
instead of a style (|.sty|) file

\section{What to put in a  .dtx file}
We started describing the |.ins| install file first. The next file we will describe is
the |.dtx| file. This holds both the code definitions as well as the user documentation.

A |dtx|\ file contains both the commented source code and the user documentation
for the package. Running a |dtx|  file through |latex| typesets the
user documentation, which usually also includes a nicely typeset version of
the commented source code.

Due to some Doc trickery, a |dtx|  file is actually evaluated twice. The first
time, only a small piece of \latex\  driver code is evaluated. The second time,
comments in the |dtx|  file are evaluated, as if there were no `\%'  preceding
them. This can lead to a good deal of confusion when writing |dtx|  files
and occasionally leads to some awkward constructions. Fortunately, once
the basic structure of a |dtx|  file is in place, filling in the code is fairly
straightforward.

\paragraph{Guards} If you open any .dtx file you will notice that the lines either start with a \%
sign or sometimes with a percentage sign and |<|\textit{guard}|>|. The latter is called a guard and they are in a way
like html tags. They have a starting and an ending tag. In the example below there are two different guards
|<*10pt>...</10pt>| and |<*11pt></11pt>|. Unlike html tags guards are boolean expressions! You can use:
\begin{quote}
\textbar  ! \&  
\end{quote}

The \textbar stands for disjunction (OR), the \& stands for conjunction (AND) and the ! (NOT) stands for
negation. The terminal is any sequence of letters and evaluates to true iff it
occurs in the list of options that have to be included.

\begin{minted}{latex}
%<*10pt|11pt|12pt>
... code
%</10pt|11pt|12pt>
\end{minted}

A longer example from KOMA shows the concept better.

\fvset{gobble=0}
\begin{minted}[
frame=lines,
framesep=2mm,
baselinestretch=1.2,
fontsize=\footnotesize,
linenos
]{latex}
%    \begin{macrocode}
\def\normalsize{%
%<*10pt>
  \@setfontsize\normalsize\@xpt\@xiipt
  \abovedisplayskip 10\p@ \@plus2\p@ \@minus5\p@
  \abovedisplayshortskip \z@ \@plus3\p@
  \belowdisplayshortskip 6\p@ \@plus3\p@ \@minus3\p@
%</10pt>
%<*11pt>
  \@setfontsize\normalsize\@xipt{13.6}%
  \abovedisplayskip 11\p@ \@plus3\p@ \@minus6\p@
  \abovedisplayshortskip \z@ \@plus3\p@
  \belowdisplayshortskip 6.5\p@ \@plus3.5\p@ \@minus3\p@
%</11pt>
... 
%    end{macrocode}
\end{minted}

If the guards only contain a one line of text, then a short form is provided as |<10pt>|. It is unecessary to provide a closing tag and the `*' is omitted. The example below from the KOMA classes shows a quite ingenious way of writing the |\ProvidesFile| macro in
the different files; one for each tag. 
Two kinds of optional code are supported: one can either have optional code
that is on one line of tex code.

To distinguish both kinds of optional code the `guard modier' has been introduced. 
The `guard modifier' is one character that immediately follows the < of
the guard. It can be either * for the beginning of a block of code, or / for the end
of a block of code. The beginning and ending guards for a block of code have to
be on a line by themselves.

When a block of code is not included, any guards that occur within that block
are not evaluated.


\begin{minted}{latex}
%    \begin{macrocode}
\ProvidesFile{%
%<10pt>  scrsize10pt.clo%
%<11pt>  scrsize11pt.clo%
%<12pt>  scrsize12pt.clo%
}[\KOMAScriptVersion\space font size class option %
%<10pt>  (10pt)%
%<11pt>  (11pt)%
%<12pt>  (12pt)%
]
%    \end{macrocode}
\end{minted}

In the |.ins| file one could write to generate the various |.clo| files.:

\begin{minted}{latex}
\generate{\usepreamble\defaultpreamble
  \file{scrsize10pt.clo}{%
    \from{scrkernel-version.dtx}{clo,10pt}%
    \from{scrkernel-fonts.dtx}{clo,10pt}%
    \from{scrkernel-paragraphs.dtx}{clo,10pt}%
  }%
  \file{scrsize11pt.clo}{%
    \from{scrkernel-version.dtx}{clo,11pt}%
    \from{scrkernel-fonts.dtx}{clo,11pt}%
    \from{scrkernel-paragraphs.dtx}{clo,11pt}%
  }%
  \file{scrsize12pt.clo}{%
    \from{scrkernel-version.dtx}{clo,12pt}%
    \from{scrkernel-fonts.dtx}{clo,12pt}%
    \from{scrkernel-paragraphs.dtx}{clo,12pt}%
  }%
}%
\end{minted}

Becareful not to introduce spurious empy lines in your generated files by having empty lines in no-man's land, that is between tags.\footnote{In the phd package, I automatically generate the default settings from the |.dtx| files. In this case pgf will complain.}

\begin{minted}{latex}
%</install>

%<install>\endbatchfile
\end{minted}

\paragraph{The character table check } The second mechanism that Doc uses to ensure that a |dtx|  file is uncorrupted
is a character table. If you put the following command verbatim into
your |dtx|  file, then \pkg{Doc} will ensure that no unexpected character translation
took place in transport:

\begin{minted}[
frame=lines,
framesep=2mm,
baselinestretch=1.2,
fontsize=\footnotesize,
linenos,gobble=0,
]{latex}
% \CharacterTable
% {Upper-case \A\B\C\D\E\F\G\H\I\J\K\L\M\N\O\P\Q\R\S\T\U\V\W\X\Y\Z
% Lower-case \a\b\c\d\e\f\g\h\i\j\k\l\m\n\o\p\q\r\s\t\u\v\w\x\y\z
% Digits \0\1\2\3\4\5\6\7\8\9
% Exclamation \! Double quote \" Hash (number) \#
% Dollar \$ Percent \% Ampersand \&
% Acute accent \’ Left paren \( Right paren \)
% Asterisk \* Plus \+ Comma \,
% Minus \- Point \. Solidus \/
% Colon \: Semicolon \; Less than \<
% Equals \= Greater than \> Question mark \?
% Commercial at \@ Left bracket \[ Backslash \\
% Right bracket \] Circumflex \^ Underscore \_
% Grave accent \‘ Left brace \{ Vertical bar \|
% Right brace \} Tilde \~}
A success message looks like this:
***************************
* Character table correct *
***************************

and an error message looks like this:
! Package doc Error: Character table corrupted.
\end{minted}

\paragraph{DoNotIndex} When producing an index, \pkg{doc} normally indexes every control sequence
(i.e., backslashed word or symbol) in the code. The problem with this level
of automation is that many control sequences are uninteresting from the
perspective of understanding the code. For example, a reader probably
doesn’t want to see every location where \cs{if} is used—or \cs{the} or \cs{let} or
\cs{begin} or any of numerous other control sequences.

As its name implies, the \cs{DoNotIndex} command gives |Doc| a list of control
sequences that should not be indexed. \cs{DoNotIndex} can be used any
number of times, and it accepts any number of control sequence names per
invocation:

\begin{minted}[
frame=lines,
framesep=2mm,
baselinestretch=1.2,
bgcolor=white,
fontsize=\footnotesize,
linenos
]{latex}
\DoNotIndex{\#,\$,\%,\&,\@,\\,\{,\},\^,\_,\~,\ }
\DoNotIndex{\@ne}
\DoNotIndex{\advance,\begingroup,\catcode,\closein}
\DoNotIndex{\closeout,\day,\def,\edef,\else,\empty, \endgroup}
\end{minted}


\subsection{User documentation}

We can finally start writing the user documentation. A typical beginning
looks like this:

\begin{minted}[
frame=lines,
framesep=2mm,
baselinestretch=1.2,
bgcolor=white,
fontsize=\footnotesize,
linenos
]{latex}
% \title{The \textsf{package} package\thanks{This document
% corresponds to \textsf{package}~\fileversion,
% dated~\filedate.}}
% \author{your name \\ \texttt{your e-mail address}}
%
% \maketitle
\end{minted}


The title can certainly be more creative, but note that it’s common for
package names to be typeset with \docAuxCommand{textsf} and for \docAuxCommand{thanks} to be used to
specify the package version and date. This yields one of the advantages
of literate programming: Whenever you change the package version (the
optional second argument to \docAuxCommand{ProvidesPackage}), the user documentation
is updated accordingly. Of course, you still have to ensure manually that
the user documentation accurately describes the updated package.

Write the user documentation as you would any \latexe document, except
that you have to precede each line with a |\%|. Note that the |ltxdoc| document
class is derived from article, so the top-level sectioning command is
|\section|, not |\chapter|.



\section{General tips for defining a Class}

Evaluate, if there is a class that is nearer to what you wish to achive. If not do a set of
requirements.

Book structure - start with book or |Octavo| if you need to hack extensively. If not use memoir, |koma| or |tufte-book|.

Paragraph looks

Lists

Figures

Bibliography and citations

Footnotes

Index

Title pages

Book Cover

Language support

Mathematics

Graphs and figures

Typography - fonts, indentations fontsize etc

headers and footers


\section{Declaring Options}

Most classes or packages will have a good deal of options. These are declared using the
\docAuxCommand{DeclareOption} command. In this part no package loading should take place.

\begin{docCommand} {DeclareOption} { \marg{option} \marg{code}}
  The argument option is the name of the option being declared and the \marg{code} is the
  code that will execute if this option is requested.
\end{docCommand}


\begin{docCommand}{DeclareOption*} { \marg{code}}
  The argument \meta{code} in the star version of the command specifies the action to be 
  taken if an unknown option is specified. Within this argument the \docAuxCommand{CurrentOption}
  refers to the name of the option in question. 
  
\end{docCommand}

For example one could pass all such options
  to another package, using:
  \begin{verbatim}
  \DeclareOption*{\PassOptionsToPackage{\CurrentOption}{A}}
  \end{verbatim}


\section{Executing Options}

Normally after the options have been defined, one would need to provide default values and 
the options need to be executed. 

\begin{docCmd} {ExecuteOptions} { \marg{option list}}
  
\end{docCmd}

You can also |\ExecuteOptions| when declaring other options. There is one caveat. This command
can only be executed prior to executing the |\ProcessOptions| command because, as one of
its last actions, the latter command reclaims all of the memory taken up by the code for
the declared options.

\begin{docCmd} {ProcessOptions*} {}
\end{docCmd}

For some packages it is preferable or essential to process options in the order they
appear in the |usepackage| commands rather than using the order given through the
sequence of the \refCmd{DeclareOption} commands. In this case it one has to use
the star version of the command, i.e, |\ProcessOptions*| rather than |\ProcessOptions|.

\section{Special Commands for class files}

It is sometimes preferable to define a new class based on another and hence to extend it.
To support this concept the \latexe kernel provides two commands, \docAuxCommand{LoadClass} and
\docAuxCommand{PassOptionsToClass}. These two commands can then be used to develop a new class, by adding and extending the functionality of the loaded class.

\begin{docCommands}
\refCom{LoadClass}{ \oarg{option list}\marg{class}\oarg{release}}
\end{docCommands}  
  
For example the |ltxdoc| class loads the standard |article| class. The \pkg{tufte-book} class loads
the |book| class. The best way to understand the concepts discussed here is to
study these classes.

\section{A minimal class}

\begin{texexample}{Model Class}{ex:modelclass}

\begin{filecontents}{phdexampleclass.cls}
\NeedsTeXFormat{LaTeX2e}
\ProvidesClass{phdexampleclass}[2015/07/07]
\renewcommand\normalsize{\fontsize{}{10pt}{12pt}\selectfont}
\setlength\textwidth{6.5in}
\setlength\textheight{5in}
\pagenumbering{arabic}
\end{filecontents}

\end{texexample}

\vfill
\endinput





















%  \chapter{Key Value Interfaces}

The key value system greatly simplifies the \tex interface for authors. As \cite{joseph2009} wrote this ease of use was not transferred into settting up key-value systems for authors of pre-packaged \tex code. This Chapter and the one that follows that focus specifically on the \pkg{pgfkeys} package provides an overview and describes some of the more difficult areas. The TUGboat article referenced earlier and written by Joseph Wright \textit{et.al} has an excellent introduction to the available packages and some longer examples for comparison. Chapter~\ref{ch:l3keys}
\nameref{ch:l3keys} discusses the |expl3| key-value functions.

\section{keyval}

The \pkgname{keyval} written by David Carlisle is still widely used by package authors to provide the means for users to easily specify numerous optional arguments for macros \cite{keyval}. The main advantages of using keyval are that  (1) the number of optional arguments is no longer limited to 9 and that (2) the arguments are named, and hence there is less chance of confusion about the syntax of a macro.

\section{xkeyval}

A more recent package, \pkgname{xkeyval} provides improvements for programming keys and  also
provides a more advanced interface for the namespacing of keys and families.
Before you start experimenting with the xkeyval package, I suggest that you load the package \pkg{xkview}. This is part of the \ctan{xkeyval}  bundle and can help you to view key value parameters in various ways. The \pkg{xkeyval} package was developed by Hendri Adriaens and Uwe Kern \citep{xkeyval}. This package is an extension of the well-known |keyval| package. The package provides more flexible commands and syntax enhancements as well as newer option processing mechanism.

The main change of the |xkeyval| package is that it provides a means to namespace the keys, which all have the form |\KV@family@keyname|, where the KV is a literal prefix to avoid collisions. They take one argument to handle user input.

The main commands of the package are the same as those of keyval. 

\begin{texexample}{xkeyval }{}
\makeatletter
\define@key{phd}{pi}{\setlength{\parindent}{#1}}
\setkeys{phd}{pi=50pt}
\makeatother
\lorem\par
\setkeys{phd}{pi=0pt}
\lorem\par
\end{texexample}

Defining a default key, i.e., a key that can be used as |indent| or |indent=30pt| will stretch your memory, as it has an optional parameter as its third argument. 

\begin{verbatim}
\define@key{family}{key}[none]{The input is: #1}
\end{verbatim}

\begin{texexample}{xkeyval }{}
\makeatletter

\define@key{phd}{pi}[30pt]{\setlength{\parindent}{#1}}

\setkeys{phd}{pi}

\lorem

or \setkeys{phd}{pi=0pt}

\lorem
\makeatother
\end{texexample}

\section{Ordinary Keys}
\makeatletter
\define@key{phd}{pi}[1em]{\setlength{\parindent}{#1}}
\makeatother


Ordinary keys are keys that have values such as \texttt{animal=elephant} and your macro can be called like \texttt{animals[animal=elephant]\{14\}}.

   

\section{Keys and values in package options}

First of all, the package supplies macros to declare class or package options, execute them and process
them. The macros are available under the usual
\latex names, but all with the suffix \textbf{X}, namely

\begin{docCommand}{DeclareOptionX}{}
\begin{docCommand}{DeclareOptionX*}{}
\begin{docCommand}{ExecuteOptionsX}{}
\begin{docCommand}{ProcessOptionsX}{}
These commands allow the user to assign a value to
an option just like when using |\setkeys|. The first
macro is based on |\define@key| and the final two
are based on |\setkeys|. Supposing that a package
|mypack| is set up with these commands, a user could
for instance do
\end{docCommand}
\end{docCommand}
\end{docCommand}
\end{docCommand}

\begin{verbatim}
\usepackage[textcolor=red,font=times]{mypack}
\end{verbatim}

These |xkeyval|macros are fully compatible with the \latex option conventions. They will allow packages to copy global options specified in the |\documentclass| command, to pass options to other classes or packages and to update the list of unused global options that will be displayed by \latex in the log file. 


\section{kvoptions}

Another package \pkgname{kvoptions} by Heiko Oberdiek is used extensively in the large suite of packages
developed by Heiko \cite{kvoptions}. The package originally formed part of the \pkgname{hyperref} and later branched into
an independent package. The package provides a number of additional commands to those found in the \latexe kernel and a comparison of the commands is shown in the table below. It is a good alternative to single purpose
package writers. An important feature of the package is its ability to process options both globally as well as locally avoiding conflicts when options are specified both globally as well as locally. Heiko provides an example from his bookmark package \cite{bookmark}, which provides the option \option{open}
that specifies whether the bookmarks are opened or closed initially. It’s values are
true or false. Since KOMA-Script version 3.00 the KOMA classes also introduces
option open with values right and any and a complete different meaning.
Such conflicts can be resolved by marking all or part of options as local by
|\DeclareLocalOption| or |\DeclareLocalOptions|. Then the packages ignores
global occurences of these options



\input{./sections/pgfmanual-en-pgfkeys}












%  \input{./sections/pgfmanual-en-pgfkeys}
%  \chapter{Colors}

\newthought{The figure below, shows the wavelengths} in nm of the visible light. It has been drawn using the \docpkg{xcolor} package and the native \latex environment \cmd{picture}. The colors can be typest using the wavelength of light.

\smallskip

\begin{texexample}{}{}
  \hbox{\color{thered} A TesT}
\end{texexample}




\newcount\WL \unitlength.75pt

\begin{figure}
\hskip-3pt\scalebox{0.9}{
\noindent

\begin{picture}(460,60)(355,-10)
\sffamily \tiny \linethickness{1.25\unitlength} \WL=360
\multiput(360,0)(1,0){456}%
{{\color[wave]{\the\WL}\line(0,1){50}}\global\advance\WL1}
\linethickness{0.25\unitlength}\WL=360
\multiput(360,0)(20,0){23}%
{\picture(0,0)
\line(0,-1){5} \multiput(5,0)(5,0){3}{\line(0,-1){2.5}}
\put(0,-10){\makebox(0,0){\the\WL}}\global\advance\WL20
\endpicture}
\end{picture}}
\caption{The visible spectrum nm}
\end{figure}

The |xcolor| package provides numerous macros for typesetting colors, using a variety of methods and color schemes. For example we can use the command \cs{color} to print a text sample in color.
\newlength\pull

\def\colorSample#1{%
\leavevmode
\parindent0pt
   \def\colorRule{\color[wave]{#1}\rule{\textwidth}{0.4pt}} 
   \colorRule
%% set to the width of the box
   \settowidth\pull{\framebox{\Large #1 nm}}
%% pull by one em
   \addtolength\pull{1em}
   \hskip -\pull{\color[wave]{#1}{{\framebox{\Large #1 nm}}}}%
   %% add story
   \hskip1em\noindent\onepar\par
   \colorRule
}
\bgroup
\colorSample{385}
\colorSample{809}
\egroup


\section{Specifying colors by name}

The easier way to specify colors is to use the pre-build names available
with the package drivers.











%  
\makeatletter\@specialtrue\makeatother
\cxset{steward,
  numbering=arabic,
  custom=stewart,
  offsety=0cm,
  image={elevendays.jpg},
  texti={An introduction to the use of font related commands. Thee chapter also gives a historical background to font selection using \tex and \latex. },
  textii={In this chapter we discuss keys that are available through the \texttt{phd} package. The image is William Hogarth's painting (c. 1755) which is the main source for `Give us our Eleven Days'.
 },
}
\chapter{Handling Dates and Time}
\label{dates}\label{ch:dates}

\parindent1.5em

\section{Problems with time and date}

\tex and \latex do not offer\cite{Thanh:TB18-4-249} any sophisticated support for date and time routines.
One can get the current system date using \cmd{\today}
Typing |\today| we get \texttt{\today}. Normally the format of |\today| would vary from class to class, as this is one of the first things class authors style. The |\today| command is build using three other commands.\footnote{It appears that there is also a time=now in IniTeX} Another issue with such commands is the fact that they are dependent on the language used and the prevalent conventions.

\begin{texexample}{Basic date example}{}
\the\day

\the\month

\the\year

\meaning\today
\end{texexample}

{
\makeatletter
|\the\month| \the\month

|\the\day| \the\day

|\the\time| \two@digits{\the\count@}:\two@digits{\the\count2}
\makeatother}

\tex offers only one macro \cmd{\time} which is the time in hours since midnight.


The code below is from the \latex kernel and can be found in the \docfile{ltdirchk.dtx}

\startlineat{126}
\begin{teX}
\count@\time
\divide\count@ 60
\count2=-\count@
\multiply\count2 60
\advance\count2 \time

\edef\today{%
  \the\year/\two@digits{\the\month}/\two@digits{\the\day}:%
  \two@digits{\the\count@}:\two@digits{\the\count2}
 }
\end{teX}

\begin{texexample}{Time in LaTeX}{}
\makeatletter
\count@\time
\divide\count@ 60
\count2=-\count@
\multiply\count2 60
\advance\count2 \time

\edef\today{%
\the\year/\two@digits{\the\month}/\two@digits{\the\day}:%
\two@digits{\the\count@}:\two@digits{\the\count2}}


\today:   \the\count2:  \the\count@

the time \the\time
\makeatother
\end{texexample}




\section{Getting the time}

\tex has a primitive register that contains “the number of minutes since midnight”; with this knowledge it’s a moderately simple programming job to print the time (one that no self-respecting Plain \tex user would bother with anyone else’s code for).

However, \latex provides no primitive for “time”, so the non-programming LaTeX user needs help.


\section*{Getting the time using pdf internal commands}

One of the problems with \tex's |\time| is that it is not possible to count seconds. One way to by-pass this is to use the pdfLaTeX or pdfTeX macro
\cmd{pdfcreationdate}.


\texttt{> \textbackslash pdfcreationdate}

\texttt{\pdfcreationdate}

As you can observe from the above, the pdf has a special format and it even includes infromation about the timezone.

PDF defines a standard date format, which closely follows that of the international standard ASN.1 (Abstract Syntax Notation One), defined in ISO/IEC 8824 (see the Bibliography). A date is a string of the form

|(D:YYYYMMDDHHmmSSOHH'mm')|

where

\begin{teX}
YYYY is the year
MM is the month
DD is the day (01-31)
HH is the hour (00-23)
mm is the minute (00-59)
SS is the second (00-59)
\end{teX}


O is the relationship of local time to Universal Time (UT), denoted by one of the characters +, -, or Z (see below)
HH followed by ' is the absolute value of the offset from UT in hours (00–23)
mm followed by ' is the absolute value of the offset from UT in minutes (00–59)

The quotation mark character (') after HH and mm is part of the syntax. All fields after the year are optional. (The prefix D:, although also optional, is strongly recommended.) The default values for MM and DD are both 01; all other numerical fields default to zero values. A plus sign (+) as the value of the O field signifies that local time is later than UT, a minus sign (-) that local time is earlier than UT, and the letter Z that local time is equal to UT. If no UT information is specified, the relationship of the specified time to UT is considered to be unknown. Whether or not the time zone is known, the rest of the date should be specified in local time.

For example, December 23, 1998, at 7:52 PM, U.S. Pacific Standard Time, is represented by the string,


|D:199812231952-08'00'|


Two packages are available, both providing ranges of ways of printing the date, as well as of the time: this question will concentrate on the time-printing capabilities, and interested users can investigate the documentation for details about dates.


\section*{Using \protect\texttt{datetime}}

The \pkg{datetime} package defines two time-printing functions: \cmd{\xxivtime} (for 24-hour time), \cmd{\ampmtime} (for 12-hour time) and \cmd{\oclock} (for time-as-words, albeit a slightly eccentric set of words).

\emphasis{xxivtime,ampmtime,oclock}

\begin{texexample}{Using DateTime}{ex:datetime}
The time is \xxivtime
The time is \ampmtime
The time is \oclock

The time is \xxivtime

The time is \ampmtime

The time is \oclock
\end{texexample}


\section{Using scrtime}

The \pkg{scrtime} package (part of the compendious KOMA-Script bundle) takes a package option (12h or 24h) to specify how times are to be printed. The command \cmd{\thistime} then prints the time appropriately (though there's no am or pm in 12h mode). The \cmd{\thistime} command also takes an optional argument, the character to separate the hours and minutes.


\begin{texexample}{Example scrtime}{ex:scrtime}
The time is \thistime
The time is \thistime[h]
\end{texexample}

\label{datesend}


The time is \thistime[ hours ] minutes 

{> The time is \thistime*[:] } 

|\thistime*| works in almost the same way as |\thistime|. The only
diffrence is that unlike with |\thistime|, with |\thistime*| the value of
the minute field is not preceded by a zero when its value is less than 10.
Thus, with |\thistime| the minute field has always two places.



\begin{comment}
%% Hack to get the time zone
%% This is based on a macro at http://tex.stackexchange.com/questions/8612/write-date-time-and-time-zone by Will Robertson



\pdfcreationdate
\newcounter{temp}
\setcounter{temp}{1}
\let\Box=\boxed

\def\Box#1{\fbox{\strut\textbf{#1}$\scriptscriptstyle\,_{\thetemp}$\stepcounter{temp} }}

\Box{D}\Box{:} \Box{\color{red}2}\Box{\color{red}0}\Box{\color{red}1}%
\Box{\color{red}1}
 \Box{0}\Box{1}
\Box{\color{purple}1} \Box{\color{purple}1}\Box{0}

\newtoks\tyear
\newtoks\tmonth
\newtoks\tday
\newtoks\thour
\newtoks\tminutes
\newtoks\tseconds
\newtoks\UTCh

\def\grabtimezone #1#2#3#4#5#6#7#8{
\tyear={#3#4#5#6}%
\tmonth{#7#8}%
\grabtimezoneB}

\def\grabtimezoneB #1#2#3#4#5#6#7#8{
  \tday={#1#2}%
  \thour={#3#4}%
  \tminutes={#5#6}%
  \tseconds={#7#8}%
\grabtimezoneC}

%\def\grabtimezoneC #1#2#3'#4'{\UTCh={sign:#1  hr: #2#3 min: #4}}
%\expandafter \grabtimezone\pdfcreationdate
%
%
%%\@namedef{timezone+0930}{CST}
%%\@namedef{timezone+1000}{EST}
%%\@namedef{timezone+1030}{CST'}
%
%\the\tyear 
%
%\the\tmonth
%
%\the\tday
%
%\the\thour
%
%\the\tminutes
%
%\the\tseconds
%
%\the\UTCh
\end{comment}

\section*{Day of the Week}
The day of the week can be calculated using the |dow| macro that 
has been around for a while

\begin{comment}
\def\DayOfWeekLong{%
%
% 	Calculate day of the week, return "Sunday", etc.
%
  \newcount\dow				% Gets day of the week
  \newcount\leap			% Leap year fingaler
  \newcount\x				% Temp register
  \newcount\y 				% Another temp register
%		leap = year + (month - 14)/12;
  \leap=\month \advance\leap by -14 \divide\leap by 12
  \advance\leap by \year
%		dow = (13 * (month + 10 - (month + 10)/13*12) - 1)/5
  \dow=\month \advance\dow by 10
  \y=\dow \divide\y by 13 \multiply\y by 12
  \advance\dow by -\y \multiply\dow by 13 \advance\dow by -1 \divide\dow by 5
%		dow += day + 77 + 5 * (leap % 100)/4
  \advance\dow by \day \advance\dow by 77
  \x=\leap \y=\x \divide\y by 100 \multiply\y by 100 \advance\x by -\y
  \multiply\x by 5 \divide\x by 4 \advance\dow by \x
%		dow += leap / 400
  \x=\leap \divide\x by 400 \advance\dow by \x
%		dow -= leap / 100 * 2;
%		dow = (dow % 7)
  \x=\leap \divide\x by 100 \multiply\x by 2 \advance\dow by -\x
  \x=\dow \divide\x by 7 \multiply\x by 7 \advance\dow by -\x
  \ifcase\dow Sunday\or Monday\or Tuesday\or Wednesday\or
	Thursday\or Friday\or Saturday\fi
}

\def\DayOfWeekShort{%
%
% 	Calculate day of the week, return "Sunday", etc.
%
  \newcount\dow				% Gets day of the week
  \newcount\leap			% Leap year fingaler
  \newcount\x				% Temp register
  \newcount\y 				% Another temp register
%		leap = year + (month - 14)/12;
  \leap=\month \advance\leap by -14 \divide\leap by 12
  \advance\leap by \year
%		dow = (13 * (month + 10 - (month + 10)/13*12) - 1)/5
  \dow=\month \advance\dow by 10
  \y=\dow \divide\y by 13 \multiply\y by 12
  \advance\dow by -\y \multiply\dow by 13 \advance\dow by -1 \divide\dow by 5
%		dow += day + 77 + 5 * (leap % 100)/4
  \advance\dow by \day \advance\dow by 77
  \x=\leap \y=\x \divide\y by 100 \multiply\y by 100 \advance\x by -\y
  \multiply\x by 5 \divide\x by 4 \advance\dow by \x
%		dow += leap / 400
  \x=\leap \divide\x by 400 \advance\dow by \x
%		dow -= leap / 100 * 2;
%		dow = (dow % 7)
  \x=\leap \divide\x by 100 \multiply\x by 2 \advance\dow by -\x
  \x=\dow \divide\x by 7 \multiply\x by 7 \advance\dow by -\x
  \ifcase\dow Sun\or Mon\or Tue\or Wed\or
	Thur\or Fri\or Sat\fi
}


\DayOfWeekLong

\DayOfWeekShort
\end{comment}

\makeatother








The \pkg{datenumber} has been developed by J\"org-Michael Schr\"oder and provides commands to convert a date into a number. Turned around a date can be calculated also by a number. Additionally there are commands for incrementing and decrementing a date. Leap years and the Gregorian calendar reform are considered.
\index{dates}\index{dates!leap year}\index{dates! Gregorian calendar}

\section{Start year}

The start of the counting is determined with \verb+\setstartyear{year}+ (standard 1800). The first day of the start year gets the number 1. The value of \texttt{startyear} must be greater 0. It may not be larger than the year of a date to be calculated. If the difference of date and \texttt{startyear} is large, the calculation can last for a long time. The correct setting of the weekdays is guaranteed only if the value of \texttt{startyear} is 1800, 1900 or 2000.


\section{Counters}
There are five counters defined \doccmd{datenumber}, \doccmd{dateyear}, \doccmd{datemonth}

\begin{description}
\item[\texttt{datenumber}:] number of the day
\item[\texttt{dateyear}:] year
\item[\texttt{datemonth}:] month
\item[\texttt{dateday}:] day
\item[\texttt{datedayname}:] weekday: 1--7 (Monday--Sunday)
\end{description}


\section{Macros}
\subsection{Macros which operate with defined counters\label{macro}}
All counters specified above are updated by these macros. \verb+\datedayname+ and \verb+\datemonthname+ are also updated.

\begin{description}
\item[\texttt{\textbackslash setdatenumber\{year\}\{month\}\{day\}}:] Sets the counter \texttt{datenumber} to a value, which corresponds to the date.
\item[\texttt{\textbackslash setdatebynumber\{number\}}:] Sets the counters \texttt{dateyear}, \texttt{datemonth}, and \texttt{dateday} to values, which corresponds to the number.
\item[\texttt{\textbackslash nextdate}:] Sets the counters \texttt{dateyear}, \texttt{datemonth}, and \texttt{dateday} to the next date.
\item[\texttt{\textbackslash prevdate}:] Sets the counters \texttt{dateyear}, \texttt{datemonth}, and \texttt{dateday} to the previous date.
\item[\texttt{\textbackslash setdate\{year\}\{month\}\{day\}}:] Sets the counters \texttt{dateyear}, \texttt{datemonth}, and \texttt{dateday} to \texttt{year}, \texttt{month}, and \texttt{day}.
\item[\texttt{\textbackslash setdatetoday}:] Sets the counters \texttt{dateyear}, \texttt{datemonth}, and \texttt{dateday} to the current date.
\item[\texttt{\textbackslash datemonthname}:] typesets the month (see section \ref{monthname}).
\item[\texttt{\textbackslash datedayname}:] typesets the weekday (see section \ref{dayname}).
\item[\texttt{\textbackslash datedate}:] typesets the date, corresponding to the counters \texttt{dateyear}, \texttt{datemonth}, \texttt{dateday}.
\end{description}


\subsection{Macros which operate with your own counters}
Only the counters you specified are updated by these macros. \verb+\datedayname+ and \verb+\datemonthname+ are not updated.
\begin{description}\sloppypar
\item[\texttt{\textbackslash setmydatenumber\{numbercount\}\{year\}\{month\}\{day\}}:] Sets the counter \texttt{numbercount} to a value, which corresponds to the date.
\item[\texttt{\textbackslash setmydatebynumber\{number\}\{yearcount\}\{monthcount\}\{daycount\}}:] Sets the counters \texttt{yearcount}, \texttt{monthcount}, and \texttt{daycount} to values, which corresponds to the number.
\item[\texttt{\textbackslash mynextdate\{yearcount\}\{monthcount\}\{daycount\}}:] Sets the counters \texttt{yearcount}, \texttt{monthcount}, and \texttt{daycount} to the next date.
\item[\texttt{\textbackslash mynextdate\{yearcount\}\{monthcount\}\{daycount\}}:]Sets the counters \texttt{yearcount}, \texttt{monthcount}, and \texttt{daycount} to the previous date.
\end{description}



\subsection{Month\label{monthname}}
The command \verb+\datemonthname+ typesets the month. It is updated by macros described in section \ref{macro}. You can do this by your own saying \verb+\setmonthname{number}+.

\subsection{Weekday\label{dayname}}
To typeset the weekday say \verb+\datedayname+. This command is updated by macros described in section \ref{macro}.
You can do this by your own saying \verb+\setmonthname{number}+ (1 for Monday and 7 for Sunday). You can also write \verb+\setdaynamebynumber{number}+, were \verb+number+ is the number of a date. If \texttt{startyear} is set to 1800, 1900 or 2000 the calculation of the weekday will work.

\section{Language}

The language options \texttt{english}, \texttt{USenglish} (standard), \texttt{french}, \texttt{spanish}, \texttt{german}, and \texttt{ngerman} are supported. Say \verb+\dateselectlanguage{language}+ to select a language. For other languages: Create a file \texttt{datenumbermylanguage.ldf}. Copy the text from \texttt{datenumberdummy.ldf}. Replace every ``dummy'' with ``mylanguage'' and change the months and weekdays. Say \verb+\usepackage{datenumber}+ \verb+\input{datenumbermylanguage.ldf}+ in your document.

\section{Examples}

\begin{teX}
\setdate{2002}{1}{1}
\thedatenumber
\end{teX}

\setdate{2000}{1}{1}



\begin{verbatim}
\setdatetoday
\addtocounter{datenumber}{10}%
\setdatebynumber{\thedatenumber}%
In 10 days is \datedate
\end{verbatim}

\setdatetoday
\addtocounter{datenumber}{10}%
\setdatebynumber{\thedatenumber}%

Result: In 10 days is \datedate


We can now find the days to Christmas

\begin{teX}
\newcounter{dateone}\newcounter{datetwo}%

\newcommand{\daydifftoday}[3]{%
  \setmydatenumber{dateone}{\the\year}{\the\month}{\the\day}%
  \setmydatenumber{datetwo}{#1}{#2}{#3}%
  \addtocounter{datetwo}{-\thedateone}%
  \thedatetwo
}
\end{teX}
\newcounter{dateone}%
\newcounter{datetwo}%
\newcommand{\daydifftoday}[3]{%
  \setmydatenumber{dateone}{\the\year}{\the\month}{\the\day}%
  \setmydatenumber{datetwo}{#1}{#2}{#3}%
  \addtocounter{datetwo}{-\thedateone}%
  \thedatetwo}

There is still \daydifftoday{\the\year}{12}{25} days to Christmas.


Result: There is still \daydifftoday{\the\year}{12}{25} days to Christmas.


\newcommand{\sd}{%
\ifcase\thedatedayname \or
    Mon\or Tue\or Wed\or Thu\or
    Fri\or Sat\or Sun\fi
}%

\newcommand{\pnext}{%
\thedateyear/%
\ifnum\value{datemonth}<10 0\fi
\thedatemonth/%
\ifnum\value{dateday}<10 0\fi
\thedateday%
\nextdate
}



\begin{verbatim}
\setdate{2001}{9}{29}%
\[\begin{tabular}{lll}
\sd & \pnext & Abc\\
\sd & \pnext & Def\\
\sd & \pnext & Ghi\\
\sd & \pnext & Jkl\\
\end{tabular}\]
\end{verbatim}


Result: \setdate{2001}{9}{29}%

\[\begin{tabular}{lll}
\sd & \pnext & Abc\\
\sd & \pnext & Def\\
\sd & \pnext & Ghi\\
\sd & \pnext & Jkl\\
\end{tabular}\]


\newthought{Get your age calculated}

\begin{teXXX}
\documentclass{article}
\usepackage{datenumber,fp}
\begin{document}
\newcounter{dateone}%
\newcounter{datetwo}%
\setmydatenumber{dateone}{1989}{08}{01}%
\setmydatenumber{datetwo}{\the\year}{\the\month}{\the\day}%
\FPsub\result{\thedatetwo}{\thedateone}
\FPdiv\myage{\result}{365.25} 
\FPround\myage{\myage}{0}\myage\ years old
\end{document}
\end{teXXX}


\subsection{Other}

Because of the Protestant Reformation, however, many Western European countries did not initially follow the Gregorian reform, and maintained their old-style systems. Eventually other countries followed the reform for the sake of consistency, but by the time the last adherents of the Julian calendar in Eastern Europe (Russia and Greece) changed to the Gregorian system in the 20th century, they had to drop 13 days from their calendars, due to the additional accumulated difference between the two calendars since 1582.

The leapyear \index{dates>leapyear} can be tested using
\cmd{\leapyear} and the date can be checked for validity using
\cmd{\ifvaliddate}. The examples below show such tests

\begin{itemize}
\item leap year test
\begin{quote}
\begin{verbatim}
The year 2012 is
\ifleapyear{2012} a \else no \fi leap year.
\end{verbatim}
Result: The year |2012| is \ifleapyear{2012} a \else no \fi leap year.
\end{quote}
\item date test
\begin{quote}
\begin{verbatim}
The 29.2.1900 is
\ifvaliddate{1900}{2}{29} a \else no \fi valid date.
\end{verbatim}


Result: The 29.2.1900 is \ifvaliddate{1900}{2}{29} a \else no \fi valid date.%
\end{quote}
\end{itemize}

\section*{Calculating the week number}
\begin{figure}%
  \centering
  \includegraphics[width=1.1\linewidth]{./graphics/babylonianmaps.jpg}
  \caption[Babylonian Imago Mundi]{\protect\footnotesize \protect\raggedright The Babylonian Imago Mundi, dated to the 6th century BC (Neo-Babylonian Empire). The map shows Babylon on the Euphrates, surrounded by a circular landmass showing Assyria, Armenia and several cities, in turn surrounded by a `bitter river' (Oceanus), with seven islands arranged around it so as to form a seven-pointed star.}
  \label{fig:eleven days}
\end{figure}

I an attempt to produce gantt charts (see Section \ref{ganttcharts}) that follow Tufte's ideas of simplicity, I came across the need to define a week number. The ISO week date system is a leap week calendar system that is part of the ISO 8601 date and time standard. The system is used (mainly) in government and business for fiscal years, as well as in timekeeping.

The system uses the same cycle of 7 weekdays as the Gregorian calendar. Weeks start with Monday. ISO week-numbering years have a year numbering which is approximately the same as the Gregorian years, but not exactly (see below). An ISO week-numbering year has 52 or 53 full weeks (364 or 371 days). The extra week is here called a leap week (ISO 8601 does not use the term).



A date is specified by the ISO week-numbering year in the format YYYY, a week number in the format ww prefixed by the letter W, and the weekday number, a digit d from 1 through 7, beginning with Monday and ending with Sunday. For example, |2006-W52-7| (or in compact form |2006W527|) is the Sunday of the 52nd week of 2006. In the Gregorian system this day is called 31 December 2006.

The system has a 400-year cycle of 146 097 days (20 871 weeks), with an average year length of exactly 365.2425 days, just like the Gregorian calendar. In every 400 years there are 71 years with 53 weeks.

\textsc{The first week of a year is the week that contains the first Thursday of the year.}

Based on this a calculation can be made using routines available from the above packages. However, how many weeks are included in a typical month it is still a problem.


\section{Summary}

This rather long chapter discussed the various options and packages available to deal with dates. The best way so far, for pdfLaTeX and pdfTeX users is to use the \cmd{pdfcreation} to access system time. Once the information made available by this command is parsed the rest of the routines can be developed. And now we have dates. Next we are going to try and develop some scheduling routines for gantt charts.

\section{phd package Internationalization of dates and time}

The phd package currently offers a range of date modules for the internationalization of dates and other strings. See the Chapter on internationalization.

























































%  \chapter{Coding Styles}

\epigraph{A survey conducted in 2010 shows that \latex is mostly a world of dwarfs.}{--Didier Verna, in \textit{Towards LaTeX coding standards} TUGboat, Volume 32 (2011), No.3}

As Didier Verna \citep{verna} wrote in a seminal article at TUGboat, most \latex macro authors are loners and there is little co-operative effort when producing packages. Once you start writing your own macros, that grow into packages and classes the matter of coding style will arise.

\section{Learning by example}

Like most programming, learning by example is the best approach. Many \latex programmers including myself started by reading code, and what other people did. In doing so as vera says: `they implicitly (and unconsiously) inherit the coding style or lack of it. This behavious actually encourages legacy (the good \textit{and} the bad and leads to a very heterogeneous code base.

\section{Some advice for readability}

\subsection{Allow spaces}

Generally programmers prefer to leave spaces before and after |=,+,-|. Consider writing:

\begin{verbatim}
$ f(x) = f(x-1) + f(x-2) $
\end{verbatim}

This is more readable than:

\begin{verbatim}
$f(x)=f(x-1)+f(x-2)$
\end{verbatim}

\section{Documentation}














%  \chapter{THE OUTPUT ROUTINE (OTR)}

\epigraph{Sherlock Holmes in "The sign of four": "'My mind," he said, "rebels at stagnation. Give me problems, give me work, give me the most abstruse cryptogram or the most intricate analysis, and I am in my own proper atmosphere.'" }{}
\normalsize
The output routine is one of the more mysterious pieces
of \tex.
and as  David Salomon noted\footnote{TUGboat/tb-11-1/tb27salomon.pdf}, advanced users hardly need to be convinced that an unerstanding of OTRs is important, since they must be used whenever, special output is desired.
 The chapter of the \texbook discussing output
routines claims that designing output routines makes one:

\begin{quotation}
achieve the level of a `\tex Grandmaster'.
As is so often the case, mastery of the concept of an
output routine in plain TEX will only barely prepare you
for the complexities awaiting you with LATEX’s variant of
an output routine.
\end{quotation}


The subject is considered complex for the following reasons:

\begin{enumerate}
\item OTRS are asynchronous with the
rest of TEX (this is explained later) and involve difficult concepts such as splitting boxes and insertions.
\item Certain features, which could be useful in OTRs are not supported by \tex. Specifically there are no commands to identify marks, rules and |whatsits| in a box and to break up a line of text into individual characters.
\end{enumerate}

\tex\ 's page breaking algorithm is simpler than the line breaking one. The reason for this is that global optimization
of page breakpoints, the way is done in the paragraph algorithm is prohibitively in terms of memory (especially in the 1980s).

Theoretically, page breaking is a more complicated \footnote{\href{test}{http://www.cs.utk.edu/~eijkhout/594-LaTeX/handouts/breaking/page-tutorial.pdf}}than line breaking. First we will briefly discuss the algoithms that \tex\ actually
uses.


\section{Page breaking algorithm}

The problem of page breaking has two components. One is that of stretching or shrinking
available glue (mostly around display math or section headings) to find typographically
desirable breakpoints. The other is that of placing ‘floating’ material, such as tables and
figures. These are typically placed at the top or the bottom of a page, on or after the first
page where they are referenced. These ‘inserts’, as they are called in TEX, considerably
complicate the page breaking algorithms, as well as the theory.

\subsection{Typographical constraints}

There are various typographical guidelines for what a page should look like, and TEX has
mechanisms that can encourage, if not always enforce, this behaviour.

\begin{enumerate}
\item The first line of every page should be at the same distance from the top. This changes
if the page starts with a section heading which is a larger type size.

\item The last line should also be at the same distance, this time from the bottom. This
is easy to satisfy if all pages only contain text, but it becomes harder if there are
figures, headings, and display math on the page. In that case, a ‘ragged bottom’ can
be specified.

\item  A page may absolutely not be broken between a section heading and the subsequent
paragraph or subsection heading.

\item It is desirable that

\begin{enumerate}
\item the top of the page does not have the last line of a paragraph started on the
preceding page

\item the bottom of the page does not have the first line of a paragraph that continues
on the next page.
\end{enumerate}

\end{enumerate}



For ordinary purposes you will probably find that \tex's automatic
method of page breaking is satisfactory. And when it occasionally gives unpleasant
results, you can force the machine to break at your favorite place by
typing |\eject|. But be careful: |eject| will cause \tex to stretch the page
out, if necessary, so that the top and bottom baselines agree with those on other
pages.  If you want to eject a short page, filling it with blank space at the bottom,
type | \vfill\eject|  instead.

\section{The current page and the recent contributions list}

The main vertical list of TEX is divided in two parts: the \emph{current page} and the list of \emph{recent
contributions}. Any material that is added to the main vertical list is appended to the recent
contributions; the act of moving the recent contributions to the current page is known as
\emph{exercising the page builder}.

Every time something is moved to the current page, TEX computes the cost of breaking the
page at that point. If it decides that it is past the optimal point, the current page up to the
best break so far is put in |box255| and the remainder of the current page is moved back
on top of the recent contributions. If the page is broken at a penalty, that value is recorded
in |outputpenalty|, and a penalty of size 10 000 is placed on top of the recent contributions;
otherwise, |outputpenalty| is set to 10 000.

If the current page is empty, discardable items that are moved from the recent contributions
are discarded. This is the mechanism that lets glue disappear after a page break and at the
top of the first page. When the first non-discardable item is moved to the current page, the
|topskip| glue is inserted; 



\section{When is the page builder activated?}


The page builder comes into play in the following circumstances.

\begin{enumerate}
\item  Around paragraphs: after the \cs{everypar} tokens have been inserted, and after the
paragraph has been added to the vertical list. See the end of this chapter for an
example.

\item  Around display formulas: after the \cs{everydisplay} tokens have been inserted, and after
the display has been added to the list.

\item  After \cs{par} commands, boxes, insertions, and explicit penalties in vertical mode.

\item  After an output routine has ended.
\end{enumerate}



In these places the page builder moves the recent contributions to the current page. Note that
\tex\  need not be in vertical mode when the page builder is exercised. In horizontal mode,
activating the page builder serves to move preceding vertical glue (for example, \cs{parskip},
\cs{abovedisplayskip}) to the page.

The \cs{end} command – which is only allowed in external vertical mode – terminates a TEX job,
but only if the main vertical list is empty and \cs{deadcycles} = 0. If this is not the case the
combination


|\hbox{}\vfill\penalty+ $-2^{30}$|

is appended, which forces the output routine to act.

\section{The depth of the current page}
The depth of the page is important since normally in good typesetting successive pages should have the same (or almost the same vertical size. (flushbottom). The height of a page is controlled and set exactly by \tex equal to |\vsize|. Consider a large |vbox| with lines of text, glue and penalties. The depth of this box, is the depth of the last component [80]. If the last component is a glue or penalty, the depth is zero. If it is a box, then its depth becomes the depth of the entire |\vbox|, except that it is limited to the value of parameter |\boxmaxdepth|.

If
|\boxmaxdepth=1pt| and the depth of the bottom box
is 1.94444pt, then the depth of the entire |\vbox|
will be 1pt and its height will be incremented
by .94444pt. This is equivalent to lowering the
reference point (or, equivalently, the baseline) of
the |\vbox| by .94444pt. In the plain format,
|\boxmaxdepth=\maxdimen| [348], so it has no effect
on the depths of boxes. However, |\boxmaxdepth|
can always be changed by the user \footnote{This \texttt{\textbackslash boxmaxdepth} setting is to ensure that deep footnotes do not overwrite the
footer (on account of the negative skip added later): it should use \texttt{\textbackslash @maxdepth}
otherwise the change is pointless when there are footnotes.
But see also its use when combining 
floats.  \latex uses a value of 5.5pt whereas plain a value of 4pt [348].}



If the last line on a page, contains letters that happen to not have any depth, the page depth will be zero. Try for example this:

\begin{teXXX}
....
\showthe\pagedepth
\bye
\end{teXXX}

You can also try it with a \latex minimal and will produce the same output.


\section{The height of a box of text}

Following the literature we denote the value of |\baselineskip| (which is normally 12pt) by $b$. 
A
large |\vbox| with text consists mainly of lines of
text, each an |\hbox|, separated by globs of glue,
normally in the (varying) amounts necessary to
separate baselines by exactly $b$, but sometimes just
the amount |\lineskip|. We assume a simple case
where no large characters or equations are used. In
such a case, all lines of text are separated by $b$. The
height of the box is thus:
\begin{gather}
b(n - 1) + \text{the height of the first line}
\end{gather}
where $n$ is the number of text lines. Remember that the first line is a special case and adjustments can be made using the value of |\topskip|.

\section{The height of \texttt{\textbackslash box255}}

In the case of |\box255|,
enough glue is placed above the first line of text
to reach to |\topskip| from the first baseline. We
denote the value of |\topskip| by $h$ (10pt in plain).
So if the baseline of the first line is now h below the
top of the page, the height H of |\box255| should
be b(n - 1) + h (Fig. 3). However, the height of
|\box255| is always set, by the page builder, to
|\vsize|. The difference between the two heights is
usually supplied by the flexible glues on the page,
the most common of which is |\parskip|

\begin{figure}[htp]
\includegraphics{./graphics/heightofpagebox}
\end{figure}


\subsection{Dead cycles.} An execution of the OTR without shipping any material is called a \texttt{dead cycle}. Dead cycles, have their uses and we will explain this a bit later on. However, long iterations that just return \textit{dead cycles} is an indication of an error somewhere. \tex counts the number of dead cycles in a counter named |\deadcycles| and stops the run if |\deadcycles >= \maxdeadcycles|.  In the \textit{plain} format |\maxdeadcycles| is set as 25 and in \latex as \the\deadcycles. |\maxdeadcycles = 100| is \the\maxdeadcycles. Each time |\shipout| is invoked, it resets |\deadcycles| to zero.

\begin{teXXX}
If the file is not included, reset \deadcycles, so that a long list of non-included
files does not generate an `Output loop' error.
115 \deadcycles\z@
116 \@nameuse{cp@#1}%
117 \fi
118 \let\@auxout\@mainaux}
\end{teXXX}


\subsection{\tex's Page Number.} The page number can come from any source. Salomon provides an example where the \textsc{OTR} typesets a page number from a |\count| variable. This is typeset centered below the printed area.

\begin{teXXX}
\newcount\pageNum
\output={
\shipout\vbox{
\box255\smallskip
\centerline{\tenrm\the\pageNum}}
\global\advance\pageNum by1}
\end{teXXX}

Notice that the output macro, just passes the contents of the box to |\shipout|. This is not actually a very good method, but is shown here to illustrate a point.

Note the |\tenrm| in the preceding example. It
is necessary because of the asynchronous nature of
the \otr. When the \otr is invoked, \tex can be
anywhere on the next page. Specifically, it could
be inside a group where a different font is used.
Without the |\tenrm|, that font (the current font)
would be used in the otr.
In the plain format, the |\count0| variable
serves as the page number, and the following two
macros are especially useful.




\subsection{The \texttt{\textbackslash vsplit} operation.} 

Supposed you have inserted the material required to go on a page on a big |\vbox|, but the material is a bit extra that what is required to fill a page exactly. You would need an operation to split the box in two. The |vsplit| operation does that. It is important to the understanding of OTR operations to have an intimate knowledge of |\vsplit|. Its syntax is: 

|\vsplit|\meta{box number} to \meta{dim}

The result of the operation is a box. Most often it appears in an assignment such as: |\setbox1=\vsplit0 to2.6in|. This sets |\box1| to a
height of 2.6in, moves material from the top of
|\box0| to |\box1|, and keeps the remainder in |\box0|.

\begin{macro}{\loremlines}
It is important to remember that most of \tex's commands work with \latex as well. In Example~\ref{ex:loremlines}, we define a box to hold |lipsum| text in a two column layout. We want to define a macro that can split the box in as many lines as we require. 
\end{macro}

\begin{texexample}{Splitting a vbox}{ex:loremlines}
\newbox\one
\newbox\two
\long\gdef\loremlines#1#2{%
   \setbox\one=\vbox {#2}
   \setbox\two=\vsplit\one to #1\baselineskip
   \unvbox\two
   \gdef\boxone{#2}
}
\begin{multicols}{2}
\small
\loremlines{16}{\lipsum[1-2]}
\end{multicols}
\boxone
\end{texexample}


\tex assumes that the new |\box1| may have to
be shipped out as part of the page. It therefore
places a glue similar to $h$ at the top of |\box1|.
This glue is called |\splittopskip| and has a plain
format value of 10pt [348].

One important thing to note is that a box can only be split \textit{between} lines of text. 
If we split a box to another size, |\box1| will come out underfull.

Here is an \otr which splits the page, ships
out the top part and returns the rest to the MVL
(actually, to the recent contributions):

\begin{teXXX}
\output={\setbox0=\vsplit255 to1in
\shipout\box0 \unvbox255}
\end{teXXX}






\section{Communicating with the OTR: Marks}

\begin{multicols}{2}
The user can pass information to the output routine through \textit{marks}. Marks have the syntax

\begin{teX}
\mark{mark text}
\end{teX}

which is put in a mark item on the current vertical list. The mark text is subject to expansion
as in \cs{edef}.
If the mark is given in horizontal mode it migrates to the surrounding vertical lists like an
insertion item (see page Text By Topic 77); however, if this is not the external vertical list, the output routine
will not find the mark.

Marks are the main mechanism through which the output routine can obtain information
about the contents of the currently broken-off page, in particular its top and bottom. TEX sets
three variables:

{\obeylines
\cs{botmark} the last mark occurring on the current page;
\cs{firstmark} the first mark occurring on the current page;
\cs{topmark} the last mark of the previous page, that is, the value of \cs{botmark} on the previous
page.
}



If no marks have occurred yet, all three are empty; if no marks occured on the current pagr, all three variables are equal to the \cs{botmark} of the previous page. 

Marks can be used to get a section heading into the headline or footline of the page.

\begin{verbatim}
\def\section#1{ ... \mark{#1} ... }
\def\rightheadline{\hbox to \hsize
    {\headlinefont \botmark\hfil\pagenumber}}
\def\leftheadline{\hbox to \hsize
   {\headlinefont \pagenumber\hfil\firstmark}}
\end{verbatim}

This places the title of the first section that starts on a left page in the left
headline, and the title of the last section that starts on the right page in
the right headline. Placing the headlines on the page is the job of the output
routine; see below.

It is important that no page breaks can occur in between the mark and the
box that places the title:

\emphasis{mark,nobreak}
\begin{teXXX}
\def\section#1{ ...
   \penalty\beforesectionpenalty
   \mark{#1}
   \hbox{ ... #1 ...}
   \nobreak
   \vskip\aftersectionskip
   \noindent}
\end{teXXX}
\end{multicols}



\section{Insertions}
Insertions are considered one of  the most  com- 
plex  topics in \tex. Many users master  topics  such 
as tokens,  file  I/O, macros,  and  even  OTRS  before 
they dare  tackle  insertions.  The  reason  is  that 
insertions  are  complex,  and  The \texbook, while 
covering all the relevant material, is somewhat cryp- 
tic regarding  insertions, and  lacks  simple examples. 
The  main  discussion  of  insertions takes  place  on 
[115-1251.  where \tex' s  registers  are also discussed. 
Examples  of  insertions are  shown, mostly  without 
explanations,  on  [363-364,  423-424].  A lot of what is described here is based on an article in TUGboat by David Salomon\footnote{ 
http://www.tug.org/TUGboat/Articles/tb11-4/tb30salomon.pdf}

Many users understand the idea of floats. Certain material to be typeset needs to be held in a buffer and inserted at different points on a page, for example a a figure that does not fit on a page it has to be inserted at the top of the next page. An \textit{insertion} is just a piece of a document that is generated at a certain point but appears at another point. Common examples are figures, footnotes and endnotes. Quoting Knuth:

\begin{quote}
  This  algorithm  is  admittedly  complicated, 
but  no  simpler  mechanism  seems  to  do  nearly 
as  much.
\end{quote}

\section{OTR Example}

\begin{figure}%
 \centering
  \includegraphics[width=0.37\linewidth]{./graphics/framedpage}
  \caption{A boxed page}
  \label{fig:framedpage}
\end{figure}

Here is an OTR for a \textit{framed} page. It surrounds the
page with double rules on all sides, and centers the
page number below the double box. Note that the
page shipped out is wider and taller than \cs{box255}.
The value of \cs{hsize} in this case is, therefore, not
the width of the final page shipped out, but the
width of the text lines in \cs{box255}.

Macro \cs{frameit} typesets text and surrounds it
with 4 rules (see [Ex. 21.3]). Parameter \#2 is the
space between the rules and the text. \#1 is a box
containing the text.

\emphasis{output,shipout}
\begin{teXXX}
\def\frameit#1#2{%
 \vbox{\hrule
  \hbox{%
    \vrule \kern#2pt
      \vbox{\kern#2pt #1
         \kern#2pt}%
      \kern#2pt\vrule}
\hrule}}

\output={
   \shipout\vbox{
   \boxit{\frameit{\box255}9}
      \medskip
      \centerline{Test Framed Page}}
  \advancepageno}
\end{teXXX}


Plain TeX has an output routine that takes care of  simple things like page numbering and insertions
using \cs{footnote} and \cs{topinsert}. 

\section{\LaTeX\  output routines}

So far we have examined the \tex OTR in detail. I hope it has given you enough understanding, not only to write your own output routine, but also to now be ready to study the \latex output routine, which is much more complicated. We have so far seen that  when \tex 
is typesetting pages of continuous text, it will gather material until it can find a least-cost page break intended to
make the gathered material fit the \cs{pagegoal size}. The
gathered material will then be placed into |\box255| and
the output routine stored in the token register \cs{output}
will be processed in a group of its own. 

Usually it will
arrange the gathered material in some way, add headers,
footlines and page numbers, and ship the gathered results out in typeset form with the \cs{shipout} command.
At the time of the \cs{shipout} command all \cs{open} and
\cs{write} commands stored in the box shipped out are expanded and written out. This is what makes it possible to have page labels corresponding to the actual page
numbers at the time of shipout: the corresponding info
is written to the |.aux| file at that time.
The output routine may decide to place material
back on the main vertical list instead of shipping it out.

\LaTeX\ output routine is described in \texttt{ltoutput.dtx}. You should also have a look at \texttt{ltfloat.dtx}. The algorithm is revisited i \latex3 and Frank Mittelbach, published a paper
\footnote{\protect\url{http://www.latex-project.org/papers/xo-pfloat.pdf}} in which he explains some of the problems facing the team, when dealing with the output routine.


Information on the output routine is rather scarce. Best source is a series of  articles in the TUGBoat by David Salomon.

\href{http://www.tug.org/TUGboat/Articles/tb11-1/tb27salomon.pdf}{Output Routines: Examples and Techniques. Part I: Introduction and Examples.}

\href{http://www.tug.org/TUGboat/Articles/tb11-2/tb28salomon.pdf}{Output Routines: Examples and Techniques. Part II: OTR Techniques}

\href{http://www.tug.org/TUGboat/Articles/tb11-4/tb30salomon.pdf}{Output Routines: Examples and Techniques. 
Part III: Insertions}

\href{http://www.tug.org/TUGboat/Articles/tb15-1/tb42salomon-output.pdf}{Output routines: Examples and techniques Part IV: Horizontal techniques}


David Kastrup's article \href{http://www.tug.org/TUGboat/Articles/tb24-3/kastrup.pdf}{Output Routine Requirements for Advanced Typesetting Tasks} (Proceedings of EuroTEX 2003) otlined some of the difficult areas and specifications for generic routines

The standard blocks are well described above and most tasks could be accomplished 
by rather working from
standard building blocks like \textit{insertion lists}, \textit{here points},
default mechanisms for \textit{margin notes} and so on.


\section*{Calling the output routine}

The output routine is called either by TeX's normal page-breaking
mechanism, or by a macro putting a penalty < or = -10000 in the output
list. In the latter case, the penalty indicates why the output
routine was called, using the following code.
penalty reason

\begin{tabular}{ll}
\toprule
penalty &reason\\
\midrule
-10000  &\ pagebreak\\
~       &\ newpage\\
-10001  &clearpage (\ penalty -10000 \ vbox{}| \ penalty -10001)|\\
-10002  &float insertion, called from horizontal mode\\
-10003 &float insertion, called from vertical mode.\\
-10004 &float insertion.\\
\bottomrule
\end{tabular}
\medskip

Note: A |float| or |marginpar| puts the following sequence in the output
list: 

\begin{enumerate}
\item a penalty of -10004,

\item a null |\vbox|

\item a penalty of -10002 or -10003.
\end{enumerate}

This solves two special problems:

\begin{enumerate}
\item If the float comes right after a |\newpage| or |\clearpage|,
then the first penalty is ignored, but the second one
invokes the output routine.

\item If there is a split footnote on the page, the second 'page'
puts out the rest of the footnote
\end{enumerate}

\latex first defines some helper routines and increase the \cs{maxdeadcycles}. The helper macros are for
manipulating lisst.

\begin{teX}
 \maxdeadcycles = 100
 \let\@elt\relax
 \def\@next#1#2#3#4{\ifx#2\@empty #4\else
   \expandafter\@xnext #2\@@#1#2#3\fi}
   \@next \CS \LIST {NONEMPTY}{EMPTY} == %% NOTE: ASSUME
\@elt = \relax
 BEGIN assume that \LIST == \@elt \B1 ... \@elt \Bn
 if n = 0
 then EMPTY
 else 
   \CS :=L \B1
   \LIST :=G \@elt \B2 ... \@elt \Bn
   NONEMPTY
 fi
END
\end{teX}


\begin{teX}
11 \def\@xnext \@elt #1#2\@@#3#4{\def#3{#1}\gdef#4{#2}}

12 \def\@testfalse{\global\let\if@test\iffalse}
13 \def\@testtrue {\global\let\if@test\iftrue}
14 \@testfalse}
   }

15 \def\@bitor#1#2{\@testfalse {\let\@elt\@xbitor
16   \@tempcnta #1\relax #2}}

17 \def\@xbitor #1{\@tempcntb \count#1
18    \ifnum \@tempcnta =\z@
19    \else
20      \divide\@tempcntb\@tempcnta
21    \ifodd\@tempcntb \@testtrue\fi
22   \fi}
\end{teX}

\begin{multicols}{2}
\subsection{Float boxes and lists.} 
A \textit{float list} consisting of the 
floats in boxes |\boxa ... \boxN| has
the form:

|\@elt \boxa ... \@elt \boxN|
where |\boxI| is defined by

|\newinsert\boxI|

Normally, |\@elt| is |\let| to |\relax|. A test can be performed on the
entire 
oat list by locally |\def|'ing |\@elt| appropriately and
executing the list.
This is a lot more efficient than looping through the list.
\LaTeX\ defines float boxes as |bx@A| to |bx@R| to make them available for 
inserts. These will be used later to define the lists that hold these boxes. 


\latex now defines the float boxes. Each one is defined as an insert.
\begin{teXXX}
\newinsert\bx@A
...
\newinsert\bx@I
\newinsert\bx@J
\newinsert\bx@K
\newinsert\bx@L
\newinsert\bx@M
\newinsert\bx@N
\newinsert\bx@O
\newinsert\bx@P
\newinsert\bx@Q
\newinsert\bx@R
\end{teXXX}


\end{multicols}
Once these boxes are defined they are inserted in the |@freelist|. At this point all the other lists are defined.

\emphasis{@freelist,@toplist,@botlist,@midlist,@currlist}
\begin{teXXX}
41 \gdef\@freelist{\@elt\bx@A\@elt\bx@B\@elt\bx@C\@elt\bx@D\@elt\bx@E
42                 \@elt\bx@F\@elt\bx@G\@elt\bx@H\@elt\bx@I\@elt\bx@J
43                 \@elt\bx@K\@elt\bx@L\@elt\bx@M\@elt\bx@N
44                 \@elt\bx@O\@elt\bx@P\@elt\bx@Q\@elt\bx@R}
\end{teXXX}

All the lists are defined initially to be empty.
\begin{teXXX}
45 \gdef\@toplist{}
46 \gdef\@botlist{}
47 \gdef\@midlist{}
48 \gdef\@currlist{}
49 \gdef\@deferlist{}
50 \gdef\@dbltoplist{}
51 \gdef\@dbldeferlist{}
\end{teXXX}


The lists are similar to those defined in \texttt{plain}.

\begin{description}
\item[\string\@freelist] : List of empty boxes for placing new 
floats.
\item[\string\@toplist] : List of 
floats to go at top of current column.
\item[\string\@midlist] : List of 
floats in middle of current column.
\item[\string\@botlist] : List of 
floats to go at bottom of current column.
\item[\string\@deferlist] : List of 
floats to go after current column.
\item[\string\@dbltoplist] : List of double-col. 
floats to go at top of current
page.
\item[\string\@dbldeferlist] : List of double-column 
floats to go on subsequent
pages.

\end{description}

\begin{multicols}{2}
Check was prudent when defining the newinsert boxes in order to reserve space and memory. The package \docpkg{morefloats} can be used to add more floats to this list. This should have definitely been included here in a revision.

\subsection{Defining Layout parameters} All the page layout parameters are defined next. 

\begin{teXXX}
52 \newdimen\topmargin
53 \newdimen\oddsidemargin
54 \newdimen\evensidemargin
55 \let\@themargin=\oddsidemargin
56 \newdimen\headheight
57 \newdimen\headsep
58 \newdimen\footskip
59 \newdimen\textheight
60 \newdimen\textwidth
61 \newdimen\columnwidth
62 \newdimen\columnsep
63 \newdimen\columnseprule
64 \newdimen\marginparwidth
65 \newdimen\marginparsep
66 \newdimen\marginparpush
\end{teXXX}

Remember  that TeX knows littel about a page. The problem is that TEX has no idea how
wide and tall the paper is. All it knows is the
left and top offsets, and the dimensions of the
printed area (|\hsize| and |\vsize|). All these dimensions need to be calculated and adjustments made within the \otr.

A document normally  starts by specifying:

\begin{teXXX}
\newdimen\paperheight
\newdimen\paperwidth
\paperheight=..in \paperwidth=..in
\end{teXXX}


\end{multicols}


\subsection*{The AtBeginDvi}
A box register is used  to put stuff that must appear before anything else
in the |.dvi| file.

The stuff in the box should not add any typeset material to the page when it
is unboxed.

\emphasis{AtBeginDvi,@begindvibox}

\begin{teXXX}
67 \newbox\@begindvibox
68 \def \AtBeginDvi #1{%
69 \global \setbox \@begindvibox
70 \vbox{\unvbox \@begindvibox #1}%
71 }
\end{teXXX}

\begin{teXXX}
72 \newdimen\@maxdepth
73 \@maxdepth = \maxdepth
\end{teXXX}


Some new registers for paperheight and paperwidth are defined:

\begin{teXXX}
74 \newdimen\paperheight
75 \newdimen\paperwidth
76 \newif \if@insert
These should definitely be global:
77 \newif \if@fcolmade
78 \newif \if@specialpage \@specialpagefalse
These should be global but are not always set globally in other les.
79 \newif \if@firstcolumn \@firstcolumntrue
80 \newif \if@twocolumn \@twocolumnfalse
Not sure about these: two questions. Should things which must apply to a whole
doument be local or global (they probably should be `preamble only' commands)?
Are these three such things?
81 \newif \if@twoside \@twosidefalse
82 \newif \if@reversemargin \@reversemarginfalse
83 \newif \if@mparswitch \@mparswitchfalse
This counter has been imported from `multicol'.
84 \newcount \col@number
85 \col@number \@ne
\end{teXXX}

and a lot of other internal registers

\begin{teX}
86 \newcount\@topnum
87 \newdimen\@toproom
88 \newcount\@dbltopnum
89 \newdimen\@dbltoproom
90 \newcount\@botnum
91 \newdimen\@botroom
92 \newcount\@colnum
93 \newdimen\@textmin
94 \newdimen\@fpmin
95 \newdimen\@colht
96 \newdimen\@colroom
97 \newdimen\@pageht
98 \newdimen\@pagedp
99 \newdimen\@mparbottom \@mparbottom\z@
100 \newcount\@currtype
101 \newbox\@outputbox
102 \newbox\@leftcolumn
103 \newbox\@holdpg
104 \def\@thehead{\@oddhead} % initialization
105 \def\@thefoot{\@oddfoot}
\end{teX}


\subsection{\texttt{\textbackslash clearpage}}

The clearpage macro is a bit complicated, as it needs to avoid a complete empty page after a |\twocolumn[..]|. This prevents the text from the argument
vanishing into a  float box, never to be seen again. We hope that it does not
produce wrong formatting in other cases.

\begin{teX}
106 \def\clearpage{%
107   \ifvmode
108   \ifnum \@dbltopnum =\m@ne
109     \ifdim \pagetotal <\topskip
110       \hbox{}%
111     \fi
112   \fi
113  \fi
114 \newpage
115 \write\m@ne{}%
116 \vbox{}%
117 \penalty -\@Mi
118 }
\end{teXXX}

\subsection{The \texttt{\textbackslash clearpagedoublepage} macro} 

This checks for odd and even pages by using the
page counter |c@page|.  It also provides switches of twoside printing. 
\TODO{Why not from auxiliary?}

\begin{teXXX}
119 \def\cleardoublepage{\clearpage\if@twoside \ifodd\c@page\else
120 \hbox{}\newpage\if@twocolumn\hbox{}\newpage\fi\fi\fi}
\end{teXXX}

Note the |\newpage| is defined a bit further on. This is a fairly simple definition, since most of the code that follows only gets a bit complicated with the twocolumn option. It sets the dimensions and the booleans to those appropriate for the |onecolumn| option. An important note we back to \tex's |\hsize|. Both the linewidth as well as the columnwidth are set to this.

\begin{teXXX}
123 \def\onecolumn{%
124   \clearpage
125   \global\columnwidth\textwidth
126   \global\hsize\columnwidth
127   \global\linewidth\columnwidth
128   \global\@twocolumnfalse
129   \col@number \@ne
130   \@floatplacement
     }
\end{teXXX}

\subsection{\string newpage.} 

The |\newpage| macro is programmed defensively. The two checks at the beginning ensure that an item label or run-in section title
immediately before a |\newpage| get printed on the correct page, the one before
the page break.
All three tests are largely to make error processing more robust; that is why
they all reset the 
flags explicitly, even when it would appear that this would be
done by a |\leavevmode|.

\begin{teXXX}
131 \def \newpage {%
132  \if@noskipsec
133    \ifx \@nodocument\relax
134      \leavevmode
135      \global \@noskipsecfalse
136    \fi
137 \fi
138 \if@inlabel
139   \leavevmode
140   \global \@inlabelfalse
141 \fi
142 \if@nobreak \@nobreakfalse \everypar{}\fi
143 \par
144 \vfil
145 \penalty -\@M}
\end{teXXX}

An empty cols is defined. There is a note here, that an invisible rule might have been a better idea.

\begin{teXXX}
146 \def \@emptycol {\vbox{}\penalty -\@M}
\end{teXXX}

\subsection{The \string twocolumn macro.} This is the longest definition so far. We will leave it for a while and then come back. There are several bug fixes to the two-column stuff here. Firstly, like the onecolumn the page parameters are set to the correct parameters.


\begin{teXXX}
147 \def \twocolumn {%
148 \clearpage
149 \global\columnwidth\textwidth
150 \global\advance\columnwidth-\columnsep
151 \global\divide\columnwidth\tw@
152 \global\hsize\columnwidth
153 \global\linewidth\columnwidth
154 \global\@twocolumntrue
155 \global\@firstcolumntrue
156 \col@number \tw@
\end{teXXX}



\section*{The output macro}

The setting of the \cs{output} is quite short but it belies its complexity.
After having checked verious parameters it redirects to |@specialoutput|. This is the heart of the routines. Notice that \latex just fills in the token list of \tex's |output| routine, it does not attempt to redefine it or save it. 
Should some hooks be defined here, life might have been made easier, however, what one can do is to first save the \latex output routine and then redefine the output as one may wish. Return to it can happen after it. If you take this approach, you should be careful of packages that redefine output, such as |multicol| and |longtable|. An approach such as this is taken by |revtex|.

\emphasis{ifnum,fi,else,ifdimen,@specialoutput}
\begin{teX}
204 \output {%
205 \let \par \@@par
206 \ifnum \outputpenalty<-\@M
207    \@specialoutput
208 \else
209    \@makecol
210    \@opcol
211    \@startcolumn
212    \@whilesw \if@fcolmade \fi
213      {%
218      \@opcol\@startcolumn}%
219 \fi
220 \ifnum \outputpenalty>-\@Miv
221 \ifdim \@colroom<1.5\baselineskip
222 \ifdim \@colroom<\textheight
223 \@latex@warning@no@line {Text page \thepage\space
224 contains only floats}%
225 \@emptycol
226 % \if@twocolumn
227 % \if@firstcolumn
228 % \else
229 % \@emptycol
230 % \fi
231 % \fi
232 \else
  233 \global \vsize \@colroom
234 \fi
235 \else
236   \global \vsize \@colroom
237 \fi
238 \else
239   \global \vsize \maxdimen
240 \fi
241 }
\end{teX}



\begin{teXXX}
244 \gdef\@specialoutput{%
245   \ifnum \outputpenalty>-\@Mii
246     \@doclearpage
247   \else
248     \ifnum \outputpenalty<-\@Miii
249         \ifnum \outputpenalty<-\@MM \deadcycles \z@ \fi
250                 \global \setbox\@holdpg \vbox {\unvbox\@cclv}%
251         \else
252         \global \setbox\@holdpg \vbox{%
253                 \unvbox\@holdpg
254                 \unvbox\@cclv
We must now remove the box added by the 
oat mechanism and the \topskip
glue therefore added above it by TEX.
255                \setbox\@tempboxa \lastbox
256                \unskip
257 }%
These two are needed as separate dimensions only by \@addmarginpar; for other
purposes we put the whole size into \@pageht (see below).
258                \@pagedp \dp\@holdpg
259                \@pageht \ht\@holdpg
260                \unvbox \@holdpg

261                \@next\@currbox\@currlist{%
262                \ifnum \count\@currbox>\z@
Putting the whole size into \@pageht (see above).
263                  \advance \@pageht \@pagedp
264                  \ifvoid\footins \else
265                    \advance \@pageht \ht\footins
266                    \advance \@pageht \skip\footins
267                    \advance \@pageht \dp\footins
268                \fi
\end{teXXX}



\subsection{The \string @doclearpage macro.} This is an emergency action. It dumps everything: footnotes first and then floats. 


\section*{The Kludgeins}

The kludgeins are simply inserts that fool \tex in enlarging a page by a small amount, normally used to allow one or two lines of text to go in the same page.

The two kludgeins mentioned in the kernel are are \cs{enlargethisspace} and its star version.\footnote{The Oxford English Dictionary (2nd ed., 1989) kludge entry cites one source for this word's earliest recorded usage, definition, and etymology: Jackson W. Granholm's 1962 "How to Design a Kludge" article, which appeared in the American computer magazine Datamation
kludge  Also kluge. [J. W. Granholm's jocular invention: see first quot.; cf. also bodge v., fudge v.]

'An ill-assorted collection of poorly-matching parts, forming a distressing whole' (Granholm); esp. in Computing, a machine, system, or program that has been improvised or 'bodged' together; a hastily improvised and poorly thought-out solution to a fault or 'bug'.

The word 'kludge' is...derived from the same root as the German Kluge..., originally meaning 'smart' or 'witty'.... 'Kludge' eventually came to mean 'not so smart' or 'pretty ridiculous'.}



\begin{teXX}
\gdef \enlargethispage{%
1198 \@ifstar
1199 {%
1203   \@enlargepage{\hbox{\kern\p@}}}%
1204 {%
1208   \@enlargepage\@empty}%
1209 }
\end{teXX}

Adds |<dim>| to the height of the current column only. On the printed page the
bottom of this column is extended downwards by exactly |<dim>| without having
any effect on the placement of the footer; this may result in an overprinting.
\cs{enlargethispage}.

Similar to |\enlargethispage| but it tries to squeeze the column to be printed
in as small a space as possible, ie it uses any shrinkability in the column. If the
column was not explicitly broken (e.g. with |\pagebreak|) this may result in an
overfull box message but except for this it will come out as expected (if you know
what to expect).
The star form of this command is dedicated to Leslie Lamport, the other we
need for ourselves (FMi, CAR).
These commands may well have unwanted if used soon before a\ldots

 




\section{Using packages to ease the pain}

OTR routines are notoriously difficult to debug and define. Some of the available packages at CTAN
can make the programming job easier.

The |everypage| package by Sergio Callegari provides hooks into the \latex\ internal commands to
to do actions on every page or on the current page. Specifically, actions  are performed \emph{before} the page is shipped, so they can be
used to put watermarks \emph{in the background} of a page, or to
set the page layout. 

The package provides two hooks:

\emphasis{AddEverypageHook,AddThisPageHook}
\begin{teXXX}
  \AddEverypageHook{Test}
  \AddThisPageHook
\end{teXXX}

The package reminds in some sense
\docpkg{bobhook} by Karsten Tinnefeld, but it differs in the way in
 which the hooks are implemented, as detailed in the following.
 In some sense it may also be related to the package
 \docpkg{everyshi} by Martin Schroeder, but again the implementation
 is different.

 
 This program adds two \LaTeX\ hooks that get run when document
 pages are finalized and output to the |.dvi| or |.pdf|
 file. Specifically, one hook gets executed on every page, while the
 other is executed for the current page. Hook actions are are performed
 \emph{before} the page is output on the medium, and this is
 important to be able to play with the page layout or to put things
 \emph{behind} the page contents (e.g., watermarks such as an image,
 framing, the ``DRAFT'' word, and the like).
 
 The package reminds in some sense \Lpack{bobhook} by Karsten
 Tinnefeld, but it differs in the way in which the hooks are
 implemented:
 


 \begin{enumerate}
 \item there is no formatting inherent in the hooks. If one wants to
   put some watermark on a page, it is his own duty to put in the
   hook the code to place the watermark in the right position. Also
   note that the hooks code should \emph{eat up no space} in the
   page.  Again, if the hooks are meant to place some material on the
   page, it is the duty of the hook programmer to put code in the
   hooks to pretend that the material has zero width and zero height.
   The implementation is \emph{lighter} than the \Lpack{bobhook} one,
   and possibly more flexible, since one is not limited by any
   pre-coded formatting for the hooks. On the other hand it is
   possibly more difficult to use. Nonetheless, it is easy to think
   of other packages relying on \Lpack{everypage} to deliver more
   user-friendly and \emph{task specific} interfaces. Already there
   are a couple of them: the package \Lpack{flippdf} produces
   mirrored pages in a PDF document and \Lpack{draftwatermark}
   watermarks document pages.
 \item similarly to \Lpack{bobhook} and \Lpack{watermark}, the
   package relies on the manipolatoin of the internal \LaTeX\ macro
   |\@begindvi| to do the job. However, the redefinition of
   |\@begindvi| is here postponed as much as possible, striving to
   avoid interference with other packages using |\AtBeginDvi| or
   anyway manipulating |\@begindvi|. Specifically \Lpack{everypage}
   makes no special assumption on the initial code that |\@begindvi|
   might contain.
 \end{enumerate}



Also in some sense \Lpack{everypage} can be related to package
 \Lpack{everyshi} by Martin Schroeder, but it differs radically from
 it in the implementation. In fact,\Lpack{everypage} operates by
 manipulation of the |\@begindvi| macro, rather than at the
 lower level |shipout| macro.


\section{How to place a background image}

One can use TikZ to place a background image on a page

First we define some utility macros:


\begin{teXXX}
  \def\bg@contents{Draft}
  \def\bg@color{red!45}
  \def\bg@angle{60}
  \def\bg@opacity{.5}
  \def\bg@scale{15}
  \def\bg@position{current page.center}
  \def\bg@anchor{}
  \def\bg@hshift{0}
  \def\bg@vshift{0}
\end{teXXX}

A new command is then developed to describe the background material

\begin{teX}
\newcommand\bg@material{%
   \begin{tikzpicture}[remember picture,overlay]
   \node [rotate=\bg@angle,scale=\bg@scale,opacity=\bg@opacity,%
   xshift=\bg@hshift,yshift=\bg@vshift,color=\bg@color]
   at (\bg@position) [\bg@anchor] {\bg@contents};
  \end{tikzpicture}}%
\end{teX}

Once the background material has been defined we can place it on the page by simply calling

\begin{teXXX}
   \newcommand\BgThispage{\AddThispageHook{\bg@material}}
\end{teXXX}

The background package has capitalized on two good packages the TikZ and the everypage. Similarly you can use your own ingenuity to design whatever you want




\section{hooking at shipout}


This package provides the hooks \cs{EveryShipout} and 
  \cs{AtNextShipout} whose arguments are executed after the output 
  routine has constructed \cs{box255}, and before \cs{shipout} is 
  called.

  An example application for this package would be a package for
  adding text to the bottom of each page.
  Such a package does exist: \docpkg{prelim2e}\cite{package!prelim2e}.

The solution  uses is based on code developed in  \textsf{quire.tex} by
 Marcel R.~van der Goot.  It is based upon \cs{afterassignment} and \cs{aftergroup}.



 









































%  \part{The LaTeX2e Kernel}
%
%  \input{./sections/latexkernel}
%  \input{./sections/kernel-ltspace}
%  \chapter{ltfloat.dtx}

 \section{Float types}

  The different types of floats are identified by a \meta{type} name,
  which is the name of the counter for that kind of float.  For
  example, figures are of type `figure' and tables are of type `table'.
  Each \meta{type} has associated a positive \meta{type number}, which
  is a power of two e.g.,\\
  figures might be have type number~1, tables type number~2, programs
  type number~4, etc. See \urlhttp://tex.stackexchange.com/questions/39017/how-to-influence-the-position-of-float-environments-like-figure-and-table-in-lat/39020#39020{}

  The locations where a float can go are specified by a
  \meta{placement specifier}, which is a list of the possible
  locations, each denoted by a letter as follows:

    \begin{center}
    \begin{tabular}{l@{ : }l@{ --- }l}
     h & here   & at the current location in the text.\\
     t & top    & at the top of a text page.\\
     b & bottom & at the bottom of a text page.\\
     p & page   & on a separate float page
    \end{tabular}
    \end{center}

  In addition, in conjunction with these, you can use `!' which means
  that the current values of the float positioning parameters are
  ignored for this float. (Has no effect on `p', float page
  positioning.)
  For example, `pht' specifies that the float can appear in any of
  three locations: page, here or top. The order of specifying the placement
  specifiers is irrelevant to the float algorithm.


\subsection{Floating Environments}
    \begin{teX}
\message{floats,}
    \end{teX}



 Where floats may appear on a page, and how many may appear there
 are specified by the following float placement parameters.  The
 numbers are named like counters so the user can set them with
 the ordinary counter-setting commands.

\begin{tabular}{lp{6cm}}

  \cs{c@topnumber}      & Number of floats allowed at the top of a column.\\
  \cs{topfraction}      & Fraction of column that can be devoted to floats.
  \cs{c@dbltopnumber}, \cs{dbltopfraction} \\
                    & Same as above, but for double-column floats.\\
  \cs{c@bottomnumber}, \cs{bottomfraction}\\ 
                    & Same as above for bottom of page.\\
  \cs{c@totalnumber}    & Number of floats allowed in a single column,
                          including in-text floats.\\
  \cs{textfraction}     &Minimum fraction of column that must contain text.\\
  \cs{floatpagefraction}& Minimum fraction of page that must be taken
                          up by float page.\\
  \cs{dblfloatpagefraction} 
                    & Same as above, for double-column floats.\\
\end{tabular}


 The document style must define the following.

\begin{longtable}{lp{6cm}}
    \cs{fps@TYPE}   & The default placement specifier for floats of type
                  TYPE. \\
    \cs{ftype@TYPE} & The type number for floats of type TYPE.\\
    \cs{ext@TYPE}   & The file extension indicating the file on which the
                  contents list for float type TYPE is stored.
                    For example,  \cs{ext@figure = 'lof'}.\\
    \cs{fnum@TYPE}  & A macro to generate the figure number for a caption.
                  For example, \cs{fnum@TYPE} == Figure \cs{thefigure}.\\
    \cs{@makecaption}{NUM}{TEXT} & 
              A macro to make a caption, with NUM the value
              produced by \cs{fnum@}... and TEXT the text of the caption.
              It can assume it's in a \cs{parbox} of the appropriate width.\\
\end{longtable}

\begin{teX}
 \@float{type}[placement] : This macro begins a float environment for a (*@ float @*)
     single-column float of type TYPE with PLACEMENT as the placement
     specifier.  The default value of PLACEMENT is defined by
     \fps@TYPE.  The environment is ended by \end@float.
     E.g., \figure == \@float{figure}, \endfigure == \end@float.

  \@float{TYPE}[PLACEMENT] ==
   BEGIN
     if hmode then \@bsphack
                          \@floatpenalty := -10002
              else      \@floatpenalty := -10003
     fi
     \@captype ==L TYPE
     \@dblflset
     \@fps     ==L PLACEMENT
     \@onelevel@sanitize \@fps 
     add default PLACEMENT if at most ! in PLACEMENT == \@fpsadddefault
     if inner
       then LaTeX Error: 'Not in outer paragraph mode.'
            \@floatpenalty := 0
       else if \@freelist nonempty
              then \@currbox  :=L head of \@freelist
                   \@freelist :=G tail of \@freelist
                   \count\@currbox :=G 32*\ftype@TYPE + 
                                          bits determined by PLACEMENT
              else \@floatpenalty := 0
                   LaTeX Error: 'Too many unprocessed floats'
            fi
     fi
     \@currbox :=G   \color@vbox
                       \normalcolor
                         \vbox{
                          %% 15 Dec 87 --
                          %% removed \boxmaxdepth :=L 0pt
                          %% that made box 0 depth because it screwed
                          %% things up. Instead, added \vskip0pt at end
                               \hsize = \columnwidth
                               \@parboxrestore
                               \@floatboxreset
   END

  \caption ==
    BEGIN
     \refstepcounter{\@captype}
     \@dblarg{\@caption{\@captype}}
    END

 In following definition, \par moved from after \addcontentsline to
 before \addcontentsline because the \write could cause
 an extra blank line to be added to the paragraph above the
 caption.  (Change made 12 Jun 87)

  \@caption{TYPE}[STEXT]{TEXT} ==
   BEGIN
     \par
     \addcontentsline{\ext@TYPE}{TYPE}{\numberline{\theTYPE}{STEXT}}
     \begingroup
       \@parboxrestore
       \@normalsize
       \@makecaption{\fnum@TYPE}{TEXT}
       \par
     \endgroup
   END


  \@dblfloat{TYPE}[PLACEMENT] : Macro to begin a float environment for
     a double-column float of type TYPE with PLACEMENT as the placement
     specifier.  The default value of PLACEMENT is 'tp'
     The environment is ended by \end@dblfloat.
     E.g., \figure* == \@dblfloat{figure}, 
           \endfigure* == \end@dblfloat.

  \@dblfloat{TYPE}[PLACEMENT] ==
     Identical to \@float{TYPE}[PLACEMENT] except \hsize and \linewidth
     are set to \textwidth.
\end{teX}

 \begin{macro}{\@floatpenalty}  
The float penalty is saved in a counter to enable ease of use.
 \end{macro}
    \begin{teX}
\newcount\@floatpenalty
    \end{teX}
 


\begin{macro}{\caption}

    This is set to be an error message outside a float since no
    captype is defined there; this may need to be changed by some 
    classes. Note if the caption is outside a float it triggers
	an error.

    \begin{teX}
\def\caption{%
   \ifx\@captype\@undefined
     \@latex@error{\noexpand\caption outside float}\@ehd
     \expandafter\@gobble
   \else
     \refstepcounter\@captype
     \expandafter\@firstofone
   \fi
   {\@dblarg{\@caption\@captype}}%
} 
    \end{teX}
 \end{macro}

 \begin{macro}{\@caption}
    \begin{teX}
\long\def\@caption#1[#2]#3{%
  \par
  \addcontentsline{\csname ext@#1\endcsname}{#1}%
    {\protect\numberline{\csname the#1\endcsname}{\ignorespaces #2}}%
  \begingroup
    \end{teX}

 The paragraph setting parameters are normalised at this point, however
 |\@parboxrestore| resets |\everypar| which is not correct in this
 context so |\@setminipage| is called if needed.

 The float mechanism, like minipage, sets the flag |@minipage| true
 before executing the user-supplied text. Many \LaTeX\ constructs
 test for this flag and do not add vertical space when it is true.
 The intention is that this emulates \TeX's `top of page' behaviour.
 The flag must be set false at the start of the first paragraph. This
 is achieved by a redefinition of |\everypar|, but the call to
 |\@parboxrestore| removes that redefinition, so it is re-inserted 
 if needed. If the flag is already false then the |\caption| was not
 the first entry in the float, and so some other paragraph has already
 activated the special |\everypar|. In this case no further action is
 needed.
    \begin{teX}
    \@parboxrestore
    \if@minipage
      \@setminipage
    \fi
    \end{teX}

    \begin{teX}
    \normalsize
    \@makecaption{\csname fnum@#1\endcsname}{\ignorespaces #3}\par
  \endgroup}
    \end{teX}
 \end{macro}

 \begin{macro}{\@float} Just a reminder that this is used to define new floating
 environments, such as \textit{figure} and \textit{table}.  The |fps@#1| has been
 defined in the standard classes and defines the default float specifier. In |book.cls|
is defined as |\def\fps@figure{tbp}|.

 \begin{macro}{\@dblflset}
    \begin{teX}
\def\@float#1{%
  \@ifnextchar[%
    {\@xfloat{#1}}%
    {\edef\reserved@a{\noexpand\@xfloat{#1}[\csname fps@#1\endcsname]}%
     \reserved@a}}
    \end{teX}
    
 \end{macro}
 \end{macro}

  \begin{macro}{\@dblfloat}

    \begin{teX}
\def\@dblfloat{%
  \if@twocolumn\let\reserved@a\@dbflt\else\let\reserved@a\@float\fi
  \reserved@a}
    \end{teX}
  \end{macro}

    
  \begin{macro}{\fps@dbl}
  Note that all double floats have default fps `tp'.
  \end{macro}
  
  \begin{macro}{\@setfps}
    This sets the fps, dealing with error conditions by adding
    the default.
  \end{macro}

  \begin{macro}{\@xfloat}

   The first part of this sets the count register that stores all
   the information about the type and fps of the float.

    We assume here that the default specifiers already contain no
    active characters.

    It may be better to store the defaults as numbers, rather than
    symbol strings.

   The |\@nodocument| is defined in the ltxerror and is the error produced if paragraphs are typeset in the preamble. So here LaTeX is checking if a float was specified in a preamble. Why here, is because
we cannot rely on user defined environments to carry out these tests.

What does the ! do? This goes to default, so it is interesting to find out why the bang was specified
in the first place. It just starts overrides all the constraints.

    \begin{teX}
\def\@xfloat #1[#2]{%
  \@nodocument 
  \def \@captype {#1}%
   \def \@fps {#2}%
   \@onelevel@sanitize \@fps 
   \def \reserved@b {!}%
   \ifx \reserved@b \@fps
     \@fpsadddefault
   \else
     \ifx \@fps \@empty
       \@fpsadddefault
     \fi
   \fi
   \ifhmode
     \@bsphack
     \@floatpenalty -\@Mii
   \else
     \@floatpenalty-\@Miii
   \fi
  \ifinner
     \@parmoderr\@floatpenalty\z@
  \else
    \@next\@currbox\@freelist
      {%
       \@tempcnta \sixt@@n
       \expandafter \@tfor \expandafter \reserved@a
         \expandafter :\expandafter =\@fps 
         \do
          {%
           \if \reserved@a h%
             \ifodd \@tempcnta
             \else
               \advance \@tempcnta \@ne
             \fi
           \fi
           \if \reserved@a t%
             \@setfpsbit \tw@
           \fi
           \if \reserved@a b%
             \@setfpsbit 4%
           \fi
           \if \reserved@a p%
             \@setfpsbit 8%
           \fi
           \if \reserved@a !%
             \ifnum \@tempcnta>15
               \advance\@tempcnta -\sixt@@n\relax
             \fi
           \fi
           }%
       \@tempcntb \csname ftype@\@captype \endcsname
       \multiply \@tempcntb \@xxxii
       \advance \@tempcnta \@tempcntb
       \global \count\@currbox \@tempcnta
       }%
    \@fltovf  %This is for too many floats error for marginpars
  \fi
    \end{teX}
    The remainder sets up the box in which the float is typeset, and
    the typesetting environment to be used.  It is essential to have
    the extra box to avoid the unwanted space that would otherwise
    often be put at the top of the float.

    It ends with a hook; not sure how useful this is but it is needed
    at present to deal with double-column floats.
    \begin{teX}
  \global \setbox\@currbox
    \color@vbox
      \normalcolor
      \vbox \bgroup
        \hsize\columnwidth
        \@parboxrestore
        \@floatboxreset
}
    \end{teX}
  \end{macro}
  
  \begin{macro}{\@floatboxreset}
    
 The rational for allowing these normally global flags to be set
 locally here, via |\@parboxrestore|, was stated originally by
 Donald Arseneau and extended by Chris Rowley.
 It is because these flags are only set globally to
 true by section commands, and these should never appear within
 marginals or floats or, indeed, in any group; and they are only ever
 set globally to false when they are definitely true.

 If anyone is unhappy with this argument then both flags should be
 treated as in |\set@nobreak|; otherwise this command will be
 redundant. 
     {Added local settings of flags: dangerous!!}
    \begin{teX}
\def \@floatboxreset {%
        \reset@font
        \normalsize
        \@setminipage
}
    \end{teX}
  \end{macro}
  
  \begin{macro}{\@setnobreak}
    \begin{teX}
\def \@setnobreak{%
  \if@nobreak
    \let\outer@nobreak\@nobreaktrue
    \@nobreakfalse
  \fi
}
    \end{teX}
  \end{macro}

  \begin{macro}{\@setminipage}
    \begin{teX}
\def \@setminipage{%
  \@minipagetrue
  \everypar{\@minipagefalse\everypar{}}%
}
    \end{teX}
  \end{macro}

 \begin{macro}{\end@float}
    \begin{teX}
\def\end@float{%
  \@endfloatbox
  \ifnum\@floatpenalty <\z@
    \end{teX}
 We make sure that we never exceed |\textheight|, otherwise float
 will never get typeset (91/03/15 FMi).
    \begin{teX}
    \@largefloatcheck
    \@cons\@currlist\@currbox
    \ifnum\@floatpenalty <-\@Mii
      \penalty -\@Miv
    \end{teX}
 Saving and restoring |\prevdepth| added 26 May 87 to prevent extra
 vertical space when used in vertical mode.
    \begin{teX}
      \@tempdima\prevdepth
      \vbox{}%
      \prevdepth\@tempdima
    \end{teX}

    \begin{teX}
      \penalty\@floatpenalty
    \end{teX}
 
    \begin{teX}
    \else
      \vadjust{\penalty -\@Miv \vbox{}\penalty\@floatpenalty}\@Esphack
    \fi
  \fi
}
    \end{teX}
 \end{macro}

 \begin{macro}{\end@dblfloat}
    \begin{teX}
\def\end@dblfloat{%
\if@twocolumn
  \@endfloatbox
  \ifnum\@floatpenalty <\z@
    \end{teX}
 We make sure that we never exceed |\textheight|, otherwise float
 will never get typeset (91/03/15 FMi).
    \begin{teX}
    \@largefloatcheck
    \@cons\@dbldeferlist\@currbox
  \fi
    \end{teX}
 RmS 92/03/18 changed |\@esphack| to |\@Esphack|.
    \begin{teX}
    \ifnum \@floatpenalty =-\@Mii \@Esphack\fi
\else
  \end@float
\fi
}
    \end{teX}
 \end{macro}
 
 \begin{macro}{\@endfloatbox}
    This macro is not intended to be a hook; it is designed to help
    maintain the integrity of this code, which is used twice and, as
    can be seen, is subject to frequent changes.
    \begin{teX}
\def \@endfloatbox{%
      \par\vskip\z@skip      %% \par\vskip\z@ added 15 Dec 87
    \end{teX}
   
    \begin{teX}
      \@minipagefalse   
      \outer@nobreak
    \egroup                  %% end of vbox
  \color@endbox
}
% 
 \begin{macro}{\outer@nobreak}
    \begin{teX}
\let\outer@nobreak\@empty
    \end{teX}
  \end{macro}
 

  \begin{macro}{\@largefloatcheck}
 
    This calculates by how much a float is oversize for the page and
    prints this in a warning message.
    
    \begin{teX}  
\def \@largefloatcheck{%
  \ifdim \ht\@currbox>\textheight
    \@tempdima -\textheight
    \advance \@tempdima \ht\@currbox
    \end{teX}

    \begin{teX}
    \@latex@warning {Float too large for page by \the\@tempdima}%
    \ht\@currbox \textheight
  \fi
}
    \end{teX}
  \end{macro}

  \begin{macro}{\@dbflt}
  \begin{macro}{\@xdblfloat}
    \begin{teX}
\def\@dbflt#1{\@ifnextchar[{\@xdblfloat{#1}}{\@xdblfloat{#1}[tp]}}
\def\@xdblfloat#1[#2]{%
  \@xfloat{#1}[#2]\hsize\textwidth\linewidth\textwidth}
    \end{teX}
  \end{macro}
  \end{macro}

    \begin{teX}
\newcount\c@topnumber
\newcount\c@dbltopnumber
\newcount\c@bottomnumber
\newcount\c@totalnumber
    \end{teX}

 An analysis of |\@floatplacement|:

 This should be called whenever |\@colht| has been set.
    \begin{teX}
\def\@floatplacement{\global\@topnum\c@topnumber
    % Textpage bit, global:
   \global\@toproom \topfraction\@colht
   \global\@botnum  \c@bottomnumber
   \global\@botroom \bottomfraction\@colht
   \global\@colnum  \c@totalnumber
    % Floatpage bit, local:
   \@fpmin   \floatpagefraction\@colht}
    \end{teX}

  \begin{macro}{\@dblfloatplacement}
 
     This should be called only within a group.  Now changed to
     provide extra checks in |\@addtodblcol|, needed when processing a
     BANG float.
    
    \begin{teX}  
\def \@dblfloatplacement {%
    \end{teX}
    Textpage bit: global, but need not be.
    \begin{teX}  
  \global \@dbltopnum \c@dbltopnumber
  \global \@dbltoproom \dbltopfraction\@colht
    \end{teX}
   This new bit uses |\@textmin| to locally store the amount of extra
   room in the column.   
    \begin{teX}
  \@textmin \@colht
  \advance \@textmin -\@dbltoproom
    \end{teX}
    Floatpage bit: must be local.
    \begin{teX}
  \@fpmin \dblfloatpagefraction\textheight
  \@fptop \@dblfptop
  \@fpsep \@dblfpsep
  \@fpbot \@dblfpbot
}
    \end{teX}
  \end{macro}

\section{Marginal Notes}

   Marginal notes use the same mechanism as floats to communicate
   with the \cs{output} routine.  Marginal notes are distinguished from
   floats by having a negative placement specification.  The command
   \cs{marginpar}[LTEXT]{RTEXT} generates a marginal note in a parbox,
   using LTEXT if it's on the left and RTEXT if it's on the right.
   (Default is RTEXT = LTEXT.)  It uses the following parameters.

% \begin{oldcomments}
   \marginparwidth : Width of marginal notes.
   \marginparsep   : Distance between marginal note and text.
        the page layout to determine how to move the marginal
        note into the margin.   E.g., \@leftmarginskip ==
        \hskip -\marginparwidth \hskip -\marginparsep .
   \marginparpush  :  Minimum vertical separation between \marginpar's

  Marginal notes are normally put on the outside of the page
  if @mparswitch = true, and on the right if @mparswitch = false.
  The command \reversemarginpar reverses the side where they
  are put.  \normalmarginpar undoes \reversemarginpar.
  These commands have no effect for two-column output.

  SURPRISE: if two marginal notes appear on the same line of
  text, then the second one could appear on the next page, in
  a funny position. (I was unable to reproduce the error).


  \marginpar [LTEXT]{RTEXT} ==
   BEGIN
     if hmode then \@bsphack
                   \@floatpenalty := -10002
              else \@floatpenalty := -10003
     fi
     if inner
       then LaTeX Error: 'Not in outer paragraph mode.'
            \@floatpenalty := 0
       else if \@freelist has two elements:
              then get \@marbox, \@currbox  from \@freelist
                   \count\@marbox :=G -1
              else \@floatpenalty := 0
                   LaTeX Error: 'Too many unprocessed floats'
                   \@currbox, \@marbox := \@tempboxa    %%use \def
            fi
     fi
     if optional argument
       then %% \@xmpar ==
            \@savemarbox\@marbox{LTEXT}
            \@savemarbox\@currbox{RTEXT}
       else %% \@ympar ==
            \@savemarbox\@marbox{RTEXT}
            \box\@currbox :=G \box\@marbox
    fi
    \@xympar 
   END

 \reversemarginpar == BEGIN \@mparbottom   :=G 0
                            @reversemargin :=G true
                      END

 \normalmarginpar  == BEGIN \@mparbottom   :=G 0
                            @reversemargin :=G false
                      END

  \end{oldcomments}
%
 \begin{macro}{\marginpar}
    \begin{teX}
\def\marginpar{%
  \ifhmode
    \@bsphack
    \@floatpenalty -\@Mii
  \else
    \@floatpenalty-\@Miii
  \fi
  \ifinner
    \@parmoderr
    \@floatpenalty\z@
  \else
    \@next\@currbox\@freelist{}{}%
    \@next\@marbox\@freelist{\global\count\@marbox\m@ne}%
       {\@floatpenalty\z@
        \@fltovf\def\@currbox{\@tempboxa}\def\@marbox{\@tempboxa}}%
  \fi
  \@ifnextchar [\@xmpar\@ympar}
    \end{teX}
 \end{macro}
%
% \begin{macro}{\@xmpar}
%    \begin{teX}
\long\def\@xmpar[#1]#2{%
  \@savemarbox\@marbox{#1}%
  \@savemarbox\@currbox{#2}%
  \@xympar}
%    \end{teX}
% \end{macro}

This is the main macro for a marginpar command that does not have left or
right text. Note it call \cs{@xympar}
 \begin{macro}{\@ympar}
    \begin{teX}
\long\def\@ympar#1{%
  \@savemarbox\@marbox{#1}%
  \global\setbox\@currbox\copy\@marbox
  \@xympar}
    \end{teX}
 \end{macro}
 
 \begin{macro}{\@savemarbox}
 sets up the vboxes including a color@vbox to correctly handle colour. 
    \begin{teX}
\long\def \@savemarbox #1#2{%
  \global\setbox #1%
    \color@vbox
      \vtop{%
        \hsize\marginparwidth
        \@parboxrestore 
        \@marginparreset
        #2%
        \@minipagefalse   
        \outer@nobreak
        }%
    \color@endbox
}
    \end{teX}
  \end{macro}
 
%  \begin{macro}{\@marginparreset}
% \changes{v1.1f}{1994/11/21}{Macro added}
%
% The rational for allowing these normally global flags to be set
% locally here, via |\@parboxrestore| was stated originally by
% Donald Arsenau and extended by Chris Rowley.
% It is because these flags are only set globally to
% true by section commands, and these should never appear within
% marginals or floats or, indeed, in any group; and they are only ever
% set globally to false when they are definitely true.
%
% If anyone is unhappy with this argument then both flags should be
% treated as in |\set@nobreak|; otherwise this command will be
% redundant. 
% \changes{v1.1p}{1996/10/24}
%     {Added local settings of flags: dangerous!!}
%    \begin{teX}
\def \@marginparreset {%
        \reset@font
        \normalsize
%        \let\if@nobreak\iffalse
%        \let\if@noskipsec\iffalse
%        \@setnobreak
        \@setminipage
}
%    \end{teX}
%  \end{macro}
%
% \begin{macro}{\@xympar}
%
%    \begin{teX}
\def \@xympar{%
  \ifnum\@floatpenalty <\z@\@cons\@currlist\@marbox\fi
  \setbox\@tempboxa
    \color@vbox
      \vbox \bgroup
  \end@float
  \@ignorefalse
  \@esphack
}
%    \end{teX}
% \end{macro}
%
 \begin{macro}{\reversemarginpar}
 \begin{macro}{\normalmarginpar}
    \begin{teX}
\def\reversemarginpar{\global\@mparbottom\z@ \@reversemargintrue}
\def\normalmarginpar{\global\@mparbottom\z@ \@reversemarginfalse}
    \end{teX}
 \end{macro}
 \end{macro}

    \begin{teX}
\message{footnotes,}
    \end{teX}

 \section{Footnotes}

We start with a summary of all user commands.

\begin{tabular}{lp{6cm}}
   \cs{footnote}\marg{text} &User command to insert a footnote.\\
   \cs{footnote}\oarg[NUM]\marg{NOTE} &User command to insert a footnote numbered, NUM, where NUM is a number -- 1, 2,
                       etc.  For example, if footnotes are numbered
                       *, **, etc. within pages, then \cs{footnote}|[2]{...}|
                       produces footnote '**'.  This command does not
                       step the footnote counter.
\end{tabular}

\begin{oldcomments}
%   \footnotemark[NUM] : Command to produce just the footnote mark in
%                        the text, but no footnote.  With no argument,
%                        it steps the footnote counter before generating
%                        the mark.
%
%   \footnotetext[NUM]{TEXT} : Command to produce the footnote but
%                              no mark.  \footnote is equivalent to
%                              \footnotemark \footnotetext .
%

%   As in PLAIN, footnotes use \insert\footins, and the following
%   parameters: 
%
%   \footnotesize   : Size-changing command for footnotes.
%
%   \footnotesep    : The height of a strut placed at the beginning of
%                     every footnote.
%   \skip\footins   : Space between main text and footnotes.  The rule
%                     separating footnotes from text occurs in this
%                     space. This space lies above the strut of height
%                     \footnotesep which is at the beginning of the
%                     first footnote.
%   \footnoterule   : Macro to draw the rule separating footnotes from
%                     text. It is executed right after a \vspace of
%                     \skip\footins. It should take zero vertical
%                     space--i.e., it should to a negative skip to
%                     compensate for any positive space it occupies.
%                     (See PLAIN.TEX.)
%
%   \interfootnotelinepenalty : Interline penalty for footnotes.
%
%   \thefootnote : In usual LaTeX style, produces the footnote number.
%                  If footnotes are to be numbered within pages, then
%                  the document style file must include an \@addtoreset
%                  command to cause the footnote counter to be reset
%                  when the page counter is stepped.  This is not a good
%                  idea, though, because the counter will not always be
%                  reset in time to ensure that the first footnote on a
%                  page is footnote number one.
%
%   \@thefnmark : Holds the current footnote's mark--e.g., \dag or '1'
%                 or 'a'. 
%
%   \@mpfnnumber  : A macro that generates the numbers for \footnote
%                  and \footnotemark commands. It == \thefootnote
%                  outside a minipage environment, but can be
%                  changed inside to generate numbers for
%                  \footnote's.
%
%   \@makefnmark : A macro to generate the footnote marker from
%                 \@thefnmark The default definition was
%                 \hbox{$^\@thefnmark$}.
%
%                 This is now replaced by
%                 \textsuperscript{\@thefnmark}
%
%   \@makefntext{NOTE} :
%        Must produce the actual footnote, using \@thefnmark as the mark
%        of the footnote and NOTE as the text.  It is called when
%        effectively  inside a \parbox, with \hsize = \columnwidth.
%          For example, it might be as simple as
%               $^{\@thefnmark}$  NOTE
%
% In a minipage environment, \footnote and \footnotetext are redefined
% so that
%    (a) they use the counter mpfootnote
%    (b) the footnotes they produce go at the bottom of the minipage.
% The switch is accomplished by letting \@mpfn == footnote or mpfootnote
% and \thempfn == \thefootnote or \thempfootnote, and by redefining
% \@footnotetext to be \@mpfootnotetext in the minipage.
%
% \footnote{NOTE}  ==
%  BEGIN
%    \stepcounter{\@mpfn}
%    begingroup
%       \protect == \noexpand
%       \@thefnmark :=G eval (\thempfn)
%    endgroup
%    \@footnotemark
%    \@footnotetext{NOTE}
%  END
%
% \footnote[NUM]{NOTE} ==
%  BEGIN
%    begingroup
%       \protect == \noexpand
%       counter \@mpfn :=L NUM
%       \@thefnmark :=G eval (\thempfn)
%    endgroup
%    \@footnotemark
%    \@footnotetext{NOTE}
%  END
%
% \footnotemark      ==
%  BEGIN \stepcounter{footnote}
%        begingroup
%           \protect == \noexpand
%           \@thefnmark :=G eval(\thefootnote)
%        endgroup
%        \@footnotemark
%  END
%
% \footnotemark[NUM] ==
%   BEGIN
%       begingroup
%         footnote counter :=L NUM
%         \protect == \noexpand
%        \@thefnmark :=G eval(\thefootnote)
%       endgroup
%       \@footnotemark
%   END
%
% \@footnotemark ==
%   BEGIN
%    \leavevmode
%    IF hmode THEN \@x@sf := \the\spacefactor FI
%    \@makefnmark          % put number in main text
%    IF hmode THEN \spacefactor := \@x@sf FI
%   END
%
% \footnotetext      ==
%    BEGIN begingroup \protect == \noexpand
%                     \@thefnmark :=G eval (\thempfn)
%          endgroup
%          \@footnotetext
%    END
%
% \footnotetext[NUM] ==
%    BEGIN begingroup  counter \@mpfn :=L NUM
%                      \protect == \noexpand
%                      \@thefnmark :=G eval (\thempfn)
%          endgroup
%          \@footnotetext
%    END
%
% \end{oldcomments}

\begin{algorithm}
\cs{footnotetext}[NUM] ==\\
\Begin{
 begingroup\\
  counter \cs{@mpfn} := L NUM\\
  \cs{protect} == \cs{noexpand}\\
  \cs{@thefnmark} :=G eval(\cs{thempfn})\\
 endgroup\\
 \cs{@footnotetext}\\
}
\end{algorithm}


 \begin{macro}{\footins}
 \LaTeX\ does use the same insert for footnotes as PLAIN.
    \begin{teX}
\newinsert\footins
    \end{teX}

 \LaTeX\ leaves these initializations for the |\footins| insert.

    \begin{teX}
\skip\footins=\bigskipamount % space added when footnote is present
\count\footins=1000 % footnote magnification factor (1 to 1)
\dimen\footins=8in % maximum footnotes per page
    \end{teX}
 \end{macro}


 \begin{macro}{\footnoterule}
 \LaTeX\ keeps PLAIN \TeX's |\footnoterule| as the default.

    \begin{teX}
\def\footnoterule{\kern-3\p@
  \hrule \@width 2in \kern 2.6\p@} % the \hrule is .4pt high
    \end{teX}
 \end{macro}

 \begin{macro}{\thefootnote}
    \begin{teX}
\@definecounter{footnote}
\def\thefootnote{\@arabic\c@footnote}
    \end{teX}
 \end{macro}

 \begin{macro}{\thempfootnote}
    The default display for the footnote counter in minipages is to
    use italic letters. We use |\itshape| not |\textit| as the latter
    would add an italic correction.
    \begin{teX}
\@definecounter{mpfootnote}
\def\thempfootnote{{\itshape\@alph\c@mpfootnote}}
    \end{teX}
 \end{macro}

 \begin{macro}{\@makefnmark}
    \begin{teX}
\def\@makefnmark{\hbox{$^{\@thefnmark}\m@th$}}
\def\@makefnmark{\hbox{\@textsuperscript{\normalfont\@thefnmark}}}
    \end{teX}
 \end{macro}

  \begin{macro}{\textsuperscript}
    This command provides superscript characters in the current text
    font. It's implementation might change!!!
    \begin{teX}
\DeclareRobustCommand*\textsuperscript[1]{%
  \@textsuperscript{\selectfont#1}}
    \end{teX}
  \end{macro}

  \begin{macro}{\@textsuperscript}
    This command should not be used directly, but may be used to define
   other commands |\textsuperscript|, |\@makefnmark|. |#1| should
   always start with a font selection command, to activate the font
   size switch.
    \begin{teX}
\def\@textsuperscript#1{%
  {\m@th\ensuremath{^{\mbox{\fontsize\sf@size\z@#1}}}}}
    \end{teX}
  \end{macro}

 \begin{macro}{\footnotesep}
    \begin{teX}
\newdimen\footnotesep
    \end{teX}
 \end{macro}

 \begin{macro}{\footnote}

    \begin{teX}
\def\footnote{\@ifnextchar[\@xfootnote{\stepcounter\@mpfn
     \protected@xdef\@thefnmark{\thempfn}%
     \@footnotemark\@footnotetext}}
    \end{teX}
 \end{macro}

 \begin{macro}{\@xfootnote}
    \begin{teX}
\def\@xfootnote[#1]{%
   \begingroup 
     \csname c@\@mpfn\endcsname #1\relax
     \unrestored@protected@xdef\@thefnmark{\thempfn}%
   \endgroup
   \@footnotemark\@footnotetext}
    \end{teX}
 \end{macro}

 \begin{macro}{\@footnotetext}
    \begin{teX}
\long\def\@footnotetext#1{\insert\footins{%
    \reset@font\footnotesize
    \interlinepenalty\interfootnotelinepenalty
    \splittopskip\footnotesep
    \splitmaxdepth \dp\strutbox \floatingpenalty \@MM
    \hsize\columnwidth \@parboxrestore
    \protected@edef\@currentlabel{%
       \csname p@footnote\endcsname\@thefnmark
    }% 
    \color@begingroup
      \@makefntext{%
        \rule\z@\footnotesep\ignorespaces#1\@finalstrut\strutbox}%
    \color@endgroup}}%
    \end{teX}
 \end{macro}

 \begin{macro}{\footnotemark}

    \cs{footnotemark}.} 
    \begin{teX}
\def\footnotemark{%
   \@ifnextchar[\@xfootnotemark
     {\stepcounter{footnote}%
      \protected@xdef\@thefnmark{\thefootnote}%
      \@footnotemark}}
    \end{teX}
 \end{macro}

 \begin{macro}{\@xfootnotemark}
    \begin{teX}
\def\@xfootnotemark[#1]{%
   \begingroup 
      \c@footnote #1\relax
      \unrestored@protected@xdef\@thefnmark{\thefootnote}%
   \endgroup
   \@footnotemark}
    \end{teX}
 \end{macro}

 \begin{macro}{\@footnotemark}

         {Add \cs{nobreak} to allow hyphenation. latex/1605}
    \begin{teX}
\def\@footnotemark{%
  \leavevmode
  \ifhmode\edef\@x@sf{\the\spacefactor}\nobreak\fi
  \@makefnmark
  \ifhmode\spacefactor\@x@sf\fi
  \relax}
    \end{teX}
 \end{macro}

 \begin{macro}{\footnotetext}
    \begin{teX}
\def\footnotetext{%
     \@ifnextchar [\@xfootnotenext
       {\protected@xdef\@thefnmark{\thempfn}%
    \@footnotetext}}
    \end{teX}
 \end{macro}

 \begin{macro}{\@xfootnotenext}
    \begin{teX}
\def\@xfootnotenext[#1]{%
  \begingroup 
     \csname c@\@mpfn\endcsname #1\relax
     \unrestored@protected@xdef\@thefnmark{\thempfn}%
  \endgroup
  \@footnotetext}
    \end{teX}
 \end{macro}

 \begin{macro}{\thempfn}
 \begin{macro}{\@mpfn}
    \begin{teX}
\def\@mpfn{footnote}
\def\thempfn{\thefootnote}

    \end{teX}
 \end{macro}
 \end{macro}

 
%  
\chapter{ltlists.dtx}
         
 \section{List, and related environments}

 The generic commands for creating an indented environment --
 |enumerate|, |itemize|, |quote|, etc -- are:
 \begin{quote}
        |\list|\marg{LABEL}\marg{COMMANDS} ... |\endlist|
 \end{quote}

 which can be invoked by the user as the list environment.  The LABEL
 argument specifies item labeling.  COMMANDS contains commands for
 changing the horizontal and vertical spacing parameters.

 Each item of the environment is begun by the command
 |\item[|ITEMLABEL|]|
 which produces an item labeled by ITEMLABEL.  If the argument is
 missing, then the LABEL argument of the |\list| command is used as the
 item label.

 The label is formed by putting |\makelabel|\marg{ITEMLABEL} in an hbox
 whose width is either its natural width or else |\labelwidth|,
 whichever is larger.  The |\list| command defines |\makelabel| to have
 the default  definition:
 \begin{quote}
     |\makelabel|\marg{ARG} == BEGIN |\hfil| ARG END
 \end{quote}
 which, for a label of width less than |\labelwidth|, puts the label
 flushright, |\labelsep| to the left of the item's text.  However,
 |\makelabel| can be |\let| to another command by the |\list|'s
 COMMANDS argument.

 A |\usecounter|\marg{foo} command in the second argument causes the
 counter \emph{foo} to be initialized to zero, and stepped by every
 |\item| command without an argument.  (|\label| commands within the
 list refer to this counter.)

 When you leave a list environment, returning either to an enclosing
 list or normal text mode, LaTeX begins a new paragraph if and only if
 you leave a blank line after the |\end| command.  This is accomplished
 by the |\@endparenv| command.

 Blank lines are ignored every other reasonable place--i.e.:
 \begin{itemize}
  \item  Between the |\begin{list}| and the first |\item|,
  \item  Between the |\item| and the text of that item.
  \item Between the end of the last item and the |\end{list}|.
 \end{itemize}

 For an environment like quotation, in which items are not labeled,
 the entire environment is a single item.  It is defined by
 letting |\quotation| == |\list{}{...}\item\relax|.  (Note the
 |\relax|, there in case the first character in the environment is a
 '['.)  The spacing parameters provide a great deal of flexability in
 designing the format, including the ability to let the indentation of
 the first paragraph be different from that of the subsequent ones.

 The trivlist environment is equivalent to a list environment
 whose second argument sets the following parameter values:
 \begin{description}
 \item[\cs{leftmargin} = 0:] causes no indentation of left margin
 \item[\cs{labelwidth} = 0:] see below for precise effect this has.
 \item[\cs{itemindent} = 0:] with a null label, makes first paragraph
        have no indentation.  Succeeding paragraphs have |\parindent|
        indentation.  To give first paragraph same indentation, set
        |\itemindent| = |\parindent| before the |\item[]|.
 \end{description}

 Every |\item| in a trivlist environment must have an argument---in
 many cases, this will be the null argument (|\item[]|).  The trivlist
 environment is mainly used for paragraphing environments, like
 verbatim, in which there is no margin change.  It provides the same
 vertical spacing as the list environment, and works reasonably well
 when it occurs immediately after an |\item| command in an enclosing
 list.



 \subsection{List and Trivlist}


 The following variables are used inside a list environment:
 \begin{description}
 \item[\cs{@totalleftmargin}] The distance that the prevailing left
     margin is indented from the outermost left margin,
 \item[\cs{linewidth}] The width of the current line.  Must be
     initialized to |\hsize|.
 \item[\cs{@listdepth}] A count for holding current list nesting depth.
 \item[\cs{makelabel}] A macro with a single argument, used to
   generate the label from the argument (given or implied)
   of the |\item| command. Initialized to |\@mklab| by the |\list|
   command.  This command must produce  some stretch---i.e., an
   |\hfil|.
 \item[\cs{@inlabel}] A switch that is false except between the time
   an |\item| is encountered and the time that \TeX{}
   actually enters horizontal mode.  Should be tested by commands
   that can be messed up by the list environment's use of |\everypar|.
 \item[\cs{box}\cs{@labels}] When |@inlabel = true|, it holds the labels
   to be put out by |\everypar|.
 \item[\texttt{@noparitem}] A switch set by |\list| when
   |@inlabel = true|.
   Handles the case of a |\list| being the first thing in an item.
 \item[\texttt{@noparlist}] A switch set true for a list that begins an
   item.  No |\topsep| space is added before or after |\item|'s such a
   list.
 \item[\texttt{@newlist}] Set true by |\list|, set false by the first
   text (by |\everypar|).
 \item[\texttt{@noitemarg}]  Set true when executing an |\item| with no
   explicit argument.  Used to save space. To save time, make two
   separate  |\@item| commands.
 \item[\texttt{@nmbrlist}] Set true by |\usecounter| command, causes
   list to be numbered.
 \item[\cs{@listctr}] |\def|'ed by |\usecounter| to name of counter.
 \item[\cs{@noskipsec}] A switch set true by a sectioning command when
    it is creating an in-text heading with |\everypar|.
 \end{description}


 Throughout a list environment, |\hsize| is the width of the current
 line, measured from the outermost left margin to the outermost right
 margin.  Environments like tabbing should use |\linewidth| instead of
 |\hsize|.

 Here are the parameters of a list that can be set by commands in
 the |\list|'s COMMANDS argument.  These parameters are all TeX
 skips or dimensions (defined by |\newskip| or |\newdimen|), so the
 usual \TeX\ or \LaTeX\ commands can be used to set them.  The
 commands will be executed in vmode if and only if the |\list| was
 preceded by a |\par| (or something like an |\end{list}|), so the
 spacing parameters can be set according to whether the list is
 inside a paragraph or is its own paragraph.


 \subsection{Vertical Spacing (skips)}
 \begin{description}
 \item[\cs{topsep}:]  Space between first item and preceding paragraph.
 \item[\cs{partopsep}:] Extra space added to \cs{topsep} when
        environment starts a new paragraph (is called in vmode).
 \item[\cs{itemsep}:] Space between successive items.
 \item[\cs{parsep}:] Space between paragraphs within an item -- the
                 \cs{parskip} for this environment.
 \end{description}

 \subsection{Penalties}
 \begin{description}

 \item[\cs{@beginparpenalty}:] put at the beginning of a list
 \item[\cs{@endparpenalty}:] put at end of list
  \item[\cs{@itempenalty}:] put between items.
  \end{description}

 \subsection{Horizontal Spacing (dimens)}
 \begin{description}
 \item[\cs{leftmargin}:] space between left margin of enclosing
   environment (or of page if top level list) and left margin of
                     this list.  Must be nonnegative.
  \item[\cs{rightmargin}:] analogous.
  \item[\cs{listparindent}:] extra indentation at beginning of every
     paragraph of a list except the one started by the \cs{item}
                      command.  May be negative!  Usually, labeled
                       lists have \cs{listparindent} equal to zero.
   \item[\cs{itemindent}:] extra indentation added right BEFORE an item
                      label.
  \item[\cs{labelwidth}:] nominal width of box that contains the label.
                      If the natural width of the
                         label $< =$ \cs{labelwidth},
                      then the label is flushed right inside a box
                      of width \cs{labelwidth} (with an \cs{hfil}).
                      Otherwise,
                      a box of the natural width is employed, which
                       causes an indentation of the text on that line.
     \item[\cs{labelsep}:] space between end of label box and text of
                      first item.
  \end{description}





 \subsection{Default Values}
 
 Defaults for the list environment are set as follows.
 First, \cs{rightmargin}, \cs{listparindent} and \cs{itemindent}
 are set
      to 0pt.  Then, one of the commands
      \cs{@listi}, \cs{@listii}, ... , \cs{@listvi}
      is called, depending upon the current level of the list.
      The \cs{@list} \ldots commands should be defined by the document
      style.  A convention that the document style should follow is
      to set \cs{leftmargin} to
      \cs{leftmargini},\ldots, \cs{leftmarginvi} for
      the appropriate level.  Items that aren't changed may be left
      alone, but everything that could possibly be changed must be
      reset.


\LinesNumbered
\begin{algorithm}
\caption{The \cs{list} environment}
  \cs{list}\marg{LABEL}\marg{COMMANDS} ==\\
   \Begin{
     \eIf{\cs{@listdepth} > 5}{
        LaTeX error: 'Too deeply nested'}{
        \cs{@listdepth} :=G \cs{@listdepth} + 1\\
     }
     \cs{rightmargin}     := 0pt\\
     \cs{listparindent}   := 0pt\\
     \cs{itemindent}      := 0pt\\
     eval(@list \cs{romannumeral}\cs{the}\cs{@listdepth})\\  
     \cs{@itemlabel}      :=L LABEL\\
     \cs{makelabel}       == \cs{@mklab}\\
     @nmbrlist        :=L false\\
     COMMANDS\\
     \cs{@trivlist}\\  
     \cs{parskip}          :=L \cs{parsep}\\
     \cs{parindent}        :=L \cs{listparindent}\\
     \cs{linewidth}        :=L \cs{linewidth} - \cs{rightmargin} -\cs{leftmargin}\\
     \cs{@totalleftmargin} :=L \cs{@totalleftmargin} + \cs{leftmargin}\\
     \cs{parshape} 1 \cs{@totalleftmargin} \cs{linewidth}\\
     \cs{ignorespaces} 
   }
\end{algorithm}

Th \cs{endlist} simply adjusts the listdepth and ends the \cs{trivlist}.

\begin{algorithm}
 \cs{endlist} == \\
  \Begin{
    \cs{@listdepth} :=G \cs{@listdepth} -1\\
    \cs{endtrivlist}\\
  }
\end{algorithm}

The \cs{@trivlist} is define as,

\begin{algorithm}
 \cs{@trivlist} ==\\
  \Begin{
    \If{@newlist = T}{\cs{@noitemerr}}
     This command removed for some forgotten reason.\\
     \cs{@topsepadd} :=L \cs{topsep}\\
     \If{@noskipsec}{leave vertical mode}
     \eIf{vertical mode}{
        \cs{@topsepadd} :=L \cs{@topsepadd} + \cs{partopsep}}{
        \cs{unskip} \cs{par}}
     \eIf{@inlabel = true}{
         @noparitem :=L true
         @noparlist :=L true}{
         @noparlist :=L false
             \cs{@topsep}   :=L \cs{@topsepadd}}
     \cs{@topsep}      :=L \cs{@topsep} + \cs{parskip}\\
      Restore paragraphing parameters\\
     \cs{leftskip}     :=L 0pt\\  
     \cs{rightskip}    :=L \cs{@rightskip}\\
     \cs{parfillskip}     :=L 0pt + 1fil\\
   NOTE: \cs{@setpar} called on every \cs{list} in case \cs{par} has been\\
   temporarily  munged before the \cs{list} command.\\
     \cs{@setpar}{if @newlist = false then {\@@par} fi}\\
     \cs{@newlist}         :=G T\\
     \cs{@outerparskip}    :=L \cs{parskip}\\
 }
\end{algorithm}

\begin{algorithm}
 \cs{trivlist}  ==\\
 \Begin{
  \cs{parsep} := \cs{parskip}\\
   @nmbrlist := F\\
  \cs{@trivlist}\\
  \cs{labelwidth} := 0\\
  \cs{leftmargin} := 0\\
  \cs{itemindent} := \cs{parindent}\\
  \cs{@itemlabel} :=L "empty"\\ 
  \cs{makelabel}\marg{LABEL} == LABEL\\
 }
\end{algorithm}



\begin{algorithm}
 \cs{endtrivlist} ==\\
 \Begin{
     \If{@inlabel = T}{\cs{indent}}
     \If{horizontal mode}{\cs{unskip} \cs{par}}
     \eIf{@noparlist = true}{}{
        \If{\cs{lastskip} > 0}{
              \cs{@tempskipa} := \cs{lastskip}
              \cs{vskip} - \cs{lastskip}
              \cs{vskip} \cs{@tempskipa} -\cs{@outerparskip} + \cs{parskip}
             }
           \cs{@endparenv}
     }
   }
\end{algorithm}


\begin{algorithm}
 \cs{@endparenv} ==
   \Begin{
    \cs{addpenalty}\marg{@endparpenalty}\\
    \cs{addvspace}\marg{\cs{@topsepadd}}\\
     ends the \cs{begin} command's \cs{begingroup}\\
    \cs{endgroup}\\
    \cs{par}  ==  \Begin{%
                  \cs{@restorepar}\\
                  \cs{everypar}|{}|
                  \cs{par}
                  }
    \cs{everypar} == \Begin{remove \cs{lastbox} \cs{everypar}|{}|}
   to match the \cs{end} commands \cs{endgroup}\\
    \cs{begingroup}  
   }
\end{algorithm}

\index{kernel>lists>\textbackslash item}
The definition of item, is fairly simple deferring the complexity
to |\@item| which follows.

\begin{algorithm}[htbp]
 \cs{item} == \Begin{
    \If{math mode}{issue warning}
    \eIf{ next char = {\tt [}}{
             \cs{@item}}{
              \cs{@noitemarg} := true\\
             \cs{@item}[\cs{@itemlabel}]}
  }
\caption{The algorithm for \textbackslash item}
\end{algorithm}


\begin{algorithm}
 \cs{@item}[LAB] ==
    \Begin{
     \eIf{@noparitem = true}{
        @noparitem := false
             % NOTE: then clause  hardly every taken,\\
             %  so made a macro \cs{@donoparitem}\\
            \cs{box}\cs{@labels} :=G\\
             \cs{hbox} \Begin{\cs{hskip} -\cs{leftmargin}\\
                                   \cs{box}\cs{@labels}\\
                                   \cs{hskip} \cs{leftmargin}}
            \If{@minipage = false}{
               \cs{@tempskipa} := \cs{lastskip}\\
               \cs{vskip} -\cs{lastskip}\\
               \cs{vskip} \cs{@tempskipa} + \cs{@outerparskip} - \cs{parskip}}
            }{
          \If{@inlabel = true}{
              then \cs{indent} \cs{par}   % previous item empty.
           }
           \If{hmode}{then 2 \cs{unskip}'s\\
                           % To remove any space at end of prev.\\
                           % paragraph that could cause a blank line.\\
                     \cs{par}\\
           }
           \eIf{if @newlist = T}{
                \eIf{@nobreak = T}{ 
                      % Kludge if list follows \cs{section}\\
                      \cs{addvspace}\marg{\cs{@outerparskip} - \cs{parskip}}}{
                       \cs{addpenalty}\marg{\cs{@beginparpenalty}}\\
                       \cs{addvspace}\marg{\cs{@topsep}}\\
                       \cs{addvspace}\marg{-\cs{parskip}}\\  
                    }}{
                \cs{addpenalty}\marg{\cs{@itempenalty}}\\
                \cs{addvspace}\marg{\cs{itemsep}}\\
            }

            @inlabel :=G true\\
     }

     \cs{everypar}\{ @minipage :=G F\\
                @newlist :=G F\\
                \If{@inlabel = true}{
                    @inlabel :=G false
                       \cs{hskip} -\cs{parindent}\\
                       \cs{box}\cs{@labels}\\
                       \cs{penalty} 0\\
                       \cs{box}\cs{@labels} :=G null\\
                }
                \cs{everypar}\{\} \}\\
     @nobreak :=G false\\
     \If{@noitemarg = true}{
        @noitemarg := false\\
            \If{@nmbrlist}{
               \cs{refstepcounter}\{\cs{@listctr}\}
            }
     }
     \cs{@tempboxa}   :=L \cs{hbox}\marg{\cs{makelabel}\marg{LAB}}\\
     \cs{box}\cs{@labels} :=G \cs{@labels}\cs{hskip}\cs{itemindent}\\
                       \cs{hskip} - (\cs{labelwidth} + \cs{labelsep})\\
                 \eIf{\cs{wd}\cs{@tempboxa} > \cs{labelwidth}}{
                          \cs{box}\cs{@tempboxa}}{
                          \cs{hbox} to \cs{labelwidth}\{\cs{unhbox}\cs{@tempboxa}\}\\
                 } 
      \cs{hskip}\cs{labelsep}\\
     \cs{ignorespaces} %gobble space up to text
  }
\end{algorithm}




\begin{algorithm}
 \cs{makelabel}\marg{LABEL} == ERROR\\
    default to catch lonely \cs{item}

 \cs{usecounter}\marg{CTR} == \Begin{
                              \cs{@nmbrlist} :=L true\\
                              \cs{@listctr} == CTR\\
                              \cs{setcounter}\marg{CTR}\marg{0}
                             }
\end{algorithm}



 DEFINE \cs{dimen}'s and \cs{count}

 \begin{macro}{\topskip}
 \begin{macro}{\partopsep}
 \begin{macro}{\itemsep}
 \begin{macro}{\parsep}
 \begin{macro}{\@topsep}
 \begin{macro}{\@topsepadd}
 \begin{macro}{\outerparskip}
    \begin{teX}
\newskip\topsep
\newskip\partopsep
\newskip\itemsep
\newskip\parsep
\newskip\@topsep
\newskip\@topsepadd
\newskip\@outerparskip
    \end{teX}
 \end{macro}\end{macro}\end{macro}\end{macro}\end{macro}\end{macro}
 \end{macro}
 \begin{macro}{\leftmargin}\begin{macro}{\rightmargin}
 \begin{macro}{\listparindent}\begin{macro}{\itemindent}
 \begin{macro}{\labelwidth}\begin{macro}{\labelsep}
 \begin{macro}{\@totalleftmargin}
    \begin{teX}
\newdimen\leftmargin
\newdimen\rightmargin
\newdimen\listparindent
\newdimen\itemindent
\newdimen\labelwidth
\newdimen\labelsep
\newdimen\linewidth
\newdimen\@totalleftmargin \@totalleftmargin=\z@
    \end{teX}
 \end{macro}\end{macro}\end{macro}\end{macro}\end{macro}
 \end{macro}\end{macro}

 \begin{macro}{\leftmargini}
 \begin{macro}{\leftmarginii}
 \begin{macro}{\leftmarginiii}
 \begin{macro}{\leftmarginiv}
 \begin{macro}{\leftmarginv}
 \begin{macro}{\leftmarginvi}
    \begin{teX}
\newdimen\leftmargini
\newdimen\leftmarginii
\newdimen\leftmarginiii
\newdimen\leftmarginiv
\newdimen\leftmarginv
\newdimen\leftmarginvi
    \end{teX}
 \end{macro}\end{macro}\end{macro}
 \end{macro}\end{macro}\end{macro}

 \begin{macro}{\@listdepth}\begin{macro}{\@itempenalty}
 \begin{macro}{\@beginparpenalty}\begin{macro}{\@endparpenalty}
    \begin{teX}
\newcount\@listdepth \@listdepth=0
\newcount\@itempenalty
\newcount\@beginparpenalty
\newcount\@endparpenalty
    \end{teX}
 \end{macro}\end{macro}\end{macro}\end{macro}

 \begin{macro}{\@labels}
    \begin{teX}
\newbox\@labels
    \end{teX}
 \end{macro}

 \begin{macro}{\if@inlabel}
 \begin{macro}{\@inlabelfalse}
 \begin{macro}{\@inlabeltrue}
    \begin{teX}
\newif\if@inlabel \@inlabelfalse
    \end{teX}
 \end{macro}\end{macro}\end{macro}

 \begin{macro}{\if@newlist}
 \begin{macro}{\@newlistfalse}
 \begin{macro}{\@newlisttrue}
    \begin{teX}
\newif\if@newlist   \@newlistfalse
    \end{teX}
 \end{macro}\end{macro}\end{macro}

 \begin{macro}{\if@noparitem}
 \begin{macro}{\@noparitemfalse}
 \begin{macro}{\@noparitemtrue}
    \begin{teX}
\newif\if@noparitem \@noparitemfalse
    \end{teX}
 \end{macro}\end{macro}\end{macro}

 \begin{macro}{\if@noparlist}
 \begin{macro}{\@noparlistfalse}
 \begin{macro}{\@noparlisttrue}
    \begin{teX}
\newif\if@noparlist \@noparlistfalse
    \end{teX}
 \end{macro}\end{macro}\end{macro}

 \begin{macro}{\if@noitemarg}
 \begin{macro}{\@noitemargfalse}
 \begin{macro}{\@noitemargtrue}
    \begin{teX}
\newif\if@noitemarg \@noitemargfalse
    \end{teX}
 \end{macro}\end{macro}\end{macro}

 \begin{macro}{\if@newlist}
 \begin{macro}{\@newlistfalse}
 \begin{macro}{\@newlisttrue}
    \begin{teX}
\newif\if@nmbrlist  \@nmbrlistfalse
    \end{teX}
 \end{macro}\end{macro}\end{macro}

 \begin{macro}{\list}
\index{kernel>lists>\textbackslash list}
 List takes two arguments and is an author command for building
other lists.
    \begin{teX}
\def\list#1#2{%
  \ifnum \@listdepth >5\relax
    \@toodeep
  \else
    \global\advance\@listdepth\@ne
  \fi
  \rightmargin\z@
  \listparindent\z@
  \itemindent\z@
  \csname @list\romannumeral\the\@listdepth\endcsname
  \def\@itemlabel{#1}%
  \let\makelabel\@mklab
  \@nmbrlistfalse
  #2\relax
  \@trivlist
  \parskip\parsep
  \parindent\listparindent
  \advance\linewidth -\rightmargin
  \advance\linewidth -\leftmargin
  \advance\@totalleftmargin \leftmargin
  \parshape \@ne \@totalleftmargin \linewidth
  \ignorespaces}
    \end{teX}
 \end{macro}

 \begin{macro}{\par@deathcycles}
    \begin{teX}
\newcount\par@deathcycles
    \end{teX}
 \end{macro}

 \begin{macro}{\@trivlist}

 Because |\par| is sometimes made a no-op it is possible for a missing
 |\item| to produce a loop that does not fill memory and so never gets
 trapped by \TeX.  We thus need to trap this here by seting |\par| to
 count the number of times a paragraph ii is called with no progress
 being made started.
    \begin{teX}
\def\@trivlist{%
  \if@noskipsec \leavevmode \fi
  \@topsepadd \topsep
  \ifvmode
    \advance\@topsepadd \partopsep
  \else
    \unskip \par
  \fi
  \if@inlabel
    \@noparitemtrue
    \@noparlisttrue
  \else
    \if@newlist \@noitemerr \fi
    \@noparlistfalse
    \@topsep \@topsepadd
  \fi
  \advance\@topsep \parskip
  \leftskip \z@skip
  \rightskip \@rightskip
  \parfillskip \@flushglue
  \par@deathcycles \z@
  \@setpar{\if@newlist
             \advance\par@deathcycles \@ne
             \ifnum \par@deathcycles >\@m
               \@noitemerr
               {\@@par}%
             \fi
           \else
             {\@@par}%
           \fi}%
  \global \@newlisttrue
  \@outerparskip \parskip}
    \end{teX}
 \end{macro}

 
 \begin{macro}{\trivlist}
    \begin{teX}
\def\trivlist{%
  \parsep\parskip
  \@nmbrlistfalse
  \@trivlist
  \labelwidth\z@
  \leftmargin\z@
  \itemindent\z@
    \end{teX}

    We initialise |\@itemlabel| so that a \texttt{trivlist} with
    an |\item| not having an optional argument doesn't produce an
    error message.
 \changes{latex2e}{1993/12/13}{Initialised \cs{@itemlabel}}
    \begin{teX}
  \let\@itemlabel\@empty
  \def\makelabel##1{##1}}
    \end{teX}
 \end{macro}

 \begin{macro}{\endlist}
    \begin{teX}
\def\endlist{%
  \global\advance\@listdepth\m@ne
  \endtrivlist}
    \end{teX}
 \end{macro}

    The definition of \cs{trivlist} used to be in ltspace.dtx 
    so that other commands could be `let to it'.  
    They now use \cs{def}.

 \begin{macro}{\endtrivlist}
 \changes{v1.2b ltspace}{1994/11/12}{Changed order of tests to make
 \cs{@noitemerror} correct: end of an era.}
 \changes{v1.0i}{1995/05/25}{Macros moved from ltspace.dtx}
 \changes{v1.0n}{1996/10/25}{Change \cs{indent} to \cs{leavevmode}}
 \changes{v1.0n}{1996/10/25}{Reset flags explicitly}
 \changes{v1.0o}{1996/10/26}{Correct typo}
    \begin{teX}
\def\endtrivlist{%
  \if@inlabel
    \leavevmode
    \global \@inlabelfalse
  \fi
  \if@newlist
    \@noitemerr
    \global \@newlistfalse
  \fi
  \ifhmode\unskip \par
    \end{teX}
    We also check if we are in math mode and issue an error message
    if so (hoping that |\@currenvir| resolves suitably). Otherwise
    the usual ``perhaps a missing item'' error will get triggered
    later which is confusing.
 \changes{v1.0s}{2002/10/28}{Check for math mode (pr/3437)}
    \begin{teX}
  \else
    \@inmatherr{\end{\@currenvir}}%
  \fi
  \if@noparlist \else
    \ifdim\lastskip >\z@
      \@tempskipa\lastskip \vskip -\lastskip
      \advance\@tempskipa\parskip \advance\@tempskipa -\@outerparskip
      \vskip\@tempskipa
    \fi
    \@endparenv
  \fi
}
    \end{teX}
 \end{macro}
 

 
 \begin{macro}{\@endparenv}
 \begin{macro}{\@doendpe}
 To suppress the paragraph indentation in text immediately following
 a paragraph-making environment, \cs{everypar} is changed to remove the
 space, and \cs{par} is redefined to restore \cs{everypar}.  Instead of
 redefining \cs{par} and \cs{everypar}, \cs{@endparenv} was changed to 
 set the @endpe switch, letting \cs{end} redefine \cs{par} and 
 \cs{everypar}.  

 This allows paragraph-making environments to work right when called 
 by other environments. (Changed 27 Oct 86)
    \begin{teX}
\def\@endparenv{%
  \addpenalty\@endparpenalty\addvspace\@topsepadd\@endpetrue}
    \end{teX}

    \begin{teX}
\def\@doendpe{\@endpetrue
     \def\par{\@restorepar\everypar{}
          \par\@endpefalse}\everypar
    \end{teX}
    
    Use |\setbox0=\lastbox| instead of   |\hskip -\parindent|   
    so that a \cs{noindent} becomes a no-op when used before 
    a line immediately following a list environment(23 Oct 86).
 \changes{v1.0k}{1995/11/07}{Enclosed \cs{setbox0} assignment by a
 group so that it leaves the contents of box $0$ intact.
    } 
    \begin{teX}
               {{\setbox\z@\lastbox}\everypar{}\@endpefalse}}
    \end{teX}
 \end{macro}
 \end{macro}

 
 \begin{macro}{\if@endpe}
 \begin{macro}{\@endpefalse}
 \begin{macro}{\@endpeltrue}
    \begin{teX}
\newif\if@endpe
\@endpefalse
    \end{teX}
 \end{macro}\end{macro}\end{macro}

 
 \begin{macro}{\@mklab}
    \begin{teX}
\def\@mklab#1{\hfil #1}
    \end{teX}
 \end{macro}

 \changes{LaTeX2.09}{1992/09/18}
     {(RmS) Added warning if \cs{item} is used in math mode}
 \changes{v1.0c}{1994/04/28}
     {Replaced \cs{@ltxnomath} by \cs{@inmatherr}}
 \changes{v1.0d}{1994/05/03}
     {Removed superfluous braces}
 \begin{macro}{\item}
    \begin{teX}
\def\item{%
  \@inmatherr\item
  \@ifnextchar [\@item{\@noitemargtrue \@item[\@itemlabel]}}
    \end{teX}
 \end{macro}
 \begin{macro}{\@donoparitem}
    \begin{teX}
\def\@donoparitem{%
  \@noparitemfalse
  \global\setbox\@labels\hbox{\hskip -\leftmargin
                               \unhbox\@labels
                                \hskip \leftmargin}%
  \if@minipage
    \else
      \@tempskipa\lastskip
      \vskip -\lastskip
      \advance\@tempskipa\@outerparskip
      \advance\@tempskipa -\parskip
      \vskip\@tempskipa
  \fi}
    \end{teX}
 \end{macro}

 \begin{macro}{\@item}
 \changes{v1.0l}{1996/07/26}{Remove unecessary \cs{global} before
                 \cs{@minipage...}}
    \begin{teX}
\def\@item[#1]{%
  \if@noparitem
    \@donoparitem
  \else
    \if@inlabel
      \indent \par
    \fi
    \ifhmode
      \unskip\unskip \par
    \fi
    \if@newlist
      \if@nobreak
        \@nbitem
      \else
        \addpenalty\@beginparpenalty
        \addvspace\@topsep
        \addvspace{-\parskip}%
      \fi
    \else
      \addpenalty\@itempenalty
      \addvspace\itemsep
    \fi
    \global\@inlabeltrue
  \fi
  \everypar{%
    \@minipagefalse
    \global\@newlistfalse
    \end{teX}
    This |\if@inlabel| check is needed in case an item starts of
    inside a group so that |\everypar| does not become empty
    outside that group. 
 \@nobreakfalse, etc etc.
    \begin{teX}
    \if@inlabel
      \global\@inlabelfalse
    \end{teX}
    The paragraph indent is now removed by using |\setbox...| since
    this makes |\noindent| a no-op here, as it should be. Thus the
    following comment is redundant but is left here for the sake of
    future historians:
    this next command was changed from an hskip to a kern to avoid
    a break point after the parindent box: the skip could cause a
    line-break if a very long label occurs in raggedright setting.
 \changes{v1.0d}{1994/05/03}{\cs{hskip} changed to \cs{kern}}
 \changes{v1.0m}{1996/10/23}{\cs{kern...} changed to \cs{setbox...}}
 \changes{v1.0r}{1997/02/21}
    {\cs{ifvoid} check added for \cs{noindent}. latex/2414}
 If |\noindent| was used after |\item| want to cancel the |\itemindent|
 skip. This case can be detected as the indentation box will be void.
    \begin{teX}
      {\setbox\z@\lastbox
       \ifvoid\z@
         \kern-\itemindent
       \fi}%
    \end{teX}

    \begin{teX}
      \box\@labels
      \penalty\z@
    \fi
    \end{teX}
    This code is intended to prevent a page break after the first
    line of an item that comes immediately after a section title. It
    may be sensible to always forbid a page break after one line of
    an item?  As with all such settings of |\clubpenalty| it is local
    so will have no effect if the item starts in a group.

    Only resetting |\@nobreak| when it is true is now
    essential since now it is sometimes set locally.
 \changes{v1.0m}{1996/10/23}{Added setting of \cs{clubpenalty} and
    set \cs{@nobreakfalse} only when necessary}
    \begin{teX}
    \if@nobreak
      \@nobreakfalse
      \clubpenalty \@M
    \else
      \clubpenalty \@clubpenalty
      \everypar{}%
    \fi}%
    \end{teX}
 \changes{v1.0l}{1996/07/26}{Remove unecessary \cs{global} before
                 \cs{@nobreak...}}
 \changes{v1.0m}{1996/10/23}{\cs{@nobreak...} moved into the
          \cs{everypar} and not executed unconditionally, see above} 
    \begin{teX}
  \if@noitemarg
    \@noitemargfalse
    \if@nmbrlist
    \end{teX}
 \changes{v1.0g}{1995/05/17}{Removed surplus braces}
    \begin{teX}
      \refstepcounter\@listctr
    \fi
  \fi
    \end{teX}
    We use |\sbox| to support colour commands.
 \changes{LaTeX2e}{1993/12/08}{use \cs{sbox} to support colour}
    \begin{teX}
  \sbox\@tempboxa{\makelabel{#1}}%
  \global\setbox\@labels\hbox{%
    \unhbox\@labels
    \hskip \itemindent
    \hskip -\labelwidth 
    \hskip -\labelsep
    \ifdim \wd\@tempboxa >\labelwidth
      \box\@tempboxa
    \end{teX}
 \changes{LaTeX2.09}{1991/11/22}
         {(RmS) Changed second call to \cs{makelabel} to
           \cs{unhbox}\cs{@tempboxa}.
          Avoids problems with side effects in \cs{makelabel} and is
               more efficient.}
    \begin{teX}
    \else
      \hbox to\labelwidth {\unhbox\@tempboxa}%
    \fi
    \hskip \labelsep}%
  \ignorespaces}
    \end{teX}
 \end{macro}

 \begin{macro}{\makelabel}
 \changes{LaTeX2.09}{1991/11/04}
         {(RmS) added default definition for \cs{makelabel},
               to produce an error message.}
    \begin{teX}
\def\makelabel#1{%
  \@latex@error{Lonely \string\item--perhaps a missing
        list environment}\@ehc}
    \end{teX}
 \end{macro}

 \begin{macro}{\@nbitem}
 \changes{v1.0g}{1995/05/17}{Removed surplus braces}
    \begin{teX}
\def\@nbitem{%
  \@tempskipa\@outerparskip
  \advance\@tempskipa -\parskip
  \addvspace\@tempskipa}
    \end{teX}
 \end{macro}

 \begin{macro}{\usecounter}
    \begin{teX}
\def\usecounter#1{\@nmbrlisttrue\def\@listctr{#1}\setcounter{#1}\z@}
    \end{teX}
 \end{macro}


 \subsection{Itemize and Enumerate}

  Enumeration is done with four counters: |enumi|, |enumii|, |enumiii|
  and |enumiv|, where |enum|N controls the numbering of the Nth level
  enumeration.  The label is generated by the commands
  \cs{labelenumi} \ldots{} \cs{labelenumiv}, which should be defined
  by the document style.
  Note that \cs{p@enum}N\cs{theenum}N defines the output
  of a \cs{ref} command.  A typical definition might be:
 \begin{verbatim}
     \def\theenumii{\alph{enumii}}
     \def\p@enumii{\theenumi}
     \def\labelenumii{(\theenumii)}
 \end{verbatim}
 which will print the labels as `(a)', `(b)', \ldots
 and print a \cs{ref} as `3a'.

 The item numbers are moved to the right of the label box, so they are
 always a distance of \cs{labelsep} from the item.

 \cs{@enumdepth} holds the current enumeration nesting depth.

 Itemization is controlled by four commands: \cs{labelitemi},
 \cs{labelitemii},
 \cs{labelitemiii}, and \cs{labelitemiv}.
 To cause the second-level list to be
 bulleted, you just define \cs{labelitemii}
 to be $\bullet$.  \cs{@itemspacing}
 and \cs{@itemdepth} are the analogs of \cs{@enumspacing} and
 \cs{@enumdepth}.

 \begin{teX}
 \enumerate ==
   BEGIN
     if \@enumdepth > 3
       then errormessage: ``Too deeply nested''.
       else \@enumdepth :=L \@enumdepth + 1
            \@enumctr :=L eval(enum@\romannumeral\the\@enumdepth)
            \list{\label(\@enumctr)}
                 {\usecounter{\@enumctr}
                  \makelabel{LABEL} ==  \hss \llap{LABEL}}
     fi
   END

 \endenumerate == \endlist
 \end{teX}

 \begin{macro}{\@enumdepth}
    \begin{teX}
\newcount\@enumdepth \@enumdepth = 0
    \end{teX}
 \end{macro}

 \begin{macro}{\c@enumi}
 \begin{macro}{\c@enumii}
 \begin{macro}{\c@enumii}
 \begin{macro}{\c@enumiv}
    \begin{teX}
\@definecounter{enumi}
\@definecounter{enumii}
\@definecounter{enumiii}
\@definecounter{enumiv}
    \end{teX}
 \end{macro}
 \end{macro}
 \end{macro}
 \end{macro}

 \begin{environment}{enumerate}
  The enumerate environment enumerates the list. The macro
  is written very efficiently and defines the basic structure. The
  typesetting parameters are left for the class files such as book
  to define them.
     \begin{teX}
\def\enumerate{%
  \ifnum \@enumdepth >\thr@@\@toodeep\else
    \advance\@enumdepth\@ne
    \edef\@enumctr{enum\romannumeral\the\@enumdepth}%
      \expandafter
      \list
        \csname label\@enumctr\endcsname
        {\usecounter\@enumctr\def\makelabel##1{\hss\llap{##1}}}%
  \fi}
    \end{teX}

    \begin{teX}
\let\endenumerate =\endlist
    \end{teX}
 \end{environment}


 \begin{teX}
  \itemize ==
    BEGIN
      if \@itemdepth > 3
        then  errormessage: 'Too deeply nested'.
        else \@itemdepth :=L \@itemdepth + 1
             \@itemitem  == eval(labelitem\romannumeral\the\@itemdepth)
             \list{\@nameuse{\@itemitem}}
                   {\makelabel{LABEL} ==  \hss \llap{LABEL}}
      fi
    END

  \enditemize ==  \endlist

 \end{teX}

 \begin{macro}{\@itemdepth}
    \begin{teX}
\newcount\@itemdepth \@itemdepth = 0
    \end{teX}
 \end{macro}

 \begin{environment}{itemize}
     \begin{teX}
\def\itemize{%
  \ifnum \@itemdepth >\thr@@\@toodeep\else
    \advance\@itemdepth\@ne
    \edef\@itemitem{labelitem\romannumeral\the\@itemdepth}%
    \end{teX}
    
 \footnote{changes v1.0j 1995/07/09 Use \textbackslash expandafter. Hard to believe the team missed it!}
    \begin{teX}
    \expandafter
    \list
      \csname\@itemitem\endcsname
      {\def\makelabel##1{\hss\llap{##1}}}%
  \fi}
    \end{teX}

    \begin{teX}
\let\enditemize =\endlist
    \end{teX}
 \end{environment}




%  %% LaTeX2e file `./manual/kernel-N-ltlength.tex'
%% generated by the `filecontents' environment
%% from source `photo-book-class-final-test' on 2011/12/21.
%%
\label{kernel:lengths}
\index{LaTeX kernel classes!File n  ltlength.dtx}
\section{File n, lengths and the ltlength.dtx}

This class defines a number of user coomands for manipulating lengths. the code is straightforward. The |\newlength| command allocates a new internal skip register using the |\newskip| command from the allocations,
class.

\let\bs\textbackslash
\index{\bs newlength}\index{\bs setlength}\index{\bs addtolength}\index{\bs settowidth}\index{\bs settoheight}
\index{\bs settodepth}
\medskip
\begin{tabular}{ll}
\verb+\newlength+  &  Declare \#1 to be a new length command.\\
\verb+\setlength+    &  Set the length command, \#1, to the value \#2.\\
|\addtolength| & Increase the value of the length command, \#1, by the value \#2.\\
|\settowidth|   & Set the length, \#1 to the width of a box containing \#2. \\
|\settoheight|  & Set the length, \#1 to the height of a box containing \#2.\\
|\settodepth|   & Set the length, \#1 to the depth of a box containing \#2.\\
|\@settodim|   & internal macro\\
|\@settopoint| & internal macro\\
\end{tabular}
\medskip

\begin{Code}
3 \def\newlength#1{\@ifdefinable#1{\newskip#1}}
4 \def\setlength#1#2{#1#2\relax}
5 \def\addtolength#1#2{\advance#1 #2\relax}
\end{Code}
\medskip

The |setto| commands use a temporary box to store the contents and then measure them using the internal macro |\@settodim|,

\medskip
\begin{Code}
6 \def\@settodim#1#2#3{\setbox\@tempboxa\hbox{{#3}}#2#1\@tempboxa
%  Clear the memory afterwards (which might be a lot).
7       \setbox\@tempboxa\box\voidb@x}
8 \def\settoheight{\@settodim\ht}
9 \def\settodepth {\@settodim\dp}
10 \def\settowidth {\@settodim\wd}
\end{Code}
\medskip

The |\@settopoint| macro takes the contents of the skip register that is supplied as its argument
and removes the fractional part to make it a whole number of points. This can be
used in class files to avoid values like 45.455pt when calulating a dimension. The method of
rounding is interesting. Also it is interesting that this macro, is not used in the kernel at all, but is defined
here for use with the standard classes (it is used to round off dimensions for page calculations).

\medskip
\begin{Code}
11 \def\@settopoint#1{\divide#1\p@\multiply#1\p@}
\end{Code}
\medskip

Example usage:
\medskip

\begin{Code}
\makeatletter
\newlength\test
\setlength\test{19.5pt}
\@settopoint{\test}

\the\test
\makeatother
\end{Code}

produces: \texttt{19pt}.

%
%	\part{The Standard Classes}
%  \part{The \LaTeX\ standard class}
\parindent0pt
\setlength\columnsep{2em}
\def\Paragraph#1{{\bf #1}\quad}
\cxset{chapter toc=true}
\chapter{The book.cls}
\index{classes>standard}
\index{book>class}

\clearpage

\includegraphics[width=\textwidth]{./graphics/anatomy.jpg}

\vspace{2\baselineskip}

\textbf{\Large DISSECTING THE BOOK CLASS}
\thispagestyle{plain}
\begin{multicols}{2}
This appendix describes the listing of the book class as defined by \latex. It is described here with extra commentary in order to enable you to understand, how it all works.

\lipsum[1-3]
\end{multicols}

\section{General}
\pagestyle{headings}

The book class starts with declaring the version of \latex, required
and naming the class it provides. The class choices are always checked for backward compatibility with the earlier version of \latex. All commands that need to be modified between two column and one column layouts, check the setting and branch accordingly. Another primary choice is if the book is to be printed on both sides or only on one side.


\begin{teX}
\NeedsTeXFormat{LaTeX2e}[1995/12/01]
\ProvidesClass{book}
              [2007/10/19 v1.4h
 Standard LaTeX document class]
\end{teX}

\begin{macro}{\cs{@ptsize}}
This is set to an empty value at start up. The original idea here was by modifying the value one could scale the
text later on. I am not aware of any such usage.
\end{macro}

\begin{teX}
\newcommand\@ptsize{}
\newif\if@restonecol
\newif\if@titlepage \@titlepagetrue
\newif\if@openright
\newif\if@mainmatter \@mainmattertrue
\end{teX}


\Paragraph{Paper size.} After checking for compatibilty with older versions the code branches to define the different standard paper sizes! The options that are declared are, |a4paper|, |a5paper|, |b5paper|, |letterpaper|, |legalpaper| and  |executivepaper|. The class will then later on process the options and set the default to |letterpaper|. \footnote{The package \texttt{geometry}, some classes such as the |Octavo| and |KOMA| classes add additional sizes to cater for other standards.}

\begin{teX}
\if@compatibility\else
\DeclareOption{a4paper}
   {\setlength\paperheight {297mm}%
    \setlength\paperwidth  {210mm}}
\DeclareOption{a5paper}
   {\setlength\paperheight {210mm}%
    \setlength\paperwidth  {148mm}}
\DeclareOption{b5paper}
   {\setlength\paperheight {250mm}%
    \setlength\paperwidth  {176mm}}
\DeclareOption{letterpaper}
   {\setlength\paperheight {11in}%
    \setlength\paperwidth  {8.5in}}
\DeclareOption{legalpaper}
   {\setlength\paperheight {14in}%
    \setlength\paperwidth  {8.5in}}
\DeclareOption{executivepaper}
   {\setlength\paperheight {10.5in}%
    \setlength\paperwidth  {7.25in}}
\end{teX}

\begin{multicols}{2}
\Paragraph{Paper orientation.} the paper orientation is set based on the |landscape| option. If it is declared it stores the |\paperheight| into one of the \latex kernel scratch registers, |\@tempdima| and then reverses the length with the |\paperwidth|.
\end{multicols}

\begin{teX}
\DeclareOption{landscape}
   {\setlength\@tempdima   {\paperheight}%
    \setlength\paperheight {\paperwidth}%
    \setlength\paperwidth  {\@tempdima}}
\fi
\end{teX}

\begin{multicols}{2}
\Paragraph{Font sizing} The class provides three font sizes |10pt|, |11pt| and |12pt|. It default to ten point text.
\end{multicols}

\begin{teX}
\if@compatibility
  \renewcommand\@ptsize{0}
\else
\DeclareOption{10pt}{\renewcommand\@ptsize{0}}
\fi
\DeclareOption{11pt}{\renewcommand\@ptsize{1}}
\DeclareOption{12pt}{\renewcommand\@ptsize{2}}
\end{teX}

\begin{multicols}{2}
\Paragraph{Recto and verso pages.} The class provides the |oneside| and |twoside| options for switching between one side printing or two side printing.  It sets the booleans |\if@twoside| and |\if@mparswitch| accordingly. These booleans are used later on for setting other variables.
\end{multicols}
\begin{teX}
\if@compatibility\else
\DeclareOption{oneside}{\@twosidefalse \@mparswitchfalse}
\fi
\DeclareOption{twoside}{\@twosidetrue  \@mparswitchtrue}
\end{teX}

\begin{multicols}{2}
\Paragraph{Draft and final options.} The options draft and final, just set the |\overfullrule| to either 1pt or 0pt. The |\overfullrule| is a \tex command and simply prints a small vertical line to indicate overfull boxes for the attention of the author. 
\end{multicols}

\begin{teX}
\DeclareOption{draft}{\setlength\overfullrule{5pt}}
\if@compatibility\else
\DeclareOption{final}{\setlength\overfullrule{0pt}}
\fi
\end{teX}

\begin{multicols}{2}
\Paragraph{Title page option.} If the book class, needed such an option is debatable. The |titlepage| option is normally set as true and results in the title being on its own page. The |notitlepage| will omit the page break and display the title on the same page with that of the opening text. Highly unlikely for any author to use it for a book. It is useful for the article class.
\end{multicols}

\begin{teX}
\DeclareOption{titlepage}{\@titlepagetrue}
\if@compatibility\else
\DeclareOption{notitlepage}{\@titlepagefalse}
\fi
\end{teX}

\begin{multicols}{2}
\Paragraph{Display of chapters.} Chapters can be set to start only on an even page or any page. The class provides the options |openright| and |openany|.
\end{multicols}


\begin{teX}
\if@compatibility
\@openrighttrue
\else
\DeclareOption{openright}{\@openrighttrue}
\DeclareOption{openany}{\@openrightfalse}
\fi
\end{teX}


\begin{teX}
\if@compatibility\else
\DeclareOption{onecolumn}{\@twocolumnfalse}
\fi
\DeclareOption{twocolumn}{\@twocolumntrue}
\DeclareOption{leqno}{\input{leqno.clo}}
\DeclareOption{fleqn}{\input{fleqn.clo}}
\DeclareOption{openbib}{%
  \AtEndOfPackage{%
   \renewcommand\@openbib@code{%
      \advance\leftmargin\bibindent
      \itemindent -\bibindent
      \listparindent \itemindent
      \parsep \z@
      }%
   \renewcommand\newblock{\par}}%
}
\end{teX}

We now execute the options and process them.
\begin{teX}
\ExecuteOptions{letterpaper,10pt,twoside,onecolumn,final,openright}
\ProcessOptions
\end{teX}

\begin{multicols}{2}
\textbf{The .clo files}\quad The book class now inputs the file .clo etc that defines the fontsizes
 for anything specific to the 10pt. These files hold quite a bit of information and size related commands for the
standard sizes provided by \latex. The |.clo| files also set many other parameters for page sizing, lists, paper sectioning, such as margins, marginpars and the like.
\end{multicols}

\begin{teX}
\input{bk1\@ptsize.clo}
\setlength\lineskip{1\p@}
\setlength\normallineskip{1\p@}
\renewcommand\baselinestretch{}
\setlength\parskip{0\p@ \@plus \p@}
\end{teX}

\begin{multicols}{2}
\Paragraph{Penalties.} Here the following penalties are set.
\end{multicols}

\begin{teX}
\@lowpenalty   51
\@medpenalty  151
\@highpenalty 301
\end{teX}

\begin{multicols}{2}
\textbf{Float control parameters.}\quad The allowable number of floats on a page are controlled by a number of parameters. These are set here. Many users overwrite these parameters in order to have more control on the placement of floats.
\end{multicols}


\begin{teX}
\setcounter{topnumber}{2}
\renewcommand\topfraction{.7}
\setcounter{bottomnumber}{1}
\renewcommand\bottomfraction{.3}
\setcounter{totalnumber}{3}
\renewcommand\textfraction{.2}
\renewcommand\floatpagefraction{.5}
\setcounter{dbltopnumber}{2}
\renewcommand\dbltopfraction{.7}
\renewcommand\dblfloatpagefraction{.5}
\end{teX}

\begin{multicols}{2}
\textbf{Running head and foot.}\quad A page header or simply header in typography is text which is separated from the main body of text and appears at the top of a printed page. Word processing programs usually provide for the creation and maintenance of page headers, which are often the same from page to page, with merely small differences in information, such as page number.

In publishing, the page header (or ``pagehead'') is often referred to as the running head. Typical running heads in a book might consist of the book title on the left-hand (verso) page, and the chapter title on the right-hand (recto) page, or chapter title on the verso and subsection title on the recto.
\end{multicols}


\begin{teX}
\if@twoside
  \def\ps@headings{%
      \let\@oddfoot\@empty\let\@evenfoot\@empty
      \def\@evenhead{\thepage\hfil\slshape\leftmark}%
      \def\@oddhead{{\slshape\rightmark}\hfil\thepage}%
      \let\@mkboth\markboth
 % chapter
  \def\chaptermark##1{%
      \markboth {\MakeUppercase{%
        \ifnum \c@secnumdepth >\m@ne
          \if@mainmatter
            \@chapapp\ \thechapter. \ %
          \fi
        \fi
        ##1}}{}}%
% section
    \def\sectionmark##1{%
      \markright {\MakeUppercase{%
        \ifnum \c@secnumdepth >\z@
          \thesection. \ %
        \fi
        ##1}}}}
\else
  \def\ps@headings{%
    \let\@oddfoot\@empty
    \def\@oddhead{{\slshape\rightmark}\hfil\thepage}%
    \let\@mkboth\markboth
    \def\chaptermark##1{%
      \markright {\MakeUppercase{%
        \ifnum \c@secnumdepth >\m@ne
          \if@mainmatter
            \@chapapp\ \thechapter. \ %
          \fi
        \fi
        ##1}}}}
\fi
\def\ps@myheadings{%
    \let\@oddfoot\@empty\let\@evenfoot\@empty
    \def\@evenhead{\thepage\hfil\slshape\leftmark}%
    \def\@oddhead{{\slshape\rightmark}\hfil\thepage}%
    \let\@mkboth\@gobbletwo
    \let\chaptermark\@gobble
    \let\sectionmark\@gobble
    }
\end{teX}

\begin{multicols}{2}\index{headings!plain}
Please note the \textit{plain} headings are not defined in the class. These are defined in the \latex kernel\footnote{See \texttt{File J: ltpage.dtx}, page 312.}

\begin{teX}
\ps@plain The plain page style: No head, centred page number in foot.
13 \def\ps@plain{\let\@mkboth\@gobbletwo
14 \let\@oddhead\@empty\def\@oddfoot{\reset@font\hfil\thepage
15 \hfil}\let\@evenhead\@empty\let\@evenfoot\@oddfoot}
\end{teX}



\textbf{Title pages.}\quad Title pages are defined between a conditional, that handle the option |titlepage|
and. The commands just take mostly of the typography. If you use the option |notitlepage| in the book class, the title will be similar for all practical purposes to that of an |article| and it will appear on the top of the first page.

The |\maketitle| sets the |\footnotesise|, the |\footnoterule| and the |\footnote|.
\end{multicols}

\begin{teX}
 \if@titlepage
  \newcommand\maketitle{\begin{titlepage}%
  \let\footnotesize\small
  \let\footnoterule\relax
  \let \footnote \thanks
  \null\vfil
  \vskip 60\p@
  \begin{center}%
    {\LARGE \@title \par}%
    \vskip 3em%
    {\large
     \lineskip .75em%
      \begin{tabular}[t]{c}%
        \@author
      \end{tabular}\par}%
      \vskip 1.5em%
    {\large \@date \par}%       % Set date in \large size.
  \end{center}\par
  \@thanks
  \vfil\null
  \end{titlepage}%
  \setcounter{footnote}{0}%
  \global\let\thanks\relax
  \global\let\maketitle\relax
  \global\let\@thanks\@empty
  \global\let\@author\@empty
  \global\let\@date\@empty
  \global\let\@title\@empty
  \global\let\title\relax
  \global\let\author\relax
  \global\let\date\relax
  \global\let\and\relax
}
\else
\newcommand\maketitle{\par
  \begingroup
    \renewcommand\thefootnote{\@fnsymbol\c@footnote}%
    \def\@makefnmark{\rlap{\@textsuperscript{\normalfont\@thefnmark}}}%
    \long\def\@makefntext##1{\parindent 1em\noindent
            \hb@xt@1.8em{%
                \hss\@textsuperscript{\normalfont\@thefnmark}}##1}%
    \if@twocolumn
      \ifnum \col@number=\@ne
        \@maketitle
      \else
        \twocolumn[\@maketitle]%
      \fi
    \else
      \newpage
      \global\@topnum\z@   % Prevents figures from going at top of page.
      \@maketitle
    \fi
    \thispagestyle{plain}\@thanks
  \endgroup
  \setcounter{footnote}{0}%
  \global\let\thanks\relax
  \global\let\maketitle\relax
  \global\let\@maketitle\relax
  \global\let\@thanks\@empty
  \global\let\@author\@empty
  \global\let\@date\@empty
  \global\let\@title\@empty
  \global\let\title\relax
  \global\let\author\relax
  \global\let\date\relax
  \global\let\and\relax
}
\def\@maketitle{%
  \newpage
  \null
  \vskip 2em%
  \begin{center}%
  \let \footnote \thanks
    {\LARGE \@title \par}%
    \vskip 1.5em%
    {\large
      \lineskip .5em%
      \begin{tabular}[t]{c}%
        \@author
      \end{tabular}\par}%
    \vskip 1em%
    {\large \@date}%
  \end{center}%
  \par
  \vskip 1.5em}
\fi
\end{teX}


\Paragraph{Section counters.}\quad 
In LaTeX all defaults all document section are numbered by default. These numbers are kept in counters, named after the section name. A series of commands are provided to access these numbers.
All the counters are in arabic numerals, with the exception of "part", which is in Roman.


\begin{teX}
\newcommand*\chaptermark[1]{}
\setcounter{secnumdepth}{2}
\newcounter {part}
\newcounter {chapter}
\newcounter {section}[chapter]
\newcounter {subsection}[section]
\newcounter {subsubsection}[subsection]
\newcounter {paragraph}[subsubsection]
\newcounter {subparagraph}[paragraph]
\renewcommand \thepart {\@Roman\c@part}
\renewcommand \thechapter {\@arabic\c@chapter}
\renewcommand \thesection {\thechapter.\@arabic\c@section}
\renewcommand\thesubsection   {\thesection.\@arabic\c@subsection}
\renewcommand\thesubsubsection{\thesubsection.\@arabic\c@subsubsection}
\renewcommand\theparagraph    {\thesubsubsection.\@arabic\c@paragraph}
\renewcommand\thesubparagraph {\theparagraph.\@arabic\c@subparagraph}
\newcommand\@chapapp{\chaptername}
\end{teX}

\begin{multicols}{2}
\textbf{Frontmatter, mainmatter and backmatter.} These are author command to set mostly, the page numbering and the clearing of pages for two page layouts. Front matter has lower roman pages numbering and the main matter has arabic numerals.
\end{multicols}

\emphasis{frontmatter,mainmatter,backmatter}
\begin{teXXX}
\newcommand\frontmatter{%
    \cleardoublepage
  \@mainmatterfalse
  \pagenumbering{roman}}

\newcommand\mainmatter{%
    \cleardoublepage
  \@mainmattertrue
  \pagenumbering{arabic}}

\newcommand\backmatter{%
  \if@openright
    \cleardoublepage
  \else
    \clearpage
  \fi
  \@mainmatterfalse}
\end{teXXX}

\begin{multicols}{2}
\Paragraph{Part.}The is the definition of part. The Part is displayed with a plain header and the it goes into the secdef. If the section depth is greater or equal -2, the start counter is increased and the part is added to the toc, using |\addcontentsline|. 
The partname i.e., default 'Part' gets printer either way except for the star version of the command.
\end{multicols}

\emphasis{@part,@spart,secdef}
\begin{teXXX}
\newcommand\part{%
  \if@openright
    \cleardoublepage
  \else
    \clearpage
  \fi
  \thispagestyle{plain}%
  \if@twocolumn
    \onecolumn
    \@tempswatrue
  \else
    \@tempswafalse
  \fi
  \null\vfil
  \secdef\@part\@spart}
\end{teXXX}

\begin{multicols}{2}
 The important command to remember here
is |\secdef|. This is defined in the kernel and not in the classes \texttt{ltsect.dtx}. Essentially in the code |\@part| calls the unstar command and the @spart calls the starred command. We copy the definition from the kernel for convenience.
\end{multicols}

\begin{teXXX}
is \secdef{unstarcmds}{unstarcmds}{starcmds}
When defining a \chapter or \section command without using \@startsection,
you can use \secdef as follows:
1. \def\chapter{ . . . \secdef \starcmd \unstarcmd}
2. \def\hstarcmdi[#1]#2{ . . . } % Command to define \chapter[. . . ]{. . . }
3. \def\unstarcmd#1{ . . . } % Command to define \chapter*{. . . }
125 \def\secdef#1#2{\@ifstar{#2}{\@dblarg{#1}}}
\end{teXXX}

The |@part| starts now,

\begin{teXXX}
\def\@part[#1]#2{%
    \ifnum \c@secnumdepth >-2\relax
      \refstepcounter{part}%
      \addcontentsline{toc}{part}{\thepart\hspace{1em}#1}%
    \else
      \addcontentsline{toc}{part}{#1}%
    \fi
    \markboth{}{}%
    {\centering
     \interlinepenalty \@M
     \normalfont
     \ifnum \c@secnumdepth >-2\relax
       \huge\bfseries \partname\nobreakspace\thepart
       \par
       \vskip 20\p@
     \fi
     \Huge \bfseries #2\par}%
    \@endpart}
\end{teXXX}

\begin{multicols}{2}
The starred version of the command is provided next. The difference the name `Part'' is not displayed. However the parameter provided by the user is displayed. A normal font is provided. Final settings depending on @openright and header styles are set and the code macro is completed.
\end{multicols}

\begin{teXXX}
\def\@spart#1{%
    {\centering
     \interlinepenalty \@M
     \normalfont
     \Huge \bfseries #1\par}%
    \@endpart}

\def\@endpart{\vfil\newpage
              \if@twoside
               \if@openright
                \null
                \thispagestyle{empty}%
                \newpage
               \fi
              \fi
              \if@tempswa
                \twocolumn
              \fi}
\end{teXXX}

\begin{multicols}{2}
\Paragraph{Chapter}. The chapter definition follows, the same pattern as that of the part definitions. It calls secdef and defines commands for the starred and unstarred versions.
\end{multicols}

\begin{teXXX}
\newcommand\chapter{\if@openright\cleardoublepage\else\clearpage\fi
                    \thispagestyle{plain}%
                    \global\@topnum\z@
                    \@afterindentfalse
                    \secdef\@chapter\@schapter}
\end{teXXX}

\Paragraph{Unstarred version}

\begin{teXXX}
\def\@chapter[#1]#2{\ifnum \c@secnumdepth >\m@ne
                       \if@mainmatter
                         \refstepcounter{chapter}%
                         \typeout{\@chapapp\space\thechapter.}%
                         \addcontentsline{toc}{chapter}%
                                   {\protect\numberline{\thechapter}#1}%
                       \else
                         \addcontentsline{toc}{chapter}{#1}%
                       \fi
                    \else
                      \addcontentsline{toc}{chapter}{#1}%
                    \fi
                    \chaptermark{#1}%
                    \addtocontents{lof}{\protect\addvspace{10\p@}}%
                    \addtocontents{lot}{\protect\addvspace{10\p@}}%
                    \if@twocolumn
                      \@topnewpage[\@makechapterhead{#2}]%
                    \else
                      \@makechapterhead{#2}%
                      \@afterheading
                    \fi}
\end{teXXX}

\begin{multicols}{2}
\Paragraph{Defining the looks of the Chapter heading.}

Good practice dictates, that when you change the chapterhead layout for the numbered version, you also change it for the star version of the command. You can do that by using two different macros, although at first glance it might be difficult to see where the difference is.
\end{multicols}


\emphasis{@makeschapterhead,@makechapterhead}
\begin{teXXX}
\def\@makechapterhead
\def\@makeschapterhead
\end{teXXX}

\begin{teXXX}
\def\@makechapterhead#1{%
  \vspace*{50\p@}%
  {\parindent \z@ \raggedright \normalfont
    \ifnum \c@secnumdepth >\m@ne
      \if@mainmatter
        \huge\bfseries \@chapapp\space \thechapter
        \par\nobreak
        \vskip 20\p@
      \fi
    \fi
    \interlinepenalty\@M
    \Huge \bfseries #1\par\nobreak
    \vskip 40\p@
  }}
\end{teXXX}

\begin{multicols}{2}
Finally the starred version of the command is called. This now checks for twocolumn or one column via an if sttaement and executes, the makeschapterhead. Another mysterious and wonderful command appears again from the LaTeX source2e.\cmd{\@afterheading}. This command 
is just a hook for custom headings? (Needs to be reviewed again).
\end{multicols}

\begin{teXXX}
\def\@schapter#1{\if@twocolumn
                   \@topnewpage[\@makeschapterhead{#1}]%
                 \else
                   \@makeschapterhead{#1}%
                   \@afterheading
                 \fi}
\end{teXXX}

And finally the |\@makeschapterhead| (remember \textbf{s} for \textbf{s}tar).

\begin{teXXX}
\def\@makeschapterhead#1{%
  \vspace*{50\p@}%
  {\parindent \z@ \raggedright
    \normalfont
    \interlinepenalty\@M
    \Huge \bfseries  #1\par\nobreak
    \vskip 40\p@
  }}
\end{teXXX}

All sorts of variations of the above two commands can be found in different classes, such as |KOMA|, |memoir| and others. The example which follows, typesets the headings as shown in \fref{fig:chapterhead-17}. The |@makechapterhead| command is modified to produce a centered heading which is displayed between two heavy rules. This style can be found in quite a number of books.

\begin{figure*}[htbp]
\includegraphics[width=\linewidth]{./graphics/chapterhead-17.png}
\caption{Modifying the way the chapterhead looks can be achieved by redefining the \texttt{\textbackslash @makechapterhead} and \texttt{\textbackslash @makeschapterhead} commands.}
\label{fig:chapterhead-17}
\end{figure*}

\section*{Full working example}

\begin{teX}
\documentclass[oneside]{book}
\usepackage[english]{babel}
\usepackage{lipsum}
\makeatletter
\def\thickhrule{\leavevmode \leaders \hrule height 1ex \hfill \kern \z@}

%% Note the difference between the commands the one is 
%% make and the other one is makes
\renewcommand{\@makechapterhead}[1]{%
  \vspace*{10\p@}%
  {\parindent \z@ \centering \reset@font
        {\Huge \scshape  \thechapter }
        \par\nobreak
        \vspace*{10\p@}%
        \interlinepenalty\@M
        \thickhrule
        \par\nobreak
        \vspace*{2\p@}%
        {\Huge \bfseries #1\par\nobreak}
        \par\nobreak
        \vspace*{2\p@}%
        \thickhrule
    \vskip 40\p@
    \vskip 100\p@
  }}

%% This is makes
\def\@makeschapterhead#1{%
  \vspace*{10\p@}%
  {\parindent \z@ \centering \reset@font
        {\Huge \scshape \vphantom{\thechapter}}
        \par\nobreak
        \vspace*{10\p@}%
        \interlinepenalty\@M
        \thickhrule
        \par\nobreak
        \vspace*{2\p@}%
        {\Huge \bfseries #1\par\nobreak}
        \par\nobreak
        \vspace*{2\p@}%
        \thickhrule
    \vskip 100\p@
  }}
\begin{document}
\chapter{The Real Numbers}
\lipsum[1-2]
\chapter*{The Imaginary Numbers}
\lipsum[1-2]
\end{document}
\end{teX}

\begin{multicols}{2}
\Paragraph{The sections.}
In this section, all the document elements besides the Chapter and the Part are Defined. They use the mother of all commands from the kernel
ltsection.dtx, named |\@startsection|. This is just a call to the kernel command. No other settings are done here. In order to remember what it does we refer to its definition in the kernel. Of interest is the sixth argument which sets the font style.

The parameter takes eight parameters, some of them optional. We discuss this command in more detail in the kernel chapter.

\end{multicols}


\begin{teX}
\newcommand\section{\@startsection {section}{1}{\z@}%
                                   {-3.5ex \@plus -1ex \@minus -.2ex}%
                                   {2.3ex \@plus.2ex}%
                                   {\normalfont\Large\bfseries}}
\newcommand\subsection{\@startsection{subsection}{2}{\z@}%
                                     {-3.25ex\@plus -1ex \@minus -.2ex}%
                                     {1.5ex \@plus .2ex}%
                                     {\normalfont\large\bfseries}}
\newcommand\subsubsection{\@startsection{subsubsection}{3}{\z@}%
                                     {-3.25ex\@plus -1ex \@minus -.2ex}%
                                     {1.5ex \@plus .2ex}%
                                     {\normalfont\normalsize\bfseries}}
\newcommand\paragraph{\@startsection{paragraph}{4}{\z@}%
                                    {3.25ex \@plus1ex \@minus.2ex}%
                                    {-1em}%
                                    {\normalfont\normalsize\bfseries}}
\newcommand\subparagraph{\@startsection{subparagraph}{5}{\parindent}%
                                       {3.25ex \@plus1ex \@minus .2ex}%
                                       {-1em}%
                                      {\normalfont\normalsize\bfseries}}
\if@twocolumn
  \setlength\leftmargini  {2em}
\else
  \setlength\leftmargini  {2.5em}
\fi
\leftmargin  \leftmargini
\setlength\leftmarginii  {2.2em}
\setlength\leftmarginiii {1.87em}
\setlength\leftmarginiv  {1.7em}
\if@twocolumn
  \setlength\leftmarginv  {.5em}
  \setlength\leftmarginvi {.5em}
\else
  \setlength\leftmarginv  {1em}
  \setlength\leftmarginvi {1em}
\fi
\setlength  \labelsep  {.5em}
\setlength  \labelwidth{\leftmargini}
\addtolength\labelwidth{-\labelsep}
\@beginparpenalty -\@lowpenalty
\@endparpenalty   -\@lowpenalty
\@itempenalty     -\@lowpenalty
\renewcommand\theenumi{\@arabic\c@enumi}
\renewcommand\theenumii{\@alph\c@enumii}
\renewcommand\theenumiii{\@roman\c@enumiii}
\renewcommand\theenumiv{\@Alph\c@enumiv}
\newcommand\labelenumi{\theenumi.}
\newcommand\labelenumii{(\theenumii)}
\newcommand\labelenumiii{\theenumiii.}
\newcommand\labelenumiv{\theenumiv.}
\renewcommand\p@enumii{\theenumi}
\renewcommand\p@enumiii{\theenumi(\theenumii)}
\renewcommand\p@enumiv{\p@enumiii\theenumiii}
\newcommand\labelitemi{\textbullet}
\newcommand\labelitemii{\normalfont\bfseries \textendash}
\newcommand\labelitemiii{\textasteriskcentered}
\newcommand\labelitemiv{\textperiodcentered}
\newenvironment{description}
               {\list{}{\labelwidth\z@ \itemindent-\leftmargin
                        \let\makelabel\descriptionlabel}}
               {\endlist}
\newcommand*\descriptionlabel[1]{\hspace\labelsep
                                \normalfont\bfseries #1}
\end{teX}

\begin{multicols}{2}
\textbf{The verse environment}\quad \latex's verse environment, can only serve for the incidental use of a few stanzas. It leaves most of the formatting to the author.  It redefines the line break |\\| to a |\centercr|.
\end{multicols}

\begin{teX}
\newenvironment{verse}
    {\let\\\@centercr(*@\protect\footnote{This is defined in ltmiscen.dtx}@*)
     \list{}{\itemsep  \z@
             \itemindent   -1.5em%
             \listparindent\itemindent
             \rightmargin  \leftmargin
             \advance\leftmargin 1.5em}%
       \item\relax}
    {\endlist}
\end{teX}
\makeatletter
\newenvironment{Verse}
    {\let\\\@centercr%
     \list{}{\itemsep1pt
             \itemindent-1.5em%
             \listparindent\itemindent
             \rightmargin\leftmargin
             \advance\leftmargin 1.5em}%
       \item\relax}
    {\endlist}
\makeatother
\begin{teX}
  \begin{Verse}
     My mobile test\\
     this is other\\
     this is last\\
  \end{Verse}
\end{teX}

The environment doesn't really do much, the way I see it but just move the poem a couple of ems inwards 
to much the definition of lists. Most people will want more from a poem environment.
\begin{Verse}
     My mobile test\\
      this is other\\
       this is last\\
\end{Verse}

The simplest thing we can add to this environment if we want to modify it, is a hook. This we can do using the |blckcntrl| package. \sidenote{From the \url{http://www.ifi.uio.no/it/latex-links/blkcntrl.pdf}}.

\begin{teX}
\renewenvironment{verse}
50 {\let\\\@centercr
51 \relax\list{}{\setlength{\itemsep}{\z@}%
52 \setlength{\itemindent}{-1.5em}%
53 \setlength{\listparindent}{\itemindent}%
54 \setlength{\rightmargin}{\leftmargin}%
55 \addtolength{\leftmargin}{1.5em}}%
56 \item\relax\PreVerse\relax}
57 {\endlist}
\end{teX}

Using the command |\PreVerse|, we can add a block at the beginning of the block. For example some code to make a poem title and insert it later on. The setting of the rightmargin to the leftmargin here is curious. It might for example give us problems with |tufte-latex| classes.

\begin{multicols}{2}
\textbf{The quote and quotation environments.}\quad The environments |quote| and |quotation| are defined next. Again they are defined using the general |\list| environment. Again the general |\list|, is used in the definition. The |listparindent| is set to 1.5 em.
\end{multicols}

\begin{teX}
\newenvironment{quotation}
               {\list{}{\listparindent 1.5em%
                        \itemindent\listparindent
                        \rightmargin\leftmargin
                        \parsep\z@ \@plus\p@}%
                \item\relax}
               {\endlist}
\end{teX}

\begin{teX}
\newenvironment{quote}
               {\list{}{\rightmargin\leftmargin}%
                \item\relax}
               {\endlist}




\section{The \protect\texttt{titlepage} environment}
\if@compatibility
\newenvironment{titlepage}
    {%
      \cleardoublepage
      \if@twocolumn
        \@restonecoltrue\onecolumn
      \else
        \@restonecolfalse\newpage
      \fi
      \thispagestyle{empty}%
      \setcounter{page}\z@
    }%
    {\if@restonecol\twocolumn \else \newpage \fi
    }
\else
\newenvironment{titlepage}
    {%
      \cleardoublepage
      \if@twocolumn
        \@restonecoltrue\onecolumn
      \else
        \@restonecolfalse\newpage
      \fi
      \thispagestyle{empty}%
      \setcounter{page}\@ne
    }%
    {\if@restonecol\twocolumn \else \newpage \fi
     \if@twoside\else
        \setcounter{page}\@ne
     \fi
    }
\fi
\end{teX}



\begin{multicols}{2}
\includegraphics[width=\linewidth]{./graphics/appendix.png}

\Paragraph{The Appendix.}
Similarly to the chapter sectioning commands, the Appendix is not defined as a section. It simply sets the chapter and section counters to zero and sets the name of the section. All the relevant counters and uses letters for the numbering of the following chapters etc. If you closely follow the code, it is all based on the chapter command, except that it defaults to Alphanumeric counting.

\begin{teX}
\newcommand\appendix{\par
  \setcounter{chapter}{0}%
  \setcounter{section}{0}%
  \gdef\@chapapp{\appendixname}(*@\sidenote{The actual literal used for \textbackslash{appendixname} is defined later on, so that you can customize the language}\label{appendixname}@*)
  \gdef\thechapter{\@Alph\c@chapter}}
\end{teX}

An Appendix page has the same looks and feel to that of a Chapter. For all practical purposes, it is a chapter, with different labels and roman numbering.
\end{multicols}

\begin{multicols}{2}
\Paragraph{General Settings.} Here, some general settings are set. These include settings for framed boxes, tabbing separators and array column separators.

\end{multicols}
\begin{teX}
\setlength\arraycolsep{5\p@}
\setlength\tabcolsep{6\p@}
\setlength\arrayrulewidth{.4\p@}
\setlength\doublerulesep{2\p@}
\setlength\tabbingsep{\labelsep}
\skip\@mpfootins = \skip\footins
\setlength\fboxsep{3\p@}
\setlength\fboxrule{.4\p@}
\end{teX}

\begin{multicols}{2}
\Paragraph{Equation numbering}
The equation counter is reset according to the chapter counter, using the \latex kernel command |\@addtoreset|. 

\end{multicols}
\begin{teX}
\@addtoreset {equation}{chapter}
\renewcommand\theequation
  {\ifnum \c@chapter>\z@ \thechapter.\fi \@arabic\c@equation}
\end{teX}

\section*{FIGURE AND TABLE ENVIRONMENTS}

\begin{multicols}{2}
\Paragraph{Figure Environment} The figure environment is defined using commands that have been provided by the kernel.  The command |\thefigure| is first redefined to display the combination of the chapter dot figure counter, all in arabic numerals. The extension for the list of figures and finally the floats for single column and double column.

\end{multicols}

\label{book:figure}
\begin{teX}
\newcounter{figure}[chapter]
\renewcommand \thefigure
     {\ifnum \c@chapter>\z@ \thechapter.\fi \@arabic\c@figure}
\def\fps@figure{tbp}
\def\ftype@figure{1}
\def\ext@figure{lof}
\def\fnum@figure{\figurename\nobreakspace\thefigure}
\newenvironment{figure}
               {\@float{figure}}
               {\end@float}
\newenvironment{figure*}
               {\@dblfloat{figure}}
               {\end@dblfloat}
\end{teX}

\begin{multicols}{2}
\Paragraph{Table Environment} Table floats are defined the same way like the figures with their respective counters and names.  
\end{multicols}

\begin{teX}
\newcounter{table}[chapter]
\renewcommand \thetable
     {\ifnum \c@chapter>\z@ \thechapter.\fi \@arabic\c@table}
\def\fps@table{tbp}
\def\ftype@table{2}
\def\ext@table{lot}
\def\fnum@table{\tablename\nobreakspace\thetable}


\newenvironment{table}
               {\@float{table}}
               {\end@float}

\newenvironment{table*}
               {\@dblfloat{table}}
               {\end@dblfloat}
\end{teX}

\begin{multicols}{2}
\Paragraph{Captions}
The captioning macros are rather short but need a bit of explanation. First
some lengths are defined. The lengths are for |abovecaptionskip| and |belowcaptionskip| are set equal to a default of 10pt as for the font-size, but the length |belowcaptionskip| is set to |opt|.
\end{multicols}

\begin{teX}
\newlength\abovecaptionskip
\newlength\belowcaptionskip
\setlength\abovecaptionskip{10\p@}
\setlength\belowcaptionskip{0\p@}

\long\def\@makecaption#1#2{%
  \vskip\abovecaptionskip
  \sbox\@tempboxa{#1: #2}
  \ifdim \wd\@tempboxa >\hsize
    #1: #2\par
  \else
    \global \@minipagefalse
    \hb@xt@\hsize{\hfil\box\@tempboxa\hfil}%
  \fi
  \vskip\belowcaptionskip}
\end{teX}

\begin{multicols}{2}
The |@makecaption| macro is also interesting. Firstly note in line  the use of a colon (:). So if you do not like to have this you know where you need to go and change it. The contents of the caption are first saved into a box. If the box is greater than |hsize| then they are written like a paragraph otherwise, they are centered. Note that the centering is done using |\hfil\box\@tempoxa\hfil|. The mysterious command |\hb@xt| is defined in the kernel and
is equivalent to |\hbox to|
\end{multicols}

\begin{teXXX}
  \hb@xt@ The next one is another 100 tokens worth.
  16 \def\hb@xt@{\hbox to}
\end{teXXX}

\begin{multicols}{2}
It is simply an abbreviation of |\hbox to|. There are many short-cut commands like this, so the command just again sets the caption in a  horizontal box. There is more to the story later on. 
\end{multicols}


\section*{Defining the old style font commands}

\begin{teX}
\DeclareOldFontCommand{\rm}{\normalfont\rmfamily}{\mathrm}
\DeclareOldFontCommand{\sf}{\normalfont\sffamily}{\mathsf}
\DeclareOldFontCommand{\tt}{\normalfont\ttfamily}{\mathtt}
\DeclareOldFontCommand{\bf}{\normalfont\bfseries}{\mathbf}
\DeclareOldFontCommand{\it}{\normalfont\itshape}{\mathit}
\DeclareOldFontCommand{\sl}{\normalfont\slshape}{\@nomath\sl}
\DeclareOldFontCommand{\sc}{\normalfont\scshape}{\@nomath\sc}
\DeclareRobustCommand*\cal{\@fontswitch\relax\mathcal}
\DeclareRobustCommand*\mit{\@fontswitch\relax\mathnormal}
\end{teX}


\section*{Table of contents}

\begin{multicols}{2}
Firstly we define the width of the box that the page number is set. Use ems so that it does not need to be redefined for every change in font size.
Toc entries are treated as rectangular areas where the text
and probably a filler will be written. Let's draw such an
area (of course, the lines themselves are not printed):
\end{multicols}


\setlength{\unitlength}{1cm}
\begin{center}
\begin{picture}(8,2.2)
\put(1,1){\line(1,0){6}}
\put(1,2){\line(1,0){6}}
\put(1,1){\line(0,1){1}}
\put(7,1){\line(0,1){1}}
\put(0,.7){\vector(1,0){1}}
\put(8,.7){\vector(-1,0){1}}
\put(0,.2){\makebox(1,.5)[b]{\textit{left}}}
\put(7,.2){\makebox(1,.5)[b]{\textit{right}}}
\end{picture}
\end{center}

The space between the left page margin and the left edge of
the area will be named |<left>|; similarly we have |<right>|.
You are allowed to modify the beginning of the first line and
the ending of the last line. For example by ``taking up'' both
places with |\hspace*{2pc}| the area becomes:
\begin{center}
\begin{picture}(8,2.2)
\put(1,1){\line(1,0){5.5}}
\put(6.5,1){\line(0,1){.5}}
\put(6.5,1.5){\line(1,0){.5}}
\put(1.5,2){\line(1,0){5.5}}
\put(1,1.5){\line(1,0){.5}}
\put(1.5,1.5){\line(0,1){.5}}
\put(1,1){\line(0,1){.5}}
\put(7,1.5){\line(0,1){.5}}
\put(0,.7){\vector(1,0){1}}
\put(8,.7){\vector(-1,0){1}}
\put(0,.2){\makebox(1,.5)[b]{\textit{left}}}
\put(7,.2){\makebox(1,.5)[b]{\textit{right}}}
\end{picture}
\end{center}
And by ``clearing'' space in both places with |\hspace*{-2pc}|
the area becomes:
\begin{center}
\begin{picture}(8,2.2)
\put(1,1){\line(1,0){6.5}}
\put(7.5,1){\line(0,1){.5}}
\put(7.5,1.5){\line(-1,0){.5}}
\put(.5,2){\line(1,0){6.5}}
\put(1,1.5){\line(-1,0){.5}}
\put(.5,1.5){\line(0,1){.5}}
\put(1,1){\line(0,1){.5}}
\put(7,1.5){\line(0,1){.5}}
\put(0,.7){\vector(1,0){1}}
\put(8,.7){\vector(-1,0){1}}
\put(0,.2){\makebox(1,.5)[b]{\textit{left}}}
\put(7,.2){\makebox(1,.5)[b]{\textit{right}}}
\end{picture}
\end{center}

\begin{multicols}{2}
If you have seen tocs, the latter should be familiar to you--
the label at the very beginning, the page at the very end:
\columnbreak

\topline
\begin{verbatim}
    3.2  This is an example showing that toc
         entries fits in that scheme . . . .   4
\end{verbatim}
\bottomline
\end{multicols}


\begin{teX}
\newcommand\@pnumwidth{1.55em}%Width of box in which page number is set.
\end{teX}

We then define the margin and the dotsep. We also set the toc counter to whatever is require (don't go too deep especially if you have an index).

\begin{teX}
\newcommand\@tocrmarg{2.55em}%Right margin indentation for all but last line of multiple-line entries.
\newcommand\@dotsep{4.5}%Separation between dots, in mu units. Should be \def'd to a number like
2 or 1.7
\end{teX}

\begin{macro}{\tableofcontents}
\Paragraph{Defining the  contents table.} The author is provided with the author command |\tableofcontents|. All format information is provided at this point.
\end{macro}

\begin{teX}
\setcounter{tocdepth}{2}
\newcommand\tableofcontents{%
    \if@twocolumn
      \@restonecoltrue\onecolumn
    \else
      \@restonecolfalse
    \fi
    \chapter*{\contentsname
        \@mkboth{%
           \MakeUppercase\contentsname}{\MakeUppercase\contentsname}}%
    \@starttoc{toc}%
    \if@restonecol\twocolumn\fi
    }
\end{teX}

\begin{teX}
\newcommand*\l@part[2]{%
  \ifnum \c@tocdepth >-2\relax
    \addpenalty{-\@highpenalty}%
    \addvspace{2.25em \@plus\p@}%
    \setlength\@tempdima{3em}%
    \begingroup
      \parindent \z@ \rightskip \@pnumwidth
      \parfillskip -\@pnumwidth
      {\leavevmode
       \large \bfseries #1\hfil \hb@xt@\@pnumwidth{\hss #2}}\par
       \nobreak
         \global\@nobreaktrue
         \everypar{\global\@nobreakfalse\everypar{}}%
    \endgroup
  \fi}

\newcommand*\l@chapter[2]{%
  \ifnum \c@tocdepth >\m@ne
    \addpenalty{-\@highpenalty}%
    \vskip 1.0em \@plus\p@
    \setlength\@tempdima{1.5em}%
    \begingroup
      \parindent \z@ \rightskip \@pnumwidth
      \parfillskip -\@pnumwidth
      \leavevmode \bfseries
      \advance\leftskip\@tempdima
      \hskip -\leftskip
      #1\nobreak\hfil \nobreak\hb@xt@\@pnumwidth{\hss #2}\par
      \penalty\@highpenalty
    \endgroup
  \fi}
\end{teX}


The five remaining levels (entry in \latex terminology, are defined next). This is done with the general \latex kernel command 

\begin{teX}
\@dottedtocline{<level>}{<indent>}{<numwidth>}{<title>}{<page>}: Macro
to produce a table of contents line with the following parameters:
\end{teX}

The commands for the remaining sections are defined as follows:

\begin{teX}
\newcommand*\l@section{\@dottedtocline{1}{1.5em}{2.3em}}
\newcommand*\l@subsection{\@dottedtocline{2}{3.8em}{3.2em}}
\newcommand*\l@subsubsection{\@dottedtocline{3}{7.0em}{4.1em}}
\newcommand*\l@paragraph{\@dottedtocline{4}{10em}{5em}}
\newcommand*\l@subparagraph{\@dottedtocline{5}{12em}{6em}}
\end{teX}

So where are the last two parameters? These are just zeroed here!


I can assure that the |dotted| type of section bothers a lot of people. Most new books will both compact the table of contents as well as remove the dots. You can use the |titlesec| and |titletoc| to do this rather than redefining the kernel commands or the standard classes styles.

\subsection*{List of figures, tables etc}
\begin{teX}
\newcommand\listoffigures{%
    \if@twocolumn
      \@restonecoltrue\onecolumn
    \else
      \@restonecolfalse
    \fi
    \chapter*{\listfigurename}%
      \@mkboth{\MakeUppercase\listfigurename}%
              {\MakeUppercase\listfigurename}%
    \@starttoc{lof}%
    \if@restonecol\twocolumn\fi
    }
\end{teX}
The interesting command here is the |@starttoc{lof}|. This simply does all the housekeeping to open a file. as you can see it is not too difficult to have file extension names other than the standard ones.

The |l@| commands for the Table of Contents are defined as per the rest of the sectioning commands.

\begin{teX}
\newcommand*\l@figure{\@dottedtocline{1}{1.5em}{2.3em}}
\newcommand\listoftables{%
    \if@twocolumn
      \@restonecoltrue\onecolumn
    \else
      \@restonecolfalse
    \fi
    \chapter*{\listtablename}%
      \@mkboth{%
          \MakeUppercase\listtablename}%
         {\MakeUppercase\listtablename}%
    \@starttoc{lot}%
    \if@restonecol\twocolumn\fi
    }
\let\l@table\l@figure
\end{teX}

\section*{Bibliographies}

\begin{multicols}{2}
\latex provides some basic bibliographic commands. Every entry is defined to be displayed in a block. It starts by defining a new length |\bibindent|. Entries are displayed using the  |\list|. The commands here are mainly to set parameters for macros already provide by the kernel.
\end{multicols}
\index{book!environments!thebibliography}

\begin{teX}
\newdimen\bibindent
\setlength\bibindent{1.5em}
\newenvironment{thebibliography}[1]
     {\chapter*{\bibname}%
      \@mkboth{\MakeUppercase\bibname}{\MakeUppercase\bibname}%
      \list{\@biblabel{\@arabic\c@enumiv}}%
           {\settowidth\labelwidth{\@biblabel{#1}}%
            \leftmargin\labelwidth
            \advance\leftmargin\labelsep
            \@openbib@code
            \usecounter{enumiv}%
            \let\p@enumiv\@empty
            \renewcommand\theenumiv{\@arabic\c@enumiv}}%
      \sloppy
      \clubpenalty4000
      \@clubpenalty \clubpenalty
      \widowpenalty4000%
      \sfcode`\.\@m}
     {\def\@noitemerr
       {\@latex@warning{Empty `thebibliography' environment}}%
      \endlist}
\newcommand\newblock{\hskip .11em\@plus.33em\@minus.07em}
\let\@openbib@code\@empty
\end{teX}

\section*{The Index Environment}
This is a short environment definition for styling the Index. It defines in line [\ref{idxitem}] the 
|@idxidtem|, which is then used to define \cmd{subitem} and \cmd{subsubitem} styling.

\begin{teX}
\newenvironment{theindex}
   {\if@twocolumn
      \@restonecolfalse
      \else
         \@restonecoltrue
      \fi
      \twocolumn[\@makeschapterhead{\indexname}]%
      \@mkboth{\MakeUppercase\indexname}%
              {\MakeUppercase\indexname}%
                \thispagestyle{plain}\parindent\z@
                \parskip\z@ \@plus .3\p@\relax
                \columnseprule \z@
                \columnsep 35\p@
                \let\item\@idxitem}
      {\if@restonecol\onecolumn\else\clearpage\fi}
\newcommand\@idxitem{\par\hangindent 40\p@} (*@\label{idxitem}@*)
\newcommand\subitem{\@idxitem \hspace*{20\p@}}
\newcommand\subsubitem{\@idxitem \hspace*{30\p@}}
\newcommand\indexspace{\par \vskip 10\p@ \@plus5\p@ \@minus3\p@\relax}
\end{teX}

\section{Footnotes}
\label{book:footnotes}
\index{footnotes>\textbackslash footnoterule}

\begin{macro}{\footnoterule}
\begin{macro}{\@makefntext}
\begin{macro}{\@makefnmark}
\Paragraph{Footnote rules.} Footnote rules are defined by renewing the command |\footnoterule|. Counters for footnotes are reset based on the chapter counters. The footnote command |\@makefntext| provides the formatting. It also gives the user the ability to use these to insert footnotes, in difficult places.
\end{macro}
\end{macro}
\end{macro}

\begin{teX}
\renewcommand\footnoterule{%
  \kern-3\p@
  \hrule\@width.4\columnwidth 
  \kern2.6\p@}

\@addtoreset{footnote}{chapter}

\newcommand\@makefntext[1]{%
    \parindent 1em%
    \noindent
    \hb@xt@1.8em{\hss\@makefnmark}#1}
\end{teX}

\section*{Catering for Other Languages}

\begin{multicols}{2}
\textbf{Structural element names.}\quad \latex does not provide by itself the means to change the structural element names to a language other than English. Howerer, their names are defined  in a series of commands, that make it easier to be overwritten to change them to another language. As they are separate from the macros that use them, it is easy to overwrite them, in order to use another language. This is what the Babel package does. Note the Section, is not defined here.
\end{multicols}

\begin{teX}
\newcommand\contentsname{Contents}
\newcommand\listfigurename{List of Figures}
\newcommand\listtablename{List of Tables}
\newcommand\bibname{Bibliography}
\newcommand\indexname{Index}
\newcommand\figurename{Figure}
\newcommand\tablename{Table}
\newcommand\partname{Part}
\newcommand\chaptername{Chapter}
\newcommand\appendixname{Appendix}
\end{teX}

\textbf{Dates} Not much of a use but the month names are also defined here in an |\ifcase| statement. Again they can be overwritten by Babel.


\begin{teX}
\def\today{\ifcase\month\or
  January\or February\or March\or April\or May\or June\or
  July\or August\or September\or October\or November\or December\fi
  \space\number\day, \number\year}
\end{teX}

\Paragraph{\bf Multicolumn gutter and rule.}\quad Here two lengths are set. The distance between two columns of text and the width of the separating rule.

\begin{teX}
\setlength\columnsep{10\p@}
\setlength\columnseprule{0\p@}
\end{teX}


\section{Final}

\begin{teX}
\pagestyle{headings}
\pagenumbering{arabic}
\if@twoside
\else
  \raggedbottom
\fi
\if@twocolumn
  \twocolumn
  \sloppy
  \flushbottom
\else
  \onecolumn
\fi
\endinput
%%
%% End of file `book.cls'.

\end{teX}

\section{Ending remarks}

It is to the credit of Lamport and his associates that he was the first one to produce a system of mark-up that structured documents, using the TeX typographical engine. The class is widely used and many variants exist. One area that can be improved is to provide more `hooks' to enable programmers to redefine classes more easily.

Since the class has been published new packages have established themselves as the `de facto` standards of defining portions of the class. For example the no-new class will attempt to define all the papers as Lamport did, but would rather use the |geometry| package to do so. Top and bottom headings are defined using the |fancyverb|. 

\begin{quotation}
It was when the code was written, but is not now (in my opinion). The current LaTeX2e kernel was release in 1992 and carries forward a lot of material from LaTeX2.09. Even with these optimizations and the old 'autoload' system, there were a lot of systems that LaTeX was too big for on release. So looked at in the early 1990s this was entirely sensible.

I'd say this is no longer needed as in most LaTeX documents today there are a lot of tokens used by things like pgf which make the modest saving in optimisation pretty meaningless. One of the things we're doing in LaTeX3 is trying to move to more logical constructs at the expense of efficiency in tokens, at least at a higher level. (Right at the core of expl3 there is still a need to watch the number of expansions, etc., and this is an area where we may yet need some more optimisation.)

\end{quotation}

You can think of the \latex classes working at three levels. 

\begin{enumerate}
\item Selecting paper sizes and defining main page elements.
\item They define how the document is section. I have called this sectioning by referring to it as structural commands.
\item It provides the typesetting of these structural elements.
\end{enumerate}

Unfortunately, they are not separated in a way that makes it easy for them to be modified. A plethora of packages assists the author in modifying every type of sectioning and formatting decisions of Lamport. Most authors will focus on the formatting commands. Some will add a bit of structure, perhaps some special sections for questions and answers. If you have used the titlesec package for modifying the sections, the caption package for modifying the way captions are displayed, the fancyhdr for headers, the titletoc for the way table of contents are displayed, one of the bibliography packages what begs to be question is what remains? Very little. You might as well at this point decide on a new class. It will be more efficient and you will have better control. Separation of structure from presentational decisions is important. Some common structural elements that are missing should be integrated in. The KOMA classes and memoir went totally overboard, in that they try to be everything to everybody. A system that is nearer to defining a structural template and then decorate it with a selction of fonts, colors, spacing and the like would have been more appropriate.

\begin{teX}
%%
%% This is file `bk10.clo',

\ProvidesFile{bk10.clo}
              [2007/10/19 v1.4h
      Standard LaTeX file (size option)]
\end{teX}

\begin{teX}
\renewcommand\normalsize{%
   \@setfontsize\normalsize\@xpt\@xiipt
   \abovedisplayskip 10\p@ \@plus2\p@ \@minus5\p@
   \abovedisplayshortskip \z@ \@plus3\p@
   \belowdisplayshortskip 6\p@ \@plus3\p@ \@minus3\p@
   \belowdisplayskip \abovedisplayskip
   \let\@listi\@listI}
\normalsize
\newcommand\small{%
   \@setfontsize\small\@ixpt{11}%
   \abovedisplayskip 8.5\p@ \@plus3\p@ \@minus4\p@
   \abovedisplayshortskip \z@ \@plus2\p@
   \belowdisplayshortskip 4\p@ \@plus2\p@ \@minus2\p@
   \def\@listi{\leftmargin\leftmargini
               \topsep 4\p@ \@plus2\p@ \@minus2\p@
               \parsep 2\p@ \@plus\p@ \@minus\p@
               \itemsep \parsep}%
   \belowdisplayskip \abovedisplayskip
}
\newcommand\footnotesize{%
   \@setfontsize\footnotesize\@viiipt{9.5}%
   \abovedisplayskip 6\p@ \@plus2\p@ \@minus4\p@
   \abovedisplayshortskip \z@ \@plus\p@
   \belowdisplayshortskip 3\p@ \@plus\p@ \@minus2\p@
   \def\@listi{\leftmargin\leftmargini
               \topsep 3\p@ \@plus\p@ \@minus\p@
               \parsep 2\p@ \@plus\p@ \@minus\p@
               \itemsep \parsep}%
   \belowdisplayskip \abovedisplayskip
}
\newcommand\scriptsize{\@setfontsize\scriptsize\@viipt\@viiipt}
\newcommand\tiny{\@setfontsize\tiny\@vpt\@vipt}
\newcommand\large{\@setfontsize\large\@xiipt{14}}
\newcommand\Large{\@setfontsize\Large\@xivpt{18}}
\newcommand\LARGE{\@setfontsize\LARGE\@xviipt{22}}
\newcommand\huge{\@setfontsize\huge\@xxpt{25}}
\newcommand\Huge{\@setfontsize\Huge\@xxvpt{30}}
\end{teX}

\begin{multicols}{2}
\Paragraph{Indentation.} Paragraph indentation is controlled by the \tex command |parindent|. It is set narrower in two column text, to avoid problems with hyphenation that can result in overfull boxes.\index{Typography rules! paragraph!parindent}

1em rule \rule{1em}{1ex}  and 15pt rule \rule{15pt}{1ex} and 1.5em \rule{1.5em}{1ex}
\end{multicols}

\begin{teX}
\if@twocolumn
  \setlength\parindent{1em}
\else
  \setlength\parindent{15\p@}
\fi

\setlength\smallskipamount{3\p@ \@plus 1\p@ \@minus 1\p@}
\setlength\medskipamount{6\p@ \@plus 2\p@ \@minus 2\p@}
\setlength\bigskipamount{12\p@ \@plus 4\p@ \@minus 4\p@}
\setlength\headheight{12\p@}
\setlength\headsep   {.25in}
\setlength\topskip   {10\p@}
\setlength\footskip{.35in}
\if@compatibility \setlength\maxdepth{4\p@} \else
\setlength\maxdepth{.5\topskip} \fi
\if@compatibility
  \if@twocolumn
    \setlength\textwidth{410\p@}
  \else
    \setlength\textwidth{4.5in}
  \fi
\else
  \setlength\@tempdima{\paperwidth}
  \addtolength\@tempdima{-2in}
  \setlength\@tempdimb{345\p@}
  \if@twocolumn
    \ifdim\@tempdima>2\@tempdimb\relax
      \setlength\textwidth{2\@tempdimb}
    \else
      \setlength\textwidth{\@tempdima}
    \fi
  \else
    \ifdim\@tempdima>\@tempdimb\relax
      \setlength\textwidth{\@tempdimb}
    \else
      \setlength\textwidth{\@tempdima}
    \fi
  \fi
\fi
\if@compatibility\else
  \@settopoint\textwidth
\fi
\if@compatibility
  \setlength\textheight{41\baselineskip}
\else
  \setlength\@tempdima{\paperheight}
  \addtolength\@tempdima{-2in}
  \addtolength\@tempdima{-1.5in}
  \divide\@tempdima\baselineskip
  \@tempcnta=\@tempdima
  \setlength\textheight{\@tempcnta\baselineskip}
\fi
\addtolength\textheight{\topskip}
\if@twocolumn
 \setlength\marginparsep {10\p@}
\else
  \setlength\marginparsep{7\p@}
\fi
\setlength\marginparpush{5\p@}
\if@compatibility
   \setlength\oddsidemargin   {.5in}
   \setlength\evensidemargin  {1.5in}
   \setlength\marginparwidth {.75in}
  \if@twocolumn
     \setlength\oddsidemargin  {30\p@}
     \setlength\evensidemargin {30\p@}
     \setlength\marginparwidth {48\p@}
  \fi
\else
  \if@twoside
    \setlength\@tempdima        {\paperwidth}
    \addtolength\@tempdima      {-\textwidth}
    \setlength\oddsidemargin    {.4\@tempdima}
    \addtolength\oddsidemargin  {-1in}
    \setlength\marginparwidth   {.6\@tempdima}
    \addtolength\marginparwidth {-\marginparsep}
    \addtolength\marginparwidth {-0.4in}
  \else
    \setlength\@tempdima        {\paperwidth}
    \addtolength\@tempdima      {-\textwidth}
    \setlength\oddsidemargin    {.5\@tempdima}
    \addtolength\oddsidemargin  {-1in}
    \setlength\marginparwidth   {.5\@tempdima}
    \addtolength\marginparwidth {-\marginparsep}
    \addtolength\marginparwidth {-0.4in}
    \addtolength\marginparwidth {-.4in}
  \fi
  \ifdim \marginparwidth >2in
     \setlength\marginparwidth{2in}
  \fi
  \@settopoint\oddsidemargin
  \@settopoint\marginparwidth
  \setlength\evensidemargin  {\paperwidth}
  \addtolength\evensidemargin{-2in}
  \addtolength\evensidemargin{-\textwidth}
  \addtolength\evensidemargin{-\oddsidemargin}
  \@settopoint\evensidemargin
\fi
\end{teX}

\begin{multicols}{2}
\Paragraph{Top margin} Next the top margin is calculated.  In earlier versions the |\topmargin| was a fixed number. In this class, it is automatically calculated form the |\paperheight| (as the user only inputs the papersize through one of the paper selection options).
\end{multicols}

\begin{teX}
\if@compatibility
  \setlength\topmargin{.75in}
\else
  \setlength\topmargin{\paperheight}
  \addtolength\topmargin{-2in}
  \addtolength\topmargin{-\headheight}
  \addtolength\topmargin{-\headsep}
  \addtolength\topmargin{-\textheight}
  \addtolength\topmargin{-\footskip}     % this might be wrong! (previously set at 0.35in)
  \addtolength\topmargin{-.5\topmargin}
  \@settopoint\topmargin
\fi
\end{teX}

The lists settings follow. Similarly all values are hard-coded based on the font size.
\begin{teX}
\setlength\footnotesep{6.65\p@}
\setlength{\skip\footins}{9\p@ \@plus 4\p@ \@minus 2\p@}
\setlength\floatsep    {12\p@ \@plus 2\p@ \@minus 2\p@}
\setlength\textfloatsep{20\p@ \@plus 2\p@ \@minus 4\p@}
\setlength\intextsep   {12\p@ \@plus 2\p@ \@minus 2\p@}
\setlength\dblfloatsep    {12\p@ \@plus 2\p@ \@minus 2\p@}
\setlength\dbltextfloatsep{20\p@ \@plus 2\p@ \@minus 4\p@}
\setlength\@fptop{0\p@ \@plus 1fil}
\setlength\@fpsep{8\p@ \@plus 2fil}
\setlength\@fpbot{0\p@ \@plus 1fil}
\setlength\@dblfptop{0\p@ \@plus 1fil}
\setlength\@dblfpsep{8\p@ \@plus 2fil}
\setlength\@dblfpbot{0\p@ \@plus 1fil}
\setlength\partopsep{2\p@ \@plus 1\p@ \@minus 1\p@}
\def\@listi{\leftmargin\leftmargini
            \parsep 4\p@ \@plus2\p@ \@minus\p@
            \topsep 8\p@ \@plus2\p@ \@minus4\p@
            \itemsep4\p@ \@plus2\p@ \@minus\p@}
\let\@listI\@listi
\@listi
\def\@listii {\leftmargin\leftmarginii
              \labelwidth\leftmarginii
              \advance\labelwidth-\labelsep
              \topsep    4\p@ \@plus2\p@ \@minus\p@
              \parsep    2\p@ \@plus\p@  \@minus\p@
              \itemsep   \parsep}
\def\@listiii{\leftmargin\leftmarginiii
              \labelwidth\leftmarginiii
              \advance\labelwidth-\labelsep
              \topsep    2\p@ \@plus\p@\@minus\p@
              \parsep    \z@
              \partopsep \p@ \@plus\z@ \@minus\p@
              \itemsep   \topsep}
\def\@listiv {\leftmargin\leftmarginiv
              \labelwidth\leftmarginiv
              \advance\labelwidth-\labelsep}
\def\@listv  {\leftmargin\leftmarginv
              \labelwidth\leftmarginv
              \advance\labelwidth-\labelsep}
\def\@listvi {\leftmargin\leftmarginvi
              \labelwidth\leftmarginvi
              \advance\labelwidth-\labelsep}
\endinput
%%
%% End of file `bk10.clo'.

\end{teX}












%  \makeatletter
\newenvironment{adjustmargins}[2]{%
 \begin{list}{}{%
 \topsep\z@%
 \listparindent\parindent%
 \parsep\parskip%
 \checkoddpage
 \ifoddpage % odd numbered page
 \@ifmtarg{#1}{\setlength{\leftmargin}{\z@}}%
 {\setlength{\leftmargin}{#1}}%
 \@ifmtarg{#2}{\setlength{\rightmargin}{\z@}}%
 {\setlength{\rightmargin}{#2}}%
 \else % even numbered page
 \@ifmtarg{#2}{\setlength{\leftmargin}{\z@}}%
 {\setlength{\leftmargin}{#2}}%
 \@ifmtarg{#1}{\setlength{\rightmargin}{\z@}}%
 {\setlength{\rightmargin}{#1}}%
\fi
}
\item[]}{\end{list}}

\makeatother


\chapter{Pages}

\parindent1em

The page is the main element in a book and its geometry and layout has been studied extensively by typographers. In this chapter we outline the typographical tradition, methods to specify layouts using \latex and associated issues, such as adjusting margins within a page.

Bringhurst notes that ``much typography is based, for the sake of economy on standard industrial sizes, from $35\times45$ inch press sheets to $3 1/2$ x 2 inch conventional business cards. Some formats as the booklets that accompany mobile telephone kits, are condemned to especially rigid restrictions of size.  

There may already be some restrictions on the page size you choose depending on your method of production and distribution. If you aim to output pages on a desktop printer then a standard size like A4 ($297\times210$)mm or US letter ($11\times 8 1/2$ inches) is advisable. If you have the opportunity and necessity of selecting the dimensions of the page you have a great opportunity to enhance the page layout of your book.

\section{Selecting  paper sizes}

Besides the limitations of the method of printing, another consideration is the size of book you writing and the
audience you are addressing it. If you are only producing a 60 page book, paper with smaller dimensions might be more appropriate than a blockbuster novel. 

History, natural science, geometry and mathematics are all relevant to typography.


\begin{figure}[ht]
\centering
\includegraphics[width=0.5\textwidth]{./images/preparing-paper.jpg}
\caption{Getting paper prepared for printing \protect\cite{moxon}.}
\end{figure}

Originally, paper sizes were determined by the moulds the paper was
made in and the use the result was put to. While many hundreds of variations have occurred throughout the centuries, in the main there have seldom been more than six categories of sizes in use since the fourteenth century. These have often come down to us bearing the names of the figures featured in the paper's watermarks, such as \emph{foolscap},
\emph{elephant}, \emph{pot}, and \emph{crown}. To enable the creation of smaller sizes from
existing larger sizes, the sheets have since the Middle Ages been proportioned
with their sides in the ratio of \(1:\sqrt{2}\). For example, quarto (4to,
formerly 4to) and octavo (8vo, formerly 8vo) sizes are obtained by cutting
or folding standard sizes four and eight times respectively.
Former British paper dimensions still used the old sizes before decimalized
versions replaced them; US dimensions still retain most of these
(untrimmed) paper sizes, in inches. Both are still encountered in specialist and bibliographic work, and in reproducing earlier or foreign formats:


\section{Canonical Layouts}

Typographers derive proportions that naturally occur in nature, and pages that embody
them recur in manuscripts and books from Rennaissance Europe, Tang and Song dynasty
China, early Egypt and ancient Rome.  
These numbers are $\pi=$3.14159\ldots , which is the circumference of a circle whose diameter
is one; $\sqrt{2}=$1.41421\ldots , which is the diagonal of a unit square; 
$e=2.71828$  \ldots ,which is the base of the natural logarithms; and $\phi=1.61803$ \ldots ,a number which is discussed later on. Certain of these proportions appear in he structure of the human body; other appear in musical scales. Indeed, one of the simplest of all systems of 
page proportions is based on the familiar intervals of the diatonic scale. Pages that
embody these basic musical proportions have been in common use in Europe for more than a thousand year.

 \begin{figure}
 \makebox[\textwidth]{\makebox[1.1\textwidth][r]{%
 \unitlength=0.0015\textwidth
 \let\ul\unitlength
	\begin{picture}(184,320)(0,-20)
	\put(0,0){\framebox(184,297){}}
	\put(27,70){\makebox(0,0)[bl]{\color{thegray}\rule{113\ul}{184\ul}}}
	\put(92,-20){\makebox(0,0)[t]{Golden number canonical layout}}
	\color{red}
	\put(92,162){\circle{184}}
	\linethickness{.2pt}
	\multiput(0,297)(1.98918918919,-3.21081081082){93}{\line(184,-297){1}}
	\end{picture}%
 \hfill
	\unitlength0.0015\textwidth
	\let\ul\unitlength
		\begin{picture}(210,330)(0,-20)
		\put(0,0){\framebox(210,297){}}
		\put(25,51){\makebox(0,0)[bl]{\color{thegray}\rule{149\ul}{210\ul}}}
		\put(105,-35){\makebox(0,0)[b]{ISO canonical layout}}
		\color{red}
		\put(105,156){\circle{210}}
		\multiput(0,297)(2.27027027027,-3.21081081082){93}{\line(210,-297){1}}
		\end{picture}%
 \hfill
   \unitlength0.001591\textwidth
   \let\ul=\unitlength
   \begin{picture}(220,330)(0,-20)
   		\put(0,0){\framebox(220,280){}}
   		\put(20.742,33.6){\makebox(0,0)[bl]{\color{thegray}\rule{172.86\ul}{220\ul}}}
   		\put(110,-35){\makebox(0,0)[b]{Letter paper canonical layout}}
   		\color{red}
   		\put(110,143.6){\circle{220}}
		\multiput(0,280)(2.37837837838,-3.02702702703){93}{\line(220,-280){1}}
   \end{picture}
 }}%
 \caption{A right page with the relevant diagonal, the text block and the canonical circle.
 In this figure the important information is the page proportions, not the scale; matter of
 fact the letter paper is 17.6 mm shorter than the A4 paper, but the drawings to the same
 height emphasize the relative proportions of the various page parts. \cite{canonicallayout}}
 \label{fig:canoniclayout}
 \end{figure}
 
The package \pkgname{xlayouts} and also Beccari’s \pkgname{canonical} layouts provide both graphical as well as settings for determining page layouts that approach canonical layouts. In reality modern book design has diverged from these principles. 

\begin{figure}[hb]
\cxset{spread xsteps=9,
          spread scale=0.20,
          spread width=0.5\textwidth}
\centering
\drawcanons
\end{figure}


\begin{figure}
  \includegraphics[width=0.5\linewidth]{./graphics/A-sizes.png}
  \caption{When a sheet whose proportions are $1$:$\surd{2}$ is folded in half, the result is a sheet half as large but with \emph{the same proportions}. Standard paper sizes on this principle have been in use in Germany since the early 1920s. The basis of this system is the A0 sheet, which has an are of 1 m$^2$. Yes because it is \textit{reciprocal with nothing but itself}, the ISO page in isolation is the least musical of all the major page shapes. It needs a textblock of another shape or contrast.}
   \label{fig:marginfig1}
\end{figure}

The advantages of basing a paper size upon an aspect ratio of $\surd{2}$ were already noted in 1786 by the German scientist Georg Christoph Lichtenberg, in a letter to Johann Beckmann[2]. The formats that became |A2|, |A3|, |B3|, |B4| and |B5| were developed in France, and published in 1798 during the French Revolution, but were subsequently forgotten. \cite{letimbre2136}

Early in the twentieth century, Dr Walter Porstmann turned Lichtenberg's idea into a proper system of different paper sizes. Porstmann's system was introduced as a DIN standard (DIN 476) in Germany in 1922, replacing a vast variety of other paper formats. Even today the paper sizes are called "DIN Ax" in everyday use in Germany.

The main advantage of this system is its scaling: if a sheet with an aspect ratio of $\surd{2}$ is divided into two equal halves parallel to its shortest sides, then the halves will again have an aspect ratio of $\surd{2}$. Folded brochures of any size can be made by using sheets of the next larger size, e.g. |A4| sheets are folded to make |A5| brochures. The system allows scaling without compromising the aspect ratio from one size to another – as provided by office photocopiers, e.g. enlarging |A4| to |A3| or reducing |A3| to |A4|. Similarly, two sheets of |A4| can be scaled down and fit exactly 1 sheet without any cutoff or margins.

%\cxset{try grid=false}
%\thispagestyle{grid}


The weight of each sheet is also easy to calculate given the basis weight in grams per square metre (g/m² or `'gsm"). Since an |A0| sheet has an area of 1m² , its weight in grams is the same as its basis weight in g/m². A standard |A4| sheet made from 80 g/m² paper weighs 5g, as it is one 16th (four halvings) of an A0 page. Thus the weight, and the associated postage rate, can be easily calculated by counting the number of sheets used.

Unlike the |A4| standard paper, the origin of the dimensions of letter size paper are lost in tradition. The American Forest and Paper Association argues that the dimension originates from the days of manual paper making, and that the 11-inch length of the page is about a quarter of ``the average maximum stretch of an experienced vatman's arms".[1] However, this does not explain the width or aspect ratio. What is known is that Ronald Reagan made this the paper size for U.S. federal forms; previously, the smaller "official" size (8 in × 10½ in or 203.2 mm × 266.7 mm) was used.[1] Letter or US Letter is the most common paper size for office use in the United States and Canada. It is 8$\frac{1}{2}$ by 11 inches (exactly 215.9 mm × 279.4 mm).

\section{The Typearea}

According to \cite{bringhurst2005}, in typography margins must do three things. They must lock the
textblock to the page and lock the facing pages to each other through the force of their proportions. Second, they must frame the textblock in a manner that suits its design. Third, they must protect the textblock, leaving it easy for the reader to see and convenient to handle. 

In most well designed books fifty per cent of the character and integrity of a printed page lies in its letterforms. Much of the other fifty per cent resides in its margins.


\subsection{Readability}

Another aspect that determines the text area, is the readability of the text. Here you need to take into account the readers of your book. For children and older persons a larger type and shorter lines are preferred.

\begin{macro}{\alphabetlength}
The macro |\alphabetlength| prints the length of the alphabet. The length of the alphabet in this text is \alphabetlength. If this is a good choice is debatable, but after all this is just a long document, with many chapters and my aim was to produce a reference and a test document. The macro is defined in the |xlayouts|  package, which is loaded automatically by the |phd| package or class. 
\end{macro}

Traditionally  a line that is approximately 1.4 times the alphabet length is considered good practice. The length of one line of text in this document is \the\textwidth giving a ratio of \alphabetsperline.

\DescribeMacro{\printreadability} prints a small table with some readability figures. If LuaTeX is used, this table is slightly longer and prints some other statistics as well. 

\begin{figure}[htbp]
\drawtriallayout
\bigskip

\printreadability
\captionof{figure}{Page layout diagram and readability statistics (using the \pkgname{xlayouts} package).}
\end{figure}

The macros described above are loaded by the |xlayouts| package, which forms part of the |phd| budle. There are macros for drawing trial layouts 


\section{Examples}
Folowing the nomenclature introduced b Bringhurst in analyzing the examples on the following pages, 
these symbols are used:

%% Align at the = sign 
\begin{table}[htbp]
\begin{tabular}{l l @{ = } p{6cm}}
\textit{Proportions:}      &P  &  page proportion $h/w$\\
~                      &T &  textblock proportion: $d/m$\\
\textit{Page size:}         &w &  width of page (trim-size)\\
~                      &h  & height of page (trim-size)\\
\textit{Textblock:}           &m & measure (width of primary textblock)\\
~                      &d  & depth of primary textblock (excluding running heads, folios etc)\\                      
~                      &$\lambda$ & line height (type size plus added lead)\\
~                      &$n$ & secondary measure (width of secondary column)\\
~                      &$c$  & column width, where there are even multiple columns\\
\textit{Margins}  &$s$  & spine margin (back margin)\\
~                      &$t$   & top margin (header margin)\\
                        &$e$  & fore-edge (front margin)\\
                        &$f$   & foot margin\\
                        &$g$  & internal gutter (on a multiple-column page)\\
\end{tabular}
\caption{Symbols used to demonstrate various ratios in books}
\end{table}
\medskip

\begin{figure}
  \includegraphics[width=\linewidth]{./graphics/page.png}
  \caption{Page nomenclature}
   \label{fig:marginfig1}
\end{figure}

More variables are necessary to specify all the variables handled by a \latex\
page. For the time being the examples refer to dimensions from historical works
in typography and should sufffice.

\subsection{Hypneroto}

\begin{figure}[htbp]
\centering
  \includegraphics[width=\linewidth]{./graphics/hypneroto.jpg}
\caption{The work is lauded for the originality of its
design. Several sequential double page
illustrations add a visual dimension to the
progression of the narrative, and act like an
early form of the strip cartoon. There is an
obsession with movement throughout which is driven
on by the illustrations, resulting in the
impression of bodies moving from one page to the
next. Other typographical innovations include
playing with the traditional layout of the text;
in the opening shown here, for example, the pages
are shaped in the form of goblets. The dimensions
of the text are: $P=1.5[2:3]$; T=1.7 (tall pentagon);
Margins: s=t=w/9; e=2 s. The text is a fantasy
novel, Francesco Colonna's Hypnerotomachia
Poliphili, set in a roman font cut by Francesco
Griffo. (Aldus Manutius, Venice, 1499). Original
size: $20.5\times31$\thinspace cm.}
\label{fig:hypneroto}
\end{figure}


%  \label{fig:layout}



The book was printed by Aldus Manutius in Venice in December 1499. The book is anonymous, but an acrostic formed by the first, elaborately decorated letter in each chapter in the original Italian reads \textsc{\small POLIAM FRATER FRANCISCVS COLVMNA PERAMAVIT}, \enquote{Brother Francesco Colonna has dearly loved Polia.} However, the book has also been attributed to Leon Battista Alberti by several scholars, and earlier, to Lorenzo de Medici. The latest contribution in this respect was the attribution to Aldus Manutius, and arguably, a Francesco Colonna, a wealthy Roman Governor. The author of the illustrations is even less certain, but contemporary opinion gives the work to Benedetto Bordone.

\section{Contemporary book layouts}

All these sound mystical with religious undertones, but we need to remember that early printers made their livelihoods from printing mostly religious books.

From the mystics to the modern, let us study Figure~\ref{fig:nudes}

\begin{figure}[htbp]
\centering

\fbox{\includegraphics[width=\linewidth]{nudes.jpg}}

\caption{In this layout, the placement of various size images on the right pages, makes the margins disappear to the eye. As the whole book, is made of similar pages\ldots }
\end{figure}

Modern designers are more cryptic. One book that I found more useful is Ambrose/Harris \textit{Layout}. The book brings together examples of layout, both contemporary  and historic, from aroudnd the world. It contains examples from leading graphic designers to provide a sample of rich and diverse possibilities for the creative use of layout.

As it will become apparent from what follows, although at first look it appears that all design principles have disappeared into post modern designs, all design is undertaken with reference to a certain set of principles, either by consciously
choosing to follow or by deliberately ignoring or subverting them. The collective body of principles represents different approaches to design and layout construction.

The principles in this section have been used
through the ages to produce designs that are
pleasing to the eye and that organise information
clearly and efficiently, two of the challenges facing
every graphic designer. These principles affect
decisions made at the heart of the design process,
as they provide the basis of how space is divided.

\section{A design must capture the spirit of the times.}

The word \emph{zeitgeist} originates from the German zeit (time) and geist (spirit),
and so literally means spirit of the age. In graphic design, each decade can
be defined by several predominant zeitgeists that somehow seem to capture
their essence. Today, in graphic design, we can see a zeitgeist for the use of
sophisticated computer graphics giving a very close approximation to reality
in addition to another, which is a backlash to this, in the form of rough-and-ready
hand-drawn designs.

\section{Objects on a Page}

How an object is placed on a page has a dramatic
impact on how it is received and interpreted by
the viewer, and the message that it delivers. We
have looked at how grids can be used to guide
element placement on a page, but maintaining a
sense of order is not the only consideration when
laying out a design.

Object placement helps form the narrative of
a design and is constructed from an understanding
of how we read a page. The narrative of a design
can be created and altered by a wide range of
placement and intervention strategies, such as
how white space is used, the balance and relative
weight given to other objects, the juxtaposition or
contrast of objects and so on.

This chapter will outline some of the
fundamental approaches to object placement.

\section{White Space}

White space is not necessarily white, as it refers to any space in the design
without text or graphic elements. Designers naturally insert white space into
their designs to help the composition and make the information the design
contains easier to access, such as leaving margins at the sides of the page that
create space around text blocks. Swiss typographer Jan Tschichold called white
space ‘the lungs of a good design’. Without white space, with every part of the
design area filled, there is a danger that a design would look cramped and
difficult to understand.

White space can instil different perceptions in a viewer depending on how it is
used and the content it is associated with. White space may give the impression
of luxury and extravagance for a full-page photograph. However, it may also give
the impression that there are gaps in a layout that is rather full, or worse, that
there is insufficient content to fill a page. Newspapers try to establish a rational
balance between giving space to page elements to meet the conflicting demands
of the need for typographical sensitivity and readability, while filling a page with
news so that the reader feels they are getting value for money. Habitually readers
expect a newspaper to be ‘full’, which means it is harder to achieve typographic
balance. In contrast, where filling space is of less concern, such as the example
below, white space becomes a more overt part of the design.



\section{Grids}

\subsection{The Baseline Grid}

The baseline grid is the (invisble) graphic foundation upon which a design is constructed and provides a visual guide for positioning and ligning page elements with an accuracy that is difficult to achieve by eye alone. Knuth's TeX focuses almost primarily on getting this one right.

\section{Pace}

It came to me as a big surprise that a books layout must have \textit{pace}. This essentially is the alternation of pages, between say images and text.

\begin{figure}[htbp]
\parindent=0pt
\includegraphics[width=\textwidth]{pace}

\end{figure}

Thumbnails are smaller versions of the spreads of a publication presented on a
page that allow a designer to gauge its pace and balance at the macro level
without focusing on details. Thumbnails allow a designer to look at the
narrative of the publication and tune it as a whole, rather than on a spread-byspread
basis.

Pictured are thumbnails for Miss X, a book for underwear retailer Agent
Provocateur art directed by Mike Figgis and published by Anova, with design by
Gavin Ambrose. The absence of folios and minimal text mean the image flow
takes prominence.

The images can let us set a method for defining such spreads. 


\chapter{Temporarily changing the text width}

\index{pagewidth>change temporarily}


Margins in a page can be changed temporarily by adjusting, the lengths of \cmd{\leftskip} and \cmd{\rightskip}. The |memoir| class provides an environment |adjustwidth| see page 422 (based on This code is based on the \pkgname{chngpage} package.) for doing so and the \class{tufte-book} class provides an environment \textit{fullwidth}. The following code is an adaptation of that found in the \class{memoir} class.


\begin{teXX}
\begin{adjustmargins}{left}{right} 
\end{teXX}


adds the given lengths to the left and
right hand margins. A positive value will shorten the text and a negative value
will widen it. The starred version of the environment will cause the margin changes to switch between odd and even pages. 



\eject
\newgeometry{left=10mm,right=10mm,bottom=1.5cm,top=1cm}

\section*{The \texttt{adjustmargins} environment}
\lorem

\vfill\vfill
\begin{multicols}{2}
\lorem
\end{multicols}

\begin{adjustmargins}{0cm}{0in}
{\leftskip1em\rule{13cm}{.4pt}\par}

\centering



\parbox{\textwidth}{{\leftskip1em\rightskip1em There are no engineers in the hottest parts of hell, because the existence of a 'hottest part' implies a temperature difference, and any marginally competent engineer would immediately use this to run a heat engine and make some other part of hell comfortably cool.  This is obviously impossible.\par}
}
\par
\medskip
\par
\noindent\includegraphics[width=0.9\textwidth]{./graphics/lilian.jpg}\par


\end{adjustmargins}

\clearpage

\restoregeometry


\lipsum[1]


\begin{adjustmargins}{-0.4\textwidth}{0.1\textwidth}
\fboxsep2pt%
\fbox{\includegraphics[width=1.2\textwidth]{./graphics/leoncroll.jpg}}
\end{adjustmargins}

\lipsum[2]

\begin{teX}
\begingroup
\makeatletter
 \catcode`\Q=3
 \long\gdef\@ifmtarg#1{\@xifmtarg#1QQ\@secondoftwo\@firstoftwo\@nil}
 \long\gdef\@xifmtarg#1#2Q#3#4#5\@nil{#4}
 \long\gdef\@ifnotmtarg#1{\@xifmtarg#1QQ\@firstofone\@gobble\@nil}
 \endgroup


\newenvironment{adjustmargins}[2]{%
  \begin{list}{}{%
    \topsep\z@%
    \listparindent\parindent%
    \parsep\parskip%
   \@ifmtarg{#1}{\setlength{\leftmargin}{\z@}}%
   {\setlength{\leftmargin}{#1}}%
   \@ifmtarg{#2}{\setlength{\rightmargin}{\z@}}%
   {\setlength{\rightmargin}{#2}}%
}
\item[]}{\end{list}}
\makeatother
\end{teX}

 
\section{Setting Dimensions in \latex}

To set dimensions for page layout in \latex is not straightforward. You need to adjust several \latex
native dimensions to place a text area where you want. If you want to center the text area in the paper
you use, for example, you have to specify native dimensions as follows:

\begin{verbatim}
\usepackage{calc}
\setlength\textwidth{7in}
\setlength\textheight{10in}
\setlength\oddsidemargin{(\paperwidth-\textwidth)/2 - 1in}
\setlength\topmargin{(\paperheight-\textheight
-\headheight-\headsep-\footskip)/2 - 1in}.
\end{verbatim}

Without package |calc|, the above example would need more tedious settings. To adjust all parameters from scratch one should have a good understanding, of \latexe's definitions of all parameters. The companion package |xlayouts| can be used to display these parameters on an actual printed page. All settings are parameterized and I find the use of colours assists in viewing the rulers better.


\subsection{The Geometry package}

The package \pkg{geometry} \cite{geometry} provides
an easy way to set page layout parameters. In this case, what you have to do is just load the package and set
the page geometry using keys.

\begin{teX}
\usepackage[text={7in,10in},centering]{geometry}.
\end{teX}

Besides centering problem, setting margins from each edge of the paper is also troublesome. But geometry
also make it easy. If you want to set each margin to 1.5in, you can type

\begin{comment}
\label{sec:geometry}

\def\OpenB{{\ttfamily\char`\{}}
 \def\Comma{{\ttfamily\char`,}}
 \def\CloseB{{\ttfamily\char`\}}}
 \def\Gm{\textsf{geometry}}
\newcommand\gpart[1]{\textsf{\textsl{\color[rgb]{.0,.45,.7}#1}}}%

\newcommand\glen[1]{\textsf{#1}}

\bgroup
\makeatletter
 \begin{figure}
  \small
  \unitlength=.65pt
  \begin{picture}(450,250)(0,-10)
  \put(20,0){\framebox(170,230){}}
  \put(20,235){\makebox(170,230)[br]{\gpart{paper}}}
  \begingroup\thicklines
  \put(40,30){\framebox(120,170){}}
  \put(40,30){\makebox(120,165)[tr]{\gpart{total body}~}}
  \put(45,30){\makebox(0,170)[l]{|height|}}
  \put(40,35){\makebox(120,0)[bc]{|width|}}
  \put(50,-20){\makebox(120,0)[bc]{|paperwidth|}}
  \put(10,45){\makebox(0,170)[r]{|paperheight|}}
  \put(90,200){\makebox(0,30)[lc]{|top|}}
  \put(90,0){\makebox(0,30)[lc]{|bottom|}}
  \put(10,70){\makebox(0,0)[r]{|left|}}
  \put(10,55){\makebox(0,0)[r]{(|inner|)}}
  \put(200,70){\makebox(0,0)[l]{|right|}}
  \put(200,55){\makebox(0,0)[l]{(|outer|)}}
  \put(80,230){\vector(0,-1){30}}\put(80,30){\vector(0,-1){30}}
  \put(80,200){\vector(0,1){30}}\put(80,0){\vector(0,1){30}}
  \put(20,70){\vector(1,0){20}}\put(40,70){\vector(-1,0){20}}
  \put(160,70){\vector(1,0){30}}\put(190,70){\vector(-1,0){30}}
  \multiput(160,30)(5,0){24}{\line(1,0){2}}
  \multiput(160,200)(5,0){24}{\line(1,0){2}}
  \begingroup\thicklines
  \put(280,30){\framebox(120,170){}}\endgroup
  \put(283,133){\makebox(0,12)[l]{|textheight|}}
  \put(295,130){\vector(0,-1){100}}\put(295,150){\vector(0,1){50}}
  \multiput(280,220)(5,0){24}{\line(1,0){3}}
  \put(280,208){\makebox(120,20)[bc]{\gpart{head}}}
  \multiput(280,207)(5,0){24}{\line(1,0){3}}
  \put(420,225){\makebox(0,0)[l]{|headheight|}}
  \put(418,225){\line(-2,-1){20}}
  \put(420,213){\makebox(0,0)[l]{|headsep|}}
  \put(418,213){\line(-2,-1){20}}
  \put(420,12){\makebox(0,0)[l]{|footskip|}}
  \put(418,12){\line(-2,1){20}}
  \put(280,40){\makebox(120,140)[c]{\gpart{body}}}
  \put(305,45){\vector(-1,0){25}}\put(375,45){\vector(1,0){25}}
  \put(80,230){\vector(0,-1){30}}\put(80,30){\vector(0,-1){30}}
  \put(280,48){\makebox(120,0)[c]{|textwidth|}}
  \put(280,15){\makebox(120,10)[c]{\gpart{foot}}}
  \multiput(280,14)(5,0){24}{\line(1,0){2}}
  \put(410,30){\dashbox{3}(30,170){}}
  \put(415,30){\makebox(30,170)[l]{\gpart{marginal note}}}
  \put(425,45){\vector(-1,0){15}}\put(425,45){\vector(1,0){15}}
  \put(450,70){\makebox(0,0)[l]{|marginparsep|}}
  \put(448,70){\line(-3,-1){43}}
  \put(450,45){\makebox(0,0)[l]{|marginparwidth|}}
  \end{picture}
\caption{Dimension names used in the geometry package. width $=$ textwidth and height $=$ textheight by default. left, right, top and bottom are margins. If margins on verso pages are swapped by twoside option, margins specified by left and right options are used for the inside and outside margins respectively. inner and outer are aliases of left and right
respectively.}
\label{fig:geometrylayout}
\end{figure}
\makeatother
\egroup
\end{comment}

 The \pkg{geometry} package provides a flexible and easy interface to page dimensions.
 You can change the page layout with intuitive parameters. For instance,
 if you want to set a margin to 2cm from each edge of the paper,
 you can type just |\usepackage[margin=2cm]{geometry}|. 
 The page layout can be changed in the middle of the document
 with \cs{newgeometry} command.  The \ref{fig:geometrylayout}, reproduced from the package documentation, illustrates the variety of parameters that can be set using the package.


\section{Footnotes}
The history of footnotes is as long and complicated as the history of scholarship and commentary. Hebrew scholars more than two thousand years ago used systems of glossing and annotation to work on religious texts. 

Scribes in the Christian tradition in the medieval period made use of annotations in their manuscript copying practices: surrounding the original text with glosses in small letters. After the advent of printing, similar kinds of marginal annotation appeared in printed texts of the late fifteenth century. 

Humanist scholars producing printed editions of classical learning in the sixteenth century also made use of the resources of typography to display both the surviving classical text and their commentary on the same page. References to classical sources - and to modern printed editions - became more systematic, as did the expectation that such references would be consistent with scholarly practices. Scholars increasingly marked their professionalism by using complex citational conventions, which by the seventeenth century were so well established as to be the subject of parody and satire. Scriblerian satire of the early eighteenth century, whose purpose was to mock the pedantry and folly of the works of the learned, frequently included extensive parodies of footnotes and the scholarly contests they encoded. Nonetheless, during the eighteenth century, to appear authoritative and learned an author had to adopt the scholarly machinery of the reference citation.

The footnote was born out of a desire to rationalise the relation between text and citation. 

Robert Connors argues that marginal notations fell out of favour for two practical reasons: they left too much blank paper at the side of the text; and they were difficult for typographers to set. The same notes placed at the bottom of the page were more efficient, both in paper and time[1]. 

Anthony Grafton's The Footnote: A Curious History suggests the modern footnote, inaugurated by Pierre Bayle's Dictionaire Critique et Historique in 1697, signalled an epistemic revolution in historical scholarship, indicating the end of credulous scholasticism and the emergence of analytical historical methodologies. Both scholars note the considerable impact of historians such as David Hume and Edward Gibbon on the stylistic development of the discursive and citational footnote as a location for the display of gentlemanly ease as much as scholarly acumen. In the nineteenth century, German scholars such as Leopold von Ranke and Alexander von Humboldt established a systematic basis for the footnote citation, creating a methodical methodological approach that all competing scholars had to obey. In this way, the idea of the footnote was established, yet no there was no general agreement on the form these footnotes should adopt. A systematic approach to the form of the footnote was needed.

In this section we will discuss how lines and paragraphs are turned into pages and how elements of pages such as footnotes, headers etc are inserted. As with the other chapters we will mix TeX basic commands with the more convenient \LaTeXe\ commnads. We will also look at some of the packages and classes that are availble to assist us with page layouts. 


Besides illustrations that are inserted at the top of a page, plain TEX will also
insert footnotes at the bottom of a page. The ootnote macro is provided
for use within paragraphs;  for example, the footnote in the present sentence was typed
in the following way:


There are two parameters to a footnote[ first comes the reference mark, which will
appear both in the paragraph** and in the footnote itself, and then comes the text of
the footnote.45 The latter text may be several paragraphs long, and it may contain
\footnote{Sidenote: ``Where God meant footnotes to go.'' ---Tufte}

\marginpar{

The history of footnotes is as long and complicated as the history of scholarship and commentary. Hebrew scholars more than two thousand years ago used systems of glossing and annotation to work on religious texts. Scribes in the Christian tradition in the medieval period made use of annotations in their manuscript copying practices: surrounding the original text with glosses in small letters. After the advent of printing, similar kinds of marginal annotation appeared in printed texts of the late fifteenth century. Humanist scholars producing printed editions of classical learning in the sixteenth century also made use of the resources of typography to display both the surviving classical text and their commentary on the same page. References to classical sources - and to modern printed editions - became more systematic, as did the expectation that such references would be consistent with scholarly practices. Scholars increasingly marked their professionalism by using complex citational conventions, which by the seventeenth century were so well established as to be the subject of parody and satire. Scriblerian satire of the early eighteenth century, whose purpose was to mock the pedantry and folly of the works of the learned, frequently included extensive parodies of footnotes and the scholarly contests they encoded. Nonetheless, during the eighteenth century, to appear authoritative and learned an author had to adopt the scholarly machinery of the reference citation.

The footnote was born out of a desire to rationalise the relation between text and citation. Robert Connors argues that marginal notations fell out of favour for two practical reasons: they left too much blank paper at the side of the text; and they were difficult for typographers to set. The same notes placed at the bottom of the page were more efficient, both in paper and time[1]. Anthony Grafton's The Footnote: A Curious History suggests the modern footnote, inaugurated by Pierre Bayle's Dictionaire Critique et Historique in 1697, signalled an epistemic revolution in historical scholarship, indicating the end of credulous scholasticism and the emergence of analytical historical methodologies. Both scholars note the considerable impact of historians such as David Hume and Edward Gibbon on the stylistic development of the discursive and citational footnote as a location for the display of gentlemanly ease as much as scholarly acumen. In the nineteenth century, German scholars such as Leopold von Ranke and Alexander von Humboldt established a systematic basis for the footnote citation, creating a methodical methodological approach that all competing scholars had to obey. In this way, the idea of the footnote was established, yet no there was no general agreement on the form these footnotes should adopt. A systematic approach to the form of the footnote was needed.}

Further reading:

Connors, Robert J., 'The Rhetoric of Citation Systems, Part I: The Development of Annotation Structures from the Renaissance to 1900', Rhetoric Review, 17 (1998), 6-48.

Connors, Robert J., 'The Rhetoric of Citation Systems, Part II: Competing Epistemic Values in Citation', Rhetoric Review, 17 (1999), 219-245.

Grafton, Anthony, The Footnote: A Curious History (London: Faber and Faber, 1997)

Grafton, Anthony, 'The Footnote from De Thou to Ranke', History and Theory, 33 (1994), 53-76

Zerby, Chuck, The Devil's Details: A History of Footnotes (Lancaster: Gazelle, 2002)

[1] Robert J. Connors, The Rhetoric of Citation Systems, Part I: The Development of Annotation Structures from the Renaissance to 1900, Rhetoric Review, 17 (1998), 6-48 (p. 30).

%  \chapter{Bibliography Management} 

\begin{figure}[p]
\includegraphics[width=\textwidth]{./images/ammar.jpg}
\caption{Wilson, Digital Collage, L. Ammar \protect\url{http://daliahammar.com/post/49217473452/wilson-digital-collage}}
\end{figure}
 
\precis{In this chapter we outline a number of experimental keys that been defined to handle Table of Contents (ToC) formatting. These keys are currently experimental.}
\addtocimage{-12pt}{-20pt}{./images/tocblock-man-02.jpg}

       
\def\bibtex{\texttt{bibTeX\xspace}}

For any academic/research writing, incorporating references into a document is an important task. Fortunately, \latex provides  a variety of features that make dealing with references much simpler, including built-in support for citing references. However, a much more powerful and flexible solution is achieved thanks to an auxiliary tool called \bibtex and if your \latex  distribution does not include it is obtainable from \url{http://www.bibtex.org}.


The style of this book places all citations to the side margin. For example, the command  \verb+\cite{Abrahams2003}+, will produce the citation \cite{Abrahams2003}. I find this type of style (suggested by \cite{Tufte1997}) more clear and relevant.

Notes in text for many centuries, before printed books were a common feature. The author picking up a different thread and not wishing to divert immediate attention away from the main body of his work. With printing, the costs of books were high and printers started placing citations and footnotes at the bottom of the page. You are not limited though to use only this style, by using |\cite{Bringhurst2005}|, \citet{Bringhurst2005}.

You can also use, the following code to get a within the text full citation:


\bibentry{Bringhurst2005}



\bibtex provides for the storage of all references in an external, flat-file database. This database can be linked to any \latex document, and citations made to any reference that is contained within the file. This is often more convenient than embedding them at the end of every document written. There is now a centralized bibliography source that can be linked to as many documents as desired (write once, read many!). 

Of course, bibliographies can be split over as many files as one wishes, so there can be a file containing references concerning General Relativity and another about Quantum Mechanics. When writing about Quantum Gravity (QG), which tries to bridge the gap between these two theories, both of these files can be linked into the document, in addition to references specific to QG.

\section{Citations}

To actually cite a given document is very easy. Go to the point where you want the citation to appear, and use the following: cite cite key, where the cite key is that of the bibitem you wish to cite. When LaTeX processes the document, the citation will be cross-referenced with the bibitems and replaced with the appropriate number citation. The advantage here, once again, is that LaTeX looks after the numbering for you. If it were totally manual, then adding or removing a reference would be a real chore, as you would have to re-number all the citations by hand.

Instead of WYSIWYG editors, typesetting systems like TeX or LaTeX \citep{lamport2004} can be used. cite{Abut1990}

\section{Referring to specific pages}

Sometimes you want to refer to a certain page, figure or theorem in a text book. For that you can use the arguments to the 

\begin{texexample}{Citation Example}{}
\cs{cite} command:
\cite[p. 215]{Mittelbach2004}
\end{texexample}

The argument, "p. 215", will show up inside the same brackets

\section{BibTeX}

I have previously introduced the idea of embedding references at the end of the document, and then using the \cs{cite} command to cite them within the text. In this tutorial, I want to do a little better than this method, as it's not as flexible as it could be. Which is why I wish to concentrate on using BibTeX.

A BibTeX database is stored as a .bib file. It is a plain text file, and so can be viewed and edited easily. The structure of the file is also quite simple. An example of a BibTeX entry:

\begin{verbatim}
@article{greenwade93,
    author  = "George D. Greenwade",
    title   = "The {C}omprehensive {T}ex {A}rchive {N}etwork ({CTAN})",
    year    = "1993",
    journal = "TUGBoat",
    volume  = "14",
    number  = "3",
    pages   = "342--351"
}
\end{verbatim}

Each entry begins with the declaration of the reference type, in the form of @type. BibTeX knows of practically all types you can think of, common ones are: book, article, and for papers presented at conferences, there is inproceedings. In this example, I have referred to an article within a journal.\sidenote{\obeylines 
book,
article,
conference
}

After the type, you must have a left curly brace '\{' to signify the beginning of the reference attributes. The first one follows immediately after the brace, which is the citation key. This key must be unique for all entries in your bibliography. It is this identifier that you will use within your document to cross-reference it to this entry. It is up to you as to how you wish to label each reference, but there is a loose standard in which you use the author's surname, followed by the year of publication. This is the scheme that I use in this tutorial.

Next, it should be clear that what follows are the relevant fields and data for that particular reference. The field names on the left are BibTeX keywords. They are followed by an equals sign (=) where the value for that field is then placed. BibTeX expects you to explicitly label the beginning and end of each value. I personally use quotation marks ("), however, you also have the option of using curly braces \verb+('{', '}')+. But as you will soon see, curly braces have other roles, within attributes, so I prefer not to use them for this job as they can get more confusing. 

A notable exception is when you want to use characters with umlauts (ü, ö, etc), since their notation is in the format \verb+\"{o}+, and the quotation mark will close the one opening the field, causing an error in the parsing of the reference.

Remember that each attribute must be followed by a comma to delimit one from another. You do not need to add a comma to the last attribute, since the closing brace will tell BibTeX that there are no more attributes for this entry, although you won't get an error if you do.

It can take a while to learn what the reference types are, and what fields each type has available (and which ones are required or optional, etc). So, look at this entry type reference and also this field reference for descriptions of all the fields. It may be worth bookmarking or printing these pages so that they are easily at hand when you need them.

\section{Authors}

BibTeX can be quite clever with names of authors. It can accept names in forename surname or surname, forename. I personally use the former, but remember that the order you input them (or any data within an entry for that matter) is customizable and so you can get BibTeX to manipulate the input and then output it however you like. If you use the forename surname method, then you must be careful with a few special names, where there are compound surnames, for example "John von Neumann". In this form, BibTeX assumes that the last word is the surname, and everything before is the forename, plus any middle names. You must therefore manually tell BibTeX to keep the 'von' and 'Neumann' together. This is achieved easily using curly braces. So the final result would be "John {von Neumann}". This is easily avoided with the surname, forename, since you have a comma to separate the surname from the forename.

Secondly, there is the issue of how to tell BibTeX when a reference has more than one author. This is very simply done by putting the keyword |and| in between every author. As we can see from another example:


\section{The natbib package}

Using the standard \latex bibliography support, you will see that each reference is numbered and each citation corresponds to the numbers. The numeric style of citation is quite common in scientific writing. In other disciplines, the author-year style, e.g., (Roberts, 2003), such as Harvard is preferred, and is in fact becoming increasingly common within scientific publications. A discussion about which is best will not occur here, but a possible way to get such an output is by the natbib package. In fact, it can supersede LaTeX's own citation commands, as |natbib| allows the user to easily switch between Harvard or numeric \docpkg{natbib}\citep{natbib2009}.


The first job is to add the following to your preamble:

\begin{verbatim}
\usepackage{natbib}
\end{verbatim}


The bibliography |.bib| file is still typed using the normal format as for example:---

\begin{verbatim}
@book{goossens93,
    author    = "Michel Goossens and Frank Mittlebach and Alexander Samarin",
    title     = "The LaTeX Companion",
    year      = "1993",
    publisher = "Addison-Wesley",
    address   = "Reading, Massachusetts"
}
\end{verbatim}



Also, you need to change the bibliography style file to be used, so edit the appropriate line at the bottom of the file so that it reads: |\bibliographystyle{plainnat}|. Once done, it is basically a matter of altering the existing \texttt{cite} commands to display the type of citation you want.


The main commands simply add a (t)  for 'textual' or (p) for 'parenthesized', to the basic \cs{cite} command. You will also notice how Natbib by default will compress references with three or more authors to the more concise 1st surname et al version. By adding an asterisk (*), you can override this default and list all authors associated with that citation. There are some other less common commands that Natbib supports, listed in the table here.

Using |natbib|, can satisfy every style required by a stern and difficult editor.

\begin{table}
\begin{tabular}{ll}
\toprule
Citation command	&Output\\
\midrule
\verb+ \citet{goossens93}+	&\citep{goossens93}\\
\verb+ \citep{goossens93}+	&\citep{goossens93}\\
\verb+ \citet*{goossens93}+	&\citet*{goossens93}\\
\verb+ \citep*{goossens93}+	&\citep*{goossens93}\\
\verb+ \citeauthor{goossens93}+	&\citeauthor{goossens93} \\
\verb+ \citeauthor*{goossens93}+	&\citeauthor*{goossens93}\\
\verb+ \citeyear{goossens93}+	&\citeyear{goossens93}\\
\verb+ \citeyearpar{goossens93}+	&\citeyearpar{goossens93}\\
\verb+ \citealt{goossens93}+	&\citealt{goossens93}\\
\verb+ \citealp{goossens93}+	&\citealp{goossens93}\\
\bottomrule
\end{tabular}
\caption{Natbib package commands}
\end{table}

When changing the bibliography style, sometimes natbib is upset because it can't interpret the data correctly.

In any case, after changing the argument to |\bibliographystyle| a run of LaTeX and one of BibTeX are necessary to get back in sync. Removing the |.bbl| and |.aux| files before those run is recommended, in order to avoid spurious error messages that might corrupt the .aux file currently being generated.\footnote{\url{http://tex.stackexchange.com/questions/54480/package-natbib-error-bibliography-not-compatible-with-author-year-citations}}

\section{Including URLs in bibliography}

As you can see, there is no field for URLs. One possibility is to include Internet addresses in howpublished field of @misc or note field of |@techreport|, |@article|,|@book|:

\begin{lstlisting}[language={[common]TeX},% 
                           alsolanguage={[LaTeX]TeX},% 
                           alsolanguage={[primitive]TeX},%
                           ]
howpublished = "\url{http://www.example.com}"
\end{lstlisting}

Note the usage of \cs{url} command to ensure proper appearance of URLs.
Another way is to use special field url and make bibliography style recognise it.

\begin{lstlisting}[language={[common]TeX},% 
                           alsolanguage={[LaTeX]TeX},% 
                           alsolanguage={[primitive]TeX},%
                           ]
URL = "http://www.example.com"
\end{lstlisting}

You need to use \texttt{usepackage{url}} in the first case or \texttt{usepackage{hyperref}} in the second case.
Styles provided by Natbib (see below) handle this field, other styles can be modified using |urlbst| program. Modifications of three standard styles (|plain|, |abbrv| and |alpha|) are provided with |urlbst|.

If you need more help about URLs in bibliography, visit FAQ of UK List of TeX.


\section{changing punctuation}

When I started using natbib I kept getting square barackets. Use
\begin{lstlisting}[language={[common]TeX},% 
                           alsolanguage={[LaTeX]TeX},% 
                           alsolanguage={[primitive]TeX},%
                           ]
    \bibpunct{(}{)}{;}{a}{,}{,}
    \bibliographystyle{plainnat}
\end{lstlisting}

\section{Error Checking}

You can check the file for errors by runing it through |bibTeX|. This will point database errors etc. 


\subsection{Entry Types}

Bibliography entries included in a .bib file are split by types. The following types are understood by virtually all |BibTeX| styles:

\subsubsection*{article}
  An article from a journal or magazine.

  Required fields: author, title, journal, year

  Optional fields: volume, number, pages, month, note, key

\emph{book}
   A book with an explicit publisher.
   Required fields: author/editor, title, publisher, year
   Optional fields: volume, series, address, edition, month, note, key

\emph{booklet}
   A work that is printed and bound, but without a named publisher or sponsoring institution.
   Required fields: title
   Optional fields: author, howpublished, address, month, year, note, key

\emph{conference}
   The same as inproceedings, included for Scribe compatibility.
   Required fields: author, title, booktitle, year
   Optional fields: editor, pages, organization, publisher, address, month, note, key

\emph{inbook}

    A part of a book, usually untitled. May be a chapter (or section or whatever) and/or a range of pages.
    Required fields: author/editor, title, chapter/pages, publisher, year
    Optional fields: volume, series, address, edition, month, note, key

\emph{incollection}

    A part of a book having its own title.
    Required fields: author, title, booktitle, year
    Optional fields: editor, pages, organization, publisher, address, month, note, key

\emph{inproceedings}

An article in a conference proceedings.
Required fields: author, title, booktitle, year
Optional fields: editor, series, pages, organization, publisher, address, month, note, key

\emph{manual}

Technical documentation.
Required fields: title
Optional fields: author, organization, address, edition, month, year, note, key

\emph{mastersthesis}

A Master's thesis.
Required fields: author, title, school, year
Optional fields: address, month, note, key

\emph{misc}

For use when nothing else fits.

Required fields: none
Optional fields: author, title, howpublished, month, year, note, key

\emph{phdthesis}

A Ph.D. thesis.

Required fields: |author|, |title|, |school|, |year|\\
Optional fields: |address|, |month|, |note|, |key|
proceedings
The proceedings of a conference.
Required fields: title, year
Optional fields: editor, publisher, organization, address, month, note, key
techreport
A report published by a school or other institution, usually numbered within a series.
Required fields: author, title, institution, year
Optional fields: type, number, address, month, note, key
unpublished
A document having an author and title, but not formally published.
Required fields: author, title, note
Optional fields: month, year, key

\section{The bibentry package}

 This package allows one to be able to place bibliographic entries anywhere
 in the text. It is to be used to produce annotated bibliographies, such as
 \begin{quote}
   For an intoduction to this topic, see Jones, J.~R., Basics on this topic,
   {\it J.\ Last Resorts}, \textbf{13}, 234--254, 1994. For more advanced
   information, see \dots.
 \end{quote}

 The idea is that the full reference is used, not just the citation Jones
 [1994].

 \section{Invoking the Package}
 The macros in this package are included in the main document
 with the |\usepackage| command of \LaTeXe,
 \begin{quote}
 |\documentclass[..]{...}|\\
 |\usepackage{|\texttt{\filename}|}|
 \end{quote}

 \section{Usage}

 \newcommand\btx{\textsc{Bib}\TeX}
 This package must be used with \btx, not with a hand-written
 \texttt{thebibliography} environment.

 More precisely, there must be a \texttt{.bbl} file external to the \LaTeX\
 file; whether this is written by hand or by BibTeX is unimportant.

| \nobibliography|
 The bibliography entries are stored with the command
 |\nobibliography| |\marg{bibfiles}|, which is like the usual
 |\bibliography| |\marg{bibfiles}| except no bibliography is printed. The
 \texttt{.bbl} file is read in as usual but the \texttt{thebibliography} is
 redefined so that all the entries are stored, not printed.


 The text of the entries may be printed with the command
 \begin{quote}
    |\bibentry| |\marg{key}|
 \end{quote}

 These commands may only be issued after |\nobibliography|, for otherwise
 the reference texts are not known.

 The final period of the original text will be missing, so that one can add
 punctuation as one pleases.

 Regular |\cite| (or the \texttt{natbib} versions) may be issued anywhere as
 usual.

|\nobibliography*|
 If a regular list of references is to be given too, with the
 |\bibliography|\sidenote{bibfiles} command, issue the starred version
 |\nobibliography*| (without argument) in order to store the bib entry texts.
 This will load the same \texttt{.bbl} file as |\bibliography|, but will avoid
 messages from BibTeX about multiple |\bibdata| commands and warnings from
 \LaTeX\ about multiply defined citations.

 The processing procedure is as usual:
 \begin{enumerate}
  \item \LaTeX\ the file;
  \item Run \btx;
  \item \LaTeX\ the file twice.
 \end{enumerate}

 \noindent
 \textbf{Note:} it is highly recommended to make use of the \docpkg{url}
 package, which will nicely format both |url| and |doi| addresses; in particular,
 they will break at convenient locations without a hyphen.\index{bibliography>doi}
\index{bibliography>url}




Here are some useful references about \LaTeX. They are
available in every worthy bookshop. Many other good documentations
might be found on the web (the FAQ of \textsf{comp.text.tex} for
instance).


\begin{verbatim}
\bibitem[GMS93]{companion} Michel Goossens, Franck Mittelbach and Alexander
Samarin, \emph{The \LaTeX{} Companion}, Addison Wesley, 1993.
\bibitem[Lam97]{lamport} Leslie Lamport, \emph{\LaTeX: A Document Preparation
System}, Addison Wesley, 1997.
\end{verbatim}

This is the main matter of the document, mentioning
[\ref{doc1}] and [\ref{doc2}], for instance.



% \fi

%  
%\iffalse
%   \includepdf[pages=1]{grid.pdf}
%\fi
%  \iffalse
%  \@specialfalse
%  \makeatletter\@specialfalse\makeatother
%%%%%%%%%%%%%%%%%%%%%%%%%%%%%%%%%%%%%%%%%%%
%%%%%%  STYLE 01
%%%%%%%%%%%%%%%%%%%%%%%%%%%%%%%%%%%%%%%%%%%


\cxset{
 name={},
 numbering=arabic,
 number font-size=\LARGE,
 number font-family=\rmfamily,
 number font-weight=\bfseries,
 number before=,
 number dot=,
 number after=,
 number position=leftname,
 chapter font-family=\sffamily,
 chapter font-weight=\normalfont,
 chapter font-size=\Large,
 chapter before={\vspace*{20pt}\par},
 chapter after={\hfill\hfill\par},
 chapter color={black!90},
 number color=\color{purple},
 title beforeskip={\vspace*{30pt}},
 title afterskip={\vspace*{40pt}\par},
 title before={},
 title after={},
 title font-family=\sffamily,
 title font-color=\color{purple},
 title font-weight=\bfseries,
 title font-size=\LARGE,
 header style=headings}

\cxset{headings ruled-01}

\chapter{Introduction to Style One}


\begin{summary}
This design is simple and its distinguishing characteristic is a short summary at the beginning of the chapter. This is almost like an abstract typeset in italic font without setting the margins in. We provide a \lstinline{summary} environment for convenience. Note the very simple line in the running head to the left of the page number.
\end{summary}

\medskip
\begin{figure}[ht]
\centering
\includegraphics[width=0.5\textwidth]{./chapters/chapter01}
\end{figure}


%  \clearpage
\makeatletter\@debugtrue\makeatother
\cxset{
 chapter toc=true,
 name=CHAPTER,
 chapter numbering=ORDINALS,
 number font-size=Large,
 number font-family=rmfamily,
 number font-weight=bfseries,
 number before=\kern0.5em,
 number dot=,
 number after=\hfill\hfill\par,
 number position=rightname,
 chapter font-family=rmfamily,
 chapter font-weight=bold,
 chapter font-size=Large,
 chapter before={\vspace*{20pt}\par\hfill},
 chapter after=,
 chapter color=black,
 number color=black,
 %
 title margin top=10pt,
 title before=\par\nointerlineskip\hfill,
 title after=\hfill\hfill\par\nointerlineskip,
 title font-family=rmfamily,
 title font-color= black,
 title font-weight=bfseries,
 title font-size=LARGE,
 chapter title width=0.8\textwidth,
 chapter title align=centering,
 title margin-left=0pt,
 author block=false}

\debugtitle
\debugchapter
\chapter[Template 2]{Mondino, the Restorer of Anatomy}

The archive.org is an extraordinary hunting ground  for typographical surprises. On a recent excursion to find some books on Versalius I stubled on a book titled \emph{Andreas Vesalius, the reformer of anatomy} by  Ball, James Moores. It is an old book published in 1910 and has a couple of unusual features. Check the figure below and see if you can identify the challenging feature.

\begin{figure}[ht]
\centering
\includegraphics[width=0.8\textwidth]{versalius}
\caption{J.B. Moore’s \emph{Andreas Versalius, the Reformer of Anatomy} has many unusual features, including chapter numbers using ordinals. }
\end{figure}

\cxset{chapter toc=true,
          chapter opening=anywhere}
          
\chapter{The Template}          
The template is called \emph{Versalius} and is stored under style02. It can be loaded in the normal way using:
\begin{verbatim}
\usepackage[style02]{phd}
\end{verbatim}

I have not reproduced the full extend of the book’s requirements, as some details are quite cumbersome to be automated through \tex. These though can easily be incorporated in a manual way. More about this later.


\section{The Table of Contents}
Another interesting aspect of this book, which is common with many books of its period is the ToC. The ToC shows the full range of the chapter pages, i.e., it is marked as Page 1-16 rather than the common practice nowdays that indicates only the starting page of the chapter. It also has “TABLE OF CONTENTS”  as a heading and not just contents as you would expect from today’s books.

\begin{figure}[ht]
\centering
\includegraphics[width=0.8\textwidth]{versalius-01}
\caption{J.B. Moore’s \emph{Andreas Versalius, the Reformer of Anatomy} has many unusual features, including chapter numbers using ordinals. }
\end{figure}

\section{List of Illustrations}

\begin{figure}[ht]
\centering
\includegraphics[width=0.8\textwidth]{versalius-02}
\caption{J.B. Moore’s \emph{Andreas Versalius, the Reformer of Anatomy} has many unusual features, including chapter numbers using ordinals. }
\end{figure}

\section{The Frontmatter}
As a foreward there is an unumbered chapter called ``Introduction’’. The chapter heading also has a head band.
\begin{figure}[ht]
\centering
\includegraphics[width=0.8\textwidth]{versalius-03}
\caption{J.B. Moore’s \emph{Andreas Versalius, the Reformer of Anatomy} has many unusual features, including chapter numbers using ordinals. }
\label{lettrine}
\end{figure}

\bgroup
\centering
\includegraphics[width=0.7\textwidth]{versalius-headband}

\LARGE\bfseries INTRODUCTION\par
\egroup
\def\dropcapversalius{%
\vbox to 0pt{\vskip6pt\leavevmode\noindent\includegraphics[width=2.39cm]{versalius-dropcap}%
}%
}
\parindent0pt

\hangindent2.6cm \hangafter0
\dropcapversalius \textsc{he dropcap will have to be inserted}, either using the lettrine package or do be achieved via a parshape command and manual entry. You can also write your own macro command using the details we provide under the Paragraphs chapter. On this page I have manually inserted it, as I used an image from the book for the dropcap. If you were to use the template for a full book, it will be then preferable to use

the lettrine package to set the dropcaps. If you observe Figure~\ref{lettrine} carefully, you will notice the first line of theopening paragraph is in small caps. As \tex typesets the full paragraph this is almost an impossible task to achieve through normal \tex commands and in order not to overcomplicate the discussion it can be achieved manually via trial and error. 

\section{Figures}

Most of the figures are wrapped illustrations. A couple are full page figures and bear no caption numbering. One such illustration is shown on page~\pageref{fig:vesalius}. Do note that the List of Illustrations does have the illustrations listed with additional information to that shown in the captions. 

\begin{figure}[p]
\centering
\includegraphics[width=\textwidth]{vesalius}
\centering
ANDREAS VESALIUS\par
(From an old copperplate engraving)\par
\label{fig:vesalius}
\end{figure}







%  \clearpage
\cxset{style04/.style={
 numbering=Roman,
 number font-size=\Large,
 number font-family=\rmfamily,
 number font-weight=\bfseries,
 number before=,
 number dot=,
 number after=,
 number position=rightname,
 chapter font-family=\rmfamily,
 chapter font-weight=\normalfont,
 chapter font-size=\Large,
 chapter before={\vspace*{20pt}\par\hfill},
 chapter after={\hfill\hfill\par\vspace*{10pt}},
 chapter color={black!90},
 number color=purple,
 title beforeskip={},
 title afterskip={\vspace*{50pt}\par},
 title before={\hfill},
 title after={\hfill\hfill\par},
 title font-family=\rmfamily,
 title font-color= purple,
 title font-weight=\normalfont,
 title font-size=\LARGE,
 section numbering=none,
 section align = center}}

\cxset{style04}

\chapter{INTRODUCTION TO STYLE FOUR}

This is a very simple design applicable perhaps to translations and commentary on older texts.
\medskip
\begin{figure}[ht]
\centering
\includegraphics[width=0.6\textwidth]{./chapters/chapter04.png}
\end{figure}

%  
\cxset{style05/.style={
 name={Chapter},
 chapter color = magenta,
 chapter toc = true,
 numbering=arabic,
 number font-size=\Large,
 number font-family=\rmfamily,
 number font-weight=\normalfont\itshape,
 number color= purple,
 number before=\hspace*{-15pt},
 number dot=,
 number after=,
 number position=rightname,
 chapter font-family=sffamily,
 chapter font-weight= \bfseries\itshape,
 chapter font-size=\Large,
 chapter before={\hrule width \columnwidth \kern12.6pt \par\hfill},
 chapter after={\hfill\hfill\par},
 chapter color={magenta},
 chapter spaceout=none,
 title beforeskip={\vspace*{10pt}},
 title afterskip={\vspace*{30pt}\par},
 title before={\hfill},
 title after={\hfill\hfill \vskip12.6pt\hrule width \columnwidth \kern2.6pt },
 title font-family=\rmfamily,
 title font-color=black!90,
 title font-weight=\bfseries,
 title font-size=\huge,
 title font-shape = normal,
 header style= headings}}

\cxset{style05}
\chapter{Introduction to Style Five}\index{ch:style5}

\tcbset{width=\textwidth}
I think this style can be improved with a bit of color. You can experiment with it quite easily. The spacing on top of this style can also be adjusted to suit your typographical taste.
\medskip
\begin{figure}[ht]
\centering
\includegraphics[width=0.6\textwidth]{./chapters/chapter05}
\end{figure}

%\section{General notes on rules}

LaTeX's default rules would normally give problems. Best is to use TeX's primitives to built them.

\index{rules!example color}

\begin{texexample}{}{}
\makeatletter
\hrule width 5cm \kern2.6\p@
AAAAAAAAAAAAAAAAAAAAA
\vskip2.6pt\hrule width 5cm
\medskip

Problem with LaTeX rules.

\rule{5cm}{0.4pt}\par
AAAAAAAAAAAAAAAAAAAAA\par%
\rule[6.5pt]{5cm}{0.4pt}

\def\rule{\@ifnextchar[\@rule{\@rule[\z@]}}
\def\@rule[#1]#2#3{%
 \leavevmode
 \hbox{%
 \setlength\@tempdima{#1}%
 \setlength\@tempdimb{#2}%
 \setlength\@tempdimc{#3}%
 \advance\@tempdimc\@tempdima%
 \vrule\@width\@tempdimb\@height\@tempdimc\@depth-\@tempdima}}

\def\thickrule{\leavevmode \leaders \hrule height 3pt \hfill \kern \z@}

{\color{teal}\hrule width 10.5cm height3pt \kern2.6\p@
    {{\color{black!80}\HUGE CHAPTER TITLE}}\vskip3pt
\hrule width 10.5cm height3pt}
\makeatother
\end{texexample}

%  % Requires the package calligra
\newfontfamily{\ovidius}{ovidius demi}
\cxset{style06/.style={%
 chapter opening=anywhere,
 chapter name=Chapter,
 chapter numbering=arabic,
 chapter number font-size=LARGE,
 chapter label font-family =ovidius,
 chapter number font-weight = bfseries,
 chapter number color= black!60,
% chapter number before=\kern3.5pt,
% chapter number dot=,
 %number after=\hfill\hfill\par\offinterlineskip,
% number position=rightname,
 chapter label font-family= ovidius,
 chapter label font-shape=itshape,
 chapter label font-weight=normal,
 chapter label font-size= LARGE,
% chapter before=\vspace*{2pt}\par\hfill,
% chapter after=,
 chapter label color=black!60,
% chapter spaceout=none,
 chapter title margin top=30pt,
 %title before=\hfill\par,
% title after=\hfill,
 chapter title font-family=ovidius,
 chapter title font-color= black!90,
 chapter title font-weight=normalfont,
 chapter title font-size=LARGE,
 %title spaceout=none,
 chapter title width=0.6\textwidth,
 chapter title align=centering,
 }}

\cxset{style06}
\cxset{chapter label font-face= ovidius}
\cxset{chapter format=traditional}

\chapter{THE RITUALS OF THE MONTHS OF THE YEAR}
\renewcommand{\DefaultLhang}{0.1}
\renewcommand{\LettrineFontHook}{\calligra}
\setlength{\DefaultFindent}{9.5pt}
\setlength{\DefaultNindent}{0pt}
\renewcommand{\LettrineFontHook}{\ovidius}
\lettrine[loversize=0.6]{\textcolor{thegray!60}{C}}{}rist\'obal de Molina’s manuscript titled \emph{Account of the Fables and Rites of the Incas (Relación
de las fábulas y ritos de los incas)}, written aroun 1575 records the rituals that were conducted in Cuzco during the last years of the Inca Empire. An excellent translation was published by
\begin{figure}[ht]
\centering
\includegraphics[width=0.45\textwidth]{./chapters/chapter06.png}
\caption{Style 5 sample}
\end{figure}
the University of Texas Press. The translation is by Brian S. Bauer, Vania Smith-Oka and Gabriel E. Cantarutti who did an excellent job. The typesetting attracted my attention by its effective simplicity and I read the book in one evening. The book size is 5.50 x 8.50 in, pages 187. Times Roman, ArnoPro and Ovidius. The Ovidius font was designed by  Thaddeus Szumilas and 
belongs to a family known as eroded fonts. It has found many devoted users especially for book covers.

This template has a lot of potential and I will come back to it and add more key hooks for lettrine settings per letter and font management. They can also come alive with a gold color.

The dropcap in the original book as well as the chapter font is given a worn style (Ovidius Demi font\footnote{Available at the fontpalace website \protect\url{http://www.fontpalace.com/font-download/Ovidius Demi/}}), I guess in order to give it a touch of style reminiscent of a manuscript. It can also look
good using |calligra| and also a bit of color. You can experiment also with many other calligraphic fonts. The example below demonstrates the use of the |calligra| font.
\bigskip

\renewcommand{\LettrineFontHook}{\calligra}
\cxset{chapter label font-family=calligra}
\cxset{chapter number font-family=calligra}
\cxset{chapter number font-weight=calligra}
\cxset{chapter number font-size=Huge}
%\cxset{chapter color=black!90,
\cxset{chapter number color=black!90}
\bigskip

\cxset{chapter opening=any}

\chapter{OF QUIPUS AND INCA YUPANQUI}

\lettrine[loversize=.6]{\textcolor{orange}{T}}{}he book by Crist\'obal de Molina’s manuscript titled \emph{Account of the Fables and Rites of the Incas (Relación
de las fábulas y ritos de los incas)}, written aroun 1575 records the rituals that were conducted in Cuzco during the last years of the Inca Empire. An excellent translation was published by
\medskip

The dropcap looks as good if not better with the |calligra| font and I have given it a colour to stand out. The chapter number has to be increased in height, so I have used |huge|. The new
settings are shown below:

\begin{verbatim}
\renewcommand{\LettrineFontHook}{\calligra}

\cxset{chapter opening=any}
\cxset{chapter label font-family=calligra}
\cxset{chapter number font-family=calligra}
\cxset{chapter number font-weight=calligra}
\cxset{chapter number font-size=Huge}
\cxset{chapter number before=\kern2.5pt}
\cxset{chapter label color=black!90,
      chapter number color=black!90}
\end{verbatim}

\cxset{section align=center,
          section numbering=none,
          section font-weight=normalfont,
          section font-family=rmfamily,
          section font-size=large,
          section color=black,
          section font-shape=scshape}


\section{THE SECTIONS}

The sections are typeset in normal font and are centered.

{\ovidius \lorem}

\bottomline 
%  <<<<<<< HEAD

\newgeometry{top=2cm,bottom=2cm,left=3cm,right=3cm}
%%%%%%%%%%%%%%%%%%%%%%%%%%%%%%%%%%%%%%%%%%%
%%%%%%  STYLE 07
%%%%%%%%%%%%%%%%%%%%%%%%%%%%%%%%%%%%%%%%%%%

\cxset{style07/.style={
 name={},
 numbering=arabic,
 number font-size=\Huge,
 number font-family=\rmfamily,
 number font-weight=\bfseries,
 number before=,
 number dot=,
 number color=\color{gray},
 number after=\par,
 number position=rightname,
 chapter font-family=\sffamily,
 chapter font-weight=\normalfont,
 chapter font-size=\Large,
 chapter before={\hfill\hfill\par},
 chapter after={},
 chapter color={black!90},
 title beforeskip={\vspace*{30pt}},
 title afterskip={\vspace*{50pt}\par},
 title before={},
 title after={\par\rule[13pt]{\textwidth}{0.4pt}},
 title font-family=\sffamily,
 title font-color=\color{purple},
 title font-weight=\bfseries,
 title font-size=\LARGE,
 title spaceout=none,
}}

\cxset{style07}
\chapter{Introduction to Style Seven}

\parindent0pt
\lipsum[1]
\medskip
\begin{figure}[ht]
\centering
\includegraphics[width=0.6\textwidth]{./chapters/chapter07}
\end{figure}
\lipsum[1]
=======

\newgeometry{top=2cm,bottom=2cm,left=3cm,right=3cm}
%%%%%%%%%%%%%%%%%%%%%%%%%%%%%%%%%%%%%%%%%%%
%%%%%%  STYLE 07
%%%%%%%%%%%%%%%%%%%%%%%%%%%%%%%%%%%%%%%%%%%

\cxset{style07/.style={
 name={},
 numbering=arabic,
 number font-size=\Huge,
 number font-family=\rmfamily,
 number font-weight=\bfseries,
 number before=,
 number dot=,
 number color=\color{gray},
 number after=\par,
 number position=rightname,
 chapter font-family=\sffamily,
 chapter font-weight=\normalfont,
 chapter font-size=\Large,
 chapter before={\hfill\hfill\par},
 chapter after={},
 chapter color={black!90},
 title beforeskip={\vspace*{30pt}},
 title afterskip={\vspace*{50pt}\par},
 title before={},
 title after={\par\rule[13pt]{\textwidth}{0.4pt}},
 title font-family=\sffamily,
 title font-color=\color{purple},
 title font-weight=\bfseries,
 title font-size=\LARGE,
 title spaceout=none,
}}

\cxset{style07}
\chapter{Introduction to Style Seven}

\parindent0pt
\lipsum[1]
\medskip
\begin{figure}[ht]
\centering
\includegraphics[width=0.6\textwidth]{./chapters/chapter07}
\end{figure}
\lipsum[1]
>>>>>>> merged

%  \clearpage

\setdefaults

\cxset{style08/.style={
 name={},
 chapter toc=true,
 numbering=arabic,
 number font-size=\LARGE,
 number font-family=\sffamily,
 number font-weight=\bfseries,
 number color= black!90,
 number before=,
 number dot=,
 number after=,
 number position=rightname,
 chapter font-family=\sffamily,
 chapter font-weight=\normalfont,
 chapter font-size=\Large,
 chapter before={\vspace*{20pt}\hfill},
 chapter after={\vspace{20pt}\par},
 chapter color={black!90},
 title beforeskip={},
 title afterskip={\vspace*{50pt}\par},
 title before={\hfill\hfill\raggedleft},
 title after={},
 title font-family=\sffamily,
 title font-color=black!90,
 title font-weight=\bfseries,
 title font-size=\LARGE,
 author block=true,
 author block format=\par\addvspace{12pt}\normalfont\large\raggedleft,
 author names=Yiannis Lazarides\par Larnaka,
 header style=empty}}

\cxset{style08}
\chapter{Introduction to Chapter Style Eight}

\lipsum[1]
\medskip
\begin{figure}[ht]
\centering
\includegraphics[width=0.5\textwidth]{./chapters/chapter08.png}
\end{figure}
\lipsum[1]

%  \cxset{author block=false}
\clearpage

\cxset{
 name={},
 numbering=arabic,
 number font-size=\LARGE,
 number font-family=\rmfamily,
 number font-weight=\bfseries,
 number before=,
 number dot=.,
 number after=\hspace{1em},
 number position=rightname,
 chapter font-family=\sffamily,
 chapter font-weight=\normalfont,
 chapter font-size=\Large,
 chapter before={\vspace*{20pt}\par\hfill},
 chapter after={},
 chapter color={black!90},
 number color= purple,
 title beforeskip={},
 title afterskip={\vspace*{50pt}\par},
 title before={},
 title after={},
 title font-family=\sffamily,
 title font-color= purple,
 title font-weight=\bfseries,
 title font-size=\LARGE}
\chapter{Introduction 09}
\lipsum[1]

\medskip
\begin{figure}[ht]
\centering
\includegraphics[width=0.8\textwidth]{./chapters/chapter09}
\end{figure}

\textit{In preparation. Patience!}

%  \makeatletter
\cxset{
 chapter opening=right,
 chapter toc=false,
 name=CHAPTER,
 numbering= WORDS, %WORDS gives errors
 number font-size=huge,
 number font-family=sffamily,
 number font-weight=bfseries,
 number before=\kern1em,
 number dot=,
 number after=,
 number position=rightname,
 % set chapter fonts 
 chapter font-family=sffamily,
 chapter font-weight=bfseries,
 chapter font-size=huge,
 chapter margin top=5cm,
 chapter margin left=0pt,
 chapter before=\par\hfill,
 chapter after=,
 chapter color=black,
 chapter spaceout=none,
 chapter title align=center,
 chapter afterindent=true,
 number color=black,
% chapter titles
 title margin top=30pt,
 title margin bottom=30pt,
 chapter title width=\textwidth,
 chapter title text-align=center,
 title font-family=sffamily,
 title font-color=black,
 title font-weight=bfseries,
 title font-size=huge,
 title font-shape=upshape,
 title before=,
 title after=,
% sections 
 section font-size=LARGE,
 section font-weight=normalfont,
 section font-family=sffamily,
 section color=black,
 section align=centering,
 section numbering=none,
 section indent=-1em,
 section beforeskip=20pt,
 section afterskip=10pt,
 section spaceout=soul,
 section font-shape=,
 pagestyle = plain,
 subsection color=black,
}

\chapter{Introduction to Style 10}

\addcontentsline{toc}{section}{Template 10 (style10)}

This style is very similar to the |verso chapter| style. I have reproduced it as close as possible to the book that gave me the inspiration titled \emph{Mind Machines}.

\begin{figure}[htb]
\centering
\fboxrule1pt
\fbox{\includegraphics[width=0.8\textwidth]{./chapters/chapter10}}
\caption{Style ten example.}
\end{figure}

Another interesting aspect is that subsections are centered and have a colon at the end of the subsection title. The setting for this is the option \lstinline{numeric=WORDS}. Use either a capital for uppercase or \lstinline{numeric=words} for lowercase number labels.

\cxset{chapter toc=true,
          chapter margin top=0pt}
\makeatother

%  \cxset{style11/.style={
 chapter opening=any,
 name=Chapter,
 numbering=arabic,
 number font-size=LARGE,
 number font-family=rmfamily,
 number font-weight=bfseries,
 number before=,
 number dot=,
 number after=,
 number before=\kern0.5em,
 number display=inline,
 number float=center,
 chapter display=block,
 chapter float=center,
 chapter font-family=rmfamily,
 chapter font-weight=bfseries,
 chapter font-size=LARGE,
 chapter before=,
 chapter after=,
 chapter color=black!90,
 chapter spaceout=none,
 chapter border-width=0pt,
 chapter border-style=none,
 number color=black!90,
 title beforeskip=,
 title afterskip=,
 title before=,
 title after=,
 title font-family=rmfamily,
 title font-color=black!90,
 title font-weight=bfseries,
 title font-size=LARGE,
 chapter title width=\textwidth,
 chapter title align=centering,
 section afterindent=true,
 section align=left,
 section numbering=arabic,
 section numbering prefix=\thechapter.,
 section numbering suffix=\space,
 section indent=0pt,
 section font-family=rmfamily,
 }}
\renewsection\renewsubsection

\cxset{style11}
\chapter{\textit{Elements} II and Babylonian Metric Algebra, Introduction to Style Eleven}

The origins of Greek Mathematics, according to the Greeks is Egypt and according to J\"oran Friberg is Babylonia. This template is based on Friberg's book \emph{Amazing Traces of a Babylonian Origin in Greek Mathematics}. The book was published by World Scientific in 2007. The book size is $5.97\times8.88$ inches and uses a variety of fonts, with the main document font in Times. 

\medskip
\begin{figure}[ht]
\centering
\fbox{\includegraphics[width=0.65\textwidth]{./chapters/chapter11.png}}
\end{figure}
\lipsum[1]

\section{Indentation}

The book follows swedish traditional typography with the paragraphs following subheadings indented. This is achieved in the template using:

\begin{verbatim}
\cxset{section afterindent=true}
\end{verbatim}

\section{Images}
\indent Images and their captions follow a \latexe style and I am sure the book must have been styled using a \latexe xml clone as the book's pdf was produced with iText\footnote{\url{http://itextpdf.com/}}.

\begin{figure}[ht]
\centering
\includegraphics[width=0.8\textwidth]{greekmaths}
\caption{Extract from the \textit{Amazing Traces of Babylonian Influence in Greek Mathematics.} Note the styling of the caption.}
\end{figure}

\testsections

% reset for following chapters
\cxset{section afterindent=false}


%  \cxset{style12/.style={
 chapter name=,
 chapter toc=true,
 chapter numbering=arabic,
 number font-size=\HUGE,
 number font-family=\rmfamily,
 number font-weight=\bfseries,
 number before=,
 number dot=,
 number color= gray,
 number after=\par,
 number position=rightname,
 chapter font-family=\sffamily,
 chapter font-weight=\normalfont,
 chapter font-size=\Huge,
 chapter before={\hfill\hfill\hfill\par},
 chapter after={},
 chapter color={black!90},
 title beforeskip={\vspace*{0pt}},
 title afterskip={\vspace*{50pt}\par},
 title before={},
 title after={\par\vspace{10pt}\rule{\textwidth}{4pt}},
 title font-family=\sffamily,
 title font-color=black!90,
 title font-weight=\bfseries,
 title font-size=\HUGE,
 title font-shape=normal,
 title spaceout=none,
}}

\cxset{style12}
\chapter{Introduction to Style Twelve}

This is a variation of Style 7, with only the lettering and the rule are thicker. In my opinion it looks better with a bit of color, so I have used a purple color with a gray.

\medskip
\begin{figure}[ht]
\centering
\includegraphics[width=0.35\textwidth]{./chapters/chapter12.png}
\end{figure}
\lipsum[1]
\chapter{Second Chapter}
^^A
%  \newgeometry{left=7cm,right=3cm,
 marginparsep=15pt, marginparwidth=4.2cm,top=2cm}

\parindent0pt
\cxset{
 name=,
 numbering=none,
 number font-size=LARGE,
 number font-family=\rmfamily,
 number font-weight=\bfseries,
 number before={\parindent0pt },
 number dot=,
 number after=,
 number position=leftname,
 chapter font-family=,
 chapter font-weight=,
 chapter font-size=,
 chapter before=,
 chapter after=,
 chapter color=blue,
 number color=blue,
 title beforeskip={\parindent0pt},
 title afterskip={\vspace*{50pt}\par},
 title before={},
 title after={},
 title font-family=rmfamily,
 title font-color=black!80,
 title font-weight=normalfont,
 title font-size=Huge,
 title font-shape=itshape,
 chapter opening=any}
\chapter{Introduction to Style Fifteen}

\parindent1em
\def\thefigure{\arabic{chapter}.\arabic{figure}}
\lorem\par

\marginpar{%
 {\centering
 \includegraphics[width=4.2cm]{./chapters/chapter15.png}\vskip5pt\par}
 {\footnotesize\protect\lorem}
}
\marginpar{%
{\centering
\includegraphics[width=4.2cm]{./chapters/chapter15.png}\par}
 
}

This is another marginpar of the same size.

\lorem

\lipsum
\marginpar{%
{\centering
\includegraphics[width=4.2cm]{./chapters/chapter15.png}\par}
}

^^A
%  \restoregeometry

\cxset{
 name={CHAPTER},
 numbering=arabic,
 number font-size=\Large,
 number font-family=\rmfamily,
 number font-weight=\normalfont,
 number before=\kern0.5em,
 number after=\hfill\hfill\par\vspace*{20pt}\centerline{\decoone}\vspace*{20pt},
 number dot={},
 number position=rightname,
 name=CHAPTER,
 chapter font-family=rmfamily,
 chapter font-weight=mdweight,
 chapter font-size=Large,
 chapter before={\vspace*{20pt}\par\hfill},
 chapter after={},
 chapter color=black!90,
 number color=black!90,
 chapter title align=center,
 chapter title text-align=center,
 title margin-left=0pt,
 title margin bottom=50pt,
 title margin top=30pt,
 title before=,
 title after=,
 title font-family=rmfamily,
 title font-shape=upshape,
 title font-color= black!90,
 title font-weight=\normalfont,
 title font-size=Huge,
 title display=block}

\chapter[Style 18]{Chapter Style Eighteen}

\parindent0pt
This design introduces an ornament. There are a number of packages on ctan that provide ornaments. If you using XeLaTeX it is also possible to use system fonts. The ornament is introduced with the key number after. At this point also we introduced all the vertical skips.
\medskip
\begin{figure}[ht]
\centering
\includegraphics[width=0.45\textwidth]{./chapters/chapter18.png}
\end{figure}

\section{Sections}
\lorem

\subsection{Subsections}
\lorem


\subsubsection{Subsubsections}
\lorem

\parindent3em
\newcommand{\wb}[2]{\fontsize{#1}{#2}\usefont{U}{webo}{xl}{n}}
\newcommand{\showb}[1]{\wb{12}{14}#1}
\newfontfamily{\minion}{MinionPro-Regular.otf}
\def\ornament{{\minion \char"2740}}

\cxset{chapter name=,
          epigraph align=center,
          epigraph text align=center,
          epigraph rule width=0pt,
          title margin top=10pt,
          number font-size=small,
          number after=\hfill\hfill\par\vspace*{5pt}\centerline{\showb{[]}}\vspace*{5pt},
           %number after=\hfill\hfill\par\vspace*{5pt}\centerline{\Large\ornament}\vspace*{5pt},
          }
\chapter{THE IMPRESSIONISTS IN NEW YORK}


\epigraph{\ldots\itshape a pile of unsung treasures \ldots}{}
\minion

\lettrine{O}{n 13 March 1886, Paul Durant-Ruel and his young son Charles were travelling} through the streets of Paris, on their way to Gare Gare Saint-Lazare. In the two decades since Paul had
inherited his father’s business, Paris had been transformed. Haussmann had
realized his dream. The city was only three years away from the
Exposition of 1889 and the erection of the new Eiffel Tower, the symbol
of modern Paris. By 1890, Baron Haussmann would be saying of his newly
created capital of Europe, ‘these days, it’s fashionable to admire old Paris,
which people only know about from books’. Some areas of Paris had
hardly changed: the poor still lived in the shacks of Montmartre or the
shanties of Belleville; there were still cholera, typhoid, deaths in childbirth
and infant mortality. But to the uninitiated, those problems were now
hidden from view. Paris had a new image: the new Republic was
streamlined and stylish, the epitome of healthy living and good taste.
Haussmann’s Paris was architecturally modern, stratified by wealth,
quintessentially urban and, above all, commercially prosperous.

\begin{figure}[ht]
\centering
\fbox{%
\includegraphics[width=0.8\textwidth]{impressionist-lives}}
\caption{Spread from the Book \emph{The Private Lives of the Impressionists} by Sue Rose and published by Harper Collins.}
\end{figure}

The ctan repository has two good packages for ornamental fonts \pkgname{webomints} and \pkgname{fourier-orns}. The one shown in the orgininal publication is from Minion Symbols Pro.

They have been inserted in the template by using the |number after| key and a custom command from the \pkgname{webomints}
\bigskip

\begin{scriptexample}{}{}
\begin{verbatim}
\newcommand{\wb}[2]{\fontsize{#1}{#2}\usefont{U}{webo}{xl}{n}}
\newcommand{\showb}[1]{\wb{12}{14}#1}
\end{verbatim}
\end{scriptexample}


\ornament

\let\oldsection\section
\long\def\section{%
\par\medskip
\addvspace{20pt}
\centerline{{\LARGE *}}%
\addvspace{20pt}}


Cézanne wanted nothing to do with any war. Taking Hortense with him,
he left their garret at 53, rue Notre-Dame-des-Champs, and made for Aix.
Zola, who had recently married, returned to Provence with his wife,
heading from there to Marseilles. Monet, still in Trouville, waited for the
time being to see how events would turn out. Degas, Renoir, Bazille and
Manet, who stayed behind, were all eligible to fight

Cézanne had been working right up to the last minute to meet the 1866
Salon deadline. On the last possible day for submitting, a wheelbarrow
arrived outside the Palais de l’Industrie, pushed and pulled by Cézanne
and Oller, his Cuban friend from Suisse’s. Cézanne rushed to unwrap his
paintings, eager to show them to anyone who wanted to see. But by now
his hopes were not particularly high. When both his paintings were
rejected he was hardly surprised. He headed straight back to Aix,
complaining to Pissarro about the ‘rotten’ family he was being forced to
rejoin, all of them ‘boring beyond measure’. 

\section

Sections are marked with a single asterisk like ornament. This is a common element
in many non-fiction as well as fiction books. Some might have anything from on to three
asterisks. Many books printed in the nineteenth century have very fancy end section ornamentation.
I like the simplicity of the one asterisk.

\let\section\oldsection
\cxset{title display=in-line block}


^^A
%  
\cxset{style19/.style={
 name={},
 numbering=arabic,
 number font-size=Huge,
 number font-family=rmfamily,
 number font-weight=bfseries,
 number before=\par\offinterlineskip,
 number after=\kern0.5em,
 number dot={ },
 number position=rightname,
 chapter font-family=rmfamily,
 chapter font-weight=mdseries,
 chapter font-size=Huge,
 chapter before=\par,
 chapter after=\par,
 chapter color=black!90,
 number  color=black!90,
 chapter title width=0.8\textwidth,
 chapter title align=left,
 title   beforeskip=,
% title afterskip={\vspace*{50pt}\par},
 title margin top=0pt,
 title margin bottom=50pt,
 title margin-left=0pt,
 title before=,
 title after=\par,
 chapter title text-align=left,
 title font-family=rmfamily,
 title font-color=black!90,
 title font-weight=bfseries,
 title font-size=Huge}}
 
\parindent1em

\cxset{style19}
\chapter{Introduction to chapter style nineteen}

I first visited the Gulf seven years after its independence. As Simon C. Smith puts it, in his book \emph{Britain’s Revival and Fall in the Gulf} the decolonization of the Gulf was a mere footnote on British history. As I have lived in and I am currently still working in the area in this footnote for about twelve years Smith’s book sheds light to the beginnings of the Gulf.

\medskip
\begin{figure}[ht]
\centering 
\includegraphics[width=0.5\textwidth]{./chapters/chapter19.png}
\end{figure}

The book’s typography is what one expects from an academic publication. The headings are simple and the text is typeset in Times Roman. I was tempted to name this template \emph{plain vanilla} but perhaps it deserves better.
There is a large section of book designers that believe that the typography of a book should be like a crystal glass.

\begin{quote}
Now the man who first chose glass instead of clay or metal to hold his wine was a ``modernist" in the sense in which I am going to use that term. That is, the first he asked of this particular object was not "How should it look?" but ``What must it do?" and to that extent all good typography is modernist.	
\end{quote}

Throughout the essay, Warde argues for the discipline and humility required to create quietly set, ``transparent" book pages.

Now, back to the template one of the difficulties we will face is that the chapter title blocks are set in Once we adjust the title to be anything less than the width of the text block, we will also need to be careful
about words in order to give it some balance.
two or three lines and they do not extend to the full length of the text block.

The main settings are as follows:

\begin{verbatim}
\cxset{reset,
 chapter title width=0.65\textwidth,
 chapter title align=raggedright,}
\end{verbatim}


\cxset{chapter opening=anywhere}
\chapter{The failure of the federal idea in the Gulf, 1950-68}

The book does not have any lower level headings. Another characteristic is a subtitle below the main chapter block on some of the chapters. The subtitle is set in normal weight and is \emph{partially} used as a heading. 

\testsections




^^A
%   <<<<<<< HEAD
\colorlet{toprule}{teal}
\colorlet{theblock}{teal}
%%%%%%%%%%%%%%
%%%%%%%%%%%%%%%%%%%%%%%%%%%%%
%%%%%%  STYLE 20
%%%%%%%%%%%%%%%%%%%%%%%%%%%%%%%%%%%%%%%%%%%
\cxset{rule color/.store in={\rulecolor@cx},
          block color/.store in={\blockcolor@cx}}
\cxset{style20/.style={
 rule color=teal!90,
 block color=teal!90,
 name={},
 numbering = arabic,
 number font-size=\HHUGE,
 number color=\color{white},
 number font-family=\sffamily,
 number font-weight=\bfseries,
 number before={\hbox to 0pt{\vbox to -10pt{\colorbox{\blockcolor@cx}{\rule{0pt}{70pt}\HHUGE \color{\blockcolor@cx}1331}}}\vskip1pt\color{\rulecolor@cx}\rule{\textwidth}{5pt}\par\vskip10pt\relax\hspace{2.5em}},
 number after=\hspace{3em},
 number dot={ },
 number position=leftname,
 chapter font-family=\rmfamily,
 chapter font-weight=\normalfont,
 chapter font-size=\huge,
 chapter before={},
 chapter after={\hskip0pt},
 chapter color={black!90},
 title beforeskip={},
 title afterskip={\vspace*{30pt}\par}, % before text
 title before={\hskip0.2em},
 title after={\par\vspace{0pt}\color{\rulecolor@cx}\rule{\textwidth}{5pt}},
 title font-family=\sffamily,
 title font-color=\color{black!90},
 title font-weight=\bfseries,
 title font-size=\HUGE,
 section color=teal,
 section font-family=\sffamily,
 section font-weight=\bfseries,
 section font-shape=\upshape\color{teal},
 section indent=-10pt,
 header style=plain}}
=======
\colorlet{toprule}{teal}
\colorlet{theblock}{teal}
%%%%%%%%%%%%%%
%%%%%%%%%%%%%%%%%%%%%%%%%%%%%
%%%%%%  STYLE 20
%%%%%%%%%%%%%%%%%%%%%%%%%%%%%%%%%%%%%%%%%%%
\cxset{rule color/.store in={\rulecolor@cx},
          block color/.store in={\blockcolor@cx}}
\cxset{style20/.style={
 rule color=teal!90,
 block color=teal!90,
 name={},
 numbering = arabic,
 number font-size=\HHUGE,
 number color=\color{white},
 number font-family=\sffamily,
 number font-weight=\bfseries,
 number before={\hbox to 0pt{\vbox to -10pt{\colorbox{\blockcolor@cx}{\rule{0pt}{70pt}\HHUGE \color{\blockcolor@cx}1331}}}\vskip1pt\color{\rulecolor@cx}\rule{\textwidth}{5pt}\par\vskip10pt\relax\hspace{2.5em}},
 number after=\hspace{3em},
 number dot={ },
 number position=leftname,
 chapter font-family=\rmfamily,
 chapter font-weight=\normalfont,
 chapter font-size=\huge,
 chapter before={},
 chapter after={\hskip0pt},
 chapter color={black!90},
 title beforeskip={},
 title afterskip={\vspace*{30pt}\par}, % before text
 title before={\hskip0.2em},
 title after={\par\vspace{0pt}\color{\rulecolor@cx}\rule{\textwidth}{5pt}},
 title font-family=\sffamily,
 title font-color=\color{black!90},
 title font-weight=\bfseries,
 title font-size=\HUGE,
 section color=teal,
 section font-family=\sffamily,
 section font-weight=\bfseries,
 section font-shape=\upshape\color{teal},
 section indent=-10pt,
 header style=plain}}
>>>>>>> merged
^^A
%  %%%%%%%%%%%%%%%%%%%%%%%%%%%%%
%%%%%%  STYLE 20a
%%%%%%%%%%%%%%%%%%%%%%%%%%%%%%%%%%%%%%%%%%%
\cxset{rule color/.store in={\rulecolor@cx},
          block color/.store in={\blockcolor@cx}}
\cxset{style20a/.style={
 rule color=teal!90,
 block color=cyan,
 name=chapter,
 numbering = arabic,
 number font-size=\HHUGE,
 number color=\color{white},
 number font-family=\sffamily,
 number font-weight=\bfseries,
 number before={\hbox to 0pt{\vbox to -10pt{\colorbox{\blockcolor@cx}{\rule{0pt}{70pt}\HHUGE \color{\blockcolor@cx}1331}}}\vskip1pt\vskip10pt\relax\hspace{2.5em}},
 number after=\hspace{3em},
 number dot={ },
 number position=rightname,
 chapter font-family=\rmfamily,
 chapter font-weight=\normalfont,
 chapter font-size=\large,
 chapter before={},
 chapter after={\hskip0pt},
 chapter color={black!90},
 title beforeskip={},
 title afterskip={\vspace*{30pt}\par}, % before text
 title before={\hskip0.2em},
 title after={\par\vspace{0pt}\color{\rulecolor@cx}\rule{\textwidth}{5pt}},
 title font-family=\sffamily,
 title font-color=\color{black!90},
 title font-weight=\bfseries,
 title font-size=\HUGE,
 section color=teal,
 section font-family=\sffamily,
 section font-weight=\bfseries,
 section font-shape=\upshape\color{teal},
 section indent=-10pt,
 header style=plain}}
\cxset{style20}
\parindent1em
\chapter{STYLE 20}

\lettrine{\textcolor{teal}{T}}{his} style is probably useful in some corporate environment. The layout has been defined traditionally using boxes and skips and can perhaps be improved tremendously via TikZ. I selected this layout from a book titled \textit{Manufacturing at Warp Speed}, Eli Schragenheim, H. William Dettmer, 2001. The original book's chapter header is not coloured. We will use this example to define color schemes and themes.
\index{color schemes}\index{themes}.
\medskip
\begin{figure}[ht]
\centering
\includegraphics[width=0.7\textwidth]{chapter20}
\end{figure}

\section{Creating themes}
The strategy we use to define themes, especially if based on color changes is to define commands to generate them. This way one could define a number of themes fairly quickly.

\section{Adding templates and themes to a library}
Templates that have a lot of themes can be considered libraries and can be loaded with the package. (See the section on libraries).

\begin{texexample}{}{}
\cxset{chapter opening=anywhere}
% #1 style to add themes
% #2 name of theme
% #3 key value list
\newcommand\maketheme[3][style20]{%
\cxset{#1 #2/.style={#1,#3 }}}
% create some themes
\maketheme[style20]{black}{rule color=black,block color=black,}
\maketheme[style20]{blue}{rule color=theblue,block color=theblue,}
\maketheme[style20]{blue}{rule color=theblue,block color=theblue,}
\maketheme[style20]{orange}{rule color=orange,block color=orange,}
\cxset{style20 black}
\chapter{A Test}
\cxset{style20 blue}
\chapter{A Test}
\cxset{style20 orange}
\chapter{A Test}
\end{texexample}

\begin{texexample}{}{}
\fboxrule0pt\fboxsep0pt
\colorbox{teal}{\fbox{\parbox[b]{3cm}{%
\vbox to 0pt{\hbox to 3cm{\hfill\large\itshape\color{white} Chapter\hfill}}
\vbox{}%
\hbox to 3cm{\hfill \color{white}\sffamily\bfseries\HHUGE39\rule{0pt}{60pt}}
\hbox to 3cm{\rule{0pt}{40pt}}
}}}\hspace{0.5em}
\fbox{\parbox[b]{13cm}{%
\huge\color{teal} Paradoxical functional facilitation\\[-1pt] and recovery in neurological\\[-1pt]
 and psychiatric conditions\par
\medskip
\vspace*{20pt}
\color{black}
\large Dr Yiannis Lazaridegj
}}

\end{texexample}
\newcommand\allbluechapter[2][]{%
\fboxrule0pt\fboxsep0pt%
\hspace*{-1em}\fbox{\colorbox{theblock}{\fbox{\parbox[b]{3cm}{%
\vbox to 0pt{\hbox to 3cm{\hfill\large\itshape\color{white} Chapter\hfill}}
\vbox{}%
\hbox to 3cm{\hfill \color{white}\sffamily\bfseries\HHUGE\thechapter\rule{0pt}{60pt}}
\hbox to 3cm{\rule{0pt}{40pt}}%
}}}\hspace{1.5em}
\fbox{\parbox[b]{10cm}{%
\huge\color{teal} #2\par
\medskip
\vspace*{10pt}
\color{black}
\large \authorblockformat@cx\authorblock@cx
}}}
\vspace{25pt}

\thispagestyle{fancy}
}
\clearpage

\@specialtrue
\cxset{custom=allbluechapter,
         header style=plain, %check why is not working
         chapter opening=any,
         subsection numbering=none,
         subsection color=teal,
         subsection font-weight=\bfseries,
         subsection font-shape=\upshape,
         subsection indent=-10pt,
         author block=true,
         author block format=\bfseries\normalfont\raggedright,
         author names={Dr Yiannis Lazarides, Maria Lazarides and Athena Lazarides}}
\renewsubsection\renewsection
\parindent1em
\chapter[Paradoxical facilitation]{Paradoxical functional facilitation\\ and recovery in neurological\\
 and psychiatric conditions}

\section{Introduction}
\lorem

\section{Author Block Formatting}

Each chapter of the book carries the names of its authors, which is typeset as shown above. The standard available fields for author blocks are programmed in the special template. The full settings are shown in Example .
\medskip

\noindent\begin{tcolorbox}
\begin{lstlisting}
\@specialtrue
\cxset{custom=allbluechapter,
         header style=plain, %check why is not working
         chapter opening=any,
         author block=true,
         author block format=\bfseries\normalfont\raggedright,
         author names=Dr Yiannis Lazarides, Maria Lazarides and Athena Lazarides}
\chapter{Paradoxical functional facilitation...}
\end{lstlisting}
\end{tcolorbox}

Remember that it is also possible to add the author with the command \cs{addauthor} or its alias macro \cs{addauthors}. Cases where people can make mistakes are normally aliased to avoid common errors.

\section{Key value interface}
\subsection{General keys}
All keys for chapters chapters can be used in the template and toc.
Additional keys are described below.


\subsection{Other formatting hooks}

When a special template is designed it is prudent to provide hooks for minor tweaks. This way it is unecessary to modify the code of a special template for such changes. Keys have been provided for all the struts etc.
All normal keys can be used, such as font selection, spacing etc.

\lipsum
^^A
%  <<<<<<< HEAD
\@specialfalse
%%%%%%%%%%%%%%%%%%%%%%%%%%%%%%%%%%%%%%%%%%%
%%%%%%  STYLE 21
%%%%%%%%%%%%%%%%%%%%%%%%%%%%%%%%%%%%%%%%%%%
\newgeometry{left=4.5cm,right=2.5cm, marginparsep=15pt, marginparwidth=4.2cm,top=2cm,%
reversemarginpar}
\cxset{
 chapter opening=right,
 name={},
 numbering=none,
 number font-size=\Large,
 number font-family=\rmfamily,
 number font-weight=\bfseries,
 number before=,
 number after=,
 number position=rightname,
 chapter font-family=\sffamily,
 chapter font-weight=\normalfont,
 chapter font-size=\Large,
 chapter before={\vspace*{0.3\textheight}},
 chapter after={\par},
 chapter color={black!90},
 number color=\color{black!90},
 title beforeskip={},
 title afterskip={\par\rule{\textwidth}{3.5pt}\vspace{20pt}},
 title before={},
 title after={},
 title font-family=\sffamily,
 title font-color=\color{black},
 title font-weight=\bfseries,
 title font-size=\Huge,
 author block=false}


\chapter{INTRODUCTION TO STYLE 21}

\lipsum[1]
\medskip
\begin{figure}[ht]
\centering
\fbox{\includegraphics[width=0.65\textwidth]{./chapters/chapter21}}
\end{figure}
\lipsum[1]
\clearpage
=======
\@specialfalse
%%%%%%%%%%%%%%%%%%%%%%%%%%%%%%%%%%%%%%%%%%%
%%%%%%  STYLE 21
%%%%%%%%%%%%%%%%%%%%%%%%%%%%%%%%%%%%%%%%%%%
\newgeometry{left=4.5cm,right=2.5cm, marginparsep=15pt, marginparwidth=4.2cm,top=2cm,%
reversemarginpar}
\cxset{
 chapter opening=right,
 name={},
 numbering=none,
 number font-size=\Large,
 number font-family=\rmfamily,
 number font-weight=\bfseries,
 number before=,
 number after=,
 number position=rightname,
 chapter font-family=\sffamily,
 chapter font-weight=\normalfont,
 chapter font-size=\Large,
 chapter before={\vspace*{0.3\textheight}},
 chapter after={\par},
 chapter color={black!90},
 number color=\color{black!90},
 title beforeskip={},
 title afterskip={\par\rule{\textwidth}{3.5pt}\vspace{20pt}},
 title before={},
 title after={},
 title font-family=\sffamily,
 title font-color=\color{black},
 title font-weight=\bfseries,
 title font-size=\Huge,
 author block=false}


\chapter{INTRODUCTION TO STYLE 21}

\lipsum[1]
\medskip
\begin{figure}[ht]
\centering
\fbox{\includegraphics[width=0.65\textwidth]{./chapters/chapter21}}
\end{figure}
\lipsum[1]
\clearpage
>>>>>>> merged

%  

\restoregeometry

\newgeometry{left=4.5cm,right=2.5cm, marginparsep=15pt, marginparwidth=4.2cm,top=2cm,
reversemarginpar}

\setdefaults
\cxset{style22/.style={
 name={},
 numbering=none,
 number font-size=\Large,
 number font-family=\rmfamily,
 number font-weight=\bfseries,
 number before=,
 number after=,
 number position=rightname,
 chapter font-family=\sffamily,
 chapter font-weight=\normalfont,
 chapter font-size=\Large,
 chapter before=\vspace*{-10pt},
 chapter after={},
 chapter color=black!90,
 number color= black!90,
 title beforeskip= \raggedleft,
 title afterskip={\vspace{70pt}},
 title before=\hspace*{-2cm},
 title after={},
 title font-family=\sffamily,
 title font-color=black,
 title font-weight=\bfseries,
 title font-size=\huge,
 section numbering=none,
 section font-family=\sffamily,
 section font-weight=\bfseries,
 section color=black,
 section indent= 10pt,
 subsection indent = 0pt,
 header style=plain}}


\cxset{style22}
\renewsection\renewsubsection

\chapter{INTRODUCTION TO STYLE TWENTY TWO}\index{style22}\index{lettrine}\index{drop cap}

\section{INTRODUCTION}

\renewcommand{\DefaultLoversize}{0.3}
\renewcommand{\LettrineTextFont}{\fontfamily{Minion Pro}\normalfont\itshape}
\renewcommand{\LettrineFontHook}{%
\fontseries{bx}\fontshape{up}\color{gray}}

\cxset{lettrine lines/.code=\global\setcounter{DefaultLines}{#1}}

\cxset{lettrine lines=5}

\lettrine[lraise=0.0, nindent=0em, slope=-.5em]{Y}{oic} \lipsum[1]

\medskip
\begin{figure}[ht]
\centering
\fbox{\includegraphics[width=0.5\textwidth]{./chapters/chapter22.png}}
\end{figure}

\lipsum[1-2]
\parindent0pt

%  \setcounter{secnumdepth}{6}

\newgeometry{left=4cm,right=4cm,bottom=2cm}

\cxset{style23/.style={
 name={Chapter},
 numbering=arabic,
 number font-size=\Large,
 number font-family=\rmfamily,
 number font-weight=\bfseries,
 number before={},
 number after={},
 number dot=,
 number position=rightname,
 chapter font-family=\sffamily,
 chapter font-weight=\normalfont,
 chapter font-size=\Large,
 chapter before={\hspace*{-50pt}\rule{\dimexpr\textwidth+50pt\relax}{0.4pt}\par\hspace*{-51pt}},
 chapter after={\par},
 chapter color= black!90,
 number color=gray,
 title beforeskip={},
 title afterskip={\vspace{30pt}},
 title before=\hspace*{-50pt},
 title after={\par\vspace*{-10pt}\hspace*{-50pt}\rule{\dimexpr\textwidth+50pt}{0.4pt}\par},
 title font-family=\rmfamily,
 title font-color=black!80,
 title font-weight=\bfseries,
 title font-size=\LARGE,
 section font-family=\rmfamily,
 section font-shape=\itshape,
 section font-weight=\bfseries,
 section numbering=arabic,
 section indent=-49pt,
 section beforeskip=\baselineskip,
 section afterskip=\baselineskip,
 subsection font-shape=\upshape,
 subsection beforeskip=\baselineskip,
 subsection afterskip=10pt,
 subsection indent=-49pt,
 subsection numbering=arabic,
 subsubsection font-family=\rmfamily,
 subsubsection font-shape=\itshape,
 subsubsection font-weight=\bfseries,
 subsubsection font-shape=\itshape,
 subsubsection font-size=\large,
 subsubsection align=,
 subsubsection beforeskip=10pt,
 subsubsection afterskip=\baselineskip,
 subsubsection indent=-49pt,
 subsubsection numbering=numeric,
 paragraph font-family=\rmfamily,
 paragraph font-shape=\itshape,
 paragraph font-weight=\normalfont,
 paragraph font-shape=\itshape,
 paragraph font-size=\large,
 paragraph align=,
 paragraph beforeskip=10pt,
 paragraph afterskip=0pt,
 paragraph indent=-49pt,
 paragraph numbering=numeric,
 subparagraph font-family=\rmfamily,
 subparagraph font-shape=\itshape,
 subparagraph font-weight=\normalfont,
 subparagraph font-shape=\itshape,
 subparagraph font-size=\large,
 subparagraph align=,
 subparagraph beforeskip=10pt,
 subparagraph afterskip=0pt,
 subparagraph indent=-49pt,
 subparagraph numbering=arabic,
 subparagraph number after=\thinspace,
 header style=empty,
 pagestyle=headings,
}}



\cxset{style23}
\chapter{Introduction to style twenty three}

\section{Introduction}

This style requires that the chapter settings as well as the
section headings are set in the margins, leaving the text after the sectioning commands to be indented. We achieve this by using negative skips.
\medskip
\begin{figure}[ht]
\centering
\fbox{\includegraphics[width=0.45\textwidth]{./chapters/chapter23}
\includegraphics[width=0.45\textwidth]{./chapters/chapter23a}}
\end{figure}

\subsection{Subsections}
The same style is applied to the subsectioning commands up to the subsubsection which is not numbered but just uses an italic font. The subsubsection is also indented into the margin. Since the book is about construction claims it follows a style found in legal and construction documents, where all paragraphs are indented with respect to the section headings.
\lipsum[1-2]
\section{Even pages}
\lipsum[2]
\subsection{Setting different margins}
\lipsum[1]
\subsubsection{Setting Subsubsections}
\lipsum[1]
\paragraph{Paragraph level. } \lipsum*[3]\par

\lipsum[1]
\subparagraph{sub-paragraph level. } \lipsum*[3]\par


\restoregeometry

%  \makeatletter
\clearpage
\cxset{
 name=Chapter,
 numbering=arabic,
 number font-size=\Large,
 number font-family=\sffamily,
 number font-weight=\normalfont,
 number before={},
 number after={\space},
 number position=rightname,
 chapter font-family=\sffamily,
 chapter font-weight=\normalfont,
 chapter font-size=\Large,
 number after={},
 number dot=,
 chapter before={},
 chapter after={\par\thinrule\vskip12pt},
 chapter color=black!90,
 number color= black!90,
 chapter spaceout=none,
 title beforeskip={},
 title afterskip={\vspace{30pt}},
 title before=,
 title after={\par},
 title font-family=\sffamily,
 title font-color= black!80,
 title font-weight=\bfseries,
 title font-size=\LARGE,
 title afterskip=\par\vspace*{3cm}\thinrule\par\bigskip\bigskip,
 section indent= 0pt,
 section font-shape=upshape,
 section font-family=upshape}



\chapter{Introduction to style twenty four}


\def\objectives@{%
 \begin{tcolorbox}[width=\linewidth,boxsep=10pt,right=10pt]
\textbf{Learning Objectives}\parindent0pt\leavevmode}
\def\stopobjectives@{\end{tcolorbox}}
\newenvironment{objectives}{\bigskip\objectives@}{\stopobjectives@\bigskip}

\parindent1em

\begin{objectives}
\par
\lipsum[1]
\bigskip\bigskip
\end{objectives}

This design is ideal for scholarly books or notes. It has a nice clean design with a shaded block for the learning objectives. \lipsum*[2-3]
\medskip
\begin{figure}[ht]
\centering
\fbox{\includegraphics[width=0.6\textwidth]{./chapters/chapter24.png}}
\end{figure}


^^A
%   \makeatletter
\cxset{author/.store in=\author@cx}
\cxset{author block=true}

\cxset{style25/.style={
 name={CHAPTER},
 numbering=arabic,
 number font-size=huge,
 number font-family=sffamily,
 number font-weight=\normalfont,
 number before=\kern0.5em,
 number position=rightname,
 chapter font-family=sffamily,
 chapter font-weight=normalfont,
 chapter font-size=huge,
 number after=\hfill\hfill\vskip1pt\hrule width\textwidth height1pt\relax\vskip1pt,
 chapter before=\hrule width\textwidth height1pt\relax\vskip1pt\hfill,
 chapter after=,
 chapter color=black!90,
 number color=black!90,
  title afterskip={\vspace{10pt}\author@cx},
 title before=\leavevmode,
 title after=\vskip1pt\hrule width\textwidth height1pt\relax,
 chapter title align=centering,
 chapter title width=.9\textwidth,
 title font-family=sffamily,
 title font-color= black!80,
 title font-weight=bfseries,
 title font-size=huge}}
\makeatother

\cxset{author=\centering\bfseries\upshape\large Yiannis Lazarides and Athena Lazarides\par\vspace{30pt}}

\cxset{style25}
\chapter{INTRODUCTION TO STYLE 25}

The interesting part of this style is that it uses roman numerals to display the counter that is in a different font than that used for the chapter name.
\lipsum*[1-2]
\begin{figure}[ht]
\centering
\fbox{\includegraphics[width=0.5\textwidth]{./chapters/chapter25.png}}
\end{figure}

^^A
%   %%%%%%%%%%%%%%%%%%%%%%%%%%%%%%%%%%%%%%%%%%%
%%%%%%  STYLE 26
%%%%%%%%%%%%%%%%%%%%%%%%%%%%%%%%%%%%%%%%%%%

\cxset{
 name={},
 numbering=arabic,
 number font-size=\huge,
 number font-family=\sffamily,
 number font-weight=\bfseries,
 number before={},
 number position=leftname,
 chapter font-family=\sffamily,
 chapter font-weight=\normalfont,
 chapter font-size=\small,
 number after={},
 chapter before={},
 chapter after={\par\vskip12pt},
 chapter color={black!90},
 number color=\color{black!90},
 title beforeskip={},
 title afterskip={\vspace{30pt}},
 title before=,
 title after={\par},
 title font-family=\sffamily,
 title font-color=\color{black!80},
 title font-weight=\bfseries,
 title font-size=\LARGE}
\chapter{Introduction to style twenty five Dr. Yiannis Lazarides and Athena Lazarides}

The interesting part of this style is that it uses roman numerals to display the counter that is in a different font than that used for the chapter name.
\medskip
\begin{figure}[ht]
\centering
\fbox{\includegraphics[width=0.6\textwidth]{./chapters/chapter26}}
\end{figure}
\lipsum[2-3]
^^A
%   \cxset{
 name={},
 numbering=arabic,
 number font-size=HUGE,
 number font-family=sffamily,
 number font-weight=bfseries,
 number before={},
 number dot=,
 number position=leftname,
 chapter font-family=sffamily,
 chapter font-weight=normalfont,
 chapter font-size=LARGE,
 number after={},
 chapter before={\vspace*{50pt}},
 chapter after={\par\vskip12pt},
 chapter color= sweet,
 number color=black!90,
 title beforeskip=,
% title afterskip={\vspace{30pt}},
 title margin bottom=30pt,
 chapter title align=left,
 title before=,
 title after=,
 title font-family=sffamily,
 title font-color= black!80,
 title font-weight=bfseries,
 title font-size=huge,
 chapter title width=\textwidth,
 chapter title align=raggedright,
 chapter afterindent=false,
 section numbering = arabic,
 section numbering prefix = \thechapter.,
 section indent=0pt,
 section align=left,
 section font-size=Large}
 
 
\chapter{Introduction to style twenty seven }
\lipsum[3]

\medskip
\begin{figure}[ht]
\centering
\fbox{\includegraphics[width=0.5\textwidth]{./chapters/chapter27.png}}
\end{figure}

\section{PATHWAY AND FORM OF RADIOACTIVE DEPOSITIONS}


\lipsum[2-3]

^^A
%   \cxset{
 name={},
 numbering=arabic,
 number font-size=\HUGE,
 number font-family=\sffamily,
 number font-weight=\bfseries,
 number before={},
 number position=leftname,
 chapter font-family=\sffamily,
 chapter font-weight=\normalfont,
 chapter font-size=\small,
 number after={},
 chapter before={},
 chapter after={\hspace*{20pt}},
 chapter color=black!90,
 number color= black!90,
 title beforeskip={},
 title afterskip={\vspace{70pt}},
 title before=,
 title after={\par},
 title font-family=\itshape,
 title font-color= black!80,
 title font-weight=\itshape,
 title font-size=\LARGE,
 author block=false}

\chapter{Introduction to Style Twenty Eight}

The interesting part of this style is that it uses roman numerals to display the counter that is in a different font than that used for the chapter name.
\medskip
\begin{figure}[ht]
\centering
\includegraphics[width=0.6\textwidth]{./chapters/chapter28.png}
\end{figure}
\lipsum[2]

\section{Sectioning Commands}
\lorem

\subsection{Subsectioning commands}
\lorem^^A
%    \makeatletter
\cxset{plain sections/.style={
 chapter name = CHAPTER,
 chapter toc = true,
 chapter color= thegray,
 chapter opening = right, 
 chapter numbering = arabic,
 chapter font-family= sffamily,
 chapter font-weight= bold,
 chapter font-size= LARGE,
 chapter before={\thinrule\vspace*{20pt}\par\hfill\hfill},
 chapter after={\vskip0pt\par},
 chapter spaceout = soul,
 number font-size= Large,
 number font-family= rmfamily,
 number font-weight= bfseries,
 number color=thegray,
 number before=\vspace*{5pt}\hfill\hfill,
 number dot=.,
 number after={\hspace*{7pt}\par},
 title beforeskip={\vspace*{10pt}},
 title afterskip={\vspace*{50pt}\par},
 title before={\hfill\hfill\raggedleft},
 title after={\par\thinrule},
 title font-family=\sffamily,
 title font-color= teal,
 title font-weight=\bfseries,
 title font-family=\sffamily,
 title font-size= Large,
 title font-shape= upshape,
 title spaceout= none,
 title beforeskip={\vspace*{10pt}},
 title afterskip={\vspace*{50pt}\par},
 title before={\hfill\hfill\raggedleft},
%
% numbers
% number font-family=\sffamily,
% number font-weight=\bfseries,
 number color=thelightgray,
 number before=\par\vspace*{5pt}\hfill\hfill,
 number dot=.,
 number after={\hspace*{7pt}\par},
 number position=rightname,
 section color= thered,     
 section beforeskip=15pt,
 section afterskip=15pt,
 section indent=0pt,
 section font-family= sffamily,
 section font-size= LARGE,
 section font-weight= bfseries,
 section font-shape=,
 section align= centering,
 section numbering prefix =,%use \thechapter. for books or add as option
 section numbering= arabic,
 section spaceout=none,
 section number after=ooo,
 subsection color= thered,
       subsection beforeskip=10pt,
       subsection afterskip=10pt,
       subsection indent=0pt,
       subsection font-family= rmfamily,
       subsection font-size= large,
       subsection font-weight= bold,
       subsection font-shape= upshape,
       subsection align= centering,
       subsection numbering prefix=\thesection.,%\S\hairsp,%add . 
       subsection numbering custom =\@arabic\c@subsection,% \two@digits{\@arabic\c@subsection},%
       subsubsection color= gray,
       subsubsection beforeskip=5pt plus3pt minus 2pt,
       subsubsection afterskip=5pt,
       subsubsection indent=0pt,
       subsubsection font-family= rmfamily,
       subsubsection font-size= normalfont,
       subsubsection font-weight= bold,
       subsubsection font-shape= itshape,
       subsubsection align= centering,
       subsubsection numbering prefix =\thesubsection.\@arabic\c@subsubsection,
       subsubsection numbering custom =, %\two@digits{\@arabic\c@subsubsection},
       subsubsection number after =, 
%
       paragraph color= thegrey,
       paragraph beforeskip=,
       paragraph afterskip=-0.5em,
       paragraph indent=0pt,
       paragraph font-family= rmfamily,
       paragraph font-size= large,
       paragraph font-weight= bfseries,
       paragraph font-shape=,
       paragraph align= centering,
       paragraph number after = 0pt,
       paragraph numbering=numeric,
       subparagraph color= thered,
       subparagraph beforeskip=0pt,
       subparagraph afterskip=-.5em,
       subparagraph indent=0pt,
       subparagraph font-family= sffamily,
       subparagraph font-size= large,
       subparagraph font-weight= normalfont,
       subparagraph font-shape= slshape,
       subparagraph align= RaggedRight,
       subparagraph number after =, % can affect all needs checking
       %subsubsection numbering prefix=\S\hairsp\thesection,%add . here if need be
       subparagraph numbering=none,
}
}
\cxset{plain sections}
\cxset{style13/.style={
 name= {\protect\pan अमुकग्रन्थे},
 chapter spaceout = none,
 numbering=arabic,
 number font-size= HUGE,
 number font-family= sffamily,
 number font-weight= bfseries,
 number color= gray!50,
 number before=\par\vspace*{5pt}\hfill\hfill,
 number dot=,
 number after={\hspace*{7pt}\par},
 number position=rightname,
 chapter font-family= sffamily,
 chapter font-weight= bold,
 chapter font-size= LARGE,
 chapter before={\tikzrule\vspace*{20pt}\par\hfill\hfill},
 chapter color= black!50,
 title beforeskip={\vspace*{10pt}},
 title afterskip={\vspace*{50pt}\par},
 title before={\hfill\hfill\raggedleft},
 chapter rule color=spot!50,
 title after=\par\tikzrule,
 title font-family= sffamily,
 title font-color= teal,
 title font-weight= bfseries,
 title font-size= huge,
 section indent=-1em,
 section align= left,
 section numbering= arabic,
 section indent=0pt,
 section beforeskip=0pt,
 section afterskip= 10pt,
 section color=teal,
 subsection align= ,
 subsection font-family= sffamily,
 subsection font-weight= bfseries,
 subsection color = teal,
 subsection font-size= large,
 subsection font-shape=,
 subparagraph number after=,
 subsubsection align=,
}
}
\cxset{style13}

\renewparagraph
\renewsection
\renewsubsection
\renewsubparagraph
\renewsubsubsection

\makeatother
  
 \cxset{style29/.style={
 name={},
 numbering=arabic,
 number font-size=normalsize,
 number font-family=sffamily,
 number font-weight=bfseries,
 number before={\vspace*{30pt}},
 number position=leftname,
 number after=\hrule width\textwidth height1pt\par,
 chapter font-family=sffamily,
 chapter font-weight=,
 chapter font-size=small,
 chapter before={\vskip2.5pt},
 chapter after=,
 chapter color= black!90,
 number color= black!90,
 title beforeskip={},
 title afterskip={\bigskip},
 title before=,
 title after={\par},
 title font-family=rmfamily,
 title font-color= black!80,
 title font-weight=bfseries,
 title font-size=\huge,
 chapter title align=raggedright,
 section indent=0pt,
 section numbering=arabic,
 section font-family=\rmfamily,
 section font-shape=\upshape,
 section numbering prefix=\thechapter.,
section numbering suffix=,
 section color=black,
 subsection number after=\quad,
 subsection number after=\quad}}
\cxset{style29}

\endinput
\renewsection\renewsubsection

\chapter{Reading Systems: An Introduction to Digital Document Processing}
\bigskip\bigskip

\textit{Lambert Schoemacher}
\bigskip\bigskip\bigskip\bigskip

\section{Introduction}

Style 29 comes from the computer world and is representative of conference publications. It is always instructive
to go back and read research undertaken decades ago to understand the present state of the art but also to study how standards emerge and the competitive forces that shape the survivability of computer software. 


\begin{figure}[ht]
\centering
\includegraphics[width=0.5\textwidth]{./chapters/chapter29.png}
\end{figure}

The template is based on a Springer-Verlag London publication dated 2007 from a series on Advance Pattern Recognition. Many of these publications were prepared using \latex itself and templates from this firm are still available and on ctan for its many journals. The introduction by Schoemacher provides background information to anyone interested to understand the evolution of document processing. 

\section{Documents}

The word document comes from the Latin word ‘documentum’, which has the same
stem as the verb ‘doceo’ (meaning ‘to teach’), plus the suffix ‘-umentum’ (indicating
a means for doing something). Hence, it is intended to denote ‘a means for teaching’
(in the same way as ‘instrument’ denotes a means to build, ‘monument’ denotes a
means to warn, etc.). Dictionary definitions of a document are the following [2]:
\begin{enumerate}
\item Proof, Evidence
\item An original or official paper relied on as basis, proof or support of something
\item Something (as a photograph or a recording) that serves as evidence or proof
\item A writing conveying information
\item A material substance (as a coin or stone) having on it a representation of thoughts
by means of some conventional mark or symbol.
\end{enumerate}

The first definition is more general. The second one catches the most intuitive
association of a document to a paper support, while the third one extends the definition
to all other kinds of support that may have the same function. While all
these definitions mainly focus on the bureaucratic, administrative or juridic aspects
of documents, the fourth and fifth ones are more interested in its role of information
bearer that can be exploited for study, research, information. Again, the former
covers the classical meaning of documents as written papers, while the latter extends
it to any kind of support and representation. Summing up, three aspects can be
considered as relevant in identifying a document: its original meaning is captured
by definitions 4 and 5, while definition 1 extends it to underline its importance as
a proof of something, and definitions 2 and 3 in some way formally recognize this
role in the social establishment.

\section {Current Landscape}

While up to recently documents were produced in paper format, and their digital
counterpart was just a possible consequence carried out for specific purposes,
nowadays we face the opposite situation: nearly all documents are produced and
exchanged in digital format, and their transposition on a tangible, human-readable
support has become a successive, in many cases optional, step. Additionally, significant
efforts have been spent in the digitization of previously existing documents


\section{The Shelf-life of Documents}

In a post at \texttt{TX.SX} Barbara Beeton mentioned that one of the advantages of \tex is that documents produced by it results in documents with a very long shelf life and that Mathematics has a long shelf life.
Of course the best way to ensure a long shelf life is to have the document printed out and stored in an archive.
Paper is still the best way to ensure long term survivability. Knuth’s idea when he insisted that \tex cannot be changed was to ensure that a document could be printed always the same way. I am not too sure if this is an absolute necessity, as what is important is for the content to survive.

^^A
%   \makeatletter
\cxset{plain sections/.style={
 chapter name = CHAPTER,
 chapter toc = true,
 chapter color= thegray,
 chapter opening = right, 
 chapter numbering = arabic,
 chapter font-family= sffamily,
 chapter font-weight= bold,
 chapter font-size= LARGE,
 chapter before={\thinrule\vspace*{20pt}\par\hfill\hfill},
 chapter after={\vskip0pt\par},
 chapter spaceout = soul,
 number font-size= Large,
 number font-family= rmfamily,
 number font-weight= bfseries,
 number color=thegray,
 number before=\vspace*{5pt}\hfill\hfill,
 number dot=.,
 number after={\hspace*{7pt}\par},
 title beforeskip={\vspace*{10pt}},
 title afterskip={\vspace*{50pt}\par},
 title before={\hfill\hfill\raggedleft},
 title after={\par\thinrule},
 title font-family=\sffamily,
 title font-color= teal,
 title font-weight=\bfseries,
 title font-family=\sffamily,
 title font-size= Large,
 title font-shape= upshape,
 title spaceout= none,
 title beforeskip={\vspace*{10pt}},
 title afterskip={\vspace*{50pt}\par},
 title before={\hfill\hfill\raggedleft},
%
% numbers
% number font-family=\sffamily,
% number font-weight=\bfseries,
 number color=thelightgray,
 number before=\par\vspace*{5pt}\hfill\hfill,
 number dot=.,
 number after={\hspace*{7pt}\par},
 number position=rightname,
 section color= thered,     
 section beforeskip=15pt,
 section afterskip=15pt,
 section indent=0pt,
 section font-family= sffamily,
 section font-size= LARGE,
 section font-weight= bfseries,
 section font-shape=,
 section align= centering,
 section numbering prefix =,%use \thechapter. for books or add as option
 section numbering= arabic,
 section spaceout=none,
 section number after=ooo,
 subsection color= thered,
       subsection beforeskip=10pt,
       subsection afterskip=10pt,
       subsection indent=0pt,
       subsection font-family= rmfamily,
       subsection font-size= large,
       subsection font-weight= bold,
       subsection font-shape= upshape,
       subsection align= centering,
       subsection numbering prefix=\thesection.,%\S\hairsp,%add . 
       subsection numbering custom =\@arabic\c@subsection,% \two@digits{\@arabic\c@subsection},%
       subsubsection color= gray,
       subsubsection beforeskip=5pt plus3pt minus 2pt,
       subsubsection afterskip=5pt,
       subsubsection indent=0pt,
       subsubsection font-family= rmfamily,
       subsubsection font-size= normalfont,
       subsubsection font-weight= bold,
       subsubsection font-shape= itshape,
       subsubsection align= centering,
       subsubsection numbering prefix =\thesubsection.\@arabic\c@subsubsection,
       subsubsection numbering custom =, %\two@digits{\@arabic\c@subsubsection},
       subsubsection number after =, 
%
       paragraph color= thegrey,
       paragraph beforeskip=,
       paragraph afterskip=-0.5em,
       paragraph indent=0pt,
       paragraph font-family= rmfamily,
       paragraph font-size= large,
       paragraph font-weight= bfseries,
       paragraph font-shape=,
       paragraph align= centering,
       paragraph number after = 0pt,
       paragraph numbering=numeric,
       subparagraph color= thered,
       subparagraph beforeskip=0pt,
       subparagraph afterskip=-.5em,
       subparagraph indent=0pt,
       subparagraph font-family= sffamily,
       subparagraph font-size= large,
       subparagraph font-weight= normalfont,
       subparagraph font-shape= slshape,
       subparagraph align= RaggedRight,
       subparagraph number after =, % can affect all needs checking
       %subsubsection numbering prefix=\S\hairsp\thesection,%add . here if need be
       subparagraph numbering=none,
}
}
\cxset{plain sections}
\cxset{style13/.style={
 name= {\protect\pan अमुकग्रन्थे},
 chapter spaceout = none,
 numbering=arabic,
 number font-size= HUGE,
 number font-family= sffamily,
 number font-weight= bfseries,
 number color= gray!50,
 number before=\par\vspace*{5pt}\hfill\hfill,
 number dot=,
 number after={\hspace*{7pt}\par},
 number position=rightname,
 chapter font-family= sffamily,
 chapter font-weight= bold,
 chapter font-size= LARGE,
 chapter before={\tikzrule\vspace*{20pt}\par\hfill\hfill},
 chapter color= black!50,
 title beforeskip={\vspace*{10pt}},
 title afterskip={\vspace*{50pt}\par},
 title before={\hfill\hfill\raggedleft},
 chapter rule color=spot!50,
 title after=\par\tikzrule,
 title font-family= sffamily,
 title font-color= teal,
 title font-weight= bfseries,
 title font-size= huge,
 section indent=-1em,
 section align= left,
 section numbering= arabic,
 section indent=0pt,
 section beforeskip=0pt,
 section afterskip= 10pt,
 section color=teal,
 subsection align= ,
 subsection font-family= sffamily,
 subsection font-weight= bfseries,
 subsection color = teal,
 subsection font-size= large,
 subsection font-shape=,
 subparagraph number after=,
 subsubsection align=,
}
}
\cxset{style13}

\renewparagraph
\renewsection
\renewsubsection
\renewsubparagraph
\renewsubsubsection

\makeatother

\cxset{style30/.style={
 name={},
 numbering=arabic,
 number font-size=HUGE,
 number font-family=sffamily,
 number font-weight=bfseries,
 number before=\rule{\textwidth}{5pt}%
                           \par\vspace*{12pt}%
                           \hspace*{20pt},%
 number after=\hspace{1em},
 number position=leftname,
 chapter font-family=sffamily,
 chapter font-weight=normalfont,
 chapter font-size=small,
 chapter before={},
 chapter after={\hspace*{20pt}},
 chapter color= black!90,
 number color= black!90,
 title beforeskip={},
% title afterskip={\vspace{30pt}},
 title before=,
 title after=,
title margin bottom=30pt,
 title font-family=sffamily,
 title font-color=black!80,
 title font-weight=\normalfont\sffamily,
 title font-size=Huge,
 chapter title align=raggedright,
 chapter title width=.7\textwidth,
 section numbering prefix=\thechapter.,
 section numbering=arabic,
 subsection number after=\quad,
}}


\cxset{style30}

\chapter[Introduction to Style Thirty]{{\language-1 Introduction to Style Thirty with a Somehow Long Title to Illustrate the Example}}

Since we do not know how long a chapter title can end up, it is best to
typeset this using two minipages or parboxes. The number is pushed down slightly although it can look as good with both the number and the text fully aligned on top.
\medskip
\begin{figure}[ht]
\centering
\includegraphics[width=0.6\textwidth]{./chapters/chapter30.png}
\end{figure}

\section{Testing}

\lipsum[1-5]




^^A
%   \makeatletter
\cxset{style31/.style={
 name={},
 numbering=arabic,
 number font-size=\HUGE,
 number font-family=\sffamily,
 number font-weight=\bfseries,
 number before=,
 number after={},
 number position=leftname,
 chapter font-family=\sffamily,
 chapter font-weight=\normalfont,
 chapter font-size=\small,
 chapter before={},
 chapter after={\vskip2.5pt{\color{gray}\rule{3cm}{5pt}\rule[3.5pt]{\dimexpr\textwidth-3cm\relax}{0.4pt}}\par},
 chapter color=gray,
 number color=gray,
 title beforeskip={},
 title afterskip={\vspace{30pt}},
 title before=,
 title after={\par{\color{gray}\rule[6pt]{3cm}{0.4pt}}\par},
 title font-family=\itshape,
 title font-color=black,
 title font-weight=itshape,
 title font-shape=itshape,
 title font-size=LARGE}}

\cxset{style31}
\chapter[Evolution of Organizations and the Environment]{The Evolution of Organizations\\ and the Environment}

This is an unusual design by all counts. I did soften the rules a bit to make them a bit less conspicuous.
\medskip
\begin{figure}[ht]
\centering
\includegraphics[width=0.6\textwidth]{./chapters/chapter31.png}
\end{figure}

\lipsum[1-2]

\section{Test}

\lipsum[1]

\lipsum[2]

^^A
%   
\newgeometry{left=5cm,right=2cm,bottom=2cm}
\cxset{style32/.style={
 name={},
 numbering=arabic,
 number font-size=\HUGE,
 number font-family=\sffamily,
 number font-weight=\bfseries,
 number before={\hspace*{-10pt}},
 number position=leftname,
 chapter font-family=\sffamily,
 chapter font-weight=\normalfont,
 chapter font-size=\small,
 number after={},
 chapter before={\vspace*{50pt}\par\hspace*{-60pt}},
 chapter after={\hspace*{20pt}},
 chapter color=black!90,
 number color=black!90,
 title beforeskip={},
 title afterskip={\vspace{70pt}},
 title before=,
 title after={\par},
 title font-family=\itshape,
 title font-color=black!80,
 title font-weight=\bfseries,
 title font-shape=\itshape,
 title font-size= Huge,
}}

\cxset{style32}
\chapter{Introduction to Style Thirty Two}

This style has a modern look to it. Its main characteristic is the large chapter number and the fact that it is drawn into the margin. A common style for computer books.
\medskip
\begin{figure}[ht]
\centering
\includegraphics[width=0.6\textwidth]{./chapters/chapter32}
\end{figure}
The example is from Python NLP book.

^^A
%   
\restoregeometry
\cxset{style33/.style={
 name=CHAPTER,
 numbering=arabic,
 number font-size= LARGE,
 number font-family= rmfamily,
 number font-weight= \normalfont,
 number before={\par\hfill},
 number after={\hfill\hfill\par},
 number position=leftname,
 name=,
 chapter font-family=\sffamily,
 chapter font-weight=\normalfont,
 chapter font-size=\small,
 chapter before={\vskip10pt},
 chapter after={\vskip10pt\par},
 chapter color= black!90,
 number color= black!90,
 title beforeskip={},
 title afterskip={\vspace{50pt}},
 title before=\hfill,
 title after={\hfill\hfill\par},
 title font-family=\normalfont,
 title font-color= black!80,
 title font-weight=\normalfont,
 title font-shape=\upshape,
 title font-size=\LARGE,
 section numbering=none,
 section font-size=\Large,
 section align= center}}

\cxset{style33}
\chapter{Introduction to Style Thirty Three}

The interesting part of this style is that it uses roman numerals to display the counter that is in a different font than that used for the chapter name.
\medskip
\begin{figure}[ht]
\centering
\includegraphics[width=0.6\textwidth]{./chapters/chapter33}
\end{figure}

^^A
%   
\cxset{
 name=CHAPTER,
 numbering=Roman,
 number font-size=\small,
 number font-family=\rmfamily,
 number font-weight=\normalfont,
 number before={},
 number position=rightname,
 chapter font-family=\sffamily,
 chapter font-weight=\normalfont,
 chapter font-size=\small,
 number after={},
 chapter before={},
 chapter after={\par},
 chapter color={black!90},
 number color= black!90,
 title beforeskip={},
 title afterskip={\vspace{50pt}},
 title before=,
 title after={\par},
 title font-family=\normalfont,
 title font-color=\color{black!80},
 title font-weight=\normalfont,
 title font-size=\LARGE}

\section{Basic astronomical phenomena}

The interesting part of this style is that it uses roman numerals to display the counter that is in a different font than that used for the chapter name.
\medskip
\begin{figure}[ht]
\centering
\includegraphics[width=0.6\textwidth]{./chapters/chapter34.png}
\end{figure}
^^A
%   \input{./styles/style35}^^A
%   
%%%%%%%%%%%%%%%%%%%%%%%%%%%%%%%%%%%%%%%%%%%
%%%%%%  STYLE 36
%%%%%%%%%%%%%%%%%%%%%%%%%%%%%%%%%%%%%%%%%%%

\cxset{numbering=none}
%% has errors
\cxset{style36/.style={
 name={},
 numbering={none},
 number font-size=\small,
 number font-family=\rmfamily,
 number font-weight=\normalfont,
 number before={},
 number position=rightname,
 chapter font-family=\sffamily,
 chapter font-weight=\normalfont,
 chapter font-size=\small,
 number after={},
 chapter before={},
 chapter after={},
 chapter color={black!90},
 number color=\color{black!90},
 title beforeskip={},
 title before=,
 title after={\vskip-12.5pt\rule{\columnwidth}{3.5pt}\vspace*{50pt}},
 title afterskip={},
 title font-family=\sffamily,
 title font-color=\color{black!80},
 title font-weight=\bfseries,
 title font-size=\Huge}}

\cxset{style36}
\chapter{Introduction to Style Thirty Six}

The interesting part of this style is that it uses roman numerals to display the counter that is in a different font than that used for the chapter name.
\medskip

\begin{figure}[ht]
\centering
\includegraphics[width=0.6\textwidth]{./chapters/chapter36}
\end{figure}
^^A
%   \cxset{style37/.style={
 name=CHAPTER,
 numbering=Roman,
 number font-size=\small,
 number font-family = sffamily,
 number font-weight= bold,
 number before={},
 number position=rightname,
 chapter font-family= sffamily,
 chapter font-weight= bold,
 chapter font-size=\small,
 chapter spaceout=soul,
 number after={},
 chapter before={},
 chapter after={\par},
 chapter color=black!90,
 number color=black!90,
 title beforeskip={},
 title afterskip={\vspace{50pt}},
 title before=,
 title after={\par},
 title font-family= sffamily,
 title font-color= black!80,
 title font-weight= sfseries,
 title font-size=\huge}}

\cxset{style37}
\chapter{Introduction to Style Thirty Seven}

The interesting part of this style is that it uses roman numerals to display the counter that is in a different font than that used for the chapter name.
\medskip

\begin{figure}[ht]
\centering
\includegraphics[width=0.6\textwidth]{./chapters/chapter37}
\end{figure}

^^A
%   \cxset{style38/.style={
 name=,
 numbering=arabic,
 number font-size= huge,
 number font-family= rmfamily,
 number font-weight= normalfont,
 number before= \centering,
 number position=leftname,
 number after =\centering,
 chapter font-family= sffamily,
 chapter font-weight= normalfont,
 number after={\vspace*{6.5pt}\par},
 chapter before=\par,
 chapter after=\par,
 chapter color= black!90,
 number color=black!90,
 title beforeskip={},
 title afterskip={\vspace{50pt}},
 title before=,
 title after=\par,
 title font-family= rmfamily,
 title font-color= black!80,
 title font-weight=\normalfont,
 title font-size= huge,
 chapter font-size=,
}}

\cxset{style38,
       author block=true,
       author block format=\centering}
\chapter{STAGES OF INITIATION IN THE \vspace{0pt} ELEUSINIAN AND\\SAMOTHRACIAN MYSTERIES\\STYLE 38}

This style uses rules to enclose both the chapter name and number as well as the title, which necessarily needs to be rather short.
\medskip

\begin{figure}[ht]
\centering
\includegraphics[width=0.6\textwidth]{./chapters/chapter38}
\end{figure}

^^A
%   \cxset{
 name=CHAPTER,
 numbering=arabic,
 number font-size=\LARGE,
 number font-family=\sffamily,
 number font-weight=\bfseries,
 number before={},
 number position=rightname,
 chapter font-family=\sffamily,
 chapter font-weight=\bfseries,
 number after={},
 chapter before={\rule{\textwidth}{2pt}\par},
 chapter after={\vskip0pt\vspace*{-8pt}\rule{\textwidth}{.4pt}\vskip-7pt},
 chapter color={black!90},
 number color=black!90,
 title beforeskip={},
 title afterskip={\vspace{50pt}},
 title before=,
 title after={\par\vskip-16.5pt\rule{\textwidth}{0.4pt}\par} ,
 title font-family=\sffamily,
 title font-color=black!80,
 title font-weight=\bfseries,
 title font-size=\LARGE,
 chapter font-size=\LARGE,
 author block=false}

\chapter{STYLE 39}

This style uses rules to enclose both the chapter name and number as well as the title, which necessarily needs to be rather short.
\medskip

\begin{figure}[ht]
\centering
\includegraphics[width=0.6\textwidth]{./chapters/chapter39.png}
\end{figure}
In the picture it does not look very attractive, but in the actual book it does. My observation is that the rule clearances are a bit tight and if you use this type of layout it is better to experiment until you get them right.



^^A
%   \makeatletter

\long\gdef\versochapter#1{
  \vspace*{3cm}
  \minipage{\textwidth}
  \hfill\includegraphics[width=0.5\textwidth]{\chapterimage@cx}\par
  \vspace*{6pt}
  \hfill\minipage{0.75\textwidth}
  {\HUGE\bfseries\flushright #1\endflushright}
  \endminipage
  \endminipage
  \newpage


\vspace*{10cm}
\@specialfalse
\@openleftfalse
\@openanyfalse
\@openrighttrue
}


\newgeometry{bottom=2.5cm}

\cxset{
   chapter image/.code={\def\chapterimage@cx{#1}},
   chapter opening/.is choice,
   chapter opening/verso/.code={\@specialtrue\@openlefttrue
   \gdef\customdesign@cx##1{\versochapter{##1}}}
}

\cxset{
 custom=versochapter,
 chapter image={vespa.jpg},
 chapter opening=verso,
 name={},
 numbering=none,
 number font-size=LARGE,
 number font-family=rmfamily,
 number font-weight=bfseries,
 number before=,
 number dot={},
 number after=,
 number position=leftname,
 chapter font-family=sffamily,
 chapter font-weight=\normalfont,
 chapter font-size=\Large,
 chapter before={\vspace*{0pt}\par},
 chapter after={\hfill\hfill\par},
 chapter color={black!90},
 number color=purple,
 title beforeskip={\vspace*{0pt}},
 title afterskip={\vspace*{0.4\textheight}\par},
 title before={},
 title after={},
 title font-family=sffamily,
 title font-color=purple,
 title font-weight=bfseries,
 title font-size=LARGE,
 header style=plain,
 pagestyle=plain,
 }

\makeatletter
\@specialtrue
\makeatother



\chapter{Verso Chapters}

\parindent1.5em

{\Huge T}he theme of this template is from a book called 
{ \textit{From Western attitudes toward death from the middle ages to the present}, Philippe Ari\'es. London, 1974.

\begin{figure}
\includegraphics[width=\textwidth]{./chapters/versochapter01.png}
\caption{Chapter opening on verso page.}
\end{figure}

I will quote Oliver Sacks words verbatim, as I cannot ever imagine that I can do it better:

\begin{quote}
A MONTH ago, I felt that I was in good health, even robust health. At 81, I still swim a mile a day. But my luck has run out — a few weeks ago I learned that I have multiple metastases in the liver. Nine years ago it was discovered that I had a rare tumor of the eye, an ocular melanoma. Although the radiation and lasering to remove the tumor ultimately left me blind in that eye, only in very rare cases do such tumors metastasize. I am among the unlucky 2 percent.

I feel grateful that I have been granted nine years of good health and productivity since the original diagnosis, but now I am face to face with dying. The cancer occupies a third of my liver, and though its advance may be slowed, this particular sort of cancer cannot be halted.

It is up to me now to choose how to live out the months that remain to me. I have to live in the richest, deepest, most productive way I can. In this I am encouraged by the words of one of my favorite philosophers, David Hume, who, upon learning that he was mortally ill at age 65, wrote a short autobiography in a single day in April of 1776. He titled it “My Own Life.”

“I now reckon upon a speedy dissolution,” he wrote. “I have suffered very little pain from my disorder; and what is more strange, have, notwithstanding the great decline of my person, never suffered a moment’s abatement of my spirits. I possess the same ardour as ever in study, and the same gaiety in company.”

I have been lucky enough to live past 80, and the 15 years allotted to me beyond Hume’s three score and five have been equally rich in work and love. In that time, I have published five books and completed an autobiography (rather longer than Hume’s few pages) to be published this spring; I have several other books nearly finished.

Hume continued, “I am ... a man of mild dispositions, of command of temper, of an open, social, and cheerful humour, capable of attachment, but little susceptible of enmity, and of great moderation in all my passions.”

Here I depart from Hume. While I have enjoyed loving relationships and friendships and have no real enmities, I cannot say (nor would anyone who knows me say) that I am a man of mild dispositions. On the contrary, I am a man of vehement disposition, with violent enthusiasms, and extreme immoderation in all my passions.

And yet, one line from Hume’s essay strikes me as especially true: “It is difficult,” he wrote, “to be more detached from life than I am at present.”

Over the last few days, I have been able to see my life as from a great altitude, as a sort of landscape, and with a deepening sense of the connection of all its parts. This does not mean I am finished with life.

On the contrary, I feel intensely alive, and I want and hope in the time that remains to deepen my friendships, to say farewell to those I love, to write more, to travel if I have the strength, to achieve new levels of understanding and insight.\footnote{Oliver Sacks, a professor of neurology at the New York University School of Medicine, is the author of many books, including “Awakenings” and “The Man Who Mistook His Wife for a Hat.”}
\end{quote}

\begin{quote}
I have been increasingly conscious, for the last 10 years or so, of deaths among my contemporaries. My generation is on the way out, and each death I have felt as an abruption, a tearing away of part of myself. There will be no one like us when we are gone, but then there is no one like anyone else, ever. When people die, they cannot be replaced. They leave holes that cannot be filled, for it is the fate — the genetic and neural fate — of every human being to be a unique individual, to find his own path, to live his own life, to die his own death.

I cannot pretend I am without fear. But my predominant feeling is one of gratitude. I have loved and been loved; I have been given much and I have given something in return; I have read and traveled and thought and written. I have had an intercourse with the world, the special intercourse of writers and readers.

Above all, I have been a sentient being, a thinking animal, on this beautiful planet, and that in itself has been an enormous privilege and adventure.
\end{quote}

To load the template just type:

\begin{verbatim}
\cxset{%
 custom=versochapter,
 chapter image=vespa.jpg,
 chapter opening=verso}
\end{verbatim}





\makeatletter
\@specialfalse
\makeatother
^^A
%   ^^A\makeatletter
\cxset{chapter author/.store in=\chapterauthor@cx}
\cxset{style41/.style={
 color=purple,
 name=CHAPTER,
 numbering=WORDS,
 number font-size=\large,
 number font-family=\rmfamily,
 number font-weight=\normalfont,
 number before={},
 number position=rightname,
 chapter font-family=\rmfamily,
 chapter font-weight=\normalfont,
 chapter before=\hfill,
 chapter spaceout=none,
 number after=,
 chapter after=\hfill\hfill\vskip0pt,
 chapter color = black!90,
 number color= black!90,
 title beforeskip=,
 title before=\begin{center},
 title after=\par\end{center},
 title spaceout=none, 
 title font-family=\rmfamily,
 title font-color= black!80,
 title font-weight=\normalfont,
 title font-size=\LARGE,
 chapter font-size=\large,
 section numbering=none,
 section indent=0pt,
 section align=\centering,
 section font-shape=\upshape,
 section font-weight=\normalfont,
 section font-size=\large,
 section spaceout=soul,
 section beforeskip=10pt,
 section afterskip=10pt,
}}

\cxset{style41,
       chapter author=Yiannis Lazarides,
       epigraph width=0.85\textwidth,
       epigraph text align=left,
       epigraph source align=right,
       epigraph rule width=0pt,
       epigraph afterskip=50pt,
       author block=true,
       author block format=\normalfont\itshape\Large\centering,
}

\renewsection

\addauthors{Dr Yiannis Lazarides}
\chapter{INTRODUCTION TO CHAPTER STYLE FORTY ONE}

\label{ch:41}
\epigraph{The existence of an area of free land, its continuous recession, and the advance of American
settlement westward explain American development.}{Frederick Jackson Turner, \textit{The Significance of the Frontier in American\\ History,} Columbian Exploration, Chicago, July 12, 1893}

A classical style chapter style with finely spaced out letters. A number of books spell out the chapter numbers. In general I find this as a good idea as sometimes the numbers don't blend in very well with the design.

\begin{figure}[ht]
\centering
\fbox{\includegraphics[width=0.5\textwidth]{./chapters/chapter41}}
\end{figure}

This book has different chapters written by different authors and the author's name appear below an ornament. Don't dismiss ornaments as old fashioned as a lot of modern books still use them.

\begin{lstlisting}
\cxset{style41,
         chapter author=Yiannis Lazarides}
\end{lstlisting}

The ornaments I used was from the \texttt{fourier-orns} package. Here is a MWE if you want to experiment with various designs.


\begin{lstlisting}
\documentclass{article}
\usepackage{fourier-orns}
\begin{document}
\Huge
\textxswup\textxswdown
\decoone\decotwo
\decothreeleft\decothreeright
\decofourleft\decofourright
\floweroneleft\floweroneright
\end{document}
\end{lstlisting}

\section{THINGS THAT ARE NOT AUTOMATED}

If you need the title to be spaced out using the soul package and you have a line break, the package will issue the error `reconstruction failed'. In this case it is better to include the spaceout commands in the title (subject to hyperref not breaking up everything).


\begin{verbatim}
\chapter{\so{INTRODUCTION TO}\\ \so{CHAPTER STYLE FORTY ONE}}
\end{verbatim}

^^A
%   \cxset{style42/.style={
 name=,
 chapter opening=right,
 numbering= arabic,
 number font-size=\huge,
 number before={},
 number position=leftname,
 chapter before=\vspace*{10pt},
 number after=\hfill\hfill,
 chapter after=\hfill\hfill\vskip20pt ,
 number color= gray,
 title font-family=\sffamily,
 title font-color= black!80,
 title font-weight=,
 title font-size=\Huge,
 title before=,
 title after=\par,
 author block=false,
 section numbering=none,
 epigraph width=0.85\textwidth,
 epigraph text align=left,
 epigraph source align=right,
 epigraph rule width=0pt,
 epigraph afterskip=30pt,
}}

\cxset{style42}
\chapter{Introduction to Style Forty Two}

\epigraph{Tell me, O Muse, of that ingenious hero who trawled far and wide after he had
sacked the famous town of Troy. Many cities did he visit, and many were the nations with whose manners
and customs he was acquainted; moreover he suffered much by sea while trying to save his own life and bring
his men safely home \ldots }{Homer, \textit{The Odyssey}}

Style 42 is shown in the following figure:

\begin{figure}[ht]
\centering
\includegraphics[width=0.6\textwidth]{./chapters/chapter42.png}
\end{figure}
The distinguishing characteristics of this chapter are that it has an epigraph and is composed of very simple stylistic elements. The epigraph is placed quite a bit lower than the chapter title. The heading style is just the page number and underlined.



^^A
%   %\newfontfamily\cambria{Cambria.ttc}
%\newfontfamily\calibri{Carlito-Regular.ttf}
%\newfontfamily\calibril{Calibril.ttf}
\cxset{syrian revolution/.style={%
 name=CHAPTER,
 number dot=,
 numbering=arabic,
 number font-size=\Large,
 number font-weight=\normalfont,
 number before=\kern1em,
 number position=rightname,
 chapter color= black!80,
 %chapter font-weight=\normalfont,
 chapter font-family=\rmfamily,
 chapter font-size=\Large,
 chapter before=\par\hfill\hfill,
 chapter spaceout=none,
 number after=,
 chapter after=,
 number color= black!95,
 chapter title align=right,
 chapter title width=\textwidth,
 title font-family=,
 title font-color= black!95,
 title font-weight=\itshape,
 title font-size=LARGE,
 title font-shape=\itshape,
 title spaceout=none,
 title beforeskip=,
 chapter title width=\textwidth,
 chapter title text-align=right,
 epigraph width=0.95\textwidth,
 epigraph font-size=\normalfont,
 author block=false}}

\cxset{syrian revolution}

\cxset{headings ruled-01}

\debugtitle
\chapter{Introduction}


\label{style43}

\epigraph{The Jebel Druse is a country of great feudal chiefs, whose efforts are
directed to preserving the powers by which they live.What we call
progress means in their eyes the loss of their privileges and later on
perhaps the partition of their lands. With regard to the inhabitants,
who are ignorant or unmindful of any better fate, they are deeply rooted
in their serfdom and are as conservative as their masters. They have no
aspirations for a system of greater social justice nor [sic] for a better
communal life.}{---Testimony to the League of Nations Permanent Mandates\\
Commission investigating the Syrian Revolt, Geneva, 1926}

\epigraph{Syrians, remember your forefathers, your history, your heroes, your
martyrs, and your national honor. Remember that the hand of God is
with us and that the will of the people is the will of God. Remember
that civilized nations that are united cannot be destroyed.

The imperialists have stolen what is yours. They have laid hands on
the very sources of your wealth and raised barriers and divided your
indivisible homeland. They have separated the nation into religious
sects and states. They have strangled freedom of religion, thought,
conscience, speech, and action.We are no longer even allowed to move
about freely in our own country.

To arms! Let us realize our national aspirations and sacred hopes.

To arms! Confirm the supremacy of the people and the freedom of
the nation.

To arms! Let us free our country from bondage.}{---Excerpt from a rebel manifesto signed\\ by Sultan
al-Atrash and issued on 23 August 1925}

\dropcap{T}{his style} is reminiscent of the stylistic elements found in Tufte's books with the chapter title set in italics.

The Great Syrian Revolt was the first episode in a contest that has defined
much of modern Syrian history. In
the \textit{The Great Syrian Revolution and the Rise of Arab Nationalism},  Michael Provence narrated this period and its main characters and described the rising of Arab nationalism.\footnote{published by the University of Texas at Austin (2005)}. From a typographical point of view the book attracted my attention due to the sometimes lengthy epgraphs and the weaving of simple features with the text. For most chapters the best part of the first page is taken by epigraphs, but as you can see from the image, the ugly ``This Page intentionally left blank'' is all over the place and I have removed it from the template  The chapter opens on an even page and bear no headers or footers. The large dropcap at the start of the chapter text balances the ragged left elements of the chapter block.

Template 43 (style43) does not specify any particular font, but a Garamond would suit this style well in print. I have set the paragraph first line indentation as 2em.

 \begin{verbatim} 
\parindent2em

\cxset{chapter font-weight=normalfont,
          chapter spaceout=none,
          chapter name=CHAPTER,
          section font-size=small,
          section font-family=rmfamily,
          section indent=2em,
          section align=left,
          section color=black,
          section afterskip=\baselineskip}
\end{verbatim}

\cxset{chapter font-weight=\normalfont,
          section font-weight=\normalfont,
          section font-size=small,
          section font-family=rmfamily,
          section indent=2em,
          section align=left,
          section color=black,
          section afterskip=0.5\baselineskip,
          section beforeskip=1\baselineskip}
              
                              
\section{SECTIONING}

This is really simple with small caps sections indented at parskip. The indentation is at least 2ems wide.
The template suits a book with a lot of text and very few figures and empty spaces. The book has only
nine figures and it includes a List of Figures, labelled `Maps and Illustrations’. I will show you how to style
this list a bit later. 
          
One notable feature is that it includes many letters, which they are quoted in italics. This is best achieved
with a special environment and we will examine the options a bit later as well.

\example
\begin{figure}[ht]
\centering
\includegraphics[width=\textwidth]{./images/syria-figures.jpg}
\caption{Figures are typeset as shown in this caption.}
\end{figure}



\begin{figure}[ht]
\centering
\includegraphics[width=0.95\textwidth]{chapter43.jpg}\par
\includegraphics[width=0.95\textwidth]{chapter43a.jpg}
\end{figure}
\everypar{}


\example Naming the template and adding it to the default templates available with your distribution. This template I have named it the \textit{Syrian Revolution} template following the title of the book and as it is
easier to remember than style43. The style43 also exists as an alias. 

\begin{verbatim}
\cxset{syrian revolution/.style={%
 name=CHAPTER,
 number dot=,
 numbering=arabic,
 number font-size=\Large,
 number font-weight=\normalfont,
 number before={},
 number position=rightname,
 chapter color= black!80,
 chapter font-family=\arial,
 chapter font-weight=\normalfont,
 chapter font-size=\Large,
 chapter before=\par\hfill\hfill,
 chapter spaceout=none,
 number after=,
 chapter after=\vskip20pt,
 number color= black!95,
 title font-family=,
 title font-color= black!95,
 title font-weight=\itshape,
 title font-size=\LARGE,
 title font-shape=\itshape,
 title spaceout=none,
 title beforeskip=\hfill,
 epigraph width=0.95\textwidth,
 epigraph font-size=\normalfont,
 author block=false}}

\end{verbatim}














^^A
%   <<<<<<< HEAD
I battled a bit to set everything and in retrospect chapters like this should have been programmed as specials. However, as soon as they are set, adjustments are easily done.



%% Style 50

\cxset{style50/.style={
 name=,
 numbering=arabic,
 number font-size=\LARGE,
 number font-weight=\bfseries,
 number before={},
 number position=rightname,
 number dot=.,
 chapter color={black!80},
 chapter font-size=,
 chapter before=,
 number after=,
 chapter after=,
 number color=\color{black!80},
 title font-family=\rmfamily,
 title font-color=\color{black!80},
 title before=,
 title after=\par,
 title font-weight=\bfseries,
 title font-size=\LARGE,
 title beforeskip=\space,
 header style=empty,
 author block=true,
 author names=\textsc{James A. Russel and\\[-1.5pt] Jos\'e Miguel Fernandez-Dols },
 author block format=\normalfont\large,
 epigraph width=0.8\textwidth, epigraph align=left}}

\cxset{style50}
\chapter[Chapter Style Fifty]{Introduction to Chapter \\Style Fifty}

\epigraph{\raggedleft The human face -- in repose and in movement, at the moment of death as in life, in silence and in speech, when seen or seemed from within, in actuality or as recorded in art or recorded by the camera}{F. Ekman}

This is an unusual book with a rather unique style. The vertical rule is simple and breaks the monotony of a book that is heavy on text.
\begin{figure}[ht]
\includegraphics[width=0.48\textwidth]{./chapters/chapter50}\hfill
\includegraphics[width=0.48\textwidth]{./chapters/chapter50a}
\caption{Style 50 from the Oxford Handbook of Cuneiform Culture.}
\end{figure}

This style is very modern and typical of many computer books. The difficulty is in integrating all the page elements to make it work flawlessly.

The psychology of facial
expression
Edited by
James A. Russell
University of British Columbia
Jose Miguel Fernandez-Dols
Universidad Autonoma de Madrid, Cambridge University Press.

=======
I battled a bit to set everything and in retrospect chapters like this should have been programmed as specials. However, as soon as they are set, adjustments are easily done.



%% Style 50

\cxset{style50/.style={
 name=,
 numbering=arabic,
 number font-size=\LARGE,
 number font-weight=\bfseries,
 number before={},
 number position=rightname,
 number dot=.,
 chapter color={black!80},
 chapter font-size=,
 chapter before=,
 number after=,
 chapter after=,
 number color=\color{black!80},
 title font-family=\rmfamily,
 title font-color=\color{black!80},
 title before=,
 title after=\par,
 title font-weight=\bfseries,
 title font-size=\LARGE,
 title beforeskip=\space,
 header style=empty,
 author block=true,
 author names=\textsc{James A. Russel and\\[-1.5pt] Jos\'e Miguel Fernandez-Dols },
 author block format=\normalfont\large,
 epigraph width=0.8\textwidth, epigraph align=left}}

\cxset{style50}
\chapter[Chapter Style Fifty]{Introduction to Chapter \\Style Fifty}

\epigraph{\raggedleft The human face -- in repose and in movement, at the moment of death as in life, in silence and in speech, when seen or seemed from within, in actuality or as recorded in art or recorded by the camera}{F. Ekman}

This is an unusual book with a rather unique style. The vertical rule is simple and breaks the monotony of a book that is heavy on text.
\begin{figure}[ht]
\includegraphics[width=0.48\textwidth]{./chapters/chapter50}\hfill
\includegraphics[width=0.48\textwidth]{./chapters/chapter50a}
\caption{Style 50 from the Oxford Handbook of Cuneiform Culture.}
\end{figure}

This style is very modern and typical of many computer books. The difficulty is in integrating all the page elements to make it work flawlessly.

The psychology of facial
expression
Edited by
James A. Russell
University of British Columbia
Jose Miguel Fernandez-Dols
Universidad Autonoma de Madrid, Cambridge University Press.

>>>>>>> merged

%  \cxset{
 name=CHAPTER,
 numbering=arabic,
 number font-size=large,
 number position=rightname,
 chapter spaceout=soul,
 chapter color=black,
 chapter font-size=large,
 chapter before=\hrule\vskip1pt\hfill,
 chapter after=,
 number before=,
 number after=\hfill\hfill\vskip1pt\hrule\vskip0pt\par,
 chapter margin left=0pt,
 number color=black,
 title font-family=bfseries,
 title font-color=black,
 title font-weight=,
 title font-size=Huge,
 title before=,
 title after=\par,
 title beforeskip=\vspace*{1cm},
 title afterskip=,
 title margin bottom=1.5cm,
 title margin-left=0pt,
 chapter title width=0.8\textwidth,
 chapter title align=centering,
 epigraph width=0.85\textwidth,
 epigraph align=center,
 header style=empty,
 epigraph rule width=0pt,
 section font-weight=bold}

\debugtitle
\chapter{The Chomskyan Revolution, Introduction to Style Fifty Two}

\epigraph{In the late forties \ldots\ it seemed to many that the conquest of syntax finally lay open before the profession. At the beginning of the fifties confidence was running high.}{--H. Allan Gleason}

\section{Looking for Mr. Goodstructure}

\lorem
\begin{figure}[ht]
\centering
\fbox{%
\includegraphics[width=0.35\textwidth]{./chapters/chapter52.png}
\includegraphics[width=0.35\textwidth]{./chapters/chapter52a.png}}

\caption{Style 50 from the Oxford Handbook of Cuneiform Culture.}
\end{figure}

This style is very modern and typical of many computer books. The difficulty is in integrating all the page elements to make it work flawlessly.

%  
%% Style 54

\cxset{appendix name/.store in=\appendix@cx}
\cxset{
 name=,
 numbering=arabic,
 number font-size=LARGE,
 number before={},
 number after=,
 number dot=,
 number position=rightname,
 number color=blue,
 chapter color=blue,
 chapter font-size=Large,
 chapter before=\par\hfill\hfill,
 chapter after=\hfill,
 title font-family=\rmfamily,
 title font-color=blue,
 title font-weight=,
 title font-size=\LARGE,
 title beforeskip=\hfill,
 title afterskip={\vspace*{20pt}},
 header style=empty}
\chapter{{STYLE FIFTY FOUR}}


This is an unusual book with a rather unique style. The vertical rule is simple and breaks the monotony of a book that is heavy on text.\index{rules!style 54}

\begin{figure}[ht]
\fbox{\includegraphics[width=0.48\textwidth]{./chapters/chapter54.png}}\hfill
\fbox{\includegraphics[width=0.48\textwidth]{./chapters/chapter54a.png}}
\caption{Style 54 from Steward's Calculus.}
\end{figure}

%  
%  \fi

