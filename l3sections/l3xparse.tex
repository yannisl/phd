\chapter{A more flexible and robust method of defining functions with LaTeX3 and xparse}
\label{ch:xparse}

\section{Introduction}

The \latex2e |\newcommand| macro is the most popular user command for creating macros. The command
provides a number of checks and also has the ability to define macros with an optional argument.
For more complex macros, users have to revert to using |\def| or use packages which extend |\newcommand|,
such as \pkg{twoopt}.\footcite{twoopt}

The \LaTeX3 Team developed the package \pkg{xparse} to provide document level 
authors with some powerful commands that extend those such as \cs{newcommand}
of \latexe. The code is been stable and the interface is not expected to change. 
Although targetted at document level, the commands offered can be used effectively to produce code used in packages.\footnote{\protect\url{http://tex.stackexchange.com/questions/98152/always-use-newdocumentcommand-instead-of-newcommand}} The functions offered by the package enable commands with star, or optional arguments to be produced easily.\tcbdocmarginnote{Revised\\ July 2018} A good introduction to the package was published in TUGboat by Joseph Wright.\footcite{wright2010} 


\section{Creating document commands}

\begin{docCommand}{DeclareDocumentCommand}{\marg{function}\marg{argument specification}\marg{code}}
This family of commands are used to create a document-level \emph{function}. The argument
specification for the function is given by \textit{arg spec}, and expanding to be replaced by the
\textit{code}. Unlike \latex's definition commands, all xparse commands take two arguments.
The first one is the \textit{argument specifier}, and the second is the \textit{code.}
\end{docCommand}

\begin{texexample}{DeclareDocumentCommand}{l3:1}
\DeclareDocumentCommand \foo { m o m } { 
    arg 1 = #1, arg 2 = #2,  arg 3 = #3 }
\foo{A}[B]{C}  

\foo{A}{B}    
\end{texexample}

In the example above |{m o m}| is the argument specifier. It tells the function  to expect, two mandatory arguments and one optional denoted by the letter \textbf{o}. There are many more specifiers. For example \textbf{O} takes an parameter as a default value.\index{argument specifier}

\begin{texexample}{DeclareDocumentCommand}{l3:1}
\DeclareDocumentCommand \foo { m O{\ldots} m } { 
    arg 1 = #1, arg 2 = #2,  arg 3 = #3 }
\foo{A}[B]{C}  

\foo{A}{B}    
\end{texexample}

The argument markers can be entered in any order. In the following example we will also add an optional argument in a curly bracket. Although this is frowned upon in certain contexts it is useful. Consider the case of a chapter title that also has a subtitle. \docAuxCommand*{Chapter}, it maybe more natural and useful to have input of the form, as shown in Example~\ref{l3:g}. 

\begin{texexample}{DeclareDocumentCommand}{l3:g}
\DeclareDocumentCommand \MyChapter { o m g } { 
\centering #2\par #3\par }
    
\MyChapter{THIS IS THE MAIN TITLE}{This is a subtitle}  

\MyChapter[]{THIS IS THE MAIN TITLE}{This is a subtitle}        
\end{texexample}

\begin{texexample}{DeclareDocumentCommand}{l4:g}
\DeclareDocumentCommand \MyChapter {s o m g } { 
\IfBooleanTF {#1} {\gdef\fonta{\bfseries\selectfont}}{\gdef\fonta{}}
\IfNoValueTF {#2} {No option\par}{#2}

\centering {\fonta #3}\par #4\par 
  }   
\MyChapter{ THIS IS THE MAIN TITLE}  

\MyChapter*[short title]{THIS IS THE MAIN TITLE}{This is a subtitle}        
\end{texexample}

As you can see, it is fairly easy to produce starred and unstarred versions of commands as well as as any form of optional arguments. Let us now see some of the other command definition functions, before we continue with other specifiers.

\begin{docCommand}{NewDocumentCommand}{\marg{function}\marg{argument specification}\marg{code}}
will issue an error if \meta{function} has already been defined
\end{docCommand}

\begin{docCommand}{RenewDocumentCommand}{\marg{function}\marg{argument specification}\marg{code}}
For changing a definition,
issuing an error message if the macro does
not already exist.
\end{docCommand}


As the \cmd{\DeclareDocumentCommand} always updates a definition, it is used for the examples in this chapter to avoid any errors.

What sets the above commands apart from \latexe \cmd{\newcommand} is the argument specification.



\begin{texexample}{DeclareDocumentCommand}{l3:1}
\DeclareDocumentCommand \teststar {s o m } { 
\IfBooleanTF {#1}
  { \typesetnormalchapter {#2} {#3} }
  { \typesetstarchapter {#3} }
}  
\newcommand\typesetnormalchapter[2][]{
  normal chapter
}
\newcommand\typesetstarchapter[1]{
  #1
}
\teststar{Test}

\teststar*{test}
\end{texexample}    

    
The argument specification \textbf{m o m} in the example enables the function to accept three arguments, two mandatory and one optional. 


\section{Argument specifications}

The basic idea of an argument specification is that each argument is listed as a single letter. 
As the argument specification is a mandatory argument, a function with no arguments still needs an arg spec. The number of letters in the argument specification tells you how many arguments a function takes, while the letters themeselves determine the type of argument. Unlike |\newcommand| or |\def| a function with no arguments still needs to be specified with an empty
argument specification.

\begin{texexample}{Empty arg spec}{}
\DeclareDocumentCommand\atest{}{some text}
\atest
\end{texexample}

There is a wide range of argument specifier letters. Mandatory arguments are created using the letter \textbf{m}.

\begin{marglist}
\item [m] Mandatory. This is a standard mandatory argument, which can either be a single token alone or multiple tokens surrounded by curly braces. Regardless of the input, the argument will
be passed to the internal code surrounded by a brace pair. This is the \pkgname{xparse} type
specifier for a normal \tex argument.
\end{marglist}

\begin{texexample}{Mandatory Values, verbatim}{}
\DeclareDocumentCommand\testverbatim{ v }{
    \ttfamily#1
}
\testverbatim+ \this is a test +

\testverbatim * &^%$#\test *

\testverbatim{\ttfamily \bfseries\normalfont test}
\end{texexample}

The \textbf{l} specifier reads its argument, until it encounters a left brace. It is equivalent to \tex \# argument. Can be used basically for |\hbox| type comands.

\begin{texexample}{Mandatory arguments l-specifier}{}
\DeclareDocumentCommand\myhbox{ l }{
   \hbox to \dimexpr(#1)\relax
}
\fbox{\myhbox 12pt+1em+13ex  {test}}
\end{texexample}

One difference between |xparse| functions and |\newcommand| is that the functions defined are not |\long|. In |xparse| the argument specidfier can b used to allow paragraph tokens or not. This is done by preceding the arg spec letter by |+|:

\begin{texexample}{Long macros with xparse}{ex:xparse1}
\DeclareDocumentCommand \mylorem {m +m }{%
  #1
  #2
}

\mylorem{\hrule}{\lorem\par}
\end{texexample}

If you expect longer texts, it is always a good idea to make the macros |long|. 

\begin{margoptionslist}
\item [o] Optional argument in  []. Returns |-NoValue-| if not present.
\item [O] As for \textbf{o}, but returns \meta{default}, if no value is given. Should be given as |O{default}|.

\item [s] Starred version
\item [v] Verbatim. Reads an argument “verbatim”, between the following character and its next occurrence,
in a way similar to the argument of the LATEX2" command \cmd{\verb}. Thus
a v-type argument is read between two matching tokens, which cannot be any of
\%, \#, \{, \}, \^ or  . The verbatim argument can also be enclosed between braces,
\{ and \}. A command with a verbatim argument will not work when it appears
within an argument of another function.
\item [l] An argument which reads everything up to the first open group token: in standard
\latex this is a left brace.
\item [u] Reads an argument “until \meta{tokens} are encountered, where the desired \meta{tokens}
are given as an argument to the specifier: |u|\meta{tokens}.
\item [d] An optional argument that is delimited. 
\item [D] As for d, but returns \meta{default} if no value is given: D\meta{token1} \meta{token2}\marg{default}.
Internally, the o, d and O types are short-cuts to an appropriated-constructed D
type argument.
\item [t]  An optional \meta{token}, which will result in a value \cs{BooleanTrue} if \meta{token} is 
            present and \cs{BooleanFalse} otherwise. Given as \meta{token}.
\item [g] An optional argument given inside a pair of \tex group tokens (in standard \latex,
              \{ . . . \}, which returns |-NoValue-| if not present.
\item [G] As  for \textbf{g} but returns \meta{default} if no value is given: |G|\marg{default}.
\end{margoptionslist}

\begin{texexample}{Default Values}{}
\DeclareDocumentCommand\testcolor{ O{red} m }{
    \textcolor{#1}{#2}
}
\testcolor{This is typeset in red}
\testcolor[blue]{This is typeset in blue.}
\end{texexample}

\section{Testing special values}

The optional arguments of a function defined using |xparse| use dedicated variables to return
information about the naure of the argument received.

\begin{docCommand}{IfNoValueTF}{\Arg{argument}\Arg{true code}\Arg{false code}}
The function tests if the argument has a value and executes the true of false code, by means
of a |-NoValue-| marker. 
\begin{texexample}{special values}{}
\DeclareDocumentCommand\doccmd{O{red} m}
    {
        \IfNoValueTF{#1}
            {\doccmdnocolor{#1}}
            {\doccmdcolor{\textcolor{#1}{#2}}}
     }
\newcommand\doccmdnocolor[1]{#1}
\newcommand\doccmdcolor[2]{#1 #2}     
This is \doccmd[blue]{text}  and this is \doccmd{text}.   
\end{texexample}
\end{docCommand}

\begin{texexample}{special values}{}
\DeclareDocumentEnvironment{allbold}{o}
    {
        \bfseries 
        \IfNoValueTF{#1}
            {\color{red}}
            {\color{#1}}
    }
    {                 }
\begin{allbold}[magenta]
\lorem
\end{allbold}
\end{texexample}

\begin{texexample}{variants}{ex:variants}
\ExplSyntaxOn
\cs_set:Npn \foo_something:Nn #1#2 {
   \csname\expandafter#1\endcsname{blue}{a a a} 
   { #2}
  }
\cs_generate_variant:Nn \foo_something:Nn { c }
%\meaning\foo_something:cn
\ExplSyntaxOff
\lorem
\end{texexample}

\section{Robustness}

|xparse| craetes functions which are naturally \enquote{robust}. This means that they can be used in section names
and so on without needing to be protected using
|\protect|. This makes using functions created using
xparse much more reliable than using those created
using |\newcommand|, particularly when there are optional
arguments.
xparse is also designed so that optional arguments
can themselves contain optional material. For
example, if you try:

\begin{teXXX}
\newcommand*\foo[2][]{%
% Code
}
\foo[\baz[arg1]{arg2}]{arg3}
\end{teXXX}

you will find that |\foo| will pick up ‘|\baz[arg1|’ as
\#1 and ‘arg2’ as \#2: not what is intended. However,
the same code with xparse

\begin{teXXX}
\DeclareDocumentCommand \foo { o m } {%
% Code
}
\foo[\baz[arg1]{arg2}]{arg3}
\end{teXXX}
will parse ‘|\baz[arg1]{}|’ as \#1 and ‘arg’ as \#2, as
anticipated.\footcite{wright2010}

\section{Using xparse in Packages}

There is no reason not to use \pkg{xparse} in your packages. The convention here is to use it as:

\begin{teX}
\NewDocumentCommand \bibnotemark { o } {
  \IfNoValueTF {#1} 
    {
      \int_gincr:N \g_@@_note_int
      \@@_mark_note:x { \@@_note_name: }
    }
    { \@@_mark_note:x { \@@_insert_refsection: #1 } }
}
\end{teX}

The strange |\@@| will be replaced by l3doc when the document is processed to print the module name and mark the macros as internal. The code used above is from |notes2bib|.

In a package it is advisable to use the NewDocumentCommand in order to ensure that the command is not overwritten by other packages or the author unintentionally.







