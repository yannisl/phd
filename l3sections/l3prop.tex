\chapter{LaTeX3 properties}


 \LaTeX3 implements a \enquote{property list} data type, which contain
 an \emph{unordered list} of entries each of which consists of a \meta{key} and
 an associated \meta{value}. The \meta{key} and \meta{value} may both be
 any \meta{balanced text}. It is possible to map functions to property lists
 such that the function is applied to every key--value pair within
 the list.

 Each entry in a property list must have a unique \meta{key}: if an entry is
 added to a property list which already contains the \meta{key} then the new
 entry will overwrite the existing one. The \meta{keys} are compared on a
 string basis, using the same method as \docAuxCommand*{str_if_eq:nn}.

 Property lists are intended for storing key-based information for use within
 code.  This is in contrast to key--value lists, which are a form of
 \emph{input} parsed by the \pkg{l3keys} module.

 \section{Creating and initialising property lists}

 \begin{docCommand}{prop_new:N or :c}{\meta{property list}}
   Creates a new \meta{property list} or raises an error if the name is
   already taken. The declaration is global. The \meta{property list} will
   initially contain no entries.
 \end{docCommand}

 \begin{docCommand}{prop_clear:N (:n:c) }{ \meta{property list}}
   Clears all entries from the \meta{property list}.
 \end{docCommand}

 \section{Adding entries to property lists}

 \begin{docCommand}{prop_put:Nnn}{ \meta{property list} \marg{key} \marg{value}}
 
   Adds an entry to the \meta{property list} which may be accessed
   using the \meta{key} and which has \meta{value}. Both the \meta{key}
   and \meta{value} may contain any \meta{balanced text}. The \meta{key}
   is stored after processing with \docAuxCommand*{tl_to_str:n}, meaning that
   category codes are ignored. If the \meta{key} is already present
   in the \meta{property list}, the existing entry is overwritten
   by the new \meta{value}.
 \end{docCommand}
 
 \begin{texexample}{Property lists}{ex:proplits}
 \ExplSyntaxOn
 \prop_new:N \g_tut_temp_prop
 \prop_gput:Nnn \g_tut_temp_prop {symbolic}{true}
 \prop_item:Nn \g_tut_temp_prop {symbolic}
 \ExplSyntaxOff
\end{texexample}

What happened here is we create the property list, using \docAuxCommand{tut_temp_prop}, adding a key and value and the using it through \docAuxCommand {tut_temp_prop}.

 \section{Recovering values from property lists}

   \begin{docCommand}{prop_get:NnN}{ \meta{property list} \marg{key} \meta{tl var}}
   Recovers the \meta{value} stored with \meta{key} from the
   \meta{property list}, and places this in the \meta{token list
   variable}. If the \meta{key} is not found in the
   \meta{property list} then the \meta{token list variable} will
   contain the special marker \docAuxCommand*{q\_no\_value}. The \meta{token list
     variable} is set within the current \TeX{} group. See also
   \docAuxCommand*{prop_get:NnNTF}.
  \end{docCommand}

Consider another example, this time we use a property lists to store information
on scripts and languages. We will use a new property list, name |l_<lang name>_prop| to store the
information.


\begin{texexample}[fontlower=\arial,]{Property Lists}{ex:proplists2}
\ExplSyntaxOn
% create new property list
\prop_new:c    {l_armenian_prop}

% save some properties
\prop_gput:cnn  {l_armenian_prop} {name     } {Armenian}
\prop_gput:cnn  {l_armenian_prop} {wfonts   } {NotoArmenian-Regular.ttf, Arial}
\prop_gput:cnn  {l_armenian_prop} {lfonts   } {NotoArmenian-Regular.ttf, Arial}
\prop_gput:cnn  {l_armenian_prop} {afonts   } {NotoArmenian-Regular.ttf, Arial}
\prop_gput:cnn  {l_armenian_prop} {group    } {Europe}
\prop_gput:cnn  {l_armenian_prop} {direction} {TLT}
\prop_gput:cnn  {l_armenian_prop} {hyphen   } {none}
\prop_gput:cnn  {l_armenian_prop} {sample   } {sample_armenian}

% get the group property and save it in a temporary variable  
\prop_get:cnN   {l_armenian_prop} {group   } \l_tmpa_tl
\tl_use:N \l_tmpa_tl
\ExplSyntaxOff
\end{texexample}

Using a property list is a much neater method than using lower level commands such as
|\csname| to store information. Property lists are used in the two largest \pkg{expl3} packages, \pkg{siunitx} and \pkg{fontspec}. As we will also see in the exampels that 
follow they can be used to automate the generation of code.


\section{Property list conditionals}







