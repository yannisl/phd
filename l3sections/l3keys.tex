\chapter{LaTeX3 Key value system}
\label{l3:keys}
The key-value system has been discussed earlier but avoided to cover the |l3keys| module of \latex3 until such time as the basics of the expl3 syntax was discussed. 


The l3keys modules provides general purpose keyval processing for |expl3| code. However, it does not interact with LaTeX2e's package or class option system. For that, you need to load some additional code, which is available in the package l3keys2e. This provides the \docAuxCommand*{ProcessKeysOptionscommand} to parse class/package options and process them using keyvals defined by l3keys.

The reason for this separation is that l3keys is intended to form part of a LaTeX3 kernel, while l3keys2e is tied to the LaTeX2e model for processing options. It seems extremely likely that a stand-alone LaTeX3 kernel will use keyval options 'natively' but with a different underlying implementation.

 The high level functions here are intended as a method to create
 key--value controls. Keys are themselves created using a key--value
 interface, minimising the number of functions and arguments
 required. Each key is created by setting one or more \emph{properties}
 of the key:
 \begin{verbatim}
   \keys_define:nn { mymodule }
     {
       key-one .code:n   = code including parameter #1,
       key-two .tl_set:N = \l_mymodule_store_tl
     }
 \end{verbatim}
 
  At a document level, |\keys_set:nn| will be used within a
 document function, for example
 \begin{verbatim}
   \DeclareDocumentCommand \MyModuleSetup { m }
     { \keys_set:nn { mymodule } { #1 }  }
   \DeclareDocumentCommand \MyModuleMacro { o m }
     {
       \group_begin:
         \keys_set:nn { mymodule } { #1 }
        ... Main code for the macro
       \group_end:
     }
 \end{verbatim}
 
 The process of incorporating a key value system into a macro or a package involves three steps. First the keys are defined then processed to set them to some values and lastly incorporated into a function or package.
 
 It is best to illustrate the process with a small example. Example\ref{ex:keyval1} defines two keys that affect the typesetting of paragraphs |parindent| and |parskip|. These are defined using the |.code|, pretty much the same way that |pgfkeys| that we discussed earlier defines keys. 
 
 \begin{texexample}{Key value}{ex:keyval1}
 \ExplSyntaxOn
 \keys_define:nn {scratch}
   {
      parindent .code:n = \parindent#1,
      parskip     .code:n = \parskip#1
   }
   
\DeclareDocumentCommand \MyModuleSetup { m }
     { \keys_set:nn { scratch } { #1 }  }
     
\DeclareDocumentCommand \MyModuleMacro { o }
     {
       \group_begin:
         \keys_set:nn { scratch } { #1 }
         % Main code for \MyModuleMacro
         \lorem\par
         \lorem\par
       \group_end:
     }
 \ExplSyntaxOff   
 \MyModuleSetup{parindent=1em, parskip=1pt}
 \MyModuleMacro [parindent=10pt, parskip=10pt]
 \end{texexample}
 
 
 The definition of the keys was achieved using the command:
 
\begin{docCommand}{keys_define:nn}{\marg{module}\marg{keyval list}}
The command parses the \meta{keyval list} and defines the keys associated there for \meta{module}. 
\end{docCommand}

The \meta{keyval list} should consist of one or more key names along with an associated
key \emph{property}. The properties of a key determine how it acts. The individual properties
are described in the following text; Note that the properties of the key begin from the dot (|.|) after the key name. The various properties available take no arguments or require one or more. All key definitions are local. 
 
 \begin{margoptionslist}
 \item [ .code:n] Stores the \meta{code} for execution when \meta{key} is used. 
 \item [.default:n] \meta{key} |.default:n| = \meta{default} This creates a \meta{default} value for \meta{key} if no value is given. This will be used if only the key name is given, but not if a blank \meta{value} is given. This behaviour is similar to the |pgfkeys| package.
 \item [.initial:n] \meta {key} |.initial:n| = \meta{value} Initialises the \meta{key} with the \meta{value}, equivalent to
|\keys_set:nn| \meta{module} \meta{key} = \meta{value}
 
 \item [.dim_set:N] \meta{key} |.dim_set:N| = \meta{dimension} Defines \meta{key} to set \meta{dimension} to \meta{value} (which must a dimension expression). If the variable does not exist, it will be created globally at the point that the key is set up.
 \end{margoptionslist}
 
%  \begin{texexample}{Key value}{ex:keyval1}
% \ExplSyntaxOn
% \dim_new:N \l_parskip
% \dim_new:N \l_parindent
% \keys_define:nn {scratch}
%   {
%      parindent .dim_set:N = \l_parindent,
%     % parindent .initial:n = 0pt,
%      parskip     .dim:n = \l_parskip,
%      %parskip     .initial:n = 1pt,
%      
%   }
%   
%\DeclareDocumentCommand \MyModuleSetup { m }
%     { \keys_set:nn { scratch } { #1 }  }
%     
%\DeclareDocumentCommand \MyModuleMacro { o }
%     {
%       \group_begin:
%         \dim_set_eq:NN \parindent \l_parindent
%         \dim_set_equal:NN \parskip \l_parskip
%         \keys_set:nn { scratch } { #1 }
%         % Main code for \MyModuleMacro
%         \lorem\par
%         \lorem\par
%       \group_end:
%     }
% \ExplSyntaxOff   
% 
% \MyModuleMacro [parindent=10pt, parskip=10pt]
% \end{texexample}

 
 \subsection{Choice keys}
 
 One of the most powerful features of modern key value packages is the ability to define and set keys for mutally exclusive values. In the |l3keys| module this can be achieved using the choice key.
 
 \begin{margoptionslist}
 \item [.choice] \meta{key} |.choice| This sets \meta{key} to act as a choice key. Each choice is then created, as discussed below:
 \end{margoptionslist}
 
 
 \begin{texexample}{Some choices}{}
 \ExplSyntaxOn
 \keys_define:nn { scratchi }
 {
    mycolor .choice:,
    mycolor/fire .code:n = {\color{red}},
    mycolor/sky .code:n = {\color{blue}},
    mycolor/orange .code:n = {\color{orange}},
    mycolor/lemon .code:n = {\color{yellow}},
    mycolor/grass .code:n = {\color{green}},
    mycolor .initial:n =sky,
    mycolor .default:n=orange,
    unknown .code:n={\color{red} ERROR},
 }

\keys_set:nn { scratchi } { mycolor=fire }  

\DeclareDocumentCommand \MyModuleSetup { m }
     { \keys_set:nn { scratchi } { #1 }  }

\DeclareDocumentCommand \MyModuleMacro { o +m}
     {
       \group_begin:
         \keys_set:nn { scratchi } { #1 }
         #2
         \group_end:
     }
     
 \ExplSyntaxOff
    
 \MyModuleSetup{mycolor=fire}

 \MyModuleMacro [mycolor=grass]{grass,} 
 
 \MyModuleMacro [mycolor]{default}
 
 \MyModuleMacro [apple]{}
 
 \MyModuleMacro [fire]{Fire}
\end{texexample} 
 
The |.choice|  key is a bit different from how it is used in the |xtemplate| package and |pgf| but probably easier to use and define. Of course our example was trivial and the colors should have been achieved with just one code key, capturing the value. It takes some practice to get used to all the types of keys available and to develop error free code easily, but by using a key value system, truly flexible, modern functions can be developed.
 

\subsection{Handling of unknown keys}
 
 Handling of unknown keys is similar to |pgf| where a key defined as |.unknown| is defined. 
 If a key has not previously been defined (is unknown), |\keys_set:nn| will look for a special
unknown key for the same module, and if this is not defined raises an error indicating that
the key name was unknown. This mechanism can be used for example to issue custom
error texts.

\begin{verbatim}
\keys_define:nn { mymodule }
{
unknown .code:n =
You~tried~to~set~key~’\l_keys_key_tl’~to~’#1’.
}
\end{verbatim}
 
 
 As for |pgf| there are many other key types and these are listed in the |l3keys| manual and are not listed here for brevity. 
 
 