\cxset{section align=left,
       section font-weight=bold}
       
\chapter{The LaTeX3 l3msg Module and how to use it for Error, Warning and other Messages}
\index{messaging>error}\index{messaging>warning}
\section{Introduction}

Messages need to be passed to the user by modules, either when errors occur or to indicate
how the code is proceeding. The l3msg module provides a consistent method for doing
this (as opposed to writing directly to the terminal or log).

The system used by l3msg to create messages divides the process into two distinct
parts. Named messages are created in the first part of the process; at this stage, no
decision is made about the type of output that the message will produce. The second
part of the process is actually producing a message. At this stage a choice of message
class has to be made, for example error, warning or info.

By separating out the creation and use of messages, several benefits are available.
First, the messages can be altered later without needing details of where they are used
in the code. This makes it possible to alter the language used, the detail level and so
on. Secondly, the output which results from a given message can be altered. This can be
done on a message class, module or message name basis. In this way, message behaviour
can be altered and messages can be entirely suppressed.

\section{Creating messages}

Messages \emph{must} be created before they can be used. This has the advantage that they can be used
over and over and also one could start thinking of internationalizing the package.

Messages can be created as either new or set and there is also a TF to check if the message exists:

\begin{docCommand}{msg_new:nnnn}{ \marg{module} \marg{message} \marg{text} \marg{more text} }
Creates a hmessagei for a given hmodulei. The message will be defined to first give htexti
and then \meta{more text} if the user requests it. If no \meta{more text} is available then a standard
text is given instead. Within htexti and more text four parameters (\#1 to \#4) can be
used: these will be supplied at the time the message is used. An error will be raised if
the \meta{message} already exists.
\end{docCommand}

\section{Messaging classes}

Messages are divided into categories termed message classes. 
These are according to \emph{severity}, \emph{fatal}, \emph{critical}, \emph{error}, \emph{warning} and \emph{info}. Each one has its own set of creation functions.

\begin{docCommand*}{msg_fatal:nnnnnn}{\marg{module} \marg{message} \marg{arg one} \marg{arg two} \marg{arg three}
\meta{arg four}}
Issues \meta{module} error \meta{message}, passing \meta{arg one} to \meta{arg four} to the text-creating
functions. After issuing a fatal error the \tex run will halt.
\end{docCommand*}



\begin{texexample}{Typical package error setup}{ex:errors}
% Error message example
%
% simulate LaTeX2e \fmversion
\def\fmversion{2000/11/12}
\makeatletter
\ExplSyntaxOn 

% create error boolean
\bool_new:N \l_mypackage_error_bool
 
% redirect package errors here  %(*@\label{l:warning}@*)
\cs_new_protected:Npn \mypackage_warning:nxx #1 #2 #3 
  {
    \bool_set_true:N \l_mypackage_error_bool
    \msg_info:nnxx { mypackage } { #1 } { #2 } { #3 }
  }
 
% define some error messages
\msg_set:nnnn { mypackage } { old-version }
  { LaTeX~source~files~more~than~5~year~old.~ Is~dated~(year:#1~date:#2-#3) }
  { Please~update~your~distribution~visit~ctan } %(*@\label{test}@*)
   

% check version number   	   	
\cs_new:Npn \mypackage_check_version:n #1 
  {
    \exp_after:wN \l_mypackage_check_version_aux:w #1\q_stop
  }

% check version number auxiliary 
% #1 relation
% #2 true code
% #3 false code  
\cs_new:Npn \l_mypackage_check_version_aux:w #1/#2/#3\q_stop 
  {
    \int_compare:nNnTF { ( \tex_year:D-#1 )*12 + (\tex_month:D-#2) } > { 65 }
      { \FAIL \mypackage_warning:nxx { old-version } { #1 } { #1 / #2 / #3 } }
      { \PASS } 
  }
  
\mypackage_check_version:n \fmversion 
\mypackage_check_version:n \fmtversion 
  
\ExplSyntaxOff
\end{texexample}

In Example~\ref{ex:errors} Line \ref{test} we define our own package command for issuing an error message, rather than typing |\msg_error:nnxx| directly. This is considered good practice and we avoid typing in \enquote{mypackage} all the time. 

\section{How to translate strings}
\index{internationalization>expl3}

I posted similar code to the |TX.SX| Q\&A site to elicit comments from other users, as to recommended best practices.
%http://tex.stackexchange.com/questions/246810/latex3-l3msg-best-practices 
At this moment in time I am not too sure if a LaTeX3 approach to internationalization is appropriate. If one looks at the complexities, it is preferable to use Lua. 





