\chapter{LaTeX3 quarks and recursion}
\label{ch:quarks}

\section{What are quarks?}
Quarks and recursion are central to the expl3 language. Quarks are a weird concept and is inherited from \tex’s way of scanning macro arguments.

But before we delve into the details of |expl3|'s quarks let us review \tex's delimited functions with an example. Consider the following example where we delimit the arguments of a macro |\test| with the control sequence |\texquark|. We do not need to define the |\texquark| and as we discussed in the section on macros it can even consist of the macro name itself. \tex will scan the input until the marker is found. It will also absorb the marker and do nothing about it.

\begin{texexample}{TeX quarks!}{}
\def\test#1\texquark{#1}
\test 123456\texquark \\
\def\test#1\test{#1}
\test 123456\test
\end{texexample}

In \latex2e macro delimiters are found all over the place, mostly in the form of \docAuxCommand{@nil}, \docAuxCommand{@nni}  or \docAuxCommand{@@}. See for example, how the \latex2e kernel defines lists.

 In \latex3 these have been termed \enquote{quarks} and \enquote{scan marks}. By convention all constants of type quark start out with |\q_| and scan marks start with |\s_|. Scan marks are reserved for internal use by the kernel and you should avoid using them in your code.\index{scan marks}\index{quarks}

They differ from the simple case above with the \tex example, in that they are used mostly indirectly. The \latex3 quarks, are defined so that they expand to themselves. As such they should never be executed directly in the code. This would cause and endless loop and cause either the program or even your computer to crash. The reason they hold a value, is that they can be tested, using |\ifx| which compares the meaning of two macros without expanding them. The equivalent construction in |expl3| is |\if_meaning:w|. We can use it at the next example. Note I gave used |\def| in th example to make it clearer, but one of course can use |\cs_set:Npn| or an equivalent function.

\begin{texexample}{Checking if is a quark}{ex:quarks}
\ExplSyntaxOn
\def\quark{\quark}
\cs_set_nopar:Npn \b {\quark}
\if_meaning:w  \quark\b
   \PASS
\else:
  \FAIL 
\fi:
\ExplSyntaxOff
\end{texexample}

This ingenious technique employed in Example~\ref{ex:quarks} depends on \tex’s ability to carry out comparisons without expanding the macros being compared. This way semantic definitions can be made for quarks and employed in generic recursive functions. 
Normally, you wouldn’t need to define your own quarks, as the ones made available by |expl3| are adequate for most tasks. If you have to create one, it can be created using:

\begin{docCommand}{quark_new:N}{ \meta{quark}}
Creates a new \meta{quark} which expands only to \meta{quark}. The \meta{quark} is defined globally, and an error message will be raised if the name was already taken.
\end{docCommand}

For example, the kernel defines two flavours of quarks to be used specifically for recursion and which we will use in the next section.

\begin{teXXX}
\quark_new:N \q_recursion_tail
\quark_new:N \q_recursion_stop
\end{teXXX}

Other flavours are lsited in the manual and summarized below:

\begin{docCommand}{q_stop}{ \meta{quark}}
Used as a marker for delimited arguments such as:
\begin{verbatim}
\cs_set:Npn \tmp:w #1#2 \q_stop {#1}
\end{verbatim}
\end{docCommand}





\section{Recursion}
 
One of the problem areas in programming recursion is to have a uniform interface to intercepting and terminating loops when one is doing recursion. \latex3 provides the building blocks.

First let us see an example:

\begin{texexample}{Recursion}{ex:l3recursion}
\ExplSyntaxOn

\cs_new:Npn \__my_decoration_fn:nn #1  {
  \str_if_eq:nnTF{e}{#1}
    {[{\bfseries\color{red}#1}]}
    {[#1]}
}

\cs_new:Npn \mymain #1 
{
      \__my_map:n #1 \q_recursion_tail\q_recursion_stop
}

\cs_new:Npn \__my_map:n #1 
  {
    \quark_if_recursion_tail_stop:n {#1}
    \__my_decoration_fn:nn  {#1} 
    \__my_map:n
  }
\ExplSyntaxOff
 
\mymain {abcdefgh}
\end{texexample}

The main function, will first call a mapping function leaving in the stream the following:
\medskip

\texttt{bcdefgh} {\hl{\textbackslash q\_recursion\_tail} \hl{\textbackslash q\_recursion\_stop}}
\medskip

On the second iteration the stream will be reduced by one token (b) and the remaing value will be:
\medskip

\texttt{cdefgh} {\hl{\textbackslash q\_recursion\_tail} \hl{\textbackslash q\_recursion\_stop}}
\medskip

This is repeated, until the quark is captured which causes the recursion to terminate. The termination is achieved by
the macro |\quark_if_recursion_tail_stop:n|. This will also absorb the |\_recursion_stop| quark. 

While the function is recursing we send the captured letter to a function to decorate and typeset it. This function can be programmed to do whatever you want to achieve. Note in the example it can only accept one argument.

Now what happens, if you wanted to capture two letters at a time or three letters at a time? The program would have to be modified as follows:

\begin{texexample}{Recursion}{ex:l3recursion}
\ExplSyntaxOn
\cs_new:Npn \__my_second_decoration_function:nn #1#2{
   {\color{red}
   [#1#2]}  
}
\cs_set:Npn \mymainother #1
{
 
   \__my_map_other:nn #1  \q_recursion_tail\q_recursion_tail\q_recursion_stop
}

\cs_new:Npn \__my_map_other:nn #1#2
  {
    \quark_if_recursion_tail_stop:n {#1}
    \quark_if_recursion_tail_stop:n {#2}
    \__my_second_decoration_function:nn  {#1}{#2} 
    \__my_map_other:nn
  }

\ExplSyntaxOff 
 
\mymainother {abcdefgh}
\end{texexample}

What just happened, we modified the custom function to accept two arguments, as well as |\_my_map_other:nn|. I also changed  their names to avoid clashes in this document.

In the next example we will iterate through two lists recursively. The first list will provide a string, which we will have to check if it consists of valid character. The valid characters are provided by the second argument of the main macro.


\begin{texexample}{Recursion}{ex:l3recursion}
\ExplSyntaxOn
\cs_new:Npn \ylcompare #1#2
  {
     \__yl_compare_auxi:nN {#2} #1 \q_recursion_tail \q_recursion_stop
  }
  
  
\cs_new:Npn \__yl_compare_auxi:nN #1#2
  {
    \quark_if_recursion_tail_stop:N #2
    \__yl_compare_auxii:nN {#1} #2
    \__yl_compare_auxi:nN {#1}
  }
 
 
\cs_new:Npn \__yl_compare_auxii:nN #1#2
  {
    \__yl_compare_auxiii:NN #2 #1 \q_recursion_tail \q_recursion_stop
  }
\cs_new:Npn \__yl_compare_auxiii:NN #1#2
  {
  % if found not found stop and print
    \quark_if_recursion_tail_stop_do:Nn #2 { \FAIL\  #1 }
  % if not the list end  
    \str_if_eq:nnT {#1} {#2}
      {
        \use_i_delimit_by_q_recursion_stop:nw { \PASS\  #1 }
      }
  % recurse     
    \__yl_compare_auxiii:NN #1
  }
\ExplSyntaxOff

\ylcompare{1234567890AAA}{-1234567890)(}
\ylcompare{text}{abcdefghijklmnopqrst} 
\end{texexample}

How would one modify the above to provide a boolean value if the string is made up only of valid characters? For example for a vowel or alphabet string. This is easy as we can define a boolean, so instead of printing the assertion we would set the boolean at false if it fails. For a number proving string, our method will fail, as we need to test for cases such as |-12345-567|, which is not a valid string also we need to think if we want to allow any spaces. This would probably have to be programmed as a special macro.

\section{Lower level functions}

As we have seen in the section for \tex iteration, one can build almost anything given patience and skills. Many examples can be found in the |expl3| package |fp|. Example~\ref{ex:fp1} is taken from the |fp| package and is a macro to trim leading zeros from a token representing a real number. All the |\@@_| are used in packages to add a prefix when processed through the doc/docstrip system, in this case it will add |fp_|.

\begin{texexample}{Weirds}{ex:fp1}
\makeatletter
\ExplSyntaxOn
 \cs_new:Npn \@@_trim_zeros:w #1 ;
  {
    \@@_trim_zeros_loop:w #1
      ; \@@_trim_zeros_loop:w 0; \@@_trim_zeros_dot:w .; \s__stop
  }
  
\cs_new:Npn \@@_trim_zeros_loop:w #1 0; #2 { #2 #1 ; #2 }

\cs_new:Npn \@@_trim_zeros_dot:w #1 .; { \@@_trim_zeros_end:w #1 ; }

\cs_new:Npn \@@_trim_zeros_end:w #1 ; #2 \s__stop { #1 }
 
 
\@@_trim_zeros:w  121200010.000; 
\ExplSyntaxOff
\end{texexample}


I have removed the |@@_| and replaced them with the |fp_| prefix to make the code more concise and readable.
The main function is delimited with a semi-colon |;| delimited function. Within the macro this is passed onto
|\fp_trim_zeros_loop:w| for further processing. 

\begin{texexample}{Weird}{ex:fp1}
\makeatletter
\ExplSyntaxOn

 \cs_new:Npn \fp_trim_zeros:w #1 ;
  {
    \fp_trim_zeros_loop:w #1;\fp_trim_zeros_loop:w 0; \fp_trim_zeros_dot:w .; \s__stop
  }
  
\cs_new:Npn \fp_trim_zeros_loop:w #1 0; #2 { #2 #1 ; #2 }

\cs_new:Npn \fp_trim_zeros_dot:w #1 .; { \fp_trim_zeros_end:w #1 ; }

\cs_new:Npn \fp_trim_zeros_end:w #1 ; #2 \s__stop { #1 }

 
\fp_trim_zeros:w  131.200010000 ; 
 \ExplSyntaxOff
\end{texexample}

This function is looking for two variables |#1 0; #2| It will scan until its end and then rescan again. The secon time it will absorb |\fp_trim_zeros_dot:w| as its second argument and then continue expanding this function.

\begin{teXXX}
\fp_trim_zeros_loop:w #1;\fp_trim_zeros_loop:w 0; {second macro}
\end{teXXX} 

Weird but wonderful functional programming.

\section{Summary}

This has brought us to almost the end of the |expl3| structures and language. There is much more to cover, but once you become proficient with the syntax and basic usage of its modules, you can pick up the rest through the documentation. 








