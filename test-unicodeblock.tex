\documentclass{article}
\usepackage{fontspec,multicol,xcolor,array,xspace,pgf,ifxetex}
\newfontfamily\meitei{NotoSansMeeteiMayek-Regular.ttf}
\newfontfamily\pan{code2000.ttf}
\newfontfamily\smallcps[Scale=0.8]{Arial Unicode MS}
\newfontfamily\brahmi{Noto Sans Brahmi}
\DeclareRobustCommand\unicodenumber[1]{{\smallcps #1\xspace}}
\def\textmeitei#1{{\meitei #1}}
\newfontfamily\aegean{Aegean.ttf}
\newfontfamily\arial{Arial Unicode MS}
\makeatletter
%    \begin{macrocode}
\def\graybox{x}
\newfontfamily\unicodenumberfam{Arial Unicode MS}
\def\textU#1{{\unicodenumberfam #1}}
\makeatletter
\def\putunicode@label#1#2;{%
\def\reformat@unicode@string##1{%
   \textU{U+}%
  \let\z\empty%
  \expandafter\@tfor\expandafter\i\expandafter:\expandafter=#2;\do{%
  \if\i;%
    \textU{x}%
  \else%
    \textU{\z}%
  \fi%
  \edef\z{\i}%
 }%
}%
  \makebox[5em]{\reformat@unicode@string{#2}\hfill}%
}

% \end{macro}
% 1#  text file #2 font command
%    \begin{macrocode}
\def\putchar@cx#1{%
   \iffontchar\font\n
     \char\the\n$_{\pgfmathparse{Hex(\the\r@cx)}\textsf{\pgfmathresult}}$%
      %
   \else
     \graybox
   \fi
\let\active@prefix\oldactive@prefix
 }

\def\urow@cx#1{%
    \n=#1% 
    \r@cx=0%
    \expandafter\putunicode@label#1;%
    \loop%
        \ifnum\n<\numexpr#1+16\relax%
        \makebox[2.1em]{\expandafter\putchar@cx{#1}}%
        \advance\r@cx by1%  
        \ifnum\r@cx>16\r@cx=1\relax\else\fi
        \advance\n by1%
    \repeat
    \par
}

\def\typeseturows@cx#1{%
\@for\next:=#1\do{%
  \urow@cx\next\vskip3pt}%
}

\newcount\r@cx%
\newcount\n%
\newcommand\unicodetable[2]{%
\bgroup
  \par
  \leavevmode%
   \r@cx=0%
   {\hbox to 5em{\ignorespaces}}%
   \loop%
    \ifnum\r@cx<16\ignorespaces 
    \makebox[2.1em]{\pgfmathparse{Hex(\the\r@cx)}\pgfmathresult}%
    \advance\r@cx by\@ne%  
   \repeat
   \vskip3pt\par
   \@nameuse{#1}%
   \typeseturows@cx{#2}%
\egroup
}

%    \end{macrocode}

\def\putdescription#1:{%
  \parindent0pt 
  \begin{minipage}[t]{4cm}
  \bgroup\aegean
  \hangindent20pt
  #1\par
  \egroup
  \end{minipage} 
}
% \end{macro}

\long\def\parsefields #1:#2\@@{%
    \ifx\par#1
    \else 
        {\small\aegean U+#1}%
         %\iffontchar\font"#1 %
          \makebox[2.1em]{\color{blue}\symbol{"#1}}% 
          \expandafter\putdescription#2\vskip3pt
        %\else
          %{\aegean \makebox[2.1em]{} Unallocated\par}%
        %\fi
    \fi  
  }%


\newread\unicodefile

% begin{macro}{\printunicodeblock}\marg{}\oarg{} The macro
% prints a unicode table from a file of definitions. This is
% printed in a two column environment by default. 
% #1 number of columns (optional) 
% #1 filename and path
% #2 font command
\DeclareDocumentCommand{\printunicodeblock}{O{2} m m }{%
  \bgroup
  \leavevmode\parindent0pt\par
  \begin{multicols}{#1}%
  #3
  \openin\unicodefile=#2
  \loop\unless\ifeof\unicodefile
    \read\unicodefile to\fileline %
    \expandafter\parsefields \fileline:\@@ 
  \repeat
  \end{multicols}%
  \egroup
  \closein\unicodefile}


\begin{document}
\arial


Meithei (Meitei) /ˈmeɪteɪ/,[4] also known as Manipuri /mænɨˈpʊəri/ ({\pan মৈতৈলোন্} \textmeitei{ꯃꯧꯇꯧꯂꯣꯟ} Meitei-lon or {\pan মৈতৈলোল্} \textmeitei{ꯃꯧꯇꯧꯂꯣꯜ} Meitei-lol), is the predominant language and lingua franca in the southeastern Himalayan state of Manipur, in northeastern India. It is the official language in government offices. Meithei is also spoken in the Indian states of Assam and Tripura, and in Bangladesh and Burma (now Myanmar).

Meithei is a Tibeto-Burman language whose exact classification remains unclear, though it shows lexical resemblances to Kuki and Tangkhul Naga.[5] The language is spoken by more than 1.5 million people.

Meithei has proven to be an integrating factor among all ethnic groups in Manipur who use it to communicate among themselves. It has been recognized (as Manipuri), by the Indian Union and has been included in the list of scheduled languages (included in the 8th schedule by the 71st amendment of the constitution in 1992). Meithei is taught as a subject up to the post-graduate level (Ph.D.) in universities of India, apart from being a medium of instruction up to the undergraduate level in Manipur.

\bgroup
\meitei
\begin{tabular}{>{\arial}l
                >{\arial}l
                >{\meitei}l
                >{\arial}l
                >{\arial}l
                >{\meitei}l
               }
1	&ama 	 &ꯑꯃ	       &11	&taramathoi	&\\
2	&ani	   &ꯑꯅꯤ	&12	 &taranithoi	&{\arial ky} \\
3	&ahum	&ꯑꯍꯨꯝ	   &13	 &tarahumdoi	&ꯇꯔꯥꯍꯨꯝꯗꯣꯢ\\
4	&mari	&ꯃꯔꯤ	   &14  &	taramari	&ꯇꯔꯥꯃꯔꯤ\\
5	&manga	 &ꯃꯉꯥ	   &15	 &taramanga	&ꯇꯔꯥꯃꯉꯥ\\
6	&taruk	 &ꯇꯔꯨꯛ	   &16	 &tarataruk	&ꯇꯔꯥꯇꯔꯨꯛ\\
7	&taret	 &ꯇꯔꯦꯠ	   &17	 &tarataret	&ꯇꯔꯥꯇꯔꯦꯠ\\
8	&nipan &ꯅꯤꯄꯥꯟ	&18	 &taranipan	&ꯇꯔꯥꯅꯤꯄꯥꯟ\\
9	&mapan	 &ꯃꯥꯄꯟ	   &19	 &taramapan	&ꯇꯔꯥꯃꯥꯄꯟ\\
10	&tara	 &ꯇꯔꯥ	   &20	 &kun	&ꯀꯨꯟ\\
\end{tabular}
\egroup

Meitei Mayek script was added to the Unicode Standard in October, 2009 with the release of version 5.2.

The Unicode block for Meitei Mayek, called Meetei Mayek, is \unicodenumber{U+ABC0–U+ABFF}.

Characters for historical orthographies are part of the Meetei Mayek Extensions block at \unicodenumber{U+AAE0–U+AAFF}.

%\begin{scriptexample}[]{Meitei}
%\unicodetable{meitei}{"ABC0,"ABCD0,"ABE0,"ABF0}
%\end{scriptexample}
%
%\begin{scriptexample}[]{Meitei}
%\unicodetable{meitei}{"AAE0,"AAF0}
%%\captionof{table}{Meetei Mayek Extensions}
%\end{scriptexample}

%\begin{multicols}{2}
\printunicodeblock{./languages/meetei-mayek.txt}{\meitei}
%\end{multicols}

\section{Brahmi}
\label{s:brahmi}
Brāhmī is the modern name given to one of the oldest writing systems used in the Indian subcontinent and in Central Asia during the final centuries BCE and the early centuries CE. Like its contemporary, Kharoṣṭhī, which was used in what is now Afghanistan and Western Pakistan, Brahmi (native to north and central India) was an \emph{abugida}.

The best-known Brahmi inscriptions are the rock-cut edicts of Ashoka in north-central India, dated to 250–232 BCE. The script was deciphered in 1837 by James Prinsep, an archaeologist, philologist, and official of the East India Company.[1] The origin of the script is still much debated, with current Western academic opinion generally agreeing (with some exceptions) that Brahmi was derived from or at least influenced by one or more contemporary Semitic scripts, but a current of opinion in India favors the idea that it is connected to the much older and as-yet undeciphered Indus script



Brahmi is a Unicode block containing characters written in India from the 3rd century BCE through the first millennium CE. It is the predecessor to all modern Indic scripts.

%\begin{scriptexample}[]{Brahmi}
\unicodetable{brahmi}{"11000,"11010,"11020,"11030,"11040,"11050,"11060,"11070}
%\end{scriptexample}


\printunicodeblock[2]{./languages/brahmi.txt}{\brahmi}

\section{Sumero Akkadian Cuneiform}
\label{s:sumero}
\newfontfamily\sumero{NotoSansSumeroAkkadianCuneiform-Regular.ttf}
In Unicode, the Sumero-Akkadian Cuneiform script is covered in two blocks:
U+12000–U+1237F Cuneiform
U+12400–U+1247F Cuneiform Numbers and Punctuation
These blocks, in version 6.0, are in the Supplementary Multilingual Plane (SMP).

The sample glyphs in the chart file published by the Unicode Consortium[2] show the characters in their Classical Sumerian form (Early Dynastic period, mid 3rd millennium BCE). The characters as written during the 2nd and 1st millennia BCE, the era during which the vast majority of cuneiform texts were written, are considered font variants of the same characters.

The character set as published in version 5.2 has been criticized, mostly because of its treatment of a number of common characters as ligatures, omitting them from the encoding standard.

\begin{scriptexample}[]{Sumero Akkadian}
\unicodetable{sumero}{"12000,"12010,"12020,"12030,"12040,"12050,"12060,"12070,
"12080,"12090,"12400,"12410,"12420,"12430}
\end{scriptexample}

\begin{table}[b]
\begin{scriptexample}[]{textbox}
From Plato's dialogue Phaedrus 14, 274c-275b:

Socrates: [274c] I heard, then, that   in Egypt, was one of the ancient gods of that country, the one whose sacred bird is called the ibis, and the name of the god himself was Theuth. He it was who [274d] invented numbers and arithmetic and geometry and astronomy, also draughts and dice, and, most important of all, letters. 

Now the king of all Egypt at that time was the god Thamus, who lived in the great city of the upper region, which the Greeks call the Egyptian Thebes, and they call the god himself Ammon. To him came Theuth to show his inventions, saying that they ought to be imparted to the other Egyptians. But Thamus asked what use there was in each, and as Theuth enumerated their uses, expressed praise or blame, according as he approved [274e] or disapproved.  

"The story goes that Thamus said many things to Theuth in praise or blame of the various arts, which it would take too long to repeat; but when they came to the letters, [274e] “This invention, O king,” said Theuth, “will make the Egyptians wiser and will improve their memories; for it is an elixir of memory and wisdom that I have discovered.” But Thamus replied, “Most ingenious Theuth, one man has the ability to beget arts, but the ability to judge of their usefulness or harmfulness to their users belongs to another; [275a] and now you, who are the father of letters, have been led by your affection to ascribe to them a power the opposite of that which they really possess.  

"For this invention will produce forgetfulness in the minds of those who learn to use it, because they will not practice their memory. Their trust in writing, produced by external characters which are no part of themselves, will discourage the use of their own memory within them. You have invented an elixir not of memory, but of reminding; and you offer your pupils the appearance of wisdom, not true wisdom, for they will read many things without instruction and will therefore seem [275b] to know many things, when they are for the most part ignorant and hard to get along with, since they are not wise, but only appear wise." 
\end{scriptexample}
\end{table}


\printunicodeblock{./languages/cuneiform.txt}{\sumero}





\newfontfamily\samaritan[RawFeature=samr]{NotoSansSamaritan-Regular.ttf}


\printunicodeblock{./languages/samaritan.txt}{\samaritan}

























\end{document}