\chapter{Image Manager}

\section{Introduction}

The scope of this Project is to develop part of an application that provides a service to store,
manipulate and serve images.

The main components of the application are the following:

\begin{enumerate}
\item  Upload images from a local disk or an ftp site raw images to an AWS S3 buckets.
       A file watcher will keep the local site and the remote buckets synchronized. If a new
       image is added to the folder this hould automatically be uploaded to S3.

\item User upload, metadata editing, cropping, publishing of optimized assets. 

\item Admins, authors and editors should be able to edit and add metada about the
      images which should be stored in an AWS Dynamo database. 

\item  A media API should eb provided in the application 

\end{enumerate}

The overall system follows the principles of microservice architecture
\href{http://martinfowler.com/articles/microservices.html}{microservices}. The server side
components will communicate asynchronously via SNS/SQS as much as possible yo improve
the overall resilience to failures.

\section{Main User Interface}

The main user interface should be developed as an AngularJS application, written in ES (aka ES6).

\section{Git Workflow}

We expect the project to follow good git practices and use open source software as far as possible.
A service such as the one we are describing has beendeveloped by the Guradian Team and is called 
grid.

\section{Documentation}

\section{Deployment}

The Developer is expected to write the necessary intallation and automation scripts for ease of deployment and to make the installation available on a cloud development environment.


\section{Image Metadata}

\begin{tabular}{ll}
id          & UUID          \\
name        & image name    \\
caption     & image caption \\
uploaded\_by & user name     \\
size        &               \\
credit      & free field might include links  \\
categories  & list of tags \\
type        & img type     \\  
\end{tabular}

The name is unique when combined with the bucket url i.e.,

\texttt{imageserver/imgservice/germany/hamburg/img1.jpg}

\section{Front end}

The front end should be minimally styled with bootstrap css, extended as necessary.






















