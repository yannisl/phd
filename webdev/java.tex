\chapter{Java}

Java technology is both a programming language and a platform. When you download and install
it you can read the claim by Oracle that it powers over 1 billion devices. This is a huge
claim, and although the language has been despised by some it has some enormous strengths
that cannot leave it out of the tookit of a professional programmer. 

\section{The Java Platform}

What's completely new is the manner in which Java technology and its runtime environment have combined them to produce a 
flexible and powerful programming system.

Developing your applications using the Java programming language results in software that is portable across multiple machine architectures, operating systems, and graphical user interfaces, secure, and high performance. With Java technology, your job as a software developer is much easier--you focus your full attention on the end goal of shipping innovative products on time, based on the solid foundation of the Java platform. The better way to develop software is here, now, brought to you by the Java platform. James Gosling and Henry McGilton
wrote a paper in 1996 describing the Java Language Environment.

\section{Too much informations and buzzwords}

If you visit the Oracle site and follow some of the tutorials and links, the language is full
of technology buzzwords.

\begin{minted}[fontsize=\footnotesize]{java}
public class HelloWorld
 {
     public static void main( String[] args )
     {
         System.out.println( "I am determined to learn how to code." );
         System.out.println( "Hello World" );
     }
 }
\end{minted}
\newmintinline{bash}{fontsize=\footnotesize,fontfamily=tt,bgcolor=black,formatcom=\color{white}}

To run the program is a two step process, first we compile it with \mint{bash}{javac HelloWorld.java}

and then we can run it using


\newminted{bash}{fontsize=\footnotesize,bgcolor=black,formatcom=\color{white},framesep=5pt}
\newminted{java}{fontsize=\footnotesize}
\begin{bashcode}

 java HelloWorld
 
 I am determined to learn how to code
 Hello World
\end{bashcode}


\begin{javacode}
/*
 Class to code and decode a string using the Ceasar cipher
 from rosettacode http://rosettacode.org/wiki/Caesar_cipher#Java
*/
public class Cipher {
    public static void main(String[] args) {
 
        String str = "The quick brown fox Jumped over the lazy Dog";
 
        System.out.println( Cipher.encode( str, 12 ));
        System.out.println( Cipher.decode( Cipher.encode( str, 12), 12 ));
    }
 
    public static String decode(String enc, int offset) {
        return encode(enc, 26-offset);
    }
 
    public static String encode(String enc, int offset) {
        offset = offset % 26 + 26;
        StringBuilder encoded = new StringBuilder();
        for (char i : enc.toCharArray()) {
            if (Character.isLetter(i)) {
                if (Character.isUpperCase(i)) {
                    encoded.append((char) ('A' + (i - 'A' + offset) % 26 ));
                } else {
                    encoded.append((char) ('a' + (i - 'a' + offset) % 26 ));
                }
            } else {
                encoded.append(i);
            }
        }
        return encoded.toString();
    }
}
\end{javacode}


\section{Program Structure}

Jave applications consist of collections of classes. Classes exist in packages but can also be nested inside other classes.

\subsection{Main method}
Whether it is a console or a graphical interface application the program must have an entrypoint of some sort. The entrypoint of the Java application is the main method. There can be more than one class with main method, but the main class is always defined externally (e.g. in a manifest file). The method must be static and is passed command-line arguments as an array of strings. Unlike C++ or C\# it never returns a value and must return void.

\begin{javacode}
public static void main(String[] args) {
}
\end{javacode}








