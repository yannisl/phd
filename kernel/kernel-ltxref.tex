\chapter{ltxref.dtx}
\label{ch:ltxref}

\section{Cross Referencing}

This section details the commands programmed in the \LaTeXe kernel that
deal with cross-referencing. 

\subsection{Author commands}

\begin{docCommand}{label}{}
The user writes  |\label|\marg{foo}  to define a reference by a name.
the name can be any character, where  foo  can be any string of characters not
             containing  `|\|', `|{|' or `|}|'.
  cross-references:\label{authorcommands}
\end{docCommand}

\begin{docCommand}{ref}{}
\begin{docCommand}{pageref}{}
   |\ref|\marg{foo}: value of most recently incremented referencable
             counter. in the current environment. (Chapter, section,
             theorem and enumeration counters counters are
             referencable, footnote counters are not.)

   |\pageref|\marg{foo}: page number at which |\label{foo}|  command
             appeared.  

  Note: The scope of the |\label| command is delimited by environments,
  so\\

\begin{verbatim}
  |\begin{theorem} \label{foo} ... \end{theorem} \label{bar}|
\end{verbatim}

  defines |\ref{foo}| to be the theorem number and |\ref{bar}| to be
  the current section number.

  Note: |\label| does the right thing in terms of spacing -- i.e.,
  leaving a space on both sides of it is equivalent to leaving
  a space on either side. \autoref{test}

\end{docCommand}
\end{docCommand}

The author commands referenced in \autoref{authorcommands}

Most of this section is just a historical note, as newer documents
inevitably use \pkgname{hyperref}, which modifies the commands and the method
of generation of references extensively. If you check the |aux| file
you will probably find label commands as:

\begin{verbatim}
\newlabel{authorcommands}{{1}{5}{Cross Referencing}{section.2.1}{}}
\end{verbatim}

 \subsection{Cross Referencing}

    \begin{teX}
\message{x-ref,}
    \end{teX}

  This is implemented as follows.  A referencable counter  CNT  is
  incremented by the command  \cs{refstepcounter}\marg{cnt} , which sets
  \cs{@currentlabel} == \marg{CNT}\marg{eval(\cs{p@cnt}\cs{theCNT})}.   The command
  \cs{label}\marg{FOO} then writes the following on file \cs{@auxout} :

\begin{verbatim}
        \newlabel{FOO}{{eval(\@currentlabel)}{eval(\thepage)}}}
\end{verbatim}

\LinesNumbered
\begin{algorithm}
\caption{The referencing algorithm}
  \cs{ref}\marg{foo} ==
    \Begin{
      \eIf{\cs{r@foo} undefined}{
         @refundefined := G T\\
              ??\\
              Warning: 'reference foo on page ... undefined'}{
        \cs{@car} \cs{eval}(\cs{r@foo})\cs{@nil}
      }
    }

  \cs{pageref}\marg{foo} =
    \Begin{
      \eIf{\cs{r@foo} undefined}{
          @refundefined := G T\\
              ??\\
              Warning: 'reference foo on page ... undefined'}{
          \cs{@cdr} \cs{eval}(\cs{r@foo})\cs{@nil}
      }
    }
\end{algorithm}


%% Check this one

  \begin{docCommand}{@refundefined}{}
    This does not save on name-space (since \cs{G@refundefinedfalse}
    was never needed) but it does make the implementation of such
    one-way switches more consistent. The extra macro to make the
    change is used since this change appears several times.
  \end{docCommand}
    \textbf{Note} despite its name, |\G@refundefinedtrue| does
    \emph{not} correspond to an |\if| command, and there is no
    matching \ldots|false|. It would be more natural to call the
    command |\G@refundefined| (as inspection of the change log will
    reveal) but unfortunately such a change would break any package
    that had defined a  |\ref|-like command that mimicked the
    definition of |\ref|, calling |\G@refundefinedtrue|. Inspection
    of the \TeX\ archives revealed several such packages, and so this
    command has been named \ldots|true| so that the definition of
    |\ref| need not be changed, and the packages will work without
    change.\label{test}.
    \begin{teX}

 \newif\ifG@refundefined
 \def\G@refundefinedtrue{\global\let\ifG@refundefined\iftrue}
 \def\G@refundefinedfalse{\global\let\ifG@refundefined\iffalse}
\def\G@refundefinedtrue{%
  \gdef\@refundefined{%
    \@latex@warning@no@line{There were undefined references}}}
\let\@refundefined\relax
    \end{teX}



  \begin{docCommand}{ref}{}
  \begin{docCommand}{@setref}{}
               
 made |\ref| and |\pageref| robust
 
 Added setting of refundefined switch.
    \begin{teX}
\def\@setref#1#2#3{%
  \ifx#1\relax
   \protect\G@refundefinedtrue
   \nfss@text{\reset@font\bfseries ??}%
   \@latex@warning{Reference `#3' on page \thepage \space
             undefined}%
  \else
   \expandafter#2#1\null
  \fi}
  
\def\ref#1{\expandafter\@setref\csname r@#1\endcsname\@firstoftwo{#1}}
\def\pageref#1{\expandafter\@setref\csname r@#1\endcsname
                                   \@secondoftwo{#1}}
    \end{teX}
  \end{docCommand}
  \end{docCommand}



  \begin{docCommand}{newlabel}{}
  
   {Remove \cs{@onlypreamble} so still defined in new \cs{enddocument}}
    This command will be written to the \texttt{.aux} file to
    pass label information from one run to another.
    
  \begin{docCommand}{@newl@bel}{}
    The internal form of |\newlabel| and |\bibcite|. Note that this
    macro does it's work inside a group. That way the local
    assignments it needs to do don't clutter the save stack. This
    prevents large documents with many labels to run out of save
    stack.

    \begin{teX}
\def\@newl@bel#1#2#3{{%
  \@ifundefined{#1@#2}%
    \relax
    {\gdef \@multiplelabels {%
       \@latex@warning@no@line{There were multiply-defined labels}}%
     \@latex@warning@no@line{Label `#2' multiply defined}}%
  \global\@namedef{#1@#2}{#3}}}
    \end{teX}
  
    \begin{teX}
\def\newlabel{\@newl@bel r}
    \end{teX}
  
    \begin{teX}
\@onlypreamble\@newl@bel
    \end{teX}
  \end{docCommand}
  \end{docCommand}


  \begin{docCommand}{if@multiplelabels}{}
  \begin{docCommand}{@multiplelabels}{}
 \changes{v1.1h}{1995/10/24}{Switch for multiplelabels removed}
    This is redefined to produce a warning if at least one label is
    defined more than once. It is executed by the |\enddocument|
    command.
    \begin{teX}
\let \@multiplelabels \relax
    \end{teX}
  \end{docCommand}
  \end{docCommand}

  \begin{docCommand}{label}{}
  \begin{docCommand}{refstepcounter}{}
     The commands |\label| and |\refstepcounter| have been changed to
    allow |\protect|'ed commands to work properly.  For example,
\begin{verbatim}
   \def\thechapter{\protect\foo{\arabic{chapter}.\roman{section}}}
\end{verbatim}
    will cause a |\label{bar}| command to define |\ref{bar}| to expand
    to something like |\foo{4.d}|.  

    \begin{teX}
\def\label#1{\@bsphack
  \protected@write\@auxout{}%
         {\string\newlabel{#1}{{\@currentlabel}{\thepage}}}%
  \@esphack}
    \end{teX}

    \begin{teX}
\def\refstepcounter#1{\stepcounter{#1}%
    \protected@edef\@currentlabel
       {\csname p@#1\endcsname\csname the#1\endcsname}%
}
    \end{teX}
  \end{docCommand}
  \end{docCommand}


  \begin{docCommand}{@currentlabel}{}
 For |\label| commands that come before any environment

    \begin{teX}
\def\@currentlabel{} 
    \end{teX}
  \end{docCommand}


 \subsection{An extension of counter referencing}


 At the moment a reference to a counter |foo| will generate the
 equivalent of |\p@foo\thefoo| although not quite in this form.  For
 some applications it would be nice of one could have |\thefoo| being
 an argument to |\p@foo| to be able to put material before and after
 the number generated by |\thefoo|. This can be easily achieved with
 a small change to one of the kernel commands as follows:

\begin{verbatim}
\def\refstepcounter#1{\stepcounter{#1}%
    \protected@edef\@currentlabel
       {\csname p@#1\expandafter\endcsname\csname the#1\endcsname}%
}
\end{verbatim}

 The trick is to ensure that |\csname the#1\endcsname| is turned into
 a single token before |\p@...| is expanded further. This way, if the
 |\p@...| command is a macro with one argument it will receive
 |\the...|. With the kernel code (i.e., without the |\expandafter|)
 it will instead pick up |\csname| which would be disastrous.

 Using |\expandafter| instead of braces delimiting the argument is
 better because, assuming that the |\p@...| command is not defined as
 a macro with one argument, the braces will stay and prohibit kerning
 that might otherwise happen between the glyphs generated by
 |\the...| and surrounding glyphs.

 We have refrained from making this change in the kernel code
 although for exisiting documents it would be 100\% backward
 compatible. The reason being that any class or package making use of
 this functionality would then horribly fail with older \LaTeX{}
 installations.

 Instead we suggest that people who are interested in using this
 functionality in a document class or package add the redefinition to
 the class file. To ensure that this redefinition is properly applied
 they might want to test for the original definition first, e.g.

\begin{verbatim}
\CheckCommand*\refstepcounter[1]{\stepcounter{#1}%
    \protected@edef\@currentlabel
       {\csname p@#1\endcsname\csname the#1\endcsname}%
}
\renewcommand*\refstepcounter[1]{\stepcounter{#1}%
    \protected@edef\@currentlabel
       {\csname p@#1\expandafter\endcsname\csname the#1\endcsname}%
}
\end{verbatim}

\begin{figure}
\centering
\begin{minipage}{\textwidth}
\mbox{}\hrulefill\mbox{}
\panunicode
\begin{quotation}
    `There's glory for you!' 

    `I don't know what you mean by ``glory''\,', Alice said. 

    Humpty Dumpty smiled contemptuously. `Of course you don't --- till I tell
you. I meant ``there's a nice knock-down argument for you''!' 

    `But ``glory'' doesn't mean ``a nice knock-down argument''\,', Alice
objected. 

    `When \emph{I} use a word', Humpty Dumpty said, in a rather scornful
tone, `it means just what I choose it to mean --- neither more nor less'.
\end{quotation}
\mbox{}\hrulefill\mbox{}
\begin{quotation}
     \enquote{There's glory for you!} 

    \enquote{I don't know what you mean by `glory,'\,} Alice said. 

    Humpty Dumpty smiled contemptuously. ``Of course you don't --- till I tell
you. I meant `there's a nice knock-down argument for you!'\,'' 

    ``But `glory' doesn't mean `a nice knock-down argument,'\,'' Alice
objected. 

    ``When \emph{I} use a word,'' Humpty Dumpty said, in a rather scornful
tone, ``it means just what I choose it to mean --- neither more nor less.''
\end{quotation}
\mbox{}\hrulefill\mbox{}
\end{minipage}
\caption{Quotation marks: top English, bottom American}\label{fig:qmarks}
\end{figure}



