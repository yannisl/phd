% \iffalse meta-comment
%
% Copyright 1993-2014
% The LaTeX3 Project and any individual authors listed elsewhere
% in this file.
%
% This file is part of the LaTeX base system.
% -------------------------------------------
%
% It may be distributed and/or modified under the
% conditions of the LaTeX Project Public License, either version 1.3c
% of this license or (at your option) any later version.
% The latest version of this license is in
%    http://www.latex-project.org/lppl.txt
% and version 1.3c or later is part of all distributions of LaTeX
% version 2005/12/01 or later.
%
% This file has the LPPL maintenance status "maintained".
%
% The list of all files belonging to the LaTeX base distribution is
% given in the file `manifest.txt'. See also `legal.txt' for additional
% information.
%
% The list of derived (unpacked) files belonging to the distribution
% and covered by LPPL is defined by the unpacking scripts (with
% extension .ins) which are part of the distribution.
\chapter{Miscellaneous class}
\label{kernel:ltmiscen}
%
\begin{teX}
%<*2ekernel>
\message{environments,}
\end{teX}
%
% \subsection{Environments}
%
%  |\begin{foo}| and |\end{foo}| are used to delimit environment |foo|.
%
%  |\begin{foo}| starts a group and calls |\foo| if it is defined,
%  otherwise it does nothing.
%
% |\end{foo}| checks to see that it matches the
%  corresponding |\begin| and if so, it calls |\endfoo| and does an
%  |\endgroup|.  Otherwise, |\end{foo}| does nothing.
%
%  If |\end{foo}| needs to ignore blanks after it, then |\endfoo| should
%  globally set the |@ignore| switch true with |\@ignoretrue|
%  (this will automatically be global).
%
%
%  NOTE: |\@@end| is defined to be the |\end| command of \TeX82.
%
%  |\enddocument| is the user's command for ending the manuscript file.
%
%  |\stop| is a panic button --- to end \TeX\ in the middle.
%
% \begin{oldcomments}
% \enddocument ==
%   BEGIN
%    \@checkend{document}   %% checks for unmatched \begin
%    \clearpage
%    \begingroup
%      if @filesw = true
%        then  close file @mainaux
%              if G@refundefined = true
%               then LaTeX Warning: 'There are undefined references.' fi
%              if @multiplelabels = true
%                then LaTeX Warning:
%                    'One or more label(s) multiply defined.'
%                else
%                \@setckpt {ARG1}{ARG2} == null
%                \newlabel{LABEL}{VAL} ==
%                    BEGIN
%                      \reserved@a == VAL
%                      if def(\reserved@a) = def(\r@LABEL)
%                        else @tempswa := true          fi
%                    END
%                \bibcite{LABEL}{VAL} == null
%                    BEGIN
%                      \reserved@a == VAL
%                      if def(\reserved@a) = def(\g@LABEL)
%                        else @tempswa := true          fi
%                    END
%                @tempswa := false
%                make @ a letter
%                \input \jobname.AUX
%                if @tempswa = true
%                  then LaTeX Warning: 'Label may have changed.
%                                  Rerun to get cross-references right.'
%       fi     fi     fi
%    \endgroup
%    finish up
%   END
%
%  \@writefile{EXT}{ENTRY} ==
%      if tf@EXT undefined
%        else \write\tf@EXT{ENTRY}
%      fi
% \end{oldcomments}
%
\begin{docCommand}{@currenvir}{\meta{environment name}}
    The name of the current environment.  Initialized to
    \texttt{document} to so that |\end{document}| works correctly.
\end{docCommand}    
\begin{teX}
\def\@currenvir{document}
\end{teX}

%
\begin{macro}{\if@ignore}
\begin{macro}{\@ignoretrue}
\begin{macro}{\@ignorefalse}
% \changes{v1.1e}{1996/07/26}{put \cs{global} into definition}
\begin{teX}
\def\@ignorefalse{\global\let\if@ignore\iffalse}
\def\@ignoretrue {\global\let\if@ignore\iftrue}
\@ignorefalse
\end{teX}
\end{macro}
\end{macro}
\end{macro}
%
%
\begin{macro}{\ignorespacesafterend}
\end{macro}
% \changes{v1.1e}{1996/07/26}{user level macro added}
\begin{teX}
\let\ignorespacesafterend\@ignoretrue
\end{teX}

%
\def\changes#1#2#3{\protect\footnote{#1,#2,#3.}}
\begin{docCommand}{enddocument}{}
\end{docCommand}
% \changes{LaTeX2.09}{1993/08/03}
%         {Changed redefinition of \cs{global} to redefinition
%               of \cs{@setckpt}.}
% \changes{LaTeX2.09}{1993/09/08}
%         {Added warning in case of undefined references.}%
% \changes{v0.9e}{1993/12/09}{Hook added}
    \begin{teX}
\def\enddocument{%
    \end{teX}
    The |\end{document}| hook is executed first. If necessary it can
    contain a |\clearpage| to output dangling floats first. In this
    position it can also contain something like |\end{foo}| so that
    the whole document effectively starts and ends with some special
    environment. However, this must be used with care, eg if two
    applications would use this without knowledge of each other the
    order of the environments will be wrong after all.
    |\AtEndDocument| is redefined
    at this point so that and such commands that get into the hook do
    not chase their tail\ldots
% \changes{v1.0y}{1995/04/27}{\cs{@checkend} moved after hook}
% \changes{v1.0z}{1995/07/13}{Set \cs{@setckpt} to \cs{@gobbletwo}
%                    instead of defining it by hand}
% \changes{v1.1i}{2000/05/19}
%            {Reset \cs{AtEndDocument} for latex/3060}
\begin{teX}
   \let\AtEndDocument\@firstofone
   \@enddocumenthook
   \@checkend{document}%
   \clearpage
   \begingroup
     \if@filesw
       \immediate\closeout\@mainaux
       \let\@setckpt\@gobbletwo
       \let\@newl@bel\@testdef
\end{teX}
% \changes{v1.0z}{1995/07/13}{Shorten redefinition of \cs{bibcite} and
%          \cs{newlabel}}
%    The previous line is equiv to setting
%\begin{verbatim}
%       \def\newlabel{\@testdef r}%
%       \def\bibcite{\@testdef b}%
%\end{verbatim}
% \changes{v1.1k}{2010/08/17}{Use braces around \cs{input} arg (pr/4124)}
% \changes{v1.1l}{2010/08/17}{Change of plan: use \cs{@@input} instead
%                             (pr/4124)}
    We use |\@@input| to load the \texttt{.aux} file, so that it doesn't    show up in the list of files produced by |\listfiles|.
\begin{teX}
       \@tempswafalse
       \makeatletter \@@input\jobname.aux
     \fi
\end{teX}
% \changes{v1.0w}{1994/11/30}
%         {(DPC) Use \cs{@dofilelist}}
\begin{teX}
     \@dofilelist
\end{teX}
    First we check for font size substitution bigger than
%    |\fontsubfuzz|. The |\relax| is necessary because this is a macro
    not a register.
% \changes{v1.0w}{1994/11/30}
%         {(DPC) Do warnings even for \cs{nofiles}}
\begin{teX}
     \ifdim \font@submax >\fontsubfuzz\relax
\end{teX}
    In case you wonder about the |\@gobbletwo| inside the message
    below, this is a horrible hack to remove the tokens |\on@line.|
    that are added by |\@font@warning| at the end.
% \changes{v1.1j}{2000/07/11}{Fix typo in warning}
\begin{teX}
       \@font@warning{Size substitutions with differences\MessageBreak
                  up to \font@submax\space have occurred.\@gobbletwo}%
     \fi
\end{teX}
    The macro |\@defaultsubs| is initially |\relax| but gets redefined
    to produce
    a warning if there have been some default font substitutions.
% \changes{v1.0z}{1995/07/13}{Use \cs{@defaultsubs} instead of switch}
\begin{teX}
     \@defaultsubs
\end{teX}
    The macro |\@refundefined| is initially |\relax| but gets redefined
    to produce a warning if there are undefined refs.
% \changes{v1.1b}{1995/10/24}{Use \cs{@refundefined} instead of switch}
\begin{teX}
     \@refundefined
\end{teX}
    If a label is defined more than once, |\@tempswa| will always be
    true and thus produce a ``Label(s) may \ldots'' warning. But
    since a rerun will not solve that problem (unless one uses a
    package like \texttt{varioref} that generates labels on the fly),
    we suppress this message.
% \changes{v1.0e}{1994/04/20}{Changed logic for producing
%                             warning messages}
% \changes{v1.1b}{1995/10/24}{Changed logic for producing
%                             warning messages and removed switch}
\begin{teX}
     \if@filesw
       \ifx \@multiplelabels \relax
         \if@tempswa
           \@latex@warning@no@line{Label(s) may have changed.
               Rerun to get cross-references right}%
         \fi
       \else
         \@multiplelabels
       \fi
     \fi
   \endgroup
   \deadcycles\z@\@@end}
\end{teX}

%
\begin{macro}{\@testdef}
\begin{teX}
\def\@testdef #1#2#3{%
  \def\reserved@a{#3}\expandafter \ifx \csname #1@#2\endcsname
 \reserved@a  \else \@tempswatrue \fi}
\end{teX}
\end{macro}
%
%
\begin{macro}{\@writefile}
% \changes{v1.0l}{1994/05/20}{Added correct setting of \cs{protect}.}
% \changes{v1.0t}{1994/11/04}{Removed setting of \cs{protect}. ASAJ.}
% \changes{v1.0z}{1995/07/13}{Added missing percent and use \cs{relax}
%  in the THEN case}
\begin{teX}
\long\def\@writefile#1#2{%
  \@ifundefined{tf@#1}\relax
    {\@temptokena{#2}%
     \immediate\write\csname tf@#1\endcsname{\the\@temptokena}%
    }%
}
\end{teX}
\end{macro}
%
\begin{macro}{\stop}
\begin{teX}
\def\stop{\clearpage\deadcycles\z@\let\par\@@par\@@end}
\end{teX}
\end{macro}
%
%
% \begin{oldcomments}
%
\begin{teX}
\everypar{\@nodocument} %% To get an error if text appears before the
\nullfont               %% \begin{document}
\end{teX}
%
% \begin, \end, and \@checkend changed so \end{document} will catch
% an unmatched \begin.  Changed 24 May 89 as suggested by
% Frank Mittelbach and Rainer Sch\"opf.
%
% \begin{NAME} ==
%  BEGIN
%    IF \NAME undefined  THEN  \reserved@a == BEGIN report error END
%                        ELSE  \reserved@a ==
%                                   (\@currenvir :=L NAME) \NAME
%    FI
%    @ignore :=G F      %% Added 30 Nov 88
%    \begingroup
%    \@endpe := F
%    \@currenvir :=L NAME
%    \NAME
%  END
%
% \end{NAME} ==
%  BEGIN
%   \endNAME
%   \@checkend{NAME}
%   \endgroup
%   IF @endpe = T              %% @endpe set True by \@endparenv
%     THEN \@doendpe           %% \@doendpe redefines \par and \everypar
%                              %% to suppress paragraph indentation in
%   FI                         %% immediately following text
%   IF @ignore = T
%     THEN @ignore :=G F
%          \ignorespaces
%   FI
%  END
%
% \@checkend{NAME} ==
%  BEGIN
%   IF \@currenvir = NAME
%     ELSE \@badend{NAME}
%   FI
%  END
%
% \end{oldcomments}
%
%
\begin{macro}{\begin}
% \changes{LaTeX2.09}{1992/03/18}{Changed \cs{@ignoretrue} to
%               \cs{@ignorefalse} (as documented)}
% \changes{LaTeX2.09}{1992/08/24}{Added code to \cs{begin} to
%      remember line number. Used by \cs{@badend} to display
%      position of non-matching \cs{begin}.}
% \changes{v1.1e}{1996/07/26}{remove \cs{global} before \cs{@ignore...}}
\begin{teX}
\def\begin#1{%
  \@ifundefined{#1}%
    {\def\reserved@a{\@latex@error{Environment #1 undefined}\@eha}}%
    {\def\reserved@a{\def\@currenvir{#1}%
     \edef\@currenvline{\on@line}%
     \csname #1\endcsname}}%
  \@ignorefalse
  \begingroup\@endpefalse\reserved@a}
\end{teX}
\end{macro}
%
\begin{macro}{\end}
% \changes{v1.1e}{1996/07/26}{remove \cs{global} before \cs{@ignore...}}
\begin{teX}
\def\end#1{%
  \csname end#1\endcsname\@checkend{#1}%
  \expandafter\endgroup\if@endpe\@doendpe\fi
  \if@ignore\@ignorefalse\ignorespaces\fi}
\end{teX}
\end{macro}
%
\begin{macro}{\@checkend}
\begin{teX}
\def\@checkend#1{\def\reserved@a{#1}\ifx
      \reserved@a\@currenvir \else\@badend{#1}\fi}
\end{teX}
\end{macro}
%
\begin{macro}{\@currenvline}
%    We do need a default value for |\@currenvline| on top-level since
%    the document environment cancels the brace group. This means that
%    a mismatch with |\begin|\allowbreak|{document}| will not produce
%    a line number. Thus the outer default must be |\@empty| or we
%    will end up with two spaces.
% \changes{v1.0q}{1994/05/24}{Use \cs{@empty} as outer default}
\begin{teX}
\let\@currenvline\@empty
\end{teX}
\end{macro}
%
%
% \subsection{Center, Flushright, Flushleft}
%
\begin{teX}
\message{center,}
\end{teX}
%
% \begin{oldcomments}
%
% \center, \flushright and \flushleft set
%   \rightskip = 0pt or \@flushglue (as appropriate)
%   \leftskip  = 0pt or \@flushglue (as appropriate)
%   \parindent = 0pt
%   \parfillskip   = 0pt. (except \flushleft)
%   \\         == \par \vskip -\parskip
%   \\[LENGTH] == \\ \vskip LENGTH
%   \\*        == \par \penalty 10000 \vskip -\parskip
%   \\*[LEN]   == \\* \vskip LENGTH
%
% They invoke the trivlist environment to handle vertical spacing before
% and after them.
%
% \centering, \raggedright and \raggedleft are the declaration analogs
% of the above.
%
% \raggedright has a more universal effect, however.  It sets
% \@rightskip := flushglue.  Every environment, like the list
% environments,
% that set \rightskip to its 'normal' value set it to \@rightskip
%
% \end{oldcomments}
%
\begin{macro}{\@centercr}
% \changes{v1.0h}{1994/05/03}{\cs{@badcrerr} replaced by \cs{@nolnerr}}
% \changes{v1.0z}{1995/07/13}{Use \cs{nobreak}}
\begin{teX}
\def\@centercr{\ifhmode \unskip\else \@nolnerr\fi
       \par\@ifstar{\nobreak\@xcentercr}\@xcentercr}
\end{teX}
\end{macro}
%
\begin{macro}{\@xcentercr}
\begin{teX}
\def\@xcentercr{\addvspace{-\parskip}\@ifnextchar
    [\@icentercr\ignorespaces}
\end{teX}
\end{macro}
%
\begin{macro}{\@icentercr}
\begin{teX}
\def\@icentercr[#1]{\vskip #1\ignorespaces}
\end{teX}
\end{macro}
%
%
% \begin{environment}{center}
% \changes{v0.9h}{1993/12/13}{Removed optional argument of \cs{item}}
% \changes{v1.0u}{1994/11/12}{Changed end macro to \cs{def}: safer and
% consistent}
%    We use |\relax| to prevent |\item| scanning too far.
\begin{teX}
\def\center{\trivlist \centering\item\relax}
\end{teX}
%
\begin{teX}
\def\endcenter{\endtrivlist}
\end{teX}
% \end{environment}
%
\begin{macro}{\centering}
\begin{teX}
\def\centering{%
  \let\\\@centercr
  \rightskip\@flushglue\leftskip\@flushglue
  \parindent\z@\parfillskip\z@skip}
\end{teX}
\end{macro}
%
\begin{macro}{\@rightskip}
\begin{teX}
\newskip\@rightskip \@rightskip \z@skip
\end{teX}
\end{macro}
%
% \begin{environment}{flushleft}
% \changes{v0.9h}{1993/12/13}{Removed optional argument of \cs{item}}
% \changes{v1.0u}{1994/11/12}{Changed end macro to \cs{def}: safer and
% consistent}
%    We use |\relax| to prevent |\item| scanning too far.
\begin{teX}
\def\flushleft{\trivlist \raggedright\item\relax}
\end{teX}
%
\begin{teX}
\def\endflushleft{\endtrivlist}
\end{teX}
% \end{environment}
%
\begin{macro}{\raggedright}
\begin{teX}
\def\raggedright{%
  \let\\\@centercr\@rightskip\@flushglue \rightskip\@rightskip
  \leftskip\z@skip
  \parindent\z@}
\end{teX}
\end{macro}
%
% \begin{environment}{flushright}
% \changes{v0.9h}{1993/12/13}{Removed optional argument of \cs{item}}
% \changes{v1.0u}{1994/11/12}{Changed end macro to \cs{def}: safer and
% consistent}
%    We use |\relax| to prevent |\item| scanning too far.
\begin{teX}
\def\flushright{\trivlist \raggedleft\item\relax}
\end{teX}
%
\begin{teX}
\def\endflushright{\endtrivlist}
\end{teX}
% \end{environment}
%
\begin{macro}{\raggedleft}
\begin{teX}
\def\raggedleft{%
  \let\\\@centercr
  \rightskip\z@skip\leftskip\@flushglue
  \parindent\z@\parfillskip\z@skip}
\end{teX}
\end{macro}
%
\begin{teX}
\message{verbatim,}
\end{teX}
%
 \subsection{Verbatim}


  The verbatim environment uses the fixed-width |\ttfamily| font, turns
  blanks into spaces, starts a new line for each carriage return (or
  sequence of consecutive carriage returns), and interprets
  \emph{every} character literally.
  I.e., all special characters |\, {, $|, etc.
   are |\catcode|'d to 'other'.

 The command |\verb| produces in-line verbatim text, where the argument
 is delimited by any pair of characters.  E.g., |\verb #...#| takes
  `|...|' as its argument, and sets it verbatim in |\ttfamily| font.

The *-variants of these commands are the same, except that spaces
print as the \TeX{}book's space character instead of as blank spaces.

\begin{macro}{\@vobeyspaces}
\begin{teX}
{\catcode`\ =\active%
\gdef\@vobeyspaces{\catcode`\ \active\let \@xobeysp}}
\end{teX}
\end{macro}
%
\begin{macro}{\@xobeysp}
% \changes{v1.0z}{1995/07/13}{Use \cs{nobreak}}
% \changes{v1.1f}{1996/09/28}{Moved to ltspace.dtx}
\end{macro}
%
%
\begin{macro}{\@xverbatim}
\begin{macro}{\@sxverbatim}
\begin{teX}
\begingroup \catcode `|=0 \catcode `[= 1
\catcode`]=2 \catcode `\{=12 \catcode `\}=12
\catcode`\\=12 |gdef|@xverbatim#1\end{verbatim}[#1|end[verbatim]]
|gdef|@sxverbatim#1\end{verbatim*}[#1|end[verbatim*]]
|endgroup
\end{teX}
\end{macro}
\end{macro}
%
\begin{macro}{\@verbatim}
% \changes{LaTeX2.09}{1991/07/24}{Added \cs{penalty}\cs{interlinepenalty}
%               to definition of \cs{par} so that \cs{samepage} works}
% \changes{v0.9h}{1993/12/13}{Removed optional argument of \cs{item}}
%    Real start of verbatim environment
%    We use |\relax| to prevent |\item| scanning too far.
% \changes{v0.9p}{1994/01/18}
%         {Add \cs{global}\cs{@inlabelfalse}}
% \changes{v1.0b}{1994/03/16}
%         {Remove \cs{global}\cs{@inlabelfalse} again.}
\begin{teX}
\def\@verbatim{\trivlist \item\relax
  \if@minipage\else\vskip\parskip\fi
  \leftskip\@totalleftmargin\rightskip\z@skip
  \parindent\z@\parfillskip\@flushglue\parskip\z@skip
\end{teX}
% \changes{LaTeX2.09}{1991/08/26}{\cs{@@par} added}
    Added |\@@par| to clear possible |\parshape| definition
    from a surrounding list (the verbatim guru says).
% \changes{v0.9p}{1994/01/18}
%         {Only add \cs{penalty} if in hmode}
\begin{teX}
  \@@par
  \@tempswafalse
  \def\par{%
    \if@tempswa
\end{teX}
    A |\leavevmode| added: needed if, for example, a blank verbatim
    line is the first thing in a list item (wow!).
% \changes{v1.0f}{1994/04/29}{\cs{leavevmode} added}
\begin{teX}
      \leavevmode \null \@@par\penalty\interlinepenalty
    \else
      \@tempswatrue
      \ifhmode\@@par\penalty\interlinepenalty\fi
    \fi}%
\end{teX}
    To allow customization we hide the font used in a separate macro.
  \changes{v0.9a}{1993/11/21}{use \cs{verbatim@font} instead of \cs{tt}}
  \changes{v0.9h}{1993/12/13}{Readded \cs{@noligs}}
  \changes{v1.1d}{1996/06/03}{Exchanged the following two code lines
           so that \cs{dospecials} cannot reset the category code
           of characters handled by \cs{@noligs}.}
  \changes{v1.1h}{2000/01/07}{Disable hyphenation even if the font allows it.}
\begin{teX}
  \let\do\@makeother \dospecials
  \obeylines \verbatim@font \@noligs
  \hyphenchar\font\m@ne
\end{teX}
To avoid a breakpoint after the labels box, we remove the penalty
put there by the list macros: another use of |\unpenalty|!
% \changes{v1.0f}{1994/04/29}{Change to \cs{everypar} added}
\begin{teX}
  \everypar \expandafter{\the\everypar \unpenalty}%
}
\end{teX}
\end{macro}
%
\begin{docCommand}{verbatim}{}
\begin{docCommand}{endverbatim}{}
%    (RmS 93/09/19) Protected against `missing item' error message
%               triggered by empty verbatim environment.
\begin{teX}
\def\verbatim{\@verbatim \frenchspacing\@vobeyspaces \@xverbatim}
\def\endverbatim{\if@newlist \leavevmode\fi\endtrivlist}
\end{teX}
\end{docCommand}
\end{docCommand}
%
\begin{macro}{\verbatim@font}
% \changes{v0.9a}{1993/11/21}{Macro added}
%    Macro to select the font  used for verbatim typesetting.
%    It also does other work if necessary for the font used.
% \changes{v0.9s}{1994/01/21}{Removed unnecessary category code
%                            hackery.}
\begin{teX}
\def\verbatim@font{\normalfont\ttfamily}
\end{teX}
\end{macro}
%
%
%  \begin{environment}{verbatim*}
\begin{teX}
\@namedef{verbatim*}{\@verbatim\@sxverbatim}
\expandafter\let\csname endverbatim*\endcsname =\endverbatim
\end{teX}
%  \end{environment}
%
\begin{macro}{\@makeother}
\begin{teX}
\def\@makeother#1{\catcode`#112\relax}
\end{teX}
\end{macro}
%
\begin{macro}{\verb@balance@group}
% \changes{LaTeX2.09}{1993/09/07}
%     {(RmS) Changed definition of \cs{verb} so that it detects a
%              missing second delimiter.}
\begin{teX}
\let\verb@balance@group\@empty
\end{teX}
\end{macro}
%
\begin{macro}{\verb@egroup}
\begin{teX}
\def\verb@egroup{\global\let\verb@balance@group\@empty\egroup}
\end{teX}
\end{macro}
%
\begin{macro}{\verb@eol@error}
\begin{teX}
\begingroup
  \obeylines%
  \gdef\verb@eol@error{\obeylines%
    \def^^M{\verb@egroup\@latex@error{%
            \noexpand\verb ended by end of line}\@ehc}}%
\endgroup
\end{teX}
\end{macro}
%
\begin{macro}{\verb}
% \changes{LaTeX2.09}{1992/08/24}
%         {Changed \cs{verb} and \cs{@sverb} to work correctly
%            in math mode}
% \changes{v0.9a}{1993/11/21}{Use \cs{verbatim@font} instead of
%                             \cs{tt}.}
% \changes{v1.1a}{1995/09/19}{Put \cs{@noligs} after
%                    \cs{verbatim@font} where it belongs.}
%    Typesetting a small piece verbatim.
%  \changes{v1.1d}{1996/06/03}{Put setting of verbatim font after
%           \cs{dospecials}
%           so that \cs{dospecials} cannot reset the category code
%           of characters handled by \cs{@noligs}.}
\begin{teX}
\def\verb{\relax\ifmmode\hbox\else\leavevmode\null\fi
  \bgroup
    \verb@eol@error \let\do\@makeother \dospecials
    \verbatim@font\@noligs
    \@ifstar\@sverb\@verb}
\end{teX}
\end{macro}
%
%
\begin{macro}{\@sverb}
% \changes{v1.0j}{1994/05/10}{Slight change in error message text.}
% Definitions of |\@sverb| and |\@verb| changed so |\verb+ foo+|
% does not lose leading blanks when it comes at the beginning of a line.
% Change made 24 May 89. Suggested by Frank Mittelbach and Rainer
% Sch\"opf.
%
\begin{teX}
\def\@sverb#1{%
  \catcode`#1\active
  \lccode`\~`#1%
  \gdef\verb@balance@group{\verb@egroup
     \@latex@error{\noexpand\verb illegal in command argument}\@ehc}%
  \aftergroup\verb@balance@group
  \lowercase{\let~\verb@egroup}}%
\end{teX}
\end{macro}
%
\begin{macro}{\@verb}
\begin{teX}
\def\@verb{\@vobeyspaces \frenchspacing \@sverb}
\end{teX}
\end{macro}
%
\begin{macro}{\verbatim@nolig@list}
% \changes{LaTeX2.09}{1993/09/03}
%         {Replaced \cs{@noligs} by extensible list}
\begin{teX}
\def\verbatim@nolig@list{\do\`\do\<\do\>\do\,\do\'\do\-}
\end{teX}
\end{macro}
%
\begin{macro}{\do@noligs}
\begin{teX}
\def\do@noligs#1{%
  \catcode`#1\active
  \begingroup
     \lccode`\~`#1\relax
     \lowercase{\endgroup\def~{\leavevmode\kern\z@\char`#1}}}
\end{teX}
\end{macro}
%
\begin{macro}{\@noligs}
%    To stay compatible with packages that use |\@noligs| we keep it.

\begin{teX}
\def\@noligs{\let\do\do@noligs \verbatim@nolig@list}
\end{teX}
\end{macro}

