 \chapter{LaTeXe Kernel Definitions Module}
 \normalfont
\label{ch:ltxdefinitions}

\section{Introduction}

 This section contains a number of commands used in defining other macros, as well as some useseful commands,
 that can be used by package authors. Some of these commands such as \cmd{\@height} were defined to
 save tokens and hence memory tokens and by now they do not add much value to a modern \latexe installation.
 However, removing them would break backward compatibility, but perhaps in your own packages you may choose
 to improve the program readability by not using them.
\medskip

 \begin{tabular}{ll}
 \refCom{two@digits} & prefix a number less than 10 with `0’\\
 |\typeout|   & display something on the terminal\\
 |\newlinechar| & newline character\\
 |\@height| & height\\
 |\@width|  & width\\
 |\@depth|  &depth\\
 \end{tabular}

  \medskip  

\begin{docCommand}{two@digits}{\marg{number}}
Prefix a number less than 10 with `0'.
\end{docCommand}

\begin{teX}
\def\two@digits#1{\ifnum#1<10 0\fi\number#1}
\end{teX}



\begin{docCommand}{typeout} {\meta{message}} 
    Display something on the terminal.
\end{docCommand}
   
\begin{teX}
\def\typeout#1{\begingroup\set@display@protect
    \immediate\write\@unused{#1}\endgroup}
\end{teX}
 
\begin{docCommand} {newlinechar}{ }
    A char to be used as new-line in output to files.
\end{docCommand}

\begin{teX}
\newlinechar`\^^J
\end{teX}
 \section{Saved versions of \TeX{} primitives}


\begin{docCommand}{@@par}{}
The TeX primitive |\foo| is saved as |\@@foo|.
The following primitives are handled in this way:
\end{docCommand}

\begin{teX}
\let\@@par=\par
\let\@@input=\input    %%% moved earlier
\let\@@end=\end        %%%
\end{teX}

\begin{docCommand}{@@hyph}{ }
\begin{docCommand}{-}{ }
  The following comment was added when these commands were first set
  up, 19 April 1986:
  the |\-| command is redefined to allow it to work in the |\ttfamily|
  type style, where automatic hyphenation is suppressed by setting
  |\hyphenchar| to~$-1$. The original primitive \TeX{} definition is
  saved as |\@@hyph| just in case anyone needs it.

  There is a need for a robust command for a discretionary hyphen
  since its exact representation depends on the glyphs available in
  the current font.  For example, with suitable fonts and the
  \texttt{T1} font encoding it is possible to use hanging hyphens.

  A suitable robust definition that allows for many possible types of
  font and encoding may be as follows:
  \begin{verbatim}
  \DeclareRobustCommand {\-}{%
    \discretionary {%
      \char \ifnum\hyphenchar\font<\z@
              \defaulthyphenchar
            \else
              \hyphenchar\font
            \fi
                    }{}{}%
  }
  \end{verbatim}

  The redefinition (via |\let|) of |\-| within tabbing also makes the
  use of a robust command advisable since then any redefinition
  of |\-| via |\DeclareRobustCommand| will not cause a conflict.

  Therefore, macro writers should be hereby warned that
  these internals will probably change! It is likely that a future
  release of \LaTeX{} will make |\-| effectively an encoding specific
  text command.

\begin{teX}
\let\@@hyph=\-        % Save original primitive definition
\def\-{\discretionary{-}{}{}}
\end{teX}
\end{docCommand}
\end{docCommand}
%
\begin{docCommand}{@dischyph}{ }

%         {Define \cs{@dischyph}, was previously in ltboxes.dtx}
\begin{teX}
\let\@dischyph=\-
\end{teX}
\end{docCommand}
%
  \begin{docCommand}{@@italiccorr}{}
 Save the original italic correction.
  \end{docCommand}
%
\begin{teX}
\let\@@italiccorr=\/
\end{teX}
 

%\endinput
\begin{docCommand}{@height} { }
\end{docCommand}

\begin{docCommand}{@depth} { }
\end{docCommand}

\begin{docCommand}{@width} { }
\end{docCommand}

\begin{docCommand}{@minus} { }
\end{docCommand}

\begin{docCommand}{@plus} { }
\end{docCommand}



The following definitions save token space  e.g., using
|\@height| instead of height saves 5 tokens at the cost in time
of one macro expansion.
    
\begin{teX}
\def\@height{height} \def\@depth{depth} \def\@width{width}
\def\@minus{minus}
\def\@plus{plus}
\end{teX}
\begin{docCommand}{hb@xt@} { }
The next one is another 100 tokens worth.
\end{docCommand}
\begin{teX}
\def\hb@xt@{\hbox to}
\end{teX}

\begin{teX}
\message{hacks,}
\end{teX}

 \subsection{Command definitions}



 This section defines the following commands:

 \DescribeMacro
  {\@namedef}\marg{NAME}\\ Expands to |\def\|\marg{NAME},
   except name can contain any characters.

 \DescribeMacro
  {\@nameuse}\marg{NAME}\\
   Expands to |\|\marg{NAME}.

 \DescribeMacro
  {\@ifnextchar} X\marg{YES}\marg{NO}\\
    Expands to \meta{YES} if next character is an `X',
          and to \meta{NO} otherwise.
          (Uses |\reserved@a|--|\reserved@c|.)
          NOTE: GOBBLES ANY SPACE FOLLOWING IT.

 \DescribeMacro
  {\@ifstar}\marg{YES}\marg{NO}\\
          Gobbles following spaces and then tests if next the
          character is a '*'.  If it is, then it gobbles the
   `*' and expands to \meta{YES}, otherwise it expands to \meta{NO}.

 \DescribeMacro
  {\@dblarg}\marg{CMD}\marg{ARG}\\
     Expands to |\|\marg{CMD}\oarg{ARG}\marg{ARG}.  Use
          |\@dblarg\CS| when |\CS| takes arguments |[ARG1]{ARG2}|,
          where default is| ARG1| = |ARG2|.

 \DescribeMacro
  {\@ifundefined}\marg{NAME}\marg{YES}\marg{NO}\\
          : If \cs{NAME} is undefined then it executes \meta{YES},
            otherwise it executes \meta{NO}.  More precisely,
            true if \cs{NAME} either undefined or = |\relax|.

 \DescribeMacro
  {\@ifdefinable}|\NAME|\marg{YES}
       Executes \meta{YES} if the user is allowed to define |\NAME|,
            otherwise it gives an error.  The user can define |\NAME|
            if |\@ifundefined{NAME}| is true, '|NAME|' $\neq$ '|relax|'
            and the first three letters of '|NAME|' are not
           '|end|', and if |\endNAME| is not defined.

 \DescribeMacro
  \newcommand|*|\marg{\cs{FOO}}\oarg{i}\marg{TEXT}\\
         User command to define |\FOO| to be a macro with
            i arguments (i = 0 if missing) having the definition
            \meta{TEXT}.  Produces an error if |\FOO| already
            defined.

            Normally the command is defined to be |\long| (ie it may
            take multiple paragraphs in its argument). In the
            star-form, the command is not defined as |\long| and a
            blank line in any argument to the command would generate
            an error.

 \DescribeMacro
  \renewcommand|*|\marg{\cs{FOO}}\oarg{i}\marg{TEXT}\\
  Same as |\newcommand|, except it checks if |\FOO| already defined.

 \DescribeMacro
  \newenvironment|*|\marg{FOO}\oarg{i}\marg{DEF1}\marg{DEF2}\\
         equivalent to:\\
         |\newcommand{\FOO}[i]{DEF1}| |\def{\endFOO}{DEF2}|\\
 (or the appropriate star forms).

 \DescribeMacro
  \renewenvironment\\ Obvious companion to |\newenvironment|.


 \DescribeMacro
  \@cons : See description of |\output| routine.

 \DescribeMacro{\@car}
  |\@car T1 T2 ... Tn\@nil| == |T1|  (unexpanded)

 \DescribeMacro{\@cdr}
  |\@cdr T1 T2 ... Tn\@ni|l == |T2 ... Tn|     (unexpanded)

 \DescribeMacro
  \typeout\marg{message}\\ Produces a warning message on the terminal.

 \DescribeMacro
  \typein\marg{message}\\
        Types message, asks the user to type in a command, then
            executes it

 \DescribeMacro
  \typein\oarg{\cs{CS}}\marg{MSG}\\
  Same as above, except defines |\CS| to be the input
                      instead of executing it.

 \changes{LaTeX209}{1992/03/18}
  {(RMS) changed input channel from 0 to \cs{@inputcheck} to avoid
     conflicts with other channels allocated by \cs{newread}}

\begin{docCommand}{typein}{}

\begin{teX}
\def\typein{%
  \let\@typein\relax
  \@testopt\@xtypein\@typein}
\end{teX}

\begin{teX}
\def\@xtypein[#1]#2{%
  \typeout{#2}%
  \advance\endlinechar\@M
  \read\@inputcheck to#1%
  \advance\endlinechar-\@M
  \@typein}
\end{teX}
\end{docCommand}

\subsection{csname abstractions }

\begin{docCommand}{@namedef}{}
The \CMDI{\@namedef} and \CMDI{\@nameuse} are used to define or use \cs{csname}. You need to watch
that the arguments are \emph{names} i.e, they are not preceded by a backslash or other active character.
(See example~\ref{ex:namedef}).

\end{docCommand}
\begin{docCommand}{@nameuse}{}
\end{docCommand}

\begin{teX}
\def\@namedef#1{\expandafter\def\csname #1\endcsname}
\def\@nameuse#1{\csname #1\endcsname}
\end{teX}


\begin{texexample}{Using \string\@namedef}{ex:namedef}
\makeatletter
\@namedef{test1}{This is the first test}
\@nameuse{test1}
\makeatother
\end{texexample}

\section{List helper macros}

\begin{docCommand}{@cons} {}
The command \cmd{\@cons} is used for the construction of lists. Its unique
characteristic being that it constsructs \emph{elt} lists. What this means is that each entry in the list is
preceded by another as yet undefined command |\@elt|. This can be later used to grub the element as
its argument and do something about it.

\begin{teX}
\def\@cons#1#2{\begingroup\let\@elt\relax\xdef#1{#1\@elt #2}\endgroup}
\end{teX}
\end{docCommand}

Assume we have a list \cmd{\mylist} which has been defined as |\mylist{\@elt one \@elt two ...}|.
We can use |\@cons| to append more elements as shown in the next example,

\begin{texexample}{Usage of cons}{ex:cons}
\makeatletter
\def\mylist{}
\@cons\mylist{one}
\@cons\mylist{two}
\def\@elt{\space}
\mylist
\makeatother
\end{texexample}


\begin{docCommand}{@car}{}
\end{docCommand}
\begin{docCommand}{@cdr}{}
The next two macros \cmd{\@car} and \cmd{\@cdr} capture either the first element of a list or the rest of the elements
except the first. They are both delimited by |\@nil|.
\end{docCommand}

\begin{teX}
\def\@car#1#2\@nil{#1}
\def\@cdr#1#2\@nil{#2}
\end{teX}



\begin{docCommand}{@carcube}{}
Given a list $T_1\dots T_n$ \cmd{\@carcube} will grub $T_1 T_2 T_3$ for |T1| |T2| |T3| , $n > 3$. If you are familiar with
lisp |car|, |cdr| and |cons| are fundamental functions. The |cons|  function is used to construct lists, and the |car|
 and |cdr| functions are used to take them apart.\footnote{The name of the cons function is not unreasonable: it is an abbreviation of the word `construct'. The origins of the names for |car| and |cdr|, on the other hand, are esoteric: |car| is an acronym from 
 the phrase `Contents of the Address part of the Register'; and cdr (pronounced `could-er') is an acronym from the phrase 
 `Contents of the Decrement part of the Register'. These phrases refer to specific pieces of hardware on 
 the very early computer on which the original Lisp was developed. Besides being obsolete, the phrases have been completely irrelevant for more than 35 years to anyone thinking about Lisp. Nonetheless, although a few brave scholars have begun to 
 use more reasonable names for these functions, the old terms are still in use. In particular, since the terms 
 are still used in the Emacs Lisp source code.}
\end{docCommand}
\begin{teX}
\def\@carcube#1#2#3#4\@nil{#1#2#3}
\end{teX}



\begin{docCommand}{@onlypreamble}{}
    This macro adds its argument to the list of commands stored in
    |\@preamblecmds| to be
    disabled after |\begin{document}|. These commands are redefined
    to generate |\@notprerr| at this point
\end{docCommand}
\begin{docCommand}{@preamblecmds}{}
\end{docCommand}

.
    
\begin{teX}
\def\@preamblecmds{}
\def\@onlypreamble#1{%
  \expandafter\gdef\expandafter\@preamblecmds\expandafter{%
       \@preamblecmds\do#1}}
       
\@onlypreamble\@onlypreamble
\@onlypreamble\@preamblecmds
\end{teX}


\section{Command Building}

\begin{docCommand}{@star@or@long} { \meta{parameter text} }
This is an interesting command in that it checks if \#1 is a star then sets \refCom{l@ngrel@x} to either |\long| or |\relax|. It is used for building \refCom{newcommand}, which is always |\long| except its star version.
\end{docCommand}

\begin{teX}
\def\@star@or@long#1{%
  \@ifstar
   {\let\l@ngrel@x\relax#1}%
   {\let\l@ngrel@x\long#1}}
\end{teX}

%

\begin{docCommand}{l@ngrel@x}{ }
 This is either |\relax| or |\long| depending on whether the |*|-form
 of a definition command is being executed.
\end{docCommand}
\begin{teX}
\let\l@ngrel@x\relax
\end{teX}

%
\begin{docCommand}{newcommand}{ \marg{cmd name} \oarg{number of arguments} \oarg{default for optional argument} \marg{replacement text}}
 User level |\newcommand|.%
 \begin{teX}
\def\newcommand{\@star@or@long\new@command}
\end{teX}
The command simply checks if it is long or short and the calls \refCom{new@command}.
\end{docCommand}
%
\begin{docCommand}{new@command}{} 
% \changes{v1.2u}{1995/10/16}{(DPC) Use \cs{@testopt} /1911}
\begin{teX}
\def\new@command#1{%
  \@testopt{\@newcommand#1}0}
\end{teX}


%
%
\begin{docCommand}{@newcommand}{ }
% \changes{LaTeX2e}{1993/11/23}{Macro added}
\end{docCommand}

\begin{docCommand}{@argdef}{}
% \changes{LaTeX2e}{1993/11/23}{Macro added}
\end{docCommand}
\begin{docCommand}{@xargdef}{}
\end{docCommand}
% \changes{LaTeX2e}{1993/11/23}{Macro interface changed}
% \changes{v1.1g}{2004/01/23}{Use kernel version of
%                             \cs{@ifnextchar} (pr/3501)}
%    Handling arguments for |\newcommand|.
\begin{teX}
\def\@newcommand#1[#2]{%
  \kernel@ifnextchar [{\@xargdef#1[#2]}%
                {\@argdef#1[#2]}}
\end{teX}
    Define |#1| if it is definable.

    Both here and in |\@xargdef| the replacement text is absorbed as
    an argument because if we are not allowed to make the definition
    we have to get rid of it completely.
    
\begin{teX}
\long\def\@argdef#1[#2]#3{%
   \@ifdefinable #1{\@yargdef#1\@ne{#2}{#3}}}
\end{teX}
%
% \changes{v1.2q}{1995/10/02}
%     {New implementation, using \cs{@test@opt}}
%    Handle the second optional argument.
\begin{teX}
\long\def\@xargdef#1[#2][#3]#4{%
  \@ifdefinable#1{%
\end{teX}

    Define the actual command to be:\\
    |\def\foo{\@protected@testopt\foo\\foo{default}}|\\
    where |\\foo| is a csname generated from applying |\csname| and
    |\string| to |\foo|, ie the actual name contains a backslash and
    therefore can't clash easily with existing command names.
    ``Default'' is the contents of the second optional argument of
    |(re)newcommand|.
    
    \begin{texexample}{newcommand internal form}{ex:newcom}
       \newcommand\mytest[2][other material] {some text}
       \meaning\mytest
    \end{texexample}
% The |\aut@global| command below is only used in the autoload
% format. If it is |\global| then a global definition will be made.
% \changes{v1.2q}{1995/10/02}
%     {Add \cs{aut@global} in autoload version}
% \changes{v1.2z2}{1998/03/04}
%     {Unnecessary \cs{expandafter} removed: pr/2758}
\begin{teX}
%<autoload>\aut@global
     \expandafter\def\expandafter#1\expandafter{%
          \expandafter
          \@protected@testopt
          \expandafter
          #1%
          \csname\string#1\endcsname
          {#3}}%
\end{teX}
    Now we define the internal macro ie |\\foo| which is supposed to
    pick up all arguments (optional and mandatory).
\begin{teX}
       \expandafter\@yargdef
          \csname\string#1\endcsname
           \tw@
           {#2}%
           {#4}}}
\end{teX}
\end{docCommand}


%
\begin{docCommand}{@testopt}{ }
 This macro encapsulates the most common call to |\@ifnextchar|, saving
 several tokens each time it is used in the definition of a command
 with an optional argument.
 
 |#1| The code to execute in the case that there is a |[| need not be
 a single token but can be any sequence of commands that `expects' to
 be followed by |[|. If this command were only used in |\newcommand|
 definitions then |#1| would be a single token and the braces could
 be omitted from |{#1}| in the definition below, saving a bit of
 memory.
\end{docCommand}
\begin{teX}
\long\def\@testopt#1#2{%
  \kernel@ifnextchar[{#1}{#1[{#2}]}}
\end{teX}


\begin{docCommand}{@protected@testopt}{ }
 Robust version of |\@testopt|. The extra argument (|#1|) must be a
 single token. If protection is needed the call expands to |\protect|
 applied to this token, and the 2nd and 3rd arguments are
 discarded (by |\@x@protect|). Otherwise |\@testopt| is called on
 the 2nd and 3rd arguments.

 This method of making commands robust avoids the need for using up
 two csnames per command, the price is the extra expansion time
 for the |\ifx| test.
 
\begin{teX}
\def\@protected@testopt#1{%%
  \ifx\protect\@typeset@protect
    \expandafter\@testopt
  \else
    \@x@protect#1%
  \fi}
\end{teX}
\end{docCommand}
%
\begin{docCommand}{@yargdef}{ }
\begin{docCommand}{@yargd@f}{}
\end{docCommand}
\end{docCommand}

    These generate a primitive argument specification, from a
    \LaTeX\ |[|\meta{digit}|]| form; in fact \meta{digit} can be
    anything such that |\number|~\meta{digit} is single digit.

    Reorganised slightly so that |\renewcommand{\reserved@a}[1]{foo}|
    works.  I am not sure this is worth it, as a following
    |\newcommand| would over-write the definition of |\reserved@a|.

    Recall that \LaTeX2.09 goes into an infinite loop with
    |\renewcommand[1]{\@tempa}{foo}| (DPC 6 October 93).

    Reorganised again (DPC 1999). Rather than make a loop to
    construct the argument spec by counting, just extract the
    required argument spec by using a delimited argument (delimited
    by the digit).  This is faster and uses less tokens. The coding
    is slightly odd to preserve the old interface (using |#2| =
    |\tw@| as the flag to surround the first argument with |[]|.  But
    the new method did not allow for the number of arguments |#3| not
    being given as an explicit digit; hence (further expansion of
    this argument and use of) |\number| was added later in 1999.

    It is not clear why these are still |\long|.

\begin{teX}
\long \def \@yargdef #1#2#3{%
  \ifx#2\tw@
    \def\reserved@b##11{[####1]}%
  \else
    \let\reserved@b\@gobble
  \fi
  \expandafter
    \@yargd@f \expandafter{\number #3}#1%
}
\end{teX}
%
% The |\aut@global| command below is only used in the autoload
% format. If it is |\global| then a global definition will be made.
% \changes{v1.2q}{1995/10/02}
%     {Add \cs{aut@global} in autoload version}
\begin{teX}
\long \def \@yargd@f#1#2{%
  \def \reserved@a ##1#1##2##{%
%<autoload>\aut@global
    \expandafter\def\expandafter#2\reserved@b ##1#1%
    }%
  \l@ngrel@x \reserved@a 0##1##2##3##4##5##6##7##8##9###1%
}
\end{teX}
%

%
\begin{docCommand}{@reargdef}{}
%  \changes{LaTeX2e}{1993/12/20}
%                {Kept old version of \cs{@reargdef}, for array.sty}
% \changes{v1.0b}{1994/03/12}{New defn, in terms of \cs{@yargdef}}
%  \changes{v1.2y}{1996/07/26}{third arg picked up by \cs{@yargdef}}
\begin{teX}
\long\def\@reargdef#1[#2]{%
  \@yargdef#1\@ne{#2}}
\end{teX}
\end{docCommand}
%
\begin{docCommand}{renewcommand} { }
    Check the command name is already used.  If not give an error
    message. Then temporarily
    disable |\@ifdefinable| then call |\newcommand|. (Previous
    version |\let#1=\relax| but this does not work too well if |#1|
\end{docCommand}
%    is |\@temp|\emph{a--e}.)
% \changes{LaTeX2e}{1993/11/23}{Macro reimplemented and extended}
% \changes{v1.1f}{1994/05/2}{Removed surplus \cs{space} in error}
\begin{teX}
\def\renewcommand{\@star@or@long\renew@command}
\end{teX}
%
\begin{docCommand}{renew@command} { }
\end{docCommand}
%  \changes{v1.2y}{1996/07/26}{use \cs{relax} in place of empty arg}
%  \changes{v1.2y}{1996/07/26}{use \cs{noexpand} instead of \cs{string}}
% \changes{v1.2z1}{1997/10/21}{Use \cs{begingroup}/\cs{endgroup} rather
%    than braces for grouping, to avoid generating empty math atom.}
\begin{teX}
\def\renew@command#1{%
  \begingroup \escapechar\m@ne\xdef\@gtempa{{\string#1}}\endgroup
  \expandafter\@ifundefined\@gtempa
     {\@latex@error{\noexpand#1undefined}\@ehc}%
     \relax
  \let\@ifdefinable\@rc@ifdefinable
  \new@command#1}
\end{teX}




\subsection{Checking if a command is definable}
%\begin{docCommand}{\@@ifdefinable}
\begin{docCommand}{@ifdefinable}{ }
 Test if user is allowed to define a command.
\end{docCommand}

\begin{docCommand}{@rc@ifdefinable}{}
   
\end{docCommand}
\begin{teX}
\long\def\@ifdefinable #1#2{%
      \edef\reserved@a{\expandafter\@gobble\string #1}%
     \@ifundefined\reserved@a
         {\edef\reserved@b{\expandafter\@carcube \reserved@a xxx\@nil}%
          \ifx \reserved@b\@qend \@notdefinable\else
            \ifx \reserved@a\@qrelax \@notdefinable\else
              #2%
            \fi
          \fi}%
         \@notdefinable}
\end{teX}
%    Saved definition of |\@ifdefinable|.

\begin{teX}
\let\@@ifdefinable\@ifdefinable
\end{teX}

    Version of |\@ifdefinable| for use with |\renewcommand|.  Does
    not do the check this time, but restores the normal definition.

\begin{teX}
\long\def\@rc@ifdefinable#1#2{%
  \let\@ifdefinable\@@ifdefinable
  #2}
\end{teX}

%\end{docCommand}

This command is not a general command for package builders but it has its uses. If the command
has been defined then it issues an error message via |\@notdefinable|. In the example
below we save the |\@notdefinable| in order to redirect the error message to the example.
We check if the counter |c@chapter| can be defined, since it has it will activate the |\@notdefinable| and
print an error. We then restore the command to its previous meaning.

\begin{texexample}{ifdefinable}{ex:ifdefinable}
\makeatletter
\let\save@notdefinable\@notdefinable
\def\@notdefinable{Not definable}
\@ifdefinable{c@chapter}{true}
\let\@notdefinable\save@notdefinable
\makeatother
\end{texexample}

\section{Environment Building commands}

\begin{docCommand}{newenvironment} {}
 Define a new user environment.
    |#1| is the environment name. |#2#| Grabs all the tokens up to
    the first |{|. These will be any optional arguments. They are not
    parsed at this point, but are just passed to |\@newenv| which
    will eventually call |\newcommand|. Any optional arguments will
    then be parsed by |\newcommand| as it defines the command that
    executes the `begin code' of the environment.

    This |#2#| trick removed with version 1.2i as it fails if a |{|
    occurs in the optional argument. Now use |\@ifnextchar| directly.

\end{docCommand}   
\begin{teX}
\def\newenvironment{\@star@or@long\new@environment}
\end{teX}
%
\begin{docCommand}{new@environment} {}

\begin{teX}
\def\new@environment#1{%
  \@testopt{\@newenva#1}0}
\end{teX}
\end{docCommand}
%


\begin{docCommand}{@newenva}{}

\begin{teX}
\def\@newenva#1[#2]{%
   \kernel@ifnextchar [{\@newenvb#1[#2]}{\@newenv{#1}{[#2]}}}
\end{teX}
\end{docCommand}
%
\begin{docCommand}{@newenvb}{}

\begin{teX}
\def\@newenvb#1[#2][#3]{\@newenv{#1}{[#2][{#3}]}}
\end{teX}

\end{docCommand}
%
%
\begin{docCommand}{renewenvironment}{}
    Redefine an environment.
    For |\renewenvironment| disable |\@ifdefinable| and then call
    |\newenvironment|.  It is OK to |\let| the argument to |\relax|
    here as there should not be a |@temp|\ldots\ environment.

\begin{teX}
\def\renewenvironment{\@star@or@long\renew@environment}
\end{teX}

\begin{docCommand}{renew@environment}{}
As a |csname| is used the name of an envronment can contain non-letters, such as dashes etc.
\begin{teX}
\def\renew@environment#1{%
  \@ifundefined{#1}%
     {\@latex@error{Environment #1 undefined}\@ehc
     }\relax
  \expandafter\let\csname#1\endcsname\relax
  \expandafter\let\csname end#1\endcsname\relax
  \new@environment{#1}}
\end{teX}
\end{docCommand}
\end{docCommand}


\begin{docCommand}{@newenv}{ }
\end{docCommand}
    The internal version of |\newenvironment|.

    Call |\newcommand| to define the \meta{begin-code} for the
    environment.  |\def| is used for the \meta{end-code} as it does
    not take arguments. (but may contain |\par|s)

 Make sure that an attempt to define a `graf' or `group' environment
 fails.
\begin{teX}
\long\def\@newenv#1#2#3#4{%
  \@ifundefined{#1}%
    {\expandafter\let\csname#1\expandafter\endcsname
                         \csname end#1\endcsname}%
    \relax
  \expandafter\new@command
     \csname #1\endcsname#2{#3}%
\end{teX}

%     {Add \cs{aut@global} in autoload version}
\begin{teX}
%<autoload>\aut@global
     \l@ngrel@x\expandafter\def\csname end#1\endcsname{#4}}
\end{teX}


\section{newif}

The \latex kernel provides the \refCom{newif} command that can be used to define boolean switches. It uses a couple of tricks to remove the |if| part of the command being defined and to |\let| the switches to true or false.
\begin{docCommand}{newif}{\marg{cmd} }
This is as the notes in the kernel mention a different type of allocation. A macro
to define boolean switches

 For example,
 |\newif\if@foo| creates |\@footrue|, |\@foofalse| to go with |\if@foo|.\footnote{Customarily these command use the `@' to make the definitions internal and also more readable.}
 
 Before we describe how this is achieved we make a small digression to test a not so widely
 known trick to get the \cmd{\string} not to print the backslash, by setting the |\escapechar|
 to -1. This is really an esoteric trick.\index{hacks}
 \end{docCommand}
 
 \begin{texexample}{escapechar}{ex:escapechar}
 \bgroup
 \def\test{This is a test}
 \makeatletter
 \escapechar\m@ne
 \test
 
 \string\test
 
 test
 \makeatother
 \egroup
 \end{texexample}
 
The \cmd{\newif} is defined in a very clever way.  

\begin{teXXX}
\def\newif#1{% 
  % save original definition 
  \count@\escapechar 
  % allocate to -1
  \escapechar\m@ne
  % set to false as a default
    \let#1\iffalse
  % define \@footrue and \@foofalse 
    \@if#1\iftrue
    \@if#1\iffalse
  \escapechar\count@}
\end{teXXX}

\begin{docCommand}{@if}{\meta{name of newif}\meta{|ifalse or iftrue|} }
The auxiliary macro for the |\newif| command.

\begin{phdverbatim}
\def\@if#1#2{% 
  \expandafter\def\csname\expandafter\@gobbletwo\string#1%
                    \expandafter\@gobbletwo\string#2\endcsname
                       {\let#1#2}}
\end{phdverbatim}
\end{docCommand}

%\begin{texexample}{Newif}{ex:newif}
%\def\Newif#1{%  \iftest 
%  \count@\escapechar \escapechar\m@ne
%    \let#1\iffalse
%    \@If#1\iftrue
%    \@If#1\iffalse
%  \escapechar\count@}
%\def\@If#1#2{% (*@\label{atif1} @*)
%  \expandafter\def\csname\expandafter\@gobbletwo\string#1%
%                    \expandafter\@gobbletwo\string#2\endcsname
%                       {\let#1#2}}  
%\end{texeexample}


This is a mind twisting way to define the |\newif| and  I am sure a method  applauded by every Byzantine
General. It also resolved a long standing query in my head as to why there was never a |@gobblethree| command in the kernel  (see \autoref{gobble}).

\section{Provide version of commands}

\begin{docCommand}{providecommand}{}

 |\providecommand| takes the same arguments as |\newcommand|, but
 discards them if |#1| is already defined, Otherwise it just acts like
 |\newcommand|. This implementation currently leaves any discarded
 definition in |\reserved@a| (and possibly |\\reserved@a|) this
 wastes a bit of space, but it will be reclaimed as soon as these
 scratch macros are redefined.


\begin{teX}
\def\providecommand{\@star@or@long\provide@command}
\end{teX}

\begin{docCommand}{provide@command}{}

\begin{teX}
\def\provide@command#1{%
  \begingroup
    \escapechar\m@ne\xdef\@gtempa{{\string#1}}%
  \endgroup
  \expandafter\@ifundefined\@gtempa
    {\def\reserved@a{\new@command#1}}%
    {\def\reserved@a{\renew@command\reserved@a}}%
   \reserved@a}%
\end{teX}
\end{docCommand}
\end{docCommand}

\begin{docCommand}{CheckCommand}{}

    |\CheckCommand| takes the same arguments as |\newcommand|. If
    the command already exists, with the same definition, then
    nothing happens, otherwise a warning is issued. Useful for
    checking the current state befor a macro package starts
    redefining things.  Currently two macros are considered to have
    the same definition if they are the same except for different
    default arguments.  That is, if the old definition was:
    |\newcommand\xxx[2][a]{(#1)(#2)}| then
    |\CheckCommand\xxx[2][b]{(#1)(#2)}| would \emph{not} generate a
    warning, but, for instance |\CheckCommand\xxx[2]{(#1)(#2)}|
    would.
\begin{teX}
\def\CheckCommand{\@star@or@long\check@command}
\end{teX}
    |\CheckCommand| is only available in the preamble part of the
    document.
\begin{teX}
\@onlypreamble\CheckCommand
\end{teX}

\begin{docCommand}{check@command}{}
\begin{teX}
\def\check@command#1#2#{\@check@c#1{#2}}
\@onlypreamble\check@command
\end{teX}
\end{docCommand}
\end{docCommand}

\begin{docCommand}{@check@c}{}
    |\CheckCommand| itself just grabs all the arguments we need,
    without actually looking for |[| optional argument forms.  Now
    define |\reserved@a|. If |\\reserved@a| is then defined, compare it
    with the ``|\#1|' otherwise compare |\reserved@a| with |#1|.
\begin{teX}
\long\def\@check@c#1#2#3{%
  \expandafter\let\csname\string\reserved@a\endcsname\relax
  \renew@command\reserved@a#2{#3}%
  \@ifundefined{\string\reserved@a}%
   {\@check@eq#1\reserved@a}%
   {\expandafter\@check@eq
           \csname\string#1\expandafter\endcsname
           \csname\string\reserved@a\endcsname}}
\@onlypreamble\@check@c
\end{teX}
\end{docCommand}
%
\begin{docCommand}{@check@eq}{}
     Complain if |#1| and |#2| are not |\ifx| equal.
\begin{teX}
\def\@check@eq#1#2{%
  \ifx#1#2\else
     \@latex@warning@no@line
               {Command \noexpand#1 has
                changed.\MessageBreak
                Check if current package is valid}%
  \fi}
\@onlypreamble\@check@eq
\end{teX}
\end{docCommand}

\subsection{Argument gobbling command utilities}
\label{gobble}

The \docAuxCommand{@gobble} macro is used to get rid of its argument. Similarly \docAuxCommand{@gobbletwo} and \docAuxCommand{@gobblefour} eat two or four arguments respectively.
\begin{teX}
\long\def \@gobble #1{}
\long\def \@gobbletwo #1#2{}
\long\def \@gobblefour #1#2#3#4{}
\end{teX}


\section{Other argument grabbers}

The commands \docAuxCommand{@firstofone}, \docAuxCommand{@firstoftwo} and \docAuxCommand{@secondoftwo} are amongst the most widely used commands in the kernel. They only grab the argumeny indicated by their name.

%    Some argument-grabbers.
\begin{teX}
\long\def\@firstofone#1{#1}
\long\def\@firstoftwo#1#2{#1}
\long\def\@secondoftwo#1#2{#2}
\end{teX}

\begin{docCommand}{@iden}{}
    |\@iden| is another name for |\@firstofone| for
    compatibility reasons.
\begin{teX}
\let\@iden\@firstofone
\end{teX}
\end{docCommand}
%
\begin{docCommand}{@thirdofthree}{}
%    Another grabber now used in the encoding specific
%    section.

\begin{teX}
\long\def\@thirdofthree#1#2#3{#3}
\end{teX}
\end{docCommand}
%
%
\begin{docCommand}{@expandtwoargs}{}
 A macro to totally expand two arguments to another macro
   
\begin{teX}
\def\@expandtwoargs#1#2#3{%
   \edef\reserved@a{\noexpand#1{#2}{#3}}\reserved@a}
\end{teX}
\end{docCommand}

\begin{texexample}{Expand two arguments}{ex:2args}
\makeatletter
\bgroup
  \def\test#1#2{#1, #2} 
  \def\xx{first argument}
  \def\yy{second argument}
  \@expandtwoargs\test{\xx}{\yy}
\egroup
\makeatother
\end{texexample}

\begin{docCommand}{@backslashchar}{}
A category code 12 backslash. See also \refCom{textbackslash}  which is a better alternative.
\begin{teX}
\edef\@backslashchar{\expandafter\@gobble\string\\ }
\end{teX}
\end{docCommand}


 \section{Robust commands and protect}

 Fragile and robust commands are one of the thornier issues in
 \LaTeX's commands.  Whilst typesetting documents, \LaTeX{} makes use
 of many of \TeX's features, such as arithmetic, defining macros, and
 setting variables.  However, there are (at least) three different
 occasions when these commands are not safe.  These are called
 `moving arguments' by \LaTeX, and consist of:
 \begin{itemize}
 \item writing information to a file, such as indexes or tables of
    contents.
 \item writing information to the screen.
 \item inside an |\edef|, |\message|, |\mark|, or other command which
    evaluates its argument fully.
 \end{itemize}
 The method \LaTeX{} uses for making fragile commands robust is to
 precede them with |\protect|.  This can have one of five possible
 values:
 
 \begin{itemize}
 \item |\relax|, for normal typesetting.  So |\protect\foo| will
    execute |\foo|.
 \item |\string|, for writing to the screen.  So |\protect\foo| will
    write |\foo|.
 \item |\noexpand|, for writing to a file.  So |\protect\foo| will
    write |\foo| followed by a space.
 \item |\@unexpandable@protect|, for writing a moving argument to a
    file.  So |\protect\foo| will write |\protect\foo| followed by a
    space.  This value is also used inside |\edef|s, |\mark|s and
    other commands which evaluate their arguments fully.
 \item |\@unexpandable@noexpand|, for performing a deferred write
    inside an |\edef|.  So |\protect\foo| will write |\foo| followed
    by a space.  If you want |\protect\foo| to be written, you should
    use |\@unexpandable@protect|. (Removed as never used).
 \end{itemize}

\begin{docCommand}{@unexpandable@protect}{}
\begin{docCommand}{@unexpandable@noexpand}{}

%    These commands are used for setting |\protect| inside |\edef|s.
\begin{teX}
\def\@unexpandable@protect{\noexpand\protect\noexpand}
%\def\@unexpandable@noexpand{\noexpand\noexpand\noexpand}
\end{teX}
\end{docCommand}
\end{docCommand}
%
% \changes{v1.2e}{1994/11/04}{Rewrote protected short commands
%    using \cs{x@protect}. ASAJ.}
%
\begin{docCommand}{DeclareRobustCommand}{}
\begin{docCommand}{declare@robustcommand}{}
    This is a package-writers command, which has the same syntax as
    |\newcommand|, but which declares a protected command.  It does
    this by having\\
    |\DeclareRobustCommand\foo|\\
    define |\foo| to be
    |\protect\foo<space>|,\\
    and then use |\newcommand\foo<space>|.\\
    Since the internal command is |\foo<space>|, when it is written
    to an auxiliary file, it will appear as |\foo|.

    We have to be a
    bit cleverer if we're defining a short command, such as |\_|, in
    order to make sure that the auxiliary file does not include a
    space after the command, since |\_ a| and |\_a| aren't the same.
    In this case we define |\_| to be:
\begin{verbatim}
    \x@protect\_\protect\_<space>
\end{verbatim}
    which expands to:
\begin{verbatim}
    \ifx\protect\@typeset@protect\else
       \@x@protect@\_
    \fi
    \protect\_<space>
\end{verbatim}
    Then if |\protect| is |\@typeset@protect| (normally |\relax|)
    then we just perform |\_<space>|, and otherwise
    |\@x@protect@| gobbles everything up and expands to
    |\protect\_|.

    \emph{Note}: setting |\protect| to any value other than |\relax|
    whilst in `typesetting' mode will cause commands to go into an
    infinite loop!  In particular, setting |\relax| to |\@empty| will
    cause |\_| to loop forever.  It will also break lots of other
    things, such as protected |\ifmmode|s inside |\halign|s.  If you
    really really have to do such a thing, then please set
    |\@typeset@protect| to be |\@empty| as well.  (This is what the
    code for |\patterns| does, for example.)

    More fun with |\expandafter| and |\csname|.
\begin{teX}
\def\DeclareRobustCommand{\@star@or@long\declare@robustcommand}
\end{teX}
%
\begin{teX}
\def\declare@robustcommand#1{%
   \ifx#1\@undefined\else\ifx#1\relax\else
      \@latex@info{Redefining \string#1}%
   \fi\fi
   \edef\reserved@a{\string#1}%
   \def\reserved@b{#1}%
   \edef\reserved@b{\expandafter\strip@prefix\meaning\reserved@b}%
\end{teX}
% \changes{v1.2s}{1995/10/06}
%     {Add \cs{aut@global} in autoload version}
\begin{teX}
%<autoload>\aut@global
   \edef#1{%
      \ifx\reserved@a\reserved@b
         \noexpand\x@protect
         \noexpand#1%
      \fi
      \noexpand\protect
      \expandafter\noexpand\csname
         \expandafter\@gobble\string#1 \endcsname
   }%
   \let\@ifdefinable\@rc@ifdefinable
   \expandafter\new@command\csname
      \expandafter\@gobble\string#1 \endcsname
}
\end{teX}
\end{docCommand}
\end{docCommand}
%
%
\begin{docCommand}{@x@protect}{}
\begin{docCommand}{x@protect}{}
%
\begin{teX}
\def\x@protect#1{%
   \ifx\protect\@typeset@protect\else
      \@x@protect#1%
   \fi
}
\def\@x@protect#1\fi#2#3{%
   \fi\protect#1%
}
\end{teX}
\end{docCommand}
\end{docCommand}
%
\begin{docCommand}{@typeset@protect}{}
%
\begin{teX}
\let\@typeset@protect\relax
\end{teX}
\end{docCommand}
%
% \changes{v1.2e}{1994/11/04}{Added commands for setting and restoring
%    \cs{protect}.  ASAJ.}
%
\begin{docCommand}{set@display@protect}{}
\begin{docCommand}{set@typeset@protect}{}
%    These macros set |\protect| appropriately for typesetting or
%    displaying.
% \changes{v1.2o}{1995/07/03}{Use \cs{@typeset@protect} for init}
\begin{teX}
\def\set@display@protect{\let\protect\string}
\def\set@typeset@protect{\let\protect\@typeset@protect}
\end{teX}
\end{docCommand}
\end{docCommand}
%
\begin{docCommand}{protected@edef}{}
\begin{docCommand}{protected@xdef}{}
\end{docCommand}
\end{docCommand}
\begin{docCommand}{unrestored@protected@xdef}{}
\begin{docCommand}{restore@protect}
    The commands |\protected@edef| and |\protected@xdef| perform
    `safe' |\edef|s and |\xdef|s, saving and restoring |\protect|
    appropriately.  For cases where restoring |\protect| doesn't
    matter, there's an `unsafe' |\unrestored@protected@xdef|, useful
    if you know what you're doing!
    
\begin{teX}
\def\protected@edef{%
   \let\@@protect\protect
   \let\protect\@unexpandable@protect
   \afterassignment\restore@protect
   \edef
}
\def\protected@xdef{%
   \let\@@protect\protect
   \let\protect\@unexpandable@protect
   \afterassignment\restore@protect
   \xdef
}
\def\unrestored@protected@xdef{%
   \let\protect\@unexpandable@protect
   \xdef
}
\def\restore@protect{\let\protect\@@protect}
\end{teX}

\end{docCommand}
\end{docCommand}
%
%
\begin{docCommand}{protect}{}
%    The normal meaning of |\protect|
% \changes{v1.2j}{1995/04/29}{Init \cs{protect} here}
\begin{teX}
\set@typeset@protect
\end{teX}
\end{docCommand}
%
\subsection{Internal defining commands}
%
% These commands are used internally to define other \LaTeX{}
% commands.
\begin{docCommand}{@ifundefined}{}
% \changes{LaTeX2e}{1993/11/23}{Redefined to remove a trailing \cs{fi}}
%    Check if first arg is undefined or \cs{relax} and execute second or
%    third arg depending,
\begin{teX}
\def\@ifundefined#1{%
  \expandafter\ifx\csname#1\endcsname\relax
    \expandafter\@firstoftwo
  \else
    \expandafter\@secondoftwo
  \fi}
\end{teX}
\end{docCommand}
%
%
\begin{docCommand}{@qend}{}
\begin{docCommand}{@qrelax}{}
% The following define |\@qend| and |\@qrelax| to be the strings
% `|end|' and `|relax|' with the characters |\catcode|d 12.
\begin{teX}
\edef\@qend{\expandafter\@cdr\string\end\@nil}
\edef\@qrelax{\expandafter\@cdr\string\relax\@nil}
\end{teX}
\end{docCommand}
\end{docCommand}


\begin{docCommand}{@ifnextchar}{}

  |\@ifnextchar| peeks at the following character and compares it
  with its first argument. If both are the same it executes its
  second argument, otherwise its third.
\begin{teX}
\long\def\@ifnextchar#1#2#3{%
  \let\reserved@d=#1%
  \def\reserved@a{#2}%
  \def\reserved@b{#3}%
  \futurelet\@let@token\@ifnch}
\end{teX}
\end{docCommand}

\begin{docCommand}{kernel@ifnextchar}{}

    This macro is the kernel version of |\@ifnextchar| which is used
    in a couple of places to prevent the AMS variant from being used
    since in some places this produced chaos (for example
    if an \texttt{fd} file
    is loaded in a random place then the optional argument to
    |\ProvidesFile| could get printed there instead of being written
    only in the log file.  This happened
    when there was a space or a newline between the mandatory and
    optional arguments! It should really be fixed in the
    \texttt{amsmath} package one day, but\ldots

    Note that there may be other places in the kernel where this version
    should be used rather than the original, but variable, version.

\begin{teX}
\let\kernel@ifnextchar\@ifnextchar
\end{teX}
\end{docCommand}
%
%
\begin{docCommand}{@ifnch}{}
    |\@ifnch| is a tricky macro to skip any space tokens that may
    appear before the character in question. If it encounters a space
    token, it calls |\@xifnch|.\footnote{There is an error in the kernel at this point in the documentation, as the
    verbatim guards have been omitted.}

\begin{teX}
\def\@ifnch{%
  \ifx\@let@token\@sptoken
    \let\reserved@c\@xifnch
  \else
    \ifx\@let@token\reserved@d
      \let\reserved@c\reserved@a
    \else
      \let\reserved@c\reserved@b
    \fi
  \fi
  \reserved@c}
\end{teX}
\end{docCommand}
%
\begin{docCommand}{@sptoken}{}
    The following code makes |\@sptoken| a space token. It is
    important here that the control sequence |\:| consists of
    a non-letter only, so that the following whitespace is
    significant. Together with the fact that the equal sign
    in a |\let| may be followed by only one optional space
    the desired effect is achieved.
    NOTE: the following hacking must precede the definition of |\:|
    as math medium space.
\begin{teX}
\def\:{\let\@sptoken= } \:  % this makes \@sptoken a space token
\end{teX}
\end{docCommand}
%
\begin{docCommand}{@xifnch}{}
    In the following definition of |\@xifnch|, |\:| is again used
    to get a space token as delimiter into the definition.
\begin{teX}
\def\:{\@xifnch} \expandafter\def\: {\futurelet\@let@token\@ifnch}
\end{teX}
\end{docCommand}

The following two commands are widely used in preambles to change the \refCom{catcode} of |@| to 11 or 12.

\begin{docCommand}{makeatletter}{}
Make internal control sequences accessible and inaccessible (\docAuxCommand{makeatother}).
\begin{teX}
\def\makeatletter{\catcode`\@11\relax}
\def\makeatother{\catcode`\@12\relax}
\end{teX}
\end{docCommand}


This |\@ifstar| command is another useful 
\begin{docCommand}{@ifstar} {\marg{true code} \marg{false code}}
 The new implementation below avoids passing the \meta{true code}
 Through one more \refCom{def} than the \meta{false code}, which previously
 meant that |#| had to be written as |####| in one argument, but |##|
 in the other. The |*| is gobbled by |\@firstoftwo|.
\begin{teX}
\def\@ifstar#1{\@ifnextchar *{\@firstoftwo{#1}}}
\end{teX}
\end{docCommand}

\begin{docCommand}{@dblarg}{}
\begin{docCommand}{@xdblarg}{}

\begin{teX}
\long\def\@dblarg#1{\kernel@ifnextchar[{#1}{\@xdblarg{#1}}}
\long\def\@xdblarg#1#2{#1[{#2}]{#2}}
\end{teX}
\end{docCommand}
\end{docCommand}
%

\begin{docCommand}{@sanitize}{}
 The command |\@sanitize| changes the catcode of all \emph{special characters}
 except for braces to `other'.  It can be used for commands like
 |\index| that want to write their arguments verbatim.  Needless to
 say, this command should only be executed within a group, or chaos
 will ensue.
\end{docCommand}

\begin{teX}
\def\@sanitize{\@makeother\ \@makeother\\\@makeother\$\@makeother\&%
\@makeother\#\@makeother\^\@makeother\_\@makeother\%\@makeother\~}
\end{teX}
%
\begin{docCommand}{@onelevel@sanitize}{}
% \changes{v1.2c}{1994/10/30}{Macro added}

    This makes the whole ``meaning'' of |#1| (its one-level
    expansion) into catcode 12 tokens: it could be used in
    |\DeclareRobustCommand|.

    If it is to be used on default float specifiers, this should be
    done when they are defined.
\begin{teX}
\def \@onelevel@sanitize #1{%
  \edef #1{\expandafter\strip@prefix
           \meaning #1}%
}
\end{teX}
\end{docCommand}

 \subsection{Commands for Autoloading}


\begin{docCommand}{aut@global}{}
% \changes{v1.2q}{1995/10/02}
%     {Macro added}
% This command is only defined in the `autoload' format. It is
% normally |\relax| but may be set to |\global|, in which case
% |\newif| and the commands based on |\newcommand| will all make
% global definitions.
\begin{teX}
\let\aut@global\relax
\end{teX}
\end{docCommand}
%
\begin{docCommand}{@autoload}{}

Use \cs{@@input} not \cs{input} to save string space and
      stops autoload files appearing in \cs{listfiles}

 This macro is only defined in the `autoload' format. It inputs a
 package `|auto#1.sty|' within a local group, and with normalised
 catcodes. |\aut@global| is set to |\global| so that |\newif|
 |\newcommand| and related commands make global definitions.
\begin{teX}
\def\@autoload#1{%
  \begingroup
  \makeatletter
  \let\aut@global\global
  \nfss@catcodes
  \catcode`\ =10
  \let\@latex@e@error\@gobble
  \@@input auto#1.sty\relax
  \endgroup}
\end{teX}
\end{docCommand}

This been a rather long discussion, but this is almost the heart of the kernel commands that are most useful
for package writers. Spend time to understand this section well, as many of the hacks described here are useful
and are used in most of the other sections of the kernel.


