\chapter{LaTeX kernel sectioning class}
\label{kernel:ltsect}

  Johannes Braams,
  David Carlisle,
  Alan Jeffrey,
  Leslie Lamport,
  Frank Mittelbach,
  Chris Rowley,
  Tobias Oetiker (Tobi updated
              the comments to `doc' conventions)
  Rainer Sch\"opf


 \section{Sectioning Commands}

 This file defines the declarations such as |\author| which are used
 by |\maketitle|. |\maketitle| itself is defined by each class, not
 in the \LaTeX{} kernel.

 The second part of the file defines the generic commands used for
 defining sectioning commands such as |\chapter|. Again the actual
 document level commands are defined in the class files, in terms of
 these commands.


    \begin{teX}
\message{title,}
    \end{teX}

 \subsection{The Title}

 \DescribeMacro{\title}
 The user defines the title and  author by the declarations
 |\title|\marg{name},
 \DescribeMacro{\author}
 |\author|\marg{name}

 \DescribeMacro{\date}
 Similarly the date is declared with
 |\date|\marg{date}.

 \DescribeMacro{\thanks}
 Inside these, the |\thanks|\marg{footnote text} command may be used
 to make acknowledgements, notice of address, etc.\ in a footnote.
 \DescribeMacro{\and}
 If there are multiple authors, they have to be separated with the
 |\and| command.

 \DescribeMacro{\maketitle}
 And finally, the |\maketitle| command produces the actual title,
 using the information previously saved with the other commands.

  \begin{docCommand}{title}{}
  \begin{docCommand}{@title}{}
% \changes{LaTeX2e}{1993/12/11}{Added default}
% |\title| for use in |\maketitle|. If not given |\maketitle| will
% produce an error message.
       \begin{teX}
\def\title#1{\gdef\@title{#1}}
\def\@title{\@latex@error{No \noexpand\title given}\@ehc}
       \end{teX}
  \end{docCommand}
  \end{docCommand}
%
  \begin{docCommand}{author}{}
  \begin{docCommand}{@author}{}
  |\author| for use in |\maketitle|. If not given |\maketitle| will
  produce a warning message.

       \begin{teX}
\def\author#1{\gdef\@author{#1}}
\def\@author{\@latex@warning@no@line{No \noexpand\author given}}
       \end{teX}
  \end{docCommand}
  \end{docCommand}
%
  \begin{docCommand}{date}{}
  \begin{docCommand}{@date}{}
%    |\date| for use in |\maketitle|. If not given |\maketitle| will
%    produce |\today| as the default.
       \begin{teX}
\def\date#1{\gdef\@date{#1}}
\gdef\@date{\today}
       \end{teX}
  \end{docCommand}
  \end{docCommand}
%
% \changes{1.0h}{1994/11/04}{(ASAJ) Added \cs{protected@xdef} to
%    \cs{thanks}.}
\begin{docCommand}{thanks}{}
       \begin{teX}
\def\thanks#1{\footnotemark
    \protected@xdef\@thanks{\@thanks
        \protect\footnotetext[\the\c@footnote]{#1}}%
}
       \end{teX}
\end{docCommand}
%
\begin{docCommand}{@thanks}{}
       \begin{teX}
\let\@thanks\@empty
       \end{teX}
\end{docCommand}


\begin{docCommand}{and}{}
 The \cmd{\and} is used to join the names of authors and to place them in a tabular!
       \begin{teX}
\def\and{%                  % \begin{tabular}
  \end{tabular}%
  \hskip 1em \@plus.17fil%
  \begin{tabular}[t]{c}}%   % \end{tabular}
       \end{teX}
\end{docCommand}
%
       \begin{teX}
\message{sectioning,}
       \end{teX}

 \subsection{Sectioning}

\begin{docCommand}{@secpenalty}{}
       \begin{teX}
\newcount\@secpenalty
\@secpenalty = -300
       \end{teX}
\end{docCommand}

\begin{docCommand}{if@noskipsec}{}
\begin{docCommand}{@noskipsectrue}{}
% \changes{1.0w}{1996/09/29}{Added documentation}
% Way back in 1991 (08/26) FMi \& RmS set the |\@noskipsec| switch
% to true for the preamble and to false in |\document|.
% This was done to trap lists and related text in the preamble but it
% does not catch everything.
       \begin{teX}
\newif\if@noskipsec \@noskipsectrue
       \end{teX}
\end{docCommand}
\end{docCommand}
%
\begin{docCommand}{@startsection}

 The |\@startsection{|\meta{name}|}{|\meta{level}|}{|%
       \meta{indent}|}{|\meta{beforeskip}|}|\\
     |{|\meta{afterskip}|}{|\meta{style}|}*[|\meta{altheading}%
     |]{|\meta{heading}|}|
 command is the mother of all the user level sectioning commands.
 The part after the |*|, including the |*| is optional.

 \begin{description}
 \item[name:] e.g., 'subsection'
 \item[level:] a number, denoting depth of section -- e.g., chapter=1,
                 section = 2, etc.
 \item[indent:] Indentation of heading from left margin
 \item[beforeskip:] Absolute value = skip to leave above the heading.
                If negative, then paragraph indent of text following
                heading is suppressed.
 \item[afterskip:] if positive, then skip to leave below heading, else
                negative of skip to leave to right of run-in heading.
 \item[style:] Commands to set style. Since June 1996 release the
               \emph{last} command in this argument may be a command
                such as |\MakeUppercase| or |\fbox| that takes an
                argument. The section heading will be supplied as the
                argument to this command. So setting |#6| to, say,
                |\bfseries\MakeUppercase| would produce bold,
                uppercase headings.
 \end{description}

  If `|*|' is  missing, then increment the counter.  If it is
  present, then there should be no |[|\meta{altheading}|]| argument.
  The command uses the counter 'secnumdepth'. It contains a pointer
  to the highest section level that is to be numbered.

  \textbf{Warning:}
  The |\@startsection| command should be at the same or higher
  grouping level as the text that follows it.  For example, you should
  \emph{not} do something like
  \begin{verbatim}
      \def\foo{ \begingroup ...
                   \paragraph{...}
                 \endgroup}
  \end{verbatim}
%
% Pseudocode for the |\@startsection| command
% \begin{oldcomments}
% \@startsection {NAME}{LEVEL}{INDENT}{BEFORESKIP}{AFTERSKIP}{STYLE} ==
%    BEGIN
%     IF  @noskipsec = T  THEN  \leavevmode  FI
%                              % true if previous section had no body.
%
%     \par
%     \@tempskipa  := BEFORESKIP
%     @afterindent := T
%     IF \@tempskipa < 0  THEN  \@tempskipa  := -\@tempskipa
%                               @afterindent := F
%     FI
%     IF @nobreak = true
%       THEN \everypar == null
%       ELSE \addpenalty{\@secpenalty}
%            \addvspace{\@tempskipa}
%     FI
%     IF * next
%       THEN \@ssect{INDENT}{BEFORESKIP}{AFTERSKIP}{STYLE}
%       ELSE \@dblarg{\@sect
%                {NAME}{LEVEL}{INDENT}
%                {BEFORESKIP}{AFTERSKIP}{STYLE}}
%     FI
% END
% \end{oldcomments}
%
\numberlineat{22}
       \begin{teX}
\def\@startsection#1#2#3#4#5#6{%
  \if@noskipsec \leavevmode \fi
  \par
  \@tempskipa #4\relax
  \@afterindenttrue
  \ifdim \@tempskipa <\z@
    \@tempskipa -\@tempskipa \@afterindentfalse
  \fi
  \if@nobreak
    \everypar{}%
  \else
    \addpenalty\@secpenalty\addvspace\@tempskipa
  \fi
  \@ifstar
    {\@ssect{#3}{#4}{#5}{#6}}%
    {\@dblarg{\@sect{#1}{#2}{#3}{#4}{#5}{#6}}}}
       \end{teX}
\end{docCommand}
%

\begin{docCommand}{@sect}{}
% Pseudocode for the |\@sect| command
% \begin{oldcomments}
% \@sect{NAME}{LEVEL}{INDENT}{BEFORESKIP}{AFTERSKIP}{STYLE}[ARG1]{ARG2}
%           ==
%   BEGIN
%    IF LEVEL > \c@secnumdepth
%      THEN \@svsec :=L null
%      ELSE \refstepcounter{NAME}
%           \@svsec :=L BEGIN \@seccntformat{#1}\relax END
%    FI
%    IF AFTERSKIP > 0
%      THEN \begingroup
%              STYLE
%              \@hangfrom{\hskip INDENT\@svsec}
%              {\interlinepenalty 10000 ARG2\par}
%           \endgroup
%           \NAMEmark{ARG1}
%           \addcontentsline{toc}{NAME}
%              { IF  LEVEL > \c@secnumdepth
%                  ELSE \protect\numberline{\theNAME}  FI
%                ARG1 }
%      ELSE \@svsechd == BEGIN  STYLE
%                               \hskip INDENT\@svsec
%                               ARG2
%                               \NAMEmark{ARG1}
%                               \addcontentsline{toc}{NAME}
%                                  { IF  LEVEL > \c@secnumdepth
%                                      ELSE
%                                        \protect\numberline{\theNAME}
%                                      FI
%                                    ARG1 }
%                        END
%    FI
%    \@xsect{AFTERSKIP}
% END
% \end{oldcomments}
%
% \changes{LaTeX2.09}{1992/08/25}
%         {(FMi) replaced explicit setting of \cs{@svsec}
%               by call to \cs{@seccntformat}}
% \changes{LaTeX2.09}{1993/08/05}
%         {(RmS) Made sure that \cs{protect} works correctly in
%               expansion of \cs{the} counter}
% \changes{1.0h}{1994/11/04}
%         {(ASAJ) Added \cs{protected@edef}.}
\numberlineat{38}
       \begin{teX}
\def\@sect#1#2#3#4#5#6[#7]#8{%
  \ifnum #2>\c@secnumdepth
    \let\@svsec\@empty
  \else
    \refstepcounter{#1}%
       \end{teX}

    Apparently since |\@seccntformat| might end with an improper |\hskip| which
    is scanning forward for |plus| or |minus| we end the definition
    of |\@svsec| with |\relax| as a precaution.

\numberlineat{43}
       \begin{teX}
    \protected@edef\@svsec{\@seccntformat{#1}\relax}%
  \fi
  \@tempskipa #5\relax
  \ifdim \@tempskipa>\z@
    \begingroup
       \end{teX}
% \changes{v1.0s}{1996/05/21}
%         {(DPC) Moved brace to allow commands like
%           \cs{MakeUppercase} in 6th argument.
%            Changed \cs{par} to \cs{endgraf} to allow non-long
%            commands. internal/2148}
% \changes{v1.0t}{1996/06/10}
%         {(DPC) Changed \cs{endgraf} to \cs{@@par}}
% This |{| used to be after the argument to |\@hangfrom| but was moved
% here to allow commands such as |\MakeUppercase| to be used at the end
% of |#6|.
\startlineat{48}
       \begin{teX}
      #6{%
        \@hangfrom{\hskip #3\relax\@svsec}%
          \interlinepenalty \@M #8\@@par}%
    \endgroup
    \csname #1mark\endcsname{#7}%
    \addcontentsline{toc}{#1}{%
      \ifnum #2>\c@secnumdepth \else
        \protect\numberline{\csname the#1\endcsname}%
      \fi
      #7}%
  \else
%
    \def\@svsechd{%
      #6{\hskip #3\relax
      \@svsec #8}%
      \csname #1mark\endcsname{#7}%
      \addcontentsline{toc}{#1}{%
        \ifnum #2>\c@secnumdepth \else
          \protect\numberline{\csname the#1\endcsname}%
        \fi
        #7}}%
  \fi
  \@xsect{#5}}
       \end{teX}
\end{docCommand}
%
\begin{docCommand}{@xsect}{}%
% Pseudocode for the |\@xsect| command
% \begin{oldcomments}
% \@xsect{AFTERSKIP} ==
%  BEGIN
%    IF AFTERSKIP > 0
%      THEN \par \nobreak
%           \vskip AFTERSKIP
%           \@afterheading
%      ELSE @nobreak :=G F
%           @noskipsec :=G T
%           \everypar{ IF @noskipsec = T
%                        THEN @noskipsec :=G F
%                             \clubpenalty :=G 10000
%                             \hskip -\parindent
%                             \begingroup
%                               \@svsechd
%                             \endgroup
%                             \unskip
%                             \hskip -AFTERSKIP \relax
%                                           %% relax added 14 Jan 91
%                        ELSE \clubpenalty :=G \@clubpenalty
%                             \everypar := NULL
%                      FI
%                    }
%    FI
%
%   END
% \end{oldcomments}

\startlineat{70}
       \begin{teX}
\def\@xsect#1{%
  \@tempskipa #1\relax
  \ifdim \@tempskipa>\z@
       \end{teX}
%    Why not combine |\@sect| and |\@xsect| and save doing the
%    same test twice? It is not possible to change this now as these
%    have become hooks!
%
%    This |\par| seems unnecessary.
       \begin{teX}
    \par \nobreak
    \vskip \@tempskipa
    \@afterheading
  \else
    \@nobreakfalse
    \global\@noskipsectrue
    \everypar{%
      \if@noskipsec
        \global\@noskipsecfalse
       {\setbox\z@\lastbox}%
        \clubpenalty\@M
        \begingroup \@svsechd \endgroup
        \unskip
        \@tempskipa #1\relax
        \hskip -\@tempskipa
      \else
        \clubpenalty \@clubpenalty
        \everypar{}%
      \fi}%
  \fi
  \ignorespaces}
       \end{teX}
\end{docCommand}
%
\begin{docCommand}{@seccntformat}{}
    This command formats the section number including the space
    following it. This command has proved a bit problematic when I parameterized
    the sectioning comands, as it is uniform for all the sectioning commands. For example
    after an inline paragraph command we might want to leave a bit more space.

\startlineat{94}
       \begin{teX}
\def\@seccntformat#1{\csname the#1\endcsname\quad}
       \end{teX}
\end{docCommand}
%
% Pseudocode for the |\@ssect| command
% \begin{oldcomments}
% \@ssect{INDENT}{BEFORESKIP}{AFTERSKIP}{STYLE}{ARG} ==
%   BEGIN
%    IF AFTERSKIP > 0
%      THEN \begingroup
%             STYLE
%             \@hangfrom{\hskip INDENT}{\interlinepenalty 10000 ARG\par}
%           \endgroup
%      ELSE \@svsechd == BEGIN STYLE
%                              \hskip INDENT
%                              ARG
%                        END
%    FI
%    \@xsect{AFTERSKIP}
%   END
% \end{oldcomments}
%
% Pseudocode for the |\@afterheading| command
% \begin{oldcomments}
% \@afterheading ==
%  BEGIN
%    @nobreak :=G true
%    \everypar := BEGIN  IF @nobreak = T
%                          THEN @nobreak  :=G false
%                               \clubpenalty :=G 10000
%                               IF @afterindent = F
%                                 THEN remove \lastbox
%                               FI
%                          ELSE \clubpenalty :=G \@clubpenalty
%                               \everypar := NULL
%                       FI
%                 END
%  END
% \end{oldcomments}
%
%
\begin{docCommand}{@ssect}{}

\startlineat{95}
       \begin{teX}
\def\@ssect#1#2#3#4#5{%
  \@tempskipa #3\relax
  \ifdim \@tempskipa>\z@
    \begingroup
       \end{teX}
% This |{| used to be after the argument to |\@hangfrom| but was moved
% here to allow commands such as |\MakeUppercase| to be used at the end
% of |#4|.
       \begin{teX}
      #4{%
        \@hangfrom{\hskip #1}%
          \interlinepenalty \@M #5\@@par}%
    \endgroup
  \else
    \def\@svsechd{#4{\hskip #1\relax #5}}%
  \fi
  \@xsect{#3}}
       \end{teX}
\end{docCommand}

\begin{docCommand}{if@afterindent}{}
\begin{docCommand}{@afterindenttrue}{}
The boolean \cmd{\@afterindent} is used to control the indentation after a sectioning command.

       \begin{teX}
\newif\if@afterindent \@afterindenttrue
       \end{teX}
\end{docCommand}\end{docCommand}
%
\begin{docCommand}{@afterheading}{}
  This hook is used in setting up custom-built headings in classes.dtx. See for example the book.dtx at
  \autoref{book:afterheading}.

\startlineat{108}
       \begin{teX}
\def\@afterheading{%
  \@nobreaktrue
  \everypar{%
    \if@nobreak
      \@nobreakfalse
      \clubpenalty \@M
      \if@afterindent \else
        {\setbox\z@\lastbox}%
      \fi
    \else
      \clubpenalty \@clubpenalty
      \everypar{}%
    \fi}}
       \end{teX}
\end{docCommand}
%
%
\begin{docCommand}{@hangfrom}{}

 |\@hangfrom{|\meta{text}|}| : Puts \meta{text} in a box, and makes a
 hanging indentation of the following material up to the first
 |\par|. Should be used in vertical mode.

       \begin{teX}
\def\@hangfrom#1{\setbox\@tempboxa\hbox{{#1}}%
      \hangindent \wd\@tempboxa\noindent\box\@tempboxa}
       \end{teX}
\end{docCommand}
%
\begin{docCommand}{c@secnumdepth}{}
\begin{docCommand}{c@tocdepth}{}

       \begin{teX}
\newcount\c@secnumdepth
\newcount\c@tocdepth
       \end{teX}
\end{docCommand}\end{docCommand}
%
\begin{docCommand}{secdef}{}
This is a user command to define star and unstarred commands without having to 
use the \refCom{@startsection}.

 |\secdef{|\meta{unstarcmds}|}{|\meta{unstarcmds}|}{|%
           \meta{starcmds}|}|\\
 When defining a |\chapter| or |\section| command without using
 |\@startsection|, you can use |\secdef| as follows:
 
 \begin{enumerate}
 \item |\def\chapter{| \ldots  |\secdef|
                |\|\meta{starcmd} |\|\meta{unstarcmd} |}|
 \item |\def\|\meta{starcmd}|[#1]#2{| \ldots |}|
            |%| Command to define |\chapter[|\ldots|]{|\ldots|}|
 \item |\def\|\meta{unstarcmd}|#1{| \ldots |}|
   |%| Command to define |\chapter*{|\ldots|}|
 \end{enumerate}

       \begin{teX}
\def\secdef#1#2{\@ifstar{#2}{\@dblarg{#1}}}
       \end{teX}
\end{docCommand}
%
 \subsubsection{Initializations}
\begin{docCommand}{sectionmark}{}
\begin{docCommand}{subsectionmark}{}
\begin{docCommand}{subsubsectionmark}{}
\end{docCommand}
\end{docCommand}
\end{docCommand}
\begin{docCommand}{paragraphmark}{}
\begin{docCommand}{subparagraphmark}{}

       \begin{teX}
\let\sectionmark\@gobble
\let\subsectionmark\@gobble
\let\subsubsectionmark\@gobble
\let\paragraphmark\@gobble
\let\subparagraphmark\@gobble
       \end{teX}
\end{docCommand}
\end{docCommand}

%
       \begin{teX}
\message{contents,}
       \end{teX}
%
 \subsection{Table of Contents etc.}

 \subsubsection{Convention}
% |\tf@|\meta{foo} = file number for output for table foo.
%       The file is opened only if |@filesw| = |true|.
%
 \subsubsection{Commands}


  A |\l@|\meta{type}|{|\meta{entry}|}{|\meta{page}|}| Macro needs to
  defined by document style for making an entry of type \meta{type}
  in a table of contents, etc.  E.g., the document style
  should define |\l@chapter|, |\l@section|, etc.

  \textbf{Note:} When the |\protect| command is
  used in the \meta{entry} or \meta{text} of one of the commands
  below, it causes the following control sequence to be written on
  the file without being expanded.  The sequence will be expanded
  when the table of contents entry is processed.

  \textbf{Surprise:} Inside an |\addcontentsline| or |\addtocontents|
  command argument, the commands: |\index|, |\glossary|,  and |\label|
  are  no-ops .  This could cause a problem if the user puts an
  |\index| or |\label| into one of the commands he writes, or into the
  optional `short version' argument of a |\section| or |\caption|
  command.

\begin{docCommand}{@starttoc}{}
 The |\@starttoc|\marg{ext} command is used to define the commands:\\
 |\tableofcontents|, |\listoffigures|, etc.

 For example:
 |\@starttoc{lof}| is used in |\listoffigures|.  This command
 reads the |.|\meta{ext} file and sets up to write the new
 |.|\meta{ext} file.

% \begin{oldcomments}
% \@starttoc{EXT} ==
%   BEGIN
%     \begingroup
%        \makeatletter
%        read file \jobname.EXT
%        IF @filesw = true
%          THEN  open \jobname.EXT as file \tf@EXT
%        FI
%        @nobreak :=G FALSE  %% added 24 May 89
%     \endgroup
%   END
% \end{oldcomments}

       \begin{teX}
\def\@starttoc#1{%
  \begingroup
    \makeatletter
    \@input{\jobname.#1}%
    \if@filesw
      \expandafter\newwrite\csname tf@#1\endcsname
      \immediate\openout \csname tf@#1\endcsname \jobname.#1\relax
    \fi
    \@nobreakfalse
  \endgroup}
       \end{teX}
\end{docCommand}

  \begin{docCommand}{addcontentsline}{}
  The |\addcontentsline{|\meta{table}|}{|\meta{type}|}{|%
  \meta{entry}|}| command allows the user to  add
  his/her own entry to a table of contents, etc. The command adds the
  entry |\contentsline{|\meta{type}|}{|\meta{entry}|}{|\meta{page}|}|
  to the |.|\meta{table} file.

  This macro is implemented as an application of |\addtocontents|.
  Note that |\thepage| is not expandable during |\protected@write|
  therefore one gets the page number at the time of the |\shipout|.

       \begin{teX}
\def\addcontentsline#1#2#3{%
  \addtocontents{#1}{\protect\contentsline{#2}{#3}{\thepage}}}
       \end{teX}
  \end{docCommand}

\begin{docCommand}{addtocontents}{}

   The |\addtocontents{|\meta{table}|}{|\meta{text}|}| command
   adds \meta{text} to the |.|\meta{table} file, with no
   page number.

       \begin{teX}
\long\def\addtocontents#1#2{%
  \protected@write\@auxout
      {\let\label\@gobble \let\index\@gobble \let\glossary\@gobble}%
      {\string\@writefile{#1}{#2}}}
       \end{teX}
\end{docCommand}
%
\begin{docCommand}{contentsline}{}
% The |\contentsline{|\meta{type}|}{|\meta{entry}|}{|\meta{page}|}|
% macro produces a \meta{type} entry in a table of contents, etc.
% It will appear in the |.toc| or other file.  For example,
% The entry for subsection 1.4.3 in the table of contents for
% example, might be produced by:
%
%  \begin{verbatim}
%       \contentsline{subsection}
%           {\makebox{30pt}[r]{1.4.3} Gnats and Gnus}{22}
%  \end{verbatim}
%
%  The |\protect| command causes command sequences to be written
%  without expanding them.
%
       \begin{teX}
\def\contentsline#1{\csname l@#1\endcsname}
       \end{teX}
\end{docCommand}

\begin{docCommand}{@dottedtocline}{\meta{level}\meta{indent}\meta{numwidth}}%
        |\meta{title}\meta{page}|:
       
   Macro to produce a table of contents line with the following
   parameters:
\end{docCommand}
   
   \begin{marglist}
   \item[level] If \meta{level} $>$ |\c@tocdepth|, then no line
                produced.
   \item[indent] Total indentation from the left margin.
   \item[numwidth] Width of box for number if the \meta{title} has a
                |\numberline| command.
                As of 25 Jan 1988, this is also the amount of extra
                indentation added to second and later lines of a
                multiple line entry.
   \item[title] Contents of entry.
   \item[page] Page number.
  \end{marglist}

  Uses the following parameters, which must be set by the document
  style. They should be defined with |\def|'s.
  
  \begin{marglist}
  \item[\cs{@pnumwidth}]  Width of box in which page number is set.
  \item[\cs{@tocrmarg}] Right margin indentation for all but last line
        of multiple-line entries.
  \item[\texttt\cs{@dotsep}] Separation between dots, in mu units.
                  Should be |\def|'d to a number like 2 or 1.7
  \end{marglist}
%
\begin{docCommand}{@dottedtocline}{}

       \begin{teX}
\def\@dottedtocline#1#2#3#4#5{%
  \ifnum #1>\c@tocdepth \else
    \vskip \z@ \@plus.2\p@
    {\leftskip #2\relax \rightskip \@tocrmarg \parfillskip -\rightskip
     \parindent #2\relax\@afterindenttrue
     \interlinepenalty\@M
     \leavevmode
     \@tempdima #3\relax
       \end{teX}
% \changes{v1.0z}{1996/12/20}{Added \cs{nobreak} for latex/2343}
       \begin{teX}
     \advance\leftskip \@tempdima \null\nobreak\hskip -\leftskip
     {#4}\nobreak
     \leaders\hbox{$\m@th
       \end{teX}
    If a document uses fonts other than computer modern, the use of a
    dot from math can be very disturbing despite the fact that this
    might be the only place in a document that then uses computer
    modern.
    Therefore we surround the dot with an |\hbox| to escape to the
    surrounding text font.
% \changes{v1.0k}{1995/04/25}{Added \cs{hbox} around dots.}
% \changes{v1.0l}{1995/05/02}{Don't reset to \cs{rmfamily}}
       \begin{teX}
        \mkern \@dotsep mu\hbox{.}\mkern \@dotsep
        mu$}\hfill
     \nobreak
     \hb@xt@\@pnumwidth{\hfil\normalfont \normalcolor #5}%
     \par}%
  \fi}
       \end{teX}
\end{docCommand}
%
% \textbf{Note:} |\nobreak|'s added 7 Jan 86 to prevent bad line break
% that left the page number dangling by itself at left edge of a new
% line.
%
% Changed 25 Jan 88 to use |\leftskip| instead of |\hangindent| so
% leaders of multiple-line contents entries would line up properly.
\begin{docCommand}{numberline}{\meta{number}}
 For use in a |\contentsline| command.
   It puts \meta{number} flushleft in a box of width |\@tempdima|
       \begin{teX}
\def\numberline#1{\hb@xt@\@tempdima{#1\hfil}}
%</2ekernel>
       \end{teX}
\end{docCommand}
%

\section{Packages and developments}

Many package authors have redefined the kernel commands to add hooks or styles to the definitions in order to make them more flexible. The \pkgname{tocstyle} \cite{tocstyle} which is still in alpha provides redefinitions so that one can specify toc styles using akey value system. It also provides a mechanism for the rdefinition of
styles. Our own \pkgname{phd} provides a flexible sytem with all the parameters provided as key value pairs based on pgf keys. By far the most popular package is Javier Bezos \pkgname{titlesec}. This forms part of a suite of
packages \pkgname{titlesec}, \pkgname{titleps} and \pkgname{titletoc}. Bezos used a different philosophy to that
used in the kernel and provides more generic commands.

\begin{verbatim}
\titleformat{\section}[runin]
{\normalfont\bfseries}
{\S\ \thesection.}{.5em}{}[.---]
\titlespacing{\section}
{\parindent}{1.5ex plus .1ex minus .2ex}{0pt}
\end{verbatim}







