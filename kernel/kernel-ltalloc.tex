% \iffalse meta-comment
%
% Copyright 1993-2016
% The LaTeX3 Project and any individual authors listed elsewhere
% in this file.
%
% This file is part of the LaTeX base system.
% -------------------------------------------
%
% It may be distributed and/or modified under the
% conditions of the LaTeX Project Public License, either version 1.3c
% of this license or (at your option) any later version.
% The latest version of this license is in
%    http://www.latex-project.org/lppl.txt
% and version 1.3c or later is part of all distributions of LaTeX
% version 2005/12/01 or later.
%
% This file has the LPPL maintenance status "maintained".
%
% The list of all files belonging to the LaTeX base distribution is
% given in the file `manifest.txt'. See also `legal.txt' for additional
% information.
%
% The list of derived (unpacked) files belonging to the distribution
% and covered by LPPL is defined by the unpacking scripts (with
% extension .ins) which are part of the distribution.
%
% \fi
%
% \iffalse
%%% From File: ltalloc.dtx
%<*driver>
% \fi
%\ProvidesFile{ltalloc.dtx}
%             [1996/07/26 v1.1c LaTeX Kernel (allocation)]
% \iffalse
%\documentclass{ltxdoc}
%\GetFileInfo{ltalloc.dtx}
%\title{\filename}
%\date{\filedate}
% \author{%
%  Johannes Braams\and
%  David Carlisle\and
%  Alan Jeffrey\and
%  Leslie Lamport\and
%  Frank Mittelbach\and
%  Chris Rowley\and
%  Rainer Sch\"opf}
%\begin{document}
%\MaintainedByLaTeXTeam{latex}
%\maketitle
% \DocInput{\filename}
%\end{document}
%</driver>
% \fi
%
%
% \changes{v1.1a}{1994/05/16}{(ASAJ) Split from ltinit.dtx.}
% \changes{v1.1b}{1995/10/25}{General doc improvements}
%
\chapter{Allocations}
\label{kernel:ltalloc}
 
 \section{Counters}
%
 This section deals with counter and other variable allocation.


 \begin{teX}
%<*2ekernel>
 \end{teX}
%
 The following are from plain \TeX:
 \begin{description}
 \item[\cs{z@}] A zero dimen or number.  It's more efficient to write
               |\parindent\z@| than |\parindent 0pt|.
 \item[\cs{@ne}]         The number 1.
 \item[\cs{m@ne}]        The number $-1$.
 \item[\cs{tw@}]         The number 2.
 \item[\cs{sixt@@n }]    The number 16.
 \item[\cs{@m}]          The number 1000.
 \item[\cs{@MM}]         The number 20000.
 \end{description}
%
 \begin{macro}{\@xxxii}
% The constant $32$.
 \begin{teX}
\chardef\@xxxii=32
 \end{teX}
 \end{macro}
%
 \begin{macro}{\@Mi}
 \begin{macro}{\@Mii}
 \begin{macro}{\@Miii}
 \begin{macro}{\@miv}
% Constants $1001$--$1004$.
    \begin{teX}
\mathchardef\@Mi=10001
\mathchardef\@Mii=10002
\mathchardef\@Miii=10003
\mathchardef\@Miv=10004
    \end{teX}
 \end{macro}
 \end{macro}
 \end{macro}
 \end{macro}
%
% \changes{v1.0d}{1994/03/28}
%     {Redefinition of `new' allocations removed.}
%
 \begin{macro}{\@tempcnta}
 \begin{macro}{\@tempcntb}
 Scratch count registers used by \LaTeX\ kernel commands.
 \begin{teX}
\newcount\@tempcnta
\newcount\@tempcntb
 \end{teX}
 \end{macro}
 \end{macro}

 \begin{macro}{\if@tempswa}
 General boolean switch used by \LaTeX\ kernel commands. To remmeber the name think of temporary switch `a'. This
 is used extensively when writing to auxiliary, toc and other list files. 
 \begin{teX}
\newif\if@tempswa
 \end{teX}
 \end{macro}

 \begin{macro}{\@tempdima}
 \begin{macro}{\@tempdimb}
 \begin{macro}{\@tempdimc}
 Scratch dimen registers used by \LaTeX\ kernel commands.
 \begin{teX}
\newdimen\@tempdima
\newdimen\@tempdimb
\newdimen\@tempdimc
 \end{teX}
 \end{macro}
 \end{macro}
 \end{macro}

 \begin{macro}{\@tempboxa}
 Scratch box register used by \LaTeX\ kernel commands.
 \begin{teX}
\newbox\@tempboxa
 \end{teX}
 \end{macro}

 \begin{macro}{\@tempskipa}
 \begin{macro}{\@tempskipb}
 Scratch skip registers used by \LaTeX\ kernel commands.
 \begin{teX}
\newskip\@tempskipa
\newskip\@tempskipb
 \end{teX}
 \end{macro}
 \end{macro}

 \begin{macro}{\@temptokena}
 Scratch token register used by \LaTeX\ kernel commands.
 \begin{teX}
\newtoks\@temptokena
 \end{teX}
 \end{macro}


 \begin{macro}{\@flushglue}
 Glue used for |\right|- \& |\leftskip|  = 0pt plus 1fil
 \begin{teX}
\newskip\@flushglue \@flushglue = 0pt plus 1fil
 \end{teX}
 \end{macro}
%
 \begin{teX}
%</2ekernel>
 \end{teX}
%
% \Finale
\endinput