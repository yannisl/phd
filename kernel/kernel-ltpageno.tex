% \iffalse meta-comment
%
% Copyright 1993-2016
% The LaTeX3 Project and any individual authors listed elsewhere
% in this file.
%
% This file is part of the LaTeX base system.
% -------------------------------------------
%
% It may be distributed and/or modified under the
% conditions of the LaTeX Project Public License, either version 1.3c
% of this license or (at your option) any later version.
% The latest version of this license is in
%    http://www.latex-project.org/lppl.txt
% and version 1.3c or later is part of all distributions of LaTeX
% version 2005/12/01 or later.
%
% This file has the LPPL maintenance status "maintained".
%
% The list of all files belonging to the LaTeX base distribution is
% given in the file `manifest.txt'. See also `legal.txt' for additional
% information.
%
% The list of derived (unpacked) files belonging to the distribution
% and covered by LPPL is defined by the unpacking scripts (with
% extension .ins) which are part of the distribution.
%
% \fi
%
% \iffalse
%%% From File: ltpageno.dtx
%
%<*driver>
% \fi
%\ProvidesFile{ltpageno.dtx}
%             [1994/05/19 v1.1a LaTeX Kernel (Page Numbering)]
%% \iffalse
%\documentclass{ltxdoc}
%\GetFileInfo{ltpageno.dtx}
%\title{\filename}
%\date{\filedate}
% \author{%
%  Johannes Braams\and
%  David Carlisle\and
%  Alan Jeffrey\and
%  Leslie Lamport\and
%  Frank Mittelbach\and
%  Chris Rowley\and
%  Rainer Sch\"opf}
%
%\begin{document}
% \MaintainedByLaTeXTeam{latex}
% \maketitle
% \DocInput{\filename}
%\end{document}
%</driver>
% \fi
%
%
% \changes{v1.0c}{1994/03/29}
%     {Create file ltcntlen from parts of ltmiscen and ltherest.}
% \changes{v1.1a}{1994/05/19}
%     {Extract file ltpageno from ltcntlen.}
%
\chapter{Page Number}
\label{kernel:ltpageno}
\section{Page Numbering}

It is rather strange that this short piece of code is in its own dtx file. One would
have expected it to perhaps be found in the file for counters. 
As expected page numbers are produced by a page counter, used just like any other
 counter.  The only difference is that |\c@page| contains the number of
 the next page to be output (the one currently being produced), rather
 than one minus it.  Thus, it is normally initialized to~1 rather
 than~0.  |\c@page| is defined to be |\count0|, rather than a count
 assigned by |\newcount|.

\refCom{pagenumbering}  The user sets the pagenumber style with the \cs{pagenumbering}\marg{foo}
 command, which sets the page counter to 1 and defines |\thepage| to be
 |\foo|.  For example, |\pagenumbering{roman}| causes pages to be
 numbered  i, ii, etc.


% 
%
\begin{teX}
%<*2ekernel>
\message{page nos.,}
\end{teX}
%

The \docAuxCommand{cl@page} is set to empty. This is a list of counters to be reset, when the page
is reset.  If we had used the |\newcounter| control sequence from the Chapter on counters (see page \pageref{kernel:ltcounts}) this would have been created automatically. 

\begin{teXXX}
\countdef\c@page=0 \c@page=1
\def\cl@page{}
\end{teXXX}

\begin{docCommand}{pagenumbering}{\marg{name}}
Sets the page numbering style. This will be called later in the output routine, when \latexe is building up the
page and getting it ready to be typeset.
\end{docCommand}
\begin{teXXX}
\def\pagenumbering#1{%
  \global\c@page \@ne \gdef\thepage{\csname @#1\endcsname
   \c@page}}
\end{teXXX}
%
\begin{teX}
%</2ekernel>
\end{teX}
%
\section{Modifications and additions by packages}

The package \pkg{perpage} developed by David Kastrup  adds the ability to reset counters per page and/or keep their
occurences sorted in order of appearance on the page. 
It works by attaching itself to the code for \refCom{stepcounter} and will then modify
the given counter according to information written to the |.aux| file, which means
that multiple passes may be needed. Since it uses the internals of the \label
mechanism, the need for additional passes will get announced by \latexe as ``labels
may have changed''.

The package provides commands to get the absolute number of a counter (even if it is reset) by creating a command |c@abs|\meta{counter}. For example if we call, |AddAbsoluteCounter{page}| it will create the counter |c@abspage|, which can later be used as we require. This can be used for example to get the total pages of a document rather than the actual. The package \pkg{babel} might also interfere with counters. 

% \Finale
%
