\cxset{fashion image=fashion-03.jpg,
       palette rouge}
       
\chapter{ltoutput.dtx}

\label{kernel:ltoutput}

\section{Introduction}

 In Chapter \ref{ch:OTR}, we described the mechanics of output routines both
 as found in Plain \tex and in \LaTeXe. This is
 a longer treatise of the subject and includes commentary on the
 actual listing as found in \LaTeXe. The Output Routine (OR) or (OTR) as is sometimes denoted in the literature, is the procedure
by which \LaTeXe\ assembles the material that makes a page by combining
text and floats, adding any other inserts such as footnotes, headers and
footers and then ships out the page to produce a |.dvi| or with some \tex engines to
be translated straight into |.pdf| output. It is a very complex process
as it has to keep a lot of different material in different lists and boxes.

The output routine as defined in the kernel covers a lot of functionality.

\begin{enumerate}
\item Defines page geometry parameters.
\item Positions floats.
\item Adds headers and footers.
\item Adds hooks.
\end{enumerate}


 \section{Floats}

The interesting and complicated part of the OR is its algorithm for handling floats.  The floating environments are defined by the standard classes. For example the |book| class defines both
the figure as well as the table environments. It is instructive to start the discussion of the algorithm from this point. 

The class defines the default float placement specifier using 
|\fps@figure| and then goes on to define the figure environment with the help of \refCom{@float} and \refCom{end@float}, which are defined in the generic \docFile{float.dtx} in the kernel.


\begin{teX}
\def\ftype@figure{1}
\def\ext@figure{lof}
\def\fnum@figure{\figurename\nobreakspace\thefigure}
\newenvironment{figure}
               {\@float{figure}}
               {\end@float}
\newenvironment{figure*}
               {\@dblfloat{figure}}
               {\end@dblfloat}
\end{teX}


A float specifier is made of two parts the float type, which is a power of two--e.g., figures in the case of the book class are type 1 and tables type 2 and the \textit{placement specification} describing where the float can be placed.  The type is defined in powers of two due to the way the specifiers are represented using a binary representation internally.

\ExplSyntaxOn
\int_to_bin:n {50}\\
\int_from_bin:n {110010}
\ExplSyntaxOff

\begin{figure}[h]
\centering
\begin{tabular}{rl}
{\fboxsep2pt   \fbox{1} \fbox{1} \fbox{0} \fbox{0} \fbox{1} \fbox{0}} &= 50\\
{\fboxsep2pt \fbox{1} \fbox{0} \fbox{1} \fbox{0} \fbox{0} \fbox{1} \fbox{0}} &= 82
\end{tabular}

\caption{Binary representation of float specifiers. Top is for figure with [t] option and the bottom is for table with the [t] option.}
\end{figure}

So a new float will need to be shifted in powers of two. The kernel defines routines to check
for various combinations. This has simplified the programming although it may not be easy to
follow. The float specifier is encoded as follows, where bit 0 is the least
 significant bit. 

\begin{table}[h]
\begin{tabular}{ll}
\toprule
  Bit    & Meaning\\
\midrule
   0     & 1 if the float may go where it appears in the text.\\
   1     & 1 if the float may go on the top of a page.\\
   2     & 1 if the float may go on the bottom of a page.\\
   3     & 1 if the float may go on a float page.\\
   4     & 1 unless the \textit{placement} includes a !\\
   5     & 1 if a type 1 float\\
   6     & 1 if a type 2 float
          etc.\\
\bottomrule
\end{tabular}
\end{table}

If a number is odd  denotes  a here placement specification [h] and if it is negative a marginpar. 
Since TeX's number limit of $ 2^{31}-1$ and the first 5 bits are taken by the float identifiers there remain 26 available float types for the adventurous. \url{http://tex.stackexchange.com/questions/32359/what-is-the-exact-purpose-of-ftypetype}.

We have seen so far how a float is defined in the standard classes and what type of parameters are coded. Once \LaTeXe\  sees the environment it will execute the macro |\@float| which is defined in the |float.dtx|. 

Once in |\@float| the bits are set as well as the relevant penalties. The penalties are distinct in order to signal to the output routines the type of float.

One needs to remember that the floats are placed based on constraints and page sizing parameters.


\section{Page Layout Parameters}

\begin{longtable}[l]{p{4.5cm}p{8.5cm}}
   |\topmargin|      & Extra space added to top of page.\\
   |\@twoside|       & boolean.  T if two-sided printing\\
   |\oddsidemargin|  & IF @twoside = T
                       THEN extra space added to left of odd-numbered
                            pages.
                       ELSE extra space added to left of all pages.\\
   |\evensidemargin| & IF @twoside = T
                       THEN extra space added to left of even-numbered
                            pages.\\
   |\headheight|     & height of head\\
   |\headsep|        & separation between head and text\\
   |\footskip|         & distance separation between baseline of last
                     line of text and baseline of foot.
                     Note difference between |\footSKIP| and |\headSEP|.\\

   |\textheight|     & height of text on page, excluding head and foot\\

   |\textwidth|      & width of printing on page\\
   |\columnsep|      & IF @twocolumn = T
                       THEN width of space between columns\\

   |\columnseprule|  & IF @twocolumn = T
                       THEN width of rule between columns (0 if none).\\

   |\columnwidth|    & IF @twocolumn = T
                       THEN |(\textwidth - \columnsep)|/2
                       ELSE |\textwidth|
                     It is set by the |\twocolumn| and
                     |\onecolumn| commands.\\

   |\@textbottom|    & Command executed at bottom of vbox holding text of
                     page (including figures).  The |\raggedbottom|
                     command almost |\let|'s this to |\vfil| (actually sets
                     it to |\vskip \z@| plus.0001fil).
                     Should have depth 0pt.\\

   |\@texttop|       & Command executed at top of vbox holding text of
                     page (including figures).  Used by letter style;
                     can also be used to produce centered pages.
                     Let to |\relax| by |\raggedbottom| and |\flushbottom|.\\
\end{longtable}


Page layout must initialize |\@colht| and |\@colroom| to |\textheight|.

\section{Page Style Parameters}

\begin{longtable}{p{3.5cm}p{7.5cm}}
   |\floatsep|       & Space left between floats.\\
   |\textfloatsep|   & Space between last top float or first bottom float
                     and the text.\\
   |\topfigrule|     & Command to place rule (or whatever) between floats
                     at top of page and text.  Executed in inner
                     vertical mode right before the |\textfloatsep| skip
                     separating the floats from the text.  Must occupy
                     zero vertical space.  (See |\footnoterule|.)\\
   |\botfigrule|     & Same as |\topfigrule|, but put after the
                     |\textfloatsep| skip separating text from the
                     floats at bottom of page.\\
   |\intextsep|      & Space left on top and bottom of an in-text float.\\
   |\dblfloatsep|    & Space between double-column floats.\\
   |\dbltextfloatsep| & Space between top double-column floats
                      and text.\\
   |\dblfigrule|     & Similar to |\topfigrule|, but for double-column
                     floats.\\
   |\@fptop|         & Glue to go at top of float column -- must be 0pt +
                     stretch\\
   |\@fpsep|         & Glue to go between floats in a float column.\\
   |\@fpbot|         & Glue to go at bottom of float column
                       -- must be 0pt +
                     stretch\\
   |\@dblfptop|, |\@dblfpsep|, |\@dblfpbot|
                   & Analogous for double-column float page in
                     two-column format.\\

   |@twocolumn|      & Boolean.  T if two columns per page globally.\\


   |\@oddhead|        & IF @twoside = T
                           THEN macro to generate head of odd-numbered
                                pages.
                           ELSE macro to generate head of all pages.\\
   |\@evenhead|      & IF @twoside = T
                           THEN macro to generate head of even-numbered
                                pages.\\
   |\@oddfoot|        & IF @twoside = T
                           THEN macro to generate foot of odd-numbered
                                pages.
                           ELSE macro to generate foot of all pages.\\
   |\@evenfoot|       & IF @twoside = T
                           THEN macro to generate foot of even-numbered
                                pages.\\
   |@specialpage|     & boolean.  T if current page is to have a special
                               format.\\
  |\@specialstyle|   & If its value is  foo then
                     IF @specialpage = T
                       THEN the command |\ps@foo| is executed to
                            temporarily reset the page style parameters
                            before composing the current page.
                            This command should execute only |\def|'s and
                            |\edef|'s, making only local definitions.\\
\end{longtable}



\section{Float placement parameters}

 The following parameters are set by the macro |\@floatplacement|.
 When |\@floatplacement| is called,
 |\@colht| is the height of the page or column being built.  I.e.:

         * For single-column page it equals |\textheight|.\\
         * For double-column page it equals |\textheight| - height
           of double-column floats on page.

 Note that some are set globally and some locally:

\begin{description}

  \item[\cs{@topnum}] = G Maximum number of floats allowed on the top of a
                  column.
  \item [\cs{@toproom}] :=G Maximum amount of top of column devoted to floats--
                  excluding |\textfloatsep| separation below the floats
                  and |\floatsep| separation between them.  For
                  two-column output, should be computed as a function
                  of |\@colht|.
\end{description}


\begin{longtable}{p{4.5cm}p{5.5cm}}
    |\@botnum|, |\@botroom|
                & Analogous to above.\\
    |\@colnum|  & G Maximum number of floats allowed in a column,
                  including in-text floats.\\
    |\@textmin| & L Minimum amount of text (excluding footnotes) that
                  must appear on a text page.
                  It is used locally in processing double
                  floats.\\
    |\@fpmin|   & L Minimum height of floats in a float column.\\
\end{longtable}


 The macro \cs{@dblfloatplacement} sets the following parameters.

\begin{tabular}{p{3.5cm}p{6cm}}
    |\@dbltopnum|  &G Maximum number of double-column floats allowed at
                     the top of a two-column page.\\
    |\@dbltoproom|  &G Maximum height of double-column floats allowed at
                     top of two-column page.\\
    |\@fpmin|      &L Minimum height of floats in a float column.\\
\end{tabular}

 It should also perform the following local assignments where necessary
 -- i.e., where the new value differs from the old one:

\begin{tabular}{p{3.5cm}p{6cm}}
     |\@fptop|    & L |\@dblfptop|\\
     |\@fpsep|      & L |\@dblfpsep|\\
     |\@fpbot|      &L |\@dblfpbot|\\
\end{tabular}


\section{Output Routine Variables}


\begin{tabular}{p{3.5cm}p{6cm}}
  |\@colht| & The total height of the current column.  In single column
            style, it equals |\textheight|.  In two-column style, it is
            |\textheight| minus the height of the double-column floats
            on the current page.  MUST BE INITIALIZED TO |\textheight|.\\

  |\@colroom| & The height available in the current column for text and
              footnotes.  It equals |\@colht| minus the height of all
              floats committed to the top and bottom of the current
              column.\\

  |\@textfloatsheight| & The total height of in-text floats on the
                       current page.\\

  |\footins| & Footnote insertion number.\\

  |\@maxdepth| & Saved value of TeX's |\maxdepth|.  Must be set
               when any routine sets |\maxdepth|.\\
\end{tabular}



\section{Calling the output routine}

 The output routine is called either by TeX's normal page-breaking
 mechanism, or by a macro putting a penalty \(\le -10000\) in the output
 list.  In the latter case, the penalty indicates why the output
 routine was called, using the following code.

\begin{table}[ht]
\centering
\begin{tabular}{lp{6cm}}
\toprule
   Penalty   & Reason\\
\midrule
   -10000    & \cs{pagebreak}\\
             & \cs{newpage}\\
   -10001    & \cs{clearpage} (\cs{penalty} -10000 \cs{vbox}|{}| \cs{penalty} -10001)\\
   -10002    & float insertion, called from horizontal mode\\
   -10003    & float insertion, called from vertical mode.\\
   -10004    & float insertion.\\
\bottomrule
\end{tabular}
\caption{Penalties when calling the output routine.}
\end{table}
Note that a float or marginpar puts the following sequence in the output
list:
\begin{enumerate} 
  \item a penalty of -10004,
  \item a null |\vbox|
  \item a penalty of -10002 or -10003.
\end{enumerate}

This solves two special problems:

\begin{enumerate}
  \item If the float comes right after a \cs{newpage} or \cs{clearpage},
        then the first penalty is ignored, but the second one
       invokes the output routine.
 \item If there is a split footnote on the page, the second 'page'
       puts out the rest of the footnote.
\end{enumerate}
             

            
\section{Functions used in the output routine}

\begin{docCommand}{@outputpage}{}
 \cs{@outputpage} : Produces an output page with the contents of box
              |\@outputbox| as the text part.
              Also sets |\@colht| :=G |\textheight|.
              The page style is determined as follows:
              
              \begin{algorithm}[H]
               \Begin{
                \If{\cs{@thispagestyle} = true}{
                   use \cs{thispagestyle} style}{
                   use ordinary page style.}}
              \end{algorithm}
\end{docCommand}


\begin{docCommand}{@tryfcolumn}{}
\begin{description}
 \item[\cs{@tryfcolumn}\cs{FLIST}]  tries to form a float column composed of floats
         from |\FLIST| (if nonempty) with the following parameters see \autoref{tryfcolumn}:
 
           \begin{tabular}{p{3cm}l}
                |\@colht|  & height of box\\
                |\@fpmin| & minimum height of floats in the box\\
                |\@fpsep|  & interfloat space\\
                |\@fptop | & glue at top of box\\
                |\@fpbot | & glue at bottom of box.\\
          \end{tabular}


  If it succeeds, then it does the following:

         \begin{tabular}{p{3cm}l}
                 |\@outputbox| & L the composed float box.\\
                 |@fcolmade|   & G true\\
                 |\FLIST|          & G |\FLIST| - floats put in box\\
                 |\@freelist|     & G |\@freelist| + floats put in box\\
         \end{tabular}

              If it fails, then:

         \begin{tabular}{ll}
                |@fcolmade| & G false\\
        \end{tabular}
           NOTE: BIT MUST BE A SINGLE TOKEN!
\end{description}
\end{docCommand}




\begin{docCommand}{@makefcolumn}{}
 |\@makefcolumn \FLIST| is similar to |\@tryfcolumn| except that it
             fails to make a float column only if |\FLIST| is empty.
             Otherwise, it makes a float column containing at least
             the first box in |\FLIST|, disregarding |\@fpmin|.
\end{docCommand}

\begin{docCommand}{@startcolumn}{}

\begin{description}
\item[ \cs{@startcolumn} ]
       Calls |\@tryfcolumn\@deferlist|.  If |\@tryfcolumn| returns with
       (globally set) @fcolmade = false, then:

\item           Globally sets |\@toplist| and |\@botlist| to floats
                  from |\@deferlist| to go at top and bottom of column,
                  deleting them from |\@deferlist|.  It does
                  this using |\@colht| as the total height, the page
                  style parameters |\@floatsep| and |\@textfloatsep|, and
                  the float placement parameters |\@topnum|, |\@toproom|,
                  |\@botnum|, |\@botroom|, |\@colnum| and |\textfraction|.

\item          Globally sets |\@colroom| to |\@colht| minus the height
                  of the added floats.
\end{description}
\end{docCommand}




\begin{docCommand}{@startdblcolumn }{}

      Calls |\@tryfcolumn\@dbldeferlist{8}|.  If |\@tryfcolumn| returns
      with (globally set) @fcolmade = false, then:

               * Globally sets |\@dbltoplist| to floats from
                 |\@dbldeferlist| to go at top and bottom of column,
                 deleting them from |\@dbldeferlist|.
                 It does this using |\textheight| as the
                 total height, and the parameters |\@dblfloatsep|, etc.

               * Globally sets |\@colht| to |\textheight| minus the height
                 of the added floats.

\end{docCommand}

\begin{docCommand}{@combinefloats}{}
 \cs{@combinefloats} Combines the text from box
          \cs{@outputbox} with the floats from \cs{@toplist} and \cs{@botlist},
          putting the new box in \cs{@outputbox}.  It uses \cs{floatsep}
          and \cs{textfloatsep} for the appropriate separations.
          It puts the elements of \cs{TOPLIST} and \cs{BOTLIST} onto
          \cs{@freelist}, and makes those lists null.

\end{docCommand}

\begin{docCommand}{@makecol}{}
|\@makecol| Makes the contents of |\box255| plus the accumulated
              footnotes, plus the floats in |\@toplist| and |\@botlist|,
              into a single column of height |\@colht| (unless the page
              height has been locally changed), which it puts
              into box |\@outputbox|.  It puts boxes in |\@midlist| back
              onto |\@freelist| and restores |\maxdepth|.
\end{docCommand}



\begin{docCommand}{@opcol}{}
 \cs{@opcol} Outputs a column whose text is in box \cs{@outputbox}
\end{docCommand}
%
%\begin{algorithm}
% \cs{@opcol}==\Begin{%    
%\eIf{@twocolumn = false}{\cs{@outputpage}\\
%  \cs{@colht} :=G \cs{textheight}\\
%  \cs{@floatplacement}}{\eIf{@firstcolumn = true}{puts box \cs{@outputbox}
%      into \cs{@leftcolumn}\\
%      @firstcolumn :=G false.}{puts out the current two-column page\\
%      any possible two-column float pages,\\
%      determine \cs{@dbltoplist} for the next page.}}
%}
%\end{algorithm}



\section{User commands  that call  affect the output routine}

\begin{docCommand}{newpage}{}
 \begin{teX}
 \newpage == BEGIN \par\vfil\penalty -10000 END
 \end{teX}
\end{docCommand}

\begin{docCommand}{clearpage}{}
\begin{verbatim}
              == BEGIN \newpage
                     \write -1{}    % Part of hack to make sure no
                     \vbox{}        % \write's get lost.
                     \penalty -10001
               END
\end{verbatim}
\begin{verbatim}
 \cleardoublepage == BEGIN \clearpage
                           if @twoside = true and c@page is even
                             then \hbox{} \newpage fi
                     END

  
 \twocolumn[BOX] : starts a new page, changing to twocolumn setting
     and puts BOX in a parbox of width \textwidth across the top.
     Useful for full-width titles for double-column pages.
     SURPRISE: The stretch from \@dbltextfloatsep will be inserted
               between the BOX and the top of the two columns.
\end{verbatim}
\end{docCommand}


\section{Float-handling mechanisms}

 The float environment obtains an insertion number B from the
 |\@freelist| (see below for a description of list manipulation), puts
 the float into box B and sets |\count| B to a FLOAT SPECIFIER.  For
 a normal (not double-column) float, it then causes a page break
 in one of the following two ways:

   - In outer hmode: |\vadjust{\penalty -10002}|
   - In vmode :      |\penalty -10003.|

 For a double-column float, it puts B onto the |\@dbldeferlist|.


 The float specifier has two components:

    * A PLACEMENT SPECIFICATION, describing where the float may
      be placed.

    * A TYPE, which is a power of two--e.g., figures might be
      type 1 floats, tables type 2 floats, programs type 4 floats, etc.

 The float specifier is encoded as follows, where bit 0 is the least
 significant bit.
\medskip


\begin{tabular}{ll}
\toprule
  Bit    & Meaning\\
\midrule
   0     & 1 if the float may go where it appears in the text.\\
   1     & 1 if the float may go on the top of a page.\\
   2     & 1 if the float may go on the bottom of a page.\\
   3     & 1 if the float may go on a float page.\\
   4     & 1 unless the PLACEMENT incluses a !\\
   5     &  if a type 1 float\\
   6     & 1 if a type 2 float
          etc.\\
\bottomrule
\end{tabular}
\medskip

A negative float specifier is used to indicate a marginal note.



\section{Macros and data structures for processing floats}

  A \textit{float list} consisting of the floats in boxes |\boxa ... \boxN| has
  the form:

  \begin{verbatim}
         \@elt \boxa ... \@elt \boxN
  \end{verbatim}

  where  |\boxI| is defined by:

  \begin{verbatim}
         \newinsert\boxI
  \end{verbatim}

  Normally, |\@elt| is |\let| to |\relax|.  A test can be performed on the
  entire float list by locally |\def|'ing |\@elt| appropriately and
  executing the list.

  This is a lot more efficient than looping through the list.

  The following macros are used for manipulating float lists. Of interest here---and a bit
  difficult to follow is |\@next| see \ref{next}.

  \begin{verbatim}
  \@next \CS \LIST {NONEMPTY}{EMPTY} ==  %% NOTE: ASSUME \@elt = \relax
    BEGIN  assume that \LIST == \@elt \B1 ... \@elt \Bn
           if n = 0
             then  EMPTY
             else   \CS    := L \B1
                     \LIST  := G \@elt \B2 ... \@elt \Bn
                   NONEMPTY
           fi
    END
  \end{verbatim}



\begin{docCommand}{@bitor}{\Arg{num}\Arg{list}}
  \cs{@bitor}|\NUM\LIST|  Globally sets switch |@test| to the disjunction for
         all I of bit  log2 |\NUM| of the float specifiers of all the
         floats in |\LIST|.

         I.e., @test is set to true iff there is at least one
         float in |\LIST| having bit  log2 |\NUM|  of its float specifier
         equal to 1.
\end{docCommand}
\begin{verbatim}
%  Note: log2 [(\count I)/32] is the bit number corresponding to the
%  type of float I.  To see if there is any float in \LIST having
%  the same type as float I, you run \@bitor with
%
%    \NUM = [(\count I)/32] * 32.
%
% \@bitor\NUM\LIST ==
%   BEGIN
%      @test :=G false
%      { \@elt \CTR ==  if \NUM <> 0 then
%                          if \count\CTR / \NUM is odd
%                             then  @test := true       fi fi
%        \LIST
%      }
%   END
%
%
% \@cons\LIST\NUM : Globally sets \LIST := \LIST * \@elt \NUM
%
% \@cons\LIST\NUM ==
%   BEGIN {  \@elt == \relax
%            \LIST :=G \LIST \@elt \NUM
%         }

\end{verbatim}



\section{Box lists for float-placement algorithms}

The \pkg{fixltx2e} the now redundant package---as it was incorporated into the fixes
provided by |fixltx2e|--- modify the output routine to correct a problem with synchronizing floats in double column texts with those in single column \cite{fix2col,fixltx2e}. Here we will describe the normal behaviour.
Additional commentary on the changes is discussed later on.
 
\begin{tabular}{p{3cm}p{6cm}}
    |\@freelist|     & List of empty boxes for placing new floats.\\
    |\@toplist|      & List of floats to go at top of current column.\\
    |\@midlist|      & List of floats in middle of current column.\\
    |\@botlist|      & List of floats to go at bottom of current column.\\
    |\@deferlist|    & List of floats to go after current column.\\
    |\@dbltoplist|   & List of double-col. floats to go at top of current
                     page.\\
    |\@dbldeferlist| & List of double-column floats to go on subsequent
                     pages.\\
\end{tabular}

\begin{texexample}{Meaning toplist}{ex:toplist}
\makeatletter
\meaning\@freelist\\
\meaning\@toplist
\makeatother
\end{texexample}

\section{Float-Placement algorithms}


\begin{docCommand}{@addtobot}{}  Tries to put insert |\@currbox| on |\@botlist|.
                     Called only when:

                  * |\ht BOX < \@colroom|\\
                  * type of |\@currbox| not on |\@deferlist|\\
                  * |\@colnum > 0|\\
                  * @insert = false\\
\end{docCommand}



               If it succeeds, then:
\begin{trivlist}
                  \item  sets @insert true\\
                  \item  decrements |\@botroom| by |\ht| BOX\\
                  \item  decrements |\@botnum| and |\@colnum| by 1\\
                  \item decrements |\@colroom| by |\ht| BOX + either |\floatsep|
                    or |\textfloatsep|, as appropriate.\\
                 \item sets |\maxdepth| to 0pt\\
\end{trivlist}



\begin{docCommand}{@addtotoporbot}{}

 Tries to put insert |\@currbox| on |\@toplist| or
                    |\@botlist|.

                    Called only under same conditions as |\@addtobot|.

                    If it succeeds, then:
                       * sets @insert true
                       * decrements |\@toproom| or |\@botroom| by |\ht| BOX
                       * decrements |\@colnum| and either |\@topnum| or
                         |\@botnum| by 1
                       * decrements |\@colroom| by |\ht| BOX + |\floatsep|
                         or |\textfloatsep|, as appropriate.
\end{docCommand}

\begin{docCommand}{@addtocurcol}{}
  Tries to add |\@currbox| to current column, setting
                 @insert true if it succeeds, false otherwise.
                 It will add |\@currbox| to top only if bit 0 of
                |\count \@currbox| is 0, and to the bottom only if
                 bit 0 = 0 or an earlier float of the same type is
                 put on the bottom.

                 If the float is put in the text, then
                 |\penalty\interlinepenalty| is put
                 right after the float, before the following |\vskip|,
                 and 
                     
                         |\outputpenalty :=L 0.|
\end{docCommand}

\begin{docCommand}{@addtonextcol}{} Tries to add |\@currbox| to the next column, setting
                  @insert true if it succeeds, false otherwise.
\end{docCommand}

\begin{docCommand}{@addtodblcol}{} Tries to add |\@currbox| to the next double-column page,
                 adding it to |\@dbltoplist| if it succeeds and
                 |\@dbldeferlist| if it fails.
\end{docCommand}


%\begin{algorithm}
%  \cs{@addmarginpar} ==\\
%   \Begin{
%     \eIf{\cs{@currlist} nonempty}{
%        remove \cs{@marbox} from \cs{@currlist}\\
%        add \cs{@marbox} and \cs{@currbox} to \cs{@freelist}\\
%         NOTE: \cs{@currbox} = left box}{
%         LaTeX error: ?  \\
%     }
%     \cs{@tempcnta} := 1\\     %% 1 = right, -1 = left
%     \eIf{@twocolumn = true}{
%       then if @firstcolumn = true
%              then \cs{@tempcnta} := -1
%            fi}{
%            \eIf{@mparswitch = true}{
%               \If{count0 odd}{}{
%                   \cs{@tempcnta} := -1
%               }
%            }
%            \If{@reversemargin = true}{
%               \cs{@tempcnta} := -\cs{@tempcnta}
%            }
%     }
%
%     \If{\cs{@tempcnta} < 0}{\cs{box}\cs{@marbox} :=G \cs{box}\cs{@currbox}}
%     
%     \cs{@tempdima}   :=L maximum(\cs{@mparbottom} - \cs{@pageht}
%                                           + ht of \cs{@marbox}, 0)\\
%
%     \If{\cs{@tempdima} > 0}{LaTeX warning: 'marginpar moved'}
%
%     \cs{@mparbottom} :=G \cs{@pageht} + \cs{@tempdima} + depth of \@cs{marbox}
%                          + \cs{marginparpush}\\
%
%     \cs{@tempdima}   :=L \cs{@tempdima} - ht of \@cs{marbox}\\
%
%     \cs{box}\cs{@marbox} :=G \cs{box}\cs{@currbox}\\
%                                \cs{vbox}\{ \cs{vskip}\cs{@tempdima}\\
%                                        \cs{box}\cs{@marbox}\\
%                                       \}\\
%     height of \cs{@marbox} :=G depth of \cs{@marbox} :=G 0\\
%     \cs{kern} -\cs{@pagedp}\\
%     \cs{nointerlineskip}\\
%     
%     hbox\{\eIf{@tempcnta > 0}{hskip columnwidth\\
%                              hskip marginparsep}{
%                             hskip -marginparsep\\
%                             hskip -marginparwidth}
%             \cs{box}\cs{@marbox}\cs{hss}
%          \}\\
%     \cs{nobreak}\\
%     \cs{nointerlineskip}\\
%     \cs{hbox}\{\cs{vrule} height=0 width=0 depth=\cs{@pagedp}\}
%  }
%\end{algorithm}

 Floats and marginpars add a lot of dead cycles.

    \begin{teX}
\maxdeadcycles = 100
    \end{teX}

    \begin{teX}
\let\@elt\relax
    \end{teX}

\begin{docCommand}{@next}{}{}{}{}
    \begin{teX}
\def\@next#1#2#3#4{\ifx#2\@empty #4\else (*@\label{next}@*)
   \expandafter\@xnext #2\@@#1#2#3\fi}
    \end{teX}
\end{docCommand}

    \begin{teX}
\def\@xnext \@elt #1#2\@@#3#4{\def#3{#1}\gdef#4{#2}}
    \end{teX}


    \begin{teX}
\def\@testfalse{\global\let\if@test\iffalse}
\def\@testtrue {\global\let\if@test\iftrue}
\@testfalse
    \end{teX}
%
\footnotechanges{v1.1v}{1996/07/26}{remove \cs{global} before \cs{@test...}}
    \begin{teX}
\def\@bitor#1#2{\@testfalse {\let\@elt\@xbitor
   \@tempcnta #1\relax #2}}
    \end{teX}
%    RmS 91/11/22: Added test for |\count#1 = 0|.
%                  Suggested by Chris Rowley.
%
%
% \changes{v1.1v}{1996/07/26}{remove \cs{global} before \cs{@test...}}
    \begin{teX}
\def\@xbitor #1{\@tempcntb \count#1
   \ifnum \@tempcnta =\z@
   \else
     \divide\@tempcntb\@tempcnta
     \ifodd\@tempcntb \@testtrue\fi
   \fi}
    \end{teX}
%
\section{Definition of Float Boxes (inserts)}

All boxes are defined using \refCom{newinsert}.  A total of eighteen insertions are defined here
and later on inserted in the freelist (See line \ref{freelist}).
    \begin{teX}
\newinsert\bx@A \newinsert\bx@B \newinsert\bx@C
\newinsert\bx@D \newinsert\bx@E \newinsert\bx@F
\newinsert\bx@G \newinsert\bx@H \newinsert\bx@I
\newinsert\bx@J \newinsert\bx@K \newinsert\bx@L
\newinsert\bx@M \newinsert\bx@N \newinsert\bx@O
\newinsert\bx@P \newinsert\bx@Q \newinsert\bx@R
    \end{teX}

\tex allows 255 classes of insertions |\insert0| to |\insert254|. It is important to remember that
every insert is tied to other registers of the same number. For example, |\insert100| is connected
with |\count100|, |\dimen100|, |\skip100| and |\box100|. plain \tex provides an allocation function
for insertions as it does for registers. Appendix B includes the command:
\begin{quotation}
|\newinsert\footins|
\end{quotation} 
which defines |\footins| as the number for footnote insertions.

\latex2e adopts similar definitions (see Chapter \ref{kernel:ltplain}). In the latest versions
allocations are extended with |\extrafloats|.

The |\@freelist| is defined next. Notice that |\@elt| is included here to enable
the manipulation of the list later on.

    \begin{teX}
\gdef\@freelist{\@elt\bx@A\@elt\bx@B\@elt\bx@C\@elt\bx@D\@elt\bx@E (*@\label{freelist} @*)
               \@elt\bx@F\@elt\bx@G\@elt\bx@H\@elt\bx@I\@elt\bx@J
                \@elt\bx@K\@elt\bx@L\@elt\bx@M\@elt\bx@N
                \@elt\bx@O\@elt\bx@P\@elt\bx@Q\@elt\bx@R}
    \end{teX}

The rest of the lists are defined below and they are initialized as empty lists.

    \begin{teX}
\gdef\@toplist{}
\gdef\@botlist{}
\gdef\@midlist{}
\gdef\@currlist{}
\gdef\@deferlist{}
\gdef\@dbltoplist{}
\gdef\@dbldeferlist{}
    \end{teX}

\begin{texexample}{Current list}{ex:currentlist}
\makeatletter
\meaning\@currlist

\meaning \bx@A

\the\bx@A

\the\dimen252

\the\count252

\the\skip252


\makeatother
\end{texexample}

\section{Page layout parameters}

The page layout parameters (all taking values by the standard classes later on
are defined here. They are important in building up the page calculations.

    \begin{teX}
\newdimen\topmargin
\newdimen\oddsidemargin
\newdimen\evensidemargin
\let\@themargin=\oddsidemargin
\newdimen\headheight
\newdimen\headsep
\newdimen\footskip
\newdimen\textheight
\newdimen\textwidth
\newdimen\columnwidth
\newdimen\columnsep
\newdimen\columnseprule
\newdimen\marginparwidth
\newdimen\marginparsep
\newdimen\marginparpush
    \end{teX}


After these preliminary definitions are made a box is defined to hold material 
that is inserterted before the dvi file is produced. This is a general hook
and widely used by package authors.

\begin{docCommand}{AtBeginDvi}{\marg{contents}}{}

Uses a box register in which to put
stuff that must appear before anything else in the
|.dvi| file.

The stuff in the box should not add any typeset material to the
page when it is unboxed.
\end{docCommand}
    
    \begin{teX}
\newbox\@begindvibox
\def \AtBeginDvi #1{%
  \global \setbox \@begindvibox
    \vbox{\unvbox \@begindvibox #1}%
}                             
    \end{teX}

  
  \begin{docCommand}{@maxdepth}{}
    This is not the right place to set this; it needs to be set in a
    class/style file when |\maxdepth| is set.

    Also, many settings to |\maxdepth| should be to |\@maxdepth|,
    probably?     

    \begin{teX}  
\newdimen\@maxdepth
\@maxdepth = \maxdepth
    \end{teX}
  \end{docCommand}


 \begin{docCommand}{paperheight}{}{}
 \begin{docCommand}{paperwidth}{}{}
 Although earlier on, page parameters have been defined, we also need to define the paper height
and width.
    \begin{teX}
\newdimen\paperheight
\newdimen\paperwidth
    \end{teX}
 \end{docCommand}
 \end{docCommand}

The following nine switches have to be defined to keep track of various options.
 \begin{docCommand}{if@insert}{}
 \end{docCommand}
 \begin{docCommand}{if@fcolmade}{}
 \end{docCommand}
 \begin{docCommand}{if@specialpage}{}
  \end{docCommand}
 \begin{docCommand}{if@firstcolumn}{}
  \end{docCommand}
 \begin{docCommand}{if@twocolumn}{}
 \end{docCommand}
 \begin{docCommand}{if@twoside}{}
 \begin{docCommand}{if@reversemarginpar}{}
 \begin{docCommand}{if@mparswitch}{}
 \begin{docCommand}{col@number}{}
 \end{docCommand}
 
 
    Local switches first:
    \begin{teX}
\newif \if@insert
    \end{teX}
    These should definitely be global:
    \begin{teX}
\newif \if@fcolmade
\newif \if@specialpage \@specialpagefalse
    \end{teX}
    These should be global but are not always set globally in other
    files. 
    \begin{teX}
\newif \if@firstcolumn \@firstcolumntrue
\newif \if@twocolumn   \@twocolumnfalse
    \end{teX}
    Not sure about these: two questions.
    Should things which must apply to a whole doument be local or
    global (they probably should be `preamble only' commands)?
    Are these three such things?
    \begin{teX}
\newif \if@twoside     \@twosidefalse
\newif \if@reversemargin \@reversemarginfalse
\newif \if@mparswitch  \@mparswitchfalse
    \end{teX}
    This counter has been imported from `multicol'.
    \begin{teX}
\newcount \col@number
\col@number \@ne
    \end{teX}
 
 \end{docCommand}
 \end{docCommand}
 \end{docCommand}


\section{Internal registers}

    \begin{teX}
\newcount\@topnum
\newdimen\@toproom
\newcount\@dbltopnum
\newdimen\@dbltoproom
\newcount\@botnum
\newdimen\@botroom
\newcount\@colnum
\newdimen\@textmin
\newdimen\@fpmin
\newdimen\@colht
\newdimen\@colroom
\newdimen\@pageht
\newdimen\@pagedp
\newdimen\@mparbottom \@mparbottom\z@
\newcount\@currtype
\newbox\@outputbox
\newbox\@leftcolumn
\newbox\@holdpg
    \end{teX}

The page headers and page footers are initialized to their odd values, this makes sense
as a document always starts at an odd number.
    \begin{teX}
\def\@thehead{\@oddhead} % 
\def\@thefoot{\@oddfoot}
    \end{teX}


  \begin{docCommand}{clearpage}{}
 The tests at the beginning are an experimental attempt to avoid a
 completely empty page after a |\twocolumn[...]|.  This prevents the
 text from the argument vanishing into a float box, never to be seen
 again.  We hope that it does not produce wrong formatting in other
 cases.
    \begin{teX}
\def\clearpage{%
  \ifvmode
    \ifnum \@dbltopnum =\m@ne
      \ifdim \pagetotal <\topskip
        \hbox{}%
      \fi
    \fi
  \fi
  \newpage
  \write\m@ne{}%
  \vbox{}%
  \penalty -\@Mi
}
    \end{teX}
 \end{docCommand}

  \begin{docCommand}{cleardoublepage}{}
  
    \begin{teX}
\def\cleardoublepage{\clearpage\if@twoside \ifodd\c@page\else
    \hbox{}\newpage\if@twocolumn\hbox{}\newpage\fi\fi\fi}
    \end{teX}
 \end{docCommand}

  \begin{docCommand}{onecolumn}{}
    \begin{teX}
\def\onecolumn{%
  \clearpage
  \global\columnwidth\textwidth
  \global\hsize\columnwidth
  \global\linewidth\columnwidth
  \global\@twocolumnfalse
  \col@number \@ne
  \@floatplacement}
    \end{teX}
 \end{docCommand}

  \begin{docCommand}{newpage}{}
    The two checks at the beginning ensure that an item label or
    run-in section title immediately before a |\newpage| get printed
    on the correct page, the one before the page break.

    All three tests are largely to make error processing more robust;
    that is why they all reset the flags explicitly, even when it
    would appear that this would be done by a |\leavevmode|. 
    \begin{teX}
\def \newpage {%
  \if@noskipsec 
    \ifx \@nodocument\relax
      \leavevmode
      \global \@noskipsecfalse 
    \fi
  \fi
  \if@inlabel
    \leavevmode
    \global \@inlabelfalse 
  \fi
  \if@nobreak \@nobreakfalse \everypar{}\fi
  \par
  \vfil
  \penalty -\@M}
    \end{teX}
  \end{docCommand}
  
  \begin{docCommand}{@emptycol}{}
    It may be better to use an invisible rule rather than an empty
    box here.  
    \begin{teX}
\def \@emptycol {\vbox{}\penalty -\@M}
    \end{teX}
  \end{docCommand}

  \begin{docCommand}{twocolumn}{}
  \begin{docCommand}{@topnewpage}{}
    There are several bug fixes to the two-column stuff here.
    \begin{teX}
\def \twocolumn {%
  \clearpage
  \global\columnwidth\textwidth
  \global\advance\columnwidth-\columnsep
  \global\divide\columnwidth\tw@
  \global\hsize\columnwidth
  \global\linewidth\columnwidth
  \global\@twocolumntrue
  \global\@firstcolumntrue
  \col@number \tw@
    \end{teX}
    There is no reason to put a |\@dblfloatplacement| here since
    |\@topnewpage| ignores these settings.
    The |\@floatplacement| is needed in case this comes after some
    changes.

    \begin{teX}
  \@ifnextchar [\@topnewpage\@floatplacement
}
    \end{teX}
    
    Note that here, getting a box from the freelist can assume
    success since this comes just after a |\clearpage|.
    \begin{teX}
\long\def \@topnewpage [#1]{%
  \@nodocument
  \@next\@currbox\@freelist{}{}%
  \global \setbox\@currbox
    \color@vbox 
      \normalcolor
      \vbox {%
        \hsize\textwidth
        \@parboxrestore
        \col@number \@ne
        #1%
        \vskip -\dbltextfloatsep
             }%
    \color@endbox
    \end{teX}
    Added size test and warning message; perhaps we should use
    an error message.

    \begin{teX}
  \ifdim \ht\@currbox>\textheight
    \ht\@currbox \textheight
  \fi
    \end{teX}

    This next line is not essential but it is more robust to make this
    value non-zero, in case of weird errors.

    This next bit is what is needed from |\@addtodblcol|, plus some
    extra checks for error trapping.
    \begin{teX}
  \global \count\@currbox \tw@
  \@tempdima -\ht\@currbox
  \advance \@tempdima -\dbltextfloatsep
  \global \advance \@colht \@tempdima
  \ifx \@dbltoplist \@empty
  \else
    \@latexerr{Float(s) lost}\@ehb
    \let \@dbltoplist \@empty
  \fi
  \@cons \@dbltoplist \@currbox
    \end{teX}
    This setting of |\@dbltopnum| is used only to change the
    typesetting in\\ |\@combinedblfloats|.
    \begin{teX}
  \global \@dbltopnum \m@ne
%<*trace>
    \tr@ce{dbltopnum set to -1 (= \the \@dbltopnum) (topnewpage)}%
%</trace>
    \end{teX}

    At points such as this we need to check that there is still a
    minimal amount of room left on the page; this uses an arbitrary
    small value at present; but note that this value is larger than
    that used when checking that page is too full of normal floats.
   
    If there is little room left we just force a page-break, OK?
    This involves producing two empty columns.  The second empty
    column may be produced by |\output|, in which case an extra,
    misleading, warning will be generated, OK?  (This happens only
    when there is too little room left on the page for any float.)
    Otherwise (ie if the size is such that it is allowed as a normal
    float) the extra |\@emptycol| will be invoked in the second
    column by the conditional code guarded by the |\if@firstcolumn|
    test.
    
    I now think that the cut-off point here should be |3\baselineskip|,
    but we make it a bit less so that 3 lines of text will be
    allowed, OK?

    Since this happens only when there is nothing on the page but the
    `top-box', the empty box should not cause any problem other than
    some overfull box messages, which is not entirely misleading.

    Here we need two page-ends since both columns need to be empty.

    \begin{teX}
  \ifdim \@colht<2.5\baselineskip
    \@latex@warning@no@line {Optional argument of \noexpand\twocolumn
                too tall on page \thepage}%
    \@emptycol
    \if@firstcolumn
    \else
      \@emptycol
    \fi
  \else
    \global \vsize \@colht
    \global \@colroom \@colht
    \@floatplacement
  \fi
}
    \end{teX}
  \end{docCommand}
  \end{docCommand}

\section{The \textbackslash output routine}

We now arrive at the interesting part. The |\output| is a token register that holds instructions
as to how the page is to be typeset. This is called automatically by \tex. Think of it as the
main function.

  \begin{docCommand}{output}{}
    This needs some small adjustments.  We cannot
    guarantee that the float mechanism will interact correctly with
    this stuff, but that mechanism does not always work properly
    with footnotes already.
    
    The reset of |\par| to the output routine.
    This avoids problems when the output routine is
    called within a list where |\par| may be a no-op.


    \begin{teX}
\output {%
  \let \par \@@par
  \ifnum \outputpenalty<-\@M
    \@specialoutput
  \else
    \@makecol
    \@opcol
    \@startcolumn
    \@whilesw \if@fcolmade \fi
      {\@opcol\@startcolumn}%
  \fi
  \ifnum \outputpenalty>-\@Miv
    \ifdim \@colroom<1.5\baselineskip
      \ifdim \@colroom<\textheight  
        \@latex@warning@no@line {Text page \thepage\space
                               contains only floats}%
        \@emptycol
      \else
        \global \vsize \@colroom
      \fi
    \else
      \global \vsize \@colroom
    \fi
  \else
    \global \vsize \maxdimen
  \fi
}
    \end{teX}
\end{docCommand}

\begin{docCommand}{@specialoutput}{}
 \end{docCommand}
    \begin{teX}
\gdef\@specialoutput{%
   \ifnum \outputpenalty>-\@Mii
     \@doclearpage
   \else
     \ifnum \outputpenalty<-\@Miii
       \ifnum \outputpenalty<-\@MM \deadcycles \z@ \fi
       \global \setbox\@holdpg \vbox {\unvbox\@cclv}%
     \else
    \end{teX}
%
%    Note that |\boxmaxdepth| should not be set here since we wish to
%    record the natural depth of the holdpg box.
%    
%    This is changed so as to not lose anything, such as writes
%    and marks, which may get into box 255 and should be returned to
%    the list.  This should only happen when the first penalty in the
%    mechanism is discarded and therefore |\@holdpg| should always be
%    void in this case.  This can happen because a penalty is
%    discarded whenever there is no box on the list.
%
%    It was just: |\setbox\@tempboxa \box \@cclv|.
%    
%    The last box which is removed is the box put there by the
%    double-penalty mechanism.  The |\unskip| then removes the
%    |\topskip| which is put there since the box is the first on the
%    page.
    \begin{teX}
       \global \setbox\@holdpg \vbox{%
                      \unvbox\@holdpg
                      \unvbox\@cclv
    \end{teX}
%    We must now remove the box added by the float mechanism and the
%    |\topskip| glue therefore added above it by \TeX.
    \begin{teX}
                      \setbox\@tempboxa \lastbox
                      \unskip
                                     }%
    \end{teX}
%    These two are needed as separate dimensions only by
%    |\@addmarginpar|; for other purposes we put the whole size into
%    |\@pageht| (see below).
    \begin{teX}
       \@pagedp \dp\@holdpg
       \@pageht \ht\@holdpg
       \unvbox \@holdpg
       \@next\@currbox\@currlist{%
         \ifnum \count\@currbox>\z@
    \end{teX}
%    Putting the whole size into |\@pageht| (see above).
    \begin{teX}
           \advance \@pageht \@pagedp
           \ifvoid\footins \else
             \advance \@pageht \ht\footins
             \advance \@pageht \skip\footins
             \advance \@pageht \dp\footins
           \fi
           \ifvbox \@kludgeins
    \end{teX}
%    We want to make the adjustment due to this insert only if the
%    non-star form is used.  The *-form will probably not work with
%    floats, but maybe it still could make some adjustment here even
%    so?
    \begin{teX}
             \ifdim \wd\@kludgeins=\z@
               \advance \@pageht \ht\@kludgeins
             \fi
           \fi
    \end{teX}
%    This version puts the inserts back just before the additional
%    material; it could be moved earlier, before unboxing the
%    page-so-far.  Neither is guaranteed not to put things on the wrong
%    page.  This version is similar to the original version.
    \begin{teX}
           \@reinserts
           \@addtocurcol
         \else
           \@reinserts
           \@addmarginpar
         \fi
         }\@latexbug
    \end{teX}
%    A 2e change: use |\addpenalty| instead of |\penalty| here.  Some 
%    penalty is needed to create a potential break-point immediately
%    after the reinserts (or the marginal).  Otherwise there can be no
%    possibility to break here and this can cause the reinserts or the
%    marginal to appear on the next page (which is often incorrect).
%    However, if the nobreak flag is true, a |\nobreak| must be
%    correct.
    \begin{teX}
       \ifnum \outputpenalty<\z@
         \if@nobreak
           \nobreak
         \else
           \addpenalty \interlinepenalty
         \fi
       \fi
     \fi
   \fi
}
    \end{teX}
 


  \begin{docCommand}{@doclearpage}{}
    This is a very much an emergency action, just dumping everything:
    footnotes first then floats.  A more sophisticated version is
    needed; but even more urgent is a bug-free version (see, for
    example, pr/3528).

    Also, it puts any left-over non-boxes (writes, specials, etc.) back
    after any float pages created: this is a very bad bug since,
    for example, a kludge insert will be in quite the wrong place
    and, worse, be irremovable and uncancelable.
    
    \begin{teX}
\def \@doclearpage {%
     \ifvoid\footins
    \end{teX}
%    We empty any left over kludge insert box here; this is a temporary fix.
%    It should perhaps be applied to one page of cleared floats, but
%    who cares?  The whole of this stuff needs completely redoing for
%    many such reasons.
    \begin{teX}
       \ifvbox\@kludgeins
         {\setbox \@tempboxa \box \@kludgeins}%
       \fi 
       \setbox\@tempboxa\vsplit\@cclv to\z@ \unvbox\@tempboxa
       \setbox\@tempboxa\box\@cclv
       \xdef\@deferlist{\@toplist\@botlist\@deferlist}%
       \global \let \@toplist \@empty
       \global \let \@botlist \@empty
       \global \@colroom \@colht
       \ifx \@currlist\@empty
       \else
          \@latexerr{Float(s) lost}\@ehb
          \global \let \@currlist \@empty
       \fi
       \@makefcolumn\@deferlist
       \@whilesw\if@fcolmade \fi{\@opcol\@makefcolumn\@deferlist}%
       \if@twocolumn
         \if@firstcolumn
           \xdef\@dbldeferlist{\@dbltoplist\@dbldeferlist}%
           \global \let \@dbltoplist \@empty
           \global \@colht \textheight
           \begingroup
              \@dblfloatplacement
              \@makefcolumn\@dbldeferlist
              \@whilesw\if@fcolmade \fi{\@outputpage
                                        \@makefcolumn\@dbldeferlist}%
           \endgroup
         \else
           \vbox{}\clearpage
         \fi
       \fi
     \else
       \setbox\@cclv\vbox{\box\@cclv\vfil}%
       \@makecol\@opcol
       \clearpage
     \fi
}
    \end{teX}
 \end{docCommand}

  \begin{docCommand}{@opcol}{}

    \begin{teX}
\def \@opcol {%
  \if@twocolumn
    \@outputdblcol
  \else
    \@outputpage
  \fi
    \end{teX}
    These do not need to be done every time |\@opcol| is used: they
    should be grouped together since they all need to be done at the
    end of the non-special output routine, or at the end of a clearpage
    one.
    \begin{teX}
  \global \@mparbottom \z@ \global \@textfloatsheight \z@
  \@floatplacement
}
    \end{teX}
  \end{docCommand}
%
% The next two functions determine what to put in box255 and which output function to be called. 

  \begin{docCommand}{@makecol}

    We must rewrite this macro to alllow for variations in page-makeup
    required by changes in page-length.
     
    This uses a different macro if a special-length column is being
    produced.


    \begin{teX}
\gdef \@makecol {%
   \ifvoid\footins
     \setbox\@outputbox \box\@cclv
   \else
     \setbox\@outputbox \vbox {%
    \end{teX}
    This |\boxmaxdepth| setting is to ensure that  deep footnotes
    do not overwrite the footer (on account of the negative skip
    added later): it should use |\@maxdepth| otherwise the change is
    pointless when there are footnotes.
   % \task{CAR}{Investigate providing an option to put the footnotes
   % below the bottom floats.}

    But see also its use when combining floats.
    
    Macro |\newinsert| computes a number (counting down from 254) and
allocates a box, a count, a dimen, and a skip register with that number. The reason
for allocating from 254 instead of 255 is that |\box255| is reserved for special OTR
use. The reason for allocating downwards is that registers |\countO, \count1|. .. are
used for the page number, and many people tend to use registers |\boxO, \box1|. .. for
temporary storage.
\begin{texexample}{footins}{ex:footins}

\the\footins

\the\skip\footins

\the\dimen\footins

\the\count\footins

\end{texexample}


    \begin{teX}
       \boxmaxdepth \@maxdepth 
       % unbox                  
       \unvbox \@cclv
       % skip 
       \vskip \skip\footins
       
       \color@begingroup
         \normalcolor
         % draw rule
         \footnoterule
         
         % unbox footins
         \unvbox \footins
       \color@endgroup
       }%
   \fi
    \end{teX}
%    The h floats have now been finally committed to this page so we
%    can reset their list.  The top and bottom floats are then added
%    to the page.
%
    \begin{teX}
   \let\@elt\relax
   \xdef\@freelist{\@freelist\@midlist}%
   \global \let \@midlist \@empty
   \@combinefloats
    \end{teX}
%    The variations start here in case |\enlargethispage| has
%    been used.
    \begin{teX}
   \ifvbox\@kludgeins
     \@makespecialcolbox
   \else
    \end{teX}
%    This extra reboxing is only needed to add the
%    |\@texttop| and |\@textbotttom| but this could be done earlier,
%    when the floats are added.
%    
%    The |\boxmaxdepth| resetting here will have no effect unless
%    |\@textbottom| ends with a box or rule.  So is this (or possibly
%    |\@maxdepth|) the correct value?
%
%    The |\vskip -\dimen@|
%    ensures that the visible depth of the box does not
%    affect the placement of anything on the page.
%    Thus very deep pages will overprint the footer; but these should
%    have been prevented by suitable settings of the maxdepths at
%    appropriate times.
%    
%    If |\@textbottom| ends with a box or rule of non-zero depth
%    then this skip adjustemnt should be done again after it.
%    
%    I think that the final boxing of the main text page could have a
%    common ending which may make it simpler to see what is going on.
%    
%    This needs further investigation, especially in the `special
%    case'.
%
%    Also, the |\boxmaxdepth| setting here affects what happens wthin
%    |\@texttop| and |\@textbottom|, should it?  Is it needed at all? 
%    
    \begin{teX}
     \setbox\@outputbox \vbox to\@colht {%
       \@texttop
       \dimen@ \dp\@outputbox
       \unvbox \@outputbox
       \vskip -\dimen@
       \@textbottom
       }%
   \fi
   \global \maxdepth \@maxdepth
}
    \end{teX}
  \end{docCommand}

  \begin{docCommand}{@reinserts}{}
    This is the code which reinserts the inserts.  It puts them all
    in one place; this can make some of them come out on the wrong
    page.
    It has been put into a separate macro to expedite experimentation.
    \begin{teX}
\gdef \@reinserts{%
  \ifvoid\footins\else\insert\footins{\unvbox\footins}\fi
}
    \end{teX}
  \end{docCommand}
%
%
%
  \begin{docCommand}{@makespecialcolbox}{}
    This implements certain variations in page-makeup.
    \begin{teX}
\gdef \@makespecialcolbox {%
    \end{teX}

    First we find the natural height of the column.
    See above for discussion of what is happening here.
    This needs further investigation, especially in this `special
    case'. 
    \begin{teX}
   \setbox\@outputbox \vbox {%
     \@texttop
     \dimen@ \dp\@outputbox
     \unvbox\@outputbox
     \vskip-\dimen@
     }%
   \@tempdima \@colht
   \ifdim \wd\@kludgeins>\z@
    \end{teX}
    Note that in this case (the *-version), the height of the
    |\@kludgeins| box is not used since its value is somewhat
    arbitrary: it need only be big enough to ensure that the
    page-break is not taken prematurely.

    Here we calculate how much vertical space needs to be added in
    order to enable the column to fit into a box of size |\@colht|
    using the best information we have about the amount of shrink
    available (another thing which is known internally about a box,
    but cannot be accessed at the \TeX{} level!).

    This needs \TeX3 otherwise |\pageshrink| is zero anyway; it may
    not be exactly the figure we wish as it is the total available
    from the all the material collected before the page-break
    decision is made.  It will, we think, always be an overestimate
    of the actual shrink in the box; therefore this should always
    force the shortest possible column with the possibility of an
    overfull box.

    This should work for bothe flush- and ragged-bottom setting since
    it makes the contents no smaller than the size (|\@colht|) of the
    box into which they are put.

    Their should perhaps be an upper limit, of 0pt?, on the extra
    space added to force shrinking.
    \task{CAR}{Further investigation of kludge-* space}

    See above for a discussion of the |\boxmaxdepth| setting here.
    
    \begin{teX}
     \advance \@tempdima -\ht\@outputbox
     \advance \@tempdima \pageshrink
     \setbox\@outputbox \vbox to \@colht {%
       \unvbox\@outputbox
       \vskip \@tempdima
       \@textbottom
       }%
    \end{teX}
    
    For the unstarred version, the final size of the page is
    precisely specified.  Therefore, at least for the flush-bottom
    case, we need to ensure that, visually, it has this size exactly.

    Thus we calculate this size and set the material in a box of this
    size, which is then put into a box of size |\@colht| with |\vss|
    at the bottom.
    \begin{teX}
   \else
     \advance \@tempdima -\ht\@kludgeins
    \end{teX}
    
    This type of final packaging could be done always; this may
    simplify all of this page-makeup.

    It is not necessary to set |\boxmaxdepth| here since the
    |\@outputbox| ends with glue.

    \begin{teX}
     \setbox \@outputbox \vbox to \@colht {%
       \vbox to \@tempdima {%
         \unvbox\@outputbox
         \@textbottom}%
       \vss}%
   \fi
    \end{teX}
    Finally we need to explicitly make the insert box void.
    \begin{teX}
   {\setbox \@tempboxa \box \@kludgeins}%
    \end{teX}
  \end{docCommand}

The following macros are just hooks and can be set to add top and
bottom glue on all pages.\footnote{http://tex.stackexchange.com/questions/40469/use-of-texttop-and-textbottom-for-vertical-positioning}\footnote{https://tex.stackexchange.com/questions/131871/vertically-centering-page-with-texttop-and-textbottom} They are currently used in the definition of \refCom{raggedbottom} and \refCom{flushbottom}. 

\begin{docCommand}{@texttop}{}
Does nothing by default, otherwise add glue at the top on all pages.
\begin{teX}
  \let \@texttop \relax
\end{teX}  
\end{docCommand}
  
  \begin{docCommand}{@textbottom}{}
    Can be used to add glue at the bottom of a page.
    \begin{teX}
\let \@textbottom \relax
    \end{teX}
  \end{docCommand}
  

\begin{docCommand}{@resetactivechars}{}
\end{docCommand}
\begin{docCommand}{@activechar@info}{}

 added hook to protect against certain active characters in the
 output routine. Default checks are for active space and end-of-line.

    \begin{teX}
\def\@activechar@info #1{%
      \@latex@info@no@line {Active #1 character found while
                            output routine is active  
                            \MessageBreak
                            This may be a bug in a package file
                            you are using}%
}
    \end{teX}
    
    Do not put any spaces in this next bit!
    \begin{teX}
\begingroup
\obeylines\obeyspaces%
\catcode`\'\active%
\gdef\@resetactivechars{%
\def^^M{\@activechar@info{EOL}\space}%
\def {\@activechar@info{space}\space}%
\let'\active@math@prime}%
\endgroup
    \end{teX}
  \end{docCommand}




  \begin{docCommand}{@outputpage}{}
  \end{docCommand}      
    The |\color@hbox| hooks here are used to avoid putting just a
    colour special into an otherwise empty box (in a header or
    footer).  These boxes are often set to be completely empty and so
    adding a special produces a very underfull box message.
    
    There has been extensive tidying up of the old code here;
    including the removal of a level of grouping.
 
    The setting of |\protect| immediately before the |\shipout|
    is needed so that protected commands within |\write|s are
    handled correctly.
 
    Within shipout's vbox it is reset to its default value, |\relax|.
 
    Resetting it to its default value after the shipout has been 
    completed (and the contents of the writes have been expanded)
    must be done by use of |\aftergroup|.
    This is because it must have the value |\relax|
    before macros coming from other uses of |\aftergroup| within
    this box are expanded.

    Putting this into the |\aftergroup| token list does not affect
    the definition used in expanding the |\write|s because the
    aftergroup token list is only constructed when popping the
    save-stack, it is not expanded until after the shipout is
    completed.

    Question: should things from an |\aftergroup| within the shipped
    out box be executed in the environment set up for the writes, or
    after it finishes?

    A lot of this code has been in-lined tp prevent mis-use of
    internal commands as hooks.
\begin{docCommand}{@outputpage}{\Arg{void}}
This essentially calls |\shipout|
\end{docCommand}        
    \startlineat{472}
    \begin{teX}
\def\@outputpage{%
\begingroup           % the \endgroup is put in by \aftergroup
    \end{teX}
    Now all the set-up stuff has been in-lined for Frank.

    First the stuff for the writes.
    
    From here \ldots\ was in the command |\@writesetup|. 
    \begin{teXX}
  \let \protect \noexpand
    \end{teXX}

    RmS 93/08/19: Redefined accents to allow changes in font encoding; 
    but exactly why was this needed?
 
    The |\catcode`\ = 10| was removed as it was considered useless 
    (presumably because nothing gets tokenised during shipout).
    
    This was put in as some error produced active spaces in a mark, I 
    think.
    
    Why was the hyphen reset?
    
    \begin{teX}
  \@resetactivechars
    \end{teX}
    If a page break happens between the start of a list and its first
    item the |@newlist| will be true and this will mess up any list
    that is used in the header or footer of the page. So we have to
    reset that flag.
    \begin{teX}
  \global\let\@@if@newlist\if@newlist
  \global\@newlistfalse
    \end{teX}
     with the new encoding setup they can use \cs{let}.
     It could also use the new internal commands?
    This next hook replaces the following:
    \begin{verbatim}
      \let\-\@dischyph
      \let\'\@acci\let\`\@accii\let\=\@acciii
      \let\\\@normalcr
      \let\par\@@par %% 15 Sep 87 (this was once inside the box)
    \end{verbatim}
%    and it does more than they did; in particular it sets:
    \begin{verbatim}
%      \parindent\z@
%      \parskip\z@skip
%      \everypar{}%
%      \leftskip\z@skip
%      \rightskip\z@skip
%      \parfillskip\@flushglue
%      \lineskip\normallineskip
%      \baselineskip\normalbaselineskip
%      \sloppy
    \end{verbatim}
%    
    \begin{teX}
  \@parboxrestore
    \end{teX}
%    \ldots\ to here was in the command |\@writesetup|. 


Finally we are ready to \docAuxCommand{shipout} the box.

    \begin{teX}
  \shipout \vbox{%
    \set@typeset@protect
    \aftergroup \endgroup
    \aftergroup \set@typeset@protect
                                % correct? or just restore by ending
                                % the group?
    \end{teX}
%    This first bit has been moved inside the shipped out box.
%    
    Now the setup inside the shipped out box; this should contain all 
    the stuff that could only affect typesetting; other stuff may need 
    to be reset for the writes also.
    
    From here \ldots\ was in the command |\@shipoutsetup|. 
    
    The \docAuxCommand{@specialpage} picks up the pagestyle using the |ps@|
    \begin{teX}
  \if@specialpage
    \global\@specialpagefalse\@nameuse{ps@\@specialstyle}%
  \fi
  \if@twoside
    \ifodd\count\z@ \let\@thehead\@oddhead \let\@thefoot\@oddfoot
         \let\@themargin\oddsidemargin
    \else \let\@thehead\@evenhead
       \let\@thefoot\@evenfoot \let\@themargin\evensidemargin
    \fi
  \fi
    \end{teX}
    
    The rest was always inside the box.

    \begin{teX}
  \reset@font 
    \end{teX}
    RmS 93/08/06 Added |\lineskiplimit=0pt| to guard against it being
              nonzero: e.g. by |\offinterlineskip| being in effect.
    
    There are probably lots of other things that may need resetting.
    
    \begin{teX}
  \normalsize
    \end{teX}
 Reset the space factors.
     {Call \cs{normalsfcodes} (from patch file) latex/2404}
    \begin{teX}
  \normalsfcodes
    \end{teX}

 Reset these here (previously reset separately for head and foot)
    \begin{teX}
  \let\label\@gobble
  \let\index\@gobble
  \let\glossary\@gobble
    \end{teX}

    \begin{teX}
  \baselineskip\z@skip \lineskip\z@skip \lineskiplimit\z@
    \end{teX}
    \ldots\ to here was in the command |\@shipoutsetup|. 
    \begin{teX}
    \@begindvi (*@\dcircle{1}@*)
    \vskip \topmargin
    \moveright\@themargin \vbox {%
      \setbox\@tempboxa \vbox to\headheight{%
        \vfil
        \color@hbox
          \normalcolor
          \hb@xt@\textwidth{\@thehead}%
        \color@endbox
        }%                        %% 22 Feb 87
      \dp\@tempboxa \z@
      \box\@tempboxa
      \vskip \headsep
      \box\@outputbox
      \baselineskip \footskip
      \color@hbox
        \normalcolor
        \hb@xt@\textwidth{\@thefoot}%
      \color@endbox
      }%
    }%
    \end{teX}
   |\endgroup| now inserted by |\aftergroup|

 Restore |\if@newlist|
    \begin{teX}
  \global\let\if@newlist\@@if@newlist
    \end{teX}

    \begin{teX}
  \global \@colht \textheight
  \stepcounter{page}%
    \end{teX}
    It is now clear that this does something useful, thanks to Piet
    van Oostrum.  It is needed because a float page is made without
    using TeX's page-builder; thus the output routine is never called
    so the marks are not updated.
    \begin{teX}
  \let\firstmark\botmark
}
    \end{teX}
%  \end{docCommand}

At \dcircle{1} the |@begindvi| was inserted for the material taht it holds. This in reality
is a hook, used to place anything we want at the page, provided it holds material that do not take
any space. \makeatletter

\the\@themargin

\vbox{\scalebox{0.5}{\@thehead}
\vskip30pt
\scalebox{0.3}{\@thefoot}}
\makeatother

 \begin{docCommand}{@begindvi}{}
 
    This unboxes stuff that must appear before anything else in the
    |.dvi| file, then returns that box register to the free list and
    cancels itself.

    The stuff in the box should not add any typeset material to the
    page.  Provided that you inserting material which is absolutely
    positioned, this is a good place to hook-in. 

\startlineat{528}        
    \begin{teX}
\def \@begindvi{%
  \unvbox \@begindvibox
  \global\let \@begindvi \@empty
}
    \end{teX}
 \end{docCommand}

The combining of the floats happens here.

 \begin{docCommand}{@combinefloats}{}
  \end{docCommand}
  
 \begin{docCommand}{@cflb}{}
 
    The |\boxmaxdepth| setting here was not made local to
    a box so was dangerous.  It is needed only within the box made
    by |\@cflt| (and not normally even there), so it has been
    moved there; this also agrees with the original pseudcode.
    
The |\@combinefloat| function will add the top and bottom floats
if their lists are not empty. |\cf|\meta{lt} is or top and |lb| for bottom.
       \begin{teX}
\def \@combinefloats {%
    \ifx \@toplist\@empty \else \@cflt \fi
    \ifx \@botlist\@empty \else \@cflb \fi
}
       \end{teX}

First define the |\@elt| 
       \begin{teX}
\def \@cflt{%
    \let \@elt \@comflelt
    \setbox\@tempboxa \vbox{}%
    \@toplist
    \setbox\@outputbox \vbox{%
                             \boxmaxdepth \maxdepth
                             \unvbox\@tempboxa
                             \vskip -\floatsep
                             \topfigrule
                             \vskip \textfloatsep
                             \unvbox\@outputbox
                             }%
                             
    % reset |\@elt| to |\relax|
    \let\@elt\relax
    
    % return |\@toplist| to the |\@freelist|
    \xdef\@freelist{\@freelist\@toplist}%
    
    % empty the |\@toplist|
    \global\let\@toplist\@empty
}
       \end{teX}
%
       \begin{teX}
\def \@cflb {%
    \let\@elt\@comflelt
    \setbox\@tempboxa \vbox{}%
    \@botlist
    \setbox\@outputbox \vbox{%
                             \unvbox\@outputbox
                             \vskip \textfloatsep
                             \botfigrule
                             \unvbox\@tempboxa
                             \vskip -\floatsep
                             }%
    \let\@elt\relax
    \xdef\@freelist{\@freelist\@botlist}%
    \global \let \@botlist\@empty
}
       \end{teX}
  \end{docCommand}
 
%
 \begin{docCommand}{@comflelt}{}
 \begin{docCommand}{@comdblflelt}{}
 \begin{docCommand}{@combinedblfloats}{}
% 
       \begin{teX}
\def\@comflelt#1{\setbox\@tempboxa
      \vbox{\unvbox\@tempboxa\box #1\vskip\floatsep}}
       \end{teX}
%
       \begin{teX}
\def\@comdblflelt#1{\setbox\@tempboxa
      \vbox{\unvbox\@tempboxa\box #1\vskip\dblfloatsep}}
       \end{teX}
%
       \begin{teX}
\def \@combinedblfloats{%
  \ifx \@dbltoplist \@empty
  \else
    \setbox\@tempboxa \vbox{}%
    \let \@elt \@comdblflelt
    \@dbltoplist
    \let \@elt \relax 
    \xdef \@freelist {\@freelist\@dbltoplist}%
    \global\let \@dbltoplist \@empty
    \setbox\@outputbox \vbox to\textheight
       \end{teX}
%
%    The setting of |\boxmaxdepth| here has no effect since the
%    |\@outputbox| should already have depth zero.  Even so, it would
%    have no effect on the layout of the page.
% \footnotechanges{v1.0l}{1994/03/15}{Removed boxmaxdepth setting.}
       \begin{teX}
      {%\boxmaxdepth\maxdepth   %% probably not needed, CAR
       \unvbox\@tempboxa\vskip-\dblfloatsep
       \end{teX}
%    Here we need different typesetting if the top float comes from
%    |\@topnewpage|. 
% \footnotechanges{v1.0n}{1994/04/30}{Removed rule in topnewpage case}
       \begin{teX}
       \ifnum \@dbltopnum>\m@ne
         \dblfigrule
       \fi
       \vskip \dbltextfloatsep
       \box\@outputbox
       }%
  \fi
}
       \end{teX}
  \end{docCommand}
  \end{docCommand}
  \end{docCommand}
%
%
  \begin{docCommand}{@startcolumn}{}
% 
    We could combine (most of) these two into |\@startcol <list>|.
   This is not quite
    as efficient but it now has the same structure as
    |\@startdblcolumn|.
%
    The empty-list test has been moved to |\@tryfcolumn|.
%
    \begin{teX}
\def \@startcolumn {%
  \global \@colroom \@colht
  \@tryfcolumn \@deferlist
  \if@fcolmade
%<*trace>
    \tr@ce{PAGE: float \if@twocolumn column \else page \fi
                completed}%
%</trace>
  \else
    \end{teX}
% \changes{v1.0h}{1993/12/12}{defs changed to lets}
    \begin{teX}
    \begingroup
      \let \reserved@b \@deferlist
      \global \let \@deferlist \@empty
      \let \@elt \@scolelt
      \reserved@b
    \endgroup
  \fi
}
    \end{teX}
%
%    This one does not need to set |\@colht|.
%
    \begin{teX}
\def \@startdblcolumn {%
    \end{teX}
% Not needed since this always comes after |\@outputpage|:
    \begin{teX}
% \global \@colht \textheight
  \@tryfcolumn \@dbldeferlist
  \if@fcolmade
%<*trace>
    \tr@ce{PAGE: double float page completed}%
%</trace>
  \else
    \end{teX}
% \changes{v1.0h}{1993/12/12}{defs changed to lets}
    \begin{teX}
    \begingroup
      \let \reserved@b \@dbldeferlist
      \global \let \@dbldeferlist \@empty
      \let \@elt \@sdblcolelt
      \reserved@b
    \endgroup
  \fi
}
    \end{teX}
  \end{docCommand}

%
  \begin{docCommand}{@tryfcolumn} {}

  Now tests if its list is empty before any further exertion.
%
    \begin{teX}
\def \@tryfcolumn #1{%   
  \global \@fcolmadefalse  (*@ \label{tryfcolumn}  @*)
  \ifx #1\@empty
  \else
%<*trace>
     \tr@ce{PAGE: try float \if@twocolumn column/page\else page\fi
                  ---\string #1}%
     \tr@ce{----- \string #1: #1}%
%</trace>
    \end{teX}
% \changes{v1.0h}{1993/12/12}{defs changed to lets}
    \begin{teX}
    \xdef\@trylist{#1}%
    \global \let \@failedlist \@empty
    \begingroup
      \let \@elt \@xtryfc \@trylist
    \endgroup
    \if@fcolmade
      \@vtryfc #1%
    \fi
  \fi
}
    \end{teX}
%
  \end{docCommand}
%
%
 \begin{docCommand}{@scolelt}{}
    \begin{teX}
\def\@scolelt#1{\def\@currbox{#1}\@addtonextcol}
    \end{teX}
 \end{docCommand}
%
 \begin{docCommand}{@sdblcolelt}{}
    \begin{teX}
\def\@sdblcolelt#1{\def\@currbox{#1}\@addtodblcol}
    \end{teX}
 \end{docCommand}
%
 \begin{docCommand}{@vtryfc}{}
    \begin{teX}
\def\@vtryfc #1{%
  \global\setbox\@outputbox\vbox{}%
  \let\@elt\@wtryfc
  \@flsucceed
  \global\setbox\@outputbox \vbox to\@colht{%
    \vskip \@fptop
    \vskip -\@fpsep
    \unvbox \@outputbox
    \vskip \@fpbot}%
  \let\@elt\relax
  \xdef #1{\@failedlist\@flfail}%
  \xdef\@freelist{\@freelist\@flsucceed}}
    \end{teX}
 \end{docCommand}
%
 \begin{docCommand}{@wtryfc}{}
    \begin{teX}
\def\@wtryfc #1{%
  \global\setbox\@outputbox\vbox{%
    \unvbox\@outputbox
    \vskip\@fpsep
    \box #1}}
    \end{teX}
 \end{docCommand}
%
 \begin{docCommand}{@xtryfc}{}
    \begin{teX}
\def\@xtryfc #1{%
  \@next\reserved@a\@trylist{}{}%
  \@currtype \count #1%
  \divide\@currtype\@xxxii
  \multiply\@currtype\@xxxii
  \@bitor \@currtype \@failedlist
  \@testfp #1%
  \ifdim \ht #1>\@colht
    \@testtrue
  \fi
  \if@test
    \@cons\@failedlist #1%
  \else
    \@ytryfc #1%
  \fi}
    \end{teX}
 \end{docCommand}
%
 \begin{docCommand}{@ytryfc}{}
    \begin{teX}
\def\@ytryfc #1{%
  \begingroup
    \gdef\@flsucceed{\@elt #1}%
    \global\let\@flfail\@empty
    \@tempdima\ht #1%
    \let\@elt\@ztryfc
    \@trylist
    \ifdim \@tempdima >\@fpmin
      \global\@fcolmadetrue
    \else
      \@cons\@failedlist #1%
    \fi
  \endgroup
  \if@fcolmade
    \let\@elt\@gobble
  \fi}
    \end{teX}
 \end{docCommand}
%
 \begin{docCommand}{@ztryfc}{}
    \begin{teX}
\def\@ztryfc #1{%
  \@tempcnta \count#1%
  \divide\@tempcnta\@xxxii
  \multiply\@tempcnta\@xxxii
  \@bitor \@tempcnta {\@failedlist \@flfail}%
  \@testfp #1%
  \@tempdimb\@tempdima
  \advance\@tempdimb \ht#1%
  \advance\@tempdimb\@fpsep
  \ifdim \@tempdimb >\@colht
    \@testtrue
  \fi
  \if@test
    \@cons\@flfail #1%
  \else
    \@cons\@flsucceed #1%
    \@tempdima\@tempdimb
  \fi}
    \end{teX}
 \end{docCommand}


 The major changes for float suppression and the changes to the float
 mechanism to make it conform to the documentation are in these next
 macros.

  \begin{docCommand}{@addtobot}{}

    \begin{teX}
%<*2ekernel|autoload|fltrace>
\def \@addtobot {%
   \@getfpsbit 4\relax
   \ifodd \@tempcnta
     \@flsetnum \@botnum
     \ifnum \@botnum>\z@
       \@tempswafalse
       \@flcheckspace \@botroom \@botlist
       \if@tempswa
    \end{teX}

%    This next line means that this page is produced with box 255 
%    having depth zero, rather than the normal maxdepth: is this
%    needed, useful? 

        \begin{teX}
         \global \maxdepth \z@
         \@flupdates \@botnum \@botroom \@botlist
         \@inserttrue
       \fi
%<*trace>
     \else
       \tr@ce{Fail: botnum = \the \@botnum:
                                  fpstype \the \@fpstype=ORD?}%
       \ifnum \@fpstype<\sixt@@n
         \tr@ce{ERROR: !b float not successful (addtobot)}%
       \fi
%</trace>
     \fi
   \fi
}
    \end{teX}
  \end{docCommand}
%
  \begin{docCommand}{@addtotoporbot}{}
%    Lots of changes.
%
    \begin{teX}
\def \@addtotoporbot {%
%<*trace>
   \tr@ce{***Start addtotoporbot}%
%</trace>
   \@getfpsbit \tw@
%<*trace>
   \tr@ce{fpstype \ifodd \@tempcnta OK \else not \fi top:
                                                     \the \@fpstype}%
%</trace>
   \ifodd \@tempcnta
     \@flsetnum \@topnum
     \ifnum \@topnum>\z@
       \@tempswafalse
       \@flcheckspace \@toproom \@toplist
       \if@tempswa
         \@bitor\@currtype{\@midlist\@botlist}%
%<*trace>
           \tr@ce{(mid+bot)list: \@midlist, \@botlist:
                              (addtotoporbot-before)}%
%</trace>
         \if@test
%<*trace>
           \tr@ce{type already on list: mid or bot---sent to addtobot}%
%</trace>
         \else
          \@flupdates \@topnum \@toproom \@toplist
%<*trace>
          \tr@ce{colroom (after-top) = \the \@colroom}%
          \tr@ce{colnum (after-top) = \the \@colnum}%
          \tr@ce{topnum (after-top) = \the \@topnum}%
          \tr@ce{***Success: top}%
%</trace>
          \@inserttrue
         \fi
       \fi
%<*trace>
     \else
       \tr@ce{Fail: topnum = \the \@topnum: fpstype
                                            \the \@fpstype=ORD?}%
       \ifnum \@fpstype<\sixt@@n
         \tr@ce{ERROR: !t float not successful (addtotoporbot)}%
       \fi
%</trace>
     \fi
   \fi
   \if@insert
   \else
%<*trace>
     \tr@ce{sent to addtobot (addtotoporbot)}%
%</trace>
     \@addtobot
   \fi
}
%</2ekernel|autoload|fltrace>
    \end{teX}
  \end{docCommand}
%
%  \begin{docCommand}{@addtocurcol}
% \changes{v1.0f}{1993/12/05}{Command changed}
% \task{CAR}{Add rules around h floats for FMi}
% \task{CAR}{Investigate pagebreak option possibilities}
%    Lots of changes.
%
    \begin{teX}
%<*2ekernel|autoload|fltrace|flafter>
\def \@addtocurcol {%
%<*trace>
  \tr@ce{***Start addtocurcol}%
%</trace>
   \@insertfalse
   \@setfloattypecounts
   \ifnum \@fpstype=8
%<*trace>
     \tr@ce{fpstype !p only (addtocurcol): \the \@fpstype = 8?}%
%</trace>
   \else
     \ifnum \@fpstype=24
%<*trace>
       \tr@ce{fpstype p only (addtocurcol): \the \@fpstype = 24?}%
%</trace>
     \else
       \@flsettextmin
  \end{teX}
% This is a new adjustment which is quite a major change in
% functionality; but it implements the documentation.
% Note that |\@reqcolroom| will include the whole of the
% page-so-far, and hence includes |\@textfloatsheight| of floats,
% so before comparing it with |\@textmin|, we add this to
% |\@textmin| also.
       \begin{teX}
%<*trace>
       \tr@ce{textfloatsheight (before) = \the \@textfloatsheight}%
%</trace>
       \advance \@textmin \@textfloatsheight
       \@reqcolroom \@pageht
       \end{teX}
% This line must be removed since |\@specialoutput| changed.
       \begin{teX}
%       \advance \@reqcolroom \@pagedp
%<*trace>
       \tr@ce{textmin + textfloatsheight: \the \@textmin}%
       \tr@ce{page-so-far: \the \@reqcolroom}%
%</trace>
       \ifdim \@textmin>\@reqcolroom
         \@reqcolroom \@textmin
%<*trace>
         \tr@ce{ORD? textmin being used}%
%</trace>
       \fi
       \advance \@reqcolroom \ht\@currbox
%<*trace>
       \tr@ce{float size = \the \ht \@currbox (addtocurcol)}%
       \tr@ce{colroom = \the \@colroom (addtocurcol)}%
       \tr@ce{reqcolroom = \the \@reqcolroom (addtocurcol)}%
%</trace>
       \ifdim \@colroom>\@reqcolroom
         \@flsetnum \@colnum
         \ifnum \@colnum>\z@
           \@bitor\@currtype\@deferlist
%<*trace>
           \tr@ce{deferlist: \@deferlist: (addtocurcol-before)}%
%</trace>
           \if@test
%<*trace>
             \tr@ce{type already on list: defer (addtocurcol)}%
%</trace>
           \else
             \@bitor\@currtype\@botlist
%<*trace>
           \tr@ce{botlist: \@botlist: (addtocurcol-before)}%
%</trace>
             \if@test
%<*trace>
               \tr@ce{type already on list: bot---sent to addtobot}%
%</trace>
               \@addtobot
             \else
%<*trace>
               \tr@ce{fpstype \ifodd \@tempcnta OK \else not \fi
                      here: \the \@fpstype}%
%</trace>
               \ifodd \count\@currbox
                 \advance \@reqcolroom \intextsep
                 \ifdim \@colroom>\@reqcolroom
                   \global \advance \@colnum \m@ne
                   \global \advance \@textfloatsheight \ht\@currbox
       \end{teX}
% This may sometimes give an overestimate.
       \begin{teX}
                   \global \advance \@textfloatsheight 2\intextsep
                   \@cons \@midlist \@currbox
%<*trace>
                   \tr@ce{***Success: here}%
                   \tr@ce{textfloatsheight (after-here) =
                        \the \@textfloatsheight}%
                   \tr@ce{colnum (after-here) = \the \@colnum}%
%</trace>
       \end{teX}
% 
% CHANGE TO |\@addtocurcol|:
% 
% |\penalty\z@| changed to |\penalty\interlinepenalty| so |\samepage|
% works properly with figure and table environments.
% (Changed 23 Oct 86)
%
% There is also an |\addpenalty\interlinepenalty| above.
%
% Since in 2e |\samepage| is no longer supported, these could be
% removed.
%
% Although it is best to use |\addvspace| in case two h floats come
% together, this makes other spacing more difficult to adjust; whereas
% if a user specifies two h floats together then they can more easily
% get the spacing correct by ad hoc commands.
%
% It is necessary to adjust for the addition of |\parskip| here in
% case the float is added betweeen paragraphs (\ie when in vertical
% mode).
%
% If the nobreak switch is true we need to reset it and clear
% |\everypar| sionce the float may not reset the flag and cannot reset
% the |\everypar| globally.
% \changes{v1.0l}{1994/03/15}{Changed \cs{addvspace} to \cs{vskip}}
% \changes{v1.1i}{1994/11/21}
%   {Added \cs{if@nobreak} test before float box}
% \changes{v1.1z}{1996/10/24}{Added \cs{nobreak}, etc as appropriate}
% 
% Typesetting starts here (we are in vertical mode).
       \begin{teX}
                   \if@nobreak
                     \nobreak
                     \@nobreakfalse
                     \everypar{}%
                   \else
                     \addpenalty \interlinepenalty
                   \fi
                   \vskip \intextsep
                   \box\@currbox
                   \penalty\interlinepenalty
                   \vskip\intextsep
                   \ifnum\outputpenalty <-\@Mii \vskip -\parskip\fi
       \end{teX}
% Typesetting ends here.
       \begin{teX}
                   \outputpenalty \z@
                   \@inserttrue
%<*trace>
                 \else
                   \tr@ce{Fail---no room at 2nd test of colroom
                                 (addtocorcol \string\intextsep)}%
%</trace>
                 \fi
               \fi
               \if@insert
               \else
%<*2ekernel|autoload|fltrace>
%<*trace>
                 \tr@ce{not here: sent to addtotoporbot}%
%</trace>
                 \@addtotoporbot
%</2ekernel|autoload|fltrace>
%<*!2ekernel&!autoload&!fltrace>
%<*trace>
                 \tr@ce{not here: sent to addtobot}%
%</trace>
                 \@addtobot
%</!2ekernel&!autoload&!fltrace>
               \fi
             \fi
           \fi
%<*trace>
         \else
           \tr@ce{Fail: colnum = \the \@colnum:
                        fpstype \the \@fpstype=ORD?}%
           \ifnum \@fpstype<\sixt@@n
             \tr@ce{ERROR: BANG float not successful (addtocurcol)}%
           \fi
%</trace>
         \fi
%<*trace>
       \else
         \tr@ce{Fail---no room: fl box ht: \the \ht \@currbox
                                                     (addtocurcol)}%
%</trace>
       \fi
     \fi
   \fi
   \if@insert
   \else
     \@resethfps
%<*trace>
     \tr@ce{put on deferlist (addtocurcol)}%
%</trace>
     \@cons\@deferlist\@currbox
%<*trace>
     \tr@ce{deferlist: \@deferlist: (addtocurcol-after)}%
%</trace>
   \fi
}
%</2ekernel|autoload|fltrace|flafter>
       \end{teX}
%  \end{docCommand}
%
\begin{docCommand}{@addtonextcol}{}
\end{docCommand}
% \changes{v1.0f}{1993/12/05}{Command changed}
%    Lots of changes.
%
       \begin{teX}
%<*2ekernel|autoload|fltrace>
\def\@addtonextcol{%
  \begingroup
%<*trace>
   \tr@ce{***Start addtonextcol}%
%</trace>
   \@insertfalse
   \@setfloattypecounts
   \ifnum \@fpstype=8
%<*trace>
     \tr@ce{fpstype not curcol: \the \@fpstype = 8?}%
%</trace>
   \else
     \ifnum \@fpstype=24
%<*trace>
       \tr@ce{fpstype not curcol: \the \@fpstype = 24?}%
%</trace>
     \else
       \@flsettextmin
%<*trace>
       \tr@ce{text-so-far: 0pt (top of col)}%
%</trace>
       \@reqcolroom \ht\@currbox
%<*trace>
       \tr@ce{float size: \the \@reqcolroom (addtonextcol)}%
%</trace>
       \advance \@reqcolroom \@textmin
%<*trace>
       \tr@ce{colroom = \the \@colroom (addtonextcol)}%
       \tr@ce{reqcolroom = \the \@reqcolroom (addtonextcol)}%
%</trace>
       \ifdim \@colroom>\@reqcolroom
         \@flsetnum \@colnum
         \ifnum\@colnum>\z@
            \@bitor\@currtype\@deferlist
%<*trace>
            \tr@ce{deferlist: \@deferlist: (addtonextcol-before)}%
%</trace>
            \if@test
%<*trace>
              \tr@ce{type already on list: defer (addtonextcol)}%
%</trace>
            \else
%<*trace>
              \tr@ce{sent to addtotoporbot (addtonextcol)}%
%</trace>
              \@addtotoporbot
            \fi
         \fi
%<*trace>
       \else
         \tr@ce{Fail---no room: fl box ht: \the \ht \@currbox
                                                  (addtonextcol)}%
%</trace>
       \fi
     \fi
   \fi
   \if@insert
   \else
%<*trace>
     \tr@ce{put back on deferlist (addtonextcol)}%
%</trace>
     \@cons\@deferlist\@currbox
%<*trace>
     \tr@ce{deferlist: \@deferlist: (addtonextcol-after)}%
%</trace>
   \fi
%<*trace>
   \tr@ce{End of addtonextcol -- locally counts:}%
   \tr@ce{ col: \the \@colnum. top: \the \@topnum. bot: \the \@botnum.}%
%</trace>
  \endgroup
%<*trace>
  \tr@ce{End of addtonextcol -- globally counts:}%
  \tr@ce{col: \the \@colnum. top: \the \@topnum. bot: \the \@botnum.}%
%</trace>
}
       \end{teX}
%  \end{docCommand}
%
%  \begin{docCommand}{@addtodblcol}
% \changes{v1.0f}{1993/12/05}{Command changed}
%    Lots of changes.
%
       \begin{teX}
\def\@addtodblcol{%
  \begingroup
%<*trace>
  \tr@ce{***Start addtodblcol}%
%</trace>
   \@insertfalse
   \@setfloattypecounts
   \@getfpsbit \tw@
%<*trace>
   \tr@ce{fpstype \ifodd \@tempcnta OK \else not \fi dbltop:
                                                     \the \@fpstype}%
%</trace>
   \ifodd\@tempcnta
     \@flsetnum \@dbltopnum
     \ifnum \@dbltopnum>\z@
       \@tempswafalse
       \ifdim \@dbltoproom>\ht\@currbox
         \@tempswatrue
%<*trace>
         \tr@ce{Space OK: \@dbltoproom =
                \the \@dbltoproom > \the \ht \@currbox
                                         (dbltoproom)}%
%</trace>
       \else
%<*trace>
         \tr@ce{fpstype: \the \@fpstype (addtodblcol)}%
%</trace>
         \ifnum \@fpstype<\sixt@@n
%<*trace>
           \tr@ce{BANG float ignoring \@dbltoproom}%
           \tr@ce{\@spaces \@dbltoproom = \the \@dbltoproom.
                           Ht float: \the \ht \@currbox-BANG}%
%</trace>
       \end{teX}
% Need to check that there is room on the page, using the local value
% of |\@textmin| to make the necessary adjustment to |\@dbltoproom|.
       \begin{teX}
           \advance \@dbltoproom \@textmin
%<*trace>
           \tr@ce{Local value of texmin: \the\@textmin}%
           \tr@ce{\@spaces space on page = \the \@dbltoproom.
                           Ht float: \the \ht \@currbox-BANG}%
%</trace>
           \ifdim \@dbltoproom>\ht\@currbox
             \@tempswatrue
%<*trace>
             \tr@ce{Space OK BANG: space on page = \the \@dbltoproom >
                                              \the \ht \@currbox}%
           \else
             \tr@ce{fpstype: \the \@fpstype}%
             \tr@ce{Fail---no room dbltoproom-BANG?:}%
             \tr@ce{\@spaces space on page = \the \@dbltoproom.
                           Ht float: \the \ht \@currbox}%
%</trace>
           \fi
           \advance \@dbltoproom -\@textmin
%<*trace>
         \else
           \tr@ce{fpstype: \the \@fpstype}%
           \tr@ce{Fail---no room dbltoproom-ORD?:}%
           \tr@ce{\@spaces \@dbltoproom = \the \@dbltoproom.
                           Ht float: \the \ht \@currbox}%
%</trace>
         \fi
       \fi
       \if@tempswa
           \@bitor \@currtype \@dbldeferlist
%<*trace>
           \tr@ce{dbldeferlist: \@dbldeferlist: (before)}%
%</trace>
           \if@test
%<*trace>
              \tr@ce{type already on list: dbldefer}%
%</trace>
           \else
              \@tempdima -\ht\@currbox
              \advance\@tempdima
                -\ifx \@dbltoplist\@empty \dbltextfloatsep \else
                                          \dblfloatsep \fi
              \global \advance \@dbltoproom \@tempdima
              \global \advance \@colht \@tempdima
              \global \advance \@dbltopnum \m@ne
              \@cons \@dbltoplist \@currbox
%<*trace>
              \tr@ce{dbltopnum (after) = \the \@dbltopnum}%
              \tr@ce{***Success: dbltop}%
%</trace>
              \@inserttrue
           \fi
       \fi
%<*trace>
     \else
       \tr@ce{Fail: dbltopnum = \the \@dbltopnum: fpstype
                                                  \the \@fpstype=ORD?}%
       \ifnum \@fpstype<\sixt@@n
         \tr@ce{ERROR: !t float not successful (addtodblcol)}%
       \fi
%</trace>
     \fi
   \fi
   \if@insert
   \else
%<*trace>
     \tr@ce{put on dbldeferlist}%
%</trace>
     \@cons\@dbldeferlist\@currbox
%<*trace>
     \tr@ce{dbldeferlist: \@dbldeferlist: (after)}%
%</trace>
   \fi
%<*trace>
   \tr@ce{End of addtodblcol -- locally count:}%
   \tr@ce{ dbltop: \the \@dbltopnum.}%
%</trace>
  \endgroup
%<*trace>
  \tr@ce{End of addtodblcol -- globally count:}%
  \tr@ce{dbltop: \the \@dbltopnum.}%
%</trace>
}
%</2ekernel|autoload|fltrace>
       \end{teX}
%  \end{docCommand}
%
%
%
\begin{docCommand}{@addmarginpar}{}
 Defining the marginpar placement function
\end{docCommand}
       \begin{teX}
%<*2ekernel|autoload>
\def\@addmarginpar{\@next\@marbox\@currlist{\@cons\@freelist\@marbox
    \@cons\@freelist\@currbox}\@latexbug\@tempcnta\@ne
    \if@twocolumn
        \if@firstcolumn \@tempcnta\m@ne \fi
    \else
      \if@mparswitch
         \ifodd\c@page \else\@tempcnta\m@ne \fi
      \fi
      \if@reversemargin \@tempcnta -\@tempcnta \fi
    \fi
    \ifnum\@tempcnta <\z@  \global\setbox\@marbox\box\@currbox \fi
    \@tempdima\@mparbottom
    \advance\@tempdima -\@pageht
    \advance\@tempdima\ht\@marbox
    \ifdim\@tempdima >\z@
      \@latex@warning@no@line {Marginpar on page \thepage\space moved}%
    \else
      \@tempdima\z@
    \fi
    \global\@mparbottom\@pageht
    \global\advance\@mparbottom\@tempdima
    \global\advance\@mparbottom\dp\@marbox
    \global\advance\@mparbottom\marginparpush
    \advance\@tempdima -\ht\@marbox
       \end{teX}
       
 Putting box movement inside the `marbox':
       \begin{teX}
    \global\setbox \@marbox
                   \vbox {\vskip \@tempdima
                          \box \@marbox}%
    \global \ht\@marbox \z@
    \global \dp\@marbox \z@
       \end{teX}
% Sticking (rather than gluing:-) the `marbox' to the line above,
% changed vskip to kern:
       \begin{teX}
    \kern -\@pagedp
    \nointerlineskip
    \hb@xt@\columnwidth
      {\ifnum \@tempcnta >\z@
          \hskip\columnwidth \hskip\marginparsep
       \else
          \hskip -\marginparsep \hskip -\marginparwidth
       \fi
       \box\@marbox \hss}%
       \end{teX}
%    For this reason the following code can vanish:
%\begin{verbatim}
%    \nobreak             %% No longer needed.  CAR92/12
%    \vskip -\@tempdima   %% No longer needed.  CAR92/12
%\end{verbatim}
       \begin{teX}
    \nointerlineskip
    \hbox{\vrule \@height\z@ \@width\z@ \@depth\@pagedp}}
%</2ekernel|autoload>
       \end{teX}
 
%
 \subsection{Kludgeins}

 This part of the file is part of the implementation of the following
 two new commands for \LaTeX2e{}.


 \begin{verbatim}
 \enlargethispage{<dim>}
 \end{verbatim}

 Adds |<dim>| to the height of the current column only. On the printed
 page the bottom of this column is extended downwards by exactly
 |<dim>| without having any effect on the placement of the footer; this
 may result in an overprinting.

 \begin{verbatim}
 \enlargethispage*{<dim>}
 \end{verbatim}

 Similar to |\enlargethispage| but it tries to squeeze the column to
 be printed in as small a space as possible, ie it uses any
 shrinkability in the column. If the column was not explicitly broken
 (\eg with |\pagebreak|) this may result in an overfull box message but
 execpt for this it will come out as expected (if you know what to
 expect).

 The star form of this command is dedicated to Leslie Lamport, the
 other we need for ourselves (FMi, CAR).

 These commands may well have unwanted effects if used soon
 before a |\clearpage|: please give keep them clear of such places.

  \begin{docCommand}{@kludgeins}{}
% \changes{v0.1c}{1993/11/23}{Insert added}
    The insert which makes \TeX{} do a lot of the necessary work.
    All we need to put into it is the amount by which the pagegoal
    should be changed.
       \begin{teX}
%<*2ekernel|def1>
\newinsert \@kludgeins
\global\dimen\@kludgeins \maxdimen
\global\count\@kludgeins 1000
%</2ekernel|def1>
       \end{teX}
  \end{docCommand}
%
%
%  \begin{docCommand}{enlargethispage}
%  \begin{docCommand}{enlargethispage*}
% \changes{v0.1c}{1993/11/23}{Commands added}
%    The user command.
       \begin{teX}
%<*2ekernel|def1>
\gdef \enlargethispage {%
   \@ifstar
     {%
%<*trace>
      \tr@ce{Enlarging page height * }%
%</trace>
      \@enlargepage{\hbox{\kern\p@}}}%
     {%
%<*trace>
      \tr@ce{Enlarging page height exactly---}%
%</trace>
      \@enlargepage\@empty}%
}
%</2ekernel|def1>
%<*autoload>
\def\enlargethispage{\@autoload{out1}\enlargethispage}
%</autoload>
       \end{teX}
%  \end{docCommand}
%  \end{docCommand}
%
%
%  \begin{docCommand}{@enlargepage}
% \changes{v0.1c}{1993/11/23}{Command added}
%    This actually inserts the insert, after checking for extreme
%    values of the change.
       \begin{teX}
%<*2ekernel|def1>
\gdef\@enlargepage#1#2{%
%<*trace>
   \tr@ce{\@spaces\@spaces by #2}%
%</trace>
   \@tempskipa#2\relax
   \ifdim \@tempskipa>.5\maxdimen
     \@latexerr{Suggested\space extra\space height\space
                (\the\@tempskipa)\space dangerously\space
                large}\@eha
   \else
     \ifdim \vsize<.5\maxdimen
%<*trace>
       \tr@ce {Kludgeins added--pagegoal before: \the\pagegoal}%
%</trace>
       \@bsphack
         \insert\@kludgeins{#1\vskip-\@tempskipa}%
       \@esphack
       \end{teX}
%    This next bit is for tracing only:
       \begin{teX}
%<*trace>
       \ifvmode \par
         \tr@ce {Kludgeins added--pagegoal after: \the \pagegoal}%
       \fi
%</trace>
     \else
       \@latexerr{Page\space height\space already\space 
                  too\space large}\@eha
     \fi
   \fi
}
%</2ekernel|def1>
       \end{teX}
%  \end{docCommand}
%
 \section{Float control}

 This part implements controllable floats and other changes
 to the float mechanism.

 It provides, at the document level, the following command for
 inclusion in \LaTeX2e{}.
%
% \begin{verbatim}
%     \suppressfloats
% \end{verbatim}
%
% This suppresses all further floats on the current page.
%
% With an optional argument it suppresses only floats only in certain
% positions on the current page.
% \begin{quote}
%  |[t]|\quad suppresses only floats at the top of the page
%  |[b]|\quad suppresses only floats at the bottom of the page
% \end{quote}
%
% It also enables the use of an extra specifier, {\tt !}, in the
% location optional argument of a float.  If this is present then,
% just for this particular float, whenever it is processed by the float
% mechanism the followinhg are ignored:
%
% \begin{itemize}
% \item  all restrictions on the number of floats which can appear;
% \item  all explicit restrictions on the amount of space which should
%   (not) be occupied by floats and/or text.
% \end{itemize}
%
% The mechanism will still attempt to ensure that pages are not
% overfull.
%
% These specifiers override, for the single float, the suppression
% commands described above.
%
%
% In its current form, it also suplies a reasonably exhaustive, and
% somewhat baroque, means of tracing some aspects of the float
% mechanism.
%
% More tracing.
%  \begin{docCommand}{tr@ce}
%  \begin{docCommand}{notrace}
%  \begin{docCommand}{tracefloats}
%  \begin{docCommand}{@traceval}
%  \begin{docCommand}{tracefloatvals}
%  \begin{docCommand}{@tracemessage}
%    Set-up tracing for floats independent of other tracing as it
%    produces mega-output.  Default is no tracing.
% \changes{v1.1j}{1995/04/24}
%   {Do not add to kernel unless `trace' specified}
% \task{???}{Make proper tracing module}
%
       \begin{teX}
%<*trace>
\def \@tracemessage #1{\typeout{LaTeX2e: #1}}
\def \tracefloats{\let \tr@ce \@tracemessage}
\def \notrace {\let \tr@ce \@gobble}
\notrace
\def \@traceval #1{\tr@ce{\string #1 = \the #1}}
\def \tracefloatvals{%
  \@dblfloatplacement
  \@floatplacement
  \@traceval\@colnum
  \@traceval\@colroom
  \@traceval\@topnum
  \@traceval\@toproom
  \@traceval\@botnum
  \@traceval\@botroom
  \@traceval\@fpmin
  \tr@ce{\string\textfraction = \textfraction}%
  \@traceval\@dbltopnum
  \@traceval\@dbltoproom
}
%</trace>
%<*flafter>
\providecommand\tr@ce[1]{}
%</flafter>
       \end{teX}
%  \end{docCommand}
%  \end{docCommand}
%  \end{docCommand}
%  \end{docCommand}
%  \end{docCommand}
%  \end{docCommand}
%

  \begin{docCommand}{suppressfloats}{}
  \begin{docCommand}{@flstop}{}
 Float suppression commands: these set the relevant counter
 globally to zero.  Thus they are overridden for a particular float
 by an ! specifier.

    \begin{teX}
\def \suppressfloats {%
   \@ifnextchar [%
     \@flstop
    {\global \@colnum \z@}%
}
    \end{teX}
 Maybe this should be a loop over |#1|?
    \begin{teX}
\def \@flstop [#1]{%
   \if t#1%
     \global \@topnum \z@
   \fi
   \if b#1%
     \global \@botnum \z@
   \fi
}
    \end{teX}
  \end{docCommand}
  \end{docCommand}


 Manipulation of float placement and type; both their strings and the
 corresponding count registers.

  \begin{docCommand}{@fpstype}{}
  \end{docCommand}
  \begin{docCommand}{@reqcolroom}{}
  \begin{docCommand}{@textfloatsheight}{}
 First a new count register to go with |\@currtype|.

 Then a new skip register, for information needed to remove the
 |\@maxsep| conservatism: it is possible that this could use a
 temporary register.

 Finally a dimension register to hold the total height of in-text
 floats on the current page.  This is needed to implement a
 major change in the functionality of |\@addtocurcol| which is,
 nevertheless, a bug fix.
 It is not local and therefore cannot be a temporary register.

       \begin{teX}
\newcount \@fpstype
\newdimen \@reqcolroom
\newdimen \@textfloatsheight
       \end{teX}
  \end{docCommand}
  \end{docCommand}
%
%  \begin{docCommand}{@fpsadddefault}
% \changes{v1.0f}{1993/12/05}{Command added}
% Adds the default placement to what is already there.
% 
% Should not need to change this, but could do it as follows:
% \begin{verbatim}
%\def \@fpsadddefault {%
%   \@temptokena \expandafter\expandafter\expandafter
%                {\csname fps@\@captype \endcsname}%
%   \edef \reserved@a {\the\@temptokena}%
%   \@onelevel@sanitize \reserved@a
%   \edef \@fps {\@fps\reserved@a}%
%}
% \end{verbatim}
%
       \begin{teX}
%<*2ekernel|autoload|fltrace>
\def \@fpsadddefault {%
%<*trace>
   \tr@ce{fps changed from: \@fps}%
%</trace>
   \edef \@fps {\@fps\csname fps@\@captype \endcsname}%
   \@latex@warning {%
     No positions in optional float specifier.\MessageBreak
     Default added (so using `\@fps')}%
}
       \end{teX}
%  \end{docCommand}
%
%  \begin{docCommand}{@setfloattypecounts}
% Sets counters |\@fpstype| and |\@currtype|.
%
% BANG $==$ bit4 of $|\count\@currbox| = 0$.
%
       \begin{teX}
\def \@setfloattypecounts {%
  \@currtype \count\@currbox
  \@fpstype \count\@currbox
  \divide\@currtype\@xxxii \multiply\@currtype\@xxxii
  \advance \@fpstype -\@currtype
%<*trace>
  \tr@ce{(mod 32) fpstype: \the \@fpstype}%
  \tr@ce{(mult of 32) currtype: \the \@currtype}%
% Tracing only: but some should be changed into real errors/warnings?
  \ifnum \@fpstype<\sixt@@n
    \ifnum \@fpstype=\z@
      \tr@ce{ERROR: no PLACEMENT, fpstype = \the \@fpstype = 0?}%
    \fi
    \ifnum \@fpstype=\@ne
      \tr@ce{WARNING: only h, fpstype = \the \@fpstype = 1?}%
    \fi
    \tr@ce{BANG float}%
  \else
    \ifnum \@fpstype=\sixt@@n
      \tr@ce{ERROR: no PLACEMENT, fpstype = \the \@fpstype = 16?}%
    \fi
    \ifnum \@fpstype=17
      \tr@ce{WARNING: only h, fpstype = \the \@fpstype = 17?}%
    \fi
    \tr@ce{ORD float}%
  \fi
%</trace>
}
       \end{teX}
%  \end{docCommand}
%
 Macros for getting, testing and setting bits of the fps.
  \begin{docCommand}{@getfpsbit}{}
 Sets |\@tempcnta| to required bit of |\count\@currbox|.

    \begin{teX}
\def \@getfpsbit {%
   \@boxfpsbit \@currbox
}
    \end{teX}
  \end{docCommand}
%
%
  \begin{docCommand}{@boxfpsbit}{}
    Used above.
    \begin{teX}
\def \@boxfpsbit #1#2{%
   \@tempcnta \count#1%
   \divide \@tempcnta #2\relax
}
    \end{teX}
  \end{docCommand}

  \begin{docCommand}{@testfp}{}
 New definition of the float page test.
    \begin{teX}
\def \@testfp #1{%
   \@boxfpsbit #18\relax % Really `#1 8' for human readers!
   \ifodd \@tempcnta
   \else
     \@testtrue
   \fi
}
    \end{teX}
  \end{docCommand}


  \begin{docCommand}{@setfpsbit}{}
 Sets required bit of |\@tempcnta| (to 1). \changes{v1.0f}{1993/12/05}{Command added}.
 This is used earlier by the @float command to set the bitset of a particular float. 

       \begin{teX}
\def \@setfpsbit #1{%
   \@tempcntb \@tempcnta
   \divide \@tempcntb #1\relax
   \ifodd \@tempcntb
   \else
     \advance \@tempcnta #1\relax
   \fi
}
       \end{teX}
  \end{docCommand}


  \begin{docCommand}{@resethfps}{}
 Globally adds t as a possible location for an h or !h only placement:
 this must be done using the count.

 Although it will leave |\@fpstype| set to 17 even if it was
 originally 1, this does not matter since it is the last thing in
 |\@addtocurcol|. 
    \begin{teX}
\def \@resethfps {%
   \let\reserved@a\@empty
   \ifnum \@fpstype=\@ne
      \def \reserved@a {!}%
      \@fpstype 17
   \fi
   \ifnum \@fpstype=17
     \global \advance \count\@currbox \tw@
     \@latex@warning@no@line {%
       `\reserved@a h' float specifier changed to `\reserved@a ht'}%
   \fi
}
    \end{teX}
  \end{docCommand}


 Special stuff for BANG floats.
 
  \begin{docCommand}{@flsetnum}{}

 Ignores any zero float counter value in case BANG.

 It uses a local assignment to the normally global counter: a bit
 naughty, perhaps?

 These assgnments are safe so long as the counter involved is only
 consulted once (\ie only for the `bang float') with the changed value.
 This is the case within |\@addtocurcol| because it is used only
 once within a call of the output routine (which forms a group).

 For |\@addtonextcol| this is achieved by putting a group around its
 code; this is needed because it is called (by |\@startcolumn|) for
 each float which was on the deferlist.  Almost identical
 considerations pertain to |\@addtodblcol|.  There may be more
 efficient ways to handle this, but the group seems to be the simplest.

       \begin{teX}
\def \@flsetnum #1{%
%<*trace>
   \tr@ce{fpstype: \the \@fpstype (flsetnum \string#1)}%
%</trace>
   \ifnum \@fpstype<\sixt@@n
     \ifnum #1=\z@
%<*trace>
       \tr@ce{BANG float resetting \string#1 to 1}%
%</trace>
       #1\@ne
     \fi
   \fi
%<*trace>
   \tr@ce{#1 (before) = \the #1}%
%</trace>
}
       \end{teX}
  \end{docCommand}


  \begin{docCommand}{@flsettextmin}{}
 This ignores |\textfraction| space restriction in case BANG.

       \begin{teX}
\def \@flsettextmin {%
%<*trace>
   \tr@ce{fpstype: \the \@fpstype (flsettextmin)}%
%</trace>
   \ifnum \@fpstype<\sixt@@n
%<*trace>
     \tr@ce{BANG ignoring textmin}%
%</trace>
     \@textmin \z@
   \else
     \@textmin \textfraction\@colht
%<*trace>
     \tr@ce{ORD textmin = \the \@textmin}%
%</trace>
   \fi
}
       \end{teX}
  \end{docCommand}


  \begin{docCommand}{@flcheckspace}{}
 This ignores space restriction in case BANG; this is still slightly
 conervative since it does not allow for the fact that, if there is
 no text in the column then |\textfloatsep| is not needed.
 Sets |@tempswa| true if there is room for |\@currbox|.
  \end{docCommand}
       \begin{teX}
\def \@flcheckspace #1#2{%
   \advance \@reqcolroom
     \ifx #2\@empty \textfloatsep \else \floatsep \fi
%<*trace>
   \tr@ce{colroom = \the \@colroom (flcheckspace \string#1 \string#2)}%
   \tr@ce{reqcolroom = \the \@reqcolroom
                                   (flcheckspace \string#1 \string#2)}%
%</trace>
   \ifdim \@colroom>\@reqcolroom
     \ifdim #1>\ht\@currbox
       \@tempswatrue
%<*trace>
       \tr@ce{Space OK: #1 = \the #1 > \the \ht \@currbox
                                   (flcheckspace \string#1 \string#2)}%
%</trace>
     \else
%<*trace>
       \tr@ce{fpstype: \the \@fpstype
                                   (flcheckspace \string#1 \string#2)}%
%</trace>
       \ifnum \@fpstype<\sixt@@n
%<*trace>
         \tr@ce{BANG float ignoring #1
                                   (flcheckspace \string#1 \string#2):}%
         \tr@ce{\@spaces #1 = \the #1.  Ht float: \the \ht \@currbox
                                                          BANG}%
%</trace>
         \@tempswatrue
%<*trace>
       \else
         \tr@ce{Fail---no room (flcheckspace \string#1 \string#2)
                       (fpstype \the \@fpstype=ORD?):}%
         \tr@ce{\@spaces #1 = \the #1.  Ht float: \the \ht \@currbox
                                                          ORD?}%
%</trace>
       \fi
     \fi
%<*trace>
   \else
     \tr@ce{Fail---no room at 2nd test of colroom
                   (flcheckspace \string#1 \string#2)}%
%</trace>
   \fi
}
       \end{teX}



  \begin{docCommand}{@flupdates}{}
    This updates everything when a float is placed.

       \begin{teX}
\def \@flupdates #1#2#3{%
   \global \advance #1\m@ne
   \global \advance \@colnum \m@ne
   \@tempdima -\ht\@currbox
   \advance \@tempdima
     -\ifx #3\@empty \textfloatsep \else \floatsep \fi
   \global \advance #2\@tempdima
   \global \advance \@colroom \@tempdima
   \@cons #3\@currbox
}
       \end{teX}
  \end{docCommand}

 Interesting facts about float mechanisms past and present, together
 with a summary of various features, some unresolved:

 \begin{enumerate}
   \item  The value |\textfraction| does not affect the processing
     of doublecol floats: this seems sensible, but should be
     documented.
   \item |\twocolumn| floatplacement was wrong: dbl not needed, ord
     needed.
   \item |\@floatplacement| was not called after |\@startdblcol|
     or |\@topnewpage|.  This has been changed; it is clearly a bug
     fix.
   \item The use |\@topnewpage| when |\dblfigrule| is non-trivial
   produced a rule in the wrong place.  This has been fixed by not
   using |\dblfigrule| when processing the `float' from
   |\@topnewpage|. 
   \item  If the specifier was just h and the float could not be put
     here, it went on the deferlist and stayed there until a clearpage.
     It now gets changed to a `th': this is only an error-recovery
     action, putting just h or !h should be deprecated.   
   \item |\@dblmaxsep| was `the maximum of |\dblfloatsep| and
     |\dbltexfloatsep|'. But it was never used!  Now gone completely,
     like |\@maxsep|.
   \item After an h float is put on a page, it was counted as text when
     applying the |\textfraction| test; this is possibly too big a
     change although it is a bug fix?
   \item  Two consecutive h floats are separated by twice |\intextsep|:
     this could be changed to one by use of |\addvspace|, OK?
     Note that it would also mean that less space is put in if an h
     float  immediaiely follows other spaces.  This is also possibly
     too big a change, at least for compatibility mode?
     Or it may be simply wrong!  It has not been changed.
   \item Now |\@addtocurcol| checks first for just p fps.  I think
     that this is an increase in efficiency, but maybe the coding
     should be made even more efficient.
   \item |\@tryfcolumn| now tests if the list is empty first, otherwise
     lots of wasted time!  Thus this test has been removed from
     |\@startcolumn|.
     As Frank pointed out, this makes |\@startcolumn| less
     efficient. But it is now the same as |\@startdblcolumn|: I can
     see no reason why they should be different, but which is best?
     
   \item Why is |\@colroom| set in |\@doclearpage|?
%   \item  Footnotes. Check what |\clearpage| does when footnotes are
%     left over.  Footnotes are not put on float pages and, also,
%     |\@addtonextcol| ignores the existence of held-over footnotes
%     in deciding what floats can go on the page.  Not changed.
%   \item  |\clearpage| can still lose non-boxes, at least when floats
%     are involved.  It also moves some to the `wrong page', but this
%     may be a coding problem.
%   \item  The ! option makes it necessary to check in |\output| that
%     there is enough room left on the page after adding a float.  (This
%     would have been necessary anyway if anyone set |\@textmin| too
%     close to zero!  A similar danger existed also if the text in a
%     |\twocolumn[text]| entity gets too large.)
%     The current implementation of this also makes the normal case a
%     little less efficient, OK?
%     Not enough room means, at present, less than  |\baselineskip|,
%     with a warning: is this OK?  Should it be made generic (another
%     parameter)?
%   \item  There are four possibilities for supporting this:
%
%     |\twocolumn[\maketitle more text]|
%
%     One is to change
%     |\maketitle| slightly to allow this.  Another is to change
%     |\@topnewpage| so that more than one |\twocolumn[]| command is
%     allowed; in this case |\maketitle\twoclumn[more text]| will work.
%     The former is more robust from the user's viewpoint, but makes the
%     code for |\maketitle| rather ad hoc (maybe it is already?).
%     Another is to misuse the global twocolumn flag locally within
%     |\@topnewpage|.
%     Yet another is to move the column count register from the multicol
%     package into the kernel.  This has beeen done.
%   \item  Where should the reinserts be put to maximise the
%     probability that footmotes come out on the correct page?
%     Or should we go for as much compatibility as possible (but see
%     next item)?
%   \item  Should we continue to support (as much as possible)
%     |\samepage|?  Some of its intended functionality is now advertised
%     as being provided by |\enlargethispage|.  Use of either is likely
%     to result in wrongly placed footnotes, marginals, etc.
%     Which should have priority: obeying the pagination instructions,
%     or correct placement of notes/marginalia?
%   \item  Is the adjustment of space to cause shrinking in the
%      kludge-* case correct?  Should it be limited to 0pt?
%   \item  Is the setting of |\boxmaxdepth| in makecol and friends
%     needed?  It only has any effect if |\@textbottom| ends with a box
%     or rule, in which case the vskip to allow for its depth should
%     also be added.  If it is kept, it should probably be the last
%     thing in the box.  It has now been removed.
%     
%     It would perhaps be better to document that |\@textbottom|
%     and |\@texttop| must have natural height 0pt.
%   \item  I cannot see why the vskip adjustement for the depth
%     is needed if boxmaxdepth is used to ensure that there is never
%     a too deep box.
%   \item  The value of |\boxmaxdepth| should be explicitly set
%     whenever necessary: it is too risky to assume that it has any
%     particular value.  Care is needed in deciding what to set it to.
%
%     It is interesting to note that the value of |\boxmaxdepth| is
%     unique in being read before the local settings for the box group
%     are reset; all other parameter settings which affect the box
%     construction use their values outside the box group.
   \item  Should |\@maxdepth| store the setting of |\maxdepth| from
     lplain?  Or should we provide a proper interface to class files
     for setting these? 
 \end{enumerate}
%
% An analysis of various other macros.
%
%    |\@opcol| should do |\@floatplacement|, but where?  Right at the
%    end, since it always occurs at the start of a column.
% \begin{verbatim}
% \def\@opcol{%
%   % Why is this done first?
%   \global \@mparbottom \z@
%   \if@twocolumn
%     \@outputdblcol
%   \else
%     \@outputpage
%     % This is not needed since it is done at the end of
%     %   |\@outputpage|:
%     \global \@colht \textheight
%   \fi}
% \end{verbatim}
%
% Only tracing has been added to these.
%
       \begin{teX}
\def\@makefcolumn #1{%
  \begingroup
    \@fpmin \z@
    \let \@testfp \@gobble
    \@tryfcolumn #1%
  \endgroup
}
       \end{teX}
 This will line up the last baselines in the two
 columns provided they are constructed in the normal way: \ie ending
 in a skip of minus the original depth, with |\@textbottom| adding
 nothing. 

 Thus again it is essential for |\@textbottom| to have depth 0pt.
 \footnotechanges{1.2g}{2000/07/12}{Ensure that rule is in \cs{normalcolor}}
       \begin{teX}
\def\@outputdblcol{%
  \if@firstcolumn
    \global \@firstcolumnfalse
    \global \setbox\@leftcolumn \box\@outputbox
%<*trace>
    \tr@ce{PAGE: first column boxed}%
%</trace>
  \else
    \global \@firstcolumntrue
    \setbox\@outputbox \vbox {%
                         \hb@xt@\textwidth {%
                           \hb@xt@\columnwidth {%
                             \box\@leftcolumn \hss}%
                           \hfil
                           {\normalcolor\vrule \@width\columnseprule}%
                           \hfil
                           \hb@xt@\columnwidth {%
                             \box\@outputbox \hss}%
                                             }%
                              }%
%<*trace>
    \tr@ce{PAGE: second column also boxed}%
%</trace>
    \@combinedblfloats
    \@outputpage
%<*trace>
    \tr@ce{PAGE: two column page completed}%
%</trace>
    \begingroup
      \@dblfloatplacement
      \@startdblcolumn
       \end{teX}
%    This loop could be replaced by an |\expandafter| tail
%    recursion in\\ |\@startdblcolumn|.
       \begin{teX}
      \@whilesw\if@fcolmade \fi
        {\@outputpage
%<*trace>
      \tr@ce{PAGE: double float page completed}%
%</trace>
         \@startdblcolumn}%
    \endgroup
  \fi
}

       \end{teX}
%
 \subsubsection{Float placement parameters}

 The main purpose of this section is to ensure that all the
 float-placement parameters which need to be set in a class file or
 package have been declared.  It also describes their use and sets
 values for them which are reasonable for typical documents using
 US letter or A4 sized paper. Unlike many other parameters that
\LaTeXe\ leaves to be determined by the classes default values are
entered here. 
 
 \subsubsection{Limits for the placement of floating objects}

 \begin{docCommand}{c@topnumber}{}
    This counter holds the maximum number of
    floats that can appear at the top of a text page or column.
    \begin{teX}
\newcount\c@topnumber
\setcounter{topnumber}{2}
    \end{teX}
 \end{docCommand}

 \begin{docCommand}{topfraction}{}  \label{topfraction}
    This macro holds the maximum proportion (as a decimal number) of
    a text page or column that can be occupied by floats at the top. In the
    |phd| package, we set this as |.85|. Many of the images we use are quite
    high and a value of |.85| is more appropriate.
    \begin{teX}
\newcommand\topfraction{.7}
    \end{teX}
 \end{docCommand}


 \begin{docCommand}{c@bottomnumber}{}
    This counter holds the maximum number of
    floats that can appear at the bottom of a text page or column. Lamport's value to
    allow only one is reasonable for many publications, but fails in many others. In the |phd|
    package we allowed a much larger number as we are aiming for more compact documents.
    \begin{teX}
\newcount\c@bottomnumber
\setcounter{bottomnumber}{1}
    \end{teX}
 \end{docCommand}

 \begin{docCommand}{bottomfraction}{}
    This macro holds the maximum proportion (as a decimal number) of
    a text page or column that can be occupied by floats at the bottom.
    \begin{teX}
\newcommand\bottomfraction{.3}
    \end{teX}
 \end{docCommand}

 \begin{docCommand}{c@totalnumber}{}
    This counter holds the maximum number of floats that can appear on
    any text page or column.
    \begin{teX}
\newcount\c@totalnumber
\setcounter{totalnumber}{3}
    \end{teX}
 \end{docCommand}

 \begin{docCommand}{textfraction}{}
    This macro holds the minimum proportion (as a decimal number) of
    a text page or column that must be occupied by text.
    \begin{teX}
\newcommand\textfraction{.2}
    \end{teX}
 \end{docCommand}

 \begin{docCommand}{floatpagefraction}{}
    This macro holds the minimum proportion (as a decimal number) of
    a page or column that must be occupied by floating objects before a
    `float page' is produced.
    \begin{teX}
\newcommand\floatpagefraction{.5}
    \end{teX}
 \end{docCommand}

 \begin{docCommand}{c@dbltopnumber}{}
    This counter holds the maximum number of double-column floats that
    can appear on the top of a two-column text page.
    \begin{teX}
\newcount\c@dbltopnumber
\setcounter{dbltopnumber}{2}
    \end{teX}
 \end{docCommand}

 \begin{docCommand}{dbltopfraction}{}
    This macro holds the maximum proportion (as a decimal number) of
    a two-column text page that can be occupied by double-column floats
    at the top.
    \begin{teX}
\newcommand\dbltopfraction{.7}
    \end{teX}
 \end{docCommand}

 \begin{docCommand}{dblfloatpagefraction}{}
    This macro holds the minimum proportion (as a decimal number) of
    a page that must be occupied by double-column floating objects
    before a `double-column float page' is produced.
    \begin{teX}
\newcommand\dblfloatpagefraction{.5}
    \end{teX}
 \end{docCommand}

 \subsection{Floats on a text page}

 \begin{docCommand}{floatsep}{}
 \begin{docCommand}{textfloatsep}{}
 \begin{docCommand}{intextsep}{}
    When a floating object is placed on a page with text, these
    parameters control the seperation between the float and the other
    objects on the page. These parameters are used for both
    one-column mode and single-column floats in two-column mode.
    They are all rubber lengths.

    |\floatsep| is the space between adjacent floats that are placed
    at the top or bottom of the text page or column.

    |\textfloatsep| is the space between the main text and floats
    at the top or bottom of the page or column.

    |\intextsep| is the space between in-text floats and the text.
    \begin{teX}
\newskip\floatsep
\newskip\textfloatsep
\newskip\intextsep
\setlength\floatsep    {12\p@ \@plus 2\p@ \@minus 2\p@}
\setlength\textfloatsep{20\p@ \@plus 2\p@ \@minus 4\p@}
\setlength\intextsep   {12\p@ \@plus 2\p@ \@minus 2\p@}
    \end{teX}
 \end{docCommand}
 \end{docCommand}
 \end{docCommand}

 \begin{docCommand}{dblfloatsep}{}
 \begin{docCommand}{dbltextfloatsep}{}
    When double-column floats (floating objects that span the whole
    |\textwidth|) are placed at the top of a text page in two-column
    mode, the separation between the float and the text is controlled
    by |\dblfloatsep| and |\dbltextfloatsep|.  They are rubber lengths.

    |\dblfloatsep| is the space between adjacent double-column floats
    placed at the top of the text page.

    |\dbltextfloatsep| is the space between the main text and
    double-column floats at the top of the page.
    \begin{teX}
\newskip\dblfloatsep
\newskip\dbltextfloatsep
\setlength\dblfloatsep    {12\p@ \@plus 2\p@ \@minus 2\p@}
\setlength\dbltextfloatsep{20\p@ \@plus 2\p@ \@minus 4\p@}
    \end{teX}
 \end{docCommand}
 \end{docCommand}

 \subsection{Floats on their own page or column}
 
    When floating objects are placed on a seperate page or column this
    is called a `float page', the layout of the page is controlled by
    these parameters, which are rubber lengths.
    
 \begin{docCommand}{@fptop}{}
 \begin{docCommand}{@fpsep}{}
 \begin{docCommand}{@fpbot}{}
   
    At the top of the page |\@fptop| is inserted;
    typically this supplies some stretchable whitespace.
    At the bottom of the page |\@fpbot| is inserted.
    Between adjacent floats |\@fpsep| is inserted.

    These parameters are used for all floating objects on a
    `float page' in one-column mode, and for single-column
    floats in two-column mode.

    Note that at least one of the two parameters |\@fptop| and
    |\@fpbot| should contain a |plus ...fil| so as to fill the
    remaining empty space.
    \begin{teX}
\newskip\@fptop
\newskip\@fpsep
\newskip\@fpbot
\setlength\@fptop{0\p@ \@plus 1fil}
\setlength\@fpsep{8\p@ \@plus 2fil}
\setlength\@fpbot{0\p@ \@plus 1fil}
    \end{teX}
 \end{docCommand}
 \end{docCommand}
 \end{docCommand}


 \begin{docCommand}{@dblfptop}{}
 \begin{teX}
   \newskip\@dblfptop
 \end{teX}
 \end{docCommand}
 
 \begin{docCommand}{@dblfpsep}{}
 \begin{docCommand}{@dblfpbot}{}
    Double-column `float pages' in two-column mode use similar
    parameters.
    \begin{teX}

\newskip\@dblfpsep
\newskip\@dblfpbot
\setlength\@dblfptop{0\p@ \@plus 1fil}
\setlength\@dblfpsep{8\p@ \@plus 2fil}
\setlength\@dblfpbot{0\p@ \@plus 1fil}
    \end{teX}
 
 \end{docCommand}
 \end{docCommand}
 
 \begin{docCommand}{topfigrule}{}
  
 \begin{docCommand}{botfigrule}{}
 \end{docCommand}
 \begin{docCommand}{dblfigrule}{}
  \end{docCommand}
    The macros can be used to put in rules between floats and text;
    whatever they insert should be vertical mode material which takes
    up zero space.
 \task{CAR}{Add more rules (for Frank in addtocurcol)}
    \begin{teX}
\let\topfigrule=\relax
\let\botfigrule=\relax
\let\dblfigrule=\relax
    \end{teX}
\end{docCommand}
 
 

