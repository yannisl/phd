\chapter{Kernel Lengths}

\epigraph{A sign that my vision, rapid like all visions, if it had not lasted the space of an ``amen,'' as the
saying goes, had lasted almost the length of a ``\textit{Dies irae}.''}{Umbeco Eco, ``The Name of the Rose'', 1983}

\label{kernel:lengths}
\index{LaTeX kernel classes!File n  ltlength.dtx}

\section{File n, lengths and the ltlength.dtx}

This class defines a number of user commands for manipulating lengths. the code is straightforward. The |\newlength| command allocates a new internal skip register using the \docAuxCommand{newskip} command from the allocations module.

\let\bs\textbackslash
\index{\bs newlength}\index{\bs setlength}\index{\bs addtolength}\index{\bs settowidth}\index{\bs settoheight}
\index{\bs settodepth}
\medskip
\begin{tabular}{ll}
\verb+\newlength+  &  Declare \#1 to be a new length command.\\
\verb+\setlength+    &  Set the length command, \#1, to the value \#2.\\
|\addtolength| & Increase the value of the length command, \#1, by the value \#2.\\
|\settowidth|   & Set the length, \#1 to the width of a box containing \#2. \\
|\settoheight|  & Set the length, \#1 to the height of a box containing \#2.\\
|\settodepth|   & Set the length, \#1 to the depth of a box containing \#2.\\
|\@settodim|   & internal macro\\
|\@settopoint| & internal macro\\
\end{tabular}
\medskip

\startlineat{3}
\begin{docCommand}{newlength} {\marg{command name}} 

The \docAuxCommand{newlength} is just syntactic sugar for PlainTeX’s \docAuxCommand{newskip}. The |\@ifdefinable| will produce an
error message if the command has already been set.  
\end{docCommand}

\begin{teXXX}
\def\newlength#1{\@ifdefinable#1{\newskip#1}}
\end{teXXX}

\begin{docCommand}{setlength} {}
\end{docCommand}
\begin{teXXX}
\def\setlength#1#2{#1#2\relax}
\end{teXXX}

\begin{docCommand}{addtolength} {}
\end{docCommand}

\begin{teXXX}
\def\addtolength#1#2{\advance#1 #2\relax}
\end{teXXX}


\begin{texexample}{Lengths}{ex:length}
\bgroup
\newlength\alength
\setlength\alength{123.12pt}
\the\alength

\addtolength\alength{12.0pt}
\the\alength
\egroup
\end{texexample}

The \docAuxCommand{setto} commands use a temporary box \docAuxCommand{@tempboxa} to store the contents and then 
measure them using the internal macro |\@settodim| which boxes the contents and then uses \tex's |ht| and|wd|.

\begin{docCommand}{@settodim}{\marg{}\marg{}\marg{}}
Sets a length to another
\end{docCommand}

\startlineat{6}
\begin{teXXX}
\def\@settodim#1#2#3{\setbox\@tempboxa\hbox{{#3}}#2#1\@tempboxa
%  Clear the memory afterwards (which might be a lot).
  \setbox\@tempboxa\box\voidb@x}
  \def\settodepth {\@settodim\dp}
\end{teXXX}

\begin{docCommand}{settoheight}{\marg{dimen}}
Sets the given length to the height of the box.
\end{docCommand}

\begin{teXXX}
\def\settoheight{\@settodim\ht}
\end{teXXX}

\begin{docCommand}{settowidth}{ \marg{arg}\marg{material}}
Sets the width to be equal to the boxed content..
\end{docCommand}
\begin{teXXX}  
  \def\settowidth {\@settodim\wd}
\end{teXXX}



\begin{docCommand}{@settopoint} {\marg{skip register}}

The \docAuxCommand{@settopoint}\marg{skip register}
 macro takes the contents of the skip register that is supplied as its argument
and removes the fractional part to make it a whole number of points. This can be
used in class files to avoid values like 45.455pt when calulating a dimension. The method of
rounding is interesting. Also it is interesting that this macro, is not used in the kernel at all, but is defined
here for use with the standard classes (it is used to round off dimensions for page calculations).
\end{docCommand}

\startlineat{100}
\begin{texexample}{Rounding dimensions}{ex:settopoint}
\bgroup 
  \makeatletter
  \def\@settopoint#1{\divide#1\p@\multiply#1\p@}
  \newlength\@test
  \setlength\@test{19.5pt}
  \@settopoint{\@test}
  \the\@test
  \makeatother
\egroup  
\end{texexample}

\section{Redefinitions and Extensions by Packages}

The commands \refCom{setlength} and \refCom{addtolength}, \refCom{@settodim} have been reimplemented by the package \pkgname{calc} to enable infix calculations.\footfullcite[See ][The pgf package does not redefine teh original commands, but rather provides numerous commands prefixed with pgfmath]{calc,pkg-pgf}The package also implements a number of other useful
commands which are described below. The \pkg{pgf} prefixes all the commands, so it has no impact on the kernel commands which can still be used. This is a much better approach. 

\medskip
\begin{tabular}{ll}
\refCom{newlength}    &  Declare \#1 to be a new length command.\\
\refCom{setlength}    &  Set the length command, \#1, to the value \#2.\\
\refCom{addtolength}  & Increase the value of the length command, \#1, by the value \#2.\\
\refCom{settowidth}   & Set the length, \#1 to the width of a box containing \#2. \\
\refCom{settoheight}  & Set the length, \#1 to the height of a box containing \#2.\\
|\settodepth|   & Set the length, \#1 to the depth of a box containing \#2.\\
|\widthof|\marg{text} & Gets the natural width of \meta{text}.\\
|\heightof|\marg{text} & \\
\docAuxCommand{depthof}\marg{text} &\\
 |\maxof|  & Select a maximum of two dimension expressions.\\
 |\minof| & Select the minimum of two dimension expressions.\\
|\@settodim|   & internal macro\\
|\@settopoint| & internal macro\\
\end{tabular}
\medskip




