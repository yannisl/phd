\documentclass{article}
\usepackage{tikz}
\usepackage{lipsum}
\usepackage{phd-lorems,varwidth}
\usetikzlibrary{positioning}
\begin{document}

\pgfdeclarelayer{background layer}
\pgfdeclarelayer{foreground layer}
\pgfsetlayers{background layer,main,foreground layer}

{\leftskip-3cm\rightskip-2cm
\parindent=0pt Testing \begin{tikzpicture}
% On main layer:
\fill[blue] (0,0) circle (1cm);
\begin{pgfonlayer}{background layer}
\fill[yellow] (-2,-1) rectangle (2,1);
\end{pgfonlayer}

\begin{pgfonlayer}{foreground layer}
\node[black, anchor=base] {CHAPTER 567};
\node[black, anchor=base] at (3,3) {CHAPTER 567};
\end{pgfonlayer}

\begin{pgfonlayer}{background layer}
\fill[black] (-.8,-.8) rectangle (.8,.8);
\end{pgfonlayer}
% On main layer again:
\fill[blue!50] (-.5,-1) rectangle (.5,1);
\end{tikzpicture}\hskip3pt\lorem
}

\lipsum[2]
\section{One}

\makeatletter
\def\foo{foo}
\pgfdeclareshape{simple shape}{%
\savedanchor{\center}{%
\pgfpointorigin}
\anchor{center}{\center}

\savedanchor{\anchorfoo}{%
  \pgf@x=1cm
\  pgf@y=0cm}

\deferredanchor{anchor \foo}{\anchorfoo}}
\begin{tikzpicture}
\node[simple shape] (Test1) at (0,0) {};
\fill (Test1.anchor foo) circle (2pt) node[below] {anchor foo anchor};
%
\def\foo{bar}
\node[simple shape] (Test2) at (4,0) {};
\fill (Test2.anchor bar) circle (2pt) node[below] {anchor bar anchor};
\end{tikzpicture}

\makeatletter
%\leftskip-4cm
\newbox\titlebox
\setbox\titlebox=\hbox{\begin{varwidth}[c]{10cm}\bf\lorem\end{varwidth}}
\leavevmode\hbox to \linewidth{\hss \begin{tikzpicture}[node distance=0ex,inner sep=0pt,outer sep=0pt,anchor=base west]
\draw[help lines] (0,0) grid (12,2);
\node[circle,inner sep=3pt,draw,fill] (X) at (0,1){\Huge\color{white}1};
\node (a) [base right=of X] {};
\node[anchor=base west] (y) at (a.base east)[xshift=0mm,yshift=0mm, align=left] {\usebox\titlebox};
\end{tikzpicture}}
\hss
\makeatother


\def\pdfliteral#1{\special{pdf:literal #1}}


\def\sqPDF#1#2{0 0 m #1 0 l #1 #1 l 0 #1 l h}
\def\trianPDF#1#2{0 0 m #1 0 l #2 4.5 l h}
\def\uptrianPDF#1#2{#2 0 m #1 4.5 l 0 4.5 l h}
\def\circPDF#1#2{#1 0 0 #1 #2 #2 cm .1 w .5 0 m
   .5 .276 .276 .5 0 .5 c -.276 .5 -.5 .276 -.5 0 c
   -.5 -.276 -.276 -.5 0 -.5 c .276 -.5 .5 -.276 .5 0 c h}
\def\diaPDF#1#2{#2 0 m #1 #2 l #2 #1 l 0 #2 l h}

\def\credCOLOR   {.54 .14 0}
\def\cblueCOLOR  {.06 .3 .54}
\def\cgreenCOLOR {0 .54 0}

\def\genbox#1#2#3#4#5#6{% #1=0/1, #2=color, #3=shape, #4=raise, #5=width, #6=width/2
    \leavevmode\raise#4bp\hbox to#5bp{\vrule height#5bp depth0bp width0bp
    \pdfliteral{q .5 w \csname #2COLOR\endcsname\space RG
                       \csname #3PDF\endcsname{#5}{#6} S Q
             \ifx1#1 q \csname #2COLOR\endcsname\space rg 
                       \csname #3PDF\endcsname{#5}{#6} f Q\fi}\hss}}

                                    % shape     raise width  width/2
\def\sqbox      #1#2{\genbox{#1}{#2}  {sq}       {0}   {4.5}  {2.25}}
\def\trianbox   #1#2{\genbox{#1}{#2}  {trian}    {0}   {5}    {2.5}}
\def\uptrianbox #1#2{\genbox{#1}{#2}  {uptrian}  {0}   {5}    {2.5}}
\def\circbox    #1#2{\genbox{#1}{#2}  {circ}     {0}   {5}    {2.5}}
\def\diabox     #1#2{\genbox{#1}{#2}  {dia}      {-.5} {6}    {3}}

%% usage:

squares: \sqbox0{cgreen}, \sqbox1{cred}, \sqbox0{cblue}.

triangles: \trianbox0{cgreen}, \trianbox1{cred}, \trianbox0{cblue}.

triangles: \uptrianbox0{cgreen}, \uptrianbox1{cred}, \uptrianbox0{cblue}.

circles: \circbox0{cgreen}, \circbox1{cred}, \circbox0{cblue}.

diamonds: \diabox0{cgreen}, \diabox1{cred}, \diabox0{cblue}.




\end{document}