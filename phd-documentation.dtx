% \iffalse meta-comment
%<*internal>
\iffalse
%</internal>
%<*readme>
----------------------------------------------------------------
phd-pkgmanager --- a package to shorten preambles
E-mail: yannislaz@gmail.com
Released under the LaTeX Project Public License v1.3c or later
See http://www.latex-project.org/lppl.txt
----------------------------------------------------------------
This file provides a phd for defining a class.
%</readme>
%<*readmemd>
###The `phd-documentation` LaTeX2e package

The `phd` latex package and the class with the same name provide
convenient methods to create new styles for books, reports
and articles. It also loads the most commonly used packages 
and resolves conflicts.

This work consists of the file  `phd-documentation.dtx`,
and the derived files   `phd-documentation.ins`,  `phd-documentation.pdf`, 
and `phd-documentation.sty`.

###Installation

run
          phd-lua.bat on windows
           pdflatex phd.dtx
           makeindex -s gind.ist -g phd 

If you have any difficulties with the package come and join us at
http://tex.stackexchange.com and post a new question or
add a comment at http://tex.stackexchange.com/a/45023/963.
or send me a message at  yannislaz at gmail.com

### Documentation

The package was written using the `doc` and `docscript` packages,
so that it is self documented in a literary programming style. 
The .pdf is a fat document, providing over fifty book styles (the
equivalent of classes) plus there is a lot of write-up on the inner
workings of TeX and LaTeX2e. However, you don't need to know much
to use it.

      \usepackage{phd}
      \input{style13}

All choices, are made via an extended key-value interface. 
Although not a compliment, it resembles CSS and the keys are a bit verbose but
attributes are easy to change and have a consistent and easy to remember interface.

To set or add a key we only use one command:

      \cxset{chapter name font-size = Huge,
             chapter number font-size = HUGE} 

### Future Development

This is still an experimental version, but I will retain the
interface in future releases. There is a large amount of
work still to be carried out to improve the template styles
provided, to test it more thoroughly and to add a number of
improvements in the special designs. At present I estimate
that I have completed about 70% of the work that needs
to be done.

__The package as it stands is not production stable.__ 


%</readmemd>
%
%<*TODO>
1. On final round add pkg options. This was left as last in order not to solve problems by adding
    options. Too many options are not a good User Interface.
2.  Finish symbol management, both text and math. Math already 60% incorporated.
3.  Better integration of indexing commands.   
4.  Revisit layout manager for Chapters. Broke again in tests.
5.  Docs. Add all references.
6.  Incorporate phd class for more flexibility.
7. Improve package manager.
8. Group script loading for better font management.
9. General font management to relook it again.
10. Add all style sections (about 100 already prepared). Once they
     are all working issue beta version.
%</TODO>
%<*internal>
\fi
\def \nameofplainTeX{plain}
\ifx\fmtname\nameofplainTeX\else
  \expandafter\begingroup
\fi
%</internal>
%<*install>
\input docstrip.tex
\keepsilent
\askforoverwritefalse
\preamble
----------------------------------------------------------------
phd --- A package to beautify documents.
E-mail: yannislaz@gmail.com
Released under the LaTeX Project Public License v1.3c or later
See http://www.latex-project.org/lppl.txt
----------------------------------------------------------------
\endpreamble

%\BaseDirectory{C:/users/admin/my documents/github/phd}
%\usedir{MWE}
\generate{\file{\jobname.sty}{
  \from{\jobname.dtx}{DOCUM}
   }
  }

%\nopreamble\nopostamble

%</install>

%<install>\endbatchfile
%<*internal>
%\usedir{tex/latex/phd}
\generate{
  \file{\jobname.ins}{\from{\jobname.dtx}{install}}
}
\nopreamble\nopostamble

\generate{
	\file{README.txt}{\from{\jobname.dtx}{readme}}
  }

\generate{
  \file{\jobname.md}{\from{\jobname.dtx}{readmemd}}
}
\generate{
  \file{TODO.tex}{\from{\jobname.dtx}{TODO}}
}

\ifx\fmtname\nameofplainTeX
  \expandafter\endbatchfile
\else
  \expandafter\endgroup
\fi
%</internal>
%<*driver>
%\listfiles
%gdef\@onlypreamble{} % TO BE REMOVED NEEDED FOR TUTS
\NeedsTeXFormat{LaTeX2e}[2017/04/15]%
\RequirePackage[2017/04/15]{latexrelease}
\documentclass[book,oneside,11pt,a4paper]{phddoc}
\usepackage[bottom=2cm]{geometry}
\savegeometry{std}
% \usepackage[style=mla]{biblatex}
\usepackage{microtype}
\usepackage{phd}
\usepackage{phd-lowersections}
\usepackage{phd-runningheads}
\usepackage{phd-documentation}
\usepackage{phd-toc}
\pagestyle{headings}


\sethyperref

\addbibresource{phd1.bib}% Syntax f
\cxset{palette unorange}
\begin{filecontents}{defaults-chapters}
%%    General Defaults for Chapters
\cxset{%    
    chapter title margin-top-width    =  0cm,
    chapter title margin-right-width  =  1cm,
    chapter title margin-bottom-width = 10pt,
    chapter title margin-left-width   = 0pt,
    chapter align                     = left,
    chapter title align               = left, %checked
    chapter name                      = hang,
    chapter format                    = hdr,
    chapter font-size                 = Huge,
    chapter font-weight               = bold,
    chapter font-family               = sffamily,
    chapter font-shape                = upshape,
    chapter color                     = black,
    chapter number prefix             = ,
    chapter number suffix             = ,
    chapter numbering                 = arabic,
    chapter indent                    = 0pt,
    chapter beforeskip                = -3cm,
    chapter afterskip                 = 30pt,
    chapter afterindent               = off,
    chapter number after              = ,
    chapter arc                       = 0mm,
    chapter background-color          = bgsexy,
    chapter afterindent               = off,
    chapter grow left                 = 0mm,
    chapter grow right                = 0mm, 
    chapter rounded corners           = northeast,
    chapter shadow                    = fuzzy halo,
    chapter border-left-width         = 0pt,
    chapter border-right-width     = 0pt,
    chapter border-top-width       = 0pt,
    chapter border-bottom-width    = 0pt,
    chapter padding-left-width     = 0pt,
    chapter padding-right-width    = 10pt,
    chapter padding-top-width      = 10pt,
    chapter padding-bottom-width   = 10pt,
    chapter number color           = white,
    chapter label color            = white,    
    }
 \cxset{    
    chapter number font-size        = huge,
    chapter number font-weight      = bfseries,
    chapter number font-family      = sffamily,
    chapter number font-shape       = upshape,
    chapter number align            = Centering,
    }
\cxset{%    
     chapter title font-size        = Huge,
     chapter title font-weight      = bold,
     chapter title font-family      = calligra,
     chapter title font-shape       = upshape,
     chapter title color            = black,
     }    
\end{filecontents}
%% LaTeX2e file `defaults-chapters'
%% generated by the `filecontents' environment
%% from source `phd-documentation' on 2018/10/28.
%%
%%    General Defaults for Chapters
\cxset{%
    chapter title margin-top-width    =  0cm,
    chapter title margin-right-width  =  1cm,
    chapter title margin-bottom-width = 10pt,
    chapter title margin-left-width   = 0pt,
    chapter align                     = left,
    chapter title align               = left, %checked
    chapter name                      = hang,
    chapter format                    = hdr,
    chapter font-size                 = Huge,
    chapter font-weight               = bvar,
    chapter font-family               = sffamily,
    chapter font-shape                = upshape,
    chapter color                     = black,
    chapter number prefix             = ,
    chapter number suffix             = ,
    chapter numbering                 = arabic,
    chapter indent                    = 0pt,
    chapter beforeskip                = -3cm,
    chapter afterskip                 = 30pt,
    chapter afterindent               = off,
    chapter number after              = ,
    chapter arc                       = 0mm,
    chapter background-color          = bgsexy,
    chapter afterindent               = off,
    chapter grow left                 = 0mm,
    chapter grow right                = 0mm,
    chapter rounded corners           = northeast,
    chapter shadow                    = fuzzy halo,
    chapter border-left-width         = 0pt,
    chapter border-right-width     = 0pt,
    chapter border-top-width       = 0pt,
    chapter border-bottom-width    = 0pt,
    chapter padding-left-width     = 0pt,
    chapter padding-right-width    = 10pt,
    chapter padding-top-width      = 10pt,
    chapter padding-bottom-width   = 10pt,
    chapter number color           = white,
    chapter label color            = white,
    }
 \cxset{
    chapter number font-size        = huge,
    chapter number font-weight      = bfseries,
    chapter number font-family      = sffamily,
    chapter number font-shape       = upshape,
    chapter number align            = Centering,
    }
\cxset{%
     chapter title font-size        = Huge,
     chapter title font-weight      = bvar,
     chapter title font-family      = calligra,
     chapter title font-shape       = upshape,
     chapter title color            = black,
     }
  


\definecolor{creamy}{HTML}{FDEBD7}
\cxset{chapter title color= creamy,
       chapter label color = creamy,
       chapter number color = creamy,
       chapter number font-size = Huge,
       subsection title color = creamy,
       chapter name = CHAPTER,
       chapter label case = upper,
       chapter number align=left,
       part format = traditional,
       part background-color=spot,
       part beforeskip                = -3cm,
       part afterskip                 = 30pt,
       subsection afterindent=off,
       section format=hang,
       }
       
\makeindex
\usepackage[cache=false]{minted} 
\usemintedstyle[latex]{borland}  
\setminted[html]{fontsize=\footnotesize,style=friendly}
\begin{document}
\DEBUGOFF
\overfullrule0pt
\parindent1em
\coverpage{monkey}{Book Design Monographs}{Camel Press}{INDEXING}{AND DOCUMENTATION} 
\pagestyle{empty}
%\coverpage{habtoor-city}{Delay Claim}{HLS-DSE/JV}{HABTOOR CITY}{MEP CLAIM} 
\secondpage
\pagestyle{empty}
\clearpage

\tableofcontents

\pagestyle{empty}
\setcounter{secnumdepth}{6}
\parskip0pt plus.1ex minus.1ex
\mainmatter
\pagenumbering{arabic}
\pagestyle{headings} 
% Input the main descriptions and guide
\makeatletter

\thispagestyle{plain}

\cxset{image={./images/breakerboys.jpg},
         subsection font-shape= upshape,}

\usemintedstyle{friendly}

\chapter{Documenting Code}

\section{Verbatims}

Verbatim code will typeset code exactly as is typed. The package \pkg{fancyvrb} is used by the \pkg{phd-documentation} for verbatims that are used in the documentation and implementation section of documents; that
is using the class |phddoc| or the class |ltxdoc| or similar and the text is inputted through |\DocInput|.
In normal documents inputted through |\input| we prefer to use |listings| or |minted|. 

\section{Listings styles}  

Many users of \latex require to typeset formatted code. There are two packages that
can be used the more conventional \pkgname{listings}\footcite{listings} and \pkgname{minted}\footcite{minted}. The
|minted| package is a more powerful and flexible package than listing, since it uses
an external program |Pygments| which is written in |Python|\footnote{See \protect\url{http://pygments.org/} for more details. You can also review and contribute to the code at \protect\url{https://bitbucket.org/birkenfeld/pygments-main}}. My recommendation to you is
to use the |minted package|. The two packages can happily co-exist and each one has
its own advantages and disantvantages. The |listings| package has all its color
parameters configurable via its \latexe key value settings, whereas the pygments program
has its own way of setting these styles, which are only accessible through \latexe
as a set of fixed styles. To create a new color scheme, you will need to write some
simple |python|, register it as a plugin or drop it at the folder holding the styles.\footnote{See documentation at \protect\url{http://pygments.org/docs/styles/}.}

For me it is the only limiting factor of pygments and which there are ways around it. However, this might be also an advantage
as users are more likely to be familiar with such code coloring schemes in their language.

\section{Python and Pygmentize}

For the |minted| package to work, you will need to have Python and Pygmentize installed on your computer. Follow normal guidelines for these. The sequence is normally to install Python version 3.6 or higher (I have re-installed Python at 3.7) ensure it is on the PATH. Then using PIP3 install pygments to install pygmentize. The scripts folder must also be on the PATH. 

\section{Using minted}

Since minted makes calls to the outside world (that is, Pygments), you need to
tell the \latex processor about this by passing it the |-shell-escape| option or it
won’t allow such calls. In effect, instead of calling the processor like this:


\begin{minted}[fontsize=\footnotesize,style=vim]{bash}
$ latex input
you need to call it like this:
$ latex -shell-escape input
\end{minted}

The same holds for other processors, such as pdf\latex or \xelatex.
You should be aware that using |-shell-escape| allows \latex to run potentially
arbitrary commands on your system. It is probably best to use -shell-escape
only when you need it, and to use it only with documents from trusted sources.

If you are on Windows and using TeXworks, you can enable |-enable-write-18| by going to Edit>Preferences and the on the tab |Typesetting| enable the write-18 by adding it on the processing tool you are using.



\subsection{A minimal example}   
 
The minted package is loaded like any other package (with or without options). 
You can then use the \docAuxEnv{minted} environment with the language we want to use
as the first argument. The environment also takes an optional argument where the numerous
settings of the package can be specified.
 
\begin{phdverbatim}[basicstyle=\small\ttfamily]
\documentclass{article}
\usepackage{minted}
\begin{document}
\begin{minted}[fontsize=\footnotesize,style=friendly]{javascript}
if (Meteor.isClient) {
  // This code only runs on the client
  Template.body.helpers({
    tasks: [
      { text: "This is task 1" },
      { text: "This is task 2" },
      { text: "This is task 3" }
    ]
  });
}
\end{minted}
\end{document}
\end{phdverbatim}

This will produce an output as:
\medskip

\begin{minted}[fontsize=\footnotesize,style=friendly]{javascript}
if (Meteor.isClient) {
  // This code only runs on the client
  Template.body.helpers({
    tasks: [
      { text: "This is task 1" },
      { text: "This is task 2" },
      { text: "This is task 3" }
    ]
  });
}
\end{minted}

If we do not need a style, the |style=default| setting will typeset as,

\begin{minted}[fontsize=\footnotesize,style=trac]{javascript}
if (Meteor.isClient) {
  // This code only runs on the client
  Template.body.helpers({
    tasks: [
      { text: "This is task 1" },
      { text: "This is task 2" },
      { text: "This is task 3" }
    ]
  });
}
\end{minted}

You can also set the style for the whole document using:

\begin{minted}[fontsize=\footnotesize,style=trac]{TeX}
\usemintedstyle{<name>}
\end{minted}
where you can get <name> by typing

\begin{minted}[fontsize=\footnotesize,style=bw]{bash}
$ pygmentize -L styles
\end{minted}
at the command prompt/terminal. For example, the minted documentation itself uses the |trac| style.

\begin{minted}{html}
<!-- First set the doctype -->
<!DOCTYPE html>
    <html>
      <head>
        <title>Canvas</title>
        <meta charset="UTF-8" />
        <style>
          #square {
            border: 1px solid black;
                    transform: scale(10) rotate(3deg) translateX(0px);
                    -moz-transform: scale(10) rotate(3deg) translateX(0px);
          }

          .box {              
                    transition-duration: 2s;
                    transition-property: transform;
                    transition-timing-function: linear;
          }
        </style>
      </head>
      <body>
        <canvas id="square" width="200" height="200"></canvas>
        <script>
                var canvas = document.createElement('canvas');
                canvas.width = 200;
                canvas.height = 200;

                var image = new Image();
                image.src = 'images/card.png';
                image.width = 114;
                image.height = 158;
                image.onload = window.setInterval(function() {
                    rotation();
                }, 1000/60);
       </script>
      </body>
    </html>
\end{minted} 
   



\begin{lstlisting}
int main() {
printf("hello, world");V\colorbox{green}{**}V
return 0;
}
\end{lstlisting}



\chapter{The phd-documentation User Manual}


\section{Documentation macros}

When developing this package the need arose to define a number of documentation macros. I~have used heavily macros and ideas present in the \pkg{doc} package, \pkg{pgf} documentation, \pkg{biblatex} documentation  and \pkg{tcolorbox} and for which I am grateful to their respective authors. The major change was to adopt the macros to use different fonts and colors and to use these from a list of key values defined at document level. More about this later. General package user documentation as opposed to package documentation that can be achieved using the |doc/docstrip| system requires that macros and environments be developed for the following:

\begin{enumerate}
\item Macros for command documentation.
\item Environments for commands and options.
\item Latex examples that need to be executed within the document as well as described.
\end{enumerate}


\section{Commands and Styles for Documenting macros}

The most commonly used commands for documenting macros are |\cs|, |\cmd|, |\meta|, |\marg|, |\oarg|. These commands have been defined by many authors and perhaps the best implementation can be found in the \pkg{doc}. Many package authors have redefined them in their documention, some if just to add a bit of colour, others to have them add the command to an index. As we also had a target to allow for
the package to be used in both normal documents as well as documentation
of packages and classes that use the \pkg{doc} and \pkg{docstrip} combination we provided many compatible macros.

\def\MacroFont{\ttfamily\color{thecs}}
\begin{environment}{macro} The environment macro is made available in this
package. This follows the definition of the macro in \pkg{doc} and can be used both in
the package documentation section as well as the implementation. In my opinion this presents
 better user interface. 
\end{environment}



\begin{macro}{\cmd} The command \cmd{\cmd} typesets its argument in
  verbatim. Typing |\cmd{\cmd}| typsets \cmd{\cmd}. If the class
  |ltxdoc| is loaded the command is defined there. We have modified
  it to accept a colour and changes to the verbatim font 
  for consistency.
\end{macro}

\begin{macro}{\meta}
The macro \cs{meta} is normally used to build other commands. On its own it can be used to typeset
examples of the argument of macros, typing:
     |\meta{Aristotle}| \\
 will typeset \meta{Aristotle}. The command provides a hook to set the font via a macro |\meta@font@select|. 
\end{macro}


|\def\meta@font@select{\upshape\color{black}}|


\subsection{Color management}
One of the first requirements for redefining some of the standard doc commands is the need to use color easily, hence we will try and define a certain amount of keys for colors.

Just a bit of a refresher, to define colors we use, either the \cs{definecolor} or the \cs{colorlet} commands.

\emphasis{definecolor,colorlet}

\begin{minted}[fontsize=\small]{TeX}
\definecolor{Hyperlink}{rgb}{0.281,0.275,0.485}
\colorlet{thehyperlink}{theblue}
\end{minted}


We use a semantic approach, where the colors are first defined with a mnemonic command such as {\bfseries\textcolor{theblue}{theblue}} and then we define a semantic command such as the\cs{option} that lets the color to the option command. This sort of double entry has proved useful in navigating through the dozen of the commands that I needed for this documentation.


\subsection{Semantic color names}
\begin{marglist}
\item [\option{theoption}] Coloring of options in margin lists.
\item [\option{themacro}] Coloring of command macros \cs{foo}.
\item [\option{hyperlink}] If we use the \texttt{hyperref} package a number of colors need to be defined for links.
\end{marglist}

\subsection{Named colors}
Standard colors that we provide are:
\begin{marglist}
\item [\textcolor{theblue}{theblue}] This color is used mainly for options.
\item [\textcolor{thered}{thered}] The color mostly used for macro commands and keys.
\item [\textcolor{thegreen}{thegreen}] used for environments.
\item [\textcolor{thelightgreen}{thelightgreen}] Used for margin lists.
\item [\textcolor{thegray}{thegray}] Used as a background to the listings.
\item [\colorbox{thegrey}{\color{white}thegrey}] Alias for the gray to satisfy both sides of the Atlantic and as I sometimes don't remeber which is which.
\item [\colorbox{theshade}{theshade}] Another slightly lighter shade.
\end{marglist}



\begin{marglist}
\item [\cs{cs}] \cs{cs} text Prints a command.
\item [\cs{cmd}] Prints a command.
\end{marglist}




\section{Lists for documentation}



The environment \env{marglist}
\begin{marglist}
\item[testing]\lorem
\item [test]\lorem
\end{marglist}

\env{keymarginlist}This environment is suitable for listing keys, set-in the margin.

\begin{keymarglist}
\item[bibliography] The term <bibliography>, also available as \cmd{\bibname}.
\item[references] The term <references>, also available as \cmd{refname}.
\item[shorthands] The term <list of shorthands> or <list of abbreviations>, also available as \cmd{losname}.
\end{keymarglist}


\env{argumentlist} This environment is suitable for listing macro arguments and their explanations.



\section{Breakable Boxes}

The \pkg{mdframed} as well as the newer versions of \pkg{tcolorbox}
offer breakable boxes.


\begin{tcolorbox}[enhanced, breakable,
  colback=blue!5!white,colframe=blue!75!black,title=Breakable box,
  watermark color=white,watermark text=\Roman{tcbbreakpart}]
  \lipsum[1-3]
\end{tcolorbox}

\section{PGF Style Code Boxes}

\begin{codeexample}[]
\begin{tikzpicture}
  \node[place,label=above:$p_1$,tokens=2]           (p1) {};
  \node[place,label=below:$p_2\ge1$,right=of p1]  (p2) {};
\end{tikzpicture}
\end{codeexample}


\section{Listings environments}

The \pkg{listings}\footcite{listings} is the main engine behind the \pkg{phd-documentation} package. The package
has been integrated to work seamlessly with the |phd| style definition sheets, as well as with the \pkg{phd-colorpalette}. We provide a number of predefined styles, as well as environments for typestting LaTeX code.

We will describe the available environments first:

The \docAuxEnvironment{teX} provides an environment for typesetting code with the \docAuxListingsStyle{simple}.

\begin{texexample}{The teX environment}{ex:teX}
\begin{teX}
% A simple TeX listing.
\TeX is a great program.
\end{teX}
\end{texexample}


\begin{teXX}
\TeX and \LaTeX are great programs.
\end{teXX}

\begin{teXXX}
\TeX,  \latexe and \pkg{Expl3} are great programs. The \IfNoValueF and \IfNoValueTF are \latex3
commands.
\end{teXXX}

\subsection{Typesetting bib files}
\begin{teXXX}
 @package{pkg:chngcntr,
    title      = {chngcntr} ,
    author     = {Peter Wilson} ,
    maintainer = {Will Robertson} ,
    date       = {2009-09-02} ,
    version    = {1.0a} ,
    url        = {http://mirror.ctan.org/macros/latex/contrib/chngcntr/}
  }
  @class{cls:exam,
    title      = {exam},
    author     = {Philip Hirschhorn},
    date       = {2015-05-07},
    version    = {2.5},
    url        = {http://mirror.ctan.org/macros/latex/contrib/exam/}
  }
  @bundle{bnd:koma-script,
    title          = {\KOMAScript} ,
    sorttitle      = {KOMA-Script} ,
    indextitle     = {\KOMAScript} ,
    indexsorttitle = {KOMA-Script} ,
    author         = {Markus Kohm},
    date           = {2015-07-02} ,
    version        = {3.18} ,
    url            = {http://mirror.ctan.org/macros/latex/contrib/koma-script/}
  }
\end{teXXX}
















% Check this as to workflow with pythontex https://news.ycombinator.com/item?id=18280594
       
%       
%\def\storyi{The best graphics package ever developed is the TikZ package. 
Its parent package is PGF which is short of a miracle that has been programmed
using \tex, a more than thirty years old program. This has taken over almost all other
packages and is very popular with newcomers to \latex. It is frustrating at first, but once 
you over the basic ideas and concepts it opens infinite possibilities for typesetting
great articles and books.}



\cxset{chapter format=stewart,
       texti=\storyi,
       textii=\storyi}

\newcommand\seepgfmanual[1]{%
    \textit{see} the PGFmanual page #1}%
    
%\cxset{chapter format = traditional}    
\chapter{TikZ}

\section{The \protect\texttt{TikZ} package}
\pkg{TikZ}, a high-level interface to \pkg{PGF}, is a language-based tool for specifying graphics.
It uses familiar graphics-related concepts, such as point, line, and circle and
has a concise and natural syntax. It meshes well with pdfLATEX in the sense that
no additional processing steps are needed. Another positive aspect of \pkg{TikZ} is
its ability to blend \tex fonts, symbols, and mathematics within the generated
graphics
.
\begin{center}
\begin{tikzpicture}

\draw (0,0) ellipse (5 and 1.8);

\filldraw [black]  (-3,-0.5) circle (1pt);

\filldraw [black] (0.5,-0.5) circle (1pt);
\filldraw [black] (1.5,-0.5) circle (1pt);
\filldraw [red] (1.0,0.5) circle (1pt);
\filldraw [green] (2.0,0.5) circle (1pt);


\draw [->>] (0.5,-0.5) .. controls (0.5,-0.2) and (0.7,0.3) .. (0.9,0.45);
\draw [->>] (1.5,-0.5) .. controls (1.5,-0.2) and (1.7,0.3) .. (1.9,0.45);

\draw (-3.1,-0.5) node[anchor=east]  {$1_G$};

\draw (0.5,-0.5) node[anchor=north] {$g$};
\draw (1.5,-0.5) node[anchor=north] {$gs$};
\draw (1.0,0.5) node[anchor=south] {$gt$};
\draw (2.0,0.5) node[anchor=south] {$gst$};
\draw (0.5,-0.5)-- (1.5,-0.5);


\end{tikzpicture}
\end{center}

All the TikZ commands can be used inline using \docAuxCommand{tikz} or within the \docAuxCommand{tikzpicture} environment. When we want to use captions and labels, we enclose it in the figure environment or use \docAuxCommand{captionof}, but it can be called anywhere in the text or math of a Tex document:

\begin{teX}
\begin{figure}
\centering
%\tikzset{external/force remake}
\begin{tikzpicture}
... TikZ commands ...
\end{tikzpicture}
\caption{A diagram drawn with TikZ.}
\label{Fig:_diagram1}
\end{figure}
\end{teX}

We can also use them in math:

\begin{teX}
\begin{align*}
\int dx\; f(x) =
\alpha
%\tikzset{external/force remake}
\begin{tikzpicture}
... TikZ commands ...
\end{tikzpicture}
\end{align*}
\end{teX}



\section{Draw simple lines}

\begin{texexample}{Draw a Line}{ex:line}
\begin{tikzpicture}
\node[draw] (S1) at (0,0) {Paris};
\node[draw] (S2) at (3,0) {Stratsbourg};
\draw (S1) -- (S2);
\end{tikzpicture}
\end{texexample}


The syntax of the command is:

|\node|\oarg{options} (\meta{name}) at (\meta{position}) |{|\meta{contents}|}|

If we look
 carefully, we see that the two writings give
Slightly different results:
- In the first case, node is an operation executed on a path. We
Can consider each node as a decoration of the point at which it
is associated. The line drawn by the draw command joins two points, the
Nodes are objects added later and centered on points. The option
Draw of the node trace operation the node outline.
- In the second case, \ node is a TikZ command which allows to define
A node, to name it and to draw it. One can then consider the
Nodes as pre-existing objects that will then be linked with the \docAuxCommand{node}.


\begin{texexample}{Draw a Line}{ex:line}
\begin{tikzpicture}
\node[draw] (S1) at (0,0) {Paris};
\node[draw] (S2) at (0,3) {Stratsbourg};
\draw[->] (S1) -- (S2);
\end{tikzpicture}
\end{texexample}

The basic building block of all pictures in \tikzname is the path. A path is a series of straight lines and curves
that are connected (that is not the whole picture, but let us ignore the complications for the moment). You
start a path by specifying the coordinates of the start position as a point in round brackets, as in (0,0).
This is followed by a series of \enquote{path extension operations.}


\begin{texexample}{Draw a Line}{ex:line}
\begin{tikzpicture}
\draw[->] (0,0) -- (1.5,0) -- (0, 1.2);
\end{tikzpicture}
\end{texexample}


\subsection*{Adding Text} 

So far we have seen how to draw lines and arcs. However, an important component is still missing the addition of text. When
\tikzname is constructing a path and it encounters the keyword |node| typically followed by some options  it reads a \textit{node specification}. Options can typically follow and then it terminates by curly brackets. 
 

\begin{texexample}{Draw a Line}{ex:line}
\begin{tikzpicture}
\draw[->] (0,0) -- (1.5,0) node {First Node} -- (0, 1.2) node[shape = circle] {Second Node};
\end{tikzpicture}
\end{texexample}


The \docAuxCommand*{node} can be used to abbreviate the operation. A longer example can demonstrate this better. How can we draw the following figure?

\begin{tikzpicture}
\node[circle,fill=black,inner sep=0.8pt,draw] (a) at (0,0) {};
\node[circle,fill=black,inner sep=0.8pt,draw] (b) at (1.5,0) {};
\node[circle,fill=black,inner sep=1.5pt,draw] (c) at (.75,2) {};
\node[circle,fill=black,inner sep=0.8pt,draw] (d) at (0.75,.75) {};
\node[circle,fill=black,inner sep=0.8pt,draw] (e) at (2,1) {};


\node () at (-0.3,0) {\tiny$1$};
\node () at (0.75,0.45) {\tiny$2$};
\node () at (0.75,2.3) {\tiny$4$};
\node () at (2,1.3) {\tiny$-1$};
\node () at (1.8,0) {\tiny$-1$};

\draw (a)--(b)--(e)--(c) --(a)--(d)--(b)--(c);
\draw (c)--(d);

\node at (3,1) {\Large{$\sim$}};

\begin{scope}[shift={(+4,0)}]
\node[circle,fill=black,inner sep=0.8pt,draw] (a) at (0,0) {};
\node[circle,fill=black,inner sep=0.8pt,draw] (b) at (1.5,0) {};
\node[circle,fill=black,inner sep=0.8pt,draw] (c) at (.75,2) {};
\node[circle,fill=black,inner sep=0.8pt,draw] (d) at (0.75,.75) {};
\node[circle,fill=black,inner sep=0.8pt,draw] (e) at (2,1) {};


\node () at (-0.3,0) {\tiny$2$};
\node () at (0.75,0.45) {\tiny$3$};
\node () at (0.75,2.3) {\tiny$0$};
\node () at (2,1.3) {\tiny$0$};
\node () at (1.8,0) {\tiny$0$};

\draw (a)--(b)--(e)--(c) --(a)--(d)--(b)--(c);
\draw (c)--(d);

\end{scope}
\end{tikzpicture}

\begin{texexample}{A larger example}{ex:larger}
\begin{tikzpicture}
\node[circle,fill=black,inner sep=0.8pt,draw] (a) at (0,0) {};
\node[circle,fill=black,inner sep=0.8pt,draw] (b) at (1.5,0) {};
\node[circle,fill=black,inner sep=1.5pt,draw] (c) at (.75,2) {};
\node[circle,fill=black,inner sep=0.8pt,draw] (d) at (0.75,.75) {};
\node[circle,fill=black,inner sep=0.8pt,draw] (e) at (2,1) {};


\node () at (-0.3,0) {\tiny$1$};
\node () at (0.75,0.45) {\tiny$2$};
\node () at (0.75,2.3) {\tiny$4$};
\node () at (2,1.3) {\tiny$-1$};
\node () at (1.8,0) {\tiny$-1$};

\draw (a)--(b)--(e)--(c) --(a)--(d)--(b)--(c);
\draw (c)--(d);

\node at (3,1) {\Large{$\sim$}};

\begin{scope}[shift={(+4,0)}]
\node[circle,fill=black,inner sep=0.8pt,draw] (a) at (0,0) {};
\node[circle,fill=black,inner sep=0.8pt,draw] (b) at (1.5,0) {};
\node[circle,fill=black,inner sep=0.8pt,draw] (c) at (.75,2) {};
\node[circle,fill=black,inner sep=0.8pt,draw] (d) at (0.75,.75) {};
\node[circle,fill=black,inner sep=0.8pt,draw] (e) at (2,1) {};


\node () at (-0.3,0) {\tiny$2$};
\node () at (0.75,0.45) {\tiny$3$};
\node () at (0.75,2.3) {\tiny$0$};
\node () at (2,1.3) {\tiny$0$};
\node () at (1.8,0) {\tiny$0$};

\draw (a)--(b)--(e)--(c) --(a)--(d)--(b)--(c);
\draw (c)--(d);

\end{scope}
\end{tikzpicture}
\captionof{figure}{The larger vertex fires once to move from the left configuration to the right configuration.}
\end{texexample}

Behind the scenes pgf uses the basic system command \docAuxCommand{pgfnode} to create the nodes. The syntax of the command is given on \seepgfmanual{1026} as:

\begin{docCommand}{pgfnode}{\marg{shape}\marg{anchor}\marg{label text}\marg{name}\marg{path usage command}}
This command creates a new node. The \marg{shape} of the node must have been declared previously using
\lstinline{pgfdeclareshape}.

The shape is shifted such that the \marg{anchor} is at the origin. In order to place the shape somewhere else,
use the coordinate transformation prior to calling this command.
The hnamei is a name for later reference. If no name is given, nothing will be “saved” for the node, it
will just be drawn.

The \marg{path usage command} is executed for the background and the foreground path (if the shape defines
them).
\end{docCommand}


A good workflow, is to first define the nodes, next label them and then draw any connecting lines.

\begin{texexample}{Named nodes}{ex:named} 
\begin{tikzpicture}
\node[circle,fill=black,inner sep=0.8pt,draw] (a) at (0,0) {};
\node[circle,fill=black,inner sep=0.8pt,draw] (b) at (1.5,0) {};
\node[circle,fill=black,inner sep=1.5pt,draw] (c) at (.75,2) {};
\node[circle,fill=black,inner sep=0.8pt,draw] (d) at (0.75,.75) {};
\node[circle,fill=black,inner sep=0.8pt,draw] (e) at (2,1) {};
\end{tikzpicture}
\end{texexample}

\begin{texexample}{Named nodes}{ex:named} 
\begin{tikzpicture}
\node[circle,fill=black,inner sep=0.8pt,draw] (a) at (0,0) {};
\node[circle,fill=black,inner sep=0.8pt,draw] (b) at (1.5,0) {};
\node[circle,fill=black,inner sep=1.5pt,draw] (c) at (.75,2) {};
\node[circle,fill=black,inner sep=0.8pt,draw] (d) at (0.75,.75) {};
\node[circle,fill=black,inner sep=0.8pt,draw] (e) at (2,1) {};
% absolute labelling
\node () at (-0.3,0) {\tiny$1$};
\node () at (0.75,0.45) {\tiny$2$};
\node () at (0.75,2.3) {\tiny$4$};
\node () at (2,1.3) {\tiny$-1$};
\node () at (1.8,0) {\tiny$-1$};
\end{tikzpicture}
\end{texexample}

\begin{texexample}{Named nodes}{ex:named} 
\begin{tikzpicture}
\pgfdeclarelayer{background}
\pgfdeclarelayer{foreground}
\pgfsetlayers{background,main,foreground}
\node[circle,fill=black,inner sep=0.8pt,draw] (a) at (0,0) {};
\node[circle,fill=black,inner sep=0.8pt,draw] (b) at (1.5,0) {};
\node[circle,fill=black,inner sep=1.5pt,draw] (c) at (.75,2) {};
\node[circle,fill=black,inner sep=0.8pt,draw] (d) at (0.75,.75) {};
\node[circle,fill=black,inner sep=0.8pt,draw] (e) at (2,1) {};
% absolute labelling
\node () at (-0.3,0) {\tiny$1$};
\node () at (0.75,0.45) {\tiny$2$};
\node () at (0.75,2.3) {\tiny$4$};
\node () at (2,1.3) {\tiny$-1$};
\node () at (1.8,0) {\tiny$-1$};
% draw connecting lines
\draw (a)--(b)--(e)--(c) --(a)--(d)--(b)--(c);
\draw (c)--(d);
%\begin{pgfonlayer}{background}
\begin{scope}[on background layer={color=blue!10}]
\node [fill=blue!10,fit=(a) (b) (c)
(d) (e)] {};
\end{scope}
%\end{pgfonlayer}
\end{tikzpicture}
\end{texexample}

Just to recap, using \docAuxCommand*{node} and the \textbf{at} we can position accurately any node. We could have used the much longer command |path node|, but in our case above this is unecessary (\seepgfmanual{49}), for more explanations if you are still unsure.

Nodes can be named or unnamed. There are two ways to name them, with the key \docValue{name} or within brackets. The second method is to be preferred. Names for nodes can be pretty arbitrary, but they should not contain commas, periods, parentheses, colons, and some other special characters. However, they can contain underscores and hyphens

\subsection{Layers and Scope}

We can add a backround layer, using the library \textit{backgrounds}, which provides key values for adding backgrounds. \pgfname\ provides a layering mechanism for composing graphics from
multiple layers. (This mechanism is not to be confused with the
conceptual ``software layers'' the \pgfname\ system is composed of.)
Layers are often used in graphic programs. The idea is that you can
draw on the different layers in any order. So you might start drawing
something on the ``background'' layer, then something on the
``foreground'' layer, then something on the ``middle'' layer, and then
something on the background layer once more, and so on. At the end, no
matter in which ordering you drew on the different layers, the layers
are ``stacked on top of each other'' in a fixed ordering to produce
the final picture. Thus, anything drawn on the middle layer would come
on top of everything of the background layer.

Normally, you do not need to use different layers since you will have
little trouble ``ordering'' your graphic commands in such a way that
layers are superfluous. However, in certain situations you only
``know'' what you should draw behind something else after the
``something else'' has been drawn.

For example, suppose you wish to draw a yellow background behind your
picture. The background should be as large as the bounding box of the
picture, plus a little border. If you know the size of the bounding box
of the picture at its beginning, this is easy to accomplish. However,
in general this is not the case and you need to create a
``background'' layer in addition to the standard ``main'' layer. Then,
at the end of the picture, when the bounding box has been established,
you can add a rectangle of the appropriate size to the picture.

\subsection{Declaring Layers}

In \pgfname\ layers are referenced using names. The standard layer,
which is a bit special in certain ways, is called |main|. If nothing
else is specified, all graphic commands are added to the |main|
layer. You can declare a new layer using the following command:

\begin{docCommand}{pgfdeclarelayer}{\marg{name}}
  This command declares a layer named \meta{name} for later
  use. Mainly, this will set up some internal bookkeeping.
\end{docCommand}

The next step toward using a layer is to tell \pgfname\ which layers
will be part of the actual picture and which will be their
ordering. Thus, it is possible to have more layers declared than are
actually used.

\begin{docCommand}{pgfsetlayers}{\marg{layer list}}
  This command tells \pgfname\ which layers will be used in
  pictures. They are stacked on top of each other in the order
  given. The layer |main| should always be part of the list. Here is
  an example:
\begin{codeexample}[code only]
\pgfdeclarelayer{background}
\pgfdeclarelayer{foreground}  
\pgfsetlayers{background,main,foreground}
\end{codeexample}

  This command should be given either outside of any picture or ``directly inside'' of a picture.
  Here, the ``directly inside'' means that there should be no further level of \TeX\ grouping between |\pgfsetlayers| and the matching |\end{pgfpicture}| (no closing braces, no |\end{...}|). It will also work if |\pgfsetlayers| is provided before |\end{tikzpicture}| (with similar restrictions).
\end{docCommand}


\subsection{Using Layers}

Once the layers of your picture have been declared, you can start to
``fill'' them. As said before, all graphics commands are normally
added to the |main| layer. Using the |{pgfonlayer}| environment, you
can tell \pgfname\ that certain commands should, instead, be added to
the given layer.

\begin{docEnvironment}{pgfonlayer}{\marg{layer name}}
\end{docEnvironment}

The whole \meta{environment contents} is added to the layer with the
name \meta{layer name}. This environment can be used anywhere inside
a picture. Thus, even if it is used inside a |{pgfscope}| or a \TeX\
group, the contents will still be added to the ``whole'' picture.
Using this environment multiple times inside the same picture will
cause the \meta{environment contents} to accumulate.

  \emph{Note:} You can \emph{not} add anything to the |main| layer
  using this environment. The only way to add anything to the main
  layer is to give graphic commands outside all |{pgfonlayer}|
  environments. 



\begin{codeexample}[]
\pgfdeclarelayer{background layer}
\pgfdeclarelayer{foreground layer}
\pgfsetlayers{background layer,main,foreground layer}
\begin{tikzpicture}
  % On main layer:
  \fill[blue] (0,0) circle (1cm);
  
  \begin{pgfonlayer}{background layer}
    \fill[yellow] (-1,-1) rectangle (1,1);
  \end{pgfonlayer}
  
  \begin{pgfonlayer}{foreground layer}
    \node[white] {foreground};
  \end{pgfonlayer}
  
  \begin{pgfonlayer}{background layer}
    \fill[black] (-.8,-.8) rectangle (.8,.8);
  \end{pgfonlayer}

  % On main layer again:
  \fill[blue!50] (-.5,-1) rectangle (.5,1);
\end{tikzpicture}
\end{codeexample}



\long\gdef\mytriangle{
\node[circle,fill=black,inner sep=0.8pt,draw] (a) at (0,0) {};
\node[circle,fill=black,inner sep=0.8pt,draw] (b) at (1.5,0) {};
\node[circle,fill=black,inner sep=1.5pt,draw] (c) at (.75,2) {};
\node[circle,fill=black,inner sep=0.8pt,draw] (d) at (0.75,.75) {};
\node[circle,fill=black,inner sep=0.8pt,draw] (e) at (2,1) {};
% absolute labelling
\node () at (-0.3,0) {\tiny$1$};
\node () at (0.75,0.45) {\tiny$2$};
\node () at (0.75,2.3) {\tiny$4$};
\node () at (2,1.3) {\tiny$-1$};
\node () at (1.8,0) {\tiny$-1$};
% draw connecting lines
\draw (a)--(b)--(e)--(c) --(a)--(d)--(b)--(c);
\draw (c)--(d);
}

\begin{texexample}{Adding backgrouns}{ex:backgrounds}
\begin{tikzpicture}
\pgfdeclarelayer{background}
\pgfdeclarelayer{foreground}
\pgfsetlayers{background,main,foreground}
\mytriangle
%\begin{pgfonlayer}{background}
\begin{scope}[on background layer={color=blue!10}]
\mytriangle
\node [fill=blue!10,fit=(a) (b) (c)
(d) (e)] {};
\end{scope}
%\end{pgfonlayer}
\end{tikzpicture}
\end{texexample}


\begin{texexample}{Adding backgrouns}{ex:backgrounds}
\begin{tikzpicture}
\pgfdeclarelayer{background}
\pgfdeclarelayer{foreground}
\pgfsetlayers{background,main,foreground}
\mytriangle
%\begin{pgfonlayer}{background}
\begin{scope}[on background layer={color=blue!10}]
\node [fill=blue!10,fit=(a) (b) (c)
(d) (e)] {};
\end{scope}

\begin{scope}[shift={(+4,0)}]
\mytriangle
\begin{pgfonlayer}{background}
\node [pattern=checkerboard light gray,fit=(a) (b) (c)
(d) (e)] {};
\end{pgfonlayer}
\end{scope}
\end{tikzpicture}
\end{texexample}

This brings us to the end of our discussion. Time for a coffee and a break.                

\section{Adding styles}

In our previous example, we cut and pasted many of the repetitive keys. \pgfname offers a way to set a new key to the values of other keys using the handler |.style|. This is a very powerful way of redefining new keys, but also simplifying the code. Styles in \tikzname can be considered similar to macros in standard LaTeX. When I made a drawing, we can still tweak the styles and look how the drawing changes, until it's perfect. You should never have to tweak each node.

\begin{texexample}{Using styles}{ex:usingstyles}
\tikzset{BN/.style = {circle,fill=black,inner sep=0.8pt,draw},
         tiny/.style = {font=\tiny}, 
}
\begin{tikzpicture}
\node[BN] (a) at (0,0) {};
\node[BN] (b) at (1,0) {};
\node[BN] (c) at (1,1) {};
\node[BN] (d) at (0,1) {};
\node[BN] (e) at (-1,0) {};

\node () at (-1.3,0) [tiny]{$v_1$};
\node () at (-.3,1)  [tiny]{$v_2$};
\node () at (1.3,0)  [tiny]{$w_1$};
\node () at (1.3,1)  [tiny]{$w_2$};

\node[tiny] () at (0.5,-0.2) {$a$};
\node[tiny] () at (0.5,1.2) {$b$};
\node[tiny] () at (0.2,0.5) {$c$};
\node[tiny] () at (-0.5,-.2) {$d$};

\draw (e) -- (a) -- (b) -- (c) -- (d) -- (a);
\draw (e) -- (d);

\end{tikzpicture}
\end{texexample}



\section{Arcs and options for lines}

\begin{texexample}{Draw a Line}{ex:line}
\begin{tikzpicture}
\draw[->] (0,0) -- (1.5,0) node[draw, ellipse] {First Node} -| (0, 1.2) node[draw,ellipse,rotate=45] {Second Node};
\end{tikzpicture}
\end{texexample}

\begin{texexample}{Drawing arcs}{ex:matharcs}
We define 
\begin{gather*}
    \bar{d}_{k,l}:=\hspace{6pt}
    \begin{tikzpicture}[baseline=(current bounding box.center)]
    \draw[->] (3,2) arc (-180:180:5mm);
	  \fill (3.95,2.2) circle [radius=2pt];
    \draw (3.95,1.8) circle [radius=2pt];
    \node at (4.2,1.8) {$l$};
    \node at (4.2,2.2) {$k$};
    \end{tikzpicture}
    \hspace{0.5cm}
    \text{and}
    \hspace{0.5cm}
    d_{k,l}:=\hspace{6pt}
    \begin{tikzpicture}[baseline=(current bounding box.center)]
    \draw[<-] (3,2) arc (-180:180:5mm);
    \fill (3.95,2.2) circle [radius=2pt];
    \draw (3.95,1.8) circle [radius=2pt];
    \node at (4.2,1.8) {$l$};
    \node at (4.2,2.2) {$k$};
    \end{tikzpicture}
    \hspace{0.5cm}
    \text{for}
    \hspace{2mm} k,l\in\mathbb{Z}_{\geq 0}.
\end{gather*}
\end{texexample}


Here is a figure that you should try and reproduce.
\newcommand{\G}{\Gamma}

\begin{tikzpicture}
\draw (-3.5,-1)--(-2.5,0); \draw (-2.5,-1)--(-3.5,0); \draw (-1.5,-1)--(-1.5,0);\draw[fill=black] (-3,-0.5) circle (0.1cm); \draw (-3.5,0)--(-3.5,1); \draw (-2.5,0)--(-1.5,1); \draw (-1.5,0)--(-2.5,1);\draw[fill=black] (-2,0.5) circle (0.1cm); \draw[->] (-3.5,1)--(-2.5,2); \draw[->] (-2.5,1)--(-3.5,2); \draw[->] (-1.5,1)--(-1.5,2); \draw[fill=black] (-3,1.5) circle (0.1cm); \draw (-3.6,0)--(-3.4,0);\draw (-2.6,0)--(-2.4,0);\draw (-1.6,0)--(-1.4,0); \draw (-3.6,1)--(-3.4,1);\draw (-2.6,1)--(-2.4,1);\draw (-1.6,1)--(-1.4,1); \node at (-3.5,-1.2) {$x_1$};\node at (-2.5,-1.2) {$x_2$};\node at (-1.5,-1.2) {$x_3$}; \node at (-3.5,2.2) {$y_1$};\node at (-2.5,2.2) {$y_2$};\node at (-1.5,2.2) {$y_3$}; \node at (-3.8,0) {$t_1$};\node at (-2.2,0) {$t_2$};\node at (-1.2,0) {$t_3$}; \node at (-3.8,1) {$t_4$};\node at (-2.8,1) {$t_5$};\node at (-1.2,1) {$t_6$}; \node at (-2.5,-1.65) {$\Gamma$};
\draw[->] (0,0)--(1,1); \draw[->] (1,0)--(0,1); \draw[fill=black] (0.5,0.5) circle (0.1cm); \draw[->] (2,0)--(3,1); \draw[->] (3,0)--(2,1); \draw[fill=black] (2.5,0.5) circle (0.1cm); \draw[->] (4,0)--(5,1); \draw[->] (5,0)--(4,1); \draw[fill=black] (4.5,0.5) circle (0.1cm); \draw[->] (6,0)--(6,1); \draw[->] (7,0)--(7,1); \draw[->] (8,0)--(8,1);
\node at (0,-.2) {$x_1$};\node at (1,-.2) {$x_2$}; \node at (2,-.2) {$t_2$};\node at (3,-.2) {$t_3$}; \node at (4,-.2) {$t_4$};\node at (5,-.2) {$t_5$}; \node at (6,-.2) {$x_3$}; \node at (7,-.2) {$t_1$}; \node at (8,-.2) {$t_6$};
\node at (0,1.2) {$t_1$};\node at (1,1.2) {$t_2$}; \node at (2,1.2) {$t_5$};\node at (3,1.2) {$t_6$}; \node at (4,1.2) {$y_1$};\node at (5,1.2) {$y_2$}; \node at (6,1.2) {$t_3$}; \node at (7,1.2) {$t_4$}; \node at (8,1.2) {$y_3$};
\node at (0.5,-0.65) {$\G_1$}; \node at (2.5,-0.65) {$\G_2$}; \node at (4.5,-0.65) {$\G_3$}; \node at (6,-0.65) {$\G_4$};\node at (7,-0.65) {$\G_5$};\node at (8,-0.65) {$\G_6$}; 
\end{tikzpicture}

This brings us to the end.




The |node| can take numerous options who are then used to set the typesetting of the text that follows:


\begin{texexample}{Draw a Line}{ex:line}
\begin{tikzpicture}
\draw[->] (0,0) -- (1.5,0) node[draw, ellipse] {First Node} -| (0, 1.2) node[draw,ellipse,rotate=45, text width=3cm, fill=creamy, text justified] {\lorem};
\end{tikzpicture}
\end{texexample}


\begin{texexample}{Draw a Line}{ex:line}
\begin{tikzpicture}[funny ellipse/.style = {draw,ellipse,rotate=45, text width=3cm, fill=creamy, text justified} ]
\draw[->] (0,0) -- (1.5,0) node[draw, ellipse] {First Node} -| (0, 1.2) node[funny ellipse] {\lorem};
\end{tikzpicture}
\end{texexample}

This can also be written by using \docAuxCommand{tikzset} for setting out all the keys. This can written just before the environment or within the scope of the environment. See \href{https://tex.stackexchange.com/questions/52372/should-tikzset-or-tikzstyle-be-used-to-define-tikz-styles}{TX.SX discussion}, for the option to set |\tikzstyle| which should not be used, even if it is quicker to write.


\begin{texexample}{Draw a Line}{ex:line}
\tikzset{funny ellipse/.style = {draw,ellipse,rotate=45, text width=3cm, fill=creamy, text justified} }
\begin{tikzpicture}
\draw[->] (0,0) -- (1.5,0) node[draw, ellipse] {First Node} -| (0, 1.2) node[funny ellipse] {\lorem};
\end{tikzpicture}
\end{texexample}

A |node| can possibly be rendered with a choice from a list of over 720 keys.

ed. 



Using the |TikZ| package you can draw figures and intermingle them with text. To draw a simple diamond as shown in \fref{fig:diamond} we use
the following commands. The package comes with a very comprehensive manual of over 500 pages long. One can state that there is nothing that you cannot draw with PGF/TikZ, if you have the patience and perseverance. TikZ's language has a syntax of its own with very little connection to what we have used so far. You will need to set aside adequate time to study this, especially if your work has a lot of specially drawn figures that you need. The result like anything else in \tex make the effort worthwhile.

\begin{texexample}{Draw a Diamond}{fig:diamond}
\begin{tikzpicture}
 \draw (1,0) -- (0,1) -- (-1,0) -- (0,-1) -- cycle;
\end{tikzpicture}
\end{texexample}


\begin{texexample}{Text long path}{ex:decorations}
\begin{tikzpicture}
\draw [help lines] grid (3,2);
\draw [red, dashed]
[postaction={decoration={text along path, text={a big juicy apple},
text align=fit to path}, decorate}]
(0,0) .. controls (0,2) and (3,2) .. (3,0);
\node (A) at (1.5,0) {!};
\end{tikzpicture}
\end{texexample}


\begin{texexample}{Text long path}{ex:decorations}

Hello \begin{pgfpicture}
\pgfpathrectangle{\pgfpointorigin}{\pgfpoint{2ex}{1ex}}
\pgfusepath{stroke}
\end{pgfpicture} World!

\end{texexample}


\emphasis{-,draw,begin,end,tikzpicture}
\begin{teXXX}
\begin{tikzpicture}
\draw (1,0) -- (0,1) -- (-1,0) -- (0,-1) -- cycle;
\end{tikzpicture}
\end{teXXX}



\makeatletter
The value of $x$ is \pgfsys@markposition{here}important.

Lots of text.
\hbox{\pgfsys@markposition{myorigin}%
\begin{pgfpicture}
% Switch of size protocol
\pgfpathmoveto{\pgfpointorigin}
\pgfusepath{use as bounding box}
\pgfsys@getposition{here}{\hereposition}
\pgfsys@getposition{myorigin}{\thispictureposition}
\pgftransformshift{\pgfpointscale{-1}{\thispictureposition}}
\pgftransformshift{\hereposition}
\pgfpathcircle{\pgfpointorigin}{1cm}
\pgfusepath{draw}
\end{pgfpicture}}

\makeatother


You cannot write directly into a picture environment. The command \docAuxCommand{pgftext} can be used. 

\begin{texexample}{Using text directly}{ex:pgftext}
\tikz{\draw[help lines] (0,0) grid (3,2);
\pgftext[base,x=1cm,y=0.5cm] {lovely}}
\end{texexample}





Sometimes it is quite useful when debugging to add a backround grid. 


\begin{centering}
\begin{tikzpicture}
\draw[step=0.25cm,color=creamy] (-1,-1) grid (1,1);
\draw [color=bgsexy](1,0) -- (0,1) -- (-1,0) -- (0,-1) -- cycle;
\end{tikzpicture}
\captionof{figure}{You can add a background grid using \texttt{step=0.25cm, color=green} as an option}
\end{centering}


\emphasis{step,color,green,grid,begin,end}
\begin{teXXX}
\begin{tikzpicture}
  \draw[step=0.25cm,color=green] (-1,-1) grid (1,1);
  \draw (1,0) -- (0,1) -- (-1,0) -- (0,-1) -- cycle;
\end{tikzpicture}
\end{teXXX}

The grid is specified by providing two diagonally opposing points: (-1,-1)
and (1, 1). The two options supplied give a step size for the grid lines and a
specification for the color of the grid lines, using the \docpkg{xcolor} package

\subsection{Specifying points and paths}

\begin{texexample}{Specifying points and paths}{ex:points}
\centering
\begin{tikzpicture}[scale=1.8]
% Define the points of a regular pentagon
\path (0,0) coordinate (origin);
\path (0:1cm) coordinate (P0);
\path (1*72:1cm) coordinate (P1);
\path (2*72:1cm) coordinate (P2);
\path (3*72:1cm) coordinate (P3);
\path (4*72:1cm) coordinate (P4);
% Draw the edges of the pentagon
\draw[color=bgsexy] (P0) -- (P1) -- (P2) -- (P3) -- (P4) -- cycle;
% Add "spokes"
\draw[color=red800] (origin) -- (P0) (origin) -- (P1) (origin) -- (P2)
(origin) -- (P3) (origin) -- (P4);
\end{tikzpicture}
\captionof{figure}{Drawing a complicated polygon, using paths and the \texttt{draw} command}
\end{texexample}


Two key ideas used in \tikzname\ are points and paths. Both of these ideas were used
in the diamond examples. Much more is possible, however. For example, points
can be specified in any of the following ways:
\begin{enumerate}
\item  Cartesian coordinates
\item  Polar coordinates
\item  Named points
\item  Relative points
\end{enumerate}



\subsection{coordinates}
The cartesian coordinates can be defined and named using the following syntax.

%\emphasis{begin,end,coordinate,at,draw}
%\begin{teXXX}
%\begin{tikzpicture}
%  \coordinate (A) at (0,0);
%  \coordinate (B) at (1.25,0.25);
%  \draw[blue] (A) -- (B);
%\end{tikzpicture}
%\end{teXXX}

\noindent This produces:
\begin{tikzpicture}
\coordinate (A) at (0,0);
\coordinate (B) at (1.25,0.25);
\draw[blue] (A) -- (B);
\end{tikzpicture}


We can add labels to the points by using the |label| option. A label is distinct from the text of a |node|.

\begin{tikzpicture}
\coordinate [label=left:\textcolor{orange}{$A$}] (A) at (0,0);
\coordinate [label=right:\textcolor{orange}{$B$}]  (B) at (1.15,0.25);
\draw[blue] (A) -- (B);
\end{tikzpicture}

\emphasis{label,left,label:,right}
\begin{teXXX}
\begin{tikzpicture}
  \coordinate [label=left:\textcolor{orange}{$A$}] (A) at (0,0);
  \coordinate [label=west:\textcolor{orange}{$B$}] (B) at (1.25,0.25);
  \draw[blue] (A) -- (B);
\end{tikzpicture}
\end{teXXX}




If you tempted to write \texttt{label=top:} it will not work, as the command accepts the following keywords.

\begin{tikzpicture}
  \coordinate [label=left:\textcolor{orange}{east}]  (A) at (0,0);
  \coordinate [label=right:\textcolor{orange}{west}] (B) at (0,0);
  \draw[blue] (A)--(B);
\end{tikzpicture}


\section{Graphic Parameters: Line Width, Line Cap, and Line Join}

The width of lines can be specified using the key:

\begin{docKey}[tikz]{line width}{=\marg{dimension}} {no default, initially 0.4pt}
Specifies the line width \seepgfmanual{166}
\end{docKey}



\bgroup
\def\mkl#1{\tikz \draw[#1] (0,0)--(1.0, 1.5ex);}
\scriptsize\arial
\begin{tabular}{|l|l|l|l|l|l|l|l|}
\hline
\mkl{line width=2pt}& \mkl{ultra thin} &\mkl{very thin} & \mkl{thin} & \mkl{semithick} & \mkl{thick} &\mkl{very thick} &\mkl{ultra thick} \\
\hline
line width=2pt &ultra thin & very thin & thin &semithick & thick & very thick & ultra thick \\
\hline
\end{tabular}
\egroup

\begin{docKey}[tikz]{line cap}{=\marg{dimension}} {no default, initially 0.4pt}
Specifies how lines “end.” Permissible types are round, rect, and butt \seepgfmanual{167}. 
\end{docKey}

\bgroup
\def\mkl#1{\begin{tikzpicture} \draw[line width=10pt, line cap=#1] (0,0)--(1.0, 1.5ex);\draw[white,line width=2pt]
(0,0 )--(1.0,1.5ex);\end{tikzpicture}}
\scriptsize\arial
\begin{tabular}{|l|l|l|}
\hline
\mkl{rect}& \mkl{butt} &\mkl{round}  \\
\hline
rect &butt & round \\
\hline
\end{tabular}
\egroup




\begin{docKey}[tikz]{line join}{=\marg{type}}{no default, initially miter}
Specifies how lines “join.” Permissible type are round, bevel, and miter. They have the following
effects:
\end{docKey}

\begin{texexample}{Joining Lines}{es:joinlines}
\begin{tikzpicture}[line width=10pt]
\draw[line join=round] (0,0) -- ++(.5,1) -- ++(.5,-1);
\draw[line join=bevel] (1.25,0) -- ++(.5,1) -- ++(.5,-1);
\draw[line join=miter] (2.5,0) -- ++(.5,1) -- ++(.5,-1);
\end{tikzpicture}
\end{texexample}


\begin{docKey}[tikz]{dash pattern}{=\marg{dash pattern}}{no default}
Sets the dashing pattern. The syntax is the same as in \metafontlogo. For example following pattern on
2pt off 3pt on 4pt off 4pt means \enquote{draw 2pt, then leave out 3pt, then draw 4pt once more, then
leave out 4pt again, repeat}.
\end{docKey}

\bgroup
\def\ml#1{\tikz \draw[ #1] (0pt,0pt) -- (50pt,0pt);}
\def\alist{solid, dotted, densely dotted, loosely dotted,% 
           dashed,densely dashed, loosely dashed, %
           dash dot, densely dash dot, loosely dash dot, %
           dash dot dot, densely dash dot dot, loosely dash dot dot.}

For patterns there are numerous settings {\arial \alist }


\scriptsize
\begin{tabular}{lll}
\hline
\ml{solid} &  & \\
solid      &  & \\
\hline
\ml{dotted} &\ml{densely dotted} & \ml{loosely dotted}\\
\textit{dotted} & densely dotted  &loosely dotted \\
\hline
\ml{dashed} & \ml{densely dashed} & \ml{loosely dashed}  \\
\textit{dashed}      & densely dashed & loosely dashed            \\
\hline

\ml{dash dot} & \ml{densely dash dot} & \ml{loosely dash dot} \\
\textit{dash dot} & densely dash dot & loosely dash dot \\
\hline

\ml{dash dot dot} & \ml{densely dash dot dot} & \ml{loosely dash dot dot} \\
\textit{dash dot dot} & densely dash dot dot & loosely dash dot dot \\
\hline
\end{tabular}
\egroup


\subsection{Pattern Library}

The library patterns can be used to draw predetermined patterns. This will be a longer than usual section as it explains how to create new patterns. Most of the content is straight from the \pgfname manual. Before we start with the creation f a new pattern let us examine how a pattern is used.

\begin{texexample}{Using Library Patterns}{ex:libpatterns}
\begin{tikzpicture}
\pattern [path fading=west,pattern=checkerboard light gray]
      (0,0) rectangle (5cm,2em);
\end{tikzpicture}
\end{texexample}


\label{section-library-patterns}


The package defines patterns for filling areas. \docAuxCommand*{usetikzlibrary}\marg{patterns}.




\subsection{Form-Only Patterns}

\begin{tabular}{ll}
  \emph{Pattern name} & \emph{Example (pattern in black, blue, and red
    on faded checkerboard)} \\ 
  \patternindex{horizontal lines} 
  \patternindex{vertical lines} 
  \patternindex{north east lines} 
  \patternindex{north west lines} 
  \patternindex{grid} 
  \patternindex{crosshatch} 
  \patternindex{dots} 
  \patternindex{crosshatch dots} 
  \patternindex{fivepointed stars} 
  \patternindex{sixpointed stars} 
  \patternindex{bricks}
  \patternindex{checkerboard}
\end{tabular}
  
\subsection{Inherently Colored Patterns}


\begin{tabular}{ll}
  \emph{Pattern name} & \emph{Example} \\
  \patternindexinherentlycolored{checkerboard light gray} 
  \patternindexinherentlycolored{horizontal lines light gray} 
  \patternindexinherentlycolored{horizontal lines gray} 
  \patternindexinherentlycolored{horizontal lines dark gray} 
  \patternindexinherentlycolored{horizontal lines light blue} 
  \patternindexinherentlycolored{horizontal lines dark blue} 
  \patternindexinherentlycolored{crosshatch dots gray} 
  \patternindexinherentlycolored{crosshatch dots light steel blue} 
\end{tabular}
  


% Copyright 2006 by Till Tantau
%
% This file may be distributed and/or modified
%
% 1. under the LaTeX Project Public License and/or
% 2. under the GNU Free Documentation License.
%
% See the file doc/generic/pgf/licenses/LICENSE for more details.


\section{Creating Patterns}

\label{section-patterns}

\subsection{Overview}

There are many ways of filling a path. First, you can fill it using a
solid color and this is also the fastest method. Second, you can also
fill it using a shading, which means that the color changes smoothly
between two (or more) different colors. Third, you can fill it using a
tiling pattern and it is explained in the following how this is done.

A tiling pattern can be imagined as a rectangular tile (hence the
name) on which a small picture is painted. There is not a single tile,
but (conceptually) an infinite number of tiles, all showing the same
picture, and these tiles are arranged horizontally and vertically to
fill the plane. When you use a tiling pattern to fill a path, what
happens is that the path clips out a ``window'' through which we see
part of this infinite plane.

Patterns come in two versions: \emph{inherently colored patterns} and
\emph{form-only patterns}. (These are often called ``color patterns''
and ``uncolored patterns,'' but these names are misleading since
uncolored patterns do have a color and the color changes. As I said,
the name is misleading\dots) An inherently colored pattern is just a
colored tile like, say, a red star with a black outline. A form-only
pattern can be imagined as a tile that is a kind of rubber stamp. When
this pattern is used, the stamp is used to print copies of the stamp
picture onto the plane, but we can use a different stamp color each
time we use a form-only pattern.

\pgfname\ provides a special support for patterns. You can declare a
pattern and then use it very much like a fill color. \pgfname\
directly maps patterns to the pattern facilities of the underlying
graphic languages (PostScript, \textsc{pdf}, and \textsc{svg}). This
means that filling a path using a pattern will be nearly as fast as if
you used a uniform color.

There are a number of pitfalls and restrictions when using
patterns. First, once a pattern has been declared, you cannot change
it anymore. In particular, it is not possible to enlarge it or change
the line width. Such flexibility would require that the repeating of
the pattern were not done by the graphic language, but on the
\pgfname\ level. This would make patterns orders of magnitude slower
to produce and to render. However, \pgfname{} does provide a
more-or-less successful emulation of ``mutable'' patterns, although
internally, a new (fixed) instance of a pattern is declared when
the parameters of a pattern change.

Second, the phase of patterns is not well-defined, that is, it is not
clear where the origin of the ``first'' tile is. To be more precise,
PostScript and \textsc{pdf} on the one hand and \textsc{svg} on the
other hand define the origin differently. PostScript and \textsc{pdf}
define a fixed origin that is independent of where the path lies. This
has the highly desirable effect that if you use the same pattern to
fill multiple paths, the outcome is the same as if you had filled a 
single path consisting of the union of all these paths. By
comparison, \textsc{svg} uses the upper-left (?) corner of the path to
be filled as the origin. However, the \textsc{svg} specification is a
bit vague on this question.


\subsection{Declaring a Pattern}

Before a pattern can be used, it must be declared. The following
command is used for this:

\begin{docCommand}{pgfdeclarepatternformonly}{%
	\oarg{variables}%
	\marg{name}%
	\marg{bottom left}%
	\marg{top right}%
	\marg{tile size}%
	\marg{code}}

	This command declares a new form-only pattern. The \meta{name} is a
  name for later reference. The two parameters \meta{lower left} and
  \meta{upper right} must describe a bounding box that is large enough
  to encompass the complete tile.
\end{docCommand}

  The size of a tile is given by \meta{tile size}, that is, a tile is
  a rectangle whose lower left   corner is the origin and whose upper
  right corner is given by \meta{tile size}. This might make you
  wonder why the second and third parameters are needed. First, the
  bounding box might be smaller than the tile size if the tile is
  larger than the picture on the tile. Second, the bounding box might
  be bigger, in which case the picture will ``bleed'' over the tile.

  The \meta{code} should be \pgfname\ code than can be protocolled. It
  should not contain any color code.


\begin{codeexample}[]
\pgfdeclarepatternformonly{stars}
{\pgfpointorigin}{\pgfpoint{1cm}{1cm}}
{\pgfpoint{1cm}{1cm}}
{
  \pgftransformshift{\pgfpoint{.5cm}{.5cm}}
  \pgfpathmoveto{\pgfpointpolar{0}{4mm}}
  \pgfpathlineto{\pgfpointpolar{144}{4mm}}
  \pgfpathlineto{\pgfpointpolar{288}{4mm}}
  \pgfpathlineto{\pgfpointpolar{72}{4mm}}
  \pgfpathlineto{\pgfpointpolar{216}{4mm}}
  \pgfpathclose%
  \pgfusepath{fill}
}
\begin{tikzpicture}
  \filldraw[pattern=stars] (0,0)   rectangle (1.5,2);
  \filldraw[pattern=stars,pattern color=red]
                           (1.5,0) rectangle (3,2);
\end{tikzpicture}
\end{codeexample}

	The optional argument \meta{variables} consists of a comma
	separated	list of macros,	registers or keys, representing the
	parameters of the pattern that may vary. If a variable is a key,
	then the full path name must be used (specifically, it must start
	with |/|).
	As an example, a list might look like the following:
	|\mymacro,\mydimen,/pgf/my key|. Note that macros and keys should
	be ``simple''. They should only store values in themselves.
	
	The effect of \meta{variables}, is the following:
  Normally, when this argument is empty, once a pattern has been
  declared, it becomes ``frozen''. This means that it is not possible
  to enlarge the pattern or change the line width later on.
  By specifying \meta{variables}, no pattern is actually created.
  Instead, the arguments are stored away
  (so the macros,	registers or keys do not have to be defined in advance).

  When the fill pattern is set, \pgfname{} checks if the pattern has
  already been created with the \meta{variables} set to their current
  values (\pgfname{} is usually ``smart enough'' to distinguish between
  macros, registers and keys). If so, this already-declared-pattern
  is used as the fill pattern.
  If not, a new instance of the pattern (which will have a
  unique internal name) is declared using the current values of
  \meta{variables}. These values are then saved and the fill pattern
  set accordingly.
	
	The following shows an example of a pattern which varies
	according to the values of the macro |\size|, the key |/tikz/radius|,
	and the \TeX{} dimension |\thickness|.

\begin{texexample}{New Pattern Example}{ex:newpattern}
\pgfdeclarepatternformonly[/tikz/radius,\thickness,\size]{rings}
{\pgfpoint{-0.5*\size}{-0.5*\size}}
{\pgfpoint{0.5*\size}{0.5*\size}}
{\pgfpoint{\size}{\size}}
{
  \pgfsetlinewidth{\thickness}
  \pgfpathcircle\pgfpointorigin{\pgfkeysvalueof{/tikz/radius}}
  \pgfusepath{stroke}
}
\newdimen\thickness
\tikzset{
  radius/.initial=4pt,
  size/.store in=\size, size=20pt,
  thickness/.code={\thickness=#1},
  thickness=0.75pt
}
\begin{tikzpicture}[rings/.style={pattern=rings}]
  \filldraw [rings, radius=2pt, size=6pt]      (0,0)   rectangle +(1.5,2);
  \filldraw [rings, radius=2pt, size=8pt]      (2,0)   rectangle +(1.5,2);
  \filldraw [rings, radius=6pt, thickness=2pt] (0,2.5) rectangle +(1.5,2);
  \filldraw [rings, radius=8pt, thickness=4pt] (2,2.5) rectangle +(1.5,2);
\end{tikzpicture}
\end{texexample}



\begin{docCommand}{pgfdeclarepatterninherentlycolored}{\oarg{variables}
    \marg{name}
    \marg{lower left}
    \marg{upper right}
    \marg{tile size}
    \marg{code}}
  This command works like |\pgfdeclarepatternuncolored|, only the
  pattern will have an inherent color. To set the color, you should
  use \pgfname's color commands, not the |\color| command, since this
  fill is not protocolled.
\end{docCommand}

\begin{texexample}{Inherently Colored}{ex:ingerentlycolored}
\pgfdeclarepatterninherentlycolored{green stars}
{\pgfpointorigin}{\pgfpoint{1cm}{1cm}}
{\pgfpoint{1cm}{1cm}}
{
  \pgfsetfillcolor{green!50!black}
  \pgftransformshift{\pgfpoint{.5cm}{.5cm}}
  \pgfpathmoveto{\pgfpointpolar{0}{4mm}}
  \pgfpathlineto{\pgfpointpolar{144}{4mm}}
  \pgfpathlineto{\pgfpointpolar{288}{4mm}}
  \pgfpathlineto{\pgfpointpolar{72}{4mm}}
  \pgfpathlineto{\pgfpointpolar{216}{4mm}}
  \pgfpathclose%
  \pgfusepath{stroke,fill}
}
\begin{tikzpicture}
  \filldraw[pattern=green stars] (0,0) rectangle (3,2);
\end{tikzpicture}
\end{texexample}



\subsection{Setting a Pattern}

Once a pattern has been declared, it can be used.

\begin{docCommand}{pgfsetfillpattern}{\marg{name}\marg{color}}
  This command specifies that paths that are filled should be filled
  with the ``color'' by the pattern \meta{name}. For an inherently
  colored pattern, the \meta{color} parameter is ignored. For
  form-only patterns, the \meta{color} parameter specifies the color
  to be used for the pattern.
\end{docCommand}
  
\begin{codeexample}[]
\begin{tikzpicture}
  \pgfsetfillpattern{stars}{red}
  \filldraw (0,0) rectangle (1.5,2);

  \pgfsetfillpattern{green stars}{red}
  \filldraw (1.5,0) rectangle (3,2);
\end{tikzpicture}
\end{codeexample}



\endinput
%To summarize, what we have been doing so far is to learn a set of primitive TikZ commands for drawing paths, drawing shapes and labeling them. All TikZ command work by passing options to them. For example to change the above line to an arrow, we just pass the option |->| to the |draw| command.
%

%\begin{tikzpicture}
%  \coordinate [label=left:\textcolor{orange}{$A$}] (A) at (0,0);
%  \coordinate [label=right:\textcolor{orange}{$B$}] (B) at (1.25,0.25);
%  \draw[->,o-stealth] (A)--(B);
%\end{tikzpicture}
%\caption{Effect of the option \protect\texttt{draw[->]}.}

%\emphasis{begin,end,->,draw}
%\begin{teXXX}
%\begin{tikzpicture}
%  ...
%  ...
%  \draw[->,blue] (A)--(B);
%\end{tikzpicture}
%\end{teXXX}
%
%\section*{Relative coordinates}
%\index{TikZ!coordinates, relative}
%A coordinate can be made "relative" by prefixing it with |++|. relative coordinates are useful in many applications.
%\medskip
%
%\noindent The code is simple, except before the coordinate you add the |++| signs. This tells the PGF engine to add the x,y dimensions of the new coordinate to that of its predecessor's. In many instances this is more intuitive and easier to determine.



%\begin{tikzpicture}
%\draw[step=0.5cm,color=gray] (-1,-1) grid (3.5,3);
%\draw[->,red,thick] (0,0) -- ++(1,0) -- ++(0,1) -- ++(-1,0) -- cycle;
%\draw[->,red,thick] (2,0) -- ++(1,0) -- ++(0,1) -- ++(-1,0) -- cycle;
%\draw[arrows=o-stealth,blue] (1.5,1.5) -- ++(1,0) -- ++(0,1) -- ++(-1,0) -- cycle;
%\end{tikzpicture}
%\caption{Example of use of the \protect\texttt{++} to specify relative coordinates.}
%\label{fig:relative}

%\begin{teXXX}
%\begin{tikzpicture}
%  \draw[step=0.5cm,color=gray] (-1,-1) grid (3.5,3);
%  \draw[red,very thick] (0,0) -- ++(1,0) -- ++(0,1) -- ++(-1,0) -- cycle;
%  \draw[red,very thick] (2,0) -- ++(1,0) -- ++(0,1) -- ++(-1,0) -- cycle;
%  \draw[->,red,very thick] (1.5,1.5) -- ++(1,0) -- ++(0,1) -- ++(-1,0) -- cycle;
%\end{tikzpicture}
%\end{teXXX}
%
%Instead of |++| you can also use a single |+|. This also specifies a relative coordinate, but it does not "update"
%the current point for subsequent usages of relative coordinates. Thus, you can use this notation to specify
%numerous points, all relative to the same "initial" point:
%

%\begin{tikzpicture}
%\draw[step=0.5cm,color=gray] (-1,-1) grid (3.5,3);
%\draw[purple, fill=white] (0,0) -- +(1,0) -- +(1,1) -- +(0,1) -- cycle;
%\draw[purple, fill=white] (2,0) -- +(1,0) -- +(1,1) -- +(0,1) -- cycle;
%\draw[purple, fill=white] (1.5,1.5) -- +(1,0) -- +(1,1) -- +(0,1) -- cycle;
%\path (0,0) node [shape=circle,draw]{(0,0)};
%\end{tikzpicture}
%\caption{Example of use of the \protect\texttt{+} to specify relative coordinates.}
%\label{fig:relative1}

%\begin{teXXX}
%  \draw (0,0) -- +(1,0) -- +(1,1) -- +(0,1) -- cycle;
%  \draw (2,0) -- +(1,0) -- +(1,1) -- +(0,1) -- cycle;
%  \draw (1.5,1.5) -- +(1,0) -- +(1,1) -- +(0,1) -- cycle;
%\end{teXXX}
%
%
%Personally, I don't favour this method of specifying co-ordinates, but it can be useful, if you are automating the production of figures through an external script\sidenote{For drawing Bezier curves, the \texttt{+} behaves differently.  You can refer to the PGF Manual for more details.}.
%
%
%\section*{Arrows}
%\index{TikZ>arrows}
%The function |->| creates a tooltip arrow. You can use different arrow tips and there is a long section for them in the PGF manual. You can even define your own.

\bgroup
%\centering
%\begin{tikzpicture}
%  \draw[->] (0,0) -- (2,0);
%  \draw[arrows=o-stealth,blue] (0,-0.3) -- (2,-0.3);
%  \draw[->,o-stealth,orange] (0,-0.6) -- (2,-0.6);
%  \draw[arrows=|-stealth,purple] (0,-0.9) -- (2,-0.9);
%\end{tikzpicture}
%\captionof{figure}{Special arrow endings}
%\label{fig:specials}
\egroup
%
%\emphasis{o,stealth,begin,end,draw}
%\begin{teXXX}
%\begin{tikzpicture}
% \draw[->] (0,0) -- (2,0);
% \draw[arrows=o-stealth,blue] (0,-0.3) -- (2,-0.3);
% \draw[->,o-stealth,orange] (0,-0.6) -- (2,-0.6);
% \draw[arrows=X-stealth,purple] (0,-0.9) -- (2,-0.9);
%\end{tikzpicture}
%\end{teXXX}

%

\begin{verbatim}
\begin{tikzpicture}
% Define the points of a regular pentagon
\path (0,0) coordinate (origin);
\path (0:1cm) coordinate (P0);
\path (1*72:1cm) coordinate (P1);
\path (2*72:1cm) coordinate (P2);
\path (3*72:1cm) coordinate (P3);
\path (4*72:1cm) coordinate (P4);
% Draw the edges of the pentagon
\draw (P0) -- (P1) -- (P2) -- (P3) -- (P4) -- cycle;
% Add "spokes"
\draw (origin) -- (P0) (origin) -- (P1) (origin) -- (P2)
(origin) -- (P3) (origin) -- (P4);
\end{tikzpicture}
\end{verbatim}





\section{Nodes}

A node is a small part of a picture. When a node is created, you provide a position where the node
should be drawn and a shape. A node of shape circle will be drawn as a |circle|, a node of shape |rectangle|
as a rectangle, and so on. A node may also contain same text, which is why they can used nodes to show text.

Finally, a node can get a name for later reference.



\emphasis{node,shape,draw}
\begin{teXXX}
\begin{tikzpicture}
\path ( 0,2) node [shape=circle,draw] {.}
( 0,1) node [shape=circle,draw] {..}
( 0,0) node [shape=circle,draw] {...}
( 1,1) node [shape=rectangle,draw] {....}
(-2,1) node [shape=rectangle,draw] {rectangle (-2,1)};
\end{tikzpicture}
\end{teXXX}
\medskip

\begin{tikzpicture}
\path ( 0,2) node [shape=circle,draw] {1}
( 0,1) node [shape=circle,draw] {\textbf{10}}
( 0,0) node [shape=circle,draw] {\textbf{100}}
( 1,1) node [shape=circle,draw] {\textbf{1000}}
(-2,1) node [shape=circle,draw] {\textbf{10000}};
\end{tikzpicture}

In the above code, this text is empty (because of the
|empty {}|). So, why do we see anything at all at all the nodes? The answer is the draw option for the node operation: It
causes the |shape| around the text" to be drawn. If you have an empty |{}|, PGF still sees the empty space as a character and justs draws around it. The reason is than TikZ automatically adds some space around the text. The amount is set
using the option |inner sep|. So, to increase the size of the nodes. Modifying the example slightly we get.



\begin{tikzpicture}
\path ( 0,2) node [shape=circle,draw] {.}
( 0,1) node [shape=circle,draw] {..}
( 0,0) node [shape=circle,draw] {...}
( 1,1) node [shape=circle,draw] {....}
(-1,1) node [shape=circle,draw] {.....};
\end{tikzpicture}

As you can observe the size of the circle has been adjusted to fit the text that is enclosing it. 
Another way to simply add a node is using the |at| syntax:

\begin{texexample}{The node command}{}
\begin{tikzpicture}
\node at (0,0) [circle, draw] {\textbf{100}};
\node at (1,1) [diamond,draw] {\textbf{100}};
\end{tikzpicture}
\end{texexample}

The \cmd{\node} is an abbreviation of the |\path| node. This is a much shorter syntax than |\path| where one would need to add a lot of redundant move-tos  \seepgfmanual{215}.

If you have many nodes another way of achieving the example outlined above is to use the |\draw| command in comination with node and at.

\begin{texexample}{The node command}{}
\begin{tikzpicture}
\tikz \draw[fill=yellow!80!black]
(0,0) node {first node}
-- (1,1) node[draw, behind path] {second node}
-- (0,2) node[fill=red!20,draw,double,rounded corners] {third node};

\node at (0,0) [circle, draw] {\textbf{100}};
\node at (1,1) [diamond,draw]{\textbf{100}};
\end{tikzpicture}
\end{texexample}

\subsection*{Drawing shapes}

PGF abd \tikzname\ come with a number of predefined shapes:
\begin{itemize}
\item rectangle
\item circle, and
\item coordinate
\end{itemize}


\begin{tikzpicture}
\draw (0,0) circle (1cm);
\draw (0.5,0) circle (0.5cm);
\draw (0,0.5) circle (0.5cm);
\draw (-0.5,0) circle (0.5cm);
\draw (0,-0.5) circle (0.5cm);
\end{tikzpicture}



A circle is specified by providing its center point and the desired radius. The
command:

\medskip

\begin{tikzpicture}
  \draw[step=0.25cm,color=green] (-1,-1) grid (1,1);
  \draw (0,0) circle (1cm);
\end{tikzpicture}
\medskip

\begin{teXXX}
\begin{tikzpicture}
  \draw (x,y) circle (dia);
\end{tikzpicture}
\end{teXXX}



You  can use one |\draw| command to draw multiple circles as shown in \fref{fig:circles}


\begin{tikzpicture} 
 \draw (0,0) 
  circle (1cm)
  circle (0.6cm)
  circle (0.2cm)
 ;
\end{tikzpicture}

\emphasis{circle,begin,end}
\begin{teXXX}
\begin{tikzpicture} 
 \draw (0,0) 
  circle (1cm)
  circle (0.6cm)
  circle (0.2cm)
 ;
\end{tikzpicture}
\end{teXXX}





\begin{center}
\begin{tikzpicture}
\draw (0,0) circle (1cm)
circle (0.6cm)
circle (0.2cm);
\end{tikzpicture}
\captionof{figure}{You can use one draw command to draw multiple circles}
\label{fig:circles}
\end{center}
\captionof{figure}{Drawing multiple circles, using mutiple \texttt{circle} commands}


\subsection{Drawing ellipses}

Ellipses can be drawn in a similar fashion to circles. As an ellipse needs two center points to be specified the command used has the following general form:

\begin{verbatim}
\draw (a,b) ellipse (r1 dim and r2 dim);
\end{verbatim}

We can draw two ellipses as shown in the figure, using the code:
\begin{teX}
\begin{tikzpicture}[scale=0.6]
\draw[color=red] (0,0) ellipse (2cm and 1cm);
\draw[color=red] (0,0) ellipse (1cm and 2cm);
\end{tikzpicture}
\end{teX}

\begin{centering}
\begin{tikzpicture}[scale=0.6]
\draw[color=red] (0,0) ellipse (2cm and 1cm);
\draw[color=red] (0,0) ellipse (1cm and 2cm);
\end{tikzpicture}
\caption[Drawing ellipses]{Use the draw command in combination with ellipse to draw ellipses}
\end{centering}


\begin{teX}
\begin{tikzpicture}
\draw (0,0) ellipse (2cm and 1cm)
ellipse (0.5cm and 1 cm)
ellipse (0.5cm and 0.25cm);
\end{tikzpicture}
\caption{Drawing multiple circles, using mutiple \texttt{draw} commands}
\end{teX}

\section{Drawing more complicated shapes}
we can place a parabola in a rectangle as shown in \fref{fig:parabola}, by using the |rectangle| and the |parabola| options.

\bgroup
\centering

\begin{tikzpicture}
\draw[color=blue] (0,0) rectangle (1,1.5)
(0,0) parabola[color=orange] (1,1.5);
\draw[xshift=1.5cm] (0,0) rectangle (1,1.5)
(0,0) parabola[bend at end] (1,1.5);
\draw[xshift=3cm] (0,0) rectangle (1,1.5)
(0,0) parabola bend (.75,1.75) (1,1.5);
\end{tikzpicture}
\captionof{figure}{Parabolas drawn using the parabola and rectangle options.}
\label{fig:parabola}
\egroup




\emphasis{parabola,rectangle}
\begin{teX}
\begin{tikzpicture}
\draw[color=blue] (0,0) rectangle (1,1.5)
(0,0) parabola[color=orange] (1,1.5);
\draw[xshift=1.5cm] (0,0) rectangle (1,1.5)
(0,0) parabola[bend at end] (1,1.5);
\draw[xshift=3cm] (0,0) rectangle (1,1.5)
(0,0) parabola bend (.75,1.75) (1,1.5);
\end{tikzpicture}
\caption{Parabolas drawn using the parabola command}
\label{fig:parabola}
\end{teX}

\subsection*{The shape library}

\begin{tikzpicture}
\draw [help lines] (0,0) grid (2,2);
\draw [blue, dashed] (1,1) circle(1cm);
\draw [red, dashed] (1,1) circle(.5cm);
\node [star, star point height=.5cm, minimum size=2cm, draw]
at (1,1) {S};
\end{tikzpicture}

\section{Iterations}
One convenient construct provided with TikZ is a |foreach| command sequence

\begin{texexample}{Tikz loops}{tz:ex}
\centering
\begin{tikzpicture}[scale=2, color=bgsexy]
\foreach \i in {1,...,4}
{
  \path (\i,0) coordinate (X\i);
  \fill (X\i) circle (1pt);
}
  \foreach \j in {1,...,3}
{
  \path (\j,1) coordinate (Y\j);
  \fill (Y\j) circle (1pt);
}
\foreach \i in {1,...,4}
{
  \foreach \j in {1,...,3}
  {
     \draw[color=bgsexy] (X\i) -- (Y\j);
  }
}
\end{tikzpicture}
\captionof{figure}{Drawing a bi-partite garph using foreach loops}
\end{texexample}



\section{The pgfplots package}



\subsection{Loading data from files}

Scientific work, especially that associated with research tends to generate
a lot of data. The data would normally come from external applications and stored in files. With |TikZ| one can import the data
by using the word |file|:

\emphasis{addplot,file,x}
\begin{teXXX}
 \addplot file {./raw/wavefunctions/wavefunc\x.dat};
\end{teXXX}

In the example we use a file with a path. The data is saved in
files with the same name but a different ending. We use a |foreach| function to add the ending i.e, the file names are |wavefunc1|, |wavefunc2| and |wavefunc3|. By using external data files and the foreach command it can substantially reduce the amount of text in the macros. This improves debugging and readability.

\begin{texexample}[colback=white]{Loading files}{ex:lfiles}
\centering
\begin{tikzpicture}[scale=0.8]
    \begin{axis}[smooth,
    xlabel=$n$,
    ylabel=$\Theta{j}{n}$]
    \foreach \x in {0,...,2}
    {
        \addplot file {./raw/wavefunctions/wavefunc\x.dat};
    }
    \legend{$j=0$,$j=1$,$j=2$};
    \end{axis}
\end{tikzpicture}
\captionof{figure}{Example plot with data imported from external files, using \texttt{file}}
\end{texexample}


\begin{teXXX}
\begin{tikzpicture}[scale=0.6]
  \begin{axis}[
    xlabel=$n$,
    ylabel=$\Theta{j}{n}$]
    \foreach \x in {0,...,2}
    {
      \addplot file {./raw/wavefunctions/wavefunc\x.dat};
    }
    \legend{$j=0$,$j=1$,$j=2$};
  \end{axis}
\end{tikzpicture}
\end{teXXX}



\section*{Plotting functions}
Functions can be defined for plotting using a variety of methods. They are powerful but generally difficult to remember.



\section{Saving Data to a file}

You can save your data to a file in many ways. One easy way is to use
the \docpkg{filecontents} package. This package extends the LaTeX environment
with the same name, but allows you to overwrite the file {\protect\ctan{filecontents}}.

\begin{teXXX}
\documentclass[justified]{tufte-book}
\usepackage{pgfplots,lipsum,booktabs}
\usepackage{pgfplotstable}
\pgfplotsset{compat=newest}
\usepackage{filecontents*}
\begin{filecontents}{my1.dat}
    Label       value       num
    Integrity     33         4
    Standalone    14         3
    Interface      6         2
    Overall       18         1
\end{filecontents*}
\begin{document}
    your code here ...
\end{document}
\end{teXXX}

It is good practice to keep, such data at the top of your file, although with
the |filecontents| package, they can be inserted anywhere. Sometimes it maybe
easier to have a number of minimal files with the type of charts you using regularly and just update the data on top. In general if the data is entered
by hand rather than generated automatically by software this is a good way
to keep your work tidy.

\newenvironment{Chart}[1][black!70!green]{%
%%  defaults
    \gdef\level##1{Level ##1}
    \def\setchartwidth##1{%
      \def\chartwidth{##1}}%
    \setchartwidth{3.9cm}%
    \def\chartcolor{#1}
    \newcommand\addTitle[2][test]{
    
    
%% For the chart title we set it in a minipage for
%% better control
    \def\charttitle{\minipage{4cm}%
       \footnotesize %
       \centering\textbf{##2}\\##1%
       \endminipage}}%
   \def\xlabel{Completion (\%)}%
%% renders the chart 
    \def\renderChart{%
%%
    \footnotesize%
%%
%%
    \IfFileExists{#1.dat}{Test}{}
   \begin{tikzpicture}
   \begin{axis}[
    xbar, width=\chartwidth,title=\charttitle,
    y=0.5cm, enlarge y limits={true, abs value=0.75},
    xmin=0, xmax=100,enlarge x limits={upper, value=0.25},
    xlabel=\xlabel,
    %ylabel=Label,
    xmajorgrids=true,
    ytick=data,
    yticklabels from table={\dataTable}{Label},
    nodes near coords, nodes near coords align=horizontal
     ]
    \addplot[draw=none, fill=\chartcolor] table [x=value, y=num]
    {\dataTable};
    \end{axis}%
    \end{tikzpicture}}}
{}

\begin{comment}
\begin{figure*}
\centering

\hskip-2cm\begin{Chart}
 \addTitle[Mechanical Systems]{Shangri-la}
 \def\dataTable{SH-mechanical.dat}
 \renderChart
\end{Chart}\hspace{0.3cm}
\begin{Chart}
 \addTitle[FM-200 System]{All areas}
 \def\dataTable{my1.dat}
 \renderChart
\end{Chart}
\begin{Chart}
 \addTitle[Electrical Works]{Merweb}
 \def\dataTable{my6.dat}
 \renderChart
\end{Chart}
\caption{Mechanical Systems Shangrila. Commissioning status}
\end{figure*}


\begin{filecontents*}{my1.dat}
Label     value       num
Integrity         33            4
Standalone      14            3
Interface        6            2
Overall           18            1
\end{filecontents*}

\begin{filecontents*}{SH-mechanical.dat}
Label     value       num
{Fan coil units}       43             8
{Air Handling Units}       13             7 
{CW Pumps}       13             6
{ECU}       11             5
{Pressurization Fans}        15             4
{Smoke Extract Fan}       5             3
{Jet fan}       5             2
{Overall}       12              1
\end{filecontents*}

\begin{filecontents*}{my6.dat}
Label    value         num   other
{Level 7}  50           11   13
L6         90           10   12
L5       80             9    16
L4       90             8    18
L3       70             7    90
L2       80             6    21
L1       70             5    22
\end{filecontents*}

\begin{filecontents*}{carparkventilation.dat}
Label    value         num   other
L5         50           11   13
L4         90           10   12
L3         80           9    16
GR         90           8    18
B1         70           7    90
B2         80           6    21
B3         70           5    22
\end{filecontents*}
%% CO SYSTEM
%% DATA
\begin{filecontents*}{carparkco.dat}
Label    value         num   other
L5         78           7   13
L4         90           6   12
L3         80           5    16
GR         90           4    18
B1         70           3    90
B2         80           2    21
B3         70           1    22
B5         50          {}    {}
\end{filecontents*}

\begin{filecontents*}{carparkco2.dat}
value,   num,   other,
78,       7,   13,
90,       6,   12,
80,       5,    16,
90,       4,    18,
70,       3,    90,
80,       2,    21,
70,       1,    22,
\end{filecontents*}
\end{comment}






















%\input{./mep/claim}  
%\input{./mep/dewa} 
%\input{./mep/busbar}
%\input{./mep/disruption}
%\input{./mep/RFI-mechanical}
%\input{./mep/RFI-electrical}
%%\input{./mep/internal}
%\input{./mep/provisional-sums}
% \begin{epigraphpage}
 \epigraph{Begin at the beginning,'' the King said, gravely, ``Then
 go till you come to the end; then stop.''}{Lewis Carroll, {\it Alice
 in Wonderland}}

 \epigraph{You can never get a cup of tea large enough or a book long enough to
 suit me''}{C. S. Lewis}
 \end{epigraphpage}

\parindent1em
%\cxset{style13}
%\cxset{title margin bottom=10pt,
%          title beforeskip=1pt}

\chapter{Introduction}
\addtocimage{-12pt}{-20pt}{../images/tocblock-fish}


\epigraph{``Begin at the beginning,'' the king said
"and then go on till you come to the end, then stop."}{
---Lewis Carroll, Alice in Wonderland}

 \parskip3pt plus 5pt 
\noindent This package and its documentation attempts to eliminate some common 
problems encountered when using \LaTeX2e. The first one is the loading of 
recommended packages for a large and perhaps complicated document and 
the second is the re-designing of styles for a document.

 \LaTeX2e, does not provide a standard library, but comes equipped with
 a package mechanism that allows code extensions to be loaded as required.
 This has created a strong vibrant community, hundreds of packages and a 
 headache to both new and seasoned users. What packages are available, when
 to use them and in which order is a common theme for many questions on
 lists and |TX.SE|.

 It is quite common during the writing of a thesis or book
 for the author to keep on adding macros and packages
 at the preamble of the document. In most cases this can
 be satisfactory but in many others it leads to
 incompatibilities and errors. This package aims at
 minimizing one's preamble, by prefetching a number of
 commonly used packages. It also aims at loading them
 in the right order and providing patches for conflicts.
 
 I am hoping that using this package, will lead to less
 frustrations with the intricacies of \LaTeX2e\ packages.

The package code is complicated, but its usage is simple. You first load the package and then
you use one of the available templates:

 \begin{commands}[]{}
 \begin{verbatim}
 \usepackage{phd}
 \usetemplate{style13}
 \end{verbatim}
 \end{commands}

This is what you need to typeset a good looking book or thesis. The rest of this book is a footnote and you can skip them if you want. 

It will be better for the longer projects to just fork the
 package and adapt it to your needs. In this respect, I have
 uploaded the package to |github|.\footnote{\url{https://github.com/yannisl/phd}}

 My goal in selecting the packages and adding a number of 
 commands for the authors was to be able to typeset a 
 document for most common use cases, without the need of
 additional packages. The packages I selected are biased
 towards academic publications, although they can find use
 in almost any fields. The package provides a mechanism via
 PGF keys to provide a settings file. 
 
 Most of the documentation can be found in the implementation part.

Browse any books in a library or bookshop and the striking thing is that their design is very individualistic. They might have similarities but their main features vary. In many respects they resemble people's faces where minor differences have striking effects.

This package arose out of a question at stackexchange. How to redefine chapter heads. Having seen the popularity of the |pgf| package \cite{pkg-pgf} I realized that \latex users prefer this method of styling rather the traditional \latex method.

The user interface can be extended to basically all major packages. The principle is to keep to a minimum changes that can affect the LaTeX core commands. If there are any additions a key setting is provided to be able to revert back to normal LaTeX.

The workflow can be simplified. In addition I want to believe that the interface can provide a useful addition to the open source community and that other people will contribute style libraries, which will be simpler to write. It is also possible
to device an easy and uncomplicated web interface to handle
such a great number of variables.


Most people when they get started with \LaTeX\ will either use one of the standard classes such as the \docFile{book.cls} or one of the generic classes notably koma-script or memoir. Most students will be forced to use on of the many thesis classes available.

\section{The key value concept}

The key-value concept that originated with \LaTeX\ has been extended many times, the last and most serious implementation of it by Tantau in the PGF package. What essentially Tantau developed is a scripting language to script TeX code. The \tikzname and pgfplots packages are two major packaged that use keys effectively. Their popularity is growing and what this package does is to offer a user interface that has been modelled to be similar to that of \texttt{css} (cascade style sheets). 
\smallskip

\begin{scriptexample}{}{}
\textit{chapter number} font-size = Large,\\
\textit{chapter number}     color = theblue
\end{scriptexample}
\smallskip

The main idea behind the package, is that you are configuring a document style by means of \emph{settings} rather than writing macros. In the example above the \emph{number, chapter} can be thought of as class or id names in css style sheets and the |font-size, color| as property settings that apply to the particular element. 


\subsection{Settings}

Settings are activated either by using the command |\cxset|  or by loading a full style sheet. In most cases you will probably import a style sheet and then modify some of the properties using |cxset|.  For example this heading has a dot after the subsection number. This was accomplished by setting,

We can de-activate it for the next and subsequent subsection headings with the setting:

\lorem

\begin{scriptexample}{}{}
\begin{verbatim}
\cxset{subsection number after=\quad}
\end{verbatim}
\end{scriptexample}




\subsection{Cascading}

Most values once set for a higher section will be seen in a cascade by all subsectioning commands in a similar fashion similar to CSS. These include properties such as color, font families and alignment. Best though to specify all of them for maximum flexibility to your users.

\section{On typography}

This package hopefully will assist in improving the typography of books set with \latexe. Any typographical comments on the various styles are just my own ramblingss and not necessarily absolute truths. Like fashion and art typography has opinions rather than absolute truths. In many styles the design is slightly adapted to blend a bit better with this manual. Also I did not select fonts as per the samples but this is left on you the user to decide.



\section{Packages and Fonts}

This manual has been typeset with numerous fonts in order to enable the typsetting of almost all the scripts provided by the Unicode standard. In order to process it from the |.dtx| file, these fonts must be available in your system, otherwise \XeLaTeX\ will have a problem finding the fonts and it will take an awful long time to process. This is especially true for the scripts section, where virtually all the Unicode defined scripts are discussed. You will need a fast computer and a fast hard disk to process the document within a reasonable time. When using \pkg{fontspec} always define your fonts with the \cmd{\newfontfamily} this will speed up processing by an order of magnitude. Compiling from the command prompt will speed up compilation. Average speed 2-3 pages per second.

Many of \tex's parameters are stretched to the limit with a complicated document such as this manual. You will require a full distribution otherwise expect some errors. Important packages is \pkg{morefloats} and \pkg{morewrites}. The package will also expect that you have |e-tex| installed. Ubuntu users are normally one year behind in updates, so you might wish to update manually. It will take upwards of 5 minutes to compile fully on an old laptop and a couple of minutes on a state of the art computer.

The |dtx| should be processed best with its own make file provided for Windows only |phd-lua.bat|. The make file will process the documentation using \lualatex. You can also process the document with \xelatex but is prone to produce errors. Using \latexe the sections on scripts etc will not be printed and a much shorter version of the manual is provided. 

\section{Scripts and Languages}

The package and the documentation offer a full repertoire of font selection keys for different scripts and languages. It hasn't been possible, however hard I tried to compile this section of the documentation with \xelatex, as it kept giving errors of too many files open. This was also not possible even with the \pkg{morewrites} package loaded. With \lualatex the document compiled with no major problems other than the font rendering being of a lower quality to that of XeLaTeX on windows, other than disabling incompatible packages and a number of commands that were redefined. 

Some good news for multi-script typesetting is the |Noto| fonts from Google. These fonts named Noto from "No Tofu" meaning you do not see any little square blocks for undefined glyphs, are fast to load. Disantvantage you need to switch between font commands fairly often.

\section{This book}

When developing the templates, I started using \emph{lorem ipsum} text as samples. Half-way through this
became a jumble mass of uninteresting pages interspersed with code. Headings and the contents of the book
determine both the structure and the selection of fonts, so I went back and wrote narratives  to accompany
the headings. Many of the narratives are semi-autobiographical in nature; others are clustered around books I read and my own interests. Some I stumbled on them accidentally and are mostly there to demonstrate some code.

Besides the templates and the code there is another narrative which is based on notes I kept on \tex and its friends over the years and are offered as a more advanced introduction to coding \latexe and \tex. The whole manual was typeset in a |ltxdoc| class, slightly modified to turn into a book class.

The implementation code is also available and it was mostly for my own benefit. The whole manual with the exception of the |\cxset| introduction, is just a test document. The notes and the “dissection” of the standard \latexe and the standard classes are there to explain the background to the many coding decisions that I took while I was developing the package.

PhD students are notorious for going in all directions and exploring many adjacent fields before they sit down and write their theses. Some become life-time students. To all these new men and women of the Renaissance that slave away to inch knowledge one thesis at a time, I dedicate this book and the name of the package.

\subsection{The TeX hacking sections}

To start programming \tex you need to have a knowldge of \tex basic commands and approach. \latex2015 is a format build on top of \tex to provide a more structured approach. To program \latexe packages you need to understand \latexe concepts, code organization and conventions. To program in \latex3, you need to learn a whole new language and you still need to understand \tex, \latexe and the expl3 language and conventions. To program using LuaTeX, other than the Lua language you need to understand \tex very well.
None of these can be found in one place.  I have gathered a lot of material and put it together. This is not a language you can master easily or quickly, but can teach you a lot about typesetting, computer science and many other interesting topics.


 \section{Version control with Git and Github}
 
 If you are involved with code or a publication that will have frequent changes, you should consider
 some type of version control system. My own recommendation is to use |git| and an online repository such
 as |github|. The latter is currently very fashionable and makes sharing code easier. Note that the |github|
 offers both public as well as private repositories. The general recommendation is that for unpublished work
 such as a thesis or code under development, it is preferable to go for a private repository. 
 
 \lorem\lorem

 \section{Ordering of Packages}
 
One package that normally leads to errors is the 
\pkg{hyperref}. The package which is an outstanding example of software engineering and supported single handledy by Heiko Oberdiek\footcite{hyperref} redefines a a lot of internal commands of the kernel. As a lot of other packages do the same it has to be loaded at the end of the preable with the exception of some packages! 
 
 This manual is typeset according to the conventions of the
 \LaTeX \textsc{docstrip} utility which enables the automatic
 extraction of the \LaTeX{} macro source files~\cite{GOOSSENS94}.

 
 \href{http://tex.stackexchange.com/questions/96350/problem-with-algorithmic-and-hyperref}{problem with algorithmic and hyperref}

 \begin{verbatim}
\usepackage{float}  % load float package first!

\usepackage{hyperref} % let hyperref patch the float package stuff
.
 \usepackage{algorithm} % let algorithm use the patched version of the float package
 \end{verbatim}
 

\section{Known problems}

Perhaps the biggest issue with the package is the speed of
compilation with \XeLaTeX\ or \LuaTeX. This is to be expected, as both engines spend a lot of resources in font management. On demand loading of packages is something I have in the back of my mind. This should be done via document styles i.e., if a book is for the humanities, perhaps only a rudimentary amount of maths packages should be loaded.

\section{Future Directions}

\latexe and \tex usage appears to be increasing. This is mostly by programs that export results with \latexe code rather than authors writing books.  The method adopted here is easier to automate all sorts of reports and automated texts. I would like too develop a web interface for processing such templates and at the same time export into html instead of just producing pdfs. I have already a prototype.   

\section{Tooling}

Some of the scripts on a Windows machine need MSYS\footnote{\url{http://mingw.org/wiki/MSYS}}









%
%\input{./styles/style01}
%\input{./styles/style02}
%\input{./styles/style03}
%\input{./styles/style04}
%\input{./styles/style05}
%\cxset{chapter format=block}
%\makeatletter
\cxset{defaults/.style ={% 
    chapter title margin-top-width    =  0cm,
    chapter title margin-right-width  =  1cm,
    chapter title margin-bottom-width = 10pt,
    chapter title margin-left-width   = 0pt,
    chapter align                     = left,
    chapter title align               = left, %checked
    chapter name                      = CHAPTER,
    chapter format                    = block,
    chapter font-size                 = Huge,
    chapter font-weight               = bold,
    chapter font-family               = sffamily,
    chapter font-shape                = upshape,
    chapter background-color          = white,
  % chapter label    
    chapter color               = black,
    chapter number prefix             = ,
    chapter number suffix             = ,
    chapter numbering                 = arabic,
    chapter indent                    = 0pt,
    chapter beforeskip                = -3cm,
    chapter afterskip                 = 30pt,
    chapter afterindent               = off,
    chapter number after              = ,
    chapter arc                       = 0mm,
    chapter label background-color    = white,
    chapter label color               = black,
   % chapter afterindent               = on,
    chapter grow left                 = 0mm,
    chapter grow right                = 0mm,
    chapter rounded corners           = northeast,
    chapter shadow                    = fuzzy halo,
    chapter border-left-width         = 0pt,
    chapter border-right-width        = 0pt,
    chapter border-top-width          = 0pt,
    chapter border-bottom-width       = 0pt,
    chapter padding-left-width        = 0pt,
    chapter padding-right-width       = 10pt,
    chapter padding-top-width         = 10pt,
    chapter padding-bottom-width      = 10pt,
    %  
    chapter number color              = black,
    chapter number background-color   = white,
    chapter number font-size        = huge,
    chapter number font-weight      = bfseries,
    chapter number font-family      = sffamily,
    chapter number font-shape       = upshape,
    chapter number align            = Centering,
    %
    chapter title font-size        = Huge,
     chapter title font-weight      = bold,
     chapter title font-family      = sffamily,
     chapter title font-shape       = upshape,
     chapter title color            = black,
     chapter title background-color = white,
     }%
   }  
\makeatother     
%\makeatletter
%\cxset{toc image=\@empty,
%       chapter toc=true,
%       title beforeskip=1pt}
%
%\@specialfalse
%
%
%\renewcommand\stewart[2][]{%
%\fancypagestyle{fancy}{%
%\lhead{}\rhead{}
%\chead{}
%\cfoot{}
%\lfoot{}
%\rfoot{\thepage}
%\def\footrule#1{{\color{blue}%
%  \hrule width\paperwidth}\vskip3pt
%}
%
%\renewcommand{\headrulewidth}{0pt}
%\renewcommand{\footrulewidth}{0.4pt}}
%
%\clearpage
%
%\begin{tikzpicture}[remember picture,overlay]
%% Main shading block
%\node [xshift=5cm,yshift=-\paperheight] at (current page.north west)
%[text width=0.98\textwidth,text height=\paperheight, fill=thecream!30,rounded corners,above right]
%{};
%\node [xshift=6.5cm,yshift=-1.5cm-\soffsety] at (current page.north west)
%[text width=0.9\textwidth,below right]{\sffamily \bfseries \huge #2};
%
%\node [xshift=3cm,yshift=-1.5cm] at (current page.north west)
%[text width=3cm,align=center,minimum height=2.5cm, fill=blue,below right]
%{\[\text{\HHUGE\bfseries\sffamily\color{white}\thechapter}\]
%\par\vspace*{3pt}
%};
%
%\node [xshift=-0.2cm,yshift=-21.5cm] at (current page.north west)
%[text width=3cm,above right]%
%{\includegraphics[width=1.0\paperwidth]{\image@cx}};
%% second box left
%\node [xshift=3cm,yshift=-19.5cm] at (current page.north west)
%[text width=9cm,minimum height=2.5cm,inner sep=0.5em, fill=blue,below right]
%{\color{white}
%  \bfseries\sffamily \texti@cx
%};
%% Last block
%\node [xshift=6.5cm,yshift=-26cm] at (current page.north west)
%[text width=12cm,above right]
%{\textii@cx
%};
%\end{tikzpicture}
%\par
%\clearpage
%}





\cxset{steward,
  chapter numbering=arabic,
  chapter format = stewart,
  offsety=0cm,
  image= {./images/hine02.jpg},
  texti={When Lamport designed the original \LaTeX\ sectioning commands he did not provide a fully comprehensive interface for modifying their design. With current tools available improvements are much easier to program and this chapter provides the details.},
  textii={\precis{In this chapter we discuss a method that allows the production of fancy chapter headings and formatting, based on a set of key values. Central  to this process is the separation of content from presentation.
We also discuss the basic formatting tools that are available and how one can modify them to mould new book designs.}
 }
}


\chapter{Designing Chapter Headings}
\addtocimage{-12pt}{-20pt}{./images/tocblock-man-01.jpg}

\section*{Introduction}

A \textls*{crowded} first page is as unsightly as a crowded title page, wrote De Vinne in \emph{Modern Methods of Book Composition} in 1904.  Not much has changed since. A new chapter must make a good impression and must give an immediate signal that a different topic is going to be discussed. Traditionally chapter openings in LaTeX are an unimpressive and dry event. Our aim is to brighten it up a bit, while keeping true separation of content from presentation, but avoiding the pit traps of over ornamenting the design. A book is to be read and we should provide minimal ornamentation. \index[phdkeys]{chapter> ornamentation}

% \usepackage{array,tabularx}
%\newcolumntype{Y}{>{\raggedleft\arraybackslash}X}% see tabularx
%\tcbset{enhanced,fonttitle=\bfseries\large,fontupper=\normalsize\sffamily,
%colback=yellow!10!white,colframe=red!50!black,colbacktitle=thecodebackground,
%coltitle=black,center title,
%tabularx={X||Y|Y|Y|Y||Y},% this sets ’before upper’ and ’after upper’
%before upper app={Group & One & Two & Three & Four & Sum\\\hline\hline} }
%
%\begin{tcolorbox}[title=My table]
%Red & 1000.00 & 2000.00 & 3000.00 & 4000.00 & 10000.00\\\hline
%Green & 2000.00 & 3000.00 & 4000.00 & 5000.00 & 14000.00\\\hline
%Blue & 3000.00 & 4000.00 & 5000.00 & 6000.00 & 18000.00\\\hline\hline
%Sum & 6000.00 & 9000.00 & 12000.00 & 15000.00 & 42000.00
%\end{tcolorbox}

\begin{figure}[htbp]
\centering
\parindent=0pt
\fbox{\includegraphics[width=\textwidth]{metropolitan-spread}}
\par
\caption{A chapter opening from the Metropolitan Museum of Art publicaion, \textit{Assyrian Reliefs and Ivories} by Vaughn. E. Crawford et. al., 1980. The spread is simple and the chapters are not numbered. This is a common characteristic of many more recently published books.}
\end{figure}


What is to us now a common occurence with instant book-printing was not always so. The cost of illustrated books was a prime factor and as Tschichold wrote:
\begin{quotation}
In the area of book design, in the last few years a revolution has taken place, until recently recognized by only a few. but which now begins to influence a much wider range of action.
It means placing much greater emphasis on the appearance of the book and a wholly contemporary use of typographic and photographic means. Before the invention of printing, literature of that time was spread around by the mouth of the author himself or by professional bards. The books of the Middle Ages - like the "Mannessische Liederhandschrift" - had
\end{quotation}

The type of book you are writing and its contents will determine an appropriate design for chapter headings and the type of design and numbering if any for subsections. Here we are merely providing a mechanism to produce them. These methods can produce a mastepiece or an ugly piece of work. Some simple suggestions follow (from my observations of styles in books I like). In general you need to think what type of book you are developing. For example a novel, should be sectioned very carefully. Many books avoid marking of sections other than chapters totally, perhaps marking them just with a soft ornament such as three centered asterisks.

\section{Numbering of Sections}


In general books do not number sections beyond subsection. You can avoid them all together, if you are not going to reference the sections extensively. 

In works of fiction, authors sometimes number their chapters eccentrically, often as a metafictional statement. For example:
Seiobo There Below by László Krasznahorkai has chapters numbered according to the Fibonacci sequence.

The Curious Incident of the Dog in the Night-Time by Mark Haddon only has chapters which are prime numbers.

At Swim-Two-Birds by Flann O'Brien has the first page titled Chapter 1, but has no further chapter divisions.

God, A Users' Guide by Seán Moncrieff is chaptered backwards (i.e., the first chapter is chapter 20 and the last is chapter 1). The novel The Running Man by Stephen King also uses a similar chapter numbering scheme.
Every novel in the series A Series of Unfortunate Events by Lemony Snicket has thirteen chapters, except the final instalment (The End), which has a fourteenth chapter formatted as its own novel.

Mammoth by John Varley has the chapters ordered chronologically from the point of view of a non-time-traveler, but, as most of the characters travel through time, this leads to the chapters defying the conventional order.


\begin{pgfpicture}
\pgfpathmoveto{\pgfpointorigin}
\pgfpathlineto{\pgfpoint{1cm}{1cm}}
\pgfpathlineto{\pgfpoint{1cm}{0cm}}
\pgfusepath{fill}
\end{pgfpicture}




\begin{figure}[tbp]
\centering
\parindent=0pt
\fbox{\includegraphics[width=\textwidth]{fantasy-architecture}}
\par
\caption{A chapter opening from the Metropolitan Museum of Art publicaion, \textit{Assyrian Reliefs and Ivories} by Vaughn. E. Crawford et. al., 1980. The spread is simple and the chapters are not numbered. This is a common characteristic of many more recent books.}
\end{figure}


\begin{figure}[tbp]
\centering
\parindent=0pt
\fbox{\includegraphics[width=\textwidth]{fantasy-architecture-02}}
\par
\caption{A chapter opening from the Metropolitan Museum of Art publicaion, \textit{Assyrian Reliefs and Ivories} by Vaughn. E. Crawford et. al., 1980. The spread is simple and the chapters are not numbered. This is a common characteristic of many more recent books.}
\end{figure}


\section*{Use of Color}

The modern books that Tschilchod was discussing have long been overwhelmed by the appearance of larger, coffee book type of books. Our brains our now conditioned by branding and graphic design is everywhere. 

Once you have decided that the book is going to be a bit more colorfull, the choice of color will follow. The decision what to color will be an important one, which brings us to color theory. The history of color is perhaps as colorfull as the rest. Attempts to formalize and recognize order date back to Aristotle (384-322 bce) but began in earnest with Leonardo da Vinci (1452-1519) and have progressed ever since. Leonardo noted that certain colors intensify each other, discovering \textit{contrary} and \textit{complementary} colors. The first color wheel was invented by Britain's Sir Isaac Newton (1642-1727), who split white light into red, orange, yellow, green, blue, indigo and violet beams, then joined the two ends of the spectrum to form a circle showing the natural progression of colors. When Newton created the color wheel, he noticed that mixing two colors from opposite positions produced a neutral or \textit{anonymous} color.


\begin{figure}[htbp]
\parindent=0pt
\centering
\fbox{\includegraphics[width=\textwidth]{line-designs} }
\caption{Spread from \textit{Beautiful Geometry}, Eli Maor and Eugen Jost, Princeton Univeristy Press, 2014. A subtle coloring of the chapter heading, de-emphasizing the chapter number and coloring the chapter title. There is no chapter label. A dropcap with the same color starts the first paragraph. This style is easy to achive with the phd system.}
\end{figure}


\begin{figure}[htbp]
\parindent=0pt
\centering
\fbox{\includegraphics[width=\textwidth]{color-book01.jpg} }
\bigskip

\fbox{\includegraphics[width=\textwidth]{color-book02.jpg} }
\end{figure}

One would expect a book written for the sole purpose of describing color theory and its application to the Graphic Arts, is expected to be colorful. Note the de-emphasizing of the label and number. 

\begin{figure}[htbp]
\parindent=0pt
\centering
\fbox{\includegraphics[width=\textwidth]{color-book-03.jpg} }
The chapter heading label and number are almost invisible. The heading text, is typeset in large bold letters, shouting what is coming next. Not your typical scintific book\ldots
\bigskip

\fbox{\includegraphics[width=\textwidth]{color-book-04.jpg} }
\end{figure}

Advertizing people understand that they need to present the message of an advertizement loud and clear so as to catch the busy eye. A heading's message is the title description. Neither the label not the chapter if any are necessary to convey the message. The chapter heading is analogous to the stop at the end of a sentence. The brain gets a signal to absorb what was written before it and get ready for the next. The heading signals the end of a topic. One must not dwell on it.


\section{Contemporary Chapter Headings}

In the book \textit{China} the designer used both a chapter heading on a spread of two images, as well as repeated the chapter number on the text pages \ref{fig:threepage}. The images distill the message of the chapter, although the chapter subtitle is almost unreadable, dominated by the surrounding text. From a technical perspective, the chapter command must paint the two images, set the right type of heading for each page and then without increasing the counter, change the counter to one that displays the chapter number in words and then continue with typesetting the text. A careful choice of images is necessary for such chapters, as well as cropping the images to match the aspect ratio of the book pages. One also needs to be carefull for \latexe not to place any floats in between the page spreads. 

\begin{figure}[htbp]
\parindent=0pt
\centering
\fbox{\includegraphics[width=\textwidth]{beijing.jpg} }\par
\vfill

\fbox{\includegraphics[width=\textwidth]{beijing-01.jpg} }\par
%\fbox{\includegraphics[width=\textwidth]{pearl-river.jpg} }
\caption{A full page chapter spread.}
\label{fig:threepage}
\end{figure}

\begin{figure}[htbp]
\parindent=0pt
\centering
\fbox{\includegraphics[width=\textwidth]{beijing.jpg} }\par
\vfill

\fbox{\includegraphics[width=\textwidth]{beijing-01.jpg} }\par
%\fbox{\includegraphics[width=\textwidth]{pearl-river.jpg} }
\caption{A full page chapter spread.}
\label{fig:threepage}
\end{figure}


\clearpage



In Figure~\ref{fig:photospread} the bands are black, but position low on the page. The size of the pages are 9.69 \texttimes 11.42. The books sections are not numbered. Text i sbroken through inserts of bigger text. Many of the examples here are from
commercial nude photography books, as they tend to break with tradition. In the 1970s and 1980s, fashion photographers began to present a
new, confrontational image of the female body. The pioneer in this
respect was the German Helmut Newton (1920–2004). Newton’s
photographs of nudes were overtly sexual, with an undertone of
menace, and although his models tended to be depicted as part
of the social elite they were often placed, apparently caught out
in reportage style, in sordid environments engaged in fantasy and
fetish. His work made him highly influential in fashion photography,
though some of it was thought too highly sexual for American
magazines and appeared only in those published in Europe.


\begin{figure}[htbp]
\parindent=0pt
\includegraphics[width=\textwidth]{baetens-01.jpg} \par
\vfill\vfill\vfill\vfill
\includegraphics[width=\textwidth]{baetens-02.jpg}\par
\caption{Chapter spread and first pages after the chapter title which is on the right page of the chapter spread. From \textit{New Photography, Art and the Craft}, Pascal Baetens, DK Publications. }
\label{fig:photospread}
\end{figure}

In the 1980s, Newton undressed the dynamic and independent
female in a series called Big Nudes. In this series the women are
indeed naked and very tall, wearing nothing but makeup and high
heels. The Big Nudes were exhibited in the form of life-size prints
that were intended to provoke the viewer by showing self-confident
women who knew what they wanted and were very aware of their
beauty and sexuality



\chapter{Package Usage}

To use the package include it just like any other package:

\begin{teXXX}
\documentclass{book}
\usepackage{phd}
\cxset{style13}
\begin{document}
\chapter{Introduction}
\end{document}
\end{teXXX}

The command \docAuxCommand{cxset} sets the default style for the example to the style defined as \meta{style13}. The package currently offers  100 templates and numerous keys to manipulate them further. Styles are similar to \enquote{themes} used in web programming; they are a collection of keys that resemble in many ways \texttt{css}. Styles can have any names and I am sure as package usage increases and evolve,they will get better names. 

\section{Background}

Before describing in detail how to specify a new layout for headings, we offer an overview of how the task can be accomplished and the design philosophy behind the approach. 

Irrespective of the technique and tools used, the creation of new layouts can always be divided into the following three tasks: constructing a document from “layout bricks”, which we can term as “blocks” or “elements”; establishing the layout semantics of each block; and finally, creating a layout engine supporting any document constructed from such blocks.

\begin{description}
\item [Canned Layouts] At one end of the spectrum, the most accessible approach consists of picking, a canned layout, such as LaTeX itself and perhaps only provide rudimentary macros to manipulate it.
\item [Constraints] Constraints offer a middle ground between canned layouts and handwritten layout engines. Constraints are arguably the most widespread and successful layout programming technique. For, instance, the foundations of \tex are laid upon constraint. CSS, the ubiquitous web template language, also relies on constraints, although in a more restricted and indirect manner.
\end{description}

\subsection{Blocks and Elements}

We define an \emph{element} as a document block, that cannot be subdivided further. For example the chapter title element, is composed of the text of the chapter title. 

A \emph{block} on the other hand is can contain other blocks and or numerous elements. We can consider the chapter headings as \emph{blocks}, composed of three blocks the chapter, number and title. Each block is then composed of elements. Each element has properties and traits. One of these mandary properties is the name. 

Blocks are either \emph{configured} (all constraints are mandatory), or flexible (there are optional/alternative constraints). By bundling optional constraints, flexible blocks make their specification customizable by non-technical users. 

\subsection{Language semantics}

One of the aims of the syntax of the templates was to offer familiar terminology and to remove the use
of \tex macros as far as possible from templates. 
\medskip

{\parindent0pt

 \textit{section}| font-family=serif,|\\
 \textit{section}| font-size=LARGE,|\\
 \textit{section}| font-weight=bold,|\\
}

The restriction I imposed is problematic when dealing with fractions of linewidths and textwidths. So
at present we allow for example |title text-width=0.5\texwidth| or |title text-width=10cm| or any other valid units. Ideas for improvements can only come from user feedback in the future.

Some experimental ideas incorporated are:

\begin{verbatim}
title text-width = 0.5 text-width,
title text-width = 1.2 text-width,
\end{verbatim}

A better parser will need to be programmed for dimensions, which are all currently handled as etex |dimexpr|. 

The syntax must allows both for microtypography as well as macro-typographical features. The former would deal with mostly fonts, spacing and text justification, where the latter deals with layouts, borders shapes and the positioning of elements on the page and also reletively to other elements or blocks.

An advantage of this approach is that it also opens the possibility of parsing the text with a language other than \tex and translating the document to another format, such as |HTML| or |XML| either fully or partially. Next we will describe both the syntax as well as the usage of the settings.

\section{Chapter opening page}

The standard \latexe classes offer only two options to either open a chapter on an odd page or at any page. This package offers five alternatives:

\begin{docKey}[phd]{chapter opening}{=\meta{any, left, right, anywhere, ifafter}}{default none, initial=any}
For documents that are primarily to be read on the web, use |any| for normal books, use \textit{right}. Some templates that we provide use |any| and the examples use |anywhere| to enable us to display the heading at any position on the page.
\end{docKey}

\begin{decription}
\item [any] Opens a chapter at any page, either \textit{verso} or \textit{recto}.
\item [left] Opens a chapter on an even page
\item [right] Opens a chapter on a right page.
\item [anywhere] Opens a chapter at the point where the \cs{chapter} is typed.
\item [none] Alias for \marg{anywhere}.
\item [ifafter] Opens a chapter at the next page if the page has material that does not exceed a certain portion of \cs{textheight}.
\end{description}

\colorlet{theoption}{bgsexy}

To change a setting you just modify the value of the key \oarg{\option{chapter opening}} to one of the values described earlier. 

\begin{dispListing}
\cxset{chapter opening = anywhere}
\end{dispListing}
 
We use this key to print the many examples typesetting chapter heads that follow (see the example~\ref{ex:anywhere}).  


\begin{texexample}{title=Inline Chapter Example}{ex:anywhere}
\cxset{examplestyle/.style = {chapter format = block,
       chapter opening = anywhere,
       chapter name = CHAPTER, 
       %label
       chapter label font-family      = sffamily,
       chapter label color            = primary,
       chapter label background-color = white,
       % number
       chapter number font-family = sffamily,
       chapter number font-size = HUGE,
       chapter number color     = primary,
       chapter label align = centering,
       chapter number background-color = white,
       %title
       chapter title font-family = rmfamily,
       chapter title align = centering,
       chapter title background-color = bgsexy!15,
       chapter title before background-color=white}}
\cxset{examplestyle}       
\lorem
\chapter{Typography Example}
\lorem
\chapter{Another Chapter Heading}
\lorem
\end{texexample}


%\cxset{toc chapter = true}
\addtocounter{chapter}{-1}

Examples for other types of chapter openings follow in the rest of the documentation.

\subsection{Blank pages before chapters}

In the standard LaTeX book class when the \texttt{openany} option is not given or in the report class when the openright is given, chapters start at odd-numbered pages. This can cause a blank page to be printed. Some book designers prefer this page to be completely empty, without any headers or footers. This cannot be done with \lstinline{\thispagestyle} as this command will have to be issued on the \textit{previous} page. However by a suitable redefinition of the
\lstinline{\clearpage} this can be done automatically.
\medskip

\begin{teXXX}
\makeatletter
\def\cleardoublepage{\clearpage\if@twoside\ifodd\c@page\else
  \hbox{}
  \vspace*{\fill}
  \begin{center}
    This page left intentionally blank.
  \end{center}
  \vspace{\fill}
  \thispagestyle{empty}
  \newpage
  \if@twocolumn\hbox{}\newpage\fi\fi\fi}
\makeatother
\end{teXXX}


This is achieved easily by setting the following options:
\bigskip

\begin{tcolorbox}
\lstinline{chapter blank page=empty}\par
\lstinline{chapter blank page text=Some text.}\par
\lstinline{chapter blank page=plain}\par
\end{tcolorbox}
\medskip



The last one refers to a \lstinline!\thispagestyle{plain}!.
\cxset{chapter opening = right, chapter format = block}
\chapter{Test}

\cxset{defaults, chapter opening= anywhere}



\section*{Keys for chapter head formatting}

A chapter heading can be considered of being constructed of several parts, the \textit{chapter number}, the chapter name typically \textit{chapter} and the \textit{title}. Predefined keys handle all the elements of formatting. Additional keys are defined to handle other elements such as inclusion of images or producing complicated examples with graphics constructed with \texttt{TikZ} and other similar packages.


\bigskip\bigskip\bigskip\bigskip
\let\oldrefkey\refKey
\let\refKey\texttt
\makeatletter
\long\def\demobox#1#2{%
\par\bigskip\bigskip\bigskip
\begin{tcolorbox}[enhanced,left=0pt, top=0pt, bottom=0pt,width=\textwidth,
  enlarge top initially by=1cm,enlarge bottom finally by=1cm,left skip=1cm,right skip=1cm,
  colframe=white,colback=white,
  colbacktitle=red!30!white,colupper=black!7!white,
  code={\appto\kvtcb@shadow{%
    \path[fill=white,draw=yellow!50!black,dashed,line width=0.4pt]
      ([xshift=-1cm,yshift=-1cm]frame.south west) rectangle
      ([xshift=1cm,yshift=1cm]frame.north east);
     \path[fill=blue!20!white, 
              opacity=0.3, draw=yellow!50!black,solid,line width=1pt]
      ([xshift=-2cm,yshift=-2cm]frame.south west) rectangle
      ([xshift=2cm,yshift=2cm]frame.north east);  
    }},
  finish={
  \draw[thick,<->] ([yshift=-1.3cm]frame.north west)-- node[below]{\texttt{#1 width}}
    ([yshift=-1.3cm]frame.north east);
  \draw[thick,<->] ([xshift=-15mm]frame.north east)-- node[above]{\refKey{#1 height}}
    ([xshift=-15mm]frame.south east);
  \draw[thick,<->] (frame.north)-- node[right]{\refKey{#1 padding-top}} +(0,1);
  \draw[thick,<->] ([yshift=1cm]frame.north)-- node[right]{\refKey{#1 margin-top}} +(0,1);
  \draw[thick,<->] (frame.south)-- node[right, align=left]{\refKey{#1 padding-bottom}}+(0,-1);
  %left padding
  \draw[thick,<->] (frame.west)-- node[below right,align=center]{\refKey{#1 padding-left }}+(-1,0);
  %left margin
  \draw[thick,<->] ([xshift=-1cm,yshift=-0.9cm]frame.west)-- node[below right,xshift=-1,align=left]{\refKey{#1 margin-left }\\\refKey{#1 grow to left by}}+(-1,0);
  %right padding
  \draw[thick,<->] (frame.east)-- node[below left,align=center]{\refKey{#1 padding-right}}+(1,0);
 %right margin
  \draw[thick,<->] ([xshift=1cm,yshift=-0.9cm]frame.east)-- node[below left,xshift=1, align=right]{\refKey{#1 margin-right}\\\refKey{#1 grow to right by}}+(1,0);
 \draw[thick,<->] ([yshift=-2cm]frame.south)-- node[right, align=left]{\refKey{#1 margin-bottom},\\ \refKey{#1 after skip}}+(0,1);
  }
    ]
#2%
%\hrule width0pt height4.5cm depth0pt\relax% \vspace*{4.5cm}% \lipsum[1]
\end{tcolorbox}\par
\bigskip\bigskip\bigskip}
\makeatother

\demobox{chapter}{\scalebox{1.17}{\HHHUGE Chapter}}

The number box is again drawn in a box similar to a chapter with all properties generalized.

\demobox{number}{\scalebox{1.15}{\HHHUGE Thirteen}}



All parameters shown in the diagram can be set using the command \cs{cxset}. The property names follow conventions similar to those of |css|, rather than typical conventions of \tikzname that are more widely known to the programming community. The prefix to these properties (in the example \textit{chapter}) can be thought of
as similar to a |class| or |id| name in |css|.  

\begin{docCommand}{cxset}{\marg{options}}
  Sets options for every following \refEnv{tcolorbox} inside the current \TeX\ group.
  By default, this does not apply to nested boxes, see \Vref{subsec:everybox}.\par
  For example, the colors of the boxes may be defined for the whole document by this:
\begin{dispListing}
\cxset{chapter numbering = Roman,
       chapter number color = blue}
\end{dispListing}
\end{docCommand}

\begin{docKey}[]{chapter padding-top}{=\meta{dimension}}{no default, initial value 0pt}
All padding keys take one argument, which is a dimension. The length is also stored in a register
\cmd{\chapterpaddingtop}. In this chapter it was set at %\the\chapterpaddingtop.
\begin{dispListing}
\cxset{colback=red!5!white,colframe=red!75!black, chapter padding-top=2pt}
\end{dispListing}
\end{docKey}



\begin{docKey}[]{chapter padding-right}{=\meta{dimension}}{no default, initial value 0pt}
All padding keys take one argument, which is a dimension. The length is also stored in a register
\cmd{\chapterpaddingright}.  In this chapter it was set at %\the\chapterpaddingright.
\end{docKey}

\begin{docKey}[]{chapter padding-bottom}{=\meta{dimension}}{no default, initial value 0pt}
All padding keys take one argument, which is a dimension. The length is also stored in a register
\cmd{\chapterpaddingbottom}.  In this chapter it was set at %\the\chapterpaddingbottom.
\end{docKey}

\begin{docKey}[]{chapter padding-left}{=\meta{dimension}}{no default, initial value 0pt}
All padding keys take one argument, which is a dimension. The length is also stored in a register
\cmd{\chapterpaddingleft}.  In this chapter it was set at %\the\chapterpaddingleft.
\end{docKey}

%% margin

\begin{docKey}[]{chapter margin-top}{=\meta{dimension}}{no default, initial value 0pt}
All padding keys take one argument, which is a dimension. The length is also stored in a register
\cmd{\chaptermargintop}. In this chapter it was set at .
\end{docKey}

\begin{docKey}[]{chapter margin-right}{=\meta{dimension}}{no default, initial value 0pt}
All padding keys take one argument, which is a dimension. The length is also stored in a register
\cmd{\chapterpaddingright}.  In this chapter it was set at %\the\chapterpaddingright.
\end{docKey}

\begin{docKey}[]{chapter margin-bottom}{=\meta{dimension}}{no default, initial value 0pt}
All padding keys take one argument, which is a dimension. The length is also stored in a register
\cmd{\chapterpaddingbottom}.  In this chapter it was set at %\the\chapterpaddingbottom.
\end{docKey}

\begin{docKey}[]{chapter margin-left}{=\meta{dimension}}{no default, initial value 0pt}
All padding keys take one argument, which is a dimension. The length is also stored in a register
\cmd{\chaptermarginleft}.  In this chapter it was set at %\the\chaptermarginleft.
\end{docKey}

\subsection{Borders}

Border have three properties \emph{width, color} and \emph{style}. They can set individually for
each side of the box or using the shorter key .

\begin{docKey}[]{chapter border-top-width}{ = \meta{dimension}}{no default, initial value 0pt}
All border keys take one argument, which is a dimension.
\end{docKey}

\begin{docKey}[]{chapter border-right-width}{=\meta{dimension}}{no default, initial value 0pt}
All border keys take one argument, which is a dimension.
\end{docKey}

\begin{docKey}[]{chapter border-bottom-width}{ = \meta{dimension}}{no default, initial value 0pt}
All border keys take one argument, which is a dimension.
\end{docKey}

\begin{docKey}[]{chapter border-left-width}{ = \meta{dimension}}{no default, initial value 0pt}
All border keys take one argument, which is a dimension.
\end{docKey}

\subsubsection{Border Colors}

The colors follow the same pattern for |border-width| and again they can be set individually or using
a shorter key to set all of them in one color. 

\begin{docKey}[]{chapter border-top-color}{=\meta{color name}}{no default, initial value black}
All border keys take one argument, which is a dimension.
\end{docKey}

\begin{docKey}[]{chapter border-right-color}{=\meta{color name}}{no default, initial value black}
All border keys take one argument, which is a dimension.
\end{docKey}

\begin{docKey}[]{chapter border-bottom-color}{=\meta{color name}}{no default, initial value black}
All border keys take one argument, which is a dimension.
\end{docKey}

\begin{docKey}[]{chapter border-left-color}{=\meta{color name}}{no default, initial value black}
This key is stored in \cmd{\chapterborderrightcolor} and the value in this chapter is 
%\ExplSyntaxOn \l_phd_chapter_border_right_color_tl.
\ExplSyntaxOff
\end{docKey}



\subsubsection{Border Styles}

Standard |css|  offers four styles \emph{dotted, solid, double, dashed}. We offer almost an unlimited set of styles.

\begin{docKey}[phd]{chapter border-top-style}{=\meta{style name}}{no default, initial value \texttt{none}}
The |border-style| properties take a value, which can be |solid, double, dotted, dashed, asterisk|.
\end{docKey}

\begin{docKey}[phd]{chapter border-right-style}{=\meta{style name}}{no default, initial value \texttt{none}}
The |border-style| properties take a value, which can be |solid, double, dotted, dashed, asterisk|.
\end{docKey}

\begin{docKey}[]{chapter border-bottom-style}{=\meta{style name}}{no default, initial value \texttt{none}}
The |border-style| properties take a value, which can be |solid, double, dotted, dashed, asterisk|.
\end{docKey}

\begin{docKey}[]{chapter border-left-style}{=\meta{style name}}{no default, initial value \texttt{none}}
The |border-style| properties take a value, which can be |solid, double, dotted, dashed, asterisk|.
\end{docKey}

\begin{docKey}[phd]{chapter border-style}{=\meta{style name}}{no default, initial value \texttt{none}}
This key sets all chapter-border-\meta{top,right,bottom,left}-style to a single value.
\end{docKey}

\subsubsection{Fonts and colors}

All font parameters can be set using individual keys. The naming scheme in general follows |css| conventions.

\begin{docKey}[phd]{chapter color}{=\meta{color name}}{no default, initial value \texttt{black}}
This key sets the color for the \textit{chapter element}. The color name is stored in \cmd{\chaptercolor@cx}.
The value in this chapter is% \makeatletter\texttt{\chaptercolor@cx}\makeatother.
\end{docKey}

\begin{docKey}[phd]{chapter font-size}{=\meta{Huge, Large}}{no default, initial value \texttt{Huge}}
This sets the size for rendering the \textit{chapter element}. Use one of the following predefined values.
Note that you can either use a command i.e, |chapter font-size=|\cmd{\huge} 
or the command name i.e., |chapter font-size=huge|. The latter is the recommended method.
\end{docKey}

\begin{marglist}
\item [tiny] renders as {\tiny tiny}.
\item[footnotesize] renders as {\footnotesize footnotesize}
\item [small] Opens a chapter on an even page
\item [large] Opens a chapter on a right page.
\item [LARGE] Opens a chapter at the point where the \cs{chapter} is typed.
\item [huge] Alias for \marg{anywhere}.
\item [Huge] Opens a chapter at the next page if the page has material that does not exceed a certain portion of
 \cs{textheight}.
 \item[HUGE] renders as {\HUGE HUGE}.
 \item[HHUGE] renders as {\HHUGE HUGE}.
\end{marglist}

\begin{texexample}{Sizing settings}{}
\cxset{
          chapter format = block,
          chapter label font-size= HUGE,
          chapter name = Chapter,
          chapter format=block,
          chapter number font-size= HUGE,
          chapter title font-size=LARGE,
         % 
         % chapter padding-top=0pt,
         % chapter padding-bottom=0pt,
         % title margin-top=3pt,
         %
          }
\chapter{Setting font-sizes}          
\lorem

\end{texexample}


\begin{docKey}{chapter font-family}{ = \meta{sffamily, rmfamily etc.}}{no default, initial value \texttt{sffamily}}
The |font-family| key accepts \latexe conventional family names or |css| names such as |serif| and |non-serif|. The
value is stored in \docAuxCommand{chapter_font_family}, in this chapter it is set as {\ExplSyntaxOn\meaning\chapter_font_family\ExplSyntaxOff}
\end{docKey}


\begin{marglist}
\item [sffamily] The \emph{chapter element} is rendered in the document default \cmd{\sffamily}.
\item [rmfamily] The \emph{chapter element} is rendered in the document default \cmd{\rmfamily}.
\end{marglist}

%% Font weights
\begin{docKey}[]{chapter font-weight}{=\meta{mdseries,bfseries,etc.}}{no default, initial value \texttt{bfseries}}
The |font-weight| key accepts \latexe conventional family names or |css| names such as |bold| and |bfseries|. The
value is stored in \cmd{\chapterfontweight@cx}, in this chapter it is set as 
{\ExplSyntaxOn\expandafter\string\l_phd_chapter_label_fontweight_tl\ExplSyntaxOff}

\begin{texexample}{Setting chapter element font-weights}{fontweight}
\cxset{chapter label font-weight=normal}
\chapter{Font-weight is normal}
\cxset{chapter label font-weight= bfseries}
\chapter{Font-weight is bfseries}
\lorem
\end{texexample}
\end{docKey}


\begin{marglist}
\item [normal] The \emph{chapter element} is rendered in the document default \cmd{\sffamily}.
\item [bold] The \emph{chapter element} is rendered in the document default \cmd{\rmfamily}.
\item[bfseries] Renders as bold.
\item[mdseries] renders as medium series.
\item[light] This is an alias for normal.
\item[\upshape\ttfamily\string\bfseries] The command version of the setting.
\item[\upshape\ttfamily\string\mdseries] The command version of the setting.
\end{marglist}



\begin{docKey}[]{chapter font-shape}{=\meta{itshape,upshape,etc.}}{no default, initial value \texttt{upshape}}
The |font-weight| key accepts \latexe conventional family names or |css| names such as |bold| and |bfseries|. The
value is stored in |chapter_font_weight|, in this chapter it is set as %\ExplSyntaxOn \texttt{\chapter_font_shape}\ExplSyntaxOff.
\end{docKey}

In |css| the |font-shape| is named as |font-style| so we alias it as well. 

%\begin{marglist}
%\item[normal] normal font-style, defaults to |upshape|.
%\item[upshape] normal font-style, defaults to |upshape|. 
%\item[italic] italic shape, renders as {\itshape italic}. For some fonts it might not be available.
%\item[itshape] italic shape, alias of |italic|.
%\item[oblique] oblique font, in \latexe is equivalent to \cmd{\slshape} and renders as {\slshape slshape}, which might be slightly different than {\itshape italic}.
%\end{marglist}


\begin{texexample}{Setting up Fonts}{chapterfonts}
\cxset{   chapter format = block,
          chapter opening=anywhere,
          chapter label font-weight=normal,
          chapter label font-shape=upshape,
          %chapter border-width=0pt,
          %chapter border-style=none,
          %chapter padding-top=0pt,
          chapter label font-size=large,
          chapter number font-size=large,
          chapter number color=black,
          %title font-size=large,
          }
\chapter[fonts]{Test Font Weights}
\lorem
\cxset{chapter label font-shape=itshape}
\chapter{Test Italic Shape}
\lorem
\cxset{chapter label font-shape=normal}
\chapter{Test normal font-shape}
\lorem
\end{texexample}



The specification of font families is somewhat problematic. In the web the |css| allows |font-family|  to hold several font names as a ``fallback” system. If the browser does not support the first font, it tries the next font.

There are two types of font family names:

\begin{description}
\item[family-name] The name of a font-family, like “times”, “courier”, “arial”, etc.
\item[generic-family] The name of a generic family, like “serif”, “sans-serif”, “cursive”, “fantasy”, “monospace”.
\end{description}

Generally in the \tex community leaving the choice of font  open to what is available on a user’s computer is frowned upon. Knuth’s original aim to render consistently documents, irrespective of a user’s computer installation has served the community well, and it is possible three decades later to produce documents identical in all respects to the original. 

If this is still a valid requirement for documents is debatable. Current document processing requirements are focusing more in the preservation of content and document structure rather than form. Typeset documents in soft copy are now widely preserved in |pdf| or |postcript|  formats. One can archive the |.tex| file as well as the |pdf| file.  Back to the provision of keys, a key defined in a 
similar fashion to those of |css| could help, but there is also the issue of slow compilation. If a font cannot be
found, with the current code, it can slow down compilation tremendously. I am leaving the choice where it belongs to the user and the package writer. It makes no harm if a more flexible definition is provided. The user can then decide to only provide one or many fonts. 

This avoids complicated and almost unintelligible commands such as:

\begin{dispListing}
\setkomafont{subsection}{\usefont{T1}{fvm}{m}{n}}
\setkomafont{section}{\usefont{T1}{fvs}{b}{n}\Large}
\end{dispListing}

Here are some examples. 

\begin{texexample}{Serif and non-serif}{ex:fontfamily}
\cxset{chapter label font-family=serif, 
       chapter opening=anywhere}
\chapter{Serif font}
\lorem
\end{texexample}


\section{Floating and Alignment} 

This particular key bothered me, as the term \emph{float} has a different meaning in \latexe. However, to
be consistent with |css| terminology I have yielded to the temptation and included it.

\begin{docKey}[]{chapter float}{=\meta{left,center,right,none}}{no default, initial value \texttt{none}}
Key that controls the horizontal alignment of the \emph{chapter element}. I order for the
element to float, its |display| property must be set to |inline|.
\end{docKey}

%\begin{texexample}{Floating}{chapter:float}
%\cxset{chapter opening=anywhere, chapter float=center}
%\chapter{Centered Chapter}
%\lorem
%\cxset{chapter float=left}
%\chapter{Left Aligned}
%\lorem
%\cxset{chapter float=right}
%\chapter{Right Aligned}
%\lorem
%\end{texexample}


\subsection{The display property}

Both the |css| box model as well as the \TeX layout engine provide numerous complex algorithms in managing the floating of elements. This is normally controlled using two properties |display| and |float|.


\makeatletter

\begin{docKey}[phd]{chapter position}{ = \meta{absolute, relative}}{no default, initial value black}
This positioning directive instructs the engine to position the element at an exact position.
\end{docKey}



\tcbox[nobeforeafter]{$box_1$}\tcbox[nobeforeafter]{$box_2$}\tcbox[nobeforeafter]{$box_3$}\dotfill\tcbox[nobeforeafter]{$box_n$}
\tcbox[before skip=0.2cm, after skip=0pt, width=1cm, enlarge left by=10cm,width=5cm,enhanced,show bounding box]{title before element}
\tcbox[before skip=0pt, width=1cm, enlarge left by=10cm,width=5cm,enhanced,show bounding box]{
\tcbox{tb}\tcbox{title}\tcbox[nobeforeafter, width=1cm,]{tb}}
\tcbox[before skip=0pt, after skip=12pt, width=1cm, enlarge left by=10cm,width=5cm,enhanced,show bounding box]{\emph{title after} element \fbox{some}}
\makeatother

\begin{docKey}[phd]{chapter float}{=\meta{left,center,right,none}}{no default, initial value \texttt{none}}
Key that controls the horizontal alignment of the \emph{chapter element}. I order for the
element to float, its |display| property must be set to |inline|.
\end{docKey}
In document preparation systems or web page development the layout is user generated, i.e., the user is expected to type the html and the |css| will then specify as to how the page will be rendered by the browser. In our case for documents we can specify how we want the headings to look. The layout manager for each element, creates other associated elements, as shown for the title here. This way most layouts can be accomplished with the declarative visual language of the \pkgname{phd} package. 

\subsubsection{In-line elements}

When an element is specified as |inline| the rendering algorithm places the boxes after each other. This is widely used in |chapter elements| to render the number inline with the chapter name.
\medskip
\bgroup

\noindent
\tcbox[nobeforeafter,width=3cm, height=1cm]{Chapter}\tcbox[nobeforeafter]{twelve}
 
When the property is set as |block| the elements are stacked below each other.
\medskip

\tcbox{chapter  display=block   CHAPTER}
\tcbox{number display=block    TWELVE}

The elements can be considered to be enclosed in a \emph{ghost} element. If the property is set to float we
\begin{figure}[htbp]
\makeatletter
\parindent0pt\fboxsep0pt
\fbox{\vbox to 0pt{\hbox to \dimexpr(\textwidth)\relax{{\hss\tcbox[capture=minipage,width=5cm, height=2cm, top=0pt]{\raggedright number display=block\\ number float=right }}%
}%
}%
}\par
\vspace*{2cm}
\makeatother
\end{figure}
signalling to the layout engine that the element must be placed to the right of the page, as shown in the figure. 


\begin{figure}[htbp]
\makeatletter
\parindent0pt\fboxsep0pt
\fbox{\vbox to 0pt{\hbox to \dimexpr(\textwidth+2cm)\relax{{\hss\tcbox[capture=minipage,width=5cm, height=2cm, top=0pt]{\raggedright number display=block\\ \emph{element} float=right }
\tcbox[capture=minipage,width=5cm, height=2cm, top=0pt]{\raggedright \emph{element} display=block\\ \emph{element} float=right }
}%
}%
}%
}\par
\vspace*{2cm}
\makeatother
\end{figure}

\subsection{Absolute positioning}

Absolute positioning mode, will place an element at an exact position on the page. They are more difficult to
achieve and inflexible. 

\begin{docKey}{position}{=\meta{absolute},\meta{relative}}{no default, initial none}{}

\end{docKey}



This positioning directive instructs the engine to position the element at an exact position.


\begin{docKey}[]{chapter float}{=\meta{left,center,right,none}}{no default, initial value \texttt{none}}
Key that controls the horizontal alignment of the \emph{chapter element}. In order for the
element to float, its |display| property must be set to |inline|.
\end{docKey}
\egroup



\section{Number Element Keys}


\subsection*{Keys for numbering}

Chapter numbering follows that of the standard \LaTeX\ classes and is extended to cover some additional cases such as fully spelled out numbers. This of course is only good for languages that use the arabic numeralsn. For other languages numerals in different formats can be added with simple keys and without the need of \pkgname{polyglossia} or \pkgname{babel}. 

Note that the package uses Heiko Oberdiek's package \pkgname{alphalph} to allow for alphabetic numbering that extends beyond the normal 26 letters of the alphabet. Examples for numbering can be seen in \ref{ex:romannumbering}


\begin{docKey}[phd]{number numbering}{= \oarg{alph,Alph,roman,Roman,none,WORDS,words,none}}{default arabic}
Style of numbering.
\end{docKey}

\begin{marglist}
\item [arabic] Despite that the Arabs call what the West calls Arabic numbers Indian numbers, we provide the value arabic to have normal numbers printed.
\item [alph] Lowercase alphabetic numbering.
\item [Alph] Uppercase alphabetic numbering.
\item [roman] Lowercase roman numbering.
\item [Roman] Uppercase roman numbering.
\item [words] The number is in lowercase words.
\item [WORDS] The number is in uppercase literal numerals.
\item [Words] Prints the number in words and capitalizes the first letter, for example the number 21 will be printed as `Twenty One'\footnote{Currently limited to the first hundred numbers}.
\index{chapter design>numbering>words}
\item [ordinals] Prints the number as ordinal.
\item [Ordinals] Prints the number as Ordinal.
\item [ORDINALS] Prinst the number as ORDINALS.
\item [none] This is equivalent to using the star version of the command. It does not print any number and does not increment the chapter counter.\footnote{I am ambivalent about this, perhaps it will be better to increment it, as it can give a more general approach.}

\end{marglist}
\begin{texexample}{Literal Numbering}{ex:literal}
\cxset{chapter numbering=WORDS} 
\chapter{Literal numbering}
\lorem
\cxset{chapter numbering=words,chapter name=chapter}
\chapter{Literal numbering} 
\lorem
\end{texexample}




\cxset{chapter opening=anywhere, chapter numbering=Roman, chapter number font-shape=upshape}
\index{chapter design>numbering>roman}

\begin{texexample}{Setting up keys for numbering}{ex:romannumberingx}
\bgroup
\cxset{chapter format = traditional, 
       chapter name = CHAPTER, 
       chapter numbering = Roman,
       chapter label color = bgsexy}
\chapter{Roman numbering}
\lorem
\egroup
\end{texexample}





To emulate some old books we also offer an ordinal numbering scheme.

\begin{texexample}{Literal Numbering}{ex:ordinals}
\cxset{chapter numbering=ORDINALS} 
\chapter{Ordinals numbering}
\lorem
\cxset{chapter numbering=words,chapter name=chapter}
\chapter{Literal numbering} 
\lorem
\end{texexample}

\cxset{chapter numbering=arabic}

\subsection{Fonts and colors}
\begin{docKey}[phd]{number color}{=\meta{color name}}{no default, initial value \texttt{black}}
This key sets the color for the \textit{number element}. The color name is stored in %\cmd{\numbercolor@cx}.
The value in this chapter is %\makeatletter\texttt{\numbercolor@cx}\makeatother.
\end{docKey}

\begin{docKey}[phd]{number font-size}{=\meta{Huge, Large}}{no default, initial value \texttt{Huge}}
This sets the size for rendering the \textit{number element}. Use one of the predefined values, as described
in the section for the \emph{chapter} element.
Note that you can either use a command i.e, |number font-size=|\cmd{\huge} 
or the command name i.e., |number font-size=huge|. The latter is the recommended method.
\end{docKey}

Letter spacing can be achieved using the soul package in a combination with the key |spaceout|.
The following examples illustrate the usage.

\index[phdkeys]{{\ttfamily phd/chapter design test}}

%\begin{texexample}{Letter Spacing}{ex:letterspacing}
%\cxset{numbering=Roman,
%        % number letter-spacing=soul,
%        % chapter spaceout=soul,
%         %title spaceout=soul,
%         title font-size=Large,
%         title font-family=rmfamily,
%         title font-shape=scshape}
%\chapter{Letter Spacing}
%
%\lorem
%\end{texexample}

\begin{docKey}[phd]{chapter number letter-spacing}{=\meta{none, true, etc.}}{no default, initial value \texttt{none}}.
\end{docKey}

\begin{marglist}
\item[none] Default value no tracking is used and the letters are spaced as per the basic font information.
\item[inherit] Inherits the letter-spacing settings from the \emph{chapter} element.
\item[true] Letter spacing is employed, using the |soul| package.
\item[false] Alias for |none|.
\item[soul] The \pkgname{soul} package is used for letter-spacing.
\item[microtype] The \pkgname{microtype} package is used for letter-spacing. When the microtype package is used more fine tuning of parameters is available.
\end{marglist}

The example that follows, explains how the features offered by the \pkgname{microtype} package can be used to
set different tracking options.

\begin{texexample}{Microtypography}{micro}
\bgroup

\SetTracking
 [ no ligatures = {f},
 spacing = {600*,-100*, },
 outer spacing = {450,250,150},
 outer kerning = {*,*} ]
 { encoding = * }
 { 100 }

{\huge \textls{Chapter Twenty}}

\SetTracking
 [ no ligatures = {f},
 spacing = {600*,-100*, },
 outer spacing = {450,250,150},
 outer kerning = {*,*} ]
 { encoding = * }
 { 200 }
 
{\huge \textls{Chapter Twenty}}

\egroup
\end{texexample}


\hbox{\drawfontbox{\huge \upshape\textls(Chapter Twenty}}

\hbox{\drawfontbox{\huge \upshape\textls{Chapter Twenty}}}


\section{Styling the chapter title}

Similarly to the number and chapter styling keys exist for styling the chapter title. We summarize the available standard keys below:

\index{chapter design!labels!letter spacing}
\begin{texexample}{Styling the Title}{ex:title} 
\cxset{chapter numbering=arabic, chapter title font-shape=itshape}
\chapter{Chapter title}
\lorem
\end{texexample}


\begin{docKey}[phd]{chapter title font-family}{=\marg{family}}{no default, initial inherit document font}
Selects a predefined font family
\end{docKey}

\begin{texexample}{Title element font styling}{}
\cxset{chapter title font-family=sffamily}
\chapter{Title font family settings}
\lorem
\cxset{chapter title font-shape=itshape}
\chapter{Title font-style settings}
\lorem
\end{texexample}


\begin{docKey}[phd]{chapter title font-weight}{ = \marg{\cs{bfseries},\cs{normalseries}}} {}
Font weight.
\end{docKey}

\begin{docKey}[phd]{chapter title font-size}{= \marg{large, Large, huge, Huge, HUGE, HHuge}}{}
Font sizing commands or their names. Both \docAuxCommand{\HUGE} and HUGE are allowed to be used as values for the key.
\end{docKey}

\begin{docKey}[phd]{chapter title color} { = \marg{color}} {}
The color of the chapter title letters. This takes any predefined color name. 
\end{docKey}


\begin{docKey}[phd]{chapter title spaceout}{ = \marg{soul,none}} {no default, initial = none}
 This key will space out the title. 
\end{docKey}

\begin{texexample}{Title element spacing}{}
\cxset{chapter name=none,
       chapter numbering=none,
       chapter title font-size=Large,
       chapter title color=black,
       chapter title width=0.6\textwidth,
       %title spaceout=soul,
         }
\chapter{The Prehistoric Period in South-East Asia: 2300 BC--AD 400}        
\lorem 
    
\end{texexample}
\cxset{defaults}


\subsection*{Adding content before and after the title element}

Like all the other elements, the title element can be decorated with additional content,
before and after the text. There are two different forms. 

\begin{docKey}[phd]{title before}{=\marg{code}}{default none}
Contents before the title (vertical material)
\end{docKey}

\begin{docKey}[phd]{title after}{=\marg{code}}{default none}
Contents after the title (vertical material)
\end{docKey}

\begin{docKey}[phd]{title content before}{=\marg{code}}{default none}
Contents before the title (horizontal material)
\end{docKey}

\begin{docKey}[phd]{title content after}{=\marg{code}}{default none}
Contents after the title (horizontal material)
\end{docKey}

The difference between the two type of settings, consider the following situation. Assume you have a title that has a rule at the top and bottom and the text is surrounded by two ornaments. The surrounding ornaments will be inserted using the |title before content|, and the rules using the |title before| form. The |title before| is a full fledged element on its own. 

%{
%\hrule
%\centering
%*** Introduction ***
%\par
%\hrule
%}
%
%{
%\MakePercentComment
%\startlineat{200}
%\lstinputlisting{./styles/style13.tex}
%\MakePercentIgnore
%}



 
\begin{docKey}{/phd/ chapter title before skip}{= \marg{soul,none}}{}
Before title string skip.
\end{docKey}

\begin{docKey}{/phd/ chapter title after skip}{ = \marg{soul,none} }{}
After title string skip.
\end{docKey}

\lorem 
%
%\begin{texexample}{letter spacing the chapter title block}{ex:title3}
%
%\cxset{chapter spaceout=none,
%         numbering=arabic}
%         
%\chapter{Chapter Title Styling}
%\end{texexample}
%
%\end{document}



\cxset{chapter opening=right}
\section{Table of Contents}\index{table of contents!key settings}

Traditionally a chapter will be added to the Table of Contents if the \cs{chapter} command is issued. The starred version will not produce a number and will not add a contents line. Since we have adopted an approach where we use a key value interface we can dispense with the starred version of the command, by setting the \option{chapter toc} option to false. For example if we want to define a command for a ``Foreward'' or ``Epiloque'' without wishing them to be added to the table of contents we can use the following setting.\index{Foreward>definitions}\index{Epilogue>definitions}



\begin{texexample}{changing the chapter label name}{}
\cxset{chapter name=Chapteris, chapter numbering=arabic,}
\chapter{Foreward}
\lorem
\end{texexample}

Note that the key \option{numbering=none} still has to be set.


Please note that when \textbf{numbering=none} the chapter number is not available anymore and yo may have to reset it if required again. Although this might be seen as rather cumbersome than simply using \cs{chapter*} the advantage is consistency in the user interface and the use of appropriate semantic definitions for all sectioning commands thus achieving a bit more separation of context from style.


%\cxset{chapter toc=true}

\section{Defining styles}

Named styles can be defined using the standard \textsc{PGF} conventions. To define a style for the forward above we can use:

\begin{texexample}{}{}
\cxset{foreward/.style={chapter numbering=none,
          chapter name=none,
          chapter title font-size= Large,
          chapter title font-family= sffamily,
          chapter numbering=none}}
\cxset{foreward}
\chapter{Foreward.}
\lorem
\end{texexample}



\cxset{chapter numbering=arabic}
\section{Creating semantic names for commands and environments}

To keep our search for semantic commands and true separation of contents it is prudent to define some macros for typesetting the  `foreward' section.

\bgroup
\begin{texexample}{defining a \textit{Foreward} macro.}{}
\begin{lstlisting}
\cxset{foreward/.style={chapter toc=false,
          name=none,
          title font-size = Large,
          title font-family = sffamily,
          numbering=none}}
\newcommand\forewardname{foreward}
\expandafter\newenvironment\expandafter{\forewardname}{%
\cxset{foreward}\chapter{Foreward}}%
{}
\begin{foreward}
\lorem
\end{foreward}
\end{lstlisting}
\end{texexample}
\egroup

Notice the use of a new command \cmd{\forewardname} to allow for internationlization using Babel or other methods. One is tempted to let the English name, but a better approach perhaps is to define both.

\makeatletter



%%\makeatletter\@specialtrue\makeatother
%\cxset{custom = stewart}
%\cxset{steward,
%  numbering=arabic,
%  custom=stewart,
%  offsety=0cm,
%  image={./images/hine03.jpg},
%  texti={When Lamport designed the original \LaTeX\ sectioning commands, limitations of computer power forced him to restrict the abstraction of complicated chapter layouts. With current tools available improvements are much easier to program.},
%  textii={In this chapter we discuss a method that allows the production of fancy section headings and formatting, based on a set of key values. Central  to this process is the separation of content from presentation.
%We also discuss the basic formatting tools that are available and how one can modify them to mould new book designs.
% }
% }


\cxset{defaults, chapter format=traditional, 
       chapter opening = left, 
       }


\chapter{Lower Level Headings}


\section{Introduction}

Good book design dictates that sectioning styles follow that the general book design and theme. An academic publication for example might have chapters and section numbered in arabic numerals, whereas a high school textbook might have sections marked in colored boxes. Most traditional books had very humble headings,
set in black ink and the reason was economics. Nowdays most publications will be read online and the use
of color can be useful.

Similarly to the chapter key value interface, the package offers a key value interface to adjust sectioning command parameters.



\cxset{section afterskip={10pt}}

\section{Section styling}

In a similar fashion to the chapter commands the following keys are provided.

\subsection{Fonts and numerals}

Font and numeral keys are shown below.
\medskip
\begin{docKey}[phd]{section font-size}{ = \marg{sizing commands}} {no default, intial=Large}
The font-size command takes arguments
of the  type |Large|, |large| both as commands or without the backslash, which is the recommended way
of setting styles with the |phd| package. 
\end{docKey}

\begin{docKey}[phd] {section font size} {= \marg{sizing commands}} {normal size} 
All the font commands, come in two flavours,
with a hyphen or without, in order to present a user interface that is similar to |pgf/TikZ| conventions for that
are familiar with \latex and another for those used to |CSS| conventions.
\end{docKey}

\begin{docKey}{/phd/section font-family}{= \marg{sizing commands}}{no default, initial value normal} The font-family key, accepts normal LateX values
related to families, but if LuaTeX or XeLaTeX are present it can also accept commands created with |\newfontfamily| 
command of the |fontspec| package, which is loaded automatically by the |phd| package. The package has a database of a number of human friendly names for fonts and commands. If one of these are detected the
family is created at run-time to avoid overloading too many fonts at start-up. 
\begin{verbatim}
\cxset{section font-family = Arial}
\cxset{section font-family = sffamily}
\cxset{section font-family = ttfamily}
\end{verbatim}
The family command family name (if undeined by the user), defaults to the human friendly version name but without the spaces. 
\end{docKey}

%
%  \keyval{section font-weight}{\marg{cmd}}{Font weight command such as \cs{bfseries.}}
%  \keyval{section font-family}{\marg{cmd}}{Font family command such as \cs{sffamily.}}
%  \keyval{section font-shape}{\marg{cmd}}{Font shape command such as \cs{itshape}}
%  \keyval{section color}{\marg{color}}{Color of section.}
%  \keyval{section numbering}{\marg{arabic|roman|Roman|alph|Alph|words|WORDS}}{Section number style.}
  \begin{marglist}
  \item [arabic] Typesers the section number in arabic numerals.
  \item [roman] Typesets the section number in lowercase roman numerals.
  \item [Roman] Typesets the section number in uppercase roman numerals.
  \item [alph] Typesets the section number in lowercase alphabetic numbering.
  \item [Alph] Typesets the section number in uppercase alphabetic numerals.
  \item [words] Typesets the numbers in words (lowercase).
  \item [WORDS] Typesets the number in words (uppercase).
  \end{marglist}

\subsection{Skip and indentation commands}

The keys for indentation and above and below skips are shown below.
\medskip

\keyval{section beforeskip}{}{}
\keyval{section afterskip}{}{}
\keyval{section indent}{\marg{dim}}{Indentation from margin as per standard LaTeX class definitions.}
\keyval{section spaceout}{}{}
\begin{marglist}
 \item[soul]
 \item[none]
\end{marglist}



\subsection{align}

\keyval{section align}{\marg{cmd}}{One of the alignment commands centering, ragged right, raggedleft}

\subsection{Hooks}

Hooks for adding material are shown in the following sketch.
\medskip

\fbox{aboveskip}

\fbox{indent} \fbox{number}\fbox{hook}\fbox{title}

\fbox{belowskip}


\section{Example usage}

In our first example we will use a predefined style for the chapter headings, so we do not need to clutter the example with the chapter commands that we have previously discussed. Our first example will number the section in lower roman, enclosed in brackets and center it.


\makeatletter\@specialfalse
\cxset{
% chapter toc=false,
% chapter  name=CHAPTER,
% numbering=arabic,
% number font-size=huge,
% number font-family=sffamily,
% number font-weight=bfseries,
% number before=,
% number dot=,
% number after=\hspace{1em},
% number position=rightname,
% chapter opening=anywhere,
% chapter font-family=sffamily,
% chapter font-weight=bfseries,
% chapter font-size=huge,
% chapter before={\vspace*{0.1\textheight}\hfill},
% chapter after={\hfill\hfill\vskip0pt\thinrule\par},
% chapter color=black!90,
% number color= black!90,
% title beforeskip={\vspace*{30pt}},
% title afterskip={\vspace*{30pt}\par},
% title before={\hfill},
% title after={\hfill\hfill},
% title font-family=\sffamily,
% title font-color= black!90,
% title font-weight=bfseries,
% title font-size=huge,
 section font-size= LARGE,
 section font-weight= bold,
 section font-family= sffamily,
 section align= centering,
 section numbering=arabic,
 section indent=0em,
 section align= centering,
 section beforeskip=20pt,
 section afterskip=10pt,
 section font-shape= itshape,
}

\cxset{book/.style={
 section numbering=arabic,
 section font-size=Large,
 section font-weight=bfseries,
 section font-family=rmfamily,
 section font-shape=normalfont,
 section align=\raggedright,
 subsection font-size=\large
 section indent=0em,
 section beforeskip=-3.5ex \@plus -1ex\@minus -0.2ex,
 section afterskip=2.3ex\@plus.2ex,
 subsection beforeskip=-3.5ex \@plus -1ex\@minus -0.2ex,
 subsection afterskip= 1.5ex \@plus .2ex,
}}
\makeatother


\begin{texexample}{Adjusting section parameters}{ex:sec}
\cxset{ section font-size= LARGE,
 section font-weight= bold,
 section font-family= sffamily,
 section font-shape=upshape,
 section numbering=(roman), 
 section indent=0em,
 section align= centering,
 section beforeskip=20pt,
 section afterskip=10pt,
 subsection afterskip=3pt,
 subsubsection afterskip=3pt,
 section align=right}
\chapter{A First Look at the Sectioning Keys}
\section{First section}
\lorem
  % adjust counter number so it does not affect the
  % rest of the document
%\addtocounter{section}{-1}
\end{texexample}


The keys are mostly self-explanatory. We have used a |beforeskip| and |afterskip| without any glue. The numbering is just a continuation of the document sections. 

One notable thing to keep in mind is that the numbering of the chapter is independent of that for the section, so if you need to have strange combinations rather define a section numbering custom.\index{section formatting>vertical space}


\cxset{section numbering=arabic}

\subsection{Adjusting vertical spaces}

Perhaps the most important issues we need to consider is the adjusting of vertical spaces; example~\ref{ex:latex}, that follows illustrates settings from the Octavo class and compare them with those of standard the \LaTeXe\ book class. The Octavo class through settings that are based on baselineskip fractions and multiples endeavours to achieve a grid layout. The class also tones down the `loudness' of some of the headings compared to those of the book class.

\makeatletter
\cxset{octavo/.style={
 section font-size=large,
 section font-weight=,
 section font-family=rmfamily,
 section font-shape=scshape,
 section indent=0em,
 section align=\centering,
 section beforeskip=-1.666\baselineskip\@minus -2\p@,
 section afterskip=0.835\baselineskip \@minus 2\p@,
 section after indent = false,
 subsection numbering=none,
 subsection font-family= rmfamily,
 subsection font-size=,
 subsection font-shape=scshape,
 subsection font-weight=,
 subsection indent=1em,
 subsection align=RaggedRight,
 subsection beforeskip=-0.666\baselineskip\@minus -2\p@,
 subsection afterskip=0.333\baselineskip \@minus 2\p@,
 subsection color=spot!50,
 subsubsection color=spot!50,
 }}


\cxset{book/.style={
 section numbering=arabic,
 section font-size= Large,
 section font-weight= bfseries,
 section font-family= rmfamily,
 section font-shape= upshape,
 section align= RaggedRight,
 subsection font-size= large,
 section indent=0em,
 section beforeskip=-3.5ex plus -1ex minus -0.2ex,
 section afterskip=2.3ex plus 0.2ex,
 subsection font-size= large,
 subsection font-weight= bfseries,
 subsection numbering=arabic,
 subsection indent=0pt,
 subsection beforeskip=-3.5ex \@plus -1ex\@minus -0.2ex,
 subsection afterskip= 1.5ex \@plus .2ex,
}}

\cxset{octavo headings/.style={
 section numbering=none,
 section font-size=Large,
 section font-weight=,
 section font-family=rmfamily, section font-shape= scshape,
 section indent=0em, 
 section align=centering, 
 section afterindent=off,
 section beforeskip=-1.666\baselineskip\@minus -2\p@,
 section afterskip=0.835\baselineskip \@minus 2\p@, 
 %
 subsection numbering=none,
 subsection font-family=\rmfamily, 
 subsection font-size=, subsection font-shape=scshape,
 subsection font-weight=, subsection indent=1em, 
 subsection align= RaggedRight,
 subsection beforeskip=-0.666\baselineskip\@minus -2\p@,
 subsection afterskip=0.333\baselineskip \@minus 2\p@,
 subsubsection numbering=none,
 subsubsection font-family= rmfamily,
 subsubsection font-size=,
 subsubsection font-shape= itshape,
 subsubsection font-weight=,
 subsubsection indent = 0em,
 subsubsection align= raggedright,
 subsubsection beforeskip= 0.666\baselineskip\@minus -.2\p@,
 subsubsection afterskip= 3pt, %1.5\baselineskip \@minus .2\p@,
 subsubsection color=spot!50,
 paragraph numbering=none,
 paragraph font-family= rmfamily,
 paragraph font-size=,
 paragraph font-shape=itfamily,
 paragraph font-weight=,
 paragraph color = spot!50,
 paragraph indent=0em,
 paragraph align= RaggedRight,
 paragraph beforeskip=10pt,
 paragraph afterskip=1em,
}}
\makeatother

%\cxset{octavo headings}


%\begin{texexample}{Octavo class headings, settings}{}
%\cxset{octavo headings/.style={
% section numbering=none,section font-size=large,
%section font-weight=,
% section font-family=rmfamily, section font-shape=scshape,
% section indent=0em, 
% paragraph numbering=none,
% paragraph font-family=rmfamily,
% paragraph font-size=,
% paragraph font-shape=,
% paragraph font-weight=,
% paragraph indent=-1em,
% paragraph align=raggedright,
% paragraph beforeskip= 0pt,
% paragraph afterskip=0pt,
%}}
%
%\cxset{octavo headings}
%\renewsection\renewsubsection\renewsubsubsection
%\section{Octavo Class Heading}
%\lorem
%\subsection{Octavo subsection}
%This is some text short text\par
%\subsubsection{Octavo sub-subsection}
%\lorem
%\paragraph{paragraph heading} This is some short text.
%\makeatother
%\end{texexample}

\begin{comment}
The following example was set using the |style| |\cxset{Octavo headings}| with some minor adaptations to enable us to show it inline with the rest of the material on this page\footnote{We set it using \cs{cxset}\marg{chapter opening = anywhere}}. We kept the use of a typical colour throughout the text, whereas the Octavo class, does not allow the use of color.

\cxset{chapter opening = anywhere,
          chapter color = spot!50,
          title font-color = spot!50,
          chapter name={},
          chapter numbering = none,
          chapter before = \addvspace{\baselineskip},
          chapter after = ,
          title spaceout=soul,
          title before =,
          title afterskip=\bigskip\bigskip,
          number before=,
          number after=,
          }
          
\bgroup
\parindent=0pt
\par

\chapter{Octavo Chapter Heading}
\lorem

\section{Octavo Class Heading (Section) }
\lorem

\subsection{Octavo subsection}
\lorem

\subsubsection{Octavo sub-subsection}
\lorem

\paragraph{Paragraph heading} This is some short text.
\lorem

\paragraph{paragraph heading} This is some short text.
\lorem

\egroup
\end{comment}

\begin{texexample}{\LaTeXe\ book class headings settings}{ex:latex}
\makeatletter
\bgroup
\cxset{book/.style={
 section number prefix = \thechapter.,
 section numbering=Roman,
 section number after=,
 section font-size= Large,
 section font-weight=bfseries,
 section font-family=rmfamily,
 section font-shape=upshape,
 section align=RaggedRight,
 section beforeskip=10pt,
 section spaceout = none,
 section color  = red,
 subsection font-size=large,
 section indent=0em,
 section beforeskip=-3.5ex plus1ex minus0.2ex,
 section afterskip=2.3ex\@plus.2ex,
 subsection color = blue,
 subsection font-size=large,
 subsection font-shape=upshape,
 subsection font-weight=bfseries,
 subsection number prefix=\thesection.,
 subsection numbering = arabic,
 subsection beforeskip=-3.5ex \@plus -1ex\@minus -0.2ex,
 subsection indent= 0pt,
 subsection afterskip= 1.5ex \@plus .2ex,
 subsubsection color=black,
}}

\cxset{book}


\section{LaTeX Book  Class Heading}
\lorem
\subsection{A subsection}
\lorem
\subsubsection{A subsubsection}
\egroup

\makeatother
\end{texexample}



\section{Grid example}

One problem sometimes is that the sectioning commands create problems with grid layouts. Example~\ref{ex:grid} shows example settings.

\begin{texexample}{Section styles from the grid package}{ex:grid}
\makeatletter
\cxset{grid/.style={
 section numbering=arabic,
 section font-size=,
 section font-weight=bfseries,
 section font-family=rmfamily,
 section font-shape=upshape,
 section beforeskip=-.999\baselineskip,
 section afterskip=0.001\baselineskip,
 section align= RaggedRight,
 subsection font-size=,
 section indent=0em,
 subsection font-shape=,
 subsection font-weight=bfseries,
 subsection numbering=arabic,
 subsection indent=0pt,
 subsection beforeskip=1\baselineskip,
 subsection afterskip= -.35\baselineskip,
 subsubsection font-shape=itshape,
 subsubsection font-weight=bfseries,
 subsubsection numbering= roman,
 subsubsection number prefix = (,
 subsubsection number suffix =),
 subsubsection indent=0pt,
 subsubsection beforeskip=1\baselineskip,
 subsubsection afterskip= 3pt, %1\baselineskip,
}}
\cxset{grid}




\begin{multicols}{2}
\section{Grid  Class Heading}
\lorem
\subsection{Grid  subsection.}
\lorem
\subsubsection{A subsubsection heading.}
\lorem
\subsubsection{Another subsection heading.}
\lorem
\end{multicols}
\makeatother
\end{texexample}



The key \option{\bfseries section numbering custom}=\marg{code} is quite powerfull and can be used to define any type of section number style. Just remember that the numbering so far depends on two counters, the \docCounter{c@chapter} and \docCounter{c@section}. What the section numbering does, it redefines the macro \docAuxCommand{thesection} to the new definition provided as argument for the key.

Although the temptation to define a lot of key combinations one would rather define new styles as a more user friendly approach.

\cxset{section numbering=arabic, section align= RaggedRight, section font-shape=upshape, section font-family=rmfamily}

\section{Handling Other Section Levels}

Other sectioning commands such as \cs{subsubsection}, \cs{paragraph} and \cs{subparagraph} have equivalent keys. Examples can be found in the chapters that follow for specific styles.

\section{Technical discussion}

The standard LaTeX classes, book report and article have sections showing dot leaders, whereas in the article class the sections are shown without the dotted lines, as the |\l@section| macro is redefined for articles. With the \pkgname{phd} the distinction is unecessary and style files can do the trick that is, either load style article or book or for that matter any other style that has the relevant settings.

\index{macros!\textbackslash @seccntformat}

\subsection{Lower Section Headings}

\LaTeXe\ offers two pathways in redefining section commands, the first one is \refCom{@startsection} and the second is \refCom{@seccntformat} \index{sectioning macros}. It also uses the macro \cs{secdef} to create the starred and unstarred versions of the sectioning commands.

 In the article document class the entry in the table of contents
 for sections looks much like the chapter entries for the report
 and book document classes.
\begin{tcolorbox}{}
\begin{lstlisting}
% \begin{macro}{\l@section}

%
%    First we make sure that if a pagebreak should occur, it occurs
%    \emph{before} this entry. Also a little whitespace is added and a
\newcommand*\l@section[2]{%
  \ifnum \c@tocdepth >\z@
    \addpenalty\@secpenalty
    \addvspace{1.0em \@plus\p@}%
%    \end{macrocode}
%
%    The macro |\numberline| requires that the width of the box that
%    holds the part number is stored in \LaTeX's scratch register
%    |\@tempdima|. Therefore we put it there. We begin a group, and
%    change some of the paragraph parameters (see also the remark at
%    \cs{l@part} regarding \cs{rightskip}).
%    \begin{macrocode}
    \setlength\@tempdima{1.5em}%
    \begingroup
      \parindent \z@ \rightskip \@pnumwidth
      \parfillskip -\@pnumwidth
%    \end{macrocode}
%    Then we leave vertical mode and switch to a bold font.
%    \begin{macrocode}
      \leavevmode \bfseries
%    \end{macrocode}
%    Because we do not use |\numberline| here, we have do some fine
%    tuning `by hand', before we can set the entry. We discourage but
%    not disallow a pagebreak immediately after a section entry.
%    \begin{macrocode}
      \advance\leftskip\@tempdima
      \hskip -\leftskip
      #1\nobreak\hfil \nobreak\hb@xt@\@pnumwidth{\hss #2}\par
    \endgroup
  \fi}
%</article>
\end{lstlisting}
\end{tcolorbox}



As you can see the dot leaders are not present in the above definition. Although we can get rid of dot leaders in other section by redefining them, it is not as easy to add them back.

As our aim is to be able to have all the classes used a common denominator we can define a command as follows (using book as a base)

\begin{tcolorbox}{}
\begin{lstlisting}
\def\articlesection{
\newcommand*\l@section[2]{%
  \ifnum \c@tocdepth >\z@
    \addpenalty\@secpenalty
    \addvspace{1.0em \@plus\p@}%
    \setlength\@tempdima{1.5em}%
    \begingroup
      \parindent \z@ \rightskip \@pnumwidth
      \parfillskip -\@pnumwidth
      \leavevmode \bfseries
      \advance\leftskip\@tempdima
      \hskip -\leftskip
      #1\nobreak\hfil \nobreak\hb@xt@\@pnumwidth{\hss #2}\par
    \endgroup
  \fi}
}
\end{lstlisting}
\end{tcolorbox}


\begin{docCommand}{@startsection}{}
The \cs{@startdsection} macro is one of those locomotive type of commands. It takes 7 required arguments and 2 optional ones and hidden within it are two booleans. The full set looks like this:

\cs{@startsection} \marg{name} \marg{level} \marg{indent} \marg{beforeskip} \marg{afterskip} \marg{style}[*]
  [\marg{altheading}]\marg{heading}.
\end{docCommand}

\begin{marglist}
\item[name] The name of the level command.
\item [level] A number denoting the depth of the section, chapter=1, section=2, etc. A section number will be printed only if \marg{level} is equal or smaller than the value of \textit{secnumdepth}
\item[indent] The indentation of the heading from the left margin.
\item[beforeskip]  The absolute value of this argument is the skip to leave above the heading. If it is negative, then the paragraph indent of the text following the heading is suppressed.
\item [afterskip] If positive, it is the skip to leave below the heading, else it is the skip to the right of a run-in heading.
\item [style] Sets the style of the heading.
\item[\textup{[*]}] When this is missing the heading is numbered and the corresponding counter is incremented.
\item[\textup{[\textit{altheading}]}] Gives an alternative heading to use in the table of contents and in the running heads. This should be present when the * form is used.
\item[heading] The heading of the new section.
\end{marglist}

%\begin{texexample}{Example formatting run-in section}{}
%\makeatletter
%\bgroup
%\renewcommand\section{
%    \@startsection{section}
%    {1}
%    {0em}
%    {-0.8em}
%    {-0.5em}
%    {\large\normalfont\scshape}}
%\makeatother
%\section[]{test}
%\lorem
%\egroup
%\end{texexample}



Note we run the example in a group so that we will not influence the formatting of this document.

As mentioned earlier there is an additional way to introduce formatting for sections and this is using the command \cs{@seccntformat}, which is responsible for typesetting the counter part of a section title. The default definition of the command typesets the \cs{the} representation of the section counter.

%\begin{texexample}{}{}
%\bgroup
%\renewcommand\section{%
%    \@startsection{section}%
%    {1}%
%    {0em}%
%    {-0.8em}%
%    {-0.5em}%
%    {\large\normalfont\scshape}}
%\renewcommand\@seccntformat[1]{\fbox
%{\csname the#1\endcsname}\hspace{0.5em}}
%\makeatother
%\section[]{test}\label{sec:ok}
%\lorem
%
%See section \ref{sec:ok}.
%\egroup
%\end{texexample}



\cxset{section color=spot!50,
          subsection color = spot!50 }
          
\section{Custom headings}

\begin{docCommand*}{@secdef}{\marg{star command} \marg{unstar command}}
So far we have used the |phd|’s keys to set keys that are affecting the standard commands used by
\latexe to set headings. Another way to achieve this,  is to use the macro
 \cs{@secdef}. Therefore, if you wish to use different definitions of \cs{@seccntformat}
for different headings, you must put the appropriate code into every heading
definition.
\end{docCommand*}



\begin{teXXX}
\newcommand\part{\secdef\starcmd\unstarcmd}
\end{teXXX}

The |part| and |chapter| and sometimes |appendix| are defined this way, but nothing stops us from doing the same for other sectioning commands. What the \cs{secdef} command does it will produce the definitions required for a star or unstarred version of the sectioning command, such as |\section|.\footnote{See \ttfamily File F: ltsect.dtx Date: 2014/09/29 Version v1.0z 360} 

\begin{texexample}{}{}
\bgroup
\makeatletter
\renewcommand\section[2] [?]{%
    \refstepcounter{section}
    \addcontentsline{toc}{section}
    {\protect\numberline{section-\thesection}#1}
    {\raggedright\large\bfseries SECTION-\thesection\par \centering#2\par}
    \sectionmark{#1}
    \@afterheading 
   \addvspace{\baselineskip}
 }%
\section[test]{Section Heading}
\lorem
\makeatother
\egroup
\end{texexample}

Many other strategies can also be implemented that are perhaps easier to grasp.

\begin{teX}
\def\@seccntformat##1{\csname the##1\endcsname{}}
\end{teX}

\begin{comment}
\begin{texexample}{}{}
\makeatletter
\bgroup
\def\strut{\vrule height12pt depth1pt width0pt}
  \renewcommand\section[2] []{% % Complex form:
  \refstepcounter{section}% % step counter/ set label
  \addcontentsline{toc}{section}% % generate toc entry
  {\protect\numberline{\thesection} }%
  {\raggedright\large\bfseries\scshape %
  \parbox[b]{\dimexpr(\linewidth-0.5\columnsep)}{\colorbox{brown!80}%
  {{\vbox{\strut\raise2pt\hbox{#2}}}}}}\vskip0pt% % and number
  \sectionmark{#1}% % add to running header
  \@afterheading % prepare indentation handling
  \vspace{\dimexpr\baselineskip+6pt}%must have a parameter
}
\chapter{Fossil Insects}
\begin{multicols*}{2}\raggedcolumns
\section[Insect Fossilization]{\raggedright \thinspace Insect Fossilization}
\lipsum[1]
\end{multicols*}
\egroup
\makeatother
\end{texexample}
\end{comment}

Of course some work is needed to center the text properly in the middle of the colour box. For all practical purposes it is lining up as per the sample.

In Chapter we discussed a forward, but this may not apply if there are no chapters or we need to treat these as sections, the example \ref{ex:forwardsection} shows such a method.


\begin{texexample}{Defining a Foreward Section}{ex:forwardsection}
\makeatletter
\newcommand\prematter@sp[1]{
\addcontentsline{toc}{section}
{\protect\numberline{}#1}
\sectionmark{#1}
{\LARGE\centering\normalfont\sffamily\colorbox{brown!80}{ \textsc{#1}}\par}%
\@afterheading
\addvspace{\baselineskip}
\@afterindentfalse
}

\newenvironment{prematter}[1]{%
   \prematter@sp{#1}}
{}
\begin{multicols}{2}
\label{theok}
\begin{prematter}{Foreward}
\lipsum[1]
\end{prematter}\ref{theok}
\end{multicols}
\makeatother
\end{texexample}


\section{underlining}

I am aware that some people have no choice but have some sections underlined as dictated by archaic regulations in some establishments for thesis submission. If nobody is forcing you to underline it is best to avoid it. We use Donald Arsenau's ulem package to achieve underlining. \footnote{\protect\url{http://tex.stackexchange.com/questions/52998/change-title-to-small-caps-but-not-in-toc}}
\endinput

\makeatletter
\gdef\sectionopen{}
\def\@sectionsuffix{}
\def\@sectionprefix{\sectionname\space}
\newif\if@sectioncase \@sectioncasefalse

\cxset{
  section special/.code =\def\specialsection@cx{#1},
  section xcolor/.store in = \sectionxcolor@cx,
  section opening/.is choice,
  section opening/openany/.code=\gdef\sectionopen{\clearpage},
  section opening/right/.code = \gdef\sectionopen{\cleardoublepage},
  section opening/none/.code = \gdef\sectionopen{},
  section top rule/.is choice, 
  section top rule/true/.code =\DeclareRobustCommand\sectiontoprule{%
        \leavevmode\par\noindent\rule{\textwidth}{1pt}\vskip3.5pt},
  section top rule/true/.code=\def\sectiontoprule{\leavevmode\par\noindent\tikzrule},      
  section top rule/false/.code=\gdef\sectiontoprule{},
  % bottom rule
  section bottom rule/.is choice, 
  section bottom rule/true/.code =\DeclareRobustCommand\sectionbottomrule{%
        \leavevmode\par\noindent\rule{\textwidth}{1pt}\vskip.5pt},
  section bottom rule/true/.code=\def\sectionbottomrule{\vskip-0.5\baselineskip\rlap{\tikzrule}},      
  section bottom rule/false/.code=\gdef\sectionbottomrule{},
  % upper and lower case - TODO in lua
  section case/.is choice,
  section case/lower/.code=\def\sectioncase@cx{\@sectioncasetrue
                             \if@sectioncase\expandafter\MakeTextLowercase\fi},
  section  case/upper/.code=\def\sectioncase@cx{\@sectioncasefalse
                    \if@sectioncase\else\expandafter\MakeTextUppercase \fi},
  section  case/none/.code=\def\sectioncase@cx{\@empty},
}
\cxset{
          section special = sectionspecialruled@cx,
          section xcolor=spot!50,
          section afterindent=false,
          section opening=right,
          section top rule=true,
          section bottom rule=true,
          section afterskip=20pt,
          section case=lower,
          section font-family=aegean
          }


%\def\specialsection@cx{sectionspecialruled@cx}
\def\secdef#1#2{\@ifstar{\@dblarg{#2}}{\@dblarg{#1}}}
%
\newcommand\sectionx{%
  \par  
  \sectionopen   %determines if it is to be treated like a chapter
  \addpenalty\@secpenalty\nobreak
  \secdef\sectionspecialruled@cx\@ssection
   } 
  

% The macro sectionspecial@cx is a more generic macro that typesets the block of tex
% for the section heading.
% 
\def\sectionspecialruled@cx[#1]#2{%
   \sectiontoprule
  \ifnum\c@secnumdepth>0\relax
     \refstepcounter{section}%
     \addcontentsline{toc}{section}{%
      \@sectionprefix\thesection\@sectionsuffix
       \texorpdfstring{\quad}{ }#1}%
  \else
     \addcontentsline{toc}{section}{#1}%
  \fi
  {% start the title
    \color{\sectionxcolor@cx}%
    \noindent\centering\interlinepenalty\@M
   \setfont@cx{\sectionfontweight@cx}%
       {\sectionfontfamily@cx}{\sectionfontsize@cx}{\sectionfontshape@cx}%
     \ifnum\c@secnumdepth>0\relax
        \@sectionprefix\thesection\@sectionsuffix
        \quad\sectioncase@cx{#2}%
    \else %
       \sectioncase@cx{#2}
      % \luadirect{tex.print(string.upper(#2))}%
   \fi%
   \sectionbottomrule
   %\expandafter\addvspace\sectionafterskip@cx\relax%
%   \tikzrule 
   %\rule{\textwidth}{3pt}%
   \afterindent@cx
   \nobreak\par}}


\def\@ssection[#1]#2{%
  \phantomsection
  \addcontentsline{toc}{section}{#1}%
  {\noindent\centering\interlinepenalty\@M
   \color{\sectioncolor@cx}
     \setfont@cx{\sectionfontweight@cx}%
       {\sectionfontfamily@cx}{\sectionfontsize@cx}{\sectionfontshape@cx}%
       \sectiontoprule
       
        \sectioncase@cx{#2}%
        \sectionbottomrule
       %\expandafter \addvspace\sectionafterskip@cx \relax
      \afterindent@cx
   \nobreak\par}}
\makeatother

\let\section\sectionx

\section{Special Sections}

When we described the usage of the chapter setting keys, we extended the system to describe commands
for specially constructed chapter heads that do not follow the normal style of \latexe.

This section describes how to design and program, sectioning styles that go a little bit more than those that
can be defined so far and that they will require you to have a bit more knowledge of \tex and \latexe programming skills.

For example, the heading of this section started on a new page and has rules above and below the title and section number. In addition the title was capitalized automatically, despite having been typed as:

\begin{verbatim}
\section{Special Sections}
\end{verbatim}

By setting the key and calling the section again, we can typeset it on the same page

\begin{verbatim}
   \cxset{section opening=none}
   \section{Another example}
\end{verbatim}

\cxset{section opening=none,
          section case=upper,
          section top rule=false,
          section bottom rule=true,
          section afterindent=false}
          
\section{Another example}

Special sections have their own user provided macros, that have been pre-defined by the user and are invoked using the key |section special|. In the example below we have predefined a macro |\sectionsspecialruled@cx|.
Do not use a command in the value just the literal name of the command as shown below,

\begin{verbatim}
\cxset{section special = ruled,}
\end{verbatim}

\cxset{section opening=none,
          section case=none,
          section top rule=true,
          section bottom rule=false}
          
The star section of the command omits the section number from the heading. It will still insert an entry into the toc. If it is provided with an optional argument it will insert the optional text into the toc.

Check the Table of Contents to see the rendering.

\begin{verbatim}
\section*{No number test}
\section*[Short Title]{No number test}
\end{verbatim}

\section*{No number test}
\lorem

\cxset{section bottom rule=true,
         section afterindent=false,
         section font-family=agean}

\section*[Short Title]{No number test}

\lorem

\cxset{chapter opening=any,
          chapter toc=true,
          chapter numbering=arabic}

One can extend these \emph{specials} to much more complicated sections (which can resemble) chapter openings.
\makeatletter 
\newif\if@debug \@debugtrue
\bgroup
\leftskip-3cm \rightskip2cm
\def\hook{\node[right=5pt, yshift=-12pt] at (0,-3) {\HUGE\color{purple} This is the  Title}; }
\def\hook{}

\cxset{chapter name = CHAPTER}
%\expandafter\ifnum\thechapter=0\stepcounter{chapter}\else\fi

\hspace*{-2cm}\begin{tikzpicture}
\if@debug\draw [help lines] (0,0) grid (18,-13);\else\fi
\draw[fill=red]  (0,0) circle (1.5pt) ;
\node[rectangle,draw, right, baseline] (x) at (0,1) {\LARGE\color{black!30}{before}\relax};
\draw[fill=red]  (0,1) circle (1.5pt) ;
\node[rectangle,draw, right=1sp] at (0,0) {\LARGE\color{black!20} \so\chaptername\relax};

\node[rectangle,draw, color=white, below right, fill=blue!50, text=white] at +(\textwidth,0) {\scalebox{2}{\HUGE \thechapter}};
\draw[fill=red]  (0,-3) circle (1.5pt) ;
% The title of the block
\node[rectangle, draw, text width=9cm,below right, yshift=-1pt] at (0,-3) {%
         \sffamily
         \HUGE Title Format\vskip1sp \medskip\Large Blue colors in jeans, dresses skirts\\ and hats.\\
         How to dress in stylish blues. \\Getting your partner to get\\ into LaTeX. }; 
\node at (12.5,-9) {\includegraphics[width=7cm]{./images/fashion.jpg}};
\hook
\end{tikzpicture}
\makeatother
\tikzrule 
\egroup 

For such complex layouts, it is always best to start from a piece of paper where you roughly outline
the design of the template. I call such layouts templates, because we will insert a number of variables
to parameterize them. All the typesetting commands will need to be inserted in a macro, which you
should give it a unique name. We will name the above template \emph{fashion} and we will later on define
a macro \cmd{\fashion}. The sectioning mechanism provided by the \pkgname{phd} will enable the
setting of such layouts to be carried out as:

\begin{verbatim}
\cxset{section custom = fashion}
\end{verbatim}

Everytime we call the above in our document settings, in the preamble or elswehere or subsequent sections will
be typeset using this format. 

Also before you get into too much detail in programming you should define the \emph{new} parameters
that may have to be introduced. In the example above most of the fields are already defined either
using the |phd|  key value interface or by LaTeX itself. What is new here is only the introduction of an image
and perhaps some rules as to its exact location. For example you can establish a rule that if half the width of
the image is less than the right margin then it should be centered at the right side of the textblock, alternatively it should be lined at the end of the page. We will see how to achieve this a bit later on.

It is also best to start with a MWE and to first achieve the layout you want without any parameters being introduced. We assume that we will be using TikZ to position the text and the image exactly where we 
want them, although nothing stops us from using either plain TeX boxes or the picture environment.
Since we are loading the TikZ package it is best though to use it for the graphical layout.

Introduce a |debug| boolean to help you with switching grid lines on and off. Depending on what you are trying to accomplish you may want to also add some hooks into the definitions. Start from the layout first.

\begin{verbatim}
\begin{tikzpicture}
\if@debug
   \draw [help lines] (0,0) grid (18,-13);
\else
\fi
...
\fashionposthook
\end{tikzpicture}
\end{verbatim}

We draw a grid of $18\times13$ cells which just happens to suit this particular layout well; The command 
\cmd{\fashionposthook} was just added to provide any further tikz instructions at runtime.

We then draw the layout first as best as we can and without too much consideration for parameterizing the layout at this stage.

\emphasis{if@debug,else,fi}
\begin{scriptexample}{}{}
\begin{teX}
\begin{tikzpicture}
\if@debug
  \draw [help lines] (0,0) grid (18,-13);
  \draw[fill=red]  (0,0) circle (1.5pt) ;
  \draw[fill=red]  (0,-3) circle (1.5pt) ;
\else
\fi
% draw debug rectangles
\node[rectangle,draw, right, baseline] (x) at (0,1) {\LARGE\color{black!30}{before}\relax};
\draw[fill=red]  (0,1) circle (1.5pt) ;
\node[rectangle,draw, right=1sp] at (0,0) {\LARGE\color{black!20} \so\chaptername\relax};

\node[rectangle,draw, color=white, below right, fill=blue!50, text=white] at +(\textwidth,0) {\scalebox{2}{\HUGE \thechapter}};

% The title of the block
\node[rectangle, draw, text width=9cm,below right, yshift=-1pt] at (0,-3) {%
         \sffamily
         \HUGE Title Format\vskip1sp \medskip\Large Blue colors in jeans, dresses skirts\\ and hats.\\
         How to dress in stylish blues. \\Getting your partner to get\\ into LaTeX. }; 
   \IfFileExists{\fashionimage@cx}%   
         {\node at (12.5,-9) {\includegraphics[width=7cm]{fashion}};}
         { \node at (12.5,-9) {\includegraphics[width=7cm]{fashion}};}
\hook
\end{tikzpicture}
\end{teX}
\end{scriptexample}

As I mentioned earlier, adding parameters increases the complexity of the layout and it might onfuse you
at first, but we do need to go back and iterate to improve the template.

\begin{description}
\item [odd or even pages]  Most opening layouts such as this one, might be redrawn differently for left or right pages. We need to check for this.
\item [fonts] You should never restrict your template to fixed size fonts or families. Here we can use all the |phd|
keys that are available.
\item [fine tuning positioning] This can be done by defining new keys.
\item [image] Some form of key for the image is required as well as checking, if the image is available or not. If the user forgot to type it in, we will just show a message  and typeset our standard template image.
\makeatletter

\begin{teXX}
\cxset{fashion image/.store in = \fashionimage@cx} (*@\label{fashionimage}@*)
\cxset{fashion image = {./images/fashion.jpg}}
\IfFileExists{\fashionimage@cx}{Found image file code}{Image File not found code}
\end{teXX}



%\IfFileExists{\fashionimage@cx}{image found code}{image not found code}


The line \ref{fashionimage} simply stores the image path and filename in the \cmd{\fashionimage@cx}. We then immediately set it to a default value, to ensure that it is always available. We could just also use a draft
key when we load the image. We will revisit this, once we get ready to test the template. Make sure that you add the \% at the end of the curly brackets when you testing, otherwise you may get weird errors. This is due to the TiKz’s parser. 

\end{description}
\makeatletter
\cxset{fashion image/.code = \gdef\fashionimage@cx{#1}}
\cxset{fashion image = shock.jpg}

\cxset{subtitle font-color/.store in=\subtitlefontcolor@cx}
\cxset{subtitle font-color=black!35}
%default value for the image width
\def\imagewidth@cx{5cm}
\def\fashionnumberbg@cx{gray!30}
\if@debug
   \tikzset{fashion/.style = rectangle, draw}
\else   
\fi
\@debugfalse
\long\gdef\fashion{%
\begin{tikzpicture}

\if@debug
  \draw [help lines] (0,0) grid (18,-13);
  \draw[fill=red]  (0,0) circle (1.5pt) ;
  \draw[fill=red]  (0,-3) circle (1.5pt) ;
\else
\fi
% draw debug rectangles
\node[fashion, right, baseline] (x) at (0,1) {\LARGE\color{black!30}{before}\relax};
\draw[fill=red]  (0,1) circle (1.5pt) ;
\node[fashion, right=1sp] at (0,0) {\LARGE\color{black!20} \so\chaptername\relax};

\node[rectangle,draw, color=white, below right, fill=\fashionnumberbg@cx, text=white] at +(13,0) {\scalebox{2}{\HUGE \thechapter}};

% The title of the block
\node[fashion, text width=9cm,below right, yshift=-1pt] at (0,-3) {%
         \sffamily
         \Huge\color{\titlefontcolor@cx}Title Format\vskip1sp \medskip\Large% 
         \color{\subtitlefontcolor@cx}Blue colors in jeans, dresses skirts\\ and hats.\\
         How to dress in stylish blues. \\Getting your partner to get\\ into LaTeX. }; 
        \IfFileExists{\fashionimage@cx}%   
           {\node at (12.5,-9) {\includegraphics[width=\imagewidth@cx]{\fashionimage@cx}};}%
           { \node at (12.5,-9) {\includegraphics[width=7cm]{shock.jpg}};}%
\end{tikzpicture}
}

At this point let us try the new code and see the small improvements we have done.

\cxset{title font-color=spot!50}
\cxset{subtitle font-color/.store in=\subtitlefontcolor@cx}
\cxset{subtitle font-color=black!35}
\cxset{fashion image=shock.jpg}

% Image needs debugging, something is not capturing it.
\fashion

We have also used a different image and as you can observe with shock, our layout has lost its appeal, will
probably offend some people and the color scheme seems messed up. What we will probably have to do
is add a few more parameters, as well as measure the image’s dimension and implement different rules for
different aspect ratios. Try at this stage and use your own code to modify the layout.

\long\def\storyi{
         In antiquity men and women saw each other as different; 
         accordingly, they developed
        complex taxonomies (philosophical explanations) 
        for understanding anatomical,
        physiological, emotional, and rational differences. \par

Some of these differences seem
profoundly odd to us moderns. Modern discussions about erotic art have often concerned the place of women: to what
extent are they objects of social manipulation, to what extent can they be subjects?
}
\long\gdef\fashion#1{%
\begin{tikzpicture}

\if@debug
  \draw [help lines] (0,0) grid (18,-13);
  \draw[fill=red]  (0,0) circle (1.5pt) ;
  \draw[fill=red]  (0,-3) circle (1.5pt) ;
\else
\fi
% draw debug rectangles
\node[fashion, right, baseline] (x) at (0,1) {\LARGE\color{black!30}{before}\relax};
\draw[fill=red]  (0,1) circle (1.5pt) ;
\node[fashion, right=1sp] at (0,0) {\LARGE\color{black!20} \so\chaptername\relax};

\node[rectangle,draw, color=white, below right, fill=\fashionnumberbg@cx, text=white] at +(12,0) {\scalebox{2}{\HUGE \thechapter}};

% The title of the block
\node[fashion, text width=9cm,below right, yshift=-1pt] at (0,-3) {%
        { \sffamily\raggedleft
        \Huge\bfseries\color{\titlefontcolor@cx}#1\par}
         \bigskip
         \Large% 
         \centering
         \color{\subtitlefontcolor@cx}%
         \raggedleft
        \storyi\par}; 
        \IfFileExists{\fashionimage@cx}%   
           {\node at (12.5,-9) {\includegraphics[width=\imagewidth@cx]{\fashionimage@cx}};}%
           { \node at (12.5,-9) {\includegraphics[width=7cm]{shock.jpg}};}%
\end{tikzpicture}
}

\fashion{SEXUALITY IN ANCIENT GREECE}
\makeatother
\bigskip

Using your document as a User Interface is  programming in a hostile environment. As mentioned
earlier, try pen and paper, it is the quickest way to get a layout right. Adding and removing text, in layouts such
as the one we have been developing is an essential part in getting the layout to get the layout aesthetics right.
Of course other people might have different taste than you and what you like would probably be distateful to other persons.
This is a common lamentation of Graphic Designers, who complain about the value systems of their Clients.

\subsection{Hooking onto LaTeX}

I think the layout is now much better and it has evolved to transform itself from a modern and colorful template to a more serious one, perhaps more appropriate for scientific work.

We have now won half the battle, the next battle is to hook into the |\section| or |\chapter| command using |\secdef|. As you might have noticed, the chapter number has not been incremented. We will need to also
add it to the Table of Contents and also get the indentation after the heading to work correctly. We do not want our users to have to worry about this and adding |\noindent|’s all over the place. At this point we will also 
add functions to add the chapter number and title to the Table of Contents. 

\makeatother

%\makeatletter\@specialfalse\makeatother
%\input{./sections/more-on-boxes}
%\input{./styles/style87}
%\cxset{section align=left}
%\cxset{section font-weight=bold}
%\cxset{section font-family=sffamily} 
%\cxset{section top rule=false,
%          section bottom rule = false,
%}
          
          
          
\makeatletter
%\@debugtrue
\makeatother
\let\bs\textbackslash
\parskip1.5pt 
\newfontfamily\tibetan{TibMachUni.ttf}
\def\deva{{\protect\tibetan\symbol{"0F7C} xx ༃}}

\chapter{Indices}
\pagebreak

\starttemplate{kroll}
\thispagestyle{empty}
    \begin{leftcolumn}
       \begin{center} 
          \huge \noindent PREPARING\\
                   INDEXES
       \end{center}
     
      \medskip

       {\justifying \small\noindent And in such indexes, although small pricks\\
To their subsequent volumes, there is seen\\
The baby figure of the giant mass\\
Of things to come at large. \par
\hfill \textit{--Shakespeare}\par
\hfill\hfill{ \RaggedRight from \textit{Troilus and Cressida}}}
\medskip
       \putimage[width=1.0\linewidth]{./images/animalium01.jpg}\par
       \aheader{Handwritten cards compiled by  Sherborn for his publication \textit{Index Animalium}}
   \end{leftcolumn}
   \begin{rightcolumn}
       \putimage[width=\linewidth]{./images/animalium.jpg}
       \onelinecaption{{\resizebox{\linewidth}{5.5pt}{\bfseries Sherborn’s `Index  Animalium'. Credit Natural History Museum, London\footnote{This monumental publication became the basis for all zoological nomenclature work having gathered together all the relevant data in one place, just as an online database does today.}}}\par}
%  \centerline{\onelineheader{TYPESETTING AN INDEX}}
      \begin{multicols}{2}
        
\parindent1em      \lettrine{T}{he} first English Language index, appeared in Christopher Marlowe's \textit{Hero and Leander} in 1593. At that period, as often as not, by an ``index to a book'' was meant what we should now call a table of contents. Among the first indexes---in the modern sense---to a book in the English language was one in Plutarch's Parallel Lives, in Sir Thomas North's 1595 translation\footnote{Borko, Harold \& Bernier, Charles L. (1978). \textit{Indexing Concepts and Methods}, ISBN 0-12-118660-1.}.  

A section entitled ``An Alphabetical Table of the most material contents of the whole book'' may be found in Henry Scobell's Acts and Ordinances of Parliament of 1658. This section comes after ``An index of the general titles comprised in the ensuing Table''. Both of these indexes predate the index to Alexander Cruden's Concordance (1737) see \citep{farrow96}, which is erroneously held to be the earliest index found in an English book.

      \end{multicols}
   \end{rightcolumn}
\stoptemplate

\pagestyle{headings}


\index{Quantum Mechanics>History|(}
\setlength{\columnsep}{2em}
\begin{multicols}{2}

\section{Preparing an index}

\index[latex]{latex>test}
In order to produce an index, we need to load the
package \pkg{makeidx}  and immediately issue the command \cmd{\makeindex}.\footnote{You will also need to at least run the file once using |MakeIndex| on |MikTex|. Check your distribution if you getting problems.}
At the place where
we want the index to be printed,we use \docAuxCmd{printindex}.
\parindent1em

In \latex, we include a word
in the index by using the command \cmd{\index}\meta{arg}, so if the word Kroll should be included in
the index, we should use the command |\index{Kroll}|.

If the word is to be printed in bold, we use

% : = |
% = = @
% \catcode `|
\bgroup
 \catcode`|=11
\gdef\idxmain#1{%
   \def\idxmainentryi##1##2##3;{%
      \index{Kroll=\textbf{#1}|textbf}
    }
   \idxmainentryi#1;    
}  
\egroup

\idxmain{Kroll}

\index{Kroll=\textbf{Kroll}>Leon Kroll}


\emphasis{makeidx,makeindex,printindex}
\begin{teX}
\documentclass{article}
\usepackage{makeidx}
\makeindex
\begin{document}
   To prepare an index, 
   just include the
   package\index{package}

\printindex
\end{document}
\end{teX}



There is a  special character |@| which is used to denote that what appears on its left side must be
typeset as it appears on its right side.  The first occurrence of perl will also be used
by the sorting algorithm. is is very useful since what is used for sorting and what
will be printed may be different! For example,we may want to have the name 
\texttt{Donald Knuth}  under the letter K., we should write


\verb+\index{Knuth@{Donald Knuth}}+

Another thing we may want to change is the way that the page number is typeset.
If we want, for example, to have the page number in bold, we would write
\verb+\index{perl|textbf}+.  Notice that we wrote "textbf"  without the backslash. Of course,
the above can be combined with the  command


\verb+\index{perl@\textbf{perl}|textit}+


will print the word perl in the index (the entry will be typeset in boldface type) sorted
as "perl",’ and its page number will be italic. A common application of this is through
the command . If we want to send the reader to another index entry, say, to send
the reader from the \verb+$\omega$+ to the \verb+$\Omega$+ command, we can write

\verb+\index{omega@$\omega$|see{$\Omega$}}+


Here, we ask for the entry to be sorted according to the word omega and, in its place,
the program must use 

\begin{teX}
$\omega$|see{$\Omega$}
\end{teX}

If a word is used repeatedly in a range of pages and we want to have this range
in the index, we do not write the relative \texttt{index} command all of the time. Instead,
we write \verb+\index\{convex\|(\}+  at the place where we have the first occurrence and
\verb+\index\{convex|)\}+  at the place where we have the last occurrence. This will produce a
page range in the index for the word "convex".

\subsection{Subindices}
Subindices can be produced using an exclamation mark. If we want the word `Zeus'
to appear in the category of Greek which is in the category of Gods, we will write:

\verb+\index{Gods!Greek!Zeus}+

The actual symbol used will depend on the |.ist| file used when the file is compiled. For example in the |ltxdoc| class this is redefined as |>|.
\DeclareRobustCommand\textat{%
  \bgroup\makeatother \egroup
}


\section{Multiple Pages}

To perform multi-page indexing, add a |( and |) to the end of the \cmd{\index} command, as in 
\index{Indexing>multi-page}.

{\small
\verb+\index{Quantum Mechanics!History|(}+

\narrower\narrower
In 1901, Max Planck released his theory of radiation dependant 
on quantized energy. While this explained the ultraviolet catastrophe
 in the spectrum of blackbody radiation, this had far larger consequences 
as the beginnings of quantum mechanics.\ldots

\verb+\index{Quantum Mechanics!History|)}+
}

\end{multicols}

\index{Quantum Mechanics>History|)}


\section{Summary of commands}

\begin{tabular}{lll}
\toprule
Example	&Index Entry	&Comment\\
\midrule
\textbackslash index\{hello\}	          &hello, 1	&Plain entry\\
\textbackslash index\{hello!Peter\}	      &Peter, 3	&Subentry under 'hello'\\
\textbackslash index\{Sam@\textbackslash textsl\{Sam\}\}	&Sam, 2	&Formatted entry\\
\textbackslash index\{Lin@\textbackslash textbf\{Lin\}\}	&\textbf{Lin}, 7	&Same as above\\
\textbackslash index\{Jennytextbf\}	     &Jenny, 3	&Formatted page number\\
\textbackslash index\{Joe textit\}	&Joe, 5	          &Same as above\\
\textbackslash index\{ecole@\'ecole\}	&école, 4	&Handling of accents\\
\textbackslash index\{Peter see\{hello\}\}	&Peter, see hello	&Cross-references\\
\textbackslash index\{Jen see also\{Jenny\}\}	&Jen, see also Jenny	 &Same as above\\
\bottomrule
\end{tabular}

\section{Indexing Class Documentation}

\index{Indexing=\textbf{Indexing}}
\index{Indexing>general}
\index{Indexing>doc}

Indexing \latex2e classes or package documentation produced with the \docClass{ltxdoc} is normally achieved using the \pkg{doc} and \pkg{docstrip} program, which are sometime difficult to use, if you need to deviate from their standard methods. The important thing here to remember is that you need to use different characters |=| |>| |*|.

\begin{tabular}{ll}
\toprule
normal    & doc \\
\midrule
\string @ & \texttt{=} \\
\string ! & \texttt{>}\\
\bottomrule
\end{tabular}

\begin{verbatim}
\index{Indexing=\textbf}
\index{Indexing>general}
\index{Indexing>doc}
\end{verbatim}

This manual, was build using a large |ltxdoc| class and these problems appeared while I was developing it. As normal with such problems, they were very time consuming to debug. There are still issues in some parts and one day, I am hoping to come back and correct them. One needs at this point to query the need to use the |doc| and |docstrip| method of documenting macros and if it shouldn't have a pre-processor written in a higher language to ease development. 

\chapter{The makeindex program}

The makeindex program was developed by Pehong Chen and Michael Harrison \fullcite{chen01} in the eighties and is still used by \latexe. It is a remarkable, flexible program, found on most distributions. It is also used in GNU EMACS. 

The input format consists of a list of \meta{specifier, attribute} tuples. These are the essential tokens and delimiters needed in scanning the input index file. There are ten such specifiers (you can think of them as the names of functions or variables) and the attributes as their value. 

\captionof{table}{Index input style parameters}
\begin{longtable}{l l l l}
\toprule
specifier &attribute &default &meaning\\
\midrule
|keyword|     & string &|"\\indexentry"| & index command\\
|arg_open|    & char   &|'{'|            & argument opening delimiter\\
|arg_close|   & char   &|'}'|            & argument closing delimiter\\
|range_open|  & char   &|'('|            & page range opening delimiter\\
|range_close| & char   &|')'|           & page range closing delimiter\\
|level|       & char   &|'!'|             & index level delimiter\\     
|actual|      & char   &|'@'|             & actual key designator\\
|encap|       & char   &  $\|$            & page number encapsulator\\
|quote|       & char & |'"'|            & quote symbol\\
|escape|      & char & |'\\'|           & symbol which escapes quote\\
|page_compositor| & string & ''-''       & composite page delimiter\\
\bottomrule
\end{longtable}


\begin{description}

\item[composite page delimiter] A page number can be a composite of one or more fields separated by a certain delimiter bound to a \textit{page\_compositor}  e.g. II-13 for page 13 of Chapter II. This attribute allows the lexical analyzer of the makeindex program to seprate these fields, making the sorting of page numbers easier.
\end{description}


\section{Output format}



The file \docFile{gind.ist} is shown below
\begin{teXXX}
actual '='
encap '|'
level '>'
quote '!'
preamble
"\n \\begin{theindex} \n \\makeatletter\\scan@allowedfalse\n"
postamble
"\n\n \\end{theindex}\n"
item_x1   "\\efill \n \\subitem "
item_x2   "\\efill \n \\subsubitem "
delim_0   "\\pfill "
delim_1   "\\pfill "
delim_2   "\\pfill "
% The next lines will produce some warnings when
% running Makeindex as they try to cover two different
% versions of the program:
%lethead_prefix   "{\\bfseries\\hfil "
%lethead_suffix   "\\hfil}\\nopagebreak\n"
%lethead_flag       1
heading_prefix   "{\\bfseries\\hfil "
heading_suffix   "\\hfil}\\nopagebreak\n"
headings_flag       1
%%
%%
%% End of file `gind.ist'.
\end{teXXX}


\section{Customization}\index{Indexing>customizing}

When creating an index with \pkgname{makeindex} one can create a \docFile{sample.ist} file that can be used together with the |makeidx| program to customize the way the index will look.

\begin{verbatim}
heading_prefix "{\\bfseries\\hfil "
heading_suffix "\\hfil}\\nopagebreak\n"
headings_flag 1
delim_0 "\\dotfill"
delim_1 "\\dotfill"
delim_2 "\\dotfill"
\end{verbatim}

This will write the first alphabet symbol in bold font, and uses dots as delimiters. This file is generally  used jointly with \texttt{makeindex} using

\begin{teX}
makeindex -s sample.ist filename.idx
\end{teX}

where |filename.idx| has been craeted by executing |latex| or one of the other engine commands such as |pdflatex| on |filename.tex|.


According to \citep{gabora}, you may use

\begin{teX}
sort_rule "\." "\b\."
sort_rule "\:" "\b\:"
sort_rule "\," "\b\,"
\end{teX}


\section{Writing custom indexing commands}

For complex documents it is easier to write a number of macros to assist with indexing and to also provide consistency. For example if you want to index the Devanagari alphabet we might need to get quite creative as to how to both index it as well as get the symbols in the index.
\DeclareRobustCommand\ta{{\tibetan ༃ }}

%\begin{texexample}{Writing Indexing Commands}{ex:zs}
%\gdef\ZZs#1{\incsyms%
%   \indexcommand[\string#1]{#1}%
%   #1}
%\DeclareRobustCommand\ta{{\tibetan ༃}\xparse}
%\ta
%\end{texexample}
%
%The command |\ZZs{\ta}| typesets the command in the document, as (\ZZs{\ta}) and also adds it to the index and typesets the symbol.
%
%
%\begin{phdverbatim}
%\def\ZZs#1{\incsyms%
%   \indexcommand[\string#1]{#1}
%   \string#1}
%\end{phdverbatim}
\endinput
























%\chapter{Unicode Math}
\tcbdocmarginnote{N 29-06-2018}
Unicode contains separate codepoints for most if not all variations of alphabet
shape one may wish to use in mathematical notation. The complete list is shown
in table 5. Some of these have been covered in the previous sections.
The math font switching commands do not nest; therefore if you want sans
serif bold, you must write |\mathbfsf{...}| rather than |\mathbf{\mathsf{...}}|.
This may change in the future.

\section{Unicode maths font setup}

The promise of Unicode is that all symbols and alphabetic variants are in one font. The \pkgname{unicode-math}
maps all the available unicode math characters of a math font to respective \latex commands. If you have patience you can actually input them directly from the keyboard rather than in commands.

The best advice that I can give you is to read the \pkgname{unicode-math} carefully. 

\begin{docCommand} {setmathfont} { \oarg{range=\meta{unicode range}, \meta{font features } } \marg{font name} }
In many cases using one font might not be adequate. Specific Unicode ranges can be assigned to separate fonts.
\end{docCommand}

\subsection{Control over maths alphabets}

\subsection{Math `versions'}

\subsection{Maths input}

\subsection{Math `style'}

\subsubsection{Bold style}

Similar as in the previous section, ISO standards differ somewhat to \tex’s conventions
(and classical typesetting) for ‘boldness’ in mathematics. In the past, it has
been customary to use bold upright letters to denote things like vectors and matrices.


$$\boldsymbol{\omega} \times \mathbf{T} = \mathbf{T'}$$

$$\mathbf{\omega} = {1\over 2}\kappa \mathbf{B} + {1\over 2}(\kappa \mathbf{B} + \tau \mathbf{T}) + {1\over 2}\tau \mathbf{T} = \kappa \mathbf{B} + \tau \mathbf{T}
$$

\[
\mathbf{e} = \frac{\mathbf{A}}{m k} = \frac{1}{m k}(\mathbf{p} \times \mathbf{L})
\]

\subsubsection{Sans serif style}

\subsubsection{Blackboard or double-struck}



\subsubsection{Caligraphic and Script variants}



\section{Growing and non-growing accents}

This are the most problematic with Unicode fonts.

%%%%%%%% INPUT INTEGRAL FILES %%%%%%%%%%
%%%%%%%%%%%%%%%%%%%%%%%%%%%%%%%
\input{mathdelimit}
\input{mathalphabetics}
%\input{mathaccents}
 \subsection{Big operators}
 \begin{multicols}{2}
 \showop\Bbbsum{2140}{}
 \showop\prod{220F}{}
 \showop\coprod{2210}{}
 \showop\sum{2211}{}
 \showop\bigwedge{22C0}{}
 \showop\bigvee{22C1}{}
 \showop\bigcap{22C2}{}
 \showop\bigcup{22C3}{}
 \showop\leftouterjoin{27D5}{*}
 \showop\rightouterjoin{27D6}{*}
 \showop\fullouterjoin{27D7}{*}
 \showop\bigbot{27D8}{*}
 \showop\bigtop{27D9}{*}
 \showop\xsol{29F8}{*}
 \showop\xbsol{29F9}{*}
 \showop\bigodot{2A00}{*}
 \showop\bigoplus{2A01}{*}
 \showop\bigotimes{2A02}{*}
 \showop\bigcupdot{2A03}{*}
 \showop\biguplus{2A04}{*}
 \showop\bigsqcap{2A05}{*}
 \showop\bigsqcup{2A06}{*}
 \showop\conjquant{2A07}{*}
 \showop\disjquant{2A08}{*}
 \showop\bigtimes{2A09}{*}
 \showop\modtwosum{2A0A}{*}
 \showop\Join{2A1D}{*}
 \showop\bigtriangleleft{2A1E}{*}
 \showop\zcmp{2A1F}{*}
 \showop\zpipe{2A20}{*}
 \showop\zproject{2A21}{*}
 \showop\biginterleave{2AFC}{}
 \showop\bigtalloblong{2AFF}{*}
 \end{multicols}
 
 \section{General manipulations test.}
 
 Here are some examples using big operators

$$\sum_{n=s}^t C\cdot f(n) = C\cdot \sum_{n=s}^t f(n),$$ where $C$ is a constant

\[\sum_{n=s}^t f(n) + \sum_{n=s}^{t} g(n) = \sum_{n=s}^t \left[f(n) + g(n)\right]\] 

\[\sum_{n=s}^t f(n) - \sum_{n=s}^{t} g(n) = \sum_{n=s}^t \left[f(n) - g(n)\right]\] 

\[\sum_{n=s}^t f(n) = \sum_{n=s+p}^{t+p} f(n-p) \] 

\[\sum_{n\in B} f(n) = \sum_{m\in A} f(\sigma(m)),\] for a bijection σ from a finite set ``$A$'' onto a finite set ``$B$''; this generalizes the preceding formula.

\[\sum_{n=s}^j f(n) + \sum_{n=j+1}^t f(n) = \sum_{n=s}^t f(n)\] 

\[\sum_{i=k_0}^{k_1}\sum_{j=l_0}^{l_1} a_{i,j} = \sum_{j=l_0}^{l_1}\sum_{i=k_0}^{k_1} a_{i,j}\] 

\[\sum_{k\le j \le i\le n} a_{i,j} = \sum_{i=k}^n\sum_{j=k}^i a_{i,j} = \sum_{j=k}^n\sum_{i=j}^n a_{i,j}\] 


\[\sum_{n=0}^t f(2n) + \sum_{n=0}^t f(2n+1) = \sum_{n=0}^{2t+1} f(n)\] 

\[\sum_{n=0}^t \sum_{i=0}^{z-1} f(z\cdot n+i) = \sum_{n=0}^{z\cdot t+z-1} f(n)\] 

\[\sum_{i=s}^m\sum_{j=t}^n {a_i}{c_j} = \sum_{i=s}^m a_i \cdot \sum_{j=t}^n c_j\] 

\[\sum_{n=s}^t \ln f(n) = \ln \prod_{n=s}^t f(n)\] 

\[c^{\left[\sum_{n=s}^t f(n) \right]} = \prod_{n=s}^t c^{f(n)}\] 
 
 
 \section{Convergence criteria}
 
The product of positive real numbers

\[\prod_{n=1}^{\infty} a_n\]
converges to a nonzero real number if and only if the sum
\[\sum_{n=1}^{\infty} \log(a_n)\]
converges. This allows the translation of convergence criteria for infinite sums into convergence criteria for infinite products. The same criterion applies to products of arbitrary complex numbers (including negative reals) if log is understood as a fixed Complex logarithm which satisfies $\log(1) = 0$, with the proviso that the infinite product diverges when infinitely many ''$a_n$'' fall outside 
the domain of log, whereas finitely many such ''$a_n$'' can be ignored in the sum.

For products of reals in which each $a_n\ge1$, written as, for instance, $a_n=1+p_n$,
where $p_n\ge 0$, the bounds

\[1+\sum_{n=1}^{N} p_n \le \prod_{n=1}^{N} \left( 1 + p_n \right) \le \exp \left( \sum_{n=1}^{N}p_n \right)\]

show that the infinite product converges precisely if the infinite sum of the $p_n$ converges. This relies on the Monotone convergence theorem. More generally, the convergence of $\prod_{n=1}^\infty(1+p_n)$ is equivalent to the convergence of $\sum_{n=1}^\infty p_n$ 
%if ''p<sub>n</sub>'' are real or complex numbers such that <math>\sum_{n=1}^\infty|p_n|^2<+\infty\], since <math>\log(1+x)=x+O(x^2)\] in a neighbourhood of 0.
%
%If the series ''p''<sub>''n''</sub> diverges, then the sequence of partial products converges to zero as a sequence.  The infinite product is said to '''diverge to zero'''.
% 
 
 
 
 
 
 
 
 
%% Relations symbols from STIX font
%% 
  \subsection{Relations}
 \begin{multicols}{2}
% \showrelsymbol*{002A}{}, \cmd\ast
% \showrelsymbol:{003A}{}
% \showrelsymbol{\less}{003C}{}, \cmd\less
% \showrelsymbol{\equal}{003D}{}, \cmd\equal
% \showrelsymbol{\greater}{003E}{}, \cmd\greater
 \showrelsymbol\closure{2050}{*}
 %\showrelsymbol\vertoverlay{20D2}{}
 \showrelsymbol\leftarrow{2190}{}, \cmd\gets
 \showrelsymbol\uparrow{2191}{}
 \showrelsymbol\rightarrow{2192}{}, \cmd\to
 \showrelsymbol\downarrow{2193}{}
 \showrelsymbol\leftrightarrow{2194}{}
 \showrelsymbol\updownarrow{2195}{}
 \showrelsymbol\nwarrow{2196}{}
 \showrelsymbol\nearrow{2197}{}
 \showrelsymbol\searrow{2198}{}
 \showrelsymbol\swarrow{2199}{}
 \showrelsymbol\nleftarrow{219A}{}
 \showrelsymbol\nrightarrow{219B}{}
 \showrelsymbol\leftwavearrow{219C}{}
 \showrelsymbol\rightwavearrow{219D}{}
 \showrelsymbol\twoheadleftarrow{219E}{}
 \showrelsymbol\twoheaduparrow{219F}{}
 \showrelsymbol\twoheadrightarrow{21A0}{}
 \showrelsymbol\twoheaddownarrow{21A1}{}
 \showrelsymbol\leftarrowtail{21A2}{}
 \showrelsymbol\rightarrowtail{21A3}{}
 \showrelsymbol\mapsfrom{21A4}{}
 \showrelsymbol\mapsup{21A5}{}
 \showrelsymbol\mapsto{21A6}{}
 \showrelsymbol\mapsdown{21A7}{}
 \showrelsymbol\hookleftarrow{21A9}{}
 \showrelsymbol\hookrightarrow{21AA}{}
 \showrelsymbol\looparrowleft{21AB}{}
 \showrelsymbol\looparrowright{21AC}{}
 \showrelsymbol\leftrightsquigarrow{21AD}{}
 \showrelsymbol\nleftrightarrow{21AE}{}
 \showrelsymbol\downzigzagarrow{21AF}{}
 \showrelsymbol\Lsh{21B0}{}
 \showrelsymbol\Rsh{21B1}{}
 \showrelsymbol\Ldsh{21B2}{}
 \showrelsymbol\Rdsh{21B3}{}
 \showrelsymbol\curvearrowleft{21B6}{}
 \showrelsymbol\curvearrowright{21B7}{}
 \showrelsymbol\circlearrowleft{21BA}{}
 \showrelsymbol\circlearrowright{21BB}{}
 \showrelsymbol\leftharpoonup{21BC}{}
 \showrelsymbol\leftharpoondown{21BD}{}
 \showrelsymbol\upharpoonright{21BE}{}, \cmd\restriction
 \showrelsymbol\upharpoonleft{21BF}{}
 \showrelsymbol\rightharpoonup{21C0}{}
 \showrelsymbol\rightharpoondown{21C1}{}
 \showrelsymbol\downharpoonright{21C2}{}
 \showrelsymbol\downharpoonleft{21C3}{}
 \showrelsymbol\rightleftarrows{21C4}{}
 \showrelsymbol\updownarrows{21C5}{}
 \showrelsymbol\leftrightarrows{21C6}{}
 \showrelsymbol\leftleftarrows{21C7}{}
 \showrelsymbol\upuparrows{21C8}{}
 \showrelsymbol\rightrightarrows{21C9}{}
 \showrelsymbol\downdownarrows{21CA}{}
 \showrelsymbol\leftrightharpoons{21CB}{}
 \showrelsymbol\rightleftharpoons{21CC}{}
 \showrelsymbol\nLeftarrow{21CD}{}
 \showrelsymbol\nLeftrightarrow{21CE}{}
 \showrelsymbol\nRightarrow{21CF}{}
 \showrelsymbol\Leftarrow{21D0}{}
 \showrelsymbol\Uparrow{21D1}{}
 \showrelsymbol\Rightarrow{21D2}{}
 \showrelsymbol\Downarrow{21D3}{}
 \showrelsymbol\Leftrightarrow{21D4}{}
 \showrelsymbol\Updownarrow{21D5}{}
 \showrelsymbol\Nwarrow{21D6}{}
 \showrelsymbol\Nearrow{21D7}{}
 \showrelsymbol\Searrow{21D8}{}
 \showrelsymbol\Swarrow{21D9}{}
 \showrelsymbol\Lleftarrow{21DA}{*}
 \showrelsymbol\Rrightarrow{21DB}{*}
 \showrelsymbol\leftsquigarrow{21DC}{}
 \showrelsymbol\rightsquigarrow{21DD}{}, \cmd\leadsto
 \showrelsymbol\barleftarrow{21E4}{*}
 \showrelsymbol\rightarrowbar{21E5}{*}
 \showrelsymbol\circleonrightarrow{21F4}{*}
 \showrelsymbol\downuparrows{21F5}{}
 \showrelsymbol\rightthreearrows{21F6}{*}
 \showrelsymbol\nvleftarrow{21F7}{*}
 \showrelsymbol\nvrightarrow{21F8}{*}
 \showrelsymbol\nvleftrightarrow{21F9}{*}
 \showrelsymbol\nVleftarrow{21FA}{*}
 \showrelsymbol\nVrightarrow{21FB}{*}
 \showrelsymbol\nVleftrightarrow{21FC}{*}
 \showrelsymbol\leftarrowtriangle{21FD}{*}
 \showrelsymbol\rightarrowtriangle{21FE}{*}
 \showrelsymbol\leftrightarrowtriangle{21FF}{*}
 \showrelsymbol\in{2208}{}
 \showrelsymbol\notin{2209}{}
 \showrelsymbol\smallin{220A}{}
 \showrelsymbol\ni{220B}{}, \cmd\owns
 \showrelsymbol\nni{220C}{}
 \showrelsymbol\smallni{220D}{}
 \showrelsymbol\propto{221D}{}
 \showrelsymbol\varpropto{221D}{}
 \showrelsymbol\mid{2223}{}
 \showrelsymbol\shortmid{2223}{}
 \showrelsymbol\nmid{2224}{}
 \showrelsymbol\nshortmid{2224}{*}
 \showrelsymbol\parallel{2225}{}
 \showrelsymbol\shortparallel{2225}{*}
 \showrelsymbol\nparallel{2226}{}
 \showrelsymbol\nshortparallel{2226}{*}
 \showrelsymbol\Colon{2237}{}
 \showrelsymbol\dashcolon{2239}{}
 \showrelsymbol\dotsminusdots{223A}{}
 \showrelsymbol\kernelcontraction{223B}{}
 \showrelsymbol\sim{223C}{}
 \showrelsymbol\thicksim{223C}{}
 \showrelsymbol\backsim{223D}{}
 \showrelsymbol\nsim{2241}{}
 \showrelsymbol\eqsim{2242}{}
 \showrelsymbol\simeq{2243}{}
 \showrelsymbol\nsime{2244}{}
 \showrelsymbol\cong{2245}{}
 \showrelsymbol\simneqq{2246}{}
 \showrelsymbol\ncong{2247}{}
 \showrelsymbol\approx{2248}{}
 \showrelsymbol\thickapprox{2248}{}
 \showrelsymbol\napprox{2249}{}
 \showrelsymbol\approxeq{224A}{}
 \showrelsymbol\approxident{224B}{}
 \showrelsymbol\backcong{224C}{}
 \showrelsymbol\asymp{224D}{}
 \showrelsymbol\Bumpeq{224E}{}
 \showrelsymbol\bumpeq{224F}{}
 \showrelsymbol\doteq{2250}{}
 \showrelsymbol\Doteq{2251}{}, \cmd\doteqdot
 \showrelsymbol\fallingdotseq{2252}{}
 \showrelsymbol\risingdotseq{2253}{}
 \showrelsymbol\coloneq{2254}{}
 \showrelsymbol\eqcolon{2255}{}
 \showrelsymbol\eqcirc{2256}{}
 \showrelsymbol\circeq{2257}{}
 \showrelsymbol\arceq{2258}{}
 \showrelsymbol\wedgeq{2259}{}
 \showrelsymbol\veeeq{225A}{}
 \showrelsymbol\stareq{225B}{}
 \showrelsymbol\triangleq{225C}{}
 \showrelsymbol\eqdef{225D}{}
 \showrelsymbol\measeq{225E}{}
 \showrelsymbol\questeq{225F}{}
 \showrelsymbol\ne{2260}{}, \cmd\neq
 \showrelsymbol\equiv{2261}{}
 \showrelsymbol\nequiv{2262}{}
 \showrelsymbol\Equiv{2263}{}
 \showrelsymbol\leq{2264}{}, \cmd\le
 \showrelsymbol\geq{2265}{}, \cmd\ge
 \showrelsymbol\leqq{2266}{}
 \showrelsymbol\geqq{2267}{}
 \showrelsymbol\lneqq{2268}{}
 \showrelsymbol\lvertneqq{2268}{}
 \showrelsymbol\gneqq{2269}{}
 \showrelsymbol\gvertneqq{2269}{}
 \showrelsymbol\ll{226A}{}
 \showrelsymbol\gg{226B}{}
 \showrelsymbol\between{226C}{}
 \showrelsymbol\nasymp{226D}{}
 \showrelsymbol\nless{226E}{}
 \showrelsymbol\ngtr{226F}{}
 \showrelsymbol\nleq{2270}{}
 \showrelsymbol\ngeq{2271}{}
 \showrelsymbol\lesssim{2272}{}
 \showrelsymbol\gtrsim{2273}{}
 \showrelsymbol\nlesssim{2274}{}
 \showrelsymbol\ngtrsim{2275}{}
 \showrelsymbol\lessgtr{2276}{}
 \showrelsymbol\gtrless{2277}{}
 \showrelsymbol\nlessgtr{2278}{}
 \showrelsymbol\ngtrless{2279}{}
 \showrelsymbol\prec{227A}{}
 \showrelsymbol\succ{227B}{}
 \showrelsymbol\preccurlyeq{227C}{}
 \showrelsymbol\succcurlyeq{227D}{}
 \showrelsymbol\precsim{227E}{}
 \showrelsymbol\succsim{227F}{}
 \showrelsymbol\nprec{2280}{}
 \showrelsymbol\nsucc{2281}{}
 \showrelsymbol\subset{2282}{}
 \showrelsymbol\supset{2283}{}
 \showrelsymbol\nsubset{2284}{}
 \showrelsymbol\nsupset{2285}{}
 \showrelsymbol\subseteq{2286}{}
 \showrelsymbol\supseteq{2287}{}
 \showrelsymbol\nsubseteq{2288}{}
 \showrelsymbol\nsupseteq{2289}{}
 \showrelsymbol\subsetneq{228A}{}
 \showrelsymbol\varsubsetneq{228A}{*}
 \showrelsymbol\supsetneq{228B}{}
 \showrelsymbol\varsupsetneq{228B}{*}
 \showrelsymbol\sqsubset{228F}{}
 \showrelsymbol\sqsupset{2290}{}
 \showrelsymbol\sqsubseteq{2291}{}
 \showrelsymbol\sqsupseteq{2292}{}
 \showrelsymbol\vdash{22A2}{}
 \showrelsymbol\dashv{22A3}{}
 \showrelsymbol\assert{22A6}{}
 \showrelsymbol\models{22A7}{}
 \showrelsymbol\vDash{22A8}{}
 \showrelsymbol\Vdash{22A9}{}
 \showrelsymbol\Vvdash{22AA}{}
 \showrelsymbol\VDash{22AB}{}
 \showrelsymbol\nvdash{22AC}{}
 \showrelsymbol\nvDash{22AD}{}
 \showrelsymbol\nVdash{22AE}{}
 \showrelsymbol\nVDash{22AF}{}
 \showrelsymbol\prurel{22B0}{}
 \showrelsymbol\scurel{22B1}{}
 \showrelsymbol\vartriangleleft{22B2}{}
 \showrelsymbol\vartriangleright{22B3}{}
 \showrelsymbol\trianglelefteq{22B4}{}
 \showrelsymbol\trianglerighteq{22B5}{}
 \showrelsymbol\origof{22B6}{}
 \showrelsymbol\imageof{22B7}{}
 \showrelsymbol\multimap{22B8}{}
 \showrelsymbol\bowtie{22C8}{}
 \showrelsymbol\backsimeq{22CD}{}
 \showrelsymbol\Subset{22D0}{}
 \showrelsymbol\Supset{22D1}{}
 \showrelsymbol\pitchfork{22D4}{}
 \showrelsymbol\equalparallel{22D5}{}
 \showrelsymbol\lessdot{22D6}{}
 \showrelsymbol\gtrdot{22D7}{}
 \showrelsymbol\lll{22D8}{}, \cmd\llless
 \showrelsymbol\ggg{22D9}{}, \cmd\gggtr
 \showrelsymbol\lesseqgtr{22DA}{}
 \showrelsymbol\gtreqless{22DB}{}
 \showrelsymbol\eqless{22DC}{}
 \showrelsymbol\eqgtr{22DD}{}
 \showrelsymbol\curlyeqprec{22DE}{}
 \showrelsymbol\curlyeqsucc{22DF}{}
 \showrelsymbol\npreccurlyeq{22E0}{}
 \showrelsymbol\nsucccurlyeq{22E1}{}
 \showrelsymbol\nsqsubseteq{22E2}{}
 \showrelsymbol\nsqsupseteq{22E3}{}
 \showrelsymbol\sqsubsetneq{22E4}{*}
 \showrelsymbol\sqsupsetneq{22E5}{*}
 \showrelsymbol\lnsim{22E6}{}
 \showrelsymbol\gnsim{22E7}{}
 \showrelsymbol\precnsim{22E8}{}
 \showrelsymbol\succnsim{22E9}{}
 \showrelsymbol\nvartriangleleft{22EA}{}
 \showrelsymbol\nvartriangleright{22EB}{}
 \showrelsymbol\ntrianglelefteq{22EC}{}
 \showrelsymbol\ntrianglerighteq{22ED}{}
 \showrelsymbol\vdots{22EE}{}
 \showrelsymbol\adots{22F0}{}
 \showrelsymbol\ddots{22F1}{}
 \showrelsymbol\disin{22F2}{*}
 \showrelsymbol\varisins{22F3}{*}
 \showrelsymbol\isins{22F4}{*}
 \showrelsymbol\isindot{22F5}{*}
 \showrelsymbol\varisinobar{22F6}{}
 \showrelsymbol\isinobar{22F7}{*}
 \showrelsymbol\isinvb{22F8}{*}
 \showrelsymbol\isinE{22F9}{*}
 \showrelsymbol\nisd{22FA}{*}
 \showrelsymbol\varnis{22FB}{*}
 \showrelsymbol\nis{22FC}{*}
 \showrelsymbol\varniobar{22FD}{}
 \showrelsymbol\niobar{22FE}{*}
 \showrelsymbol\bagmember{22FF}{*}
 \showrelsymbol\frown{2322}{}
 \showrelsymbol\smallfrown{2322}{*}
 \showrelsymbol\smile{2323}{}
 \showrelsymbol\smallsmile{2323}{*}
 \showrelsymbol\APLnotslash{233F}{}
 \showrelsymbol\vartriangle{25B5}{*}
 \showrelsymbol\perp{27C2}{*}
 \showrelsymbol\bsolhsub{27C8}{}
 \showrelsymbol\suphsol{27C9}{}
 \showrelsymbol\upin{27D2}{*}
 \showrelsymbol\pullback{27D3}{*}
 \showrelsymbol\pushout{27D4}{*}
 \showrelsymbol\DashVDash{27DA}{*}
 \showrelsymbol\dashVdash{27DB}{*}
 \showrelsymbol\multimapinv{27DC}{*}
 \showrelsymbol\vlongdash{27DD}{*}
 \showrelsymbol\longdashv{27DE}{*}
 \showrelsymbol\cirbot{27DF}{*}
 \showrelsymbol\UUparrow{27F0}{*}
 \showrelsymbol\DDownarrow{27F1}{*}
 \showrelsymbol\acwgapcirclearrow{27F2}{*}
 \showrelsymbol\cwgapcirclearrow{27F3}{*}
 \showrelsymbol\rightarrowonoplus{27F4}{*}
 \showrelsymbol\longleftarrow{27F5}{*}
 \showrelsymbol\longrightarrow{27F6}{*}
 \showrelsymbol\longleftrightarrow{27F7}{*}
 \showrelsymbol\Longleftarrow{27F8}{*}
 \showrelsymbol\Longrightarrow{27F9}{*}
 \showrelsymbol\Longleftrightarrow{27FA}{*}
 \showrelsymbol\longmapsfrom{27FB}{*}
 \showrelsymbol\longmapsto{27FC}{*}
 \showrelsymbol\Longmapsfrom{27FD}{*}
 \showrelsymbol\Longmapsto{27FE}{*}
 \showrelsymbol\longrightsquigarrow{27FF}{*}
 \showrelsymbol\nvtwoheadrightarrow{2900}{*}
 \showrelsymbol\nVtwoheadrightarrow{2901}{*}
 \showrelsymbol\nvLeftarrow{2902}{*}
 \showrelsymbol\nvRightarrow{2903}{*}
 \showrelsymbol\nvLeftrightarrow{2904}{*}
 \showrelsymbol\twoheadmapsto{2905}{*}
 \showrelsymbol\Mapsfrom{2906}{*}
 \showrelsymbol\Mapsto{2907}{*}
 \showrelsymbol\downarrowbarred{2908}{*}
 \showrelsymbol\uparrowbarred{2909}{*}
 \showrelsymbol\Uuparrow{290A}{*}
 \showrelsymbol\Ddownarrow{290B}{*}
 \showrelsymbol\leftbkarrow{290C}{*}
 \showrelsymbol\rightbkarrow{290D}{*}
 \showrelsymbol\leftdbkarrow{290E}{*}, \cmd\dashleftarrow
 \showrelsymbol\dbkarow{290F}{*}, \cmd\dashrightarrow
 \showrelsymbol\drbkarow{2910}{*}
 \showrelsymbol\rightdotarrow{2911}{*}
 \showrelsymbol\baruparrow{2912}{*}
 \showrelsymbol\downarrowbar{2913}{*}
 \showrelsymbol\nvrightarrowtail{2914}{*}
 \showrelsymbol\nVrightarrowtail{2915}{*}
 \showrelsymbol\twoheadrightarrowtail{2916}{*}
 \showrelsymbol\nvtwoheadrightarrowtail{2917}{*}
 \showrelsymbol\nVtwoheadrightarrowtail{2918}{*}
 \showrelsymbol\lefttail{2919}{*}
 \showrelsymbol\righttail{291A}{*}
 \showrelsymbol\leftdbltail{291B}{*}
 \showrelsymbol\rightdbltail{291C}{*}
 \showrelsymbol\diamondleftarrow{291D}{*}
 \showrelsymbol\rightarrowdiamond{291E}{*}
 \showrelsymbol\diamondleftarrowbar{291F}{*}
 \showrelsymbol\barrightarrowdiamond{2920}{*}
 \showrelsymbol\nwsearrow{2921}{*}
 \showrelsymbol\neswarrow{2922}{*}
 \showrelsymbol\hknwarrow{2923}{*}
 \showrelsymbol\hknearrow{2924}{*}
 \showrelsymbol\hksearow{2925}{*}
 \showrelsymbol\hkswarow{2926}{*}
 \showrelsymbol\tona{2927}{*}
 \showrelsymbol\toea{2928}{*}
 \showrelsymbol\tosa{2929}{*}
 \showrelsymbol\towa{292A}{*}
 \showrelsymbol\rightcurvedarrow{2933}{*}
 \showrelsymbol\leftdowncurvedarrow{2936}{*}
 \showrelsymbol\rightdowncurvedarrow{2937}{*}
 \showrelsymbol\cwrightarcarrow{2938}{*}
 \showrelsymbol\acwleftarcarrow{2939}{*}
 \showrelsymbol\acwoverarcarrow{293A}{*}
 \showrelsymbol\acwunderarcarrow{293B}{*}
 \showrelsymbol\curvearrowrightminus{293C}{*}
 \showrelsymbol\curvearrowleftplus{293D}{*}
 \showrelsymbol\cwundercurvearrow{293E}{*}
 \showrelsymbol\ccwundercurvearrow{293F}{*}
 \showrelsymbol\acwcirclearrow{2940}{*}
 \showrelsymbol\cwcirclearrow{2941}{*}
 \showrelsymbol\rightarrowshortleftarrow{2942}{*}
 \showrelsymbol\leftarrowshortrightarrow{2943}{*}
 \showrelsymbol\shortrightarrowleftarrow{2944}{*}
 \showrelsymbol\rightarrowplus{2945}{*}
 \showrelsymbol\leftarrowplus{2946}{*}
 \showrelsymbol\rightarrowx{2947}{*}
 \showrelsymbol\leftrightarrowcircle{2948}{*}
 \showrelsymbol\twoheaduparrowcircle{2949}{*}
 \showrelsymbol\leftrightharpoonupdown{294A}{*}
 \showrelsymbol\leftrightharpoondownup{294B}{*}
 \showrelsymbol\updownharpoonrightleft{294C}{*}
 \showrelsymbol\updownharpoonleftright{294D}{*}
 \showrelsymbol\leftrightharpoonupup{294E}{*}
 \showrelsymbol\updownharpoonrightright{294F}{*}
 \showrelsymbol\leftrightharpoondowndown{2950}{*}
 \showrelsymbol\updownharpoonleftleft{2951}{*}
 \showrelsymbol\barleftharpoonup{2952}{*}
 \showrelsymbol\rightharpoonupbar{2953}{*}
 \showrelsymbol\barupharpoonright{2954}{*}
 \showrelsymbol\downharpoonrightbar{2955}{*}
 \showrelsymbol\barleftharpoondown{2956}{*}
 \showrelsymbol\rightharpoondownbar{2957}{*}
 \showrelsymbol\barupharpoonleft{2958}{*}
 \showrelsymbol\downharpoonleftbar{2959}{*}
 \showrelsymbol\leftharpoonupbar{295A}{*}
 \showrelsymbol\barrightharpoonup{295B}{*}
 \showrelsymbol\upharpoonrightbar{295C}{*}
 \showrelsymbol\bardownharpoonright{295D}{*}
 \showrelsymbol\leftharpoondownbar{295E}{*}
 \showrelsymbol\barrightharpoondown{295F}{*}
 \showrelsymbol\upharpoonleftbar{2960}{*}
 \showrelsymbol\bardownharpoonleft{2961}{*}
 \showrelsymbol\leftharpoonsupdown{2962}{*}
 \showrelsymbol\upharpoonsleftright{2963}{*}
 \showrelsymbol\rightharpoonsupdown{2964}{*}
 \showrelsymbol\downharpoonsleftright{2965}{*}
 \showrelsymbol\leftrightharpoonsup{2966}{*}
 \showrelsymbol\leftrightharpoonsdown{2967}{*}
 \showrelsymbol\rightleftharpoonsup{2968}{*}
 \showrelsymbol\rightleftharpoonsdown{2969}{*}
 \showrelsymbol\leftharpoonupdash{296A}{*}
 \showrelsymbol\dashleftharpoondown{296B}{*}
 \showrelsymbol\rightharpoonupdash{296C}{*}
 \showrelsymbol\dashrightharpoondown{296D}{*}
 \showrelsymbol\updownharpoonsleftright{296E}{*}
 \showrelsymbol\downupharpoonsleftright{296F}{*}
 \showrelsymbol\rightimply{2970}{*}
 \showrelsymbol\equalrightarrow{2971}{*}
 \showrelsymbol\similarrightarrow{2972}{*}
 \showrelsymbol\leftarrowsimilar{2973}{*}
 \showrelsymbol\rightarrowsimilar{2974}{*}
 \showrelsymbol\rightarrowapprox{2975}{*}
 \showrelsymbol\ltlarr{2976}{*}
 \showrelsymbol\leftarrowless{2977}{*}
 \showrelsymbol\gtrarr{2978}{*}
 \showrelsymbol\subrarr{2979}{*}
 \showrelsymbol\leftarrowsubset{297A}{*}
 \showrelsymbol\suplarr{297B}{*}
 \showrelsymbol\leftfishtail{297C}{*}
 \showrelsymbol\rightfishtail{297D}{*}
 \showrelsymbol\upfishtail{297E}{*}
 \showrelsymbol\downfishtail{297F}{*}
 \showrelsymbol\rtriltri{29CE}{*}
 \showrelsymbol\ltrivb{29CF}{*}
 \showrelsymbol\vbrtri{29D0}{*}
 \showrelsymbol\lfbowtie{29D1}{*}
 \showrelsymbol\rfbowtie{29D2}{*}
 \showrelsymbol\fbowtie{29D3}{*}
 \showrelsymbol\lftimes{29D4}{*}
 \showrelsymbol\rftimes{29D5}{*}
 \showrelsymbol\dualmap{29DF}{*}
 \showrelsymbol\lrtriangleeq{29E1}{*}
 \showrelsymbol\eparsl{29E3}{*}
 \showrelsymbol\smeparsl{29E4}{*}
 \showrelsymbol\eqvparsl{29E5}{*}
 \showrelsymbol\gleichstark{29E6}{*}
 \showrelsymbol\ruledelayed{29F4}{*}
 \showrelsymbol\veeonwedge{2A59}{*}
 \showrelsymbol\eqdot{2A66}{}
 \showrelsymbol\dotequiv{2A67}{}
 \showrelsymbol\equivVert{2A68}{*}
 \showrelsymbol\equivVvert{2A69}{*}
 \showrelsymbol\dotsim{2A6A}{}
 \showrelsymbol\simrdots{2A6B}{*}
 \showrelsymbol\simminussim{2A6C}{*}
 \showrelsymbol\congdot{2A6D}{}
 \showrelsymbol\asteq{2A6E}{}
 \showrelsymbol\hatapprox{2A6F}{}
 \showrelsymbol\approxeqq{2A70}{}
 \showrelsymbol\eqqsim{2A73}{}
 \showrelsymbol\Coloneq{2A74}{*}
 \showrelsymbol\eqeq{2A75}{*}
 \showrelsymbol\eqeqeq{2A76}{*}
 \showrelsymbol\ddotseq{2A77}{*}
 \showrelsymbol\equivDD{2A78}{*}
 \showrelsymbol\ltcir{2A79}{*}
 \showrelsymbol\gtcir{2A7A}{*}
 \showrelsymbol\ltquest{2A7B}{*}
 \showrelsymbol\gtquest{2A7C}{*}
 \showrelsymbol\leqslant{2A7D}{}
 \showrelsymbol\geqslant{2A7E}{}
 \showrelsymbol\lesdot{2A7F}{*}
 \showrelsymbol\gesdot{2A80}{*}
 \showrelsymbol\lesdoto{2A81}{*}
 \showrelsymbol\gesdoto{2A82}{*}
 \showrelsymbol\lesdotor{2A83}{*}
 \showrelsymbol\gesdotol{2A84}{*}
 \showrelsymbol\lessapprox{2A85}{*}
 \showrelsymbol\gtrapprox{2A86}{*}
 \showrelsymbol\lneq{2A87}{}
 \showrelsymbol\gneq{2A88}{}
 \showrelsymbol\lnapprox{2A89}{}
 \showrelsymbol\gnapprox{2A8A}{}
 \showrelsymbol\lesseqqgtr{2A8B}{*}
 \showrelsymbol\gtreqqless{2A8C}{*}
 \showrelsymbol\lsime{2A8D}{*}
 \showrelsymbol\gsime{2A8E}{*}
 \showrelsymbol\lsimg{2A8F}{*}
 \showrelsymbol\gsiml{2A90}{*}
 \showrelsymbol\lgE{2A91}{*}
 \showrelsymbol\glE{2A92}{*}
 \showrelsymbol\lesges{2A93}{*}
 \showrelsymbol\gesles{2A94}{*}
 \showrelsymbol\eqslantless{2A95}{}
 \showrelsymbol\eqslantgtr{2A96}{}
 \showrelsymbol\elsdot{2A97}{*}
 \showrelsymbol\egsdot{2A98}{*}
 \showrelsymbol\eqqless{2A99}{*}
 \showrelsymbol\eqqgtr{2A9A}{*}
 \showrelsymbol\eqqslantless{2A9B}{*}
 \showrelsymbol\eqqslantgtr{2A9C}{*}
 \showrelsymbol\simless{2A9D}{}
 \showrelsymbol\simgtr{2A9E}{}
 \showrelsymbol\simlE{2A9F}{*}
 \showrelsymbol\simgE{2AA0}{*}
 \showrelsymbol\Lt{2AA1}{*}
 \showrelsymbol\Gt{2AA2}{*}
 \showrelsymbol\partialmeetcontraction{2AA3}{*}
 \showrelsymbol\glj{2AA4}{*}
 \showrelsymbol\gla{2AA5}{*}
 \showrelsymbol\ltcc{2AA6}{*}
 \showrelsymbol\gtcc{2AA7}{*}
 \showrelsymbol\lescc{2AA8}{*}
 \showrelsymbol\gescc{2AA9}{*}
 \showrelsymbol\smt{2AAA}{*}
 \showrelsymbol\lat{2AAB}{*}
 \showrelsymbol\smte{2AAC}{*}
 \showrelsymbol\late{2AAD}{*}
 \showrelsymbol\bumpeqq{2AAE}{*}
 \showrelsymbol\preceq{2AAF}{}
 \showrelsymbol\npreceq{XXXX}{*}
 \showrelsymbol\succeq{2AB0}{}
 \showrelsymbol\nsucceq{XXXX}{*}
 \showrelsymbol\precneq{2AB1}{*}
 \showrelsymbol\succneq{2AB2}{*}
 \showrelsymbol\preceqq{2AB3}{*}
 \showrelsymbol\succeqq{2AB4}{*}
 \showrelsymbol\precneqq{2AB5}{*}
 \showrelsymbol\succneqq{2AB6}{*}
 \showrelsymbol\precapprox{2AB7}{*}
 \showrelsymbol\succapprox{2AB8}{*}
 \showrelsymbol\precnapprox{2AB9}{*}
 \showrelsymbol\succnapprox{2ABA}{*}
 \showrelsymbol\Prec{2ABB}{*}
 \showrelsymbol\Succ{2ABC}{*}
 \showrelsymbol\subsetdot{2ABD}{}
 \showrelsymbol\supsetdot{2ABE}{}
 \showrelsymbol\subsetplus{2ABF}{*}
 \showrelsymbol\supsetplus{2AC0}{*}
 \showrelsymbol\submult{2AC1}{*}
 \showrelsymbol\supmult{2AC2}{*}
 \showrelsymbol\subedot{2AC3}{*}
 \showrelsymbol\supedot{2AC4}{*}
 \showrelsymbol\subseteqq{2AC5}{}
 \showrelsymbol\nsubseteqq{XXXX}{*}
 \showrelsymbol\supseteqq{2AC6}{}
 \showrelsymbol\nsupseteqq{XXXX}{*}
 \showrelsymbol\subsim{2AC7}{*}
 \showrelsymbol\supsim{2AC8}{*}
 \showrelsymbol\subsetapprox{2AC9}{*}
 \showrelsymbol\supsetapprox{2ACA}{*}
 \showrelsymbol\subsetneqq{2ACB}{}
 \showrelsymbol\varsubsetneqq{2ACB}{*}
 \showrelsymbol\supsetneqq{2ACC}{}
 \showrelsymbol\varsupsetneqq{2ACC}{*}
 \showrelsymbol\lsqhook{2ACD}{}
 \showrelsymbol\rsqhook{2ACE}{}
 \showrelsymbol\csub{2ACF}{}
 \showrelsymbol\csup{2AD0}{}
 \showrelsymbol\csube{2AD1}{}
 \showrelsymbol\csupe{2AD2}{}
 \showrelsymbol\subsup{2AD3}{}
 \showrelsymbol\supsub{2AD4}{}
 \showrelsymbol\subsub{2AD5}{}
 \showrelsymbol\supsup{2AD6}{}
 \showrelsymbol\suphsub{2AD7}{}
 \showrelsymbol\supdsub{2AD8}{}
 \showrelsymbol\forkv{2AD9}{}
 \showrelsymbol\topfork{2ADA}{}
 \showrelsymbol\mlcp{2ADB}{}
 \showrelsymbol\forks{2ADC}{}
 \showrelsymbol\forksnot{2ADD}{}
 \showrelsymbol\shortlefttack{2ADE}{}
 \showrelsymbol\shortdowntack{2ADF}{}
 \showrelsymbol\shortuptack{2AE0}{}
 \showrelsymbol\vDdash{2AE2}{}
 \showrelsymbol\dashV{2AE3}{}
 \showrelsymbol\Dashv{2AE4}{}
 \showrelsymbol\DashV{2AE5}{}
 \showrelsymbol\varVdash{2AE6}{}
 \showrelsymbol\Barv{2AE7}{}
 \showrelsymbol\vBar{2AE8}{}
 \showrelsymbol\vBarv{2AE9}{}
 \showrelsymbol\barV{2AEA}{}
 \showrelsymbol\Vbar{2AEB}{}
 \showrelsymbol\Not{2AEC}{}
 \showrelsymbol\bNot{2AED}{}
 \showrelsymbol\revnmid{2AEE}{}
 \showrelsymbol\cirmid{2AEF}{}
 \showrelsymbol\midcir{2AF0}{}
 \showrelsymbol\nhpar{2AF2}{}
 \showrelsymbol\parsim{2AF3}{}
 \showrelsymbol\lllnest{2AF7}{}
 \showrelsymbol\gggnest{2AF8}{}
 \showrelsymbol\leqqslant{2AF9}{}
 \showrelsymbol\geqqslant{2AFA}{}
 \showrelsymbol\circleonleftarrow{2B30}{*}
 \showrelsymbol\leftthreearrows{2B31}{*}
 \showrelsymbol\leftarrowonoplus{2B32}{*}
 \showrelsymbol\longleftsquigarrow{2B33}{*}
 \showrelsymbol\nvtwoheadleftarrow{2B34}{*}
 \showrelsymbol\nVtwoheadleftarrow{2B35}{*}
 \showrelsymbol\twoheadmapsfrom{2B36}{*}
 \showrelsymbol\twoheadleftdbkarrow{2B37}{*}
 \showrelsymbol\leftdotarrow{2B38}{*}
 \showrelsymbol\nvleftarrowtail{2B39}{*}
 \showrelsymbol\nVleftarrowtail{2B3A}{*}
 \showrelsymbol\twoheadleftarrowtail{2B3B}{*}
 \showrelsymbol\nvtwoheadleftarrowtail{2B3C}{*}
 \showrelsymbol\nVtwoheadleftarrowtail{2B3D}{*}
 \showrelsymbol\leftarrowx{2B3E}{*}
 \showrelsymbol\leftcurvedarrow{2B3F}{*}
 \showrelsymbol\equalleftarrow{2B40}{*}
 \showrelsymbol\bsimilarleftarrow{2B41}{*}
 \showrelsymbol\leftarrowbackapprox{2B42}{*}
 \showrelsymbol\rightarrowgtr{2B43}{*}
 \showrelsymbol\rightarrowsupset{2B44}{*}
 \showrelsymbol\LLeftarrow{2B45}{*}
 \showrelsymbol\RRightarrow{2B46}{*}
 \showrelsymbol\bsimilarrightarrow{2B47}{*}
 \showrelsymbol\rightarrowbackapprox{2B48}{*}
 \showrelsymbol\similarleftarrow{2B49}{*}
 \showrelsymbol\leftarrowapprox{2B4A}{*}
 \showrelsymbol\leftarrowbsimilar{2B4B}{*}
 \showrelsymbol\rightarrowbsimilar{2B4C}{*}
 \showrelsymbol\ngeqq{XXXX}{}
 \showrelsymbol\ngeqslant{XXXX}{}
 \showrelsymbol\nleqslant{XXXX}{}
 \showrelsymbol\nleqq{XXXX}{}
% \showrelsymbol\ncongdot{XXXX}{}
% \showrelsymbol\napproxeqq{XXXX}{}
% \showrelsymbol\nll{XXXX}{}
% \showrelsymbol\ngg{XXXX}{}
% \showrelsymbol\nsqsubset{XXXX}{}
% \showrelsymbol\nsqsupset{XXXX}{}
% \showrelsymbol\nBumpeq{XXXX}{}
% \showrelsymbol\nbumpeq{XXXX}{}
%\showrelsymbol\neqsim{XXXX}{}
 %\showrelsymbol\nvarisinobar{XXXX}{}
% \showrelsymbol\nvarniobar{XXXX}{}
% \showrelsymbol\neqslantless{XXXX}{}
 %\showrelsymbol\neqslantgtr{XXXX}{}
 \showrelsymbol\lhook{XXXX}{}
 \showrelsymbol\rhook{XXXX}{}
 \showrelsymbol\relbar{XXXX}{}
 \showrelsymbol\Relbar{XXXX}{}
 %\showrelsymbol\Rrelbar{XXXX}{*}
% \showrelsymbol\RRelbar{XXXX}{*}
 \showrelsymbol\mapsfromchar{XXXX}{}
 \showrelsymbol\mapstochar{XXXX}{}
 \end{multicols}
\input{mathintegrals}


%%%%%%%%%%%%%%%%%%%%%%%%%%%%%%%
 \subsection{Ordinary symbols}
 \begin{multicols}{2}
 %\showsymbol\#{0023}{}
 %\showsymbol\mathdollar{0024}{}
 % \showsymbol\%{0025}{}
 %\showsymbol\&{0026}{}
% \showsymbol.{002E}{}
% \showsymbol/{002F}{}
% \showsymbol?{003F}{}
% \showsymbol@{0040}{}
 \showsymbol\backslash{005C}{}
 \showsymbol\mathsterling{00A3}{}
 \showsymbol\mathsection{00A7}{}
 \showsymbol\neg{00AC}{}, \cmd\lnot
 \showsymbol\mathparagraph{00B6}{}
 \showsymbol\eth{00F0}{}
 \showsymbol\Zbar{01B5}{*}
 \showsymbol\digamma{03DD}{}
 \showsymbol\varkappa{03F0}{}
 \showsymbol\backepsilon{03F6}{}
 \showsymbol\upbackepsilon{03F6}{}
 \showsymbol\enleadertwodots{2025}{}
 \showsymbol\mathellipsis{2026}{}
 \showsymbol\prime{2032}{}
 \showsymbol\dprime{2033}{}
 \showsymbol\trprime{2034}{}
 \showsymbol\backprime{2035}{}
 \showsymbol\backdprime{2036}{}
 \showsymbol\backtrprime{2037}{}
 \showsymbol\caretinsert{2038}{}
 \showsymbol\Exclam{203C}{}
 \showsymbol\hyphenbullet{2043}{*}
 \showsymbol\Question{2047}{}
 \showsymbol\qprime{2057}{}
 \showsymbol\enclosecircle{20DD}{}\indexmathcmd[Circles]{\enclosecircle}
 \showsymbol\enclosesquare{20DE}{*}
 \showsymbol\enclosediamond{20DF}{*}
 \showsymbol\enclosetriangle{20E4}{}
 \showsymbol\Eulerconst{2107}{}
 \showsymbol\hbar{210F}{*}
 \showsymbol\hslash{210F}{}
 \showsymbol\Im{2111}{}
 \showsymbol\ell{2113}{}
 \showsymbol\wp{2118}{}
 \showsymbol\Re{211C}{}
 \showsymbol\mho{2127}{}
 \showsymbol\turnediota{2129}{}
 \showsymbol\Angstrom{212B}{}
 \showsymbol\Finv{2132}{}
 \showsymbol\aleph{2135}{}
 \showsymbol\beth{2136}{}
 \showsymbol\gimel{2137}{}
 \showsymbol\daleth{2138}{}
 \showsymbol\Game{2141}{*}
 \showsymbol\sansLturned{2142}{*}
 \showsymbol\sansLmirrored{2143}{*}
 \showsymbol\Yup{2144}{*}
 \showsymbol\PropertyLine{214A}{*}
 \showsymbol\updownarrowbar{21A8}{}
 \showsymbol\linefeed{21B4}{}
 \showsymbol\carriagereturn{21B5}{}
 \showsymbol\barovernorthwestarrow{21B8}{}
 \showsymbol\barleftarrowrightarrowbar{21B9}{}
 \showsymbol\acwopencirclearrow{21BA}{}\indexmathcmd[Circles]{\acwopencirclearrow}
 \showsymbol\cwopencirclearrow{21BB}{}\indexmathcmd[Circles]{\cwopencirclearrow}
 \showsymbol\nHuparrow{21DE}{*}
 \showsymbol\nHdownarrow{21DF}{*}
 \showsymbol\leftdasharrow{21E0}{*}
 \showsymbol\updasharrow{21E1}{*}
 \showsymbol\rightdasharrow{21E2}{*}
 \showsymbol\downdasharrow{21E3}{*}
 \showsymbol\leftwhitearrow{21E6}{}
 \showsymbol\upwhitearrow{21E7}{}
 \showsymbol\rightwhitearrow{21E8}{}
 \showsymbol\downwhitearrow{21E9}{}
 \showsymbol\whitearrowupfrombar{21EA}{}
 \showsymbol\forall{2200}{}
 \showsymbol\complement{2201}{}
 \showsymbol\exists{2203}{}
 \showsymbol\nexists{2204}{}
 \showsymbol\varnothing{2205}{}
 \showsymbol\emptyset{2205}{}
 \showsymbol\increment{2206}{}
 \showsymbol\QED{220E}{*}
 \showsymbol\infty{221E}{}
 \showsymbol\rightangle{221F}{}
 \showsymbol\angle{2220}{}
 \showsymbol\measuredangle{2221}{}
 \showsymbol\sphericalangle{2222}{}
 \showsymbol\therefore{2234}{}
 \showsymbol\because{2235}{}
 \showsymbol\sinewave{223F}{}
 \showsymbol\top{22A4}{}
 \showsymbol\bot{22A5}{}
 \showsymbol\hermitmatrix{22B9}{}
 \showsymbol\measuredrightangle{22BE}{}
 \showsymbol\varlrtriangle{22BF}{}
 %\showsymbol\cdots{22EF}{} % TO FIX
 \showsymbol\diameter{2300}{*}
 \showsymbol\house{2302}{}
 \showsymbol\invnot{2310}{}
 \showsymbol\sqlozenge{2311}{*}
 \showsymbol\profline{2312}{*}
 \showsymbol\profsurf{2313}{*}
 \showsymbol\viewdata{2317}{*}
 \showsymbol\turnednot{2319}{}
 \showsymbol\varhexagonlrbonds{232C}{*}
 \showsymbol\conictaper{2332}{*}
 \showsymbol\topbot{2336}{}
 \showsymbol\APLnotbackslash{2340}{*}
 \showsymbol\APLboxupcaret{2353}{*}
 \showsymbol\APLboxquestion{2370}{*}
 \showsymbol\rangledownzigzagarrow{237C}{*}
 \showsymbol\hexagon{2394}{*}
 \showsymbol\bbrktbrk{23B6}{}
 \showsymbol\varcarriagereturn{23CE}{*}
 \showsymbol\obrbrak{23E0}{}
 \showsymbol\ubrbrak{23E1}{}
 \showsymbol\trapezium{23E2}{*}
 \showsymbol\benzenr{23E3}{*}
 \showsymbol\strns{23E4}{*}
 \showsymbol\fltns{23E5}{*}
 \showsymbol\accurrent{23E6}{*}
 \showsymbol\elinters{23E7}{*}
 \showsymbol\mathvisiblespace{2423}{}
 \showsymbol\circledR{24C7}{}
 \showsymbol\circledS{24C8}{}
 \showsymbol\mdlgblksquare{25A0}{*}, \cmd\blacksquare
 \showsymbol\mdlgwhtsquare{25A1}{*}, \cmd\square, \cmd\Box
 \showsymbol\squoval{25A2}{*}
 \showsymbol\blackinwhitesquare{25A3}{*}
 \showsymbol\squarehfill{25A4}{*}
 \showsymbol\squarevfill{25A5}{*}
 \showsymbol\squarehvfill{25A6}{*}
 \showsymbol\squarenwsefill{25A7}{*}
 \showsymbol\squareneswfill{25A8}{*}
 \showsymbol\squarecrossfill{25A9}{*}
 \showsymbol\smblksquare{25AA}{*}
 \showsymbol\smwhtsquare{25AB}{*}
 \showsymbol\hrectangleblack{25AC}{*}
 \showsymbol\hrectangle{25AD}{*}
 \showsymbol\vrectangleblack{25AE}{*}
 \showsymbol\vrectangle{25AF}{*}
 \showsymbol\parallelogramblack{25B0}{*}
 \showsymbol\parallelogram{25B1}{*}
 \showsymbol\bigblacktriangleup{25B2}{*}
 \showsymbol\blacktriangle{25B4}{*}
 \showsymbol\blacktriangleright{25B6}{*}
 \showsymbol\smallblacktriangleright{25B8}{*}
 \showsymbol\smalltriangleright{25B9}{*}
 \showsymbol\blackpointerright{25BA}{*}
 \showsymbol\whitepointerright{25BB}{*}
 \showsymbol\bigblacktriangledown{25BC}{*}
 \showsymbol\bigtriangledown{25BD}{}
 \showsymbol\blacktriangledown{25BE}{*}
 \showsymbol\triangledown{25BF}{*}
 \showsymbol\blacktriangleleft{25C0}{*}
 \showsymbol\smallblacktriangleleft{25C2}{*}
 \showsymbol\smalltriangleleft{25C3}{*}
 \showsymbol\blackpointerleft{25C4}{*}
 \showsymbol\whitepointerleft{25C5}{*}
 \showsymbol\mdlgblkdiamond{25C6}{*}
 \showsymbol\mdlgwhtdiamond{25C7}{*}
 \showsymbol\blackinwhitediamond{25C8}{*}
 \showsymbol\fisheye{25C9}{*}
 \showsymbol\mdlgwhtlozenge{25CA}{}, \cmd\lozenge, \\ \cmd\Diamond
 \showsymbol\dottedcircle{25CC}{*}
 \showsymbol\circlevertfill{25CD}{*}
 \showsymbol\bullseye{25CE}{*}
 \showsymbol\mdlgblkcircle{25CF}{*}
 \showsymbol\circlelefthalfblack{25D0}{*}
 \showsymbol\circlerighthalfblack{25D1}{*}
 \showsymbol\circlebottomhalfblack{25D2}{*}
 \showsymbol\circletophalfblack{25D3}{*}
 \showsymbol\circleurquadblack{25D4}{*}
 \showsymbol\blackcircleulquadwhite{25D5}{*}
 \showsymbol\blacklefthalfcircle{25D6}{*}
 \showsymbol\blackrighthalfcircle{25D7}{*}
 \showsymbol\inversebullet{25D8}{*}
 \showsymbol\inversewhitecircle{25D9}{*}
 \showsymbol\invwhiteupperhalfcircle{25DA}{*}
 \showsymbol\invwhitelowerhalfcircle{25DB}{*}
 \showsymbol\ularc{25DC}{*}
 \showsymbol\urarc{25DD}{*}
 \showsymbol\lrarc{25DE}{*}
 \showsymbol\llarc{25DF}{*}
 \showsymbol\topsemicircle{25E0}{*}
 \showsymbol\botsemicircle{25E1}{*}
 \showsymbol\lrblacktriangle{25E2}{*}
 \showsymbol\llblacktriangle{25E3}{*}
 \showsymbol\ulblacktriangle{25E4}{*}
 \showsymbol\urblacktriangle{25E5}{*}
 \showsymbol\circ{25E6}{}, \cmd\smwhtcircle
 \showsymbol\squareleftblack{25E7}{*}
 \showsymbol\squarerightblack{25E8}{*}
 \showsymbol\squareulblack{25E9}{*}
 \showsymbol\squarelrblack{25EA}{*}
 \showsymbol\trianglecdot{25EC}{}
 \showsymbol\triangleleftblack{25ED}{*}
 \showsymbol\trianglerightblack{25EE}{*}
 \showsymbol\lgwhtcircle{25EF}{*}
 \showsymbol\squareulquad{25F0}{*}
 \showsymbol\squarellquad{25F1}{*}
 \showsymbol\squarelrquad{25F2}{*}
 \showsymbol\squareurquad{25F3}{*}
 \showsymbol\circleulquad{25F4}{*}
 \showsymbol\circlellquad{25F5}{*}
 \showsymbol\circlelrquad{25F6}{*}
 \showsymbol\circleurquad{25F7}{*}
 \showsymbol\ultriangle{25F8}{*}
 \showsymbol\urtriangle{25F9}{*}
 \showsymbol\lltriangle{25FA}{*}
 \showsymbol\mdwhtsquare{25FB}{*}
 \showsymbol\mdblksquare{25FC}{*}
 \showsymbol\mdsmwhtsquare{25FD}{*}
 \showsymbol\mdsmblksquare{25FE}{*}
 \showsymbol\lrtriangle{25FF}{*}
 \showsymbol\bigstar{2605}{*}
 \showsymbol\bigwhitestar{2606}{*}
 \showsymbol\astrosun{2609}{}
 \showsymbol\danger{2621}{}
 \showsymbol\blacksmiley{263B}{}
 \showsymbol\sun{263C}{}
 \showsymbol\rightmoon{263D}{}
 \showsymbol\leftmoon{263E}{}
 \showsymbol\female{2640}{}
 \showsymbol\male{2642}{}
 \showsymbol\spadesuit{2660}{*}
 \showsymbol\heartsuit{2661}{*}
 \showsymbol\diamondsuit{2662}{*}
 \showsymbol\clubsuit{2663}{*}
 \showsymbol\varspadesuit{2664}{}
 \showsymbol\varheartsuit{2665}{}
 \showsymbol\vardiamondsuit{2666}{}
 \showsymbol\varclubsuit{2667}{}
 \showsymbol\quarternote{2669}{}
 \showsymbol\eighthnote{266A}{}
 \showsymbol\twonotes{266B}{}
 \showsymbol\flat{266D}{}
 \showsymbol\natural{266E}{}
 \showsymbol\sharp{266F}{}
 \showsymbol\acidfree{267E}{*}
 \showsymbol\dicei{2680}{}
 \showsymbol\diceii{2681}{}
 \showsymbol\diceiii{2682}{}
 \showsymbol\diceiv{2683}{}
 \showsymbol\dicev{2684}{}
 \showsymbol\dicevi{2685}{}
 \showsymbol\circledrightdot{2686}{}
 \showsymbol\circledtwodots{2687}{}
 \showsymbol\blackcircledrightdot{2688}{}
 \showsymbol\blackcircledtwodots{2689}{}
 \showsymbol\Hermaphrodite{26A5}{}
 \showsymbol\mdwhtcircle{26AA}{}
 \showsymbol\mdblkcircle{26AB}{}
 \showsymbol\mdsmwhtcircle{26AC}{}
 \showsymbol\neuter{26B2}{}
 \showsymbol\checkmark{2713}{}
 \showsymbol\maltese{2720}{}
 \showsymbol\circledstar{272A}{}
 \showsymbol\varstar{2736}{}
 \showsymbol\dingasterisk{273D}{}
 \showsymbol\draftingarrow{279B}{*}
 \showsymbol\threedangle{27C0}{*}
 \showsymbol\whiteinwhitetriangle{27C1}{*}
 \showsymbol\subsetcirc{27C3}{*}
 \showsymbol\supsetcirc{27C4}{*}
 \showsymbol\diagup{27CB}{*}
 \showsymbol\diagdown{27CD}{*}
 \showsymbol\diamondcdot{27D0}{*}
 \showsymbol\rdiagovfdiag{292B}{*}
 \showsymbol\fdiagovrdiag{292C}{*}
 \showsymbol\seovnearrow{292D}{*}
 \showsymbol\neovsearrow{292E}{*}
 \showsymbol\fdiagovnearrow{292F}{*}
 \showsymbol\rdiagovsearrow{2930}{*}
 \showsymbol\neovnwarrow{2931}{*}
 \showsymbol\nwovnearrow{2932}{*}
 \showsymbol\uprightcurvearrow{2934}{*}
 \showsymbol\downrightcurvedarrow{2935}{*}
 \showsymbol\mdsmblkcircle{2981}{*}
 \showsymbol\fourvdots{2999}{*}
 \showsymbol\vzigzag{299A}{*}
 \showsymbol\measuredangleleft{299B}{*}
 \showsymbol\rightanglesqr{299C}{*}
 \showsymbol\rightanglemdot{299D}{*}
 \showsymbol\angles{299E}{*}
 \showsymbol\angdnr{299F}{*}
 \showsymbol\gtlpar{29A0}{*}
 \showsymbol\sphericalangleup{29A1}{*}
 \showsymbol\turnangle{29A2}{*}
 \showsymbol\revangle{29A3}{*}
 \showsymbol\angleubar{29A4}{*}
 \showsymbol\revangleubar{29A5}{*}
 \showsymbol\wideangledown{29A6}{*}
 \showsymbol\wideangleup{29A7}{*}
 \showsymbol\measanglerutone{29A8}{*}
 \showsymbol\measanglelutonw{29A9}{*}
 \showsymbol\measanglerdtose{29AA}{*}
 \showsymbol\measangleldtosw{29AB}{*}
 \showsymbol\measangleurtone{29AC}{*}
 \showsymbol\measangleultonw{29AD}{*}
 \showsymbol\measangledrtose{29AE}{*}
 \showsymbol\measangledltosw{29AF}{*}
 \showsymbol\revemptyset{29B0}{*}
 \showsymbol\emptysetobar{29B1}{*}
 \showsymbol\emptysetocirc{29B2}{*}
 \showsymbol\emptysetoarr{29B3}{*}
 \showsymbol\emptysetoarrl{29B4}{*}
 \showsymbol\obot{29BA}{*}
 \showsymbol\olcross{29BB}{*}
 \showsymbol\odotslashdot{29BC}{*}
 \showsymbol\uparrowoncircle{29BD}{*}
 \showsymbol\circledwhitebullet{29BE}{*}
 \showsymbol\circledbullet{29BF}{*}
 \showsymbol\cirscir{29C2}{*}
 \showsymbol\cirE{29C3}{*}
 \showsymbol\boxonbox{29C9}{*}
 \showsymbol\triangleodot{29CA}{*}
 \showsymbol\triangleubar{29CB}{*}
 \showsymbol\triangles{29CC}{*}
 \showsymbol\iinfin{29DC}{*}
 \showsymbol\tieinfty{29DD}{*}
 \showsymbol\nvinfty{29DE}{*}
 \showsymbol\laplac{29E0}{*}
 \showsymbol\thermod{29E7}{*}
 \showsymbol\downtriangleleftblack{29E8}{*}
 \showsymbol\downtrianglerightblack{29E9}{*}
 \showsymbol\blackdiamonddownarrow{29EA}{*}
 \showsymbol\blacklozenge{29EB}{}
 \showsymbol\circledownarrow{29EC}{*}
 \showsymbol\blackcircledownarrow{29ED}{*}
 \showsymbol\errbarsquare{29EE}{*}
 \showsymbol\errbarblacksquare{29EF}{*}
 \showsymbol\errbardiamond{29F0}{*}
 \showsymbol\errbarblackdiamond{29F1}{*}
 \showsymbol\errbarcircle{29F2}{*}
 \showsymbol\errbarblackcircle{29F3}{*}
 \showsymbol\perps{2AE1}{}
 \showsymbol\topcir{2AF1}{}
 \showsymbol\squaretopblack{2B12}{}
 \showsymbol\squarebotblack{2B13}{}
 \showsymbol\squareurblack{2B14}{}
 \showsymbol\squarellblack{2B15}{}
 \showsymbol\diamondleftblack{2B16}{}
 \showsymbol\diamondrightblack{2B17}{}
 \showsymbol\diamondtopblack{2B18}{}
 \showsymbol\diamondbotblack{2B19}{}
 \showsymbol\dottedsquare{2B1A}{}
 \showsymbol\lgblksquare{2B1B}{}
 \showsymbol\lgwhtsquare{2B1C}{}
 \showsymbol\vysmblksquare{2B1D}{}
 \showsymbol\vysmwhtsquare{2B1E}{}
 \showsymbol\pentagonblack{2B1F}{}
 \showsymbol\pentagon{2B20}{}
 \showsymbol\varhexagon{2B21}{}
 \showsymbol\varhexagonblack{2B22}{}
 \showsymbol\hexagonblack{2B23}{}
 \showsymbol\lgblkcircle{2B24}{}
 \showsymbol\mdblkdiamond{2B25}{}
 \showsymbol\mdwhtdiamond{2B26}{}
 \showsymbol\mdblklozenge{2B27}{}
 \showsymbol\mdwhtlozenge{2B28}{}
 \showsymbol\smblkdiamond{2B29}{}
 \showsymbol\smblklozenge{2B2A}{}
 \showsymbol\smwhtlozenge{2B2B}{}
 \showsymbol\blkhorzoval{2B2C}{}
 \showsymbol\whthorzoval{2B2D}{}
 \showsymbol\blkvertoval{2B2E}{}
 \showsymbol\whtvertoval{2B2F}{}
 \showsymbol\medwhitestar{2B50}{}
 \showsymbol\medblackstar{2B51}{}
 \showsymbol\smwhitestar{2B52}{}
 \showsymbol\rightpentagonblack{2B53}{}
 \showsymbol\rightpentagon{2B54}{}
 \showsymbol\postalmark{3012}{}
 \showsymbol\hzigzag{3030}{}
 \showsymbol\Bbbk{1D55C}{}
 \showsymbol\bracevert{XXXX}{*}
 \end{multicols}

\input{mathsbinops}





\DocInput{\jobname.dtx}
\nocite{*}
\printbibliography

\printindex 
 %
% 
\def\contentsname{Contents}
\end{document}
%</driver>
% \fi
% 
%  \CheckSum{0}
%  \CharacterTable
%  {Upper-case    \A\B\C\D\E\F\G\H\I\J\K\L\M\N\O\P\Q\R\S\T\U\V\W\X\Y\Z
%   Lower-case    \a\b\c\d\e\f\g\h\i\j\k\l\m\n\o\p\q\r\s\t\u\v\w\x\y\z
%   Digits        \0\1\2\3\4\5\6\7\8\9
%   Exclamation   \!     Double quote  \"     Hash (number) \#
%   Dollar        \$     Percent       \%     Ampersand     \&
%   Acute accent  \'     Left paren    \(     Right paren   \)
%   Asterisk      \*     Plus          \+     Comma         \,
%   Minus         \-     Point         \.     Solidus       \/
%   Colon         \:     Semicolon     \;     Less than     \<
%   Equals        \=     Greater than  \>     Question mark \?
%   Commercial at \@     Left bracket  \[     Backslash     \\
%   Right bracket \]     Circumflex    \^     Underscore    \_
%   Grave accent  \`     Left brace    \{     Vertical bar  \|
%   Right brace   \}     Tilde         \~}
%
%
%
% \changes{1.0}{2013/01/26}{Converted to DTX file}
%
% \DoNotIndex{\newcommand,\newenvironment}
% \GetFileInfo{phd.dtx}
% 
%  \def\fileversion{v1.0}          
%  \def \filedate{2012/03/06}
% \title{The \textsf{phd} package.
% \thanks{This
%        file (\texttt{phd.dtx}) has version number \fileversion, last revised
%        \filedate.}
% }
% \author{Dr. Yiannis Lazarides \\ \url{yannislaz@gmail.com}}
% \date{\filedate}
%
%
% 
% ^^A\maketitle
% 
% ^^A\frontmatter
%  ^^A\coverpage{./images/hine02.jpg}{Book Design }{Camel Press}{}{}
%  \newpage
% ^^A\secondpage
% \pagestyle{empty}
%
%
% 
%
%
% \pagestyle{headings}
% \raggedbottom
%  \OnlyDescription
%
%  ^^A\StopEventually{\printindex}

% \CodelineNumbered
% \pagestyle{headings}
% 
% 
% ^^A\part{IMPLEMENTATION AND FRIENDS}
% 
%
% \chapter{Code Implementation Objectives and Strategy}
% 
% \epigraph{
% ``Lord Campbell [John Campbell, 1st Baron Campbell, 1779-1861] proposed that any author who published a book without an index should be deprived of the benefits of the Copyright Act; and the Hon. Horace Binney, LL.D. [1780-1875], a distinguished American lawyer, held the same views, and would have condemned the culprit to the same punishment.''}
%{-- H[enry] B[enjamin] Wheatley, How To Make An Index
%(New York: A. C. Armstrong \& Son, 1902), 82}
%
% This package aims at providing authors with a set of tools and settings
% that can improve the typesetting of documentation and especially indices.
% 
% For the normal author there are both mark-up related macros, as well as a set
% of settings for indices.

% My main motivation for developing this package was to group all the special
% documentation macros that I have used in developing the phd package. I also saw
% the need to hook up these settings with the concept of color palettes as 
% descdribed in the phd-colorpalettes package. This enables the integration
% of full document templates.
%
% \begin{enumerate}
% \item To provide a declarative interface to enable users to modify index
%       entries by setting keys, rather than writing macros.  
%
% \item To provide a number of templates that cover most of the typical use case.
% \item To provide a plug-in architecture for extensions.

% \end{enumerate}
% 
% \section{Terminology}
%
%  \begin{description}
%  \item [document] Any written item, as a book, article, or letter, especially 
%                  of a factual or informative nature.
%  \item [heading] A division of a document or document series. For a normal
%        book headings are chapters, sections etc. However we allow for
%        specifying a more complex document divided into books, volumes
%        parts etc. For example the Bible has Books, chapters and verses,
%        where a legal document might require divisions such as clauses.
%        In general these divisions are numbered. These document divisions
%        are stored in the comma list \refCom{phd_book_divisions_clist}.
%  \item [head] A typeset heading, such as chapter head, or section head.
%        This can include a counter, label and title for example, 
%        \emph{Chapter 1 Introduction}.
%  \item [dom] This is a programming interface that provides a structured
%        representation of the document (a tree) and it defines a way
%        that the structure can be accessed. Although \latexe does not
%        offer a standard way to build such a tree (mainly because
%        \tex does not require the marking of paragraphs, it is 
%        useful to think of the document as a tree structure. We also
%        allow for a semi-automated way to build such a tree (with the 
%        exception that paragraphs are not included).
% \item [element] A part of the document tree that can be styled on
%       its own. For example the chapter label, or the section number.
%
% \end{description}
%
% \section{Users}
%  We classify users according to the \LaTeX3 terminology as a) programmers b) template designers
%  and c) authors.
% \subsection{Author}
%  We assume that the author has an exising template which she is using but might want to do
%  some minor modifications, for example use an italic shape for the font of the mark, but an 
%  upright font for the page numbers. 
%
%
% We follow the idea of representing the basic elements of documents
% as elements, each one having a parent in order to specify
% the element we need to style as accurate as possible. One can think of
% this approach being congruent with objects in other languages.
% As a matter fact nothing stops us from defining a key value
% interface as shown below.
%
% {\obeylines 
%~~ |\cxset|
%~~~~~|{| 
%~~~~~~~~\textit{header.even.mark.font.size}   = |Large,|
%~~~~~~~~\textit{header.even.mark.font.family} = |serif,|
%~~~~~|}|
%}  
%
% This would pehaps make it easier for the template designer, but I have rejected
% the idea as my aim is to make it easy for the author, who can search the template
% and just enter a couple of new proerty values.
%
% \subsection{Template designer}
% \pagestyle{headings}
% The template designer in the example above would have selected the format style
% from a number of predefined formats (templates) or would have created a style
% called \textit{apa} from an existing template and modified it using declarative
% key style.
%
% \subsection{The programmer}
%
% The programmer in the example above could have created the basic format
% \textit{apa} by using both declarative as well as defining or using existing
% macros. To the programmer we offer an extension mechanism, where the contents
% of a |ps@| command are defined. For example the programmer can define a new
% style using \tikzname, but without having to worry about defining full |ps@|
% and their interface.
%
% \section{Preliminaries}
%
%  Standard file identification. We first announce the package 
%	 and require that it be used with \LaTeX2e. 
%
% \section{Acknowledgements}
%
% This package couldn't have been possible if it was not for the documentation
% section of tcolorbox. I have liberally taken ideas and code from Dr. Thomas F. Sturm's 
% package, which in turn draws strongly from the |PGF| manual. I am grateful to both.
% 
% \iffalse
%<*DOCUM>
% \fi
%  
% \section{Preliminaries}
%
% We declare that we use \latexe and name the package. The code has been
% moved to the latest version of \latexe to take advantage of more allocations.
% Package works well with LuaLaTeX.\tcbdocmarginnote{7/8/2017}
%
%    \begin{macrocode}
\NeedsTeXFormat{LaTeX2e}[2017/04/15]%
\RequirePackage[2017/04/15]{latexrelease}
\ProvidesFile{phd-documentation}[2017/04/15 v1.0 less preamble (YL)]%
%    \end{macrocode}
%
% We also need the \pkg{refcount}. 
%
% \section{tcolorbox}
% We load \pkg{tcolorbox} with options theorems, skins, documentation etc
% for internal and external use.
%
% We also provide an interface, between the \pkg{tcolorbox} documentation
% keys and our own.
% 
% The indexing keys are still to be sorted out with other sections of the
% documentation, but they seem to be working for the moment.
% 
%    \begin{macrocode}
\RequirePackage[theorems, skins, documentation,
                breakable,listings]{tcolorbox}
%                
                \tcbset{index format=pgfchapter,
                        index actual={=},
                        index level = {>},
                        index quote = {!},
                        index german settings,
                        color hyperlink = thelinkcolor ,
                        color definition =thelinkcolor,
                   }
\tcbset{halostyle/.style={fuzzy halo=2mm with magenta!5}}                   

           
\def\tcb@Print@Com#1{\textcolor{\kvtcb@col@command}{\ttbf\char`\\ \detokenize{#1}}}
%    \end{macrocode}                
%
\cxset {doc command color/.code = \tcbset{color command = #1}}
\cxset {doc command color= thecmdcolor}
%
%    \begin{macrocode}
\lstdefinelanguage{extras}{morekeywords={%
      poemtitle, poemtoc, versewidth, 
      vin, poemlines,poemtitlefont, 
      ProvidesClass,IfFileExists,
      RequirePackage,ifthenelse,chapter,
      includegraphics, newarray,readarray,of
}}
\lstloadlanguages{[LaTeX]TeX, [primitive]TeX, extras}
%    \end{macrocode}
%
% Note the |gobble=1| option. We use this to make the colorboxes
% with code not to show the `\%` sign in this documentation.
% Ideally you should fork the code below and adapt it to 
% your own needs.
%
% Also note that this is the default that is to be used in
% \pkg{tcolorbox} commands.
% 
%    \begin{macrocode}
\newtcolorbox{scriptexample}[2][shavian]{colback=thecodebackground,
boxrule=0pt,toprule=0pt,colframe=white}

\newtcolorbox{commands}[2][shavian]{colback=thecodebackground,
boxrule=0pt,toprule=0pt,colframe=white}

\lstset{language={[LaTeX]TeX},
       escapeinside={{(*@}{@*)}}, 
       numbers=left, 
       gobble=0,
       stepnumber=1,
       numbersep=5pt, 
       numberstyle={\footnotesize\color{gray}},
      % firstnumber=last,
       breaklines=false,
       framesep=5pt,
       basicstyle=\small\ttfamily,
       showstringspaces=false,
       stringstyle={\color{orange}\footnotesize},
       commentstyle=\color{black},
       rulecolor=\color{theshade},
       breakatwhitespace=true,
       showspaces=false, 
       xleftmargin=10pt,
       xrightmargin=10pt,
       aboveskip=3pt plus1pt minus1pt, 
       belowskip=7pt plus1pt minus1pt,  
       backgroundcolor=\color{theshade},
}
%    \end{macrocode}
%	
%	
% 	The environment |\begin{TeX}..\end{TeX}| provides a listings environment
% 	for typesetting, either TeX or LaTeX code.
% 	
%    \begin{macrocode}
\lstnewenvironment{teX}[1][]
  {\lstset{language=[LaTeX]TeX}\lstset{%
      breaklines=true,
      framesep=5pt,
      basicstyle=\verbatimfamily,
      showstringspaces=false,
      keywordstyle=\verbatimfamily,
      stringstyle={\color{gray!90}},
	   commentstyle={\color{gray!90}},
	   rulecolor=\color{theshade},
      breakatwhitespace=true,
	   xleftmargin=15pt,
	   xrightmargin=5pt,
	   aboveskip=\medskipamount,
	   belowskip=\medskipamount,
      backgroundcolor=\color{white}, #1
}}
{}


\lstnewenvironment{teXX}[1][]
  {\lstset{language=[LaTeX]TeX}\lstset{%
      breaklines=true,
      framesep=5pt,
      basicstyle=\verbatimfamily,
      showstringspaces=false,
      keywordstyle=\verbatimfamily,
      stringstyle=\color{maroon},
	  commentstyle=\color{black},
	  rulecolor=\color{gray!10},
      breakatwhitespace=true,
	  xleftmargin=0pt,
	  xrightmargin=5pt,
	  aboveskip=\medskipamount,
	  belowskip=\medskipamount,
      backgroundcolor=\color{white}, #1
}}
{}
%    \end{macrocode}

% \begin{docCommand}{continuelinenumber}{}
%   Continues code numbers from previous block
% \end{docCommand} 
%    \begin{macrocode}
\newcommand\continuelinenumber{\lstset{firstnumber=last}}
%    \end{macrocode}
% {startnumberat} 
%  The macro \cs{continueLineNumber}, provides a command
%  to start the next block of code with the code numbers continuing.
%  This requires the |listings| which is already included.
%  
%    \begin{macrocode}
% Always I forget this so I created some aliases
\newcommand\startlineat[1]{\lstset{firstnumber=#1}}
\let\numberlineat\startlineat
\let\startnumberat\numberlineat
%    \end{macrocode}
% 
% 
%
%    \begin{macrocode}
\newcommand\emphasis[2][black!80]{%
   \lstset{%
     emph={write, writeln,#2},
     escapeinside={(*@}{@*)},
     emphstyle={\verbatimfont%
                \bfseries%
                \textcolor{#1}%
                },
   }%
}%changed to textbf
      
   
\lstnewenvironment{teXXX}[1][]
  {\lstset{language=[LaTeX]TeX}%
    \lstset{%
      emph={cs, use,new,seq,map,inline,eq,gincr,incr,IfNoValueF,if,If,exist,protect,nopar,gset,%
      set,undefine,define,add,gadd,remove,div,newcounter%
      round,truncate,max,min,mod,gzero,int,%
      zero,newcount,protected,msg,error,DeclareDocumentCommand},
      emphstyle=\ttfamily\color{theemphasiscolor},
      firstnumber=last,
      stepnumber=1,
      escapeinside={{(*@}{@*)}},
      breaklines=true,
      framesep=5pt,
      basicstyle= \ttfamily,
      showstringspaces=false,
      keywordstyle=\ttfamily\color{blue},%\color{primary},
      stringstyle=\color{black!50},
      commentstyle=\color{black!70},
	   rulecolor=\color{gray!10},
      breakatwhitespace=true,
      prebreak={\Righttorque},
      postbreak={\space\Lefttorque},
      showspaces=false,  % shows spacing symbol
      upquote=true,
	   %xleftmargin=0pt,
	   %xrightmargin=5pt,
	   xleftmargin=15pt,
	   xrightmargin=5pt,
	 %  aboveskip=0pt, % compact the code looks ugly in type
	  % belowskip=0pt,  % user responsible to insert any skips
	   aboveskip=\medskipamount,
	   belowskip=\medskipamount,
      backgroundcolor=,
       #1
}}
{}

\lstnewenvironment{phdverbatim}[1][]
  {\lstset{language=[LaTeX]TeX}%
    \lstset{%
      emph={cs, use,new,seq,map,inline,eq,gincr,incr,IfNoValueF,if,If,exist,protect,nopar,gset,%
      set,undefine,define,add,gadd,remove,div,%
      round,truncate,max,min,mod,gzero,int,%
      zero,newcount,protected,msg,error,DeclareDocumentCommand},
      emphstyle=\verbatimfont\bfseries\color{black!80},
      numbers=none,
     % stepnumber=1,
      escapeinside={{(*@}{@*)}},
      breaklines=false,
      framesep=5pt,
      basicstyle= {\small\ttfamily},
      showstringspaces=false,
      keywordstyle=\ttfamily\color{thekeywordstyle},
      stringstyle=\color{black!50},
      commentstyle=\color{black!50},
	   rulecolor=\color{gray!10},
      breakatwhitespace=true,
      showspaces=false,  % shows spacing symbol
	   xleftmargin=15pt,
	   xrightmargin=5pt,
	 %  aboveskip=0pt, % compact the code looks ugly in type
	  % belowskip=0pt,  % user responsible to insert any skips
	  aboveskip=\medskipamount,
	  belowskip=\medskipamount,
      backgroundcolor=,
       #1
}}
{}
%    \end{macrocode}
% 
%
%
%    \begin{macrocode}
\lstnewenvironment{lualisting}[1][]
{\lstset{language=[LaTeX]TeX,
  basicstyle           = \ttfamily,
  showstringspaces     = false,
  upquote              = true,
  keywordstyle         =\color{blue},
  commentstyle         =\color{black!50},
  stringstyle          =\color{black!80},
  backgroundcolor      =\color{white},
  xleftmargin          =15pt,
  xrightmargin         =5pt,
  aboveskip            =\medskipamount,
  belowskip	            =\medskipamount,
  #1}}
{}

%    \end{macrocode}
% \section{LaTeX code demo environments}
%
%	To demonstrate \latex code it is sometimes desirable to have the code
%	be executed. This was pioneered in a number of packages. One of
%	the better packages to do so is \pkg{tcolorbox}. We use it to define
%	a special environment.
%

% \begin{docEnvironment}{texexample}{ \marg{title} \marg{label for referencing} } 
% The environment |texexample| will list the code
%	using the \pkgname{listings} package, so we can have a nice box and shows the
%	output at the bottom section.
%	\end{docEnvironment}
%	
%	First we define a new counter which resets at every chapter. If |c@chapter|
%	is not defined we reset it based on sections.
%
% \begin{enumerate}
%	\item [\#1] Title of the example
%	\item [\#2] label for referencing
% \end{enumerate}
% 
%    \begin{macrocode}
  \ifx\c@chapter\@undefined
    \newcounter{texexp}[section]
    \@addtoreset{c@texexp}{c@section}
  \else
    \newcounter{texexp}[chapter]
    \@addtoreset{c@texexp}{c@chapter}
  \fi 
%    \end{macrocode}
%
%	
%    \begin{macrocode}
%\tcbset{listing utf8=latin1}% optional; ’latin1’ is the default.
\def\dcircle#1{\ding{\numexpr181 + #1\relax}}
\def\thetexexp{\@arabic\c@section.\arabic{texexp}}
%    \end{macrocode}
%    \begin{macrocode}    
\tcbset{texexp/.style={% 
    fonttitle=\small\ttfamily, 
    fontupper=\small, 
    fontlower=\small,
    coltitle=black,
    colback = thecodebackground,% background
    colframe=thecodebackground, 
   % process code={\def\dcircle##1{\ding{\numexpr181 + ##1}}},
      %colupper=spot!,
   },
   listing options = {%
     keywordstyle=\color{thekeywordstyle},
     belowskip=0pt, 
     escapeinside={(*@}{@*)},%
     breaklines=true,%
     backgroundcolor=\color{thecodebackground},%
     %firstnumber=last,%
     stepnumber=1,%
     upquote=true,%
     alsoletter={_,:},%
     commentstyle=\color{thecommentstyle},%
     emph={cs,new,seq,map,inline,eq,gincr,incr,IfNoValueF,if,%
            If,exist,protect,nopar,gset,%
            set,undefine,define,add,gadd,remove,div,%
            round,truncate,max,min,mod,gzero,int,exp,put,left,args,%
            zero,newcount,protected,msg,error,%
            eval,to,arabic,alph,Alph,roman,Roman,dim%
            DeclareDocumentCommand,%
            NewDocumentCommand,%
            RenewDocumentCommand,includegraphics,
            function,local,return,break,
         },%
           %
          % For LaTeX3 we need to add these, note % is important
          % dn’t miss, at the end...
          moretexcs    = {DeclareDocumentCommand,IfBooleanTF,tex_def:D,%
          cs_new:Nn,cs_new:Npn,cs_new:cn,cs_set_nopar:Npn,token_to_meaning:N,%
          %primitives
          cs:w,cs_end:,tex_underline,group_begin:, group_end:,%
          %coffins
          NewCoffin,JoinCoffins,SetHorizontalCoffin,TypesetCoffin,%
          %properties
          prop_new:N,prop_new:c,prop_put:Nnn,%
          %boolean
          bool_new:N,bool_set_true:N,bool_set_false:N,%
          bool_if:NTF,%
          hbox_to_wd:nn,%
          IfNoValueTF,%
          %token lists
          tl_new:N,tl_set:Nn,tl_concat:NNN,%
          token_to_meaning:N,%
          seq_pop_left:NN,%
          %
          %int
          int_if_exist:cT,int_use:c,int_new:c,int_new:N,int_eval:n,%
          int_add,int_use,int_to_roman,%
          %boxes
          box_new:c,hbox_set:cn,box_use:c,vbox_set:cn,box_move_down:nn,%
          %string
          str_if_eq_x:nnTF,%
          tl_tail:n,%
          DeclareObjectType,%
          DeclareTemplateInterface,%
          DeclareTemplateCode,%
          DeclareInstance,UseInstance,AssignTemplateKeys%
          keys_set,keys_define,%      
          },%
     emphstyle=\verbatimfont\bfseries\color{theemphasiscolor},%
          %
   },%close listings options
      % added for better control
      arc=0pt,  
      outer arc=0pt,
      example1/.code 2 args={\refstepcounter{texexp}{\ifx#2\empty\else\label{#2}\fi}}%Reference
     \pgfkeysalso{texexp, enhanced, breakable, title={Example \thetexexp\ #1}%
 },
}

\def\emphasize#1{%
\tcbset{texexp/.style={% 
    fonttitle=\small\ttfamily, 
    fontupper=\small, 
    fontlower=\small,
    coltitle=black,
    colback = thecodebackground,% background
    colframe=thecodebackground, 
    %process code={\def\dcircle##1{\ding{\numexpr181 + ##1}}},
      %colupper=spot!,
   },
   listing options = {%
     keywordstyle=\color{thekeywordstyle},
     belowskip=0pt, 
     escapeinside={(*@}{@*)},%
     breaklines=true,%
     backgroundcolor=\color{thecodebackground},%
     %firstnumber=last,%
     stepnumber=1,%
     upquote=true,%
     alsoletter={_,:},%
     commentstyle=\color{thecommentstyle},%
     emph={cs,new,seq,map,inline,eq,gincr,incr,IfNoValueF,if,%
            If,exist,protect,nopar,gset,%
            set,undefine,define,add,gadd,remove,div,%
            round,truncate,max,min,mod,gzero,int,exp,put,left,args,%
            zero,newcount,protected,msg,error,%
            eval,to,arabic,alph,Alph,roman,Roman,dim%
            DeclareDocumentCommand,%
            NewDocumentCommand,%
            RenewDocumentCommand,includegraphics,
            function,local,return,#1,
         },%
           %
          % For LaTeX3 we need to add these, note % is important
          % dn’t miss, at the end...
     moretexcs    = {DeclareDocumentCommand,IfBooleanTF,tex_def:D,%
          cs_new:Nn,cs_new:Npn,cs_new:cn,cs_set_nopar:Npn,token_to_meaning:N,%
          %primitives
          cs:w,cs_end:,tex_underline,group_begin:, group_end:,%
          %coffins
          NewCoffin,JoinCoffins,SetHorizontalCoffin,TypesetCoffin,%
          %properties
          prop_new:N,prop_new:c,prop_put:Nnn,%
          %boolean
          bool_new:N,bool_set_true:N,bool_set_false:N,%
          bool_if:NTF,%
          hbox_to_wd:nn,%
          IfNoValueTF,%
          %token lists
          tl_new:N,tl_set:Nn,tl_concat:NNN,%
          token_to_meaning:N,%
          seq_pop_left:NN,%
          %
          %int
          int_if_exist:cT,int_use:c,int_new:c,int_new:N,int_eval:n,%
          int_add,int_use,int_to_roman,%
          %boxes
          box_new:c,hbox_set:cn,box_use:c,vbox_set:cn,box_move_down:nn,%
          %string
          str_if_eq_x:nnTF,%
          tl_tail:n,%
          DeclareObjectType,%
          DeclareTemplateInterface,%
          DeclareTemplateCode,%
          DeclareInstance,UseInstance,AssignTemplateKeys%
          keys_set,keys_define,%      
          },%
     emphstyle=\verbatimfont\bfseries\color{theemphasiscolor},%
          %
   },%close listings options
      % added for better control
      arc=0pt,  
      outer arc=0pt,
  }%close style
}%close command
%
\newenvironment{texexp}[1]{\tcblisting{texexp,#1}}{\endtcblisting}

\newenvironment{example1}[3][]{\tcblisting{example1={#2}{#3},#1}}%
    {\endtcblisting}
%
%    \end{macrocode}
%    
%    \begin{docEnvironment}{texexample} { \oarg{} \marg{Title} \meta{label} }
%      
%    \end{docEnvironment}
%    \begin{macrocode}
\newenvironment{texexample}[3][]{\noindent\tcblisting{example1={#2}{#3},#1}}%
    {\endtcblisting }
\newenvironment{texcode}[3][listing only]{\noindent\tcblisting{example1={#2}{#3},#1}}%
    {\endtcblisting }    
%    
% Need to fix
\let\luaexample\texexample        
\let\endluaexample\endtexexample    
%    \end{macrocode}
%     
%    \begin{macrocode}
%\tcbset{luacode/.style={%
%      fonttitle=\small\ttfamily, 
%      fontupper=\small, 
%      fontlower=\small,
%      coltitle=black,
%      colback = thecodebackground,% background
%      colframe=thecodebackground, 
%      %colupper=spot!,
%      },
%      listing options = {
%          language={[5.2]Lua},
%          belowskip=0pt, 
%          escapeinside={(*@}{@*)},%
%          breaklines=true,%
%          backgroundcolor=\color{thecodebackground},%
%          firstnumber=last,%
%          stepnumber=1,%
%          upquote=true,%
%          alsoletter={_,:},%
%          commentstyle=\bfseries\color{black!90},%
%          stringstyle = \color{black!90},
%          emphstyle=\verbatimfont\bfseries\color{black!80},%
%          keywordstyle= \bfseries\color{black!80},%
%          },
%      % added for better control
%      arc=0pt,  
%      outer arc=0pt,
%      luaexp1/.code 2 args={\refstepcounter{texexp}\label{#2}}%Reference
%     \pgfkeysalso{luacode, enhanced, breakable, title={Example \thetexexp\ #1}},
%}
%\newenvironment{luaexp1}[1]{\tcblisting{luacode,#1}}{\endtcblisting}
%
%\newenvironment{luaexample}[3][]{\noindent\tcblisting{luaexp1={#2}{#3},#1}}%
%    {\endtcblisting}
%%
%    \end{macrocode} 
%
% The following demonstrates the usage.
%
% 	\begin{texexample}[]{atest}{This is a comment?}
	  \def\demomacro{Hello World!}
%	\end{texexample}
%
% 	\begin{example}{A Test}{test}{This is a comment?}
%	  \def\demomacro{Hello World!}
%	\end{example}
%
% \section{makeidx}    
%\docAuxCmd{printindex} has been defined. If the test is positive then an indexing package has been loaded, otherwise we load the \pkgname{makeidx}\footcite{makeidx}.
%  
%    \begin{macrocode}
\ExplSyntaxOn
\cs_if_exist:cTF {printindex}
  { }
  {
    \RequirePackage{makeidx}[2000/03/29]
  }
\ExplSyntaxOff  
%    \end{macrocode}
%
% \section{Refcount}
%
%References are not numbers, however they often store numerical data
%such as section or page numbers. \cs{ref} or \cs{pageref} cannot be used for
%counter assignments or calculations because they are not expandable, generate
%warnings, or can even be links. The package provides expandable macros
%to extract the data from references. Packages \pkgname{hyperref}, \pkgname{nameref}\footfullcite{nameref}, 
% \pkgname{titleref}\footfullcite{}, and
% \pkgname{babel} are supported.
%    \begin{macrocode}  
\RequirePackage{refcount}[2011/10/16]
%    \end{macrocode}
%    \begin{macrocode}
\ExplSyntaxOn
\cs_set:Npn \colDef#1{\textcolor{\phdkv@col@command}{#1}}
\cs_set:Npn \colOpt#1{\textcolor{\phdkv@col@opt}{#1}}
\ExplSyntaxOff

\lstdefinestyle{tcbdocumentation}{language={[LaTeX]TeX},
    aboveskip={0\p@ \@plus 6\p@},
    belowskip={0\p@ \@plus 6\p@},
    columns=fullflexible,
    keepspaces=true,
    breaklines=true,
    prebreak={\Righttorque},
    postbreak={\space\Lefttorque},
    breakatwhitespace=true,
    basicstyle=\ttfamily\footnotesize,
    extendedchars=true,
    nolol,
    inputencoding = \phdkv@listingencoding}
%    \end{macrocode}
% The following macros are taken from ltxdoc and modified accordingly.
%
% \begin{docCommand} {phdcs} { \marg{macro name}}
%   We modify the standard |\cs| and save to a new name to be able to use
%   underscores. Maybe there are better ways of doing it as well.
% \end{docCommand}
%
%    \begin{macrocode}
\DeclareRobustCommand\phdcs[1]{{\color{thecs}{\texttt{\char`\\\detokenize{#1}}}}}
\ExplSyntaxOn
\let\phd@doc@org@meta\meta%
%    \end{macrocode}
%
% \begin{docCommand}{meta}{\marg{argument}}
% We modify the standard \meta{arguments} to allow for colour settings.  The \docColor{themeta} is defined in the \pkg{phd-colorpalette} package. The braces are not colored
% as they do not look very good if they are. 
% \end{docCommand}
%    \begin{macrocode}
\cs_set:Npn \meta#1{
   \group_begin:
   \rmfamily\phd@doc@org@meta{{\color{themeta}#1}}
   \group_end:
}
%    \end{macrocode}
%
%    \begin{macrocode}
\cs_set:Npn \marg #1
  {
    {\ttfamily\char`\{}\rmfamily\phd@doc@org@meta{\color{theoarg}#1}{\ttfamily\char`\}}
  }
%    \end{macrocode}
% 
% \begin{docCommand}{oarg}{\marg{argument}}
% Typesets an optional argument, as found in \latexe commands. The command:
% \begin{quote}
%  |\test\oarg{style=two}\marg{mandatory arguments}| 
% \end{quote}
% will typeset:
%  \begin{quote}
%   |\test|\oarg{style=two}\marg{mandatory arguments}
%  \end{quote}
% \end{docCommand} 
%    \begin{macrocode}
\cs_set:Npn \oarg #1
  {
    {\ttfamily[}\meta{#1}{\ttfamily]}
  }
\ExplSyntaxOff
%    \end{macrocode}
%
%    \begin{macrocode}
\newif\ifphd@doc@toindex
\newif\ifphd@doc@colorize
\newif\ifphd@doc@annotate
%    \end{macrocode}
%

%    \begin{macrocode}
% language specific texts
\cxset{
  pageshort/.store in=\phdkv@text@pageshort,
  doclang/.cd,
  color/.store in=\phdkv@text@color,
  colors/.store in=\phdkv@text@colors,
  environment content/.store in=\phdkv@text@envcontent,
  environment/.store in=\phdkv@text@env,
  environments/.store in=\phdkv@text@envs,
  key/.store in=\phdkv@text@key,
  keys/.store in=\phdkv@text@keys,
  index/.store in=\phdkv@text@index,
  pageshort/.store in=\phdkv@text@pageshort,
  value/.store in=\phdkv@text@value,
  values/.store in=\phdkv@text@values,
}   
%    \end{macrocode}
%
% \section{Documentation commands key definitions}
%
%    \begin{macrocode}
\cxset
  {
    documentation listing options/.store in=\phdkv@doclistingoptions,%
    documentation listing style/.style={documentation listing options={style=#1}},%
    documentation minted style/.store in=\phdkv@docmintstyle,
    documentation minted options/.store in=\phdkv@docmintoptions,
  }
%    \end{macrocode}

%    \begin{macrocode}  
\cxset{    
    color command/.store in=\phdkv@col@command,
    color environment/.store in=\phdkv@col@environment,
    color key/.store in=\phdkv@col@key,
    color value/.store in=\phdkv@col@value,
    color color/.store in=\phdkv@col@color,
    color definition/.style={color command={#1},color environment={#1},color key={#1},
                           color value={#1},color color={#1}},
    color option/.store in=\phdkv@col@opt,
    color hyperlink/.store in=\phdkv@colhyper,
    color frame/.store in=\phdkv@colhyper,
%    
    before example/.store in=\phdkv@beforeexample,
    after example/.store in=\phdkv@afterexample,
}
%    \end{macrocode}
%
% \section{Index settings}
%
%  We consider indices to be composed of three elements, the index heading 
%  i.e., the word Index typeset in a specific language, the entries and the
%  page numbers. The following keys relate to settings that must have a 
%  one to one relationship with the settings of the |.ist| file. Unfortunately
%  there is no easy way to achieve this. A better strategy is togenerate the \docextension{.ist} file 
%  automatically by writing the parameters to a file.\tcbdocmarginnote{16/08/2017}
%
%    \begin{macrocode}
\cxset{    
    index actual/.store in   = \idx@actual,
    index quote/.store in    = \idx@quote,
    index level/.store in    = \idx@level,
    index format/.store in   = \idx@format,
    index encap/.store in    = \idx@encap,
    index colorize/.is if    = phd@doc@colorize,%
    index annotate/.is if    = phd@doc@annotate,%
  }
%    \end{macrocode}
%
%  The following keys relate to heading describing commands, keys environments and the
%  like.
%
%    \begin{macrocode}  
\cxset{    
    doc left/.dimstore in=\phdkv@doc@left,
    doc right/.dimstore in=\phdkv@doc@right,
    doc left indent/.dimstore in=\phdkv@doc@indentleft,
    doc right indent/.dimstore in=\phdkv@doc@indentright,
    doc head command/.style={doc@head@command/.style={#1}},
    doc head environment/.style={doc@head@environment/.style={#1}},
    doc head key/.style={doc@head@key/.style={#1}},
    doc head/.style={doc head command={#1},doc head environment={#1},doc head key={#1}},
    doc description/.store in=\phdkv@doc@description,%
    doc into index/.is if=phd@doc@toindex,%
  }
%    \end{macrocode}
%
% 
%    \begin{macrocode}
% styles
\cxset{
  docexample/.style={colframe=ExampleFrame,colback=ExampleBack,fontlower=\footnotesize},
  documentation minted style=,
  documentation minted options={tabsize=2,fontsize=\small},
  index default settings/.style={index actual={@},index quote={"},index level={!}},
  index german settings/.style={index actual={=},index quote={!},index level={>}},
  english language/.code={\cxset{doclang/.cd,
    color=color,colors=Colors,
    environment content=environment content,
    environment=environment,environments=Environments,
    key=key,keys=Keys,
    index=Index,
    pageshort={P.},
    value=value,values=Values}},
}

\AtBeginDocument{%
  \csname phd@doc@index@\idx@format\endcsname%
  \hypersetup{
  citecolor=\phdkv@colhyper,
  linkcolor=\phdkv@colhyper,
  urlcolor=\phdkv@colhyper,
  filecolor=\phdkv@colhyper,
  menucolor=\phdkv@colhyper
}}

%    
%\cs_set:Npn \dispExample{\cxset{docexample}\phdwritetemp}
%
%\cs_set:Npn \enddispExample{%
%  \endtcbwritetemp%
%  \phdkv@beforeexample\begin{tcolorbox}%
%  \phd@doc@usetemplisting%
%  \phdlower%
%  \phdusetemp%
%  \end{tcolorbox}\phdkv@afterexample%
%}
%
%\newenvironment{dispExample*}[1]{%
%  \cxset{docexample,#1}\phdwritetemp%
%  }{\enddispExample}
%
%    \end{macrocode}
%
% Set a basic style for dispListing
%    \begin{macrocode}
\lstdefinestyle{smalldisplay}{numbers=none, backgroundcolor=\color{thecodebackground}, xleftmargin=0pt}

\tcbset{documentation listing style=smalldisplay}
\tcbset{
docexample/.style={colframe=thecodeframe, colback=thecodebackground,
before skip=\medskipamount,after skip=\medskipamount,
fontlower=\footnotesize}
}
%    \end{macrocode}
%    \begin{macrocode}
%\newenvironment{dispListing*}[1]{%
%  \phd@layer@pushup\cxset{docexample,#1}\phdwritetemp%
%  }{\enddispListing}

% index auxiliary macros
%    \end{macrocode}

%    \begin{macrocode}
\ExplSyntaxOn
\cs_set:Npn \phdindexprintca #1#2#3 {%
  \ifphd@doc@colorize
    \textcolor{#2}
    { \texttt{#1} }
  \else
    \texttt{#1}
  \fi%
  \ifphd@doc@annotate\ 
   #3
  \fi%
}


\cs_set:Npn \phd_index_print_c#1#2{%
  \ifphd@doc@colorize
    \textcolor{#2}{\texttt{#1}}
  \else\texttt{#1}
  \fi%
}


\NewDocumentCommand{\phdindexprintcomc}{ m }
  {
    \phd_index_print_c {\phdcs{#1}}{\phdkv@col@command}
  }
\ExplSyntaxOff
%    \end{macrocode}
%
%  \begin{docCommand} {phd_print_com} { \marg{cs name} }
%    Prints a command. 
%  \end{docCommand}
%
%    \begin{macrocode}
\ExplSyntaxOn
\cs_new:Npn \phd_print_com #1
  {
    \textcolor{black}{\ttfamily\bfseries\phdcs{#1}}%\phdkv@col@command
  }
\ExplSyntaxOff  
%    \end{macrocode}

%    \begin{macrocode}
\ExplSyntaxOn
\newrobustcmd{\phdindexprintenvca}[1]
  {
    \phdindexprintca{#1}{\phdkv@col@environment}{\phdkv@text@env}
  }

\newrobustcmd{\phdindexprintenvc}[1]
  {
    \phd_index_print_c{#1}{\phdkv@col@environment}
  }
\ExplSyntaxOff
%    \end{macrocode}
%
%    \begin{macrocode}
\ExplSyntaxOn
\cs_set:Npn \phd_print_env#1
  {
    \textcolor{\phdkv@col@environment}{\ttfamily\bfseries#1}
  }

\newrobustcmd{\phdindexprintkeyca}[1]
  {
    \phdindexprintca{#1}{\phdkv@col@key}{\phdkv@text@key}
  }

\newrobustcmd{\phdindexprintkeyc}[1]{\phd_index_print_c{#1}{\phdkv@col@key}}

\cs_set:Npn \phd_print_key #1
  {
    \textcolor{\phdkv@col@key}{\ttfamily\bfseries#1}
  }

\newrobustcmd {\phdindexprintvalca}[1]
  {
    \phdindexprintca{#1}{\phdkv@col@value}{\phdkv@text@value}
  }

\newrobustcmd {\phdIndexPrintValC}[1]
  {
    \phd_index_print_c{#1}{\phdkv@col@value}
  }

\cs_set:Npn \phd@Print@Val #1 
  {
    \textcolor {\phdkv@col@value} {\ttfamily\bfseries#1}
  }

\newrobustcmd{\phdindexprintcolca}[1]
  {
    \phdindexprintca{#1}{\phdkv@col@color}{\phdkv@text@color}
  }

\newrobustcmd{\phdindexprintcolc}[1]
  {
    \phd_index_print_c{#1}{\phdkv@col@color}
  }

\cs_set:Npn \phd_print_col #1 
  {
    \textcolor{\phdkv@col@color}{\ttfamily\bfseries#1}
  }



\cs_set:Npn \phdindexcom #1 
  {
    \ifphd@doc@toindex
      \index
       {
         #1
         \idx@actual
         \phdindexprintcomc{#1}
       }
    \fi
  }


\cs_set:Npn \phd_index_env #1
  {
    \ifphd@doc@toindex
      \index
        {#1
          \idx@actual
          \phdindexprintenvca{#1}
        }
      \index
        {
          \phdkv@text@envs
          \idx@level#1
          \idx@actual
          \phdindexprintenvc{#1}
        }
    \fi
  }

\cs_set:Npn \phd_index_key #1 
  {
    \ifphd@doc@toindex
      \index{#1\idx@actual\phdindexprintkeyca{#1}}
      \index
        {
          \phdkv@text@keys
          \idx@level#1
          \idx@actual
          \phdindexprintkeyc{#1}
        }
    \fi
  }


\cs_set:Npn \phd_index_key_path #1#2
  {
    \ifphd@doc@toindex\index{#2\idx@actual  
      \phdindexprintkeyca{#2}}
      \index{\phdkv@text@keys
         \idx@level#1
        \idx@actual
        \phdindexprintkeyc{/#1/}
        \idx@level#2
        \idx@actual
        \phdindexprintkeyc{#2}
      }
    \fi
  }
%    \end{macrocode}
%
% \begin{docCmd} {phd_index_val} { \marg {value} }
% \end{docCmd}
%    \begin{macrocode}
\cs_set:Npn \phd_index_val #1  
  {
    \ifphd@doc@toindex
      \index
      {
        #1\idx@actual
        \phdindexprintvalca{#1}
      }
      \index
        {
          \phdkv@text@values
          \idx@level #1
          \idx@actual
          \phdIndexPrintValC{#1}
        }
    \fi
  }
%    \end{macrocode}
%
% \begin{docCmd} {phd_index_col} { \marg{color name} }
% \end{docCmd}
%    \begin{macrocode}
\cs_set:Npn \phd_index_col #1
  {
    \ifphd@doc@toindex
    \index
      {
        #1
        \idx@actual
        \phdindexprintcolca{#1}
      }
    \index
      {
        \phdkv@text@colors \idx@level #1
        \idx@actual\phdindexprintcolc {#1}
      }
    \fi
  }

\ExplSyntaxOff
%    \end{macrocode}

%    \begin{macrocode}
\ExplSyntaxOn
\cs_set:Npn \phd_brackets #1
  {
    {\ttfamily\char`\{}#1{\ttfamily\char`\}}
  }
\ExplSyntaxOff
%    \end{macrocode}
%
% \begin{docEnv}{phd@manual@entry}{}
% Internal command for setting an index entry
% \end{docEnv}
%    \begin{macrocode}
\ExplSyntaxOn
\newenvironment{phd@manual@entry}
  {
   \begin{list}{}
    {
     \setlength{\leftmargin}{\phdkv@doc@left}%
     \setlength{\itemindent}{0pt}%
     \setlength{\itemsep}{0pt}%
     \setlength{\parsep}{0pt}%
     \setlength{\rightmargin}{\phdkv@doc@right}%
    }\item
  }
  {\end{list}}

\cs_set:Npn \phd_manual_top #1
  {
    \itemsep=0pt
    \parskip=0pt
    \item\strut{#1}\par
    \topsep=0pt
  }

\cs_set:Npn \phd_doc_do_description:
  {
    \ifx\phdkv@doc@description\@empty
    \else\phdlower
      \raggedleft(\phdkv@doc@description)
    \fi
  }
\ExplSyntaxOff
%    \end{macrocode} 
%
%  \begin{docEnv} {phd@doc@head} { \marg{additional options} }
%  \end{docEnv}
%
%    \begin{macrocode}
\ExplSyntaxOn
\newtcolorbox{phd@doc@head}[1]
 {
  blank,
  colback=white,
  colframe=white,
  code={\tcbdimto\tcb@temp@grow@left{-\phdkv@doc@indentleft}%
        \tcbdimto\tcb@temp@grow@right{-\phdkv@doc@indentright}},
  grow~to~left~by=\tcb@temp@grow@left,%
  grow~to~right~by=\tcb@temp@grow@right,
  sidebyside,
  sidebyside~align=top,
  sidebyside~gap=-\tcb@w@upper@real,
  phantom=\phantomsection,%
  enlarge~bottom~by=-0.2\baselineskip,
  #1
 }
\ExplSyntaxOff
%    \end{macrocode}
%
%    \begin{macrocode}
\ExplSyntaxOn
\newenvironment{docCmd}[3][]{
  \cxset{#1}%
  \begin{phd@manual@entry}%
  \begin{phd@doc@head}{doc@head@command}%
  \phd_print_com{#2}
  \phdindexcom{#2}
  \protected@edef\@currentlabel{\noexpand\phdcs{#2}}
  \label{com:#2}{\ttfamily #3}%
  \phd_doc_do_description:%
  \end{phd@doc@head}}%
  {\end{phd@manual@entry}}
\ExplSyntaxOff
%    \end{macrocode}
%

%    \begin{macrocode}
\ExplSyntaxOn
\newenvironment{docCmd*}
  {
    \bgroup
    \phd@doc@toindexfalse
    \begin{docCmd}
  }
  {
    \end{docCmd}
    \egroup
  }

\newenvironment{docEnv}[3][]{\cxset{#1}%
  \begin{phd@manual@entry}%
  \begin{phd@doc@head}{doc@head@environment}%
  \strut
  \phdcs{begin}
  \phd_brackets{\phd_print_env{#2}}
  \phd_index_env{#2}
  \protected@edef\@currentlabel{#2}\label{env:#2}{\ttfamily #3}%\par%
  \strut~~\meta{\phdkv@text@envcontent}%\par%
  \strut\phdcs{end}
  \phd_brackets{\phd_print_env{#2}}%
  \phd_doc_do_description:%
  \end{phd@doc@head}}%
  {\end{phd@manual@entry}}
\ExplSyntaxOff  
%    \end{macrocode}
%
% \begin{docEnv}{docEnv*} {}{}
% \end{docEnv}
%    \begin{macrocode}
\newenvironment{docEnv*}
  {
    \bgroup
    \phd@doc@toindexfalse
    \begin{docEnv}
  }
  { \end{docEnv}\egroup }
%    \end{macrocode}

%% \begin{docEnv}{docKey} {}{}
%% \end{docEnv}
%%    \begin{macrocode} 
%\ExplSyntaxOn
%\renewenvironment{docKey}[4][\@empty]{\begin{phd@manual@entry}%
%  \cxset{doc description={#4}}%
%  \begin{phd@doc@head}{doc@head@key}%
%  \ifx#1\@empty%
% \phd_print_key{#2}
%  \phd_index_key{#2}
%  \protected@edef\@currentlabel{#2}
%  \label{key:#2}{\ttfamily #3}%
%  \else
%   \phd_print_key{/#1/#2}
%    \phd_index_key_path{#1}{#2}
%    \protected@edef\@currentlabel{/#1/#2}
%    \label{key:/#1/#2}{\ttfamily#3}%
%  \fi%
%  \phd_doc_do_description:%
%  \end{phd@doc@head}}%
%  {\end{phd@manual@entry}}
%
%\renewenvironment{docKey*}
%  {\bgroup\phd@doc@toindexfalse\begin{docKey}}
%  {\end{docKey}\egroup}
%
%\cs_set:Npn \phdmakedocSubKey#1#2{%
%  \newenvironment{#1}[4][\@empty]{%
%    \ifx##1\@empty
%      \cs_set:Npn \phd@key@path {#2}
%    \else
%     \cs_set:Npn \phd@key@path{#2/##1}
%    \fi%
%    \begin{docKey}[\phd@key@path]{##2}{##3}{##4}}%
%    {\end{docKey}}%
%  \newenvironment{#1*}{\bgroup\phd@doc@toindexfalse\begin{#1}}{\end{#1}\egroup}%
%}
\ExplSyntaxOff
%    \end{macrocode} 

% \begin {docCmd} {docAuxCmd} { \meta{*} \meta{cs name} }
% \end {docCmd} 
%
%    \begin{macrocode}
\ExplSyntaxOn
\cs_set:Npn \docAuxCmd: #1
  {
    \phdindexprintcomc {#1}
    \phdindexcom{#1}
  }
  
\cs_set:Npn \docAuxCmd_star #1 
  {
    \phdindexprintcomc {#1}
  }

\NewDocumentCommand \docAuxCmd { s m }
  {
    \IfBooleanTF {#1}
    {\docAuxCmd_star {#2} } 
    {\docAuxCmd: {#2} }  
  } 
\ExplSyntaxOff
%    \end{macrocode}
%
%    \begin{macrocode}
\ExplSyntaxOn
\cs_set:Npn \doc_aux_env: #1
  {
    \phd_print_env{#1}
    \phd_index_env{#1}
  }
  
\cs_set:Npn \doc_aux_env_star #1
  {
    \phd_print_env{#1}
  }
%    \end{macrocode}
%
%  \begin{docCmd} {docAuxEnv} { \meta{star} \marg{arg1} }
%  \end{docCmd}
%
%    \begin{macrocode}
\NewDocumentCommand\docAuxEnv { s m } 
  {
    \IfBooleanTF {#1}   
    {\doc_aux_env_star{#2} }
    {\doc_aux_env: {#2} }
  }  
\ExplSyntaxOff  
%    \end{macrocode}
%
% \begin{docCmd} {docAuxKey} { \oarg{path} \marg{} }
% \end{docCmd}
%    \begin{macrocode}
\ExplSyntaxOn
\NewDocumentCommand {\doc_aux_key:} {O{\@empty} m}
  {%
     \ifx#1\@empty%
     \phd_print_key{#2}
     \phd_index_key{#2}%
     \else%
     \phd_print_key{/#1/#2}\phd_index_key_path{#1}{#2}%
  \fi
  }%

\newcommand{\doc_aux_key_star}[2][\@empty]{%
  \ifx#1\@empty%
   \phd_print_key {#2}%
  \else%
   \phd_print_key {/#1/#2}%
  \fi}%
  
\DeclareDocumentCommand {\docAuxKey} { s m }
  {
    \IfBooleanTF {#1}
      { \doc_aux_key_star {#2} } { \doc_aux_key: [#1]{#2} }
  }  
\ExplSyntaxOff
%    \end{macrocode} 
%
% \begin{docCmd}{docColor} { \marg{color name} } 
%   Typesets a color name and also adds it onto the index. This is identical
%   with tcolorbox, which we overwrite.
% \end{docCmd}
%
% The \docColor{bgsexy} is the main color used for backgrounds.
%
%    \begin{macrocode}
\ExplSyntaxOn

\cs_set:Npn \doc_color_aux #1
  {
    \phd_print_col {#1}
    \phd_index_col {#1}
  }
  
\cs_set:Npn \doc_color_star: #1
  {
    \phd_print_col{#1}
  }

\DeclareDocumentCommand {\docColor} { s m }
  {
    \IfBooleanTF {#1} 
      { \doc_color_star: {#2} }
      { \doc_color_aux {#2} }
  }
\ExplSyntaxOff  
%    \end{macrocode}

% \begin{docCmd}{docValue}{ \meta{*} \marg{value} }
% \end{docCmd}
%
%    \begin{macrocode}
\ExplSyntaxOn
\cs_set:Npn \docValue@#1{\phd@Print@Val{#1}\phd_index_val{#1}}%
\cs_set:Npn \docValue@star#1{\phd@Print@Val{#1}}%

\DeclareDocumentCommand \docValue { s m } 
  {
    \IfBooleanTF {#1}
      { \docValue@star {#2} }
      { \docValue@ {#2} }
  }
\ExplSyntaxOff  
%    \end{macrocode}
%
% \begin{docCmd} {phd_ref_doc} { \marg{ref label} }
% \end{docCmd}
%
% We use \pkgname{hyperref} to add links. The \docAuxCmd{hyperref}\oarg{label}\marg{text}
% is used to create the link. We use |\ding{213}| \ding{213} for the page see \refCmd{docColor}.
% This is a great technique pioneered in the PGF and PGFPlots manuals for cross referencing
% and the code below is an adaptation.
%
%    \begin{macrocode}
\ExplSyntaxOn
\setrefcountdefault{-1}

\cs_set:Npn \phd_ref_doc #1
{
  \hyperref[#1]
  {\texttt{\ref*{#1}}%
    \ifnum\getpagerefnumber{#1}=\thepage
    \else%
      \textsuperscript
      {
        \ding{213}\,
        \phdkv@text@pageshort\,
        \pageref*{#1}
      }
   \fi
   }
 }

\cs_set:Npn \phd_ref_doc_star#1
  {
    \hyperref[#1]{\texttt{\ref*{#1}}}
  }
\ExplSyntaxOff
%    \end{macrocode}
%
% \begin{docCmd}{refCmd} { \oarg{*} \marg{cmd name} }
% \end{docCmd}
% The \refCmd {refCmd} references a command. This is similar to
% \index{maths>mathematical}
%
%    \begin{macrocode}
\ExplSyntaxOn
\cs_set:Npn \ref_com: #1 
  {
    \phd_ref_doc{com:#1}
  }
  
\cs_set:Npn \ref_com_star #1
  {
    \phd_ref_doc_star{com:#1}
  }


\DeclareDocumentCommand {\refCmd} { s m } 
  {
    \IfBooleanTF {#1}
    { \ref_com_star {#2} } { \ref_com: {#2} }
  }  
\ExplSyntaxOff
%    \end{macrocode}
%
%    \begin{macrocode}
\ExplSyntaxOn
\cs_set:Npn \refEnv: #1 {\phd_ref_doc{env:#1}}
\cs_set:Npn \refEnv_star#1{\phd_ref_doc_star{env:#1}}

\DeclareDocumentCommand {\refEnv} { s m }
  {
    \IfBooleanTF {#1}
      { \refEnv_star {#2} }
      { \refEnv: {#2}     }
  }
\ExplSyntaxOff
%    \end{macrocode}
%
% \begin{docCmd}{refKey} { \meta{*} \marg{ref text} }
% \end{docCmd}
%
%    \begin{macrocode}
\ExplSyntaxOn
\cs_set:Npn \refKey@#1{\phd_ref_doc{key:#1}}
\cs_set:Npn \refKey@star#1{\phd_ref_doc_star{key:#1}}
\DeclareDocumentCommand {\refKey} { s m }
  {
    \IfBooleanTF { #1 }
      { \refKey@star {#2} }
      { \refKey@ {#2}       }
  }
\ExplSyntaxOff
%    \end{macrocode}
%
% \begin{docCmd} {refAux} {}
% \end{docCmd}
%    \begin{macrocode}
\ExplSyntaxOn
\cs_set:Npn \refAux#1{\textcolor{\phdkv@colhyper}{\ttfamily #1}}
\cs_set:Npn \refAuxcs#1{\textcolor{\phdkv@colhyper}{\phdcs{#1}}}
\ExplSyntaxOff
%    \end{macrocode}
% 
% \section{Indexing}
%  Most of the indexing macros that follow have been adapted from the pgfmanual-en-macros
%  or the tcolorbox documentation code and transliterated to expl3 language.
%
%    \begin{macrocode}
\ExplSyntaxOn
\cs_set:Npn \phd@doc@index@pgf@
  {
    \c@IndexColumns=2%
    \cs_set:Npn \theindex
      {
        \@restonecoltrue
        \columnseprule 0pt  
        \columnsep 28\p@
        \twocolumn[\index@prologue]%
        \parindent -30pt%
        \columnsep 15pt%
        \parskip 0pt plus 1pt%
        \leftskip 30pt%
        \rightskip 0pt plus 2cm%
        \small%
        \cs_set:Npn \@idxitem{\par}%
        \let\item\@idxitem\ignorespaces
      }
    \cs_set:Npn \endtheindex{\onecolumn}%
    \cs_set:Npn \noindexing{\let\index=\@gobble}%
  }
%    \end{macrocode}
%
%  \subsection{Index heading and prologue}
%
% The index prologue is text that is entered just before the indexing entries start.
% Most indices do not have any text.
% We also need to distinguish between using a chapter type heading or a section type heading.
%
%    \begin{macrocode} 
\cs_set:Npn \phd@doc@index@pgfsection{%
  \cs_set:Npn \index@prologue
    {
      \section*{\phdkv@text@index}
      \addcontentsline{toc}{section}{\phdkv@text@index}
      \par\noindent%
   }
  \phd@doc@index@pgf@
}
%    \end{macrocode}
%    \begin{macrocode}
\cs_set:Npn \phd@doc@index@pgfchapter{%
  \cs_set:Npn \index@prologue{\ifdefined\phantomsection\phantomsection\fi    
  \@makeschapterhead{\phdkv@text@index}
  \addcontentsline{toc}{chapter}{\phdkv@text@index}}%
  \phd@doc@index@pgf@%
}
%    
\let\phd@doc@index@pgf=\phd@doc@index@pgfsection%
%    \end{macrocode}
%
%    \begin{macrocode}
\cs_set:Npn \phd@doc@index@doc
  {
    \let\phdindexcom      = \SpecialMainIndex
    \let\phd_index_env    = \SpecialMainEnvIndex
    \cxset{index german settings}
    \EnableCrossrefs
    \PageIndex
}

\cs_set:Npn \phd@doc@index@off{}%
\ExplSyntaxOff
%    \end{macrocode}
%    \begin{macrocode}
\cxset{%
  reset@documentation/.style={%
    index format=pgf,
    english language,
    documentation listing style = tcbdocumentation,
    index default settings,
    color option=Option,
    color definition=Definition,
    color hyperlink=Hyperlink,
    before example=\par\smallskip,
    after example=,
    doc left=0em,
    doc right=0pt,
    doc left indent=-2em,
    doc right indent=0pt,
    doc head=,
    doc description=,
    doc into index=true,
    index colorize = true,
    index annotate= false,
    },
%  initialize@reset=reset@documentation,
}
\cxset{reset@documentation}
%    \end{macrocode}
%
% We set the \docAuxKey{index format}=\docValue{pgf}  and the rest of the keys to the
% |german| settings that are suitable for |doc|.
%
%    \begin{macrocode}
\cxset{index format  =  pgfchapter,
       index actual={=},
       index level = {>},
       index quote = {!},
       index german settings,
       color hyperlink = thelinkcolor,  % links with color palette
       color definition =thelinkcolor,  % links with color palette
       pageshort       = {$\sigma{}$},
   }      
%\def\main#1{\underline{#1}}
%    \end{macrocode}
%
% \section {Unicode math index functions}
%
% The functions that follow typeset unicode math tables.
%
%  \begin{docCommand} {showsymbolalpha} { \marg{cmd} \marg{unicode point} \marg{note symbol} }
%    Indexes and typesets all the alphabetic letters available in math, 
%    mostly greek and the dotless j and i.
%  \end{docCommand}
%
%    \begin{macrocode}
\newcommand\showsymbolalpha[3]
  {
    \par\noindent\hangindent=3em%
    \makebox[2em][l]{$#1$} \makebox[3.5em][l]{\texttt{U+#2}} 
    \cmd{#1}$^{#3}$
    \indexmathcmd [Math alphabetics] {#1}
  }
%    \end{macrocode}

%  \begin{docCommand} {showsymbol} { \marg{cmd} \marg{unicode point} \marg{note symbol} }
%  \end{docCommand}
%    \begin{macrocode}
\newcommand\showsymbol[3]{\par\noindent\hangindent=3em%
\makebox[2em][l]{$#1$} \makebox[3.5em][l]{\texttt{U+#2}} 
\cmd{#1}$^{#3}$\indexmathcmd [Math ordinary] {#1}}
%    \end{macrocode}
%
%  \begin{docCommand} {showsymbolbin} { \marg{cmd} \marg{unicode point} \marg{note symbol} }
%  \end{docCommand}
%    \begin{macrocode}
\newcommand\showsymbolbin[3]{\par\noindent\hangindent=3em%
\makebox[2em][l]{$#1$} \makebox[3.5em][l]{\texttt{U+#2}} 
\cmd{#1}$^{#3}$\indexmathcmd [Math bin operators] {#1}}
%    \end{macrocode}
%
%  \begin{docCommand} {showrelsymbol} { \marg{cmd} \marg{unicode point} \marg{note symbol} }
%  \end{docCommand}
%    \begin{macrocode}
\newcommand\showrelsymbol[3]{\par\noindent\hangindent=3em%
\makebox[2em][l]{$#1$} \makebox[3.5em][l]{\texttt{U+#2}} 
\cmd{#1}$^{#3}$\indexmathcmd [Math relations] {#1}}
%    \end{macrocode}
%
%  \begin{docCommand} {integralsymbol} { \marg{cmd} \marg{unicode point} \marg{note symbol} }
%    Typesets and inserts into index integral symbols
%  \end{docCommand}
%    \begin{macrocode}
\newcommand\integralsymbol[3]{\par\noindent\hangindent=3em%
\makebox[2em][l]{$#1$} \makebox[3.5em][l]{\texttt{U+#2}} 
\cmd{#1}$^{#3}$\indexmathcmd [Math integrals] {#1}}
%    \end{macrocode}
%
%  \begin{docCommand} {showop} { \marg{cmd} \marg{unicode point} \marg{note symbol} }
%    Typesets and inserts into index integral symbols
%  \end{docCommand}
% 
%    \begin{macrocode}
\newcommand\showop[3]{\par\noindent\hangindent=6em%
  \makebox[5em][l]{$#1$\hfill$\displaystyle#1$\hfill}
  \makebox[3.5em][l]{\small\texttt{U+#2}} \cmd{#1}$^{#3}$ 
  \indexmathcmd [Math big operators] {#1} }
%    \end{macrocode}
%
%  \begin{docCommand} {showmbrace} { \marg{cmd} \marg{unicode point} \marg{note symbol} }
%    Typesets and inserts middle brace symbols
%  \end{docCommand}
% 
%    \begin{macrocode}
\newcommand\showmbrace[3]{\par\noindent\hangindent=6em%
  \makebox[5em][l]{${#1}{\bigm#1}{\Bigm#1}{\biggm#1}{\Biggm#1}$}
  \makebox[3.5em][l]{\small\texttt{U+#2}} \cmd{#1}$^{#3}$ 
  \indexmathcmd [Math delimiters] {#1}  }
%    \end{macrocode}
%
%  \begin{docCommand} {showlbrace} { \marg{cmd} \marg{unicode point} \marg{note symbol} }
%    left braces
%  \end{docCommand}
%    \begin{macrocode}
\newcommand\showlbrace[3]{\par\noindent\hangindent=6em%
  \makebox[5em][l]{$\Biggl#1\biggl#1\Bigl#1\bigl#1#1$}
  \makebox[3.5em][l]{\small\texttt{U+#2}} \cmd{#1}$^{#3}$
  [Math delimiters] {#1}
  }
%    \end{macrocode}
%
%  \begin{docCommand} {showrbrace} { \marg{cmd} \marg{unicode point} \marg{note symbol} }
%    right braces
%  \end{docCommand}
%    \begin{macrocode}
\newcommand\showrbrace[3]{\par\noindent\hangindent=6em%
  \makebox[5em][l]{$#1\bigr#1\Bigr#1\biggr#1\Biggr#1$}
  \makebox[3.5em][l]{\small\texttt{U+#2}} \cmd{#1}$^{#3}$
  [Math delimiters] {#1}
  }
%    \end{macrocode}
%
%  \begin{docCommand} {wide accents} { \marg{cmd} \marg{unicode point} \marg{note symbol} }
%    wide accents
%  \end{docCommand}
%
%    \begin{macrocode}
\DeclareDocumentCommand \showwideaccent { m m m} {\par\noindent\hangindent=4em%
  \makebox[3em][l]{$#1{xxx}$}\makebox[3.5em][l]{\small\texttt{U+#2}} \cmd{#1}$^{#3}$
  \indexmathcmd [Math accents] {#1{abc}}
  }
%    \end{macrocode}
%
%  \begin{docCommand} {showaccent} { \marg{cmd} \marg{unicode point} \marg{note symbol} }
%    right braces
%  \end{docCommand}
%
%    \begin{macrocode}
\DeclareDocumentCommand\showaccent { m m m} {\par\noindent\hangindent=4em%
  \makebox[3em][l]{$#1b$}\makebox[3.5em][l]{\small\texttt{U+#2}} \cmd{#1}$^{#3}$
  \indexmathcmd [Math accents] {#1 b}
  }
%    \end{macrocode}
%
%  \begin{docCommand} {showrover} { \marg{cmd} \marg{unicode point} \marg{note symbol} }
%   
%  \end{docCommand}  
%    \begin{macrocode}  
\newcommand\showover[3]{\par\noindent\hangindent=6em%
  \makebox[5em][l]{$#1{xxxxxx}$}
  \makebox[3.5em][l]{\small\texttt{U+#2}} 
  \cmd{#1}$^{#3}$
  \indexmathcmd [Math over and under brackets] {#1{xxxxxx}}
  }
%    \end{macrocode}  
% 
% \section{Miscellaneous doc commands}
% For consistency all commands that typeset their content, as well as index it and perhaps, also reference
% it have the prefix |doc|.
%
% \begin{docCommand} {docFile} { \marg{file name} }
%   Typesets and index a file. 
% \end{docCommand}
%    \begin{macrocode}
\ExplSyntaxOn
\cs_set:Npn \docFile #1
  {
    \texttt {#1}
    \index{files>#1}
  }
\cs_set:Npn \docExtension #1
  {
    \texttt {#1}
    \index{file extensions\idx@level#1}
  }  
\let\docextension\docExtension  
\ExplSyntaxOff  
%    \end{macrocode}
% 
%\section{Other Indexing functions}
%
% \begin{docCommand}{indexmany}{ \oarg{category} \marg{clist} }
% This function indexes a comma delimited list of items. It is convenient
% when you have paragraphs with a lot of terms.
% 
% \end{docCommand}
%    \begin{macrocode}
 \ExplSyntaxOn
 \DeclareDocumentCommand\indexmany {o m }
 {
   \clist_gset:Nn \indexmany: {#2} 
   \IfValueTF {#1}
    { 
      \clist_map_inline:Nn\indexmany: 
        {
          \index{#1\idx@level##1}\index{##1}
        }
    }
    { 
     \clist_map_inline:Nn\indexmany: 
      {
        \index{##1}
      } 
    }
 }
 \ExplSyntaxOff
%    \end{macrocode}
%
% \begin{docCommand} {indexboth} { \marg{arg1}  \marg{arg2} }
%  Indexes both arguments for example mathematical symbols
% \end{docCommand}
%    \begin{macrocode} 
\newcommand{\idxboth}[2]{\mbox{}\index{#1 #2}\index{#2>#1}}
\newcommand{\idxbothbegin}[2]{\mbox{}\index{#1 #2|(}\index{#2>#1|(}}
\newcommand{\idxbothend}[2]{\mbox{}\index{#1 #2|)}\index{#2>#1|)}}
\ExplSyntaxOn
\cs_gset_eq:NN \indexboth\idxboth
\cs_gset_eq:NN \indexbothbegin \idxbothbegin
\cs_gset_eq:NN \indexbothend\idxbothend
\ExplSyntaxOff
%    \end{macrocode}
% 
%  
%    \begin{macrocode}
\DeclareRobustCommand{\idxfont}[1]{\index{#1 (font)}\texttt{#1}\xspace}%
\DeclareRobustCommand{\idxlanguage}[1]{\index{#1 (script)}\index{scripts>#1}\texttt{#1}\xspace}%
%    \end{macrocode}
%
%  
%
% We define a related macro for indexing accents.  In a previous version
% of this file, \indexaccent additionally included "see also accents" in
% the index.  This became distracting so I made \indexaccent a synonym
% for \indexcommand for the time being.  Because punctuation marks can
% be problematic for makeindex, we define an \indexpunct macro that
% sorts its argument under the comparatively innocuous "\_".
%
% \begin{docCommand}{sanitize}{}
%  Delimited macro (!!!) that sanitizes macros. Classic TeXBook style.
% \end{docCommand}
%    \begin{macrocode}
\begingroup
 \catcode`\|=0
 \catcode`\\=12
 |gdef|sanitize#1#2!!!{%
   |ifx#1\%
     #2%
   |else
     #1#2%
   |fi
}
|endgroup
%    \end{macrocode}
%
%  \begin{docCommand}{indexcommand}{\oarg{}\marg{command} }
%    Index a \emph{symbol}, which may or may not begin with a \emph{backslash}.  (Is
%  there a better way to do this?)  Also, if symbol is given as an
%    optional argument is given, typeset that symbol in the index, as well
% \end{docCommand}
%
%  
%    \begin{macrocode}
\NewDocumentCommand \indexcommand { o m }  
  {
    \edef\sanitized{\expandafter\sanitize\string#2!!!}%
    %\def\first@arg{#1}%
    \IfNoValueTF{#1}
    {
       \expandafter\index\expandafter{\sanitized=\string\verb+\string#2+}%
    }
    {
       \expandafter\index\expandafter{\sanitized=\string\verb+\string#2+ (#1)}%
    }
  }
%    \end{macrocode}
%
% \indexcommand{\test}
%  
% \begin{docCommand} {indexcypriot} { \oarg{arg1} \marg{arg2} }
%    Index helper function for indexing Cypriot script. Only used
%    in the phd documentation.
% \end{docCommand}
%
%    \begin{macrocode}
\NewDocumentCommand \indexcypriot { o m }  
  {
    \edef\sanitized{\expandafter\sanitize\string#2!!!}%
    \IfNoValueTF{#1}
    {
       \expandafter\index\expandafter{Cypriot>\sanitized=\string\verb+\string#2+}%
    }
    {
       \expandafter\index\expandafter{Cypriot>\sanitized=\string\verb+\string#2+ (#1)}%
    }
  }
%    \end{macrocode}  
%
% \begin{docCommand} {indexstaves} { \oarg{arg1} \marg{arg2} }
%    Index helper function for indexing Icelandic staves. Only used
%    in the phd documentation.
% \end{docCommand}
%
%    \begin{macrocode}
\NewDocumentCommand \indexstaves { o m }  
  {
    \edef\sanitized{\expandafter\sanitize\string#2!!!}%
    \IfNoValueTF{#1}
    {
       \expandafter\index\expandafter{Staves>\sanitized=\string\verb+\string#2+}%
    }
    {
       \expandafter\index\expandafter{Staves>\sanitized=\string\verb+\string#2+ (#1)}%
    }
  }
%    \end{macrocode} 
% 
%  
% \begin{docCommand} {indexlinearb} { \oarg{arg1} \marg{arg2} }
%    Index helper function for indexing the linearb script. Only used
%    in the phd documentation.
% \end{docCommand}
%  
%    \begin{macrocode}
\NewDocumentCommand \indexlinearb { o m }  
  {
    \edef\sanitized{\expandafter\sanitize\string#2!!!}%
    \IfNoValueTF{#1}
    {
       \expandafter\index\expandafter{Linear B>\sanitized=\string\verb+\string#2+}%
    }
    {
       \expandafter\index\expandafter{Linearb>\sanitized=\string\verb+\string#2+ (#1)}%
    }
  }
%    \end{macrocode} 
%
% \begin{docCommand} {indexugar} { \oarg{arg1} \marg{arg2} }
%    Index helper function for indexing Ugaritic scripts. Only used
%    in the phd documentation.
% \end{docCommand}
% 
%    \begin{macrocode}
\NewDocumentCommand \indexugar { o m }  
  {
    \edef\sanitized{\expandafter\sanitize\string#2!!!}%
    \IfNoValueTF{#1}
    {
       \expandafter\index\expandafter{Ugarite>\sanitized=\string\verb+\string#2+}%
    }
    {
       \expandafter\index\expandafter{Ugarite>\sanitized=\string\verb+\string#2+ (#1)}%
    }
  }
%    \end{macrocode} 

% \begin{docCommand} {indexoldpersian} { \oarg{arg1} \marg{ag2} }
%   Indexing and doc command for Old Persian tables.
% \end{docCommand}
%
%    \begin{macrocode}
\NewDocumentCommand \indexoldpersian { o m }  
  {
    \edef\sanitized{\expandafter\sanitize\string#2!!!}%
    \IfNoValueTF{#1}
    {
       \expandafter\index\expandafter{Old Persian>\sanitized=\string\verb+\string#2+}%
    }
    {
       \expandafter\index\expandafter{Old Persian>\sanitized=\string\verb+\string#2+ (#1)}%
    }
  }
%    \end{macrocode} 
%
% \begin{docCommand} {indexsoutharabian} { }
%    Indexing and doc command for symbols tables.
% \end{docCommand}
%
%    \begin{macrocode}
\NewDocumentCommand \indexsoutharabian { o m }  
  {
    \edef\sanitized{\expandafter\sanitize\string#2!!!}%
    \IfNoValueTF{#1}
    {
       \expandafter\index\expandafter{South Arabian>\sanitized=\string\verb+\string#2+}%
    }
    {
       \expandafter\index\expandafter{South Arabian>\sanitized=\string\verb+\string#2+ (#1)}%
    }
  }
%    \end{macrocode} 
%
% \section{Indexing mathematical symbols}
%
% The currently available fonts 
% The following indexing commands are auxiliary commands to
% index unicode symbols for maths. 
% \tcbdocmarginnote{26-06-2015}
%    \begin{macrocode}
\NewDocumentCommand \indexmathcmd { o m }  
  {
    \edef\sanitized{\expandafter\sanitize\string#2!!!}%
    \IfNoValueTF{#1}
    {
       \expandafter\index\expandafter{#1>\sanitized=\string\verb+\string#2+
       ($#2$)}
       % put command also
      \expandafter\index\expandafter{\string#1=\string\verb+\string#2+ ($\string#2$)*}%
    }
    {
      \expandafter\index\expandafter{#1>\sanitized=\string\verb+\string#2+ ($#2$)}%
      \expandafter\index\expandafter{\string#1=\string\verb+\string#2+ ($\string#2$)}%
    }
  }
%    \end{macrocode} 
%
% \begin{docCommand}{indexaccent}{}
%   Syntactic sugar identical to \refCom{indexcommand}
% \end{docCommand}
%
%    \begin{macrocode}
\ExplSyntaxOn
\cs_gset_eq:NN \indexaccent\indexcommand
\cs_new:Npn \CLSLpipe {|}
\ExplSyntaxOff  
%    \end{macrocode}
%   
%
% \begin{docCommand} {indexpunct} { \oarg {arg1} \marg{arg2}} 
%   Indexing punctuation marks for latin scripts.
% \end{docCommand}
%
%    \begin{macrocode}
  \newcommand{\indexpunct}[2][]{%
    \def\first@arg{#1}%
    \def\second@arg{#2}%
    \ifx\first@arg\@empty
      \ifx\second@arg\CLSLpipe
        \index{_=\magicvertname}%
      \else
        \index{_=\string\verb+\string#2+}%
      \fi
    \else
      \ifx\second@arg\CLSLpipe
        \index{_=\magicvertname{} (#1)}%
      \else
        \index{_=\string\verb+\string#2+ (#1)}%
      \fi
    \fi
  }
%    \end{macrocode}
% 
%
%    \begin{macrocode}
%\usepackage{longdiv}
\newcommand\FC{\pkgname{fc}}
\newcommand\VIET{\pkgname{vietnam}}
%\newcommand\ABX{\pkgname{mathabx}}
%    \end{macrocode}
%

% \begin{docCommand} {incsyms} { \meta{void}}
%  We define an integer counter to keep track of all the symbols we load
%  and list.\footnote{Unicode characters are counted separately}
%  These are symbols which can be produced using command sequences.
% \end{docCommand}
% Define a counter to keep track of how many symbols are listed.
% Output this counter to the log file at the end of each run.
% Define |\prevtotalsymbols| to be the total number of symbols from
% the previous run.
%   
%    \begin{macrocode}
\ExplSyntaxOn
  \int_new:c {totalsymbols}
  \cs_new:Npn \incsyms { \int_gincr:c {totalsymbols} }
  \cs_new:Npn \thetotalsymbols {\int_use:c {totalsymbols} }
\ExplSyntaxOff
%    \end{macrocode}
%
% \begin{docCommand}{graybox} { \meta{void}}
% \end{docCommand}
%    \begin{macrocode}
\newcommand*{\graybox}{\textcolor{thegray!60}{\rule[-\adp]{\awd}{\aht}}}
 
% Define |\blackacc| to display an accented box, given an accent command.
% Define |\blackacchack| to display an accented "a" and then black out
% the "a".
\newlength\awd
\newlength\aht
\newlength\adp
\settowidth{\awd}{\normalfont m}
\settoheight{\aht}{\normalfont a}
\settodepth{\adp}{\normalfont m}
\advance\adp by 0.06pt    % In Computer Modern, "a" extends slightly below its bounding box.
\advance\aht by \adp


\gdef\blackacchack#1{#1a\llap{\graybox}}
\gdef\blackacc#1{#1{\graybox}}
\gdef\blackacctwo#1{#1{\graybox}{\graybox}}
%    \end{macrocode}
% 
% 
%
% Symbol+verbatim for various types of symbols
%    \begin{macrocode}
\def\E#1{%
  \begingroup
    \lccode`|=`\\
    \def\EStruename{ES#1T}
    \lowercase{\incsyms\index{#1=\string\verb+\string|#1+ (\string|\EStruename)}}
  \endgroup
  \csname ES#1T\endcsname 
  & \csname ES#1D\endcsname 
  &
  \ttfamily\expandafter\string\csname#1\endcsname
}
%    \end{macrocode}
%    
% \subsection{Indexing archaic symbols}  
% 
% These commands are here to be able to index these symbols for the index and to typeset
% them in the symbols appendix.
% 
% \begin{docCommand} {Kcyp} {\oarg{text cmd} \marg{symbol command}}
%   Indexes and prints the Cypriot archaic font symbols.
%   
% \begin{verbatim}
% \Kcyp[\textcypr{\Ca}]\Ca
% \end{verbatim}
% 
% \end{docCommand}
%    \begin{macrocode}
\def\Kcyp@opt@arg[#1]#2{\incsyms\indexcypriot[\textcypr{#1}]{#2}#1 &\ttfamily\string#2}
\def\Kcyp@no@opt@arg#1{\incsyms\indexcypriot[\textcypr{#1}]{#1}#1 &\ttfamily\string#1}
\def\Kcyp{\@ifnextchar[{\Kcyp@opt@arg}{\Kcyp@no@opt@arg}}
%    \end{macrocode}
%    
% \begin{docCommand} {Kstav} { \oarg{cmd} \marg{stave cmd}}      
%   Indexes and prints an Icelandic  stave. 
% \end{docCommand}
%    \begin{macrocode}
\ExplSyntaxOn

\cs_set:Npn \Kstav_opt_arg [#1]#2
  {
    \incsyms\indexstaves[#1]{#2}# 1 &\ttfamily\string#2
  }
 
\cs_set:Npn \Kstav_no_opt_arg #1
  {
    \incsyms\indexstaves[#1]{#1}#1 &\ttfamily\string#1
  }

\NewDocumentCommand\Kstav {o m} {
  \IfNoValueTF {#1} 
    {
      \Kstav_no_opt_arg {#2}
    }
    {
      \Kstav_opt_arg [#1] {#2}
    }
}
\ExplSyntaxOff
%    \end{macrocode}
%    
% \begin{docCommand}{K} { \oarg{} \marg{cmd} }    
%    Adds a symbol cmd to a table and the index.
% \end{docCommand}
%
%    \begin{macrocode}
\ExplSyntaxOn
\cs_set:Npn \K@opt@arg#1#2 
   {
      \incsyms
      \indexcommand[#1]{#2}#1 &\ttfamily\string#2
   }
   
\cs_set:Npn \K@no@opt@arg#1
  {
    \incsyms\indexcommand[#1]{#1}#1 &\ttfamily\string#1
  }

\NewDocumentCommand {\K} { o m } 
{
  \IfNoValueTF {#1} { \K@no@opt@arg {#2} } {\K@opt@arg {#1}{#2}}
}
\ExplSyntaxOff
%    \end{macrocode}
%    
%    \begin{macrocode}
\def\Kp#1{\incsyms\indexpunct[$#1$]{#1}#1 &\ttfamily\string#1}

\def\KED[#1][#2][#3]#4{\incsyms\indexcommand[#1]{#2}#3 &\ttfamily\string#4}
\def\Kfeyn#1{\incsyms\indexcommand[\string\feyn{#1}]{\feyn{#1}}\feyn{#1} &\ttfamily\string\feyn\string{\string#1\string}}

\def\Kp#1{\incsyms\indexpunct[$#1$]{#1}#1 &\ttfamily\string#1}

\def\Kpig#1{\incsyms\index{pigpenfont #1=\string\verb+{\string\pigpenfont\space#1}+\space(\string\CLSLpig{#1})}\CLSLpig{#1} &\ttfamily\string{\string\pigpenfont\space\string#1\string}}
%    \end{macrocode}
%
%  \begin{docCommand} {Ks} { \marg{cmd} }
%    Ks index and doc command, asterisk for note that is not available in |OT1|, as
%    superscript.
%  \end{docCommand}
%
%    \begin{macrocode}
\ExplSyntaxOn
  \cs_set:Npn \Ks #1
    {
      \incsyms
      \indexcommand[\string\encone{\string#1}] {#1}
      { \encone{#1} }  & \ttfamily\string#1$^*$
    }
\ExplSyntaxOff
%    \end{macrocode}
%
% This macro is also from the comprehensive and takes
% the symbol command as its only argument. It provides
% |T1| encoding and also adds the command to the index.
% 
%    \begin{macrocode}
\ExplSyntaxOn   
\cs_set:Npn \Kt #1
  {
    \incsyms
    \indexcommand[\string\encone{\string#1}] {#1}
    {\encone{#1}} & \ttfamily \string #1
  }
\ExplSyntaxOff  
%    \end{macrocode}
%
%  \begin{docCommand} {Kv} { \marg{cmd} }
%    T5 encoding
%  \end{docCommand}
%
%    \begin{macrocode}
\def\Kv#1{\incsyms\indexcommand[\string\encfive{\string#1}]{#1}{\encfive{#1}} &\ttfamily\string#1}
%    \end{macrocode}
%
%    \begin{macrocode}
\def\Kgr@opt@arg[#1]#2{\incsyms\indexcommand[\string\encgreek{\string#1}]{#2}{\encgreek{#1}} &\ttfamily\string#2}
  \def\Kgr@no@opt@arg#1{\incsyms\indexcommand[\string\encgreek{\string#1}]{#1}{\encgreek{#1}} &\ttfamily\string#1}
 
\def\Kgr{\@ifnextchar[{\Kgr@opt@arg}{\Kgr@no@opt@arg}}
\def\KN[#1][#2]#3{\incsyms\indexcommand[\string#1]{#3} #1 & #2 & \ttfamily\string#3}
\def\KNbig[#1][#2]#3{\incsyms\indexcommand[\string#2]{#3} #1 & #2 & \ttfamily\string#3}

\def\Knoidx#1{\incsyms#1 &\ttfamily\string#1}
%    \end{macrocode}
%
% \begin{docCommand}{N} {}
%   Big delimiters auxiliary command for doc and index. 
% \end{docCommand}
%
%    \begin{macrocode}
\ExplSyntaxOn
\cs_set:Npn \N@opt@arg #1 #2 
  {
    \incsyms
    \indexcommand[$\string#1$]{#2}
    $#1$ & $\Big#1$ &\ttfamily\string#2
  }

\cs_set:Npn \N@no@opt@arg#1 
  {
    \incsyms\indexcommand[$\string#1$]{#1}
    $#1$ & $\Big#1$ &\ttfamily\string#1
  }
  
\NewDocumentCommand {\N} { o m } 
  {
    \IfNoValueTF {#1} 
      { \N@no@opt@arg {#2}  }
      { \N@opt@arg {#1}{#2} }
  }

\ExplSyntaxOff  
%    \end{macrocode}  
%    \begin{macrocode}  
  \def\Nn[#1]#2{%
    \incsyms\indexcommand[$\string\nathdouble\string#1$]{#2}%
    $\nathdouble#1$ & $\nathdouble{\Big#1}$ & \ttfamily\string#2}
  \def\Nnt#1[#2]#3{%
    \incsyms\indexcommand{\triple}%
    $\nathtriple#2$ & $\nathtriple{\Big#2}$ &
    \ttfamily\expandafter\string\csname#1triple\endcsname\string#3}
  \def\Np@opt@args[#1]{\@ifnextchar[{\Np@two@opt@args[#1]}{\Np@one@opt@arg[#1]}}
  \def\Np@two@opt@args[#1][#2]#3{\incsyms\index{_=\string#2{} ($\string#1$)}$#1$ & $\Big#1$ &\ttfamily\string#3}
  \def\Np@one@opt@arg[#1]#2{\incsyms\indexpunct[$\string#1$]{#2}$#1$ & $\Big#1$ &\ttfamily\string#2}
  \def\Np@no@opt@args#1{\incsyms\indexpunct[$\string#1$]{#1}$#1$ & $\Big#1$ &\ttfamily\string#1}
  \def\Np{\@ifnextchar[{\Np@opt@args}{\Np@no@opt@args}}
  \def\Nbig[#1]#2{\incsyms\indexcommand[$\string\Big\string#1$]{#2}$#1$ & $\Big#1$ &\ttfamily\string#2}
%    \end{macrocode}
%
%  \begin{docCommand} {Q} {}
%  \end{docCommand}
%
%    \begin{macrocode}  
\ExplSyntaxOn
\cs_set:Npn \Q@opt@arg#1#2
  {
    \incsyms\indexaccent[\string\blackacchack{\string#1}]{#2}#1{A}#1{a} &
           \ttfamily\string#2\string{A\string}\string#2\string{a\string}
  }
  
\cs_set:Npn \Q@no@opt@arg#1
  {
    \incsyms\indexaccent[\string\blackacchack{\string#1}]{#1}#1{A}#1{a} &
    \ttfamily\string#1\string{A\string}\string#1\string{a\string}
  }
           
\NewDocumentCommand {\Q} { o m }
  {
    \IfNoValueTF {#1}
      { \Q@no@opt@arg {#2} }
      { \Q@opt@arg    {#1}{#2} }
  }
\ExplSyntaxOff  
%    \end{macrocode}
%
%    \begin{macrocode}
\def\Qc#1{\incsyms\indexaccent[\string\blackacc{\string#1}]{#1}#1{A}#1{a} &
         \ttfamily\string#1\string{A\string}\string#1\string{a\string}}
%    \end{macrocode}

%    \begin{macrocode}         
\def\Qe[#1][#2]#3{%
  \incsyms\incsyms\index{_=\string#2{} (\string\blackacchack{\string#1})}%
  #3{A}#3{a} &
  \ttfamily\string#3\string{A\string}\string#3\string{a\string}}
%    \end{macrocode}
%
%    \begin{macrocode}  
\def\Qt#1{\incsyms\indexaccent[\string\encone{\string\blackacc{\string#1}}]{#1}{\encone{#1{A}#1{a}}} &
          \ttfamily\string#1\string{A\string}\string#1\string{a\string}}
%    \end{macrocode}
%    \begin{macrocode}
\def\Qpc#1#2{\incsyms\indexcommand{#2}{\raisebox{1pt}{\tiny[#1]}} &
             \ttfamily\string#2\string{A\string}\string#2\string{a\string}}
%    \end{macrocode}
%
%    \begin{macrocode}             
\def\Qpfc[#1]#2{\incsyms\indexaccent[\string\encfour{\string\blackacchack{\string#1}}]{#2}\encfour{#1{A}#1{a}} &
           \ttfamily\string#2\string{A\string}\string#2\string{a\string}}
%    \end{macrocode}
%    \begin{macrocode}
\newif\ifFC\FCfalse
\ifFC
  \def\Qiv#1#2{\incsyms\indexaccent[\string\encfour{\string\blackacchack{\string#1}}]{#1}\encfour{#1{A}#1{a}} &
               \ttfamily\string#1\string{A\string}\string#1\string{a\string}$^#2$}
               
  \def\QivBAR#1{\incsyms\index{_=\string\magicVertname{}
                (\string\encfour{\string\blackacchack{\string\FCbar}})}
                \encfour{\FCbar{A}\FCbar{a}} &
                \ttfamily\string\|\string{A\string}\string\|\string{a\string}$^#1$}
\else
  \def\Qiv#1#2{\Qpc{T4}{#1}$^#2$}
  \def\QivBAR#1{\Qpc{T4}{\|}$^#1$}
\fi
%    \end{macrocode}
%    \begin{macrocode}
\newif\ifVIET\VIETfalse
\ifVIET
  \def\Qv#1#2{\incsyms\indexaccent[\string\encfive{\string\blackacchack{\string#1}}]{#1}{\encfive{#1{A}#1{a}}} &
              \ttfamily\string#1\string{A\string}\string#1\string{a\string}$^#2$}
\else
  \def\Qv#1#2{\Qpc{T5}{#1}$^#2$}\def\Qv#1#2{Err}%TODO
\fi
%    \end{macrocode}
%
% \begin{docCommand}{R} { \oarg{ams cmd} \marg {cmd} }
%   Used for variable size math operators
% \end{docCommand}
%
%    \begin{macrocode}
\ExplSyntaxOn
\cs_set:Npn \R@opt@arg#1#2
  {
    \incsyms
    \indexcommand[$\string#1$]{#2}
     $#1$ & $\displaystyle#1$ &\ttfamily\string#2
  }
  
\cs_set:Npn \R@no@opt@arg#1
  {
    \incsyms
    \indexcommand[$\string#1$]{#1}
    $#1$ & $\displaystyle#1$ &\ttfamily\string#1
  }

\NewDocumentCommand {\R} { o m}
  {
    \IfNoValueTF {#1}
      { \R@no@opt@arg {#2}      }
      { \R@opt@arg    {#1} {#2} }
  }
\ExplSyntaxOff
%% T commands
%    \end{macrocode}
%
% \begin{docCommand}{indexDing} { \marg{ ding symbol number }}
%   Auxiliary function to index and print in a table ding symbols. originally
%   from Comprehensive.
% \end{docCommand}
%
%    \begin{macrocode}
\ExplSyntaxOn
\newcommand \indexDing [1] 
  {
    \incsyms
    \indexcommand{\ding}
    \ding{#1} & 
    \ttfamily\string\ding \string{#1\string}
  }
\ExplSyntaxOff
%    \end{macrococode}
%
%    \begin{macrocode}
\def\Tm#1{\incsyms\indexcommand{\maya}$\mayadigit{#1}$ &\ttfamily\string\maya\string{#1\string}}
\def\Tmoon#1{\incsyms\indexcommand{\MoonPha}\MoonPha{#1} &\ttfamily\string\MoonPha\string{#1\string}}
%    \end{macrocode}
%
% \begin{docCommand}{indexTextcomp} {\oarg{ltx cmd} \marg{symbol arg}}
%   This command typesets its command argument in a table row of two
%   (used for textcomp symbols).
% \end{docCommand}  
% 
%    \begin{macrocode}
\newcommand{\indexTextcomp}[2][]{%
   \incsyms#1 & 
   \indexcommand[#2]{#2}% necessary to put symbol \text
   #2%  
   &\ttfamily\string#2
}
%    \end{macrocode}
%
% \begin{docCommand} {Vp} {}
%  Commands that work both in math and text mode
% \end{docCommand}
%    \begin{macrocode}   
\newcommand{\Vp}[2][]{\incsyms#1 & \indexpunct[$#2$]{#2}#2 &\ttfamily\string#2}
%    \end{macrocode}
%
%    \begin{macrocode}
\def\W@opt@arg[#1]#2#3{%
    \incsyms\indexaccent[$\string\blackacc{\string#1}$]{#2}%
    $#1{#3}$ &\ttfamily\string#2\string{#3\string}}

\def\W@no@opt@arg#1#2{%
    \incsyms\indexaccent[$\string\blackacc{\string#1}$]{#1}%
    $#1{#2}$ &\ttfamily\string#1\string{#2\string}}
    
\def\W{\@ifnextchar[{\W@opt@arg}{\W@no@opt@arg}}
%    \end{macrocode}
%
%    \begin{macrocode}
\def\Wf#1#2{\incsyms\indexcommand{#1}$#1{#2}$ &\ttfamily\string#1\string{#2\string}}
\def\Ww#1#2#3{\incsyms\indexcommand{#2}$#1{#3}$ &\ttfamily\string#2\string{#3\string}}
\def\Wul#1#2#3{%
  \incsyms\indexaccent[$\string\blackacctwo{\string#1}$]{#1}%
  $#1{#2}{#3}$ &\ttfamily\string#1\string{#2\string}\string{#3\string}}
%    \end{macrocode}

% \begin{docCommand}{X} { \oarg{command} \marg{command} }
%   Typesets its arguments as commands and also the resulting symbol in 
%   math. Used for symbol tables in the documentation.
% \end{docCommand}
%
% \tcbdocmarginnote{U 25-6-2015}
%    \begin{macrocode}
\def\X@opt@arg#1#2 {\incsyms\indexcommand[$\string#1$]{#2}$#1$ &\ttfamily\string#2}
\def\X@no@opt@arg#1{\incsyms\indexcommand[$\string#1$]{#1}$#1$ &\ttfamily\string#1}

%\def\X{\@ifnextchar[{\X@opt@arg}{\X@no@opt@arg}}

\NewDocumentCommand {\X} { o m}
  {
    \IfNoValueTF{#1}
      { \X@no@opt@arg  {#2}   }
      { \X@opt@arg {#1} {#2}  }
  }
%    \end{macrocode}

%    \begin{macrocode}
\def\Y#1{\incsyms\indexcommand[$\string\big\string#1$]{#1}$\big#1$ & $\Bigg#1$ &\ttfamily\string#1}
%    \end{macrocode}
%
% \begin{docCommand} {Z} { \marg{arg1} }
%  Typesets and index its arguments.
% \end{docCommand}
%    \begin{macrocode}
\ExplSyntaxOn
\cs_set:Npn \Z #1
  {
    \incsyms
    \indexcommand[$\string#1$] {#1}
    \ttfamily
    \string #1
  }
\ExplSyntaxOff
%    \end{macrocode}
% \begin{docCommand} {utfviii}  { \meta{void} }
%  Typesets UTF-8.
% \end{docCommand}
%    \begin{macrocode}
\newcommand{\utfviii}{\mbox{UTF-8}\index{UTF-8}\xspace}

% Index TeXbook symbols and the CTAN repository.
\newcommand{\idxTBsyms}{%
  \index{symbols>TeXbook=\TeX{}book}% 
  \index{TeXbook, The=\TeX{}book, The>symbols from}%
}
%    \end{macrocode}
%
% \begin{docCmd}{pkgname}{ \marg{package name}}
% Typesets and indexes a \latex package.
% \end{docCmd}
%    \begin{macrocode}
\newcommand{\pkgname}[1]{%
  \href{http://ctan.org/pkg/#1}{\bfseries{#1}}%
  \index{#1=\texttt{#1} (package)}%
  \index{packages>#1=\texttt{#1}}
 }
 
\let\pkg\pkgname
\let\Lpack\pkgname

\newcommand*\opt[1]{\texttt{#1}}

\newcommand*\feat[1]{\texttt{#1}}


\newcommand{\optname}[2]{%
  \textsf{#2}%
  \index{#2=\textsf{#2} (\textsf{#1} package option)}%
  \index{package options>#2=\textsf{#2} (\textsf{#1})}}
%    \end{macrocode}
%
%    \begin{macrocode}
\newcommand{\docClass}[1]{%
  \href{http://ctan.org/pkg/#1}{\bfseries{#1}}%
  \index{#1=\texttt{#1} (class)}%
  \index{classes>#1=\texttt{#1}}}
\let\pkg\pkgname
\let\Lpack\pkgname
%    \end{macrocode}
% 
% This macro and all similar macros starting from doc
% typeset their argument and also add the argument to the 
% index.
%
% \begin{docCmd}{docfilename}{ \marg{file name}}
% Typesets and indexes a file name.
% \end{docCmd}
%    \begin{macrocode}
\newcommand{\docfilename}[1]{%
  \texttt{#1}
  \index{#1=\phdindexprintcomc{#1}(file)}}
  

\let\docFilename\docfilename  
%    \end{macrocode}
% 
% 
% \begin{docCmd}{docfileextension}{ \marg{file extension}}
% Typesets and indexes a file extension, such as \refCmd{docfileextension}\marg{.tex}  (\docfileextension{.tex}). You type
% the dot if you want it to appear in the index, which is a good idea.
% \end{docCmd}
%    \begin{macrocode}
\newcommand{\docfileextension}[1]{%
  \texttt{#1}%
  \index{#1=\texttt{#1} (file extension)}}
   \index{#1=\texttt{#1}}
   
\newcommand{\PSfont}[1]{%
  #1%
  \index{#1 (font)}%
  \index{fonts\index@level#1}%
}
%    \end{macrocode}
% 
%    \begin{macrocode}
\NewDocumentCommand{\person} { m m } {#1\index{#2, #1} #2}
%    \end{macrocode}
%
% \begin{docCommand}{ctan}{\marg{package name}}
% Provides a link to the ctan package repository
% \end{docCommand}
%    \begin{macrocode}
\DeclareRobustCommand\ctan[1]{%
  \textcolor{green}{%
      \href{http://www.ctan.org/pkg/#1} {{\bfseries #1}}%
  \footnote{\protect\url{http://www.ctan.org/pkg/#1}}}
  \index{Packages>#1}%
}
%    \end{macrocode}
%    \begin{macrocode}
\newcommand{\idxCTAN}{%
  \index{Comprehensive TeX Archive Network=Comprehensive \string\TeX{} Archive Network}}
% Typeset a string in various encodings.
\newcommand{\encone}[1]{{\fontencoding{T1}\selectfont#1}}
\newcommand{\encfour}[1]{{\fontencoding{T4}\selectfont#1}}
\newcommand{\encfive}[1]{{\fontencoding{T5}\selectfont#1}}
\newcommand{\encgreek}[1]{{\fontencoding{LGR}\selectfont#1}}

% Various punctuation marks confuse makeindex when used directly.
\let\magicrbrack=]
\let\magicequal=\=
\DeclareRobustCommand{\magicequalname}{\texttt{\string\=}}
\DeclareRobustCommand{\magicvertname}{\texttt{|}}
\DeclareRobustCommand{\magicVertname}{\texttt{\string\|}}

% Vertically center a text-mode symbol.
\newsavebox{\tvcbox}
\newcommand*{\textvcenter}[1]{%
  \savebox{\tvcbox}{#1}%
  \raisebox{0.5\dp\tvcbox}{\raisebox{-0.5\ht\tvcbox}{\usebox{\tvcbox}}}%
}
% Many tables have notes beneath them.  Define an environment in which to
% display such a note, with an optional, superscripted math symbol
% preceding it.
\newenvironment{tablenote}[1][]{
  \makebox[1em]{\ensuremath{^{#1}}}%
  \begin{minipage}[t]{0.75\textwidth}%
  \setlength{\parskip}{2ex}
}{%
  \end{minipage}%
}

% Define various messages we reuse repeatedly.
\newcommand{\twosymbolmessage}{%
  \begin{tablenote}
    Where two symbols are present, the left one is the ``faked'' symbol
    that \latexe provides by default, and the right one is the ``true''
    symbol that \TC\ makes available.
  \end{tablenote}
}

\newcommand{\notpredefinedmessage}{%
  \begin{tablenote}[*]
    Not predefined in \latexe.  Use one of the packages
    \pkgname{latexsym}, \pkgname{amsfonts}, \pkgname{amssymb},
    \pkgname{txfonts}, \pkgname{pxfonts}, or \pkgname{wasysym}.
  \end{tablenote}
}

\newcommand{\notpredefinedmessageABX}{%
  \begin{tablenote}[*]
    Not predefined in \latexe.  Use one of the packages
    \pkgname{latexsym}, \pkgname{amsfonts}, \pkgname{amssymb},
    \pkgname{mathabx}, \pkgname{txfonts}, \pkgname{pxfonts}, or
    \pkgname{wasysym}.
  \end{tablenote}
}

\newcommand{\usetextmathmessage}[1][]{%
  \begin{tablenote}[#1]
    It's generally preferable to use the corresponding symbol from
    \vref{math-text} because the symbols in that table work
    properly in both text mode and math mode.
  \end{tablenote}
}



\newcommand{\usefontcmdmessage}[2]{%
  These symbols must appear either within the argument to \cmd{#1} or
  following the \cmd{#2} font-selection command within a scope%
}
% Define an environment in which to write a single table of symbols.  The
% environment looks a lot like a table, but it doesn't float, and it gets
% an entry in the table of contents as opposed to the list of tables.
%
% The first argument is a conditional.  The table will appear only if
% the value of the conditional is true.  The second argument is the
% table's caption.

\def\fnum@table{\tablename~\thetable}

\newenvironment{symtable}[2][true]{%
  \bgroup
  \expandafter\global\expandafter\let%
    \expandafter\ifshowsymtable\csname if#1\endcsname
  \ifshowsymtable
    \noindent%
    \begin{minipage}[t]{\linewidth}    % Prevent page breaks
    \begin{center}
    \refstepcounter{table}%
    \phantomsection
    \addcontentsline{toc}{subsection}{%
      \protect\numberline{\tablename~\thetable:}{#2}}%
    \@makecaption{\fnum@table}{#2}\medskip
    \let\next=\relax
  \else
    % The following was taken verbatim from verbatim.sty.
    \let\do\@makeother\dospecials\catcode`\^^M\active
    \let\verbatim@startline\relax
    \let\verbatim@addtoline\@gobble
    \let\verbatim@processline\relax
    \let\verbatim@finish\relax
    \let\next=\verbatim@
  \fi
  \next
}{%
  \ifshowsymtable
    \end{center}
    \end{minipage}
    \vskip 8ex minus 2ex
  \fi
  \egroup
}

\newenvironment{nonsymtable}[1]{%
  \begin{table}[htbp]
  \centering
  \caption{#1}\medskip
}{%
  \end{table}
}
%    \end{macrocode}
%
%    \begin{macrocode}
{
  \global\let\myempty=\@empty
  \global\let\mygobble=\@gobble
  \catcode`\@=12
  \gdef\getridofats#1@#2\relax{%
    \def\getridtest{#2}%
    \ifx\getridtest\myempty%
      \expandafter\def\expandafter\strippedat\expandafter{\strippedat#1}
    \else%
      \expandafter\def\expandafter\strippedat\expandafter{\strippedat#1\protect\printanat}
      \getridofats#2\relax%
    \fi%
  }

  \gdef\removeats#1{%
    \let\strippedat\myempty%
    \edef\strippedtext{\stripcommand#1}%
    \expandafter\getridofats\strippedtext @\relax%
  }
  
  \gdef\stripcommand#1{\expandafter\mygobble\string#1}
}


\def\printanat{\char`\@}

\def\declare{\afterassignment\pgfmanualdeclare\let\next=}
\def\pgfmanualdeclare{\ifx\next\bgroup\bgroup\color{red!75!black}\else{\color{red!75!black}\next}\fi}


\let\textoken=\command
\let\endtextoken=\endcommand

\def\myprintocmmand#1{\texttt{\char`\\#1}}

\def\example{\par\smallskip\noindent\textit{Example: }}
\def\themeauthor{\par\smallskip\noindent\textit{Theme author: }}


\def\indexoption#1{%
  \index{#1@\protect\texttt{#1} option}%
  \index{Graphic options and styles!#1@\protect\texttt{#1}}%
}

\def\itemcalendaroption#1{\item \declare{\texttt{#1}}%
  \index{#1@\protect\texttt{#1} date test}%
  \index{Date tests!#1@\protect\texttt{#1}}%
}
%    \end{macrocode}
% \begin{docEnvironment}{class}{\marg{class}}
% \end{docEnvironment}
%
% \begin{class}{{article}{[10pt,oneside]}}
% \end{class}
%    \begin{macrocode}
\def\class#1{%
  \list{}% 
    {\leftmargin=2em\itemindent-\leftmargin\def\makelabel##1{\hss##1}}%
   \extractclass#1@\par\topsep=0pt
}
\def\endclass{\endlist}

\def\extractclass#1#2@{%
\item{{{\ttfamily\char`\\documentclass}#2{\ttfamily\char`\{\declare{#1}\char`\}}}}%
  \index{#1@\protect\texttt{#1} class}%
  \index{Classes!#1@\protect\texttt{#1}}}



\def\index@prologue{\section*{Index}\addcontentsline{toc}{section}{My Index}
  This index only contains automatically generated entries. A good
  index should also contain carefully selected keywords. This index is
  not a good index.
  \bigskip
}
\@ifundefined{c@IndexColumns}{\newcount\c@IndexColumns}{}
\c@IndexColumns=2
  \def\theindex{\@restonecoltrue
    \columnseprule \z@  \columnsep 29\p@
    \twocolumn[\index@prologue]%
       \parindent -30pt
       \columnsep 15pt
       \parskip 0pt plus 1pt
       \leftskip 30pt
       \rightskip 0pt plus 2cm
       \small
       \def\@idxitem{\par}%
    \let\item\@idxitem \ignorespaces}
  \def\endtheindex{\onecolumn}
\def\noindexing{\let\index=\@gobble}



\newcommand\symarrow[1]{
  \index{#1@\protect\texttt{#1} arrow tip}%
  \index{Arrow tips!#1@\protect\texttt{#1}}
  \texttt{#1}& yields thick  
  \begin{tikzpicture}[arrows={#1-#1},thick,baseline]
    \useasboundingbox (0pt,-0.5ex) rectangle (1cm,2ex);
    \draw (0pt,.5ex) -- (1cm,.5ex);
  \end{tikzpicture} and thin
  \begin{tikzpicture}[arrows={#1-#1},thin,baseline]
    \useasboundingbox (0pt,-0.5ex) rectangle (1cm,2ex);
    \draw (0pt,.5ex) -- (1cm,.5ex);
  \end{tikzpicture}
}

\newcommand\sarrow[2]{
  \index{#1@\protect\texttt{#1} arrow tip}%
  \index{Arrow tips!#1@\protect\texttt{#1}}
  \index{#2@\protect\texttt{#2} arrow tip}%
  \index{Arrow tips!#2@\protect\texttt{#2}}
  \texttt{#1-#2}& yields thick  
  \begin{tikzpicture}[arrows={#1-#2},thick,baseline]
    \useasboundingbox (0pt,-0.5ex) rectangle (1cm,2ex);
    \draw (0pt,.5ex) -- (1cm,.5ex);
  \end{tikzpicture} and thin
  \begin{tikzpicture}[arrows={#1-#2},thin,baseline]
    \useasboundingbox (0pt,-0.5ex) rectangle (1cm,2ex);
    \draw (0pt,.5ex) -- (1cm,.5ex);
  \end{tikzpicture}
}

\newcommand\carrow[1]{
  \index{#1@\protect\texttt{#1} arrow tip}%
  \index{Arrow tips!#1@\protect\texttt{#1}}
  \texttt{#1}& yields for line width 1ex
  \begin{tikzpicture}[arrows={#1-#1},line width=1ex,baseline]
    \useasboundingbox (0pt,-0.5ex) rectangle (1.5cm,2ex);
    \draw (0pt,.5ex) -- (1.5cm,.5ex);
  \end{tikzpicture}
}
\def\myvbar{\char`\|}
\newcommand\plotmarkentry[1]{%
  \index{#1@\protect\texttt{#1} plot mark}%
  \index{Plot marks!#1@\protect\texttt{#1}}
  \texttt{\char`\\pgfuseplotmark\char`\{\declare{#1}\char`\}} &
  \tikz\draw[color=black!25] plot[mark=#1,mark options={fill=examplefill,draw=black}] coordinates{(0,0) (.5,0.2) (1,0) (1.5,0.2)};\\
}
\newcommand\plotmarkentrytikz[1]{%
  \index{#1@\protect\texttt{#1} plot mark}%
  \index{Plot marks!#1@\protect\texttt{#1}}
  \texttt{mark=\declare{#1}} & \tikz\draw[color=black!25]
  plot[mark=#1,mark options={fill=examplefill,draw=black}] 
    coordinates {(0,0) (.5,0.2) (1,0) (1.5,0.2)};\\
}



\ifx\scantokens\@undefined
  \PackageError{phd}{You need to use extended latex
    (elatex) or (pdfelatex) to process this document}{}
\fi

\begingroup
\catcode`|=0
\catcode`[= 1
\catcode`]=2
\catcode`\{=12
\catcode `\}=12
\catcode`\\=12 |gdef|find@example#1\end{codeexample}[|endofcodeexample[#1]]
|endgroup

\begingroup
\catcode`\^=7
\catcode`\^^M=13
\catcode`\ =13%
\gdef\returntospace{\catcode`\ =13\def {\space}\catcode`\^^M=13\def^^M{}}%
\endgroup

\begingroup
\catcode`\%=13
\catcode`\^^M=13
\gdef\commenthandler{\catcode`\%=13\def%{\@gobble@till@return}}
\gdef\@gobble@till@return#1^^M{}
\gdef\@gobble@till@return@ignore#1^^M{\ignorespaces}
\gdef\typesetcomment{\catcode`\%=13\def%{\@typeset@till@return}}
\gdef\@typeset@till@return#1^^M{{\def%{\char`\%}\textsl{\char`\%#1}}\par}
\endgroup

\define@key{codeexample}{width}{\setlength\codeexamplewidth{#1}}
\define@key{codeexample}{graphic}{\colorlet{thecodebackground}{#1}}
\define@key{codeexample}{code}{\colorlet{thecodebackground}{#1}}
\define@key{codeexample}{execute code}{\csname code@execute#1\endcsname}
\define@key{codeexample}{code only}[]{\code@executefalse}
\define@key{codeexample}{pre}{\def\code@pre{#1}}
\define@key{codeexample}{post}{\def\code@post{#1}}
\define@key{codeexample}{vbox}[]{\def\code@pre{\vbox\bgroup\setlength{\hsize}{\linewidth-6pt}}\def\code@post{\egroup}}
\define@key{codeexample}{ignorespaces}[]{\let\@gobble@till@return=\@gobble@till@return@ignore}
\define@key{codeexample}{leave comments}[]{\def\code@catcode@hook{\catcode`\%=12}\let\commenthandler=\relax\let\typesetcomment=\relax}
\def\code@pre{}
\def\code@post{}
\def\code@catcode@hook{}

\newdimen\codeexamplewidth
\newif\ifcode@execute
\newbox\codeexamplebox
\def\codeexample[#1]{%
  \begingroup%
  \code@executetrue
  \setlength\codeexamplewidth{4cm+7pt}
  \setkeys{codeexample}{#1}%
  \parindent0pt
  \begingroup%
  \par%
  \medskip%
  \let\do\@makeother%
  \dospecials%
  \obeylines%
  \@vobeyspaces%
  \catcode`\%=13%
  \catcode`\^^M=13%
  \code@catcode@hook%
  \relax%
  \find@example}
\def\endofcodeexample#1{%
  \endgroup%
  \ifcode@execute%
    \setbox\codeexamplebox=\hbox{%
      {%
        {%
          \returntospace%
          \commenthandler%
          \xdef\code@temp{#1}% removes returns and comments
        }%
        \colorbox{thecodebackground}{\color{black}\ignorespaces%
          \code@pre\expandafter\scantokens\expandafter{\code@temp\ignorespaces}\code@post\ignorespaces}%
      }%
    }%
    \ifdim\wd\codeexamplebox>\codeexamplewidth%
      \def\code@start{\par}%
      \def\code@flushstart{}\def\code@flushend{}%
      \def\code@mid{\parskip2pt\par\noindent}%
      \def\code@width{\linewidth-6pt}%
      \def\code@end{}%
    \else%
      \def\code@start{%
        \linewidth=\textwidth%
        \parshape \@ne 0pt \linewidth
        \leavevmode%
        \hbox\bgroup}%
      \def\code@flushstart{\hfill}%
      \def\code@flushend{\hbox{}}%
      \def\code@mid{\hskip6pt}%
      \def\code@width{\linewidth-12pt-\codeexamplewidth}%
      \def\code@end{\egroup}%
    \fi%
    \code@start%
    \noindent%
    \begin{minipage}[t]{\codeexamplewidth}\raggedright
      \hrule width0pt%
      \small%\vskip-1em%
      \code@flushstart\box\codeexamplebox\code@flushend%
      \vskip-1ex
      \leavevmode%
    \end{minipage}%
  \else%
    \def\code@mid{\par}
    \def\code@width{\linewidth-6pt}
    \def\code@end{}
  \fi%
  \code@mid%  
  \colorbox{thecodebackground}{%
    \begin{minipage}[t]{\code@width}%
      {%
        \let\do\@makeother
        \dospecials
        \frenchspacing\@vobeyspaces
        \normalfont\ttfamily%\footnotesize
        \typesetcomment%
        \@tempswafalse
        \def\par{%
          \if@tempswa
          \leavevmode \null \@@par\penalty\interlinepenalty
          \else
          \@tempswatrue
          \ifhmode\@@par\penalty\interlinepenalty\fi
          \fi}%
        \obeylines
        \everypar \expandafter{\the\everypar \unpenalty}%
        #1}
    \end{minipage}}%
  \code@end%
  \par%
  \medskip
  \end{codeexample}
}

\def\endcodeexample{\endgroup}
%    \end{macrocode}
%
% 
% From pgfplots manual
% 
%    \begin{macrocode}
\long\def\codeexamplenl{\noexpand\par}%
\pgfqkeys{/codeexample}{%
	every codeexample/.style={
		width=3.9cm,
		/pgfplots/every axis/.append style={legend style={fill=thecodebackground}}
	},
	narrow/.style={width=6.9cm},
	%tabsize=4,
	%pre={\begin{minipage}{\linewidth}\begingroup},
	%post={\endgroup\end{minipage}},
	%vbox,
	%newline=\codeexamplenl,
}
%    \end{macrocode}
%
%
% 
%	The macro \cs{keyval} typesets, key value lists and their options.
%	\medskip
%
%    \keyval{test}{\marg{option1|option2|option2|option4}}{ As per this example.}
%    \keyval{test}{\marg{option1|option2|option2|option4}}{ As per this example.}
%
%	We first measure the width of the option and not use it (want to make it a bit
%	flexible at a later stage. We also ensure that the catcode of \verb+|+ is set properly
%	in case anyone is using short verbatim commands, as we do in this document.
%
%    \begin{macrocode}
\newlength\temp@cx
\def\keyval{%
  \bgroup
  \catcode`|=11
  \@keyval}
%
\def\@keyval#1#2#3{%
  \settowidth\temp@cx{#1}%
  \parindent-30pt
  \hangindent30pt
  \par\leavevmode%
{\color{teal}\bfseries #1}\thinspace=\thinspace#2% 
\hspace*{.5em}#3%
\par\addvspace{1.5pt}%
\egroup
}
%
%    \end{macrocode}
%  Typesets a sample of bib
%    \begin{macrocode}
\newenvironment{bibsample}
  {\trivlist\samepage
   \setlength{\itemsep}{0pt}}
  {\endtrivlist}
%% doccommands
\newcommand*{\marglistfont}{\itshape}

\newcommand*{\margoptionfont}{\ttfamily}

\newcommand*{\margnotefont}{}

\newcommand*{\optionlistfont}{\bfseries}

\newcommand*{\ltxsyntaxfont}{\ttfamily}

\newcommand*{\ltxsyntaxlabelfont}{\bfseries}

\newcommand*{\changelogfont}{\normalfont}

\newcommand*{\changeloglabelfont}{\bfseries}

%% needed for listings????
\newcommand*{\verbatimfont}{\ttfamily}%

\let\displayverbfont\ttfamily

\renewcommand*{\verbatim@font}{\verbatimfamily}

%\def\cmd#1{\cs{\expandafter\cmd@to@cs\string#1}}%

%\def\cmd@to@cs#1#2{\char\number`#2\relax}

\newrobustcmd*{\env}[1]{\mbox{\verbatimfont\bfseries\textcolor{thegreen}{#1}}}

\newrobustcmd*{\len}[1]{\mbox{\verbatimfont\textbackslash#1}}

\newrobustcmd*{\cnt}[1]{\mbox{\verbatimfont#1}}

\newlength{\marglistsep}

\newlength{\marglistwidth}

\setlength{\marglistwidth}{(\oddsidemargin+1in)*85/100}%

\deflength{\marglistsep}{10pt}
%% This needs thorough checking as to restore previous definitions
%% of parsep we want parsep to be a bit higher than standard enumerated lists.
\global\newlength\oldparsep
\newenvironment*{marglist}
  {\setlength\oldparsep{\parsep}\list{}{%
     \parsep 3.5\p@ \@plus0\p@ \@minus\p@
     \setlength{\labelwidth}{\marglistwidth}%
     \setlength{\labelsep}{\marglistsep}%
     \setlength{\leftmargin}{0pt}%
     \renewcommand*{\makelabel}[1]{\hss\marglistfont##1}}}
  {\endlist\setlength\parsep{\oldparsep}}

% tt 
\newenvironment*{margoptionslist}
  {\setlength\oldparsep{\parsep}\list{}{%
     \parsep 3.5\p@ \@plus0\p@ \@minus\p@
     \setlength{\labelwidth}{\marglistwidth}%
     \setlength{\labelsep}{\marglistsep}%
     \setlength{\leftmargin}{0pt}%
     \renewcommand*{\makelabel}[1]{\hss\margoptionfont\detokenize{##1}}}}
  {\endlist\setlength\parsep{\oldparsep}}
  
  
%    \end{macrocode}
%
% \begin{docEnvironment} {keymarglist} { \meta{void} } 
%   Typesets a key options list in the margin.
% \end{docEnvironment}
%
%    \begin{macrocode}
\newenvironment*{keymarglist}
  {\marglist
   \setlength{\itemsep}{0pt}%
   \raggedright}
  {\endmarglist}
% color definitions
\def\colDef#1{\textcolor{themacro}{#1}}
% color for options
\def\colOpt#1{\textcolor{theoption}{\texttt{#1}}}

\newcommand{\option}[1]{\colOpt{#1}}
%    \end{macrocode}
%
%  \subsection{Creating a Small Verbatim Environment}
%  This is a modified version from Cambridge classes
%    \begin{macrocode}
\begingroup \catcode `|=0 
\catcode `[= 1
\catcode`]=2 
\catcode `\{=12 
\catcode `\}=12
\catcode`\\=12 
|gdef|@xsmallverbatim#1\end{smallverbatim}[#1|end[smallverbatim]]
|gdef|@sxsmallverbatim#1\end{smallverbatim*}[#1|end[smallverbatim*]]
|endgroup
\def\@smallverbatim{\trivlist \item\relax
  \if@minipage\else\vskip\parskip\fi
  \leftskip\@totalleftmargin\rightskip\z@skip
  \parindent\z@\parfillskip\@flushglue\parskip\z@skip
  \@@par
  \@tempswafalse
  \def\par{%
    \if@tempswa
      \leavevmode \null \@@par\penalty\interlinepenalty
    \else
      \@tempswatrue
      \ifhmode\@@par\penalty\interlinepenalty\fi
    \fi}%
  \let\do\@makeother \dospecials
  \obeylines \small \@noligs%\smallverbatim@font to FIX
  \hyphenchar\font\m@ne
  \everypar \expandafter{\the\everypar \unpenalty}%
}
\def\smallverbatim{\@smallverbatim \frenchspacing\@vobeyspaces \@xsmallverbatim}
\def\endsmallverbatim{\if@newlist \leavevmode\fi\endtrivlist}
\def\smallverbatim@font{\normalfont\smallverbatimsize\ttfamily}
%    \end{macrocode}
% 
% This is a short test. \lorem
%  \begin{smallverbatim}
%  \ifx\bhj
%  \else
%  \fi
%  \end{smallverbatim}
% \lorem
% \begin{docEnv}{docCommands}{}
% \end{docEnv}
%    \begin{macrocode}
\newenvironment{docCommands}{%
\bgroup
\par
\parindent=0pt
\parskip=3.5pt plus0.5pt
\everypar{\hangindent2em}%
\addvspace\belowdisplayskip\relax}%
{\everypar{}%
 \par
 \vskip\belowdisplayskip\egroup\par}
\long\def\auxm#1(#2);{%
  \def\Xtemp{#1}%
  \def\Ytemp{#2}%
  \parindent=0pt
  \addvspace{1.5pt}%
  \par\leavevmode
  \hangafter=1\relax   \hangindent=1em\relax
  \bgroup  
   \bfseries\sffamily\color{red}\Xtemp\,\color{black}(\textit{\Ytemp})\hskip0.1em
  \egroup
}

   
\newenvironment{docLua}[1]{%
  \auxm#1;
 }{%
\@@par
\smallskip\parindent=1em } 

\def\docFont#1{
\index{fonts!#1}%
#1%
} 

\newenvironment{handler}[2]{%
 \begin{phd@manual@entry}
  {\ttfamily{\color{thered}#1}\color{black}#2}
 \end{phd@manual@entry}
 }
{}

%    \end{macrocode}

%</DOCUM>
\endinput

