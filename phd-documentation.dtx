% \iffalse meta-comment
%<*internal>
\iffalse
%</internal>
%<*readme>
----------------------------------------------------------------
phd-documentation - version 1.0 (2018-10-26)
E-mail: yannislaz@gmail.com
Released under the LaTeX Project Public License v1.3c or later
See http://www.latex-project.org/lppl.txt

This work has the LPPL maintenance status `author-maintained'

This work consists of all files listed in README.md
%</readme>
%<*readmemd>
###The `phd-documentation` LaTeX2e package

The `phd` latex package and the class with the same name provide
convenient methods to create new styles for books, reports
and articles. It also loads the most commonly used packages 
and resolves conflicts.

This work consists of the file  `phd-documentation.dtx`,
and the derived files   `phd-documentation.ins`,  `phd-documentation.pdf`, 
and `phd-documentation.sty`.

###Installation

run
           phd-lua.bat on windows
           pdflatex phd.dtx
           makeindex -s gind.ist -g phd 

If you have any difficulties with the package come and join us at
http://tex.stackexchange.com and post a new question or
add a comment at http://tex.stackexchange.com/a/45023/963.
or send me a message at  yannislaz at gmail.com

### Documentation

The package was written using the `doc` and `docscript` packages,
so that it is self documented in a literary programming style. 
The .pdf is a fat document, providing over fifty book styles (the
equivalent of classes) plus there is a lot of write-up on the inner
workings of TeX and LaTeX2e. However, you don't need to know much
to use it.

      \usepackage{phd}
      %%%%%%%%%%%%%%%%%%%%%%%%%%%%%%%%%%%%%%%%%%%
%%%%%%  STYLE 13
%%%%%%%%%%%%%%%%%%%%%%%%%%%%%%%%%%%%%%%%%%%

\cxset{style13/.style={
 name={Chapter},
 numbering=arabic,
 number font-size=\HUGE,
 number font-family=\sffamily,
 number font-weight=\bfseries,
 number color=\color{gray!50},
 number before=\par\vspace*{5pt}\hfill\hfill,
 number dot=,
 number after={\hspace*{7pt}\par},
 number position=rightname,
 chapter font-family=\sffamily,
 chapter font-weight=\normalfont,
 chapter font-size=\LARGE,
 chapter before={\thickrule\vspace*{20pt}\par\hfill\hfill},
 chapter after={\vskip0pt\par},
 chapter color={black!50},
 title beforeskip={\vspace*{10pt}},
 title afterskip={\vspace*{50pt}\par},
 title before={\hfill\hfill\raggedleft},
 title after={},
 title font-family=\sffamily,
 title font-color=\color{thered},
 title font-weight=\bfseries,
 title font-size=\huge,
 section indent=-1em,
 section align=\raggedright,
 section numbering=arabic,
 section indent=0pt,
 section beforeskip=0pt,
 section afterskip=\baselineskip,
 subsection align=\raggedright,
 subsection font-family=\sffamily,
 subsection font-weight=\bfseries,
 subsection font-size=\large,
 subsection font-shape=\itshape,
 subparagraph number after=\space,
}
}

\def\setstyle#1{\cxset{style#1}%
 \renewsection\renewsubsection\renewsubsubsection%
 \renewparagraph\renewsubparagraph}

\setstyle{13}


\chapter{Introduction to Chapter\\ Style Thirteen}

\section{A Brief History of Biomedical\\ Fluid Mechanics}
\lorem
\medskip
\begin{figure}[ht]
\centering
\includegraphics[width=0.45\textwidth]{./chapters/chapter14}
\includegraphics[width=0.45\textwidth]{./chapters/chapter14a}
\end{figure}
\lorem


All choices, are made via an extended key-value interface. 
Although not a compliment, it resembles CSS and the keys are a bit verbose but
attributes are easy to change and have a consistent and easy to remember interface.

To set or add a key we only use one command:

      \cxset{chapter name font-size = Huge,
             chapter number font-size = HUGE} 

### Future Development

This is still an experimental version, but I will retain the
interface in future releases. There is a large amount of
work still to be carried out to improve the template styles
provided, to test it more thoroughly and to add a number of
improvements in the special designs. At present I estimate
that I have completed about 70% of the work that needs
to be done.

__The package as it stands is not production stable.__ 


%</readmemd>
%
%<*TODO>
1.  Split package into three diffferent parts. One for listings settings. Use def, docCommands and
    indexing commands. Indexing commands remove symbols defs into sybpackage.
2.  Finish symbol management, both text and math. Math already 80% incorporated.
%</TODO>
%<*internal>
\fi
\def \nameofplainTeX{plain}
\ifx\fmtname\nameofplainTeX\else
  \expandafter\begingroup
\fi
%</internal>
%<*install>
\input docstrip.tex
\keepsilent
\askforoverwritefalse
\preamble
----------------------------------------------------------------
phd --- A package to beautify documents.
E-mail: yannislaz@gmail.com
Released under the LaTeX Project Public License v1.3c or later
See http://www.latex-project.org/lppl.txt
----------------------------------------------------------------
\endpreamble
%\BaseDirectory{C:/users/admin/my documents/github/phd}
%\usedir{MWE}
\generate{\file{\jobname.sty}{%
  \from{\jobname.dtx}{DOCUM}%
   }%
  }%
\generate{%
  \file{\jobname-defaults.def}{%
  \from{phd-fontmanager.dtx}{DFLT}%
  \from{phd-colorpalette.dtx}{DFLT}% 
  \from{phd-lowersections.dtx}{DFLT}%
  \from{\jobname.dtx}{DFLT}}%
}
%\nopreamble\nopostamble
%</install>
%<install>\endbatchfile
%<*internal>
%\usedir{tex/latex/phd}
\generate{
  \file{\jobname.ins}{\from{\jobname.dtx}{install}}
}
\nopreamble\nopostamble
\generate{
	\file{README.txt}{\from{\jobname.dtx}{readme}}
  }
\generate{
  \file{\jobname.md}{\from{\jobname.dtx}{readmemd}}
}
\generate{
  \file{\jobname-todo.md}{\from{\jobname.dtx}{TODO}}
}
\ifx\fmtname\nameofplainTeX
  \expandafter\endbatchfile
\else
  \expandafter\endgroup
\fi
%</internal>
%<*driver>
%\listfiles
%gdef\@onlypreamble{} % TO BE REMOVED NEEDED FOR TUTS
\NeedsTeXFormat{LaTeX2e}[2017/04/15]%
\RequirePackage[2017/04/15]{latexrelease}
\documentclass[book,oneside,10pt,a4paper]{phddoc}
\usepackage[bottom=2cm]{geometry}
\savegeometry{std}
% \usepackage[style=mla]{biblatex}
\usepackage{microtype}
\usepackage{phd}
\usepackage{phd-lowersections}
\usepackage{phd-runningheads}
\usepackage{phd-documentation}
\usepackage{phd-toc}
\pagestyle{headings}
%
\sethyperref
% add bib resource
\addbibresource{phd1.bib}% Syntax f
\cxset{palette unorange}
\begin{filecontents}{defaults-chapters}
%%    General Defaults for Chapters
\cxset{%    
    chapter title margin-top-width    =  0cm,
    chapter title margin-right-width  =  1cm,
    chapter title margin-bottom-width = 10pt,
    chapter title margin-left-width   = 0pt,
    chapter align                     = left,
    chapter title align               = left, %checked
    chapter name                      = hang,
    chapter format                    = hdr,
    chapter font-size                 = Huge,
    chapter font-weight               = bvar,
    chapter font-family               = sffamily,
    chapter font-shape                = upshape,
    chapter color                     = black,
    chapter number prefix             = ,
    chapter number suffix             = ,
    chapter numbering                 = arabic,
    chapter indent                    = 0pt,
    chapter beforeskip                = -3cm,
    chapter afterskip                 = 30pt,
    chapter afterindent               = off,
    chapter number after              = ,
    chapter arc                       = 0mm,
    chapter background-color          = bgsexy,
    chapter afterindent               = off,
    chapter grow left                 = 0mm,
    chapter grow right                = 0mm, 
    chapter rounded corners           = northeast,
    chapter shadow                    = fuzzy halo,
    chapter border-left-width         = 0pt,
    chapter border-right-width     = 0pt,
    chapter border-top-width       = 0pt,
    chapter border-bottom-width    = 0pt,
    chapter padding-left-width     = 0pt,
    chapter padding-right-width    = 10pt,
    chapter padding-top-width      = 10pt,
    chapter padding-bottom-width   = 10pt,
    chapter number color           = white,
    chapter label color            = white,    
    }
 \cxset{    
    chapter number font-size        = huge,
    chapter number font-weight      = bfseries,
    chapter number font-family      = sffamily,
    chapter number font-shape       = upshape,
    chapter number align            = Centering,
    }
\cxset{%    
     chapter title font-size        = Huge,
     chapter title font-weight      = bvar,
     chapter title font-family      = calligra,
     chapter title font-shape       = upshape,
     chapter title color            = black,
     }    
\end{filecontents}
%% LaTeX2e file `defaults-chapters'
%% generated by the `filecontents' environment
%% from source `phd-lowersections' on 2015/07/21.
%%
%%    General Defaults for Chapters
\cxset{%
    chapter title margin-top-width    =  0cm,
    chapter title margin-right-width  =  1cm,
    chapter title margin-bottom-width = 10pt,
    chapter title margin-left-width   = 0pt,
    chapter align                     = RaggedLeft,
    chapter title align               = Centering, %checked
    chapter name                      = Chapter,
    chapter format                    = block,
    chapter font-size                 = HHUGE,
    chapter font-weight               = bold,
    chapter font-family               = sffamily,
    chapter font-shape                = upshape,
    chapter color                     = black,
    chapter number prefix             = ,
    chapter number suffix             = ,
    chapter numbering                 = arabic,
    chapter indent                    = 0pt,
    chapter beforeskip                = -1sp,
    chapter afterskip                 = 30pt,
    chapter afterindent               = off,
    chapter number after              = ,
    chapter arc                       = 0mm,
    chapter background-color       = bgsexy,
    chapter afterindent            = off,
    chapter grow left              = 0mm,
    chapter grow right             = 0mm,
    chapter rounded corners        = northeast,
    chapter shadow                 = fuzzy halo,
    chapter border-left-width      = 0pt,
    chapter border-right-width     = 0pt,
    chapter border-top-width       = 0pt,
    chapter border-bottom-width    = 0pt,
    chapter padding-left-width     = 0pt,
    chapter padding-right-width    = 10pt,
    chapter padding-top-width      = 10pt,
    chapter padding-bottom-width   = 10pt,
    chapter number color           = white,
    chapter label color            = white,
    }
 \cxset{
    chapter number font-size        = huge,
    chapter number font-weight      = bfseries,
    chapter number font-family      = sffamily,
    chapter number font-shape       = upshape,
    chapter number align            = Centering,
    }
\cxset{%
     chapter title font-size        = HHUGE,
     chapter title font-weight      = bold,
     chapter title font-family      = sffamily,
     chapter title font-shape       = upshape,
     chapter title color            = white,
     }
  


\definecolor{creamy}{HTML}{FDEBD7}
\cxset{chapter title color= creamy,
       chapter label color = creamy,
       chapter number color = creamy,
       chapter number font-size = Huge,
       subsection title color = creamy,
       chapter name = CHAPTER,
       chapter label case = upper,
       chapter number align=left,
       part format = traditional,
       part background-color=spot,
       part beforeskip                = -3cm,
       part afterskip                 = 30pt,
       subsection afterindent=off,
       section format=hang,
       }
       
\makeindex
\usepackage[cache=false]{minted} 
\usemintedstyle[latex]{borland}  
\setminted[html]{fontsize=\footnotesize,style=friendly}
\newfontfamily\lineara{Aegean.ttf}
\newfontfamily\cypriote{Aegean.ttf}
\begin{document}
\DEBUGOFF
\overfullrule0pt
\parindent1em
\coverpage{monkey}{Book Design Monographs}{Camel Press}{INDEXING}{AND DOCUMENTATION} 
\pagestyle{empty}
%\coverpage{habtoor-city}{Delay Claim}{HLS-DSE/JV}{HABTOOR CITY}{MEP CLAIM} 
\secondpage
\pagestyle{empty}
\clearpage

\tableofcontents

\pagestyle{empty}
\setcounter{secnumdepth}{6}
\parskip0pt plus.1ex minus.1ex
\mainmatter
\pagenumbering{arabic}
\pagestyle{headings} 
% Input the main descriptions and guide
\let\sidenote\footnote
\let\citep\footcite
\chapter{How to Develop your Own Class or Package}

\cxset {epigraph width=0.67\textwidth}

\epigraph{First there was one user and I took a lot of time to satisfy myself. Then I had 10 users, and a whole new level of difficulties arose. Then I had a hundred users and another level of things happened. I had a thousand users, I had ten thousand each of those were special phases in the development, important. I couldn't have gone with ten thousand until I'd done
it with a thousand. But each time a new wave of
changes came along, the idea was to have \tex get
better, and not get more diverse as it needed to handle
new things.}{Donald Knuth}

\parindent1em

\section{Introduction}


To \emph{make} a book is an interesting and somewhat involved process\footcite{town}. The text is set in type and printed on pages, the pages are gathered and folded into signatures and these are gathered and folded into signatures and these are then bound and covered. Many of the aspects of this process that has passed down to us by previous generations is discussed extensively in other sections of this book.  Class authors have to distill this knowledge in a set of typographical rules to be described in a class file. The first thing such an author must do is to describe the \emph{rationale} of developing such a class. The \docClass{octavo}\citep{octavo} class was developed to enable printing books in dimensions that follow traditional styles. The \citep{memoir}  class to offer a flexible system on which other classes could be based and so does \citep{koma}. The |tufte-book| and |tufte-handout| classes to provide a style that resembles those found in Tufte books. Many Universities offer \emph{Thesis} classes to standardize the way these are produced. Many of these Universities, translated the styles previously typed and the results are a typographical disaster, only mitigated by the ability to display beautiful mathematics. As these are printed on standard \emph{photocopy paper} one cannot do much with the layout. 
\section{What is a class?}
A class is simply a file with the extension \docExtension{cls} containg a set of macros. 
A class can load another class.
\section{Identifying your class}

The first thing a class must do is to identify any other formats it needs and to announce
its name. This is accomplished using the two commands 
\refCom{NeedsTeXFormat} and \refCom{ProvidesClass}.

The following example, delares the version of \LaTeXe\ that it requires and then
gives the class name. It can be found in the preable of most well written classes. You should also put some remarks to identify you as the author, the version number and other similar details. These are discussed in more detail in the next Chapter, where you will see how to automate documentation for your class.

\begin{teX}
\NeedsTeXFormat{LaTeX2e}[1994/06/01]
\ProvidesClass{myclass-book}[2010/12/11 v3.5.0 myclass-book]
\end{teX}

The above syntax must be followed exactly so that this information can be
used by \texttt{LoadClass} or \texttt{documentclass} (for classes) or \docAuxCommand{RequirePackage}
 or\cmd{usepackage} (for packages) to test that the release is not too old.
The whole of this $<release-info>$ information is displayed by \docAuxCommand{listfiles} and
should therefore not be too long.

\begin{teX}
% Load the common style elements
\input{myclass-common.def}
\end{teX}


Another command that can be used is \docAuxCommand{ProvidesFile}. 
This is similar to the two previous commands except that here the fullname,
including the extension, must be given. It is used for declaring any files other
than main class and package files.

This is useful, if you decide to have your main definitions in a separate file.

\section{Class Options}

Before we see in detail how to add options to a class, we need to review a package called
\pkgname{xkeyval}. Unless you are in the business of re-discovering wheels, this is an absolute must
for developing, readable and maintenable code and your class is to provide many options. 
\begin{teX}
\usepackage[textcolor=red,font=times]{mypack}
\end{teX}

Class options are best set by using booleans\docAuxCommand{newboolean}.

We first set a new boolean that we |name@myclass@afourpaper.| This is used using the package
\texttt{ifthen}\sidenote{The ifthen package was developed by 
David Carlisle, can be downloaded at \url{ http://www.ifi.uio.no/it/latex-links/ifthen.pdf }} 
Then we can |DecalareOptionX| and we set the boolean to default to true. If the user then types

myclass[a4paper]

The a4paper options will be set. This is a much better and concise way of defining options.
\cmd{newboolean}


\begin{teX}
\newboolean{@myclass@afourpaper}
\DeclareOptionX[myclass]<common>{a4paper}
  {
   \setboolean{@myclass@afourpaper}
   {true}
  }
\end{teX}
\medskip

Note that the command provide by \texttt{ifthen} \docAuxCommand{setboolean} takes true or false, as \#2, and sets \#1 accordingly. In the above code we set the option as true. 


It is much easier and most programmers use the \texttt{ifthen} package to check
for option booleans

\begin{teX}
\ifthenelse{\boolean{@myclass@afourpaper}}
  {\geometry{
        a4paper,
        left=24.8mm,
        top=27.4mm,
        headsep=2\baselineskip,
        textwidth=107mm,
        marginparsep=8.2mm,
        marginparwidth=49.4mm,
        textheight=49\baselineskip,
        headheight=\baselineskip
    }
  }
 {}
\end{teX}

\section{Set-up the font sizes}

LaTeX does not provide definitions of all the font-sizes. Unless you are
extending an existing class, this is one of the first tasks you need to 
do in your new class.

Normally class authors will define all the commonly defined size commands,
such as  \cmd{small}, \cmd{normalsize} and other similar commands.

In the example shown below, we first start by defining the \cmd{normalsize} font
size. In this book the \cmd{\normalsize}  is defined as 14pt. We also define the vertical
spaces that we need to have abovedisplay and belowdisplayskip. These are all very difficult to
remember and once you have something you are happy with, just copy from class to class
or even define a samll definition file to keep them all together.


{\fontfamily{phv}\selectfont Helvetica looks like this}
and {\fontencoding{OT1}\fontfamily{ppl} Palatino looks like this}.


 The user has access to a number of commands which change the size of
 the fount, relative to the `main' size used for the bulk of the text.


 These \cmd{size} commands issue a \cmd{@setfontsize}\index{Latex kernel!@setfontsize} 
 command.

\begin{teX}
  \@setfontsize\size\font-size{baselineskip} where:
\end{teX}



  \begin{description}
    \item {font-size} The absolute size of the fount to use from
        now on.
    \item{baselineskip} The normal value of \cmd{baselineskip}
        for the size of the fount selected. (The actual value will be
       % |\baselinestretch| * \meta{baselineskip}.)
    \end{description}

A number of commands, defined in the \LaTeX  kernel, shorten the
following  definitions and are used throughout. These are:

    \begin{center}
    \begin{tabular}{ll@{\qquad}ll@{\qquad}ll}
    \verb=\@vpt= & 5 & \verb=\@vipt= & 6 & \verb=\@viipt= & 7 \\
    \verb=\@viiipt= & 8 & \verb=\@ixpt= & 9 & \verb=\@xpt= & 10 \\
    \verb=\@xipt= & 10.95 & \verb=\@xiipt= & 12 & \verb=\@xivpt= & 14.4\\
    \ldots
    \end{tabular}
    \end{center}


\subsection{Setting up the normalsize}
 The user command to obtain the `main' size is \cmd{normalsize}. \LaTeX\
 uses \cmd{@normalsize} \index{Latex kernel!@normalsize} when referring to the main size and maintains this
 value even if \docAuxCommand{normalsize} is redefined. The \docAuxCommand{normalsize} macro also
  sets values for \cmd{abovedisplayskip}, \cmd{abovedisplayshortskip} and 
\cmd{belowdisplayshortskip}.



\begin{teX}
%%
% Set the font sizes and baselines to match Tufte's books
% normalsize
%%
\renewcommand\normalsize{%
   \@setfontsize\normalsize\@xpt{14}%
   \abovedisplayskip 10\p@ \@plus2\p@ \@minus5\p@
   \abovedisplayshortskip \z@ \@plus3\p@
   \belowdisplayshortskip 6\p@ \@plus3\p@ \@minus3\p@
   \belowdisplayskip \abovedisplayskip
   \let\@listi\@listI}

\normalbaselineskip=14pt
\normalsize
\end{teX}



\begin{teX}
\renewcommand\small{%
   \@setfontsize\small\@ixpt{12}%
   \abovedisplayskip 8.5\p@ \@plus3\p@ \@minus4\p@
   \abovedisplayshortskip \z@ \@plus2\p@
   \belowdisplayshortskip 4\p@ \@plus2\p@ \@minus2\p@
   \def\@listi{\leftmargin\leftmargini
               \topsep 4\p@ \@plus2\p@ \@minus2\p@
               \parsep 2\p@ \@plus\p@ \@minus\p@
               \itemsep \parsep}%
   \belowdisplayskip \abovedisplayskip
}
\renewcommand\footnotesize{%
   \@setfontsize\footnotesize\@viiipt{10}%
   \abovedisplayskip 6\p@ \@plus2\p@ \@minus4\p@
   \abovedisplayshortskip \z@ \@plus\p@
   \belowdisplayshortskip 3\p@ \@plus\p@ \@minus2\p@
   \def\@listi{\leftmargin\leftmargini
               \topsep 3\p@ \@plus\p@ \@minus\p@
               \parsep 2\p@ \@plus\p@ \@minus\p@
               \itemsep \parsep}%
   \belowdisplayskip \abovedisplayskip
}
\renewcommand\scriptsize{\@setfontsize\scriptsize\@viipt\@viiipt}
\renewcommand\tiny{\@setfontsize\tiny\@vpt\@vipt}
\renewcommand\large{\@setfontsize\large\@xipt{15}}
\renewcommand\Large{\@setfontsize\Large\@xiipt{16}}
\renewcommand\LARGE{\@setfontsize\LARGE\@xivpt{18}}
\renewcommand\huge{\@setfontsize\huge\@xxpt{30}}
\renewcommand\Huge{\@setfontsize\Huge{24}{36}}

%% Define a HUGE for fun
\newcommand\HUGE{\@setfontsize\Huge{38}{47}}  
\end{teX}


\section{Adjusting paragraph parameters}

 The parameters which control \TeX 's behaviour when typesetting
 paragraphs receive a bit of a tweak here. Contrary to the usual
 behaviour of modifying the grid with glue when difficulties are
 encountered with vertical space, here we shall try to counteract
 these tendencies and enforce as much as possible uniformity of the 
 grid of lines.

A good value for paragraph indentation is \texttt{parindent 0.5pt}, for vertical spacing between
paragraphs that are indented use 0pt. At this point if you are using any marginals it is a good idea
to allow hyphenation with the \docpkg{ragged2e} package. Since marginals use very narrow paragraphs you may
get a very funny looking marginal text. Using the package, adjustments can be made to hyphenate
the marginal text.

\begin{teXXX}
%%
% \RaggedRight allows hyphenation

\RequirePackage{ragged2e}
\setlength{\RaggedRightRightskip}{\z@ plus 0.08\hsize}
\setlength{\RaggedRightParindent}{1pc}

% Paragraph indentation and separation for normal text
\newcommand{\@tufte@reset@par}{%
  \setlength{\RaggedRightParindent}{1.0pc}%
  \setlength{\parindent}{1pc}%
  \setlength{\parskip}{0pt}%
}
\@tufte@reset@par

% Paragraph indentation and separation for marginal text
\newcommand{\@tufte@margin@par}{%
  \setlength{\RaggedRightParindent}{0.5pc}%
  \setlength{\parindent}{0.5pc}%
  \setlength{\parskip}{0pt}%
}
\end{teXXX}


\section{Formatting Chapters and Sections}

The section on Chapters etc, has more on this, but we will touch on it briefly.
Most recent class developerss use the \pkg{titlesec} and \pkg{titletoc} package to handle the complexity 
of these commands. With the |phd| package this is unecessary. 

\begin{teXXX}
\titleformat{\subsection}%
  [hang]% shape
  {\normalfont\large}% format applied to label+text removed \itshape
  {\thesubsection}% label
  {1em}% horizontal separation between label and title body
  {}% before the title body
  []% after the title body
\end{teXXX}

These are normally followed by the ``titlespacing" commands to define the space around these sections.

\begin{teXXX}
%% We set the titlespacing using the package titlesec and titletoc
%
\titlespacing*{\chapter}{0pt}{20pt}{40pt}
\titlespacing*{\section}{0pt}{3.5ex plus 1ex minus .2ex}{2.3ex plus .2ex}
\titlespacing*{\subsection}{0pt}{3.25ex plus 1ex minus .2ex}{1.5ex plus.2ex}
\end{teXXX}

\section{Adjusting the Index}

For classes representing books, the index is treated like a chapter whereas for others it is normally
treated like a section. Whatever your document ends up like, indices are best done in a multi-column environment.
One possibility is shown below, using the package "multcol".

\begin{teXXX}
\RequirePackage{multicol}
\renewenvironment{theindex}
  {\begin{fullwidth}%
    \small%
    \ifthenelse{\equal{\@tufte@class}{book}}%
      {\chapter{\indexname}}%
      {\section*{\indexname}}%
    \parskip0pt%
    \parindent0pt%
    \let\item\@idxitem%
    \begin{multicols}{3}%
  }
  {\end{multicols}%
    
\renewcommand\@idxitem{\par\hangindent 2em}
\renewcommand\subitem{\par\hangindent 3em\hspace*{1em}}
\renewcommand\subsubitem{
    \par\hangindent 4em\hspace*{2em}
}
\renewcommand\indexspace{
    \par\addvspace{
       1.0\baselineskip plus 0.5ex minus 0.2ex}\relax
    }%
%we now  swallow the letter heading in the index
\newcommand{\lettergroup}[1]{}

\end{teXXX}

The code, renews the "theindex" environment, with minor tweaks and defines it as a three column
layout at "fullwidth".

\section{Provide some hooks}
It is useful at the end of the class to allow for localization of the class
by importing a local file. This is easily achieved by checking if the file exists
and then loading it.  If there is a |myclass-book-local.sty|  file, load it.

\begin{teX}
\IfFileExists{myclass-book-local.tex}
  {input{myclass-book-local}
   \MyClassInfoNL{Loading myclass-book-local.tex}}
  {}
\end{teX}

If you intent to publish your class, you may also want to consider adding a hook for a patch-file.


\section{The final act of kindness to your users}
Many common classes, such as the |memoir| use such a tactic to avoid breaking old code.\index{IfFileExists}

\begin{teX}
 \IfFileExists{mypatch.sty}{%
 \RequirePackage{mypatch}}{}
\end{teX}


\parindent1em
\chapter{How to Package Your Class}

In the previous chapter we have outlined the main sections that you probably need
to define in your class. In the examples we have used we just typed the examples
as |example.cls| or |package.sty|.

In this chapter we will go over the packaging of the class
and automating the generation of user documentation, using the |doc| and \pkg{DocStrip}\footcite{docstrip}
programs in files with an extension |.dtx|. The DocStrip program is an amazing piece of code that was originally
created by Frank Mittelbach to accompany the |doc| package. The idea behind it was to remove comment lines
in order to reduce the execution time of the program. Having created the DocStrip program to remove comment lines from  programs it became feasible to do more than just strip comments.
Wouldn't it be nice to have a way to include parts of the code only when some
condition is set true? Wouldn't it be as nice to have the possibility to split the
source of a \tex program into several smaller files and combine them later into
one `executable'? Both these wishes have been implemented in the DocStrip program.



You should also be
familiar with ``LaTeX2e'' for Class and Package Writers”, which is available
from CTAN (\url{http://www.ctan.org}) and comes with most LaTeX2e" distributions
in a file called clsguide.dvi.\footcite{pakin2004}  Finally, you should know how to
install packages that are shipped as a \texttt{.dtx} file plus a \texttt{.ins} file.

style (.sty) file is primarily a collection of macro and
environment definitions. One or more style files (e.g., a main style file that
\cs{input}  or \cs{RequirePackages} multiple helper files) is called a package.
Packages are loaded into a document with \cs{usepackage}\marg{main .sty fille}.
In the rest of this document, we use the notation \meta{package} to represent
the name of your package.


Motivation The important parts of a package are the code, the documentation
of the code, and the user documentation. Using the \docpkg{Doc}  and
DocStrip programs, it’s possible to combine all three of these into a single,
documented LATEX(.dtx) file. The primary advantage of a .dtx file is that
it enables you to use arbitrary LATEX constructs to comment your code.
Hence, macros, environments, code stanzas, variables, and so forth can be
explained using tables, figures, mathematics, and font changes. Code can
be organized into sections using LATEX’s sectioning commands. Doc even
facilitates generating a unified index that indexes both macro definitions (in
the LATEX code) and macro descriptions (in the user documentation). 

This emphasis on writing verbose, nicely typeset comments for code—essentially
treating a program as a book that describes a set of algorithms—is known
as literate programming \cite{literate} and has been in use since the early days of \tex\ .

Furthermore,
this tutorial shows how to write a single file that serves as both documentation
and driver file, which is a more typical usage of the \texttt{Doc} system than
using separate files.

\subsection{The .ins file}

The first step in preparing a package for distribution is to write an installer
(|.ins|) file. An installer file extracts the code from a |.dtx| file, uses \pkg{docstrip}
to strip off the comments and documentation, and outputs a |.sty| file. The
good news is that a |.ins| file is typically fairly short and doesn’t change
significantly from one package to another.

\paragraph{License} The |ins| files usually start with comments specifying the copyright and license
information:

\begin{minted}{latex}
%%
%% Copyright (C) year by your name %%
%% This file may be distributed and/or modified under the
%% conditions of the LaTeX Project Public License, either
%% version 1.2 of this license or (at your option) any later
%% version. The latest version of this license is in:
%%
%% http://www.latex-project.org/lppl.txt
%%
%% and version 1.2 or later is part of all distributions of
%% LaTeX version 1999/12/01 or later.
%%
\end{minted}

The LATEX Project Public License (LPPL) is the license under which most
packages—and LATEX itself—are distributed. Of course, you can release your
package under any license you want; the LPPL is merely the most common
license for LATEX packages. The LPPL specifies that a user can do whatever
he wants with your package—including sell it and give you nothing in return.
The only restrictions are that he must give you credit for your work, and
he must change the name of the package if he modifies anything to avoid
versioning confusion.
The next step is to load DocStrip:

\begin{teXXX}
%%\input docstrip.tex
%%\keepsilent
\end{teXXX}



By default, DocStrip gives a line-by-line account of its activity. These messages
aren’t terribly useful, so most people turn them off, by using the command \docAuxCommand{keepsilent}:

\begin{teXXX}
\keepsilent
\end{teXXX}


A system administrator can specify the base directory under which all
TEX-related files should be installed, e.g., \texttt{/usr/share/texmf}. (See
\cmd{\BaseDirectory} in the DocStrip manual.) The |ins| file specifies where
its files should be installed relative to that. The following is typical:

\begin{teXXX}
\usedir{tex/latex/packagename}
\preamble
htexti \endpreamble
\end{teXXX}



The next step is to specify a preamble, which is a block of commentary that
will be written to the top of every generated file:

\begin{minted}{latex}
\preamble
----------------------------------------------------------------
phddoc --- A class to typeset LaTeX code.
E-mail: yannislaz@gmail.com
Released under the LaTeX Project Public License v1.3c or later
See http://www.latex-project.org/lppl.txt
----------------------------------------------------------------
\endpreamble
\end{minted}


The preceding preamble would cause |package.sty|  to begin as follows:

\begin{minted}{latex}
%%
%% This is file `phddoc.cls',
%% generated with the docstrip utility.
%%
%% The original source files were:
%%
%% phddoc.dtx  (with options: `class')
%% ----------------------------------------------------------------
%% phddoc --- A class to typeset LaTeX code.
%% E-mail: yannislaz@gmail.com
%% Released under the LaTeX Project Public License v1.3c or later
%% See http://www.latex-project.org/lppl.txt
%% ----------------------------------------------------------------
\end{minted}

We now reach the most important part of a .ins file: the specification of
what files to generate from the |.dtx| file. The following tells DocStrip to
generate hpackagei.sty from hpackagei.dtx by extracting only those parts
marked as `package'  in the .dtx file. (Marking parts of a .dtx file is
described later on.)

\begin{teXXX}
\generate{\file{<package>.sty}{\from{<package>.dtx}{package}}}
\end{teXXX}

\cmd{\generate} can extract any number of files from a given .dtx file. It can
even extract a single file from multiple |.dtx| files. See the DocStrip manual
for details.

Personally I also generate README.md files in |markdown| format as well, so that
when they get uploaded to |github| they can be rendered nicely.

\begin{minted}{latex}
\generate{\file{\jobname.md}{\from{\jobname.dtx}{readmemd}}}
\end{minted}

The text has to be wriiten using `guards' with the tag |readmd|

\begin{minted}{latex}
%<*readmemd>
# The `phddoc` LaTeX2e class

The `phd` latex package and the class with the same name provide
convenient methods to create new styles for books, reports
and articles. It also loads the most commonly used packages 
and resolves conflicts.
%</readmemd>
\end{minted}

\subsection{Generating messages} 

The next part of a |.ins| file consists of commands to output a message to
the user, telling him what files need to be installed and reminding him how
to produce the user documentation. The following set of \cmd{Msg} commands is
typical:

\begin{minted}[
frame=lines,
framesep=2mm,
baselinestretch=1.2,
fontsize=\footnotesize,
linenos
]{latex}
\obeyspaces
\Msg{****************************************************}
\Msg{* *}
\Msg{* To finish the installation you have to move the *}
\Msg{* following file into a directory searched by TeX: *}
\Msg{* *}
\Msg{* packagei.sty *}
\Msg{* *}
\Msg{* To produce the documentation run the file *}
\Msg{* package.dtx through LaTeX. *}
\Msg{* *}
\Msg{* Happy TeXing! *}
\Msg{* *}
\Msg{****************************************************}
Note the use of \obeyspaces to inhibit \tex from collapsing multiple spaces
into one.
\endbatchfile
\end{minted}


Appendix A.1 lists a complete, skeleton .ins file. Appendix A.2 is similar
but contains slight modifications intended to produce a class (|.cls|) file
instead of a style (|.sty|) file

\section{What to put in a  .dtx file}
We started describing the |.ins| install file first. The next file we will describe is
the |.dtx| file. This holds both the code definitions as well as the user documentation.

A |dtx|\ file contains both the commented source code and the user documentation
for the package. Running a |dtx|  file through |latex| typesets the
user documentation, which usually also includes a nicely typeset version of
the commented source code.

Due to some Doc trickery, a |dtx|  file is actually evaluated twice. The first
time, only a small piece of \latex\  driver code is evaluated. The second time,
comments in the |dtx|  file are evaluated, as if there were no `\%'  preceding
them. This can lead to a good deal of confusion when writing |dtx|  files
and occasionally leads to some awkward constructions. Fortunately, once
the basic structure of a |dtx|  file is in place, filling in the code is fairly
straightforward.

\paragraph{Guards} If you open any .dtx file you will notice that the lines either start with a \%
sign or sometimes with a percentage sign and |<|\textit{guard}|>|. The latter is called a guard and they are in a way
like html tags. They have a starting and an ending tag. In the example below there are two different guards
|<*10pt>...</10pt>| and |<*11pt></11pt>|. Unlike html tags guards are boolean expressions! You can use:
\begin{quote}
\textbar  ! \&  
\end{quote}

The \textbar stands for disjunction (OR), the \& stands for conjunction (AND) and the ! (NOT) stands for
negation. The terminal is any sequence of letters and evaluates to true iff it
occurs in the list of options that have to be included.

\begin{minted}{latex}
%<*10pt|11pt|12pt>
... code
%</10pt|11pt|12pt>
\end{minted}

A longer example from KOMA shows the concept better.

\fvset{gobble=0}
\begin{minted}[
frame=lines,
framesep=2mm,
baselinestretch=1.2,
fontsize=\footnotesize,
linenos
]{latex}
%    \begin{macrocode}
\def\normalsize{%
%<*10pt>
  \@setfontsize\normalsize\@xpt\@xiipt
  \abovedisplayskip 10\p@ \@plus2\p@ \@minus5\p@
  \abovedisplayshortskip \z@ \@plus3\p@
  \belowdisplayshortskip 6\p@ \@plus3\p@ \@minus3\p@
%</10pt>
%<*11pt>
  \@setfontsize\normalsize\@xipt{13.6}%
  \abovedisplayskip 11\p@ \@plus3\p@ \@minus6\p@
  \abovedisplayshortskip \z@ \@plus3\p@
  \belowdisplayshortskip 6.5\p@ \@plus3.5\p@ \@minus3\p@
%</11pt>
... 
%    end{macrocode}
\end{minted}

If the guards only contain a one line of text, then a short form is provided as |<10pt>|. It is unecessary to provide a closing tag and the `*' is omitted. The example below from the KOMA classes shows a quite ingenious way of writing the |\ProvidesFile| macro in
the different files; one for each tag. 
Two kinds of optional code are supported: one can either have optional code
that is on one line of tex code.

To distinguish both kinds of optional code the `guard modier' has been introduced. 
The `guard modifier' is one character that immediately follows the < of
the guard. It can be either * for the beginning of a block of code, or / for the end
of a block of code. The beginning and ending guards for a block of code have to
be on a line by themselves.

When a block of code is not included, any guards that occur within that block
are not evaluated.


\begin{minted}{latex}
%    \begin{macrocode}
\ProvidesFile{%
%<10pt>  scrsize10pt.clo%
%<11pt>  scrsize11pt.clo%
%<12pt>  scrsize12pt.clo%
}[\KOMAScriptVersion\space font size class option %
%<10pt>  (10pt)%
%<11pt>  (11pt)%
%<12pt>  (12pt)%
]
%    \end{macrocode}
\end{minted}

In the |.ins| file one could write to generate the various |.clo| files.:

\begin{minted}{latex}
\generate{\usepreamble\defaultpreamble
  \file{scrsize10pt.clo}{%
    \from{scrkernel-version.dtx}{clo,10pt}%
    \from{scrkernel-fonts.dtx}{clo,10pt}%
    \from{scrkernel-paragraphs.dtx}{clo,10pt}%
  }%
  \file{scrsize11pt.clo}{%
    \from{scrkernel-version.dtx}{clo,11pt}%
    \from{scrkernel-fonts.dtx}{clo,11pt}%
    \from{scrkernel-paragraphs.dtx}{clo,11pt}%
  }%
  \file{scrsize12pt.clo}{%
    \from{scrkernel-version.dtx}{clo,12pt}%
    \from{scrkernel-fonts.dtx}{clo,12pt}%
    \from{scrkernel-paragraphs.dtx}{clo,12pt}%
  }%
}%
\end{minted}

Becareful not to introduce spurious empy lines in your generated files by having empty lines in no-man's land, that is between tags.\footnote{In the phd package, I automatically generate the default settings from the |.dtx| files. In this case pgf will complain.}

\begin{minted}{latex}
%</install>

%<install>\endbatchfile
\end{minted}

\paragraph{The character table check } The second mechanism that Doc uses to ensure that a |dtx|  file is uncorrupted
is a character table. If you put the following command verbatim into
your |dtx|  file, then \pkg{Doc} will ensure that no unexpected character translation
took place in transport:

\begin{minted}[
frame=lines,
framesep=2mm,
baselinestretch=1.2,
fontsize=\footnotesize,
linenos,gobble=0,
]{latex}
% \CharacterTable
% {Upper-case \A\B\C\D\E\F\G\H\I\J\K\L\M\N\O\P\Q\R\S\T\U\V\W\X\Y\Z
% Lower-case \a\b\c\d\e\f\g\h\i\j\k\l\m\n\o\p\q\r\s\t\u\v\w\x\y\z
% Digits \0\1\2\3\4\5\6\7\8\9
% Exclamation \! Double quote \" Hash (number) \#
% Dollar \$ Percent \% Ampersand \&
% Acute accent \’ Left paren \( Right paren \)
% Asterisk \* Plus \+ Comma \,
% Minus \- Point \. Solidus \/
% Colon \: Semicolon \; Less than \<
% Equals \= Greater than \> Question mark \?
% Commercial at \@ Left bracket \[ Backslash \\
% Right bracket \] Circumflex \^ Underscore \_
% Grave accent \‘ Left brace \{ Vertical bar \|
% Right brace \} Tilde \~}
A success message looks like this:
***************************
* Character table correct *
***************************

and an error message looks like this:
! Package doc Error: Character table corrupted.
\end{minted}

\paragraph{DoNotIndex} When producing an index, \pkg{doc} normally indexes every control sequence
(i.e., backslashed word or symbol) in the code. The problem with this level
of automation is that many control sequences are uninteresting from the
perspective of understanding the code. For example, a reader probably
doesn’t want to see every location where \cs{if} is used—or \cs{the} or \cs{let} or
\cs{begin} or any of numerous other control sequences.

As its name implies, the \cs{DoNotIndex} command gives |Doc| a list of control
sequences that should not be indexed. \cs{DoNotIndex} can be used any
number of times, and it accepts any number of control sequence names per
invocation:

\begin{minted}[
frame=lines,
framesep=2mm,
baselinestretch=1.2,
bgcolor=white,
fontsize=\footnotesize,
linenos
]{latex}
\DoNotIndex{\#,\$,\%,\&,\@,\\,\{,\},\^,\_,\~,\ }
\DoNotIndex{\@ne}
\DoNotIndex{\advance,\begingroup,\catcode,\closein}
\DoNotIndex{\closeout,\day,\def,\edef,\else,\empty, \endgroup}
\end{minted}


\subsection{User documentation}

We can finally start writing the user documentation. A typical beginning
looks like this:

\begin{minted}[
frame=lines,
framesep=2mm,
baselinestretch=1.2,
bgcolor=white,
fontsize=\footnotesize,
linenos
]{latex}
% \title{The \textsf{package} package\thanks{This document
% corresponds to \textsf{package}~\fileversion,
% dated~\filedate.}}
% \author{your name \\ \texttt{your e-mail address}}
%
% \maketitle
\end{minted}


The title can certainly be more creative, but note that it’s common for
package names to be typeset with \docAuxCommand{textsf} and for \docAuxCommand{thanks} to be used to
specify the package version and date. This yields one of the advantages
of literate programming: Whenever you change the package version (the
optional second argument to \docAuxCommand{ProvidesPackage}), the user documentation
is updated accordingly. Of course, you still have to ensure manually that
the user documentation accurately describes the updated package.

Write the user documentation as you would any \latexe document, except
that you have to precede each line with a |\%|. Note that the |ltxdoc| document
class is derived from article, so the top-level sectioning command is
|\section|, not |\chapter|.



\section{General tips for defining a Class}

Evaluate, if there is a class that is nearer to what you wish to achive. If not do a set of
requirements.

Book structure - start with book or |Octavo| if you need to hack extensively. If not use memoir, |koma| or |tufte-book|.

Paragraph looks

Lists

Figures

Bibliography and citations

Footnotes

Index

Title pages

Book Cover

Language support

Mathematics

Graphs and figures

Typography - fonts, indentations fontsize etc

headers and footers


\section{Declaring Options}

Most classes or packages will have a good deal of options. These are declared using the
\docAuxCommand{DeclareOption} command. In this part no package loading should take place.

\begin{docCommand} {DeclareOption} { \marg{option} \marg{code}}
  The argument option is the name of the option being declared and the \marg{code} is the
  code that will execute if this option is requested.
\end{docCommand}


\begin{docCommand}{DeclareOption*} { \marg{code}}
  The argument \meta{code} in the star version of the command specifies the action to be 
  taken if an unknown option is specified. Within this argument the \docAuxCommand{CurrentOption}
  refers to the name of the option in question. 
  
\end{docCommand}

For example one could pass all such options
  to another package, using:
  \begin{verbatim}
  \DeclareOption*{\PassOptionsToPackage{\CurrentOption}{A}}
  \end{verbatim}


\section{Executing Options}

Normally after the options have been defined, one would need to provide default values and 
the options need to be executed. 

\begin{docCmd} {ExecuteOptions} { \marg{option list}}
  
\end{docCmd}

You can also |\ExecuteOptions| when declaring other options. There is one caveat. This command
can only be executed prior to executing the |\ProcessOptions| command because, as one of
its last actions, the latter command reclaims all of the memory taken up by the code for
the declared options.

\begin{docCmd} {ProcessOptions*} {}
\end{docCmd}

For some packages it is preferable or essential to process options in the order they
appear in the |usepackage| commands rather than using the order given through the
sequence of the \refCmd{DeclareOption} commands. In this case it one has to use
the star version of the command, i.e, |\ProcessOptions*| rather than |\ProcessOptions|.

\section{Special Commands for class files}

It is sometimes preferable to define a new class based on another and hence to extend it.
To support this concept the \latexe kernel provides two commands, \docAuxCommand{LoadClass} and
\docAuxCommand{PassOptionsToClass}. These two commands can then be used to develop a new class, by adding and extending the functionality of the loaded class.

\begin{docCommands}
\refCom{LoadClass}{ \oarg{option list}\marg{class}\oarg{release}}
\end{docCommands}  
  
For example the |ltxdoc| class loads the standard |article| class. The \pkg{tufte-book} class loads
the |book| class. The best way to understand the concepts discussed here is to
study these classes.

\section{A minimal class}

\begin{texexample}{Model Class}{ex:modelclass}

\begin{filecontents}{phdexampleclass.cls}
\NeedsTeXFormat{LaTeX2e}
\ProvidesClass{phdexampleclass}[2015/07/07]
\renewcommand\normalsize{\fontsize{}{10pt}{12pt}\selectfont}
\setlength\textwidth{6.5in}
\setlength\textheight{5in}
\pagenumbering{arabic}
\end{filecontents}

\end{texexample}

\vfill
\endinput





















\chapter{Pattern Matching}

The |string| library of Lua, offers a number of pattern matching functions. Unlike many other languages it does not offer a full regular expression library. One can take this at first to be a disantvantage, but the reasons are explained in \textit{Programming in Lua} by Roberto Ierusalimschy, clearly:

\begin{quote}
Unlike several other scripting languages, Lua uses neither \texttt{POSIX} regex nor
Perl regular expressions for pattern matching. The main reason for this decision
is size: a typical implementation of \texttt{POSIX} regular expressions takes more than
4000 lines of code. This is about the size of all Lua standard libraries together.
In comparison, the implementation of pattern matching in Lua has less than
600 lines. Of course, pattern matching in Lua cannot do all that a full POSIX
implementation does. Nevertheless, pattern matching in Lua is a powerful tool,
and includes some features that are difficult to match with standard POSIX
implementations.
\end{quote}

\CMDI{string.find()}
searches for a pattern inside a given subject string. The simplest form of a \textit{pattern} is a word, which matches only a copy of itself.

For instance, the pattern ‘Lua’ will search for the substring “Lua” inside the
subject string. When |find| finds its pattern, it returns two values: the \textit{index} where the match begins and the index where the match ends. If it does not find a match, it returns |nil|. Example~\ref{ex:patterns1} demonstrates the |find| function. We use the |luacode| environment to avoid dealing with catcodes and we format the results in maths mode.

\begin{texexample}{Pattern Matching with find}{ex:patterns1}
\begin{luacode*}
s = "hello Lua world"
i, j = string.find(s,"Lua")
tex.print("indices $i="..i..",".."j="..j.."$")
\end{luacode*}
\end{texexample}

\CMDI{string.sub}
When a match succeeds, we can call string.sub with the values returned by
string.find to get the part of the subject string that matched the pattern. For
simple patterns, this is the pattern itself.
The string.find function has an optional third parameter: an index that
tells where in the subject string to start the search. This parameter is useful
when we want to process all the indices where a given pattern appears: we
search for a new match repeatedly, each time starting after the position where
we found the previous one. As an example, the following code makes a table
with the positions of all newlines in a string:

We will see later a simpler way to write such loops, using the string.gmatch
iterator.

\begin{texexample}{Iterating over all matches}{}
\edef\tempstring{\string\\par..is this is a is test \string\\par and is this another?}
\begin{luacode}
local s = "\tempstring".."This is is a string and this is another"
local t = {}
local i= 0
while true do
    i = string.find(s, "\\par", i+1)
    if i == nil then break end
    t[#t + 1] = i
    tex.print(i)
end
   
\end{luacode}
\end{texexample}

\section{Patterns}

Patterns can be more useful with \textit{character classes}. 

\begin{longtable}{>{\color{blue}}ll}
. & all characters\\
\%a & letters\\
\%c &control characters\\
\%d &digits\\ 
\%g &printable characters except spaces\\
\%l &lower-case letters\\
\%p &punctuation characters\\
\%s &space characters\\
\%u  &upper-case letters\\
\%w  &alphanumeric characters\\
\%x  &hexadecimal digits\\
\end{longtable}





\makeatletter
\long\def\auxm#1(#2);{%
  \def\Xtemp{#1}
  \def\Ytemp{#2}
  \parindent=0pt
  \par\leavevmode
  \hangafter=1\relax   \hangindent=1em\relax
  \bgroup  
   \bfseries\sffamily\color{red}\Xtemp\,\color{black}(\texttt{\Ytemp})\hskip0.8em
  \egroup
}

   
\newenvironment{docLpeg}[1]{
  \auxm#1;
  }  
{%
\@@par
\smallskip\parindent=1em }
\makeatother

\chapter{Lua LPeg}

\section{Pattern Matching}
\parskip6pt

LPeg is a powerful notation for matching text data, which is more capable than Lua string patterns and standard regular expressions. However, like any language you need to know the basic words and how to combine them. These patterns like their  cousins the regular expressions in languages like Perl or Javascript have a steep learning curve. The Lpeg name derives from the acronym PEG which stands for Parsing Expression Grammars. 
See also \href{http://www.inf.puc-rio.br/~roberto/lpeg/}{Parsing Expression Grammars For Lua}. 

The |lpeg| library provides functions whose names start with the prefix |lpeg|. such as |lpeg.P|, |lpeg.R|, |lpeg.S|. We can avoid having to write these prefixes by using local variables. In Lua it is better to use local variables wherever possible. RiscLua has the extra facility that we can make statements like

|   local P,R,S = lpeg.P, lpeg.R, lpeg.S|

so that P,R,S become local variables standing for lpeg.P,lpeg.R,lpeg.S. Such statements help to make code less verbose and more efficient. All our examples of code will presume that these and similar abbreviations have occurred previously.

The whole tangle of magic characters and unreadable syntax, which bedevils RegExp notation, arises from the elementary error of not distinguishing between different datatypes. For lpeg, patterns are objects in their own right, and are not strings. 

One of the recurring themes in the development of programming has been that trouble usually comes from conflating things that are not the same. Booleans are not really integers. Floating point numbers and whole numbers are different beasts. We are used to distinguishing between the number 123 and the string "123". So now we have to smarten up and recognize the difference between these and the patterns P(123) and P("123"). The former is a pattern that matches any string that is at least 123 characters long, and the latter is a pattern that matches any string that starts with the prefix "123". The function lpeg.P is an all-purpose converter of things into patterns. For example, P(true)is a pattern that matches anything, whereas P(false) matches nothing. P can convert functions and tables, too, if they are of an appropriate kind. It converts patterns to themselves. But what are patterns? That they are an abstract datatype is an answer that will probably not satisfy your curiosity. The best way to understand them is to see examples, understand how they can be combined, and see them in action the self running examples here.

Given a string $s$ and a pattern $p$ we say that p \textit{succeeds} on s if s begins with a substring that matches p. Otherwise we say that p fails on s. The expression |p:match(s)| returns a non-nil value if p succeeds on s and nil if it fails. The non-nil value depends on what sort of pattern is being used, in particular on what the pattern captures. The default situation is that the non-nil value is one plus the index of the last character matched. In the case of literal string patterns like |P"Poly"|, the value returned will be the index of the character following the matching prefix.


\subsection{String Equality}

The simplest pattern we can write is one that checks if a string is equal to another string. If there is no match L

\begin{texexample}{String Equality}{ex:strequality}
\bgroup
\catcode`\#=11
\begin{luacode}
if type(tex) == 'table' then print = tex.print end
local temp = (lpeg.P("hello"):match("world")) 
if temp then
   print(z)
   else print('no match returned nil') 
end   
local temp1 = (lpeg.P("+hello"):match("+hello world")) 
print(temp1)
\end{luacode}
\egroup
\end{texexample}


Always when we use an |lpeg| expression and pass it to the \tex engine, we need to check for a nil value. We can handle errors
in a more sophisticated way later on.


\emphasis{lpeg, match,P}
\startnumberat{10}
\begin{texexample}{}{}
\begin{luacode}
   local match = lpeg.match   
   local P = lpeg.P           
   tex.print(match(P('Za'), 'Zaza'))
   if match(P('za'), 'Zaza') then 
      tex.print(match(P('za'), 'Zaza')) 
   else tex.print('nil') 
   end
\end{luacode}
\end{texexample}

To match a pattern against a block of text, we need to first define a \emph{pattern} or as it is sometimes referred in the literature a needle, as in `find the needle in haystack'. We first start by defining some shorthands in line~\ref{match}-\ref{match1}



The \luacmd{lpeg.P} is an \emph{operator} and operates on the string, whereas \luacmd{lpeg.match} is a function. The LPeg patterns are first class objects. These patterns are regular Lua values (represented by userdata). In the example above the match matches a string exactly. The verbosity of the code must not discourage you as the aptterns are composable. All the operators are represented by single letters.  The above pattern matches at the beginning of the string and returns the end position of the string. 

To capture a range of characters, for example the lower case letters of the English alphabet  ['a-z'] we have to use \luacmd{lpeg.R}. Notice the \textbf{R} stands for range.

\emphasis{R}
\begin{texexample}{Matching Patterns lpeg.P}{}
\begin{luacode}
   local match = lpeg.match 
   local R = lpeg.R 
   local P = lpeg.P
   tex.print(match(R'az'^1 * -1, 'abcdefgi'))
   if match(R'ac', 'abcdefgh') then 
      tex.print(match(R'ac', 'abcdefgh')) 
   else 
      tex.print('nil') 
   end
   
local str =  (P'Poly'):match("Polymath")  
tex.print(str)  --> 5   
\end{luacode}
\end{texexample}

What happened in the example, we defined the letters |R'az\^1 * -1|. Lots of symbols but they will start making more sense in a minute. The  |(\^1)| means a sequence of at least one. Unlike with Lua patterns or regular expressions, you don't have to worry about escaping 'magic' characters - every character in a string stands for itself: '(','+','*', '\%', etc are just their ASCII equivalents. 

Earlier I have mentioned that patterns are composable. The + operator means \emph{either} one or the other pattern. There are numerous operators, covering all the cases as shown in Table~\ref{tbl:peg}

\medskip

\bgroup
\captionof{table}{Basic operations for creating patterns}\label{tbl:peg}
\begin{longtable}[c]{lp{8cm}}
\toprule
Operator	          &Description\\
\midrule
|lpeg.P(string)|	  &Matches string literally\\
|lpeg.P(n)|	       &Matches exactly \meta{n} characters\\
|lpeg.S(string)|  &Matches any character in string (Set)\\
|lpeg.R("xy")|	   &Matches any character between \meta{x} and \meta{y} (Range)\\
|patt^n|	         &Matches at least \meta{n} repetitions of |patt|\\
|patt^-n|	       &Matches at most n repetitions of |patt|\\
|patt1 * patt2|	 &Matches |patt1| followed by |patt2|\\
|patt1 + patt2|	 &Matches |patt1| or |patt2| (ordered choice)\\
|patt1 - patt2|	 &Matches |patt1| if |patt2| does not match\\
|-patt|	          &Equivalent to ("" - patt)\\
|#patt|	          &Matches patt but consumes no input\\
|lpeg.B(patt)|	 &Matches patt behind the current position, consuming no input\\
\bottomrule
\end{longtable}
\egroup

\begin{longtable}{ll}
P & literal\\
S & set \\
R & Range \\
* & And\\
+ & Or\\
\^& At least\\
1 & Any single\\
- & Except\\
\end{longtable}

Composition of patterns

Lua identifier

\begin{teX}
local identifier = (lpeg.R('az', 'AZ') + '_') * (lpeg.R('az','AZ','09') + '_')^0
\end{teX}

\section{Functions}

\begin{docLpeg}{lpeg.match (pattern, subject [, init])}
The matching function. It attempts to match the given pattern against the subject string. If the match succeeds, returns the index in the subject of the first character after the match, or the captured values (if the pattern captured any value).
\end{docLpeg}

An optional numeric argument init makes the match start at that position in the subject string. As usual in Lua libraries, a negative value counts from the end.

Unlike typical pattern-matching functions, match works only in anchored mode; that is, it tries to match the pattern with a prefix of the given subject string (at position init), not with an arbitrary substring of the subject. So, if we want to find a pattern anywhere in a string, we must either write a loop in Lua or write a pattern that matches anywhere. This second approach is easy and quite efficient; see examples.


\begin{docLpeg}{lpeg.type (value)}
If the given value is a pattern, returns the string "pattern". Otherwise returns nil.
\end{docLpeg}

\begin{docLpeg}{lpeg.version ()}
Returns a string with the running version of LPeg.
\end{docLpeg}

\begin{docLpeg}{lpeg.setmaxstack (max)}
Sets a limit for the size of the backtrack stack used by LPeg to track calls and choices. (The default limit is 400.) Most well-written patterns need little backtrack levels and therefore you seldom need to change this limit; before changing it you should try to rewrite your pattern to avoid the need for extra space. Nevertheless, a few useful patterns may overflow. Also, with recursive grammars, subjects with deep recursion may also need larger limits.
\end{docLpeg}


\begin{texexample}{Matching Patterns lpeg.P}{}
\begin{luacode}
local P, match = lpeg.P, lpeg.match
local print = tex.print
       
local either_ab = (P'a' + P'b')^1
        tex.print(match(either_ab,  'ababcdefg'))
     str = "Fetchez; la vache!"
print(lpeg.anywhere("a" ):match(str))
print(lpeg.anywhere("la"):match(str))
print(lpeg.anywhere("ac"):match(str))  



 str = "aaababaab,a;"
rep1 = {
    { "a",  "\\textcolor{red}{x}" },
    { "aa", "y" },
}

rep2 = {
    { "aa", "\\textcolor{blue}{y}" },
    { "a",  "x" },
    {lpeg.patterns.semicolon, "\\textcolor{purple}{semicolon;}"},
    {lpeg.patterns.comma, "\\textcolor{purple}{comma,}"}
}

print(lpeg.replacer(rep1):match(str),'\\par')

print(lpeg.replacer(rep2):match(str), '\\par') 

str ='+12.267'
print(lpeg.match(lpeg.patterns.sign, str))
\end{luacode}
\end{texexample}


It attempts to match the given pattern against the subject string. If the match succeeds, returns the index in the subject of the first character \emph{after} the match, or the values of captured values (if the pattern captured any value).

An optional numeric argument init makes the match starts at that position in the subject string. As usual in Lua libraries, a negative value counts from the end.

Unlike typical pattern-matching functions, match works only in anchored mode; that is, it tries to match the pattern with a prefix of the given subject string (at position init), not with an arbitrary substring of the subject. So, if we want to find a pattern anywhere in a string, we must either write a loop in Lua or write a pattern that matches anywhere. This second approach is easy and quite efficient.

The lualibs library which we load by default when LuaLaTeX is detected comes with dozens of predefined patterns and a number of utilities.

\section[\textbackslash lpegreplace]{\textbackslash lpegreplace(table, text)}

\begin{docLpeg}{lpeg.replacer(table)}
Accepts a list of pairs and returns a pattern that substitutes any first elements of a given pair by its second element. The latter can be a string, a hashtable, or a function (whatever is permissible with |lpegs.Cs|).
The order of the elements of a table is important, as lpeg's match the leftmost element of a disjunction first.
\end{docLpeg}


\section{Basic Captures}

Getting the index after a match is all very well, and you can then use \luacmd{string.sub} to extract the strings. But there are ways of explicitly asking for captures:

\begin{phdverbatim}
> C = lpeg.C  -- captures a match
> Ct = lpeg.Ct -- a table with all captures from the pattern
\end{phdverbatim}


\begin{texexample}{Matching Patterns lpeg.P}{ex:peg0}
\begin{luacode}
 local C = lpeg.C
 Ct = lpeg.Ct
 digit = lpeg.R'09'   --  anything from '0' to '9'
 digits = digit^1     --  a sequence of at least one digit
 cdigits= C(digits)   --  capture digits
 letters = lpeg.R'az'^1 --  captures letters
 cletters= C(letters)  --  captures letters
 tex.print(cdigits:match '123A9AAAA')
 tex.print(cletters:match 'onlylowercaseAAAA')
\end{luacode}
\end{texexample}

In Example~\ref{ex:peg} we define two functions $tobold(n)$ and $toitalic(char)$, which will be
mapped to the capture. 

\begin{texexample}{Matching Patterns lpeg.P}{ex:peg}
\begin{luacode}
local tobold = function(n)
         return "\\textbf{"..tostring(n).."}%\n" 
      end
      
local toitalic = function(a)      
        return "\\textit{"..a.."}%\n"
      end  
      
local C  = lpeg.C
local Cf = lpeg.Cf
local Ct = lpeg.Ct
 
digit = lpeg.R'09'/tobold   --  anything from '0' to '9'
digits = digit              --  a sequence of at least one digit
letters = lpeg.R'AZ'/toitalic
cdigits= (Cf(digits, tobold) + Cf(letters, toitalic))^1  --  capture digits
 
 local function test(str)
    if (cdigits:match(str)) then
      print(cdigits:match(str))
    else
      print("No capture")
    end
  end
  
test("123456HAMBURGERVIS")
\end{luacode}
\end{texexample}


A capture is a pattern that produces values (the so called semantic information) according to what it matches. LPeg offers several kinds of captures, which produces values based on matches and combine these values to produce new values. Each capture may produce zero or more values.

The following table summarizes the basic captures:
\captionof{table}{Basic operations for creating patterns}\label{tbl:peg}
\nobreak
\begin{longtable}[c]{|l|p{8cm}|}
\hline
Operation	   &What it Produces\\
\hline

lpeg.Cf(patt, func)	&a folding of the captures from patt\\
lpeg.Cg(patt [, name])	&the values produced by patt, optionally tagged with name\\
lpeg.Cp()	&the current position (matches the empty string)\\
lpeg.Cs(patt)	&the match for patt with the values from nested captures replacing their matches\\
lpeg.Ct(patt)	&a table with all captures from patt\\
patt / string	&string, with some marks replaced by captures of patt\\
patt / number	&the n-th value captured by patt, or no value when number is zero.\\
patt / table	&table[c], where c is the (first) capture of patt\\
patt / function	&the returns of function applied to the captures of patt\\
lpeg.Cmt(patt, function)	&the returns of function applied to the captures of patt; the application is done at match time\\
\hline
\end{longtable}





\begin{docLpeg}{lpeg.C(patt)}
the match for |patt| plus all captures made by patt
\end{docLpeg} 

\begin{docLpeg}{lpeg.Carg(n)}	
The value of the nth extra argument to lpeg.match (matches the empty string).
\end{docLpeg}

\begin{docLpeg}{lpeg.Cb(name)}
The values produced by the previous group capture named name (matches the empty string)
\end{docLpeg}

\begin{docLpeg}{lpeg.Cc(values)}
The given values (matches the empty string)
\end{docLpeg}

\begin{docLpeg}{lpeg.Cf(patt, func)}
a folding of the captures from patt
\end{docLpeg}

\begin{docLpeg}{lpeg.Cg(patt [, name])}
the values produced by patt, optionally tagged with name
\end{docLpeg}

\begin{docLpeg}{lpeg.Cp()}
the current position (matches the empty string)
\end{docLpeg}

\begin{docLpeg}{lpeg.Cs(patt)}
the match for patt with the values from nested captures replacing their matches
\end{docLpeg}


\section{Built in patterns}

The following is an extract from the |ConTeXt| guide. The example captures all words in square brackets. 

\begin{texexample}{Matching Patterns lpeg.P}{}

\begin{luacode}
if type(tex) == 'table' then print = tex.print end
local P, R, C, Ct = lpeg.P, lpeg.R, lpeg.C, lpeg.Ct
local pattern = Ct((P("[") * C(R("az")^0) * P(']') + P(1))^0)
local words = lpeg.match(pattern,"a [ first ] and [second] word, some utf [third]")
tex.print(words)  

 
\end{luacode}
\end{texexample}




\footnote{\protect\url{http://www.inf.puc-rio.br/~roberto/lpeg/lpeg.html\#intro}}

\endinput

local alpha, cntrl, digit, graph, lower, punct, space, upper, alnum, xdigit =
   lpeg.alpha, lpeg.cntrl, lpeg.digit, lpeg.graph, lpeg.lower, lpeg.punct,
   lpeg.space, lpeg.upper, lpeg.alnum, lpeg.xdigit

local pattern = lpeg.Ct(lpeg.upper)   
local upwords = lpeg.match(pattern "Some words") 

http://leafo.net/guides/parsing-expression-grammars.html#what-is-a-peg
%% Check on why fonttable gives problems in Index
\parindent1em
\chapter{Symbols}

\section{Introduction}
\label{ch:comprehensivesymbols}

The \pkgname{phd} package, preloads a number of packages, to provide as
many symbols as possible irrespective of the \tex engine used. Many of these
symbols can easily be replaced by the use of \pkgname{fontspec} and if
the package is set to unicode math with the use of and suitable fonts Open Type Fonts. What follows has been largely copied from \emph{The Comprehensive List of \latexe Symbols}, which has been the authoritative publication, using almost all available symbols. 
The publication lists many symbols which I have dropped due to having exceeded the number of math alphabets allowed by \tex. 


With the newer engines \xetex \luatex you can now use any symbol you can imagine, but there is still room and advantages for using commands. It is at least for me faster than looking up a symbol's unicode character or trying it out with a screen keyboard (unless of course you are using a foreign language keyboard). 

The sections that follow describe the commands and packages that are
available, by simply including the \pkgname{phd} package. Most of the conflicts have been resolved and I am hoping that in the next version we will add some more symbols. 
Currently these are over 1500 as commands and in excess of 60,000 unicode glyphs, provided you have access to the fonts.\footnote{This document has been compiled using \luatex.}
  

\subsection{Reserved Symbols}
\tex has a number of symbols that need to be escaped, as they have 
special meanings during processing see Table~\vref{special-escapable} and also Chapter~\vref{ch:characters}.


\begin{symtable}{\latexe{} Escapable ``Special'' Characters}
\index{special characters=``special'' characters}
\index{escapable characters}
\index{underline}
\label{special-escapable}
\begin{tabular}{*6{ll@{\qqquad}}ll}
\K\$   & \K\%   & \K\_$\,^*$  & \Kp\}  & \K\&   & \K\#   & \Kp\{   \\
\end{tabular}
\end{symtable}

The \latexe kernel command \refCom{@sanitize} changes the catcode of these characters so they can be included in commands such as |\index|. In text just escape them with a (\textbackslash).


\begin{longsymtable}{Predefined \latexe{} Text-mode Commands}
\index{inequalities}
\index{tilde}
\index{underline}
\index{copyright}
\idxboth{dot}{symbols}
\index{dots (ellipses)} \index{ellipses (dots)}
\idxboth{legal}{symbols}
\label{text-predef}
\begin{longtable}{lll@{\qqquad}lll}
\indexTextcomp\textasciicircum$^*$    					& \indexTextcomp\textless                             \\
\indexTextcomp\textasciitilde$^*$     						& \indexTextcomp[\ltextordfeminine]\textordfeminine   \\
\indexTextcomp\textasteriskcentered   					& \indexTextcomp[\ltextordmasculine]\textordmasculine \\
\indexTextcomp{\textbackslash}          				    & \indexTextcomp\textparagraph$^\dag$                 \\
texbar                                              & \indexTextcomp\textperiodcentered                   \\
\indexTextcomp{textbraceleft}           $^\dag$   & \indexTextcomp\textquestiondown                     \\
\indexTextcomp\textbraceright$^\dag$  & \indexTextcomp\textquotedblleft                     \\
\indexTextcomp\textbullet             & \indexTextcomp\textquotedblright                    \\
\indexTextcomp[\ltextcopyright]\textcopyright$^\dag$
                          & \indexTextcomp\textquoteleft                        \\
\indexTextcomp\textdagger$^\dag$      & \indexTextcomp\textquoteright                       \\
\indexTextcomp\textdaggerdbl$^\dag$   & \indexTextcomp[\ltextregistered]\textregistered     \\
\indexTextcomp\textdollar$^\dag$      & \indexTextcomp\textsection$^\dag$                   \\
\indexTextcomp\textellipsis$^\dag$    & \indexTextcomp\textsterling$^\dag$                  \\
\indexTextcomp\textemdash             & \indexTextcomp[\ltexttrademark]\texttrademark       \\
\indexTextcomp\textendash             & \indexTextcomp\textunderscore$^\dag$                \\
\indexTextcomp\textexclamdown         & \indexTextcomp\textvisiblespace                     \\
\indexTextcomp\textgreater                                                      \\
\end{longtable}

\bigskip
\twosymbolmessage

\bigskip
\begin{tablenote}[*]
  \docAuxCommand{^} and
%  \cmdI[\string\~{}]{\~{}}\verb|{}| can be used instead of
  \docAuxCommand{textasciicircum} and \docAuxCommand{textasciitilde}.  See the
  discussion of ``\texttt{\textasciitilde}'' \vpageref[below]{page:tildes}.
\end{tablenote}

\bigskip
\usetextmathmessage[\dag]
\end{longsymtable}



\begin{symtable}{\latexe{} Commands Defined to Work in Both Math and Text Mode}
\index{dots (ellipses)} \index{ellipses (dots)}
\index{copyright}
\idxboth{legal}{symbols}
\label{math-text}
\begin{tabular}{*3{lll@{\qqquad}}lll}
\indexTextcomp\$ & \indexTextcomp\_              & \indexTextcomp\ddag    & \Vp\{ \\
\indexTextcomp\P & \indexTextcomp[\ltextcopyright]\copyright
                         & \indexTextcomp\dots    & \Vp\} \\
 & \indexTextcomp\dag            & \indexTextcomp\pounds          \\%V\S removed
\end{tabular}

\bigskip
\twosymbolmessage
\end{symtable}

\begin{symtable}{AMS Commands Defined to Work in Both Math and Text Mode}
\index{check marks}
\label{ams-math-text}
\begin{tabular}{*2{ll@{\qquad}}ll}
\X\checkmark & \X\circledR & \X\maltese
\end{tabular}
\end{symtable}


\begin{symtable}{Non-ASCII Letters (Excluding Accented Letters)}
\index{letters>non-ASCII} %\K\l to fix
\index{ASCII}
\label{non-ascii}
\begin{tabular}{*4{ll@{\qqquad}}ll}
\K\aa      & \Ks\DH     & \Ks\L      & \K\o       & \K\ss                   \\
\K\AA      & \Ks\dh     & &          & \K\O       & \K\SS                   \\
\K\AE      & \Ks\DJ     & \Ks\NG     & \K\OE      & \Ks\TH                  \\
\K\ae      & \Ks\dj     & \Ks\ng     & \K\oe      & \Ks\th                  \\
\end{tabular}

\bigskip
\begin{tablenote}[*]
  Not available in the OT1 \fntenc[OT1].  Use the \pkgname{fontenc}
  package to select an alternate \fntenc[T1], such as T1.
\end{tablenote}
\end{symtable}

\section{Punctuation marks}

\begin{longsymtable}{Punctuation Marks Not Found in OT1}
\index{punctuation}
\label{punc-no-OT1}
\begin{longtable}{*8l}
\Kt\guillemotleft  & \Kt\guilsinglleft & \Kt\quotedblbase & \Kt\textquotedbl \\
\Kt\guillemotright & \Kt\guilsinglright & \Kt\quotesinglbase \\
\end{longtable}
\end{longsymtable}


\begin{longsymtable}[PI]{\PI\ Decorative Punctuation Marks}
\index{punctuation}
\label{pi-punctuation}
\begin{longtable}{*5{ll}}
\indexDing{123} & \indexDing{125} & \indexDing{161} & \indexDing{163} \\
\indexDing{124} & \indexDing{126} & \indexDing{162} \\
\end{longtable}
\medskip
\begin{tablenote}
  To get these symbols, use the \pkgname{fontenc} package to select an
  alternate \fntenc[T1], such as~T1.
\end{tablenote}

\end{longsymtable}

\section{Accents}
\begin{symtable}{Text-mode Accents}
\index{accents}
\index{accents>acute=acute (\blackacchack\')}   
\index{accents>arc=arc (\blackacchack\newtie)}
\index{accents>breve=breve (\blackacchack\u)}   
\index{accents>caron=caron (\blackacchack\v)} 
\index{accents>cedilla=cedilla (\blackacc\c)} 
\index{accents>circumflex=circumflex (\blackacchack\^)}  
\index{accents>diaeresis=di\ae{}resis (\blackacchack\")}  
\index{accents>dot=dot (\blackacchack\. or \blackacc\d)} 
\index{accents>double acute=double acute (\blackacchack\H)}  
\index{accents>grave=grave (\blackacchack\`)}  
\index{accents>ogonek=ogonek (\encone{\blackacc\k})} 
\index{accents>ring=ring (\blackacchack\r)} 
\label{text-accents}
\begin{tabular}{*3{ll@{\qqquad}}ll}
\Q\"                                & \Q\`         & \Q\d         & \Q\r        \\
\Q\'                                & \QivBAR\ddag & \Qiv\G\ddag  & \Q\t        \\
\Q\.                                & \Q\~         & \Qv\h\S      &        \\ %Q/u removed
\Qe[\magicequal][\magicequalname]\= & \Q\b         & \Q\H         & \Qiv\U\ddag \\
\Q\^                                & \Q\c         & \Qt\k$^\dag$ & \Q\v        \\
\end{tabular}
\par\medskip
\begin{tabular}{ll@{\qqquad}ll}
\Q\newtie$^*$ & \Qc\textcircled
\end{tabular}

\bigskip
\begin{tablenote}[*]
  Requires the \TC\ package.
\end{tablenote}

\medskip
\begin{tablenote}[\dag]
  Not available in the OT1 \fntenc[OT1].  Use the \pkgname{fontenc}
  package to select an alternate \fntenc[T1], such as T1.
\end{tablenote}

\medskip
\begin{tablenote}[\ddag]
  Requires the T4 \fntenc[T4], provided by the \FC\ package.
\end{tablenote}

\medskip
\begin{tablenote}[\S]
  Requires the T5 \fntenc[T5], provided by the \VIET\ package.
\end{tablenote}

\bigskip
\begin{tablenote}
  \index{dotless i=dotless $i~(\imath)$>text mode} \index{dotless
  j=dotless $j~(\jmath)$>text mode} Also note the existence of
  \docAuxCommand{i} and \docAuxCommand{j}, which produce dotless versions of ``i'' and
  ``j'' (viz., ``\i'' and ``\j'').  These are useful when the accent
  is supposed to replace the dot in encodings that need to
  composite\index{composited accents} (i.e.,~combine) letters and
  accents.  For example, ``\verb|na\"{\i}ve|'' always produces a
  correct ``na\"{\i}ve'', while ``\verb|na\"{i}ve|'' yields the rather
  odd-looking na\"{i}ve
  \makeatletter
  ``na\add@accent{127}{i}ve''\index{i=\add@accent{127}{i}}
  \makeatother
  when using the OT1 \fntenc[OT1] and older versions of \latex.  Font
  encodings other than OT1 and newer versions of \latex properly
  typeset ``\verb|na\"{i}ve|'' as ``na\"{\i}ve''.
\end{tablenote}
\end{symtable}

\section{Diacritics and Accents}

Again the most convenient way to get diagritics is to use the
\pkgname{textcomp}. The \TC\ package defines all of the above as ordinary characters,
  and not as accents. Of course with Unicode and True Type fonts, the worlds accents and
  diagritics, make these tables pale in comparison. 

\begin{longsymtable}{\TC\ Diacritics}
\index{accents}
\index{accents>acute=acute (\blackacchack\')}   
\index{accents>breve=breve (\blackacchack\u)}  
\index{accents>caron=caron (\blackacchack\v)}  
\index{accents>diaeresis=di\ae{}resis (\blackacchack\")} 
\index{accents>double acute=double acute (\blackacchack\H)}
\index{accents>grave=grave (\blackacchack\`)}  
\index{diacritics}
  
\label{tc-accent-chars}
\begin{longtable}{*3{ll}}
\K\textacutedbl      & \K\textasciicaron    & \K\textasciimacron \\
\K\textasciiacute    & \K\textasciidieresis & \K\textgravedbl    \\
\K\textasciibreve    & \K\textasciigrave                         \\
\end{longtable}
\end{longsymtable}


\begin{longsymtable}{\TC\ Currency Symbols}
\idxboth{currency}{symbols}
\idxboth{monetary}{symbols}
\index{euro signs}
\label{tc-currency}
\begin{longtable}{*4{ll}}
\K\textbaht          & \K\textdollar$^*$     & \K\textguarani  & \K\textwon \\
\K\textcent          & \K\textdollaroldstyle & \K\textlira     & \K\textyen \\
\K\textcentoldstyle  & \K\textdong           & \K\textnaira    \\
\K\textcolonmonetary & \K\texteuro           & \K\textpeso     \\
\K\textcurrency      & \K\textflorin         & \K\textsterling$^*$ \\
\end{longtable}
\end{longsymtable}

\begin{symtable}[MARV]{\MARV\ Currency Symbols}
\idxboth{currency}{symbols}
\idxboth{monetary}{symbols}
\index{euro signs}
\label{marv-currency}
\begin{tabular}{*4{ll}ll}
\K\Denarius   & \K\EUR    & \K\EURdig   & \K\EURtm      & \K\Pfund      \\
\K\Ecommerce  & \K\EURcr  & \K\EURhv    & \K\EyesDollar & \K\Shilling   \\
{\arial \char"20AC}                      &                &                   &                      &                   \\
\end{tabular}

\bigskip

\begin{tablenote}
  The different euro signs are meant to be visually compatible with
  different fonts---\PSfont{Courier} (\texttt{\string\EURcr}),
  \PSfont{Helvetica} (\texttt{\string\EURhv}), \PSfont{Times Roman}
  (\texttt{\string\EURtm}), and the \MARV\ digits listed in
  \ref{marv-digits} (\texttt{\string\EURdig}).
%  
%
%\ifMDES
%  The \MDES\ package redefines \cmdI[\MDEStexteuro]{\texteuro} to be
%  visually compatible with one of three additional fonts:
%  \PSfont{Utopia}~({\usefont{TS1}{mdput}{m}{n}\char"BF}),
%  \PSfont{Charter}~({\usefont{TS1}{mdbch}{m}{n}\char"BF}), or
%  \PSfont{Garamond}~({\usefont{TS1}{mdugm}{m}{n}\char"BF}).
%\fi
%
\end{tablenote}
\end{symtable}


\begin{symtable}[WASY]{\WASY\ Currency Symbols}
\idxboth{currency}{symbols}
\idxboth{monetary}{symbols}
\label{wasy-currency}
\begin{tabular}{ll@{\qquad}ll}
\K\cent & \K\currency \\
\end{tabular}
\end{symtable}

There is another package providing Euro related signs the \pkgname{eurosym}. The package provides the commands, \docAuxCommand{geneuro}, \docAuxCommand{geneuronarrow}, \docAuxCommand{geneurowide} and \cmd{\officialeuro}. You can read more at \url{http://www.theiling.de/eurosym.html}. \texttt{eurosym}  provides a new symbol to be used for the European currency, the Euro. The specifications were taken from a picture in the c't magazine 11/98 p.211 and from Encyclopaedia Britannica, Book of the Year 2002 (thanks to Dr. Werner Gans).

\texttt{eurosym}'s Euro symbol is implemented in \texttt{MetaFont}, and thus fits smoothly into a \texttt{LaTeX} installation. It is now part of major Linux distributions, including Debian, Suse, Mandrake and probably others.

Apart from the official form, the eurosym package provides some generalisations that fit non-roman font faces better.

\ifEUSYM
\begin{symtable}[EUSYM]{\EUSYM\ Euro Signs}
\idxboth{currency}{symbols}
\idxboth{monetary}{symbols}
\index{euro signs}
\label{eurosym-euros}
\begin{tabular}{*4{ll}}
\K\geneuro & \K\geneuronarrow & \K\geneurowide & \K\officialeuro \\
\end{tabular}

\bigskip

\begin{tablenote}
  \cmd{\euro} is automatically mapped to one of the above---by
  default, \docAuxCommand{officialeuro}---based on a \EUSYM\ package option.
  \seedocs{\EUSYM}.  The \verb|\geneuro|\dots{} characters are
  generated from the current body font's ``C'' character and therefore
  may not appear exactly as shown.
\end{tablenote}

\begin{tablenote}
To use the symbol with fontspec see the package documentation.
\end{tablenote}
\end{symtable}
\fi



\begin{symtable}[CHINA]{\CHINA\ Currency Symbols}
\idxboth{currency}{symbols}
\idxboth{monetary}{symbols}
\index{euro signs}
\label{china-euro}
\begin{tabular}{ll@{\qquad}ll}
  \K\Euro & \K\Pound \\
\end{tabular}
\end{symtable}



\begin{symtable}{\TC\ Legal Symbols}
\index{copyright}
\idxboth{legal}{symbols}
\label{tc-legal}
\begin{tabular}{*2{lll@{\qquad}}lll}
\indexTextcomp\textcircledP & \indexTextcomp[\ltextcopyright]\textcopyright   
&\indexTextcomp\textservicemark \\

\indexTextcomp\textcopyleft 
& \indexTextcomp[\ltextregistered]\textregistered 
& \indexTextcomp[\ltexttrademark]\texttrademark \\
\end{tabular}

\bigskip
\twosymbolmessage
\medskip
\begin{tablenote}
  \hspace*{15pt}%
  See \url{http://www.tex.ac.uk/cgi-bin/texfaq2html?label=tradesyms}
  for solutions to common problems that occur when using these symbols
  (e.g.,~getting a~``\textcircled{r}'' when you expected to get
  a~``\textregistered'').
\end{tablenote}
\end{symtable}


\begin{symtable}[CCLIC]{\CCLIC\ Creative Commons License Icons}
\index{Creative Commons licenses}
\index{copyright}
\idxboth{legal}{symbols}
\label{creativecommons}
\begin{tabular}{*4{ll@{\qqquad}}ll}
\K\cc & \K\ccby & \K\ccnc$^*$ & \K\ccnd & \K\ccsa$^*$ \\
\end{tabular}

\bigskip
\begin{tablenote}[*]
  These symbols utilize the \pkgname{rotating} package and therefore
  display improperly in some DVI\index{DVI} viewers.
\end{tablenote}
\end{symtable}


\begin{symtable}{\TC\ Old-style Numerals}
\idxboth{old-style}{digits}
\index{numerals>old style}
\label{old-style-nums}
\begin{tabular}{*3{ll}}
\K\textzerooldstyle  & \K\textfouroldstyle  & \K\texteightoldstyle \\
\K\textoneoldstyle   & \K\textfiveoldstyle  & \K\textnineoldstyle  \\
\K\texttwooldstyle   & \K\textsixoldstyle   \\
\K\textthreeoldstyle & \K\textsevenoldstyle \\
\end{tabular}

\bigskip
\begin{tablenote}
  Rather than use the bulky \cmd{\textoneoldstyle},
  \cmd{\texttwooldstyle}, etc.\ commands shown above, consider using
  \docAuxCommand{oldstylenums}\verb|{|$\ldots$\verb|}| to typeset an old-style eg. abcde{\oldstylenums 123456789}fgh. These type of
symbols and commands become redundant with the correct font, as the old style numbers are a feature of the font. Not all fonts provide old style numbers.
\end{tablenote}
\end{symtable}

\section{Miscellaneous Symbols}

\begin{longsymtable}{Miscellaneous \TC\ Symbols}
\idxboth{musical}{symbols}
\index{tilde}
\label{tc-misc}
\begin{longtable}{lll@{\qquad}lll}
\indexTextcomp\textasteriskcentered & \indexTextcomp[\ltextordfeminine]\textordfeminine   \\
\indexTextcomp\textbardbl           & \indexTextcomp[\ltextordmasculine]\textordmasculine \\
\indexTextcomp\textbigcircle        & \indexTextcomp\textparagraph$^*$                    \\
\indexTextcomp\textblank            & \indexTextcomp\textperiodcentered                   \\
\indexTextcomp\textbrokenbar        & \indexTextcomp\textpertenthousand                   \\
\indexTextcomp\textbullet           & \indexTextcomp\textperthousand                      \\
\indexTextcomp\textdagger$^*$       & \indexTextcomp\textpilcrow                          \\
\indexTextcomp\textdaggerdbl$^*$    & \indexTextcomp\textquotesingle                      \\
\indexTextcomp\textdblhyphen        & \indexTextcomp\textquotestraightbase                \\
\indexTextcomp\textdblhyphenchar    & \indexTextcomp\textquotestraightdblbase             \\
\indexTextcomp\textdiscount         & \indexTextcomp\textrecipe                           \\
\indexTextcomp\textestimated        & \indexTextcomp\textreferencemark                    \\
\indexTextcomp\textinterrobang      & \indexTextcomp\textsection$^*$                      \\
\indexTextcomp\textinterrobangdown  & \indexTextcomp\textthreequartersemdash              \\
\indexTextcomp\textmusicalnote      & \indexTextcomp\texttildelow                         \\
\indexTextcomp\textnumero           & \indexTextcomp\texttwelveudash                      \\
\indexTextcomp\textopenbullet                                                 \\
\end{longtable}

\bigskip
\twosymbolmessage

\bigskip
\usetextmathmessage[*]

\end{longsymtable}
%
%\begin{symtable}[WASY]{Miscellaneous \WASY\ Text-mode Symbols}
%\label{wasy-text}
%\begin{tabular}{ll}
%\K\permil \\
%\end{tabular}
%\end{symtable}
%\idxbothend{body-text}{symbols}



\section{Mathematical symbols}
\label{math-symbols}
\idxbothbegin{mathematical}{symbols}


Most, but not all, of the symbols in this section are math-mode only.
That is, they yield a ``\texttt{Missing~\$ inserted}''\index{Missing
\$ inserted=``\texttt{Missing~\$ inserted}''} error message if not
used within \verb|$|$\ldots$\verb|$|, \verb|\[|$\ldots$\verb|\]|, or
another math-mode environment.  Operators marked as ``variable-sized''
are taller in displayed formulas, shorter in in-text formulas, and
possibly shorter still when used in various levels of superscripts or
subscripts.

% The following definition is used both in the discussion of disjoint
% union and in the "Joining and overlapping existing symbols" section.

\newcommand{\dotcup}{\ensuremath{\mathaccent\cdot\cup}}


Alphanumeric symbols (e.g., $\mathscr{L}$, and
|\varmathbb{Z}|) are usually produced using one of the math
alphabets in \ref{alphabets} rather than with an explicit symbol
command.  Look there first if you need a symbol for a transform,
number set, or some other alphanumeric.

Although there have been many requests on \ctt for a
contradiction\idxboth{contradiction}{symbols} symbol, the ensuing
discussion invariably reveals innumerable ways to represent
contradiction in a proof, including ``|\blitza|''~(\cmd{\blitza}),
``$\Rightarrow\Leftarrow$''~(\docAuxCommand{Rightarrow}\docAuxCommand{Leftarrow}),
``$\bot$''~(\docAuxCommand{bot}),
``$\nleftrightarrow$''~(\docAuxCommand{nleftrightarrow}), and
%``\textreferencemark''~(\docAuxCommand{textreferencemark}).  Because of the
%lack of notational consensus, it is probably better to spell out
%``Contradiction!''\ than to use a symbol for this purpose.  Similarly,
%discussions on \ctt have revealed that there are a variety of ways to
%indicate the mathematical notion of ``is
%defined\idxboth{definition}{symbols} as''.  Common candidates include
%``$\triangleq$''~(\docAuxCommand{triangleq}), ``$\equiv$''~(\docAuxCommand{equiv}),
%``$\coloneqq$''~(\emph{various}\footnote{In \TX, \PX, and \MTOOLS\ the
%symbol is called \docAuxCommand{coloneqq}.  In |\ABX\| and MNS\footnote{Do not use it uses too many aplhabets} it's called
%\cmdI[$\string\ABXcoloneq$]{\coloneq}.  In \CEQ\ it's called
%colonequals}.}), and ``$\stackrel{\text{\tiny
%def}}{=}$''~(\cmd{\stackrel}\verb|{|\cmd{\text}\verb|{\tiny|
%\verb|def}}{=}|).  See also the example of \cmd{\equalsfill}
%\vpageref[below]{equalsfill-ex}.  Depending upon the context,
%disjoint\index{disjoint union} union may be represented as
%``$\coprod$''~(\docAuxCommand{coprod}), ``$\sqcup$''~(\docAuxCommand{sqcup}),
%``$\dotcup$''~(\docAuxCommand{dotcup}), ``$\oplus$''~(\docAuxCommand{oplus}), or any
%of a number of other symbols.\footnote{\person{Bob}{Tennent} listed
%these and other disjoint-union symbol possibilities in a November~2007
%post to \ctt.}  Finally, the average\index{average} value of a
%variable~$x$ is written by some people as
%``$\overline{x}$''~(\verb|\overline{x}|)\incsyms\indexaccent[$\string\blackacc{\string\overline}$]{\overline},
%by some people as ``$\langle x \rangle$''~(\docAuxCommand{langle} \texttt{x}
%\docAuxCommand{rangle}), and by some people as ``$\diameter x$'' or
%``$\varnothing x$''~(\docAuxCommand{diameter} \texttt{x} or \docAuxCommand{varnothing}
%\texttt{x}).  The moral of the story is that you should be careful
%always to explain your notation to avoid confusing your readers.



\bigskip

\begin{symtable}{Math-Mode Versions of Text Symbols}
\index{underline}
\label{math-text-vers}
\begin{tabular}{*3{ll}}
\X\mathdollar   & \X\mathparagraph & \X\mathsterling   \\
\X\mathellipsis & \X\mathsection   & \X\mathunderscore \\
\end{tabular}

\bigskip
\usetextmathmessage

\end{symtable}

\subsection{CMLL}
The \pkgname{cmll} defines a handful of symbols useful in linear logic and not found in other
%font packages \cite{cmll}. The package defines unary operators, binary operators, large operators, binary relations and letter-like symbols |\Bot| $\Bot$ and $\simbot$
%and |\simbot|.

\begin{symtable}[CMLL]{\CMLL\ Unary Operators}
\idxboth{unary}{operators}
\idxboth{linear logic}{symbols}
\label{cmll-unary}
\begin{tabular}{*2{ll@{\qquad}}ll}
\K[!]\oc$^*$         & \K[\CMLLshneg]\shneg & \K[?]\wn$^*$ \\
\K[\CMLLshift]\shift & \K[\CMLLshpos]\shpos &              \\
\end{tabular}

\bigskip

\begin{tablenote}[*]
  \docAuxCommand{oc} and \docAuxCommand{wn} differ from~``!''  and~``?'' in
  terms of their math-mode spacing: \verb|$A=!B$| produces ``$A=!B$'',
  for example, while \verb|$A=\oc B$| produces ``$A=\mathord{!}B$''.
\end{tablenote}
\end{symtable}


`Linear implication' is not included in the grammar of connectives, but is definable in CLL using linear negation and multiplicative disjunction, by $A⊸B:=A^{{\pan ⊥}}$.


\begin{symtable}{Binary Operators}
\idxboth{binary}{operators}
\index{division}
\idxboth{linear logic}{symbols}
\label{bin}
\begin{tabular}{*4{ll}}
\X\amalg           & \X\cup          & \X\oplus    & \X\times           \\
\X\ast             & \X\dagger       & \X\oslash   & \X\triangleleft    \\
\X\bigcirc         & \X\ddagger      & \X\otimes   & \X\triangleright   \\
\X\bigtriangledown & \X\diamond      & \X\pm       & \X\unlhd$^*$       \\
\X\bigtriangleup   & \X\div          & \X\rhd$^*$  & \X\unrhd$^*$       \\
\X\bullet          & \X\lhd$^*$      & \X\setminus & \X\uplus           \\
\X\cap             & \X\mp           & \X\sqcap    & \X\vee             \\
\X\cdot            & \X\odot         & \X\sqcup    & \X\wedge           \\
\X\circ            & \X\ominus       & \X\star     & \X\wr              \\
\end{tabular}

\bigskip
\notpredefinedmessage
\end{symtable}


\begin{symtable}{AMS Binary Operators}
\idxboth{binary}{operators}
\index{semidirect products}
\label{ams-bin}
\begin{tabular}{*3{ll}}
\X\barwedge        & \X\circledcirc     & \X\intercal$^*$    \\
\X\boxdot          & \X\circleddash     & \X\leftthreetimes  \\
\X\boxminus        & \X\Cup             & \X\ltimes          \\
\X\boxplus         & \X\curlyvee        & \X\rightthreetimes \\
\X\boxtimes        & \X\curlywedge      & \X\rtimes          \\
\X\Cap             & \X\divideontimes   & \X\smallsetminus   \\
\X\centerdot       & \X\dotplus         & \X\veebar          \\
\X\circledast      & \X\doublebarwedge  \\
\end{tabular}

\bigskip

\begin{tablenote}[*]
  \newcommand{\trpose}{{\mathpalette\raiseT{\intercal}}}
  \newcommand{\raiseT}[2]{\raisebox{0.25ex}{$#1#2$}}
%
  Some people use a superscripted \docAuxCommand{intercal} for matrix
  transpose\index{transpose}: ``\verb|A^\intercal|''~$\mapsto$
  ``$A^\intercal$''.  (See the May~2009 \ctt thread, ``raising math
  symbols'', for suggestions about altering the height of the
  superscript. and se.tex question \footnote{\url{http://tex.stackexchange.com/questions/30619/what-is-the-best-symbol -for-vector-matrix-transpose}})  \docAuxCommand{top} (\vref*{letter-like}), \verb|T|, and
  \verb|\mathsf{T}| are other popular choices: ``$A^\top$'',
  ``$A^T$'', ``$A^{\text{\textsf{T}}}$''.
\end{tablenote}

\end{symtable}



\subsection{St Mary Road Binary Operators}

%\begin{symtable}[ST]{\ST\ Binary Operators}
\idxboth{binary}{operators}
\idxboth{linear logic}{symbols}
\label{st-bin}
\captionof{table}{\ST\ Binary Operators}
\begin{longtable}{*3{ll}}
\X\baro                & \X\interleave          & \X\varoast             \\
\X\bbslash             & \X\leftslice           & \X\varobar             \\
\X\binampersand        & \X\merge               & \X\varobslash          \\
\X\bindnasrepma        & \X\minuso              & \X\varocircle          \\
\X\boxast              & \X\moo                 & \X\varodot             \\
\X\boxbar              & \X\nplus               & \X\varogreaterthan     \\
\X\boxbox              & \X\obar                & \X\varolessthan        \\
\X\boxbslash           & \X\oblong              & \X\varominus           \\
\X\boxcircle           & \X\obslash             & \X\varoplus            \\
\X\boxdot              & \X\ogreaterthan        & \X\varoslash           \\
\X\boxempty            & \X\olessthan           & \X\varotimes           \\
\X\boxslash            & \X\ovee                & \X\varovee             \\
\X\curlyveedownarrow   & \X\owedge              & \X\varowedge           \\
\X\curlyveeuparrow     & \X\rightslice          & \X\vartimes            \\
\X\curlywedgedownarrow & \X\sslash              & \X\Ydown               \\
\X\curlywedgeuparrow   & \X\talloblong          & \X\Yleft               \\
\X\fatbslash           & \X\varbigcirc          & \X\Yright              \\
\X\fatsemi             & \X\varcurlyvee         & \X\Yup                 \\
\X\fatslash            & \X\varcurlywedge       \\
\end{longtable}


%\end{symtable}


\begin{symtable}[WASY]{\WASY\ Binary Operators}
\idxboth{binary}{operators}
\label{wasy-bin}
\begin{tabular}{*4{ll}}
\X\lhd & \X\ocircle & \X\RHD   & \X\unrhd \\
\X\LHD & \X\rhd     & \X\unlhd            \\
\end{tabular}
\end{symtable}


\section{Unicode Binary Operators}
 \subsection{Binary operators}
 \index{binary operators}
 \begin{multicols}{2}
% \showsymbolbin+{000B}{}
 \showsymbolbin\pm{00B1}{}
 \showsymbolbin\cdotp{00B7}{}%?, \cmd\centerdot
 \showsymbolbin\times{00D7}{}
 \showsymbolbin\div{00F7}{}
 \showsymbolbin\dagger{2020}{}
 \showsymbolbin\ddagger{2021}{}
 \showsymbolbin\smblkcircle{2022}{}
 \showsymbolbin\fracslash{2044}{}
 \showsymbolbin\upand{214B}{}
% \showsymbolbin-{000D}{}
 \showsymbolbin\mp{2213}{}
 \showsymbolbin\dotplus{2214}{}
 \showsymbolbin\smallsetminus{2216}{}
 \showsymbolbin\ast{2217}{}
 \showsymbolbin\vysmwhtcircle{2218}{}
 \showsymbolbin\vysmblkcircle{2219}{}, {\small\cmd\bullet}
 \showsymbolbin\wedge{2227}{}, \cmd\land
 \showsymbolbin\vee{2228}{}, \cmd\lor
 \showsymbolbin\cap{2229}{}
 \showsymbolbin\cup{222A}{}
 \showsymbolbin\dotminus{2238}{}
 \showsymbolbin\invlazys{223E}{}
 \showsymbolbin\wr{2240}{}
 \showsymbolbin\cupleftarrow{228C}{}
 \showsymbolbin\cupdot{228D}{}
 \showsymbolbin\uplus{228E}{}
 \showsymbolbin\sqcap{2293}{}
 \showsymbolbin\sqcup{2294}{}
 \showsymbolbin\oplus{2295}{}
 \showsymbolbin\ominus{2296}{}
 \showsymbolbin\otimes{2297}{}
 \showsymbolbin\oslash{2298}{}
 \showsymbolbin\odot{2299}{}
 \showsymbolbin\circledcirc{229A}{}
 \showsymbolbin\circledast{229B}{}
 \showsymbolbin\circledequal{229C}{}
 \showsymbolbin\circleddash{229D}{}
 \showsymbolbin\boxplus{229E}{}
 \showsymbolbin\boxminus{229F}{}
 \showsymbolbin\boxtimes{22A0}{}
 \showsymbolbin\boxdot{22A1}{}
 \showsymbolbin\intercal{22BA}{}
 \showsymbolbin\veebar{22BB}{}
 \showsymbolbin\barwedge{22BC}{}
 \showsymbolbin\barvee{22BD}{}
 \showsymbolbin\diamond{22C4}{}, \cmd\smwhtdiamond
 \showsymbolbin\cdot{22C5}{*}
 \showsymbolbin\star{22C6}{}
 \showsymbolbin\divideontimes{22C7}{}
 \showsymbolbin\ltimes{22C9}{}
 \showsymbolbin\rtimes{22CA}{}
 \showsymbolbin\leftthreetimes{22CB}{}
 \showsymbolbin\rightthreetimes{22CC}{}
 \showsymbolbin\curlyvee{22CE}{}
 \showsymbolbin\curlywedge{22CF}{}
 \showsymbolbin\Cap{22D2}{}, \cmd\doublecap
 \showsymbolbin\Cup{22D3}{}, \cmd\doublecup
 \showsymbolbin\varbarwedge{2305}{*}
 \showsymbolbin\vardoublebarwedge{2306}{*}
 \showsymbolbin\obar{233D}{}
 \showsymbolbin\triangle{25B3}{}, \cmd\bigtriangleup
 \showsymbolbin\lhd{22B2}{}
 \showsymbolbin\rhd{22B3}{}
 \showsymbolbin\unlhd{22B4}{}
 \showsymbolbin\unrhd{22B5}{}
 \showsymbolbin\mdlgwhtcircle{25CB}{*}
 \showsymbolbin\boxbar{25EB}{*}
 \showsymbolbin\veedot{27C7}{*}
 \showsymbolbin\wedgedot{27D1}{*}
 \showsymbolbin\lozengeminus{27E0}{*}
 \showsymbolbin\concavediamond{27E1}{*}
 \showsymbolbin\concavediamondtickleft{27E2}{*}
 \showsymbolbin\concavediamondtickright{27E3}{*}
 \showsymbolbin\whitesquaretickleft{27E4}{*}
 \showsymbolbin\whitesquaretickright{27E5}{*}
 \showsymbolbin\typecolon{2982}{*}
 \showsymbolbin\circlehbar{29B5}{*}
 \showsymbolbin\circledvert{29B6}{}
 \showsymbolbin\circledparallel{29B7}{}
 \showsymbolbin\obslash{29B8}{}
 \showsymbolbin\operp{29B9}{*}
 \showsymbolbin\olessthan{29C0}{}
 \showsymbolbin\ogreaterthan{29C1}{}
 \showsymbolbin\boxdiag{29C4}{}
 \showsymbolbin\boxbslash{29C5}{}
 \showsymbolbin\boxast{29C6}{}
 \showsymbolbin\boxcircle{29C7}{}
 \showsymbolbin\boxbox{29C8}{*}
 \showsymbolbin\triangleserifs{29CD}{*}
 \showsymbolbin\hourglass{29D6}{*}
 \showsymbolbin\blackhourglass{29D7}{*}
 \showsymbolbin\shuffle{29E2}{*}
 \showsymbolbin\mdlgblklozenge{29EB}{*}
 \showsymbolbin\setminus{29F5}{*}
 \showsymbolbin\dsol{29F6}{*}
 \showsymbolbin\rsolbar{29F7}{*}
 \showsymbolbin\doubleplus{29FA}{*}
 \showsymbolbin\tripleplus{29FB}{*}
 \showsymbolbin\tplus{29FE}{*}
 \showsymbolbin\tminus{29FF}{*}
 \showsymbolbin\ringplus{2A22}{}
 \showsymbolbin\plushat{2A23}{}
 \showsymbolbin\simplus{2A24}{}
 \showsymbolbin\plusdot{2A25}{}
 \showsymbolbin\plussim{2A26}{}
 \showsymbolbin\plussubtwo{2A27}{}
 \showsymbolbin\plustrif{2A28}{*}
 \showsymbolbin\commaminus{2A29}{*}
 \showsymbolbin\minusdot{2A2A}{}
 \showsymbolbin\minusfdots{2A2B}{}
 \showsymbolbin\minusrdots{2A2C}{*}
 \showsymbolbin\opluslhrim{2A2D}{*}
 \showsymbolbin\oplusrhrim{2A2E}{*}
 \showsymbolbin\vectimes{2A2F}{*}
 \showsymbolbin\dottimes{2A30}{}
 \showsymbolbin\timesbar{2A31}{}
 \showsymbolbin\btimes{2A32}{}
 \showsymbolbin\smashtimes{2A33}{*}
 \showsymbolbin\otimeslhrim{2A34}{*}
 \showsymbolbin\otimesrhrim{2A35}{*}
 \showsymbolbin\otimeshat{2A36}{*}
 \showsymbolbin\Otimes{2A37}{*}
 \showsymbolbin\odiv{2A38}{*}
 \showsymbolbin\triangleplus{2A39}{*}
 \showsymbolbin\triangleminus{2A3A}{*}
 \showsymbolbin\triangletimes{2A3B}{*}
 \showsymbolbin\intprod{2A3C}{*}
 \showsymbolbin\intprodr{2A3D}{*}
 \showsymbolbin\fcmp{2A3E}{*}
 \showsymbolbin\amalg{2A3F}{}
 \showsymbolbin\capdot{2A40}{*}
 \showsymbolbin\uminus{2A41}{*}
 \showsymbolbin\barcup{2A42}{*}
 \showsymbolbin\barcap{2A43}{*}
 \showsymbolbin\capwedge{2A44}{*}
 \showsymbolbin\cupvee{2A45}{*}
 \showsymbolbin\cupovercap{2A46}{*}
 \showsymbolbin\capovercup{2A47}{*}
 \showsymbolbin\cupbarcap{2A48}{*}
 \showsymbolbin\capbarcup{2A49}{*}
 \showsymbolbin\twocups{2A4A}{*}
 \showsymbolbin\twocaps{2A4B}{*}
 \showsymbolbin\closedvarcup{2A4C}{*}
 \showsymbolbin\closedvarcap{2A4D}{*}
 \showsymbolbin\Sqcap{2A4E}{*}
 \showsymbolbin\Sqcup{2A4F}{*}
 \showsymbolbin\closedvarcupsmashprod{2A50}{*}
 \showsymbolbin\wedgeodot{2A51}{*}
 \showsymbolbin\veeodot{2A52}{*}
 \showsymbolbin\Wedge{2A53}{*}
 \showsymbolbin\Vee{2A54}{*}
 \showsymbolbin\wedgeonwedge{2A55}{*}
 \showsymbolbin\veeonvee{2A56}{*}
 \showsymbolbin\bigslopedvee{2A57}{*}
 \showsymbolbin\bigslopedwedge{2A58}{*}
 \showsymbolbin\wedgemidvert{2A5A}{*}
 \showsymbolbin\veemidvert{2A5B}{*}
 \showsymbolbin\midbarwedge{2A5C}{*}
 \showsymbolbin\midbarvee{2A5D}{*}
 \showsymbolbin\doublebarwedge{2A5E}{}
 \showsymbolbin\wedgebar{2A5F}{*}
 \showsymbolbin\wedgedoublebar{2A60}{*}
 \showsymbolbin\varveebar{2A61}{*}
 \showsymbolbin\doublebarvee{2A62}{*}
 \showsymbolbin\veedoublebar{2A63}{}
 \showsymbolbin\dsub{2A64}{*}
 \showsymbolbin\rsub{2A65}{*}
 \showsymbolbin\eqqplus{2A71}{}
 \showsymbolbin\pluseqq{2A72}{}
 \showsymbolbin\interleave{2AF4}{}
 \showsymbolbin\nhVvert{2AF5}{}
 \showsymbolbin\threedotcolon{2AF6}{}
 \showsymbolbin\trslash{2AFB}{}
 \showsymbolbin\sslash{2AFD}{}
 \showsymbolbin\talloblong{2AFE}{}
 \end{multicols}





\begin{symtable}{Variable-sized Math Operators}
\idxboth{variable-sized}{symbols}
\idxboth{linear logic}{symbols}
\index{integrals}
\label{op}
\renewcommand{\arraystretch}{1.75}  
\begin{tabular}{*3{l@{$\:$}ll@{\qquad}}l@{$\:$}ll}
\R\bigcap    & \R\bigotimes & \R\bigwedge  & \R\prod      \\
\R\bigcup    & \R\bigsqcup  & \R\coprod    & \R\sum       \\
\R\bigodot   & \R\biguplus  & \R\int       \\
\R\bigoplus  & \R\bigvee    & \R\oint      \\
\end{tabular}
\end{symtable}




\begin{symtable}[AMS]{\AmS Variable-sized Math Operators}
\idxboth{variable-sized}{symbols}
\index{integrals}
\label{ams-large}
\renewcommand{\arraystretch}{2.5}  
\begin{tabular}{l@{$\:$}ll@{\qquad}l@{$\:$}ll}
% removed optional \R[\AMSiint]
\R\iint     & \R\iiint       \\
\R\iiiint & \R\idotsint \\
\end{tabular}
\end{symtable}


\begin{symtable}[ST]{\ST\ Variable-sized Math Operators}
\idxboth{variable-sized}{symbols}
\label{st-large}
\renewcommand{\arraystretch}{1.75} 
\begin{tabular}{*2{l@{$\:$}ll@{\qquad}}l@{$\:$}ll}
\R\bigbox        & \R\biginterleave & \R\bigsqcap                            \\
\R\bigcurlyvee   & \R\bignplus      & \R[\STbigtriangledown]\bigtriangledown \\
\R\bigcurlywedge & \R\bigparallel   & \R[\STbigtriangleup]\bigtriangleup     \\
\end{tabular}
\end{symtable}


\begin{symtable}[WASY]{\WASY\ Variable-sized Math Operators}
\idxboth{variable-sized}{symbols}
\index{integrals}
\label{wasy-large}
\renewcommand{\arraystretch}{2.5}  
\begin{tabular}{*2{l@{$\:$}ll@{\qquad}}l@{$\:$}ll}
\R[\varint]\int$^\dag$ & \R\iint        & \R\iiint \\
\R\varint$^*$          & \R\varoint$^*$ & \R\oiint \\
\end{tabular}

\bigskip
\begin{tablenote}
  None of the preceding symbols are defined when \WASY\ is passed the
  \optname{wasysym}{nointegrals} option.
\end{tablenote}

\medskip
\begin{tablenote}[*]
  Not defined when \WASY\ is passed the \optname{wasysym}{integrals} option.
\end{tablenote}

\medskip
\begin{tablenote}[\dag]
  Defined only when \WASY\ is passed the \optname{wasysym}{integrals}
  option.  Otherwise, the default \latex \docAuxCommand{int} glyph (as shown
  in \ref{op}) is used.
\end{tablenote}
\end{symtable}

\begin{symtable}{Negated Binary Relations}
\index{binary relations>negated}
\index{relational symbols>negated binary}
\label{ams-nrel}
\begin{tabular}{*3{ll}}
\X\ncong     & \X\nshortparallel & \X\nVDash      \\
\X\nmid      & \X\nsim           & \X\precnapprox \\
\X\nparallel & \X\nsucc          & \X\precnsim    \\
\X\nprec     & \X\nsucceq        & \X\succnapprox \\
\X\npreceq   & \X\nvDash         & \X\succnsim    \\
\X\nshortmid & \X\nvdash                          \\
\end{tabular}
\end{symtable}


%\begin{symtable}[ST]{\ST\ Binary Relations}
%\index{binary relations}
%\index{relational symbols>binary}
%\label{st-rel}
%\begin{tabular}{*2{ll}}
%\X\inplus & \X\niplus \\
%\end{tabular}
%\end{symtable}


\begin{symtable}[WASY]{\WASY\ Binary Relations}
\index{binary relations}
\index{relational symbols>binary}
\label{wasy-rel}
\begin{tabular}{*3{ll}}
\X\invneg & \X\leadsto & \X\wasypropto \\
\X\Join   & \X\logof                   \\
\end{tabular}
\end{symtable}


\begin{symtable}[CMLL]{\CMLL\ Binary Relations}
\index{binary relations}
\index{relational symbols>binary}
\idxboth{linear logic}{symbols}
\label{cmll-rel}
\begin{tabular}{ll@{\hspace*{2em}}ll}
\K[\CMLLcoh]\coh     & \K[\CMLLscoh]\scoh     \\
\K[\CMLLincoh]\incoh & \K[\CMLLsincoh]\sincoh \\
\end{tabular}
\end{symtable}



\begin{symtable}{Subset and Superset Relations}
\index{binary relations}
\index{relational symbols>binary}
\index{subsets}
\index{supersets}
\index{symbols>subset and superset}
\label{subsets}
\begin{tabular}{*3{ll}}
\X\sqsubset$^*$ & \X\sqsupseteq & \X\supset   \\
\X\sqsubseteq   & \X\subset     & \X\supseteq \\
\X\sqsupset$^*$ & \X\subseteq                 \\
\end{tabular}

\bigskip
\notpredefinedmessageABX
\end{symtable}

\section{Inequalities}
\begin{symtable}{Inequalities}
\index{binary relations}\index{relational symbols>binary}
\index{inequalities}
\label{inequal-rel}
\begin{tabular}{*5{ll}}
\X\geq & \X\gg & \X\leq & \X\ll & \X\neq \\
\end{tabular}
\end{symtable}
\begin{symtable}{ Subset and Superset Relations}
\index{binary relations}
\index{relational symbols>binary}
\index{subsets}
\index{supersets}
\index{symbols>subset and superset}
\label{ams-subsets}
\begin{tabular}{*3{ll}}
\X\nsubseteq  & \X\subseteqq  & \X\supsetneqq    \\
\X\nsupseteq  & \X\subsetneq  & \X\varsubsetneq  \\
\X\nsupseteqq & \X\subsetneqq & \X\varsubsetneqq \\
\X\sqsubset   & \X\Supset     & \X\varsupsetneq  \\
\X\sqsupset   & \X\supseteqq  & \X\varsupsetneqq \\
\X\Subset     & \X\supsetneq                     \\
\end{tabular}
\end{symtable}


%\begin{symtable}[ST]{\ST\ Subset and Superset Relations}
%\index{binary relations}
%\index{relational symbols>binary}
%\index{subsets}
%\index{supersets}
%\index{symbols>subset and superset}
%\label{st-subsets}
%\begin{tabular}{*2{ll}}
%\X\subsetplus   & \X\supsetplus   \\
%\X\subsetpluseq & \X\supsetpluseq \\
%\end{tabular}
%\end{symtable}


\begin{symtable}[WASY]{\WASY\ Subset and Superset Relations}
\index{binary relations}
\index{relational symbols>binary}
\index{subsets}
\index{supersets}
\index{symbols>subset and superset}
\label{wasy-subset}
\begin{tabular}{*2{ll}}
\X\sqsubset & \X\sqsupset \\
\end{tabular}
\end{symtable}


\begin{symtable}{AMS Triangle Relations}
\index{triangle relations}\index{relational symbols>triangle}
\label{ams-triangle-rel}
\begin{tabular}{*3{ll}}
\X\blacktriangleleft  & \X\ntriangleright    & \X\trianglerighteq  \\
\X\blacktriangleright & \X\ntrianglerighteq  & \X\vartriangleleft  \\
\X\ntriangleleft      & \X\trianglelefteq    & \X\vartriangleright \\
\X\ntrianglelefteq    & \X\triangleq         &                     \\
\end{tabular}
\end{symtable}


%\begin{symtable}[ST]{\ST\ Triangle Relations}
%\index{triangle relations}\index{relational symbols>triangle}
%\label{st-triangle-rel}
%\begin{tabular}{*2{ll}}
%\X\trianglelefteqslant  & \X\trianglerighteqslant  \\
%\X\ntrianglelefteqslant & \X\ntrianglerighteqslant \\
%\end{tabular}
%\end{symtable}




\begin{symtable}{Arrows}
\index{arrows}
\label{arrow}
\begin{tabular}{*3{ll}}
\X\Downarrow          & \X\longleftarrow      & \X\nwarrow     \\
\X\downarrow          & \X\Longleftarrow      & \X\Rightarrow  \\
\X\hookleftarrow      & \X\longleftrightarrow & \X\rightarrow  \\
\X\hookrightarrow     & \X\Longleftrightarrow & \X\searrow     \\
\X\leadsto$^*$        & \X\longmapsto         & \X\swarrow     \\
\X\leftarrow          & \X\Longrightarrow     & \X\uparrow     \\
\X\Leftarrow          & \X\longrightarrow     & \X\Uparrow     \\
\X\Leftrightarrow     & \X\mapsto             & \X\updownarrow \\
\X\leftrightarrow     & \X\nearrow$^\dag$     & \X\Updownarrow \\
\end{tabular}

\bigskip
\notpredefinedmessage

\bigskip
\begin{tablenote}[\dag]
  See the note beneath \ref{extensible-accents} for information
  about how to put a diagonal arrow across a mathematical expression%
%\ifhavecancel
%  ~(as in ``$\cancelto{0}{\nabla \cdot \vec{B}}\quad$'')
%\fi
.
\end{tablenote}
\end{symtable}


\begin{symtable}{Harpoons}
\index{harpoons}
\label{harpoons}
\begin{tabular}{*3{ll}}
\X\leftharpoondown   & \X\rightharpoondown  & \X\rightleftharpoons \\
\X\leftharpoonup     & \X\rightharpoonup                           \\
\end{tabular}
\end{symtable}


\begin{symtable}{\TC\ Text-mode Arrows}
\index{arrows}
\label{tc-arrows}
\begin{tabular}{*2{ll}}
\K\textdownarrow & \K\textrightarrow \\
\K\textleftarrow & \K\textuparrow    \\
\end{tabular}
\end{symtable}


\begin{symtable}{AmS Arrows}
\index{arrows}
\label{ams-arrows}
\begin{tabular}{*3{ll}}
\X\circlearrowleft    & \X\leftleftarrows          & \X\rightleftarrows   \\
\X\circlearrowright  & \X\leftrightarrows       & \X\rightrightarrows  \\
\X\curvearrowleft   & \X\leftrightsquigarrow & \X\rightsquigarrow   \\
\X\curvearrowright & \X\Lleftarrow              & \X\Rsh               \\
\X\dashleftarrow     & \X\looparrowleft        & \X\twoheadleftarrow  \\
\X\dashrightarrow  & \X\looparrowright      & \X\twoheadrightarrow \\
\X\downdownarrows   & \X\Lsh                   & \X\upuparrows        \\
\X\leftarrowtail       & \X\rightarrowtail        &                      \\
\end{tabular}
\end{symtable}


\begin{symtable}{\AmS Negated Arrows}
\index{arrows>negated}
\label{ams-narrows}
\begin{tabular}{*3{ll}}
\X\nLeftarrow       & \X\nLeftrightarrow  & \X\nRightarrow     \\
\X\nleftarrow       & \X\nleftrightarrow   & \X\nrightarrow     \\
\end{tabular}
\end{symtable}


\begin{symtable}{\AmS Harpoons}
\index{harpoons}
\label{ams-harpoons}
\begin{tabular}{*3{ll}}
\X\downharpoonleft  & \X\leftrightharpoons   & \X\upharpoonleft  \\
\X\downharpoonright & \X\rightleftharpoons & \X\upharpoonright \\
\end{tabular}
\end{symtable}




\section{Log-like Symbols}
\begin{symtable}{Log-like Symbols}
\idxboth{log-like}{symbols}
\index{atomic math objects}
\index{limits}
\label{log}
\begin{tabular}{*8l}
\Z\arccos & \Z\cos  & \Z\csc & \Z\exp & \Z\ker    & \Z\limsup & \Z\min & \Z\sinh \\
\Z\arcsin & \Z\cosh & \Z\deg & \Z\gcd & \Z\lg     & \Z\ln     & \Z\Pr  & \Z\sup  \\
\Z\arctan & \Z\cot  & \Z\det & \Z\hom & \Z\lim    & \Z\log    & \Z\sec & \Z\tan  \\
\Z\arg    & \Z\coth & \Z\dim & \Z\inf & \Z\liminf & \Z\max    & \Z\sin & \Z\tanh
\end{tabular}

\bigskip
\begin{tablenote}
  Calling the above ``symbols'' may be a bit
  misleading.\footnotemark{} Each log-like symbol merely produces the
  eponymous textual equivalent, but with proper surrounding spacing.
  See \ref{math-spacing} for more information about log-like
  symbols.  As \cmd{\bmod} and \cmd{\pmod} are arguably not symbols we
  refer the reader to the Short Math Guide for
  \latex~\cite{Downes:smg} for samples.
\end{tablenote}
\end{symtable}
\footnotetext{Michael\index{Downes, Michael J.} J. Downes prefers the
more general term, ``atomic\index{atomic math objects} math objects''.}


\begin{symtable}{AMS Log-like Symbols}
\idxboth{log-like}{symbols}
\index{atomic math objects}
\index{limits}
\label{ams-log}
\renewcommand{\arraystretch}{1.5} 
\begin{tabular}{*2{ll@{\qquad}}ll}
\X\injlim     & \X\varinjlim  & \X\varlimsup  \\
\X\projlim    & \X\varliminf  & \X\varprojlim
\end{tabular}

\bigskip
\begin{tablenote}
  Load the \pkgname{amsmath} package to get these symbols.  See
  \ref{math-spacing} for some additional comments regarding
  log-like symbols.  As \cmd{\mod} and \cmd{\pod} are arguably not
  symbols we refer the reader to the Short Math Guide for
  \latex~\cite{Downes:smg} for samples.
\end{tablenote}
\end{symtable}



\section{Greek Letters}
   
  For usage see also Chapter \vref{ch:maths}. Greek letters are fundamental
  for most mathematical documents and the control sequences to use them are shown in 
  Table~\vref{greek}.
   
  The remaining Greek majuscules\index{majuscules} can be produced
  with ordinary Latin letters.  The symbol ``M'', for instance, is
  used for both an uppercase ``m'' and an uppercase ``$\mu$''.

  See \ref{bold-math} for examples of how to produce bold Greek
  letters.\index{Greek>bold}

  The symbols in this table are intended to be used in mathematical
  typesetting.  Greek body text can be typeset using the
  \pkgname{babel} package's \optname{babel}{greek} (or
  \optname{babel}{polutonikogreek}\idxboth{polytonic}{Greek})
  option---and, of course, a font that provides the glyphs for the
  Greek alphabet.

\begin{longsymtable}{Greek Letters}
\index{Greek}\index{alphabets>Greek}
\label{greek}
\begin{longtable}{*8l}
\X\alpha        &\X\theta       &\X o           &\X\tau         \\
\X\beta         &\X\vartheta    &\X\pi          &\X\upsilon     \\
\X\gamma        &\X\iota        &\X\varpi       &\X\phi         \\
\X\delta        &\X\kappa       &\X\rho         &\X\varphi      \\
\X\epsilon      &\X\lambda      &\X\varrho      &\X\chi         \\
\X\varepsilon   &\X\mu          &\X\sigma       &\X\psi         \\
\X\zeta         &\X\nu          &\X\varsigma    &\X\omega       \\
\X\eta          &\X\xi                                          \\
                                                                \\
\X\Gamma        &\X\Lambda      &\X\Sigma       &\X\Psi         \\
\X\Delta        &\X\Xi          &\X\Upsilon     &\X\Omega       \\
\X\Theta        &\X\Pi          &\X\Phi
\end{longtable}
\end{longsymtable}


\begin{symtable}[AMS]{\AmS\ Greek Letters}
\index{Greek}\index{alphabets>Greek}
\label{ams-greek}
\begin{tabular}{*4l}
\X\digamma      &\X\varkappa
\end{tabular}
\end{symtable}



\section{Hebrew letters}
\begin{symtable}{AMS Hebrew Letters}
\index{Hebrew}\index{alphabets>Hebrew}
\label{ams-hebrew}
\begin{tabular}{*6l}
\X\beth & \X\gimel & \X\daleth
\end{tabular}

\bigskip
\begin{tablenote}
\docAuxCommand{aleph}~($\aleph$) appears in \vref{ord}.
\end{tablenote}
\end{symtable}

\section{Letter-like Symbols}

\begin{symtable}{Letter-like Symbols}
\idxboth{letter-like}{symbols}
\index{tacks}
\idxboth{linear logic}{symbols}
\label{letter-like}
\begin{tabular}{*5{ll}}
\X\bot    & \X\forall & \X\imath & \X\ni      & \X\top \\
\X\ell    & \X\hbar   & \X\in    & \X\partial & \X\wp  \\
\X\exists & \X\Im     & \X\jmath & \X\Re               \\
\end{tabular}
\end{symtable}


\begin{symtable}{\AmS Letter-like Symbols}
\idxboth{letter-like}{symbols}
\label{ams-letter-like}
\begin{tabular}{*3{ll}}
\X\Bbbk       & \X\complement & \X\hbar    \\
\X\circledR   & \X\Finv       & \X\hslash  \\
\X\circledS   & \X\Game       & \X\nexists \\
\end{tabular}
\end{symtable}


\section{Variable-sized delimiters}

\begin{symtable}{Variable-sized Delimiters}
\index{delimiters}
\index{delimiters>variable-sized}
\label{dels}
\renewcommand{\arraystretch}{1.75} 
\begin{tabular}{lll@{\qquad}lll@{\hspace*{1.5cm}}lll@{\qquad}lll}
  \N\downarrow & \N\Downarrow &               & \N[\magicrbrack]{. } \\
  \N\langle         & \N\rangle         & \Np[\vert][\magicvertname]|
                                                                          & \Np[\Vert][\magicVertname]\| \\
  \N\lceil            & \N\rceil             & \N\uparrow      & \N\Uparrow          \\
  \N\lfloor          & \N\rfloor           & \N\updownarrow  & \N\Updownarrow      \\
  \N(                  & \N)                   & \Np\{           & \Np\}               \\
  \N/                  & \N\backslash                                         \\
\end{tabular}

\bigskip
\begin{tablenote}
  When used with \cmd{\left} and \cmd{\right}, these symbols expand to
  the height of the enclosed math expression.  Note that \docAuxCommand{vert}
  is a synonym for \verb+|+\index{_=\magicvertname{} ($\vert$)}, and
  \docAuxCommand{Vert} is a synonym for \verb+\|+\index{_=\magicVertname{}
  ($\Vert$)}.

  $\varepsilon$-\TeX{}\index{e-tex=$\varepsilon$-\TeX} provides a
  \cmd{\middle} analogue to \cmd{\left} and \cmd{\right}.
  \cmd{\middle} can be used, for example, to make an internal ``$\vert$''
  expand to the height of the surrounding \cmd{\left} and \cmd{\right}
  symbols.  (This capability is commonly needed when typesetting
  adjacent bras\index{bra} and kets\index{ket} in Dirac\index{Dirac
  notation} notation: ``$\langle\phi\vert\psi\rangle$'').  A similar
  effect can be achieved in conventional \latex using the
  \pkgname{braket} package.
\end{tablenote}
\end{symtable}



\begin{symtable}[ST]{\ST\ Variable-sized Delimiters}
\index{delimiters}
\index{delimiters>variable-sized}
\index{semantic valuation}
\label{st-var-del}
\begin{tabular}{lll@{\qquad}lll}
\N\llbracket & \N\rrbracket
\end{tabular}
\end{symtable}

\begin{symtable}{\TC\ Text-mode Delimiters}
\index{delimiters}
\index{delimiters>text-mode}
\label{tc-delimiters}
\begin{tabular}{*2{ll}}
\K\textlangle    & \K\textrangle    \\
\K\textlbrackdbl & \K\textrbrackdbl \\
\K\textlquill    & \K\textrquill    \\
\end{tabular}
\end{symtable}




%%problematic skip for the moment
\section{Math-mode Accents}
%
%
%\begin{symtable}{Math-mode Accents}
%\index{accents}
%\index{accents>acute=acute (\blackacchack\')}   
%\index{accents>breve=breve (\blackacchack\u)}   
%\index{accents>caron=caron (\blackacchack\v)}   
%\index{accents>circumflex=circumflex (\blackacchack\^)}   
%\index{accents>diaeresis=di\ae{}resis (\blackacchack\")} 
%\index{accents>dot=dot (\blackacchack\. or \blackacc\d)}  
%\index{accents>grave=grave (\blackacchack\`)}   
%  
%\index{accents>ring=ring (\blackacchack\r)}     
%\index{tilde}
%\label{math-accents}
%\begin{tabular}{*4{ll}}
%\W\acute{a}    & \W\check{a}    & \W\grave{a}    & \W\tilde{a} \\
%\W\bar{a}      & \W\ddot{a}     & \W\hat{a}      & \W\vec{a}   \\
%\W\breve{a}    & \W\dot{a}      & \W\mathring{a}               \\
%\end{tabular}
%
%
%\bigskip
%
%\begin{tablenote}
%  \index{dotless i=dotless $i~(\imath)$>math mode}
%  \index{dotless j=dotless $j~(\jmath)$>math mode}
%  Also note the existence of \docAuxCommand{imath} and \docAuxCommand{jmath}, which
%  produce dotless versions of ``\textit{i}'' and ``\textit{j}''.  (See
%  \vref{ord}.)  These are useful when the accent is supposed to
%  replace the dot.  For example, ``\verb|\hat{\imath}|'' produces a
%  correct ``$\,\hat{\imath}\,$'', while ``\verb|\hat{i}|'' would yield
%  the rather odd-looking ``\,$\hat{i}\,$''.
%\end{tablenote}
%\end{symtable}
%
%
\begin{symtable}{AMS Math-mode Accents}
\index{accents}
\label{ams-math-accents}
\begin{tabular}{ll@{\hspace*{2em}}ll}
\W\dddot{a}    & \W\ddddot{a} \\
\end{tabular}

\bigskip

\begin{tablenote}
  These accents are also provided by the ABX and \pkgname{accents}
  packages and are redefined by the MDOTS package if the
  \pkgname{amsmath} and \pkgname{amssymb} packages have previously
  been loaded.  All of the variations except for the original AMS
  ones tighten the space between the dots%


\end{tablenote}
\end{symtable}

%
%
\subsection{Extensible Accents}
%
\begin{longsymtable}{Extensible Accents}
\index{accents}
\idxboth{extensible}{accents}
\idxboth{extensible}{arrows}
\index{underline}
\index{tilde}
\index{tilde>extensible}
\index{extensible tildes}
\index{symbols>extensible}
\index{accents>circumflex=circumflex (\blackacchack\^)}  
\label{extensible-accents}
\renewcommand{\arraystretch}{1.5}
\begin{longtable}{*4l}
\W\widetilde{abc}$^*$         & \W\widehat{abc}$^*$    \\
\W\overleftarrow{abc}$^\dag$  & \W\overrightarrow{abc}$^\dag$ \\
\W\overline{abc}              & \W\underline{abc}      \\
\W\overbrace{abc}             & \W\underbrace{abc}     \\[5pt]
\W\sqrt{abc}$^\ddag$                                   \\
\end{longtable}

\bigskip

\begin{tablenote}
  \def\longdivsign{%
    \ensuremath{\overline{\vphantom{)}%
      \hbox{\smash{\raise3.5\fontdimen8\textfont3\hbox{$)$}}}%
      abc}}}

  \index{long division|(}
  \index{division|(}
  \index{polynomial division|(}

  As demonstrated in a 1997 TUGboat\index{TUGboat} article about
  typesetting long-division problems~\cite{Gibbons:longdiv}, an
  extensible long-division sign (``\,\longdivsign\,'') can be faked by
  putting a ``\verb|\big)|'' in a \texttt{tabular} environment with an
  \verb|\hline| or \verb|\cline| in the preceding row.  The article
  also presents a piece of code (uploaded to CTAN as
  \texttt{longdiv.tex}%
  \index{longdiv=\textsf{longdiv} (package)}%
  \index{packages>\textsf{longdiv}}) that automatically solves and
  typesets---by putting an \docAuxCommand{overline} atop ``\verb|\big)|'' and
  the desired text---long-division problems.  Of course now we have
  a not so good unicode character for it \texttt{U+27cc} {{\pan3\char"27CC 123456}},
  which you can use with a font that supports it. 
  See also the
  \pkgname{polynom} package, which automatically solves and typesets
  polynomial-division problems in a similar manner.

  \index{long division|)}
  \index{division|)}
  \index{polynomial division|)}
\end{tablenote}

\bigskip

\begin{tablenote}[*]
  These symbols are made more extensible by the MNS package and even
  more extensible by the \pkgname{yhmath} package.
\end{tablenote}

\bigskip

\begin{tablenote}[\dag]
  If you're looking for an extensible \emph{diagonal} line or arrow to
  be used for canceling or reducing mathematical
  subexpressions\index{arrows>diagonal, for reducing subexpressions}
\ifhavecancel
  %(e.g.,~``$\cancel{x + -x}$'' or ``$\cancelto{5}{3+2}\quad$'')
\fi
  then consider using the \pkgname{cancel} package.
\end{tablenote}

\bigskip

\begin{tablenote}[\ddag]
  With an optional argument, \verb|\sqrt| typesets nth roots.  For
  example, ``\verb|\sqrt[3]{abc}|'' produces~``$\!\sqrt[3]{abc}$\,''
  and ``\verb|\sqrt[n]{abc}|'' produces~``$\!\sqrt[n]{abc}$\,''.
\end{tablenote}
\end{longsymtable}


The \pkgname{ymath} package provides some very wide and extensible accents, as well as the |\widetriangle{XYZ}| triangular hat. The latter is used in France to show that the notation $ABC$ where $A,B,C$ are three points means a triangle $\widetriangle{ABC}$ and not an 
angle $\wideparen{ABC}$ \citep{ymath}. 
\index{triangular hat accent}\index{wide triangle accent} 


\medskip
\bgroup
%\begin{longsymtable}[YH]{yhmath Extensible Accents}
\idxboth{extensible}{accents}
\index{symbols>extensible}
\index{accents>arc=arc (\blackacchack\newtie)} 
\label{yhmath-extensible-accents}
\renewcommand{\arraystretch}{1.5}
\begin{longtable}{*4l}
\W\wideparen{ABC}    & \W\widetriangle{ABC} \\[5pt]
\W\widering{ABC}     & \W\wideparen {ABC}      \\
%\W\widebar{ABC}
\end{longtable}
\captionof{table}{yhmath Extensible Accents}
\egroup
\medskip

Yiannis Haralambous stated that he called the |widering| 
because it plays the r\^ole of a wide
 symbol (and since the ring can't be widened, a parenthesis is used).
 
Here are some more examples from the documentation (the first one coded as |\ring{A}|):
 
\begin{texexample}{The ymath package} {ex:ymath}
 $$
 \ring{A},
 \widering{AB},
 \widering{ABC},
 \widering{ABCD},
 \widering{ABCDE},
 \widering{ABCDEF},
 \widering{ABCDEFG},
 $$
\end{texexample} 
 
%In this paper we give a Clifford bundle motivated approach to the wave equation of a free spin $1/2$ fermion in the de Sitter manifold, a brane with topology $M=\mathrm{S0}(4,1)/\mathrm{S0}(3,1)$ living in the bulk spacetime $\mathbb{R}^{4,1}=(\mathring{M}=\mathbb{R}^{5},\bm{\mathring{g}})$ and equipped with a metric field $\bm{g:=-i}^{\ast}\bm{\mathring{g}}$ with $\bm{i}:M\rightarrow\mathring{M}$ being the inclusion map. To obtain the analog of Dirac equation in Minkowski spacetime in the structure $\mathring{M}$ we appropriately factorize the two Casimir invariants $C_{1}$ and $C_{2}$ of the Lie algebra of the de Sitter group \ldots.

 \begin{gather}
 \begin{pmatrix} a & b\\ c & d\end{pmatrix}
 \begin{pmatrix} a & b & c\\ d & e & f\\ g & h & i\end{pmatrix}
 \begin{pmatrix} a & b & c & d\\ e & f & g & h\\ i & j & k & l\\
 m & n & o & p\end{pmatrix}
 \\
 \begin{pmatrix} a & b & c & d & e\\ f & g & h & i & j\\
 k & l & m & n & o\\ p & q & r & s & t\\ u & v & w & x & y\end{pmatrix}
 \begin{pmatrix} a & b & c & d & e & f \\ g & h & i & j & k & l \\
 m & n & o & p & q & r \\ s & t & u & v & w & x \\ y & z & \alpha &
 \beta & \gamma & \delta\end{pmatrix}
 \end{gather}

%A Kakeya set is a subset of ${\mathbb R}^d$ that contains a unit line segment in every direction. Let $\mathring S^{d-1}$ denote the unit sphere in ${\mathbb R}^d$ with antipodal points identified. We encode a Kakeya set in ${\mathbb R}^d$ as a bounded map $f:\mathring S^{d-1}\to{\mathbb R}^d$, where $f(x)$ gives the centre of the unit line segment orientated in the $x$ direction. Denoting by $B(\mathring S^{d-1})$ the collection of all such maps equipped with the supremum norm, we show that (i) for a dense set of $f$ the corresponding Kakeya set has positive Lebesgue measure and (ii) the set of those $f$ for which the corresponding Kakeya set has maximal upper box-counting (Minkowski) dimension $d$ is a residual subset of $B(\widering S^{d-1})$. We also give a very simple proof that the lower box-counting dimension of any Kakeya set is at least $d/2$.


\begin{symtable}[MTOOLS]{\MTOOLS\ Extensible Accents}
\idxboth{extensible}{accents}
\index{symbols>extensible}
\label{mathtools-extensible-accents}
\renewcommand{\arraystretch}{1.5}
\begin{tabular}{ll@{\qquad}ll}
\W[\MTOOLSoverbrace]\overbrace{abc}         & \W[\MTOOLSunderbrace]\underbrace{abc}         \\
\W[\MTOOLSoverbracket]\overbracket{abc}$^*$ & \W[\MTOOLSunderbracket]\underbracket{abc}$^*$ \\
\end{tabular}

\bigskip

\begin{tablenote}[*]
  \verb|\overbracket| and \verb|\underbracket| accept optional
  arguments that specify the bracket height and thickness.
  \seedocs{\MTOOLS}.
\end{tablenote}
\end{symtable}




\subsection{Extensible Arrows}

\begin{symtable}{AMS Extensible Arrows}
\index{arrows}
\idxboth{extensible}{arrows}
\index{symbols>extensible}
\label{ams-extensible-arrows}
\begin{tabular}{ll@{\qquad}ll}
\W\xleftarrow{abc} & \W\xrightarrow{abc} \\
\end{tabular}
\end{symtable}



\section{Dots}

%\begin{symtable}{Dots}
%\idxboth{dot}{symbols}
%\index{dots (ellipses)} \index{ellipses (dots)}
%\label{dots}
%\begin{tabular}{*{3}{ll@{\hspace*{1.5cm}}}ll}
%\X\cdotp & \X\colon$^*$    & \X\ldotp & \X\vdots$^\dag$ \\
%\X\cdots & \X\ddots$^\dag$ & \X\ldots                   \\
%\end{tabular}
%
%\bigskip
%
%\begin{tablenote}[*]
%  While ``\texttt{:}'' is valid in math mode, \cmd{\colon} uses
%  different surrounding spacing.  See \ref{math-spacing} and the
%  Short Math Guide for \latex~\cite{Downes:smg} for more information on
%  math-mode spacing.
%\end{tablenote}
%
%\bigskip
%
%\begin{tablenote}[\dag]
% \ifMDOTS
%    \let\mdcmdX=\cmdX
%  \else
%    \let\mdcmdX=\cmd
%  \fi
% The \MDOTS\ package redefines \docAuxCommand{ddots} and \docAuxCommand{vdots} to
%  make them scale properly with font size.  (They normally scale
%  horizontally but not vertically.)  \mdcmdX{\fixedddots} and
%  \mdcmdX{\fixedvdots} provide the original, fixed-height
%  functionality of \latexe's \docAuxCommand{ddots} and \docAuxCommand{vdots} macros.
%\end{tablenote}
%\end{symtable}
%
%
%
%\begin{symtable}{\AmS Dots}
%\idxboth{dot}{symbols}
%\index{dots (ellipses)} \index{ellipses (dots)}
%\label{ams-dots}
%\begin{tabular}{*{2}{ll@{\hspace*{1.5cm}}}ll}
%\X\because$^*$   & \X[\cdots]\dotsi & \X\therefore$^*$ \\
%\X[\cdots]\dotsb & \X[\cdots]\dotsm &                  \\
%\X[\ldots]\dotsc & \X[\ldots]\dotso &                  \\
%\end{tabular}
%
%\bigskip
%
%\begin{tablenote}[*]
%  \docAuxCommand{because} and \docAuxCommand{therefore} are defined as binary
%  relations and therefore also appear in \vref{ams-rel}.
%\end{tablenote}
%
%\bigskip
%
%\begin{tablenote}
%  The \AmS \verb*|\dots| symbols are named
%  according to their intended usage: \cmdI[$\string\cdots$]{\dotsb}
%  between pairs of binary operators/relations,
%  \cmdI[$\string\ldots$]{\dotsc} between pairs of commas,
%  \cmdI[$\string\cdots$]{\dotsi} between pairs of integrals,
%  \cmdI[$\string\cdots$]{\dotsm} between pairs of multiplication
%  signs, and \cmdI[$\string\ldots$]{\dotso} between other symbol
%  pairs.
%\end{tablenote}
%\end{symtable}
%


%\begin{symtable}{WASY Dots}
%\idxboth{dot}{symbols}
%\label{wasy-dots}
%\begin{tabular}{ll}
%\K\wasytherefore
%\end{tabular}
%\end{symtable}



\begin{symtable}{Miscellaneous \latexe{} Math Symbols}
\idxboth{miscellaneous}{symbols}
\index{card suits}
\index{diamonds (suit)}
\index{hearts (suit)}
\index{clubs (suit)}
\index{spades (suit)}
\idxboth{musical}{symbols}
\index{dots (ellipses)}
\index{ellipses (dots)}
\index{null set}
\index{dotless i=dotless $i~(\imath)$>math mode}
\index{dotless j=dotless $j~(\jmath)$>math mode}
\index{angles}
\label{ord}
\AMSfalse
\ifAMS
  \def\AMSfn{$^\ddag$}
\else
  \def\AMSfn{}
\fi
\begin{tabular}{*4{ll}}
\X\aleph          & \X\Diamond$^*$    & \X\infty   & \X\prime     \\
\X\angle          & \X\diamondsuit    & \X\mho$^*$ & \X\sharp     \\
\X\backslash      & \X\emptyset\AMSfn & \X\nabla   & \X\spadesuit \\
\X\Box$^{*,\dag}$ & \X\flat           & \X\natural & \X\surd      \\
\X\clubsuit       & \X\heartsuit      & \X\neg     & \X\triangle  \\
\end{tabular}

\bigskip
\begin{tablenote}[*]
  Not predefined in \latexe.  Use one of the packages
  \pkgname{latexsym}, \pkgname{amsfonts}, \pkgname{amssymb},
  \pkgname{txfonts}, \pkgname{pxfonts}, or \pkgname{wasysym}.  Note,
  however, that \pkgname{amsfonts} and \pkgname{amssymb} define
  \docAuxCommand{Diamond} to produce the same glyph as
  the other packages produce a squarer \docAuxCommand{Diamond} as depicted above.
\end{tablenote}

\bigskip
\begin{tablenote}[\dag]
  To use \docAuxCommand{Box}---or any other symbol---as an end-of-proof
  (Q.E.D\@.)\index{Q.E.D.}\index{end of proof}\index{proof, end of}
  marker, consider using the \pkgname{ntheorem} package, which
  properly juxtaposes a symbol with the end of the proof text.
\end{tablenote}
\end{symtable}



\subsection{Miscellaneous Text-mode Math Symbols}

\subsection{Biological Symbols}
\begin{symtable}[MARV]{\MARV\ Biological Symbols}
\idxboth{biological}{symbols}
\index{male}
\index{female}
\label{marv-bio}
\begin{tabular}{*3{ll}ll}
\K\Female        & \K\FemaleMale    & \K\MALE          & \K\Neutral       \\
\K\FEMALE        & \K\Hermaphrodite & \K\Male          \\
\K\FemaleFemale  & \K\HERMAPHRODITE & \K\MaleMale      \\
\end{tabular}
\end{symtable}

\begin{symtable}[WASY]{\WASY\ Biological Symbols}
\index{male}
\index{female}
\label{wasy-bio}
\begin{tabular}{*2{ll}}
\K\female & \K\male \\
\end{tabular}
\end{symtable}

\begin{symtable}[MARV]{\MARV\ Safety-related Symbols}
\idxboth{safety-related}{symbols}
\label{marv-safety}
\begin{tabular}{*3{ll}ll}
\K\Biohazard     & \K\CEsign        & \K\Explosionsafe & \K\Radioactivity \\
\K\BSEfree       & \K\Estatically   & \indexlinearb\Laserbeam     & \K\Stopsign      \\
\end{tabular}
\end{symtable}

\idxbothend{scientific}{symbols}
\idxbothend{technological}{symbols}


\section{Dingbats}
\idxbothbegin{dingbat}{symbols}

Dingbats are symbols such as stars, arrows, and geometric shapes.
They are commonly used as bullets in itemized lists or, more
generally, as a means to draw attention to the text that follows.

The \PI\ dingbat package warrants special mention.  Among other
capabilities, \PI\ provides a \latex\ interface to the \PSfont{Zapf
Dingbats} font (one of the standard~35 \postscript\index{PostScript
fonts} fonts).  However, rather than name each of the dingbats
individually, \PI\ merely provides a single \cmd{\ding} command, which
outputs the character that lies at a given position in the font.  The
consequence is that the \PI\ symbols can't be listed by name in this
document's index, so be mindful of that fact when searching for a
particular symbol.

\bigskip


\begin{symtable}[DING]{\DING\ Arrows}
\label{bbding-arrows}
\begin{tabular}{*3{ll}}
\K\ArrowBoldDownRight    & \K\ArrowBoldRightShort  & \K\ArrowBoldUpRight \\
\K\ArrowBoldRightCircled & \K\ArrowBoldRightStrobe \\
\end{tabular}
\end{symtable}


\begin{symtable}[PI]{\PI\ Arrows}
\index{arrows}
\idxboth{fletched}{arrows}
\label{pi-arrows}
\begin{tabular}{*5{ll}}
\indexDing{212} & \indexDing{221} & \indexDing{230} & \indexDing{239} & \indexDing{249} \\
\indexDing{213} & \indexDing{222} & \indexDing{231} & \indexDing{241} & \indexDing{250} \\
\indexDing{214} & \indexDing{223} & \indexDing{232} & \indexDing{242} & \indexDing{251} \\
\indexDing{215} & \indexDing{224} & \indexDing{233} & \indexDing{243} & \indexDing{252} \\
\indexDing{216} & \indexDing{225} & \indexDing{234} & \indexDing{244} & \indexDing{253} \\
\indexDing{217} & \indexDing{226} & \indexDing{235} & \indexDing{245} & \indexDing{254} \\
\indexDing{218} & \indexDing{227} & \indexDing{236} & \indexDing{246} \\
\indexDing{219} & \indexDing{228} & \indexDing{237} & \indexDing{247} \\
\indexDing{220} & \indexDing{229} & \indexDing{238} & \indexDing{248} \\
\end{tabular}
\end{symtable}



\begin{symtable}[MARV]{\MARV\ Scissors}
\index{scissors}
\label{marv-scissors}
\begin{tabular}{*3{ll}}
\K\Cutleft       & \K\Cutright      & \indexlinearb\Leftscissors  \\
\K\Cutline       & \K\Kutline       & \K\Rightscissors \\
\end{tabular}
\end{symtable}


\begin{symtable}[DING]{\DING\ Scissors}
\index{scissors}
\label{scissors}
\begin{tabular}{*2{ll}}
\K\ScissorHollowLeft        & \K\ScissorLeftBrokenTop     \\
\K\ScissorHollowRight       & \K\ScissorRight             \\
\K\ScissorLeft              & \K\ScissorRightBrokenBottom \\
\K\ScissorLeftBrokenBottom  & \K\ScissorRightBrokenTop    \\
\end{tabular}
\end{symtable}


\begin{symtable}[PI]{\PI\ Scissors}
\index{scissors}
\label{pi-scissors}
\begin{tabular}{*4{ll}}
\indexDing{33} & \indexDing{34} & \indexDing{35} & \indexDing{36} \\
\end{tabular}
\end{symtable}

\begin{symtable}[DING]{\DING\ Pencils and Nibs}
\index{pencils}
\index{nibs}
\label{pencils-nibs}
\begin{tabular}{*3{ll}}
\K\NibLeft         & \K\PencilLeft      & \K\PencilRightDown \\
\K\NibRight        & \K\PencilLeftDown  & \K\PencilRightUp   \\
\K\NibSolidLeft    & \K\PencilLeftUp    \\
\K\NibSolidRight   & \K\PencilRight     \\
\end{tabular}
\end{symtable}


\begin{symtable}[PI]{\PI\ Pencils and Nibs}
\index{pencils}
\index{nibs}
\label{pi-pencils}
\begin{tabular}{*5{ll}}
\indexDing{46} & \indexDing{47} & \indexDing{48} & \indexDing{49} & \indexDing{50} \\
\end{tabular}
\end{symtable}

\begin{symtable}[DING]{\DING\ Fists}
\index{fists}
\label{hands}
\begin{tabular}{*3{ll}}
\K\HandCuffLeft    & \K\HandCuffRightUp & \K\HandPencilLeft  \\
\K\HandCuffLeftUp  & \K\HandLeft        & \K\HandRight       \\
\K\HandCuffRight   & \K\HandLeftUp      & \K\HandRightUp     \\
\end{tabular}
\end{symtable}


\begin{symtable}[PI]{\PI\ Fists}
\index{fists}
\label{pi-hands}
\begin{tabular}{*4{ll}}
\indexDing{42} & \indexDing{43} & \indexDing{44} & \indexDing{45} \\
\end{tabular}
\end{symtable}

\begin{symtable}[DING]{\DING\ Crosses and Plusses}
\index{crosses}
\index{plusses}
\index{crucifixes}
\label{crosses-plusses}
\begin{tabular}{*3{ll}}
\K[\dingCross]\Cross  & \K\CrossOpenShadow    & \K\PlusOutline        \\
\K\CrossBoldOutline   & \K\CrossOutline       & \K\PlusThinCenterOpen \\
\K\CrossClowerTips    & \K\Plus               \\
\K\CrossMaltese       & \K\PlusCenterOpen     \\
\end{tabular}
\end{symtable}


\begin{symtable}[PI]{\PI\ Crosses and Plusses}
\index{symbols>crosses}
\index{symbols>plusses}
\index{symbols>crucifixes}
\label{pi-crosses-plusses}
\begin{tabular}{*4{ll}}
\indexDing{57} & \indexDing{59} & \indexDing{61} & \indexDing{63} \\
\indexDing{58} & \indexDing{60} & \indexDing{62} & \indexDing{64} \\
\end{tabular}
\end{symtable}


\begin{symtable}[DING]{\DING\ Xs and Check Marks}
\index{symbols>check marks}
\index{symbols>Xs}
\label{ding-check-marks}
\begin{tabular}{*3{ll}}
\K\Checkmark     & \K\XSolid        & \K\XSolidBrush   \\
\K\CheckmarkBold & \K\XSolidBold    \\
\end{tabular}
\end{symtable}


\begin{symtable}[PI]{\PI\ Xs and Check Marks}
\index{check marks}
\index{Xs}
\label{pi-check-marks}
\begin{tabular}{*3{ll}}
\indexDing{51} & \indexDing{53} & \indexDing{55} \\
\indexDing{52} & \indexDing{54} & \indexDing{56} \\
\end{tabular}
\end{symtable}


\begin{symtable}[WASY]{\WASY\ Xs and Check Marks}
\index{check marks}
\index{Xs}
\label{wasy-check-marks}
\begin{tabular}{*6l}
\K\CheckedBox & \K\Square & \K\XBox \\
\end{tabular}
\end{symtable}


\begin{symtable}[PI]{\PI\ Circled Numbers}
\index{circled numbers}
\index{numbers>circled}
\label{circled-numbers}
\begin{tabular}{*4{ll}}
\indexDing{172} & \indexDing{182} & \indexDing{192} & \indexDing{202} \\
\indexDing{173} & \indexDing{183} & \indexDing{193} & \indexDing{203} \\
\indexDing{174} & \indexDing{184} & \indexDing{194} & \indexDing{204} \\
\indexDing{175} & \indexDing{185} & \indexDing{195} & \indexDing{205} \\
\indexDing{176} & \indexDing{186} & \indexDing{196} & \indexDing{206} \\
\indexDing{177} & \indexDing{187} & \indexDing{197} & \indexDing{207} \\
\indexDing{178} & \indexDing{188} & \indexDing{198} & \indexDing{208} \\
\indexDing{179} & \indexDing{189} & \indexDing{199} & \indexDing{209} \\
\indexDing{180} & \indexDing{190} & \indexDing{200} & \indexDing{210} \\
\indexDing{181} & \indexDing{191} & \indexDing{201} & \indexDing{211} \\
\end{tabular}

\bigskip

\begin{tablenote}
  \PI\ (part of the \pkgname{psnfss} package) provides a
  \cmd{dingautolist} environment which resembles \texttt{enumerate}
  but uses circled numbers as bullets.\footnotemark{}
  \seedocs{\pkgname{psnfss}}.
\end{tablenote}
\end{symtable}
\footnotetext{In fact, \cmd{\dingautolist} can use any set of
  consecutive \PSfont{Zapf Dingbats} symbols.}


\begin{symtable}[WASY]{\WASY\ Stars}
\index{stars}
\index{Jewish star}\index{Star of David}
\label{wasy-stars}
\begin{tabular}{*6l}
\K\davidsstar & \K\hexstar & \K\varhexstar
\end{tabular}
\end{symtable}


\begin{symtable}[DING]{\DING\ Stars, Flowers, and Similar Shapes}
\index{asterisks}
\index{clovers}
\index{flowers}
\index{ornaments}
\index{sparkles}
\index{snowflakes}
\index{stars}
\index{Jewish star}\index{Star of David}
\label{star-like}
\begin{tabular}{*3{ll}}
\K\Asterisk                & \K\FiveFlowerPetal      & \K\JackStar                  \\
\K\AsteriskBold            & \K\FiveStar             & \K\JackStarBold              \\
\K\AsteriskCenterOpen      & \K\FiveStarCenterOpen   & \K\SixFlowerAlternate        \\
\K\AsteriskRoundedEnds     & \K\FiveStarConvex       & \K\SixFlowerAltPetal         \\
\K\AsteriskThin            & \K\FiveStarLines        & \K\SixFlowerOpenCenter       \\
\K\AsteriskThinCenterOpen  & \K\FiveStarOpen         & \K\SixFlowerPetalDotted      \\
\K\DavidStar               & \K\FiveStarOpenCircled  & \K\SixFlowerPetalRemoved     \\
\K\DavidStarSolid          & \K\FiveStarOpenDotted   & \K\SixFlowerRemovedOpenPetal \\
\K\EightAsterisk           & \K\FiveStarOutline      & \K\SixStar                   \\
\K\EightFlowerPetal        & \K\FiveStarOutlineHeavy & \K\SixteenStarLight          \\
\K\EightFlowerPetalRemoved & \K\FiveStarShadow       & \K\Snowflake                 \\
\K\EightStar               & \K\FourAsterisk         & \K\SnowflakeChevron          \\
\K\EightStarBold           & \K\FourClowerOpen       & \K\SnowflakeChevronBold      \\
\K\EightStarConvex         & \K\FourClowerSolid      & \K\Sparkle                   \\
\K\EightStarTaper          & \K\FourStar             & \K\SparkleBold               \\
\K\FiveFlowerOpen          & \K\FourStarOpen         & \K\TwelweStar                \\
\end{tabular}
\end{symtable}

\begin{symtable}[WASY]{\WASY\ Geometric Shapes}
\index{polygons}
\index{geometric shapes}
\label{wasy-geometrical}
\begin{tabular}{*8l}
\K\hexagon & \K\octagon & \K\pentagon & \K\varhexagon
\end{tabular}
\end{symtable}

\begin{symtable}[DING]{\DING\ Geometric Shapes}
\index{circles}
\index{diamonds}
\index{ellipses (ovals)}
\index{geometric shapes}
\index{ovals}
\index{rectangles}
\index{squares}
\index{triangles}
\label{ding-geometrical}
\begin{tabular}{*3{ll}}
\K\CircleShadow    & \K\Rectangle                   & \K\SquareShadowTopLeft     \\
\K\CircleSolid     & \K\RectangleBold               & \K\SquareShadowTopRight    \\
\K\DiamondSolid    & \K\RectangleThin               & \K\SquareSolid             \\
\K\Ellipse         & \K[\dingSquare]\Square         & \K\TriangleDown            \\
\K\EllipseShadow   & \K\SquareCastShadowBottomRight & \K\TriangleUp              \\
\K\EllipseSolid    & \K\SquareCastShadowTopLeft     \\
\K\HalfCircleLeft  & \K\SquareCastShadowTopRight    \\
\K\HalfCircleRight & \K\SquareShadowBottomRight     \\
\end{tabular}
\end{symtable}


\begin{symtable}[PI]{\PI\ Geometric Shapes}
\index{circles}
\index{diamonds}
\index{geometric shapes}
\index{rectangles}
\index{squares}
\index{triangles}
\label{pi-geometrical}
\begin{tabular}{*5{ll}}
\indexDing{108} & \indexDing{111} & \indexDing{114} & \indexDing{117} & \indexDing{121} \\
\indexDing{109} & \indexDing{112} & \indexDing{115} & \indexDing{119} & \indexDing{122} \\
\indexDing{110} & \indexDing{113} & \indexDing{116} & \indexDing{120} \\
\end{tabular}
\end{symtable}\begin{symtable}[DING]{Miscellaneous \DING\ Dingbats}
\idxboth{miscellaneous}{symbols}
\index{envelopes}
\label{bbding-misc}
\begin{tabular}{*4{ll}}
\K\Envelope             & \K\Peace & \K\PhoneHandset & \K\SunshineOpenCircled \\
\K\OrnamentDiamondSolid & \K\Phone & \K\Plane        & \K\Tape                \\
\end{tabular}
\end{symtable}


\begin{symtable}[PI]{Miscellaneous \PI\ Dingbats}
\idxboth{miscellaneous}{symbols}
\index{card suits}
\index{diamonds (suit)}
\index{hearts (suit)}
\index{clubs (suit)}
\index{spades (suit)}
\index{fleurons}
\index{leaves}
\index{ornaments}
\label{pi-misc}
\begin{tabular}{*5{ll}}
\indexDing{37} & \indexDing{40}  & \indexDing{164} & \indexDing{167} & \indexDing{171} \\
\indexDing{38} & \indexDing{41}  & \indexDing{165} & \indexDing{168} & \indexDing{169} \\
\indexDing{39} & \indexDing{118} & \indexDing{166} & \indexDing{170} \\
\end{tabular}
\end{symtable}
\idxbothend{dingbat}{symbols}

\begin{symtable}{\TC\ Genealogical Symbols}
\idxboth{genealogical}{symbols}
\label{genealogical}
\begin{tabular}{*3{ll}}
\K\textborn     & \K\textdivorced & \K\textmarried  \\
\K\textdied     & \K\textleaf     \\
\end{tabular}
\end{symtable}


\begin{symtable}[WASY]{\WASY\ General Symbols}
\index{symbols>general}
\index{smiley faces}
\index{frowny faces}
\index{faces}
\idxboth{clock}{symbols}
\index{check marks}
\label{wasy-general}
\begin{tabular}{*4{ll}}
\K\ataribox    & \K[\WASYclock]\clock & \indexlinearb\LEFTarrow  & \K\smiley      \\
\K\bell        & \K\diameter          & \K\lightning  & \K\sun         \\
\K\blacksmiley & \K\DOWNarrow         & \K\phone      & \K\UParrow     \\
\K\Bowtie      & \K\frownie           & \K\pointer    & \K\wasylozenge \\
\K\brokenvert  & \K\invdiameter       & \K\recorder                    \\
\K\checked     & \K\kreuz             & \K\RIGHTarrow                  \\
\end{tabular}
\end{symtable}


\begin{symtable}[WASY]{\WASY\ Circles}
\index{circles}
\label{wasy-circles}
\begin{tabular}{*8l}
\K\CIRCLE         & \indexlinearb\LEFTcircle     & \K\RIGHTcircle    & \K\rightturn      \\
\K\Circle         & \indexlinearb\Leftcircle     & \K\Rightcircle    \\
\indexlinearb\LEFTCIRCLE     & \K\RIGHTCIRCLE    & \K\leftturn       \\
\end{tabular}
\end{symtable}


\begin{symtable}[WASY]{\WASY\ Musical Symbols}
\idxboth{musical}{symbols}
\label{wasy-music}
\begin{tabular}{*{10}l}
\K\eighthnote & \K\halfnote    & \K\twonotes &
\K\fullnote   & \K\quarternote \\
\end{tabular}

\bigskip
\begin{tablenote}
  See also \docAuxCommand{flat}, \docAuxCommand{sharp}, and \docAuxCommand{natural}
  (\vref*{ord}).
\end{tablenote}
\end{symtable}

\begin{symtable}[MARV]{\MARV\ Navigation Symbols}
\idxboth{navigation}{symbols}
\label{marv-navigation}
\begin{tabular}{*3{ll}ll}
\K\Forward        & \K\MoveDown  & \K\RewindToIndex  & \K\ToTop \\
\K\ForwardToEnd   & \K\MoveUp    & \K\RewindToStart  \\
\K\ForwardToIndex & \K\Rewind    & \K\ToBottom       \\
\end{tabular}
\end{symtable}


\begin{symtable}[MARV]{\MARV\ Laundry Symbols}
\idxboth{laundry}{symbols}
\label{marv-laundry}
\begin{tabular}{*3{ll}}
\K\AtForty            & \K\Handwash           & \K\ShortNinetyFive    \\
\K\AtNinetyFive       & \K\IroningI           & \K\ShortSixty         \\
\K\AtSixty            & \K\IroningII          & \K\ShortThirty        \\
\K\Bleech             & \K\IroningIII         & \K\SpecialForty       \\
\K\CleaningA          & \K\NoBleech           & \K\Tumbler            \\
\K\CleaningF          & \K\NoChemicalCleaning & \K\WashCotton         \\
\K\CleaningFF         & \K\NoIroning          & \K\WashSynthetics     \\
\K\CleaningP          & \K\NoTumbler          & \K\WashWool           \\
\K\CleaningPP         & \K\ShortFifty         \\
\K\Dontwash           & \K\ShortForty         \\
\end{tabular}
\end{symtable}


\begin{symtable}[MARV]{\MARV\ Information Symbols}
\idxboth{information}{symbols}
\index{check marks}
\index{Xs}
\idxboth{clock}{symbols}
\label{marv-info}
\begin{tabular}{*3{ll}ll}
\K\Bicycle      & \K\Football     & \K\Pointinghand \\
\K\Checkedbox   & \K\Gentsroom    & \K\Wheelchair   \\
\K\Clocklogo    & \K\Industry     & \K\Writinghand  \\
\K\Coffeecup    & \K\Info         \\
\K\Crossedbox   & \indexlinearb\Ladiesroom   \\
\end{tabular}
\end{symtable}


\begin{symtable}[MARV]{Other \MARV\ Symbols}
\idxboth{miscellaneous}{symbols}
\index{crosses}
\index{crucifixes}
\index{smiley faces}
\index{frowny faces}
\index{faces}
\index{man}
\index{woman}
\index{globe}
\index{world}
\label{marv-other}
\begin{tabular}{*4{ll}}
\K\Ankh        & \K\Cross        & \K\Heart       & \K\Smiley      \\
\K\Bat         & \K\FHBOlogo     & \K\MartinVogel & \K\Womanface   \\
\K\Bouquet     & \K\FHBOLOGO     & \K\Mundus      & \K\Yinyang     \\
\K\Celtcross   & \K\Frowny       & \K\MVAt                         \\
\K\CircledA    & \K\FullFHBO     & \K\MVRightarrow                 \\
\end{tabular}
\end{symtable}

\section{Alphabets}

\begin{symtable}[CYPR]{\CYPR\ Cypriot Letters}
\index{Cypriot}
\index{alphabets>Cypriot}
\label{cypriot}
\begin{tabular}{*5{ll@{\qquad}}ll}
\Kcyp[{\Ca}]\Ca   & \Kcyp[{\Cku}]\Cku & \Kcyp[{\Cmu}]\Cmu & \Kcyp[{\Cpo}]\Cpo & \Kcyp[{\Cso}]\Cso & \Kcyp[{\Cwi}]\Cwi \\
\Kcyp[{\Ce}]\Ce   & \Kcyp[{\Cla}]\Cla & \Kcyp[{\Cna}]\Cna & \Kcyp[{\Cpu}]\Cpu & \Kcyp[{\Csu}]\Csu & \Kcyp[{\Cwo}]\Cwo \\
\Kcyp[{\Cga}]\Cga & \Kcyp[{\Cle}]\Cle & \Kcyp[{\Cne}]\Cne & \Kcyp[{\Cra}]\Cra & \Kcyp[{\Cta}]\Cta & \Kcyp[{\Cxa}]\Cxa \\
\Kcyp[{\Ci}]\Ci   & \Kcyp[{\Cli}]\Cli & \Kcyp[{\Cni}]\Cni & \Kcyp[{\Cre}]\Cre & \Kcyp[{\Cte}]\Cte & \Kcyp[{\Cxe}]\Cxe \\
\Kcyp[{\Cja}]\Cja & \Kcyp[{\Clo}]\Clo & \Kcyp[{\Cno}]\Cno & \Kcyp[{\Cri}]\Cri & \Kcyp[{\Cti}]\Cti & \Kcyp[{\Cya}]\Cya \\
\Kcyp[{\Cjo}]\Cjo & \Kcyp[{\Clu}]\Clu & \Kcyp[{\Cnu}]\Cnu & \Kcyp[{\Cro}]\Cro & \Kcyp[{\Cto}]\Cto & \Kcyp[{\Cyo}]\Cyo \\
\Kcyp[{\Cka}]\Cka & \Kcyp[{\Cma}]\Cma & \Kcyp[{\Co}]\Co   & \Kcyp[{\Cru}]\Cru & \Kcyp[{\Ctu}]\Ctu & \Kcyp[{\Cza}]\Cza \\
\Kcyp[{\Cke}]\Cke & \Kcyp[{\Cme}]\Cme & \Kcyp[{\Cpa}]\Cpa & \Kcyp[{\Csa}]\Csa & \Kcyp[{\Cu}]\Cu   & \Kcyp[{\Czo}]\Czo \\
\Kcyp[{\Cki}]\Cki & \Kcyp[{\Cmi}]\Cmi & \Kcyp[{\Cpe}]\Cpe & \Kcyp[{\Cse}]\Cse & \Kcyp[{\Cwa}]\Cwa &                         \\
\Kcyp[{\Cko}]\Cko & \Kcyp[{\Cmo}]\Cmo & \Kcyp[{\Cpi}]\Cpi & \Kcyp[{\Csi}]\Csi & \Kcyp[{\Cwe}]\Cwe &                         \\
\end{tabular}

\bigskip
\begin{tablenote}
  \usefontcmdmessage{}{\cyprfamily}.  Single-character
  shortcuts are also supported: Both
  ``\verb+{\Cpa\Cki\Cna}+'' and ``\verb+{pcn}+''
  produce ``{pcn}'', for example.  \seedocs{\CYPR}.
\end{tablenote}
\end{symtable}


\begin{symtable}[PRSN]{\PRSN\ Cuneiform Letters}
\index{cuneiform}
\index{alphabets>Old Persian (cuneiform)}
\label{oldprsn}
\begin{tabular}{*4{ll@{\qquad}}ll}
\indexoldpersian[\textcopsn{\Oa}]\Oa     & \indexoldpersian[\textcopsn{\Oga}]\Oga   & \indexoldpersian[\textcopsn{\Ola}]\Ola   & \indexoldpersian[\textcopsn{\Oru}]\Oru   & \indexoldpersian[\textcopsn{\Ovi}]\Ovi   \\
\indexoldpersian[\textcopsn{\Oba}]\Oba   & \indexoldpersian[\textcopsn{\Ogu}]\Ogu   & \indexoldpersian[\textcopsn{\Oma}]\Oma   & \indexoldpersian[\textcopsn{\Osa}]\Osa   & \indexoldpersian[\textcopsn{\Oxa}]\Oxa   \\
\indexoldpersian[\textcopsn{\Oca}]\Oca   & \indexoldpersian[\textcopsn{\Oha}]\Oha   & \indexoldpersian[\textcopsn{\Omi}]\Omi   & \indexoldpersian[\textcopsn{\Osva}]\Osva & \indexoldpersian[\textcopsn{\Oya}]\Oya   \\
\indexoldpersian[\textcopsn{\Occa}]\Occa & \indexoldpersian[\textcopsn{\Oi}]\Oi     & \indexoldpersian[\textcopsn{\Omu}]\Omu   & \indexoldpersian[\textcopsn{\Ota}]\Ota   & \indexoldpersian[\textcopsn{\Oza}]\Oza   \\
\indexoldpersian[\textcopsn{\Oda}]\Oda   & \indexoldpersian[\textcopsn{\Oja}]\Oja   & \indexoldpersian[\textcopsn{\Ona}]\Ona   & \indexoldpersian[\textcopsn{\Otha}]\Otha &                            \\
\indexoldpersian[\textcopsn{\Odi}]\Odi   & \indexoldpersian[\textcopsn{\Oji}]\Oji   & \indexoldpersian[\textcopsn{\Onu}]\Onu   & \indexoldpersian[\textcopsn{\Otu}]\Otu   &                            \\
\indexoldpersian[\textcopsn{\Odu}]\Odu   & \indexoldpersian[\textcopsn{\Oka}]\Oka   & \indexoldpersian[\textcopsn{\Opa}]\Opa   & \indexoldpersian[\textcopsn{\Ou}]\Ou     &                            \\
\indexoldpersian[\textcopsn{\Ofa}]\Ofa   & \indexoldpersian[\textcopsn{\Oku}]\Oku   & \indexoldpersian[\textcopsn{\Ora}]\Ora   & \indexoldpersian[\textcopsn{\Ova}]\Ova   &                            \\
\end{tabular}

\bigskip
\begin{tablenote}
  \usefontcmdmessage{\textcopsn}{\copsnfamily}.  Single-character
  shortcuts are also supported: Both
  ``\verb+\textcopsn{\Opa\Oka\Ona}+'' and ``\verb+\textcopsn{pkn}+''
  produce ``\textcopsn{pkn}'', for example.  \seedocs{\PRSN}.
\end{tablenote}
\end{symtable}


\begin{symtable}[PRSN]{\PRSN\ Cuneiform Numerals}
\index{cuneiform}
\index{numerals>cuneiform}
\label{oldprsn-nums}
\begin{tabular}{*4{ll@{\qquad}}ll}
\indexoldpersian[\textcopsn{\Oone}]\Oone & \indexoldpersian[\textcopsn{\Otwo}]\Otwo & \indexoldpersian[\textcopsn{\Oten}]\Oten & \indexoldpersian[\textcopsn{\Otwenty}]\Otwenty & \indexoldpersian[\textcopsn{\Ohundred}]\Ohundred \\
\end{tabular}

\bigskip
\begin{tablenote}
  \usefontcmdmessage{\textcopsn}{\copsnfamily}.
\end{tablenote}
\end{symtable}


\begin{symtable}[PRSN]{\PRSN\ Cuneiform Words}
\index{cuneiform}
\label{oldprsn-objs}
\begin{tabular}{*3{ll@{\qquad}}ll}
\indexoldpersian[\textcopsn{\OAura}]\OAura         & \indexoldpersian[\textcopsn{\Ocountrya}]\Ocountrya & \indexoldpersian[\textcopsn{\Ogod}]\Ogod           &                                      \\
\indexoldpersian[\textcopsn{\OAurb}]\OAurb         & \indexoldpersian[\textcopsn{\Ocountryb}]\Ocountryb & \indexoldpersian[\textcopsn{\Oking}]\Oking         &                                      \\
\indexoldpersian[\textcopsn{\OAurc}]\OAurc         & \indexoldpersian[\textcopsn{\Oearth}]\Oearth       & \indexoldpersian[\textcopsn{\Owd}]\Owd             &                                      \\
\end{tabular}

\bigskip
\begin{tablenote}
  \usefontcmdmessage{\textcopsn}{\copsnfamily}.
\end{tablenote}
\end{symtable}

\subsection{Ugaritic}

\begin{symtable}[UGAR]{\UGAR\ Cuneiform Letters}
\index{cuneiform}
\index{alphabets>Ugarite (cuneiform)}
\label{ugarite}
\begin{tabular}{*4{ll@{\qquad}}ll}
\indexugar[\textcugar{\Arq}]\Arq & \indexugar[\textcugar{\Az}]\Az   & \indexugar[\textcugar{\Am}]\Am   & \indexugar[\textcugar{\Asd}]\Asd & \indexugar[\textcugar{\Au}]\Au   \\
\indexugar[\textcugar{\Ab}]\Ab   & \indexugar[\textcugar{\Ahd}]\Ahd & \indexugar[\textcugar{\Adb}]\Adb & \indexugar[\textcugar{\Aq}]\Aq   & \indexugar[\textcugar{\Asg}]\Asg \\
\indexugar[\textcugar{\Ag}]\Ag   & \indexugar[\textcugar{\Atd}]\Atd & \indexugar[\textcugar{\An}]\An   & \indexugar[\textcugar{\Ar}]\Ar   & \indexugar[\textcugar{\Awd}]\Awd \\
\indexugar[\textcugar{\Ahu}]\Ahu & \indexugar[\textcugar{\Ay}]\Ay   & \indexugar[\textcugar{\Azd}]\Azd & \indexugar[\textcugar{\Atb}]\Atb &                          \\
\indexugar[\textcugar{\Ad}]\Ad   & \indexugar[\textcugar{\Ak}]\Ak   & \indexugar[\textcugar{\As}]\As   & \indexugar[\textcugar{\Agd}]\Agd &                          \\
\indexugar[\textcugar{\Ah}]\Ah   & \indexugar[\textcugar{\Asa}]\Asa & \indexugar[\textcugar{\Alq}]\Alq & \indexugar[\textcugar{\At}]\At   &                          \\
\indexugar[\textcugar{\Aw}]\Aw   & \indexugar[\textcugar{\Al}]\Al   & \indexugar[\textcugar{\Ap}]\Ap   & \indexugar[\textcugar{\Ai}]\Ai   &                          \\
\end{tabular}

\bigskip
\begin{tablenote}
  \usefontcmdmessage{\textcugar}{\cugarfamily}.  Single-character
  shortcuts and various aliases are also supported:
  ``\verb+\textcopsn{\Ap\Aq\An}+'',
  ``\verb+\textcopsn{\Ape\Aqoph\Anun}+'', and
  ``\verb+\textcopsn{pqn}+'' all produce ``\textcopsn{pqn}'', for
  example.  \seedocs{\UGAR}.
\end{tablenote}
\end{symtable}


\begin{longsymtable}[SARAB]{\SARAB\ South Arabian Letters}
\index{South Arabian alphabet}
\index{alphabets>South Arabian}
\label{sarabian}
\begin{longtable}{*4{ll@{\qquad}}ll}
\indexsoutharabian[\textsarab{\SAa}]\SAa   & \indexsoutharabian[\textsarab{\SAz}]\SAz   & \indexsoutharabian[\textsarab{\SAm}]\SAm   & \indexsoutharabian[\textsarab{\SAsd}]\SAsd & \indexsoutharabian[\textsarab{\SAdb}]\SAdb \\
\indexsoutharabian[\textsarab{\SAb}]\SAb   & \indexsoutharabian[\textsarab{\SAhd}]\SAhd & \indexsoutharabian[\textsarab{\SAn}]\SAn   & \indexsoutharabian[\textsarab{\SAq}]\SAq   & \indexsoutharabian[\textsarab{\SAtb}]\SAtb \\
\indexsoutharabian[\textsarab{\SAg}]\SAg   & \indexsoutharabian[\textsarab{\SAtd}]\SAtd & \indexsoutharabian[\textsarab{\SAs}]\SAs   & \indexsoutharabian[\textsarab{\SAr}]\SAr   & \indexsoutharabian[\textsarab{\SAga}]\SAga \\
\indexsoutharabian[\textsarab{\SAd}]\SAd   & \indexsoutharabian[\textsarab{\SAy}]\SAy   & \indexsoutharabian[\textsarab{\SAf}]\SAf   & \indexsoutharabian[\textsarab{\SAsv}]\SAsv & \indexsoutharabian[\textsarab{\SAzd}]\SAzd \\
\indexsoutharabian[\textsarab{\SAh}]\SAh   & \indexsoutharabian[\textsarab{\SAk}]\SAk   & \indexsoutharabian[\textsarab{\SAlq}]\SAlq & \indexsoutharabian[\textsarab{\SAt}]\SAt   & \indexsoutharabian[\textsarab{\SAsa}]\SAsa \\
\indexsoutharabian[\textsarab{\SAw}]\SAw   & \indexsoutharabian[\textsarab{\SAl}]\SAl   & \indexsoutharabian[\textsarab{\SAo}]\SAo   & \indexsoutharabian[\textsarab{\SAhu}]\SAhu & \indexsoutharabian[\textsarab{\SAdd}]\SAdd \\
\end{longtable}

\bigskip
\begin{tablenote}
  \usefontcmdmessage{\textsarab}{\sarabfamily}.  Single-character
  shortcuts are also supported: Both
  ``\verb+\textsarab{\SAb\SAk\SAn}+'' and ``\verb+\textsarab{bkn}+''
  produce ``\textsarab{bkn}'', for example.  \seedocs{\SARAB}.
\end{tablenote}
\end{longsymtable}

\begin{longsymtable}[LINA]{\LINA\ Linear~A Script}
\index{Linear A}
\index{alphabets>Linear A}
\label{linearA}
\begin{longtable}{*3{ll@{\quad}}ll}
\multicolumn{8}{l}{\small\textit{(continued from previous page)}} \\[1ex]
\endhead
\endfirsthead
\\[3ex]
\multicolumn{8}{r}{\small\textit{(continued on next page)}}
\endfoot
\endlastfoot
\indexlinearb\LinearAI           & \indexlinearb\LinearAXCIX        & \indexlinearb\LinearACXCVII      & \indexlinearb\LinearACCXCV       \\
\indexlinearb\LinearAII          & \indexlinearb\LinearAC           & \indexlinearb\LinearACXCVIII     & \indexlinearb\LinearACCXCVI      \\
\indexlinearb\LinearAIII         & \indexlinearb\LinearACI          & \indexlinearb\LinearACXCIX       & \indexlinearb\LinearACCXCVII     \\
\indexlinearb\LinearAIV          & \indexlinearb\LinearACII         & \indexlinearb\LinearACC          & \indexlinearb\LinearACCXCVIII    \\
\indexlinearb\LinearAV           & \indexlinearb\LinearACIII        & \indexlinearb\LinearACCI         & \indexlinearb\LinearACCXCIX      \\
\indexlinearb\LinearAVI          & \indexlinearb\LinearACIV         & \indexlinearb\LinearACCII        & \indexlinearb\LinearACCC         \\
\indexlinearb\LinearAVII         & \indexlinearb\LinearACV          & \indexlinearb\LinearACCIII       & \indexlinearb\LinearACCCI        \\
\indexlinearb\LinearAVIII        & \indexlinearb\LinearACVI         & \indexlinearb\LinearACCIV        & \indexlinearb\LinearACCCII       \\
\indexlinearb\LinearAIX          & \indexlinearb\LinearACVII        & \indexlinearb\LinearACCV         & \indexlinearb\LinearACCCIII      \\
\indexlinearb\LinearAX           & \indexlinearb\LinearACVIII       & \indexlinearb\LinearACCVI        & \indexlinearb\LinearACCCIV       \\
\indexlinearb\LinearAXI          & \indexlinearb\LinearACIX         & \indexlinearb\LinearACCVII       & \indexlinearb\LinearACCCV        \\
\indexlinearb\LinearAXII         & \indexlinearb\LinearACX          & \indexlinearb\LinearACCVIII      & \indexlinearb\LinearACCCVI       \\
\indexlinearb\LinearAXIII        & \indexlinearb\LinearACXI         & \indexlinearb\LinearACCIX        & \indexlinearb\LinearACCCVII      \\
\indexlinearb\LinearAXIV         & \indexlinearb\LinearACXII        & \indexlinearb\LinearACCX         & \indexlinearb\LinearACCCVIII     \\
\indexlinearb\LinearAXV          & \indexlinearb\LinearACXIII       & \indexlinearb\LinearACCXI        & \indexlinearb\LinearACCCIX       \\
\indexlinearb\LinearAXVI         & \indexlinearb\LinearACXIV        & \indexlinearb\LinearACCXII       & \indexlinearb\LinearACCCX        \\
\indexlinearb\LinearAXVII        & \indexlinearb\LinearACXV         & \indexlinearb\LinearACCXIII      & \indexlinearb\LinearACCCXI       \\
\indexlinearb\LinearAXVIII       & \indexlinearb\LinearACXVI        & \indexlinearb\LinearACCXIV       & \indexlinearb\LinearACCCXII      \\
\indexlinearb\LinearAXIX         & \indexlinearb\LinearACXVII       & \indexlinearb\LinearACCXV        & \indexlinearb\LinearACCCXIII     \\
\indexlinearb\LinearAXX          & \indexlinearb\LinearACXVIII      & \indexlinearb\LinearACCXVI       & \indexlinearb\LinearACCCXIV      \\
\indexlinearb\LinearAXXI         & \indexlinearb\LinearACXIX        & \indexlinearb\LinearACCXVII      & \indexlinearb\LinearACCCXV       \\
\indexlinearb\LinearAXXII        & \indexlinearb\LinearACXX         & \indexlinearb\LinearACCXVIII     & \indexlinearb\LinearACCCXVI      \\
\indexlinearb\LinearAXXIII       & \indexlinearb\LinearACXXI        & \indexlinearb\LinearACCXIX       & \indexlinearb\LinearACCCXVII     \\
\indexlinearb\LinearAXXIV        & \indexlinearb\LinearACXXII       & \indexlinearb\LinearACCXX        & \indexlinearb\LinearACCCXVIII    \\
\indexlinearb\LinearAXXV         & \indexlinearb\LinearACXXIII      & \indexlinearb\LinearACCXXI       & \indexlinearb\LinearACCCXIX      \\
\indexlinearb\LinearAXXVI        & \indexlinearb\LinearACXXIV       & \indexlinearb\LinearACCXXII      & \indexlinearb\LinearACCCXX       \\
\indexlinearb\LinearAXXVII       & \indexlinearb\LinearACXXV        & \indexlinearb\LinearACCXXIII     & \indexlinearb\LinearACCCXXI      \\
\indexlinearb\LinearAXXVIII      & \indexlinearb\LinearACXXVI       & \indexlinearb\LinearACCXXIV      & \indexlinearb\LinearACCCXXII     \\
\indexlinearb\LinearAXXIX        & \indexlinearb\LinearACXXVII      & \indexlinearb\LinearACCXXV       & \indexlinearb\LinearACCCXXIII    \\
\indexlinearb\LinearAXXX         & \indexlinearb\LinearACXXVIII     & \indexlinearb\LinearACCXXVI      & \indexlinearb\LinearACCCXXIV     \\
\indexlinearb\LinearAXXXI        & \indexlinearb\LinearACXXIX       & \indexlinearb\LinearACCXXVII     & \indexlinearb\LinearACCCXXV      \\
\indexlinearb\LinearAXXXII       & \indexlinearb\LinearACXXX        & \indexlinearb\LinearACCXXVIII    & \indexlinearb\LinearACCCXXVI     \\
\indexlinearb\LinearAXXXIII      & \indexlinearb\LinearACXXXI       & \indexlinearb\LinearACCXXIX      & \indexlinearb\LinearACCCXXVII    \\
\indexlinearb\LinearAXXXIV       & \indexlinearb\LinearACXXXII      & \indexlinearb\LinearACCXXX       & \indexlinearb\LinearACCCXXVIII   \\
\indexlinearb\LinearAXXXV        & \indexlinearb\LinearACXXXIII     & \indexlinearb\LinearACCXXXI      & \indexlinearb\LinearACCCXXIX     \\
\indexlinearb\LinearAXXXVI       & \indexlinearb\LinearACXXXIV      & \indexlinearb\LinearACCXXXII     & \indexlinearb\LinearACCCXXX      \\
\indexlinearb\LinearAXXXVII      & \indexlinearb\LinearACXXXV       & \indexlinearb\LinearACCXXXIII    & \indexlinearb\LinearACCCXXXI     \\
\indexlinearb\LinearAXXXVIII     & \indexlinearb\LinearACXXXVI      & \indexlinearb\LinearACCXXXIV     & \indexlinearb\LinearACCCXXXII    \\
\indexlinearb\LinearAXXXIX       & \indexlinearb\LinearACXXXVII     & \indexlinearb\LinearACCXXXV      & \indexlinearb\LinearACCCXXXIII   \\
\indexlinearb\LinearAXL          & \indexlinearb\LinearACXXXVIII    & \indexlinearb\LinearACCXXXVI     & \indexlinearb\LinearACCCXXXIV    \\
\indexlinearb\LinearAXLI         & \indexlinearb\LinearACXXXIX      & \indexlinearb\LinearACCXXXVII    & \indexlinearb\LinearACCCXXXV     \\
\indexlinearb\LinearAXLII        & \indexlinearb\LinearACXL         & \indexlinearb\LinearACCXXXVIII   & \indexlinearb\LinearACCCXXXVI    \\
\indexlinearb\LinearAXLIII       & \indexlinearb\LinearACXLI        & \indexlinearb\LinearACCXXXIX     & \indexlinearb\LinearACCCXXXVII   \\
\indexlinearb\LinearAXLIV        & \indexlinearb\LinearACXLII       & \indexlinearb\LinearACCXL        & \indexlinearb\LinearACCCXXXVIII  \\
\indexlinearb\LinearAXLV         & \indexlinearb\LinearACXLIII      & \indexlinearb\LinearACCXLI       & \indexlinearb\LinearACCCXXXIX    \\
\indexlinearb\LinearAXLVI        & \indexlinearb\LinearACXLIV       & \indexlinearb\LinearACCXLII      & \indexlinearb\LinearACCCXL       \\
\indexlinearb\LinearAXLVII       & \indexlinearb\LinearACXLV        & \indexlinearb\LinearACCXLIII     & \indexlinearb\LinearACCCXLI      \\
\indexlinearb\LinearAXLVIII      & \indexlinearb\LinearACXLVI       & \indexlinearb\LinearACCXLIV      & \indexlinearb\LinearACCCXLII     \\
\indexlinearb\LinearAXLIX        & \indexlinearb\LinearACXLVII      & \indexlinearb\LinearACCXLV       & \indexlinearb\LinearACCCXLIII    \\
\indexlinearb\LinearAL           & \indexlinearb\LinearACXLVIII     & \indexlinearb\LinearACCXLVI      & \indexlinearb\LinearACCCXLIV     \\
\indexlinearb\LinearALI          & \indexlinearb\LinearACXLIX       & \indexlinearb\LinearACCXLVII     & \indexlinearb\LinearACCCXLV      \\
\indexlinearb\LinearALII         & \indexlinearb\LinearACL          & \indexlinearb\LinearACCXLVIII    & \indexlinearb\LinearACCCXLVI     \\
\indexlinearb\LinearALIII        & \indexlinearb\LinearACLI         & \indexlinearb\LinearACCXLIX      & \indexlinearb\LinearACCCXLVII    \\
\indexlinearb\LinearALIV         & \indexlinearb\LinearACLII        & \indexlinearb\LinearACCL         & \indexlinearb\LinearACCCXLVIII   \\
\indexlinearb\LinearALV          & \indexlinearb\LinearACLIII       & \indexlinearb\LinearACCLI        & \indexlinearb\LinearACCCXLIX     \\
\indexlinearb\LinearALVI         & \indexlinearb\LinearACLIV        & \indexlinearb\LinearACCLII       & \indexlinearb\LinearACCCL        \\
\indexlinearb\LinearALVII        & \indexlinearb\LinearACLV         & \indexlinearb\LinearACCLIII      & \indexlinearb\LinearACCCLI       \\
\indexlinearb\LinearALVIII       & \indexlinearb\LinearACLVI        & \indexlinearb\LinearACCLIV       & \indexlinearb\LinearACCCLII      \\
\indexlinearb\LinearALIX         & \indexlinearb\LinearACLVII       & \indexlinearb\LinearACCLV        & \indexlinearb\LinearACCCLIII     \\
\indexlinearb\LinearALX          & \indexlinearb\LinearACLVIII      & \indexlinearb\LinearACCLVI       & \indexlinearb\LinearACCCLIV      \\
\indexlinearb\LinearALXI         & \indexlinearb\LinearACLIX        & \indexlinearb\LinearACCLVII      & \indexlinearb\LinearACCCLV       \\
\indexlinearb\LinearALXII        & \indexlinearb\LinearACLX         & \indexlinearb\LinearACCLVIII     & \indexlinearb\LinearACCCLVI      \\
\indexlinearb\LinearALXIII       & \indexlinearb\LinearACLXI        & \indexlinearb\LinearACCLIX       & \indexlinearb\LinearACCCLVII     \\
\indexlinearb\LinearALXIV        & \indexlinearb\LinearACLXII       & \indexlinearb\LinearACCLX        & \indexlinearb\LinearACCCLVIII    \\
\indexlinearb\LinearALXV         & \indexlinearb\LinearACLXIII      & \indexlinearb\LinearACCLXI       & \indexlinearb\LinearACCCLIX      \\
\indexlinearb\LinearALXVI        & \indexlinearb\LinearACLXIV       & \indexlinearb\LinearACCLXII      & \indexlinearb\LinearACCCLX       \\
\indexlinearb\LinearALXVII       & \indexlinearb\LinearACLXV        & \indexlinearb\LinearACCLXIII     & \indexlinearb\LinearACCCLXI      \\
\indexlinearb\LinearALXVIII      & \indexlinearb\LinearACLXVI       & \indexlinearb\LinearACCLXIV      & \indexlinearb\LinearACCCLXII     \\
\indexlinearb\LinearALXIX        & \indexlinearb\LinearACLXVII      & \indexlinearb\LinearACCLXV       & \indexlinearb\LinearACCCLXIII    \\
\indexlinearb\LinearALXX         & \indexlinearb\LinearACLXVIII     & \indexlinearb\LinearACCLXVI      & \indexlinearb\LinearACCCLXIV     \\
\indexlinearb\LinearALXXI        & \indexlinearb\LinearACLXIX       & \indexlinearb\LinearACCLXVII     & \indexlinearb\LinearACCCLXV      \\
\indexlinearb\LinearALXXII       & \indexlinearb\LinearACLXX        & \indexlinearb\LinearACCLXVIII    & \indexlinearb\LinearACCCLXVI     \\
\indexlinearb\LinearALXXIII      & \indexlinearb\LinearACLXXI       & \indexlinearb\LinearACCLXIX      & \indexlinearb\LinearACCCLXVII    \\
\indexlinearb\LinearALXXIV       & \indexlinearb\LinearACLXXII      & \indexlinearb\LinearACCLXX       & \indexlinearb\LinearACCCLXVIII   \\
\indexlinearb\LinearALXXV        & \indexlinearb\LinearACLXXIII     & \indexlinearb\LinearACCLXXI      & \indexlinearb\LinearACCCLXIX     \\
\indexlinearb\LinearALXXVI       & \indexlinearb\LinearACLXXIV      & \indexlinearb\LinearACCLXXII     & \indexlinearb\LinearACCCLXX      \\
\indexlinearb\LinearALXXVII      & \indexlinearb\LinearACLXXV       & \indexlinearb\LinearACCLXXIII    & \indexlinearb\LinearACCCLXXI     \\
\indexlinearb\LinearALXXVIII     & \indexlinearb\LinearACLXXVI      & \indexlinearb\LinearACCLXXIV     & \indexlinearb\LinearACCCLXXII    \\
\indexlinearb\LinearALXXIX       & \indexlinearb\LinearACLXXVII     & \indexlinearb\LinearACCLXXV      & \indexlinearb\LinearACCCLXXIII   \\
\indexlinearb\LinearALXXX        & \indexlinearb\LinearACLXXVIII    & \indexlinearb\LinearACCLXXVI     & \indexlinearb\LinearACCCLXXIV    \\
\indexlinearb\LinearALXXXI       & \indexlinearb\LinearACLXXIX      & \indexlinearb\LinearACCLXXVII    & \indexlinearb\LinearACCCLXXV     \\
\indexlinearb\LinearALXXXII      & \indexlinearb\LinearACLXXX       & \indexlinearb\LinearACCLXXVIII   & \indexlinearb\LinearACCCLXXVI    \\
\indexlinearb\LinearALXXXIII     & \indexlinearb\LinearACLXXXI      & \indexlinearb\LinearACCLXXIX     & \indexlinearb\LinearACCCLXXVII   \\
\indexlinearb\LinearALXXXIV      & \indexlinearb\LinearACLXXXII     & \indexlinearb\LinearACCLXXX      & \indexlinearb\LinearACCCLXXVIII  \\
\indexlinearb\LinearALXXXV       & \indexlinearb\LinearACLXXXIII    & \indexlinearb\LinearACCLXXXI     & \indexlinearb\LinearACCCLXXIX    \\
\indexlinearb\LinearALXXXVI      & \indexlinearb\LinearACLXXXIV     & \indexlinearb\LinearACCLXXXII    & \indexlinearb\LinearACCCLXXX     \\
\indexlinearb\LinearALXXXVII     & \indexlinearb\LinearACLXXXV      & \indexlinearb\LinearACCLXXXIII   & \indexlinearb\LinearACCCLXXXI    \\
\indexlinearb\LinearALXXXVIII    & \indexlinearb\LinearACLXXXVI     & \indexlinearb\LinearACCLXXXIV    & \indexlinearb\LinearACCCLXXXII   \\
\indexlinearb\LinearALXXXIX      & \indexlinearb\LinearACLXXXVII    & \indexlinearb\LinearACCLXXXV     & \indexlinearb\LinearACCCLXXXIII  \\
\indexlinearb\LinearALXXXX       & \indexlinearb\LinearACLXXXVIII   & \indexlinearb\LinearACCLXXXVI    & \indexlinearb\LinearACCCLXXXIV   \\
\indexlinearb\LinearAXCI         & \indexlinearb\LinearACLXXXIX     & \indexlinearb\LinearACCLXXXVII   & \indexlinearb\LinearACCCLXXXV    \\
\indexlinearb\LinearAXCII        & \indexlinearb\LinearACLXXXX      & \indexlinearb\LinearACCLXXXVIII  & \indexlinearb\LinearACCCLXXXVI   \\
\indexlinearb\LinearAXCIII       & \indexlinearb\LinearACXCI        & \indexlinearb\LinearACCLXXXIX    & \indexlinearb\LinearACCCLXXXVII  \\
\indexlinearb\LinearAXCIV        & \indexlinearb\LinearACXCII       & \indexlinearb\LinearACCLXXXX     & \indexlinearb\LinearACCCLXXXVIII \\
\indexlinearb\LinearAXCV         & \indexlinearb\LinearACXCIII      & \indexlinearb\LinearACCXCI       & \indexlinearb\LinearACCCLXXXIX   \\
\indexlinearb\LinearAXCVI        & \indexlinearb\LinearACXCIV       & \indexlinearb\LinearACCXCII      &                       \\
\indexlinearb\LinearAXCVII       & \indexlinearb\LinearACXCV        & \indexlinearb\LinearACCXCIII     &                       \\
\indexlinearb\LinearAXCVIII      & \indexlinearb\LinearACXCVI       & \indexlinearb\LinearACCXCIV      &                       \\
\end{longtable}
\end{longsymtable}

\begin{longsymtable}[LINB]{\LINB\ Linear~B Basic and Optional Letters}
\index{Linear B}
\index{alphabets>Linear B}
\label{linearB}
\begin{longtable}{*5{ll@{\qquad}}ll}
\indexlinearb[\textlinb{\Ba}]\Ba         & \indexlinearb[\textlinb{\Bja}]\Bja       & \indexlinearb[\textlinb{\Bmu}]\Bmu       & \indexlinearb[\textlinb{\Bpte}]\Bpte     & \indexlinearb[\textlinb{\Broii}]\Broii   & \indexlinearb[\textlinb{\Bto}]\Bto       \\
\indexlinearb[\textlinb{\Baii}]\Baii     & \indexlinearb[\textlinb{\Bje}]\Bje       & \indexlinearb[\textlinb{\Bna}]\Bna       & \indexlinearb[\textlinb{\Bpu}]\Bpu       & \indexlinearb[\textlinb{\Bru}]\Bru       & \indexlinearb[\textlinb{\Btu}]\Btu       \\
\indexlinearb[\textlinb{\Baiii}]\Baiii   & \indexlinearb[\textlinb{\Bjo}]\Bjo       & \indexlinearb[\textlinb{\Bne}]\Bne       & \indexlinearb[\textlinb{\Bpuii}]\Bpuii   & \indexlinearb[\textlinb{\Bsa}]\Bsa       & \indexlinearb[\textlinb{\Btwo}]\Btwo     \\
\indexlinearb[\textlinb{\Bau}]\Bau       & \indexlinearb[\textlinb{\Bju}]\Bju       & \indexlinearb[\textlinb{\Bni}]\Bni       & \indexlinearb[\textlinb{\Bqa}]\Bqa       & \indexlinearb[\textlinb{\Bse}]\Bse       & \indexlinearb[\textlinb{\Bu}]\Bu         \\
\indexlinearb[\textlinb{\Bda}]\Bda       & \indexlinearb[\textlinb{\Bka}]\Bka       & \indexlinearb[\textlinb{\Bno}]\Bno       & \indexlinearb[\textlinb{\Bqe}]\Bqe       & \indexlinearb[\textlinb{\Bsi}]\Bsi       & \indexlinearb[\textlinb{\Bwa}]\Bwa       \\
\indexlinearb[\textlinb{\Bde}]\Bde       & \indexlinearb[\textlinb{\Bke}]\Bke       & \indexlinearb[\textlinb{\Bnu}]\Bnu       & \indexlinearb[\textlinb{\Bqi}]\Bqi       & \indexlinearb[\textlinb{\Bso}]\Bso       & \indexlinearb[\textlinb{\Bwe}]\Bwe       \\
\indexlinearb[\textlinb{\Bdi}]\Bdi       & \indexlinearb[\textlinb{\Bki}]\Bki       & \indexlinearb[\textlinb{\Bnwa}]\Bnwa     & \indexlinearb[\textlinb{\Bqo}]\Bqo       & \indexlinearb[\textlinb{\Bsu}]\Bsu       & \indexlinearb[\textlinb{\Bwi}]\Bwi       \\
\indexlinearb[\textlinb{\Bdo}]\Bdo       & \indexlinearb[\textlinb{\Bko}]\Bko       & \indexlinearb[\textlinb{\Bo}]\Bo         & \indexlinearb[\textlinb{\Bra}]\Bra       & \indexlinearb[\textlinb{\Bswa}]\Bswa     & \indexlinearb[\textlinb{\Bwo}]\Bwo       \\
\indexlinearb[\textlinb{\Bdu}]\Bdu       & \indexlinearb[\textlinb{\Bku}]\Bku       & \indexlinearb[\textlinb{\Bpa}]\Bpa       & \indexlinearb[\textlinb{\Braii}]\Braii   & \indexlinearb[\textlinb{\Bswi}]\Bswi     & \indexlinearb[\textlinb{\Bza}]\Bza       \\
\indexlinearb[\textlinb{\Bdwe}]\Bdwe     & \indexlinearb[\textlinb{\Bma}]\Bma       & \indexlinearb[\textlinb{\Bpaiii}]\Bpaiii & \indexlinearb[\textlinb{\Braiii}]\Braiii & \indexlinearb[\textlinb{\Bta}]\Bta       & \indexlinearb[\textlinb{\Bze}]\Bze       \\
\indexlinearb[\textlinb{\Bdwo}]\Bdwo     & \indexlinearb[\textlinb{\Bme}]\Bme       & \indexlinearb[\textlinb{\Bpe}]\Bpe       & \indexlinearb[\textlinb{\Bre}]\Bre       & \indexlinearb[\textlinb{\Btaii}]\Btaii   & \indexlinearb[\textlinb{\Bzo}]\Bzo       \\
\indexlinearb[\textlinb{\Be}]\Be         & \indexlinearb[\textlinb{\Bmi}]\Bmi       & \indexlinearb[\textlinb{\Bpi}]\Bpi       & \indexlinearb[\textlinb{\Bri}]\Bri       & \indexlinearb[\textlinb{\Bte}]\Bte       &                               \\
\indexlinearb[\textlinb{\Bi}]\Bi         & \indexlinearb[\textlinb{\Bmo}]\Bmo       & \indexlinearb[\textlinb{\Bpo}]\Bpo       & \indexlinearb[\textlinb{\Bro}]\Bro       & \indexlinearb[\textlinb{\Bti}]\Bti       &                               \\
\end{longtable}

\bigskip
\begin{tablenote}
  \usefontcmdmessage{\textlinb}{\linbfamily}.  Single-character
  shortcuts are also supported: Both
  ``\verb+\textlinb{\Bpa\Bki\Bna}+'' and ``\verb+\textlinb{pcn}+''
  produce ``\textlinb{pcn}'', for example.  \seedocs{\LINB}.
\end{tablenote}
\end{longsymtable}


\begin{symtable}[LINB]{\LINB\ Linear~B Numerals}
\index{Linear B}
\index{numerals>Linear B}
\index{tally markers}
\label{linearB-nums}
\begin{tabular}{*4{ll@{\qquad}}ll}
\indexlinearb[\textlinb{\BNi}]\BNi       & \indexlinearb[\textlinb{\BNvii}]\BNvii   & \indexlinearb[\textlinb{\BNxl}]\BNxl     & \indexlinearb[\textlinb{\BNc}]\BNc       & \indexlinearb[\textlinb{\BNdcc}]\BNdcc   \\
\indexlinearb[\textlinb{\BNii}]\BNii     & \indexlinearb[\textlinb{\BNviii}]\BNviii & \indexlinearb[\textlinb{\BNl}]\BNl       & \indexlinearb[\textlinb{\BNcc}]\BNcc     & \indexlinearb[\textlinb{\BNdccc}]\BNdccc \\
\indexlinearb[\textlinb{\BNiii}]\BNiii   & \indexlinearb[\textlinb{\BNix}]\BNix     & \indexlinearb[\textlinb{\BNlx}]\BNlx     & \indexlinearb[\textlinb{\BNccc}]\BNccc   & \indexlinearb[\textlinb{\BNcm}]\BNcm     \\
\indexlinearb[\textlinb{\BNiv}]\BNiv     & \indexlinearb[\textlinb{\BNx}]\BNx       & \indexlinearb[\textlinb{\BNlxx}]\BNlxx   & \indexlinearb[\textlinb{\BNcd}]\BNcd     & \indexlinearb[\textlinb{\BNm}]\BNm       \\
\indexlinearb[\textlinb{\BNv}]\BNv       & \indexlinearb[\textlinb{\BNxx}]\BNxx     & \indexlinearb[\textlinb{\BNlxxx}]\BNlxxx & \indexlinearb[\textlinb{\BNd}]\BNd       &                               \\
\indexlinearb[\textlinb{\BNvi}]\BNvi     & \indexlinearb[\textlinb{\BNxxx}]\BNxxx   & \indexlinearb[\textlinb{\BNxc}]\BNxc     & \indexlinearb[\textlinb{\BNdc}]\BNdc     &                               \\
\end{tabular}

\bigskip
\begin{tablenote}
  \usefontcmdmessage{\textlinb}{\linbfamily}.
\end{tablenote}
\end{symtable}


\begin{symtable}[LINB]{\LINB\ Linear~B Weights and Measures}
\index{Linear B}
\label{linearB-weights}
\begin{tabular}{*4{ll@{\qquad}}ll}
\indexlinearb[\textlinb{\BPtalent}]\BPtalent & \indexlinearb[\textlinb{\BPvolb}]\BPvolb     & \indexlinearb[\textlinb{\BPvolcf}]\BPvolcf   & \indexlinearb[\textlinb{\BPwtb}]\BPwtb       & \indexlinearb[\textlinb{\BPwtd}]\BPwtd       \\
\indexlinearb[\textlinb{\BPvola}]\BPvola     & \indexlinearb[\textlinb{\BPvolcd}]\BPvolcd   & \indexlinearb[\textlinb{\BPwta}]\BPwta       & \indexlinearb[\textlinb{\BPwtc}]\BPwtc       &                                   \\
\end{tabular}

\bigskip
\begin{tablenote}
  \usefontcmdmessage{\textlinb}{\linbfamily}.
\end{tablenote}
\end{symtable}


\begin{symtable}[LINB]{\LINB\ Linear~B Ideograms}
\index{Linear B}
\index{arrows}
\index{animals}
\label{linearB-objs}
\begin{tabular}{*3{ll@{\qquad}}ll}
\indexlinearb[\textlinb{\BPamphora}]\BPamphora       & \indexlinearb[\textlinb{\BPchassis}]\BPchassis       & \indexlinearb[\textlinb{\BPman}]\BPman               & \indexlinearb[\textlinb{\BPwheat}]\BPwheat           \\
\indexlinearb[\textlinb{\BParrow}]\BParrow           & \indexlinearb[\textlinb{\BPcloth}]\BPcloth           & \indexlinearb[\textlinb{\BPnanny}]\BPnanny           & \indexlinearb[\textlinb{\BPwheel}]\BPwheel           \\
\indexlinearb[\textlinb{\BPbarley}]\BPbarley         & \indexlinearb[\textlinb{\BPcow}]\BPcow               & \indexlinearb[\textlinb{\BPolive}]\BPolive           & \indexlinearb[\textlinb{\BPwine}]\BPwine             \\
\indexlinearb[\textlinb{\BPbilly}]\BPbilly           & \indexlinearb[\textlinb{\BPcup}]\BPcup               & \indexlinearb[\textlinb{\BPox}]\BPox                 & \indexlinearb[\textlinb{\BPwineiih}]\BPwineiih       \\
\indexlinearb[\textlinb{\BPboar}]\BPboar             & \indexlinearb[\textlinb{\BPewe}]\BPewe               & \indexlinearb[\textlinb{\BPpig}]\BPpig               & \indexlinearb[\textlinb{\BPwineiiih}]\BPwineiiih     \\
\indexlinearb[\textlinb{\BPbronze}]\BPbronze         & \indexlinearb[\textlinb{\BPfoal}]\BPfoal             & \indexlinearb[\textlinb{\BPram}]\BPram               & \indexlinearb[\textlinb{\BPwineivh}]\BPwineivh       \\
\indexlinearb[\textlinb{\BPbull}]\BPbull             & \indexlinearb[\textlinb{\BPgoat}]\BPgoat             & \indexlinearb[\textlinb{\BPsheep}]\BPsheep           & \indexlinearb[\textlinb{\BPwoman}]\BPwoman           \\
\indexlinearb[\textlinb{\BPcauldroni}]\BPcauldroni   & \indexlinearb[\textlinb{\BPgoblet}]\BPgoblet         & \indexlinearb[\textlinb{\BPsow}]\BPsow               & \indexlinearb[\textlinb{\BPwool}]\BPwool             \\
\indexlinearb[\textlinb{\BPcauldronii}]\BPcauldronii & \indexlinearb[\textlinb{\BPgold}]\BPgold             & \indexlinearb[\textlinb{\BPspear}]\BPspear           &                                           \\
\indexlinearb[\textlinb{\BPchariot}]\BPchariot       & \indexlinearb[\textlinb{\BPhorse}]\BPhorse           & \indexlinearb[\textlinb{\BPsword}]\BPsword           &                                           \\
\end{tabular}

\bigskip
\begin{tablenote}
  \usefontcmdmessage{\textlinb}{\linbfamily}.
\end{tablenote}
\end{symtable}


\begin{longsymtable}[LINB]{\LINB\ Unidentified Linear~B Symbols}
\index{Linear B}
\label{linearB-unknown}
\begin{longtable}{*4{ll@{\qquad}}ll}
\indexlinearb[\textlinb{\BUi}]\BUi       & \indexlinearb[\textlinb{\BUiv}]\BUiv     & \indexlinearb[\textlinb{\BUvii}]\BUvii   & \indexlinearb[\textlinb{\BUx}]\BUx       & \indexlinearb[\textlinb{\Btwe}]\Btwe     \\
\indexlinearb[\textlinb{\BUii}]\BUii     & \indexlinearb[\textlinb{\BUv}]\BUv       & \indexlinearb[\textlinb{\BUviii}]\BUviii & \indexlinearb[\textlinb{\BUxi}]\BUxi     &                               \\
\indexlinearb[\textlinb{\BUiii}]\BUiii   & \indexlinearb[\textlinb{\BUvi}]\BUvi     & \indexlinearb[\textlinb{\BUix}]\BUix     & \indexlinearb[\textlinb{\BUxii}]\BUxii   &                               \\
\end{longtable}

\bigskip
\begin{tablenote}
  \usefontcmdmessage{\textlinb}{\linbfamily}.
\end{tablenote}
\end{longsymtable}

\section{Magical Staves}

\begin{longsymtable}[STAVE]{\STAVE\ Magical Staves}
\index{symbols>staves}
\index{symbols>magical signs}
\index{magical signs}
\index{staves}
\index{Icelandic staves}
\label{staves}
\small
\begin{longtable}{*2{ll@{\qqquad}}ll}
\multicolumn{6}{l}{\small\textit{(continued from previous page)}} \\[3ex]
\endhead
\endfirsthead
\\[3ex]
\multicolumn{6}{r}{\small\textit{(continued on next page)}}
\endfoot
\endlastfoot
\Kstav\staveI     & \Kstav\staveXXIV    & \Kstav\staveXLVII  \\
\Kstav\staveII    & \Kstav\staveXXV     & \Kstav\staveXLVIII \\
\Kstav\staveIII   & \Kstav\staveXXVI    & \Kstav\staveXLIX   \\
\Kstav\staveIV    & \Kstav\staveXXVII   & \Kstav\staveL      \\
\Kstav\staveV     & \Kstav\staveXXVIII  & \Kstav\staveLI     \\
\Kstav\staveVI    & \Kstav\staveXXIX    & \Kstav\staveLII    \\
\Kstav\staveVII   & \Kstav\staveXXX     & \Kstav\staveLIII   \\
\Kstav\staveVIII  & \Kstav\staveXXXI    & \Kstav\staveLIV    \\
\Kstav\staveIX    & \Kstav\staveXXXII   & \Kstav\staveLV     \\
\Kstav\staveX     & \Kstav\staveXXXIII  & \Kstav\staveLVI    \\
\Kstav\staveXI    & \Kstav\staveXXXIV   & \Kstav\staveLVII   \\
\Kstav\staveXII   & \Kstav\staveXXXV    & \Kstav\staveLVIII  \\
\Kstav\staveXIII  & \Kstav\staveXXXVI   & \Kstav\staveLIX    \\
\Kstav\staveXIV   & \Kstav\staveXXXVII  & \Kstav\staveLX     \\
\Kstav\staveXV    & \Kstav\staveXXXVIII & \Kstav\staveLXI    \\
\Kstav\staveXVI   & \Kstav\staveXXXIX   & \Kstav\staveLXII   \\
\Kstav\staveXVII  & \Kstav\staveXL      & \Kstav\staveLXIII  \\
\Kstav\staveXVIII & \Kstav\staveXLI     & \Kstav\staveLXIV   \\
\Kstav\staveXIX   & \Kstav\staveXLII    & \Kstav\staveLXV    \\
\Kstav\staveXX    & \Kstav\staveXLIII   & \Kstav\staveLXVI   \\
\Kstav\staveXXI   & \Kstav\staveXLIV    & \Kstav\staveLXVII  \\
\Kstav\staveXXII  & \Kstav\staveXLV     & \Kstav\staveLXVIII \\
\Kstav\staveXXIII & \Kstav\staveXLVI    &                \\
\end{longtable}

\bigskip

\begin{tablenote}
  The meanings of these symbols are described on the Web site for the
  Museum of Icelandic Sorcery and Witchcraft\index{Museum of Icelandic
  Sorcery and Witchcraft} at
  \url{http://www.galdrasyning.is/index.php?option=com_content&task=category&sectionid=5&id=18&Itemid=60}
  (TinyURL: \url{http://tinyurl.com/25979m}).  For example,
  \docAuxCommand{staveL}~(``\staveL'') is intended to ward off
  ghosts\index{ghosts} and evil\index{evil spirits} spirits.
\end{tablenote}
\end{longsymtable}


\subsection{Resizing symbols}
\label{resizing-symbols}
\index{symbols>resize}

Mathematical symbols listed in this document as
``variable-sized\idxboth{variable-sized}{symbols}'' are designed to
stretch vertically.  Each
variable-sized\idxboth{variable-sized}{symbols} symbol comes in one or
more basic sizes plus a variation comprising both stretchable and
nonstretchable segments.  Table \vref{var-sized-syms} presents the
symbols %\docAuxCommand{}}
 and \docAuxCommand{uparrow} in their default size, in their
\cmd{\big}, \cmd{\Big}, \cmd{\bigg}, and \cmd{\Bigg} sizes, in an even
larger size achieved using \cmd{\left}\slash\cmd{\right}, and---for
contrast---in a large size achieved by changing the font size using
\latexe's \cmd{\fontsize} command.  Because the symbols shown belong
to the \PSfont{Computer Modern} family, the \pkgname{type1cm} package
needs to be loaded to support font sizes larger than 24.88\,pt.

\begin{nonsymtable}{Sample resized delimiters}
\idxboth{variable-sized}{symbols}
\label{var-sized-syms}
\newcommand{\maketall}[1]{\ensuremath{\left.\rule{0pt}{1.5cm}\right#1}}
\newcommand{\makebig}[1]{\fontsize{3cm}{3cm}\selectfont\ensuremath{#1}}

\begin{tabular}{@{}*8c@{}}
  \toprule
  Symbol &
  Default size &
  \cmd{\big} &
  \cmd{\Big} &
  \cmd{\bigg} &
  \cmd{\Bigg} &
  \cmd{\left}\,/\,\cmd{\right} &
  \cmd{\fontsize} \\
  \midrule

  \verb|\}| &
  $\}$ &
  $\big\}$ &
  $\Big\}$ &
  $\bigg\}$ &
  $\Bigg\}$ &
  \maketall\} &
  \makebig\} \\

  \verb|\uparrow| &
  $\uparrow$ &
  $\big\uparrow$ &
  $\Big\uparrow$ &
  $\bigg\uparrow$ &
  $\Bigg\uparrow$ &
  \maketall\uparrow &
  \makebig\uparrow \\
  \bottomrule
\end{tabular}
\end{nonsymtable}

All variable-sized delimiters are defined (by the corresponding
\texttt{.tfm} file) in terms of up to five segments, as illustrated by
\vref{extensible-brace}.  The top, middle, and bottom segments
are of a fixed size.  The top-middle and middle-bottom segments (which
are constrained to be the same character) are repeated as many times
as necessary to achieve the desired height.

\begin{figure}[htbp]
\centering
\renewcommand{\arraystretch}{2}
\newcommand{\cmexchar}{\usefont{OMX}{cmex}{m}{n}\selectfont\char}
\newlength{\braceheight}
\setlength{\braceheight}{6.5\baselineskip}
\begin{tabular}{@{}ccl@{}}
  \multirow{5}*{$\left.\rule{0pt}{\braceheight}\right\} \longrightarrow$}
  & \cmexchar'71 & top \\
  & \cmexchar'76 & top-middle (extensible) \\
  & \cmexchar'75 & middle \\
  & \cmexchar'76 & middle-bottom (extensible) \\
  & \cmexchar'73 & bottom \\
  \\
\end{tabular}
\index{symbols>extensible}
\caption{Implementation of variable-sized delimiters}
\label{extensible-brace}
\end{figure}

  
\subsubsection{Reflecting and rotating existing symbols}

 
  \index{symbols>reversed|(}
  \index{symbols>rotated|(}
  \index{symbols>upside-down|(}
  \index{symbols>inverted|(}
  \index{reversed symbols|(}
  \index{rotated symbols|(}
  \index{upside-down symbols|(}
  \index{inverted symbols|(}
  
  
  \begin{texexample}{Create an Irony mark}{}
  \DeclareRobustCommand{\irony}{\textsuperscript{\reflectbox{?}}}
  \end{texexample}
  \DeclareRobustCommand{\irony}{\textsuperscript{\reflectbox{?}}}
  A common request on \ctt is for a reversed or rotated version of an
  existing symbol.  As a last resort, these effects can be achieved
  with the \pkgname{graphicx} (or \pkgname{graphics}) package's
  \cmd{\reflectbox} and \cmd{\rotatebox} macros.
  \newcommand{\definitedescription}{\rotatebox[origin=c]{180}{$\iota$}}
  For example, \verb|\textsuperscript{\reflectbox{?}}| produces an
  irony\index{irony mark=irony mark (\irony)} mark~(``\,\irony\,'';
  cf.~\url{http://en.wikipedia.org/wiki/Irony_mark}), and
  \verb|\rotatebox[origin=c]{180}{$\iota$}| produces the
  definite-description\index{definite-description operator
  (\definitedescription)}\index{iota, upside-down}
  operator~(``\rotatebox[origin=c]{180}{$\iota$}'').  
  
  The disadvantage
  of the \pkgname{graphicx}/\pkgname{graphics} approach is that not
  every \tex backend handles graphical transformations.\footnote{As an
  example, Xdvi\index{Xdvi} ignores both \cmd{\reflectbox} and
  \cmd{\rotatebox}.}  Far better is to find a suitable font that
  contains the desired symbol in the correct orientation.  For
  instance, if the PHON package is available, then
  \verb|\textit{\riota}| will yield a
  backend-independent~``\textit{\cmd{\riota}}''.
  Similarly,\label{page:such-that} \TIPA's
  \docAuxCommand{textrevepsilon}~(``\textrevepsilon'') or \WIPA's
  \docAuxCommand{textrevepsilon}~(``\textrevepsilon'') may be used to express the
  mathematical notion of ``such\index{such that} that'' in a cleaner
  manner than with \cmd{\reflectbox} or
  \cmd{\rotatebox}.\footnote{More common symbols for representing
  ``such\index{such that} that'' include ``\texttt{\textbar}'',
  ``\texttt{:}'', and ``\texttt{s.t.}''.}
  \index{symbols>reversed|)}
  \index{symbols>rotated|)}
  \index{symbols>upside-down|)}
  \index{symbols>inverted|)}
  \index{reversed symbols|)}
  \index{rotated symbols|)}
  \index{inverted symbols|)}



\begin{texexample}{Enlarging Delimiters}{ex:type1cm}
\newcommand{\makeBIG}[1]{\fontsize{1cm}{1cm}\selectfont\ensuremath{#1}}
  \makeBIG\>

\end{texexample}

\subsection{Where can I find the symbol for~\dots?}

\label{combining-symbols}

An easy way to find a symbol is to use \url{http://detexify.com}. This is a website service that you can use to identify a symbol by drawing it. The menu always adds the necessary package to a symbol after presenting possible matches to what you have drawn. But you also can click on the ``symbols'' button and enter the command. Here, too, the necessary package is added. 

If you can't find some symbol you're looking for in this document, there
are a few possible explanations:

\begin{itemize}
  \item The symbol isn't intuitively named.  As a few examples, the
  \IFS\ command to draw dice\index{dice} is
  ``\docAuxCommand{Cube}''; a plus sign with a circle around it
  (``exclusive or''\index{exclusive or} to computer engineers) is
  ``\docAuxCommand{oplus}''; and lightning bolts in fonts designed by German
  speakers may have ``blitz'' in their names as in the
  ULSY package.  The moral of the story is to be creative with
  synonyms when searching the index.

  \item The symbol is defined by some package that I overlooked (or
  deemed unimportant).  

  \item The symbol isn't defined in any package whatsoever.
\end{itemize}


  Even in the last case, all is not lost.  Sometimes, a symbol exists
  in a font, but there is no \latex{} binding for it.  For example,
  the \postscript \PSfont{Symbol} font contains a
  ``\Pisymbol{psy}{191}''\index{arrows} symbol, which may be useful
  for representing a carriage\index{carriage return} return, but there
  is no package (as far as I know) for accessing that symbol.  To
  produce an unnamed symbol, you need to switch to the font explicitly
  with \latexe's low-level font commands~\cite{fntguide} and use
  \tex's primitive \cmd{\char} command~\cite{Knuth:ct-a} to request a
  specific character number in the font.\footnote{\pkgname{pifont}
  defines a convenient \cmd{\Pisymbol} command for accessing symbols
  in \postscript\index{PostScript fonts} fonts by number.  For example,
  ``\cmd{\Pisymbol}\texttt{\string{psy\string}\string{191\string}}''
  produces ``\Pisymbol{psy}{191}''.}
   
  In fact, \cmd{\char} is not strictly necesssary; the character can
  often be entered symbolically.
 

  For example, the symbol for an impulse train or Tate-Shafarevich
  group (``{|\string\fontencoding{OT2}\string\selectfont SH|}'') is actually an
  uppercase \textit{sha} in the Cyrillic\index{alphabets>Cyrillic}
  alphabet.  (Cyrillic is supported by the OT2 \fntenc[OT2], for
  instance).  While a \textit{sha} can be defined numerically as
  
  it may be more intuitive to use the OT2 \fntenc[OT2]'s ``SH''
  ligature:
  
 
The \pkgname{slashed} package \citep{slashed}, although originally designed for
producing Feynman\index{Feynman slashed character notation}
slashed-character\idxboth{slashed}{letters} notation, in fact
facilitates the production of \emph{arbitrary} overlapped symbols.
\ifhaveslashed
  \newcommand{\rqm}{{\declareslashed{}{\text{-}}{0.04}{0}{I}\slashed{I}}}
  The default behavior is to overwrite a given character with ``$/$''.
  For example, \cmd{\slashed}\verb|{D}| produces ``$\slashed{D}$''.
  However, the \cmd{\declareslashed} command provides the flexibility
  to specify the mathematical context of the composite character
  (operator, relation, punctuation, etc., as will be discussed in
  \ref{math-spacing}), the overlapping symbol, horizontal and
  vertical adjustments in symbol-relative units, and the character to
  be overlapped.  Consider, for example, the symbol for reduced
  quadrupole moment~(``$\rqm$'').  This can be declared as follows:

\begin{verbatim}
    \newcommand{\rqm}{{%
      \declareslashed{}{\text{-}}{0.04}{0}{I}\slashed{I}}}
\end{verbatim}

  \noindent
  \newcommand{\curlyarg}{\texttt{\char`\{}$\cdot$\texttt{\char`\}}}%

  \cmd{\declareslashed}\curlyarg\curlyarg\curlyarg\curlyarg\verb|{I}|
  affects the meaning of all subsequent \cmd{\slashed}\verb|{I}|
  commands in the same scope.  The preceding definition of \docAuxCommand{rqm}
  therefore uses an extra set of curly braces to limit that scope to a
  single \cmd{\slashed}\verb|{I}|.  In addition, \docAuxCommand{rqm} uses
  \pkgname{amstext}'s \cmd{\text} macro
  (described~\vpageref[below]{text-macro}) to make
  \cmd{\declareslashed} use a text-mode hyphen~(``-'') instead of a
  math-mode minus sign~(``$-$'') and to ensure that the hyphen scales
  properly in size in subscripts and superscripts.
\fi  

See \pkgname{slashed}'s documentation (located in
\docfilename{slashed.sty} itself) for a detailed usage description of the
\cmd{\slashed} and \cmd{\declareslashed} commands.

Somewhat simpler than \pkgname{slashed} is the \pkgname{centernot}
package.  \pkgname{centernot} provides a single command,
\cmd{\centernot}, which, like \cmd{\not}, puts a slash over the
subsequent mathematical symbol.  However, instead of putting the slash
at a fixed location, \cmd{\centernot} centers the slash over its
argument.  \cmd{\centernot} might be used, for example, to create a
``does\index{does not imply} not imply'' symbol%

\ifhavecenternot
%   \begin{center}
%    \renewcommand{\arraystretch}{1.25}%
%    \begin{tabular}{cl}
%      $\not\Longrightarrow$       & \verb|\not\Longrightarrow| \\
%      \multicolumn{2}{c}{vs.} \\
%      $\centernot\Longrightarrow$ & \verb|\centernot\Longrightarrow| \\
%    \end{tabular}
%  \end{center}
\else
  .
\fi   
\seedocs{\pkgname{centernot}}


\subsection{How to make new symbols work in superscripts and subscripts}

\index{subscripts>new symbols used in|(}
\index{superscripts>new symbols used in|(}


To make composite symbols work properly within subscripts and
superscripts, you may need to use \tex's \cmd{\mathchoice} primitive.
\cmd{\mathchoice} evaluates one of four expressions, based on whether
the current math style is display, text, script, or scriptscript.
(See \TeXbook for a more complete description.)  For example, the
following \latex code---posted to \ctt by
\person{Torsten}{Bronger}---composes a sub/superscriptable
``\cmd{\topbot}'' symbol out of \docAuxCommand{top} and \docAuxCommand{bot} (``$\top$''
and ``$\bot$''):



\indexcommand{\displaystyle}%
\indexcommand{\textstyle}%
\indexcommand{\scriptstyle}%
\indexcommand{\scriptscriptstyle}%
\label{code:topbot}%

\begin{verbatim}
   \def\topbotatom#1{\hbox{\hbox to 0pt{$#1\bot$\hss}$#1\top$}}
   \newcommand*{\topbot}{\mathrel{\mathchoice{\topbotatom\displaystyle}
                                    {\topbotatom\textstyle}
                                    {\topbotatom\scriptstyle}
                                    {\topbotatom\scriptscriptstyle}}}
\end{verbatim}
\index{superscripts>new symbols used in|)}
\index{subscripts>new symbols used in|)}

\begin{texexample}{mathchoice}{ex:mathchoice}
\bgroup
\def\topbotatom#1{\hbox{\hbox to 0pt{$#1\bot$\hss}$#1\top$}}
   \def\topbot{\mathrel{\mathchoice{\topbotatom\displaystyle}
                                    {\topbotatom\textstyle}
                                    {\topbotatom\scriptstyle}
                                    {\topbotatom\scriptscriptstyle}}}
\[ a_{\topbot} + b^{\topbot} \]
\egroup
\end{texexample}


\subsection{Modifying \latex-generated symbols}

\index{dots (ellipses)|(}
\index{ellipses (dots)|(}
\index{dot symbols|(}
\index{symbols>dot|(}

Oftentimes, symbols composed in the \latexe source code can be
modified with minimal effort to produce useful variations.  For
example, \fontdefdtx composes the \docAuxCommand{ddots} symbol (see
\vref{dots}) out of three periods, raised~7\,pt., 4\,pt., and
1\,pt., respectively:

\begin{verbatim}
   \def\ddots{\mathinner{\mkern1mu\raise7\p@
       \vbox{\kern7\p@\hbox{.}}\mkern2mu
       \raise4\p@\hbox{.}\mkern2mu\raise\p@\hbox{.}\mkern1mu}}
\end{verbatim}

\noindent
\cmd{\p@} is a \latexe{} shortcut for ``\texttt{pt}'' or
``\texttt{1.0pt}''.  The remaining commands are defined in \TeXbook.
To\label{revddots} draw a version of \docAuxCommand{ddots} with the dots going
along the opposite diagonal, we merely have to reorder the
\verb|\raise7\p@|, \verb|\raise4\p@|, and \verb|\raise\p@|:

\begin{texexample}{revddots}{ex:revddots}
\makeatletter
   \def\revddots{\mathinner{\mkern1mu\raise\p@
      \vbox{\kern7\p@\hbox{.}}\mkern2mu
       \raise4\p@\hbox{.}\mkern2mu\raise7\p@\hbox{.}\mkern1mu}}
\makeatother

\[\revddots \]
\end{texexample}


    \makeatletter
      \def\revddots{\mathinner{\mkern1mu\raise\p@
        \vbox{\kern7\p@\hbox{.}}\mkern2mu
        \raise4\p@\hbox{.}\mkern2mu\raise7\p@\hbox{.}\mkern1mu}}
    \makeatother
\indexcommand[$\string\revddots$]{\revddots}

\noindent
\docAuxCommand{revddots} is essentially identical to the \MDOTS\
package's
\ifMDOTS
  \docAuxCommand{iddots}
\else
  \cmd{\iddots}
\fi
command or the \YH\ package's
%\ifYH
%  \docAuxCommand{adots}
%\else
  \cmd{\adots}
%\fi
command.
\index{symbols>dot|)}
\index{dot symbols|)}
\index{ellipses (dots)|)}
\index{dots (ellipses)|)}




\section{ASCII and Latin~1 quick reference}
\label{ascii-quickref}

\index{ASCII|(}

\vref{ascii-table} amalgamates data from various other tables in this
document into a convenient reference for \latexe typesetting of \texttt{ascii}
characters, i.e., the characters available on a typical U.S. computer
keyboard.  The first two columns list the character's \texttt{ascii} code in
decimal and hexadecimal.  The third column shows what the character
looks like.  The fourth column lists the \latexe command to typeset
the character as a text character.  And the fourth column lists the
\latexe command to typeset the character within a
\verb|\texttt{|$\ldots$\verb|}| command (or, more generally, when
\verb|\ttfamily| is in effect).


\index{ASCII|)}

\begin{nonsymtable}{\latexe ASCII Table}
  \index{ASCII>table}
  \label{ascii-table}
  ^^A Define an equivalent of \vdots that's the height of a "9".
  \newlength{\digitheight}
  \settoheight{\digitheight}{9}
  \newcommand{\digitvdots}{\raisebox{-1.5pt}[\digitheight]{$\vdots$}}

 ^^A Replace all glyphs in a row with vertical dots.
  \makeatletter
  \newcommand{\skipped}{%
    \settowidth{\@tempdima}{99} \makebox[\@tempdima]{\digitvdots} &
    \settowidth{\@tempdima}{99} \makebox[\@tempdima]{\digitvdots} &
    \digitvdots &
    \digitvdots &
    \digitvdots \\
  }
  \makeatother

  ^^A Typesetting a symbol by prefixing it with a "\".
  \newcommand{\bscommand}[1]{#1 & \cmd{#1} & \cmd{#1}}

  \begin{tabular}[t]{@{}*2{>{\ttfamily}r}c*2{>{\ttfamily}l}l@{}} \\ \toprule
    \multicolumn{1}{@{}c}{Dec} &
    \multicolumn{1}{c}{Hex} &
    \multicolumn{1}{c}{Char} &
    \multicolumn{1}{c}{Body text} &
    \multicolumn{1}{c@{}}{\ttfamily\string\texttt} \\ \midrule

    33 & 21 & ! & ! & ! \\
    34 & 22 & {\fontencoding{T1}\selectfont\textquotedbl} &
      \string\textquotedbl & " \\      ^^A Not available in OT1
    35 & 23 & \bscommand{\#} \\
    36 & 24 & \bscommand{\$} \\
    37 & 25 & \bscommand{\%} \\
    38 & 26 & \bscommand{\&} \\
    39 & 27 & ' & ' & ' \\
    40 & 28 & ( & ( & ( \\
    41 & 29 & ) & ) & ) \\
    42 & 2A & * & * & * \\
    43 & 2B & + & + & + \\
    44 & 2C & , & , & , \\
    45 & 2D & - & - & - \\
    46 & 2E & . & . & . \\
    47 & 2F & / & / & / \\
    48 & 30 & 0 & 0 & 0 \\
    49 & 31 & 1 & 1 & 1 \\
    50 & 32 & 2 & 2 & 2 \\
    \skipped
    57 & 39 & 9 & 9 & 9 \\
    58 & 3A & : & : & : \\
    59 & 3B & ; & ; & ; \\
    60 & 3C & \textless & \docAuxCommand{textless} & < \\       ^^A Or $<$
    61 & 3D & = & = & = \\ \bottomrule
  \end{tabular}
  \hfil
  \begin{tabular}[t]{@{}*2{>{\ttfamily}r}c*2{>{\ttfamily}l}l@{}} \\ \toprule
    \multicolumn{1}{@{}c}{Dec} &
    \multicolumn{1}{c}{Hex} &
    \multicolumn{1}{c}{Char} &
    \multicolumn{1}{c}{Body text} &
    \multicolumn{1}{c@{}}{\ttfamily\string\texttt} \\ \midrule

    62 & 3E & \textgreater & \docAuxCommand{textgreater} & > \\   
    63 & 3F & ? & ? & ? \\
    64 & 40 & @ & @ & @ \\
    65 & 41 & A & A & A \\
    66 & 42 & B & B & B \\
    67 & 43 & C & C & C \\
    \skipped
    90 & 5A & Z & Z & Z \\
    91 & 5B & [ & [ & [ \\
    92 & 5C & \textbackslash & \docAuxCommand{textbackslash} &
      \verb|\char`\\| \\   ^^A \textbackslash works in non-OT1
    93 & 5D & ] & ] & ] \\
    94 & 5E & \^{} & \verb|\^{}| & \verb|\^{}| \\   ^^A Or \textasciicircum
    95 & 5F & \_ & \verb|\_| & \verb|\char`\_| \\   ^^A \_ works in non-OT1
    96 & 60 & ` & ` & ` \\
    97 & 61 & a & a & a \\
    98 & 62 & b & b & b \\
    99 & 63 & c & c & c \\
    \skipped
   122 & 7A & z & z & z \\
   123 & 7B & \{ & \verb|\{| & \verb|\char`\{| \\   
   124 & 7C & \textbar & \docAuxCommand{textbar} & \textbar \\    
   125 & 7D & \} & \verb|\}| & \verb|\char`\}| \\   
   126 & 7E & \~{} & \verb|\~{}| & \verb|\~{}| \\   
   \\
   \bottomrule
  \end{tabular}
\end{nonsymtable}

The following are some additional notes about the contents of
\ref{ascii-table}:

\begin{itemize}
  \item
  ``\indexcommand[\string\encone{\string\textquotedbl}]{\textquotedbl}{\encone{\textquotedbl}}''
  is not available in the OT1 \fntenc[OT1].

  \item \ref{ascii-table} shows a close quote for character~39 for
    consistency with the open quote shown for character~96.  A
    straight quote can be typeset using \docAuxCommand{textquotesingle}
    (cf.~\ref{tc-misc}).

  \item
  The\label{upside-down}\index{symbols>upside-down|(}\index{upside-down
  symbols|(} characters ``\texttt{<}'', ``\texttt{>}'', and
  ``\texttt{\textbar}'' do work as expected in math mode, although they
  produce, respectively, ``<'', ``>'', and ``\textbar'' in text mode when
  using the OT1 \fntenc[OT1].\footnote{Donald\index{Knuth, Donald E.}
  Knuth didn't think such symbols were important outside of
  mathematics so he omitted them from his text fonts.} The following
  are some alternatives for typesetting ``\textless'',
  ``\textgreater'', and ``\textbar'':

  \begin{itemize}
    \item Specify a document \fntenc{} other than OT1 (as
    described~\vpageref[above]{altenc}).

    \item Use the appropriate symbol commands from
    \vref{text-predef}, viz.~\docAuxCommand{textless},
    \docAuxCommand{textgreater}, and \docAuxCommand{textbar}.

    \item Enter the symbols in math mode instead of text mode,
    i.e.,~\verb+$<$+, \verb+$>$+, and \verb+$|$+.
  \end{itemize}

  \noindent
  Note that for typesetting metavariables many people prefer
  \docAuxCommand{textlangle} and \docAuxCommand{textrangle} to \docAuxCommand{textless} and
  \docAuxCommand{textgreater}; i.e., ``\meta{filename}'' instead of
  ``$<$\textit{filename}$>$''.\index{symbols>upside-down|)}\index{upside-down
  symbols|)}

  \item Although ``\texttt{/}'' does not require any special
  treatment, \latex additionally defines a \docAuxCommand{slash} command which
  outputs the same glyph but permits a line~break afterwards.  That
  is, ``\texttt{increase/decrease}'' is always typeset as a single
  entity while ``\verb|increase\slash{}decrease|'' may be typeset with
  ``increase/'' on one line and ``decrease'' on the next.

  \item \label{page:tildes} \index{tilde|(} \docAuxCommand{textasciicircum}
  can be used instead of 
 % \cmdI[\string\^{}]{\^{}}\verb|{}|, 
  and
  \docAuxCommand{textasciitilde} can be used instead of
%  \cmdI[\string\~{}]{\~{}}\verb|{}|.  Note that
  \docAuxCommand{textasciitilde} and 
  %\cmdI[\string\~{}]{\~{}}\verb|{}|
  produce raised, diacritic tildes.  ``Text''
  (i.e.,~vertically\index{tilde>vertically centered} centered)
  tildes can be generated with either the math-mode \docAuxCommand{sim}
  command (shown in \vref{rel}), which produces a somewhat wide
  ``$\sim$'', or the \TC\ package's \docAuxCommand{texttildelow} (shown in
  \vref{tc-misc}), which produces a vertically centered
  ``{\fontfamily{ptm}\selectfont\texttildelow}'' in most fonts but a
  baseline-oriented ``\texttildelow'' in \PSfont{Computer Modern},
  \TX, \PX, and various other fonts originating from the
  \tex\ world.  If your goal is to typeset tildes in URLs or Unix
  filenames, your best bet is to use the \pkgname{url} package,
  which has a number of nice features such as proper line-breaking
  of such names.\index{tilde|)}

  \item The various \cmd{\char} commands within \verb|\texttt| are
  necessary only in the OT1 \fntenc[OT1].  In other encodings
  (e.g.,~T1)\index{font encodings>T1}, commands such as 
%  \cmdIp{\{},
  %\cmdIp{\}}, \
  %docAuxCommand{_}, 
  and \docAuxCommand{textbackslash} all work properly.

  \item The code\index{code page 437} page~437 (IBM~PC\index{IBM PC})
  version of \texttt{ascii} characters~1 to~31 can be typeset
  using the \ASCII\ package.
\ifASCII
  See \vref{ibm-ascii}.
\fi

  \item To replace~``\verb|`|'' and~``\verb|'|'' with the more
  computer-like (and more visibly distinct) ``\texttt{\char18}''
  and~``\texttt{\char13}'' within a \texttt{verbatim} environment,
  use the \pkgname{upquote} package.  Outside of \texttt{verbatim},
  you can use \cmd{\char}\texttt{18} and \cmd{\char}\texttt{13} to
  get the modified quote characters.  (The former is actually a
  grave accent.)
\end{itemize}





\subsection{Unicode characters}
\label{unicode-chars}

\index{Unicode|(}

\href{http://www.unicode.org/}{Unicode} is a ``universal character
set''---a standard for encoding (i.e.,~assigning unique numbers to)
the symbols appearing in many of the world's languages.  While \texttt{ascii}
can represent 128 symbols and Latin~1 can represent 256 symbols,
Unicode can represent an astonishing 1,114,112 symbols.

Because \tex and \latex{} predate the Unicode standard and Unicode
fonts by almost a decade, support for Unicode has had to be added to
the base \tex{} and \latex{} systems.  Note first that \latex{}
distinguishes between \emph{input} encoding---the characters used in
the \texttt{.tex} file---and \emph{output} encoding---the characters
that appear in the generated \texttt{.dvi}, \texttt{.pdf}, etc.\ file.
For a discusiion on Unicode for Mathematics see \citep{beetona}.

\begin{texexample}{How to add symbols}{unicodesymbols}
\ifxetex
  \newfontfamily{\codetwothousand}{code2000.ttf}
  \codetwothousand\char"1F050 \char"2603\char"2617
  \symbol{9825}
  \newfontfamily{\codetwothousandone}{code2001.ttf}
  \newfontfamily{\symbola}{symbola.ttf}
  {\codetwothousand \symbol{9742} \symbol{9743}
    Katakana (片仮名, カタカナ)
   \codetwothousandone \symbol{57508}
   \symbola \symbol{9816}
   
  }
\else
   Compile the document with XeTeX to see the example
\fi
\end{texexample}

\subsubsection{Inputting Unicode characters}

To include Unicode characters in a \texttt{.tex} file, load the
\pkgname{ucs} package and load the \pkgname{inputenc} package with the
\optname{inputenc}{utf8x} (``\utfviii extended'')
option.\footnote{\utfviii is the 8-bit Unicode Transformation Format,
  a popular mechanism for representing Unicode symbol numbers as
  sequences of one to four bytes.}  These packages enable \latex{} to
translate \utfviii sequences to \latex{} commands, which are
subsequently processed as normal.  For example, the \utfviii text
``\texttt{Copyright~\textcopyright\ \the\year}''---``\texttt{\textcopyright}''
is not an \texttt{ascii} character and therefore cannot be input directly
without packages such as \pkgname{ucs}/\pkgname{inputenc}---is
converted internally by \pkgname{inputenc} to ``\texttt{Copyright}
\verb+\textcopyright{}+ \texttt{\the\year}'' and therefore typeset as
``Copyright~\textcopyright\ \the\year''.

The \pkgname{ucs}\slash\pkgname{inputenc} combination supports only a
tiny subset of Unicode's million-plus symbols.  Additional symbols can
be added manually using the \cmd{\DeclareUnicodeCharacter} command.
\cmd{\DeclareUnicodeCharacter} takes two arguments: a Unicode number
and a \latex{} command to execute when the corresponding Unicode
character is encountered in the input.  For example, the Unicode
character ``degree celsius''~(``\,\textcelsius\,'') appears at
character position U+2103.\footnote{The Unicode convention is to
  express character positions as ``U+\meta{hexadecimal number}''.}
However, ``\,\texttt{\textcelsius}\,'' is not one of the characters
that \pkgname{ucs} and \pkgname{inputenc} recognize.  

The following
document shows how to use \cmd{\DeclareUnicodeCharacter} to tell
\latex{} that the ``\,\texttt{\textcelsius}\,'' character should be
treated as a synonym for \docAuxCommand{textcelsius}:

\begin{verbatim}
   \documentclass{article}
   \usepackage{ucs}
   \usepackage[utf8x]{inputenc}
   \usepackage{textcomp}

   \DeclareUnicodeCharacter{"2103}{\textcelsius} % Enable direct input of U+2103.
\end{verbatim}
\noindent
\verb|   \begin{document}| \\
\verb|   |\texttt{It was a balmy 21\textcelsius.} \\
\verb|   \end{document}|

\medskip

\noindent
which produces

\begin{quotation}
  It was a balmy 21\textcelsius.
\end{quotation}

\seedocs{\pkgname{ucs}} and for descriptions of the various options that control \pkgname{ucs}'s behavior.


\subsection{Outputting Unicode characters}

Orthogonal to the ability to include Unicode characters in a
\latex\ input file is the ability to include a given Unicode character
in the corresponding output file.  By far the easiest approach is to
use \xelatex instead of pdf\LaTeX\index{pdfLaTeX=pdf\LaTeX} or
ordinary \latex.  \xelatex handles Unicode input and output natively
and can utilize system fonts directly without having to expose them
via \texttt{.tfm}, \texttt{.fd}, and other such files.  To output a
Unicode character, a \xelatex document can either include that
character directly as \utfviii text or use \tex's \cmd{\char}
primitive, which \xelatex extends to accept numbers larger than~255.

\DeclareRobustCommand{\trafficsign}{\includegraphics[height=10pt]{./images/traffic-sign-01.png}
}
Suppose we need to declare a traffic sign \trafficsign and for which we have some images ready.


\newfontfamily{\codetwothousand}{code2000.ttf}
\newfontfamily{\codetwothousandone}{code2001.ttf}
  \newfontfamily{\symbola}{symbola.ttf}
  
\DeclareRobustCommand{\versicle}{%
  \raisebox{-2.2bp}{\includegraphics{./images/versicle.jpg}}\kern-1pt}
\DeclareRobustCommand{\response}{%
  \raisebox{-1.2bp}{\includegraphics{./images/response.jpg}}\kern-1pt}
\newcommand{\versicleIDX}{\index{versicle=versicle (\versicle)}}
\newcommand{\responseIDX}{\index{response=response (\response)}}

Suppose we want to output the symbols for
versicle\versicleIDX~(``\versicle'') and
response\responseIDX~(``\response'') in a document.  The Unicode
charts list ``versicle\versicleIDX'' at position~U+2123 ({\codetwothousand\char"2123}) and
``response\responseIDX'' at position~U+211F ({\codetwothousand\char"211F}).  We therefore need to
install a font that contains those characters at their proper
positions.  One such font that is freely available from CTAN\idxCTAN{}
is Junicode Regular (\docfilename{Junicode-Regular.ttf}) from the
\pkgname{junicode} package.  

The \pkgname{fontspec} package makes it
easy for a \xelatex or \lualatex document to utilize a system font.  The following
example defines a \texttt{\string\textjuni} command that uses
\pkgname{fontspec} to typeset its argument in Junicode Regular:

\begin{verbatim}
   \documentclass{article}
   \usepackage{fontspec}

   \newcommand{\textjuni}[1]{{\fontspec{Junicode-Regular}#1}}

   \begin{document}
   We use ``\textjuni{\char"2123}'' for a versicle
   and ``\textjuni{\char"211F}'' for a response.
   \end{document}
\end{verbatim}

\noindent
which produces

\begin{quotation}
  We use ``\versicle'' for a versicle\versicleIDX\ and ``\response''
  for a response\responseIDX.
\end{quotation}

\noindent
(Typesetting the entire document in Junicode Regular would be even
easier.  \seedocs{\pkgname{fontspec}} regarding font selection.)  Note
how the preceding example uses \cmd{\char} to specify a Unicode
character by number.  The double quotes before the number indicate
that the number is represented in hexadecimal instead of decimal.

\index{Unicode|)}

\section{XeLaTeX and fontspec}

\index{maths>fontspec}
The best option so far for math fonts using XeLaTeX and \pkgname{fontspec} is to use the option |no-math|. When typesetting this document for example there were numerous problems with accents (I lost the |ring| accent, until I used this option. Unicode math fonts are not available in large numbers.



% Because the Math Alphabets table is a bit different from the symbol
% tables in this document we start it on its own page to emphasize it
% and to include enough room for some of the table notes.
\clearpage

\begin{symtable}{Math Alphabets}
\idxboth{math}{alphabets}
\label{alphabets}
\begin{tabular}{@{}*3l@{}}
\toprule
Font sample & Generating command & Required package           \\
\midrule
\Wf\mathrm{ABCabc123}    & \textit{none}                      \\
\Ww\textit\mathit{ABCabc123}    & \textit{none}               \\
\Wf\mathnormal{ABCabc123}& \textit{none}                      \\
|\Ww\CMcal\mathcal{ABC}|   & \textit{none}                      \\

\ifx\mathscr\undefined\else
\Wf\mathscr{ABC}         & \pkgname{mathrsfs} \\
\multicolumn{1}{r@{}}{\emph{or}}
        &\verb|\mathcal{ABC}|
                         & \pkgname{calrsfs} \\
\fi
%
%\ifEU
%\Wf\mathcal{ABC}         & \pkgname{euscript} with the
%                           \optname{euscript}{mathcal} option \\
%\multicolumn{1}{r@{}}{\emph{or}}
%        &\verb|\mathscr{ABC}|
%                         & \pkgname{euscript} with the
%                           \optname{euscript}{mathscr} option \\
%\fi

\ifx\mathpzc\undefined\else
\Wf\mathpzc{ABCdef123}   & \textit{none}; manually defined$^*$    \\
\fi

\ifx\mathbb\undefined\else
\Wf\mathbb{ABC}          & \pkgname{amsfonts},%
                           \ifx\MSYMmathbb\undefined\else$^\S$~\fi
                           \pkgname{amssymb}, \pkgname{txfonts}, or
                           \pkgname{pxfonts} \\
\fi

\ifx\varmathbb\undefined\else
\Wf\varmathbb{ABC}       & \pkgname{txfonts} or \pkgname{pxfonts} \\
\fi

%\ifx\BBmathbb\undefined\else
%\Ww\BBmathbb\mathbb{ABCdef123}
%                         & \pkgname{bbold} or \pkgname{mathbbol}$^\dag$  \\
%\fi
%
%\ifx\MBBmathbb\undefined\else
%\Ww\MBBmathbb\mathbb{ABCdef123}
%                         & \pkgname{mbboard}$^\dag$              \\
%\fi

%\ifx\mathbbm\undefined\else
%\Wf\mathbbm{ABCdef12}    & \pkgname{bbm}                         \\
%\Wf\mathbbmss{ABCdef12}  & \pkgname{bbm}                         \\
%\Wf\mathbbmtt{ABCdef12}  & \pkgname{bbm}                         \\
%\fi

\ifx\mathds\undefined\else
\Wf\mathds{ABC1}         & \pkgname{dsfont}                      \\
\Ww\mathdsss\mathds{ABC1}
                         & \pkgname{dsfont} with the
                           \optname{dsfont}{sans} option         \\
\fi

\ifx\symA\undefined\else
\symA\symB\symC & \docAuxCommand{symA}\docAuxCommand{symB}\docAuxCommand{symC}
                         & \pkgname{china2e}$^\ddag$             \\
\fi

\ifx\mathfrak\undefined\else
\Wf\mathfrak{ABCdef123}  & \pkgname{eufrak}                      \\
\fi

\ifx\textfrak\undefined\else
\Wf\textfrak{ABCdef123}  & \pkgname{yfonts}$^\P$                 \\
\Wf\textswab{ABCdef123}  & \pkgname{yfonts}$^\P$                 \\
\Wf\textgoth{ABCdef123}  & \pkgname{yfonts}$^\P$                 \\
\fi
\bottomrule
\end{tabular}
\end{symtable}
\unskip



\begin{center}
\ifx\mathpzc\undefined\else
\bigskip
\begin{tablenote}[*]
  Put ``\verb|\DeclareMathAlphabet{\mathpzc}{OT1}{pzc}{m}{it}|'' in your
  document's preamble to make \verb|\mathpzc| typeset its argument in
  \PSfont{Zapf Chancery}.
\ifx\textcalligra\undefined\else
  As a similar trick, you can typeset the \PSfont{Calligra} font's
  script ``{\Large\textcalligra{r}\,}'' (or other calligraphic symbols)
  in math mode by loading the \pkgname{calligra} package and putting
  ``\verb|\DeclareMathAlphabet{\mathcalligra}{T1}{calligra}{m}{n}|''
  in your document's preamble to make \verb|\mathcalligra| typeset its
  argument in the \PSfont{Calligra} font.  (You may also want to
  specify
  ``\verb|\DeclareFontShape{T1}{calligra}{m}{n}{<->s*[2.2]callig15}{}|''
  to set \PSfont{Calligra} at 2.2~times its design size for a better
  blend with typical body fonts.)
\fi
\end{tablenote}
\fi

\ifx\BBmathbb\undefined\else
\bigskip
\begin{tablenote}[\dag]
  The \pkgname{mathbbol} package defines some additional blackboard bold
  characters: parentheses, square brackets, angle brackets, and---if
  the \optname{mathbbol}{bbgreekl} option is passed to
  \pkgname{mathbbol}---Greek\index{Greek>blackboard bold} letters.  For
  instance,
%  ``$\BBmathbb{\char`<\char`[\char`(\char"0B\char"0C\char"0D\char`)\char`]\char`>}$''
%  is produced by
%  ``\cmd{\mathbb}\verb|{|\docAuxCommand{Langle}\linebreak[1]%
%  \docAuxCommand{Lbrack}\linebreak[1]\docAuxCommand{Lparen}\linebreak[1]%
%  \docAuxCommand{bbalpha}\linebreak[1]\docAuxCommand{bbbeta}\linebreak[1]%
%  \docAuxCommand{bbgamma}\linebreak[1]\docAuxCommand{Rparen}\linebreak[1]%
%  \docAuxCommand{Rbrack}\linebreak[1]\docAuxCommand{Rangle}\verb|}|''.

  \ifx\MBBmathbb\undefined
    \pkgname{mbboard} extends the blackboard bold symbol set
    significantly further.  It supports not only the
    Greek\index{Greek>blackboard bold}\index{alphabets>Greek}
    alphabet---including ``Greek-like'' symbols such as
    \cmd{\bbnabla}---but also \emph{all} punctuation marks, various
    currency\idxboth{currency}{symbols}\idxboth{monetary}{symbols}
    symbols such as \cmd{\bbdollar} and \cmd{\bbeuro},\index{euro
    signs>blackboard bold} and the
    Hebrew\index{Hebrew}\index{alphabets>Hebrew} alphabet.
  \else
    \pkgname{mbboard} extends the blackboard bold symbol set
    significantly further.  It supports not only the
    Greek\index{Greek>blackboard bold}\index{alphabets>Greek}
    alphabet---including ``Greek-like'' symbols such as
    \docAuxCommand{bbnabla}~(``\bbnabla'')---but also \emph{all} punctuation
    marks, various
    currency\idxboth{currency}{symbols}\idxboth{monetary}{symbols}
    symbols such as \docAuxCommand{bbdollar}~(``\bbdollar'') and
    \docAuxCommand{bbeuro}~(``\bbeuro''),\index{euro signs>blackboard bold}
    and the Hebrew\index{Hebrew}\index{alphabets>Hebrew}
    alphabet~(e.g.,~``\docAuxCommand{bbfinalnun}\linebreak[1]\docAuxCommand{bbyod}%
    \linebreak[1]\docAuxCommand{bbqof}\linebreak[1]\docAuxCommand{bbpe}''~$\rightarrow$
    ``\bbfinalnun\bbyod\bbqof\bbpe'').
  \fi    t
\end{tablenote}
\fi

\ifx\symA\undefined\else
\bigskip
\begin{tablenote}[\ddag]
  The \verb|\sym|\dots\ commands provided by the \pkgname{package} are
  actually text-mode commands.  They are included in \ref{alphabets}
  because they resemble the blackboard-bold symbols that appear in the
  rest of the table.  In addition to the 26 letters of the English
  alphabet, \CHINA\ provides three umlauted%
  \index{accents>diaeresis=di\ae{}resis (\blackacchack\")}  % 
  blackboard-bold letters:
  \docAuxCommand{symAE}~(``\symAE''), \docAuxCommand{symOE}~(``\symOE''), and
  \docAuxCommand{symUE}~(``\symUE'').  Note that \CHINA\ does provide
  math-mode commands for the most common number-set symbols.  These
  are presented in \vref{china-numsets}.
\end{tablenote}
\fi

\ifx\textfrak\undefined\else
\bigskip
\begin{tablenote}[\P]
  As their \verb|\text|\dots{} names imply, the fonts provided by the
  \pkgname{yfonts} package are actually text fonts.  They are
  included in \ref{alphabets} because they are frequently used
  in a mathematical context.
\end{tablenote}
\fi

\ifx\MSYMmathbb\undefined\else
\bigskip
\begin{tablenote}[\S]
  An older (i.e.,~prior to~1991) version of the \AMS's fonts rendered
  $\mathbb{C}$, $\mathbb{N}$, $\mathbb{R}$, $\mathbb{S}$,
  and~$\mathbb{Z}$ as $\MSYMmathbb{C}$, $\MSYMmathbb{N}$,
  $\MSYMmathbb{R}$, $\MSYMmathbb{S}$, and~$\MSYMmathbb{Z}$.  As some
  people prefer the older glyphs---much to the \AMS's surprise---and
  because those glyphs fail to build under modern versions of
  \metafont, \person{Berthold}{Horn} uploaded \postscript fonts for
  the older blackboard-bold glyphs to CTAN\idxCTAN{}, to the
  \texttt{fonts/msym10} directory.  As of this writing, however, there
  are no \latexE packages for utilizing the now-obsolete glyphs.
\end{tablenote}
\fi
\end{center}


\idxbothend{mathematical}{symbols}


\bgroup
\renewcommand\arraystretch{1.4}
\newcommand\leg[1]{{\tiny\tt\char92#1}}
\newcommand\sho[1]{{\large #1}}
\begin{tabular}{|*{10}{c}|} \hline
\leg{Pickup} &
\leg{Letter} & 
\leg{Mobilefone} &
\leg{Telefon} &
\leg{fax} &
\leg{FAX} &
\leg{Faxmachine} &
\leg{Email} &
\leg{Lightning} &
\leg{EmailCT} \\
\sho{\Pickup} &
\sho{\Letter} &
\sho{\Mobilefone} &
\sho{\Telefon} &
\sho{\fax} &
\sho{\FAX} &
\sho{\Faxmachine} &
\sho{\Email} &
\sho{\Lightning} &
\sho{\EmailCT} \\
\hline
\end{tabular}

\begin{tabular}{|*{8}{c}|} \hline
\leg{Beam} &
\leg{Bearing} &
\leg{LooseBearing} &
\leg{FixedBearing} &
\leg{LeftTorque} &
\leg{RightTorque} &
\leg{Lineload} &
\leg{MVArrowDown} \\
\sho{\Beam} &
\sho{\Bearing} &
\sho{\LooseBearing} &
\sho{\FixedBearing} &
\sho{\LeftTorque} &
\sho{\RightTorque} &
\sho{\Lineload} &
\sho{\MVArrowDown} \\
\hline
\leg{OktoSteel} &
\leg{HexaSteel} &
\leg{SquareSteel} & 
\leg{RectSteel} &
\leg{CircSteel} &
\leg{SquarePipe} &
\leg{RectPipe} &
\leg{CircPipe}
\\
\sho{\OktoSteel} &
\sho{\HexaSteel} &
\sho{\SquareSteel} &
\sho{\RectSteel} &
\sho{\CircSteel} &
\sho{\SquarePipe} &
\sho{\RectPipe} &
\sho{\CircPipe}
\\ \hline
\leg{LSteel} &
\leg{RoundedLSteel} &
\leg{TSteel} &
\leg{RoundedTSteel} &
\leg{TTsteel} &
\leg{RoundedTTSteel} &
\leg{FlatSteel} &
\leg{Valve}
\\
\sho{\LSteel} &
\sho{\RoundedLSteel} &
\sho{\TSteel} &
\sho{\RoundedTSteel} &
\sho{\TTSteel} &
\sho{\RoundedTTSteel} &
\sho{\FlatSteel} &
\sho{\Valve}
\\ \hline
\end{tabular}

\subsection{Information}

\begin{tabular}{|*{8}{c}|} \hline
\leg{Industry} &
\leg{Coffeecup} &
\leg{LeftScissors} &
\leg{CuttingLine} &
\leg{RightScissors} &
\leg{Football} &
\leg{Bicycle} & \\
\sho{\Industry} &
\sho{\Coffeecup} &
\sho{\LeftScissors} &
\sho{\CuttingLine} &
\sho{\RightScissors} &
\sho{\Football} &
\sho{\Bicycle} & \\
\hline
\leg{Info} &
\leg{ClockLogo} &
\leg{CutRight} &
\leg{CutLineine} &
\leg{CutLeft} &
\leg{Wheelchair} &
\leg{Gentsroom} &
\leg{Ladiesroom} \\
\sho{\Info} &
\sho{\ClockLogo} &
\sho{\CutRight} &
\sho{\CutLine} &
\sho{\CutLeft} &
\sho{\Wheelchair} &
\sho{\Gentsroom} &
\sho{\Ladiesroom} \\
\hline
\leg{Checkedbox} &
\leg{CrossedBox} &
\leg{HollowBox} &
\leg{PointingHand} &
\leg{WritingHand} &
\leg{MineSign} &
\leg{Recycling} &
\leg{PackingWaste} \\
\sho{\Checkedbox} &
\sho{\CrossedBox} &
\sho{\HollowBox} &
\sho{\PointingHand} &
\sho{\WritingHand} &
\sho{\MineSign} &
\sho{\Recycling} &
\sho{\PackingWaste} \\
\hline
\end{tabular}

\subsection{Laundry}

\begin{tabular}{|*{8}{c}|} \hline
\leg{WashCotton} &
\leg{WashSynthetics} &
\leg{WashWool} &
\leg{HandWash} &
\leg{NoWash} &
\leg{Tumbler} &
\leg{NoTumbler} &
\leg{NoChemicalCleaning} \\
\sho{\WashCotton} &
\sho{\WashSynthetics} &
\sho{\WashWool} &
\sho{\HandWash} &
\sho{\NoWash} &
\sho{\Tumbler} &
\sho{\NoTumbler} &
\sho{\NoChemicalCleaning} \\
\hline
\leg{Bleech} &
\leg{NoBleech} &
\leg{CleaningA} &
\leg{CleaningP} &
\leg{CleaningPP} &
\leg{CleaningF} &
\leg{CleaningFF} & \\
\sho{\Bleech} &
\sho{\NoBleech} &
\sho{\CleaningA} &
\sho{\CleaningP} &
\sho{\CleaningPP} &
\sho{\CleaningF} &
\sho{\CleaningFF} & \\
\hline
\leg{IroningI} &
\leg{IroningII} &
\leg{IroningIII} &
\leg{NoIroning} &
\leg{AtNinetyFive} &
\leg{ShortNinetyFive} &
\leg{AtSixty} &
\leg{ShortSixty} \\
\sho{\IroningI} &
\sho{\IroningII} &
\sho{\IroningIII} &
\sho{\NoIroning} &
\sho{\AtNinetyFive} &
\sho{\ShortNinetyFive} &
\sho{\AtSixty} &
\sho{\ShortSixty} \\
\hline
\leg{ShortFifty} &
\leg{AtForty} &
\leg{ShortForty} &
\leg{SpecialForty} &
\leg{ShortThirty} &&& \\
\sho{\ShortFifty} &
\sho{\AtForty} &
\sho{\ShortForty} &
\sho{\SpecialForty} &
\sho{\ShortThirty} &&& \\
\hline
\end{tabular}

\subsection{Currency}

\begin{tabular}{|*{11}{c}|} \hline
\leg{EUR} &
\leg{EURdig} &
\leg{EURhv} &
\leg{EURcr} &
\leg{EURtm} &
\leg{Ecommerce} &
\leg{Shilling} &
\leg{Denarius} &
\leg{Pfund} &
\leg{EyesDollar} &
\leg{Florin} \\
 &
\leg{EurDig} &
\leg{EurHv} &
\leg{EurCr} &
\leg{EurTm} &
\leg{EstimatedSign} &
 &
\leg{Deleatur} &
 &
 &
 \\
\sho{\EUR} &
\sho{\EurDig} &
\sho{\EurHv} &
\sho{\EurCr} &
\sho{\EurTm} &
\sho{\EstimatedSign} &
\sho{\Shilling} &
\sho{\Deleatur} &
\sho{\Pfund} &
\sho{\EyesDollar} &
\sho{\Florin} \\
\hline
\end{tabular}
\label{currencysymbols} 

\subsection{Safety}

\begin{tabular}{|*{8}{c}|} \hline
\leg{Stopsign} &
\leg{CESign} &
\leg{Estatically} &
\leg{Explosionsafe} &
\leg{Laserbeam} &
\leg{Biohazard} &
\leg{Radioactivity} &
\leg{BSEFree} \\
\sho{\Stopsign} &
\sho{\CESign} &
\sho{\Estatically} &
\sho{\Explosionsafe} &
\sho{\Laserbeam} &
\sho{\Biohazard} &
\sho{\Radioactivity} &
\sho{\BSEFree} \\
\hline
\end{tabular}

\subsection{Navigation}

\begin{tabular}{|*{10}{c}|} \hline
\leg{RewindToIndex} &
\leg{RewindToStart} &
\leg{Rewind} &
\leg{Forward} &
\leg{ForwardToEnd} &
\leg{ForwardToIndex} &
\leg{MoveUp} &
\leg{MoveDown} &
\leg{ToTop} &
\leg{ToBottom} \\
\sho{\RewindToIndex} &
\sho{\RewindToStart} &
\sho{\Rewind} &
\sho{\Forward} &
\sho{\ForwardToEnd} &
\sho{\ForwardToIndex} &
\sho{\MoveUp} &
\sho{\MoveDown} &
\sho{\ToTop} &
\sho{\ToBottom} \\
\hline
\end{tabular}

\subsection{Computers}

\begin{tabular}{|*{6}{c}|} \hline
\leg{ComputerMouse} &
\leg{SerialInterface} &
\leg{Keyboard} &
\leg{SerialPort} &
\leg{ParallelPort} &
\leg{Printer} \\
\sho{\ComputerMouse} &
\sho{\SerialInterface} &
\sho{\Keyboard} &
\sho{\SerialPort} &
\sho{\ParallelPort} &
\sho{\Printer} \\
\hline
\end{tabular}

\subsection{Numbers}

\begin{tabular}{|*{10}{c}|} \hline
\leg{MVZero} &
\leg{MVOne} &
\leg{MVTwo} &
\leg{MVThree} &
\leg{MVFour} &
\leg{MVFive} &
\leg{MVSix} &
\leg{MVSeven} &
\leg{MVEight} &
\leg{MVNine} \\
\sho{\MVZero} &
\sho{\MVOne} &
\sho{\MVTwo} &
\sho{\MVThree} &
\sho{\MVFour} &
\sho{\MVFive} &
\sho{\MVSix} &
\sho{\MVSeven} &
\sho{\MVEight} &
\sho{\MVNine} \\
\hline
\end{tabular}

\subsection{Maths}

\begin{tabular}{|*{8}{c}|} \hline
\leg{MVLeftBracket} &
\leg{MVRightBracket} &
\leg{MVComma} &
\leg{MVPeriod} &
\leg{MVMinus} &
\leg{MVPlus} &
\leg{MVDivision} &
\leg{MVMultiplication} \\
\sho{\MVLeftBracket} &
\sho{\MVRightBracket} &
\sho{\MVComma} &
\sho{\MVPeriod} &
\sho{\MVMinus} &
\sho{\MVPlus} &
\sho{\MVDivision} &
\sho{\MVMultiplication} \\
\hline
% \end{tabular}
% 
% \begin{tabular}{|*{10}{c}|} \hline
\leg{Conclusion} &
\leg{Equivalence} &
\leg{barOver} &
\leg{BarOver} &
\leg{arrowOver} &
\leg{ArrowOver} &
\leg{StrikingThrough} &
\leg{MultiplicationDot} \\
\sho{\Conclusion} &
\sho{\Equivalence} &
\sho{\barOver} &
\sho{\BarOver} &
\sho{\arrowOver} &
\sho{\ArrowOver} &
\sho{\StrikingThrough} &
\sho{\MultiplicationDot} \\
\hline
% \end{tabular}
 
% \begin{tabular}{|*{10}{c}|} \hline
\leg{LessOrEqual} &
\leg{LargerOrEqual} &
\leg{AngleSign} &
\leg{Corresponds} &
\leg{Congruent} &
\leg{NotCongruent} &
\leg{Divides} &
\leg{DividesNot} \\
\sho{\LessOrEqual} &
\sho{\LargerOrEqual} &
\sho{\AngleSign} &
\sho{\Corresponds} &
\sho{\Congruent} &
\sho{\NotCongruent} &
\sho{\Divides} &
\sho{\DividesNot} \\
\hline
\end{tabular}

 \subsection{Biology}
 
 \begin{tabular}{|*{10}{c}|} \hline
 \leg{Neutral} &
 \leg{Male} &
 \leg{Hermaphrodite} &
 \leg{Female} &
 \leg{MALE} &
 \leg{HERMAPHRODITE} &
 \leg{FEMALE} &
 \leg{MaleMale} &
 \leg{FemaleFemale} &
 \leg{FemaleMale} \\
 \sho{\Neutral} &
 \sho{\Male} &
 \sho{\Hermaphrodite} &
 \sho{\Female} &
 \sho{\MALE} &
 \sho{\HERMAPHRODITE} &
 \sho{\FEMALE} &
 \sho{\MaleMale} &
 \sho{\FemaleFemale} &
 \sho{\FemaleMale} \\
 \hline
 \end{tabular}

\subsection{Biology}

\begin{tabular}{|*{4}{c}|} \hline
\leg{Female} &
\leg{Male} &
\leg{Hermaphrodite} &
\leg{Neutral} \\
\sho{\Female} &
\sho{\Male} &
\sho{\Hermaphrodite} &
\sho{\Neutral} \\
\hline
\leg{FEMALE} &
\leg{MALE} &
\leg{HERMAPHRODITE} & \\
\sho{\FEMALE} &
\sho{\MALE} &
\sho{\HERMAPHRODITE} & \\
\hline
\leg{FemaleFemale} &
\leg{MaleMale} &
\leg{FemaleMale} & \\
\sho{\FemaleFemale} &
\sho{\MaleMale} &
\sho{\FemaleMale} & \\
\hline
\end{tabular}

\subsection{Astronomy}

\begin{tabular}{|*{11}{c}|} \hline
\leg{Sun} &
\leg{Moon} &
\leg{Mercury} &
\leg{Venus} &
\leg{Mars} &
\leg{Jupiter} &
\leg{Saturn} &
\leg{Uranus} &
\leg{Neptune} &
\leg{Pluto} &
\leg{Earth} \\
\sho{\Sun} &
\sho{\Moon} &
\sho{\Mercury} &
\sho{\Venus} &
\sho{\Mars} &
\sho{\Jupiter} &
\sho{\Saturn} &
\sho{\Uranus} &
\sho{\Neptune} &
\sho{\Pluto} &
\sho{\Earth} \\
\hline
\end{tabular}

\subsection{Astrology}



\begin{tabular}{|*{12}{c}|} \hline
\leg{Aries} &
\leg{Taurus} &
\leg{Gemini} &
\leg{Cancer} &
\leg{Leo} &
\leg{Virgo} &
\leg{Libra} &
\leg{Scorpio} &
\leg{Sagittarius} &
\leg{Capricorn} &
\leg{Aquarius} &
\leg{Pisces} \\
\sho{\Aries} &
\sho{\Taurus} &
\sho{\Gemini} &
\sho{\Cancer} &
\sho{\Leo} &
\sho{\Virgo} &
\sho{\Libra} &
\sho{\Scorpio} &
\sho{\Sagittarius} &
\sho{\Capricorn} &
\sho{\Aquarius} &
\sho{\Pisces} \\
\hline
\end{tabular}

\subsection{Others}

\begin{tabular}{|*{10}{c}|} \hline
\leg{YinYang} &
\leg{MVRightArrow} &
\leg{MVAt} &
\leg{BOLogo} &
\leg{BOLogoL} &
\leg{BALogoP} &
\leg{Mundus} &
\leg{Cross} &
\leg{CeltCross} &
\leg{Ankh} \\
\sho{\YinYang} &
\sho{\MVRightArrow} &
\sho{\MVAt} &
\sho{\BOLogo} &
\sho{\BOLogoL} &
\sho{\BOLogoP} &
\sho{\Mundus} &
\sho{\Cross} &
\sho{\CeltCross} &
\sho{\Ankh} \\
\hline
\leg{Heart} &
\leg{CircledA} &
\leg{Bouquet} &
\leg{Frowny} &
\leg{Smiley} &
\leg{PeaceDove} &
\leg{Bat} &
\leg{WomanFace} &
\leg{ManFace} & \\
\sho{\Heart} &
\sho{\CircledA} &
\sho{\Bouquet} &
\sho{\Frowny} &
\sho{\Smiley} &
\sho{\PeaceDove} &
\sho{\Bat} &
\sho{\WomanFace} &
\sho{\ManFace} & \\
\hline
\end{tabular}


\egroup

\thetotalsymbols











 





























\makeatletter

\thispagestyle{plain}

\cxset{image={./images/breakerboys.jpg},
       subsection font-shape= upshape,}

\usemintedstyle{friendly}

\chapter{Listings styles}  

Many users of \latex require to typeset formatted code. There are two packages that
can be used the more conventional \pkgname{listings}\footcite{listings} and \pkgname{minted}\footcite{minted}. The
|minted| package is a more powerful and flexible package than listing, since it uses
an external program |Pygments| which is written in |Python|\footnote{See \protect\url{http://pygments.org/} for more details. You can also review and contribute to the code at \protect\url{https://bitbucket.org/birkenfeld/pygments-main}}. My recommendation to you is
to use the |minted package|. The two packages can happily co-exist and each one has
its own advantages and disantvantages. The listings package has all its color
parameters configurable via its \latex key value settings, whereas the pygments program
has its own way of setting these styles, which are only accessible through \latex
as a set of fixed styles. To create a new color scheme, you will need to write some
simple |python|, register it as a plugin or drop it at the folder holding the styles.\footnote{See documentation at \protect\url{http://pygments.org/docs/styles/}.}

For me
it is the only limiting factor of pygments and which there are ways around it. However, this might be also an advantage
as users are more likely to be familiar with such code coloring schemes in their language.

\section{Using minted}

Since minted makes calls to the outside world (that is, Pygments), you need to
tell the \latex processor about this by passing it the |-shell-escape| option or it
won’t allow such calls. In effect, instead of calling the processor like this:

\usemintedstyle{friendly}
\begin{minted}[fontsize=\footnotesize,style=vim]{bash}
$ latex input
you need to call it like this:
$ latex -shell-escape input
\end{minted}

The same holds for other processors, such as pdf\latex or \xelatex.
You should be aware that using -shell-escape allows \latex to run potentially
arbitrary commands on your system. It is probably best to use -shell-escape
only when you need it, and to use it only with documents from trusted sources.

\subsection{A minimal example}   
 
The minted package is loaded like any other package (with or without options). 
You can then use the \docAuxEnv{minted} environment with the language we want to use
as the first argument. The environment also takes an optional argument where the numerous
settings of the package can be specified.
 
\begin{phdverbatim}[basicstyle=\small\ttfamily]
\documentclass{article}
\usepackage{minted}
\begin{document}
\begin{minted}[fontsize=\footnotesize,style=friendly]{javascript}
if (Meteor.isClient) {
  // This code only runs on the client
  Template.body.helpers({
    tasks: [
      { text: "This is task 1" },
      { text: "This is task 2" },
      { text: "This is task 3" }
    ]
  });
}
\end{minted}
\end{document}
\end{phdverbatim}

This will produce an output as:
\medskip

\begin{minted}[fontsize=\footnotesize,style=friendly]{javascript}
if (Meteor.isClient) {
  // This code only runs on the client
  Template.body.helpers({
    tasks: [
      { text: "This is task 1" },
      { text: "This is task 2" },
      { text: "This is task 3" }
    ]
  });
}
\end{minted}

If we do not need a style, the |style=default| setting will typeset as,

\begin{minted}[fontsize=\footnotesize,style=trac]{javascript}
if (Meteor.isClient) {
  // This code only runs on the client
  Template.body.helpers({
    tasks: [
      { text: "This is task 1" },
      { text: "This is task 2" },
      { text: "This is task 3" }
    ]
  });
}
\end{minted}

You can also set the style for the whole document using:

\begin{minted}[fontsize=\footnotesize,style=trac]{TeX}
\usemintedstyle{<name>}
\end{minted}
where you can get <name> by typing

\begin{minted}[fontsize=\footnotesize,style=bw]{bash}
$ pygmentize -L styles
\end{minted}
at the command prompt/terminal. For example, the minted documentation itself uses the |trac| style.

\begin{minted}[fontsize=\footnotesize,style=friendly]{html}
<!-- First set the doctype -->
<!DOCTYPE html>
    <html>
      <head>
        <title>Canvas</title>
        <meta charset="UTF-8" />
        <style>
          #square {
            border: 1px solid black;
                    transform: scale(10) rotate(3deg) translateX(0px);
                    -moz-transform: scale(10) rotate(3deg) translateX(0px);
          }

          .box {              
                    transition-duration: 2s;
                    transition-property: transform;
                    transition-timing-function: linear;
          }
        </style>
      </head>
      <body>
        <canvas id="square" width="200" height="200"></canvas>
        <script>
                var canvas = document.createElement('canvas');
                canvas.width = 200;
                canvas.height = 200;

                var image = new Image();
                image.src = 'images/card.png';
                image.width = 114;
                image.height = 158;
                image.onload = window.setInterval(function() {
                    rotation();
                }, 1000/60);
       </script>
      </body>
    </html>
\end{minted} 
   



\begin{lstlisting}
int main() {
printf("hello, world");V\colorbox{green}{**}V
return 0;
}
\end{lstlisting}



\chapter{Documentation Macros}


\section{Documentation macros}

When developing this package the need arose to define a number of documentation macros. I~have used heavily macros and ideas present in the \pkg{doc} package, \pkg{pgf} documentation, \pkg{biblatex} documentation  and \pkg{tcolorbox} and for which I am grateful to their respective authors. The major change was to adopt the macros to use different fonts and colors and to use these from a list of key values defined at document level. More about this later. General package user documentation as opposed to package documentation that can be achieved using the |doc/docstrip| system requires that macros and environments be developed for the following:

\begin{enumerate}
\item Macros for command documentation.
\item Environments for commands and options.
\item Latex examples that need to be executed within the document as well as described.
\end{enumerate}


\section{Commands and Styles for Documenting macros}

The most commonly used commands for documenting macros are |\cs|, |\cmd|, |\meta|, |\marg|, |\oarg|. These commands have been defined by many authors and perhaps the best implementation can be found in the \pkg{doc}. Many package authors have redefined them in their documention, some if just to add a bit of colour, others to have them add the command to an index. As we also had a target to allow for
the package to be used in both normal documents as well as documentation
of packages and classes that use the \pkg{doc} and \pkg{docstrip} combination we provided many compatible macros.

\begin{environment}{macro} The environment macro is made available in this
package. 
\end{environment}

\DescribeMacro{macro} The environment macro is made available in this package. 

\begin{macro}{\cmd} The command \cmd{\cmd} typesets its argument in
  verbatim. Typing |\cmd{\cmd}| typsets \cmd{\cmd}. If the class
  |ltxdoc| is loaded the command is defined there. We have modified
  it to accept a colour and changes to the verbatim font 
  for consistency.
\end{macro}

\begin{macro}{meta}
The macro \cs{meta} is normally used to build other commands. On its own it can be used to typeset
examples of the argument of macros, typing |meta{Aristotle}| will typeset meta{Aristotle}. The command provides a hook to set the font via a macro |\meta@font@select|. 
\end{macro}


|\def\meta@font@select{\upshape\color{black}}|


\subsection{Color management}
One of the first requirements for redefining some of the standard doc commands is the need to use color easily, hence we will try and define a certain amount of keys for colors.

Just a bit of a refresher, to define colors we use, either the \cs{definecolor} or the \cs{colorlet} commands.

\emphasis{definecolor,colorlet}

\begin{minted}[fontsize=\small]{TeX}
\definecolor{Hyperlink}{rgb}{0.281,0.275,0.485}
\colorlet{thehyperlink}{theblue}
\end{minted}


We use a semantic approach, where the colors are first defined with a mnemonic command such as {\bfseries\textcolor{theblue}{theblue}} and then we define a semantic command such as the\cs{option} that lets the color to the option command. This sort of double entry has proved useful in navigating through the dozen of the commands that I needed for this documentation.


\subsection{Semantic color names}
\begin{marglist}
\item [\option{theoption}] Coloring of options in margin lists.
\item [\option{themacro}] Coloring of command macros \cs{foo}.
\item [\option{hyperlink}] If we use the \texttt{hyperref} package a number of colors need to be defined for links.
\end{marglist}

\subsection{Named colors}
Standard colors that we provide are:
\begin{marglist}
\item [\textcolor{theblue}{theblue}] This color is used mainly for options.
\item [\textcolor{thered}{thered}] The color mostly used for macro commands and keys.
\item [\textcolor{thegreen}{thegreen}] used for environments.
\item [\textcolor{thelightgreen}{thelightgreen}] Used for margin lists.
\item [\textcolor{thegray}{thegray}] Used as a background to the listings.
\item [\colorbox{thegrey}{\color{white}thegrey}] Alias for the gray to satisfy both sides of the Atlantic and as I sometimes don't remeber which is which.
\item [\colorbox{theshade}{theshade}] Another slightly lighter shade.
\end{marglist}



\begin{marglist}
\item [\cs{cs}] \cs{cs} text Prints a command.
\item [\cs{cmd}] Prints a command.
\end{marglist}




\section{Lists for documentation}



The environment \env{marglist}
\begin{marglist}
\item[testing]\lorem
\item [test]\lorem
\end{marglist}

\env{keymarginlist}This environment is suitable for listing keys, set-in the margin.

\begin{keymarglist}
\item[bibliography] The term <bibliography>, also available as \cmd{\bibname}.
\item[references] The term <references>, also available as \cmd{refname}.
\item[shorthands] The term <list of shorthands> or <list of abbreviations>, also available as \cmd{losname}.
\end{keymarglist}


\env{argumentlist} This environment is suitable for listing macro arguments and their explanations.



\section{Breakable Boxes}

The \pkg{mdframed} as well as the newer versions of \pkg{tcolorbox}
offer breakable boxes.


\begin{tcolorbox}[enhanced, breakable,
  colback=blue!5!white,colframe=blue!75!black,title=Breakable box,
  watermark color=white,watermark text=\Roman{tcbbreakpart}]
  \lipsum[1-3]
\end{tcolorbox}

\section{PGF Style Code Boxes}

\begin{codeexample}[]
\begin{tikzpicture}
  \node[place,label=above:$p_1$,tokens=2]        (p1) {};
  \node[place,label=below:$p_2\ge1$,right=of p1] (p2) {};
\end{tikzpicture}
\end{codeexample}





       

%\makeatletter\@specialfalse\makeatother
%%%%%%%%%%%%%%%%%%%%%%%%%%%%%%%%%%%%%%%%%%%
%%%%%%  STYLE 01
%%%%%%%%%%%%%%%%%%%%%%%%%%%%%%%%%%%%%%%%%%%


\cxset{
 name={},
 numbering=arabic,
 number font-size=\LARGE,
 number font-family=\rmfamily,
 number font-weight=\bfseries,
 number before=,
 number dot=,
 number after=,
 number position=leftname,
 chapter font-family=\sffamily,
 chapter font-weight=\normalfont,
 chapter font-size=\Large,
 chapter before={\vspace*{20pt}\par},
 chapter after={\hfill\hfill\par},
 chapter color={black!90},
 number color=\color{purple},
 title beforeskip={\vspace*{30pt}},
 title afterskip={\vspace*{40pt}\par},
 title before={},
 title after={},
 title font-family=\sffamily,
 title font-color=\color{purple},
 title font-weight=\bfseries,
 title font-size=\LARGE,
 header style=headings}

\cxset{headings ruled-01}

\chapter{Introduction to Style One}


\begin{summary}
This design is simple and its distinguishing characteristic is a short summary at the beginning of the chapter. This is almost like an abstract typeset in italic font without setting the margins in. We provide a \lstinline{summary} environment for convenience. Note the very simple line in the running head to the left of the page number.
\end{summary}

\medskip
\begin{figure}[ht]
\centering
\includegraphics[width=0.5\textwidth]{./chapters/chapter01}
\end{figure}


%\clearpage
\makeatletter\@debugtrue\makeatother
\cxset{
 chapter toc=true,
 name=CHAPTER,
 chapter numbering=ORDINALS,
 number font-size=Large,
 number font-family=rmfamily,
 number font-weight=bfseries,
 number before=\kern0.5em,
 number dot=,
 number after=\hfill\hfill\par,
 number position=rightname,
 chapter font-family=rmfamily,
 chapter font-weight=bold,
 chapter font-size=Large,
 chapter before={\vspace*{20pt}\par\hfill},
 chapter after=,
 chapter color=black,
 number color=black,
 %
 title margin top=10pt,
 title before=\par\nointerlineskip\hfill,
 title after=\hfill\hfill\par\nointerlineskip,
 title font-family=rmfamily,
 title font-color= black,
 title font-weight=bfseries,
 title font-size=LARGE,
 chapter title width=0.8\textwidth,
 chapter title align=centering,
 title margin-left=0pt,
 author block=false}

\debugtitle
\debugchapter
\chapter[Template 2]{Mondino, the Restorer of Anatomy}

The archive.org is an extraordinary hunting ground  for typographical surprises. On a recent excursion to find some books on Versalius I stubled on a book titled \emph{Andreas Vesalius, the reformer of anatomy} by  Ball, James Moores. It is an old book published in 1910 and has a couple of unusual features. Check the figure below and see if you can identify the challenging feature.

\begin{figure}[ht]
\centering
\includegraphics[width=0.8\textwidth]{versalius}
\caption{J.B. Moore’s \emph{Andreas Versalius, the Reformer of Anatomy} has many unusual features, including chapter numbers using ordinals. }
\end{figure}

\cxset{chapter toc=true,
          chapter opening=anywhere}
          
\chapter{The Template}          
The template is called \emph{Versalius} and is stored under style02. It can be loaded in the normal way using:
\begin{verbatim}
\usepackage[style02]{phd}
\end{verbatim}

I have not reproduced the full extend of the book’s requirements, as some details are quite cumbersome to be automated through \tex. These though can easily be incorporated in a manual way. More about this later.


\section{The Table of Contents}
Another interesting aspect of this book, which is common with many books of its period is the ToC. The ToC shows the full range of the chapter pages, i.e., it is marked as Page 1-16 rather than the common practice nowdays that indicates only the starting page of the chapter. It also has “TABLE OF CONTENTS”  as a heading and not just contents as you would expect from today’s books.

\begin{figure}[ht]
\centering
\includegraphics[width=0.8\textwidth]{versalius-01}
\caption{J.B. Moore’s \emph{Andreas Versalius, the Reformer of Anatomy} has many unusual features, including chapter numbers using ordinals. }
\end{figure}

\section{List of Illustrations}

\begin{figure}[ht]
\centering
\includegraphics[width=0.8\textwidth]{versalius-02}
\caption{J.B. Moore’s \emph{Andreas Versalius, the Reformer of Anatomy} has many unusual features, including chapter numbers using ordinals. }
\end{figure}

\section{The Frontmatter}
As a foreward there is an unumbered chapter called ``Introduction’’. The chapter heading also has a head band.
\begin{figure}[ht]
\centering
\includegraphics[width=0.8\textwidth]{versalius-03}
\caption{J.B. Moore’s \emph{Andreas Versalius, the Reformer of Anatomy} has many unusual features, including chapter numbers using ordinals. }
\label{lettrine}
\end{figure}

\bgroup
\centering
\includegraphics[width=0.7\textwidth]{versalius-headband}

\LARGE\bfseries INTRODUCTION\par
\egroup
\def\dropcapversalius{%
\vbox to 0pt{\vskip6pt\leavevmode\noindent\includegraphics[width=2.39cm]{versalius-dropcap}%
}%
}
\parindent0pt

\hangindent2.6cm \hangafter0
\dropcapversalius \textsc{he dropcap will have to be inserted}, either using the lettrine package or do be achieved via a parshape command and manual entry. You can also write your own macro command using the details we provide under the Paragraphs chapter. On this page I have manually inserted it, as I used an image from the book for the dropcap. If you were to use the template for a full book, it will be then preferable to use

the lettrine package to set the dropcaps. If you observe Figure~\ref{lettrine} carefully, you will notice the first line of theopening paragraph is in small caps. As \tex typesets the full paragraph this is almost an impossible task to achieve through normal \tex commands and in order not to overcomplicate the discussion it can be achieved manually via trial and error. 

\section{Figures}

Most of the figures are wrapped illustrations. A couple are full page figures and bear no caption numbering. One such illustration is shown on page~\pageref{fig:vesalius}. Do note that the List of Illustrations does have the illustrations listed with additional information to that shown in the captions. 

\begin{figure}[p]
\centering
\includegraphics[width=\textwidth]{vesalius}
\centering
ANDREAS VESALIUS\par
(From an old copperplate engraving)\par
\label{fig:vesalius}
\end{figure}







%\newgeometry{top=-10pt,bottom=2cm}

\tcbset{width=\paperwidth,boxrule=0pt,right=3cm,arc=0pt}

\cxset{style03/.style={
 name={},
 numbering=arabic,
 number font-size= HUGE,
 number font-family= rmfamily,
 number font-weight= bfseries,
 number before=\par\vspace*{10pt}\hfill\hfill,
 number dot=.,
 number after=,
 number position=rightname,
 chapter font-family= sffamily,
 chapter font-weight=normalfont,
 chapter font-size= Large,
 chapter before={\hspace*{-2.5cm}\vbox\bgroup\tcolorbox\bgroup\vspace*{20pt}\hfill\hfill},
 chapter after={\par\vspace*{15pt}},
 chapter color=black!90,
 number color= thered,
 title beforeskip={},
 title afterskip={\vspace*{10pt}\par},
 title before={\hfill\hfill},
 title after={\vspace*{60pt}\egroup\endtcolorbox\egroup},
 title font-family=\sffamily,
 title font-color= thered,
 title font-weight=\bfseries,
 title font-size=\Huge}}

\cxset{style03}

\chapter{Introduction Style Three}

This is not an exact reproduction as I am still thinking as to how to use
specials with the package. You can vary it by setting the tcolorbox settings as well as the geometry settings.
\medskip

\begin{figure}[ht]
\centering
\includegraphics[width=0.39\textwidth]{./chapters/chapter03}
\end{figure}

This setting involves changing the geometry of the page as well as adding the chapter name and title in a color box. For this I have used the \lstinline{tcolorbox}. Of course you can use any other shaded environment you feel comfortable with such as mdframed. It is important to set the colorbox parameters.

\begin{lstlisting}
\newgeometry{top=-10pt}
\tcbset{width=\paperwidth,boxrule=0pt,right=3cm,arc=0pt}
\end{lstlisting}

Note that we set the width of the \lstinline{tcolorbox} to \lstinline{\paperwidth} in order for the shading to extend to the full width of the page.

\restoregeometry

%\clearpage
\cxset{style04/.style={
 numbering=Roman,
 number font-size=\Large,
 number font-family=\rmfamily,
 number font-weight=\bfseries,
 number before=,
 number dot=,
 number after=,
 number position=rightname,
 chapter font-family=\rmfamily,
 chapter font-weight=\normalfont,
 chapter font-size=\Large,
 chapter before={\vspace*{20pt}\par\hfill},
 chapter after={\hfill\hfill\par\vspace*{10pt}},
 chapter color={black!90},
 number color=purple,
 title beforeskip={},
 title afterskip={\vspace*{50pt}\par},
 title before={\hfill},
 title after={\hfill\hfill\par},
 title font-family=\rmfamily,
 title font-color= purple,
 title font-weight=\normalfont,
 title font-size=\LARGE,
 section numbering=none,
 section align = center}}

\cxset{style04}

\chapter{INTRODUCTION TO STYLE FOUR}

This is a very simple design applicable perhaps to translations and commentary on older texts.
\medskip
\begin{figure}[ht]
\centering
\includegraphics[width=0.6\textwidth]{./chapters/chapter04.png}
\end{figure}

%
\cxset{style05/.style={
 name={Chapter},
 chapter color = magenta,
 chapter toc = true,
 numbering=arabic,
 number font-size=\Large,
 number font-family=\rmfamily,
 number font-weight=\normalfont\itshape,
 number color= purple,
 number before=\hspace*{-15pt},
 number dot=,
 number after=,
 number position=rightname,
 chapter font-family=sffamily,
 chapter font-weight= \bfseries\itshape,
 chapter font-size=\Large,
 chapter before={\hrule width \columnwidth \kern12.6pt \par\hfill},
 chapter after={\hfill\hfill\par},
 chapter color={magenta},
 chapter spaceout=none,
 title beforeskip={\vspace*{10pt}},
 title afterskip={\vspace*{30pt}\par},
 title before={\hfill},
 title after={\hfill\hfill \vskip12.6pt\hrule width \columnwidth \kern2.6pt },
 title font-family=\rmfamily,
 title font-color=black!90,
 title font-weight=\bfseries,
 title font-size=\huge,
 title font-shape = normal,
 header style= headings}}

\cxset{style05}
\chapter{Introduction to Style Five}\index{ch:style5}

\tcbset{width=\textwidth}
I think this style can be improved with a bit of color. You can experiment with it quite easily. The spacing on top of this style can also be adjusted to suit your typographical taste.
\medskip
\begin{figure}[ht]
\centering
\includegraphics[width=0.6\textwidth]{./chapters/chapter05}
\end{figure}

%\section{General notes on rules}

LaTeX's default rules would normally give problems. Best is to use TeX's primitives to built them.

\index{rules!example color}

\begin{texexample}{}{}
\makeatletter
\hrule width 5cm \kern2.6\p@
AAAAAAAAAAAAAAAAAAAAA
\vskip2.6pt\hrule width 5cm
\medskip

Problem with LaTeX rules.

\rule{5cm}{0.4pt}\par
AAAAAAAAAAAAAAAAAAAAA\par%
\rule[6.5pt]{5cm}{0.4pt}

\def\rule{\@ifnextchar[\@rule{\@rule[\z@]}}
\def\@rule[#1]#2#3{%
 \leavevmode
 \hbox{%
 \setlength\@tempdima{#1}%
 \setlength\@tempdimb{#2}%
 \setlength\@tempdimc{#3}%
 \advance\@tempdimc\@tempdima%
 \vrule\@width\@tempdimb\@height\@tempdimc\@depth-\@tempdima}}

\def\thickrule{\leavevmode \leaders \hrule height 3pt \hfill \kern \z@}

{\color{teal}\hrule width 10.5cm height3pt \kern2.6\p@
    {{\color{black!80}\HUGE CHAPTER TITLE}}\vskip3pt
\hrule width 10.5cm height3pt}
\makeatother
\end{texexample}

%\cxset{chapter format=block}
%\makeatletter
\cxset{defaults/.style ={% 
    chapter title margin-top-width    =  0cm,
    chapter title margin-right-width  =  1cm,
    chapter title margin-bottom-width = 10pt,
    chapter title margin-left-width   = 0pt,
    chapter align                     = left,
    chapter title align               = left, %checked
    chapter name                      = CHAPTER,
    chapter format                    = block,
    chapter font-size                 = Huge,
    chapter font-weight               = bold,
    chapter font-family               = sffamily,
    chapter font-shape                = upshape,
    chapter background-color          = white,
  % chapter label    
    chapter color               = black,
    chapter number prefix             = ,
    chapter number suffix             = ,
    chapter numbering                 = arabic,
    chapter indent                    = 0pt,
    chapter beforeskip                = -3cm,
    chapter afterskip                 = 30pt,
    chapter afterindent               = off,
    chapter number after              = ,
    chapter arc                       = 0mm,
    chapter label background-color    = white,
    chapter label color               = black,
   % chapter afterindent               = on,
    chapter grow left                 = 0mm,
    chapter grow right                = 0mm,
    chapter rounded corners           = northeast,
    chapter shadow                    = fuzzy halo,
    chapter border-left-width         = 0pt,
    chapter border-right-width        = 0pt,
    chapter border-top-width          = 0pt,
    chapter border-bottom-width       = 0pt,
    chapter padding-left-width        = 0pt,
    chapter padding-right-width       = 10pt,
    chapter padding-top-width         = 10pt,
    chapter padding-bottom-width      = 10pt,
    %  
    chapter number color              = black,
    chapter number background-color   = white,
    chapter number font-size        = huge,
    chapter number font-weight      = bfseries,
    chapter number font-family      = sffamily,
    chapter number font-shape       = upshape,
    chapter number align            = Centering,
    %
    chapter title font-size        = Huge,
     chapter title font-weight      = bold,
     chapter title font-family      = sffamily,
     chapter title font-shape       = upshape,
     chapter title color            = black,
     chapter title background-color = white,
     }%
   }  
\makeatother     
%\makeatletter
%\cxset{toc image=\@empty,
%       chapter toc=true,
%       title beforeskip=1pt}
%
%\@specialfalse
%
%
%\renewcommand\stewart[2][]{%
%\fancypagestyle{fancy}{%
%\lhead{}\rhead{}
%\chead{}
%\cfoot{}
%\lfoot{}
%\rfoot{\thepage}
%\def\footrule#1{{\color{blue}%
%  \hrule width\paperwidth}\vskip3pt
%}
%
%\renewcommand{\headrulewidth}{0pt}
%\renewcommand{\footrulewidth}{0.4pt}}
%
%\clearpage
%
%\begin{tikzpicture}[remember picture,overlay]
%% Main shading block
%\node [xshift=5cm,yshift=-\paperheight] at (current page.north west)
%[text width=0.98\textwidth,text height=\paperheight, fill=thecream!30,rounded corners,above right]
%{};
%\node [xshift=6.5cm,yshift=-1.5cm-\soffsety] at (current page.north west)
%[text width=0.9\textwidth,below right]{\sffamily \bfseries \huge #2};
%
%\node [xshift=3cm,yshift=-1.5cm] at (current page.north west)
%[text width=3cm,align=center,minimum height=2.5cm, fill=blue,below right]
%{\[\text{\HHUGE\bfseries\sffamily\color{white}\thechapter}\]
%\par\vspace*{3pt}
%};
%
%\node [xshift=-0.2cm,yshift=-21.5cm] at (current page.north west)
%[text width=3cm,above right]%
%{\includegraphics[width=1.0\paperwidth]{\image@cx}};
%% second box left
%\node [xshift=3cm,yshift=-19.5cm] at (current page.north west)
%[text width=9cm,minimum height=2.5cm,inner sep=0.5em, fill=blue,below right]
%{\color{white}
%  \bfseries\sffamily \texti@cx
%};
%% Last block
%\node [xshift=6.5cm,yshift=-26cm] at (current page.north west)
%[text width=12cm,above right]
%{\textii@cx
%};
%\end{tikzpicture}
%\par
%\clearpage
%}





\cxset{steward,
  chapter numbering=arabic,
  chapter format = stewart,
  offsety=0cm,
  image= {./images/hine02.jpg},
  texti={When Lamport designed the original \LaTeX\ sectioning commands he did not provide a fully comprehensive interface for modifying their design. With current tools available improvements are much easier to program and this chapter provides the details.},
  textii={\precis{In this chapter we discuss a method that allows the production of fancy chapter headings and formatting, based on a set of key values. Central  to this process is the separation of content from presentation.
We also discuss the basic formatting tools that are available and how one can modify them to mould new book designs.}
 }
}


\chapter{Designing Chapter Headings}
\addtocimage{-12pt}{-20pt}{./images/tocblock-man-01.jpg}

\section*{Introduction}

A \textls*{crowded} first page is as unsightly as a crowded title page, wrote De Vinne in \emph{Modern Methods of Book Composition} in 1904.  Not much has changed since. A new chapter must make a good impression and must give an immediate signal that a different topic is going to be discussed. Traditionally chapter openings in LaTeX are an unimpressive and dry event. Our aim is to brighten it up a bit, while keeping true separation of content from presentation, but avoiding the pit traps of over ornamenting the design. A book is to be read and we should provide minimal ornamentation. \index[phdkeys]{chapter> ornamentation}

% \usepackage{array,tabularx}
%\newcolumntype{Y}{>{\raggedleft\arraybackslash}X}% see tabularx
%\tcbset{enhanced,fonttitle=\bfseries\large,fontupper=\normalsize\sffamily,
%colback=yellow!10!white,colframe=red!50!black,colbacktitle=thecodebackground,
%coltitle=black,center title,
%tabularx={X||Y|Y|Y|Y||Y},% this sets ’before upper’ and ’after upper’
%before upper app={Group & One & Two & Three & Four & Sum\\\hline\hline} }
%
%\begin{tcolorbox}[title=My table]
%Red & 1000.00 & 2000.00 & 3000.00 & 4000.00 & 10000.00\\\hline
%Green & 2000.00 & 3000.00 & 4000.00 & 5000.00 & 14000.00\\\hline
%Blue & 3000.00 & 4000.00 & 5000.00 & 6000.00 & 18000.00\\\hline\hline
%Sum & 6000.00 & 9000.00 & 12000.00 & 15000.00 & 42000.00
%\end{tcolorbox}

\begin{figure}[htbp]
\centering
\parindent=0pt
\fbox{\includegraphics[width=\textwidth]{metropolitan-spread}}
\par
\caption{A chapter opening from the Metropolitan Museum of Art publicaion, \textit{Assyrian Reliefs and Ivories} by Vaughn. E. Crawford et. al., 1980. The spread is simple and the chapters are not numbered. This is a common characteristic of many more recently published books.}
\end{figure}


What is to us now a common occurence with instant book-printing was not always so. The cost of illustrated books was a prime factor and as Tschichold wrote:
\begin{quotation}
In the area of book design, in the last few years a revolution has taken place, until recently recognized by only a few. but which now begins to influence a much wider range of action.
It means placing much greater emphasis on the appearance of the book and a wholly contemporary use of typographic and photographic means. Before the invention of printing, literature of that time was spread around by the mouth of the author himself or by professional bards. The books of the Middle Ages - like the "Mannessische Liederhandschrift" - had
\end{quotation}

The type of book you are writing and its contents will determine an appropriate design for chapter headings and the type of design and numbering if any for subsections. Here we are merely providing a mechanism to produce them. These methods can produce a mastepiece or an ugly piece of work. Some simple suggestions follow (from my observations of styles in books I like). In general you need to think what type of book you are developing. For example a novel, should be sectioned very carefully. Many books avoid marking of sections other than chapters totally, perhaps marking them just with a soft ornament such as three centered asterisks.

\section{Numbering of Sections}


In general books do not number sections beyond subsection. You can avoid them all together, if you are not going to reference the sections extensively. 

In works of fiction, authors sometimes number their chapters eccentrically, often as a metafictional statement. For example:
Seiobo There Below by László Krasznahorkai has chapters numbered according to the Fibonacci sequence.

The Curious Incident of the Dog in the Night-Time by Mark Haddon only has chapters which are prime numbers.

At Swim-Two-Birds by Flann O'Brien has the first page titled Chapter 1, but has no further chapter divisions.

God, A Users' Guide by Seán Moncrieff is chaptered backwards (i.e., the first chapter is chapter 20 and the last is chapter 1). The novel The Running Man by Stephen King also uses a similar chapter numbering scheme.
Every novel in the series A Series of Unfortunate Events by Lemony Snicket has thirteen chapters, except the final instalment (The End), which has a fourteenth chapter formatted as its own novel.

Mammoth by John Varley has the chapters ordered chronologically from the point of view of a non-time-traveler, but, as most of the characters travel through time, this leads to the chapters defying the conventional order.


\begin{pgfpicture}
\pgfpathmoveto{\pgfpointorigin}
\pgfpathlineto{\pgfpoint{1cm}{1cm}}
\pgfpathlineto{\pgfpoint{1cm}{0cm}}
\pgfusepath{fill}
\end{pgfpicture}




\begin{figure}[tbp]
\centering
\parindent=0pt
\fbox{\includegraphics[width=\textwidth]{fantasy-architecture}}
\par
\caption{A chapter opening from the Metropolitan Museum of Art publicaion, \textit{Assyrian Reliefs and Ivories} by Vaughn. E. Crawford et. al., 1980. The spread is simple and the chapters are not numbered. This is a common characteristic of many more recent books.}
\end{figure}


\begin{figure}[tbp]
\centering
\parindent=0pt
\fbox{\includegraphics[width=\textwidth]{fantasy-architecture-02}}
\par
\caption{A chapter opening from the Metropolitan Museum of Art publicaion, \textit{Assyrian Reliefs and Ivories} by Vaughn. E. Crawford et. al., 1980. The spread is simple and the chapters are not numbered. This is a common characteristic of many more recent books.}
\end{figure}


\section*{Use of Color}

The modern books that Tschilchod was discussing have long been overwhelmed by the appearance of larger, coffee book type of books. Our brains our now conditioned by branding and graphic design is everywhere. 

Once you have decided that the book is going to be a bit more colorfull, the choice of color will follow. The decision what to color will be an important one, which brings us to color theory. The history of color is perhaps as colorfull as the rest. Attempts to formalize and recognize order date back to Aristotle (384-322 bce) but began in earnest with Leonardo da Vinci (1452-1519) and have progressed ever since. Leonardo noted that certain colors intensify each other, discovering \textit{contrary} and \textit{complementary} colors. The first color wheel was invented by Britain's Sir Isaac Newton (1642-1727), who split white light into red, orange, yellow, green, blue, indigo and violet beams, then joined the two ends of the spectrum to form a circle showing the natural progression of colors. When Newton created the color wheel, he noticed that mixing two colors from opposite positions produced a neutral or \textit{anonymous} color.


\begin{figure}[htbp]
\parindent=0pt
\centering
\fbox{\includegraphics[width=\textwidth]{line-designs} }
\caption{Spread from \textit{Beautiful Geometry}, Eli Maor and Eugen Jost, Princeton Univeristy Press, 2014. A subtle coloring of the chapter heading, de-emphasizing the chapter number and coloring the chapter title. There is no chapter label. A dropcap with the same color starts the first paragraph. This style is easy to achive with the phd system.}
\end{figure}


\begin{figure}[htbp]
\parindent=0pt
\centering
\fbox{\includegraphics[width=\textwidth]{color-book01.jpg} }
\bigskip

\fbox{\includegraphics[width=\textwidth]{color-book02.jpg} }
\end{figure}

One would expect a book written for the sole purpose of describing color theory and its application to the Graphic Arts, is expected to be colorful. Note the de-emphasizing of the label and number. 

\begin{figure}[htbp]
\parindent=0pt
\centering
\fbox{\includegraphics[width=\textwidth]{color-book-03.jpg} }
The chapter heading label and number are almost invisible. The heading text, is typeset in large bold letters, shouting what is coming next. Not your typical scintific book\ldots
\bigskip

\fbox{\includegraphics[width=\textwidth]{color-book-04.jpg} }
\end{figure}

Advertizing people understand that they need to present the message of an advertizement loud and clear so as to catch the busy eye. A heading's message is the title description. Neither the label not the chapter if any are necessary to convey the message. The chapter heading is analogous to the stop at the end of a sentence. The brain gets a signal to absorb what was written before it and get ready for the next. The heading signals the end of a topic. One must not dwell on it.


\section{Contemporary Chapter Headings}

In the book \textit{China} the designer used both a chapter heading on a spread of two images, as well as repeated the chapter number on the text pages \ref{fig:threepage}. The images distill the message of the chapter, although the chapter subtitle is almost unreadable, dominated by the surrounding text. From a technical perspective, the chapter command must paint the two images, set the right type of heading for each page and then without increasing the counter, change the counter to one that displays the chapter number in words and then continue with typesetting the text. A careful choice of images is necessary for such chapters, as well as cropping the images to match the aspect ratio of the book pages. One also needs to be carefull for \latexe not to place any floats in between the page spreads. 

\begin{figure}[htbp]
\parindent=0pt
\centering
\fbox{\includegraphics[width=\textwidth]{beijing.jpg} }\par
\vfill

\fbox{\includegraphics[width=\textwidth]{beijing-01.jpg} }\par
%\fbox{\includegraphics[width=\textwidth]{pearl-river.jpg} }
\caption{A full page chapter spread.}
\label{fig:threepage}
\end{figure}

\begin{figure}[htbp]
\parindent=0pt
\centering
\fbox{\includegraphics[width=\textwidth]{beijing.jpg} }\par
\vfill

\fbox{\includegraphics[width=\textwidth]{beijing-01.jpg} }\par
%\fbox{\includegraphics[width=\textwidth]{pearl-river.jpg} }
\caption{A full page chapter spread.}
\label{fig:threepage}
\end{figure}


\clearpage



In Figure~\ref{fig:photospread} the bands are black, but position low on the page. The size of the pages are 9.69 \texttimes 11.42. The books sections are not numbered. Text i sbroken through inserts of bigger text. Many of the examples here are from
commercial nude photography books, as they tend to break with tradition. In the 1970s and 1980s, fashion photographers began to present a
new, confrontational image of the female body. The pioneer in this
respect was the German Helmut Newton (1920–2004). Newton’s
photographs of nudes were overtly sexual, with an undertone of
menace, and although his models tended to be depicted as part
of the social elite they were often placed, apparently caught out
in reportage style, in sordid environments engaged in fantasy and
fetish. His work made him highly influential in fashion photography,
though some of it was thought too highly sexual for American
magazines and appeared only in those published in Europe.


\begin{figure}[htbp]
\parindent=0pt
\includegraphics[width=\textwidth]{baetens-01.jpg} \par
\vfill\vfill\vfill\vfill
\includegraphics[width=\textwidth]{baetens-02.jpg}\par
\caption{Chapter spread and first pages after the chapter title which is on the right page of the chapter spread. From \textit{New Photography, Art and the Craft}, Pascal Baetens, DK Publications. }
\label{fig:photospread}
\end{figure}

In the 1980s, Newton undressed the dynamic and independent
female in a series called Big Nudes. In this series the women are
indeed naked and very tall, wearing nothing but makeup and high
heels. The Big Nudes were exhibited in the form of life-size prints
that were intended to provoke the viewer by showing self-confident
women who knew what they wanted and were very aware of their
beauty and sexuality



\chapter{Package Usage}

To use the package include it just like any other package:

\begin{teXXX}
\documentclass{book}
\usepackage{phd}
\cxset{style13}
\begin{document}
\chapter{Introduction}
\end{document}
\end{teXXX}

The command \docAuxCommand{cxset} sets the default style for the example to the style defined as \meta{style13}. The package currently offers  100 templates and numerous keys to manipulate them further. Styles are similar to \enquote{themes} used in web programming; they are a collection of keys that resemble in many ways \texttt{css}. Styles can have any names and I am sure as package usage increases and evolve,they will get better names. 

\section{Background}

Before describing in detail how to specify a new layout for headings, we offer an overview of how the task can be accomplished and the design philosophy behind the approach. 

Irrespective of the technique and tools used, the creation of new layouts can always be divided into the following three tasks: constructing a document from “layout bricks”, which we can term as “blocks” or “elements”; establishing the layout semantics of each block; and finally, creating a layout engine supporting any document constructed from such blocks.

\begin{description}
\item [Canned Layouts] At one end of the spectrum, the most accessible approach consists of picking, a canned layout, such as LaTeX itself and perhaps only provide rudimentary macros to manipulate it.
\item [Constraints] Constraints offer a middle ground between canned layouts and handwritten layout engines. Constraints are arguably the most widespread and successful layout programming technique. For, instance, the foundations of \tex are laid upon constraint. CSS, the ubiquitous web template language, also relies on constraints, although in a more restricted and indirect manner.
\end{description}

\subsection{Blocks and Elements}

We define an \emph{element} as a document block, that cannot be subdivided further. For example the chapter title element, is composed of the text of the chapter title. 

A \emph{block} on the other hand is can contain other blocks and or numerous elements. We can consider the chapter headings as \emph{blocks}, composed of three blocks the chapter, number and title. Each block is then composed of elements. Each element has properties and traits. One of these mandary properties is the name. 

Blocks are either \emph{configured} (all constraints are mandatory), or flexible (there are optional/alternative constraints). By bundling optional constraints, flexible blocks make their specification customizable by non-technical users. 

\subsection{Language semantics}

One of the aims of the syntax of the templates was to offer familiar terminology and to remove the use
of \tex macros as far as possible from templates. 
\medskip

{\parindent0pt

 \textit{section}| font-family=serif,|\\
 \textit{section}| font-size=LARGE,|\\
 \textit{section}| font-weight=bold,|\\
}

The restriction I imposed is problematic when dealing with fractions of linewidths and textwidths. So
at present we allow for example |title text-width=0.5\texwidth| or |title text-width=10cm| or any other valid units. Ideas for improvements can only come from user feedback in the future.

Some experimental ideas incorporated are:

\begin{verbatim}
title text-width = 0.5 text-width,
title text-width = 1.2 text-width,
\end{verbatim}

A better parser will need to be programmed for dimensions, which are all currently handled as etex |dimexpr|. 

The syntax must allows both for microtypography as well as macro-typographical features. The former would deal with mostly fonts, spacing and text justification, where the latter deals with layouts, borders shapes and the positioning of elements on the page and also reletively to other elements or blocks.

An advantage of this approach is that it also opens the possibility of parsing the text with a language other than \tex and translating the document to another format, such as |HTML| or |XML| either fully or partially. Next we will describe both the syntax as well as the usage of the settings.

\section{Chapter opening page}

The standard \latexe classes offer only two options to either open a chapter on an odd page or at any page. This package offers five alternatives:

\begin{docKey}[phd]{chapter opening}{=\meta{any, left, right, anywhere, ifafter}}{default none, initial=any}
For documents that are primarily to be read on the web, use |any| for normal books, use \textit{right}. Some templates that we provide use |any| and the examples use |anywhere| to enable us to display the heading at any position on the page.
\end{docKey}

\begin{decription}
\item [any] Opens a chapter at any page, either \textit{verso} or \textit{recto}.
\item [left] Opens a chapter on an even page
\item [right] Opens a chapter on a right page.
\item [anywhere] Opens a chapter at the point where the \cs{chapter} is typed.
\item [none] Alias for \marg{anywhere}.
\item [ifafter] Opens a chapter at the next page if the page has material that does not exceed a certain portion of \cs{textheight}.
\end{description}

\colorlet{theoption}{bgsexy}

To change a setting you just modify the value of the key \oarg{\option{chapter opening}} to one of the values described earlier. 

\begin{dispListing}
\cxset{chapter opening = anywhere}
\end{dispListing}
 
We use this key to print the many examples typesetting chapter heads that follow (see the example~\ref{ex:anywhere}).  


\begin{texexample}{title=Inline Chapter Example}{ex:anywhere}
\cxset{examplestyle/.style = {chapter format = block,
       chapter opening = anywhere,
       chapter name = CHAPTER, 
       %label
       chapter label font-family      = sffamily,
       chapter label color            = primary,
       chapter label background-color = white,
       % number
       chapter number font-family = sffamily,
       chapter number font-size = HUGE,
       chapter number color     = primary,
       chapter label align = centering,
       chapter number background-color = white,
       %title
       chapter title font-family = rmfamily,
       chapter title align = centering,
       chapter title background-color = bgsexy!15,
       chapter title before background-color=white}}
\cxset{examplestyle}       
\lorem
\chapter{Typography Example}
\lorem
\chapter{Another Chapter Heading}
\lorem
\end{texexample}


%\cxset{toc chapter = true}
\addtocounter{chapter}{-1}

Examples for other types of chapter openings follow in the rest of the documentation.

\subsection{Blank pages before chapters}

In the standard LaTeX book class when the \texttt{openany} option is not given or in the report class when the openright is given, chapters start at odd-numbered pages. This can cause a blank page to be printed. Some book designers prefer this page to be completely empty, without any headers or footers. This cannot be done with \lstinline{\thispagestyle} as this command will have to be issued on the \textit{previous} page. However by a suitable redefinition of the
\lstinline{\clearpage} this can be done automatically.
\medskip

\begin{teXXX}
\makeatletter
\def\cleardoublepage{\clearpage\if@twoside\ifodd\c@page\else
  \hbox{}
  \vspace*{\fill}
  \begin{center}
    This page left intentionally blank.
  \end{center}
  \vspace{\fill}
  \thispagestyle{empty}
  \newpage
  \if@twocolumn\hbox{}\newpage\fi\fi\fi}
\makeatother
\end{teXXX}


This is achieved easily by setting the following options:
\bigskip

\begin{tcolorbox}
\lstinline{chapter blank page=empty}\par
\lstinline{chapter blank page text=Some text.}\par
\lstinline{chapter blank page=plain}\par
\end{tcolorbox}
\medskip



The last one refers to a \lstinline!\thispagestyle{plain}!.
\cxset{chapter opening = right, chapter format = block}
\chapter{Test}

\cxset{defaults, chapter opening= anywhere}



\section*{Keys for chapter head formatting}

A chapter heading can be considered of being constructed of several parts, the \textit{chapter number}, the chapter name typically \textit{chapter} and the \textit{title}. Predefined keys handle all the elements of formatting. Additional keys are defined to handle other elements such as inclusion of images or producing complicated examples with graphics constructed with \texttt{TikZ} and other similar packages.


\bigskip\bigskip\bigskip\bigskip
\let\oldrefkey\refKey
\let\refKey\texttt
\makeatletter
\long\def\demobox#1#2{%
\par\bigskip\bigskip\bigskip
\begin{tcolorbox}[enhanced,left=0pt, top=0pt, bottom=0pt,width=\textwidth,
  enlarge top initially by=1cm,enlarge bottom finally by=1cm,left skip=1cm,right skip=1cm,
  colframe=white,colback=white,
  colbacktitle=red!30!white,colupper=black!7!white,
  code={\appto\kvtcb@shadow{%
    \path[fill=white,draw=yellow!50!black,dashed,line width=0.4pt]
      ([xshift=-1cm,yshift=-1cm]frame.south west) rectangle
      ([xshift=1cm,yshift=1cm]frame.north east);
     \path[fill=blue!20!white, 
              opacity=0.3, draw=yellow!50!black,solid,line width=1pt]
      ([xshift=-2cm,yshift=-2cm]frame.south west) rectangle
      ([xshift=2cm,yshift=2cm]frame.north east);  
    }},
  finish={
  \draw[thick,<->] ([yshift=-1.3cm]frame.north west)-- node[below]{\texttt{#1 width}}
    ([yshift=-1.3cm]frame.north east);
  \draw[thick,<->] ([xshift=-15mm]frame.north east)-- node[above]{\refKey{#1 height}}
    ([xshift=-15mm]frame.south east);
  \draw[thick,<->] (frame.north)-- node[right]{\refKey{#1 padding-top}} +(0,1);
  \draw[thick,<->] ([yshift=1cm]frame.north)-- node[right]{\refKey{#1 margin-top}} +(0,1);
  \draw[thick,<->] (frame.south)-- node[right, align=left]{\refKey{#1 padding-bottom}}+(0,-1);
  %left padding
  \draw[thick,<->] (frame.west)-- node[below right,align=center]{\refKey{#1 padding-left }}+(-1,0);
  %left margin
  \draw[thick,<->] ([xshift=-1cm,yshift=-0.9cm]frame.west)-- node[below right,xshift=-1,align=left]{\refKey{#1 margin-left }\\\refKey{#1 grow to left by}}+(-1,0);
  %right padding
  \draw[thick,<->] (frame.east)-- node[below left,align=center]{\refKey{#1 padding-right}}+(1,0);
 %right margin
  \draw[thick,<->] ([xshift=1cm,yshift=-0.9cm]frame.east)-- node[below left,xshift=1, align=right]{\refKey{#1 margin-right}\\\refKey{#1 grow to right by}}+(1,0);
 \draw[thick,<->] ([yshift=-2cm]frame.south)-- node[right, align=left]{\refKey{#1 margin-bottom},\\ \refKey{#1 after skip}}+(0,1);
  }
    ]
#2%
%\hrule width0pt height4.5cm depth0pt\relax% \vspace*{4.5cm}% \lipsum[1]
\end{tcolorbox}\par
\bigskip\bigskip\bigskip}
\makeatother

\demobox{chapter}{\scalebox{1.17}{\HHHUGE Chapter}}

The number box is again drawn in a box similar to a chapter with all properties generalized.

\demobox{number}{\scalebox{1.15}{\HHHUGE Thirteen}}



All parameters shown in the diagram can be set using the command \cs{cxset}. The property names follow conventions similar to those of |css|, rather than typical conventions of \tikzname that are more widely known to the programming community. The prefix to these properties (in the example \textit{chapter}) can be thought of
as similar to a |class| or |id| name in |css|.  

\begin{docCommand}{cxset}{\marg{options}}
  Sets options for every following \refEnv{tcolorbox} inside the current \TeX\ group.
  By default, this does not apply to nested boxes, see \Vref{subsec:everybox}.\par
  For example, the colors of the boxes may be defined for the whole document by this:
\begin{dispListing}
\cxset{chapter numbering = Roman,
       chapter number color = blue}
\end{dispListing}
\end{docCommand}

\begin{docKey}[]{chapter padding-top}{=\meta{dimension}}{no default, initial value 0pt}
All padding keys take one argument, which is a dimension. The length is also stored in a register
\cmd{\chapterpaddingtop}. In this chapter it was set at %\the\chapterpaddingtop.
\begin{dispListing}
\cxset{colback=red!5!white,colframe=red!75!black, chapter padding-top=2pt}
\end{dispListing}
\end{docKey}



\begin{docKey}[]{chapter padding-right}{=\meta{dimension}}{no default, initial value 0pt}
All padding keys take one argument, which is a dimension. The length is also stored in a register
\cmd{\chapterpaddingright}.  In this chapter it was set at %\the\chapterpaddingright.
\end{docKey}

\begin{docKey}[]{chapter padding-bottom}{=\meta{dimension}}{no default, initial value 0pt}
All padding keys take one argument, which is a dimension. The length is also stored in a register
\cmd{\chapterpaddingbottom}.  In this chapter it was set at %\the\chapterpaddingbottom.
\end{docKey}

\begin{docKey}[]{chapter padding-left}{=\meta{dimension}}{no default, initial value 0pt}
All padding keys take one argument, which is a dimension. The length is also stored in a register
\cmd{\chapterpaddingleft}.  In this chapter it was set at %\the\chapterpaddingleft.
\end{docKey}

%% margin

\begin{docKey}[]{chapter margin-top}{=\meta{dimension}}{no default, initial value 0pt}
All padding keys take one argument, which is a dimension. The length is also stored in a register
\cmd{\chaptermargintop}. In this chapter it was set at .
\end{docKey}

\begin{docKey}[]{chapter margin-right}{=\meta{dimension}}{no default, initial value 0pt}
All padding keys take one argument, which is a dimension. The length is also stored in a register
\cmd{\chapterpaddingright}.  In this chapter it was set at %\the\chapterpaddingright.
\end{docKey}

\begin{docKey}[]{chapter margin-bottom}{=\meta{dimension}}{no default, initial value 0pt}
All padding keys take one argument, which is a dimension. The length is also stored in a register
\cmd{\chapterpaddingbottom}.  In this chapter it was set at %\the\chapterpaddingbottom.
\end{docKey}

\begin{docKey}[]{chapter margin-left}{=\meta{dimension}}{no default, initial value 0pt}
All padding keys take one argument, which is a dimension. The length is also stored in a register
\cmd{\chaptermarginleft}.  In this chapter it was set at %\the\chaptermarginleft.
\end{docKey}

\subsection{Borders}

Border have three properties \emph{width, color} and \emph{style}. They can set individually for
each side of the box or using the shorter key .

\begin{docKey}[]{chapter border-top-width}{ = \meta{dimension}}{no default, initial value 0pt}
All border keys take one argument, which is a dimension.
\end{docKey}

\begin{docKey}[]{chapter border-right-width}{=\meta{dimension}}{no default, initial value 0pt}
All border keys take one argument, which is a dimension.
\end{docKey}

\begin{docKey}[]{chapter border-bottom-width}{ = \meta{dimension}}{no default, initial value 0pt}
All border keys take one argument, which is a dimension.
\end{docKey}

\begin{docKey}[]{chapter border-left-width}{ = \meta{dimension}}{no default, initial value 0pt}
All border keys take one argument, which is a dimension.
\end{docKey}

\subsubsection{Border Colors}

The colors follow the same pattern for |border-width| and again they can be set individually or using
a shorter key to set all of them in one color. 

\begin{docKey}[]{chapter border-top-color}{=\meta{color name}}{no default, initial value black}
All border keys take one argument, which is a dimension.
\end{docKey}

\begin{docKey}[]{chapter border-right-color}{=\meta{color name}}{no default, initial value black}
All border keys take one argument, which is a dimension.
\end{docKey}

\begin{docKey}[]{chapter border-bottom-color}{=\meta{color name}}{no default, initial value black}
All border keys take one argument, which is a dimension.
\end{docKey}

\begin{docKey}[]{chapter border-left-color}{=\meta{color name}}{no default, initial value black}
This key is stored in \cmd{\chapterborderrightcolor} and the value in this chapter is 
%\ExplSyntaxOn \l_phd_chapter_border_right_color_tl.
\ExplSyntaxOff
\end{docKey}



\subsubsection{Border Styles}

Standard |css|  offers four styles \emph{dotted, solid, double, dashed}. We offer almost an unlimited set of styles.

\begin{docKey}[phd]{chapter border-top-style}{=\meta{style name}}{no default, initial value \texttt{none}}
The |border-style| properties take a value, which can be |solid, double, dotted, dashed, asterisk|.
\end{docKey}

\begin{docKey}[phd]{chapter border-right-style}{=\meta{style name}}{no default, initial value \texttt{none}}
The |border-style| properties take a value, which can be |solid, double, dotted, dashed, asterisk|.
\end{docKey}

\begin{docKey}[]{chapter border-bottom-style}{=\meta{style name}}{no default, initial value \texttt{none}}
The |border-style| properties take a value, which can be |solid, double, dotted, dashed, asterisk|.
\end{docKey}

\begin{docKey}[]{chapter border-left-style}{=\meta{style name}}{no default, initial value \texttt{none}}
The |border-style| properties take a value, which can be |solid, double, dotted, dashed, asterisk|.
\end{docKey}

\begin{docKey}[phd]{chapter border-style}{=\meta{style name}}{no default, initial value \texttt{none}}
This key sets all chapter-border-\meta{top,right,bottom,left}-style to a single value.
\end{docKey}

\subsubsection{Fonts and colors}

All font parameters can be set using individual keys. The naming scheme in general follows |css| conventions.

\begin{docKey}[phd]{chapter color}{=\meta{color name}}{no default, initial value \texttt{black}}
This key sets the color for the \textit{chapter element}. The color name is stored in \cmd{\chaptercolor@cx}.
The value in this chapter is% \makeatletter\texttt{\chaptercolor@cx}\makeatother.
\end{docKey}

\begin{docKey}[phd]{chapter font-size}{=\meta{Huge, Large}}{no default, initial value \texttt{Huge}}
This sets the size for rendering the \textit{chapter element}. Use one of the following predefined values.
Note that you can either use a command i.e, |chapter font-size=|\cmd{\huge} 
or the command name i.e., |chapter font-size=huge|. The latter is the recommended method.
\end{docKey}

\begin{marglist}
\item [tiny] renders as {\tiny tiny}.
\item[footnotesize] renders as {\footnotesize footnotesize}
\item [small] Opens a chapter on an even page
\item [large] Opens a chapter on a right page.
\item [LARGE] Opens a chapter at the point where the \cs{chapter} is typed.
\item [huge] Alias for \marg{anywhere}.
\item [Huge] Opens a chapter at the next page if the page has material that does not exceed a certain portion of
 \cs{textheight}.
 \item[HUGE] renders as {\HUGE HUGE}.
 \item[HHUGE] renders as {\HHUGE HUGE}.
\end{marglist}

\begin{texexample}{Sizing settings}{}
\cxset{
          chapter format = block,
          chapter label font-size= HUGE,
          chapter name = Chapter,
          chapter format=block,
          chapter number font-size= HUGE,
          chapter title font-size=LARGE,
         % 
         % chapter padding-top=0pt,
         % chapter padding-bottom=0pt,
         % title margin-top=3pt,
         %
          }
\chapter{Setting font-sizes}          
\lorem

\end{texexample}


\begin{docKey}{chapter font-family}{ = \meta{sffamily, rmfamily etc.}}{no default, initial value \texttt{sffamily}}
The |font-family| key accepts \latexe conventional family names or |css| names such as |serif| and |non-serif|. The
value is stored in \docAuxCommand{chapter_font_family}, in this chapter it is set as {\ExplSyntaxOn\meaning\chapter_font_family\ExplSyntaxOff}
\end{docKey}


\begin{marglist}
\item [sffamily] The \emph{chapter element} is rendered in the document default \cmd{\sffamily}.
\item [rmfamily] The \emph{chapter element} is rendered in the document default \cmd{\rmfamily}.
\end{marglist}

%% Font weights
\begin{docKey}[]{chapter font-weight}{=\meta{mdseries,bfseries,etc.}}{no default, initial value \texttt{bfseries}}
The |font-weight| key accepts \latexe conventional family names or |css| names such as |bold| and |bfseries|. The
value is stored in \cmd{\chapterfontweight@cx}, in this chapter it is set as 
{\ExplSyntaxOn\expandafter\string\l_phd_chapter_label_fontweight_tl\ExplSyntaxOff}

\begin{texexample}{Setting chapter element font-weights}{fontweight}
\cxset{chapter label font-weight=normal}
\chapter{Font-weight is normal}
\cxset{chapter label font-weight= bfseries}
\chapter{Font-weight is bfseries}
\lorem
\end{texexample}
\end{docKey}


\begin{marglist}
\item [normal] The \emph{chapter element} is rendered in the document default \cmd{\sffamily}.
\item [bold] The \emph{chapter element} is rendered in the document default \cmd{\rmfamily}.
\item[bfseries] Renders as bold.
\item[mdseries] renders as medium series.
\item[light] This is an alias for normal.
\item[\upshape\ttfamily\string\bfseries] The command version of the setting.
\item[\upshape\ttfamily\string\mdseries] The command version of the setting.
\end{marglist}



\begin{docKey}[]{chapter font-shape}{=\meta{itshape,upshape,etc.}}{no default, initial value \texttt{upshape}}
The |font-weight| key accepts \latexe conventional family names or |css| names such as |bold| and |bfseries|. The
value is stored in |chapter_font_weight|, in this chapter it is set as %\ExplSyntaxOn \texttt{\chapter_font_shape}\ExplSyntaxOff.
\end{docKey}

In |css| the |font-shape| is named as |font-style| so we alias it as well. 

%\begin{marglist}
%\item[normal] normal font-style, defaults to |upshape|.
%\item[upshape] normal font-style, defaults to |upshape|. 
%\item[italic] italic shape, renders as {\itshape italic}. For some fonts it might not be available.
%\item[itshape] italic shape, alias of |italic|.
%\item[oblique] oblique font, in \latexe is equivalent to \cmd{\slshape} and renders as {\slshape slshape}, which might be slightly different than {\itshape italic}.
%\end{marglist}


\begin{texexample}{Setting up Fonts}{chapterfonts}
\cxset{   chapter format = block,
          chapter opening=anywhere,
          chapter label font-weight=normal,
          chapter label font-shape=upshape,
          %chapter border-width=0pt,
          %chapter border-style=none,
          %chapter padding-top=0pt,
          chapter label font-size=large,
          chapter number font-size=large,
          chapter number color=black,
          %title font-size=large,
          }
\chapter[fonts]{Test Font Weights}
\lorem
\cxset{chapter label font-shape=itshape}
\chapter{Test Italic Shape}
\lorem
\cxset{chapter label font-shape=normal}
\chapter{Test normal font-shape}
\lorem
\end{texexample}



The specification of font families is somewhat problematic. In the web the |css| allows |font-family|  to hold several font names as a ``fallback” system. If the browser does not support the first font, it tries the next font.

There are two types of font family names:

\begin{description}
\item[family-name] The name of a font-family, like “times”, “courier”, “arial”, etc.
\item[generic-family] The name of a generic family, like “serif”, “sans-serif”, “cursive”, “fantasy”, “monospace”.
\end{description}

Generally in the \tex community leaving the choice of font  open to what is available on a user’s computer is frowned upon. Knuth’s original aim to render consistently documents, irrespective of a user’s computer installation has served the community well, and it is possible three decades later to produce documents identical in all respects to the original. 

If this is still a valid requirement for documents is debatable. Current document processing requirements are focusing more in the preservation of content and document structure rather than form. Typeset documents in soft copy are now widely preserved in |pdf| or |postcript|  formats. One can archive the |.tex| file as well as the |pdf| file.  Back to the provision of keys, a key defined in a 
similar fashion to those of |css| could help, but there is also the issue of slow compilation. If a font cannot be
found, with the current code, it can slow down compilation tremendously. I am leaving the choice where it belongs to the user and the package writer. It makes no harm if a more flexible definition is provided. The user can then decide to only provide one or many fonts. 

This avoids complicated and almost unintelligible commands such as:

\begin{dispListing}
\setkomafont{subsection}{\usefont{T1}{fvm}{m}{n}}
\setkomafont{section}{\usefont{T1}{fvs}{b}{n}\Large}
\end{dispListing}

Here are some examples. 

\begin{texexample}{Serif and non-serif}{ex:fontfamily}
\cxset{chapter label font-family=serif, 
       chapter opening=anywhere}
\chapter{Serif font}
\lorem
\end{texexample}


\section{Floating and Alignment} 

This particular key bothered me, as the term \emph{float} has a different meaning in \latexe. However, to
be consistent with |css| terminology I have yielded to the temptation and included it.

\begin{docKey}[]{chapter float}{=\meta{left,center,right,none}}{no default, initial value \texttt{none}}
Key that controls the horizontal alignment of the \emph{chapter element}. I order for the
element to float, its |display| property must be set to |inline|.
\end{docKey}

%\begin{texexample}{Floating}{chapter:float}
%\cxset{chapter opening=anywhere, chapter float=center}
%\chapter{Centered Chapter}
%\lorem
%\cxset{chapter float=left}
%\chapter{Left Aligned}
%\lorem
%\cxset{chapter float=right}
%\chapter{Right Aligned}
%\lorem
%\end{texexample}


\subsection{The display property}

Both the |css| box model as well as the \TeX layout engine provide numerous complex algorithms in managing the floating of elements. This is normally controlled using two properties |display| and |float|.


\makeatletter

\begin{docKey}[phd]{chapter position}{ = \meta{absolute, relative}}{no default, initial value black}
This positioning directive instructs the engine to position the element at an exact position.
\end{docKey}



\tcbox[nobeforeafter]{$box_1$}\tcbox[nobeforeafter]{$box_2$}\tcbox[nobeforeafter]{$box_3$}\dotfill\tcbox[nobeforeafter]{$box_n$}
\tcbox[before skip=0.2cm, after skip=0pt, width=1cm, enlarge left by=10cm,width=5cm,enhanced,show bounding box]{title before element}
\tcbox[before skip=0pt, width=1cm, enlarge left by=10cm,width=5cm,enhanced,show bounding box]{
\tcbox{tb}\tcbox{title}\tcbox[nobeforeafter, width=1cm,]{tb}}
\tcbox[before skip=0pt, after skip=12pt, width=1cm, enlarge left by=10cm,width=5cm,enhanced,show bounding box]{\emph{title after} element \fbox{some}}
\makeatother

\begin{docKey}[phd]{chapter float}{=\meta{left,center,right,none}}{no default, initial value \texttt{none}}
Key that controls the horizontal alignment of the \emph{chapter element}. I order for the
element to float, its |display| property must be set to |inline|.
\end{docKey}
In document preparation systems or web page development the layout is user generated, i.e., the user is expected to type the html and the |css| will then specify as to how the page will be rendered by the browser. In our case for documents we can specify how we want the headings to look. The layout manager for each element, creates other associated elements, as shown for the title here. This way most layouts can be accomplished with the declarative visual language of the \pkgname{phd} package. 

\subsubsection{In-line elements}

When an element is specified as |inline| the rendering algorithm places the boxes after each other. This is widely used in |chapter elements| to render the number inline with the chapter name.
\medskip
\bgroup

\noindent
\tcbox[nobeforeafter,width=3cm, height=1cm]{Chapter}\tcbox[nobeforeafter]{twelve}
 
When the property is set as |block| the elements are stacked below each other.
\medskip

\tcbox{chapter  display=block   CHAPTER}
\tcbox{number display=block    TWELVE}

The elements can be considered to be enclosed in a \emph{ghost} element. If the property is set to float we
\begin{figure}[htbp]
\makeatletter
\parindent0pt\fboxsep0pt
\fbox{\vbox to 0pt{\hbox to \dimexpr(\textwidth)\relax{{\hss\tcbox[capture=minipage,width=5cm, height=2cm, top=0pt]{\raggedright number display=block\\ number float=right }}%
}%
}%
}\par
\vspace*{2cm}
\makeatother
\end{figure}
signalling to the layout engine that the element must be placed to the right of the page, as shown in the figure. 


\begin{figure}[htbp]
\makeatletter
\parindent0pt\fboxsep0pt
\fbox{\vbox to 0pt{\hbox to \dimexpr(\textwidth+2cm)\relax{{\hss\tcbox[capture=minipage,width=5cm, height=2cm, top=0pt]{\raggedright number display=block\\ \emph{element} float=right }
\tcbox[capture=minipage,width=5cm, height=2cm, top=0pt]{\raggedright \emph{element} display=block\\ \emph{element} float=right }
}%
}%
}%
}\par
\vspace*{2cm}
\makeatother
\end{figure}

\subsection{Absolute positioning}

Absolute positioning mode, will place an element at an exact position on the page. They are more difficult to
achieve and inflexible. 

\begin{docKey}{position}{=\meta{absolute},\meta{relative}}{no default, initial none}{}

\end{docKey}



This positioning directive instructs the engine to position the element at an exact position.


\begin{docKey}[]{chapter float}{=\meta{left,center,right,none}}{no default, initial value \texttt{none}}
Key that controls the horizontal alignment of the \emph{chapter element}. In order for the
element to float, its |display| property must be set to |inline|.
\end{docKey}
\egroup



\section{Number Element Keys}


\subsection*{Keys for numbering}

Chapter numbering follows that of the standard \LaTeX\ classes and is extended to cover some additional cases such as fully spelled out numbers. This of course is only good for languages that use the arabic numeralsn. For other languages numerals in different formats can be added with simple keys and without the need of \pkgname{polyglossia} or \pkgname{babel}. 

Note that the package uses Heiko Oberdiek's package \pkgname{alphalph} to allow for alphabetic numbering that extends beyond the normal 26 letters of the alphabet. Examples for numbering can be seen in \ref{ex:romannumbering}


\begin{docKey}[phd]{number numbering}{= \oarg{alph,Alph,roman,Roman,none,WORDS,words,none}}{default arabic}
Style of numbering.
\end{docKey}

\begin{marglist}
\item [arabic] Despite that the Arabs call what the West calls Arabic numbers Indian numbers, we provide the value arabic to have normal numbers printed.
\item [alph] Lowercase alphabetic numbering.
\item [Alph] Uppercase alphabetic numbering.
\item [roman] Lowercase roman numbering.
\item [Roman] Uppercase roman numbering.
\item [words] The number is in lowercase words.
\item [WORDS] The number is in uppercase literal numerals.
\item [Words] Prints the number in words and capitalizes the first letter, for example the number 21 will be printed as `Twenty One'\footnote{Currently limited to the first hundred numbers}.
\index{chapter design>numbering>words}
\item [ordinals] Prints the number as ordinal.
\item [Ordinals] Prints the number as Ordinal.
\item [ORDINALS] Prinst the number as ORDINALS.
\item [none] This is equivalent to using the star version of the command. It does not print any number and does not increment the chapter counter.\footnote{I am ambivalent about this, perhaps it will be better to increment it, as it can give a more general approach.}

\end{marglist}
\begin{texexample}{Literal Numbering}{ex:literal}
\cxset{chapter numbering=WORDS} 
\chapter{Literal numbering}
\lorem
\cxset{chapter numbering=words,chapter name=chapter}
\chapter{Literal numbering} 
\lorem
\end{texexample}




\cxset{chapter opening=anywhere, chapter numbering=Roman, chapter number font-shape=upshape}
\index{chapter design>numbering>roman}

\begin{texexample}{Setting up keys for numbering}{ex:romannumberingx}
\bgroup
\cxset{chapter format = traditional, 
       chapter name = CHAPTER, 
       chapter numbering = Roman,
       chapter label color = bgsexy}
\chapter{Roman numbering}
\lorem
\egroup
\end{texexample}





To emulate some old books we also offer an ordinal numbering scheme.

\begin{texexample}{Literal Numbering}{ex:ordinals}
\cxset{chapter numbering=ORDINALS} 
\chapter{Ordinals numbering}
\lorem
\cxset{chapter numbering=words,chapter name=chapter}
\chapter{Literal numbering} 
\lorem
\end{texexample}

\cxset{chapter numbering=arabic}

\subsection{Fonts and colors}
\begin{docKey}[phd]{number color}{=\meta{color name}}{no default, initial value \texttt{black}}
This key sets the color for the \textit{number element}. The color name is stored in %\cmd{\numbercolor@cx}.
The value in this chapter is %\makeatletter\texttt{\numbercolor@cx}\makeatother.
\end{docKey}

\begin{docKey}[phd]{number font-size}{=\meta{Huge, Large}}{no default, initial value \texttt{Huge}}
This sets the size for rendering the \textit{number element}. Use one of the predefined values, as described
in the section for the \emph{chapter} element.
Note that you can either use a command i.e, |number font-size=|\cmd{\huge} 
or the command name i.e., |number font-size=huge|. The latter is the recommended method.
\end{docKey}

Letter spacing can be achieved using the soul package in a combination with the key |spaceout|.
The following examples illustrate the usage.

\index[phdkeys]{{\ttfamily phd/chapter design test}}

%\begin{texexample}{Letter Spacing}{ex:letterspacing}
%\cxset{numbering=Roman,
%        % number letter-spacing=soul,
%        % chapter spaceout=soul,
%         %title spaceout=soul,
%         title font-size=Large,
%         title font-family=rmfamily,
%         title font-shape=scshape}
%\chapter{Letter Spacing}
%
%\lorem
%\end{texexample}

\begin{docKey}[phd]{chapter number letter-spacing}{=\meta{none, true, etc.}}{no default, initial value \texttt{none}}.
\end{docKey}

\begin{marglist}
\item[none] Default value no tracking is used and the letters are spaced as per the basic font information.
\item[inherit] Inherits the letter-spacing settings from the \emph{chapter} element.
\item[true] Letter spacing is employed, using the |soul| package.
\item[false] Alias for |none|.
\item[soul] The \pkgname{soul} package is used for letter-spacing.
\item[microtype] The \pkgname{microtype} package is used for letter-spacing. When the microtype package is used more fine tuning of parameters is available.
\end{marglist}

The example that follows, explains how the features offered by the \pkgname{microtype} package can be used to
set different tracking options.

\begin{texexample}{Microtypography}{micro}
\bgroup

\SetTracking
 [ no ligatures = {f},
 spacing = {600*,-100*, },
 outer spacing = {450,250,150},
 outer kerning = {*,*} ]
 { encoding = * }
 { 100 }

{\huge \textls{Chapter Twenty}}

\SetTracking
 [ no ligatures = {f},
 spacing = {600*,-100*, },
 outer spacing = {450,250,150},
 outer kerning = {*,*} ]
 { encoding = * }
 { 200 }
 
{\huge \textls{Chapter Twenty}}

\egroup
\end{texexample}


\hbox{\drawfontbox{\huge \upshape\textls(Chapter Twenty}}

\hbox{\drawfontbox{\huge \upshape\textls{Chapter Twenty}}}


\section{Styling the chapter title}

Similarly to the number and chapter styling keys exist for styling the chapter title. We summarize the available standard keys below:

\index{chapter design!labels!letter spacing}
\begin{texexample}{Styling the Title}{ex:title} 
\cxset{chapter numbering=arabic, chapter title font-shape=itshape}
\chapter{Chapter title}
\lorem
\end{texexample}


\begin{docKey}[phd]{chapter title font-family}{=\marg{family}}{no default, initial inherit document font}
Selects a predefined font family
\end{docKey}

\begin{texexample}{Title element font styling}{}
\cxset{chapter title font-family=sffamily}
\chapter{Title font family settings}
\lorem
\cxset{chapter title font-shape=itshape}
\chapter{Title font-style settings}
\lorem
\end{texexample}


\begin{docKey}[phd]{chapter title font-weight}{ = \marg{\cs{bfseries},\cs{normalseries}}} {}
Font weight.
\end{docKey}

\begin{docKey}[phd]{chapter title font-size}{= \marg{large, Large, huge, Huge, HUGE, HHuge}}{}
Font sizing commands or their names. Both \docAuxCommand{\HUGE} and HUGE are allowed to be used as values for the key.
\end{docKey}

\begin{docKey}[phd]{chapter title color} { = \marg{color}} {}
The color of the chapter title letters. This takes any predefined color name. 
\end{docKey}


\begin{docKey}[phd]{chapter title spaceout}{ = \marg{soul,none}} {no default, initial = none}
 This key will space out the title. 
\end{docKey}

\begin{texexample}{Title element spacing}{}
\cxset{chapter name=none,
       chapter numbering=none,
       chapter title font-size=Large,
       chapter title color=black,
       chapter title width=0.6\textwidth,
       %title spaceout=soul,
         }
\chapter{The Prehistoric Period in South-East Asia: 2300 BC--AD 400}        
\lorem 
    
\end{texexample}
\cxset{defaults}


\subsection*{Adding content before and after the title element}

Like all the other elements, the title element can be decorated with additional content,
before and after the text. There are two different forms. 

\begin{docKey}[phd]{title before}{=\marg{code}}{default none}
Contents before the title (vertical material)
\end{docKey}

\begin{docKey}[phd]{title after}{=\marg{code}}{default none}
Contents after the title (vertical material)
\end{docKey}

\begin{docKey}[phd]{title content before}{=\marg{code}}{default none}
Contents before the title (horizontal material)
\end{docKey}

\begin{docKey}[phd]{title content after}{=\marg{code}}{default none}
Contents after the title (horizontal material)
\end{docKey}

The difference between the two type of settings, consider the following situation. Assume you have a title that has a rule at the top and bottom and the text is surrounded by two ornaments. The surrounding ornaments will be inserted using the |title before content|, and the rules using the |title before| form. The |title before| is a full fledged element on its own. 

%{
%\hrule
%\centering
%*** Introduction ***
%\par
%\hrule
%}
%
%{
%\MakePercentComment
%\startlineat{200}
%\lstinputlisting{./styles/style13.tex}
%\MakePercentIgnore
%}



 
\begin{docKey}{/phd/ chapter title before skip}{= \marg{soul,none}}{}
Before title string skip.
\end{docKey}

\begin{docKey}{/phd/ chapter title after skip}{ = \marg{soul,none} }{}
After title string skip.
\end{docKey}

\lorem 
%
%\begin{texexample}{letter spacing the chapter title block}{ex:title3}
%
%\cxset{chapter spaceout=none,
%         numbering=arabic}
%         
%\chapter{Chapter Title Styling}
%\end{texexample}
%
%\end{document}



\cxset{chapter opening=right}
\section{Table of Contents}\index{table of contents!key settings}

Traditionally a chapter will be added to the Table of Contents if the \cs{chapter} command is issued. The starred version will not produce a number and will not add a contents line. Since we have adopted an approach where we use a key value interface we can dispense with the starred version of the command, by setting the \option{chapter toc} option to false. For example if we want to define a command for a ``Foreward'' or ``Epiloque'' without wishing them to be added to the table of contents we can use the following setting.\index{Foreward>definitions}\index{Epilogue>definitions}



\begin{texexample}{changing the chapter label name}{}
\cxset{chapter name=Chapteris, chapter numbering=arabic,}
\chapter{Foreward}
\lorem
\end{texexample}

Note that the key \option{numbering=none} still has to be set.


Please note that when \textbf{numbering=none} the chapter number is not available anymore and yo may have to reset it if required again. Although this might be seen as rather cumbersome than simply using \cs{chapter*} the advantage is consistency in the user interface and the use of appropriate semantic definitions for all sectioning commands thus achieving a bit more separation of context from style.


%\cxset{chapter toc=true}

\section{Defining styles}

Named styles can be defined using the standard \textsc{PGF} conventions. To define a style for the forward above we can use:

\begin{texexample}{}{}
\cxset{foreward/.style={chapter numbering=none,
          chapter name=none,
          chapter title font-size= Large,
          chapter title font-family= sffamily,
          chapter numbering=none}}
\cxset{foreward}
\chapter{Foreward.}
\lorem
\end{texexample}



\cxset{chapter numbering=arabic}
\section{Creating semantic names for commands and environments}

To keep our search for semantic commands and true separation of contents it is prudent to define some macros for typesetting the  `foreward' section.

\bgroup
\begin{texexample}{defining a \textit{Foreward} macro.}{}
\begin{lstlisting}
\cxset{foreward/.style={chapter toc=false,
          name=none,
          title font-size = Large,
          title font-family = sffamily,
          numbering=none}}
\newcommand\forewardname{foreward}
\expandafter\newenvironment\expandafter{\forewardname}{%
\cxset{foreward}\chapter{Foreward}}%
{}
\begin{foreward}
\lorem
\end{foreward}
\end{lstlisting}
\end{texexample}
\egroup

Notice the use of a new command \cmd{\forewardname} to allow for internationlization using Babel or other methods. One is tempted to let the English name, but a better approach perhaps is to define both.

\makeatletter



%
\@specialtrue
\cxset{steward,
  numbering=arabic,
  custom=stewart,
  offsety=0cm,
  image=hine03,
  texti={When Lamport designed the original \LaTeX\ sectioning commands, limitations of computer power forced him to restrict the abstraction of complicated chapter layouts. With current tools available improvements are much easier to program.},
%
  textii={In this chapter we discuss a method that allows the production of fancy sectionr headings and formatting, based on a set of key values. Central  to this process is the separation of content from presentation.
We also discuss the basic formatting tools that are available and how one can modify them to mould new book designs.
 }
}



\raggedbottom

\chapter{Lower Level Headings}
\@specialfalse

\section{Introduction}

Good book design dictates that sectioning styles follow that of the general book design and theme. An academic publication for example might have chapters and section numbered in arabic numerals, whereas a high school textbook might have sections marked in colored boxes.

Similarly to the chapter key value interface, the package offers a key value interface to adjust sectioning command parameters.



\cxset{section beforeskip={10pt},
      section indent=0pt}
\cxset{section afterskip={10pt}}
\renewsection

\section{Section styling}

In a similar fashion to the chapter commands the following keys are provided.

\subsection{Fonts and numerals}

Font and numeral keys are shown below.
\medskip

  \keyval{section font-size}{\marg{cmd}}{Font size command such as \cs{large.}}
  \keyval{section font-weight}{\marg{cmd}}{Font weight command such as \cs{bfseries.}}
  \keyval{section font-family}{\marg{cmd}}{Font family command such as \cs{sffamily.}}
  \keyval{section font-shape}{\marg{cmd}}{Font shape command such as \cs{itshape}}
  \keyval{section color}{\marg{color}}{Color of section.}
  \keyval{section numbering}{\marg{arabic|roman|Roman|alph|Alph|words|WORDS}}{Section number style.}
  \begin{marglist}
  \item [arabic] Typesers the section number in arabic numerals.
  \item [roman] Typesets the section number in lowercase roman numerals.
  \item [Roman] Typesets the section number in uppercase roman numerals.
  \item [alph] Typesets the section number in lowercase alphabetic numbering.
  \item [Alph] Typesets the section number in uppercase alphabetic numerals.
  \item [words] Typesets the numbers in words (lowercase).
  \item [WORDS] Typesets the number in words (uppercase).
  \end{marglist}

\subsection{Skip and indentation commands}

The keys for indentaion and above and below skips are shown below.
\medskip

\keyval{section beforeskip}{}{}
\keyval{section afterskip}{}{}
\keyval{section indent}{\marg{dim}}{Indentation from margin as per standard LaTeX class definitions.}
\keyval{section spaceout}{}{}
\begin{marglist}
 \item[soul]
 \item[none]
\end{marglist}

\subsection{align}

\keyval{section align}{\marg{cmd}}{One of the alignment commands centering, ragged right, raggedleft}

\subsection{Hooks}

Hooks for adding material are shown in the following sketch.
\medskip

\fbox{aboveskip}

\fbox{indent} \fbox{number}\fbox{hook}\fbox{title}

\fbox{belowskip}

%\lipsum

\section{Example usage}

\cxset{
 chapter toc=false,
 name=CHAPTER,
 numbering=arabic,
 number font-size=\huge,
 number font-family=\sffamily,
 number font-weight=\bfseries,
 number before=,
 number dot=,
 number after=\hspace{1em},
 number position=rightname,
 chapter opening=anywhere,
 chapter font-family=\sffamily,
 chapter font-weight=\bfseries,
 chapter font-size=\huge,
 chapter before={\vspace*{0.1\textheight}\hfill},
 chapter after={\hfill\hfill\vskip0pt\thinrule\par},
 chapter color={black!90},
 number color=\color{black!90},
 title beforeskip={\vspace*{30pt}},
 title afterskip={\vspace*{30pt}\par},
 title before={\hfill},
 title after={\hfill\hfill},
 title font-family=\sffamily,
 title font-color=\color{black!90},
 title font-weight=\bfseries,
 title font-size=\huge,
%%%%%%%%%% Sections
 section font-size=\LARGE,
 section font-weight=\normalfont,
 section font-family=\sffamily,
 section align=\centering,
 section numbering=arabic,
 section indent=0em,
 section align=\centering,
 section beforeskip=20pt,
 section afterskip=10pt,
 section spaceout=soul,
 section font-shape=\itshape,
}
\cxset{book/.style={
 section numbering=arabic,
 section font-size=\Large,
 section font-weight=\bfseries,
 section font-family=\rmfamily,
 section font-shape=\normalfont,
 section align=\raggedright,
 %section numbering custom=\color{gray}{Section} (\thechapter-\@arabic\c@section),
 subsection font-size=\large
 section indent=0em,
 section beforeskip=-3.5ex \@plus -1ex\@minus -0.2ex,
 section afterskip=2.3ex\@plus.2ex,
 subsection beforeskip=-3.5ex \@plus -1ex\@minus -0.2ex,
 subsection afterskip= 1.5ex \@plus .2ex,
}}


\begin{example}{Adjusting section parameters}{}
\cxset{ section font-size=\LARGE,
 section font-weight=\normalfont,
 section font-family=\sffamily,
 section align=\centering,
 section numbering=(roman),
 section indent=0em,
 section align=\centering,
 section beforeskip=20pt,
 section afterskip=10pt,}
\chapter{A First Look at the Sectioning Keys}
\section{First section}
\lorem
\end{example}

One notable thing to keep in mind is that the numbering of the chapter is independent of that for the section, so if you need to have strange combinations rather define a section numbering custom.\index{section formatting!vertical space}

\cxset{section numbering=arabic}
\subsection{Adjusting vertical spaces}

Perhaps the most important issues we need to consider is the adjusting of vertical spaces; example~\ref{ex:latex}, that follows illustrates settings from the Octavo class and compare them with those of standard the \LaTeXe\ book class. The Octavo class through settings that are based on baselineskip fractions and multiples endeavours to achieve a grid layout. The class also tones down the `loudness' of some of the headings compared to those of the book class.


\cxset{octavo/.style={
 section font-size=\large,
 section font-weight=\normalfont,
 section font-family=\rmfamily,
 section font-shape=\scshape,
 section indent=0em,
 section align=\centering,
 section beforeskip=-1.666\baselineskip\@minus -2\p@,
 section afterskip=0.835\baselineskip \@minus 2\p@,
 subsection numbering=none,
 subsection font-family=\rmfamily,
 subsection font-size=\normalfont,
 subsection font-shape=\scshape,
 subsection font-weight=\normalfont,
 subsection indent=1em,
 subsection align=\raggedright,
 subsection beforeskip=-0.666\baselineskip\@minus -2\p@,
 subsection afterskip=0.333\baselineskip \@minus 2\p@
 }}




\cxset{book/.style={
 section numbering=arabic,
 section font-size=\Large,
 section font-weight=\bfseries,
 section font-family=\rmfamily,
 section font-shape=\normalfont,
 section align=\raggedright,
 %section numbering custom=\color{gray}{Section} (\thechapter-\@arabic\c@section),
 subsection font-size=\large,
 section indent=0em,
 section beforeskip=-3.5ex \@plus -1ex\@minus -0.2ex,
 section afterskip=2.3ex\@plus.2ex,
 subsection font-size=\large,
 subsection font-weight=\bfseries,
 subsection numbering=arabic,
 subsection indent=0pt,
 subsection beforeskip=-3.5ex \@plus -1ex\@minus -0.2ex,
 subsection afterskip= 1.5ex \@plus .2ex,
}}

\cxset{octavo headings/.style={%
 section numbering=none,section font-size=\large,section font-weight=\normalfont,
 section font-family=\rmfamily, section font-shape=\scshape,
 section indent=0em, section align=\centering, section beforeskip=-1.666\baselineskip\@minus -2\p@,
 section afterskip=0.835\baselineskip \@minus 2\p@, subsection numbering=none,
 subsection font-family=\rmfamily, subsection font-size=\normalfont, subsection font-shape=\scshape,
 subsection font-weight=\normalfont, subsection indent=1em, subsection align=\raggedright,
 subsection beforeskip=-0.666\baselineskip\@minus -2\p@,
 subsection afterskip=0.333\baselineskip \@minus 2\p@,
 subsubsection numbering=none,
 subsubsection font-family=\rmfamily,
 subsubsection font-size=\normalfont,
 subsubsection font-shape=\itshape,
 subsubsection font-weight=\normalfont,
 subsubsection indent=1em,
 subsubsection align=\raggedright,
 subsubsection beforeskip=-0.666\baselineskip\@minus -2\p@,
 subsubsection afterskip=0.333\baselineskip \@minus 2\p@,
 paragraph numbering=none,
 paragraph font-family=\rmfamily,
 paragraph font-size=\normalfont,
 paragraph font-shape=\normalfont,
 paragraph font-weight=\normalfont,
 paragraph indent=-1em,
 paragraph align=\raggedright,
 paragraph beforeskip=\z@,
 paragraph afterskip=0\p@,
% subparagraph numbering=none,
% subparagraph font-family=\rmfamily,
% subparagraph font-size=\normalfont,
% subparagraph font-shape=\normalfont,
% subparagraph font-weight=\normalfont,
% subparagraph indent=0em,
% subparagraph align=\raggedright,
% subparagraph beforeskip=\z@,
% subparagraph afterskip=0\p@,
}}
\cxset{octavo headings}
\renewsection\renewsubsection\renewsubsubsection\renewparagraph

\begin{example}{Octavo class headings, settings}{}
\cxset{octavo headings/.style={%
 section numbering=none,section font-size=\large,section font-weight=\normalfont,
 section font-family=\rmfamily, section font-shape=\scshape,
 section indent=0em, section align=\centering, section beforeskip=-1.666\baselineskip\@minus -2\p@,
 section afterskip=0.835\baselineskip \@minus 2\p@, subsection numbering=none,
 subsection font-family=\rmfamily, subsection font-size=\normalfont, subsection font-shape=\scshape,
 subsection font-weight=\normalfont, subsection indent=1em, subsection align=\raggedright,
 subsection beforeskip=-0.666\baselineskip\@minus -2\p@,
 subsection afterskip=0.333\baselineskip \@minus 2\p@,
 subsubsection numbering=none,
 subsubsection font-family=\rmfamily,
 subsubsection font-size=\normalfont,
 subsubsection font-shape=\itshape,
 subsubsection font-weight=\normalfont,
 subsubsection indent=1em,
 subsubsection align=\raggedright,
 subsubsection beforeskip=-0.666\baselineskip\@minus -2\p@,
 subsubsection afterskip=0.333\baselineskip \@minus 2\p@,
 paragraph numbering=none,
 paragraph font-family=\rmfamily,
 paragraph font-size=\normalfont,
 paragraph font-shape=\normalfont,
 paragraph font-weight=\normalfont,
 paragraph indent=-1em,
 paragraph align=\raggedright,
 paragraph beforeskip=\z@,
 paragraph afterskip=0\p@,}}

\cxset{octavo headings}
\renewsection\renewsubsection\renewsubsubsection\renewparagraph
\section{Octavo Class Heading}
\lorem
\subsection{Octavo subsection}
This is some text short text\par
\subsubsection{Octavo sub-subsection}
\lorem
\paragraph{paragraph heading} This is some short text.
\end{example}

\begin{example}{}{}
\cxset{octavo}
\section{Octavo Class Heading}
\lorem
\subsection{Octavo subsection}
\lorem
\subsubsection{Octavo sub-subsection}
\lorem
\paragraph{paragraph heading} This is some short text.
\lorem
\paragraph{paragraph heading} This is some short text.
\lorem
\end{example}



\begin{example}{\LaTeXe\ book class headings settings}{ex:latex}
\cxset{book/.style={
 section numbering=arabic,
 section font-size=\Large,
 section font-weight=\bfseries,
 section font-family=\rmfamily,
 section font-shape=\normalfont,
 section align=\raggedright,
 %section numbering custom=\color{gray}{Section} (\thechapter-\@arabic\c@section),
 subsection font-size=\large,
 section indent=0em,
 section beforeskip=-3.5ex \@plus -1ex\@minus -0.2ex,
 section afterskip=2.3ex\@plus.2ex,
 subsection font-size=\large,
 subsection font-shape=\normalfont,
 subsection font-weight=\bfseries,
 subsection numbering=arabic,
 subsection indent=0pt,
 subsection beforeskip=-3.5ex \@plus -1ex\@minus -0.2ex,
 subsection afterskip= 1.5ex \@plus .2ex,
}}
\cxset{book}
\renewsubsection
\section{LaTeX Book  Class Heading}
\lorem
\subsection{A subsection}
\lorem
\end{example}

\section{Grid example}

One problem sometimes is that the sectioning commands create problems with grid layouts. Example~\ref{ex:grid} shows example settings.

\begin{example}{Section styles from the grid package}{ex:grid}
\cxset{grid/.style={
 section numbering=arabic,
 section font-size=\normalsize,
 section font-weight=\bfseries\mathversion{bold},
 section font-family=\rmfamily,
 section font-shape=\normalfont\bfseries\mathversion{bold},
 section beforeskip=-.999\baselineskip,
 section afterskip=0.001\baselineskip,
 section align=\raggedright,
 %section numbering custom=\color{gray}{Section} (\thechapter-\@arabic\c@section),
 subsection font-size=\normalsize,
 section indent=0em,
% section beforeskip=-3.5ex \@plus -1ex\@minus -0.2ex,
 %section afterskip=2.3ex\@plus.2ex,
 subsection font-shape=,
 subsection font-weight=\bfseries\mathversion{bold},
 subsection numbering=arabic,
 subsection indent=0pt,
 subsection beforeskip=\baselineskip,
 subsection afterskip= -.35\baselineskip,
% subsub section
 subsubsection font-shape=\itshape,
 subsubsection font-weight=\bfseries\mathversion{bold},
 subsubsection numbering=numeric,
 subsubsection indent=0pt,
 subsubsection beforeskip=\baselineskip,
 subsubsection afterskip= -.35\baselineskip,
}}
\cxset{grid}
\renewsubsection
\begin{multicols}{2}
\section{Grid  Class Heading}
\lorem
\subsection{Grid  subsection.}
\lorem
\subsubsection{A subsection grid.}
\lorem
\subsubsection{Another subsection grid.}
\lorem
\end{multicols}
\end{example}



The key \option{\bfseries section numbering custom}=\marg{code} is quite powerfull and can be used to define any type of section number style. Just remember that the numbering so far depends on two counters, the c@chapter and c@section. What the section numbering does, it redefines the macro \cs{thesection} to the new definition provided as argument for the key.

Although the temptation to define a lot of key combinations one would rather define new styles as a more user friendly approach.

\cxset{section numbering=arabic, section align=\raggedright, section font-shape=\upshape, section font-family=\rmfamily}
\section{Handling Other Section Levels}

Other sectioning commands such as \cs{subsubsection}, \cs{paragraph} and \cs{subparagraph} have equivalent keys. Examples can be found in the chapters that follow for specific styles.

\section{Technical discussion}

The standard LaTeX classes, book report and article have sections showing dot leaders, whereas in the article class the sections are shown without the dotted lines, as the l@section macro is redefined for articles.

\index{macros!\textbackslash @seccntformat}

\subsection{Indexing of Lower Section Headings}
\LaTeXe\ offers two pathways in redefining section commands, the first one is @startsection and the second is \cs{@seccntformat} \index{sectioning macros}. It also uses the macro \cs{secdef} to create the starred and unstarred versions of the sectioning commands.

\begin{tcolorbox}{}
\begin{lstlisting}
% \begin{macro}{\l@section}
%    In the article document class the entry in the table of contents
%    for sections looks much like the chapter entries for the report
%    and book document classes.
%
%    First we make sure that if a pagebreak should occur, it occurs
%    \emph{before} this entry. Also a little whitespace is added and a
%    group begun to keep changes local.
% \changes{v1.0h}{1993/12/18}{Replaced -\cs{@secpenalty} by
%    \cs{@secpenalty}.  ASAJ.}
% \changes{v1.2i}{1994/04/28}{Don't print a toc line when the tocdepth
%    counter is less than 1.}
% \changes{v1.4a}{1998/10/12}{we should use \cs{@tocrmarg}; see PR/2881.}
%    \begin{macrocode}
%<*article>
\newcommand*\l@section[2]{%
  \ifnum \c@tocdepth >\z@
    \addpenalty\@secpenalty
    \addvspace{1.0em \@plus\p@}%
%    \end{macrocode}
%
%    The macro |\numberline| requires that the width of the box that
%    holds the part number is stored in \LaTeX's scratch register
%    |\@tempdima|. Therefore we put it there. We begin a group, and
%    change some of the paragraph parameters (see also the remark at
%    \cs{l@part} regarding \cs{rightskip}).
%    \begin{macrocode}
    \setlength\@tempdima{1.5em}%
    \begingroup
      \parindent \z@ \rightskip \@pnumwidth
      \parfillskip -\@pnumwidth
%    \end{macrocode}
%    Then we leave vertical mode and switch to a bold font.
%    \begin{macrocode}
      \leavevmode \bfseries
%    \end{macrocode}
%    Because we do not use |\numberline| here, we have do some fine
%    tuning `by hand', before we can set the entry. We discourage but
%    not disallow a pagebreak immediately after a section entry.
%    \begin{macrocode}
      \advance\leftskip\@tempdima
      \hskip -\leftskip
      #1\nobreak\hfil \nobreak\hb@xt@\@pnumwidth{\hss #2}\par
    \endgroup
  \fi}
%</article>
\end{lstlisting}
\end{tcolorbox}

As you can see the dot leaders are not present in the above definition. Although we can get rid of dot leaders in other section by redefining them, it is not as easy to add them back.

As our aim is to be able to have all the classes used a common denominator we can define a command as follows (using book as a base)

\begin{tcolorbox}{}
\begin{lstlisting}
\def\articlesection{
\newcommand*\l@section[2]{%
  \ifnum \c@tocdepth >\z@
    \addpenalty\@secpenalty
    \addvspace{1.0em \@plus\p@}%
    \setlength\@tempdima{1.5em}%
    \begingroup
      \parindent \z@ \rightskip \@pnumwidth
      \parfillskip -\@pnumwidth
      \leavevmode \bfseries
      \advance\leftskip\@tempdima
      \hskip -\leftskip
      #1\nobreak\hfil \nobreak\hb@xt@\@pnumwidth{\hss #2}\par
    \endgroup
  \fi}
}
\end{lstlisting}
\end{tcolorbox}

%\articlesection

The \cs{@starredsection} macro is one of those locomotive type of commands. It takes 7 required arguments and 2 optional ones and hidden within it are two booleans. The full set looks like this:

\cs{@startsection} \marg{name} \marg{level} \marg{indent} \marg{beforeskip} \marg{afterskip} \marg{style}[*]
  [\marg{altheading}]\marg{heading}.

\begin{marglist}
\item[name] The name of the level command.
\item [level] A number denoting the depth of the section, chapter=1, section=2, etc. A section number will be printed only if \marg{level} is equal or smaller than the value of \textit{secnumdepth}
\item[indent] The indentation of the heading from the left margin.
\item[beforeskip]  The absolute value of this argument is the skip to leave above the heading. If it is negative, then the paragraph indent of the text following the heading is suppressed.
\item [afterskip] If positive, it is the skip to leave below the heading, else it is the skip to the right of a run-in heading.
\item [style] Sets the style of the heading.
\item[\textup{[*]}] When this is missing the heading is numbered and the corresponding counter is incremented.
\item[\textup{[\textit{altheading}]}] Gives an alternative heading to use in the table of contents and in the running heads. This should be present when the * form is used.
\item[heading] The heading of the new section.
\end{marglist}

\begin{example}{Example formatting run-in section}{}
\makeatletter
\bgroup
\renewcommand\section{%
    \@startsection{section}%
    {1}%
    {0em}%
    {-0.8em}%
    {-0.5em}%
    {\large\normalfont\scshape}}
\makeatother
\section[]{test}
\lorem
\egroup
\end{example}

Note we run the example in a group so that we will not influence the formatting of this document.

As mentioned earlier there is an additional way to introduce formatting for sections and this is using the command \cs{@seccntformat}, which is responsible for typesetting the counter part of a section title. The default definition of the command typesets the \cs{the} representation of the section counter.

\begin{example}{}{}
\bgroup
\renewcommand\section{%
    \@startsection{section}%
    {1}%
    {0em}%
    {-0.8em}%
    {-0.5em}%
    {\large\normalfont\scshape}}
\renewcommand\@seccntformat[1]{\fbox
{\csname the#1\endcsname}\hspace{0.5em}}
\makeatother
\section[]{test}\label{sec:ok}
\lorem

See section \ref{sec:ok}.
\egroup
\end{example}

The definition of \cs{@seccntformat} applies to all headings
defined with the \cs{@startsection} command (which is described in the next
section). Therefore, if you wish to use different definitions of \cs{@seccntformat}
for different headings, you must put the appropriate code into every heading
definition.

\begin{tcolorbox}
\begin{lstlisting}
\def\@seccntformat##1{\csname the##1\endcsname{}}
\end{lstlisting}
\end{tcolorbox}

\section{Custom headings}

It is also possible to define section headings without resorting to any of the above. To do this.

\begin{tcolorbox}
\begin{lstlisting}
\newcommand\part{\secdef\cmda\cmdb}
\end{lstlisting}
\end{tcolorbox}

the part and chapter and sometimes appendix are defined this way, but nothing stops us from doing the same for other sections. A generic section command can be defined as follows:

\begin{example}{}{}
\bgroup
\renewcommand\section[2] [?]{% % Complex form:
\refstepcounter{section}% % step counter/ set label
\addcontentsline{toc}{appendix}% % generate toe entry
{\protect\numberline{section-\thesection}#1}%
{\raggedright\large\bfseries section %\appendixname\ % typeset the title
\thesection\par \centering#2\par}% % and number
\sectionmark{#1}% % add to running header
\@afterheading % prepare indentation handling
%\addvspace{\baselineskip}
}
\section{Test}
\lorem
\egroup
\end{example}

Many other strategies can also be implemented that are perhaps easier to grasp.

\begin{example}{}{}
\bgroup
\def\strut{\vrule height12pt depth1pt width0pt}
\renewcommand\section[2] []{% % Complex form:
\refstepcounter{section}% % step counter/ set label
\addcontentsline{toc}{section}% % generate toc entry
{\protect\numberline{\thesection} }%
{\raggedright\large\bfseries\scshape %
\parbox[b]{\dimexpr(\linewidth-0.5\columnsep)}{\colorbox{brown!80}%
{{\vbox{\strut\raise2pt\hbox{#2}}}}}}\vskip0pt% % and number
\sectionmark{#1}% % add to running header
\@afterheading % prepare indentation handling
\vspace{\dimexpr\baselineskip+6pt}%must have a parameter
}
\chapter{Fossil Insects}
\begin{multicols*}{2}\raggedcolumns
\section[Insect Fossilization]{\raggedright \thinspace Insect Fossilization}
\lipsum[1]
\end{multicols*}
\egroup
\end{example}
% To answer http://tex.stackexchange.com/questions/52998/change-title-to-small-caps-but-not-in-toc

Of course some work is needed to center the text properly in the middle of the colour box. For all practical purposes it is lining up as per the sample.

In Chapter we discussed a forward, but this may not apply if there are no chapters or we need to treat these as sections, the example \ref{ex:forwardsection} shows such a method.

\begin{example}{Defining a Foreward Section}{ex:forwardsection}

\newcommand\prematter@sp[1]{% % Complex form:
%\refstepcounter{section}% % step counter/ set label
\addcontentsline{toc}{section}% % generate toe entry
{\protect\numberline{}\textsc{#1}}%
\sectionmark{#1}% % add to running header
{\LARGE\centering\normalfont\sffamily\colorbox{brown!80}{ \textsc{#1}}\par}%
\@afterheading % prepare indentation handling
\addvspace{\baselineskip}
\@afterindentfalse
}

\newenvironment{prematter}[1]{%
   \prematter@sp{#1}}
{}
\begin{multicols}{2}
\label{theok}
\begin{prematter}{Foreward}
\lipsum[1]
\end{prematter}\ref{theok}
\end{multicols}
\end{example}

\section{underlining}

I am aware that some people have no choice but have some sections underlined as dictated by archaic regulations in some establishments for thesis submission. If nobody is forcing you to underline it is best to avoid it. We use Donald Arsenau's ulem package to achieve underlining.

\makeatletter
%\@debugtrue
\makeatother
%\parskip1ex 
\newfontfamily\tibetan{TibMachUni.ttf}
\def\deva{{\protect\tibetan\symbol{"0F7C} xx ༃}}

\chapter{Indices}
\pagebreak

\starttemplate{kroll}
\thispagestyle{empty}
    \begin{leftcolumn}
       \begin{center} 
          \huge \noindent PREPARING\\
                   INDEXES
       \end{center}
     
      \medskip

       {\justifying \small\noindent And in such indexes, although small pricks\\
To their subsequent volumes, there is seen\\
The baby figure of the giant mass\\
Of things to come at large. \par
\hfill \textit{--Shakespeare}\par
\hfill\hfill{ \RaggedRight from \textit{Troilus and Cressida}}}
\medskip
       \putimage[width=1.0\linewidth]{./images/animalium01.jpg}\par
       \aheader{Handwritten cards compiled by  Sherborn for his publication \textit{Index Animalium}}
   \end{leftcolumn}
   \begin{rightcolumn}
       \putimage[width=\linewidth]{./images/animalium.jpg}
       \onelinecaption{{\resizebox{\linewidth}{5.5pt}{\bfseries Sherborn’s `Index  Animalium'. Credit Natural History Museum, London\footnote{This monumental publication became the basis for all zoological nomenclature work having gathered together all the relevant data in one place, just as an online database does today.}}}\par}
%  \centerline{\onelineheader{TYPESETTING AN INDEX}}
      \begin{multicols}{2}
        
\parindent1em      \lettrine{T}{he} first English Language index, appeared in Christopher Marlowe's \textit{Hero and Leander} in 1593. At that period, as often as not, by an ``index to a book'' was meant what we should now call a table of contents. Among the first indexes---in the modern sense---to a book in the English language was one in Plutarch's Parallel Lives, in Sir Thomas North's 1595 translation\footnote{Borko, Harold \& Bernier, Charles L. (1978). \textit{Indexing Concepts and Methods}, ISBN 0-12-118660-1.}.  

A section entitled ``An Alphabetical Table of the most material contents of the whole book'' may be found in Henry Scobell's Acts and Ordinances of Parliament of 1658. This section comes after ``An index of the general titles comprised in the ensuing Table''. Both of these indexes predate the index to Alexander Cruden's Concordance (1737) see \citep{farrow96}, which is erroneously held to be the earliest index found in an English book.

      \end{multicols}
   \end{rightcolumn}
\stoptemplate

\pagestyle{headings}


\index{Quantum Mechanics>History|(}
\setlength{\columnsep}{2em}
\begin{multicols}{2}

\section{Preparing an index}

In order to produce an index, we need to load the
package \pkg{makeidx}  and immediately issue the command \cmd{\makeindex}.\footnote{You will also need to at least run the file once using |MakeIndex| on |MikTex|. Check your distribution if you getting problems.}
At the place where
we want the index to be printed,we use \cmd{\printindex}.
\parindent1em

In \latex, we include a word
in the index by using the command \cmd{\index}\meta{arg}, so if the word Kroll should be included in
the index, we should use the command |\index{Kroll}|.

If the word is to be printed in bold, we use

% : = |
% = = @
% \catcode `|
\bgroup
 \catcode`|=11
\gdef\idxmain#1{%
   \def\idxmainentryi##1##2##3;{%
      \index{Kr0=\textbf{#1}|textbf}
    }
   \idxmainentryi#1;    
}  


\egroup

\idxmain{Kroll}

\index{Kroll=\textbf{Kroll}>Leon Kroll}


\emphasis{usepackage,makeidx,makeindex,index,printindex}
\begin{teXXX}
\documentclass{article}
\usepackage{makeidx}
\makeindex
\begin{document}
   To prepare an index, just include the
   package\index{package}

\printindex
\end{document}
\end{teXXX}


There is a  special character |\@| is used to denote that what appears on its left side must be
typeset as it appears on its right side.  the first occurrence of perl will also be used
by the sorting algorithm. is is very useful since what is used for sorting and what
will be printed may be different! For example,we may want to have the name 
\texttt{Donald Knuth}  under the letter K., we should write


\verb+\index{Knuth@{Donald Knuth}}+

Another thing we may want to change is the way that the page number is typeset.
If we want, for example, to have the page number in bold, we would write
\verb+\index{perl|textbf}+.  Notice that we wrote "textbf"  without the backslash. Of course,
the above can be combined with the  command


\verb+\index{perl@\textbf{perl}|textit}+


will print the word perl in the index (the entry will be typeset in boldface type) sorted
as "perl",’ and its page number will be italic. A common application of this is through
the command . If we want to send the reader to another index entry, say, to send
the reader from the \verb+$\omega$+ to the \verb+$\Omega$+ command, we can write

\verb+\index{omega@$\omega$|see{$\Omega$}}+


Here, we ask for the entry to be sorted according to the word omega and, in its place,
the program must use 

\begin{teX}
$\omega$|see{$\Omega$}
\end{teX}

If a word is used repeatedly in a range of pages and we want to have this range
in the index, we do not write the relative \texttt{index} command all of the time. Instead,
we write \verb+\index\{convex\|(\}+  at the place where we have the first occurrence and
\verb+\index\{convex|)\}+  at the place where we have the last occurrence. This will produce a
page range in the index for the word "convex".

\subsection{Subindices}
Subindices can be produced using an exclamation mark. If we want the word `Zeus'
to appear in the category of Greek which is in the category of Gods, we will write:

\verb+\index{Gods!Greek!Zeus}+

The actual symbol used will depend on the |.ist| file used when the file is compiled. For example in the |ltxdoc| class this is redefined as |>|.
\DeclareRobustCommand\textat{%
  \bgroup\makeatother \egroup
}


\robustify{\ttdefault}
\robustify{\ttfamily}
\robustify{\color}

 \makeatletter
\def\IDX#1#2{%
   \def\xx{\expandafter\@gobble\string#2}
   \index{\bgroup\small\texttt{\textbackslash #1}>\small\texttt{\textbackslash \xx}\egroup}
}
 
 \makeatother

\IDX{chapter}{\@introduction}
\IDX{chapter}{\intro@duction}
\IDX{abr}{\@some@other}

\meaning\ttfamily \\
\meaning\ttdefault\\

\section{Multiple Pages}

To perform multi-page indexing, add a |( and |) to the end of the \cmd{\index} command, as in 
\index{Indexing>multi-page}.

{\small
\verb+\index{Quantum Mechanics!History|(}+

\narrower\narrower
In 1901, Max Planck released his theory of radiation dependant 
on quantized energy. While this explained the ultraviolet catastrophe
 in the spectrum of blackbody radiation, this had far larger consequences 
as the beginnings of quantum mechanics.\ldots

\verb+\index{Quantum Mechanics!History|)}+
}

\end{multicols}

\index{Quantum Mechanics>History|)}


\section{Summary of commands}

\begin{tabular}{lll}
\toprule
Example	&Index Entry	&Comment\\
\midrule
\textbackslash index\{hello\}	          &hello, 1	&Plain entry\\
\textbackslash index\{hello!Peter\}	      &Peter, 3	&Subentry under 'hello'\\
\textbackslash index\{Sam@\textbackslash textsl\{Sam\}\}	&Sam, 2	&Formatted entry\\
\textbackslash index\{Lin@\textbackslash textbf\{Lin\}\}	&\textbf{Lin}, 7	&Same as above\\
\textbackslash index\{Jennytextbf\}	     &Jenny, 3	&Formatted page number\\
\textbackslash index\{Joe textit\}	&Joe, 5	          &Same as above\\
\textbackslash index\{ecole@\'ecole\}	&école, 4	&Handling of accents\\
\textbackslash index\{Peter see\{hello\}\}	&Peter, see hello	&Cross-references\\
\textbackslash index\{Jen see also\{Jenny\}\}	&Jen, see also Jenny	 &Same as above\\
\bottomrule
\end{tabular}

\section{Indexing Class Documentation}

\index{Indexing=\textbf{Indexing}}
\index{Indexing>general}
\index{Indexing>doc}

Indexing \latex2e classes is normally achieved using the |doc| and |docstrip| program, which they are not so clear with examples, if you need to deviate from the standard methods. The important thing here to remember is that you need to use different characters |=| |>| |*|.

\begin{tabular}{ll}
\toprule
normal    & doc \\
\midrule
\string @ & \texttt{=} \\
\string ! & \texttt{>}\\
\bottomrule
\end{tabular}

\begin{verbatim}
\index{Indexing=\textbf}
\index{Indexing>general}
\index{Indexing>doc}
\end{verbatim}

This manual, was build using a large |ltxdoc| class and these problems appeared while I was developing it. As normal with such problems, they were very time consuming to debug. There are still issues in some parts and one day, I am hoping to come back and correct them. One needs at this point to query the need to use the |doc| and |docstrip| method of documenting macros and if it shouldn't have a pre-processor written in a higher language to ease development. 

\section{Customization}\index{Indexing>customizing}

When creating an index with \pkgname{makeindex} one can create a \docfile{sample.ist} file that can be used together with the |makeidx| program to customize the way the index will look.

\begin{verbatim}
heading_prefix "{\\bfseries\\hfil "
heading_suffix "\\hfil}\\nopagebreak\n"
headings_flag 1
delim_0 "\\dotfill"
delim_1 "\\dotfill"
delim_2 "\\dotfill"
\end{verbatim}

This will write the first alphabet symbol in bold font, and uses dots as delimiters. This file is generally  used jointly with \texttt{makeindex} using

\verb|makeindex -s sample.ist filename.idx|

where |filename.idx| has been craeted by executing |latex| or one of the other engine commands such as |pdflatex| on |filename.tex|.


According to \citep{gabora}, you may use

\begin{verbatim}
sort_rule "\." "\b\."
sort_rule "\:" "\b\:"
sort_rule "\," "\b\,"
\end{verbatim}


\section{Writing custom indexing commands}

For complex documents it is easier to write a number of macros to assist with indexing and to also provide consistency. For example if you want to index the Devanagari alphabet we might need to get quite creative as to how to both index it as well as get the symbols in the index.
\DeclareRobustCommand\ta{{\tibetan ༃ }}

%\begin{texexample}{Writing Indexing Commands}{ex:zs}
%\gdef\ZZs#1{\incsyms%
%   \indexcommand[\string#1]{#1}%
%   #1}
%\DeclareRobustCommand\ta{{\tibetan ༃}\xparse}
%\ta
%\end{texexample}
%
%The command |\ZZs{\ta}| typesets the command in the document, as (\ZZs{\ta}) and also adds it to the index and typesets the symbol.
%
%
%\begin{phdverbatim}
%\def\ZZs#1{\incsyms%
%   \indexcommand[\string#1]{#1}
%   \string#1}
%\end{phdverbatim}
























%\chapter{Unicode Math}
\tcbdocmarginnote{N 29-06-2018}
Unicode contains separate codepoints for most if not all variations of alphabet
shape one may wish to use in mathematical notation. The complete list is shown
in table 5. Some of these have been covered in the previous sections.
The math font switching commands do not nest; therefore if you want sans
serif bold, you must write |\mathbfsf{...}| rather than |\mathbf{\mathsf{...}}|.
This may change in the future.

\section{Unicode maths font setup}

The promise of Unicode is that all symbols and alphabetic variants are in one font. The \pkgname{unicode-math}
maps all the available unicode math characters of a math font to respective \latex commands. If you have patience you can actually input them directly from the keyboard rather than in commands.

The best advice that I can give you is to read the \pkgname{unicode-math} carefully. 

\begin{docCommand} {setmathfont} { \oarg{range=\meta{unicode range}, \meta{font features } } \marg{font name} }
In many cases using one font might not be adequate. Specific Unicode ranges can be assigned to separate fonts.
\end{docCommand}

\subsection{Control over maths alphabets}

\subsection{Math `versions'}

\subsection{Maths input}

\subsection{Math `style'}

\subsubsection{Bold style}

Similar as in the previous section, ISO standards differ somewhat to \tex’s conventions
(and classical typesetting) for ‘boldness’ in mathematics. In the past, it has
been customary to use bold upright letters to denote things like vectors and matrices.


$$\boldsymbol{\omega} \times \mathbf{T} = \mathbf{T'}$$

$$\mathbf{\omega} = {1\over 2}\kappa \mathbf{B} + {1\over 2}(\kappa \mathbf{B} + \tau \mathbf{T}) + {1\over 2}\tau \mathbf{T} = \kappa \mathbf{B} + \tau \mathbf{T}
$$

\[
\mathbf{e} = \frac{\mathbf{A}}{m k} = \frac{1}{m k}(\mathbf{p} \times \mathbf{L})
\]

\subsubsection{Sans serif style}

\subsubsection{Blackboard or double-struck}



\subsubsection{Caligraphic and Script variants}



\section{Growing and non-growing accents}

This are the most problematic with Unicode fonts.

%%%%%%%% INPUT INTEGRAL FILES %%%%%%%%%%
%%%%%%%%%%%%%%%%%%%%%%%%%%%%%%%
 %% Too many errors here need to revisit.
 \subsection{Delimiters}
 \begin{multicols}{2}
% \showmbrace/{002F}{}
% \showlbrace({0028}{}
% \showlbrace[{005B}{}
% \showlbrace\lbrace{007B}{}
% \showmbrace\backslash{005C}{}
% \showrbrace){0029}{}
% \showrbrace]{005D}{}
% \showrbrace\rbrace{007D}{}
% \showlbrace\lceil{2308}{}
% \showlbrace\lfloor{230A}{}
% \showlbrace\lmoustache{23B0}{*}
% \showlbrace\lbrbrak{2772}{*}
% \showlbrace\lBrack{27E6}{*}
% \showlbrace\langle{27E8}{}, \cmd<
% \showlbrace\lAngle{27EA}{*}
% \showlbrace\lgroup{27EE}{*}
% \showlbrace\lBrace{2983}{*}
% \showlbrace\lParen{2985}{*}
% \showrbrace\rceil{2309}{}
 %\showrbrace\rfloor{230B}{}
% \showrbrace\rmoustache{23B1}{*}
% \showrbrace\rbrbrak{2773}{*}
% \showrbrace\rBrack{27E7}{*}
% \showrbrace\rangle{27E9}{}, \cmd>
% \showrbrace\rAngle{27EB}{*}
% \showrbrace\rgroup{27EF}{*}
 %\showrbrace\rBrace{2984}{*}
% \showrbrace\rParen{2986}{*}
 \end{multicols}

 \begin{multicols}{2}
% \showmbrace\vert{007C}{}, \cmd|
 \showmbrace\Vert{2016}{*}, \cmd\|
 \showmbrace\Vvert{2980}{}
 \showmbrace\uparrow{2191}{}
 \showmbrace\downarrow{2193}{}
 \showmbrace\updownarrow{2195}{}
 \showmbrace\Uparrow{21D1}{}
 \showmbrace\Downarrow{21D3}{}
 \showmbrace\Updownarrow{21D5}{}
% \showmbrace\Uuparrow{290A}{*}
% \showmbrace\Ddownarrow{290B}{*}
% \showmbrace\UUparrow{27F0}{*}
% \showmbrace\DDownarrow{27F1}{*}
% \showmbrace\arrowvert{XXXX}{}
% \showmbrace\Arrowvert{XXXX}{}
% \showmbrace\bracevert{XXXX}{*}
 \end{multicols}

 \subsection{Other bracess}
 \begin{multicols}{2}
 \showsymbol\ulcorner{231C}{*}
 \showsymbol\urcorner{231D}{*}
 \showsymbol\llcorner{231E}{*}
 \showsymbol\lrcorner{231F}{*}
 \showsymbol\Lbrbrak{27EC}{*}
 \showsymbol\Rbrbrak{27ED}{*}
 \showsymbol\llparenthesis{2987}{*}
 \showsymbol\rrparenthesis{2988}{*}
 \showsymbol\llangle{2989}{*}
 \showsymbol\rrangle{298A}{*}
 \showsymbol\lbrackubar{298B}{*}
 \showsymbol\rbrackubar{298C}{*}
 \showsymbol\lbrackultick{298D}{*}
 \showsymbol\rbracklrtick{298E}{*}
 \showsymbol\lbracklltick{298F}{*}
 \showsymbol\rbrackurtick{2990}{*}
 \showsymbol\langledot{2991}{*}
 \showsymbol\rangledot{2992}{*}
 \showsymbol\lparenless{2993}{*}
 \showsymbol\rparengtr{2994}{*}
 \showsymbol\Lparengtr{2995}{*}
 \showsymbol\Rparenless{2996}{*}
 \showsymbol\lblkbrbrak{2997}{*}
 \showsymbol\rblkbrbrak{2998}{*}
 \showsymbol\lvzigzag{29D8}{*}
 \showsymbol\rvzigzag{29D9}{*}
 \showsymbol\Lvzigzag{29DA}{*}
 \showsymbol\Rvzigzag{29DB}{*}
 \showsymbol\lcurvyangle{29FC}{*}
 \showsymbol\rcurvyangle{29FD}{*}
 \showsymbol\lbrbrak{2772}{*}
 \showsymbol\rbrbrak{2773}{*}
 \showsymbol\lbag{27C5}{*}
 \showsymbol\rbag{27C6}{*}
 \showsymbol\Lbrbrak{27EC}{*}
 \showsymbol\Rbrbrak{27ED}{*}
 \end{multicols}

 \section{Alphabetics}
 \begin{multicols}{2}
\showsymbolalpha\Gamma{0393}{}
\showsymbolalpha\Delta{0394}{}
\showsymbolalpha\Theta{0398}{}
\showsymbolalpha\Lambda{039B}{}
\showsymbolalpha\Xi{039E}{}
\showsymbolalpha\Pi{03A0}{}
\showsymbolalpha\Sigma{03A3}{}
\showsymbolalpha\Upsilon{03A5}{}
\showsymbolalpha\Phi{03A6}{}
\showsymbolalpha\Psi{03A8}{}
\showsymbolalpha\Omega{03A9}{}
\showsymbolalpha\alpha{03B1}{}
\showsymbolalpha\beta{03B2}{}
\showsymbolalpha\gamma{03B3}{}
\showsymbolalpha\delta{03B4}{}
\showsymbolalpha\epsilon{03B5}{}
\showsymbolalpha\zeta{03B6}{}
\showsymbolalpha\eta{03B7}{}
\showsymbolalpha\theta{03B8}{}
\showsymbolalpha\iota{03B9}{}
\showsymbolalpha\kappa{03BA}{}
\showsymbolalpha\lambda{03BB}{}
\showsymbolalpha\mu{03BC}{}
\showsymbolalpha\nu{03BD}{}
\showsymbolalpha\xi{03BE}{}
\showsymbolalpha\pi{03C0}{}
\showsymbolalpha\rho{03C1}{}
\showsymbolalpha\sigma{03C3}{}
\showsymbolalpha\tau{03C4}{}
\showsymbolalpha\upsilon{03C5}{}
\showsymbolalpha\phi{03D5}{}
\showsymbolalpha\chi{03C7}{}
\showsymbolalpha\psi{03C8}{}
\showsymbolalpha\omega{03C9}{}
\showsymbolalpha\varepsilon{03F5}{}
\showsymbolalpha\vartheta{03D1}{}
\showsymbolalpha\varpi{03D6}{}
\showsymbolalpha\varrho{03F1}{}
\showsymbolalpha\varsigma{03C2}{}
\showsymbolalpha\varphi{03C6}{}
\showsymbolalpha\nabla{2207}{}
\showsymbolalpha\partial{2202}{}
\showsymbolalpha\imath{1D6A4}{}
\showsymbolalpha\jmath{1D6A5}{}
 \end{multicols}
% \subsection{Accents}
 \begin{multicols}{2}
 \showaccent\grave{0300}{}
 \showaccent\acute{0301}{}
 \showaccent\hat{0302}{}
 \showaccent\tilde{0303}{}
 \showaccent\bar{0304}{}
 \showaccent\breve{0306}{}
 \showaccent\dot{0307}{}
 \showaccent\ddot{0308}{}
 \showaccent\ovhook{0309}{}
 \showaccent\mathring{030A}{}
 \showaccent\check{030C}{}
 \showaccent\candra{0310}{}
 \showaccent\oturnedcomma{0312}{}
 \showaccent\ocommatopright{0315}{}
 \showaccent\droang{031A}{}
 \showaccent\leftharpoonaccent{20D0}{}
 \showaccent\rightharpoonaccent{20D1}{}
 %\showaccent\leftarrowaccent{20D6}{}
 \showaccent\vec{20D7}{}, \cmd\rightarrowaccent
 %\showaccent\leftrightarrowaccent{20E1}{}
 \showaccent\dddot{20DB}{}
 \showaccent\ddddot{20DC}{}
 \showaccent\annuity{20E7}{}
 \showaccent\widebridgeabove{20E9}{}
 \showaccent\asteraccent{20F0}{}
 \end{multicols}

 \begin{multicols}{2}
 \showwideaccent\widehat{0302}{*}
 \showwideaccent\widetilde{0303}{*}
% \showwideaccent\widecheck{030C}{*}
 \showwideaccent\overleftarrow{20D6}{}
 \showwideaccent\overrightarrow{20D7}{}
 \showwideaccent\underrightarrow{20EF}{}
 \showwideaccent\underleftarrow{20EE}{}
 \showwideaccent\overleftrightarrow{20E1}{}
 \showwideaccent\underleftrightarrow{034D}{}
% \showwideaccent\overleftharpoon{20D0}{}
% \showwideaccent\overrightharpoon{20D1}{}
% \showwideaccent\underleftharpoon{20EC}{}
% \showwideaccent\underrightharpoon{20ED}{}
 \end{multicols}

 OpenType STIX fonts include a number of under accents that can be used in
 math mode, but \TeX\ does not support under accents natively so such glyphs
 can not be used directly. Under accents can be set using regular accents and
 commands like |\underaccent| from the \pkgname{accents} package, for example
 gives |\(\underaccent{\hat}{X}\)|. The
 \pkgname{undertilde} package provides |\utilde| for extensible under tilde
 accent \citep{undertilde}.
 
 
 % \subsection{Over and under brackets}
 \begin{multicols}{2}
 %\showover\overbracket{23B4}{} conflict mathtools consider remove mathtools for UNICODE
 \showover\overparen{23DC}{}
 \showover\overbrace{23DE}{}
 \showover\underbracket{23B5}{} %conflict mathtools?
 \showover\underparen{23DD}{}
 \showover\underbrace{23DF}{}
 \end{multicols}

 \subsection{Radicals}
 \begin{multicols}{2}
 \showaccent\sqrt{221A}{}
 \showaccent\longdivision{27CC}{*}
 \end{multicols}
 
 
 
 
 
 
 \subsection{Big operators}
 \begin{multicols}{2}
 \showop\Bbbsum{2140}{}
 \showop\prod{220F}{}
 \showop\coprod{2210}{}
 \showop\sum{2211}{}
 \showop\bigwedge{22C0}{}
 \showop\bigvee{22C1}{}
 \showop\bigcap{22C2}{}
 \showop\bigcup{22C3}{}
 \showop\leftouterjoin{27D5}{*}
 \showop\rightouterjoin{27D6}{*}
 \showop\fullouterjoin{27D7}{*}
 \showop\bigbot{27D8}{*}
 \showop\bigtop{27D9}{*}
 \showop\xsol{29F8}{*}
 \showop\xbsol{29F9}{*}
 \showop\bigodot{2A00}{*}
 \showop\bigoplus{2A01}{*}
 \showop\bigotimes{2A02}{*}
 \showop\bigcupdot{2A03}{*}
 \showop\biguplus{2A04}{*}
 \showop\bigsqcap{2A05}{*}
 \showop\bigsqcup{2A06}{*}
 \showop\conjquant{2A07}{*}
 \showop\disjquant{2A08}{*}
 \showop\bigtimes{2A09}{*}
 \showop\modtwosum{2A0A}{*}
 \showop\Join{2A1D}{*}
 \showop\bigtriangleleft{2A1E}{*}
 \showop\zcmp{2A1F}{*}
 \showop\zpipe{2A20}{*}
 \showop\zproject{2A21}{*}
 \showop\biginterleave{2AFC}{}
 \showop\bigtalloblong{2AFF}{*}
 \end{multicols}
 
 \section{General manipulations test.}
 
 Here are some examples using big operators

$$\sum_{n=s}^t C\cdot f(n) = C\cdot \sum_{n=s}^t f(n),$$ where ``'C'' is a constant

\[\sum_{n=s}^t f(n) + \sum_{n=s}^{t} g(n) = \sum_{n=s}^t \left[f(n) + g(n)\right]\] 

\[\sum_{n=s}^t f(n) - \sum_{n=s}^{t} g(n) = \sum_{n=s}^t \left[f(n) - g(n)\right]\] 

\[\sum_{n=s}^t f(n) = \sum_{n=s+p}^{t+p} f(n-p) \] 

\[\sum_{n\in B} f(n) = \sum_{m\in A} f(\sigma(m)),\] for a bijection σ from a finite set ``$A$'' onto a finite set ``$B$''; this generalizes the preceding formula.

\[\sum_{n=s}^j f(n) + \sum_{n=j+1}^t f(n) = \sum_{n=s}^t f(n)\] 

\[\sum_{i=k_0}^{k_1}\sum_{j=l_0}^{l_1} a_{i,j} = \sum_{j=l_0}^{l_1}\sum_{i=k_0}^{k_1} a_{i,j}\] 

\[\sum_{k\le j \le i\le n} a_{i,j} = \sum_{i=k}^n\sum_{j=k}^i a_{i,j} = \sum_{j=k}^n\sum_{i=j}^n a_{i,j}\] 


\[\sum_{n=0}^t f(2n) + \sum_{n=0}^t f(2n+1) = \sum_{n=0}^{2t+1} f(n)\] 

\[\sum_{n=0}^t \sum_{i=0}^{z-1} f(z\cdot n+i) = \sum_{n=0}^{z\cdot t+z-1} f(n)\] 

\[\sum_{i=s}^m\sum_{j=t}^n {a_i}{c_j} = \sum_{i=s}^m a_i \cdot \sum_{j=t}^n c_j\] 

\[\sum_{n=s}^t \ln f(n) = \ln \prod_{n=s}^t f(n)\] 

\[c^{\left[\sum_{n=s}^t f(n) \right]} = \prod_{n=s}^t c^{f(n)}\] 
 
 
 \section{Convergence criteria}
 
The product of positive real numbers

\[\prod_{n=1}^{\infty} a_n\]
converges to a nonzero real number if and only if the sum
\[\sum_{n=1}^{\infty} \log(a_n)\]
converges. This allows the translation of convergence criteria for infinite sums into convergence criteria for infinite products. The same criterion applies to products of arbitrary complex numbers (including negative reals) if log is understood as a fixed Complex logarithm which satisfies $\log(1) = 0$, with the proviso that the infinite product diverges when infinitely many ''$a_n$'' fall outside 
the domain of log, whereas finitely many such ''$a_n$'' can be ignored in the sum.

For products of reals in which each $a_n\ge1$, written as, for instance, $a_n=1+p_n$,
where $p_n\ge 0$, the bounds

\[1+\sum_{n=1}^{N} p_n \le \prod_{n=1}^{N} \left( 1 + p_n \right) \le \exp \left( \sum_{n=1}^{N}p_n \right)\]

show that the infinite product converges precisely if the infinite sum of the $p_n$ converges. This relies on the Monotone convergence theorem. More generally, the convergence of $\prod_{n=1}^\infty(1+p_n)$ is equivalent to the convergence of $\sum_{n=1}^\infty p_n$ 
%if ''p<sub>n</sub>'' are real or complex numbers such that <math>\sum_{n=1}^\infty|p_n|^2<+\infty\], since <math>\log(1+x)=x+O(x^2)\] in a neighbourhood of 0.
%
%If the series ''p''<sub>''n''</sub> diverges, then the sequence of partial products converges to zero as a sequence.  The infinite product is said to '''diverge to zero'''.
% 
 
 
 
 
 
 
 
 
%% Relations symbols from STIX font
%% 
  \subsection{Relations}
 \begin{multicols}{2}
% \showrelsymbol*{002A}{}, \cmd\ast
% \showrelsymbol:{003A}{}
 \showrelsymbol{\less}{003C}{}, \cmd\less
 \showrelsymbol{\equal}{003D}{}, \cmd\equal
 \showrelsymbol{\greater}{003E}{}, \cmd\greater
 \showrelsymbol\closure{2050}{*}
 %\showrelsymbol\vertoverlay{20D2}{}
 \showrelsymbol\leftarrow{2190}{}, \cmd\gets
 \showrelsymbol\uparrow{2191}{}
 \showrelsymbol\rightarrow{2192}{}, \cmd\to
 \showrelsymbol\downarrow{2193}{}
 \showrelsymbol\leftrightarrow{2194}{}
 \showrelsymbol\updownarrow{2195}{}
 \showrelsymbol\nwarrow{2196}{}
 \showrelsymbol\nearrow{2197}{}
 \showrelsymbol\searrow{2198}{}
 \showrelsymbol\swarrow{2199}{}
 \showrelsymbol\nleftarrow{219A}{}
 \showrelsymbol\nrightarrow{219B}{}
 \showrelsymbol\leftwavearrow{219C}{}
 \showrelsymbol\rightwavearrow{219D}{}
 \showrelsymbol\twoheadleftarrow{219E}{}
 \showrelsymbol\twoheaduparrow{219F}{}
 \showrelsymbol\twoheadrightarrow{21A0}{}
 \showrelsymbol\twoheaddownarrow{21A1}{}
 \showrelsymbol\leftarrowtail{21A2}{}
 \showrelsymbol\rightarrowtail{21A3}{}
 \showrelsymbol\mapsfrom{21A4}{}
 \showrelsymbol\mapsup{21A5}{}
 \showrelsymbol\mapsto{21A6}{}
 \showrelsymbol\mapsdown{21A7}{}
 \showrelsymbol\hookleftarrow{21A9}{}
 \showrelsymbol\hookrightarrow{21AA}{}
 \showrelsymbol\looparrowleft{21AB}{}
 \showrelsymbol\looparrowright{21AC}{}
 \showrelsymbol\leftrightsquigarrow{21AD}{}
 \showrelsymbol\nleftrightarrow{21AE}{}
 \showrelsymbol\downzigzagarrow{21AF}{}
 \showrelsymbol\Lsh{21B0}{}
 \showrelsymbol\Rsh{21B1}{}
 \showrelsymbol\Ldsh{21B2}{}
 \showrelsymbol\Rdsh{21B3}{}
 \showrelsymbol\curvearrowleft{21B6}{}
 \showrelsymbol\curvearrowright{21B7}{}
 \showrelsymbol\circlearrowleft{21BA}{}
 \showrelsymbol\circlearrowright{21BB}{}
 \showrelsymbol\leftharpoonup{21BC}{}
 \showrelsymbol\leftharpoondown{21BD}{}
 \showrelsymbol\upharpoonright{21BE}{}, \cmd\restriction
 \showrelsymbol\upharpoonleft{21BF}{}
 \showrelsymbol\rightharpoonup{21C0}{}
 \showrelsymbol\rightharpoondown{21C1}{}
 \showrelsymbol\downharpoonright{21C2}{}
 \showrelsymbol\downharpoonleft{21C3}{}
 \showrelsymbol\rightleftarrows{21C4}{}
 \showrelsymbol\updownarrows{21C5}{}
 \showrelsymbol\leftrightarrows{21C6}{}
 \showrelsymbol\leftleftarrows{21C7}{}
 \showrelsymbol\upuparrows{21C8}{}
 \showrelsymbol\rightrightarrows{21C9}{}
 \showrelsymbol\downdownarrows{21CA}{}
 \showrelsymbol\leftrightharpoons{21CB}{}
 \showrelsymbol\rightleftharpoons{21CC}{}
 \showrelsymbol\nLeftarrow{21CD}{}
 \showrelsymbol\nLeftrightarrow{21CE}{}
 \showrelsymbol\nRightarrow{21CF}{}
 \showrelsymbol\Leftarrow{21D0}{}
 \showrelsymbol\Uparrow{21D1}{}
 \showrelsymbol\Rightarrow{21D2}{}
 \showrelsymbol\Downarrow{21D3}{}
 \showrelsymbol\Leftrightarrow{21D4}{}
 \showrelsymbol\Updownarrow{21D5}{}
 \showrelsymbol\Nwarrow{21D6}{}
 \showrelsymbol\Nearrow{21D7}{}
 \showrelsymbol\Searrow{21D8}{}
 \showrelsymbol\Swarrow{21D9}{}
 \showrelsymbol\Lleftarrow{21DA}{*}
 \showrelsymbol\Rrightarrow{21DB}{*}
 \showrelsymbol\leftsquigarrow{21DC}{}
 \showrelsymbol\rightsquigarrow{21DD}{}, \cmd\leadsto
 \showrelsymbol\barleftarrow{21E4}{*}
 \showrelsymbol\rightarrowbar{21E5}{*}
 \showrelsymbol\circleonrightarrow{21F4}{*}
 \showrelsymbol\downuparrows{21F5}{}
 \showrelsymbol\rightthreearrows{21F6}{*}
 \showrelsymbol\nvleftarrow{21F7}{*}
 \showrelsymbol\nvrightarrow{21F8}{*}
 \showrelsymbol\nvleftrightarrow{21F9}{*}
 \showrelsymbol\nVleftarrow{21FA}{*}
 \showrelsymbol\nVrightarrow{21FB}{*}
 \showrelsymbol\nVleftrightarrow{21FC}{*}
 \showrelsymbol\leftarrowtriangle{21FD}{*}
 \showrelsymbol\rightarrowtriangle{21FE}{*}
 \showrelsymbol\leftrightarrowtriangle{21FF}{*}
 \showrelsymbol\in{2208}{}
 \showrelsymbol\notin{2209}{}
 \showrelsymbol\smallin{220A}{}
 \showrelsymbol\ni{220B}{}, \cmd\owns
 \showrelsymbol\nni{220C}{}
 \showrelsymbol\smallni{220D}{}
 \showrelsymbol\propto{221D}{}
 \showrelsymbol\varpropto{221D}{}
 \showrelsymbol\mid{2223}{}
 \showrelsymbol\shortmid{2223}{}
 \showrelsymbol\nmid{2224}{}
 \showrelsymbol\nshortmid{2224}{*}
 \showrelsymbol\parallel{2225}{}
 \showrelsymbol\shortparallel{2225}{*}
 \showrelsymbol\nparallel{2226}{}
 \showrelsymbol\nshortparallel{2226}{*}
 \showrelsymbol\Colon{2237}{}
 \showrelsymbol\dashcolon{2239}{}
 \showrelsymbol\dotsminusdots{223A}{}
 \showrelsymbol\kernelcontraction{223B}{}
 \showrelsymbol\sim{223C}{}
 \showrelsymbol\thicksim{223C}{}
 \showrelsymbol\backsim{223D}{}
 \showrelsymbol\nsim{2241}{}
 \showrelsymbol\eqsim{2242}{}
 \showrelsymbol\simeq{2243}{}
 \showrelsymbol\nsime{2244}{}
 \showrelsymbol\cong{2245}{}
 \showrelsymbol\simneqq{2246}{}
 \showrelsymbol\ncong{2247}{}
 \showrelsymbol\approx{2248}{}
 \showrelsymbol\thickapprox{2248}{}
 \showrelsymbol\napprox{2249}{}
 \showrelsymbol\approxeq{224A}{}
 \showrelsymbol\approxident{224B}{}
 \showrelsymbol\backcong{224C}{}
 \showrelsymbol\asymp{224D}{}
 \showrelsymbol\Bumpeq{224E}{}
 \showrelsymbol\bumpeq{224F}{}
 \showrelsymbol\doteq{2250}{}
 \showrelsymbol\Doteq{2251}{}, \cmd\doteqdot
 \showrelsymbol\fallingdotseq{2252}{}
 \showrelsymbol\risingdotseq{2253}{}
 \showrelsymbol\coloneq{2254}{}
 \showrelsymbol\eqcolon{2255}{}
 \showrelsymbol\eqcirc{2256}{}
 \showrelsymbol\circeq{2257}{}
 \showrelsymbol\arceq{2258}{}
 \showrelsymbol\wedgeq{2259}{}
 \showrelsymbol\veeeq{225A}{}
 \showrelsymbol\stareq{225B}{}
 \showrelsymbol\triangleq{225C}{}
 \showrelsymbol\eqdef{225D}{}
 \showrelsymbol\measeq{225E}{}
 \showrelsymbol\questeq{225F}{}
 \showrelsymbol\ne{2260}{}, \cmd\neq
 \showrelsymbol\equiv{2261}{}
 \showrelsymbol\nequiv{2262}{}
 \showrelsymbol\Equiv{2263}{}
 \showrelsymbol\leq{2264}{}, \cmd\le
 \showrelsymbol\geq{2265}{}, \cmd\ge
 \showrelsymbol\leqq{2266}{}
 \showrelsymbol\geqq{2267}{}
 \showrelsymbol\lneqq{2268}{}
 \showrelsymbol\lvertneqq{2268}{}
 \showrelsymbol\gneqq{2269}{}
 \showrelsymbol\gvertneqq{2269}{}
 \showrelsymbol\ll{226A}{}
 \showrelsymbol\gg{226B}{}
 \showrelsymbol\between{226C}{}
 \showrelsymbol\nasymp{226D}{}
 \showrelsymbol\nless{226E}{}
 \showrelsymbol\ngtr{226F}{}
 \showrelsymbol\nleq{2270}{}
 \showrelsymbol\ngeq{2271}{}
 \showrelsymbol\lesssim{2272}{}
 \showrelsymbol\gtrsim{2273}{}
 \showrelsymbol\nlesssim{2274}{}
 \showrelsymbol\ngtrsim{2275}{}
 \showrelsymbol\lessgtr{2276}{}
 \showrelsymbol\gtrless{2277}{}
 \showrelsymbol\nlessgtr{2278}{}
 \showrelsymbol\ngtrless{2279}{}
 \showrelsymbol\prec{227A}{}
 \showrelsymbol\succ{227B}{}
 \showrelsymbol\preccurlyeq{227C}{}
 \showrelsymbol\succcurlyeq{227D}{}
 \showrelsymbol\precsim{227E}{}
 \showrelsymbol\succsim{227F}{}
 \showrelsymbol\nprec{2280}{}
 \showrelsymbol\nsucc{2281}{}
 \showrelsymbol\subset{2282}{}
 \showrelsymbol\supset{2283}{}
 \showrelsymbol\nsubset{2284}{}
 \showrelsymbol\nsupset{2285}{}
 \showrelsymbol\subseteq{2286}{}
 \showrelsymbol\supseteq{2287}{}
 \showrelsymbol\nsubseteq{2288}{}
 \showrelsymbol\nsupseteq{2289}{}
 \showrelsymbol\subsetneq{228A}{}
 \showrelsymbol\varsubsetneq{228A}{*}
 \showrelsymbol\supsetneq{228B}{}
 \showrelsymbol\varsupsetneq{228B}{*}
 \showrelsymbol\sqsubset{228F}{}
 \showrelsymbol\sqsupset{2290}{}
 \showrelsymbol\sqsubseteq{2291}{}
 \showrelsymbol\sqsupseteq{2292}{}
 \showrelsymbol\vdash{22A2}{}
 \showrelsymbol\dashv{22A3}{}
 \showrelsymbol\assert{22A6}{}
 \showrelsymbol\models{22A7}{}
 \showrelsymbol\vDash{22A8}{}
 \showrelsymbol\Vdash{22A9}{}
 \showrelsymbol\Vvdash{22AA}{}
 \showrelsymbol\VDash{22AB}{}
 \showrelsymbol\nvdash{22AC}{}
 \showrelsymbol\nvDash{22AD}{}
 \showrelsymbol\nVdash{22AE}{}
 \showrelsymbol\nVDash{22AF}{}
 \showrelsymbol\prurel{22B0}{}
 \showrelsymbol\scurel{22B1}{}
 \showrelsymbol\vartriangleleft{22B2}{}
 \showrelsymbol\vartriangleright{22B3}{}
 \showrelsymbol\trianglelefteq{22B4}{}
 \showrelsymbol\trianglerighteq{22B5}{}
 \showrelsymbol\origof{22B6}{}
 \showrelsymbol\imageof{22B7}{}
 \showrelsymbol\multimap{22B8}{}
 \showrelsymbol\bowtie{22C8}{}
 \showrelsymbol\backsimeq{22CD}{}
 \showrelsymbol\Subset{22D0}{}
 \showrelsymbol\Supset{22D1}{}
 \showrelsymbol\pitchfork{22D4}{}
 \showrelsymbol\equalparallel{22D5}{}
 \showrelsymbol\lessdot{22D6}{}
 \showrelsymbol\gtrdot{22D7}{}
 \showrelsymbol\lll{22D8}{}, \cmd\llless
 \showrelsymbol\ggg{22D9}{}, \cmd\gggtr
 \showrelsymbol\lesseqgtr{22DA}{}
 \showrelsymbol\gtreqless{22DB}{}
 \showrelsymbol\eqless{22DC}{}
 \showrelsymbol\eqgtr{22DD}{}
 \showrelsymbol\curlyeqprec{22DE}{}
 \showrelsymbol\curlyeqsucc{22DF}{}
 \showrelsymbol\npreccurlyeq{22E0}{}
 \showrelsymbol\nsucccurlyeq{22E1}{}
 \showrelsymbol\nsqsubseteq{22E2}{}
 \showrelsymbol\nsqsupseteq{22E3}{}
 \showrelsymbol\sqsubsetneq{22E4}{*}
 \showrelsymbol\sqsupsetneq{22E5}{*}
 \showrelsymbol\lnsim{22E6}{}
 \showrelsymbol\gnsim{22E7}{}
 \showrelsymbol\precnsim{22E8}{}
 \showrelsymbol\succnsim{22E9}{}
 \showrelsymbol\nvartriangleleft{22EA}{}
 \showrelsymbol\nvartriangleright{22EB}{}
 \showrelsymbol\ntrianglelefteq{22EC}{}
 \showrelsymbol\ntrianglerighteq{22ED}{}
 \showrelsymbol\vdots{22EE}{}
 \showrelsymbol\adots{22F0}{}
 \showrelsymbol\ddots{22F1}{}
 \showrelsymbol\disin{22F2}{*}
 \showrelsymbol\varisins{22F3}{*}
 \showrelsymbol\isins{22F4}{*}
 \showrelsymbol\isindot{22F5}{*}
 \showrelsymbol\varisinobar{22F6}{}
 \showrelsymbol\isinobar{22F7}{*}
 \showrelsymbol\isinvb{22F8}{*}
 \showrelsymbol\isinE{22F9}{*}
 \showrelsymbol\nisd{22FA}{*}
 \showrelsymbol\varnis{22FB}{*}
 \showrelsymbol\nis{22FC}{*}
 \showrelsymbol\varniobar{22FD}{}
 \showrelsymbol\niobar{22FE}{*}
 \showrelsymbol\bagmember{22FF}{*}
 \showrelsymbol\frown{2322}{}
 \showrelsymbol\smallfrown{2322}{*}
 \showrelsymbol\smile{2323}{}
 \showrelsymbol\smallsmile{2323}{*}
 \showrelsymbol\APLnotslash{233F}{}
 \showrelsymbol\vartriangle{25B5}{*}
 \showrelsymbol\perp{27C2}{*}
 \showrelsymbol\bsolhsub{27C8}{}
 \showrelsymbol\suphsol{27C9}{}
 \showrelsymbol\upin{27D2}{*}
 \showrelsymbol\pullback{27D3}{*}
 \showrelsymbol\pushout{27D4}{*}
 \showrelsymbol\DashVDash{27DA}{*}
 \showrelsymbol\dashVdash{27DB}{*}
 \showrelsymbol\multimapinv{27DC}{*}
 \showrelsymbol\vlongdash{27DD}{*}
 \showrelsymbol\longdashv{27DE}{*}
 \showrelsymbol\cirbot{27DF}{*}
 \showrelsymbol\UUparrow{27F0}{*}
 \showrelsymbol\DDownarrow{27F1}{*}
 \showrelsymbol\acwgapcirclearrow{27F2}{*}
 \showrelsymbol\cwgapcirclearrow{27F3}{*}
 \showrelsymbol\rightarrowonoplus{27F4}{*}
 \showrelsymbol\longleftarrow{27F5}{*}
 \showrelsymbol\longrightarrow{27F6}{*}
 \showrelsymbol\longleftrightarrow{27F7}{*}
 \showrelsymbol\Longleftarrow{27F8}{*}
 \showrelsymbol\Longrightarrow{27F9}{*}
 \showrelsymbol\Longleftrightarrow{27FA}{*}
 \showrelsymbol\longmapsfrom{27FB}{*}
 \showrelsymbol\longmapsto{27FC}{*}
 \showrelsymbol\Longmapsfrom{27FD}{*}
 \showrelsymbol\Longmapsto{27FE}{*}
 \showrelsymbol\longrightsquigarrow{27FF}{*}
 \showrelsymbol\nvtwoheadrightarrow{2900}{*}
 \showrelsymbol\nVtwoheadrightarrow{2901}{*}
 \showrelsymbol\nvLeftarrow{2902}{*}
 \showrelsymbol\nvRightarrow{2903}{*}
 \showrelsymbol\nvLeftrightarrow{2904}{*}
 \showrelsymbol\twoheadmapsto{2905}{*}
 \showrelsymbol\Mapsfrom{2906}{*}
 \showrelsymbol\Mapsto{2907}{*}
 \showrelsymbol\downarrowbarred{2908}{*}
 \showrelsymbol\uparrowbarred{2909}{*}
 \showrelsymbol\Uuparrow{290A}{*}
 \showrelsymbol\Ddownarrow{290B}{*}
 \showrelsymbol\leftbkarrow{290C}{*}
 \showrelsymbol\rightbkarrow{290D}{*}
 \showrelsymbol\leftdbkarrow{290E}{*}, \cmd\dashleftarrow
 \showrelsymbol\dbkarow{290F}{*}, \cmd\dashrightarrow
 \showrelsymbol\drbkarow{2910}{*}
 \showrelsymbol\rightdotarrow{2911}{*}
 \showrelsymbol\baruparrow{2912}{*}
 \showrelsymbol\downarrowbar{2913}{*}
 \showrelsymbol\nvrightarrowtail{2914}{*}
 \showrelsymbol\nVrightarrowtail{2915}{*}
 \showrelsymbol\twoheadrightarrowtail{2916}{*}
 \showrelsymbol\nvtwoheadrightarrowtail{2917}{*}
 \showrelsymbol\nVtwoheadrightarrowtail{2918}{*}
 \showrelsymbol\lefttail{2919}{*}
 \showrelsymbol\righttail{291A}{*}
 \showrelsymbol\leftdbltail{291B}{*}
 \showrelsymbol\rightdbltail{291C}{*}
 \showrelsymbol\diamondleftarrow{291D}{*}
 \showrelsymbol\rightarrowdiamond{291E}{*}
 \showrelsymbol\diamondleftarrowbar{291F}{*}
 \showrelsymbol\barrightarrowdiamond{2920}{*}
 \showrelsymbol\nwsearrow{2921}{*}
 \showrelsymbol\neswarrow{2922}{*}
 \showrelsymbol\hknwarrow{2923}{*}
 \showrelsymbol\hknearrow{2924}{*}
 \showrelsymbol\hksearow{2925}{*}
 \showrelsymbol\hkswarow{2926}{*}
 \showrelsymbol\tona{2927}{*}
 \showrelsymbol\toea{2928}{*}
 \showrelsymbol\tosa{2929}{*}
 \showrelsymbol\towa{292A}{*}
 \showrelsymbol\rightcurvedarrow{2933}{*}
 \showrelsymbol\leftdowncurvedarrow{2936}{*}
 \showrelsymbol\rightdowncurvedarrow{2937}{*}
 \showrelsymbol\cwrightarcarrow{2938}{*}
 \showrelsymbol\acwleftarcarrow{2939}{*}
 \showrelsymbol\acwoverarcarrow{293A}{*}
 \showrelsymbol\acwunderarcarrow{293B}{*}
 \showrelsymbol\curvearrowrightminus{293C}{*}
 \showrelsymbol\curvearrowleftplus{293D}{*}
 \showrelsymbol\cwundercurvearrow{293E}{*}
 \showrelsymbol\ccwundercurvearrow{293F}{*}
 \showrelsymbol\acwcirclearrow{2940}{*}
 \showrelsymbol\cwcirclearrow{2941}{*}
 \showrelsymbol\rightarrowshortleftarrow{2942}{*}
 \showrelsymbol\leftarrowshortrightarrow{2943}{*}
 \showrelsymbol\shortrightarrowleftarrow{2944}{*}
 \showrelsymbol\rightarrowplus{2945}{*}
 \showrelsymbol\leftarrowplus{2946}{*}
 \showrelsymbol\rightarrowx{2947}{*}
 \showrelsymbol\leftrightarrowcircle{2948}{*}
 \showrelsymbol\twoheaduparrowcircle{2949}{*}
 \showrelsymbol\leftrightharpoonupdown{294A}{*}
 \showrelsymbol\leftrightharpoondownup{294B}{*}
 \showrelsymbol\updownharpoonrightleft{294C}{*}
 \showrelsymbol\updownharpoonleftright{294D}{*}
 \showrelsymbol\leftrightharpoonupup{294E}{*}
 \showrelsymbol\updownharpoonrightright{294F}{*}
 \showrelsymbol\leftrightharpoondowndown{2950}{*}
 \showrelsymbol\updownharpoonleftleft{2951}{*}
 \showrelsymbol\barleftharpoonup{2952}{*}
 \showrelsymbol\rightharpoonupbar{2953}{*}
 \showrelsymbol\barupharpoonright{2954}{*}
 \showrelsymbol\downharpoonrightbar{2955}{*}
 \showrelsymbol\barleftharpoondown{2956}{*}
 \showrelsymbol\rightharpoondownbar{2957}{*}
 \showrelsymbol\barupharpoonleft{2958}{*}
 \showrelsymbol\downharpoonleftbar{2959}{*}
 \showrelsymbol\leftharpoonupbar{295A}{*}
 \showrelsymbol\barrightharpoonup{295B}{*}
 \showrelsymbol\upharpoonrightbar{295C}{*}
 \showrelsymbol\bardownharpoonright{295D}{*}
 \showrelsymbol\leftharpoondownbar{295E}{*}
 \showrelsymbol\barrightharpoondown{295F}{*}
 \showrelsymbol\upharpoonleftbar{2960}{*}
 \showrelsymbol\bardownharpoonleft{2961}{*}
 \showrelsymbol\leftharpoonsupdown{2962}{*}
 \showrelsymbol\upharpoonsleftright{2963}{*}
 \showrelsymbol\rightharpoonsupdown{2964}{*}
 \showrelsymbol\downharpoonsleftright{2965}{*}
 \showrelsymbol\leftrightharpoonsup{2966}{*}
 \showrelsymbol\leftrightharpoonsdown{2967}{*}
 \showrelsymbol\rightleftharpoonsup{2968}{*}
 \showrelsymbol\rightleftharpoonsdown{2969}{*}
 \showrelsymbol\leftharpoonupdash{296A}{*}
 \showrelsymbol\dashleftharpoondown{296B}{*}
 \showrelsymbol\rightharpoonupdash{296C}{*}
 \showrelsymbol\dashrightharpoondown{296D}{*}
 \showrelsymbol\updownharpoonsleftright{296E}{*}
 \showrelsymbol\downupharpoonsleftright{296F}{*}
 \showrelsymbol\rightimply{2970}{*}
 \showrelsymbol\equalrightarrow{2971}{*}
 \showrelsymbol\similarrightarrow{2972}{*}
 \showrelsymbol\leftarrowsimilar{2973}{*}
 \showrelsymbol\rightarrowsimilar{2974}{*}
 \showrelsymbol\rightarrowapprox{2975}{*}
 \showrelsymbol\ltlarr{2976}{*}
 \showrelsymbol\leftarrowless{2977}{*}
 \showrelsymbol\gtrarr{2978}{*}
 \showrelsymbol\subrarr{2979}{*}
 \showrelsymbol\leftarrowsubset{297A}{*}
 \showrelsymbol\suplarr{297B}{*}
 \showrelsymbol\leftfishtail{297C}{*}
 \showrelsymbol\rightfishtail{297D}{*}
 \showrelsymbol\upfishtail{297E}{*}
 \showrelsymbol\downfishtail{297F}{*}
 \showrelsymbol\rtriltri{29CE}{*}
 \showrelsymbol\ltrivb{29CF}{*}
 \showrelsymbol\vbrtri{29D0}{*}
 \showrelsymbol\lfbowtie{29D1}{*}
 \showrelsymbol\rfbowtie{29D2}{*}
 \showrelsymbol\fbowtie{29D3}{*}
 \showrelsymbol\lftimes{29D4}{*}
 \showrelsymbol\rftimes{29D5}{*}
 \showrelsymbol\dualmap{29DF}{*}
 \showrelsymbol\lrtriangleeq{29E1}{*}
 \showrelsymbol\eparsl{29E3}{*}
 \showrelsymbol\smeparsl{29E4}{*}
 \showrelsymbol\eqvparsl{29E5}{*}
 \showrelsymbol\gleichstark{29E6}{*}
 \showrelsymbol\ruledelayed{29F4}{*}
 \showrelsymbol\veeonwedge{2A59}{*}
 \showrelsymbol\eqdot{2A66}{}
 \showrelsymbol\dotequiv{2A67}{}
 \showrelsymbol\equivVert{2A68}{*}
 \showrelsymbol\equivVvert{2A69}{*}
 \showrelsymbol\dotsim{2A6A}{}
 \showrelsymbol\simrdots{2A6B}{*}
 \showrelsymbol\simminussim{2A6C}{*}
 \showrelsymbol\congdot{2A6D}{}
 \showrelsymbol\asteq{2A6E}{}
 \showrelsymbol\hatapprox{2A6F}{}
 \showrelsymbol\approxeqq{2A70}{}
 \showrelsymbol\eqqsim{2A73}{}
 \showrelsymbol\Coloneq{2A74}{*}
 \showrelsymbol\eqeq{2A75}{*}
 \showrelsymbol\eqeqeq{2A76}{*}
 \showrelsymbol\ddotseq{2A77}{*}
 \showrelsymbol\equivDD{2A78}{*}
 \showrelsymbol\ltcir{2A79}{*}
 \showrelsymbol\gtcir{2A7A}{*}
 \showrelsymbol\ltquest{2A7B}{*}
 \showrelsymbol\gtquest{2A7C}{*}
 \showrelsymbol\leqslant{2A7D}{}
 \showrelsymbol\geqslant{2A7E}{}
 \showrelsymbol\lesdot{2A7F}{*}
 \showrelsymbol\gesdot{2A80}{*}
 \showrelsymbol\lesdoto{2A81}{*}
 \showrelsymbol\gesdoto{2A82}{*}
 \showrelsymbol\lesdotor{2A83}{*}
 \showrelsymbol\gesdotol{2A84}{*}
 \showrelsymbol\lessapprox{2A85}{*}
 \showrelsymbol\gtrapprox{2A86}{*}
 \showrelsymbol\lneq{2A87}{}
 \showrelsymbol\gneq{2A88}{}
 \showrelsymbol\lnapprox{2A89}{}
 \showrelsymbol\gnapprox{2A8A}{}
 \showrelsymbol\lesseqqgtr{2A8B}{*}
 \showrelsymbol\gtreqqless{2A8C}{*}
 \showrelsymbol\lsime{2A8D}{*}
 \showrelsymbol\gsime{2A8E}{*}
 \showrelsymbol\lsimg{2A8F}{*}
 \showrelsymbol\gsiml{2A90}{*}
 \showrelsymbol\lgE{2A91}{*}
 \showrelsymbol\glE{2A92}{*}
 \showrelsymbol\lesges{2A93}{*}
 \showrelsymbol\gesles{2A94}{*}
 \showrelsymbol\eqslantless{2A95}{}
 \showrelsymbol\eqslantgtr{2A96}{}
 \showrelsymbol\elsdot{2A97}{*}
 \showrelsymbol\egsdot{2A98}{*}
 \showrelsymbol\eqqless{2A99}{*}
 \showrelsymbol\eqqgtr{2A9A}{*}
 \showrelsymbol\eqqslantless{2A9B}{*}
 \showrelsymbol\eqqslantgtr{2A9C}{*}
 \showrelsymbol\simless{2A9D}{}
 \showrelsymbol\simgtr{2A9E}{}
 \showrelsymbol\simlE{2A9F}{*}
 \showrelsymbol\simgE{2AA0}{*}
 \showrelsymbol\Lt{2AA1}{*}
 \showrelsymbol\Gt{2AA2}{*}
 \showrelsymbol\partialmeetcontraction{2AA3}{*}
 \showrelsymbol\glj{2AA4}{*}
 \showrelsymbol\gla{2AA5}{*}
 \showrelsymbol\ltcc{2AA6}{*}
 \showrelsymbol\gtcc{2AA7}{*}
 \showrelsymbol\lescc{2AA8}{*}
 \showrelsymbol\gescc{2AA9}{*}
 \showrelsymbol\smt{2AAA}{*}
 \showrelsymbol\lat{2AAB}{*}
 \showrelsymbol\smte{2AAC}{*}
 \showrelsymbol\late{2AAD}{*}
 \showrelsymbol\bumpeqq{2AAE}{*}
 \showrelsymbol\preceq{2AAF}{}
 \showrelsymbol\npreceq{XXXX}{*}
 \showrelsymbol\succeq{2AB0}{}
 \showrelsymbol\nsucceq{XXXX}{*}
 \showrelsymbol\precneq{2AB1}{*}
 \showrelsymbol\succneq{2AB2}{*}
 \showrelsymbol\preceqq{2AB3}{*}
 \showrelsymbol\succeqq{2AB4}{*}
 \showrelsymbol\precneqq{2AB5}{*}
 \showrelsymbol\succneqq{2AB6}{*}
 \showrelsymbol\precapprox{2AB7}{*}
 \showrelsymbol\succapprox{2AB8}{*}
 \showrelsymbol\precnapprox{2AB9}{*}
 \showrelsymbol\succnapprox{2ABA}{*}
 \showrelsymbol\Prec{2ABB}{*}
 \showrelsymbol\Succ{2ABC}{*}
 \showrelsymbol\subsetdot{2ABD}{}
 \showrelsymbol\supsetdot{2ABE}{}
 \showrelsymbol\subsetplus{2ABF}{*}
 \showrelsymbol\supsetplus{2AC0}{*}
 \showrelsymbol\submult{2AC1}{*}
 \showrelsymbol\supmult{2AC2}{*}
 \showrelsymbol\subedot{2AC3}{*}
 \showrelsymbol\supedot{2AC4}{*}
 \showrelsymbol\subseteqq{2AC5}{}
 \showrelsymbol\nsubseteqq{XXXX}{*}
 \showrelsymbol\supseteqq{2AC6}{}
 \showrelsymbol\nsupseteqq{XXXX}{*}
 \showrelsymbol\subsim{2AC7}{*}
 \showrelsymbol\supsim{2AC8}{*}
 \showrelsymbol\subsetapprox{2AC9}{*}
 \showrelsymbol\supsetapprox{2ACA}{*}
 \showrelsymbol\subsetneqq{2ACB}{}
 \showrelsymbol\varsubsetneqq{2ACB}{*}
 \showrelsymbol\supsetneqq{2ACC}{}
 \showrelsymbol\varsupsetneqq{2ACC}{*}
 \showrelsymbol\lsqhook{2ACD}{}
 \showrelsymbol\rsqhook{2ACE}{}
 \showrelsymbol\csub{2ACF}{}
 \showrelsymbol\csup{2AD0}{}
 \showrelsymbol\csube{2AD1}{}
 \showrelsymbol\csupe{2AD2}{}
 \showrelsymbol\subsup{2AD3}{}
 \showrelsymbol\supsub{2AD4}{}
 \showrelsymbol\subsub{2AD5}{}
 \showrelsymbol\supsup{2AD6}{}
 \showrelsymbol\suphsub{2AD7}{}
 \showrelsymbol\supdsub{2AD8}{}
 \showrelsymbol\forkv{2AD9}{}
 \showrelsymbol\topfork{2ADA}{}
 \showrelsymbol\mlcp{2ADB}{}
 \showrelsymbol\forks{2ADC}{}
 \showrelsymbol\forksnot{2ADD}{}
 \showrelsymbol\shortlefttack{2ADE}{}
 \showrelsymbol\shortdowntack{2ADF}{}
 \showrelsymbol\shortuptack{2AE0}{}
 \showrelsymbol\vDdash{2AE2}{}
 \showrelsymbol\dashV{2AE3}{}
 \showrelsymbol\Dashv{2AE4}{}
 \showrelsymbol\DashV{2AE5}{}
 \showrelsymbol\varVdash{2AE6}{}
 \showrelsymbol\Barv{2AE7}{}
 \showrelsymbol\vBar{2AE8}{}
 \showrelsymbol\vBarv{2AE9}{}
 \showrelsymbol\barV{2AEA}{}
 \showrelsymbol\Vbar{2AEB}{}
 \showrelsymbol\Not{2AEC}{}
 \showrelsymbol\bNot{2AED}{}
 \showrelsymbol\revnmid{2AEE}{}
 \showrelsymbol\cirmid{2AEF}{}
 \showrelsymbol\midcir{2AF0}{}
 \showrelsymbol\nhpar{2AF2}{}
 \showrelsymbol\parsim{2AF3}{}
 \showrelsymbol\lllnest{2AF7}{}
 \showrelsymbol\gggnest{2AF8}{}
 \showrelsymbol\leqqslant{2AF9}{}
 \showrelsymbol\geqqslant{2AFA}{}
 \showrelsymbol\circleonleftarrow{2B30}{*}
 \showrelsymbol\leftthreearrows{2B31}{*}
 \showrelsymbol\leftarrowonoplus{2B32}{*}
 \showrelsymbol\longleftsquigarrow{2B33}{*}
 \showrelsymbol\nvtwoheadleftarrow{2B34}{*}
 \showrelsymbol\nVtwoheadleftarrow{2B35}{*}
 \showrelsymbol\twoheadmapsfrom{2B36}{*}
 \showrelsymbol\twoheadleftdbkarrow{2B37}{*}
 \showrelsymbol\leftdotarrow{2B38}{*}
 \showrelsymbol\nvleftarrowtail{2B39}{*}
 \showrelsymbol\nVleftarrowtail{2B3A}{*}
 \showrelsymbol\twoheadleftarrowtail{2B3B}{*}
 \showrelsymbol\nvtwoheadleftarrowtail{2B3C}{*}
 \showrelsymbol\nVtwoheadleftarrowtail{2B3D}{*}
 \showrelsymbol\leftarrowx{2B3E}{*}
 \showrelsymbol\leftcurvedarrow{2B3F}{*}
 \showrelsymbol\equalleftarrow{2B40}{*}
 \showrelsymbol\bsimilarleftarrow{2B41}{*}
 \showrelsymbol\leftarrowbackapprox{2B42}{*}
 \showrelsymbol\rightarrowgtr{2B43}{*}
 \showrelsymbol\rightarrowsupset{2B44}{*}
 \showrelsymbol\LLeftarrow{2B45}{*}
 \showrelsymbol\RRightarrow{2B46}{*}
 \showrelsymbol\bsimilarrightarrow{2B47}{*}
 \showrelsymbol\rightarrowbackapprox{2B48}{*}
 \showrelsymbol\similarleftarrow{2B49}{*}
 \showrelsymbol\leftarrowapprox{2B4A}{*}
 \showrelsymbol\leftarrowbsimilar{2B4B}{*}
 \showrelsymbol\rightarrowbsimilar{2B4C}{*}
 \showrelsymbol\ngeqq{XXXX}{}
 \showrelsymbol\ngeqslant{XXXX}{}
 \showrelsymbol\nleqslant{XXXX}{}
 \showrelsymbol\nleqq{XXXX}{}
% \showrelsymbol\ncongdot{XXXX}{}
% \showrelsymbol\napproxeqq{XXXX}{}
% \showrelsymbol\nll{XXXX}{}
% \showrelsymbol\ngg{XXXX}{}
% \showrelsymbol\nsqsubset{XXXX}{}
% \showrelsymbol\nsqsupset{XXXX}{}
% \showrelsymbol\nBumpeq{XXXX}{}
% \showrelsymbol\nbumpeq{XXXX}{}
%\showrelsymbol\neqsim{XXXX}{}
 %\showrelsymbol\nvarisinobar{XXXX}{}
% \showrelsymbol\nvarniobar{XXXX}{}
% \showrelsymbol\neqslantless{XXXX}{}
 %\showrelsymbol\neqslantgtr{XXXX}{}
 \showrelsymbol\lhook{XXXX}{}
 \showrelsymbol\rhook{XXXX}{}
 \showrelsymbol\relbar{XXXX}{}
 \showrelsymbol\Relbar{XXXX}{}
 %\showrelsymbol\Rrelbar{XXXX}{*}
% \showrelsymbol\RRelbar{XXXX}{*}
 \showrelsymbol\mapsfromchar{XXXX}{}
 \showrelsymbol\mapstochar{XXXX}{}
 \end{multicols}
 %%% TODO HOW TO HANDLE BOTH LOCALLY
 %%%
 \subsection{Integrals}
 \label{integrals}
 Integrals come in two styles, the slanted versions shown below (|$\intsl$|,
 etc.)\ and right versions such as~|$\int$|. By default, the symbol names
 listed below will give you the slanted style, but if you specify the |int|
 package option, they will give you the corresponding right symbols.

 It is highly recommended that authors stick to the names below and use the
 |int| package option to choose a style globally for their document.
 However, in recognition of the fact that it might occasionally be necessary
 to mix the two styles, alternative names have been provided for all
 integrals. Append |sl| or || to the names below to request either the
 \emph{sl}anted or the \emph{}right variant. Thus, |$\intsl$| will always
 yield~|$\intsl$| and |$\int$| will always yield~|$\int$|, and similarly for
 the other integrals.
%
 \begin{multicols}{2}
 \integralsymbol\int{222B}{}
 \integralsymbol\iint{222C}{}
 \integralsymbol\iiint{222D}{}
 \integralsymbol\oint{222E}{}
 \integralsymbol\oiint{222F}{}
 \integralsymbol\oiiint{2230}{}
 \integralsymbol\intclockwise{2231}{}
 \integralsymbol\varointclockwise{2232}{}
 \integralsymbol\ointctrclockwise{2233}{}
 \integralsymbol\sumint{2A0B}{}
 \integralsymbol\iiiint{2A0C}{}
 \integralsymbol\intbar{2A0D}{}
 \integralsymbol\intBar{2A0E}{}
 \integralsymbol\fint{2A0F}{}
 \integralsymbol\cirfnint{2A10}{}
 \integralsymbol\awint{2A11}{}
 \integralsymbol\rppolint{2A12}{}
 \integralsymbol\scpolint{2A13}{}
 \integralsymbol\npolint{2A14}{}
 \integralsymbol\pointint{2A15}{}
 \integralsymbol\sqint{2A16}{}
 \integralsymbol\intlarhk{2A17}{}
 \integralsymbol\intx{2A18}{}
% \integralsymbol\intcapup{2A19}{}
 %\integralsymbol\intc{2A1A}{}
% \integralsymbol\intcup{2A1B}{}
 \integralsymbol\lowint{2A1C}{}
 \end{multicols}

\endinput



%%%%%%%%%%%%%%%%%%%%%%%%%%%%%%%
 \subsection{Ordinary symbols}
 \begin{multicols}{2}
 %\showsymbol\#{0023}{}
 %\showsymbol\mathdollar{0024}{}
 % \showsymbol\%{0025}{}
 %\showsymbol\&{0026}{}
% \showsymbol.{002E}{}
% \showsymbol/{002F}{}
% \showsymbol?{003F}{}
% \showsymbol@{0040}{}
 \showsymbol\backslash{005C}{}
 \showsymbol\mathsterling{00A3}{}
 \showsymbol\mathsection{00A7}{}
 \showsymbol\neg{00AC}{}, \cmd\lnot
 \showsymbol\mathparagraph{00B6}{}
 \showsymbol\eth{00F0}{}
 \showsymbol\Zbar{01B5}{*}
 \showsymbol\digamma{03DD}{}
 \showsymbol\varkappa{03F0}{}
 \showsymbol\backepsilon{03F6}{}
 \showsymbol\upbackepsilon{03F6}{}
 \showsymbol\enleadertwodots{2025}{}
 \showsymbol\mathellipsis{2026}{}
 \showsymbol\prime{2032}{}
 \showsymbol\dprime{2033}{}
 \showsymbol\trprime{2034}{}
 \showsymbol\backprime{2035}{}
 \showsymbol\backdprime{2036}{}
 \showsymbol\backtrprime{2037}{}
 \showsymbol\caretinsert{2038}{}
 \showsymbol\Exclam{203C}{}
 \showsymbol\hyphenbullet{2043}{*}
 \showsymbol\Question{2047}{}
 \showsymbol\qprime{2057}{}
 \showsymbol\enclosecircle{20DD}{}\indexmathcmd[Circles]{\enclosecircle}
 \showsymbol\enclosesquare{20DE}{*}
 \showsymbol\enclosediamond{20DF}{*}
 \showsymbol\enclosetriangle{20E4}{}
 \showsymbol\Eulerconst{2107}{}
 \showsymbol\hbar{210F}{*}
 \showsymbol\hslash{210F}{}
 \showsymbol\Im{2111}{}
 \showsymbol\ell{2113}{}
 \showsymbol\wp{2118}{}
 \showsymbol\Re{211C}{}
 \showsymbol\mho{2127}{}
 \showsymbol\turnediota{2129}{}
 \showsymbol\Angstrom{212B}{}
 \showsymbol\Finv{2132}{}
 \showsymbol\aleph{2135}{}
 \showsymbol\beth{2136}{}
 \showsymbol\gimel{2137}{}
 \showsymbol\daleth{2138}{}
 \showsymbol\Game{2141}{*}
 \showsymbol\sansLturned{2142}{*}
 \showsymbol\sansLmirrored{2143}{*}
 \showsymbol\Yup{2144}{*}
 \showsymbol\PropertyLine{214A}{*}
 \showsymbol\updownarrowbar{21A8}{}
 \showsymbol\linefeed{21B4}{}
 \showsymbol\carriagereturn{21B5}{}
 \showsymbol\barovernorthwestarrow{21B8}{}
 \showsymbol\barleftarrowrightarrowbar{21B9}{}
 \showsymbol\acwopencirclearrow{21BA}{}\indexmathcmd[Circles]{\acwopencirclearrow}
 \showsymbol\cwopencirclearrow{21BB}{}\indexmathcmd[Circles]{\cwopencirclearrow}
 \showsymbol\nHuparrow{21DE}{*}
 \showsymbol\nHdownarrow{21DF}{*}
 \showsymbol\leftdasharrow{21E0}{*}
 \showsymbol\updasharrow{21E1}{*}
 \showsymbol\rightdasharrow{21E2}{*}
 \showsymbol\downdasharrow{21E3}{*}
 \showsymbol\leftwhitearrow{21E6}{}
 \showsymbol\upwhitearrow{21E7}{}
 \showsymbol\rightwhitearrow{21E8}{}
 \showsymbol\downwhitearrow{21E9}{}
 \showsymbol\whitearrowupfrombar{21EA}{}
 \showsymbol\forall{2200}{}
 \showsymbol\complement{2201}{}
 \showsymbol\exists{2203}{}
 \showsymbol\nexists{2204}{}
 \showsymbol\varnothing{2205}{}
 \showsymbol\emptyset{2205}{}
 \showsymbol\increment{2206}{}
 \showsymbol\QED{220E}{*}
 \showsymbol\infty{221E}{}
 \showsymbol\rightangle{221F}{}
 \showsymbol\angle{2220}{}
 \showsymbol\measuredangle{2221}{}
 \showsymbol\sphericalangle{2222}{}
 \showsymbol\therefore{2234}{}
 \showsymbol\because{2235}{}
 \showsymbol\sinewave{223F}{}
 \showsymbol\top{22A4}{}
 \showsymbol\bot{22A5}{}
 \showsymbol\hermitmatrix{22B9}{}
 \showsymbol\measuredrightangle{22BE}{}
 \showsymbol\varlrtriangle{22BF}{}
 %\showsymbol\cdots{22EF}{} % TO FIX
 \showsymbol\diameter{2300}{*}
 \showsymbol\house{2302}{}
 \showsymbol\invnot{2310}{}
 \showsymbol\sqlozenge{2311}{*}
 \showsymbol\profline{2312}{*}
 \showsymbol\profsurf{2313}{*}
 \showsymbol\viewdata{2317}{*}
 \showsymbol\turnednot{2319}{}
 \showsymbol\varhexagonlrbonds{232C}{*}
 \showsymbol\conictaper{2332}{*}
 \showsymbol\topbot{2336}{}
 \showsymbol\APLnotbackslash{2340}{*}
 \showsymbol\APLboxupcaret{2353}{*}
 \showsymbol\APLboxquestion{2370}{*}
 \showsymbol\rangledownzigzagarrow{237C}{*}
 \showsymbol\hexagon{2394}{*}
 \showsymbol\bbrktbrk{23B6}{}
 \showsymbol\varcarriagereturn{23CE}{*}
 \showsymbol\obrbrak{23E0}{}
 \showsymbol\ubrbrak{23E1}{}
 \showsymbol\trapezium{23E2}{*}
 \showsymbol\benzenr{23E3}{*}
 \showsymbol\strns{23E4}{*}
 \showsymbol\fltns{23E5}{*}
 \showsymbol\accurrent{23E6}{*}
 \showsymbol\elinters{23E7}{*}
 \showsymbol\mathvisiblespace{2423}{}
 \showsymbol\circledR{24C7}{}
 \showsymbol\circledS{24C8}{}
 \showsymbol\mdlgblksquare{25A0}{*}, \cmd\blacksquare
 \showsymbol\mdlgwhtsquare{25A1}{*}, \cmd\square, \cmd\Box
 \showsymbol\squoval{25A2}{*}
 \showsymbol\blackinwhitesquare{25A3}{*}
 \showsymbol\squarehfill{25A4}{*}
 \showsymbol\squarevfill{25A5}{*}
 \showsymbol\squarehvfill{25A6}{*}
 \showsymbol\squarenwsefill{25A7}{*}
 \showsymbol\squareneswfill{25A8}{*}
 \showsymbol\squarecrossfill{25A9}{*}
 \showsymbol\smblksquare{25AA}{*}
 \showsymbol\smwhtsquare{25AB}{*}
 \showsymbol\hrectangleblack{25AC}{*}
 \showsymbol\hrectangle{25AD}{*}
 \showsymbol\vrectangleblack{25AE}{*}
 \showsymbol\vrectangle{25AF}{*}
 \showsymbol\parallelogramblack{25B0}{*}
 \showsymbol\parallelogram{25B1}{*}
 \showsymbol\bigblacktriangleup{25B2}{*}
 \showsymbol\blacktriangle{25B4}{*}
 \showsymbol\blacktriangleright{25B6}{*}
 \showsymbol\smallblacktriangleright{25B8}{*}
 \showsymbol\smalltriangleright{25B9}{*}
 \showsymbol\blackpointerright{25BA}{*}
 \showsymbol\whitepointerright{25BB}{*}
 \showsymbol\bigblacktriangledown{25BC}{*}
 \showsymbol\bigtriangledown{25BD}{}
 \showsymbol\blacktriangledown{25BE}{*}
 \showsymbol\triangledown{25BF}{*}
 \showsymbol\blacktriangleleft{25C0}{*}
 \showsymbol\smallblacktriangleleft{25C2}{*}
 \showsymbol\smalltriangleleft{25C3}{*}
 \showsymbol\blackpointerleft{25C4}{*}
 \showsymbol\whitepointerleft{25C5}{*}
 \showsymbol\mdlgblkdiamond{25C6}{*}
 \showsymbol\mdlgwhtdiamond{25C7}{*}
 \showsymbol\blackinwhitediamond{25C8}{*}
 \showsymbol\fisheye{25C9}{*}
 \showsymbol\mdlgwhtlozenge{25CA}{}, \cmd\lozenge, \\ \cmd\Diamond
 \showsymbol\dottedcircle{25CC}{*}
 \showsymbol\circlevertfill{25CD}{*}
 \showsymbol\bullseye{25CE}{*}
 \showsymbol\mdlgblkcircle{25CF}{*}
 \showsymbol\circlelefthalfblack{25D0}{*}
 \showsymbol\circlerighthalfblack{25D1}{*}
 \showsymbol\circlebottomhalfblack{25D2}{*}
 \showsymbol\circletophalfblack{25D3}{*}
 \showsymbol\circleurquadblack{25D4}{*}
 \showsymbol\blackcircleulquadwhite{25D5}{*}
 \showsymbol\blacklefthalfcircle{25D6}{*}
 \showsymbol\blackrighthalfcircle{25D7}{*}
 \showsymbol\inversebullet{25D8}{*}
 \showsymbol\inversewhitecircle{25D9}{*}
 \showsymbol\invwhiteupperhalfcircle{25DA}{*}
 \showsymbol\invwhitelowerhalfcircle{25DB}{*}
 \showsymbol\ularc{25DC}{*}
 \showsymbol\urarc{25DD}{*}
 \showsymbol\lrarc{25DE}{*}
 \showsymbol\llarc{25DF}{*}
 \showsymbol\topsemicircle{25E0}{*}
 \showsymbol\botsemicircle{25E1}{*}
 \showsymbol\lrblacktriangle{25E2}{*}
 \showsymbol\llblacktriangle{25E3}{*}
 \showsymbol\ulblacktriangle{25E4}{*}
 \showsymbol\urblacktriangle{25E5}{*}
 \showsymbol\circ{25E6}{}, \cmd\smwhtcircle
 \showsymbol\squareleftblack{25E7}{*}
 \showsymbol\squarerightblack{25E8}{*}
 \showsymbol\squareulblack{25E9}{*}
 \showsymbol\squarelrblack{25EA}{*}
 \showsymbol\trianglecdot{25EC}{}
 \showsymbol\triangleleftblack{25ED}{*}
 \showsymbol\trianglerightblack{25EE}{*}
 \showsymbol\lgwhtcircle{25EF}{*}
 \showsymbol\squareulquad{25F0}{*}
 \showsymbol\squarellquad{25F1}{*}
 \showsymbol\squarelrquad{25F2}{*}
 \showsymbol\squareurquad{25F3}{*}
 \showsymbol\circleulquad{25F4}{*}
 \showsymbol\circlellquad{25F5}{*}
 \showsymbol\circlelrquad{25F6}{*}
 \showsymbol\circleurquad{25F7}{*}
 \showsymbol\ultriangle{25F8}{*}
 \showsymbol\urtriangle{25F9}{*}
 \showsymbol\lltriangle{25FA}{*}
 \showsymbol\mdwhtsquare{25FB}{*}
 \showsymbol\mdblksquare{25FC}{*}
 \showsymbol\mdsmwhtsquare{25FD}{*}
 \showsymbol\mdsmblksquare{25FE}{*}
 \showsymbol\lrtriangle{25FF}{*}
 \showsymbol\bigstar{2605}{*}
 \showsymbol\bigwhitestar{2606}{*}
 \showsymbol\astrosun{2609}{}
 \showsymbol\danger{2621}{}
 \showsymbol\blacksmiley{263B}{}
 \showsymbol\sun{263C}{}
 \showsymbol\rightmoon{263D}{}
 \showsymbol\leftmoon{263E}{}
 \showsymbol\female{2640}{}
 \showsymbol\male{2642}{}
 \showsymbol\spadesuit{2660}{*}
 \showsymbol\heartsuit{2661}{*}
 \showsymbol\diamondsuit{2662}{*}
 \showsymbol\clubsuit{2663}{*}
 \showsymbol\varspadesuit{2664}{}
 \showsymbol\varheartsuit{2665}{}
 \showsymbol\vardiamondsuit{2666}{}
 \showsymbol\varclubsuit{2667}{}
 \showsymbol\quarternote{2669}{}
 \showsymbol\eighthnote{266A}{}
 \showsymbol\twonotes{266B}{}
 \showsymbol\flat{266D}{}
 \showsymbol\natural{266E}{}
 \showsymbol\sharp{266F}{}
 \showsymbol\acidfree{267E}{*}
 \showsymbol\dicei{2680}{}
 \showsymbol\diceii{2681}{}
 \showsymbol\diceiii{2682}{}
 \showsymbol\diceiv{2683}{}
 \showsymbol\dicev{2684}{}
 \showsymbol\dicevi{2685}{}
 \showsymbol\circledrightdot{2686}{}
 \showsymbol\circledtwodots{2687}{}
 \showsymbol\blackcircledrightdot{2688}{}
 \showsymbol\blackcircledtwodots{2689}{}
 \showsymbol\Hermaphrodite{26A5}{}
 \showsymbol\mdwhtcircle{26AA}{}
 \showsymbol\mdblkcircle{26AB}{}
 \showsymbol\mdsmwhtcircle{26AC}{}
 \showsymbol\neuter{26B2}{}
 \showsymbol\checkmark{2713}{}
 \showsymbol\maltese{2720}{}
 \showsymbol\circledstar{272A}{}
 \showsymbol\varstar{2736}{}
 \showsymbol\dingasterisk{273D}{}
 \showsymbol\draftingarrow{279B}{*}
 \showsymbol\threedangle{27C0}{*}
 \showsymbol\whiteinwhitetriangle{27C1}{*}
 \showsymbol\subsetcirc{27C3}{*}
 \showsymbol\supsetcirc{27C4}{*}
 \showsymbol\diagup{27CB}{*}
 \showsymbol\diagdown{27CD}{*}
 \showsymbol\diamondcdot{27D0}{*}
 \showsymbol\rdiagovfdiag{292B}{*}
 \showsymbol\fdiagovrdiag{292C}{*}
 \showsymbol\seovnearrow{292D}{*}
 \showsymbol\neovsearrow{292E}{*}
 \showsymbol\fdiagovnearrow{292F}{*}
 \showsymbol\rdiagovsearrow{2930}{*}
 \showsymbol\neovnwarrow{2931}{*}
 \showsymbol\nwovnearrow{2932}{*}
 \showsymbol\uprightcurvearrow{2934}{*}
 \showsymbol\downrightcurvedarrow{2935}{*}
 \showsymbol\mdsmblkcircle{2981}{*}
 \showsymbol\fourvdots{2999}{*}
 \showsymbol\vzigzag{299A}{*}
 \showsymbol\measuredangleleft{299B}{*}
 \showsymbol\rightanglesqr{299C}{*}
 \showsymbol\rightanglemdot{299D}{*}
 \showsymbol\angles{299E}{*}
 \showsymbol\angdnr{299F}{*}
 \showsymbol\gtlpar{29A0}{*}
 \showsymbol\sphericalangleup{29A1}{*}
 \showsymbol\turnangle{29A2}{*}
 \showsymbol\revangle{29A3}{*}
 \showsymbol\angleubar{29A4}{*}
 \showsymbol\revangleubar{29A5}{*}
 \showsymbol\wideangledown{29A6}{*}
 \showsymbol\wideangleup{29A7}{*}
 \showsymbol\measanglerutone{29A8}{*}
 \showsymbol\measanglelutonw{29A9}{*}
 \showsymbol\measanglerdtose{29AA}{*}
 \showsymbol\measangleldtosw{29AB}{*}
 \showsymbol\measangleurtone{29AC}{*}
 \showsymbol\measangleultonw{29AD}{*}
 \showsymbol\measangledrtose{29AE}{*}
 \showsymbol\measangledltosw{29AF}{*}
 \showsymbol\revemptyset{29B0}{*}
 \showsymbol\emptysetobar{29B1}{*}
 \showsymbol\emptysetocirc{29B2}{*}
 \showsymbol\emptysetoarr{29B3}{*}
 \showsymbol\emptysetoarrl{29B4}{*}
 \showsymbol\obot{29BA}{*}
 \showsymbol\olcross{29BB}{*}
 \showsymbol\odotslashdot{29BC}{*}
 \showsymbol\uparrowoncircle{29BD}{*}
 \showsymbol\circledwhitebullet{29BE}{*}
 \showsymbol\circledbullet{29BF}{*}
 \showsymbol\cirscir{29C2}{*}
 \showsymbol\cirE{29C3}{*}
 \showsymbol\boxonbox{29C9}{*}
 \showsymbol\triangleodot{29CA}{*}
 \showsymbol\triangleubar{29CB}{*}
 \showsymbol\triangles{29CC}{*}
 \showsymbol\iinfin{29DC}{*}
 \showsymbol\tieinfty{29DD}{*}
 \showsymbol\nvinfty{29DE}{*}
 \showsymbol\laplac{29E0}{*}
 \showsymbol\thermod{29E7}{*}
 \showsymbol\downtriangleleftblack{29E8}{*}
 \showsymbol\downtrianglerightblack{29E9}{*}
 \showsymbol\blackdiamonddownarrow{29EA}{*}
 \showsymbol\blacklozenge{29EB}{}
 \showsymbol\circledownarrow{29EC}{*}
 \showsymbol\blackcircledownarrow{29ED}{*}
 \showsymbol\errbarsquare{29EE}{*}
 \showsymbol\errbarblacksquare{29EF}{*}
 \showsymbol\errbardiamond{29F0}{*}
 \showsymbol\errbarblackdiamond{29F1}{*}
 \showsymbol\errbarcircle{29F2}{*}
 \showsymbol\errbarblackcircle{29F3}{*}
 \showsymbol\perps{2AE1}{}
 \showsymbol\topcir{2AF1}{}
 \showsymbol\squaretopblack{2B12}{}
 \showsymbol\squarebotblack{2B13}{}
 \showsymbol\squareurblack{2B14}{}
 \showsymbol\squarellblack{2B15}{}
 \showsymbol\diamondleftblack{2B16}{}
 \showsymbol\diamondrightblack{2B17}{}
 \showsymbol\diamondtopblack{2B18}{}
 \showsymbol\diamondbotblack{2B19}{}
 \showsymbol\dottedsquare{2B1A}{}
 \showsymbol\lgblksquare{2B1B}{}
 \showsymbol\lgwhtsquare{2B1C}{}
 \showsymbol\vysmblksquare{2B1D}{}
 \showsymbol\vysmwhtsquare{2B1E}{}
 \showsymbol\pentagonblack{2B1F}{}
 \showsymbol\pentagon{2B20}{}
 \showsymbol\varhexagon{2B21}{}
 \showsymbol\varhexagonblack{2B22}{}
 \showsymbol\hexagonblack{2B23}{}
 \showsymbol\lgblkcircle{2B24}{}
 \showsymbol\mdblkdiamond{2B25}{}
 \showsymbol\mdwhtdiamond{2B26}{}
 \showsymbol\mdblklozenge{2B27}{}
 \showsymbol\mdwhtlozenge{2B28}{}
 \showsymbol\smblkdiamond{2B29}{}
 \showsymbol\smblklozenge{2B2A}{}
 \showsymbol\smwhtlozenge{2B2B}{}
 \showsymbol\blkhorzoval{2B2C}{}
 \showsymbol\whthorzoval{2B2D}{}
 \showsymbol\blkvertoval{2B2E}{}
 \showsymbol\whtvertoval{2B2F}{}
 \showsymbol\medwhitestar{2B50}{}
 \showsymbol\medblackstar{2B51}{}
 \showsymbol\smwhitestar{2B52}{}
 \showsymbol\rightpentagonblack{2B53}{}
 \showsymbol\rightpentagon{2B54}{}
 \showsymbol\postalmark{3012}{}
 \showsymbol\hzigzag{3030}{}
 \showsymbol\Bbbk{1D55C}{}
 \showsymbol\bracevert{XXXX}{*}
 \end{multicols}

 \subsection{Binary operators}
 \index{binary operators}
 \begin{multicols}{2}
% \showsymbolbin+{000B}{}
 \showsymbolbin\pm{00B1}{}
 \showsymbolbin\cdotp{00B7}{}%?, \cmd\centerdot
 \showsymbolbin\times{00D7}{}
 \showsymbolbin\div{00F7}{}
 \showsymbolbin\dagger{2020}{}
 \showsymbolbin\ddagger{2021}{}
 \showsymbolbin\smblkcircle{2022}{}
 \showsymbolbin\fracslash{2044}{}
 \showsymbolbin\upand{214B}{}
% \showsymbolbin-{000D}{}
 \showsymbolbin\mp{2213}{}
 \showsymbolbin\dotplus{2214}{}
 \showsymbolbin\smallsetminus{2216}{}
 \showsymbolbin\ast{2217}{}
 \showsymbolbin\vysmwhtcircle{2218}{}
 \showsymbolbin\vysmblkcircle{2219}{}, {\small\cmd\bullet}
 \showsymbolbin\wedge{2227}{}, \cmd\land
 \showsymbolbin\vee{2228}{}, \cmd\lor
 \showsymbolbin\cap{2229}{}
 \showsymbolbin\cup{222A}{}
 \showsymbolbin\dotminus{2238}{}
 \showsymbolbin\invlazys{223E}{}
 \showsymbolbin\wr{2240}{}
 \showsymbolbin\cupleftarrow{228C}{}
 \showsymbolbin\cupdot{228D}{}
 \showsymbolbin\uplus{228E}{}
 \showsymbolbin\sqcap{2293}{}
 \showsymbolbin\sqcup{2294}{}
 \showsymbolbin\oplus{2295}{}
 \showsymbolbin\ominus{2296}{}
 \showsymbolbin\otimes{2297}{}
 \showsymbolbin\oslash{2298}{}
 \showsymbolbin\odot{2299}{}
 \showsymbolbin\circledcirc{229A}{}
 \showsymbolbin\circledast{229B}{}
 \showsymbolbin\circledequal{229C}{}
 \showsymbolbin\circleddash{229D}{}
 \showsymbolbin\boxplus{229E}{}
 \showsymbolbin\boxminus{229F}{}
 \showsymbolbin\boxtimes{22A0}{}
 \showsymbolbin\boxdot{22A1}{}
 \showsymbolbin\intercal{22BA}{}
 \showsymbolbin\veebar{22BB}{}
 \showsymbolbin\barwedge{22BC}{}
 \showsymbolbin\barvee{22BD}{}
 \showsymbolbin\diamond{22C4}{}, \cmd\smwhtdiamond
 \showsymbolbin\cdot{22C5}{*}
 \showsymbolbin\star{22C6}{}
 \showsymbolbin\divideontimes{22C7}{}
 \showsymbolbin\ltimes{22C9}{}
 \showsymbolbin\rtimes{22CA}{}
 \showsymbolbin\leftthreetimes{22CB}{}
 \showsymbolbin\rightthreetimes{22CC}{}
 \showsymbolbin\curlyvee{22CE}{}
 \showsymbolbin\curlywedge{22CF}{}
 \showsymbolbin\Cap{22D2}{}, \cmd\doublecap
 \showsymbolbin\Cup{22D3}{}, \cmd\doublecup
 \showsymbolbin\varbarwedge{2305}{*}
 \showsymbolbin\vardoublebarwedge{2306}{*}
 \showsymbolbin\obar{233D}{}
 \showsymbolbin\triangle{25B3}{}, \cmd\bigtriangleup
 \showsymbolbin\lhd{22B2}{}
 \showsymbolbin\rhd{22B3}{}
 \showsymbolbin\unlhd{22B4}{}
 \showsymbolbin\unrhd{22B5}{}
 \showsymbolbin\mdlgwhtcircle{25CB}{*}
 \showsymbolbin\boxbar{25EB}{*}
 \showsymbolbin\veedot{27C7}{*}
 \showsymbolbin\wedgedot{27D1}{*}
 \showsymbolbin\lozengeminus{27E0}{*}
 \showsymbolbin\concavediamond{27E1}{*}
 \showsymbolbin\concavediamondtickleft{27E2}{*}
 \showsymbolbin\concavediamondtickright{27E3}{*}
 \showsymbolbin\whitesquaretickleft{27E4}{*}
 \showsymbolbin\whitesquaretickright{27E5}{*}
 \showsymbolbin\typecolon{2982}{*}
 \showsymbolbin\circlehbar{29B5}{*}
 \showsymbolbin\circledvert{29B6}{}
 \showsymbolbin\circledparallel{29B7}{}
 \showsymbolbin\obslash{29B8}{}
 \showsymbolbin\operp{29B9}{*}
 \showsymbolbin\olessthan{29C0}{}
 \showsymbolbin\ogreaterthan{29C1}{}
 \showsymbolbin\boxdiag{29C4}{}
 \showsymbolbin\boxbslash{29C5}{}
 \showsymbolbin\boxast{29C6}{}
 \showsymbolbin\boxcircle{29C7}{}
 \showsymbolbin\boxbox{29C8}{*}
 \showsymbolbin\triangleserifs{29CD}{*}
 \showsymbolbin\hourglass{29D6}{*}
 \showsymbolbin\blackhourglass{29D7}{*}
 \showsymbolbin\shuffle{29E2}{*}
 \showsymbolbin\mdlgblklozenge{29EB}{*}
 \showsymbolbin\setminus{29F5}{*}
 \showsymbolbin\dsol{29F6}{*}
 \showsymbolbin\rsolbar{29F7}{*}
 \showsymbolbin\doubleplus{29FA}{*}
 \showsymbolbin\tripleplus{29FB}{*}
 \showsymbolbin\tplus{29FE}{*}
 \showsymbolbin\tminus{29FF}{*}
 \showsymbolbin\ringplus{2A22}{}
 \showsymbolbin\plushat{2A23}{}
 \showsymbolbin\simplus{2A24}{}
 \showsymbolbin\plusdot{2A25}{}
 \showsymbolbin\plussim{2A26}{}
 \showsymbolbin\plussubtwo{2A27}{}
 \showsymbolbin\plustrif{2A28}{*}
 \showsymbolbin\commaminus{2A29}{*}
 \showsymbolbin\minusdot{2A2A}{}
 \showsymbolbin\minusfdots{2A2B}{}
 \showsymbolbin\minusrdots{2A2C}{*}
 \showsymbolbin\opluslhrim{2A2D}{*}
 \showsymbolbin\oplusrhrim{2A2E}{*}
 \showsymbolbin\vectimes{2A2F}{*}
 \showsymbolbin\dottimes{2A30}{}
 \showsymbolbin\timesbar{2A31}{}
 \showsymbolbin\btimes{2A32}{}
 \showsymbolbin\smashtimes{2A33}{*}
 \showsymbolbin\otimeslhrim{2A34}{*}
 \showsymbolbin\otimesrhrim{2A35}{*}
 \showsymbolbin\otimeshat{2A36}{*}
 \showsymbolbin\Otimes{2A37}{*}
 \showsymbolbin\odiv{2A38}{*}
 \showsymbolbin\triangleplus{2A39}{*}
 \showsymbolbin\triangleminus{2A3A}{*}
 \showsymbolbin\triangletimes{2A3B}{*}
 \showsymbolbin\intprod{2A3C}{*}
 \showsymbolbin\intprodr{2A3D}{*}
 \showsymbolbin\fcmp{2A3E}{*}
 \showsymbolbin\amalg{2A3F}{}
 \showsymbolbin\capdot{2A40}{*}
 \showsymbolbin\uminus{2A41}{*}
 \showsymbolbin\barcup{2A42}{*}
 \showsymbolbin\barcap{2A43}{*}
 \showsymbolbin\capwedge{2A44}{*}
 \showsymbolbin\cupvee{2A45}{*}
 \showsymbolbin\cupovercap{2A46}{*}
 \showsymbolbin\capovercup{2A47}{*}
 \showsymbolbin\cupbarcap{2A48}{*}
 \showsymbolbin\capbarcup{2A49}{*}
 \showsymbolbin\twocups{2A4A}{*}
 \showsymbolbin\twocaps{2A4B}{*}
 \showsymbolbin\closedvarcup{2A4C}{*}
 \showsymbolbin\closedvarcap{2A4D}{*}
 \showsymbolbin\Sqcap{2A4E}{*}
 \showsymbolbin\Sqcup{2A4F}{*}
 \showsymbolbin\closedvarcupsmashprod{2A50}{*}
 \showsymbolbin\wedgeodot{2A51}{*}
 \showsymbolbin\veeodot{2A52}{*}
 \showsymbolbin\Wedge{2A53}{*}
 \showsymbolbin\Vee{2A54}{*}
 \showsymbolbin\wedgeonwedge{2A55}{*}
 \showsymbolbin\veeonvee{2A56}{*}
 \showsymbolbin\bigslopedvee{2A57}{*}
 \showsymbolbin\bigslopedwedge{2A58}{*}
 \showsymbolbin\wedgemidvert{2A5A}{*}
 \showsymbolbin\veemidvert{2A5B}{*}
 \showsymbolbin\midbarwedge{2A5C}{*}
 \showsymbolbin\midbarvee{2A5D}{*}
 \showsymbolbin\doublebarwedge{2A5E}{}
 \showsymbolbin\wedgebar{2A5F}{*}
 \showsymbolbin\wedgedoublebar{2A60}{*}
 \showsymbolbin\varveebar{2A61}{*}
 \showsymbolbin\doublebarvee{2A62}{*}
 \showsymbolbin\veedoublebar{2A63}{}
 \showsymbolbin\dsub{2A64}{*}
 \showsymbolbin\rsub{2A65}{*}
 \showsymbolbin\eqqplus{2A71}{}
 \showsymbolbin\pluseqq{2A72}{}
 \showsymbolbin\interleave{2AF4}{}
 \showsymbolbin\nhVvert{2AF5}{}
 \showsymbolbin\threedotcolon{2AF6}{}
 \showsymbolbin\trslash{2AFB}{}
 \showsymbolbin\sslash{2AFD}{}
 \showsymbolbin\talloblong{2AFE}{}
 \end{multicols}







\DocInput{\jobname.dtx}
\nocite{*}
\printbibliography

\printindex 
 %
% 
\def\contentsname{Contents}
\end{document}
%</driver>
% \fi
% 
%  \CheckSum{0}
%  \CharacterTable
%  {Upper-case    \A\B\C\D\E\F\G\H\I\J\K\L\M\N\O\P\Q\R\S\T\U\V\W\X\Y\Z
%   Lower-case    \a\b\c\d\e\f\g\h\i\j\k\l\m\n\o\p\q\r\s\t\u\v\w\x\y\z
%   Digits        \0\1\2\3\4\5\6\7\8\9
%   Exclamation   \!     Double quote  \"     Hash (number) \#
%   Dollar        \$     Percent       \%     Ampersand     \&
%   Acute accent  \'     Left paren    \(     Right paren   \)
%   Asterisk      \*     Plus          \+     Comma         \,
%   Minus         \-     Point         \.     Solidus       \/
%   Colon         \:     Semicolon     \;     Less than     \<
%   Equals        \=     Greater than  \>     Question mark \?
%   Commercial at \@     Left bracket  \[     Backslash     \\
%   Right bracket \]     Circumflex    \^     Underscore    \_
%   Grave accent  \`     Left brace    \{     Vertical bar  \|
%   Right brace   \}     Tilde         \~}
%
%
%
% \changes{1.0}{2013/01/26}{Converted to DTX file}
%
% \DoNotIndex{\newcommand,\newenvironment}
% \GetFileInfo{phd.dtx}
% 
%  \def\fileversion{v1.0}          
%  \def \filedate{2012/03/06}
% \title{The \textsf{phd} package.
% \thanks{This
%        file (\texttt{phd.dtx}) has version number \fileversion, last revised
%        \filedate.}
% }
% \author{Dr. Yiannis Lazarides \\ \url{yannislaz@gmail.com}}
% \date{\filedate}
%
%
% 
% ^^A\maketitle
% 
% ^^A\frontmatter
%  ^^A\coverpage{./images/hine02.jpg}{Book Design }{Camel Press}{}{}
%  \newpage
% ^^A\secondpage
% \pagestyle{empty}
%
%
% 
%
%
% \pagestyle{headings}
% \raggedbottom
%  \OnlyDescription
%
%  ^^A\StopEventually{\printindex}
%
% \CodelineNumbered
% \pagestyle{headings}
% 
% 
% ^^A\part{IMPLEMENTATION AND FRIENDS}
% 
%
% \chapter{Code Implementation Objectives and Strategy}
% \parindent1em
% \epigraph{
% ``Lord Campbell [John Campbell, 1st Baron Campbell, 1779-1861] proposed that any author who published a book without an index should be deprived of the benefits of the Copyright Act; and the Hon. Horace Binney, LL.D. [1780-1875], a distinguished American lawyer, held the same views, and would have condemned the culprit to the same punishment.''}
%{-- H[enry] B[enjamin] Wheatley, How To Make An Index
%(New York: A. C. Armstrong \& Son, 1902), 82}
%
% This package aims at providing authors with a set of tools and settings
% that can improve the typesetting of documentation and especially indices.
% 
% For the normal author there are both mark-up related macros, as well as a set
% of settings for indices.
%
% My main motivation for developing this package was to group all the special
% documentation macros that I have used in developing the \pkg{phd} package. I also saw
% the need to hook up these settings with the concept of color palettes as 
% described in the \pkg{phd-colorpalette} package. This enables the integration
% of full document templates.
%
% \section{Requirements Specification}
% \begin{enumerate}
% \item To provide a declarative interface to enable users to modify documentation
%       and index  entries by setting keys, rather than writing macros.  It is suggested
%       that the author has a single style sheet that he loads to decorate a full
%       document such as an article or a book or a document that describes code.
% \item Setting of parameters will generally be with one command |\cxset|
% \item Provide a consistent `look and feel' for all code displays.
% \item Support minted and listings.
% \item Provide a set of environments and commands for documenting symbols and fonts.
% \item Seamless integration with  doc type classes. As a minimal to support |ltxdoc| and 
%          \docClass{phddoc}. The latter is provided as part of the |phd| budle.
% \item To provide a number of templates that cover most of the typical use case.
% \item To provide an easy  plug-in architecture for extensions.
% \item Provide a detailed user manual.
% \end{enumerate}
% 
% The development of this package predates |tcolorbox| and |minted|. As they are
%  both excellent packages, I have abandoned some of my code and incorporated
% code from |tcolorbox| extensively. 
%
% \section{Terminology}
%
%  \begin{description}
%  \item [document] Any written item, as a book, article, or letter, especially 
%                  of a factual or informative nature.
%  \item [heading] A division of a document or document series. For a normal
%        book headings are chapters, sections etc. However we allow for
%        specifying a more complex document divided into books, volumes
%        parts etc. For example the Bible has Books, chapters and verses,
%        where a legal document might require divisions such as clauses.
%        In general these divisions are numbered. These document divisions
%        are stored in the comma list \refCom{phd_book_divisions_clist}.
%  \item [head] A typeset heading, such as chapter head, or section head.
%        This can include a counter, label and title for example, 
%        \emph{Chapter 1 Introduction}.
%  \item [dom] This is a programming interface that provides a structured
%        representation of the document (a tree) and it defines a way
%        that the structure can be accessed. Although \latexe does not
%        offer a standard way to build such a tree (mainly because
%        \tex does not require the marking of paragraphs, it is 
%        useful to think of the document as a tree structure. We also
%        allow for a semi-automated way to build such a tree (with the 
%        exception that paragraphs are not included).
% \item [element] A part of the document tree that can be styled on
%       its own. For example the chapter label, or the section number.
%
% \end{description}
%
% \section{Users}
%  We classify users according to the \LaTeX3 terminology as a) programmers b) template designers
%  and c) authors.
% \subsection{Author}
%  We assume that the author has an exising template which she is using but might want to do
%  some minor modifications, for example use an italic shape for the font of the mark, but an 
%  upright font for the page numbers. 
%
%
% We follow the idea of representing the basic elements of documents
% as elements, each one having a parent in order to specify
% the element we need to style as accurate as possible. One can think of
% this approach being congruent with objects in other languages.
% As a matter fact nothing stops us from defining a key value
% interface as shown below.
%
% {\obeylines 
%~~ |\cxset|
%~~~~~|{| 
%~~~~~~~~\textit{header.even.mark.font.size}   = |Large,|
%~~~~~~~~\textit{header.even.mark.font.family} = |serif,|
%~~~~~|}|
%}  
%
% This would pehaps make it easier for the template designer, but I have rejected
% the idea as my aim is to make it easy for the author, who can search the template
% and just enter a couple of new proerty values.
%
% \subsection{Template designer}
% \pagestyle{headings}
% The template designer in the example above would have selected the format style
% from a number of predefined formats (templates) or would have created a style
% called \textit{apa} from an existing template and modified it using declarative
% key style.
%
% \subsection{The programmer}
%
% The programmer in the example above could have created the basic format
% \textit{apa} by using both declarative as well as defining or using existing
% macros. To the programmer we offer an extension mechanism, where the contents
% of a |ps@| command are defined. For example the programmer can define a new
% style using \tikzname, but without having to worry about defining full |ps@|
% and their interface.
%
%
% \section{Acknowledgements}
%
% This package couldn't have been possible if it was not for the documentation
% section of tcolorbox. I have liberally taken ideas and code from Dr. Thomas F. Sturm's 
% package, which in turn draws strongly from the |PGF| manual. I am grateful to both.
% 
% \iffalse
%<*DOCUM>
% \fi
%  
% \section{Preliminaries}
%
% We declare that we use \latexe and name the package. The code has been
% moved to the latest version of \latexe to take advantage of more allocations.
% Package works well with LuaLaTeX.\tcbdocmarginnote{7/8/2017}
%
%    \begin{macrocode}
\NeedsTeXFormat{LaTeX2e}[2017/04/15]%
\ProvidesFile{phd-documentation}[2017/04/15 v1.0 less preamble (YL)]%
%    \end{macrocode}
%
% We also need the \pkg{refcount}. 
%
% \section{tcolorbox}
% We load \pkg{tcolorbox} with options theorems, skins, documentation etc
% for internal and external use.
%
% We also provide an interface, between the \pkg{tcolorbox} documentation
% keys and our own.
% 
% The indexing keys are still to be sorted out with other sections of the
% documentation, but they seem to be working for the moment.
% 
% 
%    \begin{macrocode}
\newcommand*{\verbatimfont}{\ttfamily}%

\let\displayverbfont\ttfamily

\renewcommand*{\verbatim@font}{\verbatimfamily}

%    \end{macrocode}
%
%    \begin{macrocode}
\RequirePackage[theorems, skins, documentation,
                breakable,listings]{tcolorbox}
%                
                \tcbset{index format=pgfchapter,
                        index actual={=},
                        index level = {>},
                        index quote = {!},
                        index german settings,
                        color hyperlink = thelinkcolor ,
                        color definition =thelinkcolor,
                   }
\tcbset{halostyle/.style={fuzzy halo=2mm with magenta!5}}                   

           
\def\tcb@Print@Com#1{\textcolor{\kvtcb@col@command}{\ttbf\char`\\ \detokenize{#1}}}
%    \end{macrocode}                
%
\cxset {doc command color/.code = \tcbset{color command = #1}}
\cxset {doc command color= thecmdcolor}
%
%    \begin{macrocode}
\lstdefinelanguage{extras}{morekeywords={%
      poemtitle, poemtoc, versewidth, 
      vin, poemlines,poemtitlefont, 
      ProvidesClass,IfFileExists,
      RequirePackage,ifthenelse,chapter,
      includegraphics, newarray,readarray,of
}}
\lstloadlanguages{[LaTeX]TeX, [primitive]TeX, extras}
%    \end{macrocode}
%
% Note the |gobble=1| option. We use this to make the colorboxes
% with code not to show the `\%` sign in this documentation.
% Ideally you should fork the code below and adapt it to 
% your own needs.
%
% Also note that this is the default that is to be used in
% \pkg{tcolorbox} commands.
% 
% \begin{docEnvironment}[doclang/environment content=text]{scriptexample}{\oarg{options}}
% Display box for examples of various languages and scripts.
%    \begin{macrocode}
\newtcolorbox{scriptexample}[2][]{colback=thecodebackground,
boxrule=0pt,toprule=0pt,colframe=white,#1}
%    \end{macrocode}
% \end{docEnvironment}
% \begin{scriptexample}[title=Script Example,]{Cypriot Syllabary}
%\unicodetable{cypriote}{"10800,"10810,"10820,"10830}
%\end{scriptexample}
%  \captionof{figure}{Example of using the \docAuxEnvironment*{scriptexample} environment. The box shows the unicode points for the Cypriote script.}
%    \begin{macrocode}
\newtcolorbox{commands}[2][]{colback=thecodebackground,
boxrule=0pt,toprule=0pt,colframe=white}
%    \end{macrocode}
%
% Set the \pkg{listings} defaults for TeX and LaTeX. All colours are harmonized with the definitions
% od the color palette, as set in the \pkg{phd-colorpalette}.
%    \begin{macrocode}
\lstset{language={[LaTeX]TeX},
       escapeinside={{(*@}{@*)}}, 
       numbers=left, 
       gobble=0,
       stepnumber=1,
       numbersep=5pt, 
       numberstyle={\footnotesize\color{thegray}},
       breaklines=false,
       framesep=5pt,
       basicstyle=\small\ttfamily,
       showstringspaces=false,
       stringstyle={\color{orange}\footnotesize},
       commentstyle=\color{black},
       rulecolor=\color{theshade},
       breakatwhitespace=true,
       showspaces=false, 
       xleftmargin=10pt,
       xrightmargin=0pt,
       aboveskip=3pt plus1pt minus1pt, 
       belowskip=7pt plus1pt minus1pt,  
       backgroundcolor=\color{theshade},
}
%    \end{macrocode}
% 
% We predefine some listings styles to use with color palettes. First we define three styles for TeX and LaTeX, as well
% as three environements.\index{listings styles>simple}\docAuxListingsStyle{simple}
%    \begin{macrocode}
\lstdefinestyle{simple}{%
       escapeinside={{(*@}{@*)}}, 
       numbers=left, 
       gobble=0,
       stepnumber=1,
       numbersep=5pt, 
       numberstyle={\footnotesize\color{gray}},
       breaklines=false,
       framesep=5pt,
       basicstyle=\small\ttfamily,
       showstringspaces=false,
       stringstyle={\color{thestringstyle}\footnotesize},
       commentstyle=\color{thecommentstyle},
       rulecolor=\color{theshade},
       breakatwhitespace=true,
       showspaces=false, 
       xleftmargin=10pt,
       xrightmargin=0pt,
       aboveskip=3pt plus1pt minus1pt, 
       belowskip=3pt plus1pt minus1pt,  
       backgroundcolor=\color{thecodebackground},
       showlines=true,
}

\lstdefinestyle{singleline}{%
       escapeinside={{(*@}{@*)}}, 
       numbers=left, 
       gobble=0,
       stepnumber=1,
       numbersep=5pt, 
       numberstyle={\footnotesize\color{gray}},
       breaklines=false,
       framesep=5pt,
       basicstyle=\small\ttfamily,
       showstringspaces=false,
       stringstyle={\color{thestringstyle}\footnotesize},
       commentstyle=\color{thecommentstyle},
       rulecolor=\color{theshade},
       breakatwhitespace=true,
       showspaces=false, 
       xleftmargin=0pt,
       xrightmargin=0pt,
       aboveskip=3pt plus1pt minus1pt, 
       belowskip=3pt plus1pt minus1pt,  
       backgroundcolor=\color{thecodebackground},
       showlines=true,
       numbers=none,
}


% prints the style
\def\phd@print@LstStyle#1{\textcolor{\kvtcb@col@counter}{\tcb@doc@bfseries\tcb@scantokens{#1}}}

\DeclareDocumentCommand\docAuxListingsStyle{sm}{%
  \phd@print@LstStyle{#2}%
  \IfBooleanTF{#1}{}{\index{\phd@text@lststyles\idx@level#2}}%
}
%    \end{macrocode}
%
%	An example of |\docAuxListingsStyle|   \docAuxListingsStyle{simplex}.
%	
% 	The environment |\begin{TeX}..\end{TeX}| provides a listings environment
% 	for typesetting, either TeX or LaTeX code.
% 	This environment always starts the number
%    \begin{macrocode}
\lstnewenvironment{teX}[1][]
  {\lstset{language=[LaTeX]TeX}\lstset{%
      style=simple,#1
}}
{}
%    \end{macrocode}
% \begin{docEnvironment}{sverbatim}{\oarg{options}}
% A listings environment for single line verbatim typesetting. It uses the single line listings style.\docAuxListingsStyle{singleline}. It typesets with smaller values  for |aboveskip| and |belowskip|.
%    \begin{macrocode}
\lstnewenvironment{sverbatim}[1][]
  {\lstset{language=[LaTeX]TeX}\lstset{%
      style=singleline,#1
}}
{}
%    \end{macrocode}
% \end{docEnvironment}
%
% The \docAuxListingsStyle{extended} is similar to the rest, except it is numbered continuouly.
%    \begin{macrocode}
\lstdefinestyle{extended}{%
      breaklines=true,
      framesep=5pt,
      basicstyle=\verbatimfamily,
      showstringspaces=false,
      keywordstyle=\small\verbatimfamily,
      stringstyle=\color{thestringstyle},
      commentstyle=\color{thecommentstyle},
      rulecolor=\color{gray!10},
      breakatwhitespace=true,
      xleftmargin=10pt,
      xrightmargin=0pt,
      aboveskip=\medskipamount,
      belowskip=\medskipamount,
      backgroundcolor=\color{thecodebackground}, 
}
%    \end{macrocode}
% \begin{docEnvironment}[doclang/environment content=code]{teXX}{\oarg{listings options}}
%   Different listings environment.
% 
%    \begin{macrocode}
\lstnewenvironment{teXX}[1][]
  {\lstset{language=[LaTeX]TeX}\lstset{%
      style=extended, #1
}}
{}
%    \end{macrocode}
% \end{docEnvironment}
%
%
% \begin{docCommand}{continuelinenumber}{}
%   Continues code numbers from previous block
% \end{docCommand} 
%    \begin{macrocode}
\newcommand\continuelinenumber{\lstset{firstnumber=last}}
%    \end{macrocode}
% {startnumberat} 
%  The macro \cs{continueLineNumber}, provides a command
%  to start the next block of code with the code numbers continuing.
%  This requires the |listings| which is already included.
%  
%    \begin{macrocode}
% \begin{docCoomand}{startlineat}{\meta{line number}}
%    Starts a listings display with \metal{line number}.
% \end{docCommand}
% Always I forget this so I created some aliases
\DeclareDocumentCommand{\startlineat}{ m }{\lstset{firstnumber=#1}}
%    \end{macrocode}
% \begin{docCommand}{number}{\meta{line number}}
%  An alias for \docAuxCommand{startlineat}.
% \end{docCommand}
%    \begin{macrocode}
\let\numberlineat\startlineat
\let\startnumberat\numberlineat
%    \end{macrocode}
% 
% 
%
%    \begin{macrocode}
\newcommand\emphasis[2][black!80]{%
   \lstset{%
     emph={write, writeln,#2},
     escapeinside={(*@}{@*)},
     emphstyle={\verbatimfont%
                \bfseries%
                \textcolor{#1}%
                },
   }%
}
%    \end{macrocode}
% \begin{docEnvironment}{teXXX}{\oarg{options}}
% Environment for typestting long tracks of code with full highlighting of keywords, uses the \docAuxListingsStyle{highlite} style.
% \end{docEnvironment}
%    \begin{macrocode}
\lstdefinestyle{highlite}{%
      alsolanguage=makeindex,
      alsolanguage=BibTeX,
      emph={cs, use,new,seq,map,inline,eq,gincr,incr,IfNoValueF,%
      if,If,exist,protect,nopar,gset,%
      set,undefine,define,add,gadd,remove,div,newcounter%
      round,truncate,max,min,mod,gzero,int,%
      zero,newcount,protected,msg,error,DeclareDocumentCommand},
      emphstyle=\ttfamily\color{thered},
      firstnumber=last,
      stepnumber=1,
      escapeinside={{(*@}{@*)}},
      breaklines=true,
      framesep=5pt,
      basicstyle= \ttfamily,
      showstringspaces=false,
      keywordstyle=\ttfamily\color{thered},%\color{primary},
      keywordstyle=[2]\ttfamily\color{black},
      stringstyle=\color{thestringstyle},
      commentstyle=\color{thecommentstyle},
      rulecolor=\color{gray!10},
      breakatwhitespace=true,
      prebreak={\Righttorque},
      postbreak={\space\Lefttorque},
      showspaces=false,  % shows spacing symbol
      upquote=true,
      xleftmargin=10pt,
      xrightmargin=0pt,
      aboveskip=\medskipamount,
      belowskip=\medskipamount,
      backgroundcolor=,
}

\lstnewenvironment{teXXX}[1][]
  {\lstset{language=[LaTeX]TeX}%
    \lstset{%
      style=highlite,  #1
}}
{}
\lstnewenvironment{phdverbatim}[1][]
  {\lstset{language=[LaTeX]TeX}%
    \lstset{%
      emph={cs, use,new,seq,map,inline,eq,gincr,incr,IfNoValueF,%
      if,If,exist,protect,nopar,gset,%
      set,undefine,define,add,gadd,remove,div,%
      round,truncate,max,min,mod,gzero,int,%
      zero,newcount,protected,msg,error,DeclareDocumentCommand},
      emphstyle=\verbatimfont\bfseries\color{black!80},
      numbers=none,
     % stepnumber=1,
      escapeinside={{(*@}{@*)}},
      breaklines=false,
      framesep=5pt,
      basicstyle= {\small\ttfamily},
      showstringspaces=false,
      keywordstyle=\ttfamily\color{thekeywordstyle},
      stringstyle=\color{black!50},
      commentstyle=\color{black!50},
	rulecolor=\color{gray!10},
      breakatwhitespace=true,
      showspaces=false,  % shows spacing symbol
	xleftmargin=15pt,
	xrightmargin=5pt,
	%  aboveskip=0pt, % compact the code looks ugly in type
	% belowskip=0pt,  % user responsible to insert any skips
	aboveskip=\medskipamount,
	belowskip=\medskipamount,
       backgroundcolor=,
       #1
}}
{}
%    \end{macrocode}
% The following have been taken from| https://github.com/cgnieder/cnltx/blob/master/cnltx-listings.sty|
%    \begin{macrocode}
\lstdefinelanguage{makeindex}{
  morekeywords = {
    actual,           % @
    arg_open,         % {
    arg_close,        % }
    encap,            % |
    escape,           % \\
    keyword,          % \\indexentry
    level,            % !
    page_compositor,  % -
    quote,            % "
    range_open,       % (
    range_close,      % )
    preamble,         % \\begin{theindex} \n
    postamble,        % \n\n \end{theindex} \n
    setpage_prefix,   % \n \\setcounter{page}{
    setpage_suffix,   % } \n
    group_skip,       % \n\n \\indexspace \n
    headings_flag,    % 0
    heading_prefix,   %
    heading_suffix,   %
    symhead_positive, % Symbols
    symhead_negative, % symbols
    numhead_positive, % Numbers
    numhead_negative, % numbers
    item_0,           % \n \\item
    item_1,           % \n \\subitem
    item_2,           % \n \\subsubitem
    item_01,          % \n \\subitem
    item_x1,          % \n \\subitem
    item_12,          % \n \\subsubitem
    item_x2,          % \n \\subsubitem
    delim_0,          % , 
    delim_1,          % , 
    delim_2,          % , 
    delim_n,          % , 
    delim_r,          % --
    delim_t,          %
    suffix_2p,        %
    suffix_3p,        %
    suffix_mp,        %
    encap_prefix,     % \\
    encap_infix,      % {
    encap-suffix,     % }
    line_max,         % 72
    indent_space,     % \t\t
    indent_length     % 16
  } ,
  morestring = [b]{"} ,
  morecomment = [l]{\%} ,
  sensitive = true
}
% --------------------------------------------------------------------------
% a listings language BibTeX:
\lstdefinelanguage{BibTeX}{
  % entry types:
  morekeywords = {
    % regular types:
    @article,
    @book,@mvbook,@inbook,@bookinbook,@suppbook,@booklet,
    @collection,@mvcollection,@incollection,@suppcollection,
    @manual,
    @misc,
    @online,
    @patent,
    @periodical,
    @suppperiodical,
    @proceedings,@mvproceedings,@inproceedings,
    @reference,@mvreference,@inreference,
    @report,
    @set,
    @thesis,
    @unpublished,
    @xdata,
    @customa,@customb,@customc,@customd,@custome,@customf,
    % type aliases:
    @conference,
    @electronic,
    @mastersthesis,
    @phdthesis,
    @techreport,
    @www,
    % unsupported types:
    @artwork,
    @audiobibnote,
    @commentary,
    @image,
    @jurisdiction,
    @legislation,
    @legal,
    @letter,
    @movie,
    @music
    @performance,
    @review,
    @software,
    @standard,
    @video,
    % cnltx types:
    @bundle,
    @class,
    @package,
  } ,
  % entry fields:
  morekeywords = [2]{
    % data fields:
    abstract,
    addendum,
    afterword,
    annotation,
    annotator,
    author,
    authortype,
    bookauthor,
    bookpagination,
    booksubtitle,
    booktitle,
    booktitleaddon,
    chapter,
    commentator,
    date,
    doi,
    edition,
    editor,editora,editorb,editorc,
    editortype,
    editoratype,editorbtype,editorctype,
    eid,
    entrysubtype,
    eprint,
    eprintclass,
    eprinttype,
    eventdate,
    eventtitle,
    eventtitleaddon,
    file,
    foreword,
    hvarer,
    howpublished,
    indextitle,
    institution,
    introduction,
    isan,
    isbn,
    ismn,
    isrn,
    issn,
    issue,
    issuesubtitle,
    issuetitle,
    iswc,
    journalsubtitle,
    journaltitle,
    label,
    language,
    library,
    location,
    mainsubtitle,
    maintitle,
    maintitleaddon,
    month,
    nameaddon,
    note,
    number,
    organization,
    origdate,
    origlanguage,
    origlocation,
    origpublisher,
    origtitle,
    pages,
    pagetotal,
    pagination,
    part,
    publisher,
    pubstate,
    reprinttitle,
    series,
    shortauthor,
    shorteditor,
    shorthand,
    shorthandintro,
    shortjournal,
    shortseries,
    shorttitle,
    subtitle,
    title,
    titleaddon,
    translator,
    type,
    url,
    urldate,
    venue,
    version,
    volume,
    volumes,
    year,
    % special fields:
    crossref,
    entryset,
    execute,
    gender,
    hyphenation,
    ids,
    indexsorttitle,
    keywords,
    options,
    presort,
    related,
    relatedoptions,
    relatedtype,
    relatedstring,
    sortkey,
    sortname,
    sortshorthand,
    sorttitle,
    sortyear,
    xdata,
    xref,
    % custom fields:
    namea,nameb,namec,
    nameatype,namebtype,namectype,
    lista,listb,listc,listd,liste,listf,
    usera,userb,userc,userd,usere,userf,
    verba,verbb,verbc,
    % field aliases:
    address,
    annote,
    archiveprefix,
    journal,
    key,
    pdf,
    primaryclass,
    school,
    % cnltx fields:
    maintainer
  } ,
  morestring = [b]{"} ,
  morecomment = [l]{\%} ,
  sensitive = false
}

% listings style for source code from igniter
\lstdefinestyle{complex}{
  basicstyle       =  \small\ttfamily,  %{\sourceformat},
  numbers          = left,
  numberstyle      = \tiny,
  xleftmargin      = 1em,
  numbersep        = .75em,
  gobble           = \@gobble ,
  columns          = fullflexible,
  literate         =
    {ä}{{\"a}}{1}
    {ö}{{\"o}}{1}
    {ü}{{\"u}}{1}
    {Ä}{{\"A}}{1}
    {Ö}{{\"O}}{1}
    {Ü}{{\"U}}{1}
    {ß}{{\ss}}{1}
    {`}{\`{}}{1}
    {'}{\textquotesingle}{1} ,
  breaklines       = true,
  keepspaces       = true,
  breakindent      = 1em,
  commentstyle     = \color{thecomment},
  keywordstyle     = \color{thecs},
  deletetexcs      =
    {
      a,o,u,A,O,U,
      begin,
      center,
      description,document,
      end,enumerate,
      equation,eqnarray,
      figure,flushleft,flushright,
      itemize,list,
      otherlanguage,
      table,tabu,tabular
    },
  deletekeywords   =
    {
      a,o,u,A,O,U,
      begin,
      center,
      description,document,
      end,enumerate,
      equation,eqnarray,
      figure,flushleft,flushright,
      itemize,list,
      otherlanguage,
      table,tabu,tabular
    },
  % \begin, \end:
  texcsstyle       = [2]\color{red}, %add color begin end
  index            = [2][texcs2],
  indexstyle       = [2]\@gobble,
  moretexcs        = [2]{begin,end},
  % control sequences that'll be indexed:
  texcsstyle       = [3]\color{cs},
  index              = [3][texcs3],
  indexstyle       = [3]\indexcs,
  % control sequences that won't be indexed:
  texcsstyle       = [4]\color{thecs},
  index            = [4][texcs4],
  indexstyle       = [4]\@gobble,
  % added environments that'll be indexed:
  texcsstyle       = [5]\color{env},
  index            = [5][texcs5],
  %indexstyle       = [5]\indexenv,   UNCOMMENT LATER
  % environments that won't be indexed:
  texcsstyle       = [6]\color{env},
  index            = [6][texcs6],
  indexstyle       = [6]\@gobble,
  moredelim        = *[s][\color{green}]{$}{$} %add color the math
}
%    \end{macrocode}
% 
%
%
%    \begin{macrocode}
\lstnewenvironment{lualisting}[1][]
{\lstset{language=[LaTeX]TeX,
  basicstyle           = \ttfamily,
  showstringspaces     = false,
  upquote              = true,
  keywordstyle         =\color{blue},
  commentstyle         =\color{black!50},
  stringstyle          =\color{black!80},
  backgroundcolor      =\color{white},
  xleftmargin          =15pt,
  xrightmargin         =5pt,
  aboveskip            =\medskipamount,
  belowskip	            =\medskipamount,
  #1}}
{}
%    \end{macrocode}
% \section{LaTeX code demo environments}
%
%	To demonstrate \latex code it is sometimes desirable to have the code
%	be executed. This was pioneered in a number of packages. One of
%	the better packages to do so is \pkg{tcolorbox}. We use it to define
%	a special environment.
%
% \begin{docEnvironment}{texexample}{ \marg{title} \marg{label for referencing} } 
% The environment |texexample| will list the code
%	using the \pkgname{listings} package, so we can have a nice box and shows the
%	output at the bottom section.
% \end{docEnvironment}
%	
%	First we define a new counter which resets at every chapter. If |c@chapter|
%	is not defined we reset it based on sections.
%
% \begin{enumerate}
%	\item [\#1] Title of the example
%	\item [\#2] label for referencing
% \end{enumerate}
% 
%    \begin{macrocode}
  \ifx\c@chapter\@undefined
    \newcounter{texexp}[section]
    \@addtoreset{c@texexp}{c@section}
  \else
    \newcounter{texexp}[chapter]
    \@addtoreset{c@texexp}{c@chapter}
  \fi 
%    \end{macrocode}
%
%	
%    \begin{macrocode}
%\tcbset{listing utf8=latin1}% optional; ’latin1’ is the default.
\def\dcircle#1{\ding{\numexpr181 + #1\relax}}
\def\thetexexp{\@arabic\c@section.\arabic{texexp}}
%    \end{macrocode}
%    \begin{macrocode}    
\tcbset{texexp/.style={% 
    fonttitle=\small\ttfamily, 
    fontupper=\small, 
    fontlower=\small,
    coltitle=black,
    colback = thecodebackground,% background
    colframe=thecodebackground, 
   % process code={\def\dcircle##1{\ding{\numexpr181 + ##1}}},
      %colupper=spot!,
   },
   listing options = {%
     keywordstyle=\color{thekeywordstyle},
     belowskip=0pt, 
     escapeinside={(*@}{@*)},%
     breaklines=true,%
     backgroundcolor=\color{thecodebackground},%
     %firstnumber=last,%
     stepnumber=1,%
     upquote=true,%
     alsoletter={_,:},%
     commentstyle=\color{thecommentstyle},%
     emph={cs,new,seq,map,inline,eq,gincr,incr,IfNoValueF,if,%
            If,exist,protect,nopar,gset,%
            set,undefine,define,add,gadd,remove,div,%
            round,truncate,max,min,mod,gzero,int,exp,put,left,args,%
            zero,newcount,protected,msg,error,%
            eval,to,arabic,alph,Alph,roman,Roman,dim%
            DeclareDocumentCommand,%
            NewDocumentCommand,%
            RenewDocumentCommand,includegraphics,
            function,local,return,break,
         },%
           %
          % For LaTeX3 we need to add these, note % is important
          % dn’t miss, at the end...
          moretexcs    = {DeclareDocumentCommand,IfBooleanTF,tex_def:D,%
          cs_new:Nn,cs_new:Npn,cs_new:cn,cs_set_nopar:Npn,token_to_meaning:N,%
          %primitives
          cs:w,cs_end:,tex_underline,group_begin:, group_end:,%
          %coffins
          NewCoffin,JoinCoffins,SetHorizontalCoffin,TypesetCoffin,%
          %properties
          prop_new:N,prop_new:c,prop_put:Nnn,%
          %boolean
          bool_new:N,bool_set_true:N,bool_set_false:N,%
          bool_if:NTF,%
          hbox_to_wd:nn,%
          IfNoValueTF,%
          %token lists
          tl_new:N,tl_set:Nn,tl_concat:NNN,%
          token_to_meaning:N,%
          seq_pop_left:NN,%
          %
          %int
          int_if_exist:cT,int_use:c,int_new:c,int_new:N,int_eval:n,%
          int_add,int_use,int_to_roman,%
          %boxes
          box_new:c,hbox_set:cn,box_use:c,vbox_set:cn,box_move_down:nn,%
          %string
          str_if_eq_x:nnTF,%
          tl_tail:n,%
          DeclareObjectType,%
          DeclareTemplateInterface,%
          DeclareTemplateCode,%
          DeclareInstance,UseInstance,AssignTemplateKeys%
          keys_set,keys_define,%      
          },%
         emphstyle=\verbatimfont\bfseries\color{theemphasiscolor},%
          %
   },%close listings options
      % added for better control
      arc=0pt,  
      outer arc=0pt,
      example1/.code 2 args={\refstepcounter{texexp}{\ifx#2\empty\else\label{#2}\fi}}%Reference
     \pgfkeysalso{texexp, enhanced, breakable, title={Example \thetexexp\ #1}%
 },
}

\def\emphasize#1{%
\tcbset{texexp/.style={% 
    fonttitle=\small\ttfamily, 
    fontupper=\small, 
    fontlower=\small,
    coltitle=black,
    colback = thecodebackground,% background
    colframe=thecodebackground, 
    %process code={\def\dcircle##1{\ding{\numexpr181 + ##1}}},
      %colupper=spot!,
   },
   listing options = {%
     keywordstyle=\color{thekeywordstyle},
     belowskip=0pt, 
     escapeinside={(*@}{@*)},%
     breaklines=true,%
     backgroundcolor=\color{thecodebackground},%
     %firstnumber=last,%
     stepnumber=1,%
     upquote=true,%
     alsoletter={_,:},%
     commentstyle=\color{thecommentstyle},%
     emph={cs,new,seq,map,inline,eq,gincr,incr,IfNoValueF,if,%
            If,exist,protect,nopar,gset,%
            set,undefine,define,add,gadd,remove,div,%
            round,truncate,max,min,mod,gzero,int,exp,put,left,args,%
            zero,newcount,protected,msg,error,%
            eval,to,arabic,alph,Alph,roman,Roman,dim%
            DeclareDocumentCommand,%
            NewDocumentCommand,%
            RenewDocumentCommand,includegraphics,
            function,local,return,#1,
         },%
           %
          % For LaTeX3 we need to add these, note % is important
          % dn’t miss, at the end...
     moretexcs    = {DeclareDocumentCommand,IfBooleanTF,tex_def:D,%
          cs_new:Nn,cs_new:Npn,cs_new:cn,cs_set_nopar:Npn,token_to_meaning:N,%
          %primitives
          cs:w,cs_end:,tex_underline,group_begin:, group_end:,%
          %coffins
          NewCoffin,JoinCoffins,SetHorizontalCoffin,TypesetCoffin,%
          %properties
          prop_new:N,prop_new:c,prop_put:Nnn,%
          %boolean
          bool_new:N,bool_set_true:N,bool_set_false:N,%
          bool_if:NTF,%
          hbox_to_wd:nn,%
          IfNoValueTF,%
          %token lists
          tl_new:N,tl_set:Nn,tl_concat:NNN,%
          token_to_meaning:N,%
          seq_pop_left:NN,%
          %
          %int
          int_if_exist:cT,int_use:c,int_new:c,int_new:N,int_eval:n,%
          int_add,int_use,int_to_roman,%
          %boxes
          box_new:c,hbox_set:cn,box_use:c,vbox_set:cn,box_move_down:nn,%
          %string
          str_if_eq_x:nnTF,%
          tl_tail:n,%
          DeclareObjectType,%
          DeclareTemplateInterface,%
          DeclareTemplateCode,%
          DeclareInstance,UseInstance,AssignTemplateKeys%
          keys_set,keys_define,%      
          },%
     emphstyle=\verbatimfont\bfseries\color{theemphasiscolor},%
          %
   },%close listings options
      % added for better control
      arc=0pt,  
      outer arc=0pt,
  }%close style
}%close command
%
\newenvironment{texexp}[1]{\tcblisting{texexp,#1}}{\endtcblisting}

\newenvironment{example1}[3][]{\tcblisting{example1={#2}{#3},#1}}%
    {\endtcblisting}
%
%    \end{macrocode}
%    
%    \begin{docEnvironment}{texexample} { \oarg{} \marg{Title} \meta{label} }
%      
%    \end{docEnvironment}
%    \begin{macrocode}
\newenvironment{texexample}[3][]{\noindent\tcblisting{example1={#2}{#3},#1}}%
    {\endtcblisting }
\newenvironment{texcode}[3][listing only]{\noindent\tcblisting{example1={#2}{#3},#1}}%
    {\endtcblisting }    
%    
% Need to fix
\let\luaexample\texexample        
\let\endluaexample\endtexexample    
%    \end{macrocode}
%     
%    \begin{macrocode}
%\tcbset{luacode/.style={%
%      fonttitle=\small\ttfamily, 
%      fontupper=\small, 
%      fontlower=\small,
%      coltitle=black,
%      colback = thecodebackground,% background
%      colframe=thecodebackground, 
%      %colupper=spot!,
%      },
%      listing options = {
%          language={[5.2]Lua},
%          belowskip=0pt, 
%          escapeinside={(*@}{@*)},%
%          breaklines=true,%
%          backgroundcolor=\color{thecodebackground},%
%          firstnumber=last,%
%          stepnumber=1,%
%          upquote=true,%
%          alsoletter={_,:},%
%          commentstyle=\bfseries\color{black!90},%
%          stringstyle = \color{black!90},
%          emphstyle=\verbatimfont\bfseries\color{black!80},%
%          keywordstyle= \bfseries\color{black!80},%
%          },
%      % added for better control
%      arc=0pt,  
%      outer arc=0pt,
%      luaexp1/.code 2 args={\refstepcounter{texexp}\label{#2}}%Reference
%     \pgfkeysalso{luacode, enhanced, breakable, title={Example \thetexexp\ #1}},
%}
%\newenvironment{luaexp1}[1]{\tcblisting{luacode,#1}}{\endtcblisting}
%
%\newenvironment{luaexample}[3][]{\noindent\tcblisting{luaexp1={#2}{#3},#1}}%
%    {\endtcblisting}
%%
%    \end{macrocode} 
%
% The following demonstrates the usage.
%
%\begin{texexample}[]{atest}{This is a comment?}
% \def\demomacro{Hello World!}
%\end{texexample}
%
% 	\begin{example}{A Test}{test}{This is a comment?}
%	  \def\demomacro{Hello World!}
%	\end{example}
%
% \section{makeidx}    
% We check that \docAuxCommand{printindex} has been defined. If the test is positive then an indexing package has been loaded, otherwise we load the \pkgname{makeidx}\footcite{makeidx}.
%  
%    \begin{macrocode}
\ExplSyntaxOn
\cs_if_exist:cTF {printindex}
  { }
  {
    \RequirePackage{makeidx}[2000/03/29]
  }
\ExplSyntaxOff  
%    \end{macrocode}
%
% \section{Refcount}
%
%References are not numbers, however they often store numerical data
%such as section or page numbers. \cs{ref} or \cs{pageref} cannot be used for
%counter assignments or calculations because they are not expandable, generate
%warnings, or can even be links. The package provides expandable macros
%to extract the data from references. Packages \pkgname{hyperref}, \pkgname{nameref}\footfullcite{nameref}, 
% \pkgname{titleref}\footfullcite{}, and
% \pkgname{babel} are supported.
%    \begin{macrocode}  
\RequirePackage{refcount}[2011/10/16]
%    \end{macrocode}
%    \begin{macrocode}
\ExplSyntaxOn
\cs_set:Npn \cvaref#1{\textcolor{\phdkv@col@command}{#1}}
\cs_set:Npn \colOpt#1{\textcolor{\phdkv@col@opt}{#1}}
\ExplSyntaxOff
\lstdefinestyle{tcbdocumentation}{language={[LaTeX]TeX},
    aboveskip={0\p@ \@plus 6\p@},
    belowskip={0\p@ \@plus 6\p@},
    columns=fullflexible,
    keepspaces=true,
    breaklines=true,
    prebreak={\Righttorque},
    postbreak={\space\Lefttorque},
    breakatwhitespace=true,
    basicstyle=\ttfamily\footnotesize,
    extendedchars=true,
    nolol,
    inputencoding = \phdkv@listingencoding}
%    \end{macrocode}
%
%  \section{Documenting Macros}
% The following macros are taken from \docClass{ltxdoc} and modified accordingly.
%
%  \begin{docEnvironment}[doclang/environment content=command description]{docCommand}{\oarg{options}\marg{name}\marg{parameters}}
% \end{docEnvironment}
%
%    \begin{macrocode}
\DeclareRobustCommand\phdcs[1]{{\color{thecs}{\texttt{\char`\\\detokenize{#1}}}}}
\let\phd@doc@orig@meta\meta%
%    \end{macrocode}
%
% \begin{docCommand}{meta}{\marg{argument}}
% We modify the standard \meta{arguments} to allow for colour settings.  The \docColor{themeta} is defined in the \pkg{phd-colorpalette} package. The braces are not colored
% as they do not look very good if they are. 
% \end{docCommand}
%
%  Use the original |meta| as defined in |doc| and only colorize and set the font parameters through the |doc|
%  provided macro hook |\meta@font@select|.
%    \begin{macrocode}
\ExplSyntaxOn
\cs_set:Npn \meta #1 {
   \group_begin:
   \def\meta@font@select{\rmfamily\itshape\color{themeta}}
   \phd@doc@orig@meta{#1}
   \group_end:
}
%    \end{macrocode}
% \begin{docCommand}{marg}{\marg{mandatory argument}}
%  Typesets a mandatory argument as \meta{argument}.
% \end{docCommand}
%    \begin{macrocode}
\cs_set:Npn \marg #1
  {
    \group_begin:
      \ttfamily\char`\{
      \def\meta@font@select{\rmfamily\itshape\color{theoarg}}
      \phd@doc@orig@meta{#1}
      \ttfamily\char`\}
    \group_end:
  }
%    \end{macrocode}
% 
% \begin{docCommand}{oarg}{\marg{argument}}
% Typesets an optional argument, as found in \latexe commands. The command:
% \begin{quote}
%  |\test\oarg{style=two}\marg{mandatory arguments}| 
% \end{quote}
% will typeset:
%  \begin{quote}
%   |\test|\oarg{style=two}\marg{mandatory arguments}
%  \end{quote}
% \end{docCommand} 
%    \begin{macrocode}
\cs_set:Npn \oarg #1
  {
    {\ttfamily[}\meta{#1}{\ttfamily]}
  }
\ExplSyntaxOff
%    \end{macrocode}
%
%\begin{docCommand}[doc new=2015-01-08]{docCounter}{\marg{name}}
%  Documents a counter with given \meta{name}. The counter is automatically indexed.
%\begin{dispExample}
%The counter \docCounter{foocounter} can be used for computation.
%\end{dispExample}
%\end{docCommand}

%\begin{docCommand}[doc new=2015-01-08]{docCounter*}{\marg{name}}
%  Identical to \refCom{docCounter}, but without index entry.
%\end{docCommand}


% \begin{docCommand}[doc updated=2017-10-11] {phdcs} { \marg{macro name}}
%   We modify the standard |\cs| and save to a new name to be able to use
%   underscores. Maybe there are better ways of doing it as well.
% \end{docCommand}

%    \begin{macrocode}
\newif\ifphd@doc@toindex
\newif\ifphd@doc@colorize
\newif\ifphd@doc@annotate
%    \end{macrocode}
%
%    \begin{macrocode}
% language specific texts
\cxset{
  pageshort/.store in=\phdkv@text@pageshort,
  doclang/.cd,
  color/.store in=\phdkv@text@color,
  colors/.store in=\phdkv@text@colors,
  environment content/.store in=\phdkv@text@envcontent,
  environment/.store in=\phdkv@text@env,
  environments/.store in=\phdkv@text@envs,
  key/.store in=\phdkv@text@key,
  keys/.store in=\phdkv@text@keys,
  index/.store in=\phdkv@text@index,
  pageshort/.store in=\phdkv@text@pageshort,
  value/.store in=\phdkv@text@value,
  values/.store in=\phdkv@text@values,
  % List Styles text in english
  lst styles/.store in=\phd@text@lststyles,
}   
\cxset{doclang/lst styles=Listings Styles}
%    \end{macrocode}
%
% \section{Documentation commands key definitions}
%
%    \begin{macrocode}
\cxset
  {
    documentation listing options/.store in=\phdkv@doclistingoptions,%
    documentation listing style/.style={documentation listing options={style=#1}},%
    documentation minted style/.store in=\phdkv@docmintstyle,
    documentation minted options/.store in=\phdkv@docmintoptions,
  }
%    \end{macrocode}
% Define keys for all related color commands. 
%    \begin{macrocode}  
\cxset{    
    color command/.store in=\phdkv@col@command,
    color environment/.store in=\phdkv@col@environment,
    color key/.store in=\phdkv@col@key,
    color value/.store in=\phdkv@col@value,
    color color/.store in=\phdkv@col@color,
    color definition/.style={color command={#1},
                                       color environment={#1},
                                       color key={#1},
                                       color value={#1},
                                       color color={#1}},
    color option/.store in=\phdkv@col@opt,
    color hyperlink/.store in=\phdkv@colhyper,
    color frame/.store in=\phdkv@colhyper,
    color meta/.store in =\pkdkv@colmeta,
%    
    before example/.store in=\phdkv@beforeexample,
    after example/.store in=\phdkv@afterexample,
}
%    \end{macrocode}
%    \begin{macrocode}
%</DOCUM>
%<*DOCUM|DFLT>
%<DOCUM>\cxset {
   color command     = thecs,
   color environment = theenvironment,
   color key              = thekey,
   color value           = thevalue,
   color color            = thevalue,
   color option          = theoption,
   color meta            = themeta,
   color frame           = the frame, 
%<DOCUM> }
%</DOCUM|DFLT>
%    \end{macrocode}
%
%<*DOCUM>
% \section{Index settings}
%
%  We consider indices to be composed of three elements, the index heading 
%  i.e., the word Index typeset in a specific language, the entries and the
%  page numbers. The following keys relate to settings that must have a 
%  one to one relationship with the settings of the |.ist| file. Unfortunately
%  there is no easy way to achieve this. A better strategy is togenerate the \docextension{.ist} file 
%  automatically by writing the parameters to a file.\tcbdocmarginnote{16/08/2017}
%
%  \begin{docKey}[phd]{index actual}{=character}{no default, initially @}
%    Sets the character for \enquote{actual} in automatic indexing.
%  \end{docKey}
%    \begin{macrocode}
\cxset{    
    % indexing
    index actual/.store in   = \idx@actual,
    index quote/.store in    = \idx@quote,
    index level/.store in      = \idx@level,
    index format/.store in   = \idx@format,
    index encap/.store in    = \idx@encap,
    index colorize/.is if       = phd@doc@colorize,%
    index annotate/.is if     = phd@doc@annotate,%
  }
%    \end{macrocode}
%
% we provide three default styles. The first style is suitable for documenting normal text without any
% commands or fancy characters in the index. The second is for the german language. These settings
% are also suitable for any language that has the |''| messed up by Babel or the @ has its catcode
% changed.
% For better mnemonics we also have one as |doc|. We default to the \docAuxKey{index doc settings}.
%    \begin{macrocode}  
 \cxset{
    index default settings/.style={index actual={@},index quote={"},index level={!}},
    index german settings/.style={index actual={=},index quote={!},index level={>}},
    index doc settings/.style={index actual={=},index quote={!},index level={>}},
 }
%</DOCUM>
%<*DFLT|DOCUM>  
%<DOCUM> \cxset{
   % indexing
   index actual  = {@},
   index quote   = {!},
   index level    = {>},
   index doc settings,
%<DOCUM> }
%</DFLT|DOCUM>
%<*DOCUM>  
%    \end{macrocode}
% \subsection{Keys for adjusting design of command displays.}
% The following keys relate to heading describing commands, keys environments and the
%  like
%  \begin{docKey}{doc left}{=\meta{length}}{no default, initially 2em}
%   Sets the lefthand \enquote{offset} of the documentation texts for the \refCom{docCommand},
%  \refCom{docEnvironment}, \refCom{docKey} to \meta{length}.
% \end{docKey}
%  \begin{docKey}{doc right}{=\meta{length}}{no default, initially 0em}
%   Sets the righthand \enquote{offset} of the documentation texts for the \refCom{docCommand},
%  \refCom{docEnvironment}, \refCom{docKey} to \meta{length}.
% \end{docKey}
%
%    \begin{macrocode}  
\cxset{    
    doc left/.dimstore in=\phdkv@doc@left,
    doc right/.dimstore in=\phdkv@doc@right,
    doc left indent/.dimstore in=\phdkv@doc@indentleft,
    doc right indent/.dimstore in=\phdkv@doc@indentright,
    doc head command/.style={doc@head@command/.style={#1}},
    doc head environment/.style={doc@head@environment/.style={#1}},
    doc head key/.style={doc@head@key/.style={#1}},
    doc head/.style={doc head command={#1},doc head environment={#1},doc head key={#1}},
    doc description/.store in=\phdkv@doc@description,%
    doc into index/.is if=phd@doc@toindex,%
  }
%    \end{macrocode}
% 
%    \begin{macrocode}
% styles
%</DOCUM>
%<*DOCUM|DFLT>
\cxset{
  docexample/.style={colframe=ExampleFrame,colback=ExampleBack,fontlower=\footnotesize},
  documentation minted style=,
  documentation minted options={tabsize=2,fontsize=\small},
  english language/.code={\cxset{doclang/.cd,
    color=color,colors=Colors,
    environment content=environment content,
    environment=environment,environments=Environments,
    key=key,keys=Keys,
    index=Index,
    pageshort={P.},
    value=value,values=Values}},
}
%</DOCUM|DFLT>
%<*DOCUM>
\AtBeginDocument{%
  \csname phd@doc@index@\idx@format\endcsname%
  \hypersetup{
  citecolor=\phdkv@colhyper,
  linkcolor=\phdkv@colhyper,
  urlcolor=\phdkv@colhyper,
  filecolor=\phdkv@colhyper,
  menucolor=\phdkv@colhyper
}}
%    
%\cs_set:Npn \dispExample{\cxset{docexample}\phdwritetemp}
%
%\cs_set:Npn \enddispExample{%
%  \endtcbwritetemp%
%  \phdkv@beforeexample\begin{tcolorbox}%
%  \phd@doc@usetemplisting%
%  \phdlower%
%  \phdusetemp%
%  \end{tcolorbox}\phdkv@afterexample%
%}
%
%\newenvironment{dispExample*}[1]{%
%  \cxset{docexample,#1}\phdwritetemp%
%  }{\enddispExample}
%
%    \end{macrocode}
%
% Set a basic style (\docAuxListingsStyle{smalldisplay}) for dispListing.
%
%    \begin{macrocode}
\lstdefinestyle{smalldisplay}{%
  numbers=none, 
  backgroundcolor=\color{thecodebackground}, 
  xleftmargin=0pt}

\tcbset{documentation listing style=smalldisplay}
\tcbset{
   docexample/.style={%
   colframe=thecodeframe, 
   colback=thecodebackground,
   before skip=\medskipamount,
   after skip=\medskipamount,
   fontlower=\footnotesize,
   }
}
%    \end{macrocode}
%    \begin{macrocode}
%
%\newenvironment{dispListing*}[1]{%
%  \phd@layer@pushup\cxset{docexample,#1}\phdwritetemp%
%  }{\enddispListing}

% index auxiliary macros
%    \end{macrocode}

%    \begin{macrocode}
\ExplSyntaxOn
\cs_set:Npn \phdindexprintca #1#2#3 {%
  \ifphd@doc@colorize
    \textcolor{#2}
    { \texttt{#1} }
  \else
    \texttt{#1}
  \fi%
  \ifphd@doc@annotate\ 
   #3
  \fi%
}


\cs_set:Npn \phd_index_print_c#1#2{%
  \ifphd@doc@colorize
    \textcolor{#2}{\texttt{#1}}
  \else\texttt{#1}
  \fi%
}


\NewDocumentCommand{\phdindexprintcomc}{ m }
  {
    \phd_index_print_c {\phdcs{#1}}{\phdkv@col@command}
  }
\ExplSyntaxOff
%    \end{macrocode}
%
%  \begin{docCommand} {phd_print_com} { \marg{cs name} }
%    Prints a command. 
%  \end{docCommand}
%
%    \begin{macrocode}
\ExplSyntaxOn
\cs_new:Npn \phd_print_com #1
  {
    \textcolor{black}{\ttfamily\bfseries\phdcs{#1}}%\phdkv@col@command
  }
\ExplSyntaxOff  
%    \end{macrocode}

%    \begin{macrocode}
\ExplSyntaxOn
\newrobustcmd{\phdindexprintenvca}[1]
  {
    \phdindexprintca{#1}{\phdkv@col@environment}{\phdkv@text@env}
  }

\newrobustcmd{\phdindexprintenvc}[1]
  {
    \phd_index_print_c{#1}{\phdkv@col@environment}
  }
\ExplSyntaxOff
%    \end{macrocode}
%
%    \begin{macrocode}
\ExplSyntaxOn
\cs_set:Npn \phd_print_env#1
  {
    \textcolor{\phdkv@col@environment}{\ttfamily\bfseries#1}
  }

\newrobustcmd{\phdindexprintkeyca}[1]
  {
    \phdindexprintca{#1}{\phdkv@col@key}{\phdkv@text@key}
  }

\newrobustcmd{\phdindexprintkeyc}[1]{\phd_index_print_c{#1}{\phdkv@col@key}}

\cs_set:Npn \phd_print_key #1
  {
    \textcolor{\phdkv@col@key}{\ttfamily\bfseries#1}
  }

\newrobustcmd {\phdindexprintvalca}[1]
  {
    \phdindexprintca{#1}{\phdkv@col@value}{\phdkv@text@value}
  }

\newrobustcmd {\phdIndexPrintValC}[1]
  {
    \phd_index_print_c{#1}{\phdkv@col@value}
  }

\cs_set:Npn \phd@Print@Val #1 
  {
    \textcolor {\phdkv@col@value} {\ttfamily\bfseries#1}
  }

\newrobustcmd{\phdindexprintcolca}[1]
  {
    \phdindexprintca{#1}{\phdkv@col@color}{\phdkv@text@color}
  }

\newrobustcmd{\phdindexprintcolc}[1]
  {
    \phd_index_print_c{#1}{\phdkv@col@color}
  }

\cs_set:Npn \phd_print_col #1 
  {
    \textcolor{\phdkv@col@color}{\ttfamily\bfseries#1}
  }



\cs_set:Npn \phdindexcom #1 
  {
    \ifphd@doc@toindex
      \index
       {
         #1
         \idx@actual
         \phdindexprintcomc{#1}
       }
    \fi
  }
%    \end{macrocode}
% \begin{docCommand}{phd_index_env}{\marg{environment name}}{}
%   Auxiliary command to add an environment to the index. The environment name need to come from the key settings.
%  This need to be integrated with the i18n package later on.
% \end{docCommand}
%    \begin{macrocode}
\cs_set:Npn \phd_index_env #1
  {
    \ifphd@doc@toindex
      \index
        {#1
          \idx@actual
          \phdindexprintenvca{#1}
        }
      \index
        {
          \phdkv@text@envs
          \idx@level#1
          \idx@actual
          \phdindexprintenvc{#1}
        }
    \fi
  }
%    \end{macrocode}
%  \begin{docCommand}{phd_index_key}{\marg{key name}}
%  Indexes a key.
%  \end{docCommand}
%    \begin{macrocode}
\cs_set:Npn \phd_index_key #1 
  {
    \ifphd@doc@toindex
      \index{#1\idx@actual \phdindexprintkeyca{#1}}
      \index
        {
          \phdkv@text@keys
          \idx@level
          #1
          \idx@actual
          \phdindexprintkeyc{#1}
        }
    \fi
  }


\cs_set:Npn \phd_index_key_path #1#2
  {
    \ifphd@doc@toindex\index{#2\idx@actual  
      \phdindexprintkeyca{#2}}
      \index{\phdkv@text@keys
         \idx@level#1
        \idx@actual
        \phdindexprintkeyc{/#1/}
        \idx@level#2
        \idx@actual
        \phdindexprintkeyc{#2}
      }
    \fi
  }
%    \end{macrocode}
%
% \begin{docCommand} {phd_index_val} { \marg {value} }
% \end{docCommand}
%    \begin{macrocode}
\cs_set:Npn \phd_index_val #1  
  {
    \ifphd@doc@toindex
      \index
      {
        #1\idx@actual
        \phdindexprintvalca{#1}
      }
      \index
        {
          \phdkv@text@values
          \idx@level #1
          \idx@actual
          \phdIndexPrintValC{#1}
        }
    \fi
  }
%    \end{macrocode}
%
% \begin{docCommand} {phd_index_col} { \marg{color name} }
% \end{docCommand}
%    \begin{macrocode}
\cs_set:Npn \phd_index_col #1
  {
    \ifphd@doc@toindex
    \index
      {
        #1
        \idx@actual
        \phdindexprintcolca{#1}
      }
    \index
      {
        \phdkv@text@colors \idx@level #1
        \idx@actual\phdindexprintcolc {#1}
      }
    \fi
  }

\ExplSyntaxOff
%    \end{macrocode}
%
%   
%    \begin{macrocode}
\ExplSyntaxOn
\cs_set:Npn \phd_brackets #1
  {
    {\ttfamily\char`\{}#1{\ttfamily\char`\}}
  }
\ExplSyntaxOff
%    \end{macrocode}
%
% \begin{docEnvironment}{phd@manual@entry}{}
% Internal command for setting a user manual entry.
% \end{docEnvironment}
%    \begin{macrocode}
\ExplSyntaxOn
\newenvironment{phd@manual@entry}%
  {%
   \begin{list}{}
    {%
     \setlength{\leftmargin}{\phdkv@doc@left}%
     \setlength{\itemindent}{0pt}%
     \setlength{\itemsep}{0pt}%
     \setlength{\parsep}{0pt}%
     \setlength{\rightmargin}{\phdkv@doc@right}%
    }\item
  }
  {\end{list}}

\cs_set:Npn \phd_manual_top #1
  {
    \itemsep=0pt
    \parskip=0pt
    \item\strut{#1}\par
    \topsep=0pt
  }

\cs_set:Npn \phd_doc_do_description:
  {
    \ifx\phdkv@doc@description\@empty
    \else\phdlower
      \raggedleft(\phdkv@doc@description)
    \fi
  }
\ExplSyntaxOff
%    \end{macrocode} 
%
%  \begin{docEnvironment} {phd@doc@head} { \marg{additional options} }
%  \end{docEnvironment}
%  \begin{phd@doc@head}{}
% \lorem
% \end{phd@doc@head}
%    \begin{macrocode}
\ExplSyntaxOn
\newtcolorbox{phd@doc@head}[1]
 {
  blank,
  colback=white,
  colframe=white,
  code={\tcbdimto\tcb@temp@grow@left{-\phdkv@doc@indentleft}%
        \tcbdimto\tcb@temp@grow@right{-\phdkv@doc@indentright}},
  grow~to~left~by=\tcb@temp@grow@left,%
  grow~to~right~by=\tcb@temp@grow@right,
  sidebyside,
  sidebyside~align=top,
  sidebyside~gap=-\tcb@w@upper@real,
  phantom=\phantomsection,%
  enlarge~bottom~by=-0.2\baselineskip,
  #1
 }
\ExplSyntaxOff
%    \end{macrocode}
%
%    \begin{macrocode}
\ExplSyntaxOn
\newenvironment{docCmd}[3][]{
  \cxset{#1}%
  \begin{phd@manual@entry}%
  \begin{phd@doc@head}{doc@head@command}%
  \phd_print_com{#2}
  \phdindexcom{#2}
  \protected@edef\@currentlabel{\noexpand\phdcs{#2}}
  \label{com:#2}{\ttfamily #3}%
  \phd_doc_do_description:%
  \end{phd@doc@head}}%
  {\end{phd@manual@entry}}
\ExplSyntaxOff
%    \end{macrocode}
%

%    \begin{macrocode}
\ExplSyntaxOn
\newenvironment{docCmd*}
  {
    \bgroup
    \phd@doc@toindexfalse
    \begin{docCommand}
  }
  {
    \end{docCommand}
    \egroup
  }

\newenvironment{docEnv}[3][]{\cxset{#1}%
  \begin{phd@manual@entry}%
  \begin{phd@doc@head}{doc@head@environment}%
  \strut
  \phdcs{begin}
  \phd_brackets{\phd_print_env{#2}}
  \phd_index_env{#2}
  \protected@edef\@currentlabel{#2}\label{env:#2}{\ttfamily #3}%\par%
  \strut~~\meta{\phdkv@text@envcontent}%\par%
  \strut\phdcs{end}
  \phd_brackets{\phd_print_env{#2}}%
  \phd_doc_do_description:%
  \end{phd@doc@head}}%
  {\end{phd@manual@entry}}
\ExplSyntaxOff  
%    \end{macrocode}
%
% \begin{docEnv}{docEnv*} {}{}
% \end{docEnv}
%    \begin{macrocode}
\newenvironment{docEnv*}
  {
    \bgroup
    \phd@doc@toindexfalse
    \begin{docEnv}
  }
  { \end{docEnv}\egroup }
%    \end{macrocode}

%% \begin{docEnv}{docKey} {}{}
%% \end{docEnv}
%%    \begin{macrocode} 
%\ExplSyntaxOn
%\renewenvironment{docKey}[4][\@empty]{\begin{phd@manual@entry}%
%  \cxset{doc description={#4}}%
%  \begin{phd@doc@head}{doc@head@key}%
%  \ifx#1\@empty%
% \phd_print_key{#2}
%  \phd_index_key{#2}
%  \protected@edef\@currentlabel{#2}
%  \label{key:#2}{\ttfamily #3}%
%  \else
%   \phd_print_key{/#1/#2}
%    \phd_index_key_path{#1}{#2}
%    \protected@edef\@currentlabel{/#1/#2}
%    \label{key:/#1/#2}{\ttfamily#3}%
%  \fi%
%  \phd_doc_do_description:%
%  \end{phd@doc@head}}%
%  {\end{phd@manual@entry}}
%
%\renewenvironment{docKey*}
%  {\bgroup\phd@doc@toindexfalse\begin{docKey}}
%  {\end{docKey}\egroup}
%
%\cs_set:Npn \phdmakedocSubKey#1#2{%
%  \newenvironment{#1}[4][\@empty]{%
%    \ifx##1\@empty
%      \cs_set:Npn \phd@key@path {#2}
%    \else
%     \cs_set:Npn \phd@key@path{#2/##1}
%    \fi%
%    \begin{docKey}[\phd@key@path]{##2}{##3}{##4}}%
%    {\end{docKey}}%
%  \newenvironment{#1*}{\bgroup\phd@doc@toindexfalse\begin{#1}}{\end{#1}\egroup}%
%}
\ExplSyntaxOff
%    \end{macrocode} 

% \begin {docCommand} {docAuxCmd} { \meta{*} \meta{cs name} }
% \end {docCommand} 
%
%    \begin{macrocode}
\ExplSyntaxOn
\cs_set:Npn \docAuxCmd: #1
  {
    \phdindexprintcomc {#1}
    \phdindexcom{#1}
  }
  
\cs_set:Npn \docAuxCmd_star #1 
  {
    \phdindexprintcomc {#1}
  }

\NewDocumentCommand \docAuxCmd { s m }
  {
    \IfBooleanTF {#1}
    {\docAuxCmd_star {#2} } 
    {\docAuxCmd: {#2} }  
  } 
\ExplSyntaxOff
%    \end{macrocode}
%
%    \begin{macrocode}
\ExplSyntaxOn
\cs_set:Npn \doc_aux_env: #1
  {
    \phd_print_env{#1}
    \phd_index_env{#1}
  }
  
\cs_set:Npn \doc_aux_env_star #1
  {
    \phd_print_env{#1}
  }
%    \end{macrocode}
%
%  \begin{docCommand} {docAuxEnv} { \meta{star} \marg{arg1} }
%  \end{docCommand}
%
%    \begin{macrocode}
\NewDocumentCommand\docAuxEnv { s m } 
  {
    \IfBooleanTF {#1}   
    {\doc_aux_env_star{#2} }
    {\doc_aux_env: {#2} }
  }  
\ExplSyntaxOff  
%    \end{macrocode}
%
% \begin{docCommand} {docAuxKey} { \oarg{path} \marg{} }
% \end{docCommand}
%    \begin{macrocode}
\ExplSyntaxOn
\NewDocumentCommand {\doc_aux_key:} {O{\@empty} m}
  {%
     \ifx#1\@empty%
     \phd_print_key{#2}
     \phd_index_key{#2}%
     \else%
     \phd_print_key{/#1/#2}\phd_index_key_path{#1}{#2}%
  \fi
  }%

\newcommand{\doc_aux_key_star}[2][\@empty]{%
  \ifx#1\@empty%
   \phd_print_key {#2}%
  \else%
   \phd_print_key {/#1/#2}%
  \fi}%
  
\DeclareDocumentCommand {\docAuxKey} { s m }
  {
    \IfBooleanTF {#1}
      { \doc_aux_key_star {#2} } { \doc_aux_key: [#1]{#2} }
  }  
\ExplSyntaxOff
%    \end{macrocode} 
%
% \begin{docCommand}{docColor} { \marg{color name} } 
%   Typesets a color name and also adds it onto the index. This is identical
%   with tcolorbox, which we overwrite.
% \end{docCommand}
%
% \example The \docColor{bgsexy} is the main color used for backgrounds.
%
%    \begin{macrocode}
\ExplSyntaxOn

\cs_set:Npn \doc_color_aux #1
  {
    \phd_print_col {#1}
    \phd_index_col {#1}
  }
  
\cs_set:Npn \doc_color_star: #1
  {
    \phd_print_col{#1}
  }

\DeclareDocumentCommand {\docColor} { s m }
  {
    \IfBooleanTF {#1} 
      { \doc_color_star: {#2} }
      { \doc_color_aux {#2} }
  }
\ExplSyntaxOff  
%    \end{macrocode}

% \begin{docCommand}{docValue}{ \meta{*} \marg{value} }
% \end{docCommand}
%
%    \begin{macrocode}
\ExplSyntaxOn
\cs_set:Npn \docValue@#1{\phd@Print@Val{#1}\phd_index_val{#1}}%
\cs_set:Npn \docValue@star#1{\phd@Print@Val{#1}}%

\DeclareDocumentCommand \docValue { s m } 
  {
    \IfBooleanTF {#1}
      { \docValue@star {#2} }
      { \docValue@ {#2} }
  }
\ExplSyntaxOff  
%    \end{macrocode}
%
% \begin{docCommand} {phd_ref_doc} { \marg{ref label} }
% \end{docCommand}
%
% We use \pkgname{hyperref} to add links. The \docAuxCommand{hyperref}\oarg{label}\marg{text}
% is used to create the link. We use |\ding{213}| \ding{213} for the page see \refCmd{docColor}.
% This is a great technique pioneered in the PGF and PGFPlots manuals for cross referencing
% and the code below is an adaptation.
%
%    \begin{macrocode}
\ExplSyntaxOn
\setrefcountdefault{-1}

\cs_set:Npn \phd_ref_doc #1
{
  \hyperref[#1]
  {\texttt{\ref*{#1}}%
    \ifnum\getpagerefnumber{#1}=\thepage
    \else%
      \textsuperscript
      {
        \ding{213}\,
        \phdkv@text@pageshort\,
        \pageref*{#1}
      }
   \fi
   }
 }

\cs_set:Npn \phd_ref_doc_star#1
  {
    \hyperref[#1]{\texttt{\ref*{#1}}}
  }
\ExplSyntaxOff
%    \end{macrocode}
%
% \begin{docCommand}{refCmd} { \oarg{*} \marg{cmd name} }
% \end{docCommand}
% The \refCmd {refCmd} references a command. This is similar to
% \index{maths>mathematical}
%
%    \begin{macrocode}
\ExplSyntaxOn
\cs_set:Npn \ref_com: #1 
  {
    \phd_ref_doc{com:#1}
  }
  
\cs_set:Npn \ref_com_star #1
  {
    \phd_ref_doc_star{com:#1}
  }


\DeclareDocumentCommand {\refCmd} { s m } 
  {
    \IfBooleanTF {#1}
    { \ref_com_star {#2} } { \ref_com: {#2} }
  }  
\ExplSyntaxOff
%    \end{macrocode}
%
%    \begin{macrocode}
\ExplSyntaxOn
\cs_set:Npn \refEnv: #1 {\phd_ref_doc{env:#1}}
\cs_set:Npn \refEnv_star#1{\phd_ref_doc_star{env:#1}}

\DeclareDocumentCommand {\refEnv} { s m }
  {
    \IfBooleanTF {#1}
      { \refEnv_star {#2} }
      { \refEnv: {#2}     }
  }
\ExplSyntaxOff
%    \end{macrocode}
%
% \begin{docCommand}{refKey} {*\marg{ref text} }
% \end{docCommand}
%
%    \begin{macrocode}
\ExplSyntaxOn
\cs_set:Npn \refKey@#1{\phd_ref_doc{key:#1}}
\cs_set:Npn \refKey@star#1{\phd_ref_doc_star{key:#1}}
\DeclareDocumentCommand {\refKey} { s m }
  {
    \IfBooleanTF { #1 }
      { \refKey@star {#2} }
      { \refKey@ {#2}       }
  }
\ExplSyntaxOff
%    \end{macrocode}
%
% \begin{docCommand} {refAux} {\marg{command name}}
% \end{docCommand}
%    \begin{macrocode}
\ExplSyntaxOn
\cs_set:Npn \refAux#1{\textcolor{\phdkv@colhyper}{\ttfamily #1}}
\cs_set:Npn \refAuxcs#1{\textcolor{\phdkv@colhyper}{\phdcs{#1}}}
\ExplSyntaxOff
%    \end{macrocode}
% 
% \section{Indexing}
%  Most of the indexing macros that follow have been adapted from the pgfmanual-en-macros
%  or the tcolorbox documentation code and transliterated to expl3 language.
%
%    \begin{macrocode}
\ExplSyntaxOn
\cs_set:Npn \phd@doc@index@pgf@
  {
    \c@IndexColumns=2%
    \cs_set:Npn \theindex
      {
        \@restonecoltrue
        \columnseprule 0pt  
        \columnsep 28\p@
        \twocolumn[\index@prologue]%
        \parindent -30pt%
        \columnsep 15pt%
        \parskip 0pt plus 1pt%
        \leftskip 30pt%
        \rightskip 0pt plus 2cm%
        \small%
        \cs_set:Npn \@idxitem{\par}%
        \let\item\@idxitem\ignorespaces
      }
    \cs_set:Npn \endtheindex{\onecolumn}%
    \cs_set:Npn \noindexing{\let\index=\@gobble}%
  }
%    \end{macrocode}
%
%  \subsection{Index heading and prologue}
%
% The index prologue is text that is entered just before the indexing entries start.
% Most indices do not have any text.
% We also need to distinguish between using a chapter type heading or a section type heading.
%
%    \begin{macrocode} 
\cs_set:Npn \phd@doc@index@pgfsection{%
  \cs_set:Npn \index@prologue
    {
      \section*{\phdkv@text@index}
      \addcontentsline{toc}{section}{\phdkv@text@index}
      \par\noindent%
   }
  \phd@doc@index@pgf@
}
%    \end{macrocode}
%    \begin{macrocode}
\cs_set:Npn \phd@doc@index@pgfchapter{%
  \cs_set:Npn \index@prologue{\ifdefined\phantomsection\phantomsection\fi    
  \@makeschapterhead{\phdkv@text@index}
  \addcontentsline{toc}{chapter}{\phdkv@text@index}}%
  \phd@doc@index@pgf@%
}
%    
\let\phd@doc@index@pgf=\phd@doc@index@pgfsection%
%    \end{macrocode}
%
%    \begin{macrocode}
\cs_set:Npn \phd@doc@index@doc
  {
    \let\phdindexcom      = \SpecialMainIndex
    \let\phd_index_env    = \SpecialMainEnvIndex
    \cxset{index german settings}
    \EnableCrossrefs
    \PageIndex
}

\cs_set:Npn \phd@doc@index@off{}%
\ExplSyntaxOff
%    \end{macrocode}
%    \begin{macrocode}
\cxset{%
    reset@documentation/.style={%
    index format=pgf,
    english language,
    documentation listing style = tcbdocumentation,
    index default settings,
    color option=Option,
    color definition=Definition,
    color hyperlink=Hyperlink,
    before example=\par\smallskip,
    after example=,
    doc left=0em,
    doc right=0pt,
    doc left indent=-2em,
    doc right indent=0pt,
    doc head=,
    doc description=,
    doc into index=true,
    index colorize = true,
    index annotate= false,
    },
%  initialize@reset=reset@documentation,
}
\cxset{reset@documentation}
%    \end{macrocode}
%
% We set the \docAuxKey{index format}=\docValue{pgf}  and the rest of the keys to the
% |german| settings that are suitable for |doc|.
%
%    \begin{macrocode}
\cxset{index format  =  pgfchapter,
       index actual={=},
       index level = {>},
       index quote = {!},
       index german settings,
       color hyperlink = thelinkcolor,  % links with color palette
       color definition =thelinkcolor,  % links with color palette
       pageshort       = {$\sigma{}$},
   }      
%\def\main#1{\underline{#1}}
%    \end{macrocode}
%

% \section{Miscellaneous doc commands}
% For consistency all commands that typeset their content, as well as index it and perhaps, also reference
% it have the prefix |doc|.
%
% \begin{docCommand} {docFile} { \marg{file name} }
%   Typesets and index a file. 
% \end{docCommand}
%    \begin{macrocode}
\ExplSyntaxOn
\cs_set:Npn \docFile #1
  {
    \texttt {#1}
    \index{files>#1}
  }
%    \end{macrocode}
% \begin{macro}{\docExtension}
%  Extension macro
%    \begin{macrocode}  
\cs_set:Npn \docExtension #1
  {
    \texttt {#1}
    \index{file extensions\idx@level#1}
  }  
\let\docextension\docExtension  
\ExplSyntaxOff  
%    \end{macrocode}
%\end{macro}
% 
%\section{Other Indexing functions}
%
% \begin{docCommand}{indexmany}{ \oarg{category} \marg{clist} }
% This function indexes a comma delimited list of items. It is convenient
% when you have paragraphs with a lot of terms.
% 
% \end{docCommand}
%    \begin{macrocode}
 \ExplSyntaxOn
 \DeclareDocumentCommand\indexmany {o m }
 {
   \clist_gset:Nn \indexmany: {#2} 
   \IfValueTF {#1}
    { 
      \clist_map_inline:Nn\indexmany: 
        {
          \index{#1\idx@level##1}\index{##1}
        }
    }
    { 
     \clist_map_inline:Nn\indexmany: 
      {
        \index{##1}
      } 
    }
 }
 \ExplSyntaxOff
%    \end{macrocode}

%    \begin{macrocode}
\DeclareRobustCommand{\idxlanguage}[1]{\index{\string #1 (script)}\index{scripts\idx@level#1}\texttt{#1}\xspace}%
%    \end{macrocode}
% 
% \section{Indexing Symbols}
%
% The commands that are defined in this section can be used to produce almost any symbol
% that is described in the Comprehensive Symbols\footcite{comprehensive}. They are mostly
% variants as described by this publication. Some modernized a bit, others using \latex3 type
% definitions. 
% 
% Some of them are just used for typesetting symbols in tables.
% 
% \begin{docCommand} {indexboth} { \marg{arg1}  \marg{arg2} }
%  Indexes both arguments. 
% \idxboth{currency}{symbols}
% \idxboth{monetary}{symbols}
%\begin{longsymtable}{\TC\ Currency Symbols}
%\idxboth{currency}{symbols}
%\idxboth{monetary}{symbols}
%\index{euro signs}
%\label{tc-currency}
%\begin{longtable}{*4{ll}}
%\K\textbaht          & \K\textdollar$^*$     & \K\textguarani  & \K\textwon \\
%\K\textcent          & \K\textdollar              & \K\textlira     & \K\textyen \\
%\K\textcent            & \K\textdong           & \K\textnaira    \\
%\K\textcolonmonetary & \K\texteuro           & \K\textpeso     \\
%\K\textcurrency      & \K\textflorin         & \K\textsterling$^*$ \\
%\end{longtable}
%\end{longsymtable}
% \end{docCommand}
%    \begin{macrocode} 
\newcommand{\idxboth}[2]{\mbox{}\index{#1 #2}\index{#2\idx@level#1}}
\newcommand{\idxbothbegin}[2]{\mbox{}\index{#1 #2|(}\index{#2\idx@level #1|(}}
\newcommand{\idxbothend}[2]{\mbox{}\index{#1 #2|)}\index{#2\idx@level #1|)}}
\ExplSyntaxOn
\cs_gset_eq:NN \indexboth\idxboth
\cs_gset_eq:NN \indexbothbegin \idxbothbegin
\cs_gset_eq:NN \indexbothend\idxbothend
\ExplSyntaxOff
%    \end{macrocode}
% 
%    
%
% We define a related macro for indexing accents.  In a previous version
% of this file, |\indexaccent| additionally included "see also accents" in
% the index.  This became distracting so I made |\indexaccent| a synonym
% for |\indexcommand| for the time being.  Because punctuation marks can
% be problematic for makeindex, we define an \indexpunct macro that
% sorts its argument under the comparatively innocuous "|\_|".
%
% \begin{docCommand}{sanitize}{}
%  Delimited macro (!!!) that sanitizes macros. Classic TeXBook style. It removes
%  the backslash. Not too sure if the word sanitize is appropriate. 
% \end{docCommand}
%    \begin{macrocode}
\def\cmd#1{\texttt{\string#1}\indexcommand{#1}}
\newcommand{\cmdI}[2][]{%
  \def\first@arg{#1}%
  \ifx\first@arg\@empty
    \texttt{\string#2}\indexcommand[#2]{#2}%
  \else
    \texttt{\string#2}\indexcommand[#1]{#2}%
  \fi
}
\newcommand{\cmdX}[1]{\cmdI[$\string#1$]{#1}}
\newcommand{\cmdW}[1]{\cmdI[$\string\blackacc{\string#1}$]{#1}}
\newcommand{\cmdIp}[2][]{%
  \def\first@arg{#1}%
  \ifx\first@arg\@empty
    \texttt{\string#2}\indexpunct[#2]{#2}%
  \else
    \texttt{\string#2}\indexpunct[#1]{#2}%
  \fi
}
\begingroup
 \catcode`\|=0
 \catcode`\\=12
 |gdef|sanitize#1#2!!!{%
   |ifx#1\%
     #2%
   |else
     #1#2%
   |fi
}
|endgroup
%    \end{macrocode}
%
%  \begin{docCommand}{indexcommand}{\oarg{}\marg{command} }
%    Index a \emph{symbol}, which may or may not begin with a \emph{backslash}.  (Is
%  there a better way to do this?)  Also, if symbol is given as an
%    optional argument is given, typeset that symbol in the index, as well
% \end{docCommand}
%
%  
%    \begin{macrocode}
 \ExplSyntaxOn
\NewDocumentCommand \indexcommand { o m }  
  {
    \edef\sanitized{\expandafter\sanitize\string#2!!!}%
    %\def\first@arg{#1}%
    \IfNoValueTF{#1}
    {
       \exp_after:wN
          \index {
             \sanitized\idx@actual
             \expandafter\phdIndexPrintCs{\sanitized}}%
    }
    {
       \expandafter\index\expandafter{\sanitized=\string\verb+\string#2+ (#1)}%
    }
  }
 \ExplSyntaxOff
%  \newcommand{\indexcommand}[2][]{%
%    \edef\sanitized{\expandafter\sanitize\string#2!!!}%
%    \def\first@arg{#1}%
%    \ifx\first@arg\@empty
%      \expandafter\index\expandafter{\sanitized=\string\verb+\string#2+}%
%    \else
%      \expandafter\index\expandafter{\sanitized=\string\verb+\string#2+ (#1)}%
%    \fi
%  }
%    \end{macrocode}
%
% \subsection{Indexing archaic scripts. }
%  
% \begin{docCommand} {indexcypriot} { \oarg{arg1} \marg{arg2} }
%    Index helper function for indexing Cypriot script. Only used
%    in the phd documentation.
% \end{docCommand}
%
%    \begin{macrocode}
\NewDocumentCommand \indexcypriot { o m }  
  {
    \edef\sanitized{\expandafter\sanitize\string#2!!!}%
    \IfNoValueTF{#1}
    {
       \expandafter\index\expandafter{Cypriot\idx@level\sanitized=\string\verb+\string#2+}%
    }
    {
       \expandafter\index\expandafter{Cypriot>\sanitized=\string\verb+\string#2+ (#1)}%
    }
  }
%    \end{macrocode}  
%
% \begin{docCommand} {indexstaves} { \oarg{arg1} \marg{arg2} }
%    Index helper function for indexing Icelandic staves. Only used
%    in the phd documentation.
% \end{docCommand}
%
%    \begin{macrocode}
\NewDocumentCommand \indexstaves { o m }  
  {
    \edef\sanitized{\expandafter\sanitize\string#2!!!}%
    \IfNoValueTF{#1}
    {
       \expandafter\index\expandafter{Staves>\sanitized=\string\verb+\string#2+}%
    }
    {
       \expandafter\index\expandafter{Staves>\sanitized=\string\verb+\string#2+ (#1)}%
    }
  }
%    \end{macrocode} 
% 
%  
% \begin{docCommand} {indexlinearb} { \oarg{typesetting command(s)} \marg{command} }
%    Index helper function for indexing the linearb script. Only used
%    in the phd documentation for scripts. 
%    In .idx it writes
%    \begin{verbatim}
%    \indexentry{Linear B >BPamphora=\phdIndexPrint {BPamphora} (\textlinb  {\BPamphora })|hyperpage}{63}
%    \end{verbatim}
% \end{docCommand}
%  
%    \begin{macrocode}

\newrobustcmd\phdIndexPrintCs[1]{{\catcode`\_=12\relax\catcode`\@11\relax\ttfamily\char`\\\scantokens{#1}\unskip}}
% 
% 
\NewDocumentCommand \indexlinearb { o m }  
  {
    \edef\sanitized{\expandafter\sanitize\string#2!!!}%
    \IfNoValueTF{#1}
    {
       \expandafter\index\expandafter{Linear B\idx@level \sanitized\idx@actual\string\verb+\string#2+}%
    }
    { % index
         \index{Linear B
         \idx@level
        \sanitized
         \idx@actual 
         \expandafter\phdIndexPrintCs{\sanitized} (#1)}%
       % typeset        
       \string#2 & (#1)
    }
  }
%    \end{macrocode} 
% 
%\index{Linear B}
%\index{arrows}
%\index{animals}
%\label{linearB-objs}
%\begin{longtable}{*3{ll@{\qquad}}ll}
%\indexlinearb[\textlinb{\BPamphora}]\BPamphora       & \indexlinearb[\textlinb{\BPchassis}]\BPchassis       & \indexlinearb[\textlinb{\BPman}]\BPman               & \indexlinearb[\textlinb{\BPwheat}]\BPwheat           \\
%\indexlinearb[\textlinb{\BParrow}]\BParrow           & \indexlinearb[\textlinb{\BPcloth}]\BPcloth           & \indexlinearb[\textlinb{\BPnanny}]\BPnanny           & \indexlinearb[\textlinb{\BPwheel}]\BPwheel           \\
%\indexlinearb[\textlinb{\BPbarley}]\BPbarley         & \indexlinearb[\textlinb{\BPcow}]\BPcow               & \indexlinearb[\textlinb{\BPolive}]\BPolive           & \indexlinearb[\textlinb{\BPwine}]\BPwine             \\
%\indexlinearb[\textlinb{\BPbilly}]\BPbilly           & \indexlinearb[\textlinb{\BPcup}]\BPcup               & \indexlinearb[\textlinb{\BPox}]\BPox                 & \indexlinearb[\textlinb{\BPwineiih}]\BPwineiih       \\
%\indexlinearb[\textlinb{\BPboar}]\BPboar             & \indexlinearb[\textlinb{\BPewe}]\BPewe               & \indexlinearb[\textlinb{\BPpig}]\BPpig               & \indexlinearb[\textlinb{\BPwineiiih}]\BPwineiiih     \\
%\indexlinearb[\textlinb{\BPbronze}]\BPbronze         & \indexlinearb[\textlinb{\BPfoal}]\BPfoal             & \indexlinearb[\textlinb{\BPram}]\BPram               & \indexlinearb[\textlinb{\BPwineivh}]\BPwineivh       \\
%\indexlinearb[\textlinb{\BPbull}]\BPbull             & \indexlinearb[\textlinb{\BPgoat}]\BPgoat             & \indexlinearb[\textlinb{\BPsheep}]\BPsheep           & \indexlinearb[\textlinb{\BPwoman}]\BPwoman           \\
%\indexlinearb[\textlinb{\BPcauldroni}]\BPcauldroni   & \indexlinearb[\textlinb{\BPgoblet}]\BPgoblet         & \indexlinearb[\textlinb{\BPsow}]\BPsow               & \indexlinearb[\textlinb{\BPwool}]\BPwool             \\
%\indexlinearb[\textlinb{\BPcauldronii}]\BPcauldronii & \indexlinearb[\textlinb{\BPgold}]\BPgold             & \indexlinearb[\textlinb{\BPspear}]\BPspear           &                                           \\
%\indexlinearb[\textlinb{\BPchariot}]\BPchariot       & \indexlinearb[\textlinb{\BPhorse}]\BPhorse           & \indexlinearb[\textlinb{\BPsword}]\BPsword           &                                           \\
%\end{longtable}
%
%
% \begin{docCommand} {indexugar} { \oarg{arg1} \marg{arg2} }
%    Index helper function for indexing Ugaritic scripts. Only used
%    in the phd documentation.
% \end{docCommand}
% 
%    \begin{macrocode}
\NewDocumentCommand \indexugar { o m }  
  {
    \edef\sanitized{\expandafter\sanitize\string#2!!!}%
    \IfNoValueTF{#1}
    {
       \expandafter\index\expandafter{Ugarite>\sanitized=\string\verb+\string#2+}%
    }
    {
       \expandafter\index\expandafter{Ugarite>\sanitized=\string\verb+\string#2+ (#1)}%
    }
  }
%    \end{macrocode} 

% \begin{docCommand} {indexvarpersian} { \oarg{arg1} \marg{arg2} }
%   Indexing and doc command for var Persian tables.
% \end{docCommand}
%
%    \begin{macrocode}
\NewDocumentCommand \indexvarpersian { o m }  
  {
    \edef\sanitized{\expandafter\sanitize\string#2!!!}%
    \IfNoValueTF{#1}
    {
       \expandafter\index\expandafter{var Persian>\sanitized=\string\verb+\string#2+}%
    }
    {
       \expandafter\index\expandafter{var Persian>\sanitized=\string\verb+\string#2+ (#1)}%
    }
  }
%    \end{macrocode} 
%
% \begin{docCommand} {indexsoutharabian} {\oarg{}\marg{} }
%    Indexing and doc command for symbols tables.
% \end{docCommand}
%
%    \begin{macrocode}
\NewDocumentCommand \indexsoutharabian { o m }  
  {
    \edef\sanitized{\expandafter\sanitize\string#2!!!}%
    \IfNoValueTF{#1}
    {
       \expandafter\index\expandafter{South Arabian>\sanitized=\string\verb+\string#2+}%
    }
    {
       \expandafter\index\expandafter{South Arabian>\sanitized=\string\verb+\string#2+ (#1)}%
    }
  }
%    \end{macrocode} 
%
% \subsection{Indexing mathematical symbols}
%
% The currently available fonts 
% The following indexing commands are auxiliary commands to
% index unicode symbols for maths. 
% \tcbdocmarginnote{26-06-2015}
%    \begin{macrocode}
\NewDocumentCommand \indexmathcmd { o m }  
  {
    \edef\sanitized{\expandafter\sanitize\string#2!!!}%
    \IfNoValueTF{#1}
    {
       \expandafter\index\expandafter{#1\idx@level\sanitized=\string\verb+\string#2+
       ($#2$)}
       % put command also
      \expandafter\index\expandafter{\string#1=\string\verb+\string#2+ ($\string#2$)*}%
    }
    {
      \expandafter\index\expandafter{#1\idx@level
      \sanitized=\string\verb+\string#2+ ($#2$)}%
      \expandafter\index\expandafter{\string#1=\string\verb+\string#2+ ($\string#2$)}%
    }
  }
%    \end{macrocode} 
%
% \begin{docCommand}{indexaccent}{}
%   Syntactic sugar identical to \refCom{indexcommand}
% \end{docCommand}
%
%    \begin{macrocode}
\ExplSyntaxOn
\cs_gset_eq:NN \indexaccent\indexcommand
\cs_new:Npn \CLSLpipe {|}
\ExplSyntaxOff  
%    \end{macrocode}
%   
%
% \begin{docCommand} {indexpunct} { \oarg {arg1} \marg{arg2}} 
%   Indexing punctuation marks for latin scripts.
% \end{docCommand}
%
%    \begin{macrocode}
  \newcommand{\indexpunct}[2][]{%
    \def\first@arg{#1}%
    \def\second@arg{#2}%
    \ifx\first@arg\@empty
      \ifx\second@arg\CLSLpipe
        \index{_=\magicvertname}%
      \else
        \index{_=\string\verb+\string#2+}%
      \fi
    \else
      \ifx\second@arg\CLSLpipe
        \index{_=\magicvertname{} (#1)}%
      \else
        \index{_=\string\verb+\string#2+ (#1)}%
      \fi
    \fi
  }
%    \end{macrocode}
% 
%
%    \begin{macrocode}
%\usepackage{longdiv}
\newcommand\FC{\pkgname{fc}}
\newcommand\VIET{\pkgname{vietnam}}
%\newcommand\ABX{\pkgname{mathabx}}
%    \end{macrocode}
%

% \begin{docCommand} {incsyms} { \meta{void}}
%  We define an integer counter \docCounter{totalsymbols} to keep track of all the symbols we load
%  and list.\footnote{Unicode characters are counted separately and are dealt with under the phd-scriptsmanager package.}
%  These are symbols which can be produced using command sequences.
% \end{docCommand}
%
% Define a counter to keep track of how many symbols are listed.
% Output this counter to the log file at the end of each run.
% Define |\prevtotalsymbols| to be the total number of symbols from
% the previous run.
%   
%    \begin{macrocode}
\ExplSyntaxOn
  \int_new:c {totalsymbols}
  \cs_new:Npn \incsyms { \int_gincr:c {totalsymbols} }
  \cs_new:Npn \thetotalsymbols {\int_use:c {totalsymbols} }
\ExplSyntaxOff
%    \end{macrocode}
%
% \begin{docCommand}{graybox} { \meta{void}}
% \end{docCommand}
%    \begin{macrocode}
\newcommand*{\graybox}{\textcolor{thecodebackground}{\rule[-\adp]{\awd}{\aht}}}
 
% Define |\blackacc| to display an accented box, given an accent command.
% Define |\blackacchack| to display an accented "a" and then black out
% the "a".
\newlength\awd
\newlength\aht
\newlength\adp
\settowidth{\awd}{\normalfont a}
\settoheight{\aht}{\normalfont a}
\settodepth{\adp}{\normalfont a}
\advance\adp by 0.06pt    % In Computer Modern, "a" extends slightly below its bounding box.
\advance\aht by \adp


\gdef\blackacchack#1{#1a\llap{\graybox}}
\gdef\blackacc#1{#1{\graybox}}
\gdef\blackacctwo#1{#1{\graybox}{\graybox}}
%    \end{macrocode}
% 
% 
%
% Symbol+verbatim for various types of symbols
%    \begin{macrocode}
\def\E#1{%
  \begingroup
    \lccode`|=`\\
    \def\EStruename{ES#1T}
    \lowercase{\incsyms\index{#1=\string\verb+\string|#1+ (\string|\EStruename)}}
  \endgroup
  \csname ES#1T\endcsname 
  & \csname ES#1D\endcsname 
  &
  \ttfamily\expandafter\string\csname#1\endcsname
}
%    \end{macrocode}
%    
%  
% 
% These commands are here to be able to index these symbols for the index and to typeset
% them in the symbols appendix.
% 
% \begin{docCommand} {Kcyp} {\oarg{text cmd} \marg{symbol command}}
%   Indexes and prints the Cypriot archaic font symbols.
% 
% \example |\Kcyp[\textcypr{\Ca}]\Ca|
%
% 
% \end{docCommand}
%    \begin{macrocode}
\def\Kcyp@opt@arg[#1]#2{\incsyms\indexcypriot[\textcypr{#1}]{#2}#1 &\ttfamily\string#2}
\def\Kcyp@no@opt@arg#1{\incsyms\indexcypriot[\textcypr{#1}]{#1}#1 &\ttfamily\string#1}
\def\Kcyp{\@ifnextchar[{\Kcyp@opt@arg}{\Kcyp@no@opt@arg}}
%    \end{macrocode}
%    
% \begin{docCommand} {Kstav} { \oarg{cmd} \marg{stave cmd}}      
%   Indexes and prints an Icelandic  stave. 
% \end{docCommand}
%    \begin{macrocode}
\ExplSyntaxOn

\cs_set:Npn \Kstav_opt_arg [#1]#2
  {
    \incsyms\indexstaves[#1]{#2}# 1 &\ttfamily\string#2
  }
 
\cs_set:Npn \Kstav_no_opt_arg #1
  {
    \incsyms\indexstaves[#1]{#1}#1 &\ttfamily\string#1
  }

\NewDocumentCommand\Kstav {o m} {
  \IfNoValueTF {#1} 
    {
      \Kstav_no_opt_arg {#2}
    }
    {
      \Kstav_opt_arg [#1] {#2}
    }
}
\ExplSyntaxOff
%    \end{macrocode}
%    
% \begin{docCommand}{K} { \oarg{} \marg{cmd} }    
%    Adds a symbol cmd to a table and the index.
% \end{docCommand}
%
%    \begin{macrocode}
\ExplSyntaxOn
\cs_set:Npn \K@opt@arg#1#2 
   {
      \incsyms
      \indexcommand[#1]{#2}#1 &\footnotesize\ttfamily\string#2
   }
   
\cs_set:Npn \K@no@opt@arg#1
  {
    \incsyms\indexcommand[#1]{#1}#1 &\footnotesize\ttfamily\string#1
  }
%    \end{macrocode}
%
% \begin{docCommand}{K}{\oarg{symbol command} \marg{symbol command}} 
%     Command used to typeset a table of symbols such as textsymbols. The mandatory argument
%     takes a command such as |texteuro|, typesets the symbol first and then the command. It also adds it to the 
%    index. It is the most common command and hence we keep it short. It was originally defined in Comprehensive \footcite{comprehensive}.
%  It also increases the counter \docCounter{incsyms}. The optional argument has an alternative symbol.
%     |\K[\dingSquare]\Square| 
% \end{docCommand} 
%    \begin{macrocode}
\NewDocumentCommand {\K} { o m } 
{
  \IfNoValueTF {#1} { \K@no@opt@arg {#2} } {\K@opt@arg {#1}{#2}}
}
\ExplSyntaxOff
%    \end{macrocode}
%
%    \begin{symtable}{\latexe{} Escapable ``Special'' Characters}
%\index{special characters=``special'' characters}
%\index{escapable characters}
%\index{underline}
%\label{special-escapable}
%\begin{tabular}{*6{ll@{\qqquad}}ll}
%\K\$   & \K\%   & \K\_$\,^*$  & \Kp\}  & \K\&   & \K\#   & \Kp\{   \\
%\end{tabular}
%\end{symtable}
%
%    \begin{macrocode}
\ExplSyntaxOn
\cs_set:Npn \Kp #1 
   {
       \incsyms
       \indexpunct[$#1$]{#1}#1 &\footnotesize\ttfamily\string#1
    }
\ExplSyntaxOff
%    \end{macrocode}
%
%    \begin{macrocode}
%\begin{symtable}[EDICE]{\EDICE\ Dice}
%\index{dice}
%\idxboth{game-related}{symbols}
%\label{edice}
%\begin{tabular}{ll@{\qquad}ll@{\qquad}ll}
%  \KED[\allepsdice][\epsdice][\epsdice{1}]\epsdice\verb|{1}| &
%  \KED[\allepsdice][\epsdice][\epsdice{3}]\epsdice\verb|{3}| &
%  \KED[\allepsdice][\epsdice][\epsdice{5}]\epsdice\verb|{5}| \\
%  \KED[\allepsdice][\epsdice][\epsdice{2}]\epsdice\verb|{2}| &
%  \KED[\allepsdice][\epsdice][\epsdice{4}]\epsdice\verb|{4}| &
%  \KED[\allepsdice][\epsdice][\epsdice{6}]\epsdice\verb|{6}| \\
%\end{tabular}
%\end{symtable}
%\newif\ifEDICE
%\newcommand\EDICE{\pkgname{epsdice}}
%\IfFileExists{epsdice.sty}
%  {\EDICEtrue
%   \usepackage{epsdice}
%   \let\origepsdice=\epsdice
%   \DeclareRobustCommand{\epsdice}[1]{\origepsdice{##1}}
%   \DeclareRobustCommand{\allepsdice}{%
%     \epsdice{1}%
%     \epsdice{2}%
%     \epsdice{3}%
%     \epsdice{4}%
%     \epsdice{5}%
%     \epsdice{6}%
%   }
%  }
%  {}
%    \end{macrocode}  
%\begin{symtable}[DOZ]{\DOZ\ Base-12 Digits}
%\index{numerals}
%\index{dozenal (base 12)>numerals}
%\index{base twelve>numerals}
%\index{duodecimal (base 12)>numerals}
%\idxboth{Pitman's base 12}{symbols}
%\label{dozenal-digits}
%\begin{tabular}{ll@{\qquad}ll}
%\K[\DOZx]\x & \K[\DOZe]\e \\
%\end{tabular}
%\end{symtable}
%
% The below are faked symbols, based on Scott Pakin's |fakedozenal.sty|. We include this file
% with the |phd| bundle distribution. With LuaLaTeX this can be included. This is a rather
% strange package with probably very few users and use. It has intrigued me and I include it
% here.\footnote{See the website of the dozenal society to understand what it is all about.
% \protect\url{http://www.dozenal.org/index.html}}
%\begin{symtable}[DOZ]{\DOZ\ Tally Markers}
%\index{dozenal (base 12)>tally markers}
%\index{base twelve>tally markers}
%\index{duodecimal (base 12)>tally markers}
%\idxboth{Pitman's base 12}{symbols}
%\index{tally markers}
%\label{dozenal-tally}
%\begin{tabular}{ll@{\qquad}ll@{\qquad}ll}
%\KED[\alldoztallies][\tally][\doz{1}]\tally\verb|{1}| & \KED[\alldoztallies][\tally][\doz{3}]\tally\verb|{3}| & \KED[\alldoztallies][\tally][\doz{5}]\tally\verb|{5}| \\
%\KED[\alldoztallies][\tally][\doz{2}]\tally\verb|{2}| & \KED[\alldoztallies][\tally][\doz{4}]\tally\verb|{4}| & \KED[\alldoztallies][\tally][\doz{6}]\tally\verb|{6}| \\
%\end{tabular}
%\end{symtable}

%    \begin{macrocode}
% Load a faked version of a package.
\newcommand{\fakeusepackage}[1]{%
 \let\origProvidesPackage=\ProvidesPackage
 \def\ProvidesPackage##1[##2]{\origProvidesPackage{##1}[##2]\endinput}
 \usepackage{#1}
 \let\ProvidesPackage=\origProvidesPackage
 \usepackage{fake#1}
}

\newif\ifDOZ
\newcommand\DOZ{\pkgname{dozenal}}
\IfFileExists{dozenal.sty}
  {\DOZtrue
   \if\luatex
      \RequirePackage[nocounters,typeone]{dozenal}
    \else  
     \fakeusepackage{dozenal}
   \fi  
   \DeclareRobustCommand{\DOZx}{\doz{X}}
   \DeclareRobustCommand{\DOZe}{\doz{E}}
   \DeclareRobustCommand{\alldoztallies}{%
     \doz{1}~%
     \doz{2}~%
     \doz{3}~%
     \doz{4}~%
     \doz{5}~%
     \doz{6}%
   }
  }
{}  
%    \end{macrocode}
% \begin{docCommand}{KED}{\oarg{}\oarg{2}\oarg{3}\oarg{4}}
%  Documentation macros for tables showing dozenal symbols. Rarely used.
% \end{docCommand}
%    \begin{macrocode}
\def\KED[#1][#2][#3]#4 {%
   \incsyms\indexcommand[#1]{#2}#3 &\small\ttfamily\string#4%
}

% feyn provides yet another math font for which we have no room.
% Fortunately, it's relatively easy to define all of its symbols in
% terms of a text font.


\def\Kfeyn#1{\incsyms\indexcommand[\string\feyn{#1}]{\feyn{#1}}\feyn{#1} &\small\ttfamily\string\feyn\string{\string#1\string}}
%    \end{macrocode}

%
%    \begin{macrocode}
% We have no math alphabets left so we trick svrsymbols.sty into defining
% all of its characters in text mode.
\newif\ifSVR
\newcommand\SVR{\pkgname{svrsymbols}}
\makeatletter
\IfFileExists{svrsymbols.sty}
  {\SVRtrue
   \let\origDeclareSymbolFont=\DeclareSymbolFont
   \let\origDeclareMathSymbol=\DeclareMathSymbol
   \let\origSetSymbolFont=\SetSymbolFont
   \renewcommand{\DeclareSymbolFont}[5]{}
   \renewcommand{\SetSymbolFont}[6]{}
   \renewcommand{\DeclareMathSymbol}[4]{%
     \let##1=\relax%  \photon and \antiproton are defined repeatedly.
     \newcommand{##1}{{\usefont{OML}{svr}{m}{it}\char##4}}%
   }
   \usepackage{svrsymbols}
   \let\DeclareSymbolFont=\origDeclareSymbolFont
   \let\DeclareMathSymbol=\origDeclareMathSymbol
   \let\SetSymbolFont=\origSetSymbolFont
  }
  {}
%   
%%
%  \newenvironment{longsymtable}[2][true]{%
%  \expandafter\global\expandafter\let
%    \expandafter\ifshowsymtable\csname if#1\endcsname
%  \ifshowsymtable
%    \mbox{}%
%    \Needspace*{13\baselineskip}%
%    \mbox{}%
%    \begin{center}%
%    \phantomsection
%    \refstepcounter{table}%
%%
%    % Inhibit longtable's implicit increment of the table counter.
%    \let\refstepcounter=\@gobble
%    \let\LT@array=\origLT@array
%    \let\LT@start=\origLT@start
%%
%    \addcontentsline{toc}{subsection}{%
%      \protect\numberline{\tablename~\thetable:}{#2}}%
%    \@makecaption{\fnum@table}{#2}%
%    \gdef\lt@indexed{}%
%    \let\next=\relax
%  \else
%    % The following was taken verbatim from verbatim.sty.
%    \let\do\@makeother\dospecials\catcode`\^^M\active
%    \let\verbatim@startline\relax
%    \let\verbatim@addtoline\@gobble
%    \let\verbatim@processline\relax
%    \let\verbatim@finish\relax
%    \let\next=\verbatim@
%  \fi
%  \next
%}{%
%  \ifshowsymtable
%    \end{center}
%    \let\@elt=\index\lt@indexed  % Close our index ranges.
%    \gdef\lt@indexed{}%
%    \vskip 8ex minus 2ex
%  \fi
%}
%    \end{macrocode}
%\begin{longsymtable}[SVR]{\SVR\ Physics Ideograms}
%\ltindex{particle-physics symbols}
%\ltindex{symbols>particle physics}
%\ltindex{fermions}
%\ltindex{subatomic particles}
%\ltindex{photons}
%\label{svrsymbols}
%\begin{longtable}{*3{ll}}
%\multicolumn{6}{l}{\small\textit{(continued from previous page)}} \\[3ex]
%\endhead
%\endfirsthead
%\\[3ex]
%\multicolumn{6}{r}{\small\textit{(continued on next page)}}
%\endfoot
%\endlastfoot
%\K\adsorbate      & \K\experimentalsym & \K\protein        \\
%\K\adsorbent      & \K\externalsym     & \K\proton         \\
%\K\antimuon       & \K\fermiDistrib    & \K\quadrupole     \\
%\K\antineutrino   & \K\fermion         & \K\quark          \\
%\K\antineutron    & \K\Gluon           & \K\quarkb         \\
%\K\antiproton     & \K\graphene        & \K\quarkc         \\
%\K\antiquark      & \K\graviton        & \K\quarkd         \\
%\K\antiquarkb     & \K\hbond           & \K\quarks         \\
%\K\antiquarkc     & \K\Higgsboson      & \K\quarkt         \\
%\K\antiquarkd     & \K\hole            & \K\quarku         \\
%\K\antiquarks     & \K\interaction     & \K\reference      \\
%\K\antiquarkt     & \K\internalsym     & \K\resistivity    \\
%\K\antiquarku     & \K\ion             & \K\rhomesonminus  \\
%\K\anyon          & \K\ionicbond       & \K\rhomesonnull   \\
%\K\assumption     & \K\Jpsimeson       & \K\rhomesonplus   \\
%\K\atom           & \K\Kaonminus       & \K\solid          \\
%\K\bigassumption  & \K\Kaonnull        & \K\spin           \\
%\K\Bigassumption  & \K\Kaonplus        & \K\spindown       \\
%\K\biggassumption & \K\magnon          & \K\spinup         \\
%\K\Bmesonminus    & \K\maxwellDistrib  & \K\surface        \\
%\K\Bmesonnull     & \K\metalbond       & \K\svrexample     \\
%\K\Bmesonplus     & \K\method          & \K\svrphoton      \\
%\K\bond           & \K\muon            & \K\tachyon        \\
%\K\boseDistrib    & \K\neutrino        & \K\tauleptonminus \\
%\K\boson          & \K\neutron         & \K\tauleptonplus  \\
%\K\conductivity   & \K\nucleus         & \K\Tmesonminus    \\
%\K\covbond        & \K\orbit           & \K\Tmesonnull     \\
%\K\dipole         & \K\phimeson        & \K\Tmesonplus     \\
%\K\Dmesonminus    & \K\phimesonnull    & \K\triplecovbond  \\
%\K\Dmesonnull     & \K\phonon          & \K\Upsilonmeson   \\
%\K\Dmesonplus     & \K\pionminus       & \K\varphoton      \\
%\K\doublecovbond  & \K\pionnull        & \K\water          \\
%\K\electron       & \K\pionplus        & \K\Wboson         \\
%\K\errorsym       & \K\plasmon         & \K\Wbosonminus    \\
%\K\etameson       & \K\polariton       & \K\Wbosonplus     \\
%\K\etamesonprime  & \K\polaron         & \K\Zboson         \\
%\K\exciton        & \K\positron        &                   \\
%\end{longtable}
%\end{longsymtable}
%%    \begin{macrocode}  
%\def\Kpig#1{\incsyms\index{pigpenfont #1=\string\verb+{\string\pigpenfont\space#1}+\space(\string\CLSLpig{#1})}\CLSLpig{#1} &\ttfamily\string{\string\pigpenfont\space\string#1\string}}
%    
%
%  \begin{docCommand} {Ks} { \marg{cmd} }
%    \cs{Ks} index and doc command, asterisk for note that is not available in |OT1|, as
%    superscript.
%  \end{docCommand}
%
%  \subsection{Arrows}
%%
% We use a faked version of old-arrows.sty provide by Pakin so as not to waste a math alphabet, if not
% using LuaLaTeX. Under the latest versions of Lua and LaTeX we load it directly.
%    \begin{macrocode}
\newif\ifARR
\newcommand\ARR{\pkgname{old-arrows}}
\IfFileExists{old-arrows.sty}
  {\ARRtrue\usepackage[old]{old-arrows}}
  {}

%  
%
%%\index{arrows}
%%\label{var-arrows}
%%\begin{tabular}{*3{ll}}
%%\X\vardownarrow\downarrow                   &  \X\varlongleftrightarrow \longleftrightarrow &  \X\varnwarrow \nwarrow         \\
%% \X\varhookleftarrow \hookleftarrow           &  \X\varlongmapsfrom \longmapsfrom$^*$         &  \X\varrightarrow \rightarrow   \\
%% \X\varhookrightarrow \hookrightarrow         &  \X\varlongmapsto \longmapsto                 &  \X\varsearrow \searrow         \\
%% \X\varleftarrow \leftarrow                   &  \X\varlongrightarrow \longrightarrow         &  \X\varswarrow \swarrow         \\
%% \X\varleftrightarrow \leftrightarrow         &  \X\varmapsfrom \mapsfrom$^*$                 &  \X\varuparrow \uparrow         \\
%% \X\varlonghookrightarrow \longhookrightarrow &  \X\varmapsto \mapsto                         &  \X\varupdownarrow \updownarrow \\
%% \X\varlongleftarrow \longleftarrow           &  \X\varnearrow \nearrow                       &                                 \\
%%\end{tabular}

%    \end{macrocode}
%    \begin{macrocode}
\ExplSyntaxOn
  \cs_set:Npn \Ks #1
    {
      \incsyms
      \indexcommand[\string\encone{\string#1}] {#1}
      { \encone{#1} }  & \ttfamily\string#1$^*$
    }
\ExplSyntaxOff
%    \end{macrocode}
%
% This macro is also from the comprehensive and takes
% the symbol command as its only argument. It provides
% |T1| encoding and also adds the command to the index.
% 
%    \begin{macrocode}
\ExplSyntaxOn   
\cs_set:Npn \Kt #1
  {
    \incsyms
    \indexcommand[\string\encone{\string#1}] {#1}
    {\encone{#1}} & \ttfamily \string #1
  }
\ExplSyntaxOff  
%    \end{macrocode}
%
%  \begin{docCommand} {Kv} { \marg{cmd} }
%    T5 encoding
%  \end{docCommand}
%
%    \begin{macrocode}
\def\Kv#1{\incsyms\indexcommand[\string\encfive{\string#1}]{#1}{\encfive{#1}} &\ttfamily\string#1}
%    \end{macrocode}
%
%    \begin{macrocode}
\def\Kgr@opt@arg[#1]#2{\incsyms\indexcommand[\string\encgreek{\string#1}]{#2}{\encgreek{#1}} &\ttfamily\string#2}
  \def\Kgr@no@opt@arg#1{\incsyms\indexcommand[\string\encgreek{\string#1}]{#1}{\encgreek{#1}} &\ttfamily\string#1}
 
\def\Kgr{\@ifnextchar[{\Kgr@opt@arg}{\Kgr@no@opt@arg}}
\def\KN[#1][#2]#3{\incsyms\indexcommand[\string#1]{#3} #1 & #2 & \ttfamily\string#3}
\def\KNbig[#1][#2]#3{\incsyms\indexcommand[\string#2]{#3} #1 & #2 & \ttfamily\string#3}

\def\Knoidx#1{\incsyms#1 &\ttfamily\string#1}
%    \end{macrocode}
%
% \begin{docCommand}{N} {}
%   Big delimiters auxiliary command for doc and index. 
% \end{docCommand}
%
%    \begin{macrocode}
\ExplSyntaxOn
\cs_set:Npn \N@opt@arg #1 #2 
  {
    \incsyms
    \indexcommand[$\string#1$]{#2}
    $#1$ & $\Big#1$ &\footnotesize\ttfamily\string#2
  }

\cs_set:Npn \N@no@opt@arg#1 
  {
    \incsyms\indexcommand[$\string#1$]{#1}
    $#1$ & $\Big#1$ &\ttfamily\string#1
  }

\NewDocumentCommand {\N} { o m } 
  {
    \IfNoValueTF {#1} 
      { \N@no@opt@arg {#2}  }
      { \N@opt@arg {#1}{#2} }
  }

\ExplSyntaxOff  
%    \end{macrocode}
% \begin{docCommand}{Nn}{\oarg{}\marg{}}  
% \end{docCommand}
%    \begin{macrocode}  
  \def\Nn[#1]#2{%
    \incsyms\indexcommand[$\string\nathdouble\string#1$]{#2}%
    $\nathdouble#1$ & $\nathdouble{\Big#1}$ & \ttfamily\string#2%
 }
    
  \def\Nnt#1[#2]#3{%
    \incsyms\indexcommand{\triple}%
    $\nathtriple#2$ & $\nathtriple{\Big#2}$ &
    \ttfamily\expandafter\string\csname#1triple\endcsname\string#3}
  \def\Np@opt@args[#1]{\@ifnextchar[{\Np@two@opt@args[#1]}{\Np@one@opt@arg[#1]}}
  \def\Np@two@opt@args[#1][#2]#3{\incsyms\index{_=\string#2{} ($\string#1$)}$#1$ & $\Big#1$ &\ttfamily\string#3}
  \def\Np@one@opt@arg[#1]#2{\incsyms\indexpunct[$\string#1$]{#2}$#1$ & $\Big#1$ &\ttfamily\string#2}
  \def\Np@no@opt@args#1{\incsyms\indexpunct[$\string#1$]{#1}$#1$ & $\Big#1$ &\ttfamily\string#1}
  \def\Np{\@ifnextchar[{\Np@opt@args}{\Np@no@opt@args}}
  \def\Nbig[#1]#2{\incsyms\indexcommand[$\string\Big\string#1$]{#2}$#1$ & $\Big#1$ &\ttfamily\string#2}
%    \end{macrocode}
%
%  \begin{docCommand} {Q} {}
%    Used to typeset accents.
% \index{accents}
% \index{accents>acute=acute (\blackacchack\')}   
% \index{accents>arc=arc (\blackacchack\newtie)}
% \index{accents>breve=breve (\blackacchack\u)}   
% \index{accents>caron=caron (\blackacchack\v)}
%  \end{docCommand}
% \begin{macro}{\Q@opt@arg}
%    \begin{macrocode}  
\ExplSyntaxOn
\cs_set:Npn \Q@opt@arg#1#2
  {
    \incsyms\indexaccent[\string\blackacchack{\string#1}]{#2}#1{A}#1{a} &
           \ttfamily\string#2\string{A\string}\string#2\string{a\string}
  }
  
\cs_set:Npn \Q@no@opt@arg#1
  {
    \incsyms\indexaccent[\protect\blackacchack{\string#1}]{#1}#1{A}#1{a} &
    \ttfamily\string#1\string{A\string}\string#1\string{a\string}
  }
           
\NewDocumentCommand {\Q} { o m }
  {
    \IfNoValueTF {#1}
      { \Q@no@opt@arg {#2} }
      { \Q@opt@arg    {#1}{#2} }
  }
\ExplSyntaxOff  
%    \end{macrocode}
% \end{macro}
%    \begin{macrocode}
\newif\ifHARM
\newcommand\HARM{\pkgname{harmony}}
\IfFileExists{harmony.sty}
  {\HARMtrue
   \let\orignewcommand=\newcommand
   \let\newcommand=\DeclareRobustCommand
   %\savesymbol{HH}
   \usepackage{harmony}
   %\restoresymbol{harm}{HH}
   \let\newcommand=\orignewcommand
  }
  {}
%    \end{macrocode}
%\begin{symtable}[HARM]{\HARM\ Musical Accents}
%\idxboth{musical}{symbols}
%\index{accents}
%\label{harmony-accents}
%\begin{tabular}{ll@{\qqquad}ll}
%\Q\Ferli$^*$  & \Q\Ohne$^*$  \\
%\Q\Fermi        & \Q\Umd$^*$   \\
%\Qc\Kr           &              \\
%\end{tabular}
% \end{symtable}
%
% \begin{docCommand}{Qc}{\marg{accent symbol command}}
%   Typesets and indexes accents on two letters (A) and (a), as two tabular cells.
%   Increases the \docCounter{incsyms} counter.
% \end{docCommand}
%
%    \begin{macrocode}
\ExplSyntaxOn
\cs_set:Npn \Qc#1
  {
    \incsyms
    \indexaccent[\string\blackacc{\string#1}]{#1} #1 {A} #1{a}  &
    \ttfamily
         \string#1
         \token_to_str:N {
            A
         \token_to_str:N }
         \token_to_str:N #1
         \token_to_str:N {
           a
         \token_to_str:N }
  }
\ExplSyntaxOff
%    \end{macrocode}
% \magicequal  and \magicequalname
% \begin{docCommand}{Qe}{\meta{arg1}\meta{arg2}\meta{arg3}}
%    Limited use command to typeset and index |magic symbols|, that is symbols
%    that might confuse MakeIndex and that they are escaped. It is easier to use
%   |text commands|.
%   \example |\Qe[\magicequal][\magicequalname]\=| produces 
%      \begin{tabular}{ll}
%      \Qe[\magicequal][\magicequalname]\=\\
%      \end{tabular}
% 
% \end{docCommand}
%    \begin{macrocode}         
\def\Qe[#1][#2]#3{%
  \incsyms
  \incsyms
  \index{_=\string#2{} (\string\blackacchack{\string#1})}%
  #3{A}#3{a} &
  \ttfamily\string#3\string{A\string}\string#3\string{a\string}}
%    \end{macrocode}
%
%    \begin{macrocode}  
\def\Qt#1{\incsyms\indexaccent[\string\encone{\string\blackacc{\string#1}}]{#1}{\encone{#1{A}#1{a}}} &
          \ttfamily\string#1\string{A\string}\string#1\string{a\string}}
%    \end{macrocode}
%    \begin{macrocode}
\def\Qpc#1#2{\incsyms\indexcommand{#2}{\raisebox{1pt}{\tiny[#1]}} &
             \ttfamily\string#2\string{A\string}\string#2\string{a\string}}
%    \end{macrocode}
%
%    \begin{macrocode}             
\def\Qpfc[#1]#2{\incsyms\indexaccent[\string\encfour{\string\blackacchack{\string#1}}]{#2}\encfour{#1{A}#1{a}} &
           \ttfamily\string#2\string{A\string}\string#2\string{a\string}}
%    \end{macrocode}
%    \begin{macrocode}
\newif\ifFC\FCfalse
\ifFC
  \def\Qiv#1#2{\incsyms\indexaccent[\string\encfour{\string\blackacchack{\string#1}}]{#1}\encfour{#1{A}#1{a}} &
               \ttfamily\string#1\string{A\string}\string#1\string{a\string}$^#2$}
               
  \def\QivBAR#1{\incsyms\index{_=\string\magicVertname{}
                (\string\encfour{\string\blackacchack{\string\FCbar}})}
                \encfour{\FCbar{A}\FCbar{a}} &
                \ttfamily\string\|\string{A\string}\string\|\string{a\string}$^#1$}
\else
  \def\Qiv#1#2{\Qpc{T4}{#1}$^#2$}
  \def\QivBAR#1{\Qpc{T4}{\|}$^#1$}
\fi
%    \end{macrocode}
%    \begin{macrocode}
\newif\ifVIET\VIETfalse
\ifVIET
  \def\Qv#1#2{\incsyms\indexaccent[\string\encfive{\string\blackacchack{\string#1}}]{#1}{\encfive{#1{A}#1{a}}} &
              \ttfamily\string#1\string{A\string}\string#1\string{a\string}$^#2$}
\else
  \def\Qv#1#2{\Qpc{T5}{#1}$^#2$}\def\Qv#1#2{Err}%TODO
\fi
%    \end{macrocode}
%
% \begin{docCommand}{R} { \oarg{ams cmd} \marg {cmd} }
%   Used for variable size math operators
% \end{docCommand}
%
%    \begin{macrocode}
\ExplSyntaxOn
\cs_set:Npn \R@opt@arg#1#2
  {
    \incsyms
    \indexcommand[$\string#1$]{#2}
     $#1$ & $\displaystyle#1$ &\ttfamily\string#2
  }
 \cs_set:Npn \R@no@opt@arg#1
  {
    \incsyms
    \indexcommand[$\string#1$]{#1}
    $#1$ & $\displaystyle#1$ &\ttfamily\string#1
  }
\NewDocumentCommand {\R} { o m}
  {
    \IfNoValueTF {#1}
      { \R@no@opt@arg {#2}      }
      { \R@opt@arg    {#1} {#2} }
  }
\ExplSyntaxOff
%% T commands
%    \end{macrocode}
%
%
%\idxboth{variable-sized}{symbols}
%\index{integrals}
%\label{ams-large}
%\renewcommand{\arraystretch}{2.5}  % Keep tall symbols from touching.
%\begin{longtable}{l@{$\:$}ll@{\qquad}l@{$\:$}ll}
%\R[\iint]\iint     & \R[\iiint]\iiint       \\
%\R[\iiiint]\iiiint & \R[\idotsint]\idotsint \\
%\end{longtable}
%
%
%
% \begin{docCommand}{indexDing} { \marg{ ding symbol number }}
%   Auxiliary function to index and print in a table ding symbols. originally
%   from Comprehensive.
% \end{docCommand}
%
%    \begin{macrocode}
\ExplSyntaxOn
\newcommand \indexDing [1] 
  {
    \incsyms
    \indexcommand{\ding}
    \ding{#1} & 
    \footnotesize\ttfamily\string\ding \string{#1\string}
  }
\ExplSyntaxOff
%    \end{macrococode}
%
%    \begin{macrocode}
\def\Tding#1{%
  \incsyms\indexcommand[\ding{#1}]{\ding{#1}}\ding{#1}\indexcommand{\ding} &
  \ttfamily\string\ding\string{#1\string}%
}

\def\Tm#1{\incsyms\indexcommand{\maya}$\mayadigit{#1}$ &\ttfamily\string\maya\string{#1\string}}
\def\Tmoon#1{\incsyms\indexcommand{\MoonPha}\MoonPha{#1} &\ttfamily\string\MoonPha\string{#1\string}}
%    \end{macrocode}
%
% \begin{docCommand}{indexTextcomp} {\oarg{ltx cmd} \marg{symbol arg}}
%   This command typesets its command argument in a table row of two
%   (used for textcomp symbols).
% \end{docCommand}  
% 
%    \begin{macrocode}
\newcommand{\indexTextcomp}[2][]{%
   \incsyms#1 & 
   \indexcommand[#2]{#2}% necessary to put symbol \text
   #2%  
   &\ttfamily\string#2
}
%    \end{macrocode}
%
% \begin{docCommand} {Vp} {}
%  Commands that work both in math and text mode
% \end{docCommand}
%    \begin{macrocode}   
\newcommand{\Vp}[2][]{\incsyms#1 & \indexpunct[$#2$]{#2}#2 &\ttfamily\string#2}
\newcommand{\V}[2][]{\incsyms\indexcommand[#1]{#2}#1 & \indexcommand[#2]{#2}#2 &\ttfamily\string#2}
\newcommand{\Vl}[1]{\incsyms\indexcommand{#1}#1 & & \ttfamily\string#1}
\newcommand{\Vpl}[1]{\incsyms\indexpunct[$#1$]{#1}#1 & & \ttfamily\string#1}
%    \end{macrocode}
%
%    \begin{macrocode}
\def\W@opt@arg[#1]#2#3{%
    \incsyms\indexaccent[$\string\blackacc{\string#1}$]{#2}%
    $#1{#3}$ &\ttfamily\string#2\string{#3\string}}

\def\W@no@opt@arg#1#2{%
    \incsyms\indexaccent[$\string\blackacc{\string#1}$]{#1}%
    $#1{#2}$ &\ttfamily\string#1\string{#2\string}}
    
\def\W{\@ifnextchar[{\W@opt@arg}{\W@no@opt@arg}}
%    \end{macrocode}
%
%    \begin{macrocode}
\def\Wf#1#2{\incsyms\indexcommand{#1}$#1{#2}$ &\ttfamily\string#1\string{#2\string}}
\def\Ww#1#2#3{\incsyms\indexcommand{#2}$#1{#3}$ &\ttfamily\string#2\string{#3\string}}
\def\Wul#1#2#3{%
  \incsyms\indexaccent[$\string\blackacctwo{\string#1}$]{#1}%
  $#1{#2}{#3}$ &\ttfamily\string#1\string{#2\string}\string{#3\string}}
%    \end{macrocode}

% \begin{docCommand}{X} { \oarg{command} \marg{command} }
%   Typesets its arguments as commands and also the resulting symbol in 
%   math. Used for symbol tables in the documentation.
%  \end{docCommand}
%
%\begin{symtable}{AMS Commands Defined to Work in Both Math and Text Mode}
%\index{check marks}
%\label{ams-math-text}
%\begin{tabular}{*2{ll@{\qquad}}ll}
%\X\checkmark & \X\circledR & \X\maltese
%\end{tabular}
%\end{symtable}
%
%
%
% \tcbdocmarginnote{U 25-6-2015}
%    \begin{macrocode}
\ExplSyntaxOn
\def\X_opt_arg#1#2 {\incsyms\indexcommand[$\string#1$]{#2}$#1$ &\ttfamily\string#2}
\def\X@no@opt@arg#1{\incsyms\indexcommand[$\string#1$]{#1}$#1$ &\ttfamily\string#1}

\NewDocumentCommand {\X} { o m}
  {
    \IfNoValueTF{#1}
      { \X@no@opt@arg  {#2}   }
      { \X_opt_arg {#1} {#2}  }
  }
\ExplSyntaxOff  
% \def\X@opt@arg[#1]#2{\incsyms\indexcommand[$\string#1$]{#2}$#1$ &\ttfamily\string#2}
%  \def\X@no@opt@arg#1{\incsyms\indexcommand[$\string#1$]{#1}$#1$ &\ttfamily\string#1}
%  \def\X{\@ifnextchar[{\X@opt@arg}{\X@no@opt@arg}}
%    \end{macrocode}
%
%
% \begin{docCommand}{Y}{\marg{math command}}    
%    \begin{macrocode}
\def\Y#1{\incsyms\indexcommand[$\string\big\string#1$]{#1}$\big#1$ & $\Bigg#1$ &\scriptsize\ttfamily\string#1}
%    \end{macrocode}
% \end{docCommand}
%
%
% \begin{docCommand} {Z} { \marg{arg1} }
%  Typesets and index its arguments.
%\idxboth{log-like}{symbols}
%\index{atomic math objects}
%\index{limits}
%\label{log}
%
%\begin{tabular}{*8l}
%\Z\arccos & \Z\cos  & \Z\csc & \Z\exp & \Z\ker    & \Z\limsup & \Z\min & \Z\sinh \\
%\Z\arcsin & \Z\cosh & \Z\deg & \Z\gcd & \Z\lg     & \Z\ln     & \Z\Pr  & \Z\sup  \\
%\Z\arctan & \Z\cot  & \Z\det & \Z\hom & \Z\lim    & \Z\log    & \Z\sec & \Z\tan  \\
%\Z\arg    & \Z\coth & \Z\dim & \Z\inf & \Z\liminf & \Z\max    & \Z\sin & \Z\tanh
% \end{tabular}
% \end{docCommand}
%    \begin{macrocode}
\ExplSyntaxOn
\cs_set:Npn \Z #1
  {
    \incsyms
    \indexcommand[$\string#1$] {#1}
    \footnotesize
    \ttfamily
    \string #1
  }
\ExplSyntaxOff
%    \end{macrocode}
%
% \begin{docCommand} {utfviii}  { \meta{void} }
%  Typesets UTF-8.
% \end{docCommand}
%    \begin{macrocode}
\newcommand{\utfviii}{\mbox{UTF-8}\index{UTF-8}\xspace}

% Index TeXbook symbols and the CTAN repository.
\newcommand{\idxTBsyms}{%
  \index{symbols>TeXbook=\TeX{}book}% 
  \index{TeXbook, The=\TeX{}book, The>symbols from}%
}
%    \end{macrocode}
%
% \begin{docCommand}{pkgname}{ \marg{package name}}
% Typesets and indexes a \latex package.
% \end{docCommand}
%    \begin{macrocode}
\newcommand{\pkgname}[1]{%
  \href{http://ctan.org/pkg/#1}{\bfseries{#1}}%
  \index{#1=\texttt{#1} (package)}%
  \index{packages>#1=\texttt{#1}}
 }
 
\let\pkg\pkgname
\let\Lpack\pkgname

\newcommand*\opt[1]{\texttt{#1}}

\newcommand*\feat[1]{\texttt{#1}}


\newcommand{\optname}[2]{%
  \textsf{#2}%
  \index{#2=\textsf{#2} (\textsf{#1} package option)}%
  \index{package options>#2=\textsf{#2} (\textsf{#1})}}
%    \end{macrocode}
%
% \begin{docCommand}{docClass}{\marg{name of class}}
%   Prints and indexes a \latex class.
% \end{docCommand}
%    \begin{macrocode}
\newcommand{\docClass}[1]{%
  \href{http://ctan.org/pkg/#1}{\bfseries{#1}}%
  \index{#1=\texttt{#1} (class)}%
  \index{classes>#1=\texttt{#1}}}

\let\Lpack\pkgname
%    \end{macrocode}
% 
% This macro and all similar macros starting from doc
% typeset their argument and also add the argument to the 
% index.
%
% \begin{docCommand}{docfilename}{ \marg{file name}}
% Typesets and indexes a file name.
% \end{docCommand}
%    \begin{macrocode}
\newcommand{\docfilename}[1]{%
  \texttt{#1}
  \index{#1=\phdindexprintcomc{#1}(file)}}
  

\let\docFilename\docfilename  
%    \end{macrocode}
% 
% 
% \begin{docCommand}{docfileextension}{ \marg{file extension}}
% Typesets and indexes a file extension, such as \refCmd{docfileextension}\marg{.tex}  (\docfileextension{.tex}). You type
% the dot if you want it to appear in the index, which is a good idea.
% \end{docCommand}
%    \begin{macrocode}
\newcommand{\docfileextension}[1]{%
  \texttt{#1}%
  \index{#1=\texttt{#1} (file extension)}}
   \index{#1=\texttt{#1}}
   
\newcommand{\PSfont}[1]{%
  #1%
  \index{#1 (font)}%
  \index{fonts\index@level#1}%
}
%    \end{macrocode}
% 
%    \begin{macrocode}
\NewDocumentCommand{\person} { m m } {#1\index{#2, #1} #2}
%    \end{macrocode}
%
% \begin{docCommand}{ctan}{\marg{package name}}
% Provides a link to the ctan package repository
% \end{docCommand}
%    \begin{macrocode}
\DeclareRobustCommand\ctan[1]{%
  \textcolor{green}{%
      \href{http://www.ctan.org/pkg/#1} {{\bfseries #1}}%
  \footnote{\protect\url{http://www.ctan.org/pkg/#1}}}
  \index{Packages>#1}%
}
%    \end{macrocode}
%    \begin{macrocode}
\newcommand{\idxCTAN}{%
  \index{Comprehensive TeX Archive Network=Comprehensive \string\TeX{} Archive Network}}
% Typeset a string in various encodings.
\newcommand{\encone}[1]{{\fontencoding{T1}\selectfont#1}}
\newcommand{\encfour}[1]{{\fontencoding{T4}\selectfont#1}}
\newcommand{\encfive}[1]{{\fontencoding{T5}\selectfont#1}}
\newcommand{\encgreek}[1]{{\fontencoding{LGR}\selectfont#1}}

% Various punctuation marks confuse makeindex when used directly.
\let\magicrbrack=]
\let\magicequal=\=
\DeclareRobustCommand{\magicequalname}{\texttt{\string\=}}
\DeclareRobustCommand{\magicvertname}{\texttt{|}}
\DeclareRobustCommand{\magicVertname}{\texttt{\string\|}}

% Vertically center a text-mode symbol.
\newsavebox{\tvcbox}
\newcommand*{\textvcenter}[1]{%
  \savebox{\tvcbox}{#1}%
  \raisebox{0.5\dp\tvcbox}{\raisebox{-0.5\ht\tvcbox}{\usebox{\tvcbox}}}%
}
% Many tables have notes beneath them.  Define an environment in which to
% display such a note, with an optional, superscripted math symbol
% preceding it.
\newenvironment{tablenote}[1][]{
  \makebox[1em]{\ensuremath{^{#1}}}%
  \begin{minipage}[t]{0.75\textwidth}%
  \setlength{\parskip}{2ex}
}{%
  \end{minipage}%
}

% Define various messages we reuse repeatedly.
\newcommand{\twosymbolmessage}{%
  \begin{tablenote}
    Where two symbols are present, the left one is the ``faked'' symbol
    that \latexe provides by default, and the right one is the ``true''
    symbol that \TC\ makes available.
  \end{tablenote}
}

\newcommand{\notpredefinedmessage}{%
  \begin{tablenote}[*]
    Not predefined in \latexe.  Use one of the packages
    \pkgname{latexsym}, \pkgname{amsfonts}, \pkgname{amssymb},
    \pkgname{txfonts}, \pkgname{pxfonts}, or \pkgname{wasysym}.
  \end{tablenote}
}

\newcommand{\notpredefinedmessageABX}{%
  \begin{tablenote}[*]
    Not predefined in \latexe.  Use one of the packages
    \pkgname{latexsym}, \pkgname{amsfonts}, \pkgname{amssymb},
    \pkgname{mathabx}, \pkgname{txfonts}, \pkgname{pxfonts}, or
    \pkgname{wasysym}.
  \end{tablenote}
}

\newcommand{\usetextmathmessage}[1][]{%
  \begin{tablenote}[#1]
    It's generally preferable to use the corresponding symbol from
    \vref{math-text} because the symbols in that table work
    properly in both text mode and math mode.
  \end{tablenote}
}



\newcommand{\usefontcmdmessage}[2]{%
  These symbols must appear either within the argument to \cmd{#1} or
  following the \cmd{#2} font-selection command within a scope%
}
% Define an environment in which to write a single table of symbols.  The
% environment looks a lot like a table, but it doesn't float, and it gets
% an entry in the table of contents as opposed to the list of tables.
%
% The first argument is a conditional.  The table will appear only if
% the value of the conditional is true.  The second argument is the
% table's caption.

\def\fnum@table{\tablename~\thetable}

\newenvironment{symtable}[2][true]{%
  \bgroup
  \expandafter\global\expandafter\let%
    \expandafter\ifshowsymtable\csname if#1\endcsname
  \ifshowsymtable
    \noindent%
    \begin{minipage}[t]{\linewidth}    % Prevent page breaks
    \begin{center}
    \refstepcounter{table}%
    \phantomsection
    \addcontentsline{toc}{subsection}{%
      \protect\numberline{\tablename~\thetable:}{#2}}%
    \@makecaption{\fnum@table}{#2}\medskip
    \let\next=\relax
  \else
    % The following was taken verbatim from verbatim.sty.
    \let\do\@makeother\dospecials\catcode`\^^M\active
    \let\verbatim@startline\relax
    \let\verbatim@addtoline\@gobble
    \let\verbatim@processline\relax
    \let\verbatim@finish\relax
    \let\next=\verbatim@
  \fi
  \next
}{%
  \ifshowsymtable
    \end{center}
    \end{minipage}
    \vskip 8ex minus 2ex
  \fi
  \egroup
}
%    \end{macrocode}
% We need a command that can typeset a symbol in the text and index it by its command name and
% also show the symbol in brackets next to it. 
% \begin{docCommand}{docSymbol}{\oarg{optional explicit formatting}\marg{command}}
% \end{docCommand}
%   \begin{macrocode}
% Display and index a command, but not its symbol (\cmd).  \cmdI shows
% the symbol in the index, with optional explicit formatting.  \cmdX is
% the same as \cmdI, but with the optional argument hardwired to the
% command displayed in math mode.  \cmdW indexes an accent.  \cmdIp is
% also similar to \cmdI but formats its argument with \indexpunct
% instead of \indexcommand.
\ExplSyntaxOn
\newcommand{\docSymbol}[2][]{%
  \def\first@arg{#1}%
  \ifx\first@arg\@empty
    \texttt{\string#2} (#2)%
    \indexcommand[#2]{#2}%
  \else
    \texttt{\string#2} (#2)%
    \indexcommand[#1]{#2}%
  \fi
}
\ExplSyntaxOff
%    \end{macrocode}
%
%
%\begin{symtable}[true]{Fontawsome Currency Symbols}
%\idxboth{currency}{symbols}
%\idxboth{monetary}{symbols}
%\index{euro signs}
%\label{fontawesome-currency}
%\begin{tabular}{*4{ll}ll}
%\K\faBtc & \K\faIls & \K\faKrw & \K\faUsd     \\
%\K\faEur & \K\faInr & \K\faRub & \K\faViacoin \\
%\K\faGbp & \K\faJpy & \K\faTry &              \\
%\end{tabular}
%
%\bigskip
%
%\begin{tablenote}
%   Fontawesome defines \docSymbol[\faBitcoin]{\faBitcoin} as a synonym for \docSymbol{\faBtc};
%  \docSymbol{\faCny}, \docSymbol{\faYen}, and \docSymbol{\faRmb} as synonyms for
%  \docSymbol{faJpy}; \docSymbol{faDollar} as a synonym for \docSymbol{faUsd};
%  \docSymbol{faEuro} as a synonym for \docSymbol{faEur}; \docSymbol{faRouble} and
%  \docSymbol{faRuble} as synonyms for \docSymbol{faRub}; \docSymbol{faRupee} as a
%  synonym for \docSymbol{faInr}; \docSymbol{faShekel} and \docSymbol{faSheqel} as
%  synonyms for \docSymbol{faIls}; \docSymbol{faTurkishLira} as a synonym for
%  \docSymbol{faTry}; and \docSymbol{\faWon} as a synonym for \docSymbol{\faKrw}.
%\end{tablenote}
%\end{symtable}


%    \begin{macrocode}
\newenvironment{nonsymtable}[1]{%
  \begin{table}[htbp]
  \centering
  \caption{#1}\medskip
}{%
  \end{table}
}
   


{
  \global\let\myempty=\@empty
  \global\let\mygobble=\@gobble
  \catcode`\@=12
  \gdef\getridofats#1@#2\relax{%
    \def\getridtest{#2}%
    \ifx\getridtest\myempty%
      \expandafter\def\expandafter\strippedat\expandafter{\strippedat#1}
    \else%
      \expandafter\def\expandafter\strippedat\expandafter{\strippedat#1\protect\printanat}
      \getridofats#2\relax%
    \fi%
  }

  \gdef\removeats#1{%
    \let\strippedat\myempty%
    \edef\strippedtext{\stripcommand#1}%
    \expandafter\getridofats\strippedtext @\relax%
  }
  
  \gdef\stripcommand#1{\expandafter\mygobble\string#1}
}


\def\printanat{\char`\@}

\def\declare{\afterassignment\pgfmanualdeclare\let\next=}
\def\pgfmanualdeclare{\ifx\next\bgroup\bgroup\color{red!75!black}\else{\color{red!75!black}\next}\fi}


\let\textoken=\command
\let\endtextoken=\endcommand

\def\myprintocmmand#1{\texttt{\char`\\#1}}
%    \end{macrocode}
%
% \begin{docCommand}{example}{\marg{void}}
%  A no parameter macro to typeset a one line example, in code.
%  \example This is an example for $\beta$.
%    \begin{macrocode}
\def\example{\par\smallskip\noindent\textit{Example: }}
%    \end{macrocode}
% \end{docCommand}
%
%    \begin{macrocode}
\def\themeauthor{\par\smallskip\noindent\textit{Theme author: }}


\def\indexoption#1{%
  \index{#1@\protect\texttt{#1} option}%
  \index{Graphic options and styles!#1@\protect\texttt{#1}}%
}

\def\itemcalendaroption#1{\item \declare{\texttt{#1}}%
  \index{#1@\protect\texttt{#1} date test}%
  \index{Date tests!#1@\protect\texttt{#1}}%
}
%    \end{macrocode}
% \begin{docEnvironment}{class}{\marg{class}}
% \end{docEnvironment}
%
% \begin{class}{{article}{[10pt,oneside]}}
% \end{class}
%    \begin{macrocode}
\def\class#1{%
  \list{}% 
    {\leftmargin=2em\itemindent-\leftmargin\def\makelabel##1{\hss##1}}%
   \extractclass#1@\par\topsep=0pt
}
\def\endclass{\endlist}

\def\extractclass#1#2@{%
\item{{{\ttfamily\char`\\documentclass}#2{\ttfamily\char`\{\declare{#1}\char`\}}}}%
  \index{#1@\protect\texttt{#1} class}%
  \index{Classes!#1@\protect\texttt{#1}}}



\def\index@prologue{\section*{Index}\addcontentsline{toc}{section}{My Index}
  This index only contains automatically generated entries. A good
  index should also contain carefully selected keywords. This index is
  not a good index.
  \bigskip
}
\@ifundefined{c@IndexColumns}{\newcount\c@IndexColumns}{}
\c@IndexColumns=2
  \def\theindex{\@restonecoltrue
    \columnseprule \z@  \columnsep 29\p@
    \twocolumn[\index@prologue]%
       \parindent -30pt
       \columnsep 15pt
       \parskip 0pt plus 1pt
       \leftskip 30pt
       \rightskip 0pt plus 2cm
       \small
       \def\@idxitem{\par}%
    \let\item\@idxitem \ignorespaces}
  \def\endtheindex{\onecolumn}
\def\noindexing{\let\index=\@gobble}



\newcommand\symarrow[1]{
  \index{#1@\protect\texttt{#1} arrow tip}%
  \index{Arrow tips!#1@\protect\texttt{#1}}
  \texttt{#1}& yields thick  
  \begin{tikzpicture}[arrows={#1-#1},thick,baseline]
    \useasboundingbox (0pt,-0.5ex) rectangle (1cm,2ex);
    \draw (0pt,.5ex) -- (1cm,.5ex);
  \end{tikzpicture} and thin
  \begin{tikzpicture}[arrows={#1-#1},thin,baseline]
    \useasboundingbox (0pt,-0.5ex) rectangle (1cm,2ex);
    \draw (0pt,.5ex) -- (1cm,.5ex);
  \end{tikzpicture}
}

\newcommand\sarrow[2]{
  \index{#1@\protect\texttt{#1} arrow tip}%
  \index{Arrow tips!#1@\protect\texttt{#1}}
  \index{#2@\protect\texttt{#2} arrow tip}%
  \index{Arrow tips!#2@\protect\texttt{#2}}
  \texttt{#1-#2}& yields thick  
  \begin{tikzpicture}[arrows={#1-#2},thick,baseline]
    \useasboundingbox (0pt,-0.5ex) rectangle (1cm,2ex);
    \draw (0pt,.5ex) -- (1cm,.5ex);
  \end{tikzpicture} and thin
  \begin{tikzpicture}[arrows={#1-#2},thin,baseline]
    \useasboundingbox (0pt,-0.5ex) rectangle (1cm,2ex);
    \draw (0pt,.5ex) -- (1cm,.5ex);
  \end{tikzpicture}
}

\newcommand\carrow[1]{
  \index{#1@\protect\texttt{#1} arrow tip}%
  \index{Arrow tips!#1@\protect\texttt{#1}}
  \texttt{#1}& yields for line width 1ex
  \begin{tikzpicture}[arrows={#1-#1},line width=1ex,baseline]
    \useasboundingbox (0pt,-0.5ex) rectangle (1.5cm,2ex);
    \draw (0pt,.5ex) -- (1.5cm,.5ex);
  \end{tikzpicture}
}



\newcommand\plotmarkentry[1]{%
  \index{#1@\protect\texttt{#1} plot mark}%
  \index{Plot marks!#1@\protect\texttt{#1}}
  \texttt{\char`\\pgfuseplotmark\char`\{\declare{#1}\char`\}} &
  \tikz\draw[color=black!25] plot[mark=#1,mark options={fill=examplefill,draw=black}] coordinates{(0,0) (.5,0.2) (1,0) (1.5,0.2)};\\
}
\newcommand\plotmarkentrytikz[1]{%
  \index{#1@\protect\texttt{#1} plot mark}%
  \index{Plot marks!#1@\protect\texttt{#1}}
  \texttt{mark=\declare{#1}} & \tikz\draw[color=black!25]
  plot[mark=#1,mark options={fill=examplefill,draw=black}] 
    coordinates {(0,0) (.5,0.2) (1,0) (1.5,0.2)};\\
}



\ifx\scantokens\@undefined
  \PackageError{phd}{You need to use extended latex
    (elatex) or (pdfelatex) to process this document}{}
\fi

\begingroup
\catcode`|=0
\catcode`[= 1
\catcode`]=2
\catcode`\{=12
\catcode `\}=12
\catcode`\\=12 |gdef|find@example#1\end{codeexample}[|endofcodeexample[#1]]
|endgroup

\begingroup
\catcode`\^=7
\catcode`\^^M=13
\catcode`\ =13%
\gdef\returntospace{\catcode`\ =13\def {\space}\catcode`\^^M=13\def^^M{}}%
\endgroup

\begingroup
\catcode`\%=13
\catcode`\^^M=13
\gdef\commenthandler{\catcode`\%=13\def%{\@gobble@till@return}}
\gdef\@gobble@till@return#1^^M{}
\gdef\@gobble@till@return@ignore#1^^M{\ignorespaces}
\gdef\typesetcomment{\catcode`\%=13\def%{\@typeset@till@return}}
\gdef\@typeset@till@return#1^^M{{\def%{\char`\%}\textsl{\char`\%#1}}\par}
\endgroup

\define@key{codeexample}{width}{\setlength\codeexamplewidth{#1}}
\define@key{codeexample}{graphic}{\colorlet{thecodebackground}{#1}}
\define@key{codeexample}{code}{\colorlet{thecodebackground}{#1}}
\define@key{codeexample}{execute code}{\csname code@execute#1\endcsname}
\define@key{codeexample}{code only}[]{\code@executefalse}
\define@key{codeexample}{pre}{\def\code@pre{#1}}
\define@key{codeexample}{post}{\def\code@post{#1}}
\define@key{codeexample}{vbox}[]{\def\code@pre{\vbox\bgroup\setlength{\hsize}{\linewidth-6pt}}\def\code@post{\egroup}}
\define@key{codeexample}{ignorespaces}[]{\let\@gobble@till@return=\@gobble@till@return@ignore}
\define@key{codeexample}{leave comments}[]{\def\code@catcode@hook{\catcode`\%=12}\let\commenthandler=\relax\let\typesetcomment=\relax}
\def\code@pre{}
\def\code@post{}
\def\code@catcode@hook{}

\newdimen\codeexamplewidth
\newif\ifcode@execute
\newbox\codeexamplebox
\def\codeexample[#1]{%
  \begingroup%
  \code@executetrue
  \setlength\codeexamplewidth{4cm+7pt}
  \setkeys{codeexample}{#1}%
  \parindent0pt
  \begingroup%
  \par%
  \medskip%
  \let\do\@makeother%
  \dospecials%
  \obeylines%
  \@vobeyspaces%
  \catcode`\%=13%
  \catcode`\^^M=13%
  \code@catcode@hook%
  \relax%
  \find@example}
\def\endofcodeexample#1{%
  \endgroup%
  \ifcode@execute%
    \setbox\codeexamplebox=\hbox{%
      {%
        {%
          \returntospace%
          \commenthandler%
          \xdef\code@temp{#1}% removes returns and comments
        }%
        \colorbox{thecodebackground}{\color{black}\ignorespaces%
          \code@pre\expandafter\scantokens\expandafter{\code@temp\ignorespaces}\code@post\ignorespaces}%
      }%
    }%
    \ifdim\wd\codeexamplebox>\codeexamplewidth%
      \def\code@start{\par}%
      \def\code@flushstart{}\def\code@flushend{}%
      \def\code@mid{\parskip2pt\par\noindent}%
      \def\code@width{\linewidth-6pt}%
      \def\code@end{}%
    \else%
      \def\code@start{%
        \linewidth=\textwidth%
        \parshape \@ne 0pt \linewidth
        \leavevmode%
        \hbox\bgroup}%
      \def\code@flushstart{\hfill}%
      \def\code@flushend{\hbox{}}%
      \def\code@mid{\hskip6pt}%
      \def\code@width{\linewidth-12pt-\codeexamplewidth}%
      \def\code@end{\egroup}%
    \fi%
    \code@start%
    \noindent%
    \begin{minipage}[t]{\codeexamplewidth}\raggedright
      \hrule width0pt%
      \small%\vskip-1em%
      \code@flushstart\box\codeexamplebox\code@flushend%
      \vskip-1ex
      \leavevmode%
    \end{minipage}%
  \else%
    \def\code@mid{\par}
    \def\code@width{\linewidth-6pt}
    \def\code@end{}
  \fi%
  \code@mid%  
  \colorbox{thecodebackground}{%
    \begin{minipage}[t]{\code@width}%
      {%
        \let\do\@makeother
        \dospecials
        \frenchspacing\@vobeyspaces
        \normalfont\ttfamily%\footnotesize
        \typesetcomment%
        \@tempswafalse
        \def\par{%
          \if@tempswa
          \leavevmode \null \@@par\penalty\interlinepenalty
          \else
          \@tempswatrue
          \ifhmode\@@par\penalty\interlinepenalty\fi
          \fi}%
        \obeylines
        \everypar \expandafter{\the\everypar \unpenalty}%
        #1}
    \end{minipage}}%
  \code@end%
  \par%
  \medskip
  \end{codeexample}
}

\def\endcodeexample{\endgroup}
%    \end{macrocode}
%
% 
% From pgfplots manual
% 
%    \begin{macrocode}
\long\def\codeexamplenl{\noexpand\par}%
\pgfqkeys{/codeexample}{%
	every codeexample/.style={
		width=3.9cm,
		/pgfplots/every axis/.append style={legend style={fill=thecodebackground}}
	},
	narrow/.style={width=6.9cm},
	%tabsize=4,
	%pre={\begin{minipage}{\linewidth}\begingroup},
	%post={\endgroup\end{minipage}},
	%vbox,
	%newline=\codeexamplenl,
}
%    \end{macrocode}
%
%
%   \begin{docCommand}{keyval}{\meta{key}\meta{options}\meta{text}}

%	The macro \cs{keyval} typesets, key value lists and their options.
%   
%	\verb+|\keyval{test}{\marg{option1|option2|option2|option4}{text}+
%   

%   \keyval{test} {\marg{option1|option2|option2|option4}} { As per this example.}
%
%   \end{docCommand}
%	We first measure the width of the option and not use it (want to make it a bit
%	flexible at a later stage. We also ensure that the catcode of \verb+|+ is set properly
%	in case anyone is using short verbatim commands, as we do in this document.
%
%    \begin{macrocode}
\newlength\temp@cx
\def\keyval{%
  \begingroup
  \catcode`|=11
  \@keyval}
%

\def\@keyval#1#2#3{%
  \settowidth\temp@cx{#1}%
  \parindent-30pt
  \hangindent30pt
  \par\leavevmode%
{\verbatimfont\color{themeta}\bfseries #1}\thinspace=\thinspace#2% 
\hspace*{.5em}#3%
\par\addvspace{1.5pt}%
\endgroup
}
%
%    \end{macrocode}
%  Typesets a sample of bib
%    \begin{macrocode}
\newenvironment{bibsample}
  {\trivlist\samepage
   \setlength{\itemsep}{0pt}}
  {\endtrivlist}
%% doccommands
\newcommand*{\marglistfont}{\itshape}

\newcommand*{\margoptionfont}{\ttfamily}

\newcommand*{\margnotefont}{}

\newcommand*{\optionlistfont}{\bfseries}

\newcommand*{\ltxsyntaxfont}{\ttfamily}

\newcommand*{\ltxsyntaxlabelfont}{\bfseries}

\newcommand*{\changelogfont}{\normalfont}

\newcommand*{\changeloglabelfont}{\bfseries}



%\def\cmd#1{\cs{\expandafter\cmd@to@cs\string#1}}%

%\def\cmd@to@cs#1#2{\char\number`#2\relax}

\newrobustcmd*{\env}[1]{\mbox{\verbatimfont\bfseries\textcolor{thegreen}{#1}}}

\newrobustcmd*{\len}[1]{\mbox{\verbatimfont\textbackslash#1}}

\newrobustcmd*{\cnt}[1]{\mbox{\verbatimfont#1}}

\newlength{\marglistsep}

\newlength{\marglistwidth}

\setlength{\marglistwidth}{(\oddsidemargin+1in)*85/100}%

\deflength{\marglistsep}{10pt}
%% This needs thorough checking as to restore previous definitions
%% of parsep we want parsep to be a bit higher than standard enumerated lists.
\global\newlength\varparsep
\newenvironment*{marglist}
  {\setlength\varparsep{\parsep}\list{}{%
     \parsep 3.5\p@ \@plus0\p@ \@minus\p@
     \setlength{\labelwidth}{\marglistwidth}%
     \setlength{\labelsep}{\marglistsep}%
     \setlength{\leftmargin}{0pt}%
     \renewcommand*{\makelabel}[1]{\hss\marglistfont##1}}}
  {\endlist\setlength\parsep{\varparsep}}

%
\newenvironment*{margoptionslist}
  {\setlength\varparsep{\parsep}\list{}{%
     \parsep 3.5\p@ \@plus0\p@ \@minus\p@
     \setlength{\labelwidth}{\marglistwidth}%
     \setlength{\labelsep}{\marglistsep}%
     \setlength{\leftmargin}{0pt}%
     \renewcommand*{\makelabel}[1]{\hss\margoptionfont\detokenize{##1}}}}
  {\endlist\setlength\parsep{\varparsep}}
  
  
%    \end{macrocode}
%
% \begin{docEnvironment} {keymarglist} { \meta{void} } 
%   Typesets a key options list in the margin.
% \end{docEnvironment}
%
%    \begin{macrocode}
\newenvironment*{keymarglist}
  {\marglist
   \setlength{\itemsep}{0pt}%
   \raggedright}
  {\endmarglist}
% color definitions
\def\cvaref#1{\verbatimfont\textcolor{themacro}{#1}}
% color for options
\def\colOpt#1{\textcolor{theoption}{\verbatimfont\texttt{#1}}}
%    \end{macrocode}
%
% \begin{docCommand}{option}{\marg{option}}
%  Typesets an option. It uses the color \docColor{theoption}, which is defined in the package \pkg{phd-colorpalette}.
%    \begin{macrocode}
\newcommand{\option}[1]{\colOpt{#1}}
%    \end{macrocode}
% \end{docCommand}
%
%  \subsection{Creating a Small Verbatim Environment}
%  This is a modified version from Cambridge classes
%    \begin{macrocode}
\begingroup \catcode `|=0 
\catcode `[= 1
\catcode`]=2 
\catcode `\{=12 
\catcode `\}=12
\catcode`\\=12 
|gdef|@xsmallverbatim#1\end{smallverbatim}[#1|end[smallverbatim]]
|gdef|@sxsmallverbatim#1\end{smallverbatim*}[#1|end[smallverbatim*]]
|endgroup
\def\@smallverbatim{\trivlist \item\relax
  \if@minipage\else\vskip\parskip\fi
  \leftskip\@totalleftmargin\rightskip\z@skip
  \parindent\z@\parfillskip\@flushglue\parskip\z@skip
  \@@par
  \@tempswafalse
  \def\par{%
    \if@tempswa
      \leavevmode \null \@@par\penalty\interlinepenalty
    \else
      \@tempswatrue
      \ifhmode\@@par\penalty\interlinepenalty\fi
    \fi}%
  \let\do\@makeother \dospecials
  \obeylines \small \@noligs%\smallverbatim@font to FIX
  \hyphenchar\font\m@ne
  \everypar \expandafter{\the\everypar \unpenalty}%
}
\def\smallverbatim{\@smallverbatim \frenchspacing\@vobeyspaces \@xsmallverbatim}
\def\endsmallverbatim{\if@newlist \leavevmode\fi\endtrivlist}
\def\smallverbatim@font{\normalfont\smallverbatimsize\ttfamily}
%    \end{macrocode}
% 
% This is a short test. \lorem
%  \begin{smallverbatim}
%  \ifx\bhj
%  \else
%  \fi
%  \end{smallverbatim}
% \lorem
% \begin{docEnvironment}{docCommands}{}
% \end{docEnvironment}
%    \begin{macrocode}
\let\luacmd\docValue
\newenvironment{docCommands}{%
\bgroup
\par
\parindent=0pt
\parskip=3.5pt plus0.5pt
\everypar{\hangindent2em}%
\addvspace\belowdisplayskip\relax}%
{\everypar{}%
 \par
 \vskip\belowdisplayskip\egroup\par}
\long\def\auxm#1(#2);{%
  \def\Xtemp{#1}%
  \def\Ytemp{#2}%
  \parindent=0pt
  \addvspace{1.5pt}%
  \par\leavevmode
  \hangafter=1\relax   \hangindent=1em\relax
  \bgroup  
   \bfseries\sffamily\color{red}\Xtemp\,\color{black}(\textit{\Ytemp})\hskip0.1em
  \egroup
}

   
\newenvironment{docLua}[1]{%
  \auxm#1;
 }{%
\@@par
\smallskip\parindent=1em } 


%    \end{macrocode}
%
% \begin{docCommand}{docFont}{\marg{font name}}
%  Typeset and indexes a font by name, such as \cs{docFont{Arial}} typesetting \docFont{Arial}. 
%    \begin{macrocode}
\DeclareRobustCommand{\phdidxfont}[1]{\index{#1 (font)}\index{Fonts\idx@level #1}}%
\def\docFont#1{
    \enquote{#1} 
    \phdidxfont{#1}%
} 
%    \end{macrocode}
% \end{docCommand}
%
%
% \begin{handler}{.fontweight}{}
%  A handler to handle fontweights. Chooses betwenn.This handler causes the default path to be set to hkeyi. Note that the default path is reset at the beginning of each call to pgfkeys to be equal to \ldots
% TODO remove second parameter as it is not needed. Add index command
% \end{handler}
%    \begin{macrocode}
\newenvironment{handler}[2][]{%
  \begin{phd@manual@entry}%
   \begingroup
   \sffamily\textbf{Key handler} \meta{key}/\bfseries\ttfamily{\color{thered}#2}\color{black}#1
   \endgroup
 \end{phd@manual@entry}
 }
{}

% From egreg's italian latex guide
% \stok takes a \<char> argoment and prints it with its
% category code or the number given as optional argument
\newcommand\stok[2][]{%
  \texttt{#2}\ensuremath{_{%
  \if!#1!
    \the\catcode`#2
  \else
    #1
  \fi}}}
%    \end{macrocode}
% \section {Unicode math index functions}
%
% The functions that follow typeset unicode math tables.
%
%  \begin{docCommand} {showsymbolalpha} { \marg{cmd} \marg{unicode point} \marg{note symbol} }
%    Indexes and typesets all the alphabetic letters available in math, 
%    mostly greek and the dotless j and i.
%  \end{docCommand}
%
%    \begin{macrocode}
\newcommand\showsymbolalpha[3]
  {
    \par\noindent\hangindent=3em%
    \makebox[2em][l]{$#1$} \makebox[3.5em][l]{\texttt{U+#2}} 
    \cmd{#1}$^{#3}$
    \indexmathcmd [Math alphabetics] {#1}
  }
%    \end{macrocode}

%  \begin{docCommand} {showsymbol} { \marg{cmd} \marg{unicode point} \marg{note symbol} }
%  \end{docCommand}
%    \begin{macrocode}
\newcommand\showsymbol[3]{\par\noindent\hangindent=3em%
\makebox[2em][l]{$#1$} \makebox[3.5em][l]{\texttt{U+#2}} 
\cmd{#1}$^{#3}$\indexmathcmd [Math ordinary] {#1}}
%    \end{macrocode}
%
%  \begin{docCommand} {showsymbolbin} { \marg{cmd} \marg{unicode point} \marg{note symbol} }
%  \end{docCommand}
%    \begin{macrocode}
\newcommand\showsymbolbin[3]{\par\noindent\hangindent=3em%
\makebox[2em][l]{$#1$} \makebox[3.5em][l]{\texttt{U+#2}} 
\cmd{#1}$^{#3}$\indexmathcmd [Math bin operators] {#1}}
%    \end{macrocode}
%
%  \begin{docCommand} {showrelsymbol} { \marg{cmd} \marg{unicode point} \marg{note symbol} }
%  \end{docCommand}
%    \begin{macrocode}
\newcommand\showrelsymbol[3]{\par\noindent\hangindent=3em%
\makebox[2em][l]{$#1$} \makebox[3.5em][l]{\texttt{U+#2}} 
\cmd{#1}$^{#3}$\indexmathcmd [Math relations] {#1}}
%    \end{macrocode}
%
%  \begin{docCommand} {integralsymbol} { \marg{cmd} \marg{unicode point} \marg{note symbol} }
%    Typesets and inserts into index integral symbols
%  \end{docCommand}
%    \begin{macrocode}
\newcommand\integralsymbol[3]{\par\noindent\hangindent=3em%
\makebox[2em][l]{$#1$} \makebox[3.5em][l]{\texttt{U+#2}} 
\cmd{#1}$^{#3}$\indexmathcmd [Math integrals] {#1}}
%    \end{macrocode}
%
%  \begin{docCommand} {showop} { \marg{cmd} \marg{unicode point} \marg{note symbol} }
%    Typesets and inserts into index integral symbols
%  \end{docCommand}
% 
%    \begin{macrocode}
\newcommand\showop[3]{\par\noindent\hangindent=6em%
  \makebox[5em][l]{$#1$\hfill$\displaystyle#1$\hfill}
  \makebox[3.5em][l]{\small\texttt{U+#2}} \cmd{#1}$^{#3}$ 
  \indexmathcmd [Math big operators] {#1} }
%    \end{macrocode}
%
%  \begin{docCommand} {showmbrace} { \marg{cmd} \marg{unicode point} \marg{note symbol} }
%    Typesets and inserts middle brace symbols
%  \end{docCommand}
% 
%    \begin{macrocode}
\newcommand\showmbrace[3]{\par\noindent\hangindent=6em%
  \makebox[5em][l]{${#1}{\bigm#1}{\Bigm#1}{\biggm#1}{\Biggm#1}$}
  \makebox[3.5em][l]{\small\texttt{U+#2}} \cmd{#1}$^{#3}$ 
  \indexmathcmd [Math delimiters] {#1}  }
%    \end{macrocode}
%
%  \begin{docCommand} {showlbrace} { \marg{cmd} \marg{unicode point} \marg{note symbol} }
%    left braces
%  \end{docCommand}
%    \begin{macrocode}
\newcommand\showlbrace[3]{\par\noindent\hangindent=6em%
  \makebox[5em][l]{$\Biggl#1\biggl#1\Bigl#1\bigl#1#1$}
  \makebox[3.5em][l]{\small\texttt{U+#2}} \cmd{#1}$^{#3}$
  [Math delimiters] {#1}
  }
%    \end{macrocode}
%
%  \begin{docCommand} {showrbrace} { \marg{cmd} \marg{unicode point} \marg{note symbol} }
%    right braces
%  \end{docCommand}
%    \begin{macrocode}
\newcommand\showrbrace[3]{\par\noindent\hangindent=6em%
  \makebox[5em][l]{$#1\bigr#1\Bigr#1\biggr#1\Biggr#1$}
  \makebox[3.5em][l]{\small\texttt{U+#2}} \cmd{#1}$^{#3}$
  [Math delimiters] {#1}
  }
%    \end{macrocode}
%
%  \begin{docCommand} {wide accents} { \marg{cmd} \marg{unicode point} \marg{note symbol} }
%    wide accents
%  \end{docCommand}
%
%    \begin{macrocode}
\DeclareDocumentCommand \showwideaccent { m m m} {\par\noindent\hangindent=4em%
  \makebox[3em][l]{$#1{xxx}$}\makebox[3.5em][l]{\small\texttt{U+#2}} \cmd{#1}$^{#3}$
  \indexmathcmd [Math accents] {#1{abc}}
  }
%    \end{macrocode}
%
%  \begin{docCommand} {showaccent} { \marg{cmd} \marg{unicode point} \marg{note symbol} }
%    right braces
%  \end{docCommand}
%
%    \begin{macrocode}
\DeclareDocumentCommand\showaccent { m m m} {\par\noindent\hangindent=4em%
  \makebox[3em][l]{$#1b$}\makebox[3.5em][l]{\small\texttt{U+#2}} \cmd{#1}$^{#3}$
  \indexmathcmd [Math accents] {#1 b}
  }
%    \end{macrocode}
%
%  \begin{docCommand} {showrover} { \marg{cmd} \marg{unicode point} \marg{note symbol} }
%   
%  \end{docCommand}  
%    \begin{macrocode}  
\newcommand\showover[3]{\par\noindent\hangindent=6em%
  \makebox[5em][l]{$#1{xxxxxx}$}
  \makebox[3.5em][l]{\small\texttt{U+#2}} 
  \cmd{#1}$^{#3}$
  \indexmathcmd [Math over and under brackets] {#1{xxxxxx}}
  }
    
%    \end{macrocode}  
% 
%</DOCUM>
\endinput
%\begin{symtable}[FEYN]{\FEYN\ Feynman Diagram Symbols}
%\index{Feynman-diagram symbols}
%\index{symbols>Feynman diagram}
%\index{particle-physics symbols}
%\index{symbols>particle physics}
%\index{bosons}
%\index{fermions}
%\index{gluons}
%\index{photons}
%\index{proper vertices}
%\index{subatomic particles}
%\label{feyn}
%\renewcommand{\arraystretch}{1.75}  % Keep tall symbols from touching.
%\begin{tabular}{*2{ll@{\qquad}}ll}
%\K\bigbosonloop    & \K\hfermion        & \K\smallbosonloopV \\
%\K\bigbosonloopA   & \K\shfermion       & \K\wfermion        \\
%\K\bigbosonloopV   & \K\smallbosonloop  & \K\whfermion       \\
%\K\gvcropped       & \K\smallbosonloopA &                    \\
%\end{tabular}
%
%\vspace{\baselineskip}
%
%\begin{tabular}{*3{ll@{\qquad}}ll}
%\Kfeyn{a}   & \Kfeyn{fu}  & \Kfeyn{glS} & \Kfeyn{hs} \\
%\Kfeyn{c}   & \Kfeyn{fv}  & \Kfeyn{glu} & \Kfeyn{hu} \\
%\Kfeyn{f}   & \Kfeyn{g}   & \Kfeyn{gu}  & \Kfeyn{m}  \\
%\Kfeyn{fd}  & \Kfeyn{g1}  & \Kfeyn{gv}  & \Kfeyn{ms} \\
%\Kfeyn{fl}  & \Kfeyn{gd}  & \Kfeyn{gvs} & \Kfeyn{p}  \\
%\Kfeyn{flS} & \Kfeyn{gl}  & \Kfeyn{h}   & \Kfeyn{P}  \\
%\Kfeyn{fs}  & \Kfeyn{glB} & \Kfeyn{hd}  & \Kfeyn{x}  \\
%\end{tabular}
%
%\bigskip
%
%\begin{tablenote}
%  All other arguments to the \verb|\feyn| command produce a
%  ``~\feyn{?}~'' symbol.
%
%  The \FEYN\ package provides various commands for composing the
%  preceding symbols into complete Feynman diagrams.  See the \FEYN\
%  documentation for examples and additional information.
%\end{tablenote}
%\end{symtable}
