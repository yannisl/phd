\bgroup
\arial


\chapter{South Asian Scripts}

The scripts of South Asia share so many characteristics that a side by side comparison of a few often reveal structural similarities even in the 
modern letterforms.
\medskip


\begin{center}
\begin{tabular}{lll}
  \hyperref[s:devanagari]{Devanagari} 
& \hyperref[s:gujarati]{Gujarati}
& \hyperref[s:telugu]{Telugu}\\
  \hyperref[s:bengali]{Bengali}
& \hyperref[s:oriya]{Oriya} 
& \hyperref[s:kannada]{Kannada}\\
  \hyperref[s:gurmukhi]{Gurmukhi} 
& \hyperref[s:tamil]{Tamil}
& \hyperref[s:malayalam]{Malayalam}\\
  \hyperref[s:sinhala]{Sinhala} 
& \hyperref[s:kaithi]{Kaithi}  
& \hyperref[s:meeteimayek]{Meetei Mayek}\\
  \hyperref[s:tibetan]{Tibetan} 
& \hyperref[s:saurashtra]{Saurashtra} 
& \hyperref[olchiki]{Ol Chiki}\\
  \hyperref[s:lepcha]{Lepcha}
& \hyperref[s:sharada]{Sharada} 
& \hyperref[s:sorasompeng]{Sora Sompeng}\\
  \hyperref[s:phagspa]{Phags-pa} 
& \hyperref[s:takri]{Takri}  
& \hyperref[s:kharoshthi]{Kharoshthi}\\
  \hyperref[s:limbu]{Limbu} 
& \hyperref[s:chakma]{Chakma}
& \hyperref[s:brahmi]{Brahmi}\\
  \hyperref[s:sylotinagra]{Syloti Nagri} 
& \hyperref[s:mro]{Mro} 
&\\
\end{tabular}
\end{center}

The sections that follow describe the scripts briefly and the |phd| settings
to activate the relevant commands and load appropriate fonts. 

\begin{figure}[htbp]
\includegraphics[width=\textwidth]{./images/indic-language-tree.jpg}
\caption{A family tree of a few of the most important Indic scripts, (adapted from \protect\cite{writing}.}
\end{figure}


\section{Sinhala Alphabet}
\label{s:sinhala}
\index{scripts>Sinhala}

The Sinhala alphabet derives from Pali.

\newfontfamily{\sinhala}{NotoSansSinhala-Regular.ttf}

\begin{docKey}[phd]{sinhala font}{ = \meta{font face}}{default none, initial NotoSansSinhala-Regular.ttf}
\end{docKey}

\begin{docCommand}{textsinhala}{\marg{text}}
Command to typeset sinhalese text.
\end{docCommand}

\begin{docEnvironment}{sinhalascript}{}{}
\end{docEnvironment}

The \refEnv{sinhalascript} environment typesets text inputted in the Sinhala script.

\begin{scriptexample}[]{Sinhala}
\unicodetable{sinhala}{"0D80,"0D90,"0DA0,"0DB0,"0DC0,"0DD0,"0DE0,"0DF0}
\end{scriptexample}

\printunicodeblock{./languages/sinhala.txt}{\sinhala}
\section{Meitei Mayek alphabet}
\label{s:meiteimayek}
\index{scripts>Meitei Mayek}
\index{Meetei Mayek}
\newfontfamily\meitei{Noto Sans Meetei Mayek}

\def\textmeitei#1{{\meitei #1}\xspace}

Meithei (Meitei) /ˈmeɪteɪ/,[4] also known as Manipuri /mænɨˈpʊəri/ ({\pan মৈতৈলোন্} \textmeitei{ꯃꯧꯇꯧꯂꯣꯟ} Meitei-lon or {\pan মৈতৈলোল্} \textmeitei{ꯃꯧꯇꯧꯂꯣꯜ} Meitei-lol), is the predominant language and lingua franca in the southeastern Himalayan state of Manipur, in northeastern India. It is the official language in government offices. Meithei is also spoken in the Indian states of Assam and Tripura, and in Bangladesh and Burma (now Myanmar).

The Meitei (also Meetei, Meithei, Manipuri) people are the majority ethnic group of Manipur, a northeastern state of India. Meitei is an endonym or autonym while Manipuri is an exonym. A significant population of the Meitei also are settled in domestic neighboring states such as Assam[1] and Tripura. They have also settled in Bangladesh[2] and Myanmar.[3]
The Meitei people are made up of seven major clans known as Salai Taret.[4] Their written history has been documented to 1445 BC.[5]

Meithei is a Tibeto-Burman language whose exact classification remains unclear, though it shows lexical resemblances to Kuki and Tangkhul Naga.[5] The language is spoken by more than 1.5 million people.

\begin{figure}[htbp]
\centering
\includegraphics[width=\linewidth]{dancing}

\caption{"Khamba-Thoibi" Jagoi 
RKCS paintings on the walls of temple of Ibudhou Thangjing at Moirang, Manipur. 
Picture Courtesy - Recky Maibram.}
\end{figure}

Meithei has proven to be an integrating factor among all ethnic groups in Manipur who use it to communicate among themselves. It has been recognized (as Manipuri), by the Indian Union and has been included in the list of scheduled languages (included in the 8th schedule by the 71st amendment of the constitution in 1992). Meithei is taught as a subject up to the post-graduate level (Ph.D.) in universities of India, apart from being a medium of instruction up to the undergraduate level in Manipur.

\bgroup
\meitei
\begin{tabular}{>{\arial}l
                >{\arial}l
                >{\meitei}l
                >{\arial}l
                >{\arial}l
                >{\meitei}l
               }
1	&ama 	 &ꯑꯃ	       &11	&taramathoi	&\\
2	&ani	   &ꯑꯅꯤ	&12	 &taranithoi	&ky \\
3	&ahum	&ꯑꯍꯨꯝ	   &13	 &tarahumdoi	&ꯇꯔꯥꯍꯨꯝꯗꯣꯢ\\
4	&mari	&ꯃꯔꯤ	   &14  &	taramari	&ꯇꯔꯥꯃꯔꯤ\\
5	&manga	 &ꯃꯉꯥ	   &15	 &taramanga	&ꯇꯔꯥꯃꯉꯥ\\
6	&taruk	 &ꯇꯔꯨꯛ	   &16	 &tarataruk	&ꯇꯔꯥꯇꯔꯨꯛ\\
7	&taret	 &ꯇꯔꯦꯠ	   &17	 &tarataret	&ꯇꯔꯥꯇꯔꯦꯠ\\
8	&nipan &ꯅꯤꯄꯥꯟ	&18	 &taranipan	&ꯇꯔꯥꯅꯤꯄꯥꯟ\\
9	&mapan	 &ꯃꯥꯄꯟ	   &19	 &taramapan	&ꯇꯔꯥꯃꯥꯄꯟ\\
10	&tara	 &ꯇꯔꯥ	   &20	 &kun	&ꯀꯨꯟ\\
\end{tabular}
\egroup





Meitei Mayek script was added to the Unicode Standard in October, 2009 with the release of version 5.2.\index{Meitei Mayek}

The Unicode block for Meitei Mayek, called Meetei Mayek, is \unicodenumber{U+ABC0–U+ABFF}.

Characters for historical orthographies are part of the Meetei Mayek Extensions block at \unicodenumber{U+AAE0–U+AAFF}.

\begin{scriptexample}[]{Meitei}
\unicodetable{meitei}{"ABC0,"ABCD0,"ABE0,"ABF0}
\end{scriptexample}

\begin{scriptexample}[]{Meitei}
\unicodetable{meitei}{"AAE0,"AAF0}
\captionof{table}{Meetei Mayek Extensions}
\end{scriptexample}


\printunicodeblock{./languages/meetei-mayek.txt}{\meitei}


% http://e-pao.net/eyek/tamba/




%\section{Tibetan}
\label{tibetan}
\index{scripts>tibetan}


Another important Northern Indian member perhaps
derived directly from Gupta---and thus a sister script to Nagari,
Sarada and Pali---is Tibetan. However, the Tibetan
language wears this foreign Indo-Aryan script most uncomfortably.\cite{writing}
The script retains the Indic consonantal alphabet with diacritic
attachments to indicate vowels – but with only one vowel
letter, the /a/, which is the same as the system’s own ‘default’ /a/.
This /a/ letter is then used to attach other diacritics in order to
indicate further vowels. Because the Tibetan language has
changed greatly since c. AD 700 (when the script was first elaborated
from Gupta) while the script has remained almost
unchanged, Tibetan is extremely difficult to read today. Its
greatest problem is that it marks none of the tones of its tonal
language. Though Tibetans have long tried to adapt written
Tibetan to spoken Tibetan, high illiteracy has been the price of
failing to achieve this. Tibetan schools in Tibet, by governmental
decree, now teach only the Chinese script and in the Chinese
language.

\newfontfamily\tibetan{TibMachUni.ttf}

\newfontfamily\tibetan{Qomolangma-Chuyig.ttf}

%A should pick it up automatically \tibetan

Fonts described in this section can be obtained from The Tibetan \& Himalayan Library
\footnote{\url{http://www.thlib.org/tools/scripts/wiki/tibetan\%20machine\%20uni.html}  }

I have tried a few \texttt{Tibetan Machine Uni (TMU)} seems to be used by a number of scholars. 

A tip when you are trying to locate fonts is to find a related article in Wikipedia, such as Tibetan alphabet and inspect the element using your browser to see what fonts are being used.


|style="font-family:'Jomolhari','Tibetan Machine Uni','DDC Uchen', 'Kailash';| 


If you cannot see the script and rather than boxes or question marks then you can search and download one of the fonts in |font-family|.



\begin{docKey}[phd]{language}{ = tibetan}{default none, initial english} 
The key |language=tibetan| sets the default language as Tibetan, using the main font given by the key |tibetan font=TibMachUni.ttf|.

It will also create an environment tibetanlanguage.
\end{docKey}

\begin{docKey}[phd]{tibetan font}{= TibMachUni.ttf} {initial = TibMachUni.ttf} 
The key |tibetan font=font-name| sets the default font for the Tibetan language. It will also create the switch \cmd{\tibetan} for typesetting text in Tibetan.
\end{docKey}


\begin{docEnvironment}{tibetan}{}
\end{docEnvironment}

The environment is created automatically
\begin{texexample}{Tibetan language setttings}{ex:tibetan}
\bgroup
\cxset{language=tibetan, tibetan font = TibMachUni.ttf}

\tibetan Tibetan: དབུ་ཅན\par
ཨོཾ་ཨཿཧཱུྂ་བཛྲ་གུ་རུ་པདྨ་སིདྡྷི་ཧཱུྂ༔\par
\egroup

\begin{tibetanlanguage}
The tibetan environment\par
ཨོཾ་ཨཿཧཱུྂ་བཛྲ་གུ་རུ་པདྨ་སིདྡྷི་ཧཱུྂ༔
\end{tibetanlanguage}
\end{texexample}


The Tibetan alphabet is an \emph{abugida} of Indic origin used to write the Tibetan language as well as Dzongkha\footnote{Spoken in Bhutan.}, the Sikkimese language, Ladakhi, and sometimes Balti. 

The printed form of the alphabet is called \textit{uchen} script (Tibetan: དབུ་ཅན་, Wylie: dbu-can; "with a head") while the hand-written cursive form used in everyday writing is called umê script (Tibetan: དབུ་མེད་, Wylie: dbu-med; "headless").

The alphabet is very closely linked to a broad ethnic Tibetan identity. Besides Tibet, it has also been used for Tibetan languages in Bhutan, India, Nepal, and Pakistan.[1] The Tibetan alphabet is ancestral to the Limbu alphabet, the Lepcha alphabet,[2] and the multilingual 'Phags-pa script.[2]


The Tibetan alphabet is romanized in a variety of ways.[3] This article employs the Wylie transliteration system.

The Tibetan alphabet has thirty basic letters, sometimes known as "radicals", for consonants.[2]

{\tibetanfontfamily
ཀ ka /ká/	ཁ kha /kʰá/	ག ga /kà, kʰà/	ང nga /ŋà/\\
ཅ ca /tʃá/	ཆ cha /tʃʰá/	ཇ ja /tʃà/	ཉ nya /ɲà/\\
ཏ ta /tá/	ཐ tha /tʰá/	ད da /tà, tʰà/	ན na /nà/\\
པ pa /pá/	ཕ pha /pʰá/	བ ba /pà, pʰà/	མ ma /mà/\\
ཙ tsa /tsá/	ཚ tsha /tsʰá/	ཛ dza /tsà/	ཝ wa /wà/ (not originally part of the alphabet)[5]\\
ཞ zha /ʃà/[6]	ཟ za /sà/	འ 'a /hà/[7]\\
ཡ ya /jà/	ར ra /rà/	ལ la /là/\\
ཤ sha /ʃá/[6]	ས sa /sá/	ཧ ha /há/[8]\\
ཨ a /á/\\
}


Tibetan is not a difficult script to read or write, but it is a very complex script to deal with in terms of computer processing (as far as complexity goes I would rate it second only to the Mongolian script). The problem is that written Tibetan comprises complex syllable units (known in Tibetan as a tsheg bar {\tibetan ཚེག་བར}) which although written horizontally may include \emph{vertical} clusters of consonants and vowel signs agglutinating around a base consonant (a vertical cluster is known as a "stack"). 

Thus most words have a horizontal and a vertical dimension, with the result that text is not laid out in a straight line as in most scripts. For example, the word bsGrogs བསྒྲོགས་ (pronounced drok ... obviously!) may be analysed as follows :

\definecolor{lavenderblush}{HTML}{FFF0F5}%
\definecolor{beige}{HTML}{F5F5DC}%


{\tibetan 
\HUGE བསྒྲོགས

{\color{beige}%
\symbol{"0F56}\color{blue!40}\color{red}\symbol{"0F66}\symbol{"0F92}\color{blue!80}\symbol{"0FB2}\color{beige}\symbol{"0F7C}\color{blue!25}\symbol{"0F42}\symbol{"0F66}\symbol{"0F0B}}



\begin{tabular}{|l|}
\symbol{"0F56}\symbol{"0F7C}\\
\symbol{"0F42}\symbol{"0F7C}\\
\symbol{"0F66}\symbol{"0F7C}\\
\symbol{"0F40}\symbol{"0F7C}\\
\end{tabular}
}

\subsection{Unicode Block Tibetan}


\bgroup\large\tibetan
\begin{tabular}{llllllllllllllll l}
\toprule
	           &|0|	&|1|	&|2|	&|3|	&|4|	&|5|	&|6|	&|7|	&|8|	&|9|	&|A|	&|B|	&|C|	&|D|	&|E|	&|F|\\
\midrule
\texttt{U+0F0x}	&ༀ	&༁	&༂	&༃	&༄	&༅	&༆	&༇	&༈	&༉	&༊	&་	&༌  &	།	&༎	&༏\\
\midrule
\texttt{U+0F1x} &༐	&༑	&༒	&༓	&༔	&༕	&༖	&༗	&༘&	༙	&༚	&༛	&༜	&༝	&༞	&༟\\
\midrule
\texttt{U+0F2x} &༠	&༡	&༢	&༣	&༤	&༥	&༦	&༧	&༨	&༩	&༪	&༫	&༬	&༭	&༮	&༯\\
\midrule
\texttt{U+0F3x}	&༰ &༱	 &༲ &༳	&༴ &༵	&༶ & ༷	&༸&	༹	&༺&	༻	&༼&	༽	&༾	&༿\\
\midrule
\texttt{U+0F4x} &ཀ	&ཁ	&ག	&གྷ	&ང	&ཅ	&ཆ	&ཇ	&	&ཉ	&ཊ	&ཋ	&ཌ	&ཌྷ	&ཎ	&ཏ\\
\midrule
\texttt{U+0F5x}	 &ཐ	&ད	&དྷ	&ན	&པ	&ཕ	&བ	&བྷ	&མ	&ཙ	&ཚ	&ཛ	&ཛྷ	&ཝ	&ཞ	&ཟ\\
\midrule
\texttt{U+0F6x} &འ	&ཡ	&ར	&ལ	&ཤ	&ཥ	&ས	&ཧ	&ཨ	&ཀྵ	&ཪ	&ཫ	&ཬ	&&&\\
^^A\texttt{U+0F7x}&&	ཱ &	& &ི	ཱི&	ུ&	ཱུ&	ྲྀ&	ཷ&	ླྀ&	ཹ&	ེ&	ཻ&	ོ&	ཽ&	&ཾ	&ཿ\\
\midrule
\texttt{U+0F8x}&    ྀ   & 	ཱྀ&	ྂ&	&ྃ &	྄	&྅&	྆	&྇	ྈ&	ྉ&	ྊ&	ྋ&	ྌ&	ྍ&	ྎ&	ྏ\\
\midrule
\texttt{U+0F9x} &	ྐ&	ྑ   & 	ྒ &	ྒྷ &	ྔ &	ྕ &	ྖ &	ྗ &		ྙ &	ྚ &	ྛ &	ྜ &	ྜྷ &	ྞ &	ྟ\\
\texttt{U+0FAx} &	ྠ &	ྡ &	ྡྷ &	ྣ &	ྤ &	ྥ &		&ྦ	&ྦྷ	ྨ&	ྩ&	ྪ&	ྫ&	ྫྷ&	ྭ&	ྮ&	ྯ\\
\midrule
\texttt{U+0FBx} 
&	  ྰ 
&	
& ྱ  	 
&ྲ	
&ླ	
&ྴ
&	ྵ
&	ྶ
&	ྷ
&ྸ
&
&
&
&	
&྾	
&྿\\
\midrule
\texttt{U+0FCx}	 &࿀&	࿁&	࿂&	࿃&	࿄&	࿅&	&࿇	&࿈	&࿉	&࿊	&࿋	&࿌	&&	࿎	&࿏\\
\midrule
\texttt{U+0FDx}	&࿐	&࿑	&࿒	&࿓	&࿔	&࿕	&࿖	&࿗	&࿘	&࿙	&࿚	&&&&&\\
\midrule
\texttt{U+0FEx} &&&&&&&&&&&&&&&&\\
\midrule
\texttt{U+0FFx}  &&&&&&&&&&&&&&&&\\
\bottomrule
\end{tabular}
\egroup




\subsection{Fonts for Tibetan}

Fonts for Tibetan need to be downloaded one set of fonts are the \texttt{Qomolangma}. They come in different flavours, but they appear
to offer advantages as compared to the Tibetan Machine Uni.
\medskip


\newfontfamily\betsu{Qomolangma-Betsu.ttf}
\newfontfamily\drutsa{Qomolangma-Drutsa.ttf}
\newfontfamily\chuyig{Qomolangma-Chuyig.ttf}
\newfontfamily\tsumachu{Qomolangma-Tsumachu.ttf}
\newfontfamily\uchensutung{Qomolangma-UchenSutung.ttf}
\newfontfamily\uchensuring{Qomolangma-UchenSuring.ttf}
\newfontfamily\uchensarchen{Qomolangma-UchenSarchen.ttf}
\newfontfamily\uchensarchung{Qomolangma-UchenSarchung.ttf}
\newfontfamily\tsuring{Qomolangma-Tsuring.ttf}
\newfontfamily\TMU{TibMachUni.ttf}
\newfontfamily\himalaya{Microsoft Himalaya}


{
\centering

\renewcommand{\arraystretch}{1.5}

\begin{tabular}{lr}
\toprule
|Qomolangma-Betsu.ttf| & {\betsu  དབུ་མེད }\\
\midrule
|Qomolangma-Chuyig.ttf| &{\chuyig  དབུ་མེད}\\
\midrule
|Qomolangma-Drutsa.ttf| &{\drutsa  དབུ་མེད}\\
\midrule
|Qomolangma-Tsumachu.ttf|&{\tsumachu  དབུ་མེད}\\
\midrule
|Qomolangma-Tsuring.ttf| &{\tsuring  དབུ་མེད}\\
\midrule
|Qomolangma-UchenSarchen.ttf| &{\uchensarchen དབུ་མེད}\\
\midrule
|Qomolangma-UchenSarchung.ttf|&{\uchensarchung དབུ་མེད }\\
\midrule
|Qomolangma-UchenSuring.ttf|&{\uchensuring དབུ་མེད}\\
\midrule
|Qomolangma-UchenSutung.ttf|&{\uchensutung དབུ་མེད }\\
\midrule
|TibMachUni.ttf| &{\TMU དབུ་མེད }\\
\midrule
|Microsoft Himalaya| &{\himalaya དབུ་མེད ཽ}\\
\bottomrule
\end{tabular}

}
\bigskip

\bgroup
\LARGE\tsuring
\noindent༆ །ཨ་ཡིག་དཀར་མཛེས་ལས་འཁྲུངས་ཤེས་བློ  འི་\par
གཏེར༑ །ཕས་རྒོལ་ཝ་སྐྱེས་ཟིལ་གནོན་གདོང་ལྔ་བཞིན།།\par
ཆགས་ཐོགས་ཀུན་བྲལ་མཚུངས་མེད་འཇམ་དབྱངསམཐུས།།\par
མཧཱ་མཁས་པའི་གཙོ་བོ་ཉིད་འགྱུར་ཅིག། །མངྒལཾ༎\par
བསྒྲོགས
\egroup

\subsubsection{Tibetan numbers}
\cxset{language=tibetan, tibetan font = TibMachUni.ttf}

{
\obeylines
\small
TIBETAN DIGIT ZERO\tibetan	༠
TIBETAN DIGIT ONE	\tibetan༡	
TIBETAN DIGIT TWO\tibetan	༢	
TIBETAN DIGIT THREE\tibetan	༣	
TIBETAN DIGIT FOUR	\tibetan ༤	
TIBETAN DIGIT FIVE\tibetan	༥	
TIBETAN DIGIT SIX	\tibetan ༦	
TIBETAN DIGIT SEVEN\tibetan	༧	
TIBETAN DIGIT EIGHT\tibetan	༨	
TIBETAN DIGIT NINE\tibetan	༩	
TIBETAN DIGIT HALF ONE	\tibetan༪	
TIBETAN DIGIT HALF TWO	༫	
TIBETAN DIGIT HALF THREE	༬
TIBETAN DIGIT HALF FOUR ༭	
TIBETAN DIGIT HALF FIVE ༯	
TIBETAN DIGIT HALF SIX	 ༯	
TIBETAN DIGIT HALF SEVEN	༰	
TIBETAN DIGIT HALF EIGHT	༱	
TIBETAN DIGIT HALF NINE	༲	
TIBETAN DIGIT HALF ZERO	༳	
}


Tibetan numbers

The usage is not certain. By some interpretations, this has the value of 9.5. Used only in some traditional contexts, these appear as the last digit of a multidigit number, eg. ༤༬ represents 42.5. These are very rarely used, however, and other uses have been postulated.


\PrintUnicodeBlock{./languages/tibetan.txt}{\himalaya}


\section{Oriya}
\label{s:oriya}
\index{Indic scripts>Oriya}
\epigraph{Oṛiyā is encumbered with the drawback of an excessively awkward and cumbrous written character. ... At first glance, an Oṛiyā book seems to be all curves, and it takes a second look to notice that there is something inside each.}{(G. A. Grierson, \textit{Linguistic Survey of India}, 1903)}

\newfontfamily\oriya[Scale=1.1,Script=Oriya]{Noto Sans Oriya}

\def\oriyatext#1{{\oriya#1}}
The Oriya script or Utkala Lipi (Oriya: \oriyatext{ଉତ୍କଳ ଲିପି}) or Utkalakshara (Oriya: \oriyatext{ଉତ୍କଳାକ୍ଷର}) is used to write the Oriya language, and can be used for several other Indian languages, for example, Sanskrit.

\centerline{\Huge\oriyatext{ଉତ୍କଳ ଲିପି}}

\bgroup
\oriya
୦୧୨୩୪୫୬୭୮୯
ଅ ଆ ଇ ଈ ଉ ଊ ଋ ୠ ଌ ୡ ଏ ଐ ଓ ଔ କ ଖ ଗ ଘ ଙ ଚ ଛ ଜ ଝ ଞ ଟ ଠ ଡ ଢ ଣ ତ ଥ ଦ ଧ ନ ପ ଫ ବ ଵ ଭ ମ ଯ ର ଳ ୱ ଶ ଷ ସ ହ ୟ ଲ
\egroup






\begin{figure}[htbp]
\centering

\includegraphics[width=\linewidth-2\parindent]{oriya-people}

\hspace*{-1em}\caption{Children dressed for celebration of Janmashtami, which marks the birth of Lord Krishna. odisha360.com}
\end{figure}

Comparison of Oṛiyā script with its neighbours

At a first look the great number of signs with round shapes suggests a closer relation to the southern neighbour Telugu than to the other neighbours Bengali in the north and Devanāgarī in the west. The reason for the round shapes in Oriya and Telugu (and also in Kannaḍa and Malayāḷam) is the former method of writing using a stylus to scratch the signs into a palm leaf. These tools do not allow for horizontal strokes because that would damage the leaf.

Oriya letters are mostly round shaped whereas in Devanāgarī and Bengali have horizontal lines. So in most cases the reader of Oṛiyā will find the distinctive parts of a letter only below the hoop. Considering this the  closer relation to Devanāgarī and Bengali exists than to any southern script, though both northern and southern scripts have the same origin, Brāhmī.

Oriya (\oriyatext{ଓଡ଼ିଆ} oṛiā), officially spelled Odia,[3][4] is an Indian language belonging to the Indo-Aryan branch of the Indo-European language family. It is the predominant language of the Indian states of Odisha, where native speakers comprise 80\% of the population,[5] and it is spoken in parts of West Bengal, Jharkhand, Chhattisgarh and Andhra Pradesh. Oriya is one of the many official languages in India; it is the official language of Odisha and the second official language of Jharkhand. [6][7][8] Oriya is the sixth Indian language to be designated a Classical Language in India, on the basis of having a long literary history and not having borrowed extensively from other languages.



\printunicodeblock{./languages/oriya.txt}{\oriya}

\section{Numerals}

{\oriya
\obeylines
୦	୧	୨	୩	୪	୫	୬	୭	୮	୯	୵	୶	୷	୲	୳	୴
{\arial 0	1	2	3	4	5	6	7	8	9	¹⁄₁₆	⅛	³⁄₁₆	¼	½	¾}
}




\section{Mro (Mru language)}
\label{s:mro}

\newfontfamily\mro{MroUnicode-Regular.ttf}
\def\textmro#1{{\mro #1\xspace}}

 Mro (or Mru) is a Tibeto-Burman language spoken primarily in Bangladesh with a few
speakers in India. 

Mru is a Tibeto-Burman language and one of the recognized languages of Bangladesh. It is spoken by a community of Mros (Mru) inhabiting the Chittagong Hill Tracts of Bangladesh and also in Burma with a population of 22,000 in Bangladesh according to the 1991 census. The Mros are the second-largest tribal group in Bandarban District of the Chittagong Hill Tracts. A small group of Mros also live in Rangamati Hill District.

The Mru language is considered "definitely endangered" by UNESCO in June 2010.[4]

The script was invented in the 1980s and is of the class of “messianic”
scripts with no genetic relationship with existing scripts. In the last 10 years there has been an acceptance
among all the Mro to use this script and literacy levels among the 100,000 Mro exceed 80\%.

Some of the characters of the Mro alphabet have a visual similarity to those from other alphabets, but this
relationship is purely coincidental, and the Mro alphabet stands alone as a unity.


The Mro script has no technical complexity: it is a simple left to right alphabet with no
combining characters or characters with special function. There are no tone marks. Some sounds are
represented by more than one letter. The sound [k] is usually represented by \textmro{𖩌} KEAAE kəɘ, as in \textmro{𖩌𖩑𖩗} kow
‘village’, \textmro{𖩄𖩑𖩁𖩌𖩑} boŋko ‘owl’, but in a few words the letter 𖩙 KOO ko is used, as in \textmro{𖩙𖩑} ko ‘gold’. The sound
[m] is usually represented by \textmro{𖩎} MAEM mɘm, as in \textmro{𖩎𖩆𖩁} maŋ ‘go’, \textmro{𖩔𖩎𖩑} śmo ‘fool’, but in a few words the
letter \textmro{𖩃} MIM mim is used, as in \textmro{𖩃𖩊𖩏} min ‘cat’, \textmro{𖩋𖩃𖩊} cmi ‘rice’. The sound [l] is usually represented by \textmro{𖩍} OL
\textmro{ɔl}, as in \textmro{𖩍𖩝𖩁} lɔŋ ‘boat’, \textmro{𖩈𖩍𖩆} khla ‘spoon’, but in a few words the letter \textmro{𖩛} LA la is used, as in \textmro{𖩛𖩆𖩎𖩖} lamɘ
‘moon’, and in a few words \textmro{𖩚} LAN lan is used (we have no example). The vowels \textmro{𖩑𖩖} oɘ are used as a
digraph to describe the vowel [ø].

We are using Philip Reimer's font which is freely available under SIL OFL licence at \href{http://phjamr.github.io/mro.html}{github}. Philip has also produced fonts for two other scripts: Lisu (Fraser) and Miao (Pollard). All three scripts were added to Unicode 7.0 in 2014.



\begin{scriptexample}[]{Mro}
\unicodetable{mro}{"16A40,"16A50,"16A60}
\end{scriptexample}


\printunicodeblock{./languages/mro.txt}{\mro}








\section{Devanagari}
\label{s:devanagari}
\parindent1em

Devanagari is part of the Brahmic family of scripts of India, Nepal, Tibet, and South-East Asia.[2] It is a descendant of the Gupta script, along with Siddham and Sharada.[2] Eastern variants of Gupta called nāgarī are first attested from the 7th century CE; from c. 1200 CE these gradually replaced Siddham, which survived as a vehicle for Tantric Buddhism in East Asia, and Sharada, which remained in parallel use in Kashmir. An early version of Devanagari is visible in the Kutila inscription of Bareilly dated to Vikram Samvat 1049 (i.e. 992 CE), which demonstrates the emergence of the horizontal bar to group letters belonging to a word.[3]

Sanskrit nāgarī is the feminine of nāgara \enquote{relating or belonging to a town or city}. It is feminine from its original phrasing with lipi ("script") as nāgarī lipi "script relating to a city", that is, probably from its having originated in some city.[4]

The use of the name devanāgarī is relatively recent, and the older term nāgarī is still common.[2] The rapid spread of the term Devanāgarī may be related to the almost exclusive use of this script to publish Sanskrit texts in print since the 1870s.[2]

In time, Devanagari became India’s principal script. It also
became one of the world’s most important, as it was used to
convey many other languages of the region, such as Hindi, Nepali, Marwari, 
Kumaoni and several non-Indo-Aryan
languages. Devanagari failed to become India’s sole script perhaps
because of the region’s long disunity. Subsequently, it
became the parent of, among other scripts, the Gurmukhi
which the Sikhs elaborated in the 1500s in order to write their
Punjabi language (illus. 78). Today, Devanagari survives in India
alongside some ten other major scripts (including the Latin and
Perso-Arabic alphabets) and about 190 others of lesser significance.\cite{writing}

On Windows use \texttt{Arial Unicode MS} or \texttt{Arial}
\medskip

%\newfontfamily\devanagari[Script=Devanagari,Scale=1.5]{Arial Unicode MS}
\newfontfamily\devanagarilohit[Script=Devanagari,Scale=1.1]{Lohit-Devanagari.ttf}
\let\devanagari\devanagarilohit

\begin{scriptexample}[]{Devanagari}
{\begin{center}\parindent0pt\devanagari

ंःअआइईउऊऋऌऍऎएऐऑऒओऔऔँ \par 

ी	ु	ू	ृ	ॄ	ॅ	ॆ	े	ै	ॉ	ॊ	ो	ौ	्	\par

\bigskip		
\begin{tabular}{lll lll lll l}
०	&१	&२	&३	&४	&५	&६	&७	&८	&९\\
0	&1	&2	&3	&4	&5	&6	&7	&8	&9\\
\end{tabular}
\end{center}	
}
\end{scriptexample}


On Linux \texttt{Lohit} is a font family designed to cover Indic scripts and released by Red Hat. The Lohit fonts currently cover 11 languages: Assamese, Bengali, Gujarati, Hindi, Kannada, Malayalam, Marathi, Oriya, Punjabi, Tamil, Telugu.[1] The fonts were supplied by Modular Infotech and licensed under the GPL. In September 2011, they were retroactively relicensed under the OFL.[2] The Lohit fonts are used as web fonts by some Wikimedia Foundation sites, like Wikipedia, since March 2012.The font currently support 21 Indian languages. 

\let\devanagarilohit\pan

\begin{scriptexample}[]{Devanagari}
\begin{center}\parindent0pt\devanagarilohit

ंःअआइईउऊऋऌऍऎएऐऑऒओऔऔँ \par 

ी	ु	ू	ृ	ॄ	ॅ	ॆ	े	ै	ॉ	ॊ	ो	ौ	्	\par

\bigskip		
\begin{tabular}{lll lll lll l}
०	&१	&२	&३	&४	&५	&६	&७	&८	&९\\
0	&1	&2	&3	&4	&5	&6	&7	&8	&9\\
\end{tabular}
\end{center}
\end{scriptexample}

\subsubsection{Punctuation} 
The end of a sentence or half-verse may be marked with a dot known as a pūrna virām or a vertical line danda: \textbar. The end of a full verse may be marked with two vertical lines: \textbar\textbar. A comma, or alpa virām, is used to denote a natural pause in speech. With expansion of English speakers in India, the full stop is also sometimes used.

\subsection{LaTeX support}

\latex2e support can be found in the \pkgname{sanskrit}. The package contains the font files and pre-processor for printing Sanskrit
text in both devanāgarī and transliterated Roman with diacritics. Another package that can be used with \XeTeX\ is support \pkgname{devnag}.  This was originally developed by Frans Velthuis for the University of Groningen, The Netherlands, and it was the first system to provide
support for the script for \tex. The package was  extended by Anshuman Pandey. The package provides both fonts as well as tranliteration macros.


\printunicodeblock{./languages/devanagari.txt}{\devanagarilohit}






\section{Bengali}
\label{s:bengali}
\idxlanguage{Bengali}
\index{Bengali fonts>Shonar Bangla}
\index{Bengali fonts>Vrinda}
\index{Bengali fonts>Noto Sans Bengali}
\index{Bengali fonts>Noto Serif Bengali}
\index{Bengali}
%\newfontfamily\bengali[Script=Bengali,Scale=1.3]{Shonar Bangla}
\newfontfamily\bengali[Script=Bengali,Scale=1.0]{Noto Serif Bengali}
There are two Windows fonts that can be used with Windows \textit{Shonar Bangla} and \textit{Vrinda}. For open source fonts one can use, \texttt{Not Serif Bengali}.

\docAuxCommand{bengali} and \docAuxCommand{textbengali} Once the key is set the command \cmd{\bengali} is available for use in typesetting Bengali text.

\bigskip

\bgroup



\bengali
\centering

  অ  আ ই  ঈ  উ  ঊ  ঋ  এ  ঐ\par

%\newfontfamily\bengal[Script=Bengali,Scale=3.2]{Vrinda}

\centering

  অ  আ ই  ঈ  উ  ঊ  ঋ  এ  ঐ\par




\centering

  অ  আ ই  ঈ  উ  ঊ  ঋ  এ  ঐ\par

\captionof{table}{The consonant{\protect\bengali{} ক (kô)} along with the diacritic form of the vowels {\protect\bengali{} অ, আ, ই, ঈ, উ, ঊ, ঋ, এ, ঐ, ও and ঔ} \textit{from Wikipedia}.}
\egroup

\def\indexindic#1{\index{Indic Languages>#1}\index{#1} }

|Bengali| is a Unicode block containing characters for the Bangla, Assamese, Bishnupriya Manipuri, Daphla, Garo, Hallam, Khasi, Mizo, Munda, Naga, Rian, and Santali languages. In its original incarnation, the code points U+0981..U+09CD were a direct copy of the Bengali characters A1-ED from the 1988 ISCII standard, as well as several Assamese ISCII characters in the U+09F0 column. The Devanagari, Gurmukhi, Gujarati, Oriya, Tamil, Telugu, Kannada, and Malayalam blocks were similarly all based on ISCII encodings. \index{Bengali}\index{Indic Languages>Bangla}\index{Indic Languages>Assamese}\index{Indic Languages>Bishnupriya Manipuri}\indexindic{Daphla}\indexindic{Garo}\indexindic{Hallam}\indexindic{Khasi}\indexindic{Mizo}\indexindic{Munda}
\indexindic{Naga}\indexindic{Rian}\indexindic{Santali}

\begin{scriptexample}[]{Bengal}
\unicodetable{bengali}{"0980,"0990,"09A0,"09B0,"09C0,"09D0,"09E0,"09F0}
\end{scriptexample}


\printunicodeblock{./languages/bengali.txt}{\bengali}



\bgroup
\bengali\LARGE
\char"0995 + \color{blue} \char"09BC + \color{red}\char"09AF  = \char"0995\char"09CD \char"09AF
\egroup

Noto has both a serif and a sans font \docFont{Noto Serif Bengali}

See also \url{http://www.nongnu.org/freebangfont/downloads.html} for additional fonts.










\section{Saurashtra}
\label{s:saurashtra}
\idxlanguage{Saurashtra}\idxlanguage{Sourashtra}

\index{Saurashtra fonts>code2000}
\newfontfamily\saurashtra{code2000.ttf}
\def\test{}
\cxset{saurashtra font/.code=\test}
\cxset{saurashtra font=code2000.ttf}

\begin{docKey}[phd]{saurashtra font}{ = \meta{fontname}} {default none, initial = code2000}
  This key sets the saurashtra font.
\end{docKey}

Saurashtra or Sourashtra or {\saurashtra ꢱꣃꢬꢵꢰ꣄ꢜ꣄ꢬꢵ} or Palkar or Patkar (Sanskrit: सौराष्ट्र, Tamil: சௌராட்டிரம்) is an Indo-Aryan language[3] spoken by the Saurashtrian community native to Gujarat, who migrated and settled in Southern India. Madurai in Tamil Nadu has the highest number of people belonging to this community and also remains as their cultural center.

The language is largely only in spoken form even though the language has its own script. The lack of schools teaching Saurashtra script and the language is often cited as a reason for the very few number of people who actually know to read and write in Saurashtra script. Latin, Devanagari or Tamil script is used as alternative for Saurashtra Script by many Saurashtrians.

Census of India places the language under Gujarati. Official figures show the number of speakers as 185,420 (2001 census).[4]


\begin{scriptexample}[]{Saurashtra}
\unicodetable{saurashtra}{"A880,"A890,"A8A0,"A8B0,"A8C0,"A8D0}
\end{scriptexample}


\begin{scriptexample}[]{Saurashtra}
\bgroup
\saurashtra

ꢮꢶꢯ꣄ꢮ ꢱꣃꢬꢵꢰ꣄ꢜ꣄ꢬꢪ꣄ ꢦꢡ꣄ꢬꢶꢒꢾ ꢱꢵꢡ꣄ꢡꢒꢸ ꢂꢮꢬꢾ
ꢮꣁꢭꢱ꣄ꢢꢵꢥꢪꢸꢒ꣄(ꣀꢵꢮꢾꢔꢹ ꢂꢮ꣄ꢬꢶꢫꣁ


\arial

Text: Vishwa Sourashtram \url{http://www.sourashtra.info/ghEr.htm}
\egroup
\end{scriptexample}


\printunicodeblock{./languages/saurashtra.txt}{\saurashtra}

\section{Gujarati}
\label{s:gujarati}
\idxlanguage{Gujarati}
\index{Unicode>Gujarati}
%FIXME
\index{languages>Gujarati}\index{languages>Gujǎrātī Lipi}
has its own writing system, distinct but related to several other Indian languages' writing systems, such as the one used to write Hindi. Strictly speaking, the Gujarati writing system is what is called an \emph{abugida} (and not an \textit{alphabet}), because the consonant characters all contain an inherent vowel, and other vowels are written as accents added on to the consonant characters. There are also symbols for stand-alone vowels.

The Gujarati script ({\gujarati{ગુજરાતી લિપિ }} Gujǎrātī Lipi), which like all Nāgarī writing systems is strictly speaking an abugida rather than an alphabet, is used to write the Gujarati and Kutchi languages. It is a variant of Devanāgarī script differentiated by the loss of the characteristic horizontal line running above the letters and by a small number of modifications in the remaining characters.
With a few additional characters, added for this purpose, the Gujarati script is also often used to write Sanskrit and Hindi.
Gujarati numerical digits are also different from their Devanagari counterparts.
\medskip

\bgroup
\newfontfamily\gujaratilohit[Script=Gujarati,Scale=1.5]{Lohit-Gujarati.ttf}
\gujarati

\centering

\underline{English/Hindi/Gujarati Alphabets}

\hskip-1.5cm\begin{tabular}{lllllllllllllllllllll}
A &B &bh &C &ch &chh &D &dh &E &F &G &gh &H &I &J &K &kh &L &M &N &O\\

अ &ब &भ &क &च &छ &ड/द &ध/ढ़ &इ &फ &ग &घ &ह &ई &ज &क &ख &ल &म &न/ण &ऑ\\

અ &બ &ભ &ક &ચ &છ &ડ/દ &ધ /ઢ &ઇ &ફ &ગ &ઘ &હ &ઈ &જ &ક &ખ &લ &મ &ન/ણ &ઓ\\

\end{tabular}
\egroup

\medskip

Gujarati has its own set of numeric signs (placed alongside their Hindu-Arabic [or Indo-Arabic] counterparts in the tables below), they are employed in much the same way as English;  that is to say, they are put together in the same manner in order to express larger numbers. It is quite possible to simply substitute the Gujarati numerals for the Hindu-Arabic ones.

The Gujarati words for 1-10 are as follows:
\medskip

\bgroup
\begin{center}
\gujarati
\begin{tabular}{ccl}
Arabic & Gujarati &Name\\
Numeral &Numeral  &\\
0	&૦	&mīṇḍuṃ or shunya\\
1	&૧	&ekaṛo or ek\\
2	&૨	&bagaṛo or bay\\
3	&૩	&tragaṛo or tran\\
4	&૪	&chogaṛo or chaar\\
5	&૫	&pāchaṛo or paanch\\
6	&૬	&chagaṛo or chah\\
7	&૭	&sātaṛo or sāt\\
8	&૮	&āṭhaṛo or āanth\\
9	&૯	&navaṛo or nav\\
10 &૧૦ &દસ das\\

\end{tabular}
\end{center}
\egroup

\chapter{Tamil}

\epigraph{Women live like bats or owls.\\Labour like beasts\\and die like worms\ldots}{Margaret of Newcastle, 1660, England}



\label{s:tamil}
\newfontfamily\tamil[Scale=1.0, Script=Tamil]{code2000.ttf}

\def\tamiltext#1{{\tamil#1}}

\section{Background and History}

Of all the Dravidian languages Tamil has the longest literary tradition, covering
more than two thousand years. The earliest records are cave inscriptions from
the second century \textsc{bce}; the earliest extant literary text is the grammar
Tolkāppiyam (100 \textsc{bce}), which describes the grammar and poetics of Tamil during
that period. The dating of the Tolkāppiyam is still disputed by scholars proposing dates from
5 \textsc{bce} to 600 \textsc{ce}. 

During its two-thousand-year uninterrupted history, Tamil distinguishes
three different stages: Old Tamil (300 \textsc{bce} to 700 \textsc{ce}), Middle Tamil (700
\textsc{ce} to 1600) and Modern Tamil (1600 \textsc{ce} to the present), each with distinct
grammatical characteristics.\index{Dravidian>Tamil}\index{Tamil}


\begin{figure}[htbp]
\bgroup
\parindent=0pt
\centering
\includegraphics[width=0.9\linewidth-2\parindent]{./images/old-tamil-inscription.jpg}

\caption{Mangulam Tamil Brahmi inscription at Dakshin Chithra, Chennai (wikipedia)}

\egroup
\end{figure}

The Tamil script (\tamiltext{தமிழ் அரிச்சுவடி} tamiḻ ariccuvaṭi) is an abugida script that is used by the Tamil people in India, Sri Lanka, Malaysia and elsewhere, to write the Tamil language, as well as to write the liturgical language Sanskrit, using consonants and diacritics not represented in the Tamil alphabet. Certain minority languages such as Saurashtra, Badaga, Irula, and Paniya are also written in the Tamil script. \index{Surashtra}\index{Badaga}\index{Irula}

The Tamil script has 12 vowels (\tamiltext{உயிரெழுத்து} uyireḻuttu ``soul-letters''), 18 consonants (\tamiltext{மெய்யெழுத்து} meyyeḻuttu ``body-letters").
An additional character, the āytam \tamiltext{ஃ (ஆய்தம்)},  is classified in Tamil grammar as being neither a consonant nor a vowel (\tamiltext{அலியெழுத்து} aliyeḻuttu ``the hermaphrodite letter''), though often considered as part of the vowel set (\tamiltext{உயிரெழுத்துக்கள்} uyireḻuttukkaḷ ``vowel class''). The script, however, is syllabic and not alphabetic. The complete script, therefore, consists of the thirty-one letters in their independent form, and an additional 216 combinant letters representing a total 247 combinations (\tamiltext{உயிர்மெய்யெழுத்து} uyirmeyyeḻuttu) of a consonant and a vowel, a mute consonant, or a vowel alone. These combinant letters are formed by adding a vowel marker to the consonant. Some vowels require the basic shape of the consonant to be altered in a way that is specific to that vowel. Others are written by adding a vowel-specific suffix to the consonant, yet others a prefix, and finally some vowels require adding both a prefix and a suffix to the consonant. In every case the vowel marker is different from the standalone character for the vowel.
The Tamil script is written from left to right.\index{hermaphrodite letter}


\section{Unicode}

Tamil is a Unicode block containing characters for the Tamil, Badaga, and Saurashtra languages of Tamil Nadu India, Sri Lanka, Singapore, and Malaysia. In its original incarnation, the code points U+0B02..U+0BCD were a direct copy of the Tamil characters A2-ED from the 1988 ISCII standard. The Devanagari, Bengali, Gurmukhi, Gujarati, Oriya, Telugu, Kannada, and Malayalam blocks were similarly all based on their ISCII encodings.

\begin{scriptexample}[]{Tamil}
\unicodetable{tamil}{"0B80,"0B90,"0BA0,"0BB0,"0BC0,"0BE0,"0BF0}

\hfill  Typeset with \cmd{\tamil} and \texttt{code2000.ttf}
\end{scriptexample}

\subsection{Tamil Numbers and Numerals}

Originally, Tamils did not use zero, nor did they use positional digits (having separate 
symbols for the numbers 10, 100 and 1000). Symbols for the numbers are similar to 
other Tamil letters, with some minor changes. 

For example, the number 3782 is not written as \tamiltext{௩௭௮௨} as in modern usage. Instead it 
is written as \tamiltext{௩ ௲ ௭ ௱ ௮ ௰ ௨}. This would be read as they are written as 
Three Thousands, Seven Hundreds, Eight Tens, Two; or in Tamil as 
\tamiltext{௩௲௭௱௮௰௨ž}.\footnote{https://cloud.github.com/downloads/raaman/Tamil-Numeral/tamilnumbers.html}

\subsection{Dates}

Once the script is loaded the day, month and year can be loaded using the command  \cmd{\tamildate}, which returns the |\today| formatted as per custom Tamil. 

\begin{center}
\bgroup
\tamil
\begin{tabular}{lll}
day	 &month	&year	\\

௳	&௴	      &௵	\\

u	&mee	      &wa	\\
\end{tabular}
\egroup
\end{center}


\section{Grantha}
\label{s:grantha}

Grantha is a Unicode block containing the ancient Grantha script characters of 6th to 19th century Tamil Nadu and Kerala for writing Sanskrit and Manipravalam. Battled to get it working, as I could not find an appropriate unicode font. The font would need remapping. Unfortunately this is a script with no Noto support.

\begin{figure}[htbp]
\bgroup
\parindent=0pt
\centering
\includegraphics[width=\linewidth]{./images/grantha.jpg}

\caption{An image of a palm leaf manuscript with Sanskrit written in Grantha script (wikipedia)}

\egroup
\end{figure}

\newfontfamily\grantha{e-Grantamil 7}%e-Grantamil 7

\begin{scriptexample}[\grantha]{Tamil}
\unicodetable{grantha}{"0D0,"0D1,"0D2,"1133,"1134,"1135,"1136,"1137}

\hfill  Typeset with \cmd{\grantha} and \texttt{e-Granthamil 7.ttf}
\end{scriptexample}

{
\grantha \char"11311

}

%\newfontfamily\freeserif{FreeSerif}
%
%
%\freeserif \lorem
%\begin{tabular}{lll}
%day	 &month	&year	\\
%
%௳	&௴	      &௵	\\
%
%u	&mee	      &wa	\\
%\end{tabular}



\section{Malayalam}
\label{s:malayalam}
\newfontfamily\malayam[Scale=1.1]{Lohit-Malayalam.ttf}

\def\malamtext#1{{\malayam#1}}


Malayalam is a language spoken by the native people of southwestern India (from Thuckalay to Talapady).According to the Indian census of 2011, there were 32,299,239 speakers of Malayalam in Kerala, making up 93.2\% of the total number of Malayalam speakers in India, and 96.74\% of the total population of the state. There were a further 701,673 (2.1\% of the total number) in Karnataka, 957,705 (2.7\%) in Tamil Nadu, and 406,358 (1.2\%) in Maharashtra. The number of Malayalam speakers in Lakshadweep is 51,100, which is only 0.15\% of the total number, but is as much as about 84\% of the population of Lakshadweep. In all, Malayalis made up 3.22\% of the total Indian population in 2011. Of the total 34,713,130 Malayalam speakers in India in 2011, 33,015,420 spoke the standard dialects, 19,643 spoke the Yerava dialect and 31,329 spoke non-standard regional variations like Eranadan.[37] As per the 1991 census data, 28.85\% of all Malayalam speakers in India spoke a second language and 19.64\% of the total knew three or more languages.


\includegraphics[width=\textwidth]{nangeli}
https://feminisminindia.com/2016/09/12/kerala-breast-tax-nangeli/


Large numbers of Malayalis have settled in Delhi, Bangalore, Hyderabad, Mumbai (Bombay), Pune and Chennai (Madras). A large number of Malayalis have also emigrated to the Middle East, the United States, and Europe. There were 179,860 speakers of Malayalam in the United States, according to the 2000 census, with the highest concentrations in Bergen County, New Jersey and Rockland County, New York.[38] There were 7,093 Malayalam speakers in Australia in 2006.[39] The 2001 Canadian census reported 7,070 people who listed Malayalam as their mother tongue, mainly in Toronto. The 2006 New Zealand census reported 2,139 speakers.[40] 134 Malayalam speaking households were reported in 1956 in Fiji. There is also a considerable Malayali population in the Persian Gulf regions, especially in Dubai and Doha. Recently a Keralite is elected as mayor in Loughten town of England.

The Malayalam script (Malayalam: \malamtext{മലയാളലിപി}, Malayāḷalipi, IPA: [mɐləjaːɭɐ lɪβɪ], also known as Kairali script (Malayalam: \malamtext{കൈരളീലിപി}), is a Brahmic script used commonly to write the Malayalam language—which is the principal language of the Indian state of Kerala, spoken by 35 million people in the world.[3] Like many other Indic scripts, it is an alphasyllabary (\textit{abugida}), a writing system that is partially “alphabetic” and partially syllable-based. The modern Malayalam alphabet has 15 vowel letters, 41 consonant letters, and a few other symbols. The Malayalam script is a Vattezhuttu script, which had been extended with Grantha script symbols to represent Indo-Aryan loanwords.[4] The script is also used to write several minority languages such as Paniya, Betta Kurumba, and Ravula.[5] The Malayalam language itself was historically written in several different scripts.

\begin{scriptexample}[]{Malayalam}
\centerline{\Huge\malamtext{കൈരളീലിപി}}
\end{scriptexample}
\section{Syloti Nagri}
\label{s:sylotinagri}
\newfontfamily\syloti{NotoSansSylotiNagri-Regular.ttf}
\newfontfamily\damase{damase_v.2.ttf}
\index{languages>Sylheti Nagari}

Sylheti or Syloti (i.e. "Silēṭī" Bengali: সিলেটী or "Silôṭī" Bengali: ছিলটী) is one of the Bengali dialects, primarily spoken in the Sylhet Division of northeast Bangladeshi district Moulvibazar,Sylhet,Sunamganj,Hobiganj and the Barak Valley region of southern Assam. (Although sometimes it is considered an independent language for not sharing grammatical mutual intelligibility), it is a similar language to Standard Bengali, with which it shares a high proportion of vocabulary: Spratt and Spratt (1987) report 70\% shared vocabulary, while Chalmers (1996) reports at least 80\% overlap.

Sylheti Nagari or Syloti Nagri (Silôṭi Nagôri) is the original script used for writing the Sylheti language. It is an almost extinct script, this is because the Sylheti Language itself was reduced to only dialect status after Bangladesh gained independence and because it did not make sense for a dialect to have its own script, its use was heavily discouraged. The government of the newly formed Bangladesh did so to promote a greater "Bengali" identity. This led to the informal adoption of the Eastern Nagari script also used for Bengali and Assamese. It is also known as Jalalabadi Nagri, Mosolmani Nagri, Ful Nagri etc.

Sylheti Nagari was added to the Unicode Standard in March, 2005 with the release of version 4.1.
The Unicode block for Sylheti Nagari is U+A800–U+A82F:

\begin{scriptexample}[]{Sylheti}
\unicodetable{damase}{"A800,"A810,"A820}
\end{scriptexample}


\printunicodeblock{./languages/syloti.txt}{\damase}



\section{Limbu}
\label{s:limbu}

The Limbu script is used to write the Limbu language. The Limbu script is an abugida derived from the Tibetan script. Limbu is a Tibeto-Burman language spoken mainly in Nepal,[3] significant communities in Bhutan, Sikkim, Darjeeling district, India by the Limbu community. Virtually all Limbus are bilingual in Nepali.

\newfontfamily\limbu{code2000.ttf}

According to traditional histories, the Limbu script was first invented in the late 9th century by King Sirijonga Haang, then fell out of use, to be reintroduced in the 18th century by Te-ongsi Sirijunga Xin Thebe.

To change the inherent vowel, a diacritic is added. Shown here on /k/ ({\limbu ᤁ}):
Appearance	IPA

\begin{table}[htb]
\centering
\begin{tabular}{>{\Large\bfseries\limbu}l>{\arial}l}
ᤁᤡ	&/ki/\\
ᤁᤣ	&/ke/\\
ᤁᤧ	&/kɛ/\\
ᤁᤠ	&/ka/\\
ᤁᤨ	&/kɔ/\\
ᤁᤥ	&/ko/\\
ᤁᤢ	&/ku/\\
ᤁᤤ	&/kai/\\
ᤁᤦ	&/kau/\\
\end{tabular}
\caption{Changing the inherent vowel, using a diacritic.}
\end{table}




\begin{scriptexample}[]{Limbu}
\unicodetable{limbu}{"1900,"1910,"1920,"1930,"1940}
\end{scriptexample}


\printunicodeblock{./languages/limbu.txt}{\limbu}



\section{Cham}
\label{s:cham}

The Cham alphabet is an abugida used to write Cham, an Austronesian language spoken by some 230,000 Cham people in Vietnam and Cambodia. It is written horizontally left to right, as is English.

\newfontfamily\cham{Noto Sans Cham}

Cham is a Unicode block containing characters for writing the Cham language, primarily used for the Eastern dialect in Cambodia.
Cham script was added to the Unicode Standard in April, 2008 with the release of version 5.1.
The Unicode block for Cham is \textsc{U+AA00–U+AA5F}:

\begin{scriptexample}[]{Cham}
\unicodetable{cham}{"AA10,"AA20,"AA30,"AA40,"AA50}
\end{scriptexample}


\printunicodeblock{./languages/cham.txt}{\cham}


\section{Sora Sompeng}
\label{s:sorasompeng}
The Sora Language is part of the Austroasiatic language family. More locally, however, it is a part of the \hyperref[s:munda]{Munda} Languages which include other tribal languages in close proximity to Sora. Sora is unique because although it is surrounded by the Indo-Aryan language \hyperref[s:oriya]{Oriya} and the Dravidian Language Telugu, Sora is more closely related to the languages of Southeast Asia such as Khmer in Cambodia than it is to the predominant languages of India. Moreover, Sora contains very little formal literature but has an abundance of folk tales and traditions. Most of their passed down knowledge is of the oral tradition. Compared to other languages in the Munda family, Sora is decreasing within the Sora tribe at a faster rate. Most speakers are concentrated in Odisha and Andhra Pradesh but smaller communities also exist in Madhya Pradesh, Tamil Nadu, and Bihar.

Sorang Sompeng script is used to write in Sora, a Munda language with 300,000 speakers in India. The script was created by Mangei Gomango in 1936 and is used in religious contexts.[1] He was familiar with Oriya, Telugu and English, so the parent systems of the script are Brahmi and Latin.[2]
The Sora language is also written in the Latin alphabet and the Telugu script.

Sorang Sompeng script was added to the Unicode Standard in January, 2012 with the release of version 6.1. In Windows Nirmala UI.ttf (Windows 10.0) can be used. 

\newfontfamily\NirmalaU{Nirmala UI}


\unicodetable{NirmalaU}{"110D0,"110E0,"110F0}
 	

The Sora Bible employs a Latin-based orthography with a number of Sora-specific
conventions. Sora has also been rendered in the Oriya script in Orissa and in
the Telugu script in Andhra Pradesh, as well as a modified phonetic alphabet in
Ramamurti’s grammatical materials and dictionary. The use of and knowledge of
Sorang Sompeng (N. Zide 1996), the indigenous script, appears to be quite limited.
In many areas Sora remains a vital and thriving language, but one that has no state
or institutional support (sermons and materials are increasingly in Oriya in Gajapati
district which we observed and were told about in Christianized Sora communities).
In other areas, Sora is reportedly being or indeed has already been replaced by Telugu
or Oriya. So, although not an endangered language in sensu stricto, Sora (except in


\section{Numerals}\label{s:munda}

The Sora are unique in their numeral system. Instead of base 10, Sora uses a base 12 system. Only a few other languages in the world share this anomaly. Ekari, for example, uses a base 60 system.[7] For example, 39 in Sora arithmetic would be thought of as (1 * 20)+ 12 + 7. Here are the first 12 numerals in the Sora language :[7]
English: one two three four five six seven eight nine ten eleven twelve

Sora: aboy bago yagi unji monloy tudru gulji thamji tinji gelji gelmuy migel
Similar to how English uses the suffix from the numeral ten after twelve (such as thirteen, fourteen, etc.), Sora also uses a suffix assignment to numerals after 12 and before 20. Thirteen in Sora is expressed as migelboy (12+1), fourteen as migelbagu (12+2), etc.[7] Between numerals 20 and 99, Sora adds the suffix kuri to the first constituent of the numeral. For example, 31 is expressed as bokuri gelmuy and 90 as unjikuri gelji.[7]



\section{Ol Chiki script}
\label{s:olchiki}
\arial

The Ol Chiki script, also known as Ol Cemetʼ (Santali: ol 'writing', cemet' 'learning'), Ol Ciki, Ol, and sometimes as the Santali alphabet, was created in 1925 by Raghunath Murmu for the Santali language.

Previously, Santali had been written with the Latin alphabet. But because Santali is not an Indo-Aryan language (like most other languages in the south of India), Indic scripts did not have letters for all of Santali's phonemes, especially its stop consonants and vowels, which made writing the language accurately in an unmodified Indic script difficult. The detailed analysis was given by Dr. Byomkes Chakrabarti in his 'Comparative Study of Santali and Bengali'. Missionaries (first of all Paul Olaf Bodding, a Norwegian) brought the Latin script, which is better at representing Santali stops, phonemes and nasal sounds with the use of diacritical marks and accents. Unlike most Indic scripts, which are derived from Brahmi, Ol Chiki is not an abugida, with vowels given equal representation with consonants. Additionally, it was designed specifically for the language, but one letter could not be assigned to each phoneme because the sixth vowel in Ol Chiki is still problematic.
Ol Chiki has 30 letters, the forms of which are intended to evoke natural shapes. Linguist Norman Zide said "The shapes of the letters are not arbitrary, but reflect the names for the letters, which are words, usually the names of objects or actions representing conventionalized form in the pictorial shape of the characters."[1] It is written from left to right.

\newfontfamily\olchiki{code2000.ttf}

\begin{scriptexample}[]{olchiki}
\bgroup
\olchiki
\obeylines

U+1C5x 	᱐	᱑	᱒	᱓	᱔	᱕	᱖	᱗	᱘	᱙	ᱚ	ᱛ	ᱜ	ᱝ	ᱞ	ᱟ
U+1C6x	   ᱠ	ᱡ	ᱢ	ᱣ	ᱤ	ᱥ	ᱦ	ᱧ	ᱨ	ᱩ	ᱪ	ᱫ	ᱬ	ᱭ	ᱮ	ᱯ
U+1C7x  	ᱰ	ᱱ	ᱲ	ᱳ	ᱴ	ᱵ	ᱶ	ᱷ	ᱸ	ᱹ	ᱺ	ᱻ	ᱼ	ᱽ	᱾	᱿
\egroup

\unicodetable{olchiki}{"1C50,"1C60,"1C70}
\end{scriptexample}

\newfontfamily\sikkim{Tibetan Machine Uni}
\newfontfamily\lepcha{Mingzat-R.ttf}
\section{Lepcha}
\label{s:lepcha}
\epigraph{``Had your independence ensured mine, I surely would have greeted you on this moment every year ...''}{
August 15, 2015\\
Chewang Pintso\\
General Secretary, SIBLAC}

\label{s:lepcha}
\index{Scripts>Lepcha}



The Lepcha are also called the Rongkup meaning the children of God and the Rong, Mútuncí Róngkup Rumkup (Lepcha:{\lepcha ᰕᰫ་ᰊᰪᰰ་ᰆᰧᰶ ᰛᰩᰵ་ᰀᰪᰱ ᰛᰪᰮ་ᰀᰪᰱ}; "beloved children of the Róng and of God"), and Rongpa (Sikkimese:{\sikkim རོང་པ་}), are among the indigenous peoples of Sikkim and number between 30,000 and 50,000. Many Lepcha are also found in western and southwestern Bhutan, Tibet, Darjeeling, the Mechi Zone of eastern Nepal, and in the hills of West Bengal. The Lepcha people are composed of four main distinct communities: the Renjóngmú of Sikkim; the Támsángmú of Kalimpong, Kurseong, and Mirik; the ʔilámmú of Ilam District, Nepal; and the Promú of Samtse and Chukha in southwestern Bhutan.[3][2][4]\index{Languages>Lepcha}
\index{Nepal Languages>Lepcha}\index{Bhutan Languages>Lepcha} The Lepcha probably do not exceed 50,000 and hence their language is on the \textsc{UNESCO} endangered list of languages.

\begin{figure}[htbp]
\centering
\includegraphics[width=\linewidth-2\parindent]{lepchas}

\caption{Lepcha manuscript}

\end{figure}

The Lepcha have their own language, also called Lepcha. It belongs to the Bodish–Himalayish group of Tibeto-Burman languages. The Lepcha write their language in their own script, called Róng or Lepcha script, which is derived from the Tibetan script. It was developed between the 17th and 18th centuries, possibly by a Lepcha scholar named Thikúng Mensalóng, during the reign of the third Chogyal (Tibetan king) of Sikkim.[7] The world's largest collection of old Lepcha manuscripts is found with the Himalayan Languages Project in Leiden, Netherlands, with over 180 Lepcha books.

The Lepcha script, or Róng script is an abugida used by the Lepcha people to write the Lepcha language. Unusually for an abugida, syllable-final consonants are written as diacritics.

The United Nations Educational, Scientific and Cultural Organization (UNESCO) lists Lepcha as an endangered language with the following characterization:

The Lepcha language is spoken in Sikkim and Darjeeling district in West Bengal of India. The 1991 Indian census counted 39,342 speakers of Lepcha. Lepcha is considered to be one of the indigenous languages of the area in which it is spoken. Unlike most other languages of the Himalayas, the Lepcha people have their own indigenous script (the world's largest collection of old Lepcha manuscripts is kept in Leiden, with over 180 Lepcha books).

Lepcha is the language of instruction in some schools in Sikkim. In comparison to other Tibeto-Burman languages, it has been given considerable attention in the literature. Nevertheless, many important aspects of the Lepcha language and culture still remain undescribed. 


\begin{figure}[htbp]
\centering
\includegraphics[width=\linewidth-2\parindent]{lepcha}

\caption{A manuscript of the Van Manen Collection at the Kern Institute of Leiden University.}
\end{figure}

{\lepcha
Consonants bear the inherent vowel, but no virama is used to kill this vowel; vowel matras modify it, and
explicit final consonants are used where there is no inherent vowel. Initial vowels are represented with
the vowel matras on the neutral letter £ A. Initial consonants can be followed by the glides ˇ§ -YA and
ˇ• -RA, both of which normally ligate with the consonant they modify; these can also combine to form
ˇˆ -rya, which is simply a glyph ligature of the other two: ÄÙ kya, Äı kra, Ĉ krya. The glide -la is also
found, but is represented not by a ligating combining mark, but by a limited set of letters containing this
glide inherently. With few exceptions, these “combined” letters do not look like a ligature of their base
letters with some mark: Ä ka Å kla, É ga Ñ gla, é pa è pla, ë fa í fla, ì ba î bla, ï ma ñ mla,
ù ha û hla.}

The Mingzat font is still under development by SIL so I am not too sure if the rendering is correct\footnote{\url{http://scripts.sil.org/cms/scripts/page.php?site_id=nrsi&id=Mingzat}}.


\section{Unicode}

Lepcha script was added to the Unicode Standard in April, 2008 with the release of version 5.1.
The Unicode block for Lepcha is U+1C00–U+1C4F:
\begin{scriptexample}[]{Lepcha}
\bgroup
\lepcha
\obeylines
 	    0	1	2	3	4	5	6	7	8	9	A	B	C	D	E	F
U+1C0x	 ᰀ	ᰁ	ᰂ	ᰃ	ᰄ	ᰅ	ᰆ	ᰇ	ᰈ	ᰉ	ᰊ	ᰋ	ᰌ	ᰍ	ᰎ	ᰏ
U+1C1x	 ᰐ	ᰑ	ᰒ	ᰓ	ᰔ	ᰕ	ᰖ	ᰗ	ᰘ	ᰙ	ᰚ	ᰛ	ᰜ	ᰝ	ᰞ	ᰟ
U+1C2x	 ᰠ	ᰡ	ᰢ	ᰣ	ᰤ	ᰥ	ᰦ	ᰧ	ᰨ	ᰩ	ᰪ	ᰫ	ᰬ	ᰭ	ᰮ	ᰯ
U+1C3x	 ᰰ	ᰱ	ᰲ	ᰳ	ᰴ	ᰵ	ᰶ	᰷	x	x	x	᰻	᰼	᰽	᰾	᰿
U+1C4x	 ᱀	᱁	᱂	᱃	᱄	᱅	᱆	᱇	᱈	᱉	x	x	x	ᱍ	ᱎ	ᱏ

\egroup
\end{scriptexample}




\section{Sharada}
\label{s:sharada}
The Śāradā, or Sharada, script (शारदा) is an abugida writing system of the Brahmic family of scripts, developed around the 8th century. It was used for writing Sanskrit and Kashmiri. The Gurmukhī script was developed from Śāradā. Originally more widespread, its use became later restricted to Kashmir, and it is now rarely used except by the Kashmiri Pandit community for ceremonial purposes. Śāradā is another name for Saraswati, the goddess of learning.
Śāradā script was added to the Unicode Standard in January, 2012 with the release of version 6.1.

The Unicode block for Śāradā script, called Sharada, is U+11180–U+111DF: Unable to locate font in unicode.




%@book{book:1681159,
%   title =     {The Munda Languages},
%   author =    {Norman H. Zide, Gregory D. S. Anderson},
%   publisher = {Routledge},
%   isbn =      {041532890X,9780415328906},
%   year =      {2008},
%   series =    {Routledge Language Family Series},
%   edition =   {},
%   volume =    {},
%   url =       {http://gen.lib.rus.ec/book/index.php?md5=CD5787A1386CD03191256D28E1D5DAD4}}


\chapter{Phags-pa}
\label{s:phagspa}
\newfontfamily\phagspa{code2000.ttf}
\arial 
The 'Phags-pa script, (Mongolian: дөрвөлжин үсэг "Square script") was an alphabet designed by the Tibetan monk and vice-king Drogön Chögyal Phags-pa for the Mongol Yuan emperor Kublai Khan as a unified script for the literary languages of the Yuan. 


It was first promulgated in 1269, although there is an inscription to testify its use before that time. ThePP alphabet is considered to have been designed for all the languages of the Mongol empire, but it appears it was almost used exclusively for Mongolian and Chinese. 

Widespread use was limited to about a hundred years during the Yuan Dynasty, and it fell out of use with the advent of the Ming dynasty. The documentation of its use provides clues about the changes in the varieties of Chinese, the Tibetic languages, Mongolian and other neighboring languages during the Yuan era.

After the fall of the Yuan dynasty in 1368, PP fell out of use, although it may have survived on seals and in some copybooks (although some hold that these descend froma Tibetan seal script rather than from PP).  

PP was based on the Tibetan alphabet and was used for writing both Chinese and Mongolian.

The script as a whole system comes down to us in two different traditions. On the one hand we have the letters and arrangements as they were recorded in the \textit{Shu shih hui yao} and the Fas shu k'ao, two 14th century works on calligraphy. Here the letters are presented in a more Buddist and Tibetan tradition.


\begin{figure}[htbp]
\includegraphics[width=1\linewidth]{./images/phags-pa.jpg}

credit \protect\url{http://turfan.bbaw.de/dta/monght/images/monght009_seite2.jpg}
\end{figure}


\begin{scriptexample}[]{Phags-pa}
\bgroup
\unicodetable{phagspa}{"A840,"A850,"A860,"A870}

\arial
\hfill Typeset with \texttt{code2000.ttf} and \cmd{\phagspa}

\egroup
\end{scriptexample}
\medskip

Phags-pa is a historical script related to Tibetan that was created as the national script of
the Mongol empire. Even though Phags-pa was used mostly in Eastern and Central Asia for
writing text in the Mongolian and Chinese languages, it is discussed in this chapter because
of its close historical connection to the Tibetan script. The script has very limited modern use. It bears similarity to Tibetan and has no case distinctions. It is written vertically in columns running for left to right, like Mongolian. Units are often composed of several syllables and sometimes are separated by whitespace.


\printunicodeblock{./languages/phags-pa.txt}{\phagspa}

\cxset{script/.code={}}
\cxset{script=phags-pa}

\begin{docKey}[phd]{script}{ = \meta{phags-pa}} {}
The key |script| will activate the commands available for typesetting the phags-pa script.
\end{docKey}















\cxset{image=chakma.jpg}
\chapter{Chakma}
\label{s:chakma}

\newfontfamily\chakma{RibengUni.ttf}

The Chakma alphabet (Ajhā pāṭh), also called Ojhapath, Ojhopath, Aaojhapath, is an abugida used for the Chakma language and which is being adapted for the Tanchangya language.[1] The forms of the letters are quite similar to those of the Burmese script.

\begin{figure}[htbp]
\includegraphics[width=\linewidth-2\parindent]{chakma}
\end{figure}

The Chakma (Chakma or ), also known as the Changma, are a Tibeto-Burman tribe of the Chittagong Hill Tracts inBangladesh. Today, the geographic distribution of Chakmas is spread across Bangladesh and parts of northeastern India, westernBurma, China and diaspora communities in North America and Europe. Within the CHT, the Chakma are the largest ethnic group and make up half of the region's population. In Burma, they are known as Daingnet people. The Chakma are divided into 46 clans orGozas. They have their own language, customs and culture, and profess Theravada Buddhism. The Chakma Royal Family is one of the major Buddhist royal houses of the South Asia.

Chakmas are Tibeto-Burman, and are thus closely related to tribes in the foothills of the Himalayas. The Chakmas are believed to be originally from Arakan who later on immigrated to Bangladesh in around fifteenth century, settling in the Cox's Bazar District, the Korpos Mohol area, and in the Indian states of Mizoram, Arunachal Pradesh, Tripura.\href{http://thechakmadiary.weebly.com/about.html}{thechakmadiary}

The Arakanese referred to the Chakmas as Saks or Theks. In 1546, when the king of Arakan, Meng Beng, was engaged in a battle with the Burmese, the Sak king appeared from the north and attacked Arakan, and occupied the Ramu of Cox's Bazar, the then territory of the kingdom of Arakan
\bgroup
\obeylines
\chakma
𑄇𑄳𑄇 Kkā = 𑄇 Kā + 𑄳 VIRAMA + 𑄇 Kā
𑄇𑄳𑄑 Ktā = 𑄇 Kā + 𑄳 VIRAMA + 𑄑 Tā
𑄇𑄳𑄖 Ktā = 𑄇 Kā + 𑄳 VIRAMA + 𑄖 Tā
𑄇𑄳𑄟 Kmā = 𑄇 Kā + 𑄳 VIRAMA + 𑄟 Mā
𑄇𑄳𑄌 Kcā = 𑄇 Kā + 𑄳 VIRAMA + 𑄌 Cā
𑄋𑄳𑄇 ńkā = 𑄋 ńā + 𑄳 VIRAMA + 𑄇 Kā
𑄋𑄳𑄉 ńkā = 𑄋 ńā + 𑄳 VIRAMA + 𑄉 Gā
𑄌𑄳𑄌 ccā = 𑄌 cā + 𑄳 VIRAMA + 𑄌 Cā

\egroup

Fonts for the script are not available easily but the
the script can be typeset using \texttt{RibengUni.ttf} which is available at \url{http://uni.hilledu.com/}. 

\begin{scriptexample}[]{Chakma}
\unicodetable{chakma}{"11100,"11110,"11120,"11130,"11140}

\texttt{RibengUni.ttf}
\end{scriptexample}


\printunicodeblock{./languages/chakma.txt}{\chakma}


\section{Brahmi}
\label{s:brahmi}
Brāhmī is the modern name given to one of the oldest writing systems used in the Indian subcontinent and in Central Asia during the final centuries BCE and the early centuries CE. Like its contemporary, Kharoṣṭhī, which was used in what is now Afghanistan and Western Pakistan, Brahmi (native to north and central India) was an \emph{abugida}.

The A´sokan Br¯ahm¯ı of the third century BCE is the mother of all major Indian scripts,
both Indo-Aryan and Dravidian. It was an alpha-syllabic script with diacritics used for
vowels occurring in postconsonantal position. It has separate symbols for the five primary
vowels a i u e o, twenty-five occlusives and eight sonorants and fricatives. The Br¯ahm¯ı
script was used in the rock edicts set up by the Mauryan Emperor A´soka to spread the
Buddhist faith in different parts of the country. The languages represented were Pali
and certain early regional varieties of Middle Indic. The origin of the Br¯ahm¯ı script is
controversial; nearly half of the characters are said to bear similarity to the consonant
symbols employed in the South Semitic script, eventually traceable to Aramaic script of
2000 BCE (Daniels and Bright 1996: §30, 373–83).

The best-known Brahmi inscriptions are the rock-cut edicts of Ashoka in north-central India, dated to 250–232 BCE. The script was deciphered in 1837 by James Prinsep, an archaeologist, philologist, and official of the East India Company.[1] The origin of the script is still much debated, with current Western academic opinion generally agreeing (with some exceptions) that Brahmi was derived from or at least influenced by one or more contemporary Semitic scripts, but a current of opinion in India favors the idea that it is connected to the much older and as-yet undeciphered Indus script

\begin{figure}[htb]
\centering
\includegraphics[width=0.6\textwidth]{./images/ashoka-pillar.jpg}
\caption{Brahmi script on Ashoka Pillar}
\end{figure}



\begin{scriptexample}[]{Brahmi}
\bgroup
\raggedleft
\brahmi

         
   

\arial
\hfill Text: Asokan Edict typeset with \texttt{NotoSansBrahmi-Regular} 
\egroup
\end{scriptexample}

Brahmi is a Unicode block containing characters written in India from the 3rd century BCE through the first millennium CE. It is the predecessor to all modern Indic scripts.

\begin{scriptexample}[]{Brahmi}
\unicodetable{brahmi}{"11000,"11010,"11020,"11030,"11040,"11050,"11060,"11070}
\end{scriptexample}


\printunicodeblock{./languages/brahmi.txt}{\brahmi}











\egroup