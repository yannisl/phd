% \iffalse meta-comment
%<*internal>
\iffalse
%</internal>
%<*readme>
----------------------------------------------------------------
phd-layouts manager --- a to improve chapter and section designs
E-mail: yannislaz@gmail.com
Released under the LaTeX Project Public License v1.3c or later
See http://www.latex-project.org/lppl.txt
----------------------------------------------------------------
This file provides a phd for defining a class.
%</readme>
%<*readmemd>
###The `phd` LaTeX2e package

The `phd` latex package and the class with the same name provide
convenient methods to create new styles for books, reports
and articles. It also loads the most commonly used packages 
and resolves conflicts.

This work consists of the file  `phd.dtx`,
and the derived files   `phd.ins`,  `phd.pdf`, and `phd.sty`.

###Installation

run
          phd-lua.bat on windows
           pdflatex phd.dtx
           makeindex -s gind.ist -g phd 

If you have any difficulties with the package come and join us at
http://tex.stackexchange.com and post a new question or
add a comment at http://tex.stackexchange.com/a/45023/963.
or send me a message at  yannislaz at gmail.com

### Documentation

The package was written using the `doc` and `docscript` packages,
so that it is self documented in a literary programming style. 
The .pdf is a fat document, providing over fifty book styles (the
equivalent of classes) plus there is a lot of write-up on the inner
workings of TeX and LaTeX2e. However, you don't need to know much
to use it.

      \usepackage{phd}
      %%%%%%%%%%%%%%%%%%%%%%%%%%%%%%%%%%%%%%%%%%%
%%%%%%  STYLE 13
%%%%%%%%%%%%%%%%%%%%%%%%%%%%%%%%%%%%%%%%%%%

\cxset{style13/.style={
 name={Chapter},
 numbering=arabic,
 number font-size=\HUGE,
 number font-family=\sffamily,
 number font-weight=\bfseries,
 number color=\color{gray!50},
 number before=\par\vspace*{5pt}\hfill\hfill,
 number dot=,
 number after={\hspace*{7pt}\par},
 number position=rightname,
 chapter font-family=\sffamily,
 chapter font-weight=\normalfont,
 chapter font-size=\LARGE,
 chapter before={\thickrule\vspace*{20pt}\par\hfill\hfill},
 chapter after={\vskip0pt\par},
 chapter color={black!50},
 title beforeskip={\vspace*{10pt}},
 title afterskip={\vspace*{50pt}\par},
 title before={\hfill\hfill\raggedleft},
 title after={},
 title font-family=\sffamily,
 title font-color=\color{thered},
 title font-weight=\bfseries,
 title font-size=\huge,
 section indent=-1em,
 section align=\raggedright,
 section numbering=arabic,
 section indent=0pt,
 section beforeskip=0pt,
 section afterskip=\baselineskip,
 subsection align=\raggedright,
 subsection font-family=\sffamily,
 subsection font-weight=\bfseries,
 subsection font-size=\large,
 subsection font-shape=\itshape,
 subparagraph number after=\space,
}
}

\def\setstyle#1{\cxset{style#1}%
 \renewsection\renewsubsection\renewsubsubsection%
 \renewparagraph\renewsubparagraph}

\setstyle{13}


\chapter{Introduction to Chapter\\ Style Thirteen}

\section{A Brief History of Biomedical\\ Fluid Mechanics}
\lorem
\medskip
\begin{figure}[ht]
\centering
\includegraphics[width=0.45\textwidth]{./chapters/chapter14}
\includegraphics[width=0.45\textwidth]{./chapters/chapter14a}
\end{figure}
\lorem


All choices, are made via an extended key-value interface. 
Although not a compliment, it resembles CSS and the keys are a bit verbose but
attributes are easy to change and have a consistent and easy to remember interface.

To set or add a key we only use one command:

      \cxset{chapter name font-size: Huge,
             chapter number font-size: HUGE} 

### Future Development

This is still an experimental version, but I will retain the
interface in future releases. There is a large amount of
work still to be carried out to improve the template styles
provided, to test it more thoroughly and to add a number of
improvements in the special designs. At present I estimate
that I have completed about 70% of the work that needs
to be done.

__The package as it stands is not production stable.__ 


%</readmemd>
%
%<*TODO>
1. On final round add pkg options. This was left as last in order not to solve problems by adding
    options. Too many options are not a good User Interface.
2.  Finish symbol management, both text and math. Math already 60% incorporated.
3.  Better integration of indexing commands.   
4.  Revisit layout manager for Chapters. Broke again in tests.
5.  Docs. Add all references.
6.  Incorporate phd class for more flexibility.
7. Improve package manager.
8. Group script loading for better font management.
9. General font management to relook it again.
10. Add all style sections (about 100 already prepared). Once they
     are all working issue beta version.
%</TODO>
%<*internal>
\fi
\def\nameofplainTeX{plain}
\ifx\fmtname\nameofplainTeX\else
  \expandafter\begingroup
\fi
%</internal>
%<*install>
\input docstrip.tex
\keepsilent
\askforoverwritefalse
\preamble
----------------------------------------------------------------
phd --- A package to beautify documents.
E-mail: yannislaz@gmail.com
Released under the LaTeX Project Public License v1.3c or later
See http://www.latex-project.org/lppl.txt
----------------------------------------------------------------
\endpreamble

%\BaseDirectory{C:/users/admin/my documents/github/phd}
%\usedir{MWE}
\generate{\file{\jobname.sty}{
  \from{\jobname.dtx}{SECT}}
  }

%\nopreamble\nopostamble

%</install>

%<install>\endbatchfile
%<*internal>
%\usedir{tex/latex/phd}
\generate{
  \file{\jobname.ins}{\from{\jobname.dtx}{install}}
}
\nopreamble\nopostamble

\generate{
	\file{README.txt}{\from{\jobname.dtx}{readme}}
  }

\generate{
  \file{README.md}{\from{\jobname.dtx}{readmemd}}
}
\generate{
  \file{TODO.tex}{\from{\jobname.dtx}{TODO}}
}

\ifx\fmtname\nameofplainTeX
  \expandafter\endbatchfile
\else
  \expandafter\endgroup
\fi
%</internal>
%<*driver>

%\listfiles
%gdef\@onlypreamble{} % TO BE REMOVED NEEDED FOR TUTS
\documentclass[oneside,11pt,a4paper]{ltxdoc}
\usepackage[bottom=2cm]{geometry}
\savegeometry{std}
% \usepackage[style=mla]{biblatex}
\usepackage{phd}
\usepackage{phd-sections}
%\usepackage{pkgindoc}             %%% danger
\sethyperref
\begin{document}
\let\bold\bfseries

\frontmatter
\tableofcontents
\mainmatter
%\makeatletter
\cxset{defaults/.style ={% 
    chapter title margin-top-width    =  0cm,
    chapter title margin-right-width  =  1cm,
    chapter title margin-bottom-width = 10pt,
    chapter title margin-left-width   = 0pt,
    chapter align                     = left,
    chapter title align               = left, %checked
    chapter name                      = CHAPTER,
    chapter format                    = block,
    chapter font-size                 = Huge,
    chapter font-weight               = bold,
    chapter font-family               = sffamily,
    chapter font-shape                = upshape,
    chapter background-color          = white,
  % chapter label    
    chapter color               = black,
    chapter number prefix             = ,
    chapter number suffix             = ,
    chapter numbering                 = arabic,
    chapter indent                    = 0pt,
    chapter beforeskip                = -3cm,
    chapter afterskip                 = 30pt,
    chapter afterindent               = off,
    chapter number after              = ,
    chapter arc                       = 0mm,
    chapter label background-color    = white,
    chapter label color               = black,
   % chapter afterindent               = on,
    chapter grow left                 = 0mm,
    chapter grow right                = 0mm,
    chapter rounded corners           = northeast,
    chapter shadow                    = fuzzy halo,
    chapter border-left-width         = 0pt,
    chapter border-right-width        = 0pt,
    chapter border-top-width          = 0pt,
    chapter border-bottom-width       = 0pt,
    chapter padding-left-width        = 0pt,
    chapter padding-right-width       = 10pt,
    chapter padding-top-width         = 10pt,
    chapter padding-bottom-width      = 10pt,
    %  
    chapter number color              = black,
    chapter number background-color   = white,
    chapter number font-size        = huge,
    chapter number font-weight      = bfseries,
    chapter number font-family      = sffamily,
    chapter number font-shape       = upshape,
    chapter number align            = Centering,
    %
    chapter title font-size        = Huge,
     chapter title font-weight      = bold,
     chapter title font-family      = sffamily,
     chapter title font-shape       = upshape,
     chapter title color            = black,
     chapter title background-color = white,
     }%
   }  
\makeatother     
%\makeatletter
%\cxset{toc image=\@empty,
%       chapter toc=true,
%       title beforeskip=1pt}
%
%\@specialfalse
%
%
%\renewcommand\stewart[2][]{%
%\fancypagestyle{fancy}{%
%\lhead{}\rhead{}
%\chead{}
%\cfoot{}
%\lfoot{}
%\rfoot{\thepage}
%\def\footrule#1{{\color{blue}%
%  \hrule width\paperwidth}\vskip3pt
%}
%
%\renewcommand{\headrulewidth}{0pt}
%\renewcommand{\footrulewidth}{0.4pt}}
%
%\clearpage
%
%\begin{tikzpicture}[remember picture,overlay]
%% Main shading block
%\node [xshift=5cm,yshift=-\paperheight] at (current page.north west)
%[text width=0.98\textwidth,text height=\paperheight, fill=thecream!30,rounded corners,above right]
%{};
%\node [xshift=6.5cm,yshift=-1.5cm-\soffsety] at (current page.north west)
%[text width=0.9\textwidth,below right]{\sffamily \bfseries \huge #2};
%
%\node [xshift=3cm,yshift=-1.5cm] at (current page.north west)
%[text width=3cm,align=center,minimum height=2.5cm, fill=blue,below right]
%{\[\text{\HHUGE\bfseries\sffamily\color{white}\thechapter}\]
%\par\vspace*{3pt}
%};
%
%\node [xshift=-0.2cm,yshift=-21.5cm] at (current page.north west)
%[text width=3cm,above right]%
%{\includegraphics[width=1.0\paperwidth]{\image@cx}};
%% second box left
%\node [xshift=3cm,yshift=-19.5cm] at (current page.north west)
%[text width=9cm,minimum height=2.5cm,inner sep=0.5em, fill=blue,below right]
%{\color{white}
%  \bfseries\sffamily \texti@cx
%};
%% Last block
%\node [xshift=6.5cm,yshift=-26cm] at (current page.north west)
%[text width=12cm,above right]
%{\textii@cx
%};
%\end{tikzpicture}
%\par
%\clearpage
%}





\cxset{steward,
  chapter numbering=arabic,
  chapter format = stewart,
  offsety=0cm,
  image= {./images/hine02.jpg},
  texti={When Lamport designed the original \LaTeX\ sectioning commands he did not provide a fully comprehensive interface for modifying their design. With current tools available improvements are much easier to program and this chapter provides the details.},
  textii={\precis{In this chapter we discuss a method that allows the production of fancy chapter headings and formatting, based on a set of key values. Central  to this process is the separation of content from presentation.
We also discuss the basic formatting tools that are available and how one can modify them to mould new book designs.}
 }
}


\chapter{Designing Chapter Headings}
\addtocimage{-12pt}{-20pt}{./images/tocblock-man-01.jpg}

\section*{Introduction}

A \textls*{crowded} first page is as unsightly as a crowded title page, wrote De Vinne in \emph{Modern Methods of Book Composition} in 1904.  Not much has changed since. A new chapter must make a good impression and must give an immediate signal that a different topic is going to be discussed. Traditionally chapter openings in LaTeX are an unimpressive and dry event. Our aim is to brighten it up a bit, while keeping true separation of content from presentation, but avoiding the pit traps of over ornamenting the design. A book is to be read and we should provide minimal ornamentation. \index[phdkeys]{chapter> ornamentation}

% \usepackage{array,tabularx}
%\newcolumntype{Y}{>{\raggedleft\arraybackslash}X}% see tabularx
%\tcbset{enhanced,fonttitle=\bfseries\large,fontupper=\normalsize\sffamily,
%colback=yellow!10!white,colframe=red!50!black,colbacktitle=thecodebackground,
%coltitle=black,center title,
%tabularx={X||Y|Y|Y|Y||Y},% this sets ’before upper’ and ’after upper’
%before upper app={Group & One & Two & Three & Four & Sum\\\hline\hline} }
%
%\begin{tcolorbox}[title=My table]
%Red & 1000.00 & 2000.00 & 3000.00 & 4000.00 & 10000.00\\\hline
%Green & 2000.00 & 3000.00 & 4000.00 & 5000.00 & 14000.00\\\hline
%Blue & 3000.00 & 4000.00 & 5000.00 & 6000.00 & 18000.00\\\hline\hline
%Sum & 6000.00 & 9000.00 & 12000.00 & 15000.00 & 42000.00
%\end{tcolorbox}

\begin{figure}[htbp]
\centering
\parindent=0pt
\fbox{\includegraphics[width=\textwidth]{metropolitan-spread}}
\par
\caption{A chapter opening from the Metropolitan Museum of Art publicaion, \textit{Assyrian Reliefs and Ivories} by Vaughn. E. Crawford et. al., 1980. The spread is simple and the chapters are not numbered. This is a common characteristic of many more recently published books.}
\end{figure}


What is to us now a common occurence with instant book-printing was not always so. The cost of illustrated books was a prime factor and as Tschichold wrote:
\begin{quotation}
In the area of book design, in the last few years a revolution has taken place, until recently recognized by only a few. but which now begins to influence a much wider range of action.
It means placing much greater emphasis on the appearance of the book and a wholly contemporary use of typographic and photographic means. Before the invention of printing, literature of that time was spread around by the mouth of the author himself or by professional bards. The books of the Middle Ages - like the "Mannessische Liederhandschrift" - had
\end{quotation}

The type of book you are writing and its contents will determine an appropriate design for chapter headings and the type of design and numbering if any for subsections. Here we are merely providing a mechanism to produce them. These methods can produce a mastepiece or an ugly piece of work. Some simple suggestions follow (from my observations of styles in books I like). In general you need to think what type of book you are developing. For example a novel, should be sectioned very carefully. Many books avoid marking of sections other than chapters totally, perhaps marking them just with a soft ornament such as three centered asterisks.

\section{Numbering of Sections}


In general books do not number sections beyond subsection. You can avoid them all together, if you are not going to reference the sections extensively. 

In works of fiction, authors sometimes number their chapters eccentrically, often as a metafictional statement. For example:
Seiobo There Below by László Krasznahorkai has chapters numbered according to the Fibonacci sequence.

The Curious Incident of the Dog in the Night-Time by Mark Haddon only has chapters which are prime numbers.

At Swim-Two-Birds by Flann O'Brien has the first page titled Chapter 1, but has no further chapter divisions.

God, A Users' Guide by Seán Moncrieff is chaptered backwards (i.e., the first chapter is chapter 20 and the last is chapter 1). The novel The Running Man by Stephen King also uses a similar chapter numbering scheme.
Every novel in the series A Series of Unfortunate Events by Lemony Snicket has thirteen chapters, except the final instalment (The End), which has a fourteenth chapter formatted as its own novel.

Mammoth by John Varley has the chapters ordered chronologically from the point of view of a non-time-traveler, but, as most of the characters travel through time, this leads to the chapters defying the conventional order.


\begin{pgfpicture}
\pgfpathmoveto{\pgfpointorigin}
\pgfpathlineto{\pgfpoint{1cm}{1cm}}
\pgfpathlineto{\pgfpoint{1cm}{0cm}}
\pgfusepath{fill}
\end{pgfpicture}




\begin{figure}[tbp]
\centering
\parindent=0pt
\fbox{\includegraphics[width=\textwidth]{fantasy-architecture}}
\par
\caption{A chapter opening from the Metropolitan Museum of Art publicaion, \textit{Assyrian Reliefs and Ivories} by Vaughn. E. Crawford et. al., 1980. The spread is simple and the chapters are not numbered. This is a common characteristic of many more recent books.}
\end{figure}


\begin{figure}[tbp]
\centering
\parindent=0pt
\fbox{\includegraphics[width=\textwidth]{fantasy-architecture-02}}
\par
\caption{A chapter opening from the Metropolitan Museum of Art publicaion, \textit{Assyrian Reliefs and Ivories} by Vaughn. E. Crawford et. al., 1980. The spread is simple and the chapters are not numbered. This is a common characteristic of many more recent books.}
\end{figure}


\section*{Use of Color}

The modern books that Tschilchod was discussing have long been overwhelmed by the appearance of larger, coffee book type of books. Our brains our now conditioned by branding and graphic design is everywhere. 

Once you have decided that the book is going to be a bit more colorfull, the choice of color will follow. The decision what to color will be an important one, which brings us to color theory. The history of color is perhaps as colorfull as the rest. Attempts to formalize and recognize order date back to Aristotle (384-322 bce) but began in earnest with Leonardo da Vinci (1452-1519) and have progressed ever since. Leonardo noted that certain colors intensify each other, discovering \textit{contrary} and \textit{complementary} colors. The first color wheel was invented by Britain's Sir Isaac Newton (1642-1727), who split white light into red, orange, yellow, green, blue, indigo and violet beams, then joined the two ends of the spectrum to form a circle showing the natural progression of colors. When Newton created the color wheel, he noticed that mixing two colors from opposite positions produced a neutral or \textit{anonymous} color.


\begin{figure}[htbp]
\parindent=0pt
\centering
\fbox{\includegraphics[width=\textwidth]{line-designs} }
\caption{Spread from \textit{Beautiful Geometry}, Eli Maor and Eugen Jost, Princeton Univeristy Press, 2014. A subtle coloring of the chapter heading, de-emphasizing the chapter number and coloring the chapter title. There is no chapter label. A dropcap with the same color starts the first paragraph. This style is easy to achive with the phd system.}
\end{figure}


\begin{figure}[htbp]
\parindent=0pt
\centering
\fbox{\includegraphics[width=\textwidth]{color-book01.jpg} }
\bigskip

\fbox{\includegraphics[width=\textwidth]{color-book02.jpg} }
\end{figure}

One would expect a book written for the sole purpose of describing color theory and its application to the Graphic Arts, is expected to be colorful. Note the de-emphasizing of the label and number. 

\begin{figure}[htbp]
\parindent=0pt
\centering
\fbox{\includegraphics[width=\textwidth]{color-book-03.jpg} }
The chapter heading label and number are almost invisible. The heading text, is typeset in large bold letters, shouting what is coming next. Not your typical scintific book\ldots
\bigskip

\fbox{\includegraphics[width=\textwidth]{color-book-04.jpg} }
\end{figure}

Advertizing people understand that they need to present the message of an advertizement loud and clear so as to catch the busy eye. A heading's message is the title description. Neither the label not the chapter if any are necessary to convey the message. The chapter heading is analogous to the stop at the end of a sentence. The brain gets a signal to absorb what was written before it and get ready for the next. The heading signals the end of a topic. One must not dwell on it.


\section{Contemporary Chapter Headings}

In the book \textit{China} the designer used both a chapter heading on a spread of two images, as well as repeated the chapter number on the text pages \ref{fig:threepage}. The images distill the message of the chapter, although the chapter subtitle is almost unreadable, dominated by the surrounding text. From a technical perspective, the chapter command must paint the two images, set the right type of heading for each page and then without increasing the counter, change the counter to one that displays the chapter number in words and then continue with typesetting the text. A careful choice of images is necessary for such chapters, as well as cropping the images to match the aspect ratio of the book pages. One also needs to be carefull for \latexe not to place any floats in between the page spreads. 

\begin{figure}[htbp]
\parindent=0pt
\centering
\fbox{\includegraphics[width=\textwidth]{beijing.jpg} }\par
\vfill

\fbox{\includegraphics[width=\textwidth]{beijing-01.jpg} }\par
%\fbox{\includegraphics[width=\textwidth]{pearl-river.jpg} }
\caption{A full page chapter spread.}
\label{fig:threepage}
\end{figure}

\begin{figure}[htbp]
\parindent=0pt
\centering
\fbox{\includegraphics[width=\textwidth]{beijing.jpg} }\par
\vfill

\fbox{\includegraphics[width=\textwidth]{beijing-01.jpg} }\par
%\fbox{\includegraphics[width=\textwidth]{pearl-river.jpg} }
\caption{A full page chapter spread.}
\label{fig:threepage}
\end{figure}


\clearpage



In Figure~\ref{fig:photospread} the bands are black, but position low on the page. The size of the pages are 9.69 \texttimes 11.42. The books sections are not numbered. Text i sbroken through inserts of bigger text. Many of the examples here are from
commercial nude photography books, as they tend to break with tradition. In the 1970s and 1980s, fashion photographers began to present a
new, confrontational image of the female body. The pioneer in this
respect was the German Helmut Newton (1920–2004). Newton’s
photographs of nudes were overtly sexual, with an undertone of
menace, and although his models tended to be depicted as part
of the social elite they were often placed, apparently caught out
in reportage style, in sordid environments engaged in fantasy and
fetish. His work made him highly influential in fashion photography,
though some of it was thought too highly sexual for American
magazines and appeared only in those published in Europe.


\begin{figure}[htbp]
\parindent=0pt
\includegraphics[width=\textwidth]{baetens-01.jpg} \par
\vfill\vfill\vfill\vfill
\includegraphics[width=\textwidth]{baetens-02.jpg}\par
\caption{Chapter spread and first pages after the chapter title which is on the right page of the chapter spread. From \textit{New Photography, Art and the Craft}, Pascal Baetens, DK Publications. }
\label{fig:photospread}
\end{figure}

In the 1980s, Newton undressed the dynamic and independent
female in a series called Big Nudes. In this series the women are
indeed naked and very tall, wearing nothing but makeup and high
heels. The Big Nudes were exhibited in the form of life-size prints
that were intended to provoke the viewer by showing self-confident
women who knew what they wanted and were very aware of their
beauty and sexuality



\chapter{Package Usage}

To use the package include it just like any other package:

\begin{teXXX}
\documentclass{book}
\usepackage{phd}
\cxset{style13}
\begin{document}
\chapter{Introduction}
\end{document}
\end{teXXX}

The command \docAuxCommand{cxset} sets the default style for the example to the style defined as \meta{style13}. The package currently offers  100 templates and numerous keys to manipulate them further. Styles are similar to \enquote{themes} used in web programming; they are a collection of keys that resemble in many ways \texttt{css}. Styles can have any names and I am sure as package usage increases and evolve,they will get better names. 

\section{Background}

Before describing in detail how to specify a new layout for headings, we offer an overview of how the task can be accomplished and the design philosophy behind the approach. 

Irrespective of the technique and tools used, the creation of new layouts can always be divided into the following three tasks: constructing a document from “layout bricks”, which we can term as “blocks” or “elements”; establishing the layout semantics of each block; and finally, creating a layout engine supporting any document constructed from such blocks.

\begin{description}
\item [Canned Layouts] At one end of the spectrum, the most accessible approach consists of picking, a canned layout, such as LaTeX itself and perhaps only provide rudimentary macros to manipulate it.
\item [Constraints] Constraints offer a middle ground between canned layouts and handwritten layout engines. Constraints are arguably the most widespread and successful layout programming technique. For, instance, the foundations of \tex are laid upon constraint. CSS, the ubiquitous web template language, also relies on constraints, although in a more restricted and indirect manner.
\end{description}

\subsection{Blocks and Elements}

We define an \emph{element} as a document block, that cannot be subdivided further. For example the chapter title element, is composed of the text of the chapter title. 

A \emph{block} on the other hand is can contain other blocks and or numerous elements. We can consider the chapter headings as \emph{blocks}, composed of three blocks the chapter, number and title. Each block is then composed of elements. Each element has properties and traits. One of these mandary properties is the name. 

Blocks are either \emph{configured} (all constraints are mandatory), or flexible (there are optional/alternative constraints). By bundling optional constraints, flexible blocks make their specification customizable by non-technical users. 

\subsection{Language semantics}

One of the aims of the syntax of the templates was to offer familiar terminology and to remove the use
of \tex macros as far as possible from templates. 
\medskip

{\parindent0pt

 \textit{section}| font-family=serif,|\\
 \textit{section}| font-size=LARGE,|\\
 \textit{section}| font-weight=bold,|\\
}

The restriction I imposed is problematic when dealing with fractions of linewidths and textwidths. So
at present we allow for example |title text-width=0.5\texwidth| or |title text-width=10cm| or any other valid units. Ideas for improvements can only come from user feedback in the future.

Some experimental ideas incorporated are:

\begin{verbatim}
title text-width = 0.5 text-width,
title text-width = 1.2 text-width,
\end{verbatim}

A better parser will need to be programmed for dimensions, which are all currently handled as etex |dimexpr|. 

The syntax must allows both for microtypography as well as macro-typographical features. The former would deal with mostly fonts, spacing and text justification, where the latter deals with layouts, borders shapes and the positioning of elements on the page and also reletively to other elements or blocks.

An advantage of this approach is that it also opens the possibility of parsing the text with a language other than \tex and translating the document to another format, such as |HTML| or |XML| either fully or partially. Next we will describe both the syntax as well as the usage of the settings.

\section{Chapter opening page}

The standard \latexe classes offer only two options to either open a chapter on an odd page or at any page. This package offers five alternatives:

\begin{docKey}[phd]{chapter opening}{=\meta{any, left, right, anywhere, ifafter}}{default none, initial=any}
For documents that are primarily to be read on the web, use |any| for normal books, use \textit{right}. Some templates that we provide use |any| and the examples use |anywhere| to enable us to display the heading at any position on the page.
\end{docKey}

\begin{decription}
\item [any] Opens a chapter at any page, either \textit{verso} or \textit{recto}.
\item [left] Opens a chapter on an even page
\item [right] Opens a chapter on a right page.
\item [anywhere] Opens a chapter at the point where the \cs{chapter} is typed.
\item [none] Alias for \marg{anywhere}.
\item [ifafter] Opens a chapter at the next page if the page has material that does not exceed a certain portion of \cs{textheight}.
\end{description}

\colorlet{theoption}{bgsexy}

To change a setting you just modify the value of the key \oarg{\option{chapter opening}} to one of the values described earlier. 

\begin{dispListing}
\cxset{chapter opening = anywhere}
\end{dispListing}
 
We use this key to print the many examples typesetting chapter heads that follow (see the example~\ref{ex:anywhere}).  


\begin{texexample}{title=Inline Chapter Example}{ex:anywhere}
\cxset{examplestyle/.style = {chapter format = block,
       chapter opening = anywhere,
       chapter name = CHAPTER, 
       %label
       chapter label font-family      = sffamily,
       chapter label color            = primary,
       chapter label background-color = white,
       % number
       chapter number font-family = sffamily,
       chapter number font-size = HUGE,
       chapter number color     = primary,
       chapter label align = centering,
       chapter number background-color = white,
       %title
       chapter title font-family = rmfamily,
       chapter title align = centering,
       chapter title background-color = bgsexy!15,
       chapter title before background-color=white}}
\cxset{examplestyle}       
\lorem
\chapter{Typography Example}
\lorem
\chapter{Another Chapter Heading}
\lorem
\end{texexample}


%\cxset{toc chapter = true}
\addtocounter{chapter}{-1}

Examples for other types of chapter openings follow in the rest of the documentation.

\subsection{Blank pages before chapters}

In the standard LaTeX book class when the \texttt{openany} option is not given or in the report class when the openright is given, chapters start at odd-numbered pages. This can cause a blank page to be printed. Some book designers prefer this page to be completely empty, without any headers or footers. This cannot be done with \lstinline{\thispagestyle} as this command will have to be issued on the \textit{previous} page. However by a suitable redefinition of the
\lstinline{\clearpage} this can be done automatically.
\medskip

\begin{teXXX}
\makeatletter
\def\cleardoublepage{\clearpage\if@twoside\ifodd\c@page\else
  \hbox{}
  \vspace*{\fill}
  \begin{center}
    This page left intentionally blank.
  \end{center}
  \vspace{\fill}
  \thispagestyle{empty}
  \newpage
  \if@twocolumn\hbox{}\newpage\fi\fi\fi}
\makeatother
\end{teXXX}


This is achieved easily by setting the following options:
\bigskip

\begin{tcolorbox}
\lstinline{chapter blank page=empty}\par
\lstinline{chapter blank page text=Some text.}\par
\lstinline{chapter blank page=plain}\par
\end{tcolorbox}
\medskip



The last one refers to a \lstinline!\thispagestyle{plain}!.
\cxset{chapter opening = right, chapter format = block}
\chapter{Test}

\cxset{defaults, chapter opening= anywhere}



\section*{Keys for chapter head formatting}

A chapter heading can be considered of being constructed of several parts, the \textit{chapter number}, the chapter name typically \textit{chapter} and the \textit{title}. Predefined keys handle all the elements of formatting. Additional keys are defined to handle other elements such as inclusion of images or producing complicated examples with graphics constructed with \texttt{TikZ} and other similar packages.


\bigskip\bigskip\bigskip\bigskip
\let\oldrefkey\refKey
\let\refKey\texttt
\makeatletter
\long\def\demobox#1#2{%
\par\bigskip\bigskip\bigskip
\begin{tcolorbox}[enhanced,left=0pt, top=0pt, bottom=0pt,width=\textwidth,
  enlarge top initially by=1cm,enlarge bottom finally by=1cm,left skip=1cm,right skip=1cm,
  colframe=white,colback=white,
  colbacktitle=red!30!white,colupper=black!7!white,
  code={\appto\kvtcb@shadow{%
    \path[fill=white,draw=yellow!50!black,dashed,line width=0.4pt]
      ([xshift=-1cm,yshift=-1cm]frame.south west) rectangle
      ([xshift=1cm,yshift=1cm]frame.north east);
     \path[fill=blue!20!white, 
              opacity=0.3, draw=yellow!50!black,solid,line width=1pt]
      ([xshift=-2cm,yshift=-2cm]frame.south west) rectangle
      ([xshift=2cm,yshift=2cm]frame.north east);  
    }},
  finish={
  \draw[thick,<->] ([yshift=-1.3cm]frame.north west)-- node[below]{\texttt{#1 width}}
    ([yshift=-1.3cm]frame.north east);
  \draw[thick,<->] ([xshift=-15mm]frame.north east)-- node[above]{\refKey{#1 height}}
    ([xshift=-15mm]frame.south east);
  \draw[thick,<->] (frame.north)-- node[right]{\refKey{#1 padding-top}} +(0,1);
  \draw[thick,<->] ([yshift=1cm]frame.north)-- node[right]{\refKey{#1 margin-top}} +(0,1);
  \draw[thick,<->] (frame.south)-- node[right, align=left]{\refKey{#1 padding-bottom}}+(0,-1);
  %left padding
  \draw[thick,<->] (frame.west)-- node[below right,align=center]{\refKey{#1 padding-left }}+(-1,0);
  %left margin
  \draw[thick,<->] ([xshift=-1cm,yshift=-0.9cm]frame.west)-- node[below right,xshift=-1,align=left]{\refKey{#1 margin-left }\\\refKey{#1 grow to left by}}+(-1,0);
  %right padding
  \draw[thick,<->] (frame.east)-- node[below left,align=center]{\refKey{#1 padding-right}}+(1,0);
 %right margin
  \draw[thick,<->] ([xshift=1cm,yshift=-0.9cm]frame.east)-- node[below left,xshift=1, align=right]{\refKey{#1 margin-right}\\\refKey{#1 grow to right by}}+(1,0);
 \draw[thick,<->] ([yshift=-2cm]frame.south)-- node[right, align=left]{\refKey{#1 margin-bottom},\\ \refKey{#1 after skip}}+(0,1);
  }
    ]
#2%
%\hrule width0pt height4.5cm depth0pt\relax% \vspace*{4.5cm}% \lipsum[1]
\end{tcolorbox}\par
\bigskip\bigskip\bigskip}
\makeatother

\demobox{chapter}{\scalebox{1.17}{\HHHUGE Chapter}}

The number box is again drawn in a box similar to a chapter with all properties generalized.

\demobox{number}{\scalebox{1.15}{\HHHUGE Thirteen}}



All parameters shown in the diagram can be set using the command \cs{cxset}. The property names follow conventions similar to those of |css|, rather than typical conventions of \tikzname that are more widely known to the programming community. The prefix to these properties (in the example \textit{chapter}) can be thought of
as similar to a |class| or |id| name in |css|.  

\begin{docCommand}{cxset}{\marg{options}}
  Sets options for every following \refEnv{tcolorbox} inside the current \TeX\ group.
  By default, this does not apply to nested boxes, see \Vref{subsec:everybox}.\par
  For example, the colors of the boxes may be defined for the whole document by this:
\begin{dispListing}
\cxset{chapter numbering = Roman,
       chapter number color = blue}
\end{dispListing}
\end{docCommand}

\begin{docKey}[]{chapter padding-top}{=\meta{dimension}}{no default, initial value 0pt}
All padding keys take one argument, which is a dimension. The length is also stored in a register
\cmd{\chapterpaddingtop}. In this chapter it was set at %\the\chapterpaddingtop.
\begin{dispListing}
\cxset{colback=red!5!white,colframe=red!75!black, chapter padding-top=2pt}
\end{dispListing}
\end{docKey}



\begin{docKey}[]{chapter padding-right}{=\meta{dimension}}{no default, initial value 0pt}
All padding keys take one argument, which is a dimension. The length is also stored in a register
\cmd{\chapterpaddingright}.  In this chapter it was set at %\the\chapterpaddingright.
\end{docKey}

\begin{docKey}[]{chapter padding-bottom}{=\meta{dimension}}{no default, initial value 0pt}
All padding keys take one argument, which is a dimension. The length is also stored in a register
\cmd{\chapterpaddingbottom}.  In this chapter it was set at %\the\chapterpaddingbottom.
\end{docKey}

\begin{docKey}[]{chapter padding-left}{=\meta{dimension}}{no default, initial value 0pt}
All padding keys take one argument, which is a dimension. The length is also stored in a register
\cmd{\chapterpaddingleft}.  In this chapter it was set at %\the\chapterpaddingleft.
\end{docKey}

%% margin

\begin{docKey}[]{chapter margin-top}{=\meta{dimension}}{no default, initial value 0pt}
All padding keys take one argument, which is a dimension. The length is also stored in a register
\cmd{\chaptermargintop}. In this chapter it was set at .
\end{docKey}

\begin{docKey}[]{chapter margin-right}{=\meta{dimension}}{no default, initial value 0pt}
All padding keys take one argument, which is a dimension. The length is also stored in a register
\cmd{\chapterpaddingright}.  In this chapter it was set at %\the\chapterpaddingright.
\end{docKey}

\begin{docKey}[]{chapter margin-bottom}{=\meta{dimension}}{no default, initial value 0pt}
All padding keys take one argument, which is a dimension. The length is also stored in a register
\cmd{\chapterpaddingbottom}.  In this chapter it was set at %\the\chapterpaddingbottom.
\end{docKey}

\begin{docKey}[]{chapter margin-left}{=\meta{dimension}}{no default, initial value 0pt}
All padding keys take one argument, which is a dimension. The length is also stored in a register
\cmd{\chaptermarginleft}.  In this chapter it was set at %\the\chaptermarginleft.
\end{docKey}

\subsection{Borders}

Border have three properties \emph{width, color} and \emph{style}. They can set individually for
each side of the box or using the shorter key .

\begin{docKey}[]{chapter border-top-width}{ = \meta{dimension}}{no default, initial value 0pt}
All border keys take one argument, which is a dimension.
\end{docKey}

\begin{docKey}[]{chapter border-right-width}{=\meta{dimension}}{no default, initial value 0pt}
All border keys take one argument, which is a dimension.
\end{docKey}

\begin{docKey}[]{chapter border-bottom-width}{ = \meta{dimension}}{no default, initial value 0pt}
All border keys take one argument, which is a dimension.
\end{docKey}

\begin{docKey}[]{chapter border-left-width}{ = \meta{dimension}}{no default, initial value 0pt}
All border keys take one argument, which is a dimension.
\end{docKey}

\subsubsection{Border Colors}

The colors follow the same pattern for |border-width| and again they can be set individually or using
a shorter key to set all of them in one color. 

\begin{docKey}[]{chapter border-top-color}{=\meta{color name}}{no default, initial value black}
All border keys take one argument, which is a dimension.
\end{docKey}

\begin{docKey}[]{chapter border-right-color}{=\meta{color name}}{no default, initial value black}
All border keys take one argument, which is a dimension.
\end{docKey}

\begin{docKey}[]{chapter border-bottom-color}{=\meta{color name}}{no default, initial value black}
All border keys take one argument, which is a dimension.
\end{docKey}

\begin{docKey}[]{chapter border-left-color}{=\meta{color name}}{no default, initial value black}
This key is stored in \cmd{\chapterborderrightcolor} and the value in this chapter is 
%\ExplSyntaxOn \l_phd_chapter_border_right_color_tl.
\ExplSyntaxOff
\end{docKey}



\subsubsection{Border Styles}

Standard |css|  offers four styles \emph{dotted, solid, double, dashed}. We offer almost an unlimited set of styles.

\begin{docKey}[phd]{chapter border-top-style}{=\meta{style name}}{no default, initial value \texttt{none}}
The |border-style| properties take a value, which can be |solid, double, dotted, dashed, asterisk|.
\end{docKey}

\begin{docKey}[phd]{chapter border-right-style}{=\meta{style name}}{no default, initial value \texttt{none}}
The |border-style| properties take a value, which can be |solid, double, dotted, dashed, asterisk|.
\end{docKey}

\begin{docKey}[]{chapter border-bottom-style}{=\meta{style name}}{no default, initial value \texttt{none}}
The |border-style| properties take a value, which can be |solid, double, dotted, dashed, asterisk|.
\end{docKey}

\begin{docKey}[]{chapter border-left-style}{=\meta{style name}}{no default, initial value \texttt{none}}
The |border-style| properties take a value, which can be |solid, double, dotted, dashed, asterisk|.
\end{docKey}

\begin{docKey}[phd]{chapter border-style}{=\meta{style name}}{no default, initial value \texttt{none}}
This key sets all chapter-border-\meta{top,right,bottom,left}-style to a single value.
\end{docKey}

\subsubsection{Fonts and colors}

All font parameters can be set using individual keys. The naming scheme in general follows |css| conventions.

\begin{docKey}[phd]{chapter color}{=\meta{color name}}{no default, initial value \texttt{black}}
This key sets the color for the \textit{chapter element}. The color name is stored in \cmd{\chaptercolor@cx}.
The value in this chapter is% \makeatletter\texttt{\chaptercolor@cx}\makeatother.
\end{docKey}

\begin{docKey}[phd]{chapter font-size}{=\meta{Huge, Large}}{no default, initial value \texttt{Huge}}
This sets the size for rendering the \textit{chapter element}. Use one of the following predefined values.
Note that you can either use a command i.e, |chapter font-size=|\cmd{\huge} 
or the command name i.e., |chapter font-size=huge|. The latter is the recommended method.
\end{docKey}

\begin{marglist}
\item [tiny] renders as {\tiny tiny}.
\item[footnotesize] renders as {\footnotesize footnotesize}
\item [small] Opens a chapter on an even page
\item [large] Opens a chapter on a right page.
\item [LARGE] Opens a chapter at the point where the \cs{chapter} is typed.
\item [huge] Alias for \marg{anywhere}.
\item [Huge] Opens a chapter at the next page if the page has material that does not exceed a certain portion of
 \cs{textheight}.
 \item[HUGE] renders as {\HUGE HUGE}.
 \item[HHUGE] renders as {\HHUGE HUGE}.
\end{marglist}

\begin{texexample}{Sizing settings}{}
\cxset{
          chapter format = block,
          chapter label font-size= HUGE,
          chapter name = Chapter,
          chapter format=block,
          chapter number font-size= HUGE,
          chapter title font-size=LARGE,
         % 
         % chapter padding-top=0pt,
         % chapter padding-bottom=0pt,
         % title margin-top=3pt,
         %
          }
\chapter{Setting font-sizes}          
\lorem

\end{texexample}


\begin{docKey}{chapter font-family}{ = \meta{sffamily, rmfamily etc.}}{no default, initial value \texttt{sffamily}}
The |font-family| key accepts \latexe conventional family names or |css| names such as |serif| and |non-serif|. The
value is stored in \docAuxCommand{chapter_font_family}, in this chapter it is set as {\ExplSyntaxOn\meaning\chapter_font_family\ExplSyntaxOff}
\end{docKey}


\begin{marglist}
\item [sffamily] The \emph{chapter element} is rendered in the document default \cmd{\sffamily}.
\item [rmfamily] The \emph{chapter element} is rendered in the document default \cmd{\rmfamily}.
\end{marglist}

%% Font weights
\begin{docKey}[]{chapter font-weight}{=\meta{mdseries,bfseries,etc.}}{no default, initial value \texttt{bfseries}}
The |font-weight| key accepts \latexe conventional family names or |css| names such as |bold| and |bfseries|. The
value is stored in \cmd{\chapterfontweight@cx}, in this chapter it is set as 
{\ExplSyntaxOn\expandafter\string\l_phd_chapter_label_fontweight_tl\ExplSyntaxOff}

\begin{texexample}{Setting chapter element font-weights}{fontweight}
\cxset{chapter label font-weight=normal}
\chapter{Font-weight is normal}
\cxset{chapter label font-weight= bfseries}
\chapter{Font-weight is bfseries}
\lorem
\end{texexample}
\end{docKey}


\begin{marglist}
\item [normal] The \emph{chapter element} is rendered in the document default \cmd{\sffamily}.
\item [bold] The \emph{chapter element} is rendered in the document default \cmd{\rmfamily}.
\item[bfseries] Renders as bold.
\item[mdseries] renders as medium series.
\item[light] This is an alias for normal.
\item[\upshape\ttfamily\string\bfseries] The command version of the setting.
\item[\upshape\ttfamily\string\mdseries] The command version of the setting.
\end{marglist}



\begin{docKey}[]{chapter font-shape}{=\meta{itshape,upshape,etc.}}{no default, initial value \texttt{upshape}}
The |font-weight| key accepts \latexe conventional family names or |css| names such as |bold| and |bfseries|. The
value is stored in |chapter_font_weight|, in this chapter it is set as %\ExplSyntaxOn \texttt{\chapter_font_shape}\ExplSyntaxOff.
\end{docKey}

In |css| the |font-shape| is named as |font-style| so we alias it as well. 

%\begin{marglist}
%\item[normal] normal font-style, defaults to |upshape|.
%\item[upshape] normal font-style, defaults to |upshape|. 
%\item[italic] italic shape, renders as {\itshape italic}. For some fonts it might not be available.
%\item[itshape] italic shape, alias of |italic|.
%\item[oblique] oblique font, in \latexe is equivalent to \cmd{\slshape} and renders as {\slshape slshape}, which might be slightly different than {\itshape italic}.
%\end{marglist}


\begin{texexample}{Setting up Fonts}{chapterfonts}
\cxset{   chapter format = block,
          chapter opening=anywhere,
          chapter label font-weight=normal,
          chapter label font-shape=upshape,
          %chapter border-width=0pt,
          %chapter border-style=none,
          %chapter padding-top=0pt,
          chapter label font-size=large,
          chapter number font-size=large,
          chapter number color=black,
          %title font-size=large,
          }
\chapter[fonts]{Test Font Weights}
\lorem
\cxset{chapter label font-shape=itshape}
\chapter{Test Italic Shape}
\lorem
\cxset{chapter label font-shape=normal}
\chapter{Test normal font-shape}
\lorem
\end{texexample}



The specification of font families is somewhat problematic. In the web the |css| allows |font-family|  to hold several font names as a ``fallback” system. If the browser does not support the first font, it tries the next font.

There are two types of font family names:

\begin{description}
\item[family-name] The name of a font-family, like “times”, “courier”, “arial”, etc.
\item[generic-family] The name of a generic family, like “serif”, “sans-serif”, “cursive”, “fantasy”, “monospace”.
\end{description}

Generally in the \tex community leaving the choice of font  open to what is available on a user’s computer is frowned upon. Knuth’s original aim to render consistently documents, irrespective of a user’s computer installation has served the community well, and it is possible three decades later to produce documents identical in all respects to the original. 

If this is still a valid requirement for documents is debatable. Current document processing requirements are focusing more in the preservation of content and document structure rather than form. Typeset documents in soft copy are now widely preserved in |pdf| or |postcript|  formats. One can archive the |.tex| file as well as the |pdf| file.  Back to the provision of keys, a key defined in a 
similar fashion to those of |css| could help, but there is also the issue of slow compilation. If a font cannot be
found, with the current code, it can slow down compilation tremendously. I am leaving the choice where it belongs to the user and the package writer. It makes no harm if a more flexible definition is provided. The user can then decide to only provide one or many fonts. 

This avoids complicated and almost unintelligible commands such as:

\begin{dispListing}
\setkomafont{subsection}{\usefont{T1}{fvm}{m}{n}}
\setkomafont{section}{\usefont{T1}{fvs}{b}{n}\Large}
\end{dispListing}

Here are some examples. 

\begin{texexample}{Serif and non-serif}{ex:fontfamily}
\cxset{chapter label font-family=serif, 
       chapter opening=anywhere}
\chapter{Serif font}
\lorem
\end{texexample}


\section{Floating and Alignment} 

This particular key bothered me, as the term \emph{float} has a different meaning in \latexe. However, to
be consistent with |css| terminology I have yielded to the temptation and included it.

\begin{docKey}[]{chapter float}{=\meta{left,center,right,none}}{no default, initial value \texttt{none}}
Key that controls the horizontal alignment of the \emph{chapter element}. I order for the
element to float, its |display| property must be set to |inline|.
\end{docKey}

%\begin{texexample}{Floating}{chapter:float}
%\cxset{chapter opening=anywhere, chapter float=center}
%\chapter{Centered Chapter}
%\lorem
%\cxset{chapter float=left}
%\chapter{Left Aligned}
%\lorem
%\cxset{chapter float=right}
%\chapter{Right Aligned}
%\lorem
%\end{texexample}


\subsection{The display property}

Both the |css| box model as well as the \TeX layout engine provide numerous complex algorithms in managing the floating of elements. This is normally controlled using two properties |display| and |float|.


\makeatletter

\begin{docKey}[phd]{chapter position}{ = \meta{absolute, relative}}{no default, initial value black}
This positioning directive instructs the engine to position the element at an exact position.
\end{docKey}



\tcbox[nobeforeafter]{$box_1$}\tcbox[nobeforeafter]{$box_2$}\tcbox[nobeforeafter]{$box_3$}\dotfill\tcbox[nobeforeafter]{$box_n$}
\tcbox[before skip=0.2cm, after skip=0pt, width=1cm, enlarge left by=10cm,width=5cm,enhanced,show bounding box]{title before element}
\tcbox[before skip=0pt, width=1cm, enlarge left by=10cm,width=5cm,enhanced,show bounding box]{
\tcbox{tb}\tcbox{title}\tcbox[nobeforeafter, width=1cm,]{tb}}
\tcbox[before skip=0pt, after skip=12pt, width=1cm, enlarge left by=10cm,width=5cm,enhanced,show bounding box]{\emph{title after} element \fbox{some}}
\makeatother

\begin{docKey}[phd]{chapter float}{=\meta{left,center,right,none}}{no default, initial value \texttt{none}}
Key that controls the horizontal alignment of the \emph{chapter element}. I order for the
element to float, its |display| property must be set to |inline|.
\end{docKey}
In document preparation systems or web page development the layout is user generated, i.e., the user is expected to type the html and the |css| will then specify as to how the page will be rendered by the browser. In our case for documents we can specify how we want the headings to look. The layout manager for each element, creates other associated elements, as shown for the title here. This way most layouts can be accomplished with the declarative visual language of the \pkgname{phd} package. 

\subsubsection{In-line elements}

When an element is specified as |inline| the rendering algorithm places the boxes after each other. This is widely used in |chapter elements| to render the number inline with the chapter name.
\medskip
\bgroup

\noindent
\tcbox[nobeforeafter,width=3cm, height=1cm]{Chapter}\tcbox[nobeforeafter]{twelve}
 
When the property is set as |block| the elements are stacked below each other.
\medskip

\tcbox{chapter  display=block   CHAPTER}
\tcbox{number display=block    TWELVE}

The elements can be considered to be enclosed in a \emph{ghost} element. If the property is set to float we
\begin{figure}[htbp]
\makeatletter
\parindent0pt\fboxsep0pt
\fbox{\vbox to 0pt{\hbox to \dimexpr(\textwidth)\relax{{\hss\tcbox[capture=minipage,width=5cm, height=2cm, top=0pt]{\raggedright number display=block\\ number float=right }}%
}%
}%
}\par
\vspace*{2cm}
\makeatother
\end{figure}
signalling to the layout engine that the element must be placed to the right of the page, as shown in the figure. 


\begin{figure}[htbp]
\makeatletter
\parindent0pt\fboxsep0pt
\fbox{\vbox to 0pt{\hbox to \dimexpr(\textwidth+2cm)\relax{{\hss\tcbox[capture=minipage,width=5cm, height=2cm, top=0pt]{\raggedright number display=block\\ \emph{element} float=right }
\tcbox[capture=minipage,width=5cm, height=2cm, top=0pt]{\raggedright \emph{element} display=block\\ \emph{element} float=right }
}%
}%
}%
}\par
\vspace*{2cm}
\makeatother
\end{figure}

\subsection{Absolute positioning}

Absolute positioning mode, will place an element at an exact position on the page. They are more difficult to
achieve and inflexible. 

\begin{docKey}{position}{=\meta{absolute},\meta{relative}}{no default, initial none}{}

\end{docKey}



This positioning directive instructs the engine to position the element at an exact position.


\begin{docKey}[]{chapter float}{=\meta{left,center,right,none}}{no default, initial value \texttt{none}}
Key that controls the horizontal alignment of the \emph{chapter element}. In order for the
element to float, its |display| property must be set to |inline|.
\end{docKey}
\egroup



\section{Number Element Keys}


\subsection*{Keys for numbering}

Chapter numbering follows that of the standard \LaTeX\ classes and is extended to cover some additional cases such as fully spelled out numbers. This of course is only good for languages that use the arabic numeralsn. For other languages numerals in different formats can be added with simple keys and without the need of \pkgname{polyglossia} or \pkgname{babel}. 

Note that the package uses Heiko Oberdiek's package \pkgname{alphalph} to allow for alphabetic numbering that extends beyond the normal 26 letters of the alphabet. Examples for numbering can be seen in \ref{ex:romannumbering}


\begin{docKey}[phd]{number numbering}{= \oarg{alph,Alph,roman,Roman,none,WORDS,words,none}}{default arabic}
Style of numbering.
\end{docKey}

\begin{marglist}
\item [arabic] Despite that the Arabs call what the West calls Arabic numbers Indian numbers, we provide the value arabic to have normal numbers printed.
\item [alph] Lowercase alphabetic numbering.
\item [Alph] Uppercase alphabetic numbering.
\item [roman] Lowercase roman numbering.
\item [Roman] Uppercase roman numbering.
\item [words] The number is in lowercase words.
\item [WORDS] The number is in uppercase literal numerals.
\item [Words] Prints the number in words and capitalizes the first letter, for example the number 21 will be printed as `Twenty One'\footnote{Currently limited to the first hundred numbers}.
\index{chapter design>numbering>words}
\item [ordinals] Prints the number as ordinal.
\item [Ordinals] Prints the number as Ordinal.
\item [ORDINALS] Prinst the number as ORDINALS.
\item [none] This is equivalent to using the star version of the command. It does not print any number and does not increment the chapter counter.\footnote{I am ambivalent about this, perhaps it will be better to increment it, as it can give a more general approach.}

\end{marglist}
\begin{texexample}{Literal Numbering}{ex:literal}
\cxset{chapter numbering=WORDS} 
\chapter{Literal numbering}
\lorem
\cxset{chapter numbering=words,chapter name=chapter}
\chapter{Literal numbering} 
\lorem
\end{texexample}




\cxset{chapter opening=anywhere, chapter numbering=Roman, chapter number font-shape=upshape}
\index{chapter design>numbering>roman}

\begin{texexample}{Setting up keys for numbering}{ex:romannumberingx}
\bgroup
\cxset{chapter format = traditional, 
       chapter name = CHAPTER, 
       chapter numbering = Roman,
       chapter label color = bgsexy}
\chapter{Roman numbering}
\lorem
\egroup
\end{texexample}





To emulate some old books we also offer an ordinal numbering scheme.

\begin{texexample}{Literal Numbering}{ex:ordinals}
\cxset{chapter numbering=ORDINALS} 
\chapter{Ordinals numbering}
\lorem
\cxset{chapter numbering=words,chapter name=chapter}
\chapter{Literal numbering} 
\lorem
\end{texexample}

\cxset{chapter numbering=arabic}

\subsection{Fonts and colors}
\begin{docKey}[phd]{number color}{=\meta{color name}}{no default, initial value \texttt{black}}
This key sets the color for the \textit{number element}. The color name is stored in %\cmd{\numbercolor@cx}.
The value in this chapter is %\makeatletter\texttt{\numbercolor@cx}\makeatother.
\end{docKey}

\begin{docKey}[phd]{number font-size}{=\meta{Huge, Large}}{no default, initial value \texttt{Huge}}
This sets the size for rendering the \textit{number element}. Use one of the predefined values, as described
in the section for the \emph{chapter} element.
Note that you can either use a command i.e, |number font-size=|\cmd{\huge} 
or the command name i.e., |number font-size=huge|. The latter is the recommended method.
\end{docKey}

Letter spacing can be achieved using the soul package in a combination with the key |spaceout|.
The following examples illustrate the usage.

\index[phdkeys]{{\ttfamily phd/chapter design test}}

%\begin{texexample}{Letter Spacing}{ex:letterspacing}
%\cxset{numbering=Roman,
%        % number letter-spacing=soul,
%        % chapter spaceout=soul,
%         %title spaceout=soul,
%         title font-size=Large,
%         title font-family=rmfamily,
%         title font-shape=scshape}
%\chapter{Letter Spacing}
%
%\lorem
%\end{texexample}

\begin{docKey}[phd]{chapter number letter-spacing}{=\meta{none, true, etc.}}{no default, initial value \texttt{none}}.
\end{docKey}

\begin{marglist}
\item[none] Default value no tracking is used and the letters are spaced as per the basic font information.
\item[inherit] Inherits the letter-spacing settings from the \emph{chapter} element.
\item[true] Letter spacing is employed, using the |soul| package.
\item[false] Alias for |none|.
\item[soul] The \pkgname{soul} package is used for letter-spacing.
\item[microtype] The \pkgname{microtype} package is used for letter-spacing. When the microtype package is used more fine tuning of parameters is available.
\end{marglist}

The example that follows, explains how the features offered by the \pkgname{microtype} package can be used to
set different tracking options.

\begin{texexample}{Microtypography}{micro}
\bgroup

\SetTracking
 [ no ligatures = {f},
 spacing = {600*,-100*, },
 outer spacing = {450,250,150},
 outer kerning = {*,*} ]
 { encoding = * }
 { 100 }

{\huge \textls{Chapter Twenty}}

\SetTracking
 [ no ligatures = {f},
 spacing = {600*,-100*, },
 outer spacing = {450,250,150},
 outer kerning = {*,*} ]
 { encoding = * }
 { 200 }
 
{\huge \textls{Chapter Twenty}}

\egroup
\end{texexample}


\hbox{\drawfontbox{\huge \upshape\textls(Chapter Twenty}}

\hbox{\drawfontbox{\huge \upshape\textls{Chapter Twenty}}}


\section{Styling the chapter title}

Similarly to the number and chapter styling keys exist for styling the chapter title. We summarize the available standard keys below:

\index{chapter design!labels!letter spacing}
\begin{texexample}{Styling the Title}{ex:title} 
\cxset{chapter numbering=arabic, chapter title font-shape=itshape}
\chapter{Chapter title}
\lorem
\end{texexample}


\begin{docKey}[phd]{chapter title font-family}{=\marg{family}}{no default, initial inherit document font}
Selects a predefined font family
\end{docKey}

\begin{texexample}{Title element font styling}{}
\cxset{chapter title font-family=sffamily}
\chapter{Title font family settings}
\lorem
\cxset{chapter title font-shape=itshape}
\chapter{Title font-style settings}
\lorem
\end{texexample}


\begin{docKey}[phd]{chapter title font-weight}{ = \marg{\cs{bfseries},\cs{normalseries}}} {}
Font weight.
\end{docKey}

\begin{docKey}[phd]{chapter title font-size}{= \marg{large, Large, huge, Huge, HUGE, HHuge}}{}
Font sizing commands or their names. Both \docAuxCommand{\HUGE} and HUGE are allowed to be used as values for the key.
\end{docKey}

\begin{docKey}[phd]{chapter title color} { = \marg{color}} {}
The color of the chapter title letters. This takes any predefined color name. 
\end{docKey}


\begin{docKey}[phd]{chapter title spaceout}{ = \marg{soul,none}} {no default, initial = none}
 This key will space out the title. 
\end{docKey}

\begin{texexample}{Title element spacing}{}
\cxset{chapter name=none,
       chapter numbering=none,
       chapter title font-size=Large,
       chapter title color=black,
       chapter title width=0.6\textwidth,
       %title spaceout=soul,
         }
\chapter{The Prehistoric Period in South-East Asia: 2300 BC--AD 400}        
\lorem 
    
\end{texexample}
\cxset{defaults}


\subsection*{Adding content before and after the title element}

Like all the other elements, the title element can be decorated with additional content,
before and after the text. There are two different forms. 

\begin{docKey}[phd]{title before}{=\marg{code}}{default none}
Contents before the title (vertical material)
\end{docKey}

\begin{docKey}[phd]{title after}{=\marg{code}}{default none}
Contents after the title (vertical material)
\end{docKey}

\begin{docKey}[phd]{title content before}{=\marg{code}}{default none}
Contents before the title (horizontal material)
\end{docKey}

\begin{docKey}[phd]{title content after}{=\marg{code}}{default none}
Contents after the title (horizontal material)
\end{docKey}

The difference between the two type of settings, consider the following situation. Assume you have a title that has a rule at the top and bottom and the text is surrounded by two ornaments. The surrounding ornaments will be inserted using the |title before content|, and the rules using the |title before| form. The |title before| is a full fledged element on its own. 

%{
%\hrule
%\centering
%*** Introduction ***
%\par
%\hrule
%}
%
%{
%\MakePercentComment
%\startlineat{200}
%\lstinputlisting{./styles/style13.tex}
%\MakePercentIgnore
%}



 
\begin{docKey}{/phd/ chapter title before skip}{= \marg{soul,none}}{}
Before title string skip.
\end{docKey}

\begin{docKey}{/phd/ chapter title after skip}{ = \marg{soul,none} }{}
After title string skip.
\end{docKey}

\lorem 
%
%\begin{texexample}{letter spacing the chapter title block}{ex:title3}
%
%\cxset{chapter spaceout=none,
%         numbering=arabic}
%         
%\chapter{Chapter Title Styling}
%\end{texexample}
%
%\end{document}



\cxset{chapter opening=right}
\section{Table of Contents}\index{table of contents!key settings}

Traditionally a chapter will be added to the Table of Contents if the \cs{chapter} command is issued. The starred version will not produce a number and will not add a contents line. Since we have adopted an approach where we use a key value interface we can dispense with the starred version of the command, by setting the \option{chapter toc} option to false. For example if we want to define a command for a ``Foreward'' or ``Epiloque'' without wishing them to be added to the table of contents we can use the following setting.\index{Foreward>definitions}\index{Epilogue>definitions}



\begin{texexample}{changing the chapter label name}{}
\cxset{chapter name=Chapteris, chapter numbering=arabic,}
\chapter{Foreward}
\lorem
\end{texexample}

Note that the key \option{numbering=none} still has to be set.


Please note that when \textbf{numbering=none} the chapter number is not available anymore and yo may have to reset it if required again. Although this might be seen as rather cumbersome than simply using \cs{chapter*} the advantage is consistency in the user interface and the use of appropriate semantic definitions for all sectioning commands thus achieving a bit more separation of context from style.


%\cxset{chapter toc=true}

\section{Defining styles}

Named styles can be defined using the standard \textsc{PGF} conventions. To define a style for the forward above we can use:

\begin{texexample}{}{}
\cxset{foreward/.style={chapter numbering=none,
          chapter name=none,
          chapter title font-size= Large,
          chapter title font-family= sffamily,
          chapter numbering=none}}
\cxset{foreward}
\chapter{Foreward.}
\lorem
\end{texexample}



\cxset{chapter numbering=arabic}
\section{Creating semantic names for commands and environments}

To keep our search for semantic commands and true separation of contents it is prudent to define some macros for typesetting the  `foreward' section.

\bgroup
\begin{texexample}{defining a \textit{Foreward} macro.}{}
\begin{lstlisting}
\cxset{foreward/.style={chapter toc=false,
          name=none,
          title font-size = Large,
          title font-family = sffamily,
          numbering=none}}
\newcommand\forewardname{foreward}
\expandafter\newenvironment\expandafter{\forewardname}{%
\cxset{foreward}\chapter{Foreward}}%
{}
\begin{foreward}
\lorem
\end{foreward}
\end{lstlisting}
\end{texexample}
\egroup

Notice the use of a new command \cmd{\forewardname} to allow for internationlization using Babel or other methods. One is tempted to let the English name, but a better approach perhaps is to define both.

\makeatletter



\DocInput{\jobname.dtx}
% \printindex
 \end{document}
 %
% 
%</driver>
% \fi
% 
%  \CheckSum{0}
%  \CharacterTable
%  {Upper-case    \A\B\C\D\E\F\G\H\I\J\K\L\M\N\O\P\Q\R\S\T\U\V\W\X\Y\Z
%   Lower-case    \a\b\c\d\e\f\g\h\i\j\k\l\m\n\o\p\q\r\s\t\u\v\w\x\y\z
%   Digits        \0\1\2\3\4\5\6\7\8\9
%   Exclamation   \!     Double quote  \"     Hash (number) \#
%   Dollar        \$     Percent       \%     Ampersand     \&
%   Acute accent  \'     Left paren    \(     Right paren   \)
%   Asterisk      \*     Plus          \+     Comma         \,
%   Minus         \-     Point         \.     Solidus       \/
%   Colon         \:     Semicolon     \;     Less than     \<
%   Equals        \=     Greater than  \>     Question mark \?
%   Commercial at \@     Left bracket  \[     Backslash     \\
%   Right bracket \]     Circumflex    \^     Underscore    \_
%   Grave accent  \`     Left brace    \{     Vertical bar  \|
%   Right brace   \}     Tilde         \~}
%
%
%
% \changes{1.0}{2013/01/26}{Converted to DTX file}
%
% \DoNotIndex{\newcommand,\newenvironment}
% \GetFileInfo{phd.dtx}
% 
%  \def\fileversion{v1.0}          
%  \def\filedate{2012/03/06}
% \title{The \textsf{phd} package.
% \thanks{This
%        file (\texttt{phd.dtx}) has version number \fileversion, last revised
%        \filedate.}
% }
% \author{Dr. Yiannis Lazarides \\ \url{yannislaz@gmail.com}}
% \date{\filedate}
%
%
% 
% ^^A\maketitle
% 
% ^^A\frontmatter
%  ^^A\coverpage{./images/hine02.jpg}{Book Design }{Camel Press}
%  \newpage
% ^^A\secondpage
% \pagestyle{empty}
%
%
% 
%
%
% \pagestyle{headings}
% \raggedbottom
%  ^^A\OnlyDescription
%
%  ^^A\StopEventually{\printindex}

% \CodelineNumbered
% \pagestyle{headings}
% 


%<*SECT>
% \subsection{Preliminaries}
%
%    Standard file identification. We first announce the package 
%	 and require that it be used with \LaTeX2e. 
%
%    \begin{macrocode}
\NeedsTeXFormat{LaTeX2e}[1994/12/01]%
\ProvidesPackage{phd-sections}[2015/1/13 v1.0 chapter head management (YL)]%
\let\ltxtoday\today
%    \end{macrocode}
% \chapter{Layout Engine}
%

% 
%  The layout engine is used to provide a declarative interface to all sectioning
%  and ancillary commands. All keys follow a common nomenclature to make them 
%  easy to remember.
% user level parameters,
%    \begin{macrocode}
\newdimen\fboxrule
\newdimen\fboxsep
\fboxrule.4pt
\fboxsep1pt
\newdimen\fboxseptop
\newdimen\fboxsepright
\newdimen\fboxsepbottom
\newdimen\fboxsepleft
\fboxseptop\fboxsep
\fboxsepright\fboxsep
\fboxsepbottom\fboxsep
\fboxsepleft\fboxsep
\newdimen\fboxruletop
   \fboxruletop\fboxrule
\newdimen\fboxruleright
   \fboxruleright\fboxrule
\newdimen\fboxrulebottom
   \fboxrulebottom\fboxrule 
\newdimen\fboxruleleft
   \fboxruleleft\fboxrule
%    \end{macrocode}
% \section{toks registers}
% We creat a number of toks registers for later usage.
%    \begin{macrocode}

\newtoks\chapterprelimtoks
\newtoks\chaptertoks
\newtoks\numbertoks
\numbertoks={}
\newtoks\titletoks
\newtoks\headingtoks 
\headingtoks={}
\newsavebox\numberbox
%    \end{macrocode}
%
% \section{Registers, booleans and preliminaries}
%
% Parametric definitions for chapters 
%    \begin{macrocode}
\@ifundefined{@openright}{%
  }{}
\def\afterindenton@cx{\def\afterindent@cx{\@afterindenttrue}}
\def\afterindentoff@cx{\def\afterindent@cx{\@afterindentfalse}}
%\edef\zeroboxalign@cx{c}
%    \end{macrocode}
%
% \subsection{Lengths}
%    \begin{macrocode}
\global\newlength\chaptermarginleft
    \setlength\chaptermarginleft{30pt}%
 

\gdef\chaptermarginleft@cx{0pt}
\def\chaptertitleblockalign{}
\newcounter{chapterdisplay} \setcounter{chapterdisplay}{0}
\newcounter{numberdisplay} \setcounter{numberdisplay}{0}
%
\gdef\numberbgcolor{spot!20}   
%    \end{macrocode}
% All chapter titles can be fully framed with borders. We create length
% registers for these and appropriate keys.
% (See style 87 for usage examples)
%    \begin{macrocode}

\ExplSyntaxOn
\gdef\chapter_number_color{blue},
%
	\dim_new:c {chapter_margin_top}
	\dim_new:c {chapter_margin_right}
	\dim_new:c {chapter_margin_bottom}
	\dim_new:c {chapter_margin_left}
	\dim_new:c {chapter_margin}
	%
	\dim_new:c {chapter_border_top_width}
	\dim_new:c {chapter_border_right_width}
	\dim_new:c {chapter_border_bottom_width}
	\dim_new:c {chapter_border_left_width}
	\dim_new:c {chapter_border}
	%
	\dim_new:c {chapter_padding_top}
	\dim_new:c {chapter_padding_right}
	\dim_new:c {chapter_padding_bottom}
	\dim_new:c {chapter_padding_left}
	\dim_new:c {chapter_padding}
%
  \tl_new:N \chapter_border_top_color
  \tl_new:N \chapter_border_right_color
  \tl_new:N \chapter_border_bottom_color
  \tl_new:N \chapter_border_left_color

%
  \dim_gzero_new:N \number_margin_top
  \dim_gzero_new:N \number_margin_right
  \dim_gzero_new:N \number_margin_bottom
  \dim_gzero_new:N \number_margin_left
%
  \dim_gzero_new:N \number_border_top_width
  \dim_gzero_new:N \number_border_right_width
  \dim_gzero_new:N \number_border_bottom_width
  \dim_gzero_new:N \number_border_left_width
%  
  \dim_new:N \number_padding_top
  \dim_new:N \number_padding_right
  \dim_new:N \number_padding_bottom
  \dim_new:N \number_padding_left  
%  
  \dim_gzero_new:N \title_margin_top
  \dim_gzero_new:N \title_margin_right
  \dim_gzero_new:N \title_margin_bottom
  \dim_gzero_new:N \title_margin_left
%
  \dim_gzero_new:N \title_padding_width
  \dim_gzero_new:N \title_padding_top_width
  \dim_gzero_new:N \title_padding_left_width
  \dim_gzero_new:N \title_padding_right_width
  \dim_gzero_new:N \title_padding_bottom_width 
% 
  \dim_gzero_new:N \title_border_width
  \dim_gzero_new:N \title_border_left_width
  \dim_gzero_new:N \title_border_top_width
  \dim_gzero_new:N \title_border_right_width 
  \dim_gzero_new:N \title_border_bottom_width 
%  
  \tl_new:N \chapter_title_display_tl
  \tl_new:N \chapter_title_float_tl
  \tl_gset:Nn \chapter_title_float_tl {center}
  \tl_new:c {chapter_title_text_align}  
% 
\ExplSyntaxOff  


%    \end{macrocode}
%

%
% \subsection{Shape handler} 
%    \begin{docCommand}{number_shape}{\meta{shape name}}
%      This handler obtains values for the shape attribute.
%    \end{docCommand} 
%
%    \begin{macrocode}
\ExplSyntaxOn
\clist_new:N \allowed_shape_options
\clist_gset:Nn \allowed_shape_options 
  {
    rectangle,rounded~rectangle,circle,ellipse, 
    diamond, starburst,none,star,custom
  }
  
  \pgfkeys{/handlers/.shape~is/.code = 
    \pgfkeysalso
      {\pgfkeyscurrentpath/.code=
        \clist_if_in:NnTF \allowed_shape_options {##1 } 
          {
           \gdef#1{##1}
          } 
          {
            \gdef#1{none} %also emit error message
          }
      }
  }
  
\ExplSyntaxOff  
%    \end{macrocode}
% \subsection{Border style option}
%    \begin{macrocode}
\ExplSyntaxOn
\clist_new:N \allowed_border_style_options
\clist_gset:Nn \allowed_border_style_options 
  {
    dash,solid,double,dotted,custom,none,
  }
  
  \pgfkeys{/handlers/.border~style~is/.code = 
    \pgfkeysalso
      {\pgfkeyscurrentpath/.code=
        \clist_if_in:NnTF \allowed_border_style_options {##1 } 
          {
           \gdef#1{##1}
          } 
          {
            \gdef#1{none} %also emit error message
          }
      }
  }
  
\ExplSyntaxOff  
%    \end{macrocode}
%
%

%





% \subsection {Chapter name handler}
%
%  Handler for chaptername to hook to i18n functions, if set to auto
%  If empty just typesets nothing.\FIRE 
%  
%    \begin{macrocode}
\ExplSyntaxOn
\clist_new:N \allowed_chapter_names_clist
\clist_gset:Nn \allowed_chapter_names_clist
  { auto, i18n, none}
\pgfkeys{/handlers/.getchaptername/.code = 
    \pgfkeysalso
      {\pgfkeyscurrentpath/.code=
        \tl_set:Nn\l_tmpa_str:N {##1}
           \str_case_x:nnTF {##1}  
             {
               { none } { \gdef#1{} } 
               { auto } { \gdef#1{\chaptername} } %hook to babel
               { i18n } { \gdef#1{\chaptername} } %hook to babl 
             }
             {            }
             {\gdef#1{#1} }
         
      }
  }

\ExplSyntaxOff   
%    \end{macrocode}
% 
% \section{Part keys and code}
%
% The Part section is pretty much similar to the chapter code. We redefine the 
% standard sectioning commands from the book class to provide for hooks and
% additional parameters. This will be the pattern for all the sectioning 
% commands that follow.
%
%    \begin{macrocode}
\ExplSyntaxOn
  \bool_new:N \part_open_left_bool     \bool_gset_false:N \part_open_left_bool
  \bool_new:N \part_open_any_bool      \bool_gset_false:N \part_open_any_bool
  \bool_new:N \part_open_anywhere_bool \bool_gset_false:N \part_open_anywhere_bool
  \bool_new:N \part_open_right_bool    \bool_gset_false:N \part_open_right_bool
\cxset
  {
    part~name/.store~in                      = \partname,
    part~color/.store~in                     = \part_color,
    part~background-color/.store~in          = \part_bgcolor,
    part~opening/.is~choice,
    part~opening/right/.code                 = \bool_gset_true:N \part_open_right_bool,
    part~opening/left/.code                  = \bool_gset_true:N \part_open_left_bool,
    part~opening/any/.code                   = \bool_gset_true:N \part_open_any_bool,
    part~opening/none/.code                  = \bool_gset_true:N \part_open_anywhere_bool,
    part~opening/anywhere/.code              = \bool_gset_true:N \part_open_anywhere_bool, 
    part~opening/ifafter/.code={},
    part~font-family/.font-family~in         = \part_font_family,
    part~font-weight/.font-weight~in         = \part_font_weight,
    part~font-size/.font-size~in             = \part_font_size,
    part~font-shape/.font-style~in           = \part_font_shape,
    part~font-style/.font-style~in           = \part_font_shape,
    part~case/.case~in                       = \part_case, 
    part~numbering/.numbering~in             = \thepart,
    part~title~text-align/.textalign         = \part_title_text_align,
    part~format/.store~in                    = \part_format,
    part~template/.style                     = {part~format=#1},
%    
    part~number~font-family/.font-family~in  = \part_number_font_family,
    part~number~font-weight/.font-weight~in  = \part_number_font_weight,
    part~number~font-size/.font-size~in      = \part_number_font_size,
    part~number~font-shape/.font-style~in    = \part_number_font_shape,
    part~number~font-style/.font-style~in    = \part_number_font_shape,
    part~number~color/.store~in              = \part_number_color,
} 
\ExplSyntaxOff  
\cxset
  {
    part opening                             = left,
    part color                               = black!90,
    part background-color                    = spot!30, 
    part name                                = PART,
    part font-size                           = Huge, 
    part font-weight                         = bold,
    part font-family                         = serif, 
    part font-shape                          = upshape,
    part number font-size                    = Huge, 
    part number font-weight                  = normal,
    part number font-family                  = sans-serif, 
    part number font-shape                   = upshape,
    part number color                        = thelightgray,
    part numbering                           = ordinals,
    part numbering                           = Roman,
    part title text-align                    = center,
    part format                              = box,  
    part template                            = stewartpart,
  }       
%    \end{macrocode}
%
% \begin{docCommand} {part} { \meta{void} }
%   The famous |\part| command can be used both for articles as well as books
%   here.
% \end{docCommand}
%
% As is normal with programming the user interface is much longer than the
% the code that does the actual work!
%    \begin{macrocode}
\ExplSyntaxOn
\renewcommand\part{%
  \bool_if:NTF \part_open_anywhere_bool {}
    {
      \bool_if:NT \part_open_right_bool { \cleardoublepage }
      \bool_if:NT \part_open_any_bool  { \clearpage       }
      \bool_if:NT \part_open_left_bool  { \clearpage       }
    }
  
  \thispagestyle{plain}
  \if@twocolumn
    \onecolumn
    \@tempswatrue
  \else
    \@tempswafalse
  \fi
  \null\vfil
  \secdef\@part\@spart}
\ExplSyntaxOff  
%    \end{macrocode}
%
% \begin{docCommand} {part} {\marg{short title} \marg{long title} }
% \end{docCommand}
%    \begin{macrocode}
\ExplSyntaxOn  
  \def\@part[#1]#2{%
    \ifnum \c@secnumdepth >-2\relax
      \refstepcounter{part}%
      \addcontentsline{toc}{part}{\thepart\hspace{1em}#1}%
    \else
      \addcontentsline{toc}{part}{#1}%
    \fi
    \markboth{}{}%
    \bgroup
    \tl_set:Nn\l_tmpa_str:N {box}
    \str_case_x:nnTF { \part_format  }  
     {
       { traditional} { \format_part_traditional:nn { part } { #1 } { #2 } }
       { box        } { \format_head_boxed:nn       { part } { #1 } { #2 } } 
       { inline     } { \format_head_inline:nn      { part } { #1 } { #2 } }
       { inmargin   } { \format_head_inmargin:nn    { part } { #1 } { #2 } } 
      }
      {                                            }
      { \cs:w stewartpart\cs_end: {part} {#1}  }
    \egroup
    \@endpart
}
 
    
            
\def\@spart#1{%
    {\centering
     \interlinepenalty \@M
     \normalfont
     \Huge \bfseries #1\par}%
     \@endpart}

\def\@endpart{%
              %\vfil %\newpage %was here
              \if@twoside
               \if@openright
                \null
                \thispagestyle{empty}%
               \fi
              \fi
              \if@tempswa
                \twocolumn
              \fi} 
\ExplSyntaxOff                 
%    \end{macrocode}
%
% \begin{docCommand} {format_part_traditional:nn} { \marg{part name } \marg{ unused} \marg{title }}
%  Formats a traditional part, set on its own page and centeres
% \end{docCommand}
%    \begin{macrocode} 
\ExplSyntaxOn
\cs_new:Npn \format_part_traditional:nn #1 #2 #3
  {
   \group_begin:
    \centering
     \interlinepenalty \@M
     \normalfont
     \set_color:nn {#1_number}{color}
     \set_font_parameters:n {#1_number}
     \ifnum \c@secnumdepth >-2\relax
       \partname\nobreakspace\thepart
       \par
       \vskip 20\p@
     \fi
     \group_begin:
       \set_color:nn {#1}{color}
       \set_font_parameters:n { #1 }
          #3
     \group_end:
     \par
   \group_end:  
  }
 
\cs_new:Npn \format_head_inmargin:nn #1 #2 #3
  {
   \tcbdocmarginnote
     {
       \group_begin:
         \parindent0pt 
         \language-1
         \normalfont
         %\set_color:nn {#1_number}{color}
         \color{blue}
         \set_font_parameters:n {#1_number}
         \ifnum \c@secnumdepth >-2\relax
           \partname\nobreakspace\thepart\\[2pt]
         \fi
         \group_begin:
           %\set_color:nn {#1_number}{color}
           \color{blue} 
           \set_font_parameters:n { #1 }
           \nobreakspace#3
         \group_end:
         \group_end:
    }  
  }
\ExplSyntaxOff  
%    \end{macrocode}
%
% \begin{docCommand} { format_part_boxed } { }
% \end{docCommand}
%
%    \begin{macrocode}
\ExplSyntaxOn
\cs_new:Npn \format_head_boxed:nn #1 #2 #3
  {
    \centering
    \begin{tcolorbox}[colback=\part_bgcolor, 
                              colframe=white, 
                              arc=5mm]
     \bgroup                         
     \interlinepenalty \@M
     \normalfont
     \color{\part_color}
     \ifnum \c@secnumdepth >-2\relax
       \huge\bfseries \partname\nobreakspace\thepart
       \par
       \vskip 20\p@
     \fi
     \set_font_parameters:n { #1 }
      #3\par
      \egroup
     \end{tcolorbox} 
  }
\ExplSyntaxOff  
%    \end{macrocode}
%
% \begin{docCommand} { set_property_from_section_name:nn } 
%                         { \marg{name of section}  \marg{suffix} }
%  Given a section name such as part gets its property.
% \end{docCommand}
%
%    \begin{macrocode}
\ExplSyntaxOn
\cs_new:Npn \get_property_from_section_name:nn #1 #2
  {
    \csname\expandafter\csname #1#2\endcsname\endcsname
  }
\cs_new:Npn \set_font_parameters:n #1
  {
     \get_property_from_section_name:nn {#1}{_font_size}
     \get_property_from_section_name:nn {#1}{_font_weight}
     \get_property_from_section_name:nn {#1}{_font_shape}
     \get_property_from_section_name:nn {#1}{_font_family}
  }

\cs_new:Npn \get_color_property:nn #1 #2  
  {
     \cs:w #1_#2 \cs_end:
  }    
\cs_new:Npn \set_color:nn #1 #2  
  {
     \color{\cs:w #1_#2 \cs_end:}
  }  
\ExplSyntaxOff  
%    \end{macrocode}
%
%
%    \begin{macrocode}  
\ExplSyntaxOn
\cs_new:Npn \format_head_inline:nn #1 #2 #3
  {
    
     \begin{tcolorbox}[colback= \get_color_property:nn {#1}{bgcolor}, 
                                colframe=white, 
                                arc=3mm]
     \interlinepenalty \@M
     \normalfont
     \set_color:nn {#1}{color}
     \cs:w #1_title_text_align \cs_end:
     \ifnum \c@secnumdepth >-2\relax
       {
         \set_font_parameters:n {part_number} 
         \set_color:nn {part_number} {color}
         \partname\nobreakspace\thepart
       }
     \fi
       \set_font_parameters:n {#1} 
     \nobreakspace #3
     \par
     \end{tcolorbox} 
  }
\ExplSyntaxOff  
%    \end{macrocode}
%
%    \begin{macrocode}
\def\gluestart{\hss}\def\glueend{\hss}
%    \end{macrocode}
% \section{Chapter keys}
%
% It is not envisioned that the chapter name key be set directly by the user.
% This should be set by the document language tag (like Babel). 
% However, the user might decide that this is an easier approach.
%  \begin{docCommand}{chaptername}{ \meta{void}} 
%    Holds the chapter name tag such as Chapter or 'CHAPTER' or whatever
%    is typed by the user. 
%    This is also hooked here for i18n routines later, hence we get it via a handler.
% 
%  \end{docCommand}
%    \begin{macrocode}

\ExplSyntaxOn 
\cxset
  {
    chapter~name/.getchaptername                = \chapternameint,
    chapter~color/.store~in                     = \chapter_color,
    chapter~background-color/.store~in          = \chapter_bgcolor,
    number~background-color/.store~in           = \numberbgcolor,
}
\ExplSyntaxOff

\cxset{chapter name= Chapter} 
\cxset{    
  chapter opening/.is choice,
  chapter opening/right/.code={\@openrighttrue},
  chapter opening/left/.code={\@openlefttrue},
  chapter opening/any/.code={\@openanytrue},
  chapter opening/none/.code={\@openanywheretrue\@openrightfalse%
                                                  \@openleftfalse\@openanyfalse},
  chapter opening/anywhere/.code={\@openanywheretrue\@openrightfalse
     \@openleftfalse\@openanyfalse},
  chapter opening/ifafter/.code={},
}
%    \end{macrocode}


%  The font options use handlers to get the values. This alows for more flexibiliy.
%    \begin{macrocode}
\ExplSyntaxOn
\cxset{%    
  chapter~font-family/.font-family~in    = \chapter_font_family,
  chapter~font-weight/.font-weight~in    = \chapter_font_weight,
  chapter~font-size/.font-size~in        = \chapter_font_size,
  chapter~font-shape/.font-style~in      = \chapter_font_shape,
  chapter~font-style/.font-style~in      = \chapter_font_shape,}
\ExplSyntaxOff  
%    \end{macrocode}
%
% \subsection{Chapter display and float properties}
%
% This generates keys for float and display. The attribute display determines if the
% element is on a line of its own or not. The float determines glue to be
% set to float the element left or right.
%
%    \begin{macrocode}
\newcounter{lastelementfloat}
     \setcounter{lastelementfloat}{-1}
\newcounter{chapterfloat} 
      \setcounter{chapterfloat}{1}  
\newcounter{numberfloat} 
      \setcounter{numberfloat}{1}        
\newcounter{currentelementfloat}
      \setcounter{currentelementfloat}{-1}
%
\global\newlength\chapterborderrightwidth
    \setlength\chapterborderrightwidth{2pt} 
\global\newlength\chapterborderleftwidth
    \setlength\chapterborderleftwidth{2pt}     
\global\newlength\chapterborderbottomwidth
    \setlength\chapterborderbottomwidth{2pt}  
\global\newlength\chapterbordertopwidth
    \setlength\chapterbordertopwidth{2pt}             
%
\global\newlength\chapterpaddingleft
    \setlength\chapterpaddingleft{10pt}
\global\newlength\chapterpaddingright
    \setlength\chapterpaddingright{10pt}  
\global\newlength\chapterpaddingtop
    \setlength\chapterpaddingtop{10pt}        
\global\newlength\chapterpaddingbottom
    \setlength\chapterpaddingbottom{10pt}  
         
\ExplSyntaxOn
\int_zero_new:c {chapterdisplaycounter}
\int_zero_new:c {chapterfloatcounter}
%    \end{macrocode}
%
% \begin{docCommand}{phdsetcounter}{ \marg{counter name} \marg{int value} }
%    Sets an integer counter to a value.
% \end{docCommand}
%    \begin{macrocode}
\cs_gset:Npn \phd_set_counter:nn #1 #2 
 {
   \int_gset:cn {#1} {#2}
 }  
\ExplSyntaxOff
%    \end{macrocode}
%
% \subsubsection{Chapter name floating and display properties}
%
%    \begin{macrocode}
\ExplSyntaxOn
\cxset{%  
  chapter~display/.is~choice,
  chapter~display/inline/.code    =\global\setcounter{chapterdisplay}{0}
                               \phd_set_counter:nn {chapterdisplaycounter}{0},
  chapter~display/block/.code     =\global\setcounter{chapterdisplay}{2}
                               \phd_set_counter:nn {chapterdisplaycounter}{2}, 
  chapter~display/none/.code      =\global\setcounter{chapterdisplay}{0}
                               \phd_set_counter:nn {chapterdisplaycounter}{0},                              
%    
  chapter~float/.is~choice,
  chapter~float/none/.code        = \global\setcounter{chapterfloat}{0}%
                            \phd_set_counter:nn {chapterfloatcounter}{0},
  chapter~float/left/.code= \global\setcounter{chapterfloat}{0}
                            \phd_set_counter:nn {chapterfloatcounter}{0},
% center = 1                             
  chapter~float/center/.code= \global\setcounter{chapterfloat}{1}%
                             \phd_set_counter:nn {chapterfloatcounter}{1}, 
% right = 2                             
  chapter~float/right/.code= \global\setcounter{chapterfloat}{2}%
                             \phd_set_counter:nn {chapterfloatcounter}{2},%   
 }
\ExplSyntaxOff 
\cxset{chapter display=block,
       chapter float=left,
       } 
%    \end{macrocode}
%
%
% \subsection{Chapter content before and after}
%
%    \begin{macrocode}   
\ExplSyntaxOn
\cxset
  {
    chapter~before~content/.store~in        = \chapter_before_content,
    chapter~before/.store~in                = \chapter_before,
    chapter~before~content = ,
    chapter~before=,
  }
\ExplSyntaxOff 
%    \end{macrocode}
% 
% \subsection{Chapter margins and padding}
%
% The dual code is interim we will avoid all these in the future
%    \begin{macrocode}
\ExplSyntaxOn

\cxset{   
  chapter~margin-top/.code= \dim_gset:cn { chapter_margin_top } { #1 },
  chapter~margin-left/.code=\setlength\chaptermarginleft{#1}%
                            \global\chaptermarginleft\chaptermarginleft\relax
                            \def\gluestart{\hskip#1}%
                            \def\glueend{\hss}
                            \dim_gset:cn {chapter_margin_left}{#1},
  chapter~margin-right/.code = \dim_gset:cn {chapter_margin_right} { #1 },
  chapter~margin-bottom/.code = \dim_gset:cn {chapter_margin_bottom} { #1 },                           
  }

\ExplSyntaxOff 
%    \end{macrocode}
%
% \subsection{Chapter borders}
%
% Next we set keys for all border width
% \subsubsection{Chapter border widths}
%    \begin{macrocode}      
\ExplSyntaxOn
\cxset{  
  chapter~border-top-width/.code    = \dim_gset:cn {chapter_border_top_width} {#1},                                            
  chapter~border-right-width/.code  = \dim_gset:cn {chapter_border_right_width} {#1},                                                                                      
  chapter~border-bottom-width/.code = \dim_gset:cn {chapter_border_bottom_width} {#1},                                                                                                                              
  chapter~border-left-width/.code   = \dim_gset:cn {chapter_border_left_width} {#1},
 }                        
\cxset{  
  chapter~border-width/.code = \pgfkeysalso{chapter~border-top-width=#1,
                                            chapter~border-right-width=#1,
                                            chapter~border-bottom-width=#1,
                                            chapter~border-left-width=#1,
                                            },
}            
\ExplSyntaxOff                                     
%    \end{macrocode}  
% 
% \subsubsection{Chapter padding}
% We now deal with padding the same way including the generic version                                             
%    \begin{macrocode}  
\ExplSyntaxOn                                             
\cxset{
  chapter~padding-left/.code   = \dim_gset:cn {chapter_padding_left}{#1},                                                                                                                           
  chapter~padding-right/.code  = \dim_gset:cn {chapter_padding_right}{#1},                                                                                                             
  chapter~padding-top/.code    = \dim_gset:cn {chapter_padding_top}{#1},                                                                                                                                                       
  chapter~padding-bottom/.code = \dim_gset:cn {chapter_padding_bottom}{#1},                                                                                                                                                                                                 
}
\ExplSyntaxOff
%    \end{macrocode}
% In retrospect this should go in a handler
%    \begin{macrocode}
\cxset{
  chapter padding/.code={
    \def\@tempa{none}%
    \def\@tempb{#1}%
    \ifx\@tempa\@tempb%
      \global\setlength\chapterpaddingleft{0pt}%
      \global\setlength\chapterpaddingright{0pt}%
      \global\setlength\chapterpaddingtop{0pt}%
      \global\setlength\chapterpaddingbottom{0pt}%
    \else
      \setlength\chapterpaddingleft{#1}%
      \global\chapterpaddingleft\chapterpaddingleft\relax                                                                                                                           
      \setlength\chapterpaddingright{#1}%
      \setlength\chapterpaddingtop{#1}%
      \setlength\chapterpaddingbottom{#1}%
    \fi}
}    

%    \end{macrocode}
%
% \subsubsection{Chapter border colors}
%
% The series of keys denoted by \meta{chapter}\meta{border}\meta{top}\meta{color}
% are used to store the colors of borders. We need to be careful here not to get
% any expansion problems.

%    \begin{macrocode}
\ExplSyntaxOn
\cxset{chapter~border-top-color/.code    = \tl_gset:Nn \chapter_border_top_color {#1},
  chapter~border-right-color/.code       = \tl_gset:Nn \chapter_border_right_color {#1},
  chapter~border-bottom-color/.code      = \tl_gset:Nn \chapter_border_bottom_color{#1},
  chapter~border-left-color/.code        = \tl_gset:Nn \chapter_border_left_color {#1},
  }
 \cxset{%
  chapter~border-top-color=sweet,
  chapter~border-right-color=sweet,
  chapter~border-bottom-color=sweet,
  chapter~border-left-color=sweet,
}%
%                            
\cxset{chapter~border-color/.code=\pgfkeysalso{chapter~border-top-color={#1},%
                                               chapter~border-right-color={#1},
                                               chapter~border-bottom-color={#1},
                                               chapter~border-left-color={#1}}}%
%                                                           
 % set some defaults                                                                
\cxset{%
  chapter~border-top-color=sweet,
  chapter~border-right-color=sweet,
  chapter~border-bottom-color=sweet,
  chapter~border-left-color=white,
  chapter~border-color=blue,
 }%
\ExplSyntaxOff          
%    \end{macrocode}    
%
% \subsubsection{Chapter letter spacing} 
% NEEDS REVISITING TO ALLOW FOR  SOUL OR MICROTYPE  LEAVE ALSO LETTER SPACING 
% ALSO TO TAKE OUT SPACEOUT
%    \begin{macrocode}   
                                   
\cxset{
  chapter after/.store in=\chapterafter@cx,
  chapter spaceout/.is choice,
  chapter spaceout/soul/.code=\@chapterspaceouttrue\@soulspaceouttrue,
  chapter spaceout/microtype/.code=\@chapterspaceouttrue\@soulspaceouttrue,
  chapter spaceout/none/.code=\@chapterspaceoutfalse\@soulspaceoutfalse,
  %  
  chapter letter-spacing/.is choice,
  chapter letter-spacing/soul/.style=\pgfkeysalso{chapter spaceout=soul},
  chapter letter-spacing/microtype/.style=\pgfkeysalso{chapter spaceout=microtype},
  chapter letter-spacing/true/.code=\@chapterspaceouttrue,
  chapter letter-spacing/none/.code=\@chapterspaceoutfalse,
  chapter letter-spacing/false/.code=\@chapterspaceoutfalse,
 }  
%    \end{macrocode}  
%
%  Next we define styles. This must be distinguished from shapes and only
%  apply to rectangular boxed content, using \cmd{\phd@fbox}
%
%    \begin{macrocode}
\ExplSyntaxOn
\tl_new:c {chapter_border_top_style}
\tl_new:c {chapter_border_right_style}
\tl_new:c {chapter_border_bottom_style}
\tl_new:c {chapter_border_left_style}
\cxset{
  chapter~border-top-style/.code      = \tl_gset:cn {chapter_border_top_style}{#1}, 
  chapter~border-right-style/.code    = \tl_gset:cn {chapter_border_right_style}{#1}, 
  chapter~border-bottom-style/.code   = \tl_gset:cn {chapter_border_bottom_style}{#1},
  chapter~border-left-style/.code     = \tl_gset:cn {chapter_border_left_style}{#1}, 
  chapter~border-style/.code          = \pgfkeysalso{chapter~border-top-style=#1,%
                              chapter~border-right-style=#1,%
                              chapter~border-bottom-style=#1,%
                              chapter~border-left-style=#1%,
  },
}
\ExplSyntaxOff
\cxset{
  chapter border-top-style=solid,
  chapter border-right-style=solid,
  chapter border-bottom-style=solid,
  chapter border-left-style=solid}  
%    \end{macrocode}

%
%  \begin{docCommand}{chaptershape} {\meta{shape name}}
%  Defines the shape for the Chapter
%  \end{docCommand}
%
%    \begin{macrocode}
%
\ExplSyntaxOn
  \cxset{chapter~shape/.shape~is = \chaptershape, }
  \cxset{chapter~shape = diamond}
\ExplSyntaxOff
%  
%    \end{macrocode}
%
% \cxset{chapter shape = diamond}
% \subsection{Chapter title keys}
% 
%   We define  key-sets for the chapter title block. The text is typeset in a minipage
%   of width \docAuxCommand {chapter_title_text_width}. Any borders and padding
%   are added by the layout engine and then the block is aligned as per the rules 
%   and settings described later on.
% 
%    \begin{macrocode}  
\ExplSyntaxOn
\dim_new:N \chapter_title_text_width
\cxset{ 
  chapter~title~width/.code =  \dim_gset:Nn \chapter_title_text_width {#1},
  title~text-width/.style = { chapter~title~width= {#1} }
}    
\ExplSyntaxOff
%    \end{macrocode}
%
%  Next we deal with the title alignment. The title is typeset in a minipage
%  We allow for the total to be positioned. The key text-align specifies the alignment
%  of the inner text block. 
%
%  TeX does not distinguish the type of boxes found in CSS. As a matter of fact TeX’s model
%  is much more complicated and also allows the different types to be nested indefinetly.
%  Rendering depends on the typesetting mode. 
%  The display block, should just add |\vskip|s and terminate horizontal mode. This might
%  avoid to have to type some keys.
%
% \begin{docKey}{title display}{ = \oarg{none,block,inline,inline-block}}{default block}
% The title display key determines how the title is aligned with its neighbours.
% It defaults to block, which it means is typeset on its own line.
%
%  |chapter_title_display = none|   none is left left aligned\\
%  |chapter_title_display = block|   block typeset in minipage\\
%  |chapter_title_display = in-line block|   in-line block minipage but cannot float\\
%  |chapter_title_display = inline|   inline equivalent to 0 consider removing \\ 
% \end{docKey}
%    \begin{macrocode}
\ExplSyntaxOn
%
\cxset{
  title~display/.is~choice,
  title~display/none/.code          = \tl_gset:Nn \chapter_title_display_tl{none},
  title~display/block/.code         = \tl_gset:Nn \chapter_title_display_tl{block},
  title~display/in-line block/.code = \tl_gset:Nn \chapter_title_display_tl{in-line block},
  title~display/inline/.code        = \tl_gset:Nn \chapter_title_display_tl{inline},
 }
\ExplSyntaxOff
\cxset{title display=block}  %!REMOVE PUT AT DEFAULTS
%    \end{macrocode}
% 
%  \begin{docCommand}{chapter_title_float_tl} {\meta{option}}
%  This key determines if the title block can float. This is used together 
%  with the display property described above. For an element to float the |title  display| must be block
%  and the |title| float to one of |left|, |right| or |center|. The last one is a
%  heretical departure from the css standard model, which uses |margin:auto| for this. 
%  Not too difficult to incorporate, maybe I should do this at the next version.
%  \end{docCommand}
% 
%    \begin{macrocode}
\ExplSyntaxOn
\cxset{
  title~float/.is~choice,
  title~float/none/.code   = \gdef\chapter_title_float_tl {none},    
  title~float/left/.code   = \gdef\chapter_title_float_tl {left},
  title~float/right/.code  = \gdef\chapter_title_float_tl {right},
  title~float/center/.code = \gdef\chapter_title_float_tl {center},
}
\ExplSyntaxOff
\cxset{title float=center}
%    \end{macrocode}
% 
% 
%  \begin{docCommand}{chapter_title_text_align} {\meta{void}}
%  Key setting to set inner block text alignment. This allows for more options than normal css.
%  For example we can do russian last line alignment. 
%  
% \end{docCommand}  
%
%    \begin{macrocode}  
\ExplSyntaxOn
\cxset{
  chapter~title~text-align/.textalign = \chapter_title_text_align,
  %/.is~choice,
%  chapter~title~text-align/center/.code = \tl_gset:cn {chapter_title_text_align}
%      {\Centering},                                                                                                 
%  chapter~title~text-align/centering/.code= \tl_gset:cn {chapter_title_text_align}
%      {\centering},
%  chapter~title~text-align/Centering/.code= \tl_gset:cn {chapter_title_text_align}
%      {\Centering},                                                                                                        
%  chapter~title~text-align/none/.code = \tl_gset:cn {chapter_title_text_align}{},                                                                                                 
%  chapter~title~text-align/justified/.code = \tl_gset:cn {chapter_title_text_align}{},
%  chapter~title~text-align/left/.code =  \tl_gset:cn {chapter_title_text_align}
%      {\RaggedRight},
%  chapter~title~text-align/raggedleft/.code =  \tl_gset:cn {chapter_title_text_align}
%      {\RaggedLeft},
%  chapter~title~text-align/right/.code =  \tl_gset:cn {chapter_title_text_align}
%      {\RaggedLeft},
%  chapter~title~text-align/raggedright/.code = \tl_gset:cn {chapter_title_text_align}
%     {\RaggedRight},
}

%
\cxset{chapter~title~text-align=left}

\tl_new:c {chapter_title_align}
\cxset{    
  % aligning the block title 
  chapter~title~align/.is~choice,
  chapter~title~align/centering/.code=
   \tl_gset:cn {chapter_title_align}{centering},
  % alias
  chapter~title~align/center/.style= {chapter~title~align=#1} , 
%   
  chapter~title~align/raggedright/.code=,
%  
  chapter~title~align/raggedleft/.code=
    \tl_gset:cn {chapter_title_align}{raggedleft},
%                                                          
  chapter~title~align/right/.code=                                                      
    \tl_gset:cn {chapter_title_align}{right},  
%  
  chapter~title~align/left/.code=
    \tl_gset:cn {chapter_title_align}{left},
%                                                 
  chapter~title~align/none/.code=
     \tl_gset:cn {chapter_title_align}{none},
}

%  
\ExplSyntaxOff
\cxset{chapter title align=centering}
%
%    \end{macrocode}
%
% \begin {docCommand} {title_font_family} {\meta{font family name}}
%    These keys set the font parameters for the element. These are high level commands
%    \pkgname{fontspec} can be used for lower level fine-tuning, such as tags for scripts,
%    languages and other features.
% \end{docCommand}
%    \begin{macrocode}
\ExplSyntaxOn
\cxset{
  title~font-face/.font-face~in               = \title_font_face, 
  title~font-family/.font-family~in           = \title_font_family,
  title~font-weight/.font-weight~in           = \title_font_weight,
  title~font-size/.font-size~in               = \title_font_size,
  title~font-color/.store~in                  = \titlefontcolor@cx,
  title~font-shape/.font-style~in             = \title_font_shape
}
\ExplSyntaxOff
%
\cxset{title font-shape=upshape}
\cxset{title font-face=pan}
%    \end{macrocode}
%
%  Letter-spacing is handled in a similar fashion defining keys both for 
%  the common \latex community terminology (spaceout) and also
%  using |letter-spacing|.
%
%  
%    \begin{macrocode}
\cxset{  
  title spaceout/.is choice,
  title spaceout/soul/.code                  = \@titlespaceouttrue,
  title spaceout/none/.code                  = \@titlespaceoutfalse,
  title spaceout/true/.code                  = \@titlespaceouttrue,
  title spaceout/false/.code                 = \@titlespaceoutfalse,
  title letter-spacing/true/.code            = \@titlespaceouttrue,
  title letter-spacing/false/.code           = \@titlespaceoutfalse,  
}
%    \end{macrocode}
%
%    \begin{macrocode}
\cxset{  
  title font/.style={title font-family=#1},
  title before/.store in=\titlebefore@cx,
  title after/.store in=\titleafter@cx,
  title beforeskip/.store in=\titlebeforeskip@cx,
  }
%    \end{macrocode}
%
% \begin{docKey}{title margin-top}{ = \meta{dim}} {default 0pt}
% This family of keys are used to set the title margins. Note these are outside 
% the containing block and for the top and bottom, we need to be in vertical mode.
% \end{docKey}
%
%   \begin{macrocode}
\ExplSyntaxOn


\cxset  {          
  title~margin-top/.code    = \dim_gset:Nn \title_margin_top {#1},
  title~margin-right/.code  = \dim_gset:Nn \title_margin_right {#1},
  title~margin-bottom/.code = \dim_gset:Nn \title_margin_bottom {#1},
  title~margin-left/.code   = \dim_gset:Nn \title_margin_left {#1},
 
}    
\ExplSyntaxOff    

%    \end{macrocode}
%
% \begin{docCommand}{title_padding_top_width} { \meta{dim} }  
%    Handles all the padding settings. The padding is added outside the minipage
%    and before the border. It is equivalent to \tikzname |inner sep| and to |\fbox|'s 
%    |\fboxsep|, so no fire works here or browser wars.
% \end{docCommand}
% 
%    \begin{macrocode}
\ExplSyntaxOn
\cxset{  
   title~padding-top/.code    = \dim_gset:Nn \title_padding_top_width{#1},
   title~padding-bottom/.code = \dim_gset:Nn \title_padding_bottom_width{#1} ,
   title~padding-left/.code   = \dim_gset:Nn \title_padding_left_width{#1},
   title~padding-right/.code  = \dim_gset:Nn \title_padding_right_width{#1},
   title~padding/.style       = {title~padding-top=#1,
                                   title~padding-right=#1,
                                   title~padding-bottom=#1,
                                   title~padding-left=#1,
                                 }, 
}
\ExplSyntaxOff
%    \end{macrocode}

%
% \begin{docCommand}{title_border_top_width}{\marg {dim} }
%    Handles all the settings for the border widths. The \docAuxCommand{title_border_width}
%    sets all borders to one value.
% \end{docCommand}
%    \begin{macrocode}  
\ExplSyntaxOn  
\cxset {                                   
  title~border-top-width/.code      = \dim_gset:Nn \title_border_top_width {#1},
  title~border-right-width/.code    = \dim_gset:Nn \title_border_right_width {#1},
  title~border-left-width/.code     = \dim_gset:Nn \title_border_left_width {#1},
  title~border-bottom-width/.code   =\dim_gset:Nn \title_border_bottom_width {#1},                             
  title~border-width/.code= \dim_gset:Nn \title_border_width{#1}
                                           \dim_gset:Nn\title_border_top_width{#1}
                                           \dim_gset:Nn \title_border_right_width{#1}
                                           \dim_gset:Nn \title_border_bottom_width{#1}
                                           \dim_gset:Nn \title_border_left_width{#1},
}                     
\ExplSyntaxOff  
%    \end{macrocode}
%
% \begin{docCommand} { title_border_color } { \meta{color} }
%   This family of commands sets colors for titles.
% \end{docCommand}
%
%    \begin{macrocode} 
\ExplSyntaxOn
\tl_new:N \title_border_color
\tl_new:N \title_border_top_color
\tl_new:N \title_border_right_color
\tl_new:N \title_border_bottom_color
\tl_new:N \title_border_left_color
\cxset{  
  title~border-left-color/.code     = \tl_gset:Nn \title_border_left_color {#1},
  title~border-top-color/.code      = \tl_gset:Nn \title_border_top_color {#1},
  title~border-right-color/.code    = \tl_gset:Nn \title_border_right_color {#1},
  title~border-bottom-color/.code   = \tl_gset:Nn \title_border_bottom_color {#1},
% 
  title~border-color/.code=\tl_gset:Nn \title_border_color{#1}%
                           \tl_gset:Nn \title_border_left_color{#1}%
                           \tl_gset:Nn \title_border_right_color{#1}%
                           \tl_gset:Nn \title_border_top_color{#1}%
                           \tl_gset:Nn \title_border_bottom_color{#1},
}
\ExplSyntaxOff
%    \end{macrocode}
%
%\cxset{title border-color=blue}
%\ExplSyntaxOn
%
%\ExplSyntaxOff

%    \begin{macrocode}
\cxset {
  %title margin-bottom/.style =\pgfkeysalso{title margin bottom=#1},      
%  title margin-left/.code=\global\setlength{\titlemarginleft}{#1}
%                                      \gdef\titlemarginleft@cx{\hspace*{#1}},%,
  title afterskip/.store in=\titleafterskip@cx,
  position/.is choice,
  position/left/.code={\@lefttrue},
  position/right/.code={\@righttrue},
  position/center/.code={\@centertrue},
}

%                  
%    \end{macrocode}
%
% \section{The number element keys}
% The numbering keys deal with the typesetting of the chapter number
% in the chapter head. We use two packages for expressing numbers into
% words. The padzeroes is to produce EWD style notes. 
%
% We define a number of shorter aliases. This also for some legacy code using it.
%    \begin{macrocode}
\cxset{
  numbering/.numbering in = \thechapter,
  chapter numbering/.numbering in = \thechapter,
%  numbering/padzeroes/.code={\gdef\thechapter{\mbox{EWD -\padzeroes[4]\decimal{chapter}}
%  }},
%  numbering/ORDINALS/.code=\gdef\thechapter{%
%  \expandafter\ordinals@cx{\@arabic\c@chapter}},
}
%    \end{macrocode}
%
%  Next we get the spaceout keys going
%
%    \begin{macrocode}
\cxset{  
  number spaceout/.is choice,
  number spaceout/soul/.code=\@numberspaceouttrue,
  number spaceout/none/.code=\@numberspaceoutfalse,
  number spaceout/inherit/.code=\let\@numberspaceout\@chapterspaceout,
  number spaceout/microtype/.code=\@numberspaceouttrue,
  number letter-spacing/.code=\pgfkeysalso{number spaceout=soul},
  number dot/.store in=\numberpunctuation@cx,}
%    \end{macrocode}
% The position keys will be dropped soon!
%    \begin{macrocode}  
\cxset{  
  number position/.is choice,
  number position/leftname/.code={\@leftnametrue\@rightnamefalse},
  number position/rightname/.code={\@rightnametrue\@leftnamefalse},
  number position/absolute/.code={},
  number position/righttitle/.code=\@righttitletrue,
  number position/lefttitle/.code=\@lefttitletrue,
}
%    \end{macrocode}
%
% The before and after keys. We need to expand the concept.
%
%    \begin{macrocode}
\ExplSyntaxOn
\cxset
  {
    number~after/.store~in               = \numberafter@cx,
    number~after~content/.store~in       = \numberaftercontent@cx,
    number~before/.store~in              = \numberbefore@cx,
    number~before~content/.store~in      = \numberbeforecontent@cx,
  }
\ExplSyntaxOff

\cxset{number after=,
       number after content=,
       number before=,
       number before content=,}  
%  
\ExplSyntaxOn
\cxset{  
  number background-color/.code=\gdef\numberbgcolor{#1},
  number color/.store in=\chapter_number_color,
}
\cxset{number color=blue}
\ExplSyntaxOff
%    \end{macrocode}
%  \begin{docCommand}{number_font_weight} {\meta{font weight name}}
%    Defining the fonts follows the same pattern as for the other elements.
%    These are passed onto the \refCom{setnumberfont} for further processing by the
%    layouts engine.
%  \end{docCommand}
%
%    \begin{macrocode}
\ExplSyntaxOn
\cxset{    
  number~font-size/.store~in         = \number_font_size,
  number~font-family/.font-family~in = \number_font_family,
  number~font-weight/.font-weight~in = \number_font_weight ,
  number~font-shape/.font-style~in   = \number_font_shape,
  number~font-style/.font-style~in   = \number_font_shape,
  number~font-name/.store~in         = \number_font_name,% CHECK USAGE
}
\ExplSyntaxOff
%
%    \end{macrocode}
%
% \subsection{Number borders}
%    \begin{macrocode}
\ExplSyntaxOn
%

\cxset
  {  
    number~margin~top/.code     = \dim_gset:Nn\number_margin_top {#1},
    number~margin~left/.code    = \dim_gset:Nn \number_margin_left {#1},
    number~margin~right/.code   = \dim_gset:nn \number_margin_right {#1},
    number~margin~bottom/.code  = \dim_gset:nn \number_margin_bottom {#1}
  }  
%  
\cxset
  {  
  % number borders
  number~border-top-width/.code    = \dim_gset:Nn\number_border_top_width{#1},
  number~border-bottom-width/.code = \dim_gset:Nn\number_border_bottom_width {#1} ,
  number~border-right-width/.code  = \dim_gset:Nn\number_border_right_width{#1},
  number~border-left-width/.code   = \dim_gset:Nn\number_border_left_width{#1},
  number~border-width/.code        = \pgfkeysalso{number~border-top-width=#1,
                                                  number~border-right-width=#1,
                                                  number~border-bottom-width=#1,
                                                  number~border-left-width=#1},}
\ExplSyntaxOff
%    \end{macrocode}
%
%    \begin{macrocode}                                                         
\cxset{  
  number display/.is choice,
  number display/inline/.code=\global\setcounter{numberdisplay}{0},
  number display/block/.code=\global\setcounter{numberdisplay}{2},}   
%  
\cxset{  
  number float/.is choice,
  number float/left/.code=\global\setcounter{numberfloat}{0},
  number float/none/.code=\global\setcounter{numberfloat}{0},                                                    
  number float/center/.code=\global\setcounter{numberfloat}{1},
  number float/right/.code=\global\setcounter{numberfloat}{2},     
}
%    \end{macrocode}
%  \subsection{Shapes}
% I wasn’t too sure how to incorporate this in a nice way, so I defined a new property key,
% shape
% A shape can also have a style, if you want to really get fancy.
%  \begin{docCommand}{numbershape} {\meta{void}}
%    The |\numbershape| token list stores common shapes that can be used to shape the
%    number. We use \tikzname to as the shaping engine except for the rectangle that
%    we use our own. The number can be shaped separately from the chapter even when they
%    are inline. 
%  \end{docCommand}
% Picked up quite a few issues here. Need to revisit to ensure everything is ok. Seems
% to be an issue with number???
%    \begin{macrocode}
\ExplSyntaxOn
\cxset{
   number~shape/.shape~is= \numbershape,
}
\ExplSyntaxOff
\cxset{number shape=star}
%
%    \end{macrocode}
%  
%
% We define border styles first individually per side and then globally with
% a short-hand key.
% CSS has a dotted solid dashed double
% 
%    \begin{macrocode}
\ExplSyntaxOn
\cxset
  {                                                         
   number~border-style/.border~style~is=\number_border_style,
  }
\ExplSyntaxOff   
\cxset{number border-style=solid}
 
% number padding
\ExplSyntaxOn

\cxset{  
  number~padding-top/.code= \dim_gset:Nn \number_padding_top {#1},
  number~padding-right/.code=\dim_gset:Nn \number_padding_right{#1},
  number~padding-bottom/.code=\dim_gset:Nn \number_padding_bottom {#1},
  number~padding-left/.code=\dim_gset:Nn \number_padding_left{#1},
  number~padding/.code=\pgfkeysalso{number~padding-top=#1,
           number~padding-right=#1,
           number~padding-bottom=#1,
           number~ padding-left=#1},%
}   
\ExplSyntaxOff
%    \end{macrocode}                                                        
%
% \subsection{Author blocks}
% 
% Author blocks are only set if the boolean |\@authorblock| is set to true.
%
%    \begin{macrocode}
\cxset{
  author block/.is choice,
  author block/true/.code={\@authorblocktrue},
  author block/false/.code={\@authorblockfalse},
  author names/.store in=\authorblock@cx,  
  author block format/.store in=\authorblockformat@cx,
  author block afterskip/.store in=\authorblockafterskip@cx,
  chapter toc/.is choice,
  chapter toc/true/.code=\@toctrue,
  chapter toc/false/.code=\@tocfalse,
  chapter toc/none/.code=\@tocfalse,
}
%    \end{macrocode}
    
%    \begin{macrocode}    
\def\debugtitle{%                                         
\cxset{title border-top-color=sweet,
          title border-top-width=10pt,
          title border-left-color=sweet,
          title border-left-width=10pt,
          title border-right-color=sweet,
          title border-right-width=20pt,
          title border-bottom-color=sweet,
          title border-bottom-width=20pt,
          title border-width=0.2pt,
          title border-color=red,
          title padding-top=50pt,
          title padding-bottom=50pt,
          title padding-left=0pt,
          title padding-right=0pt,
          title padding=0pt,
          }
   }                
\cxset{chapter margin-top=0pt,
          chapter margin-left=20pt,
          chapter title align=left,
          chapter background-color=white,
          chapter border-left-width=0pt,
          chapter border-right-width=0pt,
          chapter border-bottom-width=0pt,
          chapter border-top-width=0pt,
          chapter font-shape=upshape,
}
\cxset{number background-color=white,
          number padding-left=0pt,
          number padding-right=0pt,
          number padding-top=0pt,
          number padding-bottom=0pt,
          number border-top-width=0pt,
          number border-bottom-width=0pt,
          number border-left-width=0pt,
          number border-right-width=0pt,
          number border-style=solid,
          number font-shape=upshape}
\cxset{author block=false,
          author block afterskip=,
          title margin-top=0pt,
          title margin-bottom=0pt,
          title margin-left=0pt,
          title margin-right=0pt,
          title border-top-color=sweet,
          title border-top-width=10pt,
          title border-left-color=sweet,
          title border-left-width=10pt,
          title border-right-color=sweet,
          title border-right-width=20pt,
          title border-bottom-color=sweet,
          title border-bottom-width=20pt,
          title border-width=0pt,
          title border-color=red,
          title padding-top=50pt,
          title padding-bottom=50pt,
          title padding-left=50pt,
          title padding-right=50pt,
          title padding=0pt,
          chapter title text-align=center,
          title display=block,
          }
\cxset{author names=}
\cxset{author block format=}
\cxset{chapter title width=0.7\textwidth}
\cxset{chapter title align=centering}
%    \end{macrocode}
%
% \begin{docCommand} {debugchapter} { \meta{void}}
% Settings for debugging a chapter heading. Shows all borders. \FIRE
% \end{docCommand}
%    \begin{macrocode}
\def\debugchapter{%
\cxset{chapter margin-top=0pt,
          chapter margin-left=0pt,
          chapter background-color=white,
%          
          chapter border-left-width=0.2pt,
          chapter border-right-width=0.2pt,
          chapter border-bottom-width=0.2pt,
          chapter border-top-width=0.2pt,
%          
          chapter padding-top=1pt,
          chapter padding-bottom=0pt,
          chapter padding-left=0pt,
          chapter padding-right=0pt,
%         
          number border-left-width=0.2pt,
          number border-right-width=0.2pt,
          number border-bottom-width=0.2pt,
          number border-top-width=0.2pt,
%          
          number padding-top=1pt,
          number padding-bottom=0pt,
          number padding-left=0pt,
          number padding-right=0pt,
}}
\debugchapter
%    \end{macrocode}
% 
%
% \section{Setting up the special chapter head mechanism}
%
% We divide chapter heads in two broad categories, the
%	standard chapter heads that utilize macros similar to
%	the standard classes and the \textit{special} chapter
%	heads that have their own typesetter commands.
%	For example we provide a special type of design for
%	this book called \textit{stewart}. The \cs{stewart}
%	is a template author defined command.
%	Any special design requires, two items. A macro defining
%	the design and setting the custom key to point to this macro.
%
%	 
% begin{macro}{custom}
% begin{macro}{customdesign@cx} 
%	This key holds the name of a macro that is to be
%	trigerred for a custom designed template. 
% 
%    \begin{macrocode}
\cxset{custom/.code=\global\@specialtrue
                \gdef\customdesign@cx{%
                      \csname#1\endcsname},
          fill/.store in=\fill@cx}
%    \end{macrocode}
%  
%
% 
%
% 	This macro  typesets the chapter label i.e., |CHAPTER|. We
%	set the font parameters as defined by the key value system.
%	The label is defined first as |CHAPTER| by the standard
%	class and later on as |Appendix|. 
%	If we need small caps or spaceout we use the |\so| command
%	from the |soul| package.
%    \begin{macrocode}
 \newcommand\inshape[2][fill=sweet,white]{%
% \rightline{\fbox{#2R}}
%\leftline{\fbox{#2}}
  %
        \begin{tikzpicture} 
         \filldraw[gray]  (0,0) circle [radius=1.5pt];%
         \node at (0,0) [%rounded rectangle,
                      trim left, 
                      name=s,
                      %anchor=midway,
                       behind path,
                       circle,
                       drop shadow={opacity=0.5,fill=sweet}, %box shadow in css
                        black,
                       % double=sweet,
                        %text height=1.5ex,
                        %text depth=1ex,
                        %anchor=s.base,
                        draw,
                        outer ysep=0pt, %no outer so that lines can align nicely
                        inner ysep=0pt,
                        inner xsep=0pt,
                        line width=1pt,%#1
                         ]{#2};
         \end{tikzpicture}%
\ignorespaces}%
%             

\def\tikzi{%
    \tikz[remember picture,overlay] 
    \draw[<->] (0,0)--(0,1.5)--++(-.2,0) node[left,fill=blue!15,text=black]%
       {{\ttfamily\footnotesize\string\chaptermarginleft}};%\space%
}%
%
%
\global\newsavebox\chapternamebox
\global\newsavebox\numbernamebox
\global\newsavebox\bothboxes
\global\newsavebox\tempboxa@cx
\global\newsavebox\tempboxb@cx
\global\newsavebox\tempboxc@cx
%    \end{macrocode}
%
%  \begin{docCommand}{set_chapter_font} {\meta{void}}
%    Set the chapter font from global keys.
%  \end{docCommand}
%
%    \begin{macrocode}
\ExplSyntaxOn
\cs_gset:Npn \set_chapter_font {%
  \exp_after:wN \setfontparam@cx\chapter_font_family;%
  \exp_after:wN \setfontparam@cx\chapter_font_size;%
  \exp_after:wN \setfontparam@cx\chapter_font_weight;%
  \exp_after:wN \setfontparam@cx\chapter_font_shape;%
}
\ExplSyntaxOff
%    \end{macrocode}
%

%
%   This is the main rendering routine for a generic block element. The element can either be
%    rendered in-line or as a block.
%
%    \#1  class of the element or id
%    \#2  the contents of the element e.g chapter or number.
%
%    Any element to be used here has to have a series of keys associated with it. They keep a naming
%    convention as for example,
%
%     |chapter border-left-width| \\
%     |chapter font-size|
%       
%    The prefix |chapter|  or |number|  or |title|  then enable to use the generic commands.
%    TeX is not an object orientated language, and future improvements are possible with LuaTeX.
%     
% {saveelementbox}
%    \begin{macrocode}
\ExplSyntaxOn
\def\saveelementbox#1#2#3{%
%    \end{macrocode}
% 
%  
%  Before we save the box, we set all its properties so we can measure it
%  correctly. As this is a generalized routine all properties use the prefix \#2
%  i.e., \meta{chapter}paddingtop etc.
%
%    \begin{macrocode}
   
   \expandafter\fboxseptop\csname#2paddingtop\endcsname
   \expandafter\fboxsepright\csname#2paddingright\endcsname
   \expandafter\fboxsepbottom\csname#2paddingbottom\endcsname
   \expandafter\fboxsepleft\csname#2paddingleft\endcsname
%
  \expandafter\fboxruletop\csname#2_border_top_width\endcsname\relax
  \expandafter\fboxruleright\csname#2_border_right_width\endcsname\relax
  \expandafter\fboxrulebottom\csname#2_border_bottom_width\endcsname\relax
  \expandafter\fboxruleleft\csname#2_border_left_width\endcsname\relax
%
%          
  \expandafter\savebox\csname#2namebox\endcsname{%
     %\fboxrule1pt\fboxsep1pt
       #3
      %\shadowbox{#3}%
       %\Ovalbox{#3}%
    
%       \doublebox{#3}
%      \hspace*{2cm}\tcbox[size=normal,
%         colframe=blue, colback =blue, borderline={2pt}{5pt}{black},
%         frame style={top color=blue, bottom color=black, 
%         left color=black, right color=black},
%            borderline west={2pt}{-2pt}{red},
%%        %
%        arc=5pt,outer arc=5pt, %!hyberbola
%        outer arc=3ptpt,rounded corners=all,
%        tikz={shape=star, text=white}]{#3}%
%      %
%        \tcbox[colframe=thelightgray,arc=5pt,%!hyberbola arcs 200
%      outer arc=5pt,rounded corners=all,
%      tikz={rotate=30}]{#3}%
 %         \phd@fbox{#3}%
  }%
}
\ExplSyntaxOff
%    \end{macrocode}
%
% \begin{docCommand} {print_chapter_name} {\marg{element name}} {\marg{element name}}
%   This function is a generalized macro that can be used to set and typeset
%   an element, working out all necessary floats.
% \end{docCommand}
%
%    \begin{macrocode}  
  \newcommand\printchaptername[2][chapter]{%
%    \end{macrocode}
% 
%    \begin{macrocode}
   \saveelementbox{}{#1}{#2}%   
%   if there is a margin on top set it      
%   #0 is inline   #2 block 
%   This decides if the element and subsequent elements are to be floated left or right. If the first element
%    is to be floated right, then all subsequent elements are floated right.
%    If we are on the first element, we set glue at the beginning to float all subsequent elements to the
%    right, if centered we do the same. 
%          <0  first element rendering
%           0    float left no glue
%           1    center inline 
%           2    right  - glue only at first element
%           3   float left and break
%           4   center and break
%           5   center no break 
%    
%     
%\global\setcounter{chapterfloat}{2}
%\global\setcounter{numberfloat}{2}
%  The following is only executed  for the first element, giving a signal as to how the next elements are to be floated
%  The first element is a negative number and hence will only be activated once.
%   
\setcounter{currentelementfloat}{\csname c@#1float\endcsname}%
      \ifcase \@arabic\c@currentelementfloat                      
                 \expandafter\renderleftblock{#1}\or         %0
                 \expandafter\rendercenterblock{#1}\or         %1
                 \expandafter\renderrightblock{#1}  \or         %2
                 \expandafter\renderinline{#1} \or          %3
      \else
                \rendercenterblock{#1}%
      \fi
%  We now can deal with any material that has to be rendered outside the |element| block, possibly material
%  such as horizontal or vertical rules.
   \ifnum\@arabic\c@numberdisplay=0
      %\hrule 
      %\csname#1after@cx\endcsname% 
     \else
     \@@par       
  \fi    
     }
    
%    \end{macrocode}
%
%  We now ready to render the text. If a border width has been defined we need to use
%  the |draw| property of the node to show it. If not we do not draw it. However, we might
%  still need to fill it, if a background color has been specified.  
%
%    
%  For block elements, i.e., elements that are allowed to float, we use a full line to float them. 
%    \begin{macrocode}
\def\rendercenterblock#1{%
       \appendtotoks{heading}{%
       \centerline{%
         \expandafter\unhcopy\csname#1namebox\endcsname
        }%
    }%
  }
%    \end{macrocode}  
% 
%    \begin{macrocode}
 \def\renderleftblock#1{%
   \appendtotoks{heading}{%
     % \leftline%
      }%
  }
  %
\def\renderrightblock#1{%
   \appendtotoks{heading}{%
     \rightline{%
         \expandafter\unhcopy\csname#1namebox\endcsname
      }% 
     }%   
}

\def\renderinline#1{%
  \appendtotoks{headingtoks}{%
     \leftline{%
         \expandafter\unhcopy\csname#1namebox\endcsname
      }%    
      }%  
    }

\def\renderboxcontents#1{%
        \drawmaybe{#1}%  
        \edef\tmp{\tempcmd@cx}
       \inshape[\expandafter\csname#1color@cx\endcsname,
                            fill=\expandafter\csname#1bgcolor\endcsname, 
                            ellipse, 
                            \expandafter\csname#1shape\endcsname, 
                            behind path,
                            line width=1pt,  %!fixme
                            \tmp,]{\expandafter\copy%
                                 \expandafter\csname#1namebox\endcsname}%
  }
%    \end{macrocode}
%
%
% \subsection{Author blocks}
% 
% {printauthorblock} 
%	An author author block is  printed for some chapter 
%	designs such as those in multi-author books, hence we provide a macro to typeset it. 
%	
%    \begin{macrocode}
\def\authorblockdebug{
\if@debug
    \tikz[remember picture,overlay] 
       \draw[<->] (0,0)--(0,0.5)--++(-.2,0)% 
              node[left,fill=blue!15,text=black]%
               {{\ttfamily\footnotesize author block=true}};%
  \fi  
}
\ExplSyntaxOn
\cs_new:Npn \print_author_block:
  {%
    \authorblockdebug
    \authorblockformat@cx\authorblock@cx
  }
 
%
%    \end{macrocode}
%   
%

%	We also provide a macro to typeset the number with appropriate
%	hooks for key value parameters.
%
%  \begin{docCommand}{setnumberfont}{}
%  sets the font for the number part of a chapter heading
% \end{docCommand}
%    \begin{macrocode}

\def\setnumberfont{%
    
    %\set_font_parameters:n {number}
  }%

%    \end{macrocode}

% 
%    \begin{macrocode}

\def\afteralignhook@cx{\par}
  \box_new:N \chapter_title_box
  \dim_new:N \chaptertitleboxwidth
  \dim_new:N \offset_for_center
  \dim_new:N \offset_for_right
  \dim_new:N \total_title_width
  \dim_new:N \title_bounding_box_width 
  \dim_set_eq:Nc \chaptertitleboxwidth {chapter_title_text_width}

%    \end{macrocode}
%
%
%   \begin{docCommand} {print_chapter_title} { \meta{contents} }
%   This function is responsible to make the chapter title. It receives one parameter
%   which comes from the |\chapter|  command and is the contents of the title.
%   \end{docCommand}
%
%    \begin{macrocode}


\cs_new:Npn \print_chapter_title #1 
 {
      \if@titlespaceout
         \long\def\SSS{{\so{#1}}}
      \else
         \long\def\SSS{#1}
      \fi%
  

			\dim_gset:Nn \total_title_width %including padding+borders
			    {
			        \chapter_title_text_width % text width
			       +\title_border_left_width
			       +\title_border_right_width
			       +\title_padding_left_width
			       +\title_padding_right_width
			     }
%	   
     \savebox \chapter_title_box {% 
        \begin{tcolorbox}[
           size=minimal,
           colback=spot!15,
           colframe=spot!15]
           \set_font_parameters:n {title}
           \chapter_title_text_align%
           \language-1
           \SSS\par %NEW NEW CHECK
        \end{tcolorbox}         
      } 
     
%%    \end{macrocode}
%%
%%   Having measured the title block, we now typeset it. Before we typeset it
%%   we will provide borders all around if required and also allow for padding. 
%%   We will not repeat the browser wars here, so we will provide the borders
%%   outside the block and the padding inside.
%%   
%%   Once we done we append everything to the heading toks.
%%  
%%    \begin{macrocode}

    \appendtotoks{heading}{%

%%    \end{macrocode}
%%  We first add the before hook
%%  This is normally used to put ornaments or rules before the title text
%%  block.
%%    \begin{macrocode}
  
    \dim_gset:Nn \title_bounding_box_width %width defined by kernel!strange error here
      {
        \chapter_title_text_width
        +\title_border_left_width
        +\title_border_right_width
        + \title_padding_left_width
        + \title_padding_right_width  
      }
%%    \end{macrocode}
%%
%%   Having measured the title block, we now typeset it. Before we typeset it
%%   we will provide borders all around if required and also allow for padding. 
%%   We will not repeat the browser wars here, so we will provide the borders
%%   outside the block and the padding inside.
%%   
%%   Once we done we append everything to the heading toks.
%    \begin{macrocode} 
    \dim_gset:Nn \offset_for_center { ( (\linewidth - \chapter_title_text_width)/2) }
    \dim_gset:Nn \offset_for_right {(\linewidth-\chapter_title_text_width)}
    \dim_gset:Nn \l_tmpa_dim {0pt}
    \dim_gset:Nn \l_tmpb_dim {0pt}
    \tcbset{centered/.style={
          width=\chapter_title_text_width, 
          boxrule=0pt,boxsep=0pt,
          left=0pt,right=0pt,
          leftright~skip=0pt,
          middle=0pt,arc=0pt,toptitle=0pt,bottomtitle=0pt,
          before={\vskip1pt},
          after={\vskip1pt},
          colback=spot!15}
             }
                
    \tcbset{floatright/.style={
             width=\chapter_title_text_width, 
             boxrule=0pt,
             boxsep=0pt,
             left=0pt,
             right=0pt,
             top=0pt,
             bottom=0pt,
             left~skip=-\offset_for_right-\offset_for_right,
            % right~skip=0pt,
%             leftright~skip=0cm,
             %middle=0pt,arc=0pt,toptitle=0pt,bottomtitle=0pt,
             before={\vskip1pt},
             after={\vskip1pt}}
                }             
%         \par\rule{\linewidth}{2pt}\vskip0pt
    \begin{tcolorbox}[
          centered,
          width=\offset_for_right,
          colback=spot!15,
          colframe=spot!30]
           % Before
         \end{tcolorbox}      
        %\hspace*{3cm}
        \begin{tcolorbox}[
           floatright,
           colback=spot!15,
           colframe=spot!50]
           \set_font_parameters:n {title}
           \chapter_title_text_align%
           \language-1
           \SSS\par %NEW NEW CHECK
        \end{tcolorbox} 
    }%appendtotoks 
    
}
\ExplSyntaxOff
%    \end{macrocode}

% 
%
% \begin{docCommand}{make_chapter_head} {\marg{optional part of chapter}}{\marg{title}}
%	 The macro calls the main typesetting activities of the chaper head
%	 We begin our typesetting by checking, if the macro has a special
%	 design which we then call.
%  This is a replacement macro for \latexe |\@make_chapter_head|. We get it here
%  with two parameters in order to set also the chaptermark
%  \end{docCommand}
%
%    \begin{macrocode}
%    \newif\if@mainmatter \@mainmatterfalse CHECK THIS NOT NECESSARY
%
%
%    \end{macrocode}
% 
%  \begin{docCommand}{appendtotoks} { \marg{toks name} \marg{contents}}  
%   The command \cmd{\appendtotoks} is just a helper macro to append tokens to a
%   token register, defined as \meta{element name}|toks|. 
%  \end{docCommand}
%
%  \begin{docCommand}{prependtotoks} { \marg{toks name} \marg{contents}}    
%   The command \cmd{\prependtotoks} is just a helper macro to prepend tokens to a
%   token register, defined as \meta{element name}|toks|. 
%  \end{docCommand}
%
%    \begin{macrocode}
\ExplSyntaxOn
\cs_new:Npn \appendtotoks #1#2
  {
    \expandafter\expandafter\expandafter\csname#1toks\endcsname\expandafter{
      \the\csname#1toks\endcsname #2}
  }
\cs_new:Npn \prependtotoks #1#2
  {
    \expandafter\expandafter\expandafter\csname#1toks\endcsname\expandafter{
      #2 \the\csname#1toks\endcsname }
  }  
\ExplSyntaxOff
%    \end{macrocode}
% 
%
%    \begin{macrocode}
\ExplSyntaxOn
\def\set_number: {
  \appendtotoks{number}{\numberbefore@cx}%
  \appendtotoks{number}{\color{blue}%chapter_number_color
  \appendtotoks{number}{\setnumberfont}%     
  \if@chapterspaceout
     \if@soulspaceout
         \expandafter\appendtotoks{number}{\so\thechapter}%
      \fi
   \else
      \expandafter\appendtotoks{number}{\thechapter}%   
   \fi   
   \appendtotoks{number}{\numberafter@cx}%
}
\ExplSyntaxOff
%    \end{macrocode}
%  
%   \begin{docCommand}{make_chapter_head}{\oarg{contents}\marg{contents}}
%     This is where everything happens. If the design is special
%     it is by-passed and send to the custom layout renderer.
%
%     The engine first assembles all the elements that make up the
%     the chapter head and inserts them into a sequence list.
%     The macros are then expanded by the typesetting section of the 
%     layout engine.
%
%   \end{docCommand}
%    \begin{macrocode}
\ExplSyntaxOn
%
\cs_gset:Npn \make_chapter_head #1 #2
  {  
%    \end{macrocode}
%
% 	We now ready to typeset the chapter heading, we run everything 
% 	within a group and
% 	activate the chapter only if the |thesecumdepth>-1| and if only we are 
% 	within mainmatter. 
%
%    \begin{macrocode}
    \tex_parindent:D 0pt 
      %\offinterlineskip
      %
    \normalfont%
%  \ifnum \c@secnumdepth>\m@ne%
%    \if@mainmatter%
%    \end{macrocode}
%
% We first check if we need to print anything before the chapter head, as for
% example an image or a graphic, we then check if the number is to the left of
% the right of the image and typeset it. We follow it by printing any chapter
% 
%    \begin{macrocode}
    \tex_lineskip:D 0pt% 
    \tex_topskip:D 0pt%check this out
    %\chaptermargintop@cx%
    \leavevmode% 
%    \end{macrocode}
%
%  The typesetting of the chapter and number combination and as a matter of fact
%  any generalized string of element blocks depends if the value of the property |display|
%  is inline, inline block or block. A block can float freely, whereas the others restrict the
%  typesetting to a linear mode.
%
%  Both token registers have a common part and a unique part
%    \begin{macrocode}
      \set_number:
%    \end{macrocode}
%
%    In terms of the language semantics, I am not too sure, what should happen, if the
%    heading does not have a name (should all its properties be nullified and be taken 
%    over by the next element? Needs thought and I will revisit this.
%
%    \begin{macrocode}  
       \xdef\xtemp{\chaptername}%
       \xdef\ytemp{}%         
       \ifx\xtemp\ytemp%
          \printchaptername[number]{\the\numbertoks}%               
       \else
%    \end{macrocode}   
%  
%  We collect all the commands in a token register. We start with the color and font
%  settings which we add in a group.
%             
%  Any before property means vertical mode by definition, we do this by means of a |vskip0pt|
%    \begin{macrocode}
       \appendtotoks{chapterprelim}{\leavevmode \chapter_before\vskip0pt}
       \appendtotoks{chapter}
         {
           \leavevmode\noindent
           \color{\chapter_color}
           \set_chapter_font
         }
       \appendtotoks{chapter}{\chapter_before_content}  
                 
           %\if@chapterspaceout
               \expandafter
  %               \so{\chaptername}}%PROBLEMS FIX
                                             
%           \else
              \appendtotoks{chapter}{\chaptername}%
 %          \fi   
         
%
%    \end{macrocode}
%
%  Next we need to add the tokens for decorating the number. We expect all headings to ne
%  numbered if the word `chapter’ is prefixed to the heading. 
%    
%
%    \begin{macrocode} 
        \ifnum\thenumberdisplay=0 %                    
          \appendtotoks{chapter}{\kern0.5em}%
          \appendtotoks{chapter}{\the\numbertoks\numberaftercontent@cx}%
%    \end{macrocode}
%
%   We are done with inline headings and we can typeset them.
%    \begin{macrocode}
%   
              \the\chapterprelimtoks                 
              \printchaptername[chapter]{\the\chaptertoks}%
                 %
%    \end{macrocode}
%
%  If both the chapter name as well as the number are displayed as blocks
%  we typeset them in two operations.
%
%    \begin{macrocode}                  
          \else
                %\the\chapterprelimtoks%
                \expandafter\printchaptername[chapter]{\the\chaptertoks}%
                \expandafter\printchaptername[chapter]{\the\numbertoks}%
        \fi%ifnum  


  \set_chapter_title:nn {#1}{#2} 
  \appendtotoks{heading}{
       \begin{tcolorbox}[
          centered,
          left=0pt,right=0pt,colback=spot!15,
          ]
         %after~title
       \end{tcolorbox}}
 \typeset_heading:   
 }      
\ExplSyntaxOff 
%    \end{macrocode}
%

% \begin{docCommand}{set_chapter_title:nn} {\marg{arg1}}{\marg{arg2}}
%   sets all the parameters for the chapter title block.
% \end{docCommand}

%    \begin{macrocode}
\ExplSyntaxOn
\cs_new:Npn \set_chapter_title:nn #1#2{
    \if@chaptertitlespecial%
       \csname ethics\endcsname{#2}%
    \else%
       \print_chapter_title { #2 }
     \fi
}     
\ExplSyntaxOff
%    \end{macrocode}       
%
% Now we ready to add the contents in an outer box or vbox will see 
%    \begin{macrocode} 
\ExplSyntaxOn
\cs_new:Npn \author_block_aux: {
   \skip_vertical:n \title_margin_bottom
      \if@authorblock 
           \print_author_block:
           \authorblockafterskip@cx
      \fi%
    \tex_par:D
    \nobreak%
   %
}%
\ExplSyntaxOff
%    \end{macrocode}
%    \begin{macrocode}
\tcbset{centered/.style={
             width=\linewidth, 
             boxrule=0pt,boxsep=0pt,left=0pt,right=0pt,
             leftright~skip=0cm,middle=0pt,arc=0pt,toptitle=0pt,bottomtitle=0pt,
             before={\vskip1pt},
             after={\vskip1pt}}
            }
%    \end{macrocode}
%    \begin{macrocode}  
\ExplSyntaxOn              
\cs_new:Npn \typeset_heading: {
 
       \parindent0pt
       \begin{tcolorbox}[
         width=\textwidth, 
         boxrule=0pt,
         boxsep=0pt,
         leftright~skip=0cm,
         middle=0pt,
         arc=3pt,
         toptitle=0pt,
         bottomtitle=0pt,
         left=0pt,
         right=0pt,
         before={\vskip1pt},
         after={\vskip1pt},
         oversize=0cm,
         colback=spot!15,
         colframe=white,
         ]
        %\parindent1em       
        \the\headingtoks
       \end{tcolorbox}
% Need to empty the toks after we typeset them!!! Or use them in a group
% Add to notes for pittraps of heading management!       
  \headingtoks{}  
  %\headingstoks{} 
  \numbertoks{}
  \chaptertoks{}
  \chapterprelimtoks {}  
  \author_block_aux:  
   
  }   
\ExplSyntaxOff
%    \end{macrocode}
% 
%
%
% {chapter} The \cs{chapter} is modified to
% 	add hooks for openings and headers. 
% 	The |book| standard class states that a chapter should
% 	always start on a new page. In reality many book styles
%	allow the chapter heading to be continuous i.e., more
%	like a section. \label{code:chapterafterindent} 
%    \begin{macrocode}
\global\newif\if@chapterafterindent@cx \@chapterafterindent@cxtrue
%    \end{macrocode}
%
%  We set keys for |\afterindent| to enable it via the key value interface
%  
%    \begin{macrocode}
\cxset{chapter afterindent/.is choice,
           chapter afterindent/true/.code=\gdef\chapterafterindent@cx{%
                                                            \global\@chapterafterindent@cxtrue},
           chapter afterindent/false/.code=\gdef\chapterafterindent@cx{%
            \global\@chapterafterindent@cxfalse},
}           
% We set this to false by default
\cxset{chapter afterindent=true}
% call it after a heading
%    \end{macrocode}
% \begin{docCommand}{chapter_after_heading:}{ \meta{void} }
%   This is a hook, that is called after the chapter heading is typeset in
%   order to control the indentation of the first paragraph.
%   It is based on similar code from \latexe and we explorify it.
% \end{docCommand}
%    \begin{macrocode}
\ExplSyntaxOn
\cs_gset:Npn \chapter_after_heading:
  {
    \@nobreaktrue
    \everypar
      {
         \if@nobreak
           \@nobreakfalse
           \clubpenalty \@M
           \if@chapterafterindent@cx \else
             {\setbox\z@\lastbox}
           \fi
         \else
         \clubpenalty \@clubpenalty
         \everypar {}
         \fi
      }
  }
\ExplSyntaxOff  
%    \end{macrocode} 
% 
%  \begin{docCommand}{chapter} {\oarg{short title}\marg{contents}}
%  We renew |\chapter| after we save its old definition
%  \end{docCommand}
%
%    \begin{macrocode}
\ExplSyntaxOn
\let\ltxchapter\chapter
%

\renewcommand\chapter
  {
    \if@openright\cleardoublepage\fi
    \if@openleft\cleartoevenpage\fi
    \if@openany\clearpage\fi
%    \end{macrocode}
%  
%	Floats are prevented from floating at the top of chapter
%	opening pages as they look out of place.
% 	|\headerstyle@cx| defaults to empty.
%    Then we suppress the indentation of the first paragraph by
%    setting the switch |\@afterindent| to |false|. We use |\secdef|
%    to specify the macros to use for actually setting the chapter
%    title.
%
%    \begin{macrocode}
    \thispagestyle{empty}
    \global\@topnum\z@~%
%    \end{macrocode}
%
% We provide a hook to handle indentation after a chapter. This would 
% also necessitate to change the afterheading macro and make it specific 
% to a chapter head.
%    \begin{macrocode}  
      \@chapterafterindent@cxfalse
   % \@afterindentfalse
%    \end{macrocode}
% 
%
% 	Everything is now ready to call |secdef|, which is defined in the
% 	kernel. This command takes two arguments and calls the auxiliary
% 	macros for starred and unstarred commands. 
%
%    \begin{macrocode}
      \secdef\__chapter\@schapter
   }%[optional]{title} follows
\ExplSyntaxOff
%    \end{macrocode}
%
% We first define the unstarred version of the command.
% This is modified to include
% our hooks.
%%    \begin{macrocode}
\newif\if@tocspecial\@tocfalse
\def\formattoctitle{}
% 
%    \end{macrocode}
%
% \begin{docCommand}{__chapter} {\marg{}\marg{}}
%    This macro is called when we have a numbered chapter. When
%    |secnumdepth| is larger than $-1$ and, in the book
%    class, |\@mainmatter| is true, we display the chapter
%    number. We also inform the user that a new chapter is about to be
%    typeset by writing a message to the terminal. We hook
%	 here to add a number of typesetting key value macros.
%	
%    \#1 what to write in toc if has optional argument
%    \#2 title
% 
%   
%   The non-star form of the command.
%  
%   \end{docCommand}
%    \begin{macrocode}
\ExplSyntaxOn
\cs_gset:Npn \__chapter [#1]#2 
  {
  %\refstepcounter{chapter}%
    \ifnum \c@secnumdepth >\m@ne%
      \if@mainmatter
        \if@toc% added extra if
          \refstepcounter{chapter}%
          \typeout{\@chapapp\space\thechapter.}%
          \def\tocchapternumber@cx{\@arabic\c@chapter}%to move over
          \phantomsection
          \addcontentsline{toc}{chapter}
            {
              \protect\numberline{Chapter~\tocchapternumber@cx~}{#1}
         %\protect\chapternumberline{\tocchapternumber@cx}{#1}{\tocimage@cx}
            }
        \fi%
            %\fi%
      \else
          \addcontentsline{toc}{chapter}{#1APAPAP}%for prelims???
      \fi%
    \else%
      \addcontentsline{toc}{chapter}{#1PPP}%????? for part
    \fi%secnumdepth
%    \end{macrocode}
%
%	After having written the entry to the table of contents we
%	store the alternative title of this chapter with |\chaptermark|
%	and add some white space to the lists of figures and tables.
%    In one column mode we call \refCom{chapter_after_heading:} which takes care 
%	of supressing the indentation after a chapter heading. Then
% we hand over the typesetting to \refCom{make_chapter_head}, which
% is the work-horse of the package. 
%
%    \begin{macrocode}
  \chaptermark{#1}%
  \addtocontents{lof}{\protect\addvspace{10\p@}}%
  \addtocontents{lot}{\protect\addvspace{10\p@}}%
     \if@twocolumn
         \@topnewpage[\make_chapter_head {#1} {#2}]%
      \else%
%         \if@special
%            \customdesign@cx{#2}%needs to go different 
%         \else
            \make_chapter {#1} {#2}%
%         \fi
%    \end{macrocode}
%
% Next we call \refCom{chapter_after_heading:} to set the indentation
% after the heading is typeset.
%    \begin{macrocode}         
          \chapter_after_heading:
      \fi
   }

% Although the css box model is good, some simpler designs have their
% own renderers.
%     
\cs_new:Npn \make_chapter #1 #2
  {

    \str_case_x:nnTF { box }  
     {
       { traditional} { \format_part_traditional:nn { chapter } { #1 } { #2 } }
       { box        } { \format_head_boxed:nn       { chapter } { #1 } { #2 } } 
       { inline     } { \format_head_inline:nn      { chapter } { #1 } { #2 } }
       { inmargin   } { \format_head_inmargin:nn    { chapter } { #1 } { #2 } }
       { css        } { \make_chapter_head {#1} {#2}                          }
      }
      {                                           }
      { \cs:w stewartpart\cs_end: {chapter} {#1}  }
   }
        
\ExplSyntaxOff   
%    \end{macrocode}
% 
%    \begin{macrocode}
\cxset{toc image/.store in = \tocimage@cx}
\cxset{toc image ={}}
%
%    \end{macrocode}
%</SECT>
\endinput    